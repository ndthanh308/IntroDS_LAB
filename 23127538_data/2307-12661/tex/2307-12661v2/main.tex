\documentclass[a4paper, 10pt]{amsproc}

\usepackage{defs}
\usepackage{amsaddr, orcidlink}

\title[Algorithmic construction of Lyapunov functions]{On the algorithmic construction of Lyapunov functions for continuous vector fields} 

\author{Raavi Gupta\,\orcidlink{0009-0003-7585-5820}, Sameep Chattopadhyay\,\orcidlink{0009-0006-4474-8266}}
\address{%
	\faGroup\ Department of Electrical Engineering\\
	\faUniversity\ Indian Institute of Technology Bombay\\
	\faMapMarker\ Powai, Mumbai 400076, India
}

\author[R. Gupta, S. Chattopadhyay, P. Paruchuri, and D. Chatterjee]{Pradyumna Paruchuri\,\orcidlink{0000-0002-8598-5069}, and Debasish Chatterjee\,\orcidlink{0000-0002-1718-653X}}
\address{%
	\faGroup\ Systems \& Control Engineering\\
	\faUniversity\ Indian Institute of Technology Bombay\\
	\faMapMarker\ Powai, Mumbai 400076, India
}
\thanks{%
	\faHome\ (DC) \url{https://www.sc.iitb.ac.in/~chatterjee}\\
	\indent	\faEnvelope\ \texttt{raavi02g@gmail.com, sameep.ch.2002@gmail.com, pradyumnaparuchuri@gmail.com, dchatter@iitb.ac.in}
}

\thanks{D.\ Chatterjee acknowledges partial support of the SERB MATRICS grant MTR/2022/000656 from the Govt.\ of India.}

\date{\DTMnow}

\begin{document}

\begin{abstract}
	This article presents a novel numerically tractable technique for synthesizing Lyapunov functions for equilibria of nonlinear vector fields. In broad strokes, corresponding to an isolated equilibrium point of a given vector field, a selection is made of a compact neighborhood of the equilibrium and a dictionary of functions in which a Lyapunov function is expected to lie. Then an algorithmic procedure based on the recent work \cite{ref:DasAraCheCha-22} is deployed on the preceding neighborhood-dictionary pair and charged with the task of finding a function satisfying a compact family of inequalities that defines the behavior of a Lyapunov function on the selected neighborhood. The technique applies to continuous nonlinear vector fields without special algebraic structures and does not even require their analytical expressions to proceed. Several numerical examples are presented to illustrate our results.
\end{abstract}

\maketitle

% Figure environment removed

\section{Introduction}
Automatic 3D reconstruction of clothed humans using image inputs has gained increasing significance due to its potential applications in a wide array of AR/VR scenarios. High-fidelity reconstructions typically depend on sophisticated capture systems, which are developed with dense camera arrays~\cite{collet2015high,joo2015panoptic,joo2018total}, programmable light-stages~\cite{Vlasic2009, guo2019relightables}, and depth sensors~\cite{newcombe2011kinectfusion,DoubleFusion,BodyFusion,dou2016fusion4d,newcombe2015dynamicfusion}. However, stringent capture environments equipped with complex hardware pose significant challenges for consumer-level applications.


In this context, considerable research effort has been dedicated to developing methods that allow for more flexible capture configurations, such as utilizing a few RGB inputs. Among these works, learning implicit functions \cite{iccv2020PIFu, saito2020pifuhd, hong2021stereopifu} has proven effective in achieving highly detailed reconstructions by integrating the advancements of deep neural networks. These methods employ large multi-layer perceptrons (MLPs) to predict the occupancy probability or truncated signed distance function (TSDF) value of every queried 3D point based on its associated local feature, which is extracted from images. They can recover a continuous surface at arbitrary resolutions without topology restrictions.


However, in typical MLP-based implicit networks, the occupancy or TSDF value at each location is solved independently with planar image features, rendering them less capable of addressing challenging cases such as occlusions. Consequently, these methods suffer from generalization and robustness issues, particularly when tackling strong occlusions caused by large motion or multiple interacting humans. 
Some follow-up studies  \cite{zheng2021deepmulticap,zheng2021pamir,huang2020arch} utilize an extra geometric model, SMPL~\cite{Loper2015}, to improve robustness by introducing strong shape priors. 
Their success typically relies on the assumption of geometrical similarity \cite{huang2020arch} between the shape prior and target reconstruction, making them intractable for handling complex cases with loose clothes and sensitive to errors in SMPL model fitting.



%\ping{this paragraph sounds like `TSDF is better than MLP/SMPL, and we use TSDF to solve the problem'. But in Sec 3, we are telling a different story, saying `MLP needs a 3D convolutional encoder'. We need to make these two sections consistent.}\sicong{I think in this paragraph we claim that the TSDF}


%We opt for Trucated Signed Distance Funtion (TSDF) volumetric representations as they are naturally suitable for convolution operations, which have shown remarkable performance for learning hierarchical features on 2D visual perception tasks \cite{SunXLW19}. 
%Meanwhile, TSDF also describes the gradual geometry change around shape surface, which is not reflected by occupancy volume. 

We instead revisit the 3D volumetric representation and resort to 3D convolutional neural networks (CNNs) for feature learning, due to their impressive performance in feature learning and the ability to incorporate spatial context. However, volumetric methods and 3D convolution involve discretization, which might raise concerns regarding whether a discretized volume can preserve subtle geometric details as continuous representations learned in implicit functions. We investigate the relationship between volume resolution and quantization error on synthetic data by converting target mesh objects to TSDF volumes, as shown in Figure~\ref{fig:quantization_error}. We observe that the quantization errors are significantly reduced by increasing volume resolution and become nearly negligible when reaching a relatively high resolution (e.g., 512 or higher). In other words, achieving fine-detailed reconstruction is not supposed to be restricted by the use of volume representations as long as a proper volume resolution is utilized. Therefore, we present a method with high-resolution feature volumes, e.g., 256 and 512, while traditional volumetric methods \cite{varol18_bodynet,gilbert2018volumetric} are often limited to much lower resolutions, such as 32 or 128.



On the other hand, an increase in volume resolution may lead to a cubic growth of memory overhead \cite{8100085}. Reducing memory costs while guaranteeing the granularity of volumetric representations is necessary for pursuing high-quality reconstruction. Thus, we adopt a coarse-to-fine approach and cull away irrelevant voxels to build a sparse high-resolution feature volume. At the coarse level, the network computes an initial TSDF by applying a U-Net with sparse 3D CNN \cite{3DSemanticSegmentationWithSubmanifoldSparseConvNet} on the sparse feature volume, which is carved by a visual hull. Through our experiments, it turns out that more than 95\% of the volume grids are discarded by the visual hull culling, making the sparse 3D CNN efficient. At the fine level, the network focuses on a narrow band near the zero-level set of the initial TSDF and discretizes the narrow band with smaller voxels. By employing this narrow-band culling, we further shrink the sampling space, resulting in a relatively small range of grid numbers (usually 300K--500K in our experiments) even with a high volume resolution of 512. The remaining voxels in the narrow band are associated with features that fuse high-frequency information from the computed normal maps upon the low-frequency shape from the coarse level to compute the TSDF at high resolution. The final mesh is then extracted from the TSDF using the Marching-Cube algorithm ~\cite{Lorensen87marchingcubes}.
% Different from the u-net sturcture to preserve global topology context, we then apply a shallow 3dcnn to compute the final TSDF $D_{final}$ which contain more local geometry detail.




% \ping{this paragraph can be expanded. It is an important contribution and often ignored by other works. stress on the novel idea of regressing blending weights instead of colors}

In addition to geometry, high-quality mesh texture is also a crucial factor contributing to visual appearance. Directly computing a color field in 3D space, as in \cite{iccv2020PIFu}, struggles to capture high-frequency texture details, while the neural radiance field (NeRF) \cite{yu2020pixelnerf} or the DoubleField~\cite{shao2022doublefield} require expensive per-instance optimization and are often unstable for sparse input images. In contrast, we adopt an image-based rendering approach to compute a texture atlas map, which is efficient and widely supported in existing computer graphics tools. 
Specifically, we compute a blending weight at each 3D point on the mesh surface to determine its color as a weighted average of the colors at its image projections. The blending weights can be computed at a relatively coarse resolution, e.g., 512 volume resolution in our case, and leave texture details to the high-resolution images, such as 1K or 2K. Unlike previous methods that generate blurry texturing results under sparse input, our method generalizes well on both synthetic and real data with just a few input views. 
Figure~\ref{fig:teaser} shows two examples reconstructed by our method. Despite the challenging garment, pose, and occlusion, our method recovers faithful shape, normal, and texture on the right.

%with a wide variety of poses and clothing styles, and it is also adaptive to handle input image with arbitrary resolutions.
%\sicong{For this concern we claim that when the resolution of dicretized volume meets certain threshold (which is 256 in our experiment), the quantization error can be neglected.} 



In summary, the main contributions of this paper are as follows:
\begin{itemize}
\vspace{-0.1in}
  \item 
  We revisit the 3D volumetric representation and demonstrate that it can support clothed human reconstruction with equal or even better performance compared to implicit representation. 
  \item 
  We develop a memory and computation-efficient method for high-resolution volumetric reconstruction using sophisticated sparse 3D CNN, coarse-to-fine estimation, and voxel culling by visual hull and narrow bands. 
  \item 
  We introduce a novel method to compute a texture atlas map, which captures rich appearance details from high-resolution input images.
  \item 
  We achieve impressive results on standard benchmark datasets Twindom and MultiHuman, significantly reducing the point-2-surface (P2S) precision to approximately 0.2cm from just six input views, with more than $50\%$ error reduction compared to the state-of-the-art methods, including DoubleField~\cite{shao2022doublefield} and PIFuHD~\cite{saito2020pifuhd}.
\end{itemize}


\vspace{-0.15cm}

\section{Strategy Templates}\label{sec:templates}

In this section, we introduce a formalization of player \Odd strategies in \Odd-fair parity games via \emph{strategy templates}.
% 
In contrast to player \Even, player \Odd winning strategies are no longer positional in \Odd-fair parity games, as illustrated by the following example. %that requires the same number of symbolic steps as the algorithm computing winning strategies for \Even in \enquote{normal} parity games.
% \vspace{-0.5em}
\begin{example}\label{ex:strategytemplates}
Consider the three different parity games depicted in Fig.~\ref{fig:Oddstrategies1}. %, three \Odd-fair parity games are depicted, with circles indicating \Ve and squares indicating \Vo. Edges in $E^\ell$ are shown by dashed lines. All nodes are labeled with their priorities.
   In all three games, \Odd has a winning strategy from all vertices, i.e., $\mathcal{W}_{Odd}=V$. %The red-colored edges indicate \Odd's strategy: if \Odd takes the red edges alternatingly from the source nodes, it wins from all nodes. 
  However, in order to win, the vertex $3$ has to be seen infinitely often in game (a) and (b), which forces \Odd to use its live edge\textbackslash s infinitely often. This prevents the existence of a positional strategy for \Odd in games (a) and (b): In (a) it needs to somehow alternate between (it's only) live edge to $4$ and a \enquote{normal} edge to $7$ (both indicated in red) in order to win, and in (b) it needs to somehow alternate between all its live edges (also indicated in red). In the game (c), \Odd can win by 'escaping' its live vertex $3$ to a \enquote{normal} vertex $5$, and thereby has a positional strategy. % (again indicated in red).
   
  Now consider the subgraph of each game formed by all colored edges (red and blue), which include the strategy choices from \Vo and \emph{all} outgoing edges from \Ve. As we have seen that \Odd needs to play all red edges repeatably, this subgraph represents the paths that \emph{can} be seen in the game depending on the \Even strategy. Hence, a node $v\in\Vl\subseteq\Vo$ can be seen infinitely often in a play (compliant with \Odd's strategy), if it lies on a cycle in this subgraph. We observe that, in games (a) and (b), node $3$ lies on cycles in this subgraph, whereas in game (c), it does not. 
  We further see that whenever a vertex  $v\in\Vl$ lies on a cycle, \Odd needs to take all its outgoing live edges (as for vertex $3$ in example (b)) and possibly one more edge (as for vertex $3$ in example (a)), for all other vertices in $\Vo$ a positional strategy suffices (as for vertex $5$ in all examples, and for vertex $3$ in example (c)). This shows that \Odd strategies are intuitively still \enquote{almost positional}.
% 
\end{example}

% Figure environment removed


\vspace*{-0.2cm}

The intuitions conveyed by Ex.~\ref{ex:strategytemplates} are formalized by the following definitions. % for \Odd strategy templates.


\begin{definition}[\Odd Strategy Template]\label{def:Oddstrategytemplate}
 Given an \Odd-fair parity game $\mathcal{G}^\ell = \ltup{\mathcal{G}, E^\ell}$ with \newline $\mathcal{G} = \langle V, \Ve, \Vo, E, \chi\rangle$, an \Odd \emph{strategy template} $\mathcal{S}$ over $\mathcal{G}^\ell$ is a subgraph of $\mathcal{G}$ given as follows: $\mathcal{S}:=\tup{V',E'}$ where $V'\subseteq V$ and $E'\subseteq E \cap (V' \times V')$ such that the following hold,
\begin{compactitem}\label{item:Oddstrtemprules}
 \item if $v \in \Vo \cap V'$ does not lie on a cycle in $(V',E')$, then $|E'(v)|=1$,
 \item if $v \in \Vo \cap V'$ lies on a cycle in $(V',E')$ then $E^\ell(v) \subseteq E'(v)$ and  $1\leq |E'(v)|\leq |E^\ell(v)| + 1$,
 \item if $v \in \Ve \cap V'$, then  $E'(v) = E(v)$.
\end{compactitem}
\end{definition}
%
\begin{definition}\label{def:compliantstrat}
 Let  $\mathcal{G}^\ell = \ltup{\mathcal{G}, E^\ell}$ be an \Odd-fair parity game with \Odd strategy template $\mathcal{S}=\tup{V',E'}$, and $V'_\Odd := V' \cap V_\Odd$. Then an
\Odd strategy $\rho$ is said to be \textbf{compliant} with $\mathcal{S}$ if  
it is a winning strategy in the game $\ltup{\gamegraph,\alpha'}$ where $\gamegraph= \tup{V,\Ve,\Vo,E}$ and 
\begin{subequations}
 \begin{align}
 \alpha':= &\textstyle\bigwedge_{v\in\Vo'}(\,\square\, (\,v \implies \bigvee_{(v,w)\in E'} \bigcirc\, w\,))\,\label{equ:alpha:a}\\
 & \textstyle\wedge \bigwedge_{v\in\Vo'} (\,\square \,\diamondsuit\, v \implies \bigwedge_{(v,w)\in E'}\square\, \diamondsuit\, (\,v \wedge \bigcirc \,w\,)).\label{equ:alpha:b}
\end{align}
\end{subequations}
\end{definition}

Intuitively, for all \Odd vertices in $\mathcal{S}$, the strategy $\rho$ compliant with $\mathcal{S}$ takes only their outgoing edges in $\mathcal{S}$ \eqref{equ:alpha:a}, and if a play visits an \Odd node $v$ infinitely often, then $\rho$ takes each of $v$'s outgoing edges in $\mathcal{S}$ infinitely often \eqref{equ:alpha:b}.
% 
For an \Odd strategy template $\mathcal{S}$, if $v \in V'_\Odd$ lies on a cycle in $\mathcal{S}$, then by Def. \ref{def:Oddstrategytemplate}, $\mathcal{S}$ contains all live outgoing edges of $v$. By \eqref{equ:alpha:b} any \Odd strategy $\rho$ compliant with $\mathcal{S}$ satisfies the fairness condition in \eqref{eq:fairness-ltl} for $v$. 
On the other hand, if $v \in V'_\Odd$ does not lie on a cycle in $\mathcal{S}$, then by \eqref{equ:alpha:a} any such $\rho$ sees $v$ at most once. Thus $\rho$ trivially satisfies \eqref{eq:fairness-ltl} for $v$. 
This observation is stated in the following proposition.


\begin{proposition}
 Given the premisses of Def.~\ref{def:compliantstrat} let $\pi$ be a play starting from a node in $V'$ that complies with $\rho$. Then $\pi \models \alpha$ where $\alpha$ if the LTL formula in~\eqref{eq:fairness-ltl}.%\vspace{-2mm}
\end{proposition}

Next, we define \Even strategy templates. Each \Even strategy template encodes a unique \Even positional strategy, which is known to exist in \Odd-fair parity games \cite{banerjee2022fast}, due to the lack of fair edges defined on \Even vertices. %, \Even strategy templates are very simple\footnote{In fact, \Even strategy templates simply encode a positional strategy and are only re-defined to make further arguments more symmetric for both players.}.
\begin{definition}\label{def:Evenstrategytemplate}
    Given an \Odd-fair parity game $\mathcal{G}^\ell = \ltup{\mathcal{G}, E^\ell}$ with \newline $\mathcal{G} = \langle V, \Ve, \Vo, E, \chi\rangle$, an \Even \emph{strategy template} $\mathcal{S}$ over $\mathcal{G}^\ell$ is a subgraph of $\mathcal{G}$ given as $\mathcal{S}:=\tup{V', E'}$ where $V'\subseteq V$ and $E'\subseteq E \cap (V' \times V')$ such that,    \begin{compactitem}\label{item:Evenstrtemprules}
     \item if $v \in \Ve \cap V'$, then $|E'(v)|=1$,
     \item if $v \in \Vo \cap V'$, then  $E'(v) = E(v)$.
    \end{compactitem}
\end{definition}

\vspace*{-0.1cm}

An \Even strategy $\rho$ is compliant with the \Even strategy template $\mathcal{S} = \tup{V', E'}$ if for all $v \in V'_\Even$, $\rho(v) = E'(v)$. In other words, $\rho$ is the positional strategy defined by $\mathcal{S}$.

Let $\rho$ be an \Odd (\Even) strategy, compliant with the \Odd (\Even) strategy template $\mathcal{S}$ and let $\pi$ be a play compliant with $\rho$. Then we call $\pi$ a play \emph{compliant with $\mathcal{S}$}.

\vspace*{-0.1cm}

\begin{definition}
An \Odd (\Even) strategy template $\mathcal{S}=\ltup{V', E'}$ is \emph{winning} in the \Odd-fair parity game $\mathcal{G}^\ell$ if all \Odd (\Even) strategies $\rho$ compliant with $\mathcal{S}$ are winning for player \Odd (\Even) in $\mathcal{G}^\ell$ from $V'$. A winning \Odd (\Even) strategy template $\mathcal{S}$ is called \emph{maximal} if $V'=\Wo$ ($\We$).%\vspace{-2mm}
\end{definition}

\vspace*{-0.2cm}
We note that maximal winning \Odd (\Even) strategy templates $\mathcal{S}$ immediately imply that for every vertex $v\in \Wo$ ($\We$) there exists a winning strategy for player \Odd (\Even) from $v$ that is compliant with $\mathcal{S}$.
% 
The existence of maximal winning \Even strategy templates follows from the existence of positional \Even strategies \cite{banerjee2022fast}. 
% 
The first main contribution of this paper is a constructive proof showing the existence of maximal winning \Odd strategy templates given in the next section. 
This result is then used in Sec.~\ref{sec:zielonka} to prove the correctness of \Odd-fair Zielonka's algorithm, which is introduced there.





\section{Experimental Results}\label{sec:results}
    \subsection{General Results}
        The basic ResSAN model is used to determine reference results which our expanded model can be compared to as it is structurally similar to ResLAN but does not possess the Lidar adaptive components of it. Further, we compare with the full-size PackNet-SAN and the unmodified NLSPN architecture. 
        As it can be seen from Tab.\,\ref{tab:sota-results}, our LiDAR-adaptive ResLAN achieves competitive performance compared to state-of-the-art standard depth completion methods, which are specialized to the unfiltered 64-beam-LiDAR. The performance differences are in the range of a few centimetres in terms of MAE, which is acceptable given the practical advantage that ResLAN can generalize to different beam patterns as will be shown below.

        Furthermore, we compared the architectures for a set of three different input types that contained 64, 32 or 16 LiDAR channels using both filter types on the metrics from the KITTI benchmark. The NLSPN model was trained for the standard depth completion task and then evaluated with different input data. As for the ResSAN models, we trained one model for each input type and tested it for the corresponding one which serve serve as the \emph{Baseline} in Tab.\,\ref{tab:overall-results}. Our ResLAN model was jointly trained for all three settings. As listed in Tab.\,\ref{tab:overall-results}, the ResLAN models outperform the challenging baseline in all metrics for FOV filtering and all but one for sparse filtering. This implies that our LiDAR adaptive model is able to outperform dedicated models in case of very sparse input depth. Fig.\,\ref{fig:comp-plot} shows this is indeed the case for 32 and even more for 16 channels. For FOV-filtered inputs with 16 channels, the ResLAN exhibits approx. $10\%$ smaller MAE than the baseline. As for the NLSPN, it becomes apparent that it is not capable of generalizing to other input types since it shows clearly worse results. The difference is especially pronounced for the FOV filtering where on average more than every fourth predicted pixel is more than $25 \%$ deviating from the ground truth\,($\delta_{1.25}$). Therefore, using a weight-adapting network in combination with differently filtered input depths allows us to train models that outperform their non-adaptive counterparts.

        \begin{table}[]
            \centering
    	    \small
            \vspace{0.4cm}
            \caption{\textbf{Depth estimation result for standard depth completion} when the ResSAN model was only trained for 64 channels and the ResLAN model for multiple tasks. The PackNet-SAN and NLSPN models were trained with the setup that was also used for our model architecture.}
            \footnotesize
            \setlength{\tabcolsep}{5pt}
            \begin{tabular}{@{}lrrrrl@{}}
            \toprule
            \multicolumn{6}{c}{\textbf{Standard LiDAR Depth Completion}}                                                                                                                         \\ \midrule
            \multicolumn{1}{l|}{Method}          & RMSE $\downarrow$            & MAE  $\downarrow$            & iRMSE $\downarrow$             & iMAE $\downarrow$ & $\delta_{1.25}$ $\uparrow$ \\
            \multicolumn{1}{l|}{}                & \multicolumn{1}{l}{{[}mm{]}} & \multicolumn{1}{l}{{[}mm{]}} & \multicolumn{1}{l}{{[}1/km{]}} & {[}1/km{]}        &                            \\ \midrule
            \multicolumn{1}{l|}{PackNet-SAN}     &  914                            &  298                            &  2.78                              &  1.4                 &  99.65 \%                          \\
            \multicolumn{1}{l|}{NLSPN}           &  \textbf{889}                            &   \textbf{263}                           &  \textbf{2.62}                              &   \textbf{1.3}                &   \textbf{99.61} \%                         \\ \midrule
            \multicolumn{1}{l|}{ResSAN (Ours)}   & 948                             &  275                            &  2.75                              &    1.4               &   99.58 \%                         \\
            \multicolumn{1}{l|}{ResLAN (Ours)} &   969                           &  283                            &   2.83                             &   1.4                &  99.56 \%                          \\ \bottomrule
            \end{tabular}
            \vspace{0.2cm}
            \label{tab:sota-results}
        \end{table}

        \begin{table}[]
    	    \centering
    	    \small
    	    \caption{\textbf{Depth estimation results of the two baseline setups and the explicit and implicit ResSAN} when evaluated on a combination of 16, 32 and 64 channel depth inputs. Please note that Specialist Methods need to train three specialized networks, one for each of the three types of inputs while our method only uses one network.}
            \footnotesize
            \setlength{\tabcolsep}{4.8pt}
            \begin{tabular}{@{}lrrrrl@{}}
                \toprule
                \multicolumn{6}{c}{\textbf{Sparse Channel Filter}}                                                                                                                                  \\ \midrule
                \multicolumn{1}{l|}{Method}        & RMSE $\downarrow$            & MAE  $\downarrow$            & iRMSE $\downarrow$             & iMAE $\downarrow$ & $\delta_{1.25}$ $\uparrow$  \\
                \multicolumn{1}{l|}{}              & \multicolumn{1}{l}{{[}mm{]}} & \multicolumn{1}{l}{{[}mm{]}} & \multicolumn{1}{l}{{[}1/km{]}} & {[}1/km{]}        &                             \\ \midrule
                \multicolumn{1}{l|}{NLSPN}         &  1396                            &  437                            & 5.54                               &  2.2                 &  98.82 \%                           \\
                \multicolumn{1}{l|}{Baseline}      & \textbf{1207}                             &  381                            & 4.41                               &  1.8                 &  \textbf{99.37} \%                           \\
                \multicolumn{1}{l|}{ResLAN (Ours)} &  1215                            &  \textbf{378}                            &  \textbf{4.27}                              &  \textbf{1.7}                 &  99.31 \%                           \\ \toprule
                \multicolumn{6}{c}{\textbf{Field-of-View Filter}}                                                                                                                                   \\ \midrule
                \multicolumn{1}{l|}{Method}        & RMSE $\downarrow$            & MAE  $\downarrow$            & iRMSE $\downarrow$             & iMAE $\downarrow$ & $\delta_{1.25}$ $\uparrow$ \\
                \multicolumn{1}{l|}{}              & \multicolumn{1}{l}{{[}mm{]}} & \multicolumn{1}{l}{{[}mm{]}} & \multicolumn{1}{l}{{[}1/km{]}} & {[}1/km{]}        &                             \\ \midrule
                \multicolumn{1}{l|}{NLSPN}         &  2738                            &  1702                            & 12.3                              &  4.3                 &  74.69 \%                           \\
                \multicolumn{1}{l|}{Baseline}      &  1556                            &  525                            &  6.8                              &  3.0                 & 98.14 \%                            \\
                \multicolumn{1}{l|}{ResLAN (Ours)} &  \textbf{1548}                            &  \textbf{519}                            &  \textbf{6.44}                              &  \textbf{2.8}                 & \textbf{98.52 \%}                            \\ \bottomrule
            \end{tabular}
            \label{tab:overall-results}
        \end{table}

        
        
        % Figure environment removed
        
        % Figure environment removed

    \subsection{Filter Effects}
        Comparing the effect of the two different types of depth input filters on the model performance, it becomes apparent that FOV filtering is the more challenging task. In that setting, reducing LiDAR channels is more detrimental to the performance than sparse filtering as it creates regions where no depth information is available. Effectively, the model is forced to perform depth prediction in these regions. These effects are highlighted in the depth images in Fig.\,\ref{fig:dense-maps} where the effect of a 16-channel sparse depth filter and a 16-channel FOV can be compared.

    \subsection{Generalization Capabilities}
        We trained three models for both filter types eaach, so the combinations and number of filtered depth inputs they receive are different. This serves the purpose of testing the generalization capabilities of the ResLAN architecture as well as the robustness to different filter settings. After training, the models were evaluated for the depth input settings they were trained for, as well as for ones they weren't exposed to. Overall, ResLAN shows good generalization capabilities. As one can gather from Fig.\,\ref{fig:explicit-comp} and Fig.\,\ref{fig:implicit-comp}, the consequences of slightly varying sets of input depth settings are limited. The most considerable deviations can be seen when the model is tasked to extrapolate. For instance, the model $\{64, 32, 16\}$ shows a noticeably higher MAE for eight-channel depth inputs than the model that was trained for it. Similar behaviour can be seen for the FOV filtering case as well for the model $\{64, 48, 32\}$ when tasked to generalize for a 16-channel input. There is no such pronounced effect for generalization tasks that lie between two filter settings the model was trained for. At most, it can be observed that models that were trained for a smaller range of filter values perform slightly better than ones that have to cover a wider range. The number of filter settings used in a fixed range does not relevantly influence the model performance, as can be seen, when comparing the two models in Fig.\,\ref{fig:implicit-comp}, which are both trained for a range of 64 to 32 channels but one with three filter settings and the other one with five.
    
    % Figure environment removed
    
    
    % Figure environment removed

%%%%%%%%%%%%%%%%%%%%%%%%%%%%%%%%%%%%%%%%%%%%%%%%%%%%%%%%%%%%%%%%%%%%%%%%%%%%%%%%%
%\newpage
%
%\section{Template (to be removed)}
%
%{\color{red} 
%
%\begin{itemize}[label=\(\circ\), leftmargin=*]
%	\item Use the macro \texttt{\textbackslash{}pmat} for matrices: check the style file for the syntax.
%
%	\item Never use vertical bars in tables unless absolutely necessary --- \textbf{vertical bars are for prisons}. Here's one example lifted from the numerical section:
%		\begin{table}[H]
%		    \begin{center}
%		    \renewcommand{\arraystretch}{1.5}
%			    \begin{tabular}{cl}
%				    \toprule
%					$\damp$ & Lyapunov Function $x\mapsto\lyapfn(x)$ \\
%					\midrule
%				    1 & $0.7819\state_1^2 - 0.3008\state_1\state_2 + 0.6712\state_2^2$ \\
%				    2 & $0.8149\state_1^2 -0.2985\state_1\state_2 + 0.6150\state_2^2$ \\
%				    3 & $0.8336\state_1^2 -0.2966\state_1\state_2 + 0.5640\state_2^2$ \\
%				    \bottomrule
%			    \end{tabular}
%			\end{center}
%			\caption{\label{table_pendulum} Lyapunov function $x\mapsto\lyapfn(x)$ for different values of $\damp$ for damped pendulum }
%		\end{table}
%
%		Here's a template for such tables from the \texttt{booktabs} package:\\
%		\begin{center}
%		\begin{tabular}{@{}llr@{}} \toprule
%			\multicolumn{2}{c}{Item} \\ \cmidrule(r){1-2}
%			Animal & Description & Price (\$)\\ \midrule
%			Gnat & per gram & 13.65 \\
%			& each & 0.01 \\
%			Gnu & stuffed & 92.50 \\
%			Emu & stuffed & 33.33 \\
%			Armadillo & frozen & 8.99 \\ \bottomrule
%		\end{tabular}
%		\end{center}
%
%	\item Use the macro \texttt{\textbackslash{}inprod} to match vertical spacing.
%	\item Use the commands of the package \texttt{derivative} to stay consistent.
%	\item Don't be sloppy: \(V(x)\) is the value of \(V\) at \(x\), let's write either \(x\mapsto V(x)\) or \(V(\cdot)\) for functions. Match parentheses according to the vertical extent of their contents, but don't blindly employ \texttt{\textbackslash{}left(, \textbackslash{}right)}, etc. Check how the previous sections are written. Each sentence should terminate with a period: `.'
%	\item Numbers and legends in the figures should be \textbf{eminently} readable.
%	\item ``Van der Pol'' \(\to\) ``van der Pol''.
%	\item There should \textbf{never} be a space \emph{before} a colon `:', but a spaace \emph{after} a colon should \textbf{never} be omitted.
%
%	\item Each example should begin with a description of the system under consideration and then a selection of a candidate Lyapunov triplet. Refer to the appropriate sections as needed. Then present the outcome of your numerical experiments in a scientific way; provide ``proof'' of convergence --- several sample paths of the annealing algorithm would suffice. Then provide a figure of the constructed Lyapunov function in 3D (to be done).
%\end{itemize}
%
%}
%
%\newpage
%%%%%%%%%%%%%%%%%%%%%%%%%%%%%%%%%%%%%%%%%%%%%%%%%%%%%%%%%%%%%%%%%%%%%%%%%%%%%%%%%

%% ============================================================================
\section{Numerical experiments}
\label{s:numerics}
%% ============================================================================

This section presents a number of numerical experiments to illustrate the technique developed above. A few points concerning these experiments:
\begin{enumerate}[label=(\roman*), align=right, leftmargin=*, widest=iii]
    \item For the purpose of the numerical experiments, we choose the convex quadratic objective function \(\R[\basisDim]\ni z\mapsto \objective(z) \Let \norm{z - \mathbf{1}}^2\), where \(\mathbf{1} \in \R[\basisDim]\) is a vector of all ones
    \item In our simulations, we excluded the function \(\upperBound(\cdot)\) --- serving as an upper bound for the Lyapunov functions to be constructed --- from our consideration; since the number of elements in our dictionaries is finite, the decrescent property of any Lyapunov function follows immediately.
    \item In each experiment, we validated the output of our algorithm by numerically computing the minimum values of the two functions
		\[
			y\mapsto \constraintA(y) \Let \lyapfn(y) - \lowerBound(\norm{y}) \quad \text{and}  \quad y\mapsto \constraintB(y) \Let -\stabilityMargin(\ypos) - \inprod{\pdv{\lyapfn}{x}(y)}{\vecfld(y)}
		\]
		over the domain under consideration, and they were confirmed to be non-negative.
	\item \texttt{python 3.10.4} was employed with the support of the libraries \texttt{numpy 1.24.3}, \texttt{scipy 1.8.1}, and \texttt{matplotlib 3.7.1}.
		\begin{itemize}[label=\(\circ\), leftmargin=*]
			\item The `inner' convex optimization problem in each case was solved by employing the \texttt{scipy.optimize.minimize()} function with the \texttt{``SLSQP''} method as an input parameter and \texttt{max iterations = 10000}.
			\item The `outer' (global) maximization problem in each case was solved by employing the \texttt{scipy.optimize.dual\_annealing()} function with \texttt{max iterations = 30}, and a vector (of appropriate dimension) of all entries equal to \(0.5\) as the initial starting point.
			\item In each case, the outcome of the MSA algorithm (which produces a Lyapunov function) was validated against the selected candidate Lyapunov triplet with the aid of the function \texttt{scipy.optimize.differential\_evolution()} together with its default parameters.
		\end{itemize}
\end{enumerate}


%% -------------------------------------------------------------------------
\subsection{A nonlinear planar vector field}
%% -------------------------------------------------------------------------

Consider a nonlinear 2-d system given by
\begin{equation}
    \label{e:random}
    \begin{aligned}
        & \dot{\state_{1}} = 2\state_1\left(1-\frac{\state_1}{2}\right)-\state_1\state_{2}, \quad \\
        & \dot{\state_{2}} = 3\state_2\left(1-\frac{\state_2}{3}\right)-2\state_1\state_{2}.
    \end{aligned}
    \end{equation}
It can be found that \eqref{e:random} has three equilibrium points, out of which the system is asymptotically stable around the equilibrium points \((2,0)\) and \((0,3)\). A circular disc of radius \(0.2\) centered at the equilibria is chosen as the domain of investigation for the equilibrium points. After shifting the origin to the respective equilibrium points, the following Lyapunov triplets were utilized for both points.

\subsubsection*{Candidate Lyapunov triplet for our experiment}
Our selections were as follows:
\begin{description}
	\item[\ref{d:nbhd}]  \(\nbhd  \Let \set[\big]{ (\state_1, \state_2) \in \R[2] \suchthat \state_1^2 + \state_2^2  \leq 0.04 }\).
	\item[\ref{d:pdf}] \(\lowerBound, \marginBound \in \classK\) were picked as \(\lowerBound(\radius) = \frac{\radius^2}{6}\) and \(\marginBound\left(\radius\right) =  \frac{\radius^2}{12}\) for \(\radius \ge 0\).
	\item[\ref{d:dictionary}] The dictionaries were selected to be:
		\begin{align*}
			\basisDict & \Let \set[\big]{ x_1^{i_1} \cdot x_2^{i_2} \suchthat (i_1, i_2) \in\Nz[2], i_1 + i_2  = 2 },\\ 
			\marginDict & \Let \set[\big]{ x_1^{i_1} \cdot x_2^{i_2} \suchthat (i_1, i_2) \in\Nz[2], i_1 + i_2  = 2, 4}.
		\end{align*}
\end{description}

The results of our procedure are collected in Table \ref{table_2DField}.
\begin{table}[tbh]
    \begin{center}
    \renewcommand{\arraystretch}{1.5}
    \begin{tabular}{cl}
    \toprule
    Equilibrium Point & Lyapunov Function $\state\mapsto\lyapfn(\state)$ \\ \midrule
    (2,0) & ${0.981\state_1}^2 + 0.922\state_1\state_2+{1.083\state_2}^2 -3.924\state_1-1.844\state_2+3.924$\\
    (0,3) & $1.316\state_1^2 + 0.261\state_1\state_2+
    0.336\state_2^2- 0.783\state_1-2.016\state_2+3.024$ \\
    \bottomrule
    \end{tabular}
		\caption{\label{table_2DField}
			Lyapunov functions $x\mapsto\lyapfn(x)$ for the two equilibrium points corresponding to \eqref{e:random} in the \emph{original space coordinates}.}
    \end{center}
\end{table}

% Figure environment removed
% Figure environment removed

% Figure environment removed

% Figure environment removed


%% -----------------------------------------------------------------------------
\subsection{The van der Pol oscillator}
%% -----------------------------------------------------------------------------

We consider a van der Pol oscillator described by
\begin{equation}
\label{e:vanderpol}
\begin{aligned}
    & \dot\state_{1} = \state_{2}, \quad \\
    & \dot\state_{2} = -\state_1 + \eps\state_2(1-\state_1^2),
\end{aligned}
\end{equation}
where \(\eps\in\R[]\) is a parameter. Observe that the stability of the above system depends on \(\eps\). To avoid an unstable limit cycle around the equilibrium point (origin), the domain is kept to be a circular disc with a radius of \(0.5\).

\subsubsection*{Candidate Lyapunov triplet for our experiment}
Our selections were as follows:
\begin{description}
	\item[\ref{d:nbhd}] \(\nbhd  \Let \set[\big]{ (\state_1, \state_2) \in \R[2] \suchthat \state_1^2 + \state_2^2  \leq 0.25 }\).
	\item[\ref{d:pdf}] \(\lowerBound, \marginBound \in \classK\) were picked as \(\lowerBound\left(\radius\right) = \frac{\radius^3}{2}\) and \(\marginBound\left(\radius\right) =  \frac{\radius^{10}}{4}\) for \(\radius \ge 0\).
	\item[\ref{d:dictionary}] The dictionaries were selected to be:
		\begin{align*}
			\basisDict & \Let \set[\big]{ x_1^{i_1} \cdot x_2^{i_2} \suchthat (i_1, i_2) \in\Nz[2],  i_1 + i_2  = 2 },\\ 
			\marginDict & \Let \set[\big]{ x_1^{i_1} \cdot x_2^{i_2} \suchthat (i_1, i_2) \in\Nz[2], i_1 + i_2  = 2, 4, \cdots, 12 }.
		\end{align*}
\end{description}
The Lyapunov function obtained from our numerical procedure was: 
\[
	\state\mapsto\lyapfn(\state) = 1.106\state_1^2 + 0.380\state_1\state_2+ 1.106\state_2^2
\]
with the value of \(\eps= -2\) for which the vector field is asymptotically stable, which happens to be the case for all \(\eps <  0\).

% Figure environment removed
% Figure environment removed

% Figure environment removed



%%--------------------------------------------------------------------------------
\subsection{A whirling pendulum}
%%----------------------------------------------------------------
Consider the whirling pendulum system lifted from \cite[Example 3]{ref:PapPra-02} and described by
\begin{equation}
    \label{e: whirl pendulum}
    \begin{aligned}
        & \dot{\state_{1}} = \state_{2},\\
        & \dot{\state_{2}} = {\dot{\theta}^2}_a\sin(\state_1)\cos(\state_1) - \frac{g}{l_a}\sin(\state_1).
    \end{aligned}
\end{equation}
The system \eqref{e: whirl pendulum} is found to be Lyapunov stable when the condition 
\begin{equation}
    \label{e:whirl pendulum condition}
	{\dot{\theta}^2}_a <\frac{g}{l_a}
\end{equation}
is met and is otherwise unstable. \cite{ref:PapPra-02} provides a Lyapunov function for the case of the above whirling pendulum system using \texttt{SOSTOOLS} after introducing two more variables to the system: $\wpstate_{1}=\sin(\state_{1})$ and $\wpstate_{2}=\cos(\state_{2})$, which transforms \eqref{e: whirl pendulum} into a 4-d polynomial vector field, and obtained a Lyapunov function of the form \(\state\mapsto a_1\state_{2}^2 + a_2\wpstate_{1}^{2} + a_3\wpstate_{2}^{2}+a_4\wpstate_{2}+a_5\). 

We now demonstrate the ability of our method to construct a Lyapunov function of a form similar to the above but without any need for  such transformations. We carried out our simulations with the numerical values of different parameters present in Table \ref{table_whirlpendulum}.
\begin{table}[tbh]
		    \begin{center}
		    \renewcommand{\arraystretch}{1.5}
			    \begin{tabular}{cc}
				    \toprule
					Parameter & Numerical Value \\
					\midrule
				    ${\dot{\theta}^2}_a$ & 1 \\
                        $\frac{g}{l_a}$ & 10\\
				    \bottomrule
			    \end{tabular}
			\end{center}
			\caption{\label{table_whirlpendulum} Parameters for the damped whirling pendulum \eqref{e: whirl pendulum}. }
\end{table}
The flexibility of our method is demonstrated in this example by the selection of both polynomial and non-polynomial functions in the dictionary for \(V(\cdot)\).

\subsubsection*{Candidate Lyapunov triplet for our experiment}

Our selections were as follows:
\begin{description}
	\item[\ref{d:nbhd}]  \(\nbhd  \Let \set[\big]{ (\state_1, \state_2) \in \R[2] \suchthat \state_1^2 + \state_2^2  \leq 1 }\).
	\item[\ref{d:pdf}] \(\lowerBound, \marginBound \in \classK\) were picked as \(\lowerBound\left(\radius\right) = \frac{\radius^2}{100}\) and \(\marginBound\left(\radius\right) =  0\) for \(\radius \ge 0\).
	\item[\ref{d:dictionary}] The dictionaries were selected to be:
		\begin{align*}
			\basisDict & \Let \set[\big]{ x_1^{i_1} \cdot x_2^{i_2} \suchthat (i_1, i_2) \in\Nz[2], i_1 + i_2  = 2 } \cup \set[\big]{ \cos\left(j_{1}x_1\right) \suchthat 0 \leq j_1 \leq 2 } ,\\ 
			\marginDict & \Let \set[\big]{ x_1^{i_1} \cdot x_2^{i_2} \suchthat (i_1, i_2) \in\Nz[2], i_1 + i_2  = 2, 4, 6 }.
		\end{align*}
\end{description}
  
The Lyapunov function for the system \eqref{e: whirl pendulum}, with parameters in Table \ref{table_whirlpendulum} and the aforementioned Lyapunov triplet, obtained from our numerical procedure was: 
\[
	\state\mapsto\lyapfn(\state) = 0.050 \state_2^2 + 0.025\cos\left(2\state_1\right)-\cos\left(\state_1\right)+0.975.
\]
Figure \ref{fig: whirling_lyapunov} depicts this function on the unit box \(\lcrc{-1}{1}^2\) (a domain containing \(\nbhd\)).

% Figure environment removed
% Figure environment removed

% Figure environment removed

%% -----------------------------------------------------------------------------
\subsection{A 5-d hyperchaotic system with linear feedback controller}
%% -----------------------------------------------------------------------------
For our next example, we introduce an illustration of a 5-d autonomous hyperchaotic system, as proposed by \cite{ref:Niu2021}. The system incorporates a linear feedback controller, effectively achieving global asymptotic stability at the origin. The system can be expressed as follows:
\begin{equation}
    \label{e:5d-system}
    \begin{aligned}
		\dot \state_{1} & = a(\state_{2} - \state_{1}) -k_1\state_1, \quad \\
		\dot \state_{2} & = (c-a)\state_1 + c\state_2 + \state_5 - \state_1\state_3 - k_2\state_2, \quad \\
		\dot \state_{3} & = -b\state_3 + \state_1\state_2 - k_3\state_3, \quad \\
		\dot \state_{4} & = m\state_5 - k_4\state_4, \quad \\
		\dot \state_{5} & = -\state_2 - h\state_4 - k_5\state_5
    \end{aligned}
\end{equation}
where \((k_{1}, k_{2}, k_{3}, k_{4},k_{5}) = (0, 30, 0, 1, 1)\). We keep \(a = 23\), \(b = 3\), \(c = 18\), \(m = 12\), and \(h = 4\) as recommended in \cite{ref:Niu2021} to obtain a global asymptotic stability at origin.

\subsubsection*{Candidate Lyapunov triplet for our experiment}
Our selections were as follows:
\begin{description}
	\item[\ref{d:nbhd}]  \(\nbhd  \Let \set[\big]{ (\state_1, \state_2, \state_3, \state_4, \state_5) \in \R[5] \suchthat \state_1, \state_2, \state_3, \state_4, \state_5 \in \lcrc{-0.5}{0.5} }\).
	\item[\ref{d:pdf}] \(\lowerBound, \marginBound \in \classK\) were picked as \(\lowerBound\left(\radius\right) = \frac{\radius^2}{100}\), and \(\marginBound\left(\radius\right) =  \frac{\radius^4}{20000}\) for \(\radius \ge 0\).
	\item[\ref{d:dictionary}] The dictionaries were selected to be:
		\begin{align*}
			\basisDict & \Let \set[\big]{ x_1^{i_1} \cdot x_2^{i_2} \cdots x_5^{i_5} \suchthat (i_1, i_2, i_3, i_4, i_5) \in\Nz[5], i_1 + i_2 + i_3 + i_4 + i_5 = 2 },\\ 
			\marginDict & \Let \set[\big]{ x_1^{i_1} \cdot x_2^{i_2} \cdots x_5^{i_5} \suchthat (i_1, i_2, i_3, i_4, i_5) \in\Nz[5], i_1 + i_2 + i_3 + i_4 + i_5  = 2, 4, 6 }.
		\end{align*}
\end{description}
Our numerical procedure led to the Lyapunov function%contained in Table \ref{table-5D}.
\begin{align*}
	\state\mapsto \lyapfn(\state) & = 1.229\state_1^2 + 0.982\state_1\state_2 + 0.891\state_1\state_3 + 0.632\state_1\state_4 + 0.236\state_1\state_5\\
	& \qquad + 0.996\state_2^2+ 1.026\state_2\state_3 + 0.663\state_2\state_4 + 0.928\state_2\state_5 + 1.116\state_3^2\\
	& \qquad + 0.406\state_3\state_4 + 1.054\state_3\state_5 + 0.523\state_4^2 + 0.158\state_4\state_5 + 1.595\state_5^2.
\end{align*}

% Figure environment removed

% Figure environment removed

%% -----------------------------------------------------------------------------
\subsection{Transient stability of classical power system models}
%% -----------------------------------------------------------------------------

For our next example, we pick a four-dimensional system from \cite{ref:AngMilPap-13} given by the equation (transformed to have the origin as its equilibrium point):
\begin{equation}
  \label{eq:power_model}
  \begin{aligned}
	  \dot \state_{1} & = \state_{2}, \quad \\
	  \dot \state_{2} & = 0.0200\cos(\state_1)\cos(\state_3) - 0.0200\cos(\state_1) - 0.9998\sin(\state_1) - 0.4000\state_2  \\
	  & \qquad + 0.4996\cos(\state_1)\sin(\state_3) - 0.4996\cos(\state_3)\sin(\state_1) + 0.0200\sin(\state_1)\sin(\state_3), \\
	  \dot \state_{3} & = \state_{4}, \\
	  \dot \state_{4} & = 0.4996\cos(\state_3)\sin(\state_1) - 0.0299\cos(\state_3) - 0.4991\sin(\state_3) \\
    & \qquad -0.0200\cos(\state_1)\cos(\state_3) - 
    0.4996\cos(\state_1)\sin(\state_3) - 0.5000\state_4 \\
    & \qquad - 0.0200\sin(\state_1)\sin(\state_3) + 0.0500.
    \end{aligned}
\end{equation}

\subsubsection*{Candidate Lyapunov triplet for our experiment}
    Our selections were as follows:
\begin{description}
	\item[\ref{d:nbhd}]  \(\nbhd  \Let \set[\big]{ (\state_1, \state_2, \state_3, \state_4) \in \R[4] \suchthat \state_1, \state_2, \state_3, \state_4 \in \lcrc{-0.2}{0.2} }\).
	\item[\ref{d:pdf}] \(\lowerBound \in \classK\): \(\lowerBound\left(\radius\right) = \frac{\radius^2}{16}\) and \(\marginBound\left(\radius\right) =  \frac{\radius^2}{200}\) for \(\radius \ge 0\).
	\item[\ref{d:dictionary}] The dictionaries were selected to be:
		\begin{align*}
			\basisDict & \Let \set[\big]{ x_1^{i_1} \cdot x_2^{i_2} \cdot x_3^{i_3} \cdot x_4^{i_4} \suchthat (i_1, i_2, i_3, i_4) \in\Nz[4],  i_1 + i_2 + i_3 + i_4 = 2 },\\ 
			\marginDict & \Let \set[\big]{ x_1^{i_1} \cdot x_2^{i_2} \cdot x_3^{i_3} \cdot x_4^{i_4} \suchthat (i_1, i_2, i_3, i_4) \in\Nz[4], i_1 + i_2 + i_3 + i_4  = 2, 4}.
		\end{align*}
\end{description} 
This example demonstrates how our algorithm effectively handles higher order systems, enabling the identification of an appropriate Lyapunov function with predefined lower bounds. Our numerical procedure led to the Lyapunov function %contained in Table \ref{table_powersys}.
\begin{align*}
	\state\mapsto\lyapfn(\state) & = 1.339\state_1^2 + 0.591\state_1\state_2 + 0.635\state_1\state_3 + 0.622\state_1\state_4 + 0.989\state_2^2\\
	& \qquad + 0.651\state_2\state_3 + 1.063\state_2\state_4 + 1.064\state_3^2 + 0.970\state_3\state_4 + 1.297\state_4^2.
\end{align*}

% Figure environment removed

% Figure environment removed


%% -----------------------------------------------------------------------------
\subsection{Flexibility of basis}
%% -----------------------------------------------------------------------------

We present a final example illustrating an interesting situation in which the vector field is defined by means of a case statement. This example is lifted from \cite[p. 68]{ref:BhaSze-70}, and the dynamics is given by
\begin{equation}
  \label{eq:bhatia's}
  \begin{aligned}
	  \dot\state_{1} & =
	  \begin{cases}
		  \state_1 & \text{if }\state_{1}^2\state_{2}^2 \geq 1,\\
				2\state_{1}^3\state_{2}^2 - \state_{1} & \text{if } \state_{1}^2\state_{2}^2 < 1,
	\end{cases}\\
	\dot\state_{2} & = -\state_{2}.
\end{aligned}
\end{equation}
Notice that the vector field \eqref{eq:bhatia's} is continuous, and the neighborhood \(\nbhd\) selected above contains both the regions differentiated by the case statement.

\subsubsection*{Candidate Lyapunov triplet for our experiment}
    Our selections were as follows:
\begin{description}
	\item[\ref{d:nbhd}]  \(\nbhd  \Let \set[\big]{ (\state_1, \state_2) \in \R[2] \suchthat \state_1, \state_2 \in [-4, 4] }\).
	\item[\ref{d:pdf}] \(\lowerBound, \marginBound \in \classK\) : \(\lowerBound\left(\radius\right) = 0\) and \(\marginBound\left(\radius\right) =  \frac{\radius^2}{2048}\) for \(\radius \ge 0\).
	\item[\ref{d:dictionary}] The dictionaries were selected to be:
		\begin{align*}
			\basisDict & \Let \set[\big]{ x_1^{i_1} \cdot x_2^{i_2} \suchthat (i_1, i_2) \in\Nz[2], 2 \leq i_1 + i_2  \leq 6 }, \\
				\marginDict & \Let \set[\big]{ x_1^{i_1} \cdot x_2^{i_2}  \suchthat (i_1, i_2) \in\Nz[2], i_1 + i_2 = 2, 4}.
		\end{align*}
\end{description}
In contrast to the rational polynomial functions mentioned in \cite{ref:BhaSze-70} corresponding to this example, we obtained a Lyapunov function within the class of polynomial functions itself without extending it to include rational functions. The outcome of our numerical experiment was the Lyapunov function
\[
	\state\mapsto\lyapfn(\state) =0.003\state_1^2 + 0.069\state_1\state_2 + 1.063\state_1^2.
\]

% Figure environment removed

% Figure environment removed

% Figure environment removed


%%==============================================================================
\subsection{Rayleigh problem with control constraints}
%%==============================================================================

In this example we address stability of an equilibrium point under \emph{continuous-time model predictive control}. It is well-known that discrete-time model predictive control (MPC) \cite{ref:XiLi-19} is a very powerful constrained control technique, and it consists of the following steps for a system of the form \(x^+ = f(x, u)\):
\begin{itemize}
	\item solve a constrained finite horizon optimal control problem for the given control system with the current state \(x\) as the initial condition, and extract the corresponding sequence of optimal control actions over the horizon,
	\item apply the entry corresponding to initial stage to obtain the next state \(x^+\), 
	\item increment the time by \(1\), and return back to the first step with \(x^+\) as the current state.
\end{itemize}
No analytical expression of the optimal control sequence is available, in general, due to the presence of constraints in the problem, but it is clear that the optimal control actions depend on the initial state \(x\); if the optimal control action is unique at the initial time, then it is a \emph{mapping} of the initial state.\footnote{This feature is exploited by the so-called \emph{explicit MPC} technique to numerically construct this implicit feedback.}

The idea of continuous-time MPC for a control system \(\dot x = f(x) + g(x) u\) follows the same route as outlined above, starting from a continuous-time finite horizon constrained control problem (given an initial condition) over control trajectories, and extracting the optimal control trajectory (assuming uniqueness). This trajectory parametrically depends on the initial condition, and therefore, the optimal control action \(u\opt(0)\) at the initial time \(0\) depends on the initial state \(x\); one denotes this dependence as \(u\opt(0; x)\). The closed-loop system under a continuous-time MPC controller, consequently, is \(\dot x = f(x) + g(x) u\opt(0; x)\); it is immediately clear that \(u\opt(0; \cdot)\) is a feedback. Solving continuous-time constrained optimal control problems is typically time consuming, and it is an extremely challenging matter to evaluate \(u\opt(0; x)\) instantaneously for the continuous-time model. However, if such an implementaion were possible, then \(u\opt(0; \cdot)\) would be evaluated by an oracle and, in general, would not admit an explicit formula.

This example studies stability of an equilibrium point of the nonlinear control system 
\[
	\pmat{\dot{x}_1(t)\\ \dot{x}_2(t)} = \pmat{x_2(t) \\ -x_1(t) + x_2(t)\bigl(1.40 - 0.14 x_2(t)^2\bigr)} + \pmat{0\\ 4} u(t)
\]
under a particular continuous-time MPC strategy; this system features in the benchmark Rayleigh's problem \cite[Example 4.6]{ref:Bet-10}, and for our purposes it will be accompanied by control constraints over a finite horizon. Note that the vector field of the nonlinear system vanishes at the triplet \((x_1, x_2, u) = (0, 0, 0)\). Our intention is to stabilize the origin \((x_1, x_2) = (0, 0)\) under continuous-time MPC in closed-loop, and with this in mind, the following underlying continuous-time optimal control problem for MPC was formulated:
\begin{equation}
\label{eq:quito}
\begin{aligned}
	\minimize_{u:\lcrc{0}{\tfin}\to\R[]} \quad & \int_{0}^{\tfin} \bigl(u(t)^2 + x_1(t)^2\bigr) \, \odif{t} \\
\sbjto \quad &
\begin{cases}
\dot{x}_1(t) = x_2(t), \\
\dot{x}_2(t) = -x_1(t) + x_2(t)\bigl(1.40 - 0.14 x_2(t)^2\bigr) + 4u(t),\\
	\bigl(x_1(0), x_2(0)\bigr) = (a_1, a_2) \text{ (given)}, \\
	\bigl(x_1(\tfin), x_2(\tfin)\bigr) = (0, 0), \\
	|u(t)| \leq 1.
\end{cases}
\end{aligned}
\end{equation}
We picked the horizon \(\tfin = 4.5\). The optimal trajectory construction algorithm QuITO (Quasi Interpolation based Trajectory Optimization) developed in \cite{ref:Ganguly2022} was employed to solve the optimal control problem \eqref{eq:quito} and obtain optimal control trajectories \(\lcrc{0}{\tfin}\ni t\mapsto u\opt\bigl(t; (a_1, a_2)\bigr)\in\R[]\) for each initial condition \((a_1, a_2)\) sufficiently close to \((0, 0)\). The step size for the uniform cardinal grid on \(\lcrc{0}{\tfin}\) was picked to be \(h = 0.09\), and the interpolating Schwarz function was chosen to be \(\R[]\ni t\mapsto \psi(t) = \frac{1}{\sqrt{\pi\mathcal D}} \epower{-\frac{t^2}{h^2 \mathcal D}}\) with the shape parameter \(\mathcal D = 5\), both of which were input parameters to QuITO.% and a quintic spline-based interpolation for the off-grid values of time was carried out. The parameters were chosen such that the error between the optimal solution and the one from QuITO was within \(\pm 0.005\%\).\todo[fancyline]{Are you sure about this figure?}

The resulting closed-loop control system was, naturally,
\begin{equation}
	\label{e:clquito}
	\pmat{\dot{x}_1(t)\\ \dot{x}_2(t)} = \pmat{x_2(t) \\ -x_1(t) + x_2(t)\bigl(1.40 - 0.14 x_2(t)^2\bigr)} + \pmat{0\\ 4} u\opt\bigl(0; x_1(t), x_2(t)\bigr).
\end{equation}
We highlight that the right-hand side of \eqref{e:clquito} lacks an analytical expression although off-the-shelf results may be employed to check the mapping \(x\mapsto u\opt(0; x)\) is continuous and it evaluates to \(0\) at \(x = (0, 0)\). This means \((0, 0)\) is an equilibrium point of the preceding closed-loop system; let us check whether it is asymptotically stable.

\subsubsection*{Candidate Lyapunov triplet for our experiment}
Our selections were as follows:
\begin{description}
	\item[\ref{d:nbhd}]  \(\nbhd  \Let \set[\big]{ (\state_1, \state_2) \in \R[2] \suchthat \state_1^2 + \state_2^2  \leq 0.2025}\).
	\item[\ref{d:pdf}] \(\lowerBound, \marginBound \in \classK\) were picked as \(\lowerBound\left(\radius\right) = \frac{\radius^2}{4}\) and \(\marginBound\left(\radius\right) =  \frac{\radius^2}{32}\) for \(\radius \ge 0\).
	\item[\ref{d:dictionary}] The dictionaries were selected to be:
		\begin{align*}
			\basisDict & \Let \set[\big]{ x_1^{i_1} \cdot x_2^{i_2} \suchthat (i_1, i_2) \in\Nz[2], i_1 + i_2  = 2 }, \\ 
			\marginDict & \Let \set[\big]{ x_1^{i_1} \cdot x_2^{i_2} \suchthat (i_1, i_2) \in\Nz[2], i_1 + i_2  = 2, 4, 6 }.
		\end{align*}
\end{description}
The Lyapunov function obtained from our numerical procedure was
\[
	\state\mapsto\lyapfn(\state) = 1.216\state_1^2 + 0.688\state_1\state_2+ 0.948\state_2^2,
\]
which shows that the equilibrium point \((0, 0)\) is indeed asymptotically stable.

% Figure environment removed

% Figure environment removed
% Figure environment removed

%% =============================================================================
\section{A quick excursion into instability via Chetaev's Theorem}
%% =============================================================================

In this brief section, we transcend beyond the stability of isolated equilibria into their instability. This gives us one more opportunity to leverage the generality and adaptability of our algorithmic procedure. We consider the problem of determining the instability of an equilibrium point by employing Chetaev's theorem \cite[Chapter V, p.\ 188]{ref:Vid-02}. We refer to the example of spinning of a rigid body provided in \cite[pp. 188-190]{ref:Vid-02}, and express the dynamics as:
\begin{equation}
  \label{eq:chetaev's}
  \begin{aligned}
	  \dot\state_{1} & = a\state_{2}\state_{3},\\
	  \dot\state_{2} & = -b\state_{1}\state_{3},\\
	  \dot\state_{3} & = c\state_{1}\state_{2},
    \end{aligned}
\end{equation}
where for the sake of simplicity, we pick \(a = b = c = 1\). We pick an equilibrium point of the above system of the form \( (0, y_0, 0)\) with \(y_0 \ge 0\) to be specific and transform coordinates as shown in \cite[Example 105, p.\ 189]{ref:Vid-02}.

\subsubsection*{Candidate Lyapunov triplet for our experiment}
    Our selections were as follows:
\begin{description}
	\item[\ref{d:nbhd}]  \(\nbhd  \Let \set[\big]{ (\state_1, \state_2, \state_3) \in \R[3] \suchthat \state_1, \state_2, \state_3 \in [0, 1] }\).
	\item[\ref{d:pdf}] \(\lowerBound \in \classK\): \(\lowerBound\left(\radius\right) = 0\) for \(\radius \ge 0\).
	\item[\ref{d:dictionary}] The dictionary for candidate Lyapunov functions was selected to be
		\begin{align*}
			\basisDict & \Let \set[\big]{ x_1^{i_1} \cdot x_2^{i_2} \cdot x_3^{i_3} \suchthat (i_1, i_2, i_3) \in\Nz[3], 2 \leq i_1 + i_2 + i_3 \leq 6 }.
		\end{align*}
\end{description}
No stability margin was incorporated in the numerical experiment, effectively considering it to be zero. Upon executing the dynamics, the resulting Lyapunov-like function is obtained as \(\state\mapsto \state_{1}^2 + \state_{2}^2 + \state_{3}^2 + \state_1\state_3\), which fulfills the constraints specified in Chetaev's theorem \cite[Theorem 99, p.\ 188]{ref:Vid-02}. This confirms the instability of the system's equilibrium point at the origin \((0, 0, 0)\).



%% =============================================================================
\section{Concluding remarks}
\label{s:concl}
%% =============================================================================

We have given an exposition of a numerically tractable technique for finding Lyapunov functions to test two types of stability properties of equilibria of continuous finite-dimensional vector fields. These two properties are representational, and our technique carries over to other qualitative properties of equilibria that are captured (equivalently) by means of appropriate pointwise behavior of Lyapunov-like functions. Indeed, the technique readily applies to robust and stochastic versions of stability and, as such, will be useful for verifying input-to-state stability (ISS) and its allied versions in the robust setting, as well as for verifying different types of stochastic stability. Extensions of our results to the synthesis of control Lyapunov functions can also be readily carried out by means of our techniques. Results along some of the aforementioned directions will be reported subsequently.


%% ----------------------------------------------------------------------

%\bibliographystyle{amsalpha}
%\bibliography{refs}

\begin{thebibliography}{10}

\bibitem{eilers2021product}Eilers, M., Meier, S. \& Müller, P. Product Programs in the Wild: Retrofitting Program Verifiers to Check Information Flow Security. {\em Computer Aided Verification (CAV)}. (2021)
\bibitem{tiwari2009complete}Tiwari, M., Wassel, H., Mazloom, B., Mysore, S., Chong, F. \& Sherwood, T. Complete information flow tracking from the gates up. {\em Proceedings Of The 14th International Conference On Architectural Support For Programming Languages And Operating Systems}. pp. 109-120 (2009)

\bibitem{tiwari2009execution}Tiwari, M., Li, X., Wassel, H., Chong, F. \& Sherwood, T. Execution leases: A hardware-supported mechanism for enforcing strong non-interference. {\em Proceedings Of The 42nd Annual IEEE/ACM International Symposium On Microarchitecture}. pp. 493-504 (2009)

\bibitem{jin2012proof}Jin, Y. \& Makris, Y. Proof carrying-based information flow tracking for data secrecy protection and hardware trust. {\em 2012 IEEE 30th VLSI Test Symposium (VTS)}. pp. 252-257 (2012)

\bibitem{li2011caisson}Li, X., Tiwari, M., Oberg, J., Kashyap, V., Chong, F., Sherwood, T. \& Hardekopf, B. Caisson: A Hardware Description Language for Secure Information Flow. {\em Proceedings Of The 32Nd ACM SIGPLAN Conference On Programming Language Design And Implementation}. pp. 109-120 (2011), http://doi.acm.org/10.1145/1993498.1993512

\bibitem{li2014sapper}Li, X., Kashyap, V., Oberg, J., Tiwari, M., Rajarathinam, V., Kastner, R., Sherwood, T., Hardekopf, B. \& Chong, F. Sapper: A Language for Hardware-level Security Policy Enforcement. {\em Proceedings Of The 19th International Conference On Architectural Support For Programming Languages And Operating Systems}. pp. 97-112 (2014), http://doi.acm.org/10.1145/2541940.2541947

\bibitem{zhang2015secverilog}Zhang, D., Wang, Y., Suh, G. \& Myers, A. A Hardware Design Language for Timing-Sensitive Information-Flow Security. {\em Proceedings Of The Twentieth International Conference On Architectural Support For Programming Languages And Operating Systems}. pp. 503-516 (2015), http://doi.acm.org/10.1145/2694344.2694372

\bibitem{bidmeshki2015vericoq}Bidmeshki, M. \& Makris, Y. VeriCoq: A Verilog-to-Coq converter for proof-carrying hardware automation. {\em 2015 IEEE International Symposium On Circuits And Systems (ISCAS)}. pp. 29-32 (2015)

\bibitem{hu2016detecting}Hu, W., Mao, B., Oberg, J. \& Kastner, R. Detecting hardware trojans with gate-level information-flow tracking. {\em Computer}. \textbf{49}, 44-52 (2016)

\bibitem{kong2017using}Kong, S., Shen, Y. \& Zhou, H. Using security invariant to verify confidentiality in hardware design. {\em Proceedings Of The On Great Lakes Symposium On VLSI 2017}. pp. 487-490 (2017)

\bibitem{ardeshiricham2017register}Ardeshiricham, A., Hu, W., Marxen, J. \& Kastner, R. Register transfer level information flow tracking for provably secure hardware design. {\em Proceedings Of The Conference On Design, Automation \& Test In Europe (DATE)}. pp. 1695-1700 (2017), http://dl.acm.org/citation.cfm?id=3130379.3130775

\bibitem{ardeshiricham2017clepsydra}Ardeshiricham, A., Hu, W. \& Kastner, R. Clepsydra: Modeling timing flows in hardware designs. {\em 2017 IEEE/ACM International Conference On Computer-Aided Design (ICCAD)}. pp. 147-154 (2017)

\bibitem{deng2017secchisel}Deng, S., Gümüşoğlu, D., Xiong, W., Gener, Y., Demir, O. \& Szefer, J. SecChisel: language and tool for practical and scalable security verification of security-aware hardware architectures. {\em Cryptology EPrint Archive}. (2017)

\bibitem{bidmeshki2017information}Bidmeshki, M., Antonopoulos, A. \& Makris, Y. Information flow tracking in analog/mixed-signal designs through proof-carrying hardware IP. {\em Design, Automation \& Test In Europe Conference \& Exhibition (DATE), 2017}. pp. 1703-1708 (2017)

\bibitem{boraten2018securing}Boraten, T. \& Kodi, A. Securing NoCs against timing attacks with non-interference based adaptive routing. {\em 2018 Twelfth IEEE/ACM International Symposium On Networks-on-Chip (NOCS)}. pp. 1-8 (2018)

\bibitem{pilato2018tainthls}Pilato, C., Wu, K., Garg, S., Karri, R. \& Regazzoni, F. Tainthls: High-level synthesis for dynamic information flow tracking. {\em IEEE Transactions On Computer-Aided Design Of Integrated Circuits And Systems}. \textbf{38}, 798-808 (2018)

\bibitem{zagieboylo2019using}Zagieboylo, D., Suh, G. \& Myers, A. Using information flow to design an ISA that controls timing channels. {\em 2019 IEEE 32nd Computer Security Foundations Symposium (CSF)}. pp. 272-27215 (2019)

\bibitem{pieper2020dynamic}Pieper, P., Herdt, V., Große, D. \& Drechsler, R. Dynamic information flow tracking for embedded binaries using SystemC-based virtual prototypes. {\em 2020 57th ACM/IEEE Design Automation Conference (DAC)}. pp. 1-6 (2020)

\bibitem{restuccia2021aker}Restuccia, F., Meza, A. \& Kastner, R. AKER: A design and verification framework for safe and secure soc access control. {\em IEEE/ACM International Conference On Computer Aided Design (ICCAD)}. (2021), https://par.nsf.gov/servlets/purl/10298115

\bibitem{restuccia2022framework}Restuccia, F., Meza, A., Kastner, R. \& Oberg, J. A Framework for Design, Verification, and Management of SoC Access Control Systems. {\em IEEE Transactions On Computers}. (2022), https://kastner.ucsd.edu/wp-content/uploads/2022/10/admin/tcomputer-aker22.pdf

\bibitem{cherupalli2017software}Cherupalli, H., Duwe, H., Ye, W., Kumar, R. \& Sartori, J. Software-based gate-level information flow security for IoT systems. {\em 50th IEEE/ACM International Symposium On Microarchitecture}. (2017), https://dl.acm.org/doi/pdf/10.1145/3123939.3123955

\bibitem{fadiheh2023exhaustive}Fadiheh, M., Wezel, A., Muller, J., Bormann, J., Ray, S., Fung, J., Mitra, S., Stoffel, D. \& Kunz, W. An Exhaustive Approach to Detecting Transient Execution Side Channels in RTL Designs of Processors. {\em IEEE Transactions On Computers}. \textbf{72}, 222-235 (2023,1)

\bibitem{wu2022exert}Wu, J., Fowze, F. \& Forte, D. EXERT: EXhaustive Integrity Analysis for Information Flow Security. {\em Asian Hardware Oriented Security And Trust Symposium (AsianHOST)}. (2022), https://dforte.ece.ufl.edu/wp-content/uploads/sites/65/2022/09/EXERT%5C_AsianHost.pdf

\bibitem{athalye2022knox}Athalye, A., Kaashoek, M. \& Zeldovich, N. Verifying Hardware Security Modules with Information-Preserving Refinement. {\em OSDI}. (2022)

\bibitem{fowze2022eisec}Fowze, F., Choudhury, M. \& Forte, D. EISec: Exhaustive Information Flow Security of Hardware Intellectual Property Utilizing Symbolic Execution. {\em Asian Hardware Oriented Security And Trust Symposium (AsianHOST)}. (2022)

\bibitem{athalye2019notary}Athalye, A., Belay, A., Kaashoek, M., Morris, R. \& Zeldovich, N. Notary: A Device for Secure Transaction Approval. {\em 27th Symposium On Operating Systems Principles (SOSP)}. (2019), https://doi.org/10.1145/3341301.3359661

\bibitem{meza2023hyperflowgraph}Meza, A. \& Kastner, R. Information Flow Coverage Metrics for Hardware Security Verification.  (2023), arXiv 2304.08263

\bibitem{ryan2023countering}Ryan, K. \& Sturton, C. Countering the Path Explosion Problem in the Symbolic Execution of Hardware Designs.  (2023), arXiv 2304.05445 

\bibitem{dorsey2020intel}Dorsey, V. \& Morhardt, C. Intel Security Development Lifecycle. (Intel,2020)

\bibitem{he2015model}He, S., Roe, N., Wood, E., Nachtigal, N. \& Helms, J. Model of the Product Development Lifecycle. (Sandia National Laboratories,2015)

\bibitem{YangSP2016}Yang, K., Hicks, M., Dong, Q., Austin, T. \& Sylvester, D. A2: Analog Malicious Hardware. {\em 2016 IEEE Symposium On Security And Privacy (SP)}. pp. 18-37 (2016)

\bibitem{or1200}. OpenRISC 1200 Implementation. , https://github.com/openrisc/or1200

\bibitem{msp430}. openMSP430. , https://opencores.org/projects/openmsp430

\bibitem{farzana2019soc}Farzana, N., Rahman, F., Tehranipoor, M. \& Farahmandi, F. SoC Security Verification using Property Checking. {\em 2019 IEEE International Test Conference (ITC)}. pp. 1-10 (2019)

\bibitem{TrustHub2}Farzana, N., Farahmandi, F. \& Tehranipoor, M. SoC Security Properties and Rules. {\em IACR Cryptol. EPrint Arch.}. \textbf{2021} pp. 1014 (2021)

\bibitem{hicks2015specs}Hicks, M., Sturton, C., King, S. \& Smith, J. SPECS: A Lightweight Runtime Mechanism for Protecting Software from Security-Critical Processor Bugs. {\em ASPLOS}. pp. 517-529 (2015)

\bibitem{bilzor2011security}Bilzor, M., Huffmire, T., Irvine, C. \& Levin, T. Security Checkers: Detecting processor malicious inclusions at runtime. {\em HOST}. (2011)

\bibitem{zhang2017scifinder}Zhang, R., Stanley, N., Griggs, C., Chi, A. \& Sturton, C. Identifying Security Critical Properties for the Dynamic Verification of a Processor. {\em ASPLOS}. pp. 541-554 (2017)

\bibitem{zhang2020transys}Zhang, R. \& Sturton, C. Transys: Leveraging Common Security Properties Across Hardware Designs. {\em Proceedings Of The Symposium On Security And Privacy (S\&P)}. (2020)

\bibitem{trippel2020ICAS}Trippel, T., Shin, K., Bush, K. \& Hicks, M. ICAS: an Extensible Framework for Estimating the Susceptibility of IC Layouts to Additive Trojans. {\em 2020 IEEE Symposium On Security And Privacy (SP)}. pp. 1742-1759 (2020)

\bibitem{Deutschbein2022JCEN}Deutschbein, C., Meza, A., Restuccia, F., Kastner, R. \& Sturton, C. Isadora: Automated Information Flow Property Generation for Hardware Security Verification. {\em Journal Of Cryptographic Engineering (JCEN)}. (2022)

\bibitem{zhang2021sidechannel}Zhang, T., Park, J., Tehranipoor, M. \& Farahmandi, F. PSC-TG: RTL Power Side-Channel Leakage Assessment with Test Pattern Generation. {\em 2021 58th ACM/IEEE Design Automation Conference (DAC)}. pp. 709-714 (2021)

\bibitem{torlak2014rosette}Torlak, E. \& Bodik, R. A Lightweight Symbolic Virtual Machine for Solver-Aided Host Languages. {\em Proceedings Of The 35th ACM SIGPLAN Conference On Programming Language Design And Implementation}. pp. 530-541 (2014), https://doi.org/10.1145/2594291.2594340

\bibitem{cha2012mayhem}Cha, S., Avgerinos, T., Rebert, A. \& Brumley, D. Unleashing Mayhem on Binary Code. {\em Proceedings Of The 2012 IEEE Symposium On Security And Privacy}. pp. 380-394 (2012)

\bibitem{bao2021symbolic}Bao, Q., Wang, Z., Li, X., Larus, J. \& Wu, D. Abacus: Precise side-channel analysis. {\em International Conference On Software Engineering (ICSE)}. pp. 797-809 (2021)

\bibitem{wang2017cached}Wang, S., Wang, P., Liu, X., Zhang, D. \& Wu, D. CacheD: Identifying cache-based timing channels in production software. {\em USENIX Security Symposium}. pp. 235-252 (2017)

\bibitem{wang2019identifying}Wang, S., Bao, Y., Liu, X., Wang, P., Zhang, D. \& Wu, D. Identifying Cache-Based Side Channels through Secret-Augmented Abstract Interpretation. {\em 28th USENIX Security Symposium (USENIX Security 19)}. pp. 657-674 (2019,8), https://www.usenix.org/conference/usenixsecurity19/presentation/wang-shuai

\bibitem{brotzman2019casym}Brotzman, R., Liu, S., Zhang, D., Tan, G. \& Kandemir, M. Casym: Cache aware symbolic execution for side channel detection and mitigation. {\em Symposium On Security And Privacy (SP)}. (2019)

\bibitem{guarnier2020spectector}Guarnieri, M., Köpf, B., Morales, J., Reineke, J. \& Sánchez, A. Spectector: Principled Detection of Speculative Information Flows. {\em 2020 IEEE Symposium On Security And Privacy (SP)}. pp. 1-19 (2020)

\bibitem{avgerinos2014automatic}Avgerinos, T., Cha, S., Rebert, A., Schwartz, E., Woo, M. \& Brumley, D. Automatic exploit generation. {\em Communications Of The ACM}. \textbf{57}, 74-84 (2014)

\bibitem{avgerinos2011automatic}Avgerinos, T., Hao, B. \& Brumley, D. Automatic exploit generation. {\em Network And Distributed System Security Symposium (NDSS)}. (2011)

\bibitem{renzelmann2012symdrive}Renzelmann, M., Kadav, A. \& Swift, M. SymDrive: Testing Drivers without Devices. {\em 10th USENIX Symposium On Operating Systems Design And Implementation}. (2012), https://www.usenix.org/conference/osdi12/technical-sessions/presentation/renzelmann

\bibitem{zhang2018end}Zhang, R., Deutschbein, C., Huang, P. \& Sturton, C. End-to-End Automated Exploit Generation for Validating the Security of Processor Designs. {\em Proceedings Of The International Symposium On Microarchitecture (MICRO)}. (2018)

\bibitem{Shen2018SymbolicEB}Shen, L., Mu, D., Cao, G., Qin, M., Blackstone, J. \& Kastner, R. Symbolic execution based test-patterns generation algorithm for hardware Trojan detection. {\em Comput. Secur.}. \textbf{78} pp. 267-280 (2018)

\bibitem{clarkson2010hyperproperties}Clarkson, M. \& Schneider, F. Hyperproperties. {\em J. Comput. Secur.}. \textbf{18}, 1157-1210 (2010,9), http://dl.acm.org/citation.cfm?id=1891823.1891830

\bibitem{Kozyri2022expressing}Kozyri, E., Chong, S. \& Myers, A. Expressing Information Flow Properties. {\em Foundations And Trends® In Privacy And Security}. \textbf{3}, 1-102 (2022), http://dx.doi.org/10.1561/3300000008

\bibitem{meza2022safety}Meza, A., Restuccia, F., Kastner, R. \& Oberg, J. Safety verification of third-party hardware modules via information flow tracking. {\em 1st Real-Time Intelligent Edge Computing Workshop (RAGE)}. (2022), https://kastner.ucsd.edu/wp-content/uploads/2022/08/admin/rage22-safety.pdf

\bibitem{Deutschbein2021Isadora}Deutschbein, C., Meza, A., Restuccia, F., Kastner, R. \& Sturton, C. Isadora: Automated Information Flow Property Generation for Hardware Designs. {\em Proceedings Of The Workshop On Attacks And Solutions In Hardware Security (ASHES)}. (2021)

\bibitem{ferraiuolo2017secverilog}Ferraiuolo, A., Xu, R., Zhang, D., Myers, A. \& Suh, G. Verification of a Practical Hardware Security Architecture Through Static Information Flow Analysis. {\em Proceedings Of The Twenty-Second International Conference On Architectural Support For Programming Languages And Operating Systems}. pp. 555-568 (2017), http://doi.acm.org/10.1145/3037697.3037739

\bibitem{ardeshiricham2019verisketch}Ardeshiricham, A., Takashima, Y., Gao, S. \& Kastner, R. VeriSketch: Synthesizing Secure Hardware Designs with Timing-Sensitive Information Flow Properties. {\em Proceedings Of The 2019 ACM SIGSAC Conference On Computer And Communications Security}. pp. 1623-1638 (2019)


\end{thebibliography}


\end{document}

