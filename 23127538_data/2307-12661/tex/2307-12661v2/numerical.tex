%% ============================================================================
\section{Numerical experiments}
\label{s:numerics}
%% ============================================================================

This section presents a number of numerical experiments to illustrate the technique developed above. A few points concerning these experiments:
\begin{enumerate}[label=(\roman*), align=right, leftmargin=*, widest=iii]
    \item For the purpose of the numerical experiments, we choose the convex quadratic objective function \(\R[\basisDim]\ni z\mapsto \objective(z) \Let \norm{z - \mathbf{1}}^2\), where \(\mathbf{1} \in \R[\basisDim]\) is a vector of all ones
    \item In our simulations, we excluded the function \(\upperBound(\cdot)\) --- serving as an upper bound for the Lyapunov functions to be constructed --- from our consideration; since the number of elements in our dictionaries is finite, the decrescent property of any Lyapunov function follows immediately.
    \item In each experiment, we validated the output of our algorithm by numerically computing the minimum values of the two functions
		\[
			y\mapsto \constraintA(y) \Let \lyapfn(y) - \lowerBound(\norm{y}) \quad \text{and}  \quad y\mapsto \constraintB(y) \Let -\stabilityMargin(\ypos) - \inprod{\pdv{\lyapfn}{x}(y)}{\vecfld(y)}
		\]
		over the domain under consideration, and they were confirmed to be non-negative.
	\item \texttt{python 3.10.4} was employed with the support of the libraries \texttt{numpy 1.24.3}, \texttt{scipy 1.8.1}, and \texttt{matplotlib 3.7.1}.
		\begin{itemize}[label=\(\circ\), leftmargin=*]
			\item The `inner' convex optimization problem in each case was solved by employing the \texttt{scipy.optimize.minimize()} function with the \texttt{``SLSQP''} method as an input parameter and \texttt{max iterations = 10000}.
			\item The `outer' (global) maximization problem in each case was solved by employing the \texttt{scipy.optimize.dual\_annealing()} function with \texttt{max iterations = 30}, and a vector (of appropriate dimension) of all entries equal to \(0.5\) as the initial starting point.
			\item In each case, the outcome of the MSA algorithm (which produces a Lyapunov function) was validated against the selected candidate Lyapunov triplet with the aid of the function \texttt{scipy.optimize.differential\_evolution()} together with its default parameters.
		\end{itemize}
\end{enumerate}


%% -------------------------------------------------------------------------
\subsection{A nonlinear planar vector field}
%% -------------------------------------------------------------------------

Consider a nonlinear 2-d system given by
\begin{equation}
    \label{e:random}
    \begin{aligned}
        & \dot{\state_{1}} = 2\state_1\left(1-\frac{\state_1}{2}\right)-\state_1\state_{2}, \quad \\
        & \dot{\state_{2}} = 3\state_2\left(1-\frac{\state_2}{3}\right)-2\state_1\state_{2}.
    \end{aligned}
    \end{equation}
It can be found that \eqref{e:random} has three equilibrium points, out of which the system is asymptotically stable around the equilibrium points \((2,0)\) and \((0,3)\). A circular disc of radius \(0.2\) centered at the equilibria is chosen as the domain of investigation for the equilibrium points. After shifting the origin to the respective equilibrium points, the following Lyapunov triplets were utilized for both points.

\subsubsection*{Candidate Lyapunov triplet for our experiment}
Our selections were as follows:
\begin{description}
	\item[\ref{d:nbhd}]  \(\nbhd  \Let \set[\big]{ (\state_1, \state_2) \in \R[2] \suchthat \state_1^2 + \state_2^2  \leq 0.04 }\).
	\item[\ref{d:pdf}] \(\lowerBound, \marginBound \in \classK\) were picked as \(\lowerBound(\radius) = \frac{\radius^2}{6}\) and \(\marginBound\left(\radius\right) =  \frac{\radius^2}{12}\) for \(\radius \ge 0\).
	\item[\ref{d:dictionary}] The dictionaries were selected to be:
		\begin{align*}
			\basisDict & \Let \set[\big]{ x_1^{i_1} \cdot x_2^{i_2} \suchthat (i_1, i_2) \in\Nz[2], i_1 + i_2  = 2 },\\ 
			\marginDict & \Let \set[\big]{ x_1^{i_1} \cdot x_2^{i_2} \suchthat (i_1, i_2) \in\Nz[2], i_1 + i_2  = 2, 4}.
		\end{align*}
\end{description}

The results of our procedure are collected in Table \ref{table_2DField}.
\begin{table}[tbh]
    \begin{center}
    \renewcommand{\arraystretch}{1.5}
    \begin{tabular}{cl}
    \toprule
    Equilibrium Point & Lyapunov Function $\state\mapsto\lyapfn(\state)$ \\ \midrule
    (2,0) & ${0.981\state_1}^2 + 0.922\state_1\state_2+{1.083\state_2}^2 -3.924\state_1-1.844\state_2+3.924$\\
    (0,3) & $1.316\state_1^2 + 0.261\state_1\state_2+
    0.336\state_2^2- 0.783\state_1-2.016\state_2+3.024$ \\
    \bottomrule
    \end{tabular}
		\caption{\label{table_2DField}
			Lyapunov functions $x\mapsto\lyapfn(x)$ for the two equilibrium points corresponding to \eqref{e:random} in the \emph{original space coordinates}.}
    \end{center}
\end{table}

% Figure environment removed
% Figure environment removed

% Figure environment removed

% Figure environment removed


%% -----------------------------------------------------------------------------
\subsection{The van der Pol oscillator}
%% -----------------------------------------------------------------------------

We consider a van der Pol oscillator described by
\begin{equation}
\label{e:vanderpol}
\begin{aligned}
    & \dot\state_{1} = \state_{2}, \quad \\
    & \dot\state_{2} = -\state_1 + \eps\state_2(1-\state_1^2),
\end{aligned}
\end{equation}
where \(\eps\in\R[]\) is a parameter. Observe that the stability of the above system depends on \(\eps\). To avoid an unstable limit cycle around the equilibrium point (origin), the domain is kept to be a circular disc with a radius of \(0.5\).

\subsubsection*{Candidate Lyapunov triplet for our experiment}
Our selections were as follows:
\begin{description}
	\item[\ref{d:nbhd}] \(\nbhd  \Let \set[\big]{ (\state_1, \state_2) \in \R[2] \suchthat \state_1^2 + \state_2^2  \leq 0.25 }\).
	\item[\ref{d:pdf}] \(\lowerBound, \marginBound \in \classK\) were picked as \(\lowerBound\left(\radius\right) = \frac{\radius^3}{2}\) and \(\marginBound\left(\radius\right) =  \frac{\radius^{10}}{4}\) for \(\radius \ge 0\).
	\item[\ref{d:dictionary}] The dictionaries were selected to be:
		\begin{align*}
			\basisDict & \Let \set[\big]{ x_1^{i_1} \cdot x_2^{i_2} \suchthat (i_1, i_2) \in\Nz[2],  i_1 + i_2  = 2 },\\ 
			\marginDict & \Let \set[\big]{ x_1^{i_1} \cdot x_2^{i_2} \suchthat (i_1, i_2) \in\Nz[2], i_1 + i_2  = 2, 4, \cdots, 12 }.
		\end{align*}
\end{description}
The Lyapunov function obtained from our numerical procedure was: 
\[
	\state\mapsto\lyapfn(\state) = 1.106\state_1^2 + 0.380\state_1\state_2+ 1.106\state_2^2
\]
with the value of \(\eps= -2\) for which the vector field is asymptotically stable, which happens to be the case for all \(\eps <  0\).

% Figure environment removed
% Figure environment removed

% Figure environment removed



%%--------------------------------------------------------------------------------
\subsection{A whirling pendulum}
%%----------------------------------------------------------------
Consider the whirling pendulum system lifted from \cite[Example 3]{ref:PapPra-02} and described by
\begin{equation}
    \label{e: whirl pendulum}
    \begin{aligned}
        & \dot{\state_{1}} = \state_{2},\\
        & \dot{\state_{2}} = {\dot{\theta}^2}_a\sin(\state_1)\cos(\state_1) - \frac{g}{l_a}\sin(\state_1).
    \end{aligned}
\end{equation}
The system \eqref{e: whirl pendulum} is found to be Lyapunov stable when the condition 
\begin{equation}
    \label{e:whirl pendulum condition}
	{\dot{\theta}^2}_a <\frac{g}{l_a}
\end{equation}
is met and is otherwise unstable. \cite{ref:PapPra-02} provides a Lyapunov function for the case of the above whirling pendulum system using \texttt{SOSTOOLS} after introducing two more variables to the system: $\wpstate_{1}=\sin(\state_{1})$ and $\wpstate_{2}=\cos(\state_{2})$, which transforms \eqref{e: whirl pendulum} into a 4-d polynomial vector field, and obtained a Lyapunov function of the form \(\state\mapsto a_1\state_{2}^2 + a_2\wpstate_{1}^{2} + a_3\wpstate_{2}^{2}+a_4\wpstate_{2}+a_5\). 

We now demonstrate the ability of our method to construct a Lyapunov function of a form similar to the above but without any need for  such transformations. We carried out our simulations with the numerical values of different parameters present in Table \ref{table_whirlpendulum}.
\begin{table}[tbh]
		    \begin{center}
		    \renewcommand{\arraystretch}{1.5}
			    \begin{tabular}{cc}
				    \toprule
					Parameter & Numerical Value \\
					\midrule
				    ${\dot{\theta}^2}_a$ & 1 \\
                        $\frac{g}{l_a}$ & 10\\
				    \bottomrule
			    \end{tabular}
			\end{center}
			\caption{\label{table_whirlpendulum} Parameters for the damped whirling pendulum \eqref{e: whirl pendulum}. }
\end{table}
The flexibility of our method is demonstrated in this example by the selection of both polynomial and non-polynomial functions in the dictionary for \(V(\cdot)\).

\subsubsection*{Candidate Lyapunov triplet for our experiment}

Our selections were as follows:
\begin{description}
	\item[\ref{d:nbhd}]  \(\nbhd  \Let \set[\big]{ (\state_1, \state_2) \in \R[2] \suchthat \state_1^2 + \state_2^2  \leq 1 }\).
	\item[\ref{d:pdf}] \(\lowerBound, \marginBound \in \classK\) were picked as \(\lowerBound\left(\radius\right) = \frac{\radius^2}{100}\) and \(\marginBound\left(\radius\right) =  0\) for \(\radius \ge 0\).
	\item[\ref{d:dictionary}] The dictionaries were selected to be:
		\begin{align*}
			\basisDict & \Let \set[\big]{ x_1^{i_1} \cdot x_2^{i_2} \suchthat (i_1, i_2) \in\Nz[2], i_1 + i_2  = 2 } \cup \set[\big]{ \cos\left(j_{1}x_1\right) \suchthat 0 \leq j_1 \leq 2 } ,\\ 
			\marginDict & \Let \set[\big]{ x_1^{i_1} \cdot x_2^{i_2} \suchthat (i_1, i_2) \in\Nz[2], i_1 + i_2  = 2, 4, 6 }.
		\end{align*}
\end{description}
  
The Lyapunov function for the system \eqref{e: whirl pendulum}, with parameters in Table \ref{table_whirlpendulum} and the aforementioned Lyapunov triplet, obtained from our numerical procedure was: 
\[
	\state\mapsto\lyapfn(\state) = 0.050 \state_2^2 + 0.025\cos\left(2\state_1\right)-\cos\left(\state_1\right)+0.975.
\]
Figure \ref{fig: whirling_lyapunov} depicts this function on the unit box \(\lcrc{-1}{1}^2\) (a domain containing \(\nbhd\)).

% Figure environment removed
% Figure environment removed

% Figure environment removed

%% -----------------------------------------------------------------------------
\subsection{A 5-d hyperchaotic system with linear feedback controller}
%% -----------------------------------------------------------------------------
For our next example, we introduce an illustration of a 5-d autonomous hyperchaotic system, as proposed by \cite{ref:Niu2021}. The system incorporates a linear feedback controller, effectively achieving global asymptotic stability at the origin. The system can be expressed as follows:
\begin{equation}
    \label{e:5d-system}
    \begin{aligned}
		\dot \state_{1} & = a(\state_{2} - \state_{1}) -k_1\state_1, \quad \\
		\dot \state_{2} & = (c-a)\state_1 + c\state_2 + \state_5 - \state_1\state_3 - k_2\state_2, \quad \\
		\dot \state_{3} & = -b\state_3 + \state_1\state_2 - k_3\state_3, \quad \\
		\dot \state_{4} & = m\state_5 - k_4\state_4, \quad \\
		\dot \state_{5} & = -\state_2 - h\state_4 - k_5\state_5
    \end{aligned}
\end{equation}
where \((k_{1}, k_{2}, k_{3}, k_{4},k_{5}) = (0, 30, 0, 1, 1)\). We keep \(a = 23\), \(b = 3\), \(c = 18\), \(m = 12\), and \(h = 4\) as recommended in \cite{ref:Niu2021} to obtain a global asymptotic stability at origin.

\subsubsection*{Candidate Lyapunov triplet for our experiment}
Our selections were as follows:
\begin{description}
	\item[\ref{d:nbhd}]  \(\nbhd  \Let \set[\big]{ (\state_1, \state_2, \state_3, \state_4, \state_5) \in \R[5] \suchthat \state_1, \state_2, \state_3, \state_4, \state_5 \in \lcrc{-0.5}{0.5} }\).
	\item[\ref{d:pdf}] \(\lowerBound, \marginBound \in \classK\) were picked as \(\lowerBound\left(\radius\right) = \frac{\radius^2}{100}\), and \(\marginBound\left(\radius\right) =  \frac{\radius^4}{20000}\) for \(\radius \ge 0\).
	\item[\ref{d:dictionary}] The dictionaries were selected to be:
		\begin{align*}
			\basisDict & \Let \set[\big]{ x_1^{i_1} \cdot x_2^{i_2} \cdots x_5^{i_5} \suchthat (i_1, i_2, i_3, i_4, i_5) \in\Nz[5], i_1 + i_2 + i_3 + i_4 + i_5 = 2 },\\ 
			\marginDict & \Let \set[\big]{ x_1^{i_1} \cdot x_2^{i_2} \cdots x_5^{i_5} \suchthat (i_1, i_2, i_3, i_4, i_5) \in\Nz[5], i_1 + i_2 + i_3 + i_4 + i_5  = 2, 4, 6 }.
		\end{align*}
\end{description}
Our numerical procedure led to the Lyapunov function%contained in Table \ref{table-5D}.
\begin{align*}
	\state\mapsto \lyapfn(\state) & = 1.229\state_1^2 + 0.982\state_1\state_2 + 0.891\state_1\state_3 + 0.632\state_1\state_4 + 0.236\state_1\state_5\\
	& \qquad + 0.996\state_2^2+ 1.026\state_2\state_3 + 0.663\state_2\state_4 + 0.928\state_2\state_5 + 1.116\state_3^2\\
	& \qquad + 0.406\state_3\state_4 + 1.054\state_3\state_5 + 0.523\state_4^2 + 0.158\state_4\state_5 + 1.595\state_5^2.
\end{align*}

% Figure environment removed

% Figure environment removed

%% -----------------------------------------------------------------------------
\subsection{Transient stability of classical power system models}
%% -----------------------------------------------------------------------------

For our next example, we pick a four-dimensional system from \cite{ref:AngMilPap-13} given by the equation (transformed to have the origin as its equilibrium point):
\begin{equation}
  \label{eq:power_model}
  \begin{aligned}
	  \dot \state_{1} & = \state_{2}, \quad \\
	  \dot \state_{2} & = 0.0200\cos(\state_1)\cos(\state_3) - 0.0200\cos(\state_1) - 0.9998\sin(\state_1) - 0.4000\state_2  \\
	  & \qquad + 0.4996\cos(\state_1)\sin(\state_3) - 0.4996\cos(\state_3)\sin(\state_1) + 0.0200\sin(\state_1)\sin(\state_3), \\
	  \dot \state_{3} & = \state_{4}, \\
	  \dot \state_{4} & = 0.4996\cos(\state_3)\sin(\state_1) - 0.0299\cos(\state_3) - 0.4991\sin(\state_3) \\
    & \qquad -0.0200\cos(\state_1)\cos(\state_3) - 
    0.4996\cos(\state_1)\sin(\state_3) - 0.5000\state_4 \\
    & \qquad - 0.0200\sin(\state_1)\sin(\state_3) + 0.0500.
    \end{aligned}
\end{equation}

\subsubsection*{Candidate Lyapunov triplet for our experiment}
    Our selections were as follows:
\begin{description}
	\item[\ref{d:nbhd}]  \(\nbhd  \Let \set[\big]{ (\state_1, \state_2, \state_3, \state_4) \in \R[4] \suchthat \state_1, \state_2, \state_3, \state_4 \in \lcrc{-0.2}{0.2} }\).
	\item[\ref{d:pdf}] \(\lowerBound \in \classK\): \(\lowerBound\left(\radius\right) = \frac{\radius^2}{16}\) and \(\marginBound\left(\radius\right) =  \frac{\radius^2}{200}\) for \(\radius \ge 0\).
	\item[\ref{d:dictionary}] The dictionaries were selected to be:
		\begin{align*}
			\basisDict & \Let \set[\big]{ x_1^{i_1} \cdot x_2^{i_2} \cdot x_3^{i_3} \cdot x_4^{i_4} \suchthat (i_1, i_2, i_3, i_4) \in\Nz[4],  i_1 + i_2 + i_3 + i_4 = 2 },\\ 
			\marginDict & \Let \set[\big]{ x_1^{i_1} \cdot x_2^{i_2} \cdot x_3^{i_3} \cdot x_4^{i_4} \suchthat (i_1, i_2, i_3, i_4) \in\Nz[4], i_1 + i_2 + i_3 + i_4  = 2, 4}.
		\end{align*}
\end{description} 
This example demonstrates how our algorithm effectively handles higher order systems, enabling the identification of an appropriate Lyapunov function with predefined lower bounds. Our numerical procedure led to the Lyapunov function %contained in Table \ref{table_powersys}.
\begin{align*}
	\state\mapsto\lyapfn(\state) & = 1.339\state_1^2 + 0.591\state_1\state_2 + 0.635\state_1\state_3 + 0.622\state_1\state_4 + 0.989\state_2^2\\
	& \qquad + 0.651\state_2\state_3 + 1.063\state_2\state_4 + 1.064\state_3^2 + 0.970\state_3\state_4 + 1.297\state_4^2.
\end{align*}

% Figure environment removed

% Figure environment removed


%% -----------------------------------------------------------------------------
\subsection{Flexibility of basis}
%% -----------------------------------------------------------------------------

We present a final example illustrating an interesting situation in which the vector field is defined by means of a case statement. This example is lifted from \cite[p. 68]{ref:BhaSze-70}, and the dynamics is given by
\begin{equation}
  \label{eq:bhatia's}
  \begin{aligned}
	  \dot\state_{1} & =
	  \begin{cases}
		  \state_1 & \text{if }\state_{1}^2\state_{2}^2 \geq 1,\\
				2\state_{1}^3\state_{2}^2 - \state_{1} & \text{if } \state_{1}^2\state_{2}^2 < 1,
	\end{cases}\\
	\dot\state_{2} & = -\state_{2}.
\end{aligned}
\end{equation}
Notice that the vector field \eqref{eq:bhatia's} is continuous, and the neighborhood \(\nbhd\) selected above contains both the regions differentiated by the case statement.

\subsubsection*{Candidate Lyapunov triplet for our experiment}
    Our selections were as follows:
\begin{description}
	\item[\ref{d:nbhd}]  \(\nbhd  \Let \set[\big]{ (\state_1, \state_2) \in \R[2] \suchthat \state_1, \state_2 \in [-4, 4] }\).
	\item[\ref{d:pdf}] \(\lowerBound, \marginBound \in \classK\) : \(\lowerBound\left(\radius\right) = 0\) and \(\marginBound\left(\radius\right) =  \frac{\radius^2}{2048}\) for \(\radius \ge 0\).
	\item[\ref{d:dictionary}] The dictionaries were selected to be:
		\begin{align*}
			\basisDict & \Let \set[\big]{ x_1^{i_1} \cdot x_2^{i_2} \suchthat (i_1, i_2) \in\Nz[2], 2 \leq i_1 + i_2  \leq 6 }, \\
				\marginDict & \Let \set[\big]{ x_1^{i_1} \cdot x_2^{i_2}  \suchthat (i_1, i_2) \in\Nz[2], i_1 + i_2 = 2, 4}.
		\end{align*}
\end{description}
In contrast to the rational polynomial functions mentioned in \cite{ref:BhaSze-70} corresponding to this example, we obtained a Lyapunov function within the class of polynomial functions itself without extending it to include rational functions. The outcome of our numerical experiment was the Lyapunov function
\[
	\state\mapsto\lyapfn(\state) =0.003\state_1^2 + 0.069\state_1\state_2 + 1.063\state_1^2.
\]

% Figure environment removed

% Figure environment removed

% Figure environment removed


%%==============================================================================
\subsection{Rayleigh problem with control constraints}
%%==============================================================================

In this example we address stability of an equilibrium point under \emph{continuous-time model predictive control}. It is well-known that discrete-time model predictive control (MPC) \cite{ref:XiLi-19} is a very powerful constrained control technique, and it consists of the following steps for a system of the form \(x^+ = f(x, u)\):
\begin{itemize}
	\item solve a constrained finite horizon optimal control problem for the given control system with the current state \(x\) as the initial condition, and extract the corresponding sequence of optimal control actions over the horizon,
	\item apply the entry corresponding to initial stage to obtain the next state \(x^+\), 
	\item increment the time by \(1\), and return back to the first step with \(x^+\) as the current state.
\end{itemize}
No analytical expression of the optimal control sequence is available, in general, due to the presence of constraints in the problem, but it is clear that the optimal control actions depend on the initial state \(x\); if the optimal control action is unique at the initial time, then it is a \emph{mapping} of the initial state.\footnote{This feature is exploited by the so-called \emph{explicit MPC} technique to numerically construct this implicit feedback.}

The idea of continuous-time MPC for a control system \(\dot x = f(x) + g(x) u\) follows the same route as outlined above, starting from a continuous-time finite horizon constrained control problem (given an initial condition) over control trajectories, and extracting the optimal control trajectory (assuming uniqueness). This trajectory parametrically depends on the initial condition, and therefore, the optimal control action \(u\opt(0)\) at the initial time \(0\) depends on the initial state \(x\); one denotes this dependence as \(u\opt(0; x)\). The closed-loop system under a continuous-time MPC controller, consequently, is \(\dot x = f(x) + g(x) u\opt(0; x)\); it is immediately clear that \(u\opt(0; \cdot)\) is a feedback. Solving continuous-time constrained optimal control problems is typically time consuming, and it is an extremely challenging matter to evaluate \(u\opt(0; x)\) instantaneously for the continuous-time model. However, if such an implementaion were possible, then \(u\opt(0; \cdot)\) would be evaluated by an oracle and, in general, would not admit an explicit formula.

This example studies stability of an equilibrium point of the nonlinear control system 
\[
	\pmat{\dot{x}_1(t)\\ \dot{x}_2(t)} = \pmat{x_2(t) \\ -x_1(t) + x_2(t)\bigl(1.40 - 0.14 x_2(t)^2\bigr)} + \pmat{0\\ 4} u(t)
\]
under a particular continuous-time MPC strategy; this system features in the benchmark Rayleigh's problem \cite[Example 4.6]{ref:Bet-10}, and for our purposes it will be accompanied by control constraints over a finite horizon. Note that the vector field of the nonlinear system vanishes at the triplet \((x_1, x_2, u) = (0, 0, 0)\). Our intention is to stabilize the origin \((x_1, x_2) = (0, 0)\) under continuous-time MPC in closed-loop, and with this in mind, the following underlying continuous-time optimal control problem for MPC was formulated:
\begin{equation}
\label{eq:quito}
\begin{aligned}
	\minimize_{u:\lcrc{0}{\tfin}\to\R[]} \quad & \int_{0}^{\tfin} \bigl(u(t)^2 + x_1(t)^2\bigr) \, \odif{t} \\
\sbjto \quad &
\begin{cases}
\dot{x}_1(t) = x_2(t), \\
\dot{x}_2(t) = -x_1(t) + x_2(t)\bigl(1.40 - 0.14 x_2(t)^2\bigr) + 4u(t),\\
	\bigl(x_1(0), x_2(0)\bigr) = (a_1, a_2) \text{ (given)}, \\
	\bigl(x_1(\tfin), x_2(\tfin)\bigr) = (0, 0), \\
	|u(t)| \leq 1.
\end{cases}
\end{aligned}
\end{equation}
We picked the horizon \(\tfin = 4.5\). The optimal trajectory construction algorithm QuITO (Quasi Interpolation based Trajectory Optimization) developed in \cite{ref:Ganguly2022} was employed to solve the optimal control problem \eqref{eq:quito} and obtain optimal control trajectories \(\lcrc{0}{\tfin}\ni t\mapsto u\opt\bigl(t; (a_1, a_2)\bigr)\in\R[]\) for each initial condition \((a_1, a_2)\) sufficiently close to \((0, 0)\). The step size for the uniform cardinal grid on \(\lcrc{0}{\tfin}\) was picked to be \(h = 0.09\), and the interpolating Schwarz function was chosen to be \(\R[]\ni t\mapsto \psi(t) = \frac{1}{\sqrt{\pi\mathcal D}} \epower{-\frac{t^2}{h^2 \mathcal D}}\) with the shape parameter \(\mathcal D = 5\), both of which were input parameters to QuITO.% and a quintic spline-based interpolation for the off-grid values of time was carried out. The parameters were chosen such that the error between the optimal solution and the one from QuITO was within \(\pm 0.005\%\).\todo[fancyline]{Are you sure about this figure?}

The resulting closed-loop control system was, naturally,
\begin{equation}
	\label{e:clquito}
	\pmat{\dot{x}_1(t)\\ \dot{x}_2(t)} = \pmat{x_2(t) \\ -x_1(t) + x_2(t)\bigl(1.40 - 0.14 x_2(t)^2\bigr)} + \pmat{0\\ 4} u\opt\bigl(0; x_1(t), x_2(t)\bigr).
\end{equation}
We highlight that the right-hand side of \eqref{e:clquito} lacks an analytical expression although off-the-shelf results may be employed to check the mapping \(x\mapsto u\opt(0; x)\) is continuous and it evaluates to \(0\) at \(x = (0, 0)\). This means \((0, 0)\) is an equilibrium point of the preceding closed-loop system; let us check whether it is asymptotically stable.

\subsubsection*{Candidate Lyapunov triplet for our experiment}
Our selections were as follows:
\begin{description}
	\item[\ref{d:nbhd}]  \(\nbhd  \Let \set[\big]{ (\state_1, \state_2) \in \R[2] \suchthat \state_1^2 + \state_2^2  \leq 0.2025}\).
	\item[\ref{d:pdf}] \(\lowerBound, \marginBound \in \classK\) were picked as \(\lowerBound\left(\radius\right) = \frac{\radius^2}{4}\) and \(\marginBound\left(\radius\right) =  \frac{\radius^2}{32}\) for \(\radius \ge 0\).
	\item[\ref{d:dictionary}] The dictionaries were selected to be:
		\begin{align*}
			\basisDict & \Let \set[\big]{ x_1^{i_1} \cdot x_2^{i_2} \suchthat (i_1, i_2) \in\Nz[2], i_1 + i_2  = 2 }, \\ 
			\marginDict & \Let \set[\big]{ x_1^{i_1} \cdot x_2^{i_2} \suchthat (i_1, i_2) \in\Nz[2], i_1 + i_2  = 2, 4, 6 }.
		\end{align*}
\end{description}
The Lyapunov function obtained from our numerical procedure was
\[
	\state\mapsto\lyapfn(\state) = 1.216\state_1^2 + 0.688\state_1\state_2+ 0.948\state_2^2,
\]
which shows that the equilibrium point \((0, 0)\) is indeed asymptotically stable.

% Figure environment removed

% Figure environment removed
% Figure environment removed

%% =============================================================================
\section{A quick excursion into instability via Chetaev's Theorem}
%% =============================================================================

In this brief section, we transcend beyond the stability of isolated equilibria into their instability. This gives us one more opportunity to leverage the generality and adaptability of our algorithmic procedure. We consider the problem of determining the instability of an equilibrium point by employing Chetaev's theorem \cite[Chapter V, p.\ 188]{ref:Vid-02}. We refer to the example of spinning of a rigid body provided in \cite[pp. 188-190]{ref:Vid-02}, and express the dynamics as:
\begin{equation}
  \label{eq:chetaev's}
  \begin{aligned}
	  \dot\state_{1} & = a\state_{2}\state_{3},\\
	  \dot\state_{2} & = -b\state_{1}\state_{3},\\
	  \dot\state_{3} & = c\state_{1}\state_{2},
    \end{aligned}
\end{equation}
where for the sake of simplicity, we pick \(a = b = c = 1\). We pick an equilibrium point of the above system of the form \( (0, y_0, 0)\) with \(y_0 \ge 0\) to be specific and transform coordinates as shown in \cite[Example 105, p.\ 189]{ref:Vid-02}.

\subsubsection*{Candidate Lyapunov triplet for our experiment}
    Our selections were as follows:
\begin{description}
	\item[\ref{d:nbhd}]  \(\nbhd  \Let \set[\big]{ (\state_1, \state_2, \state_3) \in \R[3] \suchthat \state_1, \state_2, \state_3 \in [0, 1] }\).
	\item[\ref{d:pdf}] \(\lowerBound \in \classK\): \(\lowerBound\left(\radius\right) = 0\) for \(\radius \ge 0\).
	\item[\ref{d:dictionary}] The dictionary for candidate Lyapunov functions was selected to be
		\begin{align*}
			\basisDict & \Let \set[\big]{ x_1^{i_1} \cdot x_2^{i_2} \cdot x_3^{i_3} \suchthat (i_1, i_2, i_3) \in\Nz[3], 2 \leq i_1 + i_2 + i_3 \leq 6 }.
		\end{align*}
\end{description}
No stability margin was incorporated in the numerical experiment, effectively considering it to be zero. Upon executing the dynamics, the resulting Lyapunov-like function is obtained as \(\state\mapsto \state_{1}^2 + \state_{2}^2 + \state_{3}^2 + \state_1\state_3\), which fulfills the constraints specified in Chetaev's theorem \cite[Theorem 99, p.\ 188]{ref:Vid-02}. This confirms the instability of the system's equilibrium point at the origin \((0, 0, 0)\).

