\documentclass[10pt,aps,prd,onecolumn,showpacs,amsmath,amssymb,nofootinbib,eqsecnum,preprintnumbers,superscriptaddress]{revtex4-2}


%%%%%%%%%%%%%%%%%%%%%%%%%%%%%%%%%%%%%%%%%%%%%%%%%%%%%%%%%%%%%%%%%%%%%%%%%%%%%%%%%%%%%%
%% PACKAGES
%\usepackage{mathtools}
\usepackage{xcolor}
\usepackage{graphicx}
\usepackage{tikz}
\usepackage[colorlinks=true,
            citecolor=red,
            linkcolor=blue,
            urlcolor=violet,
            filecolor=cyan,
            backref=false]{hyperref}
\usepackage{multirow}

%%%%%%%%%%%%%%%%%%%%%%%%%%%%%%%%%%%%%%%%%%%%%%%%%%%%%%%%%%%%%%%%%%%%%%%%%%%%%%%%%%%%%%
%% MACROS
\newcommand\bs[1]{\boldsymbol{#1}}
\newcommand\feq{\mathrel{\phantom{=}}}

%%%%%%%%%%%%%%%%%%%%%%%%%%%%%%%%%%%%%%%%%%%%%%%%%%%%%%%%%%%%%%%%%%%%%%%%%%%%%%%%%%%%%%
%% DOCUMENT

\begin{document}


%%%%%%%%%%%%%%%%%%%%%%%%%%%%%%%%%%%%%%%%%%%%%%%%%%%%%%%%%%%%%%%%%%%%%%%%%%%%%%%%%%%%%%
%% TITLE


\title{
Carrollian limit of quadratic gravity
}

\author{Poula Tadros}
\email{poulatadros9@gmail.com}
\affiliation{Institute of Theoretical Physics, Faculty of Mathematics and Physics, Charles University,
V Hole\v{s}ovi\v{c}k\'ach 2, Prague 180 00, Czech Republic}
\affiliation{Department of Applied Physics, Aalto University School of Science, FI-00076 Aalto, Finland}

\author{Ivan Kol\'a\v{r}}
\email{ivan.kolar@matfyz.cuni.cz}
\affiliation{Institute of Theoretical Physics, Faculty of Mathematics and Physics, Charles University,
V Hole\v{s}ovi\v{c}k\'ach 2, Prague 180 00, Czech Republic}

\date{\today}

\begin{abstract}
We study the Carrollian limit of the (general) quadratic gravity in four dimensions. We find that in order for the Carrollian theory to be a modification of the Carrollian limit of general relativity, the parameters in the action must depend on the speed of light in a specific way. By focusing on the leading and the next-to-leading orders in the Carrollian expansion, we show that there are four such nonequivalent Carrollian theories. Imposing conditions to remove tachyons (from the linearized theory), we end up with a classification of Carrollian theories according to the leading-order and next-to-leading-order actions. All modify the Carrollian limit of general relativity with quartic terms of the extrinsic curvature. To the leading order, we show that two theories are equivalent to general relativity, one to $R+R^2$ theory, and one to the general quadratic gravity. To the next-to-leading order, two are equivalent to $R+R^2$ while the other two to the general quadratic gravity.

\end{abstract}

\maketitle


%%%%%%%%%%%%%%%%%%%%%%%%%%%%%%%%%%%%%%%%%%%%%%%%%%%%%%%%%%%%%%%%%%%%%%%%%%%%%%%%%%%%%%
%% INTRODUCTION

\section{Introduction}
%Literature \textcolor{blue}{[add \& categorize anything useful (remove if irrelevant)]}: 
%\begin{itemize}
 %  \item Carrollian limit \cite{Perez:2021abf,Perez:2022jpr,Hartong:2015xda,Figueroa-OFarrill:2021sxz,Campoleoni:2022ebj,Hansen:2021fxi,Donnay:2019jiz,Gomis:2020wxp,Grumiller:2019tyl,Figueroa-OFarrill:2022pus,Bergshoeff:2022qkx,Guerrieri:2021cdz,Redondo-Yuste:2022czg,Marsot:2022imf,Freidel:2022vj,Marsot:2022qkx,Figueroa-OFarrill:2022mcy,Bergshoeff:2022eog,Herfray:2021qmp,deBoer:2021jej,Chandrasekaran:2021hxc,Henneaux:2021yzg,Anabalon:2021wjy,Hansen:2020wqw,Bagchi:2019clu,Morand:2018tke,Ciambelli:2019lap,Levy-Leblond,Gupta,Price:1986yy,1986bhmp.book.....T,Damour:1978cg,PhysRevD.106.085004,Anderson:2002zn,Marsot:2021tvq,Bergshoeff:2017btm,Henneaux:2021yzg,Marsot:2022imf,4860f44e-649d-341b-9a70-b912b6531bea}
 %  \item Quadratic gravity \cite{Stelle:1977ry,Stelle:1976gc,Nenmeli:2021orl}
   % \item BHs in Quadratuc gravity \cite{Lu:2015cqa,Lu:2015psa,Podolsky:2019gro,Pravdova:2023nbo}
   % \item near-horizon dynamics \cite{Gray:2022svz,Bicak:2023vxs}
    %\item Quadratic gravity from string theory \cite{ZUMINO1986109,ZWIEBACH1985315,Forger:1996vj,Myers:1987yn,Alvarez-Gaume:2015rwa}
   % \item Carroll particles \cite{Bergshoeff_2014,Marsot:2021tvq,Zhang:2023jbi}
    %\item Carrollian limit in condensed matter physics \cite{Bagchi:2022eui,Kubakaddi_2021,Kononov_2021}
   %\item Carrollian field theories \cite{PhysRevD.106.085004,Chen:2023pqf,Bergshoeff:2022eog}
    %\item Carrollian CFT \cite{Bagchi:2019xfx,Bagchi:2019clu,Bagchi:2021gai,PhysRevD.103.105001}
   % \item Carrollian fluids \cite{Bagchi:2023ysc,Ciambelli:2018wre,Ciambelli:2018xat,campoleoni2019two,Ciambelli:2020eba,10.21468/SciPostPhys.9.2.018}
   % \item Carrollian strings \cite{PhysRevLett.123.111601,Bagchi:2021ban,Cardona:2016ytk}
%\end{itemize}

The \textit{quadratic gravity} can be derived as an effective field theory by truncating the expansion of the bosonic section of string theory with the first order being \textit{general relativity (GR)} \cite{ZUMINO1986109,ZWIEBACH1985315,Forger:1996vj,Myers:1987yn,Alvarez-Gaume:2015rwa} or by imposing a maximal momentum to strings \cite{Nenmeli:2021orl}. It has been studied even before the connection to string theory as a renormalizable theory of gravity \cite{Stelle:1977ry,Stelle:1976gc,Julve:1978xn}. It admits a wide class of black-hole and other spherically symmetric (exact) solutions \cite{Lu:2015cqa,Lu:2015psa,Podolsky:2019gro,Pravdova:2023nbo}. Nevertheless, in general, it suffers from the presence of unphysical ghost and tachyonic degrees of freedom \cite{Stelle:1976gc}.




The \textit{Carrollian limit} was first considered independently by Levy-Leblond \cite{Levy-Leblond} and Sen Gupta \cite{Gupta} as the ultralocal limit of the Poincar\'e group where the speed of light $c$ approaches zero, ${c\to0}$. However, at the time, due to the lack of physical application of this limit, it was only studied by mathematicians until 40 years later when the Carrollian limit was linked to many applications in physics. Now, Carrollian physics and Carrollian structures are studied in the context of representations of the Carroll group i.e. Carroll particles \cite{Zhang:2023jbi,Bergshoeff_2014,Marsot:2021tvq,Marsot:2022imf}, condensed matter physics \cite{Bagchi:2022eui,Kubakaddi_2021,Kononov_2021}, field theory \cite{PhysRevD.106.085004,Chen:2023pqf,Bergshoeff:2022eog,Henneaux:2021yzg}, conformal field theory \cite{Bagchi:2019xfx,Bagchi:2019clu,Bagchi:2021gai,PhysRevD.103.105001}, fluid mechanics \cite{Bagchi:2023ysc,Ciambelli:2018wre,Ciambelli:2018xat,campoleoni2019two,Ciambelli:2020eba,10.21468/SciPostPhys.9.2.018}, cosmology \cite{deBoer:2021jej,Bonga:2020fhx}, string theory \cite{PhysRevLett.123.111601,Bagchi:2021ban,Cardona:2016ytk}, gravity \cite{Perez:2021abf,Perez:2022jpr,Hartong:2015xda,Figueroa-OFarrill:2021sxz,Hansen:2021fxi,Gomis:2020wxp,Bergshoeff:2022qkx,Guerrieri:2021cdz,Hansen:2020wqw} (it is regarded as the strong coupling limit of gravity theories \cite{Anderson:2002zn}), black holes \cite{Donnay:2019jiz,Grumiller:2019tyl,Redondo-Yuste:2022czg,Marsot:2022imf,Anabalon:2021wjy}, null boundaries \cite{Herfray:2021qmp,Chandrasekaran:2021hxc,Bagchi:2019clu,Ciambelli:2019lap} and dynamics of particles near black-hole horizons \cite{Gray:2022svz,Bicak:2023vxs,Marsot:2022qkx}.

The connection between the Carrollian limit and physics near black-hole horizons was shown in \cite{Donnay:2019jiz} utilizing the membrane paradigm \cite{Price:1986yy,1986bhmp.book.....T,Damour:1978cg} which is a paradigm showing that the physics of a black hole on a stretched horizon is dual to that of a relativistic fluid on a ${(2+1)}$-dimensional submanifold. Taking the Carrollian limit of both sides gives a duality between physics on the horizon and a Carrollian fluid. It was shown afterwards that there are two nonequivalent Carrollian limits of a relativistic theory called the \textit{electric} and \textit{magnetic limits}. The electric limit comes directly from the \textit{leading order (LO)} in the \textit{Carrollian expansion}, i.e., the expansion in $c$, while the magnetic limit is a certain truncation of the \textit{next-to-leading order (NLO)} of this expansion.

In this paper we analyze the electric Carrollian limit of quadratic gravity, which is the first step towards the analysis of dynamics of particles near black-hole horizons. Throughout the paper we use the units where is Newton's constant $G$ is set to ${G={1}/{(16\pi)}}$. The paper is organized as follows:
\begin{itemize}   
    \item In Sec.~\ref{sec:3}, we review the mathematical aspects of Carrollian physics from algebraic and geometric points of view and explain the duality to physics near black holes' horizons in more detail.
    \item In Sec.~\ref{sec:2}, we review the pre-ultralocal (PUL) parametrization, which is suitable for the Carrollian expansion, and calculate the PUL versions of various tensors appearing in a general four-dimensional quadratic gravity action.
    \item In Sec.~\ref{sec:4}, we review the electric Carrollian limit of GR and show the ultralocality of the spacetime evolution.
    \item In Sec.~\ref{sec:5}, we perform the Carrollian expansion of quadratic gravity action. We show that the parameters $\alpha$ and $\beta$ in the action must depend on $c$ in a specific way otherwise the resulting theory will be drastically different than the Carrollian limit of GR. Requiring the resulting theory to be a modification to the Carrollian limit of GR to LO or NLO gives four nonequivalent Carrollian theories.
    \item In Sec.~\ref{sec:6}, we study those limits one by one and derive conditions on $\alpha$ and $\beta$ to remove tachyons (from the linearized theory) in each case to the LO and NLO.
    \item The paper is concluded with a brief summary and discussions of our results in Sec.~\ref{sec:concl}.
\end{itemize}

    
\section{Mathematical Overview and relation to black hole physics}\label{sec:3}
The Carroll algebra is given by Carrollian limit of the Poincar\'e group, also considered as the ultralocal Inonu-Wigner contraction for the Poincar\'e group \cite{4860f44e-649d-341b-9a70-b912b6531bea}. It was first constructed independently by Levy-Leblond \cite{Levy-Leblond} and Sen Gupta \cite{Gupta} as the limit of the Poincar\'e group as the speed of light tends to zero or equivalently, when the time separation is much less than the space separation. In this limit the light cone converges into a line making any motion of particles with non-zero energy impossible. However, due to the lack of physical applications in this limit, the Carroll group and the geometry associated with it i.e. Carrollian geometry were studied solely by mathematicians and mathematical physicist for a long time. The study of Carrollian physics by physicists began when a connection between the Carrollian limit and physics near black-hole horizons was established \cite{Donnay:2019jiz}. In their paper they showed that any null hypersurface is endowed with a Carrollian structure. Since then many papers came out studying Carrollian limit of GR, dynamics of particles near black-hole horizons, and the geometry on horizons of different gravity theories as well as mathematical aspects \cite{Figueroa-OFarrill:2021sxz,Figueroa-OFarrill:2022mcy,Figueroa-OFarrill:2022pus,Freidel:2022vjq,Herfray:2021qmp,Morand:2018tke,Campoleoni:2022ebj}. For a review on Carrollian geometry and its relation with Galilean geometry i.e. Newton-Cartan geometry see \cite{Bergshoeff:2017btm}. In this section we review the mathematical structures and properties of the Carroll group and the Carrollian geometric structures and their relation to black holes.

\subsection{Algebraic structure}

Beginning with the Poincar\'e group $ISO(1,3)$. The group can be decomposed into $ISO(1)$ generated by the time translation generator $P_0$, and $ISO(3)$ generated by space translation generators $P_i$ and space rotation generators $M_{ij}$, where $i,j,\ldots=1,2,3$, and Lorentz boost generators $M_{0i}$ relating the two subgroups. To perform the ultralocal contraction\footnote{In the literature (for example in \cite{Gupta:2020dtl,Hartong:2015xda,Bagchi:2019xfx}), it is called the ultrarelativistic contraction and the ultrarelativistic limit. However, we use the term `ultralocal' instead since `ultrarelativistic' was used in the case $v\rightarrow c$ not $c \rightarrow 0$ and the defining feature of the Carrollian limit is ultralocality.} we define a parameter $\omega$ and re-scale $ISO(1)$ and the boosts as follows: $P_0 \rightarrow \omega P_0$, $M_{0i} \rightarrow \omega M_{0i}$. Then, we take the limit $\omega \rightarrow \infty$. In order to derive the commutation relations between the generators of the new algebra, we begin with the Poincar\'e algebra and consider only the relations with consistent dependency on $\omega$. The resulting commutation relations are
\begin{equation}\label{eq:24}
    \begin{aligned}
        \big[M_{ij},M_{0k}\big]&=2\delta_{k[i}M_{j]0},\\
       \big[M_{0i},M_{0j}\big]&=0,\\
       \big[P_i,M_{0j}\big]&= \delta_{ij}P_0,\\
    \big[M_{ij},P_k\big]&=2\delta_{k[i}P_{j]},\\
       \big[M_{ij},M_{kl}\big]&=2\big(\delta_{i[l}M_{k]j}-\delta_{j[l}M_{k]i}\big).
    \end{aligned}
\end{equation}

This algebra is called the Carroll algebra and it is the symmetry algebra of all Carrollian theories. The described contraction aligns with our intuition about Carrollian limits since we rescale the time translation and boost generators and send them to infinity implying that space generators are very small in comparison. Another aspect is that, unlike the full Poincar\'e algebra, the commutation relations show that space generators (space rotations and space translations) get transformed into time generators (time translation) or boosts but not the other way around. This means that space motion eventually get swiped away in the favour of time translation and boosts. This is equivalent to saying that space translations are negligible compared to time translations and boosts i.e. the motion in space is negligible compared to motion in time, or the closure of the light cone to a line. This is opposite to the Galilean contraction where we take the limit $\omega \to 0$, or equivalently rescale space translations and boosts and send the parameter to infinity i.e. time translations are negligible compared to space translations and boosts. This results in the opening of the light cone to a sheet.  


% Figure environment removed

%% Figure environment removed

Note that the ultralocal limit of a relativistic theory will result in a theory with Carrollian symmetries by construction. However, not all Carrollian theories originate from the ultralocal limit of a relativistic theory. There exist other ways to construct Carrollian theories, for example, by defining a theory directly on a Carrollian manifolds. This leads to a richer symmetry group and geometric structures \cite{Gupta:2020dtl,Hartong:2015xda}.

\subsection{Geometric structure}\label{sec:geomstructure}

By geometric structure we mean the description of the mathematical structure using manifolds and bundles. Carrollian geometric structures can be described as an \textit{intrinsic} or \textit{extrinsic} structure of a manifold. For more detailed mathematical description see \cite{Ciambelli:2019lap,Figueroa-OFarrill:2021sxz}

The intrinsic description defines a \textit{weak Carrollian spacetime} $\mathcal{C}$ of dimension ${d+1}$ as a fiber bundle with a degenerate metric with a base space of a sphere $S$ and a one-dimensional fiber. This structure comes with two mappings $\pi:\mathcal{C} \rightarrow S$ such that $\pi^{-1}(S)$ is one-dimensional (this represents time), and $d\pi: T\mathcal{C} \rightarrow TS$, where $T\mathcal{C}$ is the tangent bundle of $\mathcal{C}$ and $TS$ is the tangent bundle of $S$. A \textit{Carrollian spacetime} is a weak Carrollian spacetime with an Ehresmann connection. It defines a smooth decomposition of the tangent space of the Carrollian spacetime into a \textit{vertical} (thought of as time), and \textit{horizontal} parts (thought of as space), ${T\mathcal{C}= \mathrm{Ver} \oplus \mathrm{Hor}}$, where ${\mathrm{Ver}=\ker(d\pi)}$. This decomposition is similar to the ADM decomposition, however, here the metric is decomposed into a timelike vector and the induced metric on spatial surfaces rather than in terms of a lapse function, a shift vector, and the induced metric.\footnote{In some cases like in the adapted coordinates used in \cite{Ciambelli:2019lap,Baiguera:2022lsw,Petkou:2022bmz}, we can identify this decomposition with the ADM decomposition with zero shift vector.} Equivalently, a weak Carrollian spacetime is the triple $(\mathcal{C},\mathcal{V},\boldsymbol{h})$ where $\mathcal{C}$ is a manifold, $\mathcal{V}$ is a vector bundle on $\mathcal{C}$ and $\boldsymbol{h}$ is a degenerate metric on $\mathcal{C}$ such that ${\boldsymbol{h}(\boldsymbol{v},.)=0}$ for every ${\boldsymbol{v}\in \mathcal{V}}$. The Carrollian theories obtained in this paper by means of the Carrollian expansion (in LO) are defined on Carrollian spacetimes.

Alternatively, we can characterize the Carrollian structure also extrinsically by means of the rigged structure on a $d$-dimensional timelike submanifold $H$ of the $(d+1)$-dimensional Lorentzian manifold $(M,\boldsymbol{g})$ (smooth manifold $M$ equipped with Lorentzian metric $\boldsymbol{g}$). Let us define a normal covector $\boldsymbol{n}$ to $H$ and a vector $\boldsymbol{k}$ that is dual to it, ${n_{\mu} k^{\mu}=1}$. The pair $(\boldsymbol{n},\boldsymbol{k})$ is called the \textit{rigged structure}. Due to the Frobenius theorem its existence is equivalent to a foliation of $M$ with $d$-dimensional leaves corresponding to surfaces of constant coordinate ${r}$ (such that ${\boldsymbol{n}=\boldsymbol{d}r}$), which we choose to be copies of $H$ and call them the \textit{stretched horizons}. Furthermore, we assume that a leaf $N$ representing the limit ${r \rightarrow 0}$ is null and call it the \textit{true horizon}. The rigged structure on $H$ defines a projection operator from $TM$ to $TH$, called the \textit{rigging projector},
\begin{equation}\label{eq:25}
    P^{\mu}_{\nu}=g_{\nu}^{\mu}-k^{\mu}n_{\nu}.
\end{equation}
Let us denote the norm of $\boldsymbol{n}$ by ${g^{\mu\nu}n_{\mu}n_{\nu}=2\rho}$ and define a tangential vector $\boldsymbol{v}$ to $H$ as
\begin{equation}\label{eq:26}
    v^{\mu}=P^{\mu}_{\nu}n^{\nu}=n^{\mu}-2\rho k^{\mu}.
\end{equation}
With this definition we can decompose the rigged projector as
\begin{equation}\label{eq:27}
    P^{\mu}_{\nu}=q^{\mu}_{\nu}+k_{\nu}v^{\mu},
\end{equation}
where $\boldsymbol{q}$ is the induced metric on $H$. Notice that ${q_{\mu\nu}v^{\mu}v^{\nu}= -2\rho + 4\rho^2k_{\mu}k^{\mu}}$. Although it is not necessary here, in the black hole applications $\boldsymbol{k}$ is typically chosen to be null \cite{Price:1986yy}. Taking the limit ${\rho \rightarrow 0}$, which corresponds to ${r\to0}$, of the triple $(\boldsymbol{P},\boldsymbol{v}, \boldsymbol{q})$, given by \eqref{eq:25}, \eqref{eq:26}, and \eqref{eq:27}, defines the \textit{Carrollian structure} on $N$. Here, $k_{\mu}=g_{\mu\nu}k^{\nu}$ plays the role of the Ehresmann connection. In other words, $\rho$, indicating the distance between $H$ and $N$ (in the limiting sense when $H$ is close to $N$), plays a role of the speed of light. This procedure is depicted in Fig.~\ref{fig:stretchedhorizon}. Remark that the metric $\boldsymbol{q}$ is regular in the limit ${\rho \rightarrow 0}$. Note that the splitting done by \eqref{eq:27} is equivalent to the splitting done by the mapping $d\pi$, hence, the extrinsic and intrinsic descriptions are equivalent.

% Figure environment removed

The extrinsic description of Carrollian structures with the concept of stretched horizons is useful for understanding the relationship between the Carrollian limit of the gravitational theories and the dynamics near black-holes horizons. The physics on a stretched horizon of the black hole is shown to be equivalent to a relativistic fluid on a $(2+1)$-dimensional sub-manifold in what is known as the membrane paradigm \cite{Damour:1978cg,1986bhmp.book.....T,Price:1986yy}. However, when trying to define the same quantities on the true horizon they diverge. These divergences can be regularized but the regularization which depends on the foliation of stretched horizons used. The way to define finite quantities on and near the true horizon is to use the identification by the membrane paradigm. Since the stretched horizons converge to the true horizon in the Carrollian limit, we can identify the physics on and near the true horizon by taking the Carrollian limit of the dual fluid on a $(2+1)$-dimensional sub-manifold. This was shown explicitly in \cite{Donnay:2019jiz}. The quantities are well defined since the metrics on stretched horizons converge to a regular metric on the true horizon as shown in the previous section. Just described relation between Carrollian limits and black holes is visualized in Fig.~\ref{fig:2}.

% Figure environment removed


\section{Pre-ultralocal parametrization}\label{sec:2}

The \textit{pre-ultralocal (PUL) parametrization} is a parametrization of the metric on a manifold using the decomposition of its tangent bundle into a vertical and horizontal subbundles (see below). It is the most convenient parametrization of the spacetime for the analysis of Carrollian gravity since it is well adapted to the ultralocal structure of the Carrollian limit and it displays the speed of light $c$ explicitly, which makes the calculations more obvious. In what follows, we briefly explain the mathematical background of the PUL parametrization. By following the calculations and notations in \cite{Hansen:2021fxi}, we present the PUL version of the Riemannian tensor which will be used to claculate terms in quadratic gravity action in later sections. 

Let $(M,\boldsymbol{g})$ be a ${(d+1)}$-dimensional Lorentzian manifold (with mostly positive signature). Let us denote the tangent bundle of $M$ by $TM$ and can define two sub-bundles of $TM$ according to the signature of the metric: The first is called the \textit{vertical bundle} $\mathrm{Ver}M$ (or the timelike bundle) and it corresponds to the timelike direction, i.e., its fibers are endowed with a vector space isomorphic to the time coordinate. The second is referred to as the \textit{horizontal bundle} $\mathrm{Hor}M$ (or the spatial bundle) and it represents the remaining $d$ spacelike directions. It is easy to prove that ${TM= \mathrm{Ver}M \oplus \mathrm{Hor}M}$. Furthermore, it generates a foliation of the manifold whose slices are the sub-manifolds of a constant time coordinate $t$. This foliation allows us to define orthogonal spatial and timelike sections as follows: Consider a covector $T_{\mu}$ and a vector $V^{\mu}$ from $\mathrm{Ver}M$, where ${\mu,\nu, \ldots= 1,2, \dots,d+1}$, and a symmetric tensor $\Pi_{\mu \nu}$ from $\mathrm{Hor}M$, which is the induced metric (or the first fundamental form), and its inverse $\Pi^{\mu \nu}$. 

By construction of the sub-bundles and the foliation we have 
\begin{subequations} \label{eq:1}
\begin{align}
T_{\mu}V^{\mu} &=-1 \label{eq:1a}, 
\\
-V^{\mu}T_{\nu} + \Pi^{\rho \mu} \Pi_{\rho \nu} &= \delta_{\mu}^{\nu} , \label{eq:1b}
\\
T_\mu \Pi^{\mu \nu} &=0, 
\\
\Pi_{\mu \nu} V^\nu &=0.
\end{align}
\end{subequations}
The PUL parametrization of the metric $g_{\mu\nu}$ is given by 
\begin{subequations} \label{eq:3}
\begin{align}
g_{\mu \nu} &= -c^2 T_{\mu}T_{\nu}+ \Pi_{\mu \nu},\label{eq:3a}\\
g^{\mu \nu} &= -\tfrac{1}{c^2} V^{\mu}V^{\nu}+ \Pi^{\mu \nu}. \label{eq:3b}
\end{align}
\end{subequations}
In terms of vielbeins, the metric, its inverse, and the spatial tensors can be written as
\begin{subequations} \label{eq:4}
\begin{align}
    g_{\mu \nu} &= \eta_{AB}E^A_{\mu}E^B_{\nu},
\\
    g^{\mu \nu} &= \eta^{AB}\Theta_A^{\mu}\Theta_B^{\nu},
\\
     \Pi_{\mu \nu} &= \eta_{ab}E^a_{\mu}E^b_{\nu},
\\
     \Pi^{\mu \nu} &= \eta^{ab}\Theta_a^{\mu}\Theta_b^{\nu},
     \end{align}
\end{subequations}
where $E^A_{\mu}$ and $\Theta_A^{\mu}$ are the vielbiens, $A,B$ are labels running from 1 to $d+1$ (the dimension of $TM$ and the labels $a,b$ run from 1 to $d$ (the dimension of the $\mathrm{Hor}M$). Comparing the PUL parametrization with the vielbien definition we get ${E^{A}_{\mu}=(cT_{\mu},E^a_{\mu})}$ and ${\Theta^{\mu}_{A}=(-c^{-1}V^{\mu},\Theta^{\mu}_{a})}$. 

Following \cite{Hansen:2021fxi}, we assume that all fields are analytic in $c^2$ and expand them as follows:
\begin{subequations}\label{eq:5}
\begin{align}
    V^{\mu} &=v^{\mu}+c^2 M^{\mu}+ O(c^4),
\\
    T_{\mu} &=\tau_{\mu}+c^2 N_{\mu}+ O(c^4),
\\
    \Theta_a^{\mu} &=\theta_a^{\mu}+c^2 \pi_a^{\mu}+ O(c^4),
\\
    E^a_{\mu} &=e^a_{\mu}+c^2 F^a_{\mu}+ O(c^4),
\\
    \Pi^{\mu\nu} &=h^{\mu\nu}+c^2 \Phi^{\mu \nu}+ O(c^4),
\\
     \Pi_{\mu\nu} &=h_{\mu\nu}+c^2 \Phi_{\mu \nu}+ O(c^4),
     \end{align}
\end{subequations}
where $v^{\mu},M^{\mu},\tau_{\mu},N_{\nu},\theta_a^{\mu},\pi_a^{\mu},e^a_{\mu},F^a_{\mu},h^{\mu\nu},\Phi^{\mu\nu}$ are fields used to define geometries in the Carrollian limit. These fields are not all independent but they are related by two constraints. Thus, we can write $\tau_{\mu}$ and $\theta_a^{\mu}$ in terms of the other fields. Including more orders in $c^2$ leads to defining more fields that interpolate between the Carrollian theory (LO in the expansion) and the full theory on the manifold. Expanding \eqref{eq:1a}, we get
\begin{equation}\label{eq:6}
    \tau_{\mu}v^{\mu} + c^2 (\tau_{\mu}M^{\mu}+N_{\nu}v^{\mu})+c^4 N_{\mu}M^{\mu}=-1.
\end{equation}
Comparing the LO and NLO terms we arrive at
 \begin{subequations}\label{eq:7}
 \begin{align}
     \tau_{\mu}v^{\mu} &=-1,\\
     \tau_{\mu}M^{\mu}+N_{\nu}v^{\mu} &=0.
      \end{align}
 \end{subequations}
Similarly, if we expand \eqref{eq:1b}, we obtain
\begin{equation}\label{eq:8}
     -\tau_{\nu}v^{\mu}+ h^{\mu \rho}h_{\rho \nu}+c^2 (h^{\mu \rho}\Phi_{\rho \nu}+ \Phi^{\mu}h_{\rho \nu}- M^{\mu} \tau_{\nu} - v^{\mu}N_{\mu}) + c^4 \Phi^{\mu \rho}\Phi_{\rho \nu}= \delta^{\mu}_{\nu},
 \end{equation}
which by comparison of LO and NLO terms gives
 \begin{subequations}\label{eq:9}
 \begin{align}
     -\tau_{\nu}v^{\mu}+ h^{\mu \rho}h_{\rho \nu} &= \delta^{\mu}_{\nu},\\
h^{\mu \rho}\Phi_{\rho \nu}+ \Phi^{\mu}h_{\rho \nu}- M^{\mu} \tau_{\nu} - v^{\mu}N_{\mu} &=0.
 \end{align}
 \end{subequations}
Now, by expanding \eqref{eq:3a} we also get
\begin{equation}\label{eq:10}
    h_{\mu\nu} + c^2 \Phi_{\mu\nu}= \delta_{ab} e^a_{\mu}e^b_{\nu} + c^2 \delta_{ab}(F^a_{\mu}e^b_{\nu}+e^a_{\mu}F^b_{\nu}) + c^4 \delta_{ab}F^a_{\mu}F^b_{\nu}.
\end{equation}
and after comparing the LO and NLO terms, we arrive at
\begin{subequations}\label{eq:11}
\begin{align}
    h_{\mu\nu} &=\delta_{ab}e^a_{\mu}e_{\nu}^b,
\\
    \Phi_{\mu\nu} &=\delta_{ab}(F^a_{\mu}e^b_{\nu}+e^a_{\mu}F^b_{\nu}).
    \end{align}
\end{subequations}
Similarly, \eqref{eq:3b} leads to
\begin{subequations}\label{eq:12}
\begin{align}
    h^{\mu\nu} &=\delta^{ab}\theta_a^{\mu}\theta^{\nu}_b,
\\
    \Phi^{\mu\nu} &=\delta^{ab}(\theta_a^{\mu}\pi^{\nu}_{b}+\pi_a^{\mu}\theta^{\nu}_{b}).
    \end{align}
\end{subequations}
Remark that the induced metric $\boldsymbol{h}$ and the set of all vectors $\boldsymbol{v}\in\mathcal{V}$, give rise to the Carrollian spacetime $(\mathcal{C},\mathcal{V},\boldsymbol{h})$ form Sec.~\ref{sec:geomstructure}, where $\mathcal{C}$ represents the limit of $M$.

To derive a compatible connection with the PUL parametrization \cite{Hansen:2021fxi,Ciambelli:2019lap}, we notice that $V^{\mu}$ and $\Pi_{\mu\nu}$ are invariant under Carroll boosts. Thus, they must be covariantly constant. Although this can not determine a connection uniquely, it was argued in Appendix B of \cite{Hansen:2021fxi} that the most convenient choice is
\begin{equation}\label{eq:13}
C^{\rho}_{\mu\nu} = -V^{\rho}\partial_{(\mu}T_{\nu)}- V^{\rho}T_{(\mu}\pounds_{\boldsymbol{V}}T_{\nu)} + \tfrac{1}{2} \Pi^{\rho \lambda}\big[ \partial_{\mu} \Pi_{\nu \lambda}+ \partial_{\nu} \Pi_{\lambda \mu}- \partial_{\lambda}\Pi_{\mu \nu}\big]- \Pi^{\rho \lambda}T_{\nu}\mathcal{K}_{\mu \lambda},
\end{equation} 
where $\mathcal{K}_{\mu \lambda}= -\tfrac{1}{2} \pounds_{\boldsymbol{V}} \Pi_{\mu \lambda}$ is the extrinsic curvature (or the second fundamental from). 
%Expanding in $c^2$ we get
%\begin{equation}\label{eq:14}
%\begin{aligned}
 % C^{\rho}_{\mu\nu} &=  -v^{\rho}\partial_{(\mu}\tau_{\nu)}-v^{\rho}\tau_{(\mu}\pounds_{\boldsymbol{V}} \tau_{\nu)}+\tfrac{1}{2}\big[h^{\rho \lambda}\partial_{\mu}h_{\nu \lambda}+ h^{\rho \lambda}\partial_{\nu}h_{\mu \lambda} - h^{\rho \lambda}\partial_{\lambda}h_{\mu \nu}\big] - h^{\rho \lambda} \tau_{\nu}K_{\mu \lambda} \\ &\feq
 %   + c^2 \big[-v^{\rho}\partial_{(\mu}N_{\nu)}-M^{\rho} \partial_{(\mu}\tau_{\nu)}- M^{\rho} \tau_{\mu}\pounds_{\boldsymbol{V}} \tau_{\nu} - v^{\rho}M_{(\mu}\pounds_{\boldsymbol{V}} \tau_{\nu)}+ v^{\rho} \tau_{(\mu}\pounds_{\boldsymbol{V}} M_{\nu)} \\&\feq
  %  + \tfrac{1}{2} \Phi^{\rho \lambda}\partial_{\mu} h_{\nu \lambda} + \tfrac{1}{2}\Phi^{\rho \lambda} \partial_{\nu}h_{\lambda \mu} - \tfrac{1}{2} \Phi^{\rho \lambda} \partial_{\lambda}h_{\mu \nu} + \tfrac{1}{2} h^{\rho \lambda} N_{\nu} \pounds_{\boldsymbol{V}} h_{\mu \lambda} + h^{\rho \lambda} \tau_{\nu} \pounds_{\boldsymbol{V}} \Phi_{\mu \lambda} \\ &\feq
  %  + \Phi^{\rho \lambda} \tau_{\nu} \pounds_{\boldsymbol{V}} h_{\mu \lambda}\big]+ c^4 \big[-M^{\rho}\partial_{(\mu}N_{\nu)} - v^{\rho} M_{(\mu} \pounds_{\boldsymbol{V}} M_{\nu)} - M^{\rho} \tau_{(\mu} \pounds_{\boldsymbol{V}} M_{\nu)} - M^{\rho} M_{(\mu} \pounds_{\boldsymbol{V}} \tau_{\nu)} \\ &\feq
  %  + \tfrac{1}{2}\Phi^{\rho \lambda} \partial_{\mu}\Phi_{\nu \lambda} + \tfrac{1}{2}\Phi^{\rho \lambda} \partial_{\nu}h_{\lambda \mu} -  \tfrac{1}{2}\Phi^{\rho \lambda} \partial _{\lambda} \Phi_{mu \nu} + \tfrac{1}{2} h^{\rho \lambda} N_{\nu} \pounds_{\boldsymbol{V}} \Phi_{\mu \lambda} + \Phi^{\rho \lambda} N_{\nu} \pounds_{\boldsymbol{V}} h_{\mu \lambda}  \\ &\feq
  %   + \Phi^{\rho \lambda} \tau_{\nu} \pounds_{\boldsymbol{V}} \Phi_{\mu \lambda}\big] + c^6 \big[- M^{\rho}M_{(\mu} \pounds_{\boldsymbol{V}} M_{\nu)}+ \tfrac{1}{2} \Phi^{\rho \lambda} N_{\nu} \pounds_{\boldsymbol{V}} \Phi_{\mu \lambda}\big],
  %  \end{aligned}
%\end{equation}
%where $K_{\mu \lambda}=-\tfrac{1}{2}\pounds_{\boldsymbol{v}} h_{\mu \lambda}$.
This connection has a torsion given by
\begin{equation}\label{eq:14}
    T^{\rho}_{\mu\nu}=\Pi^{\rho\lambda}T_{[\mu}\mathcal{K}_{\nu]\lambda},
\end{equation}
which, to the LO, reads
\begin{equation}\label{torsion}
    T^{\rho}_{\mu\nu}=h^{\rho\lambda}\tau_{[\mu}K_{\nu]\lambda}.
\end{equation}

To proceed parameterizing the Riemann tensor, we write the Christoffel symbols in terms of the PUL fields using \eqref{eq:3} and \eqref{eq:4}. The result is
\begin{equation}\label{eq:15}
\begin{aligned}
    \Gamma^{\rho}_{\mu \nu} &= \tfrac{1}{c^2}\big[-\tfrac{1}{2}V^{\rho}V^{\lambda}\partial_{\mu}\Pi_{\nu \lambda}- \tfrac{1}{2}V^{\rho} V^{\lambda} \partial_{\nu}\Pi_{\lambda \mu} + \tfrac{1}{2}V^{\rho}V^{\lambda} \partial_{\lambda}\Pi_{\mu \nu}\big] + \tfrac{1}{2}\big[\Pi^{\rho \lambda} \partial_{\mu}\Pi_{\nu \lambda}
    \\
    &\feq+ \Pi^{\rho \lambda} \partial_{\nu}\Pi_{\lambda \mu} - \Pi^{\rho \lambda} \partial_{\lambda}\Pi_{\mu \nu} + V^{\rho}V^{\lambda} \partial_{\mu}(T_{\nu} T_{\lambda}) + 
    V^{\rho}V^{\lambda} \partial_{\nu}(T_{\mu} T_{\lambda})
    \\ 
    &\feq- V^{\rho}V^{\lambda} \partial_{\lambda}(T_{\nu} T_{\mu})\big]
    + c^2 \big[ \Pi^{\rho \lambda}\partial_{\mu}(T_{\nu}T_{\lambda})- \Pi^{\rho \lambda}\partial_{\nu}(T_{\mu}T_{\lambda}) + \Pi^{\rho \lambda}\partial_{\lambda}(T_{\nu}T_{\mu})\big]\;.
\end{aligned}    
\end{equation}
With the help of the definition of the Lie derivative we can rewrite the Christoffel symbol as
\begin{equation}\label{eq:16}
    \Gamma^{\rho}_{\mu \nu}= \tfrac{1}{c^2}\big[-V^{\rho} \mathcal{K}_{\mu \nu}\big] + \big[ C^{\rho}_{\mu\nu} + \Pi^{\rho \lambda}T_{\nu}\mathcal{K}_{\mu \lambda} \big] + c^2 \big[-T_{(\mu}\Pi^{\rho \lambda}(dT)_{\nu)\lambda}\big], 
\end{equation}
where $(dT)_{\mu\nu}=\partial_{\mu}T_{\nu}-\partial_{\nu}T_{\mu}$ is the exterior derivative of the covector $T_{\mu}$, which is the same as Eq. (2.21) in \cite{Hansen:2021fxi}. Finally, we are equipped to parameterize the Riemann tensor. In terms of the Christoffel symbols, the Riemann tensor is given by
\begin{equation}\label{eq:17}
    R^{\rho}_{\lambda \mu \nu}= \partial_{\mu}\Gamma^{\rho}_{\nu \lambda} - \partial_{\nu} \Gamma^{\rho}_{\mu \lambda} + \Gamma^{\rho}_{\mu \sigma} \Gamma^{\sigma}_{\nu \lambda} - \Gamma^{\rho}_{\nu \sigma}\Gamma^{\sigma}_{\mu \lambda}.
\end{equation}
Inserting \eqref{eq:15}, we obtain
\begin{equation}\label{eq:18}
\begin{aligned}
    R^{\rho}_{\lambda \mu \nu} &= \tfrac{1}{c^2}\big[-\partial_{\mu}(V^{\rho}\mathcal{K}_{\nu \lambda})+\partial_{\nu}(V^{\rho}\mathcal{K}_{\mu \lambda})-V^{\rho}C^{\sigma}_{\nu \lambda} \mathcal{K}_{\mu \sigma}+V^{\rho}C^{\sigma}_{\mu \lambda} \mathcal{K}_{\nu \sigma}- V^{\rho}\mathcal{K}_{\mu \sigma}\Pi^{\sigma \alpha}T_{\lambda}\mathcal{K}_{\nu \alpha} \\ &\feq
    + V^{\rho}\mathcal{K}_{\nu \sigma}\Pi^{\sigma \alpha}T_{\lambda}\mathcal{K}_{\mu \alpha} - C^{\rho}_{\mu \sigma}V^{\sigma}\mathcal{K}_{\nu \lambda} + C^{\rho}_{\nu \sigma}V^{\sigma}\mathcal{K}_{\mu \lambda} - V^{\sigma}\mathcal{K}_{\nu \lambda}\Pi^{\rho \alpha}T_{\sigma}\mathcal{K}_{\mu \alpha} + V^{\sigma}\mathcal{K}_{\mu \lambda}\Pi^{\rho \alpha}T_{\sigma}\mathcal{K}_{\nu \alpha}\big] \\ &\feq
    +\big[\partial_{\mu}C^{\rho}_{\nu \lambda}+ \partial_{\mu}(\Pi^{\rho \alpha}T_{\lambda}\mathcal{K}_{\mu \alpha})-\partial_{\nu}C^{\rho}_{\mu \lambda} - \partial_{\nu}(\Pi^{\rho \alpha}T_{\lambda}\mathcal{K}_{\nu \alpha})+ V^{\rho}\mathcal{K}_{\mu \sigma}T_{(\nu}\Pi^{\sigma \alpha}(dT)_{\lambda ) \alpha} \\ &\feq
    - V^{\rho}\mathcal{K}_{\nu \sigma}T_{(\mu}\Pi^{\sigma \alpha}(dT)_{\lambda ) \alpha} + T_{(\mu}\Pi^{\rho \alpha}(dT)_{\sigma ) \alpha}V^{\sigma} \mathcal{K}_{\nu \lambda} - T_{(\nu}\Pi^{\rho \alpha}(dT)_{\sigma ) \alpha}V^{\sigma} \mathcal{K}_{\mu \lambda} \\ &\feq
    + C^{\rho}_{\mu \sigma}C^{\sigma}_{\nu \lambda} - C^{\rho}_{\nu \sigma}C^{\sigma}_{\mu \lambda} + C^{\rho}_{\mu \sigma}\Pi^{\sigma \alpha}T_{\lambda}\mathcal{K}_{\nu \alpha} - C^{\rho}_{\nu \sigma}\Pi^{\sigma \alpha}T_{\lambda}\mathcal{K}_{\mu \alpha} + \Pi^{\rho \alpha}T_{\sigma}\mathcal{K}_{\mu \alpha}C^{\sigma}_{\nu \lambda} - \Pi^{\rho \alpha}T_{\sigma}\mathcal{K}_{\nu \alpha}C^{\sigma}_{\mu \lambda}\big] \\ &\feq
    +c^2\big[-\partial_{\mu}(T_{(\nu}\Pi^{\rho \alpha}(dT)_{\lambda ) \alpha}) + \partial_{\nu}(T_{(\mu}\Pi^{\rho \alpha}(dT)_{\lambda ) \alpha}) - C^{\rho}_{\mu \sigma}T_{(\nu}\Pi^{\sigma \alpha}(dT)_{\lambda ) \alpha} + C^{\rho}_{\nu \sigma}T_{(\mu}\Pi^{\sigma \alpha}(dT)_{\lambda ) \alpha} \\ &\feq
    - T_{(\mu}\Pi^{\rho \alpha}(dT)_{\sigma ) \alpha} C^{\sigma}_{\nu \lambda} + T_{(\nu}\Pi^{\rho \alpha}(dT)_{\sigma ) \alpha} C^{\sigma}_{\mu \lambda} - T_{( \mu}\Pi^{\rho \alpha}(dT)_{\sigma ) \alpha} \Pi^{\sigma \beta}T_{\lambda}\mathcal{K}_{\nu \beta} + T_{( \nu}\Pi^{\rho \alpha}(dT)_{\sigma ) \alpha} \Pi^{\sigma \beta}T_{\lambda}\mathcal{K}_{\mu \beta}\big] \\ &\feq
    +c^4\big[T_{( \mu}\Pi^{\rho \alpha}(dT)_{\sigma ) \alpha}T_{( \nu}\Pi^{\sigma \beta}(dT)_{\lambda ) \beta} - T_{( \nu}\Pi^{\rho \alpha}(dT)_{\sigma ) \alpha}T_{( \mu}\Pi^{\sigma \beta}(dT)_{\lambda ) \beta}\big].
    \end{aligned}
\end{equation}

\section{Carrollian expansion of GR}\label{sec:4}

Having derived the PUL parametrization of the Riemann tensor in \eqref{eq:18}, we can now review the Carrollian expansion of the GR. Recall that the Einstein-Hilbert action in four dimensions ($d=3$) is,
\begin{equation}\label{eq:GRaction}
    S=c^3\int R \sqrt{-g}d^{4}x\;.
\end{equation}


Let us first calculate the PUL parametrization of the Ricci scalar $R$. By contracting $\rho$ and $\mu$ in \eqref{eq:18}, we can write the Ricci tensor in the form
\begin{equation}\label{eq:19}
    \begin{aligned}
        R_{\lambda \nu} &= \tfrac{1}{c^2}\big[-\nabla_{\mu}(V^{\mu}\mathcal{K}_{\nu \lambda}) - 2 V^{\mu}C^{\sigma}_{[\mu \lambda]}\mathcal{K}_{\nu \sigma}+ \mathcal{K}_{\nu \lambda} \mathcal{K} - \mathcal{K}_{\mu \lambda}\Pi^{\mu \alpha} \mathcal{K}_{\nu \alpha}\big]
        + \big[R^{c}_{\lambda \nu} + \nabla_{\mu}(\Pi^{\mu \alpha}T_{\lambda}\mathcal{K}_{\nu \alpha})- \nabla_{\nu}(T_{\lambda}\mathcal{K})\\ &\feq
        +2 C^{\mu}_{[\nu \beta]}\Pi^{\beta \alpha}T_{\lambda}\mathcal{K}_{\mu \alpha} + \mathcal{K}_{(\nu}^{\alpha}(dT)_{\lambda) \alpha} - \tfrac{1}{2} V^{\mu} \mathcal{K}_{\nu \sigma}T_{\lambda}\Pi^{\sigma \alpha}(dT)_{\mu \alpha} - \tfrac{1}{2}T_{\nu}\Pi^{\mu \alpha}V^{\sigma} (dT)_{\sigma \alpha} \mathcal{K}_{\mu \lambda}\big] \\ &\feq
        +c^2\big[-\nabla_{\mu}(T_{(\nu}\Pi^{\mu \alpha}(dT)_{\lambda) \alpha}) +2 C^{\sigma}_{[\nu \mu]}T_{(\sigma}\Pi^{\mu \alpha}(dT)_{\lambda) \alpha}
       + T_{(\nu}\Pi^{\mu \alpha}(dT)_{\sigma) \alpha} \Pi^{\sigma \beta} T_{\lambda} \mathcal{K}_{\mu \beta} \big] \\ &\feq
       + c^4 \big[-\tfrac{1}{4}T_{\nu}\Pi^{\mu \alpha}(dT)_{\sigma \alpha} T_{\lambda} \Pi^{\sigma \beta} (dT)_{\mu \beta}\big],
    \end{aligned}
\end{equation}
where we introduced the trace of the extrinsic curvature, ${\mathcal{K}=\Pi^{\mu\nu}\mathcal{K}_{\mu\nu}}$, and the Ricci tensor of the connection $C^{\rho}_{\mu\nu}$, ${R_{\lambda \nu}^c= \partial_{\mu}C^{\mu}_{\nu \lambda} - \partial_{\nu}C^{\mu}_{\mu \lambda} + C^{\mu}_{\mu \sigma}C^{\sigma}_{\nu \lambda} - C^{\mu}_{\nu \sigma}C^{\sigma}_{\mu \lambda}}$. The PUL parametrization of the Ricci scalar is obtained by contraction with the inverse metric and employing $\Pi^{\lambda \nu}\nabla_{\mu}(V^{\mu}\mathcal{K}_{\nu\lambda}) = \nabla_{\nu}(V^{\nu}\mathcal{K})$. The result is
\begin{equation}\label{eq:20}
    \begin{aligned}
        R &= \tfrac{1}{c^2}\big[\mathcal{K}^2 - \Pi^{\lambda \nu} \mathcal{K}_{\mu \lambda} \Pi^{\mu \alpha} \mathcal{K}_{\nu \alpha}
         - 2 \nabla_{\nu}(V^{\nu}\mathcal{K})\big]\\ &\feq
        + \big[-\Pi^{\lambda \nu}R^{c}_{\lambda \nu} + \Pi^{\lambda \nu} \nabla_{\mu}(\Pi^{\mu \alpha} T_{\lambda} \mathcal{K}_{\nu \alpha})
        - \Pi^{\lambda \nu} \nabla_{\nu}(T_{\lambda} \mathcal{K}) +  V^{\lambda}V^{\nu}\nabla_{\mu}(T_{(\nu} \Pi^{\mu \alpha} (dT)_{\lambda)\alpha}) - V^{\lambda}V^{\nu}\nabla_{\nu}(T_{(\mu} \Pi^{\mu \alpha} (dT)_{\lambda)\alpha})\big]\\ &\feq
        + c^2\big[-\Pi^{\lambda \nu} \nabla_{\mu}(T_{(\nu} \Pi^{\mu \alpha} (dT)_{\lambda) \alpha}) + \Pi^{\lambda \nu}\nabla_{\nu}(T_{(\nu} \Pi^{\mu \alpha} (dT)_{\lambda) \alpha}) - \tfrac{1}{4} (dT)_{\mu\nu} (dT)^{\mu \nu}\big],
    \end{aligned}
\end{equation}
where we used $V^{\mu}R^c_{\mu\nu}=0$ in the calculations.

Using the relation $\nabla_{\rho}\Pi^{\mu\nu}=-V^{(\mu}\Pi^{\nu)\sigma}(dT)_{\sigma \lambda}[\delta_{\rho}^{\lambda}-V^{\lambda}T_{\rho}]$, we can find that $\Pi^{\lambda \nu}\nabla_{\mu}(T_{\lambda} \mathcal{K}_{\nu}^{\mu})=0$, $\Pi^{\lambda \nu}\nabla_{\nu}(T_{\lambda}\mathcal{K})=0$, $V^{\lambda}V^{\nu} \nabla_{\mu}(T_{(\nu}\Pi^{\mu\alpha}(dT)_{\lambda)\alpha})=-V^{\lambda}\nabla_{\mu}((dT)_{\lambda}^{\ \mu})$ and $-\Pi^{\lambda \nu}\nabla_{\mu}(T_{(\nu}\Pi^{\mu\alpha}(dT)_{\lambda)\alpha})=\tfrac{1}{2}(dT)^{\mu\nu}(dT)_{\mu\nu}$. Employing these identities, the Ricci scalar simplifies to
\begin{equation}\label{eq:21}
    \begin{aligned}
        R &= \tfrac{1}{c^2}\big[\mathcal{K}^2-\mathcal{K}^{\mu\nu}\mathcal{K}_{\mu\nu}-2\nabla_{\nu}(V^{\nu}\mathcal{K})\big]+\big[-R^{c}-\nabla_{\mu}(V^{\nu}(dT)_{\nu}^{\ \mu})]
        +c^2[\tfrac{1}{4}(dT)^{\mu\nu}(dT)_{\mu\nu}\big]
    \end{aligned}
\end{equation}
where $R^{c}=\Pi^{\mu\nu}R^c_{\mu\nu}$. Furthermore, we can separate the total derivative terms as they corresponds to the boundary terms in actions of physical theories. Finally, the PUL parametrization of the Ricci scalar can be written in the form
\begin{equation}\label{eq:22}
    \begin{split}
        R= \tfrac{1}{c^2}\big[\mathcal{K}^2-  \mathcal{K}^{\mu \nu} \mathcal{K}_{\mu \nu} ] + [- R^c]+ c^2[ \tfrac{1}{4} (dT)^{\mu\nu}(dT)_{\mu \nu}\big] + \text{boundary terms},
    \end{split}
\end{equation}
where we collected all the boundary terms from all orders. Note that the boundary terms will be used in the calculation of quadratic curvature terms however, they are not important in this section since we compute the LO of GR.


 Hence, the PUL parameterization of the Einstein-Hilbert action is then
\begin{equation}\label{eq:28}
    S= c^2 \int \tfrac{1}{c^2}\big[\mathcal{K}^2-  \mathcal{K}^{\mu \nu} \mathcal{K}_{\mu \nu} ] + [-R^c]+ c^2[ \tfrac{1}{4} (dT)^{\mu\nu}(dT)_{\mu \nu}\big] Ed^4x,
\end{equation}
where $E= \det(T_{\mu},E^a_{\mu})$, so after the Carrollian expansion to the LO, we arrive at
\begin{equation}\label{eq:CarrolActionGR}
    S= c^2 \int \big[K^2-  K^{\mu \nu} K_{\mu \nu}\big] ed^4x,
\end{equation}
where $K_{\mu\nu}=-\tfrac{1}{2}\pounds_{\boldsymbol{v}} h_{\mu\nu}$, $e=\det(\tau_{\mu},e^a_{\mu})$. 

In order to get the field equations, we first vary with respect to $v^{\mu}$ and equate to zero:
\begin{equation}\label{eq:29}
    \tfrac{1}{2}\tau_{\mu}\big[K^2-  K^{\mu \nu} K_{\mu \nu}] - h^{\nu \alpha} \nabla'_{\alpha}[K_{\mu\nu}-Kh_{\mu\nu}\big]=0,
\end{equation}
where $\nabla'$ is the covariant derivative with respect to Levi-Civita connection associated to $h_{\mu\nu}$. Since $\tau_{\mu}$ and $h^{\mu \nu}$ are independent we can write two field equations as
\begin{equation}
\begin{aligned}
    K^2-  K^{\mu \nu} K_{\mu \nu} &=0,
    \\
    h^{\nu \alpha} \nabla'_{\alpha}\big[K_{\mu\nu}-Kh_{\mu\nu}\big] &=0.
\end{aligned}
\end{equation}
Now varying \eqref{eq:CarrolActionGR} with respect to $h^{\mu\nu}$ and equating to zero we get
\begin{equation}\label{eq:32}
     \tfrac{1}{2}h_{\mu\nu}\big[K^2-  K^{\mu \nu} K_{\mu \nu}\big] - KK_{\mu\nu} + v^{\alpha} \nabla'_{\alpha}K_{\mu\nu} + h_{\mu\nu}K^2 - v^{\alpha}\nabla'(h_{\mu\nu}K)=0.
\end{equation}  
From equation \eqref{eq:32} the first two terms vanish and we end up with the third field equations
\begin{equation}\label{eq:33}
    - KK_{\mu\nu} + v^{\alpha} \nabla'_{\alpha}K_{\mu\nu} + h_{\mu\nu}K^2 - v^{\alpha}\nabla_{\alpha}'(h_{\mu\nu}K) = 0.
\end{equation}
Following \cite{Hansen:2021fxi}, we can rearrange the last equation into
\begin{equation}\label{eq:34}
    \pounds_{\boldsymbol{v}} K_{\mu\nu}= -2 K_{\mu}^{\alpha}K_{\nu \alpha} + KK_{\mu\nu}.
\end{equation}
The left hand side is the Lie derivative of the extrinsic curvature of the leaves of the foliation (equal time submanifolds) with respect to $\boldsymbol{v}$, thus it tells us about the evolution of each point as we move along the integral curves of $v$ i.e. the evolution of each point on the leaves with time. Since it is only dependent on the extrinsic curvature $K_{\mu\nu}$, which is a Lie derivative of the induced metric $h_{\mu\nu}$, the evolution of the spacetime in GR is ultralocal. However, unlike the nonrelativistic limit \cite{Hansen:2020wqw}, the three field equations at the LO are non trivial and define a distinct theory. In order to support motion we need to go to the NLO terms, which represent corrections to Carrollian theory towards the full Lorentzian theory of GR.

\section{Carrollian expansion of quadratic gravity}\label{sec:5}

Quadratic gravity is a theory where quadratic curvature terms are added to the action, which makes it renormalizable \cite{Stelle:1976gc,Stelle:1977ry}. It also emerges from string theory by imposing a cutoff for the maximum possible momenta \cite{Nenmeli:2021orl}. The action for the theory is given by
\begin{equation}\label{eq:35}
    S= c^3\int   \big[R-\alpha R^{\mu \nu}R_{\mu\nu} + \beta R^2\big]\sqrt{-g}d^4x.
\end{equation}

In Sec.~\ref{sec:4} we computed the PUL parametrization of $R$. Now, we will do the same also for the other two terms in the action, $R^{\mu \nu}R_{\mu\nu}$ and $R^2$. Using \eqref{eq:19}, we can find the PUL parametrization of $R^{\mu \nu}R_{\mu\nu}$,
\begin{equation}\label{eq:36}
\begin{aligned}
    R^{\mu \nu}R_{\mu\nu} &= \tfrac{1}{c^4}\big[\Pi^{\nu\alpha}\Pi^{\lambda\beta} \nabla_{\mu}(V^{\mu}\mathcal{K}_{\alpha \beta})\nabla_{\rho}(V^{\rho}\mathcal{K}_{\nu \lambda}) -2 \mathcal{K}^{\alpha\beta}\mathcal{K}\nabla_{\mu}(V^{\mu}\mathcal{K}_{\alpha \beta}) + \mathcal{K}^{\lambda \nu}\mathcal{K}_{\lambda \nu} \mathcal{K}^2 \\ &\feq
    - V^{\lambda}V^{\nu}(\nabla_{\mu}\mathcal{K})(\nabla_{\nu}\mathcal{K}) + 2 \mathcal{K}_{\alpha\beta} \mathcal{K}^{\alpha \beta} V^{\nu}\nabla_{\nu}\mathcal{K}-(\mathcal{K}^{\mu\nu}\mathcal{K}_{\mu\nu})^2\big]\\&\feq
    + \tfrac{1}{c^2}\big[-2R^{\lambda \nu} \nabla_{\mu}(V^{\mu}\mathcal{K}_{\lambda \nu})- \Pi^{\nu \alpha}\nabla_{\mu}(V^{\mu}\mathcal{K}_{\alpha\beta})\mathcal{K}^{\rho \beta} (dT)_{\nu \rho} - \tfrac{1}{2}\Pi^{\lambda \beta} \mathcal{K}^{\rho \alpha} (dT)_{\lambda \rho} + 2R_{\lambda\nu}\mathcal{K}^{\lambda \nu} \mathcal{K}\\ &\feq
    + \mathcal{K}^{\lambda \nu} \mathcal{K} \mathcal{K}^{\alpha}_{\lambda} (dT)_{\nu \alpha} - \Pi^{\nu\alpha}\nabla_{\mu}(\mathcal{K}^{\mu}_{\alpha})\nabla_{\rho}\mathcal{K}^{\rho}_{\nu} + 2\Pi^{\nu\alpha}(\nabla_{\mu}\mathcal{K}^{\mu}_{\alpha})\nabla_{\nu}\mathcal{K} + V^{\lambda} (\nabla_{\mu}\mathcal{K}^{\mu}_{\alpha}) \mathcal{K}^{\rho \alpha} (dT)_{\lambda \rho} \\ &\feq
    - \Pi^{\nu\alpha}(\nabla_{\nu}\mathcal{K})(\nabla_{\alpha}\mathcal{K}) - V^{\lambda}(\nabla_{\alpha}\mathcal{K})\mathcal{K}^{\alpha \epsilon} (dT)_{\lambda \epsilon} - 2V^{\nu}V^{\alpha} \nabla_{\mu}(\Pi^{\mu\beta} (dT)_{\alpha\beta})\nabla_{\nu}\mathcal{K}\\ &\feq
    - V^{\alpha}\mathcal{K}^{\sigma \lambda} (dT)_{\sigma\alpha} V^{\rho} \mathcal{K}^{\beta}_{\lambda} (dT)_{\rho \beta} + 2 V^{\lambda} \nabla_{\mu}(\Pi^{\mu\rho} (dT_{\lambda\rho})) \mathcal{K}^{\alpha\beta}\mathcal{K}_{\alpha\beta}\\ &\feq
      - (dT)^{\lambda \nu} \nabla_{\mu}(V^{\mu}\mathcal{K}_{\lambda\nu}) + \tfrac{3}{2} \Pi^{\beta \nu} \Pi^{\alpha \lambda}(dT)_{\alpha \beta}\mathcal{K}_{\lambda\nu}\mathcal{K}\big]\\ &\feq
    +\big[\tfrac{1}{2} \mathcal{K}^{\alpha\beta}\mathcal{K}_{\alpha\beta}(dT)^{\mu\nu}(dT)_{\mu\nu} - \Pi^{\nu\alpha}(\nabla_{\alpha}\mathcal{K})\nabla_{\rho}(\Pi^{\rho\beta}(dT)_{\nu\beta}) - \tfrac{1}{4} V^{\alpha} \mathcal{K}^{\nu\rho} (dT)_{\sigma\rho} \nabla_{\mu}(\Pi^{\mu\alpha}(dT)_{\nu\alpha})\\ &\feq
     - \tfrac{1}{2} V^{\lambda} V^{\alpha} \nabla_{\mu}(\Pi^{\mu\beta}(dT)_{\alpha\beta})\nabla_{\nu}(\Pi^{\nu\rho}(dT)_{\lambda\rho}) + \tfrac{3}{2}R^{\alpha\lambda}\mathcal{K}^{\beta}_{\lambda}(dT)_{\alpha\beta}\\ &\feq
     + R^{\mu\nu}R_{\mu\nu} + \tfrac{1}{4} \mathcal{K}_{\beta\lambda}\mathcal{K}^{\rho\lambda}(dT)^{\nu\beta}(dT)_{\nu\rho} + \tfrac{1}{4}\mathcal{K}^{\beta\lambda}\mathcal{K}^{\alpha\rho}(dT)_{\alpha\beta}(dT)_{\lambda\rho}\big]\\ &\feq
    + c^2\big[\tfrac{-1}{4}(dT)^{\alpha\beta}(dT)_{\alpha\beta} \mathcal{K}^{\mu\nu}(dT)_{\mu\nu}\big]
    + c^4 \big[\tfrac{1}{16}((dT)^{\alpha\beta}(dT)_{\alpha\beta})^2\big].
\end{aligned}  
\end{equation}
By expanding this expression to the LO, we arrive at
\begin{equation}\label{eq:37}
\begin{aligned}
     R^{\mu \nu}R_{\mu\nu} &= \tfrac{1}{c^4}\big[h^{\nu\alpha}h^{\lambda\beta} \nabla_{\mu}(v^{\mu}K_{\alpha \beta})\nabla_{\rho}(v^{\rho}K_{\nu \lambda}) -2 K^{\alpha\beta} K \nabla_{\mu}(v^{\mu} K_{\alpha \beta}) + K^{\lambda \nu} K_{\lambda \nu} K^2 \\ &\feq
    - v^{\lambda}v^{\nu}\nabla_{\mu}(K)\nabla_{\nu}K + 2 K_{\alpha\beta} K^{\alpha \beta} v^{\nu}\nabla_{\nu}K-(K^{\mu\nu}K_{\mu\nu})^2\big].
    \end{aligned}
\end{equation}
The PUL parametrization of $R^2$ can be computed from \eqref{eq:20}, 
\begin{equation}\label{eq:38}
    \begin{aligned}
        R^2 &= \tfrac{1}{c^4}\big[\mathcal{K}^4 - 2\mathcal{K}^2\mathcal{K}^{\mu\nu}\mathcal{K}_{\mu\nu} - 4\mathcal{K}^2\nabla_{\nu}(V^{\nu}\mathcal{K})+ (\mathcal{K}^{\mu\nu}\mathcal{K}_{\mu\nu})^2 + 4\mathcal{K}^{\mu\nu}\mathcal{K}_{\mu\nu}\nabla_{\nu}(V^{\nu}\mathcal{K}) + 4\nabla_{\mu}(V^{\mu}\mathcal{K})\nabla_{\nu}(V^{\nu}\mathcal{K})\big]\\ &\feq
        + \tfrac{1}{c^2}\big[-\mathcal{K}^2R^c - \mathcal{K}^2\nabla_{\mu}(V^{\lambda}(dT)_{\lambda}^{\ \mu}) + \mathcal{K}_{\mu\nu}\mathcal{K}^{\mu\nu}R^c + \mathcal{K}_{\mu\nu}\mathcal{K}^{\mu\nu}\nabla_{\rho}(V^{\lambda}(dT)_{\lambda}^{\ \rho}) + 2R^c\nabla_{\mu}(V^{\mu}\mathcal{K}) \\ &\feq
        + 2\nabla_{\mu}(V^{\mu}\mathcal{K})\nabla_{\nu}(V^{\lambda}(dT)_{\lambda}^{\ \nu})\big] + \big[\tfrac{1}{2}\mathcal{K}(dT)^{\mu\nu}(dT)_{\mu\nu} - \tfrac{1}{2}\mathcal{K}^{\mu\nu}\mathcal{K}_{\mu\nu}(dT)^{\sigma \rho}(dT)_{\sigma \rho} - (dT)^{\mu\nu}(dT)_{\mu\nu} \nabla_{\rho}(V^{\rho}\mathcal{K}) \\ &\feq
        + (R^c)^2 + \nabla_{\mu}(V^{\lambda}(dT)_{\lambda}^{\ \mu})\nabla_{\rho}(V^{\sigma}(dT)_{\sigma}^{\ \rho}) + 2R^c \nabla_{\mu}(V^{\lambda}(dT)_{\lambda}^{\ \mu})\big] + c^2\big[-\tfrac{1}{4} R^c (dT)^{\mu\nu}(dT)_{\mu\nu} \\ &\feq
        - (dT)^{\mu\nu}(dT)_{\mu\nu} \nabla_{\rho}(V^{\sigma}(dT)_{\sigma}^{\ \rho})\big] + c^4\big[\tfrac{1}{16}((dT)_{\mu\nu}(dT)^{\mu\nu})^2\big],
    \end{aligned}
\end{equation}
and its Carrollian expansion to the LO is
\begin{equation}\label{eq:39}
    \begin{split}
        R^2 &= \tfrac{1}{c^4}\big[K^4 - 2K^2K^{\mu\nu}K_{\mu\nu} - 4K^2\nabla_{\nu}(v^{\nu}K)+ (K^{\mu\nu}K_{\mu\nu})^2 + 4K^{\mu\nu}K_{\mu\nu}\nabla_{\nu}(v^{\nu}K) + 4\nabla_{\mu}(v^{\mu}K)\nabla_{\nu}(v^{\nu}K)\big].
    \end{split}
\end{equation}

Substituting \eqref{eq:22}, \eqref{eq:37}, \eqref{eq:39} into the action \eqref{eq:35} we get
\begin{equation}\label{eq:40}
    \begin{aligned}
        S =  \int  &\big\{c^2\big[K^2- K^{\mu \nu} K_{\mu \nu}\big]- \alpha \big[h^{\nu\alpha}h^{\lambda\beta} \nabla_{\mu}(v^{\mu}K_{\alpha \beta})\nabla_{\rho}(v^{\rho}K_{\nu \lambda}) -2 K^{\alpha\beta} K \nabla_{\mu}(v^{\mu} K_{\alpha \beta}) + K^{\lambda \nu} K_{\lambda \nu} K^2\\&\feq
    - v^{\lambda}v^{\nu}\nabla_{\mu}(K)\nabla_{\nu}K + 2 K_{\alpha\beta} K^{\alpha \beta} v^{\nu}\nabla_{\nu}K-(K^{\mu\nu}K_{\mu\nu})^2\big]+ \beta \big[K^4 - 2K^2K^{\mu\nu}K_{\mu\nu} 
    \\ &\feq
    - 4K^2\nabla_{\nu}(v^{\nu}K)+ (K^{\mu\nu}K_{\mu\nu})^2 + 4K^{\mu\nu}K_{\mu\nu}\nabla_{\nu}(v^{\nu}K) + 4\nabla_{\mu}(v^{\mu}K)\nabla_{\nu}(v^{\nu}K)\big]\big\} e d^4 x  .
    \end{aligned}
\end{equation}
Interestingly, this formula can be rewritten in terms of Lie derivatives of the extrinsic curvature. In order to do that, we first write
\begin{equation}
    \pounds_{\boldsymbol{v}}K_{\mu\nu}=v^{\sigma}\nabla_{\sigma}K_{\mu\nu}+K_{\sigma\nu}\nabla_{\mu}v^{\sigma}+K_{\sigma\mu}\nabla_{\nu}v^{\sigma} - K_{\mu\nu}\nabla_{\sigma}v^{\sigma}+v^{\sigma}T^{\rho}_{\sigma\mu}K_{\rho\nu} + v^{\sigma}T^{\rho}_{\sigma\nu}K_{\rho\mu},
\end{equation}
where $T^{\rho}_{\mu\nu}$ is the torsion of the connection defined in Sec.~\ref{sec:2}. Since in the PUL-parameterization vector $v^{\mu}$ is covariantly constant by definition and $T^{\rho}_{\mu\nu}$ is given by \eqref{torsion}, the relation reduces to
\begin{equation}\label{derivative relation}
    \pounds_{\boldsymbol{v}} K_{\mu\nu}=v^{\sigma}\nabla_{\sigma}K_{\mu\nu}- K^{\sigma}_{(\mu}K_{\nu)\sigma}.
\end{equation}
Substituting in \eqref{eq:40} and using the fact that $v^{\sigma}\nabla_{\sigma}$ acts on scalars simply as $\pounds_{\boldsymbol{v}}$, we get
\begin{equation}\label{lie der action}
    \begin{aligned}
      S =  \int &\big\{c^2\big[K^2- K^{\mu \nu} K_{\mu \nu}\big]- \alpha  \big[   h^{\nu\alpha}h^{\lambda\beta}\pounds_{\boldsymbol{v}}(K_{\alpha\beta})\pounds_{\boldsymbol{v}}(K_{\nu\lambda}) + 2 \pounds_{\boldsymbol{v}}(K_{\nu\lambda})K^{\sigma(\nu}K^{\lambda)}_{\sigma} \\ 
      &\phantom{=} + K^{\sigma}_{(\alpha}K_{\beta)\sigma}K^{\rho(\alpha}K^{\beta)}_{\rho} -2K^{\alpha\beta}K\pounds_{\boldsymbol{v}}K_{\alpha\beta}-2K^{\alpha\beta}KK^{\sigma}_{(\alpha}K_{\beta)\sigma}+ K^2K^{\mu\nu}K_{\mu\nu} \\ 
      &\phantom{=}- (\pounds_{\boldsymbol{v}} K)^2 +2K_{\mu\nu}K^{\mu\nu}\pounds_{\boldsymbol{v}} K - (K_{\mu\nu}K^{\mu\nu})^2 \big] + \beta \big[  
 K^4 - 2K^2K_{\mu\nu}K^{\mu\nu} \\
 &\phantom{=} - 4K^2\pounds_{\boldsymbol{v}} K + (K_{\mu\nu}K^{\mu\nu})^2 + 4K_{\mu\nu}K^{\mu\nu} \pounds_{\boldsymbol{v}} K + (\pounds_{\boldsymbol{v}} K)^2\big]
      \big\}ed^4 x   
    \end{aligned}
\end{equation}
Note that only the first two terms have the $c^2$ factor. Thus, assuming $\alpha$ and $\beta$ being independent of $c$, the Carrollian limit of the theory will exclude the first two terms coming from the Carrollian limit of the Ricci scalar. This means that the resulting theory will not couple to $R$ and it will be drastically different than the Carrollian limit of GR. This means that $\alpha$ and $\beta$ should depend on $c$. In this case, we get an infinite number of nonequivalent Carrollian theories, only 4 of them modify GR to LO or NLO.

 \section{Theories from the Carrollian limit of quadratic gravity}\label{sec:6}

In this section, we study Carrollian theories resulting from the Carrollian limit of quadratic gravity. Different (nonequivalent) theories arise from assuming different dependencies of $\alpha$ and $\beta$ on the speed of light $c$ in \eqref{lie der action}. Thus, we classify them as such and denote them by $(n,m)$, where $\alpha = c^n\alpha'$  and $\beta = c^m\beta'$. The relevant theories are listed in Tab.~\ref{table:1}. As mentioned above, not all theories are modifications to GR. For example, the theories with negative powers of $c$ in $\alpha$ or $\beta$ but also (0,0), (0,2) and (2,0) are not physically interesting since they are drastically different from GR at LO. It is easy to see that dependencies with odd powers of $c$ ultimately converge to one of the theories in Tab. \ref{table:1}. Theories with higher power dependencies on $c$ can not modify GR to the LO nor the NLO but to higher orders however, since $\alpha$ and $\beta$ dependencies on $c$ is a non perturbative assumption, having high powers of $c$ in the action without being an overall factor can lead to inconsistencies in the Galilean limit. Thus, in what follows, we focus only on the the four interesting Carrollian theories (2,2), (2,4), (4,2), and (4,4).
 
\setlength{\arrayrulewidth}{0.5mm}
\renewcommand{\arraystretch}{2}
\setlength{\tabcolsep}{10pt}
\begin{table}
 \begin{tabular}{ |p{0.9cm}|p{6.6cm}|p{7cm}|  }
 \hline
 \multicolumn{3}{|c|}{Carrollian theories from quadratic gravity} \\
 \hline
 Theory  & Action contributing to the LO & Type of modification to the Carrollian limit of GR\\
 \hline
 (0,0) & $ S= c^3 \int  \big[-\alpha R^{\mu\nu}R_{\mu\nu}+\beta R^2\big] \sqrt{-g}d^4 x  $ & \textit{Not a modification of GR} \\
 \hline
 (0,2)   & $S= c^3 \int \big[-\alpha R^{\mu\nu}R_{\mu\nu}\big] \sqrt{-g}d^4 x$ & \textit{Not a modification of GR}\\
 \hline
 (2,0) & $ S= c^3 \int \big[\beta R^2\big] \sqrt{-g}d^4 x $ & \textit{Not a modification of GR} \\
 \hline
 (2,2)    & $S= c^3 \int \big[\frac{1}{16\pi G}R-\alpha R^{\mu \nu}R_{\mu\nu} + \beta R^2\big] \sqrt{-g}d^4 x$ & \textit{Modifies GR to the LO}\\
 \hline
 (2,4) &  $S= c^3 \int \big[\frac{1}{16\pi G}R-\alpha R^{\mu \nu}R_{\mu\nu} \big]\sqrt{-g}d^4 x$   & \textit{Modifies GR to the LO with $R^{\mu\nu}R_{\mu\nu}$ terms and the NLO by $R^2$ terms}\\
 \hline
 (4,2) & $S= c^3 \int \big[\frac{1}{16\pi G}R + \beta R^2\big] \sqrt{-g}d^4 x$  & \textit{Modifies GR to the LO with $R^2$ terms and the NLO by $R^{\mu\nu}R_{\mu\nu}$ terms} \\
 \hline
 (4,4) & $S= c^3\int \big[\frac{1}{16\pi G}R\big]\sqrt{-g}d^4 x$ & \textit{Modifies GR in the NLO}\\
 \hline
\end{tabular}
\caption{This table summarizes some possible Carrollian theories arising from quadratic gravity that couple to $R$ at most in the NLO. We list the theories with factors of $c$ with non-negative powers since negative $c$ dependencies are clearly not modifications of the Carrollian limit of GR. For example, although (0,0) can not be a modification to the Carrollian limit of GR, we can say that $R$ terms are a NLO modification of this theory. There are other geometries which are modifications to the listed geometries like (0,4) which can be regarded as a next to next to leading order modification of (0,2) while GR itself is the NLO. We can extend the list indefinitely adding more geometries modifying GR to higher orders but here we focus on the LO and NLO.}
\label{table:1}
\end{table}



\subsection{(2,2) Carrollian theory}

Consider the case where $\alpha$ and $\beta$ are quadratic in the speed of light, ${\alpha = c^2 \alpha'}$, ${\beta = c^2\beta'}$, with $\alpha'$ and $\beta'$ being constants independent of the speed of light $c$. We will study the resulting action to the LO. From Tab.~\ref{table:1}, the action is
\begin{equation}\label{eq:41}
    S=  c^3\int \big[R-\alpha R^{\mu \nu}R_{\mu\nu} + \beta R^2\big] \sqrt{-g}d^4x.
\end{equation}
Writing $\alpha= c^2 \alpha'$ and $\beta = c^2 \beta'$, where $\alpha'$ and $\beta'$ are $c$ independent constants, we can write the action as
\begin{equation}\label{eq:42}
 S= \int  c^3 \big[R-c^2\alpha' R^{\mu \nu}R_{\mu\nu} + c^2 \beta' R^2\big]\sqrt{-g}d^4x, 
\end{equation}
which in the LO of the Carrollian expansion gives
\begin{equation}\label{eq:43}
\begin{aligned}
S =  c^2\int    &\big\{ \big[K^2- K^{\mu \nu} K_{\mu \nu}\big]- \alpha'  \big[   h^{\nu\alpha}h^{\lambda\beta}(\pounds_{\boldsymbol{v}}K_{\alpha\beta})(\pounds_{\boldsymbol{v}}K_{\nu\lambda}) + 2 (\pounds_{\boldsymbol{v}}K_{\nu\lambda})K^{\sigma(\nu}K^{\lambda)}_{\sigma} \\ 
      &\phantom{=} + K^{\sigma}_{(\alpha}K_{\beta)\sigma}K^{\rho(\alpha}K^{\beta)}_{\rho} -2K^{\alpha\beta}K\pounds_{\boldsymbol{v}}K_{\alpha\beta}-2K^{\alpha\beta}KK^{\sigma}_{(\alpha}K_{\beta)\sigma}+ K^2K^{\mu\nu}K_{\mu\nu} \\ 
      &\phantom{=}- (\pounds_{\boldsymbol{v}}K)^2 +2K_{\mu\nu}K^{\mu\nu}\pounds_{\boldsymbol{v}}K - (K_{\mu\nu}K^{\mu\nu})^2 \big] + \beta' \big[  
 K^4 - 2K^2K_{\mu\nu}K^{\mu\nu} \\
 &\phantom{=} - 4K^2\pounds_{\boldsymbol{v}}K + (K_{\mu\nu}K^{\mu\nu})^2 + 4K_{\mu\nu}K^{\mu\nu} \pounds_{\boldsymbol{v}}K + (\pounds_{\boldsymbol{v}}K)^2\big]
      \big\} ed^4 x
\end{aligned}
\end{equation}

Since the Carrollian expansion and the weak field regime are not conflicting, the conditions to find tachyons remain the same. In \cite{Stelle:1977ry} it was found that the additional degrees of freedom have masses of\footnote{Remark that $\alpha$ and $\beta$ in our convention have opposite signs to the convention used in \cite{Stelle:1977ry}.} 
\begin{subequations}\label{eq:44}
    \begin{align}
         m_0 &=\frac{\sqrt{2}}{32 \pi G}\frac{1}{\sqrt{-\alpha}},\\ 
        m_2 &= \frac{1}{32 \pi G} \frac{1}{\sqrt{\alpha-3\beta}}.
    \end{align}
\end{subequations}
The conditions to avoid tachyons are (at any order of the Carrollian expansion)
\begin{subequations}\label{tachyonremoving}
    \begin{align}
        \alpha \leq 0,\\
        \alpha -3\beta \geq 0,
    \end{align}
\end{subequations}
which translates to
\begin{subequations}\label{eq:45}
    \begin{align}
        \alpha' \leq 0,\\
        \alpha' -3\beta' \geq 0,
    \end{align}
\end{subequations}
in the case of (2,2) theory.

\subsection{(2,4) Carrollian theory}

Let us now investigate the case where ${\alpha = c^2 \alpha'}$ and ${\beta = c^4\beta'}$. The action is
\begin{equation}\label{eq:46}
    S= c^3\int \big[R-c^2\alpha' R^{\mu \nu}R_{\mu\nu} + c^4 \beta' R^2\big]\sqrt{-g}d^4x.
\end{equation}
To the LO in the Carrollian expansion, we get the action
\begin{equation}\label{eq:47}
\begin{aligned}
    S= c^2 \int   &  \big\{\big[K^2-K^{\mu\nu}K_{\mu\nu}\big] - \alpha' \big[   h^{\nu\alpha}h^{\lambda\beta}(\pounds_{\boldsymbol{v}}K_{\alpha\beta})(\pounds_{\boldsymbol{v}}K_{\nu\lambda}) + 2 (\pounds_{\boldsymbol{v}}K_{\nu\lambda})K^{\sigma(\nu}K^{\lambda)}_{\sigma} \\ 
      &\phantom{=} + K^{\sigma}_{(\alpha}K_{\beta)\sigma}K^{\rho(\alpha}K^{\beta)}_{\rho} -2K^{\alpha\beta}K\pounds_{\boldsymbol{v}}K_{\alpha\beta}-2K^{\alpha\beta}KK^{\sigma}_{(\alpha}K_{\beta)\sigma}+ K^2K^{\mu\nu}K_{\mu\nu} \\ 
      &\phantom{=}- (\pounds_{\boldsymbol{v}}K)^2 +2K_{\mu\nu}K^{\mu\nu}\pounds_{\boldsymbol{v}}K - (K_{\mu\nu}K^{\mu\nu})^2 \big\}e d^4x.
\end{aligned}
\end{equation}

Notice that this theory is the same as the Carrollian limit of $R-\alpha R_{\mu\nu}R^{\mu\nu}$. The conditions \eqref{tachyonremoving} to the LO reduce to $\alpha' = 0$. Thus, to the LO, the theory without tachyons is the same as the Carrollian limit of GR.

Assuming $\alpha'$ and $\beta'$ to be of the same numerical order, the conditions to the LO and NLO respectively are 
\begin{subequations}\label{eq:48}
\begin{align}
     \alpha'=0,\\
     \beta' \leq 0.
     \end{align}
\end{subequations}
Thus, the theory without tachyons to the NLO would be
\begin{equation} \label{(2,4)theory}
    S= c^3\int\big[ R_{NLO} 
+ c^4 \beta'(R^2)_{LO} \big] \sqrt{-g}d^4x,
\end{equation}
where $R_{NLO}$ is the Ricci scalar expanded to the NLO and $(R^2)_{LO}$ is the LO of the Carrollian expansion of $R^2$.

\subsection{(4,2) Carrollian theory}

Considering the dependencies are $\alpha = c^4\alpha'$ and $\beta = c^2 \beta'$, the action is
\begin{equation}\label{eq:49}
     S= c^3\int \big[R-c^4\alpha' R^{\mu \nu}R_{\mu\nu} + c^2 \beta' R^2\big]  \sqrt{-g}d^4x.
\end{equation}
The corresponding LO action reads
\begin{equation}\label{eq:50}
    \begin{aligned}
    S=c^2\int & \big[\big(K^2-K^{\mu\nu}K_{\mu\nu}\big)+ \beta' \big[  
 K^4 - 2K^2K_{\mu\nu}K^{\mu\nu} \\
 &\phantom{=} - 4K^2\pounds_{\boldsymbol{v}}K + (K_{\mu\nu}K^{\mu\nu})^2 + 4K_{\mu\nu}K^{\mu\nu} \pounds_{\boldsymbol{v}}K + (\pounds_{\boldsymbol{v}}K)^2\big]e d^4x.
    \end{aligned}
\end{equation}

In this case the conditions \eqref{tachyonremoving} then reduce to ${\beta' \leq 0}$. Expanding the conditions to the NLO we obtain
\begin{subequations}
    \begin{align}
        \beta' \leq 0,\\
        \alpha'=0.
    \end{align}
\end{subequations}
Hence, this theory is equivalent to the Carrollian limit of $R-\beta R^2$ theory to all orders with NLO action being the same as \eqref{(2,4)theory}.

\subsection{(4,4) Carrollian theory}

If we consider $\alpha = c^4\alpha'$ and $\beta = c^4 \beta'$, then the action reads
\begin{equation}\label{eq:52}
     S= c^3\int   \big[R-c^4\alpha' R^{\mu \nu}R_{\mu\nu} + c^4 \beta' R^2\big]\sqrt{-g}d^4x.
\end{equation}
For this theory, the LO action is the same as GR. At the NLO and higher orders it will receive corrections from both $R^2$ and $R_{\mu\nu}R^{\mu\nu}$ terms. The conditions \eqref{tachyonremoving} are the same as in the (2,2) case.


\section{Conclusions} \label{sec:concl}

In this paper, we studied the electric Carrollian limit of quadratic gravity. We calculated the PUL parametrization of terms with quadratic curvature in the action. After the Carrollian expansion, we saw that such terms are of the order of ${c^{-4}}$ while the Ricci scalar term is only of the order of ${c^{-2}}$. From that, we concluded that the Carrollian limit of quadratic gravity requires $\alpha$ and $\beta$ to depend on $c$ in a particular way so that the resulting theory is a modification of GR. We classified different limits according to the dependencies of $\alpha$ and $\beta$ on $c$. For example, the three of them $(0,0)$ (no dependence on $c$), $(0,2)$, and $(2,0)$ are not GR modifications because to the leading order only terms of order ${c^{-4}}$ survive, i.e., only the quadratic terms in curvature but not the Ricci scalar. The only four theories that are modifications of GR (to the LO and NLO) are summarized in Tab.~\ref{table:2} together with the corresponding modifications.
\setlength{\arrayrulewidth}{0.5mm}
\renewcommand{\arraystretch}{2}
\setlength{\tabcolsep}{10pt}
\begin{table}
 \begin{tabular}{ |p{0.9cm}|p{5cm}|p{5.7cm}|p{1.7cm}|}
 \hline
 \multicolumn{4}{|c|}{Carrollian theories from quadratic gravity after removing tachyons} \\
 \hline
 Theory & Action contributing to the LO & Action contributing to the NLO & Conditions\\
 \hline
 (2,2)    & 
 $\begin{aligned}[t]
     S= c^3{\textstyle\int} \big[R_{LO}-c^2\alpha' (R^{\mu \nu}R_{\mu\nu})_{LO} \\ + c^2\beta' (R^2)_{LO}\big]\sqrt{-g}d^4x
     \end{aligned}$ & $\begin{aligned}[t] S= c^3{\textstyle\int} \big[R_{NLO} -c^2\alpha' (R^{\mu \nu}R_{\mu\nu})_{NLO}\\  + c^2\beta' (R^2)_{NLO}\big]\sqrt{-g}d^4x \end{aligned}$ & $\begin{aligned}[t] &\alpha' \leq 0, \\ &\alpha' -3\beta' \geq 0\end{aligned}$\\
 \hline
 (2,4) &  $\begin{aligned}[t] S= c^3{\textstyle\int} \big[R_{LO} \big]\sqrt{-g}d^4x\end{aligned}$   & $\begin{aligned}[t] S= c^3{\textstyle\int} \big[R_{NLO} + c^4\beta'(R^2)_{LO} \big]\sqrt{-g}d^4x \end{aligned}$ & $\begin{aligned}[t] \beta' \leq 0\end{aligned}$\\
 \hline
 (4,2) & $\begin{aligned}[t] S=  c^3{\textstyle\int} \big[R_{LO} + c^2\beta' (R^2)_{LO}\big]\sqrt{-g}d^4x \end{aligned}$  & $\begin{aligned}[t] S= c^3{\textstyle\int} \big[R_{NLO} + c^4 \beta' (R^2)_{NLO}\big]\sqrt{-g}d^4x \end{aligned}$ & $\beta' \leq 0$\\
 \hline
 (4,4) & $\begin{aligned}[t] S= c^3{\textstyle\int} \big[R_{LO}\big]\sqrt{-g}d^4x \end{aligned}$ & $\begin{aligned}[t] S= c^3{\textstyle\int} \big[R_{NLO} -c^4\alpha' (R^{\mu \nu}R_{\mu\nu})_{NLO} \\+ c^4\beta' (R^2)_{NLO}\big]\sqrt{-g}d^4x \end{aligned}$ & $\begin{aligned}[t] &\alpha' \leq 0, \\ &\alpha' -3\beta' \geq 0 \end{aligned}$\\
 \hline
\end{tabular}
\caption{After imposing the conditions to remove tachyons, the set of resulting theories consists either of the full Stelle gravity to various orders or variations of $R+R^2$ theories. It is worth mentioning that, as said before, theories with odd powers of $c$ will be equivalent to one of the theories above, and higher powers of $c$ may be problematic in the Galilean limit. Note that the LO actions possess Carrollian symmetries by construction so they are Carrollian theories, but the NLO action do not. The NLO of the Carrollian expansion does not preserve Carrollian symmetry in general, however, certain truncation recovers the symmetries resulting in the magnetic Carrollian limit of the theory.}
\label{table:2}
\end{table}

More work has to be done to study the field equations for these theories to the LO and NLO. It would be interesting to compare each case with GR to understand what modifications can arise from different quartic terms of the extrinsic curvature. Another direction for future research is to calculate the Galilean limit of quadratic gravity. Since the dependence of $\alpha$ and $\beta$ on $c$ is not a perturbative assumption, the higher powers of $c$ in the action may be problematic in the Galilean limit. In the current classification the most attractive options for future study are $(2,4)$ and $(4,2)$ since, after imposing the tachyon removing conditions, we get the Carrollian limit of $R+\beta R^2$, a renormalizable theory with no ghosts or tachyons (only if $\beta$ is positive) which is deduced directly from the string theory. We plan to study the magnetic Carrollian limit of this theory. Since the theory has more black hole solutions than GR, it is interesting to analyze the dynamics of Carrollian particles on horizons of various black hole solutions and compare the dynamics with that of \cite{Bicak:2023vxs} and study the modifications arising from the quartic terms. 

%%%%%%%%%%%%%%%%%%%%%%%%%%%%%%%%%%%%%%%%%%%%%%%%%%%%%%%%%%%%%%%%%%%%%%%%%%%%%%%%%%%%%%
%% ACKNOWLEDGEMENTS

\section*{Acknowledgements}

The authors would like to thank Eric Bergshoeff (Groningen, Netherlands), Pavel Krtou\v{s}, David Kubiz\v{n}\'ak (Prague, Czechia), and Marc Henneaux (Brussels, Belgium) for stimulating discussions. P.T. and I.K. were supported by Primus grant PRIMUS/23/SCI/005 from Charles University.


%%%%%%%%%%%%%%%%%%%%%%%%%%%%%%%%%%%%%%%%%%%%%%%%%%%%%%%%%%%%%%%%%%%%%%%%%%%%%%%%%%%%%%
%% Appendix

%\appendix



%%%%%%%%%%%%%%%%%%%%%%%%%%%%%%%%%%%%%%%%%%%%%%%%%%%%%%%%%%%%%%%%%%%%%%%%%%%%%%%%%%%%%%
%% REFERENCES

%apsrev4-2.bst 2019-01-14 (MD) hand-edited version of apsrev4-1.bst
%Control: key (0)
%Control: author (8) initials jnrlst
%Control: editor formatted (1) identically to author
%Control: production of article title (0) allowed
%Control: page (0) single
%Control: year (1) truncated
%Control: production of eprint (0) enabled
\begin{thebibliography}{75}%
\makeatletter
\providecommand \@ifxundefined [1]{%
 \@ifx{#1\undefined}
}%
\providecommand \@ifnum [1]{%
 \ifnum #1\expandafter \@firstoftwo
 \else \expandafter \@secondoftwo
 \fi
}%
\providecommand \@ifx [1]{%
 \ifx #1\expandafter \@firstoftwo
 \else \expandafter \@secondoftwo
 \fi
}%
\providecommand \natexlab [1]{#1}%
\providecommand \enquote  [1]{``#1''}%
\providecommand \bibnamefont  [1]{#1}%
\providecommand \bibfnamefont [1]{#1}%
\providecommand \citenamefont [1]{#1}%
\providecommand \href@noop [0]{\@secondoftwo}%
\providecommand \href [0]{\begingroup \@sanitize@url \@href}%
\providecommand \@href[1]{\@@startlink{#1}\@@href}%
\providecommand \@@href[1]{\endgroup#1\@@endlink}%
\providecommand \@sanitize@url [0]{\catcode `\\12\catcode `\$12\catcode
  `\&12\catcode `\#12\catcode `\^12\catcode `\_12\catcode `\%12\relax}%
\providecommand \@@startlink[1]{}%
\providecommand \@@endlink[0]{}%
\providecommand \url  [0]{\begingroup\@sanitize@url \@url }%
\providecommand \@url [1]{\endgroup\@href {#1}{\urlprefix }}%
\providecommand \urlprefix  [0]{URL }%
\providecommand \Eprint [0]{\href }%
\providecommand \doibase [0]{https://doi.org/}%
\providecommand \selectlanguage [0]{\@gobble}%
\providecommand \bibinfo  [0]{\@secondoftwo}%
\providecommand \bibfield  [0]{\@secondoftwo}%
\providecommand \translation [1]{[#1]}%
\providecommand \BibitemOpen [0]{}%
\providecommand \bibitemStop [0]{}%
\providecommand \bibitemNoStop [0]{.\EOS\space}%
\providecommand \EOS [0]{\spacefactor3000\relax}%
\providecommand \BibitemShut  [1]{\csname bibitem#1\endcsname}%
\let\auto@bib@innerbib\@empty
%</preamble>
\bibitem [{\citenamefont {Zumino}(1986)}]{ZUMINO1986109}%
  \BibitemOpen
  \bibfield  {author} {\bibinfo {author} {\bibfnamefont {B.}~\bibnamefont
  {Zumino}},\ }\bibfield  {title} {\bibinfo {title} {Gravity theories in more
  than four dimensions},\ }\href
  {https://doi.org/https://doi.org/10.1016/0370-1573(86)90076-1} {\bibfield
  {journal} {\bibinfo  {journal} {Physics Reports}\ }\textbf {\bibinfo {volume}
  {137}},\ \bibinfo {pages} {109} (\bibinfo {year} {1986})}\BibitemShut
  {NoStop}%
\bibitem [{\citenamefont {Zwiebach}(1985)}]{ZWIEBACH1985315}%
  \BibitemOpen
  \bibfield  {author} {\bibinfo {author} {\bibfnamefont {B.}~\bibnamefont
  {Zwiebach}},\ }\bibfield  {title} {\bibinfo {title} {Curvature squared terms
  and string theories},\ }\href
  {https://doi.org/https://doi.org/10.1016/0370-2693(85)91616-8} {\bibfield
  {journal} {\bibinfo  {journal} {Physics Letters B}\ }\textbf {\bibinfo
  {volume} {156}},\ \bibinfo {pages} {315} (\bibinfo {year}
  {1985})}\BibitemShut {NoStop}%
\bibitem [{\citenamefont {Forger}\ \emph {et~al.}(1996)\citenamefont {Forger},
  \citenamefont {Ovrut}, \citenamefont {Theisen},\ and\ \citenamefont
  {Waldram}}]{Forger:1996vj}%
  \BibitemOpen
  \bibfield  {author} {\bibinfo {author} {\bibfnamefont {K.}~\bibnamefont
  {Forger}}, \bibinfo {author} {\bibfnamefont {B.~A.}\ \bibnamefont {Ovrut}},
  \bibinfo {author} {\bibfnamefont {S.~J.}\ \bibnamefont {Theisen}},\ and\
  \bibinfo {author} {\bibfnamefont {D.}~\bibnamefont {Waldram}},\ }\bibfield
  {title} {\bibinfo {title} {{Higher derivative gravity in string theory}},\
  }\href {https://doi.org/10.1016/S0370-2693(96)01175-6} {\bibfield  {journal}
  {\bibinfo  {journal} {Phys. Lett. B}\ }\textbf {\bibinfo {volume} {388}},\
  \bibinfo {pages} {512} (\bibinfo {year} {1996})},\ \Eprint
  {https://arxiv.org/abs/hep-th/9605145} {arXiv:hep-th/9605145} \BibitemShut
  {NoStop}%
\bibitem [{\citenamefont {Myers}(1987)}]{Myers:1987yn}%
  \BibitemOpen
  \bibfield  {author} {\bibinfo {author} {\bibfnamefont {R.~C.}\ \bibnamefont
  {Myers}},\ }\bibfield  {title} {\bibinfo {title} {{Higher Derivative Gravity,
  Surface Terms and String Theory}},\ }\href
  {https://doi.org/10.1103/PhysRevD.36.392} {\bibfield  {journal} {\bibinfo
  {journal} {Phys. Rev. D}\ }\textbf {\bibinfo {volume} {36}},\ \bibinfo
  {pages} {392} (\bibinfo {year} {1987})}\BibitemShut {NoStop}%
\bibitem [{\citenamefont {Alvarez-Gaume}\ \emph {et~al.}(2016)\citenamefont
  {Alvarez-Gaume}, \citenamefont {Kehagias}, \citenamefont {Kounnas},
  \citenamefont {L\"ust},\ and\ \citenamefont
  {Riotto}}]{Alvarez-Gaume:2015rwa}%
  \BibitemOpen
  \bibfield  {author} {\bibinfo {author} {\bibfnamefont {L.}~\bibnamefont
  {Alvarez-Gaume}}, \bibinfo {author} {\bibfnamefont {A.}~\bibnamefont
  {Kehagias}}, \bibinfo {author} {\bibfnamefont {C.}~\bibnamefont {Kounnas}},
  \bibinfo {author} {\bibfnamefont {D.}~\bibnamefont {L\"ust}},\ and\ \bibinfo
  {author} {\bibfnamefont {A.}~\bibnamefont {Riotto}},\ }\bibfield  {title}
  {\bibinfo {title} {{Aspects of Quadratic Gravity}},\ }\href
  {https://doi.org/10.1002/prop.201500100} {\bibfield  {journal} {\bibinfo
  {journal} {Fortsch. Phys.}\ }\textbf {\bibinfo {volume} {64}},\ \bibinfo
  {pages} {176} (\bibinfo {year} {2016})},\ \Eprint
  {https://arxiv.org/abs/1505.07657} {arXiv:1505.07657 [hep-th]} \BibitemShut
  {NoStop}%
\bibitem [{\citenamefont {Nenmeli}\ \emph {et~al.}(2021)\citenamefont
  {Nenmeli}, \citenamefont {Shankaranarayanan}, \citenamefont {Todorinov},\
  and\ \citenamefont {Das}}]{Nenmeli:2021orl}%
  \BibitemOpen
  \bibfield  {author} {\bibinfo {author} {\bibfnamefont {V.}~\bibnamefont
  {Nenmeli}}, \bibinfo {author} {\bibfnamefont {S.}~\bibnamefont
  {Shankaranarayanan}}, \bibinfo {author} {\bibfnamefont {V.}~\bibnamefont
  {Todorinov}},\ and\ \bibinfo {author} {\bibfnamefont {S.}~\bibnamefont
  {Das}},\ }\bibfield  {title} {\bibinfo {title} {{Maximal momentum GUP leads
  to quadratic gravity}},\ }\href
  {https://doi.org/10.1016/j.physletb.2021.136621} {\bibfield  {journal}
  {\bibinfo  {journal} {Phys. Lett. B}\ }\textbf {\bibinfo {volume} {821}},\
  \bibinfo {pages} {136621} (\bibinfo {year} {2021})},\ \Eprint
  {https://arxiv.org/abs/2106.04141} {arXiv:2106.04141 [gr-qc]} \BibitemShut
  {NoStop}%
\bibitem [{\citenamefont {Stelle}(1978)}]{Stelle:1977ry}%
  \BibitemOpen
  \bibfield  {author} {\bibinfo {author} {\bibfnamefont {K.~S.}\ \bibnamefont
  {Stelle}},\ }\bibfield  {title} {\bibinfo {title} {{Classical Gravity with
  Higher Derivatives}},\ }\href {https://doi.org/10.1007/BF00760427} {\bibfield
   {journal} {\bibinfo  {journal} {Gen. Rel. Grav.}\ }\textbf {\bibinfo
  {volume} {9}},\ \bibinfo {pages} {353} (\bibinfo {year} {1978})}\BibitemShut
  {NoStop}%
\bibitem [{\citenamefont {Stelle}(1977)}]{Stelle:1976gc}%
  \BibitemOpen
  \bibfield  {author} {\bibinfo {author} {\bibfnamefont {K.~S.}\ \bibnamefont
  {Stelle}},\ }\bibfield  {title} {\bibinfo {title} {{Renormalization of Higher
  Derivative Quantum Gravity}},\ }\href
  {https://doi.org/10.1103/PhysRevD.16.953} {\bibfield  {journal} {\bibinfo
  {journal} {Phys. Rev. D}\ }\textbf {\bibinfo {volume} {16}},\ \bibinfo
  {pages} {953} (\bibinfo {year} {1977})}\BibitemShut {NoStop}%
\bibitem [{\citenamefont {Julve}\ and\ \citenamefont
  {Tonin}(1978)}]{Julve:1978xn}%
  \BibitemOpen
  \bibfield  {author} {\bibinfo {author} {\bibfnamefont {J.}~\bibnamefont
  {Julve}}\ and\ \bibinfo {author} {\bibfnamefont {M.}~\bibnamefont {Tonin}},\
  }\bibfield  {title} {\bibinfo {title} {{Quantum Gravity with Higher
  Derivative Terms}},\ }\href {https://doi.org/10.1007/BF02748637} {\bibfield
  {journal} {\bibinfo  {journal} {Nuovo Cim. B}\ }\textbf {\bibinfo {volume}
  {46}},\ \bibinfo {pages} {137} (\bibinfo {year} {1978})}\BibitemShut
  {NoStop}%
\bibitem [{\citenamefont {Lu}\ \emph {et~al.}(2015)\citenamefont {Lu},
  \citenamefont {Perkins}, \citenamefont {Pope},\ and\ \citenamefont
  {Stelle}}]{Lu:2015cqa}%
  \BibitemOpen
  \bibfield  {author} {\bibinfo {author} {\bibfnamefont {H.}~\bibnamefont
  {Lu}}, \bibinfo {author} {\bibfnamefont {A.}~\bibnamefont {Perkins}},
  \bibinfo {author} {\bibfnamefont {C.~N.}\ \bibnamefont {Pope}},\ and\
  \bibinfo {author} {\bibfnamefont {K.~S.}\ \bibnamefont {Stelle}},\ }\bibfield
   {title} {\bibinfo {title} {{Black Holes in Higher-Derivative Gravity}},\
  }\href {https://doi.org/10.1103/PhysRevLett.114.171601} {\bibfield  {journal}
  {\bibinfo  {journal} {Phys. Rev. Lett.}\ }\textbf {\bibinfo {volume} {114}},\
  \bibinfo {pages} {171601} (\bibinfo {year} {2015})},\ \Eprint
  {https://arxiv.org/abs/1502.01028} {arXiv:1502.01028 [hep-th]} \BibitemShut
  {NoStop}%
\bibitem [{\citenamefont {L\"u}\ \emph {et~al.}(2015)\citenamefont {L\"u},
  \citenamefont {Perkins}, \citenamefont {Pope},\ and\ \citenamefont
  {Stelle}}]{Lu:2015psa}%
  \BibitemOpen
  \bibfield  {author} {\bibinfo {author} {\bibfnamefont {H.}~\bibnamefont
  {L\"u}}, \bibinfo {author} {\bibfnamefont {A.}~\bibnamefont {Perkins}},
  \bibinfo {author} {\bibfnamefont {C.~N.}\ \bibnamefont {Pope}},\ and\
  \bibinfo {author} {\bibfnamefont {K.~S.}\ \bibnamefont {Stelle}},\ }\bibfield
   {title} {\bibinfo {title} {{Spherically Symmetric Solutions in
  Higher-Derivative Gravity}},\ }\href
  {https://doi.org/10.1103/PhysRevD.92.124019} {\bibfield  {journal} {\bibinfo
  {journal} {Phys. Rev. D}\ }\textbf {\bibinfo {volume} {92}},\ \bibinfo
  {pages} {124019} (\bibinfo {year} {2015})},\ \Eprint
  {https://arxiv.org/abs/1508.00010} {arXiv:1508.00010 [hep-th]} \BibitemShut
  {NoStop}%
\bibitem [{\citenamefont {Podolsk\'y}\ \emph {et~al.}(2020)\citenamefont
  {Podolsk\'y}, \citenamefont {\v{S}varc}, \citenamefont {Pravda},\ and\
  \citenamefont {Pravdova}}]{Podolsky:2019gro}%
  \BibitemOpen
  \bibfield  {author} {\bibinfo {author} {\bibfnamefont {J.}~\bibnamefont
  {Podolsk\'y}}, \bibinfo {author} {\bibfnamefont {R.}~\bibnamefont
  {\v{S}varc}}, \bibinfo {author} {\bibfnamefont {V.}~\bibnamefont {Pravda}},\
  and\ \bibinfo {author} {\bibfnamefont {A.}~\bibnamefont {Pravdova}},\
  }\bibfield  {title} {\bibinfo {title} {{Black holes and other exact spherical
  solutions in Quadratic Gravity}},\ }\href
  {https://doi.org/10.1103/PhysRevD.101.024027} {\bibfield  {journal} {\bibinfo
   {journal} {Phys. Rev. D}\ }\textbf {\bibinfo {volume} {101}},\ \bibinfo
  {pages} {024027} (\bibinfo {year} {2020})},\ \Eprint
  {https://arxiv.org/abs/1907.00046} {arXiv:1907.00046 [gr-qc]} \BibitemShut
  {NoStop}%
\bibitem [{\citenamefont {Pravdova}\ \emph {et~al.}(2023)\citenamefont
  {Pravdova}, \citenamefont {Pravda},\ and\ \citenamefont
  {Ortaggio}}]{Pravdova:2023nbo}%
  \BibitemOpen
  \bibfield  {author} {\bibinfo {author} {\bibfnamefont {A.}~\bibnamefont
  {Pravdova}}, \bibinfo {author} {\bibfnamefont {V.}~\bibnamefont {Pravda}},\
  and\ \bibinfo {author} {\bibfnamefont {M.}~\bibnamefont {Ortaggio}},\
  }\bibfield  {title} {\bibinfo {title} {{Topological black holes in higher
  derivative gravity}},\ }\href
  {https://doi.org/10.1140/epjc/s10052-023-11338-9} {\bibfield  {journal}
  {\bibinfo  {journal} {Eur. Phys. J. C}\ }\textbf {\bibinfo {volume} {83}},\
  \bibinfo {pages} {180} (\bibinfo {year} {2023})},\ \Eprint
  {https://arxiv.org/abs/2301.10720} {arXiv:2301.10720 [gr-qc]} \BibitemShut
  {NoStop}%
\bibitem [{\citenamefont {Levy-Leblond}(1965)}]{Levy-Leblond}%
  \BibitemOpen
  \bibfield  {author} {\bibinfo {author} {\bibfnamefont {J.-M.}\ \bibnamefont
  {Levy-Leblond}},\ }\bibfield  {title} {\bibinfo {title} {{Une nouvelle limite
  non-relativiste du groupe de Poincaré}},\ }\href
  {https://doi.org/http://archive.numdam.org/item/AIHPA_1965__3_1_1_0/}
  {\bibfield  {journal} {\bibinfo  {journal} {Annales de l'institut Henri
  Poincaré. Section A, Physique Théorique}\ }\textbf {\bibinfo {volume}
  {3}},\ \bibinfo {pages} {1} (\bibinfo {year} {1965})}\BibitemShut {NoStop}%
\bibitem [{\citenamefont {Sen~Gupta}(1966)}]{Gupta}%
  \BibitemOpen
  \bibfield  {author} {\bibinfo {author} {\bibfnamefont {N.}~\bibnamefont
  {Sen~Gupta}},\ }\bibfield  {title} {\bibinfo {title} {On an analogue of the
  galilei group},\ }\href {https://doi.org/https://doi.org/10.1007/BF02740871}
  {\bibfield  {journal} {\bibinfo  {journal} {Nuovo Cimento A (1965-1970)}\
  }\textbf {\bibinfo {volume} {44}},\ \bibinfo {pages} {512} (\bibinfo {year}
  {1966})}\BibitemShut {NoStop}%
\bibitem [{\citenamefont {Zhang}\ \emph {et~al.}(2023)\citenamefont {Zhang},
  \citenamefont {Zeng},\ and\ \citenamefont {Horvathy}}]{Zhang:2023jbi}%
  \BibitemOpen
  \bibfield  {author} {\bibinfo {author} {\bibfnamefont {P.~M.}\ \bibnamefont
  {Zhang}}, \bibinfo {author} {\bibfnamefont {H.-X.}\ \bibnamefont {Zeng}},\
  and\ \bibinfo {author} {\bibfnamefont {P.~A.}\ \bibnamefont {Horvathy}},\
  }\href@noop {} {\bibinfo {title} {{MultiCarroll dynamics}}} (\bibinfo {year}
  {2023}),\ \Eprint {https://arxiv.org/abs/2306.07002} {arXiv:2306.07002
  [gr-qc]} \BibitemShut {NoStop}%
\bibitem [{\citenamefont {Bergshoeff}\ \emph {et~al.}(2014)\citenamefont
  {Bergshoeff}, \citenamefont {Gomis},\ and\ \citenamefont
  {Longhi}}]{Bergshoeff_2014}%
  \BibitemOpen
  \bibfield  {author} {\bibinfo {author} {\bibfnamefont {E.}~\bibnamefont
  {Bergshoeff}}, \bibinfo {author} {\bibfnamefont {J.}~\bibnamefont {Gomis}},\
  and\ \bibinfo {author} {\bibfnamefont {G.}~\bibnamefont {Longhi}},\
  }\bibfield  {title} {\bibinfo {title} {Dynamics of carroll particles},\
  }\href {https://doi.org/10.1088/0264-9381/31/20/205009} {\bibfield  {journal}
  {\bibinfo  {journal} {Classical and Quantum Gravity}\ }\textbf {\bibinfo
  {volume} {31}},\ \bibinfo {pages} {205009} (\bibinfo {year}
  {2014})}\BibitemShut {NoStop}%
\bibitem [{\citenamefont {Marsot}(2022)}]{Marsot:2021tvq}%
  \BibitemOpen
  \bibfield  {author} {\bibinfo {author} {\bibfnamefont {L.}~\bibnamefont
  {Marsot}},\ }\bibfield  {title} {\bibinfo {title} {{Planar Carrollean
  dynamics, and the Carroll quantum equation}},\ }\href
  {https://doi.org/10.1016/j.geomphys.2022.104574} {\bibfield  {journal}
  {\bibinfo  {journal} {J. Geom. Phys.}\ }\textbf {\bibinfo {volume} {179}},\
  \bibinfo {pages} {104574} (\bibinfo {year} {2022})},\ \Eprint
  {https://arxiv.org/abs/2110.08489} {arXiv:2110.08489 [math-ph]} \BibitemShut
  {NoStop}%
\bibitem [{\citenamefont {Marsot}\ \emph
  {et~al.}(2022{\natexlab{a}})\citenamefont {Marsot}, \citenamefont {Zhang},
  \citenamefont {Chernodub},\ and\ \citenamefont {Horvathy}}]{Marsot:2022imf}%
  \BibitemOpen
  \bibfield  {author} {\bibinfo {author} {\bibfnamefont {L.}~\bibnamefont
  {Marsot}}, \bibinfo {author} {\bibfnamefont {P.~M.}\ \bibnamefont {Zhang}},
  \bibinfo {author} {\bibfnamefont {M.}~\bibnamefont {Chernodub}},\ and\
  \bibinfo {author} {\bibfnamefont {P.~A.}\ \bibnamefont {Horvathy}},\
  }\href@noop {} {\bibinfo {title} {{Hall effects in Carroll dynamics}}}
  (\bibinfo {year} {2022}{\natexlab{a}}),\ \Eprint
  {https://arxiv.org/abs/2212.02360} {arXiv:2212.02360 [hep-th]} \BibitemShut
  {NoStop}%
\bibitem [{\citenamefont {Bagchi}\ \emph
  {et~al.}(2023{\natexlab{a}})\citenamefont {Bagchi}, \citenamefont {Banerjee},
  \citenamefont {Basu}, \citenamefont {Islam},\ and\ \citenamefont
  {Mondal}}]{Bagchi:2022eui}%
  \BibitemOpen
  \bibfield  {author} {\bibinfo {author} {\bibfnamefont {A.}~\bibnamefont
  {Bagchi}}, \bibinfo {author} {\bibfnamefont {A.}~\bibnamefont {Banerjee}},
  \bibinfo {author} {\bibfnamefont {R.}~\bibnamefont {Basu}}, \bibinfo {author}
  {\bibfnamefont {M.}~\bibnamefont {Islam}},\ and\ \bibinfo {author}
  {\bibfnamefont {S.}~\bibnamefont {Mondal}},\ }\bibfield  {title} {\bibinfo
  {title} {{Magic fermions: Carroll and flat bands}},\ }\href
  {https://doi.org/10.1007/JHEP03(2023)227} {\bibfield  {journal} {\bibinfo
  {journal} {JHEP}\ }\textbf {\bibinfo {volume} {03}},\ \bibinfo {pages}
  {227}},\ \Eprint {https://arxiv.org/abs/2211.11640} {arXiv:2211.11640
  [hep-th]} \BibitemShut {NoStop}%
\bibitem [{\citenamefont {Kubakaddi}(2021)}]{Kubakaddi_2021}%
  \BibitemOpen
  \bibfield  {author} {\bibinfo {author} {\bibfnamefont {S.~S.}\ \bibnamefont
  {Kubakaddi}},\ }\bibfield  {title} {\bibinfo {title} {Giant thermopower and
  power factor in magic angle twisted bilayer graphene at low temperature},\
  }\href {https://doi.org/10.1088/1361-648x/abf0c2} {\bibfield  {journal}
  {\bibinfo  {journal} {Journal of Physics: Condensed Matter}\ }\textbf
  {\bibinfo {volume} {33}},\ \bibinfo {pages} {245704} (\bibinfo {year}
  {2021})}\BibitemShut {NoStop}%
\bibitem [{\citenamefont {Kononov}\ \emph {et~al.}(2021)\citenamefont
  {Kononov}, \citenamefont {Endres}, \citenamefont {Abulizi}, \citenamefont
  {Qu}, \citenamefont {Yan}, \citenamefont {Mandrus}, \citenamefont {Watanabe},
  \citenamefont {Taniguchi},\ and\ \citenamefont
  {Schönenberger}}]{Kononov_2021}%
  \BibitemOpen
  \bibfield  {author} {\bibinfo {author} {\bibfnamefont {A.}~\bibnamefont
  {Kononov}}, \bibinfo {author} {\bibfnamefont {M.}~\bibnamefont {Endres}},
  \bibinfo {author} {\bibfnamefont {G.}~\bibnamefont {Abulizi}}, \bibinfo
  {author} {\bibfnamefont {K.}~\bibnamefont {Qu}}, \bibinfo {author}
  {\bibfnamefont {J.}~\bibnamefont {Yan}}, \bibinfo {author} {\bibfnamefont
  {D.~G.}\ \bibnamefont {Mandrus}}, \bibinfo {author} {\bibfnamefont
  {K.}~\bibnamefont {Watanabe}}, \bibinfo {author} {\bibfnamefont
  {T.}~\bibnamefont {Taniguchi}},\ and\ \bibinfo {author} {\bibfnamefont
  {C.}~\bibnamefont {Schönenberger}},\ }\bibfield  {title} {\bibinfo {title}
  {Superconductivity in type-{II} weyl-semimetal \text{WTe}$_2$ induced by a
  normal metal contact},\ }\href {https://doi.org/10.1063/5.0021350} {\bibfield
   {journal} {\bibinfo  {journal} {Journal of Applied Physics}\ }\textbf
  {\bibinfo {volume} {129}},\ \bibinfo {pages} {113903} (\bibinfo {year}
  {2021})}\BibitemShut {NoStop}%
\bibitem [{\citenamefont {Rivera-Betancour}\ and\ \citenamefont
  {Vilatte}(2022)}]{PhysRevD.106.085004}%
  \BibitemOpen
  \bibfield  {author} {\bibinfo {author} {\bibfnamefont {D.}~\bibnamefont
  {Rivera-Betancour}}\ and\ \bibinfo {author} {\bibfnamefont {M.}~\bibnamefont
  {Vilatte}},\ }\bibfield  {title} {\bibinfo {title} {Revisiting the carrollian
  scalar field},\ }\href {https://doi.org/10.1103/PhysRevD.106.085004}
  {\bibfield  {journal} {\bibinfo  {journal} {Phys. Rev. D}\ }\textbf {\bibinfo
  {volume} {106}},\ \bibinfo {pages} {085004} (\bibinfo {year}
  {2022})}\BibitemShut {NoStop}%
\bibitem [{\citenamefont {Chen}\ \emph {et~al.}(2023)\citenamefont {Chen},
  \citenamefont {Liu}, \citenamefont {Sun},\ and\ \citenamefont
  {Zheng}}]{Chen:2023pqf}%
  \BibitemOpen
  \bibfield  {author} {\bibinfo {author} {\bibfnamefont {B.}~\bibnamefont
  {Chen}}, \bibinfo {author} {\bibfnamefont {R.}~\bibnamefont {Liu}}, \bibinfo
  {author} {\bibfnamefont {H.}~\bibnamefont {Sun}},\ and\ \bibinfo {author}
  {\bibfnamefont {Y.-f.}\ \bibnamefont {Zheng}},\ }\href@noop {} {\bibinfo
  {title} {{Constructing Carrollian Field Theories from Null Reduction}}}
  (\bibinfo {year} {2023}),\ \Eprint {https://arxiv.org/abs/2301.06011}
  {arXiv:2301.06011 [hep-th]} \BibitemShut {NoStop}%
\bibitem [{\citenamefont {Bergshoeff}\ \emph {et~al.}(2022)\citenamefont
  {Bergshoeff}, \citenamefont {Figueroa-O'Farrill},\ and\ \citenamefont
  {Gomis}}]{Bergshoeff:2022eog}%
  \BibitemOpen
  \bibfield  {author} {\bibinfo {author} {\bibfnamefont {E.}~\bibnamefont
  {Bergshoeff}}, \bibinfo {author} {\bibfnamefont {J.}~\bibnamefont
  {Figueroa-O'Farrill}},\ and\ \bibinfo {author} {\bibfnamefont
  {J.}~\bibnamefont {Gomis}},\ }\href@noop {} {\bibinfo {title} {{A
  non-lorentzian primer}}} (\bibinfo {year} {2022}),\ \Eprint
  {https://arxiv.org/abs/2206.12177} {arXiv:2206.12177 [hep-th]} \BibitemShut
  {NoStop}%
\bibitem [{\citenamefont {Henneaux}\ and\ \citenamefont
  {Salgado-Rebolledo}(2021)}]{Henneaux:2021yzg}%
  \BibitemOpen
  \bibfield  {author} {\bibinfo {author} {\bibfnamefont {M.}~\bibnamefont
  {Henneaux}}\ and\ \bibinfo {author} {\bibfnamefont {P.}~\bibnamefont
  {Salgado-Rebolledo}},\ }\bibfield  {title} {\bibinfo {title} {{Carroll
  contractions of Lorentz-invariant theories}},\ }\href
  {https://doi.org/10.1007/JHEP11(2021)180} {\bibfield  {journal} {\bibinfo
  {journal} {JHEP}\ }\textbf {\bibinfo {volume} {11}},\ \bibinfo {pages}
  {180}},\ \Eprint {https://arxiv.org/abs/2109.06708} {arXiv:2109.06708
  [hep-th]} \BibitemShut {NoStop}%
\bibitem [{\citenamefont {Bagchi}\ \emph
  {et~al.}(2019{\natexlab{a}})\citenamefont {Bagchi}, \citenamefont {Mehra},\
  and\ \citenamefont {Nandi}}]{Bagchi:2019xfx}%
  \BibitemOpen
  \bibfield  {author} {\bibinfo {author} {\bibfnamefont {A.}~\bibnamefont
  {Bagchi}}, \bibinfo {author} {\bibfnamefont {A.}~\bibnamefont {Mehra}},\ and\
  \bibinfo {author} {\bibfnamefont {P.}~\bibnamefont {Nandi}},\ }\bibfield
  {title} {\bibinfo {title} {{Field Theories with Conformal Carrollian
  Symmetry}},\ }\href {https://doi.org/10.1007/JHEP05(2019)108} {\bibfield
  {journal} {\bibinfo  {journal} {JHEP}\ }\textbf {\bibinfo {volume} {05}},\
  \bibinfo {pages} {108}},\ \Eprint {https://arxiv.org/abs/1901.10147}
  {arXiv:1901.10147 [hep-th]} \BibitemShut {NoStop}%
\bibitem [{\citenamefont {Bagchi}\ \emph {et~al.}(2020)\citenamefont {Bagchi},
  \citenamefont {Basu}, \citenamefont {Mehra},\ and\ \citenamefont
  {Nandi}}]{Bagchi:2019clu}%
  \BibitemOpen
  \bibfield  {author} {\bibinfo {author} {\bibfnamefont {A.}~\bibnamefont
  {Bagchi}}, \bibinfo {author} {\bibfnamefont {R.}~\bibnamefont {Basu}},
  \bibinfo {author} {\bibfnamefont {A.}~\bibnamefont {Mehra}},\ and\ \bibinfo
  {author} {\bibfnamefont {P.}~\bibnamefont {Nandi}},\ }\bibfield  {title}
  {\bibinfo {title} {{Field Theories on Null Manifolds}},\ }\href
  {https://doi.org/10.1007/JHEP02(2020)141} {\bibfield  {journal} {\bibinfo
  {journal} {JHEP}\ }\textbf {\bibinfo {volume} {02}},\ \bibinfo {pages}
  {141}},\ \Eprint {https://arxiv.org/abs/1912.09388} {arXiv:1912.09388
  [hep-th]} \BibitemShut {NoStop}%
\bibitem [{\citenamefont {Bagchi}\ \emph {et~al.}(2021)\citenamefont {Bagchi},
  \citenamefont {Dutta}, \citenamefont {Kolekar},\ and\ \citenamefont
  {Sharma}}]{Bagchi:2021gai}%
  \BibitemOpen
  \bibfield  {author} {\bibinfo {author} {\bibfnamefont {A.}~\bibnamefont
  {Bagchi}}, \bibinfo {author} {\bibfnamefont {S.}~\bibnamefont {Dutta}},
  \bibinfo {author} {\bibfnamefont {K.~S.}\ \bibnamefont {Kolekar}},\ and\
  \bibinfo {author} {\bibfnamefont {P.}~\bibnamefont {Sharma}},\ }\bibfield
  {title} {\bibinfo {title} {{BMS field theories and Weyl anomaly}},\ }\href
  {https://doi.org/10.1007/JHEP07(2021)101} {\bibfield  {journal} {\bibinfo
  {journal} {JHEP}\ }\textbf {\bibinfo {volume} {07}},\ \bibinfo {pages}
  {101}},\ \Eprint {https://arxiv.org/abs/2104.10405} {arXiv:2104.10405
  [hep-th]} \BibitemShut {NoStop}%
\bibitem [{\citenamefont {Banerjee}\ \emph {et~al.}(2021)\citenamefont
  {Banerjee}, \citenamefont {Basu}, \citenamefont {Mehra}, \citenamefont
  {Mohan},\ and\ \citenamefont {Sharma}}]{PhysRevD.103.105001}%
  \BibitemOpen
  \bibfield  {author} {\bibinfo {author} {\bibfnamefont {K.}~\bibnamefont
  {Banerjee}}, \bibinfo {author} {\bibfnamefont {R.}~\bibnamefont {Basu}},
  \bibinfo {author} {\bibfnamefont {A.}~\bibnamefont {Mehra}}, \bibinfo
  {author} {\bibfnamefont {A.}~\bibnamefont {Mohan}},\ and\ \bibinfo {author}
  {\bibfnamefont {A.}~\bibnamefont {Sharma}},\ }\bibfield  {title} {\bibinfo
  {title} {Interacting conformal carrollian theories: Cues from
  electrodynamics},\ }\href {https://doi.org/10.1103/PhysRevD.103.105001}
  {\bibfield  {journal} {\bibinfo  {journal} {Phys. Rev. D}\ }\textbf {\bibinfo
  {volume} {103}},\ \bibinfo {pages} {105001} (\bibinfo {year}
  {2021})}\BibitemShut {NoStop}%
\bibitem [{\citenamefont {Bagchi}\ \emph
  {et~al.}(2023{\natexlab{b}})\citenamefont {Bagchi}, \citenamefont {Kolekar},\
  and\ \citenamefont {Shukla}}]{Bagchi:2023ysc}%
  \BibitemOpen
  \bibfield  {author} {\bibinfo {author} {\bibfnamefont {A.}~\bibnamefont
  {Bagchi}}, \bibinfo {author} {\bibfnamefont {K.~S.}\ \bibnamefont
  {Kolekar}},\ and\ \bibinfo {author} {\bibfnamefont {A.}~\bibnamefont
  {Shukla}},\ }\bibfield  {title} {\bibinfo {title} {{Carrollian Origins of
  Bjorken Flow}},\ }\href {https://doi.org/10.1103/PhysRevLett.130.241601}
  {\bibfield  {journal} {\bibinfo  {journal} {Phys. Rev. Lett.}\ }\textbf
  {\bibinfo {volume} {130}},\ \bibinfo {pages} {241601} (\bibinfo {year}
  {2023}{\natexlab{b}})},\ \Eprint {https://arxiv.org/abs/2302.03053}
  {arXiv:2302.03053 [hep-th]} \BibitemShut {NoStop}%
\bibitem [{\citenamefont {Ciambelli}\ \emph
  {et~al.}(2018{\natexlab{a}})\citenamefont {Ciambelli}, \citenamefont
  {Marteau}, \citenamefont {Petkou}, \citenamefont {Petropoulos},\ and\
  \citenamefont {Siampos}}]{Ciambelli:2018wre}%
  \BibitemOpen
  \bibfield  {author} {\bibinfo {author} {\bibfnamefont {L.}~\bibnamefont
  {Ciambelli}}, \bibinfo {author} {\bibfnamefont {C.}~\bibnamefont {Marteau}},
  \bibinfo {author} {\bibfnamefont {A.~C.}\ \bibnamefont {Petkou}}, \bibinfo
  {author} {\bibfnamefont {P.~M.}\ \bibnamefont {Petropoulos}},\ and\ \bibinfo
  {author} {\bibfnamefont {K.}~\bibnamefont {Siampos}},\ }\bibfield  {title}
  {\bibinfo {title} {{Flat holography and Carrollian fluids}},\ }\href
  {https://doi.org/10.1007/JHEP07(2018)165} {\bibfield  {journal} {\bibinfo
  {journal} {JHEP}\ }\textbf {\bibinfo {volume} {07}},\ \bibinfo {pages}
  {165}},\ \Eprint {https://arxiv.org/abs/1802.06809} {arXiv:1802.06809
  [hep-th]} \BibitemShut {NoStop}%
\bibitem [{\citenamefont {Ciambelli}\ \emph
  {et~al.}(2018{\natexlab{b}})\citenamefont {Ciambelli}, \citenamefont
  {Marteau}, \citenamefont {Petkou}, \citenamefont {Petropoulos},\ and\
  \citenamefont {Siampos}}]{Ciambelli:2018xat}%
  \BibitemOpen
  \bibfield  {author} {\bibinfo {author} {\bibfnamefont {L.}~\bibnamefont
  {Ciambelli}}, \bibinfo {author} {\bibfnamefont {C.}~\bibnamefont {Marteau}},
  \bibinfo {author} {\bibfnamefont {A.~C.}\ \bibnamefont {Petkou}}, \bibinfo
  {author} {\bibfnamefont {P.~M.}\ \bibnamefont {Petropoulos}},\ and\ \bibinfo
  {author} {\bibfnamefont {K.}~\bibnamefont {Siampos}},\ }\bibfield  {title}
  {\bibinfo {title} {{Covariant Galilean versus Carrollian hydrodynamics from
  relativistic fluids}},\ }\href {https://doi.org/10.1088/1361-6382/aacf1a}
  {\bibfield  {journal} {\bibinfo  {journal} {Class. Quant. Grav.}\ }\textbf
  {\bibinfo {volume} {35}},\ \bibinfo {pages} {165001} (\bibinfo {year}
  {2018}{\natexlab{b}})},\ \Eprint {https://arxiv.org/abs/1802.05286}
  {arXiv:1802.05286 [hep-th]} \BibitemShut {NoStop}%
\bibitem [{\citenamefont {Campoleoni}\ \emph {et~al.}(2019)\citenamefont
  {Campoleoni}, \citenamefont {Ciambelli}, \citenamefont {Marteau},
  \citenamefont {Petropoulos},\ and\ \citenamefont
  {Siampos}}]{campoleoni2019two}%
  \BibitemOpen
  \bibfield  {author} {\bibinfo {author} {\bibfnamefont {A.}~\bibnamefont
  {Campoleoni}}, \bibinfo {author} {\bibfnamefont {L.}~\bibnamefont
  {Ciambelli}}, \bibinfo {author} {\bibfnamefont {C.}~\bibnamefont {Marteau}},
  \bibinfo {author} {\bibfnamefont {P.~M.}\ \bibnamefont {Petropoulos}},\ and\
  \bibinfo {author} {\bibfnamefont {K.}~\bibnamefont {Siampos}},\ }\bibfield
  {title} {\bibinfo {title} {Two-dimensional fluids and their holographic
  duals},\ }\href@noop {} {\bibfield  {journal} {\bibinfo  {journal} {Nuclear
  Physics B}\ }\textbf {\bibinfo {volume} {946}},\ \bibinfo {pages} {114692}
  (\bibinfo {year} {2019})}\BibitemShut {NoStop}%
\bibitem [{\citenamefont {Ciambelli}\ \emph {et~al.}(2020)\citenamefont
  {Ciambelli}, \citenamefont {Marteau}, \citenamefont {Petropoulos},\ and\
  \citenamefont {Ruzziconi}}]{Ciambelli:2020eba}%
  \BibitemOpen
  \bibfield  {author} {\bibinfo {author} {\bibfnamefont {L.}~\bibnamefont
  {Ciambelli}}, \bibinfo {author} {\bibfnamefont {C.}~\bibnamefont {Marteau}},
  \bibinfo {author} {\bibfnamefont {P.~M.}\ \bibnamefont {Petropoulos}},\ and\
  \bibinfo {author} {\bibfnamefont {R.}~\bibnamefont {Ruzziconi}},\ }\bibfield
  {title} {\bibinfo {title} {{Gauges in Three-Dimensional Gravity and
  Holographic Fluids}},\ }\href {https://doi.org/10.1007/JHEP11(2020)092}
  {\bibfield  {journal} {\bibinfo  {journal} {JHEP}\ }\textbf {\bibinfo
  {volume} {11}},\ \bibinfo {pages} {092}},\ \Eprint
  {https://arxiv.org/abs/2006.10082} {arXiv:2006.10082 [hep-th]} \BibitemShut
  {NoStop}%
\bibitem [{\citenamefont {de~Boer}\ \emph {et~al.}(2020)\citenamefont
  {de~Boer}, \citenamefont {Hartong}, \citenamefont {Have}, \citenamefont
  {Obers},\ and\ \citenamefont {Sybesma}}]{10.21468/SciPostPhys.9.2.018}%
  \BibitemOpen
  \bibfield  {author} {\bibinfo {author} {\bibfnamefont {J.}~\bibnamefont
  {de~Boer}}, \bibinfo {author} {\bibfnamefont {J.}~\bibnamefont {Hartong}},
  \bibinfo {author} {\bibfnamefont {E.}~\bibnamefont {Have}}, \bibinfo {author}
  {\bibfnamefont {N.~A.}\ \bibnamefont {Obers}},\ and\ \bibinfo {author}
  {\bibfnamefont {W.}~\bibnamefont {Sybesma}},\ }\bibfield  {title} {\bibinfo
  {title} {{Non-boost invariant fluid dynamics}},\ }\href
  {https://doi.org/10.21468/SciPostPhys.9.2.018} {\bibfield  {journal}
  {\bibinfo  {journal} {SciPost Phys.}\ }\textbf {\bibinfo {volume} {9}},\
  \bibinfo {pages} {018} (\bibinfo {year} {2020})}\BibitemShut {NoStop}%
\bibitem [{\citenamefont {de~Boer}\ \emph {et~al.}(2022)\citenamefont
  {de~Boer}, \citenamefont {Hartong}, \citenamefont {Obers}, \citenamefont
  {Sybesma},\ and\ \citenamefont {Vandoren}}]{deBoer:2021jej}%
  \BibitemOpen
  \bibfield  {author} {\bibinfo {author} {\bibfnamefont {J.}~\bibnamefont
  {de~Boer}}, \bibinfo {author} {\bibfnamefont {J.}~\bibnamefont {Hartong}},
  \bibinfo {author} {\bibfnamefont {N.~A.}\ \bibnamefont {Obers}}, \bibinfo
  {author} {\bibfnamefont {W.}~\bibnamefont {Sybesma}},\ and\ \bibinfo {author}
  {\bibfnamefont {S.}~\bibnamefont {Vandoren}},\ }\bibfield  {title} {\bibinfo
  {title} {{Carroll Symmetry, Dark Energy and Inflation}},\ }\href
  {https://doi.org/10.3389/fphy.2022.810405} {\bibfield  {journal} {\bibinfo
  {journal} {Front. in Phys.}\ }\textbf {\bibinfo {volume} {10}},\ \bibinfo
  {pages} {810405} (\bibinfo {year} {2022})},\ \Eprint
  {https://arxiv.org/abs/2110.02319} {arXiv:2110.02319 [hep-th]} \BibitemShut
  {NoStop}%
\bibitem [{\citenamefont {Bonga}\ and\ \citenamefont
  {Prabhu}(2020)}]{Bonga:2020fhx}%
  \BibitemOpen
  \bibfield  {author} {\bibinfo {author} {\bibfnamefont {B.}~\bibnamefont
  {Bonga}}\ and\ \bibinfo {author} {\bibfnamefont {K.}~\bibnamefont {Prabhu}},\
  }\bibfield  {title} {\bibinfo {title} {{BMS-like symmetries in cosmology}},\
  }\href {https://doi.org/10.1103/PhysRevD.102.104043} {\bibfield  {journal}
  {\bibinfo  {journal} {Phys. Rev. D}\ }\textbf {\bibinfo {volume} {102}},\
  \bibinfo {pages} {104043} (\bibinfo {year} {2020})},\ \Eprint
  {https://arxiv.org/abs/2009.01243} {arXiv:2009.01243 [gr-qc]} \BibitemShut
  {NoStop}%
\bibitem [{\citenamefont {Bagchi}\ \emph
  {et~al.}(2019{\natexlab{b}})\citenamefont {Bagchi}, \citenamefont
  {Banerjee},\ and\ \citenamefont {Parekh}}]{PhysRevLett.123.111601}%
  \BibitemOpen
  \bibfield  {author} {\bibinfo {author} {\bibfnamefont {A.}~\bibnamefont
  {Bagchi}}, \bibinfo {author} {\bibfnamefont {A.}~\bibnamefont {Banerjee}},\
  and\ \bibinfo {author} {\bibfnamefont {P.}~\bibnamefont {Parekh}},\
  }\bibfield  {title} {\bibinfo {title} {Tensionless path from closed to open
  strings},\ }\href {https://doi.org/10.1103/PhysRevLett.123.111601} {\bibfield
   {journal} {\bibinfo  {journal} {Phys. Rev. Lett.}\ }\textbf {\bibinfo
  {volume} {123}},\ \bibinfo {pages} {111601} (\bibinfo {year}
  {2019}{\natexlab{b}})}\BibitemShut {NoStop}%
\bibitem [{\citenamefont {Bagchi}\ \emph
  {et~al.}(2022{\natexlab{a}})\citenamefont {Bagchi}, \citenamefont {Banerjee},
  \citenamefont {Chakrabortty},\ and\ \citenamefont
  {Chatterjee}}]{Bagchi:2021ban}%
  \BibitemOpen
  \bibfield  {author} {\bibinfo {author} {\bibfnamefont {A.}~\bibnamefont
  {Bagchi}}, \bibinfo {author} {\bibfnamefont {A.}~\bibnamefont {Banerjee}},
  \bibinfo {author} {\bibfnamefont {S.}~\bibnamefont {Chakrabortty}},\ and\
  \bibinfo {author} {\bibfnamefont {R.}~\bibnamefont {Chatterjee}},\ }\bibfield
   {title} {\bibinfo {title} {{A Rindler road to Carrollian worldsheets}},\
  }\href {https://doi.org/10.1007/JHEP04(2022)082} {\bibfield  {journal}
  {\bibinfo  {journal} {JHEP}\ }\textbf {\bibinfo {volume} {04}},\ \bibinfo
  {pages} {082}},\ \Eprint {https://arxiv.org/abs/2111.01172} {arXiv:2111.01172
  [hep-th]} \BibitemShut {NoStop}%
\bibitem [{\citenamefont {Cardona}\ \emph {et~al.}(2016)\citenamefont
  {Cardona}, \citenamefont {Gomis},\ and\ \citenamefont
  {Pons}}]{Cardona:2016ytk}%
  \BibitemOpen
  \bibfield  {author} {\bibinfo {author} {\bibfnamefont {B.}~\bibnamefont
  {Cardona}}, \bibinfo {author} {\bibfnamefont {J.}~\bibnamefont {Gomis}},\
  and\ \bibinfo {author} {\bibfnamefont {J.~M.}\ \bibnamefont {Pons}},\
  }\bibfield  {title} {\bibinfo {title} {{Dynamics of Carroll Strings}},\
  }\href {https://doi.org/10.1007/JHEP07(2016)050} {\bibfield  {journal}
  {\bibinfo  {journal} {JHEP}\ }\textbf {\bibinfo {volume} {07}},\ \bibinfo
  {pages} {050}},\ \Eprint {https://arxiv.org/abs/1605.05483} {arXiv:1605.05483
  [hep-th]} \BibitemShut {NoStop}%
\bibitem [{\citenamefont {P\'erez}(2021)}]{Perez:2021abf}%
  \BibitemOpen
  \bibfield  {author} {\bibinfo {author} {\bibfnamefont {A.}~\bibnamefont
  {P\'erez}},\ }\bibfield  {title} {\bibinfo {title} {{Asymptotic symmetries in
  Carrollian theories of gravity}},\ }\href
  {https://doi.org/10.1007/JHEP12(2021)173} {\bibfield  {journal} {\bibinfo
  {journal} {JHEP}\ }\textbf {\bibinfo {volume} {12}},\ \bibinfo {pages}
  {173}},\ \Eprint {https://arxiv.org/abs/2110.15834} {arXiv:2110.15834
  [hep-th]} \BibitemShut {NoStop}%
\bibitem [{\citenamefont {P\'erez}(2022)}]{Perez:2022jpr}%
  \BibitemOpen
  \bibfield  {author} {\bibinfo {author} {\bibfnamefont {A.}~\bibnamefont
  {P\'erez}},\ }\bibfield  {title} {\bibinfo {title} {{Asymptotic symmetries in
  Carrollian theories of gravity with a negative cosmological constant}},\
  }\href {https://doi.org/10.1007/JHEP09(2022)044} {\bibfield  {journal}
  {\bibinfo  {journal} {JHEP}\ }\textbf {\bibinfo {volume} {09}},\ \bibinfo
  {pages} {044}},\ \Eprint {https://arxiv.org/abs/2202.08768} {arXiv:2202.08768
  [hep-th]} \BibitemShut {NoStop}%
\bibitem [{\citenamefont {Hartong}(2015)}]{Hartong:2015xda}%
  \BibitemOpen
  \bibfield  {author} {\bibinfo {author} {\bibfnamefont {J.}~\bibnamefont
  {Hartong}},\ }\bibfield  {title} {\bibinfo {title} {{Gauging the Carroll
  Algebra and Ultra-Relativistic Gravity}},\ }\href
  {https://doi.org/10.1007/JHEP08(2015)069} {\bibfield  {journal} {\bibinfo
  {journal} {JHEP}\ }\textbf {\bibinfo {volume} {08}},\ \bibinfo {pages}
  {069}},\ \Eprint {https://arxiv.org/abs/1505.05011} {arXiv:1505.05011
  [hep-th]} \BibitemShut {NoStop}%
\bibitem [{\citenamefont {Figueroa-O'Farrill}\ \emph
  {et~al.}(2022{\natexlab{a}})\citenamefont {Figueroa-O'Farrill}, \citenamefont
  {Have}, \citenamefont {Prohazka},\ and\ \citenamefont
  {Salzer}}]{Figueroa-OFarrill:2021sxz}%
  \BibitemOpen
  \bibfield  {author} {\bibinfo {author} {\bibfnamefont {J.}~\bibnamefont
  {Figueroa-O'Farrill}}, \bibinfo {author} {\bibfnamefont {E.}~\bibnamefont
  {Have}}, \bibinfo {author} {\bibfnamefont {S.}~\bibnamefont {Prohazka}},\
  and\ \bibinfo {author} {\bibfnamefont {J.}~\bibnamefont {Salzer}},\
  }\bibfield  {title} {\bibinfo {title} {{Carrollian and celestial spaces at
  infinity}},\ }\href {https://doi.org/10.1007/JHEP09(2022)007} {\bibfield
  {journal} {\bibinfo  {journal} {JHEP}\ }\textbf {\bibinfo {volume} {09}},\
  \bibinfo {pages} {007}},\ \Eprint {https://arxiv.org/abs/2112.03319}
  {arXiv:2112.03319 [hep-th]} \BibitemShut {NoStop}%
\bibitem [{\citenamefont {Hansen}\ \emph {et~al.}(2022)\citenamefont {Hansen},
  \citenamefont {Obers}, \citenamefont {Oling},\ and\ \citenamefont
  {S\o{}gaard}}]{Hansen:2021fxi}%
  \BibitemOpen
  \bibfield  {author} {\bibinfo {author} {\bibfnamefont {D.}~\bibnamefont
  {Hansen}}, \bibinfo {author} {\bibfnamefont {N.~A.}\ \bibnamefont {Obers}},
  \bibinfo {author} {\bibfnamefont {G.}~\bibnamefont {Oling}},\ and\ \bibinfo
  {author} {\bibfnamefont {B.~T.}\ \bibnamefont {S\o{}gaard}},\ }\bibfield
  {title} {\bibinfo {title} {{Carroll Expansion of General Relativity}},\
  }\href {https://doi.org/10.21468/SciPostPhys.13.3.055} {\bibfield  {journal}
  {\bibinfo  {journal} {SciPost Phys.}\ }\textbf {\bibinfo {volume} {13}},\
  \bibinfo {pages} {055} (\bibinfo {year} {2022})},\ \Eprint
  {https://arxiv.org/abs/2112.12684} {arXiv:2112.12684 [hep-th]} \BibitemShut
  {NoStop}%
\bibitem [{\citenamefont {Gomis}\ \emph {et~al.}(2021)\citenamefont {Gomis},
  \citenamefont {Hidalgo},\ and\ \citenamefont
  {Salgado-Rebolledo}}]{Gomis:2020wxp}%
  \BibitemOpen
  \bibfield  {author} {\bibinfo {author} {\bibfnamefont {J.}~\bibnamefont
  {Gomis}}, \bibinfo {author} {\bibfnamefont {D.}~\bibnamefont {Hidalgo}},\
  and\ \bibinfo {author} {\bibfnamefont {P.}~\bibnamefont
  {Salgado-Rebolledo}},\ }\bibfield  {title} {\bibinfo {title}
  {{Non-relativistic and Carrollian limits of Jackiw-Teitelboim gravity}},\
  }\href {https://doi.org/10.1007/JHEP05(2021)162} {\bibfield  {journal}
  {\bibinfo  {journal} {JHEP}\ }\textbf {\bibinfo {volume} {05}},\ \bibinfo
  {pages} {162}},\ \Eprint {https://arxiv.org/abs/2011.15053} {arXiv:2011.15053
  [hep-th]} \BibitemShut {NoStop}%
\bibitem [{\citenamefont {Bergshoeff}\ \emph {et~al.}(2023)\citenamefont
  {Bergshoeff}, \citenamefont {Gomis},\ and\ \citenamefont
  {Kleinschmidt}}]{Bergshoeff:2022qkx}%
  \BibitemOpen
  \bibfield  {author} {\bibinfo {author} {\bibfnamefont {E.~A.}\ \bibnamefont
  {Bergshoeff}}, \bibinfo {author} {\bibfnamefont {J.}~\bibnamefont {Gomis}},\
  and\ \bibinfo {author} {\bibfnamefont {A.}~\bibnamefont {Kleinschmidt}},\
  }\bibfield  {title} {\bibinfo {title} {{Non-Lorentzian theories with and
  without constraints}},\ }\href {https://doi.org/10.1007/JHEP01(2023)167}
  {\bibfield  {journal} {\bibinfo  {journal} {JHEP}\ }\textbf {\bibinfo
  {volume} {01}},\ \bibinfo {pages} {167}},\ \Eprint
  {https://arxiv.org/abs/2210.14848} {arXiv:2210.14848 [hep-th]} \BibitemShut
  {NoStop}%
\bibitem [{\citenamefont {Guerrieri}\ and\ \citenamefont
  {Sobreiro}(2021)}]{Guerrieri:2021cdz}%
  \BibitemOpen
  \bibfield  {author} {\bibinfo {author} {\bibfnamefont {A.}~\bibnamefont
  {Guerrieri}}\ and\ \bibinfo {author} {\bibfnamefont {R.~F.}\ \bibnamefont
  {Sobreiro}},\ }\bibfield  {title} {\bibinfo {title} {{Carroll limit of
  four-dimensional gravity theories in the first order formalism}},\ }\href
  {https://doi.org/10.1088/1361-6382/ac345f} {\bibfield  {journal} {\bibinfo
  {journal} {Class. Quant. Grav.}\ }\textbf {\bibinfo {volume} {38}},\ \bibinfo
  {pages} {245003} (\bibinfo {year} {2021})},\ \Eprint
  {https://arxiv.org/abs/2107.10129} {arXiv:2107.10129 [gr-qc]} \BibitemShut
  {NoStop}%
\bibitem [{\citenamefont {Hansen}\ \emph {et~al.}(2021)\citenamefont {Hansen},
  \citenamefont {Hartong}, \citenamefont {Obers},\ and\ \citenamefont
  {Oling}}]{Hansen:2020wqw}%
  \BibitemOpen
  \bibfield  {author} {\bibinfo {author} {\bibfnamefont {D.}~\bibnamefont
  {Hansen}}, \bibinfo {author} {\bibfnamefont {J.}~\bibnamefont {Hartong}},
  \bibinfo {author} {\bibfnamefont {N.~A.}\ \bibnamefont {Obers}},\ and\
  \bibinfo {author} {\bibfnamefont {G.}~\bibnamefont {Oling}},\ }\bibfield
  {title} {\bibinfo {title} {{Galilean first-order formulation for the
  nonrelativistic expansion of general relativity}},\ }\href
  {https://doi.org/10.1103/PhysRevD.104.L061501} {\bibfield  {journal}
  {\bibinfo  {journal} {Phys. Rev. D}\ }\textbf {\bibinfo {volume} {104}},\
  \bibinfo {pages} {L061501} (\bibinfo {year} {2021})},\ \Eprint
  {https://arxiv.org/abs/2012.01518} {arXiv:2012.01518 [hep-th]} \BibitemShut
  {NoStop}%
\bibitem [{\citenamefont {Anderson}(2004)}]{Anderson:2002zn}%
  \BibitemOpen
  \bibfield  {author} {\bibinfo {author} {\bibfnamefont {E.}~\bibnamefont
  {Anderson}},\ }\bibfield  {title} {\bibinfo {title} {{Strong coupled
  relativity without relativity}},\ }\href
  {https://doi.org/10.1023/B:GERG.0000010474.63835.2c} {\bibfield  {journal}
  {\bibinfo  {journal} {Gen. Rel. Grav.}\ }\textbf {\bibinfo {volume} {36}},\
  \bibinfo {pages} {255} (\bibinfo {year} {2004})},\ \Eprint
  {https://arxiv.org/abs/gr-qc/0205118} {arXiv:gr-qc/0205118} \BibitemShut
  {NoStop}%
\bibitem [{\citenamefont {Donnay}\ and\ \citenamefont
  {Marteau}(2019)}]{Donnay:2019jiz}%
  \BibitemOpen
  \bibfield  {author} {\bibinfo {author} {\bibfnamefont {L.}~\bibnamefont
  {Donnay}}\ and\ \bibinfo {author} {\bibfnamefont {C.}~\bibnamefont
  {Marteau}},\ }\bibfield  {title} {\bibinfo {title} {{Carrollian Physics at
  the Black Hole Horizon}},\ }\href {https://doi.org/10.1088/1361-6382/ab2fd5}
  {\bibfield  {journal} {\bibinfo  {journal} {Class. Quant. Grav.}\ }\textbf
  {\bibinfo {volume} {36}},\ \bibinfo {pages} {165002} (\bibinfo {year}
  {2019})},\ \Eprint {https://arxiv.org/abs/1903.09654} {arXiv:1903.09654
  [hep-th]} \BibitemShut {NoStop}%
\bibitem [{\citenamefont {Grumiller}\ and\ \citenamefont
  {Merbis}(2020)}]{Grumiller:2019tyl}%
  \BibitemOpen
  \bibfield  {author} {\bibinfo {author} {\bibfnamefont {D.}~\bibnamefont
  {Grumiller}}\ and\ \bibinfo {author} {\bibfnamefont {W.}~\bibnamefont
  {Merbis}},\ }\bibfield  {title} {\bibinfo {title} {{Near horizon dynamics of
  three dimensional black holes}},\ }\href
  {https://doi.org/10.21468/SciPostPhys.8.1.010} {\bibfield  {journal}
  {\bibinfo  {journal} {SciPost Phys.}\ }\textbf {\bibinfo {volume} {8}},\
  \bibinfo {pages} {010} (\bibinfo {year} {2020})},\ \Eprint
  {https://arxiv.org/abs/1906.10694} {arXiv:1906.10694 [hep-th]} \BibitemShut
  {NoStop}%
\bibitem [{\citenamefont {Redondo-Yuste}\ and\ \citenamefont
  {Lehner}(2022)}]{Redondo-Yuste:2022czg}%
  \BibitemOpen
  \bibfield  {author} {\bibinfo {author} {\bibfnamefont {J.}~\bibnamefont
  {Redondo-Yuste}}\ and\ \bibinfo {author} {\bibfnamefont {L.}~\bibnamefont
  {Lehner}},\ }\href@noop {} {\bibinfo {title} {{Non-linear black hole dynamics
  and Carrollian fluids}}} (\bibinfo {year} {2022}),\ \Eprint
  {https://arxiv.org/abs/2212.06175} {arXiv:2212.06175 [gr-qc]} \BibitemShut
  {NoStop}%
\bibitem [{\citenamefont {Anabal\'on}\ \emph {et~al.}(2021)\citenamefont
  {Anabal\'on}, \citenamefont {Brenner}, \citenamefont {Giribet},\ and\
  \citenamefont {Montecchio}}]{Anabalon:2021wjy}%
  \BibitemOpen
  \bibfield  {author} {\bibinfo {author} {\bibfnamefont {A.}~\bibnamefont
  {Anabal\'on}}, \bibinfo {author} {\bibfnamefont {S.}~\bibnamefont {Brenner}},
  \bibinfo {author} {\bibfnamefont {G.}~\bibnamefont {Giribet}},\ and\ \bibinfo
  {author} {\bibfnamefont {L.}~\bibnamefont {Montecchio}},\ }\bibfield  {title}
  {\bibinfo {title} {{Closer look at black hole pair creation}},\ }\href
  {https://doi.org/10.1103/PhysRevD.104.024044} {\bibfield  {journal} {\bibinfo
   {journal} {Phys. Rev. D}\ }\textbf {\bibinfo {volume} {104}},\ \bibinfo
  {pages} {024044} (\bibinfo {year} {2021})},\ \Eprint
  {https://arxiv.org/abs/2103.05782} {arXiv:2103.05782 [hep-th]} \BibitemShut
  {NoStop}%
\bibitem [{\citenamefont {Herfray}(2022)}]{Herfray:2021qmp}%
  \BibitemOpen
  \bibfield  {author} {\bibinfo {author} {\bibfnamefont {Y.}~\bibnamefont
  {Herfray}},\ }\bibfield  {title} {\bibinfo {title} {{Carrollian manifolds and
  null infinity: a view from Cartan geometry}},\ }\href
  {https://doi.org/10.1088/1361-6382/ac635f} {\bibfield  {journal} {\bibinfo
  {journal} {Class. Quant. Grav.}\ }\textbf {\bibinfo {volume} {39}},\ \bibinfo
  {pages} {215005} (\bibinfo {year} {2022})},\ \Eprint
  {https://arxiv.org/abs/2112.09048} {arXiv:2112.09048 [gr-qc]} \BibitemShut
  {NoStop}%
\bibitem [{\citenamefont {Chandrasekaran}\ \emph {et~al.}(2022)\citenamefont
  {Chandrasekaran}, \citenamefont {Flanagan}, \citenamefont {Shehzad},\ and\
  \citenamefont {Speranza}}]{Chandrasekaran:2021hxc}%
  \BibitemOpen
  \bibfield  {author} {\bibinfo {author} {\bibfnamefont {V.}~\bibnamefont
  {Chandrasekaran}}, \bibinfo {author} {\bibfnamefont {E.~E.}\ \bibnamefont
  {Flanagan}}, \bibinfo {author} {\bibfnamefont {I.}~\bibnamefont {Shehzad}},\
  and\ \bibinfo {author} {\bibfnamefont {A.~J.}\ \bibnamefont {Speranza}},\
  }\bibfield  {title} {\bibinfo {title} {{Brown-York charges at null
  boundaries}},\ }\href {https://doi.org/10.1007/JHEP01(2022)029} {\bibfield
  {journal} {\bibinfo  {journal} {JHEP}\ }\textbf {\bibinfo {volume} {01}},\
  \bibinfo {pages} {029}},\ \Eprint {https://arxiv.org/abs/2109.11567}
  {arXiv:2109.11567 [hep-th]} \BibitemShut {NoStop}%
\bibitem [{\citenamefont {Ciambelli}\ \emph {et~al.}(2019)\citenamefont
  {Ciambelli}, \citenamefont {Leigh}, \citenamefont {Marteau},\ and\
  \citenamefont {Petropoulos}}]{Ciambelli:2019lap}%
  \BibitemOpen
  \bibfield  {author} {\bibinfo {author} {\bibfnamefont {L.}~\bibnamefont
  {Ciambelli}}, \bibinfo {author} {\bibfnamefont {R.~G.}\ \bibnamefont
  {Leigh}}, \bibinfo {author} {\bibfnamefont {C.}~\bibnamefont {Marteau}},\
  and\ \bibinfo {author} {\bibfnamefont {P.~M.}\ \bibnamefont {Petropoulos}},\
  }\bibfield  {title} {\bibinfo {title} {{Carroll Structures, Null Geometry and
  Conformal Isometries}},\ }\href {https://doi.org/10.1103/PhysRevD.100.046010}
  {\bibfield  {journal} {\bibinfo  {journal} {Phys. Rev. D}\ }\textbf {\bibinfo
  {volume} {100}},\ \bibinfo {pages} {046010} (\bibinfo {year} {2019})},\
  \Eprint {https://arxiv.org/abs/1905.02221} {arXiv:1905.02221 [hep-th]}
  \BibitemShut {NoStop}%
\bibitem [{\citenamefont {Gray}\ \emph {et~al.}(2022)\citenamefont {Gray},
  \citenamefont {Kubiznak}, \citenamefont {Perche},\ and\ \citenamefont
  {Redondo-Yuste}}]{Gray:2022svz}%
  \BibitemOpen
  \bibfield  {author} {\bibinfo {author} {\bibfnamefont {F.}~\bibnamefont
  {Gray}}, \bibinfo {author} {\bibfnamefont {D.}~\bibnamefont {Kubiznak}},
  \bibinfo {author} {\bibfnamefont {T.~R.}\ \bibnamefont {Perche}},\ and\
  \bibinfo {author} {\bibfnamefont {J.}~\bibnamefont {Redondo-Yuste}},\
  }\href@noop {} {\bibinfo {title} {{Carrollian Motion in Magnetized Black Hole
  Horizons}}} (\bibinfo {year} {2022}),\ \Eprint
  {https://arxiv.org/abs/2211.13695} {arXiv:2211.13695 [gr-qc]} \BibitemShut
  {NoStop}%
\bibitem [{\citenamefont {Bicak}\ \emph {et~al.}(2023)\citenamefont {Bicak},
  \citenamefont {Kubiznak},\ and\ \citenamefont {Perche}}]{Bicak:2023vxs}%
  \BibitemOpen
  \bibfield  {author} {\bibinfo {author} {\bibfnamefont {J.}~\bibnamefont
  {Bicak}}, \bibinfo {author} {\bibfnamefont {D.}~\bibnamefont {Kubiznak}},\
  and\ \bibinfo {author} {\bibfnamefont {T.~R.}\ \bibnamefont {Perche}},\
  }\href@noop {} {\bibinfo {title} {{Monarch Migration of Carrollian Particles
  on the Black Hole Horizon}}} (\bibinfo {year} {2023}),\ \Eprint
  {https://arxiv.org/abs/2302.11639} {arXiv:2302.11639 [gr-qc]} \BibitemShut
  {NoStop}%
\bibitem [{\citenamefont {Marsot}\ \emph
  {et~al.}(2022{\natexlab{b}})\citenamefont {Marsot}, \citenamefont {Zhang},\
  and\ \citenamefont {Horvathy}}]{Marsot:2022qkx}%
  \BibitemOpen
  \bibfield  {author} {\bibinfo {author} {\bibfnamefont {L.}~\bibnamefont
  {Marsot}}, \bibinfo {author} {\bibfnamefont {P.-M.}\ \bibnamefont {Zhang}},\
  and\ \bibinfo {author} {\bibfnamefont {P.}~\bibnamefont {Horvathy}},\
  }\bibfield  {title} {\bibinfo {title} {{Anyonic spin-Hall effect on the black
  hole horizon}},\ }\href {https://doi.org/10.1103/PhysRevD.106.L121503}
  {\bibfield  {journal} {\bibinfo  {journal} {Phys. Rev. D}\ }\textbf {\bibinfo
  {volume} {106}},\ \bibinfo {pages} {L121503} (\bibinfo {year}
  {2022}{\natexlab{b}})},\ \Eprint {https://arxiv.org/abs/2207.06302}
  {arXiv:2207.06302 [gr-qc]} \BibitemShut {NoStop}%
\bibitem [{\citenamefont {Price}\ and\ \citenamefont
  {Thorne}(1986)}]{Price:1986yy}%
  \BibitemOpen
  \bibfield  {author} {\bibinfo {author} {\bibfnamefont {R.~H.}\ \bibnamefont
  {Price}}\ and\ \bibinfo {author} {\bibfnamefont {K.~S.}\ \bibnamefont
  {Thorne}},\ }\bibfield  {title} {\bibinfo {title} {{Membrane Viewpoint on
  Black Holes: Properties and Evolution of the Stretched Horizon}},\ }\href
  {https://doi.org/10.1103/PhysRevD.33.915} {\bibfield  {journal} {\bibinfo
  {journal} {Phys. Rev. D}\ }\textbf {\bibinfo {volume} {33}},\ \bibinfo
  {pages} {915} (\bibinfo {year} {1986})}\BibitemShut {NoStop}%
\bibitem [{\citenamefont {{Thorne}}\ \emph {et~al.}(1986)\citenamefont
  {{Thorne}}, \citenamefont {{Price}},\ and\ \citenamefont
  {{MacDonald}}}]{1986bhmp.book.....T}%
  \BibitemOpen
  \bibfield  {author} {\bibinfo {author} {\bibfnamefont {K.~S.}\ \bibnamefont
  {{Thorne}}}, \bibinfo {author} {\bibfnamefont {R.~H.}\ \bibnamefont
  {{Price}}},\ and\ \bibinfo {author} {\bibfnamefont {D.~A.}\ \bibnamefont
  {{MacDonald}}},\ }\href@noop {} {\emph {\bibinfo {title} {{Black holes: The
  membrane paradigm}}}}\ (\bibinfo {year} {1986})\BibitemShut {NoStop}%
\bibitem [{\citenamefont {Damour}(1978)}]{Damour:1978cg}%
  \BibitemOpen
  \bibfield  {author} {\bibinfo {author} {\bibfnamefont {T.}~\bibnamefont
  {Damour}},\ }\bibfield  {title} {\bibinfo {title} {{Black Hole Eddy
  Currents}},\ }\href {https://doi.org/10.1103/PhysRevD.18.3598} {\bibfield
  {journal} {\bibinfo  {journal} {Phys. Rev. D}\ }\textbf {\bibinfo {volume}
  {18}},\ \bibinfo {pages} {3598} (\bibinfo {year} {1978})}\BibitemShut
  {NoStop}%
\bibitem [{\citenamefont {Inonu}\ and\ \citenamefont
  {Wigner}(1953)}]{4860f44e-649d-341b-9a70-b912b6531bea}%
  \BibitemOpen
  \bibfield  {author} {\bibinfo {author} {\bibfnamefont {E.}~\bibnamefont
  {Inonu}}\ and\ \bibinfo {author} {\bibfnamefont {E.~P.}\ \bibnamefont
  {Wigner}},\ }\bibfield  {title} {\bibinfo {title} {On the contraction of
  groups and their representations},\ }\href
  {http://www.jstor.org/stable/88703} {\bibfield  {journal} {\bibinfo
  {journal} {Proceedings of the National Academy of Sciences of the United
  States of America}\ }\textbf {\bibinfo {volume} {39}},\ \bibinfo {pages}
  {510} (\bibinfo {year} {1953})}\BibitemShut {NoStop}%
\bibitem [{\citenamefont {Figueroa-O'Farrill}\ \emph
  {et~al.}(2022{\natexlab{b}})\citenamefont {Figueroa-O'Farrill}, \citenamefont
  {Have}, \citenamefont {Prohazka},\ and\ \citenamefont
  {Salzer}}]{Figueroa-OFarrill:2022mcy}%
  \BibitemOpen
  \bibfield  {author} {\bibinfo {author} {\bibfnamefont {J.}~\bibnamefont
  {Figueroa-O'Farrill}}, \bibinfo {author} {\bibfnamefont {E.}~\bibnamefont
  {Have}}, \bibinfo {author} {\bibfnamefont {S.}~\bibnamefont {Prohazka}},\
  and\ \bibinfo {author} {\bibfnamefont {J.}~\bibnamefont {Salzer}},\
  }\bibfield  {title} {\bibinfo {title} {{The gauging procedure and carrollian
  gravity}},\ }\href {https://doi.org/10.1007/JHEP09(2022)243} {\bibfield
  {journal} {\bibinfo  {journal} {JHEP}\ }\textbf {\bibinfo {volume} {09}},\
  \bibinfo {pages} {243}},\ \Eprint {https://arxiv.org/abs/2206.14178}
  {arXiv:2206.14178 [hep-th]} \BibitemShut {NoStop}%
\bibitem [{\citenamefont
  {Figueroa-O'Farrill}(2023)}]{Figueroa-OFarrill:2022pus}%
  \BibitemOpen
  \bibfield  {author} {\bibinfo {author} {\bibfnamefont {J.}~\bibnamefont
  {Figueroa-O'Farrill}},\ }\bibfield  {title} {\bibinfo {title} {{Lie algebraic
  Carroll/Galilei duality}},\ }\href {https://doi.org/10.1063/5.0132661}
  {\bibfield  {journal} {\bibinfo  {journal} {J. Math. Phys.}\ }\textbf
  {\bibinfo {volume} {64}},\ \bibinfo {pages} {013503} (\bibinfo {year}
  {2023})},\ \Eprint {https://arxiv.org/abs/2210.13924} {arXiv:2210.13924
  [math.DG]} \BibitemShut {NoStop}%
\bibitem [{\citenamefont {Freidel}\ and\ \citenamefont
  {Jai-akson}(2022)}]{Freidel:2022vjq}%
  \BibitemOpen
  \bibfield  {author} {\bibinfo {author} {\bibfnamefont {L.}~\bibnamefont
  {Freidel}}\ and\ \bibinfo {author} {\bibfnamefont {P.}~\bibnamefont
  {Jai-akson}},\ }\href@noop {} {\bibinfo {title} {{Carrollian hydrodynamics
  and symplectic structure on stretched horizons}}} (\bibinfo {year} {2022}),\
  \Eprint {https://arxiv.org/abs/2211.06415} {arXiv:2211.06415 [gr-qc]}
  \BibitemShut {NoStop}%
\bibitem [{\citenamefont {Morand}(2020)}]{Morand:2018tke}%
  \BibitemOpen
  \bibfield  {author} {\bibinfo {author} {\bibfnamefont {K.}~\bibnamefont
  {Morand}},\ }\bibfield  {title} {\bibinfo {title} {{Embedding Galilean and
  Carrollian geometries I. Gravitational waves}},\ }\href
  {https://doi.org/10.1063/1.5130907} {\bibfield  {journal} {\bibinfo
  {journal} {J. Math. Phys.}\ }\textbf {\bibinfo {volume} {61}},\ \bibinfo
  {pages} {082502} (\bibinfo {year} {2020})},\ \Eprint
  {https://arxiv.org/abs/1811.12681} {arXiv:1811.12681 [hep-th]} \BibitemShut
  {NoStop}%
\bibitem [{\citenamefont {Campoleoni}\ \emph {et~al.}(2022)\citenamefont
  {Campoleoni}, \citenamefont {Henneaux}, \citenamefont {Pekar}, \citenamefont
  {P\'erez},\ and\ \citenamefont {Salgado-Rebolledo}}]{Campoleoni:2022ebj}%
  \BibitemOpen
  \bibfield  {author} {\bibinfo {author} {\bibfnamefont {A.}~\bibnamefont
  {Campoleoni}}, \bibinfo {author} {\bibfnamefont {M.}~\bibnamefont
  {Henneaux}}, \bibinfo {author} {\bibfnamefont {S.}~\bibnamefont {Pekar}},
  \bibinfo {author} {\bibfnamefont {A.}~\bibnamefont {P\'erez}},\ and\ \bibinfo
  {author} {\bibfnamefont {P.}~\bibnamefont {Salgado-Rebolledo}},\ }\bibfield
  {title} {\bibinfo {title} {{Magnetic Carrollian gravity from the Carroll
  algebra}},\ }\href {https://doi.org/10.1007/JHEP09(2022)127} {\bibfield
  {journal} {\bibinfo  {journal} {JHEP}\ }\textbf {\bibinfo {volume} {09}},\
  \bibinfo {pages} {127}},\ \Eprint {https://arxiv.org/abs/2207.14167}
  {arXiv:2207.14167 [hep-th]} \BibitemShut {NoStop}%
\bibitem [{\citenamefont {Bergshoeff}\ \emph {et~al.}(2017)\citenamefont
  {Bergshoeff}, \citenamefont {Gomis}, \citenamefont {Rollier}, \citenamefont
  {Rosseel},\ and\ \citenamefont {ter Veldhuis}}]{Bergshoeff:2017btm}%
  \BibitemOpen
  \bibfield  {author} {\bibinfo {author} {\bibfnamefont {E.}~\bibnamefont
  {Bergshoeff}}, \bibinfo {author} {\bibfnamefont {J.}~\bibnamefont {Gomis}},
  \bibinfo {author} {\bibfnamefont {B.}~\bibnamefont {Rollier}}, \bibinfo
  {author} {\bibfnamefont {J.}~\bibnamefont {Rosseel}},\ and\ \bibinfo {author}
  {\bibfnamefont {T.}~\bibnamefont {ter Veldhuis}},\ }\bibfield  {title}
  {\bibinfo {title} {{Carroll versus Galilei Gravity}},\ }\href
  {https://doi.org/10.1007/JHEP03(2017)165} {\bibfield  {journal} {\bibinfo
  {journal} {JHEP}\ }\textbf {\bibinfo {volume} {03}},\ \bibinfo {pages}
  {165}},\ \Eprint {https://arxiv.org/abs/1701.06156} {arXiv:1701.06156
  [hep-th]} \BibitemShut {NoStop}%
\bibitem [{\citenamefont {Gupta}\ and\ \citenamefont
  {Suryanarayana}(2021)}]{Gupta:2020dtl}%
  \BibitemOpen
  \bibfield  {author} {\bibinfo {author} {\bibfnamefont {N.}~\bibnamefont
  {Gupta}}\ and\ \bibinfo {author} {\bibfnamefont {N.~V.}\ \bibnamefont
  {Suryanarayana}},\ }\bibfield  {title} {\bibinfo {title} {{Constructing
  Carrollian CFTs}},\ }\href {https://doi.org/10.1007/JHEP03(2021)194}
  {\bibfield  {journal} {\bibinfo  {journal} {JHEP}\ }\textbf {\bibinfo
  {volume} {03}},\ \bibinfo {pages} {194}},\ \Eprint
  {https://arxiv.org/abs/2001.03056} {arXiv:2001.03056 [hep-th]} \BibitemShut
  {NoStop}%
\bibitem [{\citenamefont {Bagchi}\ \emph
  {et~al.}(2022{\natexlab{b}})\citenamefont {Bagchi}, \citenamefont
  {Banerjee},\ and\ \citenamefont {Muraki}}]{Bagchi:2022nvj}%
  \BibitemOpen
  \bibfield  {author} {\bibinfo {author} {\bibfnamefont {A.}~\bibnamefont
  {Bagchi}}, \bibinfo {author} {\bibfnamefont {A.}~\bibnamefont {Banerjee}},\
  and\ \bibinfo {author} {\bibfnamefont {H.}~\bibnamefont {Muraki}},\
  }\bibfield  {title} {\bibinfo {title} {{Boosting to BMS}},\ }\href
  {https://doi.org/10.1007/JHEP09(2022)251} {\bibfield  {journal} {\bibinfo
  {journal} {JHEP}\ }\textbf {\bibinfo {volume} {09}},\ \bibinfo {pages}
  {251}},\ \Eprint {https://arxiv.org/abs/2205.05094} {arXiv:2205.05094
  [hep-th]} \BibitemShut {NoStop}%
\bibitem [{\citenamefont {Baiguera}\ \emph {et~al.}(2023)\citenamefont
  {Baiguera}, \citenamefont {Oling}, \citenamefont {Sybesma},\ and\
  \citenamefont {S\o{}gaard}}]{Baiguera:2022lsw}%
  \BibitemOpen
  \bibfield  {author} {\bibinfo {author} {\bibfnamefont {S.}~\bibnamefont
  {Baiguera}}, \bibinfo {author} {\bibfnamefont {G.}~\bibnamefont {Oling}},
  \bibinfo {author} {\bibfnamefont {W.}~\bibnamefont {Sybesma}},\ and\ \bibinfo
  {author} {\bibfnamefont {B.~T.}\ \bibnamefont {S\o{}gaard}},\ }\bibfield
  {title} {\bibinfo {title} {{Conformal Carroll scalars with boosts}},\ }\href
  {https://doi.org/10.21468/SciPostPhys.14.4.086} {\bibfield  {journal}
  {\bibinfo  {journal} {SciPost Phys.}\ }\textbf {\bibinfo {volume} {14}},\
  \bibinfo {pages} {086} (\bibinfo {year} {2023})},\ \Eprint
  {https://arxiv.org/abs/2207.03468} {arXiv:2207.03468 [hep-th]} \BibitemShut
  {NoStop}%
\bibitem [{\citenamefont {Petkou}\ \emph {et~al.}(2022)\citenamefont {Petkou},
  \citenamefont {Petropoulos}, \citenamefont {Betancour},\ and\ \citenamefont
  {Siampos}}]{Petkou:2022bmz}%
  \BibitemOpen
  \bibfield  {author} {\bibinfo {author} {\bibfnamefont {A.~C.}\ \bibnamefont
  {Petkou}}, \bibinfo {author} {\bibfnamefont {P.~M.}\ \bibnamefont
  {Petropoulos}}, \bibinfo {author} {\bibfnamefont {D.~R.}\ \bibnamefont
  {Betancour}},\ and\ \bibinfo {author} {\bibfnamefont {K.}~\bibnamefont
  {Siampos}},\ }\bibfield  {title} {\bibinfo {title} {{Relativistic fluids,
  hydrodynamic frames and their Galilean versus Carrollian avatars}},\ }\href
  {https://doi.org/10.1007/JHEP09(2022)162} {\bibfield  {journal} {\bibinfo
  {journal} {JHEP}\ }\textbf {\bibinfo {volume} {09}},\ \bibinfo {pages}
  {162}},\ \Eprint {https://arxiv.org/abs/2205.09142} {arXiv:2205.09142
  [hep-th]} \BibitemShut {NoStop}%
\end{thebibliography}%


\end{document}
