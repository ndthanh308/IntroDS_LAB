
\begin{table}
	\begin{center}
    	\caption{The polarised disc brightness relative to the total intensity of the system.}
    	\label{tab:polarisedtototal}
    	\begin{tabular}{clcc} 
    		\hline
    		 \#ID& Name & Lower limit & Upper limit \\
    		\hline
    		\multicolumn{4}{c}{Full discs}\\
    		\hline
            1 & U\,Mon$^{*}$ & 0.37\%&1.58\%\\
            2 & IRAS\,08544-4431& 0.58\%&1.30\%\\
            3 & IW\,Car & 0.95\%&1.93\%\\
            4 & HR 4049 & 0.39\%&0.79\%\\
            5&IRAS 15469-5311 &0.36\%&1.60\%\\
            6&IRAS 17038-4815 & 0.31\%&2.24\%\\
            \hline
    		\multicolumn{4}{c}{Transition discs}\\
            \hline
            7 & RU\,Cen & 0.46\%&1.32\%\\
            8 & AC\,Her & 1.28\%&1.94\%\\
            \hline
    	\end{tabular}
    \end{center}
	\begin{tablenotes}
    \small
    \item \textbf{Notes:} The lower limit of the polarised disk brightness includes only resolved emission from the disc, while the upper limit also includes unresolved central polarisation (see Section~\ref{sec: polarised_to_tot}) for more details. We note that $^{*}$ indicates that we use the mean combined result of two independent observations for U Mon (see Section~\ref{sec:observations}) to improve the signal-to-noise ratio.\\
    \end{tablenotes}
\end{table}
