\begin{table*}
    \caption{Derived disc properties for all targets in the sample.}
    \begin{center}
            
        \begin{tabular}{ |l|c|c|c|l|c|c|c }
            \hline
        \#ID& Name & \begin{tabular}[c]{@{}l@{}}
        $a$ \\ {[}mas{]}\end{tabular} &  \begin{tabular}[c]{@{}l@{}}$b$\\ {[}mas{]}\end{tabular}&
        \begin{tabular}[c]{@{}l@{}}
        $i$ \\  {[}$^\circ${]}
        \end{tabular} & \begin{tabular}[c]{@{}l@{}}
        $PA$ \\  {[}$^\circ${]}
        \end{tabular}& $e$ & \begin{tabular}[c]{@{}l@{}} Percentage of  polarised light \\on different substructures\end{tabular} \\
            \hline
    		\multicolumn{8}{c}{Full discs}\\
            \hline
            1 & U\,Mon$^{*}$ & 31.6$^{+3}_{-2.5}$& 28.6$^{+2.4}_{-2.7}$& 25$^{+14}_{-18}$ & 144$^{+10}_{-15}$ & 0.42&80.1; 5.1\\
            2 & IRAS\,08544-4431 &38.6$^{+1.2}_{-1.2}$ & 35.5$^{+1.5}_{-1.4}$& 23$^{+7}_{-15}$ & 121$^{+27}_{-30}$ & 0.39& 78.9; 6.8; 2.2\\
            3& IW\,Car & 41.2$^{+3.3}_{-2.6}$ & 30.7$^{+2}_{-2}$& 41.7$^{+6.5}_{-7.7}$ & 161$^{+12}_{-10}$ & 0.67 & 72; 7.5; 4.5; 3.6; 1.6\\
            4 & HR 4049 & 36.5$^{+2.3}_{-1.7}$ & 34.9$^{+2}_{-2.3}$& 16.8$^{+14}_{-14}$& 174 $^{+28}_{-30}$ & 0.29&--\\
            
            4$^{\dag}$ & HR 4049 &  & & 49$^{+3}_{-3}$& 63$^{+7}_{-6}$ & &\\
            5&IRAS 15469-5311 & 37.5$^{+3.6}_{-2.4}$& 35.4$^{+2.6}_{-3.9}$& 19.6$^{+18}_{-16}$ & 139$^{+27}_{-30}$ & 0.33& 53.4; 26\\
            6&IRAS 17038-4815 &  32.6$^{+12.5}_{-6.8}$ & 26.5$^{+5.6}_{-8.5}$& 35.7$^{+29}_{-29}$ & 123$^{+28}_{-30}$ & 0.58& --\\
            \hline
    		\multicolumn{8}{c}{Transition discs}\\
            \hline
            7& RU\,Cen & 39.4& 23.9& 52.8& 156 & 0.80&--\\
            8 & AC\,Her & -- & -- & -- & -- & -- & 74.5; 13.1; 1.7\\
            \hline
        \end{tabular}
    \end{center}
    \begin{tablenotes}
     \small
    \item \textbf{Notes:} $a$ and $b$ represent the major and minor half-axes of the disc in units of mas, $i$ indicates the inclination, $e$ represents the eccentricity. The position angle ($PA$) is presented in degrees and rises counterclockwise from the vertical axis (North) to the first principal radius (major axis). Due to the limited number of pixels with a significant SNR, we could only provide a rough estimation of the parameters for RU\,Cen (see Section~\ref{sec:individual_study}). Therefore, we did not specify any uncertainties for these measurements. \\ $^{*}$ indicates that we use the mean combined result of two independent observations  of the U\,Mon (see Section~\ref{sec:observations}) to improve the signal-to-noise ratio. More details on the tabulated information can be found in Section~\ref{sec:orientation} and Section~\ref{sec:morphology}. \\
    $^{\dag}$  represents the inclination and PA values defined from the IR interferometric data \citep{Kluska2019A&A...631A.108K}. We adopt these values for the rest of the paper (see Section~\ref{sec:individual_study} for more details). 
    \end{tablenotes}
    \label{tab:fit}
    
    
\end{table*}
