\begin{table}
	\centering
	\caption{Peak signal-to-noise ratio (SNR) for all targets in our sample.\label{tab:snr}}
	
	\begin{center}
    	    
    	\begin{tabular}{|c|l|c|} 
    		\hline
    		 \#ID & Name & Peak S/N\\
    		\hline
    		\multicolumn{3}{c}{Full discs}\\
    		\hline
            1 & U\,Mon & 274\\
            2 & IRAS\,08544-4431 & 181\\
            3 & IW\,Car & 152\\
            4& HR 4049 & 139\\
            5&IRAS 15469-5311 & 111\\
            6&IRAS 17038-4815 & 78\\
            \hline
    		\multicolumn{3}{c}{Transition discs}\\
    		\hline
            7 & RU\,Cen  & 231\\
            8 & AC\,Her & 829\\
            \hline
    	\end{tabular}
    \end{center}
	\begin{tablenotes}
        
    \small
\item \textbf{Notes:} To improve the signal-to-noise ratio for the U\,Mon, we use the mean combined reduced polarised images of two independent observations (see Section~\ref{sec:observations}).\\
    \end{tablenotes}
	
\end{table}
