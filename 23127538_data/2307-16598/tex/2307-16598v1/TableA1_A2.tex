\begin{landscape}
    
    \begin{table}
        \caption{Additional stellar and orbital parameters for each target in the sample.}
        \begin{tabular}{ c|l|l|c|c|c|c|c|c|c|c|c|c|c|l }
            \hline
        \#ID& Name & \begin{tabular}[c]{@{}l@{}}$\alpha$ 2000\\ {[}h m s{]}\end{tabular} & \begin{tabular}[c]{@{}l@{}}$\delta$ 2000\\ {[} $^{\circ}$ ’ ”{]}\end{tabular} & \begin{tabular}[c]{@{}l@{}}$H$ \\ {[}mag{]}\end{tabular}& \begin{tabular}[c]{@{}l@{}}distance \\ {[}pc{]}\end{tabular} & \begin{tabular}[c]{@{}l@{}}Spectral \\ type\end{tabular}& \begin{tabular}[c]{@{}l@{}}$T_{\rm eff}$ \\ {[}K{]}\end{tabular}&  \begin{tabular}[c]{@{}l@{}}$P_{\rm puls}$\\ {[}days{]}\end{tabular} & RVb & [Zn/Fe] &[C/H]& \begin{tabular}[c]{@{}l@{}}$T_{\rm turn-off}$ \\ {(}K{)}\end{tabular}&  \begin{tabular}[c]{@{}l@{}}Depletion \\ profile\end{tabular}& Ref. \\
            \hline
    		\multicolumn{15}{c}{Full discs}\\
            \hline
            1& U\,Mon & 07 30 47 & -09 46 37 &3.9& 800$^{+117}_{-87}$ & G0I & 5000 & 92$\pm$3  & y & 0.1&-0.18&-&N& 1, 2, 6, 15 \\
            2& IRAS 08544-4431 & 08 56 14.1 & -44 43 10.7 &4.7& 1575$^{+124}_{-126}$ & F3 & 7250  & 72 & n &0.4&-0.02&1200&S & 1, 2, 7,14  \\
            3& IW\,Car   & 09 26 53 & -63 37 49 &5.1& 1314$^{+55}_{-71}$&F7I&6700& 72$\pm$1 & y & 1&0.32&1100&S& 3, 2, 8, 15 \\
            4& HR 4049  & 10 18 07.5 & -28 59 31.2 &4.2& 1449$^{+310}_{-186}$&A4Ib/II&7600&-&  y & 3.5&-&800&S & 1, 9 \\
            5&IRAS 15469-5311 & 15 50 44 & -53 20 43 &6.2& 3471$^{+375}_{-320}$     &F3&7500& 50 & n &0.3&0.3&1300&S & 1, 2, 11,13\\
            6&IRAS 17038-4815& 17 07 37 & -48 19 08 &7.1& 4718$^{+1014}_{-504}$     &G2p&4750& 37.9$\pm$1.5 &   y  &0.3&0.3&1400&S & 1, 4, 11, 13 \\
            \hline
    		\multicolumn{15}{c}{Transition discs}\\
            \hline
            7& RU\,Cen   & 12 09 23 & -45 25 35 &7.2& 4618$^{+1311}_{-1084}$ &F6I&6000&  &  n & 0.9&-0.42&800&P& 1, 10, 15 \\
            8& AC\,Her & 18 30 16.24   & 21 52 00.6  &5.3& 1627$^{+99}_{-90}$ &F2Iep&5500& 75$\pm$2 &  y &  0.7&-0.35&1200&U& 1, 5, 12, 13, 15 \\
            \hline
        \end{tabular}
        \begin{tablenotes}
        
        \small
        \item \textbf{Notes:}  The target sample is separated into full discs and transition discs based on the disc category of \citet{Kluska2022}. See Section~\ref{sec:target_sel} for more details. RA and Dec. coordinates are given for the J2000 epoch. $H$ represents the observed photometric $H$-band magnitude. The distances to binary post-AGB stars were adopted from $Gaia$\,EDR3 \citep{Bailer-Jones2021AJ....161..147B}. However, we note that these distances are uncertain because: i) they are too far away and therefore not flagged as astrometric binaries ii) the orbital motion of the binary results in an angular displacement comparable to the parallax. $T_{\rm eff}$ represents the spectroscopically determined effective temperature. $P_{\rm puls}$ represents the pulsating period of the post-AGB star in days. RVb represents the presence of RVb phenomenon with 'y' indicating 'yes' and 'n' indicating 'no'. $T_{\rm turn-off}$ represents the turn-off temperature (which defines the limit for which elements with higher condensation temperatures become depleted). Depletion profile shapes were adopted from \citet{Oomen2019A&A...629A..49O}. Saturated profiles are assigned by 'S', plateau profiles are assigned by 'P', in cases where the distinction is not clear - 'U', and the stars that are not depleted - 'N'. More details on the tabulated information can be found in the individual studies mentioned in column 'Ref': 1 - \citet{Oomen2018}, 2 - \citet{Kiss2007MNRAS.375.1338K}, 3 - \citet{Kiss2017}, 4 -\citet{Manick2017}, 5 - \citet{Samus2009yCat....102025S}, 6 - \citet{Giridhar2000ApJ...531..521G}, 7 - \citet{Maas2003A&A...405..271M}, 8 - \citet{Giridhar1994ApJ...437..476G}, 9 - \citet{VanWinckel1995PhDT........31V}, 10 - \citet{Maas2002}, 11 - \citet{Maas2005}, 12 - \citet{VanWinckel1998A&A...336L..17V}, 13 - \citet{Giridhar1998ApJ...509..366G}, 14 - \citet{Maas2003A&A...405..271M}, 15 - \citet{Bodi2019}. \\
       
        \end{tablenotes}
       
        \label{tab:initialdata}
    \end{table}   
    
    \begin{table}
        \centering
         \caption{Geometrical modelling results for VLTI/PIONIER and VLTI/MIDI  surveys}
     	\label{tab:rt-modelling}
        \begin{tabular}{c|l|c|l|l|l|l|l|c|} 
            \hline
            & & \multicolumn{6}{c}{Geometrical modelling for VLTI/PIONIER survey} & \multicolumn{1}{c}{A mid-IR interferometric survey with VLTI/MIDI}\\ 
            & & \multicolumn{6}{c}{Kluska et al. 2019} & \multicolumn{1}{c}{Hillen et. al. 2017}\\ 
    
            \hline
            \#ID& Name & Model & Tring & $\theta$, mas & $\delta\theta$ & $i$ & PA  & $\theta$, mas \\ 
            \hline
            1& U\,Mon & br5 & 2619$^{+138}_{-132}$&5.49$^{+0.03}_{-0.03}$& 0.02$^{+0.02}_{-0.01}$ & 57.9$^{+1.6}_{-1.5}$ & 45$^{+1}_{-1}$ & 50$^{+0.5}_{-0.5}$ $^{*}$ \\ 
           
            2&IRAS 08544-4431 & br5 & 875$^{+10}_{-9}$&14.3$^{+0.1}_{-0.1}$& 0.48$^{+0.01}_{-0.01}$& 21.3$^{+0.8}_{-0.8}$ & 12$^{+3}_{-3}$& 46$^{+2}_{-2}$\\ 
            
            3& HR 4049 & br5 & 711$^{+24}_{-23}$& 16.4$^{+0.5}_{-0.5}$& 0.8$^{+0.1}_{-0.1}$& 49.3$^{+3.2}_{-3.3}$ & 63$^{+7}_{-6}$& 42$^{+2}_{-2}$\\ 
    
            4& IW\,Car & br5 & 1047$^{+15}_{-15}$& 23.0$^{+0.7}_{-0.6}$& 0.71$^{+0.02}_{-0.02}$& 44.6$^{+1.7}_{-1.8}$ & 155$^{+2}_{-2}$&38$^{+3}_{-3}$ \\ 
    
            
            5&IRAS 15469-5311 &br4 & 818$^{+17}_{-17}$&10.4$^{+0.3}_{-0.3}$&0.39$^{+0.03}_{-0.03}$& 53.5$^{+1.8}_{-2.2}$ & 64$^{+2}_{-2}$ &36$^{+3}_{-4}$  \\ 
    
            6&IRAS 17038-4815& sr6& 2132$^{+118}_{-104}$ &5.3$^{+0.3}_{-0.3}$ & 1.2$^{+0.1}_{-0.1}$ & 36.7$^{+5.1}_{-7.8}$ & 158$^{+7}_{-8}$&  
        19.8$^{+0.8}_{-0.9}$ $^{*}$  \\ 
            7& RU\,Cen & b2 & - & -& -& -& -&-\\ 
            8& AC\,Her & s0& 5812$^{+2713}_{-2239}$ & - &-& -& -&65.7$^{+0.7}_{-0.7}$\\
            \hline
         
        \end{tabular}
        \begin{tablenotes}
            \small
            \item \textbf{Notes:} \citet{Kluska2019A&A...631A.108K}: $\theta$ represents an inner rim diameter for the fitted ring, $\delta\theta$ represents the width of the ring in the units of the ring radius. Column 'Model' represents components of the model that were used for the analysis: 'br5' - the primary star, the secondary star, a ring, and a background, 'br4' -  the primary star, the secondary star, a ring, and a background, 'b2' - two stars and a background flux, 'sr6' - the primary star, the ring, and the background, 's0' - a single star and background flux.\\
            \citet{Hillen2017}: $\theta$ represents an outer diameter of the disc. This parametric modelling assumed that the dust disc starts at the sublimation radius, which is not applicable for AC\,Her as it has confirmed inner disc radius is much larger than the dust-sublimation radius \citep[][]{Hillen2015A&A...578A..40H}.\\
            $^{*}$ - observations were made during the maximum brightness of the system.
        \end{tablenotes}
    \end{table}
    
   
\end{landscape}
