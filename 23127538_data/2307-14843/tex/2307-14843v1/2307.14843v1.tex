\documentclass[a4paper]{amsart} %feb-28-2023
\usepackage{amsmath,amssymb,amsthm}
\usepackage[backref=page, colorlinks, linkcolor=blue]{hyperref}
%\usepackage[notref, notcite]{showkeys}
\usepackage[utf8]{inputenc}
\usepackage{enumitem}
%\usepackage{marginnote}

\newtheorem{theorem}{Theorem}[section]
\newtheorem{lemma}[theorem]{Lemma}
\newtheorem{corollary}[theorem]{Corollary}
\newtheorem{proposition}[theorem]{Proposition}
%\newtheorem*{theorem*}{Theorem}
%\newtheorem*{corollary*}{Corollary}

\theoremstyle{definition}
\newtheorem{definition}[theorem]{Definition}
\newtheorem{example}[theorem]{Example}
\newtheorem{remark}[theorem]{Remark}

\begin{document}
\title[Uniform distribution mod $1$ for continued fractions]{Uniform distribution mod $1$ for sequences of ergodic  sums and continued fractions}
\author{Albert M. Fisher}
\address{Instituto de Matemática e Estatística, Universidade de São Paulo}
\email{afisher@ime.usp.br}
\thanks{AF was partially supported by FAPESP grant \#2016/25053-8}
\author{Xuan Zhang}
\address{Instituto de Matemática e Estatística, Universidade Federal Fluminense}
\email{xuanz@id.uff.br}
\date{\today}
\thanks{XZ was partially supported by FAPESP grant \#2018/15088-4.}
\subjclass[2010]{11K50, 11K06, 37A50}%, 60G10}

\begin{abstract}
  We establish a necessary and sufficient
  condition for a sequence of ergodic sums (i.e.~  Birkhoff partial
  sums)
  to be almost surely uniformly distributed mod $1$. Applications are given when the sequence is generated by a Gibbs-Markov map. In particular,
  we show that for almost every real number, the sequence of denominators of the convergents of its continued fraction expansion satisfies Benford's law.
\end{abstract}
\maketitle

\section{Introduction}
For $x\in (0,1)$ irrational,  denote by $[a_1, a_2, \ldots]$ its continued fraction expansion  and write $p_n/q_n$ for  the $n$-th convergent $[a_1, \ldots, a_n]=1/(a_1+\cdots +1/a_n)$. The sequence $(\log q_n(x))_{n\in\mathbb N}$ behaves much like  a sum of i.i.d.~random variables in that it obeys a range of  classical limit theorems 
(\cite{Philipp1971}). In this note we are interested in uniform distribution  (u.d.) mod $1$. Recalling the definition,   a sequence $(x_n)$ of real numbers is said to be  u.d.~ mod $1$ iff for every $t\in [0,1)$,
\begin{equation*}%\label{eq:equivud}
\lim_{N\to\infty}\frac 1N\#\{1\leqslant n\leqslant N: \{ x_n\}\leqslant t\}=t,
\end{equation*}
where $\{x_n\}$ represents the fractional part of $x_n$. Equivalently,
 on the circle $\mathbb T=\mathbb R/\mathbb Z$ with Lebesgue measure $dt$, then for every continuous function $f\in C(\mathbb T)$ we have:
$$\lim_{N\to\infty}\frac1N \sum_{n=1}^{N}f(x_n)=\int_{\mathbb T} f(t) dt.$$

Uniform distribution mod $1$ is closely related to {\em Benford's law}  or {\em the first-digit law}, that is, the proportion of entries with the first digit $d$, $d\in\{1,\ldots, 9\}$, tends to $\lg (1+1/d)$. Here $\lg=\log_{10}$ is the decimal logarithm. First noted by Benford as well as Newcomb, this specific distribution
has been observed to occur in a variety of social science statistics and data sets as well as in many number-theoretical experiments. We refer to \cite{BergerHill2011} for a comprehensive survey of Benford's law. For a nonzero
%% I PUT  "nonzero"
real number $x$, let $s(x)$ be its decimal significand, which is the unique number $t\in [1,10)$ such that $|x|=10^kt$ holds for some $k\in\mathbb Z$. A sequence of real numbers $(x_n)_{n\in\mathbb N}$ is said to be a {\em Benford sequence} if the corresponding sequence of decimal significands  $(s(x_n))_{n\in\mathbb N}$ is distributed in $[1, 10)$ with density $\lg$, that is,
$$\lim_{N\to\infty}\frac1N\#\{1\leqslant n\leqslant N: s(x_n)\leqslant t\}=\lg (t), \quad \forall t\in [1, 10).$$
%  Equivalently, one can consider the density of significant digits. The first significant decimal
% digit of $x$, denoted by $D_1(x)$, is the unique integer $j\in\{1,2, \ldots ,9\}$ such that $10^k j\leqslant |x|<10^k (j + 1)$ holds for some $k\in\mathbb Z$.
% Similarly, for every $m\geqslant 2, m\in\mathbb N$, the $m$-th significant decimal digit of $x$,
% denoted by $D_m (x)$, is defined inductively as the unique integer $j\in\{0,1,...,9\}$ such that
% $$10^k\left(\sum_{i=1}^{m-1} D_i(x) 10^{m-i}+j\right)\leqslant |x|<10^k\left(\sum_{i=1}^{m-1} D_i(x) 10^{m-i}+j+1\right)$$ holds for some $k\in\mathbb Z$. Then $(a_n)_{n\in\mathbb N}$ is Benford if for every $m\in\mathbb N$, $d_1\in\{1,2,\ldots,9\}$ and $d_k\in\{0,1,...,9\}$,
% $k\geqslant 2$,
% $$\lim_{N\to\infty}\frac1N \#\{1\leqslant n \leqslant N: D_j (a_n) = d_j,  ~j=1, 2,\ldots, m\}=\lg\Big(1+\Big(\sum_{j=1}^m 10^{m-j}d_j\Big)^{-1}\Big).$$
% Note that the definition of a Benford sequence determines not only the density of each significant digit $D_m$ but also the joint densities.
Many sequences are known to be Benford; examples are $(2^n)_{n\in\mathbb N}$ and the Fibonacci sequence. The sequences of all natural numbers and of the prime numbers are not Benford, but are {\em weakly Benford} (with the usual  Ces\'aro density being
% (I PUT " Ces\'aro")
replaced by logarithmic density). %More examples of Benford sequences can be obtained from the extensively-studied uniformly distributed sequences (also called equidistributed sequences, see e.g.~ \cite{KuipersNiederreiter1974, DrmotaTichy1997}),
Benford's law is related to uniform distribution mod $1$ via the following equivalent condition (see e.g.~\cite[Theorem 4.2]{BergerHill2011}):
% ( I replaced relation by "condition")
\begin{center}$(x_n)_{n\in\mathbb N}$ is a Benford sequence if and only if $(\lg x_n)_{n\in\mathbb N}$ is u.d.~mod $1$. 
\end{center}
Hence the question under consideration can be rephrased by whether $(q_n(x))_{n\in\mathbb N}$ is a Benford sequence.

When $x$ is a quadratic irrational, several authors at about  the same time (\cite{JagerLiardet1988, KanemitsuNagasakaRauzyEtAl1988, SchatteNagasaka1991}) showed that $(q_n(x))_{n\in\mathbb N}$ is indeed a Benford sequence. %Note that $(p_n(x))_{n\in\mathbb N}$ is therefore also Benford by using Weyl's criterion and the fact that $p_n/q_n$ converges to $x$ fast enough.
In particular this reconfirms that the Fibonacci sequence, being equal to $(q_n(\frac{1+\sqrt 5}{2}))_{n\in\mathbb N}$, is Benford. Jager asked whether (as mentioned in \cite{KanemitsuNagasakaRauzyEtAl1988})  $(q_n(x))_{n\in\mathbb N}$ is Benford for almost every $x$. In this note we answer this question affirmatively.
\begin{theorem}\label{thm:benford}
$(q_n(x))_{n\in\mathbb N}$ is a Benford sequence for almost every $x$. Equivalently, $(\lg q_n(x))_{n\in\mathbb N}$ is u.d.~mod $1$ for almost every $x$.
\end{theorem}
We note that Schatte (\cite{Schatte1990}) presented a proof. %As we had some difficulties understanding parts of it (the estimates of $\lambda_{xyz}$ in there),
We give a new proof, introducing  methods which we hope  lead to new insight on this and related questions. For example, it follows immediately from our proof that for any constant $\rho\in\mathbb R$ the sequence $(\lg q_n(x)+ n\rho)_{n\in\mathbb N}$ is also u.d.~mod $1$ for almost every $x$; the same holds for the sequences $(a_1+\cdots + a_n)_{n\in\mathbb N}$ and $(\lg(a_1\cdots a_n))_{n\in\mathbb N}$.

In what follows will use the natural logarithm $\log$ in  place of $\lg$; it will be clear that the proofs work for both.


The almost sure behavior of the sequence $(\log q_n(x))_{n\in\mathbb N}$ is somewhat different from the situation where $x$ is a quadratic irrational, which equivalently
has an eventually periodic continued fraction expansion. In this  case, as shown by Jager-Liardet (\cite{JagerLiardet1988}), $\log q_n(x)=\frac{n}{l(x)}\alpha(x)+c_n(x)+o(1)$ for large $n$, where $\alpha(x)$ is an irrational number, $l(x)$ is the period in the continued fraction expansion of $x$, and $c_n=c_{n+l}$ takes a finite number of values periodically. One can then deduce uniform distribution mod $1$ of the sequence $(\log q_n(x))_{n\in\mathbb N}$ from that of the irrational rotation $(n\alpha(x))_{n\in\mathbb N}$, the classical Weyl's Theorem. %and by applying a difference theorem of van der Corput.  (\cite[Chapter 1, Theorem 3.3]{KuipersNiederreiter1974}).
In the almost sure case, $\lim_{n\to\infty}\log q_n(x)/n$ converges to the Lévy constant $\frac{\pi^2}{12\log 2}$ and the rate of convergence is of the order $O(\sqrt{n^{-1}\log\log n})$ by the law of iterated logarithm of Philipp-Stackelberg (\cite{PhilippStackelberg1969}). However we do not know if the L\'evy constant is irrational; moreover this rate of convergence is too weak to  show almost sure uniform distribution of $(\log q_n(x))_{n\in \mathbb N}$ as a direct consequence of that for a rotation; an explanation is that this gives  lim sup rather than the sought-for time average information.
Nevertheless, one might still guess that $(\log q_n(x))_{n\in\mathbb N}$ is u.d.~mod $1$ for almost every $x$, given that the statistics of $\log q_n(x)$ resembles well that of a partial sum of i.i.d. random variables and that, as shown by Robbins (\cite{Robbins1953}),  the sequence of partial sums of i.i.d. random variables  (if not supported on a rational lattice) is almost surely uniformly distributed (a.s.u.d.) mod $1$.
% Moreover Schatte (\cite{Schatte1988a, Schatte1991}) obtained a law of iterated logarithm regarding the rate of convergence.

The main idea of our proof is to approximate the sequence $(\log q_n(x))_{n\in\mathbb N}$ fast enough by a series of sequences of ergodic 
sums (in other words, sequences of partial sums of stationary processes), each  of which is a.s.u.d.~mod $1$ (Lemma \ref{lem:aprox}). %Almost sure uniform distribution mod $1$ for sequences of partial sums of i.i.d. random variables was first studied by Robbins (\cite{Robbins1953}), and estimates of the rate of convergence were given by Schatte (\cite{Schatte1988a, Schatte1991}) and more recently by Berkes and Weber (\cite{BerkesWeber2009}) among others. 
Sequences of partial sums of stationary processes were studied by Holewijn (\cite{Holewijn1969/70}) and more recently by Chenavier, Mass\'e and Schneider (\cite{ChenavierMasseSchneider2018}) where they found a necessary and sufficient condition for such a sequence to be a.s.u.d.~mod $1$. We use a classical argument of Furstenberg
% The spelling is "Furstenberg"- ( in German Furstenberg means mountain of the %prince while Furstenburg means city of the prince   !!)
to find a (very different, also equivalent)  condition in terms of coboundaries for sequences of the partial sums
%changed to "partial sums"
generated by ergodic maps (Corollary \ref{cor:ud}). When the sequence is generated by a Gibbs-Markov map we give examples where the coboundary condition is verified (Theorem \ref{thm:ud_a}), and these provide the approximating sequences to $(\log (q_n))$.
%
%Using this result, we characterize the observables that form a.s.u.d.~mod $1$ sequences of ergodic sums generated by mixing Gibbs-Markov maps (Theorem \ref{thm:ud_gm}) and by mixing Gibbs-Markov maps with the b.i.p.~property (Corollary \ref{cor:ud_gm}). In particular this applies to sequences of ergodic sums generated by the continued fraction map $x\mapsto\{1/x\}$. %As an application, we can deduce that the sequence $(a_1\cdots a_n)_{n\in\mathbb N}$ is Benford a.e. (Theorem \ref{thm:ud_a}). 
%Then we consider a class of sequences that can be uniformly approximated by a series of a.s.u.d.~mod $1$ sequences (Theorem \ref{thm:ud_aerg}) and show in the last section that $(\log q_n)_{n\in\mathbb N}$ belongs to this class.

% One may ask what is the almost sure rate of convergence of the density of $(\lg q_n)_{n\in\mathbb N}$ to the uniform distribution, that is, the rate of the discrepancy. Note that Schatte (\cite{Schatte1988a}) obtained an estimate of $O(\sqrt{\log\log N}/\sqrt N)$ for sequences of partial sums of generic i.i.d. random variables. So far we can only obtain an estimate of $O((\log N)^{2+\epsilon}/\sqrt N)$ for sequences of partial sums of some special stationary processes. The rate regarding $(\lg q_n)_{n\in\mathbb N}$ is yet unknown to us.

\section{Sequence of ergodic sums}
Let $(X, \mu)$ be a standard probability space, $T:X\to X $ be a $\mu$-preserving transformation, and $f:X\to \mathbb R$ be a measurable function (observable). For any $n\in\mathbb N$, $x\in X$, form the ergodic sum $\mathcal S_n f(x)=f(x)+\cdots + f(T^{n-1}x)$. 


Assuming  ergodicity of $T$, we apply an argument of  Furstenberg (\cite {Furstenberg1961}, [Lemma 2.1]; see also Theorem 4.4 of \cite{BedfordFisherUrbanski02}) to show that $\mathcal S_nf(x)$ is almost surely uniformly distributed (a.s.u.d.) mod $1$ under a coboundary condition for $f$. Note that we do not assume the compactness of $X$ as done
% "don"e not "did"
in \cite{Furstenberg1961}.

\begin{theorem}\label{thm:asud} %Let $(X, \mathcal A, \mu)$ be a probability space.
%Suppose $X$ is a compact metric space and $(X, \mu)$ a probability space.
Let $T$ be an ergodic $\mu$-preserving transformation and $f:X\to \mathbb R$ measurable. Suppose the equation $e^{2\pi i k f(x)}=g(x)/g(Tx)$ a.s.~ has no solution for any integer $k\neq 0$ and $g\in L^2(X)$; then $\mathcal S_nf(x)$ is a.s.u.d. mod $1$.
\end{theorem}

\begin{proof}
We write $\mathbb T={\mathbb R}/{\mathbb Z}$ with the Lebegue measure $dt$.
Let $Y=X\times \mathbb T$ and define the skew product  $T_f: Y \to Y$ by $T_f(x,
t)=(Tx, t+f(x))$. Then $T_f^n(x,t)=(T^nx, t+\mathcal S_nf(x))$. 
Now firstly
% added "now"
$\lambda=\mu\times dt$ is an invariant measure for $T_f$; this is because for any $F\in L^\infty_\lambda(Y)$
\begin{align*}
\int_Y F\circ T_f d\lambda&=\iint_Y F(Tx, t+f(x))dt d\mu =\iint_Y F(Tx, t)dtd\mu\\
&=\iint_Y F(x, t)dt d\mu=\int_Y F d\lambda
\end{align*}
by the invariance of $dt$ under translation and the invariance of $\mu$ under $T$.
Moreover the skew product transformation is ergodic; we follow  Furstenberg's proof of this part.  If not, then there is a
non-constant invariant function $F\in L^2_\lambda(Y)$ for $T_f$, that
is, $F\circ T_f=F$ almost everywhere. Now for almost every $x$, $F(x,
\cdot)\in L^2(\mathbb T)$.
We write  its Fourier expansion,  $F(x, t)= \sum_{n\in\mathbb Z}a_n(x)e^{2\pi i n t},$
with $a_n\in L^2_\mu(X)$. Then $$F\circ T_f(x, t) = \sum_{n\in\mathbb Z}a_n(Tx)e^{2\pi i n f(x)}e^{2\pi i n t}.$$ $F\circ T_f= F$ implies $a_n(x)=a_n(Tx)e^{2\pi inf(x)}$, hence $|a_n(x)|=|a_n(Tx)|$, for every $n\in\mathbb Z$ and almost every $x$. The ergodicity of $T$ implies that $|a_n(x)|$ is a.s.~ a constant  for every $n\in\mathbb Z$. Since $F$ is not  constant, one of $|a_n(x)|$ with $n\neq 0$  is not $0$. Say for $k\neq 0$, $|a_k(x)|$ is a nonzero constant. Then $e^{2\pi i k f(x)}=a_k(x)/a_k(Tx)$ a.s., a contradiction to the hypothesis,
proving ergodicity of $T_f$.

Next we show how u.d.~ in $\mathbb T$ follows from this.  Let $h\in C(\mathbb T)$. We extend this to $H: Y\to \mathbb R$ by
$ H(x,t)= h(t)$. This is a measurable function which is continuous
on the fiber over $x$, for each $x$. Note that $H\in L^1_\lambda(Y)$.

The Birkhoff ergodic theorem implies that for every such $H$, hence for each  continuous function $h\in C(\mathbb T)$,  
there is a
full measure set $E_h$ such that for every $(x, t)\in E_h\subseteq Y,$
\begin{equation}\label{eq:1}
\lim_{N\to\infty}\frac1N \sum_{n=0}^{N-1}H\circ T_f^{n}(x,t)=\int_Y
H d\lambda=\int_{\mathbb T}h dt.
\end{equation}
Since $\mathbb T$ is compact, $C(\mathbb T)$ is separable. Suppose $\{h_i\}_{i\in
  \mathbb N}$ is a dense subset of $C(\mathbb T)$, then \eqref{eq:1} holds
for every $h_i$ and all $(x,t)$ in $\cap_{i\in \mathbb N} E_{h_i}$. This passes over
to every $h\in C(\mathbb T)$:  we approximate $h$ $\varepsilon$-uniformly by functions
$h_i$, and  both sides of \eqref{eq:1} are within $\varepsilon$.

In conclusion,  $\lambda$-almost every $(x,t)$ is a {\em fiber generic point}, that is, \eqref{eq:1} holds for every $h\in C(\mathbb T)$. 

Suppose $(x,t)$ is a fiber generic point, then $(x, t+s)$ is also
a fiber generic point for any $s\in \mathbb T$. This is because, for every $h\in
C(\mathbb T)$ defining $h_s(\cdot)=h(\cdot+s)\in C(\mathbb T)$ and
$H_s(x,t)=h_s(t)$,
\begin{equation*}
\lim_{N\to\infty}\frac1N \sum_{n=0}^{N-1}H\circ
  T_f^{n}(x, t+s)=\lim_{N\to\infty}\frac1N
                      \sum_{n=0}^{N-1}H_s\circ
                      T_f^{n}(x,t)=\int_{\mathbb T} h_sdt=\int_{\mathbb T} h d t.
\end{equation*}
So the set of all fiber generic points is $E\times \mathbb T$ for some
$E\subset X$. Since this set has full $\lambda$-measure, 
$\mu(E)=1$. Now for every $x\in E$ $(x,0)$ is a fiber generic point. This means that
for every  $h\in  C(\mathbb T)$,
 $$\lim_{N\to\infty}\frac1N \sum_{n=0}^{N-1}h(\mathcal S_nf(x))=\int_{\mathbb T} hd t,$$
%Using continuous functions $h$ to approximate $ \mathbf 1_{[a,b)}$, one then gets that 
%$$\lim_{N\to\infty}\frac1N \sum_{n=0}^{N-1}\mathbf 1_{[a,b)}(\{S_nf(x)\})=b-a,$$
 equivalently $\mathcal S_nf(x)$ is u.d. mod $1$ for every $x\in E$.
\end{proof}




The coboundary condition is close to being 
a necessary as well as sufficient condition for a.s.u.d. Recall
Weyl's criterion (e.g.~ \cite{KuipersNiederreiter1974}): $(x_n)$ is u.d. mod $1$ if and only if for every integer $k\neq 0$
$$\lim_{N\to\infty}\frac1N\sum_{n=1}^N e^{2\pi i k x_n}=0.$$
%To this end, we use a criterion of 
%Chenavier, Massé and Schneider (\cite{ChenavierMasseSchneider2018}) on when the partial sums of a stationary process is a.s.u.d. mod $1$. In there the result is stated in terms of Benford's law. Denote by $\mathbb Z^\ast$ the set of all non-zero integers.
%\begin{theorem}\label{thm:ud}\cite[Theorem 3.3]{ChenavierMasseSchneider2018} $(\mathcal S_n f(x))_{n\in\mathbb N}$ is a.s.u.d. mod $1$ if and only if for every $m\in\mathbb Z^\ast$ 
%$$\lim_{N\to\infty}\frac{1}{N}\sum_{n=1}^N \int e^{2\pi i m \mathcal S_n f(x)}d\mu=0.
%$$
%\end{theorem}

\begin{corollary}\label{cor:ud} Let $T$ be an ergodic $\mu$-preserving transformation. Then for   $f$ measurable, $\mathcal S_nf(x)$ is a.s.u.d. mod $1$ if and only if  the equation $e^{2\pi i k f(x)}=g(x)/g(Tx)$ a.s.~ has no solution for any integer $k\neq 0$ and $g\in L^2$ with $\int g d\mu \neq0$.
\end{corollary}
\begin{proof}
Suppose for some integer $k\neq 0$ and $g\in L^2$, $e^{2\pi i k f(x)}=g(x)/g(Tx)$ a.s. Then $|g|=|g\circ T|$. We can assume that $|g|$=1 by ergodicity of $T$. So $$\frac1N \sum_{n=1}^N  e^{2\pi i k \mathcal S_nf(x)}  =\frac 1N\sum_{n=1}^N  g(x)\bar{g}(T^nx) \to g(x) \int \bar{g} d\mu\ \text{a.s.}$$
by ergodic theorem. It follows from Weyl's criterion that $\mathcal S_nf(x)$ is a.s.u.d. mod $1$ if and only if $\int g d\mu =0$. The Corollary then follows from Theorem \ref{thm:asud}.
\end{proof}


For circle-valued skew products the coboundary condition is called
{\em aperiodicity} in \cite{AaronsonDenker2001}. This terminology makes sense in view of Proposition 4.3 of \cite{BedfordFisherUrbanski02}. However that
fact for the skew product  does not directly imply  the non-uniform distribution,  which is why we give this   argument.


\subsection{Gibbs-Markov maps} When $T$ is a Gibbs-Markov map, we give some sufficient conditions for the coboundary condition to hold that are easier to verify.
First recall the definition of a Gibbs-Markov map from \cite{AaronsonDenkerUrbanski1993},
\cite{AaronsonDenker2001}. Let $(\Omega,\mathcal B, \mu)$ be a probability
space and let $T$ be a non-singular transformation as considered by Aaronson and Denker;
these are generally noninvertible maps with a countable number of branches, each branch of which is thus individually non-singular; % in practice  $T$ is usually assumed to be measure-preserving.
Consider a countable partition $\alpha$ of $\Omega\mod \mu$. Denote by $\alpha_0^{n-1}$ the refined partition $\bigvee_{i=0}^{n-1}T^{-i}\alpha$ and by $\sigma(\cdot)$ the $\sigma$-algebra generated by a partition.
\begin{definition}%\label{def:gm} 
$T$ is called a Gibbs-Markov map if the following conditions are satisfied.
\begin{enumerate}[label=(\alph*)]
	\item $\alpha$ is a strong generator of $\mathcal{B}$ under $T$, i.e. $\sigma(\{T^{-n}\alpha:n\in\mathbb N_0\})=\mathcal{B} \mod \mu$.
	\item  $\inf\limits_{a\in\alpha}\mu(Ta)>0$.
	\item For every $a\in\alpha$, $Ta\in \sigma(\alpha) \mod \mu$, moreover the restriction $T|_{a}$ is invertible and non-singular. 
	\item For every $n\in\mathbb N$ and $a\in\alpha_0^{n-1}$, denote the non-singular inverse branch of $T^{-n}$ on $T^n a$ by $v_a: T^n a\rightarrow a$ and its Radon-Nikodym derivative by $v'_a$. There exist $r\in (0,1)$ and $M>0$ such that for any $n\in \mathbb N, a\in\alpha_0^{n-1}$ and $x,y\in T^n a$ a.e.
			$$\left|\dfrac{v'_a(x)}{v'_a(y)}-1\right|\leqslant M\cdot r(x,y),
			$$
\end{enumerate}
where $r(x,y)$ is the metric $r^{m(x,y)}$ where $$m(x,y)=\min\{n\in\mathbb N: T^{n-1}(x) \text{ and } T^{n-1}(y) \text{ belong to different elements of } \alpha \}.$$
\end{definition}
$T$ is said to be topologically mixing if for any $a, b\in\alpha$ there is $n_{a,b}\in\mathbb N$ such that for any $n\geqslant n_{a,b}$, $T^n a \supset b$.  Note that a topologically mixing and measure preserving Gibbs-Markov map is always ergodic (\cite[Corollary]{AaronsonDenker2001}). A function $f$ is said to be locally Lipschitz if $$D_\alpha (f)=\sup_{a\in\alpha}\sup_{x,y\in a}\frac{|f(x)-f(y)|}{r(x,y)}<\infty,$$
and Lipschitz if $$D(f)=\sup_{x,y\in \Omega}\frac{|f(x)-f(y)|}{r(x,y)}<\infty.$$ Let $\beta$ be the finest partition such that $\sigma(T\alpha)=\sigma(\beta)$. 

\begin{proposition}\label{prop:cob} Let $T$ be a topologically mixing and $\mu$-preserving Gibbs-Markov map.  Suppose $e^{2\pi i k f(x)}=g(x)/g(Tx)$ a.s.~ for some integer $k\neq 0$ and measurable $g$.
\begin{enumerate}
\item If $f$ is $\alpha$-measurable, then $g$ is $\beta$-measurable.
\item Suppose $p$ is a fixed point of $T$ and, with respect to the metric $r$, $p$ lies in the interior of a partition member and every open ball around $p$ has positive measure. If $p$ is a continuity point of both $T$ and $f$ and if $f$ is locally Lipschitz, then $e^{2\pi i k f(p)}=1$.
\end{enumerate}
\end{proposition}
\begin{proof}
\begin{enumerate}
\item This is \cite[Theorem 3.1]{AaronsonDenker2001}.
\item We can assume that $|g|$=1 by ergodicity of $T$. According to \cite[Corollary 2.2]{AaronsonDenker2001}, the condition that $f$ is locally Lipschitz implies that $g$ is Lipschitz. %The ergodicity of $T$ implies that $|g|$ is a constant, say $|g|=1$. Then $g/g\circ T=g\cdot \bar{g}\circ T$ is also Lipschitz, as $$|g(x)\bar{g}(Tx)-g(y)\bar{g}(Ty)|\leqslant |g(x)-g(y)|+|g(Tx)-g(Ty)|\leqslant (M+ M/r)\cdot r(x,y),$$ where $M=D(g)$. 
For a fixed point $p$ in the assumption, there is a sequence $x_n$ tending to $p$ such that $|g(x_n)-g(Tx_n)|\leqslant D(g) r(x_n, Tx_n)$ and $e^{2\pi i k f(x_n)}=g(x_n)/g(Tx_n)$. Since both $Tx_n$ and $x_n$ tend to $p$ and since $f$ is continuous at $p$, it follows that $e^{2\pi i k f(p)}=1$.
\end{enumerate}
\end{proof}

\section{Applications to continued fractions}%\label{sec:ctf}
We return to continued fractions and prove Theorem \ref{thm:benford} in this section. 
Let $\Omega=(0,1)$, $\mu$ be the Gauss measure $d\mu=\frac{1}{\log 2}\frac{1}{1+x} dx$,  and $T$ be the continued fraction map $x\mapsto \{1/x\}$. Denoting the continued fraction expansion of an irrational number $x$ by $[a_1(x), a_2(x), \ldots]$, then $T$ is equal to the shift map $[a_1, a_2,\ldots]\mapsto [a_2, a_3, \ldots]$. Let $\alpha$ be the partition into cylinder sets determined by $a_1$, i.e. $\alpha=\{\{x: a_1(x)=n\}: n\in \mathbb N\}$. Let $r(x,y)={(1/2)}^{\min\{n:\,a_j(x)=a_j(y), 1\leqslant j\leqslant n\}}$. The constant $r=1/2$ is chosen so that $r^2\cdot |(T^2)'|\geqslant 1$. One can check that $T$ is a topologically mixing, $\mu$-preserving Gibbs-Markov map (\cite[Example 2]{AaronsonDenker2001}), so we can apply results from the previous section to the continued fraction map.

\begin{proposition}\label{prop:cob_cf}
The equation $e^{2\pi i k f(x)}=g(x)/g(Tx)$ has no solution for any integer $k\neq 0$ and measurable $g$ in each of the following cases.
\begin{enumerate}
\item $f(x)=a_1(x)$
\item $f(x)=\log (a_1(x))$
%\item $f(x)=-\log x$.
\item $f(x)=-\log (T^lx)$, where $l\in\mathbb N\cup\{0\}$.
%\item $f(x)=-\log x+\rho$, where $\rho\in \mathbb R$.
\end{enumerate}
\end{proposition}
\begin{proof}
This follows immediately form Proposition \ref{prop:cob}, since 
\begin{enumerate}
\item $a_1(x)$ is $\alpha$-measurable and $\beta=\{\Omega\}$. For $g$ to be $\beta$-measurable, it has to be constant, but $e^{2\pi i k a_1}$ is not.
\item The same reason as (1).
\item First let $l=0$. Note that $\log x$ is locally Lipschitz, because $D_{[n]}(\log)$ has the order of $\frac 1n$. Consider the fixed point $p=\frac{\sqrt{5}+1}{2}=[1,1,\ldots]$, which is in the interior of the cylinder set $\{x: a_1(x)=1\}$. Both $T$ and $\log$ are continuous at $p$, but $e^{-2\pi i k \log(p)}\neq 1$ for any integer $k\neq 0$. For $l\in\mathbb N$,
$\log\circ T^l$ is locally Lipschitz, as $D_{[n]}(\log\circ T^l)$ has the order of $\frac{1}{r^l n}$. But again $e^{-2\pi i k \log(T^lp)}\neq 1$ for any integer $k\neq 0$. 
%\item $e^{2\pi i k (-\log(x)+\rho)}$ cannot be $1$ on both $[1,1,\ldots]$ and $[2,2,\ldots]$.
\end{enumerate}
\end{proof}

Consequently, applying Theorem \ref{thm:asud},
\begin{theorem}\label{thm:ud_a} With respect to $\mu$, hence also with respect to  Lebesgue measure, each of the following sequences is a.s.u.d.~mod $1$.
\begin{enumerate}[label=(\roman*)]
\item $(a_1(x)+\cdots a_n(x))_{n\in\mathbb N}$
\item
$(\log(a_1(x)\cdots a_n(x)))_{n\in\mathbb N}$
%\item $(-\mathcal S_n \log(x))_{n\in\mathbb N}$ is a.s.u.d.~mod $1$.
\item $(-\mathcal S_n \log(T^l x))_{n\in\mathbb N}$, where $l\in\mathbb N\cup\{0\}$.
\end{enumerate}
\end{theorem}
%\begin{proof}
%Apply Corollary \ref{cor:ud_gm} with $h_1(x)=\log a_1(x)$ and $h_2(x)=-\log(x)$ respectively for each statement. Note that both $h_1$ and $h_2$ are unbounded and that $D_{[n]} h_1=0$ and $D_{[n]} h_2$ is of the order of $1/n$.%$\lg(1+\frac 1n)$.
%\end{proof}
\begin{remark}
Item (ii) implies that $(a_1(x)\cdots a_n(x))_{n\in\mathbb N}$ is a Benford sequence a.s.
\end{remark}

Now we prepare for the proof of Theorem  \ref{thm:benford}. Recalling that $q_n$ denotes the denominator of the convergent $[a_1,\ldots, a_n]$,  to extend the previous theorem to sequences such as $(\log q_n)$ that are not necessarily ergodic sums, we make use of a simple approximation lemma. Given a sequence $(u_n(x))_{n\in\mathbb N}$ and a series of sequences $(u_n^{(k)}(x))_{n\in\mathbb N}$, where $k\in\mathbb N$ and $x\in\Omega$,
\begin{lemma}\label{lem:aprox}
Suppose that for every $k\in\mathbb N$ the sequence $(u_n^{(k)}(x))_{n\in\mathbb N}$ is a.s.u.d.~mod $1$ and that $$\lim_{k\to\infty}\sup_{n> k}\|u_n^{(k)}-u_n\|_\infty=0,$$
then $(u_n(x))_{n\in\mathbb N}$ is also a.s.u.d.~mod $1$.
\end{lemma}
\begin{proof}
For every $m\in\mathbb N$, choose $k=k(m)$ such that $$2\pi\sup_{n> k}\|u_n^{(k)}-u_n\|_\infty<1/m.$$ 
So there is a full-measure set $\Omega_m'$ such that for every $x\in\Omega_m'$, 
$$2\pi\sup_{n> k}|u_n^{(k)}(x)-u_n(x)|<1/m.$$ 
Again by assumption, there is a full-measure set $\Omega_m$ such that for every $x\in\Omega_m$, $(u_n^{(k(m))}(x))_{n\in\mathbb N}$ is u.d.~mod $1$. Put $\Omega_0=\cap_{m\in\mathbb N} \Omega_m'\cap\Omega_m$, which is a full-measure set. Let $x\in\Omega_0$. Since $(u^{(k(m))}_n(x))_{n\in\mathbb N}$ is u.d.~mod $1$, in view of Weyl's criterion, choose $N_0>2km$ such that for all $N\geqslant N_0$
$$\frac 1N\left|\sum_{n=1}^N e^{2\pi i u^{(k)}_n(x)}\right|\leqslant 1/m.$$
So for all $N\geqslant N_0$ 
\begin{align*}
\frac1N \left|\sum_{n=1}^N e^{2\pi i u_n(x)}\right| & \leqslant \frac1N\left|\sum_{n=1}^N e^{2\pi i u^{(k)}_n(x)}\right|+\frac1N\sum_{n=1}^N \left|e^{2\pi i u^{(k)}_n(x)}-e^{2\pi i u_n(x)}\right|\\
&\leqslant \frac1N\left|\sum_{n=1}^N e^{2\pi i u^{(k)}_n(x)}\right|+\frac{2k}{N}+2\pi\sup_{n>k}|u_n^{(k)}(x)- u_n(x)|\\
&\leqslant 3/m.
\end{align*}
Hence $\lim_{N\to\infty}\frac 1N\sum_{n=1}^N e^{2\pi i u_n(x)}=0$, and $(u_n(x))_{n\in\mathbb N}$ is u.d.~mod $1$.
\end{proof}

Now consider the difference $\delta_n(x)=\log q_n(x)-(-\mathcal S_n\log(x))$. It is well-known that  $\delta_n(x)$ is uniformly bounded; in fact (see e.g.~ \cite{Khinchin1997}),

\begin{lemma}\label{lem:delta}
$$\delta_n(x)=-\log(1+T^{n}x\cdot [a_{n},\ldots, a_1])$$
where $[a_n,\ldots, a_1]$ represents the finite continued fraction $1/(a_n+\cdots +1/a_1).$ 
%Especially $|\delta_n(x)|\leqslant \lg 2.$
\end{lemma}
\begin{proof}
Because $$x=\frac{p_{n-1}r_n+p_{n-2}}{q_{n-1}r_n+q_{n-2}},$$ where $r_n=1/(T^{n-1}x)$,
one finds $$T^nx=\frac{xq_n-p_n}{p_{n-1}-xq_{n-1}}.$$
So
$$q_n\cdot x\cdots T^{n-1}x=(-1)^{n-1}q_n(xq_{n-1}-p_{n-1}).$$
Because $p_nq_{n-1}-p_{n-1}q_n=(-1)^{n-1}$ and $\frac{q_{n-1}}{q_n}=[a_n, \ldots, a_1]$, it follows that
\begin{align*}
\delta_n
&=\log(q_n\cdot x\cdots T^{n-1}x)\\
%&=\lg(q_n\langle xq_{n-1}\rangle)\\
&=-\log\left(1+\frac{x-\frac{p_n}{q_n}}{\frac{p_{n-1}}{q_{n-1}}-x}\right)\\
&=-\log(1+T^{n}x\cdot [a_{n},\ldots, a_1]).
\end{align*}
\end{proof}


Moreover, we can approximate $(\delta_n(x))_{n\in\mathbb N}$ by a dynamical sequence up to a uniformly small error.
\begin{lemma}\label{lem:aprox_delta} For all $k\in\mathbb N$, $n>k$ and $x\in (0,1)$,
$$|\delta_k(T^{n-k}x)-\delta_n(x)|<-\log(1-2^{-k/2}).$$
\end{lemma}
\begin{proof}
Fix $n\in\mathbb N$. For every $1\leqslant k\leqslant n$, let ${\tilde p_k}/{\tilde q_k}=[a_n,\ldots, a_{n-k+1}]$ be the $k$-th convergent of the finite continued fraction $[a_n,\ldots, a_1]$, $\tilde r_k=1/[a_{n-k+1}, \ldots, a_1]$ be the $k$-th remainder and let $\tilde p_0=0, \tilde q_0=1$, then
$$[a_n, \ldots, a_1]=\frac{\tilde p_n}{\tilde q_n}=\frac{\tilde p_k\tilde r_{k+1}+\tilde p_{k-1}}{\tilde q_k\tilde r_{k+1}+\tilde q_{k-1}}, \quad 1\leqslant k<n.$$
One calculates
\begin{align*}
&\quad \delta_k(T^{n-k}x)-\delta_n(x)\\
&=-\log(1+T^{n}x\cdot [a_{n},\ldots, a_{n-k+1}])+\log(1+T^{n}x\cdot [a_{n},\ldots, a_1])\\
&=-\log\left(1+\frac{T^nx\cdot([a_n,\ldots, a_{n-k+1}]-[a_n, \ldots, a_1])}{1+T^nx\cdot [a_n, \ldots, a_1]}\right)\\
&=-\log\left(1+\frac{T^nx\cdot \left(\tilde p_k\tilde q_{k-1}-\tilde p_{k-1}\tilde q_k\right)}{\tilde q_k\left(\tilde q_k\tilde r_{k+1}+\tilde q_{k-1}\right)+T^nx\cdot \tilde q_k\left(\tilde p_k\tilde r_{k+1}+\tilde p_{k-1}\right)}\right)\\
&=-\log\left(1+\frac{T^nx\cdot (-1)^{k-1}}{\tilde q_k\left(\tilde q_k\tilde r_{k+1}+\tilde q_{k-1}\right)+T^nx\cdot \tilde q_k\left(\tilde p_k\tilde r_{k+1}+\tilde p_{k-1}\right)}\right),
\end{align*}
hence the lemma follows from the estimate
\begin{align*}
&\quad \left|\frac{T^nx\cdot (-1)^{k-1}}{\tilde q_k\left(\tilde q_k\tilde r_{k+1}+\tilde q_{k-1}\right)+T^nx\cdot \tilde q_k\left(\tilde p_k\tilde r_{k+1}+\tilde p_{k-1}\right)}\right|\\
&< \frac{1}{\tilde q_k\left(\tilde q_k\tilde r_{k+1}+\tilde q_{k-1}\right)}\leqslant \frac{1}{\tilde q_k\left(\tilde q_k a_{n-k}+\tilde q_{k-1}\right)}
=\frac{1}{\tilde q_k\tilde q_{k+1}}\leqslant 2^{-k/2}.
\end{align*}
\end{proof}

\begin{proof}[Proof of Theorem \ref{thm:benford}]
Let $h(x)=-\log(x)$, and for every $k\in\mathbb N$ let $$h^{(k)}=h\circ T^k-\delta_k+\delta_k\circ T.$$  Note that
for all $n> k$, $$\log q_n(x)=\mathcal S_nh+\delta_n=\mathcal S_{n-k}h^{(k)}+\mathcal S_kh+\delta_k-\delta_k\circ T^{n-k}+\delta_n.$$
In order to show that  $(\log q_n(x))_{n\in\mathbb N}$ is a.s.u.d.~mod $1$, applying Lemmas \ref{lem:aprox} and \ref{lem:aprox_delta},  it suffices to show that for every $k$ the sequence $$(\mathcal S_{n-k}h^{(k)}(x)+\mathcal S_k h(x)+\delta_k(x))_{n> k}$$ is a.s.u.d.~mod $1$. For every $k$, subtract $\mathcal S_k h(x)+\delta_k(x)$  from every entry of the latter sequence. Since a translation does not change equidistribution,  the equidistribution of the sequence $(\mathcal S_{n-k}h^{(k)}(x))_{n> k}=(\mathcal S_nh^{(k)}(x))_{n\in\mathbb N}$ will complete the proof.

According to Proposition \ref{prop:cob_cf}, the equation $e^{2\pi i m h\circ T^k}=g/g\circ T$ has no solution for any integer $m\neq 0$ and measurable $g$, so $e^{2\pi i m h^{(k)}}=g/g\circ T$ also has no solution. Thus,  by Theorem \ref{thm:asud},  $(\mathcal S_nh^{(k)}(x))_{n\in\mathbb N}$ is a.s.u.d. mod $1$.

\end{proof}

\begin{remark} Lemma \ref{lem:aprox_delta} may be of independent interest. With $T$ being $\mu$-preserving, Lemma \ref{lem:aprox_delta} implies that $\lim_{n\to\infty}\int \delta_n(x) d\mu$ exists and together with the ergodic theorem this implies that for almost every $x$ $$\lim_{n\to\infty}\frac1n\sum_{j=1}^n{\delta_j}(x)=\lim_{n\to\infty}\int \delta_n d\mu.$$
The limit on the left hand side can be explicitly calculated. Let $t_n(x)=T^nx\cdot \frac{q_{n-1}(x)}{q_{n}(x)}$, then Bosma, Jager and Wiedijk (\cite[Theorem 3]{BosmaJagerWiedijk1983}) found that for almost every $x$,
$\frac{1}{n}\sum_{j=1}^n\mathbf 1_{(0,z]}(t_j(x))$ converges to $F(z)$ at every $z\in[0,1]$, where
$$F(z)=\frac{1}{\log 2}\left(\log(1+z)-\frac{z}{1+z}\log z\right).$$
Since $\delta_n(x)=-\log(1+t_n(x))$ and since $\log(1+z)$ is a bounded and continuous function on $[0,1]$, one has that for almost every $x$
$$\lim_{n\to\infty}\frac1n\sum_{j=1}^n{\delta_j}(x)=\int_0^1-\log(1+z)dF(z)=-1-\frac12\log 2+\frac{\pi^2}{12\log2}.$$
So $\lim_{n\to\infty}\int \delta_n d\mu$ is equal to this number as well.

Let $$\theta_n(x)=q_n|xq_n-p_n|.$$ Since $\delta_n=\log\theta_{n-1}+\log \frac{q_n}{q_{n-1}}$, it follows that for almost every $x$, $$\lim_{n\to\infty}\frac1n\sum_{j=1}^n\log {\theta_j}(x)=-1-\frac12\log 2.$$
This gives an alternative proof of a result of Haas (\cite{Haas2005}).
\end{remark}




%\bibliography{bib0}
\bibliographystyle{alpha}




%\end{document}


\begin{thebibliography}{KNRS88}

\bibitem[AD01]{AaronsonDenker2001}
Jon Aaronson and Manfred Denker.
\newblock Local limit theorems for partial sums of stationary sequences
  generated by {G}ibbs-{M}arkov maps.
\newblock {\em Stoch. Dyn.}, 1(2):193--237, 2001.

\bibitem[ADU93]{AaronsonDenkerUrbanski1993}
Jon Aaronson, Manfred Denker, and Mariusz Urba{\'n}ski.
\newblock Ergodic theory for {M}arkov fibred systems and parabolic rational
  maps.
\newblock {\em Trans. Amer. Math. Soc.}, 337(2):495--548, 1993.

\bibitem[BFU02]{BedfordFisherUrbanski02}
T.~Bedford, A.~M. Fisher, and M.~Urba{\'n}ski.
\newblock The scenery flow for hyperbolic {J}ulia sets.
\newblock {\em Proc. London Math. Soc.}, 2(85):467--492, 2002.

\bibitem[BH11]{BergerHill2011}
Arno Berger and Theodore~P. Hill.
\newblock A basic theory of {B}enford's law.
\newblock {\em Probab. Surv.}, 8:1--126, 2011.

\bibitem[BJW83]{BosmaJagerWiedijk1983}
W.~Bosma, H.~Jager, and F.~Wiedijk.
\newblock Some metrical observations on the approximation by continued
  fractions.
\newblock {\em Nederl. Akad. Wetensch. Indag. Math.}, 45(3):281--299, 1983.

\bibitem[CMS18]{ChenavierMasseSchneider2018}
Nicolas Chenavier, Bruno Mass{\'e}, and Dominique Schneider.
\newblock Products of random variables and the first digit phenomenon.
\newblock {\em Stochastic Process. Appl.}, 128(5):1615--1634, 2018.

\bibitem[Fur61]{Furstenberg1961}
H.~Furstenberg.
\newblock Strict ergodicity and transformation of the torus.
\newblock {\em Amer. J. Math.}, 83:573--601, 1961.

\bibitem[Haa05]{Haas2005}
Andrew Haas.
\newblock An ergodic sum related to the approximation by continued fractions.
\newblock {\em New York J. Math.}, 11:345--349, 2005.

\bibitem[Hol70]{Holewijn1969/70}
P.~J. Holewijn.
\newblock On the uniform distribution of sequences of random variables.
\newblock {\em Z. Wahrscheinlichkeitstheorie und Verw. Gebiete}, 14:89--92,
  1969/70.

\bibitem[JL88]{JagerLiardet1988}
Hendrik Jager and Pierre Liardet.
\newblock Distributions arithm{\'e}tiques des d{\'e}nominateurs de convergents
  de fractions continues.
\newblock {\em Nederl. Akad. Wetensch. Indag. Math.}, 50(2):181--197, 1988.

\bibitem[Khi97]{Khinchin1997}
A.~Ya. Khinchin.
\newblock {\em Continued fractions}.
\newblock Dover Publications, Inc., Mineola, NY, translated from the third
  (1961) russian edition edition, 1997.
\newblock With a preface by B. V. Gnedenko, Reprint of the 1964 translation.

\bibitem[KN74]{KuipersNiederreiter1974}
L.~Kuipers and H.~Niederreiter.
\newblock {\em Uniform distribution of sequences}.
\newblock Wiley-Interscience [John Wiley \& Sons], New York-London-Sydney,
  1974.
\newblock Pure and Applied Mathematics.

\bibitem[KNRS88]{KanemitsuNagasakaRauzyEtAl1988}
Shigeru Kanemitsu, Kenji Nagasaka, G{\'e}rard Rauzy, and Jau-Shyong Shiue.
\newblock On {B}enford's law: the first digit problem.
\newblock In {\em Probability theory and mathematical statistics ({K}yoto,
  1986)}, volume 1299 of {\em Lecture Notes in Math.}, pages 158--169.
  Springer, Berlin, 1988.

\bibitem[Phi71]{Philipp1971}
Walter Philipp.
\newblock {\em Mixing sequences of random variables and probablistic number
  theory}.
\newblock American Mathematical Society, Providence, R. I., 1971.
\newblock Memoirs of the American Mathematical Society, No. 114.

\bibitem[PS69]{PhilippStackelberg1969}
Walter Philipp and Olaf~P. Stackelberg.
\newblock Zwei {G}renzwerts{\"a}tze f{\"u}r {K}ettenbr{\"u}che.
\newblock {\em Math. Ann.}, 181:152--156, 1969.

\bibitem[Rob53]{Robbins1953}
Herbert Robbins.
\newblock On the equidistribution of sums of independent random variables.
\newblock {\em Proc. Amer. Math. Soc.}, 4:786--799, 1953.

\bibitem[Sch90]{Schatte1990}
Peter Schatte.
\newblock On {B}enford's law for continued fractions.
\newblock {\em Math. Nachr.}, 148:137--144, 1990.

\bibitem[SN91]{SchatteNagasaka1991}
P.~Schatte and K.~Nagasaka.
\newblock A note on {B}enford's law for second order linear recurrences with
  periodical coefficients.
\newblock {\em Z. Anal. Anwendungen}, 10(2):251--254, 1991.

\end{thebibliography}


\end{document}



%\begin{thebibliography}{10}
%
%\bibitem{AaronsonDenker2001}
%Jon Aaronson and Manfred Denker.
%\newblock Local limit theorems for partial sums of stationary sequences
%  generated by {G}ibbs-{M}arkov maps.
%  \newblock {\em Stoch. Dyn.}, 1(2):193--237, 2001.
%
%  \bibitem{AaronsonDenkerUrbanski3MarkovFibered1993}
%Jon Aaronson, Manfred Denker, and Mariusz Urba{\'n}ski.
%\newblock Ergodic theory for {M}arkov fibred systems and parabolic rational maps.
%\newblock {\em Transactions of the American Mathematical Society},
%  337(2):495--548, 1993.
%
%\bibitem{BergerHill2011}
%Arno Berger and Theodore~P. Hill.
%\newblock A basic theory of {B}enford's law.
%\newblock {\em Probab. Surv.}, 8:1--126, 2011.
%
%\bibitem{BosmaJagerWiedijk1983}
%W.~Bosma, H.~Jager, and F.~Wiedijk.
%\newblock Some metrical observations on the approximation by continued
%  fractions.
%\newblock {\em Nederl. Akad. Wetensch. Indag. Math.}, 45(3):281--299, 1983.
%
%\bibitem{ChenavierMasseSchneider2018}
%Nicolas Chenavier, Bruno Mass{\'e}, and Dominique Schneider.
%\newblock Products of random variables and the first digit phenomenon.
%\newblock {\em Stochastic Process. Appl.}, 128(5):1615--1634, 2018.
%
%\bibitem{Furstenberg1961}
%H.~Furstenberg.
%\newblock Strict ergodicity and transformation of the torus.
%\newblock {\em Amer. J. Math.}, 83:573--601, 1961.
%
%\bibitem{Haas2005}
%Andrew Haas.
%\newblock An ergodic sum related to the approximation by continued fractions.
%\newblock {\em New York J. Math.}, 11:345--349, 2005.
%
%\bibitem{Holewijn1969/70}
%P.~J. Holewijn.
%\newblock On the uniform distribution of sequences of random variables.
%\newblock {\em Z. Wahrscheinlichkeitstheorie und Verw. Gebiete}, 14:89--92,
%  1969/70.
%
%\bibitem{JagerLiardet1988}
%Hendrik Jager and Pierre Liardet.
%\newblock Distributions arithm{\'e}tiques des d{\'e}nominateurs de convergents
%  de fractions continues.
%\newblock {\em Nederl. Akad. Wetensch. Indag. Math.}, 50(2):181--197, 1988.
%
%\bibitem{KanemitsuNagasakaRauzyEtAl1988}
%Shigeru Kanemitsu, Kenji Nagasaka, G{\'e}rard Rauzy, and Jau-Shyong Shiue.
%\newblock On {B}enford's law: the first digit problem.
%\newblock In {\em Probability theory and mathematical statistics ({K}yoto,
%  1986)}, volume 1299 of {\em Lecture Notes in Math.}, pages 158--169.
%  Springer, Berlin, 1988.
%
%\bibitem{Khinchin1997}
%A.~Ya. Khinchin.
%\newblock {\em Continued fractions}.
%\newblock Dover Publications, Inc., Mineola, NY, translated from the third
%  (1961) russian edition edition, 1997.
%\newblock With a preface by B. V. Gnedenko, Reprint of the 1964 translation.
%
%\bibitem{KuipersNiederreiter1974}
%L.~Kuipers and H.~Niederreiter.
%\newblock {\em Uniform distribution of sequences}.
%\newblock Wiley-Interscience [John Wiley \& Sons], New York-London-Sydney,
%  1974.
%\newblock Pure and Applied Mathematics.
%
%\bibitem{Philipp1971}
%Walter Philipp.
%\newblock {\em Mixing sequences of random variables and probablistic number
%  theory}.
%\newblock American Mathematical Society, Providence, R. I., 1971.
%\newblock Memoirs of the American Mathematical Society, No. 114.
%
%\bibitem{PhilippStackelberg1969}
%Walter Philipp and Olaf~P. Stackelberg.
%\newblock Zwei {G}renzwerts{\"a}tze f{\"u}r {K}ettenbr{\"u}che.
%\newblock {\em Math. Ann.}, 181:152--156, 1969.
%
%\bibitem{Robbins1953}
%Herbert Robbins.
%\newblock On the equidistribution of sums of independent random variables.
%\newblock {\em Proc. Amer. Math. Soc.}, 4:786--799, 1953.
%
%\bibitem{SchatteNagasaka1991}
%P.~Schatte and K.~Nagasaka.
%\newblock A note on {B}enford's law for second order linear recurrences with
%  periodical coefficients.
%\newblock {\em Z. Anal. Anwendungen}, 10(2):251--254, 1991.
%
%\bibitem{Schatte1990}
%Peter Schatte.
%\newblock On {B}enford's law for continued fractions.
%\newblock {\em Math. Nachr.}, 148:137--144, 1990.
%
%\end{thebibliography}
%
%\end{document}

