\documentclass[aps,prb,twocolumn,showpacs,floatfix,superscriptaddress,nofootinbib]{revtex4}
\usepackage{graphicx}
\usepackage{amssymb}
\usepackage{amsmath}
\usepackage{lmodern}
\usepackage{color}
\usepackage{hyperref}
\usepackage{empheq}
\usepackage{epstopdf}
\usepackage{amsmath}
%\usepackage{lipsum, babel}
\newcommand{\lap}[2]{\mathcal{L} \left\{ {#1} \right\} \left( {#2} \right) }
\DeclareMathOperator{\sech}{sech}

\begin{document}


\title{The shocks in Josephson transmission line revisited}

\author{Eugene Kogan}
\email{Eugene.Kogan@biu.ac.il}
\affiliation{Department of Physics, Bar-Ilan University, Ramat-Gan 52900, Israel}
\affiliation{Donostia International Physics Center (DIPC)\\
San Sebastian/Donostia, Spain}

\begin{abstract}
We continue our previous studies of the localized travelling waves,
more specifically, of the shocks and the kinks, propagating in the series-connected   Josephson transmission line (JTL). The paper
consists of two parts. In the first part we  calculate the scattering of the "sound' (small amplitude small wave vector harmonic wave)   on the shock wave.
In the second part we   study the similarities and the dissimilarities between  the shocks and the kinks in the lossy JTL. We also find
the particular cases, when the nonlinear equation, describing weak travelling  wave in the lossy JTL can be integrated in terms of elementary functions.
\end{abstract}

\date{\today}

\maketitle

\section{Introduction}

The interest in studies of  nonlinear electrical transmission lines, in particular of lossy nonlinear transmission lines, has started some time ago \cite{rosenau,chen,mohebbi}, but it became even more pronounced recently
\cite{ricketts,houwe,katayama,sekulic}. A very recent and complete review of studies of nonlinear electric transmission networks one can find in Ref. \cite{malomed2}.

We studied previously the shock waves in the lossy Josephson transmission line (JTL) JTL \cite{kogan1,kogan2} and kinks (and solitons) in the lossless (actually, without any shunting at al) JTL \cite{kogan2}.
The present work had several aims. First we would like to
analyse the interaction between the "sound" (small amplitude small wave vector harmonic wave) and the shock wave. Second  we would like to establish the relation between the shock waves and the kinks. And third, we would like to additionally study the weak  waves, and, in particular, to look for the cases when the nonlinear  equation, describing such waves in the JTL, can be integrated in terms of elementary functions.

 The rest of the article is constructed as follows.
In Section \ref{con} we rederive the circuit equations describing the JTL in the continuum approximation.
In  Section \ref{sou} we consider scattering of the "sound" wave
by the shock wave and calculate the appropriate reflection and transmission coefficients.
In Section \ref{unity} we show that the kinks, which we previously believed to exist only in the lossless JTL, exist also in the lossy JTL and show the connection between the shocks and the kinks.
We  also  integrate the wave equation  describing weak waves in the lossy JTL
in terms of elementary functions for the specific value of the losses parameter.
We conclude in Section \ref{concl}. In the Appendix \ref{real} we present a physically appealing model of the JTL, composed of superconducting grains. In the Appendix \ref{dif} we explain the condition for the applicability of the continuum approximation used in the paper. Some mathematical details are relegated to Appendix \ref{ana}.

\section{The circuit equations: continuum approximation}
\label{con}

Consider
the discrete model of the  Josephson transmission line (JTL), constructed from the identical Josephson junctions (JJs)  capacitors and resistors is shown on Fig. \ref{trans5}.  (Possible physical realization of the model  is presented in the Appendix \ref{real}.)
We take
as the dynamical variables  the phase differences (which we for brevity will call just phases) $\varphi_n$ across the  JJs
and the voltages $v_n$ of the ground capacitors.
The circuit equations are
\begin{subequations}
\begin{alignat}{4}
\frac{\hbar}{2e}\frac{d \varphi_n}{d t}&=v_{n-1}-v_{n} \label{a8a}\\
C\frac{dv_n}{dt} &=  I_c\sin\varphi_n- I_c\sin\varphi_{n+1}\nonumber\\
&+\left(\frac{1}{R_J}
+C_J\frac{d}{d t}\right)\frac{\hbar}{2e}\frac{d}{d t}\left(\varphi_{n}-\varphi_{n+1}\right),\label{a8b}
\end{alignat}
\end{subequations}
where    $C$ is the capacitance,  $I_c$ is the critical current of the JJ,
 and $C_J$ and $R_J$ are the capacitor and the ohmic resistor shunting the JJ.
% Figure environment removed

In the continuum approximation we  treat $n$  as the continuous variable $Z$ and approximate the finite differences in the r.h.s. of the equations by the first derivatives with respect to $Z$,   after which the equations take the form
\begin{subequations}
\label{ve9c}
\begin{alignat}{4}
\frac{\partial \varphi}{\partial\tau}&= -\frac{\partial V}{\partial Z}, \label{vb0}\\
\frac{\partial V}{\partial\tau} &=  -\frac{\partial }{\partial Z}\left(\sin\varphi
+\frac{Z_J}{R_J}\frac{\partial\varphi}{\partial\tau}
+\frac{C_J}{C}\frac{\partial^2\varphi}{\partial\tau^2}\right).\label{vb}
\end{alignat}
\end{subequations}
where we  introduced the dimensionless time $\tau=t/\sqrt{L_JC}$ and the dimensionless voltage $V=v/(Z_JI_c)$; $L_J\equiv\hbar/(2eI_c)$ is the "inductance" of the  JJ and  $Z_J\equiv\sqrt{L_J/C}$ is the "characteristic impedance" of the JTL.
The condition for the applicability of the continuum approximation is formulated explicitly  in the Appendix \ref{dif}.

\section{The sound scattering by the shock wave}
\label{sou}

\subsection{The sound waves and the shock waves}

Equation  (\ref{vb})  being nonlinear, the system (\ref{vb0}), (\ref{vb}) has a lot of different types of solutions.
In this Section we'll be interested in only two types of those. First type -
small amplitude small wave vector harmonic waves
on a homogeneous background $ \varphi_0$.
For such waves Eq. (\ref{vb}) is simplified to
\begin{eqnarray}
\label{ve9bb}
\frac{\partial V}{\partial\tau} =  -\cos\varphi_0\frac{\partial \varphi}{\partial Z}.
\end{eqnarray}
We ignored the shunting terms in r.h.s. of (\ref{vb0}) because they contain higher order derivatives in comparison with the main term, and small wave vector means also small frequency.

The harmonic wave solutions of Eqs. (\ref{vb0}), (\ref{vb}) (which, for brevity we'll call the sound)
are
\begin{subequations}
%\label{e9b}
\begin{alignat}{4}
\varphi&= \varphi_0+\varphi^{(h)} e^{ikz-i\omega \tau},\label{e9ba}\\
V&=V_0+ V^{(h)} e^{ikz-i\omega \tau},\label{e9bb}
\end{alignat}
\end{subequations}
where
\begin{eqnarray}
\label{uu}
\omega=\overline{u}\left(\varphi_0\right)k,\hskip 1cm \overline{u}^2\left(\varphi_0\right)=\cos\varphi_0,
\end{eqnarray}
is the normalized sound velocity. In this paper the normalized  velocity $\equiv$ physical velocity times $\sqrt{L_J C}/\Lambda$,  where $\Lambda$ is the JTL period. Note that the stability of  a homogeneous background $ \varphi_0$  demands
\begin{eqnarray}
\label{k}
\cos\varphi_0>0.
\end{eqnarray}

The second type of solutions we'll be (mostly) interested in, is shock waves \cite{kogan1,kogan2}. In this Section we'll ignore the structure of the shock wave
and consider it as the discontinuity of the dynamical variables.
The property of the shocks, which will be proven in the next Section,  connects the discontinuities of $\varphi$ and $V$  with the shock velocity:
\begin{subequations}
%\label{av78}
\begin{alignat}{4}
\overline{U}\left(\varphi_2-\varphi_1\right)&=V_1-V_2  ,\label{av78a}\\
\overline{U}\left(V_2-V_1\right)&= \sin\varphi_1-\sin\varphi_2,\label{av78b}
\end{alignat}
\end{subequations}
where $\varphi_1$ and $V_1$ are the phase and the voltage before the shock,  $\varphi_2$ and $V_2$ - after the shock, and
 $\overline{U}$ is the normalized shock wave velocity. Note also  the obvious result of (\ref{av78a}), (\ref{av78b}):
\begin{eqnarray}
\label{foa}
\overline{U}_{\varphi_2,\varphi_1}^2
=\frac{\sin\varphi_1-\sin\varphi_2}{\varphi_1-\varphi_2}.
\end{eqnarray}


\subsection{The reflection and the transmission coefficients}

In this Section we'll be interested in two problems \cite{landau}.
The first one: A  sound wave is incident  from the rear  on a shock wave.
Determine the sound reflection coefficient. The situation is shown in Fig. \ref{reflection}.
% Figure environment removed
The second problem: A  sound wave is incident  from the front on a shock wave. Determine the sound
transmission coefficient. The situation is shown in Fig. \ref{transmission}.
% Figure environment removed
While formulating both problems we took into account the equation, which will be derived in Section \ref{unity}
\begin{eqnarray}
\overline{u}_b^2<\overline{U}_{\varphi_a,\varphi_b}^2<\overline{u}_a^2.
\end{eqnarray}
where
$\varphi_b$ and $\varphi_a$  are the phases before and after the shock in the absence of the sound respectively. Also,
\begin{eqnarray}
\label{ine}
\overline{U}_{\varphi_2,\varphi_1}&=\overline{U}_{\varphi_a,\varphi_b}
+\delta\overline{U}^{(r,t)}.
\end{eqnarray}

For the first problem mentioned above we have
\begin{subequations}
%\label{in}
\begin{alignat}{4}
\varphi_1&=\varphi_b,\label{ina}\\
V_1&=V_b,\\
\varphi_2&=\varphi_a+\varphi^{(in)}+\varphi^{(r)},\\
V_2&=V_a+V^{(in)}+ V^{(r)},\label{ind}
\end{alignat}
\end{subequations}
where   (in) stands for the incident sound wave and (r) for the reflected sound wave.
Substituting (\ref{ine}) - (\ref{ind})  into (\ref{av78a}), (\ref{av78b})
 in the first order approximation we obtain
\begin{subequations}
%\label{in2}
\begin{alignat}{4}
&\delta\overline{U}\left(\varphi_a-\varphi_b\right)
+\overline{U}\left(\varphi^{(in)}+\varphi^{(r)}\right)
= V^{(in)}+V^{(r)}  ,\\
&\delta\overline{U}\left(V_a-V_b\right)+\overline{U}\left( V^{(in)}+ V^{(r)}\right)
= \overline{u}^2(\varphi_a)\left(\varphi^{(in)}+\varphi^{(r)}\right).
\end{alignat}
\end{subequations}
Taking into account the relations
\begin{subequations}
\begin{alignat}{4}
V^{(in)}&=\overline{u}\left(\varphi_a\right) \varphi^{(in)},\\
 V^{(r)}&=-\overline{u}\left(\varphi_a\right) \varphi^{(r)}
\end{alignat}
\end{subequations}
(the difference in the signs is because of the opposite directions of propagation of the two waves) and excluding $\delta\overline{U}$ we obtain
\begin{eqnarray}
\label{rr}
r\equiv \frac{ \varphi^{(r)}}{ \varphi^{(in)}}=-\frac{\left[\overline{u}\left(\varphi_a\right)
-\overline{U}\right]^2}
{\left[\overline{u}\left(\varphi_a\right)+\overline{U}\right]^2}
=-\frac{\overline{u}_{in}^2}{\overline{u}_{r}^2},
\end{eqnarray}
where $\overline{u}_{in}=\overline{u}\left(\varphi_a\right)-\overline{U}$ is the velocity of the incident sound wave relative to the shock wave, and $\overline{u}_{r}=\overline{u}\left(\varphi_a\right)+\overline{U}$ is the velocity of the reflected sound wave relative to the shock wave.
As one could have expected, the modulus of the sound reflection coefficient is less than one, and it goes to zero when the intensity of the shock wave decreases, that is when $\varphi_a\to \varphi_b$,
in other words, when the shock wave itself nearly becomes the sound wave.


Now let us turn to the second problem.
We have
\begin{subequations}
%\label{out}
\begin{alignat}{4}
\varphi_1&=\varphi_b+\varphi^{(in)},\label{outa}\\
V_1&=V_b+ V^{(in)},\\
\varphi_2&=\varphi_a+\varphi^{(t)},\\
V_2&=V_a+ V^{(t)},\label{outd}
\end{alignat}
\end{subequations}
where (t) stands for the transmitted wave.
Substituting (\ref{ine}), (\ref{outa}) - (\ref{outd}) into (\ref{av78a}), (\ref{av78b}),
in the first order approximation we obtain
\begin{subequations}
%\label{in2}
\begin{alignat}{4}
\delta\overline{U}\left(\varphi_a-\varphi_b\right)
+\overline{U}\left(\varphi^{(t)}-\varphi^{(in)}\right)
= V^{(t)}- V^{(in)} \\
\delta\overline{U}\left(V_a-V_b\right)+\overline{U}\left( V^{(t)}- V^{(in)}\right)
\nonumber\\
= \overline{u}^2(\varphi_a) \varphi^{(t)}- \overline{u}^2(\varphi_b) \varphi^{(in)}.
\end{alignat}
\end{subequations}
Taking into account the relations
\begin{subequations}
\begin{alignat}{4}
V^{(in)}&=-\overline{u}\left(\varphi_b\right) \varphi^{(in)},\\
V^{(t)}&=-\overline{u}\left(\varphi_a\right) \varphi^{(t)}
\end{alignat}
\end{subequations}
 and excluding $\delta\overline{U}$ we obtain
\begin{eqnarray}
\label{rr4}
t\equiv \frac{ \varphi^{(t)}}{ \varphi^{(in)}}
=\frac{\left[\overline{u}\left(\varphi_b\right)+\overline{U}\right]^2}
{\left[\overline{u}\left(\varphi_a\right)+\overline{U}\right]^2}
=\frac{\overline{u}_{in}^2}{\overline{u}_{t}^2},
\end{eqnarray}
where $\overline{u}_{in}=\overline{u}\left(\varphi_b\right)
+\overline{U}$ is the velocity of the incident sound wave relative to the shock wave, and $\overline{u}_{t}=\overline{u}\left(\varphi_a\right)+\overline{U}$ is the velocity of the transmitted sound wave relative to the shock wave.
As one could have expected, the sound transmission coefficient is less than one, and goes to one when the intensity of the shock wave decreases, that is when $\varphi_a\to \varphi_b$.

Looking back at the derivation of (\ref{rr}) and (\ref{rr4}) we understand that the equations will be valid also for a generalized Josephson law for the supercurrent $I_s$:
\begin{eqnarray}
I_s=I_cf(\varphi).
\end{eqnarray}
where $f$ is a (nearly) arbitrary function. The difference from the case considered above is that the sound velocity in the general  case is
\begin{eqnarray}
\overline{u}^2\left(\varphi_0\right)=f'(\varphi_0),
\end{eqnarray}
and the shock velocity is given by the equation
\begin{eqnarray}
\label{foa2}
\overline{U}_{\varphi_2,\varphi_1}^2
=\frac{f(\varphi_1)-f(\varphi_2)}{\varphi_1-\varphi_2}.
\end{eqnarray}
The validity of (\ref{foa2}) will become obvious after we present the proof
of its particular case (\ref{foa}) in the next Section.

\section{The  shocks and the kinks}
\label{unity}

\subsection{The  travelling waves}


In this Section we would like to study the structure of the shock wave, so we return to Eqs. (\ref{vb0}), (\ref{vb})  in their full glory.
For the travelling waves,
\begin{eqnarray}
\varphi(\tau,Z)=\varphi(\tau-Z/\overline{U}),\hskip .5cm
V(\tau,Z)=V(\tau-Z/\overline{U}),
\end{eqnarray}
where $\overline{U}$ is the travelling wave velocity.
Making the  ansatz  we obtain
\begin{subequations}
%\label{vc}
\begin{alignat}{4}
\overline{U}\frac{d \varphi}{d\tau}&= \frac{d V}{d \tau}, \label{vca}\\
\overline{U}\frac{d V}{d\tau} &=  -\frac{d }{d \tau}\left(\sin\varphi
+\frac{Z_J}{R_J}\frac{d\varphi}{d\tau}
+\frac{C_J}{C}\frac{d^2\varphi}{d\tau^2}\right).\label{vcb}
\end{alignat}
\end{subequations}

Consider a solution  which for $\tau\in (-\infty,+\infty)$ stays in the finite region of $(\varphi,V)$ phase space. The limit cycles are excluded for our problem,
and strange attractors are excluded  in a 2d phase space in general \cite{strogatz}. Hence the trajectory begins  in a fixed point and ends  in a fixed point
\begin{subequations}
%\label{bou}
\begin{alignat}{4}
\lim_{\tau\to -\infty}\varphi&=\varphi_1,  \hskip 1cm
\lim_{\tau\to +\infty}\varphi=\varphi_2 \label{12}\\
\lim_{\tau\to -\infty}V&=V_1,  \hskip 1cm
\lim_{\tau\to +\infty}V=V_2.\label{127}
\end{alignat}
\end{subequations}
Integrating  (\ref{vca}),  (\ref{vcb})  with respect to $\tau$ from $-\infty$ to $+\infty$ and taking into account the boundary conditions, we obtain Eqs. (\ref{av78a}), (\ref{av78b}), which are the basis of our consideration in the previous Section.
Note that the shunting of the JJ doesn't influence the shock velocity \cite{kogan1,kogan2}.

Excluding $V$ from (\ref{vca}),  (\ref{vcb}) and integrating the resulting equation
 we obtain closed equation for $\varphi$
\begin{eqnarray}
\label{system26}
\frac{d^2\varphi}{d \tilde{\tau}^2 }+\gamma\frac{d\varphi}{d\tilde{\tau}}
=\overline{U}^2\varphi-\sin\varphi+F,
\end{eqnarray}
where
$\tilde{\tau}\equiv \tau\sqrt{C/C_J}=t/\sqrt{L_JC_J}$,
$\gamma\equiv\sqrt{L_J/C_J}\big/R_J$ and $F$  is
the constant of integration.
Taking into account the boundary conditions (\ref{12}),
we can write down   (\ref{system26})  as
\begin{eqnarray}
\label{system3}
\frac{d^2\varphi}{d \tilde{\tau}^2 }
+\gamma\frac{d\varphi}{d\tilde{\tau}}+\sin\varphi
=\overline{I}(\varphi),
\end{eqnarray}
where
\begin{eqnarray}
\overline{I}(\varphi)
\equiv\frac{(\varphi-\varphi_2)\sin\varphi_1-(\varphi-\varphi_1)\sin\varphi_2}
{\varphi_1-\varphi_2},
\end{eqnarray}
which reminds the equation
\begin{eqnarray}
\frac{d^2\varphi}{d \tilde{\tau}^2 }+\gamma\frac{d\varphi}{d\tilde{\tau}}
+\sin\varphi=I/I_c,
\end{eqnarray}
describing current biased JJ within the RCSJ model \cite{tinkham}.

\subsection{Unity and struggle of opposites}

Alternatively  (\ref{system26}) can be written down as
\begin{eqnarray}
\label{system4}
\frac{d^2\varphi}{d \tilde{\tau}^2 }+\gamma\frac{d\varphi}{d\tilde{\tau}}
=-\frac{d\Pi(\varphi)}{d\varphi},
\end{eqnarray}
where
\begin{eqnarray}
\label{w}
\Pi(\varphi)=\frac{\left(\varphi-\varphi_1\right)^2\sin\varphi_2-
\left(\varphi-\varphi_2\right)^2\sin\varphi_1}{2(\varphi_1-\varphi_2)}
-\cos\varphi.
\end{eqnarray}
Equation (\ref{system4}) is Newton equation,  describing  motion with friction of the fictitious particle in the potential well $\Pi(\varphi)$. The motion starts at $\tilde{\tau}=-\infty$ at $\varphi=\varphi_1$ and ends at $\tilde{\tau}=+\infty$ at $\varphi=\varphi_2$.
Because of the   invariance of the system when all phases are shifted by $2\pi$,
it is enough to consider $\varphi_1\in (-\pi,\pi)$. Because of the condition (\ref{k}), we should consider only $\varphi_1\in (-\pi/2,\pi/2)$.
We consider in this paper only the case $\varphi_2\in (-\pi/2,\pi/2)$.
Because the sine function monotonically increases between $-\pi/2$ and $\pi/2$, the r.h.s. of (\ref{foa}) is always positive in this case.

The potential (\ref{w}) should have maximum at $\varphi_1$, while
the point $\varphi_2$ should be a stationary point of the potential with the property
\begin{eqnarray}
\Pi(\varphi_2)<\Pi(\varphi_1),
\end{eqnarray}
from which follows $-\varphi_1<\varphi_2<\varphi_1$.
The point $\varphi_2$ can be either  a minimum or a maximum.
The boundary between these two cases  (when $\varphi_2$ is an inflexion point) we can find by equating
the second derivative  of the potential at the point $\varphi_2$ to zero
\begin{eqnarray}
\label{in}
\frac{\sin\varphi_1-\sin\varphi_2}{\varphi_1-\varphi_2}-\cos\varphi_2=0.
\end{eqnarray}
The approximate solution of (\ref{in}) is $\varphi_2=-\varphi_1/2$.

What was said above can be reformulated in a slightly different way.
Because the physics is obviously symmetric with respect to simultaneous inversion of all phases $\varphi \to -\varphi$, in the following
 we  consider only
$\varphi_1\in (0,\pi/2)$.
If   $\varphi_2$  is  positive, it is inevitably the point of a minimum of the potential.
In fact, the stationary points of the potential are given by the equation
\begin{eqnarray}
\label{rhs}
\sin\varphi=\overline{U}^2\varphi+F.
\end{eqnarray}
Because $\sin\varphi$ is concave downward
for $0 < \varphi< \pi/2$, the straight line, crossing the sine curve at the points $\pi/2>\varphi_1,\varphi_2>0$ can't cross the curve in  between. Hence there are no stationary points between $\varphi_1$ and $\varphi_2$.
The potential  $\Pi(\varphi)$  for positive $\varphi_2$  is illustrated in Fig. \ref{we2}.
% Figure environment removed

On the other hand, for  $\varphi_2<0$  the potential  $\Pi(\varphi)$ can have either a minimum or a maximum at $\varphi_2$, as it  is
 illustrated in  Fig. \ref{we}.
% Figure environment removed

Looking at  Fig.  \ref{we} (left)  we realize that for  the solution with $\varphi_1$ and $\varphi_2$ having opposite signs to exist, the effective friction coefficient $\gamma$ should be large enough to prevent escape of the particle above the potential barrier to the left of $\varphi_2$.
(There is no such restriction for the shock wave with $\varphi_1$ and $\varphi_2$ having the same sign, because in this case the left potential barrier is higher than the right one, as it is illustrated in Fig. \ref{we2}.)

The minimum of the potential at $\varphi_2$ situation
corresponds to the shock wave  and was discussed at length in our previous publications \cite{kogan1,kogan2}.
Equation  (\ref{system4})  can  be easily integrated numerically. The result of such integration is presented in Fig. \ref{sho}.
% Figure environment removed

The maximum situation we considered previously  only for the  particular case of the JTL  in the absence of shunting \cite{kogan2}.
We called such travelling waves the kinks.
Now we understand that similar kinks exist also in the lossy JTL (for $-\varphi_1<\varphi_2<\varphi_1/2$).
Looking at  Fig.  \ref{we} (right), presenting the potential for the kink,  we realize, that  since the particle  stops at the unstable equilibrium point, for kink to exist,
 the fine tuning of $\gamma$ is necessary. Saying it in different words,
for a given $\varphi_1$ and given $\gamma$, only the kink with the definite value of $\varphi_2$ can exist.
In particular,    in the absence of losses ($\gamma=0$)  only the kinks with $\varphi_2=-\varphi_1$, are possible \cite{kogan2}.

Everywhere above we considered the travelling wave going to the right, but, of course, by interchanging $\varphi_1$ and $\varphi_2$ we obtain the wave going to the left. So the conditions for the shocks and for the kinks
in the whole phase plane of the boundary conditions $(\varphi_1,\varphi_2)$
are shown in Fig. \ref{col}. Two additional straight lines on this Figure $\varphi_2=-\varphi_1$ and $\varphi_2=\varphi_1$ present the kinks and the solitons respectively, which can exist in the bare-bones (unshunted) JTL \cite{kogan2} and propagate in both directions.
% Figure environment removed

\subsection{The shocks velocity vs. the kinks velocity}

Differentiating the r.h.s. of (\ref{system26}) with respect to $\varphi$ we obtain
\begin{eqnarray}
\label{long}
\frac{d^2\Pi(\varphi)}{d\varphi^2}=\cos\varphi-\overline{U}_{\varphi_2,\varphi_1}^2.
\end{eqnarray}

For the shock   $\varphi_1$ is  the point of a maximum of $\Pi(\varphi)$
and $\varphi_2$ is  the point of a minimum.
Hence
the second derivative of the potential with respect to $\varphi$ is  negative
at $\varphi_1$ and positive at $\varphi_2$.  Thus
\begin{eqnarray}
\label{30}
\overline{u}^2(\varphi_2)>\overline{U}^2_{\varphi_2,\varphi_1}>\overline{u}^2(\varphi_1).
\end{eqnarray}
The inequalities (\ref{30}) reflect the well-known in the nonlinear
waves theory fact: the shock velocity is smaller than the sound
velocity in the region behind the shock but larger than the sound
velocity in the region before the shock \cite{whitham}.

From the inequalities (\ref{30}) we can prove that a shock can not split into two shocks. Actually we can make even stronger statement: two shocks moving in the same direction will merge. In fact, let there is the first shock   $\varphi_2\leftarrow\varphi_3$ and the second shock $\varphi_3\leftarrow\varphi_1$ ahead of it. Because of inequalities (\ref{30}) the velocity of the first shock is larger,  and the velocity of the second shock is smaller that $\overline{u}(\varphi_3)$. The statement is proved.
Note that due to a one-dimensional nature of our problem we don't have to consider the corrugation instability of the shock wave
\cite{dyakov1,dyakov2,lubchich,semenko,robinet}.

For the kink both  $\varphi_1$
and $\varphi_2$ are  the points of minima.
Hence
the second derivative of the potential with respect to $\varphi$ is  positive at both points.  Thus
\begin{eqnarray}
\label{31}
\overline{U}^2_{\varphi_2,\varphi_1}>\overline{u}^2(\varphi_2)>\overline{u}^2(\varphi_1).
\end{eqnarray}
The kink is supersonic  from the point of view both of the region before and after it.

\subsection{Weak  waves}
\label{weak}

For
weak  wave, characterized by the condition
$\varphi_1-\varphi_2\ll 1$ (for the sake of definiteness we have chosen $\varphi_1>0$),  the r.h.s. of (\ref{system26})
can be approximated as
\begin{eqnarray}
\label{ystem44}
\text{r.h.s. of (\ref{system26})}
=\alpha(\varphi-\varphi_1)(\varphi-\varphi_2)
(\varphi-\varphi_3),
\end{eqnarray}
where
\begin{eqnarray}
\overline{\varphi}\equiv \frac{\varphi_1+\varphi_2}{2},\hskip .5cm
\varphi_3\equiv\overline{\varphi}-3\tan\overline{\varphi},\hskip .5cm \alpha\equiv \frac{\cos\overline{\varphi}}{6}.
\end{eqnarray}
Note that for the kink, $\varphi_3$ is between $\varphi_1$ and $\varphi_2$, for the shock, $\varphi_3$ is outside this interval.

As the result, (\ref{system26}) can be simplified to
\begin{eqnarray}
\label{system444}
\varphi_{\tilde{\tau}\tilde{\tau}}+\gamma\varphi_{\tilde{\tau}}
=\alpha(\varphi-\varphi_1)(\varphi-\varphi_2)(\varphi-\varphi_3).
\end{eqnarray}

Using the results of  Appendix \ref{ana} we state that
if the constants in (\ref{system444}) satisfy the condition
\begin{eqnarray}
\label{g}
\gamma=\sqrt{\frac{\alpha}{2}}(\varphi_1+\varphi_2-2\varphi_3)
=\sqrt{3\cos\overline{\varphi}}\tan\overline{\varphi},
\end{eqnarray}
the solution of (\ref{system444})  satisfying the boundary conditions
(\ref{12}) is
\begin{eqnarray}
\label{fi}
\varphi(\tilde{\tau})=\varphi_2+\frac{\varphi_1-\varphi_2}
{e^{\beta\tilde{\tau}}+1},
\end{eqnarray}
where
\begin{eqnarray}
\label{g60}
\beta=\sqrt{\frac{\alpha}{2}}(\varphi_1-\varphi_2)
=\sqrt{\frac{\cos\overline{\varphi}}{12}}(\varphi_1-\varphi_2).
\end{eqnarray}
A simple example: for $\varphi_2=-\varphi_1$,
Eqs. (\ref{g}), (\ref{fi}) and  (\ref{g60})  give $\gamma=0$ and
\begin{eqnarray}
\varphi(\tilde{\tau})=-\varphi_1
\tanh\left(\frac{\varphi_1\tilde{\tau}}{2\sqrt{3}}\right).
\end{eqnarray}


Let us return to Eq. (\ref{system444})  and strengthen the assumption which lead to the latter to $\varphi_1-\varphi_2\ll\varphi_1$.
In this case the equation can be approximated as
\begin{eqnarray}
\label{system444b}
\varphi_{\tilde{\tau}\tilde{\tau}}+\gamma\varphi_{\tilde{\tau}}
=\alpha'(\varphi-\varphi_1)(\varphi-\varphi_2),
\end{eqnarray}
where $\alpha'\equiv \sin\overline{\varphi}/2$.
Again using the results of  Appendix \ref{ana} we state that
if $\gamma$ satisfies the condition
\begin{eqnarray}
\label{g3}
\gamma=5\sqrt{\frac{\alpha'(\varphi_1-\varphi_2)}{6}}
=5\sqrt{\frac{\sin\overline{\varphi}(\varphi_1-\varphi_2)}{12}},
\end{eqnarray}
the solution of (\ref{system444b}) with the boundary conditions (\ref{12})  is
\begin{eqnarray}
\label{fi8}
\varphi(\tilde{\tau})=\varphi_2+\frac{\varphi_1-\varphi_2}
{\left(e^{\beta'\tilde{\tau}}+1\right)^2},
\end{eqnarray}
where
\begin{eqnarray}
\beta'=\sqrt{\frac{\alpha'(\varphi_1-\varphi_2)}{6}}
=\sqrt{\frac{\sin\overline{\varphi}(\varphi_1-\varphi_2) }{12}} .
\end{eqnarray}
Pay attention that though (\ref{system444b}) is an approximation to (\ref{system444}), it can be integrated in terms of elementary functions for totally different value of $\gamma$ (and hence the  solutions (\ref{fi8}) and (\ref{fi}) are totally different).

\section{Conclusions}
\label{concl}

The interaction of the sound waves with the shock waves is well studied in fluid mechanics. In Section \ref{sou} we considered similar problem for the JTL.
The formulas for the reflection coefficient in one case and the transmission coefficient in the other case (Eqs. (\ref{rr}) and (\ref{rr4})) turned out to be very simple and appealing.

We established the relation between the shocks existing in the lossy JTL and the kinks, which as we now understand, exist both in the lossy and in the lossless JTL.
However the solitons, we studied previously in the lossless JTL, are absent in the lossy JTL.

We found the particular cases when nonlinear equation describing weak travelling waves in the lossy JTL can be integrated in terms of elementary functions.


\begin{appendix}


\section{The JTL composed of superconducting grains}
\label{real}

A physically appealing model of the JTL composed of superconducting grains is presented in Fig. \ref{gr1}. (For simplicity in this Appendix we ignore the shunting capacitor.)
% Figure environment removed
Here,
we take
as the dynamical variables  the phases at the grains $\Phi_n$ and the potentials of the grains $V_n$. The circuit equations are
\begin{subequations}
%\label{ne}
\begin{alignat}{4}
\frac{\hbar}{2e}\frac{d \Phi_n}{d t}&=v_n \label{nea}\\
C\frac{dv_n}{dt} &=  I_c\sin\left(\Phi_{n-1}- \Phi_{n}\right)
-I_c\sin\left(\Phi_{n}- \Phi_{n+1}\right)\nonumber\\
&+\frac{1}{R_J}\left(v_{n-1}-2v_{n}+v_{n+1}\right).\label{neb}
\end{alignat}
\end{subequations}
We realise that Eqs. (\ref{a8a}), (\ref{a8b})  (in the absence of the shunting capacitor) follows from Eqs. (\ref{nea}),  (\ref{neb})
if we substitute $\varphi_n=\Phi_{n-1}-\Phi_n$.
Also, if we exclude $v_n$ from (\ref{nea}), (\ref{neb}) we obtain
\begin{eqnarray}
\label{mal}
\frac{d^2 \Phi_n}{d \tau^2}&=&  \sin\left(\Phi_{n-1}- \Phi_{n}\right)
-\sin\left(\Phi_{n}- \Phi_{n+1}\right)\nonumber\\
&+&\frac{Z_J}{R_J}\frac{d}{d\tau}\left(\Phi_{n-1}-2\Phi_{n}+\Phi_{n+1}\right),
\end{eqnarray}
which is the particular case of the Fermi-Pasta-Ulam-Tsingou  equation (with losses).

It is interesting to compare (\ref{mal}) with the equation from Ref. \cite{malomed}, describing the chain of interacting particles with friction
\begin{eqnarray}
m\frac{d^2y_n}{d\tau^2}&=&-\frac{\partial}{\partial y_n}\left[U\left(y_{n-1}-y_{n}\right)
+U\left(y_{n+1}-y_{n}\right)\right]\nonumber\\
&-&\alpha\frac{dy_n}{d\tau},
\end{eqnarray}
where $m$ is the mass and $y_n$ are  displacements of particles in the chain, $U(z)$ is the potential of the interparticle interaction, and $\alpha$ is the friction
coefficient.
Comparison shows the substantially different character of the losses in the systems.

It is also interesting to compare the  JTL with the one-dimensional Josephson-junction array.
The equation describing
the fluxon dynamics in the array is  the discretized version of the perturbed sine-Gordon
equation \cite{ustinov}
\begin{eqnarray}
\label{us}
\frac{d^2\varphi_n}{d\tau^2}+\alpha\frac{d\varphi_n}{d\tau}+\sin\varphi_n\nonumber\\
-\frac{1}{\alpha^2}\left(\varphi_{n-1}-2\varphi_n+\varphi_{n+1}\right)=I/I_c,
\end{eqnarray}
where $\alpha$ is the dissipation coefficient.
It is appropriate to compare (\ref{us}) with the equation obtained by excluding $v_{n}$ from (\ref{a8a}), (\ref{a8b})
\begin{eqnarray}
\label{nsg}
\frac{d^2 \varphi_n}{d \tau^2}=\sin\varphi_{n-1}+\sin\varphi_n-\sin\varphi_{n+1}
\nonumber\\
-\frac{Z_J}{R_J}\frac{d}{d \tau}\left(\varphi_{n-1}-2\varphi_{n}+\varphi_{n+1}\right).
\end{eqnarray}
Again, the comparison shows the substantially different character of the losses in the systems. But even in the absence of losses (\ref{nsg}) is different from the sine-Gordon equation.
Neither does (\ref{nsg}) in the continuum approximation
\begin{eqnarray}
\label{nsg5}
\frac{\partial^2 \varphi}{\partial \tau^2}=\frac{\partial^2 \sin\varphi}{\partial Z^2}
-\frac{Z_J}{R_J}\frac{\partial^3 \varphi}{\partial \tau\partial Z^2},
\end{eqnarray}
coincides with the sine-Gordon equation with losses \cite{landauer}.



\section{The continuum approximation}
\label{dif}

Natural question is how good is
the continuum approximation used everywhere in this paper?
To answer this question let us return to Eqs. (\ref{a8a}), (\ref{a8b})  and exclude $v_n$. We obtain
\begin{eqnarray}
\label{di}
\frac{d ^2\varphi_n}{d \tau^2}=
\sin\varphi_{n+1}-2\sin\varphi_n+\sin\varphi_{n+1}\nonumber\\
+\left(\frac{Z_J}{R_J}
+\frac{C_J}{C}\frac{d}{d\tau}\right)\frac{d}{d\tau}
\left(\varphi_{n+1}-2\varphi_{n}+\varphi_{n+1}\right).\label{b}
\end{eqnarray}
The continuum approximation (in the narrow sense) consists in promoting the discrete variable $n$ to the continuous variable $Z$ and approximating the discrete second order derivatives in the r.h.s. of (\ref{di}) by the continuous derivatives:
\begin{subequations}
\begin{alignat}{4}
\sin\varphi_{n+1}-2\sin\varphi_n+\sin\varphi_{n+1}
&=\frac{\partial^2 \sin\varphi}{\partial Z^2} \label{qq}\\
\varphi_{n+1}-2\varphi_n+\varphi_{n+1}
&=\frac{\partial^2\varphi}{\partial Z^2}.   \label{qq2}
\end{alignat}
\end{subequations}

To find the limits of the applicability of this approximation, let us
consider continuum approximation in the broad sense and
generalize, say,  (\ref{qq2}) to
\begin{eqnarray}
\label{cqq}
\varphi_{n+1}-2\varphi_n&+&\varphi_{n+1}
=\frac{\partial^2 \varphi}{\partial Z^2}+\frac{1}{12}\frac{\partial^4 \varphi}{\partial Z^4}+\frac{1}{360}\frac{\partial^6 \varphi}{\partial Z^6}+\dots\nonumber\\
\end{eqnarray}
We realize that   if shunting is strong, that is either
$C_J/C\gg 1$  or $Z_J/R_J\gg 1$ (the condition implied in this paper),
the continuum approximation (in the narrow sense)
can be justified
when $\Delta\varphi\ll 1$, where $\Delta\varphi\equiv |\varphi_1-\varphi_2|$.
In fact,
from  (\ref{system444}) follows that in this case
the time scale of the solution
is proportional to $1/\sqrt{\Delta\varphi}$, if $\gamma\ll 1$, and to $1/\Delta\varphi$, if $\gamma\gg 1$. So the forth order derivative term in (\ref{cqq}) has an additional $\Delta\varphi$  ($(\Delta\varphi)^2$)  factor with respect to the second order derivative terms, the sixth order derivative term  - an additional $(\Delta\varphi)^2$ ($(\Delta\varphi)^4$) factor with respect to the second order derivative terms and so on.

In our previous publication \cite{kogan2} we considered also the case of zero shunting. In this case, even if $\Delta\varphi\ll 1$,
the continuum approximated has to be upgraded to the quasi-continuum approximation
\begin{eqnarray}
\label{cqq2}
\sin\varphi_{n+1}-2\sin\varphi_n&+&\sin\varphi_{n+1}
=\frac{\partial^2\sin \varphi}{\partial Z^2}+\frac{1}{12}\frac{\sin\partial^4 \varphi}{\partial Z^4}.\nonumber\\
\end{eqnarray}
So, the equation describing the localized travelling wave for an arbitrary strength of the shunting (for $\Delta\varphi\ll 1$) is
\begin{eqnarray}
\label{system29}
\frac{1}{12\overline{U}^2}\frac{d^2\sin\varphi}{d \tau^2}
+\frac{C_J}{C}\frac{d^2\varphi}{d \tau^2 }+\frac{Z_J}{R_J} \frac{d\varphi}{d \tau }
=\overline{U}^2\varphi-\sin\varphi+F.\nonumber\\
\end{eqnarray}
Thus we were able to study the kinks (and the solitons) in the absence of shunting.

\section{Solutions of the ODE}
\label{ana}

We consider here the ODE for the function $p(\tau)$ in the form
\begin{eqnarray}
\label{44}
p_{\tau\tau}+\gamma p_{\tau}
=\alpha p\left(p^n-p_1\right)\left(p^n-p_3\right),
\end{eqnarray}
where $\gamma$, $\alpha$, $n$, $p_1$ and $p_3$ are some constants.
(For the sake of definiteness, we consider $p_1$ being positive.)

It would be good to get read of the first derivative with respect to $\tau$ in (\ref{44}). We  can achieve it by making the change of variable
\begin{eqnarray*}
p(\tau)=e^{-\gamma\tau/2}w(\tau)\hskip 2cm ?
\end{eqnarray*}
but let us make a more general transformation
\begin{eqnarray}
\label{5.10}
p(\tau)=e^{-m\tau}w(\tau),
\end{eqnarray}
where  the  parameter $m$ will be determined later.
This change of variables turns  Eq. (\ref{44d}) into
\begin{eqnarray}
\label{5.12}
w_{\tau\tau}
+(\gamma-2m)w_{\tau}=\alpha e^{-2nm\tau}w^{2n+1}\nonumber\\
-\alpha(p_1+p_3)e^{-nm\tau}w^{n+1}-\left(m^2-m\gamma-\alpha p_1p_3\right)w.
\end{eqnarray}
We see that the choice $m=\gamma/2$ in fact would cancel the first derivative in the equation, but the price is too high - the coefficients before the nonlinear terms in the equation would become explicitly $\tau$-dependent.

However we have another idea.  Let us kill the last term in (\ref{5.12}) by choosing
$m$ satisfying the equation
\begin{eqnarray}
\label{mumu}
m^2-m\gamma-\alpha p_1p_3=0.
\end{eqnarray}
Let us also, in addition to the change of dependent variable (\ref{5.10}) we did previously, make the change of the independent variable
\begin{eqnarray}
\label{5.13}
\tau\to z(\tau)=e^{-nm\tau},
\end{eqnarray}
 Eq. (\ref{5.12}) becomes
\begin{eqnarray}
\label{5.15}
n^2m^2w_{zz}+nm\left[(n+2)m-\gamma\right]\frac{w_z}{z}\nonumber\\
=\alpha  w^{2n+1}-\alpha (p_1+p_3) \frac{w^{n+1}}{z}.
\end{eqnarray}

\subsection{Integration by quadrature}

Consider the particular case $p_3=-p_1$, that is the case when (\ref{44}) has the form
\begin{eqnarray}
\label{44d}
p_{\tau\tau}+\gamma p_{\tau}=\alpha p\left(p^{2n}-p_1^2\right).
\end{eqnarray}
In this (\ref{5.15}) can be easily integrated in quadrature, provided $m$ satisfies the relation
\begin{eqnarray}
\label{5.17}
(n+2)m-\gamma=0.
\end{eqnarray}
The condition of compatibility of (\ref{mumu}) and (\ref{5.17}) is
\begin{eqnarray}
\label{cond}
\gamma=\sqrt{\frac{\alpha }{n+1}}(n+2)p_1,
\end{eqnarray}
and $m$ can be presented as
\begin{eqnarray}
\label{meme}
m=\sqrt{\frac{\alpha }{n+1}}p_1.
\end{eqnarray}

So finally,  Eq.(\ref{5.15}) becomes
\begin{eqnarray}
\label{5.19}
w_{zz}-\frac{n+1}{n^2p_1^2}w^{2n+1}=0.
\end{eqnarray}
Multiplying Eq.(\ref{5.19}) by $w_{z}$ and integrating with respect to $z$ we have
\begin{eqnarray}
\label{5.20}
w_{z}^2-\frac{1}{n^2p_1^2}w^{2n+2}=2E,
\end{eqnarray}
where $E$ is the integration constant. Equation (\ref{5.20}) can be easily integrated by quadrature.

Equation (\ref{5.19})  can be given obvious mechanical interpretation: it describes motion ($w$ being the coordinate and $z$ being the time) of the fictitious
particle  in the absence of friction
(this  gives us the opportunity to  integrate it by quadrature)
in the potential
\begin{eqnarray}
\label{pi}
\Pi(w)=-\frac{1}{2n^2p_1^2}w^{2n+2}.
\end{eqnarray}

\subsubsection{The elementary solution}

Let us demand that the solution of (\ref{44d}) satisfies the boundary conditions
\begin{eqnarray}
\label{condi}
\lim_{\tau\to-\infty}p(\tau)=p_1^{1/n},\hskip 1 cm \lim_{\tau\to+\infty}p(\tau)=0,
\end{eqnarray}
that is the solution of (\ref{5.20}) satisfies the boundary conditions
\begin{eqnarray}
\label{conditor}
\lim_{z\to 0}w(z)=p_1,\hskip 1 cm \lim_{z\to+\infty}w(z)=0,
\end{eqnarray}
 Hence
the  particle ends its motion in the infinite future at the potential $\Pi(w)$
maximum or inflexion point, depending whether $n$ is even or odd,
and we should substitute $E=0$ in (\ref{5.20}). This gives us the opportunity to integrate the equation not by quadrature, but in terms of elementary function:
\begin{eqnarray}
\label{5.22}
w(z)=\frac{p_1^{1/n}}{\left(1+z\right)^{1/n}}.
\end{eqnarray}
Substituting (\ref{5.22}) into  (\ref{5.10}), we obtain the solution of (\ref{44d}) with the initial conditions (\ref{condi})   in the form
\begin{eqnarray}
\label{fii}
p(\tau)=\frac{p_1^{1/n}.}
{\left[\exp\left(\sqrt{\frac{\alpha}{n+1}}np_1\tau\right)+1\right]^{1/n}}.
\end{eqnarray}

\subsection{The  expanded applicability of the elementary solution}
\label{ana3}

Now we have a pleasant surprise: the elementary solution (\ref{5.22}) solves
(\ref{5.15}) also when $p_3\neq -p_1$. In fact,
substituting (\ref{5.22}) into (\ref{5.15}) we obtain
\begin{eqnarray}
\label{44b}
\frac{(n+1)m^2}{(1+z)^2}- \frac{m\left[(n+2)m-\gamma\right] }
{z(1+z)}\nonumber\\
= \frac{\alpha p_1^2}{(1+z)^2}-  \frac{\alpha (p_1+p_3)p_1}{z(1+z)}.
\end{eqnarray}
The first term in the l.h.s. of the equation cancels the first term in the r.h.s., provided (\ref{meme}) is valid, and the equality between the remaining terms demands
\begin{eqnarray}
\label{h}
m\left[(n+2)m-\gamma\right]=\alpha (p_1+p_3)p_1,
\end{eqnarray}
which together with (\ref{meme}) gives
\begin{eqnarray}
\label{g2}
\gamma=\sqrt{\frac{\alpha}{n+1}}[p_1-(n+1)p_3].
\end{eqnarray}
So finally, if the constants of Eq. (\ref{44}) are connected by the relation (\ref{g2}), the solution of (\ref{44})
satisfying the boundary conditions (\ref{condi}) is
\begin{eqnarray}
\label{fii}
p(\tau)=\frac{p_1^{1/n}}
{\left[\exp\left(\sqrt{\frac{\alpha}{n+1}}np_1\tau\right)+1\right]^{1/n}}.
\end{eqnarray}

Mechanics analogy helps us to understand why $\gamma$, as given by Eq. (\ref{g2}) changes sign at $p_1=(n+1)p_3$. In fact, Eq. (\ref{44}) describes motion (with friction) of the fictitious particle in the potential
\begin{eqnarray}
\widetilde{\Pi}(p)=- \frac{\alpha p^{2n+2}}{2(n+1)}+ \frac{\alpha(p_1+p_3) p^{n+2}}{n+2}-\frac{\alpha p_1p_3p^2}{2}.\nonumber\\
\end{eqnarray}
 The motion starts at the equilibrium position $p=p_1^{1/n}$ and ends at the equilibrium position $p=0$. Simple algebra shows that $\widetilde{\Pi}(p_1^{1/n})>\widetilde{\Pi}(0)$ for
 $p_1>(n+1)p_3$. On the other hand, $\widetilde{\Pi}(p_1^{1/n})<\widetilde{\Pi}(0)$
 for  $p_1<(n+1)p_3$,  and the motion demands negative friction.

\end{appendix}

\begin{acknowledgments}

I am grateful to  A. Abanov, M. Goldstein and Jesus Cuevas Maraver for the discussion. I am also very grateful to Donostia International Physics Center (DIPC)
for the hospitality during my visit, when the paper was finalised.

\end{acknowledgments}
%\section{Conflict of Interest}

%The author declare no conflict of interest.

\begin{thebibliography}{99}

\bibitem{rosenau} P. Rosenau,  Phys. Lett. A {\bf 118},  222 (1986);  Phys. Scripta {\bf 34},  827 (1986).

\bibitem{chen} G. J. Chen and M. R. Beasley, IEEE
Trans. Appl. Supercond. {\bf 1},  140 (1991).

\bibitem{mohebbi} H. R. Mohebbi and A. H. Majedi, IEEE Trans. Appl.
Supercond. {\bf 19}, 891 (2009);  IEEE Transactions on Microwave Theory and Techniques {\bf 57},  1865 (2009).

\bibitem{ricketts} D. S. Ricketts and D. Ham, {\it Electrical Solitons: Theory, Design, and Applications}, CRC Press  (2018).


\bibitem{houwe} A. Houwe, S. Abbagari, M. Inc, G. Betchewe, S. Y. Doka, K. T. Crepin, and K. S. Nisar, Results in Physics {\bf 18}, 103188 (2020).

\bibitem{katayama} H. Katayama, N. Hatakenaka, and T. Fujii,
Phys. Rev. D {\bf 102}, 086018 (2020).

\bibitem{sekulic} D. L. Sekulic, N. M.  Samardzic,
Z. Mihajlovic,   and M. V. Sataric, Electronics
 {\bf 10}, 2278 (2021).

\bibitem{malomed2} E. Kengne, W. M. Liu, L. Q. English,and B. A. Malomed,
%Ginzburg–Landau models of nonlinear electric transmission networks.
 Phys. Rep.  {\bf 982}, 1 (2022).

\bibitem{kogan1} E. Kogan,
% Shock wave in series connected Josephson
%transmission line: Theoretical foundations
%and effects of resistive elements.
Journal of Applied Physics {\bf 130}, 013903 (2021). https://doi: 10.1063/5.0056886

\bibitem{kogan2}
E. Kogan,
% The Kinks, the Solitons and the Shocks in Series-
%Connected Discrete Josephson Transmission Lines.
Phys. Stat. Sol. (b) {\bf 259}, 2200160 (2022). https://doi: 10.1002/pssb.202200160



\bibitem{landau} L. D. Landau  and  E. M. Lifshitz, {\it Fluid Mechanics: Landau and Lifshitz: Course of Theoretical Physics, Volume 6 (Vol. 6)}, Elsevier (2013).

\bibitem{tinkham} M. Tinkham, {\it Introduction to superconductivity}, Courier Corporation (2004).

\bibitem{strogatz} S. H. Strogatz, {\it Nonlinear dynamics and chaos with student solutions manual: With applications to physics, biology, chemistry, and engineering}, CRC press, (2018).

\bibitem{whitham} G. B. Whitham, {\it Linear and Nonlinear Waves}, John Wiley \& Sons Inc.,
New York (1999).

\bibitem{dyakov1} S. P. D'yakov,
%The interaction of shock waves with small perturbations. i."
Sov. Phys. JETP {\bf 6}, 729 (1958).

\bibitem{dyakov2} S. P. D'yakov,
%The interaction of shock waves with small perturbations. i."
Sov. Phys. JETP {\bf 6}, 739 (1958).

\bibitem{lubchich} A. A. Lubchich and M. I. Pudovkin,
%"Interaction of small perturbations with shock waves."
Phys. Fluids {\bf 16} 4489 (2004).

\bibitem{semenko} E. V. Semenko,
%Linear problem of the shock wave disturbance in a non-classical case."
Phys.  Fluids {\bf 29},  066101 (2017).

\bibitem{robinet} J.-Ch. Robinet and G. Casalis,
 %"Critical interaction of a shock wave with an acoustic wave."
 Phys. Fluids {\bf 13}, 1047 (2001).

\bibitem{montroll} F. W. Montroll, in {\it Statistical Mechanics, edited by S. A.
Rice, K. F. Freed, and J. C. Light} (University of Chicago,
Chicago (1972).



%\bibitem{magy} E. Magyari,  Journal of Physics A, {\bf 15},  L139 (1982).


\bibitem{malomed} B. A. Malomed,
% "Propagating solitons in damped ac-driven chains."
Phys. Rev. A {\bf 45},  4097 (1992).

\bibitem{ustinov} A. V. Ustinov,  M. Cirillo, and B. A. Malomed,
% "Fluxon dynamics in one-dimensional Josephson-junction arrays."
Phys. Rev. B {\bf 47}, 8357 (1993).

\bibitem{landauer} R. Landauer,
%Phase transition waves: Solitons versus shock waves."
J. Appl. Phys. {\bf 51} 5594  (1980).


\end{thebibliography}


\end{document}

