% This is samplepaper.tex, a sample chapter demonstrating the
% LLNCS macro package for Springer Computer Science proceedings;
% Version 2.20 of 2017/10/04
%
\documentclass[runningheads]{llncs}
%
\usepackage{graphicx}
\usepackage[misc]{ifsym}
\usepackage{amsmath}
\usepackage{amssymb}
\usepackage{multirow}
\usepackage{bbding}
\usepackage{pifont}
\usepackage{wasysym}
\usepackage{utfsym}
\usepackage{fontawesome}
\usepackage{url}
\usepackage{subcaption}
\usepackage{mathtools}
\usepackage{color}
\usepackage{booktabs}
\usepackage{makecell,rotating}
%\usepackage{pifont}
% \usepackage{hyperref}
% \urlstyle{same}
\usepackage[misc]{ifsym}
\usepackage{threeparttable}
\newcommand{\xmli}[1]{{\color[rgb]{0.1,0.1,0.9}{[XM:#1]}}}
\newcommand{\hn}[1]{{\color{red}{#1}}}

\newcommand{\repeatthanks}{\textsuperscript{\thefootnote}}
\newcommand{\methodname}{DHC}

\def\etal{\textit{et al}. }
\def\ie{\textit{i.e.}}
\def\eg{\textit{e.g. }}
\def\vs{\textit{v.s. }}

\usepackage{hyperref}
\hypersetup{hidelinks,
	colorlinks=true,
	allcolors=black,
	pdfstartview=Fit,
	breaklinks=true}




% Used for displaying a sample figure. If possible, figure files should
% be included in EPS format.
%
% If you use the hyperref package, please uncomment the following line
% to display URLs in blue roman font according to Springer's eBook style:
\renewcommand\UrlFont{\color{blue}\rmfamily}

\begin{document}
%
% \title{Contribution Title\thanks{Supported by organization x.}}
% \title{Distilled Contrastive Learning for Surgical Video Pre-training}

% \title{Free Lunch for Self-supervised Learning in Surgical Video}
\title{DHC: Dual-debiased Heterogeneous Co-training Framework for Class-imbalanced Semi-supervised Medical Image Segmentation}


\titlerunning{Dual-debiased Heterogeneous Co-training Framework}
% If the paper title is too long for the running head, you can set
% an abbreviated paper title here
%

\authorrunning{H. Wang \& X. Li}
% % First names are abbreviated in the running head.
% % If there are more than two authors, 'et al.' is used.
% %


\author{Haonan Wang, Xiaomeng Li\textsuperscript{(\Letter)}} 
% index{Haonan, Wang}
% index{Xiaomeng, Li}

\institute{Department of Electronic and Computer Engineering, The Hong Kong University
of Science and Technology, Hong Kong, China
 \\ \email{eexmli@ust.hk}}

% % First names are abbreviated in the running head.
% % If there are more than two authors, 'et al.' is used.
% %

%
\maketitle              % typeset the header of the contribution
%

\begin{abstract}
The volume-wise labeling of 3D medical images is expertise-demanded and time-consuming; hence semi-supervised learning (SSL) is highly desirable for training with limited labeled data. 
\textit{Imbalanced class distribution} is a severe problem that bottlenecks the real-world application of these methods but was not addressed much.
Aiming to solve this issue, we present a novel \textbf{D}ual-debiased \textbf{H}eterogeneous \textbf{C}o-training (\textbf{DHC}) framework for semi-supervised 3D medical image segmentation. 
Specifically, we propose two loss weighting strategies, namely Distribution-aware Debiased Weighting (DistDW) and Difficulty-aware Debiased Weighting (DiffDW), which leverage the pseudo labels dynamically to guide the model to solve data and learning biases.
The framework improves significantly by co-training these two \textit{diverse and accurate} sub-models.
We also introduce more representative benchmarks for class-imbalanced semi-supervised medical image segmentation, which can fully demonstrate the efficacy of the class-imbalance designs.
Experiments show that our proposed framework brings significant improvements by using pseudo labels for debiasing and alleviating the class imbalance problem. 
More importantly, our method outperforms the state-of-the-art SSL methods, demonstrating the potential of our framework for the more challenging SSL setting. Code and models are available at: \href{https://github.com/xmed-lab/DHC}{https://github.com/xmed-lab/DHC}
\keywords{Semi-supervised learning \and Class imbalance \and 3D medical image segmentation \and CT image }
\end{abstract}


\section{Introduction} \label{intro}
The shortage of labeled data is a significant challenge in medical image segmentation, as acquiring large amounts of labeled data is expensive and requires specialized knowledge. This shortage limits the performance of existing segmentation models. 
To address this issue, researchers have proposed various semi-supervised learning (SSL) techniques that incorporate both labeled and unlabeled data to train models for both natural~\cite{tarvainen2017meanteacher,sohn2020fixmatch,chen2021cps,wang2022u2pl,wang2022depl,chen2022dst} and medical images~\cite{yu2019uamt,luo2021urpc,wang2022gbdl,wu2022cdcl,lin2022cld,you2022cvrl,wu2022ssnet}. 
However, most of these methods do not consider the class imbalance issue, which is common in medical image datasets. For example, multi-organ segmentation from CT scans requires to segment esophagus, right adrenal gland, left adrenal gland, \textit{etc.}, where the class ratio is quite imbalanced; see Fig~\ref{analysis}(a). As for liver tumor segmentation from CT scans, usually the ratio for liver and tumor is larger than 16:1.

\if 1
Recently, more attention has been given to addressing the class imbalance problem in semi-supervised classification tasks through techniques such as leveraging unlabeled data~\cite{wei2021crest,fan2022cossl,simis}, re-balancing data distributions in loss~\cite{hong2021lade}, and re-sampling learning~\cite{yu2022instancediff}. However, directly adopting these methods to medical image segmentation leads to unsatisfactory results (Table~\ref{sota}, especially those colored with \textcolor{red}{red}), which are mainly due to the more imbalanced classes and complex features in medical images. 
%Thus, these methods often result in inferior performance in the medical image segmentation domain (see results in Table.~\ref{sota}). 
Thus, designing a robust class-imbalanced semi-supervised segmentation framework tailored for medical images is necessary.
\fi 

%The limited data is an intrinsic bottleneck of medical image segmentation since accessing large volumes of labeled data is difficult owing to the high cost and required domain-specific expertise.
%On the other hand, using a lesser amount of training data significantly affects the model’s performance.
%To solve this issue, many methods on natural images~\cite{tarvainen2017meanteacher,sohn2020fixmatch,chen2021cps,wang2022u2pl} and medical images~\cite{yu2019uamt,luo2021urpc,wang2022gbdl,wu2022cdcl,lin2022cld,you2022cvrl,wu2022ssnet} were proposed targeting to Semi-Supervised Learning (SSL) which exploits unlabeled data information to compensate for the substantial requirement of data annotation.
%Most methods did not consider the class imbalanced issue. 
%However, the in-the-wild data distribution is quite imbalanced, making these methods fail to capture representative features of the minority classes.
%Recently, more and more researchers focused on solving the class-imbalance issue of semi-supervised classification tasks by different means, e.g., leveraging unlabeled data~\cite{wei2021crest,fan2022cossl,simis}, re-balancing data distributions in loss~\cite{hong2021lade}, debiased learning~\cite{chen2022dst,wang2022depl} etc.
%Unlike natural image domain, medical image segmentation domain naturally have more imbalanced classes, more complicated features, etc. Existing imbalanced SSL methods fail to handle these issues, resulting in inferior performances (see Table~\ref{sota}). 


Recently, some researchers proposed class-imbalanced semi-supervised methods~\cite{basak2022addressing,lin2022cld} and demonstrated substantial advances in medical image segmentation tasks. 
%These methods mainly focus on addressing data bias and learning bias, \ie, to prevent the learning of the majority classes and well-performed classes from overwhelming the training process, respectively.
Concretely, Basak \etal~\cite{basak2022addressing} introduced a robust class-wise sampling strategy to address the \textit{learning bias} by maintaining performance indicators on the fly and using fuzzy fusion to dynamically obtain the class-wise sampling rates. 
However, the proposed indicators can not model the difficulty well, and the benefits may be overestimated due to the non-representative datasets used (Fig.~\ref{analysis}(a)).
Lin \textit{et al.}~\cite{lin2022cld} proposed CLD to address the \textit{data bias} by weighting the overall loss function based on the voxel number of each class. However, this method fails due to the easily over-fitted CPS (Cross Pseudo Supervision)~\cite{chen2021cps} baseline, ignoring unlabeled data in weight estimation and the fixed class-aware weights. 
%during the training stage.
%Despite their success, these methods still have limitations: 1) the data bias and learning bias are not sufficiently addressed due to the limited baselines, the neglecting of the unlabeled data, the fixed class-aware weights and \textit{etc}.
%2) the benefits of the re-balance design may be overestimated due to the non-representativeness of the datasets used (as shown in Fig.~\ref{analysis}(a)). 



%However, they mainly conducted experiments on MMWHS, ACDC datasets which are not representative, i.e., with limited classes and relatively small imbalance degrees (as can be seen in Fig.~\ref{analysis}). 
%These characteristics make the tasks not tricky enough to measure the actual merits brought by the class re-balance strategy design, so when the dataset is truly imbalanced, these methods fail (see Table~\ref{sota}). 

% Figure environment removed


% Figure environment removed

In this work, we explore the importance of heterogeneity in solving the over-fitting problem of CPS (Fig.~\ref{heterogeneous}) and propose a novel \textbf{DHC} (\textbf{D}ual-debiased \textbf{H}eterogeneous \textbf{C}o-training) framework with two distinct dynamic weighting strategies leveraging both labeled and unlabeled data, to tackle the class imbalance issues and drawbacks of the CPS baseline model.
The key idea of heterogeneous co-training is that individual learners in an ensemble model should be both \textit{accurate and diverse}, as stated in the error-ambiguity decomposition~\cite{krogh1994neural}. 
To achieve this, we propose \textbf{DistDW} (\textbf{Dist}ribution-aware \textbf{D}ebiased \textbf{W}eighting) and \textbf{DiffDW} (\textbf{Diff}iculty-aware \textbf{D}ebiased \textbf{W}eighting) strategies
to guide the two sub-models to tackle different biases, leading to heterogeneous learning directions.
%In this work, we focus on solving two biases which are insufficiently studied in the previous works~\cite{basak2022addressing,lin2022cld}: 1) data bias, i.e., majority classes are easier to overwhelm model training whereas the minority classes are ignored;
%2) learning bias, i.e., easier classes with regular shapes, clear boundaries, large number of voxels, etc., are easier to overwhelm model training compared with harder classes with opposite characters.
%We further present a novel SSL framework named DHC (Dual-debiased Heterogeneous Co-training Framework) by alleviating the above two biases and leveraging significant performance improvements of CPS when co-trained with heterogeneous sub-models with two novel supplementary weighting strategies.
%Most existing methods leveraged the class numbers to re-balance the data distribution. 
%We propose two adaptive weighting strategies to solve the above biases: the distribution-aware debiased weighting strategy and the difficulty-aware debiased weighting strategy. 
Specifically, DistDW solves the \textit{data bias} by calculating the imbalance ratio with the unlabeled data and forcing the model to focus on extreme minority classes through careful function design.
%the log function to keep the weight of the most majority class as 0 which keeps the merit of \cite{simis} that eliminate the redundancy of the most majority class. 
Then, after observing the inconsistency between the imbalance degrees and the performances (see Fig.~\ref{analysis}(b)), DiffDW is designed to solve the \textit{learning bias}. We use the labeled samples and the corresponding labels to measure the learning difficulty from learning speed and Dice value aspects and slow down the speeds of the easier classes by setting smaller weights. 
%When the model is almost converged, the learning speeds are similar across all the classes. Then we use the current reversed Dice values to force the model to pay more attention to the classes with smaller Dice values.
%The weighting schemes are all calculated on-the-fly and update the loss weights in each iteration.
DistDW and DiffDW are diverse and have complementary properties (Fig.~\ref{analysis}(c)), which satisfies the design ethos of a heterogeneous framework.
%%Many previous work employed consistency regularization under perturbation of different initialization, augmentations and features. 
%% We employ the powerful framework CPS~\cite{} as the baseline as it achieves the state-of-the-art performance on the SSL segmentation task for natural images.

The key contributions of our work can be summarized as follows: 1) we first state the homogeneity issue of CPS and improve it with a novel dual-debiased heterogeneous co-training framework targeting the class imbalance issue; 2) we propose two novel weighting strategies, DistDW and DiffDW, which effectively solve two critical issues of SSL: data and learning biases; 3) we introduce two public datasets, Synapse~\cite{synapse} and AMOS~\cite{amos}, as new benchmarks for class-imbalanced semi-supervised medical image segmentation.
These datasets include sufficient classes and significant imbalance ratios ($>500:1$), making them ideal for evaluating the effectiveness of class-imbalance-targeted algorithm designs.
%since it has sufficient classes and a large enough imbalance ratio to estimate the capability of the algorithm design targeting to class imbalance issue. Extensive experiments and ablation studies are conducted to validate the effectiveness of the proposed methods.


\iffalse
\section{Related Work}
\subsubsection{Semi-supervised Segmentation}
Semi-supervised segmentation aims to explore tremendous unlabeled data with supervision from limited labeled data.
%, which is most relevant to domain adaptive segmentation where labeled data is obtained from another domain.
%\textbf{General Semi-supervised Segmentation.}
Recently, self-training-based methods~\cite{chen2021cps,wang2022u2pl,fan2022ucc} have become the mainstream of this domain. Approaches with consistency regularization strategies~\cite{ouali2020cct,chen2021cps,fan2022ucc} achieved good performance by encouraging high similarity between the predictions of two perturbed networks for the same input image, which highly improved the generalization ability.
%% Wang et al. argued that every pixel matters to the model training, even its prediction is ambiguous. Based on this insight, they developed **U2PL** which adaptively separated reliable and unreliable pixels via the entropy of predictions, push each unreliable pixel to a category-wise queue that consists of negative samples, and manage to train the model with all candidate pixels.
In the medical image domain, the data limitation issue is more natural and serious.
%% Existing approaches to combat the limited data include uncertainty suppression and consistency loss~\cite{luo2021urpc}, contrast learning sampling strategy by utilizing the most valuable knowledge from unlabeled data \cite{wu2022cdcl}, rethinking bayesian deep learning methods \cite{wang2022gbdl}, exploring the pixel-level smoothness and inter-class separation \cite{wu2022ssnet}, and improved Global Local CL~\cite{chaitanya2020glcl} by capturing 3D spatial context and rich anatomical information along both the feature and the batch dimensions~\cite{you2022cvrl}。
Existing approaches~\cite{luo2021urpc,wu2022cdcl,wang2022gbdl,wu2022ssnet,you2022cvrl} to combat the limited data have achieved great success but are bottlenecked by the class imbalance problem, which is quite common and challenging in real-world application scenarios. Recently, several approaches~\cite{basak2022addressing,lin2022cld} have been proposed to tackle this issue, but they still have major limitations, as stated in Section~\ref{intro}.
Class-imbalanced semi-supervised segmentation is still an under-studied area.
%Basak \textit{et al.}~\cite{basak2022addressing} introduced a robust class-wise sampling strategy to address the imbalance problem. They maintained a confidence array to record class-wise performance during training based on entropy, variance and confidence.
%Lin \textit{et al.}~\cite{lin2022cld} considered the imbalance issue and calibrated the label distribution by weighting the overall loss function with the voxel numbers of all classes. They also leveraged the uncertainty-aware sampling supervision and probability-aware random cropping augmentation to further enhance the learning of minority classes.


\subsubsection{Class-imbalanced Semi-supervised Learning}
Recently, many state-of-the-art methods leveraged the unlabeled data to alleviate the class imbalance issues~\cite{wei2021crest,simis,oh2022daso,lai2022sar}.
Specifically, CReST~\cite{wei2021crest} dynamically added some unlabeled pseudo labels to the corresponding classes of the labeled set, depending on the imbalance ratios. 
However, it still added some samples to the majority classes which makes it hard to balance.
SimiS~\cite{simis} solve this problem by supplementing the tail classes according to their difference in class distribution from the head classes, which makes it focuses more on the tail classes.
SimiS fails when the one of the majority classes has very large number of samples and the other classes will have no significant margins even though they are still quite imbalanced. We proposed a new strategy to solve this issue.
%% Semantic pseudo-labels are reversely biased towards the tail side compared with the linear labels which biased to the head side. DASO~\cite{oh2022daso} leverages this complementary property of the similarity-based classifier to complement the vanilla linear classifier. With the mix-up of these two kinds of labels, the class-imbalance problem is alleviated and the overall bias is reduced.
Moreover, Yu \textit{et al.}~\cite{yu2022instancediff} proposed a instance-difficulty-aware re-sampling method by estimating the prediction variations in the learning and the unlearning directions to compensate the existing failures on the difficult majority classes.

\fi



\section{Methods}


Fig.~\ref{framework} shows the overall framework of the proposed DHC framework. DHC leverages the benefits of combining two \textit{diverse and accurate} sub-models with two distinct learning objectives: alleviating data bias and learning bias.
%\xmli{not very clear and precise.}
To achieve this, we propose two dynamic loss weighting strategies, DistDW (Distribution-aware Debiased Weighting) and DiffDW (Difficulty-aware Debiased Weighting), to guide the training of the two sub-models.
DistDW and DiffDW demonstrate complementary properties.
Thus, by incorporating multiple perspectives and sources of information with DistDW and DiffDW, the overall framework reduces over-fitting and enhances the generalization capability.



% Figure environment removed



\subsection{Heterogeneous Co-training Framework with Consistency Supervision}

%After implementing the two proposed strategies to the baseline framework, we observed that the difficulty-aware knowledge and the data-distribution-aware knowledge have complementary properties. As shown in Fig.~\ref{analysis}(b), the distribution-aware model performs better on the least minority classes. In contrast, the difficulty-aware model is better in some difficult classes, such as stomach, but lower in the least minority classes.
%Our model complements the distribution bias with the learning bias through the cross-pseudo consistency regularization.


%To fully leverage the complementary properties of the distribution-aware debiased model and the difficulty-aware debiased model, we employ the CPS as the semi-supervised segmentation baseline. The cross pseudo consistency supervision





%Instead of using different initializations, we consider the two weighting strategies as the perturbation.
Assume that the whole dataset consists of $N_L$ labeled samples $\{(x_i^l,y_i)\}_{i=1}^{N_L}$ and $N_U$ unlabeled samples $\{x_i^u\}_{i=1}^{N_U}$, where $x_i \in \mathbb{R}^{D\times H\times W}$ is the input volume and $y_i \in \mathbb{R}^{K\times D\times H\times W}$ is the ground-truth annotation with $K$ classes (including background). 
%Denote the output probability map of the segmentation model as $p_i^\theta$ and the pseudo label as $\hat{y}_i^\theta = argmax(p_i^\theta)$, where $\theta=A,B$.
%\xmli{may delete this sentence. Explain the meaning of $p_i^\A$}
The two sub-models of DHC complement each other by minimizing the following objective functions with two \textit{diverse and accurate} weighting strategies:
\begin{equation}
    \overline{\mathcal{L}_s}=\frac{1}{N_L}\frac{1}{K}\sum_{i=0}^{N_L} [ {\color[RGB]{33,101,168} W^{diff}_i}\mathcal{L}_{s}(p^A_i, y_i)+ { \color[RGB]{153,97,132}W^{dist}_i}\mathcal{L}_{s}(p^B_i, {y}_i)]
 \end{equation}
\begin{equation}
    \overline{\mathcal{L}_u}=\frac{1}{N_L+N_U}\frac{1}{K}\sum_{i=0}^{N_L+N_U} [{\color[RGB]{33,101,168}W^{diff}_i}\mathcal{L}_{u}(p^A_i, \hat{y}_i^B)+ { \color[RGB]{153,97,132}W^{dist}_i}\mathcal{L}_{u}(p^B_i, \hat{y}_i^A)]
\end{equation}
where $p_i^{(\cdot)}$ is the output probability map and $\hat{y}_i^{(\cdot)} = \mathbf{argmax}\{p_{i,k}^{(\cdot)}\}_{k=0}^K$ is the pseudo label of $i^{th}$ sample.
$\mathcal{L}_s(x,y)=\mathcal{L}_{CE}(x,y)$ is the supervised cross entropy loss function to supervise the output of labeled data, and $\mathcal{L}_u(x,y) = \frac{1}{2}[\mathcal{L}_{CE}(x,y)+\mathcal{L}_{Dice}(x,y)]$ is the unsupervised loss function to measure the prediction consistency of two models by taking the same input volume $x_i$. 
Note that both labeled and unlabeled data are used to compute the unsupervised loss. 
Finally, we can obtain the total loss: $\mathcal{L}_{total}=\overline{\mathcal{L}_s}+\lambda \overline{\mathcal{L}_u}$, we empirically set $\lambda$ as 0.1 and follow~\cite{lin2022cld} to use the epoch-dependent Gaussian ramp-up strategy to gradually enlarge the ratio of unsupervised loss.
${\color[RGB]{33,101,168}W^{diff}_i}$ and ${ \color[RGB]{153,97,132}W^{dist}_i}$ are the dynamic class-wise loss weights obtained by the proposed weighting strategies, which will be introduced next.


% Figure environment removed

\subsection{Distribution-aware Debiased Weighting (DistDW)}
%Aiming to solve the over-supplementary problem of the majority class in CReST, SimiS supplemented the tail classes of the labeled set according to their difference in class distribution from the head classes, which makes it focuses more on the tail classes. 
%However, when the most majority classes have huge number of samples (\eg, more than the summation of all the other classes, Fig~\ref{analysis}), using the difference between class distribution from the most frequent class will make other classes have no remarkable differences on the class marginal distribution. In this case, all the other classes will degrade to minority classes.
%We proposed a new way to eliminate the data distribution bias, which inherits the advantage of SimiS that eliminate the weight of the most majority class while keep the weights of minority classes apart from being similar.

To mitigate the data distribution bias, we propose a simple yet efficient re-weighing strategy, DistDW. DistDW combines the benefits of the SimiS~\cite{simis}, which eliminate the weight of the largest majority class, while preserving the distinctive weights of the minority classes (Fig.~\ref{curves}(c)). The proposed strategy re-balances the learning process by forcing the model to focus more on the minority classes.
Specifically, we utilize the class-wise distribution of the unlabeled pseudo labels $p^u$ by counting the number of voxels for each category, denoted as $N_k , k=0, . . . , K$. We construct the weighting coefficient for $k^{th}$ category as follows:
\begin{equation}
    w_k= \frac{log(P_k)} {\mathbf{max}\{log(P_i)\}^K_{i=0}}, \quad  P_k = \frac {\mathbf{max}\{N_i\}^K_{i=0}} {N_k}, \quad k=0,1,...,K
\end{equation}
\begin{equation}
    W_t^{dist} \leftarrow \beta W^{dist}_{t-1} + (1-\beta) W^{dist}_t,\quad W_t^{dist} = [w_1, w_2, ..., w_K]
\end{equation}
where $\beta$ is the momentum parameter, set to 0.99 experimentally. 
%The effectiveness of DistDW will be further analyzed with weights curves in Section~\ref{ablation}.




\subsection{Difficulty-aware Debiased Weighting (DiffDW)}

After analyzing the proposed DistDW, we found that some classes with many samples present significant learning difficulties. For instance, despite having the second highest number of voxels, the stomach class has a much lower Dice score than the aorta class, which has only 20\% of the voxels of the stomach (Fig.~\ref{analysis}(b)). Blindly forcing the model to prioritize minority classes may further exacerbate the learning bias, as some challenging classes may not be learned to an adequate extent.
%we found some classes with quite many samples are even very hard to learn (as can be seen in Fig.~\ref{analysis}(b), the stomach has $2^{nd}$ most voxels, but the Dice score is much lower than the aorta class with only 20\% voxels compared with the stomach). Thus, blindly forcing the model to pay more attention to the minority classes will enlarge the learning bias, some hard classes can not be learned sufficiently.
To alleviate this problem, we design DiffDW to force the model to focus on the most difficult classes (\textit{i.e.} the classes learned slower and with worse performances) rather than the minority classes.
The difficulty is modeled in two ways: learning speed and performance.
%and $Dice(p^l, y^l) = [\lambda_0,\lambda_1,...,\lambda_K]$ denote the Dice scores of all classes. 
We use Population Stability Index~\cite{jeffreys1946psi} to measure the learning speed of each class after the $t^{th}$ iteration:
\begin{equation}
    du_{k,t}=\sum^t_{t-\tau} \mathbb{I}(\triangle \leq 0) \mathrm{ln}(\frac {\lambda_{k,t}} {\lambda_{k,t-1}}), \quad 
    dl_{k,t}=\sum^t_{t-\tau} \mathbb{I}(\triangle > 0) \mathrm{ln}(\frac {\lambda_{k,t}} {\lambda_{k,t-1}})
\end{equation}
where $\lambda_k$ denotes the Dice score of $k^{th}$ class in $t^{th}$ iteration and $\triangle=\lambda_{k,t}-\lambda_{k,t-1}$. $du_{k,t}$ and $dl_{k,t}$ denote classes not learned and learned after the $t^{th}$ iteration. $\mathbb{I(\cdot)}$ is the indicator function. $\tau$ is the number accumulation iterations and set to 50 empirically.
Then, we define the difficulty of $k^{th}$ class after $t^{th}$ iteration as $d_{k,t}= \frac {{du_{k,t}}+\epsilon} {{dl_{k,t}}+\epsilon}$, where $\epsilon$ is a smoothing item with minimal value. The classes learned faster have smaller $d_{k,t}$, the corresponding weights in the loss function will be smaller to slow down the learn speed.
%Difficulty after $t^{th}$ iteration is defined as: $D_t(p^l, y^l)=[d_{0,t},d_{1,t},...,d_{K,t}]$
After several iterations, the training process will be stable, and the difficulties of all classes defined above will be similar. Thus, we also accumulate $1-\lambda_{k,t}$ for $\tau$ iterations to obtain the reversed Dice weight $w_{\lambda_{k,t}}$ and weight $d_{k,t}$. 
In this case, classes with lower Dice scores will have larger weights in the loss function, which forces the model to pay more attention to these classes.
The overall difficulty-aware weight of $k^{th}$ class is defined as:
$w_k^{diff}= w_{\lambda_{k,t}} \cdot (d_{k,t})^\alpha$. $\alpha$ is empirically set to $\frac 1 5$ in the experiments to alleviate outliers.
The difficulty-aware weights for all classes are $W_t^{diff} = [w_1, w_2, ..., w_K]$.









\section{Experiments}

%\subsection{Dataset and Implementation Details}
\subsubsection{Dataset and Implementation Details.} We introduce two new benchmarks on the Synapse~\cite{synapse} and AMOS~\cite{amos} datasets for class-imbalanced semi-supervised medical image segmentation. 
The Synapse dataset has 13 foreground classes, including spleen (Sp), right kidney ({RK}), left kidney ({LK}), gallbladder ({Ga}), esophagus ({Es}), liver({Li}), stomach({St}), aorta ({Ao}), inferior vena cava ({IVC}), portal \& splenic veins ({PSV}), pancreas ({Pa}), right adrenal gland ({RAG}), left adrenal gland ({LAG}) with one background and 30 axial contrast-enhanced abdominal CT scans.
We randomly split them as 20, 4 and 6 scans for training, validation, and testing, respectively.
Compared with Synapse, the AMOS dataset excludes PSV but adds three new classes: duodenum(Du), bladder(Bl) and prostate/uterus(P/U). 360 scans are divided into 216, 24 and 120 scans for training, validation, and testing.
We ran experiments on Synapse three times with different seeds to eliminate the effect of randomness due to the limited samples.
%We implement the proposed framework with PyTorch, using a single NVIDIA A100 GPU. The network parameters are optimized with SGD with a momentum of 0.9 and an initial learning rate of 0.03. We employ a “poly” decay strategy follow~\cite{isensee2021nnunet}.
%In the training stage, simple data augmentations (random cropping and random flipping) are used to avoid over-fitting.
%We trained the networks 300 epochs with batch size 4, consisting of 2 labeled and 2 unlabeled data.
%In the inference stage, we use the average of the two sub-networks for prediction for all the CPS-based methods to avoid over-fitting to one of the perturbations.
%We evaluate the prediction of the network with two metrics, including Dice and the average surface distance (ASD).
%Final segmentation results are obtained using a sliding window strategy with a stride size of $32 \times 32 \times 16$.
\textbf{More training details are in the supplementary material.}


\begin{table}[!ht]
\scriptsize
\caption{Quantitative comparison between DHC and SOTA SSL segmentation methods on \textbf{20\% labeled Synapse dataset}. `General' or `Imbalance' indicates whether the methods consider the class imbalance issue or not. Results of 3-times repeated experiments are reported in the `mean$\pm$std' format. Best results are boldfaced, and $2^{nd}$ best results are underlined. %\xmli{$\dagger$: we implement existing semi-supervised segmentation methods on our dataset. $\star$: we extend existing semi-supervised classification methods to segmentation with CPS as the baseline.}
%Sp: spleen, RK: right kidney, LK: left kidney, Ga: gallbladder, ES: esophagus, Li: liver, St: stomach, Ao: Aorta, IVC: inferior vena cava, PSV: portal \& splenic veins, Pa: pancreas, RAG: right adrenal gland, LAG: left adrenal gland.
}
\label{sota1}
\resizebox*{\linewidth}{!}{

\begin{tabular}{c|c|cc|cccccccc@{\ }ccccc}
\toprule
\multicolumn{2}{c|}{\multirow{2}{*}{Methods}}  & \multirow{2}{*}{Avg. Dice} & \multirow{2}{*}{Avg. ASD} & \multicolumn{13}{c}{Average Dice of Each   Class}                                        \\ 
\multicolumn{2}{c|}{}                          &                            &                           & Sp   & RK   & LK   & Ga   & Es   & Li   & St   & Ao   & IVC  & PSV  & PA   & RAG  & LAG  \\\midrule

\multirow{9}{*}{\rotatebox{90}{General}}         & V-Net (fully)      & 62.09±1.2	&10.28±3.9	& 84.6	& 77.2	& 73.8	& 73.3	& 38.2	& 94.6	& 68.4	& 72.1	& 71.2	& 58.2	& 48.5	& 17.9	& 29.0 \\ \midrule

& UA-MT~\cite{yu2019uamt}$^\dagger$      & 20.26±2.2	&71.67±7.4	& 48.2	& 31.7	& 22.2	& \textcolor{red}{0.0}	& \textcolor{red}{0.0}	& 81.2	& 29.1	& 23.3	& 27.5	& \textcolor{red}{0.0}	& \textcolor{red}{0.0}	& \textcolor{red}{0.0}	& \textcolor{red}{0.0}  \\
                 
& URPC~\cite{luo2021urpc}$^\dagger$       & 25.68±5.1	&72.74±15.5	& \textbf{66.7}	& 38.2	& 56.8	& \textcolor{red}{0.0}	& \textcolor{red}{0.0}	& 85.3	& 33.9	& 33.1	& 14.8	& \textcolor{red}{0.0}	& 5.1	& \textcolor{red}{0.0}	& \textcolor{red}{0.0}  \\

& CPS~\cite{chen2021cps}$^\dagger$        & 33.55±3.7	&41.21±9.1	& 62.8	& 55.2	& 45.4	& 35.9	& \textcolor{red}{0.0}	& \textbf{91.1}	& 31.3	& 41.9	& 49.2	& 8.8	& 14.5	& \textcolor{red}{0.0}	& \textcolor{red}{0.0}  \\

& SS-Net~\cite{wu2022ssnet}$^\dagger$    & 35.08±2.8	&50.81±6.5	& 62.7	& 67.9	& \textbf{60.9}	& 34.3	& \textcolor{red}{0.0}	& 89.9	& 20.9	& 61.7	& 44.8	& \textcolor{red}{0.0}	& 8.7	& 4.2	& \textcolor{red}{0.0}  \\

& DST~\cite{chen2022dst}$^\star$        & 34.47±1.6	&37.69±2.9	& 57.7	& 57.2	& 46.4	& 43.7	& \textcolor{red}{0.0}	& 89.0	& 33.9	& 43.3	& 46.9	& 9.0	& \underline{21.0}	& \textcolor{red}{0.0}	& \textcolor{red}{0.0}  \\

& DePL~\cite{wang2022depl}$^\star$       & 36.27±0.9	&36.02±0.8	& 62.8	& 61.0	& 48.2	& 54.8	& \textcolor{red}{0.0}	& \underline{90.2}	& \underline{36.0}	& 42.5	& 48.2	& 10.7	& 17.0	& \textcolor{red}{0.0}	& \textcolor{red}{0.0}  \\ \midrule

\multirow{6}{*}{\rotatebox{90}{Imbalance}} 
& Adsh~\cite{guo2022adsh}$^\star$       & 35.29±0.5	&39.61±4.6	& 55.1	& 59.6	& 45.8	& 52.2	& \textcolor{red}{0.0}	& 89.4	& 32.8	& 47.6	& 53.0	& 8.9	& 14.4	& \textcolor{red}{0.0}	& \textcolor{red}{0.0}  \\ 

& CReST~\cite{wei2021crest}$^\star$      & 38.33±3.4	&\underline{22.85±9.0}	& 62.1	& 64.7	& 53.8	& 43.8	& \underline{8.1}	& 85.9	& 27.2	& 54.4	& 47.7	& 14.4	& 13.0	& \underline{18.7}	& 4.6  \\

& SimiS~\cite{simis}$^\star$      & 40.07±0.6	&32.98±0.5	& 62.3	& \underline{69.4}	& 50.7	& 61.4	& \textcolor{red}{0.0}	& 87.0	& 33.0	& 59.0	& \underline{57.2}	& \underline{29.2}	& 11.8	& \textcolor{red}{0.0}	& \textcolor{red}{0.0}  \\



& Basak \textit{et al.}~\cite{basak2022addressing}$^\dagger$        &33.24±0.6	&43.78±2.5	& 57.4	& 53.8	& 48.5	& 46.9	& \textcolor{red}{0.0}	& 87.8	& 28.7	& 42.3	& 45.4	& 6.3	& 15.0	& \textcolor{red}{0.0}	& \textcolor{red}{0.0}   \\
                                 
& CLD~\cite{lin2022cld}$^\dagger$        &\underline{41.07±1.2}	&32.15±3.3	& 62.0	& 66.0	& \underline{59.3}	& \underline{61.5}	& \textcolor{red}{0.0}	& 89.0	& 31.7	& \underline{62.8}	& 49.4	& 28.6	& 18.5	& \textcolor{red}{0.0}	& \underline{5.0}  \\
 
& \textbf{DHC (ours)}   & \textbf{48.61±0.9}	&\textbf{10.71±2.6}	& \underline{62.8}	& \textbf{69.5}	& 59.2	& \textbf{66.0}	& \textbf{13.2}	& 85.2	& \textbf{36.9}	& \textbf{67.9}	& \textbf{61.5}	& \textbf{37.0}	& \textbf{30.9}	& \textbf{31.4}	& \textbf{10.6} \\ \bottomrule
\end{tabular}
}
\begin{threeparttable}
 \begin{tablenotes}
        \scriptsize
        \item[$\dagger$] we implement semi-supervised segmentation methods on our dataset.
        \item[$\star$] we extend semi-supervised classification methods to segmentation with CPS as the baseline.
\end{tablenotes}
\end{threeparttable}

\end{table}


% Figure environment removed



\begin{table}[t]
\scriptsize
\caption{Quantitative comparison between DHC and SOTA SSL segmentation methods on \textbf{5\% labeled AMOS dataset}. 
}
\label{sota2}
\resizebox*{\linewidth}{!}{

\begin{tabular}{c|c|c@{\ \ }c|cccccccc@{\ }ccccccc}
\toprule
\multicolumn{2}{c|}{\multirow{2}{*}{Methods}}  & Avg. & Avg. & \multicolumn{15}{c}{Average Dice of Each   Class}                                        \\ 
\multicolumn{2}{c|}{}                          &    Dice                        &      ASD                     & Sp   & RK   & LK   & Ga   & Es   & Li   & St   & Ao   & IVC   & PA   & RAG  & LAG  & Du   & Bl  & P/U  \\\midrule
\multirow{9}{*}{\rotatebox{90}{General}}         & V-Net (fully)      & 76.50 	&2.01 	&92.2	&92.2	&93.3	&65.5	&70.3	&95.3	&82.4	&91.4	&85.0	&74.9	&58.6	&58.1	&65.6	&64.4	&58.3 \\ \midrule

& UA-MT~\cite{yu2019uamt}$^\dagger$      &42.16 	&15.48 	&59.8 	&64.9 	&64.0 	&35.3 	&\underline{34.1} 	&77.7 	&37.8 	&61.0 	&46.0 	&33.3 	&\underline{26.9} 	&12.3 	&18.1 	& 29.7	&31.6  \\

& URPC~\cite{luo2021urpc}$^\dagger$       & 44.93 	&27.44 	&67.0 	&64.2 	&67.2 	&36.1 	&\textcolor{red}{0.0} 	&83.1 	&\underline{45.5} 	&67.4 	&54.4 	&\textbf{46.7} 	&\textcolor{red}{0.0} 	&\textbf{29.4} 	&\textbf{35.2} 	&44.5	&33.2  \\

& CPS~\cite{chen2021cps}$^\dagger$        & 41.08 	&20.37 	&56.1 	&60.3 &	59.4 	&33.3 &	25.4 	&73.8 	&32.4 	&65.7 	&52.1 	&31.1 	&25.5 	&6.2 	&18.4 	&40.7	&35.8  \\

& SS-Net~\cite{wu2022ssnet}$^\dagger$    & 33.88 	&54.72 	&65.4 	&68.3 	&69.9 	&37.8 &\textcolor{red}{0.0} 	&75.1 	&33.2 	&68.0 	&\underline{56.6} 	&33.5 	&\textcolor{red}{0.0} 	&\textcolor{red}{0.0}	&\textcolor{red}{0.0} 	&0.2	&0.2  \\

& DST~\cite{chen2022dst}$^\star$        & 41.44 	&21.12 	&58.9 	&63.3 	&63.8 	&37.7 	&29.6 	&74.6 	&36.1 	&66.1 	&49.9 	&32.8 	&13.5 	&5.5 	&17.6 	&39.1	&33.1  \\

& DePL~\cite{wang2022depl}$^\star$       & 41.97 	&20.42 	&55.7 	&62.4 	&57.7 	&36.6 	&31.3 	&68.4 	&33.9 	&65.6 	&51.9 	&30.2 	&23.3 	&10.2 	&20.9 	&43.9	&\textbf{37.7}  \\ \midrule
\multirow{6}{*}{\rotatebox{90}{Imbalance}} 

& Adsh~\cite{guo2022adsh}$^\star$       & 40.33	& 24.53	& 56.0	& 63.6	& 57.3	& 34.7	& 25.7	& 73.9	& 30.7	& 65.7	& 51.9	& 27.1	& 20.2	& 0.0	& 18.6	& 43.5	& 35.9  \\ 

& CReST~\cite{wei2021crest}$^\star$      & 46.55	& 14.62	& 66.5	& 64.2	& 65.4	& 36.0	& 32.2	& 77.8	& 43.6	& 68.5	& 52.9	& 40.3	& 24.7	& 19.5	& 26.5	& 43.9	& 36.4  \\

& SimiS~\cite{simis}$^\star$      & \underline{47.27}	& \textbf{11.51}	& \textbf{77.4}	& \textbf{72.5}	& 68.7	& 32.1	& 14.7	& \textbf{86.6}	& \textbf{46.3}	& \textbf{74.6}	& 54.2	& 41.6	& 24.4	& 17.9	& 21.9	& \textbf{47.9}	& 28.2  \\


& Basak \textit{et al.}~\cite{basak2022addressing}$^\dagger$        & 38.73	& 31.76	& \underline{68.8}	& 59.0	& 54.2	& 29.0	& \textcolor{red}{0.0}	& \underline{83.7}	& 39.3	& 61.7	& 52.1	& 34.6	& \textcolor{red}{0.0}	& \textcolor{red}{0.0}	& 26.8	& \underline{45.7}	& 26.2   \\
                                 
& CLD~\cite{lin2022cld}$^\dagger$        & 46.10	& 15.86	& 67.2	& 68.5	& \textbf{71.4}	& \underline{41.0}	& 21.0	& 76.1	& 42.4	& 69.8	& 52.1	& 37.9	& 24.7	& 23.4	& 22.7	& 38.1	& 35.2  \\
 
                                 
& \textbf{DHC (ours)}   & \textbf{49.53}	& \underline{13.89}	& 68.1	& \underline{69.6}	& \underline{71.1}	& \textbf{42.3}	& \textbf{37.0}	& 76.8	& 43.8	& \underline{70.8}	& \textbf{57.4}	& \underline{43.2}	& \textbf{27.0}	& \underline{28.7}	& \underline{29.1}	& 41.4	& \underline{36.7} \\ \bottomrule
\end{tabular}
}
\begin{threeparttable}
 \begin{tablenotes}
        \scriptsize
        \item[$\dagger$] we implement semi-supervised segmentation methods on our dataset.
        \item[$\star$] we extend semi-supervised classification methods to segmentation with CPS as the baseline.
\end{tablenotes}
\end{threeparttable}

\end{table}



\begin{table}[t]
\scriptsize
\caption{Ablation study on \textbf{20\% labeled Synapse dataset}. 
The first four rows show the effectiveness of our proposed methods; the last four rows verify the importance of the heterogeneous design by combining our proposed modules with the existing class-imbalance methods, and serve as detailed results of the $3^{rd}$ and $4^{th}$ columns in Fig.~\ref{heterogeneous}.}
\label{ablation_v2}
\resizebox*{\linewidth}{!}{
\begin{tabular}{c|cc|cccccccc@{\ }ccccc}
\toprule
{\multirow{2}{*}{ModelA-ModelB}}  & \multirow{2}{*}{Avg. Dice} & \multirow{2}{*}{Avg. ASD} & \multicolumn{13}{c}{Average Dice of Each   Class}                                        \\ 
& & & Sp   & RK   & LK   & Ga   & Es   & Li   & St   & Ao   & IVC  & PSV  & Pa   & RAG  & LAG  \\\midrule

Org-Org (CPS)       & 33.55±3.65	&41.21±9.08	& 62.8	& 55.2	& 45.4	& 35.9	& 0.0	& \textbf{91.1}	& 31.3	& 41.9	& 49.2	& 8.8	& 14.5	& 0.0	& 0.0  \\

\textcolor[RGB]{153,97,132}{DistDW}-\textcolor[RGB]{153,97,132}{DistDW}        & 43.41±1.46	&\underline{17.39±1.93}	& 56.1	& 66.7	& \underline{60.2}	& 40.3	& \textbf{23.5}	& 75.7	& 10.3	& \underline{70.1}	& \underline{61.2}	& 30.4	& 26.4	& \textbf{32.2}	& \underline{11.2} \\

\textcolor[RGB]{33,101,168}{DiffDW}-\textcolor[RGB]{33,101,168}{DiffDW} & 42.75±0.6	&18.64±5.17	& \underline{71.9}	& 65.7	& 49.8	& 58.9	& 6.7	& 88.3	& 32.0	& 59.5	& 51.8	& 29.0	& 13.4	& 22.6	& 6.2 \\

\textbf{\textcolor[RGB]{33,101,168}{DiffDW}-\textcolor[RGB]{153,97,132}{DistDW} }  & \textbf{48.61±0.91}	&\textbf{10.71±2.62}	& 62.8	& 69.5	& 59.2	& 66.0	& \underline{13.2}	& 85.2	& \underline{36.9}	& 67.9	& \textbf{61.5}	& \textbf{37.0}	& \textbf{30.9}	& \underline{31.4}	& 10.6 \\ \midrule


\textcolor[RGB]{153,97,132}{DistDW}-CReST & 45.61±2.98	&21.58±3.89	& 68.7	& \textbf{70.5}	& 58.6	& 60.0	& 12.4	& 79.7	& 31.5	& 60.9	& 58.3	& 30.7	& \underline{29.0}	& 27.4	& 5.3  \\

\textcolor[RGB]{153,97,132}{DistDW}-SimiS & 41.92±2.52	&30.29±4.49	& 70.3	& 68.9	& \textbf{63.6}	& 56.2	& 3.1	& 77.0	& 14.0	& \textbf{75.4}	& 57.1	& 26.8	& 27.5	& 3.3	& 1.7  \\

\textcolor[RGB]{33,101,168}{DiffDW}-CReST & \underline{46.52±1.5}	&19.47±2.1	& 59.5	& 64.6	& 60.1	& \underline{67.3}	& 0.0	& 87.2	& 34.4	& 65.5	& 60.4	& 31.9	& 28.5	& 30.6	& \textbf{14.8} \\

\textcolor[RGB]{33,101,168}{DiffDW}-SimiS & 43.85±0.5	& 32.55±1.4	& \textbf{73.5}	& \underline{70.2}	& 54.9	& \textbf{69.7}	& 0.0	& \underline{89.4}	& \textbf{41.1}	& 67.5	& 51.8	& \underline{34.8}	& 16.7	& 0.6	& 0.0 \\
                                 
\bottomrule
\end{tabular}
}
\end{table}








\subsubsection{Comparison with State-of-the-Art Methods.}
%In our study, we evaluated several state-of-the-art semi-supervised segmentation methods (UA-MT~\cite{yu2019uamt}, URPC~\cite{luo2021urpc},SS-Net~\cite{wu2022ssnet}, CPS~\cite{chen2021cps}) and re-implemented SSL classification methods (CReST~\cite{wei2021crest}, SimiS~\cite{simis}, Adsh~\cite{guo2022adsh}, DST~\cite{chen2022dst}, DePL~\cite{wang2022depl}) for comparison. 
We compare our method with several state-of-the-art semi-supervised segmentation methods~\cite{yu2019uamt,luo2021urpc,wu2022ssnet,chen2021cps}. 
Moreover, simply extending the state-of-the-art semi-supervised classification methods~\cite{wei2021crest,simis,guo2022adsh,chen2022dst,wang2022depl}, including class-imbalanced designs~\cite{wei2021crest,simis,guo2022adsh} to segmentation, is a straightforward solution to our task. Therefore, we extend these methods to segmentation with CPS as the baseline. %\xmli{please confirm.}
As shown in Table~\ref{sota1}\&\ref{sota2}, the general semi-supervised methods which do not consider the class imbalance problem fail to capture effective features of the minority classes and lead to terrible performances (\textcolor{red}{colored with red}). 
%Thus, it is necessary to consider the class-imbalance problem in semi-supervised segmentation tasks and introduce a more representative benchmark.
The methods considered the class imbalance problem have better results on some smaller minority classes such as gallbladder, portal \& splenic veins and \textit{etc.}
However, they still fail in some minority classes (Es, RAG, and LAG) since this task is highly imbalanced. 
Our proposed DHC outperforms these methods, especially in those classes with very few samples.
Note that our method performs better than the fully-supervised method for the RAG segmentation. %\xmli{for LAG segmentation? Actually not better. please check.}
%% \subsection{Qualitative Comparison with State-of-the-Art Methods}
Furthermore, our method outperforms SOTA methods on Synapse by larger margins than the AMOS dataset, demonstrating the more prominent stability and effectiveness of the proposed DHC framework in scenarios with a severe lack of data.
Visualization results in Fig.~\ref{vis_SOTA} show our method performs better on minority classes which are pointed with green arrows.
More results on datasets with different labeled ratios can be found in the supplementary material.




\subsubsection{Ablation Study.}\label{ablation}
To validate the effectiveness of the proposed DHC framework and the two learning strategies, DistDW and DiffDW, we conduct ablation experiments, as shown in Table~\ref{ablation_v2}. 
%% We can see that WL improves the Dice scores of two cartilages 1.1\% and 9.3\% but brings a 2.3\% performance drop for hard tissues, which means WL can improve the learning of cartilages, but in turn negatively affect the learning of hard tissues. 
%% DUS maintains the improvements on cartilages and alleviates the performance drop of hard tissues. PRC can further boost the performance of both soft cartilages and hard tissues.
DistDW (`\textcolor[RGB]{153,97,132}{DistDW}-\textcolor[RGB]{153,97,132}{DistDW}') alleviates the bias of baseline on majority classes and thus segments the minority classes (RA, LA, ES, \textit{etc.}) very well. However, it has unsatisfactory results on the spleen and stomach, which are difficult classes but down-weighted due to the larger voxel numbers.
DiffDW (`\textcolor[RGB]{33,101,168}{DiffDW}-\textcolor[RGB]{33,101,168}{DiffDW}') shows complementary results with DistDW, it has better results on difficult classes (\eg, stomach since it is hollow inside).
%We visualized the weights curves of CSDD, SimiS and CReST to show the under-going properties of these methods.
%As shown in Fig.~\ref{curves}, 
%From the results we can oberserve that DiffDW and DistDW are quite radical on their targeting classes but have inferior results on some of the other classes, \eg, only 5.3\% on esophagus with DiffDW and 10.1\% on stomach with DistDW in terms of Dice.
When combining these two weighting strategies in a heterogeneous co-training way (`\textcolor[RGB]{33,101,168}{DiffDW}-\textcolor[RGB]{153,97,132}{DistDW}', namely DHC), the Dice score has 5.12\%, 5.6\% and 13.78\% increase compared with DistDW, DiffDW, and the CPS baseline.
These results highlight the efficacy of incorporating heterogeneous information in avoiding over-fitting and enhancing the performance of the CPS baseline.
%which verifies the effectiveness of complementing diverse information to further improve the CPS baseline.
%DHC is like a medium of DiffDW and DistDW, it is not that radical as each of them but takes both strengths and achieves best overall performance in a more balanced way.


\section{Conclusion}
This work proposes a novel Dual-debiased Heterogeneous Co-training framework for class-imbalanced semi-supervised segmentation. We are the first to state the homogeneity issue of CPS and solve it intuitively in a heterogeneous way.
To achieve it, we propose two diverse and accurate weighting strategies: DistDW for eliminating the data bias of majority classes and DiffDW for eliminating the learning bias of well-performed classes.
By combining the complementary properties of DistDW and DiffDW, the overall framework can learn both the minority classes and the difficult classes well in a balanced way.
Extensive experiments show that the proposed framework brings significant improvements over the baseline and outperforms previous SSL methods considerably.


\section*{Acknowledgement}
This work was supported in part by a grant from Hong Kong Innovation and Technology Commission (Project no. ITS/030/21) and in part by a research grant from Beijing Institute of Collaborative Innovation (BICI) under collaboration with HKUST under Grant HCIC-004 and in part by grants from Foshan HKUST Projects under Grants FSUST21-HKUST10E and FSUST21-HKUST11E.




\bibliographystyle{splncs04}
\bibliography{miccai_bib}
%
% \begin{thebibliography}{8}
% \bibitem{ref_article1}
% Author, F.: Article title. Journal \textbf{2}(5), 99--110 (2016)

% \bibitem{ref_lncs1}
% Author, F., Author, S.: Title of a proceedings paper. In: Editor,
% F., Editor, S. (eds.) CONFERENCE 2016, LNCS, vol. 9999, pp. 1--13.
% Springer, Heidelberg (2016). \doi{10.10007/1234567890}

% \bibitem{ref_book1}
% Author, F., Author, S., Author, T.: Book title. 2nd edn. Publisher,
% Location (1999)

% \bibitem{ref_proc1}
% Author, A.-B.: Contribution title. In: 9th International Proceedings
% on Proceedings, pp. 1--2. Publisher, Location (2010)

% \bibitem{ref_url1}
% LNCS Homepage, \url{http://www.springer.com/lncs}. Last accessed 4
% Oct 2017
% \end{thebibliography}









\newpage

\title{Supplementary Material}

\author{Haonan Wang, Xiaomeng Li\textsuperscript{(\Letter)}} 
% index{Haonan, Wang}
% index{Xiaomeng, Li}

\institute{Department of Electronic and Computer Engineering, The Hong Kong University
of Science and Technology, Hong Kong, China
 \\ \email{eexmli@ust.hk}}

\maketitle







\section*{Training Details}
We implement the proposed framework with PyTorch, using a single NVIDIA A100 GPU. The network parameters are optimized with SGD with a momentum of 0.9 and an initial learning rate of 0.03. We employ a “poly” decay strategy follow~\cite{isensee2021nnunet}.
In the training stage, simple data augmentations (random cropping and random flipping) are used to avoid over-fitting.
We trained the networks 300 epochs with batch size 4, consisting of 2 labeled and 2 unlabeled data.
In the inference stage, we use the average of the two sub-networks for prediction for all the CPS-based methods to avoid over-fitting to one of the perturbations.
We evaluate the prediction of the network with two metrics, including Dice and the average surface distance (ASD).
Final segmentation results are obtained using a sliding window strategy with a stride size of $32 \times 32 \times 16$.



\begin{table}[!ht]
\scriptsize
\caption{Quantitative comparison between DHC and SOTA SSL segmentation methods on \textbf{10\% labeled Synapse dataset}. `General' or `Imbalance' indicate whether the methods consider class-imbalance issue or not.
Sp: spleen, RK: right kidney, LK: left kidney, Ga: gallbladder, Es: esophagus, Li: liver, St: stomach, Ao: aorta, IVC: inferior vena cava, PSV: portal \& splenic veins, Pa: pancreas, RAG: right adrenal gland, LAG: left adrenal gland.
}
\label{sota}
\resizebox*{\linewidth}{!}{
\begin{tabular}{c|c|cc|cccccccc@{\ }ccccc}
\toprule
\multicolumn{2}{c|}{\multirow{2}{*}{Methods}}  & \multirow{2}{*}{Avg. Dice} & \multirow{2}{*}{Avg. ASD} & \multicolumn{13}{c}{Average Dice of Each   Class}                                        \\ 
\multicolumn{2}{c|}{}                          &                            &                           & Sp   & RK   & LK   & Ga   & Es   & Li   & St   & Ao   & IVC  & PSV  & PA   & RAG  & LAG  \\\midrule
\multirow{9}{*}{\rotatebox{90}{General}}         & V-Net (fully)      & 62.09±1.2	&10.28±3.9	& 84.6	& 77.2	& 73.8	& 73.3	& 38.2	& 94.6	& 68.4	& 72.1	& 71.2	& 58.2	& 48.5	& 17.9	& 29.0 \\ \midrule

& UA-MT~\cite{yu2019uamt}$^\dagger$     &18.07±1.2	&57.64±1.8	& 27.1	& 7.1	& 17.0	& 24.4	& \textcolor{red}{0.0}	& 80.6	& 15.6	& 39.3	& 16.7	& 4.4	& 2.7	& \textcolor{red}{0.0}	& \textcolor{red}{0.0} \\


& URPC~\cite{luo2021urpc}$^\dagger$       &26.37±1.5	&53.95±11.3	& \textbf{51.7}	& 35.1	& 26.4	& 7.3	& \textcolor{red}{0.0}	& \textbf{83.8}	& 21.3	& \textbf{69.0}	& \textbf{41.0}	& 1.9	& 5.2	& \textcolor{red}{0.0} 	&\textcolor{red}{0.0}  \\

& CPS~\cite{chen2021cps}$^\dagger$        &21.96±1.2	&55.42±4.6	& 37.9	& 31.8	& 19.0	& 31.9	& \textcolor{red}{0.0}	& 65.1	& 15.5	& 44.8	& 29.6	& 4.3	& 5.5	& \textcolor{red}{0.0}	& \textcolor{red}{0.0}  \\

& SS-Net~\cite{wu2022ssnet}$^\dagger$     &17.5±3.0	&66.17±8.0	& 45.6	& 11.6	& \textbf{42.3}	& 2.4	& \textcolor{red}{0.0}	& 74.5	& 6.0	& 32.6	& 2.8	& \textcolor{red}{0.0}	& \textcolor{red}{0.0}	& 3.8	& 5.8  \\

& DST~\cite{chen2022dst}$^\star$        &20.91±5.9	&61.33±18.3	& 43.3	& 32.8	& 16.0	& 24.9	& \textcolor{red}{0.0}	& 75.8	& \textbf{22.4}	& 27.6	& 19.4	& 3.8	& 5.4	& 0.3	& \textcolor{red}{0.0}\\

& DePL~\cite{wang2022depl}$^\star$       &21.01±3.3	&58.42±6.3	& 34.2	& 32.2	& 17.3	& 27.2	& \textcolor{red}{0.0}	& 65.7	& 16.8	& 40.8	& 29.3	& 2.8	& 6.8	& \textcolor{red}{0.0}	& \textcolor{red}{0.0} \\ \midrule
\multirow{8}{*}{\rotatebox{90}{Imbalance}} 

& Adsh~\cite{guo2022adsh}$^\star$       &22.8±0.9	&46.18±4.0	& 36.0	& 35.7	& 20.0	& 31.0	& \textcolor{red}{0.0}	& 74.7	& 18.3	& 32.3	& 27.8	& 11.7	& 7.3	& 1.7	& \textcolor{red}{0.0}  \\ 

& CReST~\cite{wei2021crest}$^\star$      &26.56±2.9	&36.17±1.0	& 37.3	& 46.5	& 25.2	& 27.1	& 1.7	& 66.3	& 14.2	& 45.2	& 35.8	& 11.2	& 6.8	& 24.2	& 3.8  \\

& SimiS~\cite{simis}$^\star$      & 25.05±3.1	&43.93±2.4	& 42.0	& 38.6	& 27.2	& 19.7	& \textcolor{red}{0.0}	& 74.2	& 16.5	& 51.7	& 35.0	& 13.6	& 5.4	& \textcolor{red}{0.0}	& 1.8  \\

& Basak \textit{et al.}~\cite{basak2022addressing}$^\dagger$         &25.3±2.2	&50.02±5.7	& 40.9	& 42.3	& 19.2	& 35.2	& \textcolor{red}{0.0}	& 75.7	& 19.2	& 44.7	& 32.8	& 5.0	& \textbf{10.4}	& 3.5	& \textcolor{red}{0.0}  \\
                                 
& CLD~\cite{lin2022cld}$^\dagger$        & 22.49±1.6	&49.74±4.1	& 39.3	& 43.9	& 25.6	& 12.8	& \textcolor{red}{0.0}	& 73.3	& 14.3	& 41.1	& 25.7	& 8.8	& 6.1	& 0.2	& 1.1  \\

& \textbf{DistDW(ours)}   & 27.21±0.9	&32.38±4.0	& 47.8	& 42.3	& 33.1	& 27.0	& 1.1	& 65.5	& 20.7	& 49.0	& 34.5	& 8.1	& 7.4	& 14.5	& 2.8 \\

& \textbf{DiffDW(ours)}   & 28.63±2.5	&24.81±4.0	& 44.1	& 33.4	& 25.3	& \textbf{37.0}	& 6.3	& 75.8	& 19.1	& 46.3	& 28.6	& \textbf{17.5}	& 7.8	& \textbf{24.7}	& 6.3\\
                                  
& \textbf{DHC(ours)}   & \textbf{31.64±0.9}	&\textbf{21.82±1.0}	& 45.1	& \textbf{47.4}	& 33.1	& 36.6	& \textbf{7.1}	& 71.4	& 17.8	& 58.9	& 34.4	& 16.5	& 9.3	& 21.8	& \textbf{12.0} \\ \bottomrule
\end{tabular}
}
\begin{threeparttable}
 \begin{tablenotes}
        \scriptsize
        \item[$\dagger$] we implement semi-supervised segmentation methods on our dataset.
        \item[$\star$] we extend semi-supervised classification methods to segmentation with CPS as the baseline.
\end{tablenotes}
\end{threeparttable}
\end{table}




\begin{table}[!ht]
\scriptsize
\caption{Quantitative comparison between DHC and SOTA SSL segmentation methods on\textbf{ 40\% labeled Synapse dataset}. 
%Sp: spleen, RK: right kidney, LK: left kidney, Ga: gallbladder, ES: esophagus, Li: liver, St: stomach, Ao: Aorta, IVC: inferior vena cava, PSV: portal \& splenic veins, Pa: pancreas, RAG: right adrenal gland, LAG: left adrenal gland.
}
\label{sota}
\resizebox*{\linewidth}{!}{
\begin{tabular}{c|c|cc|cccccccc@{\ }ccccc}
\toprule
\multicolumn{2}{c|}{\multirow{2}{*}{Methods}}  & \multirow{2}{*}{Avg. Dice} & \multirow{2}{*}{Avg. ASD} & \multicolumn{13}{c}{Average Dice of Each   Class}                                        \\ 
\multicolumn{2}{c|}{}                          &                            &                           & Sp   & RK   & LK   & Ga   & Es   & Li   & St   & Ao   & IVC  & PSV  & PA   & RAG  & LAG  \\\midrule
\multirow{9}{*}{\rotatebox{90}{General}}         & V-Net (fully)      & 62.09±1.2	&10.28±3.9	& 84.6	& 77.2	& 73.8	& 73.3	& 38.2	& 94.6	& 68.4	& 72.1	& 71.2	& 58.2	& 48.5	& 17.9	& 29.0 \\ \midrule

& UA-MT~\cite{yu2019uamt}$^\dagger$      &17.09±2.97	&91.86±7.93	& 4.2	& 57.8	& 32.2	& \textcolor{red}{0.0}	& \textcolor{red}{0.0}	& 91.0	& 37.0	& \textcolor{red}{0.0}	& \textcolor{red}{0.0}	& \textcolor{red}{0.0}	& \textcolor{red}{0.0}	& \textcolor{red}{0.0}	& \textcolor{red}{0.0}  \\

& URPC~\cite{luo2021urpc}$^\dagger$       &24.83±8.19	&74.44±17.01	& 42.6	& 44.8	& 51.4	& \textcolor{red}{0.0}	& \textcolor{red}{0.0}	& 86.7	& 37.8	& 25.8	& 27.0	& \textcolor{red}{0.0}	& 6.6	& \textcolor{red}{0.0}	& \textcolor{red}{0.0}  \\

& CPS~\cite{chen2021cps}$^\dagger$       &33.07±1.07	&60.46±2.25	& 68.4	& 72.7	& 64.2	& \textcolor{red}{0.0}	& \textcolor{red}{0.0}	& 91.9	& 42.1	& 66.2	& 22.3	& \textcolor{red}{0.0}	& 2.2	& \textcolor{red}{0.0}	& \textcolor{red}{0.0} \\

& SS-Net~\cite{wu2022ssnet}$^\dagger$     &32.98±10.99	&71.18±20.77	& 49.0	& 68.9	& 71.4	& 22.9	& \textcolor{red}{0.0}	& 92.0	& 34.7	& 51.7	& 38.1	& \textcolor{red}{0.0}	& \textcolor{red}{0.0}	& \textcolor{red}{0.0}	& \textcolor{red}{0.0} \\

& DST~\cite{chen2022dst}$^\star$        &35.57±1.54	&55.69±1.43	& 73.8	& \textbf{73.2}	& 64.2	& \textcolor{red}{0.0}	& \textcolor{red}{0.0}	& \textbf{92.1}	& 41.3	& 71.8	& 40.7	& \textcolor{red}{0.0}	& 5.2	& \textcolor{red}{0.0}	& \textcolor{red}{0.0}  \\

& DePL~\cite{wang2022depl}$^\star$       &36.16±2.08	&56.14±7.61	& 72.7	& 72.4	& 64.4	& 13.3	& \textcolor{red}{0.0}	& 91.7	& 42.8	& 63.8	& 47.0	& \textcolor{red}{0.0}	& 1.9	& \textcolor{red}{0.0}	& \textcolor{red}{0.0}  \\ \midrule
\multirow{8}{*}{\rotatebox{90}{Imbalance}} 

& Adsh~\cite{guo2022adsh}$^\star$       &35.91±6.17	&53.7±6.95	& 66.8	& 72.5	& 64.4	& 19.1	& \textcolor{red}{0.0}	& 91.8	& 43.8	& 62.0	& 39.8	& \textcolor{red}{0.0}	& 6.5	& \textcolor{red}{0.0}	& \textcolor{red}{0.0}  \\ 

& CReST~\cite{wei2021crest}$^\star$      &41.6±2.49	&27.82±5.07	& 53.8	& 69.5	& 58.1	& 35.3	& 17.7	& 85.0	& 36.0	& 60.3	& 45.2	& 21.5	& 24.2	& 23.3	& 10.8 \\

& SimiS~\cite{simis}$^\star$      &47.09±2.33	&33.46±1.75	& 75.4	& 66.9	& 69.0	& 62.6	& \textcolor{red}{0.0}	& 81.6	& \textbf{53.1}	& 80.4	& 56.2	& 29.9	& 37.1	& \textcolor{red}{0.0}	& \textcolor{red}{0.0} \\

& Basak \textit{et al.}~\cite{basak2022addressing}$^\dagger$   &35.03±3.68	&60.69±6.57	& 69.1	& 72.8	& 67.5	& \textcolor{red}{0.0}	& \textcolor{red}{0.0}	& 91.6	& 45.0	& 62.2	& 39.1	& \textcolor{red}{0.0}	& 8.0	& \textcolor{red}{0.0}	& \textcolor{red}{0.0}   \\
                                 
& CLD~\cite{lin2022cld}$^\dagger$        &48.23±1.02	&28.79±3.64	& 78.3	& 73.1	& \textbf{73.8}	& 57.0	& \textcolor{red}{0.0}	& 87.6	& 51.7	& 82.8	& 50.5	& 38.0	& 31.8	& 1.5	& 0.8 \\

& \textbf{DistDW(ours)}   &51.86±4.13	&14.68±4.31	& 78.0	& 69.7	& 66.2	& 65.9	& \textbf{34.4}	& 74.3	& 23.2	& 76.9	& 58.0	& 32.4	& 33.4	& 29.6	& 32.1 \\

& \textbf{DiffDW(ours)}   &50.84±2.78	&19.24±11.48	& 76.7	& 67.8	& 68.4	& 58.6	& 3.2	& 82.5	& 42.8	& 83.6	& 51.0	& 44.2	& 42.3	& 22.9	& 16.9 \\
                                
& \textbf{DHC(ours)}   &\textbf{57.13±0.8}	&\textbf{11.66±2.7}	& \textbf{82.5}	& 72.8	& 73.5	& \textbf{69.8}	& 10.7	& 71.9	& 41.2	& \textbf{83.7}	& \textbf{66.1}	& \textbf{53.8}	& \textbf{47.4}	& \textbf{36.8}	& \textbf{32.7} \\ \bottomrule
\end{tabular}
}
\begin{threeparttable}
 \begin{tablenotes}
        \scriptsize
        \item[$\dagger$] we implement semi-supervised segmentation methods on our dataset.
        \item[$\star$] we extend semi-supervised classification methods to segmentation with CPS as the baseline.
\end{tablenotes}
\end{threeparttable}
\end{table}




\begin{table}[!ht]
\scriptsize
\caption{Quantitative comparison between DHC and SOTA SSL segmentation methods on \textbf{2\% labeled AMOS dataset}. 
}
\label{sota2}
\resizebox*{\linewidth}{!}{

\begin{tabular}{c|c|c@{\ \ }c|cccccccc@{\ }ccccccc}
\toprule
\multicolumn{2}{c|}{\multirow{2}{*}{Methods}}  & Avg. & Avg. & \multicolumn{15}{c}{Average Dice of Each   Class}                                        \\ 
\multicolumn{2}{c|}{}                          &    Dice                        &      ASD                     & Sp   & RK   & LK   & Ga   & Es   & Li   & St   & Ao   & IVC   & PA   & RAG  & LAG  & Du   & Bl  & P/U  \\\midrule
\multirow{9}{*}{\rotatebox{90}{General}}         & V-Net (fully)      & 76.50 	&2.01 	&92.2	&92.2	&93.3	&65.5	&70.3	&95.3	&82.4	&91.4	&85.0	&74.9	&58.6	&58.1	&65.6	&64.4	&58.3 \\ \midrule

& UA-MT~\cite{yu2019uamt}$^\dagger$      & 33.96	& 22.43	& \textbf{62.5}	& \textbf{61.7}	& \textbf{59.8}	& 17.5	& 13.8	& 73.4	& 39.4	& 34.6	& 32.4	& 26.5	& 12.1	& 6.5	& 15.3	& 32.4	& 21.7  \\

& URPC~\cite{luo2021urpc}$^\dagger$       & \textbf{38.39}	& 37.58	& 60.8	& 57.7	& 56.5	& \textbf{34.6}	& \textcolor{red}{0.0}	& \textbf{78.4}	& \textbf{41.4}	& \textbf{53.3}	& \textbf{49.6}	& \textbf{40.4}	& \textcolor{red}{0.0}	& \textcolor{red}{0.0}	& \textbf{30.1}	& 42.5	& \textbf{30.6}  \\

& CPS~\cite{chen2021cps}$^\dagger$        & 31.78	& 39.23	& 55.9	& 46.9	& 53.1	& 27.7	& \textcolor{red}{0.0}	& 66.4	& 25.2	& 41.8	& 45.2	& 29.4	& 0.1	& \textcolor{red}{0.0}	& 22.1	& 38.7	& 24.2  \\

& SS-Net~\cite{wu2022ssnet}$^\dagger$    & 17.47	& 59.05	& 37.7	& 20.1	& 26.3	& 9.0	& 3.3	& 57.1	& 25.1	& 28.4	& 28.2	& \textcolor{red}{0.0}	& \textcolor{red}{0.0}	& \textcolor{red}{0.0}	& \textcolor{red}{0.0}	& 26.5	& 0.2  \\

& DST~\cite{chen2022dst}$^\star$        & 31.94	& 39.15	& 50.9	& 52.4	& 56.9	& 24.6	& \textcolor{red}{0.0}	& 59.4	& 31.5	& 41.8	& 43.1	& 26.2	& \textcolor{red}{0.0}	& \textcolor{red}{0.0}	& 23.8	& 42.6	& 25.9  \\

& DePL~\cite{wang2022depl}$^\star$       & 31.56	& 40.70	& 57.1	& 49.3	& 54.3	& 26.6	& 0.1	& 69.2	& 26.2	& 41.1	& 46.7	& 23.9	& \textcolor{red}{0.0}	& \textcolor{red}{0.0}	& 16.7	& 40.3	& 21.8  \\ \midrule
\multirow{6}{*}{\rotatebox{90}{Imbalance}} 

& Adsh~\cite{guo2022adsh}$^\star$       & 30.30	& 42.48	& 53.9	& 45.1	& 51.2	& 28.5	& \textcolor{red}{0.0}	& 62.1	& 27.0	& 41.4	& 42.7	& 25.0	& \textcolor{red}{0.0}	& \textcolor{red}{0.0}	& 20.3	& 35.8	& 21.6  \\ 

& CReST~\cite{wei2021crest}$^\star$      & 34.13	& 20.15	& 57.9	& 51.5	& 49.1	& 22.7	& 13.2	& 66.2	& 34.4	& 39.4	& 40.4	& 24.6	& 17.2	& 10.2	& 24.4	& 36.5	& 24.4  \\

& SimiS~\cite{simis}$^\star$     & 36.89	& 26.16	& 57.8	& 58.6	& 58.6	& 22.9	& \textcolor{red}{0.0}	& 70.9	& 38.0	& 52.0	& 47.0	& 32.4	& 20.2	& \textbf{11.5}	& 18.1	& 39.9	& 25.5  \\


& Basak \textit{et al.}~\cite{basak2022addressing}$^\dagger$        & 29.87	& 35.55	& 50.7	& 47.7	& 44.1	& 21.1	& \textcolor{red}{0.0}	& 61.8	& 27.7	& 38.1	& 40.4	& 21.8	& 9.6	& 9.5	& 14.6	& 36.5	& 24.5   \\
                                 
& CLD~\cite{lin2022cld}$^\dagger$        & 36.23	& 27.63	& 55.8	& 55.8	& 59.1	& 23.9	& \textcolor{red}{0.0}	& 69.9	& 38.2	& 50.1	& 44.5	& 32.3	& 18.9	& 9.2	& 18.8	& 42.2	& 24.9  \\
 
                                 
& \textbf{DHC (ours)}   & 38.28	& \textbf{20.34}	& 62.1	& 59.5	& 57.8	& 25.0	& \textbf{20.5}	& 66.0	& 38.2	& 51.3	& 47.9	& 26.8	& \textbf{26.4}	& 7.0	& 17.8	& \textbf{43.2}	& 24.8 \\ \bottomrule
\end{tabular}
}
\begin{threeparttable}
 \begin{tablenotes}
        \scriptsize
        \item[$\dagger$] we implement semi-supervised segmentation methods on our dataset.
        \item[$\star$] we extend semi-supervised classification methods to segmentation with CPS as the baseline.
\end{tablenotes}
\end{threeparttable}

\end{table}



\begin{table}[!ht]
\scriptsize
\caption{Quantitative comparison between DHC and SOTA SSL segmentation methods on \textbf{10\% labeled AMOS dataset}. 
}
\label{sota2}
\resizebox*{\linewidth}{!}{

\begin{tabular}{c|c|c@{\ \ }c|cccccccc@{\ }ccccccc}
\toprule
\multicolumn{2}{c|}{\multirow{2}{*}{Methods}}  & Avg. & Avg. & \multicolumn{15}{c}{Average Dice of Each   Class}                                        \\ 
\multicolumn{2}{c|}{}                          &    Dice                        &      ASD                     & Sp   & RK   & LK   & Ga   & Es   & Li   & St   & Ao   & IVC   & PA   & RAG  & LAG  & Du   & Bl  & P/U  \\\midrule
\multirow{9}{*}{\rotatebox{90}{General}}         & V-Net (fully)      & 76.50 	&2.01 	&92.2	&92.2	&93.3	&65.5	&70.3	&95.3	&82.4	&91.4	&85.0	&74.9	&58.6	&58.1	&65.6	&64.4	&58.3 \\ \midrule

& UA-MT~\cite{yu2019uamt}$^\dagger$      & 40.60	& 38.45	& 61.0	& 75.4	& 58.8	& 0.1	& \textcolor{red}{0.0}	& 84.4	& 45.2	& 72.8	& 61.6	& 36.2	& \textcolor{red}{0.0}	& \textcolor{red}{0.0}	& 30.7	& 46.5	& 36.3  \\

& URPC~\cite{luo2021urpc}$^\dagger$       & 49.09	& 29.69	& 81.7	& 77.5	& 77.2	& 38.1	& \textcolor{red}{0.0}	& 87.7	& 57.9	& 75.0	& 62.7	& 52.1	& \textcolor{red}{0.0}	& \textcolor{red}{0.0}	& 35.9	& 48.8	& 41.7  \\

& CPS~\cite{chen2021cps}$^\dagger$        & 54.51	& 7.84	& 80.7	& 79.8	& 74.3	& 35.2	& 44.4	& 90.5	& 51.1	& 76.1	& 65.6	& 48.6	& 31.6	& 21.8	& 33.6	& 47.0	& 37.3  \\

& SS-Net~\cite{wu2022ssnet}$^\dagger$    & 38.91	& 53.43	& 73.4	& 73.4	& 72.2	& 42.4	& \textcolor{red}{0.0}	& 83.5	& 46.7	& 74.1	& 69.6	& \textcolor{red}{0.0}	& \textcolor{red}{0.0}	& \textcolor{red}{0.0}	&\textcolor{red}{0.0}	& 48.3	& 0.2 \\

& DST~\cite{chen2022dst}$^\star$        & 52.24	& 17.66	& 81.7	& 80.2	& 78.6	& 39.5	& 41.0	& 89.8	& 52.8	& 78.5	& 65.9	& 51.1	& 4.3	& 0.1	& 34.2	& 48.8	& 37.2  \\

& DePL~\cite{wang2022depl}$^\star$      & 56.76	& 6.70	& 81.9	& 80.6	& 79.5	& 41.0	& 42.6	& 89.3	& 57.6	& 79.1	& 66.0	& 53.2	& 34.6	& 21.8	& 34.9	& 48.4	& 40.9  \\ \midrule
\multirow{6}{*}{\rotatebox{90}{Imbalance}} 

& Adsh~\cite{guo2022adsh}$^\star$       & 54.92	& 8.07	& 81.6	& 78.5	& 76.6	& 40.1	& 43.4	& 90.1	& 53.0	& 76.7	& 64.4	& 48.3	& 25.9	& 24.2	& 34.7	& 48.7	& 37.7  \\ 

& CReST~\cite{wei2021crest}$^\star$      & 60.74	& 4.65	& 85.3	& 84.5	& 84.0	& 43.2	& 50.8	& 89.9	& 58.7	& 84.7	& 73.0	& 54.2	& \textbf{41.8}	& 31.6	& 41.0	& 52.8	& 35.8  \\

& SimiS~\cite{simis}$^\star$      & 57.48	& 4.46	& 83.1	& 80.9	& 80.0	& 39.6	& 45.9	& 90.0	& 57.1	& 78.0	& 66.3	& 54.1	& 35.8	& 26.9	& 39.9	& 49.3	& 35.4  \\


& Basak \textit{et al.}~\cite{basak2022addressing}$^\dagger$       & 53.66	& 8.50	& 80.3	& 78.2	& 79.0	& 36.3	& 40.3	& 88.6	& 53.2	& 76.8	& 65.6	& 46.8	& 23.9	& 16.1	& 31.4	& 49.7	& 38.6   \\
                                 
& CLD~\cite{lin2022cld}$^\dagger$        & 61.55	& 4.21	& 86.0	& 85.3	& 84.8	& 44.5	& 51.9	& \textbf{90.8}	& 59.7	& 83.7	& 73.1	& 55.7	& 40.2	& \textbf{37.2}	& 41.4	& 53.0	& 36.1 \\
 
                                 
& \textbf{DHC (ours)}   & \textbf{64.16}	& \textbf{3.51}	& \textbf{87.4}	& \textbf{86.6}	& \textbf{87.1}	& \textbf{45.8}	& \textbf{57.0}	& 89.8	& \textbf{64.7}	& \textbf{86.0}	& \textbf{75.0}	& \textbf{62.5}	& 39.8	& 36.8	& \textbf{44.0}	& \textbf{56.5}	& \textbf{43.6} \\ \bottomrule
\end{tabular}
}
\begin{threeparttable}
 \begin{tablenotes}
        \scriptsize
        \item[$\dagger$] we implement semi-supervised segmentation methods on our dataset.
        \item[$\star$] we extend semi-supervised classification methods to segmentation with CPS as the baseline.
\end{tablenotes}
\end{threeparttable}

\end{table}


\end{document}
