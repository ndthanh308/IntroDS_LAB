\documentclass[notitlepage,superscriptaddress,aps,prl,twocolumn,noeprint]{revtex4-2}% http://ctan.org/pkg/revtex4-1
%\documentclass[twocolumn]{article}
\usepackage{amsmath}
\usepackage{amssymb}
\usepackage[sort&compress]{natbib}
\setcitestyle{numbers,comma,square}
\usepackage{upgreek}
\usepackage{mathtools}
\usepackage{graphicx}
\usepackage{afterpage}
\usepackage{amssymb}
\usepackage{epstopdf}
\usepackage[colorlinks=true,citecolor=blue,linkcolor=magenta]{hyperref}
\epstopdfsetup{outdir=./}
\usepackage{tikz}
\usepackage{color}
\usepackage{bm}
\raggedbottom
\usepackage[caption=false,position=top,singlelinecheck=off,justification=raggedright]{subfig}

\newcommand{\samedit}[1]{{\color[rgb]{0.5,0.5,0} #1}}

\usepackage{xr}
\makeatletter


\newcommand*{\addFileDependency}[1]{% argument=file name and extension
\typeout{(#1)}% latexmk will find this if $recorder=0
% however, in that case, it will ignore #1 if it is a .aux or 
% .pdf file etc and it exists! If it doesn't exist, it will appear 
% in the list of dependents regardless)
%
% Write the following if you want it to appear in \listfiles 
% --- although not really necessary and latexmk doesn't use this
%
\@addtofilelist{#1}
%
% latexmk will find this message if #1 doesn't exist (yet)
\IfFileExists{#1}{}{\typeout{No file #1.}}
}\makeatother

\newcommand*{\myexternaldocument}[1]{%
\externaldocument{#1}%
\addFileDependency{#1.tex}%
\addFileDependency{#1.aux}%
}
%------------End of helper code--------------

% put all the external documents here!
\myexternaldocument{Superfluid_NDC_main}

\renewcommand{\theequation}{S\arabic{equation}}
\renewcommand{\thefigure}{S\arabic{figure}}
\renewcommand{\bibnumfmt}[1]{[S#1]}
\renewcommand{\citenumfont}[1]{S#1}
\newcommand{\mjd}[1]{ {\color{blue} \textbf{MJD:} #1}}


\DeclarePairedDelimiter\bra{\langle}{\rvert}  
\DeclarePairedDelimiter\ket{\lvert}{\rangle}       
\DeclarePairedDelimiterX\braket[2]{\langle}{\rangle}{#1 \delimsize\vert #2}      


\begin{document}
\title{Supplemental Material for ``Nonequilibrium  Transport in a Superfluid Josephson Junction Chain: Is There Negative Differential Conductivity?"}

\author{Samuel E. Begg}
\affiliation{Asia Pacific Center for Theoretical Physics, Pohang 37673, Korea}
\affiliation{Australian Research Council Centre of Excellence in Future Low-Energy Electronics Technologies, School of Mathematics and Physics, University of Queensland, St Lucia, Queensland 4072, Australia.}
\author{Matthew J. Davis}
\affiliation{Australian Research Council Centre of Excellence in Future Low-Energy Electronics Technologies, School of Mathematics and Physics, University of Queensland, St Lucia, Queensland 4072, Australia.}
\author{Matthew T. Reeves}
\affiliation{Australian Research Council Centre of Excellence in Future Low-Energy Electronics Technologies, School of Mathematics and Physics, University of Queensland, St Lucia, Queensland 4072, Australia.}
\date{\today}

\maketitle

This supplemental material is organised as follows.  We begin with an overview of the experimental procedure.  We then provide further details of the simulations, including the choice of parameters for the c-field simulations, how the current is calculated, and discuss the robustness of the results to varying the initial conditions and the energy cut-off of the c-field region. We then analyse the emergence of a condensate on the central site and examine the adiabaticity of the dynamics of the refilling of the system site.  Finally, we provide more results on the dependence of the dynamics on the initial coherence between the two sides of the Josephson chain, and describe simulations performed at finite temperature with a limited condensate fraction to demonstrate the absence of acceleration of filling without coherence.

\section{Review of the Experiment}
The experiment \cite{Labouvie2015} consists of an approximately 65 site 1D chain of 2D harmonic traps, see Fig.~\ref{fig:Schematic} of the main text, with a lattice spacing in the $z$-direction $a = 549$~nm and a radial trap-frequency $\omega_r= 165 \times 2\pi $ rad/s.  Traps near the centre of the chain hold approximately $N = 700$ atoms of Rb-87 with mass $m$ and an s-wave scattering length $a_s = 5.8$ nm. Sites near the ends of the array contain comparatively few sites due to the weak harmonic potential in the $z$-dimension. 

The experimental system was initialized as follows.  After an initial cooling process to produce the bulk condensate, a one-dimensional optical lattice is ramped up so that every site is effectively isolated in a deep well for 9~ms (note that $1 ~\text{ms} \approx  \omega_r^{-1}$). During this time the central site is also depleted via one-body losses provided by a site-selective electron beam. The final atom number on the central site is in the range of $5\%$ to $10\%$ of the full wells. The lattice height is then lowered to the final value over a time-scale of 2~ms, with the dissipation still turned on. 

The dissipation is then turned off and unitary dynamics ensues.  After the system evolves undisturbed for a period of time, electron impact ionization facilitates a measurement of the atom number. Since this process removes the atoms from the trap, only a single measurement can be made in a given experimental run. The entire procedure is then repeated for each time $t$ to obtain a statistically relevant sample. A detailed breakdown of the procedure can be found in Ref.~\cite{LabouvieThesis}.

\section{Details of simulations}
\subsection{Numerical method}
In this section we provide additional details of the c-field (classical field) numerical technique which we employ. For a broad overview of c-field methods see Ref.  \cite{Blakie2008}. The projection operator which defines the classical field region  is prescribed by an energy cutoff within the appropriate single-particle basis associated with Eq.~(\ref{eqn:GPEoperator}). We use a cartesian Hermite-Gauss computational basis, which diagonalizes the single-particle Hamiltonian associated with Eq.~(\ref{eqn:GPEoperator}) (i.e.  $g_2 = 0$); {this allows for the definition of the projector~Eq.~(\ref{eq:Projector}) in a suitable basis, and  provides an efficient computational basis, enabling the simulation of a large number of sites}. We define the Hermite-Gauss functions $\varphi_n(\bm{x})$ of the 2D harmonic oscillator, and the associated quantum numbers $n = n_x + n_y$. The cutoff $n_{\rm cut}$ is hence given via $\mathcal{C} = \{n \; | \; n_x + n_y \leq n_{\rm cut}\}$ and 
\begin{align}
\mathcal{P}\{f(\bm{x})\} = \sum_{n \in \mathcal{C}} \varphi_n(\bm{x}) \int d^2 \bm{x}' \varphi_n^*(\bm{x}') f(\bm{x}'). 
\label{eq:Projector}
\end{align}
Gaussian quadrature allows the matrix transforms to be evaluated exactly~\cite{Blakie2008,Boyd2001}. The number of retained modes (defined by $n_{\rm{cut}}$) is chosen to be sufficiently high that the relevant energy scales are captured, while ensuring that the results are not significantly affected by this choice. Specifically, we use $n_{\rm{cut}} = 3\mu/\hbar \omega_r$, where $\mu$ is the chemical potential of the ground-state.  This is chosen since collisions between excited atoms with energy $\mu$ that have tunnelled on to the depleted site can, occasionally, scatter to energies near $0$ and $2\mu$ respectively. Hence,  $2\mu$ is a potentially important scale. Accessing higher energy modes requires multiple scatterings of energetic atoms. Some of these processes are captured by using a cut-off of  $3\mu$. 

% Figure environment removed 

Figure~\ref{fig:supp_fullns} shows the filling for cut-offs of $n_{\rm{cut}} = \{2\mu/\hbar \omega_r , 3\mu/\hbar \omega_r ,4\mu/\hbar \omega_r \}$. Due to the 2D nature of each site, $4\mu/\hbar \omega_r$  corresponds to approximately $3.5$ times the number of modes as  $2\mu/\hbar \omega_r$. 
The results suggest that the lower cut-off $n_{\rm{cut}} = 2\mu/\hbar \omega_r$ only changes the results slightly from the larger cut-offs, primarily for the case of $J = 0.097 \hbar \omega_r$. We suspect that at low $J$ values the slower dynamics simply allow more time for successive scattering events that can populate the high energy modes, and therefore more modes are required. The results for cut-offs of  $3\mu$ and $4\mu$ are almost identical, suggesting they are sufficiently high to capture the scales of interest. Discrepancies between these results are of the order of the sampling error for the $4\mu$ case (not shown). 



\subsection{Simulation parameters} \label{sec:param}
We now discuss how the model parameters were determined. Using the details provided in the previous section, the 3D interaction strength can be calculated as $g = 4\pi\hbar^2 a_s/m$. Since the model Eq.~(\ref{eq:gpe}) considers the $z$-direction to be discrete due to the strong confining potential, we obtain the 2D interaction strength $g_2 = g \int dz ~ w(z)^4 ,$ where $w(z)$ is the Wannier band in the $z$-direction, which is obtained numerically. The chemical potential is obtained via the Thomas-Fermi approximation: $\mu =\sqrt{g_2 m \omega_r^2 N /\pi}$. 

 We use the specific values of $J$ provided in Ref.~\cite{Labouvie2015}, which can be calculated from first principles by considering the corresponding lattice potential $V_0 \sin^2(k z)$ in the $z$-direction. We determine $V_0$ using a harmonic oscillator approximation in the center of the well: $V_0 = \hbar^2 \omega_z^2/4 E_r$, where $E_r = \hbar^2 k^2/2m$ is the lattice recoil energy, and we define the lattice wavevector as $k= 2 \pi/\lambda$, with lattice spacing $a$ and lattice wavelength $\lambda = 2a$.
 Using the tight-binding approximation, the tunneling can be obtained as $J = \Delta E/4$ where $\Delta E$ is the band-width of the lowest Bloch band \cite{Arzamasovs2017}. We obtain $\Delta E$ by diagonalizing $\hat{H} = - \hslash^2 \partial_z^2/2m + V_0 \sin^2(2 \pi z / \lambda)$ numerically. Since $J$ is in one-to-one correspondence with $\omega_z$, we check that fixing $J$ produces $\omega_z$ values in the range proposed by \cite{Labouvie2015}. We further check that the results do not deviate significantly from the exact expression obtained using Matheiu functions in the `deep lattice' limit $V_0 \gg E_r$  \cite{Zwerger2003}.  The above procedure uniquely determines $\omega_z$, $g_2$, and $\mu$, which are required for simulations. 

\subsection{Initial conditions}
While the experiment Ref.~\cite{Labouvie2015} contained 65 sites, for computational expediency, we consider a somewhat smaller but comparable uniform system of 21 sites. As the total atom number is conserved by Eq.~(\ref{eq:gpe}), the atom number in equilibrium is thus reduced by $\sim5\%$ across each site to refill the central site (cf.~$\sim$1--2\% for the 65-site chain). For the tunneling strengths $J$ relevant to the experiment (determined by lattice height $V_0$), the effective 2D interaction strength is $g_2 \sim 0.2 \hbar\omega_r l_r^2$, and the chemical potential of an isolated site is $\mu \sim 7 \hbar \omega_r$ relative to the bottom of the central site. Note these values vary slightly as $J$ is varied (via $V_0$).

To initialize simulations, we first find the system's zero temperature ground-state by evolving  Eq.~(\ref{eq:gpe}) in imaginary time. We then remove almost all atoms on the central site ($i=0$), such that the initial relative atom number is $\sim5\%$, consistent with the experimentally reported values. 

The removal of the atoms at the central site effectively separates the half-chains on each side into two isolated systems; the relative phase between left and right chains are thus expected to drift due to quantum (or  thermal) fluctuations~\cite{Sinatra2008} while the two subchains are disconnected.  Figure~1 shows how the relative phase  evolves due to quantum fluctuations; for the parameters of the experiment.  The phase difference is shown to grow over a time of $30 \sim 40 ~~ \omega_r^{-1}$, after which point it has reached approximately $\pi/2$, indicating complete randomness. In the experiment, the half-chains are uncoupled for 11 ms.  Our results suggest this is long enough for significant phase randomness to develop between the half-chains. For small values of $J$ the initial part of the filling process is quite slow, and the dephasing should continue even after the dissipation is turned off.  Furthermore, thermal fluctuations will further contribute to the decoherence.  Considering these factors, we conclude that the phase of each half-chain in the experiment may be modelled, to a good degree of approximation, as independent random variables. Thus, for the simulations presented in the main text,  for each trajectory, we therefore multiply the right $R = \{i | i > 0\}$ subset of the chain by random phase $e^{i \theta}$, $\theta \in [0,2\pi]$, to mimic this phase diffusion. 
Finally, half a particle per mode of noise is added to all sites, to capture vacuum fluctuations in accordance with the TWA~\cite{Steel1998,Polkovnikov2003a,Blakie2008}.


 % Figure environment removed








\subsection{Dependence on the initial atom number}
We briefly discuss the dependence of the filling dynamics on the initial population of the central site. We find that the results are insensitive to moderate changes in the initial atom number. This is demonstrated in Fig.~\ref{fig:diffinit}, in which we compare the atom number vs time for the cases in which the initial population is 0\%, 5\% and 10\% of a full well. The 5\% and 10\% cases correspond to the limits of the experimental values reported in \cite{Labouvie2015}. It can be seen that the filling curves follow broadly the same trend, with the 10\% case giving faster filling. This is anticipated from the increased Franck-Condon factor Eq.~(\ref{eqn:FrankCondon}) in this case.


\subsection{Current calculation}
The current $I_t \equiv {dN_0(t)}/{dt}$ for an individual trajectory is calculated directly from Eq.~(4) of the main text at each time $t$.
At a given time, we also calculate $\Delta N(t) = (N_f-N_0(t))/N_f$ for each trajectory, which gives  us the ``current-voltage'' (current-atom number difference) pair $(dN_0(t)/dt,\Delta N(t))$. To calculate the statistics of these quantities as shown in Fig.~\ref{fig:mean_vs_trajectories}(c), the data for every time-step is placed in narrow bins according to the value of $\Delta N(t)$. This process is repeated for every trajectory. The average and fluctuations can then be obtained from the statistics within a given bin.

In contrast, the quantity we call $I_m$ is calculated from the mean (trajectory-averaged) atom number $\langle N_0(t)\rangle$ at each time $t$, i.e. $I_m = d\langle N_0(t)\rangle/dt = [{\langle N_0(t)\rangle - \langle N_0(t-\delta) \rangle}]/{\delta}.$  This is how the current was inferred in Ref. \cite{Labouvie2015}, since only average values are available due to the destructive measurement protocol. 



% Figure environment removed


 
 % Figure environment removed
 \section{Dynamics of condensation}
 In this section we consider the dynamics of the condensate fraction on the central site.  The condensate fraction can be determined in c-field simulations by calculating the single-particle density matrix $\rho_{ij} = \langle \psi_i^*|\psi_j\rangle$, and then diagonalizing it.  The largest eigenvalue is then the condensate number.  The average can either be calculated over trajectories in non-equilibrium scenarios, or as a time-average from a single trajectory in a quasi-equilibrium situation.  See Ref.~\cite{Blakie2008} for further details.
 
 Figure~\ref{fig:cond_things}(a) shows the time evolution of the condensate fraction $f$ for a variety of coupling values, which match to those displayed in Fig.~\ref{fig:relativePhase} of the main text. We observe that for weak couplings ($J = 0.05 \hbar \omega_r$) the condensate fraction drops to  very small values before rising as the filling time is approached. The incoherent behavior at early times explains the observations in Fig.~\ref{fig:relativePhase}(a) of the main text, which demonstrated independence of the filling time on the initial phase difference across the central site. Although outside the scope of the present work, the peak at early times for $J = 0.6 \hbar\omega_r$ is due to the trajectories for which $|\Delta \Phi_0|$ is initially small. The slow rise that occurs after this is due to the trajectories for which $|\Delta \Phi_0|$ is initially larger, 
   such that the neighboring wells are out of phase. This is related to the dark solitons discussed in Ref.~\cite{Ceulemans2023} and will be reported on elsewhere.
 
 
 Figure \ref{fig:cond_things}(b) shows an estimate of the short-time averaged condensate fraction for four individual trajectories in the case of  $J = 0.097 \hbar\omega_r$. The single-particle density matrix
 at a given time is calculated by time-averaging the single-particle density matrix over a rolling time-window of $5~\omega_r^{-1}$ using 20 samples of $\psi_i$.
 We choose the same trajectories as in Fig.~\ref{fig:mean_vs_trajectories}(a) with matching colors for the lines. A comparison of the data against the atom number $N_0/N_f$ for each trajectory (dashed lines) shows that the onset of rapid growth in the atom number is highly correlated with the onset of significant condensation. 
 

% Figure environment removed


\section{Far-from-equilibrium nature of system dynamics}
In this section we demonstrate that in the experiment of Labouvie~\emph{et al.}~\cite{Labouvie2015} the system is in fact far-from-equilibrium during the filling process, and hence cannot be considered to have an effective chemical potential. To do this we perform a numerical experiment in which we initially add an energy shift of $(\mu_R - \hbar\omega_r)$ to the central depleted site.  This means that that the  chemical potential of the undepleted sites is equal to the vacuum energy of the two-dimensional harmonic trap on the central site, and hence it remains unfilled in equilibrium.
 
Beginning from this equilibrium state with the central site empty and the other sites full, we then  decrease the energy shift on the central site to zero linearly in time at a rate $\nu$. This introduces a  potential difference between the central site and the rest of the lattice that results in atoms tunnelling onto the central site and refilling it. When the energy shift is eventually removed, the central site will fill until it  contains the same number of atoms as on the sites of the rest of the lattice. 

The relevant  coupled Gross-Pitaevskii equation for the system site is given by
\begin{align}
i \hbar \partial_t \psi_0 = \mathcal{L}\psi_0 + J (\psi_{-1}  + \psi_{1}) + \Delta V(t) \psi_0 , \label{eq:gperamp}
\end{align}
with the energy shift for $0 \le \nu t \le 1$ being 
\begin{eqnarray}
\Delta V(t)  &=& (1-\nu t)(\mu_R -\hbar \omega_r),\nonumber\\
&=& (\mu_R -\hbar \omega_r) - V_s(t), \label{eq:ramp}
\end{eqnarray}
where $\nu$ is the ramp rate, $V_s(t) = \nu t(\mu_R -\hbar \omega_r)$, and the equation for other sites 
is unchanged
as in Eq.~(\ref{eq:gpe}).  


% Figure environment removed


If this ramp is conducted slowly enough, the 
system will remain in quasi-equilibrium, and the central site will have the same occupation for a given value of $V_s(t)$ regardless of the ramp rate.
Figure~\ref{fig:ramp} shows the atom number $N_0$ vs the energy $V_s(t)$ for different ramp times $1/\nu$, with $J/\hbar\omega_r = 0.05$. For slow ramps, conducted over a time of roughly $20~\omega_r^{-1}$ or more, we observe that all ramps result in essentially the same $N_0$  vs $V_s$ curve, showing that the system is in a quasi-equilibrium state. These scenarios allow for the definition of an on-site chemical potential $\mu_S(t) = V_s(t) + \hbar \omega_r$. Provided the atom number $N_0$ exceeds 200 or so, the Thomas-Fermi prediction of $N_0 \sim (\Delta \mu)^2 $ is a reasonable approximation, as shown by the dotted line in the inset. 

For ramp times shorter than $20~\omega_r^{-1}$, the system no longer follows the same curve and hence the dynamics are non-equilibrium, and a chemical potential for the system cannot be defined.  In the experiment of Labouvie~\emph{et al.}~\cite{Labouvie2015}, the rapid change of the lattice height to initialise the refilling dynamics occurs over a timescale of approximately $2$~ms or $\sim 2 \omega_r^{-1}$ during the preparation period.  The results presented here clearly show that the system is  far-from-equilibrium. 





 \section{Phase Dynamics}

 In Fig.~4 of the main text, we showed the system filling dynamics is dependent on the initial phase difference across the central well, $\Delta \Phi_0$, at high $J$ but not at low $J$. This section provides further discussion on this dependence. Further insight into the role of the phase difference is gained by inspecting the statistics of $\Delta \Phi_0$ across different trajectories; to this end, in Fig.~\ref{fig:ph_time_colorplot}, we show the probability distribution $P(\Delta \Phi_0)$, for small (a) and large (b) coupling strengths $J$.  The top panels (i) show the situation in which the left and right chains are initialized randomly, while the bottom panels (ii)  show the case where the chains are in phase ($\Delta \Phi_0=0 $).




 In Fig.~\ref{fig:phasehist}(a),  (small $J$) [Fig.~\ref{fig:phasehist}(a)], it does not matter whether the relative phase between the left and right chains is initially random ($\Delta \Phi_0 \in [0,2\pi]$ uniformly, top row) or synchronized ($\Delta \Phi_0 =0$, bottom row). The initial phase is quickly forgotten, and rephasing occurs at a similar time in both cases, $t_f \omega_r \sim 320-350$, which is within the sampling error. By contrast, for large $J$ [Fig.~\ref{fig:phasehist}(b)] the initial phase is crucial to the refilling dynamics. For random phases (top) some trajectories gravitate towards $\Delta \Phi = \pm \pi$ and remain there for some time, dramatically slowing the filling. This is related to the formation of dark solitons \cite{Ceulemans2023}, and will be discussed elsewhere. Meanwhile, for synchronized phases (bottom), the filling is comparatively rapid. 


% Figure environment removed

 \section{Finite Temperature Case}
In the main text, we argue that the acceleration of the current seen in trajectories is due to the onset of condensation and phase coherence. To provide further evidence of this interpretation, in this section we provide additional simulations of the system but at finite temperature (i.e., low initial condensate fraction on the filled sites). 



To prepare a finite temperature initial state, we unitarily evolve an arbitrary initial state with $E/N_f = 10 \hbar \omega_r,$ allowing it to thermalize to the equilibrium state with this energy. We ensure that the atom number is comparable to the atom number in the  experiment. The initial condensate fraction on the filled sites for this scenario is less than $5$\%. The results for some individual trajectories are shown in Fig.~\ref{fig:zerotemp}(i), for a variety of $J$ values. For comparison, Fig.~\ref{fig:zerotemp}(ii) shows data for the zero temperature cases considered in the main text. In the finite temperature case, the vast majority of the trajectories do not show an  acceleration in the filling rate during the later stages of the filling dynamics. Rather, in this case the current often slows down over time. This supports our interpretation of this process being driven by condensation and phase coherence. There is some evidence of an NDC characteristic for low $J$ and early times, which would align with the `incoherent' filling interpretation as was put forward in Ref.~\cite{Labouvie2015}. However, this is not a prominent feature compared to some individual trajectories for the zero temperature cases in Fig.~\ref{fig:zerotemp}(ii). Interestingly, a direct comparison of these figures suggests that the fluctuations are actually significantly reduced at finite temperature. This may be due to the lack of Josephson oscillations at high temperatures, as these cause large fluctuations in the coherent case.

Further argument for our interpretation in terms of coherence can be drawn from the spatial overlap integral in Eq.~(\ref{eqn:FrankCondon}) of the main text; this equation shows that the current into site $i$ is  $\propto \sin(\varphi_{ij})$, the relative phase difference between neighbouring sites. When the depleted site has no well-defined phase, this term averages to a value near zero. However, once condensation occurs, the depleted site adopts a definite phase, allowing this term to be $\sim \mathcal{O}(1)$.   This leads to a larger current in Eq.~(\ref{eq:fillrate}), which would appear to explain the late time acceleration in the filling observed in Fig.~\ref{fig:mean_vs_trajectories}(a) of the main text. However, since the overlap is complex this does not rule out Josephson oscillations; it is also necessary that atoms tunneling into the central site can be irreversibly scattered into a large number of modes. This is guaranteed since the condensate fraction, while finite and growing, is still far from unity as shown in Fig.~\ref{fig:cond_things}.
The effective driving frequency is also changing with time, which may cause some damping. 



%\bibliography{References.bib}
\bibliographystyle{apsrev4-2}

%apsrev4-2.bst 2019-01-14 (MD) hand-edited version of apsrev4-1.bst
%Control: key (0)
%Control: author (72) initials jnrlst
%Control: editor formatted (1) identically to author
%Control: production of article title (-1) disabled
%Control: page (0) single
%Control: year (1) truncated
%Control: production of eprint (0) enabled
\begin{thebibliography}{10}%
\makeatletter
\providecommand \@ifxundefined [1]{%
 \@ifx{#1\undefined}
}%
\providecommand \@ifnum [1]{%
 \ifnum #1\expandafter \@firstoftwo
 \else \expandafter \@secondoftwo
 \fi
}%
\providecommand \@ifx [1]{%
 \ifx #1\expandafter \@firstoftwo
 \else \expandafter \@secondoftwo
 \fi
}%
\providecommand \natexlab [1]{#1}%
\providecommand \enquote  [1]{``#1''}%
\providecommand \bibnamefont  [1]{#1}%
\providecommand \bibfnamefont [1]{#1}%
\providecommand \citenamefont [1]{#1}%
\providecommand \href@noop [0]{\@secondoftwo}%
\providecommand \href [0]{\begingroup \@sanitize@url \@href}%
\providecommand \@href[1]{\@@startlink{#1}\@@href}%
\providecommand \@@href[1]{\endgroup#1\@@endlink}%
\providecommand \@sanitize@url [0]{\catcode `\\12\catcode `\$12\catcode
  `\&12\catcode `\#12\catcode `\^12\catcode `\_12\catcode `\%12\relax}%
\providecommand \@@startlink[1]{}%
\providecommand \@@endlink[0]{}%
\providecommand \url  [0]{\begingroup\@sanitize@url \@url }%
\providecommand \@url [1]{\endgroup\@href {#1}{\urlprefix }}%
\providecommand \urlprefix  [0]{URL }%
\providecommand \Eprint [0]{\href }%
\providecommand \doibase [0]{https://doi.org/}%
\providecommand \selectlanguage [0]{\@gobble}%
\providecommand \bibinfo  [0]{\@secondoftwo}%
\providecommand \bibfield  [0]{\@secondoftwo}%
\providecommand \translation [1]{[#1]}%
\providecommand \BibitemOpen [0]{}%
\providecommand \bibitemStop [0]{}%
\providecommand \bibitemNoStop [0]{.\EOS\space}%
\providecommand \EOS [0]{\spacefactor3000\relax}%
\providecommand \BibitemShut  [1]{\csname bibitem#1\endcsname}%
\let\auto@bib@innerbib\@empty
%</preamble>
\bibitem [{\citenamefont {Labouvie}\ \emph {et~al.}(2015)\citenamefont
  {Labouvie}, \citenamefont {Santra}, \citenamefont {Heun}, \citenamefont
  {Wimberger},\ and\ \citenamefont {Ott}}]{Labouvie2015}%
  \BibitemOpen
  \bibfield  {author} {\bibinfo {author} {\bibfnamefont {R.}~\bibnamefont
  {Labouvie}}, \bibinfo {author} {\bibfnamefont {B.}~\bibnamefont {Santra}},
  \bibinfo {author} {\bibfnamefont {S.}~\bibnamefont {Heun}}, \bibinfo {author}
  {\bibfnamefont {S.}~\bibnamefont {Wimberger}},\ and\ \bibinfo {author}
  {\bibfnamefont {H.}~\bibnamefont {Ott}},\ }\href
  {https://journals.aps.org/prl/abstract/10.1103/PhysRevLett.115.050601}
  {\bibfield  {journal} {\bibinfo  {journal} {Phys. Rev. Lett.}\ }\textbf
  {\bibinfo {volume} {115}},\ \bibinfo {pages} {050601} (\bibinfo {year}
  {2015})}\BibitemShut {NoStop}%
\bibitem [{\citenamefont {Labouvie}(2015)}]{LabouvieThesis}%
  \BibitemOpen
  \bibfield  {author} {\bibinfo {author} {\bibfnamefont {R.}~\bibnamefont
  {Labouvie}},\ }\href@noop {} {\bibfield  {journal} {\bibinfo  {journal} {PhD
  Thesis, \textit{Non-equilibrium dynamics in ultracold quantum gases with
  localized dissipation}, Technischen Universit\"{a}t Kaiserslautern}\ }
  (\bibinfo {year} {2015})}\BibitemShut {NoStop}%
\bibitem [{\citenamefont {Blakie}\ \emph {et~al.}(2008)\citenamefont {Blakie},
  \citenamefont {Bradley}, \citenamefont {Davis}, \citenamefont {Ballagh},\
  and\ \citenamefont {Gardiner}}]{Blakie2008}%
  \BibitemOpen
  \bibfield  {author} {\bibinfo {author} {\bibfnamefont {P.~B.}\ \bibnamefont
  {Blakie}}, \bibinfo {author} {\bibfnamefont {A.~S.}\ \bibnamefont {Bradley}},
  \bibinfo {author} {\bibfnamefont {M.~J.}\ \bibnamefont {Davis}}, \bibinfo
  {author} {\bibfnamefont {R.~J.}\ \bibnamefont {Ballagh}},\ and\ \bibinfo
  {author} {\bibfnamefont {C.~W.}\ \bibnamefont {Gardiner}},\ }\href
  {https://www.tandfonline.com/doi/full/10.1080/00018730802564254?casa_token=bHbRxWesdhEAAAAA%3AcAKKqK6F1YQQdqqnCXZlKqtMITGU65aOGRg0HgBVnmtCadskeNNkBOK5L7v9YMyh0Z50sAeFTu-Rqg}
  {\bibfield  {journal} {\bibinfo  {journal} {Adv. Phys}\ }\textbf {\bibinfo
  {volume} {57}},\ \bibinfo {pages} {363} (\bibinfo {year} {2008})}\BibitemShut
  {NoStop}%
\bibitem [{\citenamefont {Boyd}(2001)}]{Boyd2001}%
  \BibitemOpen
  \bibfield  {author} {\bibinfo {author} {\bibfnamefont {J.~P.}\ \bibnamefont
  {Boyd}},\ }\href {https://link.springer.com/book/9783540514879} {\emph
  {\bibinfo {title} {Chebyshev and Fourier spectral methods}}}\ (\bibinfo
  {publisher} {Dover Publications},\ \bibinfo {year} {2001})\BibitemShut
  {NoStop}%
\bibitem [{\citenamefont {Arzamasovs}\ and\ \citenamefont
  {Liu}(2017)}]{Arzamasovs2017}%
  \BibitemOpen
  \bibfield  {author} {\bibinfo {author} {\bibfnamefont {M.}~\bibnamefont
  {Arzamasovs}}\ and\ \bibinfo {author} {\bibfnamefont {B.}~\bibnamefont
  {Liu}},\ }\href
  {https://iopscience.iop.org/article/10.1088/1361-6404/aa8d2c/meta?casa_token=j7a4GYhfYj4AAAAA:xkEkN8d_4jSBuiKHykPrs23DpgCZGGil3o4YyN_WfAU9eWdKxYpWwqh52kDgbZq199UXX5gSbMg}
  {\bibfield  {journal} {\bibinfo  {journal} {Eur. J. Phys.}\ }\textbf
  {\bibinfo {volume} {38}},\ \bibinfo {pages} {065405} (\bibinfo {year}
  {2017})}\BibitemShut {NoStop}%
\bibitem [{\citenamefont {Zwerger}(2003)}]{Zwerger2003}%
  \BibitemOpen
  \bibfield  {author} {\bibinfo {author} {\bibfnamefont {W.}~\bibnamefont
  {Zwerger}},\ }\href
  {https://iopscience.iop.org/article/10.1088/1464-4266/5/2/352/meta?casa_token=B-7T-aTveLgAAAAA:XROBCEGbM9d_0gvnhACZc_3BD1XEex6FwpalNWejAqe38ub8vbU2mdCuE7XHpP45nTmaVqG_e8I}
  {\bibfield  {journal} {\bibinfo  {journal} {J. Opt. B}\ }\textbf {\bibinfo
  {volume} {5}},\ \bibinfo {pages} {S9} (\bibinfo {year} {2003})}\BibitemShut
  {NoStop}%
\bibitem [{\citenamefont {Sinatra}\ and\ \citenamefont
  {Castin}(2008)}]{Sinatra2008}%
  \BibitemOpen
  \bibfield  {author} {\bibinfo {author} {\bibfnamefont {A.}~\bibnamefont
  {Sinatra}}\ and\ \bibinfo {author} {\bibfnamefont {Y.}~\bibnamefont
  {Castin}},\ }\href {https://doi.org/10.1103/PhysRevA.78.053615} {\bibfield
  {journal} {\bibinfo  {journal} {Phys. Rev. A}\ }\textbf {\bibinfo {volume}
  {78}},\ \bibinfo {pages} {053615} (\bibinfo {year} {2008})}\BibitemShut
  {NoStop}%
\bibitem [{\citenamefont {Steel}\ \emph {et~al.}(1998)\citenamefont {Steel},
  \citenamefont {Olsen}, \citenamefont {Plimak}, \citenamefont {Drummond},
  \citenamefont {Tan}, \citenamefont {Collett}, \citenamefont {Walls},\ and\
  \citenamefont {Graham}}]{Steel1998}%
  \BibitemOpen
  \bibfield  {author} {\bibinfo {author} {\bibfnamefont {M.~J.}\ \bibnamefont
  {Steel}}, \bibinfo {author} {\bibfnamefont {M.~K.}\ \bibnamefont {Olsen}},
  \bibinfo {author} {\bibfnamefont {L.~I.}\ \bibnamefont {Plimak}}, \bibinfo
  {author} {\bibfnamefont {P.~D.}\ \bibnamefont {Drummond}}, \bibinfo {author}
  {\bibfnamefont {S.~M.}\ \bibnamefont {Tan}}, \bibinfo {author} {\bibfnamefont
  {M.~J.}\ \bibnamefont {Collett}}, \bibinfo {author} {\bibfnamefont {D.~F.}\
  \bibnamefont {Walls}},\ and\ \bibinfo {author} {\bibfnamefont
  {R.}~\bibnamefont {Graham}},\ }\href
  {https://journals.aps.org/pra/abstract/10.1103/PhysRevA.58.4824} {\bibfield
  {journal} {\bibinfo  {journal} {Phys. Rev. A}\ }\textbf {\bibinfo {volume}
  {58}},\ \bibinfo {pages} {4824} (\bibinfo {year} {1998})}\BibitemShut
  {NoStop}%
\bibitem [{\citenamefont {Polkovnikov}(2003)}]{Polkovnikov2003a}%
  \BibitemOpen
  \bibfield  {author} {\bibinfo {author} {\bibfnamefont {A.}~\bibnamefont
  {Polkovnikov}},\ }\href
  {https://journals.aps.org/pra/abstract/10.1103/PhysRevA.68.053604} {\bibfield
   {journal} {\bibinfo  {journal} {Phys. Rev. A}\ }\textbf {\bibinfo {volume}
  {68}},\ \bibinfo {pages} {053604} (\bibinfo {year} {2003})}\BibitemShut
  {NoStop}%
\bibitem [{\citenamefont {Ceulemans}\ and\ \citenamefont
  {Wouters}(2023)}]{Ceulemans2023}%
  \BibitemOpen
  \bibfield  {author} {\bibinfo {author} {\bibfnamefont {R.}~\bibnamefont
  {Ceulemans}}\ and\ \bibinfo {author} {\bibfnamefont {M.}~\bibnamefont
  {Wouters}},\ }\href
  {https://journals.aps.org/pra/abstract/10.1103/PhysRevA.108.013314}
  {\bibfield  {journal} {\bibinfo  {journal} {Phys. Rev. A}\ }\textbf {\bibinfo
  {volume} {108}},\ \bibinfo {pages} {013314} (\bibinfo {year}
  {2023})}\BibitemShut {NoStop}%
\end{thebibliography}%


\clearpage








\end{document}
