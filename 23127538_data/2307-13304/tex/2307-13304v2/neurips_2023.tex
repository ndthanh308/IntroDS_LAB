\documentclass{article}


% if you need to pass options to natbib, use, e.g.:
%     \PassOptionsToPackage{numbers, compress}{natbib}
% before loading neurips_2023
\PassOptionsToPackage{numbers,compress}{natbib}

% ready for submission
% \usepackage{neurips_2023}


% to compile a preprint version, e.g., for submission to arXiv, add add the
% [preprint] option:
% \usepackage[preprint]{neurips_2023}


% to compile a camera-ready version, add the [final] option, e.g.:
\usepackage[final]{neurips_2023}


% to avoid loading the natbib package, add option nonatbib:
%    \usepackage[nonatbib]{neurips_2023}

\usepackage{amsmath,amssymb,amsthm}

\usepackage[utf8]{inputenc} % allow utf-8 input
\usepackage[T1]{fontenc}    % use 8-bit T1 fonts
\usepackage{hyperref}       % hyperlinks
\usepackage{url}            % simple URL typesetting
% \usepackage{booktabs}       % professional-quality tables
\usepackage{amsfonts}       % blackboard math symbols
\usepackage{nicefrac}       % compact symbols for 1/2, etc.
\usepackage{microtype}      % microtypography
\usepackage{xcolor}         % colors
\usepackage{algorithm}
\usepackage{algpseudocode}

\newtheorem{theorem}{Theorem}
\newtheorem{corollary}{Corollary}[theorem]
\newtheorem{assumption}{Assumption}
\newtheorem{lemma}[theorem]{Lemma}
\newtheorem{definition}{Definition}


% User defined packages
\usepackage{todonotes} %comments [disable] to hide todos
\usepackage{soul} %strikethrough text
%\usepackage[bibliography=common]{apxproof} %proofs automatically in appendix
\usepackage[bibliography=separate]{apxproof} %proofs automatically in appendix
%\newcommand{\customapxsubsectiontitle}[2]{\begin{toappendix}\subsection*{Proofs for Subsection~[#1]~([#2])}\end{toappendix}}
\renewcommand{\appendixbibliographystyle}{plainnat}
\newtheoremrep{theorem}{Theorem}
\newtheoremrep{corollary}{Corollary}[theorem]
\newtheoremrep{assumption}{Assumption}
\newtheoremrep{lemma}[theorem]{Lemma}
\newtheoremrep{definition}{Definition}
\usepackage{wrapfig}
\usepackage{multirow}
\newcommand\Tstrut{\rule{0pt}{2.6ex}}         % = `top' strut
\newcommand\Bstrut{\rule[-0.9ex]{0pt}{0pt}}   % = `bottom' strut
\usepackage{listings} % code 
\usepackage{adjustbox}

% User defined commands
\newcommand{\round}{\mathcal{Q}}
\newcommand{\alglin}{\mathcal{A}}
\newcommand{\Lworst}{\mathcal{L}_{\operatorname{worst}}}
\newcommand{\Lavg}{\mathcal{L}_{\operatorname{avg}}}
\newcommand{\eig}[1]{\operatorname{eig}(#1)}
\newcommand{\diag}[1]{\operatorname{diag}(#1)}
\newcommand{\ldl}{\mathsf{LDLQ}}
\newcommand{\optq}{\mathsf{OPTQ}}
\newcommand{\near}{\mathsf{Near}}
\newcommand{\stoch}{\mathsf{Stoch}}
\newcommand{\quip}{\mathsf{QuIP}}
\newcommand{\optquip}{\mathsf{OPTQuIP}}

\newcommand{\norm}[1]{\left\|#1\right\|}
\newcommand{\Abs}[1]{\left|#1\right|}
\newcommand{\Exv}[2][]{\mathbf{E}_{#1}\left[#2\right]}
\newcommand{\Var}[1]{\operatorname{Var}\left(#1\right)}
\newcommand{\trace}[1]{\operatorname{tr}\left(#1\right)}
\newcommand{\R}{\mathbb{R}}


\title{QuIP: 2-Bit Quantization of \\ Large Language Models With Guarantees}


% The \author macro works with any number of authors. There are two commands
% used to separate the names and addresses of multiple authors: \And and \AND.
%
% Using \And between authors leaves it to LaTeX to determine where to break the
% lines. Using \AND forces a line break at that point. So, if LaTeX puts 3 of 4
% authors names on the first line, and the last on the second line, try using
% \AND instead of \And before the third author name.


\author{
  Jerry Chee \\
  % Department of Computer Science \\
  Cornell University \\
  \texttt{jerrychee@cs.cornell.edu} \\
  \And
  Yaohui Cai \\
  % Department of  Electrical and\\  Computer Engineering \\
  Cornell University \\
  \texttt{yc2632@cornell.edu} \\
  \AND
  Volodymyr Kuleshov \\
  % Department of Computer Science \\
  Cornell University \\
  \texttt{kuleshov@cornell.edu} \\
  \And
  Christopher    De Sa \\
  % Department of Computer Science \\
  Cornell University \\
  \texttt{cdesa@cs.cornell.edu} \\
  % David S.~Hippocampus\thanks{Use footnote for providing further information
  %   about author (webpage, alternative address)---\emph{not} for acknowledging
  %   funding agencies.} \\
  % Department of Computer Science\\
  % Cranberry-Lemon University\\
  % Pittsburgh, PA 15213 \\
  % \texttt{hippo@cs.cranberry-lemon.edu} \\
  % examples of more authors
  % \And
  % Coauthor \\
  % Affiliation \\
  % Address \\
  % \texttt{email} \\
  % \AND
  % Coauthor \\
  % Affiliation \\
  % Address \\
  % \texttt{email} \\
  % \And
  % Coauthor \\
  % Affiliation \\
  % Address \\
  % \texttt{email} \\
  % \And
  % Coauthor \\
  % Affiliation \\
  % Address \\
  % \texttt{email} \\
}



\begin{document}


\maketitle

\begin{abstract}
This work studies post-training parameter quantization in large language models (LLMs). 
We introduce quantization with incoherence processing (QuIP), a new method based on the insight that quantization benefits from {\em incoherent} weight and Hessian matrices, i.e., 
from the weights being even in magnitude
and the directions in which it is important to round them accurately being unaligned with the coordinate axes.
QuIP consists of two steps: 
(1)~an adaptive rounding procedure minimizing a quadratic proxy objective;
(2)~efficient pre- and post-processing that ensures weight and Hessian incoherence via multiplication by random orthogonal matrices.
%
We complement QuIP with the first theoretical analysis for an LLM-scale quantization algorithm, and show that our theory also applies to an existing method, OPTQ.
%
Empirically, we find that our incoherence preprocessing improves several existing quantization algorithms and yields the first LLM quantization methods that produce viable results using only two bits per weight.
Our code can be found at \url{https://github.com/Cornell-RelaxML/QuIP}.
\end{abstract}








%% Supplement (Put before proofs) ===========================================================
\begin{toappendix}
%\setcounter{section}{3} % for constitent numbering with main paper submission
\newpage
\appendix

\section{Proof of Lemma \ref{lemma, equivalence of two def of MDDO}}
\begin{proof}
For any ${\bs{\beta}}\in\mc H$, according to the definition of $G_{\bs s}$ (see Definition $\ref{def: MDDO}$), one has
\begin{align*}
\langle G_{\bs s},{\bs{\beta}}\rangle&=\int_{[0,1]} G_{\bs s}(t){\bs{\beta}}(t)~\mathrm{d}t=\int_{[0,1]}\mathrm{cov}\hspace{-0.9mm}\left(\bs{X}(t),\mathrm{e}^{\mi\langle \bs s,\Y\rangle}\right){\bs{\beta}}(t)~\mathrm{d}t\\
&=\int_{[0,1]}\mathrm{cov}\hspace{-0.9mm}\left(\bs{X}(t){\bs{\beta}}(t),\mathrm{e}^{\mi \langle \bs s,\Y\rangle}\right)~\mathrm{d}t.
\end{align*}
By Fubini theorem, under Assumption $\ref{as:joint distribution assumption}$, one can exchange the order of integration and covariance above and get that
\begin{align*}
 \langle G_{\bs s},{\bs{\beta}}\rangle&=\int_{[0,1]}\mathrm{cov}\hspace{-0.9mm}\left(\bs{X}(t){\bs{\beta}}(t),\mathrm{e}^{\mi \langle \bs s,\Y\rangle}\right)~\mathrm{d}t\\ &=\mathrm{cov}\hspace{-0.9mm}\left(\int_{[0,1]}\bs{X}(t){\bs{\beta}}(t)~\mathrm{d}t,\mathrm{e}^{\mi \langle \bs s,\Y\rangle}\right)=\mathrm{cov}\hspace{-0.9mm}\left(\langle \bs{X},{\bs{\beta}}\rangle,\mathrm{e}^{\mi \langle \bs s ,\Y\rangle}\right).
\end{align*}
Thus for any $\bs\alpha(t),{\bs{\beta}}(t)\in\mc H$, one can get
\begin{align*}
\big\langle \big(G_{\bs s}\otimes \overline{G}_{\bs s}\big)\bs\alpha,{\bs{\beta}}\big\rangle=\langle G_{\bs s},\bs\alpha\rangle\langle \overline{G}_{\bs s},{\bs{\beta}}\rangle=\mathrm{cov}\hspace{-0.9mm}\left(\langle \bs{X},\bs\alpha\rangle,\mathrm{e}^{\mi \langle \bs s,\Y\rangle}\right)\hspace{-0.9mm}\mathrm{cov}\hspace{-0.9mm}\left(\langle \bs{X},{\bs{\beta}}\rangle,\mathrm{e}^{-\mi\langle \bs s,\Y\rangle}\right)\\
=\mb{E}\hspace{-0.9mm}\left(\langle \bs{X},\bs\alpha\rangle\mathrm{e}^{\mi \langle \bs s,\Y\rangle}\right)\mb{E}\hspace{-0.8mm}\left(\langle \bs{X},{\bs{\beta}}\rangle\mathrm{e}^{-\mi \langle \bs s,\Y\rangle}\right)=\mb{E}\Big(\langle \bs{X},\bs\alpha\rangle\langle \bs{X}',{\bs{\beta}}\rangle\mathrm{e}^{\mi \langle \bs s,\Y-\Y'\rangle}\Big).
\end{align*}
Considering that $\mb{E}\big(\langle \bs{X},\alpha\rangle\langle \bs{X}',{\bs{\beta}}\rangle\big)=0$, one has
\begin{align*}
\big\langle \big(G_{\bs s}\otimes \overline{G}_{\bs s}\big)\bs\alpha,{\bs{\beta}}\big\rangle
=- \mb{E}\Big(\langle \bs{X},\bs\alpha\rangle\langle \bs{X}',{\bs{\beta}}\rangle\big(1-\mr{e}^{\mi \langle \bs s,\Y-\Y'\rangle}\big)\Big)&\\
=- \mb{E}\Big(\langle \bs{X},\bs\alpha\rangle\langle \bs{X}',{\bs{\beta}}\rangle\big[1-\cos\big(\langle \bs s,\Y-\Y'\rangle\big)\big]\Big)&\\
+\mi\mb{E}\Big(\langle \bs{X},\bs\alpha\rangle\langle \bs{X}',{\bs{\beta}}\rangle\big[\sin\big(\langle\bs s,\Y-\Y'\rangle\big)\big]\Big)&.
\end{align*}
It is easy to check that
\[\int_{\mb R^q}\frac{\sin \big(\langle\bs s,\Y-\Y'\rangle)\big)}{\|\bs s\|^{1+q}}~\mr{d}\bs s=\lim_{\varepsilon\to0^+}\int_{\bs s\in\mb{R}^q:\varepsilon\leqslant\|\bs s\|\leqslant \varepsilon^{-1}}\frac{\sin \big(\langle \bs s,\Y-\Y'\rangle\big)}{\|\bs s\|^{1+q}}~\mr{d}\bs s=0,\]
because the integrand is an odd function. By Lemma 1 in \cite{szekely2007measuring},  one can also get
\[\int_{\R^q}\frac{1-\cos\big(\langle \bs s,\Y-\Y'\rangle\big)}{\|\bs s\|^{1+q}}~\mr{d}\bs s=c_q\|\Y-\Y'\|.
\]
Combining above results with Definition $\ref{def: MDDO}$, one can obtain that 
\begin{align}\label{proof: lemma MDDO}
\langle\mathrm{MDDO}(\bs{X}|Y)\bs\alpha,{\bs{\beta}}\rangle=- \mb{E}\Big(\langle \bs{X},\bs\alpha\rangle\langle \bs{X}',{\bs{\beta}}\rangle\|\Y-\Y'\|\Big) .
\end{align}
Then by the arbitrariness of $\bs\alpha,{\bs{\beta}}\in\mc H$, the proof is completed. 
\end{proof}

\section{Proof of Theorem \ref{theorem, MDDO and conditional mean independence}}



According to \eqref{proof: lemma MDDO}, one can get the following useful lemma.
\begin{lemma}\label{lemma, MDDO and FMDD}
Under Assumption $\ref{as:joint distribution assumption}$, for all ${\bs{\beta}}\in\mathcal H$, $\|{\bs{\beta}}\|=1$, we have
\begin{align*}
\langle \mathrm{MDDO}(\boldsymbol{X}|\Y)({\bs{\beta}}),{\bs{\beta}}\rangle &=- \mathbb E\Big[ \langle \boldsymbol{X},{\bs{\beta}}\rangle \langle \boldsymbol{X}',{\bs{\beta}}\rangle \|\Y-\Y'\|\Big]\\
&=- \mathbb E\Big[\big\langle\langle \boldsymbol{X},{\bs{\beta}}\rangle{\bs{\beta}},\langle \boldsymbol{X}',{\bs{\beta}}\rangle{\bs{\beta}}\big\rangle\|\Y-\Y'\|\Big].
\end{align*}
\end{lemma}
This conclusion links MDDO with functional martingale
difference divergence  (FMDD, \citealt{lee2020testing}). 
Next we give the following two lemmas to finish the proof of Theorem $\ref{theorem, MDDO and conditional mean independence}$.
\begin{lemma}\label{lem: Txx=0tuiTx=0}If $T$ is a positive semi-definite operator on a Hilbert space $\wt{\mathcal{H}}$, then for all $x\in\wt{\mathcal{H}}$, one has $\langle Tx,x\rangle=0\Longleftrightarrow Tx=0$.
\end{lemma}
\begin{proof}
`$\Longleftarrow$': It is obvious.

`$\Longrightarrow$': It is easy to check that $f(a,b)=\langle Ta,b\rangle$ $(a,b\in\wt{\mc H})$ is a 
positive semi-definite Hermitian form. Thus, for any $y\in\wt{\mathcal{H}}$, one can use Cauchy inequality to get
\[|\langle Tx,y\rangle|^2\leqslant\langle Tx,x\rangle\langle Ty,y\rangle=0\Longrightarrow \langle Tx,y\rangle=0.\]
By the arbitrariness of $y\in\wt{\mc H}$, one has $Tx=0$.
\end{proof}

Our proof of Theorem $\ref{theorem, MDDO and conditional mean independence}$ is mainly inspired by the following property of
FMDD in \cite{lee2020testing}.
\begin{lemma}[Proposition 1 of \cite{lee2020testing}]\label{lem:prop1inlee}
If $\E[\|\X\|+\|\Y\|]<\infty$ and $\E[\|\bs X\|\|\Y\|]<\infty$, then we have
\[\E[\langle \X,\X'\rangle\|\Y-\Y'\|]=0\Longleftrightarrow \E[\X|\Y]=0\quad\text{almost surely},\]
where $(\X',\Y')$ is an i.i.d. copy of $(\X,\Y)$.
\end{lemma}
\paragraph{Proof of Theorem $\ref{theorem, MDDO and conditional mean independence}$}
\begin{proof}
Clearly, (ii) is a direct consequence of Lemma $\ref{lemma, equivalence of two def of MDDO}$ and the following lemma.

\begin{lemma}[Lemma 15 in \citealt{chen2023optimality}]\label{lem:cov TX}
If $T$ is an operator defined on $\mc H_1\to\mc H_2$ where $\mc H_i,i=1,2$ is a Hilbert space. $\bs X\in\mc H_1$ is a random element satisfying $\mb E[\bs X]=0$ . Then we have $\mr{var}(T\bs X)=T\mr{var}(\bs X)T^*$.
\end{lemma}

Now we start  to prove (i).
 First, one has
\begin{align*}\mathrm{MDDO}(\boldsymbol{X}|\Y)=0 &\Longleftrightarrow \mathrm{MDDO}(\boldsymbol{X}|\Y)({\bs{\beta}})=0,\quad\forall{\bs{\beta}}\in\mb{S}_{\mathcal H};\\
\mathbb E[\boldsymbol{X}|\Y]=0~~\text{a.s.}&\Longleftrightarrow\langle\mb E[\boldsymbol{X}|\Y],{\bs{\beta}}\rangle{\bs{\beta}}=0~~\text{a.s.} \quad\forall{\bs{\beta}}\in\mb{S}_{\mathcal H},
\end{align*}
where $\mb{S}_\mc{H}=\{{\bs{\beta}}\in\mc H:\|{\bs{\beta}}\|=1\}$. Second, from Lemma $\ref{lem: Txx=0tuiTx=0}$, one knows that
\begin{align*}
\mathrm{MDDO}(\boldsymbol{X}|\Y)({\bs{\beta}})=0&\Longleftrightarrow\langle\mathrm{MDDO}(\boldsymbol{X}|\Y)({\bs{\beta}}),{\bs{\beta}}\rangle=0.
\end{align*}
 Then under Assumption $\ref{as:joint distribution assumption}$, by Lemma $\ref{lemma, MDDO and FMDD}$ and $\ref{lem:prop1inlee}$, one has
\begin{align*}
&\langle\mathrm{MDDO}(\boldsymbol{X}|\Y)({\bs{\beta}}),{\bs{\beta}}\rangle=0\Longleftrightarrow\mathbb E[\big\langle\langle \boldsymbol{X},{\bs{\beta}}\rangle{\bs{\beta}},\langle \boldsymbol{X}',{\bs{\beta}}\rangle{\bs{\beta}}\rangle\|\Y-\Y'\|]=0\\
&\qquad\qquad\qquad\qquad\qquad\Longleftrightarrow\mathbb E[\langle \bs X,{\bs{\beta}}\rangle{\bs{\beta}}|\Y]=\langle \mb E[\boldsymbol{X}|\Y],{\bs{\beta}}\rangle{\bs{\beta}}=0~~\text{a.s.}
\end{align*}
This finishes the proof of Theorem $\ref{theorem, MDDO and conditional mean independence}$.
\end{proof}

% {\color{blue}\paragraph{Proof of Lemma \ref{lem:cov TX} (Repeated)}
% \begin{proof}
% For any $\u_1,\u_2\in\mc H_2$, we have
% \begin{align*}
% &\left\langle  T\mr{var}(\vX)T^*\u_1,\u_2  \right\rangle=\left\langle  T\mb E[\vX\otimes\vX]T^*\u_1,\u_2  \right\rangle
% =\left\langle  \mb E[\vX\otimes\vX]T^*\u_1,T^*\u_2  \right\rangle    
% \end{align*}
% since $\mb E[\vX]=0$. By the definition of convariance operator and expectation, we have 
% \begin{align*}
% \left\langle  \mb E[\vX\otimes\vX]T^*\u_1,T^*\u_2  \right\rangle=&\left\langle  \mb E[\left\langle\vX,  T^*\u_1 \right\rangle       \vX            ],T^*\u_2  \right\rangle
% =\mb E[  \left\langle\vX,  T^*\u_1 \right\rangle      \left\langle \vX            ,T^*\u_2  \right\rangle].
% \end{align*}
% Similarly, we have
% \begin{align*}
%  \left\langle  \mr{var}(T\vX)\u_1,\u_2  \right\rangle=\left\langle  \mb E[T\vX\otimes T\vX]\u_1,\u_2  \right\rangle=\mb E[  \left\langle T\vX,  \u_1 \right\rangle      \left\langle T\vX            ,\u_2  \right\rangle].\\    
% \end{align*}
% Then the proof is completed by noticing the following
% \begin{align*}
% \mb E[  \left\langle T\vX,  \u_1 \right\rangle      \left\langle T\vX            ,\u_2  \right\rangle]=\mb E[  \left\langle\vX,  T^*\u_1 \right\rangle      \left\langle \vX            ,T^*\u_2  \right\rangle].
% \end{align*}
% \end{proof}}



\section{Proof of Lemma \ref{lemma: SE=GammaS}}
Recall the following fact in FSIR.
\begin{lemma}\label{lemma, direct result of linearity condition}~\\
Under Assumption $\ref{as:Linearity condition and Coverage condition}~ \boldsymbol{\mathrm{{i)}}}$, we have $\mathcal S_{\mathbb E(\boldsymbol{X}|\Y)}\subseteq \Gamma \mc S_{\Y|\bs X}\subseteq \mc H$.
\end{lemma}
It is a trivial generalization of    \cite[Theorem 2.1]{ferre2003functional} from univariate response to multivariate response.
\paragraph{Proof of Lemma $\ref{lemma: SE=GammaS}$}
\begin{proof}
First, we prove that $\mathcal{S}_{\mathbb{E}(\bs X|\Y)}^\perp\subseteq \mathrm{Im}\{\mathrm{var(\mb{E}(\bs X|\Y))}\}^\perp$. For any ${\bs{\beta}}\in\mathcal{S}_{\mathbb{E}(\bs X|\Y)}^\perp$, one has $\langle{\bs{\beta}},\mb{E}(\bs X|\Y)\rangle=0$ a.s. Then for any $\bs\alpha\in\mathcal{H}$, one can get
\begin{align*}
\langle{\bs{\beta}},\mathrm{var}(\mb{E}(\bs X|\Y))\bs\alpha\rangle&=\langle{\bs{\beta}},\mb E\lmi\mb{E}(\bs X|\Y)\otimes \mb{E}(\bs X|\Y)\rmi\bs\alpha\rangle\\
&=\mb E\big(\langle\mb{E}(\bs X|\Y),\bs\alpha\rangle\langle{\bs{\beta}},\mb{E}(\bs X|\Y)\rangle\big)=0,
\end{align*}
which means that ${\bs{\beta}}\in\mathrm{Im}\{\mathrm{var}(\mb{E}(\bs X|\Y))\}^\perp$. Moreover, one has
\begin{align*}\mathcal{S}_{\mathbb{E}(\bs X|\Y)}^\perp\subseteq \mathrm{Im}\{\mathrm{var}(\mb{E}(\bs X|\Y))\}^\perp
%&\Rightarrow\left(\mathcal{S}_{\mathbb{E}(\bs X|Y)}^\perp\right)^\perp\supseteq \left(\mathrm{Im}\{\mathrm{var(\mb{E}(\bs X|Y))}\}^\perp\right)^\perp\\&
\Longrightarrow\overline{\mathcal{S}_{\mathbb{E}(\bs X|\Y)}}\supseteq\overline{\mathrm{Im}}\{\mathrm{var}(\mb{E}(\bs X|\Y))\}.
\end{align*}
Thus, $\overline{\mathrm{Im}}\{\mathrm{var}(\mb{E}(\bs X|\Y))\}\subseteq\overline{\mathcal{S}_{\mathbb{E}(\bs X|\Y)}}\subseteq\overline{\Gamma\mathcal{S}_{\Y|\bs X}}$ by Lemma $\ref{lemma, direct result of linearity condition}$. According to Assumption $\ref{as:Linearity condition and Coverage condition}$ \textbf{ii)}, one can get
\[\mathrm{dim}\left(\overline{\mathrm{Im}}\{\mathrm{var}(\mb{E}(\bs X|\Y))\}\right)=\mathrm{dim}\left(\overline{\mathcal{S}_{\mathbb{E}(\bs X|\Y)}}\right)=\mathrm{dim}(\overline{\Gamma\mathcal{S}_{\Y|\bs X}})=d.\]
One can complete the proof since finite dimension subspaces are closed.
\end{proof}

\section{Proof of Theorem \ref{theorem, MDDO and IRS}}
\begin{proof}
For convenience, we abbreviate $\mathrm{MDDO}(\boldsymbol{X}|\Y)$ to ${M}$. According to Theorem $\ref{theorem, MDDO and conditional mean independence}$ and Lemma $\ref{lem: Txx=0tuiTx=0}$, one can get
\begin{align*}{\bs{\beta}}\in\mathcal S_{\mb E(\boldsymbol{X}|\Y)}^\perp&\Longleftrightarrow\langle {\bs{\beta}},\mathbb E(\boldsymbol{X}|\Y)\rangle=0~~\text{a.s.}\Longleftrightarrow\mathbb E(\langle {\bs{\beta}},\boldsymbol{X}\rangle|\Y)=0~~\text{a.s.}\\
&\Longleftrightarrow\mathrm{MDDO}(\langle {\bs{\beta}},\boldsymbol{X}\rangle|\Y)=0\Longleftrightarrow\langle {M}{\bs{\beta}},{\bs{\beta}}\rangle=0\\
&\Longleftrightarrow{M}{\bs{\beta}}=0\Longleftrightarrow {\bs{\beta}}\in\mathrm{null}(M)=\overline{\mathrm{Im}}(M)^\perp,
\end{align*}
which means that $\mathcal S_{\mb E(\boldsymbol{X}|\Y)}^\perp=\overline{\mathrm{Im}}(M)^\perp$ and $\overline{\mathcal S_{\mb E(\boldsymbol{X}|\Y)}}=\overline{\mathrm{Im}}(M)$.
One can complete the proof since finite dimension subspaces are closed.
\end{proof}
\section{Proof of Lemma \ref{lemma, way of estimate truncate central subspace}}
Before proving Lemma $\ref{lemma, way of estimate truncate central subspace}$, we give the following lemma.
\begin{lemma}\label{lem: colPBP equal colPB operator}
Assume that $P$ is a bounded linear operator from a Hilbert space $\wt{\mc H}$ to itself and $B$ is a positive semi-definite operator from $\wt{\mc H}$ to itself. 
Then we have $\overline{\mathrm{Im}}(PBP^*)=\overline{\mathrm{Im}}(PB)$.
\end{lemma}
\begin{proof}
It suffices to show that $\mnull(BP^*)=\mnull(PBP^*)$. First, since $B$ is positive semi-definite, one has $\langle x,PBP^*x\rangle = \langle P^*x,BP^*x \rangle\geqslant 0~(\forall x\in\wt{\mc H})$. Thus $PBP^*$ is a positive semi-definite operator on $\wt{\H}$.
For any $y\in\wt{\H}$, we have 
\begin{align*}
PBP^*y=0\overset{(a)}{\Longleftrightarrow}\langle y,PBP^*y\rangle = \langle P^*y,BP^*y \rangle=0\overset{(b)}{\Longleftrightarrow} BP^*y=0. 
\end{align*}
where $(a)$ and $(b)$ come from Lemma $\ref{lem: Txx=0tuiTx=0}$.
Thus $\mnull(PBP^*)=\mnull(BP^*)$.
\end{proof}

\paragraph{Proof of Lemma $\ref{lemma, way of estimate truncate central subspace}$}
\begin{proof}
For convenience, we abbreviate $\mathrm{MDDO}(\boldsymbol{X}|\Y)$ and $\mathrm{MDDO}(\boldsymbol{X}^{(m)}|\Y)$ to ${M}$ and $M_m$ respectively. 

By Corollary $\ref{corollary, MDDO and central subspace}$, one can get $\Gamma\mathcal{S}_{\Y|\boldsymbol{X}}=\mathrm{Im}(M)$. Thus,
\begin{align}\label{eq: corollary, MDDO and central subspace}
\Pi_m\Gamma\mathcal{S}_{\Y|\boldsymbol{X}}=\Pi_m\mathrm{Im}(M)=\mathrm{Im}(\Pi_m M).
\end{align}
It is easy to check that
\begin{align}
\Gamma_m&:=\mathrm{var}(\bs X^{(m)})=\Pi_m\Gamma\Pi_m=\Pi_m\Gamma=\Gamma\Pi_m=\sum\limits_{i=1}^m\lambda_i\phi_i\otimes\phi_i.\label{eq: Gamma m def}
\end{align}
On the one hand, by the definition of $\mathcal{S}^{(m)}_{{\Y|\boldsymbol{X}}}$ and $\Gamma_m$ (see \eqref{def: truncated central subspace} and \eqref{eq: Gamma m def}), one can get
\begin{align}\label{eq:Pim Gamma S}
\Pi_m\Gamma\mathcal{S}_{\Y|\boldsymbol{X}}&=\Pi_m\Gamma\Pi_m\mathcal{S}_{\Y|\boldsymbol{X}}=(\Pi_m\Gamma)(\Pi_m\mathcal{S}_{\Y|\boldsymbol{X}})=\Gamma_m\mathcal{S}^{(m)}_{{\Y|\boldsymbol{X}}}.
\end{align}
On the other hand, one has $\overline{\mathrm{Im}}(\Pi_m M)=\overline{\mathrm{Im}}(\Pi_m M\Pi_m)$ by Lemma $\ref{lem: colPBP equal colPB operator}$. Since $\Pi_m M$ and $\Pi_m M\Pi_m$ are both of finite rank, one can further get
\begin{align*}
\mathrm{Im}(\Pi_m M)&=\overline{\mathrm{Im}}(\Pi_m M)=\overline{\mathrm{Im}}(\Pi_m M\Pi_m)=\mathrm{Im}(\Pi_mM\Pi_m).
\end{align*}
Then according to Theorem $\ref{theorem, MDDO and conditional mean independence}$(ii), one has
\begin{align}
\mathrm{Im}(\Pi_m M)=\mathrm{Im}(\Pi_mM\Pi_m)=\mathrm{Im}(M_m).\label{eq:Pim span M}
\end{align}
Combining \eqref{eq:Pim Gamma S}, \eqref{eq:Pim span M} with \eqref{eq: corollary, MDDO and central subspace}, one has $\Gamma_m\mathcal{S}^{(m)}_{{\Y|\boldsymbol{X}}}=\mathrm{Im}\{M_m\}$.
Finally, one can get $ \Gamma_m^\dagger\mathrm{Im}\{M_m\}=\Gamma_m^\dagger\Gamma_m\mathcal{S}^{(m)}_{{\Y|\boldsymbol{X}}}=\Pi_m\mathcal{S}^{(m)}_{{\Y|\boldsymbol{X}}}=\mathcal{S}^{(m)}_{{\Y|\boldsymbol{X}}}$.
\end{proof}

\section{Wely Inequality for a Self-adjoint and Compact Operator}\label{ap:Wely inequality for self-adjoint and compact operators}
First, we show the following three results in standard functional analysis textbook.
\begin{lemma}[Spectral theorem]\label{thm: Spectral theorem}Let $\wt{\mathcal{H}}$ be a Hilbert space and $A:\wt{\mc{H}}\to\wt{\mc{H}}$ be a compact, self-adjoint operator. There is an at most countable orthonormal basis $\{\wt e_j\}_{j\in J}$ ($J=\{1,\cdots,n\}$ or $\mathbb{Z}_{\geqslant1}$) of $\wt{\mathcal{H}}$ and eigenvalues $\{\wt\lambda_j\}_{j\in J}$ with $|\wt\lambda_1|\geqslant|\wt\lambda_2|\geqslant\cdots\geqslant0$ converging to zero, such that
\begin{align*}
x=\sum_{j\in J}\langle x,\wt e_j\rangle \wt e_j;\qquad Ax=\sum_{j\in J}\wt\lambda_j\langle x,\wt e_j\rangle \wt e_j,\qquad x \in\wt{\mathcal{H}}.
\end{align*}
\end{lemma}

\begin{lemma}[Rayleigh's principle]\label{lem:Rayleigh operator}Let $A$ be a compact, self-adjoint operator. If $\{\wt e_j\}_{j\in J}$ and $\{\wt\lambda_j\}_{j\in J}$ are eigenvectors and eigenvalues define in Lemma $\ref{thm: Spectral theorem}$ respectively. Then
\[|\wt\lambda_1|=\mathop{\sup\limits_{\|u\|=1}}|\langle Au,u\rangle|;\qquad|\wt\lambda_n|=\mathop{\sup\limits_{\|u\|=1}}_{u\in\{\wt e_1,\cdots,\wt e_{n-1}\}^\perp}|\langle Au,u\rangle|~(n\geqslant 2).\]
\end{lemma}
\begin{lemma}[Minimax theorem]\label{lem:minimax operator}
Assume that $A$ is a positive semi-definite and compact operator with its eigenvalues $\{\wt\lambda_i\}$ ordered as $\wt\lambda_1\geqslant\dots\geqslant \wt\lambda_n\geqslant\dots\geqslant 0$, then
$$
\wt\lambda_n=\inf_{E_{n-1}}\sup_{x\in E_{n-1}^\perp,\|x\|=1}\langle Ax,x\rangle
$$
where $E_{n-1}$ with dimension $n-1$ is a closed linear subspace of $\wt{\mc H}$.
\end{lemma}
Then we give the Wely inequality for a self-adjoint and compact operator.
\begin{proposition}\label{prop: wely operator}
Let $M=N+R$ where $M$, $N$ and $R$ are three self-adjoint and compact operators defined on a Hilbert space $\wt{\mc H}$. Also, $M$ and $N$ are positive semi-definite with their respective eigenvalues $\{\mu_i\},\{\nu_i\}$ ordered as follows
\begin{align*}
M:\mu_1\geqslant\dots\geqslant \mu_n\geqslant\dots\geqslant 0;\qquad
N:\nu_1\geqslant\dots\geqslant \nu_n\geqslant\dots\geqslant 0,
\end{align*}
while $R$'s eigenvalues are $\{\rho_i\}$ ordered as follows:
\[R:|\rho_1|\geqslant\dots\geqslant |\rho_n|\geqslant\dots\geqslant 0.\]
Then the following inequalities hold: $|\mu_k-\nu_k|\leqslant|\rho_1|=\|R\| $, $k\geqslant1$.
\end{proposition}
\begin{proof}
From Lemma $\ref{lem:minimax operator}$, we have:
\[\mu_n=\inf_{E_{n-1}}\sup_{x\in E_{n-1}^\perp,\|x\|=1}\langle Mx,x\rangle;\qquad\nu_n=\inf_{E_{n-1}}\sup_{x\in E_{n-1}^\perp,\|x\|=1}\langle Nx,x\rangle,\]
where $E_{n-1}$ with dimension $n-1$ is a closed linear subspace of $\wt{\mc H}$.
By Lemma $\ref{lem:Rayleigh operator}$, we have:
$$
\sup_{\|u\|=1}|\langle Ru,u\rangle|=|\rho_1|.
$$
Since $\langle Mu,u\rangle=\langle Nu,u\rangle+\langle Ru,u\rangle$, for any $\|u\|=1$, we have:
$$
\langle Nu,u\rangle-|\rho_1|\leqslant\langle Mu,u\rangle \leqslant \langle Nu,u\rangle+|\rho_1|.
$$
Then for any given $n-1$ dimensional closed linear subspace of $\wt{\mc H}$, we conclude
\begin{equation}\label{eq:max ineq}
\sup_{u\in E_{n-1}^\perp,\|u\|=1}\langle Nu,u\rangle-|\rho_1|\leqslant\sup_{u\in E_{n-1}^\perp,\|u\|=1}\langle Mu,u\rangle\leqslant \sup_{u\in E_{n-1}^\perp,\|u\|=1}\langle Nu,u\rangle+|\rho_1|.
\end{equation}
Take the infimum with respective to $E_{n-1}$ in \eqref{eq:max ineq}, we have
\[\nu_n-|\rho_1|\leqslant\mu_n\leqslant \nu_n+|\rho_1|\]
by Lemma $\ref{lem:minimax operator}$.
\end{proof}
The next result is a direct corollary of Proposition $\ref{prop: wely operator}$.
\begin{corollary}\label{coro:wely ineq operator}
Let $M$ and $N$ be two self-adjoint, positive semi-definite and compact operators defined on a Hilbert space $\wt{\mc H}$ with their respective eigenvalues $\{\mu_i\},\{\nu_i\}$ ordered as follows
\begin{align*}
M:\mu_1\geqslant\dots\geqslant \mu_n\geqslant\dots\geqslant 0\quad\text{and}\quad
N:\nu_1\geqslant\dots\geqslant \nu_n\geqslant\dots\geqslant 0.
\end{align*}
Then the following inequalities hold: $|\mu_k-\nu_k|\leqslant\|M-N\| $, $ k\geqslant1$.
\end{corollary}




\section{Proof of Proposition \ref{prop:bound hatMmd Mm}}
Before proving Proposition $\ref{prop:bound hatMmd Mm}$, we give the following conclusion, whose proof is deferred to the end of this section.
\begin{proposition}\label{proposition, concentration of MDDO}
Under Assumptions $\ref{as:joint distribution assumption}$ and $\ref{assumption: sub-Gaussian}$, for all $\gamma\in(0,1/2)$, there exist positive constants $D_0=D_0(\gamma,\sigma_0,\sigma_1)$, $D_1=D_1(\sigma_1)$, $D_2=D_2(\sigma_0,\sigma_1)$ and $n_0=n_0(\gamma,\sigma_0,\sigma_1)$ such that for all $n\geqslant n_0$ and
\[C\in \l D_0n^{\frac{2\gamma}{5}}-\ln\l D_1m^2n \r,D_2 n^{\frac{1}{5}}-\ln\l D_1m^2n \r \rmi,\]
we have
\begin{equation*}
\mathbb{P}\l\left\|\wh M_m- M_m\right\| <\l \frac{C+\ln( D_1m^2n)}{D_2}\r^{\frac52}\frac{12m}{\sqrt n}\r\geqslant 1-\exp(- C).
\end{equation*}
\end{proposition}
\paragraph{Proof of Proposition $\ref{prop:bound hatMmd Mm}$}
\begin{proof}




Using Corollary $\ref{coro:wely ineq operator}$, one can get
$
\lambda_i\l\wh M_m\r\leqslant \lno\wh M_m-M_m\rno +\lambda_i\l M_m\r
$. 
Since $\rank(M_m)=d$, one can get $\lambda_i(M_m)=0,~i\geqslant d+1$. Thus by Proposition $\ref{proposition, concentration of MDDO}$, one has
\begin{align}\label{eq:lambdai hat Mm upper bound}
\mathbb{P}\l\lambda_{d+1}(\wh M_m)<\l \frac{C+\ln\l D_1m^2n\r}{D_2}\r^{\frac52}\frac{12m}{\sqrt n}\r\geqslant 1-\exp(- C)\qquad(i\geq d+1). 
\end{align}
Notice that 
\begin{align*}\lno\wh M_m^d- M_m\rno &\leqslant\lno M_m-\wh M_m\rno +\lno\wh M_m-\wh M_m^d\rno ;\\
\lno\wh M_m-\wh M_m^d\rno &=\left\|\sum_{i=d+1}^\infty\wh\mu_i\wh\gamma_i\otimes \wh\gamma_i\right\| =\widehat{\lambda}_{d+1}=\lambda_{d+1}(\widehat{M}_m)
\end{align*}
by \eqref{wh M_m spectral decomposition}.
Then combing Proposition $\ref{proposition, concentration of MDDO}$ with \eqref{eq:lambdai hat Mm upper bound} can complete the proof.
\end{proof}


\paragraph{Proof of Proposition \ref{proposition, concentration of MDDO}}
\begin{proof}
Note that $\boldsymbol{X}^{(m)}=\sum\limits_{j=1}^m\langle \boldsymbol{X},\phi_j\rangle\phi_j$, then a simple calculation leads to
\begin{align*}
M_m&=-\sum_{i,j=1}^m\mathbb E\big[\langle \boldsymbol{X},\phi_i\rangle\langle \boldsymbol{X}',\phi_{j}\rangle\|\Y-\Y'\|\big]\phi_i\otimes\phi_j;\\
\widehat{M}_m&=-\sum_{i,j=1}^m\frac1{n^2}\sum_{k,\ell=1}^n\langle \boldsymbol{X}_k,\phi_i\rangle\langle \boldsymbol{X}_\ell,\phi_j\rangle\|\Y_k-\Y_\ell\|\phi_i\otimes\phi_j.
\end{align*}

For a operator $\Gamma'$ that can be expanded as $\Gamma':=\sum\limits_{i,j=1}^\infty a_{ij}\phi_i\otimes\phi_{j}$, let us define its maximal norm as $\|\Gamma'\|_{\mathrm{max}}=\sup\limits_{i,j}|a_{ij}|$.



\begin{lemma}\cite[Theorem 1]{mai2021slicing}\label{lemma, concentration of MDDOnm}
Under Assumptions $\ref{as:joint distribution assumption}$ and $\ref{assumption: sub-Gaussian}$, for all
$\gamma\in(0,1/2)$, there exist positive
constants $C_0=C_0(\gamma,\sigma_0,\sigma_1)$, $C_1=C_1(\sigma_1)$, $C_2 = C_2(\sigma_0;\sigma_1)$ and $n_0 = n_0(\gamma,\sigma_0,\sigma_1)$
such that for all $n\geqslant n_0$ and $\varepsilon\in(C_0 n^{-(1/2-\gamma)},1]$, we have
\begin{equation*}
\mathbb{P}\l\lno \widehat{M}_m-M_m\rno_{\max}>12\varepsilon\r\leqslant C_1 m^2n\exp\l- C_2\l\varepsilon^2 n\r^{1/5}\r.
\end{equation*}
\end{lemma}
\noindent Since $\lno\widehat{M}_m-M_m\rno \leqslant m\lno\widehat{M}_m-M_m\rno_{\mathrm{max}}$, one has
\begin{equation*}
\mathbb{P}\l\lno\widehat{M}_m-M_m\rno >12m\varepsilon\r\leqslant C_1 m^2n\exp\l-C_2\l\varepsilon^2 n\r^{1/5}\r.
\end{equation*}
Let $C=C_2\l\ve^2n\r^{1/5}-\ln\l C_1m^2n\r$ satisfying 
\begin{align*}
C\in\l C_2C_0^{2/5}n^{2\gamma/5}-\ln\l C_1m^2n\r,C_2n^{1/5}-\ln\l C_1m^2n\r\rmi,
\end{align*}
then one has
\begin{equation*}
\mathbb{P}\l\lno\widehat{M}_m-M_m\rno \leqslant\l \frac{C+\ln\l C_1m^2n\r}{C_2}\r^{\frac52}\frac{12m}{\sqrt{n}}\r>1- \exp(- C).
\end{equation*}
Then in order to complete the proof, one only need to choose $D_0$, $D_1$ and $D_2$ to be $C_2C_0^{2/5}$, $C_1$ and $C_2$ respectively. 
\end{proof}





\section{Properties of Sub-Gaussian Random Vectors}
We first review the definition of sub-Gaussian random variables.
\begin{definition}[Sub-Gaussian random variable and its upper-exponentially bounded constant]\label{def:sub gaussian variable}
A random variable $X$ is called a sub-Gaussian random variable if $X$ satisfies one of the following equivalent properties:
\begin{itemize}
 \item[1).] Tails. $\P(|X|>t)\leqslant \exp(1-t^{2}/K^{2}_{1})$ for any $t>0$;
 \item[2).] Moments. $\E[|X|^{p}]^{1/p}\leqslant K_{2}\sqrt{p}$ for any $p\geqslant 1$;
 \item[3).]Super-exponential moment: $\E[\exp(X^{2}/K^{2}_{3})]\leqslant \mr{e}$.

\noindent Moreover, if $\E[X]=0$, then the properties $1)-3)$ are also equivalent to the following one:
\item[4).] Moment generating function: $\E[\exp(tX)]\leqslant \exp(t^{2}K^{2}_{4})$ for all $t\in\R$.
\end{itemize}
Here $K_1$, $K_2$, $K_3$ and $K_4$ are four constants.
$K$ is called an upper-exponentially bounded constant of $X$ if 
$K\geqslant \max\{K_{1},K_{2},K_{3},K_{4}\}$.
\end{definition}
\begin{definition}[Sub-Gaussian random vector and its upper-exponentially bounded constant]\label{def,sub-Gaussian random vector,upper-exponentially bounded constant}
 ${X}\in\R^m$ is called a sub-Gaussian random vector if for all $x\in\R^m$, one-dimensional marginal $\langle{X},x\rangle$ is sub-Gaussian random variable. $K$ is called an upper-exponentially bounded constant of $X$ if $K$ satisfies:
 \begin{align*}
K\geqslant \sup_{x\in\mb{S}^{m-1}}K(\langle X,x\rangle) 
 \end{align*}
 where $K(\langle X,x\rangle)$ denotes an upper-exponentially bounded constant of $\langle X,x\rangle$.
Moreover, $K$ is called a uniform (about $m$) upper-exponentially bounded constant of $X$ if $K$ satisfies:
 \begin{align*}
K\geqslant \sup_m\sup_{x\in\mb{S}^{m-1}}K\l \langle X,x\rangle\r.
 \end{align*}
Furthermore, $X$ is called a uniform (about $m$) sun-Gaussian random vector.
 \end{definition}
The following is an application of sub-Gaussian random vectors.
\begin{lemma}[\citealt{vershynin2010introduction}]\label{lem:esgrm}
 Let $\M=[\bs m_1~\cdots~\bs m_n]$ be an $m\times n$ matrix ($n>m$) whose columns $\m_{i}$ are 
 independent centered sub-Gaussian random vectors with 
 covariance matrix $\mathbf{I}_{m}$. Let $\sigma^{+}_{\min}(\M)$ and $\sigma_{\max}(\M)$ be the infimum and supremum of positive singular values of $\M$ respectively. Then, for any $t>0$, with probability at least $1-2\exp(- C^{\prime}t^{2})$, we have
 \begin{equation*}
 \sqrt{n}-C_0\sqrt{m}-t\leqslant \sigma^{+}_{\min}(\M)\leqslant \sigma_{\max}(\M)\leqslant \sqrt{n}+C_0\sqrt{m}+t
 \end{equation*}
 where $C'$ and $C_0$ are two positive constants depending only on $K(\bs m_1)$:
 the upper-exponentially bounded constant of $\bs m_1$.
\end{lemma}
\noindent Let $t=\sqrt m$, then one can easily get
\begin{align}\label{equation, min max eval}
\begin{split}
\lambda_{\max}\left(\frac1n \M\M^\top\right)\leqslant \left(1+\frac{(C_0+1)\sqrt m}{\sqrt n}\right)^2;\\
\lambda_{\min}^+\left(\frac1n \M\M^\top\right)\geqslant \left(1-\frac{(C_0+1)\sqrt m}{\sqrt n}\right)^2, 
\end{split}
\end{align}
with probability at least $1-2\exp(- C'm)$ where $\lambda^{+}_{\min}(\cdot)$ and $\lambda_{\max}(\cdot)$ stands for the infimum and supremum of the positive spectrum respectively.



\begin{lemma}\label{lemma, estiamtion error of inverse sample cov}
Assume that $\x_1,\x_2,...,\x_n$ are $n$ i.i.d. samples from an $m$-dimensional centered sub-Gaussian vector with an invertible covariance matrix $\Sigma$. Let $\wh\Sigma:=\frac1n\sum_i \x_i\x_i^\top$.
Then there exists a positive constant $n_1'=n_1'(K(\bs m_1),c_1)$ ($c_1$ is defined in \eqref{eq: m n relationship}), such that when $n\geqslant n_1'$, we have
\begin{align*}
\lno\wh{\Sigma}-\Sigma\rno\hspace{-1.5mm}&\leqslant (C_0+2)^2\lambda_{\max}(\Sigma)\sqrt{\frac mn}~~\text{and}~~ \lno\wh{\Sigma}^{-1}-\Sigma^{-1}\rno\hspace{-1.5mm}\leqslant \frac{4(C_0+2)^2}{\lambda_{\min}(\Sigma)}\sqrt{\frac mn},
 \end{align*}
 with probability at least $1-2\exp(- C'm)$, where $C_0$ is defined in Lemma $\ref{lem:esgrm}$.
\end{lemma}
\begin{proof}
Let $\x_i=\Sigma^{\frac12}\m_i$ and $\bs{M}=[\bs m_1~\cdots~\bs m_n]$ where $\m_i$ is a centered sub-Gaussian random vector with covariance $\mathbf I_m$. Then one has 
\begin{align*}
\lno\wh\Sigma-\Sigma\rno&\leqslant\lno\Sigma^{\frac12}\rno\cdot\left\|\frac1n \M\M^\top-\mathbf I\right\|\cdot\lno\Sigma^{\frac12}\rno\\
&= \lambda_{\max}(\Sigma)\cdot\left[\lambda_{\max}\left(\frac1n \M\M^\top\right)-1\right]
\end{align*}
and 
\begin{align*}
\lno\wh{\Sigma}^{- 1}-\Sigma^{- 1}\rno
&\leqslant \lno\Sigma^{-\frac12}\rno\cdot\left\|\frac1n \M\M^\top-\mathbf I\right\|\cdot\lno\l\frac1n \M\M^\top\r^{-1}\rno\cdot\lno\Sigma^{-\frac12}\rno\\
&=\frac{1}{\lambda_{\min}(\Sigma)}\left[\lambda_{\max}\left(\frac1n \M\M^\top\right)-1\right]\cdot\lambda_{\min}\left(\frac1n \M\M^\top\right)^{-1}.
\end{align*}
By \eqref{equation, min max eval}, it is easy to check that
\begin{align*}&\lambda_{\max}\left(\frac1n \M\M^\top\right)-1\leqslant\left(1+\frac{(C_0+1)\sqrt m}{\sqrt n}\right)^2-1\leqslant\frac{(C_0+2)^2\sqrt m}{\sqrt n};\\
&\lambda_{\min}\left(\frac1n \M\M^\top\right)\geqslant \left(1-\frac{(C_0+1)\sqrt m}{\sqrt n}\right)^2\geqslant \frac14~\text{for}~n\geqslant [2(C_0+1)]^{\frac2{1-c_1}},
\end{align*}
with probability at least $1-2\exp(- C'm)$. Thus the proof is completed by choosing $n_1'(C_0,c_1):=[2(C_0+1)]^{\frac{2}{1-c_1}}$. 
\end{proof}

\section{Proof of Proposition \ref{prop:concentration Gammam dag Mmd}}\label{ap:concentration inequality}
We first give the following lemma whose proof is deferred to the end of this section.
\begin{lemma}\label{lem:PimTPimtoT}If $T$ is of finite rank, then we have $\lim\limits_{m\to \infty}\|\Pi_m T\Pi_m-T\| =0$.
\end{lemma}
A direct corollary of this lemma is as follows.
\begin{corollary}\label{lemma, M go to Mm}
%For any $\varepsilon>0$, one has $\|M-M_m\| <\varepsilon$ when $m$ is sufficiently large.
Under Assumptions $\ref{as:joint distribution assumption}$ and $\ref{as:Linearity condition and Coverage condition}$, we have $\lim\limits_{m\to\infty}\|M-M_m\| =0$.
\end{corollary}
\noindent We denote by $m_M(\varepsilon)$ the minimal integer $m_M$ satisfying $\|M-M_m\| \leqslant \varepsilon$ for all $m\geqslant m_M$.

Proposition $\ref{prop:concentration Gammam dag Mmd}$ is a direct corollary of the following Proposition.
\begin{proposition}
\label{prop:bound of finite estimate}
 Suppose that Assumptions $\ref{as:joint distribution assumption}$ to $\ref{assumption: rate-type condition}$ hold, then $\forall \gamma\in(0,1/2)$, there exist positive constants
 \begin{align*}
 n_1=n_1(\gamma,\sigma_0,\sigma_1,\bs K,m_M(1),c_1),\quad D_3=D_3(\|M\| ,\wt C,\bs K) 
 \end{align*}
and $C'=C'(\bs K)$
, such that when $n\geqslant n_1$, we have
\begin{equation*}
\begin{aligned}
\mb P\l \lno\widehat\Gamma_m^\dagger \widehat M_m^d-\Gamma_m^\dagger M_m\rno  \leqslant \left[\frac{C+\ln(D_1m^2n)}{D_2}\right]^{\frac52}\frac{24m^{\alpha_1+1}}{\wt C\sqrt n}+D_3\frac{m^{(2\alpha_1+1)/2}}{n^{1/2}} \r&\\
\geqslant 1-\exp(- C)-2\exp(- C'm).&
\end{aligned}
\end{equation*}
Here $D_1,D_2$ and $C$ are defined in Proposition $\ref{prop:bound hatMmd Mm}$ and $\bs K$ is the uniform upper-exponentially bounded constant of $(\sqrt{\lambda_1}w_1,\dots,\sqrt{\lambda_m}w_m)$. 
\end{proposition}
\begin{proof}
By triangle inequality, one has
\begin{align*}
&\lno\widehat{\Gamma}_m^\dagger \widehat M_m^d-\Gamma_m^\dagger M_m\rno 
=\lno\widehat\Gamma_m^\dagger \widehat M_m^d-\wh\Gamma_m^\dagger M_m+\wh\Gamma_m^\dagger M_m-\Gamma_m^\dagger M_m\rno 
\\&\qquad\leqslant\lno\Gamma_m^\dagger\rno \cdot \lno\widehat M_m^d-M_m\rno +\lno\widehat\Gamma_m^\dagger-\Gamma_m^\dagger\rno \cdot \lno M_m\rno .
\end{align*}
Thus one can bound $\lno\Gamma_m^{\dag}M_m-\widehat\Gamma_m^{\dag}\widehat M_m^d\rno $ by bound $\lno\Gamma_m^\dagger\rno $, $\lno\widehat\Gamma_m^\dagger-\Gamma_m^\dagger\rno $, $\lno\widehat M_m^d-M_m\rno $ and $\lno M_m\rno $ respectively.
\begin{itemize}
 \item\textbf{Bound of $\lno\Gamma_m^\dagger\rno $}: By Assumption $\ref{assumption: rate-type condition}$, one has 
\begin{align}\label{eq:bound Gammam dagger}
\lambda_j\geqslant \wt C j^{-\alpha_1}\Rightarrow\lno\Gamma_m^\dagger\rno =\lambda_m^{-1}\leqslant \wt{C}^{-1} m^{\alpha_1}. 
\end{align} 
 \item\textbf{Bound of $\lno\widehat\Gamma_m^\dagger-\Gamma_m^\dagger\rno $}:
 Let us define $\mc H_m:=\mathrm{span}\{\phi_1,\dots,\phi_m\}$ where $\{\phi_i\}$ is introduced in Equation $\eqref{eq:X expansion}$. It is easy to check that
$\lno\widehat\Gamma_m^\dagger-\Gamma_m^\dagger\rno =\lno(\widehat\Gamma_m^\dagger-\Gamma_m^\dagger)|_{\mc H_m}\rno $ since $\l\widehat\Gamma_m^\dagger-\Gamma_m^\dagger\r{\bs{\beta}}=0$ for any ${\bs{\beta}}\in\mc{H}_m^\perp$. 
Because $\l\widehat\Gamma_m^\dagger-\Gamma_m^\dagger\r|_{\mc H_m}$ can be represented by matrix $\widehat{\Sigma}^{-1}-\Sigma^{-1}$ defined in Lemma $\ref{lemma, estiamtion error of inverse sample cov}$ under orthonormal basis $\{\phi_i\}_{i=1}^m$, one can get $\lno\widehat\Gamma_m^\dagger-\Gamma_m^\dagger\rno =\|\widehat{\Sigma}^{-1}-\Sigma^{-1}\|$.
Similarly, one can also get $\lno\Gamma_m^\dagger\rno =\lno\Sigma^{-1}\rno=\lambda_{\min}^{-1}(\Sigma)$. Thus, by Lemma $\ref{lemma, estiamtion error of inverse sample cov}$ one has
\[\mb P\l\lno\widehat\Gamma_m^\dagger-\Gamma_m^\dagger\rno \leqslant {4(C_0+2)^2}\lno\Gamma^{\dag}_m\rno \sqrt{\frac mn}\r\geqslant 1-2\exp(- C'm)\]
for sufficiently large $n\geqslant n_1'(\bs K,c_1)$
. Combing with $\lno\Gamma_m^\dagger\rno \hspace{-1mm}\leqslant \wt{C}^{-1} m^{\alpha_1}$, one can get
\begin{equation}\label{eq: distance hat gamma m dagger hat gamma m dagger}
\mb P\l\lno\widehat\Gamma_m^\dagger-\Gamma_m^\dagger\rno \leqslant \frac{4(C_0+2)^2m^{(2\alpha_1+1)/2}}{\wt Cn^{1/2}}\r\geqslant 1-2\exp(- C'm)
\end{equation}
for sufficiently large $n\geqslant n_1'(\bs K,c_1)$.
 \item\textbf{Bound of $\lno\widehat M_m^d-M_m\rno $}:
 See Proposition $\ref{prop:bound hatMmd Mm}$.
 \item \textbf{Bound of $\lno M_m\rno $}: By Corollary $\ref{lemma, M go to Mm}$, $\|M-M_m\| \leqslant 1$ for sufficiently large $m\geqslant m_M(1)$. Then by triangle inequality, one can get
\[\|M_m\| -\|M\| \leqslant \|M-M_m\| \leqslant 1.\]
Hence,
\begin{align}\label{eq:Mm leq M C}
\|M_m\| \leqslant \|M\| +1.
\end{align}
\end{itemize}
Combing \eqref{eq:bound Gammam dagger}, \eqref{eq: distance hat gamma m dagger hat gamma m dagger}, Proposition $\ref{prop:bound hatMmd Mm}$ with \eqref{eq:Mm leq M C}, one can choose $D_3$ and $n_1$ to be $\frac{4(C_0+2)^2(\|M\| +1)}{\wt C}$ and $\max\{n_0,n_1'(\bs K,c_1),m_M(1)^{1/c_1}\}$ respectively to
complete the proof where $n_0$ is defined in Proposition $\ref{prop:bound hatMmd Mm}$.
\end{proof}

\paragraph{Proof of Lemma \ref{lem:PimTPimtoT}}
\begin{proof}By the triangle inequality and compatibility of operator norm, one has
\begin{align*}
\|\Pi_m T\Pi_m-T\| &\leqslant\|\Pi_mT\Pi_m-\Pi_mT\| +\|\Pi_mT-T\| \\
&\leqslant\|(\Pi_m-I)T^*\| +\|(\Pi_m-I)T\| 
\end{align*}
where $I=\sum\limits_{i=1}^\infty\phi_i\otimes\phi_i$ for $\{\phi_i\}_{i\in\mb{Z}_{\geqslant 1}}$ defined in \eqref{eq:X expansion} being an orthonormal basis of $\mc H$. 
% Since the adjoint of $M(\Pi_m-I)$ is $(\Pi_m-I)M$, we have
% \begin{align*}&\|M(\Pi_m-I)\| +\|(\Pi_m-I)M\| \\
% =&
% \end{align*}

Since $T$ is of finite rank, let us assume that $\{e_i\}_{i=1}^k$ is an orthonormal basis of $\mathrm{Im}(T)$ where $k=\mr{rank}(T)$. For any ${\bs{\beta}}\in\mathcal{H}$ such that $\|{\bs{\beta}}\|=1$, one has $\|T{\bs{\beta}}\|\leqslant\|T\| \|{\bs{\beta}}\|=\|T\| $, so one can assume that $T{\bs{\beta}}\in\mathrm{Im}(T)$ admits the following expansion under basis $\{e_i\}_{i=1}^k$:
\[T{\bs{\beta}}=\sum_{i=1}^k b_ie_i,\quad \sum_{i=1}^k b^2_i\leqslant\|T\| ^2<\infty.\]
Thus
\[\|(I-\Pi_m)T{\bs{\beta}}\|=\left\|\sum_{i=1}^k(I-\Pi_m) b_ie_i\right\|\leqslant\sum_{i=1}^k |b_i|\cdot\|(I-\Pi_m) e_i\|.\]
Clearly, $\|(\Pi_m-I)\alpha\|~(\forall\alpha\in\H)$ tends to $0$ as $m\to\infty$ since 
\[(I-\Pi_m)\alpha=\left(\sum_{i={m+1}}^\infty\phi_i\otimes\phi_i\right)\left(\sum\limits_{i=1}^\infty c_i\phi_i\right)=\sum_{i=m+1}^\infty c_i\phi_i\xrightarrow{m\to\infty} 0\]
where we have assumed that $\alpha=\sum\limits_{i=1}^\infty c_i\phi_i$ .

Thus $\forall\varepsilon>0$, there exists some $N_i>0$ such that $\forall m> N_i$ one has $\|(\Pi_m-I)e_i\|<\varepsilon$, $(\forall i=1,...,k)$. Let $N=\max\{N_1,\cdots,N_k\}$, then $\forall m>N$ one has
\[\|(I-\Pi_m)T{\bs{\beta}}\|\leqslant\sum_{i=1}^k |b_i|\cdot\|(I-\Pi_m) e_i\|\leqslant\sum_{i=1}^k |b_i|\varepsilon\leqslant k\varepsilon\|T\| ,\]
which means that $\forall m>N$, one has
\begin{align*}
\|(\Pi_m-I)T\| &=\sup_{\|{\bs{\beta}}\|=1}\|(\Pi_m-I)T{\bs{\beta}}\|\leqslant k\varepsilon\|T\| . 
\end{align*}
Thus $\lim\limits_{m\to\infty}\|(\Pi_m-I)T\| =0$. 

Similarly, one can also get $\lim\limits_{m\to\infty}\|(\Pi_m-I)T^*\| =0$. Then the proof of Lemma $\ref{lem:PimTPimtoT}$ is completed.
\end{proof}
% \section{Sin Theta Theorem}\label{ap:Sin Theta theorem}
% \subsection{Sin Theta Theorem for Self-adjoint Operators}
% \begin{lemma}[Proposition 2.3 in \cite{seelmann2014notes}]\label{lemma, sin theta of infinite dimension operator}
% Let $B$ be a self-adjoint operator on a separable Hilbert space $\widetilde{\mathcal{H}}$, and let ${V}\in\mathcal{L}(\widetilde{\mathcal{H}})$ be another self-adjoint operator where $\mathcal{L}\l\widetilde{\mc H}\r$ stands for the space of bounded linear operators from a Hilbert space $\widetilde{\mc H}$ to $\widetilde{\mc H}$.
% Write \[\mathrm{spec}( B)=\sigma\cup\Sigma\quad\text{and}\quad \mathrm{spec}( B+ V)=\omega\cup\Omega
% \]
% with $\sigma\cap\Sigma=\varnothing=\omega\cap\Omega$, and suppose that there is $\widehat d>0$ such that
% \[\mathrm{dist}(\sigma,\Omega)\geqslant \widehat d\quad\text{and}\quad\mathrm{dist}(\Sigma,\omega)\geqslant \wh d\]
% where $\mathrm dist(\sigma,\Sigma):=\min\{|a-b|:a\in\sigma,b\in\Omega\}$.
% Then, the operator angle $\Theta=\Theta(P_{ B}(\sigma),P_{ B+ V}(\omega))$ satisfies the bound
% \[\|\sin\Theta\|:=\|P_{{B}}(\sigma)-P_{{B}+{V}}(\omega)\| \leqslant\frac\pi2\frac{\| V\| }{\wh d}\]
% where $P_{ B}(\sigma)$ denotes the spectral projection for $ B$ associated with $\sigma$, i.e., 
% \[P_{B}(\sigma):=\frac{1}{2\pi\mathrm{i}}\oint_{\gamma}\frac{\mathrm{d}z}{z-B},\]
% where $\gamma$ is a contour on $\mathbb{C}$ that encloses $\sigma$ but no other elements of $\mathrm{spec}( B)$.
% \end{lemma}
% \begin{remark}
% We note that, 
% if further $ B$ is compact, 
% the spectral projection coincide with projection operator onto the closure of the space spanned by the eigenfunctions associated with the eigenvalues in $\sigma$.

% If $B$ is compact, by the spectral decomposition theorem one has
% \[B=\sum_{i=1}^\infty\mu_ie_i\otimes e_i\quad\text{and}\quad(z- B)^{-1}=\sum_{i=1}^\infty(z-\mu_i)^{-1}e_i\otimes e_i,\]
% where $\mr{spec}(B):=\{\mu_i\}_{i=1}^\infty$ satisfies $|\mu_i|\xrightarrow{i\to\infty} 0$.
% Then $\forall v\in \mathcal{H}$,
% \begin{align*}P_{B}(\sigma)v&=\frac{1}{2\pi\mathrm{i}}\oint_{\gamma}({z-B})^{-1}v~{\mathrm{d}z}=\frac{1}{2\pi\mathrm{i}}\oint_{\gamma}\sum_{i=1}^\infty(z-\mu_i)^{- 1}\langle e_i,v\rangle e_i~{\mathrm{d}z}\\
% &=\sum_{i=1}^\infty\left[\left(\frac{1}{2\pi\mathrm{i}}\oint_{\gamma}(z-\mu_i)^{-1}~{\mathrm{d}z}\right)\langle e_i,v\rangle e_i\right]=\sum_{i\in\{i:\mu_i\in\sigma\}}\langle e_i,v\rangle e_i.
% \end{align*}
% Especially, if $\sigma=\mr{spec}(B)\backslash\{0\}$, then $P_{B}(\sigma)$ is the projection operator onto the $\overline{\mathrm{Im}}(B)$.
% \end{remark}
% Splitting eigenvalues into nonzero part and zero part yields the following useful corollary.
% \begin{corollary}\label{cor: sin theta self adjoint}
% Let $B$ and $B'$ be two positive semi-definite {and compact} operators with finite rank on a separable Hilbert space $\widetilde{\mathcal{H}}$. Let $\lambda_{\min}^+( B)$ and $\lambda_{\min}^+(B')$ be the infimum of the positive eigenvalues of ${B}$ and ${B}'$ respectively. Then we have
% \[\left\|P_{ B}-P_{ B'}\right\| \leqslant\frac\pi2\frac{\| B- B'\| }{\min\{\lambda_{\min}^+( B),\lambda_{\min}^+( B')\}}.\]
% \end{corollary}
% \subsection{Sin Theta Theorem for General Operators}
% When ${B}$ and ${V}$ in Lemma $\ref{lemma, sin theta of infinite dimension operator}$ are not self-adjoint, we use the symmetrization trick, which mainly depends on the following Lemma.
% \begin{lemma}\label{lem:projection equality}
% $P_A=P_{AA^*}$ for any bounded linear operator $A$ from a Hilbert space $\wt\H$ to $\wt\H$. Especially, $P_A=P_{AA^{\top}}$ for any matrix $A$.
% \end{lemma}
% \begin{proof}This lemma is a direct corollary of Lemma $\ref{lem: colPBP equal colPB operator}$.
% \end{proof}

% Then we have the following Sin Theta theorem for general operator.
% \begin{lemma}\label{lemma, sin theta of nonadjoint operator}
% Let $ B,B'\in\mathcal{L}(\widetilde{\mathcal{H}})$ be two compact operators (not necessarily self-adjoint) with finite rank.
% Then we have
% \begin{align*}
% \left\|P_{ B}-P_{ B'}\right\| &\leqslant\frac\pi2\frac{\| B B^*- B'B'^*\| }{\min\lb\sigma_{\min}^+( B)^2,\sigma_{\min}^+(B')^2\rb}\\
% &\leqslant \frac\pi2\frac{\| B- B'\| ^2+2\| B- B'\| \| B'\| }{\min\lb\sigma_{\min}^+( B)^2,\sigma_{\min}^+( B')^2\rb}.
% \end{align*}
% \end{lemma}
% \begin{proof}By Lemma $\ref{lem:projection equality}$, one can get $\left\|P_{ B}-P_{ B'}\right\| =\left\|P_{ B B^*}-P_{ B' B'^*}\right\| $.
% Since $ BB^*, B'B'^*$ are both self-adjoint and compact, by Lemma $\ref{cor: sin theta self adjoint}$, one has
% \begin{align*}
% \left\|P_{ B B^*}-P_{ B' B'^*}\right\| \leqslant \frac{\pi}{2}\frac{\| B B^*- B' B'^*\| }{\min\lb\lambda_{\min}^+\l B B^*\r,\lambda_{\min}^+\l B' B'^*\r\rb}.
% \end{align*}
% Then the proof is completed in view of the following inequality:
% % of $\| B B^*- B' B'^*\| $:
% \begin{align}
% \lno B B^*- B' B'^*\rno &= \|( B- B')( B- B')^*\hspace{-0.5mm}+\hspace{-0.5mm}( B-B')(B')^*\hspace{-0.5mm}+\hspace{-0.5mm} B'( B- B')^*\| \nonumber\\
% &\leqslant \| B- B'\| ^2+2\| B- B'\| \| B'\| . \label{eq:sy ineq}
% \end{align}
% \end{proof}


\section{Sin Theta Theorem}\label{ap:Sin Theta theorem}
\subsection{Sin Theta Theorem for Self-adjoint Operators}
\begin{lemma}[Proposition 2.3 in \cite{seelmann2014notes}]\label{lemma, sin theta of infinite dimension operator}
Let $B$ be a self-adjoint operator on a separable Hilbert space $\widetilde{\mathcal{H}}$, and let ${V}\in\mathcal{L}(\widetilde{\mathcal{H}})$ be another self-adjoint operator where $\mathcal{L}\left(\widetilde{\mc H}\right)$ stands for the space of bounded linear operators from a Hilbert space $\widetilde{\mc H}$ to $\widetilde{\mc H}$.
Write the spectra of $B$ and $B+V$ as \[\mathrm{spec}( B)=\sigma\cup\Sigma\quad\text{and}\quad \mathrm{spec}( B+ V)=\omega\cup\Omega
\]
with $\sigma\cap\Sigma=\varnothing=\omega\cap\Omega$, and suppose that there is $\widehat d>0$ such that
\[\mathrm{dist}(\sigma,\Omega)\geqslant \widehat d\quad\text{and}\quad\mathrm{dist}(\Sigma,\omega)\geqslant \wh d\]
where $\mathrm dist(\sigma,\Sigma):=\min\{|a-b|:a\in\sigma,b\in\Omega\}$.
Then it holds that
\[\|P_{{B}}(\sigma)-P_{{B}+{V}}(\omega)\| \leqslant\frac\pi2\frac{\| V\| }{\wh d}\]
where $P_{ B}(\sigma)$ denotes the spectral projection for $ B$ associated with $\sigma$, i.e., 
\[P_{B}(\sigma):=\frac{1}{2\pi\mathrm{i}}\oint_{\gamma}\frac{\mathrm{d}z}{z-B},\]
where $\gamma$ is a contour on $\mathbb{C}$ that encloses $\sigma$ but no other elements of $\mathrm{spec}( B)$.
\end{lemma}
\begin{remark}
We note that, 
if further $ B$ is compact, 
the spectral projection coincide with projection operator onto the closure of the space spanned by the eigenfunctions associated with the eigenvalues in $\sigma$. 
% For more details, see, e.g., Remark 1 in \cite{chen2023optimality}.

Specifically, if $B$ is compact, by the spectral decomposition theorem one has
\[B=\sum_{i=1}^\infty\mu_ie_i\otimes e_i\quad\text{and}\quad(z- B)^{-1}=\sum_{i=1}^\infty(z-\mu_i)^{-1}e_i\otimes e_i,\]
where $\mr{spec}(B):=\{\mu_i\}_{i=1}^\infty$ satisfies $|\mu_i|\xrightarrow{i\to\infty} 0$.
Then $\forall v\in \mathcal{H}$, it holds that
\begin{align*}P_{B}(\sigma)v&=\frac{1}{2\pi\mathrm{i}}\oint_{\gamma}({z-B})^{-1}v~{\mathrm{d}z}=\frac{1}{2\pi\mathrm{i}}\oint_{\gamma}\sum_{i=1}^\infty(z-\mu_i)^{- 1}\langle e_i,v\rangle e_i~{\mathrm{d}z}\\
&=\sum_{i=1}^\infty\left[\left(\frac{1}{2\pi\mathrm{i}}\oint_{\gamma}(z-\mu_i)^{-1}~{\mathrm{d}z}\right)\langle e_i,v\rangle e_i\right]=\sum_{i\in\{i:\mu_i\in\sigma\}}\langle e_i,v\rangle e_i.
\end{align*}
In particular, if $\sigma=\mr{spec}(B)\backslash\{0\}$, then $P_{B}(\sigma)$ is the projection operator onto the $\overline{\mathrm{Im}}(B)$.
\end{remark}

Splitting eigenvalues into nonzero part and zero part yields the following useful corollary.
\begin{corollary}\label{cor: sin theta self adjoint}
Let $B$ and $B'$ be two positive semi-definite {and compact} operators with finite rank on a separable Hilbert space $\widetilde{\mathcal{H}}$. Let $\lambda_{\min}^+( B)$ and $\lambda_{\min}^+(B')$ be the infimum of the positive eigenvalues of ${B}$ and ${B}'$ respectively. Then we have
\[\left\|P_{ B}-P_{ B'}\right\| \leqslant\frac\pi2\frac{\| B- B'\| }{\min\{\lambda_{\min}^+( B),\lambda_{\min}^+( B')\}}.\]
\end{corollary}
\subsection{Sin Theta Theorem for General Operators}
When ${B}$ and ${V}$ in Lemma $\ref{lemma, sin theta of infinite dimension operator}$ are not self-adjoint, we use the symmetrization trick, which mainly depends on the following Lemma.
\begin{lemma}\label{lem:projection equality}
$P_A=P_{AA^*}$ for any bounded linear operator $A$ from a Hilbert space $\wt\H$ to $\wt\H$. Especially, $P_A=P_{AA^{\top}}$ for any matrix $A$.
\end{lemma}
\begin{proof}First we show that the null space of  $A^*$ is the same as the null space of $AA^*$.
On the one hand, 
\[x\in\mathrm{null}(A^*)\Longrightarrow
A^*x=0\Longrightarrow AA^*x=0\Longrightarrow x\in\mathrm{null}(AA^*); 
\]
One the other hand,
\begin{align*}x\in\mathrm{null}(AA^*)&\Longrightarrow
AA^*x=0\Longrightarrow \langle x,AA^*x\rangle=\langle A^*x,A^*x\rangle=\|A^*x\|^2=0\\
&\Longrightarrow A^*x=0\Longrightarrow x\in\mathrm{null}(A^*).
\end{align*}
Hence, we have $\mathrm{null}(A^*)=\mathrm{null}(AA^*)$. Take the orthogonal complement of the both sides of this equality, we can get
\[\mathrm{null}(A^*)^{\perp}=\mathrm{null}(AA^*)^{\perp}\Longrightarrow {\mathrm{Im}(A)}={\mathrm{Im}(AA^*)}.\]
\end{proof}
Then we have the following Sin Theta theorem for general operator.
\begin{lemma}\label{lemma, sin theta of nonadjoint operator}
Let $ B,B'\in\mathcal{L}(\widetilde{\mathcal{H}})$ be two compact operators (not necessarily self-adjoint) with finite rank.
Then we have
\begin{align*}
\left\|P_{ B}-P_{ B'}\right\| &\leqslant\frac\pi2\frac{\| B B^*- B'B'^*\| }{\min\left\{\sigma_{\min}^+( B)^2,\sigma_{\min}^+(B')^2\right\}}\\
&\leqslant \frac\pi2\frac{\| B- B'\| ^2+2\| B- B'\| \| B'\| }{\min\left\{\sigma_{\min}^+( B)^2,\sigma_{\min}^+( B')^2\right\}}.
\end{align*}
\end{lemma}
\begin{proof}By Lemma $\ref{lem:projection equality}$, one can get $\left\|P_{ B}-P_{ B'}\right\| =\left\|P_{ B B^*}-P_{ B' B'^*}\right\| $.
Since $ BB^*, B'B'^*$ are both self-adjoint and compact, by Lemma $\ref{cor: sin theta self adjoint}$, one has
\begin{align*}
\left\|P_{ B B^*}-P_{ B' B'^*}\right\| \leqslant \frac{\pi}{2}\frac{\| B B^*- B' B'^*\| }{\min\left\{\lambda_{\min}^+\left( B B^*\right),\lambda_{\min}^+\left( B' B'^*\right)\right\}}.
\end{align*}
Then the proof is completed in view of the following inequality:
% of $\| B B^*- B' B'^*\| $:
\begin{align}
\left\| B B^*- B' B'^*\right\| &= \|( B- B')( B- B')^*\hspace{-0.5mm}+\hspace{-0.5mm}( B-B')(B')^*\hspace{-0.5mm}+\hspace{-0.5mm} B'( B- B')^*\| \nonumber\\
&\leqslant \| B- B'\| ^2+2\| B- B'\| \| B'\| . \label{eq:sy ineq}
\end{align}
\end{proof}


\section{Proof of Theorem \ref{theorem, total convergence rate}}
Thanks to the triangle inequality, one can bound the subspace estimation error by bounding the error term (i): $\mathbf{ Loss}_1:=\left\|P_{\mc S_{\Y|\X}^{(m)}}-P_{ \widehat {\mc S}_{\Y|\X}^{(m)}}\right\| $ and error term (ii): $\mathbf{ Loss}_2:= \left\|P_{\mathcal S_{\Y|\boldsymbol{X}}}-P_{\mathcal S_{\Y|\boldsymbol{X}}^{(m)}}\right\| $ respectively.
\subsection{Upper bound of error term (i)}
We first give the following lemmas, whose proofs are all deferred to the end of this section.
\begin{lemma}\label{lem:Gammam dagger Mm uniformly bounded}
% Under Assumptions $\ref{as:joint distribution assumption}$ and $\ref{as:Linearity condition and Coverage condition}$,
% $\{\|\Gamma_m^\dagger M_m\| \}_{m=1}^\infty$ is uniformly (about $m$) bounded by $\|\Gamma^{-1}M\| $.
Under Assumptions $\ref{as:joint distribution assumption}$ and $\ref{as:Linearity condition and Coverage condition}$, it holds that $\|\Gamma_m^\dagger M_m\| \leq \|\Gamma^{-1}M\| (\forall m).$
% \begin{align*}
% \|\Gamma_m^\dagger M_m\| \leq \|\Gamma^{-1}M\| \quad\forall m.
% \end{align*}
% $\{\|\Gamma_m^\dagger M_m\| \}_{m=1}^\infty$ is uniformly (about $m$) bounded by $\|\Gamma^{-1}M\| $.
\end{lemma}
\begin{lemma}\label{lem: Gamma inverse M to Gammam dagger Mm}Under Assumptions $\ref{as:joint distribution assumption}$ and $\ref{as:Linearity condition and Coverage condition}$, we have \[\lim\limits_{m\to\infty}\lno\Gamma^{-1}M-\Gamma_m^\dagger M_m\rno =0.\]
\end{lemma}
\noindent We denote by $m_T(\varepsilon)$ the minimal integer $m_T$ satisfying $\lno\Gamma^{- 1}M-\Gamma_m^\dagger M_m\rno \hspace{-1mm}\leqslant \varepsilon$ for all $m\geqslant m_T$ and define an event 
$$\ttE:=\lb \left\|\widehat\Gamma_m^\dagger \widehat M_m^d-\Gamma_m^\dagger M_m\right\|  \leqslant\hspace{-0.5mm}\left(\tfrac{D_0+1}{D_2}\right)^{\frac52}\tfrac{24}{\wt C}n^{c_1(\alpha_1+1)+\gamma-\frac{1}{2}}+D_3n^{\frac{c_1(2\alpha_1+1)-1}{2}}\rb.$$
Then by taking $C$ to be $(D_0+1)n^{\frac{2\gamma}{5}}-\ln\l D_1m^2n \r$ in  Proposition \ref{prop:bound of finite estimate}, one has: for $n\geqslant \l\frac{D_0+1}{D_2}\r^{\frac{5}{1-2\gamma}}$,
$$\P(\ttE)\geq 1-D_1m^2n\exp\left[-(D_0+1)n^{\frac{2\gamma}{5}}\right] -2\exp(- C'm).$$
\begin{lemma}\label{lem:lower bound sigma min total}
Introducing $
\bigtriangleup :=\max\lb \frac{\sigma_d(\Gamma^{-1} M)}{2},\frac{\sigma_d(\Gamma^{-1} M)^2}{4\|\Gamma^{-1}M\| } \rb$.
Suppose that Assumptions $\ref{as:joint distribution assumption}$ to $\ref{assumption: rate-type condition}$ hold, $c_1(2\alpha_1+1)-1<0$ and $2(c_1(\alpha_1+1)+\gamma)-1<0$. Then there exists a positive constant
\begin{align*}
n_2'=n_2'\l\sigma_d(\Gamma^{-1}M),\|\Gamma^{-1}M\| , \gamma,\sigma_0,\sigma_1,\bs K,m_M(1),c_1,m_T\l \tfrac{\bigtriangleup}{2}\r,\wt C,\alpha_1\r
\end{align*}
such that when $n\geqslant n_2'$, we have
\begin{align}
\sigma_{\min}^+(\Gamma_m^{\dagger} M_m)^2\geqslant \tfrac{\sigma_d(\Gamma^{-1}M)^2}{2} \label{eq: lower bound of sigma min}. 
\end{align}
Furthermore, Conditioning on $\ttE$, we have
\begin{align}\label{eq: lower bound of sigma min hat}
&\sigma_{\min}^+(\wh\Gamma_m^{\dagger}\wh M_m^d)^2\geqslant \tfrac{\sigma_d(\Gamma^{-1}M)^2}{2}.
\end{align}
\end{lemma}
The following proposition is an upper bound of error term (i):
\begin{proposition}\label{proposition, estimation error}
Positive constants $D_1$, $D_2$ and $C'$  as in Proposition $\ref{prop:bound of finite estimate}$,
suppose that Assumptions $\ref{as:joint distribution assumption}$ to $\ref{assumption: rate-type condition}$ hold, then $\forall \gamma\in(0,1/2)$, if $c_1$ satisfies $2c_1(\alpha_1+1)+2\gamma-1<0$ and $c_1(2\alpha_1+1)-1<0$, there exists a positive constant $C_1:=C_1\l \|\Gamma^{-1}M\| ,\sigma_d(\Gamma^{-1}M) ,\wt C,\gamma,\sigma_0,\sigma_1\r$ such that
\begin{align*}
\P\l
\lno P_{\mc{S}_{\Y|\X}^{(m)}}-P_{ \widehat{\mc{S}}_{\Y|\X}^{(m)}}\rno \leqslant C_1\frac{m^{\alpha_1+1}}{n^{1/2-\gamma}}\r\geqslant1-2\exp(- C'm)&\\
- D_1m^2n\exp\l -(D_0+1)n^{\frac{2\gamma}{5}} \r&,
\end{align*}
when 
\begin{align*}
n\geqslant\max\Bigg\{ n_1,\l\tfrac{D_0+1}{D_2}\r^{\frac{5}{1-2\gamma}},\left[\tfrac{\|\Gamma^{-1}M\|  \wt C}{48}\l\tfrac{D_2}{D_0+1}\r^{\frac52}\right]^{\frac{2}{2(c_1(\alpha_1+1)+\gamma)-1}}&,\\
\l \tfrac{\|\Gamma^{-1}M\| }{2D_3}\r^{\frac{2}{c_1(2\alpha_1+1)-1}},n_2',\left[ \tfrac{D_3\wt C}{24}\l \tfrac{D_2}{D_0+1} \r^{\frac52} \right]^{\frac2{2\gamma+c_1}}&\Bigg\}
\end{align*}
where $n_2'$ is defined in Lemma $\ref{lem:lower bound sigma min total}$.
\end{proposition}
\begin{proof}
By Lemma $\ref{lemma, way of estimate truncate central subspace}$, $\eqref{def: estimator central subspace}$ and Lemma $\ref{lemma, sin theta of nonadjoint operator}$, one has
\begin{align}
&\left\|P_{\mc S_{\Y|\vX}^{(m)}}-P_{\wh{\mc{S}}_{\Y|\vX}^{(m)}}\right\| =\left\|P_{\Gamma_m^{\dagger}M_m}-P_{\wh\Gamma_m^{\dagger}\wh M_m^d}\right\| \nonumber\\
&\qquad\leqslant\frac{\pi}{2}\frac{\lno\widehat\Gamma_m^\dagger \widehat M_m^d-\Gamma_m^\dagger M_m\rno ^2+\lno\widehat\Gamma_m^\dagger \widehat M_m^d-\Gamma_m^\dagger M_m\rno \lno\Gamma_m^\dagger M_m\rno }{\min\lb\sigma_{\min}^+\l\wh\Gamma_m^\dagger \wh M_m^d\r^2,\sigma_{\min}^+\l\Gamma_m^\dagger M_m\r^2\rb}\label{eq: PS minus P hat S norm}.
% &\leqslant C_5\|\widehat\Gamma_m^\dagger \widehat M_m^d-\Gamma_m^\dagger M_m\|\\
% &=\widetilde O_{\mathbb{P}}\l\frac{m^{\alpha_1+1}}{n^{1/2}}\r,
\end{align}
% with probability at least $1-\exp(- C)-2\exp(- C'm)$.
Because of $c_1(2\alpha_1+1)-1<0$ and $2(c_1(\alpha_1+1)+\gamma)-1<0$, it is easy to check that when
\[n\geqslant\max\lb\left[\tfrac{\|\Gamma^{-1}M\|  \wt C}{48}\l\tfrac{D_2}{D_0+1}\r^{\frac52}\right]^{\frac{2}{2(c_1(\alpha_1+1)+\gamma)-1}},\l \tfrac{\|\Gamma^{-1}M\| }{2D_3}\r^{\frac{2}{c_1(2\alpha_1+1)-1}}\rb,\]
both $\l\tfrac{D_0+1}{D_2}\r^{\frac52}\tfrac{24}{\wt C}n^{c_1(\alpha_1+1)+\gamma-\frac{1}{2}}$ and $D_3n^{\frac{c_1(2\alpha_1+1)-1}{2}}$ are less than or equal to $\frac{\|\Gamma^{-1}M\| }{2}$. Thus, on the event $\ttE$,
\begin{align}\label{eq: high prob upper bound is Gamma minus 1 M}
\lno\widehat\Gamma_m^\dagger \widehat M_m^d-\Gamma_m^\dagger M_m\rno \leqslant \lno\Gamma^{-1}M\rno .
\end{align}
By Lemma $\ref{lem:Gammam dagger Mm uniformly bounded}$, inserting \eqref{eq: high prob upper bound is Gamma minus 1 M} into \eqref{eq: PS minus P hat S norm} leads to
$$
\lno P_{\mc{S}_{\Y|\X}^{(m)}}-P_{ \widehat{\mc{S}}_{\Y|\X}^{(m)}}\rno
\leqslant \frac{\pi\lno\widehat\Gamma_m^\dagger \widehat M_m^d-\Gamma_m^\dagger M_m\rno \lno\Gamma^{-1}M\rno }{\min\lb\sigma_{\min}^+\l\wh\Gamma_m^\dagger \wh M_m^d\r^2,\sigma_{\min}^+\l\Gamma_m^\dagger M_m\r^2\rb},
$$
on the event $\ttE$.
Furthermore, when $n\geqslant \left[ \frac{D_3\wt C}{24}\l \frac{D_2}{D_0+1} \r^{\frac52} \right]^{\frac2{2\gamma+c_1}}$ and $n\geq n_2'$, one can get
$\l \tfrac{D_0+1}{D_2}\r^{\frac52}\tfrac{24m^{\alpha_1+1}}{\wt C n^{1/2-\gamma}}$ is greater than or equal to $D_3\tfrac{m^{(2\alpha_1+1)/2}}{n^{1/2}}$
and then on the event $\ttE$,
\begin{align*}
\lno P_{\mc{S}_{\Y|\X}^{(m)}}-P_{ \widehat{\mc{S}}_{\Y|\X}^{(m)}}\rno \leqslant \tfrac{96\pi\|\Gamma^{-1}M\| }{\sigma_d(\Gamma^{-1}M)^2}\l \tfrac{D_0+1}{D_2}\r^{\frac52}\tfrac{m^{\alpha_1+1}}{\wt C n^{1/2-\gamma}}.
\end{align*}
 by
Lemma $\ref{lem:lower bound sigma min total}$.
Then choosing $C_1=\tfrac{96\pi\|\Gamma^{-1}M\| }{\wt C\sigma_d(\Gamma^{-1}M)^2}\l \tfrac{D_0+1}{D_2}\r^{\frac52}$ can complete the proof.
\end{proof}



\paragraph{Proof of Lemma \ref{lem:Gammam dagger Mm uniformly bounded}}
\begin{proof}
First, it is easy to check that:
\begin{align}
\Gamma^\dag_m=\Pi_m\Gamma^{-1}\Pi_m=\Pi_m\Gamma^{-1}=\Gamma^{-1}\Pi_m=\sum\limits_{i=1}^m\lambda_i^{-1}\phi_i\otimes\phi_i.\label{eq: Gamma m dag def}
\end{align}
According to \eqref{eq: Gamma m dag def} and $M_m=\Pi_mM\Pi_m$, it is easy to check that $\Gamma_m^\dagger M_m=\Pi_m \Gamma^{- 1}M\Pi_m$. Then by the compatibility of operator norm, one can get
\begin{align*}
\lno\Gamma_m^\dagger M_m\rno =\lno\Pi_m \Gamma^{-1}M\Pi_m\rno \leqslant \lno\Pi_m\rno  \lno\Gamma^{-1}M\rno \lno\Pi_m\rno =\lno\Gamma^{-1}M\rno .
\end{align*}
Note that $\Gamma^{-1}M$ is bounded since $\Gamma^{-1}M$ is of finite rank by Corollary $\ref{corollary, MDDO and central subspace}$. Thus the proof is completed. 
\end{proof}


\paragraph{Proof of Lemma \ref{lem: Gamma inverse M to Gammam dagger Mm}}
\begin{proof}
It is easy to check that
$\Gamma_m^\dagger M_m=\Pi_m\Gamma^{-1}M\Pi_m$ and $\Gamma^{-1}M$ is of finite rank by Corollary $\ref{corollary, MDDO and central subspace}$.
Thus the proof is completed by Lemma $\ref{lem:PimTPimtoT}$.
\end{proof}
\paragraph{Proof of Lemma \ref{lem:lower bound sigma min total}}
\begin{proof}
We first prove \eqref{eq: lower bound of sigma min}.
By Corollary $\ref{corollary, MDDO and central subspace}$ and Lemma $\ref{lem:projection equality}$, one has $\rank(\Gamma^{- 1}M)=\rank\l\Gamma^{- 1}M(\Gamma^{- 1}M)^*\r=d$. Thus
\begin{align*}
\sigma_{\min}^+(\Gamma^{-1}M)^2=\lambda_{\min}^+\l\Gamma^{-1}M(\Gamma^{-1}M)^*\r=\lambda_d\l \Gamma^{-1}M(\Gamma^{-1}M)^*\r. 
\end{align*}
 It is easy to see $\rank(\Gamma_m^\dagger M_m)=\rank\l \Gamma_m^\dagger M_m(\Gamma_m^\dagger M_m)^*\r\leqslant d$ by $\Gamma_m^\dagger M_m=\Pi_m \Gamma^{-1} M \Pi_m$ and Lemma $\ref{lem:projection equality}$, thus one can assume that 
 \begin{align*}
\sigma_{\min}^+(\Gamma^\dagger_m M_m)^2=\lambda_{\min}^+\l\Gamma_m^\dagger M_m(\Gamma_m^\dagger M_m)^*\r=\lambda_j\l \Gamma_m^\dagger M_m(\Gamma_m^\dagger M_m)^*\r
\end{align*}
for some $j\leqslant d$.
By Corollary $\ref{coro:wely ineq operator}$, $\eqref{eq:sy ineq}$ and
% (Notice that $M_m$ and $M$ are both compact and self-adjoint)
Lemma $\ref{lem: Gamma inverse M to Gammam dagger Mm}$
%and Lemma \ref{lem:Gammam dagger Mm uniformly bounded}
, one has
\begin{align*}
&\left|\sigma_{\min}^+(\Gamma^\dagger_m M_m)^2\hspace{-0.5mm}-\hspace{-0.5mm}\sigma_j(\Gamma^{-1} M)^2\right|\hspace{-0.5mm}=\hspace{-0.5mm}\left|\lambda_{j}\hspace{-1mm}\l\Gamma^\dagger_m M_m(\Gamma^\dagger_m M_m)^{*}\hspace{-0.5mm}\r\hspace{-0.5mm}-\hspace{-0.5mm}\lambda_j\hspace{-1mm}\l \Gamma^{-1} M(\Gamma^{-1} M)^*\hspace{-0.5mm}\r\right|\\
&\qquad\leqslant
\|\Gamma^{-1} M(\Gamma^{-1} M)^*- \Gamma_m^\dagger M_m(\Gamma_m^\dagger M_m)^*\| \\
&\qquad\leqslant \|\Gamma^{-1} M- \Gamma_m^\dagger M_m\| ^2+
\|\Gamma^{-1} M- \Gamma_m^\dagger M_m\| \cdot\|\Gamma^{-1} M\| \xrightarrow{m\to\infty} 0. 
% &\leqslant\|\Gamma^{-1} M- \Gamma_m^\dagger M_m\|\cdot3\|\Gamma^{-1} M\|
\end{align*}
Thus for 
$
n\geqslant m_T(\bigtriangleup)^{\frac1{c_1}}=m_T\l\max\lb\frac{\sigma_d(\Gamma^{-1} M)}{2},\frac{\sigma_d(\Gamma^{-1} M)^2}{4\|\Gamma^{-1}M\| }\rb\r^{\frac1{c_1}}, 
$
one has $\|\Gamma^{-1} M- \Gamma_m^\dagger M_m\| ^2$ and $\|\Gamma^{-1} M- \Gamma_m^\dagger M_m\| \cdot\|\Gamma^{-1} M\| $ are both less than or equal to $\frac{1}{4}\sigma_d(\Gamma^{-1} M)^2$. Hence one can get
$\left|\sigma_{\min}^+(\Gamma^\dagger_m M_m)^2-\sigma_j(\Gamma^{-1} M)^2\right|\leqslant\frac{1}{2}\sigma_d(\Gamma^{-1} M)^2$
% \begin{align*}\label{eq:sigma min Mm}
% \left|\sigma_{\min}^+(\Gamma^\dagger_m M_m)^2-\sigma_j(\Gamma^{-1} M)^2\right|\leqslant\frac{1}{2}\sigma_d(\Gamma^{-1} M)^2
% \|
% \lambda_j\l \Gamma_m^\dagger M_m\l\Gamma_m^\dagger M_m\r^*\r\geqslant \lambda_j\l \Gamma^{-1} M\l\Gamma^{-1} M\r^*\r-\frac{\lambda_d\l \Gamma^{-1} M\l\Gamma^{-1} M\r^*\r}{2}
% \geqslant\frac{\lambda_d\l \Gamma^{-1} M\l\Gamma^{-1} M\r^*\r}{2}. 
% \end{align*}
and
\begin{equation}
\sigma_{\min}^+(\Gamma^\dagger M_m)^2\geqslant \sigma_j(\Gamma^{-1} M)^2-\frac{1}{2}\sigma_d(\Gamma^{-1} M)^2\geqslant\frac{1}{2}\sigma_d(\Gamma^{-1} M)^2
\end{equation}
for sufficiently large $n$. This completes the proof of \eqref{eq: lower bound of sigma min}.





Next we prove $\eqref{eq: lower bound of sigma min hat}$. Combining Proposition $\ref{prop:bound of finite estimate}$ with Lemma $\ref{lem: Gamma inverse M to Gammam dagger Mm}$ leads to that on the event $\ttE$, 
$$
\lno\wh \Gamma_m^\dag\wh M^d_m- \Gamma^{-1}M\rno \leqslant\ve+\l\tfrac{D_0+1}{D_2}\r^{\frac52}\tfrac{24}{\wt C}n^{c_1(\alpha_1+1)+\gamma-\frac{1}{2}}+D_3n^{\frac{c_1(2\alpha_1+1)-1}{2}}
$$
for  $n\geqslant \max\{n_1,m_T( \ve)^{1/c_1}\}$.
Assuming that $c_1(2\alpha_1+1)-1<0$ and $2(c_1(\alpha_1+1)+\gamma)-1<0$, it is easy to check that when $$n\geqslant\max\lb\left[\frac{\bigtriangleup \wt C}{96}\l\frac{D_2}{D_0+1}\r^{\frac52}\right]^{\frac{2}{2(c_1(\alpha_1+1)+\gamma)-1}},\l \frac{\bigtriangleup}{4D_3}\r^{\frac{2}{c_1(2\alpha_1+1)-1}}\rb$$, both $\l\tfrac{D_0+1}{D_2}\r^{\frac52}\tfrac{24}{\wt C}n^{c_1(\alpha_1+1)+\gamma-\frac{1}{2}}$ and $D_3n^{\frac{c_1(2\alpha_1+1)-1}{2}}$ are less than or equal to $\frac{\bigtriangleup}{4}$. Letting $\varepsilon=\frac12\bigtriangleup$, one can get on the event $\ttE$,
when
\begin{align*}
n&\geqslant n_2'=n_2'\hspace{-0.5mm}\l\hspace{-0.5mm}\sigma_d(\Gamma^{-1}M),\|\Gamma^{-1}M\| , \gamma,\sigma_0,\sigma_1,\bs K,m_M(1),c_1,m_T\l \tfrac{\bigtriangleup}{2}\r,\wt C,\alpha_1\hspace{-0.5mm}\r\\
&:=\max\bigg\{ n_1,m_T\l \tfrac{\bigtriangleup}{2}\r^{1/c_1}, \left[\tfrac{\bigtriangleup \wt C}{96}\l\tfrac{D_2}{D_0+1}\r^{\frac52}\right]^{\frac{2}{2(c_1(\alpha_1+1)+\gamma)-1}},\l \tfrac{\bigtriangleup}{4D_3}\r^{\frac{2}{c_1(2\alpha_1+1)-1}}\bigg\},
\end{align*}
one has $\lno\wh \Gamma_m^\dag\wh M^d_m- \Gamma^{-1}M\rno \leqslant\bigtriangleup$ and further
$\sigma_{\min}^+(\wh\Gamma^\dagger \wh M^d_m)^2\hspace{-1mm}\geqslant\hspace{-1mm} \tfrac{\sigma_d(\Gamma^{-1} M)^2}{2}$ by the same argument as the proof of \eqref{eq: lower bound of sigma min}.
 This completes the proof of \eqref{eq: lower bound of sigma min hat}.
Considering that $m_T(\bigtriangleup)\leqslant m_{T}\l\frac\bigtriangleup2\r$, one can also get $\eqref{eq: lower bound of sigma min}$ when $n\geqslant n_2'$. Thus the proof is completed.
\end{proof}
\subsection{Upper bound of error term (ii)}\label{ap, subs, truncation error}
\begin{proposition}\label{proposition, truncation error}
Under Assumption $\ref{assumption: rate-type condition}$, there exists a positive constant $C_2:=C_2\l d,\wt C,\lambda_d(\mc{B}),\alpha_2\r$ where $\mc{B}:=\sum\limits_{i=1}^d {\bs{\beta}}_i\otimes{\bs{\beta}}_i$ for ${\bs{\beta}}_i$ defined in \eqref{def: central subspace}, such that when $n\geqslant \l \frac{\lambda_d({\mc{B}})}{4d\wt C^2}\sqrt{\frac{2\alpha_2-1}{\zeta(2\alpha_2)}}\r^{\frac{2}{c_1(1-2\alpha_2)}}$, we have
\begin{equation}\label{equation, truncation error}
 \left\|P_{\mathcal S_{\Y|\boldsymbol{X}}}-P_{\mathcal S_{\Y|\boldsymbol{X}}^{(m)}}\right\| \leqslant C_2m^{-\frac{2\alpha_2-1}{2}},
\end{equation}
where $\zeta(\cdot)$ is Riemann $\zeta$ function.
% \begin{equation}
% \|P_{\mathcal S_{Y|\boldsymbol{X}}}-P_{\mathcal S_{Y|\boldsymbol{X}}^{(m)}}\|\leqslant O_{\mathbb{P}}(dn^{-(\alpha_2-1)/(2\alpha_1+\alpha_2)}) 
% \end{equation}
\end{proposition}
\begin{proof}
Let ${\mc{B}^{(m)}}:=\sum\limits_{i=1}^d {\bs{\beta}}_i^{(m)}\otimes{\bs{\beta}}_i^{(m)}$ for ${\bs{\beta}}_i^{(m)}$ defined in \eqref{def: truncated central subspace}.
Combing with Equation $\eqref{def: central subspace}$, it is easy to check that $\left\|P_{\mathcal S_{\Y|\boldsymbol{X}}}-P_{\mathcal S_{\Y|\boldsymbol{X}}^{(m)}}\right\| =\|P_{\mc{B}}-P_{\mc{B}^{(m)}}\| $. By Corollary $\ref{cor: sin theta self adjoint}$, we have
\begin{align}\label{eq:sin theta for B Bm}
\|P_{\mc{B}}-P_{\mc{B}^{(m)}}\| \leqslant \frac{\pi}{2}\frac{\|{\mc{B}}-{\mc{B}^{(m)}}\| }{\min\{\lambda_{\min}^+({\mc{B}}),\lambda_{\min}^+({\mc{B}^{(m)}})\}}.
\end{align}

Note that ${\mc{B}}-{\mc{B}^{(m)}}$ is self-adjoint, then
\begin{align*}
&\lno{\mc{B}}-{\mc{B}^{(m)}}\rno =\sup_{{\bs{\beta}}\in\mathbb{S}_{ \mathcal H}}|\langle ({\mc{B}}-{\mc{B}^{(m)}})({\bs{\beta}}),{\bs{\beta}}\rangle|=\sup_{{\bs{\beta}}\in\mathbb{S}_{\mathcal H}}|\langle {\mc{B}}{\bs{\beta}},{\bs{\beta}}\rangle-\langle {\mc{B}^{(m)}}{\bs{\beta}},{\bs{\beta}}\rangle|\\
&~~=\sup_{{\bs{\beta}}\in\mathbb{S}_{\mathcal H}}\hspace{-0.9mm}\left|\sum_{i=1}^d\hspace{-0.9mm}\left[\langle{\bs{\beta}}_i,{\bs{\beta}}\rangle^2-\langle{\bs{\beta}}_i^{(m)},{\bs{\beta}}\rangle^2\right]\right|=\sup_{{\bs{\beta}}\in\mathbb{S}_{\mathcal H}}\hspace{-0.9mm}\left| \sum_{i=1}^d\langle{\bs{\beta}}_i-{\bs{\beta}}_i^{(m)},{\bs{\beta}}\rangle\langle{\bs{\beta}}_i+{\bs{\beta}}_i^{(m)},{\bs{\beta}}\rangle\right|\\
&~~\leqslant\sup_{{\bs{\beta}}\in\mathbb{S}_{\mathcal H}}\sum_{i=1}^d\left| \langle{\bs{\beta}}_i-{\bs{\beta}}_i^{(m)},{\bs{\beta}}\rangle\langle{\bs{\beta}}_i+{\bs{\beta}}_i^{(m)},{\bs{\beta}}\rangle\right|
\leqslant\sum_{i=1}^d\left\|{\bs{\beta}}_i-{\bs{\beta}}_i^{(m)}\right\|\left\|{\bs{\beta}}_i+{\bs{\beta}}_i^{(m)}\right\|,
\end{align*}
where the first inequality comes from the triangle inequality, and the 
second inequality comes from the Cauchy-Schwarz inequality and $\|{\bs{\beta}}\|=1$. 
 Then one has ${\bs{\beta}}_i=\sum\limits_{j=1}^\infty b_{ij}\phi_j$ and 
\[{\bs{\beta}}^{(m)}_i=\Pi_m{\bs{\beta}}_i=\sum_{j'=1}^m\phi_{j'}\otimes\phi_{j'}\sum_{j=1}^\infty b_{ij}\phi_j=\sum_{j'=1}^m\sum_{j=1}^\infty\langle\phi_{j'},\phi_j\rangle b_{ij}\phi_{j'}=\sum_{j=1}^mb_{ij}\phi_j.\]
According to Assumption $\ref{assumption: rate-type condition}$, one can get
\begin{align*}
\left\|{\bs{\beta}}_i-{\bs{\beta}}_i^{(m)}\right\|&=\left\|\sum_{j=m+1}^\infty b_{ij}\phi_j\right\|=\sqrt{\sum_{j=m+1}^\infty b_{ij}^2}\leqslant \wt C\sqrt{\sum_{j=m+1}^\infty j^{-2\alpha_2}};\\
\left\|{\bs{\beta}}_i+{\bs{\beta}}_i^{(m)}\right\|&\leqslant\|{\bs{\beta}}_i\|+\lno{\bs{\beta}}_i^{(m)}\rno\leqslant2\|{\bs{\beta}}_i\|=2\sqrt{\sum_{j=1}^\infty b_{ij}^2}\leqslant 2\wt C\sqrt{\sum_{j=1}^\infty j^{- 2\alpha_2}}.
\end{align*}
Because $\alpha_2>1/2$, one has
\[\sum\limits_{j=m+1}^\infty \frac{1}{j^{2\alpha_2}}\leqslant \frac{1}{2\alpha_2-1}\frac{1}{m^{2\alpha_2-1}};\qquad \sum_{j=1}^\infty \frac 1{j^{2\alpha_2}}=\zeta(2\alpha_2)\text{ is convergent},\]
where $\zeta(\cdot)$ is Riemann $\zeta$ function. Thus, one can get
\begin{equation}\label{eq: upper bound of operator norm of A minus B}
\lno{\mc{B}}-{\mc{B}^{(m)}}\rno \leqslant 2d\wt C^2\sqrt{\frac{\zeta(2\alpha_2)}{2\alpha_2-1}}m^{-\frac{2\alpha_2-1}{2}}.
\end{equation}

Furthermore, 
{since $\mr{rank}(\mc{B})=d$, one can get that $\lambda_{\min}^+(\mc{B})=\lambda_{d}(\mc{B})$. It is easy to see $\rank(\mc{B}^{(m)})\leqslant d$ by $\mc{B}^{(m)}=\Pi_m \mc{B} \Pi_m$, thus one can assume that $\lambda_{\min}^+(\mc{B}^{(m)})=\lambda_j( \mc{B}^{(m)})$ for some $j\leqslant d$.
By Corollary $\ref{coro:wely ineq operator}$
% (Notice that $M_m$ and $M$ are both compact and self-adjoint)
and \eqref{eq: upper bound of operator norm of A minus B}, one has:
$$
|\lambda_j( \mc{B}^{(m)})-\lambda_j\l \mc{B}\r|\leqslant\lno \mc{B}-\mc{B}^{(m)}\rno \leqslant 2d\wt C^2\sqrt{\frac{\zeta(2\alpha_2)}{2\alpha_2-1}}m^{-\frac{2\alpha_2-1}{2}}.
$$
Thus for sufficiently large {$n\geqslant \l \frac{\lambda_d({\mc{B}})}{4d\wt C^2}\sqrt{\frac{2\alpha_2-1}{\zeta(2\alpha_2)}} \r^{\frac{2}{c_1(1-2\alpha_2)}}$}, one has
\begin{align}
&\lambda_j\l \mc{B}^{(m)}\r\geqslant \lambda_j\l \mc{B}\r-\frac{\lambda_d\l \mc{B}\r}{2}
\geqslant\frac{\lambda_d\l \mc{B}\r}{2}\nonumber\\
&\qquad\Longrightarrow \min\{\lambda_{\min}^+({\mc{B}}),\lambda_{\min}^+({\mc{B}^{(m)}})\}\geqslant \frac{\lambda_d({\mc{B}})}{2}. \label{eq:lower bound lambda min plus B Bm}
\end{align}}
Inserting \eqref{eq: upper bound of operator norm of A minus B} and \eqref{eq:lower bound lambda min plus B Bm} into \eqref{eq:sin theta for B Bm} leads to
\begin{align*}
\left\|P_{\mathcal S_{\Y|\boldsymbol{X}}}-P_{\mathcal S_{\Y|\boldsymbol{X}}^{(m)}}\right\| \leqslant \frac{2\pi d\wt C^2}{\lambda_{d}(\mc{B})}\sqrt{\frac{\zeta(2\alpha_2)}{2\alpha_2-1}}m^{-\frac{2\alpha_2-1}{2}}.
\end{align*}
Then choosing $C_2:=\frac{2\pi d\wt C^2}{\lambda_d({\mc{B}})}\sqrt{\frac{\zeta(2\alpha_2)}{2\alpha_2-1}}$ can complete the proof.
\end{proof}



\subsection{Proof of Theorem \ref{theorem, total convergence rate}}
\begin{proof}
Note that
\begin{equation}
\begin{aligned}
\left\|P_{\mc{S}_{\Y|\X}}-P_{\widehat{\mc{S}}_{\Y|\X}^{(m)}}\right\| 
&\leqslant \left\|P_{\mc{S}_{\Y|\X}}-P_{\mc{S}_{\Y|\X}^{(m)}}\right\| +\left\|P_{\mc{S}_{\Y|\X}^{(m)}}-P_{ \widehat{\mc{S}}_{\Y|\X}^{(m)}}\right\| .\\
\end{aligned}
\end{equation}
Next we select $m$ to be $n^{\frac{1-2\gamma}{2\alpha_1+2\alpha_2+1}}$, i.e.,  $c_1:=\frac{1-2\gamma}{2\alpha_1+2\alpha_2+1}$. And it is easy to check that $c_1$ satisfies $2c_1(\alpha_1+1)+2\gamma-1=-\frac{(1-2\gamma)(2\alpha_2-1)}{2\alpha_1+2\alpha_2+1}<0$ and $c_1(2\alpha_1+1)-1=-\frac{2[\gamma(2\alpha_1+1)+\alpha_2]}{2\alpha_1+2\alpha_2+1}<0$.
Then combining Proposition $\ref{proposition, estimation error}$ with Proposition $\ref{proposition, truncation error}$ leads to
\begin{align*}
\P\left[\left\|P_{\mc S_{\Y|\X}}-P_{\widehat{\mc{S}}_{\Y|\X}^{(m)}}\right\| \leqslant\hspace{-0.5mm} (C_1+C_2)n^{-\frac{(2\alpha_2-1)(1-2\gamma)}{2(2\alpha_1+2\alpha_2+1)}}\right]\hspace{-1mm}\geqslant\hspace{-1mm} 1-2\exp\hspace{-0.5mm}\l\hspace{-1mm}- C'n^{\frac{1-2\gamma}{2\alpha_1+2\alpha_2+1}}\r&\\
-\exp\left[\ln\l D_1n^{\frac{2\alpha_1+2\alpha_2+3-4\gamma}{2\alpha_1+2\alpha_2+1}} \r-(D_0+1)n^{\frac{2\gamma}{5}}\right]&
\end{align*}
when $n\geqslant n_3'$, where
\begin{align*}
n_3'=\max\Bigg\{n_1,n_2',\left[\tfrac{\|\Gamma^{-1}M\|  \wt C}{48}\l\tfrac{D_2}{D_0+1}\r^{\frac52}\right]^{\frac{2}{2(c_1(\alpha_1+1)+\gamma)-1}}\hspace{-0.9mm},\l \tfrac{\|\Gamma^{-1}M\| }{2D_3}\r^{\frac{2}{c_1(2\alpha_1+1)-1}}\hspace{-0.9mm},\\
\l\tfrac{D_0+1}{D_2}\r^{\frac{5}{1-2\gamma}},\left[ \tfrac{D_3\wt C}{24}\l \tfrac{D_2}{D_0+1} \r^{\frac52} \right]^{\frac2{2\gamma+c_1}},\l\tfrac{\lambda_d(\mc{B})}{4d\wt C^2}\sqrt{\tfrac{{2\alpha_2-1}}{\zeta(2\alpha_2)}} \r^{\frac{2}{c_1(1-2\alpha_2)}}\Bigg\}
\end{align*}

It is easy to check that as long as $\frac{2\gamma}{5}<\frac{1-2\gamma}{2\alpha_1+2\alpha_2+1}\Longrightarrow\gamma<\frac{5}{4(\alpha_1+\alpha_2+3)}$, 
there exists a constant $n_3''=n_3''\l \gamma,\alpha_1,\alpha_2,D_0,D_1,C'\r$ such that when $n\geqslant n_3'$ further, we have 
\begin{align*}
\P\l\left\|P_{\mc S_{\Y|\X}}-P_{\widehat{\mc{S}}_{\Y|\X}^{(m)}}\right\| \leqslant (C_1+C_2)n^{-\frac{(2\alpha_2-1)(1-2\gamma)}{2(2\alpha_1+2\alpha_2+1)}} \r
\geqslant1-2\exp\l-\tfrac{D_0+1}{2}n^{\frac{2\gamma}{5}} \r.
\end{align*}
Thus one can choose $n_3=\max\{n_3',n_3''\}$ to get the following conclusion.
\begin{proposition}
Under Assumptions $\ref{as:joint distribution assumption}$ to $\ref{assumption: rate-type condition}$, for any $\gamma\in\l0,\tfrac{5}{4(\alpha_1+\alpha_2+3)}\r$, choosing 
$m=n^{\frac{1-2\gamma}{2\alpha_1+2\alpha_2+1}}$ (i.e.,  $c_1=\frac{1-2\gamma}{2\alpha_1+2\alpha_2+1}$) yields a positive constant
\begin{align*}
D_4:=D_4\l \|\Gamma^{-1}M\| ,\sigma_d(\Gamma^{-1}M) ,\gamma,\sigma_0,\sigma_1,d,\wt C,\lambda_d\l\sum\limits_{i=1}^d {\bs{\beta}}_i\otimes{\bs{\beta}}_i\r,\alpha_2\r 
\end{align*}
such that when $n$ is sufficiently large, we have:
\begin{align*}
\P\l\left\|P_{\mc{S}_{\Y|\X}}-P_{\widehat{\mc{S}}_{\Y|\X}^{(m)}}\right\| \leqslant D_4n^{-\frac{(2\alpha_2-1)(1-2\gamma)}{2(2\alpha_1+2\alpha_2+1)}} \r
\geqslant1-2\exp\l -\tfrac{D_0+1}{2}n^{\frac{2\gamma}{5}} \r,
\end{align*}
where $D_0$ and $D_1$ are defined in Proposition $\ref{prop:bound hatMmd Mm}$.
\end{proposition}
\noindent
% Theorem $\ref{theorem, total convergence rate}$ is a direct corollary of above proposition.
Define 
$$\mathtt F:=\left\{\left\|P_{\mc{S}_{\Y|\X}}-P_{\widehat{\mc{S}}_{\Y|\X}^{(m)}}\right\| \leqslant D_4n^{-\frac{(2\alpha_2-1)(1-2\gamma)}{2(2\alpha_1+2\alpha_2+1)}}\right\}.$$
Then 
\begin{align*}
 \mb E\left[\left\|P_{\mc{S}_{\Y|\X}}-P_{\widehat{ \mc{S}}_{\Y|\X}^{(m)}}\right\|^2\right] =&
  \mb E\left[\left\|P_{\mc{S}_{\Y|\X}}-P_{\widehat{ \mc{S}}_{\Y|\X}^{(m)}}\right\|^21_{\mathtt{F}}\right] +
   \mb E\left[\left\|P_{\mc{S}_{\Y|\X}}-P_{\widehat{ \mc{S}}_{\Y|\X}^{(m)}}\right\|^21_{\mathtt{F}^c}\right]\\ 
 \leqslant &
 D_4^2n^{-\frac{(2\alpha_2-1)(1-2\gamma)}{2\alpha_1+2\alpha_2+1}}+4\mb P\left( \mathtt F^c\right)\\
 \lesssim&n^{-\frac{(2\alpha_2-1)(1-2\gamma)}{2\alpha_1+2\alpha_2+1}}+\exp\l -\tfrac{D_0+1}{2}n^{\frac{2\gamma}{5}} \r\\
\lesssim&n^{-\frac{(2\alpha_2-1)(1-2\gamma)}{2\alpha_1+2\alpha_2+1}}.
\end{align*}
This completes the proof of  Theorem \ref{theorem, total convergence rate}.
\end{proof}







\section{Additional Simulation Results of Section \ref{sec:Synthetic}}
This section contains the additional  simulation results  of Sections \ref{sec:Synthetic}  when $\varepsilon\sim N(0,1)$.



We show the average $\mc D(\bs B;\bs{\wh B})$ with different $m$ or $\rho$ for three methods under $\mc M_1$ to $\mc M_3$ in Figure \ref{fig:error 3models,noise1},
where we mark minimal error in each model with red `$\times$'. The shaded areas represent the standard error associated with these estimates and all of them are less than  $0.01$. For FSFIR, the  minimal errors for $\mc M_1-\mc M_3$ are  $0.08,0.02,0.01$ respectively.
For TFSIR, the  minimal errors are  $0.08,0.02,0.01$ and for regularized FSIR,  the  minimal errors are $0.13,0.06,0.01$.  

% Figure environment removed


Figure \ref{fig:error 3models,noise1} shows that FSFIR attains the best performance among  all models. 
Moreover, FSFIR is easier to practice as it does not need a slice number $H$ in advance. 





%\setcounter{section}{0} % for constitent numbering with main paper submission
\end{toappendix}


\section{Introduction}

Large language models (LLMs) have enabled advances in text generation, few-shot learning, reasoning, protein sequence modeling, and other tasks~\citep{brown2020LLMfewshot,workshop2023bloom,meta2022opt}.
The massive size of these models---often reaching into hundreds of billions of parameters---requires sophisticated deployment methods and motivates research into efficient inference algorithms.

This work studies the post-training quantization of LLM parameters as a way to improve their runtime efficiency~\citep{dettmers2022int8,frantar2023gptq,park2023lutgemm,xiao2023smooth,yao2022zero,yuan2023rptq}.
Our key insight is that quantization can be most effective when weight and proxy Hessian matrices are {\em incoherent}---that the weights themselves are even in magnitude,  
and the directions in which it is important to have good rounding accuracy are not too large in any one coordinate.
Intuitively, incoherence can be thought of as a principled form of outlier reduction,
which makes it easier to adaptively round the weights to a finite set of compressed values.
We use this intuition to develop theoretically sound two-bit quantization algorithms that scale to LLM-sized models.

Specifically, we introduce quantization with incoherence processing (QuIP),
a new method motivated by the above insight. QuIP consists of two steps: 
(1)~an adaptive rounding~\cite{nagel2020up} procedure, which minimizes a quadratic proxy objective $\ell(\hat W) = \operatorname{tr}((\hat W - W) H (\hat W - W)^T)$ of the error between the original weights $W$ and the quantized weights $\hat W$ using an estimate of the Hessian $H$;
(2)~efficient pre- and post- processing that ensures that the weight and Hessian matrices are incoherent by multiplying them by a Kronecker product of random orthogonal matrices.
We denote ``incoherence processing'' as both the pre- and post- processing steps of our procedure.
Incoherence processing can be viewed as a form of outlier suppression across the weights and the activation space.

We complement our method with a theoretical analysis---the first for a quantization algorithm that scales to LLM-sized models---which analyzes the role of incoherence and shows that our quantization procedure is optimal within a general class of rounding methods. Interestingly, we find that QuIP without incoherence processing yields a more efficient implementation of an earlier algorithm, OPTQ~\cite{frantar2023gptq}; our paper thus also provides the first theoretical analysis for that method.

Empirically, we find that incoherence processing greatly improves the quantization of large models, especially at higher compression rates, and yields the first LLM quantization method that produces viable results using only two bits per weight. For large LLM sizes (>2B parameters), we observe small gaps between 2-bit and 4-bit compression that further decrease with model size, hinting at the feasibility of accurate 2-bit inference in LLMs.

\textbf{Contributions.}\;
In summary, this paper makes the following contributions: (1) we propose QuIP, a quantization method based on the insight that model parameters should ideally be incoherent; (2) we provide a theoretical analysis for a broad class of adaptive rounding methods that encompass QuIP and OPTQ; (3) we demonstrate that QuIP makes two-bit LLM compression viable for the first time.


\section{Related Work}
\section{Related Work}
\label{appsec: related work}
Bayesian causal discovery literature has primarily focused on inference in linear models with closed-form posteriors or marginalized parameters. Early works considered sampling directed acyclic graphs (DAGs) for discrete~\cite{cooper1992bayesian, madigan1995bayesian, heckerman2006bayesian} and Gaussian random variables~\cite{friedman2003being, tong2001active} using Markov chain Monte Carlo (MCMC) in the DAG space. However, these approaches exhibit slow mixing and convergence~\cite{eaton2012bayesian,grzegorczyk2008improving}, often requiring restrictions on number of parents~\cite{kuipers2017partition}. %Alternative exact dynamic programming methods are limited to small settings~\cite{koivisto2012advances}. 

Recent advances in variational inference~\cite{zhang2018advances} have facilitated graph inference in DAG space, with gradient-based methods employing the NOTEARS DAG penalty \cite{zheng2018dags}.\cite{annadani2021variational} samples DAGs from autoregressive adjacency matrix distributions, while \cite{lorch2021dibs} utilizes Stein variational approach \cite{liu2016stein} for DAGs and causal model parameters. \cite{cundy2021bcd} proposed a variational inference framework on node orderings using the gumbel-sinkhorn gradient estimator \cite{mena2018learning}. \cite{deleu2022bayesian,nishikawa2022bayesian} employ the GFlowNet framework \cite{bengio2021gflownet} for inferring the DAG posterior. Most methods, except\cite{lorch2021dibs} are restricted to linear models, while \cite{lorch2021dibs} has high computational costs and lacks DAG generation guarantees compared to our method.
% at least quadratic scaling complexity, both with respect to the number of nodes (due to the DAG penalty) as well as number of posterior samples. Our proposed approach instead has linear complexity with respect to number of posterior samples and does not require any additional DAG penalty.     

In contrast, \emph{quasi-Bayesian} methods, such as DAG bootstrap \cite{friedman2013data}, demonstrate competitive performance. DAG bootstrap resamples data and estimates a single DAG using PC \cite{spirtes2000causation}, GES \cite{chickering2002optimal}, or similar algorithms, weighting the obtained DAGs by their unnormalized posterior probabilities. Recent neural network-based works employ variational inference to learn DAG distributions and point estimates for nonlinear model parameters \cite{charpentier2022differentiable,geffner2022deep}.

\section{Quantization With Incoherence Processing: Adaptive Rounding Step }
\label{secQuIPquant}
This section introduces quantization with incoherence processing (QuIP), a new method consisting of:
(1)~an adaptive rounding step;
(2)~efficient pre- and post-processing that ensures weight and Hessian incoherence. We define and analyze step (1) in this section; the next section focuses on step (2).

Following existing state-of-the-art post-training quantization methods, we round weights per-layer by minimizing the ``adaptive rounding'' proxy objective, as in \citet{nagel2020up},
\begin{equation}
    \label{eqnAdaptiveEll}
    \textstyle
    \ell(\hat W)
    =
    \mathbf{E}_x\left[ \norm{ (\hat W - W) x }^2 \right]
    =
    \trace{ (\hat W - W) H (\hat W - W)^T }.
\end{equation}
Here, $W \in \mathbb{R}^{m \times n}$ is the original weight matrix for a given linear layer, $\hat W \in \mathbb{R}^{m \times n}$ are the quantized weights, $x \in \mathbb{R}^n$ is an input vector drawn uniformly at random from a calibration set, and $H$ is the second moment matrix of these vectors, interpreted as a proxy Hessian. Crucially, this formulation lets the quantization be run in parallel across neurons, which 
is tractable for
large language models~\citep{frantar2023gptq}. 
For simplicity, we will focus in this section on rounding to the integers; subsequent sections will extend the analysis to finite grids.


\subsection{LDLQ: An Optimal Adaptive Rounding Method}
Our strategy is to define a family of adaptive rounding methods for optimizing objective (\ref{eqnAdaptiveEll}) and then define LDLQ, the optimal method within that class. Our defined methods iteratively perform the following update for $k=1,2,...,n$:
\[
    \hat W_k = \round( W_k + (W_{1:(k-1)} - \hat W_{1:(k-1)}) a_k),
\]
where $W_k$ denotes the $k$-th column, $W_{1:(k-1)}$ denotes the first $k-1$ columns, the subroutine $\mathcal{Q}$ denotes either nearest rounding or standard unbiased rounding to the integers (which rounds up or down such that $\Exv{\mathcal{Q}(z)} = z$), and $a_k \in \R^{k - 1}$ is some sequence of vectors. 
This scheme rounds columns one at a time; at each step, it adds a ``correction'' term that is a linear function of the residual from the rounding we have done so far. The final $\hat W$ satisfies the following matrix equation:
\begin{equation}
  \label{eqnVectorQuant}
  \hat W = \round( W + (W - \hat W) U),  
\end{equation} 
where $U$ is a strictly upper-triangular matrix whose columns are the vectors $a_k$ and $\mathcal{Q}$ acts elementwise. 
Because $U$ is upper-triangular, $\hat W_k$ only depends on $\hat W_{1:(k-1)}$.

If we let $\eta = \mathcal{Q}( W + (W - \hat W) U) - ( W + (W - \hat W) U)$ denote the quantization error of $\mathcal{Q}$, we find that $\hat W - W = \eta (U + I)^{-1}$ and we can rewrite objective (\ref{eqnAdaptiveEll}) as
\begin{equation}
    \label{eqnProxyEq}
    \operatorname{tr}((\hat W - W) H (\hat W - W)^T)
    =
    \operatorname{tr}(\eta (U + I)^{-1} H (U + I)^{-T}  \eta^T ).
\end{equation} 

\paragraph{The LDLQ Method}
How should we specify $U$, the linear feedback from the quantization error of preceding columns in \eqref{eqnVectorQuant}?
Equation~\ref{eqnProxyEq} provides an answer.
If we choose $U \gets \grave U$ such that the LDL decomposition of $H$ is 
\begin{equation}
    \label{eqnH=LDL}
    H = (\grave U + I) D (\grave U + I)^T,
\end{equation}
where $D$ is a (non-negative) diagonal matrix and $\grave U$ is upper unit triangular, then the terms $(U + I)$ in Eq.~\eqref{eqnProxyEq} cancel.
We denote as LDLQ the rounding procedure in Eq.~\eqref{eqnVectorQuant} with $U \gets \grave U$ as the LDL assignment from Eq.~\eqref{eqnH=LDL}. 
We will now see that the LDL assignment of $U$ is in fact optimal. 

\newcommand{\titleLDLQopt}{Deriving the Optimality of the LDLQ Adaptive Rounding Procedure} % for apxproof subsection titles
\subsection{\titleLDLQopt}\label{secLDLopt}
\begin{toappendix}
\subsection*{Subsection~\ref{secLDLopt} (\titleLDLQopt)} % for aprxproof subsection titles
\end{toappendix}

In order to reason about optimality, 
we consider weights which are worst and average-case for the proxy loss.
Let $\alglin$ denote a rounding method,
and let $\alglin(W,H)$ be the resulting quantized weights.
Define the \emph{worst-case} ($\Lworst$) and \emph{average} ($\Lavg$) proxy losses with respect to the input weights as
\begin{align}
    \Lworst(\alglin, H) &= \sup_{W \in \R^{m \times n}} \Exv{ \trace{ ( \alglin(W, H) - W) H (\alglin(W, H) - W)^T } } 
    \label{eqnLworst}\\
    \Lavg(\alglin, H)  &= \Exv[{W \sim \operatorname{Unif}[0,1]^{m \times n}}]{ \trace{ ( \alglin(W, H) - W) H (\alglin(W, H) - W)^T } }.
    \label{eqnLavg}
\end{align}

\begin{theoremrep}\label{thmLDLopt}
$\ldl$ is worst and average-case optimal amongst rounding methods which specify the linear feedback $U$ as a function of $H$ (not of $W$), and when rounding to the integers.
That is, for all rounding methods $\alglin$ in the class described by Eq.~\eqref{eqnVectorQuant}, for all positive semi-definite $H$, and for $\round$ as either nearest or stochastic rounding,
\[
    \textstyle
    \frac{m}{4} \operatorname{tr}(D) = \Lworst(\ldl, H) \leq \Lworst(\alglin, H)
    \;\;\text{and}\;\;
    \frac{m}{c} \operatorname{tr}(D) = \Lavg(\ldl, H) \leq \Lavg(\alglin, H),
\]
where $D$ is the matrix from the LDL decomposition of $H$, and $c=12$ for nearest, $c=6$ for stochastic.
\end{theoremrep}
\begin{proof}
Let $X$ be the strictly upper triangular matrix associated with the rounding procedure $\alglin$ such that $U \gets X$ in Eq.~\eqref{eqnVectorQuant}.
Let $B \equiv (X + I)^{-1} (\grave U + I)$ where $\grave U$ is from the LDL decomposition of $H$ in Eq.~\eqref{eqnH=LDL}.
The proxy loss is then, 
\begin{align}
    \trace{ (\alglin(W, H) - W) H (\alglin(W, H))^T }
    &\stackrel{\eqref{eqnProxyEq},\eqref{eqnH=LDL}}{=}
    \trace{ \eta (X + I)^{-1} (\grave U + I) D (\grave U + I)^T (X + I)^{-T} \eta^T }\nonumber\\
    &= 
    \trace{ \eta B D B^T \eta^T }.
    \label{eqnThm1proofa} 
\end{align}
With the LDL assignment of $U$, we further have that,
\begin{equation}
    \label{eqnThm1proofb} 
    \trace{ \eta B D B^T \eta^T }
    =
    \trace{ \eta D \eta^T }.
\end{equation}

First, consider the worst-case loss, $\Lworst$.
The goal is to construct a particularly bad case where the entries of $\tilde W$ are $1/2 \pm \epsilon$, and thus when rounding to the integers we will always have error 1/2.
Construct a weight matrix $\tilde W \in \R^{m \times n}$ such that each entry satisfies,
\[
    \tilde W_{ij} 
    = 
    \begin{cases} 
        0.5 - \epsilon &w.p. \; 1/2 \\ 
        0.5 + \epsilon &w.p. \; 1/2 
    \end{cases}
    \;\;\Rightarrow\;\;
    \eta_{ij}
    =
    \begin{cases} 
        +0.5  &w.p. \; 1/2 \\ 
        -0.5 &w.p. \; 1/2 
    \end{cases},
\]
and the quantization errors $\eta \in \R^{m \times n}$ are for each entry $\{+1/2, -1/2\}$ with equal probability.
For this particular $\tilde W$, $\alglin$ achieves proxy loss $\Lworst(\alglin, H) \stackrel{\eqref{eqnThm1proofa}}{=} \Exv{ \trace{ \eta B D B^T \eta^T } } = \frac m4 \trace{ B D B^T }$, with $\round$ as either nearest or stochastic rounding.
It follows from the supremum in the definition of $\Lworst$ in Eq.~\eqref{eqnLworst} that,
$
    \Lworst(\alglin, H) 
    \geq
    \frac m4 \trace{ B D B^T }
$.
For the LDL assignment of $U$, the worst case expected quantization error rounding to the integers is $1/2$.
Therefore,
$
    \Lworst(\ldl, H)
    \stackrel{\eqref{eqnThm1proofb}}{=}
    \frac m4 \trace{ D }
$, again for $\round$ as either nearest or stochastic rounding.
$B$ must be a unit triangular matrix since it is the product of unit triangular matrices.
Therefore $\trace{ B D B^T }$ is minimized when $B = I$, and 
\[
    \Lworst(\ldl, H)
    \leq
    \Lworst(\alglin, H).
\]

Next, consider the average loss, $\Lavg$, where $W \sim Unif[0,1]^{m \times n}$.
For $\round$ as nearest rounding, the entries of the quantization error $\eta$ are $Unif[-\frac 12, \frac 12]$, because each entry is independent and uniformly distributed.
It follows that for any entry of $\eta$, $\Exv{ \eta_{ij}^2 } = \int_{-1/2}^{1/2} x^2 dx = \frac{1}{12}$.
Therefore,
$
    \Lavg(\alglin, H)
    \stackrel{\eqref{eqnThm1proofa}}{=}
    \Exv[{W \sim Unif[0,1]^{m \times n}}]{\trace{ \eta B D B^T \eta^T } }
    =
    \frac{m}{12} \trace{ B D B^T }
$.
For $\round$ as stochastic rounding, the entries of the quantization error $\eta$ are $Unif[-1,1]$.
It follows that for any entry of $\eta$, $\Exv{ \eta_{ij}^2 } = \int_0^1 x (1-x) dx = \frac16$.
Note that for stochastic rounding, the quantization error will be $x$ with probability $(1-|x|)$.
Therefore,
$
    \Lavg(\alglin, H)
    = 
    \frac m6 \trace{ B D B^T }
$. 
Based on these same calculations of $\Exv{ \eta_{ij}^2 }$, we have that $\Lavg(LDL, H) \stackrel{\eqref{eqnThm1proofa}}{=} \frac{m}{12} \trace{ D }$ with $\round$ as nearest , and $= \frac m6 \trace { D }$ with $\round$ as stochastic rounding.
By the same reasoning on the minimization of $\trace{ B D B^T }$,
\[
    \Lavg(\ldl, H)
    \leq
    \Lavg(\alglin, H).
\]

\end{proof}

\textbf{Remarks.}
The number of rows being quantized is $m$, and each quantization method operates across the $n$ entries of each row.
For all rounding methods described by Eq.~\eqref{eqnVectorQuant}, and for all positive semi-definite $H$, $\round$ as nearest rounding achieves the same worst-case proxy loss as stochastic rounding, but achieves better average proxy loss.


Moving beyond a generic algorithm $\alglin$ within our framework, we consider the common baselines of nearest and stochastic rounding.
These methods are represented within our framework by choosing the appropriate $\round$ subroutine, and setting all entries of the linear feedback to zero.
For these baseline methods, their optimality gap to LDLQ is governed by $\trace{D}$ vs. $\trace{H}$.
For any non-diagonal $\tilde H \succeq 0$, LDLQ achieves strictly lower worst and average-case proxy loss because $\trace{D} < \operatorname{tr}(\tilde {H})$.
Let $\mathcal{B} = \{\near, \stoch\}$. Then,
$\Lworst(\ldl, \tilde H) < \Lworst(\stoch, \tilde H)$
and
$\Lavg(\ldl, \tilde H) < \Lavg(\mathcal{B}, \tilde H)$.
Across OPT models 125m to 2.7b,
$\trace{D}/\trace{H} \leq 0.65$---empirically verifying that the gap is not insignificant.
See Supplement~\ref{suppAddResults} for full details.


\newcommand{\titleIncoherOpt}{Incoherence: Optimality with a Spectral Bound}
\subsection{\titleIncoherOpt}\label{secIncoherOpt}
\begin{toappendix}
\subsection*{Subsection~\ref{secIncoherOpt} (\titleIncoherOpt)}
\end{toappendix}


\begin{wrapfigure}{r}{0.35\textwidth}
\vspace{-4em}
\begin{center}
% Figure removed
\end{center}
\vspace{-1em}
\caption{$\eig H$ from OPT-2.7b.}
\label{figEigH}
\vspace{-1em}
\end{wrapfigure}
Theorem~\ref{thmLDLopt} gives exact expressions for the proxy loss, albeit with $\trace{D}$, which can be difficult to reason about.
In Figure~\ref{figEigH}, we
empirically observe that $H$ is approximately low-rank: we visualize the spectrum of several randomly chosen $H$ from OPT-2.7b, and observe that the spectrum decays rapidly.
In fact, across all layers of OPT-125m to 2.7b models, a vast majority of $H$ matrices have fewer than a quarter of eigenvalues $>1\%$ of the max eigenvalue; see Supplement~\ref{suppAddResults} for full details.
Given this observation about the low rank of $H$, can we bound the behavior of LDLQ, and thus $\trace{D}$, using the spectrum of $H$?

We do this building on a variant of the incoherence assumption that is specialized to our case~\cite{desa2015matrix, jain2013complete}.
\begin{definitionrep}
We say a symmetric Hessian matrix $H \in \R^{n \times n}$ is $\mu$-incoherent if it has an eigendecomposition $H = Q \Lambda Q^T$ such that for all $i$ and $j$,
$\Abs{ Q_{ij} } = \Abs{ e_i^T Q e_j } \le \mu / \sqrt{n}$. By extension, we say a weight matrix $W \in \mathbb{R}^{m \times n}$ is $\mu$-incoherent if all $i$ and $j$,
$\Abs{ W_{ij} } = \Abs{ e_i^T W e_j } \le \mu \norm{W}_F / \sqrt{mn}$.
\end{definitionrep}
Note that ``most'' $n \times n$ matrices are incoherent with $\mu = \mathcal{O}(\sqrt{\log n}) = \tilde{\mathcal{O}}(1)$ because a random orthogonal matrix has entries with squared-magnitudes that concentrate around their mean of $1/n$.
Wanting $W$ to be incoherent is very natural: a small bound on the magnitude of its entries means that we do not need to scale it as much to make it fit in the finite range of representable low-precision numbers.
Making $H$ incoherent is less intuitive, but its utility is motivated by the following lemma.

\begin{toappendix}    
\begin{lemma}
\label{lemmaSigmaRecurrence}
Let $H \in \R^{n \times n}$ be a positive semi-definite symmetric matrix, and let $a_1,\ldots,a_n$ be a sequence of vectors in $\R^n$. Consider the recurrence given by $\Sigma_0 = 0 \in \R^{n \times n}$ and from $k=0$ to $n-1$
\[
    \Sigma_{k+1} = (I - e_k a_k^T) \Sigma_k (I - a_k e_k^T) + e_k e_k^T.
\]
Let $\ell(a_1,\ldots,a_n) = \trace{H \Sigma_n}$.
Then if $H = L D L^T$ is the LDL decomposition of $H$, a global minimum of $\ell$ occurs when $a_k$ is the $k$th column of $L$, and at this minimum, $\ell = \trace{D}$.
\end{lemma}
\begin{proof}
First observe that at step $k$, $\Sigma_k$ will be $0$ in all entries $(\Sigma_k)_{ij}$ if $\min(i,j) \ge k$. This means that changing the last $n-k$ entries of $a_k$ does not change $\Sigma$ (or $\ell$) at all. Without loss of generality, set those entries of $a_k$ to $0$. If $A$ is the matrix whose $k$th row is $a_k$, this is equivalent to saying that $A$ is strictly lower triangular.

Next, let $\eta$ be a random Gaussian sampled from $\mathcal{N}(0, I)$, and consider the recurrence given by $x_0 = 0 \in \R^n$ and
\[
    x_{k+1} = x_k - e_k a_k^T x_k + e_k e_k^T \eta.
\]
It's straightforward to see that $\Sigma_k = \Exv{ x_k x_k^T }$. But it's also easy to see that the step-$k$ update only modifies/assigns the $k$th entry of $x$, and does so based only on earlier entries of $x$. Since $e_k^T x_k = 0$, and no later step assigns the $k$-or-lower entries of $x$,
\[
    e_k^T x_n = e_k^T x_{k+1} = 0 - a_k^T x_k + e_k^T \eta = - a_k^T x_n + e_k^T \eta,
\]
which in vector form yields
\[
    (I + A) x_n = \eta.
\]
In particular, this immediately implies that
\[
    \Sigma_n = (I + A)^{-1} (I + A)^{-T}
\]
and
\[
    \ell = \trace{H \Sigma_n} = \trace{ (I + A)^{-T} H (I + A)^{-1} } = \trace{ B^{-T} H B^{-1} }.
\]
where $B = I+A$.
Differentiating with respect to $B$ in strictly lower triangular direction $\Delta$ (the only direction in which we have degress of freedom, since the diagonal of $B$ must be unit) yields 
\[
   -2 \trace{ B^{-T} H B^{-1} \Delta B^{-1} }.
\]
It's not hard to see that if $H = L D L^T$ is the LDL decomposition of $H$, and $B^T = L$, that the gradient is
\[
   -2 \trace{ D \Delta B^{-1} }
   =
   -2 \trace{ \Delta B^{-1} D }
   =
   -2 \langle \Delta^T, B^{-1} D \rangle.
\]
Since $\Delta^T$ is strictly upper triangular, but $B^{-1} D$ must be lower triangular, this is $0$ so we have a minimum. The uniqueness of this minimum (up to assignments of the lower-triangular elements of $A$ or $B$, which have no effect on $\ell$) also immediately follows from the recurrence relation. This implies the minimum is global. This is what we wanted to show.
\end{proof}
\end{toappendix}


\begin{lemmarep}\label{lemDincoherentBd}
Let $H \in \R^{n \times n}$ be a $\mu$-incoherent positive semi-definite symmetric matrix
and let $H = (\grave U + I) D (\grave U + I)^T$ be its LDL Cholesky decomposition, where $\grave U$ is a strictly upper triangular matrix and $D$ is a (non-negative) diagonal matrix. Then,
\[
    \trace{D} \le \frac{\mu^2}{n} \trace{ H^{1/2} }^2.
\]
\end{lemmarep}
\begin{proof}
By continuity of $\trace{D}$ and $\trace{H^{1/2}}$, it suffices to prove the lemma for positive definite $H$.
First, the closure of positive definite symmetric matrices is the set of positive semi-definite symmetric matrices.
Second, consider the set of $H$ that are positive definite and satisfy $\frac{\mu^2}{n} \trace{H^{1/2}}^2 - \trace{D} \geq 0$, i.e. are non-negative.
The closure of this set (i.e. $H \succeq 0$) must also satisfy that the inequality is non-negative.

Let $H = Q \Lambda Q^T$ be the eigendecomposition of $H$.
First, observe that by incoherence,
\[
    e_k^T H^{1/2} e_k
    =
    \sum_{i=1}^n \lambda_i^{1/2} (e_i^T Q e_k)^2
    \le
    \frac{\mu^2}{n} \sum_{i=1}^n \lambda_i^{1/2}
    =
    \frac{\mu^2}{n} \trace{H^{1/2}}.
\]

Set
\[
    \alpha = \frac{\mu^2}{n} \trace{H^{1/2}},
\]
and consider the recurrence from Lemma~\ref{lemmaSigmaRecurrence} with
\[
    a_k = \frac{H^{1/2} e_k}{\alpha}
\]
Then
\[
    \Sigma_{k+1} = \left(I - \alpha^{-1} e_k e_k^T H^{1/2} \right) \Sigma_k \left(I - \alpha^{-1}H^{1/2} e_k e_k^T \right) + e_k e_k^T.
\]
Suppose by way of induction that for some scalar the covariance $\Sigma_k \preceq \alpha H^{-1/2}$. For the base case, this obviously holds since $\Sigma_0 = 0$. At step $k$,
\begin{align*}
    \Sigma_{k+1} 
    &\preceq
     \left(I - \alpha^{-1} e_k e_k^T H^{1/2} \right) \alpha H^{-1/2} \left(I - \alpha^{-1}H^{1/2} e_k e_k^T \right) + e_k e_k^T
    \\&=
    \alpha H^{-1/2} - 
    2 e_k e_k^T
    +
    \alpha^{-1} e_k e_k^T H^{1/2} e_k e_k^T
    +
    e_k e_k^T
    \\&\preceq
    \alpha H^{-1/2}.
\end{align*}
Note that with this assignment,
\[
    a_k^T \Sigma_k a_k
    \le
    (\alpha^{-1} e_k^T H^{1/2}) (\alpha H^{-1/2}) (\alpha^{-1} H^{1/2} e_k)
    =
    \alpha^{-1} e_k^T H^{1/2} e_k
    \le
    1.
\]


So, by induction it follows that
\[
    \Sigma_n \preceq \frac{\mu^2}{n} \trace{H^{1/2}} \cdot H^{-1/2},
\]
and so
\[
    \trace{H \Sigma_n} \le \frac{\mu^2}{n} \trace{H^{1/2}} \trace{H \cdot H^{-1/2}}
    = \frac{\mu^2}{n} \trace{H^{1/2}}^2.
\]
But from Lemma~\ref{lemmaSigmaRecurrence}, we know that $\trace{D}$ is the global minimum of $\trace{H \Sigma_n}$ for any assignment of $a_k$. This immediately gives us the desired result.
\end{proof}
To the best of our knowledge, this is a novel result  using incoherence to obtain a bound on $\trace{D}$ that depends only on the spectrum of $H$.
To help interpret this result, we derive explicit proxy losses for plain nearest and stochastic rounding, which we will then compare to what LDLQ gets via Lemma~\ref{lemDincoherentBd}.

\begin{lemmarep}\label{lemNearStochProxyH}
Let $H$ be symmetric positive definite.
In the worst case stochastic rounding achieves $\Lworst(\stoch, H) = (m/4) \trace{H}$.
In the average case nearest and stochastic rounding achieve $\Lavg(\{\near, \stoch\}, H) = (m/c) \trace{H}$, where $c=12$ for nearest, and $c=6$ for stochastic.
\end{lemmarep}
\begin{proof}
For nearest and stochastic rounding, set the linear feedback $U$ in Eq.~\eqref{eqnVectorQuant} to be zero.
Stochastic rounding achieves worst-case loss,
\begin{align}
    \Lworst(\stoch, H)
    &\stackrel{\eqref{eqnProxyEq}}{=}
    \sup_{W \in \R^{m \times n}} \Exv{ \trace{ \eta H \eta^T } }
    = \frac m4 \trace{ H }.
    \label{eqnThm3proofb}
\end{align}
For the average-case proxy loss, recall the computations of $\Exv{ \eta_{ij}^2 }$ from the proof of Theorem~\ref{thmLDLopt}.
\begin{align}
    \Lavg(\near, H)
    &\stackrel{\eqref{eqnProxyEq}}{=}
    \Exv[{W \sim Unif[0,1]^{m \times n}}]{ \trace{ \eta H \eta^T } }
    = \frac{m}{12} \trace{ H } 
    \label{eqnThm3proofc}\\
    \Lavg(\stoch, H)
    &\stackrel{\eqref{eqnProxyEq}}{=}
    \Exv[{W \sim Unif[0,1]^{m \times n}}]{ \trace{ \eta H \eta^T } }
    = \frac m6 \trace{ H }.
    \label{eqnThm3proofd}
\end{align}
\end{proof}



To interpret this result, consider $H$ rank-$k$ with $\mu^2 k < n$.
By Cauchy-Schwarz, $\operatorname{tr}( H^{1/2} )^2 \leq k \trace{ H }$. Combining Lemma~\ref{lemDincoherentBd} with the LDLQ proxy losses of Theorem~\ref{thmLDLopt} and comparing with Lemma~\ref{lemNearStochProxyH},
{\small\begin{align*}
   \Lworst(\ldl, H) 
   &\leq 
   \frac{m \mu^2}{4n} \trace{ H^{1/2} }^2 
   \le
   \frac{m \mu^2 k}{4n} \trace{H}
   \leq
   \frac{m}{4} \trace{H}
   =
   \Lworst(\stoch, H) \\
   \Lavg(\ldl, H) 
   &\leq 
   \frac{m \mu^2}{cn} \trace{ H^{1/2} }^2 
   \le
   \frac{m \mu^2 k}{cn} \trace{ H } 
   \leq 
   \frac{m}{c} \trace{H}
   =
   \Lavg(\mathcal{B}, H),
\end{align*}}
where $\mathcal{B} \in \{\near, \stoch\}$, and $c$ is as given in Theorem~\ref{thmLDLopt}.
This shows that for sufficiently low-rank $H$, LDLQ is asymptotically better than plain nearest and stochastic rounding by a factor of $\mu^2 k / n$.

\newcommand{\titleNoIncohSpectral}{Without incoherence: no improvement with a spectral bound}
\paragraph{\titleNoIncohSpectral.}
By assuming incoherence, we were able to show LDLQ gets an asymptotically better bound in terms of just the spectrum of $H$. We might ask: \emph{was the incoherence assumption necessary to get this result?} The following theorem answers this question in the affirmative by showing that without incoherence, the best spectral bound for LDLQ cannot differentiate it from the nearest and stochastic rounding baselines.
\begin{toappendix}
\subsection*{\titleNoIncohSpectral}
\end{toappendix}


\begin{theoremrep}\label{thmOPTQequivNearStoch}
Consider all $\tilde H$ with the same spectrum as $H$.
For any positive semi-definite $H$, the following holds.
On the worst-case loss $\ldl$ achieves the same error as stochastic rounding,
\[
    \sup_{\tilde H s.t. \eig{\tilde H} = \eig{H}} \Lworst(\ldl, \tilde H)
    =
    \Lworst(\stoch, H) 
    = 
    \frac m4 \trace{ H }.
\]
On the average-case loss $\ldl$ achieves the same error as the corresponding rounding routine. Let $\mathcal{B} = \{\near, \stoch\}$ and $c=12$ for nearest, $c=6$ for stochastic.
\[
    \sup_{\tilde H s.t. \eig{\tilde H} = \eig{H}} \Lavg(\ldl^\ast, \tilde H)
    =
    \Lavg(\mathcal{B}, H) 
    = 
    \frac{m}{c} \trace { H }.
\]
\end{theoremrep}
\begin{proof}
See Lemma~\ref{lemNearStochProxyH} for calculations on the proxy loss for nearest and stochastic rounding.

For LDLQ, we will derive lower and upper bounds on $\sup_{\tilde H s.t. \eig{\tilde H} = \eig{H}} \Lworst(\ldl, \tilde H)$ and $\sup_{\tilde H s.t. \eig{\tilde H} = \eig{H}} \Lavg(\ldl, \tilde H)$, and show they are equal.
To construct a lower bound, consider $\tilde H = I \Lambda I$ where $\Lambda$ are the eigenvalues of $H$.
This decomposition is also the LDL decomposition of $\tilde H$, rewritten as $\tilde H=(U + I) D (U + I)^{-1}$.
It follows that $\trace{ D } = \trace{ \tilde H }$ for this $\tilde H$.
Combine this result with the worst and average-case losses calculated in the proof of Theorem~\ref{thmLDLopt}.
For the worst-case loss from the proof of Theorem~\ref{thmLDLopt}, $\geq \frac m4 \trace{H}$.
The lower bound for the average-case loss is $\geq \frac{m}{12} \trace{H}$ for $\round$ as nearest, and $\geq \frac m6 \trace{H}$ for $\round$ as stochastic.
Now upper bounds are derived using the preceding calculations in Eq.~\eqref{eqnThm3proofb}-\eqref{eqnThm3proofd}, and using the worst and average-case optimality of LDLQ proven in Theorem~\ref{thmLDLopt}.
The lower and upper bounds are tight, proving our result.
\end{proof}

Note that the worst case for comparing LDLQ against these baselines occurs when $H$ is diagonal, see Theorem~\ref{thmLDLopt} and Lemma~\ref{lemNearStochProxyH}.
Assuming incoherence as we do is a natural way to exclude such cases.



\section{Quantization With Incoherence Processing: Incoherence Processing Step }

\begin{algorithm}[t]
\caption{QuIP - Incoherence Pre-Processing}\label{algQuIPpre}
\begin{algorithmic}[1]
\Require $b \in \mathbb{N}$, $H \in \mathbb{R}^{n \times n}$ SPD, original $W \in \mathbb{R}^{m \times n}$, $\rho \in \R_+$, $\alpha \in [0,1]$
\State \textbf{seeded sample} random two-factor orthogonal matrices $U \in \mathbb{R}^{m \times m}$ and $V \in \mathbb{R}^{n \times n}$
\State $H = H + \alpha * \operatorname{mean}(\operatorname{diag}(H)) I$ \Comment{from OPTQ}
\State $\tilde D \gets \sqrt[4]{ \diag{H} / \diag{W^T W} }$ \Comment{$\sqrt[4]{\;\;}$ applies element-wise}
\State $W \gets W \tilde D$; 
\;
$H \gets \tilde D^{-1} H \tilde D^{-1}$ 
\Comment{diagonal rescaling}\label{alg1rescale_a}
\State $W \gets U W V^T$;
\;
$H \gets V H V^T$
\Comment{incoherence}\label{alg1incoherent_a}
\State $s \gets \rho \|W\|_F / \sqrt{mn}$;
\;
$W \gets \frac 12 (\frac 1s W + 1)$
\Comment{reduced quantization range due to incoherency}\label{alg1reducedquant}
\State $W \gets \operatorname{clamp}(W * (2^b-1), 0, 2^b-1)$ \Comment{rescale $W$ to lie within $[0, 2^b-1]$}
\State \Return $W, H, s, \tilde D$
\end{algorithmic}
\end{algorithm}

\begin{algorithm}[t]
\caption{QuIP - Incoherence Post-Processing}\label{algQuIPpost}
\begin{algorithmic}[1]
\Require $b \in \mathbb{N}$, $H \in \mathbb{R}^{n \times n}$ SPD, quantized $W \in [0, 2^b-1]^{m \times n}$, $s \in \R$ \& $\tilde D \in \R^{n \times n}$ (Alg~\ref{algQuIPpre})
\State \textbf{seeded sample} random two-factor orthogonal matrices $U \in \mathbb{R}^{m \times m}$ and $V \in \mathbb{R}^{n \times n}$
\State $W \gets s * \left( ( W / (2^b - 1) ) * 2 - 1 \right)$
\State $W \gets U^T W V$;
\;
$H \gets V^T H V$
\Comment{revert incoherence}
\State \Return $W \gets W \tilde D^{-1}$ \Comment{revert diagonal rescaling}
\end{algorithmic}
\end{algorithm}



Next, we leverage the above incoherence analysis to introduce \emph{incoherence processing}, the second step of the QuIP algorithm.
Our strategy will be to pre-process weight and Hessian matrices to ensure the favorable incoherence properties outlined above.
One straightforward way to make a symmetric matrix incoherent is to conjugate it by a uniform random orthogonal matrix: this will result in each of its eigenvectors being a random unit vector, whose entries will concentrate around magnitude $n^{-1/2}$. 

Specifically, let $U \in \R^{m \times m}$ and $V \in \R^{n \times n}$ be two random orthogonal matrices.
(Let's temporarily ignore how these matrices are generated, or how we would efficiently perform inference.)
We ensure the weight and Hessian are incoherent with high probability through random orthogonal multiplications $\tilde H \gets V H V^T$ and $\tilde W \gets U W V^T$.
Importantly, this transformation preserves the proxy quadratic form since $\operatorname{tr}(\tilde W \tilde H \tilde W^T) = \operatorname{tr}( (U W V^T) (V H V^T) (V W^T U^T) ) = \operatorname{tr}( W H W^T )$.

\newcommand{\titleFastOrth}{Incoherence via Efficient Orthogonal Multiplication}
\subsection{\titleFastOrth}\label{secFastOrth}
\begin{toappendix}
\subsection*{Subsection~\ref{secFastOrth} (\titleFastOrth)}
\end{toappendix}
If all we wanted to do was to store or transmit the weights of the quantized neural network, the above procedure would introduce no overhead, since we can generate a random orthogonal matrix from a seed---making it essentially free to store. However, for running \emph{inference} on a DNN, we need to multiply by the  weight matrix $W$, and here the need to manifest and multiply by $n \times n$ random orthogonal matrices $U, V$ would be prohibitive.

To handle this, we propose to instead use a distribution over random orthogonal matrices for which multiplication is fast. Let $n = pq$ be a factorization of $n$ (where $p \approx q \approx \sqrt{n}$), and set $U = U_L \otimes U_R$ where $U_L$ is sampled uniformly from the $p \times p$ orthogonal matrices and $U_R$ is sampled uniformly from the $q \times q$ orthogonal matrices.  
Multiplication of a vector $x \in \R^n$ by the matrix $U$ can be accomplished by reshaping to a $p \times q$ matrix, multiplying on the left by $U_L$ and the right by $U_R^T$, and then reshaping back: this takes $O(n (p + q)) = o(n^2)$ operations. Using more than two factors in this way is also possible, but using two suffices to make this preprocessing asymptotically non-dominant.

\begin{toappendix}    
\begin{lemma}[Theorem 2.4 from~\citet{lalley2018prob}
]
There exist constants $C$ and $A$ independent of $n$ such that for any function $F$ from the unit sphere in $n$ dimensions to $\R$ that is 1-Lipschitz relative to the Riemannian metric on the sphere,
\[
    \mathbf{P}_{x \sim \mathcal{S}_n}\left( F(x) - \mathbf{E}_{x \sim \mathcal{S}_n}[F(x)] \ge t \right)
    \le
    C \exp\left( -\frac{n t^2}{A} \right)
\]
\label{lemmaSphere}
\end{lemma}

\begin{lemma}
    Let $B \in \mathbb{R}^{m \times n}$ be a matrix, and let $x$ be a random vector uniformly distributed on the unit sphere in $\R^n$. Then there exist global constants $A > 0$ and $C > 0$ independent of $m$ and $n$ such that
    \[
        \mathbf{P}\left( \norm{Bx}^2 \ge \frac{A \norm{B}_F^2}{n} \log\left( \frac{C}{\delta} \right) \right)
        \le
        \delta,
    \]
    \label{lemmaOneStep}
\end{lemma}
\begin{proof}
Let
\[
    F(x) = \frac{\norm{Bx}}{\norm{B}_F}.
\]
Observe that
\[
    \nabla F(x) = \frac{B^T B x}{\norm{B x} \cdot \norm{B}_F},
\]
and so
\[
    \norm{\nabla F(x)} \le 1.
\]
Also observe that for $x$ drawn uniformly from the sphere in $n$ dimensions,
\[
    \Exv{F(x)} \le \sqrt{\Exv{F(x)^2}} = \frac{1}{\norm{B}_F} \cdot \sqrt{\Exv{ \norm{B x}^2 }}
    =
    \frac{1}{\sqrt{n}}.
\]
So, applying Lemma~\ref{lemmaSphere},
\[
    \mathbf{P}\left( \frac{\norm{Bx}}{\norm{B}_F} - \frac{1}{\sqrt{n}} \ge t \right)
    \le
    C \exp\left( -\frac{n t^2}{A} \right).
\]
If we let $\delta$ be
\[
    \delta = C \exp\left( -\frac{n t^2}{A} \right),
\]
then
\[
    \frac{A}{n} \log\left( \frac{C}{\delta} \right) = t^2
\]
Trivially, then, for some modified global constants $A'$ and $C'$,
\[
    \frac{A'}{n} \log\left( \frac{C'}{\delta} \right) = \left(t + \frac{1}{\sqrt{n}} \right)^2
\]
This means that
\[
    \mathbf{P}\left( \frac{\norm{Bx}^2}{\norm{B}_F^2} \ge \frac{A'}{n} \log\left( \frac{C'}{\delta} \right) \right)
    \le
    \delta,
\]
i.e.
\[
    \mathbf{P}\left( \norm{Bx}^2 \ge \frac{A' \norm{B}_F^2}{n} \log\left( \frac{C'}{\delta} \right) \right)
    \le
    \delta,
\]
This is what we wanted to prove.
\end{proof}
\end{toappendix}

\begin{lemmarep}\label{lemFastInco}
Let $H$ be a positive semi-definite matrix on $\R^{n \times n}$ and $W$ a matrix on $\R^{m \times n}$, and suppose that $m = p_1 \cdot p_2 \cdots p_k$ and $n = q_1 \cdot q_2 \cdots q_k$. Let $U_1, U_2, \ldots, U_k, V_1, V_2, \ldots, V_k$ be independent random orthogonal matrices on $\R^{p_i \times p_i}$ and $\R^{q_i \times q_i}$ respectively.
Set $U$ as the Kronecker product $U = U_1 \otimes U_2 \otimes \cdots \otimes U_k$ and $V$ as $V = V_1 \otimes V_2 \otimes \cdots \otimes V_k$
Then $V H V^T$ is $\mu_H$-incoherent with probability at least $1 - \delta$, and $U W V^T$ is $\mu_W$-incoherent with probability at least $1 - \delta$,
where
\[
    \mu_H
    =
    A^{k/2} \log\left( \frac{C k n^2}{\delta} \right)^{k/2}
    =
    \tilde{\mathcal{O}}\left( 1 \right)
    \;\;\text{and}\;\;
    \mu_W
    =
    A^k \log\left( \frac{2 C k m n}{\delta} \right)^k
    =
    \tilde{\mathcal{O}}\left( 1 \right)
\]
for some global constants $A$ and $C$ independent of $n$ and $k$.
\end{lemmarep}
\begin{proof}
    First we will prove what we want to prove about $H$; then we will prove what we want to prove about $W$. Let $Q$ be a matrix of eigenvectors of $H$.
    Observe that since $Q$ is an orthogonal matrix (by the spectral theorem, because $H$ is symmetric), $Q e_j$ is a unit vector, i.e. $\norm{ Q e_j } = 1$. Call $Q e_j = y$. Also observe that
    \[
        e_i^T (U_1 \otimes U_2 \otimes \cdots \otimes U_k)
        =
        ((e_{i_1}^T U_1) \otimes (e_{i_2}^T U_2) \otimes \cdots \otimes (e_{i_k}^T U_k))
    \]
    for some indices $i_j$.
    Call $e_{i_j}^T U_j = x_j^T$, and observe that the $x_j$ are all independent unit random vectors. So,
    \[
        \left( (U_1 \otimes U_2 \otimes \cdots \otimes U_k) Q \right)_{ij}
        =
        (x_1 \otimes x_2 \otimes \cdots \otimes x_k)^T y
    \]
    for random unit vectors $x_1, \ldots, x_k$ and unit vector $y$.
    We can easily bound this with $k$ applications of Lemma~\ref{lemmaOneStep} and a union bound, yielding
    \[
        \mathbf{P}\left( \left( (x_1 \otimes x_2 \otimes \cdots \otimes x_k)^T y \right)^2 \ge \frac{A^k}{n} \log\left( \frac{C}{\delta} \right)^k \right)
        \le
        k \delta,
    \]
    Setting $\delta \mapsto \frac{\delta}{k n^2}$ yields
    \[
        \mathbf{P}\left( \left( (x_1 \otimes x_2 \otimes \cdots \otimes x_k)^T y \right)^2 \ge \frac{A^k}{n} \log\left( \frac{C k n^2}{\delta} \right)^k \right)
        \le
        \frac{\delta}{n^2},
    \]
    and unioning over all the entries of the large orthogonal matrix,
    \[
        \mathbf{P}\left( \max_{i,j} \; \Abs{ \left( (U_1 \otimes U_2 \otimes \cdots \otimes U_k) Q \right)_{ij} } \ge \sqrt{ \frac{A^k}{n} \log\left( \frac{C k n^2}{\delta} \right)^k } \right)
        \le
        \delta.
    \]
    
    Next, for $W$, observe that if we flatten $W$, then $W/\norm{W}_F$ is a unit vector. Then any entry of the resulting matrix can be written as
    \[
        (x_1 \otimes x_2 \otimes \dots \otimes x_k)^T W (y_1 \otimes y_2 \otimes \dots \otimes y_k)
    \]
    where $x_1, \dots, x_k$ and $y_1, \dots, y_k$ are $k$ independent random unit vectors.
    We can easily bound this with $2k$ applications of Lemma~\ref{lemmaOneStep} and a union bound, yielding
    \[
        \mathbf{P}\left( \left( 
        (x_1 \otimes x_2 \otimes \dots \otimes x_k)^T W (y_1 \otimes y_2 \otimes \dots \otimes y_k)
        \right)^2 \ge \frac{A^{2k}}{mn} \log\left( \frac{C}{\delta} \right)^{2k} \right)
        \le
        2 k \delta,
    \]
    Setting $\delta \mapsto \frac{\delta}{2 k m n}$ yields
    \[
        \mathbf{P}\left( \left( 
        (x_1 \otimes x_2 \otimes \dots \otimes x_k)^T W (y_1 \otimes x_2 \otimes \dots \otimes y_k)
        \right)^2 \ge \frac{A^{2k}}{mn} \log\left( \frac{2 C k mn}{\delta} \right)^{2k} \right)
        \le
        \frac{\delta}{m n},
    \]
    and unioning over all the $mn$ entries of the large orthogonal matrix,
    \[
        \mathbf{P}\left( \max_{i,j} \; \Abs{ e_i^T (U_1 \otimes U_2 \otimes \dots U_k) W (V_1 \otimes V_2 \otimes \dots \otimes V_k) e_j }  \ge \sqrt{ \frac{A^{2k}}{mn} \log\left( \frac{2 C k mn}{\delta} \right)^{2k} } \right)
        \le
        \delta.
    \]
    This is what we wanted to show.

\end{proof}

\textbf{Remarks.} This lemma means that multiplying by a random matrix in this family suffices to make a matrix incoherent with parameter $\mu$ only poly-logarithmic in the matrix size.
In our experiments we use $k=2$ factors to construct the orthogonal matrices $U, V$.


\subsection{Additional Heuristics}\label{secAddHeur}
We outline QuIP pre-processing and post-processing in Algorithms \ref{algQuIPpre} and \ref{algQuIPpost}, respectively.
In line~\ref{alg1incoherent_a} of Algorithm~\ref{algQuIPpre}, we apply the aforementioned fast orthogonal multiplication procedure to ensure $W$ and $H$ are incoherent.
We also randomly permute entries at the fast matrix multiplication step to prevent any correlation between attention heads from worsening performance. We introduce a number of additional heuristic improvements that further improve performance.

\textbf{Incoherence-Based Heuristics.}
Line~\ref{alg1rescale_a} diagonally rescales $W$ and $H$ to minimize $\ell(\hat W) \approx \trace{H} \|W\|_F^2$, effectively trading off the spectrum of these matrices to find a minimum.
Motivated by the incoherence of $W$, Line~\ref{alg1reducedquant} computes the quantization range depending on the spectrum $\|W\|_F$, instead of the typical $\max_{i,j} |W_{ij}|$.
Our full QuIP procedure is described in Algorithm~\ref{algQuIP}, which contains calls to the pre- and post-processing sub-steps in Algorithms~\ref{algQuIPpre} and~\ref{algQuIPpost}.

\textbf{Greedy local search.}
Our basic procedure yields a good initial guess with error guarantees. 
We can further lower the proxy loss by running coordinate descent after LDLQ (but before post-processing), updating the weights in the same order as in the initial pass. See Supplement~\ref{suppExtraMethod} for full details.



\begin{algorithm}[t]
\caption{QuIP: Quantization with Incoherence Processing}\label{algQuIP}
\begin{algorithmic}[1]
\Require $b \in \mathbb{N}$, $H \in \mathbb{R}^{n \times n}$ SPD, $W \in \R^{m \times n}$, $\round \in \{\near, \stoch\}$, $\rho \in \R_+$, $\alpha \in [0,1]$
\State $\hat W, H, s, \tilde D \gets \operatorname{Alg~\ref{algQuIPpre}}(b, H, W,  \rho, \alpha)$ \Comment{QuIP Incoherence Pre-Procesing}
\State $H = (\grave U + I) D (\grave U + I)^{-1}$ \Comment{LDL decomposition}
\State \textbf{for} $k \in \{1, \dots, n\}$ \textbf{do} 
$\hat W_k \gets \operatorname{clamp}(\round( W_k + (W - \hat W) \grave U_k ), 0, 2^b-1)$ \Comment{LDLQ}
\State \Return $\hat W \gets \operatorname{Alg~\ref{algQuIPpost}}(b, H, \hat W, s, \tilde D)$ \Comment{QuIP Incoherence Post-Processing}
\end{algorithmic}
\end{algorithm}




\section{Extensions and Further Analyses}
\newcommand{\titleOptqEquiv}{OPTQ is a Special Case of LDLQ}
\subsection{\titleOptqEquiv}\label{secOPTQequiv}
\begin{toappendix}
\subsection*{Subsection~\ref{secOPTQequiv} (\titleOptqEquiv)}
\end{toappendix}
We prove a novel theoretical insight: QuIP without incoherence processing (i.e.,~LDLQ) is equivalent to a more efficient version of the OPTQ algorithm. % in Algorithm~\ref{algQuIPquant}.
That is, OPTQ falls under our class of adaptive rounding procedures with linear feedback, and is within-class optimal. 

\begin{theoremrep}
OTPQ~\citep{frantar2023gptq} falls within the class of adaptive rounding procedures with linear feedback as described by Eq.~\eqref{eqnVectorQuant}, and is equivalent to LDLQ in Section~\ref{secQuIPquant}.
\end{theoremrep}
\begin{proof}
OPTQ works in the following way.
After OPTQ has quantized the first $t-1$ components of the row vector $w$, it minimizes the proxy loss over the remaining $n-t+1$ elements, keeping the first $t-1$ elements fixed.
It then quantizes the $t$th element using nearest rounding to the grid and clamping.
It then proceeds to the next column.
If we let $\Delta = \hat w - w$, this proxy loss that it minimizes can be written in block form as
$$
\ell = \Delta_{1:(t-1)} H_{1:(t-1),1:(t-1)} \Delta_{1:(t-1)}^T + 2 \Delta_{1:(t-1)} H_{1:(t-1),t:n} + \Delta_{t:n} H_{t:n,t:n} \Delta_{t:n}^T
$$
and its minimum over $\Delta_{t:n}$ will occur when
$$
0 = \Delta_{1:(t-1)} H_{1:(t-1),t:n} + \Delta_{t:n} H_{t:n,t:n},
$$
i.e.
$$
\Delta_{t:n} = -\Delta_{1:(t-1)} H_{1:(t-1),t:n} \left( H_{t:n,t:n} \right)^{-1}.
$$
Now, suppose that $H = \tilde U D \tilde U^T$ is the LDL decomposition of $H$, where $\tilde U$ is unit upper triangular and $D$ is diagonal. Since $\tilde U$ is upper triangular,
$$
H_{t:n,t:n} = \tilde U_{t:n,t:n} D_{t:n,t:n} \tilde U_{t:n,t:n}^T.
$$
Similarly,
$$
H_{1:(t-1),t:n} = \tilde U_{1:(t-1),t:n} D_{t:n,t:n} \tilde U_{t:n,t:n}^T.
$$
This means that
$$
\Delta_{t:n} = -\Delta_{1:(t-1)} \tilde U_{1:(t-1),t:n} \left( \tilde U_{t:n,t:n} \right)^{-1}.
$$
Now, the only part of the value of $\Delta_{t:n}$ which matters is the first entry, since this is the one that's going to be used to make the next quantization decision. But since $\tilde U_{t:n,t:n}$ is unit upper triangular and so is its inverse, $\left( \tilde U_{t:n,t:n} \right)^{-1} e_t = e_t$, and so
$$
\Delta_t = \Delta_{t:n} e_1 = -\Delta_{1:(t-1)} \tilde U_{1:(t-1),t:n} e_t = -\Delta_{1:(t-1)} \tilde U_{1:(t-1),t} = -\Delta (\tilde U - I) e_t.
$$
Finally, we quantize the $t$-th weight as
\[
    \hat w_t
    =
    \round( w_t - (\hat W - W) (\tilde U - I) e_t ).
\]
This update is equivalent to our adaptive rounding with linear feedback procedure in Eq.~\eqref{eqnVectorQuant}, with $U$ assigned from the LDL decomposition of $H$.
\end{proof}

\textbf{Remarks.}
To the best of our knowledge, this equivalence yields the first theoretical analysis of OPTQ.
Even though the two methods are equivalent, LDLQ is more efficient.
OPTQ's implementation requires a matrix inversion of $H$, and two Cholesky decompositions.
Our implementation of LDLQ performs no matrix inversion, and only one Cholesky decomposition.

\textbf{Empirical Verification.}
The quantized outputs of the OPTQ implementation~\cite{frantar2023gptq} are shown to be exactly identical to the outputs of our LDLQ implementation.
Synthetic random data was used, with $W \sim \operatorname{Unif}[0,1]^{1000 \times 1000}$.
Full details can be found in Supplement~\ref{suppAddResults}.

% \newpage
\newcommand{\titleFiniteGrid}{A Bound for Rounding to a Finite Grid}
\subsection{\titleFiniteGrid}\label{secFiniteGrid}
\begin{toappendix}
\subsection*{Subsection~\ref{secFiniteGrid} (\titleFiniteGrid)}
\end{toappendix}

\begin{wrapfigure}{r}{0.4\textwidth}
\vspace{-4em}
  \begin{center}
  % Figure removed
  \end{center}
\vspace{-1em}
  \caption{LDLQ underperforms.}
  \label{figLDLQbad}
\vspace{-1em}
\end{wrapfigure}


In Section~\ref{secQuIPquant}, we saw that LDLQ (equivalently, OPTQ) is optimal for minimizing the adaptive rounding objective. However, this analysis assumed rounding to the integers. In practice, we do not want to round $W$ just to the integers, but instead to scale it, shift it, and round it a finite subset corresponding to a $b$-bit integer. To do this, the ``real'' LDLQ algorithm uses a clamp operation to restrict the range of quantized values.  Is LDLQ still optimal when this small change is made? It turns out that the answer is \emph{no}, as the following concrete example illustrates.


\textbf{Finite Grid Counterexample.}
Figure~\ref{figLDLQbad} illustrates the behavior of LDLQ and other rounding methods---when restricted via clamping to a finite 4-bit grid $[0,15]$---on a particular example where $H$ is a (cleverly chosen) small perturbation of $(I_n + \mathbf{1}_{n \times n} - e_n e_n^T)/n$, and $W$ has $m = 16$ and is a small perturbation of $\mathbf{1}_{m \times n} / 2$. Details of the setup appear in Supplement~\ref{suppAddResults}. The figure shows that clamped LDLQ with nearest rounding is asymptotically worse, and the clamping to the finite grid is what causes it to be worse in this case.
 
Note that in our experiments in practice, OPTQ has been shown to soundly beat nearest rounding.
This clamping issue does not seem to arise in practice; however, since it is \emph{possible} we do need to take it into account to prove useful end-to-end bounds.

\textbf{A Procedure With a Bound.} In order to address the above issues in theory, here we describe a method that acts to restrict the value of $| \hat W_{ij} - W_{ij} |$, so that the rounded weights will remain inside the grid if $W$ is sufficiently far inside. We do this via the optimization problem with hyperparameter $c$
\begin{align}
    \mbox{minimize: } & \trace{H R^T R} \nonumber\\
    \mbox{over: } & R \text{ unit upper triangular} \label{eqnADMMobj}\\
    \mbox{subject to: } & e_i^T R^T R e_i \le 1 + c, \; \forall i \in \{1,\ldots,n\}.\nonumber
\end{align}

\begin{toappendix}
Algorithm~\ref{alg:qcvx} presents a quantization procedure which theoretically address OPTQ's clamping issue, by incorporating a restriction of $|\hat W_{ij} - W_{ij}|$ into objective~\eqref{eqnADMMobj}.
Note that for simplicity, here we present the explicit case where only two factors are used in each Kronecker product of orthogonal matrices; however, the proof should generalize to any number of factors.
\begin{algorithm}
\caption{``Fixed'' Rounding via a Convex Program}\label{alg:qcvx}
\begin{algorithmic}
\Require $W \in \R^{m \times n}$, $H \in \R^{n \times n}$, $c > 0$, $\rho > 0$
\Require factorization $m = p_1 p_2$, $n = p_3 p_4$
\State \textbf{draw } $U_1 \in \mathbb{R}^{p_1 \times p_1}$ uniformly from the set of orthogonal matrices using seed $\mathsf{seed}(U_1)$
\State \textbf{draw } $U_2 \in \mathbb{R}^{p_2 \times p_2}$ uniformly from the set of orthogonal matrices using seed $\mathsf{seed}(U_2)$
\State \textbf{draw } $U_3 \in \mathbb{R}^{p_3 \times p_3}$ uniformly from the set of orthogonal matrices using seed $\mathsf{seed}(U_3)$
\State \textbf{draw } $U_4 \in \mathbb{R}^{p_4 \times p_4}$ uniformly from the set of orthogonal matrices using seed $\mathsf{seed}(U_4)$
\State $W \gets (U_1 \otimes U_2) W (U_3 \otimes U_4)$
\State $H \gets (U_3^T \otimes U_4^T) H (U_3 \otimes U_4)$
\State $W \gets \frac{2^b - 1}{2} \left( \frac{W}{\rho} + 1 \right)$ elementwise
\State $W \gets \operatorname{clamp}(W, \min=0, \max=2^{b}-1))$ elementwise
\State use ADMM or some other solver to solve
\begin{align*}
    \mbox{minimize: } & \trace{H L^T L} \\
    \mbox{over: } & L \text{ unit upper triangular} \\
    \mbox{subject to: } & e_i^T L^T L e_i \le 1 + c, \; \forall i \in \{1,\ldots,n\}.
\end{align*}
\State note that when $c = \infty$, $L^{-1}$ is the factor from the LDL decomposition of $H$
\State $\grave{U} \gets L^{-1} - I$
\State \textbf{for} $k \in \{1, \dots, n\}$ \textbf{do} 
$\hat W_k \gets \operatorname{clamp}(\round( W_k + (W - \hat W) \grave{U}_k ), 0, 2^b-1)$ \Comment{round with LF}
\State $\hat W \gets \rho \left( \frac{2 \hat W}{2^b - 1} - 1 \right)$
\State $\hat W \gets (U_1^T \otimes U_2^T) \hat W (U_3^T \otimes U_4^T)$
\State \Return $\hat W$ encoded as a tuple of the integer rounded values, the scale factor $\rho$, and the seeds
\end{algorithmic}
\end{algorithm}


\begin{lemma}
Suppose that for positive definite $\mu$-incoherent matrix $H \in \mathbb{R}^{n \times n}$ and scalar $c > 0$, $L$ is the solution to the optimization problem
\begin{align*}
    \mbox{minimize: } & \trace{H L^T L} \\
    \mbox{over: } & L \text{ unit upper triangular} \\
    \mbox{subject to: } & e_i^T L^T L e_i \le 1 + c, \; \forall i \in \{1,\ldots,n\}.
\end{align*}
Then the solution satisfies
\[
	\trace{H L^T L} = \frac{\mu^2}{n \cdot \min(1,c)} \trace{H^{1/2}}^2.
\]
\end{lemma}
\begin{proof}
Let $\eta \in \mathbb{R}^{1 \times n}$ be a random standard Gaussian variable as a row vector, let $A$ be a matrix, and consider the recurrence relation over $x_t \in \mathbb{R}^{1 \times n}$ given by $x_0 = 0$ and
\[
	x_t = x_{t-1} - x_{t-1} A e_i e_i^T + \eta e_i e_i^T 
\]
We first note that since $x_t$ is supported only on $\{1,\ldots,t\}$, if $M$ denotes the strictly upper triangular mask, this update step is equivalent to
\[
	x_t = x_{t-1} - x_{t-1} (A \odot M) e_i e_i^T + \eta e_i e_i^T.
\]
From here, it's fairly easy to see by induction that
\[
	x_n = -x_n (A \odot M) + \eta,
\]
and so
\[
	x_n (I + A \odot M) = \eta,
\]
or
\[
	x_n = \eta (I + A \odot M)^{-1}.
\]
Now, since $I + A \odot M$ is a unit upper triangular matrix, its inverse is also a unit upper triangular matrix. If we let $L = (I + A \odot M)^{-1}$, then $L$ is a unit upper triangular matrix and
\[
	\Exv{x_n^T x_n} = L^T L.
\]
We are going to choose $A$ such that $L$ is a feasible solution to our optimization problem and has the desired objective.
Next, let $\Sigma_t = \Exv{x_t^T x_t}$, and observe that
\[
	\Sigma_t = \left( I - A e_i e_i^T \right)^T \Sigma_{t-1} \left( I - A e_i e_i^T \right) + e_i e_i^T.
\]
Let $\alpha > 0$ be some constant to be set later, and set $A = \alpha H^{1/2}$. Suppose by way of induction that for some constant $\beta > 0$ to be set later, $\Sigma_t \preceq \beta H^{-1/2}$.
The base case clearly holds since $\Sigma_0 = 0$. For the inductive step,
\begin{align*}
	\Sigma_t 
	&\preceq 
	\beta \left( I - \alpha H^{1/2} e_i e_i^T \right)^T H^{-1/2} \left( I - \alpha H^{1/2} e_i e_i^T \right) + e_i e_i^T
	\\&= 
	\beta H^{-1/2} - 2 \alpha \beta e_i e_i^T + \alpha^2 \beta e_i e_i^T H^{1/2} e_i e_i^T + e_i e_i^T.
\end{align*}
This inductive step will hold if, letting $h = \max_i e_i^T H^{1/2} e_i$,
\[
	2 \alpha \beta \ge 1 + \alpha^2 \beta h
\]
On the other hand,
\begin{align*}
	e_i^T L^T L e_i 
	&= \Exv{(x_n e_i)^2} 
	\\&=  \Exv{\left( -x_{i-1} A e_i + \eta e_i \right)^2}
	\\&=  \Exv{\left( -x_{i-1} A e_i \right)^2} + 1
	\\&= e_i^T A^T \Sigma_{i-1} A e_i + 1
	\\&= \alpha^2 e_i^T H^{1/2} \Sigma_{i-1} H^{1/2} e_i + 1
	\\&\le \alpha^2 \beta e_i^T H^{1/2} H^{-1/2} H^{1/2} e_i + 1
	\\&\le \alpha^2 \beta e_i^T H^{1/2} e_i + 1.
\end{align*}
So the constraint of our optimization problem will be satisfied if
\[
	\alpha^2 \beta h \le c.
\]
To satisfy these constraints, set $\beta = \max(h, h/c)$ and $\alpha = \beta^{-1}$. Then
\[
	2 \max(h, h/c)^{-1} \cdot \max(h, h/c) \ge 1 + \max(h, h/c)^{-2} \cdot \max(h, h/c) \cdot h,
\]
and
\[
	\max(h, h/c)^{-2} \cdot \max(h, h/c) \cdot h \le c.
\]
Also, the objective will be bounded by
\[
	\trace{H L^T L} = \trace{H \Sigma_n} \le \beta \trace{H^{1/2}} = \max(1, c^{-1}) \cdot h \cdot \trace{H^{1/2}}.
\]
Now, applying incoherence to bound $h$, where $H = U \Lambda U^T$ is the eigendecomposition of $H$,
\[
	e_i^T H^{1/2} e_i 
	= 
	\sum_{j=1}^n \lambda_j^{1/2} (e_i^T U e_j)^2
	\le
	\sum_{j=1}^n \lambda_j^{1/2} \frac{\mu^2}{n}
	=
	\frac{\mu^2}{n} \trace{H^{1/2}}.
\]
So this yields a whole bound of
\[
	\trace{H L^T L} = \frac{\mu^2}{n \cdot \min(1,c)} \trace{H^{1/2}}^2.
\]
This is what we wanted to show.
\end{proof}

\begin{lemma}
Suppose that we quantize the row vector $w \in \mathbb{R}^{1 \times n}$ using $L$ the solution to the optimization problem
\begin{align*}
    \mbox{minimize: } & \trace{H L^T L} \\
    \mbox{over: } & L \text{ unit upper triangular} \\
    \mbox{subject to: } & e_i^T L^T L e_i \le 1 + c, \; \forall i \in \{1,\ldots,n\}
\end{align*}
and
\[
	\hat w = \mathcal{Q}_{\operatorname{stoch}}\left( w - (\hat w - w) (L^{-1} - I) \right),
\]
where $\mathcal{Q}_{\operatorname{stoch}}$ denotes elementwise unbiased stochastic rounding. Then for any $u \in \mathbb{R}^n$ and any $\delta > 0$
\[
	\mathbf{P}\left( \Abs{ (\hat w - w) u } \ge \norm{L u} \sqrt{\frac{1}{2} \log\left( \frac{2}{\delta} \right) } \right) \le \delta.
\]
In particular,
\[
	\mathbf{P}\left( \Abs{ (\hat w - w) (L^{-1} - I) e_i } \ge \sqrt{\frac{c}{2} \log\left( \frac{2}{\delta} \right) } \right) \le \delta.
\]
\end{lemma}
\begin{proof}
Let $\eta$ be the error of stochastic rounding, and observe that each entry is, conditioned on earlier steps, zero mean and supported on two values that differ by $1$.
Also observe that
\[
	\hat w = \left( w - (\hat w - w) (L^{-1} - I) \right) + \eta,
\]
and so
\[
	\hat w - w = \eta L
\]
and
\[
	\Exv{\exp\left( (\hat w - w) u \right)} = \Exv{\exp\left( \eta L u \right)}.
\]
From a repeated application of Hoeffding's lemma, we get
\[
	\Exv{\exp\left( (\hat w - w) u \right)} \le \exp\left( \frac{1}{8} \norm{ L u }^2 \right).
\]
Setting $u \mapsto \gamma u$ for $\gamma > 0$,
\[
	\Exv{\exp\left( \gamma (\hat w - w) u \right)} \le \exp\left( \frac{\gamma^2}{8} \norm{ L u }^2 \right).
\]
And by Markov's inequality,
\[
	\mathbf{P}\left( \exp\left( \gamma (\hat w - w) u \right) \ge \exp(\gamma R) \right) \le \exp(-\gamma R) \exp\left( \frac{\gamma^2}{8} \norm{ L u }^2 \right),
\]
i.e.
\[
	\mathbf{P}\left( (\hat w - w) u \ge R \right) \le \exp\left( -\gamma R + \frac{\gamma^2}{8} \norm{ L u }^2 \right).
\]
Minimizing the right side over $\gamma$ yields $\gamma = 4 R \norm{L u}^{-2}$ and
\[
	\mathbf{P}\left( (\hat w - w) u \ge R \right) \le \exp\left( -2 R^2 \norm{L u}^{-2} \right).
\]
By a union bound,
\[
	\mathbf{P}\left( \Abs{ (\hat w - w) u } \ge R \right) \le 2 \exp\left( -2 R^2 \norm{L u}^{-2} \right).
\]
Now setting the right side equal to $\delta$,
\[
	\mathbf{P}\left( \Abs{ (\hat w - w) u } \ge \norm{L u} \sqrt{\frac{1}{2} \log\left( \frac{2}{\delta} \right) } \right) \le \delta.
\]
This is what we wanted to show. The second statement follows from the fact that
\[
	 \norm{L (L^{-1} - I) e_i}^2
	 =
	 \norm{e_i - L e_i}^2
	 =
	e_i^T e_i - e_i^T L e_i - e_i^T L^T e_i + e_i^T L^T L e_i
	\le
	1 - 1 - 1 + (1 + c)
	=
	c.
\]
\end{proof}

\begin{lemma}
Suppose that we quantize the row vector $w \in \mathbb{R}^{1 \times n}$ using $L$ the solution to the optimization problem
\begin{align*}
    \mbox{minimize: } & \trace{H L^T L} \\
    \mbox{over: } & L \text{ unit upper triangular} \\
    \mbox{subject to: } & e_i^T L^T L e_i \le 1 + c, \; \forall i \in \{1,\ldots,n\}
\end{align*}
and
\[
	\hat w = \mathcal{Q}_{\operatorname{stoch}}\left( w - (\hat w - w) (L^{-1} - I) \right),
\]
where $\mathcal{Q}_{\operatorname{stoch}}$ denotes elementwise unbiased stochastic rounding.
Suppose that for some integer $b$, $1 \le w_{ij} \le 2^{b} - 2$.
Then if we set
\[
	c = 2 \left( \log\left( \frac{4 m n}{\delta} \right) \right)^{-1},
\]
then with probability at least $1 - \delta$, $0 \le \hat w_{ij} \le 2^{b} - 1$ and
\[
	\trace{ (\hat w - w) H (\hat w - w)^T } \le \frac{\mu^2 m}{4n} \trace{H^{1/2}}^2 \left( \log\left( \frac{4 mn}{\delta}\right)^2 \right).
\]
\end{lemma}
\begin{proof}
First, from the previous lemmas, if $U e_i$ is the $i$th eigenvector of $H$, with eigenvalue $\lambda_i$ since
\[
	\mathbf{P}\left( \lambda_i ( e_j^T (\hat w - w) U e_i )^2 \ge \lambda_i \norm{L U e_i}^2 \cdot \frac{1}{2} \log\left( \frac{2}{\delta} \right) \right) \le \delta.
\]
By the union bound,
\[
	\mathbf{P}\left( \exists i,j, \; \lambda_i ( e_j^T (\hat w - w) U e_i )^2 \ge \lambda_i \norm{L U e_i}^2 \cdot \frac{1}{2} \log\left( \frac{2 mn}{\delta} \right) \right) \le \delta.
\]
And so
\[
	\mathbf{P}\left( \sum_{i,j} \lambda_i ( e_j^T (\hat w - w) U e_i )^2 \ge \sum_{i,j} \lambda_i \norm{L U e_i}^2 \cdot \frac{1}{2} \log\left( \frac{2 mn}{\delta} \right) \right) \le \delta,
\]
which simplifies to
\[
	\mathbf{P}\left( \trace{ (\hat w - w) H (\hat w - w)^T } \ge m \trace{H L^T L} \cdot \frac{1}{2} \log\left( \frac{2 mn}{\delta} \right) \right) \le \delta.
\]
Now applying the other lemma,
\[
	\mathbf{P}\left( \trace{ (\hat w - w) H (\hat w - w)^T } \ge \frac{\mu^2 m}{2n \cdot \min(1,c)} \trace{H^{1/2}}^2 \log\left( \frac{2 mn}{\delta} \right) \right) \le \delta.
\]
And substituting $\delta \mapsto \delta/2$,
\[
	\mathbf{P}\left( \trace{ (\hat w - w) H (\hat w - w)^T } \ge \frac{\mu^2 m}{2n \cdot \min(1,c)} \trace{H^{1/2}}^2 \log\left( \frac{4 mn}{\delta} \right) \right) \le \frac{\delta}{2}.
\]
On the other hand, again by a union bound from the previous lemma,
\[
	\mathbf{P}\left( \exists i,j, \; \Abs{ e_j^T (\hat w - w) (L^{-1} - I) e_i } \ge \sqrt{\frac{c}{2} \log\left( \frac{4 m n}{\delta} \right) } \right) \le \frac{\delta}{2}.
\]
Setting
\[
	c = 2 \left( \log\left( \frac{4 m n}{\delta} \right) \right)^{-1}
\]
yields
\[
	\mathbf{P}\left( \exists i,j, \; \Abs{ e_j^T (\hat w - w) (L^{-1} - I) e_i } \ge 1 \right) \le \frac{\delta}{2}.
\]
And so by another union bound, the probability that
\[
	\trace{ (\hat w - w) H (\hat w - w)^T } \le \frac{\mu^2 m}{4n} \trace{H^{1/2}}^2 \left( \log\left( \frac{4 mn}{\delta}\right) \right)^2
\]
and
\[
	\max_{i,j} \; \Abs{ e_j^T (\hat w - w) (L^{-1} - I) e_i } \le 1
\]
is no less than $1 - \delta$. It's clear that if this second inequality holds, the value we pass in to the stochastic quantizer will be in range, and thus so will the output. This proves what we want.
\end{proof}

\begin{theorem}
Suppose that we are given an input matrix $w$ with bounded maximum entry magnitude $\norm{w}_{\infty}$ and we want to quantize it using $b$ bits. Suppose that we first re-scale the entries of $w$ by mapping
\[
	w_{ij} \mapsto \frac{2^b - 3}{2} \left( \frac{w_{ij}}{\norm{w}_{\infty}} + 1 \right) + 1;
\]
this guarantees that $1 \le w_{ij} \le 2^b - 2$. Then, suppose we quantize using the procedure described in the previous lemma. Finally, we undo the scaling. Then then with probability at least $1 - \delta$, all the quantized weights will be in range (no overflow or need for clipping) and
\[
	\trace{ (\hat w - w) H (\hat w - w)^T } \le \frac{\mu^2 m}{n (2^b - 3)^2} \trace{H^{1/2}}^2 \norm{w}_{\infty}^2 \left( \log\left( \frac{4 mn}{\delta}\right)^2 \right).
\]
\end{theorem}
\begin{proof}
This is a straightforward consequence of the previous lemma.
\end{proof}

\begin{theorem}
Suppose that we are given an input matrix $w$ with bounded $\norm{w}_{F}$ and we want to quantize it using $b$ bits. Suppose that we first multiply by two-factor orthogonal matrices, and then we re-scale the entries of $w$ by mapping
\[
	w_{ij} \mapsto \frac{2^b - 3}{2} \left( \frac{w_{ij}}{\norm{w}_F \sqrt{ \frac{A^2}{mn} \log\left( \frac{2C m n}{\delta} \right)^2 }} + 1 \right) + 1;
\]
this guarantees that $1 \le w_{ij} \le 2^b - 2$. Then, suppose we quantize using the procedure described in the previous lemma. Finally, we undo the scaling and multiplication. Then then with probability at least $1 - \delta$, all the quantized weights will be in range (no overflow or need for clipping) and
\begin{align*}
	\trace{ (\hat w - w) H (\hat w - w)^T } &\le \frac{A^4}{n^2 (2^b - 3)^2} \trace{H^{1/2}}^2 \norm{w}_{F}^2 \left( \log\left( \frac{12 C m n^2}{\delta}\right) \right)^6
	\\&=
	\tilde{\mathcal{O}}\left( \frac{1}{n^2 4^b} \trace{H^{1/2}}^2 \norm{w}_{F}^2 \right).
\end{align*}
\end{theorem}
\begin{proof}
It is a straightforward consequence of Lemma~\ref{lemFastInco}, that unioning over the three bounds on the infinity norm of $w$, the incoherence of $H$, and the stochastic rounding, with probability at least $1 - 3 \delta$,
\begin{align*}
\trace{ (\hat w - w) H (\hat w - w)^T } &\le \frac{m}{n (2^b - 3)^2} \trace{H^{1/2}}^2 \norm{w}_{F}^2 \left( \log\left( \frac{4 mn}{\delta}\right)^2 \right) \\&\hspace{4em}\cdot A^2 \log\left(\frac{2 C n^2}{\delta}\right)^2 \cdot  \frac{A^2}{mn} \log\left( \frac{2 C n}{\delta} \right)^2.
\end{align*}
Substituting $\delta \mapsto \delta/3$, 
\begin{align*}
	\trace{ (\hat w - w) H (\hat w - w)^T } &\le \frac{1}{n (2^b - 3)^2} \trace{H^{1/2}}^2 \norm{w}_{F}^2 \left( \log\left( \frac{12 mn}{\delta}\right)^2 \right) \\&\hspace{4em}\cdot A^2 \log\left(\frac{6 C n^2}{\delta}\right)^2 \cdot  \frac{A^2}{n} \log\left( \frac{6 C n}{\delta} \right)^2.
\end{align*}
And this right side is clearly less than
\[
	\trace{ (\hat w - w) H (\hat w - w)^T } \le \frac{A^4}{n^2 (2^b - 3)^2} \trace{H^{1/2}}^2 \norm{w}_{F}^2 \left( \log\left( \frac{12 C m n^2}{\delta}\right) \right)^6.
\]
This is what we wanted to show.
\end{proof}


\end{toappendix}

Our ``fixed'' algorithm solves this convex problem (e.g. with ADMM), then runs QuIP using stochastic rounding and $U = R^{-1} - I$ in place of the LDL decomposition. Observe that for sufficiently large $c$, this is exactly equivalent to base QuIP, since the solution of that optimization problem is given by the LDL decomposition when the constraint is dropped. Doing this (the full algorithm is given in the supplemental) yields the following theorem.

\begin{theoremrep}
Suppose that we run Algorithm~\ref{alg:qcvx} (Supplement) to quantize a matrix $W \in \mathbb{R}^{m \times n}$ by solving the objective~\eqref{eqnADMMobj}. Then there exists an assignment of the algorithm's hyperparameters $c$ and $\rho$ such that with probability at least $1 - \delta$, all the quantized weights will be in range (no overflow or need for clipping) and
\[
	\trace{ (\hat W - W) H (\hat W - W)^T }
	=
	\tilde{\mathcal{O}}\left( \frac{1}{n^2 4^b} \trace{H^{1/2}}^2 \norm{W}_{F}^2 \right).
\]
\end{theoremrep}
\begin{proof}
This follows directly from the previous theorem, which says explicitly what the hyperparameter assignments should be.
\end{proof}

In practice, because clamping rarely causes issues, and because of the significant additional compute needed to solve this program, we always just use QuIP as described in the previous sections, which is equivalent to setting $c$ large and using nearest rounding.




\section{Experiments}\label{secExp}
\section{Experiments}
% \haizhou{Follow the same way of introduction as we did in Section2.}
% \noindent In this section, we will introduce datasets and experimental setups that we used. Then we evaluate our method, other self-supervised methods, and supervised methods under different distribution shifts (\ie, concept shifts and covariate shifts) under common settings (\ie, transductive, inductive settings). It has to note that we focus on node-level tasks (\eg, node classification) in this work. As for graph-level tasks, we leave it as our future work and some simple experiments can be found in Appendix~\ref{app:graph_classification}. 
In this section, we first introduce the experimental setup including datasets, training, and evaluation protocol in Section~\ref{sec:dataset}~and~\ref{sec:unsupervised}. 
% Next, we present our experimental setup and conduct extensive experiments to evaluate our method in Section~\ref{sec:unsupervised}. 
We then perform an ablation study to demonstrate the effectiveness of each proposed component in Section~\ref{sec:ablation}. 
Additionally, we analyze the impact of important hyper-parameters in Section~\ref{sec:sensitivity}. 
Subsequently, we integrate our method with various encoding models, showcasing the model-agnostic nature of our recipe in Section~\ref{sec:other_models}. 
Finally, we provide some qualitative results such as feature visualization in Section~\ref{sec:vis}.
It is important to note that we focus on node-level tasks (\eg, node classification) in this work. As for graph-level tasks, we leave it as our future work, while some simple experiments are also provided in Appendix~\ref{app:graph_classification}.

\subsection{Datasets}\label{sec:dataset}
There exist some benchmarks for evaluating graph out-of-distribution generalization~\cite{good,ji2022drugood,gds}. 
Among them, GOOD~\cite{good} is the most representative and comprehensive benchmark that curates more diverse graph datasets with diverse tasks, including single/multi-task graph classification, graph regression, and node classification involving more distribution shifts (\ie, concept shifts and covariate shifts). Hence in this work, we follow the evaluation protocol proposed in \cite{good}. Furthermore, we validate the effectiveness of our method in the datasets (\ie, Amazon-Photo, Elliptic) that are used in EERM~\cite{eerm}. The statistics and detailed introduction to these datasets can be found in Table~\ref{tab:dataset} and Appendix~\ref{app:datasets}.

\begin{table*}[htp]
\caption{The descriptions of datasets. ``Domain-Level'' means splitting by graphs, ``Time-Aware'' denotes splitting according to chronological order.``Word'' and ``Degree'' represent splitting according to word diversity and node degree respectively. ``Language'' means splitting by user language, suggesting the prediction should not be impacted by the language the user use. ``University'' denotes splitting according to the domain university, implying that the prediction of webpages should be based on word contents and link connections rather than university features. ``Color'' means that nodes are split according to node differences in covariate shift and color-label correlations in concept shift.}
\label{tab:dataset}
\centering
\begin{tabular}{cccccccc}
\toprule
Datasets     & Network Type        & \#Nodes & \#Edges & \#Attributes &\#Classes& Train/Val/Test Split     & Metric   \\
% Cora         & Artificial Transformation & 2,703   &         &              &         &                      & Accuracy \\
Amazon-Photo\footnotemark
             & Co-purchasing network      & 7,650   & 119,081   & 755          & 10      & Domain-Level         & Accuracy \\
Elliptic\footnotemark  
             & Bitcoin transactions       & 203,769 & 234,355   & 165          & 2       & Time-Aware           & F1-Score \\
GOOD-Cora    & Scientific publications    & 19,793  & 126,842   & 8,710         & 70      & Word/Degree          & Accuracy \\
% GOOD-Arxiv   & arXiv papers               & 169,343 & 2,315,598 & 128          & 40      & Time/Degree          & Accuracy \\
GOOD-Twitch  & Gamer network              & 34,120  & 892,346   & 128          & 2       & Language             & ROC-AUC  \\
GOOD-CBAS    & A BA-house graph           & 700     & 3,962     & 4             & 4       & Color                & Accuracy \\
GOOD-WebKB   & Webpage network            & 617     & 1,138     & 1,703         & 5       & University           & Accuracy \\
\bottomrule
\end{tabular}
\end{table*}
\footnotetext[5]{This dataset is adopted from~\cite{yang2016revisiting}. \cite{eerm} constructs ten graphs with different environment id’s for each graph.} 
\footnotetext[6]{The original is available on \hyperlink{https://www.kaggle.com/ellipticco/elliptic-data-set}{https://www.kaggle.com/ellipticco/elliptic-data-set}}

\subsection{Unsupervised Representation Learning}\label{sec:unsupervised}
\subsubsection{Transductive Setting}~\label{sec:trans}
% \noindent\textbf{Baselines.}\quad We conduct experiments with 12 baselines which consist of three categories: supervised methods and self-supervised generative methods, self-supervised contrastive methods. Specifically, we compare with three supervised baselines: empirical risk minimization~(ERM)~\cite{erm}, invariant risk minimization (IRM)~\cite{irm}, and a recent proposed graph OOD method dubbed EERM~\cite{eerm}. We also compare various unsupervised node-level representation learning methods: three self-supervised generative methods including GAE~\cite{gae}, VGAE~\cite{gae}, GraphMAE~\cite{gmae} and seven self-supervised contrastive methods: DGI~\cite{dgi}, MVGRL~\cite{mvgrl}, GRACE~\cite{grace}, RoSA~\cite{rosa}, BGRL~\cite{bgrl}, COSTA~\cite{costa}, SwAV~\cite{swav}. The descriptions of these methods can be found in Appendix~\ref{app:baselines}.
In this subsection, we focus on validating our proposed algorithm under the transductive setting, where the test nodes will participate in message passing~\cite{gilmer2017neural} during training following~\cite{good}. 

\noindent\textbf{Baselines.} We conduct experiments with 12 baselines from three categories: (i)~supervised methods, including empirical risk minimization~(\textbf{ERM})~\cite{erm}, invariant risk minimization (\textbf{IRM})~\cite{irm}, and a recent proposed graph OOD method \textbf{EERM}~\cite{eerm}; (ii)~self-supervised generative methods including Graph Autoencoder (\textbf{GAE})~\cite{gae}, Variational Graph Autoencoder (\textbf{VGAE})~\cite{gae}, Self-Supervised Masked Graph Autoencoders (\textbf{GraphMAE})~\cite{gmae}; (iii)~self-supervised contrastive methods including Deep Graph Infomax (\textbf{DGI})~\cite{dgi}, Contrastive Multi-View Representation Learning on Graphs (\textbf{MVGRL})~\cite{mvgrl}, Deep Graph Contrastive Representation Learning (\textbf{GRACE})~\cite{grace}, A Robust Self-Aligned Framework for Node-Node Graph Contrastive Learning (\textbf{RoSA})~\cite{rosa}, Bootstrapped Representation Learning on Graphs (\textbf{BGRL})~\cite{bgrl}, Covariance-Preserving Feature Augmentation for Graph Contrastive Learning (\textbf{COSTA})~\cite{costa}, Unsupervised Learning of Visual Features by Contrasting Cluster Assignments (\textbf{SwAV})~\cite{swav}. The detailed descriptions of these baselines can be found in Appendix~\ref{app:baselines}.

\noindent\textbf{Experimental setup.} We use the same graph encoder across different datasets for a fair comparison following~\cite{good}. We use grid search to find other hyper-parameters (\eg, learning rate, epochs) for different methods. For all experiments, we select the best checkpoints for ID and OOD tests according to results on ID and OOD validation sets following~\cite{good}, respectively. Experimental details and hyper-parameter selections are provided in Appendix~\ref{app:hyper}. For evaluating unsupervised methods, a linear classifier will be built on the frozen trained encoder after finishing pre-training. The reported results are the mean performance with standard deviation after 10 runs following~\cite{good}.

\noindent\textbf{Analysis.}\quad Based on the experimental results listed in Table~\ref{tab:trans_concept} and \ref{tab:trans_covariate}, we can draw the following conclusions: firstly, we find strong self-supervised methods (\eg, GRACE, BGRL, COSTA) are more robust to distribution shifts (concept shift in Table~\ref{tab:trans_concept} and covariate shift in Table~\ref{tab:trans_covariate}) compared to supervised methods. For instance, on GOOD-CBAS and GOOD-WebKB datasets, GRACE surpasses the best supervised method by large margins (over 6\% absolute improvement). Interestingly, we find the methods designed for OOD generalization (\ie, IRM) and graph OOD generalization (\ie, EERM) do not attain superior performance than the standard ERM on most of the datasets. For example, EERM shows superior OOD performance compared to ERM in only one experiment, and IRM outperforms ERM in four out of ten experiments across the conducted evaluations. This phenomenon is also observed in \cite{good,ahuja2020empirical,rosenfeld2021risks}, showcasing the challenge of achieving invariant prediction in non-Euclidean graph settings. 

Furthermore, our method surpasses other SOTA self-supervised methods on the OOD test set of all datasets by a considerable margin while achieving comparable performance in the in-distribution test set. For instance, on small datasets such as GOOD-CBAS and GOOD-WebKB, our method outperforms GRACE\footnote{MARIO is built up on GRACE according to our recipe. So, we make a comparison with GRACE here.} by over 2\% absolute accuracy on the OOD test set. On larger datasets such as GOOD-Cora and GOOD-Twitch, our method still outperforms other methods which shows its superiority. For instance, under covariate shift, MARIO surpasses other methods by over 7\% absolute accuracy on the GOOD-Twitch OOD test set. These statistics prove the effectiveness of our design.


\begin{table*}[htp]
\caption{Experimental results of all methods under concept shift. The bold font means the top-1 performance and the underline represents the second performance across the unsupervised methods. 'ID' represents in-distribution test performance and 'OOD' means out-of-distribution test performance. (OOM: out-of-memory on a GPU with 24GB memory)}
\label{tab:trans_concept}
\centering
\scalebox{0.95}{
\begin{tabular}{l|cc|cc|cc|cc|cc}
\toprule
\toprule
\multirow{3}{*}{concept shift} & \multicolumn{4}{c|}{GOOD-Cora}                   & \multicolumn{2}{c|}{GOOD-CBAS} & \multicolumn{2}{c|}{GOOD-Twitch} & \multicolumn{2}{c}{GOOD-WebKB} \\
                           & \multicolumn{2}{c}{word} & \multicolumn{2}{c|}{degree}& \multicolumn{2}{c|}{color}    & \multicolumn{2}{c|}{language}   & \multicolumn{2}{c}{university} \\
                           & ID         & OOD         & ID          & OOD          & ID            & OOD           & ID             & OOD            & ID            & OOD            \\
\midrule
ERM                        & 66.38±0.45 & 64.44±0.18  & 68.60±0.40  & 60.76±0.34   & 89.79±1.39    & 83.43±1.19    & 80.80±1.00     & 56.92±0.92     & 62.67±1.53    & 26.33±1.09     \\
IRM                        & 66.42±0.41 & 64.29±0.31  & 68.57±0.35  & 61.45±0.24   & 89.64±1.21    & 82.29±1.14    & 78.87±1.04     & 59.30±1.79     & 62.67±1.10    & 26.88±1.42     \\
EERM                       & 65.10±0.44 & 62.45±0.19  & 66.95±0.44  & 56.58±0.25   & 79.07±2.12    & 64.50±1.01    & OOM            & OOM            & 62.50±2.01    & 28.07±3.23      \\
\midrule
% Random-Init                & 37.53±1.74 & 32.12±1.24  & 37.82±1.71  & 27.74±1.14   &               &               &                &                & 60.33±2.21    & 27.07±1.70     \\
GAE                        & 60.65±0.89 & 58.00±0.55  & 62.59±1.11  & 53.44±0.80   & 75.28±1.36    & 68.07±2.05    & 81.25±0.81     & 51.51±1.05     & 62.17±3.34    & 25.78±1.85     \\
VGAE                       & 63.19±0.53 & 60.35±0.47  & 61.65±0.66  & 54.28±0.28   & 76.50±0.50    & 59.07±0.56    & 80.46±0.53     & 55.56±4.53     & 62.50±2.38    & 24.40±2.57     \\
GraphMAE                   & \underline{66.44±0.46} & \underline{64.87±0.30}  & 67.95±0.46  & 59.41±0.39   & 89.14±0.89    & 82.93±0.93    & 80.05±0.64     & 59.38±1.49     & 61.83±3.37    & 29.27±2.15     \\
DGI                        & 63.33±0.56 & 60.71±0.49  & 65.93±1.02  & 55.83±0.53   & 91.22±1.47    & 85.00±1.66    & 80.05±0.87     & 59.16±1.88     & 61.83±2.83    & 28.63±1.92      \\
MVGRL                      & OOM        & OOM         & OOM         & OOM          & 88.57±1.15    & 76.50±1.17    & OOM            & OOM            & 62.00±3.79    & 28.26±4.20     \\
GRACE                      & 65.61±0.61 & 63.92±0.44  & \textbf{68.59±0.35}  & 60.15±0.45   & 92.00±1.39    & 88.64±0.67    & \textbf{83.43±0.63}     & \underline{60.45±1.46}     & 64.00±3.43    & \underline{34.86±3.43}  \\
RoSA                       & 64.06±0.67 & 62.44±0.39  & 67.07±0.65  & 57.68±0.44   & 90.78±2.27    & 85.93±2.14    & 82.39±0.42     & 57.45±2.16     & 64.17±4.10    & 32.20±2.15     \\
BGRL                       & 65.18±0.43 & 63.43±0.45  & 66.83±0.80  & 59.63±0.38   & 92.36±1.16    & 87.14±1.60    & 82.52±0.60     & 55.48±1.48     & 63.67±2.33    & 31.47±3.43     \\
COSTA                      & 65.05±0.80 & 62.37±0.45  & 66.76±0.87  & 55.73±0.36   & \underline{93.50±2.62}    & \underline{89.29±3.11}    & 83.15±0.30 & 55.03±3.22     & 61.66±2.58    & 32.39±2.13 \\
% ArCL                       &            &             & 67.64±0.57  & 59.71±0.44   &               &               &                &                & 65.00±3.94    & 35.41±1.97 \\      
SwAV                       & 62.22±0.53 & 59.79±0.53  & 64.65±0.94  & 55.06±0.39   & 89.00±0.79    & 81.72±0.66    & \underline{83.32±0.15}     & 59.69±1.97     & \underline{65.17±3.76}    & 29.36±2.01    \\
\midrule
MARIO                       & \textbf{67.11±0.46} & \textbf{65.28±0.34}  & \underline{68.46±0.40}  & \textbf{61.30±0.28}   & \textbf{94.36±1.21}    & \textbf{91.28±1.10}    & 82.31±0.54     & \textbf{63.33±1.72}     & \textbf{65.67±2.81}    & \textbf{37.15±2.37}     \\
\bottomrule
\end{tabular}}
\end{table*}

\begin{table*}[htp]
\caption{Experimental results of all methods under covariate shift. The bold font means the top-1 performance and the underline represents the second performance across the unsupervised methods. 'ID' represents in-distribution test performance and 'OOD' means out-of-distribution test performance. (OOM: out-of-memory on a GPU with 24GB memory)}
\label{tab:trans_covariate}
\centering
\scalebox{0.95}{
\begin{tabular}{l|cc|cc|cc|cc|cc}
\toprule
\toprule
\multirow{3}{*}{covariate shift} & \multicolumn{4}{c|}{GOOD-Cora}                                   & \multicolumn{2}{c|}{GOOD-CBAS} & \multicolumn{2}{c|}{GOOD-Twitch} & \multicolumn{2}{c}{GOOD-WebKB} \\
                           & \multicolumn{2}{c}{word} & \multicolumn{2}{c|}{degree}& \multicolumn{2}{c|}{color}    & \multicolumn{2}{c|}{language}   & \multicolumn{2}{c}{university} \\
                           & ID         & OOD         & ID          & OOD          & ID            & OOD           & ID             & OOD            & ID            & OOD            \\
\midrule
ERM                        & 70.50±0.41 & 64.69±0.33  & 72.46±0.49  & 55.53±0.50   & 92.00±3.08    & 77.57±1.29    & 70.98±0.41     & 49.35±5.09     & 39.34±1.79    & 14.52±3.14   \\
IRM                        & 70.48±0.26 & 64.53±0.57  & 71.98±0.34  & 53.72±0.46   & 90.86±2.41    & 78.86±1.67    & 69.81±0.95     & 49.11±2.82     & 38.52±3.30    & 13.97±2.80     \\
EERM                       & OOM        & OOM         & OOM         & OOM          & 65.00±2.57    & 57.43±3.60    & OOM            & OOM            & 46.07±4.55    & 27.40±7.65     \\
\midrule
GAE                        & 56.63±0.79 & 48.93±0.93  & 66.30±0.88  & 34.01±0.87   & 73.00±2.16    & 60.86±3.01    & 67.24±1.23     & 47.65±2.49     & 45.08±6.32    & 28.02±6.29    \\
VGAE                       & 62.02±0.66 & 54.12±0.86  & 69.41±0.57  & 44.20±1.29   & 62.29±2.04    & 63.29±1.11    & 66.99±1.43     & \underline{50.48±4.58}     & 48.85±4.68    & 20.87±6.69     \\
GraphMAE                   & 68.14±0.43 & 64.00±0.33  & \textbf{73.36±0.56}  & 53.75±0.55   & 67.28±3.03    & 67.28±1.49    & 68.84±1.20     & 48.02±2.79     & 48.03±4.34    & 30.00±8.09     \\
DGI                        & 60.85±0.75 & 57.03±0.67  & 68.97±0.41  & 41.75±0.88   & 69.57±4.09    & 59.71±3.43    & 68.43±1.05     & 44.83±1.61     & 48.52±5.04    & 21.11±7.50     \\
MVGRL                      & OOM        & OOM         & OOM         & OOM          & 65.00±1.94    & 64.15±0.77    & OOM            & OOM           & \textbf{54.10±5.39}    & 16.59±6.51     \\
GRACE                      & \underline{68.77±0.33} & \underline{64.21±0.41}  & 72.69±0.34  & \underline{56.10±0.63}   & \underline{93.57±1.83}    & \underline{89.29±3.40}    & \underline{71.12±0.87} & 46.21±1.54 & 49.67±5.82    & 28.10±4.68    \\
RoSA                       & 68.19±0.56 & 62.48±0.61  & 71.04±0.62  & 52.72±0.79   & 84.71±4.14    &79.14±3.51     & 70.58±0.36     & 45.83±1.72     & 52.30±4.24    & \underline{34.24±7.92}     \\
BGRL                       & 67.23±0.43 & 61.33±0.36  & 72.11±0.39  & 49.15±0.73   & 89.00±2.56    & 79.86±3.29    & \textbf{71.43±0.53}     & 43.86±0.94     & 51.80±5.55    & 30.32±7.61    \\
COSTA                      & 65.28±0.60 & 60.33±0.53  & 70.65±0.62  & 54.03±0.28   & 92.29±1.59    & 82.71±2.74    & 69.29±1.37     & 49.07±2.13     & 50.49±3.01    & 29.84±4.75   \\
SwAV                       & 63.29±1.01 & 56.98±0.94  & 70.27±0.73  & 43.00±0.52   & 89.57±1.12    & 81.43±1.69    & 69.19±0.93     & 49.37±2.96     & 49.84±4.82    & 30.55±6.72   \\
\midrule
MARIO                       & \textbf{69.99±0.54} & \textbf{65.06±0.34}  & \underline{72.73±0.43}  & \textbf{57.73±0.45}  & \textbf{94.57±2.46}    & \textbf{91.00±2.48}     & 68.31±0.78 & \textbf{57.37±1.37}     & \underline{53.94±3.23}    & \textbf{35.24±4.98}   \\
\bottomrule
\end{tabular}}

\end{table*}

\subsubsection{Inductive Setting}
In this subsection, we conduct experiments under the inductive settings, where the test nodes are kept unseen during training. This setting is more suitable for domain generalization.
% But we think it is more convincing that conduct experiments under inductive settings which means test nodes are unseen during training. This setting is more appropriate for domain generalization.

\noindent\textbf{Baselines:} For GOOD-WebKB and GOOD-CBAS datasets, we adopt ERM, IRM, GraphMAE, and GRACE as our baselines. And for Amazon-Photo and Elliptic datasets, we select ERM, EERM, and GRACE as our baselines.

\noindent\textbf{Experimental setup:} For GOOD-WebKB and GOOD-CBAS datasets, we use the same model configuration in Section~\ref{sec:trans}.
% Besides, we add experiments on Amazon-Photo dataset~\cite{yang2016revisiting} and Elliptic~\cite{elliptic} dataset in this subsection. 
For Amazon-Photo dataset~\cite{yang2016revisiting} and Elliptic~\cite{elliptic} dataset, they consist of many snapshots (training data and testing data use different snapshots) which are naturally inductive. For Amazon-Photo dataset, we use 2-layer GCN~\cite{gcn} as the encoder and for elliptic dataset, we use 5-layer GraphSAGE~\cite{sage} as encoder following~\cite{eerm}.

% Figure environment removed

\noindent\textbf{Analysis:}
According to Figure~\ref{fig:amazon},\ref{fig:elliptic},\ref{fig:ind_con},\ref{fig:ind_cov}, we can draw following conclusions:
firstly, based on Figure~\ref{fig:amazon}, it is evident that our method outperforms other representative supervised and self-supervised methods on all test graphs (T1$\sim$T8). This superiority is reflected in the larger median value of our method compared to others. For instance, MARIO achieves over a 3\% absolute improvement compared to ERM in terms of the mean value of eight median values. Additionally, our method demonstrates higher stability across different random initializations, as indicated by the closer proximity of the first and third quartile values to the median value~(\eg, the difference of first and third quartile values of ERM, EERM, GRACE and MARIO are 4.2, 3.3, 6.7 and 1.0 on T8 respectively which indicates MARIO is much more stable than other methods). Furthermore, our method exhibits consistent performance across different graphs (\eg, The standard deviation of median values on T1$\sim$T8 for ERM, EERM, GRACE, and MARIO are 0.4, 1.1, 1.2, and 0.3, respectively.), indicating its robustness to environmental variations and its ability to extract invariant features: $g(G^e) \approx g(G^{e'})$ for all $e, e' \in \mathcal{E}^\text{train}$. In summary, our method showcases enhanced OOD generalization capabilities.
% $g(G^e)g(G^e^\prime)$ where $any e, e^\prime in \mathcal{E}^{train}$

Secondly, from the results presented in Figure~\ref{fig:elliptic}, we can observe that our method averagely harvests 10.9\% absolute improvement over GRACE and 12.5\% absolute improvement over EERM in terms of F1 scores on Elliptic dataset. This demonstrates the effectiveness of our method in handling distribution shifts and improving performance compared to existing approaches. It is worth noting that GRACE's performance worsens over time, indicating its inability to handle distribution shifts effectively. In contrast, our method consistently achieves better F1 scores, except for T9, which is caused by the dark market shutdown occurred after T7~\cite{elliptic}. The emergence of such an event introduces significant variations in data distributions, which subsequently results in performance degradation for all methods. Indeed, this event serves as an unpredictable external factor that introduces significant challenges for models trained on limited training data. The results indicate that the performance heavily depends on available training data. Nonetheless, our approach outperforms other methods even in such an extreme case. This highlights the effectiveness of our method in addressing distribution shifts and improving generalization performance.

Finally, based on the observations from Figure~\ref{fig:ind_con} and Figure~\ref{fig:ind_cov} MARIO demonstrates the best performances on both ID and OOD test sets for GOOD-WebKB and GOOD-CBAS datasets, under both concept shift and covariate shift. Notably, MARIO outperforms other methods by more than 3\% and 10\% absolute improvement on GOOD-WebKB and GOOD-CBAS, respectively, under covariate shift. We can draw similar conclusions as discussed in Section~\ref{sec:trans}. Even under the inductive setting, our method continues to demonstrate excellent OOD generalization capabilities and achieves comparable or even improved in-distribution test performance. These statistical results further validate the effectiveness of our method in handling distribution shifts and enhancing generalization performance.

Overall, the observations we have made provide strong evidence of the great capacity of our method for handling distribution shifts, validating its effectiveness and potential for real-world applications.



% Figure environment removed

% Figure environment removed


% Figure environment removed


\subsection{Ablation Studies}\label{sec:ablation}
\noindent Table~\ref{tab:aba} provides a detailed analysis of the effect of each component according to our proposed recipe for improving OOD generalization in graph contrastive learning. Let's examine the different variants of our method and their impact on performance.
Specifically, MARIO~(w/o ad) represents MARIO without  adversarial augmentation. MARIO~(w/o cmi) denotes we only maximize the mutual information between positive pairs without considering conditional mutual information. MARIO~(w/o cmi, ad) means a vanilla graph contrastive method that is similar to GRACE. 

From Table~\ref{tab:aba}, we can find MARIO~(w/o cmi) lags far behind MARIO on OOD test set which demonstrates appropriately minimizing the redundant information (\ie, conditional mutual information) is essential to improve OOD generalization of GCL methods. And adversarial augmentation can also boost OOD generalization because it can approximately serve as a supermum operator to learn more invariant features  discussed in Section~\ref{sec:aug}. Based on the analysis of these variants, it is evident that the proposed improvements on data augmentation and contrastive loss in the recipe are both effective in enhancing graph OOD generalization. Each component contributes to the overall performance improvement, and their combination leads to a stronger self-supervised graph learner in terms of graph OOD generalization. 

In short, the findings from Table~\ref{tab:aba} support the rationale behind your proposed recipe and provide empirical evidence of the effectiveness of each proposed component. By incorporating these enhancements, our method achieves superior performance in handling distribution shifts and improving graph OOD generalization in graph contrastive learning.
\begin{table*}[htp]
\caption{Ablation studies for MARIO by masking each component.}
\label{tab:aba}
\centering
\scalebox{0.9}{
\begin{tabular}{l|cc|cc|cc|cc|cc}
\toprule
\toprule
\multirow{3}{*}{concept shift} & \multicolumn{4}{c|}{GOOD-Cora}                       & \multicolumn{2}{c|}{GOOD-CBAS} & \multicolumn{2}{c|}{GOOD-Twitch} & \multicolumn{2}{c}{GOOD-WebKB} \\
                           & \multicolumn{2}{c}{word} & \multicolumn{2}{c|}{degree}& \multicolumn{2}{c|}{color}    & \multicolumn{2}{c|}{language}   & \multicolumn{2}{c}{university} \\
                           & ID         & OOD         & ID          & OOD          & ID            & OOD           & ID             & OOD            & ID            & OOD            \\
\midrule
MARIO                      & \textbf{67.11±0.46} & \textbf{65.28±0.34}  & \textbf{68.46±0.40}  & \textbf{61.30±0.28}      & \textbf{94.36±1.21}  & \textbf{91.28±1.10}    & 82.31±0.54     & \textbf{63.33±1.72}     & \textbf{65.67±2.81}    & \textbf{37.15±2.37}     \\
MARIO(w/o ad)              & 66.23±0.53 & 64.02±0.18  & 67.88±0.38  & 60.46±0.29   & 93.21±1.25    & 90.29±0.91    & 82.42±0.73     & 60.50±1.02     & 64.83±2.83    & 36.51±3.25    \\
MARIO(w/o cmi)             & 65.32±0.60 & 63.51±0.32  & 68.14±0.32  & 61.19±0.34   & 94.15±1.23    & 90.57±1.96    & \textbf{82.51±0.56}     & 61.41±2.63     & 64.50±4.35    & 35.78±2.53     \\
MARIO(w/o cmi, ad)         & 64.67±0.55 & 63.11±0.32  & 67.95±0.65  & 60.01±0.57   & 93.36±1.66    & 89.64±1.73    & 81.90±0.75     & 60.12±1.60     & 64.17±3.67    & 34.13±2.38     \\
\bottomrule
\end{tabular}}
\end{table*}
% & 65.32±0.60 & 63.51±0.32 exchange 64.67±0.55 & 63.11±0.32
% 68.14±0.32       id ood test: 60.95±0.43       ood ood test: 61.19±0.34


\subsection{Sensitivity Analysis}\label{sec:sensitivity}
\noindent In this subsection, we will analyze some important hyper-parameters of our method. We conduct sensitivity analysis on GOOD-WebKB dataset with concept shift, we chose two sensitive hyper-parameters (\ie, the coefficient $\gamma$ of condition mutual information in Equation~\ref{equ:cmi} and the number of prototypes $|C|$ in Equation~\ref{equ:pq}). The coefficient of CMI range in $[0.001, 0.01, 0.1, 0.5, 1]$ and the number of prototypes $|C|$ ranges in $[10, 50, 100, 200, 300]$. From Figure~\ref{fig:sensitivity}, we can observe that $\gamma$ reaches 0.1 and $|C|$ reaches 100 or 200 can achieve the best OOD test accuracy. Both higher and lower values of $\gamma$ result in suboptimal performance. This finding aligns with previous research such as DIB~\cite{dib}, indicating that an appropriate compression level is crucial for achieving optimal performance. Extremely high or low compression values are not ideal. 

Regarding the number of prototypes $|C|$, based on the results shown in Figure~\ref{fig:sensitivity}, it is found that setting $|C|=100$ leads to the best performance in terms of OOD test accuracy. This choice provides a moderate number of pseudo labels, which is beneficial for the learning process. 

Based on the sensitivity analysis, we determined that setting $\gamma=0.1$ and $|C|=100$ on most datasets. These hyperparameter values strike a balance between compression level and the number of prototypes, resulting in improved graph OOD generalization.
% Figure environment removed


\subsection{Integrated with Other Models}\label{sec:other_models}
% Figure environment removed

\begin{table}[htp]
\caption{Results of different learning approaches with different encoding models (\ie, GCN, GraphSAGE, GAT).}
\label{tab:others}
\centering
\scalebox{0.9}{
\begin{tabular}{cc|cc|cc}
\toprule
\toprule
\multirow{3}{*}{Model}& \multirow{3}{*}{Method} & \multicolumn{2}{c|}{GOOD-CBAS} & \multicolumn{2}{c}{GOOD-WebKB} \\
                & & \multicolumn{2}{c|}{color}    & \multicolumn{2}{c}{university} \\
                &   & ID          & OOD         & ID          & OOD            \\
\midrule
\multirow{3}{*}{GCN} 
&ERM               & 89.79±1.39 & 83.43±1.19  &  62.67±1.53 & 26.33±1.09         \\
&GRACE             & 92.00±1.39 & 88.64±0.67  &  64.00±3.43 & 34.86±3.43        \\
&MARIO             & 94.36±1.21 & 91.28±1.10  &  65.67±2.81 & 37.15±2.37        \\ \bottomrule
\multirow{3}{*}{SAGE} 
&ERM               & 95.07±1.51 & 75.14±1.19  & 73.67±2.08  & 46.33±3.42       \\
&GRACE             & 95.29±1.11 & 74.43±2.36  & 70.50±5.06  & 49.54±3.83        \\
&MARIO             & 96.00±1.07 & 76.29±3.01  & 71.00±3.82  & 51.74±4.63        \\ \bottomrule
\multirow{3}{*}{GAT} 
&ERM               & 78.64±3.63 & 72.93±2.64  & 61.33±3.71  & 28.99±2.63        \\
&GRACE             & 84.57±1.79 & 78.36±1.60  & 59.50±2.36  & 35.78±3.26        \\
&MARIO             & 84.93±1.95 & 80.43±1.89  & 62.17±4.78  & 38.17±3.10        \\
\bottomrule
\end{tabular}}
\end{table}



\noindent In the subsection, we demonstrate the model-agnostic nature of the recipe by integrating it with various graph neural network (GNN) models, including GCN, GraphSAGE, and GAT.

From Table~\ref{tab:others}, it can be observed that regardless of the specific GNN model used as the encoder, our method consistently achieves the best performance on the OOD test set. This indicates the effectiveness and robustness of our method across different GNN models.
By achieving superior performance across different GNN models, MARIO demonstrates its versatility and ability to improve the OOD generalization of various graph neural models. This highlights the broad applicability and effectiveness of our recipe in enhancing the performance of different GNN encoders.

Furthermore, we integrate our recipe with other GCL methods in Appendix~\ref{app:other_methods}. The results demonstrate our recipe can boost the OOD generalization ability of various GCL methods which means our recipe can serve as a plug-in for many current classical GCL methods.

% Figure environment removed

\subsection{Visualization}\label{sec:vis}
\subsubsection{Metric Score Curves}
We present metric score curves for ERM and MARIO, including training, ID validation, ID testing, OOD validation, and OOD testing accuracy, in Figure~\ref{fig:curve2}. Notably, MARIO demonstrates superior convergence with approximately 10\% absolute improvement on the OOD test set compared to ERM. Furthermore, MARIO effectively narrows the performance gap between in-distribution and out-of-distribution performance, showcasing its efficacy in enhancing OOD generalization for graph data. More metric score curves can be found in Appendix~\ref{app:curves}.


\subsubsection{Feature Visualization}
In order to assess the quality of learned embeddings, we adopt t-SNE~\cite{tsne} to visualize the node embedding on GOOD-Cora dataset (concept shift in word domain) using random-init of GCN, EERM, GRACE, and MARIO, where different classes have different colors in Figure~\ref{fig:vis}. For clarity, we select eight classes with the largest number of nodes to enhance the informativeness and interpretability of the visualization. We can observe that the 2D projection of node embeddings learned by MARIO has a better separation of clusters, which indicates the model can help learn representative features for downstream tasks. It has to note that we depict both ID nodes and OOD nodes in the same figure. 

Besides, we also separately visualize ID nodes and OOD nodes in the different figures in the Appendix~\ref{app:feature}. And we can find MARIO performs a clearer separation of clusters whether on ID nodes or OOD nodes compared to other methods.




\section{Conclusion}
This paper introduced quantization with incoherence processing (QuIP), an algorithm consisting of (1) an optimal adaptive rounding procedure which minimizes a quadratic proxy of the weight error, and (2) efficient pre- and post-processing to ensure the incoherence of the weight and Hessian matrices by multiplying them by a Kronecker product of random orthogonal matrices.
We showed that QuIP quantization is optimal in a general class of adaptive rounding methods with linear feedback; this theoretical analysis is the first for any quantization algorithm that scales to LLM-sized models. 

Empirically, QuIP achieves the first viable two-bit quantization results for LLMs, especially at large model sizes,
hinting at the feasibility of accurate 2-bit inference in LLMs.

\section*{Acknowledgements and Disclosure of Funding}
This work was partially funded by the National Science Foundation under awards DGE-1922551, CAREER awards 2046760 and 2145577, by the National Institute of Health under award MIRA R35GM151243, and a gift from CISCO.

\newpage

\bibliography{references}
\bibliographystyle{plainnat}

\end{document}