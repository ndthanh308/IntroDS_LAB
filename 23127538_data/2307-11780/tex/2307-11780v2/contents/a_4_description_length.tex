\section{Description Length Calculation}

\label{app:mdl}

This section provides the complete mathmatical calculation of the description length.

As the basis, we consider the length of each code, which can be computed by Shannon entropy \cite{shannon1948mathematical}.
For pattern $p$, the length of the pattern code $code_p(p)$ is the negative log-likelihood
\begin{displaymath}
    L\left(code_p\left(p\right)\right)=-lg2\left(\frac{usage\left(p\right)}{\sum_{p_i \in CT}usage\left(p_i\right)}\right),
\end{displaymath}
where $lg2(k)$ means the logarithm of $k$ to the base $2$, and $usage(p)$ is the number of pattern codes of pattern $p$ used to encode $S$.
Similarly, we can give the calculation formulas for gap code $code_g(p)$, fill code $code_f(p)$, and miss code $code_m(p)$ as follows.
\begin{displaymath}
    \begin{aligned}
        L\left(code_g\left(p\right)\right)=&-lg2\left(\frac{gaps\left(p\right)}{gaps\left(p\right)+fills\left(p\right)+misses\left(p\right)}\right), \\
        L\left(code_f\left(p\right)\right)=&-lg2\left(\frac{fills\left(p\right)}{gaps\left(p\right)+fills\left(p\right)+misses\left(p\right)}\right), \\
        L\left(code_m\left(p\right)\right)=&-lg2\left(\frac{misses\left(p\right)}{gaps\left(p\right)+fills\left(p\right)+misses\left(p\right)}\right) \\ &+ L_N\left(~|~A~|~\right),
    \end{aligned}
\end{displaymath}
where $gaps(p)$, $fills(p)$, and $misses(p)$ are the number of gap codes, fill codes, and miss codes, respectively, of pattern $p$.
Miss code $code_m(p)$ contains additional bits for the index of missing attributes, which is denoted by $L_N(|A|)$.
Function $L_N(k)$ represents the number of bits required to encode integer $k$, where the MDL optimal Universal code for integers is considered \cite{grunwald2007minimum}.

\textbf{The encoded length of the code table.} To obtain a minimum description length, we treat patterns in $ST$ and patterns in $CT^*$ differently, where $L(P)=L(ST)+L(CT^*)$.

For $ST$, we consider each attribute separately and encode the number of optional values and their supports:
\begin{displaymath}
    L\left(ST\right)=\sum_{1\leq k \leq ~|~A~|~}\left(L_N\left(~|~V_k~|~\right)
    + log\left(
        \begin{array}{c}
                ~|~S^k~|~ \\
                ~|~V_k~|~
            \end{array}
        \right)
    \right),
\end{displaymath}
where $S^k$ is a univariate dataset that preserves the $k$-th attribute of each event in dataset $S$.

For $CT^*$, we encode the number of patterns, the sum of their usages, the distribution of their usages over different patterns, and the original patterns:
\begin{displaymath}
    \begin{aligned}
        L\left(CT^*\right)= & L_N\left(~|~P^*~|~\right)+L_N\left(usage\left(P^*\right)\right)            \\
                            & +log\left(\begin{array}{c}
                ~|~usage\left(P^*\right)~|~ \\
                ~|~P^*~|~
            \end{array}\right)+\sum_{p_i\in P^*}L\left(p_i\right),
    \end{aligned}
\end{displaymath}
where $P^*$ is the set of all patterns in $CT^*$.
Considering that $|~P^*~|$ and $usage(P^*)$ can be zero, we define $L_N(0)=0$.
For a non-singleton pattern $p_i$, we encode the number of events, the number of values, the number of gaps, the number of misses, and the first column (i.e., each value in the pattern) as
\begin{displaymath}
    \begin{aligned}
        L\left(p_i\right) = & L_N\left(~|~p_i~|~\right)+L_N\left(~||~p_i~||~\right)                            \\
                            & +L_N\left(gaps\left(p_i\right)+1\right)+L_N\left(misses\left(p_i\right)+1\right) \\
                            & +\sum_{v\in p_i}L\left(code_p\left(v~|~ST\right)\right),
    \end{aligned}
\end{displaymath}
where $L(code_p(v ~|~ ST))$ represents the encoded length of a value $v$ in pattern $p_i$.
Here, we directly use the pattern code of singleton value $v$ in $ST$.

\textbf{The encoded length of all code streams} is the sum of the description length of four types of codes:
\begin{displaymath}
    \begin{aligned}
        L\left(S~|~P\right)= & \sum_{p_i \in CT}usage\left(p_i\right)L\left(code_p\left(p_i\right)\right)     \\
                          & + \sum_{p_i \in CT}gaps\left(p_i\right)L\left(code_g\left(p_i\right)\right)    \\
                          & + \sum_{p_i \in CT}fills\left(p_i\right)L\left(code_f\left(p_i\right)\right)   \\
                          & + \sum_{p_i \in CT}misses\left(p_i\right)L\left(code_m\left(p_i\right)\right).
    \end{aligned}
\end{displaymath}