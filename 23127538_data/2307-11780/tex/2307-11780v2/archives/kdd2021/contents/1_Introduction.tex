\section{Introduction}

\label{sec:Introduction}

Event sequence data -- i.e. data consisting of time-stamped or ordered events --- is widely used in many fields, including medical data analysis, sports data analysis, and consumer behavior analysis.
% Sequential data is widely used in many fields, including genetics, medical data analysis, sports data analysis, and consumer behavior analysis.
To better comprehend large sets of event sequences and obtain useful insights, analysts in these fields often use pattern mining algorithms~\cite{mabroukeh2010taxonomy,fournier2017survey} to discover and summarize patterns, i.e., the subsequences that occur frequently.

% multiple attributes bring two requirements
As information technology continues to progress, enabling the collection of more specific data, real-world event sequence data becomes more complex \cite{gan2018survey}.
% With the development of information technology in these fields, real-world event sequences data becomes more complex \cite{gan2018survey}.
For example, data produced during a table tennis game may now include the players' position, the ball's position, the technique used to hit, the spin direction of the ball, and more.
To work well with event sequences that include multiple attributes --- also known as multivariate event sequences --- pattern mining algorithms should be both \textbf{effective} and \textbf{efficient}.
% For example, to describe a hit event in tennis, we need to record the players' position, the ball's position, the technique used to hit, and the spin direction of the ball, et al.
% The multivariate event sequence data brings two requirements to current pattern mining algorithms: \textbf{effectiveness} and \textbf{efficiency}.

\textbf{Effective} pattern mining algorithms can discover patterns that reveal meaningful information about the sequences. To be effective, a pattern mining algorithm for multivariate event sequences must fulfill three conditions:
(1) The correlations among multiple attributes within sequences should be preserved for domain analysis.
For example, in table tennis, tactical patterns that reveal the players' playing style can involve multiple attributes.
One tactical pattern may be that when a player hits the ball to a certain \textit{position} on the table, the opponent always uses a specific \textit{technique} in response.
Univariate patterns cannot reveal the multivariate tactical patterns comprehensively.
% (1) It must be able to preserve correlations among multiple attributes within extracted patterns for domain analysis.
(2) The algorithm should have a high tolerance for single-value noises.
For example, in table tennis, when a player applies a tactical pattern, he/she may only change the technique of one hit to a similar technique, retaining the overall playing style.
We regard the changes on only one attribute as single-value noises, which should not prevent the algorithm from detecting the overall pattern.
% (2) The algorithm should have a high tolerance for noise data --- only one attribute is allowed to have a noise value in the data record.
% (2) It must have a high tolerance for noise data. Because noise is an essential problem in multivariate sequences, effective algorithms allow only one attribute that has a noise value in the data record.
(3) The number of returned patterns should be manageable. Given that multivariate patterns are complicated, analyzing them is time-consuming.
Only a small number of meaningful patterns should be returned to reduce the burden of analysis.
% (3) It must identify a manageable number of helpful patterns.

% Second, we need an efficient pattern mining algorithm that can summarize patterns within a few seconds.
% Second, we need an efficient pattern mining algorithm whose response time is acceptable---typically within a few seconds---for an interactive application.
% The outputs of a pattern mining algorithm eventually should be interpreted by human users.
% Thanks to the human eye's inherence to perceive information in parallel, visualization is a preferred way for analysts to explore multivariate event sequences~\cite{loorak2015timespan,wu2020visual}.
% Thanks to the human eye's inherence to perceive information in parallel, visualization is a preferred way for domain experts to explore multivariate sequential data~\cite{loorak2015timespan,wu2020visual}.
% In a visual analytics system, interactivity plays an essential role in supporting comprehensive data exploration.
% A pattern mining algorithm with high efficiency is the basis of seamless interactions.
\textbf{Efficient} pattern mining algorithms can mine multivariate patterns within an acceptable response time.
% Multivariate patterns may contain values of different attributes.
In a real-world scenario, the time allowed for pattern analysis is limited and cannot go forever.
Another scenario is that some parameters should be adjusted based on the analysts' feedback so that the patterns can satisfy the analysts' requirements.
In both scenarios, the algorithm is expected to return the results as soon as possible.
However, the multivariate patterns may contain values of different attributes, thus resulting in the huge search space of multivariate patterns.
A truly efficient algorithm is able to explore this huge, multivariate search space in a reasonable amount of time.

To the best of our knowledge, existing multivariate pattern mining algorithms cannot satisfy these two requirements simultaneously.
One set of traditional sequential pattern mining algorithms is based on thresholding \cite{fournier2017survey}. These algorithms tend to return an enormous number of patterns rather than seeking the most meaningful ones.
Some multivariate pattern mining algorithms \cite{morchen2007efficient,chen2010efficient,bertens2014characterising} transform multivariate sequences into univariate ones, such that extracted patterns cannot retain any correlations between attributes.
DITTO~\cite{bertens2016keeping} summarizes patterns with correlations between multiple attributes, but does not have high noise tolerance.

% our algorithm
In this paper, we propose \textsc{Beep}, a novel sequential pattern-mining algorithm which \textbf{B}alances \textbf{E}ffectiveness and \textbf{E}fficiency when finding \textbf{P}atterns.
Our algorithm is based on the Minimum Description Length (MDL) principle \cite{grunwald2007minimum}, chosen for its ability to summarize numerous sequences with a small number of informative patterns.
\textsc{Beep} considers a set of multivariate patterns as the model to preserve correlations among multiple attributes.
The main contributions of \textsc{Beep} consists of two parts.
First, for each pattern in the model, \textsc{Beep} introduces a new encoding scheme to encode the noise values in a dataset.
Second, to efficiently discover such an informative model, \textsc{Beep} applies a tailored acceleration method based on \textit{Locality Sensitive Hashing} so that the patterns with high frequencies can be found in a short time.
% \ldy{More description here about our algorithm to highlight what is novel here? What we have done to make the satisfaction of the two requirements possible?}

% evaluation
To evaluate the effectiveness of \textsc{Beep}, we worked with analysts to conduct an empirical study on a real-world dataset, namely the table tennis data.
Our tests show that \textsc{Beep} can mine a small number of informative multivariate patterns and handle noise well.
We conducted further quantitative experiments with synthetic datasets to compare \textsc{Beep} with the current state-of-art algorithm.
% To empirically evaluate our algorithm, we conduct experiments on both real-world datasets and synthetic datasets.
% Based on the results of the experiments, we further compare our algorithm with the state-of-art algorithms.
The results of the experiments prove that \textsc{Beep} can discover patterns that summarize the dataset better and save a great amount of time.

% organization of this paper
% We organize the rest of this paper as follows.
% We first define the data structure in Section \ref{sec:Predefinitions} and briefly introduce the MDL principle in Section \ref{sec:MDL}.
% Then, we illustrate our algorithm in detail in Section \ref{sec:F4M} and discuss related work in Section \ref{sec:RelatedWork}.
% Furthermore, we evaluate our algorithm in Section \ref{sec:Experiments}.
% Finally, we discuss and conclude our work in Section \ref{sec:Discussion} and Section \ref{sec:Conclusion}, respectively.