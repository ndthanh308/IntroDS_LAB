\subsection{Empirical Study}

\subsubsection{Dataset}

To empirically evaluate the effectiveness of our algorithm, we conducted a case study on a real-world table tennis dataset.
Table tennis data is a typical example of multivariate event sequence data.
In table tennis singles matches, two players alternately hit the ball, beginning with one player's serve and ending with one player winning the game.
We considered each hit as an event and the consecutive hits as a sequence.
For each hit, we considered four attributes; namely, the technique the player used to hit the ball (\textit{Tech}), the area where the ball impacted the table (\textit{Ball}), the spin of the ball (\textit{Spin}), and the position of the player (\textit{Player}).
The optional values of these four attributes can be found in the Appendix \ref{app:values}.
We collected sequences from 10 matches (quarterfinals or later) played in 2019 by \textit{Ito Mima}, one of the top players in the world, against Chinese players.
We filtered sequences where \textit{Ito} served the ball in order to analyze her tactical patterns in this situation.
We were left with 716 sequences, of which the average length was 5.91.

\input{figures/caseTT}

\begin{table}
    \caption{The results of the case study on the table tennis data. We compared two algorithms, namely \textsc{Ditto} and \textsc{Beep}, on the number of patterns ($|P|$), the average frequency of each pattern ($avg.$ $freq.$), and the number of missing values in 716 sequences ($miss$, only for \textsc{Beep}), and the runtime ($t$).}
    \label{tab:empirical}
    \begin{tabular}{rrrrr}
        \toprule

        % $|S|$&\textit{avg.} $|s_i|$&$|A|$&$|E|$ &
        \textit{Algorithm} & $|P|$ & $avg.$ $freq.$ & $miss$ & $t$(s) \\

        \midrule

        \textsc{Ditto} & 161 & 4.36 & -- & 1747 \\

        \textsc{Beep} & 62 & 12.94 & 280 & 425 \\

        \bottomrule
    \end{tabular}
\end{table}

\subsubsection{Results analysis}

We ran both \textsc{Ditto} and \textsc{Beep} on this dataset and summarized the results (shown as Table \ref{tab:empirical}) as follows.

\textbf{\textsc{Beep} can summarize a smaller set of informative patterns than \textsc{Ditto}.}
\textsc{Beep} summarized 63 non-singleton patterns, while \textsc{Ditto} summarized 161.
According to the analysts we worked with, patterns with more than 4 values and used more than 10 times tend to be considered more worthy of analysis. Of these, \textsc{Beep} summarized 7 (used 530 times), while \textsc{Ditto} summarized 29 (used 546 times).
This was because many of the patterns that \textsc{Ditto} summarized as distinct differed in only one value, and \textsc{Beep} instead summarized these as one pattern with missing values.
As a result, \textsc{Beep} reduces the analysis burden of analysts.
Considering that the number of patterns is completely controlled by the algorithm, rather than manually input parameters, we believe that \textsc{Beep} is better than \textsc{Ditto} in summarizing a smaller set of informative patterns.

\textbf{\textsc{Beep} is more efficient than \textsc{Ditto} and performs similarly in information compression.}
We elaborate on these findings through quantitative experiments described in Section \ref{sec:quantitative}.

\subsubsection{Insights}

We discussed the summarized patterns with two analysts, each of whom has over three years of experience analyzing table tennis data for one of the best national teams in the world.
The patterns afforded the two analysts many insights into \textit{Ito}'s playing.
Due to space limitations, we'll focus here on one of the most common tactical patterns, shown as $p$ in Figure \ref{fig:tt}.

\begin{table*}
    \caption{The results of the quantitative experiments. Given 5 synthetic datasets, one for each row, we compared five algorithms, namely, \textsc{Ditto}, \textsc{Beep}, \textsc{Beep}-miss, \textsc{Beep}-LSH, and \textsc{Beep}-none. For each dataset, we reported the number of sequences ($|S|$), the length of each sequence ($|s|$), the number of attributes ($|A|$), and the number of optional values of each attribute ($|V_k|$). For each algorithm, we reported the number of patterns ($|P|$), the description length reduction compared to the singleton-only model ($\Delta L\%$), the number of missing values for \textsc{Beep}-miss and \textsc{Beep} ($miss$), and the runtime in seconds ($t$). (\tablet{orange}: the shortest time; \tabledL{blue}: the largest description length reduction.)}
    \label{tab:quantitative}
    \begin{tabular}{rrrr|rrr|rrrr|rrrr|rrr|rrr}
      \toprule

      \multicolumn{4}{l|}{Synthetic Datasets}&\multicolumn{3}{l|}{\textsc{Ditto}}&\multicolumn{4}{l|}{\textsc{Beep}}&\multicolumn{4}{l|}{\textsc{Beep}-miss}&\multicolumn{3}{l|}{\textsc{Beep}-LSH}&\multicolumn{3}{l}{\textsc{Beep}-none} \\

      \cmidrule(lr){1-4}\cmidrule(lr){5-7}\cmidrule(lr){8-11}\cmidrule(lr){12-15}\cmidrule(lr){16-18}\cmidrule(lr){19-21}

      $|S|$&  $|s_i|$&      $|A|$&  $|V_k|$&
      $|P|$&  $\Delta L\%$& $t$(s)&
      $|P|$&  $\Delta L\%$& $miss$& $t$(s)&
      $|P|$&  $\Delta L\%$& $miss$& $t$(s)&
      $|P|$&  $\Delta L\%$& $t$(s)&
      $|P|$&  $\Delta L\%$& $t$(s) \\

      \midrule

      50& 20& 5& 100&
      19&   24.5&     399&
      9&   \tabledL{27.2}&     33& 79&
      9&   \tabledL{27.2}&     33&455&
      16&   24.4&     \tablet{43}&
      19&   24.5&     380
      \\

      \textbf{70}& 20& 5& 100&
      12&   19.0&     987&
      14&   \tabledL{21.3}&     11& 215&
      14&   \tabledL{21.3}&     16&1238&
      12&   18.9&     \tablet{29}&
      12&   19.0&     994
      \\

      50& \textbf{30}& 5& 100&
      20&   25.8&     772&
      13&   26.6&     28& 402&
      14&   \tabledL{26.7}&     29& 1276&
      20&   25.9&     \tablet{120}&
      20&   25.8&     753
      \\

      50& 20& \textbf{7}& 100&
      15&   27.7&     1040&
      9&   29.1&     30& 251&
      9&   \tabledL{29.2}&     31& 1141&
      14&   27.7&     \tablet{43}&
      15&   27.7&     1028
      \\

      50& 20& 5& \textbf{200}&
      17&   24.1&     175&
      9&   \tabledL{25.3}&     39& \tablet{18}&
      10&   25.2&     39& 149&
      16&   24.1&     19&
      17&   24.1&     168
      \\

      \bottomrule
    \end{tabular}
  \end{table*}

\textbf{\textsc{Beep} can find multivariate patterns that reveal the correlations among multiple attributes.}
In pattern $p$, \textit{Ito} first positioned herself at the \textit{Backhand} area and served with the \textit{Pendulum} technique.
Next, \textit{Ito}'s opponent stood at \textit{Backhand} and received the ball as it hit the table at the \textit{Short Backhand} position, using the control technique \textit{Push}.
Finally, \textit{Ito} stood at the \textit{Forehand} position and received the ball at \textit{Half-long Forehand} hitting with the technique \textit{Attack} and spinning type \textit{Sink}.
This pattern showed that \textit{Ito} preferred to use the attack technique at the third hit when her opponent used a control technique at the second hit.
At the same time, \textit{Ito} stood where she could hit the ball with her forehand, which results in a faster hit.
Furthermore, the \textit{Sink} spin made the ball lose height, giving her opponent less of a chance to return it well.
The analysts concluded that an excellent tactical pattern, which led to a high winning rate, benefits from various details.

\textbf{Missing values contribute to data analysis.}
The analysts found two sequences, $s_1$ and $s_2$ in Figure \ref{fig:tt}, that contain this pattern.
Sequence $s_1$ has a missing value at the second hit, where \textit{Ito}'s opponent used the technique \textit{Touch Short}, a control technique similar to \textit{Push}.
The analysts believed that $p$ summarized $s_1$ well, despite the missing value.
Sequence $s_2$ has a missing value at the third hit, where \textit{Ito} received the ball at \textit{Half-long Backhand}.
However, given that \textit{Ito} stood at the \textit{Forehand} area, it didn't seem possible for her to receive this ball.
The analysts checked the data and found this point to be an anomaly.
If we had instead used the \textsc{Ditto} algorithm, these two sequences would have been encoded by other short patterns and thus unable to reveal these insights.

% \subsubsection{ECG data}

% % data description
% ECG data consists of real-valued time series detected by sensors\footnote{physionet.org/physiobank/database/stdb}.
% Here we demonstrate how \textsc{Beep} discovers multivariate patterns in time series data.
% Since \textsc{Beep} only accepts event sequence data as input, we first need to transform the time series data into event sequence data.
% In our case, we chose two time series as detected by two sensors to construct a long sequence with two attributes.
% In order to discover patterns that could reveal overall characteristics, we sampled the two time so that every fifty real values were summarized by their average value.
% Next, we discretised the real values into categorical attributes by SAX \cite{lin2007experiencing}.
% Finally, we obtained a sequence of 10740 events with two attributes.
% Each attribute had five optional values.

% % insights
% \fillblank{A paragraph to describe insights}

% % Figure environment removed
