\section{Discussion}

\label{sec:Discussion}

% significance
Through the case study on the real-world dataset, we empirically evaluated the effectiveness of \textsc{Beep}.
% The quantitative experiments on synthetic data show that \textsc{Beep} can strike a good balance between effectiveness and efficiency.
In practice, the analysts usually focused on the top five patterns, which had more events, more values, and higher support.
These patterns provided more insights into correlations among multiple attributes.
In the large dataset, \textsc{Beep} detected many missing values, where most missing values were substitutions of similar values and some were anomalies.
Thanks to the detection of missing values, some further studies could be conducted, such as analyzing the distribution of similar values, cleaning the data, and so on.

We quantitatively evaluated the efficiency of \textsc{Beep} through an ablation study on synthetic data. The results showed that the introduction of missing values could result in larger patterns and smaller description length, and that LSH-based acceleration could enhance runtime performance.
When both are employed, \textsc{Beep} strikes a good balance between effectiveness and efficiency.

% limitations:
% 1. different set of attributes for different events
% 2. different weight of different attributes
Through the analysts' feedback, we find two limitations.
First, some datasets may have different attributes in different events, such that correlations among attributes are more complicated.
For example, in soccer, the shooting angle is an essential attribute in a shooting event, but is useless in a passing event.
A possible solution is to assume each event has all attributes, and fill the irrelevant ones with empty values so that all events have the same attributes.
Second, some attributes may be more important for analysis, and could be assigned greater weights.
One possible solution for this is to give a longer code length to singletons of more important attributes, so that more non-singleton patterns related to these attributes can be summarized to compress the description.

% future work
% 1. interactive pattern mining
% 2. GPU acceleration
In the future, we plan to extend \textsc{Beep} for interactive pattern mining.
Although MDL considers the pattern set that minimizes the description length as the best, in practice, some ``best'' patterns cannot be explained by the analysts.
We expect an algorithm that can keep humans in the loop, so that analysts can interactively guide the algorithm to discover interpretable patterns.
Another future project could involve using GPU to accelerate the algorithm by running the covering and summarizing algorithms in parallel.