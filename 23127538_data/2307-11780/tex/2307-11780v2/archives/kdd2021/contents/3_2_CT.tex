\subsection{Code Table}

\label{sec:ct}

In prior work, code tables (\textit{CT}), which provide an encoding scheme for every pattern, have been widely used for model design.
We follow the design of code tables and further extend it to support two advanced features: gap events (\ref{feature:gap}) and missing values (\ref{feature:miss}).
As shown in Fig. \ref{fig:ct}, each row of a code table is associated with a pattern $p_i$, which is recorded in the first column.
To preserve the correlations between multiple attributes, the pattern can be multivariate.
The four columns on the right record four types of codes.
The second column records a pattern code $code_p(p_i)$ that represents the first event of the pattern, which indicates the occurrence of the pattern.
The gap code $code_g(p_i)$ (the third column) and the fill code $code_f(p_i)$ (the fourth column) support the gap events feature.
The two codes are introduced and well evaluated in prior works \cite{tatti2012long, bertens2016keeping}.
A gap code represents a gap event in the sequence, and a fill code represents an event in the pattern to be filled in the sequence (with the exception of the first event represented by the pattern code).

We propose a new code (the fifth column), namely the miss code $code_m(p_i)$, to allow for the missing values feature \ref{feature:miss}.
A miss code is a number to indicate the index of the missing value.
After an event is encoded by a pattern code or a fill code, there may exist a miss code to show which attribute of the event is missing.

When we implement the code table, two more constraints arise.
First, following our limitations on gap events and missing values, for a pattern $p_i$, the number of the gap code must be smaller than $|p_i|$, the number of the fill code must be $|p_i| - 1$, and the number of the missing code must be smaller than $\lfloor||p_i|| \div 10 + 0.5\rfloor$.
Given these limitations, we find it useless to record the gap code and the fill code for a pattern $p$ that satisfies $|p| = 1$ and to record the miss code for a pattern $p$ that satisfies $||p|| < 5$.
Thus, we delete these useless codes to shorten the description length.

Second, to ensure that a sequence can be completely covered by the model, we further extend $CT$ with all singleton values, which are regarded as patterns with only one value.
The extended rows are regarded as the standard code table ($ST$) because it is the minimum $CT$.
Obviously, the patterns in $ST$ cannot deliver insight into the sequences.
To distinguish the informative patterns with more than one value from these singleton patterns, we define $CT^*$ as the table of all the patterns in $CT$ but not in $ST$.
The output of our algorithm is all the patterns in $CT^*$.

\input{figures/encoding}