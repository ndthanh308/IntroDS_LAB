%%
%% This is file `sample-authordraft.tex',
%% generated with the docstrip utility.
%%
%% The original source files were:
%%
%% samples.dtx  (with options: `authordraft')
%%
%% IMPORTANT NOTICE:
%%
%% For the copyright see the source file.
%%
%% Any modified versions of this file must be renamed
%% with new filenames distinct from sample-authordraft.tex.
%%
%% For distribution of the original source see the terms
%% for copying and modification in the file samples.dtx.
%%
%% This generated file may be distributed as long as the
%% original source files, as listed above, are part of the
%% same distribution. (The sources need not necessarily be
%% in the same archive or directory.)
%%
%% The first command in your LaTeX source must be the \documentclass command.
% \documentclass[sigconf,anonymous]{acmart}
\documentclass[sigconf,anonymous]{acmart}
% \documentclass[manuscript,screen,review,anonymous]{acmart}
%% NOTE that a single column version may be required for
%% submission and peer review. This can be done by changing
%% the \doucmentclass[...]{acmart} in this template to
%% \documentclass[manuscript,screen,review]{acmart}
%%
%% To ensure 100% compatibility, please check the white list of
%% approved LaTeX packages to be used with the Master Article Template at
%% https://www.acm.org/publications/taps/whitelist-of-latex-packages
%% before creating your document. The white list page provides
%% information on how to submit additional LaTeX packages for
%% review and adoption.
%% Fonts used in the template cannot be substituted; margin
%% adjustments are not allowed.
%%
%% \BibTeX command to typeset BibTeX logo in the docs

\usepackage{enumitem}
\usepackage[ruled,linesnumbered,noend]{algorithm2e}

\newcommand\mycommfont[1]{\footnotesize\ttfamily\textcolor{black}{#1}}
\SetCommentSty{mycommfont}

\newcommand{\TODO}[1]{\textcolor{blue}{[\textit{\textbf{TODO:~}}#1]}}
\newcommand{\fillblank}[1]{\textcolor{gray}{#1}}
\newcommand{\ldy}{\textcolor[rgb]{1, 0, 0}}
\newcommand{\tabledL}[1]{\textcolor[rgb]{0.215, 0.494, 0.722}{\textbf{#1}}}
\newcommand{\tablet}[1]{\textcolor[rgb]{1.0, 0.498, 0.0}{\textbf{#1}}}

\AtBeginDocument{%
  \providecommand\BibTeX{{%
    \normalfont B\kern-0.5em{\scshape i\kern-0.25em b}\kern-0.8em\TeX}}}

%% Rights management information.  This information is sent to you
%% when you complete the rights form.  These commands have SAMPLE
%% values in them; it is your responsibility as an author to replace
%% the commands and values with those provided to you when you
%% complete the rights form.
\setcopyright{acmcopyright}
\copyrightyear{2021}
\acmYear{2021}
% \acmDOI{10.1145/1122445.1122456}

%% These commands are for a PROCEEDINGS abstract or paper.
\acmConference[KDD '21]{KDD '21: ACM SIGKDD Conference on Knowledge Discovery and Data Mining}{August 14--18, 2021}{Singapore}
\acmBooktitle{KDD '21: ACM SIGKDD Conference on Knowledge Discovery and Data Mining, August 14--18, 2021, Singapore}
% \acmPrice{15.00}
% \acmISBN{978-1-4503-XXXX-X/18/06}


%%
%% Submission ID.
%% Use this when submitting an article to a sponsored event. You'll
%% receive a unique submission ID from the organizers
%% of the event, and this ID should be used as the parameter to this command.
\acmSubmissionID{31}

%%
%% The majority of ACM publications use numbered citations and
%% references.  The command \citestyle{authoryear} switches to the
%% "author year" style.
%%
%% If you are preparing content for an event
%% sponsored by ACM SIGGRAPH, you must use the "author year" style of
%% citations and references.
%% Uncommenting
%% the next command will enable that style.
%%\citestyle{acmauthoryear}

%%
%% end of the preamble, start of the body of the document source.
\begin{document}

%%
%% The "title" command has an optional parameter,
%% allowing the author to define a "short title" to be used in page headers.
%%
%% If your title is lengthy, you must define a short version to be used
%% in the page headers, to prevent overlapping text. The \verb|title|
%% command has a ``short title'' parameter:
%% \begin{verbatim}
%%   \title[short title]{full title}
%% \end{verbatim}
\title{\textsc{Beep}: An Effective and Efficient Pattern Mining Algorithm\\for Multivariate Event Sequence Data}

%%
%% The "author" command and its associated commands are used to define
%% the authors and their affiliations.
%% Of note is the shared affiliation of the first two authors, and the
%% "authornote" and "authornotemark" commands
%% used to denote shared contribution to the research.
\author{Jiang Wu}
\email{wujiang5521@zju.edu.cn}
\affiliation{%
  \institution{Zhejiang Lab}
  \city{Hangzhou}
  \state{Zhejiang}
  \country{China}
}

\author{Dongyu Liu}
\email{dongyu@mit.edu}
\affiliation{%
  \institution{Massachusetts Institute of Technology}
  \city{Cambridge}
  \state{MA}
  \country{USA}
}

\author{Ziyang Guo}
\email{ziyangguo27@zju.edu.cn}
\affiliation{%
  \institution{Zhejiang University}
  \city{Hangzhou}
  \state{Zhejiang}
  \country{China}
}

\author{Yingcai Wu}
\authornote{Yingcai Wu is the corresponding author.}
\email{ycwu@zju.edu.cn}
\affiliation{%
  \institution{Zhejiang Lab}
  \city{Hangzhou}
  \state{Zhejiang}
  \country{China}
}

%%
%% By default, the full list of authors will be used in the page
%% headers. Often, this list is too long, and will overlap
%% other information printed in the page headers. This command allows
%% the author to define a more concise list
%% of authors' names for this purpose.
% \renewcommand{\shortauthors}{Wu, et al.}

%%
%% The abstract is a short summary of the work to be presented in the
%% article.
\begin{abstract}
  Pattern mining is essential for the analysis of event sequence data.
  However, analysis of multivariate event sequence data, a common type of real-world data where each event includes multiple attributes, requires pattern mining algorithms to be highly effective and efficient.
  In this paper, we propose \textsc{Beep}, a novel algorithm that can efficiently discover a small set of meaningful patterns based on the Minimum Description Length principle.
  In particular, \textsc{Beep} introduces a new encoding scheme to discover patterns that can reveal correlations among multiple attributes and are highly tolerant of noise.
  Furthermore, \textsc{Beep} applies a tailored acceleration method based on Locality-Sensitive Hashing, which can greatly reduce the time for summarizing patterns.
  Empirical study on a real-world dataset shows that patterns discovered by \textsc{Beep} help analysts gain new and effective insights, while quantitative experiments on multi-scaled synthetic data show that our algorithm can discover these meaningful patterns about 5 times faster than the current state-of-art algorithm.
\end{abstract}

%dyu
% Identifying frequent sequential patterns among event sequence data is essential tasks in many disciplines.
% However, in real world, 

%%
%% The code below is generated by the tool at http://dl.acm.org/ccs.cfm.
%% Please copy and paste the code instead of the example below.
%%
\begin{CCSXML}
<ccs2012>
<concept>
<concept_id>10003752.10003809.10010031.10010032</concept_id>
<concept_desc>Theory of computation~Pattern matching</concept_desc>
<concept_significance>500</concept_significance>
</concept>
</ccs2012>
\end{CCSXML}

\ccsdesc[500]{Theory of computation~Pattern matching}

%%
%% Keywords. The author(s) should pick words that accurately describe
%% the work being presented. Separate the keywords with commas.
\keywords{efficient algorithm, multivariate event sequence, MDL principle}

%% A "teaser" image appears between the author and affiliation
%% information and the body of the document, and typically spans the
%% page.
% \begin{teaserfigure}
%   % Figure removed
%   \caption{Seattle Mariners at Spring Training, 2010.}
%   \Description{Enjoying the baseball game from the third-base
%   seats. Ichiro Suzuki preparing to bat.}
%   \label{fig:teaser}
% \end{teaserfigure}

%%
%% This command processes the author and affiliation and title
%% information and builds the first part of the formatted document.
\maketitle

%%%%%%%%%%%%%%%%%%%%%%%%%%%%%%%%%%%%%%%%%%%%%%%%%%%%%%%%%%%%%%%%%%%%%%%%%%%%%%%%
\section{Introduction}

Autonomous driving (AD) %with deep learning networks 
has shown promising achievements and is considered an important technological breakthrough that could revolutionize the future of transportation. Currently, ensuring the safety of autonomous driving systems has become a topic of extensive development.
% There has been much discussion on how to verify the safety of autonomous driving systems.
One traditional solution for safety tests is to exhaustively enumerate real scenarios for validation. Nevertheless, this process is not only labor-intensive and costly but also dangerous. Simulation has emerged as a robust, safe, and efficient alternative for training and evaluating AD software and algorithms~\cite{li2019aads, amini2020learning, amini2022vista}.

% Figure environment removed

Recently, neural radiance field (NeRF)~\cite{mildenhall2020nerf} has gained significant attention in AD simulation~\cite{drivesim}. This approach leverages multi-view images to construct a 3D scene and enable novel view synthesis for both indoor and outdoor applications. When it comes to constructing NeRF models in AD simulation, there are two options available: 1) collecting a large amount of data to cover as many viewpoints as possible, and constructing a fine-grained scene offline; 2) directly using log data from road tests to quickly create an environment and dynamically simulate driving scenarios. The first choice can deliver high-quality simulation~\cite{tancik2022block} by transforming the problem of view extrapolation into view interpolation through the use of large amounts of data. However, it is time- and cost-intensive, which makes it challenging to generalize. As for the second choice, the collected images from log data are usually similar to each other along the running trajectory, which may result in unsatisfactory outcomes, particularly when the camera pose is placed out-of-trajectory (see \figref{figSupportComp} as an example), semantic consistency cannot be guaranteed when synthesizing images from deviated views. We observe this problem under this data condition in all neural radiance approaches, and to the best of our knowledge, none of the existing work has solved this issue.
In our opinion, semantic consistency is crucial for AD simulation, and synthesizing on deviated views is unavoidable for scalability.

AD simulation usually involves map data for planning and control, which can be obtained from a prebuilt High-Definition Map (HD Map) or an online mapping module. While the map data may not be pixel-perfect, it can provide semantic-level information that is useful for enhancing the semantic consistency of the trained neural radiance field.
In this paper, we propose incorporating map priors into neural radiance fields to enhance the semantic consistency and rendering quality of deviated driving view synthesis. Firstly, we employ ground information from maps to supervise the density field of NeRF, providing a more reliable road base for semantic entities. Next, we propose sampling rays to simulate unseen views. Unlike most NeRF augmentation methods~\cite{zhang2022ray, chen2022geoaug}, we utilize ground and lane information in sampling computations to guide the radiance field. More importantly, we model the above two supervision methods as weak supervision by using an uncertainty parameter and propose an uncertainty tempering scheme to increase the uncertainty. This ensures that map priors only guide the training process rather than enforce it towards their absolute values. As a result, our proposed method not only improves the rendering quality of interpolated novel view synthesis quantitatively but also enhances the semantic consistency of deviated novel view synthesis. 
Our contributions can be summarized as follows:
% We summarize the contributions of this paper as follows.



% To overcome the limitations of the collected data, this paper proposes a novel approach that leverages map information to enhance the semantic consistency of the synthesized driving views. 

% Autonomous driving (AD) vehicles are being trained with the help of deep learning networks and have shown promising achievements. This technology is considered to be a breakthrough that could change the way of transportation in the near future. However, there are many discussions on how to verify or judge the safety of autonomous driving systems.
% A straightforward solution towards the safety tests is to exhaustively enumerate real scenarios for validation as many as possible. However, the process of implementing different real scenarios is not only labor-intensive and costly, but also dangerous. Simulation has been proved to be an alternative, which is robust, safe, efficient in training, and evaluating AD software and algorithms.
% Now, the emerging technology of neural radiance field (NeRF)~\cite{} leverages multi-view images to construct a 3D scene and enable novel view synthesis for many indoor and outdoor applications. For AD simulation, there are two choices for constructing NeRF models: 1) collect a large amount of data, such as LiDAR and camera data, similar to mapping, to construct a fine-grained scene offline; or 2) directly use the log file (typically in the format of ROS bag) to rapidly create an environment and then dynamically simulate the driving scenarios.
% The first choice can achieve high-quality simulation, but it is time-consuming and expensive, making it difficult to generalize to very large scales. On the other hand, the second option is fast but can lead to low-quality simulation due to the data being sparse and similar to each other in log data. This paper tackles the problem raised by choosing the latter option and attempts to improve the quality of out-of-trajectory driving view synthesis by incorporating map information. This approach is practical for many autonomous driving tests.
% In conclusion, the use of NeRF technology for AD simulation is a promising avenue for training and evaluating AD software and algorithms. While both options for constructing NeRF models have their pros and cons, this paper addresses the challenges of the second option and proposes a potential solution to improve the quality of simulation.

%There exist a few attempts to facilitate training a NeRF model for synthesizing out-of-trajectory (or called as extrapo trajectory) views.


\begin{itemize}
    \item We propose a novel method to incorporate commonly used map priors in AD scenes into neural radiance fields to improve the out-of-trajectory driving view synthesis.
    \item We explicitly model the uncertainty in map priors as a parameter and propose an uncertainty tempering scheme to guide the training process of the neural radiance field.
    \item Experiments demonstrated that the proposed method can improve the semantic consistency of out-of-trajectory views and the rendering quality of novel view trajectory interpolation.
\end{itemize}

Our proposed method is easy to implement, can be easily plugged into existing NeRF algorithms, and has the capability of extending to other formats of priors.
%
\label{sec:RelatedWork}

\paragraph{Estimating a scoring function}
The scoring function $f$ in Section~\ref{sec:ProblemFormulation} con be approximated using past screening data to obtain $\hat{f}$. In terms of the initial screening order, $\hat{f}$ represents the algorithmic screener. Two main approaches exist for obtaining $\hat{f}$: score-based ranking (SBR) \cite{Zehlike2023_FairRanking_P1} and learning-to-rank (LTR) \cite{Zehlike2023_FairRanking_P2}, where in the first a function is given to calculate the scores while in the second the function is learnt. The initial screening problem, we argue, requires elements of both approaches. 
What is clear to us from the interaction with the company G (Section~\ref{sec:Generali}), is that the human-like screener acts within the SBR framework (as underlined by $U^{*}$ in Section~\ref{sec:ProblemFormulation}). If it were (humanly) possible, such screener would go over all candidates in the candidate pool $\mathcal{C}$. The issue for obtaining $\hat{f}$ is not only that $f$ remains a mental process, but so do the scores $Y$: the \textit{selected}-$k$ is recorded, but never their individual scores.
In that regard, the LTR framework is appealing. Based on past candidate screenings and using LTR methods like RankNet~\cite{Burges2010ranknet} and ListNet~\cite{Cao2007learning}, we could learn $\hat{f}$ based on the characteristics of the candidates within and outside past \textit{selected}-$k$ sets. These methods are often used for providing suggestions to the user. The issue for obtaining $\hat{f}$, though, now becomes the lack of order within \textit{selected}-$k$. There is a difference in training a ranking algorithm that recommends the best item for the user over one that recommends $k$ items all interchangeable for the user (as underlined by $\underline{U}^{k}$ in Section~\ref{sec:ProblemFormulation}). 

We argue that the initial screening order problem, opens new formulations for both SBR and LTR (the fatigued scored \eqref{eq:FatiguedScores} in Section~\ref{sec:ISO:FatiguedScreener}, e.g., hints at this line of work). Estimating $\hat{f}$ to aid HR officers is an ultimate goal, but we need $\hat{f}$ to capture the role of fatigue, the influence of the initial order (and the fact that HR officers can alter it), and the objective of not always selecting the best possible candidates. With the initial screening order problem, we have highlighted a complex problem that, to the best of our knowledge, has not been tackled explicitly by the ranking literature.

\paragraph{Modeling position bias}
Position bias states that there is a premium for being, usually, at the top of a platform like a list or a website. Human are predisposed to favor those items, leading to biased decisions \cite{DBLP:journals/cacm/Baeza-Yates18}. Position bias falls under technical bias in the fair ranking literature \cite{DBLP:journals/vldb/PitouraSK22,Zehlike2023_FairRanking_P1,Zehlike2023_FairRanking_P2}. 
This behaviour was formalized (e.g., \cite{CraswellZTR08_ExperimentsClickPositionBias}) and tested (e.g., \cite{DBLP:journals/jcmc/PanHJLGG07, DBLP:conf/clef/GrotovCMSXR15, DBLP:conf/www/RichardsonDR07}) early on by the click model literature. The name choice of the two search procedures in Section~\ref{sec:ISO:SearchAlgorithms} is a reference to this line of work. Position bias, probably given its link to click models and current recommender system problems, is addressed within the LTR framework \cite{Zehlike2023_FairRanking_P2}. 

Fairness-wise, two lines of work---probability-based fairness (PBF) and exposure-based fairness (EBF)---tackle position bias. In PBF \cite{DBLP:conf/ssdbm/YangS17, DBLP:conf/cikm/ZehlikeB0HMB17}, we constraint the ranking algorithm to (re-)arrange the candidates under some fair distribution like a tossing a fair coin. In EBF \cite{DBLP:journals/cacm/Baeza-Yates18, DBLP:journals/sigir/JoachimsGPHG17}, we instead model attention, which decreases geometrically or logarithmically as the user faces a longer list of items, and constraint the ranking algorithm to (re-)arrange the candidates such that candidates receive similar exposure (e.g., \cite{DBLP:conf/kdd/SinghJ18}). The initial screening order, with its focus on fatigue, relates to the EBF works. We interpret fatigue as inversely proportional to attention: the more tired the screener, the less attention it gives to a candidate. Such link allows to use EBF to tackle our problem.

The initial screening problem, however, challenges current works on position bias. It is important to separate the bias itself from its effect on deriving a fair ranking. Position bias seems to be inherent to humans. EBF handles this by pre-emptily accounting for it and returning rankings that, e.g., are fair in exposure. In EBF problems we are returning rankings, meaning the user is expecting to go over a meaningful list of ordered items (like a Google search result or a recommendation of candidates to hire). This is not, however, the case for the initial screening order problem where the HR officer plans to make its own assessment of the list of candidates and, thus, attaches no meaning to it. That is way we make the distinction between the candidate pool $\mathcal{C}$ and the initial order $\theta$. EBF methods could be useful for providing a $\theta$ that accounts for fatigue, but it should do so without providing it in the form of recommendations. It is unclear from our experience with company G (Section~\ref{sec:Generali}) whether obtaining $\hat{f}$ is possible under such complex process; hence, it is unclear to us how the solutions proposed under EBF or even PBF can address the position bias when the goal is to return a fair initial order (Def.~\ref{def:FairIO}) instead of a fair ranking of candidates. 

We highlight recent work by \citet{DBLP:conf/chi/EchterhoffYM22} that focuses in capturing and balancing anchoring bias \cite{Kahneman2011Thinking} in sequential decision-making. Such bias occurs, e.g., when a given candidate is evaluated following, say, two very bad versus two very good previous candidates: that same candidate has a higher chance of a positive decision in the former scenario. This is because the screener anchors its expectations on a lower reference point despite each candidate being independent from each other. These researchers worked closely with a university to understand and model their admission process and proposed an algorithmic procedure to balance the anchoring bias. In that sense, our work also prioritizes the role of the user in the problem formulation and, like \cite{DBLP:conf/chi/EchterhoffYM22}, links future work between LTR, EBF, and theories on human decision making (see \cite[Ch10]{DBLP:books/daglib/0033056} and, e.g., \cite{DBLP:conf/chi/CarabanKGC19, DBLP:journals/isr/AdomaviciusBCZ13}).

\paragraph{Understanding candidate screening.}
With the exception of \cite{DBLP:conf/chi/EchterhoffYM22} and \cite{SukumarMH18_PeacanPie}, no other paper offers detailed insights on the process of candidate screening. Both of these papers focused on college admissions, with \cite{DBLP:conf/chi/EchterhoffYM22} proposing an algorithmic procedure for anchoring bias while \cite{SukumarMH18_PeacanPie} the use of visual tools to support the decision maker. We join these paper in stressing the complexity of candidate screening and share their view in using technology to aid the human decision maker. To the best of our knowledge, we offer the first formalization of hiring in candidate screening. 

%
% ESO
%

% \paragraph{Wed. 08/03.} In going over once again Meike's fair ranking survey, I came across the \textit{click model} literature. These ones were hidden under \textit{technical bias} and refer to practices that encourage bias. In particular, it refers to \textit{position bias}. Humans, for example, tend to read from top-to-bottom and, thus, associate items positioned at the top of a list as more important. Unsurprisingly, the ranking technologies reflect this bias: Google searches will put on top the most relevant search to your query; football leagues will order the classification board in terms of the leader, and so it goes. 

% The click model literature is interesting as it precedes fairness and the overall ML boom but still identifies the issue of bias. Based on user experiments (mostly done using eye-tracking technology), this literature assumes the position bias and tries to explain it via clicking models to address it. From ``In Google We Trust'', e.g., researchers found that humans will click on the top ranked item despite being the one with less relevance to the search. Both ``A Comparative Study of Click Models for Web Search'' and ``An Experimental Comparison of Click Position-Bias Models'', to name a few, propose/study several click models to see which one explains human behaviour better. The consensus seems to be: one, there is such a thing a position bias, two, no click model seems to outperform the other even when using several performance measures.

% Now, it is important to distinguish the setting from click models from the one studied in Generali. In the former, the user assumes that the search engine (or overall ranker) provides a ranking that is meaningful: i.e., the positions reflect the relevance of the item. In the latter, this is not the case: the HR platform offered many, non-meaningful ways to order the list of candidates but the HR screener was well aware that the first candidate on the list was not necessarily the better suited or more relevant one. The issue, of course, is whether this position bias can have unconscious effects. In that sense, the notion of fatigue, $\omega(t)$, is interesting as we just say the screener gets tired. The position bias arises because nobody wants to be reading a list of candidates all day... It is conceptually appealing to frame it as such. It also aligns with the notion of the \textit{decaying attention curve} used in click models where it is known that attention is scarce and it decays as the user goes over the ranked items. Hence, the position bias seems equivalent across these framings of the problem and, to an extent, unavoidable.

% Formally, say for candidate list $\mathcal{C}$ with $|\mathcal{C}|=n$ candidates, let $\mathcal{R}$ represent the set of all possible rankings of $\mathcal{C}$. In other words, it denotes all possible permutations. Clearly, $\mathcal{R}$ includes (assuming no tied items/candidates) the optimal initial ordering $r_{io}^*$, meaning the \textit{meaningful ordering for the screener} such that $r_{io}^*[1]$ represents the best suited candidates for the job or, in general, the most relevant item. $\mathcal{R}$ includes a given $r_{io}$ \textit{initial order ranking} that is not meaningful to the screener (though, it is possible for $r_{io} = r_{io}^*$ despite the screener being unable to know that: i.e., it does not matter to the problem setting. It is also very unlikely, with probability: $1/n!$.

% Therefore, in terms of related work, we need to \textit{(i)} consider the potential unconsciousness/unconscious ways the position bias materializes despite knowing that $r_{io}$ is not meaningful and \textit{(ii)} whether the human search algorithms (or models of behavior) relate to existing models like the clicking ones?

% Regarding \textit{(ii)}, I liked the models presented in ``An Experimental Comparison of Click Position-Bias Models''. It presents four models, though the mixed model would only apply if we allowed the screener to read the same $r_{io}$ more than once and also to be inconsistent in his/her search strategy (future work maybe?). We consider the following and link them to the IO problem:
% %
% \begin{itemize}
%     \item Baseline Model: there is no position bias or, equivalently, the screener never gets tired: $\forall t \omega(t)=0$. 
%     \item Examination Model: similar to the ExhaustiveSearch.
%     \item Cascade Model: similar to the LazySearch.
% \end{itemize}
% %
% where we must consider that these models are based on ``clicking'' once and stopping while we consider the case where we ``click'' $k$ times to reach \textit{select-k}. For instance, in the \textit{examination model}, a user looks over the ranking almost exhaustively and then decides where to click. Also, we must consider that these models are based on \textit{the user assuming that the initial order of the ranking is meaningful}. We, thus, should refer to them and, if we choose to use the same terminology, expand them according to our setting.

% Regarding \textit{(i)}, it seems it is a question on all possible $r_{io}$ rankings and, overall, a question \textit{sequential decision-making}: how can bias manifest itself when looking at many items sequentially with or without a meaningful order? In that sense, we need consider, one, these biases and, two, if there are studies on ExhaustiveSearch and LazySearch strategies.

% There are some interesting works in the FindHR folder, though I cannot find ones that relate directly to the setting studies here. We have a could on HR hiring and how AI could help/worsen fairness/bias. We also have one (``Sifting and Sorting'') that examines the hiring practices in a bank (though in 1997) and how personal contacts influence the sequential decisions. None of these seem to consider fatigue as a parameter (the one on HR tools to reduce bias do so indirectly). In that sense, the literature on judgement heuristics / nudging is closer to us in the sense that ``we known there is a bias due to mental processes and want th platform or ADM to help reduce this brisk of bias''. Still, in this line of work, I do not seem to find explicit formalizations on search strategies (as, e.g., the click model literature does for the WWW users). Maybe we have a found a bridge here?

% Similarly, we must be honest about what is implicit in our translation of the human screener: some degree of optimal decision making on the basis of rational thinking. At a minimum, we assume an agent that wants to reach a goal and minimize the task duration to some degree: time is limited; otherwise, the screener could spend as much as needed to build the \textit{select-k} set without getting tired.

% Along the lines of decision-making models, some of interest: \textit{the effort accuracy framework}, and \textit{the preference construction framework}. In the former, the agent knows what it wants and will try to find it while keeping in mind the accuracy-effort trade-off, meaning that the better suited candidate is not the optimal one if it lies at the bottom of the list... In the latter, instead, the agent does not know what it wants explicitly and forms its preferences as it starts searching, meaning the way the list is presented influences the final decisions. It seems that both frameworks can affect our screener in the Lazy and Exhaustive searches, meaning that it will depend on what we want to assume. Under the first framework, we are in a more algorithmic setting: try to minimize the risk that the desired candidates require a disproportionally high search effort. While under the second framework, we in a more cognitive setting: make sure that the platform or list nudges/helps the agent to make the right decision buy building the best preferences. For the Generali setting and ranking problem formulation, the effort accuracy framework comes more natural. The preference construction framework seems more natural for human-computer interaction works. Still, both frameworks are Related Work, no? Consider the Chapter 10 from the FindHR folder.

% Under preference construction, relevant to us are: \textit{primacy/recency effects}; \textit{priming}; \textit{defaults}... though I feel like, one, this is not much relative to the 1970s Judgment Heuristics, and, two, the nudging literature seems more robust than this. Also, the clicking models indirectly address some of these issues... its main default is that we assume one click, which may not translate to all decision-making scenarios. That said: we could frame the two searches in terms of \textit{personalities}. In particular, \textit{the maximizer} for the LazySearch and \textit{the satisficer} for the ExhaustiveSearch. Though, these are more psychology definitions... I have a preference for defining the agents in terms of fatigue or even some sort of \textit{memory parameter} to capture the effects of primacy/recency or priming or default. 

\section{Definitions}

\label{sec:Predefinitions}

In this section, we formally define the input and the output of \textsc{Beep}.

\textbf{Input: }
We define the input as a dataset $S$ of multivariate event sequences, where the number of sequences is denoted by $|S|$.
% We consider a dataset $S$ of $|S|$ multivariate event sequences as the input of our algorithm.
Each sequence $s_i \in S$ is a vector of $|s_i|$ events, denoted $s_i = (e_1, e_2, ..., e_{|s_i|})$.
We further define $||S||=\sum_i{|s_i|}$ as the number of events in the whole dataset.
For each event $e$, we consider a same set of categorical attributes $A=\{a_1, a_2, ..., a_{|A|}\}$.
Thus, an event $e$ can be considered a vector of values, denoted $e=(v_1, v_2, ..., v_{|A|})$, where for each $1 \leq k \leq |A|$, $v_k$ is an optional value of attribute $a_k$.
We define $V_k$ as a finite set of all optional values of attribute $a_k$.
Thus, the set $E$ of all distinct events has a maximum size of $|E|_{max} = \prod_{1\leq k\leq |A|}{|V_k|}$.
% The dataset $S$ is over multiple categorical attributes $A=\{a_1, a_2,\\ ..., a_{|A|}\}$.
% Finite set $V_k$ consists of all the optional values for the $k$-th attribute $a_k$.
% Furthermore, an event $e$ can be considered as a vector of values, denoted $e=(v_1, v_2, ..., v_{|A|})$, where the $k$-th value $v_k \in V_k$.
% We define $E$ as the set of all distinct events, where the size $|E|$ has a maximum value of $|E|_{max} = \prod_{1\leq k\leq |A|}{|V_k|}$.

\textbf{Output: }
The output of \textsc{Beep} is a set $P$ of $|P|$ multivariate patterns. To define a pattern we must first define two other pre-elements.
First, we define $e^a\preceq e^b$ to indicate that event $e^a$ preserves some attributes of event $e^b$ and drops others, i.e., $e^a$ is part of $e^b$ (e.g., in Figure \ref{fig:patterndef}, $e_1$ of $p_1$, which drops the first attribute, is part of $e_1$ of $s_1$).
Formally, assume that $I$ is the set that includes the indexes of all the attributes to be preserved (i.e., some integers between 1 and $|A|$). For each $1\leq k\leq |A|$, if $k \in I$, $e^a$ has the same $k$-th value as $e^b$.
Otherwise, the $k$-th value of $e^a$ is empty.
Second, we define $s^a \subseteq s^b$ to indicate that sequence $s^a=(e^a_1, ..., e^a_n)$ is a subsequence of sequence $s_b=(e^b_1, ..., e^b_m)$, if there exist integers $1\leq i_1\le i_2\le ...\le i_n\leq m$ such that $e^a_j\preceq e^b_{i_j}$ for each $1\leq j\leq n$.

Based on these two pre-definitions, we define a pattern $p_i \in P$ as a subsequence of numerous original sequences in the dataset $S$, where $|p_i|$ indicates the length and $||p_i||$ indicates the number of values.
For example, in Figure \ref{fig:patterndef}, there exist four sequences ($s_1$ to $s_4$) in a dataset over two attributes ($a_1$ and $a_2$).
The dataset can be summarized by two patterns ($p_1$ and $p_2$), where $|p_1|=|p_2|=3$, $||p_1||=4$, and $||p_2||=6$.
Sequence $s_1$ covered by pattern $p_1$ is the simplest case.
We further support three features as follows to obtain informative patterns in real-world dataset.

\input{figures/patternDefinition}

\begin{enumerate}[label={\bf F{{\arabic*}}}]
  \item \label{feature:gap} \textbf{Gap events.}
%   A gap event can separate two consecutive events in the pattern into non-consecutive events in the sequence, such as $e_2$ of $s_2$.
  A gap event can separate two consecutive events in the pattern into non-consecutive events in the sequence. For example, sequence $s_2$ is covered by pattern $p_2$, but the gap event $e_2$ is not captured by $p_2$.
  We allow gaps to ignore events that may be noise.

  \item \label{feature:miss} \textbf{Missing values.}
  A missing value occurs in the pattern but not in the sequence, e.g., $s_3$ misses the second value of $e_2$ when it is covered by $p_2$.
  In the strictest interpretation, pattern $p_2$ is not a subsequence of sequence $s_3$.
  However, we must consider the possibility (common in real-world data) that a missing value is actually noise.
  Value $z$ may be similar to value $y$ because they have similar contexts (e.g., in tennis, $drive$ and $topspin$ are both offensive techniques to hit hard and low), or it may be an anomaly (e.g., an error caused by an automatic data acquisition system).
  Ignoring the substitution of similar values can keep the pattern simple, while detecting the pattern despite anomalies can help to inspect and dig into the data.
  Note that we are the first work that introduces missing values into multivariate sequential pattern mining.

  \item \label{feature:interleaveing} \textbf{Interleaving patterns.}
  Two interleaving patterns overlap over a period of time\cite{tatti2012long}, e.g., pattern $p_1$ and $p_2$ are interleaving in sequence $s_4$.
  We allow interleaving patterns to discover simultaneous patterns.
\end{enumerate}

An informative pattern should be compact (i.e., the events in the pattern should occur within a short period of time in the original sequences) and authentic (i.e., there should exist as few missing values as possible).
Thus, we limit the maximum number of gap events in a pattern $p$ to $|p|-1$ and the maximum number of missing values in a pattern $p$ to $\lfloor||p|| \div 10 + 0.5\rfloor$.
Moreover, we allow only one missing value in an event.

\section{Basic Theory}

\label{sec:MDL}

In this section, we briefly introduce the Minimum Description Length (MDL) principle and how we apply it in sequence mining.

\subsection{MDL principle}

The MDL principle was originally theorized in the context of data compression \cite{grunwald2007minimum}.
In order to minimize storage space, MDL describes the original dataset $D$ with a model $M$, which usually suggests the regularity in the data.
The MDL principle works under the assumption that the optimal model results in the shortest description length.
More formally, an optimal model $M$ can minimize $L(M) + L(D|M)$, where $L(M)$ is the description length of model $M$, and $L(D|M)$ is the description length of the original dataset $D$ when $D$ is described by $M$.
An MDL-based algorithm usually aims to solve three problems; namely, how to design a model, how to encode the dataset using a model, and how to obtain the optimal model.

In the past two decades, MDL has been applied to sequential pattern mining \cite{siebes2006item,tatti2012long,bertens2016keeping,kawabata2018streamscope}, and there exists a robust framework to solve the three problems mentioned above.
In our paper, we follow this framework and extend it to meet our requirements for effectiveness and efficiency.
Our solutions to the three aforementioned problems are as follows.
We define our model in Section \ref{sec:ct}, introduce the process of encoding in Section \ref{sec:encoding}, and propose our algorithm for obtaining model in Section \ref{sec:F4M}.
% Our solutions to the three questions are as follows.
\subsection{Code Table}

\label{sec:ct}

In prior work, code tables (\textit{CT}), which provide an encoding scheme for every pattern, have been widely used for model design.
We follow the design of code tables and further extend it to support two advanced features: gap events (\ref{feature:gap}) and missing values (\ref{feature:miss}).
As shown in Fig. \ref{fig:ct}, each row of a code table is associated with a pattern $p_i$, which is recorded in the first column.
To preserve the correlations between multiple attributes, the pattern can be multivariate.
The four columns on the right record four types of codes.
The second column records a pattern code $code_p(p_i)$ that represents the first event of the pattern, which indicates the occurrence of the pattern.
The gap code $code_g(p_i)$ (the third column) and the fill code $code_f(p_i)$ (the fourth column) support the gap events feature.
The two codes are introduced and well evaluated in prior works \cite{tatti2012long, bertens2016keeping}.
A gap code represents a gap event in the sequence, and a fill code represents an event in the pattern to be filled in the sequence (with the exception of the first event represented by the pattern code).

We propose a new code (the fifth column), namely the miss code $code_m(p_i)$, to allow for the missing values feature \ref{feature:miss}.
A miss code is a number to indicate the index of the missing value.
After an event is encoded by a pattern code or a fill code, there may exist a miss code to show which attribute of the event is missing.

When we implement the code table, two more constraints arise.
First, following our limitations on gap events and missing values, for a pattern $p_i$, the number of the gap code must be smaller than $|p_i|$, the number of the fill code must be $|p_i| - 1$, and the number of the missing code must be smaller than $\lfloor||p_i|| \div 10 + 0.5\rfloor$.
Given these limitations, we find it useless to record the gap code and the fill code for a pattern $p$ that satisfies $|p| = 1$ and to record the miss code for a pattern $p$ that satisfies $||p|| < 5$.
Thus, we delete these useless codes to shorten the description length.

Second, to ensure that a sequence can be completely covered by the model, we further extend $CT$ with all singleton values, which are regarded as patterns with only one value.
The extended rows are regarded as the standard code table ($ST$) because it is the minimum $CT$.
Obviously, the patterns in $ST$ cannot deliver insight into the sequences.
To distinguish the informative patterns with more than one value from these singleton patterns, we define $CT^*$ as the table of all the patterns in $CT$ but not in $ST$.
The output of our algorithm is all the patterns in $CT^*$.

\input{figures/encoding}
\subsection{Dataset Encoding}

\label{sec:encoding}

To encode the dataset with the code table, our algorithm takes three steps: covering, encoding, and calculating the encoded length.

\subsubsection{Covering}

Covering is the process of using patterns to describe a sequence.
A sequence $s$ is first separated into several patterns in code table $CT$ without overlapped values. The result is called a cover $C=cover(s~|~CT)$.
In the formal manner, given a sequence $s$ and a code table $CT$, a cover $C$ records which patterns are used and the location of each value in the pattern.
For example, sequence $s$ in Fig. \ref{fig:ct} is covered by pattern $p_1$, pattern $p_2$ and singleton value $y$.
For pattern $p_1$, cover $C$ records that it covers the first three events, and that there exists one miss value at the second attribute of $e_2$, and no gap events.
For pattern $p_2$, cover $C$ records that it covers the events $e_2$, $e_4$, and $e_5$, and that there exists one gap event and no miss value.
For singleton value $y$, cover $C$ simply records its position.

Note that for a given code table, there may exist many ways of varying utility to cover a sequence.
The simplest way is to use only singleton values to cover the sequence.
However, this is also the least useful way, because it cannot compress any information or extract any patterns.
In section \ref{sec:coverAlgorithm}, we explain how \textsc{Beep} finds an optimal way to cover a sequence with our code table design.

\subsubsection{Encoding}

Encoding is the process of using codes in the code table to describe a sequence, rather than patterns.
Given a multivariate sequence $s$, a code table $CT$, and a cover $C$, we encode the sequence into a code stream $cs$ through four steps.
(1) Scan the sequence $s$ left-to-right and top-to-bottom to find the first value that is not encoded, and look up the pattern $p_i$ to which the value belongs in the cover $C$.
(2) Traverse the sequence $s$ from the first event of $p_i$ to the last event of $p_i$.
During the traversal, the first event of $p_i$ is encoded as the pattern code $code_p(p_i)$, the gap events are encoded as gap codes, and the other events are encoded as fill codes.
All these codes are sorted in the order they are traversed.
(3) Find all the miss values in cover $C$.
For each miss value, assuming that it belongs to event $e$, we insert the miss code after the pattern code or the fill code that encodes $e$.
(4) Concatenate the code stream of the pattern to the entire code stream and mark all the values in the pattern as encoded.
We repeat these four steps until all the values in the sequence are encoded.

For example, Fig. \ref{fig:ct} demonstrates how a sequence $s$ is encoded as a code stream $CS$ given a cover $C$ and a code table $CT$.
We encode $s$ with three patterns, namely, $p_1$, $p_2$, and singleton $y$, in the order of traversal.
We first encode $p_1$ with no gap events and a missing value at $a_2$ of $e_2$.
Then, we encode $p_2$ with one gap event and no missing values.
Finally, we encode the singleton value $y$.


\subsubsection{Calculating the Encoded Length}

After encoding, we need to calculate the encoded length of the dataset to evaluate the model.
We consider the encoded length in two parts, namely the encoded length of the code table $L(CT~|~C)$ and the encoded length of the dataset $L(S~|~CT)$.
The optimization goal is to minimize the total encoded length $L(CT~|~C)+L(S~|~CT)$.

As the basis, we consider the length of each code.
Given that each code should be unique in order to avoid ambiguity during decoding, we use prefix codes \cite{cover1999elements} to optimize the length of distinct codes.
Given the set of all code streams $CS=\{cs_i~for~each~sequence~s_i\}$, the length of each type of code can be computed by Shannon entropy \cite{shannon1948mathematical}.
For pattern $p$, the length of the pattern code $code_p(p)$ is the negative log-likelihood
\begin{displaymath}
    L\left(code_p\left(p\right)\right)=-lg2\left(\frac{usage\left(p\right)}{\sum_{p_i \in CT}usage\left(p_i\right)}\right),
\end{displaymath}
where $lg2(k)$ means the logarithm of $k$ to the base $2$, and $usage(p)$ is the number of pattern codes of pattern $p$ in $CS$.
Similarly, we can give the calculation formulas for gap code $code_g(p)$, fill code $code_f(p)$, and miss code $code_m(p)$ as follows.
\begin{displaymath}
    \begin{aligned}
        L\left(code_g\left(p\right)\right)=&-lg2\left(\frac{gaps\left(p\right)}{gaps\left(p\right)+fills\left(p\right)+misses\left(p\right)}\right), \\
        L\left(code_f\left(p\right)\right)=&-lg2\left(\frac{fills\left(p\right)}{gaps\left(p\right)+fills\left(p\right)+misses\left(p\right)}\right), \\
        L\left(code_m\left(p\right)\right)=&-lg2\left(\frac{misses\left(p\right)}{gaps\left(p\right)+fills\left(p\right)+misses\left(p\right)}\right) \\ &+ L_N\left(~|~A~|~\right),
    \end{aligned}
\end{displaymath}
where $gaps(p)$, $fills(p)$, and $misses(p)$ are the number of gap codes, fill codes, and miss codes, respectively, of pattern $p$ in $CS$.
Miss code $code_m(p)$ contains additional bits for the index of missing attributes, which is denoted by $L_N(|A|)$.
Function $L_N(k)$ represents the number of bits required to encode integer $k$, where the MDL optimal Universal code for integers is considered \cite{grunwald2007minimum}.

\textbf{The encoded length of the code table.} To obtain a minimum description length, we treat patterns in $ST$ and patterns in $CT^*$ differently, where $L(CT~|~C)=L(ST)+L(CT^*)$.

For $ST$, we consider each attribute separately and encode the number of optional values and their supports:
\begin{displaymath}
    L\left(ST\right)=\sum_{1\leq k \leq ~|~A~|~}\left(L_N\left(~|~V_k~|~\right)
    + log\left(
        \begin{array}{c}
                ~|~S^k~|~ \\
                ~|~V_k~|~
            \end{array}
        \right)
    \right),
\end{displaymath}
where $S^k$ is a univariate dataset that preserves the $k$-th attribute of each event in dataset $S$.

For $CT^*$, we encode the number of patterns, the sum of their usages, the distribution of their usages over different patterns, and the original patterns:
\begin{displaymath}
    \begin{aligned}
        L\left(CT^*\right)= & L_N\left(~|~P^*~|~\right)+L_N\left(usage\left(P^*\right)\right)            \\
                            & +log\left(\begin{array}{c}
                ~|~usage\left(P^*\right)~|~ \\
                ~|~P^*~|~
            \end{array}\right)+\sum_{p_i\in P^*}L\left(p_i\right),
    \end{aligned}
\end{displaymath}
where $P^*$ is the set of all patterns in $CT^*$.
Considering that $|~P^*~|$ and $usage(P^*)$ can be zero, we define $L_N(0)=0$.
For a non-singleton pattern $p_i$, we encode the number of events, the number of values, the number of gaps, the number of misses, and the first column (i.e., each value in the pattern) as
\begin{displaymath}
    \begin{aligned}
        L\left(p_i\right) = & L_N\left(~|~p_i~|~\right)+L_N\left(~||~p_i~||~\right)                            \\
                            & +L_N\left(gaps\left(p_i\right)+1\right)+L_N\left(misses\left(p_i\right)+1\right) \\
                            & +\sum_{v\in p_i}L\left(code_p\left(v~|~ST\right)\right),
    \end{aligned}
\end{displaymath}
where $L(code_p(v ~|~ ST))$ represents the encoded length of a value $v$ in pattern $p_i$.
Here, we directly use the pattern code of singleton value $v$ in $ST$.

\textbf{The encoded length of the whole dataset given a code table.} We encode the number of sequences $|~S~|$, the length $|~s_i~|$ of each sequence $s_i$, the number of attributes $|~A~|$, and the codes in $CS$.
Thus, we obtain the equation
\begin{displaymath}
    \begin{aligned}
        L\left(S~|~CT\right)= & L_N\left(~|~S~|~\right) + \sum_{s_i \in S}L_N\left(~|~s_i~|~\right) + L_N\left(~|~A~|~\right) + L\left(CS\right),
    \end{aligned}
\end{displaymath}
where $L(CS)$ is the description length for $CS$ -- the sum of the description length of four types of codes:
\begin{displaymath}
    \begin{aligned}
        L\left(CS\right)= & \sum_{p_i \in CT}usage\left(p_i\right)L\left(code_p\left(p_i\right)\right)     \\
                          & + \sum_{p_i \in CT}gaps\left(p_i\right)L\left(code_g\left(p_i\right)\right)    \\
                          & + \sum_{p_i \in CT}fills\left(p_i\right)L\left(code_f\left(p_i\right)\right)   \\
                          & + \sum_{p_i \in CT}misses\left(p_i\right)L\left(code_m\left(p_i\right)\right).
    \end{aligned}
\end{displaymath}
\subsection{Problem Definition}

Before introducing our algorithm for obtaining the optimal model, we formally define our problem as follows.

\textbf{Optimal Pattern Set Problem.} \textit{Suppose that $A$ is a set of categorical attributes with finite values, and $S$ is a dataset of multivariate event sequences over these $|A|$ attributes. Find a set of multivariate patterns $P$ so that a cover $C$ of $S$ using $P$ can be found to obtain the minimum description length L(CT~|~C) + L(S~|~CT).}

For a problem like this, there exists a large search space, which leads to high runtime complexity.
First, because any subsequences of a sequence $s_i \in S$ can be a pattern, and thousands of sequences may exist, the set of patterns can be innumerable.
Second, given a set of patterns, we can cover $S$ in different ways.
We need to search for both the optimal set of patterns and the optimal way to cover the dataset.
When we introduce missing values, the search space becomes even larger. A pattern may not be a subsequence of any sequence in the dataset. And when using a pattern to cover a sequence, some values in the pattern may not necessarily appear in the sequence. In the next section, we explain how \textsc{Beep} can search this large space efficiently.

\section{Algorithms}

\label{sec:F4M}

In this section, we introduce our algorithm, \textbf{BEEP}, to tackle the two challenges of covering the dataset and summarizing the optimal model.
The covering algorithm (Section \ref{sec:coverAlgorithm}) accepts the model summarized by the summarizing algorithm (Section \ref{sec:summarizing}) as input.
The summarizing algorithm optimizes the model iteratively according to the cover provided by the covering algorithm.

\subsection{Covering the Dataset}

\label{sec:coverAlgorithm}

Given a set of patterns $CT$ and a dataset $S$, the covering algorithm targets the optimal cover $C$ for each sequence $s_i \in S$ with the patterns in $CT$.
Considering that each sequence is independent, we cover the sequences separately.
Algorithm \ref{alg:cover} shows the process of obtaining the optimal cover $C$ given a sequence $s$ and patterns in $CT$.
First, we initialize three variables with an empty set (line 1), where $C$ is the output of the algorithm, $marks$ records which pattern each value in $s$ belongs to, and $misses$ records the position of all the miss values.
Note that $marks$ does not record any miss values, and $misses$ may record a value that is missed by two patterns.

The core idea is to traverse each pattern $p$ in $CT^*$ (line 2) and try to cover $s$ by $p$ (line 3\textasciitilde7) until all the values in $s$ are marked as covered by a pattern (line 8\textasciitilde9).
If there exist values not marked (line 10\textasciitilde12), they will be marked as covered by the corresponding singleton patterns in $ST$ (line 13\textasciitilde15).
To ensure that the final cover $C$ is optimal, we follow the \textsc{Krimp} algorithm \cite{vreeken2011krimp} and employ a greedy algorithm.
We traverse the patterns in $CT^*$ in a fixed order (Cover Order): $\downarrow ||~p~||$, $\downarrow support(p~|~S)$, and $\uparrow$ lexicographically.
This order ensures that the patterns that have more values and higher frequency, which are more meaningful, will be used first.
% To ensure that the final cover $C$ is optimal, we follow the \textsc{Krimp} algorithm \cite{vreeken2011krimp} and employ a greedy algorithm that traverses the patterns in $CT^*$ in a fixed order.
% We first consider patterns with more values, which are obviously more informative.
% We then consider patterns with higher frequency because higher frequency means more regularity.
% Finally, a lower likelihood of the cover are expected.
% Formally, the \textbf{Cover Order} is: $\downarrow ||~p~||$, $\downarrow support(p~|~S)$, and $\uparrow$ lexicographically.

Given a pattern $p$, a simple DFS-based algorithm (Appendix \ref{app:search}) searches pattern $p$ in sequence $s$ and finds every occurrence (line 3), where the limitations on gap events and missing values are considered.
For each time pattern where $p$ occurs, the searching algorithm returns the position of each value and the missing values, namely $marks_p$ and $misses_p$ respectively.
If all the positions in $marks_p$ are not marked, the algorithm uses pattern $p$ to cover these positions and updates the three variables.

\begin{algorithm}[t]
    \caption{Covering Algorithm}
    \label{alg:cover}
    \KwIn{A sequence $s$, a code table $CT$}
    \KwOut{An optimal cover $C$}
    \tcc{the explanation can be found in Sec. \ref{sec:coverAlgorithm}}
    $C \leftarrow \emptyset$, $marks \leftarrow \emptyset$, $misses \leftarrow \emptyset$\;
    \For{{\bf each} $p\in CT^*$ in Cover Order} {
        \For{{\bf each} $(marks\_p, misses\_p) \in search(p, s)$}{
            \If{$marks \cap marks\_p = \emptyset$}{
                $C \leftarrow C \cup (marks\_p, misses\_p)$\;
                $marks \leftarrow marks \cup marks\_p$\;
                $misses \leftarrow misses \cup misses\_p$\;
            }
        }
        \If{$|marks| = |s| \times |A|$}{
            {\bf break}\;
        }
    }
    \For{{\bf each} event $e \in s$} {
        \For{{\bf each} value $v \in e$} {
            \If{$v$ is not marked by $marks$} {
                $mark$ = \{cover $v$ by singleton pattern $v$ in $ST$\}\;
                $C \leftarrow C \cup (mark, \emptyset)$\;
                $marks \leftarrow marks \cup mark$\;
            }
        }
    }
    \Return{$C$}
\end{algorithm}
\subsection{Summarizing the Optimal Model}

\label{sec:summarizing}

Given a dataset $S$ and all singleton values in $ST$, the summarizing algorithm aims to find the optimal code table $CT$.
However, due to the large search space, finding the optimal $CT$ is time-consuming.
Inspired by \textsc{DITTO} \cite{bertens2016keeping}, we propose a heuristic algorithm to efficiently find an approximate optimal code table $CT$.
Algorithm \ref{alg:summarizing} demonstrates the algorithm overview.

The algorithm first initializes $CT$ with $ST$ (line 1).
Then a set of candidate patterns $Cand$ is generated (line 2) based on $S$ and $CT$.
The core idea of the algorithm is to examine whether each pattern $p$ in $Cand$ can bring information compression (line 3\textasciitilde4) and update $CT$ with $p$ that can do so (line 5\textasciitilde7).
To ensure priority access to a pattern that can bring higher information compression, the algorithm traverses the patterns $p$ in $Cand$ in the \textbf{Candidate Order}: $\downarrow support(p~|~S)$, $\downarrow |p|$, and $\uparrow$ lexicographically.

After confirming the ability of pattern $p$ to compress description length, the algorithm updates $CT$ with $p$ with three steps (line 5\textasciitilde7).
(1) The code table $CT$ is pruned to remove patterns that have become redundant with the addition of the new pattern $p$.
(2) The algorithm generates several variations of pattern $p$ to extend the code table efficiently.
(3) The algorithm updates the set of candidates for $CT$ based on the new code table $CT$.

We introduce three main functions below: the function for generating candidate patterns (Sec. \ref{sec:candidates}); the function for pruning (Sec. \ref{sec:pruning}); and the function for variations (Sec. \ref{sec:variations}).
We further introduce our method for speeding up the algorithm and the motivation for employing it (Sec. \ref{sec:lsh})

\input{algorithms/summarizing.tex}

\subsubsection{Generating Candidates}

\label{sec:candidates}

\input{figures/candidate}

A candidate pattern can be generated by combining two patterns that exist in $CT$, allowing us to construct complex patterns with simple ones.
Two patterns can be combined at different alignments.
To give the simplest example of univariate patterns, pattern $\{a, b\}$ and pattern $\{c, d\}$ can construct four candidates, namely, $\{a, b, c, d\}$, $\{a, c, b, d\}$, $\{c, a, d, b\}$, and $\{c, d, a, b\}$.
The combinations $\{a, c, d, b\}$ and $\{c, a, b, d\}$ are invalid because the number of gaps in $a, b$ and $c, d$ should be at most 1.
When constructing multivariate patterns, more candidates may arise, because the patterns may be related to different attributes such that the two patterns can be overlapped.
Furthermore, we need to consider the feature of missing values.
For example, in Fig. \ref{fig:candidate}, we list all the candidate patterns ($cp_1$ to $cp_6$) generated based on multivariate pattern $p_1$ and $p_2$ over three attributes.
When aligning the two events $e_2$ of $p_1$ and $p_2$, a conflict exists on value $x$ and $y$.
We resolve this conflict by regarding it as a missing value of one of the original patterns so that there exist two candidate patterns $cp_4$ and $cp_5$.
However, considering the limitation on the number of missing values, we regard these two candidates as invalid in practice.
We use $construct(p_i, p_j)$ to represent the set of all valid candidate patterns generated based on patterns $p_i$ and $p_j$.
We traverse each pair of patterns in $CT$ and construct candidate patterns.

% \input{algorithms/candidate}

\subsubsection{Pruning Patterns}

\label{sec:pruning}

If we add a new pattern $p^*$ into $CT$, some patterns may become redundant.
For example, if pattern $p_1={a}$ is always followed by pattern $p_2={b}$, all sequences containing $p_1$ can be covered by a new pattern $p^*={a,~b}$.
To keep the code table simple, we remove these redundant patterns.
Algorithm \ref{alg:pruning} demonstrates the process.
We first find all the patterns that have a lower usage after adding $p*$ (line 1\textasciitilde2).
Note that even if a singleton pattern has a lower support, we preserve it in $CT$ so that $CT$ always contains all singleton patterns.
Then, we traverse all the patterns to be pruned in \textbf{Prune Order}, where a pattern $p$ with a higher usage change $\Delta support(p)=usage(p~|~CT) - usage(p~|~CT_p)$ will be considered first.
If removing pattern $p$ can bring a smaller description length, we delete it from $CT$ (line 4\textasciitilde5).

\input{algorithms/pruning}

\subsubsection{Variations}

\label{sec:variations}

Constructing a large pattern is time-consuming, because we usually need to combine two patterns multiple times and examine all possible combinations.
To tackle this challenge, we follow the extremely effective solution used by \textsc{Ditto}: \textit{variations}.
The core idea of this algorithm is to quickly extend a pattern $p$ with the gap events that occur when using $p$ to cover the dataset.
More details can be found in Sec. 4.4 of \cite{bertens2016keeping}.

% \begin{algorithm}[t]
%     \caption{Function for Variations}
%     \label{alg:pruning}
%     \KwIn{The dataset $S$, a pattern $p^*$ and a code table $CT$}
%     \KwOut{A code table $CT$ with variations of $p$}
%     \tcc{the explanation can be found in Sec. \ref{sec:variations}}
%     $Cand \leftarrow p^* \times gapEvents(p^*)$\;
%     \For{{\bf each} $p \in Cand$}{
%         \If{$L(S,~CT \cup \{p\}) < L(S,~CT)$}{
%             $CT \leftarrow prune(S,~p,~CT) \cup \{p\}$\;
%             $CT \leftarrow variations(S,~p,~CT)$\;
%         }
%     }
%     \Return{$CT$}
% \end{algorithm}

\subsubsection{Speedup with Locality Sensitive Hashing (LSH)}

\label{sec:lsh}

The most time-consuming step was calculating the description length.
This is straightforward, as this was the only step that considered thousands of original sequences.
However, we found that, in line 4, Algorithm \ref{alg:summarizing}, a candidate $p$ is usually not an ideal pattern after comparing $L$, indicating that this step was wasting a lot of time examining useless candidates.
A faster algorithm would filter out those useless candidates before the description length step.

Since the candidate pattern is constructed from two patterns, when these two patterns often occur simultaneously or consecutively, the candidate pattern is more likely to be a real pattern.
Thus, we employ weighted \textit{Locality Sensitive Hashing} (LSH) \cite{ioffe2010improved} for efficiently determining whether two patterns can be combined.
If two sets of numbers have a weighted Jaccard similarity larger than a threshold $th$, the weighted LSH algorithm will generate the same hash value for them.

In our algorithm, for each pattern $p$, we record the positions where it occurs when covering the dataset.
A position is an index of a segment of original sequences, where a segment contains at most $l$ events (default $l$ is 20), and a long sequence will be cut into several segments.
If a pattern spans two segments, we record both positions.
Then we apply weighted LSH to examine whether the two patterns occur in similar sets of positions.
If the weighted LSH algorithm returns the same hash value, the candidate pattern constructed from these two patterns is indicated to be promising.

% 这里不太清楚是否需要加入一段,对应用MDL的内存消耗、时间消耗进行理论分析,还是只要最后进行实验就好了?
% Finally, we consider the runtime complexity and the memory requirements of LSH.
% Collecting the positions of a pattern in code table is not time-consuming.
% Before adding a pattern into code table, we have to computing the description length bases on the cover $C$ to examine whether the pattern brings information compression, where all the positions can be found $C$.

\section{Experiments}

\label{sec:Experiments}

We implemented \textsc{Beep} in C++ and have open-sourced the code for research purposes\footnote{https://github.com/BEEP-algorithm/BEEP-algorithm}.
A computer with a 2 GHz CPU and 16 GB of memory was used in all experiments.
We first worked with analysts to conduct an empirical study on a real-world dataset, collecting their feedback on the number of patterns, the effectiveness of missing values, and the correlations among multiple attributes.
We further conducted quantitative experiments on multi-scaled synthetic datasets to compare \textsc{Beep} with the state-of-art multivariate pattern mining algorithm to evaluate the performance of \textsc{Beep}.

\subsection{Empirical Study}

\subsubsection{Dataset}

To empirically evaluate the effectiveness of our algorithm, we conducted a case study on a real-world table tennis dataset.
Table tennis data is a typical example of multivariate event sequence data.
In table tennis singles matches, two players alternately hit the ball, beginning with one player's serve and ending with one player winning the game.
We considered each hit as an event and the consecutive hits as a sequence.
For each hit, we considered four attributes; namely, the technique the player used to hit the ball (\textit{Tech}), the area where the ball impacted the table (\textit{Ball}), the spin of the ball (\textit{Spin}), and the position of the player (\textit{Player}).
The optional values of these four attributes can be found in the Appendix \ref{app:values}.
We collected sequences from 10 matches (quarterfinals or later) played in 2019 by \textit{Ito Mima}, one of the top players in the world, against Chinese players.
We filtered sequences where \textit{Ito} served the ball in order to analyze her tactical patterns in this situation.
We were left with 716 sequences, of which the average length was 5.91.

\input{figures/caseTT}

\begin{table}
    \caption{The results of the case study on the table tennis data. We compared two algorithms, namely \textsc{Ditto} and \textsc{Beep}, on the number of patterns ($|P|$), the average frequency of each pattern ($avg.$ $freq.$), and the number of missing values in 716 sequences ($miss$, only for \textsc{Beep}), and the runtime ($t$).}
    \label{tab:empirical}
    \begin{tabular}{rrrrr}
        \toprule

        % $|S|$&\textit{avg.} $|s_i|$&$|A|$&$|E|$ &
        \textit{Algorithm} & $|P|$ & $avg.$ $freq.$ & $miss$ & $t$(s) \\

        \midrule

        \textsc{Ditto} & 161 & 4.36 & -- & 1747 \\

        \textsc{Beep} & 62 & 12.94 & 280 & 425 \\

        \bottomrule
    \end{tabular}
\end{table}

\subsubsection{Results analysis}

We ran both \textsc{Ditto} and \textsc{Beep} on this dataset and summarized the results (shown as Table \ref{tab:empirical}) as follows.

\textbf{\textsc{Beep} can summarize a smaller set of informative patterns than \textsc{Ditto}.}
\textsc{Beep} summarized 63 non-singleton patterns, while \textsc{Ditto} summarized 161.
According to the analysts we worked with, patterns with more than 4 values and used more than 10 times tend to be considered more worthy of analysis. Of these, \textsc{Beep} summarized 7 (used 530 times), while \textsc{Ditto} summarized 29 (used 546 times).
This was because many of the patterns that \textsc{Ditto} summarized as distinct differed in only one value, and \textsc{Beep} instead summarized these as one pattern with missing values.
As a result, \textsc{Beep} reduces the analysis burden of analysts.
Considering that the number of patterns is completely controlled by the algorithm, rather than manually input parameters, we believe that \textsc{Beep} is better than \textsc{Ditto} in summarizing a smaller set of informative patterns.

\textbf{\textsc{Beep} is more efficient than \textsc{Ditto} and performs similarly in information compression.}
We elaborate on these findings through quantitative experiments described in Section \ref{sec:quantitative}.

\subsubsection{Insights}

We discussed the summarized patterns with two analysts, each of whom has over three years of experience analyzing table tennis data for one of the best national teams in the world.
The patterns afforded the two analysts many insights into \textit{Ito}'s playing.
Due to space limitations, we'll focus here on one of the most common tactical patterns, shown as $p$ in Figure \ref{fig:tt}.

\begin{table*}
    \caption{The results of the quantitative experiments. Given 5 synthetic datasets, one for each row, we compared five algorithms, namely, \textsc{Ditto}, \textsc{Beep}, \textsc{Beep}-miss, \textsc{Beep}-LSH, and \textsc{Beep}-none. For each dataset, we reported the number of sequences ($|S|$), the length of each sequence ($|s|$), the number of attributes ($|A|$), and the number of optional values of each attribute ($|V_k|$). For each algorithm, we reported the number of patterns ($|P|$), the description length reduction compared to the singleton-only model ($\Delta L\%$), the number of missing values for \textsc{Beep}-miss and \textsc{Beep} ($miss$), and the runtime in seconds ($t$). (\tablet{orange}: the shortest time; \tabledL{blue}: the largest description length reduction.)}
    \label{tab:quantitative}
    \begin{tabular}{rrrr|rrr|rrrr|rrrr|rrr|rrr}
      \toprule

      \multicolumn{4}{l|}{Synthetic Datasets}&\multicolumn{3}{l|}{\textsc{Ditto}}&\multicolumn{4}{l|}{\textsc{Beep}}&\multicolumn{4}{l|}{\textsc{Beep}-miss}&\multicolumn{3}{l|}{\textsc{Beep}-LSH}&\multicolumn{3}{l}{\textsc{Beep}-none} \\

      \cmidrule(lr){1-4}\cmidrule(lr){5-7}\cmidrule(lr){8-11}\cmidrule(lr){12-15}\cmidrule(lr){16-18}\cmidrule(lr){19-21}

      $|S|$&  $|s_i|$&      $|A|$&  $|V_k|$&
      $|P|$&  $\Delta L\%$& $t$(s)&
      $|P|$&  $\Delta L\%$& $miss$& $t$(s)&
      $|P|$&  $\Delta L\%$& $miss$& $t$(s)&
      $|P|$&  $\Delta L\%$& $t$(s)&
      $|P|$&  $\Delta L\%$& $t$(s) \\

      \midrule

      50& 20& 5& 100&
      19&   24.5&     399&
      9&   \tabledL{27.2}&     33& 79&
      9&   \tabledL{27.2}&     33&455&
      16&   24.4&     \tablet{43}&
      19&   24.5&     380
      \\

      \textbf{70}& 20& 5& 100&
      12&   19.0&     987&
      14&   \tabledL{21.3}&     11& 215&
      14&   \tabledL{21.3}&     16&1238&
      12&   18.9&     \tablet{29}&
      12&   19.0&     994
      \\

      50& \textbf{30}& 5& 100&
      20&   25.8&     772&
      13&   26.6&     28& 402&
      14&   \tabledL{26.7}&     29& 1276&
      20&   25.9&     \tablet{120}&
      20&   25.8&     753
      \\

      50& 20& \textbf{7}& 100&
      15&   27.7&     1040&
      9&   29.1&     30& 251&
      9&   \tabledL{29.2}&     31& 1141&
      14&   27.7&     \tablet{43}&
      15&   27.7&     1028
      \\

      50& 20& 5& \textbf{200}&
      17&   24.1&     175&
      9&   \tabledL{25.3}&     39& \tablet{18}&
      10&   25.2&     39& 149&
      16&   24.1&     19&
      17&   24.1&     168
      \\

      \bottomrule
    \end{tabular}
  \end{table*}

\textbf{\textsc{Beep} can find multivariate patterns that reveal the correlations among multiple attributes.}
In pattern $p$, \textit{Ito} first positioned herself at the \textit{Backhand} area and served with the \textit{Pendulum} technique.
Next, \textit{Ito}'s opponent stood at \textit{Backhand} and received the ball as it hit the table at the \textit{Short Backhand} position, using the control technique \textit{Push}.
Finally, \textit{Ito} stood at the \textit{Forehand} position and received the ball at \textit{Half-long Forehand} hitting with the technique \textit{Attack} and spinning type \textit{Sink}.
This pattern showed that \textit{Ito} preferred to use the attack technique at the third hit when her opponent used a control technique at the second hit.
At the same time, \textit{Ito} stood where she could hit the ball with her forehand, which results in a faster hit.
Furthermore, the \textit{Sink} spin made the ball lose height, giving her opponent less of a chance to return it well.
The analysts concluded that an excellent tactical pattern, which led to a high winning rate, benefits from various details.

\textbf{Missing values contribute to data analysis.}
The analysts found two sequences, $s_1$ and $s_2$ in Figure \ref{fig:tt}, that contain this pattern.
Sequence $s_1$ has a missing value at the second hit, where \textit{Ito}'s opponent used the technique \textit{Touch Short}, a control technique similar to \textit{Push}.
The analysts believed that $p$ summarized $s_1$ well, despite the missing value.
Sequence $s_2$ has a missing value at the third hit, where \textit{Ito} received the ball at \textit{Half-long Backhand}.
However, given that \textit{Ito} stood at the \textit{Forehand} area, it didn't seem possible for her to receive this ball.
The analysts checked the data and found this point to be an anomaly.
If we had instead used the \textsc{Ditto} algorithm, these two sequences would have been encoded by other short patterns and thus unable to reveal these insights.

% \subsubsection{ECG data}

% % data description
% ECG data consists of real-valued time series detected by sensors\footnote{physionet.org/physiobank/database/stdb}.
% Here we demonstrate how \textsc{Beep} discovers multivariate patterns in time series data.
% Since \textsc{Beep} only accepts event sequence data as input, we first need to transform the time series data into event sequence data.
% In our case, we chose two time series as detected by two sensors to construct a long sequence with two attributes.
% In order to discover patterns that could reveal overall characteristics, we sampled the two time so that every fifty real values were summarized by their average value.
% Next, we discretised the real values into categorical attributes by SAX \cite{lin2007experiencing}.
% Finally, we obtained a sequence of 10740 events with two attributes.
% Each attribute had five optional values.

% % insights
% \fillblank{A paragraph to describe insights}

% % Figure environment removed

\subsection{Quantitative Experiments}

\label{sec:quantitative}

\subsubsection{Synthetic data}

We performed quantitative experiments on multi-scaled synthetic data shown in Table \ref{tab:quantitative}.
We generated 15 datasets in 5 levels, varying in terms of the number of sequences ($|S|$), the length of each sequence ($|s_i|$), the number of attributes ($|A|$), and the number of optional values of each attribute ($|V_k|$).
The data generation consisted of three steps.
First, we generated sequences randomly, ensuring that all the optional values had similar frequencies.
More precisely, for any two optional values of an attribute, the frequency of one value must be higher than 90\% of the frequency of the other value.
Second, we generated 5 multivariate patterns, each of which contained 5 values and a random length.
Third, we added these patterns to the dataset generated in the first step.
We ensured that each pattern covered 10\% of the events in the dataset.
At the same time, for each pattern, we randomly chose two occurrences and set one value as a missing value for each occurrence, resulting in 10 total missing values.

\subsubsection{Experiments Setup}

We compared \textsc{Beep} with the state-of-art MDL-based multivariate pattern mining algorithm, \textsc{Ditto}, the source code of which has been implemented in C++ and published for research purposes\footnote{http://eda.mmci.uni-saarland.de/ditto/}.
Given that \textsc{Beep} mainly introduces two improvements -- namely, the miss codes and the LSH-based acceleration -- we performed an ablation study to evaluate how each improvement contributes to \textsc{Beep}.
Specifically, we considered five algorithms: (1) \textsc{Ditto}, (2) \textsc{Beep}-\,- (no miss codes; no LSH-based acceleration), (3) \textsc{Beep}-miss (no miss codes), (4) \textsc{Beep}-LSH (no LSH-based acceleration), and (5) the complete \textsc{Beep} algorithm.
Furthermore, we set the parameters of these algorithms, where the \textit{min support} of a pattern was $0.1 \times |S|$ for \textsc{Ditto}, and the threshold \textit{th} of LSH was $0.05 \times |S|$ for \textsc{Beep}-miss and \textsc{Beep}.
We ran the five algorithms on the 15 datasets and calculate the average performance of each algorithm on each level of datasets.

\subsubsection{Results}

The results are shown in Table \ref{tab:quantitative} and summarized as follows.

\textbf{\textsc{Beep} can discover all the planted patterns and missing values.} All five algorithms found all the planted patterns. Moreover, \textsc{Beep}-LSH and \textsc{Beep} detected all the planted missing values. Both \textsc{Ditto} and \textsc{Beep} work well in this scenario.

\textbf{\textsc{Beep}-\,- and \textsc{Ditto} have similar performance.} This proves that we can regard \textsc{Ditto} as the baseline algorithm and compare \textsc{Beep} with it fairly.

\textbf{LSH-based acceleration contributes to efficiency but sacrifices effectiveness.}
Comparing \textsc{Beep}-LSH with \textsc{Beep} and comparing \textsc{Beep}-\,- with \textsc{Beep}-miss, we find that LSH-based acceleration reduces runtime substantially.
However, when applying LSH-based acceleration, some patterns may not be found, meaning that the total description length may be longer.

\textbf{Miss codes help compress information but sacrifice efficiency.}
Comparing \textsc{Beep}-miss with \textsc{Beep} and comparing \textsc{Beep}-\,- with \textsc{Beep}-LSH, we find that miss codes can reduce the number of patterns and compress more information.
We believe that miss codes can filter out similar patterns and preserve only one, leaving a smaller set of meaningful patterns.
Miss codes also allow for the encoding of sequences with long patterns with missing values, rather than multiple short patterns, so that we obtain a short description length.
However, an algorithm that allows missing values spends more time searching for patterns.

\textbf{\textsc{Beep} strikes a good balance between effectiveness and efficiency.}
Comparing \textsc{Beep} with \textsc{Beep}-miss and \textsc{Beep}-LSH, we find that \textsc{Beep} balances the strengths of \textsc{Beep}-miss and \textsc{Beep}-LSH.
Although \textsc{Beep}-miss and \textsc{Beep}-LSH showed the best performance on efficiency and information compression, respectively, \textsc{Beep} has a nearly best performance on both efficiency and information compression.
In practice, analysts can choose one of these three algorithms according to their needs, although \textsc{Beep} will be the best choice in most scenarios.

\section{Discussion and Conclusion}
\label{sec:discussion}
Mobile user engagement is commonly pursued through a range of techniques; however, there is a tendency to apply these techniques uniformly to all users, adopting a one-size-fits-all approach. To investigate the possibility that individuals may perceive engagement techniques differently, we created HarrySpotter, a location-based augmented reality (AR) app that enables users to annotate real-world objects using six distinct engagement techniques. By deploying HarrySpotter and analyzing data from 29 users, we found that agreeable users were not motivated by competition, while conscientious users were motivated by it to a greater extent. As expected, individuals open to new experiences and extroverts were motivated by exploration, while neurotics exhibited a stronger drive towards personal achievements. These preferences for specific engagement techniques also predicted personality traits to different extents (e.g., Extraversion with an Adj. $R^2$ of 0.61, while Conscientiousness with an Adj. $R^2$ of 0.16), suggesting that engagement strategies should be tailored to one's personality.

From a theoretical perspective, our work is situated within the domain of adaptive user interfaces (AUI). The effectiveness of an AUI hinges on the ability to construct and utilize individual user profiles, allowing for the delivery of personalized versions of the user interface~\cite{jameson2007adaptive,constantinides2015exploring,constantinides2018framework,constantinides2016user}. Building upon this foundation, we envision that our methodology could be employed to enhance user models with specific characteristics, such as personality traits. This, in turn, could facilitate the personalization of user interfaces in various contexts, such as advancing levels in a gamified app or completing tasks. From a practical standpoint, our findings can inform the design of personalized engagement strategies. To illustrate, let us consider the scenario of mobile crowdsourcing systems, where a one-size-fits-all approach has proven ineffective in engaging users for specific tasks, such as object annotation~\cite{rula2014no}. By incorporating brief personality questionnaires, for example during account setup (e.g., the TIPI~\cite{gosling2003very} questionnaire, which can be completed in a minute), mobile developers can implement in-app mechanisms to dynamically infer personality traits, thereby adapting engagement strategies based on users' interactions.

Our work has three limitations that warrant further research efforts. Firstly, our findings are specific to the HarrySpotter game and this particular cohort. Future studies could extend our methodology to different types of mobile apps. For example, developing tailored strategies for Conscientiousness could enhance prediction accuracy by incorporating logging mechanisms for organized individuals. Secondly, the slight skew in Extraversion may be due to self-selection bias, as introverted individuals are less likely to engage with such apps. Future studies should replicate our methodology with larger and culturally diverse populations. Lastly, while our two-week study provided ample data, longer deployments can explore user retention and preferences more comprehensively, thereby enhancing our understanding of personalized engagement strategies.
\section{Conclusion}
\label{sec:conclusion}


In this comprehensive study, we conducted a multi-faceted examination of cryptocurrency wallets. Our research began with the formulation of an exhaustive taxonomy of crypto-wallets, shedding light on their distinctive characteristics. Following this, we deconstructed prevalent vulnerabilities and associated attack vectors, taxonomising attack based on exiting literature. We alsl
oanalyzed extant defense mechanisms and mitigation strategies targeted at neutralizing these threats. In addition to this, we conducted a comparative study of diverse wallet types, summarizing their respective security attributes and inherent risks. We also underscored the rising significance of semi-custodial wallets and quantum-resistant crypto-wallets in the dynamic landscape of digital asset security.

Overall, this research represents a concerted effort to deepen the understanding of wallet security and, thereby, contribute to the advancement of a resilient and secure cryptocurrency ecosystem.

% \old{
%     Our research provides a comprehensive analysis of crypto-wallets, focusing on their design, vulnerabilities, and defense mechanisms. We explored the cryptographic algorithms used in building wallets and categorized them based on funds controlability, internet connectivity, and infrastructure. Our primary focus was on identifying and analyzing various attacks that target crypto-wallets, as well as the available defense mechanisms.

% 	Our findings highlight that wallet attacks targeting the application software are the most prevalent from the user's perspective. To enhance security, users are advised to choose reputable wallet applications and consider using multiple wallets, combining offline and online options. We have also determined that paper wallets and brain wallets offer the highest level of security for storing cryptocurrencies. However, they require users to manage their own private keys and take full responsibility for their wallet security. Hardware wallets, while providing a good balance between security and convenience, may present usability challenges for some users. On the other hand, smart contract wallets offer ease of use but are prone to coding errors, including vulnerabilities such as re-entrancy attacks, transaction ordering issues, gas cost inefficiencies, and the risk of being a destroyable contract.

% 	We recognize the emerging importance of semi-custodial wallets, which strike a balance between user control and platform convenience. Exploring the security aspects of these wallets and developing robust defense mechanisms specific to their architecture will be crucial. Additionally, given the growing concerns regarding the potential threat posed by quantum computers to current cryptographic algorithms, investigating and developing post-quantum secured crypto-wallets will be of paramount importance in ensuring the long-term security of digital assets.
%     }

% Our research provides a comprehensive analysis of crypto-wallets, focusing on their design, vulnerabilities, and defense mechanisms. By categorizing wallets based on funds controlability, internet connectivity, and infrastructure, we shed light on the diverse cryptographic algorithms utilized in their construction. Our findings underscore the prevalence of attacks targeting wallet application software from the user's perspective, emphasizing the importance of selecting reputable wallet applications and utilizing a combination of offline and online wallets for enhanced security.

% Additionally, our investigation reveals that paper wallets and brain wallets offer the highest level of security for storing cryptocurrencies, albeit demanding users to manage their private keys independently. Hardware wallets strike a good balance between security and convenience but may pose usability challenges for some users. Meanwhile, smart contract wallets offer ease of use, but they are susceptible to coding errors and vulnerabilities. Furthermore, the emerging significance of semi-custodial wallets requires further exploration to develop specific defense mechanisms tailored to their architecture. In light of potential quantum computing threats, we advocate researching and developing post-quantum secured crypto-wallets to ensure the long-term security of digital assets. To fortify the digital vaults of crypto-wallets comprehensively, future research should address vulnerabilities unique to iOS-based wallet applications and conduct thorough security assessments of key recovery mechanisms. By advancing our understanding of wallet security and developing targeted defense measures, we can contribute to a more secure and resilient cryptocurrency ecosystem, instilling confidence among users and promoting wider adoption in the decentralized financial landscape.
\input{contents/9_acknowledgement}

% \section{Tables}

% \begin{table}
%   \caption{Frequency of Special Characters}
%   \label{tab:freq}
%   \begin{tabular}{ccl}
%     \toprule
%     Non-English or Math&Frequency&Comments\\
%     \midrule
%     \O & 1 in 1,000& For Swedish names\\
%     $\pi$ & 1 in 5& Common in math\\
%     \$ & 4 in 5 & Used in business\\
%     $\Psi^2_1$ & 1 in 40,000& Unexplained usage\\
%   \bottomrule
% \end{tabular}
% \end{table}

% \section{Math Equations}
% You may want to display math equations in three distinct styles:
% inline, numbered or non-numbered display.  Each of the three are
% discussed in the next sections.

% \subsection{Inline (In-text) Equations}

% \begin{math}
%   \lim_{n\rightarrow \infty}x=0
% \end{math},

% \subsection{Display Equations}

% Now, we'll enter an unnumbered equation:
% \begin{displaymath}
%   \sum_{i=0}^{\infty} x + 1
% \end{displaymath}
% and follow it with another numbered equation:
% \begin{equation}
%   \sum_{i=0}^{\infty}x_i=\int_{0}^{\pi+2} f
% \end{equation}
% just to demonstrate \LaTeX's able handling of numbering.

% \section{Figures}

% % Figure environment removed

% \subsection{The ``Teaser Figure''}

% \begin{verbatim}
%   \begin{teaserfigure}
%     % Figure removed
%     \caption{figure caption}
%     \Description{figure description}
%   \end{teaserfigure}
% \end{verbatim}

%%
%% The next two lines define the bibliography style to be used, and
%% the bibliography file.
\bibliographystyle{ACM-Reference-Format}
\bibliography{f4m}

%%
%% If your work has an appendix, this is the place to put it.
\appendix

\section{Optional values in the table tennis dataset}
\label{app:values}

There exist four attributes in the table tennis dataset.
All the optional values for each attribute are shown as Table \ref{tab:values}.

\begin{table}[!htb]
    \caption{The optional values of four attributes in the table tennis dataset.}
    \label{tab:values}
    \begin{tabular}{ll}
        \toprule

        Attributes&Optional values \\

        \midrule

        The technique used to hit   &Pendulum serving \\
        (16 optional values)        &Reverse serving \\
                                    &Tomahawk serving \\
                                    &Topspin \\
                                    &Smash \\
                                    &Attack \\
                                    &Flick \\
                                    &Twist \\
                                    &Push \\
                                    &Slide \\
                                    &Touch short \\
                                    &Block \\
                                    &Lob \\
                                    &Chopping \\
                                    &Pimpled techniques \\
                                    &Other techniques \\

        \cmidrule(lr){1-2}

        The area where the ball hit on the table    &Backhand serving \\
        (12 optional values)                        &Forehand serving \\
                                                    &Short forehand \\
                                                    &Half-long forehand \\
                                                    &Long forehand \\
                                                    &Short backhand \\
                                                    &Half-long backhand \\
                                                    &Long backhand \\
                                                    &Short middle \\
                                                    &Half-long middle \\
                                                    &Long middle \\
                                                    &Net or edge \\

        \cmidrule(lr){1-2}

        The type of ball's spinning &Strong topspin \\
        (7 optional values)         &Normal topspin \\
                                    &Strong downspin \\
                                    &Normal downspin \\
                                    &No spin \\
                                    &Sink \\
                                    &Without touching \\

        \cmidrule(lr){1-2}

        The position of the player  &Backhand serving \\
        (6 optional values)         &Forehand serving \\
                                    &Pivot \\
                                    &Backhand \\
                                    &Forehand \\
                                    &Back turn \\

        \bottomrule
    \end{tabular}
\end{table}
\section{Search a pattern in a sequence}

\label{app:search}

Shown as Algorithm \ref{alg:search} and Algorithm \ref{alg:check}, we use a simple DFS to search all the occurrences of pattern $p$ in sequence $s$.

\begin{algorithm}[!htb]
    \caption{Searching Algorithm}
    \label{alg:search}
    \KwIn{A sequence $s$, a pattern $p$}
    \KwOut{All occurrences $occurs$}
    \tcc{the explanation can be found in Appendix \ref{app:search}}
    $occurs \leftarrow \emptyset$\;
    $max\_gaps \leftarrow |p| - 1$, $max\_misses \leftarrow (||p|| + 5) / 10$\;
    \For{$i \leftarrow$ 1 to $|s|$} {
        $found$, $marks$, $misses$ = dfs($s$, $p$, $i$, 1, $max\_gaps$, $max\_misses$)\;
        \If{$found$} {
            $occurs \leftarrow occurs + (marks, misses)$\;
        }
    }
    \Return{$occurs$}
\end{algorithm}

\begin{algorithm}[!htb]
    \caption{DFS Algorithm}
    \label{alg:check}
    \KwIn{A sequence $s$, a pattern $p$, the start index in sequence $si$, the start index in pattern $pi$, max number of gap events $max\_gaps$, max number of missing values $max\_misses$}
    \KwOut{Whether the patter occurs $found$, all positions $marks$, and positions of missing values $misses$}
    \tcc{the explanation can be found in Appendix \ref{app:search}}
    \If{$|s| - si < |p| - pi$} {
        \Return{{\bf false}}\;
    }

    $es \leftarrow$ the $si$-th event of $s$\;
    $ep \leftarrow$ the $pi$-th event of $p$\;
    $marks \leftarrow \emptyset$\;
    $num\_misses \leftarrow 0$, $misses \leftarrow \emptyset$\;
    \For{$k \leftarrow$ 1 to $|A|$} {
        \If{$ep[k]$ is not empty} {
            \eIf{$ep[k] != es[k]$} {
                $num\_misses += 1$\;
                $misses \leftarrow misses + (si, k)$
            } {
                $marks \leftarrow marks + (si, k)$
            }
        }
    }
    \If{$num_misses > max_misses$} {
        \Return{{\bf false}}\;
    }

    \For{$i \leftarrow 0$ to $max\_gaps$} {
        $found$, $marks^*$, $misses^*$ = dfs($s$, $p$, $si+1+i$, $pi+1$, $max\_gaps-i$, $max\_misses-num\_misses$)\;
        \If{$found$} {
            $marks \leftarrow marks \cup marks^*$\;
            $misses \leftarrow misses \cup misses^*$\;
            \Return{{\bf true}, marks, misses}\;
        }
    }
    \Return{{\bf false}}\;
\end{algorithm}

\end{document}
\endinput
%%
%% End of file `sample-authordraft.tex'.
