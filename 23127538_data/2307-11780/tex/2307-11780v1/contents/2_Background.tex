% \section{Definitions}
\section{Problem Formulation}

\label{sec:Predefinitions}

\textsc{Beep} accepts a multivariate event sequence dataset $S$ as input and outputs a pattern set $P$, demonstrated as Figure \ref{fig:patterndef}.

\textbf{Each sequence} $s \in S$ is a vector of events, denoted $s = (e_1, e_2, ...,\\ e_{|s|})$.
All sequences are over the same set of categorical attributes $A=\{a_1, a_2, ..., a_{|A|}\}$.
Thus, each event $e$ can be considered a vector of values, denoted $e=(a_1=v_1, a_2=v_2, ..., a_{|A|}=v_{|A|})$.

\textbf{Each pattern} $p \in P$ is a subsequence of numerous original sequences in the dataset $S$, where $|p|$ indicates the length and $||p||$ indicates the number of non-empty values.
% For example, in Figure \ref{fig:patterndef}, four sequences ($s_1$ to $s_4$) can be summarized by two patterns ($p_1$ and $p_2$).
For example, in Figure \ref{fig:patterndef}, four sequences ($s_1$ to $s_4$) can be summarized by pattern $p_1$ ($||p_1||=4$) and pattern $p_2$ ($||p_2||=6$).
The formal definition of ``subsequence'' can be found in Appendix \ref{app:subsequence}.
\textsc{Beep} further supports four features as follows to obtain informative patterns in real-world datasets.

\input{figures/patternDefinition}

\begin{itemize}[leftmargin=10pt]
  \item \textbf{Empty values} in a pattern can match any values.
  For example, in Figure \ref{fig:patterndef}, $v_2$ of $e_3$ in $p_1$ is empty, which matches $z$ in $s_1$ and $y$ in $s_4$.
  Empty values allow \textsc{Beep} to tolerate the ever-changing values in a pattern, which were less important.

  \item \textbf{Missing values} allow patterns to tolerate some noise values in sequences.
  For example, $s_3$ misses $v_2$ of $e_2$ when it is covered by $p_2$.
  \textsc{Beep} considers the possibility (common in real-world data) that the missing value is actually a noise value.

  \item \textbf{Gap events} can separate two consecutive events in a pattern into non-consecutive events in sequences.
  For example, sequence $s_2$ has pattern $p_2$, but the gap event $e_2$ of $s_2$ is not captured by $p_2$.
  \textsc{Beep} allows gaps to ignore events that may be noise.

  \item \textbf{Interleaving patterns} overlap over a period of time\cite{tatti2012long}, e.g., pattern $p_1$ and $p_2$ are interleaving in $s_4$.
  \textsc{Beep} allows interleaving patterns to discover simultaneous patterns.
\end{itemize}

In practice, domain experts usually expect patterns to be informative (i.e., fewer empty values), authentic (i.e., fewer missing values), and compact (i.e., fewer gaps).
Thus, for each pattern $p$, we limit the maximum number of empty values ($|p|$), missing values ($||p|| \div 10$, and at most 1 for each event), and gap events ($|p|-1$).
