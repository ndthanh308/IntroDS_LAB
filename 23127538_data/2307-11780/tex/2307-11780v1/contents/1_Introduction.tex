\section{Introduction}

\label{sec:Introduction}

Multivariate event sequence data is widely analyzed in racket sports, such as table tennis \cite{wang2019tac, wang2021tac, Lan2021RallyComparator}, tennis \cite{polk2019courttime,wu2022rasipam}, and badminton \cite{wu2020visual,wu2021tacticflow}.
These works usually model each hit as an event with multiple attributes, each for a hitting feature (e.g., the ball position, the hitting technique, etc.).
Consecutive hits, starting with one player serving the ball and ending with one player winning a point, constitute a multivariate event sequence.
Based on such a data model, pattern mining algorithms can discover frequent patterns (shown as Figure \ref{fig:patterndef}), which can be regarded as players' tactics, to help domain experts in sports obtain insights into players' playing styles and thereby improve their performance.
However, in five years of collaboration with domain experts in racket sports, we found that multivariate event sequences placed high-level requirements on both \textbf{effectiveness} and \textbf{efficiency} of pattern mining algorithms.

\textbf{Effective} pattern mining algorithms can discover patterns that reveal meaningful information about the sequences. To be effective, a pattern mining algorithm for multivariate event sequences must fulfill three conditions:
(1) The correlations among multiple attributes within sequences should be preserved for domain analysis.
For example, in table tennis, a tactical pattern may be that when a player hits the ball to a certain \textit{position} on the table, the opponent always uses a specific \textit{technique} in response.
(2) The algorithm should have a high tolerance for single-value noises (i.e., changes on only one attribute).
For example, in table tennis, when a player applies a tactical pattern, he/she may only change the technique of one hit to a similar technique, retaining the overall playing style.
(3) The number of returned patterns should be manageable.
Given that multivariate patterns are complicated, analyzing them is both time-consuming and mentally overwhelming.

\textbf{Efficient} pattern mining algorithms can mine multivariate patterns within an acceptable response time.
In practice, the time allowed for pattern analysis is limited (e.g., one hour).
Moreover, some parameters should be adjusted based on the analysts' feedback to satisfy the analysts' requirements.
Thus, the algorithm is expected to return the results in several minutes.

To the best of our knowledge, existing multivariate pattern mining algorithms cannot satisfy these two requirements simultaneously.
Some algorithms \cite{morchen2007efficient,chen2010efficient,bertens2014characterising} transform multivariate sequences into univariate ones, such that extracted patterns cannot retain any correlations between attributes.
Algorithms based on SPM (Sequential Pattern Mining) \cite{oates1996searching,tatti2011mining,wu2013mining,fournier2017survey} retain the correlations between attributes but usually return an enormous number of patterns rather than seeking the most meaningful ones, due to the well-known problem of \textit{pattern explosion} \cite{menger2015experimental}.
Algorithms based on MDL (Minimum Description Length) summarize a set of patterns to describe the entire sequence dataset instead of searching for each pattern \cite{bertens2016keeping,kawabata2018streamscope,wu2020visual}, thus avoiding \textit{pattern explosion}.
However, the current MDL-based methods have no tailored algorithmic design to handle single-value noises in sports and are usually time-consuming.

In this paper, we propose \textsc{Beep}, a novel pattern mining algorithm which \textbf{B}alances \textbf{E}ffectiveness and \textbf{E}fficiency when finding \textbf{P}atterns in racket sports.
The contributions are mainly as follows.
\begin{itemize}[leftmargin=10pt]
    \item We introduce a new encoding scheme for the noise values in a pattern, enhancing \textsc{Beep}'s effectiveness (i.e., tolerance of noises).
    \item We propose a tailored acceleration method based on \textit{Locality Sensitive Hashing} so that the patterns with high frequencies can be found in a short time, enhancing \textsc{Beep}'s efficiency.
    \item We conducted an empirical study with analysts in table tennis to demonstrate that \textsc{Beep} can help analysts obtain insights into players' playing styles.
    We further compared \textsc{Beep} with the current SOTA algorithm on multi-scaled synthetic datasets, proving that \textsc{Beep} was about five times faster than the SOTA algorithm.
\end{itemize}
