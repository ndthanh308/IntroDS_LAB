\section{Conclusion}

In this paper, we have presented a basic operator called 3D deformable attention (DFA3D), built upon which we develop a novel feature lifting approach. Such an approach not only takes depth into consideration to tackle the problem of depth ambiguity but also benefits from the multi-layer refinement mechanism. 
We seek assistance from math to simplify DFA3D and develop a memory-efficient implementation. The simplified DFA3D makes querying features in 3D space through deformable attention possible and efficient. The experimental results show a consistent improvement, demonstrating the superiority and generalization ability of DFA3D-based feature lifting.

\textbf{Limitations and Future Work.} In this paper, we simply take monocular depth estimation to provide depth information for DFA3D, which is not accurate and stable enough. Recently proposed methods~\cite{sparse4dv2, streampetr} have proved that long temporal information can provide the network with a better depth sensing ability. How to make full use of the superior depth sensing ability to generate high-quality depth maps explicitly, and utilize them to help improve the performance of DFA3D-based feature lifting is still an issue, which we leave as future work.