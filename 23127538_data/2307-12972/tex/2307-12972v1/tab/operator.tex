\begin{table}[t]
\centering
\renewcommand\arraystretch{1.2}
\caption{Comparisons of feature lifting methods. DFA2D denotes the 2D deformable attention. 
We use the same architecture and supervision for depth estimation as our {\methodname} for the Lift-Splat-based method. We also leverage SECOND FPN~\cite{yan2018second} to enable multi-scale feature maps for the Lift-Splat-based method.
}
\resizebox{1.01\linewidth}{!}{
\setlength{\tabcolsep}{1.5pt}
\begin{tabular}{l|c|c|ccccc|c} 
\shline
\textbf{Method}            & \textbf{\#layers} &
\textbf{mAP}$\uparrow$  & \textbf{mATE}$\downarrow$ & \textbf{mASE}$\downarrow$  & \textbf{mAOE}$\downarrow$ & \textbf{mAVE}$\downarrow$ & \textbf{mAAE}$\downarrow$ & \textbf{NDS}$\uparrow$ \\
\shline
PointAttn & 1            & 35.0 & 76.3 & 27.8  & 42.9  & \textbf{84.9} & \textbf{20.7} & 42.2  \\
DFA2D & 1          & 35.9 & 74.6 & 27.8  & 42.5  & 85.8 & 20.9 & 42.8  \\
Lift-Splat      & 1                    & 36.9 & \textbf{73.1} & 28.0  & 44.7  & 85.0 & 23.6 & 43.0  \\
 \rowcolor{gray!15} {\methodname} (Ours) & 1          & \textbf{37.3} & 73.4 & \textbf{27.5}  & \textbf{41.7}  & 85.2 & 22.5 & \textbf{43.6}  \\
\shline
DFA2D & 2          & 37.1 & 72.7 & \textbf{27.7}  & 43.5  & 81.4 & \textbf{20.6} & 44.0  \\
 \rowcolor{gray!15} {\methodname} (Ours) & 2     & \textbf{37.9} & \textbf{72.4} & \textbf{27.7}  & \textbf{38.1}  & \textbf{78.1} & 22.3 & \textbf{45.1}  \\
\shline
\end{tabular}
}
\vspace{2pt}

\label{tab:operator}
\end{table}