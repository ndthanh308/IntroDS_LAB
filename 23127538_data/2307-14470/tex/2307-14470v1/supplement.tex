%%%% Header %%%%%%%%%%%%%%%%%%%%%%%%%%%%%%%%%%%%%%%%%

%%%% Packages %%%%%%%%%%%%%%%%%%%%%%%%%%%%%%%%%%%%%%
\documentclass[12pt]{article}
\usepackage[utf8]{inputenc}
\usepackage{graphicx}
\graphicspath{{figures/}}
\usepackage{color}
\usepackage{amsmath}
\usepackage{float}
\usepackage{hyperref}
\usepackage{comment}
\usepackage[margin=1in]{geometry}
\setcounter{secnumdepth}{0} % do not number sections
\pagenumbering{arabic} % {gobble}
\usepackage[style=nature,doi=false,isbn=false,url=false]{biblatex}
\addbibresource{references.bib}
\newcommand{\ordinal}[2]{${#1}^{\text{#2}}$} 
\usepackage[labelfont=bf]{caption}

%%%% Math Functions %%%%%%%%%%%%%%%%%%%%%%%%%%%%%%%
\DeclareMathOperator*{\argmax}{arg\,max}
\DeclareMathOperator*{\argmin}{arg\,min}

%%%% Markup functions %%%%%%%%%%%%%%%%%%%%%%%%%%%%%%%
\newcommand{\addvalue}{\textcolor{red}{ADD VALUE}}
\newcommand{\addcite}{\textcolor{red}{CITE HERE}}
\newcommand{\addcomment}[3]{\textcolor{#1}{{\bf\boldmath {#2}:} {#3}}} 

\usepackage[utf8]{inputenc}
\usepackage[T1]{fontenc}
\usepackage{mathptmx}
\usepackage{multirow}
\usepackage{chemformula}
\usepackage{upgreek}
\usepackage{float}
\usepackage{subcaption}
\usepackage{cleveref}
\captionsetup{justification=raggedright,singlelinecheck=false}
% \captionsetup{justification=justified,singlelinecheck=false}
\usepackage[labelfont=bf, format=plain]{caption}
\setcounter{figure}{0}
\renewcommand{\figurename}{Fig.}
\renewcommand{\thefigure}{S\arabic{figure}}
\renewcommand{\thetable}{S\arabic{table}}
\newcommand{\BR}[1]{{\color{red}{#1}}}
\newcommand{\GMS}[1]{{\color{blue}{#1}}}

%%%% Title %%%%%%%%%%%%%%%%%%%%%%%%%%%%%%%%%%%%%%%%%

\title{Supplementary Materials for ``Predicting Relative Populations of Protein Conformations without a Physics Engine Using AlphaFold2"}

\author{{\bf Gabriel Monteiro da Silva and Jennifer Y. Cui} \\
\textit{Brown University Department of Molecular Biology, Cell Biology, } \\
\textit{and Biochemistry, Providence, RI, USA} \\ \\
{\bf David C. Dalgarno} \\
\textit{Dalgarno Scientific LLC, Brookline, MA, USA} \\ \\
{\bf George P. Lisi and Brenda M. Rubenstein} \\
\textit{Brown University Department of Molecular Biology, Cell Biology,} \\
\textit{and Biochemistry} \\
\textit{Brown University Department of Chemistry} \\
\textit{Providence, RI, USA} \\ \\
}

%%%% Body %%%%%%%%%%%%%%%%%%%%%%%%%%%%%%%%%%%%%%%%%
\begin{document}
\include{MyCommand}
\maketitle





\maketitle


% -------------------------------------------------------------------------------%
\section{Abl1 Ortholog Sequences Used to Generate Multiple Sequence Alignments}

% Figure environment removed

\section{Comparisons Between AF2 Predictions and Molecular Dynamics Snapshots along the Ground to I2 Transition}

% Figure environment removed




% Figure environment removed

% Figure environment removed

% Figure environment removed


\section{GMCSF Chemical Shift Perturbations}
% Figure environment removed


\section{Optimization of AF2 Parameters for the GMCSF Protein}

% Figure environment removed


\section{GMCSF Dynamics Prediction}

% Figure environment removed

% Figure environment removed

\section{Optimization of AF2 Parameters for the Abl1 Protein}

\begin{table}[H]
    \centering
    \caption{Optimized AF2 parameters for predicting Abl1 ensembles.} 
    \begin{tabular}{lllllll}
        \textbf{parameter\_test} & \textbf{max\_seq} & \textbf{extra\_seq} & \textbf{n\_recycles} & \textbf{n\_models} & \textbf{n\_seeds} & \textbf{\%\_notground} \\ \hline
        \textbf{t\_max\_extra\_1} & 32 & 64 & 4 & 5 & 32 & 2 \\ 
        \textbf{t\_max\_extra\_2} & 64 & 128 & 4 & 5 & 32 & 5 \\ 
        \textbf{t\_max\_extra\_3} & 128 & 256 & 4 & 5 & 32 & 9 \\ 
        \textbf{t\_max\_extra\_4} & 256 & 512 & 4 & 5 & 32 & 18 \\ 
        \textbf{t\_max\_extra\_5} & 512 & 1024 & 4 & 5 & 32 & 15 \\ 
        \textbf{t\_max\_extra\_6} & 2048 & 4096 & 4 & 5 & 32 & 7 \\ 
        \textbf{t\_max\_extra\_7} & 4098 & 8192 & 4 & 5 & 32 & 6 \\ 
        \textbf{t\_max\_extra\_8} & 512 & 32 & 4 & 5 & 32 & 18 \\ 
        \textbf{t\_max\_extra\_9} & 32 & 512 & 4 & 5 & 32 & 1 \\ 
        \textbf{t\_nseeds\_1} & 256 & 512 & 4 & 5 & 128 & 12 \\ 
        \textbf{t\_nseeds\_2} & 256 & 512 & 4 & 5 & 300 & 12 \\ 
        \textbf{t\_nrecycles\_1} & 32 & 64 & 8 & 5 & 128 & 0 \\ 
        \textbf{t\_nrecycles\_2} & 32 & 64 & 8 (kept) & 5 & 128 & 2 \\ 
        \textbf{t\_nrecycles\_3} & 256 & 512 & 8 & 5 & 128 & 8 \\ 
        \textbf{t\_nrecycles\_4} & 256 & 512 & 8 (kept) & 5 & 128 & 21 \\ 
    \label{Table S1}
    \end{tabular}
\end{table}

\section{AF2 Predictions of the Relative State Populations of Abl1 Kinase Core Mutants}

\begin{table}[H]
    \centering
    \caption{Abl1 kinase core mutants and their observed or expected effects on the relative populations of the active (Ground), inactive 1 (I1), or inactive 2 (I2) states.}
    \label{table:S2}
    \begin{tabular}{lllllll}
        \textbf{} & \textbf{Ground} & \textbf{I1} & \textbf{I2} \\ \hline
        \textbf{Wild-Type} & 88 & 6 & 6 \\ 
        \textbf{} & ~ & ~ & ~ \\ 
        \textbf{M290L} & 55 & 10 & 35 \\ 
        \textbf{L301I} & 25 & 10 & 65 \\ 
        \textbf{M290L + L301I} & 8 & 10 & 82 \\ 
        \textbf{} & ~ & ~ & ~ \\ 
        \textbf{F382L} & 90 & 0 & 10 \\ 
        \textbf{F382Y} & 10 & 0 & 90 \\ 
        \textbf{F382V} & 5 & 0 & 95 \\ 
        \textbf{} & ~ & ~ & ~ \\ 
        \textbf{I2M} & 10 & 0 & 90 \\ 
        \textbf{E255V (I2M background)} & nr & nr & 45 \\ 
        \textbf{T315I (I2M background)} & 93 & 0 & 7 \\ 
        \textbf{E255V + T315i} & nr & nr & nr
    \end{tabular}
\end{table}


\section{Molecular Dynamics and WESTPA2 Simulations\label{misc}}
Molecular dynamics simulations of wild-type Abl1 were conducted using the OpenMM software package (Eastman 2017) with the amber99sb-ildn force field (Lindorff Larsen 2010) and the tip3p water model (Mark 2001) at 300 K and 1 atm. The lowest energy Abl1 structure from the PDB 6XR6 NMR ensemble was solvated within a dodecahedron box and charges were neutralized by replacing a number of solvent atoms with chloride and sodium ions. Following solvation, we minimized the energy of each system using a steepest-descent algorithm until the maximum force on any given atom was less than 1000 kJ/mol/min or until 50,000 minimization steps were conducted. We ran the simulations with a 1 fs time step during the equilibration phase and a 2 fs time step during the production phase. We equilibrated solvent atoms first for 1 ns in the NVT ensemble and then for 1 ns in the NPT ensemble with solute heavy atoms restrained using the LINCS algorithm with a spring constant of 1,000 kJ/mol/m$^2$ (Hess 1997). The production phase (in the NPT ensemble) followed the equilibration phase but without restraints.

We used the WESTPA2 (Bogetti 2022) enhanced-sampling method to access the timescales necessary to simulate the inactivation pathway of Abl1. This was done via two WESTPA2 simulations (ground to I1 and I1 to I2). As progress coordinates for the ground to I1 transition, we defined the distance between the backbone oxygen of V299 and the center of mass of the carboxyl group of D381 as PC1; and the angle formed by the center of mass of the carboxyl group of D381, the backbone oxygen of K379, and the center of mass of the aromatic ring of F382 as PC2. For the I1 to I2 transition, we defined the distance between the backbone oxygen of L409 and the backbone oxygen of E377 as PC1; and the distance between backbone oxygen of L409 and the backbone oxygen of G4598 as PC2. Representative illustrations of the progress coordinates used in this protocol are in Figure S10, and their distributions and start/end state definitions are described in Fig S11. We ran WESTPA2 for 300 iterations for each leg of the transition, with the number of walkers per iteration varying from 64 to 512 due to the adaptive binning scheme, and 100 ps per iteration, totaling over 9 us of aggregate simulation time for each leg of the transition.

% Figure environment removed

% Figure environment removed
% \section{References}
% \printbibliography[heading=none]



\end{document}