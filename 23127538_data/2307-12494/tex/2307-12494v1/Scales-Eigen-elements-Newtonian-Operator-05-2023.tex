\documentclass[11pt,leqno,twoside]{amsart}

%\documentclass[leqno]{amsart}
\usepackage{fancyvrb}
\usepackage{fancyhdr}
\usepackage{etoolbox}
%\usepackage{Tikz, tcolorbox}
\usepackage{enumerate}
%\usepackage{tikz}
\usepackage{pgfplots}
\pgfplotsset{compat=1.16, width = 10cm}
%\usepackage{axis}

%\usepackage{cancel}

\usepackage{amssymb}
\usepackage{placeins}
%\usepackage[pagewise]{lineno}\linenumbers
%\usepackage{german}
\usepackage[english]{babel}
\usepackage{amssymb,amsthm,amsmath,eucal,mathrsfs}
\usepackage{bm}


\setlength{\textwidth}{16.3cm}

\setlength{\textheight}{21.6cm}

\hoffset=-55pt

\newcommand*{\QEDA}{\hfill\ensuremath{\blacksquare}}%
\newcommand*{\QEDB}{\hfill\ensuremath{\square}}%

%%%%%%%%%%%%%%%%%%%%%%%%%%%%%%%%%%%%
%%%%%%%%%%%%%%%%%%%%%%%%%%%%%%%%%%%%
\usepackage{amsmath}
\usepackage{amsfonts}
\usepackage{float}
\usepackage{amsmath}
\usepackage{amssymb}
\usepackage{graphicx}
%\usepackage[utf8]{inputenc}
%\usepackage[new]{old-arrows}
\usepackage{lscape}
\usepackage{amstext}
\usepackage{amsthm}
\usepackage{color}
\usepackage{float}
%\usepackage[lite]{mtpro2}
%\usepackage{ulem}
%\usepackage{cancel}
%\usepackage{geometry}
\usepackage{mathrsfs}
\usepackage{epsfig}
\usepackage{url}
\usepackage{fancyhdr}
\usepackage{pspicture}
%\usepackage{wrapfig}
\usepackage{graphicx}
%\usepackage{multirow}
%\usepackage{lmodern}
%%%%%%%%%%%%%%%%%%%%%%%%%%%%%%%%%
%%%%%%%%%%%%%%%%%%%%%%%%%%%%%%%%%%%%
%%%%%%%%%%%%%%%%%%%%%%%%%%%%%%%%%%%%







%\usepackage[latin1]{inputenc}  

%\usepackage[T1]{fontenc}       

%\usepackage[ngerman]{babel}

%\usepackage{tikz, subfigure, xcolor} 

%\usepackage{pgfplots}
%\usepackage{Comment}
\usepackage{cite}

\usepackage{amsmath,verbatim}

\usepackage{amsthm}

\usepackage{amssymb}

\usepackage{amsfonts}

%\usepackage{showkeys}
%\usepackage{cancel}


\usepackage{dsfont}



\usepackage{hyperref}



\newtheorem{theorem}{Theorem}[section]

\newtheorem{lemma}[theorem]{Lemma}

\newtheorem{proposition}[theorem]{Proposition}

\newtheorem{corollary}[theorem]{Corollary}

\newtheorem{assumption}[theorem]{Assumption}

%\newtheorem{claim}[theorem]{Claim}
\newtheorem{claim}{Claim}
\newtheorem{Hypotheses}{Hypotheses}
%

\theoremstyle{definition}
\newtheorem{definition}[theorem]{Definition}
\newtheorem{remark}[theorem]{Remark}

\newtheorem{example}{Example}[section]

%\newenvironment{Example}{\begin{example}}{\hfill\qed\end{example}}



\numberwithin{equation}{section}



%Mathematische Symbole

\newcommand{\diag}{\mathrm{diag}}      % diagonal

\newcommand{\dist}{\mathrm{dist}}      % distance

\newcommand{\diam}{\mathrm{diam}}      % diameter

\newcommand{\codim}{\mathrm{codim}}    % co-dimension

\newcommand{\supp}{\mathrm{supp}}      % support

\newcommand{\clo}{\mathrm{clo}}        %closure

\newcommand{\beweisqed}{\hfill$\Box$\par} % Beweisende

\newcommand{\Span}{\mathrm{span}}       % span

\renewcommand{\Im}{{\ensuremath{\mathrm{Im\,}}}} %Imagin????rteil nicht alsfraktur

\renewcommand{\Re}{{\ensuremath{\mathrm{Re\,}}}} %Realteil nicht als fraktur

\renewcommand{\div}{\mathrm{div}\,}    %div anstatt geteilt







\providecommand{\norm}[1]{\lVert#1\rVert} %Norm

\providecommand{\abs}[1]{\lvert#1\rvert} % absolut value

%\providecommand{\au}[1]{\underline{a}} % length of the edges



\DeclareMathOperator{\Hom}{Hom}

\DeclareMathOperator{\End}{End} 

\DeclareMathOperator{\Imt}{Im} 

\DeclareMathOperator{\Ran}{Ran} 

\DeclareMathOperator{\ran}{Ran} 

\DeclareMathOperator{\Rank}{Rank} 

\DeclareMathOperator{\sign}{sign} 

\DeclareMathOperator{\sgn}{sign} 

\DeclareMathOperator{\tr}{tr}

\DeclareMathOperator{\Ker}{Ker}

\DeclareMathOperator{\Dom}{Dom}

\DeclareMathOperator{\dom}{dom}

\DeclareMathOperator{\grad}{grad}

\DeclareMathOperator{\Gr}{Gr}





\DeclareMathOperator{\dd}{\mbox{d}} %differential

\DeclareMathOperator{\ddx}{\mbox{dx}} %dx

\DeclareMathOperator{\ddy}{\mbox{dy}} %dx



\DeclareMathOperator{\Sm}{\mathfrak{S}} %coefficient matrix

\DeclareMathOperator{\Cm}{\mathfrak{C}} %coefficient matrix

\DeclareMathOperator{\Jm}{\mathfrak{J}} %coefficient matrix

\DeclareMathOperator{\jm}{\mathfrak{j}} %coefficient matrix







\newcommand{\au}{\underline{a}}

\newcommand{\bu}{\underline{b}}

%mathematische Schriften

\newcommand{\HH}{\mathbb{H}}

\newcommand{\Q}{\mathbb{Q}}

\newcommand{\R}{\mathbb{R}}

\newcommand{\T}{\mathbb{T}}

\newcommand{\C}{\mathbb{C}}

\newcommand{\Z}{\mathbb{Z}}

\newcommand{\N}{\mathbb{N}}

\newcommand{\bP}{\mathbb{P}}

\newcommand{\PP}{\mathbb{P}}

\newcommand{\1}{\mathbb{I}}

\newcommand{\E}{\mathbb{E}}

\newcommand{\EE}{\mathsf{E}}







\newcommand{\fQ}{\mathfrak{Q}}

\newcommand{\fH}{\mathfrak{H}}

\newcommand{\fR}{\mathfrak{R}}

\newcommand{\fF}{\mathfrak{F}}

\newcommand{\fB}{\mathfrak{B}}

\newcommand{\fS}{\mathfrak{S}}

\newcommand{\fD}{\mathfrak{D}}

\newcommand{\fE}{\mathfrak{E}}

\newcommand{\fG}{\mathfrak{G}}

\newcommand{\fP}{\mathfrak{P}}

\newcommand{\fT}{\mathfrak{T}}

\newcommand{\fC}{\mathfrak{C}}

\newcommand{\fM}{\mathfrak{M}}

\newcommand{\fI}{\mathfrak{I}}

\newcommand{\fL}{\mathfrak{L}}

\newcommand{\fW}{\mathfrak{W}}

\newcommand{\fK}{\mathfrak{K}}

\newcommand{\fA}{\mathfrak{A}}

\newcommand{\fV}{\mathfrak{V}}



\newcommand{\fa}{\mathfrak{a}}

\newcommand{\fb}{\mathfrak{b}}

\newcommand{\fh}{\mathfrak{h}}

\newcommand{\ft}{\mathfrak{t}}

\newcommand{\fs}{\mathfrak{s}}

\newcommand{\fv}{\mathfrak{v}}

\newcommand{\fq}{\mathfrak{q}}

%%%%%%%%%%%%%%%%%%%%%%%%%%%%%%%%

%Kaliegraphie

%%%%%%%%%%%%%%%%%%%%%%%%%%%%%%%%

\newcommand{\cA}{{\mathcal A}}

\newcommand{\cB}{{\mathcal B}}

\newcommand{\cC}{{\mathcal C}}

\newcommand{\cD}{{\mathcal D}}

\newcommand{\cE}{{\mathcal E}}

\newcommand{\cF}{{\mathcal F}}

\newcommand{\cG}{{\mathcal G}}

\newcommand{\cH}{{\mathcal H}}

\newcommand{\cI}{{\mathcal I}}

\newcommand{\cJ}{{\mathcal J}}

\newcommand{\cK}{{\mathcal K}}

\newcommand{\cL}{{\mathcal L}}

\newcommand{\cM}{{\mathcal M}}

\newcommand{\cN}{{\mathcal N}}

\newcommand{\cO}{{\mathcal O}}

\newcommand{\cP}{{\mathcal P}}

\newcommand{\cQ}{{\mathcal Q}}

\newcommand{\cR}{{\mathcal R}}

\newcommand{\cS}{{\mathcal S}}

\newcommand{\cT}{{\mathcal T}}

\newcommand{\cU}{{\mathcal U}}

\newcommand{\cV}{{\mathcal V}}

\newcommand{\cX}{{\mathcal X}}

\newcommand{\cW}{{\mathcal W}}

\newcommand{\cZ}{{\mathcal Z}}

%Doppelt definieret symbole

\newcommand{\Me}{{\mathcal M}}

\newcommand{\He}{{\mathcal H}}

\newcommand{\Ke}{{\mathcal K}}

\newcommand{\Ge}{{\mathcal G}}

\newcommand{\Ie}{{\mathcal I}}

\newcommand{\Ee}{{\mathcal E}}

\newcommand{\De}{{\mathcal D}}

\newcommand{\Qe}{{\mathcal Q}}

\newcommand{\We}{{\mathcal W}}

\newcommand{\Be}{{\mathcal B}}



\newcommand{\lpso}{L^p_{{\sigma}}(\Omega)}



\newcommand\restr[2]{{% we make the whole thing an ordinary symbol
  \left.\kern-\nulldelimiterspace % automatically resize the bar with \right
  #1 % the function
  \vphantom{\big|} % pretend it's a little taller at normal size
  \right|_{#2} % this is the delimiter
  }}





% David's macros

\newcommand{\cf}{\emph{cf.}}

\newcommand{\ie}{{\emph{i.e.}}}

\newcommand{\eg}{{\emph{e.g.}}}

\usepackage{eucal}

\newcommand{\verts}{{\mathcal V}}

%%%%%%%%%%%%%%%%%%%%%%%%%%%%%%%%%%%%%%%%%%%%%%%%%%%%%%
\date{\today}   

 

%\title[Photo-acoustic inversion using plasmonics]{Photo-acoustic inversion using plasmonic contrast agents:\\ The full Maxwell model}

\title[Logarithmic potential]{Estimation of the eigenvalues and the integral of the eigenfunctions of the Newtonian potential operator}


\author[Alsenafi,, Ghandriche and Sini]{Abdulaziz Alsenafi$^{*}$, Ahcene Ghandriche  $^{**}$ and Mourad Sini$^{\ddag}$}
\thanks{$^*$ Department of Mathematics, Faculty of Science, Kuwait University, P.O. Box 5969, Safat 13060, Kuwait. Email: abdulaziz.alsenafi@ku.edu.kw}
\thanks{$^*$ Nanjing Center for Applied Mathematics, Nanjing 211135, People's Republic of China. Email: gh.hsen@njcam.org.cn.}% This author is supported by the Austrian Science Fund (FWF): P 30756-NBL}
\thanks{$^{\ddag}$ RICAM, Austrian Academy of Sciences, Altenbergerstrasse 69, A-4040, Linz, Austria. Email: mourad.sini@oeaw.ac.at.} %This author is partially supported by the Austrian Science Fund (FWF): P 30756-NBL}


\begin{document}

\subjclass[2010]{31B10, 35R30, 35C20}
\keywords{Logarithmic potential operator, asymptotic expansions, Bessel functions, min-max principle, spectral theory.}
\maketitle



\begin{abstract}
We consider the problem of estimating the  eigenvalues and the integral of the corresponding eigenfunctions, associated to the Newtonian potential operator, defined in a bounded domain $\Omega \subset \mathbb{R}^{d}$, where $d=2,3$, in terms of the maximum radius of $\Omega$. %This operator arises naturally in the integral formulation of the acoustic/electric field propagation in $\mathbb{R}^{d}$. %As an application the electric $\cdots \cdots$ in two dimensional space is discussed.
We first provide these estimations in the particular case of a ball and a disc. Then we extend them to general shapes using a, derived, monotonicity property of the eigenvalues of the Newtonian operator. The derivation of the lower bounds are quite tedious for the 2D-Logarithmic potential operator. Such upper/lower bounds appear naturally while estimating the electric/acoustic fields propagating in $\mathbb{R}^{d}$ in the presence of small scaled and highly heterogeneous particles.
\end{abstract}


\bigskip

\section{Introduction and statement of the results}
\subsection{Introduction}\label{SubsectionI}
The Newtonian potential integral operator, in $\mathbb{R}^{d}$, $d=2,3$, is of considerable interest in both scattering and potential theory, without being exhaustive we refer the readers to \cite{cartan1945theorie, ammari2019subwavelength, alsenafi2022foldy, ghandriche2022mathematical, colton2019inverse, ammari2018super}. 

%This integral operator occurs while using integral equation methods for the boundary value problem given by the Laplacian operator with non-local boundary conditions. More exactly, in \cite[Theorem 2.1]{kalmenov2011boundary}, the authors proved that a function $u(\cdot)$ is a solution of the following problem
%\medskip
%\newline 
%\begin{equation}
%\left\{
%\begin{array}{rll}
% - \, \Delta u  &=& f, \quad \text{in} \quad \Omega,\\
% && \\
%    - \frac{1}{2} \, u(\cdot) + \int_{\partial \Omega} \partial_{\nu} \phi_{0}(\cdot,y) \, u(y) \, d\sigma(y) - \int_{\partial \Omega} \phi_{0}(\cdot,y) \partial_{\nu} u(y) \, d\sigma(y) &=& 0, \quad \text{on} \quad \partial \Omega, 
%    \end{array}
%\right.
%\end{equation}
%if and only if 
Let $\Omega$ be a bounded and Lipschitz-regular domain of $\mathbb{R}^{d}$, $d=2,3$. The Newtonian operator correspond to any function $f$ the potential
\begin{equation}\label{defN}
u(x) = N_{\Omega}\left( f \right)(x) := \int_{\Omega} \phi_{0}(x,y) \, f(y) \, dy, \quad x \in \Omega,
\end{equation}
 where 
\begin{equation}\label{FundamentalSolution}
\phi_{0}(x,y) := \left\{
\begin{array}{rll}
  \dfrac{-1}{2 \, \pi} \log\left\vert x - y \right\vert \quad & \text{in} & \quad \mathbb{R}^{2},\\
 && \\
    \dfrac{1}{4 \, \pi \, \left\vert x-y \right\vert} \quad & \text{in} & \quad \mathbb{R}^{3}, 
    \end{array}
\right.
\end{equation}
is the fundamental solution of the Laplacian operator, i.e., 
\begin{equation*}
- \Delta \phi_{0}(x,y) = \delta(x,y), \quad \text{in} \quad \mathbb{R}^{d} \quad \text{for} \; d=2,3,
\end{equation*}
where $\delta(\cdot,\cdot)$ is the Dirac function. It is well known that $N_{\Omega}(\cdot)$, 
%when $\Omega$ is such that\footnote{Add ref} $\diam\left( \Omega \right) < 1$, 
is a linear, compact, self-adjoint non-negative operator on $\mathbb{L}^{2}(\Omega)$ and it carries $\mathbb{L}^{2}(\Omega)$ to $\mathbb{H}^{2}(\Omega)$. For other useful properties  of Newtonian potential operator we refer the readers to \cite{Anderson, kellogg, ruzhansky2016isoperimetric, kalmenov2011boundary}. Therefore, $N_{\Omega}(\cdot)$ has a countable decreasing sequence of eigenvalues, with possible multiplicities, that we denote in the sequel by $\left\{ \lambda_{n}(\Omega) \right\}_{n \geq 0}$, with the corresponding eigenfunctions as a basis of the space $\mathbb{L}^{2}(\Omega)$. Unfortunately, the literature on computing explicitly its eigenvalues and the corresponding eigenfunctions is not that rich and it has focused only on 'simple' domains, e.g. symmetric domains, see \cite{kalmenov2011boundary, Anderson}.  For the particular case of a ball, in $\mathbb{R}^{3}$, of radius $a$, the authors of \cite[Theorem 4.2]{kalmenov2011boundary} have given explicit expressions for the eigenvalues defined by 
\begin{equation}\label{EigValFractionalOrder}
\lambda_{l,j} = \frac{a^{2}}{\left[ \mu_{j}^{(l + \frac{1}{2})} \right]^{2}}, \quad l \geq 0 \quad \text{and} \quad j \geq 1, 
\end{equation}
where $\mu_{j}^{(l + \frac{1}{2})}$ are the roots of the transcendental equation
\begin{equation}\label{muFractionalOrder}
\left(2 \, l + 1 \right) \; \begin{LARGE}
\textbf{J}_{l + \frac{1}{2}}
\end{LARGE}\left( \mu_{j}^{(l + \frac{1}{2})} \right) + \frac{\mu_{j}^{(l + \frac{1}{2})}}{2} \; \left(     \begin{LARGE}
\textbf{J}_{l - \frac{1}{2}}
\end{LARGE}\left( \mu_{j}^{(l + \frac{1}{2})}\right)   -  \begin{LARGE}
\textbf{J}_{l + \frac{3}{2}}
\end{LARGE}\left( \mu_{j}^{(l + \frac{1}{2})} \right) \right) = 0,
\end{equation}
where $\begin{LARGE}
\textbf{J}_{\nu}
\end{LARGE}\left( \cdot \right)$, \, for $\nu \in \mathbb{R}$, refers to the Bessel function of fractional order. We recall that $\begin{LARGE}
\textbf{J}_{\nu}
\end{LARGE}\left( \cdot \right)$, admits the following representation 
\begin{equation}\label{BFFOS}
\begin{LARGE}
\textbf{J}_{\nu}
\end{LARGE}\left( x \right) = \sum_{k = 0}^{\infty} \frac{(-1)^{k}}{\Gamma(k+1) \, \Gamma(k+1+\nu)} \; \frac{x^{2k+\nu}}{2^{2k+\nu}},
\end{equation}
where $\Gamma(\cdot)$ stands for the Gamma function. The eigenfunctions corresponding to each eigenvalue $\lambda_{l,j}$ can be represented, in spherical coordinates, in the form
\begin{equation}\label{EigFctFO}
u_{l,j,m}(r, \phi, \theta) = \begin{LARGE}
\textbf{J}_{l + \frac{1}{2}}
\end{LARGE}\left(\sqrt{\lambda_{l,j}} \, r \right) \mathbb{Y}_{l}^{m}(\phi, \theta),
 \quad \text{with} \;\; \left\vert m \right\vert \leq l, 
\end{equation}
where 
\begin{equation}\label{SphericalHarmonics}
\mathbb{Y}_{l}^{m}(\phi, \theta) = \left\{
\begin{array}{rll}
  \mathbb{P}_{l}^{m}\left( \cos(\theta) \right) \; \cos(m \phi) \quad & \text{for} & \quad m=0,\cdots,l ,\\
 && \\
   \mathbb{P}_{l}^{\left\vert m \right\vert}\left( \cos(\theta) \right) \; \sin(\left\vert m \right\vert \phi) \quad & \text{for} & \quad m=-1,\cdots,-l , 
    \end{array}
\right.
\end{equation}
where $\mathbb{P}_{l}^{m}$ are the associated Legendre polynomials. It is clear, from $(\ref{muFractionalOrder})$, that $\mu_{j}^{(l + \frac{1}{2})} \sim 1$, for $l \geq 0$ and $j \geq 1$. Hence, from $(\ref{EigValFractionalOrder})$, we obtain
\begin{equation}\label{EVNB}
\lambda_{l,j} \, \sim \, a^{2}, \quad \text{for} \; l \geq 0 \; \text{and} \; j \geq 1. 
\end{equation}
For the integral of the eigenfunctions, we have 
\begin{equation*}
\int_{D} u_{l,j,m}(x) \, dx \overset{(\ref{EigFctFO})}{=} \int_{0}^{a} \, \begin{LARGE}
\textbf{J}_{l + \frac{1}{2}}
\end{LARGE}\left(\sqrt{\lambda_{l,j}} \, r \right) \,  r^{2} \, dr \, \int_{0}^{2 \, \pi} \, \int_{0}^{\pi}  \mathbb{Y}_{l}^{m}(\phi, \theta) \, \cos(\theta) \, d\theta \, d\phi.  
\end{equation*}
Observe, from $(\ref{SphericalHarmonics})$, that for $m \neq 0$
\begin{equation*}
\int_{0}^{2 \, \pi} \, \int_{0}^{\pi}  \mathbb{Y}_{l}^{m}(\phi, \theta) \, \cos(\theta) \, d\theta \, d\phi = 0. 
\end{equation*}
Then, by taking $m=0$ we obtain
\begin{equation*}
\int_{D} u_{l,j,0}(x) \, dx = 2 \, \pi \, \int_{0}^{a} \, \begin{LARGE}
\textbf{J}_{l + \frac{1}{2}}
\end{LARGE}\left(\sqrt{\lambda_{l,j}} \, r \right) \,  r^{2} \, dr \,\, \int_{0}^{\pi}  \mathbb{P}_{l}^{0}(\cos(\theta)) \, \cos(\theta) \, d\theta.  
\end{equation*}
We can check that\footnote{This can be proved using the series representation of $\mathbb{P}_{2l}^{0}(\cdot)$, given by
\begin{equation*}
\mathbb{P}_{2l}^{0}(x) = \frac{1}{4^{l}} \sum_{m=0}^{l} (-1)^{m} \frac{(4l-2m)!}{m! \, (2l-m)! \, (2l-2m)!} \, x^{2l-2m}.
\end{equation*}
} for $l$ even, we have
\begin{equation*}
\int_{0}^{\pi}  \mathbb{P}_{l}^{0}(\cos(\theta)) \, \cos(\theta) \, d\theta = 0.
\end{equation*}
Hence, by keeping only the odd index, we end up with
\begin{eqnarray*}
\int_{D} u_{2l+1,j,0}(x) \, dx &=& 2 \, \pi \, \int_{0}^{a} \, \begin{LARGE}
\textbf{J}_{2 l + \frac{3}{2}}
\end{LARGE}\left(\sqrt{\lambda_{2l+1,j}} \, r \right) \,  r^{2} \, dr \,\, \int_{0}^{\pi}  \mathbb{P}_{2l+1}^{0}(\cos(\theta)) \, \cos(\theta) \, d\theta \\ 
&\overset{(\ref{EigValFractionalOrder})}{=}& 2 \, \pi \, \left( \dfrac{\mu_{j}^{(2 l + \frac{3}{2})}}{a} \right)^{3} \, \int_{0}^{\dfrac{a^{2}}{\mu_{j}^{(2 l + \frac{3}{2})}}} \, \begin{LARGE}
\textbf{J}_{2 l + \frac{3}{2}}
\end{LARGE}\left( r \right) \,  r^{2} \, dr \,\, \int_{0}^{\pi}  \mathbb{P}_{2l+1}^{0}(\cos(\theta)) \, \cos(\theta) \, d\theta.  
\end{eqnarray*}
Using the series representation of $\mathbb{P}_{2l+1}^{0}(\cdot)$, we obtain: 
\begin{equation*}
\int_{0}^{\pi}  \mathbb{P}_{2l+1}^{0}(\cos(\theta)) \, \cos(\theta) \, d\theta = \sum_{m=0}^{l} (-1)^{m} \, \frac{\left( 4 l + 2 - 2m  \right)! \, \pi}{2^{2(2l+1-m)} \, (l-m)! \; (l+1-m)!} \; \sim \; 1.
\end{equation*}
Then, knowing that $\mu_{j}^{(2 l + \frac{3}{2})} \, \sim \, 1$, we deduce
\begin{equation}\label{ApprInt}
\int_{D} u_{2l+1,j,0}(x) \, dx  \sim  a^{-3}  \, \int_{0}^{a^{2}} \, \begin{LARGE}
\textbf{J}_{2 l + \frac{3}{2}}
\end{LARGE}\left( r \right) \,  r^{2} \, dr.  
\end{equation}
Now, using $(\ref{BFFOS})$,
\begin{equation*}
\int_{0}^{a^{2}} \, \begin{LARGE}
\textbf{J}_{2 l + \frac{3}{2}}
\end{LARGE}\left( r \right) \,  r^{2} \, dr = a^{9 + 4 l} \sum_{k = 0}^{\infty} \frac{(-1)^{k}}{\Gamma(k+1) \, \Gamma\left(k+2l+\frac{5}{2}\right)} \; \frac{1}{2^{2k+2l+\frac{3}{2}}} \; \frac{a^{4k}}{\frac{9}{2} + 2k + 2l} \; \sim \; a^{9 + 4 l}.
\end{equation*}
Hence, 
\begin{equation}\label{NU}
\int_{D} u_{2l+1,j,0}(x) \, dx  \, \sim \,  a^{6+4l}.  
\end{equation}
For the $\mathbb{L}^{2}(D)$ norm of $u_{2l+1,j,0}(\cdot)$, similarly to $(\ref{ApprInt})$, we can derive 
\begin{equation}\label{DU}
\left\Vert u_{2l+1,j,0} \right\Vert^{2}_{\mathbb{L}^{2}(D)} \; \sim \; a^{-3} \, \, \int_{0}^{a^{2}} \left\vert \begin{LARGE}
\textbf{J}_{2 l + \frac{3}{2}}
\end{LARGE}\left( r \right) \right\vert^{2} \,  r^{2} \, dr \; \overset{(\ref{BFFOS})}{\sim} \; a^{9 + 8 l}.  
\end{equation}
By setting $v_{2l+1,j,0} := \dfrac{u_{2l+1,j,0}}{\left\Vert u_{2l+1,j,0} \right\Vert_{\mathbb{L}^{2}(D)}}$ to be the normalized eigenfunctions and combining $(\ref{NU})$ with $(\ref{DU})$, we end up with the following estimation
\begin{equation}\label{EFNB}
\int_{D} v_{2l+1,j,0}(x) \, dx \; \sim \; a^{\frac{3}{2}}.
\end{equation} 
Finally, in the case of ball of radius $a$, we deduce from  $(\ref{EVNB})$ and $(\ref{EFNB})$, the following behaviour 
\begin{equation}\label{3DEigS}
\lambda_{l,j} \, \sim \, a^{2}  \quad \text{and} \quad \int_{D} v_{2l+1,j,0}(x) \, dx \; \sim \; a^{\frac{3}{2}} \quad \text{for } l \geq 0 \;\, \text{and} \;\, j \geq 1. 
\end{equation}
Deriving analogous formula for $(\ref{3DEigS})$, in the case of two dimension space is more complicated. The reason for this is the additive logarithmic term appearing after scaling the fundamental solution in two dimension. More precisely, from $(\ref{FundamentalSolution})$, we see that 
\begin{equation}\label{KernelScale}
\phi_{0}(x,y) = a^{-1} \, \phi_{0}\left( \tilde{x},\tilde{y} \right), \;\; \text{in 3D}, \;\; \text{and} \;\; \phi_{0}(x,y) = \frac{-1}{2 \, \pi} \log(a) + \phi_{0}\left( \tilde{x},\tilde{y} \right), \;\; \text{in 2D},
\end{equation}
where $x = z + a \, \tilde{x}$ and $y = z + a \, \tilde{y}$. The goal of this work is to analyse in detail, for an arbitrary domain $\Omega$ the scale of the eigenvalues and the integral of their corresponding eigenfunctions for the two dimensional Newtonian potential operator, that is also called in the literature the Logarithmic potential operator. As done previously, we start with the simplest case of a disc. For the case of a disc of radius $a$, we recall from \cite[Theorem 4.1]{kalmenov2011boundary} the explicit expressions of the eigenvalues 
\begin{equation*}
\lambda_{k,j} = a^{2} \, \left( \mu_{j}^{(k)} \right)^{-2}, \qquad k=0,1,2,\cdots \quad \text{and} \quad j=1,2,\cdots 
\end{equation*} 
and the corresponding eigenfunctions are given by
\begin{equation*}
u_{k,j}(r,\phi) = \begin{LARGE}
\textbf{J}_{k}
\end{LARGE}\left(\mu_{j}^{(k)} \, \frac{r}{a}\right) \, e^{i k \phi}, \qquad r \in [0,a] \quad \text{and} \quad \phi \in [0,2\pi],
\end{equation*} 
where $\begin{LARGE}
\textbf{J}_{k}
\end{LARGE}$ is the Bessel function of the first kind of order $k$ and $\mu_{j}^{(k)}$ are the roots of the following transcendental equation
\begin{equation*}
k \, \begin{LARGE}
\textbf{J}_{k}
\end{LARGE}\left( \mu_{j}^{(k)} \right) + \frac{\mu_{j}^{(k)}}{2} \, \left( \begin{LARGE}
\textbf{J}_{k-1}
\end{LARGE}\left( \mu_{j}^{(k)} \right) - \begin{LARGE}
\textbf{J}_{k+1}
\end{LARGE}\left( \mu_{j}^{(k)} \right) \right) = 0, \qquad k=1,2,\cdots,
\end{equation*}
and, for $k=0$, 
\begin{equation*}
\begin{LARGE}
\textbf{J}_{0}
\end{LARGE}
\left( \mu_{j}^{(0)} \right) + 2 \, \log(a) \, \mu_{j}^{(0)} \, \begin{LARGE}
\textbf{J}_{1}
\end{LARGE}
\left( \mu_{j}^{(0)} \right) = 0.
\end{equation*}
\medskip
\newline
Writing such explicit formulas for the eigenvalues and the eigenfunctions for an arbitrary domain $\Omega$ is out of reach. To overcome this difficulty, we propose a two-steps method allowing us to get the scale of the eigenvalues and the integral of eigenfunctions of the Newtonian potential operator defined over an arbitrary domain $\Omega$. First, we estimate the scale of the eigenvalues and the integral of eigenfunctions of Newtonian operator defined over a disc of radius $a$. Afterwards, the idea is to encircle the  domain $\Omega$, from inside and outside, between two discs, that we denote in the sequel by $D_{1}$ and $D_{2}$, with radius of each of them proportional to $a$ and then we make use of the \textit{property domain monotonicity} of the eigenvalues of the Newtonian potential operator that we prove in Section \ref{SectionII}, to derive the scale of the eigenvalues and eigenfunctions of the Newtonian potential operator. To accomplish this, the coming assumption, regarding the shape domain $\Omega$, is needed to derive the \textit{ property of domain monotonicity} for the Newtonian potential operator.
\begin{Hypotheses}\label{Hyp}
The domains $ \Omega $ are taken to be Lipschitz-regular domain of $\mathbb{R}^2$ and satisfy the following property 
\begin{equation}\label{EH1}
D(z_{1};\rho_{1}) := D_{1} \subset \Omega \subset D_{2} := D(z_{2};\rho_{2}),
\end{equation}
where, for $j=1,2$, we have 
\begin{equation}\label{EH2}
z_{j} \in \Omega \;\; \text{and} \;\; \rho_{j} \; \sim \; a,
\end{equation}
and $D_{j}$ is a disc.
\end{Hypotheses}
The previous hypotheses suggest that
\begin{equation*}
\left\vert \Omega \right\vert \;\; \sim \;\; \pi \, \rho_{j}^{2} \;\; \sim \;\; a^{2}.
\end{equation*}

\bigskip
\subsection{Statement of the results}
In the following theorem, we state the scales of the eigenvalues and the integral of eigenfunctions of the Newtonian potential operator, defined by $(\ref{defN})$, for an arbitrary domain $\Omega$ satisfying \textbf{Hypotheses \ref{Hyp}}.

\begin{theorem}\label{PrincipleResult}
Assume that $\Omega \subset \mathbb{R}^{2}$ satisfies \textbf{Hypotheses \ref{Hyp}}. Let\footnote{To write short formula we omit to mark the dependency of the eigenfunctions with respect to the domain $\Omega$.} $\left( \lambda_{n}(\Omega); e_{n} \right)_{n \geq 0}$ be the eigen-system associated to the 2D-Newtonian potential operator $N_{\Omega}(\cdot)$, defined by $(\ref{defN})$, then we have the following behavior
\begin{enumerate}
\item[]
\item For $n=0$, we have 
\begin{equation}\label{FirstEigValOmega}
\lambda_{0}\left( \Omega \right) \sim a^{2} \, \left\vert \log(a) \right\vert \quad \text{and} \quad \int_{\Omega} e_{0}(x) \, dx \sim a.
\end{equation}
\item[]
\item For $n \geq 1$, we have 
\begin{equation*}
\lambda_{n}\left( \Omega \right) \sim a^{2}  \quad \text{and} \quad \vert \int_{\Omega} e_{n}(x) \, dx \vert \lesssim a \, \left\vert \log(a) \right\vert^{- 1}.
\end{equation*}
\end{enumerate}
In the particular case when $\Omega$ is a disc of radius $a$, i.e. $\Omega = D$, we have the following behavior 
\begin{enumerate}
\item[]
\item For $n=0$, we have\footnote{Because the eigenvalues of the Newtonian operator are decreasing and as we see that $\lambda_{0,1}$ is the highest one, we refer to $\lambda_{0,1}$ to be the first eigenvalue.} 
\begin{equation}\label{FirstEigValDisc}
\lambda_{0,1}\left( D \right) \sim a^{2} \, \left\vert \log(a) \right\vert \quad \text{and} \quad \int_{D} e_{0,1}(x) \, dx \sim a.
\end{equation}
\begin{equation*}
\qquad \qquad \quad \lambda_{0,j}\left( D \right) \sim a^{2} \,  \quad \text{and} \quad \int_{D} e_{0,j}(x) \, dx \sim  \; a \; \left\vert \log(a) \right\vert^{-1}, \quad j \geq 2.
\end{equation*}
\item[]
\item For $n \geq 1$, we have 
\begin{equation}\label{VanishingInt}
\lambda_{n,j}\left( D \right) \sim a^{2}  \quad \text{and} \quad \int_{D} e_{n,j}(x) \, dx = 0, \quad j \geq 1.
\end{equation}
\end{enumerate}
\end{theorem}
\bigskip
%\newline
%\section{Application in photo-acoustic imaging}
%From the previous theorem, regarding the scale of the first eigenvalue of the Newtonian potential operator and the non vanishing character of the integral of their corresponding eigenfunctions, see for instance $(\ref{FirstEigValOmega})$ and $(\ref{FirstEigValDisc})$, we suggest the following corollary.
%\begin{corollary}\label{CoroI}
%\end{corollary}
%\begin{proof}
%\end{proof}
We have seen in Subsection \ref{SubsectionI}, how the eigenvalues and the integral of their corresponding eigenfunctions scales with respect to the radius of the used ball, that we have denoted by $'a'$, see for instance $(\ref{3DEigS})$. Based on the previous theorem, the goal of the coming proposition is to generalize the obtained scales, i.e. $(\ref{3DEigS})$, when the 3D-Newtonian potential operator is defined over an arbitrary shape domain $\Omega \subset \mathbb{R}^{3}$. 
\begin{proposition}\label{PropoII}
Assume that $\Omega \subset \mathbb{R}^{3}$ such that $\left\vert \Omega \right\vert \sim a^{3}$. Let $\left( \lambda_{n}(\Omega); e_{n} \right)_{n \geq 0}$ the eigen-system associated to the 3D-Newtonian potential operator $N_{\Omega}(\cdot)$, defined by $(\ref{defN})$. Then, for $n \geq 0$, we have the following behaviour
\begin{equation*}
\lambda_{n}(\Omega) \, \sim \, a^{2} \;\; \text{and} \;\; \left\vert \int_{\Omega} e_{n}(x) \, dx \right\vert \, \lesssim \, a^{\frac{3}{2}}.
\end{equation*}
\end{proposition}
\begin{proof}{of \textbf{Proposition \ref{PropoII}}}
\newline
By definition we have 
\begin{equation}\label{OmegaEigSys}
N_{\Omega}\left( e_{n} \right) = \lambda_{n}(\Omega) \; e_{n},
\end{equation}
and, thanks to $(\ref{KernelScale})$, we get after scaling from $\Omega$ to $\Omega^{\star}$,  
\begin{equation*}
a^{2} \;  N_{\Omega^{\star}}\left( \tilde{e}_{n} \right) = \lambda_{n}(\Omega) \; \tilde{e}_{n},
\end{equation*}
where $\Omega^{\star}$ is such that $\left\vert \Omega^{\star} \right\vert \, \sim \, 1$ and $\Omega = z + a \, \Omega^{\star}$. We see that $\tilde{e}_{n}$ are eigenfunctions of $N_{\Omega^{\star}}(\cdot)$, in the domain $\Omega^{\star}$, related to the eigenvalues $\lambda_{n}(\Omega^{\star}) = a^{-2} \, \lambda_{n}(\Omega)$, where obviously $\lambda_{n}(\Omega^{\star}) \, \sim \, 1$, with respect to the parameter $a$. This indicate, 
\begin{equation}\label{EstEigOmega3D}
\lambda_{n}(\Omega) \, \sim \, a^{2}, \qquad \forall \, n \geq 0.
\end{equation}
Furthermore, by integrating both sides of  $(\ref{OmegaEigSys})$ and taking the modulus in both sides, we obtain  
\begin{equation*}
\lambda_{n}(\Omega) \; \left\vert \int_{\Omega} e_{n}(x) \, dx \right\vert  = \left\vert \int_{\Omega} N_{\Omega}\left( e_{n} \right)(x) \, dx \right\vert \leq \left\Vert 1 \right\Vert_{\mathbb{L}^{2}(\Omega)} \; \left\Vert N_{\Omega}(\cdot) \right\Vert_{\mathcal{L}\left(\mathbb{L}^{2}(\Omega) ; \mathbb{L}^{2}(\Omega) \right)} \; \left\Vert e_{n} \right\Vert_{\mathbb{L}^{2}(\Omega)}.
\end{equation*}
It is know from the spectral theory that $\left\Vert N_{\Omega}(\cdot) \right\Vert_{\mathcal{L}\left(\mathbb{L}^{2}(\Omega) ; \mathbb{L}^{2}(\Omega) \right)} \leq \underset{n}{Sup}\left(\lambda_{n}(\Omega) \right)=\lambda_{0}(\Omega)$, where the last equality is a consequence of the fact that the sequence of eigenvalues is decreasing. In addition, we know that the sequence $\left\{ e_{n}(\cdot) \right\}_{n \geq 0}$ is orthonormalized in $\mathbb{L}^{2}(\Omega)$. Then, 
\begin{equation*}
 \left\vert \int_{\Omega} e_{n}(x) \, dx \right\vert   \leq \left\vert \Omega \right\vert^{\frac{1}{2}} \; \frac{\lambda_{0}(\Omega)}{\lambda_{n}(\Omega)} \; \overset{(\ref{EstEigOmega3D})}{\sim} \; \left\vert \Omega \right\vert^{\frac{1}{2}} \; \sim  \; a^{\frac{3}{2}}.
\end{equation*}
This concludes the proof of Proposition \ref{PropoII}.
\end{proof}
\begin{remark}
As we can see in the proof of Proposition \ref{PropoII}, the \textit{property of domain monotonicity} for the Newtonian potential operator is no longer used. Consequently, for the 3D-Newtonian potential operator, the  \textbf{Hypotheses \ref{Hyp}} is no longer needed. 
\end{remark}

\bigskip

%The paper contains two section and it is organised as follows. Section 
%\ref{SectionII} is devoted to  give a detailed proof of Theorem \ref{PrincipleResult}. Section \ref{BJI} is intended to correct an error that have been discovered in our work  \cite[Appendix A]{ghandriche2022mathematical}.

%\bigskip

\section{Proof of Theorem \ref{PrincipleResult}}\label{SectionII}
We split the proof into two steps, in the first one we justify the result in the case of a disc and in the second step we prove the result for a general shape satisfying \textbf{Hypotheses \ref{Hyp}}. 
\begin{enumerate}
\item[]
\item The case of a disc $D$ of radius $a$. 
\medskip
\newline
We recall, from \cite[Theorem 4.1]{kalmenov2011boundary}, that the eigenvalues  of the logarithmic potential operator for a disc are given by: 
\begin{equation}\label{EigValDef}
\lambda_{k,j} = a^{2} \, \left( \mu_{j}^{(k)} \right)^{-2}, \qquad k=0,1,2,\cdots \quad \text{and} \quad j=1,2,\cdots 
\end{equation} 
and the corresponding eigenfunctions given by
\begin{equation}\label{Eigfcts}
u_{k,j}(r,\phi) = \begin{LARGE}
\textbf{J}_{k}
\end{LARGE}\left(\mu_{j}^{(k)} \, \frac{r}{a}\right) \, e^{i k \phi}, \qquad r \in [0,a] \quad \text{and} \quad \phi \in [0,2\pi],
\end{equation} 
where $\begin{LARGE}
\textbf{J}_{k}
\end{LARGE}$ is the Bessel function of the first kind of order $k$ and $\mu_{j}^{(k)}$ are the roots of the following transcendental equation
\begin{equation}\label{EigValk>1}
k \, \begin{LARGE}
\textbf{J}_{k}
\end{LARGE}\left( \mu_{j}^{(k)} \right) + \frac{\mu_{j}^{(k)}}{2} \, \left( \begin{LARGE}
\textbf{J}_{k-1}
\end{LARGE}\left( \mu_{j}^{(k)} \right) - \begin{LARGE}
\textbf{J}_{k+1}
\end{LARGE}\left( \mu_{j}^{(k)} \right) \right) = 0, \qquad k=1,2,\cdots,
\end{equation}
and, for $k=0$, 
\begin{equation}\label{EquaBessel}
\begin{LARGE}
\textbf{J}_{0}
\end{LARGE}
\left( \mu_{j}^{(0)} \right) + 2 \, \log(a) \, \mu_{j}^{(0)} \, \begin{LARGE}
\textbf{J}_{1}
\end{LARGE}
\left( \mu_{j}^{(0)} \right) = 0.
\end{equation}
In \cite[Appendix IV, Table III]{carslaw1947conduction}, for some specific values of the parameter $a$, the authors  computed the first six roots of the transcendental equation $(\ref{EquaBessel})$. It is clear, from $(\ref{EquaBessel})$, that the solution $\left\{ \mu_{j}^{(0)} \right\}_{j \geq 1}$ will be dependent on the parameter $a$. To see closely how the solutions depends on $a$, we start   by plotting the graph\footnote{These graphs were produced with Mathematica.} associated to the function, with parameter $a$, defined by 
\begin{equation}\label{DefPsi}
\Psi_{a}(x) := \begin{LARGE}
\textbf{J}_{0}
\end{LARGE}
\left( x \right) + 2 \, \log(a) \, x \, \begin{LARGE}
\textbf{J}_{1}
\end{LARGE}
\left( x \right).
\end{equation}
% Figure environment removed 
From the graphs, we can see clearly that the first root, i.e $\mu_{1}^{(0)}$, is small and the other roots $\mu_{j}^{(0)}$, for $j \geq 1$, are moderate. To determine the order of smallness of $\mu_{1}^{(0)}$, we use the asymptotic behavior of $\begin{LARGE}
\textbf{J}_{0}
\end{LARGE}$  and $\begin{LARGE}
\textbf{J}_{1}
\end{LARGE}$ for small argument. More precisely, for $0 < x \ll \sqrt{n+1}$, see \cite[Equation 25]{LANDAU}, we have
\begin{equation}\label{BesselNearZero}
\begin{LARGE}
\textbf{J}_{n}
\end{LARGE}\left( x \right) \sim \frac{1}{n!} \, \left( \frac{x}{2} \right)^{n}. 
\end{equation}
Now, using $(\ref{BesselNearZero})$ we approximate $(\ref{EquaBessel})$ as   
\begin{equation*}
1 +  \log(a) \, \left( \mu_{1}^{(0)} \right)^{2}=0,
\end{equation*}
and this implies that 
\begin{equation*}
\mu_{1}^{(0)} \; \sim \; \frac{1}{\sqrt{\left\vert \log(a) \right\vert}}.
\end{equation*}
Consequently, 
\begin{equation*}
\lambda_{0,1} \; \sim \; a^{2} \, \left\vert \log(a) \right\vert.
\end{equation*}
To study the other roots, we start by
setting $\alpha_{k,j}$, for $k=0,1$ and $j \in \mathbb{N}$, to be the root of order $j$ associated to the Bessel function  $\begin{LARGE}
\textbf{J}_{k}
\end{LARGE}$. Then, thanks to Dixon's theorem, see \cite{watson}, we  know that $\mu_{j}^{(0)}$ will be interlaced between the roots of $\begin{LARGE}
\textbf{J}_{0}
\end{LARGE}$  and the roots of $\begin{LARGE}
\textbf{J}_{1}
\end{LARGE}$. More precisely, 
\begin{equation}\label{DixonResults}
\mu_{1}^{(0)} < \alpha_{0,1} < \alpha_{1,1} \quad \text{and} \quad \alpha_{1,j-1} < \mu_{j}^{(0)} < \alpha_{0,j}, \quad j \geq 2. 
\end{equation}
Using the previous relation and knowing that 
 $ \alpha_{1,1} = 3.8317$, we deduce that $\mu_{j}^{(0)} > 3.8317$, for $j \geq 2$. Therefore, for $j \geq 2$, we have $\mu_{j}^{(0)} \; \sim \; 1$. Hence, from $(\ref{EigValDef})$, we obtain
\begin{equation*}
\lambda_{0,j} \; \sim \; a^{2}, \quad \text{for} \quad j \geq 2.
\end{equation*}
Next, contrary to $(\ref{EquaBessel})$, because the equation $(\ref{EigValk>1})$ is independent on the parameter $a$, we deduce that $\mu_{j}^{(k)} \sim 1$. Hence, the eigenvalues $\lambda_{k,j}$ defined by $(\ref{EigValDef})$ behave as  
\begin{equation*}
\lambda_{k,j} \; \sim \; a^{2}, \quad \text{for} \quad k \geq 1 \quad \text{and} \quad j \geq 1. 
\end{equation*} 
Arranging the obtained results we get: 
\begin{equation}\label{BehaviourEigVal}
\lambda_{0,1} \; \sim \; a^{2} \, \left\vert \log(a) \right\vert  \quad \text{and} \quad \lambda_{k,j} \; \sim \; a^{2}, \quad \text{for} \quad (k,j) \neq (0,1).
\end{equation}
\medskip
\newline
In what follows, when we talk about the first eigenvalue of the Newtonian potential operator in the disc we refer to $\lambda_{0,1}$. This is because the eigenvalues of the Newtonian operator are decreasing and, as proved by $(\ref{BehaviourEigVal})$, $\lambda_{0,1}$ is the highest one.
\medskip
\newline
Similarly to $(\ref{BehaviourEigVal})$, we derive an analogous  result for the integrals of the associated  eigenfunctions $u_{k,j}(\cdot,\cdot)$, defined by $(\ref{Eigfcts})$. We start with the coming computations. 
\begin{eqnarray*}
\int_{D} u_{k,j}(x) \, dx = \int_{0}^{2\pi} \int_{0}^{a}  u_{k,j}(r,\phi) \, r \ dr \, d\phi &=& \int_{0}^{2\pi} \int_{0}^{a}  \begin{LARGE}
\textbf{J}_{k}
\end{LARGE}\left(\mu_{j}^{(k)} \, \frac{r}{a}\right) \, e^{i k \phi} \, r \ dr \, d\phi \\
&=& 2 \, \pi \, \int_{0}^{a}  \begin{LARGE}
\textbf{J}_{0}
\end{LARGE}\left(\mu_{j}^{(0)} \, \frac{r}{a}\right)  \, r \ dr \,\, \delta_{0,k} \\
&=& 2 \, \pi \, \left( \frac{a}{\mu_{j}^{(0)}} \right)^{2} \, \int_{0}^{\mu_{j}^{(0)}}  \begin{LARGE}
\textbf{J}_{0}
\end{LARGE}\left( r \right)  \, r \ dr \,\, \delta_{0,k},
\end{eqnarray*}
where $\delta_{\cdot,\cdot}$ is the Kronecker-symbol. Thanks to \cite[Formula (5), page 206]{carslaw1947conduction}, we know that
\begin{equation*}
x \, \begin{LARGE}
\textbf{J}_{1}
\end{LARGE}\left( x \right) = \int_{0}^{x} r \, \begin{LARGE}
\textbf{J}_{0}
\end{LARGE}\left( r \right) \, dr.
\end{equation*}
Then, 
\begin{equation}\label{IntEigFcts}
\int_{D} u_{k,j}(x) \, dx = 2 \, \pi \,  \frac{a^{2}}{\mu_{j}^{(0)}}  \,  \begin{LARGE}
\textbf{J}_{1}
\end{LARGE}\left( \mu_{j}^{(0)} \right)  \, \delta_{0,k}.
\end{equation}
Later, to define the normalized eigenfunctions, we compute the $\left\Vert u_{0,j} \right\Vert_{\mathbb{L}^{2}(D)}$. We have, 
\begin{eqnarray*}
\left\Vert u_{0,j} \right\Vert^{2}_{\mathbb{L}^{2}(D)}  :=  \int_{D} \left\vert u_{0,j} \right\vert^{2}(x) \, dx = \int_{0}^{2 \, \pi} \, \int_{0}^{a} \left\vert u_{0,j} \right\vert^{2}(r, \phi) \, r \, dr \, d\phi & = & 2 \, \pi \, \int_{0}^{a} \left\vert \begin{LARGE}
\textbf{J}_{0}
\end{LARGE}\left(\mu_{j}^{(0)} \, \frac{r}{a}\right) \right\vert^{2} \, r \, dr \\
&=& 2 \, \pi \, \left( \frac{a}{\mu_{j}^{(0)}} \right)^{2} \, \int_{0}^{\mu_{j}^{(0)}} \left\vert \begin{LARGE}
\textbf{J}_{0}
\end{LARGE}\left(r \right) \right\vert^{2} \, r \, dr.
\end{eqnarray*}
Combining the previous formula with $(\ref{IntEigFcts})$, gives us a formula for the integral of the normalized eigenfunctions. More precisely, 
\begin{equation}\label{IntNEigFcts}
\int_{D} v_{0,j}(x) \, dx := \frac{\int_{D} u_{0,j}(x) \, dx}{\left\Vert u_{0,j} \right\Vert_{\mathbb{L}^{2}(D)}} = \dfrac{\sqrt{2 \, \pi} \; a \; \begin{LARGE}
\textbf{J}_{1}
\end{LARGE}\left( \mu_{j}^{(0)} \right)}{\left[\int_{0}^{\mu_{j}^{(0)}} \left\vert \begin{LARGE}
\textbf{J}_{0}
\end{LARGE}\left(r \right) \right\vert^{2} \, r \, dr\right]^{\frac{1}{2}}}.
\end{equation} 
As a result of the different behaviors of $\left\{ \mu_{j}^{(0)} \right\}_{j \geq 1}$, with respect to the parameter $a$, to estimate $(\ref{IntNEigFcts})$ we split our computations into two steps.
\begin{enumerate}
\item[]
\item For $j=1$, as $\mu_{1}^{(0)}$ is small, we have  
\begin{equation*}
\int_{D} v_{0,1}(x) \, dx  = \frac{\sqrt{2 \, \pi} \; a \; \begin{LARGE}
\textbf{J}_{1}
\end{LARGE}\left( \mu_{1}^{(0)} \right)}{\left[\int_{0}^{\mu_{1}^{(0)}} \left\vert \begin{LARGE}
\textbf{J}_{0}
\end{LARGE}\left(r \right) \right\vert^{2} \, r \, dr\right]^{\frac{1}{2}}} \overset{(\ref{BesselNearZero})}{\sim}  \frac{\sqrt{2 \, \pi} \; a \; \dfrac{\mu_{1}^{(0)}}{2}}{\left[\int_{0}^{\mu_{1}^{(0)}} r \, dr\right]^{\frac{1}{2}}} = \sqrt{\pi} \, a \; \sim \; a .
\end{equation*} 
\item[]
\item For $j \geq 2$, as $\mu_{j}^{(0)}$ is moderate, we have 
\begin{equation*}
\int_{D} v_{0,j}(x) \, dx  = \dfrac{\sqrt{2 \, \pi} \; a \; \begin{LARGE}
\textbf{J}_{1}
\end{LARGE}\left( \mu_{j}^{(0)} \right)}{\left[\int_{0}^{\mu_{j}^{(0)}} \left\vert \begin{LARGE}
\textbf{J}_{0}
\end{LARGE}\left(r \right) \right\vert^{2} \, r \, dr\right]^{\frac{1}{2}}}.
\end{equation*}
By induction on the formula $(\ref{DixonResults})$, we obtain for $j \geq 2$ the following relation 
\begin{equation*}
\alpha_{0,j-1} < \mu_{j}^{(0)} < \alpha_{0,j}.
\end{equation*}
Then, 
\begin{equation*}
\dfrac{\sqrt{2 \, \pi} \; a \; \left\vert \begin{LARGE}
\textbf{J}_{1}
\end{LARGE}\left( \mu_{j}^{(0)} \right) \right\vert}{\left[\int_{0}^{\alpha_{0,j}} \left\vert \begin{LARGE}
\textbf{J}_{0}
\end{LARGE}\left(r \right) \right\vert^{2} \, r \, dr\right]^{\frac{1}{2}}} < \left\vert \int_{D} v_{0,j}(x) \, dx \right\vert  < \dfrac{\sqrt{2 \, \pi} \; a \; \left\vert \begin{LARGE}
\textbf{J}_{1}
\end{LARGE}\left( \mu_{j}^{(0)} \right) \right\vert}{\left[\int_{0}^{\alpha_{0,j-1}} \left\vert \begin{LARGE}
\textbf{J}_{0}
\end{LARGE}\left(r \right) \right\vert^{2} \, r \, dr\right]^{\frac{1}{2}}},
\end{equation*}
or, equivalently, 
\begin{equation*}
\dfrac{\sqrt{2 \, \pi} \; a \; \left\vert \begin{LARGE}
\textbf{J}_{1}
\end{LARGE}\left( \mu_{j}^{(0)} \right) \right\vert}{\alpha_{0,j} \, \left[\int_{0}^{1} \left\vert \begin{LARGE}
\textbf{J}_{0}
\end{LARGE}\left(\alpha_{0,j} \, r \right) \right\vert^{2} \, r \, dr\right]^{\frac{1}{2}}} < \left\vert \int_{D} v_{0,j}(x) \, dx \right\vert  < \dfrac{\sqrt{2 \, \pi} \; a \; \left\vert \begin{LARGE}
\textbf{J}_{1}
\end{LARGE}\left( \mu_{j}^{(0)} \right) \right\vert}{\alpha_{0,j-1} \, \left[\int_{0}^{1} \left\vert \begin{LARGE}
\textbf{J}_{0}
\end{LARGE}\left(\alpha_{0,j-1} \, r \right) \right\vert^{2} \, r \, dr\right]^{\frac{1}{2}}}.
\end{equation*}
Now, thanks to \cite[Formula (2), page 205]{carslaw1947conduction}, we know that
\begin{equation*}
\int_{0}^{1} \left( \begin{LARGE}
\textbf{J}_{0}
\end{LARGE}\left(\alpha_{0,k} \, r \right) \right)^{2} \, r \, dr = \frac{1}{2} \, \left( \begin{LARGE}
\textbf{J}_{1}
\end{LARGE}\left(\alpha_{0,k} \right) \right)^{2} \sim 1, \quad \text{for} \quad k \in \mathbb{N},
\end{equation*}
hence,
\begin{equation*}
\dfrac{2 \, \sqrt{ \pi} \; a \; \left\vert \begin{LARGE}
\textbf{J}_{1}
\end{LARGE}\left( \mu_{j}^{(0)} \right) \right\vert}{\alpha_{0,j} \, \left\vert \begin{LARGE}
\textbf{J}_{1}
\end{LARGE}\left(\alpha_{0,j} \right) \right\vert} < \left\vert \int_{D} v_{0,j}(x) \, dx \right\vert  < \dfrac{2 \, \sqrt{ \pi} \; a \; \left\vert \begin{LARGE}
\textbf{J}_{1}
\end{LARGE}\left( \mu_{j}^{(0)} \right) \right\vert}{\alpha_{0,j-1} \, \left\vert \begin{LARGE}
\textbf{J}_{1}
\end{LARGE}\left(\alpha_{0,j-1} \right) \right\vert}.
\end{equation*}
As $\alpha_{0,k} \sim 1$ and $\left\vert \begin{LARGE}
\textbf{J}_{1}
\end{LARGE}\left(\alpha_{0,k} \right) \right\vert \sim 1$, for $k=j-1$ or $k=j$, with respect to the parameter $a$, we deduce that
\begin{equation*}
 a \; \left\vert \begin{LARGE}
\textbf{J}_{1}
\end{LARGE}\left( \mu_{j}^{(0)} \right) \right\vert \lesssim \left\vert \int_{D} v_{0,j}(x) \, dx \right\vert \lesssim a \; \left\vert \begin{LARGE}
\textbf{J}_{1}
\end{LARGE}\left( \mu_{j}^{(0)} \right) \right\vert, 
\end{equation*}
or, equivalently, 
\begin{equation*}
\left\vert \int_{D} v_{0,j}(x) \, dx \right\vert \; \sim \; a \; \left\vert \begin{LARGE}
\textbf{J}_{1}
\end{LARGE}\left( \mu_{j}^{(0)} \right) \right\vert \overset{(\ref{EquaBessel})}{=} \frac{a \; \left\vert \begin{LARGE}
\textbf{J}_{0}
\end{LARGE}\left( \mu_{j}^{(0)} \right) \right\vert}{2 \, \mu_{j}^{(0)} \, \left\vert \log(a) \right\vert}.
\end{equation*}
The final step consist in using the fact that\footnote{This can be proved using the integral representation of the Bessel function 
\begin{equation*}
\begin{LARGE}
\textbf{J}_{0}
\end{LARGE}\left( x \right) := \frac{1}{\pi} \, \int_{0}^{\pi} \cos\left( x \, \sin(\tau)\right) \, d\tau, 
\end{equation*}
see \cite[Formula 9.19, page 230]{Temme}.}  $\left\vert \begin{LARGE}
\textbf{J}_{0}
\end{LARGE}\left( x \right) \right\vert \leq 1, \, \forall \, x \in \mathbb{R}$, see \cite[Formula (5), page 31]{watson}. Hence,  
\begin{equation*}
\left\vert \int_{D} v_{0,j}(x) \, dx \right\vert \; \sim \; a \;  \left\vert \log(a) \right\vert^{-1}.
\end{equation*}
\end{enumerate} 
Analogously to $(\ref{BehaviourEigVal})$, after arranging the obtained results we deduce that: 
\begin{equation}\label{IranProtests}
\int_{D} v_{0,1}(x) \, dx \;\; \sim \;\; a  \quad \text{and} \quad \int_{D} v_{0,j}(x) \, dx \;\; \sim \;\; a \, \left\vert \log(a) \right\vert^{-1}, \quad j \geq 2.
\end{equation}
Obviously, 
\begin{equation*}
\int_{D} v_{k,j}(x) \, dx = 0, \quad k \geq 1 \quad \text{and}  \quad j \geq 1.
\end{equation*}
\item[]
\item The case of an arbitrary shape $\Omega$.\medskip
\newline
To estimate the behavior of the eigenvalues of the Newtonian potential operator defined over an arbitrary domain $\Omega$, we proceed in two steps. 
\begin{enumerate}
\item[]
\item \label{PDM}
 We start by proving the \textit{ property of domain monotonicity} for the Newtonian potential operator: eigenvalues of the Newtonian potential operator monotonously increase when the domain is enlarged, i.e., $\lambda_{n}\left( \Omega_{1} \right) \leq \lambda_{n}\left( \Omega_{2} \right)$ if $ \Omega_{1} \subset \Omega_{2}$. To accomplish this, we define an extension operator $\bm{P}$, as follows: 
\smallskip
\newline
\begin{eqnarray}\label{Extension}
\nonumber
\bm{P} := \mathbb{L}^{2}\left( \Omega_{1} \right) & \longrightarrow & \mathbb{L}^{2}\left( \Omega_{2} \right) \\
f(\cdot) & \longrightarrow & \bm{P}\left( f \right)(\cdot) := \tilde{f}(\cdot) := f(\cdot) \; \underset{\Omega_{1}}{\chi}(\cdot) \; + \; 0 \; \underset{\Omega_{2} \setminus \Omega_{1}}{\chi}(\cdot),
\end{eqnarray}  
where we have assumed that $\Omega_{1} \subset \Omega_{2}$. Manifestly, we have the following injection 
\begin{equation}\label{Injection}
\mathbb{L}^{2}\left( \Omega_{1} \right) \hookrightarrow \mathbb{L}^{2}\left( \Omega_{2} \right).
\end{equation}
In similar manner we define the extension of a function represented by the Newtonian operator 
%\bigskip
%\newline
%In the sequel we assume that $\Omega \subset B_{1} := B(z, \rho_{1})$, where $B_{1}$ has a disc shape with center  $z \in \mathbb{R}^{2}$ and radius $\rho_{1}$. In addition we assume the existence of a constant $\rho_{1}^{\star}$, depending only on the geometry of $\Omega$ and independent on the parameter $a$, 
%such that $\left\vert \Omega \right\vert = \rho_{1}^{\star} \, \left\vert B_{1} \right\vert = \mathcal{O}\left( a^{2} \right)$.
%\medskip
%\newline
%From Surugan??, we know that $u(\cdot)$, solution of ??, will be represented like a Newtonian operator with some density $f \in \mathbb{L}^{2}(\Omega)$, i.e. 
\begin{equation*}
u(x) = \int_{\Omega_{1}} \Phi_{0}(x,y) \, f(y) \, dy, \quad x \in \Omega_{1},   
\end{equation*} 
to the domain $\Omega_{2}$,  as follows
\begin{equation*}
\tilde{u}(x) = \int_{\Omega_{2}} \Phi_{0}(x,y) \, \tilde{f}(y) \, dy, \quad x \in \Omega_{2},   
\end{equation*}
where $\tilde{f}(\cdot)$ is the extension of $  f(\cdot)$, with zero in $\Omega_{2} \setminus \Omega_{1}$, defined by $(\ref{Extension})$.
%We set also, 
%\begin{eqnarray*}
%\mathcal{D}\left( \textbf{A}_{\Omega} \right) & := & \left\{u \in \mathbb{L}^{2}(\Omega) \quad \; \, \text{such that} \quad  N_{\Omega}\left(u\right)_{|_{\Omega}}   \in \mathbb{L}^{2}(\Omega)  \right\} \\
%\mathcal{D}\left( \textbf{A}_{B_{1}} \right) & := & \left\{u \in \mathbb{L}^{2}(B_{1}) \quad \text{such that} \quad  N_{B_{1}}\left(u\right)_{|_{B_{1}}}   \in \mathbb{L}^{2}(B_{1})  \right\},
%\end{eqnarray*}
%and as $\Omega \subset B_{1}$, we deduce that $\mathcal{D}\left( \textbf{A}_{B_{1}} \right) \subset \mathcal{D}\left( \textbf{A}_{\Omega} \right)$. 
Now, from the min-max principle applied to the Newtonian potential operator, see  \cite[Theorem 4, page 318-319]{lax2002functional}, we have\footnote{In $(\ref{min-max})$, the positivity of the operator $N_{\Omega_{1}}(\cdot)$ ensures that all eigenvalues are non-negative.} 
\begin{equation}\label{min-max}
\lambda_{k}\left( \Omega_{1} \right) = \underset{\Xi_{k}}{Sup} \quad \underset{u \in \left( \Xi_{k} \cap  \mathbb{L}^{2}(\Omega_{1}) \right)}{Inf} \quad \dfrac{\int_{\Omega_{1}} N_{\Omega_{1}}\left(u\right)(x) u(x) \; dx}{\int_{\Omega_{1}} \left\vert u \right\vert^{2}(x) \; dx},
\end{equation}
where $\Xi_{k} \subset \mathbb{L}^{2}(\mathbb{R}^{2})$ is $k$ dimensional subspace.  Now, using the injection $(\ref{Injection})$ we deduce that:
\begin{equation}\label{Black}
\lambda_{k}\left( \Omega_{1} \right) \leq  \underset{\Xi_{k}}{Sup} \quad \underset{u \in \left( \Xi_{k} \cap   \mathbb{L}^{2}(\Omega_{2}) \right)}{Inf} \quad \dfrac{\int_{\Omega_{1}} N_{\Omega_{1}}\left(u\right)(x) u(x) \; dx}{\int_{\Omega_{1}} \left\vert u \right\vert^{2}(x) \; dx}.
\end{equation}
Exploiting the properties of the extension function, defined by $(\ref{Extension})$, we obtain 
\begin{equation*}
\int_{\Omega_{1}} N_{\Omega_{1}}\left(u\right)(x) u(x) \; dx = \int_{\Omega_{2}} N_{\Omega_{2}}\left(\tilde{u}\right)(x) \tilde{u}(x) \; dx \quad \text{and} \quad \int_{\Omega_{1}} \left\vert u \right\vert^{2}(x) \; dx = \int_{\Omega_{2}} \left\vert \tilde{u} \right\vert^{2}(x) \; dx.
\end{equation*}
Hence, $(\ref{Black})$ becomes, 
\begin{equation*}
\lambda_{k}\left( \Omega_{1} \right) \leq  \underset{\Xi_{k}}{Sup} \quad \underset{\tilde{u} \in \left( \Xi_{k} \cap \mathbb{L}^{2}(\Omega_{2}) \right)}{Inf} \quad \dfrac{\int_{\Omega_{2}} N_{\Omega_{2}}\left(\tilde{u}\right)(x) \tilde{u}(x) \; dx}{\int_{\Omega_{2}} \left\vert \tilde{u} \right\vert^{2}(x) \; dx} = \lambda_{k}\left( \Omega_{2} \right).
\end{equation*}
Finally, 
\begin{equation}\label{Monotonicity}
\text{if} \quad \Omega_{1} \subset \Omega_{2} \quad \text{we obtain} \quad \lambda_{k}\left( \Omega_{1} \right) \leq   \lambda_{k}\left( \Omega_{2} \right),
\end{equation}
and this ends the proof of the monotonicity property for the eigenvalues of the Newtonian potential operator. 
\item[]
\item At this stage, we use the proved monotonicity property, given by $(\ref{Monotonicity})$, by assuming the existence of two discs $B_{1} := B(z_{1},\rho_{1})$ and $B_{2} := B(z_{2},\rho_{2})$, where $z_{j} \in \Omega$ and  $\rho_{j} > 0$, for $j=1,2$, such that: 
\begin{equation*}
B_{1} \subset \Omega \subset B_{2} \quad \text{and} \quad \left\vert B_{j} \right\vert \sim a^{2}, \;\; j=1,2.
\end{equation*}
Thanks to $(\ref{Monotonicity})$ we derive, from the previous formula, the following relation, 
\begin{equation}\label{B1OmegaB2}
\lambda_{k}\left( B_{1} \right) \leq \lambda_{k}\left( \Omega \right) \leq \lambda_{k}\left( B_{2} \right).
\end{equation} 
\medskip
\newline
Now, by combining $(\ref{B1OmegaB2})$ with $(\ref{BehaviourEigVal})$, we derive the following behavior of the eigenvalues of the Newtonian potential operator, defined over an arbitrary shape $\Omega$.
\begin{equation}\label{EstimationEigOmega}
\lambda_{0}\left( \Omega \right) \, \sim \, a^{2} \, \left\vert \log(a) \right\vert  \quad \text{and} \quad \lambda_{k}\left( \Omega \right) \, \sim \, a^{2}, \quad \text{for} \quad k \geq 1.
\end{equation}
\end{enumerate}
\medskip
Regarding the estimation of the integral of the eigenfunction $f_{n}$, of the Newtonian potential operator defined over $\Omega$, i.e. 
\begin{equation}\label{AddedEqua}
N_{\Omega}\left( f_{n} \right) = \lambda_{n}(\Omega) \; f_{n},\, \quad \text{in} \quad \Omega,
\end{equation}
and as suggested by the behavior of the eigenvalues $\left\{ \lambda_{n}\left( \Omega \right) \right\}_{n \geq 0}$, i.e. $(\ref{EstimationEigOmega})$, we split the study into two cases.
\begin{enumerate}
\item[]
\item[i)] \label{EstimationFirstEigFct}
 For $n = 0$, we know from $(\ref{EstimationEigOmega})$, that $\lambda_{0}(\Omega) = \beta_{0} \, a^{2}\, \left\vert \log(a) \right\vert$, where $\beta_{0}$ is a positive constant independent on the parameter $a$. Moreover, after scaling the equation $(\ref{AddedEqua})$, from $\Omega$ to $\Omega^{\star}$, and taking the inner product with respect to $\tilde{f}_{n}$, we end up with 
\begin{equation*}
\left\langle \tilde{f}_{n}; N_{\Omega^{\star}}\left( \tilde{f}_{n} \right) \right\rangle_{\mathbb{L}^{2}(\Omega^{\star})} = \frac{\lambda_{n}(\Omega)}{a^{2}} \; \left\Vert \tilde{f}_{n} \right\Vert_{\mathbb{L}^{2}(\Omega^{\star})} + \frac{1}{2 \, \pi} \, \log(a) \, \left( \int_{\Omega^{\star}} \tilde{f}_{n} \right)^{2}. 
\end{equation*}
Next, we set $\overline{f}_{n} := \dfrac{ \tilde{f}_{n}}{\left\Vert \tilde{f}_{n} \right\Vert_{\mathbb{L}^{2}\left( \Omega^{\star} \right)}}$ and $\tilde{\lambda}_{n} := \left\langle \tilde{f}_{n}; N_{\Omega^{\star}}\left( \tilde{f}_{n} \right) \right\rangle_{\mathbb{L}^{2}(\Omega^{\star})}$ we obtain from the previous equation
\begin{equation}\label{Formulafrom2DWork}
\tilde{\lambda}_{n} = \frac{\lambda_{n}(\Omega)}{a^{2}} + \frac{1}{2 \, \pi} \, \log(a) \, \left( \int_{\Omega^{\star}} \overline{f}_{n}(x) \, dx \right)^{2},
\end{equation}
where $\Omega^{\star} = \dfrac{-z+\Omega}{a}, \, \left\vert \Omega^{\star} \right\vert \, \sim \, 1 $ and $\left\{ \tilde{\lambda}_{n} \right\}_{n \geq 0}$ is a positive and uniformly bounded sequence, i.e. $0 < \tilde{\lambda}_{n} \leq \underset{n}{Sup} \; \tilde{\lambda}_{n} < C^{te}$, where $C^{te}$ is a uniformly constant. Using the scale of $\lambda_{0}(\Omega)$ and the previous formula give us 
\begin{equation}\label{Equa1}
\tilde{\lambda}_{0} = \frac{\beta_{0} \, a^{2} \, \left\vert \log(a) \right\vert}{a^{2}} + \frac{1}{2 \, \pi} \, \log(a) \, \left( \int_{\Omega^{\star}} \overline{f}_{0}(x) \, dx \right)^{2},
\end{equation}
and this implies 
\begin{equation}\label{Equa2}
\left\vert \int_{\Omega^{\star}} \overline{f}_{0}(x) \, dx \right\vert = \sqrt{2 \, \pi \, \left( \beta_{0} - \frac{\tilde{\lambda}_{0}}{\left\vert \log(a) \right\vert} \right)} \;\;\sim \;\; 1.
\end{equation} 
\item[]
\item[ii)] For $n \geq 1$, we know from $(\ref{EstimationEigOmega})$, that $\lambda_{n}(\Omega) = \beta_{n} \, a^{2}\, $, where $\beta_{n}$ is a positive constant independent on the parameter $a$. Hence, using $(\ref{Formulafrom2DWork})$, we obtain
\begin{equation*}
\tilde{\lambda}_{n} = \frac{\beta_{n} \, a^{2} }{a^{2}} + \frac{1}{2 \, \pi} \, \log(a) \, \left( \int_{\Omega^{\star}} \overline{f}_{n}(x) \, dx \right)^{2},
\end{equation*}
and, knowing that $\tilde{\lambda}_{n} > 0$,  
\begin{equation*}
\left( \int_{\Omega^{\star}} \overline{f}_{n}(x) \, dx \right)^{2} \leq \beta_{n} \; 2 \, \pi \; \left\vert \log(a) \right\vert^{-1}.
\end{equation*}
This implies, 
\begin{equation*}
\left\vert \int_{\Omega^{\star}} \overline{f}_{n}(x) \, dx \right\vert  \;\; \lesssim \;\; \left\vert \log(a) \right\vert^{-\frac{1}{2}}.
\end{equation*}
Next, we show that the previous obtained estimation can be improved to be of order $\left\vert \log(a) \right\vert^{- 1}$, instead of $\left\vert \log(a) \right\vert^{-\frac{1}{2}}$. Now, after scaling the equation $(\ref{AddedEqua})$ and integrating again the obtained equation we obtain   
\begin{equation}\label{LBII}
\int_{\Omega^{\star}} \overline{f}_{n}(x) \, dx = \frac{\int_{\Omega^{\star}} N_{\Omega^{\star}}\left(1 \right)(x) \overline{f}_{n}(x) \, dx}{\left[  \frac{\lambda_{n}\left( \Omega \right)}{a^{2}} - \frac{1}{2 \, \pi} \, \left\vert \log(a) \right\vert \, \left\vert \Omega^{\star} \right\vert \right]},
\end{equation}
and knowing that $\lambda_{n}\left( \Omega \right) = \beta_{n} \, a^{2}$ we get
\begin{equation*}
\int_{\Omega^{\star}} \overline{f}_{n}(x) \, dx = \frac{\int_{\Omega^{\star}} N_{\Omega^{\star}}\left(1 \right)(x) \overline{f}_{n}(x) \, dx}{\left[ \beta_{n} - \frac{1}{2 \, \pi} \, \left\vert \log(a) \right\vert \, \left\vert \Omega^{\star} \right\vert \right]}.
\end{equation*}
By taking the modulus, in both sides, of the previous relation, using the fact that $\left\Vert N_{\Omega^{\star}} \right\Vert_{\mathcal{L}\left(\mathbb{L}^{2}(\Omega^{\star}) ; \mathbb{L}^{2}(\Omega^{\star})\right)} \; \sim \; 1$ and recalling that $\overline{f}_{n}(\cdot)$ are orthonormalized eigenfunctions in $\mathbb{L}^{2}(\Omega^{\star})$, we deduce  that 
\begin{equation*}
\left\vert \int_{\Omega^{\star}} \overline{f}_{n}(x) \, dx \right\vert \; \lesssim \;  \left\vert \log(a) \right\vert^{-1}. 
\end{equation*}

\item[]
\end{enumerate}

Finally, correspondingly to $(\ref{IranProtests})$ and for an arbitrary shape domain $\Omega$, we obtain after rescaling back to $\Omega$ the following behaviour of the integral of the eigenfunctions of the Newtonian potential operator with respect to the parameter $a$.
\begin{equation}\label{Iran'sWomen}
\int_{\Omega} f_{0}(x) \, dx \;\; \sim \;\; a \quad \text{and} \quad  \vert \int_{\Omega} f_{n}(x) \, dx \vert \;\; \lesssim \;\; a \, \left\vert \log(a) \right\vert^{-1}, \quad \text{for} \;\; n \geq 1.
\end{equation}
\end{enumerate}




%\section{2D plots}
%\begin{tikzpicture}
%\begin{axis}[xmin=-1, xmax=6, ymin=-5, ymax= 36, axis lines=middle, xlabel = $s$, ylabel = $p$]
%\addplot{x^2};
%\addplot{\arctan x};
%\addplot[color=red, domain=0:5]{x^2};
%\end{axis}
%\end{tikzpicture}

%\section{3D plots}
%\url{https://www.youtube.com/watch?v=5jmIHOWpEg0}

%\begin{tikzpicture}
%\begin{axis}[xlabel = $\Re(\omega)$, ylabel = $\Im(\omega)$, zlabel = $f(\omega)$]
%\addplot3[surf, shader=interp]{$\frac{1}{1-x^2-y^2}$};
%\end{axis}
%\end{tikzpicture}
%\thispagestyle{empty}
%\mbox{}


%\bigskip

\begin{thebibliography}{10}

%Cancel
%\bibitem{AhnDyaRae99}
%J.~F.~Ahner, V.~V.~Dyakin, V.~Ya.~Raevskii and R.~Ritter,
%\newblock{On series solutions of the magnetostatic integral equation,}
%\newblock{Zh. Vychisl. Mat. Mat. Fiz, number 4, volume 39, pages 630-637, 1999.}

%%%%%%%%%%%%%%%%%%%%%%%%%%%%%%%%%%%

%Cancel
%\bibitem{PhysRevA.82.055802}
%A. Akyurtlu and A.G. Kussow, 
%\newblock{Relationship between the Kramers-Kronig relations and negative index of refraction},
%\newblock{Phys. Rev. A, volume 82, American Physical Society, 2010.}

%%%%%%%%%%%%%%%%%%%%%%%%%%%%%%%%%%%%%%%%%%

%\bibitem{alsenafi2022foldy}
%A. Alsenafi, A. Ghandriche and M. Sini, 
%\newblock{The Foldy-Lax approximation is valid for nearly resonating frequencies, arXiv:2206.10899, 2022.}

%%%%%%%%%%%%%%%%%%%%%%%%%%%%%%%%

\bibitem{alsenafi2022foldy}
A. Alsenafi, A. Ghandriche and M. Sini,
\newblock{The Foldy–Lax approximation is valid for nearly resonating frequencies. Z. Angew.  Math. Phys. $\bm{74}$, 11 (2022).}





%%%%%%%%%%%%%%%%%%%%%%%%%%%%%%%%%%%%%%%%


%\bibitem{amirov2014integral}
%A. Kh. Amirov,
%\newblock{Integral geometry and inverse problems for kinetic equations, De Gruyter, 2014.}

%%%%%%%%%%%%%%%%%%%%%%%%%%%%%%%%%%%%%%%%%%%%

%\bibitem{Habib-book}
%H. Ammari,
%\newblock{An introduction to mathematics of emerging biomedical imaging,}
%\newblock{ Springer-Verlag, Volume 62, 2008.}


%%%%%%%%%%%%%%%%%%%%%%%%%%%%%%%%%%%
%Cancelled
%\bibitem{Ammari_2019}
%H. Ammari, D. P. Challa,  A. P. Choudhury and M. Sini,   
%\newblock{The point-interaction approximation for the fields generated by contrasted bubbles at arbitrary fixed frequencies,}
%\newblock{Journal of Differential Equations, volume 267, number 4, pages 2104-2191, 2019.}


%%%%%%%%%%%%%%%%%%%%%%%%%%%%%%%%

%Cancelled
%\bibitem{Ammari-Li-Zou}
%Habib Ammari, Bowen Li and Jun Zou, 
%\newblock{Super-resolution in Recovering Embedded Electromagnetic Sources in High Contrast Media,}
%\newblock{SIAM Journal on Imaging Sciences, volume 13, number 3,
%1467-1510, 2020.}

%%%%%%%%%%%%%%%%%%%%%%%%%%%%%%%%%%%
%\Cancelled
%\bibitem{Ammari-Zhang} 
%Habib Ammari and Hai Zhang,  
%\newblock{Super-resolution in high-contrast media,
%Proceedings of the Royal Society A: Mathematical, Physical and Engineering Sciences, volume 471, number 2178, 2015.}
%%%%%%%%%%%%%%%%%%%%%%%%%%%%%%%%%%%%%

\bibitem{ammari2018super}
H. Ammari, Y. T. Chow and J. Zou,
\newblock{Super-resolution in imaging high contrast targets from the perspective of scattering coefficients},
\newblock{Journal de Math{\'e}matiques Pures et Appliqu{\'e}es, volume 111, pages 191--226, Elsevier, 2018.}

%%%%%%%%%%%%%%%%%%%%%%%%%%%%%%%%%%%%%

\bibitem{ammari2019subwavelength}
H. Ammari, A. Dabrowski, B. Fitzpatrick, P. Millien and M. Sini,
\newblock{Subwavelength resonant dielectric nanoparticles with high refractive indices},
\newblock{Mathematical Methods in the Applied Sciences, volume 42, number 18, pages 6567--6579, Wiley Online Library, 2019.}


%%%%%%%%%%%%%%%%%%%%%%%%%%%%%%%%%%%%%

%Cancel
%\bibitem{Amrouche-Houda}
%C. Amrouche, N. El Houda Seloula. 
%\newblock{$\mathbb{L}^{p}$-Theory for vector potentials and Sobolev's inequalities for vector fields. Application to the Stokes equations with pressure boundary conditions. 2011. hal-00686230.}


%%%%%%%%%%%%%%%%%%%%%%%%%%%%%%%%%%%%%%

%\Cancelled
%\bibitem{amrouche1998vector}
%Ch\'erif Amrouche, Christine Bernardi, Monique Dauge and Vivette Girault, 
%\newblock{Vector potentials in three-dimensional non-smooth domains, Mathematical Methods in the Applied Sciences, volume 21,
%number 9, 1998.}

%%%%%%%%%%%%%%%%%%%%%%%%%%%%%%%%%%%%
\bibitem{Anderson}
J. M. Anderson, D. Khavinson and V. Lomonosov,
\newblock{Spectral properties of some integral operators arising in potential theory, The Quarterly Journal of Mathematics, volume 43,
number 4, pages 387--407, Oxford University Press, 1992.}
%%%%%%%%%%%%%%%%%%%%%%%%%%%%%%%%%%%%%

%\bibitem{anikonov1997uniqueness}
%Yu. E. Anikonov and V. G. Romanov, 
%\newblock{On uniqueness of determination of a form of first degree by its integrals along geodesics, Walter de Gruyter, 1997.}

%%%%%%%%%%%%%%%%%%%%%%%



%%%%%%%%%%%%%%%%%%%%%%%%%%%%%%%%%%%%%%%
%\Cancelled
%\bibitem{Ali}
%Ali Bouzekri and Mourad Sini, 
%\newblock{Foldy-Lax approximation of the electromagnetic fields generated by anisotropic inhomogeneities in the mesoscale regime with complements for the perfectly conducting case, arXiv, 2019.}

%%%%%%%%%%%%%%%%%%%%%%%%%%%%%%%%%%%%%%%

%Cancel
%\bibitem{B:2014}
%G.~Bal.
%\newblock Hybrid inverse problems and redundant systems of partial differential
%  equations.
%\newblock In {Inverse problems and applications}, volume 615 of {\em
%  Contemp. Math.}, pages 15--47. Amer. Math. Soc., Providence, RI, 2014.

%%%%%%%%%%%%%%%%%%%%%%%%%%%%%%%%
%Cancel
%\bibitem{B-B-M-T}
%G.~Bal, E.~Bonnetier, F.~Monard, and F.~Triki.
%\newblock Inverse diffusion from knowledge of power densities.
%\newblock {\em Inverse Probl. Imaging}, 7(2):353-375, 2013.

%%%%%%%%%%%%%%%%%%%%%%%%%%%%%%%%%

%\bibitem{B-U:2010}
%G. Bal and G. Uhlmann,
%\newblock Inverse diffusion theory of photoacoustics.
%\newblock{\em Inverse Problems}, 26 (2010). 085010.

%%%%%%%%%%%%%%%%%%%%%%%%%%%%%%%%%%%%%%%

%\bibitem{B-E-K-S:2018}
% A. Beigl, P. Elbau, K. Sadiq, O. Scherzer, 
%\newblock{ Quantitative photoacoustic imaging in the acoustic regime using SPIM.}
%\newblock{ Inverse Problems 34 (2018), no. 5, 054003, 5 pp.}

%%%%%%%%%%%%%%%%%%%%%%%%%%%%%%%%%%

%\bibitem{B-G-S:2016}
%Z. Belhachmi, T. Glatz, O. Scherzer,
%\newblock{ A direct method for photoacoustic tomography with inhomogeneous sound speed.}
%\newblock{ Inverse Problems 32 (2016), no. 4, 045005, 25 pp.}

%%%%%%%%%%%%%%%%%%%%%%%%%%%%%%

%\bibitem{caflisch1985wave}
%R. E. Caflisch, M. J. Miksis, G. C. Papanicolaou and L. Ting, 
%\newblock{Wave propagation in bubbly liquids at finite volume fraction, Journal of Fluid Mechanics,
%vol. 160, 1--14, Cambridge University Press, 1985.}

%%%%%%%%%%%%%%%%%%%%%%%%%%%%%%%%%%%%%%%%%%%%%%

%\Cancelled
%\bibitem{brezis}
%Haim Brezis, 
%\newblock{Functional analysis, Sobolev spaces and partial differential equations, 2010, Springer Science \& Business Media.}


%%%%%%%%%%%%%%%%%%%%%%%%%%%%%%%%%%%
%\Cancelled
%\bibitem{BCS}
%A. Buffa, M. Costabel and D. Sheen,
%\newblock{On traces for $\bm{H(curl,\Omega)}$ in Lipschitz domains},
%\newblock{Journal of Mathematical Analysis and Applications,
%Volume 276, Issue 2, 2002, Pages 845-867.}

%%%%%%%%%%%%%%%%%%%%%%%%%%%%%%%%%%%%%%%%%%
% Cancel
%\bibitem{CZ}
%A. P. Calder{\'o}n and A. Zygmund,
%\newblock{Singular integral operators and differential equations},
%\newblock{American Journal of mathematics, volume 79, number = 4, pages 901-921, 1957, JSTOR.}

%%%%%%%%%%%%%%%%%%%%%%%%%%%%%%

%\bibitem{carton2007cours}
%H. Carton,
%\newblock{Cours de calcul diff\'{e}rentiel, Herman, 2007.}

%%%%%%%%%%%%%%%%%%%%%%%%%%%%%%%%%%%%%%%%%%

\bibitem{carslaw1947conduction}
H. S. Carslaw and J. C. Jaeger,
\newblock{Conduction of heat in solids, 1947.}

%%%%%%%%%%%%%%%%%%%%%%%%%%%%%%%%%%%%%%%%%%

\bibitem{cartan1945theorie}
H. Cartan,
\newblock{Th{\'e}orie du potentiel newtonien: {\'e}nergie, capacit{\'e}, suites de potentiels, Bulletin de la Soci{\'e}t{\'e} Math{\'e}matique de France, volume 73, pages 74--106, 1945.}



%%%%%%%%%%%%%%%%%%%%%%%%%%%%%%%%%%%%%%%%%%

%\Cancelled
%\bibitem{Ma2019MathematicalAN}
%Ma Chupeng, Z. Yongwei and J. Zou, 
%\newblock{Mathematical and numerical analysis of a nonlocal Drude model in nanoplasmonics, ArXiv, 2019.}

%%%%%%%%%%%%%%%%%%%%%%%%%%%%%%%%%%%%

\bibitem{colton2019inverse}
D. Colton and R. Kress,
\newblock{Inverse acoustic and electromagnetic scattering theory, 93, 2019, Springer Nature.}


%%%%%%%%%%%%%%%%%%%%%%%%%%%%%%%%%%%%%%%
%Cancel
%\bibitem{Costabel}
%M. Costabel, 
%\newblock{Some historical remarks on the positivity of boundary integral operators. In: Boundary element analysis. vol. 29 of Lect. Notes Appl. Comput. Mech. Springer, Berlin; 2007. p. 1–27.}

%%%%%%%%%%%%%%%%%%%%%%%%%%%
%\Cancelled
%\bibitem{Costabel.D.K}
%M. Costabel, E. Darrigrand and E.H. Koné,
%\newblock{Volume and surface integral equations for electromagnetic scattering by a dielectric body, Journal of Computational and Applied Mathematics, Volume 234, Issue 6,
%2010, Pages 1817-1825.}

%%%%%%%%%%%%%%%%%%%%%%%%
%Cancel
%\bibitem{C-A-B:2007}
%B. T. Cox, S. R. Arridge, and P. C. Beard,
%\newblock{Photoacoustic tomography with a limited-aperture planar sensor and a reverberant cavity,}
%\newblock{\em Inverse Problems,} 23, pp. S95-S112, 2007.


%%%%%%%%%%%%%%%%%%%%%%%%%%%%%%%%%%%%%%%
%Cancel
%\bibitem{cutzach1998existence}
%P. M. Cutzach and C. Hazard, 
%\newblock{Existence, uniqueness and analyticity properties for electromagnetic scattering in a two-layered medium},
%\newblock{Mathematical methods in the applied sciences, volume 21, number 5, pages 433--461, 1998.}


%%%%%%%%%%%%%%%%%%%%%%%%%
%Cancel
%\bibitem{Dyakin-Rayevskii}
%V.V. Dyakin and V.Ya. Rayevskii,
%\newblock{Investigation of an equation of electrophysics, U.S.S.R Computational Mathematics and Mathematical Physics, Volume 30, Number 1, Pages 213-217, 1990.}

%%%%%%%%%%%%%%%%%%%%%%%%%%%%%%%%%%%%%%%%ù
%Cancel
%\bibitem{Dautry-Lions}
%R. Dautray and J. L. Lions,
%\newblock{Mathematical Analysis and Numerical Methods for Science and Technology Volume 3 Spectral Theory and Applications.}
%\newblock{1st ed. Berlin, Heidelberg: Springer Berlin Heidelberg, 2000.}

%%%%%%%%%%%%%%%%%%%%%%%%%%%%%%%%%%%%%%
%\Cancelled
%\bibitem{Di_Fratta_2016}
%Giovanni Di Fratta,
%\newblock{The Newtonian potential and the demagnetizing factors of the general ellipsoid},
%\newblock{Proceedings of the Royal Society A: Mathematical, Physical and Engineering Sciences, volume 472, number 2190, 2016.}

%%%%%%%%%%%%%%%%%%%%%%%%%%%%%%%%%%
%Cancel
%\bibitem{engheta2006metamaterials}
%N. Engheta and R. W. Ziolkowski,
%\newblock{Metamaterials: physics and engineering explorations, John Wiley \& Sons, 2006.}

%%%%%%%%%%%%%%%%%%%%%%%%%%%%%%%%%%%%%%%%%%%%
%\bibitem{Evans}
%L.C. Evans,
%\newblock{Partial Differential Equations, American Mathematical Society, 2010.}

%%%%%%%%%%%%%%%%%%%%%%%%%%%%%%%%%%%%%%%%%%%%
%Cancel
%\bibitem{FHR}
%D. Finch, M. Haltmeier and Rakesh,
%\newblock{Inversion of spherical means and the wave equation in even dimensions,}
%\newblock{SIAM Journal on Applied Mathematics, V. 68, n. 2, pp. 392-412, 2007.}

%%%%%%%%%%%%%%%%%%%%%%%%%%%%%%%%%%%%%%%%%%%%
%Cancel
%\bibitem{Foias1978}
%C. Foias and R. Temam,
%\newblock{Remarques sur les \'equations de Navier-Stokes stationnaires et les ph\'enom\`enes successifs de bifurcation},
%\newblock{Annali della Scuola Normale Superiore di Pisa-Classe di Scienze, Scuola normale superiore, Number 1, pages 29-63, Volume 5, 1978.}

%%%%%%%%%%%%%%%%%%%%%%%%%%%%%%%%%%%%%%%%%%%
%Cancel
%\bibitem{friedman1980mathematical}
%M. J. Friedman,
%\newblock{Mathematical study of the nonlinear singular integral magnetic field equation. \bf{I}.}
%\newblock{SIAM Journal on Applied Mathematics, volume 39, number 1, pages 14-20, 1980.}

%%%%%%%%%%%%%%%%%%%%%%%%%%%%%%%%%%%%%%%%%%%%%%%%
%Cancel
%\bibitem{friedman1981mathematical}
%M. J. Friedman,
%\newblock{Mathematical study of the nonlinear singular integral magnetic field equation. \bf{III}.}
%\newblock{SIAM Journal on Mathematical Analysis, volume 12, number 4, pages 536-540, 1981.}
  
%%%%%%%%%%%%%%%%%%%%%%%%%%%%%%%%%%%
%Cancel
%\bibitem{10.2307/2008286}
%M. J. Friedman and J. E. Pasciak,
%\newblock{Spectral Properties for the Magnetization Integral Operator, Mathematics of Computation, number = 168, pages  447-453, volume 43, 1984.}

%%%%%%%%%%%%%%%%%%%%%%%%%%%%%%%%%%

%\bibitem{Gelfand1967GeneralizedFV}
%I. M. Gel'fand and G. E. Shilov, 
%\newblock{Generalized Functions, Volume 1: Properties and Operations, American Mathematical Monthly, volume 377, pages 1026, 1967.}

%%%%%%%%%%%%%%%%%%%%%%%%%%%%%%%%%%%%%
%Cancel
%\bibitem{Ahcene-Mourad-IICM}
%A. Ghandriche and M. Sini,
%\newblock{An Introduction To The Mathematics Of The Imaging Modalities Using Small Scaled Contrast Agents, arXiv, physics.optics, 2020.}

%%%%%%%%%%%%%%%%%%%%%%%%%%%%%%%%%%%
%Cancel
%\bibitem{Ghandriche}
%A. Ghandriche and M. Sini,
%\newblock{Mathematical analysis of the photo-acoustic imaging modality using resonating dielectric nano-particles: The 2D TM-model, Journal of Mathematical Analysis and Applications,
%volume 506, number 2, 2022.}

%%%%%%%%%%%%%%%%%%%%%%%%%%%%%%%%%%%%%

%\bibitem{AhceneMouradMaxwell}
%A. Ghandriche and M. Sini,
%\newblock{Photo-acoustic inversion using plasmonic contrast agents: The full Maxwell model, arXiv:2111.06269, 2021.}

%%%%%%%%%%%%%%%%%%%%%%%%%%%%%%%%%%%%%%%%%%

\bibitem{ghandriche2022mathematical}
A. Ghandriche and M. Sini,
\newblock{Mathematical analysis of the photo-acoustic imaging modality using resonating dielectric nano-particles: The 2D TM-model},
\newblock{Journal of Mathematical Analysis and Applications, volume 506, number 2, pages 125658, Elsevier, 2022.}

%%%%%%%%%%%%%%%%%%%%%%%%%%%%%%%%

%\bibitem{gilbarg2001elliptic}
%D. Gilbarg and N. S. Trudinger,
%\newblock{Elliptic partial differential equations of second order, 2001, Springer.}


%%%%%%%%%%%%%%%%%%%%%%%%%%%%%%%%%%

%\bibitem{henrot2006extremum}
%A. Henrot,
%\newblock{Extremum problems for eigenvalues of elliptic operators, Springer Science \& Business Media, 2006.}


%%%%%%%%%%%%%%%%%%%%%%%%%%%%%%%%%%%%%%%%%


%\Cancelled
%\bibitem{girault2012finite}
%Vivette Girault and Raviart Pierre-Arnaud,   
%\newblock{Finite element methods for Navier-Stokes equations: theory and algorithms, volume 5, Springer Science \& Business Media, 2012.}

%%%%%%%%%%%%%%%%%%%%%%%%%%%%%%%%%%%%%%%
%\Cancelled
%\bibitem{li2012time}
%Y. Huang and J. Li, 
%\newblock{Time-Domain Finite Element Methods for Maxwell's Equations in Metamaterials, 2012, Springer Berlin Heidelberg.}


%%%%%%%%%%%%%%%%%%%%%%%%%%%%%%%%%%%%%
%\Cancelled
%\bibitem{Sini-India}
%Manas Kar and Mourad Sini,
%\newblock{Reconstruction of interfaces using CGO solutions for the Maxwell equations, Journal of Inverse and Ill-posed Problems, vol. 22, no. 2, 2014, pp. 169-208.} 

%%%%%%%%%%%%%%%%%%%%%%%%%%%%%%%%%%%%%%
%\Cancel
%\bibitem{lgeleyen2010}
%İ, Gölgeleyen,
%\newblock{An integral geometry problem along geodesics and a computational approach},
%\newblock{number 2, pages 91-112, volume 18, Analele Ştiinţifice ale Universităţii “Ovidius" Constanţa. Seria: Matematică, 2010.}

%%%%%%%%%%%%%%%%%%%%%%%%%%%%%%%%%

\bibitem{kalmenov2011boundary}
T. Kalmenov and D. Suragan,
\newblock{A boundary condition and spectral problems for the Newton potential, Modern aspects of the theory of partial differential equations, pages 187--210, Springer, 2011.}

%%%%%%%%%%%%%%%%%%%%%%%%%%%%%%%%%%%%

\bibitem{kellogg}
O. D. Kellogg,
\newblock{Foundations of potential theory,
  volume 31, Courier Corporation, 1953.}

%%%%%%%%%%%%%%%%%%%%%%%%%%%%%%%%%%%%

%\bibitem{Kirsch-Scherzer}
%A. Kirsch and O. Scherzer, 
%\newblock{Simultaneous reconstructions of absorption density and wave speed with photoacoustic measurements,} \newblock{SIAM J. Appl. Math. 72 (2012), no. 5, 1508-1523.}


%%%%%%%%%%%%%%%%%%%%%%%%%%%%%ù%%%%
%\Cancelled
%\bibitem{Khavinson_2014}
%D. Khavinson and E. Lundberg,
%\newblock{A Tale of Ellipsoids in Potential Theory, Notices of the American Mathematical Society, volume 61, number 02, 2014.}

%%%%%%%%%%%%%%%%%%%%%%%%%%%%%%%%%%%%%%%%%%%%%%
%Cancel
%\bibitem{Kirsch}
%A. Kirsch,
%\newblock{An Integral Equation for Maxwell's Equations in a Layered Medium with an Application to the Factorization Method},
%\newblock{Journal of Integral Equations and Applications, volume 9, number 3, pages 333 - 358,
%2007.}

%%%%%%%%%%%%%%%%%%%%%%%%%%%%%%%%%%%%%%%%%%


%\bibitem{K-K:2010} 
%P. Kuchment and L. Kunyansky,
%\newblock{Mathematics of thermoacoustic and photoacoustic tomography,} 
%\newblock{in Handbook of Mathematical Methods in Imaging,
%O. Scherzer, ed., Springer-Verlag,} pp. 817-866, 2010.

%%%%%%%%%%%%%%%%%%%%%%%%%%%%%%%%%%%%%%%%%%%

%\bibitem{KuchmentKunyansky} 
%P. Kuchment and L. Kunyansky,  
%\newblock{Mathematics of thermoacoustic tomography, \,
%European Journal of Applied Mathematics,}
%\newblock{Volume 19, Number 02, 2008.}


%%%%%%%%%%%%%%%%%%%%%%%%%%%%%%%%%%%%%%
%\Cancelled
%\bibitem{Mitrea}
%D. Mitrea, M. Mitrea and J. Pipher, 
%\newblock{Vector potential theory on nonsmooth domains in $\mathbb{R}^{3}$ and applications to electromagnetic scattering. The Journal of Fourier Analysis and Applications 3, 131–192 (1997).}

%%%%%%%%%%%%%%%%%%%%%%%%%%%%%%%%%%%%%%

%\bibitem{lavrent1967certain}
%M. M. Lavrent'ev and Yu. E. Anikonov, 
%\newblock{A certain class of problems in integral geometry},
%\newblock{Doklady Akademii Nauk, volume 176, number 5, pages 1002--1003, Russian Academy of Sciences, 1967.}

%%%%%%%%%%%%%%%%%%%%%%%%%%%%%%%%%%%%%%

\bibitem{LANDAU}
L.J. Landau, 
\newblock{Ratios of Bessel Functions and Roots of $\alpha \, \begin{LARGE}
\textbf{J}_{\nu}
\end{LARGE}\left( x \right) + x \, \begin{LARGE}
\textbf{J}^{\prime}_{\nu}
\end{LARGE}\left( x \right) = 0$},
\newblock{Journal of Mathematical Analysis and Applications, volume  240, number 1, pages 174-204, 1999.}

%%%%%%%%%%%%%%%%%%%%%%%%%%%%%%%%%%%%%%

\bibitem{lax2002functional}
P. D. Lax,
\newblock{Functional analysis, volume 55, John Wiley \& Sons, 2002.}

%%%%%%%%%%%%%%%%%%%%%%%%%%%%%%%%%%%%%%
%Cancel
%\bibitem{McLean}
%W. McLean, 
%\newblock{Strongly elliptic systems and boundary integral equations, Cambridge university press, 2000.}
%%%%%%%%%%%%%%%%%%%%%%%%%%%%%%%%%






%\bibitem{N-S:2014}
%W. Naetar and O. Scherzer,
%\newblock{Quantitative photoacoustic tomography with piecewise constant material parameters,}
%\newblock{\em SIAM J.
%Imag. Sci.,} V. 7, pp. 1755-1774, 2014.

%%%%%%%%%%%%%%%%%%%%%%%%%%%%%%
%Cancel
%\bibitem{Natterer}
%F. Natterer,
%\newblock{The Mathematics of Computerized Tomography,}
%\newblock{Society for Industrial and Applied Mathematics, \; 2001}.


%%%%%%%%%%%%%%%%%%%%%%%%%%%%%%%%%%%%
%Cancel
%\bibitem{Neil}
%P. V. O'Neil,
%\newblock{Beginning partial differential equations, 2011,  John Wiley \& Sons.}


%%%%%%%%%%%%%%%%%%%%%%%%%%%%%%%%%%%%
%\Cancelled
%\bibitem{powell2011calculating}
%Philip D. Powell,
%\newblock{Calculating Determinants of Block Matrices, arXiv, 2011.}  

%%%%%%%%%%%%%%%%%%%%%%%%%%%%%%%
%\bibitem{P-P-B:2015}
%A. Prost, F. Poisson and E. Bossy.
%\newblock{ Photoacoustic generation by gold nanosphere: From linear to nonlinear thermoelastic in the 
%long-pulse illumination regime}
%\newblock arkiv:1501.04871v4

%%%%%%%%%%%%%%%%%%%%%%%%%%%%%%%%%%%%%%%%%
%\Cancelled

%\bibitem{Ritter}
%Stefan Ritter,
%\newblock{On the computation of Lam\`e functions, of eigenvalues and eigenfunctions of some potential operators,}
%\newblock{Z. f. angew. Math. u. Mech. 78, pages 66-72, 1998.}

%%%%%%%%%%%%%%%%%%%%%%%%%%%%%%%%%%%%%%%%%ù
%Cancel
%\bibitem{Stein}
%E. M. Stein,
%\newblock{Singular integrals and differentiability properties of functions, Princeton Univ, 1970.}


%%%%%%%%%%%%%%%%%%%%%%%%%%%%%%%%%%%%%%%%%ù


%Cancel
%\bibitem{Raevskii1994}
%V. Ya. Raevskii,
%\newblock{Some properties of the operators of potential theory and their application to the investigation of the basic equation of electrostatics and magnetostatics,} 
%\newblock{Theoretical and Mathematical Physics, Volume 100, Number 3, pages 1040-1045, 1994.}

%%%%%%%%%%%%%%%%%%%%%%%%%%%%%%%%%%%%%%

%\bibitem{romanovBook}
%V. G. Romanov,
%\newblock{Inverse problems of mathematical physics, De Gruyter, 2018.}

%%%%%%%%%%%%%%%%%%%%%%%%%%%%%%%%%%%%%%%
%Cancel
%\bibitem{romanov1978integral}
%V. G. Romanov, 
%\newblock{Integral geometry on geodesics of an isotropic Riemannian metric,}
%\newblock{Doklady Akademii Nauk, vol. 241, number 2, pages 290--293, Russian Academy of Sciences, 1978.}

%%%%%%%%%%%%%%%%%%%%%%%%%%%%%%%%%%%%%%

%\bibitem{romanov2009smoothness}
%V. G. Romanov,
%\newblock{On smoothness of a fundamental solution to a second order hyperbolic equation},
%\newblock{Siberian Mathematical Journal, Springer,
%volume 50, number 4, pages 700--705, 2009.}

%%%%%%%%%%%%%%%%%%%%%%%%%%%%%%%%%%%%%%%

%\bibitem{romanov2013integral}
%V. G. Romanov, 
%\newblock{Integral geometry and inverse problems for hyperbolic equations,
%  volume 26, Springer Science \& Business Media, 2013.}

%%%%%%%%%%%%%%%%%%%%%%%%%%%%%%%%%%%%%%

\bibitem{ruzhansky2016isoperimetric}
M. Ruzhansky and D. Suragan,
\newblock{Isoperimetric inequalities for the logarithmic potential operator, Journal of Mathematical Analysis and Applications, volume 434, number 2, pages 1676--1689, Elsevier, 2016.}


%%%%%%%%%%%%%%%%%%%%%%%%%%%%%%%%%%%%%%

%\bibitem{salo}
%M. Salo,
%\newblock{Stability for solutions of wave equations with $C^{1, 1}$ coefficients, arXiv preprint math/0611457, 2006.}

%%%%%%%%%%%%%%%%%%%%%%%%%%%%%%%%%%%%%%%%

%\bibitem{S:2010}
%O. Scherzer,
%\newblock{Handbook of Mathematical Methods in Imaging,}
%\newblock {\em Springer-Verlag}, 2010.

%%%%%%%%%%%%%%%%%%%%%%%%%%%%%%%%%%%%%%%%%%

%\bibitem{smith}
%H. F. Smith,
%\newblock{A parametrix construction for wave equations with $C^{1,1}$ coefficients, Annales de l'institut Fourier, vol. 48, num. 3, 797--835, 1998.}


%%%%%%%%%%%%%%%%%%%%%%%%%%%%%%%%%%%%%%%%%%
%\bibitem{S-U:2009}
%P. Stefanov and G. Uhlmann,
%\newblock{Thermoacoustic tomography with variable sound speed,} 
%\newblock {\em Inverse Problems}, 25, 075011, 2009.

%%%%%%%%%%%%%%%%%%%%%%%%%%%%%%%%%%%%

%\bibitem{Triki-Vauthrin:2017}
%F. Triki and M. Vauthrin, 
%\newblock{Mathematical modelization of the Photoacoustic effect generated by the heating of metallic nanoparticles, arXiv, 2017.}

%%%%%%%%%%%%%%%%%%%%%%%%%%%%%%%%%%%%%%%

\bibitem{Temme}
N. M. Temme,
\newblock{Special functions: An introduction to the classical functions of mathematical physics, John Wiley \& Sons, 1996.}


%%%%%%%%%%%%%%%%%%%%%%%%%%%%%%%%%%%%%%%


\bibitem{watson}
G. N. Watson,
\newblock{A treatise on the theory of Bessel functions, volume 3, The University Press,1922.}


%%%%%%%%%%%%%%%%%%%%%%%%%%%%%%%%%%%%%%%%%

%\bibitem{wlcx}
%C. H. Wilcox,
%\newblock{Multiparameter acoustic imaging in the born approximation, Mathematical methods in the applied sciences, vol. 5, num. 1, 276--291, Wiley Online Library, 1983.}

%%%%%%%%%%%%%%%%%%%%%%%%%%%%%%%%%%%%%%

%\bibitem{wilcox1977spectral}
%C. H. Wilcox,
%\newblock{Spectral and asymptotic analysis of acoustic wave propagation},
%\newblock{ booktitle: Boundary Value Problems for Linear Evolution Partial Differential Equations,
%385--473, Springer, 1977.}


%%%%%%%%%%%%%%%%%%%%%%%%%%%%%%%%%%%%%%%%%%

%\Cancelled
%\bibitem{Valdivia}
%Nicolas Valdivia,
%\newblock{Uniqueness in inverse obstacle scattering with conductive boundary conditions, Applicable Analysis, volume  83, number 8, pages 825-851, 2004.}
%%%%%%%%%%%%%%%%%%%%%%%%%%%%%%%%%%%%5

 



\end{thebibliography}








\end{document}