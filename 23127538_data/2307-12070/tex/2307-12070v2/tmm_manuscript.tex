\documentclass[lettersize,journal]{IEEEtran}
\usepackage{hyperref}
\usepackage{amsmath,amsfonts}
%\usepackage{algorithmic}
\usepackage{algorithm}
\usepackage{array}
\usepackage[caption=false,font=normalsize,labelfont=sf,textfont=sf]{subfig}
\usepackage{textcomp}
\usepackage{stfloats}
\usepackage{url}
\usepackage{verbatim}
\usepackage{graphicx}
\usepackage{cite}
\usepackage[table,xcdraw]{xcolor}

\usepackage{booktabs}
\usepackage{amssymb}
\usepackage{mathtools}
\usepackage{algorithm}
\usepackage{algpseudocode}
\usepackage{float}
\usepackage{multirow}

\usepackage{amsthm}
\theoremstyle{plain}
\newtheorem{assumption}{Assumption}
\newtheorem{proposition}{Proposition}
\newtheorem{remark}{Remark}
\newtheorem{definition}{Definition}
\newtheorem{theorem}{Theorem}

\hyphenation{op-tical net-works semi-conduc-tor IEEE-Xplore}
\usepackage[normalem]{ulem}
\useunder{\uline}{\ul}{}
% updated with editorial comments 8/9/2021

%%%%% NEW MATH DEFINITIONS %%%%%
\newtheorem{property}{Property}
\newtheorem{definition}{Definition}
\newtheorem{theorem}{Theorem}
\newtheorem{lemma}{Lemma}
\newtheorem{corollary}{Corollary}
\DeclarePairedDelimiter\abs{\lvert}{\rvert}
\DeclarePairedDelimiter\norm{\lVert}{\rVert}
\makeatletter
\let\oldabs\abs
\def\abs{\@ifstar{\oldabs}{\oldabs*}}
\let\oldnorm\norm
\def\norm{\@ifstar{\oldnorm}{\oldnorm*}}
\makeatother

% Mark sections of captions for referring to divisions of figures
\newcommand{\figleft}{{\em (Left) }}
\newcommand{\figcenter}{{\em (Center) }}
\newcommand{\figright}{{\em (Right)}}
\newcommand{\figtop}{{\em (Top) }}
\newcommand{\figbottom}{{\em (Bottom) }}
\newcommand{\captiona}{{\em (a) }}
\newcommand{\captionb}{{\em (b) }}
\newcommand{\captionc}{{\em (c) }}
\newcommand{\captiond}{{\em (d) }}

% Highlight a newly defined term
\newcommand{\newterm}[1]{{\bf #1}}


\def\figref#1{figure~\ref{#1}}
\def\Figref#1{Figure~\ref{#1}}
\def\twofigref#1#2{figures \ref{#1} and \ref{#2}}
\def\quadfigref#1#2#3#4{figures \ref{#1}, \ref{#2}, \ref{#3} and \ref{#4}}
\def\secref#1{section~\ref{#1}}
\def\Secref#1{Section~\ref{#1}}
\def\twosecrefs#1#2{sections \ref{#1} and \ref{#2}}
\def\secrefs#1#2#3{sections \ref{#1}, \ref{#2} and \ref{#3}}
\def\eqref#1{equation~\ref{#1}}
\def\Eqref#1{Equation~\ref{#1}}
% A raw reference to an equation---avoid using if possible
\def\plaineqref#1{\ref{#1}}
% Reference to a chapter, lower-case.
\def\chapref#1{chapter~\ref{#1}}
% Reference to an equation, upper case.
\def\Chapref#1{Chapter~\ref{#1}}
% Reference to a range of chapters
\def\rangechapref#1#2{chapters\ref{#1}--\ref{#2}}
% Reference to an algorithm, lower-case.
\def\algref#1{algorithm~\ref{#1}}
% Reference to an algorithm, upper case.
\def\Algref#1{Algorithm~\ref{#1}}
\def\twoalgref#1#2{algorithms \ref{#1} and \ref{#2}}
\def\Twoalgref#1#2{Algorithms \ref{#1} and \ref{#2}}
% Reference to a part, lower case
\def\partref#1{part~\ref{#1}}
% Reference to a part, upper case
\def\Partref#1{Part~\ref{#1}}
\def\twopartref#1#2{parts \ref{#1} and \ref{#2}}

% Random variables
\def\reta{{\textnormal{$\eta$}}}
\def\ra{{\textnormal{a}}}

% Random vectors
\def\rvepsilon{{\mathbf{\epsilon}}}
\def\rvtheta{{\mathbf{\theta}}}
\def\rva{{\mathbf{a}}}

% Elements of random vectors
\def\erva{{\textnormal{a}}}
\def\ervb{{\textnormal{b}}}

% Random matrices
\def\rmA{{\mathbf{A}}}
\def\rmB{{\mathbf{B}}}

% Elements of random matrices
\def\ermA{{\textnormal{A}}}
\def\ermB{{\textnormal{B}}}

\def\fvec{{\mathbf{f}}}
\def\bff{{\mathbf{f}}}
\def\bfg{{\mathbf{g}}}
% Vectors
\def\vzero{{\bm{0}}}
\def\vone{{\bm{1}}}
\def\vmu{{\bm{\mu}}}
\def\vtheta{{\bm{\theta}}}
\def\va{{\bm{a}}}
\def\vb{{\bm{b}}}
\def\vc{{\bm{c}}}
\def\vd{{\bm{d}}}
\def\ve{{\bm{e}}}
\def\vf{{\bm{f}}}
\def\vg{{\bm{g}}}
\def\vh{{\bm{h}}}
\def\vi{{\bm{i}}}
\def\vj{{\bm{j}}}
\def\vk{{\bm{k}}}
\def\vl{{\bm{l}}}
\def\vm{{\bm{m}}}
\def\vn{{\bm{n}}}
\def\vo{{\bm{o}}}
\def\vp{{\bm{p}}}
\def\vq{{\bm{q}}}
\def\vr{{\bm{r}}}
\def\vs{{\bm{s}}}
\def\vt{{\bm{t}}}
\def\vu{{\bm{u}}}
\def\vv{{\bm{v}}}
\def\vw{{\bm{w}}}
\def\vx{{\bm{x}}}
\def\vy{{\bm{y}}}
\def\vz{{\bm{z}}}

% Matrix
\def\mA{{\bm{A}}}

% Tensor
\DeclareMathAlphabet{\mathsfit}{\encodingdefault}{\sfdefault}{m}{sl}
\SetMathAlphabet{\mathsfit}{bold}{\encodingdefault}{\sfdefault}{bx}{n}
\newcommand{\tens}[1]{\bm{\mathsfit{#1}}}
\def\tA{{\tens{A}}}
\def\tB{{\tens{B}}}
\def\tC{{\tens{C}}}
\def\tD{{\tens{D}}}
\def\tE{{\tens{E}}}
\def\tF{{\tens{F}}}
\def\tG{{\tens{G}}}
\def\tH{{\tens{H}}}
\def\tI{{\tens{I}}}
\def\tJ{{\tens{J}}}
\def\tK{{\tens{K}}}
\def\tL{{\tens{L}}}
\def\tM{{\tens{M}}}
\def\tN{{\tens{N}}}
\def\tO{{\tens{O}}}
\def\tP{{\tens{P}}}
\def\tQ{{\tens{Q}}}
\def\tR{{\tens{R}}}
\def\tS{{\tens{S}}}
\def\tT{{\tens{T}}}
\def\tU{{\tens{U}}}
\def\tV{{\tens{V}}}
\def\tW{{\tens{W}}}
\def\tX{{\tens{X}}}
\def\tY{{\tens{Y}}}
\def\tZ{{\tens{Z}}}


% Graph
\def\gA{{\mathcal{A}}}
\def\gB{{\mathcal{B}}}
\def\gC{{\mathcal{C}}}
\def\dataset{{\mathcal{D}}}
\def\gE{{\mathcal{E}}}
\def\gF{{\mathcal{F}}}
\def\fourier{{\mathcal{F}}}
\def\gG{{\mathcal{G}}}
\def\gH{{\mathcal{H}}}
\def\gI{{\mathcal{I}}}
\def\gJ{{\mathcal{J}}}
\def\gK{{\mathcal{K}}}
\def\gL{{\mathcal{L}}}
\def\loss{{\mathcal{L}}}
\def\gM{{\mathcal{M}}}
\def\gN{{\mathcal{N}}}
\def\normal{{\mathcal{N}}}
\def\gaussian{{\mathcal{N}}}
\def\gO{{\mathcal{O}}}
\def\gP{{\mathcal{P}}}
\def\gQ{{\mathcal{Q}}}
\def\gR{{\mathcal{R}}}
\def\gS{{\mathcal{S}}}
\def\gT{{\mathcal{T}}}
\def\gU{{\mathcal{U}}}
\def\uniform{{\mathcal{U}}}
\def\gV{{\mathcal{V}}}
\def\gW{{\mathcal{W}}}
\def\gX{{\mathcal{X}}}
\def\gY{{\mathcal{Y}}}
\def\gZ{{\mathcal{Z}}}

\def\algebra{{\mathscr{A}}}
\def\borel{{\mathscr{B}}}
\def\manifold{{\mathscr{M}}}

% Sets
\def\sA{{\mathbb{A}}}
\def\sB{{\mathbb{B}}}
\def\complex{{\mathbb{C}}}
\def\sD{{\mathbb{D}}}
\def\expectation{{\mathbb{E}}}
\newcommand{\E}{\mathbb{E}}
\def\sF{{\mathbb{F}}}
\def\sG{{\mathbb{G}}}
\def\sH{{\mathbb{H}}}
\def\sI{{\mathbb{I}}}
\def\sJ{{\mathbb{J}}}
\def\sK{{\mathbb{K}}}
\def\sL{{\mathbb{L}}}
\def\sM{{\mathbb{M}}}
\def\natural{{\mathbb{N}}}
\def\sO{{\mathbb{O}}}
\def\sP{{\mathbb{P}}}
\def\rational{{\mathbb{Q}}}
\def\real{{\mathbb{R}}}
\newcommand{\R}{\mathbb{R}}
\def\sS{{\mathbb{S}}}
\def\sphere{{\mathbb{S}}}
\def\sT{{\mathbb{T}}}
\def\sU{{\mathbb{U}}}
\def\sV{{\mathbb{V}}}
\def\sW{{\mathbb{W}}}
\def\sX{{\mathbb{X}}}
\def\sY{{\mathbb{Y}}}
\def\integer{{\mathbb{Z}}}
\def\indicator{{\mathbbm{1}}}

% Entries of a matrix
\def\emLambda{{\Lambda}}
\def\emA{{A}}
\def\emB{{B}}
\def\emC{{C}}
\def\emD{{D}}
\def\emE{{E}}
\def\emF{{F}}
\def\emG{{G}}
\def\emH{{H}}
\def\emI{{I}}
\def\emJ{{J}}
\def\emK{{K}}
\def\emL{{L}}
\def\emM{{M}}
\def\emN{{N}}
\def\emO{{O}}
\def\emP{{P}}
\def\emQ{{Q}}
\def\emR{{R}}
\def\emS{{S}}
\def\emT{{T}}
\def\emU{{U}}
\def\emV{{V}}
\def\emW{{W}}
\def\emX{{X}}
\def\emY{{Y}}
\def\emZ{{Z}}
\def\emSigma{{\Sigma}}

% entries of a tensor
% Same font as tensor, without \bm wrapper
\newcommand{\etens}[1]{\mathsfit{#1}}
\def\etLambda{{\etens{\Lambda}}}
\def\etA{{\etens{A}}}
\def\etB{{\etens{B}}}
\def\etC{{\etens{C}}}
\def\etD{{\etens{D}}}
\def\etE{{\etens{E}}}
\def\etF{{\etens{F}}}
\def\etG{{\etens{G}}}
\def\etH{{\etens{H}}}
\def\etI{{\etens{I}}}
\def\etJ{{\etens{J}}}
\def\etK{{\etens{K}}}
\def\etL{{\etens{L}}}
\def\etM{{\etens{M}}}
\def\etN{{\etens{N}}}
\def\etO{{\etens{O}}}
\def\etP{{\etens{P}}}
\def\etQ{{\etens{Q}}}
\def\etR{{\etens{R}}}
\def\etS{{\etens{S}}}
\def\etT{{\etens{T}}}
\def\etU{{\etens{U}}}
\def\etV{{\etens{V}}}
\def\etW{{\etens{W}}}
\def\etX{{\etens{X}}}
\def\etY{{\etens{Y}}}
\def\etZ{{\etens{Z}}}

\def\ceil#1{\lceil #1 \rceil}
\def\floor#1{\lfloor #1 \rfloor}
\def\eps{{\epsilon}}

\newcommand{\pder}[1]{\frac{\partial}{\partial #1}}

\newcommand{\half}{\frac{1}{2}}
\newcommand{\limNinf}{\lim_{N \to \infty}}
\newcommand{\limTzero}{\lim_{\tau \to 0}}


\newcommand{\cmark}{\ding{51}}
\newcommand{\xmark}{\ding{55}}

\newcommand{\layer}{\mathcal{H}}
\newcommand{\defeq}{\triangleq}
%\newcommand{\defeq}{vcentcolon=}
\newcommand{\domain}{\Omega}
\newcommand{\grad}{\nabla}

\newcommand{\cin}{c_{\rm{in}}}
\newcommand{\cout}{c_{\rm{out}}}
\newcommand{\intdomain}{\int_{\domain}}
\newcommand{\network}{\gT}
\newcommand{\subnet}{\gK}
\newcommand{\map}{\gR} %\gR

\newcommand{\innerproduct}[2]{\langle #1, #2 \rangle}
\newcommand{\mcsum}[1][j]{\frac{1}{N}\sum_{#1=1}^N}

\newcommand{\inrspace}[1][c]{\gF_{#1}}

\DeclareMathOperator*{\argmax}{arg\,max}
\DeclareMathOperator*{\argmin}{arg\,min}

\let\ab\allowbreak


\DeclareMathOperator*{\argminA}{arg\,min}

\definecolor{myPurple}{rgb}{0.4, .0, .8}
\definecolor{myGreen}{rgb}{0, .8, .3}
\definecolor{myRed}{rgb}{0.8, .2, .2}
\definecolor{myOrange}{rgb}{0.7, 0.45, 0.2}
\definecolor{myBlue}{rgb}{.0, .0, 1.0}
\definecolor{myBlue2}{rgb}{.0, .0, 0.5}
\definecolor{myBlack}{rgb}{.0, .0, 0.0}
\newcommand{\lkm}[1]{{\color[rgb]{0.2,0.8,0.2}LKM: #1}}


\begin{document}

% \title{Iterative Reconstruction Based on Latent Diffusion Model for Sparse Data Reconstruction}
\title{Fast and Stable Diffusion Inverse Solver with History Gradient Update}

\author{Linchao He, Hongyu Yan, Mengting Luo, Hongjie Wu, Kunming Luo, Wang Wang, Wenchao Du, Hu Chen~\IEEEmembership{Member,~IEEE,}, Hongyu Yang, Yi Zhang~\IEEEmembership{Senior Member,~IEEE,}, Jiancheng Lv~\IEEEmembership{Senior Member,~IEEE,}
        % <-this % stops a space
\thanks{Corresponding author: Hu Chen e-mail: huchen@scu.edu.cn). Linchao He and Hongyu Yan have equal contribution.}% <-this % stops a space
\thanks{Linchao He and Mengting Luo are with the Department of National Key Laboratory of Fundamental Science on Synthetic Vision, Sichuan University, Chengdu 610065, China. (e-mail: hlc@stu.scu.edu.cn).}
\thanks{Yi Zhang is with the School of Cyber Science and Engineering, Sichuan University, Chengdu 610065, China (e-mail: yzhang@scu.edu.cn).}
\thanks{Linchao He, Mengting Luo, Hongjie Wu, Wenchao Du, Hu Chen, Jiancheng Lv, and Hongyu Yang are with the College of Computer Science, Sichuan University, Chengdu 610065, China. (e-mail:  \{hlc, lmt\}@stu.scu.edu.cn; \{wuhongjie0818, wenchaodu.scu\}@gmail.com; \{huchen, lvjiancheng\}@scu.edu.cn; yanghongyu\_scu@163.com).}
\thanks{Hongyu Yan and Kunming Luo are with the Hong Kong University of Science and Technology, Hong Kong, China. (e-mail: \{hyanar, kluoad\}@connect.ust.hk)}
\thanks{Wang Wang is with the Institute of Space and Earth Information Science, The Chinese University of Hong Kong, Shatin, N.T., Hong Kong, China. (e-mail: wangw00821@gmail.com)}}

% The paper headers
\markboth{Journal of \LaTeX\ Class Files,~Vol.~14, No.~8, August~2021}%
{Shell \MakeLowercase{\textit{et al.}}: A Sample Article Using IEEEtran.cls for IEEE Journals}

%\IEEEpubid{0000--0000/00\$00.00~\copyright~2021 IEEE}
% Remember, if you use this you must call \IEEEpubidadjcol in the second
% column for its text to clear the IEEEpubid mark.

\maketitle

\begin{abstract}
%Reconstructing Computed tomography (CT) images from sparse measurement is a well-known ill-posed inverse problem with limited dataset available. The Iterative Reconstruction (IR) algorithm is a solution to inverse problems. However, previous IR methods require large-scale paired data to train their models. To address the above problem, we present Latent Diffusion Iterative Reconstruction (LDIR), a pioneering zero-shot method that only depends on a small-scale unpaired dataset to achieve better performance than supervised method. By approximating the prior distribution with an unconditional latent diffusion model, LDIR is the first method to successfully integrate diffusion-based inverse solvers with LDM in an unsupervised manner. Moreover, LDIR utilizes the gradient from the data-fidelity term to guide the sampling process of the LDM which make the process more stable and efficiency, therefore, LDIR does not need the approximation of the inverse projection matrix and can solve various CT reconstruction tasks with a single model. Additionally, for enhancing the sample consistency of the reconstruction, we introduce a novel approach that uses historical gradient information to guide the gradient. Our experiments on extremely sparse CT data reconstruction tasks show that LDIR outperforms other state-of-the-art unsupervised and even exceeds supervised methods, establishing it as a leading technique in terms of both quantity and quality. Furthermore, LDIR also achieves competitive performance on nature image tasks. It is worth noting that LDIR also exhibits significantly faster execution times and lower memory consumption compared to methods with similar network settings. 
% Diffusion models have recently been recognized as efficient inverse problem solvers because of their ability to produce high-quality reconstruction results without relying on training on paired data. However, existing diffusion-based solvers only use pixel-based diffusion models as their prior models, which result in failure to learn effective data distributions from small-scale data. To address this issue, we propose a method for diffusion-based solvers, which incorporates latent-based prior models with these solvers while improves overall reconstruction efficiency. This allow us to recover high-quality data even for small-scale dataset which is very important for medical applications. Furthermore, we propose a new method which utilizes history gradient from previous solving steps to guide current optimization. This approach can make the whole solve process stable and the final reconstruction having better quality. This approach is suitable to all pixel-based and latent-based diffusion solvers, owing to it only change the gradient decent process. By combining the above techniques, we propose a new solver, named Latent Diffusion Iterative Reconstruction (LDIR), which outperforms other state-of-the-art unsupervised and even exceeds supervised methods on small-scale medical dataset on a variety of tasks and also presents competitive performance on nature image tasks.
Diffusion models have recently been recognised as efficient inverse problem solvers due to their ability to produce high-quality reconstruction results without relying on pairwise data training. Existing diffusion-based solvers utilize Gradient Descent strategy to get a optimal sample solution.
However, these solvers only calculate the current gradient and have not utilized any history information of sampling process, thus resulting in unstable optimization progresses and suboptimal solutions. To address this issue, we propose to utilize the history information of the diffusion-based inverse solvers. In this paper, we first prove that, in previous work, using the gradient descent method to optimize the data fidelity term is convergent. Building on this, we introduce the incorporation of historical gradients into this optimization process, termed History Gradient Update (HGU). We also provide theoretical evidence that HGU ensures the convergence of the entire algorithm. It's worth noting that HGU is applicable to both pixel-based and latent-based diffusion model solvers. Experimental results demonstrate that, compared to previous sampling algorithms, sampling algorithms with HGU achieves state-of-the-art results in medical image reconstruction, surpassing even supervised learning methods. Additionally, it achieves competitive results on natural images.
\end{abstract}

\begin{IEEEkeywords}
Diffusion model, inverse problem, CT reconstruction.
\end{IEEEkeywords}

% Figure environment removed

\section{Introduction}
%Computed tomography (CT) is a crucial medical imaging technique in contemporary medicine to aid physicians in diagnosing relevant conditions. Measurements in CT are obtained by X-rays projections of an object from different views. However, the use of X-rays in CT exposes the human body to potentially harmful doses of radiation, raising concerns among the public regarding radiation-induced diseases. Therefore, reducing the exposure dose, such as sparse-view and limited-angle imaging, while maintaining the quality of imaging has beneficial implications for both public health and medical diagnosis, specifically in intraoperative CT. Due to the sparse information, the CT reconstruction processes are well-known ill-posed inverse problems.
%In the past decade, numerous works have focused on Iterative Reconstruction (IR), which is considered a solution to CT reconstruction \cite{donoho2006compressed, candes2006robust}. Iterative reconstruction aims to recover signals $x$ from noisy measurements $y = \mathbf{A}x + n$, where $n$ represents the noise in the measuring process and $\mathbf{A}$ is the linear projection matrix that typically maps $x$ to a lower dimension. As a result, a typical IR process can be formulated as follows:
%\begin{align}
%	\hat{x} = \argminA_x \left|\left|\mathbf{A}x - y\right|\right|_2^2 + \lambda R\left(x\right),
%	\label{eq:ir_formualtion}
%\end{align}
%here, $\left|\left|\mathbf{A}x - y\right|\right|_2^2$ is the data-fidelity term that ensures the reconstructed results are consistent with the measurements, while $\lambda R\left(x\right)$ is a prior term that ensures the reconstructed results are realistic and follow the distribution $p\left(x\right)$ of the ground truth images. 
Inverse problem solving is of paramount significance, as it typically entails the recovery of missing data from sparse measurement $\y$ which is usually formulated as:
\begin{align}
	\hat{\x} = \argmin_\x \left|\left|\Ac \x - \y\right|\right|_2^2 + \lambda R\left(\x\right),
	\label{eq:ir_formualtion}
\end{align}
where $\left|\left|\Ac \x - \y\right|\right|_2^2$ is the data-fidelity term that ensures the reconstructed results are consistent with the measurements, while $\lambda R\left(\x\right)$ is a prior term that ensures the reconstructed results are realistic and follow the ground truth image distribution $p\left(\x\right)$. Due to the sparsity of $\y$, the solving processes are well-known ill-posed problems. Therefore, it is important to take advantage of good prior models to generate high-quality results. Commonly, pre-trained generative models are used as the prior models, including GANs~\cite{bora2017compressed,goodfellow2014generative}, VAEs~\cite{bora2017compressed,kingma2013auto}, and diffusion models~\cite{ho2020denoising,chung2022improving,song2021solving}.

% Previous work on optimization problems within diffusion-based inverse problem solvers, particularly with respect to the data fidelity term optimization, has traditionally employed the most basic Gradient Descent (GD) strategy~\cite{chung2022improving,chung2022solving,chung2022diffusion,Song2023PseudoinverseGuidedDM}. However, the optimization process of the data fidelity term is a stochastic process which is naturally unstable. And, the GD strategy ignores the historical gradient and only depends the gradient in a single step which also leads to be unstable. These factors make the current diffusion-based solvers slowly convergence and unstable optimization which finally lead to suboptimal samples. Furthermore, current research has yet to explore this issue, and no studies have provided evidence of the convergence properties of the GD strategy when optimizing the data fidelity term.

% Previous work on optimization problems within diffusion-based inverse problem solvers, particularly with respect to the data fidelity term optimization, has traditionally employed the most basic Gradient Descent (GD) strategy~\cite{chung2022improving,chung2022solving,chung2022diffusion,Song2023PseudoinverseGuidedDM}. However, the loss-guided optimization process of the data fidelity term is usually a stochastic process which is naturally unstable. Furthermore, the GD strategy only depends the gradient in a single step to optimize the target. These above factors make the current diffusion-based solvers hard to optimize and finally lead to inaccurate samples. One of solutions for this issue is to integrate historical gradient information to the optimization processes which can make the processes more stable and yield high-quality samples. However, current researches have yet to explore the historical gradient information, and no studies have provided evidence of the convergence properties of the GD strategy when optimizing the data fidelity term.

Previous research on optimization problems in diffusion-based inverse problem solvers, specifically with regards to optimizing the data fidelity term, has traditionally utilized the most basic Gradient Descent (GD) strategy~\cite{chung2022improving,chung2022solving,chung2022diffusion,Song2023PseudoinverseGuidedDM} to obtain the generated result. However, the loss-guided optimization process of the data fidelity term is typically a stochastic process that is naturally unstable. Additionally, the GD strategy only relies on the gradient in a single step to optimize the target. These facts make the current diffusion-based solvers difficult to optimize and ultimately lead to inaccurate samples. One solution to this issue is to incorporate historical gradient information into the optimization processes, which can make the processes more stable and produce high-quality samples. However, current research has yet to explore the historical gradient information, and no studies have provided evidence of the convergence properties of the GD strategy when optimizing the data fidelity term.

% These factors make the current diffusion-based solvers slowly convergence and unstable optimization which finally lead to suboptimal samples. Furthermore, current research has yet to explore this issue, and no studies have provided evidence of the convergence properties of the GD strategy when optimizing the data fidelity term.

Pixel-based diffusion models have been widely used to solve inverse problems in recent research~\cite{lugmayr2022repaint,song2021solving,chung2022diffusion,kawar2022denoising,chung2022improving,wang2022zero,Song2023PseudoinverseGuidedDM}. These models can effectively act as prior models and generate high-quality samples in an iterative optimization framework. However, they have a significant limitation: they can learn accurate prior information from large-scale data, but inaccurate from small-scale data. This limits their efficiency and applicability to data-constrained tasks, such as medical image inverse problems. To the best of our knowledge, no previous work has investigated the applicability of diffusion-based solvers on small-scale datasets.
% All of them have only considered large-scale datasets.

% Previous work on optimization problems within diffusion-based inverse problem solvers, particularly with respect to data fidelity term optimization, has traditionally employed the most basic Gradient Descent (GD) strategy. Similar to training neural networks, the use of GD strategy may render the optimization process unstable, thereby impacting the final optimization outcome. This issue becomes more pronounced in the context of inverse problem solving with limited data scales. Furthermore, current research has yet to explore this issue, and no studies have provided evidence of the convergence properties of the GD strategy when optimizing the data fidelity term.

In this paper, we aim to address the above issues by introducing a new optimization method, which greatly expands the applicability of diffusion-based solvers. Specifically, \textbf{(1)} to address the optimization problems within diffusion-based solvers, we prove the evidence of the convergence properties of the GD strategy. \textbf{(2)} Based on this evidence, we develop a new strategy for the optimization of data fidelity term which use the historical gradient from previous optimization steps to adaptively adjust sample-level gradient, namely History Gradient Update (HGU), thereby stabilizing the whole optimization process and improving the quality of final samples. \textbf{(3)} We introduce the latent diffusion model (LDM) to the diffusion-based solver, such that we can learn accurate prior information from small-scale data.

% Specifically, \textbf{1)} we introduce the latent diffusion model (LDM) to diffusion-based solver, such that we can learn accurate prior information from small-scale data. Furthermore, \textbf{2)} to address the optimization problems within diffusion-based solvers, we proves the evidence of the convergence properties of the GD strategy. \textbf{3)}  Based on this evidence, we develop a new guidance strategy for the optimization of data fidelity term which use the history gradient from previous optimization steps to adaptively adjust sample-level gradient, thereby stabilizing the whole optimization process and improving the quality of final reconstructions. \textbf{4)} We named the framework that includes the above methods Latent Diffusion Iterative Reconstruction (LDIR).

Extensive experiments of various inverse problems demonstrate that HGU with LDMs outperforms state-of-the-art supervised and unsupervised methods on small-scale medical datasets. Additionally, as a general method of inverse problem solvers, we extend HGU to the natural image restoration task and HGU achieves competitive performance compared to other state-of-the-art zero-shot methods. HGU provides a valuable optimization tool for solving the inverse problem, allowing the community to solve the inverse problem for different modalities without regrading the scale of the dataset and leveraging the vast amount of available LDMs such as Stable Diffusion. Fig.~\ref{fig:coverfig} shows some representative visual results of the proposed method.

% The key challenge of IR is to find appropriate data prior or sparse transformation to generate the prior term. While some representative works have been proposed to address this problem, such as total variation, nonlocal means, wavelets, and low rank methods, it remains challenging to obtain reconstruction results with high consistency and realism. These methods rely on designing a handcrafted prior that is meant to apply to all conditions, but achieving this is essentially impossible. Moreover, these methods require a backprojection operation to maintain data consistency, which can be challenging to perform accurately. In recent years, data-driven methods have demonstrated significant advantages over handcrafted priors in solving inverse problems in iterative reconstruction~\cite{bora2017compressed,chen2018learn,song2021solving}. Most of these methods require large-scale paired data to train their networks which directly project the measurement to the results and they also need to retrain while the detector geometry changes. Although~\cite{song2021solving} has addressed the above problems, they still need to know the inverse matrix of the projection matrix which is difficult to be obtained in the real-world application. Additionally, they use direct projection to replace the data-fidelity term which makes they make them need to design different sampling procedures for different detector geometries.

% However, it is still a challenge problem to learn generalized prior from data for various inverse problems.
% To address these issues, a zero-shot diffusion-based inverse solvers~\cite{song} have been proposed. This method condition the data-fidelity term on a pretrained unconditional diffusion model, enabling the use of a single model to solve various inverse problems of nature images. However, they mainly focus on solving nature image inverse problems at the pixel level and make strong assumptions about the data-fidelity term, which significantly underestimates the complexity of real-world problems. 

%The key challenge of IR is to find appropriate data prior or sparse transformation to generate the prior term. Traditional IR methods~\cite{beck2009fast,kim2016non,zhao2000x} leverage the total variation, nonlocal means, or wavelets to gain the reconstructed results by a handcrafted prior and the approximation of the inverse projection matrix. Inspired by the success of deep learning and neural network, recent data-driven methods~\cite{bora2017compressed,chen2018learn} achieve impressive performance by learning the prior and data-fidelity terms. However, most of these methods require large-scale paired data to train their networks. Besides, they directly project the measurement to the results and they also need to retrain while the detector geometry changes. Although method~\cite{song2021solving} solves these problems mentioned above by introducing a score-based generative model, it still needs to know the approximation of the inverse projection matrix which is difficult to be obtained in the real-world application. Additionally, these methods use direct projection to replace the data-fidelity term which makes they make them need to design different sampling procedures for different detector geometries.

%	Data-driven methods~\cite{bora2017compressed,chen2018learn} has been proven that they have a significant advantage over the handcrafted priors. Very recently, diffusion model techniques have been introduced into compressive sensing field and achieve impressive performance. These methods~\cite{rombach2022high,saharia2022image,gao2023implicit} use conditional diffusion models to iteratively denoise the Gaussian noise to obtain the reconstructions. However these approaches are limited, since they rely on the conditional diffusion models which need paired data to train and can only handle a specific task without retraining. Thus, many zero-shot diffusion-based inverse solvers~\cite{lugmayr2022repaint,song2021solving,kawar2022denoising,chung2022diffusion,chung2022improving,wang2022zero} have proposed to address the above issues. They condition the data-fidelity term rather than the prior term which is a unconditional diffusion model pretrained on a target dataset. They make it success to utilize a single pretrained diffusion model to solve various inverse problems. However, they all focus on solving the inverse problems on the pixel-level space and they also make a strong assumption on the data-fidelity term, which significantly underestimate the complexity of real-world problem.

%	In this work, we propose Deep Latent Iterative Reconstruction (LDIR), a method that extends the traditional IR with a latent diffusion model for sparse data restoration. Through guiding the reverse diffusion sampling on the latent-level space, we offer a zero-shot tool to simultaneously generate high-resolution and multiple images with high consistency and realness. Since LDIR do not make any assumption on the data-fidelity term, we can use any measurement function to keep the data consistency even with the network-based functions (Perceptual loss~\cite{johnson2016perceptual} and LPIPS~\cite{zhang2018perceptual} or the frequency-based function~\cite{jiang2021focal}) as long as the functions are differentiable. In addition, we propose novel guidance policies which utilize the history gradient to fix the current gradient, thereby making the above zero-shot diffusion guidance method performing better. Our approach offers the community a useful zero-shot tool toward solving inverse problem with latent diffusion models, as the latent diffusion models have been a necessary component for current pretrained AIGC models.

% In this paper, we introduce Latent Diffusion Iterative Reconstruction (LDIR), a novel zero-shot method that extends traditional IR techniques by incorporating a latent diffusion model and do not need any paired data to train network, for sparse CT data reconstruction. By guiding the reverse diffusion sampling process in the latent space without the inverse of the measurement matrix, our zero-shot method can generate high-resolution images with better performance. Since LDIR do not make any assumption on the data-fidelity term, we can use any measurement function to keep the data consistency as long as the functions are differentiable. Moreover, we propose novel guidance policies that leverage the history gradient to fix the current gradient, thereby improving the performance of our zero-shot diffusion guidance method. Our method achieves state-of-the-art performance for extremely sparse CT data reconstruction, outperforming both supervised and unsupervised learning methods. Additionally, due to the commonality of iterative reconstruction on CT images and natural images, we extend LDIR to the natural image restoration task, and our approach achieves competitive performance compared to other state-of-the-art zero-shot methods. Our approach provides a valuable zero-shot tool for solving inverse problems with latent diffusion models, allowing us to leverage the vast amount of available latent diffusion models.

%In this paper, we introduce a novel Latent Diffusion Iterative Reconstruction (LDIR) for sparse CT data reconstruction in a zero-shot manner. The LDIR is the first to extend traditional IR techniques by incorporating a latent diffusion model. Specifically, we train a latent unconditional diffusion model to learn the data distribution. In the reverse process, the trained diffusion model is utilized to replace the prior term in the regular IR. In particular, to generate specified prior from the unconditional diffusion model, the gradient from the data-fidelity term is applied to guide the sampling process of the diffusion model. Therefore, we do not need any paired data or inverse of the measurement matrix to train our network. Moreover, LDIR can solve various CT reconstruction tasks with a single model. In addition, by guiding the reverse diffusion sampling process in the latent space, our zero-shot method can generate high-resolution images with impressive performance. Since LDIR does not make any assumption on the data-fidelity term, we can use any differentiable measurement function to keep the consistency of data. Further on, we propose a novel guidance strategy to adaptively adjust the sample-level gradient by fusing the history gradient, thereby improving the performance of our zero-shot diffusion model.

% In summary, our work makes the following \textbf{contributions}:

%	(i) Based on traditional iterative reconstruction process, we propose a novel iterative reconstruction method with diffusion model as a better data prior. The diffusion model do not need paired data to train, thus, our method can work as a zero-shot reconstruction method. Our method also do not need to know the inverse of the projection matrix or need to retrain the networks when the projection matrix changes. Also, we do not need to make any noise assumption on the data-fidelity term.

% (i) In this paper, we propose a novel iterative reconstruction method that incorporates a diffusion model as a more effective data prior compared to traditional iterative reconstruction processes. Our method operates as a zero-shot reconstruction method since the diffusion model does not require paired data for training. Furthermore, our approach does not require knowledge of the inverse of the projection matrix or retraining of the networks when the projection matrix changes. Additionally, our method does not rely on any assumptions in the data-fidelity term.

%	(ii) To the best of our knowledge, this is the first time that zero-shot diffusion-based IR methods are extended from the pixel space to the latent space. Compared to pixel-based approaches, we significantly decrease computational costs while achieve better performance and nearly $2\times$ speed up with similar network settings. We can also restore high-resolution images by processing in the latent space.
% (ii) To the best of our knowledge, this is the first time that zero-shot diffusion-based IR methods is extended from the pixel space to the latent space. In comparison to pixel-based approaches, our method significantly reduces computational costs while achieving better performance and nearly doubling the speed with similar network settings. Moreover, our approach can restore high-resolution images by processing in the latent space, demonstrating the effectiveness of our method.

% (iii) We show that, following the huge successes of optimizers, we can leverage the history gradient information to further enhance the quality of the reconstruction results with minimal computational cost.

%	(iv) We achieve state-of-the-art performance on the sparse medical data reconstruction over other supervised learning and unsupervised learning competitors. Additionally, on the nature image restoration task (super-resolution), we achieve the competitive performance with other state-of-the-art zero-shot methods.

% (iv) Our method achieves state-of-the-art performance for sparse medical data reconstruction, outperforming both supervised and unsupervised learning methods. Additionally, we extend LDIR to nature image restoration task, and our approach achieves competitive performance compared to other state-of-the-art zero-shot methods.

\section{Background}

\subsection{Diffusion models}
Consider a Gaussian diffusion process, where $\x_t \in \Rd^n, t\in[0,T]$ and initial $\x_0$ is sampled from the original data distribution $P_\text{data}$. We define the forward diffusion process using stochastic differential equation (SDE)~\cite{song2021scorebased}:
\begin{align}
	d\x = f\left(\x,t\right) \, dt + g\left(t\right) \, d\w,
	\label{eq:forward_sde}
\end{align}
where $f\left(\cdot,t\right): \, \mathbb{R}^n \to \mathbb{R}^n > 0$ is a drift coefficient function, $g\left(t\right) \in \mathbb{R}$ is defined as a diffusion coefficient function, and $\w \in \mathbb{R}^n$ is a standard $n$-dimensional Brownian motion. Thus, the reverse SDE of Eq.~\eqref{eq:forward_sde} can also defined as:
\begin{align}
	d\x = \left[f\left(\x_t,t\right)-g\left(t\right)^2\nabla_{\x_t} \log p_t\left(\x_t\right)\right] \, dt + g\left(t\right) \, d\bar{\w},
	\label{eq:reverse_sde}
\end{align}
where $dt$ is a negative infinitesimal time step and $d\bar{\w}$ is the backward process of $d\w$. The reverse SDE defines a generative process that transforms standard Gaussian noise into meaningful content. To accomplish this transformation, the score function $\nabla_{\x_t} \log p_t\left(\x_t\right)$ needs to be matching, which is typically replaced with $\nabla_{\x_t} \log p_{0 | t}\left(\x_t \middle| \x_0\right)$ in practice. Therefore, we can train a score model $\s_\theta\left(\x_t,t\right)$, so that $\s_\theta\left(\x_t,t\right) \approx \nabla_{\x_t} \log p_t\left(\x_t\right) \approx \nabla_{\x_t} \log p_{0 | t}\left(\x_t \middle| \x_0\right)$ using the following score-matching objective:
\begin{multline}
	\label{eq:sde_objective}
	\min_\theta \Ed_{t \in \left[1,\dots,T\right], \x_0 \sim P_\text{data}, \x_t \sim p_{0|t}\left(\x_t \middle| \x_0 \right)} \\ \left[ \left|\left| \s_\theta\left(\x_t,t\right)- \nabla_{\x_t} \log p_{0|t}\left(\x_t \middle| \x_0 \right) \right|\right| _2^2\right] \, .
\end{multline}
Subsequently, the reverse SDE can yield meaningful contents $\x_0 \sim P_\text{data}$ from random noises $\x_{T} \sim \Nc(\bm{0}, \Ib)$ by iteratively using $\s_\theta \left(\x_t,t\right)$ to estimate the scores $\nabla_{\x_t} \log p_t\left(\x_t\right)$. In our experiments, we adopt the standard Denoising Diffusion Probabilistic Models (DDPM)~\cite{ho2020denoising} which is equivalent to the above variance preserving SDE (VP-SDE)~\cite{song2021scorebased}.

\subsection{Diffusion model for inverse problem solving}
To solve the inverse problems using the diffusion model, various workarounds are proposed~\cite{rombach2022high,saharia2022image,gao2023implicit,luo2023image}. These methods use conditional diffusion models to iteratively denoise Gaussian noise and obtain reconstructed samples. However, these approaches have limitations, as they rely on conditional diffusion models that require paired data for training and can only handle specific tasks without retraining. To address these issues, several zero-shot diffusion-based inverse solvers~\cite{lugmayr2022repaint,song2021solving,kawar2022denoising,chung2022improving,wang2022zero} have been proposed. Typically, for each denoising step, they~\cite{lugmayr2022repaint,song2021solving,wang2022zero} unconditionally estimate new denoised samples based on the previous step, followed by replacing the corresponding items in the denoised samples using the measurement $\y$, which is also known as range-null space decomposition~\cite{wang2022zero}. This approach ensures data consistency, but it fails in the case of noisy measurements, since $\Ac^{-1} \y$ is not a correct corresponding item for the denoised samples. To deal with noisy measurements, alternative methods~\cite{chung2022diffusion,chung2022parallel} have been proposed to solve the inverse problems with noised measurements. Rather than directly replace items, these approaches use the gradient of $\left|\left|\y-\Ac \x \right|\right|_2^2$ to conditionally guide the generative process. These methods are robust to noise and can process nonlinear projection operators, which can be formulated as
\begin{align}
	& \nabla_{\x_t} \log p(\y|\x_t) \simeq \nabla_{\x_t} \log p\left(\y|\Ed \left[\x_0|\x_t\right] \right), \nonumber \\
	& \Ed \left[\x_0|\x_t\right] = \frac{1}{\sqrt{\bar{\alpha}}} \left(\x_t-(1-\bar{\alpha})\s_\theta(\x_t,t)\right).
	\label{eq:dps}
\end{align}

%However, these methods try to solve inverse problems on the pixel space and make strong assumptions on the data-fidelity term, which significantly underestimates the complexity of real-world problems and can not reconstruct high-resolution results.
Despite the achievements, these methods try to solve the inverse problem on the pixel space which are very effective when having large-scale training dataset for DDPM models, while obtaining bad performance on small-scale datasets which limits their availability in solving the inverse problem of medical imaging.

\section{Method}

\subsection{Latent diffusion solver}
% Our review of previous works on diffusion-based data reconstruction, including~\cite{chung2022diffusion,chung2022improving,chung2022parallel,chung2022solving,song2021solving,wang2022zero}, reveals that they all perform reconstruction in the pixel space, which requires significant computational resources and are sensitive to the scale of data. To address this limitation, we draw inspiration from the Latent Diffusion Models (LDMs) proposed by~\cite{rombach2022high}. We propose a new inverse solver named Latent Diffusion Solver (LDS) which aims to solve general inverse problems on the latent space.

% Our review of previous works on diffusion-based data reconstruction, including~\cite{chung2022diffusion,chung2022improving,chung2022parallel,chung2022solving,song2021solving,wang2022zero}, reveals that they all perform reconstruction in the pixel space, which requires significant computational resources and are sensitive to the scale of data. To address this limitation, we draw inspiration from the Latent Diffusion Models (LDMs) proposed by~\cite{rombach2022high}. We propose a new inverse solver named Latent Diffusion Solver (LDS) which aims to solve general inverse problems on the latent space.

Previous works on diffusion-based inverse problem solving, such as~\cite{chung2022diffusion,chung2022improving,chung2022parallel,chung2022solving,song2021solving,wang2022zero}, have solved the problems in the pixel space, which is sensitive to the scale of data. To overcome this limitation, we are inspired by the Latent Diffusion Models (LDMs)~\cite{rombach2022high}. We introduce a new inverse solver called Latent Diffusion Solver (LDS) that aims to solve general inverse problems on the latent space for small-scale datasets.
\begin{assumption}\label{autoencoder}
For paired encoder $\mathcal{E}$ and decoder $\mathcal{D}$, $\mathcal{E}$ can compress any data $\x \sim p_\text{data}$ into a distinct low-dimensional latent $\z \sim p_\text{latent}$, and subsequently $\mathcal{D}$ can restore $\z$ from $\x$.
\begin{equation}
\z = \mathcal{E}(\x), \x = \mathcal{D}(\z), \; \; \; x \sim p_\text{data}
\end{equation}	
\end{assumption}
Although the above Assumption~\ref{autoencoder} is a strong assumption, in real-world applications, variational autoencoders~\cite{kingma2013auto,van2017neural} have revealed their strong performance in compressing and restoring. When both the encoder and decoder are well-trained, we posit that the aforementioned assumptions hold. So that we can obtain a score model on latent space denoted as $\epsilon^\z_\theta$ as shown in Proposition~\ref{scoremodel}.
\begin{proposition}\label{scoremodel}
	Considering a data $\x \sim p_\text{data}$, we can get its unique latent $\z$ by $\z = \mathcal{E}(\x)$. By minimizing the below score matching function, we can get $\epsilon^\z_\theta$:
	\begin{multline}
		\min_\theta \Ed_{t \in \left[1,\dots,T\right], \z_0 \sim p_\text{latent}, \z_t \sim p_{0|t}\left(\z_t \middle| \z_0 \right)} \\ \left[ \left|\left| \s^\z_\theta\left(\z_t,t\right)- \nabla_{\z_t} \log p_{0|t}\left(\z_t \middle| \z_0 \right) \right|\right| _2^2\right] \, , \z = \mathcal{E}(\x).
	\end{multline}
\end{proposition}
Also, given the well-trained decoder $\mathcal{D}$, we can have a new data consistency term on the latent space as shown in the following Proposition.
\begin{proposition}	\label{dataconsistency}
	Considering a latent $\z_t \sim p_\text{latent}(z_i|z_0)$ where $i \in [0, 1]$ and $\z_0 \sim p_\text{latent}$, the data consistency term on the latent space can be computed by 
	\begin{align}
		& p(\y| \z_i) = \gU (\y, \Ac \mathcal{D}\left(\z_i\right)).
	\end{align}
\end{proposition}
Based on the above assumption and proposition, we can solve inverse problems on the latent space without losing any generalization ability. Similarly, we can use the posterior mean method from DPS~\cite{chung2022diffusion} to replace $\z_i$ with $\hat{\z}_0$ to derive LDS. The following theorem indicates this manner.
\begin{theorem}[Latent Diffusion Solver]\label{eq:lds}
Considering Assumption~\ref{autoencoder}, Proposition~\ref{scoremodel} and~\ref{dataconsistency}, we can derive Eq.~\ref{eq:dps} to the latent space. Formally,
	\begin{align}
		\nabla_{\z_t} \log p(\y|\z_t) & \simeq \nabla_{\z_t} \log p\left(\y|\Ed \left[\z_0|\z_t\right] \right) \nonumber \\
		& =\gU\left( \y, \Ac \mathcal{D}\left(\Ed \left[\z_0|\z_t\right]\right) \right), \nonumber \\
		\Ed \left[\z_0|\z_t\right] &= \frac{1}{\sqrt{\bar{\alpha}}} \left(\z_t-(1-\bar{\alpha})\s^\z_\theta(\z_t,t)\right), \\
		\nabla_{\z_t} \log p(\z_t|\y) & \simeq \s^\z_\theta(\z_t,t) - \epsilon \nabla_{\z_t} \gU \left( \y, \Ac \mathcal{D}\left(\Ed \left[\z_0|\z_t\right]\right) \right)
	\end{align}
where $\epsilon$ is the guidance rate, which can balance the realness and consistency of results, and $\gU$ is the evaluation function which evaluates the difference between the measurement $\y$ and predicted measurement $\Ac \mathcal{D}\left(\Ed \left[\z_0|\z_t\right]\right)$.
\end{theorem}

%% Figure environment removed

% Figure environment removed

\subsection{History gradient update}

% Figure environment removed

% Based on our latent diffusion solver, we propose a new framework to solve inverse problems, named Latent Diffusion Iterative Reconstruction (LDIR). Here, we present how our LDIR to iteratively solve inverse problem with two different history gradient update strategies.

% To utilize the historical gradient information, we propose a new optimization method for diffusion-based solvers, named History Gradient Update. Here, we present how our HGU solves inverse problem with two different gradient update variants.

Latent diffusion solvers typically generate superior samples~\cite{rombach2022high}. However, both pixel and latent-based diffusion solvers still face challenges from unstable optimization processes and the gradient descent strategy, which result in sub-optimal samples. To improve the stability of solvers and achieve the best quality samples, we propose a new optimization method for diffusion-based solvers, called History Gradient Update. In this section, we explain how our HGU solves the inverse problem with two variants.

\textbf{Momentum-variant. } Given a well-trained latent-based score model $\s^\z_\theta$, we have the posterior mean:
\begin{equation}
	\Ed \left[\z_0|\z_t\right] = \frac{1}{\sqrt{\bar{\alpha}}} \left(\z_t-\sqrt{1-\bar{\alpha}}\s^\z_\theta(\z_t,t)\right), t \in [0, T],
\end{equation}
where $\z_T \sim \Nc(\bm{0}, \Ib)$. Thus we can add historical gradient to the reconstruction as:
\begin{align}
	\z_{t-1} &= \z'_{t-1} - \eps_t \m^t, \\
	\z'_{t-1} &= \frac{\sqrt{\alpha_t}(1-\bar{\alpha}_{t-1})}{1-\bar{\alpha}_t}\z_t + \frac{\sqrt{\bar{\alpha}_{t-1}}\beta_t}{1-\bar{\alpha}_t}\Ed \left[\z_0|\z_t\right] + g(t)\kb ,\\
	\m_t &= \eta \m_{t-1} + (1-\eta) \nabla_{\z_t} \gU \left( \y, \Ac \mathcal{D}\left(\Ed \left[\z_0|\z_t\right]\right) \right),
\end{align}
where $\kb \sim \Nc(\bm{0}, \Ib)$. The final results can be obtained by decoding the latent $\z_0$ as $\hat{\x}_0 = \mathcal{D}(\z_0)$.

\textbf{Improved-Momentum-variant. } Similarly to the Momentum-variant, we have the same predicted posterior mean $\Ed \left[\z_0|\z_t\right]$. The Momentum-variant only considers the first momentum information. Thus, we develop a new variant, named Improved-Momentum-variant HGU (iGDM), which includes the both first momentum and second momentum information~\cite{kingma2014adam}. We derive the Improved-Momentum-variant HGU as:
\begin{align}
	\z_{t-1} &= \z'_{t-1} - \eps_t \frac{\m^t}{\sqrt{\vbb^t} + \varepsilon}, \\
	\m_t &= \eta_1 \m_{t-1} + (1 - \eta_1) \gU \left( \y, \Ac \mathcal{D}\left(\Ed \left[\z_0|\z_t\right]\right) \right), \\
	\vbb_t &= \eta_2 \vbb_{t-1} + (1 - \eta_2) \gU \left( \y, \Ac \mathcal{D}\left(\Ed \left[\z_0|\z_t\right]\right) \right)^2,
\end{align}
where $\varepsilon$ helps improve the numerical stability. The detail pseudocode of the above variant HGU is provided in Algorithm~\ref{algo:ldir_momentum} and~\ref{algo:ldir_adam}.

% It should be noted that our iGDM and Adam~\cite{kingma2014adam} both use the first and second momentum to optimize variables. However, Adam is designed to be an optimizer for the optimization of the neural network training, while, our iGDM is designed to keep the measurement consistent during the solving process. Thus, based on their purposes, there are two major differences between our iGDM and Adam: (1) the Adam optimizer assumes that the parameters in the neural network are following the $L2$ distribution, so, Adam applies a weight decay algorithm to keep this assumption which is usually set to a non-zero value. While, our iGDM does not make a prior assumption on the variable, because the prior is approximated by the diffusion models. Thus, iGDM does not include the weight decay algorithm. (2) The second-momentum in the Adam optimizer is usually unstable which needs a term added to the denominator to improve numerical stability. The value of this term would greatly influence the performance of the neural network. However, this term in iGDM has little influence on the performance (shown in Tab.~\ref{tab:function_optim}) which may be because our algorithm can provide a stable second moment.

It should be noted that our iGDM and Adam~\cite{kingma2014adam} both use the first and second momentum to optimize variables. However, Adam is designed as an optimizer for neural network training, while our iGDM is designed to maintain consistent measurements during the solving process. Therefore, based on their purposes, there are two major differences between our iGDM and Adam: (1) The Adam optimizer assumes that the parameters in the neural network follow the $L2$ distribution, so it applies a weight decay algorithm to preserve this assumption. In contrast, our iGDM does not make any prior assumption on the variables, because the prior is approximated by the diffusion models. Thus, iGDM does not include the weight decay algorithm. (2) The second momentum in the Adam optimizer is often unstable and requires a term added to the denominator to improve numerical stability. The value of this term would greatly affect the performance of the neural network. However, this term in iGDM has little influence on the performance (shown in Tab.~\ref{tab:function_optim}), which may be because our algorithm can provide a stable second moment. This term is necessary to prevent the unlikely case where the second momentum is zero in iGDM.

% Figure environment removed

\subsection{Theoretical analysis of history gradient update}
\label{sec:hgu}
In the following, we show the theoretical analysis of the gradient decent algorithm and our history gradient update in the context of diffusion-based solvers. The basic formulation of gradient guidance in Eq.~\ref{eq:lds} corresponds to a simple gradient descent scheme. However, previous work~\cite{chung2022improving,chung2022diffusion} did not provide convergence proofs for the use of gradient descent algorithms on the data fidelity term. Here, we give the proofs for the use of gradient descent algorithms on the data fidelity term. We also demonstrate through theoretical analysis that incorporating historical gradients into updates of the data fidelity term can still convergence.
The following definitions shows our definition on the data fidelity term and the convergence criteria:
\begin{definition}[Data fidelity term]
	For all $t \in \Rd$, the data fidelity term $U_t(\z)$ is a convex function related to latent variable $\z$, then any $\z_1$ and $\z_2$ in a given domain and $\forall a \in (0, 1)$ have:
	\begin{align}
		U_t(a\z_1 + (1-a)\z_2) & \leq a U_t(\z_1) + (1 - a)U_t(\z_2), \\
		U_t(\z_2) & \geq U_t(\z_1) + \left\langle \nabla U_t(\z_1),\z_2-\z_1\right\rangle .
	\end{align}
\end{definition}
\begin{definition}[Convergence Criteria]
	When $U_t(\z)$ is a convex function, the regret algorithm $R(T)$ from~\cite{Zinkevich2003OnlineCP} is chosen as the statistical quantity:
	\begin{equation}
		R(T) = \sum_{t=1}^T U_t \left(\z^{(t)}\right) - \min_\z \sum_{t=1}^T U_t (\z)	.
	\end{equation}
	When $T \to \infty$ and $R(T) / T \to 0$, we can conduct that the gradient descent algorithm is convergent, i.e., $\z \to \argmin_\z \sum_{t=1}^{T}U_t(\z) \triangleq \z^*$, not only does it converge to some value, but this value minimizes the data fidelity term. Subject to convergence of the algorithm, we generally assume that \textbf{1.} the slower $R(T)$ grows with $T$, the faster the algorithm converges, and \textbf{2.} the slower the guidance rate $\epsilon$ decays for the same growth rate, the faster the algorithm converges.
\end{definition}
Based on the above definitions, we make assumptions on the latent variable $\z$ and the gradient $\g$:
\begin{assumption}\label{var_bounded}
	Latent variable $\z$ is bounded, i.e., $\|\z - \z'\|_2 \leq D, \forall \z, \z'$, and the	gradient $\g = \nabla_{\z_t} \gU \left( \y, \Ac \mathcal{D}\left(\Ed \left[\z_0|\z_t\right]\right)\right) $ is also bounded, i.e., $\|\g_t\|_2 \leq G, \forall t$.
\end{assumption}
This assumption is very common and has been widely adopted in neural network analysis. Now we can provide a theoretical proof of convergence for the GD algorithm.
\begin{theorem}
	For the gradient descent algorithm used in~\cite{chung2022improving,chung2022diffusion}, we have the upper bound of GD as:
	\begin{equation}
		R(T) = \frac{1}{2\eps_T} D^2 + \frac{G^2}{2} \sum_{t=1}^{T} \eps_t^2.
	\end{equation}
	When $T \to \infty$, the GD algorithm is tend to convergence:
	\begin{equation}
		\lim_{T \to \infty} \frac{R(T)}{T} \leq \lim_{T \to \infty} \frac{1}{T} \left[\frac{1}{2\eps_T} D^2 + \frac{G^2}{2} \sum_{t=1}^{T} \eps_t^2\right] = 0.
	\end{equation}
	Considering $\eps_t$ is a function w.r.t $t$: $\eps_t = \epsilon(t)$, and a polynomial decay with a constant $C$ is employed: $\eps_t = C/t^n, n \geq 0$:
	\begin{equation}
		R(T) \leq D^2 \frac{T^n}{2 C} + \frac{G^2C}{2}\left(\frac{1}{1-n}T^{1-n} - \frac{n}{1-n}\right).
	\end{equation}
	Thus, $R(T) = \gO\left(T^{\max(n, 1-n)}\right)$. When $n=1/2$, $R(T)$ attain the optimal upper bound $\gO\left(T^{1/2}\right)$.
	\label{proofofgd}
\end{theorem}
By leveraging the result of Theorem~\ref{proofofgd}, we can ensure the GD algorithm can converge on the data fidelity term. In the training of neural networks, the integration of historical gradient information with gradient descent algorithms has achieved significant success~\cite{sutskever2013importance,kingma2014adam}. Therefore, similarly, we introduce historical gradient information into the optimization process for the data fidelity term to form our history gradient update method. Here, we demonstrate a variants of gradient update policies based on first and second moments. For any dimension $i$ of the variable $\z$, we have:
\begin{align}
	z^{(t+1)}_i & = z_i^{(t)} - \eps_t \frac{\hat{m}_i^{(t)}}{\sqrt{\hat{v}_i^{(t)}}} \nonumber \\
	             & = z_i^{(t)} - \frac{\eps_t}{1 - \prod_{s=1}^t \eta_{1,s}} \frac{\eta_{1,t} m^{(t-1)}_i + (1-\eta_{1,t})g_{t,i}}{\sqrt{\hat{v}_i^{(t)}}}, \nonumber \\
	             \hat{m}_i^{(t)} & =\frac{1-\eta_1}{1-\eta_1^t}\sum_{s=1}^{t}\eta_{1}^{t-s}g_s, \quad \hat{v}_i^{(t)}=\frac{1-\eta_2}{1-\eta_2^t}\sum_{s=1}^{t}\eta_{2}^{t-s}g_s^2. 
\end{align}
Where $\hat{m}_i^{(t)}, \hat{v}_i^{(t)}$ are the first and second momentum of gradient. $(\eta_1, \eta_2)$ are the coefficient for $\hat{m}_i^{(t)}, \hat{v}_i^{(t)}$. Based on the Assumption~\ref{var_bounded}, we have a theoretical proof convergence for Improved-Momentum-variant history gradient update:
\begin{theorem}
	For the Improved-Momentum-variant history gradient update, we have the upper bound as:
	\begin{align}
		& R(T) \leq \frac{\sum_{i=1}^{d}D^2_i G_i}{2\eps_T(1-\eta_{1,1})} + \left(\sum_{i=1}^{d}D_iG_i\right)\left(\sum_{t=1}^{T}\frac{\eta_{1,t}}{1-\eta_{1,t}}\right) + \nonumber \\
		& \left(\sum_{i=1}^{d}G_i\right)\left[\sum_{t=1}^{T}\frac{\gamma_t}{2(1-\eta_{1,t})} \cdot \sum_{s=1}^{t}\frac{(1-\eta_{1,s})^2\left(\prod_{r=s+1}^t\eta_{1,r}\right)^2}{(1-\eta_2)\eta_2^{t-s}}\right],
	\end{align}
	where $\gamma_t = \frac{\eps_t}{1 - \prod_{s=1}^t \eta_{1,s}}$. Similarly, 	When $T \to \infty$, this Improved-Momentum-variant history gradient update algorithm is tend to convergence. Considering $\eta_{1,t} \in (0, 1), \forall t, \eta_{1,1} \geq \eta_{1,2} \geq \dots \geq \eta_{1,T} \geq \dots$ and $\eta_2 \in (0,1), \frac{\eta_{1,t}}{\sqrt{\eta_2}}\leq\sqrt{c} < 1$, we have:
	\begin{align}
		& R(T) \leq \frac{\sum_{i=1}^{d}D^2_i G_i}{2\eps_T(1-\eta_{1,1})} + \left(\sum_{i=1}^{d}D_iG_i\right)\left(\sum_{t=1}^{T}\eta_{1,t}\right) + \nonumber \\
		& \left(\sum_{i=1}^{d}G_i\right)\left[\frac{\sum_{t=1}^{T}\eps_t}{2(1-\eta_{1,1})^2(1-\eta_2)(1-c)}\right].
	\end{align}
	Similar to Theorem~\ref{proofofgd}, when $n=1/2$, $R(T)$ attain the optimal upper bound $\gO\left(T^{1/2}\right)$.
	\label{proofofadam}
\end{theorem}
\begin{remark}
Note that the GD algorithm and our Improved-Momentum-variant history gradient update algorithm have similar upper bound which mean they can both converge when optimize the data fidelity term. Although Theorem~\ref{proofofgd}and~\ref{proofofadam} both require $T \to \infty$, in real-world applications, both the algorithms can converge with limited iterations. In addition, although both the algorithms have the same optimal upper bound $\gO\left(T^{1/2}\right)$, the Improved-Momentum-variant history gradient update has a faster and more stable convergence process than the naive GD algorithm when deals with the various inverse problems in real-world applications. While we have only proven the convergence of GD and iGDM, these proofs can also be extended to update algorithms using historical gradient information, such as Gradient Descent with Momentum (GDM)~\cite{sutskever2013importance}.
\end{remark}
It is worth noting that our history gradient update method is a generalize optimizing method for the data fidelity term, thus, we can apply it to our latent-diffusion-based solver~\cite{song2023solving2} and previous pixel-diffusion-based solvers (\textit{e.g.,\xspace} MCG~\cite{chung2022improving} and DPS~\cite{chung2022diffusion}). Details of the proof process are presented in the supporting document.


%The guidance rate $\epsilon$ can be thought of as the learning rate value in the Stochastic Gradient Descent (SGD). Thus, in order to further improve the data consistency, we adopt gradient information from the previous steps. Because the history gradient information can provide sample-level information to decide the optimization direction of the guidance process. This is also known as the \textit{first-order} gradient-based optimization. Here, we demonstrate two variants of gradient update policies based on two typical optimizers.
%
%\textbf{Momentum-like gradient update policy. } Similar to the Momentum optimizer~\cite{sutskever2013importance}, we consider to use the moving averages $m^t$ of the gradients $\nabla_{x_t} \mathcal{U}\left(\mathbf{A}x_t, y\right)$ to perform gradient descent:
%\begin{align}
%	m^{t} &= \eta m^{t-1} + \left(1 - \eta\right) \nabla_{x_t} \mathcal{U}\left(\mathbf{A}x_t, y\right), \\ 
%	x_{t-1}  &= x_t - \epsilon m^{t},
%\end{align}
%where $\eta$ is a hyper-parameter to adjust the factor of momentum. 
%
%\textbf{Adam-like gradient update policy. } Adam optimizer~\cite{kingma2014adam} uses momentum and adaptive learning rate to perform gradient descent:
%\begin{align}
%	m^{t} &= \eta_1 m^{t-1} + \left(1 - \eta_1\right) \nabla_{x_t} \mathcal{U}\left(\mathbf{A}x_t, y\right), \\ 
%	v^{t} &= \eta_2 v^{t-1} + \left(1 - \eta_2\right) \nabla_{x_t} \mathcal{U}\left(\mathbf{A}x_t, y\right)^2, \\
%	%	\hat{m}^t &= \frac{m^{t}}{1-\eta_1}, \, \hat{v}^t = \frac{v^t}{1-\eta_2}, \\
%	x_{t-1}  &= x_t - \epsilon \frac{\hat{m}^t}{\sqrt{\hat{v}^t} + \varepsilon},
%\end{align}
%where $\left(\eta_1, \eta_2\right)$ are the coefficients used to calculate the exponentially weighted moving averages of gradient and its square, while $\varepsilon$ helps improve the numerical stability. With the help of the moving averages $m^t$ and $v^t$, we can efficiently locate the flat minima.

%It is worth noting that the choice of $\epsilon$ and gradient policy is dependent on the evaluation function $\mathcal{U}$ (Details and ablation studies can be found in Appendix.~\hyperref[ablation_func]{C}). Our gradient update policies are compatible with DIR, LDIR, and DPS whether on pixel space or latent space. The details are demonstrated in the ablation studies. The detail of LDIR algorithms with the above gradient update policies are presented in Appendix~\hyperref[app:LDIR_hgu]{A}.

%	We find that the previous works~\cite{chung2022diffusion,chung2022improving,chung2022parallel,chung2022solving,song2021solving,wang2022zero} all perform data restoration on the pixel space, which causes huge demands in computation. Thus, based on previous works of Latent Diffusion Models (LDM)~\cite{rombach2022high} and our DIR, we propose a novel data restoration diffusion model named Deep Latent Iterative Reconstruction (LDIR). This approach offers us several advantages: (i) Instead of processing images in the pixel space, we can encode high-resolution images into a low-dimensional space and process them at the same time with less computation. (ii) Compared to the pixel space, the latent space contains much higher information density so that we can ensure the data are conform to some natural priors, e.g., sparsity.

%Our review of previous works on diffusion-based data reconstruction, including~\cite{chung2022diffusion,chung2022improving,chung2022parallel,chung2022solving,song2021solving,wang2022zero}, reveals that they all perform reconstruction in the pixel space, which requires significant computational resources. To address this limitation, we draw inspiration from the Latent Diffusion Models (LDMs) proposed by~\cite{rombach2022high} and our DIR. We introduce a novel data reconstruction diffusion model called Latent Diffusion Iterative Reconstruction (LDIR). LDIR offers several advantages over previous methods: (i) Instead of processing images in the pixel space, we encode images into a low-dimensional latent space, enabling us to process them more efficiently with fewer computational demands. (ii) The latent space contains significantly higher information density compared to the pixel space, allowing us to incorporate natural priors such as sparsity and improve the quality of the restored data.
%
%\textbf{From pixels to latents. } 
%To encode pixels to latents, we construct an autoencoder comprising an encoder $\mathcal{E}$ and a decoder $\mathcal{D}$. Specifically, given an input image $x$ in the pixel space, $\mathcal{E}$ maps $x$ to a low-dimensional latent vector $\ell = \mathcal{E}\left(x\right)$. $\mathcal{D}$ reconstructs the image $\bar{x} = \mathcal{D}\left(\mathcal{E}\left(x\right)\right)$ from $\ell$. To incorporate the sparsity prior of $x$, we use the \textit{Vector Quantized Variational Autoencoder} (\textit{VQ-VAE}) proposed by ~\cite{kingma2013auto,rezende2014stochastic,esser2021taming} with the quantization layer~\cite{van2017neural}.
%	As mentioned above, we need to construct an autoencoder to encode pixels to latents. Without loss of generality, given an image $x$ in pixel space, the encoder $\mathcal{E}$ maps $x$ to a low-dimensional latent $l=\mathcal{E}(x)$, and the decoder $\mathcal{D}$ reconstructs the image $\bar{x} = \mathcal{D}(\mathcal{E}(x))$ from $l$. To conform the sparsity prior of $x$, we use \textit{VQ}-VAE~\cite{kingma2013auto,rezende2014stochastic,esser2021taming} with the quantization layer~\cite{van2017neural}.

%\textbf{Score matching for latents. } With our semantic compression model $\mathcal{E}$ and $\mathcal{D}$, we can now establish the score of latents $\nabla_\ell \, \log p_t\left(\ell\right)$ and use score model $s_{\theta_\ell}$ to approx it with the following objective:
%\begin{multline}
%	\label{eq:latent_sde_objective}
%	\min_\theta \mathbb{E}_{t \in \left[0,\dots,T-1\right], \ell_0 \sim p_{\ell}, \ell_t \sim p_{0|t}\left(\ell_t \middle| \ell_0 \right)} \\ \left[\left|\left|s_{\theta_\ell}\left(\ell,t\right)-\nabla_\ell \log p_{0|t}\left(\ell_t \middle| \ell_0 \right)\right|\right|_2^2\right],
%\end{multline}
%where $p\left(\ell\right) = p\left(\mathcal{E}\left(x\right)\right)$ and $x \sim p\left(x\right)$.
%
%\textbf{Conditional guidance process on the latent space. } Similar to the conditional guidance process in the pixel space, we begin by using the score model $s_{\theta_\ell}$ to generate latents from standard Gaussian noise:
%\begin{align}
%	\ell'_{t-1} = \ell_t - f\left(\ell,t\right) - g\left(t\right)^2 s_{\theta_\ell}\left(\ell,t\right) + g\left(t\right)z, z \sim \mathcal{N}\left(0,\mathbf{I}\right),
%\end{align}
%We need to use $\mathcal{D}$ to decode $\ell$ to pixel space, and compute the data consistency term $\mathcal{U}\left(\mathbf{A}\mathcal{D}\left(\ell\right), y\right)$, which can be derived to:
%\begin{align}
%	\ell_{t-1} = \ell'_{t-1} -  \epsilon \nabla_{\ell_t} \, \mathcal{U}\left(\mathbf{A}\mathcal{D}\left(\ell_t\right), y\right).
%\end{align}
%Thus, we can get the conditional guidance algorithm of LDIR as:
%\begin{multline}
%	\ell_{t-1} \simeq \ell_t - \underbrace{\left(\lambda_t \, s_{\theta_\ell}\left(\ell_t,t\right) - g\left(t\right)z\right)}_{\text{Prior term}} - \\
%	 \underbrace{ \epsilon \nabla_{\ell_t} \, \mathcal{U}\left(\mathbf{A}\mathcal{D}\left(\ell_t\right), y\right)}_{\text{Data-fidelity term}}, z \sim \mathcal{N}\left(0,\mathbf{I}\right).
%	\label{eq:LDIR_sde}
%\end{multline}
%The final results can be obtained by decoding the final latent $x_0 = \mathcal{D}\left(\ell_0\right)$ using the decoder $\mathcal{D}$.
%	It should be noted that some methods relied on range-null space decomposition are not compatible with the latent generative process. Because, they use $\mathbf{A}^{-1}$ to projection the difference $y - \mathbf{A}\mathcal{D}(x_{0|t})$ to update the latents which may be inconsistent with $\mathcal{E}$, \textit{i.e.,} $\mathbf{A}^{-1}(y - \mathbf{A}\mathcal{D}(x_{0|t})) \nsim p(x)$. 
\begin{table*}[t]
	\centering
	\caption{Quantitative evaluation (PSNR, SSIM) of medical image reconstruction on AAPM test $256\times256$ dataset. We mark \textbf{bold} for the best and \underline{underline} for the second best. CLEAR~\cite{zhang2021clear} is a supervised method.}
		\begin{tabular}{@{\extracolsep{4pt}}lcccccccc@{}}
			\toprule
			\multicolumn{1}{l}{\multirow{3}{*}{\textbf{Method}}} & \multicolumn{4}{c}{\textbf{Sparse view}}                 & \multicolumn{4}{c}{\textbf{Limited angle }}   \\ \cmidrule{2-5} \cmidrule{6-9} & \multicolumn{2}{c}{18} & \multicolumn{2}{c}{32} & \multicolumn{2}{c}{45} & \multicolumn{2}{c}{90} \\ \cmidrule{2-3} \cmidrule{4-5} \cmidrule{6-7} \cmidrule{8-9} 
			\multicolumn{1}{c}{}  & PSNR $\uparrow$      & SSIM $\uparrow$     & PSNR $\uparrow$      & SSIM $\uparrow$     & PSNR $\uparrow$      & SSIM $\uparrow$     & PSNR $\uparrow$      & SSIM $\uparrow$  \\ \midrule
			FBP                                         &     24.76       &    0.5296       &     28.03       &     0.6779      &      16.65      &      0.5422     &      20.35      &     0.5113      \\
			FISTA-TV                                    &     24.86       &     0.5408      &      28.14      &      0.6888     &     16.66       &     0.5463      &    20.40        &    0.5241      \\
			CLEAR                                       &     \underline{32.28}      &     \underline{0.8798}     &      \underline{36.24}      &     \underline{0.9257}      &     25.71       &      \underline{0.8559}     &       \underline{31.60}     &    \textbf{0.9223}      \\
			MCG                                         &      28.54      &  0.8135         &     28.98       &     0.8242     &      26.08      &     0.7418      &     28.44      &     0.8079     \\
			DPS   & 28.55 & 0.8140 & 28.97 & 0.8242 & \underline{28.25} & 0.8204 & 28.25 & 0.8088 \\ \midrule
			LHGU  & \textbf{39.01} & \textbf{0.9552} & \textbf{39.77} & \textbf{0.9612} & \textbf{30.05} & \textbf{0.8747} & \textbf{32.68} & \underline{0.9032} \\ \bottomrule
		\end{tabular}%	
	\label{tab:ct_images_256}
\end{table*}

\section{Experiments}

\subsection{Experimental setup}
\textbf{Models and datasets.} For medical image reconstruction, we train our DDPM and LDM model on the 2016 American Association of Physicists in Medicine (AAPM) grand challenge dataset~\cite{mccollough2017low}. The dataset has normal-dose data from 10 patients. 9 patients’ data are used for training, and 1 for validation which contains 526 images. To simulate low-dose imaging, a parallel-beam imaging geometry with 180 degrees was employed. Regarding inpainting and super-resolution tasks, we test our method on CelebAHQ 1k $256\times256$ dataset~\cite{liu2015faceattributes} and LSUN-bedroom $256\times256$ dataset~\cite{yu2015lsun}. We utilize pretrained DDPM and LDM models from the open-source model repository from~\cite{ho2020denoising,rombach2022high}. All the images are normalized to range $\left[0,1\right]$. More details including the hyper-parameters are listed in Appendix.~\hyperref[app:experiment]{B}. We call the Latent diffusion solver with HGU as LHGU.

\textbf{Measurement operators.} For sparse-view CT reconstruction, we uniformly sample 18 and 32 views. For limited-angle CT reconstruction, we restrict the imaging degree range to $45$ and $90$ degrees with 128 views using parallel beam geometry. For random inpainting, following~\cite{chung2022improving,chung2022diffusion}, we mask out $99\%$ of the total pixels (including all the channels). For super-resolution, we use $8\times$ bilinear downsampling. Gaussian noise is added in the nature image evaluation after a forward operation performed with $\sigma=0.05$. The medical data are evaluated without noise.

\begin{table*}[h]
	\centering
	\caption{Quantitative evaluation (PSNR, SSIM) of medical image reconstruction on AAPM test $512\times512$ dataset for zero-shot methods. We mark \textbf{bold} for the best and \underline{underline} for the second best.}
		\begin{tabular}{@{\extracolsep{4pt}}lcccccccc@{}}
			\toprule
			\multicolumn{1}{c}{\multirow{3}{*}{\textbf{Method}}} & \multicolumn{4}{c}{\textbf{Sparse view}}                 & \multicolumn{4}{c}{\textbf{Limited angle }}   \\ \cmidrule{2-5} \cmidrule{6-9} & \multicolumn{2}{c}{18} & \multicolumn{2}{c}{32} & \multicolumn{2}{c}{45} & \multicolumn{2}{c}{90} \\ \cmidrule{2-3} \cmidrule{4-5} \cmidrule{6-7} \cmidrule{8-9} 
			\multicolumn{1}{c}{}  & PSNR $\uparrow$       & SSIM $\uparrow$     & PSNR $\uparrow$      & SSIM $\uparrow$     & PSNR $\uparrow$      & SSIM $\uparrow$     & PSNR $\uparrow$      & SSIM $\uparrow$  \\ \midrule
			FBP                                         &      23.48      &     0.5096      &     26.70       &     0.6423      &     16.53       &     0.5480      &     19.88       &     0.4932      \\
			FISTA-TV                                    &     \underline{23.93}       &     \underline{0.5566}      &     \underline{27.11}       &     \underline{0.6768}      &      \underline{16.59}      &    \underline{0.5695}       &     \underline{20.08}       &     \underline{0.5348}     \\ \midrule
			LHGU                                        &     \textbf{36.94}       &    \textbf{0.9216}       &      \textbf{37.63}      &      \textbf{0.9373}     &      \textbf{30.47}      &     \textbf{0.8579}      &       \textbf{33.41}     &     \textbf{0.8837}     \\ \bottomrule
		\end{tabular}%
	\label{tab:ct_images_512}
\end{table*}
\subsection{Evaluation on medical data}

To assess the performance of HGU in reconstructing medical sparse data, we compare it with several recent state-of-the-art methods: manifold constrained gradients (MCG)~\cite{chung2022improving}, diffusion posterior sampling (DPS)~\cite{chung2022diffusion}, comprehensive learning enabled adversarial reconstruction (CLEAR)~\cite{zhang2021clear}, fast iterative shrinkage-thresholding algorithm with total-variation (FISTA-TV), and the analytical reconstruction method, filtered back projection (FBP). Peak-signal-to-noise-ratio (PSNR) and structural similarity index measure (SSIM) are used for quantitative evaluation. 

The quantitative results of medical sparse data reconstruction are demonstrated in Tab.~\ref{tab:ct_images_256} and Tab.~\ref{tab:ct_images_512}. LHGU outperforms all other state-of-the-art methods by a significant margin across all experiment settings. We also compare our method in the high-resolution CT image reconstruction task with zero-shot methods. However, due to the large memory consumption of DDPM, it is challenging to train DDPM models for high-resolution reconstruction. Thus, we exclude MCG and DPS which rely on DDPM from Tab.~\ref{tab:ct_images_512}. The results show that LHGU provides noise-free reconstruction results, although there is still a significant gap between the reconstructed images and the ground truth. In contrast, other zero-shot methods fail to reconstruct meaningful results. 

% % Figure environment removed

The qualitative results of medical sparse image reconstruction are demonstrated in Fig.~\ref{fig:mayo256vis} which are consistent with the quantitative results reported in Tab.~\ref{tab:ct_images_256}. In Fig.~\ref{fig:mayo256vis}, we compare our method with the state-of-the-art zero-shot unsupervised and supervised methods. We observe that LHGU can provide high-quality reconstructions, especially for the sparse view reconstruction task. Specifically, LHGU can provide better overall structure and nearly artifact-free reconstruction. Additionally, our method also provides better reconstructions than other methods in limited angle reconstruction tasks. 
%(More qualitative results of medical sparse data reconstruction can be found in Appendix.~\hyperref[app:medical]{D}).

% Figure environment removed
% Figure environment removed

\begin{table*}[t]
	\centering
	% 加说明
	\caption{Quantitative evaluation (PSNR, SSIM, LPIPS) of nature image reconstruction on CelebAHQ and LSUN-bedroom dataset. We mark \textbf{bold} for the best and \underline{underline} for the second best.}
	\resizebox{\textwidth}{!}{%
		\begin{tabular}{@{\extracolsep{1pt}}lccccccccccccc@{}}
			\toprule
			\multicolumn{1}{c}{\multirow{3}{*}{\textbf{Method}}}     & \multirow{3}{*}{\textbf{Type}}                                      & \multicolumn{6}{c}{\textbf{CelebAHQ}}                           & \multicolumn{6}{c}{\textbf{LSUN-bedroom}}                             \\ \cmidrule{3-8} \cmidrule{9-14}
			\multicolumn{2}{c}{}                                                                   & \multicolumn{3}{c}{\textbf{Inpaint}} & \multicolumn{3}{c}{\textbf{SR (8$\times$)}} & \multicolumn{3}{c}{\textbf{Inpaint}} & \multicolumn{3}{c}{\textbf{SR (8$\times$)}} \\ \cmidrule{3-5} \cmidrule{6-8} \cmidrule{9-11} \cmidrule{12-14}
			\multicolumn{2}{c}{}                                                                    & PSNR $\uparrow$    & SSIM $\uparrow$   & LPIPS $\downarrow$   & PSNR $\uparrow$  & SSIM $\uparrow$  & LPIPS $\downarrow$ & PSNR $\uparrow$    & SSIM $\uparrow$   & LPIPS $\downarrow$  & PSNR $\uparrow$ & SSIM $\uparrow$ & LPIPS $\downarrow$ \\ \midrule
			PnP-ADMM &       Traditional IR                                                 &    3.97     &     0.3017    &    0.8916     &   22.94    &   0.6303    &    0.6820    &    5.059     &    0.3236     &    0.8940     &    \textbf{20.14}   &   0.5458    &    0.7944    \\
			MCG      & Pixel Diffusion                                                          &    18.91     &     0.5600    &    0.2544     &    12.47   &   0.1655    &   0.6713     &     16.89    &    0.4555     &    0.5486     &    9.39   &    0.0606   &   0.8698     \\
			DPS          & Pixel Diffusion                                                      &    \underline{18.95}     &     \underline{0.5614}    &    \underline{0.2543 }    &   \underline{24.36}    &    \underline{0.7116}   &   \underline{0.1089}     &     \underline{17.03}    &    \underline{0.4587}     &    \underline{0.5414}     &   19.15    &    \underline{0.5614}   &    \textbf{0.3074}    \\ \midrule
			LHGU         & Latent Diffusion                                                      &    \textbf{22.14}     &     \textbf{0.6647 }   &     \textbf{0.2280}    &   \textbf{25.27}    &    \textbf{0.7530}   &     \textbf{0.0878}   &    \textbf{20.33}     &    \textbf{0.5845}     &   \textbf{0.4858}      &   \underline{19.83}    &    \textbf{0.5762}   &    \underline{0.3253}    \\
			\bottomrule
		\end{tabular}%
	}
	\label{tab:nature_images}
\end{table*}

\subsection{Evaluation on nature images}
In order to further test the performance of our method, we compare our method against state-of-the-art methods, namely, MCG, DPS, and plug-and-play alternating direction method of multipliers (PnP-ADMM)~\cite{chan2016plug}. For quantitative analysis, we utilize three widely used perceptual evaluation metrics: LPIPS distance, PSNR, and SSIM.

The quantitative results of nature image reconstruction are illustrated in Tab.~\ref{tab:nature_images}. Our method achieves competitive results compared to the previous state-of-the-art. Specifically, we observe that our method is able to accurately reconstruct the original data and preserve the most data consistency, even when dealing with highly sparse measurements such as $99\%$ random inpainting. Additionally, we note that LHGU gains some advantages over the previous best method on the super-resolution task. 
% This is likely due to the high information density of the super-resolution task, which provides enough information for all methods to reconstruct the original data, making the advantage of LDIR in recovering extremely sparse data less pronounced.

The qualitative results of nature sparse image reconstruction are demonstrated in Fig.~\ref{fig:naturevis} and Fig.~\ref{fig:naturevis_sr}. Notably, the traditional iterative method PnP-ADMM failed to produce satisfactory results for both the inpainting and super-resolution tasks due to its limited prior terms. In contrast, our method outperforms the comparison methods, particularly in terms of color and structure in the inpainting task. In the super-resolution task, the results obtained by MCG exhibit many artifacts, which are likely due to the projection step\cite{chung2022diffusion}. Our method, on the other hand, achieves competitive results with DPS, the most advanced method, with small gaps.
% (More qualitative results of nature sparse image reconstruction can be found in Appendix.~\hyperref[app:nature]{E}).

% \begin{table*}[t]
% 	% add pixel and latent 
% 	\centering
% 	\caption{Ablation evaluation (PSNR, SSIM) on the effect of latent-based iterative reconstruction. We mark \textbf{bold} for the best and \underline{underline} for the second best.}
% 	\resizebox{\textwidth}{!}{%
% 		\begin{tabular}{@{\extracolsep{4pt}}lccccccccccc@{}}
% 			\toprule
% 			\multicolumn{1}{c}{\multirow{3}{*}{\textbf{Method}}} & \multicolumn{1}{c}{\multirow{3}{*}{\textbf{Type}}} & \multicolumn{4}{c}{\textbf{Sparse view}}                                     & \multicolumn{4}{c}{\textbf{Limited angle}}                 & \multicolumn{1}{c}{\multirow{3}{*}{\textbf{Speed(iter/s) $\uparrow$ }}} & \multicolumn{1}{c}{\multirow{3}{*}{\textbf{Memory(MB) $\downarrow$}}}                  \\ \cmidrule{3-6} \cmidrule{7-10} 
% 			\multicolumn{2}{c}{}                        & \multicolumn{2}{c}{18}           & \multicolumn{2}{c}{32}           & \multicolumn{2}{c}{45}           & \multicolumn{2}{c}{90}       \\ \cmidrule{3-4} \cmidrule{5-6} \cmidrule{7-8} \cmidrule{9-10} 
% 			\multicolumn{2}{c}{}                        & PSNR $\uparrow$          & SSIM $\uparrow$           & PSNR $\uparrow$          & SSIM $\uparrow$           & PSNR $\uparrow$          & SSIM $\uparrow$           & PSNR $\uparrow$          & SSIM $\uparrow$           \\ \midrule
% 			DPS        & Pixel Diffusion                                 & 28.55          & 0.8140          & 28.97          & 0.8242          & \underline{28.25}    & \underline{0.8204}    & 28.25          & 0.8088  & \underline{20.88} & \underline{6338}        \\ \midrule
% 			DPS with HGU (Momentum-variant)    & Pixel Diffusion                                   & \underline{31.45}    & \underline{0.8654}    & \underline{32.82}    & \underline{0.8898}    & 27.31          & 0.8133          & \underline{28.98}    & \underline{0.8280} & 20.75 & \underline{6338}   \\
% 			LDIR (Ours, Adam-variant)    & Latent Diffusion                                  & \textbf{39.01} & \textbf{0.9552} & \textbf{39.77} & \textbf{0.9612} & \textbf{29.60} & \textbf{0.8779} & \textbf{32.89} & \textbf{0.9116} & \textbf{36.67} & \textbf{4268} \\ \bottomrule
% 		\end{tabular}%
% 	}
% 	\label{tab:ablation_dir}
% \end{table*}

\begin{table*}[]
\centering
\caption{Ablation evaluation (PSNR, SSIM) on the effect of our method. HGU denotes the Improved-Momentum-variant HGU. Model A is equal to DPS~\cite{chung2022diffusion} We mark \textbf{bold} for the best and \underline{underline} for the second best.}
\label{tab:ablation_dir}
\begin{tabular}{@{\extracolsep{1pt}}lcccccccccccc@{}}
\toprule
\multirow{3}{*}{\textbf{Model}} &
  \multirow{3}{*}{\textbf{\begin{tabular}[c]{@{}c@{}}Diffusion\\ Type\end{tabular}}} &
  \multirow{3}{*}{\textbf{\begin{tabular}[c]{@{}c@{}}Update\\ Strategy\end{tabular}}} &
  \multicolumn{4}{c}{\textbf{Sparse View}} &
  \multicolumn{4}{c}{\textbf{Limited Angle}} &
  \multirow{3}{*}{\textbf{\begin{tabular}[c]{@{}c@{}}Speed\\ (iter/s)\end{tabular}}} &
  \multirow{3}{*}{\textbf{\begin{tabular}[c]{@{}c@{}}Memory\\ (MB)\end{tabular}}} \\ \cmidrule(lr){4-7} \cmidrule(lr){8-11}  
 &
   &
   &
  \multicolumn{2}{c}{18} &
  \multicolumn{2}{c}{32} &
  \multicolumn{2}{c}{45} &
  \multicolumn{2}{c}{90} &
   &
   \\ \cmidrule(lr){4-5} \cmidrule(lr){6-7} \cmidrule(lr){8-9} \cmidrule(lr){10-11}   
 &
   &
   &
  PSNR &
  SSIM &
  PSNR &
  SSIM &
  PSNR &
  SSIM &
  PSNR &
  SSIM &
   &
   \\ \midrule
A (DPS) &
  Pixel &
  GD &
  28.55 &
  0.8140 &
  28.97 &
  0.8242 &
  28.25 &
  0.8204 &
  28.25 &
  0.8088 &
  20.88 &
  {\ul 6338} \\
B &
  Latent &
  GD &
  31.37 &
  0.8775 &
  31.88 &
  0.8891 &
  \textbf{29.82} &
  {\ul 0.8684} &
  31.65 &
  0.8884 &
  \textbf{37.03} &
  \textbf{4268} \\
C &
  Pixel &
  HGU &
  {\ul 32.83} &
  {\ul 0.8892} &
  {\ul 34.29} &
  {\ul 0.9171} &
  29.40 &
  0.8562 &
  {\ul 32.42} &
  {\ul 0.9047} &
  20.62 &
  {\ul 6338} \\
D (Ours, LHGU) &
  Latent &
  HGU &
  \textbf{39.01} &
  \textbf{0.9552} &
  \textbf{39.77} &
  \textbf{0.9612} &
  {\ul 29.60} &
  \textbf{0.8779} &
  \textbf{32.89} &
  \textbf{0.9116} &
  {\ul 36.67} &
  \textbf{4268} \\ \bottomrule
\end{tabular}
\end{table*}

\subsection{Ablation studies}
\label{manu:ablation}
We conducted ablation studies to validate the effectiveness of our approach. we compared the performance of our approach against a pixel-based iterative reconstruction approach. To ensure a fair comparison, we conducted these ablation studies on the medical image reconstruction task, as both the DDPM and LDM models were trained using the same protocol.
% (2) we investigated how the choice of measure function $\mathcal{U}$ and history gradient update policies influenced the results. To ensure a fair comparison, we conducted these ablation studies on the medical image reconstruction task, as both the DDPM and LDM models were trained using the same protocol.

In Table~\ref{tab:ablation_dir}, we can observe that our method outperforms the pixel-based iterative reconstruction method, DPS, by a large margin. This result confirms that the latent-based approach is superior to the pixel-based approach in terms of both speed and accuracy. Additionally, we can see that using HGU in the pixel space can improve the performance of DPS, allowing it to surpass the naive DPS. Compared to pixel-space models, LHGU achieves significant speed-up with less memory consumption. Although LHGU decodes latent into pictures at each step, it still has a greater advantage than processing directly in pixel space.

From Table~\ref{tab:function_optim}, we can observe that our history gradient update method significantly enhances performance, regardless of the evaluation function used, even for non-strong-convex functions such as LPIPS~\cite{johnson2016perceptual} and FFL~\cite{jiang2021focal}. This suggests that our history gradient update is a versatile method applicable to various scenarios. Similarly, from Fig.~\ref{fig:eval_func}, we can observe that in the case of using Improved-Momentum-variant, all evaluation functions achieve relatively stable optimization. In Fig.~\ref{fig:hgu_func}, we present a comparison between our history gradient update and the GD algorithm. It is evident that our algorithm provides a more stable optimization process and does not exhibit an increase in error during the later stages of optimization. This indicates that historical gradient information contributes to a more stable optimization of the data fidelity term.

% Figure environment removed

\begin{table*}[t]
	\centering
	\caption{Ablation evaluation (PSNR, SSIM) on the effect of evaluation function and history gradient update variants. We mark \textbf{bold} for the best and \underline{underline} for the second best.}
	\label{tab:function_optim}
%	\resizebox{\textwidth}{!}{%
		\begin{tabular}{@{}lcccccccccccc@{}}
			\toprule
			Evaluation function     & L1     & L1     & L1              & L2 & L2     & L2              & FFL    & FFL    & FFL          & LPIPS  & LPIPS  & LPIPS  \\
			\midrule
			Gradient update method  & GD     & GDM    & iGDM            & GD                & GDM    & iGDM            & GD     & GDM    & iGDM         & GD     & GDM    & iGDM   \\ \midrule
			PSNR $\uparrow$                   & 27.07  & 38.87  & {\ul 39.77}     & 32.23             & 32.36  & \textbf{39.86}  & 23.18  & 34.30  & 39.56        & 19.03  & 23.30  & 39.54  \\
			SSIM $\uparrow$                   & 0.8857 & 0.9551 & \textbf{0.9612} & 0.8933            & 0.8948 & \textbf{0.9612} & 0.7822 & 0.9009 & {\ul 0.9598} & 0.6144 & 0.6409 & 0.9597 \\ \bottomrule
		\end{tabular}%
%	}
\end{table*}

% \textbf{Measure function and update policy.} \alert{Missing Table, waiting the results} The selection of the measure function $\mathcal{U}$ and history gradient update policy is a crucial component in achieving successful reconstruction results. As shown in Tab.~, we observe that different update policies favor different guidance rate $\epsilon$, and the choice of $\mathcal{U}$ can also influence the selection of update policy. It should be noted that the selection of the measure function, update policy, and guidance rate is highly dependent on the specific task. However, in general, the two \textit{first-order} update policies proposed in Sec.~\ref{sec:hgu} tend to outperform the naive update policy used in~\cite{chung2022improving,chung2022diffusion,chung2022parallel}.

% Please add the following required packages to your document preamble:
% \usepackage{booktabs}
% \usepackage[table,xcdraw]{xcolor}
% Beamer presentation requires \usepackage{colortbl} instead of \usepackage[table,xcdraw]{xcolor}
\begin{table*}[]
	\centering
	\caption{Hyper-parameters evaluation on the Improved-Momentum-variant history gradient update. We mark \textbf{bold} for the best.}
	\label{tab:adam}
	\begin{tabular}{@{}lccccccccc<{}}
		\toprule
		Model & Baseline & A      & B      & C      & D      & E      & F      & G      & H      \\ \midrule
		$\varepsilon$ &
		$1e^{-8}$ &
		\cellcolor[HTML]{EFEFEF}$1e^{-3}$ &
		\cellcolor[HTML]{EFEFEF}$1$ &
		$1e^{-8}$ &
		$1e^{-8}$ &
		$1e^{-8}$ &
		$1e^{-8}$ &
		$1e^{-8}$ &
		$1e^{-8}$ \\
		$\eta_1$ &
		$0.9$ &
		$0.9$ &
		$0.9$ &
		\cellcolor[HTML]{C0C0C0}$0.99$ &
		\cellcolor[HTML]{C0C0C0}$0.5$ &
		\cellcolor[HTML]{C0C0C0}$0.1$ &
		$0.9$ &
		$0.9$ &
		$0.9$ \\
		$\eta_2$ &
		$0.999$ &
		$0.999$ &
		$0.999$ &
		$0.999$ &
		$0.999$ &
		$0.999$ &
		\cellcolor[HTML]{9B9B9B}$0.9$ &
		\cellcolor[HTML]{9B9B9B}$0.5$ &
		\cellcolor[HTML]{9B9B9B}$0.1$ \\ \midrule
		PSNR  & \textbf{39.77}    & 39.74  & \textbf{39.77}  & 30.85  & 34.11  & 30.83  & 29.88  & 20.02  & 20.85  \\
		SSIM  & \textbf{0.9612 }  & 0.9609 & 0.9610 & 0.8465 & 0.9200 & 0.8720 & 0.8245 & 0.6252 & 0.6414 \\ \bottomrule
	\end{tabular}
\end{table*}
\subsection{Hyper-parameter studies}
To analyze the impact of the history gradient update on the entire reconstruction process, we conducted multiple experiments on the hyperparameters of its Improved-Momentum-variant. Table~\ref{tab:adam} presents the experimental results of adjusting the three hyperparameters, $\varepsilon, \eta_1, \eta_2$. From the results, it is evident that adjusting $\varepsilon$ does not significantly impact the final performance. However, modifications to $\eta_1$ and $\eta_2$ result in a substantial reduction in performance, a phenomenon reminiscent of using the Adam optimizer for neural network optimization. On the other hand, this phenomenon underscores the necessity of incorporating historical gradients into the optimization process. Proper hyperparameter selection can lead to a significant performance improvement.

\section{Conclusion}
% In this paper, we propose Latent Diffusion Iterative Reconstruction (LDIR) as a novel approach for reconstructing CT sparse data in a zero-shot manner. We show theoretically that by utilizing the unconditional latent diffusion models as better prior term, we can achieve various CT reconstruct tasks with a single model and do not need paired data to our train network. In addition, we generate the prior term in the latent space instead of the pixel space, which encourages us to form a high-resolution image with lower computational complexity. Furthermore, we use history gradient information from data-fidelity term to guide the sample-level reconstruction process which provides high-quality results. Our experimental results demonstrate that LDIR outperforms state-of-the-art methods including supervised method on sparse CT data reconstruction and achieves competitive results on nature image restoration. We believe that our work offers the community a promising tool for leveraging the rapidly growing field of latent diffusion models to restore high-quality and high-resolution data from degraded measurements.

In this paper, we introduce the History Gradient Update (HGU) as a powerful optimization tool for solving general inverse problems. We ensured the convergence of optimizing the data fidelity term using a gradient descent algorithm through theoretical derivation. Simultaneously, we demonstrated that incorporating historical gradient information into the optimization process accelerates its convergence. In addition, we generate the prior term in the latent space instead of the pixel space, which has proven its effectiveness on small-scale datasets and also leads to faster sampling speed. Our experimental results demonstrate that the latent diffusion solver with HGU outperforms state-of-the-art methods including the supervised method on sparse CT data reconstruction and achieves competitive results on nature image restoration. We believe that our work offers the community a promising tool for leveraging the rapidly growing field of latent diffusion models to restore high-quality and high-resolution data from degraded measurements with stable and fast optimizations.



% We demonstrate that utilizing the latent diffusion model within LDIR as a prior can reduce the amount of training data required. In addition, we generate the prior term in the latent space instead of the pixel space, which encourages us to form a high-resolution image with lower computational complexity and sampling time. We ensured the convergence of optimizing the data fidelity term using gradient descent algorithm through theoretical derivation. Simultaneously, we also demonstrated that incorporating historical gradient information into the optimization process accelerates its convergence in LDIR. Our experimental results demonstrate that LDIR outperforms state-of-the-art methods including the supervised method on sparse CT data reconstruction and achieves competitive results on nature image restoration. We believe that our work offers the community a promising tool for leveraging the rapidly growing field of latent diffusion models to restore high-quality and high-resolution data from degraded measurements.

% We show theoretically that by utilizing the unconditional latent diffusion models as better prior term, we can achieve various CT reconstruct tasks with a single model and do not need paired data to our train network. In addition, we generate the prior term in the latent space instead of the pixel space, which encourages us to form a high-resolution image with lower computational complexity. Furthermore, we use history gradient information from data-fidelity term to guide the sample-level reconstruction process which provides high-quality results.

%	\section*{References}

\section*{Acknowledgments}
This work were supported in part by the Sichuan Science and Technology Program under Grant 2022JDJQO045 and Grant 2021JDJQ0024, in part by the National Natural Science Foundation of China under Grant 61871277 and Grant 62271335, in part by the Sichuan Health Commission Research Project under Grant 19PJ007, in part by the Chengdu Municipal Health Commission Research Project under Grant 2022053, in part by the Chengdu Key Research and development Support project under Grant 2021YF0501788SN.

%{\appendix[Proof of the Zonklar Equations]
%Use $\backslash${\tt{appendix}} if you have a single appendix:
%Do not use $\backslash${\tt{section}} anymore after $\backslash${\tt{appendix}}, only $\backslash${\tt{section*}}.
%If you have multiple appendixes use $\backslash${\tt{appendices}} then use $\backslash${\tt{section}} to start each appendix.
%You must declare a $\backslash${\tt{section}} before using any $\backslash${\tt{subsection}} or using $\backslash${\tt{label}} ($\backslash${\tt{appendices}} by itself
% starts a section numbered zero.)}

%{\appendices
%\section*{Proof of the First Zonklar Equation}
%Appendix one text goes here.
% You can choose not to have a title for an appendix if you want by leaving the argument blank
%\section*{Proof of the Second Zonklar Equation}
%Appendix two text goes here.}




 
% argument is your BibTeX string definitions and bibliography database(s)
%\bibliography{IEEEabrv,../bib/paper}
%
\bibliographystyle{IEEEtran}
\bibliography{tmm}
%\begin{thebibliography}{1}

%\newpage

%\section{Biography Section}
%If you have an EPS/PDF photo (graphicx package needed), extra braces are
% needed around the contents of the optional argument to biography to prevent
% the LaTeX parser from getting confused when it sees the complicated
% $\backslash${\tt{includegraphics}} command within an optional argument. (You can create
% your own custom macro containing the $\backslash${\tt{includegraphics}} command to make things
% simpler here.)
% 
%\vspace{11pt}
%
%\bf{If you include a photo:}\vspace{-33pt}
%\begin{IEEEbiography}[{% Figure removed}]{Michael Shell}
%Use $\backslash${\tt{begin\{IEEEbiography\}}} and then for the 1st argument use $\backslash${\tt{includegraphics}} to declare and link the author photo.
%Use the author name as the 3rd argument followed by the biography text.
%\end{IEEEbiography}
%
%\vspace{11pt}
%
%\bf{If you will not include a photo:}\vspace{-33pt}
%\begin{IEEEbiographynophoto}{John Doe}
%Use $\backslash${\tt{begin\{IEEEbiographynophoto\}}} and the author name as the argument followed by the biography text.
%\end{IEEEbiographynophoto}
%
%
%
%
%\vfill

\clearpage

\appendices

\section*{Proofs}
\setcounter{theorem}{1}
\begin{theorem}
	For the gradient descent algorithm used in~\cite{chung2022improving,chung2022diffusion}, we have the upper bound of GD as:
	\setcounter{equation}{13}
	\begin{equation}
		R(T) = \frac{1}{2\eps_T} D^2 + \frac{G^2}{2} \sum_{t=1}^{T} \eps_t^2.
	\end{equation}
	When $T \to \infty$, the GD algorithm is tend to convergence:
	\begin{equation}
		\lim_{T \to \infty} \frac{R(T)}{T} \leq \lim_{T \to \infty} \frac{1}{T} \left[\frac{1}{2\eps_T} D^2 + \frac{G^2}{2} \sum_{t=1}^{T} \eps_t^2\right] = 0.
	\end{equation}
	Considering $\eps_t$ is a function w.r.t $t$: $\eps_t = \epsilon(t)$, and a polynomial decay with a constant $C$ is employed: $\eps_t = C/t^n, n \geq 0$:
	\begin{equation}
		R(T) \leq D^2 \frac{T^n}{2 C} + \frac{G^2C}{2}\left(\frac{1}{1-n}T^{1-n} - \frac{n}{1-n}\right).
	\end{equation}
	Thus, $R(T) = \gO\left(T^{\max(n, 1-n)}\right)$. When $n=1/2$, $R(T)$ attain the optimal upper bound $\gO\left(T^{1/2}\right)$.
%	\label{proofofgd}
\end{theorem}
\begin{proof}
	\setcounter{equation}{26}
	Considering $\z^* = \argmin_\z \sum_{t=1}^{T} U_t (\z)$, we have:
	\begin{align}
		R(T) &= \sum_{t=1}^T U_t \left(\z^{(t)}\right) - \min_\z \sum_{t=1}^T U_t (\z) \\
		     &= \sum_{t=1}^T U_t \left(\z^{(t)}\right) - \sum_{t=1}^T U_t (\z^*) \\
		     &= \sum_{t=1}^T \left[\left(\z^{(t)}\right) - U_t (\z^*) \right],
	\end{align}
	because $U_t(\z)$ is a convex function, we have:
	\begin{align}
		& U_t (\z^*) \geq U_t \left(\z^{(t)}\right) + \left\langle \g_t, \z^{*} - \z^{(t)}\right\rangle \\
		\Longrightarrow & U_t \left(\z^{(t)}\right) - U_t (\z^*) \leq \left\langle \g_t,\z^{(t)} - \z^{*}\right\rangle.
	\end{align}
	Hence, we can substitute the above expression into $R(T)$:
	\begin{equation}
		R(T) \leq \sum_{t=1}^{T} \left\langle \g_t,\z^{(t)} - \z^{*}\right\rangle.
	\end{equation}
	For the GD algorithm, we have:
	\begin{align}
						\z^{(t+1)}  = & \z^{(t)} - \eps_t \g_t \\
		\Longrightarrow \z^{(t+1)} - \z^*  = & \z^{(t)} - \z^* - \eps_t \g_t \\
		\Longrightarrow \left\|\z^{(t+1)} - \z^*\right\|_2^2  = & \left\|\z^{(t)} - \z^* - \eps_t \g_t\right\|_2^2 \\
		\Longrightarrow \left\|\z^{(t+1)} - \z^*\right\|_2^2  = & \left\|\z^{(t)} - \z^*\right\|_2^2 - 2 \eps_t \left\langle \g_t,\z^{(t)} - \z^{*}\right\rangle \nonumber \\
		& + \eps_t^2 \|\g_t\|_2^2 \\
		\Longrightarrow \left\langle \g_t,\z^{(t)} - \z^{*}\right\rangle = & \frac{1}{2\eps_t}\left[\left\|\z^{(t)} - \z^*\right\|_2^2 - \left\|\z^{(t+1)} - \z^*\right\|_2^2 \right] \nonumber \\
		& + \frac{\eps_t}{2} \|\g_t\|_2^2.
	\end{align}
	Thus, the upper bound of $R(T)$ can be:
	\begin{align}
		R(T) & \leq \sum_{i=1}^T \frac{1}{2\eps_t}\left[\left\|\z^{(t)} - \z^*\right\|_2^2 - \left\|\z^{(t+1)} - \z^*\right\|_2^2 \right] + \frac{\eps_t}{2} \|\g_t\|_2^2 \\
			&=\underbrace{\sum_{i=1}^T \frac{1}{2\eps_t}\left[\left\|\z^{(t)} - \z^*\right\|_2^2 - \left\|\z^{(t+1)} - \z^*\right\|_2^2 \right]}_{(1)} \nonumber \\
			&+\underbrace{\sum_{i=1}^T \frac{\eps_t}{2} \|\g_t\|_2^2}_{(2)}. \label{eq:upper_bound_group}
	\end{align}
	For the $(1)$ in Eq.~\ref{eq:upper_bound_group}, we have:
	\begin{align}
		& \sum_{i=1}^T \frac{1}{2\eps_t}\left[\left\|\z^{(t)} - \z^*\right\|_2^2 - \left\|\z^{(t+1)} - \z^*\right\|_2^2 \right] \\
	   =& \frac{1}{2\eps_1}\left\| \z^{(1)} - \z^* \right\|_2^2 - \frac{1}{2\eps_1}\left\| \z^{(2)} - \z^* \right\|_2^2 + \\
	    & \frac{1}{2\eps_2}\left\| \z^{(2)} - \z^* \right\|_2^2 - \frac{1}{2\eps_2}\left\| \z^{(3)} - \z^* \right\|_2^2 + \cdots + \\
	    & \frac{1}{2\eps_T}\left\| \z^{(T)} - \z^* \right\|_2^2 - \frac{1}{2\eps_T}\left\| \z^{(T+1)} - \z^* \right\|_2^2 \\
	   =& \frac{1}{2\eps_1}\left\| \z^{(1)} - \z^* \right\|_2^2 + \sum_{t=2}^T\left(\frac{1}{2\eps_t} - \frac{1}{2\eps_{t-1}}\right)\left\|\z^{(t)} - \z^*\right\|_2^2 \\
	   &- \frac{1}{2\eps_T}\left\| \z^{(T+1)} - \z^* \right\|_2^2.
	\end{align}
	Since the latent variable $\z$ is bounded, we have:
	\begin{equation}
		\frac{1}{2\eps_1} \left\| \z^{(1)} - \z^* \right\|_2^2 \leq \frac{1}{2\eps_1} D^2.
	\end{equation}
	Since $\left\lbrace \eps_t\right\rbrace $ is monotonically non-decreasing and the latent variable $\z$ is bounded, we have:
	\begin{equation}
		\sum_{t=2}^T \left(\frac{1}{2\eps_t} - \frac{1}{2\eps_{t-1}}\right) \left\|\z^{(t)} - \z^*\right\|_2^2 \leq \sum_{t=2}^T \left(\frac{1}{2\eps_t} - \frac{1}{2\eps_{t-1}}\right) D^2.
	\end{equation}
	Clearly, $- \frac{1}{2\eps_T}\left\| \z^{(T+1)} - \z^* \right\|_2^2$ is less than or equal to 0. Therefore, we can rescale equation $(1)$ as:
	\begin{align}
		& \sum_{i=1}^T \frac{1}{2\eps_t}\left[\left\|\z^{(t)} - \z^*\right\|_2^2 - \left\|\z^{(t+1)} - \z^*\right\|_2^2 \right] \\
   \leq & \frac{1}{2\eps_1} D^2 + \sum_{t=2}^T \left(\frac{1}{2\eps_t} - \frac{1}{2\eps_{t-1}}\right) D^2 + 0 = \frac{1}{2\eps_T}D^2.
	\end{align}
	And for $(2)$, considering the gradient $\g$ is bounded, we have:
	\begin{equation}
		\sum_{i=1}^T \frac{\eps_t}{2} \|\g_t\|_2^2 \leq \sum_{i=1}^T \frac{\eps_t}{2} G^2 = \frac{G^2}{2} \sum_{t=1}^T \eps_t^2.
	\end{equation}
	Thus, we have the upper bound for $R(T)$:
	\begin{equation}
		R(T) = \frac{1}{2\eps_T} D^2 + \frac{G^2}{2} \sum_{t=1}^{T} \eps_t^2.
	\end{equation}
	Clearly, when $T \to \infty$, the above equation is tend to $0$:
	\begin{equation}
		\lim_{T \to \infty} \frac{R(T)}{T} \leq \lim_{T \to \infty} \frac{1}{T} \left[\frac{1}{2\eps_T} D^2 + \frac{G^2}{2} \sum_{t=1}^{T} \eps_t^2\right] = 0.
	\end{equation}
	Thus, we can conclude that the GD algorithm is tend to convergence when $T \to \infty$. Considering $\eps_t$ is a function w.r.t $t$: $\eps_t = \epsilon(t)$, and a polynomial decay with a constant $C$ is employed: $\eps_t = C/t^n, n \geq 0$:
	\begin{align}
		R(T) &\leq \frac{T^p}{2C} D^2 + \frac{G^2}{2} \sum_{t=1}^{T} \frac{C}{t^p} \\
		     &\leq \frac{T^p}{2C} D^2 + \frac{G^2C}{2} \left(1 + \int_{1}^{T}\frac{dt}{t^p}\right) \\
		     &= \frac{T^p}{2C} D^2 + \frac{G^2C}{2} \left(\frac{1}{1-p}T^{1-p} - \frac{p}{1-p}\right).
	\end{align}
	Thus, $R(T) = \gO\left(T^{\max(n, 1-n)}\right)$. When $n=1/2$, $R(T)$ attain the optimal upper bound $\gO\left(T^{1/2}\right)$.	
\end{proof}

\begin{theorem}
	\setcounter{equation}{17}
	For the Improved-Momentum-variant history gradient update, we have the upper bound as:
	\begin{align}
		& R(T) \leq \frac{\sum_{i=1}^{d}D^2_i G_i}{2\eps_T(1-\eta_{1,1})} + \left(\sum_{i=1}^{d}D_iG_i\right)\left(\sum_{t=1}^{T}\frac{\eta_{1,t}}{1-\eta_{1,t}}\right) + \nonumber \\
		& \left(\sum_{i=1}^{d}G_i\right)\left[\sum_{t=1}^{T}\frac{\gamma_t}{2(1-\eta_{1,t})} \cdot \sum_{s=1}^{t}\frac{(1-\eta_{1,s})^2\left(\prod_{r=s+1}^t\eta_{1,r}\right)^2}{(1-\eta_2)\eta_2^{t-s}}\right],
	\end{align}
	where $\gamma_t = \frac{\eps_t}{1 - \prod_{s=1}^t \eta_{1,s}}$. Similarly, 	When $T \to \infty$, this Improved-Momentum-variant history gradient update algorithm is tend to convergence. Considering $\eta_{1,t} \in (0, 1), \forall t, \eta_{1,1} \geq \eta_{1,2} \geq \dots \geq \eta_{1,T} \geq \dots$ and $\eta_2 \in (0,1), \frac{\eta_{1,t}}{\sqrt{\eta_2}}\leq\sqrt{c} < 1$, we have:
	\begin{align}
		& R(T) \leq \frac{\sum_{i=1}^{d}D^2_i G_i}{2\eps_T(1-\eta_{1,1})} + \left(\sum_{i=1}^{d}D_iG_i\right)\left(\sum_{t=1}^{T}\eta_{1,t}\right) + \nonumber \\
		& \left(\sum_{i=1}^{d}G_i\right)\left[\frac{\sum_{t=1}^{T}\eps_t}{2(1-\eta_{1,1})^2(1-\eta_2)(1-c)}\right].
	\end{align}
	Similar to Theorem~\ref{proofofgd}, when $n=1/2$, $R(T)$ attain the optimal upper bound $\gO\left(T^{1/2}\right)$.
%	\label{proofofadam}
\end{theorem}
\begin{proof}
	\setcounter{equation}{55}
	For the Improved-Momentum-variant algorithm, we have:
	\begin{align}
		\m^{(t)} &= \eta_1 \m^{(t-1)} + (1-\eta_1)\g_t, \hat{\m}^{(t)} = \frac{\m^{(t)}}{1-\eta_1^t}, \m^{(0)} = 0, \\
		\vbb^{(t)} &= \eta_2 \vbb^{(t-1)} + (1-\eta_2)\g_t^2, \hat{\vbb}^{(t)} = \frac{\vbb^{(t)}}{1-\eta_2^t},\vbb^{(0)}=0, \\
		\z^{(t+1)} &= \z^{(t)} - \eps_t \frac{\hat{\m}^{(t)}}{\sqrt{\hat{\vbb}^{(t)}}}, \; \g_t = \nabla U_t(\z^{(t)}).
	\end{align}
	We can calculate its closed-form solution based on the iterative equation for $\m^{(t)}$ as:
	\begin{align}
		\m^{(t)} =& \eta_1\m^{(t-1)} + (1-\eta_1)\g_t \\
		         =& \eta_1^2\m^{(t-2)} + \eta_1(1-\eta_1)\g_{t-1} + (1-\eta_1)\g_t \\
		         =& \eta_1^3\m^{(t-3)} + \eta_1^2(1-\eta_1)\g_{t-2} + \eta_1(1-\eta_1)\g_{t-1} + \nonumber \\ & (1-\eta_1)\g_t \\
		         =& \eta_1^t\m^{(0)} + \eta_1^{t-1}(1-\eta_1)\g_1 + \cdots + \eta_1^2(1-\eta_1)\g_{t-2} + \nonumber \\ & \eta_1(1-\eta_1)\g_{t-1} + (1-\eta_1)\g_t \\
		         \overset{\text{(a)}}{=} & \eta_1^{t-1}(1-\eta_1)\g_1 + \cdots + \eta_1^2(1-\eta_1)\g_{t-2} + \nonumber \\ & \eta_1(1-\eta_1)\g_{t-1} + (1-\eta_1)\g_t \\
		         =& (1-\eta_1)\sum_{s=1}^{t}\eta_1^{t-s}\g_s.
	\end{align}
	Where (a) holds because $\m^{(0)}=0$. Similarly, we have:
	\begin{equation}
		\vbb^{(t)} = (1-\eta_2)\sum_{s=1}^{t}\eta_2^{t-s}\g_s^2.
	\end{equation}
	For $\Ed\left[\m^{(t)}\right]$ and $\Ed\left[\vbb^{(t)}\right]$, we have:
	\begin{align}
		& \Ed\left[\m^{(t)}\right] = (1-\eta_1)\sum_{s=1}^{t}\eta_1^{t-s} \Ed \left[\g_s\right] = \Ed \left[\g_s\right] \cdot (1-\eta_1^t), \\ 
		&\Ed\left[\vbb^{(t)}\right] = (1-\eta_2)\sum_{s=1}^{t}\eta_2^{t-s} \Ed \left[\g_s^2\right] = \Ed \left[\g_s^2\right] \cdot (1-\eta_2^t).
	\end{align}
	In order to make $\Ed\left[\m^{(t)}\right]=\Ed \left[\g_s\right]$ and $\Ed\left[\vbb^{(t)}\right] = \Ed \left[\g_s^2\right]$, we need to make the following corrections:
	\begin{align}
		\m^{(t)} &\rightarrow \hat{\m}^{(t)} = \frac{\m^{(t)}}{1-\eta_1^t}, \\
		\vbb^{(t)} &\rightarrow  \hat{\vbb}^{(t)} = \frac{\vbb^{(t)}}{1-\eta_2^t}.
	\end{align}
	We break down the upper bound of $R(T)$ into individual dimensions of the variables as:
	\begin{align}
		R(T) \leq& \sum_{t=1}^{T} \left\langle \g_t,\z^{(t)} - \z^{*}\right\rangle \\
			 	=& \sum_{t=1}^{T} \sum_{i=1}^{d} g_{t,i}\left(z_i^{(t)} - z_i^*\right) \\
			 	=& \sum_{i=1}^{d} \sum_{t=1}^{T} g_{t,i}\left(z_i^{(t)} - z_i^*\right).
	\end{align}
	For any dimension $i$ of the variable, we have:
	\begin{align}
		z_i^{(t+1)} &= z_i^{(t+1)} - \eps_t \frac{\hat{m}^{(t)}_i}{\sqrt{\hat{v}^{(t)}_i}} \\
					&= z_i^{(t+1)} - \eps_t \frac{1}{1-\eta_1^t}\frac{m^{(t)}_i}{\sqrt{\hat{v}_i^{(t)}}} \\
					&= z_i^{(t+1)} - \eps_t\frac{1}{1-\eta_1^t}\frac{\eta_1 m_i^{(t-1)} + (1-\eta_1)g_{t,i}}{\sqrt{\hat{v}_i^{(t)}}}.
	\end{align}
	Therefore, according to~\cite{kingma2014adam}, we define $\eta_1=\eta_{1,t}$, allowing $\eta_1$ to change with an increase in the number of iterations, and $\eta_{1,t}$ is non-decreasing. This implies that momentum gradually diminishes, and eventually, $m_i$ approaches $g_i$. Such that, $z_i^{(t+1)}$ changes to:
	\begin{align}
	&	z_i^{(t+1)} = z_i^{(t)} - \gamma_t \frac{\eta_{1,t}m_i^{t-1}+\left(1-\eta_{1,t}\right)g_{t,i}}{\sqrt{\hat{v}_i^{(t)}}} \\
		&\Longrightarrow \left(z_i^{(t+1)} - z_i^*\right)^2 \nonumber \\
		&= \left[\left( z_i^{(t)}  - z_i^*\right) - \gamma_t \frac{\eta_{1,t}m_i^{t-1}+\left(1-\eta_{1,t}\right)g_{t,i}}{\sqrt{\hat{v}_i^{(t)}}}\right]^2 \\
		&\Longrightarrow 2\gamma_t \frac{\eta_{1,t}m_i^{t-1}+\left(1-\eta_{1,t}\right)g_{t,i}}{\sqrt{\hat{v}_i^{(t)}}}\left( z_i^{(t)}  - z_i^*\right) = \nonumber \\
		& \left( z_i^{(t)}  - z_i^*\right)^2 - \left(z_i^{(t+1)} - z_i^*\right)^2 \nonumber \\
		&+\gamma_t \frac{\left[ \eta_{1,t}m_i^{t-1} +  \left(1-\eta_{1,t}\right)g_{t,i}\right] ^2}{\sqrt{\hat{v}_i^{(t)}}}.
%		\gamma_t = \eps_t \frac{1}{1 - \prod_{s=1}^{t}\eta_{1,s}}
	\end{align}
	Here, $\gamma_t = \eps_t \frac{1}{1 - \prod_{s=1}^{t}\eta_{1,s}}$. Considering $m^{(t)}_i=\eta_{1,t}m^{(t-1)} + (1-\eta_{1,t})g_{t,i}$, we have:
	\begin{align}
		& g_{t,i}\left(z^{(t)}_i - z^*_i\right) = \underbrace{\frac{\sqrt{\hat{v}_i^{(t)}}\left[\left( z_i^{(t)}  - z_i^*\right)^2 - \left(z_i^{(t+1)} - z_i^*\right)^2\right]}{2\gamma_t(1-\eta_{1,t})}}_{(1)} \nonumber \\
		& - \underbrace{\frac{\eta_{1,t}}{1 - \eta_{1,t}}m_i^{(t-1)}\left( z_i^{(t)}  - z_i^*\right)}_{(2)} + \underbrace{\frac{\gamma_t}{2(1-\eta_{1,t})}\frac{\left(m_i^{(t)}\right)^2}{\sqrt{\hat{v}_i^{(t)}}}}_{(3)}.
		\label{eq:adam_item}
	\end{align}
	In order to get the upper bound of $R(T)$, we need to rescale $(1)$, $(2)$, and $(3)$ in the Eq.~\ref{eq:adam_item}. For $(1)$, we have:
	\begin{align}
		&\sum_{t=1}^{T} \frac{\sqrt{\hat{v}_i^{(t)}}\left[\left( z_i^{(t)}  - z_i^*\right)^2 - \left(z_i^{(t+1)} - z_i^*\right)^2\right]}{2\gamma_t(1-\eta_{1,t})} \\
	   =&\sum_{t=1}^{T} \frac{\sqrt{\hat{v}_i^{(t)}}\left[\left( z_i^{(t)}  - z_i^*\right)^2 - \left(z_i^{(t+1)} - z_i^*\right)^2\right]}{2  \eps_t \frac{1}{1 - \prod_{s=1}^{t}\eta_{1,s}} (1-\eta_{1,t})} \\
	   =&\sum_{t=1}^{T} \frac{\sqrt{\hat{v}_i^{(t)}}\left[\left( z_i^{(t)}  - z_i^*\right)^2 - \left(z_i^{(t+1)} - z_i^*\right)^2\right]\left(1 - \prod_{s=1}^{t}\eta_{1,s}\right)}{2  \eps_t (1-\eta_{1,t})} \\
	   \leq&\sum_{t=1}^{T} \frac{\sqrt{\hat{v}_i^{(t)}}\left[\left( z_i^{(t)}  - z_i^*\right)^2 - \left(z_i^{(t+1)} - z_i^*\right)^2\right]}{2\eps_t(1-\eta_{1,1})}.
	\end{align}
	We obtain the result using permutation recombination and summation:
	\begin{align}
		&\sum_{t=1}^{T} \frac{\sqrt{\hat{v}_i^{(t)}}\left[\left( z_i^{(t)}  - z_i^*\right)^2 - \left(z_i^{(t+1)} - z_i^*\right)^2\right]}{2\eps_t(1-\eta_{1,1})} \\
		=&\sum_{t=1}^{T} \frac{\sqrt{\hat{v}_i^{(t)}}\left(z_i^{(t)} - z_i^* \right)^2}{2\eps_t(1-\eta_{1,1})} - \frac{\sqrt{\hat{v}_i^{(t)}}\left(z_i^{(t+1)} - z_i^* \right)^2}{2\eps_t(1-\eta_{1,1})} \\
		=&\frac{\sqrt{\hat{v}_i^{(1)}}\left(z_i^{(1)} - z_i^* \right)^2}{2\eps_1 (1-\eta_{1,1})} - \frac{\sqrt{\hat{v}_i^{(T)}}\left(z_i^{(T+1)} - z_i^* \right)^2}{2\eps_T(1-\eta_{1,1})} + \nonumber \\
		& \sum_{t=2}^{T} \left(z_i^{(t)} - z_i^* \right)^2 \cdot \left[\frac{\sqrt{\hat{v}_i^{(t)}}}{2\eps_t(1-\eta_{1,1})} - \frac{\sqrt{\hat{v}_i^{(t-1)}}}{2\eps_{t-1}(1-\eta_{1,1})}\right].
	\end{align}
	We mainly focus on the last term. Considering $\frac{\sqrt{\hat{v}_i^{(t)}}}{2\eps_t(1-\eta_{1,1})} \geq \frac{\sqrt{\hat{v}_i^{(t-1)}}}{2\eps_{t-1}(1-\eta_{1,1})}, \forall t$, we have:
	\begin{align}
		& \sum_{t=2}^{T} \left(z_i^{(t)} - z_i^* \right)^2 \cdot \left[\frac{\sqrt{\hat{v}_i^{(t)}}}{2\eps_t(1-\eta_{1,1})} - \frac{\sqrt{\hat{v}_i^{(t-1)}}}{2\eps_{t-1}(1-\eta_{1,1})}\right] \\
	\leq& \sum_{t=2}^{T} D^2_i \cdot \left[\frac{\sqrt{\hat{v}_i^{(t)}}}{2\eps_t(1-\eta_{1,1})} - \frac{\sqrt{\hat{v}_i^{(t-1)}}}{2\eps_{t-1}(1-\eta_{1,1})}\right] \\
	   =& D_i^2 \left[\frac{\sqrt{\hat{v}_i^{(T)}}}{2\eps_T(1-\eta_{1,1})} - \frac{\sqrt{\hat{v}_i^{(1)}}}{2\eps_{1}(1-\eta_{1,1})}\right].
	\end{align}
	Because $\frac{\sqrt{\hat{v}_i^{(1)}}\left(z_i^{(1)} - z_i^* \right)^2}{2\eps_1 (1-\eta_{1,1})} \leq \frac{D^2_i \sqrt{\hat{v}_i^{(1)}}}{2\eps_1(1-\eta_{1,1})}, \frac{\sqrt{\hat{v}_i^{(T)}}\left(z_i^{(T+1)} - z_i^* \right)^2}{2\eps_T(1-\eta_{1,1})} \leq 0$, we can rescale $(1)$ to:
	\begin{align}
		& \sum_{t=1}^{T} \frac{\sqrt{\hat{v}_i^{(t)}}\left[\left( z_i^{(t)}  - z_i^*\right)^2 - \left(z_i^{(t+1)} - z_i^*\right)^2\right]}{2\gamma_t(1-\eta_{1,t})} \\
		\leq & \left[\frac{^2_i\sqrt{\hat{v}_i^{(T)}}}{2\eps_T(1-\eta_{1,1})} - \frac{D^2_i\sqrt{\hat{v}_i^{(1)}}}{2\eps_1(1-\eta_{1,1})}\right] + \frac{D_i^2 \sqrt{\hat{v}_i^{(1)}}}{2\eps_T(1-\eta_{1,1})} \\
		\leq & \frac{^2_i\sqrt{\hat{v}_i^{(T)}}}{2\eps_T(1-\eta_{1,1})}.
	\end{align}
	We focus on $\hat{v}_i^{(T)}$ for further scaling. As mentioned earlier, $v_i^{(t)} = (1 - \eta_2)\sum_{s=1}^{t}\eta_{2,t-s}g^2_{s,i}$, so we can explore its boundedness based on the earlier gradient bounded assumption:
	\begin{align}
		& \left.\begin{array}{cc}
			v_i^{(t)} & \leq \\
			\hat{v}_i^{(t)} & = \\
		\end{array}\right\rbrace \frac{v_i^{(t)}}{1-\eta_{2}^{t-s}} \leq \frac{(1-\eta_2)\sum_{s=1}^{t}\eta_{2}^{t-s}G_i^2}{1-\eta_{2,t}} \\
		& = \frac{G_i^2(1-\eta_{2}^{t-s})}{1-\eta_{2,t}} = G_i^2.
	\end{align}
	So, finally, $(1)$ is scaled to:
	\begin{align}
		& \sum_{t=1}^{T} \frac{\sqrt{\hat{v}_i^{(t)}}\left[\left( z_i^{(t)}  - z_i^*\right)^2 - \left(z_i^{(t+1)} - z_i^*\right)^2\right]}{2\gamma_t(1-\eta_{1,t})} \\
	\leq& \frac{D^2_i\sqrt{\hat{v}_i^{(T)}}}{2\eps_T(1-\eta_{1,1})} \leq \frac{D_i^2 G_i}{2\eps_T(1-\eta_{1,1})}.
	\end{align}
	Regarding $(2)$, we first apply the variable bounded assumption:
	\begin{align}
		& \sum_{t=1}^{T}\frac{- \eta_{1,t}}{1-\eta_{1,t}}\left(z^{(t)}_i - z^*_i\right) \\
	   =& \sum_{t=1}^T\frac{\eta_{1,t}}{1-\eta_{1,t}}m^{(t-1)}_i \left[-\left(z^{(t)}_i - z^*_i\right)\right] \\
	\leq&  \sum_{t=1}^T\frac{\eta_{1,t}}{1-\eta_{1,t}}m^{(t-1)}_i \vert m_i^{(t-1)} \vert D_i.
	\end{align}
	Now, we focuses on $m_i^{(t-1)}$:
	\begin{align}
		m_i^{(t)} 
		=& \eta_{1,t} m_i^{(t-1)} + (1- \eta_{1,t}) g_{t,i} \\
		=& \eta_{1,t} \eta_{1,t-1} m_i^{(t-2)} + \eta_{1,t}(1-\eta_{1,t-1}) g_{t-1,i} + \nonumber \\ 
		&(1 - \eta_{1,t})g_{t,i} \\
		=& \eta_{1,t} \eta_{1,t-1} \eta_{1,t-2} m_i^{(t-3)} + \eta_{1,t}\eta_{1,t-1}(1-\eta_{1,t-2})g_{t-2,i} \nonumber \\
		& + \eta_{1,t}(1-\eta_{1,t-1}) g_{t-1,i} + (1 - \eta_{1,t})g_{t,i} \\
		=& \eta_{1,t} \eta_{1,t-1} \cdots \eta_{1,1} m_i^{(0)}+\eta_{1,t}\eta_{1,t-1}\cdots(1-\eta_{1,1})g_{1,i} + \nonumber \\
		&\cdots+\eta_{1,t}\eta_{1,t-1}(1-\eta_{1,t-2})g_{t-2,i}+ \nonumber \\ & \eta_{1,t}(1-\eta_{1,t-1}) g_{t-1,i} + (1 - \eta_{1,t})g_{t,i} \\
		\overset{\text{(a)}}{=}& \eta_{1,t}\eta_{1,t-1}\cdots(1-\eta_{1,1})g_{1,i} +\cdots+ \nonumber \\
		& \eta_{1,t}\eta_{1,t-1}(1-\eta_{1,t-2})g_{t-2,i}+ \nonumber \\ 
		& \eta_{1,t}(1-\eta_{1,t-1}) g_{t-1,i} + (1 - \eta_{1,t})g_{t,i} \\
		=& \sum_{s=1}^t (1 - \eta_{1,s}) \left(\prod_{k=s+1}^t \eta_{1,k}\right) g_{s,i}.
	\end{align}
	Where (a) holds because $m_i ^{(0)}=0$. Here, we apply the gradient bounded assumption, for any $t$ we have:
	\begin{align}
		\vert m_i^{(t)} \vert & \leq \sum_{s=1}^t (1-\eta_{1,s}) \left(\prod_{k=s+1}^t \eta_{1,k} \right) \vert g_{s,i} \vert \\
		& \leq \sum_{s=1}^t (1-\eta_{1,s}) \left(\prod_{k=s+1}^t \eta_{1,k} \right) G_i \\
		& = G_i \left(1 - \sum_{s=1}^t \eta_{1,s} \right) \leq G_i.
	\end{align}
	This way, we can scale equation $(2)$:
	\begin{align}
		& \sum_{t=1}^T \frac{-\eta_{1,t}}{1 - \eta_{1,t}} m_i^{(t-1)} \left(z_i^{(t)} - z_i^*\right) \\
   \leq & \sum_{t=1}^T \frac{\eta_{1,t}}{1 - \eta_{1,t}} m_i^{(t-1)} G_i D_i \\
      = & G_i D_i \sum_{t=1}^{T} \frac{\eta_{1,t}}{1-\eta_{1,t}}.
	\end{align}
	About $(3)$, we mainly focus on $\frac{\left(m_i^{(t)}\right)^2}{\sqrt{\hat{v}_i^{(t)}}} = \sqrt{1 - \eta_2^t} \frac{\left(m_i^{(t)}\right)^2}{\sqrt{\hat{v}_i^{(t)}}} \leq \frac{\left(m_i^{(t)}\right)^2}{\sqrt{\hat{v}_i^{(t)}}}$, we have:
	\begin{align}
		m_i^{(t)} &= \sum_{s=1}^t (1 - \eta_{1,s}) \left(\prod_{k=s+1}^t \eta_{1,k} \right) g_{s,i}, \\
		v_i^{(t)} &= (1 - \eta_2) \sum_{s=1}^{t} \eta_2^{t-s}.
	\end{align}
	Then, we transform $\left(m_i^{(t)}\right)^2$ to:
	\begin{align}
		\left(m_i^{(t)}\right)^2 =& \left(\sum_{s=1}^{t} \frac{(1-\eta_{1,s})\left(\prod_{k=s+1}^t\eta_{1,k}\right)}{\sqrt{(1-\eta_2) \eta_2^{t-s}}} \cdot \sqrt{(1-\eta_2)\eta_2^{t-s}}g_{s,i} \right)^2 \\
		=& \sum_{s=1}^t \left(\frac{(1-\eta_{1,s})\left(\prod_{k=s+1}^t\eta_{1,k}\right)}{\sqrt{(1-\eta_2) \eta_2^{t-s}}}\right)^2 \nonumber \\
		& \cdot \sum_{s=1}^t \left( \sqrt{(1-\eta_2)\eta_2^{t-s}}g_{s,i} \right)^2 \\
		=& \sum_{s=1}^t \frac{(1-\eta_{1,s})^2\left(\prod_{k=s+1}^t\eta_{1,k}\right)^2}{(1-\eta_2) \eta_2^{t-s}} \cdot \sum_{s=1}^t (1-\eta_2)\eta_2^{t-s}g_{s,i}^2.
	\end{align}
	So that, we can derive $(3)$ to:
	\begin{align}
		& \sum_{t=1}^T \frac{\gamma_t}{2(1-\eta_{1,t})} \frac{\left(m_i^{(t)}\right)^2}{\sqrt{\hat{v}_i^{(t)}}} \leq \sum_{t=1}^T \frac{\gamma_t}{2(1-\eta_{1,t})} \cdot \\
		&\sum_{s=1}^t \frac{(1-\eta_{1,s})^2\left(\prod_{k=s+1}^{t} \eta_{1,k} \right)^2}{(1-\eta_2)\eta_2^{t-s}} \cdot \frac{\sum_{s=1}^t (1-\eta_2)\eta_2^{t-s}g_{s,i}^2}{\sqrt{\sum_{s=1}^t (1-\eta_2)\eta_2^{t-s}g_{s,i}^2}} \\
	   =&\sum_{t=1}^{T} \frac{\gamma_t}{2(1-\eta_{1,t})} \sum_{s=1}^t \frac{(1-\eta_{1,s})^2\left(\prod_{k=s+1}^t\eta_{1,k}\right)^2}{(1-\eta_2)\eta_2^{t-s}}\sqrt{v_i^{(t)}} \\
	\leq&\sum_{t=1}^T \frac{\gamma_t}{2(1-\eta_{1,t})} \sum_{s=1}^t \frac{(1-\eta_{1,s})^2\left(\prod_{k=s+1}^t\eta_{1,k}\right)^2}{(1-\eta_2)\eta_2^{t-s}} \cdot G_i.
	\end{align}
	Finally, we have the upper bound of $R(T)$:
	\begin{align}
		R(T) & \leq \frac{\sum_{i=1}^d D_i^2 G_i}{2\eps_T(1-\eta_{1,1})} + \sum_{i=1}^{d} G_i D_i \sum_{t=1}^T \frac{\eta_{1,t}}{1-\eta_{1,t}} \nonumber \\
		& + \sum_{i=1}^{d} G_i \sum_{t=1}^T \frac{\gamma_t}{2(1-\eta_{1,t})} \cdot \nonumber \\
		& \sum_{s=1}^{t} \frac{(1-\eta_{1,s})^2\left(\prod_{k=s+1}^t \eta_{1,k}\right)}{(1-\eta_2)\eta_2^{t-s}} \\
		& = \frac{\sum_{i=1}^{d}D^2_i G_i}{2\eps_T(1-\eta_{1,1})} + \left(\sum_{i=1}^{d}D_iG_i\right)\left(\sum_{t=1}^{T}\frac{\eta_{1,t}}{1-\eta_{1,t}}\right) + \nonumber \\
		& \left(\sum_{i=1}^{d}G_i\right)\left[\sum_{t=1}^{T}\frac{\gamma_t}{2(1-\eta_{1,t})} \cdot \right. \nonumber \\
		& \left. \sum_{s=1}^{t}\frac{(1-\eta_{1,s})^2\left(\prod_{r=s+1}^t\eta_{1,r}\right)^2}{(1-\eta_2)\eta_2^{t-s}}\right],
	\end{align}
	Clearly, when $T \to \infty$, the above equation is tend to $0$:
	\begin{align}
		\lim_{T \to \infty} \frac{R(T)}{T} & \leq \lim_{T \to \infty} \frac{1}{T} \left[ \frac{\sum_{i=1}^{d}D^2_i G_i}{2\eps_T(1-\eta_{1,1})} \right. + \nonumber \\
		&\left(\sum_{i=1}^{d}D_iG_i\right)\left(\sum_{t=1}^{T}\frac{\eta_{1,t}}{1-\eta_{1,t}}\right) + \nonumber \\
		&\left(\sum_{i=1}^{d}G_i\right)\left[\sum_{t=1}^{T}\frac{\gamma_t}{2(1-\eta_{1,t})} \cdot \right. \nonumber \\
		& \left. \sum_{s=1}^{t}\frac{(1-\eta_{1,s})^2\left(\prod_{r=s+1}^t\eta_{1,r}\right)^2}{(1-\eta_2)\eta_2^{t-s}}\right] = 0.
	\end{align}
	 Thus, we can conclude that the Improved-Momentum-variant history gradient update algorithm is tend to convergence. Considering $\eta_{1,t} \in (0, 1), \forall t, \eta_{1,1} \geq \eta_{1,2} \geq \dots \geq \eta_{1,T} \geq \dots$ and $\eta_2 \in (0,1), \frac{\eta_{1,t}}{\sqrt{\eta_2}}\leq\sqrt{c} < 1$, about $(2)$, we have:
	 \begin{align}
	 	& \left(\sum_{i=1}^{d}D_iG_i\right)\left(\sum_{t=1}^{T}\frac{\eta_{1,t}}{1-\eta_{1,t}}\right) \nonumber \\
	 \leq&
	     \left(\sum_{i=1}^{d}D_iG_i\right)\left(\frac{1}{1-\eta_{1,t}}\sum_{t=1}^{T}\eta_{1,t}\right).
	 \end{align}
	 About $(3)$, we have:
	 \begin{align}
	 	& \left(\sum_{i=1}^{d}G_i\right)\left[\sum_{t=1}^{T}\frac{\gamma_t}{2(1-\eta_{1,t})} \cdot \sum_{s=1}^{t}\frac{(1-\eta_{1,s})^2\left(\prod_{r=s+1}^t\eta_{1,r}\right)^2}{(1-\eta_2)\eta_2^{t-s}}\right] \\
	   =& \left(\sum_{i=1}^{d}G_i\right) \cdot \sum_{t=1}^T \frac{\eps_t}{2(1-\eta_{1,t})\left(1-\prod_{s=1}^t\eta_{1,s}\right)} \cdot \nonumber \\
	    &\sum_{s=1}^t \left(\frac{1-\eta_{1,s}}{\sqrt{1-\eta_2}}\right)^2 \cdot \prod_{k=s+1}^t\left(\frac{\eta_{1,k}}{\sqrt{\eta_2}}\right) \\
	   \overset{\text{(a)}}{\leq}& \left(\sum_{i=1}^{d}G_i\right) \cdot \sum_{t=1}^T \frac{\eps_t}{2(1-\eta_{1,t})^2 (1-\eta_2)} \sum_{s=1}^t \prod_{k=s+1}^t c \\
	   =& \left(\sum_{i=1}^{d}G_i\right) \cdot \sum_{t=1}^T \frac{\eps_t}{2(1-\eta_{1,t})^2(1-\eta_2)(1-c)} \\
	   =& \left(\sum_{i=1}^{d}G_i\right) \cdot \frac{  \sum_{t=1}^T \eps_t}{2(1-\eta_{1,t})^2(1-\eta_2)(1-c)}.
	 \end{align}
	Where (a) holds because the assumption $\frac{\eta_{1,t}}{\sqrt{\eta_2}}\leq\sqrt{c} < 1$. 	Similar to Theorem~\ref{proofofgd}, when $n=1/2$, $R(T)$ attain the optimal upper bound $\gO\left(T^{1/2}\right)$.
\end{proof}

\section*{Experimental Details}
\label{app:experiment}
\subsection{Model Details}
Here, we present the parameters utilized in our experiments for each dataset.
\begin{table*}[t]
	\centering
	\caption{Model hyperparameters for the unconditional DDPM and LDMs used in the AAPM CT evaluation experiments. All models are trained on a single RTX 4090.}
	\label{tab:hyperparameters}
	\setlength{\tabcolsep}{18pt}
	\begin{tabular}{@{}lccc@{}}
			\toprule
			& \textbf{DDPM} $256\time256$   & \textbf{LDM} $256\time256$    & \textbf{LDM}  $512\time512$      \\ \midrule
			Latent shape             & -       & $32 \times 32 \times 4$     & $32 \times 32 \times 4$         \\
			Input shape              & $256 \times 256$     & $32 \times 32$      & $32 \times 32$         \\
			Diffusion steps          & 1000    & 1000    & 1000       \\
			Noise schedule           & linear  & linear  & linear     \\
			U-Net Channels           & 128     & 128     & 128        \\
			U-Net Channel Multiplier & 1,2,2,4 & 1,2,2,4 & 1,2,2,4    \\
			U-Net Depth              & 2       & 2       & 2          \\
			U-Net Batch size         & 4       & 32      & 32         \\
			U-Net Epochs             & 500     & 500     & 500        \\
			U-Net Learning Rate      & 1e-4    & 1e-4    & 1e-4       \\
			VQVAE Channels           & -       & 128     & 32         \\
			VQVAE Channel Multiplier & -       & 1,1,2,4 & 1,4,8,8,16 \\
			VQ Dimension             & -       & 4       & 4          \\
			Numbers of VQ embedding  & -       & 16384   & 16384      \\
			VQVAE Batch size         & -       & 4       & 4          \\
			VQVAE Epochs             & -       & 251     & 251        \\
			VQVAE Learning rate      & -       & 4.5e-5  & 4.5e-5     \\ \bottomrule
		\end{tabular}
\end{table*}

\begin{itemize}
	\item \textbf{CelebaHQ. } This pretrained DDPM model can be accessed from the \href{https://huggingface.co/google/ddpm-ema-celebahq-256}{\textcolor{blue}{huggingface model zoo}}. The pretrained LDM model can be accessed from the \href{https://huggingface.co/CompVis/ldm-celebahq-256}{\textcolor{blue}{huggingface model zoo}}.
	\item \textbf{LSUN Bedroom. } This pretrained DDPM model can be accessed from the \href{https://huggingface.co/google/ddpm-ema-bedroom-256}{\textcolor{blue}{huggingface model zoo}}. The pretrained LDM model can be accessed from the \href{https://github.com/CompVis/latent-diffusion#pretrained-ldms}{\textcolor{blue}{Github repository}}.
	\item \textbf{AAPM CT. } Both the DDPM and LDM models are trained from scratch using the AAPM CT dataset. The hyperparameters employed for the models can be found in Table~\ref{tab:hyperparameters}.
\end{itemize}
All models were trained using a single RTX 4090 GPU, with the training process taking approximately 1 day, 2 days, and 5 days for each respective model. The reconstruction process running on a single RTX 4090 GPU.

\subsection{Details of the measurement operators}
\textbf{CT Reconstruction. } The measurement operator of CT reconstruction can be defined as:
\begin{align}
	\y &= \Ac\x + \n \\
	\y &= \mathbf{P}(\mathbf{\Lambda})\x + \n,
\end{align}
here, we define $\mathbf{P}$ as the discretized Radon transform, which is utilized in CT to generate Sinogram data. Additionally, $\mathbf{\Lambda}$ represents the selection mask matrix, determining the chosen views for measurement. Consequently, the disparity between sparse-view and limited-angle CT reconstruction lies solely in the variation of $\mathbf{\Lambda}$. Throughout our experiments, we set the noise variable $\n$ to 0.

\textbf{Random Inpainting. } The measurement operator of random inpainting can be defined as:
\begin{align}
	\y &= \Ac \x + \n \\
	\y &= \mathbf{M} \circ \x + \n , \n \sim \Nc(\bm{0}, \Ib),
\end{align}
where we define $\mathbf{M}$ as a random masking matrix with the same shape as $\x$ and comprising elementary unit vectors. The symbol $\circ$ represents the Hadamard product. In our experiments, we set $\sigma$ to a value of 0.05.

\textbf{Super Resolution. } The measurement operator of super-resolution can be defined as:
\begin{align}
	\y &= \Ac \x + \n \\
	\y &= \mathbf{H} \x + \n , \n \sim \Nc(\bm{0}, \Ib),
\end{align}
where we denote $\mathbf{H}$ as the bilinear downsampling catalecticant matrix. In our experiments, we set the value of $\sigma$ to 0.05.

The hyperparameters used for the LHGU guiding process with different measurement operators are presented in Table~\ref{tab:exp_params}.

\begin{table*}[t]
	\centering
	\caption{Guidance hyperparameters of LHGU for AAPM CT and nature image experiments.}
	\label{tab:exp_params}
%	\resizebox{\textwidth}{!}{%
		\begin{tabular}{@{}lcccc@{}}
				\toprule
				& \multicolumn{1}{c} {\textbf{Sparse view CT}} & \multicolumn{1}{c}{\textbf{Limited angle CT}} & \multicolumn{1}{c}{\textbf{Inpainting}} & \multicolumn{1}{c}{\textbf{Super-Resolution}} \\ \midrule
				Evaluation function $\mathcal{U}$           & $L1$                                          & $L2$                                            & $L2$                                      & $L2$                                            \\
				Guidance rate $\epsilon$                  & 0.5                                         & 0.1                                           & 0.05                                    & 0.001                                         \\
				History gradient update & Improved-Momentum-variant                                   & Improved-Momentum-variant                                     & Improved-Momentum-variant                               & Improved-Momentum-variant                                     \\ \bottomrule
			\end{tabular}
%	}
\end{table*}

\end{document}


