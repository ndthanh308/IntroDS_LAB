\begin{abstract}

%YouTube has unveiled features in recent years to give users agency over their own recommendations. These features let users indicate disinterest towards topics and videos, and are purported to alter future recommendations in order to better accommodate their requests. In theory, this allows users to ``train" their recommender system.

% Introduce ``Training'' term

YouTube provides features for users to indicate disinterest when presented with unwanted recommendations, such as the ``Not interested'' and ``Don't recommend channel'' buttons. These buttons are purported to allow the user to correct ``mistakes'' made by the recommendation system. Yet, relatively little is known about the empirical efficacy of these buttons. Neither is much known about users' awareness of and confidence in them. To address these gaps, 
we simulated YouTube users with sock puppet agents. Each agent first executed a ``stain phase'', where it watched many videos of one assigned topic; then it executed a ``scrub phase'', where it tried to remove recommendations of the assigned topic. Each agent repeatedly applied a single scrubbing strategy, either indicating disinterest in one of the videos visited in the stain phase (disliking it or deleting it from the watch history), or indicating disinterest in a video recommended on the homepage (clicking the ``not interested" or ``don't recommend channel" button or opening the video and clicking the dislike button).
We found that the stain phase significantly increased the fraction of the recommended videos on the user's homepage dedicated to the assigned topic.
\al{
%\marginnote{SPC-1}
For the scrub phase, using the ``Not interested'' button worked best, significantly reducing such recommendations in all topics tested, on average removing 88\% of them.}
Neither the stain phase nor the scrub phase, however, had much effect on videopage recommendations (those given to users while they watch a video). 
We also ran a survey ($N$=300) asking adult YouTube users in the US whether they were aware of and used these buttons before, as well as how effective they found these buttons to be. We found that 44\% of participants were not aware that the ``Not interested'' button existed. However, those who were aware of this button often used it to remove unwanted recommendations (82.8\%) and found it to be modestly effective (3.42 out of 5).
\end{abstract}



%YouTube has introduced various features on its platform to allow users to remove content they don't like. In this study, we investigate whether these features are indeed effective at removing unwanted recommendations by simulating users trying to remove content from different topics. We further examined whether users were aware of and used these features in their own experience, and how effective they found them to be.
%We first found that it was impossible for our scrubbing strategies to change the prevalence topical videopage recommendations (those presented when watching a video), when the video being watched belonged to that topic.
%Focusing on the YouTube homepage, we found that using the ``not-interested'' button was the most effective strategy: It was the only strategy to significantly reduce topical recommendations for all topics investigated; On average, it also reduced topical recommendations by the greatest amount (-97\%). Videopage recommendations (those presented when watching a video) did not experience significant changes as a result of using any button.
%We further found that not-interested performed well in reducing recommendations from both channels whose videos it pressed ``not-interested'' on, as well as other topical channels it didn't explicitly interact with (0-2\% of recommendations).
%Looking at the results of a survey of real users' interactions with these features we find that 44\% of users were not aware that the not-interested button existed. Additionally, those who knew about the not-interested button actually used it to remove unwanted recommendations at a higher rate (83\%) than those who knew about the less effective strategies. Such findings point to the need for YouTube to more widely inform its users of effective ways to remove unwanted content. 

%Recent years have seen many social media companies adopt platform features that allow their users to remove unwanted content from their feed and recommendations.
%YouTube, one of the world's most popular video-sharing platforms, is no exception; Recognizing that their recommendation engine sometimes ``makes mistakes'', it released features such as the ``not-interested'' and ``no-channel'' button that purport to allow users to better tailor their content.
%Yet, relatively little is known about the empirical efficacy, as well as users' understanding and relationship with, these buttons.
%In this study, we simulated YouTube users (agents) that interacted with the platform to try and remove content belonging an assigned topic by repeatedly using one of these platform-provided buttons. We collected recommendations from agents along the way and labeled them as ``stain'' if they belonged to their assigned topic.
%We first found that watching many videos of their assigned topic (the ``stain phase'') significantly increased stain on the agent's homepage to varying degrees, though never reaching more than half. 
%Then, during the ``scrub phase'', agents repeatedly used their assigned button to try and remove their stain.
%Our key finding here was that the ``Not interested'' button worked best by significantly reducing stain in all topics tested, on average removing 97\% of it.
%Across the stain and scrub phase, the videopage rarely experienced any significant change.
%Lastly, to understand users' relationship with these buttons, we additionally ran a survey (N=300) asking users whether they were aware of and used them, as well as how effective they found them to be. 
%Our key finding here is we estimate that 44\% of adult YouTube users in the US are not aware that the ``Not interested'' button exists.
%Overall, we shed light on the need for YouTube to better publicize effective user controls.