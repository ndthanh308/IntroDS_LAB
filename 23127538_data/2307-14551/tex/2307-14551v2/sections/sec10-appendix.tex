\appendix

% \section*{Appendix}
% ====================================================

\section{Annotation Codebook}
\label{app:codebook}

\subsection{Alt-Right}

\subsubsection{Process}
\begin{itemize}
    \item ``Carefully inspect each one of the channels in this table, taking a look at the most popular videos, and watching, altogether, at least 5 minutes of content from that channel.'' -- \citet{ribeiro_auditing_2020}
    \item Search around for former connections to Alt-Right, e.g., Google them and then cross-reference with \citet{marantz_alt-right_2017}.
    \item Can also check existing lists of alt-right channels \cite{chen_exposure_2022,ribeiro_auditing_2020} and online forum.\footnote{\url{https://www.reddit.com/r/Fuckthealtright/}}
\end{itemize}

\subsubsection{Distinguishing the Alt-Right from within the Alternative Influence Network (AIN)}
\begin{itemize}
    \item Read \citet{noauthor_alt_2019} and \citet{marantz_alt-right_2017}, and \cite{weiss_opinion_2018}.
    \item The Alt-Right is the most extreme of three groups of the AIN \cite{lewis_alternative_2018}. The other two groups are the Alt-Lite (less extreme) and Intellectual Dark Web (least extreme).
    \item To distinguish between the Alt-Right and Alt-Lite, note that they are similar in their hatred of feminists, immigrants, social justice warriors, Jewish people, and others, but at the same time the Alt-Lite purport to reject white supremacist thinking that the Alt-Right espouses~\cite{marantz_alt-right_2017}.
    %\item Alt-lite "rejects" explicit white supremacy: "Today, the alt lite, sometimes referred to as the New Right, is loosely-connected movement whose adherents generally shun white supremacist thinking, but who are in step with the alt right in their hatred of feminists and immigrants, among others." \cite{marantz_alt-right_2017}. %Proud Boys are an example of Alt-lite. 
    %\item The main difference is that both sides hate the same things (feminism, SJW, Jewish people, immigrants), but only the Alt-right pair their espoused enemies with an explicit belief that whites are superior to others \cite{noauthor_alt_2019}.
    %\item Here is an example of an Alt-right white-supremicist belief: "Richard Spencer, the keynote speaker and leader of the alt right, and Nathan Damigo, the founder of Identity Evropa, did not focus on Jews, but both talked about preserving white identity in America. Damigo said that America was founded by white people for white people and was not founded to be a multiracial or multicultural society." \cite{noauthor_alt_2019}
    %\item "It is important to stress the difference between civic nationalism and racial nationalism… Please consider the Alt-right label only to the most extreme content. You are encouraged to search on the internet for the name of the content creator to help you make your decision." \cite{ribeiro_auditing_2020}
    %\item One specific example of the divide between Alt-right and Alt-lite is the rally in 2017 at the White House \cite{noauthor_alt_2019}. 

\end{itemize}


\subsection{Antitheism}

\subsubsection{Process}
\begin{itemize}
    \item Watch intro video (if exists) and several other videos
    \item Look for anti-religion videos that could indicate potential antitheism
    \item Look for discussion of the channel/personality's own ideology -- do they talk about being atheist?
\end{itemize}

\subsubsection{Definition}
\begin{itemize}
    \item ``The (a) self-identified atheist who is also (b) actively critical of religion. Also called New Atheists or Street Epistemologists." -- \citet{ledwich_algorithmic_2019}
    \item The keyword here is ``self-identified". They should state their own religious beliefs somewhere in their videos. You can do some web searching to confirm.
    \item Some channels are scientific explanations of phenomena that may be traditionally explained by religion (for instance, explaining ancient floods with scientific reasoning). These videos should not be evidence for antitheism.
    \item Some channels are anti-religion, but only for certain religions (e.g., ex-Jehova's Witness members). However, if they do not talk about their own religious beliefs as being atheist, then it should not count as antitheism.
\end{itemize}



\subsection{Politics}

There are two types of requirements to satisfy here: (1) the format of the content and (2) the content itself. This codebook applies to both the politics-left and right topics.

\subsubsection{Decide whether the channel is ``political''}
\begin{itemize}
    \item Must cover US news and/or some sort of current events
    \item Must be in English
    \item Channels reposting clips of others can count as political
    \item Can be satire as long as the target’s political identity is clear and consistent
    \item If a channel only talks about religious or biblical content, then it is not political
\end{itemize}

\subsubsection{If the channel is political, then decide its political leaning}
\begin{itemize}
    \item Use the MBFC definition of left and right-leaning\footnote{\url{https://mediabiasfactcheck.com/left-vs-right-bias-how-we-rate-the-bias-of-media-sources/}}
    \item Can check both MBFC and All-Sides for leaning labels
\end{itemize}

\subsubsection{Special case - local news outlets}
\begin{itemize}
    \item First, check MBFC and All-Sides for label
    \item If that station does not have a label, then use that of the parent company
\end{itemize}

\subsubsection{Special case - late night shows}
\begin{itemize}
    \item If a show primarily makes fun of politicians, regardless of their affiliated political parties, then it is not political
    \item If a show consistently demotes politicians from one side, or consistently promotes issues fitted into one side’s agenda, it may be political-left or political-right
\end{itemize}

% \section{Screenshots of relevant pages and features on the YouTube platform} 

% To provide visual clarity to our sock puppet actions, as well as the buttons that we asked about during the survey, we provide screenshots of relevant pages and features that they interact with on YouTube.

% On the homepage (Figure \ref{fig:screenshot_homepage}), recommendations are presented to the user as "title cards" consisting of a picture, title, and publishing channel (A). Recommendations are ranked, naturally, in a left-to-right, up-to-down order. The first 10 of these are collected for analysis. Additionally, this ordering becomes relevant when a bot with a recommendation-based strategy encounters multiple videos published by channels in our channel list, and must select the first one (i.e. highest-ranked). A user can explore further options on the platform by pressing the three-dots button (B), where the ``Not interested" and ``Don't recommend channel" buttons can be found (C).

% The videopage (Figure \ref{fig:screenshot_videopage}) is the page that loads when a user watches a given video (A). Recommendations are given in the right-hand bar (B). The first of such recommendations is often referred to as the ``up-next" recommendation \cite{tomlein_audit_2021}, because it is the next video that automatically plays next if a user does not click on any videos. The feature on this page that is relevant to our study is the dislike button (C), which is clicked on for both the \emph{Dislike} and \emph{Dislike recommendation} strategies, and asked about during the survey.

% Lastly, the watch history page (Figure \ref{fig:screenshot_watch-history}) is the page that displays a user's watch history (A). The videos in the history are displayed by recency (by default), with the most recently-watched video being the top video displayed. Bots of with the \emph{Delete} strategy load this page and delete the most recent video by pressing the X button (B). We also ask survey participants about this button.

% % Figure environment removed

% % Figure environment removed

% % Figure environment removed
