\section{Ethical Considerations}

We now discuss the ethical concerns of our study. For our sock puppet audits, since they are computer scripted, we do not run the risk of making real users watch potentially harmful content, such as those from the \emph{Alt-Right} channels. However, making the bots watch a lot of content from a given topic may still increase its prevalence on YouTube by boosting its general popularity, making real platform users see it more often than they would have had our bots not artificially promoted it. Also, pressing the ``Dislike" button on channels may cause them to be demoted in recommendations, limiting YouTube creators' ability to generate advertising revenue. 

While this is a possibility, we do not find these costs to outweigh the benefits of our study. 
\sw{
%\marginnote{SPC-2}
First, we consider the potential cost to content creators of negative interactions with the system, such as pressing the ``Dislike'', ``Not interested'', ``Don't recommend channel'', and ``Delete from watch history'' buttons. 
Here we note that (a) our bots collectively injected up to 3 such interactions per video, which we expect is small compared to the number of ``authentic" ones, (b) we cleared all the revertible actions caused by the audit when it exited its experimental runs, and (c) the average lifetime of an audit in this study is less than six hours, including both the stain phase and scrub phase. This means that not only is the effect of negative interactions small per video, it is also both short-lived and fully reversed.

Another potential cost of our study is that our audits would irreversibly alter two public metrics -- total view count and total watch time. However, the costs are small because we do not expect to affect videos' and/or channels' overall prevalence by much because the number of views we are ``artificially" introducing to the YouTube world are minuscule in relation to the amount of ``authentic" views that the videos have received. 

Our study's findings are a benefit to all YouTube users alike, because they can inform users on how best to deal with and get rid of unwanted recommendations. We think these benefits outweigh the minimal harms.
}

As for the survey participants, we must make sure that they are not being put in the way of any harm. Because we did not ask them to view any actual content, but rather recall times in their life in which they'd interacted with YouTube, we are only at the risk of having participants revisit potentially-triggering or traumatic events if they have had any. However, we include at our introduction of the survey a description of the survey, which describes the questions we'd like to ask them. Thus, if a participant were truly going to be adversely harmed by potentially triggering questions, they would've not consented to the survey and  therefore been removed from the panel before they even saw the questions.
