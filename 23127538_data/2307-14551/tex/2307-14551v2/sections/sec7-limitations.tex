\section{Limitations}

Our findings add to a growing chorus of studies investigating problematic and unwanted recommendations on YouTube. However, our study is not without its limitations. First, we perform channel-level, rather than video-level labeling. Performing labeling at this level may result in labeling channels as a certain topic even if not all of its videos are of that topic (e.g. an \emph{Alt-Right} channel sometimes posting music videos), or, conversely, labeling a channel as off-topic even if just a few of its videos are topical (e.g. a science vlogger who occasionally talks about their journey to atheism). However, we are encouraged by the fact that other studies have taken this approach \cite{tomlein_audit_2021,chen_exposure_2022}. Analyzing channels as a whole is still important because they indicate a high number of videos of that topic, and users may be encouraged to subscribe to these channels even if not all videos in the channel are topical.

Another limitation is that of generalizing from the particular settings our sock puppets used. In particular, our sock puppets tested just four topics, using geolocation of the US East AWS center in August of 2021. While we found that certain strategies work in these conditions to remove recommendations, we cannot be sure it does in other conditions. New topics may be harder or easier to scrub due to more or less general interest in that topic in the broader YouTube ecosystem, respectively. Our code is publicly available \footnote{\url{https://github.com/avliu-um/youtube-disinterest}} so that we and other researchers may continue to understand more general effects of our scrubbing strategies. 

%Also, because we do not perform manual labeling in the staining phase analysis (Subsection \ref{results_stain}), our estimates for topical prevalence in that section are underestimated. Alternative Influence Network analysis is also underestimated in Subsection \ref{results_scrub} for the same reason. As we saw in the scrubbing analysis, there are plenty of false negatives. Thus, while there were enough videos at the end of the staining phase for us to pursue the scrubbing section (i.e. we confirmed there was indeed some stain we could scrub), we cannot make accurate claims about the upper limit on the prevalence of topics.

% If we don't include scrubbing strategy characterizing, then we need to discuss it here as a limitation