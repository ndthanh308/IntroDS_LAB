%----------------------------------------------------------------------------------------
% PACKAGES
%----------------------------------------------------------------------------------------

%%%	 define own colours, here mainly for hyperlinks and notes
\usepackage[dvipsnames,hyperref]{xcolor}
			\definecolor{dark-gray}{gray}{0.1}
			\definecolor{dark-blue}{RGB}{0,0,80}
			\definecolor{light-gray}{gray}{0.9}


%%%%% Language related stuff
\usepackage[english]{babel}
\usepackage[autostyle]{csquotes}			% language dependent citation styles, Loading recommended due to polyglossia
			\MakeOuterQuote{"}

%%%%% Math related
\usepackage{amsmath} 					% AMS = American Math Society
			\delimitershortfall=-1pt
\usepackage{mathtools}				% Fixes some amsmath quirks and adds some useful settings
			\mathtoolsset{showonlyrefs,mathic}
\usepackage{amssymb}					% Math symbols and fonts such as \mathbb, \mathfrak
\usepackage{aliascnt}					% Allows to have one parent counter and different \autoref names for the different structures below
\usepackage{amsthm} 					% Theorem Formatting
			\theoremstyle{definition}
				% Definition. Parent Counter
				\newtheorem{definition}{Definition}[section]
				\providecommand*{\definitionautorefname}{Definition}
				\newtheorem*{definition*}{Definition}
			\theoremstyle{plain}
				% Theorem. Descendant Counter & Unnumbered
				\newaliascnt{theorem}{definition} 								% provides alias counter: theorem=definition
				\newtheorem{theorem}[theorem]{Theorem}						% creates the theorem-environment. Parent counter is theorem
				\aliascntresetthe{theorem} 												% resets the alias counter?
				\providecommand*{\theoremautorefname}{Theorem}		% provides the name for \autoref
				\newtheorem*{theorem*}{Theorem}										% unnumbered theorems
				% Proposition. Descendant Counter
				\newaliascnt{proposition}{definition}
				\newtheorem{proposition}[proposition]{Proposition}
				\aliascntresetthe{proposition}
				\providecommand*{\propositionautorefname}{Proposition}
				% Corollary. Descendant Counter
				\newaliascnt{corollary}{definition}
				\newtheorem{corollary}[corollary]{Corollary}
				\aliascntresetthe{corollary}
				\providecommand*{\corollaryautorefname}{Corollary}
				\newtheorem*{corollary*}{Corollary}
				% Lemma. Descendant Counter
				\newaliascnt{lemma}{definition}
				\newtheorem{lemma}[lemma]{Lemma}
				\aliascntresetthe{lemma}
				\providecommand*{\lemmaautorefname}{Lemma}
				\newtheorem*{lemma*}{Lemma}
				% Big Theorem. Own Counter
				\newtheorem{bigtheorem}{Theorem}
				\renewcommand{\thebigtheorem}{\Alph{bigtheorem}}
				\providecommand*{\bigtheoremautorefname}{Theorem}
			\theoremstyle{remark}
				% Remarks and Examples. No numbering
				\newtheorem*{remark}{Remark}
				\newtheorem*{remarks}{Remarks}
				\newtheorem*{example}{Example}
				\newtheorem*{examples}{Examples}

\usepackage{array}						% adjustments to array, matrix and tabular environment
\usepackage{multicol}
\usepackage[inline]{enumitem}					% generalise the appearance of enumerated lists: \begin{enumerate}[(A)] ... \end{enumerate}
			\renewcommand{\labelenumi}{(\roman{enumi})}	 % Use lower case roman numbers by default (i), (ii), (iii)




%%%%% Literaturverzeichnis; nutzt biblatex mit backend=biber
\usepackage[
	backend=biber,
	bibstyle=authoryear,
	citestyle=alphabetic,
	hyperref,
	sorting=ynt,
	giveninits=true,
	maxbibnames=99,
	backref=false,
	doi=false
	]{biblatex}
	\addbibresource{literature.bib}
			% Do not include In: infront of Journal
			\renewbibmacro{in:}{%
				\ifentrytype{article}{}{\printtext{\bibstring{in}\intitlepunct}}
			}
			% If using backreference, shorten notation
			\DefineBibliographyStrings{english}{backrefpage = {see p.}, backrefpages = {see pp.}}
			% Adjust bibmacro doi+eprint+url to only print urls if DOI and ArXiv identifier are missing
			\renewbibmacro*{doi+eprint+url}{
				\iftoggle{bbx:doi}{\printfield{doi}}{}
				\iftoggle{bbx:eprint}{\usebibmacro{eprint}}{}
				\iftoggle{bbx:url}{
					\iffieldundef{doi}{
						\iffieldundef{eprint}{
							\usebibmacro{url+urldate}
						}{}
					}{}
				}
			}
			% Do not include ISSN for articles.
			\AtEveryBibitem{\ifentrytype{article}{\clearfield{issn}}{}}
			% Need to change alphabetic.bbx in order to include the shorthands in the bibliography.
			% begin: excerpt from `alphabetic.bbx', changed for my needs
			\DeclareFieldFormat{labelalphawidth}{\mkbibbrackets{#1}}

			\defbibenvironment{bibliography}
				{\list
					 {\printtext[labelalphawidth]{%
							\printfield{labelprefix}%
					\printfield{labelalpha}%
							\printfield{extraalpha}}}
					 {\setlength{\labelwidth}{\labelalphawidth}%
						\setlength{\leftmargin}{\labelwidth}%
						\setlength{\labelsep}{\biblabelsep}%
						\addtolength{\leftmargin}{\labelsep}%
						\setlength{\itemsep}{\bibitemsep}%
						\setlength{\parsep}{\bibparsep}}%
						\renewcommand*{\makelabel}[1]{##1\hss}}
				{\endlist}
				{\item}
			 %end: excerpt from `alphabetic.bbx'


%%%%% Hyperlinks; \href, \url, \hyperref
\usepackage[
	pdfusetitle,
	colorlinks=true, pdfborder={0 0 0},
	citecolor={dark-blue}, urlcolor={dark-blue},	linkcolor={dark-blue}
	]{hyperref}
			% let autoref refer to all (sub)subsections as section
			%\let\subsectionautorefname\sectionautorefname
			%\let\subsubsectionautorefname\sectionautorefname
			%\def\equationautorefname~#1\null{eq.~(#1)\null}
\usepackage[numbered, open, openlevel=0]{bookmark}





%----------------------------------------------------------------------------------------
% SHORTHAND MAKROS
%----------------------------------------------------------------------------------------

% convenient stuff
\newcommand{\thalf}{\tfrac{1}{2}}

%%%% Mathematical notation
\newcommand{\del}{\partial}
\DeclareMathOperator{\Tr}{Tr}
\DeclareMathOperator{\tr}{tr}

\newcommand{\after}{\circ}

\newcommand{\abs}[1]{\left| #1 \right|}
\newcommand{\norm}[1]{\left\lVert #1 \right\rVert}

\newcommand{\set}[1]{\left\{\ #1 \ \right\} } 							% set notation. puts a set of objects into curly bracket
\newcommand{\suchthat}{\mathrel{}\middle|\mathrel{}}		% set notation. automatically scales with brackets if used

\newcommand{\spans}[1]{\operatorname{span}\left\{ #1 \right\}}		% generating set = span of a basis. puts a set of objects into angle brackets

\newcommand{\hodge}{\star}
\DeclareMathOperator{\vol}{vol}
\DeclareMathOperator{\Ric}{Ric}

% restriction sign f|_A  via \restrict{f}{A} (\kern-\nulldelimiterspace makes the bar stand closer to the argument)
\newcommand\Restrict[2]{{\left. \kern-\nulldelimiterspace #1 \right\vert_{#2} }}								 % scaling version
\newcommand\restrict[1]{\raisebox{-.2ex}{\ensuremath \vert}_{#1}}  % non-scaling version, e.g. in textmode and without large brackets
\newcommand\restr{\raisebox{-.2ex}{\ensuremath \vert}}  					 % non-scaling version without subscript

% double stroke notation for fields
\newcommand{\N}{\mathbb{N}}			% natural numbers N
\newcommand{\Z}{\mathbb{Z}} 		% integers Z
\newcommand{\Q}{\mathbb{Q}} 		% rational numbers Q
\newcommand{\R}{\mathbb{R}} 		% real numbers R
\newcommand{\C}{\mathbb{C}}			% complex numbers C

% Lie groups and algebras notational stuff
%\newcommand{\su}{\mathfrak{su}}
%\renewcommand{\sl}{\mathfrak{sl}}
\DeclareMathOperator{\ad}{ad}
\DeclareMathOperator{\Ad}{Ad}
\DeclareMathOperator{\Hom}{Hom}


% Differential Operators
\newcommand{\HW}[1][]{{\mathop{\mathbf{HW}}{\hspace{-0.2em}}_{#1\, \/}}}
\newcommand{\KW}[1][]{{\mathop{\mathbf{KW}}{\hspace{-0.2em}}_{#1\, \/}}}
\newcommand{\VW}[1][]{{\mathop{\mathbf{VW}}{\hspace{-0.2em}}_{#1\, \/}}}
\newcommand{\dHW}[1][]{{\mathop{\mathbf{dHW}}{\hspace{-0.2em}}_{#1\, \/}}}
\newcommand{\dKW}{{\mathop{\mathbf{dKW}}}}
\newcommand{\EBE}[1][]{{\mathop{\mathbf{EBE}}{}_{#1\, \/}}}
\newcommand{\TEBE}[1][]{{\mathop{\mathbf{TEBE}}{}_{#1\, \/}}}
\newcommand{\Nahm}[1][]{{\mathop{\mathbf{Nahm}}{}_{#1\, \/}}}


\AtBeginDocument{%
\let\Re\relax
\DeclareMathOperator{\Re}{Re}
\let\Im\relax
\DeclareMathOperator{\Im}{Im}
\let\div\relax
\DeclareMathOperator{\div}{div}
}
