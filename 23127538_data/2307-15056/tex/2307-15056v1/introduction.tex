\section{Introduction}
Let $(M^5, g)$ be a five-manifold with poly-cylindrical ends, where ends may be located at either finite or infinite geodesic distance.
Assume $M^5$ admits a non-vanishing unit vector field $v\/$ and that the subbundle $\ker g(v,\cdot) \subset TM^5 $ admits an almost Hermitian structure.
This means that there is an almost complex structure $J:\ker g(v,\cdot) \to \ker g(v,\cdot)$ that is compatible with the metric, i.e. $g(J\cdot, J\cdot) = g(\cdot, \cdot)$.

In this article we investigate the Haydys-Witten equations on manifolds $(M^5, g, v, J)$.
The existence of $J$ provides a specialization of the equations that we call \emph{decoupled} Haydys-Witten equations.
Crucially, the decoupled version of the equations exhibits a Hermitian Yang-Mills structure, which provides additional tools in solving the equations.
This structure becomes most apparent in the $4\mathcal{D}$-formulation of the Haydys-Witten equations, an extension of Witten's $3\mathcal{D}$-formulation of the extended Bogomolny equations (EBE)~\cite{Witten2011}, which is introduced below and used throughout the introduction.
The main contribution of this article consists in working out conditions under which the Haydys-Witten equations reduce to the decoupled version.

Curiously, in the context of Witten's gauge theoretic approach to homological knot invariants~\cite{Witten2011}, manifolds are generally equipped with the additional structure $(g,v,J)$.
In that situation one considers the Haydys-Witten equations on five-manifolds of the form $M^5= \R_s \times X^3 \times \R_y^+$, equipped with a product metric $g$, and sets $v = \del_y$.
The subbundle $\ker g(v,\cdot)$ is then simply the tangent space of $\R_s \times X^3$, which always admits an almost complex structure, as it is an open and orientable four-manifold.
Conjecturally, when $X^3=\R^3$ or $S^3$ and in the presence of a magnetically charged knot $K \subset X^3$, the homology groups obtained from $\theta$-Kapustin-Witten solutions (stationary Haydys-Witten solutions) at $s\to\pm \infty$ modulo Haydys-Witten instantons reproduces a knot invariant known as Khovanov homology.
Hence, the results presented here may offer a fresh perspective on the gauge theoretic approach to Khovanov homology.

In the following we briefly introduce the $4\mathcal{D}$-formulation of the Haydys-Witten equations and the decoupled version of the equations, see \autoref{sec:vanishings-weitzenböck} for a more detailed, global discussion.
For this, let $G$ denote a compact Lie group, $G_{\mathbb{C}}$ its complexification, $E\to M^5$ a $G$-principal bundle, and $E_{\mathbb{C}}$ the associated $G_{\mathbb{C}}$-principal bundle.
Furthermore, let $\mathcal{A}(E)$ denote gauge connections and write $\Omega^2_{v,+}(M^5)$ for Haydys' self-dual two-forms with respect to $v$~\cite{Haydys2010}.

Given a pair $(A,B) \in \mathcal{A}(E)\times \Omega^2_{v,+}(M^5,\ad E)$ and an almost complex structure $J$, one can locally define four differential operators $\mathcal{D}_\mu$ that act on sections of $\ad E_{\mathbb{C}}$.
To that end, consider normal coordinates $(x^i,y)_{i=0,1,2,3}$ near a point $p$, chosen in such a way that $v = \del_y$ and that $J$ takes the canonical form with respect to the coordinate vector fields $\del_i$ at $p$.
In these coordinates, $B = \sum_{a,b,c = 1}^3 \phi_a (dx^0\wedge dx^a + \thalf \epsilon_{abc} dx^b\wedge dx^c)$.
The four differential operators are defined by
\begin{align}
	\mathcal{D}_0 &= \nabla^A_0 + i\nabla^A_1 &
	\mathcal{D}_1 &= \nabla^A_2 + i\nabla^A_3 \\
	\mathcal{D}_2 &= \nabla^A_y - i [\phi_1, \cdot] &
	\mathcal{D}_3 &= [\phi_2, \cdot] - i [\phi_3,\cdot]
\end{align}
There is a complex conjugation, induced from $\ad E_{\mathbb{C}}$, that we denote by $\overline{\mathcal{D}_\mu}$.
Furthermore, $G_{\mathbb{C}}$-valued gauge transformations act on the operators by conjugation, i.e. $\mathcal{D}_\mu \mapsto g \mathcal{D}_\mu g^{-1}$.

In this formulation, the Haydys-Witten equations $\HW[v](A,B) = 0$ are given by
\begin{align}
	[\overline{\mathcal{D}_0} , \overline{\mathcal{D}_i} ] - \thalf \varepsilon_{ijk} [\mathcal{D}_j, \mathcal{D}_k] &= 0 \ , \qquad i = 1,\ldots,3\ , \label{eq:vanishings-HW-4Ds-gauged-Nahm}\\
	\sum_{\mu = 0}^3 [\mathcal{D}_\mu, \overline{\mathcal{D}_\mu}] &= 0\ . \label{eq:vanishings-HW-4Ds-moment-map-condition}
\end{align}
A typical approach in solving equations of this type is to utilize their symmetry properties.
Since there is an action by complex gauge transformation, one natural idea is to use a Donaldson-Uhlenbeck-Yau type approach, where one first extracts some underlying holomorphic data from $G_{\C}$-invariant parts of the equations, and subsequently hopes to find a complex gauge transformation that ensures also the remaining equations are satisfied.

Unfortunately, although the Haydys-Witten equations are invariant under $G$-valued gauge transformations and the action lifts naturally to $G_{\C}$, neither~\eqref{eq:vanishings-HW-4Ds-gauged-Nahm} nor~\eqref{eq:vanishings-HW-4Ds-moment-map-condition} are invariant under $G_{\C}$-valued gauge transformations.
There is, however, a subset of solutions for which the three equations in~\eqref{eq:vanishings-HW-4Ds-gauged-Nahm} decompose into their $G_{\C}$-invariant parts.
This is given by solutions that satisfy the following equations:
\begin{align}\label{eq:vanishings-dHW-4D}
\begin{split}
	[ \mathcal{D}_\mu, \mathcal{D}_\nu ] &= 0 \ , \qquad \mu,\nu = 0, \ldots, 3\ , \\
	\sum_{\mu = 0}^3 [\mathcal{D}_\mu, \overline{\mathcal{D}_\mu}] &= 0\ .
\end{split}
\end{align}
We refer to~\eqref{eq:vanishings-dHW-4D} as \emph{decoupled Haydys-Witten equations} and denote them by ${\dHW[v, J](A,B)=0}$.
A global version of these equations is provided in \autoref{sec:vanishings-weitzenböck}.
The commutativity equations are $G_{\mathbb{C}}$-invariant and can be interpreted as a complex moment map condition in a hyper-Kähler reduction, while the remaining equation provides the real moment map condition.
Put differently, the decoupled equations exhibit a Hermitian Yang-Mills structure, such that a Donaldson-Uhlenbeck-Yau type approach and other powerful tools become available.

Clearly, whenever $\dHW[v,J](A,B) = 0$, then $\HW[v](A,B) = 0$.
Here, we prove that in certain situations the reverse is true, such that the Haydys-Witten equations reduce to the decoupled Haydys-Witten equations.
This is controlled, on the one hand by the (asymptotic) geometry, and on the other hand by the asymptotic behaviour of the fields $(A,B)$, at boundaries and cylindrical ends of $M^5$.
Crucially, we need to assume that
\begin{enumerate*}[label=(\alph*)]
\item boundaries of $M^5$ are flat and $(A,B)$ satisfies Nahm pole boundary conditions with knot singularities, and
\item non-compact ends of $M^5$ are asymptotically locally Euclidean (ALE) or flat (ALF) gravitational instantons and $(A,B)$ approaches a finite energy solution of the $\theta$-Kapustin-Witten equations.
\end{enumerate*}
For the sake of brevity, we omit further technical conditions for now and instead refer to assumptions (A1)~-~(A4) in \autoref{sec:vanishings-proof-main-result} for details.
With that understood, our main result is as follows (this is \autoref{thm:vanishings-decoupling}).
\begin{bigtheorem}\label{bigthm:decoupling}
Let $G=SU(2)$, $M^5$ a manifold with poly-cylindrical ends, $v$ a non-vanishing vector field that approaches ends at a constant angle, and $J$ an almost Hermitian structure on $\ker g(v,\cdot)$.
Assume $\HW[v](A,B)=0$ and that assumptions \textnormal{(A1)}~-~\textnormal{(A4)} are satisfied, then $\dHW[v,J](A,B)=0$.
\end{bigtheorem}

The proof of \autoref{bigthm:decoupling} is based on a Weitzenböck formula of the form
\begin{align} \label{eq:vanishings-intro-weitzenböck}
	\int_{M^5} \norm{\HW[v](A,B)}^2 = \int_{M^5} \norm{\dHW[v,J](A,B)}^2 + \int_{M^5} d \chi
\end{align}
From this it's clear that whenever $\HW[v](A,B) = 0$ and any boundary contributions (including contributions from non-compact ends) in $\int_{M^5} d\chi$ vanish, then also $\dHW[v,J](A,B) = 0$.
The key insights of this article lie in determining conditions under which all boundary contributions vanish, if one imposes Nahm pole boundary conditions at finite distances and asymptotically stationary solutions at infinity.
Due to the two different flavours of boundary conditions, we rely on two distinct facts:
Elliptic regularity of the Nahm pole boundary conditions, on the one hand, and a vanishing theorem for solutions of $\theta$-Kapustin-Witten solutions on ALE and ALF spaces, on the other.
Let us shortly explain how these two facts appear in the proof.

First, the Nahm pole boundary conditions for $(A,B)$ state, in particular, that at order $y^{-1}$ the fields satisfy the extended Bogomolny equations (EBE), which in the $4\mathcal{D}$ formalism correspond to
\begin{align}
	\mathcal{D}_0 &= 0\ , &
	[ \mathcal{D}_i, \mathcal{D}_j ] &= 0 \qquad i,j = 1, \ldots, 3\ , &
	\sum_{i = 1}^3 [\mathcal{D}_i, \overline{\mathcal{D}_i}] &= 0\ .	
\end{align}
This means that the leading order terms are already solutions of the decoupled Haydys-Witten equations.
As will be discussed in detail below, elliptic regularity of the Haydys-Witten equations states that deviations from the EBE-solutions can only appear at order $y^{1+\delta}$, for some $\delta>0$~\cite{Mazzeo2013a, Mazzeo2017, He2018a}.
This, in turn, implies that $\chi$ only involves terms of order $y^\delta$.
As a consequence, any contributions to $\int d \chi$ from boundaries with Nahm pole boundary conditions vanish.

We expect that \autoref{bigthm:decoupling} is also generally true in the presence of knot singularities.
Unfortunately, the twisted knot singularity solutions are only known implicitly (by a continuation argument)~\cite{Dimakis2022}, such that extracting information from elliptic regularity is difficult.
Although we currently have no proof for this extension of the result, we include a discussion of the relevant boundary conditions and state a necessary condition that is known to be satisfied in the untwisted case.

Second, and perhaps more surprisingly, the boundary terms also vanish at asymptotic ends when the fields approach stationary solutions of the Haydys-Witten equations, or equivalently solutions of the $\theta$-Kapustin-Witten equations, with finite energy.
This relies on a vanishing theorem that was originally conjectured by Nagy and Oliveira~\cite{Nagy2021} and for which a proof appeared recently in~\cite{Bleher2023a}.
The vanishing theorem states that for a finite energy solution of the $\theta$-Kapustin-Witten equations on an ALE or ALF gravitational instanton $A$ is flat, $\phi$ is $\nabla^A$-parallel, and $[\phi \wedge \phi] = 0$.

This article is structured as follows.
In \autoref{sec:vanishings-weitzenböck} we summarize Haydys' geometry and the Haydys-Witten equations, define the decoupled Haydys-Witten equations, and establish the promised Weitzenböck formula.
In \autoref{sec:vanishings-setting} we further specify the basic geometric setting that is necessary to specify boundary conditions and which is used in evaluating the integral of the exact term in the Weitzenböck formula.
A key step in this article is an investigation of the polyhomogeneous expansion of a twisted Nahm pole solution of the Haydys-Witten equations, which is presented in \autoref{sec:vanishings-polyhomogeneous-expansion-of-nahm-pole-solutions}.
Subsequently, in \autoref{sec:vanishings-asymptotics-of-chi}, we determine the asymptotic behaviour of $\chi$ at the various boundaries and ends, filling in the details of the remaining boundary conditions as we go.
Finally, in \autoref{sec:vanishings-proof-main-result}, we bring everything together and show that in certain situations the boundary term in~\eqref{eq:vanishings-intro-weitzenböck} vanishes, which immediately implies \autoref{bigthm:decoupling}.