\paragraph{Abstract.}
The Haydys-Witten equations are partial differential equations on five-dimensional Riemannian manifolds that are equipped with a non-vanishing vector field $v$.
Conjecturally, their solutions determine the Floer differential in a gauge-theoretic approach to Khovanov homology.
This article introduces a certain decoupled version of the Haydys-Witten equations, a specialization of the Haydys-Witten equations that exhibits a Hermitian Yang-Mills structure.
These equations exist whenever the vector bundle defined by the orthogonal complement of $v$ admits an almost Hermitian structure.
We investigate the relation between the full Haydys-Witten equations and their decoupled version on manifolds with poly-cylindrical ends and boundaries, and find conditions under which the Haydys-Witten equations reduce to the decoupled equations.
This relies on a Weitzenböck-like formula that shows that the difference between the full Haydys-Witten equations and the decoupled equations is governed by the asymptotic behaviour of solutions near boundaries and non-compact ends.
Regarding the analysis near boundaries, we provide a detailed analysis of the polyhomogeneous expansion of Haydys-Witten solutions with twisted Nahm pole boundary conditions, generalizing work of Siqi He in the untwisted case. 
The corresponding analysis at non-compact ends relies on a vanishing theorem by Nagy-Oliveira that was recently generalized by the author.