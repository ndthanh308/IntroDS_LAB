\documentclass[12pt,a4paper]{article}
%\documentclass[tikz,letterpaper,12pt]{article}
%\documentclass[tikz,border=10pt]{standalone}
\pdfoutput=1
\usepackage{amsmath,amssymb,exscale}
\usepackage{amsfonts,amsthm}
\usepackage{graphicx, caption, float, epsfig}
\usepackage{mathrsfs}
\usepackage{slashed}
\usepackage{textcomp}
\usepackage[T1]{fontenc}
\usepackage[utf8]{inputenc}
\usepackage[table]{xcolor}
\usepackage[lowtilde]{url}
\usepackage{cancel}
\usepackage[font={small}]{caption}
%\usepackage[colorlinks=true,urlcolor=blue,anchorcolor=blue,citecolor=blue,filecolor=blue,linkcolor=blue,menucolor=blue,pagecolor=blue,linktocpage=true]{hyperref}
\numberwithin{equation}{section}
\usepackage{verbatim} 
\usepackage{appendix}
\usepackage{hyperref}
%\usetikzlibrary{decorations.pathmorphing}
%\tikzset{zigzag/.style={decorate, decoration=zigzag}}

\newcommand{\be}{\begin{equation}}
\newcommand{\ee}{\end{equation}}
\newcommand{\ba}{\begin{eqnarray}}
\newcommand{\ea}{\end{eqnarray}}
\newcommand{\bay}{\begin{array}{rcl}}
\newcommand{\eay}{\end{array}}
\newcommand{\ra}{\rightarrow}
\newcommand{\la}{\leftarrow}
\newcommand{\rh}{r_{\rm H}}
\newcommand{\gn}{G_{\rm N}}
\newcommand{\bra}{\langle}
\newcommand{\ket}{\rangle}

\def\LP{l_{\rm Pl}}
\def\TP{T_{\rm Pl}}
\def\MP{M_{\rm Pl}}
\def\tP{t_{\rm Pl}}

\thispagestyle{empty} \hoffset=-.3in \topmargin -0.15truein
\setlength{\textheight}{8.5in} \setlength{\textwidth}{6in}
\flushbottom

\begin{document}

\title{The fate of supersymmetry in topological quantum field theories}

\author{Risto Raitio\footnote{E-mail: risto.raitio@gmail.com}\\	
02230 Espoo, Finland}

\date{July 23, 2023}  \maketitle 

\abstract{\noindent
We analyze the role of supersymmetry in nature. We extend our previous model of particles and cosmology beyond its critical energy scale at about $10^{16}$ Gev. We assume that there are three main phases in the evolving universe. The first is topological gravity phase, the second a brief Chern-Simons phase, and the third the standard model (SM) gauge phase. In our scenario supersymmetry (SUSY) appears in all phases but in the third phase confined in topological preons, which form quarks and leptons. The confined SUSY (cSUSY) is supported by the lack of observation of squarks and sleptons. cSUSY also provides a natural mechanism for matter-antimatter asymmetry. The possible relationship of this tentative scenario to quantum gravity and the role of UV-completeness are disclosed.}

\vskip 1.5cm
\noindent
 
\vskip 1.5cm

\noindent
\textit{Keywords:} Topological field theory, Supersymmetry, Chern-Simons model, Baryon asymmetry


\newpage

\tableofcontents
\vskip 2.0cm

\section{Introduction}
\label{intro}

String theory has been under active study for about 50 years. The beauty of it has not so far been realized in phenomenological success. In spite of that, stringy features like dualities have been introduced with success in field theoretic model building, together with topological concepts. The alleged UV-completeness of string theory is another motivator for active research at present. 

We use in this note supersymmetry, T-duality, topological models and aim towards UV-completeness. With these properties we extend our previous scenario of the universe beyond the critical scale $\Lambda_{cr} \sim 10^{16-13}$ GeV up to Planck scale. Matter in the present scenario goes through two phase transitions between Planck time universe and the present baryon asymmetric one. The mechanisms of these phase transitions are defined. We propose our model of topological, supersymmetric matter as an attempt to look for the answer to the ontic question.

The article is organized as follows. In section \ref{earlyuniv} we consider some general features, like the three different phases of the universe, the two phase transitions and motivation for preons (called here chernons). To indicate the nature of problem of phase I matter, two models of topological gravity are briefly reviewed in section \ref{topolgrav}, namely those of Witten, and Fang and Gu. Comparison of the present scenario and standard model inflation is made in section \ref{compar}. In section \ref{chernon} transition from topological phase is discussed, based on Chern-Simons (CS) matter creating mass, and metric spacetime, by Higgs mechanism. CS matter finally confines itself into ordinary visible and dark matter. A process for creating baryon asymmetry in the universe is recapped from our previous article in section \ref{barasym}. Conclusions and outlook are given in section \ref{conclusions}, with a philosophical paragraph about the relevance of UV-completion. An appendix with table \ref{tab:table4} of CS particle - SM particle correspondence is provided. - The nature of this note is mainly to collect into single coherent form physical ideas from different articles (including our own).


%*********************************************
\section{Three phases of the universe}
\label{earlyuniv}

The common view is that as we go far enough back in time in the contracting universe we will reach a point, defined here as time t = $\rm{t}_0$, or just t = 0, (see figure \ref{fig:figure1}) where the degrees of freedom that our universe is made of may disappear and get replaced by other light degrees of freedom \cite{Vafa&al}. This kind of idea appears also in \cite{Rai_1, Rai_2} where at energy scales $\Lambda_{cr} > 10^{16}$ GeV new degrees of freedom replace standard model particles, before phase I own objects become dominating. 

% Figure environment removed

We assume in this note that the form of matter we obsrve depends upon the environment, basically on the temperature, or energy scale. E.g. nuclei consist of nucleons in ordinary laboratory conditions. When bombarded with enough energy the nucleons get unbound and free. On the next level quarks are liberated inside nucleons. We go one step further by assuming that quarks, and leptons, consist of preons above $\Lambda_{cr}$, before phase I enters. 

There may be some need to disclose arguments for preons in general. To implement our personal view of supersymmetry, standard model \cite{Rai_1} and baryon asymmetry \cite{Rai_2}, we split quarks and leptons in three pointlike constituents, called in this note chernons (synonym for preon\footnote{The term was coined by Pati and Salam in 1974 \cite{Pati_S}.} or superon). Of the many preon models in the literature there are two of them which are like ours. One of them was a gauge theory proposed by Harari \cite{Harari}, and simultaneously by Shupe \cite{Shupe}. The model of Finkelstein \cite{Finkelstein} was developed from a different basis, namely by the quantum group SLq(2) and knot theory in the form of plane projections of e.g. of trefoils of figure \ref{fig:figure4} where the three outer loops "visualize" preons. This model turned out to agree with the model of Harari and Shupe \cite{Harari, Shupe}. The major difference between the above models and our model \cite{Rai_0, Rai_1} is that ours has its basis in unbroken global supersymmetry where superpartners are in the model initially, not as new sparticles to be found in the future.

The era $t < 0$, or phase I, is a topological phase between, say $\Lambda_{cr}$ and $\LP$. In the T-dual \cite{Sathiapalan} second phase $t > 0$, or $E < \Lambda_{cr}$, there exist the standard model matter, dark matter and dark energy \cite{Vafa&al}. We make a proposal about what may happen in the transition between the two phases I and II. What ends the topological phase I, and what is the destiny of supersymmetry in phase II? 

Briefly, we propose that between the two phases I and II there is a brief interpolating phase O for transfering supersymmetry to phase II, though in confined form, in close analogy with SU(3) color confinement. Our assumption, or rather prediction, is that no SM sparticles, like squaks and sleptons, exist in nature because SUSY is in confined state. At present, there is no experimental evidence for SM sparticles, after a long search.

We start with SUSY in phase I and want keep it towards phase II, where it is a priori not guaranteed to exist. Then a SUSY conserving intermediate process in phase O is needed at $t \sim 0$ when both derivatives of $\rho^I_m$ and $\rho^{II}_m$ are non-zero, see figure \ref{fig:figure1} green area. In this process topological objects (called chernons in section \ref{chernon}) take over supersymmetry of pase I. 

In the next transition, phase O to phase II, chernons form composite states, i.e. quarks and leptons, by a Chern-Simons interaction. SUSY suffers now confinement inside quarks and leptons, as supersymmetric chernons. This resembles QCD color being hidden inside hadrons or plasma cooling down to atomic matter.

Physics in phase II after reheating is well described by a thermal distribution of SM matter (and the dark components). The notion of time is common to both phases of the universe. This leads to energy being common to both phases. In addition there are weak long range correlations that originate from phase I modes that are non-local in phase II. This yields proper initial conditions for Friedmann–Lemaître–Robertson–Walker (FLRW) metric cosmology. The horizon problem is solved simply because the locality, relevant in our universe in phase II, is not natural in phase I. The light modes of phase I are non-local as viewed from phase II. A known example are the winding modes of the string gas cosmology \cite{Brandenberger:2004}. Fluctuations visible in phase II are not part of the degrees of freedom of phase I.

% Figure environment removed

How does phase I look from the perspective of phase II \cite{Vafa&al}? In phase I there should not be any position dependent observables. Let us assume the state in phase I is given by a state $| I\rangle$. We would expect $n$-point correlations of physical observables in this state
%
$$\langle I | {\cal O}^{i_1}(x_1)\ldots {\cal O}^{i_n}(x_n)|I \rangle =A^{i_1,\ldots,i_n}$$
%
to be position independent when all $\partial _j A^{i_1,\ldots,i_n} = 0$. This is a key feature of a topological quantum field theory. We are thus led to view phase I as a topological phase as viewed from the perspective of frame II. It is curious that the reverse is also true:  phase II can be viewed from the perspective of frame I as a topological theory \cite{Vafa&al}. This is illustrated in Figure \ref{fig:figure2}.
 

%**********************************
\section{Topological models in phase I}
\label{topolgrav}

\subsection{BRST formalism}
\label{brst}

In topological field theories observables must be a measure of global features. Consequently, there are no propagating signals. This property is achieved in the Becchi-Rouet-Stora-Tyutin (BRST) \cite{BRS, Tyutin} formalism by the presence of a Grassmann odd charge operator $Q$.

This operator $Q$ is nilpotent, %$Q^​2$ = 0, 
hermitian, and it commutes with the Hamiltonian, $[H,Q] = 0$. The action of the charge operator on fields $\Phi$ is given by 
\be 
\delta\Phi = i \epsilon [Q,\Phi] 
\label{deltaphi}
\ee
where $\epsilon$ is a Grassmann parameter, a supernumber that anticommutes with all other Grassmann variables. $Q$ is also the Noether charge for the BRST symmetry. The action (see (\ref{WWaction})) combines together bosonic and fermionic fields in a way similar to the pairing in supersymmetric theories. Physical states in the Hilbert space are $Q$-cohomology classes: these states are $Q$-closed (i.e. $|\psi\rangle$ satisfying $Q|\psi\rangle=0$) modulo $Q$-exact (i.e. $|\psi\rangle$ such that $|\psi\rangle = Q|\chi\rangle$ for some $|\chi\rangle$). This latter requirement implies that the fermionic partners of bosonic fields are in fact ghosts so that all degrees of freedom cancel in the BRST sense.

If we assume that the vacuum is $Q$-invariant, then $Q$-exact operators have a vanishing expectation value $\langle [ Q,\mathcal{O} ]\rangle = 0$. In topological field theories, the energy-momentum tensor (given by the variation of the action with respect to the metric) is $Q$-exact, i.e. $T_{\alpha\beta} = \{Q, \lambda_{\alpha\beta}\}$ for some $\lambda$. This implies that the partition function is invariant under metric variations
\begin{align*}
	\delta Z &= \int \mathcal{D}\Phi e^{-S}\left(-\delta S\right)
	= - \int \mathcal{D}\Phi e^{-S}\{Q,\int \sqrt{g}\delta g^{\alpha\beta} \lambda_{\alpha\beta}\}\\
	&= -\langle \{Q,\int \sqrt{g}\delta g^{\alpha\beta} \lambda_{\alpha\beta}\} \rangle = 0
\end{align*}
provided the integration measure is BRST invariant.

Another way to illuminate background independence in a topological theory in general is based on calculating Wilson loops in 3D Chern-Simons (CS) theory \cite{Witten_0}.\footnote{CS theory is discussd later in section \ref{chernon}} Wilson loops give a natural class of gauge invariant observables that do not require a choice of metric. Let C be an oriented closed curve in M. Intrinsically C is simply a circle, but the topological classification of embeddings of a circle in M may be complicated, as we can imagine in figure \ref{fig:figure4}. Let R be an irreducible representation of G. One then defines the Wilson loop $W_R(C)$ to be the following functional of the connection $A_i$. One computes the holonomy of $A_i$ around C, getting an element of G that is well-defined up to conjugacy, and then one takes the trace of this element in the representation R. Thus, the definition is 
\be
W_R(C) = \rm{Tr}_R P \exp \int_C A_i dx^i
\ee
The crucial property of this definition is that there is no need to introduce a metric, so general covariance is maintained. 
% Figure environment removed
%In topological gravity, the partition function $Z$ includes an integral over the metric degrees of freedom $g$ (discarding gauge redundancies as usual). 
Consider the partition function $Z$, defined as 
\be
Z = \int \mathcal{D}\mathcal{A}\exp(i\mathcal{L})\prod_i {W_R}_i (C_i)
\label{partitionfcn}
\ee
where $\mathcal{D}\mathcal{A}$ represents Feynman integral over all gauge orbits, the $C_i$ are non-intersecting knots and $R_i$ representation assigned to $C_i$. The partition function Z is thus automatically independent of any background metric. However, there is still a question of whether the theory contains local excitations. 

\subsection{Witten's topological gravity}
\label{wittenth}

Witten's theory \cite{Witten:1988xi} is defined as  follows.\footnote{It can be obtained by applying the Batalin-Vilkovisky (BV) formalism \cite{Batalin:1981jr, Batalin:1984jr, Batalin:1985qj} to the topological action $W\wedge W$ where $W$ is the Weyl tensor~\cite{Brooks:1988jm}.}  
If one ignores the ghosts, then the dynamics of gravity would be governed by a self-dual Weyl action
\begin{equation}
	S_g = \int d^4x \sqrt{g} \frac{1}{2}(W + \star W)^2
	\label{WWaction}
\end{equation}
where $\star$ is the Hodge dual. This action is scale invariant classically but would generally have a conformal anomaly at the quantum level. In addition, conformal symmetry is broken in Witten's topological gravity by a vev of a scalar field (denoted as $\Phi$) that is required for the action to be non-degenerate. Despite the fact that the usual Weyl tensor square gravity has ghosts and is non-unitary, this topological theory is unitary~\cite{Witten:1988xi} as the non-unitary correlations are not allowed observables of the topological theory. We also note that the Einstein-Hilbert term $\mathcal{R}$ is not generated in Witten's topological gravity because it is forbidden by the BRST symmetry. To see this, one would have to review in more detail the field content and the BRST transformations.

\begin{table}[ht]
	\begin{center}
		\captionsetup{width=.8\linewidth}
		\begin{tabular}{|c||c|c|}
			\hline
			$ \text{field} $~ & ght & ~~$[ Q , \text{field} \}$~~ \tabularnewline
			\hline
			\hline
			$ C_{A\dot A} $ & $2$ & $\psi_{AB,\dot A \dot B} C^{B \dot B}$
			\tabularnewline
			\hline
			$ \psi_{AB, \dot A \dot B} $ & $1$ & $- \frac{i}{4} \big( e^{\alpha}_{A \dot A} D_{\alpha} C_{B \dot B} + e^{\alpha}_{B \dot A} D_{\alpha} C_{A \dot B} + e^{\alpha}_{A \dot B} D_{\alpha} C_{B \dot A} + e^{\alpha}_{B \dot B} D_{\alpha} C_{A \dot A} \big)$
			\tabularnewline
			\hline
			$ e_{\alpha A \dot A} $ & $0$ & $e_{\alpha}^{B \dot B} \psi_{AB,\dot A \dot B}$
			\tabularnewline
			\hline
			$ W_{ABCD} $ & $0$ & $\frac{1}{6} ({\psi_{AB,}}^{\dot A \dot B} R_{CD,\dot A \dot B} - e_{C \dot C}^{\alpha} e_{D \dot D}^{\beta} D_{\alpha} D_{\beta} {\psi_{AB,}}^{\dot C \dot D})$
			\tabularnewline
			& & $+~5~\text{permutations of}~A,B,C,D$
			\tabularnewline
			\hline
			$ \chi_{ABCD} $ & $-1$ & $-i W_{ABCD}$
			\tabularnewline
			\hline
			$ \lambda_{A \dot A} $ & $-1$ & $-i C^{\alpha} D_{\alpha} B_{A \dot A} + \text{``}(\psi \psi + e D C) B\text{''} - \frac{i}{4} B_{A \dot A} e^{\beta}_{X \dot X} D_{\beta} C^{X \dot X}$
			\tabularnewline
			\hline
			$ B_{A \dot A} $ & $-2$ & $\lambda_{A \dot A}$
			\tabularnewline
			\hline
		\end{tabular}
		\par\end{center}
	\caption{Field content of Witten's topological gravity \cite{Witten:1988xi}.\\Second column is ghost number.}
	\label{tab:fields}
\end{table}

%In addition to the metric (tetrad), Witten's topological
Witten's topological gravity includes the metric, or tetrad, bosonic fields $C_{A\dot{A}}$ and $B_{A\dot{A}}$ and fermionic fields $\lambda_{A \dot{A}}$, $\psi_{AB\dot{A}\dot{B}}$ and $\chi_{ABCD}$, where the dotted and undotted indices are the $SU(2)_L \times SU(2)_R$ spinor indices in four dimensions. The transformations of these fields are summarized in table~\ref{tab:fields}. They determine the conditions the bosonic backgrounds must satisfy in order to have a BRST-invariant vacuum. As in supersymmetry, these conditions are obtained by requiring that the variations of the fermionic fields vanish. Included in the variations is the condition $\delta \chi_{ABCD} = W_{ABCD} = 0$, which implies that the universe in phase I must be conformally half-flat. %This will be discussed further below. 

Using the variation of the fermionic field $\chi_{ABCD}$ one can show that the Einstein-Hilbert term $\mathcal{R}$ does not appear among the manifestly $Q$-exact terms. 

In order to give the fields a conventional kinetic term it was proposed in ~\cite{Witten:1988xi} that topological gravity is coupled, in addition to the fields discussed above, to topological matter and a topological invariant field $\Phi$ couples to some of the fields in the topological theory whose vev
$\langle \Phi \rangle =v_0^2$ will give rise to the desired kinetic term.  This term breaks scale invariance, but we will be assuming that the vev $v_0$ is sufficiently small, as not to break the scaling symmetry of the topological theory.

Gravitational theories of such a topological nature have an intriguing physical interpretation \cite{Witten:1988ze}. They are believed to be confined phases of gravity where general covariance is unbroken. Once the metric acquires an expectation value (i.e. there is a background spacetime) then this symmetry is spontaneously broken and local gravitational excitations, gravitons, emerge. Here an analogy can again be made to QCD with an unbroken local symmetry and no massless gauge bosons.

Finally, there is the question of observables in topological gravity. As in all topological theories, these would be position independent expectation values of operators in $Q$-cohomology. In addition, the absence of spin-2 excitations implies the absence of tensor modes in cosmological observables.

\subsection{Fang and Gu's topological gravity}
\label{guth}

We consider another topological theory by Fang and Gu  \cite{Gu:2017, Gu:2021}. The TQFT approach can not be easily generalized into 3+1D because consistency with Einstein's gravity in 3+1D contains propagating a mode, the graviton, therefore it is obviously not a case for TQFT in the usual sense. Secondly, there is no Chern-Simons like action in 3+1D. Fang and Gu have shown that Einstein gravity might emerge by adding a topological mass term of the 2-form gauge field. Physically, such a phenomenological theory might describe a loop condensing phase, i.e. flux lines in the context of gauge theory.

Due to the recent developments in the classification of topological phases of quantum matter in higher dimensions \cite{Chenlong,Wenscience,cobordism,Wencoho}, new types of TQFT have been discovered in 3+1D to describe the so-called three-loop-braiding statistics. It is argued that such types of TQFT are closely related to Einstein gravity and that gravitational field will disappear at extremely high energy scale. 3+1D quantum gravity (QG) would be controlled by a TQFT renormalization group fixed point. At intermediate energy scales, Einstein gravity and classical spacetime would emerge via loop (flux lines) condensation of the underlying TQFT.\footnote{The uncondensed loop-like excitation are a natural candidate of dark matter. Such kind of dark matter will not contribute scalar curvature but  will be a direct source of torsion. Normal matter, like Dirac fermions, will not contribute to torsion.}

Let us begin with the topological gravity theory in 3+1D \cite{Topgravity}. Consider the following topological invariant action:
\begin{eqnarray}
	S_{top}&=&\frac{k_{1}}{4\pi }\int \varepsilon
	_{abcd} R^{ab}\wedge e^{c}\wedge e^{d} +\frac{k_{2}}{2\pi }\int B_{ab} \wedge  R^{ab} \nonumber\\ &&+\frac{k_{3}}{2\pi }%
	\int \widetilde{B}_{a} \wedge  T^{a}, \label{action}
\end{eqnarray}
where $e$ is the tetrad field, $R$ is the curvature tensor, $T$ is the torsion tensor and $B,\widetilde{B}$ are 2-form gauge fields. Like in the CS theory, the values of $k_i$ are quantized. Without loss of generality, the following values can be chosen $k_1=k_2=2$ and $k_3=1$ for convenience. The above action is invariant under the following (twisted) 1-form and 2-form gauge transformations, respectively:
\begin{eqnarray}
	e^{a} &\rightarrow &e^{a}+Df^{a} \nonumber\\ 
	B_{ab} &\rightarrow &B_{ab}-\frac{k_{3}}{2k_{2}}\left( \widetilde{B}%
	_{a}f_{b}-\widetilde{B}_{b}f_{a}\right)  \nonumber \\
	\widetilde{B}_{a} &\rightarrow &\widetilde{B}_{a}-\frac{k_{1}}{k_{3}}
	\varepsilon _{abcd}f^{b}R^{cd}, \label{1form}
\end{eqnarray}
and
\begin{eqnarray}
	B_{ab} & \rightarrow & B_{ab}+D\xi _{ab}, \label{2forma}\\
	\widetilde{B}_{a} &\rightarrow &\widetilde{B}_{a}+D \tilde{\xi} _{a} 
	\nonumber\\
	B_{ab} &\rightarrow &B_{ab}-\frac{k_{3}}{2k_{2}}\left( \tilde{\xi}  _{a}\wedge
	e_{b}-\tilde{\xi}  _{b}\wedge e_{a}\right).  \label{2formb}
\end{eqnarray}
Such an action can be regarded as the non-Abelian generalization of $AAdA+BF$ type TQFT \cite{aada1,aada2,aada3} of the Poincare gauge group. Physically, it has been shown that such kind of TQFT describes the three-loop-braiding statistics \cite{loop1,loop2}. As a TQFT, the action Eq. (\ref{action}) is a super-renormalizable theory. The coefficient quantization and canonical quantization of such a theory are discussed in \cite{Gu:2021}. 

SUSY generalization of 3 + 1D topological gravity is discussed in \cite{Gu:2017}. One needs to introduce the gauge connection of super Poincare group and write the action as $\int sTr[ A\wedge A\wedge (dA+A\wedge A)]+\int sTr(B \wedge F)$. For the $N=1$ case, one can express $A$, $B$ and $F$ as follows
\begin{eqnarray}
	&&A_{\mu}\equiv \frac{1}{2}\omega_{\mu}^{ab} M_{ab}+e_\mu^a P_a+\bar{\psi}_{\mu\alpha} Q^\alpha \nonumber\\
	&&B_{\mu\nu}\equiv \frac{1}{2}{B_{\mu\nu}}^{ab} M_{ab}+{\tilde{B}_{\mu\nu}}^a P_a+\mathfrak{B}_{\mu\nu\alpha} Q^\alpha \nonumber\\
	&& F_{\mu\nu}\equiv
	\frac{1}{2}{R_{\mu\nu}}^{ab} M_{ab}+T_{\mu\nu}^a P_a+\bar{R}_{\mu\nu\alpha} Q^\alpha
\end{eqnarray}
Here $\bar{R}_{\mu\nu\alpha}$ is the super curvature tensor defined as $\bar{R}_{\mu\nu\alpha}=D_{\mu} \bar{\psi}_{\nu\alpha}-D_{\nu} \bar{\psi}_{\mu\alpha}$ where $D_{\mu}$ is the covariant derivative for spinor fields. Fermionic loops (flux lines) cannot be condensed. Therefore supersymmetry breaking happens at very high energy scale when bosonic loops condense and classical space-time emerges.

Although the total action S is super-renormalizable, it does not imply a UV-complete quantum gravity theory due to explicit breaking of 2-form gauge symmetries by the $ S_\theta = -\frac{\theta}{2\pi} \int B_{ab}\wedge B^{ab}$ term. The algebraic tensor 2-category theory \cite{TCAT1,TCAT1} may provide an equivalent UV-complete description for a topological quantum gravity theory in 3+1D.


\subsection{Flatness}
\label{flat}

We are assuming no observables of frame I will distinguish positions, so the metric should be homogeneous, i.e. a constant curvature metric. We note that the time direction is picked out as an invariant concept in both phases.  We would like to determine the consequences of this for the geometry in phase I as viewed from the frame II perspective. The most general metric with these symmetries is
%
\begin{equation}
	ds^2=-dt^2+a^2(t)\left[\frac{dr^2}{ (1-k r^2)} +  r^2 d\Omega^2\right]
	\label{FRWansatz}
\end{equation}
%
where $k=+1,0,-1$ for positive, flat or negative curvature spaces.
However, as discussed above, the solutions to BRST \cite{BRS, Tyutin} invariant configurations in 4D topological gravity are conformally flat self-dual geometries, which have
%
\begin{align}
	W_{ABCD}=0.
\end{align}
%
This condition by itself allows all three possibilities above. We will view time as a continuous element between phase I and phase II. Thus, a natural assumption is that the metric can be expressed as a flat metric up to a conformal factor that is only dependent on time, which is the only duality invariant coordinate. This is equivalent to having an FLRW metric \eqref{FRWansatz} with $k=0$
\begin{align}
	ds^2 = a^2(\eta)(-d\eta^2+ dx^i dx^i)
\end{align}
Moreover in phase II, since the metric should smoothly connect, we learn that at the beginning of the FLRW cosmology, the universe is spatially flat, which is proper for phase O. 


%*******************
\section{Topological early phases versus inflation}
\label{compar}

In this section we compare and contrast the three phase topological scenario with the inflationary scenario.  There are a number of common features in the two approaches  as can be seen in Fig.~\ref{fig:figure3}.

The end result for both is the FLRW scenario.  Both of them involve a kind of phase transition.  In the case of inflation the transition is marked by the end of inflation and reheating as the inflaton settles to the minimum of the potential.  In the case of the topological scenario the phase transition takes place by a topology and symmetry change process \cite{Horowitz, Tanaka_N} followed immediately by confinement of the new topological objects, discussed in the next section \ref{chernon}. In both scenarios we have a nearly homogeneous thermal initial condition for FLRW in phase II. In both scenarios the homogeneity of space is described by a novel phenomenon:  in the inflationary scenario by the exponential expansion of the space and in the topological phase by the fact that gravity is described by a topological theory. In the inflationary scenario the fluctuations of the inflaton field leads to scalar fluctuation, whereas in the topological phase which involves only global/zero modes and only through scale anomalies do we get fluctuations in the otherwise thermal background. Detailed properties and predictions of the topological inflation are presented in in \cite{Vafa&al}. Briefly said, processes take place as well as in other successful models. After reheating everything goes as in the standard model of cosmology.

% Figure environment removed


%**********************
\section{Chern-Simons model in phase O}
\label{chernon}

The initial 4D topological universe in phase I transforms first to Chern-Simons topological phase O and finally dynamically by attractive chernon interactions to present universe in phase II. Chernon interactions are 2+1 dimensional inside a 3+1D world. The topological space of phase I makes phase transition into metric spacetime generated by the newly formed chernon masses (see (\ref{masspar})). The boson and fermion states correspond each other in phases I and O. A summary of the three phases and their properties is given in table \ref{tab:table3}.

\begin{table}[h!]
	\begin{center}
		\captionsetup{width=.8\linewidth}
		\begin{tabular}{|l||l|l|l|} 
			\hline
			Ph. & Particles & Dimension & Symmetry \\ 
			\hline 
			I & Witten theory & 4D topol. & $SU(2)_L \times SU(2)_R$; SUSY \\				 
			O & chernons & 3D top. $\in$ 4D & $SU(3) [\times SU(2)]\times U(1)$; SUSY \\  		
			II & SM particles & metric space & $SU(3) \times SU(2) \times U(1)$; \cancel{SUSY} \\
			\hline
		\end{tabular}
		\caption{\small Development of the universe from phase I to phase O and finally to phase II. Particles of Witten's theory are in table \ref{tab:fields}. The phase O's role is to retain supersymmetry, create SM matter, spacetime metric and baryon asymmetry in the universe. The term $[\times SU(2)]$ indicates appearance of weak interaction "automatically" between u- and d-quarks as well as between e and $\nu$.} 
		\label{tab:table3}
	\end{center}
\end{table}

Chern-Simons-Maxwell (CSM) models have been studied in condensed matter physics, e.g. \cite{Deser_J_T, Giro_G_d_M_N, Beli_D_F_H}. In this note we apply the CSM model in particle physics phenomenology at high energy in the early universe. 

We construct the visible matter of two fermionic chernons: (i) one charged $m^-$, (ii) one neutral $m^0_V$, V = R, G, B, carrying QCD color, and the photon. The Wess-Zumino \cite{WZ} type  action \cite{Rai_1} is supersymmetric as well as C symmetric. The chernons have zero (or very small) mass. Weak interactions operate below $\Lambda_{cr}$ between quarks and leptons, just as in SM. The chernon baryon (B) and lepton (L)  numbers are zero. Given these quantum numbers, quarks consist of three chernons, as indicated in table \ref{tab:table4}.\footnote{There are more combinations of states like those containing an $m^+m^-$ pair. This state annihilates immediately into other chernons, which form later leptons and quarks.}

In \cite{Beli_D_F_H} a 2+1 dimensional Chern-Simons (CS) action \cite{cs, Witten_0} was used to derive chernon-chernon interaction, which turns out to trigger the second phase transition between O and II. In 2+1 dimensions, a fermionic field has its spin polarization fixed up by the sign of mass \cite{Binegar}. The model includes two positive-energy spinors (two spinor families) and a complex scalar $\varphi$. The fermions obey Dirac equation, each one with one polarization state according to the sign of the mass parameter. The vacuum expectation value $v$ of the scalar field $\varphi$ is given by: 
\be
\langle \varphi ^{\ast }\varphi \rangle =v^{2}=-\zeta /\left( 2\lambda
\right) +\left[ \left( \zeta /\left( 2\lambda \right) \right) ^{2}-\mu
^{2}/\lambda \right] ^{1/2}
\label{vev}
\ee
The condition for its minimum is $\mu ^{2}+\frac{\zeta }{2}%
v^{2}+\lambda v^{4}=0$. After the spontaneous symmetry breaking, the
scalar complex field can be parametrized by $\varphi = v+H+i\theta $, where $H$ represents the Higgs scalar field and $\theta $ the would-be Goldstone boson. For manifest renormalizability one adopts the 't Hooft gauge by adding the gauge fixing term $S_{R_{\xi}}^{gt}=\int d^{3}x[-\frac{1}{2\xi }(\partial ^{\mu }A_{\mu }-\sqrt{2}\xi M_{A}\theta)^{2}]$ to the broken action. Keeping only the bilinear and the Yukawa interaction terms one has the following action

\begin{align}
	{S}_{{\rm {CS-QED}}}^{{\rm SSB}} & =\int d^{3}x\biggl\{-\frac{1}{4}F^{\mu \nu}
	F_{\mu \nu }+\frac{1}{2}M_{A}^{2}A^{\mu }A_{\mu } \nonumber \\
	&~~ -\frac{1}{2\xi }(\partial^{\mu }A_{\mu })^{2}+\overline{\psi }_+(i\cancel\partial -m_{eff})\psi _{+} \nonumber \\
	&~~ +\overline{\psi }_{-}(i\cancel\partial +m_{eff})\psi _{-}+ \frac{1}{2}
	\theta \epsilon ^{\mu v\alpha }A_{\mu }\partial _{v}A_{\alpha }  \nonumber \\
	&~~ +\partial ^{\mu }H\partial _{\mu }H-M_{H}^{2}H^{2} +\partial ^{\mu }\theta
	\partial _{\mu }\theta -M_{\theta }^{2}\theta ^{2} \nonumber \\
	&~~ -2yv(\overline{\psi }_{+}\psi _{+}-\overline{\psi }_{-}\psi _{-})H-e_{3}\left( \overline{\psi }
	_{+}\cancel A\psi _{+}+\overline{\psi }_{-}\cancel A\psi _{-}\right) \biggr\}  \label{actionMCS3}
\end{align}
where the mass parameters 
\begin{equation}
	M_{A}^{2}=2v^{2}e_{3}^{2},~~m_{eff}=m_{ch}+yv^{2},~~M_H^{2}=2v^{2}(\zeta +2\lambda v^{2}),~~M_{\theta }^{2}=\xi M_{A}^{2}
	\label{masspar}
\end{equation}
depend on the SSB mechanism. The Proca mass $M_A^2$ represents the mass acquired by the photon through the Higgs mechanism. The Higgs mass, $M_{H}^{2}$, is associated with the real scalar field. The Higgs mechanism also contributes to the chernon mass $m_{ch}$, resulting in an effective mass $m_{eff}$. There are two photon mass-terms in (\ref{actionMCS3}), the Proca and the topological one. 

The chernon-chernon scattering amplitude in the non-relativistic approximation is obtained by calculating the t-channel exchange diagrams of the Higgs scalar and the massive gauge field. The propagators of the two exchanged particles and the vertex factors are calculated from the action (\ref{actionMCS3}) \cite{Beli_D_F_H}.

The gauge invariant effective potential for the scattering considered is obtained in \cite{Kogan, Dobroliubov}
\begin{equation}
	V_{{\rm MCS}}(r)=\frac{e^{2}}{2\pi }\left[ 1-\frac{\theta }{m_{ch}}\right]
	K_{0}(\theta r)+\frac{1}{m_{ch}r^{2}}\left\{ l-\frac{e^{2}}{2\pi \theta }%
	[1-\theta rK_{1}(\theta r)]\right\} ^{2} 
\label{Vmcs}
\end{equation}
where $K_{0}(x)$ and $K_{1}(x)$ are the modified Bessel functions and $l$ is the angular momentum ($l=0$ in this note). In (\ref{Vmcs}) the first term $[~]$ corresponds to the electromagnetic potential, the second one $\{~\}^2$ contains the centrifugal barrier $\left(l/mr^{2}\right)$, the Aharonov-Bohm term and the two photon exchange term.

One sees from (\ref{Vmcs}) the first term may be positive or negative while the second term is always positive. The function $K_{0}(x)$ diverges as $x \ra 0$ and approaches zero for $x \ra \infty$ and $K_{1}(x)$ has qualitatively similar behavior. For our scenario we need negative potential between equal charge chernons. Being embarrassed of having no data points for several parameters in (\ref{Vmcs}) we can give one relation between these parameter values for a binding potential. We must require the condition\footnote{For applications to condensed matter physics, one must require $\theta \ll m_{e}$, and the scattering potential given by (\ref{Vmcs}) then comes out positive  \cite{Beli_D_F_H}.}
\be
\theta \gg m_{ch}
\label{condition}
\ee
The potential (\ref{Vmcs}) also depends on $v^{2}$, the vacuum expectation value, and on $y$, the parameter that measures the coupling between fermions and Higgs scalar. Being a free parameter, $v^{2}$ indicates the energy scale of the spontaneous breakdown of the $U(1)$ local symmetry. 


%**********************************
\section{Baryon asymmetry in phase II}
\label{barasym}

We now examine the potential (\ref{Vmcs}) in the early universe. Consider large number of groups of twelve chernons each group consisting of four $m^+$, four $m^-$ and four $m^0$ particles \cite{Rai_2}. Any bunch may form only electron and proton (hydrogen atoms H), only positron and antiproton ($\bar{\rm{H}}$) or some combination of both H and $\bar{\rm{H}}$ atoms. This is achieved by arranging the chernons appropriately (mod 3) using table \ref{tab:table4}. This way the transition from matter-antimatter symmetric universe to matter-antimatter asymmetric one happens straightforwardly.

Because the Yukawa force (\ref{Vmcs}) is the strongest force the light $e^-$, $e^+$ and the neutrinos combine first from three chernons at the very onset of inflation. To obey condition $B-L=0$ of baryon-lepton balance and to sustain charge conservation, for one electron made of three chernons, nine other chernons have to be created simultaneously, these form a proton.\footnote{Note that instead of particle-antiparticle charge symmetry we form effectively $e^-p^+$ charge symmetry to get baryon asymmetry.} Correspondingly for positrons. One neutrino requires a neutron to be created. The $m^0$ carries in addition color enhancing neutrino formation. This makes neutrinos different from other leptons and the quarks. 

Later, when the protons were formed, because chernons had the freedom to choose whether they are constituents of $\rm{H}$ or $\bar{\rm{H}}$ there are regions of space of various sizes dominated by $\rm{H}$ or $\bar{\rm{H}}$ atoms. Since the universe is the largest statistical system it is expected that there is only a very slight excesses of $\rm{H}$ atoms (or $\bar{\rm{H}}$ atoms which only means a charge sign redefinition) which remain after the equal amounts of $\rm{H}$ and $\bar{\rm{H}}$ atoms have annihilated. The ratio $n_B/n_{\gamma}$ is thus predicted to be $\ll 1$. 


%***********************
\section{Conclusions}
\label{conclusions}

The treatment of topology phase O, SUSY transfer to it, birth of matric spacetime, and SUSY confinement in phase II of the universe are the main points in this note. In order to explore aspects of the early universe in more detail we need a more precise description of phases I and O. 

Here we have extended our previous preon/chernon model to scales above $\Lambda_{cr}$ up to $\MP$. In that purpose we have considered two models for 4D topological gravity in phase I, one proposed some time ago by Witten \cite{Witten:1988xi} and the other more recently by Fang and Gu \cite{Gu:2017, Gu:2021}. The latter seems to show more potential in the three phase evolutionary scenario, including QG tentatively as an effective field theory.

There are three possibilities for the fate of supersymmetry: no SUSY at all, highly broken SM SUSY, and confined SUSY (in chernons or in some other way). We consider the first case unlikely. The second case has been studied thoroughly with certain success but the sparticles are still missing. The third case, described above, agrees with the standard model particle spectrum (1st generation) and provides an answer to matter-antimatter asymmetry by the mechanism presented in \cite{Rai_2} and recapped in section \ref{barasym}. We conclude it is premature to consider supersymmetry nonexistent.

Finally, a word of philosophical caution from the article of Karen Crowther and Niels Linnemann \cite{Crother_L}. We cite them: "There is no requirement that QG be valid to arbitrarily high-energy scales (or to the shortest length scales), and thus, UV-completion cannot be taken as a criterion of theory acceptance. Instead, the necessary requirement is more modest: that the theory be {\it UV-better} (than what we have now)—i.e., that it be valid at the Planck scale. UV-completion only makes sense as criterion within approaches whose goal is a ToE—yet, most approaches to QG do not have this aim." The problem with "Everything" is that we do not know what surprises future experiments will reveal of the universe. Our goal is an "all-inclusive" model of the known universe rather than ToE, preferably UV-complete.


\newpage
%\vskip 2cm
%*********************************************
\appendix
\section{Chernon-particle correspondence}
\label{appdx}

The table \ref{tab:table4} gives the chernon content of SM matter and a proposal for dark matter.

\begin{table}[h!]
	\begin{center}
		\captionsetup{width=.8\linewidth}
		\begin{tabular}{|l|l|} 
			\hline
			SM Matter & Chernon state \\ 
			\hline                                                                        				  % Mass
			$\nu_e$ & $m^0_R m^0_G m^0_B$ \\      % 0 
			$u_R$ & $m^+ m^+ m^0_R$ \\			  % 2.4       	c 10    t 175 - 220 MeV
			$u_G$ & $m^+ m^+ m^0_G$ \\  			
			$u_B$ & $m^+ m^+ m^0_B$ \\
			
			$e^-$ & $m^- m^-m^-$ \\		          % 0.5 tau 2.1   
			$d_R$ & $m^- m^0_G m^0_B $ \\		  % 4.9         s 9.5     b  4.2 MeV
			$d_G$ & $m^- m^0_B m^0_R$ \\			
			$d_B$ & $m^- m^0_R m^0_G$ \\
			\hline
			Dark Matter & Chernon state  \\
			\hline
			boson (or BC) & axion(s), $s^0$ \\
			$e'$ & axino $n$ \\
			meson, baryon $o$ & $n\bar{n}, 3n$ \\
			nuclei (atoms with $\gamma ')$ & multi $n$ \\
			celestial bodies & any dark stuff \\	 
			black holes & any chernon \\
			\hline
		\end{tabular}
		\caption{\small Visible and Dark Matter with corresponding particles. $m^0$ is color triplet, $m^{\pm}$ are color singlets. $e'$ and $\gamma '$ refer to dark electron and dark photon, respectively. BC stands for Bose condensate. Chernons obey anyon statistics.}
		\label{tab:table4}
	\end{center}
\end{table}


%\newpage
%\vskip 2cm

\begin{thebibliography}{99}
	
\bibitem{Vafa&al} Prateek Agrawal, Sergei Gukov, Georges Obied and Cumrun Vafa, Topological Gravity as the Early Phase of Our Universe.
\href{https://arxiv.org/abs/2009.10077}{{\ttfamily arXiv:2009.10077}}
%arXiv:2009.10077 

\bibitem{Rai_1} Risto Raitio, Supersymmetric preons and the standard model, Nuclear Physics B931, 283–290 (2018). \href{https://doi.org://10.1016/j.nuclphysb.2018.04.021} {\tt doi:10.1016/j.nuclphysb.2018.04.021} \href{https://arxiv.org/pdf/1805.03013.pdf} {\tt arXiv:1805.03013}; Risto Raitio, A stringy model of pointlike particles, Nuclear Physics B980 (2022) 115826. \href{https://doi.org/10.1016/j.nuclphysb.2022.115826} {\tt doi:10.1016/j.nuclphysb.2022.115826}

\bibitem{Rai_2} Risto Raitio, A Chern-Simons model for baryon asymmetry, Nuclear Physics B990 (2023) 116174. \href{https://doi.org://10.1016/j.nuclphysb.2023.116174} {\tt doi:10.1016/j.nuclphysb.2023.116174}
\href{https://arxiv.org/pdf/2301.10452.pdf} {\tt arXiv:2301.10452}
%arXiv:2301.10452

\bibitem{Pati_S} J. Pati, A. Salam, Phys. Rev.D10 (1974) 275.

\bibitem{Harari} Haim Harari, Phys. Lett. B86, (1979) 83.

\bibitem{Shupe} M. A. Shupe, Phys. Lett. B86 (1979) 87.

\bibitem{Finkelstein} Robert Finkelstein, Int. J. Mod. Phys A 30(16) (2015) 1530037.

\bibitem{Rai_0} Risto Raitio, A Model of Lepton and Quark Structure. Physica Scripta, 22, 197 (1980). \href{https://dx.doi.org/10.1088/0031-8949/22/3/002} {\tt PS22,197}. \href{http://vixra.org/abs/1903.0224} {viXra:1903.0224}
{\small The core of this model was conceived in November 1974 at SLAC. I proposed that the c-quark would be an excitation of the u-quark, both composites of three 'subquarks'. The idea was opposed by the community and was therefore not written down until five years later.}

\bibitem{Sathiapalan} Sathiapalan, Bala "Duality in Statistical Mechanics and String Theory". Physical Review Letters. 58 (16): 1597–9 (1987). doi:10.1103/PhysRevLett.58.1597

\bibitem{Brandenberger:2004} Robert Brandenberger, String Gas Cosmology, Progress of Theoretical Physics, Vol. 111, No. 4 (2004).
\href{https://arxiv.org/pdf/0808.0746.pdf} {\tt arXiv:0808.0746}

\bibitem{BRS} C. Becchi, A. Rouet and R. Stora, Renormalization of the Abelian Higgs-Kibble Model, Commun. Math. Phys. 42, 127 (1975).

\bibitem{Tyutin} I.V. Tyutin, Gauge Invariance in Field Theory and Statistical Physics in Operator Formalism, Lebedev Institute preprint N39 (1975).

\bibitem{Witten_0} Edward Witten, Quantum Field Theory and the Jones Polynomial, Commun. Math. Phys. 121, 351-399 (1989).

\bibitem{Witten:1988xi}
E.~Witten, Topological Gravity, \href{http://dx.doi.org/10.1016/0370-2693(88)90704-6}{ Phys. Lett. B}{206} (1988) 601--606.

\bibitem{Batalin:1981jr}
I.~Batalin and G.~Vilkovisky, Gauge Algebra and Quantization, \href{http://dx.doi.org/10.1016/0370-2693(81)90205-7}{ Phys. Lett. B}{102} (1981) 27-31.

\bibitem{Batalin:1984jr}
I.~Batalin and G.~Vilkovisky, Quantization of Gauge Theories with Linearly Dependent Generators, \href{http://dx.doi.org/10.1103/PhysRevD.28.2567}{ Phys. Rev. D} {28} (1983) 2567-2582. [Erratum: Phys.Rev.D 30, 508 (1984)].

\bibitem{Batalin:1985qj}
I.~Batalin and G.~Vilkovisky, Existence Theorem for Gauge Algebra, \href{http://dx.doi.org/10.1063/1.526780}{J. Math. Phys.} {26} (1985) 172-184.

\bibitem{Brooks:1988jm}
R.~Brooks, D.~Montano, and J.~Sonnenschein, Gauge Fixing and Renormalization in Topological Quantum Field Theory, \href{http://dx.doi.org/10.1016/0370-2693(88)90458-3}{ Phys. Lett. B}{214} (1988) 91-97.

\bibitem{Witten:1988ze}
E.~Witten, Topological Quantum Field Theory,
\href{http://dx.doi.org/10.1007/BF01223371}{ Commun. Math. Phys.} {117} (1988) 353.
	
\bibitem{Gu:2017} 
Zheng-Cheng Gu, The emergence of 3+1D Einstein gravity from topological gravity theory. Festschrift in Honor of the C. N. Yang Centenary, pp. 175-183 (2022).  DOI:\href{https://doi.org/10.1142/9789811264153_0009}{ Festschrift C. N. Yang}
%1709.09806

\bibitem{Gu:2021} 
Tianyao Fang and Zheng-Cheng Gu, Topological gravity in 3+1D and a possible origin of dark matter.
DOI:\href{https://doi.org/10.48550/arXiv.2106.10242}  {2106.10242}
%2106.10242

\bibitem{Chenlong} X. Chen, Z.-C. Gu, Z.-X. Liu, and X.-G. Wen, 
Symmetry protected topological orders and the group cohomology of their symmetry group, Phys. Rev. B87, 155114 (2013).

\bibitem{Wenscience} X. Chen, Z.-C. Gu, Z.-X. Liu, X.-G. Wen, 
Symmetry-Protected Topological Orders in Interacting Bosonic Systems, Science 338, 1604 (2012).

\bibitem{cobordism} A. Kapustin, Symmetry Protected Topological Phases, Anomalies, and Cobordisms: Beyond Group Cohomology, arXiv:1403.1467; A. Kapustin, Bosonic Topological Insulators and Paramagnets: a view from cobordisms, arXiv:1404.6659.

\bibitem{Wencoho} X.-G. Wen, Construction of bosonic symmetry-protected-trivial states and their topological invariants via $G\times SO(\infty)$ non-linear $\sigma$-models,  Phys. Rev. B91, 205101 (2015).

\bibitem{Topgravity}Zheng-Cheng Gu,
The emergence of 3+ 1D Einstein gravity from topological gravity, arXiv:1709.09806 

\bibitem{aada1} Peng Ye and Zheng-Cheng Gu,
Topological quantum field theory of three-dimensional bosonic Abelian-symmetry-protected topological phases, Phys. Rev. B93, 205157 (2016).

\bibitem{aada2} Juven Wang, Xiao-Gang Wen, and Shing-Tung Yau,
Quantum Statistics and Spacetime Surgery, arXiv:1602.05951  

\bibitem{aada3} Pavel Putrov, Juven Wang, and Shing-Tung Yau,
Braiding Statistics and Link Invariants of Bosonic/Fermionic Topological Quantum Matter in 2+1 and 3+1 dimensions, Annals of Physics 384, 254 (2017).

\bibitem{loop1} Chenjie Wang and Michael Levin, 
Braiding statistics of loop excitations in three dimensions, Phys. Rev. Lett. 113, 080403 (2014).

\bibitem{loop2} Shenghan Jiang, Andrej Mesaros, and Ying Ran, Generalized modular transformations in 3+1D topologically ordered phases and triple linking invariant of loop braiding, Phys. Rev. X 4, 031048 (2014).

\bibitem{TCAT1} Liang Kong and Xiao-Gang Wen,
Braided fusion categories, gravitational anomalies, and the mathematical framework for topological orders in any dimensions, arXiv:1405.5858

\bibitem{TCAT2} Tian Lan, Liang Kong, and Xiao-Gang Wen, 
A classification of 3+1D bosonic topological orders (I): the case when point-like excitations are all boson, arXiv:1704.04221 

\bibitem{Horowitz} Gary Horowitz, Topology Change in General Relativity, Proceedings of the Sixth Marcel Grossman Meeting held in Kyoto, Japan, June 24-29, 1991. Edited By: Humitaka Sato and Takashi Nakamura. doi:10.1142/1644 

\bibitem{Tanaka_N} Izumi Tanaka and Seiji Nagami, Gauge Group and Topology Change, Int. J. Geom. Methods Mod. Phys. 08, 1225 (2011).
doi:/10.1142/S0219887811005622
		
\bibitem{Deser_J_T}	S. Deser, R. Jackiw and S. Templeton, Phys. Rev. Lett. 48, 975 (1982) and Ann. Phys. (NY) 140, 372 (1982).  

\bibitem{Giro_G_d_M_N} H. O. Girotti, M. Gomes, J. L. deLyra, R. S. Mendes, and J. R. S. Nascimento, Electron–Electron Bound States in QED$_3$. \href{https://arxiv.org/pdf/hep-th/9210161.pdf} {\tt arXiv:hep-th/9210161}
%hep-th/9210161

\bibitem{Beli_D_F_H} H. Belich, O. M. Del Cima, M. M. Ferreira Jr. and J. A. Helay\"{e}l-Neto, Electron-Electron Bound States in Maxwell-Chern-Simons-Proca QED$_3$, Eur. Phys. J. B 32, 145–155 (2003).
\href{https://arxiv.org/pdf/hep-th/0212285.pdf} {\tt arXiv:hep-th/0212285}
%hep-th/0212285

\bibitem{WZ} J. Wess, and B. Zumino, (1974) Nucl. Phys. B 70, 39.

\bibitem{cs} S.-S. Chern and J. Simons, Characteristic forms and geometric invariants, Annals of Mathematics. 99 (1): 48–69 (1974).

\bibitem{Binegar} B. Binegar, J. Math. Phys. 23, 1511 (1982); S. Deser and R. Jackiw, Phys. Lett. B263, 431 (1991); R. Jackiw and V. P. Nair, Phys. Rev. D43, 1933 (1991); \ J. Fr\"{o}hlich and P. A. Marchetti, Lett. in Math. Phys. 16, 347 (1988).

\bibitem{Kogan} Ya. I. Kogan, JETP Lett. 49, 225 (1989).

\bibitem{Dobroliubov} M. I. Dobroliubov, D. Eliezer, I. I. Kogan, G.W. Semenoff and R.J. Szabo, Mod. Phys. Lett. A, 8, 2177 (1993).

\bibitem{Crother_L} Karen Crowther and Niels Linnemann, Renormalizability, fundamentality and a final theory:
The role of UV-completion in the search for
quantum gravity, Brit. J. Phil. Sci. 70 (2019) 2, 377-406.
\href{https://arxiv.org/pdf/1705.06777.pdf} {\tt arXiv:1705.06777}
%arXiv:1705.06777

\end{thebibliography}

\end{document}

