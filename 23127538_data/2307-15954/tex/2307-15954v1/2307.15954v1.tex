\documentclass[reqno]{article}

\textwidth135mm
\textheight227mm
\hoffset-5mm
\voffset-10mm

\usepackage{xcolor}
%\usepackage{latexsym}
%\usepackage{amsbsy}
%\usepackage{amsthm}
%\usepackage{epsfig}
%\usepackage{booktabs}
%\usepackage{bm}
\usepackage{amssymb}
\usepackage{amsfonts}
\usepackage{amsmath}
%\usepackage{mathtools}   
%\usepackage{float}
\numberwithin{equation}{section}
\usepackage{enumerate}
%\usepackage[notcite,notref]{showkeys}
%\usepackage[inputenc.sty]{inputenc}

\newtheorem{theorem}{Theorem}[section]
\newtheorem{proposition}[theorem]{Proposition}
\newtheorem{lemma}[theorem]{Lemma}
\newtheorem{corollary}[theorem]{Corollary}
\newtheorem{definition}[theorem]{Definition}
\newtheorem{example}[theorem]{Example}
\newtheorem{xca}[theorem]{Exercise}

%\theoremstyle{remark}
\newtheorem{remark}[theorem]{Remark}
\newtheorem{assumption}[theorem]{Assumption}

\newcommand{\ran}{{\rm ran}}
\newcommand{\hol}{{\rm hol}}
\newcommand{\dom}{{\rm dom}}
\newcommand{\mul}{{\rm mul}}

\title {Green's boundary relation model in a Krein space}
 
\author{Author: Muhamed Borogovac}

\begin{document}

\maketitle 

%\textbf{A concise and informative title}:  Green's boundary relation
%\\
%\textbf{The affiliation and address of the author}: 
%BML - Actuarial Department, Canton, MA 02021, USA
%\textbf{The e-mail address, and telephone of the author}: 
muhamed.borogovac@gmail.com
%1-857-991-8779 
\\

The 16-digit ORCID of the author: \textbf{0000-0001-6857-8309}
\\
\\
\textbf{Abstract:} Given Krein and Hilbert spaces $\left( \mathcal{K},[.,.] \right)$ and $\left( \mathcal{H},\, \left( .,. \right) \right)$, respectively. We generalize the concept of the boundary triple $\Pi =(\mathcal{H}, \Gamma _{0}, \Gamma_{1})$, i.e. we introduce a general concept of the \textit{Green's boundary relation} simply as an isometric relation $\Gamma$ between Krein spaces $\left( \mathcal{K}^{2}, \left[ .,.\right]_{\mathcal{K}^{2}} \right) $ and $\left(\mathcal{H}^{2}, \left[ .,.\right]_{\mathcal{H}^{2}} \right) $, without any conditions on $\dom\, \Gamma$ and $\ran\, \Gamma$. The main properties of the Green's boundary relations are proved. In the process, a new characterization of unitary linear relations between Krein spaces is proved. The theorem about the main transformation between self-adjoint and unitary relations is strengthened. Then two statements about generalized Nevanlinna families are generalized by means of the Green's boundary relation approach. In addition, some already known boundary triples with the Hilbert space $\mathcal{K}$, such as AB-generalized, B-generalized, ordinary, isometric, unitary, quasi-boundary and S-generalized boundary triples are generalized from Hilbert to the Krein space $\mathcal{K}$, by means of the Green's boundary relation.

\textbf{Key words:} 

Krein space, boundary relation, self-adjoint linear relation, unitary linear relation 

\textbf{MSC (2020)} 46C20 47B50 47B25 47A06 

\section{Preliminaries and introduction }\label{s2}

\textbf{1.1.} By $\mathbb{N}$, $\mathbb{R}$, $\mathbb{C}$ we denote sets of positive integers, 
real numbers, and complex numbers, respectively. Let $\left( \mathcal{K},\, \left[ .,. 
\right] \right)$ denote a Krein space. That is a complex vector space on 
which a scalar product, i.e. a Hermitian sesquilinear form $\left[ .,. 
\right]$, is defined such that the following decomposition of $\mathcal{K}$ exists
\[
\mathcal{K}=\mathcal{K}_{+}[+]\mathcal{K}_{-}
\]
where $\left( \mathcal{K}_{+},[.,.] \right)$ and $\left( \mathcal{K}_{-},-[.,.] \right)$ are 
Hilbert spaces which are mutually orthogonal with respect to the form $[.,.]$; elements $x,y \in \mathcal{K} $ are \textit{orthogonal} if it holds $\left[ x,y\right]=0$, which we denote by $x \left[ \perp \right]y$. Every Krein space $\left( \mathcal{K},[.,.] \right)$ is \textit{associated} with a Hilbert space $\left( \mathcal{K},(.,.) \right),$ which is defined as a direct and orthogonal sum of the Hilbert spaces $\left( \mathcal{K}_{+},[.,.] \right)$ and $\left( \mathcal{K}_{-},-[.,.] \right)$. Topology in the Krein space $\mathcal{K}$ is introduced by means of the associated Hilbert space $\left( \mathcal{K},(.,.) \right)$. \textit{Orthogonal companion} $A^{\left[ \perp \right]}$  of the set $A$ is defined by $A^{\left[ \perp \right]}:=\left\lbrace y\in \mathcal{K}:x\left[ \perp \right]y,\forall x \in A\right\rbrace $ and the \textit{isotropic} part $M$ of $A$ is defined by  $M:= A \cap A^{\left[ \perp \right]}$. A set $A \subseteq \mathcal{K}$ is called \textit{neutral} if it satisfies $A \subseteq A^{\left[ \perp \right]}$. A set $A \subseteq \mathcal{K}$ is called \textit{hyper-maximal neutral} if $A = A^{\left[ \perp \right]}$. For properties of Krein spaces one can see e.g. \cite[Chapter V]{Bog}.

If the scalar product $\left[ .,. \right]$ has $\kappa \in \mathbb{N}$ negative squares, then we call it a \textit{Pontryagin space of negative index} $\kappa $. If $\kappa =0$, then it is a Hilbert space. More information about Pontryagin space can be found e.g. in \cite {IKL}. 
\\

\textbf{1.2.} The following definitions of a linear relation and basic concepts related to it can be found in \cite {A,S,DS,DM2}. In the sequel $X$, $Y$, $W$ are Krein spaces which include Pontryagin and Hilbert spaces. 

A \textit{linear relation} $T: X\rightarrow Y$ is a linear manifold $T \subseteq X \times Y $. If $ X=Y $, then $T$ is said to be a \textit{linear relation in} $X$. A linear relation $T$ is closed if it is a (closed) subspace with respect to the product topology of $X \times Y $. As usually, for a linear relation $ T: X \rightarrow Y $, or $ T \subseteq X \times Y $, the symbols $ \dom \, T $, $ \ran\, T $ and $ \ker T $ mean domain, range and kernel, respectively. In addition, we will use the following concepts and notation for two linear relations, $T$ and $S$ from $X$ into $Y$ and a linear relation $U$ from $Y$ into $W$: 
\[
\mul\, T:=\left\{ g\in Y : \left\{ 0,g \right\}\in T \right\},
\]
\[
T\left( f \right):=\left\{ g\in Y, : \left\{ f,g \right\}\in T \right\}, (f\in D\left( T \right)),
\]
\[
T^{-1}:=\left\{ \left\{ g,f \right\}\in Y \times X : \left\{ f,g \right\}\in T \right\},
\]
\[
zT:=\left\{ \left\{ f,zg \right\}\in X \times Y : \left\{ f,g \right\}\in T \right\}, (z\in \mathbb{C}),
\]
\[
S+T:=\left\{ \left\{ f,g+k \right\} : \left\{ f,g \right\}\in S,\left\{ f,k \right\}\in T \right\},
\]
\[
S\hat{+}T:=\left\{ \left\{ f+h,g+k \right\} : \, \left\{ f,g \right\}\in S,\left\{ h,k \right\}\in T \right\},
\]
\[
S\dot{+}T:=\left\{ \{f+h,g+k\} : \left\{ f,g \right\}\in S,\left\{ h,k \right\}\in T, S\cap T=\lbrace 0\rbrace \right\},
\]
\[
UT:=\left\{ \left\{ f,k \right\}\in X \times W \, : \, \left\{ f,g \right\}\in T,\left\{ g,k \right\}\in U\, for\, some\, g\in Y \right\},
\]
\[
T^{\ast}:=\left\{ \left\{ k,h \right\}\in Y \times X : \left[ f,h \right]=\left[ g,k \right]\, for\, all\, \left\{ f,g \right\}\in T \right\},
\]
\[
T_{\infty }:=\left\{ \left\{ 0,g \right\}\in T \right\}=\lbrace 0 \rbrace \times \mul\, {T}.
\]
If $\mul\, {T}=\{ 0\}$, we say that T is \textit{single-valued} linear relation, i.e.\textit{ operator}. The sets of closed linear relations, closed operators, and bounded operators in X are denoted by $\tilde{C}(X)$, $C(X)$, $B(X)$, respectively.  

A linear relation $V\subseteq X\times Y$ is called \textit{isometric} (\textit{unitary}) if $V^{-1}\subseteq V^{\ast }\, \left( V^{-1}=V^{\ast} \right)$, see \cite[p.5360]{DHMS1}. Let $A$ be a linear relation in a Krein space $\mathcal{K}$. When $X=Y=\mathcal{K}$ we use the notation $A^{+}$ rather than $A^{\ast}$. We say that $A$ is \textit{symmetric} (\textit{selfadjoint}) if it satisfies $A\subseteq A^{+}$ ($A=A^{+})$.  

Every point $\alpha \in \mathbb{C}$ for which $\left\{ f,\alpha f \right\}\in A$, with some $f\ne 0$, is 
called a \textit{finite eigenvalue}, denoted by $\alpha \in \sigma_{p}(A)$. The corresponding vectors 
are \textit{eigenvectors belonging to the eigenvalue} $\alpha $. If for some $z \in \mathbb{C}$ the operator $(A-z)^{-1}$ is bounded, not necessarily densely defined in $\mathcal{K}$, then $z$ is a \textit{point of regular type of} $A$, symbolically, $z\in \hat{\rho }\left( A \right)$. If for $z\in \mathbb{C}$ the relation $\left( A-z \right)^{-1}$ is a bounded operator and $\overline{\ran \left(A-z \right)} = \mathcal{K}$ ($\ran \left(A-z \right)= \mathcal{K}$, if $A$ is closed), then $z$ is a \textit{regular point of} $A$, symbolically $z\in \rho \left( A \right)$.  
\\

\textbf{1.3. Introduction}

Let $(\mathcal{K}\left[ .,. \right])$ be a separable Krein (Pontryagin or Hilbert) space. Typically an investigation of a boundary triple $\Pi=(\mathcal{H},\Gamma_{0}, \Gamma_{1})$, begins with the assumption that there exists a closed symmetric linear operator or relation $S\subseteq \mathcal{K}^{2}$, with equal, finite or infinite defect numbers, c.f. \cite{DM1,DM2,BHS,DHMS1,DHMS2,BDHS}. In time, in those papers, the concept of ordinary boundary triple $\Pi$ for symmetric operator $S$ in a Hilbert space $\mathcal{K}$ with a surjective reduction operator $\Gamma$, see e.g. \cite{DM2, BHS, DHM}, generalizes towards symmetric relations $S$ in a Krein space $\mathcal{K}$ with reduction relation $\Gamma$. In all above papers, the assumption that the operator or relation $S\subseteq \mathcal{K}^{2}$ exists was essential to define the triple $\Pi$. The domain of the reduction operator or relation $\Gamma=\lbrace\Gamma_{0}, \Gamma_{1}\rbrace : \mathcal{K}^{2} \rightarrow \mathcal{H}^{2}$ was always defined by means of $S$, i.e. by menas of the adjoint relation $S^{+}\subseteq \mathcal{K}^{2}$. The following is an of the generalizations of the ordinary boundary triple, the formal definition of the boundary triple $\Pi$ when $\mathcal{K}$ is a Krein space, see e.g. \cite{D1,D2,BDHS}. 

\begin{definition}\label{definition12} A triple $\Pi =(\mathcal{H}, \Gamma _{0}, \Gamma_{1})$, where $\mathcal{H}$ is a Hilbert space and $\Gamma_{0}, \Gamma_{1}$ are bounded operators from $S^{+}$ to $\mathcal{H}$, is called an ordinary boundary triple for the relation $S^{+}$ if the abstract Green's identity 
\begin{equation}
\label{eq12}
\left[ f',g \right]-\left[ f,g' \right]=\left( \Gamma_{1}\hat{f},\, 
\Gamma_{0}\hat{g} \right)_{\mathcal{H}}-\left( \Gamma_{0}\hat{f},\, \Gamma 
_{1}\hat{g} \right)_{\mathcal{H}}, \forall \hat{f}, \hat{g}\in S^{+}, 
\end{equation}
holds, and the mapping $\Gamma :\hat{f}\to \left( {\begin{array}{*{20}c}
\Gamma_{0}\hat{f}\\
\Gamma_{1}\hat{f}\\
\end{array} } \right)$ from $S^{+}$ to $\mathcal{H}\times \mathcal{H}$ is surjective. The operator $\Gamma $ is called boundary or reduction operator.
\end{definition}

In other kinds of boundary triples $\Pi =(\mathcal{H}, \Gamma _{0}, \Gamma_{1})$ the existence of a symmetric reltion $S$ is also assumed, see \cite{BL,DHM}. However, in applications we usually do not know the symmetric relation or operator $S$ in advance. We usually know only a differential expression and the corresponding differential equation. 

The following is one of the simplest examples. Let us denote $\mathcal{K}=L^{2}(0,\infty )$. Then the equation is 
\[
\frac{d^{2}y}{dx^{2}}-\lambda y=g\left( x \right),
\]
and the differential expression is $l:=-\frac{d^{2}}{dx^{2}}$, $x \in \left( 0,\infty \right)$. The expression $l$ is regular at 0 and the limit-point case at $\infty $. We know that the maximal domain $\mathcal{D}$ of $l$ consists of all functions $g(x)\in L^{2}(0,\infty )$ for which $g\left( x \right), \frac{dg}{dx}$ are absolutely continuous and $\frac{d^{2}g}{dx^{2}}\in L^{2}(0,\infty )$. We do not know in advance what might be the closed linear relation or operator $S\subseteq \mathcal{K}^{2}$.  Then, if boundary operator $\Gamma =\lbrace \Gamma _{0}, \Gamma_{1}\rbrace$ is defined by
\[
\mathcal{K}:=L^{2}(0;\infty ); \mathcal{H}=\mathbb{C};  \Gamma_{0}f:=f\left( 0 \right); \Gamma_{1}f:=f'\left( 0 \right).
\]
one finds out that $S:=\lbrace f \in \mathcal{D} : f\left( 0 \right)=0 \wedge f'\left( 0 \right)=0 \rbrace$ is the closed symmetric linear relation that satisfies the abstract Green's identity (\ref{eq12}). 

The point is that in theory we assume in advance the existence of $S$ while in applications, such as this example, we do not know $S$ in advance. We can even determine the Weyl solution of the equation $\frac{d^{2}y}{dx^{2}}-\lambda y=0$ and the defect number without ever determining the symmetric linear relation $S$. This motivated us to introduce a general model, which we call the \textit{Green's boundary relation (GBR)}, without assuming in advance existence of the closed symmetric linear relation $S$.

In \cite[Theorem 2.3.(b)]{B2} we found additional motivation to introduce GBR. In that theorem, starting only from a given regular function $Q\in N_{\kappa }\left( \mathcal{H} \right)$, the existence of a unique closed symmetric linear relation $S$ has been proved. Then the existence of an ordinary boundary triple $\Pi=(\mathcal{H}, \Gamma_{0}, \Gamma_{1})$ for $S^{+}$, such that $Q$ is the Weyl function associated with $S$ and $\Pi$ was proved. Again, the existence of the linear relation $S$ has not been assumed in advance, that assumption was not necessary again. 

Motivated by the above two examples, we introduced the Green's boundary relation, see Definition \ref{definition24}, where we only assumed existence of a separable Krein space $\left( \mathcal{K}, \left[ .,. \right] \right)$, existence of an auxiliary Hilbert space $\left( \mathcal{H}, \left( .,. \right) \right)$ and existence of a linear relation $\Gamma : \mathcal{K}^{2} \rightarrow\mathcal{H}^{2}$ that satisfies identity (\ref{eq212}), also known as the abstract Green's identity. 

In Section \ref{s4}, in addition to Definition \ref{definition24}, some properties of the Green's boundary relation are proved. For example, in Proposition \ref{proposition28}, a condition on the GBR that is equivalent to existence of the symmetric closed relation $S\subseteq \mathcal{K}^{2}$ that satisfies $ S^{+}=\overline{\dom}\, \Gamma \subseteq \mathcal{K}^{2}$ is given. This way, a Green's boundary relation model is created in which we can fit all kinds of boundary triples, including ordinary, unitary, and other well known triples, see e.g.  \cite{DM1, DM2, DHM}. 

In Section \ref{s6}, for the given Krein spaces $X$ and $Y$, the Krein space $\left( X\times Y, \left[ .,. \right]_{X\times Y} \right)$ is introduced. This space is used to prove Proposition \ref{proposition34}, a generalization to the Krein space $\mathcal{K}$ of the \cite[Proposition 1.3.2]{BDHS} which holds for a Hilbert space. The Krein space $\left( X\times Y, \left[ .,. \right]_{X\times Y} \right)$ is also used to prove some other statements. This space has some nice properties; for example an isometric (unitary) relation $V:X\rightarrow Y$ is a neutral manifold (hyper-maximal neutral subspace, respectively) in the Krein space $X\times Y$, see Propositions \ref{proposition38}.

In Section \ref{s8}, some general properties of the isometric linear relation $V:X\to Y$, where $X$ and $Y$ are Krein spaces, are proved, that we need in the research of Green's boundary relation, see Proposition \ref{proposition42}. The most important result of Section 4 is Theorem \ref{theorem46} where we prove a characterization of unitary relations between Krein spaces. We also give generalizations of \cite[Lemma 1.8.1]{BHS} and generalizations of some statements of  \cite[Corollary 4.2]{DHMS1}.

In Section \ref{s10}, the results of Section \ref{s8} and properties of the space $\left( X\times Y, \left[ .,. \right]_{X\times Y} \right)$, when $X=\mathcal{K}^{2}, Y=\mathcal{H}^{2}, V=\Gamma$, are applied on the Green's boundary relation $\Gamma : \mathcal{K}^{2} \rightarrow \mathcal{H}^{2}$ in order to prove some important properties of $\Gamma$, see Proposition \ref{proposition52}. In Theorem \ref{theorem55} we prove a characterization of unitary boundary relation $\Gamma : \mathcal{K}^{2} \rightarrow \mathcal{H}^{2}$. 

If $\Gamma: \mathcal{K}^{2} \rightarrow \mathcal{H}^{2} $ is a reduction operator of an ordinary boundary triple $\Pi$, then the closed symmetric linear relation $S$ associated with $\Pi$ coincides with $N:= \ker \Gamma $ and with the isotropic manifold $M$ of $\dom\, \Gamma $. It is not necessarily the case for more general Green's boundary relations. In Corollary \ref{corollary56} we give sufficient conditions for $S=N=M$ to hold if $\Gamma$ is a Green's boundary relation. In Proposition \ref{proposition510} we prove $S=N=M$ if $\Gamma$ is a unitary boundary relation. 

In subsection 5.2, we learn that isometric and unitary boundsry relation $\Gamma: \mathcal{K}^{2} \rightarrow \mathcal{H}^{2} $, that satisfy condition $\mathcal{K} \cap \mathcal{H} =\lbrace0\rbrace$, have very specific properties. Our Proposition \ref{proposition520} strengthen \cite[Proposition 2.10]{DHMS1} when $\mathcal{K} \cap \mathcal{H} =\lbrace0\rbrace$.

In Section \ref{s12}, the previously proven properties of Green's boundary relationare are used to generalize two statements about generalized Nevanlinna families. One generalization is Proposition \ref{proposition64} and the other is Theorem \ref{theorem66} which generalizes \cite[Theorem 4.8]{BDHS}. 

In Section \ref{s14}, we study Green's boundary relations $\Gamma$ with non-degenerate $\ran\, \Gamma$, including $\overline{\ran}\, \Gamma=\mathcal{H}^{2}$, the condition which makes those relations \textit{Green's boundary operators}. For example, we study the closure $\bar{\Gamma}$ of the Green's boundary relation with non-degenerate  $\ran\, \Gamma$.  

In Section \ref{s16}, we show another use of the Green's boundary model. The well-known boundary triples, previously studied in \cite{DHM}, are fitted into Green's boundary relation model, i.e. the boundary triples with Hilbert space $\mathcal{K}$, such as ordinary, isometric, unitary, AB-generalized, B-generalized, quasi-boundary and S-boundary relations are defined in terms of the Green's boundary relations with Krein space $\mathcal{K}$. This use of GBR reveals some relationships between various boundary triples. 

\section{Green's boundary relation }\label{s4}

Let $(\mathcal{K},\left[ .,. \right])$ and $(\mathcal{H},(.,.))$ be Krein and Hilbert spaces, respectively, and let $(\mathcal{K}^{2}, \left[ .,. \right])\,$ and $(\mathcal{H}^{2}, (.,.))$ be the corresponding \textit{product spaces}, with inner products $\left[ .,. \right]$ and $(.,.)$ defined in the following way
\begin{equation}
\label{eq22}
\left[ \hat{f},\hat{g} \right]=\left[ f,g \right]+\left[ f',g' 
\right],\hat{f}=\left( {\begin{array}{*{20}c}
f\\
f'\\
\end{array} } \right)\in \mathcal{K}^{2}, \hat{g}=\left( {\begin{array}{*{20}c}
g\\
g'\\
\end{array} } \right)\in \mathcal{K}^{2}, 
\end{equation}
\begin{equation}
\label{eq24}
\left( \hat{h},\hat{k} \right)=\left( h,k \right)+\left( h',k' 
\right), \hat{h}=\left( {\begin{array}{*{20}c}
h\\
h'\\
\end{array} } \right)\in \mathcal{H}^{2}, \hat{k}=\left( {\begin{array}{*{20}c}
k\\
k'\\
\end{array} } \right)\in \mathcal{H}^{2}.
\end{equation}
Let us introduce the involution
\begin{equation}
\label{eq26}
J_{\mathcal{K}}:=\left( {\begin{array}{*{20}c}
0 & -iI_{\mathcal{K}}\\
iI_{\mathcal{K}} & 0\\
\end{array} } \right), J_{\mathcal{H}}:=\left( {\begin{array}{*{20}c}
0 & -iI_{\mathcal{H}}\\
iI_{\mathcal{H}} & 0\\
\end{array} } \right),
\end{equation}
where $I_{\mathcal{K}}$ and $I_{\mathcal{H}}$ are identities in the corresponding spaces, and let $\left( \mathcal{K}^{2},\left[ .,. \right]_{\mathcal{K}^{2}} \right)$ and $\left( \mathcal{H}^{2}, \left[ .,. \right]_{\mathcal{H}^{2}} \right)$ be the inner product spaces obtained by means of the following inner products 
\begin{equation}
\label{eq28}
\left[ \hat{f},\hat{g} \right]_{\mathcal{K}^{2}}:=\left[ J_{\mathcal{K}}\hat{f},\hat{g} 
\right],\hat{f},\hat{g}\in \mathcal{K}^{2},
\end{equation}
\begin{equation}
\label{eq210}
\left[ \hat{h},\hat{k} \right]_{\mathcal{H}^{2}}:=\left( J_{\mathcal{H}}\hat{h},\hat{k} 
\right), \hat{h},\hat{k}\in \mathcal{H}^{2},
\end{equation}
respectively. 

In the following lemma we collect several well known facts.

\begin{lemma}\label{lemma22} Let $(\mathcal{K},\left[ .,. \right])$ be a Krein space and let $\left( \mathcal{K},( .,.) \right) $ be the Hilbert space associated with $(\mathcal{K},\left[ .,. \right])$ by means of the fundamental symmetry $G$, i.e. $\left[x,y \right]=\left( Gx,y \right),  \forall x,y \in \mathcal{K}$. The spaces $\left( \mathcal{K}^{2}, \left[ .,. \right]_{\mathcal{K}^{2}}\right) $ and $\left( \mathcal{H}^{2},  \left[ .,. \right]_{\mathcal{H}^{2}}\right) $ with scalar products (\ref{eq28}) and (\ref{eq210}), respectively, are Krein spaces. They are associated with the \textbf{product} Hilbert spaces $(\mathcal{K}^{2}, \left( .,. \right))$ and $(\mathcal{H}^{2},\left( .,. \right))$ with the fundamental symmetries
\[
\mathcal{G}=\left( {\begin{array}{*{20}c}
0 & -iG\\
iG & 0\\
\end{array} } \right) \,\wedge \, \mathcal{J}=\left( {\begin{array}{*{20}c}
0 & -iI_{\mathcal{H}}\\
iI_{\mathcal{H}} & 0\\
\end{array} } \right)
\]
respectively. Equivalently,
\begin{equation}
\label{eq211.5}
\left[ \hat{f},\hat{g} \right]_{\mathcal{K}^{2}}=\left(\mathcal{G} \hat{f},\hat{g} \right)_{\mathcal{K}^{2}}, \forall \hat{f}, \hat{g} \in \mathcal{K}^{2}.
\end{equation}
and (\ref{eq210}), respectively, hold.
\end{lemma}
If $\left( X,  {[.,.]}_{X} \right)$ and $\left(Y, {[.,.]}_{Y} \right)$ are Krein spaces, then a linear relation $V\subseteq X\times Y$ is called \textit{isometric} if $V^{-1}\subseteq V^{\ast }$, see e.g. \cite[p.5360]{DHMS1}. Obviously, it is equivalent with 
\[
\left[ f' ,g' \right]_{Y}=\left[ f,g \right]_{X},\forall 
\left( {\begin{array}{*{20}c}
f\\
f'\\
\end{array} } \right),\, \left( {\begin{array}{*{20}c}
g\\
g'\\
\end{array} } \right)\in V,
\]
and $V$ preserves orthogonality. A linear relation $V\subseteq X\times Y$ is called \textit{unitary} if $V^{-1}=V^{\ast }$. 
\begin{definition}\label{definition24} Let $\Gamma :\hat{f}\to \hat{h}, \hat{f}=\left( 
{\begin{array}{*{20}c}
f\\
f'\\
\end{array} } \right)\in \mathcal{K}^{2},\hat{h}=\left( {\begin{array}{*{20}c}
h\\
h'\\
\end{array} } \right)\in \mathcal{H}^{2}$, be a linear relation from the Krein space $(\mathcal{K}^{2},\left[ .,. \right]_{\mathcal{K}^{2}})$ to Krein space $\left( \mathcal{H}^{2},\left[ .,. \right]_{\mathcal{H}^{2}} \right)$ which satisfies 
\begin{equation}
\label{eq212}
\left[ f',g \right]-\left[ f,g' \right]=\left( h',k \right)-\left( 
h,k' \right),\forall \left( {\begin{array}{*{20}c}
\hat{f}\\
\hat{h}\\
\end{array} } \right), \left( {\begin{array}{*{20}c}
\hat{g}\\
\hat{k}\\
\end{array} } \right)\in \Gamma \subseteq \mathcal{K}^{2}\times H^{2},
\end{equation}
or equivalently
\begin{equation}
\label{eq214}
\left[ \hat{f},\hat{g} \right]_{\mathcal{K}^{2}}=\left[ \hat{h},\hat{k} 
\right]_{\mathcal{H}^{2}},\forall \left( {\begin{array}{*{20}c}
\hat{f}\\
\hat{h}\\
\end{array} } \right),\left( {\begin{array}{*{20}c}
\hat{g}\\
\hat{k}\\
\end{array} } \right)\in \Gamma \subseteq \mathcal{K}^{2}\times \mathcal{H}^{2}.
\end{equation}
Then the relation $\Gamma $ is called \textbf{Green's boundary relation}, abbreviated as GBR.
\end{definition}
Note, the \textbf{Green's boundary relation} (GBR) $\Gamma :\mathcal{K}^{2}\to \mathcal{H}^{2}$ introduced here is a generalization of the \textbf{boundary relations} defined in \cite[Definition 3.1]{DHMS1} and \cite[Definition 3.1.]{BDHS} in the following aspects:
\begin{itemize}
\item We do not have to assume the existence of a closed symmetric linear relation $S\subseteq \mathcal{K}^{2}$ in order to define a GBR.
\item We do not set any specific assumptions either on $\dom\, \Gamma \subseteq \mathcal{K}^{2}$ or on $\ran\, \Gamma \subseteq \mathcal{H}^{2}$. 
\item We do not assume that the condition \cite[Definition 3.1.G2]{DHMS1} is satisfied, i.e. we do \textbf{not} assume that $\Gamma $ is a unitary relation.  
\end{itemize}

The Green's boundary relation $\Gamma$ is associated with two relations defined by
\begin{equation}
\label{eq220}
\Gamma_{0}\left( {\begin{array}{*{20}c}
f\\
f'\\
\end{array} } \right):=\left\{ \left( {\begin{array}{*{20}c}
h\\
0\\
\end{array} } \right):\left( {\begin{array}{*{20}c}
\hat{f}\\
\hat{h}\\
\end{array} } \right)\in \, \Gamma \right\}\, \wedge \, \Gamma_{1}\left( 
{\begin{array}{*{20}c}
f\\
f'\\
\end{array} } \right):=\left\{ \left( {\begin{array}{*{20}c}
0\\
h'\\
\end{array} } \right):\, \left( {\begin{array}{*{20}c}
\hat{f}\\
\hat{h}\\
\end{array} } \right)\in \, \Gamma \right\}.
\end{equation}
\begin{lemma}\label{lemma26} Let $\Gamma :\mathcal{K}^{2} \to \mathcal{H}^{2}$ be a Green's boundary relation, then
\begin{enumerate}[(i)]%, (i), (ii),...
\item $\ker \Gamma_{i} \subseteq {\left( \ker \Gamma_{i} \right)}^{+}, i=0, 1$,
\item $\ker \Gamma \subseteq \left( \ker \Gamma \right)^{+}$,
\item $\mul\, \Gamma_{i} \subseteq \left(\mul\,\Gamma_{i} \right)^{+}, i=0, 1$,
\item $\mul\, \Gamma \subseteq {\left(\mul\,\Gamma \right)}^{+}$.
\end{enumerate}
\end{lemma}

\textbf{Proof.} 

(i) If $\hat{f}, \hat{g}\in \ker \Gamma_{0}$, then 
$\left\lbrace \hat{f} , \left( {\begin{array}{*{20}c}
0\\
h'\\
\end{array} } \right)\right\rbrace  \in \Gamma $ and $\left\lbrace \hat{g} , \left( 
{\begin{array}{*{20}c}
0\\
k'\\
\end{array} } \right)\right\rbrace  \in \Gamma$. Assume that $\hat{f}\in \ker \Gamma_{0}$ is 
arbitrarily selected. Then for every $\hat{g}\in \ker \Gamma_{0}$, 
according to (\ref{eq212}), we have 
\[
\left( \left[ f',g \right]-\left[ f,g' \right]=0 \right)\Rightarrow \hat{f}\in \left( \ker \Gamma_{0} \right)^{+}.
\]
Hence, the relation $\ker \Gamma_{0}$ is symmetric. Similar reasoning can be repeated for $\ker \Gamma_{1}$.

Claims (ii), (iii) and (iv) follow also from (\ref{eq212}) directly. \hfill $\square$

\begin{proposition}\label{proposition28} Let $\Gamma :\mathcal{K}^{2} \to \mathcal{H}^{2}$ be a Green's boundary relation. Then:

\begin{enumerate}[(i)]%, (i), (ii),...
\item The condition 
\begin{equation}
\label{eq222}
\left( \dom\, \Gamma \right)^{+} \subseteq \overline{\dom}\, \Gamma 
\end{equation}
or, equivalently,
\[
\left( \left[ f',g \right]-\left[ f,g' \right]=0,\forall \hat{g}\in \dom\, \Gamma \right)\Rightarrow \hat{f}\in  \overline{\dom}\, \Gamma
\]
holds if and only if there exists a closed symmetric linear relation 
$S\subseteq \mathcal{K}^{2}$ such that $S^{+}=\overline{\dom}\, \Gamma $.
\item If 
\begin{equation}
\label{eq223}
\left( \dom\, \Gamma \right)^{+}\subseteq \dom\, \Gamma ,
\end{equation}
then $M=S$, where $M$ denotes the isotropic manifold of $dom \Gamma$, i.e.
\begin{equation}
\label{eq223.5}
M=\dom\, \Gamma \cap \left(\dom\, \Gamma \right)^{\left[ \bot \right]_{\mathcal{K}^{2}}}.
\end{equation}
\end{enumerate}
\end{proposition}

\textbf{Proof.} 

(i) Assume that $\Gamma :\mathcal{K}^{2} \to \mathcal{H}^{2}$ satisfies (\ref{eq222}). We define 
\begin{equation}
\label{eq224}
S:=\left(\dom\, \Gamma  \right)^{+}=\left( \dom\, \Gamma \right)^{\left[ \bot \right]_{\mathcal{K}^{2}}}.
\end{equation}
Then $ {S}^{+}=\overline{\dom}\, \Gamma$ and $S\subseteq {S\, }^{+}$. Indeed, according to (\ref{eq224}), $S^{+}= {\left(\dom\, \Gamma  \right)^{+}}^{+}=\overline{\dom}\, \Gamma$, and
\[
\hat{g}\in S\Rightarrow \left( \left[ f',g \right]-\left[ f,g' \right]=0, \forall \hat{f}\in \dom\, \Gamma \right) \Rightarrow \hat{g}\in \left(\dom\, \Gamma  \right)^{+}.
\]
Then from (\ref{eq222}) it follows
\[
\hat{g}\in \overline{\dom}\, \Gamma =S^{+}.
\]
Conversely, assume that $S$ is a closed symmetric relation and that its adjoint relation satisfies $S^{+}=\overline{\dom}\Gamma $. Then $S\subseteq S^{+}$ and $S=\left(\dom\, \Gamma \right)^{+}$. To prove $(\ref{eq222})$, assume 
\[
\left[ f',g \right]-\left[ f,g' \right]=\left[ \hat{f},\hat{g} \right]_{\mathcal{K}^{2}}=0,\forall \hat{g}\in \dom\, \Gamma .
\]
According to Lemma \ref{lemma22}, the scalar product $\left[ .,. \right]_{\mathcal{K}^{2}}$ is continuous with 
respect to the strong topology of the product Hilbert space $\left( \mathcal{K}^{2}, \left( .,. \right) \right)$. Thus $\left[ \hat{f},\hat{g} \right]_{\mathcal{K}^{2}}=0,\, \forall \hat{g}\in \overline{\dom}\,\Gamma $. That means 
\[
\hat{f}\in \left( \overline{\dom}\, \Gamma  \right)^{+}=S \subseteq S^{+}=\overline{\dom}\, \Gamma.
\]
This proves implication (\ref{eq222}).

(ii) From (\ref{eq223}), (\ref{eq223.5}) and (\ref{eq224}) we get 
\[
M=\left(\dom\, \Gamma \right)^{\left[ \bot \right]_{\mathcal{K}^{2}}}=S .
\] \hfill$\square$ 
\begin{definition}\label{definition28} The linear relation $S$ defined by (\ref{eq224}) is called a \textit{closed symmetric relation associated with the Green's boundary relation}.  
\end{definition}
\begin{definition}\label{definition210}The Green's boundary relation $\Gamma :\mathcal{K}^{2} \to \mathcal{H}^{2}$ that satisfies the maximality condition (\ref{eq222}) or equivalently,
\begin{equation}
\label{eq228}
\left(\dom\, \Gamma  \right)^{\left[ \bot \right]_{\mathcal{K}^{2}}}\subseteq  \overline{\dom}\, \Gamma,
\end{equation}
is called an isometric boundary relation.
\end{definition} 
Note, every Green's relation $\Gamma :\mathcal{K}^{2} \to \mathcal{H}^{2}$ is an isometric relation in the sense of (\ref{eq214}) but it is not an isometric \textbf{boundary} relation in the sense of this definition.
\begin{definition}\label{definition214} The Green's boundary relation $ \Gamma :\mathcal{K}^{2} \to \mathcal{H}^{2}$  that satisfies:
\[
\Gamma^{-1}=\Gamma^{\ast} 
\]
is called \textit{unitary boundary relation}.
\end{definition}
Later, in Proposition \ref{proposition510} we will prove that $\Gamma ^{-1}=\Gamma ^{\ast }$ implies (\ref{eq223.5}), which means that every unitary boundary relation has a corresponding symmetric relation $S$, and that it holds $S=M$.

We will sometimes use notation $N:=\ker \Gamma$ and $T:=\dom\, \Gamma$. Then $S=T^{+} $. Proposition \ref{proposition28} (i) shows us that the condition (\ref{eq222}) is essential because it makes the relation $S$ defined by (\ref{eq224}) a symmetric relation. Proposition \ref{proposition28} (ii) gives us a sufficient condition for $S=M$. Proposition \ref{proposition52} (i) will give us a sufficient condition for $M=N$. Corollary \ref{corollary56} will give us a couple of sufficient condition for $S=N=M$.

\begin{corollary}\label{corollary216} Let $\Gamma : \mathcal{K}^{2} \to \mathcal{H}^{2}$ be a Green's boundary relation. If there exists an extensions $A$ of $S:={\left(\dom\, \Gamma \right)}^{+}$ which satisfies $S\subseteq A\subseteq A^{+} \subseteq \dom\, \Gamma$, then both $\Gamma$ and $\tilde{\Gamma }:=\Gamma_{\vert A^{+}}$ are isometric boundary relations. 
\end{corollary}

\textbf{Proof.} We need to prove that both $\Gamma$ and $\tilde{\Gamma }$ satisfy (\ref{eq222}). From $S:={\left( \dom\, \Gamma \right)}^{+}$ and Proposition \ref{proposition28} (i) it follows that $\Gamma$ satisfies (\ref{eq222}).

Obviously $A^{+} \subseteq S^{+} $ and $\dom\, \tilde{\Gamma}=A^{+}$. Then 
\[
\left( \left[ f',g \right]-\left[ f,g' \right]=0,\forall \hat{g}\in \dom\, \tilde{\Gamma}=A^{+} \right)\Rightarrow \hat{f}\in {A^{+}}^{+}=\bar{A}\subseteq A^{+} = \overline{\dom}\, \tilde{\Gamma} .
\]\hfill  $\square$ 

\section{Krein space $\left( X\times Y, \left[ .,. \right]_{X\times Y} \right)$}\label{s6}
\textbf{3.1.} Let $\left( X, \left( .,. \right)_{X} \right)$ and $\left( Y, \left( .,. \right)_{Y} \right)$, $X \cap Y =\lbrace 0 \rbrace$ be the Hilbert spaces associated with the Krein space$\left( X, {[.,.]}_{X} \right)$ and $\left( Y, {[.,.]}_{Y} \right)$, respectively. This means 
\[
\left[ u_{x},v_{x} \right]_{X}=\left( J_{X}u_{x},v_{x} \right)_{X},\, \left[ u_{y},v_{y} \right]_{Y}=\left( J_{Y}u_{y},v_{y} \right)_{Y}, \forall \left( {\begin{array}{*{20}c}
u_{x}\\
u_{y}\\
\end{array} } \right),\left( {\begin{array}{*{20}c}
v_{x}\\
v_{y}\\
\end{array} } \right)\in X\times Y,
\]
where $J_{X}$ and $J_{Y}$ are fundamental symmetries of the respective Krein spaces $X$ and $Y$. The \textit{product Hilbert space} $\left( X\times Y,\left( .,. \right)_{X\times Y} \right)$ is defined as usually
\begin{equation}
\label{eq31}
\left( \left( {\begin{array}{*{20}c}
u_{x}\\
u_{y}\\
\end{array} } \right),\left( {\begin{array}{*{20}c}
v_{x}\\
v_{y}\\
\end{array} } \right) \right)_{X\times Y}:=\left( u_{x},v_{x} 
\right)_{X}+\left( u_{y},v_{y} \right)_{Y};\, \, \left( 
{\begin{array}{*{20}c}
u_{x}\\
u_{y}\\
\end{array} } \right),\left( {\begin{array}{*{20}c}
v_{x}\\
v_{y}\\
\end{array} } \right)\in X\times Y.
\end{equation}
We introduce the indefinite scalar product in the space $X\times Y$ by: 
\begin{equation}
\label{eq32}
\left[\left( {\begin{array}{*{20}c}
u_{x}\\
u_{y}\\
\end{array} } \right),\left( {\begin{array}{*{20}c}
v_{x}\\
v_{y}\\
\end{array} } \right) \right]_{X\times Y}:=\left[ u_{x},v_{x} 
\right]_{X}-\left[ u_{y},v_{y} \right]_{Y}; \left({\begin{array}{*{20}c}
u_{x}\\
u_{y}\\
\end{array} } \right),\left( {\begin{array}{*{20}c}
v_{x}\\
v_{y}\\
\end{array} } \right)\in X\times Y.
\end{equation} 

We introduced the Krien space $\left( X\times Y, \left[ .,. \right]_{X\times Y} \right)$ because it is useful in the following considerations and it has some nice properties, see propositions \ref{proposition38} and \ref{proposition310}. We use notation $X$ and $Y$ rather than $\mathcal{K}$ and $\mathcal{H}$ in order to avoid confusion because in further considerations we frequently substitute the Krein space $\mathcal{K}^{2}$ introduced by (\ref{eq28}) for $X$ and the Krein space $\mathcal{H}^{2}$ introduced by (\ref{eq210}) for $Y$. 

\begin{lemma}\label{lemma32} The space $\left( X\times Y,\left[ .,. \right]_{X\times Y} \right)$, where the scalar product is defined by (\ref{eq32}) is a Krein space with the fundamental symmetry 
\[
J_{X \times Y}:=\left( {\begin{array}{*{20}c}
J_{X} & 0\\
0 & -J_{Y}\\
\end{array} } \right)
\]
with respect to the product Hilbert space $\left( X\times Y,{(.,.)}_{X\times Y} \right)$ defined by (\ref{eq31}). 
\end{lemma}

\textbf{Proof.} It is easy to verify $\left[ \left( {\begin{array}{*{20}c}
u_{x}\\
0\\
\end{array} } \right),\left( {\begin{array}{*{20}c}
0\\
v_{y}\\
\end{array} } \right) \right]_{X\times Y}=0$. Therefore, we can consider 
\[
X\times Y=X\left[ + \right]Y,
\]
where $\left[ + \right]$ is the direct and orthogonal sum with respect to scalar product (\ref{eq32}). The lemma is now evident. \hfill$\square$

\begin{remark}\label{remark33} Let us observe that $\left( X\times Y,{[.,.]}_{X\times Y} \right)$ is still a Krein space, even if vector spaces $X$ and $Y$ satisfy $X=Y$. However, in that case some results may be very different then in the case $X \cap Y = \lbrace 0 \rbrace$, see Lemma \ref{lemma518} and Example \ref{example519}.
\end{remark}

\textbf{3.2.} We will need the following proposition. It can be found for Hilbert spaces as \cite[Proposition 1.3.2]{BHS}. Some of the statements have been proven for Krein spaces in \cite[(2.7)]{DHMS1} by means of the corresponding statements for Hilbert spaces and fundamental symmetries. See also \cite[Lemma 2.1]{W} and papers cited there. For convenience of the reader we will prove the proposition for Krein spaces. We will do it by means of the space $\left( X\times Y, \left[ .,. \right]_{X\times Y} \right)$ to show how that Krein space can be used.

As usually $\bar{H}$ denotes the closure of $H\subseteq X\times Y$ with respect to product Hilbert space $X \times Y$. 

\begin{proposition}\label{proposition34} Let $\left( X, {[.,.]}_{X} \right)$ 
and $\left( Y, {[.,.]}_{Y} \right)$ be Krein spaces, and let $H:X\to Y$ 
be a linear relation. Let $H^{\ast }:Y \to X$ be the adjoint linear 
\textbf{relation} of $H$. Then 

\begin{enumerate}[(i)]%, (i), (ii),...
\item $H^{\ast }$ is a closed linear relation,
\item $\left( \bar{H} \right)^{\ast }=H^{\ast }$, 
\item $\bar{H}={H^{\ast }}^{\ast }$,
\item $\left( \dom\, H\, \right)^{\left[ \bot \right]_{X}}=\mul\, H^{\ast }$, 
\item $\ker\, H^{\ast }=\left( \ran\, H\, \right)^{\left[ \bot \right]_{Y}}$. 
%\item $\left( \ker H^{\ast }\, \right)^{\left[ \bot \right]_{Y}}=\overline{\ran}\, H.$
\end{enumerate}
\end{proposition}

\textbf{Proof}.

(i) By definition it holds: 
\[
\left( {\begin{array}{*{20}c}
g_{y}\\
g_{x}\\
\end{array} } \right)\in H^{\ast }\Longleftrightarrow \left[ f_{x},g_{x} 
\right]_{X}-\left[ f_{y},g_{y} \right]_{Y}=0, \forall \left( 
{\begin{array}{*{20}c}
f_{x}\\
f_{y}\\
\end{array} } \right)\in H \Longleftrightarrow
\left( 
{\begin{array}{*{20}c}
g_{x}\\
g_{y}\\
\end{array} } \right)\in H^{\left[ \bot \right]_{X\times Y}} 
\]
\begin{equation}
\label{eq36}
\Longleftrightarrow H^{\ast }={ \left( H^{\left[ \bot \right]_{X\times Y}} 
\right)}^{-1}.
\end{equation}
Since the orthogonal companion $H^{\left[ \bot \right]_{X\times Y}}$is closed $H^{\ast }$ is also closed. 

(ii) Let us now select any $\left( {\begin{array}{*{20}c}
g_{y}\\
g_{x}\\
\end{array} } \right)\in H^{\ast }$. Because of the continuity of scalar 
product (\ref{eq32}), the relation $
\left[ f_{x},g_{x} \right]_{X}-\left[ f_{y},g_{y} \right]_{Y}=0,\, 
\forall \left( {\begin{array}{*{20}c}
f_{x}\\
f_{y}\\
\end{array} } \right)\in H
$ extends to all $\left( {\begin{array}{*{20}c}
f_{x}\\
f_{y}\\
\end{array} } \right)\in \bar{H}$. Hence, $H^{\ast }\subseteq \left( \bar{H} 
\right)^{\ast }$. This together with the obvious $\left( \bar{H} \right)^{\ast 
}\subseteq H^{\ast }$ gives $\left( \bar{H} 
\right)^{\ast }=H^{\ast }$.

(iii) For a linear relation $T\subseteq Y\times X$ one can easily verify
\begin{equation}
\label{eq38}
\left( T^{-1} \right)^{\left[ \bot \right]_{X\times Y}}=\left( T^{\left[ 
\bot \right]_{Y\times X}} \right)^{-1}.
\end{equation}
If we substitute $H^{\ast }\subseteq Y\times X$ for $H \subseteq X \times Y$ into (\ref{eq36}), we obtain
\[
{H^{\ast }}^{\ast }={\, \left( \left( H^{\ast } \right)^{\left[ \bot 
\right]_{Y\times X}} \right)}^{-1}.
\]
If we substitute $H^{\ast }$ for $T$ in (\ref{eq38}) and observe that the 
right sides are equal we get
\[
{H^{\ast }}^{\ast }=\left( \left( H^{\ast } \right)^{-1} \right)^{\left[ \bot 
\right]_{X\times Y}}.
\]
According to (\ref{eq36}) it holds $\left( H^{\ast } \right)^{-1}=H^{\left[ 
\bot \right]_{X\times Y}}$. Finally we get 
\[
{H^{\ast }}^{\ast }=\left( H^{\left[ \bot \right]_{X\times Y}} \right)^{\left[ 
\bot \right]_{X\times Y}}.
\]
Because, $\left( X\times Y,\left[ .,. \right]_{X\times Y} \right)$ is a 
Krein space, all conditions of \cite[p.68, Theorem 6.1]{Bog} are satisfied. Therefore $\left( H^{\left[ \bot \right]_{X\times Y}} \right)^{\left[ \bot \right]_{X\times 
Y}}=\bar{H}$. This proves ${H^{\ast }}^{\ast }=\bar{H}$.

(iv)  
\[
f'\in \left( \dom\, H\, \right)^{\left[ \bot \right]_{X}}\Longleftrightarrow
\left[ h,f' \right]_{X}-\left[ h',0 \right]_{Y}=0,\, \, \forall \left( 
{\begin{array}{*{20}c}
h\\
h'\\
\end{array} } \right)\in H\, \Longleftrightarrow \left( {\begin{array}{*{20}c}
0\\
f'\\
\end{array} } \right)\in H^{\ast}.
\]

(v) $\left( {\begin{array}{*{20}c}
h\\
0\\
\end{array} } \right)\in H^{\ast } \Longleftrightarrow \left[ h,f' 
\right]=0, \forall \left( {\begin{array}{*{20}c}
f\\
f'\\
\end{array} } \right)\in H \Longleftrightarrow h\in \left( \ran\, H\, 
\right)^{\left[ \bot \right]_{Y}}$. \hfill  $\square $
\\

\textbf{3.3.} In the following two propositions we will characterize isometric and symmetric linear relations in terms of the Krein spaces defined on the vector space $\mathcal{K}\times \mathcal{K}$ by different scalar products (\ref{eq32}) and (\ref{eq28}).
 
\begin{proposition}\label{proposition38}. Let $\left( X,\, {[.,.]}_{X} \right)$ and $\left( Y,\, {[.,.]}_{Y} \right)$ be Krein spaces. Then $V\subseteq X\times Y$ is a neutral manifold (hyper-maximal neutral subspace) in the Krein space $\left( X\times Y, \left[ .,. \right]_{X\times Y} \right)$ if and only if $V$ is an isometric (unitary) linear relation. Symbolically,
\[
V \subseteq V^{\left[ \bot\right]_{X \times Y}} \Longleftrightarrow V^{-1} \subseteq V^{*} \left(V=V^{\left[ \bot\right]_{X \times Y}} \Longleftrightarrow V^{-1} = V^{*} \right).
\]
\end{proposition}

\textbf{Proof}. Assume that $V$ is an isometric relation, i.e. $V^{-1}\subseteq V^{*}$. According to (\ref{eq36}), this is equivalent to $V^{-1} \subseteq { \left( V^{\left[ \bot \right]_{X\times Y}} \right)}^{-1}$. This is further equivalent to $V \subseteq V^{\left[ \bot\right]_{X \times Y}}$. By definition this means that $V\subseteq X\times Y$ is an neutral subspace. 

The statement for the unitary relation $V$ follows from this if we replace "$\subseteq$" by "$=$" and "neutral manifold" by "hyper-maximal neutral subspace". \hfill $\square$

The following proposition is slight generalization of some statements from \cite[Proposition 2.9]{DHMS1}. In the proof we again use use the Krein space defined by (\ref{eq32}).
  
\begin{proposition}\label{proposition310} Let $(\mathcal{K}, \left[ .,. \right])$ be a Krein space and let $(\mathcal{K}, \left( .,. \right))$ be the corresponding Hilbert space with the fundamental symmetry $G$, and $(\mathcal{K}^{2}, \left[ .,. \right]_{\mathcal{K}^{2}})$ and $(\mathcal{K}^{2}, \left( .,. \right)_{\mathcal{K}^{2}})$ are spaces as in Lemma \ref{lemma22}. In addition, let $\left( \mathcal{K}\times \mathcal{K}, \left[ .,. \right]_{\mathcal{K}\times \mathcal{K}} \right)$ be the Krein space defined by the scalar product (\ref{eq32}) with $X=Y=\mathcal{K}$.
Then it holds
\begin{enumerate}[(i)]%, (i), (ii),...
\item 
\begin{equation}
\label{eq318}
\left[ \hat{f},\hat{g} \right]_{\mathcal{K}^{2}}= \left[ \left( {\begin{array}{*{20}c}
0 & -i\\
-i & 0\\
\end{array} } \right)\hat{f},\hat{g} \right]_{\mathcal{K}\times \mathcal{K}}, \, \hat{f}, \hat{g} \in\mathcal{K}\times \mathcal{K}.
\end{equation} 
\item A liner relation $A $ in $ (\mathcal{K}, \left[ .,. \right])$ is symmetric if and only if it is neutral manifold in $(\mathcal{K}^{2}, \left[ .,. \right]_{\mathcal{K}^{2}})$, if and only if $A^{-1} \subseteq A^{\left[ \bot \right]_{\mathcal{K}\times \mathcal{K}}}$.
\item A liner relation $A $ in $ (\mathcal{K}, \left[ .,. \right])$ is self-adjoint if and only if it is a hyper-maximal neutral subspace in $(\mathcal{K}^{2}, \left[ .,. \right]_{\mathcal{K}^{2}})$. Symbolically 
\[
A = A^{+} \Longleftrightarrow A=A^{\left[ \bot \right]_{\mathcal{K}^{2}}} \left( \Longleftrightarrow A^{-1} = A^{\left[ \bot \right]_{\mathcal{K}\times \mathcal{K}}} \right).
\]
\end{enumerate}
\end{proposition}

\textbf{Proof.} 

(i) According to Lemma \ref{lemma32}, the fundamental symmetry of $\mathcal{K} \times \mathcal{K}$ is
\[
J_{\mathcal{K} \times \mathcal{K}}:=\left( {\begin{array}{*{20}c}
G & 0\\
0 & -G\\
\end{array} } \right).
\]
According to Lemma \ref{lemma22}, the fundamental symmetry of $\mathcal{K}^{2}$ is  
\[
\mathcal{G}=\left( {\begin{array}{*{20}c}
0 & -iG\\
iG & 0\\
\end{array} } \right).
\]
The identity (\ref{eq318}) now follows from
\[
\mathcal{G}= J_{\mathcal{K} \times \mathcal{K}}\left( {\begin{array}{*{20}c}
0 & -i\\
-i & 0\\
\end{array} } \right).
\] 

(ii) According to (\ref{eq318}) 
\[
\left({\begin{array}{*{20}c}
f\\
f' \\
\end{array} } \right)  \left[ \bot \right]_{\mathcal{K}^{2}} \left({\begin{array}{*{20}c}
g\\
g' \\
\end{array} } \right) \Longleftrightarrow \left({\begin{array}{*{20}c}
f\\
f' \\
\end{array} } \right)  \left[ \bot \right]_{\mathcal{K}\times \mathcal{K}} \left({\begin{array}{*{20}c}
g'\\
g \\
\end{array} } \right).
\]
Therefore, $A \subseteq A^{+} = A^{\left[ \bot \right]_{\mathcal{K}^{2}}} \Longleftrightarrow A^{-1} \subseteq A^{\left[ \bot \right]_{\mathcal{K}^\times \mathcal{K}}}$.

The statement (iii) follows from (ii) when we substitute "$=$" for "$\subseteq$". \hfill $\square $ 

Note that according to previous two propositions, isometric relations are characterized as neutral in $\left( \mathcal{K} \times \mathcal{K}, \left[ .,.\right]_{\mathcal{K} \times \mathcal{K}} \right) $ and symmetric relations are characterized as neutral in $\left( \mathcal{K}^{2},\left[ .,.\right]_{\mathcal{K}^{2}} \right)$. Therefore, the space $\left( \mathcal{K} \times \mathcal{K}, \left[ .,.\right]_{\mathcal{K} \times \mathcal{K}} \right) $ plays the same role for isometric relations in $\mathcal{K}$ as the space $\left( \mathcal{K}^{2},\left[ .,.\right]_{\mathcal{K}^{2}} \right)$ plays for symmetric relations in $\mathcal{K}$.

\section{Isometric relations in Krein spaces}\label{s8}

The definition of a unitary operator ($V^{-1}=V^{\ast }\, \wedge \mul\, V=\left\{ 0 \right\}$ ) in \cite[p. 76]{BHS} is different from the definition in \cite[p. 128]{Bog}, where the unitary operator is defined as an isometric operator $V:X\to Y$ with $\dom\, V=X\, \wedge \ran\, V=Y$. The latter operator is a completely invertible isometric, and by many authors it is called \textit{standard unitary operator}. 

In the following lemma, we collect all statements about isometric relation $V:X\to Y$, where $X$ and $Y$ are Krein spaces, which we will need in this study of Green's boundary relations. Note we will denote by $M$ the \textit{isotropic manifold} of $\dom\, V$, i.e. $M= \dom\, V \cap \left( \dom\, V \right)^{[\bot]}$. By the same token $\tilde{M}:= \ran\, V \cap \left( \ran\, V \right)^{[\bot]}$.

\begin{proposition}\label{proposition42} Let $\left( X, {[.,.]}_{X} \right)$ and $\left( Y, {[.,.]}_{Y} \right)$ be Krein spaces, let $V:X\to Y$ be an isometric relation. Then:

\begin{enumerate}[(i)]%, (i), (ii),...
\item If $\ran\, V$ does not degenerate, then for the isotropic manifold $M$ it holds $M=\ker V$. 
\item If $\dom\, V \subseteq X$ is closed and $\overline{\ran}\, V=Y$, then $V$ is a bounded operator. 
\item If $\ran\, V$ is closed and $\overline{\dom}\, V=X$, then $V^{-1}$ is a bounded operator.
\item If $\overline{\dom}\, V=X$ and $\ran \, V=Y$, then $V$ is a unitary operator.
\item If $\overline{\ran}\, V=Y$ and $\dom\, V=X$, then $V$ is a unitary operator.
\end{enumerate}
\end{proposition}

\textbf{Proof.}

(i) Assume that $\ran \, V$ is non-degenerate and $f\in M $. It holds
\begin{equation}
\label{eq42}
\left( f\in M \wedge \left( {\begin{array}{*{20}c}
f\\
f'\\
\end{array} } \right)\in V \right) \Longleftrightarrow \left( \left[ f,g \right]=\left[ f',g' 
\right]=0,\, \forall \left( {\begin{array}{*{20}c}
g\\
g'\\
\end{array} } \right)\in V \right).
\end{equation}
\[
(\ref{eq42}) \Rightarrow f'=0  \Rightarrow f \in \ker V \Rightarrow M \subseteq \ker V. 
\]

We assume now $\ran \, V$ is non-degenerate $f\in \ker V $. Then 
\[
(\ref{eq42}) \Rightarrow f'=0 \Rightarrow f \in M \Rightarrow \ker V \subseteq M. 
\]
This proves $M = \ker V$. 

(ii) If $\dom\, V$ is closed and $\ran\, V$ is non-degenerate, then obviously $M=\ker V$ is also closed. 

Because $V$ is isometric, it satisfies $V^{-1}\subseteq V^{\ast }$. This and $\overline{\ran} \, V=Y$ imply $\overline{\dom}\, V^{\ast }=Y$. Then, according to Proposition \ref{proposition34} (iv) we have $\mul\, \bar{V}=\left(\dom\, V^{\ast } \right)^{{[\bot ]}_{Y}}=\left\{ 0 \right\}$. Hence, $\bar{V}$ and $V$ are operators. 

Let us prove that $V$ is closed. Assume
\[
\left( {\begin{array}{*{20}c}
f_{n}\\
f_{n}'\\
\end{array} } \right)\in V\, \wedge \left( {\begin{array}{*{20}c}
f_{n}\\
f_{n}'\\
\end{array} } \right)\to \left( {\begin{array}{*{20}c}
f\\
f'\\
\end{array} } \right)\in \bar{V}\, \, (n\to \infty ).
\]
Because $\dom\, V$ is closed, $f\in \dom\, V$. Therefore, there exists $g'$ such that $\left( {\begin{array}{*{20}c}
f\\
g'\\
\end{array} } \right)\in V$. Because $V$ is single-valued, it must be $\left( {\begin{array}{*{20}c}
f\\
g'\\
\end{array} } \right)=\left( {\begin{array}{*{20}c}
f\\
f'\\
\end{array} } \right)\in V$. This proves that $V$ is closed. The continuity of $V$ follows from the closed graph theorem. 

(iii) Because $V^{-1}$ is also isometric we can substitute $V^{-1}$ for $V$ into (i) in order to prove (ii).

(iv) Assume $\overline{\dom}\, V=X$ and $V$ is a surjective relation. Let us prove $V^{\ast }\subseteq V^{-1}$. 

Let us arbitrarily select $\left( {\begin{array}{*{20}c}
g'\\
g_{0}\\
\end{array} } \right)\in V^{\ast }\subseteq Y\times X$. Because $V^{-1}\subseteq V^{\ast }$, for $g'\in Y=\ran\, V$ there exists $g\in \dom\, V$ such that it holds $\left( {\begin{array}{*{20}c}
g\\
g'\\
\end{array} } \right)\in V$. We need to prove $\left( {\begin{array}{*{20}c}
g'\\
g_{0}\\
\end{array} } \right)\in V^{-1}$. Indeed, via definitions of isometric $V$ and adjoint $V^{\ast }$, respectively, we get 
\[
\left[ f,g \right]_{X}=\left[ f',g' \right]_{Y}=\left[ f,g_{0} \right]_{X}, \forall \left( {\begin{array}{*{20}c}
f\\
f'\\
\end{array} } \right)\in V.
\]
Because $\dom\, V$ is dense in $X$, we conclude $g_{0}=g$. Hence, $\left( 
{\begin{array}{*{20}c}
g'\\
g_{0}\\
\end{array} } \right)=\left( {\begin{array}{*{20}c}
g'\\
g\\
\end{array} } \right)\in V^{-1}$. Together with $V^{-1}\subseteq V^{\ast }$ this gives $V^{-1}=V^{\ast }$. Hence, $V$ is a closed relation, and, according to Proposition \ref{proposition34} (v), $\ker V^{\ast }=\mul \, V=\left\{ 0 \right\}$, i.e. $V$ is an operator. 

(v) Substitute $V^{-1}$ for $V$ into (iii) in order to prove (iv).
\hfill $\square$

Note that Proposition \ref{proposition42} (ii) is a generalization of \cite[Lemma 1.8.1]{BHS} because here we deal with arbitrary Krein spaces $X$ and $Y$ and we deal with a linear isometric relation $V$ with $\overline{\ran}\, V=Y$ instead of a surjective operator. Recall, a similar research for unitary relations has been done in \cite[Corollary 4.2]{DHMS1}.

Note also that in order to claim continuity of the operator $V$ when $X$ and $Y$ are Krein spaces, we had to assume $\overline{\ran}\, V=Y$ and $\dom\, V$ is closed, in Proposition \ref{proposition42} (ii). A more general statement holds when $X$ and $Y$ are Pontryagin spaces, see \cite[Lemma 2.1]{B3}.

\begin{theorem}\label{theorem46} Let $\left( X,{[.,.]}_{X} \right)$ and $\left( Y, {[.,.]}_{Y} \right)$ be Krein spaces and let $V:X\to Y$ be an isometric relation. 
\begin{enumerate}[(i)]%, (i), (ii),...
\item $\tilde{M} = \mul\, V \Leftrightarrow M=\ker V$.
\item If $V$ is a unitary relation, then 
\begin{equation}
\label{eq43}
(\ran\, V)^{[\bot]_{Y}}=\mul\, V \, \wedge \, \left( \dom\, V \right)^{[\bot]_{X}}=\ker V.  
\end{equation}
Conversely, 
\[
\left( (\ran\, V)^{[\bot]_{Y}}=\mul\, V \wedge\left( \dom\, V \right)^{[\bot]_{X}}=\ker V \right) \Rightarrow V^{*}=\bar{V}^{-1}.
\]
Hence, if $V$ is closed, we have a characterization of unitary relations in Krein spaces. 
\item If $V:X\to Y$ is an isometric relation and either $\left( \overline{\dom}\, V=X \wedge \ran \, V=Y\right)$ or $\left( \overline{\ran}\, V=Y \wedge \dom\, V=X\right) $, then $V$ is a completely invertible isometric operator, i.e. a standard unitary operator.
\end{enumerate}
\end{theorem}

\textbf{Proof.} 

(i)  Let us prove $\tilde{M} = \mul\, V  \Rightarrow \ker V = M$. 

Assume $\tilde{M} = \mul\, V $ and $f\in M $. It holds
\begin{equation}
\label{eq44}
\left( f\in M \wedge \left( {\begin{array}{*{20}c}
f\\
f'\\
\end{array} } \right)\in V \right) \Longleftrightarrow \left( \left[ f,g \right]=\left[ f',g' 
\right]=0,\, \forall \left( {\begin{array}{*{20}c}
g\\
g'\\
\end{array} } \right)\in V \right).
\end{equation}
\[
(\ref{eq44}) \Rightarrow \left( f'=0 \vee 0 \neq f'\in \mul\, V\right)  \Rightarrow f \in \ker V \Rightarrow M \subseteq \ker V. 
\]
Assume now $\tilde{M} = \mul\, V $ and $f\in \ker V $. Then 
\[
(\ref{eq44}) \Rightarrow f'=0 \Rightarrow f \in M \Rightarrow \ker V \subseteq M. 
\]
This proves: $\tilde{M} = \mul\, V  \Rightarrow M=\ker V $. 

The converse $M = \ker V \Rightarrow \tilde{M}=\mul\, V$ follows immediately if we substitute the isometric $V^{-1}$ for the isometric $V$ in the previous implication. This completes the proof of (i). 

(ii) We assume $V^{-1}=V^{\ast }$. According to Proposition \ref{proposition34} (iv) it holds $(\dom\, V^{*})^{[\bot]_{Y}}=\mul\, V$. Therefore,  $(\ran\, V)^{[\bot]_{Y}}=\mul\, V$. 

The identity $\left( \dom\, V \right)^{[\bot]_{X}}=\ker V $ follows when we apply the previous identity on the unitary relation $V^{-1}$.

This completes the proof of the first part of (ii). 

Conversely, by definition, it holds
\[
\left( {\begin{array}{*{20}c}
f\\
f'\\
\end{array} } \right)\in V^{*} \Rightarrow \left[ f,g' \right]_{Y}=\left[ f',g \right]_{X}, \forall \left( {\begin{array}{*{20}c}
g\\
g'\\
\end{array} } \right)\in V .
\]
From this and from 
\[
\left( \lbrace 0 \rbrace \times \mul\, V \subseteq V \, \wedge \, \ker V \times \lbrace 0 \rbrace \subseteq V\right) 
\]
it follows
\[
\left( f [ \bot] \mul\, V \, \wedge \, f' [ \bot] \ker\, V \right) \Rightarrow \left( f \in \overline{\ran}\, V \, \wedge \, f' \in \overline{\dom}\, V \right) 
\]
\[
\Rightarrow \exists \left\lbrace \left( {\begin{array}{*{20}c}
f_{n}'\\
f_{n}\\
\end{array} } \right)\in V :  n \in \mathbb{N} \right\rbrace \wedge \left( {\begin{array}{*{20}c}
f_{n}'\\
f_{n}\\
\end{array} } \right)\rightarrow \left( {\begin{array}{*{20}c}
f'\\
f\\ \end{array} } \right) \Rightarrow \left( {\begin{array}{*{20}c}
f'\\
f\\ \end{array} } \right) \in \bar{V}.
\]
Therefore, $V^{*} \subseteq \bar{V}^{-1}$.

(ii) This statement follows from Proposition \ref{proposition42} (iv) and (v). \hfill $\square$

Note that the identities (\ref{eq43}) have been proven in \cite{DHMS1} by means of the same identities for Hilbert spaces and fundamental symmetries. We state them here to have a complete characterization of unitary relations in the Krein spaces in Theorem \ref{theorem46} (ii). In addition, we gave a direct Krein space proof.

\section{Properties of the Green's boundary relation }\label{s10}

\textbf{5.1}. We proved the general results of Section \ref{s8} in order to obtain the results about Green's boundary relation in this section. 

\begin{proposition}\label{proposition52} Let $\Gamma : \mathcal{K}^{2}\to \mathcal{H}^{2}$ be a Green's boundary relation. Then:

\begin{enumerate}[(i)]%, (i), (ii),...
\item If $\ran\, \Gamma $ does not degenerate, then for the isotropic manifold $M$ it holds $M=\ker \Gamma $. 
\item If $\dom\, \Gamma \subseteq \mathcal{K}^{2}$ is closed and $\overline{\ran}\, \Gamma=\mathcal{H}^{2}$, then $\Gamma$ is a bounded operator. 
\item If $\ran\, \Gamma $ is closed and $\overline{\dom}\, \Gamma =\mathcal{K}^{2}$, then $\Gamma^{-1}$ is a bounded operator.
\item If $\overline{\dom}\, \Gamma=\mathcal{K}^{2}$ and $\ran \, \Gamma=\mathcal{H}^{2}$, then $\Gamma$ is a unitary operator.
\item If $\overline{\ran}\, \Gamma=\mathcal{H}^{2}$ and $\dom\, \Gamma=\mathcal{K}^{2}$, then $\Gamma$ is a unitary operator.
\end{enumerate}
\end{proposition}

\textbf{Proof.} Recall that the indefinite scalar products (\ref{eq28}) and (\ref{eq210}) define the Krein spaces $\left( \mathcal{K}^{2}, \left[ .,. \right]_{\mathcal{K}^{2}} \right)$, $(\mathcal{H}^{2}, \left[ .,. \right]_{\mathcal{H}^{2}})$, respectively. Since $\Gamma : \mathcal{K}^{2}\to \mathcal{H}^{2}$ satisfies (\ref{eq214}), $\Gamma$ is an isometric relation between those Krein spaces. We can substitute $\left( \mathcal{K}^{2},\, \left[ .,. \right]_{\mathcal{K}^{2}} \right)$, $(\mathcal{H}^{2},\, \left[ .,. \right]_{\mathcal{H}^{2}})$ and $\Gamma $ for $\left( X, {[.,.]}_{X} \right)$, $\left( Y,, {[.,.]}_{Y} \right)$ and $V$, respectively, in Proposition \ref{proposition42}. Also observe that the isotropic part of $\dom\, \Gamma $ in $\left( \mathcal{K}^{2},\, \left[ .,. \right]_{\mathcal{K}^{2}} \right)$ is $M=\dom\, \Gamma \cap \left(\dom\, \Gamma \right)^{+}$. Then statement (i)-(v) are consequences of the corresponding statements in Proposition \ref{proposition42}.\hfill $\square$

Note, Proposition \ref{proposition52} (ii) is a generalization of \cite[Lemma 1.8.1]{BHS} because here we allow $\mathcal{K}$ to be a Krein space, rather than a Hilbert space, and $\Gamma $ can be a linear relation with $\overline{\\ran}\, \Gamma= \mathcal{H}^{2}$ rather than a surjective operator.

\begin{theorem}\label{theorem55} Let $\Gamma : \mathcal{K}^{2}\to \mathcal{H}^{2}$ be a Green's boundary relation, not necessarily unitary.
\begin{enumerate}[(i)]%, (i), (ii),...
\item $\tilde{M} = \mul\, \Gamma \Leftrightarrow M=\ker \Gamma$.
\item If $\Gamma $ is a unitary relation, then 
\begin{equation}
\label{eq51}
(\ran\, \Gamma)^{[\bot]_{\mathcal{H}^{2}}}=\mul\, \Gamma \, \wedge \, \left( \dom\, \Gamma \right)^{[\bot]_{\mathcal{K}^{2}}}=\ker \Gamma.  
\end{equation}
Conversely, 
\[
\left( (\ran\, \Gamma)^{[\bot]_{\mathcal{H}^{2}}}=\mul\, \Gamma \wedge\left( \dom\, \Gamma \right)^{[\bot]_{\mathcal{K}^{2}}}=\ker \Gamma \right) \Rightarrow \Gamma^{*}=\bar{\Gamma}^{-1}.
\]
Therefore, $\Gamma$ is a unitary boundary relation if and only if $\Gamma$ is closed and conditions (\ref{eq51}) are satisfied. 
\item If $\Gamma: \mathcal{K}^{2} \to \mathcal{H}^{2}$ is an isometric relation and either $\left( \overline{\dom}\, \Gamma=\mathcal{K}^{2} \wedge \ran \, \Gamma=Y\right)$ or $\left( \overline{\ran}\, \Gamma=\mathcal{H}^{2} \wedge \dom\, \Gamma=\mathcal{K}^{2}\right) $, then $\Gamma$ is a completely invertible isometric operator, i.e. a standard unitary operator.
\end{enumerate}
\end{theorem}

\textbf{Proof.} Both statements are consequences of Theorem \ref{theorem46}. \hfill $\square$

Note, statements (i), (ii) and (iii) of Proposition \ref{proposition52} can be proved by means of Theorem \ref{theorem55} (i) and (ii). We decided to give the presented proof of Proposition \ref{proposition52} because it is direct, based on definitions.

\begin{corollary}\label{corollary56} Let $\Gamma :\mathcal{K}^{2}\to \mathcal{H}^{2}$ be a Green's boundary relation. If $\ran\, \Gamma$ is non-degenerate and if condition (\ref{eq223}) is satisfied, then $M=N=S$.
\end{corollary}

\textbf{Proof.} According to Proposition \ref{proposition28}, (ii) $M=S$ and according to Proposition \ref{proposition52} (i) $N=M$. \hfill $\square$ 

\begin{corollary}\label{corollary58} Let $\Gamma :\mathcal{K}^{2} \to \mathcal{H}^{2}$ be a Green's boundary relation. Then, $\dom\, \Gamma$ is closed, $\ran\, \Gamma=\mathcal{H}^{2}$ and (\ref{eq228}) holds if and only if $\Pi=(\mathcal{H}, \Gamma_{0}, \Gamma_{1})$ is an ordinary boundary triple for $S^{+}$.
\end{corollary}

\textbf{Proof.} $\dom\, \Gamma$ is closed and (\ref{eq228}) imply that (\ref{eq223}) holds. According to Corollary \ref{corollary56}, there exists a closed symmetric relation $S=N=\ker \Gamma$. According to Proposition \ref{proposition52} (ii), $\Gamma$ is a continuous operator that satisfies (\ref{eq214}). Because, it is also surjective, the triple $\Pi=(\mathcal{H}, \Gamma_{0}, \Gamma_{1})$ is ordinary.

Conversely, if the triple $\Pi=(\mathcal{H}, \Gamma_{0}, \Gamma_{1})$ is an ordinary boundary triple for $S^{+}$, then $\dom\, \Gamma = S^{+}$ is closed, $\ran\, \Gamma =\mathcal{H}^{2}$. \hfill $\square$ 

\begin{proposition}\label{proposition510}. Let $\Gamma : \mathcal{K}^{2} \rightarrow  \mathcal{H}^{2}$ be a 
Green's boundary relation. Then
\begin{enumerate}[(i)]%, (i), (ii),...
\item$\Gamma $ is a unitary relation if and only if $\Gamma $ satisfies the condition
\begin{equation}
\label{eq54}
\left( \left[ \hat{f},\hat{g} \right]_{\mathcal{K}^{2}}=\left[ \hat{h},\hat{k} 
\right]_{\mathcal{H}^{2}}, \forall \left( {\begin{array}{*{20}c}
\hat{f}\\
\hat{h}\\
\end{array} } \right)\in \Gamma \right)\Rightarrow \left( 
{\begin{array}{*{20}c}
\hat{g}\\
\hat{k}\\
\end{array} } \right)\in \Gamma .
\end{equation}
\item (\ref{eq54}) $\Rightarrow $ (\ref{eq223}) $\Rightarrow $ (\ref{eq222}).
\item If $\Gamma$ is a unitary boundary relation, then $M=N=S$ and  $\tilde{M}=\tilde{N}=\tilde{S}=\left( \ran\, \Gamma \right)^{+}$, where $\tilde{M}$ is the isotropic manifold of $\ran\, \Gamma$, $\tilde{N}:= \mul\, \Gamma$, and $\tilde{S}:= \left( \ran\, \Gamma\right)^{+}$.
\end{enumerate}
\end{proposition}

\textbf{Proof.} 

(i) If we substitute $\mathcal{K}^{2}$ for $X$, $\mathcal{H}^{2}$ for $Y$, and $\Gamma$ for $V$, then $ \left(  X\times Y, \left[ .,. \right]_{X \times Y} \right) = \left(  \mathcal{K}^{2}\times \mathcal{H}^{2}, \left[ .,. \right]_{\mathcal{K}^{2}\times \mathcal{H}^{2}} \right) $. The above condition (\ref{eq54}) is equivalent with  $\Gamma ^{\left[ \perp \right]_{\mathcal{K}^{2}\times \mathcal{H}^{2}}} \subseteq \Gamma $. The converse implication $\Gamma  \subseteq \Gamma ^{\left[ \perp \right]_{\mathcal{K}^{2}\times \mathcal{H}^{2}}} $ holds according to Proposition \ref{proposition38} for $\Gamma=V$.  Then $\Gamma ^{\left[ \perp \right]_{\mathcal{K}^{2}\times \mathcal{H}^{2}}} = \Gamma $ means that $\Gamma $ is a hyper-maximal neutral relation in $ \left(\mathcal{K}^{2}\times \mathcal{H}^{2}, \left[ .,. \right]_{\mathcal{K}^{2}\times \mathcal{H}^{2}} \right)$. Now the claim follows from unitary part of Proposition \ref{proposition38}.

(ii) Assume that (\ref{eq54}) holds. Then
\[
\left( \left[ f',g \right]-\left[ f,g' \right]=0, \forall \hat{f}\in 
\dom\, \Gamma \right)\Rightarrow \left( \left[ \hat{f},\hat{g} 
\right]_{\mathcal{K}^{2}}=\left[ \hat{h},\hat{k} \right]_{\mathcal{H}^{2}}=0, \forall \left( 
{\begin{array}{*{20}c}
\hat{f}\\
\hat{h}\\
\end{array} } \right)\in \Gamma \right).
\]
According to (\ref{eq54}), $\left( {\begin{array}{*{20}c}
\hat{g}\\
\hat{k}\\
\end{array} } \right)\in \Gamma $. This means $\hat{g}\in \dom\, \Gamma$. Hence, (\ref{eq223}) holds.  The implication (\ref{eq223}) $\Rightarrow$ (\ref{eq222}) is obvious. This proves (ii).

(iii) We assume that $\Gamma$ is a unitary Green's boundary relation. According to (i), (\ref{eq54}) holds, and according to (ii), (\ref{eq223}) holds for $\Gamma$. Then the it holds $M=S$. From (\ref{eq51}) it follows $S=N$. This proves the first claim of (iii).

The claim $\tilde{M}=\tilde{N}=\tilde{S}$ follows similarly using $\Gamma^{-1}$ instead of $\Gamma$. \hfill $\square$

\textbf{5.2.} In this sub-section we will strengthen \cite[Proposition 2.10]{DHMS1}. In order to do that we will need the concept of trivial isometric (unitary) relation. We say that the isometric (unitary) relation $V \subseteq  X \times Y$ is \textit{trivial} if $V := \ker V \times \mul\, V $. In that case $\ker V \wedge \mul\, V$ are neutral (hyper-maximal neutral) manifolds. In Proposition \ref{proposition520}, we will see that the trivial unitary relations play an important role among unitary boundary relations.

Let us observe the vector spaces: $\mathcal{K}^{2}$, $\mathcal{H}^{2}$, and $\mathcal{K}\times \mathcal{H}$. Elements of $\mathcal{K}^{2}$ are $\hat{f}=\left( {\begin{array}{*{20}c}
f\\
f'\\
\end{array} } \right),\hat{g}=\left( {\begin{array}{*{20}c}
g\\
g'\\
\end{array} } \right)$, elements of $\mathcal{H}^{2}$ are $\hat{h}=\left( 
{\begin{array}{*{20}c}
h\\
h'\\
\end{array} } \right), \hat{k}=\left( {\begin{array}{*{20}c}
k\\
k'\\
\end{array} } \right)$ and elements of $\mathcal{K}\times \mathcal{H}$ are $\left( 
{\begin{array}{*{20}c}
f\\
h\\
\end{array} } \right), \left( {\begin{array}{*{20}c}
f'\\
h'\\
\end{array} } \right), \left( {\begin{array}{*{20}c}
g\\
k\\
\end{array} } \right), \left( {\begin{array}{*{20}c}
g'\\
k'\\
\end{array} } \right)$ etc. Let the linear mapping $\mathcal{J}:\mathcal{K}^{2}\times \mathcal{H}^{2}\to \left( \mathcal{K}\times \mathcal{H} \right)\times \left( \mathcal{K}\times \mathcal{H} \right)$ be the well known \textit{main transformation}, see e.g. \cite[(2.16)]{DHMS1}. It satisfies 
\begin{equation}
\label{eq56}
\mathcal{J}\left( {\begin{array}{*{20}c}
\hat{f}\\
\hat{h}\\
\end{array} } \right)
=\left( {\begin{array}{*{20}c}
\left( {\begin{array}{*{20}c}
1 & 0\\
0 & 0\\
\end{array} } \right) & \left( {\begin{array}{*{20}c}
0 & \, 0\\
1 & \, 0\\
\end{array} } \right)\\
\left( {\begin{array}{*{20}c}
0 & 1\\
0 & 0\\
\end{array} } \right) & \left( {\begin{array}{*{20}c}
0 & 0\\
0 & -1\\
\end{array} } \right)\\
\end{array} } \right)\left( {\begin{array}{*{20}c}
\left( {\begin{array}{*{20}c}
f\\
f'\\
\end{array} } \right)\\
\left( {\begin{array}{*{20}c}
h\\
h'\\
\end{array} } \right)\\
\end{array} } \right)=\left( {\begin{array}{*{20}c}
\left( {\begin{array}{*{20}c}
f\\
h\\
\end{array} } \right)\\
\left( {\begin{array}{*{20}c}
f'\\
-h'\\
\end{array} } \right)\\
\end{array} } \right).
\end{equation}
In the sequel we will deal with Krein spaces $(\mathcal{K}^{2},\left[ .,. \right]_{\mathcal{K}^{2}})$, $(\mathcal{H}^{2},\left[ .,. \right]_{\mathcal{H}^{2}})$ and with the product Krein space $\mathcal{K}\times \mathcal{H}$ endowed  with the inner product 
\begin{equation}
\label{eq58}
\left[ \left( {\begin{array}{*{20}c}
f\\
h\\
\end{array} } \right),\left( {\begin{array}{*{20}c}
g\\
k\\
\end{array} } \right) \right]_{\mathcal{K}\times \mathcal{H}}:=\left[ f,g \right]_{\mathcal{K}}+\left( h,k 
\right)_{\mathcal{H}}.
\end{equation}

Let us also note that the matrix $\mathcal{J}$ is an \textit{involution}, i.e. it satisfies $\mathcal{J}^{*}=\mathcal{J}^{-1}=\mathcal{J}$. 

We will need the following lemma. %Because, the lemma was stated wrongly in \cite{B1}, we will state it correctly and prove it here, for convenience of the reader. The error in  \cite[Lemma 2.1]{B1} did not cause any further errors in \cite{B1}.

\begin{lemma}\label{lemma518} Assume that $\mathcal{K}_{1}$ and $\mathcal{K}_{2}$ are Krein spaces, $\mathcal{K}_{1}\cap \mathcal{K}_{2}=\lbrace 0 \rbrace$, and $A_{l}: \mathcal{K}_{l}\rightarrow\mathcal{K}_{l},\thinspace l=1,\thinspace 2$, are linear relations. We can define direct orthogonal sum
\[
\tilde{\mathcal{K}}:=\mathcal{K}_{1}\left[ + \right]\mathcal{K}_{2}, \,\, \, \mathcal{K}_{1}\cap \mathcal{K}_{2}= \lbrace 0 \rbrace
\]
and
\begin{equation}
\label{eq518}
\tilde{A}=A_{1}\left[ + \right]A_{2}:=\left\{ \left( {\begin{array}{*{20}c}
h_{1} \left[ + \right] h_{2}\\
h_{1}' \left[ + \right] h_{2}'\\
\end{array} } \right ) : \left( {\begin{array}{*{20}c}
h_{l}\\
h_{l}'\\
\end{array} } \right)\in A_{l},\thinspace \thinspace l=1,\thinspace 2 \right\}\subseteq \tilde{\mathcal{K}}\times \tilde{\mathcal{K}}.
\end{equation}
If the linear relation $\tilde{A}:=A_{1}\left[ + \right]A_{2}$ is symmetric (self-adjoint) in $\tilde{\mathcal{K}}$, then the linear relations $A_{l}\subseteq \mathcal{K}_{l}^{2},\thinspace \thinspace l=1,\thinspace 2$, are symmetric (self-adjoint).
\end{lemma} 

\textbf{Proof.} Let us assume $A_{1}\left[ + \right]A_{2} \subseteq \left( A_{1}\left[ + \right]A_{2}\right)^{+}$. Then
\[
A_{1} \subseteq  A_{1}\left[ + \right]A_{2} \subseteq \left( A_{1}\left[ + \right]A_{2}\right)^{+} \subseteq A_{1}^{+}.
\]
By the same token it holds  $A_{2} \subseteq A_{2}^{+}$.

Now we assume $\left( A_{1}\left[ + \right]A_{2}\right)^{+}=A_{1}\left[ + \right]A_{2}$. We have 
\[ 
\left( {\begin{array}{*{20}c}
k_{1} \\
k_{1}'\\
\end{array} } \right ) \in A_{1}^{+} \Rightarrow \left[ h_{1},k_{1}' \right] = \left[ h_{1}',k_{1} \right], 
\forall \left( {\begin{array}{*{20}c}
h_{1} \\
h_{1}'\\
\end{array} } \right ) \in A_{1} 
\]
\[ 
\Rightarrow  \left[ h_{1}+h_{2},k_{1}' \right] = \left[ h_{1}'+h_{2}',k_{1} \right], 
\forall \left( {\begin{array}{*{20}c}
h_{1}+h_{2} \\
h_{1}'+h_{2}'\\
\end{array} } \right ) \in A_{1}\left[ + \right]A_{2}
\]
\[
\Rightarrow \left( {\begin{array}{*{20}c}
k_{1} \\
k_{1}'\\
\end{array} } \right ) \in \left( A_{1}\left[ + \right]A_{2}\right)^{+}=A_{1}\left[ + \right]A_{2} \Rightarrow 
\left( {\begin{array}{*{20}c}
k_{1} \\
k_{1}'\\
\end{array} } \right ) \in A_{1}. 
\]
This proves $  A_{1}^{+} \subseteq  A_{1}$. Note, in the last step we used the assumption $\mathcal{K}_{1}\cap \mathcal{K}_{2}=\lbrace 0 \rbrace$.

By the same token it holds $A_{2}^{+} \subseteq A_{2}$. \hfill $\square$

The assumption $\mathcal{K}_{1}\cap \mathcal{K}_{2}=\lbrace 0 \rbrace $ in Lemma \ref{lemma518} that satisfy $\mathcal{K}_{1}$ and $\mathcal{K}_{2}$ a\textbf{s vector spaces}, cannot be avoided. The following example shows that. 

\begin{example}\label{example519} Let the vector spaces $\mathcal{K}$ and $\mathcal{H}$ that satisfy $\mathcal{K}=\mathcal{H}=\mathbb{C}$, be endowed with (different) scalar products that satisfy
\[
\left[f,g \right] =-\left(f,g \right) :=-f\bar{g}: f,g \in \mathbb{C}, 
\]
respectively. Then $\mathcal{K}$ and $\mathcal{H}$ are Krein and Hilbert space, respectively, and 
\[
A_{1} := \left\{\left( {\begin{array}{*{20}c}
f_{K}\\
f'_{K}\\
\end{array} } \right )\right\} ,\,  A_{2} :=\left\{\left( {\begin{array}{*{20}c}
f_{H}\\
f'_{H}\\
\end{array} } \right)\right\}, \,  f_{K}, f'_{K}, f_{H}, f'_{H}\in  \mathbb{C}.
\]
are not self-adjoint linear relations in $\mathcal{K}$ and $\mathcal{H}$, respectively, because $A_{i}^{+}=\lbrace 0 \rbrace$. We will see that it is possible to select $A_{1}$ and $A_{2}$ as above so that $\tilde{A}= A_{1}\left[ + \right]A_{2}$ is a self-adjoint relation.
\end{example} 

Because vector spaces $\mathcal{K}$ and $\mathcal{H}$ satisfy $\mathcal{K}=\mathcal{H}$, we can select 
\begin{equation}
\label{eq520}
\tilde{A}= A_{1}\left[ + \right]A_{2}=\left\{\left( {\begin{array}{*{20}c}
f\\
f\\
\end{array} } \right ), \left( {\begin{array}{*{20}c}
f'\\
f'\\
\end{array} } \right)\right\},
\end{equation}
i.e. $f_{K}=f_{H}=f$ and $f'_{K}= f'_{H}=f'$. Then it holds $\tilde{A}^{+} \subseteq \tilde{A} $, i.e. $\tilde{A}$ is a self-adjoint relation in $\mathcal{K} [+] \mathcal{H}$ endowed with the scalar product (\ref{eq58}). Indeed, for 
\[
\left\{\left( {\begin{array}{*{20}c}
g_{K}\\
g_{H}\\
\end{array} } \right ), \left( {\begin{array}{*{20}c}
g_{K}'\\
g_{H}'\\
\end{array} } \right)\right\} \in \left( \tilde{A}\right)^{+} 
\]
\[
\Rightarrow \left[ \left( {\begin{array}{*{20}c}
f\\
f\\
\end{array} } \right ), \left( {\begin{array}{*{20}c}
g_{K}'\\
g_{H}'\\
\end{array} } \right)\right] - \left[ \left( {\begin{array}{*{20}c}
f'\\
f'\\
\end{array} } \right ), \left( {\begin{array}{*{20}c}
g_{K}\\
g_{H}\\
\end{array} } \right)\right]=0, \forall \left\{\left( {\begin{array}{*{20}c}
f\\
f\\
\end{array} } \right ), \left( {\begin{array}{*{20}c}
f'\\
f'\\
\end{array} } \right)\right\} \in \tilde{A}
\]
Because, $f \in \mathbb{C}$ and $f' \in \mathbb{C}$ run independently, it follows 
\[
 g_{K}=g_{H}=:g \wedge f'_{K}=g'_{H}=:g'.
\]
\[
\Rightarrow A^{+}=\left\{\left( {\begin{array}{*{20}c}
g\\
g\\
\end{array} } \right ), \left( {\begin{array}{*{20}c}
g'\\
g'\\
\end{array} } \right)\right\} \in \tilde{A}
\]

Hence, $\tilde{A}$ is self-adjoint even thought $A_{1}$ and $A_{2}$ are not. This proves that the assumption $\mathcal{K}_{1}\cap \mathcal{K}_{2}=\lbrace 0 \rbrace$ in Lemma \ref{lemma518} is essential. \hfill $\square$

\begin{proposition}\label{proposition520} Let $\mathcal{K}$ and $\mathcal{H}$ be a Krein and Hilbert space, respectively, and $\mathcal{K}\cap \mathcal{H}=\lbrace 0 \rbrace$. 
\begin{enumerate}[(i)]%, (i), (ii),...
\item Let $\tilde{A}:\mathcal{K}\times\mathcal{H} \rightarrow \mathcal{K}\times\mathcal{H}$ be a asymmetric (self-adjoint) linear relation, and let $\mathcal{J}: \left( \mathcal{K}^{2}\times \mathcal{H}^{2}\right) \rightarrow \left( \mathcal{K}\times\mathcal{H}\right)\times \left( \mathcal{K}\times\mathcal{H}\right)$ be the linear mapping given by (\ref{eq56}). Then 
\[
\Gamma = \mathcal{J}^{-1}\left( \tilde{A} \right) 
\]
is a trivial isometric (unitary) relation.
\item Let $\Gamma : \mathcal{K}^{2} \rightarrow  \mathcal{H}^{2}$ be an isometric (unitary) boundary relation. Then $\Gamma $ is a trivial isometric (unitary) boundary relation.  
\end{enumerate}
\end{proposition}

\textbf{Proof.}

(i) Let us assume that $\tilde{A}$ is a self-adjoint relation. We will use Lemma \ref{lemma518}, after we adjust notation. We will substitute $\mathcal{K}:=\mathcal{K}_{1}$, and $\mathcal{H}:=\mathcal{K}_{2}$. Obviously, then (\ref{eq518}) is equivalent to 
\[
\tilde{A}=A_{1}\left[ + \right]A_{2}:=\left\{ \left\{\left( {\begin{array}{*{20}c}
h_{1}\\
h_{2}\\
\end{array} } \right ), \left( {\begin{array}{*{20}c}
h_{1}'\\
h_{2}'\\
\end{array} } \right)\right\}:\thinspace \left( {\begin{array}{*{20}c}
h_{l}\\
h_{l}'\\
\end{array} } \right)\in A_{l}, \thinspace l=1,\thinspace 2 \right\}\subseteq \left( \mathcal{K}\times \mathcal{H}\right) \times \left( \mathcal{K}\times \mathcal{H}\right).
\]
According to Lemma \ref{lemma518}, $A_{1}$ is a self-adjoint linear relation in $\mathcal{K}$ and $-A_{2}$ is a self-adjoint linear relation in $\mathcal{H}$. According to \cite[Proposition 2.10]{DHMS1}, $\Gamma = \mathcal{J}^{-1}\left( \tilde{A} \right)$ is a unitary relation, $ \dom\, \Gamma =A_{1}$ and $ \ran\, \Gamma =-A_{2} $. 

Because, $A_{1}$ and $-A_{2}$ are self-adjoint we have $ \left( \dom\, \Gamma\right)^{+} =A_{1}$ and $ \left( \ran\, \Gamma\right)^{+}=-A_{2} $.  Because $\Gamma$ is a unitary relation we have $\ker \Gamma = \left( \dom\, \Gamma\right)^{+} = A_{1}$. Similarly, it holds $ \mul\, \Gamma = \left( \ran\, \Gamma\right)^{+}=-A_{2}$. Therefore
\[
\dom\, \Gamma = \ker\Gamma \wedge \ran\, \Gamma = \mul\, \Gamma.
\]
Because, $A_{1}$ and $A_{2}$ are self-adjoint,  $\dom\, \Gamma$ and $\mul\, \Gamma$ are hyper-maximal neutral sub-spaces. By definition, $\Gamma $ is a trivial unitary relation.  

If $\tilde{A}$ is symmetric, then $A_{1}$ and $A_{2}$ are symmetric relations, and therefore $\dom\, \Gamma$ and $\mul\, \Gamma$ are neutral manifolds. By definition, $\Gamma $ is a trivial isometric relation. 

(ii) If $\Gamma$ is an isometric (unitary) relation, then according to \cite[Proposition 2.10]{DHMS1} $\tilde{A}$ is a symmetric relation. Then the statement (ii) follows form (i). \hfill $\square$

%\begin{example}\label{example522} Let a self-adjoint linear relation $\tilde{A}$ be given by (\ref{eq520}). Then
%\[
%\Gamma := \mathcal{J}^{-1}\left( \tilde{A} \right)=\left\{ \left\{\left( {\begin{array}{*{20}c}
%f\\
%f'\\
%\end{array} } \right ), \left( {\begin{array}{*{20}c}
%f\\
%-f'\\
%\end{array} } \right)\right\}:\thinspace \left( {\begin{array}{*{20}c}
%f\\
%f'\\
%\end{array} } \right)\in  \mathbb{C}^{2} \right\}\subseteq \mathcal{K}^{2}\times \mathcal{H}^{2}
%\]
%is one-to-one unitary operator with $\dom\, \Gamma=A_{1}=\mathcal{K}^{2}$and $\ran\, \Gamma = -A_{2}=\mathcal{H}^{2}$.  
%\end{example}
%
\section{An application of the Green's boundary relations in Pontryagin spaces}\label{s12}
In this section we use the concept of the Green's boundary relation to prove generalizations of \cite[Lemma 3.15]{BDHS} and \cite[Theorem 4.8]{BDHS}.  

For a linear relation $T\subseteq \mathcal{K}^{2}$ the following concepts are defined:
\[
R_{z}\left( T \right):=\ker \left( T-z \right), \hat{R}_{z}\left( T 
\right):=\left\{ \left( {\begin{array}{*{20}c}
f_{z}\\
zf_{z}\\
\end{array} } \right):f_{z}\in R_{z}\left( T \right) \right\}, z\in \mathbb{C}.
\]
Let $\Gamma :\mathcal{K}^{2} \to \mathcal{H}^{2}$ be a Green's boundary relation. If we denote 
$T:=\dom\, \Gamma $, then for $\hat{f_{z}}:= \left( {\begin{array}{*{20}c}
f_{z}\\
zf_{z}\\
\end{array} } \right) \in \hat{R}_{z}\left( T 
\right)$ and $\hat{h}=\left( {\begin{array}{*{20}c}
h\\
h'\\
\end{array} } \right)\in \mathcal{H}^{2}$ the function 
%\begin{equation}
%\label{eq62}
%\gamma \left( z \right)=\left\{ \left( {\begin{array}{*{20}c}
%h\\
%f_{z}\\
%\end{array} } \right):\left( {\begin{array}{*{20}c}
%\hat{f_{z}}\\
%\hat{h}\\
%\end{array} } \right)\in \Gamma ,\, \hat{f_{z}}\in \hat{R}_{z}\left( T \right) \right\}
%\end{equation}
%then the $\gamma $\textit{-field and the Weyl family associated with the boundary relation} $\Gamma $ are defined %analogously as $\gamma $-field and the Weyl function in the case of ordinary boundary triple, for example the relation 
\begin{equation}
\label{eq64}
M\left( z \right)=\left\{ \hat{h}:\left( {\begin{array}{*{20}c}
\hat{f_{z}}\\
\hat{h}\\
\end{array} } \right)\in \Gamma :\hat{f_{z}}\in \hat{R}_{z}\left( T \right) 
\right\}=\Gamma \left( \hat{R}_{z}\left( T \right) \right),
\end{equation}
is called the \textit{Weyl family associated with the boundary relation} $\Gamma $, c.f. \cite[Definition 3.6]{BDHS}.

The following definition is a generalization of \cite[Definition 3.11]{BDHS} and \cite[Definition 3.4]{DHMS1}. In those definitions $\Gamma$ is a unitary relation, while in the following definition we assume only that the weaker condition (\ref{eq222}) is satisfied. 

\begin{definition}\label{definition62} A Green's boundary relation $\Gamma :\mathcal{K}^{2}\to \mathcal{H}^{2}$ that satisfies condition (\ref{eq222}), with $T:=\dom\, \Gamma $, is called minimal if 
\begin{equation}
\label{eq66}
\mathcal{K}=\overline{clos}\left\{ R_{z}\left( T \right):z\in \hat{\rho }\left( S \right) \right\},
\end{equation}
\end{definition} 
where $S:=\left( \dom\, T\right)^{+}$. It is necessary to have $\hat{\rho }\left( S \right)\ne \emptyset $; otherwise it would be 
\[
\overline{clos}\left\{ R_{z}\left( T \right):z\in \hat{\rho }\left( S \right) \right\}=\emptyset. 
\]
\begin{proposition}\label{proposition64} Let $\mathcal{K}$ be a Pontryagin space and let $\Gamma :\mathcal{K}^{2}\to \mathcal{H}^{2}$ be a minimal Green's boundary relation. Then $S$ is a symmetric operator with $\sigma_{p}(S)=\emptyset$.
\end{proposition}
\textbf{Proof}. Because the minimal Green's boundary relation $\Gamma$ satisfies condition (\ref{eq222}), according to Proposition \ref{proposition28} (i), $S$ is a closed symmetric linear relation in the Pontryagin space $\mathcal{K}$. The following is a proof that the relation $S$ does not have any eigenvalues in $\mathbb{C}\cup\lbrace\infty\rbrace$. That proof is repetition of the proof of \cite[Lemma 3.15]{BDHS}. We repeat it here to make sure that it still works in this more general situation.  

Assume that the claim is not true. Let us first assume that there exist a finite $\lambda \in \sigma_{p}(S)$ with $\left( {\begin{array}{*{20}c}
f_{\lambda}\\
\lambda f_{\lambda}\\
\end{array} } \right) \in S$.  Let $\eta \in \hat{\rho }\left( S \right)$ be an arbitrarily selected complex number and $g_{\eta} \in R_{\eta}(T):=\ker \left( T-\eta I \right) $, i.e. $\left( {\begin{array}{*{20}c}
g_{\eta}\\
\eta g_{\eta}\\
\end{array} } \right) \in T  $. Because $S=T^{+}$ it follows
\[\left[ \left( {\begin{array}{*{20}c}
f_{\lambda}\\
\lambda f_{\lambda}\\
\end{array} } \right), \left( {\begin{array}{*{20}c}
g_{\eta}\\
\eta g_{\eta}\\
\end{array} } \right) \right]_{\mathcal{K}^{2}}=0 \Rightarrow \left( \bar{\eta}-\lambda\right)\left[f_{\lambda},g_{\eta} \right] =0 \Rightarrow f_{\lambda} \left[ \perp \right]_{\mathcal{K}} R_{\eta}(T).
\]
Because $\eta \in \hat{\rho }\left( S \right)$ was arbitrarily selected, according to (\ref{eq66}), $f_{\lambda}=0$ i.e. $\lambda$ is not an eigenvalue of $S$. 

The assumption $\lambda = \infty$ is an eigenvalue of $S$ means $\left( {\begin{array}{*{20}c}
0\\
f_{\infty}\\
\end{array} } \right) \in S$. We have 
\[\left[ \left( {\begin{array}{*{20}c}
0\\
f_{\infty}\\
\end{array} } \right), \left( {\begin{array}{*{20}c}
g_{\eta}\\
\eta g_{\eta}\\
\end{array} } \right) \right]_{\mathcal{K}^{2}}=0 \Rightarrow \left[f_{\infty},g_{\eta} \right] =0
\Rightarrow f_{\infty} \left[ \perp \right]_{\mathcal{K}} R_{\eta}(T).
\] 
Again, according to (\ref{eq66}) it holds $f_{\infty}=0$, which means that $S$ is an operator. 
\hfill $\square$

Proposition \ref{proposition64} is a generalization of \cite[Lemma 3.15]{BDHS} because here we do not assume that $\Gamma$ is a unitary linear relation, as was implicitly assumed in \cite[Definition 3.11]{BDHS}. Here we assume only that the weaker condition (\ref{eq222}) holds.

\begin{theorem}\label{theorem66} Let $\mathcal{K}$ be a Pontryagin space with $\kappa $ negative squares. Let $\Gamma :\mathcal{K}^{2}\to \mathcal{H}^{2}$ be a minimal unitary boundary relation and let $M\left( z \right)$ be the Weyl family associated with $\Gamma $, i.e. the Weyl family defined by (\ref{eq66}). Then the family $M\left( z \right)$ is a generalized Nevanlinna family with $\kappa $ negative squares. 
\end{theorem}

\textbf{Proof.} According to Proposition \ref{proposition510}, unitary relation $\Gamma$ satisfies condition (\ref{eq222}). Then, according to Proposition \ref{proposition28} (i), $S:=T^{+}$ is a closed symmetric linear relation in the Pontryagin space $\mathcal{K}$ that satisfies $S^{+}=\overline{\dom}\, \Gamma$. This means that the \textbf{unitary boundary relation} $\Gamma $ in the sense of Definition \ref{definition214} is a \textbf{boundary relation} $\Gamma $ for $S^{+}$ in terms of \cite{BDHS}. Then Theorem \ref{theorem66} follows from \cite[Theorem 4.8]{BDHS}.  \hfill $\square$

Theorem \ref{theorem66} is a generalization of \cite[Theorem 4.8]{BDHS} because existence of the closed linear relation $S$ with $\overline{\dom}\, \Gamma= S^{+}$ is not assumed in the statement of the theorem as was assumed in \cite[Theorem 4.8]{BDHS}. The existence of such $S$ is a consequence of the assumption that $\Gamma$ is a unitary boundary relation, as we saw earlier. 
\section{Green's boundary relations with a non-degenerate range }\label{s14}

In this chapter, we assume that $\ran\, \Gamma$ is a non-degenerate manifold. Sometimes, we use the stronger assumption $\overline{\ran}\, \Gamma = \mathcal{H}^{2}$. In the latter case the relation $\Gamma$ becomes a \textit{Green's boundary operator}. As usually, $\bar{\Gamma}$ denotes a closure of $\Gamma :\mathcal{K}^{2}\to \mathcal{H}^{2}$. 

\begin{proposition}\label{proposition72} Let $\Gamma : \mathcal{K}^{2} \to \mathcal{H}^{2}$ be a Green's boundary relation with non-degenerate $\ran\, \Gamma$. Then:
\begin{enumerate}[(i)]%, (i), (ii),...
\item $\bar{\Gamma} :\mathcal{K}^{2} \rightarrow \mathcal{H}^{2}$ is a Green's boundary relation.
\item $ \ker \bar{\Gamma }$ is a closed symmetric relation.
\item If $\Gamma$ satisfies condition (\ref{eq222}), then also $\bar{\Gamma }$ satisfies condition (\ref{eq222}).
\item If $S$ and $\hat{S}$ are closed symmetric relations associated with $\Gamma$ and $\bar{\Gamma}$, respectively, then $S=\hat{S}$. 
\item If $\hat{M}$ denotes isotropic manifold of $\dom\, \bar{\Gamma}$ then $\hat{M} \subseteq \hat{S}$.
\end{enumerate}
\end{proposition}

\textbf{Proof}. 

(i) The continuity of the scalar products (\ref{eq28}) and (\ref{eq210}) implies that the closure $\bar{\Gamma }$ satisfies (\ref{eq214}). Hence, (i) holds. 

(ii) The relation $\ker \bar{\Gamma }$ is closed as a kernel of a closed relation, and it is symmetric according to Lemma \ref{lemma26} (ii). 

(iii) Let us first prove the claim
\begin{equation}
\label{eq72}
\overline{\dom}\, \bar{\Gamma }=\overline{\dom}\, \Gamma.
\end{equation}
Assume, $f\in \dom\, \bar{\Gamma }$. Then
$\left( {\begin{array}{*{20}c}
f\\
f'\\
\end{array} } \right)\in \bar{\Gamma }$, for some $f'\in \ran\, \bar{\Gamma}$. This means that there exist the sequences $\left( \left\{ f_{n} \right\} \subseteq \dom\, \Gamma \wedge  \left\{ f_{n}' \right\} \subseteq \ran\, \Gamma \right)  $ such that $ \left( f_{n}\to f  \wedge f_{n}' \to f' \right) $, when $n\rightarrow \infty$. This further means $f \in  \overline {\dom}\, \Gamma  \wedge f' \in \overline{\ran}\, \Gamma $. Hence, $\dom\, \bar{\Gamma } \subseteq \overline{\dom}\, \Gamma$. This implies $\overline{\dom}\, \bar{\Gamma } \subseteq \overline{\dom}\, \Gamma$. Because the converse inclusion is obvious, the identity (\ref{eq72}) holds.

Now, we can prove (\ref{eq222}) for $\bar{\Gamma}$. Indeed, 
\[
\left[ \hat{f},\hat{g} \right]_{\mathcal{K}^{2}}=0,\, \forall \hat{g}\in \dom\, 
\bar{\Gamma }\Rightarrow \left[ \hat{f},\hat{g} \right]_{\mathcal{K}^{2}}=0,\, \forall 
\hat{g}\in \dom\, \Gamma .
\]
According to assumption that (\ref{eq222}) holds for $\Gamma $, it follows $\hat{f}\in \overline{\dom}\, 
\Gamma =\overline{\dom}\, \bar{\Gamma }$. This proves (iii). 

(iv) This claim follows directly from (\ref{eq72}) and definition of $\hat{S}$. 

(v) From definitions of an isotropic manifold and (i), it follows $\hat{M}=\dom\, \bar{\Gamma} \cap \left( \dom\, \bar{\Gamma}\right)^{+} \subseteq \left( \dom\, \bar{\Gamma}\right)^{+} = \hat{S}$. 

\begin{lemma}\label{lemma73}Let $\Gamma : \mathcal{K}^{2}\to \mathcal{H}^{2}$,  be a Green's boundary relation. If $\overline{\ran}\,\Gamma =\mathcal{H}^{2}$, then $\overline{\Gamma_{i}\left( \ker \Gamma_{j} \right)}=\mathcal{H},  i \neq j; i,j=0,1$.
\end{lemma}

\textbf{Proof.} Let us arbitrarily select a point $h \in \ran\, \Gamma_{0}$ i.e. $\left( {\begin{array}{*{20}c}
h\\
0\\
\end{array} } \right)\in \ran\, \Gamma \subseteq 
\mathcal{H}^{2}$. There exists $\hat{s}\in \dom\, \Gamma \subseteq \mathcal{K}^{2}$ such that 
$\left( {\begin{array}{*{20}c}
h\\
0\\
\end{array} } \right)=\left( {\begin{array}{*{20}c}
\Gamma_{0}\hat{s}\\
\Gamma_{1}\hat{s}\\
\end{array} } \right)$. This means that $\hat{s}\in \ker \Gamma_{1}$. Therefore 
\[
\ran\, \Gamma_{0}=\Gamma_{0}\left( \ker \Gamma_{1} \right). 
\]
If we assume that the claim $\overline{\Gamma_{0}\left( \ker \Gamma_{1} \right)}=\mathcal{H}$ does not hold, then there exists $t\in \mathcal{H}, 0\ne t$, such that $t\left( \bot \right) \ran\, \Gamma_{0}$. Then
$\ran\, \Gamma  \left[ \bot \right]_{\mathcal{H}^{2}} \left( {\begin{array}{*{20}c}
0\\
t\\
\end{array} } \right) $.
This contradicts to the assumption that $\ran\, \Gamma $ is dense in $\mathcal{H}^{2}$. 
This proves that $\overline{\Gamma_{0}\left( \ker \Gamma_{1} \right)}=\mathcal{H}$. By the same token $\overline{\Gamma_{1}\left( \ker \Gamma_{0} \right)}=\mathcal{H}$ holds. \hfill $\square$

\begin{proposition}\label{proposition74} Let $\Gamma : \mathcal{K}^{2}\to \mathcal{H}^{2}$,  be a Green's boundary relation. Assume
\begin{itemize}
\item Maximality condition (\ref{eq222}) is fulfilled,
\item $\overline{\ran}\,\Gamma =\mathcal{H}^{2}$,
\item $\dom\, \Gamma $ is closed.
\end{itemize}
Then: 
\begin{enumerate}[(i)]%, (i), (ii),...
\item Linear relations $S_{i}:=\ker \Gamma_{i},\, i=0,1$, are self-adjoint.
\item $S:=\left( \dom\, \Gamma \right)^{+}$ is a closed symmetric linear relation that satisfies $S=\ker \Gamma_{0} \cap \ker \Gamma_{1}$, i.e. extensions $S_{i}:=\ker \Gamma_{i}, i=0,1$, of $S$ are disjoint. 
\item $S^{+}=S_{0} \hat{+}S_{1}$, i.e. extensions $S_{i}, i=0,1$, of $S$ are transversal. 
\end{enumerate} 
\end{proposition}

\textbf{Proof}. According to Proposition \ref{proposition52} (ii), the assumption $\overline{\ran}\,\Gamma =\mathcal{H}^{2}$ means that $\Gamma, \Gamma_{i}, i=0,1,$ are operators.

(i) The proof consists of the following four steps. 

\textbf{Step 1.} $\ker \Gamma_{i}\subseteq \left( \ker \Gamma_{i} \right)^{+}, \, i=0,1$. 

This holds according to Lemma \ref{lemma26} (i).

\textbf{Step 2.} If $\ran\, \Gamma $ is dense in $\mathcal{H}^{2}$, then $\overline{\Gamma_{i}\left( \ker \Gamma_{j} \right)}=\mathcal{H},  i \neq j; i,j=0,1$. 

This claim holds according to Lemma \ref{lemma73}.

\textbf{Step 3.} $\left( \ker \Gamma_{i} \right)^{+}\subseteq \dom\, \Gamma, \, i=0, 1$.

Indeed, because we assume that (\ref{eq222}) holds, we can define $S^{+}:=\overline{\dom}\, \Gamma $. According to Proposition \ref{proposition28} (i),  $S={(\overline{\dom}\, \Gamma)}^{+} \subseteq S^{+}$. Because,  $\overline{\ran}\, \Gamma =\mathcal{H}^{2}$ and $\dom\, \Gamma$ is closed, according to Corollary \ref{corollary56} (i), it holds $M=N=S$, i.e. $\left(\dom\, \Gamma \right)^{+}\cap \dom\, \Gamma =\ker \Gamma =S$. Then, obviously, $S\subseteq \ker \Gamma_{i}$. This further implies $ \left( \ker \Gamma_{i} \right)^{+}\subseteq S^{+}=\overline{\dom}\, \Gamma =\dom\, \Gamma $, which proves Step 3. 

\textbf{Step 4.} $\left( \ker \Gamma_{i} \right)^{+}\subseteq \ker \Gamma _{i}, i \neq j; i,j=0,1$. 

Indeed, assume e.g. $\hat{f}\in \left( \ker \Gamma_{1} \right)^{+}$. According to Step 3, $\Gamma_{1}\hat{f}$ is defined. It holds 
\[
0=\left[ f',g \right]-\left[ f,g' \right]=\left( \Gamma_{1}\hat{f}, \Gamma_{0}\hat{g} \right), \forall \hat{g} \in \ker \Gamma_{1}.
\]
According to Step 2, $\Gamma_{0}\left( \ker \Gamma_{1} \right)$ is dense in $H$. We conclude $\Gamma_{1}\hat{f}=0$. This proves, $\left( \ker \Gamma_{1} \right)^{+}\subseteq \ker \Gamma_{1}$. 

By the same arguments we prove $\left( \ker \Gamma_{0} \right)^{+}\subseteq \ker \Gamma_{0}$, which proves Step 4. 
\\

Statement (i) follows from steps 1 and 4, because  
\[
Step \,1 \wedge Step \, 4 \Rightarrow \left( \ker \Gamma_{i} \right)^{+}=\ker \Gamma_{i}, \, i=0,1.
\]

(ii) In the proof of Step 3, we saw $S=\ker \Gamma $. Because $\ker \Gamma=\ker \Gamma _{0} \cap \ker \Gamma _{1}$, the statement (ii) holds. 

(iii) We will prove that every $\hat{k}=\left( {\begin{array}{*{20}c}
k\\
k'\\
\end{array} } \right)\in \dom\, \Gamma = S^{+}$ can be decomposed as $\hat{k}=\hat{u}+\hat{r}, \, \hat{u} \in \ker\, \Gamma _{0}, \hat{r} \in \ker\, \Gamma _{1}$. 

Assume $\hat{h}=\Gamma \hat{k}$. Then, 
\[
\left( {\begin{array}{*{20}c}
h\\
h'\\
\end{array} } \right)=\left( {\begin{array}{*{20}c}
0\\
h'\\
\end{array} } \right)+\left( {\begin{array}{*{20}c}
h\\
0\\
\end{array} } \right)=\Gamma \hat{t}+\Gamma \hat{r}, \, \hat{t} \in \ker\, \Gamma _{0}, \hat{r} \in \ker\, \Gamma _{1}.
\]
This means $ \hat{s}:=\hat{k}-\hat{t}-\hat{r} \in S \subseteq ker\, \Gamma _{0}$. Hence,
$\hat{k}:=\left( \hat{s}+ \hat{t}\right)+\hat{r} =: \hat{u}+ \hat{r} \in ker\, \Gamma _{0} \hat{+} ker\, \Gamma _{1}$ This proves (iii). \hfill $\square$

\section{Properties of boundary triples}\label{s16}

When $\mathcal{K}$ is a separable \textbf{Hilbert} space, and $S\subseteq \mathcal{K}^{2}$ is a closed symmetric linear \textbf{operator} with equal finite or infinite deficiency indexes, the following kinds of boundary 
triples were defined in \cite{BL,DHM}: \textbf{ordinary}, \textbf{B-generalized},\textbf{ AB-generalized,} 
\textbf{isometric}, \textbf{unitary}, \textbf{S-generalized}, \textbf{quasi boundary triple}, and some others. 

In this section, we will fit all above-mentioned boundary triples into the general Green's boundary relation model and we will generalize all above boundary triples to a Krein space $\mathcal{K}$ by means of GBR. 

Definitions of isometric and unitary boundary relations in terms of Green's boundary relations were given earlier in this paper, see definitions \ref{definition210} and \ref{definition214}. Those definitions are generalizations of isometric and unitary triples given by \cite[Definition 1.9]{ DHM}.

Definitions of AB-generalized and B-generalized boundary triples \cite[definitions 1.5 and 1.8]{DHM} can 
be generalized in terms of Green's boundary relation and in the \textbf{Krein space} $\mathcal{K}$ in the following way:

\begin{definition}\label{definition82} A Green's boundary relation $\Gamma : \mathcal{K}^{2}\to \mathcal{H}^{2}$ is called an almost B-generalized, or AB-generalized boundary relation if it satisfies:

\begin{itemize}
\item Condition (\ref{eq222}),
\item $\overline{\ran}\, \Gamma_{0}=\mathcal{H}$, and
\item $A:=\ker \Gamma_{0}$ is a self-adjoint linear \textbf{relation} in $\mathcal{K}$.
\end{itemize}
If additionally ${\ran\, \Gamma }_{0}=\mathcal{H}$, then the relation $\Gamma$ is called B-generalized boundary relation.
\end{definition}
According to Definition \ref{definition210}, AB-generalized and B-generalized boundary relations are isometric boundary relations. Note also that every GBR that satisfies conditions of Proposition \ref{proposition74} iz an AB-generalized boundary operator. 

For the definition of a \textit{quasi boundary triple} for $S^{+}$ in the Hilbert space $\mathcal{K}$ see \cite[Definition 2.1]{BL}. Here are generalizations  to Krein space $\mathcal{K}$ in terms of Green's boundary relation.

\begin{definition}\label{definition810} A Green's boundary relation $\Gamma : \mathcal{K}^{2}\to \mathcal{H}^{2}$ is called quasi-boundary if it satisfies: 

\begin{itemize}
\item Condition (\ref{eq222}),
\item $\overline{\ran}\, \Gamma =\mathcal{H}^{2},$ \textit{and }
\item $A\, :=\ker \Gamma_{0}$ is a self-adjoint operator in $\mathcal{K}$.
\end{itemize}
\end{definition}
Note, a quasi-boundary relation $\Gamma$ is always an isometric boundary closable operator.

\begin{definition}\label{definition812} A Green's boundary relation $\Gamma : \mathcal{K}^{2}\to \mathcal{H}^{2}$ is called an S-generalized boundary relation if:

\begin{itemize}
\item It is a unitary relation,
\item $A:=\ker \Gamma_{0}$ is a self-adjoint operator.
\end{itemize}
\end{definition} 
\textbf{The Compliance with Ethical Standards Statements:}

\textbf{Conflict of Interest:} The author states that there is no conflict of interest. 

\textbf{Funding:} No funding was received to assist with the preparation of this manuscript.

\textbf{Ethical Conduct:} This is an original research. I am submitting this manuscript only to "Integral Equations and Operator Theory". All previous results used in this paper are properly cited. 

\textbf{Data Availability Statements:} 

This is pure mathematical research. The data that support this paper are only mathematical statements and they are openly available in cited references. There are no any other data than mathematical statements. 

\begin{thebibliography}{99}

\bibitem {A} R. Arens, Operational calculus of linear relations, Pacific J. Math., 11 (1961), 9-23.

%\bibitem {AG} N. I. Akhiezer, I.M. Glazman, Theory of Linear Operators in Hilbert 
Space, Dover Publications, Inc. New York, 1993.

%\bibitem {BK} J. Behrndt, H. Kreusler, Boundary relations and generalized resolvents of symmetric relations in Krein spaces. Integral Equations Operator Theory 59 (2007), 309-327


\bibitem {BL} J. Behrndt, M. Langer, Boundary value problems for elliptic partial differential operators on bounded domains, J. Functional Analysis, 243, (2007), 536--565.

\bibitem {BDHS} J. Behrndt, V. A. Derkach,  S. Hassi, H. de Snoo, A realization theorem for generalized Nevanlinna families, Operators and Matrices, 4, (2011) 679-706

\bibitem {BHS} J. Behrndt, S. Hassi, H. de Snoo, Boundary Value Problems, Weyl Functions, and Differential Operators, Open access eBook, https://doi.org/10.1007/978-3-030-36714-5

\bibitem {Bog} J. Bognar, Indefinite Inner Product Spaces,Springer-Verlag, Berlin, Heidelberg, New York, 1974.

\bibitem {B3} M. Borogovac, Points of nondegenerate range of closed Hermitian operator in Pontryagin space, Glasnik Matematicki, Vol. 21 (41) (1986), 123-135.

%\bibitem {B} M. Borogovac, Inverse of generalized Nevanlinna function that is holomorphic at Infinity, North-Western European Journal of Mathematics, 6 (2020), 19-43.

\bibitem {B1} M. Borogovac, Reducibility of self-adjoint linear relations and application to generalized Nevanlinna functions, to appear in Ukr. Math. J., Vol. 74 (2022)

\bibitem {B2} M. Borogovac, Characterization of Weyl functions in the class of regular generalized Nevanlinna functions $Q \in N_{\kappa}(\mathcal{H})$, preprint,
https://www.researchgate.net/profile/Muhamed-Borogovac/research 

%\bibitem {BLu} M. Borogovac, A. Luger, Analytic characterizations of Jordan Chains by pole cancellation functions of higher order, J. Funct. Anal. (2014), http://dx.doi.org/10.1016/j.jfa.2014.09.025

\bibitem {D1} V. A. Derkach, On generalized resolvents of Hermitian relations in Krein spaces, Journal of Math. Sci, Vol. 97, No. 5, 1999.

\bibitem {D2} V. A. Derkach, Boundary Triplets, Weyl Functions, and the Krein Formula, In book: Operator Theory, Chapter 10, pp.183-218 (2014); DOI: 10.1007/978-3-0348-0667-1\textunderscore 32 

\bibitem {DHM}. V. Derkach, S. Hassi, M. Malamud, Generalized boundary triples, I. Some classes of isometric and unitary boundary pairs and realization problems for subclasses of Nevanlinna functions, Math. Nachr. 293 (2020), 1278–1327.

%\bibitem {DL} K. Daho and H. Langer. Matrix functions of the class $N_{\kappa }^{nxn}$. Math. Nachr. 120:275 294, 1985.

%\bibitem {DLS} A. Dijksma, H. Langer and H. S. V. de Snoo, Eigenvalues and pole functions of Hamiltonian systems with eigenvalue depending boundary conditions, Math. Nachr. 161 (1993) 107-154. 

%\bibitem {DHM} V. A. Derkach, S. Hassi, M. Malamoud, Generalied Boundary Triples, Weyl Functions and Inverse Problems, preprint arXiv:math/0610299v1 .

\bibitem {DHMS1} V. A. Derkach, S. Hassi, M.M. Malamud, and H.S.V. de Snoo, Boundary relations and their Weyl families, Trans. Amer. Math. Soc., 358 (2006), 5351–5400.

\bibitem {DHMS2} V. A. Derkach, S. Hassi, M.M. Malamud, and H.S.V. de Snoo, Boundary relations and generalized resolvents of symmetric operators, Russ. J. Math. Phys. 16 (2009), 17–60.

\bibitem {DM1} V. A. Derkach, M. M. Malamud, Generalized Resolvents and the Boundary Value Problems for Hermitian Operators with Gaps, J. Funct. Anal. 95, 1-95 (1991)  

\bibitem {DM2} V. A. Derkach, M. M. Malamud, The extension theory of hermitian operators and the moment problem, J. Math. Sciences, 73 (1995), 141–242.

\bibitem {DS} A. Dikjsma, H.S.V. de Snoo, Linear relations in indefinite inner product spaces, Argonne national labaratory report ANL-84-73.  

%\bibitem {HL} S. Hassi, A. Luger, Generalized zeros and poles $N_{\kappa }$-functions: on the underlying spectral structure, Methods of Functional Analysis and Topology Vol. 12 (2006), no. 2, pp. 131-150.

%\bibitem {HSW} S. Hassi, H.S.V. de Snoo, and H. Woracek: Some interpolation problems of NevanlinnaPick type, Oper. Theory Adv. Appl.106 (1998), 201-216.

\bibitem {IKL} I. S. Iohvidov, M. G. Krein, H. Langer, Introduction to the Spectral Theory of Operators in Spaces with an Indefinite Metric, Akademie-Verlag, Berlin, 1982.

%\bibitem {KL1} M. G. Krein and H. Langer, $\ddot{U}$ber die $Q$-Funktion eines $\pi $-hermiteschen Operatos im Raume $\Pi_{\kappa }$, Acta Sci. Math. 34, 190-230 (1973). 

%\bibitem {KL2} M. G. Krein and H. Langer, $\ddot{U}$ber einige Fortsetzungsprobleme, die eng mit der Theorie hermitescher Operatoren im Raume $\Pi_{\kappa }$ zusammenhangen, I. Einige Funktionenklassen und ihre Darstellungen, Math. Nachr. 77, 187-236 (1977). 

%\bibitem {L} H. Langer, Spectral functions of definitizable operators in Krein spaces, In: Functional Analysis, Proceedings, Dubrovnik 1981, Lecture Notes Math. (984) Springer-Verlag, Berlin and New York, 1982.

%\bibitem {LaLu} H. Langer and A. Luger, A class of 2x2-matrix functions, Glasnik Matematicki, 35(55) 149-160, (2000)

%\bibitem {LT} H. Langer and B. Textorius, On generalized resolvents and Q-functions of symmetric linear relations (subspaces) in Hilbert spaces, Pacific Journal of Mat., Vol 72, No. 1, (1997)

%\bibitem {Lu1} A. Luger, A factorization of regular generalized Nevanlinna functions, Inetgr. Equ. Oper. Theory 43 (2002) 326-345.

%\bibitem {Lu3} A. Luger, Generalized Nevanlinna Functions: Operator Representations, Asymptotic Behavior, In book: Operator Theory, Chapter 15, pp.345-371 (2014); DOI: 10.1007/978-3-0348-0667-1\textunderscore 35 

% \bibitem{Lu4} A. Luger, A characterization of generalized poles of generalized Nevanlinna functions, Math. Nachr. 279, 891-910 (2006). 

%\bibitem{N} C. Neuner, Generalized Tichmarsh-Weyl functions and super singular perturbations, Ph. D. Thesis, Stockholm University, 2015, \\
%http://su.diva-portal.org/smash/get/diva2:784627/FULLTEXT01.pdf . 

\bibitem {S} P. Sorjonen, On linear relations in an indefinite 
inner product space, Annales Academiae Scientiarum Fennica, Series A. I. Mathematica, Vol.4, 
1978/1979, 169-192 

\bibitem {W} R. Wietsma, Block representations for classes of isometric operators between Krein spaces, Oper. Matrices 7 (2013), 651–685

\end{thebibliography}

%BML, Actuarial Department

%120 Royall Street. 

%Canton, MA 02131,USA

%muhamed.borogovac@gmail.com 

\end{document}

