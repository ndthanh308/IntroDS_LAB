% \documentclass[acmsmall]{acmart}
\documentclass[manuscript,authorversion,nonacm]{acmart}

% \usepackage{pbox}
\usepackage[usestackEOL]{stackengine}
\usepackage{xcolor}
\usepackage{multirow}


\usepackage{lscape}
% \usepackage{refcheck}

\renewcommand{\arraystretch}{1.2}
\usepackage[english=american,threshold=40,thresholdtype=words]{csquotes}


\begin{document}

\strutlongstacks{T}

%%
%% The "title" command has an optional parameter,
%% allowing the author to define a "short title" to be used in page headers.
\title[Method for Generating Dynamic Responsible AI Guidelines]{A Method for Generating Dynamic Responsible AI Guidelines for Collaborative Action}


\author{Marios Constantinides}
\affiliation{
  \institution{Nokia Bell Labs}
  \city{Cambridge}
  \country{UK}
}
\email{marios.constantinides@nokia-bell-labs.com}


\author{Edyta Bogucka}
\affiliation{
  \institution{Nokia Bell Labs}
  \city{Cambridge}
  \country{UK}
}
\email{edyta.bogucka@nokia-bell-labs.com}


\author{Daniele Quercia}
\affiliation{
  \institution{Nokia Bell Labs}
  \city{Cambridge}
  \country{UK}
}
\email{daniele.quercia@nokia-bell-labs.com}

\author{Susanna Kallio}
\affiliation{
  \institution{Nokia}
  \city{Espoo}
  \country{Finland}
}
\email{susanna.kallio@nokia.com}


\author{Mohammad Tahaei}
\orcid{0000-0001-9666-2663}
\affiliation{
  \institution{Nokia Bell Labs}
  \streetaddress{21 J.J. Thomson Avenue}
  \postcode{CB3 0FA}
  \city{Cambridge}
  \country{UK}
}
\email{mohammad.tahaei@nokia-bell-labs.com}


\renewcommand{\shortauthors}{Constantinides et al.}

\begin{abstract}
To improve the development of responsible AI systems, developers are increasingly utilizing tools such as checklists or guideline cards to ensure fairness, transparency, and sustainability. However, these tools face two main challenges. First, they are static and are not meant to keep pace with the latest responsible AI literature and international standards. Second, they tend to prioritize individual usage over fostering collaboration among AI practitioners. To overcome these limitations, we propose a method that enables easy updates of responsible AI guidelines by incorporating research papers and ISO standards, ensuring that the content remains relevant and up to date, while emphasizing actionable guidelines that can be implemented by a wide range of AI practitioners. We validated our method in a case study at a large tech company by designing and deploying a tool that recommends interactive and actionable guidelines, which were generated by a team of engineers, standardization experts, and a lawyer using our method. Through the study involving AI developers and engineers, we assessed the usability and effectiveness of the tool, showing that the guidelines were considered practical and actionable. The guidelines encouraged self-reflection and facilitated a better understanding of the ethical considerations of AI during the early stages of development,  significantly contributing to the idea of ``Responsible AI by Design''---a design-first approach that considers responsible AI values throughout the development lifecycle and across business roles. 
\end{abstract}

%%
%% The code below is generated by the tool at http://dl.acm.org/ccs.cfm.
%% Please copy and paste the code instead of the example below.
%%
\begin{CCSXML}
<ccs2012>
   <concept>
       <concept_id>10003120.10003130.10011762</concept_id>
       <concept_desc>Human-centered computing~Empirical studies in collaborative and social computing</concept_desc>
       <concept_significance>500</concept_significance>
       </concept>
   <concept>
       <concept_id>10003120.10003121.10003129</concept_id>
       <concept_desc>Human-centered computing~Interactive systems and tools</concept_desc>
       <concept_significance>500</concept_significance>
       </concept>
   <concept>
       <concept_id>10010147.10010257</concept_id>
       <concept_desc>Computing methodologies~Machine learning</concept_desc>
       <concept_significance>500</concept_significance>
       </concept>
   <concept>
       <concept_id>10010147.10010178</concept_id>
       <concept_desc>Computing methodologies~Artificial intelligence</concept_desc>
       <concept_significance>500</concept_significance>
       </concept>
 </ccs2012>
\end{CCSXML}

\ccsdesc[500]{Human-centered computing~Empirical studies in collaborative and social computing}
\ccsdesc[500]{Human-centered computing~Interactive systems and tools}
\ccsdesc[500]{Computing methodologies~Machine learning}
\ccsdesc[500]{Computing methodologies~Artificial intelligence}

%%
%% Keywords. The author(s) should pick words that accurately describe
%% the work being presented. Separate the keywords with commas.

\keywords{responsible AI, AI ethics, AI guidelines, system development, co-design}

%% A "teaser" image appears between the author and affiliation
%% information and the body of the document, and typically spans the
%% page.

% \textbf{Responsible AI by Design.}
\begin{teaserfigure}
  % Figure removed
  \caption{(A) An adaptable and updatable method for generating actionable guidelines for responsible AI; (B) design and evaluation of a tool with responsible AI guidelines; (C) tool revisions.}
  \label{fig:card-elements}
\end{teaserfigure}

%%
%% This command processes the author and affiliation and title
%% information and builds the first part of the formatted document.
\maketitle
%%%%%%%%%%%%%%%%%%%%%%%%%%%%%%%%%%%%%%%%%%%%%%%%%%%%%%%%%%%%%%%%%%%%%%%%%%%%%%%%
\section{Introduction}

Autonomous driving (AD) %with deep learning networks 
has shown promising achievements and is considered an important technological breakthrough that could revolutionize the future of transportation. Currently, ensuring the safety of autonomous driving systems has become a topic of extensive development.
% There has been much discussion on how to verify the safety of autonomous driving systems.
One traditional solution for safety tests is to exhaustively enumerate real scenarios for validation. Nevertheless, this process is not only labor-intensive and costly but also dangerous. Simulation has emerged as a robust, safe, and efficient alternative for training and evaluating AD software and algorithms~\cite{li2019aads, amini2020learning, amini2022vista}.

% Figure environment removed

Recently, neural radiance field (NeRF)~\cite{mildenhall2020nerf} has gained significant attention in AD simulation~\cite{drivesim}. This approach leverages multi-view images to construct a 3D scene and enable novel view synthesis for both indoor and outdoor applications. When it comes to constructing NeRF models in AD simulation, there are two options available: 1) collecting a large amount of data to cover as many viewpoints as possible, and constructing a fine-grained scene offline; 2) directly using log data from road tests to quickly create an environment and dynamically simulate driving scenarios. The first choice can deliver high-quality simulation~\cite{tancik2022block} by transforming the problem of view extrapolation into view interpolation through the use of large amounts of data. However, it is time- and cost-intensive, which makes it challenging to generalize. As for the second choice, the collected images from log data are usually similar to each other along the running trajectory, which may result in unsatisfactory outcomes, particularly when the camera pose is placed out-of-trajectory (see \figref{figSupportComp} as an example), semantic consistency cannot be guaranteed when synthesizing images from deviated views. We observe this problem under this data condition in all neural radiance approaches, and to the best of our knowledge, none of the existing work has solved this issue.
In our opinion, semantic consistency is crucial for AD simulation, and synthesizing on deviated views is unavoidable for scalability.

AD simulation usually involves map data for planning and control, which can be obtained from a prebuilt High-Definition Map (HD Map) or an online mapping module. While the map data may not be pixel-perfect, it can provide semantic-level information that is useful for enhancing the semantic consistency of the trained neural radiance field.
In this paper, we propose incorporating map priors into neural radiance fields to enhance the semantic consistency and rendering quality of deviated driving view synthesis. Firstly, we employ ground information from maps to supervise the density field of NeRF, providing a more reliable road base for semantic entities. Next, we propose sampling rays to simulate unseen views. Unlike most NeRF augmentation methods~\cite{zhang2022ray, chen2022geoaug}, we utilize ground and lane information in sampling computations to guide the radiance field. More importantly, we model the above two supervision methods as weak supervision by using an uncertainty parameter and propose an uncertainty tempering scheme to increase the uncertainty. This ensures that map priors only guide the training process rather than enforce it towards their absolute values. As a result, our proposed method not only improves the rendering quality of interpolated novel view synthesis quantitatively but also enhances the semantic consistency of deviated novel view synthesis. 
Our contributions can be summarized as follows:
% We summarize the contributions of this paper as follows.



% To overcome the limitations of the collected data, this paper proposes a novel approach that leverages map information to enhance the semantic consistency of the synthesized driving views. 

% Autonomous driving (AD) vehicles are being trained with the help of deep learning networks and have shown promising achievements. This technology is considered to be a breakthrough that could change the way of transportation in the near future. However, there are many discussions on how to verify or judge the safety of autonomous driving systems.
% A straightforward solution towards the safety tests is to exhaustively enumerate real scenarios for validation as many as possible. However, the process of implementing different real scenarios is not only labor-intensive and costly, but also dangerous. Simulation has been proved to be an alternative, which is robust, safe, efficient in training, and evaluating AD software and algorithms.
% Now, the emerging technology of neural radiance field (NeRF)~\cite{} leverages multi-view images to construct a 3D scene and enable novel view synthesis for many indoor and outdoor applications. For AD simulation, there are two choices for constructing NeRF models: 1) collect a large amount of data, such as LiDAR and camera data, similar to mapping, to construct a fine-grained scene offline; or 2) directly use the log file (typically in the format of ROS bag) to rapidly create an environment and then dynamically simulate the driving scenarios.
% The first choice can achieve high-quality simulation, but it is time-consuming and expensive, making it difficult to generalize to very large scales. On the other hand, the second option is fast but can lead to low-quality simulation due to the data being sparse and similar to each other in log data. This paper tackles the problem raised by choosing the latter option and attempts to improve the quality of out-of-trajectory driving view synthesis by incorporating map information. This approach is practical for many autonomous driving tests.
% In conclusion, the use of NeRF technology for AD simulation is a promising avenue for training and evaluating AD software and algorithms. While both options for constructing NeRF models have their pros and cons, this paper addresses the challenges of the second option and proposes a potential solution to improve the quality of simulation.

%There exist a few attempts to facilitate training a NeRF model for synthesizing out-of-trajectory (or called as extrapo trajectory) views.


\begin{itemize}
    \item We propose a novel method to incorporate commonly used map priors in AD scenes into neural radiance fields to improve the out-of-trajectory driving view synthesis.
    \item We explicitly model the uncertainty in map priors as a parameter and propose an uncertainty tempering scheme to guide the training process of the neural radiance field.
    \item Experiments demonstrated that the proposed method can improve the semantic consistency of out-of-trajectory views and the rendering quality of novel view trajectory interpolation.
\end{itemize}

Our proposed method is easy to implement, can be easily plugged into existing NeRF algorithms, and has the capability of extending to other formats of priors.
\section{Related Work}
\label{sec:related}
We surveyed various lines of research that our work draws upon, and grouped them into two main areas: \emph{i)} AI regulation and governance, and \emph{ii)} responsible AI practices and toolkits. 

\subsection{AI Regulation and Governance}
The landscape of AI regulation and governance is constantly evolving~\cite{jobin2019global, mittelstadt2016ethics}. At the time of writing, the European Union (EU) has endorsed new transparency and risk-management rules for AI systems known as the EU AI Act~\cite{eu_ai_act}, which is expected to become law in 2023. Similarly, the United States (US) has recently passed a blueprint of the AI Bill of Rights in late 2022~\cite{us_ai_bill}. This bill comprises \emph{``five principles and associated practices to help guide the design, use, and deployment of automated systems to protect the rights of the American public in the age of AI.''} While both the EU and US share a conceptual alignment on key principles of responsible AI, such as fairness and explainability, as well as the importance of international standards (e.g., ISO), the specific AI risk management regimes they are developing are potentially diverging, creating an ``artificial divide''~\cite{ecfr}. The EU aims to become the leading regulator for AI globally, while the US takes the view that excessive regulation may impede innovation.

Notable predecessors to AI regulations include the EU GDPR law on data protection and privacy~\cite{eu_gdpr}, the US Anti-discrimination Act~\cite{us_anti_discrimination}, and the UK Equality Act 2010~\cite{uk_equality}. GDPR's Article 25 mandates that data controllers must implement appropriate technical and organizational measures during the design and implementation stages of data processing to safeguard the rights of data subjects. The Anti-discrimination Act prohibits employment decisions based on an individual's race, color, religion, sex (including gender identity, sexual orientation, and pregnancy), national origin, age (40 or older), disability, or genetic information. This legislation ensures fairness in AI-assisted hiring systems. Similarly, the UK Equality Act provides legal protection against discrimination in the workplace and wider society.

The National Institute of Standards and Technology (NIST), a renowned organization for developing frameworks and standards, recently published an AI risk management framework~\cite{nist2023aiRisk}. According to the NIST framework, an AI system is defined as \emph{``an engineered or machine-based system capable of generating outputs such as predictions, recommendations, or decisions that influence real or virtual environments, based on a given set of objectives. These systems are designed to operate with varying levels of autonomy.''} Similarly, the Principled Artificial Intelligence white paper from the Berkman Klein Center~\cite{fjeld2020principled} highlights eight key thematic trends that represent a growing consensus on responsible AI. These themes include privacy, accountability, safety and security, transparency and explainability, fairness and non-discrimination, human control of technology, professional responsibility, and the promotion of human values.

As AI regulation and governance continue to evolve, AI practitioners are faced with the challenge of staying updated with the changing guidelines and regulations, requiring significant time and effort. Therefore, the focus of this work is to develop an adaptable methodology for generating responsible AI guidelines.

\subsection{Responsible AI Practices and Toolkits}
\subsubsection{Responsible AI Practices.} 
\label{sec:sub-raipractices}
A growing body of research, typically discussed in conferences with a long-standing commitment to human-centered design, such as the Conference on Human Factors in Computing Systems (CHI) and the Conference on Computer-Supported Cooperative Work and Social Computing (CSCW), as well as in newer conferences like the Conference on AI, Ethics, and Society (AIES) and the Conference on Fairness, Accountability, and Transparency (FAccT), focuses on the work practices of AI practitioners in addressing responsible AI issues. This strand of research encompasses various aspects of responsible AI, including fairness, explainability, sustainability, and best practices for data and model documentation and evaluation.

Fairness is a fundamental value in responsible AI, but its definition is complex and multifaceted~\cite{narayanan21fairness}. To assess bias in classification outputs, various research efforts have introduced quantitative metrics such as disparate impact and equalized odds, as discussed by Dixon et al.~\cite{dixon2018measuring}. Another concept explored in the literature is ``equality of opportunity,'' advocated by Hardt et al.~\cite{hardt2016equality}, which ensures that predictive models are equally accurate across different groups defined by protected attributes like race or gender.

Explainable AI (XAI) is another aspect of responsible AI. XAI involves tools and frameworks that assist end users and stakeholders in understanding and interpreting predictions made by machine learning models~\cite{arrieta2020explainable, kulesza2015principles, gunning2019xai,liao2021human, ehsan2020human, ibm2019fairness}. Furthermore, the environmental impact of training AI models should also be considered. Numerous reports have highlighted the significant carbon footprint associated with deep learning and large language models~\cite{sharir2020cost, hao2019training, strubell2019energy}. 

Best practices for data documentation and model evaluation have also been developed to promote fairness in AI systems. Gebru et al.~\cite{gebru2021datasheets} proposed ``Datasheets for Datasets'' as a comprehensive means of providing information about a dataset, including data provenance, key characteristics, relevant regulations, test results, and potential biases. Similarly, Bender et al.\cite{bender2018data} introduced ``data statements'' as qualitative summaries that offer crucial context about a dataset's population, aiding in identifying biases and understanding generalizability. For model evaluation, Mitchell et al.~\cite{mitchell2019model} suggested the use of model cards, which provide standardized information about machine learning models, including their intended use, performance metrics, potential biases, and data limitations. Transparent reporting practices, such as the TRIPOD statement by Collins et al.~\cite{collins2015transparent} in the medical domain, emphasize standardized and comprehensive reporting to enhance credibility and reproducibility of AI prediction models.\\

\subsubsection{Responsible AI Toolkits} Translating these practices into practical responsible AI is another area of growing research. New tools and frameworks are being proposed to assist developers in mitigating biases~\cite{bird2020fairlearn, gebru2021datasheets}, explaining algorithmic decisions~\cite{arya2019one}, and ensuring privacy-preserving AI systems~\cite{fjeld2020principled}.

Fairness auditing tools typically offer a set of metrics to test for potential biases, and algorithms to mitigate biases that may arise in AI models~\cite{saleiro2018aequitas, baeza2018bias}. For instance, Google's fairness-indicators toolkit~\cite{google2022fairness} enables developers to evaluate the distribution of datasets, performance of models across user-defined groups, and delve into individual slices to identify root causes and areas for improvement. IBM's AI Fairness 360~\cite{ibm2022ai} implements metrics for comparing subgroups of datasets (e.g., differential fairness and bias amplification~\cite{foulds2020intersectional}) and algorithms for mitigating biases (e.g., learning fair representations~\cite{zemel2013learning}, adversarial debiasing~\cite{zhang2018mitigating}). Microsoft's Fairlearn provides metrics to assess the negative impact on specific groups by a model and compare multiple models in terms of fairness and accuracy metrics. It also offers algorithms to mitigate unfairness across various AI tasks and definitions of fairness~\cite{fairlearn2022}.

Explainable AI systems are typically achieved through interpretable models or model-agnostic methods. Interpretable models employ simpler models like linear or logistic regression to explain the outputs of black-box models. On the other hand, model-agnostic methods (e.g., LIME~\cite{ribeiro2016should} or SHAP~\cite{lundberg2017unified}) have shown effectiveness with any model. IBM's AI Explainability 360 provides metrics that serve as quantitative proxies for the quality of explanations and offers guidance to developers and practitioners on ensuring AI explainability~\cite{ibm2022ai}. Another research direction introduced new genres of AI-related visualizations for explainability, drawing inspiration from domains such as visual storytelling, uncertainty visualizations, and visual analytics. Examples include Google's explorables, which are interactive visual explanations of the internal workings of AI techniques~\cite{google2022pair}; model and data cards that support model transparency and accountability (e.g., NVIDIA's Model Card++)\cite{nvidia2022}; computational notebook additions for data validations like AIF360\cite{ibm2022ai}, Fairlearn~\cite{fairlearn2022}, and Aequitas~\cite{saleiro2018aequitas}; and data exploration dashboards such as Google's Know Your Data~\cite{google2022know} and Microsoft's Responsible AI dashboard~\cite{microsoft2022aiLab}.

Ensuring privacy-preserving AI systems is commonly attributed to the practice of ``Privacy by Design''~\cite{cavoukian2009privacy, cavoukian2010privacy}, which involves integrating data privacy considerations throughout the AI lifecycle, particularly during the design stage to ensure compliance with laws, regulations, and standards~\cite{fjeld2020principled} such as the European General Data Protection Regulation (GDPR)~\cite{eu_gdpr}. IBM's AI Privacy 360 is an example of a toolkit that assesses privacy risks and helps mitigate potential privacy concerns. It includes modules for data anonymization (training a model on anonymized data) and data minimization (collecting only relevant and necessary data for model training) to evaluate privacy risks and ensure compliance with privacy regulations.

While many toolkits and frameworks emphasize the importance of involving stakeholders from diverse roles and backgrounds, they often lack sufficient support for collaborative action. Wong et al.~\cite{wong2023seeing} have also highlighted the ``mismatch between the promise of toolkits and their current design'' in terms of supporting collaboration. Collaboration is key to enhance creativity by allowing AI practitioners to share knowledge with other stakeholders. To address this gap, we aim to develop a set of actionable guidelines that will facilitate the engagement of a diverse range of stakeholders in AI ethics. By doing so, we hope to take a significant step forward in fostering collaboration and inclusivity within the field.
\section{Author Positionality Statement}
\label{sec:positionality}
Understanding researcher positionality is crucial for transparently examining our perspectives on methodology, data collection, and analyses~\cite{frluckaj2022gender, havens2020situated}. In this paper, we situate ourselves in a Western country during the 21\textsuperscript{st} century, writing as authors primarily engaged in academic and industry research. Our team comprises three males and two females from Southern, Eastern, and North Europe, and Middle East with diverse ethnic and religious backgrounds. Our collective expertise spans various fields, including human-computer interaction (HCI), ubiquitous computing, software engineering, artificial intelligence, data visualization, and digital humanities.

It is important to recognize that our backgrounds and experiences have shaped our positionality. As HCI researchers affiliated with a predominantly Western organization, we acknowledge the need to expand the understanding of the research questions and methodology presented in this paper. Consequently, our positionality may have influenced the subjectivity inherent in framing our research questions, selecting our methodology, designing our study, and interpreting and analyzing our data.

\section{Method for Generating Responsible AI Guidelines}
\label{sec:method}
To answer our \textbf{RQ\textsubscript{1}}, we followed a four-step process (Figure~\ref{fig:steps}), based on the methodology proposed by Michie et al.~\cite{michie2013behavior}. This process allowed us to identify the essential element of a guideline, referred to as the ``active ingredient,'' focusing on the ``what'' rather than the ``how''~\cite{michie2011strengthening}. A similar parallel can be drawn in software engineering, where the ``what'' represents the software requirements and the ``how'' represents the software design, both of which are important for a successful software product~\cite{aggarwal2005software}. However, by shifting the focus to the ``what,'' AI practitioners can develop a clearer understanding of the objectives and goals they need to achieve, fostering a deeper comprehension of complex underlying ethical concepts. Throughout this process, we actively engaged a diverse group of stakeholders, including AI engineers, researchers, and experts in law and standardization. As a result, we were able to develop a total of \textbf{22} responsible AI guidelines. \\

% Figure environment removed

\subsection{Compiling a List of Papers on Responsible AI}
In the first step, we compiled a list of key scientific articles focusing on responsible AI guidelines for AI practitioners (discussed in detail in \S\ref{sec:sub-raipractices}). We created this list by targeting influential papers published in renowned computer science conferences, such as the ACM CHI, CSCW, FAccT, AAAI/ACM Conference on AI, Ethics, and Society (AIES), and scientific literature from the medical domain (e.g., the Annals of Internal Medicine). Note that we did not conduct a systematic literature but rather relied on snowball sampling by identifying key publications related to the topic at hand. Overall, we identified 17 key papers that covered a broad range of responsible AI aspects, including fairness, explainability, sustainability, and best practices for data and model documentation and evaluation.

\subsection{Creating a Catalog of Responsible AI Guidelines From the Papers}
For each source, we compiled a list of techniques that could be employed to create responsible AI guidelines, focusing on the actions developers should consider during AI development. Following the methodology proposed by Michie et al.~\cite{michie2013behavior} (which was also used to identify community engagement techniques by Dittus et al.~\cite{dittus2017community}), we sought techniques that describe the ``active ingredient'' of what needs to be done. This means that the phrasing of the technique should focus on what developers need to do (\emph{what}), rather than the specific implementation details (\emph{how}). In total, we formulated a set of 16 techniques based on relevant literature sources~\cite{mitchell2019model, madaio2020co, dixon2018measuring, hardt2016equality, mitchell2018prediction, verma2018fairness, arrieta2020explainable, kulesza2015principles, fjeld2020principled, bender2018data, gebru2021datasheets, holland2018dataset, wang2020revise, collins2015transparent, sharir2020cost, hao2019training}.

For instance, a recommended practice for ensuring fairness involves evaluating an AI system across different demographic groups~\cite{madaio2020co, dixon2018measuring, hardt2016equality}. In this case, the technique specifies ``what'' needs to be done rather than ``how'' it should be implemented (e.g., using common fairness metrics such as demographic parity or equalized odds). We then conducted an iterative review of the collection of techniques to identify duplicates, which were instances where multiple sources referred to the same technique. For example, four sources indicated that data biases could affect the model~\cite{mitchell2019model, gebru2021datasheets, bender2018data, holland2018dataset}, emphasizing the need to report the characteristics of training and testing datasets. We consolidated such instances by retaining the specific actions to be taken (e.g., reporting dataset characteristics). This process resulted in an initial list of 16 distinct techniques. We provided a concise summary sentence for each technique, utilizing active verbs to emphasize the recommended actions for developers.\\

\subsection{Examining the Catalog With AI Developers and Standardization Experts Through Interviews}
The catalog of techniques underwent eleven iterations to ensure clarity and comprehensive thematic coverage. The iterations were carried out by two authors, with the first author conducting interviews with five AI researchers and developers. During the interviews, the participants were asked to consider their current AI projects and provide insights on the implementation of each technique, focusing on the ``how'' aspect. This served two purposes: firstly, to identify any statements that were unclear or vague, prompting suggestions for alternative phrasing; and secondly, to expand the catalog further. The interviews yielded two main recommendations for improvement: \emph{i)} mapping duplicate techniques to the same underlying action(s); and \emph{ii)} adding examples to support each technique.

In addition to the interviews, the two authors who developed the initial catalog conducted a series of six 1-hour workshops with two standardization experts from a large organization. The purpose of these workshops was to review the initial catalog for ISO compliance. The standardization experts examined six AI-related ISOs, including ISO 38507, ISO 23894, ISO 5338, ISO 24028, ISO 24027, and ISO 24368, which were developed at the time of writing (we provide a high-level summary\footnote{Note that the summary provided is a brief and simplified description due to a paywall restriction.} of each ISO next). The experts provided input on any missing techniques and mapped each technique in the initial catalog to the corresponding ISO that covers it. As a result of this exercise, six new techniques (\#2, \#7, \#12, \#13, \#14, \#21) were added to the catalog.\\

\noindent \textbf{ISO 38507 (Governance, 28 pages).} It provides guidance to organizations on how to effectively and responsibly govern the use of AI (e.g., identify potential harms and risks for each intended use(s) of the systems). It offers recommendations to the governing body of an organization, as well as various stakeholders such as executive managers, external specialists, public authorities, service providers, assessors, and auditors. The standard is applicable to organizations of all types and sizes, regardless of their reliance on data or information technologies, addressing both current and future uses of AI and their implications. 

\smallskip
\noindent \textbf{ISO 23894 (Risk Management, 26 pages).} It offers guidance to organizations involved in the development, production, deployment, or use of products, systems, and services utilizing AI to effectively manage AI-related risks (e.g., develop mechanisms for incentivizing reporting of system harms). It provides recommendations on integrating risk management into AI activities and functions, along with describing processes for the successful implementation and integration of AI risk management. The guidance is adaptable to suit the specific needs and context of any organization.

\smallskip
\noindent \textbf{ISO 5338 (AI Lifecycle Process, 27 pages).} It establishes a framework for describing the life cycle of AI systems that rely on machine learning and heuristic systems. It defines processes and concepts (e.g., through reporting of harms and risks, obtaining approval of intended uses) that enable the effective definition, control, management, execution, and enhancement of AI systems throughout their life cycle stages. These processes are applicable to organizations or projects involved in the development or procurement of AI systems, providing a structured approach to their development and implementation. 

\smallskip
\noindent \textbf{ISO 24028 (Trustworthiness, 43 pages).} It provides an overview of trustworthiness in AI systems, covering various aspects. It explores approaches to establish trust in AI systems through transparency, explainability, and controllability. It also addresses potential engineering pitfalls, associated threats, and risks to AI systems, offering mitigation techniques. Additionally, it discusses approaches to assess and ensure the availability, resiliency, reliability, accuracy, safety, security, and privacy of AI systems. However, it does not specify the levels of trustworthiness for AI systems. 

\smallskip
\noindent \textbf{ISO 24027 (Bias, 39 pages).} It focuses on bias (i.e., related to protected attributes, such as age, gender, and ethnicity, being used in the training of AI) in AI systems, particularly in the context of AI-aided decision-making. It provides techniques and methods for measuring and assessing bias, with the objective of identifying and addressing vulnerabilities related to bias. The standard covers all phases of the AI system lifecycle, encompassing data collection, training, continual learning, design, testing, evaluation, and use. 

\smallskip
\noindent \textbf{ISO 24368 (Ethical and Societal Concerns, 48 pages).} It offers a broad introduction to ethical and societal concerns related to AI. It provides information on principles, processes, and methods in this field and is aimed at technologists, regulators, interest groups, and society as a whole. It does not promote any specific set of values or value systems. Additionally, the document provides an overview of existing International Standards that tackle issues arising from ethical and societal concerns in AI. \\

While our method is comprehensive, it is important to note that the responsible AI guidelines were checked against six ISOs in their current form. However, the flexibility of our approach allows for amending or adding new responsible AI techniques as scientific literature advances and ISO standards evolve. At the time of writing, there are additional ISOs that are in the committee draft stage and can be included in the guidelines, such as ISO 42001 (AI management system), ISO 5469 (Functional safety), ISO 5259 (Data quality), ISO 6254 (Explainability), and ISO 12831 (Testing). 

\begin{table}
\caption{\textbf{Responsible AI guidelines.} 22 techniques that describe a responsible AI guideline (i.e., an actionable item that a developer should consider during the AI development lifecycle). These techniques are grounded in the scientific literature (main sources are reported), and were checked for ISO compliance: ISO 38507 (Governance); ISO 23894 (Risk management); ISO 5338 (AI lifecycle processes); ISO 24028 (Trustworthiness); ISO 24027 (Bias); ISO 24368 (Ethical considerations). They were also cross-referenced with the EU AI Act's articles~\cite{eu_ai_act_2022}, following guidance from~\cite{golpayegani2023high}. Each technique is followed by an example, and the techniques are categorized thematically into six categories, concerning the \emph{intended uses, harms, system, data, oversight, and team}. }
\label{tbl:techniques}
\resizebox{\textwidth}{!}{%
\begin{tabular}{lllll}
\toprule
\textbf{Number} & \textbf{Technique} & \textbf{Source(s)} & \textbf{ISO} & \textbf{AI Act}\\ \midrule
\multicolumn{2}{l}{\textbf{INTENDED USES}} &  &  \\
1 & \begin{tabular}[t]{l}Work with relevant parties to identify intended uses. \\ (e.g., identify the system's usage, deployment, and contextual conditions)\end{tabular} &
\begin{tabular}[t]{l} \cite{mitchell2019model} \end{tabular}
 & \begin{tabular}[t]{l} 5338, 38507, 23894, \\ 24027, 24368\end{tabular} & Art. 6\\

2 & \begin{tabular}[t]{l}Obtain approval from an Ethics Committee or similar body for intended uses.\\ (e.g., Obtain Ethics Committee approval for the intended use, aligned with sustainability goals)\end{tabular} & \begin{tabular}[t]{l} ---\end{tabular} & \begin{tabular}[t]{l}38507, 5338, 23894\end{tabular} & Art. 6, 9 \\ 

\midrule
\multicolumn{2}{l}{\textbf{HARMS}} &  &  \\
3 & \begin{tabular}[t]{l}Identify potential harms and risks associated with the intended uses. \\  (e.g., prevent privacy violation, discrimination, and adversarial attacks, provide interpretable output)\end{tabular} & 
\begin{tabular}[t]{l} \cite{madaio2020co} \end{tabular} & \begin{tabular}[t]{l}23894, 24028, 38507, \\ 24368\end{tabular}  & Art. 9\\

4 & \begin{tabular}[t]{l}Provide mechanism(s) for incentivizing reporting of system harms. \\  (e.g.,provide contact emails and feedback form to raise concerns)\end{tabular} &
\begin{tabular}[t]{l} \cite{madaio2020co} \end{tabular} & \begin{tabular}[t]{l}38507, 23894\end{tabular} & Art. 9\\

5 & \begin{tabular}[t]{l}Develop strategies to mitigate identified harms or risks for each intended use. \\ (e.g., use stratified sampling and safeguards against adversarial attacks during training)\end{tabular} &
\begin{tabular}[t]{l} \cite{mitchell2019model} \end{tabular}
 & \begin{tabular}[t]{l}24368, 23894\end{tabular} & Art. 9\\ 

\midrule
\multicolumn{2}{l}{\textbf{SYSTEM}} &  &  \\
6 & \begin{tabular}[t]{l}Document all system components, including the AI models, to enable reproducibility and scrutiny. \\ (e.g., create UML diagrams, flowcharts, and specify model types, versions, hardware architecture)\end{tabular} & \begin{tabular}[t]{l} \cite{madaio2020co} \end{tabular} & \begin{tabular}[t]{l}5338, 23894, 24027\end{tabular} & Art. 15\\

7 & \begin{tabular}[t]{l}Review the code for reliability\\ (e.g., manage version control using software.)\end{tabular} & \begin{tabular}[t]{l} --- \end{tabular} & \begin{tabular}[t]{l}5338\end{tabular} & Art. 15\\

8 & \begin{tabular}[t]{l}Report evaluation metrics for various groups based on factors such as age, gender, and ethnicity. \\ (e.g., evaluate false positive/negative, AUC, and feature importance across protected attributes)\end{tabular} & 
\begin{tabular}[t]{l} \cite{madaio2020co, dixon2018measuring, hardt2016equality} \\ \cite{mitchell2018prediction, verma2018fairness}\end{tabular} & \begin{tabular}[t]{l}23894, 5338, 24028, \\ 24027\end{tabular} & Art. 10, 13, 15\\

9 & \begin{tabular}[t]{l} Provide mechanisms for interpretable outputs and auditing. \\ (e.g., output feature importance and provide human-understandable explanations)\end{tabular} & 
\begin{tabular}[t]{l} \cite{arrieta2020explainable, kulesza2015principles}  \end{tabular} & \begin{tabular}[t]{l}38507, 24028\end{tabular} & Art. 13\\

10 & \begin{tabular}[t]{l}Document the security of all system components in consultation with experts. \\ (e.g., guard against adversarial attacks and unauthorized access)\end{tabular} & \begin{tabular}[t]{l} \cite{fjeld2020principled} \end{tabular} & \begin{tabular}[t]{l}24028, 24368\end{tabular} & Art. 15\\

11 & \begin{tabular}[t]{l}Provide an environmental assessment of the system. \\ (e.g., report the number of GPU hours used in training and deployment)\end{tabular} &
\begin{tabular}[t]{l} \cite{sharir2020cost, hao2019training} \end{tabular}
 & \begin{tabular}[t]{l}38507, 23894,\\ 5338, 24368\end{tabular} & Art. 15, 17\\

12 & \begin{tabular}[t]{l}Develop feedback mechanisms to update the system. \\ (e.g., provide contact email, feedback form, and notification of new knowledge extracted)\end{tabular} & \begin{tabular}[t]{l} ---\end{tabular} & \begin{tabular}[t]{l}24028\end{tabular} & Art. 15\\

13 & \begin{tabular}[t]{l}Ensure safe system decommissioning.\\ (e.g., ensure decommissioned data is either deleted or restricted to authorized personnel.)\end{tabular} & \begin{tabular}[t]{l} ---\end{tabular} & \begin{tabular}[t]{l}38507, 24368\end{tabular} & Art. 10, 15\\

14 & \begin{tabular}[t]{l}Redocument model information and contractual requirements at every system update.\\  (e.g., update the model information when re-training the system or using datasets with new contractual requirements)\end{tabular} & \begin{tabular}[t]{l} ---\end{tabular} & \begin{tabular}[t]{l}23894, 5338, 24368 \end{tabular} & Art. 10, 15\\ 

\midrule
\multicolumn{2}{l}{\textbf{DATA}} &  &  \\

15 & \begin{tabular}[t]{l}Ensure compliance with agreements and legal requirements when handling data. \\  (e.g., create data sharing and non-disclosure agreements and secure servers)\end{tabular} & \begin{tabular}[t]{l} --- \end{tabular} & \begin{tabular}[t]{l}38507, 23894, 5338 \end{tabular} & Art. 10\\ 

16 & \begin{tabular}[t]{l}Compare the quality, representativeness, and fit of training and testing datasets with the intended uses. \\  (e.g., report dataset details such as public/private, personal information, demographics, and data provenance)\end{tabular} & 
\begin{tabular}[t]{l}\cite{bender2018data, gebru2021datasheets, holland2018dataset, wang2020revise}\\ \cite{madaio2020co, mitchell2018prediction, verma2018fairness}\end{tabular}
& \begin{tabular}[t]{l}38507, 5338,\\ 24028, 24027\end{tabular} & Art. 10 \\

17 & \begin{tabular}[t]{l}Identify any measurement errors in input data and their associated assumptions. \\ (e.g., account for potential input errors in the input device, text data, audio, and video)\end{tabular} & \begin{tabular}[t]{l} \cite{collins2015transparent} \end{tabular} & \begin{tabular}[t]{l}38507\end{tabular} & Art. 10 \\

18 & \begin{tabular}[t]{l}Protect sensitive variables in training/testing datasets. \\ (e.g., protect sensitive data and use techniques such as k-anonymity and differential privacy)\end{tabular} & \begin{tabular}[t]{l} \cite{dworkdifferential} \end{tabular} & \begin{tabular}[t]{l}38507, 24028\end{tabular} & Art. 10 \\ 

\midrule
\multicolumn{2}{l}{\textbf{OVERSIGHT}} &  &  \\
19 & \begin{tabular}[t]{l}Continuously monitor metrics and utilize guardrails or rollbacks to ensure the system's output stays within a desired range. \\ (e.g., validate against concept drift and test with diverse testers and compliance and adversarial cases)\end{tabular} & \begin{tabular}[t]{l} \cite{fjeld2020principled} \end{tabular}  & \begin{tabular}[t]{l}38507, 5338, 24028, \\ 24027, 24368\end{tabular} & Art. 14 \\

20 & \begin{tabular}[t]{l}Ensure human control over the system, particularly for designers, developers, and end-users.  \\ (e.g., include human in the loop with the ability to inspect data, models, and training methods)\end{tabular} & \begin{tabular}[t]{l} ---\end{tabular} & \begin{tabular}[t]{l}38507, 5338,\\ 24028, 24368\end{tabular} & Art. 14\\ 

\midrule
\multicolumn{2}{l}{\textbf{TEAM}} &  &  \\

21 & \begin{tabular}[t]{l}Ensure team diversity.  \\ (e.g., consider diversity in gender, neurotypes, personality traits, and thinking styles)\end{tabular} & \begin{tabular}[t]{l} --- \end{tabular} & \begin{tabular}[t]{l}38507, 5338,\\ 24028, 24368\end{tabular} & Art. 17\\ 

22 & \begin{tabular}[t]{l}Train team members on ethical values and regulations. \\ (e.g., train on privacy regulations, ethical issues, and raising concerns)\end{tabular} & \begin{tabular}[t]{l} \cite{fjeld2020principled}\end{tabular}  & \begin{tabular}[t]{l}38507, 24368\end{tabular} & Art. 17\\ 

\bottomrule
\end{tabular}%
}
\end{table}

\subsection{Revising the Catalog}
In response to the interviews with AI developers and standardization experts, we incorporated an example for each guideline. For instance, under the guideline on system interpretability (guideline \#9), the example provided reads: ``output feature importance and provide human-understandable explanations.'' Furthermore, we simplified the language by avoiding domain-specific or technical jargon. We also categorized each guideline into six thematically distinct categories, namely \emph{intended uses}, \emph{harms}, \emph{system}, \emph{data}, \emph{oversight}, and \emph{team}.

Recognizing that certain guidelines may only be applicable at specific stages (e.g., monitoring AI after deployment), we assigned them to three phases based on previous research (e.g., \cite{madaio2020co, mitchell2019model}). These phases are development (designing and coding the system), deployment (transferring the system into the production stage), and use (actual usage of the system). For example, guidelines like identifying the system's intended uses (guideline \#1) are relevant to all three phases, while those related to system updates (guideline \#14) or decommissioning (guideline \#13) are applicable during the use phase. The revised and final catalog, consisting of 22 unique guidelines, is presented in Table~\ref{tbl:techniques}. To ensure the timeliness and relevance of our guidelines, we cross-referenced them with the articles of the EU AI Act~\cite{eu_ai_act_2022} by following guidance provided by Golpayegani et al.~\cite{golpayegani2023high}.
\section{Evaluation of a tool using our Responsible AI Guidelines}
\label{sec:userstudy}

\subsection{Populating a Tool Using Our Responsible AI Guidelines}

\subsubsection{Eliciting Requirements of the Tool Through a Formative Study}
To determine the design requirements for implementing our 22 guidelines into a responsible AI tool, we conducted a formative study involving interviews with AI practitioners. We used open-ended questions during these interviews to gather insights and perspectives from the participants. \\

\noindent\textbf{Participants and Procedure.}
We recruited 10 AI practitioners (4 females and 6 males) who were in their 30s and 40s and employed at a large tech company. The participants had a range of work experience, spanning from 1 to 8 years, and were skilled in areas such as data science, data visualization, UX design, natural language processing, and machine learning. Participants were asked to consider their ongoing AI projects, and the interviewer guided them through the 22 guidelines. The interviewer prompted participants to reflect on how these guidelines could be incorporated into an interactive responsible AI tool. \\

\noindent\textbf{Design Requirements.}
By conducting an inductive thematic analysis of the interview transcripts~\cite{saldana2015coding, miles1994qualitative, mcdonald2019reliability, braun2006thematic}, two authors identified three key design requirements, supported by quotes from our participants (referred to as FP). These requirements pertained to \textbf{unpacking complexity of AI Ethics}, to \textbf{increasing awareness}, and to \textbf{providing examples and reading materials}: The first requirement was about offering an easy-to-use functionality that unpacks the complexity of AI Ethics. The tool's functionality should be user-friendly, allowing for effortless interaction and navigation through complex terminology to enhance the overall user experience. As expressed by FP5, \emph{``the sheer number of the guidelines is the main difficulty [...] they should be separated in bite-sized questions and allow me to understand the complex terminology used''}. To assist users in systematically moving through the guidelines, FP9 suggested that \emph{``the system should provide clear navigation [...] for example, using a progress bar.''}. The second requirement was about increasing awareness. The tool should increase users' awareness of ethical considerations. FP5 emphasized the importance of \emph{``gaining insights while engaging with the 22 guidelines,''} while FP8 described this need as having \emph{``visual feedback or a score that shows how responsible [their] AI system is.''}. Yet, the user FP2 suggested that the implementation of the feedback \emph{``should not make me anxious and feel like I have not done enough''}. FP5 also recommended that the tool should store user's answers and produce a documentation of their tool experience: \emph{``there should be some functionality there that captures the answers I gave, so it'd help me reflect''}. The third requirement was about providing examples and reading materials. The tool should incorporate examples to assist users in comprehending and effectively utilizing the system. FP9 suggested that \emph{``references to these guidelines or practical examples could be added. These additions would enhance the sense of credibility.''} \\

% Figure environment removed

\subsubsection{Designing and Populating the Tool} We then describe the content and design choices, as well as the tool's flow and interactions.

\smallskip
\noindent\textbf{Content and Design Choices.} Using the 22 guidelines (Table~\ref{tbl:techniques}), we designed and developed an interactive responsible AI tool. Each guideline is presented as a rectangular box, with both the front and back sides being interactive boxes---creating a notion of a digital card. The front side includes a symbolic graphic collage representing the guideline, a brief guideline name, and a concise textual description. On the other hand, the back side includes an example illustrating how the guideline can be applied in an AI system, along with input fields where users can document their specific implementation of the guideline within their context (Figure~\ref{fig:card-elements}).

Digital cards often replicate the appearance and interactions of physical cards, allowing for gestures like stacking, shuffling, and swiping~\cite{cardInteractions_2020}. In our interactive responsible AI tool, guidelines can be viewed from both sides by using the flipping button located in the bottom-left corner, and users have the option to put the guidelines back into a stack for further consideration. We explored different layout options for displaying the guidelines, considering previous research that involved scrolling through a deck or organizing them into multiple groups~\cite{dittus2017community}. However, due to the limited screen size and repetitive guidelines for each phase, we opted to stack the guidelines into three groups based on the phase of the AI system: \emph{i)} development (designing and coding), \emph{ii)} deployment (transitioning into production), and \emph{iii)} use (actual usage of the system). The number of guidelines in each group varied: 20 for development, 19 for deployment, and 21 for use (\S\ref{sec:method}, Step 4) to accommodate the specific requirements of each phase. \\

\noindent\textbf{Flow and Interactions.} The interactive tool includes two follow-up questions for each guideline, as shown in Figure~\ref{fig:game-sorting}. These questions offer users a systematic approach to consider each guideline within a specific context of their own projects. The first question asks the developer whether the guideline has been successfully implemented in their AI system. For example, a guideline related to fairness asks the developer to consider if they have reported evaluation metrics for various groups based on factors such as age, gender, and ethnicity (technique \#8 in Table~\ref{tbl:techniques}). This prompts the developer to evaluate whether fairness has been addressed in their AI system. If the developer answers ``yes,'' they are then prompted to provide specific details on how fairness was implemented. Upon sharing this information, the tool moves the guideline to the ``successfully implemented'' stack. In contrast, if the developer answers ``no,'' the tool asks a second follow-up question regarding whether the guideline should be implemented in a future iteration of the AI system. If the developer answers ``yes,'' they are prompted to provide specific details on how to implement it. The tool then moves the guideline to the ``should be considered'' stack. However, if the developer answers ``no'' to both questions, indicating that the guideline is not applicable to their AI system, the tool moves the guideline to the ``inapplicable'' stack.

The layout of the tool consists of three sections, as shown in Figure~\ref{fig:ui-sections}. In the first section, users can enter the name of the developed AI system (Figure \ref{fig:ui-sections}A) and select the phase it belongs to (Figure \ref{fig:ui-sections}B). Once the phase is selected, the second section displays a stack of guidelines (Figure \ref{fig:ui-sections}C). As users interact with the stack, a counter on the left side color-codes the guidelines and indicates their assignment to the three stacks. The counter also shows the number of remaining guidelines. Blue leaves represent guidelines that have been successfully used, magenta leaves represent guidelines for future considerations, and empty leaves represent inapplicable guidelines. After completing the sorting process, the user is presented with an automatically generated report (available for download as a PDF) that separates the guidelines into the three distinct stacks (Figure \ref{fig:ui-sections}D). If desired, the user can repeat the guideline sorting procedure for other phases (Figure \ref{fig:ui-sections}E).

% Figure environment removed

\subsection{Evaluating the Tool’s Usability and Revising It}
To answer our \textbf{RQ\textsubscript{2}} and evaluate the tool, we conducted an interview study with 14 additional AI researchers and developers. 

\subsubsection{Participants}
We recruited participants from the same large research-intensive technology company.\footnote{Participants who took part in the formative study were not eligible to participate in this evaluation study.} The recruitment process took place in October and November 2022. All participants had significant expertise in AI, including areas such as machine learning, deep learning, and computer vision. Additionally, each participant was actively involved in at least one ongoing AI project during the time of the interviews. Table~\ref{tab:demographics} summarizes participants' demographics.


\begin{table}
\centering
\caption{User study participants' demographics.}
\label{tab:demographics}
\resizebox{\textwidth}{!}{%
\begin{tabular}{llllll}
\toprule
\textbf{ID} & \textbf{Gender} & \textbf{Yrs of expr. In AI} & \textbf{Education} & \textbf{Current continent} & \textbf{Expertise} \\ \midrule
1 & Male & 6 & Ph.D. & EU & Deep learning, computer vision \\
2 & Male & +10 & Ph.D. & North America & Machine learning, computer vision \\
3 & Male & 8 & Ph.D. & EU & Machine learning \\
4 & Male & 4 & Ph.D. & North America & Deep learning, IoT, computer vision \\
5 & Female & 5 & Ph.D. & EU & Machine learning \\
6 & Female & 8 & Ph.D. & EU & Computer vision \\
7 & Male & 2 & Ph.D. & North America & Computer vision \\
8 & Male & 10 & Ph.D. & EU & Machine learning \\
9 & Male & 4 & Ph.D. & North America & Computer vision \\
10 & Male & +10 & M.S. & EU & Machine learning, natural language processing \\
11 & Male & +10 & Ph.D. & EU & Machine learning \\
12 & Male & 6 & Ph.D. & EU & Machine learning \\
13 & Male & 4 & Ph.D. & EU & Reinforcement learning, decision making \\
14 & Male & 8 & Ph.D. & EU & Computer vision, robotics \\ \bottomrule
\end{tabular}%
}
\end{table}

\subsubsection{Procedure}
Ahead of the interviews, we sent an email to all participants, providing a concise explanation of the study along with a brief demographics survey. The survey consisted of questions regarding participants' age, domain of expertise, and years of experience in AI system development. The survey is available in Appendix~\ref{app:demographics-survey}. It is important to note that our organization approved the study, and we adhered to established guidelines for user studies, ensuring that no personal identifiers were collected, personal information was removed, and the data remained accessible solely to the research team.

During the interview session, we presented our tool to the participants and allocated 20 minutes (or less if they completed the task sooner) for them to interact with the guidelines. To make the scenario as realistic as possible, we encouraged participants to reflect on their ongoing AI projects and consider how the guidelines could be applied in those specific contexts. Subsequently, we administered the System Usability Scale (SUS)~\cite{brooke1996sus} to assess the usability of the tool. We further engaged participants by asking about their preferences, dislikes, and the relevance of the guidelines to their work. We also sought their suggestions for improvements and potential use cases for the tool. The session concluded with a discussion on enhancing the tool's usability through shared ideas and insights.

We piloted our study with two researchers (1 female, 1 male), which helped us make minor changes to the study guide (e.g., clarifying question-wording and changing the order of questions for a better interview flow). These pilot interviews were not included in the analysis.

\subsubsection{Analysis}
Two authors conducted an inductive thematic analysis (bottom-up) of the interview transcripts, following established coding methodologies~\cite{saldana2015coding, miles1994qualitative, mcdonald2019reliability}. The authors used sticky notes on the Miro platform~\cite{miro2022} to capture the participants' answers, and collaboratively created affinity diagrams based on these notes. They held seven meetings, totaling 14 hours, to discuss and resolve any disagreements that arose during the analysis process. Feedback from the last author was sought during these meetings. In some cases, a single note was relevant to multiple themes, leading to overlap between themes. All themes included quotes from at least two participants, indicating that data saturation had been achieved~\cite{guest2006many}. As a result, participant recruitment was concluded after the $14^{\text{th}}$ interview. The resulting themes, along with their corresponding codes, are provided in Table~\ref{tab:codebook-interviews} in the Appendix.

\subsubsection{Results}
\label{sec:users-results}
First, we present the results regarding the usability and effectiveness of our tool. Then, we provide feedback from our participants regarding potential improvements to the tool. \\

\noindent \textbf{Usability and Effectiveness}. The guidelines were generally well-received by the participants, with a majority considering them as a valuable tool for raising awareness and facilitating self-learning about responsible AI (12 out of 14 participants). For example, one participant expressed, \blockquote[P3]{It made me reflect on my previous choices and how I would describe my decisions when I had to develop the system.} Additionally, seven participants acknowledged the usefulness of the provided examples, which helped them think about potential scenarios and make the guidelines more actionable. Some participants also appreciated the visual simplicity of the guidelines (mentioned by 3 out of 14 participants) and the sequential flow of information, which allowed them to have a more pleasant experience and sufficient time to digest the information (mentioned by 2 out of 14 participants).

Participants, on average, rated the guidelines' usability with a score of 66 out of 100 in SUS, with a standard deviation of 16.01. This indicates a generally positive user experience~\cite{sauro2011practical}. However, participants also identified areas where improvements could be made, which we discuss next.\\

\noindent \textbf{Improvements.} 
Although participants found the guidelines to be a valuable starting point for reflection and engagement with AI ethical considerations, they also provided recommendations for improvement. Many participants expressed the desire for guidelines tailored to their specific project and role (10/14): \blockquote[P6]{It would be helpful if the guidelines were tailored to the specific tasks or challenges I encounter in my project, such as processing the data set and seeking feedback from other people.} Additionally, five participants suggested making the tool more collaborative, allowing them to engage with other experts and stakeholders: \blockquote[P1]{I appreciated that the guidelines emphasized the importance of seeking support from experts when needed. It would be great if the tool facilitated collaboration and discussions with other stakeholders.} Some participants found the tool's summary to be less useful and were uncertain about its future application (3/14), while others expressed the need to keep the examples visible throughout the process (currently, they disappear after consideration).

% Figure environment removed

\subsubsection{Revisions}
Based on the feedback, we made three revisions to our tool (Figure~\ref{fig:prompts-revision}). That is, we: \emph{a)} added roles; \emph{b)} implemented features to foster collaboration; and \emph{c)} improved user experience.\\

\noindent\textbf{Adding Roles.} To assign roles to each guideline (Table~\ref{tbl:techniques_roles} in the Appendix), we referred to previous literature that focused on understanding the best practices of AI practitioners and the development and evaluation of responsible AI toolkits. Wang et al.~\cite{wang2023designing} interviewed UX practitioners and responsible AI experts to understand their work practices. UX practitioners included designers, researchers, and engineers, while responsible AI experts included ethics advisors and specialists. Wong et al.~\cite{wong2023seeing} analyzed 27 ethics toolkits to identify the intended audience of these toolkits, specifically those who are expected to engage in AI ethics work. The intended audience roles identified included software engineers, data scientists, designers, members of cross-functional or cross-disciplinary teams, risk or internal governance teams, C-level executives, and board members. Additionally, Madaio et al.~\cite{madaio2020co} co-designed a fairness checklist with a diverse set of stakeholders, including product managers, data scientists and AI/ML engineers, designers, software engineers, researchers, and consultants. Following guidance therefore from these studies~\cite{wang2023designing,wong2023seeing,madaio2020co}, we formulated three roles as follows:
\begin{enumerate}
    \item Decision-maker or Advisor: This role includes individuals such as product managers, C-suite executives, ethics advisors/responsible AI consultants, and ethical board members.
    \item Engineer or Researcher: This role includes AI/ML engineers, AI/ML researchers, data scientists, software engineers, UX engineers, and UX researchers.
    \item Designer: This role includes interaction designers and UX designers.
\end{enumerate} 

\noindent\textbf{Fostering Collaboration.} Our participants also recognized the potential to enhance the collaborative nature of the tool, allowing them to share knowledge with other stakeholders. To achieve this, we implemented two user interface features. First, we introduced a feature that enables users to keep a history of their interactions with the tool by revising their answers and tracking the changes made. When users revise the content of the guidelines, the newly generated content is stored in a ``responsible AI knowledge base.'' This functionality allows distributed teams to leverage this content, fostering a shared understanding of the project at hand. Additionally, the interface utilizes color coding to indicate the relevance or irrelevance of guidelines at different points in time. Second, we redesigned the PDF summary to include responsible AI blindspots, which encompass specific actions to be taken and shared among the development team. This improvement enhances the utility of the summary by highlighting areas that require attention and providing actionable insights.
\smallskip

\noindent\textbf{Improving User Experience.} While our participants appreciated the simplicity of the user interface, we modified our initial idea of mimicking physical interactions, such as flipping. This decision was made based on feedback from users who found it challenging to remember the guideline while writing the corresponding action after the guideline was flipped. Instead, in the revised version of the tool, we adopted a simplified approach. Both the guideline and its corresponding example(s) are now visible at all times to ensure better usability. We achieved this by dividing each guideline into two side-by-side parts: the left side displays the guideline, while the right side presents its examples in interactive boxes. By making both the guideline and examples consistently visible, users can easily refer to the information they need while formulating their responses. This design change aims to improve the user experience and enhance the effectiveness of the tool in guiding responsible AI practices. \\



\section{Discussion}
\label{sec:discussion}
To assist AI practitioners in navigating the rapidly evolving landscape of AI ethics, governance, and regulations, we have developed a method for generating actionable guidelines for responsible AI. This method enables easy updates of guidelines based on research papers and ISO standards, ensuring that the content remains relevant and up-to-date. We validated this method through a use case study at a large tech company, where we designed and evaluated a tool that uses our responsible AI guidelines. We conducted a formative study involving 10 AI practitioners to design the tool, and further evaluated it through an interview study with an additional 14 AI practitioners. The results indicate that the guidelines were perceived as practical and actionable, promoting self-reflection and enhancing understanding of the ethical considerations associated with AI during the early stages of development. In light of these results, we discuss how our method contributes to the idea of ``Responsible AI by Design'', that is, a design-first approach that considers responsible AI values throughout the development lifecycle and across business roles. We discuss the inherent problem of decontextualization in responsible AI toolkits, the concept of meta-responsibility, and provide practical recommendations for designing responsible AI toolkits with the aim of fostering collaboration and enabling organizational accountability.

\subsection{Theoretical Implications}

\subsubsection{Decontextualization}
The inherent challenge in responsible AI toolkits lies in their attempt to reconcile the tension between scalability and context specificity~\cite{wong2023seeing}. Traditional approaches to toolkit development have often favored a universal, top-down approach that assumes a one-size-fits-all solution~\cite{kelty_participatory_toolkit, mattern_toolkit}. However, participatory development, such as the methodology we followed in designing and populating a responsible AI toolkit with our guidelines, emphasizes the importance of tailoring responsible AI guidelines to specific contexts and job roles needs. It is crucial therefore to recognize that different AI practitioners, such as designers, developers, engineers, and advisors, have distinct requirements and considerations that cannot be treated as identical. This highlights the complexity of developing toolkits that cater to a diverse range of practitioners while accounting for their unique roles and settings---the problem of decontextualization in responsible AI toolkits~\cite{wong2023seeing}.

To tackle the problem of decontextualization, our proposed method incorporates two key elements: \emph{actionable guidelines} and \emph{follow-up questions}. Firstly, the integration of actionable guidelines, tailored to different roles and projects, provides practical steps and recommendations that technical practitioners can easily implement, or C-level executives can make informed decisions upon. These guidelines serve as a starting point for ethical decision-making throughout the AI lifecycle, contributing to the vision of responsible AI by design (borrowing from the idea of `privacy by design'\footnote{``Privacy by design'' is a standard practice for incorporating data protection into the design of technology. In other words, data protection is achieved when it is already integrated into the technology during its design and development~\cite{cavoukian2009privacy}.}). Secondly, the inclusion of follow-up questions enhances our toolkit's ability to capture the complexities of different social and organizational contexts. Expanding upon the concept that follow-up questions are an effective means of communication~\cite{weger2014relative}, as they help in gaining deeper insights, clarifying responses, and uncovering underlying meanings, AI practitioners can engage with these questions to explore the ethical considerations and challenges that are unique to their deployment context.

\subsubsection{Meta-responsibility} Scholars have long recognized the need for a socio-technical approach that considers the contextual factors governing the use of AI systems, including social, organizational, and cultural factors~\cite{tahaei2023toward}. In fact, Ackerman~\cite{ackerman2000intellectual} introduced the concept of socio-technical gap to highlight the disparity between human requirements in technology deployment contexts (socio-requirements) and the technical solutions. This gap arises due to the flexible and nuanced nature of human activity compared to the rigid and brittle nature of computational mechanisms, resulting from necessary formalization and abstraction. Along these lines, \citet{stahl2023embedding} introduced the concept of meta-responsibility to stress that AI systems should be viewed as systems of systems (ecosystems) rather than single entities. To establish a regime of meta-responsibility, Stahl argued for an adaptive governance structure to effectively respond to new insights and external influences (e.g., upcoming AI regulation), and for a knowledge base that equips AI stakeholders with technical, ethical, legal, and social understanding. By integrating ethical, legal, and social knowledge into the AI development process---what Stahl referred to as adaptive governance structure, and offering recommendations for areas that require additional attention (i.e., responsible AI blindspots), our work contribute to this line of research by providing empirical evidence to it and pushing the theoretical boundaries further.

\subsection{Practical Implications}

\subsubsection{Recommendations for designing responsible AI toolkits}
Our responsible AI guidelines, populated in a usable tool, leverage the concept of nudging to encourage users to consider the ethical implications of AI systems. Nudging has demonstrated effectiveness in various domains, such as mitigating the dissemination of misinformation on social media through the use of checklists~\cite{jahanbakhsh2021exploring}, or guiding users towards more private and secure choices~\cite{acquisti2017nudges, tahaei2021deciding}.

Nudges can be implemented in various ways. For instance, the \emph{confront} type of nudge incorporates elements of ``reminding consequences'' and ``providing multiple viewpoints,'' encouraging users to consider alternative directions and diverse perspectives~\cite{caraban2019ways}. In the case of our guidelines, these two concepts are utilized to remind AI developers about the ethical considerations of AI systems and to prompt them to think critically about alternative viewpoints, thus helping them avoid confirmation bias. Further research could explore additional types of nudges, such as incorporating visual cues (e.g., \emph{just-in-time nudges} within development tools), facilitating positive behavior (e.g., \emph{enabling social comparisons} by recognizing and appreciating developers who promote ethical values within the organization), or fostering empathy (e.g., \emph{instigating empathy} by presenting the environmental impact of an AI system through easily understandable animations).

While the format of our tool proved to be useful, it offers a starting point to explore other formats and interactions for populating and contextualizing the guidelines. For example, structuring the guidelines into a narrative might be useful to unpack the complexity of particularly complex guidelines, such as guideline \#15---\emph{ensuring compliance with agreements and legal requirements when handling data.} This guideline can be further sub-divided into sequential steps providing more context and explanations. Moreover, future responsible AI tools can incorporate configurable parameters or customization widgets to align with specific requirements of the developed AI systems or user preferences. Additionally, the use of Language Models (LLMs) can be explored to further customize and adapt the provided examples within the tool. Finally, more research can be done on exploring responsible AI tools as a method for artifact creation. This includes automatic generation of summary reports, model cards, or responsible AI certificates.

\subsubsection{Recommendations for fostering collaboration and enabling organizational accountability}
While individual adoption of responsible AI best practices is crucial, promoting collaboration among diverse AI stakeholders is equally important. Many existing responsible AI toolkits prioritize individual usage~\cite{wong2023seeing}. However, addressing complex ethical and societal challenges associated with AI systems requires collaborative actions. Our interactive tool populated with actionable guidelines addresses this need by offering features that facilitate collaboration. First, the tool stores users' inputs in a responsible AI knowledge base, enabling distributed teams to access and leverage this knowledge for a shared understanding of a particular AI system. This promotes collaboration and a collective approach to ethical considerations. Second, the tool automatically generates a report that summarizes the user's considerations. This report can be downloaded as a PDF and includes responsible AI blindspots, which are specific actions to be taken by individuals or shared among the development team. Highlighting these blindspots fosters awareness and prompts collective action towards responsible AI practices.

In addition to fostering collaboration, our interactive tool can be used to enable organizational accountability. Similar to Google's five-stage internal algorithmic auditing framework~\cite{raji2020closing}, our guidelines serve as a practical tool for closing the AI accountability gap. The automatically generated report plays a crucial role in this process by providing a summary of the guidelines that were effectively implemented, those that should be considered for future development, and the non-applicable ones. These reports establish an additional chain of accountability that can be shared with stakeholders at various levels, including managers, senior leadership, and AI engineers. By offering more oversight and the ability to troubleshoot if needed, these reports help mitigate unintentional harm. However, it is important to note that when an organization adopts our guidelines, it should establish clear ethical guidelines for their intended uses. Our tool is not intended to discourage developers from using it due to the fear of being held accountable for their responses. On the contrary, developers' responses, as documented in the report, provide an opportunity to identify potential ethical issues and address them early in the design stages. This proactive approach prevents the need for post-hoc fixes and repairs, aligning with the principle of addressing ethical considerations during the development process rather than as an afterthought~\cite{sambasivan2018toward}---the idea of \emph{Responsible AI by Design}.

\subsection{Limitations and Future Work} Our work has four main limitations that highlight the need for future research efforts. 

Firstly, although we followed a rigorous four-step process involving multiple stakeholders, the list of 22 guidelines may not be exhaustive. The rapidly evolving nature of AI ethics, governance, and regulations necessitates an ongoing effort to stay abreast of emerging developments. However, one of the strengths of our method lies in its modular design, which allows for ongoing refinement and expansion of the set of guidelines. This ensures that our responsible AI tool maintains its relevance and stays up to date in the ever-evolving landscape of AI ethics, governance, and regulations. As new ISOs are established, addressing specific aspects of AI systems such as functional safety (ISO 5469), data quality (ISO 5259), and explainability (ISO 6254), our tool can be readily extended to include these guidelines. Moreover, as the scientific community progresses in its understanding of ethical considerations in AI, our tool can incorporate new insights and recommendations to enhance its comprehensive coverage.

Secondly, it is important to consider the qualitative nature of our user study, which involved in-depth interviews and analysis of participants' responses. The findings from this study should be interpreted with caution, understanding that the reported frequency of themes should be viewed in a comparative context rather than taken at face value~\cite{fossey2002understanding}. This approach helps to avoid potential misinterpretation or overgeneralization of the results.

Thirdly, we need to acknowledge the limitations associated with the sample size and demographics of our user study. The study was conducted with a specific group of participants, and therefore, the findings may not fully represent the practices and perspectives of all AI practitioners. Our sample predominantly consisted of male participants, which aligns with the gender distribution reported in Stack Overflow's 2022 Developer Survey, where 92.85\% of professional developer respondents identified as male~\cite{stackoverflow2022survey}. Additionally, our participants were drawn from a large research-focused technology company. While the results may offer insights into practices within certain companies, they serve as a case study for future research. Furthermore, we did not explicitly consider participants' specific roles, despite their expertise spanning various domains and levels. Future studies could explore the considerations of ethical values in AI systems across organizations and different roles and areas of expertise. Previous research has indicated different understandings of responsible AI values between practitioners and the general public~\cite{maurice2022how}, suggesting the potential for similar research methods to be applied in this area.

Last but not least, our qualitative data suggests indicators of ease of use for AI practitioners but does not provide direct information on the actual effectiveness of the guidelines. Understanding the impact of guidelines (or other AI toolkits~\cite{wong2023seeing}) requires long-term studies that consider multiple projects, with some utilizing the toolkit and others not. One potential avenue, as suggested by clinical researchers developing deep learning tools for patient care~\cite{beede2020human}, is to conduct observational studies with users of the AI system to assess its performance. Another approach is to use proxies, such as measuring users' attitudes, beliefs, and mindset regarding ethical values before and after utilizing the guidelines. We intend to explore these directions in future research.
\section{Conclusion}
\label{sec:conclusion}
We proposed a method that allows for easy updating of responsible AI guidelines derived from research papers and ISO standards, ensuring that the content of responsible AI tools remains relevant and up to date. To validate the effectiveness of our method, we developed and deployed an interactive tool that provides actionable guidelines, which were generated by a team comprising engineers, standardization experts, and a lawyer. Our findings indicate that the guidelines were well-received, as they were perceived as practical and actionable, fostering self-reflection and facilitating a deeper understanding of the ethical considerations associated with AI during the initial phases of design and development.

%%
%% The acknowledgments section is defined using the "acks" environment
%% (and NOT an unnumbered section). This ensures the proper
%% identification of the section in the article metadata, and the
%% consistent spelling of the heading.
% \begin{acks}

% \end{acks}

%%
%% The next two lines define the bibliography style to be used, and
%% the bibliography file.
% \nocite{*}
\bibliographystyle{ACM-Reference-Format}
\bibliography{main}

%%
%% If your work has an appendix, this is the place to put it.
% \appendix
% \newpage
\appendix
\clearpage

\section{Additional Materials For the User Study}
\label{app:demographics-survey}
\begin{itemize}
    \item How old are you?
    \item What is your gender? [Male, Female, Non-binary, Prefer not to say, Open-ended option]
    \item How many years of experience do you have in AI systems?
    \item What's your educational background?
    \item In which country do you currently reside?
    \item What is domain or sector of your work? (e.g., health, energy, education, finance, technology, food)
    \item What is your current role?
    \item What kinds of AI systems do you work on? (e.g., machine learning, computer vision, NLP, game theory, robotics)
\end{itemize}

\begin{table}[h]
\centering

\caption{Constructed themes for the user study based on how our participants saw the application of guidelines, what worked well and what could have been improved.}
\label{tab:codebook-interviews}
% \resizebox{\textwidth}{!}{%
\begin{tabular}{ll}
\toprule
\textbf{Theme} & \textbf{Participants} \\ \midrule
\quad Raising awareness, facilitating self-learning & 12 \\
\quad Aligning with roles & 10 \\
\quad Aligning with regulations & 10 \\
\quad Providing helpful examples & 7 \\
\quad Engaging team members and external experts & 5 \\
\quad Maintaining the visual simplicity of the guidelines & 3 \\
\quad Documenting guidelines in a concise summary PDF & 3 \\
\quad Providing a systematic flow of information and guidelines & 2 \\
\end{tabular}%
% }
\end{table}

\clearpage

\section{Mapping Guidelines with EU AI Act Articles}
\label{app:mapping_guidelines}

\noindent\textbf{Article 6 (\emph{Classification rules for high-risk AI systems}):} It states that an AI system shall be considered high-risk when \emph{``it [the AI system] is intended to be used as a safety component of a product, or is itself a product''.} This article aligns with \textbf{guideline \#1} as it mandates the identification of an AI system's intended use to determine whether its use poses a low or high risk.
\smallskip

\noindent\textbf{Article 9 (\emph{Risk management system}):} 
It states that \emph{``a risk management system shall be established, implemented, documented and maintained throughout the entire lifecycle of a high-risk AI system''.} This article aligns with \textbf{guidelines \#1, \#3-5, and \#13} as it is about the identification of harms and risks of the AI system's intended use.
\smallskip

\noindent\textbf{Article 10 (\emph{Data and data governance}):} 
It states that \emph{``training, validation and testing data sets shall be subject to appropriate data governance and management practices''.} This article aligns with \textbf{guidelines \#8 and \#15-18} as it discusses the management and quality of data for training, validation, and testing, including aspects of diversity and minimizing biases.
\smallskip

\noindent\textbf{Article 11 (\emph{Technical documentation}):}  
It states that the technical documentation of a high-risk AI system shall \emph{``be drawn up before that system is placed on the market or put into service and shall be kept up-to date''}, and \emph{``provide national competent authorities and notified bodies with all the necessary information to assess the compliance of the AI system''.} This article aligns with \textbf{guidelines \#2, \#6, \#14} as it about documentation of the system and its contractual requirements, which may also be needed for obtaining ethical approvals.
\smallskip

\noindent\textbf{Article 12 (\emph{Record-keeping}):} 
It states that high-risk AI systems shall include \emph{``logging capabilities to enable the monitoring of the operation of the high-risk AI system with respect to the occurrence of situations that may result in the AI system presenting a risk''.} This article aligns with \textbf{guidelines \#6, \#9, \#10, and \#14} as it is about providing mechanisms for interpretable outputs and auditing, and improving the security of the system. 
\smallskip

\noindent\textbf{Article 13 (\emph{Transparency and provision of information to users}):} 
It states that \emph{``high-risk AI systems shall be designed and developed in such a way to ensure that their operation is sufficiently transparent to enable users to interpret the system's output and use it appropriately''.} This article aligns with \textbf{guidelines \#8-10, \#16-18, and \#20} as it is about quality, representativeness, and fit of training and testing datasets with the intended use.
\smallskip

\noindent\textbf{Article 14 (\emph{Human oversight}):} 
It states that \emph{``high-risk AI systems shall be designed and developed in such a way, including with appropriate human-machine interface tools, that they can be effectively overseen by natural persons during the period in which the AI system is in use''.} This article aligns with \textbf{guidelines \#9 and \#20} as it about ensuring human control over the system.
\smallskip

\noindent\textbf{Article 15 (\emph{Accuracy, robustness and cybersecurity}):} 
It states that \emph{``high-risk AI systems shall be designed and developed in such a way that they achieve, in the light of their intended purpose, an appropriate level of accuracy, robustness and cybersecurity, and perform consistently in those respects throughout their lifecycle''.} This article aligns with \textbf{guideline \#10} as it is about documenting the security of all system components.
\smallskip

\noindent\textbf{Article 16 (\emph{Obligations of providers of high-risk AI systems}):} 
It states that \emph{``providers of high-risk AI systems shall draw-up the technical documentation of the high-risk AI system''.} This article aligns with \textbf{guideline \#6} as it is about system documentation.
\smallskip

\noindent\textbf{Article 17 (\emph{Quality management system}):} 
It states that ``an AI system shall be documented in a systematic and orderly manner in the form of written policies, procedures and instructions''. This article aligns with \textbf{guidelines \#6, \#7, \#10, and \#14-18} because it is about documentation of all system components, including AI models and testing and validation procedures.
\smallskip

\noindent\textbf{Article 18 (\emph{Obligation to draw up technical documentation}):}    
It states that \emph{``providers of high-risk AI systems shall draw up the technical documentation ''.} This article aligns with \textbf{guideline \#6} as it is about system documentation.
\smallskip

\noindent\textbf{Article 20 (\emph{Automatically generated logs}):} 
It states that \emph{``providers of high-risk AI systems shall keep the logs automatically generated by their high-risk AI systems, to the extent such logs are under their control by virtue of a contractual arrangement with the user or otherwise by law''.} This article aligns with \textbf{guideline \#19} as it is about monitoring of the system.
\smallskip

\noindent\textbf{Article 29 (\emph{Obligations of users of high-risk AI systems}):} 
It states that users shall \emph{``monitor the operation of the high-risk AI system on the basis of the instructions of use.''}, and \emph{``inform the provider or distributor when they have identified any serious incident or any malfunctioning and interrupt the use of the AI system''.}  This article aligns with \textbf{guideline \#19} as it about monitoring of the system and utilizing guardrails or rollbacks.
\smallskip

\noindent\textbf{Article 50 (\emph{Document retention}):} 
It states that \emph{``the provider shall, for a period ending 10 years after the AI system has been placed on the market or put into service, keep at the disposal of the national competent authorities the technical documentation''.} This article aligns with \textbf{guideline \#6} as it about system documentation.
\smallskip

\noindent\textbf{Article 60 (\emph{EU database for stand-alone high-risk AI systems}):} 
It states that information contained in the EU database shall \emph{``be accessible to the public''} and \emph{``include the names and contact details of natural persons who are responsible for registering the system and have the legal authority to represent the provider''.} This article aligns with \textbf{guideline \#4} as it is about providing mechanisms for reporting system harms.
\smallskip

\noindent\textbf{Article 61 (\emph{Post-market monitoring by providers and post-market monitoring plan for high-risk AI systems}):} 
It states that \emph{``the post-market monitoring system shall actively and systematically collect, document and analyse relevant data provided by users or collected through other sources on the performance of high-risk AI systems throughout their lifetime''.} This article aligns with \textbf{guidelines \#12, \#14, \#15, \#19} as it is about data handling and model updates when the AI system is in use.
\smallskip

\noindent\textbf{Article 62 (\emph{Reporting of serious incidents and of malfunctioning}):} 
It states that \emph{``providers of high-risk AI systems placed on the Union market shall report any serious incident or any malfunctioning of those systems which constitutes a breach of obligations under Union law intended to protect fundamental rights to the market surveillance authorities of the Member States where that incident or breach occurred''.} This article aligns with \textbf{guideline \#4} as it is about incentivizing the reporting of system harms.
\smallskip

\noindent\textbf{Article 63 (\emph{Market surveillance and control of AI systems in the Union market}):} 
It states that \emph{``the national supervisory authority shall report to the Commission on a regular basis the outcomes of relevant market surveillance activities. ''.} This article aligns with \textbf{guideline \#4} as it about incentivizing the reporting of system harms.
\smallskip

\noindent\textbf{Article 64 (\emph{Access to data and documentation}):} 
It states that \emph{``access to data and documentation in the context of their activities, the market surveillance authorities shall be granted full access to the training, validation and testing datasets used by the provider, including through application programming interfaces (`API') or other appropriate technical means and tools enabling remote access''.} This article aligns with \textbf{guidelines \#16 and \#17} as it is about data documentation.
\smallskip

\noindent\textbf{Article 65 (\emph{Procedure for dealing with AI systems presenting a risk at national level}):} 
It states that \emph{``AI systems presenting a risk shall be understood as a product presenting a risk defined in Article 3, point 19 of Regulation (EU) 2019/1020 insofar as risks to the health or safety or to the protection of fundamental rights of persons are concerned''.} This article aligns with \textbf{guideline \#3} as it is about harms and risks identification.
\smallskip

\noindent\textbf{Article 67 (\emph{Compliant AI systems which present a risk}):} 
It states that if the AI system is compliant with the EU AI Act but still presents a risk to the health or safety of persons, the market surveillance authority \emph{``shall require the relevant operator to take all appropriate measures to ensure that the AI system concerned, when placed on the market or put into service, no longer presents that risk, to withdraw the AI system from the market or to recall it within a reasonable period, commensurate with the nature of the risk, as it may prescribe''.} This article aligns with \textbf{guideline \#5} as it is about mitigation strategies about the identified harms and risks.
\smallskip

\noindent\textbf{Article 69 (\emph{Codes of conduct}):} 
It states that \emph{``the Commission and the Board shall encourage and facilitate the drawing up of codes of conduct intended to foster the voluntary application to AI systems of requirements related for example to environmental sustainability, accessibility for persons with a disability, stakeholders participation in the design and development of the AI systems and diversity of development teams on the basis of clear objectives and key performance indicators to measure the achievement of those objectives''.} This article aligns with \textbf{guidelines \#2, \#11, \#21, \#22} as it is about the environmental assessment of the system, the ethical approvals obtained from ethics committees and boards, and the characteristics of the development team.


\end{document}
\endinput
%%
%% End of file `sample-authordraft.tex'.
