\section{Conclusion}
\label{sec:conclusion}
We proposed a method that allows for easy updating of responsible AI guidelines derived from research papers and ISO standards, ensuring that the content of responsible AI tools remains relevant and up to date. To validate the effectiveness of our method, we developed and deployed an interactive tool that provides actionable guidelines, which were generated by a team comprising engineers, standardization experts, and a lawyer. Our findings indicate that the guidelines were well-received, as they were perceived as practical and actionable, fostering self-reflection and facilitating a deeper understanding of the ethical considerations associated with AI during the initial phases of design and development.