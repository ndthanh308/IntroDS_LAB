\section{Modeling the Physical Capabilities of a CAN Attacker as a Closed-Loop Attack Policy}
\label{sec:prelims}
%
In this section we briefly review a variety of security threats to automotive CAN networks that an adversary can exploit for targeting the braking ECUs (see, e.g.,~\cite{froschle2017analyzing,palanca2017stealth,miller2019lessons} for further details). This section also provides a justification for modeling the cyber-physical threat capabilities of an adversary as a closed-loop attack policy design for the vehicle actuators.

Under certain assumptions, Fr{\"o}schle and St{\"u}hring~\cite{froschle2017analyzing} have outlined a collection of six possible attacks on the CAN bus with cyber-physical implications. These attacks include: (i) blocking messages by priority; (ii) disrupting the target network; (iii) silencing a target node by dominant bits; (iv) silencing a target node by collisions; (v) suppressing a  target message; and (vi) modification attacks via either impersonating a target node or modifying target messages by suppress and inject. Through a combination of these six attacks,  Fr{\"o}schle and St{\"u}hring~\cite{froschle2017analyzing} investigate the cyber-physical implications for manipulation of steering and braking, e.g., steering a Jeep at any speed. The systemic investigation by Fr{\"o}schle and St{\"u}hring~\cite{froschle2017analyzing} was motivated by the celebrated hack of Miller and Valasek~\cite{miller2013adventures,miller2019lessons}, in which remote re-flashing of the firmware of a target microcontroller was demonstrated. 

Under the assumption of re-flashing the firmware of a target braking ECU, the adversary threat capability can be modeled by assuming complete authority over the brake actuators and full knowledge of the states of the vehicle traction dynamics by reading from the in-vehicle network, e.g., the CAN bus. \emph{In essence, the adversary can act as a feedback controller by sending malicious commands to the brake actuators while using the sensed states of the vehicle for computing these commands.}  In general, the adversary does have a very limited knowledge of the plant dynamics and its interactions with the ambient environment (vehicle traction dynamics and its interaction with the road). Consequently, the closed-loop attack policies on a vehicle actuator can be located within the cyber-physical attack space due to Teixeira \emph{et al.}~\cite{teixeira2015secure} according to Figure~\ref{fig:newFig}(right).  %provides a  classification of the proposed attack policy in this paper by  
 %
%In this section we provide some preliminary definitions from finite-time stability theory. The interested reader is referred to~\cite{polyakov2014stability,zhou2016asymptotic,sanchez2015predefined,polyakov2015finite,song2019time} for further details. 
%
%Consider the dynamical system
%%
%\begin{equation}
%\dot{y} = f(t,y;\zeta)
%\label{eq:prelim1}
%\end{equation}
%% 
%where $t\in [t_0, \infty)$ for some $t_0>0$, $y\in \mathbb{R}^n$, and $\zeta \in \mathbb{R}^m$ denote time, the state, and the parameters of the system in~\eqref{eq:prelim1}, respectively. We denote the solutions to~\eqref{eq:prelim1} starting from $y_0=y(t_0)$  by $y(t,y_0)$. We say that the non-empty set $\Gamma \subset \mathbb{R}^n$ is \emph{locally finite-time stable}  for~\eqref{eq:prelim1} if $\Gamma$ is locally asymptotically stable for~\eqref{eq:prelim1} and there exists a neighborhood of $\Gamma$, denoted by $\mathcal{N}(\Gamma)$, such that any solution $y(t,y_0)$ of~\eqref{eq:prelim1} with $y_0\in \mathcal{N}(\Gamma)$  reaches $\Gamma$ in a finite time moment $t^\ast=T(y_0)$, where the \emph{settling-time function} $T: \mathcal{N}(\Gamma) \to \rpos$ is locally bounded. We say that~\eqref{eq:prelim1} is \emph{locally fixed-time stable} if it is locally finite-time stable and there exists some positive constant $T_{max}$, called the \emph{settling-time}, such that $T(y_0) \leq T_{max}$ for all $y_0\in \mathcal{N}(\Gamma)$. If $\mathcal{N}(\Gamma) = \mathbb{R}^n$, the definitions will become global. 

%We denote the set of non-negative real numbers by $\mathbb{R}_{\geq 0}$. Following the notation in~\cite{zhou2016asymptotic}, given the interval $J=[t_0,\infty)\subset \mathbb{R}_{\geq 0}$ where $t_0$ is some positive real number, we let $\mathbb{PC}(J,\mathbb{R})$ denote the union of the sets of all real-valued piecewise continuous and continuously differentiable functions defined on $J$. Consider the LTV system 
%%
%\begin{equation}
%\dot{x} = a(t) x(t), 
%\label{eq:prelim2}
%\end{equation}
%% 
%where $a(.) \in \mathbb{PC}(J,\mathbb{R})$ and $x\in \mathbb{R}$. We say that the LTV system in~\eqref{eq:prelim2} is \emph{uniformly exponentially stable} if there exist positive constants $\gamma$ and $k$ such that $|x(t)| \leq k |x(t_0)| \exp(-\gamma (t-t_0) )$ for all $t, t_0 \in J$ with $t\geq t_0$.
 