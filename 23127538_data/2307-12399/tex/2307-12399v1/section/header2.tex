% Header.tex

\usepackage{array,pdfsync,enumerate,tikz,enumerate}
\usepackage[dvips]{epsfig}
\usepackage{amssymb,amsmath}
\usepackage{caption}
\usepackage{subcaption}
\usepackage{multirow}
%
\usepackage{color}
\definecolor{linkcol}{rgb}{0,0.0,1.0}
\definecolor{citecol}{rgb}{0.0,0.6,0.0}


\IEEEoverridecommandlockouts
% The preceding line is only needed to identify funding in the first footnote. If that is unneeded, please comment it out.
\usepackage{cite}
\usepackage{amsmath,amssymb,amsfonts}
\usepackage{algorithmic}
\usepackage{graphicx}
\usepackage{textcomp}
%\def\BibTeX{{\rm B\kern-.05em{\sc i\kern-.025em b}\kern-.08em
%		T\kern-.1667em\lower.7ex\hbox{E}\kern-.125emX}}

%\overrideIEEEmargins
%\IEEEoverridecommandlockouts                              % This command is only
%\usepackage{macros,macros2}


\usepackage{hyperref}
\hypersetup{colorlinks,breaklinks,
	linkcolor=linkcol,urlcolor=citecol,
	anchorcolor=linkcol,citecolor=citecol,bookmarks=true}
%----------------------------------
\newcounter{MYtempeqncnt}
\usepackage{caption}
\usepackage{subcaption}
%%%%%%%%%%%%%%%%%%%%%%%%%%%%%%%%
\newcommand{\thbf}{\boldsymbol\theta}
\DeclareMathOperator{\diag}{diag}
\DeclareMathOperator{\rank}{rank}
%%%%%%%%%%%%%%%%%%%%%%%%%%%%%%%%
%% %----------------------------
%% % THEOREMS
%% % ---------------------------------------------------------------
%% %%%%%%%%%%%%%%%%%%%%%%%
%\theoremstyle{definition}
\newtheorem{theorem}{Theorem}[section]
\newtheorem{lemma}[theorem]{Lemma}
%\newtheorem{corollary}[theorem]{Corollary}
\newtheorem{proposition}[theorem]{Proposition}
%\newtheorem{defi}[theorem]{Definition}
\newtheorem{remark}[theorem]{Remark}
%%\newtheorem{assumption}[theorem]{Assumption}
%\newtheorem{exmple}[theorem]{Example}
%\newcommand{\tx}[1]{\text{#1}}
%%\newtheorem{procedure}[theorem]{Procedure}
%%%%%%%%%%%%%%%%%%%%%%%%%%%%%%%%%%%
%\newenvironment{example}{
%	\begin{exmple}\rm}{\hfill $\bigtriangleup$%
%	\end{exmple}
%}
%
%\newenvironment{remark}{
%	\begin{rmrk}\rm}{ \hfill{ }$\bigtriangleup$%
%	\end{rmrk}
%}
%
%
%\newenvironment{definition}{
%	\begin{defi}\rm}{\hfill $\bigtriangleup$%
%	\end{defi}
%}
%%%%%%%%%%%%%%%%%%%%%%%%%%%%%%%%%%%
%\renewcommand\thetheorem{\arabic{section}.\arabic{theorem}}
%
%\title{\Large \bf }
%==============================================
\newcommand{\rpos}{\mathbb{R}_{\geq 0}}
\newcommand{\ex}{\hat{e}_x}
\newcommand{\ez}{\hat{e}_z}
\newcommand{\ethr}{\hat{e}_{\theta_r}}
\newcommand{\ethrp}{\hat{e}^{\perp}_{\theta_r}}
\newcommand{\tx}[1]{\text{#1}}

\usepackage{graphicx}      % include this line if your document contains figures
\usepackage{cite}
%\usepackage{natbib}        % required for bibliography
%===============================================================================
\renewcommand\floatpagefraction{.9}
\renewcommand\dblfloatpagefraction{.9} % for two column documents
\renewcommand\topfraction{.9}
\renewcommand\dbltopfraction{.9} % for two column documents
\renewcommand\bottomfraction{.9}
\renewcommand\textfraction{.1}   
\setcounter{totalnumber}{50}
\setcounter{topnumber}{50}
\setcounter{bottomnumber}{50}
%================================================================================
\begin{document}
	
	\title{Investigation of Wheel Lockup Attacks on Nonlinear Dynamics of Vehicle Traction Using
		Frictional Brakes}
	
	\author{Alireza Mohammadi and Hafiz Malik\\% <-this % stops a space
		Emails: {\tt \{amohmmad,hmalik\}@umich.edu}
		%\thanks{This work was not supported by any organization}% <-this % stops a space
		%\thanks{%This work is supported by NSF Award 2035770. A. Mohammadi and H. Malik are with the
			%Department of Electrical and Computer Engineering, University of Michigan-Dearborn, MI 48127 USA. M. Abbaszadeh is with 
			%GE Global Research, NY 12309 USA.  
		%	Emails: {\tt\small \{amohmmad,hmalik\}@umich.edu}.}%
	}
	

\maketitle		
		
%		\begin{abstract}                % Abstract of not more than 250 words.
%			 There is ample evidence in the automotive cybersecurity literature that the car brake ECUs can be maliciously reprogrammed. 
%			 Motivated by such threat, this paper investigates the capabilities of an adversary who can directly control the frictional brake actuators and would like to induce wheel lockup conditions leading to catastrophic road injuries.  Simulations under various road conditions demonstrate the effectiveness of the proposed attack policy.          
%		\end{abstract}
		
	%	\begin{keyword}
	%	quadratic programming, autonomous robots, nonlinear control systems, saturation, guidance systems.
	%	\end{keyword}
		
		
		
%	\end{frontmatter}

%~~~~~~~~~~~~~~~~~~~~~~~~~~~~~~~~~~~~~~~~~
%\include{PhDBiblio}	
%===============================================================================