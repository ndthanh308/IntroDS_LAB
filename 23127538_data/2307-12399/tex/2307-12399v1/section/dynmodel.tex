\section{Preliminaries}

\subsection{Prior Literature on Impedance-Based Control of Haptic Shared Control Steering Systems}
% provide control schemes for HSC with specific attention to impedance control 
Impedance control  is finding its way in the design of efficient HSC controllers in ADAS applications~\cite{chugh2018comparison}. The \emph{key idea in impedance control} for HSC steering systems is to  create a proper virtual inertia-spring-damper behavior resulting in a good steering feel and effective task sharing between the human driver and the automation. Izadi and Ghasemi~\cite{izadi2020adaptive,izadi2021modulation} have proposed an adaptive impedance control scheme to smoothly balance the authority of control in-between the human driver and his/her automation assistance system  through a motorized steering wheel. Chugh \emph{et al.}~\cite{chugh2020approach} have utilized open-loop driving maneuvers to design reference impedance models for the generation of appropriate haptic feedback for HSC steering systems in the linear range of vehicle handling dynamics. To ensure a stable interaction between the human driver and the steering wheel system, passivity techniques from the field of robotics teleoperation have also been utilized for designing controllers in HSC steering applications. For instance, Lee \emph{et al.}~\cite{lee2022haptic} have designed a haptic control algorithm with passivity guarantees for steering wheel systems to remove the possibility of having limit cycles and unstable behavior due to the unwanted effects of sampling and quantization.

\subsection{The Dynamics of the Steering System}
\label{sec:dynModel}
%
Sensors, ECUs, and servo motors form the backbone of HSC steering systems, which aim at transmitting the driver input to the servo motor mounted on the steering gear assembly in an electronic manner. In turn, for creating a proper `feel' of the road through haptic feedback to the human, the self-aligning reaction torques due to the tire--road interface forces are reflected back to the driver through the servo motor located on the steering wheel. The coordination between the torques transmitted to the steering column and the ones reflected back to the driver at the steering wheel are handled through the ECUs and proper control algorithms coded on them. Figure~\ref{fig:steerBlock} depicts a schematic block diagram of the HSC steering system investigated in this paper. 

% Figure environment removed

The steering system dynamics, through which the driver/driving-assist-system interaction are governed,  can be written as (see, e.g.,~\cite{lazcano2021mpc,setlur2006trajectory}) 
% 
\begin{subequations}
	\begin{align}
		\Sigma_{\theta_{\text{sw}}, 
			\dot{\theta}_{\text{sw}}}:\, I_{\text{sw}} \ddot{\theta}_{\text{sw}} + N_{\text{sw}}(\theta_{\text{sw}}, 
		\dot{\theta}_{\text{sw}}) &= \alpha_{\text{sw}} \tau_{\text{sw}} + T_{\text{sw}}, 
		\label{eq:dynSW}
		\\
		\Sigma_{\theta_{\text{c}}, 
			\dot{\theta}_{\text{c}}}:\,I_{\text{c}} \ddot{\theta}_{\text{c}} + N_{\text{c}}(\theta_{\text{c}}, 
		\dot{\theta}_{\text{c}}) &= \alpha_{\text{c}} \tau_{\text{c}} + T_{\text{c}}. 
		\label{eq:dynC}
	\end{align}
	\label{eq:dynMain}
\end{subequations}
%
\hspace{-2ex} In~\eqref{eq:dynMain}, the driver input device dynamics $\Sigma_{\theta_{\text{sw}},\dot{\theta}_{\text{sw}}}$ are given by~\eqref{eq:dynSW}, where  $\theta_{\text{sw}},\dot{\theta}_{\text{sw}},$ and $\ddot{\theta}_{\text{sw}}$, denote the driver input device angular position, its rate of change, and its associated angular acceleration, respectively. Furthermore, the inertia of the driver's input device is given by $I_{\text{sw}}$. The driver's arm is assumed to be rigidly coupled with the steering wheel and be exerting the torque $\tau_{\text{sw}}$ on it. Additionally, $T_{\text{sw}}$ is the control/attack input torque applied to the steering  wheel. Similarly, the steering column dynamics $\Sigma_{\theta_{\text{c}},\dot{\theta}_{\text{c}}}$ are given by~\eqref{eq:dynC}, in which $\theta_{\text{c}},\dot{\theta}_{\text{c}},$ and $\ddot{\theta}_{\text{c}}$, denote the steering column angular position, its rate of change, and its associated angular acceleration, respectively. The steering column dynamics $\Sigma_{\theta_{\text{c}},\dot{\theta}_{\text{c}}}$, whose inertia is given by $I_{\text{c}}$, is under the influence of the tire/road reaction torque $\tau_{\text{c}}$ and an additional control/attack input  $T_{\text{c}}$. The torque input scaling factors  $\alpha_{\text{sw}}$ and $\alpha_{\text{c}}$ are positive constants that might arise due to system gearing. Finally, the damping and friction effects are captured by the functions $N_{\text{sw}}(\cdot)$ and $N_{\text{c}}(\cdot)$, respectively. 

It is assumed that the damping and friction functions $N_{\text{sw}}(\cdot)$ and $N_{\text{c}}(\cdot)$ can be expressed as 
%
\begin{subequations}
	\begin{align}
	N_{\text{sw}}(\theta_{\text{sw}}, \dot{\theta}_{\text{sw}})  &= B_{{\text{sw}}} \dot{\theta}_{\text{sw}} + K_{{\text{sw}}} {\theta}_{\text{sw}},
	\label{eq:linFunc1}
	\\
	N_{\text{c}}(\theta_{\text{c}}, \dot{\theta}_{\text{c}})  &= B_{{\text{c}}} \dot{\theta}_{\text{c}} + K_{{\text{c}}} {\theta}_{\text{c}},
	\label{eq:linFunc2}
	\end{align}
	\label{eq:linFunc}
\end{subequations}
% 
\hspace{-2ex} where $B_i$ and $K_i$, $i\in \{ \text{sw}, \text{c}\}$ are the damping and stiffness coefficients of the driver input device and the steering column, respectively. It can be seen that the functions in~\eqref{eq:linFunc1} and~\eqref{eq:linFunc2} can be written in the following linearly parametrizable forms 
%
\begin{subequations}
	\begin{align}
		N_{\text{sw}}(\theta_{\text{sw}}, \dot{\theta}_{\text{sw}})  &= Y_{N_{\text{sw}}}(\theta_{\text{sw}}, \dot{\theta}_{\text{sw}}) \phi_{N_{\text{sw}}},
		\label{eq:linParam1}
		\\
		N_{\text{c}}(\theta_{\text{c}}, \dot{\theta}_{\text{c}})  &= Y_{N_{\text{c}}}(\theta_{\text{c}}, \dot{\theta}_{\text{c}}) \phi_{N_{\text{c}}}. 
		\label{eq:linParam2}
	\end{align}
	\label{eq:linParam}
\end{subequations}
% 
\hspace{-2ex} where $Y_{N_{i}}(\theta_{i}, \dot{\theta}_{i}):=[\theta_{i},\, \dot\theta_{i}]\in \mathbb{R}^{1\times 2}$, $i\in \{ \text{sw}, \text{c}\}$, is a regression matrix consisting of the measurable quantities $\theta_{i}$ and  $\dot{\theta}_{i}$. Furthermore, $\phi_{N_{i}}:=[K_{i},\, B_{i}]^\top\in \mathbb{R}^{2\times 1}$, $i\in \{ \text{sw}, \text{c}\}$, is a constant vector of the damping and stiffness coefficients. 

As it will be shown in Section~\ref{sec:attackDesign}, the linearly parametrizable form in~\eqref{eq:linFunc} will be used for designing adaptive attack policies and updating the unknown parameters of the driver/driving-assist-system dynamics  according to properly designed adaptive update laws.  