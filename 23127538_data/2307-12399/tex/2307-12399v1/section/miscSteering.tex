My note: reducing the driving load; transition from autonomy to manual; increasing the joyful experience of driving

@incollection{abbink2010neuromuscular,
	title={Neuromuscular Analysis as a Guideline in designing Shared Control},
	author={Abbink, DA and Mulder, M},
	booktitle={Advances in Haptics},
	pages={499--516},
	year={2010},
	publisher={Intech}
}


@article{nekouei2021randomized,
	title={A randomized filtering strategy against inference attacks on active steering control systems},
	author={Nekouei, Ehsan and Pirani, Mohammad and Sandberg, Henrik and Johansson, Karl H},
	journal={IEEE Transactions on Information Forensics and Security},
	volume={17},
	pages={16--27},
	year={2021},
	publisher={IEEE}
}


% Dawson papers: 
@inproceedings{mandhata2012evaluation,
	title={Evaluation of a customizable haptic feedback system for ground vehicle steer-by-wire interfaces},
	author={Mandhata, Uday B and Jensen, Matthew J and Wagner, John R and Switzer, Fred S and Dawson, Darren M and Summers, Joshua D},
	booktitle={2012 American Control Conference (ACC)},
	pages={2781--2787},
	year={2012},
	organization={IEEE}
}


@article{baviskar2008adjustable,
	title={An adjustable steer-by-wire haptic-interface tracking controller for ground vehicles},
	author={Baviskar, Abhijit and Wagner, John R and Dawson, Darren M and Braganza, David and Setlur, Pradeep},
	journal={IEEE Transactions on Vehicular Technology},
	volume={58},
	number={2},
	pages={546--554},
	year={2008},
	publisher={IEEE}
}


% The raw material for paper writing: 
@inproceedings{ferraguti2013tank,
	title={A tank-based approach to impedance control with variable stiffness},
	author={Ferraguti, Federica and Secchi, Cristian and Fantuzzi, Cesare},
	booktitle={2013 IEEE international conference on robotics and automation},
	pages={4948--4953},
	year={2013},
	organization={IEEE}
}

Humans interact with the surrounding environment constantly, e.g. moving an utensil, catching a ball, opening a door. The growing interest in reproducing human-like behavior in many robotic tasks brought the control of the interaction between a manipulator and the environment to be a key component for the success of many manipulation tasks. These interactions may often be unstable, especially when the task involves a tool.

The concept of active control of a manipulator’s interactive behavior is formally treated as an aspect of impedance control [2]. The impedance control provides a compliant behavior during the interaction and regulates the dynamic response of the robot end-effector to interaction forces by establishing a suitable virtual mass-spring-damper system on the end-effector.

Excessive contact force between the manipulator and the environment should be prevented. Since humans control the force exerted on an object by adapting their arm stiffness, in a similar way the robot should be able to change the stiffness of its arm while performing an interaction task. In this paper we presented a passivity-based impedance control algorithm with variable stiffness. The passification strategy involves the exploitation of an energy tank that store back the energy dissipated by the system.

~~~~~~~~~~~~~~~~~~~~~~~~~~~~~~~~~~~~~~~~~~~~~~~~~~~~~~~~~~~~~~~~~~~~~~~~~~~~~~~~~

@article{kronander2016stability,
	title={Stability considerations for variable impedance control},
	author={Kronander, Klas and Billard, Aude},
	journal={IEEE Transactions on Robotics},
	volume={32},
	number={5},
	pages={1298--1305},
	year={2016},
	publisher={IEEE}
}

Variable impedance control allows not only controlling the dynamic relationship between external forces and robot movements, but in addition gives the flexibility to change these dynamics in a continuous manner during the task.

There is hence a need for analysis or control methods that can guarantee stable execution of variable impedance tasks. This
issue has been recently addressed in [8], wherein a tank-based approach to passive varying stiffness was proposed. Their system uses the total energy of the manipulator (kinetic plus virtual potential energy coming from stiffness term), but does not constrain this function to strictly decrease. Instead, any dissipated energy is added to a virtual energy tank, from which energy can be extracted in order to implement stiffness variations. The tank is given an initial level of energy and a maximum allowed level of energy. These levels determine to what extent the system will accept stiffness variations. 

It has two important shortcomings, which are as follows. 1) It depends on the state of the robot, with the consequence that
one cannot guarantee beforehand the execution of a desired
impedance profile. 2) The performance depends strongly on the intial and threshold
levels of energy in the tank. 

Our aim with this work is to provide a stability condition for varying stiffness and damping that is state independent. The
most important practical advantage of such a constraint is that it can be verified offline, before execution of the task. Any standard impedance control architecture can subsequently be used for task execution, with a reassuring guarantee that the system cannot go unstable.

We focus on variable stiffness and damping, which are the impedance parameters that are most commonly varied for im-
proving task performance. We then propose a stability condition that relates the stiffness, damping, and their rates of change. The constraint arises from the choice of a Lyapunov candidate function in which mixed position and velocity terms appear in the time derivative. Our main contribution in this paper is using this function to derive stability constraints for stiffness and damping profiles, and evaluating how these constraints can be used in practice.

~~~~~~~~~~~~~~~~~~~~~~~~~~~~~~~~~~~~~~~~~~~~~~~~~~~~~~~~~~~~~~~~~~~~~~~~~~~~~~~~~

@article{lazcano2021mpc,
	title={MPC-based haptic shared steering system: a driver modeling approach for symbiotic driving},
	author={Lazcano, Andrea Michelle Rios and Niu, Tenghao and Akutain, Xabier Carrera and Cole, David and Shyrokau, Barys},
	journal={IEEE/ASME Transactions on Mechatronics},
	volume={26},
	number={3},
	pages={1201--1211},
	year={2021},
	publisher={IEEE}
}

The exponential growth of advanced driver assistance systems  the years has a direct impact on increased safety and reduction of mental workload while driving [1]. However, automation can also lead to unsatisfactory user acceptance when the driver's intention or expectation does not match the behavior of the driving assist system [2]. Moreover, the different projections toward the d ems (ADAS) over the years. 

The shared control approach is particularly suitable for the steering task as forces can be exchanged at the steering wheel to accomplish a common objective. Through haptic shared control (HSC), the authority of the driving task is balanced between the driving assist system and the driver. However, although HSC can lead to less steering control activity and increased safety [4], drivers sometimes resist the assist system’s guidance [5]. This can be due to, for example, a mismatch between the driver's cognitive intentions and the controller's objective, or, from a neuromuscular level, the reflex action of the muscle spindles [6].

The introduction of the steering system dynamics is key to investigate the interaction between driver and driving assist system. The steering dynamics are rigidly coupled to the arms dynamics at the steering wheel, where torques are exchanged.

Thereby resulting in a lumped inertia that is the sum of the inertia of the arms Iarms and the inertia of the steering wheel Isw. The neuromuscular dynamics of the arms are described in detail in Section III-B. The proposed control strategy tackles the need to blend driver and assist system through driver modeling in an HSC strategy. 

~~~~~~~~~~~~~~~~~~~~~~~~~~~~~~~~~~~~~~~~~~~~~~~~~~~~~~~~~~~~~~~~~~~~~~~~~~~~~~~~~

@article{lv2021human,
	title={Human--Machine Collaboration for Automated Driving Using an Intelligent Two-Phase Haptic Interface},
	author={Lv, Chen and Li, Yutong and Xing, Yang and Huang, Chao and Cao, Dongpu and Zhao, Yifan and Liu, Yahui},
	journal={Advanced Intelligent Systems},
	volume={3},
	number={4},
	pages={2000229},
	year={2021},
	publisher={Wiley Online Library}
}

~~~~~~~~~~~~~~~~~~~~~~~~~~~~~~~~~~~~~~~~~~~~~~~~~~~~~~~~~~~~~~~~~~~~~~~~~~~~~~~~~

@article{mars2014analysis,
	title={Analysis of human-machine cooperation when driving with different degrees of haptic shared control},
	author={Mars, Franck and Deroo, Mathieu and Hoc, Jean-Michel},
	journal={IEEE Transactions on Haptics},
	volume={7},
	number={3},
	pages={324--333},
	year={2014},
	publisher={IEEE}
}


This study investigated human-machine cooperation when driving with different degrees of a shared control system. By

means of a direct intervention on the steering wheel, shared control systems partially correct the vehicle’s trajectory and, at the same
 time, provide continuous haptic guidance to the driver. A crucial point is to determine the optimal level of steering assistance for
 effective cooperation between the two agents. Five system settings were compared with a condition in which no assistance was
 present. In addition, road visibility was manipulated by means of additional fog or self-controlled visual occlusions. Several
 performance indicators and subjective assessments were analyzed. The results show that the best repartition of control in terms of
 cooperation between human and machine can be identified through an analysis of the steering wheel reversal rate, the steering effort
 and the mean lateral position of the vehicle. The best cooperation was achieved with systems of relatively low-level haptic authority,
 although more intervention may be preferable in poor visibility conditions. Increasing haptic authority did not yield higher benefits in
 terms of steering behavior, visual demand or subjective feeling.

~~~~~~~~~~~~~~~~~~~~~~~~~~~~~~~~~~~~~~~~~~~~~~~~~~~~~~~~~~~~
@inproceedings{ludwig2018comparison,
	title={A comparison of concepts for control transitions from automation to human},
	author={Ludwig, Julian and Haas, Andreas and Flad, Michael and Hohmann, Soren},
	booktitle={2018 IEEE International Conference on Systems, Man, and Cybernetics (SMC)},
	pages={3201--3206},
	year={2018},
	organization={IEEE}
}

Until upcoming autonomous vehicles can handle
 every scenario in all operation domains there will be situations, in

which the human driver needs to retrieve control of the vehicle.
 This can lead to safety issues as literature shows impaired driving
 performance after taking over from highly automated driving.
 A discrete switch of control might be a valid option to hand
 over to an attentive human driver. However, it is questionable if

this is a safe way for a transition of control to somebody who is
 distracted at the time of the takeover request. Instead, gradually
 shifting control via a period of haptic shared control can provide
 guidance during the process of establishing situation awareness.
 Here, we present different design concepts for handing over
 control from a controller to a human partner. To analyze these
 concepts, we designed an experiment, in which the participants
 need to take over a simplified steering task when requested by
 the automation. We conducted a study comparing two discrete

switching transitions and three concepts in which the authority of
 the controller fades out slowly. The results show that in situations,
 needing a continuation of the control action of the automation, an
 early switch-off underperformed compared to the other concepts.
 If the situation demanded action during the transition the shared
 control approaches achieved the lowest tracking error. A survey
 of the participants revealed that gradually shifting transitions
 lead to a sense of better performance and are also regarded as
 more comfortable. The best overall results were obtained by a

concept which models human behavior and considers it within the
 takeover process. This approach provides a systematic method for
 the design of authority transitions by finding the best trade-off
 between guidance and handover.

~~~~~~~~~~~~~~~~~~~~~~~~~~~~~~~~~~~~~~~~~~~~~~~~~~~~~~~~~~~~

@article{marcano2020review,
	title={A review of shared control for automated vehicles: {T}heory and applications},
	author={Marcano, Mauricio and D{\'\i}az, Sergio and P{\'e}rez, Joshu{\'e} and Irigoyen, Eloy},
	journal={IEEE Transactions on Human-Machine Systems},
	volume={50},
	number={6},
	pages={475--491},
	year={2020},
	publisher={IEEE}
}

The last decade has shown an increasing interest on
 advanced driver assistance systems (ADAS) based on shared control, where automation is continuously supporting the driver at the
 control level with an adaptive authority. A first look at the literature
 offers two main research directions: 1) an ongoing effort to advance
 the theoretical comprehension of shared control, and 2) a diversity
 of automotive system applications with an increasing number of
 works in recent years. Yet, a global synthesis on these efforts is
 not available. To this end, this article covers the complete field of
 shared control in automated vehicles with an emphasis on these
 aspects: 1) concept, 2) categories, 3) algorithms, and 4) status of
 technology. Articles from the literature are classified in theory- and
 application-oriented contributions. From these, a clear distinction
 is found between coupled and uncoupled shared control. Also,
 model-based and model-free algorithms from these two categories
 are evaluated separately with a focus on systems using the steering
 wheel as the control interface. Model-based controllers tested by at
 least one real driver are tabulated to evaluate the performance of
 such systems.Results showthatthe inclusion of a driver model helps
 to reduce the conflicts at the steering. Also, variables such as driver
 state, driver effort, and safety indicators have a high impact on
 the calculation of the authority. Concerning the evaluation, driverin-the-loop simulators are the most common platforms, with few
 works performed in real vehicles. Implementation in experimental
 vehicles is expected in the upcoming years.

~~~~~~~~~~~~~~~~~~~~~~~~~~~~~~~~~~~~~~~~~~~~~~~~~~~~~~~~~~~~~~~~~~

@article{zwaan2019haptic,
	title={Haptic shared steering control with an adaptive level of authority based on time-to-line crossing},
	author={Zwaan, Hugo M and Petermeijer, Sebastiaan M and Abbink, David A},
	journal={IFAC-PapersOnLine},
	volume={52},
	number={19},
	pages={49--54},
	year={2019},
	publisher={Elsevier}
}

Traditional driver-automation interaction trades control over the vehicle back and forth between driver and automation. Haptic shared control offers an alternative by continuously sharing the control through torques on the steering wheel and pedals. When designing additional feedback torques, part of the design choice lies in the stiffness around the neutral steering point: also called the Level of Haptic Authority (LoHA), which is usually static and tuned to balance safety benefits (better at high LoHA) with conflicts torques in case of different intentions between automation and driver (higher conflict torques with increased LoHA). In this paper we explore the idea of situation-adaptive LoHA: in this case during lane-keeping by changing the LoHA based on time to lane crossing (TLC). Consequently, when safety margins are high (e.g., when driving on a wide road) the LoHA is low, but the LoHA would only increase when safety margins decrease. We propose two alternative design approaches to apply the LoHA: symmetrically and asymmetrically (i.e., only increase of LoHA in the direction of the low TLC). We compared these design in an explorative driving simulator study (n=14) to driving with two static LoHA designs (low and high). We found that compared to the high LoHA controller, both adaptive LoHA controllers designs resulted in similar safety margins, but at decreased conflict torques. Hence, a TLC-based adaptive LoHA controller seems to be an effective approach to mitigate conflicts while maintaining the safety benefits associated with HSC.

~~~~~~~~~~~~~~~~~~~~~~~~~~~~~~~~~~~~~~~~~~~~~~~~~~~~~~~~~~~~~~~~~~
@article{katzourakis2014haptic,
	title={Haptic steering support for driving near the vehicle's handling limits: Test-track case},
	author={Katzourakis, Diomidis I and Velenis, Efstathios and Holweg, Edward and Happee, Riender},
	journal={IEEE Transactions on Intelligent Transportation Systems},
	volume={15},
	number={4},
	pages={1781--1789},
	year={2014},
	publisher={IEEE}
}


Abstract—Current vehicle dynamic control systems from simple
 yaw control to high-end active steering support systems are designed to primarily actuate on the vehicle itself, rather than stimulate the driver to adapt his/her inputs for better vehicle control.
 The driver though dictates the vehicle’s motion, and centralizing
 him/her in the control loop is hypothesized to promote safety and
 driving pleasure. Exploring the above statement, the goal of this
 paper is to develop and evaluate a haptic steering support when
 driving near the vehicle’s handling limits [Haptic Support near
 the Limits (HSNL)]. The support aims to promote the driver’s
 perception of the vehicle’s behavior and handling capacity (the
 vehicle’s internal model) by providing haptic cues on the steering
 wheel. The HSNL has been evaluated in a test track where 17 test
 subjects drove around a narrow-twisting tarmac circuit, a vehicle
 (Opel Astra G/B) equipped with a steering system able to provide
 variable steering feedback torque. The drivers were instructed to
 achieve maximum velocity through corners while receiving haptic
 steering feedback cues related to the vehicle’s cornering potentials.
 The test-track tests led to the conclusion that haptic support
 reduced drivers’ mental and physical demand without affecting
 their driving performance.