\section{Adversarial Time-Varying Impedance Target Dynamics}
\label{sec:TargetDyn}
%
%
In this section we present a non-passive and  time-varying impedance target dynamics, which are fed with a linear combination of the human driver and the steering column torques from the torque sensor readings of the HSC steering system. We will demonstrate that it is possible for the adversary to generate in real-time reference angular commands for the driver input device and the directional control steering assembly of the vehicle where the target dynamics (i.e., the system $\Sigma_{\theta_{\text{d}}, \dot{\theta}_{\text{d}}}$ in Figure~\ref{fig:steerBlock}) become a non-passive system. 

Consider the following time-varying target dynamics 
%
\begin{equation}
\Sigma_{\theta_{\text{d}}, \dot{\theta}_{\text{d}}}: I_T \ddot{\theta}_{\text{d}} + N_T(\theta_{\text{d}}, \dot{\theta}_{\text{d}},t) = \alpha_{T_{\text{sw}}} \tau_{\text{sw}} + \alpha_{T_{\text{c}}} \tau_{\text{c}}, 
\label{eq:refModel}
\end{equation}
%
where $\theta_{\text{d}}$, $\dot{\theta}_{\text{d}}$, and $\ddot{\theta}_{\text{d}}$ are the target angular position, its rate of change, and the angular acceleration, respectively. The time-varying function $N_T(\theta_{\text{d}}, \dot{\theta}_{\text{d}},t)$ represents an auxiliary target dynamic function, consisting of time-varying damping and stiffness terms; and, the design parameters $\alpha_{T_{\text{sw}}}$ and $\alpha_{T_{\text{c}}}$ are scaling constants that weigh the driver's input torque $\tau_{{\text{sw}}}$ and the reaction torque $\tau_{{\text{c}}}$ from the steering column. 

\noindent{\textbf{The Adversarial Design Objective.}} The adversarial design objective for the auxiliary function $N_T(\theta_{\text{d}}, \dot{\theta}_{\text{d}},t)$ is to inject energy into the dynamical system $\Sigma_{\theta_{\text{d}}, \dot{\theta}_{\text{d}}}$ to create an unstable behavior for the reference trajectories $\theta_{\text{d}}$, $\dot{\theta}_{\text{d}}$, and $\ddot{\theta}_{\text{d}}$, which get modulated in response to the driver's input torque $\tau_{{\text{sw}}}$ and the reaction torque $\tau_{{\text{c}}}$ from the steering column. As demonstrated later in Section~\ref{sec:attackDesign}, these trajectories will be fed to an adaptive attack policy that will make the angular positions $\theta_{\text{sw}}$ and $\theta_{\text{c}}$ to follow $\theta_{\text{d}}(t)$ closely. 

To design the adversarial target dynamic function $N_T(\theta_{\text{d}}, \dot{\theta}_{\text{d}},t)$, we consider 
%
\begin{equation}
N_T(\theta_{\text{d}}, \dot{\theta}_{\text{d}},t) = B_T(t) \dot{\theta}_{\text{d}} + K_T(t) {\theta}_{\text{d}}, 
\label{eq:NTtarget}
\end{equation}
%
where  $B_T(\cdot)$ and $K_T(\cdot)$ are two continuously differentiable time-varying stiffness and damping profiles. Therefore, the \emph{adversarial target dynamical system} takes the form (see, also, the block diagram in Figure~\ref{fig:steerTargetDyn})
%
\begin{equation}
I_T \ddot{\theta}_{\text{d}} + B_T(t) \dot{\theta}_{\text{d}} + K_T(t) {\theta}_{\text{d}} = \alpha_{T_{\text{sw}}} \tau_{\text{sw}} + \alpha_{T_{\text{c}}} \tau_{\text{c}}.
\label{eq:targetMain}
\end{equation}
%
\begin{remark}
	In contrast to the conventional robotics literature (see, e.g.,~\cite{buchli2011learning,ferraguti2013tank,kronander2016stability,ochoa2021impedance}), where the goal is to improve the robotic task performance by designing proper variable stiffness and damping profiles while ensuring stability, the goal of the adversary in this paper, who is targeting the vehicle HSC steering system, is to create an unstable behavior by inducing non-passive dynamics through a proper choice of the variable damping  $B_T(\cdot)$ and impedance $K_T(\cdot)$ functions.\hfill$\blacklozenge$ 
	\label{rem:timeVarying}
\end{remark}
%
%
% Figure environment removed

\noindent{\textbf{The Adversarial Target Dynamics Storage Function.}} Let us consider the storage function, which was originally proposed by Kronander and Billard~\cite{kronander2016stability} for studying stability of time-varying impedance controllers (see, also, Remark~\ref{rem:timeVarying}), 
%
\begin{equation}
V_a(\theta_{\text{d}}, \dot{\theta}_{\text{d}},t) = \frac{1}{2} I_T (\dot{\theta}_{\text{d}}+ \alpha \theta_{\text{d}})^2 + \frac{1}{2} \beta(t) \theta_{\text{d}}^2, 
\label{eq:lyapCandid}
\end{equation}
%
for the target dynamics in~\eqref{eq:targetMain}. The  positive constant $\alpha$ in the storage function given by~\eqref{eq:lyapCandid} needs to be chosen in a way that the inequality  
%
\begin{equation}
\beta(t) := K_T(t) + \alpha B_T(t)  - \alpha^2 I_T \geq 0, 
\label{eq:betaFunc}
\end{equation}
%
holds for all $t>0$. 

\begin{remark}
	In addition to the storage function $V_a(\cdot)$ given by~\eqref{eq:lyapCandid}, there are other possibilities for defining candidate storage functions. For instance, consider the normalized unforced adversarial target dynamical system $\ddot{\theta}_d + \tfrac{B_T(t) }{I_T} \dot{\theta}_{\text{d}} + \tfrac{K_T(t)}{I_T} {\theta}_{\text{d}}=0$. It is possible to define a \emph{time-dependent Hamiltonian} as another potential candidate storage function for the normalized target dynamics. Defining the canonical momentum to be $p:=\zeta(t)^2 \dot{\theta}_d$, where $\zeta(t) := C_1 \exp(\tfrac{\int B_T(t)dt}{2I_T})$, the time-dependent Hamiltonian of these unforced dynamics are given by (see, e.g.,~\cite{kanasugi1995systematic} for further details) $H(\theta_d, p, t) = \frac{1}{2} \Big\{ \frac{p^2}{\zeta^2(t)} + \zeta^2(t) \frac{K_T(t)}{I_T} \theta_d^2 \Big\}$.\hfill$\blacklozenge$ 
	%
	%\begin{equation}
	%
	%	\label{eq:timeDepHamil}
	%\end{equation} 
	\label{rem:timeDep}
\end{remark}

Taking the derivative of the storage function in~\eqref{eq:lyapCandid} along the trajectories of the target dynamics~\eqref{eq:targetMain}, we arrive at 
%
\begin{equation}
\begin{aligned}
\dot{V}_a &= \big\{ \alpha I_T - B_T(t) \big\} \dot{\theta}_{\text{d}}^2 + \big\{ \frac{1}{2} \dot{K}_T(t) + 
 \frac{\alpha}{2} \dot{B}_T(t) \\ 
 & - \alpha K_T(t) \big\} \theta_{\text{d}}^2 +  \dot{\theta}_{\text{d}} \big( \alpha_{T_{\text{sw}}} \tau_{\text{sw}} + \alpha_{T_{\text{c}}} \tau_{\text{c}} \big). 
\end{aligned}
\label{eq:lyapCandidDeriv}
\end{equation}
%

It can be easily seen that if the time-varying stiffness $K_T(t)$, the time-varying damping $B_T(t)$, and their rates of change $\dot{K}_T(t)$ and  $\dot{B}_T(t)$ satisfy the inequalities 
%
\begin{subequations}
	\begin{align}
	\alpha I_T - B_T(t)  > 0,
	\label{eq:ineq1}
	\\
	\frac{1}{2} \dot{K}_T(t) + \frac{\alpha}{2} \dot{B}_T(t) - \alpha K_T(t) > 0,  
	\label{eq:ineq2}
	\end{align}
	\label{eq:targetIneq}
\end{subequations}
% 
\hspace{-2ex} then the rate of change of the storage function $V_a(\cdot)$ satisfies the inequality 
%
\begin{equation}
\dot{V}_a > \dot{\theta}_{\text{d}} \big( \alpha_{T_{\text{sw}}} \tau_{\text{sw}} + \alpha_{T_{\text{c}}} \tau_{\text{c}} \big) + \frac{\rho_1}{2} {\theta}_{\text{d}}^2+ \frac{\rho_2}{2} \dot{\theta}_{\text{d}}^2,     
\label{eq:nonPassiveProof}
\end{equation}
%
where $\rho_1 = \inf\limits_{t>0} \big( \alpha I_T - B_T(t) \big)$ and $\rho_2 =  \inf\limits_{t>0} \big( \tfrac{1}{2} \dot{K}_T(t) + \tfrac{\alpha}{2} \dot{B}_T(t) - \alpha K_T(t) \big)$, for all $(\theta_{\text{d}}, \dot\theta_{\text{d}})\neq (0,0)$ and $t>0$.  which  renders the target dynamics in~\eqref{eq:targetMain} \emph{non-passive} with respect to the input-output pair $(\dot{\theta}_{\text{d}},\, \alpha_{T_{\text{sw}}} \tau_{\text{sw}} + \alpha_{T_{\text{c}}} \tau_{\text{c}})$. The adversary can utilize the weighting factors  $\alpha_{T_{\text{sw}}}$ and $\alpha_{T_{\text{c}}}$ as design variables. For instance, if the adversary is interested in creating a non-passive target dynamics in response to the driver input, they can choose  the driver input weighting factor $\alpha_{T_{\text{sw}}}$ to be positive and the steering column weighting factor $\alpha_{T_{\text{c}}}$ to be zero.


We remark that the inequalities given in~\eqref{eq:ineq1} and~\eqref{eq:ineq2} arise due to the particular choice of the storage function in~\eqref{eq:lyapCandid}, where mixed angular velocity and position terms appear in the derivative of the storage function along the reference dynamic system trajectories in~\eqref{eq:targetMain}. As it can be seen from~\eqref{eq:lyapCandidDeriv}, the choice of the storage function in~\eqref{eq:lyapCandid} results in the inequalities given by~\eqref{eq:ineq1} and~\eqref{eq:ineq2} that relate the time-varying damping, the time-varying stiffness, and their rates of change to ensure a non-passive behavior from the target dynamical system in response to the driver's input $\tau_{\text{sw}}$ and the steering column reaction torque $\tau_{\text{c}}$.  



\noindent{\textbf{Design of the Adversarial Target Dynamics Impedance Profiles.}} To find conditions on the time-varying stiffness $K_T(t)$ and damping $B_T(t)$ profiles, which satisfy the inequalities given by~\eqref{eq:betaFunc},~\eqref{eq:ineq1}, and~\eqref{eq:ineq2}, we define 
%
\begin{equation}
	\kappa(t) := -K_T(t) - \alpha B_T(t) + \alpha^2 I_T. 
	\label{eq:kappa}
\end{equation}
%

Using the definition given by~\eqref{eq:kappa}, it can be seen that $\kappa(t)\leq 0$ due to~\eqref{eq:betaFunc} and  $-\tfrac{1}{2\alpha}\dot\kappa - K_T(t)> 0$ due to~\eqref{eq:ineq2}. Adding the sides of the inequality in~\eqref{eq:ineq1} to $-\tfrac{1}{2\alpha}\dot\kappa - K_T(t)> 0$ yields
%
\begin{equation}
	\dot{\kappa} <  2\alpha \kappa(t).  
	\label{eq:kappaIneq}
\end{equation}
% 

From a straightforward application of the Gr\"{o}nwall-Bellman inequality (see, e.g.,~\cite[Lemma A.1, p. 651]{khalil2002nonlinear}) to~\eqref{eq:kappaIneq}, it can be seen that $\kappa(t) < \kappa(0) \exp(2\alpha t)$ holds for all $t>0$. Therefore, the definition in~\eqref{eq:kappa} yields the following constraint on the time-varying stiffness and damping profiles 
%
\begin{equation}
	K_T(t) +  \alpha B_T(t) >  \alpha^2 I_T + \big( K_{0} + \alpha B_{0} - \alpha^2 I_T \big) \exp(2\alpha t), 
	\label{eq:finalConst}
\end{equation}
%  
where $B_{0} = B_T(0)$ and $K_0 = K_T(0)$ for all $t>0$.

% Figure environment removed

Consequently, if the time-varying stiffness and damping profiles satisfy~\eqref{eq:ineq1} and~\eqref{eq:finalConst}, then the rate of change of the storage function $V_a(\cdot)$ satisfies~\eqref{eq:nonPassiveProof} and the adversarial target dynamics in~\eqref{eq:targetMain} become non-passive with respect to the input-output pair $(\dot{\theta}_{\text{d}},\, \alpha_{T_{\text{sw}}} \tau_{\text{sw}} + \alpha_{T_{\text{c}}} \tau_{\text{c}})$. As demonstrated in Figure~\ref{fig:svgSynth}, to ensure non-passive adversarial target dynamics for all $t>0$, $K_T(t) +  \alpha B_T(t)$ should belong to the red area above the curve $y=\alpha^2 I_T + \big( K_{0} + \alpha B_{0} - \alpha^2 I_T \big) \exp(2\alpha t)$ while $B_T(t)$ should belong to the blue area under $y=\alpha I_T$. 



%A special case for the inequalities in~\eqref{eq:ineq1} and~\eqref{eq:ineq2} arise when the damping is fixed, namely, when %$B_T(t)=B_0$. In this special case, the inequalities in~\eqref{eq:ineq1} and~\eqref{eq:ineq2} become 
%
%\begin{subequations}
%	\begin{align}
%	\alpha I_T - B_0  > 0,
%	\label{eq:ineq1b}
%	\\
%	\frac{1}{2} \dot{K}_T(t) - \alpha K_T(t) > 0.  
%	\label{eq:ineq2b}
%	\end{align}
%	\label{eq:targetIneqb}
%\end{subequations}
% 


%
%\begin{equation}
%	\alpha  > \frac{B_0}{I_T},\; {K}_T(t) > K_T(0) \exp(2\alpha t),  
%	\label{eq:targetIneqc}
%\end{equation}
% 
%then the target dynamics in~\eqref{eq:targetMain} become non-passive. 