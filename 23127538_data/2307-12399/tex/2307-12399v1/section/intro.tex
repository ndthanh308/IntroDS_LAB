\section{Introduction}
\label{sec:intro}

% HSC intro: 
Haptic shared control (HSC), through which the driving task authority is balanced between the human driver and the driving assist system via an exchange of force at a motorized steering wheel, is becoming an indispensable part of advanced driver assistance systems (ADAS). In particular, HSC steering systems can achieve a common vehicle steering task objective between the automation and the human such as avoiding an obstacle or lane keeping. In these shared tasks,  the automation provides a continuous support to the human driver at the control level~\cite{marcano2020review}.  As demonstrated by Wang \emph{et al.}~\cite{wang2017effect}, haptic guidance can support drivers who have become passively fatigued during lengthy and monotonous trips.  Moreover, the perceived task load by the human driver can be reduced significantly during near-handling-limit driving scenarios as evidenced via the experiments by Katzourakis \emph{et al.}~\cite{katzourakis2014haptic}. In addition to guidance and reducing the driving perceived load, HSC steering systems can be used to make safe  transitions from humans to automation and vice versa~\cite{ludwig2018comparison,marcano2020review}. 

%% Motivate by attacks on steering systems. One of the systems that can be target is the steering system. Say some examples. 
Given the safety-critical nature of steering, there is no wonder that cybersecurity researchers are interested in demonstrating cyber-physical vulnerabilities of vehicles through design of various attacks against their steering systems~\cite{kim2021cybersecurity} and analyzing the capabilities of an adversary who ``has made it to the last stage''~\cite{froschle2017analyzing} (see Figure~\ref{fig:attackSteer} for several possible cyber-physical attack vectors against the HSC steering system).

% Figure environment removed

One of the main motivators for designing steering attacks can be traced back to the celebrated hacks by Miller and Valasek~\cite{miller2019lessons} in the field of vehicle cybersecurity. In a scentless manner and with leaving almost no forensic evidence behind, Miller and Valasek managed to steer a 2014 Jeep Cherokee into a ditch. Another hack against the active steering systems include the one proposed by Nekouei \emph{et al.}~\cite{nekouei2021randomized}, where the gains of the vehicle steering closed-loop controller are inferred by infiltrating the vehicular ad-hoc network. Chernikova \emph{et al.}~\cite{chernikova2019self}  have designed effective evasion attacks against  the deep neural networks that are used to predict the vehicle steering angle from images. Gao \emph{et al.}~\cite{gao2020distributed} have considered a class of CAN-based deception attacks against the steering mechanism of a four-in-wheel motor drive vehicle. 

\noindent{\textbf{Gap in the Literature.}} Despite the rich line of research on design of attacks against vehicle steering systems, no attack policy in the literature has directly focused on  destabilizing  interactions between the driver and the HSC steering system. Given the important role of human drivers in automated driving and the fact that up to \emph{Level 4 Automation}, humans should be capable of operating in driving conditions that are not supported by the automation (see, e.g.,~\cite{kyriakidis2019human}),  \emph{cyber-physical attacks that aim at destabilizing the human driver-vehicle interactions need to be thoroughly investigated}. 

This paper investigates the adversarial capabilities for destabilizing the interaction dynamics between human drivers and HSC steering systems. We assume a cyber-physical attack scenario where the adversary has access to full disruption and full disclosure resources while having a limited knowledge of the HSC steering system parameters under attack (see, e.g., the inexpensive device depicted in Figure~\ref{fig:palanaca}  for accessing the controller area network (CAN) bus through the OBD-II port). We demonstrate that using a properly designed non-passive and  time-varying impedance target dynamics, which are fed with a linear combination of the human driver and the steering column torques, it is possible for the adversary to generate in real-time a reference angular command for the driver input device and the directional control steering assembly of the vehicle. Furthermore, we show that the adversary can make the steering wheel and the vehicle steering column angular positions to follow the reference command generated by the time-varying impedance target dynamics using proper adaptive control strategies while removing the need for the exact knowledge of the HSC steering system parameters.  

\noindent\textbf{Contributions of the Paper.} This paper contributes to the emerging field of cybersecurity in robotics and automation (see, e.g.,~\cite{khalil2021fault,Vilches22}) in several important ways. The first contribution of the paper is founded in using variable impedance control synthesis techniques (see, e.g.,~\cite{ferraguti2013tank,kronander2016stability}) for generation of cyber-physical closed-loop attacks against the HSC steering systems. In contrast to the traditional robotics literature, where the main objective is to render the human-robot interaction dynamics stable by ensuring passivity (see, e.g.,~\cite{buchli2011learning,ferraguti2013tank,kronander2016stability,ochoa2021impedance}), this paper takes the exact opposite route; namely, this paper proposes synthesizing time-varying impedance profiles that result in a non-passive interaction between the human driver and the vehicle HSC steering system. Second, to the best of our knowledge, no other attack policy has directly aimed at destabilizing  the interaction between the driver and the HSC steering system. Given the safety critical nature of HSC steering systems, which play an integral role in several applications such as seamless transition from automation to human and creation of a symbiotic driving experience, this paper contributes to understanding the physical damaging capabilities of adversaries who target HSC steering systems. 

%Finally, the data generated through the developed attacks in this paper can be utilized for training fault/attack classification algorithms in %connected and  autonomous vehicles using supervised machine learning  (see, e.g.,~\cite{khalil2021fault}). 
%
% Figure environment removed
%

The rest of this paper is organized as follows. First, we review the necessary preliminaries in Section~\ref{sec:dynModel}. Next, we present the time-varying impedance target dynamics and provide conditions under which these dynamics become non-passive in Section~\ref{sec:TargetDyn}. Thereafter,  in Section~\ref{sec:attackDesign}, we present  our closed-loop attack policy based on using an adaptive control scheme for making the steering wheel and the vehicle steering column angular positions to follow the reference command generated by the target dynamics. After validating our proposed attack policy through simulation results in Section~\ref{sec:sims}, we conclude the paper with further remarks, insights, and directions for future research in Section~\ref{sec:conc}. 