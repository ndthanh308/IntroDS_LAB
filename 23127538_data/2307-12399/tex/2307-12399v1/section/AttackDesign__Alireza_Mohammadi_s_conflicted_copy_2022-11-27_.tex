\section{Adaptive Attack Policy}
\label{sec:attackDesign}

In this section we will design adaptive attack policies that will make the angular position of the driver's input device, i.e.,~$\theta_{\text{sw}}$, to closely follow the angular position reference $\theta_{\text{d}}$, which is generated by the target dynamical system given by~\eqref{eq:targetMain}. Our adaptive control strategy for achieving this tracking objective follows the control design proposed by Baviskar \emph{et al.}~\cite{baviskar2008adjustable}. 

We first define the driver experience error signal, i.e., the angular position tracking error between the driver input device and the target dynamics, as 
%
\begin{equation}
e_{\text{sw}} = \theta_{\text{d}} - \theta_{\text{sw}},\, 
\label{eq:errorSigsw}
\end{equation}
%
and the locked tracking error signal, i.e., the angular position tracking error between the driver input device and the steering column, as 
%
\begin{equation}
e_{\text{c}} = \theta_{\text{sw}} - \theta_{\text{c}}. 
\label{eq:errorSigc}
\end{equation}
%

Using the angular position tracking errors in~\eqref{eq:errorSigsw} and~\eqref{eq:errorSigc}, one can additionally define the filtered error signals as 
%
\begin{equation}
r_{\text{sw}} = \dot{e}_{\text{sw}} + \mu_{\text{sw}} e_{\text{sw}},\, r_{\text{c}} = \dot{e}_{\text{c}} + \mu_{\text{c}} e_{\text{c}}, 
\label{eq:filteredErr}
\end{equation}
%
where the positive gains $\mu_{\text{sw}}$ and $\mu_{\text{c}}$ are control parameters. 

To express the adaptive attack policy, in addition to the filtered error signals in~\eqref{eq:filteredErr}, we also need to define the regression matrices  
%
\begin{equation}
Y_{\text{sw}} = [Y_{N_{\text{sw}}}, -\tau_{\text{sw}}, \ddot{\theta}_{d1}  + \mu_{\text{sw}} \dot{e}_{\text{sw}}] \in \mathbb{R}^{1\times 4}, 
\label{eq:Ysw}
\end{equation}
%
and 
\begin{equation}
Y_{\text{c}} = [-Y_{N_{\text{sw}}}, \tau_{\text{sw}}, T_{\text{sw}}, Y_{N_{\text{c}}}, -\tau_{\text{c}}, \mu_{\text{c}}\dot{e}_{\text{c}}]\in \mathbb{R}^{1\times 8}, 
\label{eq:Yc}
\end{equation}
% 
where  $Y_{N_{\text{sw}}}$ and $Y_{N_{\text{c}}}$ are the regression vectors defined in~\eqref{eq:linParam}. The regression matrices $Y_{\text{sw}}$ and $Y_{\text{c}}$ consist of measurable quantities that the adversary is assumed to be able to read from the in-vehicle communication networks and/or accurately estimate. Furthermore, let us define the vectors $\phi_{\text{sw}}$, $\phi_{\text{c}}$, which stack the unknown constant parameters of the ESC steering dynamics in~\eqref{eq:dynMain} in two unknwon constant vectors, as 
%
\begin{equation}
\phi_{\text{sw}} = [\phi_{N_{\text{sw}}}, \alpha_{\text{sw}}, I_{\text{sw}}]^\top,
\label{eq:phisw}
\end{equation}
%
and
%
\begin{equation}
\phi_{\text{c}} = [\frac{I_{\text{c}}}{I_{\text{sw}}} \phi_{N_{\text{sw}}},  \frac{I_{\text{c}}}{I_{\text{sw}}} \alpha_{1}, \frac{I_{\text{c}}}{I_{\text{sw}}}, \phi_{N_{\text{c}}}, \alpha_{\text{c}}, I_{\text{c}}]^\top.  
\label{eq:phic}
\end{equation}
%
In~\eqref{eq:phisw} and~\eqref{eq:phic},  the constant unknown vectors $\phi_{N_{\text{sw}}}$ and $\phi_{N_{\text{c}}}$ are defined in~\eqref{eq:linParam}. 

Using the filtered error signals defined in~\eqref{eq:filteredErr}, the regression matrices defined in~\eqref{eq:Ysw} and~\eqref{eq:Yc}, and the constant unknown vectors defined in~\eqref{eq:phisw} and~\eqref{eq:phic}, the adversary can utilize the following control inputs 
%
\begin{equation}
T_{\text{sw}} = k_{\text{sw}} r_{\text{sw}} + Y_{\text{sw}} \hat{\phi}_{\text{sw}},\, T_{\text{c}} = k_{\text{c}} r_{\text{c}} + Y_{\text{c}} \hat{\phi}_{\text{c}}, 
\label{eq:Tattack}
\end{equation}
%
where $k_{\text{sw}}$ and $k_{\text{c}}$ are constant positive control gains, and ${\hat{\phi}}_{\text{sw}}$ and ${\hat{\phi}}_{\text{c}}$ are adaptive estimates for the unknown parameter vectors $\phi_{\text{sw}}$ and $\phi_{\text{c}}$, whose evolution are governed by the following adaptive update laws 
%
\begin{equation}
\dot{\hat{\phi}}_{\text{sw}} = \Gamma_{\text{sw}} Y_{\text{sw}}^\top r_{\text{sw}},\, \dot{\hat{\phi}}_{\text{c}} = \Gamma_{\text{c}} Y_{\text{c}}^\top r_{\text{c}},
\label{eq:adaptationLaws}
\end{equation}
%
where $\Gamma_{\text{sw}}$ and $\Gamma_{\text{c}}$ are positive constant diagonal gain matrices.


%Based on the subsequent stability analysis and the structure
% of the open-loop error system given in (9) and (10), control
% inputs $T_{\text{sw}}$ and $T_{\text{c}}$ are designed as
% and
\[
\tilde{\phi}_{\text{sw}} = \phi_{\text{sw}} - \hat\phi_{\text{sw}},\, \tilde{\phi}_{\text{c}} = \phi_{\text{c}} - \hat\phi_{\text{c}}
\]


%\[
%\dot{\hat\tau}_{\text{sw}} = -(\beta_{\text{sw}} + K_s + 1) \hat\tau_{\text{sw}} - \frac{I_{\text{sw}}}{\alpha_{\text{sw}}} \big[ (\beta_{\text{sw}}  + K_s(\beta_{\text{sw}}+1) ) \dot{e}_{\text{sw}}  + 
%K_s  \beta_{\text{sw}} e_{\text{sw}} + \rho_{\text{sw}} \text{sgn}(p_{\text{sw}})
% \big]
%\]
%
%\[
%\dot{\hat\tau}_{\text{c}} = -(\beta_{\text{sw}} + K_s + 1) \hat\tau_{\text{c}} - \frac{I_{\text{c}}}{\alpha_{\text{c}}} \big[ \rho_{\text{c}} \text{sgn}(p_{\text{c}}) 
%\big] - \frac{I_{\text{c}}}{\alpha_{\text{c}}} \big[ (\beta_{\text{sw}} + K_s(\beta_{\text{sw}}+1))\dot{e}_{\text{c}} + K_s \beta_{\text{sw}} e_{\text{c}} \big]
%\]
%
%\[
%+ \frac{I_{\text{c}}}{\alpha_{\text{c}}} \big[ \frac{\alpha_{\text{sw}}}{I_{\text{sw}}} (\dot{\hat\tau}_{\text{sw}} + (\beta_{\text{sw}}+K_s+1)\hat\tau_{\text{sw}}) \big]
%\]
%
%\[
%p_{\text{sw}} = \dot{e}_{\text{sw}} + \beta_{\text{sw}} e_{\text{sw}},\, p_{\text{c}} = \dot{e}_{\text{c}} + \beta_{\text{sw}} e_{\text{c}}
%\]
%
%
%\[
%T_{\text{sw}} = N_{\text{sw}}(\cdot) + \big(\frac{I_{\text{sw}}}{I_T}\big) (-N_T(\cdot) + \alpha_{T1} \hat\tau_{\text{sw}} + \alpha_{T2} \hat\tau_{\text{c}}) -\alpha_{\text{sw}}\hat\tau_{\text{sw}}
%\]
%
%\[
%T_{\text{c}} = N_{\text{c}}(\cdot) + \big(\frac{I_{\text{c}}}{I_{\text{sw}}}\big) (-N_{\text{sw}}(\cdot) + T_{\text{sw}} + \alpha_{1} \hat\tau_{\text{sw}}) -\alpha_{\text{c}}\hat\tau_{\text{c}}
%\]


%\section{Vehicle Traction Dynamical Model}
%\label{sec:dynmodel} 
%%
%Detailed models for the conventional and power-assisted
% steering systems exist within the literature (e.g., [5] and [22]).
% Steering models considering tire forces and vehicle dynamics have been derived in [6]. The representation of steering
% dynamics chiefly follows from the work in [5], [6], and [22].
%
%is an auxiliary nonlinear function
% that describes the dynamics on the driver side,  denotes the driver input torque, and represents a
% control input torque applied to the driver input device. denote the angular position, velocity,
% and acceleration, respectively, of the vehicle directional control
% assembly.  an auxiliary nonlinear function
% that describes the dynamics of the vehicle directional control assembly, represents the reaction torque between the
% actuator on the directional control assembly and the mechanical subsystem actuated by the directional control assembly, and . denotes the control input torque applied to the
%
%%%% driver input device; represents an auxiliary
% target dynamic function for the driver input device; and are scaling constants. The structure for (5) is
% motivated by the following philosophy: If T1(t) was exactly
%
%%%% equal to α2τ2(t) in (1), then the dynamics given by (1) offer the
% driver a realistic experience, provided that auxiliary term N1(·)
% may be designed or constructed according to some desirable
% mechanical response. Thus, the NT (·) term in (5) is designed
% to simulate the desired driving experience, and hence, this
%
%%%% dynamic equation serves as a desired trajectory generator for
% the driver input device control design.
%
%%%% 
%%%% Remark 2: The target dynamic function NT (·) must be
% selected to ensure that the desired driver input device trajectory tory and its first two derivatives remain bounded at all times.
% Suppose that NT (·) is selected as where BT and KT are some positive design constants.
% If NT (·) is designed according to (6), then standard linear
% system arguments can be used to show that θd1(t), θd1(t), and
% θd1(t) remain bounded. Furthermore, NT (·) can be constructed
% as a nonlinear damping function by utilizing Lyapunov-type
% arguments.
%%% 
%%%% INTERACTION FORCES: PART I (MEASURABLE)
% In this section, the adaptive steer-by-wire tracking controller
% is developed. The first control objective requires that both
% the target system tracking error and the locked tracking error
% signals be asymptotically regulated to zero while adapting for
% the uncertain system parameters. The control design is based
% on the assumption that the signals θ1(t), θ2(t), θ˙1(t), and θ˙2(t)
% are measurable and that driver input torque τ1(t) and the torque
% acting on the directional control system τ2(t) are measurable.
%%% 
%%%% Open-Loop Error System Development
% To quantify the mismatches between the target and primary
% systems (or driver experience tracking error), as well as the
% primary and secondary systems (or locked tracking error), the
%filtered error signals r1(t), r2(t)  are defined as
% r1 = ˙ e1 + µ1e1 r2 = ˙ e2 + µ2e2 (7)
% where µ1 and µ2 ∈ ℜ1 represent positive control gains, and
% error signals e1(t), e2(t) are
% e1 = θd1 − θ1 e2 = θ1 − θ2. (8)
% After taking the first time derivative of (7) and substituting
% the dynamics in (1) and (2), and (5) the open-loop error
% systems are
% I1r˙1 = Y1φ1 − T1 (9)
% I2r˙2 = Y2φ2 − T2 (10)
% where Y1(θ1, θ˙1, τ1, θ¨d1, θ˙d1) ∈ ℜ1×r and Y2(θ1, θ˙1, θ2, θ˙2, τ1,
% τ2, T1) ∈ ℜ1×s are regression matrices consisting of measurable quantities, and φ1  ℜr×1, φ2 are constant unknown vectors explicitly defined as follows:
%
%%%% Y1 = [YN1, −τ1, θ¨d1 + µ1e˙1] (11)
% φ1 = [φN1, α1, I1]T (12)
% Y2 = [−YN1, τ1, T1, YN2, −τ2, µ2e˙2] (13)
% φ2 = I I2 1 φN1, II21 α1, II2 1 , φN2, α2, I2T . (14)
%
%%%% 
%%%% Remark 3: Based on the definition of r1(t) and r2(t) given in
%(7), standard arguments [23] can be used to prove that 1) if r1(t)
%and r2(t) remain bounded, then e1(t), e2(t), and e˙1(t), e˙2(t)
%are also bounded, and 2) i r1(t) and r2(t) are asymptotically
% regulated, then e1(t) and e2(t) are asymptotically regulated.
%%%
%%%%In this section, we provide a brief overview of the single-wheel model of rubber-tired vehicles that are subject to  straight-ahead braking conditions. This single-wheel dynamic model can capture the steady and transient tractive performance while demonstrating how a vehicle can undergo lockup or stable braking~\cite{olson2003nonlinear,johansen2003gain,de2012torque,li2018hierarchical}. The dynamic states are often chosen to be the tire/wheel rate of  rotation and the forward vehicle speed. Hence, the quarter-car dynamics that govern the vehicle longitudinal motion during braking  are given by (see, e.g.,~\cite{johansen2003gain,de2012torque})
%%%%%
%%%%\begin{subequations}\label{eq:tractionDyn1}
%%%%	\begin{align}
%%%%	\dot{v} = -g_\alpha \mu(\lambda) - \frac{\Delta_v(t,v)}{M},\\
%%%%	\dot{\omega} = \frac{M g_\alpha r}{J} \mu(\lambda) - \frac{T_a}{J} - \frac{\Delta_w(t,\omega)}{J}, 
%%%%	\end{align}
%%%%\end{subequations}
%%%%%
%%%%\noindent where the parameters $M$, $r$, and $J$  are the quarter-car mass, wheel radius, and wheel inertia, respectively.  Additionally, during braking, the vehicle speed $v$ and the wheel rotational speed $\omega$ vary within in the set $\mathcal{D}_b:=\{ (v,\omega) | v > 0, \; 0 \leq r\omega \leq v \}$. The braking torque $T_a$ is the input to the dynamical system in~\eqref{eq:tractionDyn1}. Furthermore, the longitudinal slip $\lambda$ that determines whether the wheel is locked is given by 
%%%%%
%%%%\begin{equation}\label{eq:slipDefn}
%%%%\lambda := \frac{v- r\omega}{\max(v, r\omega)}.
%%%%\end{equation}
%%%%
%%%%%% Figure environment removed
%%%%%%
%%%%While braking actuators are engaged, we have $\lambda = \tfrac{v- r\omega}{v}$ and $(v,\omega) \in \mathcal{D}_b$. It follows that $\lambda \in [0,1]$ during braking. We let the constant $g_\alpha$ denote $g \cos(\alpha)$ where $\alpha$ is the road slope. Finally, $\mu(\lambda)$, $\Delta_v(t,v)$, and $\Delta_w(t,\omega)$ denote the uncertain nonlinear friction coefficient, the force, and the torque disturbances resulting from unmodeled dynamics, respectively. 
%%%%
%%%%%\begin{remark}
%%%%%	\label{rem:mu}
%%%%There are a variety of ways to represent the function $\mu(\cdot)$ including the Magic Formula and Burckhardt representation (see, e.g.,~\cite{de2011optimal}). For instance, equations like Burckhardt  model (see, e.g.,~\cite{de1999model}) where 
%%%%%
%%%%\begin{equation}
%%%%\mu(\lambda) = c_1 (1-\exp(-c_2\lambda)) - c_3 \lambda,
%%%%\label{eq:burck}
%%%%\end{equation}
%%%%%
%%%%are empirical equations based on fitting coefficients that are widely utilized for modeling the interaction between the road pavement and tire tread. The longitudinal force on the tire arising from this interaction is computed by $-\mu(\lambda) g_\alpha$. 
%%%%
%%%%In this paper, we do not assume any particular closed-form representation for the nonlinear friction coefficient function $\mu(\cdot)$ and only assume that $\mu : \Lambda \to \mathbb{R}$ is a continuous function on the closed interval $\Lambda := [0,1]$. Accordingly,  $\mu(\cdot)$ attains its maximum $\mu_{\tx{max}}$ and minimum $\mu_{\tx{min}}$ on the closed interval $\Lambda$ because of the well-known properties of continuous functions on compact sets.
%%%%%\end{remark}
%%%%
%%%%It will be assumed that the disturbances $\Delta_v(t,v)$, and $\Delta_w(t,\omega)$ satisfy the following uniform bounds (see, e.g.,~\cite{de2012torque})
%%%%%
%%%%\begin{eqnarray}\label{eq:distBound}
%%%%\nonumber
%%%%|\Delta_v(t,v)| \leq \bar{\Delta}_v,\;  |\Delta_w(t,\omega)| \leq \bar{\Delta}_\omega, \\
%%%%\text{ for all } (t,v,\omega) \in [0,\infty) \times \mathcal{D}_b. 
%%%%\end{eqnarray}
%%%%
%%%%It is possible to transform the longitudinal dynamics by a change of coordinates from $(v,\omega)$ to $(v, \lambda)$. Under this change of coordinates, the longitudinal dynamics become (see, e.g.,~\cite{de2012torque,olson2003nonlinear} for the details of derivation) 
%%%%\begin{subequations}\label{eq:tractionDyn2}
%%%%	\begin{align}
%%%%	\dot{v} &= -g_\alpha \mu(\lambda) - \frac{\Delta_v(t,v)}{M},\\
%%%%	\dot{\lambda} &= \frac{g_\alpha}{v}\big\{ (\lambda - 1 - \nu) \mu(\lambda) + \Upsilon_a + \Upsilon_{\Delta, w} + (\lambda -1)\Upsilon_{\Delta, v} \big\}, 
%%%%	\end{align}
%%%%\end{subequations}
%%%%\noindent where $\nu:=\tfrac{M R^2}{J}$ is the dimensionless ratio of vehicle to wheel inertia, $\Upsilon_a := \frac{r}{J g_\alpha} T_a$ is the dimensionless brake torque, and $\Upsilon_{\Delta, w} := \frac{r}{J g_\alpha} \Delta_w(t, \omega)$, $\Upsilon_{\Delta, v} := \frac{\Delta_v(t,v)}{M g_\alpha}$ are the dimensionless force and torque disturbances acting on speed and slip dynamics, respectively. 
%%%%
%%%%The dynamical equations in~\eqref{eq:tractionDyn1} and~\eqref{eq:tractionDyn2} capture the coupling in-between the wheel slip $\lambda$ dynamics with that of the vehicle speed $v$ in case of~\eqref{eq:tractionDyn2}, or the tire/wheel angular speed $\omega$ dynamics with that of the vehicle speed $v$ in case of~\eqref{eq:tractionDyn1}. In $(v, \lambda)$ coordinates, the set  $\mathcal{D}_b$ reads as 
%%%%%
%%%%\begin{equation}
%%%%\mathcal{D}_b=\big\{ (v,\lambda) | v > 0, \; \lambda \in \Lambda := [0,1] \big\}.
%%%%\label{eq:Db}
%%%%\end{equation}
%%%%%
%%%% 
%%%% Using the brake input $\Upsilon_a$, the adversary would like to induce unstable braking conditions corresponding to lockup. The most severe case of lockup happens when $\lambda=1$. Therefore, following Olson \emph{et al.}~\cite{olson2003nonlinear}, we define \textbf{the lockup manifold} in the following way
%%%%%
%%%%\begin{equation}
%%%%\mathcal{W}_b^L := \big\{ (v,\lambda) \big| v>0,\, \lambda = 1 \big\}.
%%%%\label{eq:lockupManifold}
%%%%\end{equation}
%%%%%
%%%%\begin{remark}
%%%%	It is remarked that the adversary can choose the slip reference value $\lambda^{\text{r}}$ in advance, which can belong to the interval $[0,1]$.  The closer $\lambda^{\text{r}}$ to one, the closer the wheel to the condition of lockup. Without loss of generality, we assume that $\lambda^{\text{r}}$ has been chosen to be equal to one. 
%%%%\end{remark}
%%%%
%%%%To model the adversarial disruption resources, we assume that the reference malicious command generated by the attack policy $\hat{\Upsilon}_a$ passes through the following first-order delay system (see, e.g.,~\cite{de2012torque,li2018hierarchical})  
%%%%%
%%%%\begin{equation}
%%%%\tau_f \dot{\Upsilon}_a  = - \Upsilon_a + \hat{\Upsilon}_a(t- \delta_f), 
%%%%\label{eq:torqueResp}
%%%%\end{equation}
%%%%% 
%%%%to generate the frictional braking torque response $\Upsilon_a$, which then gets applied to the traction dynamics in~\eqref{eq:tractionDyn2}.  It is remarked that the attacker does not have any knowledge of either the friction brake time constant $\tau_f$ or the friction brake deadtime $\delta_f$. In designing our attack policies in the next section, we assume that $\tau_f \approx 0$ and $\delta_f \approx 0$. However, the simulation results in Section~\ref{sec:sims} demonstrate the effectiveness of the attack policies when these assumptions do not hold.

%In this section, we present an attack policy that can achieve the wheel lockup attack policy objective in the previous section. The proof of the stated propositions are removed for the sake of brevity. The  attack policy relies on a feedback and a feedforward control action. The feedback control input is generated through a predefined-time controller~\cite{sanchez2015predefined} that can cause wheel lockups if the tire-road interaction characteristics and other relevant parameters in the vehicle traction dynamics are completely known. Against the lack of such information,  it is shown that the  adversary can employ an additional feedforward control input that is generated by a nonlinear disturbance observer. %The NDOB will compensate for the adversarial limited knowledge of the vehicle traction dynamics. 
%%We start with a simple attack policy based on the concept of predefined-time controllers proposed in~\cite{sanchez2015predefined} and assume that %the adversary has an almost perfect knowledge of the vehicle dynamics. 
%%The sophistication level of the attack policy analysis and design by assuming less knowledge of the dynamical model. 
%
%\noindent\textbf{Predefined-time controller design.} Following the notation in~\cite{sanchez2015predefined}, we let 
%%
%\begin{equation}
%\Phi_p(x):=\frac{\exp(|x|^p)}{p} |x|^{1-p} \tx{sign}(x),  
%\label{eq:phip}
%\end{equation}
%%
%for any $x\in \mathbb{R}$ and some real constant $0 < p < 1$, and   
%\begin{equation}
%\Phi_1(x):=\exp(|x|) \tx{sign}(x),
%\label{eq:phi1}
%\end{equation}
%%
%for any $x\in \mathbb{R}$. Furthermore, we define the lockup error as
%%
%\begin{equation}
%\label{eq:eL}
%e_L := \lambda - 1. 
%\end{equation}
%%
%\noindent Hence, if $e_L=0$ and $v>0$, the wheel is in a locked stated. As it will be demonstrated in this section, if the attack objective is met, the wheel will be locked in finite time. Hence, the vehicle speed will satisfy  
%%
%\begin{equation}
%\label{eq:speedBound}
%v \in [v_\tx{min}, v_\tx{max}], 
%\end{equation}
%%  
%during a successful attack, for some positive $v_\tx{min}$ and $v_\tx{max}$. 
%%Using the functions defined in~\eqref{eq:phip} and~\eqref{eq:phi1}, we define the following 
%%feedback control laws 
%%%
%%\begin{eqnarray}
%%u_{np}^a(e_\tx{L}) &:=& -k_a \tx{sign}(e_L) -\frac{1}{T_c} \Phi_p(e_L), 
%%\label{eq:unp}
%%\\
%%u_{n1}^a(e_\tx{L}) &:=& -\big(\frac{1}{T_c} + k_a \big) \Phi_1(e_L),
%%\label{eq:un1}
%%\end{eqnarray}
%%%
%%\noindent where $k_a$, $T_c$, and $0<p<1$  are some positive real constants. 
%%
%\begin{proposition}
%	\label{prop:prop1}
%	Consider the vehicle longitudinal dynamics $\mathcal{P}$ in~\eqref{eq:tractionDyn2} with the attacker's \emph{a priori} knowledge $\hat{\mathcal{P}}$ in~\eqref{eq:tractionDyn2approx} and the frictional braking response given by~\eqref{eq:torqueResp} with $\tau_f \approx 0$ and $\delta_f \approx 0$.  Suppose $\Upsilon_{\Delta, w}=0$, $\hat{\nu}=\nu$, and $\hat{\mu}(\cdot)=\mu(\cdot)$. Additionally, assume that the uniform bound on $\Delta_v(t,v)$ given by~\eqref{eq:distBound} holds. Given any positive constant $T_c$, the attack policy
%	\begin{equation}
%	\label{eq:attPolicyNoDob}
%	\hat{\Upsilon}_a = \frac{v}{g_\alpha} u_{np}^a(e_\tx{L}) + \hat{\nu} \hat{\mu}(\lambda),  
%	\end{equation}
%	with $u_{np}^a(e_\tx{L})=  -(\tfrac{1}{T_c} + k p ) \Phi_p(e_L)$, where $0 < p < 1$, $k \geq k^{\ast\prime}:=\tfrac{M g_\alpha \mu_{\max} + \bar{\Delta}_v}{M v_{\tx{min}}}$, and $\Phi_p(\cdot)$ given by~\eqref{eq:phip}, makes the lockup manifold $\mathcal{W}_b^L$  globally finite-time stable with settling-time $T_c$. 
%\end{proposition}
%
%%\noindent\textbf{Proof.} The closed-loop dynamics of $e_L$ under the stated assumptions become  
%%\begin{equation}\label{eq:errorDyn0}
%%\dot{e}_L  =   u_{np}^a(e_\tx{L})  + \Delta_e(t,e_L),  
%%\end{equation}
%%%
%%\noindent where $\Delta_e(t,e_L) = \tfrac{g_\alpha}{v} e_L \{\mu(\lambda) + \Upsilon_{\Delta, v} \}$ is a vanishing 
%%perturbation satisfying $|\Delta_e(t,e_L)| \leq \tfrac{M g_\alpha \mu_{\max} + \bar{\Delta}_v}{M v_{\tx{min}}} |e_L|$. Therefore, 
%%by Lemma~4.1 in~\cite{sanchez2015predefined}, the proposition is proved.\hfill$\blacksquare$
%
%\begin{remark}
%Proposition~\ref{prop:prop1} assumes an almost perfect knowledge of the vehicle's model, where the only unknown is the 
%disturbance force $\Delta_v(t,v)$ acting on the vehicle speed dynamics in~\eqref{eq:tractionDyn2}. The next proposition 
%removes these restrictions further.  
%\end{remark} 
%
%\begin{proposition}
%	\label{prop:prop2}
%	Consider the stated assumptions in Proposition~\ref{prop:prop1} with $\hat{\nu}$ arbitrary, and     $\Delta_w(t,\omega)$, $\Delta_v(t,v)$ satisfying~\eqref{eq:distBound}. Furthermore, assume  that $\hat{\mu}: \Lambda \to \mathbb{R}$ is a continuous function. Then, given $0 < T_c < 1$, the attack policy~\eqref{eq:attPolicyNoDob} with $u_{np}^a(e_\tx{L})=  -k_a \tx{sign}(e_L) -\tfrac{1}{T_c} \Phi_p(e_L)$, $0 < p < 1$, $\Phi_p(\cdot)$ given by~\eqref{eq:phip}, and $k_a \geq k^\ast$ in which  
%	\begin{equation}
%	\label{eq:kBound}
%		k^\ast := \frac{M g_\alpha \mu_{\max} + \bar{\Delta}_v}{M v_{\tx{min}}} + \frac{g_\alpha}{v_{\tx{min}}}\big( \hat\nu \hat\mu_{\text{max}} + \nu \mu_{\text{max}} + \frac{r \bar{\Delta}_w}{J g_\alpha} \big), 
%	\end{equation}
%	makes the lockup manifold $\mathcal{W}_b^L$ globally finite-time stable with settling-time $T_c$. 
%\end{proposition}
%
%
%%\noindent\textbf{Proof.} The closed-loop dynamics of $e_L$ under the stated assumptions read as 
%%	\begin{equation}\label{eq:errorDyn2}
%%	\dot{e}_L  =   u_{np}^a(e_\tx{L})   + \Delta_e^{\prime}(t,e_L),   
%%	\end{equation}
%%	% 
%%	\noindent where 
%%	\begin{eqnarray}
%%	\nonumber
%%	\Delta_e^{\prime}(t,e_L) & = & \frac{g_\alpha}{v} e_L \big\{\mu(\lambda) +  \Upsilon_{\Delta, v}  \big\} + \cdots \\
%%    &  & \frac{g_\alpha}{v} \big\{\hat{\nu}\hat{\mu}(\lambda) - \nu\mu(\lambda)\big\} + \frac{g_\alpha}{v} \Upsilon_{\Delta, w},
%%	\label{eq:kbound2}
%%	\end{eqnarray}
%%	%  
%%	is a non-vanishing perturbation. Due to the definition of wheel slip, we have $|e_L| \leq 1$ for all values of $\lambda \in \Lambda$. Therefore, it follows that $|\Delta_e^{\prime}(t,e_L)| \leq k^\ast$, where $k^\ast$ is given by~\eqref{eq:errorDyn2}. Consequently, 
%%	by Lemma~4.3 in~\cite{sanchez2015predefined}, the proposition is proved.\hfill$\blacksquare$
%
%The following proposition, whose proof is omitted for the sake of brevity, removes the restrictions on the settling-time in Proposition~\ref{prop:prop2}. 
%
%\begin{proposition}
%Consider the stated assumptions in Proposition~\ref{prop:prop2}. Given any positive constant $T_c$, the attack policy 
%\begin{equation}
%\label{eq:attPolicyNoDob3}
%\hat{\Upsilon}_a = \frac{v}{g_\alpha} u_{n1}^a(e_\tx{L}) + \hat{\nu} \hat{\mu}(\lambda),  
%\end{equation}
%with $u_{n1}^a(e_\tx{L})=  -(\tfrac{1}{T_c} + k_a) \Phi_1(e_L)$, where $k_a \geq k^\ast$, $k^\ast$ given by~\eqref{eq:kBound}, and $\Phi_1(\cdot)$ given by~\eqref{eq:phi1}, makes the lockup manifold $\mathcal{W}_b^L$ globally finite-time stable with  settling-time $T_c > 0$.  
%\label{prop:prop3}	
%\end{proposition}
%%
%%
%%\begin{eqnarray}\label{eq:errorDyn}
%%\nonumber
%%\dot{e}_L & =   \frac{g_\alpha}{\hat{g}_\alpha} u_n^a(e_\tx{L}) - \frac{g_\alpha}{v} \hat{d}_a  + \frac{g_\alpha}{v} e_L \big\{\mu(\lambda) + \Upsilon_{\Delta, v}  \big\} + \\
%%& \frac{g_\alpha}{v} \big\{\hat{\nu}\hat{\mu}(\lambda) - \nu\mu(\lambda)\big\} + \frac{g_\alpha}{v} \Upsilon_{\Delta, w} 
%%\end{eqnarray}
%%locally finite-time convergent to a compact set $\Omega_b^L$ containing $\mathcal{W}_b^L$ such that $|e_L| \leq f(| d |_\infty)$. 
%%
%\noindent\textbf{Nonlinear disturbance observer design.} Thus far, the presented family of attack policies in Propositions~\ref{prop:prop1}--\ref{prop:prop3} depend on some \emph{a priori} knowledge of the vehicle  longitudinal dynamics $\mathcal{P}$ in~\eqref{eq:tractionDyn2}. Against the lack of such information in realistic scenarios, we add a feedforward compensation term to the proposed attack policies. In particular, we extend the brake attack policies in~\eqref{eq:attPolicyNoDob} and~\eqref{eq:attPolicyNoDob3} according to 
%%
%\begin{equation}
%\label{eq:attPolicyDob}
%\hat{\Upsilon}_a = \frac{v}{g_\alpha} u_{ni}^a(e_\tx{L}) + \hat{\nu} \hat{\mu}(\lambda) - \hat{d}_a, \; i=1,\, p, 
%\end{equation}
%%
%where $\hat{d}_a \in \mathbb{R}$ is the output of the following NDOB (see, e.g.,~\cite{chen2004disturbance,mohammadi2017nonlinear} for details of derivation)
%%
%\begin{subequations}\label{eq:dobDyn}
%	\begin{align}
%	\dot{z}_a &= -L_d z_a - L_d \big\{ u_{ni}^a + \frac{g_\alpha}{v} (-\hat{d}_a + 
%	\hat{\nu}\hat{\mu}(\lambda)  + p_a) \big\},\\
%	\hat{d}_a &= z_a + p_a, 
%	\end{align}
%\end{subequations}
%%
%\noindent where $z_a \in \mathbb{R}$ is the state of the NDOB, and the relationship $p_a = L_d e_L$ between $L_d$, namely, the 
%NDOB gain, and $p_a$, namely, the NDOB auxiliary variable, holds. 
%%
%%\begin{equation}
%%\label{eq:pLrel}
%%p_a = L_d e_L,   
%%\end{equation}
%%
%\noindent Therefore, it follows that $L_d = \tfrac{\partial p_a}{\partial e_L}$. 
%
%%The NDOB in~\eqref{eq:dobDyn}  has been designed based on considering the wheel slip error dynamics 
%%%
%%\begin{equation}\label{eq:errorDyn3}
%%\dot{e}_L  =   u_{ni}^a(e_\tx{L})   + \Delta_e^{\prime}(t,e_L) - \hat{d}_a, \, i=1,\, p,   
%%\end{equation} 
%%% 
%%where $\Delta_e^{\prime}(t,e_L)$ denotes the lumped disturbance. The NDOB output $\hat{d}_a$, which is being calculated by~\eqref{eq:dobDyn}, tries to cancel the lumped disturbance $\Delta_e^{\prime}(t,e_L)$. In the ideal case, when $\hat{d}_a \approx \Delta_e^{\prime}(t,e_L)$, complete disturbance cancellation is achieved and it appears to the predefined-time controller, which was designed in Step~1, that it is controlling a system with no disturbances. The block diagram of the proposed attack policy is depicted in Figure~\ref{fig:blockDiag1}. 
%
%% Figure environment removed
%
%The convergence properties of the disturbance tracking error are well-studied in the literature (see, e.g.,~\cite{li2014disturbance,chen2004disturbance,mohammadi2017nonlinear}) and for the sake of brevity we refer the readers to the aforementioned references. 
%
%\begin{remark}\label{rem:choice}
%	The NDOB in~\eqref{eq:dobDyn} has only one dynamic state and it does not rely on having a particular representation such as the Burckhardt closed-form for the nonlinear friction coefficient function $\mu(\cdot)$. Indeed, whenever no knowledge of $\mu(\cdot)$ is available, the adversary can set $\hat{\mu}(\lambda)$ to be equal to zero in~\eqref{eq:dobDyn}.   This NDOB-based disturbance compensation technique is unlike the adaptive algorithms in~\cite{de2012torque,li2018hierarchical} where a particular representation of the friction coefficient function is needed and several parameters need to get updated simultaneously. As it will be seen in the simulations, even when $\hat{\mu}(\lambda)$ is set to zero, corresponding to a complete lack of knowledge by the adversary, the attack policy using the NDOB will meet its objectives. 
%\end{remark}

