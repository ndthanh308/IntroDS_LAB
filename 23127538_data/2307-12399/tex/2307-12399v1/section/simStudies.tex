\section{Simulation Results}
\label{sec:sims}
%
In this section we present numerical simulation results to demonstrate the effectiveness of the proposed attack. The simulation parameters are given in Table~\ref{tab:TableSim}, where the HSC steering system  parameters are directly adopted from~\cite{baviskar2008adjustable}. 

	\begin{table}[t]
	\centering
	\begin{tabular}{|c|c|c|c|}
		\hline
		\rowcolor[HTML]{93C47D} 
		{\color[HTML]{93C47D} } 	Variable			 &   Value &  	Variable  & Value   \\ \hline
		%
		%
		$I_\text{sw}$ &  $1.16\times 10^{-2}$ kg.m\textsuperscript{2} &   $I_\text{c}$  & $2.35\times 10^{-2}$ kg.m\textsuperscript{2}  \\ %\hline
		%
		%
		$B_\text{sw}$ &  $1.9\times 10^{-2}$ kg.m\textsuperscript{2}/s& 	$B_\text{c}$ &  $6 \times 10^{-2}$  kg.m\textsuperscript{2}/s\\ %\hline
		%
		%
		$K_\text{sw}$ & $0$ N.m &      $K_\text{c}$ &  $0$ N.m  \\ %\hline
		%
		$\alpha_\text{sw}$ & $1$ &      $\alpha_\text{c}$ &  $1$  \\ %\hline
		%
		%
		$\alpha_{T_{\text{sw}}}$ & $1$ &      $\alpha_{T_{\text{c}}}$ &  $0.15$  \\ %\hline
		%
		$\gamma$ & $0.02$ &      $C_d$ &  $150$ N.m  \\ \hline
	\end{tabular}
	\caption{\label{tab:TableSim} \small Numerical simulation parameters and their values.}
	%\vspace{-2ex}
\end{table} 

Due to the tire–road interface reaction forces, a reaction torque will be exerted on the directional control assembly (see the block diagram in Figure~\ref{fig:steerBlock}). In the simulations, we have assumed the following functional form for the reaction torque $\tau_{\text{c}}$ (see, e.g.,~\cite{setlur2006trajectory}) 
%
\begin{equation}
	\tau_{\text{c}} = -C_d \tanh(\gamma \theta_{\text{c}}), 
	\label{eq:tauCForm}
\end{equation}
%
\noindent where the values of $\gamma$ and $C_d$ are given in Table~\ref{tab:TableSim}. 

In addition to the HSC steering system dynamics,   we utilize a bicycle model for the vehicle dynamics (see Figure~\ref{fig:bikeModel}) to study the impact of the proposed attacks on the trajectory of a vehicle that is performing an obstacle clearing maneuver. In the bicycle model, we have  employed a hybrid physical/dynamic tire/road friction model.  The readers are referred to~\cite{li2013hybrid} for further details about this modeling approach. 
%
% Figure environment removed

In the first collection of numerical simulations, we assume that there exists no cyber-physical attack on the HSC steering control system while the driver is performing a simple obstacle clearing maneuver. The nominal HSC steering control system is utilizing the asymptotically stable control schemes originally proposed by~\cite{baviskar2008adjustable}. Figures~\ref{fig:Nominal1} and~\ref{fig:Nominal2} depict the nominal HSC steering system torque  and angular position/speed time profiles, respectively. Figure~\ref{fig:Nominal3} depicts the trajectory of the vehicle under the nominal obstacle clearing maneuver.
 
% Figure environment removed
%
% Figure environment removed
%
% Figure environment removed

In the second collection of numerical simulations under the proposed time-varying impedance attack against the target vehicle HSC steering system, we have assumed that the adversarial inertia and time-varying stiffness and damping profiles are given by $I_T=1\times 10^{-2}$ kg.m\textsuperscript{2}, $K_T(t)=2.8\times 10^{-2}\exp(2.1\alpha t)$ N.m, where $\alpha=0.5$, and $B_T(t)=4.99\times 10^{-3}$ kg.m\textsuperscript{2}/s, respectively. These time-varying stiffness and damping time profiles satisfy~\eqref{eq:ineq1} and~\eqref{eq:finalConst}, which guarantee a non-passive adversarial target dynamics. Furthermore, we have considered the weighting factors  $\alpha_{T_{\text{sw}}}$ and $\alpha_{T_{\text{c}}}$ in~\eqref{eq:targetMain} to be equal to $0.15$ and $1$, respectively. The parameters of the closed-loop adaptive attack policy inputs $T_{\text{sw}}$ and $T_{\text{c}}$ and their update laws in~\eqref{eq:adaptiveUpdate} and~\eqref{eq:Tattack} are chosen to be  $\mu_{\text{sw}}=\mu_{\text{c}}=1.01$, $k_{\text{sw}}=k_{\text{c}}=1$, $\Gamma_{\text{c}}=80\mathbf{I}_8$, and $\Gamma_{\text{sw}}=80\mathbf{I}_4$, respectively. Note that we denote the identity matrix of size $N$ by $\mathbf{I}_N$. 

Figures~\ref{fig:unstable1} and~\ref{fig:unstable2} depict the attacked HSC steering system torque  and angular position/speed time profiles, respectively. As it can be seen from Figure~\ref{fig:unstable2}, the target dynamics  are injecting energy into the driver input device and the steering column systems by providing the closed-loop adaptive control policy with angular reference commands that get modulated in response to the driver's input torque and the steering column torque. Additionally, from the same figure, it can be seen that the closed-loop adaptive control policy is making the driver's input device and the steering column angular positions to closely follow the malicious reference command $\theta_d$ generated by the adversarial target dynamics in~\eqref{eq:targetMain}. 
%
% Figure environment removed

% Figure environment removed

Figure~\ref{fig:unstable3} depicts the trajectory of the vehicle when the obstacle clearing maneuver is taking place under the proposed time-varying impedance attack input. As it can be seen from this figure, the driver fails to clear the obstacle and the targeted vehicle collides with the obstacle. Intuitively, this collision can be explained by noticing that the attacked HSC steering system dynamics are stiffening  under the influence of the time-varying stiffness $K_T(t)=2.8\times 10^{-2}\exp(2.1\alpha t)$ in the adversarial target dynamics. This is why the attacked HSC steering system cannot make a proper response to the human driver's input in contrast with the nominal obstacle clearing maneuver depicted in Figure~\ref{fig:Nominal3}. This stiffening effect can also be observed when comparing the nominal steering column torque time profile $\tau_{c}(t)$ in Figure~\ref{fig:Nominal1} with its counterpart (under attack scenario) in Figure~\ref{fig:unstable1}. 
%
% Figure environment removed

Figure~\ref{fig:unstable4} depicts the evolution of the elements of the estimated vectors  $\hat\phi_{\text{sw}}$ and  $\hat\phi_{\text{c}}$ under the adversarial adaptive update laws given by~\eqref{eq:adaptiveUpdate}. It should be noted the class of adaptive update laws, which rely on the linear parametrization property of the HSC steering dynamics, only guarantee convergence properties of angular position tracking errors. However, convergence to the true values of the unknown parameter vectors $\phi_{\text{sw}}$ and  $\phi_{\text{c}}$ is not a guaranteed feature of these adaptation rules.
%
% Figure environment removed
%
%% Figure environment removed
%
%In the numerical simulations, two driver input torque profiles from~\cite{baviskar2008adjustable}  for performing slalom and circular path tracking maneuvers are considered. In particular, we consider the following two driver torque inputs 
%%
%\begin{equation}
%\tau_{\text{sw}}(t) = 0.8\sin(5t)(1 - \exp(-3t) ), \tau_{\text{sw}}(t) = 0.9(1 - \exp(-3t) ), 
%\end{equation}
%%
%where the first one represents a slalom maneuver and the second one represents a circular path tracking maneuver.
%
%% Figure environment removed
%
%% Figure environment removed
%
%% Figure environment removed
%
%
%% Figure environment removed
%
%
%% Figure environment removed
%
%Since the adversary would like to 
%induce an almost complete wheel lockup condition during braking, it is desired that the trajectories $(v(t),\lambda(t))$ of the vehicle nonlinear traction dynamics in~\eqref{eq:tractionDyn2} converge to a very near vicinity of the lockup manifold $\mathcal{W}^b_L$ defined in~\eqref{eq:lockupManifold} within a relatively small amount of time (here, $T_c=0.95$ seconds).  
%Out of the five attack policies employed by the adversary, one of them corresponds to applying a constant brake torque to the wheel, which is indeed a naive attack based on the assumption that with a relatively large brake torque the adversary can induce lockup in the wheels. The other four attacks employ the presented predefined-time controllers in the paper, where two of them that are given by~\eqref{eq:attPolicyNoDob}, with $p=0.15$ and $T_c=0.95$, do not possess any NDOB-based dynamic compensation mechanism. On the other hand, the last two predefined-time controllers, with $p=0.15$ and $T_c=0.95$, are employing the control policy in~\eqref{eq:attPolicyDob} with the disturbance estimate generated by the NDOB given by~\eqref{eq:attPolicyDob} with $L_d=2.65$. 
%
%
% The nonlinear friction coefficient function is modeled using the three-parameter Burckhardt model in~\eqref{eq:burck}.  It is assumed that the adversary \emph{has no knowledge} of the nonlinear friction coefficient function. Accordingly, in all of the four non-constant adopted attack policies, $\hat{\mu}(\lambda)$ is set equal to zero. In Figure~\ref{tab:simTab}, the coefficients associated with dry and wet asphalt road conditions are given by $c_i$ and $c_i^\prime$, $1 \leq i \leq 3$, respectively. 
%Furthermore, it is assumed that the adversary \emph{has no knowledge} of either the friction brake time constant $\tau_f$ or the friction brake deadtime $\delta_f$. Finally, it is assumed that the adversary \emph{has no knowledge} of the lower bounds $k^{\ast\prime}$ and $k^\ast$  in Propositions~\ref{prop:prop1} and~\ref{prop:prop2}. Therefore, $k$ in all four cases is set equal to zero. Finally, in the presented simulations, the external disturbances not related to road-tire interaction forces, i.e., $\Delta_v(t,v)$ and $\Delta_w(t,\omega)$, are set equal to zero. 
%
%Figure~\ref{fig:simTimeProf} depicts the speed, wheel slip, and disturbance profiles from the simulations. As it can be seen from the Figure, in the three scenarios where the adversary does not employ the NDOB in~\eqref{eq:attPolicyNoDob3}, the attack objective is not met. 
%
%It remains an open question how an adversary can devise an attacking device for realizing the proposed wheel lockup attacks in this paper. An initial direction could be the line of work by Palanca \emph{et al.} in~\cite{palanca2017stealth}, where they crafted an inexpensive attacking device that utilizes an Arduino Uno Rev 3, a Microchip MCP2551 E/P, and an SAE J1962 Male Connector. Their device, which was powered by a simple 12V battery, could be physically plugged into the OBD-II port of their target vehicle, namely, a 2012 Alfa
%Romeo Giulietta. 
%
%
%%We remark that had the adversary known an approximate representation of the tire-road interaction characteristics as well as the lower bounds $k^{\ast\prime}$ and $k^\ast$  in Propositions~\ref{prop:prop1} and~\ref{prop:prop2} (as opposed to setting all of them equal to zero), we still would have expected that the predefined-time attack policies without NDOBs to be successful on their own. The NDOB here is playing its well-known ``add-on'' role described in the DOB design literature~\cite{li2014disturbance,chen2004disturbance}. Indeed, it is \emph{the interplay in-between the predefined-time controller and the NDOB that makes the proposed braking attack policy less dependent on having a proper adversarial knowledge of the tire-road interaction characteristics.} 
%%
%
%
%%% Figure environment removed
%	
%%	\begin{subfigure}{0.4\textwidth}
%%		% Figure removed 
%%		\caption{}
%%	\end{subfigure}
%%    \begin{subfigure}{0.4\textwidth}
%%    	% Figure removed 
%%    	\caption{}
%%    \end{subfigure}
%
%%	\label{fig:path}
%%\end{figure*}