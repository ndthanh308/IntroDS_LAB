\section{Concluding Remarks and Future Research Directions}
\label{sec:conc}
%
The aim of this study was to explore the potential of adversarial techniques in disrupting the interaction dynamics between human drivers and vehicle haptic shared control (HSC) steering systems. While previous research in robotics has focused on ensuring stability and passivity in human-robot interaction dynamics, this paper took a different approach. It proposed the synthesis of time-varying impedance profiles that would create a non-passive interaction between the human driver and the HSC steering system. The study highlighted the importance of HSC steering systems in critical safety scenarios, such as facilitating smooth transitions from automation to human control and reducing conflicts between drivers and vehicles. Through practical experiments, this paper demonstrated the physical capabilities of adversaries who aim to destabilize the stability of driver-vehicle interaction. Future research will include modeling the neuromuscular dynamics of the drivers in reaction to such attacks and devising real-time passivity monitoring algorithms (see, e.g.,~\cite{welikala2022line}) for detection of such attacks against the vehicle HSC steering system.  