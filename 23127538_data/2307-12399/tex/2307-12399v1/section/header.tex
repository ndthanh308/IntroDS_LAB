%% Modified for NDSS 2023 by DB on 2022/04/04
% Header.tex

\usepackage{array,pdfsync,enumerate}
\usepackage[dvips]{epsfig}
\usepackage{amssymb,amsmath}
\usepackage{caption}
\usepackage{subcaption}
%\usepackage{multirow}
%
\usepackage{color}
\definecolor{linkcol}{rgb}{0,0.0,1.0}
\definecolor{citecol}{rgb}{0.0,0.6,0.0}

%\usepackage[dvipsnames]{xcolor}
\usepackage[table,xcdraw]{xcolor}

\IEEEoverridecommandlockouts
% The preceding line is only needed to identify funding in the first footnote. If that is unneeded, please comment it out.
\usepackage{cite}

\usepackage{amsmath,amssymb,amsfonts}
\usepackage{algorithmic}
\usepackage{graphicx}
\usepackage{textcomp}
%\def\BibTeX{{\rm B\kern-.05em{\sc i\kern-.025em b}\kern-.08em
%		T\kern-.1667em\lower.7ex\hbox{E}\kern-.125emX}}

%\overrideIEEEmargins
%\IEEEoverridecommandlockouts                              % This command is only
%\usepackage{macros,macros2}


\usepackage[pdfa]{hyperref}
\hypersetup{colorlinks,breaklinks,
	linkcolor=linkcol,urlcolor=citecol,
	anchorcolor=linkcol,citecolor=citecol,bookmarks=true}
%----------------------------------
\newcounter{MYtempeqncnt}
\usepackage{caption}
\usepackage{subcaption}
%%%%%%%%%%%%%%%%%%%%%%%%%%%%%%%%
\newcommand{\thbf}{\boldsymbol\theta}
\DeclareMathOperator{\diag}{diag}
\DeclareMathOperator{\rank}{rank}
%%%%%%%%%%%%%%%%%%%%%%%%%%%%%%%%
%% %----------------------------
%% % THEOREMS
%% % ---------------------------------------------------------------
%% %%%%%%%%%%%%%%%%%%%%%%%
%\theoremstyle{definition}
\newtheorem{theorem}{Theorem}[section]
\newtheorem{lemma}[theorem]{Lemma}
%\newtheorem{corollary}[theorem]{Corollary}
\newtheorem{proposition}[theorem]{Proposition}
%\newtheorem{proof}[theorem]{Lemma}
%\newtheorem{defi}[theorem]{Definition}
\newtheorem{remark}[theorem]{Remark}
%%\newtheorem{assumption}[theorem]{Assumption}
%\newtheorem{exmple}[theorem]{Example}
%\newcommand{\tx}[1]{\text{#1}}
%%\newtheorem{procedure}[theorem]{Procedure}
%%%%%%%%%%%%%%%%%%%%%%%%%%%%%%%%%%%
%\newenvironment{example}{
%	\begin{exmple}\rm}{\hfill $\bigtriangleup$%
%	\end{exmple}
%}
%
%\newenvironment{remark}{
%	\begin{rmrk}\rm}{ \hfill{ }$\bigtriangleup$%
%	\end{rmrk}
%}
%
%
%\newenvironment{definition}{
%	\begin{defi}\rm}{\hfill $\bigtriangleup$%
%	\end{defi}
%}
%%%%%%%%%%%%%%%%%%%%%%%%%%%%%%%%%%%
%\renewcommand\thetheorem{\arabic{section}.\arabic{theorem}}
%
%\title{\Large \bf }
%==============================================
\newcommand{\rpos}{\mathbb{R}_{\geq 0}}
\newcommand{\ex}{\hat{e}_x}
\newcommand{\ez}{\hat{e}_z}
\newcommand{\ethr}{\hat{e}_{\theta_r}}
\newcommand{\ethrp}{\hat{e}^{\perp}_{\theta_r}}
\newcommand{\tx}[1]{\text{#1}}

\usepackage{graphicx}      % include this line if your document contains figures
\usepackage{cite}
%\usepackage{natbib}        % required for bibliography
%===============================================================================
\renewcommand\floatpagefraction{.9}
\renewcommand\dblfloatpagefraction{.9} % for two column documents
\renewcommand\topfraction{.9}
\renewcommand\dbltopfraction{.9} % for two column documents
\renewcommand\bottomfraction{.9}
\renewcommand\textfraction{.1}   
\setcounter{totalnumber}{50}
\setcounter{topnumber}{50}
\setcounter{bottomnumber}{50}
%================================================================================
\begin{document}
	
	\title{Generation of Time-Varying Impedance Attacks Against Haptic Shared Control Steering Systems}
		\author{Alireza Mohammadi\textsuperscript{$\dagger$}, and Hafiz Malik% <-this % stops a space
			\thanks{This work is supported by NSF Award 2035770. A. Mohammadi and H. Malik are with the
				Department of Electrical and Computer Engineering, University of Michigan-Dearborn, MI 48127 USA.   
				Emails: {\tt\small \{amohmmad,hafiz\}@umich.edu}. \textsuperscript{$\dagger$}Corresponding Author: A. Mohammadi.}%
		}
	
	\IEEEoverridecommandlockouts
	\makeatletter\def\@IEEEpubidpullup{6.5\baselineskip}\makeatother

\maketitle		
		
\begin{abstract}                % Abstract of not more than 250 words.
The safety-critical nature of vehicle steering is one of the main motivations for exploring the space of possible cyber-physical attacks against the steering systems of modern vehicles.  This paper investigates the adversarial capabilities for destabilizing the interaction dynamics between human drivers and vehicle haptic shared control (HSC) steering systems. In contrast to the conventional robotics literature, where the main objective is to render the human-automation interaction dynamics stable by ensuring passivity, this paper takes the exact opposite route. In particular, to investigate the damaging capabilities of a successful cyber-physical attack, this paper demonstrates that an attacker who targets the HSC steering system can destabilize the interaction dynamics between the human driver and the vehicle HSC steering system  through synthesis of time-varying impedance profiles. Specifically, it is shown that the adversary can utilize a properly designed non-passive and  time-varying adversarial impedance target dynamics, which are fed with a linear combination of the human driver and the steering column torques. Using these target dynamics, it is possible for the adversary to generate in real-time a reference angular command for the driver input device and the directional control steering assembly of the vehicle. Furthermore, it is shown that the adversary can make the steering wheel and the vehicle steering column angular positions to follow the reference command generated by the time-varying impedance target dynamics using proper adaptive control strategies. Numerical simulations demonstrate the effectiveness of such time-varying impedance attacks, which result in a non-passive and inherently unstable interaction between the driver and the HSC steering system.          
\end{abstract}