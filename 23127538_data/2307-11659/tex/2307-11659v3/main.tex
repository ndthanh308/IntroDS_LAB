\documentclass{amsart}
\usepackage{amssymb,amsmath}
\usepackage{comment}
\usepackage{graphicx}
\usepackage{subfigure}
\usepackage{ upgreek }
\usepackage{mathtools} 
\usepackage{bbm}
\usepackage{tikz}
\usepackage{pgfplots}
\usepackage{refcount}
%\usepackage[pdftex]{hyperref}
\usepackage[all]{xy}
\usetikzlibrary{matrix}
\usetikzlibrary{calc}

\newcommand{\bC}{{\mathbb C}}
\newcommand{\bP}{{\mathbb P}}
\newcommand{\bR}{{\mathbb R}}
\newcommand{\bZ}{{\mathbb Z}}
\newcommand{\bN}{{\mathbb N}}
\newcommand{\bQ}{{\mathbb Q}}
\newcommand{\bS}{{\mathbb S}}
\newcommand{\bF}{{\mathbb F}}
\newcommand{\bT}{{\mathbb T}}
\newcommand{\bK}{{\mathbb K}}
\newcommand{\bfk}{{\mathbf k}}
\newcommand{\bfq}{{\mathbf q}}

\newcommand{\cF}{\mathcal F}
\newcommand{\cG}{\mathcal G}
\newcommand{\cA}{\mathcal A}
\newcommand{\cB}{\mathcal B}
\newcommand{\cC}{\mathcal C}
\newcommand{\cE}{\mathcal E}
\newcommand{\cK}{\mathcal{K}}
\newcommand{\cY}{\mathcal Y}
\newcommand{\cS}{\mathcal S}
\newcommand{\cN}{\mathcal N}
\newcommand{\cM}{\mathcal M}
\newcommand{\cO}{\mathcal O}
\newcommand{\cP}{\mathcal P}
\newcommand{\cU}{\mathcal U}
\newcommand{\cL}{\mathcal L}
\newcommand{\cR}{\mathcal R}
\newcommand{\cT}{\mathcal T}
\newcommand{\cH}{\mathcal H}
\newcommand{\oD}{\overline{D}}
\newcommand{\cPM}{{\mathcal{PM}}}
\newcommand{\scrS}{\EuScript S}
\newcommand{\scrC}{\EuScript C}
\newcommand{\scrD}{\EuScript D}
\newcommand{\scrE}{\EuScript E}
\newcommand{\scrF}{\EuScript F}
\newcommand{\scrG}{\EuScript G}
\newcommand{\scrL}{\EuScript L}
\newcommand{\scrM}{\EuScript M}
\newcommand{\scrO}{\EuScript O}
\newcommand{\scrP}{\EuScript P}
\newcommand{\scrU}{\EuScript U}
\newcommand\isomto{\stackrel{\sim}{\smash{\longrightarrow}\rule{0pt}{0.4ex}}}
\DeclareMathOperator{\val}{val}
\DeclareMathOperator{\skel}{Skel}
\DeclareMathOperator{\id}{id}
\DeclareMathOperator{\im}{im}
\DeclareMathOperator{\hoSC}{hoSC}
\DeclareMathOperator{\whoSC}{\widehat{hoSC}}
\DeclareMathOperator{\SC}{SC}
\DeclareMathOperator{\wSC}{\widehat{SC}}
\DeclareMathOperator{\wtel}{\widehat{tel}}
\DeclareMathOperator{\Sp}{span}
\DeclareMathOperator{\tel}{tel}
\DeclareMathOperator{\Per}{Per}
\newtheorem{theorem}{Theorem}
\newtheorem{tm}{Theorem}[section]
\newtheorem{proposition}{Proposition}
\newtheorem{cy}[tm]{Corollary}
\newtheorem{lm}[tm]{Lemma}
\newtheorem{prp}[tm]{Proposition}
\newtheorem{pblm}[tm]{Problem}
\newtheorem{rem}[tm]{Remark}
\newtheorem{df}[tm]{Definition}
\newtheorem{ques}{Question}
\newtheorem{conj}{Conjecture}
\newtheorem{const}{Construction}
\newtheorem{whyp}{Working Hypothesis}
\newtheorem{ass}{Assumption}
\newtheorem{obs}{Observation}
\newtheorem{ex}[tm]{Example}
\begin{document}

\title{The local  Floer cohomology of indicator functions}
\author{Yoel Groman}
\address{ Yoel Groman,
Hebrew University of Jerusalem,
Mathematics Department\\
Email: yoel.groman@mail.huji.ac.il}
\begin{abstract}
For a compact set $K$ with contact type boundary in a symplectic manifold $M$ we construct a spectral sequence from the local Floer homology of the Reeb orbits, as studied by \cite{Mclean2012}, to the relative symplectic cohomology of $K$ in $M$ over the Novikov ring.  The spectral sequence is functorial with respect to inclusions which are not required to be exact. This functoriality is key to the closed string reconstruction problem near the singularity of an SYZ fibration. We illustrate this in the case of dimension $2n=4$ for symplectic cluster manifolds. In higher dimension, an additional ingredient, the locality spectral sequence, is required, and is the subject of a forthcoming work in progress. 

\end{abstract}
\maketitle

\tableofcontents
\section{Introduction}

Let $M$ be a symplectic manifold which is either closed or geometrically bounded and let $K\subset M$ be an arbitrary compact subset. Hamiltonian Floer theory associates to $K\subset M$  the cohomology group $SH^*_K(M)$, the \emph{symplectic cohomology of $K$ relative to $M$}.  This invariant plays a central role in many recent works.  An important motivation for the present study is its role in SYZ mirror symmetry: Mirror symmetry heuristics suggest that when $K$ is an invariant neighborhood of a fiber of a Maslov $0$ Lagrangian torus fibration, which may be singular, $SH^*_M(K)$ is the ring of analytic functions on the corresponding subset of the mirror. This idea has been pursued in \cite{GromanVarolgunes2022} for the case of symplectic cluster manifolds in dimension $2n=4$. The present work is part of a program to give a general construction of the mirror via relative $SH$.


As is typical in Floer theory, it is impossible to compute relative $SH$ from its definition. However, one can hope to make progress by analyzing its underlying chain complex $SC^*_M(K)$ as the deformation of something simpler.  The idea that we pursue in the present paper is that $SC^*_M(K)$ is the deformation of \emph{local Floer cohomology} \footnote{There is an unfortunate conflict of terminology, as what is here referred to as relative SH has sometimes been called, including by the author, local SH, whereas the term local Floer cohomology is used in the literature to what we call below the unweighted Floer cohomology of an orbit.}, a version of Floer cohomology which takes into account only Floer trajectories of infinitesimally small energy.  It turns out, as explained in \S\ref{SubsecSYZ}, that in settings of SYZ mirror symmetry this goes a long way.  

Though we are motivated by an application to SYZ mirror symmetry, we formulate all our results for compact sets with arbitrary contact boundary.  This of course contains a lot more than singularities of Lagrangian torus fibrations.  On the other hand,  the contact hypothesis is a bit restrictive for the case of Lagrangian torus fibrations. For example, the symplectic cluster manifolds studied in \cite{GromanVarolgunes2021} are not generally exact. The removal of the contact type hypothesis will be taken up in forthcoming work \cite{GromanToAppear2}. 
\subsection{Statement of the main results}
Let $K\subset M$ have contact boundary. Suppose with respect to some contact primitive of $\omega|_{\partial K}$  the set of Reeb  orbits decomposes, as a subset of the loop space with the continuous topology, into a collection of isolated path connected components so that
\begin{itemize}
\item any path of Reeb orbits has vanishing $\omega$ flux, and,
\item the Chern class evaluates to zero on any closed path of Reeb orbits.
\end{itemize}
Note that we do not assume these components are Morse-Bott, or even that they are smooth manifolds. Fix any ground ring $R$. To each Reeb component $\gamma$ of $\partial K$ we can associate an $R$-module,  $SH^*_{uw}(\gamma)$, the \emph{local Floer cohomology} of the Reeb component as studied by \cite{Mclean2012}.  A precise definition in the present context is given at the end of \S\ref{SubsecSHReeb}.  Before proceeding we point out some basic facts about $SH^*_{uw}(\gamma)$.
\begin{itemize}
\item The $R$-module $SH^*_{uw}(\gamma)$ depends only on the multiplicity of $\gamma$ and the germ near $\gamma$ of the pair $(M,\partial K)$ up to symplectomorphism.   In fact, it depends on the germ only up to isolating isotopy.  See Definition \ref{dfIsolatingIsotopy}.
\item If $\gamma$ is Morse-Bott, the group $SH^*_{uw}(\gamma)$ over $\bZ$ is, up to a shift of grading,  the integral singular homology of $\gamma$ twisted by a local system $\cL$ coming from Floer theory.  See Theorem \ref{tmMorseBottCascades}. 
\item When $\gamma$ is a Maslov $0$ Lagrangian torus and $\partial K$ is integral affine convex at $\gamma$, the local system $\cL$ is trivial by the work of \cite{BourgeoisOancea2009}. %\marginpar{Perhaps the connection between the local system of  Theorem \ref{tmMorseBottCascades}, the local system alluded to here, and the work of  \cite{BourgeoisOancea2009} should be made more explicit inside the paper.} 
\end{itemize}


Denote by $\cR$ the set of all Reeb components. Then we can associate to $K$ a graded BV algebra over the ground ring $R$ whose underlying $R$-module is
\begin{equation}
RH^*(K):=H^*(K;R)\oplus\bigoplus_{\gamma\in\cR(\partial K)}SH^*_{uw}(\gamma).
\end{equation}
We can think of the component $H^*(K;R)$ as the local Floer cohomology of the component of constant orbits inside $K$. 

We first give a formulation of our result  under the assumption that \emph{the period spectrum is gapped}.  This means that for any homology class $\alpha\in H_1(\partial K;\bZ)$ there is a $\delta>0$ so that the absolute value of the difference of any two distinct periods of orbits representing $\alpha$ is at least $\delta$. We later relax this assumption. 

The underlying complex $SC^*_M(K)$ of relative symplectic cohomology is defined over the universal Novikov field defined in equation \eqref{eqNovikovField}. As such it comes equipped with a $T$-adic filtration.  In particular, for any real number $\hbar$ we have a $\bZ$ filtration of $SC^*_M(K)$ defined by $F^pSC^*_M(K):=T^{p\hbar}SC^*_M(K)$. This gives rise to a spectral sequence $E_*^{*,*}=E_*^{*,*}(M,K)\Rightarrow SH^*_M(K) $ associated with the filtration. 


\begin{theorem}\label{mainThmA}
For $\hbar>0$  small enough, the page $E_1$ is naturally isomorphic to  
\begin{equation}
E_1^{p,q}=R^{p+q}(K)\otimes \Lambda_{[-(p+1)\hbar,-p\hbar)}.
\end{equation}
\end{theorem}

\begin{rem}\label{remOutsideGenerators}

Theorem \ref{mainThmA} gives rigorous meaning to the folklore interpretation of relative $SH$ that it is the homology of a complex generated by critical points inside $K$ and by the Reeb orbits of the boundary.  The contribution of $M$ is through the  connecting trajectories in $M$ which are not necessarily confined to $K$. A major difficulty in making this precise is that in general the relative $SH$ is computed by $S$-shaped Hamiltonians. These have  too many generators. Namely there are outside critical points and each Reeb component appears with at least two incarnations. In the  case where either $M$ is not exact, or  $K$ is not exactly embedded, it is far from clear how to filter out these extra generators. 
\end{rem}


To make the statement actually useful we need to discuss functoriality and naturality with respect to a class of inclusions. The assignment $K\mapsto E_*^{*,*}(M,K)$ is clearly contravariantly functorial with respect to all inclusions. For functoriality of $RH^*(K)$ we consider a more limited class. We say that an inclusion $K_1\subset K_2$ is \emph{admissible} if  
\begin{enumerate}
\item each Morse-Bott component of either $\partial K_1$ or $\partial K_2$ is either contained in $\partial K_1\cap\partial K_2$ or is disjoint of  $\partial K_1\cap\partial K_2$.  
\item
The restriction of $\omega$ to a neighborhood of $\partial K_1\cup \partial K_2$ is exact.
\end{enumerate}

\begin{rem}
Crucially, we \emph{do not} require the existence of a primitive on a neighborhood of $\partial K_1\cup\partial  K_2$ which is of contact type. See Figure \ref{FigAdmInc0}. This is essential for studying restriction maps for non-exact embeddings.  One of the main advantages of relative $SH$ over other approaches to symplectic cohomology is that it can handle non-exact embeddings. 
\end{rem}

% Figure environment removed

\begin{rem}
    The admissibility condition for inclusions is closely related to the barrier condition appearing in the work \cite{varolgunes} on the Mayer Vietoris property in relative $SH$. 
\end{rem}
We make the assignment $K\mapsto RH^*(K)$ contravariantly functorial with respect to admissible inclusions. Given an admissible inclusion $K_1\subset K_2$, the induced map $R^*(K_2)\to R^*(K_1)$ is described as follows. For the critical components it is the restriction map in ordinary homology. For a Reeb components it is the identity map on components contained in $\partial K_1\cap\partial K_2$ and the trivial map otherwise. 

\begin{theorem}\label{mainThmB}
For admissible inclusions there is a commutative diagram
\begin{equation}
\xymatrix{E_1^{p,q}(M,K_2)\ar[r]\ar[d]&RH^*(K_2)\otimes \Lambda_{[-(p+1)\hbar,-p\hbar)}\ar[d]\\
E_1^{p,q}(M,K_1)\ar[r]& RH^*(K_1)\otimes\Lambda_{[-(p+1),\hbar,-p\hbar)}}
\end{equation}
where the vertical arrow on the left is the restriction map in relative $SH$ and the map on the right is the functorial map in $RH$ on the first factor.
\end{theorem}




Our main application of these results is to the computation of relative $SH$ in the neighborhood of a singularity of an  SYZ fibration \emph{as a sheaf over the base. } For these applications, the gappedness requirement is too strict. It appears reasonable that at least in these settings, the gappedness requirement could be removed, but we do not know how to prove this. Rather we formulate a version Theorems  \ref{mainThmA} and \ref{mainThmB} which holds in the non-gapped setting. It appears the correct framework would be to consider spectral sequences whose pages are continuously indexed, but we have not found this notion explored in the literature. 

We define the infinitesimal symplectic cohomology of $K$ as 
\begin{equation}
SH^*_{M,t^+}(K):=\varprojlim_{\epsilon\to 0}SH^*_{M,[t,t+\epsilon)}(K). 
\end{equation}
We then have
\begin{theorem}\label{mainThmC}
There is a natural isomorphism $SH^*_{M,t^+}(K)\simeq RH^*(K)$. This isomorphism commutes with  restriction maps for admissible inclusions. 
\end{theorem}



An important notion which comes out of this discussion is the \emph{spectral symbol} $\sigma:SH^*_M(K)\to RH^*(K)$ defined when the symplectic cohomology $SH^*_{M,0}(K)$ over the Novikov ring is torsion free, and which associates to an element its leading term. For precise definitions see \S\ref{SecSpectral}




We also highlight the commutative diagram
\begin{equation}
\xymatrix{H^*(K_2)\ar[d]\ar[r]&H^*(K_1)\ar[d]\\
SH^*_{M,[0,\delta)}(K_2)\ar[r]& SH_{M,[0,\delta)}^*(K_1)}
\end{equation}
for admissible inclusions $K_1\subset K_2$ and $\delta>0$ small enough. This can be seen as a weak version of Viterbo functoriality. Note that for general ambient manifold $M$, we generally don't have a natural map $H^*(K)\to SH^*_M(K)$. 

\subsection{SYZ mirror symmetry near a singularity}\label{SubsecSYZ}
We apply the results above to the study of mirror symmetry in a neighborhood of a singularity of an SYZ fibration. To be precise, let $B$ be a topological manifold and let $\pi:M\to B$ be a Maslov $0$ Lagrangian torus fibration with singularities. In the broadest sense this means that 
\begin{enumerate}
\item for any pair of functions $f_1,f_2:B\to\bR$ so that the functions  $g_1:=f_1\circ \pi,g_2:=f_2\circ\pi $ are smooth we have that $\{g_1,g_2\}=0$.
\item  there is a subset $\Delta$ which deformation retracts to a codimension $2$ simplicial complex so that the fibers over $B\setminus \Delta$ are smooth Maslov $0$ Lagrangian tori.  
\end{enumerate}

\begin{pblm}\label{pblmNonAdHoc}
Construct in a non-ad hoc way a rigid analytic variety $M^{\vee}$ over the universal Novikov field together with a Non-Archimedean torus fibration $\pi^{\vee}:M^{\vee}\to B$, as defined in \cite{KoSo}, which is mirror to  $\pi:M\to B$ in the sense of, say, homological mirror symmetry. 
\end{pblm}
The framework of family Floer theory is sufficient for carrying out mirror symmetry over the complement of $\Delta$ \cite{Abouzaid2014b, Tu2014, Yuan2020}.  It runs into difficulties, however, when studying neighborhoods of singular fibres. 

To date, to the best of the author's knowledge, there is no satisfactory answer to Problem \ref{pblmNonAdHoc}. The work of  \cite{pomerleano} indicates that a good starting point for a general answer is the \emph{closed string mirror} pieced together from relative $SH$ of invariant sets \cite{GromanVarolgunes2022}. Note that relative $SH$ is an invariant of domains in $M$, so singular fibers pose no particular foundational difficulties. An additional appealing feature is the expected relation of higher $SH$ to polyvector fields \cite{Ganatra2013}. This produces a path to classical closed string mirror symmetry which does not go through homological mirror symmetry. These ideas will be expanded upon elsewhere.

We thus pose  the \emph{closed string reconstruction problem} which can be stated as follows. 
\begin{pblm}[\textbf{The Closed string reconstruction problem}]\label{pblmReconstruction}
Construct a rigid analytic variety $M^{\vee}$ over the universal Novikov field together with a Non-Archimedean torus fibration $\pi^{\vee}:M^{\vee}_{cs}\to B$ which is determined canonically up to isomorphism by the requirement that for any polytope $P\subset B$ we have $SH^0_M(\pi^{-1}(P))$ is the ring of analytic functions on $(\pi^{\vee})^{-1}(P)$ and satisfies that  $SH^*_M(\pi^{-1}(P))$ is the analytic of polyvector fields on $(\pi^{\vee})^{-1}(P)$.
\end{pblm}
A concise reference for basic notions in rigid analytic geometry that we refer to here and below is \cite{Conrad2008}.

This leads to
\begin{pblm}\label{pblmAffinoid}
Fix $P\subset B$. When is $SH^0(\pi^{-1}(P))$ an affinoid algebra of dimension the same as $B$? When is $SH^*(\pi^{-1}(P))$ the ring of polyderivations? Fix $P_1\subset P_2$. When is the restriction map from $P_2\to P_1$ dual to an open inclusion?   
\end{pblm}

In the present paper we address this in dimension $n=2$.  We lay foundations necessary for addressing this in dimension $n>2$.  An additional necessary ingredient, the locality spectral sequence,  is the subject of a forthcoming work \cite{GromanToAppear2}.   We briefly discuss this below for the positive singularity. 


\subsubsection{The case $n=2$}




For $P\subset B$ denote $\cF^*(P):=SH^*_M(\pi^{-1}(P))$.  We abbreviate $\cF(P)=\cF^0(P)$. 

The Arnold-Liouville integral affine structure on $B_{reg}:=B\setminus \Delta$ allows us to define the notion of convex polygons. These are polygons whose boundary is in $B_{reg}$ and so that this boundary is convex.  We shall assume henceforth that $P$ is rational, convex and \emph{Delzant smooth}.  See \S\ref{Sec4DimSYZ} for precise definitions. For the remainder of the discussion we add the assumption that that \emph{the periods associated with the set of edges are rationally independent \footnote{We expect this restriction to be easily liftable. See Remark \ref{remRestrictLift}.}.}   We call such $P$ \emph{admissible}. 

To a rational convex polygon we associate a partial monoid $P_{trop}(\bZ)$.  As a set it is the set of integral points in the dual fan of $P$\footnote{The dual fan in this case lives not in $\bR^2$ but in some integral affine manifold with singularities which is dual  in an appropriate sense to $P$.}. The partial monoid structure is given by standard addition within cones of the fan.  




Let $P_1\subset P_2$ be admissible. We say that the inclusion is \emph{admissible} if $\partial P_1\cap\partial P_2$ is a   codimension $1$ subset whose boundary points are interior points of the $1$ dimensional strata of $\partial P_2$. See Figure \ref{FigAdm}. It is easy to see this property is  transitive. Note that $\partial P_1\cap\partial P_2$  is a union of  edges of $\partial P_1$. 

Given an admissible inclusion $Q\subset P$ there is an induced (partial) morphism of partial monoids $P_{trop}(\bZ)\to Q_{trop}(\bZ)$.  It is induced in the obvious way by mapping dual generators of common edges to themselves. 

\begin{rem}
It is useful to think the partial monoid $P_{trop}(\bZ)$ as the monomial basis of the underlying abelian group of the Stanley-Reisner ring associated with $\partial P$. That is, the quotient of the polymonial algebra generated by the edges by the ideal generated by products of non-adjacent edges. Note that a product of basis elements is either a basis element or $0$ in which case we consider it undefined. 
\end{rem}

To each element $x\in P_{trop}(\bZ)$ associate an $R$-module $M_x^*$ as follows. For the $0$ element take $M_0:=H^*(P;R)$. For $x=me_i+ne_{i+1}$ let $T_x$ be the $(m,n)$-cover of the torus formed by taking the quotient $T^*_vB$ by the dual to the lattice generated by the primitive tangents to $e_1,e_2$. Let $M^*_x:=H^*(T_x;\bZ)$.

\begin{theorem}\label{tmInfSHn2}
Let $P\subset B$ be admissible and let $K=\pi^{-1}(P)$. Then 
\begin{enumerate}
\item
 The  infinitesimal Floer cohomology of $K$ is the direct sum 
 \begin{equation}
 SH^*_{M,t^{+}}(K)=M(P_{trop}(\bZ)):= \bigoplus_{x\in P_{trop}(\bZ)}M_x^*
 \end{equation}
 in each $\bR$-degree. 
\item
If  $Q\subset P$ is an admissible inclusion then the restriction map in infinitesimal Floer cohomology is induced by the map of partial monoids $P_{trop}(\bZ)\to Q_{trop}(\bZ)$.
\end{enumerate}
\end{theorem}
\begin{rem}
The domains considered in Theorem \ref{tmInfSHn2} have corners rather than boundaries. To deduce it from  Theorem \ref{mainThmC} we use  convex smoothing. See Section \S\ref{SecNSCase}. This approach  forces us to impose the irrationality assumption on the periods. This might pose a problem if we wish to apply the reasoning to analyze restriction maps in connection with isotopies of domains. We expect a more careful approach developing a version of Theorem \ref{mainThmC} for domains with corners would alleviate this restriction.
\end{rem}
\begin{rem}
The degree $0$ part $SH^0_{M,t^{+}}(K)$ can be alternatively described as the underlying module of the Stanley Reisner ring associated with  the intersection complex of $P$. Setting $t=0$, it is easy to see that is true also as a statement about algebra structure.  We do not pursue this in the present work. 
\end{rem}
Armed with the computation of infinitesimal $SH$, we proceed to address Problem \ref{pblmAffinoid}. We first add an assumption on $\pi$ which should be viewed as the closed string version of FOOO unobstructedness.  To state it, consider $P\subset B$ which contains no singular point. We can associate with $P$ a $2$-dimensional vector space $C_P$ be taking the tangent space to any point in the interior of $P$.  Then $P_{trop}(\bZ)$ is the integral lattice in $C^*_P$. 
\begin{df}
We say that the regular fibers of a Lagrangian torus fibration $\pi$ are \emph{undeformed} if for any convex $P\subset  B_{reg}$,  writing $K=\pi^{-1}(P)$, there is  an isomorphism of BV algebras $SH^*_M(K)\simeq SH_{\overline{K}}(K)$ where $\overline{K}\simeq T^*T^n$ is the completion of $K$ by attaching a cone.  Moreover, we assume this isomorphism respects spectral symbols. 
\end{df}
To justify this definition we spell out the settings where it is known or expected to be satisfied.
\begin{enumerate}
 \item One class of examples comes from complete embeddings considered in \cite{GromanVarolgunes2021}.  Namely,  if $M$ is a symplectic cluster manifold then the regular fibers of $\pi$ are undeformed. This follows immediately from the locality theorem of \cite{GromanVarolgunes2021}.    
\item In forthcoming work we prove that the regular fibers are undeformed if and only if $SH^*_M(K)$ considered over the Novikov ring is torsion free.  The latter assumption is verified in the same way as FOOO unobstructedness of the regular fibers is verified. Namely, either by flux considerations as in \cite{ShelukhinEtAl} or by an anti-symplectic involution as in \cite{Solomon}.  
\end{enumerate}

We now restrict attention again to the case $n=2$. 
\begin{theorem}\label{tmLocSHn2}
Suppose the regular fibers of $\pi$ are undeformed.   Suppose the inclusion of a regular fiber into $\pi^{-1}(P)$ induces an injection of $H^*(\pi^{-1}(P);\bZ)$ into $H^*(\bT^2;\bZ)$.  Then
\begin{enumerate}
\item \label{tmLocSHn2:pt1} There exists an isomorphism $\cF^*\simeq M^*(P_{trop}(\bZ))\otimes\Lambda$ which preserves norms and leading symbols. A corresponding isomorphism holds when considering truncation windows $[a,b)$. See \S\ref{SecConventions} for definitions. 
\item \label{tmLocSHn2:pt2} For $i$ indexing the edges of $\partial P$,  let $z_i\in \cF_P$ be an element corresponding to the generator associated with the $i$th edge under an isomorphism as in the previous part.   Then $\cF_P$ is generated as a Banach algebra by the collection $z_i$. In particular, $\cF_P$ is affinoid. 
\item For an admissible inclusion $Q\subset P$  the restriction map $\cF_P\to\cF_Q$ is the inclusion of a Laurent domain whenever $Q$ is contained in a small enough neighborhood of the boundary. 
\item There is a natural injection of $\cF^*(P)$ into the ring of polyvector fields on $\cF^0(P)$. 
\end{enumerate}
\end{theorem}

\begin{rem}
    Part  \ref{tmLocSHn2:pt1}  of Theorem \ref{tmLocSHn2} is equivalent to the assertion that relative $SH$ for convex Delzant polygons is torsion free over the Novikov ring.  As the proof of the remaining parts of Theorem \ref{tmLocSHn2} indicates, \emph{establishing torsion freedom is the key to the closed string reconstruction in general.} The main contribution of the present paper is a technique for deducing torsion freedom for neighborhoods of the singular fiber from the torsion freedom of the regular fibers. Below we discuss how the matter of applying this in dimension $2n>4$. 
\end{rem}

\begin{rem}
Theorem \ref{tmLocSHn2} implies in particular that $\cF^0(P)$ is an integral domain and that $
\cF^*(P)$ satisfies uniqueness of analytic continuation with respect to restriction to admissible polygons. Further properties, such as normality and Gorenstein will be discussed in forthcoming work. 
\end{rem}
\begin{rem}
Parts \ref{tmLocSHn2:pt1} and \ref{tmLocSHn2:pt2} of Theorem \ref{tmLocSHn2} can be extracted in the case of exact symplectic cluster manifolds from the work of \cite{Pascaleff}.
\end{rem}

\begin{theorem}\label{tmLocSHn3}
Suppose $\cF^0(P)$ is a smooth algebra and that $\pi$ admits a section over $P$. Then
\begin{enumerate}
\item  the dual cohomology class of the section maps to a generator $\sigma$ of $\cF^2$ over $\cF_0$ corresponding to a nowhere vanishing bi vector field. 
\item
The injection of  $\cF^*(P)$  into the polyvector fields over $\cF^0(P)$ is an isomorphism
\item 
The BV operator on $\cF^*(P)$  is the divergence operator associated with the isomorphism $\cF^0\simeq\cF^2$ induced by the bivector field $\sigma$. 
\end{enumerate}
\end{theorem}

The question of smoothness for symplectic cluster manifolds is discussed in forthcoming work. We point out that that this holds for neighborhoods of the nodal singularity in a symplectic cluster manifold. A simple example where $\cF^0(P)$ is not smooth occurs when $P$ contains a node with multiplicity greater than $1$. Both assertions are established in  \cite{GromanVarolgunes2022}. In forthcoming work it will be shown that this the only way in which non-smoothness occurs for symplectic cluster manifolds. 





\subsubsection{The case $n>2$}
The analogues of Theorem \ref{tmInfSHn2}  also holds in higher dimensions for an appropriately defined notion of convex polytope. A detailed discussion is taken up in forthcoming work. The main new phenomenon is that singularities meet the boundary. See Figure \ref{figPrism}.  From the technical standpoint, the results of the present paper are largely sufficient for establishing the analogue  of Theorem \ref{tmInfSHn2} in all dimensions.  In the body of the paper we only discuss the case $n=2$ in detail.  We sketch the higher dimensional version of Theorem  \ref{tmInfSHn2} for one representative three dimensional example in section \ref{SecPosSing}.

Analogues of Theorem \ref{tmLocSHn2} and Theorem \ref{tmLocSHn3} also hold mutatis mutandis in all dimensions.  However,  to prove the torsion freedom we need to show that undeformedness of the regular fibers implies torsion freedom of the faces.  This done in \cite{GromanToAppear} by an inductive argument using the locality spectral sequence. 



\subsection{Local and infinitesimal Floer homology of indicator functions}
We now turn to explain the technical ingredients going into the proof of the main Theorems. 

The invariant $SH^*_M(K)$ is the Floer cohomology of the indicator function 
\begin{equation}\label{eqHK}
H_K(x):=\begin{cases}
0&\quad x\in K,\\
\infty&\quad x\not\in K.\notag
\end{cases}
\end{equation}
The underlying Floer complex is defined up to homotopy equivalence as a colimit 
$$
CF^*(H_K)=hoColim_{i}CF^*(H_i,J_i).
$$
over any monotone sequence of non-degenerate Hamiltonians converging to $H_K$ on compact sets. The main Theorems should be seen as extending to the case of generalized Hamiltonians the truism that the Floer complex of a smooth non-degenerate Hamiltonian is generated by its 1-periodic orbits. They do this by relating Floer cohomology of $H_K$ in small action windows with local Floer cohomology groups $SH^*_{uw}(\gamma)$ associated with the Reeb orbits.

To explain what goes in to the proof of this relation we first discuss the related case of smooth Hamiltonians but possibly degenerate Hamiltonians. The Floer complex $CF^*(H)$ is again defined up to homotopy equivalence as a colimit over a monotone sequence of non-degenerate Hamiltonians $H_i$ approximating $H$ from below
$$
CF^*(H)=hoColim_{i}CF^*(H_i,J_i).
$$
If the 1-periodic orbits of $H$ are isolated, a slight strengthening of a result by \cite{Hein}, referred to henceforth as Hein's Lemma, shows that if all the $J_i$ are close to some given almost complex structure then for $\epsilon$ small enough the truncated complex $CF^*_{[a,a+\epsilon)}(H)$\footnote{We caution the reader that \cite{Hein} considers the Floer complex generated by capped orbits and truncates with respect to the action filtration, whereas we consider Floer complexes generated over the Novikov ring by (uncapped, non-contractible even,) periodic orbits and consider truncations with respect to the adic filtration.} decomposes into direct summands $\oplus CF^*_{[a,a+\epsilon)}(H;U_{\gamma})$ consisting  of periodic orbits and Floer solutions that are contained in an isolating neighborhood $U_{\gamma}$ of the component $\gamma$. 

On the other hand, one considers local Floer cohomology \cite{Pozniak, Ginzburg} which associates a chain complex over the ground ring $R$  to each component by slightly perturbing $H$ and considering only local Floer trajectories. It can be defined using the same Floer data as the one defining $CF^*(H)$ provided all the $H_i$ are sufficiently $C^2$ close to $H$. We use the notation $CF^*_{uw}(\gamma)$ for the corresponding complex. The subscript indicates that this complex is unweighted by contrast to  $CF^*_{[a,a+\epsilon)}(H;U_{\gamma})$ which is defined over the $\epsilon$ truncated Novikov ring. The relation between these two versions is given in Lemma \ref{lmIsotropicComp} which produces a natural isomorphism
\begin{equation}\label{eqLocInf}
CF^*_{[a,a+\epsilon)}(H;U_{\gamma})=CF^*_{uw}(\gamma)\otimes \Lambda_{[a, a+\epsilon)}.
\end{equation}
This immediately leads to a spectral sequence similar to the one in Theorem \ref{mainThmA} from the sum of the unweighted local Floer cohomologies of the periodic components of $H$ to the $HF^*(H)$ over the Novikov ring. Moreover, given an isolated homotopy, this allows one to understand the leading term of the continuation map. Namely, an isotopy from $\gamma^0$ to $\gamma^1$ induces a homotopy equivalence from $CF^*_{uw}(\gamma^0)\to CF^*_{uw}(\gamma^1)$. With this identification, the map  $CF^*_{[a,a+\epsilon)}(H;U_{\gamma^0})\to CF^*_{[a,a+\epsilon)}(H;U_{\gamma^1})$, defined for monotone homotopies, is given by rescaling by appropriate action differences.

We would like to imitate this type of reasoning to the case of the function $H_K$. Before proceeding, we wish to clarify that in this case the discussion is non-trivial even of all the Reeb components are transversally non-degenerate. 

 It turns out that it is advantageous to first do this for non-smooth  but continuous Hamiltonians $H_{K,I}$. $I$ here denotes a choice of Liouville coordinate and slope, and $H_{K,I}$ is a strictly $S$ shaped Hamiltonian. See figure \ref{figSShaped} and Definition \ref{dfSShaped}.
 We show below that if $I$ is such that the slope is not in the period spectrum of $\partial K$ then an equation similar to \eqref{eqLocInf} holds for the 1-periodic components of $H_{K,I}$ for $\epsilon=\epsilon(I)$ small enough depending on $I$. The proof of this claim involves showing that Hein's lemma, in its strengthened version Lemma \ref{lmEpsilonSeparation}, applies for a monotone sequence of smooth Hamiltonians approximating  $H_{K,I}$ from below. We point out that due to the non-smoothness of  $H_{K,I}$ this claim is non-trivial. Indeed, we don't have uniform estimates for the derivatives of such a sequence in any neighborhood of the periodic components we care about.  

Note that the set of $1$-periodic orbits of $H_{K,I}$ consists of inner orbits associated with the  bottom of the graph of $H_{K,I}$ and outer orbits associated with the top. Thus for small action windows we can write 
\begin{equation}
CF^*_{[a,a+\epsilon(I))}(H_{K,I})=In^*_{[a,a+\epsilon(I))}(H_{K,I})\oplus Out_{[a,a+\epsilon(I))}(H_{K,I}),
\end{equation}
where we sum over inner and outer generators respectively. Unfortunately, this splitting is only valid in action windows of size $\epsilon(I)$ and we have no control over $\epsilon(I)$. Crucially, and this is where we invoke the fact that $\partial K$ is of contact type, we prove there is an $\hbar>0$, independent of $I$, so that we have an exact sequence 
\begin{equation}
0\to  Out_{[a,a+\hbar)}(H_{K,I})\to CF^*_{[a,a+\hbar)}(H_{K,I})\to In^*_{[a,a+\hbar)}(H_{K,I})\to 0. 
\end{equation} 
More precisely, we show that in small action windows, Floer trajectories connecting Reeb components localize to a symplectization neighborhood in which case this property is established in \cite{CieliebakOancea}. We emphasize again that obtaining a uniform $\hbar$ giving this is a non-trivial matter as  $H_{K,I}$ satisfies no uniform estimate in $I$ and is also highly degenerate. A crucial consequence of this locality is  that the sub-complex $Out_{[a,a+\delta)}(H_{K,I})$ maps to $0$ under the continuation map to  $CF^*_{[a,a+\delta)}(H_{K,I'})$ for $I'$ big enough.  

To reach a conclusion concerning $H_K$ we first observe that for any truncation window we have 
\begin{equation}
CF_{[a,a+\hbar)}^*(H_K)=hoColim_ICF^*_{[a,a+\hbar)}(H_{K,I})=hoColim_IIn^*_{[a,a+\hbar)}(H_{K,I}).
\end{equation}
To utilize this, let us first make the gappedness assumption, and WLOG let the gap $\delta$ be less than $\hbar$. The proof of Theorem \ref{mainThmA} follows by establishing a commutative diagram of isomorphisms. 
\begin{equation}
\xymatrix{In^*_{[0,\delta)}(\gamma;H_{K,I})\ar[d]\ar[r]& CF^*_{uw}(H_K,\gamma)\otimes \Lambda_{[0,\delta)}\ar[d]\\
In^*_{[0,\delta)}(\gamma;H_{K,I'})\ar[r]& CF^*_{uw}(H_K,\gamma)\otimes \Lambda_{[0,\delta)}}.
\end{equation}
We now remove the gappedness assumption. Even in this case, the localization property gives us  an action filtration of $CF^*_{[a,a+\hbar)}(H_{K,I})$ which is respected by the continuation maps $I\to I'$. Below each action level, we have a finite number of generators, and therefore a gap which is independent of $I$. From this and some general nonsense about commuting limits and colimits we obtain Theorem \ref{mainThmC}. 







\subsubsection{Removing the exactness assumption?}
It is clear from the above outline that the contact type assumption plays a crucial role in the proof. To be more precise, the argument extending Hein's lemma to Hamiltonians such as $H_{K,I}$ is completely general and is applicable to stable Hamiltonian hyper-surfaces. However, the argument for localization to a symplectization neighborhood in a uniform window uses the action filtration in an essential way.  In forthcoming work we produce a homological perturbation type algorithm which constructs a chan level model for $CF^*(H_K)$ whose underlying module is a direct sum of one Morse-Bott complex for each Reeb component. For this algorithm the exactness assumption is not crucial, and we expect it can be used to establish a version of Theorem \ref{mainThmC} whenever $\partial K$ is stable Hamiltonian. This concern is not just about mindless generality. Even if we limit ourselves to the setting of SYZ mirror symmetry, imposing a contact type assumption on the boundary is too restrictive. For example, in a forthcoming work we show that for symplectic cluster manifolds a smoothing of  the singularities of $mSpec(SH)$ is achieved by deforming the symplectic form so that we no longer have exactness near the boundary. 


We part with a fundamental question raised by the above discussion. Let $M$ be a geometrically bounded symplectic manifold and let $H:M\to\bR$ be a proper Hamiltonian which is bounded from below.  For concreteness, consider $M$ the twisted cotangent bundle of a smooth compact manifold and $H$ a mechanical Hamiltonian. We don't necessarily know that $H$ satisfies $C^0$ estimates for the direct definition of Hamiltonian Floer cohomology. However, it is shown in \cite{Groman} that one can define it using a sequence of Lipschitz Hamiltonians which converge to $H$ on compact sets.  Moreover, the definition is independent of the choice of approximating sequence.
\begin{pblm}
 Does the analogue of Theorem \ref{mainThmA} hold? Namely,  suppose $H$ has isolated periodic components,  is $HF^*_{0^+}(H)$ the direct sum of components associated with the periodic components of $H$. 
 \end{pblm}
 Note in the case of the magnetic cotangent bundle, the approximating scheme involves drastically slowing down the Hamiltonian which introduces many periodic orbit not coming from $H$. The problem is thus akin to the problem mentioned in Remark \ref{remOutsideGenerators}.
%\subsection{Outline}
%In Section \S\ref{SecConventions} we set our conventions for Hamiltonian Floer theory.  In Section \S\ref{SecInfFloer} we develop the relation \eqref{eqLocInf} for smooth degenerate Hamiltonians. In Section \S\ref{secInfFH} we study unweighted local Floer cohomology and Hein's lemma for non-smooth Hamiltonians of the form $H_{K,I}$. In Section \S\ref{SecTruncSH} we develop a relation between truncated and unweighted Floer cohomology for the generalized function $H_K$.In Section \S\ref{SecProofs} we prove Theorems \ref{mainThmA}, \ref{mainThmB}, \ref{mainThmC}, and a version for domains with corners. In Section \S\ref{SecSYZ} we apply the tools developed so far to study 

\subsection{Acknowledgements}
The author would like to thank Mohammed Abouzaid for helpful comments on Morse Bott Floer theory and Umut Varolgunes for useful comments on an early draft and for long term collaboration on the topic of relative $SH$ and SYZ mirror symmetry.  

The work was supported by the ISF (grant no. 2445/20).
 
\section{Conventions for Floer cohomology}\label{SecConventions}
Fix a ground field $R$ for the remainder of the discussion. 

We denote by $\Lambda$ the Novikov field 
\begin{equation}\label{eqNovikovField}
\Lambda:=\left\{\sum a_iT^{\lambda_i}|\lambda_i\in\mathbb{R},a_i\in R,\lim_{i\to\infty}\lambda_i=\infty\right\}.
\end{equation}

We consider $\Lambda$ as a normed vector space (over itself) with norm given by $|x|=e^{-\val}(x)$ where for $x=\sum a_iT^{\lambda_i}$ we take $\val(x)=\inf\{\lambda_i|a_i\neq 0\}$. 
Denote by $\Lambda_a$ the lattice consisting of elements of norm $<e^{a}$. Equivalently,
\[
\Lambda_{a}:=\left\{\sum a_iT^{\lambda_i}|\lambda_i>-b,a_i\in R,\lim_{i\to\infty}\lambda_i=\infty\right\}.
\]
For an arbitrary interval $[a,b)$ we consider the module $\Lambda_{[a,b)}:=\Lambda_b/\Lambda_a$.

We assume the reader is familiar with the basic definitions of Floer cohomology. The purpose of the remainder of the sections is to set conventions. 

Let $H:\bR/\bZ\times M\to\bR$ be a smooth function and let $J$ be an $\omega$ compatible periodically time dependent almost complex structure. We assume that $(H,J)$ is regular for the definition of Floer cohomology. We write $X_H$ for the unique vector field satisfying $\omega(X_H,\cdot)=dH$. Denote by $\Per(H)$ the set of periodic orbits of $H$.  Given a pair of elements $\gamma_1,\gamma_2\in\Per(H)$ a \emph{Floer trajectory} from $\gamma_1$ to $\gamma_2$ is a solution $u$ to Floer's equation
\begin{equation}
\partial_su+J(\partial_t-X_H)=0,
\end{equation}
such that $\lim_{s\to-\infty}u(s,t)=\gamma_1(t)$ and $\lim_{s\to\infty}u(s,t)=\gamma_2(t)$. We define the \emph{topological energy} of $u$ by the formula
\begin{equation}
E_{top}(u):=\int u^*\omega+\int_{\bR/\bZ}\left(H_t(\gamma_2(t))-H_t(\gamma_1(t))\right)dt.
\end{equation}
It is a fact that
\begin{equation}\label{eqGeoTopEn1}
E_{top}(u)=\int\|\partial_su\|^2ds\geq 0.
\end{equation}
The quantity on the right hand side is referred to as the \emph{geometric energy} $E_{geo}(u)$. 

Denote by $$\overline{\cM}_1(\gamma_1,\gamma_2,E)\subset \overline{\cM}(\gamma_1,\gamma_2,E)$$ the subset consisting of solutions of index difference $1$. 



We  proceed to define the Floer complex  $CF^*(H,J)$. We take the set of periodic orbits $\Per(H)$ to be  graded by a group $R$ with a map to $\bZ/2\bZ$. The reader who wish to do so may just take $R=\bZ/2\bZ$. For simplicity assume first that $\Per(H)$ is a finite set. Then as an $R$-graded $\Lambda$-module 
\begin{equation}\label{eqFloerComplex}
CF^*(H,J)=\bigoplus_{k\in R} CF^k(H,J),
\end{equation}
where 
\begin{equation}
CF^k(H,J):=\oplus_{\gamma\in Per^k(H)}\Lambda \langle\gamma\rangle
\end{equation}
and $Per^k(H)$ is the set of $1$-periodic orbits of $H$ with degree $k$. We consider $CF^*(H,J)$ as a finite dimension non-archimedean normed space.


For each periodic orbit $\gamma$ we denote by
$o_{\gamma}$ is the orientation line associated with $\gamma$. Each $u\in \cM(\gamma_1,\gamma_2)$ for $\gamma_1,\gamma_2$ with index difference $1$ induces an isomorphism
\[
d_u:o_{\gamma_1}\to o_{\gamma_2}.
\]
The differential is defined by
\begin{equation}
d|_{\Lambda_{\geq 0}\langle\gamma_1\rangle}=\sum_{\gamma_2:i_{CZ}(\gamma_2)-i_{CZ}(\gamma_1)=1}\sum_{u\in \cM(\gamma_1,\gamma_2)}T^{E_{top}(u)}d_u.
\end{equation}

For a pair $F_1=(H_1,J_1), F_2=(H_2,J_2)$, a \emph{monotone homotopy from $F_1$ to $F_2$} is a family $(H^s,J^s)$ which coincides with $(H_1,J_1)$ for $s\ll0$, with $(H_2,J_2)$ for $s\gg0$, and satisfies
\begin{equation}
\partial_sH^s\geq 0.
\end{equation}
Evidently, a monotone homotopy exists if only if $H_{1,t}(x)\leq H_{2,t}(x)$ for all $t\in S^1$ and $x\in M$. Solutions to the Floer equation corresponding to a monotone datum satisfy the variant of estimate \eqref{eqGeoTopEn1}
\begin{equation}\label{eqGeoTopEn2}
E_{top}(u)\geq\int\|\partial_su\|^2ds\geq 0.
\end{equation}
There is an induced chain map
\begin{equation}\label{eqContinuationMap}
f_{H^s,J^s}:CF^*(H_1,J_1)\to CF^*(H_2,J_2).
\end{equation}
These are again defined by counting appropriate Floer solutions weighted by their topological energy. Moreover, these maps are defined over the Novikov ring.

\subsection{The Floer complex for general Hamiltonians}
The naive definition of the Floer complex requires considering smooth non-degenerate Hamiltonians, and a choice of $J$ (and, in the fully general case, some scheme for virtual counts). The definition extends to more general Hamiltonians $H$ by considering a monotone sequence $(H_i,J_i)$ of generic Floer data such that $H_i$ converges pointwise to $H$ and defining 
$$
CF^*(H)=\widehat{hoColim}_iCF^*(H_i,J_i).
$$
Here  $hoColim$ denotes some unspecified model for the homotopy colimit, a popular choice being the telescope construction. In Appendix \ref{SecAcceleration} we recall the construction for indicator functions. The widehat denotes completion with respect to the adic norm. The complex $CF^*(H)$ is well defined up to contractible choice. 

The Hamiltonian Floer cohomology $HF^*(H)$ is defined as the homology of $CF^*(H)$. This is a vector space over the Novikov field. Note that $CF^*(H)$, being freely generated over $\Lambda$, is endowed with a natural norm by assigning to the generators the norm $1$. The differential respects this norm and we can thus define subcomplexes $CF^*_a(H)\subset CF^*(H)$ consisting of elements of norm $<e^a$. For a half interval $[a,b)$ the truncated Floer cohomology $HF^*_{[a,b)}(H)$ to be the homology of the sub-quotient $CF^*_b(H)/CF^*_a(H)$. It is shown in \cite{Groman}
that for any proper Hamiltonian $H$ we have  
\begin{equation}\label{eqTruncHom}
HF^*_{[a,b)}(H):=\varinjlim_{H'}HF^*_{[a,b)}(H'), 
\end{equation}
where the colimit is taken over all dissipative non-degenerate Hamiltonians. 

\begin{rem}
There is a slight discrepancy in the notation $HF^*_{[a,b)}(H)$ in comparison to \cite{Groman}. Here the filtration is with respect to the norm whereas there the filtration is with respect to the action filtration which is a monotone function of the norm. 
\end{rem}

\begin{df}
For a compact set $K\subset M$ the \emph{relative symplectic cohomology} of $K\subset M$ is the Hamiltionian Floer cohomology of the indicator function
\begin{equation}
    SH^*_M(K)=HF^*(H_K),
\end{equation}
where 
\begin{equation}
H_K(x):=\begin{cases}
0&\quad x\in K,\\
\infty&\quad x\not\in K.
\end{cases}
\end{equation}
Similarly, $H^*_{M,[a,b)}(K)$ denotes the corresponding truncated homology. 
\end{df}

%Within the text below we shall have occasion to refer to the Floer complex $CF^*(H)$ for $H$ which is not smooth, or even, as in the case of indicator functions, discontinuous.  

\section{Floer cohomology of $1$-periodic orbits of autonomous Hamiltonians.}\label{SecInfFloer}
As a warm up for the main results which concern Floer cohomology for generalized functions we discuss similar results for smooth Hamiltonians. Our result are a slight strengthening and a repackaging of results that have been established in the literature. Nevertheless we urge the reader to pay attention to Lemmas \ref{lmEpsilonSeparation} and \ref{lmIsotropicComp}, and, at least in the latter case, the proof thereof which will be re-used in the later sections.  
\subsection{Floer cohomology in small truncation windows}
We have seen that given an arbitrary proper Hamiltonian $H$, with no assumptions regarding non-degeneracy or dissipativity, we can define the truncated Hamiltonian Floer cohomology $HF^*_{[a,b)}(H)$ via equation \eqref{eqTruncHom} or the chain level version thereof. 


The following lemma expresses the sense in which $HF^*_{[a,b)}(H)$, for small enough intervals, is about the dynamics of $H$, not a sequence approximating it. For simplicity of presentation, assume in the following $M$ is closed. The adjustment for $M$ geometrically bounded is completely straightforward.

\begin{lm}\label{lmEpsilonSeparation}
Let $H$ be an autonomous Hamiltonian on $M$. Let $\gamma$ be an isolated family of $1$-periodic points of $H$. That is, there is an open neighborhood of $\gamma$ with smooth boundary containing no other $1$-periodic points besides those in $\gamma$. Fix a bounded ball $\cB$ in the space of $\omega$-compatible almost complex structures considered with $C^k$ topology for some large fixed $k$.   For any small enough isolating neighborhood $V$ of $\gamma$ there is an epsilon $>0$ and a neighborhood $\cN$ of $H$ in $C^1$ considered as a function on $S^1\times M$ such that the following hold.
\begin{enumerate}
\item For any $H'\in\cN$ and any $J\in \cB$ in the Floer complex $CF^*_{[a,a+\epsilon)}(H',J)$ splits as
\begin{equation}
CF^*_{[a,a+\epsilon)}(H',J)=CF^*_{[a,a+\epsilon)}(V;H',J)\oplus CF^*_{[a,a+\epsilon)}(M\setminus V;H',J) 
\end{equation}
corresponding to $1$-periodic points in $V$ and in $M\setminus V$ respectively. 
\item 
For any pair $H_1\leq H_2\in \cN$ and any pair $J_1,J_2\in\cB$ the continuation map $CF^*(H_1,J_1)\to CF^*(H_2,J_2)$ for any monotone path respects the splitting of the previous item. 
\end{enumerate}
Accordingly we denote by $HF^*_{[a,a+\epsilon)}(V;H,\cB)$ the corresponding component of $HF^*_{[a,a+\epsilon)}(H)$ defined as the colimit over a monotone sequence of non-degenerate Hamiltonians converging to $H_0$. 
\end{lm}

\begin{rem}
If $V$ is a sub-level set of $H$ this is an immediate consequence of a Lemma by \cite{Hein}.  However we wish to separate a small neighborhood of a family of $1$-periodic orbits, not a sublevel set.  The proof requires a slight adjustment of Hein's proof to estimate the energy of a path. 
\end{rem}


\begin{proof}
Fix a compatible almost complex structure $J\in \cB$. Consider the continuous function $f:M\to\bR_+$ given by $f(x)=d_J(x,\gamma)$ for $d_J$ the distance with respect to the metric induced by $J$ and $\gamma$ considered as a subset.  Fix real numbers $0<r_0<r_1<r_2<r_3$ with the following properties. Denote by $U_i:=f^{-1}([0,r_i))$.
\begin{enumerate}
\item $U_3\setminus U_0$ contains no $1$-periodic points of $H$.
\item The minimal flow-time of $X_H$ from $U_2$ to the boundary of $U_3$ and from $U_2\setminus U_1$ to the boundary of $U_0$ is at least $2$. 
\end{enumerate}
Such numbers exist by continuity of the flow and the fact that all points of $\gamma$ are periodic since $\gamma$ is compact and isolated.  Fix  a neighborhood $\cN$ of $H$ which is sufficiently small for the above to hold for all $H\in\cN$. 

Fix an $R>0$. Write $V=U_3$. Let $\delta_0$ be a bound from below on distance $d(x,\psi_1(x))$ of any $x\in U_3\setminus U_0$ to its image under the time $1$ flow of $H$. Let $\delta_1$ be a bound from below on the distance $d(\psi_t(x),\partial (U_3\setminus U_0))$ for $x\in U_2\setminus U_1$ and for $t\in[0,1]$. All bounds are assumed to hold for metrics determined by any almost complex structure in $\cB$. Fix $(H',J')\in \cN\times \cB$. Let $u$ be an $(H',J')$ Floer trajectory meeting both $V$ and $M\setminus V$ with input or output a periodic orbit contained in $U_0\subset V$. Then the pre-image $u^{-1}(U_2\setminus U_1)$ contains a connected component $S\subset\bR\times S^1$. Let $[a,b]$ be the image of $S$ under projection to the first factor. Then there is a constant $\delta_2$ so that for any $s\in [a,b]$ there are a $t_0,t_1\in[0,1]$ such that 
$$d(u_s(t_1),\psi_{t_1-t_0}(u_s(t_0)))>\delta_2.$$ Indeed, either $u_s:=u|_{s\times S^1}$ remains inside $U_3\setminus U_0$  or $u_s$ connects $U_2\setminus U_1$ with $\partial (U_3\setminus U_0)$. From this we deduce using Proposition \ref{PrpDistEn} that for a constant $C=C(J,H)$ we have
\begin{equation}\label{eqLoopwiseEst}
\int\|X_{H'_t}\circ u(s,t)-\partial_su(s,t)\|^2dt>C\delta_2. 
\end{equation}



We conclude that $E^{geo}(u)\geq (b-a)\delta_1$. On the other hand, monotonicity (Lemma \ref{lmMonEst++}) gives the estimate $E^{geo}+(b-a)\geq C\delta_0$. Combining the two estimates we get an a priori estimate on the energy of any trajectory connecting a periodic orbit in $V$ with one in $M\setminus V$.  Note that the constants all depend on $\cN$ and $\cB$ only. This proves the first part of the claim. 

For the second part, given $J_1,J_2\in \cB$ and $H_1\leq H_2\in\cN$ note that for any continuation trajectory between them we have a similar estimate depending only on the path of Floer data connecting them. To see this observe first that the constant the estimate in \eqref{eqLoopwiseEst}  can be made uniform for all almost complex structures in $\cB$ and Hamiltonians in $\cN$ at least if the metric on the space of almost complex structures takes into account sufficiently many derivatives. Moreover, this estimate depends on $s$ only through the value of $J_s$ on the loop $u_s$. For the monotonicity estimate note that the metric on $\bR\times S^1\times M$ induced by Gromov trick for the family of almost complex structures and Hamiltonians is  equivalent to the one associated with a fixed almost complex structure and Hamiltonians with equivalence constant depending on $|\partial_s(H_a,J_s)|$ which in turn can be made to depend only on $\cN,\cB$. By the last clause in Lemma \ref{lmMonEst++} and Remark \ref{rmMonotonicity} this produces a monotonicity estimate with constants depending only on  $\cN,\cB$. \end{proof}










\begin{lm}\label{lmGermHF}
Let $\gamma$ be an isolated family of $1$-periodic points of $H$. Then for any small enough neighborhood $V$ of $\gamma$ and for $\epsilon>0$ small enough, the group $HF^*_{[0,\epsilon)}(V;H,\cB)$  depends only on the restriction of $H,J$ to $V$. That is, given two pairs of data $(H_1,J_1)$ and $(H_2,J_2)$ whose restrictions to $V$ are the same, there is a canonical isomorphism between the corresponding groups. 
\end{lm}
\begin{proof}
For this observe that the estimate derived in Lemma \ref{lmEpsilonSeparation} says that if a Floer trajectory connects periodic orbits in $V$ and has energy $\leq\epsilon$ it stays inside $V$.  This observation was made by \cite{Hein}. Moreover, the estimates are \emph{robust} in the sense that they  depend only on the behavior of $(H,J)$ in $V$. Hence the claim.
\end{proof}

\subsection{The unweighted local Floer complex}

\begin{df}
Let $H$ be a Hamiltonian. A family $\gamma$ of $1$- periodic orbits of $H$ is said to be a \emph{well isolated $1$-periodic component} if 
\begin{enumerate}
\item $\gamma$ is path connected. 
\item $\gamma$ is isotropic in the sense that for any path in $\gamma$ the corresponding cylinder in $M$ has vanishing symplectic area \footnote{The latter is defined for any arbitrary continuous path of smooth loops by homotoping to a smooth path with fixed endpoints.}.  
\item $\gamma$ has an isolating neighbourhood $V$ such that $\omega|_V$ is exact.
\item Choosing an almost complex structure $J$ on $M$, the restriction to $V$ of the complexified canonical bundle is trivial. 
\end{enumerate}
We call such $V$ a \emph{well-isolating neighborhood.}
\end{df}
We will associate with any well isolated component $\gamma$ a Floer type complex $CF^*_{uw}(\gamma;H)$   \emph{over the ground ring $R$} constructed from Floer solutions in a small neighbourhood of $\gamma$.   We will then relate it to the truncated Floer cohomology in small action windows. 

To proceed we introduce an enhanced version of a notion introduced by Pozniak \cite{Pozniak}. 

We use the letter $\cH$ to denote a \emph{Floer theoretic diagram} by which we mean 
\begin{itemize}
\item a collection, also denoted by $\cH$, of smooth regular Floer data $F=(H,J)$, 
\item for each pair $F_1,F_2$ of such Floer data a set $\cH(F_1,F_2)$ of continuation data $(H_s,J_s)$ from $F_1$ to $F_2$, and,
\item  for each pair $k_1,k_2\in \cH(F_1,F_2)$ a contractible set of homotopies  from $k_1$ to $k_2$. 
\end{itemize}
We say that $\cH$ is \emph{contractible} if for any pair $F_1, F_2$ of Floer data in $\cH$ there is a Floer datum $F_3$ so that $\cH(F_1,F_3)\neq\emptyset$ and $\cH(F_2,F_3)\neq\emptyset$. 



The typical example to keep in mind is as follows.  Consider  an open set $\cN$ in the space of time dependent Hamiltonians in the $C^2$ topology and a bounded ball $\cB$ in the space of $\omega$-compatible almost complex structures with $C^k$ topology for sufficiently large $k$.  Denote by $\cH(\cN,\cB)$ the following Floer diagram. 
\begin{itemize}
\item The Floer data are all the pairs $(H,J)\in\cN\times\cB$  so that $(H,J)$ is regular,
\item the continuation data between $(H_1,J)$ and $(H_2,J)$ is the set of all monotone continuation data whose first derivative is supported in $[-1,1]\times S^1$ and is bounded above by $1$ in the region $[-1,1]\times S^1$.
\item The homotopies  between two continuation data are all the regular  paths in the space obtained from the previous part by dropping the regularity requirement. 
\end{itemize}
Observe that if there is a lower semi continuous function $H$ which dominates all the functions in $\cN$ and  $\cN$ contains a monotone sequence converging pointwise to $H$ then $\cH(\cN,\cB)$ is contractible. In this case we say that $\cH(\cN,\cB)$ \emph{converges to $H$}.


Let $\cU\subset\cL(M)$ be a subset of loop-space. Denote by $F_{\cH}(\cU)$ the set of all paths in $\cU$ which arise as a Floer solution associated with $\cH$. By this we mean any map $u:\bR\times S^1\to M$ that solves the Floer  equation associated with a Floer datum $F\in\cH$, a continuation datum in $\cH(F_1,F_2)$ or as part of the homotopy between two such continuation data.

We denote by $S_{\cH}(\cU)\subset\cU$ the image of $\bR\times F_{\cU}$ under the evaluation map
$(s,u)\mapsto u(s,\cdot)$.
\begin{df}
We say that $S_{\cH}(\cU)$ is \emph{isolated} if the closure of $S_{\cH}(\cU)$ is contained in $\cU$. 
\end{df}

Consider in particular  the case where $\cU$ is the set of loops contained in some open set $V$. We denote this by $\cU_V$.  Suppose $V$ is an exact isolating neighbourhood of a well isolated periodic component  $\gamma$  of some Hamiltonian $H$. 
A basic consequence of proof of Lemma \ref{lmEpsilonSeparation} is
\begin{lm}
For any real number $R$ and almost complex structure  $J$ if $\cN$ is a sufficiently small open neighbourhood of $H$ in the $C^2$ topology then letting $\cH:=\cH(\cN,J,R)$ we have that $S_{\cH}(\cU_V)$ is isolated.
\end{lm}
\begin{proof}
Fix a primitive on $V$ and any $\epsilon>0$.  Since $\gamma$ is isotropic and path connected,  the action of any periodic orbit in the family $\gamma$ equals a fixed constant which we may take to be $0$.  By taking $\cN$ an arbitrarily small neighbourhood of $H$  we guarantee that the $1$-periodic orbits contained in $V$ of any $H'\in\cN$ are arbitrarily close to $\gamma$ and so have actions which are arbitrarily close to $0$.  We can thus guarantee for any $\epsilon$ that for any pair of $1$-periodic orbits $\gamma_1,\gamma_2$ associated with Hamiltonians $H_1,H_2\in\cN$ and contained in $V$ we have that $|\Delta\cA|:=|\cA_{H_1}(\gamma_1)-\cA_H(\gamma_2)|<\epsilon$.  

The proof of Lemma \ref{lmEpsilonSeparation} now shows that all the relevant Floer solutions map into $V$.   
\end{proof}

An observation made by Pozniak \cite{Pozniak} is that for any $F\in \cH$ one can build a complex Floer complex  $CF^*_{uw}(\cU;F)$ over $R$ whose generators are periodic orbits in $\cU$ and whose connecting trajectories are paths in $\cU$. Indeed, by the assumption and Gromov-Floer compactness we the connecting trajectories square to $0$. In a similar way, given Floer data $F_1,F_2\in H$ so that $\cH(F_1,F_2)\neq\emptyset$ we get an induced map $HF^*_{uw}(\cU;F_1)\to HF^*_{uw}(\cU;F_2)$. 


In particular if $\cH$ is  contractible, we obtain a Hamiltonian Floer group $$HF^*_{uw}(\cU;\cH)=\varinjlim_{(H,J)\in\cH}HF^*_{uw}(\cU;H,J).$$  
\begin{df}
In case of $\gamma$, $V$ and $\cH$ as above we write
\begin{equation}
HF^*_{uw}(\gamma;H):= HF^*_{uw}(\cH(\cU_V;\cN,\cB)).
\end{equation}
We refer to this as the \emph{unweighted local Floer cohomology of $\gamma$. }
\end{df}


Dropping $V$ and $\cB$ from the notation is justified by the following Lemma. 
\begin{lm}\label{lmInfSHGermDep}
$HF^*_{uw}(\gamma;H)$ depends only on the germ of $M$ and $H$ near $\gamma$ in the sense that a symplectomorphism of the germs induces a canonical isomorphism of the corresponding unweighted Floer groups.
\end{lm}
\begin{proof}
First observe that if $\cN'\subset \cN$ is still an open neighborhood of $H$ then  $HF^*_{uw}(\cU_V;\cH(\cN',\cB))=HF^*_{uw}(\cU_V;\cH(\cN,\cB))$.  If $\cN'$ is chosen small enough so that the estimates guarantee that all Floer trajectories arising in the Floer diagram $\cH'=\cH(\cN',\cB)$ are contained in some small neighborhood $V'\subset V$,  then the sets $S_{\cH'}(\cU_V)$ and $S_{\cH'}(\cU_{V'})$ are the same.  Here we rely on the robustness of the $C^0$ estimates mentioned in the proof of Lemma \ref{lmGermHF}.  In this way we obtain independence of the isolating set $V$. Independence of $\cB$ is obtained in the same way. Namely shrinking from $\cN$ to $\cN'$ allows to enlarge $\cB$ to $\cB'$. But the diagram $\cH(\cN',\cB)$ is cofinal in the diagram $\cH(\cN',\cB')$ for $\cB\subset\cB'$.  
\end{proof}

We denote the underlying chain complex of $HF^*_{uw}(\gamma;H)$ by $CF^*_{uw}(\gamma;H)$. It is well defined up to contractible choice. Moreover, it can be modeled by the periodic orbits and Floer trajectories of a single Floer datum $(H',J)$ for $H'$ close enough to $H$. That is,  for an isolating neighborhood $V$ of $\gamma$ and for $H'$ close enough to $H$ and so that $(H,J)$ is regular we have a well defined chain complex $CF^*_{uw}(V;H',J)$ generated by the $1$ periodic orbits of $H'$ that are contained in $V$ and the differential is defined by the unweighted count of Floer solutions contained in $V$.  This complex $CF^*_{uw}(V;H',J)$ is a chain level model for $HF^*_{uw}(\gamma;H)$. 
 
\subsubsection{The Morse-Bott case}


We briefly treat the Morse-Bott case.  Let $H$ be a Hamiltonian. Let $\gamma$ be a Morse-Bott family of $1$-periodic orbits which is well isolated. 
Let $U$ be an open neighborhood of $\gamma$ containing no other $1$-periodic orbits, so that $\omega|_U$ is exact and so that that the canonical bundle over $U$ is still trivial. 

Fix a trivialization of the canonical bundle over $\gamma$ and denote by $\iota_{CZ}(\gamma)$ the Conley-Zehnder index of $\gamma$ according to this trivialization. Each point $x$ of $\gamma$ corresponds to a $1$-periodic orbit.  We thus can associate a Fredholm operator $D_{x}$ on the space of maps from the disc into $R^{2n}$ up to homotopy \cite{SeidelBook}. The orientation line  associated with $D_x$ is well defined up to canonical isomorphism.
\begin{df}
The local system $\cL=\cL(\gamma)$ is the local system of $\bZ$-coefficients whose stalk over each point $x\in\gamma$, where $\gamma$ is considered as a manifold whose points are $1$-periodic orbits of $H$, is the orientation line of the Fredholm operator $D_x$. 
\end{df}
We can now formulate 
\begin{tm}\label{tmMorseBottCascades}
There is a natural isomorphism $HF^*_{uw}(\gamma;H)\simeq HM^{*+\iota_{CZ}(\gamma)}(\gamma;\cL)$ where the right hand side is the Morse homology of $\gamma$ with respect to an arbitrary Morse function with coefficients twisted by $\cL$.
\end{tm}

\begin{proof}
We model $FH^*_{uw}(\gamma;H)$ using the Floer complex of a sufficiently close non-degenerate Hamiltonian $H'$.  We pick a Morse function $f$ on $\gamma$.  Denote by $n$ the dimension of the family $\gamma$.

The map is constructed by considering configurations consisting of a Floer trajectory $u$ with input a periodic orbit $x$ of $H'$ and output a periodic point $y'$ in the family $\gamma$ followed by an upward gradient trajectory $\alpha$ flowing into a critical point $y$ of $f$.  The virtual dimension is given by the formula $i_{CZ}(\gamma)+i_{morse}(y)-i_{CZ}(x)$. 

The map is given as follows. Denote by $D_u$ the Fredholm operator associated with Floer solutions whose input is $x$ and whose output is anywhere on $\gamma$. Then $\delta_x=|D_u|\otimes \cL_{y'}$. Given an orientation of $\gamma$ the latter is isomorphic to $T_{y'}W^u(y)\otimes\cL_{y'}$. The gradient trajectory $\alpha$ then gives an isomorphism 
$$
T_{y'}W^u(y)\otimes\cL_{y'}=W^u(y)\otimes\cL_{y}.
$$
The claim that this is a chain map and homotopy invertible is standard and we omit it. 
\end{proof}
\subsection{From truncated to unweighted Floer cohomology}
In the following abbreviate $\Lambda_{[0,\epsilon)}:=T^{-\epsilon}\Lambda_{>0}/\Lambda_{>0}$. This is the module over $\Lambda_{>0}$ consisting of elements whose norm is in the interval $\left[1,e^{\epsilon}\right)$. 
\begin{lm}\label{lmIsotropicComp}
Let $H$ be a smooth Hamiltonian and let $\gamma$ be a well isolated $1$-periodic component of $H$. 
 Then for any  choice of $\cB$ and an isolating neighborhood $V$ there is canonical isomorphism for $\epsilon>0$ small enough 
\begin{equation}\label{eqIsoInfFH}
HF^*_{[0,\epsilon)}(V;H,J)=HF^*_{uw}(\gamma;H)\otimes \Lambda_{[0,\epsilon)}
\end{equation}
for any $J\in\cB$. 
\end{lm}
\begin{rem}
More precisely, it can be shown there is a  map of the underlying chain complexes in equation \eqref{eqIsoInfFH} which is canonical up to contractible choice and induces an isomorphism on homology. Since $H$ is degenerate the underlying complex of the left hand side involves choosing a model for the homotopy colimit over functions converging to $H$. The proof is rather tedious which is why we formulate and prove only the cohomology level claim. 
\end{rem}


\begin{proof}
Fix $\cN$ a small enough neighborhood of $H$ so $S_{\cH(\cN,\cB)}(\cU_V)$ is isolated.  Consider a monotone sequence $H_i\in  \cN$ of non-degenerate Hamiltonians converging uniformly to $H$ and assume $(H_i,J)$ is regular. Assume $\epsilon>0$ is small enough so that all Floer trajectories connecting orbits in $V$ stay in $V$.  Then $$HF^*_{[0,\epsilon)}(V;H,J)=\varinjlim_i HF^*_{[0,\epsilon)}(V;H_i,J). $$On the other hand, for any $i$ we can model $CF^*_{uw}(\gamma;H)$ by $CF^*_{uw}(V;H_i,J)$. 

Exactness of $\omega|_V$ gives rise to an additional filtration on $CF^*_{[0,\epsilon)}(H_i;J_i;V)$ by action $\cA_{H_i}$. The definition of the action involves the choice of a primitive of $\omega$. For convenience we may assume by adding constants that $\cA_H(\gamma)\equiv 0$.  For any $i$ let $\delta_i<\epsilon$ such that the action spectrum of $H_i$ is contained in the open interval $(-\delta_i,\delta_i)$. We may take $\delta_i\to 0$ as $i\to\infty$.  
In particular,
$$HF^*_{[0,\epsilon)}(V;H,J)=\varinjlim_i HF^*_{[0,\epsilon-\delta_i)}(V;H_i,J).$$


Thus we can define a map 
$$\alpha_{i}:CF^*_{[0,\epsilon-\delta_i)}(V;H_i,J) \to CF^*_{uw}(V;H_i,J)\otimes  \Lambda_{[0,\epsilon)}$$
where we weight generators whose input is $x$  by $T^{-\cA_{H_i}(x)}$. Unwinding definitions this is well defined as a map of $\Lambda_{>0}$-modules. To check it is a chain map note that $\langle\alpha_i d x,y\rangle=nT^{-\cA_i(y)}T^{\cA_i(y)-\cA_i(x)}=nT^{-\cA_i(x)}$ for $n$ the number of Floer differentials from $x$ to $y$. On the other hand $\langle d\alpha_i  x,y\rangle=nT^{-\cA_i(x)}$.  Similarly, we have a homotopy commutative diagram
\begin{equation}\label{LclFHDiag}
\xymatrix{CF^*_{[0,\epsilon-\delta_i)}(V;H_i,J)\ar[d]\ar[r]&CF^*_{uw}(V;H_i,J)\otimes \Lambda_{[0,\epsilon)}\ar[d]\\
		CF^*_{[0,\epsilon-\delta_i)}(V;H_{i+1},J)\ar[r]&CF^*_{uw}(V;H_{i+1},J)\otimes  \Lambda_{[0,\epsilon)}}
\end{equation}
since the same is true for the unweighted version and the weights are again seen to depend only on the input. It follows that we have a well define map $\alpha$ between the corresponding colimits.


It remains to verify that $\alpha$ is an isomorphism. Abbreviate $A_i=CF^*_{[0,\epsilon-\delta_i)}(V;H_i,J)$ and $B_i=CF^*_{uw}(V;H_i,J)\otimes \Lambda_{[0,\epsilon)}$.  Denote by $\kappa_{ij}:A_i\to A_j,\eta_{ij}:B_i\to B_j$ the structural maps in the colimit.  Note the $\eta_{ij}$ are all homotopy equivalences. Suppose $\alpha(x)=0$. Then $x$ lifts to an element of $x\in A_i$ so that $\alpha_i(x)$ is a boundary. We have that $|x|<e^{\epsilon}$. That is, there is $\delta>0$ such that $T^{-\delta}\kappa_{ij}(x)\in A_j$ for all $j$. Pick $k>i$ such that  that the action spectrum of $H_k$ is contained in the interval $(-\delta/2,\delta/2)$. Then $\eta_{ik}\circ \alpha_i(x)$ is a boundary and it must be killed by an element $y$ such that $|y|<e^{\delta}|x|$. Thus $y$ can be lifted to an element of $B_k$ which kills $\kappa_{ik}(x)$. It follows that $x$ is null-homologous in the colimit. Injectivity follows.

We now prove surjectivity. Given an element $z=T^{-\lambda}y$ for $y\in HF^*_{uw}(\gamma;H)$ and $\lambda\in[0,\epsilon)$ we lift $z$ to some $z_i\in B_i$ whose action spectrum is contained in $(-\delta/2,\delta/2)$ for $\delta<\epsilon-\lambda$.  Let $y_i\in B_i$ be the element corresponding to $z_i$ under the naive identification of $A_i$ with $B_i$. Then $T^{-\delta'}(y_i)$ is well defined as an element of $A_i$ for all $\delta'$ in the action spectrum of $H_i$ so  $z_i$ can be lifted to $A_i$ and surjectivity follows.

\end{proof}

Let $\tau\mapsto H^{\tau}$ be a continuous monotone family of smooth Hamiltonians. 
\begin{df}\label{dfIsolatingIsotopy0}
An \emph{isotopy of well  isolated $1$-periodic components} is a family $\tau\mapsto\gamma^{\tau},\tau\in[0,1],$ of well isolated $1$-periodic components of $H^{\tau}$ such that there exists a homotopy of embeddings $f:[0,1]\times U\to M$ where 
\begin{itemize}
\item $U$ is a compact manifold with boundary,
\item $U^{\tau}:=f(\{\tau\}\times U)$ is a well isolating neighborhood of $\gamma^{\tau}$, and,
\item we can choose the primitives of $\omega$ and the trivializations of the complexified canonical bundle on $U^{\tau}$ so that they are locally constant in $\tau$.
\end{itemize}
\end{df}
Given such an isotopy define a function $\cA_t:[0,1]\to\bR$ by $t\mapsto \cA_{H^{t}}(\gamma^{t})$ where the action is defined using primitive which is locally constant in $t$.  

\begin{lm}\label{lmIstopyInfMap}
An isotopy of well isolated  components induces an isomorphism 
\begin{equation}\label{eqIsoInfFH2}
HF^*_{uw}(\gamma^0;H^0)\to HF^*_{uw}(\gamma^{\tau};H^{\tau}). 
\end{equation}
Moreover, for fixed $J,r$, there is an $\epsilon>0$ such that for any $\tau_1<\tau_2$ so that $\Delta:=\cA_{\tau_2}-\cA_{\tau_1}<\epsilon$, we have that $HF^*_{[0,\epsilon)}(U^{\tau_1};H^{\tau_1},J)$ maps into $HF^*_{[0,\epsilon)}(U^{\tau_2};H^{\tau_2},J)$ and with respect to the identifications \eqref{eqIsoInfFH} and \eqref{eqIsoInfFH2} the map is given by rescaling by $T^{\Delta}$. 
\end{lm}
\begin{proof}
After fixing a $J$ we can subdivide the interval $[0,1]$ into intervals of size $\delta$ which are so small that for appropriate homotopy $s\mapsto H_s$ from $H^{k\delta}$ to $H^{(k+1)\delta}$, the small energy continuation solutions $u$ satisfy that $u(s,\cdot)\subset U^{\tau}$ where $\tau$ is some fixed value in the interval. We can assume that the difference  $H^{(k+1)\delta}-H^{k\delta}$ is so small that the continuation maps for the unweighted complexes are defined for both sides. The isomorphism \eqref{eqIsoInfFH2} is constructed as  a composition of isomorphisms $HF^*_{uw}(\gamma^{k\delta};H^{k\delta})\to HF^*_{uw}(\gamma^{(k+1)\delta};H^{(k+1)\delta})$. This map is independent of the choice of small enough $\delta$. 

For the second part of the theorem pick $\epsilon$ so that Floer trajectories of energy $\leq\epsilon$ are local as in the previous paragraph. The claim then follows by picking arbitrarily close non-degenerate Hamiltonians and following the map as defined in the proof of Lemma \ref{lmIsotropicComp}.

\end{proof}



\section{Infinitesimal symplectic cohomology}\label{secInfFH}

Let $K\subset M$ be a compact set with smooth boundary of positive contact type.  The aim of this section and the next two is develop analogous results to those of the previous section in the case of a Hamiltonian which is the indicator function 
\begin{equation}
H_K(x):=\begin{cases}
0&\quad x\in K,\\
\infty&\quad x\not\in K.\notag
\end{cases}
\end{equation}
In other words, our goal is to relate the Floer cohomology of $H_K$ in small action windows to certain unweighted Floer cohomology groups associated with the Reeb components.  





The unweighted invariant is essentially the one defined in \cite{Mclean2012}.  For an isolated Reeb orbit $\gamma$ in the boundary $\partial K$, \cite{Mclean2012} associates a group $SH^*(\gamma)$. However, in order to relate the unweighted invariant to the truncated Floer cohomology we need a somewhat more refined construction.
\subsection{Symplectic cohomology of Reeb orbits}\label{SubsecSHReeb}


Let $\alpha$ be a contact form on $\partial K$ arising as a primitive of $\omega|_{\partial K}$. Assume for any $T>0$, the simple Reeb orbits of period $\leq T$ occur in a  discrete sequence $\gamma_i$ of isolated isotropic path connected components. We will denote by $\gamma_0$ the component consisting of the points of $K$. These are the periodic points which are fixed. We denote by $\cR(K)$ the set of all Reeb components including $\gamma_0$.

Before proceeding we point out that the set of periodic points under the Reeb flow depends only on $\omega$ and $\partial K$, not on the choice of primitive. Indeed the Reeb vector field is determined up to scaling as lying on the characteristic line field determined by $\omega$ and $\partial K$. The choice of $\alpha$ affects the period of each periodic point. Our isotropy assumption is that the latter is constant on each component. It will be important for the discussion below to allow $\alpha$ to change. We shall assume that $\alpha$ is of \emph{positive contact type} meaning that the Liouville field $Z$ determined by $\iota_Z\omega=\alpha$ points outwards of $K$. 

For a choice of $\alpha$ we denote by  $T(\gamma_i,\alpha):=-\int\beta_i^*\alpha$ be for $\beta_i$ the flow-line through any point of $\gamma_i$. By the isotropy condition, this is independent of the chosen point and is referred to as the period of $\gamma_i$ with respect to $\alpha$. We assume $T_i\leq T_{i+1}$ for all $i$. We refer to the set $\{T_i\}$ as the \emph{period spectrum of $\partial K$ with respect to the contact form $\alpha$}. If $\gamma_i$ is a critical component we take $T_i=0$. 






A choice of $1$-form $\alpha$ gives rise to a radial coordinate $r$ in a neighborhood $V$ of $\partial K$ in which the symplectic form is given by $\omega|_V=d(e^{r}\alpha)$ so that $\partial K$ is given by $\{r=0\}$. We call such a neighborhood $V$ a \emph{symplectization neighborhood}. 

Our study of the Floer cohomology of the non-continuous function $H_K$ will involve three stages of approximation. Namely, smooth non-degenerate $\to$ smooth $\to$ continuous $\to H_K$. First we introduce strictly S-shaped Hamiltonians, which are continuous but not smooth. 

\begin{df}\label{dfSShaped}
Fix a primitive $\alpha$ and a real number $\epsilon>0$ such that there is symplectization neighborhood $V$ of $\partial K$ with proper Liouville coordinate $t:V\to(-\epsilon,\epsilon)$. Define the Hamiltonian
\begin{equation}
H_{K,a,\epsilon,\alpha}(x):=\begin{cases}0,&\quad x\in K,\\  at,&\quad x\in t^{-1}([0,\epsilon)),\\ a\epsilon, &\quad x\in M\setminus K\cup V.
\end{cases}
\end{equation}
\end{df}

Note that for fixed $\epsilon$ and $\alpha$ we have that $H_{K,a,\epsilon,\alpha}\to H_K$ as $a\to\infty$. From now on we abbreviate the data of the parameters as $I=(\alpha,\epsilon,a)$.  To refer to the individual parameters we will write $I$ as a subscript: $a_I,\epsilon_I,$ and $\alpha_I$.  Note there is  a partial order on the space of parameters $I$ by $I_1\leq I_2 \iff H_{K,I_1}\leq H_{K,I_2}$. 

% Figure environment removed

For $a>0$ let $\delta(a)$ be the minimal gap in $Spec(\partial K,\alpha)\cap [0,a]$.

Given a loop $\gamma$ and an isolating open neighborhood $V$ of $\gamma$ in $\partial K$. For any real number $r<\epsilon/10$ let $U_{r}\subset M$ be given in symplectization coordinates by $V\times (-r,r)$. Let $\cU_I(\gamma,r)$ be the set of loops in $U_{r}$  whose action is in the $\delta$ interval centered at $T(\gamma)$.

For a parameter $I$ we say that a subset $\cN\subset C^2(S^1\times M))$ is \emph{$I$-thick} if it is open and contains a cofinal sequence converging to $H_{K,I}$. 
\begin{prp}\label{prpFloerDiagGamma}
Suppose $a_I$ is not in the period spectrum. 
Then
\begin{enumerate}
\item \label{prpFloerDiagGamma:it1} For any real number $r$ and any ball $\cB$ of compatible almost complex structures there is an $I$-thick set $\cN=\cN(I,\cB,r)\subset C^2(S^1\times M)$ such that writing $\cH=\cH(\cN,\cB)$  we have that $S_{\cH}(\cU_I(\gamma,r))$ is isolated.  While the construction involves a choice,  every two choices are equivalent in the sense that their intersection is still $I$-thick. 
\item \label{prpFloerDiagGamma:it2}
For $r_2\leq r_1$ and $\cB_1\subset \cB_2$ write $\cN_i=\cN(I,r_i,\cB_i)$. Then $\cN_1\cap\cN_2$ is $I$-thick. %\cH_i=\cH_{K,I,r_i,R}$.  
Writing $\cH'_2:=\cH(\cN_1\cap\cN_2,\cB_2)$ we have
%Then $\cH'_2$ is still open and converges to $H_{K,I}$ and satisfies 
$S_{\cH'_2}(\cU_I(\gamma,r_1))=S_{\cH'_2}(\cU_I(\gamma,r_2))$. 


\end{enumerate}
\end{prp}

The best way to understand the content of this notation heavy statement is to discuss its consequence which we turn to do.  The proof will be given in the next subsection.

\begin{cy}
Let
\begin{equation}
HF^*_{uw}(\gamma;I,\cB,r):=\varinjlim_{(H,J)\in\cN(I,\cB,r)}HF^*_{uw}(\cU_I(\gamma,r);H,J). 
\end{equation}
Then the Floer groups $HF^*_{uw}(\gamma;I,\cB,r)$ depend only on $\gamma$ and $I$. That is, for any two choices for the other parameters there is a canonical isomorphism between the corresponding groups.  Thus we may abbreviate the push-out of all these isomorphisms by $HF^*_{uw}(\gamma;I).$

\end{cy}
\begin{proof}

First observe that independence of $J$ is subsumed under independence of $R$ by definition.   Given $r_1\leq r_2$ and $\cB_1\geq \cB_2$,    \ref{prpFloerDiagGamma}\ref{prpFloerDiagGamma:it2} gives


$$
HF^*_{uw}(\cU_1;\cH(\cN_1,\cB_1))=HF^*_{uw}(\cU_2;\cH(\cN_1,\cB_1))=HF^*_{uw}(\cU_2;\cH(\cN_2,\cB_2))
$$

\end{proof} 
In fact, we will show
\begin{prp}\label{prpInfSHGermDep}
The Floer group $HF^*_{uw}(\gamma;H_{K,I})$ depends only on the germ of $(M,\partial K)$ and $I$ in a neighbourhood of $\gamma$ in the sense that a symplectomorphism of the germs induces an isomorphism of the corresponding groups. 
\end{prp}

Finally,  we will prove the independence on the parameter $I$.   Given an $I$ thick set $\cN\subset C^2(S^1\times M)$  a subset $\cN'\subset\cN$ is called \emph{almost full} if there is an $H<H_{K,I}$ such that $\cN'$ contains all elements of $\cN$ which are greater than $H$. 


\begin{prp}\label{prpFloerDiagMor}
Given a monotone sequence $I_1\leq\cdots \leq I_n$, numbers $r_i$ and ball $\cB_i$ for $i=1,\dots n$ then there is an open $\cN$ such that letting $r=min\{r_i\}$, $\cB$ a ball containing the $\cB_i$,  and $\cH=\cH(\cN,\cB)$, the following hold
\begin{enumerate}
\item For each $i$ the intersection $\cN\cap\cN_i$ is an almost full subset of $\cN_i$. 
\item $\cN_n$ is cofinal in $\cN$ 
\item $S_{\cH}(\cU(\gamma,r))$ is isolated.  
\end{enumerate}
We thus get induced maps  $HF^*_{uw}(\gamma;I_1)\to HF^*_{uw}(\gamma;I_2)$ for $I_1\leq I_2$. These map are functorial and are isomorphisms.  
\end{prp}

As a result, for each periodic component we denote by $SH^*_{uw}(\gamma)$ the push-out of all these isomorphisms. In more down to earth terms, $SH^*_{uw}(\gamma)$ can be computed by taking a Floer datum $(H,J)$ for $H$ a Hamiltonian which is so close to being S-shaped that Floer trajectories formed by connecting periodic orbits which are near $\gamma$ are confined to an isolating neighborhood of $\gamma$. We then take $SH^*_{uw}(\gamma)$ to be the unweighted complex formed by these orbits and trajectories. The Floer diagram is needed to keep track of how the unweighted groups obtained by different choices of $(H,J)$ are identified.

\subsection{Proofs}


\subsubsection{Construction of the Floer diagrams}\label{SubsecConstr}


We would like to approximate the continuous functions $H_{K,I}$ by smooth functions. 

 Fix any $\eta\ll \min\{\epsilon,\delta(a)\}$ and denote by $\cK(K,I,\eta)$ the set of all Hamiltonians $H$ such that



\begin{enumerate}
\item $H|_{M\setminus t^{-1}(0,\epsilon)}$ is $C^2$ small enough to have no non-trivial periodic orbits. 
\item In the region $t^{-1}(-\epsilon,\epsilon)$ we have that $H$ factors as $H=h\circ e^t$ and we require the following behavior of $h$ and $H$.
\begin{enumerate}
\item $0>H|_K>-\eta/8$.
\item $h'\equiv a$ on the interval $(\eta/8,\epsilon-\eta/8)$.
\item $h$ is convex on $[0,\eta/8)$ and concave on $[\epsilon-\eta/8,\epsilon]$. 
\end{enumerate}

\end{enumerate}

 

For each $\eta\ll\epsilon$ fix once and for all an element $H_{K,I,\eta}\in\cK(K,I,\eta)$. We refer to $\eta$ as the \emph{smoothing parameter}. If $\eta_i\to 0$ then  $H_{K,I,\eta_i}$  converges uniformly to $H_{K,I}$. Fix once and for all a  sequence $\eta_i\to 0$ such that $K_{K,I,\eta_i}$ is monotone. 





For $H=H_{K,I,\eta}$ with $a_I$ not in the period spectrum, the set of $1$-periodic components $Per(H)$ can be described as follows. There are the fixed points inside, which we may take to be just one component $\gamma_0$ which coincides with $K$. Then each Reeb component $\gamma$ with $T(\gamma,\alpha)<a$ occurs in two incarnations, an \emph{inner incarnation} $\gamma^{i}$ and  and \emph{outer incarnation} $\gamma^{o}$. Finally, we have orbits which occur away from the union of $K$ with the symplectization neighborhood of $V$ of  $\partial K$. 


We refer to the components in the inner incarnation together with the fixed points inside $K$ as \emph{convex orbits} and to components in the second one together with critical points outside $V$ as \emph{concave orbits}. 

Even though we are in a non-exact setting, since the functions $H_{K,I,\eta}$ have no non-trivial periodic orbits outside of the symplectization neighborhood we can unambiguously assign an action to each $1$-periodic component. These behave as follows:
\begin{equation}
\cA(\gamma)=T(\gamma,\alpha)+O(\eta)
\end{equation} 
for convex orbits, and,
\begin{equation}
\cA(\gamma)\leq(1+\epsilon)T(\gamma,\alpha)-(\epsilon-2\eta)a,
\end{equation}
for concave orbits,
where for $\gamma$ a $1$-periodic orbit of $H$ we denote by $T(\gamma,\alpha)$ the alpha period of the underlying Reeb component. If $\gamma$ happens to be a fixed point we just take $T(\gamma,\alpha)=0$. Note we typically take $\eta\ll\epsilon$. So, for $T(\gamma,\alpha)$ fixed, the action goes to $-\infty$ as $a$ increases to $\infty$.

 


\subsubsection{Proof of Proposition \ref{prpFloerDiagGamma}}


We will take $\cN(I,\cB,r)$ to be a sufficiently small neighborhood of $\cK(K,I,\eta)$ for $\eta$ small enough.  Clearly this is an $I$-thick set. We need to prove that by taking the neighborhood and $\eta$ small enough we have isolatedness.  Let us formulate this as a proposition. 
\begin{prp}\label{prpActionWindowComponent}
If $\eta$ is small enough and $\cN$ is a small enough  neighborhood of $\cK(K,I,\eta)$ in $C^2$ then  for $\cH=\cH(\cN,\cB)$ we have that $S_{\cH}(\cU_I(\gamma,r))$ is isolated. 
\end{prp}

Note this does not follow from the argument of Lemma \ref{lmEpsilonSeparation} since $\cK(K,I,\eta)$ does not consist of Hamiltonians that are $C^k$ close to each other on the neighborhood of some $1$ periodic orbit of a fixed Hamiltonian. The proof of Proposition \ref{prpActionWindowComponent} will require the following observation. Fix as before a symplectization neighborhood $V\simeq\partial K\times (-\epsilon,\epsilon)$ and let $\rho$ be the coordinate in the $\bR$ direction. Denote by $H_1(x,\rho)=e^\rho$ the Hamiltonian whose flow is the Reeb flow in a symplectization neighborhood. Let $H_2=h\circ H_1(x,\rho)$ for some monotone function $h$. Let $\gamma=(\alpha,a):[0,T]\to V$ be a smooth path.  Consider the function $\tau(t):=\int_0^t h'\circ a$ on the interval $[0,T]$. It is monotone, and so, invertible. For any $H$ and $t$  and for any $p\in\bR_+$ denote by $E^p_H(\gamma;t)$ the energy $\int_0^t\|\gamma'(x)-X_H\circ\gamma(x)\|^pdx.$ When there is no superscript, assume $p=2$.
\begin{lm}\label{lmReparam}
We have
\begin{equation}
E^1_{H_2}(\gamma;t)=E^1_{H_1}(\gamma\circ \tau^{-1},\tau(t)).
\end{equation}
\end{lm}
\begin{proof}
Let $g=\tau^{-1}$. Then $g'(s)={h'(a(g(s))}^{-1}$. 
Substituting in the integral $dt=g'(s)ds$ and using $X_{H_2}=(h'\circ a )X_{H_1}$, we obtain
\begin{align*}
    \int_0^t\|\gamma'(t)-X_{H_2}\circ\gamma(t)\|dt&=\int_0^{\tau(t)}\|\gamma'\circ g(s)-h'(a(g(s))) X_{H_1}\circ\gamma(g(s))\|g'(s)ds\\
    &=\int_0^{\tau(t)}\|(\gamma\circ g)'(s)-X_{H_1}\circ(\gamma\circ g)\|ds\\
    &=E^1_{H_1}(\gamma\circ \tau^{-1};\tau(t)).
\end{align*}
%Rearranging, the right hand side becomes $E^1_{H_1}(\gamma\circ \tau^{-1};\tau(t))$. 
\end{proof}



\begin{proof}[Proof of Proposition \ref{prpActionWindowComponent}]

The proof is similar to that of Lemma \ref{lmEpsilonSeparation} with some adjustments which we spell out. It is assumed to be read in conjunction with the proof of Lemma \ref{lmEpsilonSeparation}. First observe can find open sets $V_0\subset V_1\subset V_2\subset V_3\subset \partial K$ so that

\begin{enumerate}
\item
The only closed Reeb orbits whose action is in the $r$-interval around $T(\gamma)$ and which meet $V_3$ are the orbits of $\gamma$.

\item The minimal flow-time from $V_2$ to the boundary of $V_3$ and from $V_0$ to the boundary of $V_1$ is at least $2a_I$. 
\end{enumerate}

In symplectization coordinates we let $U_i=V_i\times[-\delta,\delta]$. Denote by $\pi_i$ for $i=1,2$ the projections to $\partial K$ and to $\bR$ respectively in symplectization coordinates. By making the neighborhood $\cN$ and $\eta$ small enough, we guarantee that all the Floer trajectories in  $S_{\cH}(\cU_I(\gamma,r))$  have arbitrarily small energy, and have ends contained in arbitrarily small neighborhoods of $\gamma$. 

The reasoning of Hein's Lemma immediately allows to confine $\pi_2\circ u$ in $(-\delta,\delta)$ for $u\in S_{\cH}(\cU_I(\gamma,r))$. Indeed, for any fixed $\delta'>0$ monotonicity applies with uniform constants in the region $\pi_2^{-1}([-\delta,-\delta+\delta']\cup [\delta',\delta])$ for all Floer data in $\cH$. The energy distance inequality applies since $a_I$ is not in the period spectrum. 

It thus remains to confine the projection $\pi_1\circ u$. Denote by $\psi_t$ the time $t$ flow under $H_1$. The pre-image $u^{-1}\circ\pi^{-1}(V_2\setminus V_1)$ contains a connected component $S\subset\bR\times S^1$. Let $[a,b]$ be the image of $S\subset\bR\times S^1$ under projection to the first factor. We now find a constant $\delta'$ such that 
$$
E^2_H(u(s,\cdot))>\delta'
$$
for all $s\in[a,b]$.
By \ref{lmReparam} $E^1_H(u(s,\cdot);1)=E^1_{H_1}(u(s,\tau(\cdot));\tau(1))$. Moreover, we have $\tau(1)\leq a_I$. If $u_s$ does not remain inside $V_3\setminus V_0$ then using our assumptions on the flow of $H_1$,  Lemma \ref{lmReparam} and Proposition \ref{PrpDistEn0}, we obtain an estimate  $E^1_H(u(s,\cdot))>\delta''$ for some $\delta''>0$ depending only on bounds on the geometry of $J$ and on $I$. If $u_s$ remains inside,  fix a real number $t_0$ to be determined momentarily. If $\tau(1)>t_0$, there is a bound from below on the distance $d(\psi_{\tau(1)}(x),x)$ for all $x\in V_2\setminus V_1$, which leads again to an estimate $E^1_H(u(s,\cdot))>\delta''$ for appropriate $\delta''$ depending on $t_0$.

Suppose on the other hand that $\tau(1)<t_0$. We have an estimate
$$
E^1_H(\gamma)=E^1_{H_1}(\gamma;\tau(1))\geq\ell(\gamma)-t_0\sup\|X_{H_1}\|. 
$$
Now note that  the action of $u_s$ is contained in a small interval around $T(\gamma)$ and so is bounded away from $0$. From this we deduce an a priori estimate from below on the length of $\gamma$. Indeed, for sufficiently small length the action is given roughly by the integral of $\omega$ on a filling of $u_s$, which by the isoperimetric inequality (see, e.g., \cite[Ch. 4]{MS2}) is estimated from below by $\ell^2(u_s)$. By making $t_0$ sufficiently small we thus obtain an estimate as required. 

Having estimated $E^1_H(u(s,\cdot))>\delta''$ we obtain an estimate from below $E^2_H(u(s,\cdot))>\delta'$ by applying Cauchy-Schwarz. 

%We then find $\delta'$ as required reasoning as in \ref{PrpDistEn} keeping in mind that $\tau(t)<a_I$. It remains to deal with the case $u_s$ does remain inside.

% If $\tau(1)>t_0$, we derive a uniform estimate from below on  $E_H(u(s,\cdot))$ , depending on $t_0$, using  Lemma \ref{lmReparam} and Proposition \ref{PrpDistEn} again. Suppose on the other hand that $\tau(1)<t_0$. We have an estimate
%$$
%E_H(\gamma)=E_{H_1}(\gamma;\tau(1))\geq\ell^2(\gamma)-t_0\sup\|X_{H_1}\|. 
%$$
%Now note that  the action of $u_s$ is contained in a small interval around $T(\gamma)$ and so is bounded away from $0$. From this we deduce an a priori estimate from below on the length of $\gamma$. Indeed, for sufficiently small length the action is given roughly by the integral of $\omega$ on a filling of $u_s$, which by the isoperimetric inequality (see, e.g., \cite[Ch. 4]{MS2}) is estimated from below by $\ell(u_s)^2$. By making $t_0$ sufficiently small we thus obtain an estimate as required. 

We conclude that $E^{geo}(u)\geq (b-a)\delta'$. To continue we observe that the monotonicity constant of Lemma \ref{lmMonEst++} is sensitive only to the $C^0$ norm of the Hamiltonian vector field. And thus, as in  the proo of Lemma \ref{lmEpsilonSeparation} we get an estimate $E^{geo}+(b-a)\geq C\delta''$. Combining the two estimates we get an a priori estimate on the energy of any trajectory associated with $\gamma$ and leaving $V_3$. 



%For any $t\in[0,a_I]$ let $d_t$ be a bound from below on distance $d(x,\psi_t(x))$ of all $x\in V_3\setminus V_0$. By our assumptions, for any $t_0>0$, $d_t$ is bounded away from $0$ on the interval $[t_0,a_I]$. Let $\delta$ be a bound from below on the distance $d(\psi_t(x),\partial (V_3\setminus V_0))$ for $x\in V_2\setminus V_1$ and for $t\in[0,a_I]$. All bounds are assumed to hold for metrics determined by any almost complex structure in $\cB$. 
\end{proof}


We now proceed to
\begin{proof}[Proof of Proposition \ref{prpFloerDiagGamma}]
\begin{enumerate}
\item We fix a small enough $\eta$ and  $\cN$ as in Proposition \ref{prpActionWindowComponent}.  For any two such choice $\cN,\cN'$ the intersection $\cN\cap\cN'$ is open and contains the set $\cK(K,I,\eta)$ and is thus $I$-thick.
\item
Given $r_1\leq r_2$ and $\cB_2\subset\cB_1$ let $\cN_1,\cN_2$ be choices of neighborhoods as in the previous part.  Then $\cN_1\cap\cN_2$ is $I$-thick since it contains  the set $\cK(K,I,\eta)$.   For the equality $S_{\cH'_2}(\cU_I(\gamma,r_1))=S_{\cH'_2}(\cU_I(\gamma,r_2))$ note that by choice of $\cN_2$ all Floer solutions arising from the diagram  $\cH'_2$ and connecting loops in $\cU_I(\gamma,r_2)$ are paths in $\cU_I(\gamma,r_2)$ regardless of the fact that we do not impose this as a requirement. 

\end{enumerate}

\end{proof}


\begin{proof}[Proof of Proposition \ref{prpInfSHGermDep}]
Our confinement estimates build on the monotonicity estimate of Lemma \ref{lmMonEst++} and on the energy distance estimate of Proposition \ref{PrpDistEn} both of which are robust in the sense that the constants are unaffected by the behaviour outside if the region under consideration. The claim follows as in the proof of Lemma \ref{lmInfSHGermDep}. 

\end{proof}









\subsubsection{Proof of Proposition \ref{prpFloerDiagMor}}

We will pick an $\eta$ and a neighborhood $\cN$ of $\cup_{i=1}^n\cK(K,H_{I_i},\eta)$ in $C^2(S^1\times M)$. Clearly,  we have that $\cN\cap\cN_i$ is almost full in $\cN_i$, and $\cN_n$ is cofinal in $\cN$. A slight adjustment of argument in the proof of Proposition \ref{prpActionWindowComponent} shows that by taking $\eta,\cN$ small enough we get the isolatedness property. 

 Then $\cN$ contains information for the construction of the Floer groups $HF_{uw}(\gamma;I_i)$, for maps $HF_{uw}(\gamma;I_i)\to HF_{uw}(\gamma;I_j)$,  for $i\leq j$ and for the homotopies witnessing functoriality.  Note that almost  fullenss guarantees independence on the choice of $\cN$. We thus get functorial maps as required.  
 
It remains to show that these maps are isomorphisms. We show first that if $I_1=(a_1,\alpha,\epsilon)$ and $I_2=(a_2,\alpha,\epsilon)$ the map is an isomorphism.  For this it suffices to find a pair of sequences cofinal in $\cN\cap\cN_i$ for $i=1,2$ respectively for which the continuation maps are invertible.  The functions in the pair are constructed as follows.  Let $H^1_{\eta}\in\cK(K,I_1,\eta)$ and let $H^2_{\eta,\zeta}\in\cK(K,I_2,\zeta)$ assuming $\eta\ll\zeta$ and $H^1_{\eta}$ coincides with $H^2_{\eta,\zeta}$ on the symplectization interval $(-\zeta/2,\zeta/2)$.  By making $\eta$  arbitrarily small while keeping $\zeta$ fixed, we guarantee that any continuation trajectory connecting orbits associated with $\gamma$ has so little energy that it cannot reach out of the the symplectization interval $(-\zeta/2,\zeta/2)$.  Then we can define inverse continuation maps on the unweighted complexes for such a pair.  Call a pair $(\eta,\zeta)$ satisfying this \emph{admissible}. It is clear that we can construct a sequence $(\eta_i,\zeta_i)$ of such pairs with $\zeta\to 0.$ The corresponding sequence of pairs is cofinal. We thus conclude that the natural map $HF^*_{uw}(\gamma;I_1)\to HF^*_{uw}(\gamma;I_2)$ is an isomorphism whenever $\alpha_{I_1}=\alpha_{I_2}$ and $\epsilon_{I_1}=\epsilon_{I_2}$.  We  can easily relax the requirement that $\epsilon_{I_1}=\epsilon_{I_2}$ by the same argument as before.  To remove the requirement on $\alpha$ not that given $I_1,I_2$ we can find $I'_1>I_1,I_2$  and $I'_2>I'_1$ such that $\alpha_{I'_i}=\alpha_{I_i}$.  Since the map $I_1\to I'_1$ is an isomorphism factoring through the map $I_2\to I'_1$, the latter is a surjection. At the same time the isomorphism $I_2\to I'_2$ factors through the same map, implying its also an injection. Thus that map $I_2\to I'_1$ is an isomorphism. By the 2 out of 3 rule, the map $I_1\to I_2$ is an isomorphism. 

\subsubsection{Behavior under isotopy}\label{subsecIsotopy}
So far we have considered changes of parameters in which $\gamma$ stays fixed, and moreover, all the parameters are in contractible spaces. We now discuss an isomorphism of a somewhat different nature which is associated with a path. 
 
\begin{df}
    
Given a Reeb component $\gamma$, a family $T\mapsto V^T, T\in[0,\infty),$ of compact domains is said to \emph{isolate $\gamma$ well} if for each $T$
\begin{itemize}
\item  $\gamma$ is contained in the interior of $V^T$, 
\item any periodic points of period $\leq T$ that are in the neighborhood $V^{T}$ lie on $\gamma$, and,
\item $\omega|_{V^T}$ is exact and the complexified canonical bundle restricted to $V^T$ is trivial. 
\end{itemize}
\end{df}

\begin{df}\label{dfIsolatingIsotopy}
Given a path $\tau\mapsto K^{\tau}$ of smooth domains with contact boundary, consider an isotopy $\tau\mapsto \gamma^{\tau}$ of Reeb components of $K^{\tau}$. We say that this isotopy is \emph{isolated} if there is a family $V^{T,\tau}$ such that 
\begin{itemize}
    \item for fixed $\tau$, the family $T\mapsto V^T$ isolates $\gamma^{\tau}$ well,
    \item for each fixed $T$ the family $\tau\mapsto V^{T,\tau}$ is a continuous isotopy. That is, there is a homotopy $f^T:[0,1]\times V^T\to M$ such that $V^{T,\tau}=f^{T}(\{\tau\}\times V^T),$ and,
    \item we can choose the primitives of $\omega$ and the trivializations of the complexified canonical bundle on $V^{T,\tau}$ so that they are locally constant in $\tau$.
\end{itemize}

\end{df}
\begin{rem}
Note the component itself might be changing topology along the isotopy, as long as for each $\tau$ it stays connected. 
\end{rem}
%for each $T>0$ there is a continuous isotopy of  compact domains $V^{T,\tau}$ isolating $\gamma^{\tau}$ in the sense that for each $\tau$ the  only periodic points of period $\leq T$ that are in the neighborhood $V^{T,\tau}$ lie on $\gamma^{\tau}$ and $\gamma^{\tau}$ is contained in the interior of $V^{T,\tau}$. 

\begin{lm}\label{lmIsotopUW}
In the setting of Definition \ref{dfIsolatingIsotopy} there is an induced isomorphism $SH^*_{uw}(\gamma^0)\to SH^*_{uw}(\gamma^1)$.
\end{lm}
\begin{rem}
This isomorphism \emph{may depend} on the path $\tau\mapsto K^{\tau}$. 
\end{rem}
\begin{proof}
By subdividing the interval into a finite number of pieces we may assume that there is a fixed primitive $\alpha$ which is of Liouville type for $\partial K_{\tau}$ for all $\tau$ in the interval. We fix a slope $a$ such that $T(\gamma^{\tau})\leq a$ for all $\tau$ in the interval.  By further dividing the interval, we may assume that $a$ is not in the period spectrum for any $\tau$. Letting $I=(\alpha,a,\epsilon)$ the family of functions $\tau\mapsto H_{K^{\tau},I}$ is monotone.  For each $\tau$ we have symplectization coordinates determined by $\alpha$ and $\partial K^{\tau}$.  We can take the monotonicity constants to be independent of $\tau$.  By further dividing the interval, we may assume without loss of generality that $a$ is not in the period spectrum for any $\tau$.  We may thus take the bound on the left hand side  of equation \eqref{eqBoundL2L1} to be bounded away from $0$ for all $\tau$ in the interval.  By compactness the Lyapunov constant may be bounded independently of $\tau$ and so the estimate \eqref{eqBoundL2L1} is independent of $\tau$.  Finally,  by compactness and further subdivision we may assume there is a fixed open set $V$ which isolates all the $\gamma^{\tau}$ from any other periodic orbit of period $\leq a$ for any $\tau$. There is then an $\epsilon_0>0$ such that any continuation trajectory leaving the open set $V$ has energy at least $\epsilon_0$. By subdividing the interval further into a finite number of pieces, we may assume that the action differences going from $\tau$ to $\tau'$ is $\leq\epsilon_0/2$. We can further assume that $|H_{K^{\tau},I}-H_{K^{\tau'},I}|$ is $\leq\epsilon/2$. 

To sum up, we can subdivide the interval into a finite number of sub intervals $\tau)<\tau_1\dots<\tau_N$ so there is a continuation map of unweighted Floer complexes from going from $\tau_i$ to $\tau_{i+1}$ and this map is invertible.  It is clear by construction that this independent of any choices.  The desired isomorphism is the composition of all these isomorphisms. 

\end{proof}


\section{From truncated to unweighted $SH$ of Reeb orbits}\label{SecTruncSH}

\subsection{The action filtration in small action windows}
We would like to prove an analogue of Lemma \ref{lmIsotropicComp} relating truncated Floer cohomology of $H_K$ to the Reeb orbit cohomologies $SH^*_{uw}(\gamma)$. However, we run into a difficulty that as we increase the slope, the size of the window for which a priori $C^0$ estimates produce  a summand associated with $\gamma$ goes to $0$.  It turns out that in order to get windows of fixed size (depending on $\gamma$) we first need to deal with the concave orbits. 


 
\begin{prp}\label{prphbarfilt}
There is an $\hbar>0$ so that in action windows of size $\hbar$  we have
\begin{enumerate}
\item If we construct $CF^*(H_{K,I})$ by taking as in \S\ref{SubsecConstr} a homotpoy colimit over functions $H=H_{K,I,\eta_i}$ for $\eta_i$ small enough then $CF^*(H_{K,I})$ is filtered by the  action $\cA_{H_{K,I}}$ and continuation maps from $H_{K,I}$ to $H_{K,I'}$ respect this filtration. 
\item 
The concave orbits form a subcomplex in $CF^*(H_{K,I}).$ 
\end{enumerate}
\end{prp}

The proof of Proposition \ref{prphbarfilt} will be given at the end of the section. 

As a consequence of Proposition \ref{prphbarfilt} we get for any $\delta\in(0,\hbar)$ a long exact sequence 
\begin{equation}\label{eqInOutLES}
\dots\to Out^*_{[0,\delta)}(H_{K,I})\to HF^*_{[0,\delta)}(H_{K,I})\to In^*_{[0,\delta)}(H_{K,I})\to \dots
\end{equation}
where $Out^*_{[0,\delta)}(H_{K,I})$ is the  truncated homology of the sub-complex generated by outside orbits in accordance with Proposition Similarly, $In^*_{[0,\delta)}(H_{K,I})$ denotes the corresponding quotient complex. Moreover, for $I'$ such that $H_{K,I}\leq H_{K,I'}$ we have natural maps of  the corresponding long exact sequences
\begin{equation}
\xymatrix{\dots\ar[r]& Out^*_{[0,\delta)}(H_{K,I})\ar[d]\ar[r]&  HF^*_{[0,\delta)}(H_{K,I})\ar[d]\ar[r]& In^*_{[0,\delta)}(H_{K,I})\ar[d]\ar[r]& \dots
\\
\dots\ar[r]& Out^*_{[0,\delta)}(H_{K,I})\ar[r]& HF^*_{[0,\delta)}(H_{K,I})\ar[r]& In^*_{[0,\delta)}(H_{K,I})\ar[r]& \dots}
\end{equation}


\subsection{A $C_0$ estimate for $S$-shaped Hamiltonians. }
For the following Proposition fix a positive real number $K$.  Let $M$ be a symplectic manifold with boundary. Let $S\subset\bR\times S^1$ be a compact Riemann surface with boundary.  Let $J_z$ be an $S$-parametrized family of  almost complex structures on $M$ with absolute value of sectional curvature bounded from above by $K$ and with radius of injectivity bounded from below by $\frac1{\sqrt{K}}$. Let $H:S\times M\to\bR$ be a domain dependent Hamiltonian satisfying $\partial_sH\geq 0$. For a solution $u$ to Floer's equation denote by $E^{geo}(u):=\int_S\|\partial_su\|^2dtds.$

\begin{prp}\label{prpSmallDiamEst}
 There is a continuous function $h:(\bR_+)^2\to\bR_+$ which converges to $1$ at $(0,0)$ and has the following significance. Let $\gamma:[0,1]\to S$ be a geodesic. Let $u:(S,\partial S)\to (M,\partial M)$  be a solution to Floer's equation. Let $d=dist(u(\gamma(0)),u(\gamma(1))$. Let $d_0=dist(u(\gamma),\partial M)$.  Assume $|X_H|<1/2$.  Then for any $\tau\in(0,1]$ such that $\|X_H\|<\tau$ and writing  $\delta=\min\left\{d_0,\frac1{\sqrt{K}}\right\}$ we have 
\begin{equation}
d<\frac1{\delta\tau}h\left(\frac{\left|\partial_sH\right|_{\infty}}{\tau^2},\frac{\|X_H\|^2_{\infty}}{\tau^2}\right)\left(E^{geo}(u)+\tau^2\ell(\gamma)\right).
\end{equation}
\end{prp}



\begin{proof}
We refer to Appendix \ref{SecEstimates} for the notions introduced during the proof. For each $\tau$ let $\tilde{\omega}_\tau:=\tau^2\pi_1^*\omega_{\Sigma}+\pi_2^*\omega_M+dH\wedge dt$. Applying the Gromov trick to the space $S\times M$ we obtain a family of (Cauchy-)complete metrics $\tilde{g}_\tau:=g_{\tilde{J}_\tau}$. For each $\tau$ let $C_\tau=C_{\tau}\left(\frac{\left|\partial_sH\right|_{\infty}}{\tau^2},\frac{\|X_H\|^2_{\infty}}{\tau^2}\right)$ be the constant introduced in Lemma \ref{lmProdCEquivalence}. Then by Lemma \ref{lmProdCEquivalence} the geometry of $\tilde{g}_\tau$ is $C_\tau$ equivalent to the product metric $\tau^2\pi_1^*g_{\Sigma}+\pi_2^*g_M$. The product metric in turn has geometry bounded by $K_\tau:=\max\left \{\sqrt{K},\frac{1}{\tau}\right\}$. Denote by $\tilde{u}\subset S\times M$ the graph of $u$. Consider the lift $\tilde{\gamma}$ of $\gamma$. 
For any integer $N$ let $r_N=\frac{d}{2N}$.  Then we claim that for each $\tau\geq \|X_H\|$  there are at least $\lfloor N/\tau\rfloor$ points $s_i\in[0,1]$ such that $d_\tau(\tilde{u}(\gamma(s_i)), \tilde{u}(\gamma(s_{j})))\geq2\tau r_N$ whenever $i\neq j$. For this, it suffices to show that the projection map $\pi_2: S\times M\to M$ does not increase norms. Referring to the formula \eqref{eqGromovMetricFormula}, the orthogonal complement to the fibers of $\pi$ can be decomposed as $\bR(X_H+\frac{\partial}{\partial t})\oplus (\bR X_H)^{\perp}$ where the second summand is the orthogonal complement of $X_H$ with respect to $g_J$ in $TM$. On this summand the projection preserves norms. On the first summand the vector $X_H+\frac{\partial}{\partial t}$ which has norm at least $\tau$ gets mapped to the vector $X_H$ which has norm at most $\tau$ by assumption.   
Let $B_i=B_{r_{N}}(\tilde{u}(\gamma(s_i)))$ where we consider balls with respect to $\tilde{g}_{
\tau}$. Suppose now that $N$ is so that $r_N\ll \delta$ and $N>1$. The monotonicity inequality of Lemma \ref{lmMonEst++} then gives 
\begin{equation}
\sum_i \tilde{\omega}_\tau(B_i) \geq \frac1{C^3_\tau} \lfloor N/\tau\rfloor \tau^2 r_N^2>\frac1{4C^3_\tau }\tau r_N{d},
\end{equation}
where $C$ is a constant that dominates $C_{\tau}$ on $(0,1]$.
On the other hand
\begin{equation}
\tilde{\omega}_\tau(B_i)= E^{geo}(u;u^{-1}(\pi_1(B_i)))+Area_\tau(\pi_1(B_i)).
\end{equation}
So,
\begin{equation}
\sum_i \tilde{\omega}_\tau(B_i) \leq E^{geo}(u)+2\tau\ell_\tau=E^{geo}(u)+2\tau^2\ell.
\end{equation}

Combining these, we obtain
\begin{equation}
d\leq  \frac{C_\tau^3}{r_N}\left(\frac1{\tau}E^{geo}(u)+2\tau\ell\right).
\end{equation}
Taking $r_N\sim\frac15\delta$ gives the claim. 
\end{proof}



For the following proposition fix an S-shaped Hamiltonian $H$ and a geometrically bounded almost complex structure $J$. 
\begin{prp}\label{prpRelEn}
If $H|_{\bR\times S^1\times M\setminus V}$ is sufficiently $C^1$ small there is an $\hbar>0$ depending only on the geometry of $J$ such that if $E^{rel}(u)\neq 0$ then $E^{geo}(u)>\hbar$. The same holds if the hypothesis is true once we replace $H$ by $H-f$ where $f:\bR\times S^1\to\bR$ is  any smooth function.
\end{prp}

\begin{rem}
The final clause is meant to make the Proposition applicable to continuation maps. The typical situation is for a continuation map from $H_0$ to $H_1$ the difference $H_1-H_0$ has to be allowed to be arbitrarily large, but the the oscillation of each of the $H_i$ on each component of $M\setminus V$ can be kept arbitrarily small. 
\end{rem}
\begin{proof}
Fix a symplectization neighborhood $V$ of $\partial K$ witnessing $H$ being of contact type. The fact that $E^{rel}(u)\neq 0$ implies that there is a component $v$ of $u\setminus u^{-1}(V)$ so that $E^{rel}(v)\neq 0$.  Let $\epsilon>0$ be dominated by the injectivity radius of $M$ as well as by the distance from $\partial K$ to its cut locus. Consider the coordinate $s$ restricted to $v$. Then there is a value $s_0$ such that a connected component of $u(s_0,\cdot)\cap v$ has diameter greater than $\epsilon$. Indeed, otherwise $v$ is contractible relative to $V$.  

Applying Proposition \ref {prpSmallDiamEst}, taking $d_0=\epsilon/2$ we and assuming without loss of generality that $\epsilon/2\leq \frac1{\sqrt{K}}$ we obtain
\begin{equation}
\epsilon<\frac2{\epsilon\tau}h\left(\frac{\left|\partial_sH\right|_{\infty}}{\tau^2},\frac{\|X_H\|^2_{\infty}}{\tau^2}\right)\left(E^{geo}(u)+\tau^2\ell(\gamma)\right),
\end{equation}
for any $\tau$ such that $\|X_H\|<\tau.$
Taking  $H$ even smaller in $C^1$ so that $h\left(\frac{\left|\partial_sH\right|_{\infty}}{\tau^2},\frac{\|X_H\|^2_{\infty}}{\tau^2}\right)\leq 2$ we obtain the estimate
\begin{equation}
    E^{geo}(u)>\epsilon^2\delta\tau/4-2\tau^2\pi.
\end{equation}
Fixing $\tau=\frac{\epsilon^2\delta}{16\pi}$
we obtain
\begin{equation}
E^{geo}(u)>\frac{\epsilon^2\delta}{16\pi}. 
\end{equation} 
For the final clause, note that $H$ determines the same equation as $H-f$ since Floer's equation only involves derivatives of $H$ in directions tangent to $M$.
\end{proof}

\subsection{Proof of Proposition \ref{prphbarfilt}}
\begin{lm}\label{lmBorgOanc}
Any monotone Floer trajectory involving Hamiltonians as in the hypothesis of Lemma \ref{prpRelEn} whose input is concave and whose output is convex has energy $>\hbar$. 
\end{lm}
\begin{proof}
By the conclusion of Lemma \ref{prpRelEn}, Floer trajectories of energy $\leq\hbar$ remain inside a symplectization neighborhood. Within such a neighborhood, the claim is \cite[Lemma 2.3]{CieliebakOancea} with suitable adjustment for sign conventions. Namely, our $X_H$ is the negative of theirs.
\end{proof}

\begin{proof}[Proof of Proposition \ref{prphbarfilt}]
Hamiltonians of the form $H_{K,I,\eta}$  satisfy the hypothesis of Lemma \ref{prpRelEn}. This is also true of we consider continuation maps fixing $K,I$ and varying $\eta$. When varying also $I$ we satisfy the hypothesis of the last clause of Proposition \ref{prphbarfilt} on the outside componet of $M\setminus V$.   The first part of the claim is now immediate from  Lemma \ref{prpRelEn}. The second part follows from this and Lemma \ref{lmBorgOanc}. 
\end{proof}
\subsection{Truncated symplectic cohomology and the unweighted Floer homology of Reeb orbits}  


For a Reeb component $\gamma$ of $K$ denote by $\delta(\gamma)$ the minimum of $\hbar$ and gap in the period spectrum at $\gamma$. Observe that while the action spectrum is dependent on the choice of primitive $\alpha$, the gaps in the period spectrum for the orbits in a fixed homology class of $\partial K$ are independent of the choice of primitive. Moreover, Proposition \ref{prpRelEn} implies that at least in windows of size $\leq\hbar$ the Floer cohomology is split into components associated with distinct classes of $H_1(\partial K;\bZ)$. 

\begin{prp}\label{prpInInfHKI}
For any Reeb component $\gamma$ and any choice of parameters $I$, the generators associated with $\gamma$ in the quotient complex underlying $In^*_{[0,\delta(\gamma))}(H_{K,I})$ form a summand. Denoting this summand by $In^*_{[0,\delta(\gamma))}(\gamma; H_{K,I})$ we have that for any $I'\geq I$  the summand  $In^*_{[0,\delta(\gamma))}(\gamma; H_{K,I})$ maps into  $In^*_{[0,\delta(\gamma))}(\gamma; H_{K,I'})$ under the natural continuation maps.
\end{prp}
\begin{proof}
By Proposition \ref{prphbarfilt}, for action intervals less than $\hbar$ the action filtration the weight of a Floer trajectory is given by the topological energy with respect to a fixed primitive (the choice is immaterial). So any Floer trajectory connecting $\gamma,\gamma'$ with $T(\gamma;\alpha)\neq T(\gamma';\alpha)$ must have energy $\geq\hbar$. Since our Hamiltonians may have degenerate orbits we need to rule out the possibility of trajectories connecting generators associated with distinct Reeb components $\gamma,\gamma'$ satisfying $T(\gamma)=T(\gamma')$. It is an immediate consequence of Proposition \ref{prpActionWindowComponent} that there are no such trajectories once the smoothing parameter  $\eta$ is made small enough. 
\end{proof}
\begin{prp}
For any choice of parameters $I$ for $K$ there is an isomorphism $In^*_{[0,\delta(\gamma))}(\gamma;H_{K,I})\to SH^*_{uw}(\gamma)\otimes \Lambda_{[0,\delta(\gamma))}$. This isomorphism is natural with respect to monotone changes in the parameters. Namely,  if $I\leq I'$ there is a commutative diagram
\begin{equation}
\xymatrix{In^*_{[0,\delta(\gamma))}(\gamma;H_{K,I})\ar[d]\ar[r]& SH^*_{uw}(\gamma)\otimes \Lambda_{[0,\delta(\gamma))}\ar[d]\\
In^*_{[0,\delta(\gamma))}(\gamma;H_{K,I'})\ar[r]& SH^*_{uw}(\gamma)\otimes \Lambda_{[0,\delta(\gamma))}}
\end{equation}
where on the vertical left arrow is the continuation map and on the right it is the identity. 
\end{prp}
\begin{proof}
The proof is nearly the same as that of Lemma \ref{lmGermHF}. We outline it briefly. We fix an isolating neighbourhood $U$ of $\gamma$, a small constant $\delta_0\ll\epsilon_I$ and consider the neighbourhood $V_I$ given in symplectization coordinates corresponding to the  primitive $\alpha_I$ as $V_I=U\times (0,\delta_0)$.  We consider a decreasing sequence $\eta_i\to 0$ and $H_i$ of smooth non-degenerate Hamiltonians close enough to some $H_{K,I,\eta_i}\in\cK(K,I,\eta_i)$ so that all Floer trajectories for the differential and continuation map contributing to $In^*_{[0,\delta)}(\gamma)$ have energy so small as to be guaranteed to be contained within $V_I$.   This can be done according to Proposition \ref{prpActionWindowComponent}. 

As in the proof of Lemma \ref{lmGermHF} we now construct a map from the underlying complex $CF^*_{In,[0,\epsilon)}(\gamma; H_{K,I})$ to  $CF^*_{uw}(\gamma;H_{K,I})\otimes \Lambda_{\epsilon}$.  Namely we model both the truncated and the weighted complexes of $H_{K,I}$ as colimits over $i$ of suitable chain complexes associated with $H_i$.  We then define the map between them as in the paragraph  preceding eq \eqref{LclFHDiag}. The proof that this map is well defined and induces an isomorphism in homology is word for word the same as in the proof of Lemma \ref{lmGermHF}. In the same way the lower horizontal map is an isomorphism. 

To conclude, we need to show that the diagram commutes. Here we need to briefly comment since the neighborhood $V_I$ and $V_{I'}$ aren't necessarily the same when they encode different primitives. However, if we   take $\delta$ small enough, $V_I$ is still an isolating neighborhood of the incarnation of $\gamma$  for the functions approximating $H_{K,I'}$. All the relevant Floer trajectories are contained in this neighborhood, and again we have naturality in the same way as in the proof of Lemma \ref{lmGermHF}.  

Since all the other arrows in the commutative diagram are isomorphisms, it follows that the left vertical arrow is also an isomorphism. 
\end{proof}

\begin{prp}\label{prpTruncIsotop}
More generally, consider an isotopy of domains $\tau\mapsto K^{\tau}$ and an isolated family of Reeb components $\gamma^{\tau}$. Let $\delta=\inf_\tau\delta(\gamma^{\tau})$. Let $I, I'$ be choices of parameters so that $H_{K^0,I}\leq H_{K^1,I}$. Then we have a commutative diagram
\begin{equation}
\xymatrix{In^*_{[0,\delta)}(H_{K^0,I})\ar[d]\ar[r]& SH^*_{uw}(\gamma^0)\otimes \Lambda_{[0,\delta(\gamma))}\ar[d]_{T^{\Delta\cA}\iota}\\
In^*_{[0,\delta)}(H_{K^1,I})\ar[r]& SH^*_{uw}(\gamma^1)\otimes \Lambda_{[0,\delta(\gamma))}}
\end{equation}
where $\iota$ is the map from Lemma \ref{lmIsotopUW}   and $\Delta\cA$ is the action difference.
\end{prp}
\begin{proof}
The map on the right hand side is constructed in the proof of Lemma \ref{lmIsotopUW} and  involves an adiabatic subdivision of the interval parametrizing the isotopy.  For each of the small intervals the commutation of the diagram is obtained by the same argument as in Lemma \ref{lmIstopyInfMap}. 
\end{proof}

An analogous argument for the outer orbits leads to 

 \begin{prp}\label{prpOutInfHKI}
For any $I$ there is an $I'>I$ such that $Out^*_{[0,\delta(I))}(H_{K,I})$ maps to $0$ under the continuation map to $Out^*_{[0,\delta(I))}(H_{K,I'})$.  
\end{prp}
\begin{proof}
For $I=(\alpha,a,\epsilon)$ consider $I'=(\alpha,b,\epsilon)$. That is, vary only the slope. 
In action windows of size $<\delta(a)$ and for $(a''-a')\epsilon< \delta(a)$ the continuation map 
$Out^*_{[0,\delta(I))}(H_{K,I'})\to Out^*_{[0,\delta(I))}(H_{K,I''})$ 
is given by rescaling by the action difference by the same argument as \ref{lmIstopyInfMap}. This action difference is given  by rescaling  by roughly $T^{(a''-a')\epsilon}$. The map from $a$ to arbitrary $b$ is given by composition of $(b-a)/\epsilon$ such rescalings. The claim follows. 
\end{proof}


 
\section{Proof of the main Theorems}\label{SecProofs}
\subsection{Proof of Theorems \ref{mainThmA} and \ref{mainThmB}}
\begin{tm}\label{tmFilteredMainTm}
$$F^aSH^*_{M,[0,\epsilon)}(K)=\oplus_{\gamma:\cA(\gamma)<a}SH^*_{uw}(\gamma)\otimes\Lambda_{[0,\delta(a))}.$$ 
\end{tm}
\begin{proof}
First observe
\begin{equation}
SH^*_{[0,\epsilon),M}(K)=\varinjlim_I HF^*_{[0,\epsilon)}(H_{K,I}). 
\end{equation}
Now, from Proposition \ref{prpInInfHKI} we deduce that for any $I$ such that $slope(I)>a$ we have
\begin{equation}\label{eqFiltSymbMap}
F^aIn^*_{[0,\epsilon)}(H_{K,I})=\oplus_{\gamma:\cA(\gamma)<a}SH^*_{uw}(\gamma)\otimes\Lambda_{[0,\delta(a))}.
\end{equation}
And moreover, this is natural with respect to continuation maps from $H_{K,I}$ to $H_{K,I'}$. From this and proposition \ref{prpOutInfHKI} we deduce that the boundary homomorphism in the long exact sequence \eqref{eqInOutLES} vanishes. Moreover, by proposition \ref{prpOutInfHKI} the continuation map from $H_{K,I}\to H_K$ factors through the quotient map to $F^aIn^*_{[0,\delta(a))}(H_{K,I})$. Finally, the induced map from 
$ F^aIn^*_{[0,\delta(a))}(H_{K,I})$ is an isomorphism by Proposition \ref{prpInInfHKI}.


\end{proof}





We now discuss functoriality.




\begin{df}
 We say that an inclusion $K_1\subset K_2$ is \emph{admissible} if  
\begin{enumerate}
\item $\partial K_1\cap\partial K_2$ is codimension $0$ sub-manifold with boundary of $\partial K_2$.
\item each Morse-Bott component of either $\partial K_1$ or $\partial K_2$ is either contained in $\partial K_1\cap\partial K_2$ or is disjoint of  $\partial K_1\cap\partial K_2$, 
\item there is a contact type primitive of $\omega$ near $\partial K_2$ which extends to a primitive (not necessarily of contact type) of $\omega$ on a neighborhood of $\partial K_1$.
\end{enumerate}
\end{df}
\begin{ex}
A basic example is when $M$ is the affine variety $\bC^2\setminus \{xy-1\}$ equipped with the symplectic form $Im\left(\frac{dx\wedge dy}{xy-1}\right)$. Then $M$ carries a Lagrangian torus fibration $\pi:M\to\bR^2$ with a nodal singular fiber over the origin. Let $B\simeq\bR^2$ be the integral affine manifold with singularities coming from Arnold-Liouville. Let $P_2\subset B$ be a convex polygon containing the singular value, and let $P_1\subset P$ be a convex polygon not intersecting the monodromy invariant line. See Figure \ref{FigAdmInc}. Take $K_i=\pi^{-1} (P_i).$ Then each of $K_i$ is a Liouville domain, but it can be shown that $K_1$ is not a Liouville subdomain of $K_2$ with respect to any contact form form on $K_1$. 
\end{ex}
% Figure environment removed

Suppose we have an admissible inclusion $\iota: K_1\subset K_2$. Let $\gamma$ be a Reeb component of $K_2$. If $\gamma$ is in the interior of the intersection $\iota(\partial K_1)\cap K_2$ then there is a natural isomorphism $\iota_*:SH^*_{uw}(\iota^{-1}(\gamma))\simeq SH^*_{uw}(\gamma)$ induced by the symplectomorphism of the germs.  If $\gamma$ is a component in the boundary of $\partial K_1\cap\partial K_2$ we still have  an isomorphism $\iota_*$ by considering an isotopy  $\tau\mapsto K^{\tau}_1$ such that for $K^0_1$ the component $\gamma$ is still in the interior, and such that $\partial K_1\cap\partial K_2\subset \partial K^{\tau}_1\cap\partial K_2$ for all $\tau$. It is straightforward to construct such an isotopy. In the following we identify $\gamma$ with $\iota(\gamma)$ and call $\gamma$ a \emph{common Reeb component}.  All other Reeb components of $K_2$ are called \emph{non-common Reeb components.}

In the following, for an admissible inclusion $K_1\subset K_2$ denote by $\delta_{12}(a)$ the minimal gap in $\left(Spec(\partial K_1)\cup Spec(\partial K_2)\right)\cap[0,a]$ where the spectrum is taken with respect to a primitive which is defined on $\partial K_1\cup\partial K_2$. We also write $F^aSH^*_{M,[0,\delta_{12}(a))}(K_i)$ for the induced filtration with respect to this primitive. Note this makes sense in small action windows also for primitives which are not necessarily of contact type. 

\begin{prp}\label{tmInfintsmlSHFunct} 
Under the isomorphism \eqref{eqFiltSymbMap}, given an admissible embedding $K_1\subset K_2$, the induced map  $F^aSH^*_{M,[0,\delta_{12}(a))}(K_2)\to  F^aSH^*_{M,[0,\delta_{12}(a))}(K_1)$ is given by the direct sum of the projection to the summands corresponding to the common orbits with the restriction map $H^*(K_1)\to H^*(K_2)$.
\end{prp}
\begin{proof}
By Theorem \ref{tmFilteredMainTm} the canonical map $ F^aSH^*_{M,[0,\delta_{12}(a))}(K_2)\to  F^aSH^*_{M,[0,\delta_{12}(a))}(K_1)$ induces a collection of canonical maps $r_{\gamma\gamma'}:SH^*_{uw}(\gamma)\to SH^*_{uw}(\gamma')$ where $\gamma$ runs over all Reeb components of $K_2$ and $\gamma'$ runs over all Reeb components of $K_1$.

By Proposition \ref{prpTruncIsotop} it suffices to study the corresponding claim for $In_{[0,\delta_{12}(a))}(H_{K_1,I_1})\to In_{[0,\delta_{12}(a))}(H_{K_2,I_2})$. We claim that for $\gamma$ a common component we have
\begin{equation}
r_{\gamma\gamma'}=\begin{cases}\iota_*,&\quad\gamma'=\iota(\gamma)\\
							0,&\quad \gamma'\neq\iota(\gamma).\end{cases}
\end{equation}
Indeed, if $\gamma'\neq\gamma$ this follows by the argument of Proposition \ref{prpInInfHKI}.  For $\gamma=\gamma'$ the Floer trajectories contributing to the continuation map are  the ones contributing to the unweighted complex which, i.e., the map $\iota_*$.

We now deal with the case that $\gamma$ is a non-common Reeb component. Let $\gamma'$ be Reeb component of $K_2$. If $\gamma'$ is trivial then any continuation Floer trajectory connecting $\gamma$ to $\gamma'$ can be considered as a disc with boundary in $K_1$. If this disc cohomologous to $0$ in homology rel $\partial K_1$, the topological energy of $u$ is arbitrarily close to $T(\gamma)>0$. Otherwise, $E(u)\geq \hbar$ by Lemma \ref{prpRelEn}. In any case we see the component $r_{\gamma\gamma'}$ vanishes.  Assume now $\gamma'$ is non-trivial. Let $\alpha$ be a contact type primitive of $\omega$ near $\partial K_1$ underlying the acceleration datum of $K_1$. By assumption it can be extended to a primitive along a neighborhood of $\partial K_2$. This primitive is not necessarily of contact type on $\partial K_2$. However for periodic orbits in a given homology class of $\partial K_1\cup\partial K_2$ the period spectrum with respect to this $\alpha$ is discrete by Morse-Bott non-degeneracy for each of the hyper-surfaces $\partial K_1,\partial K_2$. Now if $u$ is a continuation trajectory whose energy is less than $\hbar$ of Lemma  \ref{prpRelEn} the $u$ is cohomologous to $0$ rel $\partial K_1\cup\partial K_2$. In this case the topological energy is given by the action difference defined via $\alpha$. This can be taken arbitrarily close to the period difference $\Delta(T)=T(\gamma)-T(\gamma')$. By discreteness there is a  $\delta>0$  so that the $\delta$-interval around $\gamma$ contains no $\alpha$-periods of $\partial K_2$ other than possibly $T(\gamma)$ itself (in case that is a period). So, either $\Delta(T)>\delta$ are $\Delta(T)=0$. In the first case it is clear that the map $r_{\gamma\gamma'}$ vanishes. In the second case using the fact that $\gamma\neq \gamma'$ we can by adjusting the perturbation data and the functions $H_1,H_2$  make the action difference on these orbits precisely zero ruling out any continuation trajectory.  

Finally, we analyze the map on $H^*(K_2)$. Given a critical point of $K_2$ there cannot be any continuation trajectory $u$ with output on a non trivial Reeb trajectory $\gamma$ so that $u$ is cohomologous to $0$ rel $\partial K_2$. Indeed, the topological energy of such a trajectory can be taken to be arbitrarily close to $-T(\gamma)$. Thus every such trajectory has energy at least $\hbar$. So, in infinitesimal Floer cohomology, only the Morse trajectories contribute. It is standard that these compute the restriction map in singular cohomology.  


\end{proof}
\begin{proof}[Proof of Theorems \ref{mainThmA} and \ref{mainThmB}]
Under the gappedness assumption, the $\delta$ in Theorem \ref{tmFilteredMainTm} can be taken independent of $a$. Thus 
$$
SH^*_{[t,t+\delta),M}(K)=\varinjlim_aF^aSH^*_{[t,t+\delta),M}(K)=RH^*(K)\otimes \Lambda_{[0,\delta)}.
$$
But, setting $t=-\delta$, the left hand side is precisely the first page in the spectral sequence in the formulation of Theorem  \ref{mainThmA}. Theorem \ref{mainThmB} now follows immediately by Theorem \ref{tmInfintsmlSHFunct}.


\end{proof}
\subsection{The non-smooth case}\label{SecNSCase}
The proof of Theorem \ref{tmInfSHn2} requires a version of Theorem \ref{mainThmC} for certain domains with corners. To formulate it when the boundary of $K$ is non-smooth  we need to first have a definition of $RH^*(K)$. 

\begin{df}\label{dfAdmSm}
Let $K\subset M$ be a domain such that in a neighborhood $N(\partial K)$ the symplectic form is exact. An \emph{admissible smoothing} of $K$ is a family $\tau\mapsto K^\tau$ such  and a choice $\alpha$ of a primitive of $\omega$ in a connected open neighborhood of $\partial K$ containing $\partial K^{\tau}$ for all $\tau\in[0,1]$  such that the following hold
\begin{enumerate}
\item $K=K^0\subset K^{\tau'}\subset K^{\tau}$ whenever $\tau'<\tau$.
\item For each $\tau>0$, the restriction $\alpha|_{\partial K^{\tau}}$ is a contact form. 
\item 
There is a countable set $\cR$ such that Reeb components decompose into families $\gamma^{\tau}$ for $\gamma\in\cR$ such that $\gamma^{\tau}$ is a well isolated Reeb component of $\partial K^{\tau}$ and the isotopy $\tau\mapsto \gamma^{\tau}_i$ is isolated in the sense of Definition \ref{dfIsolatingIsotopy}.

%The set of primitive periodic Reeb components  decomposes for each $\tau$ into a disjoint union of path connected isotropic components.
%\item For each $T$ the set of Reeb components of period $\leq T$ is \emph{isolated uniformly in $\tau$}. This means  there is a continuous  isotopy $\tau\mapsto U^{T,\tau}$ of embedded domains decomposing into a disjoint union of isotopies of connected components  $U^{T,\tau}(\gamma^{\tau})$ with the property that for each $\tau$ the interior of $U^{T,\tau}(\gamma^{\tau})$  is a well isolating neighborhood of a unique primitive connected Reeb  component $\gamma^{\tau}$ and moreover, each Reeb component of period $\leq T$ is contained in  $U^{T,\tau}$. 
\end{enumerate} 

Define for $\gamma\in\cR$ 
\begin{equation}
    SH^*(\gamma)=\varinjlim_{\tau\to 0}SH^*(\gamma^{\tau})
\end{equation}
where the direct limit is over the isomorphisms of Lemma \ref{lmIsotopUW} for $\tau>0$. 
\end{df}



%For $K$ with such an admissible smoothing denote by $\cR$ a set parametrizing the Reeb components, and for each $\gamma\in\cR$ denote by $\gamma^{\tau}$ the corresponding Reeb component of  $\partial K^{\tau}$.

In the setting of Definition \ref{dfAdmSm}, 
for each $a$ let $\delta(a,\tau)$ be the minimal non-zero gap in the period spectrum of $\partial K^{\tau}$ up to action value $a$. For fixed $a$ the function $\delta(a,\tau)$ is continuous as a function of $\tau$. 


%Suppose now we have smoothing $\tau\mapsto K^{\tau}$ of $K=K^0$ where for each closed subinterval $I\subset[0,1)$ the restriction of the family to $\tau\in I$  is as in Definition \ref{dfAdmSm}.   We then denote for $\gamma\in\cR$ the direct limit over the isomorphisms of Lemma \ref{lmIsotopUW} for $\tau>0$ by $SH^*_{uw}(\gamma)$. 
 



% now conclude 
\begin{tm}\label{tmInfintsmlSHns}
Suppose the limiting periods are all distinct and let $\delta(a)=\inf_\tau\delta(a,\tau)>0$. Then
\begin{equation}
F^aSH^*_{[0,\delta(a))}(K)=\left(H^*(K)\oplus_{\gamma\in\cR}SH^*_{uw}(\gamma)\right)\otimes\Lambda_{[0,\delta)}.
\end{equation}
 
\end{tm}
\begin{rem}
The assumption of distinct limiting periods is imposed on us because our current basic approach is to build on Theorems \ref{tmFilteredMainTm} and \ref{tmInfintsmlSHFunct}. A more complete approach, which would suffer from less limitations, is to prove versions of these theorems from the start for appropriate domains with corners. This will be pursued elsewhere. 

\end{rem}
\begin{proof}[Proof of Theorem \ref{tmInfintsmlSHns}]
We first observe that since colimits commute with colimits we have for any $\epsilon>0$ 
\begin{equation}
SH^*_{M,[0,\epsilon)}(K^0)=\varinjlim_{\tau\to0}SH^*_{M,[0,\epsilon)}(K^{\tau}).
\end{equation}
More crucially, for each $a$, we have 
\begin{equation}
F^aSH^*_{M,[0,\epsilon)}(K^0)=\varinjlim_{\tau\to0}F^aSH^*_{M,[0,\epsilon)}(K^{\tau}).
\end{equation}

The claim now follow from Theorem \ref{tmFilteredMainTm} and Proposition \ref{prpTruncIsotop}.


\end{proof}


To discuss functoriality we need the notion of compatible smoothings. 
\begin{df}\label{dfCompSmooth}
Given domains $K_1\subset K_2\subset M$ as in Definition \ref{dfAdmSm} we say that a pair $\{K_1^{\tau}\}\subset \{K_2^{\tau}\}$ of smoothings is \emph{compatible} if for each $\tau$  the inclusion $K_1^{\tau}\subset K_2^{\tau}$ is admissible, and for each action $a$ with respect to some fixed primitive, the set of common components of action $\leq a$ stabilizes as $\tau\to\infty$. We call the intersection over $\tau$ of the common components for $K_1^{\tau}\subset K_2^{\tau}$ \emph{the common components} for $K_1\subset K_2$. 
\end{df}
\begin{ex}\label{exComp}
Consider $K_1,K_2$ be the pre-images of respectively  a rectangle and a square under the moment map $T^*\bT^2=\bR^2\times\bT^2\to\bR^2$. Figure \ref{FigComp} depicts an example of a compatible and incompatible smoothing. In this case there are choices of smoothings which are compatible. In Figure \ref{FigComp2} we have a convex polygon inside an $L$-shaped polygon. It can be shown that no compatible smoothings exist in this case. 
\end{ex}


% Figure environment removed



% Figure environment removed

  









\begin{rem}\label{remRestrictLift}
The framework set up here is rather restrictive in two ways. Firstly we impose the assumption that the limiting periods are all distinct. Second, our definition of admissibility does not allow a Reeb component  of $K_2$ which is a proper subset of a component of $K_1$. Both restrictions can be removed. This will be done in forthcoming work. \end{rem}

\begin{tm}\label{tmInfintsmlSHnsFunct}
Given a pair of compatible inclusions the restriction map in infinitesimal Floer cohomology is given by projection to the common components. 
\end{tm}
\begin{proof}
As in the proof of Theorem \ref{tmInfintsmlSHns}, it suffices to prove that for any $a$ and for $\epsilon>0$ small enough the corresponding claim holds for $F^aSH^*_{[0,\epsilon)}(K_0)$. Consider the commutative diagram

$$
\xymatrix{ F^aSH^*_{[0,\epsilon)}(K_2^\tau)\ar[d]\ar[r]&F^aSH^*_{[0,\epsilon)}(K_1^\tau))\ar[d]\\
F^aSH^*_{[0,\epsilon)}(K_2^0)\ar[r]&F^aSH^*_{[0,\epsilon)}(K_1^0)}
$$
 

Given  Theorem \ref{tmInfintsmlSHFunct}  and the identifications of Theorems \ref{tmInfintsmlSHns}, the horizontal maps are direct sums of the maps $r^0_{\gamma\gamma'}$ and $r^{\tau}_{\gamma\gamma'}$ respectively. Moreover, $r^0_{\gamma\gamma'}$ is the colimit over $\tau$ of the $r^{\tau}_{\gamma\gamma'}$. For $\tau$ large enough we hav that $r^{\tau}_{\gamma\gamma'}=0$ if $\gamma'\neq\gamma$. So $r^0_{\gamma\gamma'}=0$ in this case. Similarly $r^0_{\gamma\gamma'}$ is the identity if $\gamma=\gamma'$.
\end{proof}
\subsection{Proof of Theorem \ref{mainThmC}}

\begin{df}
For $K\subset M$ a compact set define the \emph{infinitesimal Floer cohomology} by
\begin{equation}
SH^*_{M,t^+}(K):=\varprojlim_{a\to t^+} SH^*_{M,[t,a)}(K).
\end{equation}
\end{df}
\begin{rem}
A similar definition makes sense for any lower semi-continuous Hamiltonian.
\end{rem}

Note that $SH^*_{M,t^+}(K)$ is an $R$-module.  Moreover,  since the Floer complex is a free normed Novikov field module with an orthonormal basis  we have a canonical isomorphism $SH^*_{M,t^+}(K)=SH^*_{M,0^+}(K)$ for all real $t$.  The isomorphism is induced by scaling by $T^{-t}$. 



\begin{proof}[Proof of Theorem \ref{mainThmC}]
Given Theorem \ref{tmFilteredMainTm} we conclude
\begin{equation}
\varprojlim_{\epsilon\to 0}F^aSH^*_{[0,\epsilon)}(K_0)=H^*(K)\oplus_{\gamma\in\cR(\partial K):T(\gamma)\leq a}SH^*_{uw}(\gamma).
\end{equation}

To conclude, it remains to prove 
\begin{equation}
SH^*_{0^+,M}(K_0)=\varinjlim_a \varprojlim_{\epsilon\to 0}F^aSH^*_{[0,\epsilon)}(K_0).
\end{equation}
This requires an argument for commutation of limit and colimit. That is, we wish to show that the natural map
\begin{equation}
\varinjlim_a\varprojlim_{\epsilon}F^aSH^*_{[0,\epsilon)}(K_0)\to\varprojlim_{\epsilon}\varinjlim_aF^aSH^*_{[0,\epsilon)}(K_0)=SH^*_{0^+,M}(K_0)
\end{equation}
is an isomorphism.

For this we observe that for any fixed $a$ and $\epsilon<\delta(a)$ we actually have a \emph{splitting} $SH^*_{[0,\epsilon)}=F^aSH^*_{[0,\epsilon)}\oplus R(a,\epsilon)$ where $R(a,\epsilon)$ is some "remainder" term. This is a consequence of the fact that below $\hbar$ all Floer theoretic interactions are weighted by action differences. In particular, for any $a$ we have a splitting
\begin{equation}
SH^*_{0^+,M}(K_0)=\varprojlim_{\epsilon\to 0}(F^aSH^*_{[0,\epsilon)})\oplus \varprojlim_{\epsilon\to 0}R(a,\epsilon).
\end{equation}
In particular, for any $a$ the map $\varprojlim_{\epsilon}F^aSH^*_{[0,\epsilon)}\to SH^*_{0^+,M}(K_0)$ is injective and therefore so is the map from $\varinjlim_a\varprojlim_{\epsilon}F^aSH^*_{[0,\epsilon)}(K_0)$. For surjectivity note that all the maps in the inverse limit respect the action filtration for $\epsilon<\hbar$. Thus the induced filtration by of  $SH^*_{0^+,M}(K_0)$ by the $\varprojlim_{\epsilon\to 0}(F^aSH^*_{[0,\epsilon)})$ is exhaustive. This proves the conclusion of Theorem first half of the present Theorem.

Naturality of the isomorphism is immediate by naturality in Theorem \ref{tmInfintsmlSHFunct}.

\end{proof}

We state the Theorem for the non-smooth case.
\begin{tm}
Theorem \ref{mainThmC} holds if one considers $K$ a domain with an admissible smoothing. Naturality holds for inclusions $K_1\subset K_2$ of domains with compatible admissible smoothings. 
\end{tm}
\begin{proof}
Given Theorems \ref{tmInfintsmlSHns} and  \ref{tmInfintsmlSHnsFunct}, the proof is the same as that of Theorem \ref{mainThmC}.
\end{proof}
\section{Relative $SH$ near the singularity of an SYZ fibration}\label{SecSYZ}
\subsection{Infinitesimal $SH$ near the singularity of an SYZ fibration}
\begin{lm}
Let $\partial K$ carry a Lagrangian torus fibration, and let the Reeb orbit $\gamma$ be a torus fiber. Assume $TM|_{\gamma}$ is trivialized so that the Robin Salamon index of $\gamma$ is $0$. Then $SH^*_0(\gamma)=H^*(T^*\bT^n;R)$. In other words, the local system of Theorem \ref{tmMorseBottCascades} is trivial. 
\end{lm}
\begin{proof}
In dimension $1$ this is \cite{BourgeoisOancea2009}. Namely, $\gamma$ is a good orbit and so the local system has trivial monodromy. In higher dimension this follows by the Kunneth formula. 
\end{proof}

\subsubsection{The 4-dimensional case}\label{Sec4DimSYZ}
Let $M$ be a symplectic $4$-manifold which is closed or geometrically bounded and satisfies $c_1(M)=0$. Let $B$ be a smooth surface and let $\pi:M\to B$ be proper. Let $B_{reg}\subset B$ be an open subset so that $\pi$ is a proper Lagrangian submersion over $B_{reg}$. In particular, the fibers over $B_{reg}$ are Lagrangian tori. We assume these tori are Maslov $0$.  

We say that a subset $P\subset B$ is \emph{admissible} if
\begin{enumerate}
\item $\partial P\subset B_{reg}$. 
\item $\partial P$ is an integral affine convex polygon which is Delzant smooth. This means, first, $\partial P$ is the union of a finite number of affine segments with respect to the Arnold Liouville structure. Second,  orienting  $\partial P$ counter clockwise, at each vertex the tangent vector to the outgoing segment is to the left of the  line through the incoming one. Finally, we require that the primitive covectors annihilating adjacent edges form a $\bZ$-basis for the covectors at the vertex. 
\item The integral affine structure is conical near $\partial P$. This means there is an atlas of  action angle coordinates $\{p,\theta\}$ such that the Euler vector field $\sum p_i\frac{\partial}{\partial P_i}$ is preserved by the transition maps and points outwards of $P$. 
\item To each edge $\partial_iP$ associate the action $\cA_i$ which is the value of the primitive $\alpha=\sum_ip_id\theta_i$ on the primitive outward covector defining $\partial P_i$ where $\alpha$ is evaluated at any point on $\partial P_i$. Then the $\cA_i$ are mutually irrational.

\end{enumerate}

\begin{ex}
$\cR\subset\bR^2$ be an eigenray diagram. Let $B_{\cR}$ be the corresponding integral affine manifold with singularities. Let $P\subset B_{reg}$ be convex and Delzant, and suppose that the eigenlines through singular values inside $P$ all meet at a point in $P$ then $P$ is admissible. For a reference see \cite[\S7]{GromanVarolgunes2021}. 
\end{ex}
Let $P_1\subset P_2$ be admissible. We say that the inclusion is \emph{admissible} if $\partial P_1\cap\partial P_2$ is a  codimension $0$ subset whose boundary points are interior points of the $1$ dimensional strata of $\partial P_2$. See Figure \ref{FigAdm}. It is easy to see this property is  transitive. Note that each connected component of $\partial P_1\cap\partial P_2$  is a union of successive edges of $\partial P_1$. 
% Figure environment removed

We define a category $\cP$ of compact sets $P\subset B$ with morphisms admissible inclusions. We also consider the category $PMon$ of \emph{partial monoids} over $\bZ$. We define a contravariant functor $Trop: \cP\to PMon$  as follows. 


To each edge $e$ associate the monoid $C_e$ consisting of the integral covectors on $P$ which are constant on $e$ and are non-negative on the outside of $P$. To each vertex $e_i\cap e_{i+1}$ associate the cone $C_{i,i+1}$ generated by the primitive generators of $C_{e_i}$ and $C_{e_{i+1}}$. Define the partial monoid 
\begin{equation}
P_{trop}:=\cup_iC_{i,i+1},
\end{equation}
where we identify the generator of $C_{e_i}\subset  C_{i-1,i}$ with $C_{e_i}\subset C_{i,i+1}$. The partial monoid structure is defined for $x,y$ which are contained in a cone and is undefined otherwise. Given an admissible inclusion $Q\subset P$  we define a map $f:Edges(P)\to Edges (Q)\cup\{*\}$ by $f(i)=*$ if the $i$th edge of $P$ does not contain any edge of $Q$, and by $f(i)=j$ for $\partial_jQ$ the unique edge contained in $\partial_i P$ otherwise. We then get an induced partial map of partial monoids $P_{trop}\to Q_{trop}$ by mapping the generator associated with $\partial_iP$ to the generator associated to $\partial_{f(i)}Q$ if $f(i)\neq *$ and is undefined otherwise.

To each element $x\in P_{trop}(\bZ)$ associate an $R$-module $M_x^*$ as follows. For the $0$ element take $M_0:=H^*(\pi^{-1}(P);R)$. For $x=me_i+ne_{i+1}$ let $T_x$ be the $(m,n)$-cover of the torus formed by taking the quotient $T^*_vB$ by the dual to the lattice generated by the primitive tangents to $e_1,e_2$. Let $M^*_x:=H^*(T_x;\bZ)$.

\setcounter{theorem}{\getrefnumber{tmInfSHn2}-1}
\renewcommand{\thetheorem}{\arabic{theorem}} % Redefine theorem numbering
We now recall and prove Theorem \ref{tmInfSHn2} from the introduction.

\begin{theorem}\label{tmInfShPtrop}
Let $P\subset B$ be admissible and let $K=\pi^{-1}(P)$.   Then 
\begin{enumerate}
\item
 The  infinitesimal Floer cohomology of $K$ is the direct sum 
 \begin{equation}\label{eqPtroInfSH}
SH^*_{0^+,M}(P)=M(P_{trop}(\bZ)):= \bigoplus_{x\in P_{trop}(\bZ)}M_x^*
 \end{equation}
 in each $\bR$-degree. 
\item
If  $Q\subset P$ is an admissible inclusion then the restriction map in infinitesimal Floer cohomology is induced by the map of partial monoids $f:P_{trop}(\bZ)\to Q_{trop}(\bZ)$. Namely, the map is identity for $x\in P_{trop}(\bZ)$ for which is defined and $0$ otherwise.
\end{enumerate}
\end{theorem}
\begin{rem}
The $R$-module $M(P_{trop}(\bZ))$ has a functorial structure of a BV-algebra over R. It should not be hard to show that the isomorphism of \eqref{eqPtroInfSH} is in fact an isomorphism of BV algebras. This could slightly simplify the proof of some claims below. We do not pursue this point further here. 
\end{rem}
We break down the proof of Theorem \ref{tmInfShPtrop} into a number of steps. We first discuss the notion of \emph{convex smoothing} of a polygon $P$. Let $C\subset B$ be a non-contractible smooth simple closed curve.  $C$ is said to be convex if orienting the curve $C$ counter clockwise, it is locally convex at each point of $C$. That is, identifying a small open neighborhood $U\subset B$ of $p\in C$ with an open set in $T_pB$ via the affine structure we have that $C\cap U$ is to the left of the oriented tangent line at $p$. For $C$ convex the \emph{outside} of $C$ is the component of $B\setminus P$ which contains points to the right of $C$ under the local identifications of the last sentence. A similar notion applies to a convex polygon and more generally any convex piecewise smooth simple closed curve. 

A convex smoothing of $P$ is  a family $P_s\subset B$ such that
\begin{enumerate}
\item $P=P_0$
\item $\partial P_s\subset B_{reg}$ is smooth and strictly convex 
\item for $s<s'$ we have $P_s\subset P_s'$. 
\item Define a characteristic line to be a curve $\gamma:t\mapsto \partial(P_t)$ so that the tangent  line to $\partial(P_t)$ at $\gamma(t)$ is locally constant in integral affine coordinates. Then $\gamma$ converges as $t\to 0$. 
\end{enumerate}


\begin{lm}
Every polygon admits a convex smoothing.
\end{lm}
\begin{proof}
The $i$th edge of $\partial P$ is defined by the equation $x_i=1$ for an appropriate integral affine function defined on a neighborhood of $\partial P$. Let $f:\bR\to \bR$ be a smooth monotone function which  is constant and equal to $1$ for $t\leq 1$ and is strictly monotone otherwise. Consider the function $h=\max_i\{f\circ x_i\}$ which extends as a constant to the interior of $P$ and is well defined on open neighborhood of $P$. Then $h$ is a continuous convex function defining $P$ as the inverse image of $1$.  For each $i$ pick a strictly convex function $g_i: e_i\to[0,1]$ which is $0$ on the boundary of $e_i$. For each $s$ let 
$$
f_{i,s}(x)=f(x_i-sg_i),
$$
and,
$$
h_s(x)=\left(\sum_if^s_{i,s}(x)\right)^{1/s}.
$$
Then $P_s:=h_s^{-1}((-\infty,1])$ is as required. 
\end{proof}
We say that an integral affine structure is \emph{conical} if there is an atlas so that the transition maps commute with local scaling. 
\begin{lm}
Suppose the integral affine structure on $B$ is conical and that $\pi$ admits a Lagrangian section over $B$. Then the family $K_s$ is an admissible smoothing in the sense of Definition \ref{dfAdmSm}.
\end{lm}
\begin{proof}
Fix a Lagrangian section $\sigma$ over $B$. Given a local integral affine chart we use the section $\sigma$ to define Arnold-Liouville coordinates. Namely, the angle coordinates are defined via the Hamiltonian flow of 
the coordinate functions of the integral affine chart with starting point $\sigma$. We then lift the Euler vector field in the action angle coordinates. The conicity assumption implies there is an atlas on $B$ so that the Euler vector field if invariant under all transition maps. 

By convexity of $f$, the Liouville field thus constructed points outward of each level set and  in particular defines a contact form. 

For each $s$ the primitive Reeb components are the pre-images under $\pi$ of points $x\in f^{-1}(s)$ at which the gradient in integral affine coordinates in a rational direction. By strict convexity there is no  birth or death of connected components \end{proof}




Let $Q\subset P$ be an admissible inclusion. We say that a convex smoothing $f_2$ of $Q$ is compatible with a convex smoothing $f_1$ of $P$ if there is an open set $V$ such that
\begin{enumerate}
\item $f_1|_V=f_2|_V$
\item The characteristic lines corresponding to elements of $Q_{trop}(\bZ)$ which are in the image of the partial map $P_{trop}(\bZ)\to Q_{trop}(\bZ)$ are all contained in $V$. 
\end{enumerate}

 
\begin{lm}
Compatible convex smoothings exist. Moreover, compatible smoothings of admissible polygons lift to compatible smoothings in  the sense of Definition \ref{dfCompSmooth} of the pre-images under $\pi$. 
\end{lm}
\begin{proof}
Let $e$ be an edge of $P$ containing a boundary point of $\partial P\cap \partial Q$. Any smoothing of $\partial P$ can be locally considered as a family of functions $s\mapsto f_s$ on $e$ with a single maximum. Moreover, the characteristic line of the element associated with the primitive conormal of $e$  is given by $(x_s,f(x_s))$ for $x_s$ the point where $f_s$ achieves its local maximum. We construct a smoothing of $P$ so that the characteristic line  goes to the interior of $e\cap\partial Q$ if the latter is non-empty. We then construct a convex smoothing of $Q$ which coincides with that of $P$ on a neighborhood of $(x_s,f(x_s))$. For illustration, see the right hand figure in Figure \ref{FigComp}. It is clear that such a pair of smoothings satisfies the requirement.

For the last part of the statement, we need to verify the stabilization requirement. For this note that as $s\to 0$, the differential $df_s$ at a point  $p\in \partial Q\cap\partial P$ converges to the covector associated to the edge of $\partial P\cap\partial Q$ containing $p$.  
\end{proof}




We can now Prove Theorem \ref{tmInfShPtrop}.


\begin{proof}[Proof of Theorem \ref{tmInfShPtrop}]
We pick compatible convex smoothings for $Q$ and $P$. Given a convex smoothing $f$ denote by $\cR(f)$ the corresponding set of families of Reeb components. Note these are the same as characteristic lines associated with rational directions. We construct a bijection $\cR(f)\to P_{trop}(\bZ)\setminus\{0\}$ as follows. If the limit point of the characteristic line $\gamma$ is in the interior of some edge we map it to the corresponding cone in accordance with its multiplicity. Similarly if the limit point is a vertex, we map it to the corresponding element in the corresponding cone in the obvious way.  This bijection is functorial with respect to admissible inclusions.

The irrationality assumption implies all the limiting  periods are distinct. The compatibility of the smoothings of $P$ and $Q$ implies compatibility in the sense of Theorem \ref{tmInfintsmlSHnsFunct} of the inclusion $\pi^{-1}(Q)\subset\pi^{-1}(P)$. The claim now follows immediately by Theorems \ref{tmInfintsmlSHns} and \ref{tmInfintsmlSHnsFunct}.  
\end{proof}



\subsubsection{The positive singularity}\label{SecPosSing}

We briefly discuss the case $n>2$. A full discussion will be given in a forthcoming work, and so the discussion in this part is somewhat less formal. 

To fix ideas we shall consider a neighborhood of the positive singularity.  This can be described as a symplectic manifold with an integral affine structure whose base is a prism as in Figure \ref{figPrism}. The fibers over the generic points of the dashed lines are products of a nodal $2$-torus with $S^1$ while the fiber over the vertex is of type $(1,2)$ in the terminology of \cite{Gross2001}. For a concrete model see \cite{Gross00} The monodromy around the singular values is such that there are global integral affine coordinates $x_1,x_2$ in terms of which the sides of $P$ are given by the equations $x_1=const,x_2=const$ and $-x_1-x_2=const$. The front and back sides are not defined by global integral affine coordinates, however there are functions $y^{\pm}$ which are integral affine on a half space containing the front and back respectively. 

 

% Figure environment removed


It is straightforward to associate a dual fan $P_{trop}(\bZ)$ to $P$. The one dimensional cones are non-negative integer multiples of the covectors $dy_{\pm},dx_1,dx_2,-d(x_1+x_2)$ with cones corresponding to double and triple intersections.  Note all these covectors define $S^1$ actions which are defined on a neighborhood of the respective face regardless of whether or not it contains a singular value. 

We can then associate with each $x\in P_{trop}(\bZ)$ a $\bZ$-module $M^*_x$.  For $x$ which is not in any of the $1$-dimensional cones associated with the sides of the prism this is a copy of the cohomology of the three torus.  However,  for elements corresponding to the periodic orbits of $dx_1,dx_2,-d(x_1+x_2)$ it can be shown that we get a copy of the cohomology of the singular fiber.  Note that in this case, the singular fiber produces periodic components which are not Morse-Bott. In this way it should be straightforward to prove the analogue the first part of  Theorem  \ref{tmLocSHn2} for the positive singularity. 

The second part of Theorem  \ref{tmLocSHn2}  is a little  more tricky as we did need to develop some theory for the restriction map in infinitesimal cohomology  from the component associated with normal to a face containing a singularity to a subface which does not contain the singularity.  A comprehensive discussion will be taken up in forthcoming work. 
\subsection{A criterion for torsion freedom}
\begin{df}
A Banach complex $B$ over $\Lambda$ is a chain complex over $\Lambda$ equipped with a complete norm such that the differential does not increase norms. We denote by $B_t$ the sub-complex consisting of elements whose norm is $<e^t$ and by $B_{[t_1,t_2)}$ the quotient complex $B_{t_2}/B_{t_1}$. We denote by $H_{t^+}(B)$ the infinitesimal cohomology
$$
\varprojlim_{\epsilon\to 0} H_{[t,t+\epsilon)}(B).
$$
\end{df}

\begin{rem}
    Some claims in this section and the next only use the norm, not the completeness property. Completeness is essential for 
\end{rem}
\begin{lm}\label{lmTorFreeInj}
Let $B$ be a Banach complex and suppose $H^*(B_0)$ is torsion free. Then for any $t<0$ the natural map 
\begin{equation}
\tau:H^*_{t^+}(B)\to H^*(B_0/B_t)
\end{equation}
is injective
\end{lm}

\begin{proof}
Let $x$ be a non-zero element in the kernel of $\tau$. Then for $\epsilon>0$ small enough $x$ is non-zero in $H^*_{[t,t+\epsilon)}(B)$. Fix such an $\epsilon$ and  pick a representative of $x$ in $B_{t+\epsilon}/B_t$, still denoted by $x$. Then $x$ is a boundary in $B_0/B_t$ but not in $B_{t+\epsilon}/B_t$.  So, let $y\in B_0/B_t$ be a primitive of $x$. Lift $y$ to an element $\overline{y}\in B_0$ and let $\overline{x}=d\overline{y}$. Then any primitive of $\overline{x}$ maps to a primitive of $x$ in $B_0/B_t$ and so is has norm greater than $e^{\epsilon+t}$. Note we can assume the norm of $\overline{x}$ is arbitrarily close to $e^t$. It follows that $z:=T^{t+\epsilon/2}\overline{x}\in B_0$. Moreover it is not a boundary in $B_0$, but has a scalar multiple which is a boundary. That is, $z$ is non-zero torsion element in $B_0$. 
\end{proof}

\begin{lm}\label{lmTorsionFreeInjInf}
Let $\phi:B\to C$ be a map of Banach complexes. Suppose  $H^*(C_0)$ is torsion free and the map $\phi_{0^+}:H^*_{0^+}(B)\to H^*_{0^+}(C)$ is injective. Then $H^*(B_0)$ is also torsion free. If $\phi_{0^+}$ is injective in a degree $i$ then $H^i(B_0)$ is torsion free.
\end{lm}
\begin{proof}
Let $x\in H^*(B_0)$ be a non-zero torsion element. Let $\lambda\leq 0$ be the supremal number such that $x\neq 0\in H^*(B_0/B_\lambda)$. Denote by $[x]$ the class of $x\in H^*(B_0/B_\lambda)$. Then $[x]\neq 0$. Indeed if $[x]=0$ then up to a boundary in $B_0$ we have  $|x|<e^{\lambda}$. That is there some $\epsilon>0$ such that $|x|<e^{\lambda-\epsilon}$ contradicting the definition of $\lambda$.  Moreover, $[x]$ is in the image of the map $H^*_{\lambda^+}(B)\to  H^*(B_0/B_\lambda)$. Let $y\in H^*_{\lambda^+}(B)$ be a pre-image, then $\phi_{\lambda^+}(y)\neq 0$ by assumption (infinitesimal cohomology is invariant under scaling). By the previous lemma the image of $\phi_{\lambda^+}(y)$ in $H^*(C_0/C_{\lambda})$ is non-zero. In particular, $[x]$ maps to a non-zero element. Considering the  commutative diagram
$$
\xymatrix{ H^*(B_0)\ar[d]\ar[r]&H^*(C_0)\ar[d]\\
H^*(B_0/B_\lambda)\ar[r]&H^*(C_0/C_{\lambda})}
$$
we see that $x$ itself maps to a nonzero element. The contradicts the torsion freedom of $C$. 
\end{proof}




\begin{lm}\label{lmTorsCrit}
Let $K$ be a smooth domain with positive contact boundary or one admitting a smoothing as in Theorem \ref{tmInfintsmlSHns} and let $\{K_i\}\subset K$ be a finite collection of admissibly embedded domains which admit compatible smoothings.  Suppose the  union of the $K_i$ form an open neighborhood of $\partial K$ in $K$. 
For an integer $j$ suppose $H^j(K)\to H^j(K_i)$ is injective and that $SH^j_M(K_i)$ is torsion free. Then $SH^j_M(K)$ is torsion free. 
\end{lm}
\begin{proof}
Consider the direct sum of maps $SH^*_{M,0^+}(K)\to SH^*_{M,0^+}(K_i)$.  It is an injection by Theorem \ref{mainThmC} . The claim now follows from Lemma \ref{lmTorsionFreeInjInf}.  
\end{proof}
\begin{rem}
The compatibility assumption in Lemma \ref{lmTorsCrit} is essential. For example, consider as in Example \ref{exComp} and Figure \ref{FigComp2} a partition of an $L$ shaped polygon into convex rectangles. Then the pre-images of the rectangles are torsion free, but that of the ambient L-shaped domain is not. See \cite[\S6.4]{GromanVarolgunes2021}.
\end{rem}


\subsection{Torsion freedom and the spectral symbol}\label{SecSpectral}
We use the notation $SH^*_{M,t}(K)$ \footnote{Note this notation, which indicates an action window of $[-\infty,t),$ collides somewhat with the notation $SH^*_{M,t^+}(K)$ which indicates an infinitesimally thin action window.} to denote the homology of the subcomplex $CF^*_{M,t}(K)\subset CF^*_M(K)$ consisting of the elements whose norm is $<e^t.$
The symplectic cohomology $SH_{M,t}(K)$ is a module over the Novikov ring $\Lambda_{\geq 0}$. As such it carries a valuation defined by $\val(a)=\sup\{r\in\bR_{\geq 0}:a\in T^rSH^*_M(K)$. We define a semi-norm by $|a|=e^{-\val(a)}$. Note the semi-norm makes sense even for torsion elements though it is only submultiplicative with respect to scalar multiplication for torsion elements. 

For $a\in SH^*_{M,0}(K)$ the statement $|a|\leq e^t$ implies that $a$ is in the image of $SH^*_{M,t+\epsilon}\to SH^*_{M,0}(K)$ for all $\epsilon>0$. 

Suppose now that  $SH^*_{M,0}(K)$ is torsion free. It then follows that there is a unique element in $SH^*_{M,t^+}(K)$ that maps to the image of $a$ in the quotient $SH^*_{M,[t,\infty)}$.  We thus get for each $t\in\bR$ a map
\begin{equation}
\tilde{\sigma}_t: SH^*_{M,t}(K)\to SH^*_{M,t^+}(K),
\end{equation} 
where
$$
 SH^*_{M,t}(K):=\varprojlim_{\epsilon\to 0} SH^*_{M,[-\infty,t+\epsilon)}(K).
$$
Combining this with the map from Theorem \ref{mainThmC} we obtain the 
\begin{equation}
\sigma_t: SH^*_{M,t}(K)\to RH^*(K).
\end{equation} 
We call $\sigma_t(x)$ the \emph{$t$-symbol} of $x$. 

From now on we assume $SH^*_{M,0}(K)$ is torsion free.  
The $t$-symbol is compatible with restriction maps for admissible inclusions of torsion free domains. Namely, the norm is not increased under the restriction map $r$. So, if $\sigma_t(x)$ is defined, so is $\sigma_{t}(r(x))$, and we have the equality $\sigma_t(r(x))=r_*\sigma_t(x)$ where $r_*$ is the functorial map $RH^*(K_2)\to RH^*(K_1)$.






We have seen that $SH^*_{M,[a,a+\hbar)}(K)$ is filtered by action. Let us  write $\sigma(x):=\sigma_{|x|}(x)$ and refer to this as the \emph{leading symbol} of $x$. For an $\epsilon\in(0,\hbar)$ we call an element of $x\in SH^*_K(M)$ \emph{$\epsilon$-good}  if $x$  is contained modulo $T^{\epsilon}$ in the submodule of action $\leq\cA(\sigma(x))$. Note that modulo $T^{\epsilon}$ the image of a good representative under $r$ is a combination of generators of action  $\leq\cA(\sigma(x))$
 

\begin{lm}\label{lmEpsGoodGenCrit}
Suppose $SH^*_M(K)$ is torsion free. Let $\epsilon>0$ and let $\cB\subset SH^*_M(K)$ be a collection of $\epsilon$ good elements. Suppose the set $\{\sigma(x):x\in \cB\}$ generates $RH^*(K)$. Then $\cB$ generates  $SH^*_M(K)$ as a topological vector space over $\Lambda$. A similar statement is true in a fixed degree $i$ assuming torsion freedom in degrees $i$ and $i+1$. 
\end{lm}
\begin{proof}
Since we are working over the Novikov field, we may assume without loss of generality that the elements of $\cB$ are normalized. For any $a$ denote by $V_a=\{x\in SH^*_M(K):|x|<e^a\}$. Observe that because of torsion freedom, $SH^*_M(K)$ is complete. For a proof of this see \cite[\S6.3]{GromanVarolgunes2021}. So, it suffices to show for some $\epsilon >0$ the image of $\cB$ in the quotient $V_\epsilon/V_0$ generates. 

For this we claim that torsion freedom implies
\begin{equation}
V_b/V_a=SH^*_{M,[a,b)}(K). 
\end{equation}
We  first define a map. Given an element $x$ of $V_b$ we get an element on the right hand side by picking a representative $y$ of $x$ of norm $<b$. For any other representative $y'$ we must have that $y-y'$ can be killed by an element of norm $<b$ or otherwise we could produce a non-zero torsion element.  So we get a well defined map from $V_b$.  This map is norm preserving by the same reasoning.  In particular, the submodule $V_a$ maps to $0$.  

We verify that the map is an isomorphism. For injectivity, if an element $x$ on the left maps to $0$ on the right it means that $x$ has a representative of norm $<a$, so it belongs to $V_a$.  For surjectivity, a non-vanishing element on the left is represented by an element $x$ of norm $\geq a$ with boundary of norm $t<a$. By torsion freedom we can kill $dx$ by an element of norm $t$. Thus we can assume that $dx=0$. In particular, it lifts to $V_b/V_a$.

As a consequence, it suffices to show for some $\epsilon>0$ that the image of $B$  generates $SH^*_{M,[0,\epsilon)}(K)$. 

For this we observe that for any action $a$ we can construct a non-canonical isomorphism $F^aSH^*_{M,[0,\hbar)}(K)\simeq F^aRH^*(K)\otimes\Lambda_{[0,\hbar)}$ in a filtration and leading term preserving manner.  More precisely,   there is an isomorphism 
$$F^aSH^*_{M,[0,\hbar)}(K)=\oplus_{\gamma:\cA(\gamma)<a}SH^*_{uw}(\gamma)\otimes\Lambda_{[0,\hbar)}$$ 
To see this note that we have an isomorphism $F^aSH^*_{M,[0,\hbar)}(K)=\varinjlim_IF^aIn_{[0,\hbar)}(H_{K,I})$ where for $I$ large enough, all the maps are isomorphisms. The torsion freedom assumption implies that $In_{[0,\hbar)}(H_{K,I})$ can be modeled as the colimit complex of the sequence $CF^*_{In,[0,\hbar)}(H_{K,I,\eta_i})$ where each complex in the sequence has vanishing differential.  Thus we can take any complex in the sequence  to be the model for  $$F^aSH^*_{M,[0,\hbar)}(K)$$ and its homology is isomorphic to $$\oplus_{\gamma:\cA(\gamma)<a}SH^*_{uw}(\gamma)\otimes\Lambda_{[0,\hbar)}.$$ 

Thus given any element $x\in SH^*_K(M)$ which we take without loss of generality to be normalized,  we consider the truncation $\tau_{\hbar}(x)$ of $x$ modulo $T^{\hbar}$. Then $\tau(x)$ is  a finite combination of good elements of with leading symbol in $\oplus_{\gamma:\cA(\gamma)<a}SH^*_{uw}(\gamma)\otimes\Lambda_{[0,\hbar)}$ for some $a$. We can then truncate further modulo $T^{\epsilon}$. Then by $\epsilon$-goodness we can write $\tau_{\epsilon}(x)$as a finite combination of elements of $\cB$. 
\end{proof}

\begin{cy}\label{cyEpsGoodGenCrit}
    Under the hypothesis of Lemma \ref{lmEpsGoodGenCrit}, suppose $\cB$ consists of normalized elements and the $R$-module generated maps injectively into $RH^*(K)$ under $x\mapsto\sigma(x)$. Then there is an isomorphism
    \begin{equation}
    SH^*_{M,0}=RH^*(K)\otimes\Lambda_0.
    \end{equation}
    Similarly, for any $a<b\in[-\infty,-\infty)$ we have
     \begin{equation}
    SH^*_{M,[a,b)]}=RH^*(K)\otimes\Lambda_[a,b).
    \end{equation}
\end{cy}

\subsection{Proof of Theorems \ref{tmLocSHn2} and \ref{tmLocSHn3}}
In this section we refer to notions from rigid analytic geometry. A detailed reference is \cite{Bosch14}.




\begin{df}
A Lagrangian sub-manifold $L\subset M$ is called \emph{undeformed} if there is a  Weinstein neighbourhood $K\subset M$ of $L$ such that, denoting by $\overline{K}$ the Liouville completion of $K$, $SH_{\overline{K}}(K)$ is torsion free and  there exists an isomorphism of BV algebras $SH^*_M(K)\simeq SH_{\overline{K}}(K)$ which respects spectral symbols. 
\end{df}
\begin{rem}
The definition makes sense also for singular Lagrangians which occur as skeleta of Liouville domains.
\end{rem}


Let $\pi:M\to B$ and $P\subset B$ be as in Theorem \ref{tmInfShPtrop}. For $P\subset B_{reg}$ a convex polygon denote by $\cF^*_{loc}(P)$ the relative $SH$ of $\pi^{-1}(P)$ inside its Liouville completion. Explicitly, fix a base-point $p$ in $P$ and  $L_p:=\pi^{-1}(p)\simeq \bT^2$. We can canonically identify $P$ with a subset of $H^1(L_p;\bR)$ so that $p$ maps to the origin. We can lift this identification to an embedding of $\pi^{-1}(P)$ into $T^*L_p$ and, abbreviating $B_{loc}:=H^1(L_p;\bR)=(H_1(L_p;\bZ)\otimes\bR)^*$, extend $\pi$ to a map $\pi_{loc}:T^*L_p\to B_{loc}$. Picking a basis $\{e_1,e_2\}$ for $H_1(L_p;\bZ)$ we get an integral affine isomorphism $B_{loc}\simeq\bR^2$. Consider the algebraic torus $mSpec(\Lambda[ H_1(L_p;\bZ)])\simeq {(\Lambda^*)^2}$ over the Novikov field. It comes equipped with a log map to $B_{loc}$ which upon picking a basis is the map $f:(\Lambda^*)^2\to\bR^2$ given by $(x,y)\mapsto(\val(x),\val(y))$. 

The choice of basis induces isomorphism $\cF^0_{loc}\simeq f_*\cO_{(\Lambda^*)^2}$ of sheaves \cite[\S5]{GromanVarolgunes2022}. Let us describe this geometrically. Say the polygon $P$ is cut out by rational half planes $P_i$. For each $i$, let $\ell_i=\langle a_i,\cdot\rangle +b_i$ be the defining affine function of $P_i$, where $a_i\in H_1(L_p;\bZ)$ is a primitive generator and $b_i\in\bR$. Denote the corresponding monomial by $T^{b_i}u^{a_i}\in H_1(L_p;\bZ)]$. Then for each $i$, the pre-image $f^{-1}(P_i)$ is the set of points where $|T^{b_i}u^{a_i}|\leq1$, or equivalently $|u^{a_i}|\leq e^{b_i}$. Thus $\cF^0_{loc}(P)$ is the algebra of analytic function on the non-Archimedean toric domain cut out by the inequalities $|u^{a_i}|\leq e^{b_i}$. Similarly, it is shown  in \cite[\S5]{GromanVarolgunes2022}  that $\cF^*_{loc}$ is canonically isomorphic to the sheaf of polyvector fields on the analytification of the algebraic torus over the Novikov field. Note that this identification of $\cF^0(P)$ involves a choice of base-point. This corresponds to choosing a primitive of the symplectic form on the completion of $\pi^{-1}(P)$. We generally work with not with the generators $u^a$ but rather with the generators $z^a:=T^bu^a$. That is, our generators always have norm $1$. This corresponds to the fact that our construction of Hamiltonian Floer cohomology over the Novikov field assigns to each generator the norm $1$ since primitives exist only in special situations. 

\begin{tm}\label{tmAffinoidPolygon}
Let $\pi:M\to B$ and $P\subset B$ be as in Theorem \ref{tmInfShPtrop}. Suppose the fibers of $\pi$ are undeformed. Suppose the inclusion of a regular fiber into $\pi^{-1}(P)$ induces an injection of $H^*(\pi^{-1}(P);\bZ)$ into $H^*(\bT^n;\bZ)$. For each $i$ let $\xi_i\in P_{trop}(\bZ)$ be the generator associated with the $i$th face. Let  $z_i\in \cF_P$ be a normalized element whose leading symbol is $\xi_i$. Abusing notation we also consider  $z_i$ as a normalized representative cycle in $SC^*_M(\pi^{-1}(P)$. Then \begin{enumerate}
\item The set $\cB$ of monomials in the $z_i$ satisfies the hypothesis of Corollary \ref{cyEpsGoodGenCrit}.
\item 
$\cF_P$ is generated as a Banach algebra by the $z_i$. In particular, $\cF_P$ is affinoid and of dimension $n$. 
\item For an admissible inclusion $Q\subset P$  the restriction map $\cF_P\to\cF_Q$ is the inclusion of a Laurent domain provided $Q$ is contained in a sufficiently small neighborhood of the boundary. 
\end{enumerate}
\end{tm}
\begin{proof}
First observe $SH^*_{M,t}(\pi^{-1}(P))$ is torsion free by Lemma \ref{lmTorsCrit}. By finiteness there is an $\epsilon\in(0,\hbar)$ so that the $z_i$ are all $\epsilon$-good.  Since the product preserves the action filtration modulo $T^{\epsilon}$ it follows that all monomials in  the $z_i$ are $\epsilon$-good. We now show that for each $i,j$ the monomial $z_i^j$  has leading symbol $\xi_i^j$ and that for adjacent edges $i_1,i_2$ and integers $j_1,j_2\geq 0$ the product $z_{i_1}^{j_1}z_{i_2}^{j_2}$ has leading symbol $\xi_{i_1}^{j_1}\xi_{i_2}^{j_2}$.  



Any $I\in P_{trop}(\bZ)$ can presented as a triple $(v, j_1,j_2)$ where $v$ is a vertex joining two adjacent edges $i_1,i_2$ and   $(j_1,j_2)$ are non-negative generators. We first prove  $\left|\xi^{(v,j_1,j_2)}-\xi_{i_1}^{j_1}\xi_{i_2}^{j_2}\right|<1$. Denote by $r$ the restriction map to a polygon sharing the corner containing $(i_1,i_2)$. Since $r$ is a homomorphism, $r(z_{i_1}^{j_1}z_{i_2}^{j_2})=r(z_{i_1}^{j_1})r(z_{i_2}^{j_2})$. By the undeformedness assumption on the product  for neighborhoods of regular fibers, $r(z_{i_1}^{j_1})r(z_{i_2}^{j_2})$ has leading symbol $\xi^{(v,j_1,j_2)}\in P_{trop}(\bZ)$. We deduce that the leading symbol of $z_{i_1}^{j_1}z_{i_2}^{j_2}$ is a sum $\xi^{(v,j_1,j_2)}+a$ for $a$ in the kernel of $P_{trop}(\bZ)\to Q_{trop}(\bZ)$.  We must have $\cA(a)\leq \cA(\xi^{(v,j_1,j_2)})$ and this inequality must be strict since distinct terms have distinct action values by the irrationality assumption. 

Thus the leading symbol of a monomial in the $z_i$ is the corresponding monomial in the $\xi_i$.  The set of monomials in the $z_i$ satisfies the hypothesis of Lemma \ref{lmEpsGoodGenCrit} and thus generate $\cF^0(P)$  as a Banach space. It follows that the $z_i$ themselves generate $\cF^0(P)$ as a Banach algebra. 

The claim concerning the dimension follows since dimension is preserved under reduction $T=1$. 



For the last part of the claim consider first $Q\subset P$ cut out by an affine line $\ell$ parallel to the edge $e_i$. We claim that since $z_i$ has leading term coinciding with that of an invertible element of $\cF(Q)$ it follows that if $\ell$ is close enough to the edge $e_i$ then  $z_i$ is itself invertible on $Q.$ Here we are relying on undeformedness of $Q$ and of our knowledge of $\cF^0_{loc}(Q)$. Namely, we can identify $\cF^0_{loc}(Q)$ with a completion of the Laurent series in variables $x_1,x_2$ in such a way that the edge $e_i$ corresponds to $\val(x_1)=0$ and making $\ell$ arbitrarily close to the edge $e_i$ has the effect of considering a subset of the domain $|x_1|\geq 1-\epsilon$ for $\epsilon$ arbitrarily close to $0$. Since $z_i$ coincides with $x_1$ up to elements of strictly lower norm, say $<e^{-\epsilon'}$, it suffices to consider $\epsilon<\epsilon'$.

Write $\epsilon=-\log|z_i^{-1}|.$ Then we have a commutative diagram 
\[
\xymatrix{
    \cF(P) \ar[r] \ar[dr] & \cF(P)\{X\}/(z_iX=T^{\epsilon} )\ar[d]^{h} \\
    & \cF(Q)
}
\]
where $h$ is uniquely determined by being identity on $\cF(P)$ and mapping $X$ to $T^{\epsilon}z_i^{-1}$.
We need to show that $h$ is an isomorphism. Injectivity we prove in Lemma \ref{lmUniqueContinuation}. For surjectivity,  let $u_1,u_2,u_3$ be the images under $h$ of the generators corresponding to the three common edges of $Q$ and $P$ , and  of $X$ respectively. Let $u_4=u_2^{-1}$. We take the generators to be cyclically ordered in the obvious way. Let $\cB$ be the set of monomials in these elements with nonnegative powers where we only consider products of adjacent generators. Using Lemma \ref{lmEpsGoodGenCrit} we see that the submodule $V$ generated by $\cB$ is dense in $\cF(Q)$. To conclude surjectivity it suffices to observe that on the set $\cB$, the map $h$ precisely preserves norms. In particular the closure of $V$ is in the image of $h$. 

Fix now a neighborhood of $\partial P$ which is the union of the polygons $Q_i$ cut out by lines which are parallel and sufficiently close to $e_i$. Then the claim for polygons in such a neighborhood follows by our knowledge of $\cF_{loc}$ and by transitivity of Laurent domains. 


\end{proof}

\begin{lm}\label{lmUniqueContinuation}[Uniqueness of analytic continuation]
Suppose the regular fibers are undeformed. Let $Q\subset P$ be an admissible inclusion such that $\partial Q\cap\partial P\neq\emptyset$. Then the restriction map $r:\cF^0_P\to\cF^0_Q$ is injective. 
\end{lm}
\begin{proof}
For each $i$ consider an admissible neighborhood $P_i\subset B_{reg}$ of the $i$th edge inside $P$. Let $i$ be such that $Q\cap P_i$ is non-empty.  If $x$ map to $0$ in $\cF^0_Q$, it maps to $0$ in $\cF^0_{P_i\cap Q}$. Since $P_i\subset B_{reg}$ and we know that for the regular polygons the restriction maps are injective it follows that $x$ vanishes in $\cF^0_{P_i}$. Therefore, it vanishes in $\cF^0_{P_j}$ for $j$ adjacent to $i$. By connectedness we get vanishing for all the $P_i$.  But  the map from $\cF^0_P$ to the direct sum of the $\cF^0_{P_i}$ is injective since the map on spectral symbols is injective according to Theorem \ref{tmInfShPtrop}.  

\end{proof}
We now consider symplectic cohomology in degrees $>0$.  For any compact set $K$ there is a bracket operation 
$$
\{,\}:SH^1_M(K)\otimes SH^0_M(K)\to SH^0_M(K)
$$
defined by 
$$
\{v,f\}:=\Delta(v*f)+(\Delta v)*f-v*\Delta f.
$$
Here $\Delta$ denotes the BV operator and $*$ the pair of pants product. 
In the cases we care about $SH^*$ is supported in non-negative degrees, so the formula simplifies to 
$$
\{v,f\}:=\Delta(v*f)+(\Delta v)*f. 
$$
For each $v\in SH^1$ we denote by  $\partial_v:SH^0\to SH^0$ the map $\partial_v:=\{v,\cdot\}$.  Then  $\partial_v$ satisfies the Leibniz rule.  For a detailed discussion see \cite{Abouzaid14}.  For a Banach algebra $\cA$ we denote by $Der(\cA)$ the set of bounded linear operators which satisfy the Leibniz rule. 

Similarly,  we can map $\cF^2_P$ to bi-derivations by taking the $v$ to $\{v,\cdot\}$ where $\{v,f\}$ is defined by 
$$
\{v,f\}:=\Delta(v*f)-(\Delta v)*f
$$
for $f\in \cF^0_P$. That this is a biderivation follows from the Poisson identity for the bracket.  See \cite{Abouzaid14}.  


\begin{lm}\label{lmInjPoly}
Suppose the regular fibers are undeformed.  Let $Q\subset P$ be an admissible inclusion of rational convex Delzant polygons such that $\partial Q\cap\partial P\neq\emptyset$ and suppose $Q\subset B_{reg}$. Then the kernel of the map  $\cF^1_P\to Der(\cF^0_P)$,  given by $v\mapsto\partial_v$,  is contained in the set of elements whose leading symbol is in  the kernel of the map $H^*(\pi^{-1}(P);R)\to H^*(\pi^{-1}(Q);R)$.
\end{lm}
\begin{proof}
Suppose $\partial_v$ is identically $0$.  Denote by $r:\cF^*_P\to\cF^*_Q$ the Floer theoretic restriction. Then $\partial_{r(v)}$ is the restriction of the operator $\partial_v$.  In particular,  the restriction of $\partial_v$ to $\cF^0_Q$  vanishes.  We know that for a convex polygon in the base of $T^*\bT^n\to\bR^n$ the map from $SH^1$ to derivations is an isomorphism.  See \cite{GromanVarolgunes2022}.  It follows that $v$ is in the kernel of $\cF^1_P\to\cF^1_Q$.  This is only possible if the leading term is in the kernel of the map in infinitesimal $SH$. The claim follows.  
\end{proof}

Suppose the fibration admits a topological section $\sigma$ . The Poincare dual to $\sigma$ defines an element, still denoted by $\sigma$ in $H^2(M;\bZ)$ which maps to an element again denoted by $\sigma$ in $\cF^2(P)$. 
\begin{lm}\label{lmPolyvectPolygon}
Suppose the regular fibers are undeformed and let $\sigma$ as above. Then the restriction of $\sigma$ to $Q\subset B_{reg}$ is nowhere vanishing bi-vector field. If in addition $\cF^0(P)$ is a smooth algebra, then $\sigma$ vanishes nowhere. In this case, the map $v\mapsto\partial v$ is surjective. Suppose in addition that $H^*(\pi^{-1}(P);R)\to H^*(\pi^{-1}(Q);R)$ for $Q\subset B_{reg}$ is injective. Then $\cF^*_P$ is canonically isomorphic to the polyvector fields.  

\end{lm}
\begin{proof}
Let us introduce the notation $
\cF^2_{loc}(Q)$ for the symplectic cohomology of $\pi^{-1}(Q)$ inside its completion. Denote by $\lambda_{loc}\in \cF^2_{loc}(Q)$ the element  corresponding to the cohomology class associated with a section of $\pi_{loc}$ under the map from ordinary cohomology to $\cF^2_{loc}(Q)$. This is the same as the Poincare dual of the point class of the torus fibers.  By standard computations of symplectic cohomology of the cotangent bundle, $\lambda_{loc}$ gives rise to the nowhere vanishing $2$-from $\frac{dx\wedge dy}{xy}$. See \cite{GromanVarolgunes2022} for a detailed discussion. Moreover, $\lambda_{loc}$ is characterized by its leading symbol together with the fact that it vanishes under the BV operator. Let  $\lambda$ be the element corresponding to $\sigma$ in $\cF^2(Q)$ and fix an identification of $\cF^2_{loc}(Q)$ with $
\cF^2(Q)$ which respects leading symbols. Then $\lambda$ has the same leading symbol as $\lambda_{loc}$ and vanishes under the $BV$ operator, so it must coincide with $\lambda_{loc}$. In particular, it vanishes nowhere on $Spec (\cF^0(Q))$. This implies the first part of the claim. 

For the second part, note that by smoothness, the top exterior power of the tangent sheaf is a locally free sheaf of rank $1$. So, the vanishing locus $s$ of $\lambda$ is of codimension $1$ or empty.  Let $f_1,\dots, f_N$ be the set of generators associated with the edges of $P$. For sufficiently small $\epsilon$ each of the Laurent domains $|f_i|>1-\epsilon$ corresponds to the restriction of $
\cF^0(P)$ to an admissible polygon $Q_i$ which is a slight thickening of the $i$the edge. So, by \cite[Proposition 1.6]{Lutkebohmert}, if $s$ is non-empty, it must meet the image of $mSpec(\cF^0(Q_i))$ inside the spectrum of $\cF^0(P)$. But this contradicts the previous paragraph. 

For the third part of the claim observe that as a BV algebra, for each $Q\subset B_{reg}$ we have that $\cF^*(Q)$ is isomorphic to to the algebra of polyvector fields with  $\Delta$ the divergence operator associated with the non-vanishing vector field coming from $\sigma$. The claim then amounts to the claim that the vector fields on a smooth affine variety are generated as a module over the functions by vector fields which are divergences of polyvector fields. This is equivalent to the observation that differential forms are generated as a module over functions by exact differential forms.

The final part is immediate from Lemma \ref{lmInjPoly} and what we have proven so far.


\end{proof}

\begin{proof}[Proof of Theorems \ref{tmLocSHn2} and \ref{tmLocSHn3}]
Theorem \ref{tmLocSHn2} follows from Theorem \ref{tmAffinoidPolygon}, Corollary \ref{cyEpsGoodGenCrit}, and Lemma \ref{lmInjPoly}. Theorem \ref{tmLocSHn3} follows from Lemma \ref{lmPolyvectPolygon}.
\end{proof}








\appendix

 \section{$C^0$ estimates for Floer trajectories}\label{SecEstimates}

In this section we recall $C^0$ estimates on Floer trajectories in the small energy regime. These are enhancements of estimates by \cite{Usher09} and \cite{Hein}. 



 
 

 


\subsection{Gromov's trick for Floer trajectories}
Denote by $\Sigma$ the cylinder $\bR\times S^1$ considered as a Riemann surface with its standard complex structure $j_{\Sigma}$. 
Let $H:\Sigma\times M\to\bR$ be a function which is $s$ independent outside of a compact set. Let $(s,t)\mapsto J_{s,t}$ be a $\Sigma$-dependent family of almost complex structures on $M$ which is $s$ independent outside of a compact set.  To this datum we associate an almost complex structure $J_{H}$ on $\Sigma\times M$ defined by
\begin{equation}\label{eqGrTrick}
J_{H}:=J_M+j_{\Sigma}+X_H\otimes ds- JX_H\otimes dt.
\end{equation}
When $H$ is $s$-dependent, we assume $\partial_sH\geq 0$. 
In this case the closed form
\begin{equation}\label{eqGrTrick2}
\omega_{H}:=\pi_1^*\omega_{\Sigma}+\pi_2^*\omega+ dH
\end{equation}
on $\Sigma\times M$ can be shown to be symplectic and  $J_{H}$ is compatible with it. Here $\omega_\Sigma$ is any symplectic form compatible with  $j_\Sigma$.  We shall take $\omega_\Sigma$ to coincide with the form $ds\wedge dt$ on the ends.
We denote the induced metric on $\Sigma\times M$ by $g_{J_{H}}$. We refer to the metric $g_{J_H}$ as the \emph{Gromov metric}. We stress that \emph{a Gromov metric depends on the choice of area form on $\Sigma$}. 

In the following lemma we consider in particular the case $\Sigma=\bR\times S^1$, $H=H_{s,t}dt$ and $\omega_{\Sigma,\tau}:=\tau^2 ds\wedge dt$. For $i=1,2$ let 
$\pi_i$ be the $\Sigma, M$ respectively. Let $g_{\Sigma}$ be the standard metric on the cylinder $\bR\times S^1$. Given a pair of Riemannian metrics $g_1,g_2$ on a smooth manifold, we say they are \emph{$C$-equivalent} for some $C>1$ if $\frac1{C}\|v\|^2_{g_2}<\|v\|^2_{g_1}<C\|v\|^2_{g_2}$


\begin{lm}\label{lmProdCEquivalence}
There is a continuous function $f:\bR_+^2\to\bR_+$ converging to $1$ at $(0,0)$ such that writing $C_{\tau}:=f\left(\left|\frac{\partial_sH}{\tau^2}\right|_{\infty},\left|\frac{\|X_H\|^2}{\tau^2}\right|_{\infty}\right)$, the metric $g_{J_H}$ determined by $J_H$ and $\omega_{\Sigma,\tau}$ is $C_{\tau}$-equivalent to the product metric $\tau^2\pi_1^*g_\Sigma+\pi_2^*g_J$.  
\end{lm}
\begin{proof}
We have
\begin{equation}\label{eqGromovMetricFormula}
g_{J_H}=(\tau^2+\partial_sH)\pi_1^*g_{\Sigma}+\|X_H\|^2_{g_J}(\pi_1^*dt)^2+g_J(X_H,\cdot)dt +\pi_2^*g_J.
\end{equation}
To relate this metric with the product metric we split $$T(\Sigma\times M)=\left(T\Sigma\oplus\bR X_H\right)\oplus (\bR X_H)^{\perp},$$ 
where  $(\bR X_H)^{\perp}$ denote the orthogonal complement in $TM$ with respect to $g_{J_H}$. Note this is also an orthogonal splitting with respect to $g_J$. Let $p_1, p_2$ be the orthogonal projections associated with the splitting. Consider the map $A_{\tau}$ on the first summand given in the basis $\left(\frac{\partial}{\partial s},\frac{\partial}{\partial t}, X_H\right),$ by the matrix
$$
\begin{pmatrix}1+\frac{\partial_sH}{\tau^2}& 0& 0\\
0& 1+\frac{\partial_sH+\|X_H\|^2}{\tau^2}& -\|X_H\|\\
0& -\|X_H\|& 1
\end{pmatrix}
$$
Then $g_J$ is represented with respect to the product metric by the linear map $A\circ p_1+p_2$. For $x,y\geq 0$ consider the matrix $A(x,y)$
$$
\begin{pmatrix}1+x& 0& 0\\
0& 1+x+y^2& -y\\
0& -y& 1
\end{pmatrix}
$$
and let $f(x,y)=\max\left\{\lambda_{max},\frac 1{\lambda_{min}}\right\} $ where we refer to the maximal and minimal eigenvalue of the positive definite symmetric matrix $A(x,y)$. Then $f(x,y)\to 1$ as $(x,y)\to (0,0).$
Moreover, for $C_{\tau}:=f\left(\left|\frac{\partial_sH}{\tau^2}\right|_{\infty},\left|\frac{\|X_H\|^2}{\tau^2}\right|_{\infty}\right)$ the metric $g_{J_H}$ is $C_{\tau}$-equivalent to the corresponding product metric.
\end{proof}

An observation known as \emph{Gromov's trick} is that $u$ is a solution to Floer's equation
\begin{equation}\label{eqFloer}
(du-X_{J})^{0,1}=0,
\end{equation}
if and only if its graph $\tilde{u}$ satisfies the Cauchy Riemann equation
\begin{equation}
\overline{\partial}_{J_{H}}\tilde{u}=0.
\end{equation}
Thus Floer trajectories can be considered as $J_{H}$-holomorphic sections of $\Sigma\times M\to\Sigma$. 
To a Floer solutions $u:\Sigma\to M$ and a subset $S\subset\Sigma$ we can now associate three different non-negative real numbers:
\begin{itemize}
\item The \emph{geometric energy} $E_{geo}(u;S):=\frac12\int_S\|(du-X_{H})\|^2$ of $u$.
\item The \emph{topological energy} $E_{top}(u;S):=\int_Su^*\omega+\tilde{u}^*dH.$
\item The \emph{symplectic energy} $E(\tilde{u};S):=\int_S\tilde{u}^*\tilde{\omega}$. 
\end{itemize}
We have the relation $E(\tilde{u};S)=E_{top}(u;S)+Area(S)$. For a monotone Floer datum we have, in addition, the relation $E_{geo}(u;S)\leq E_{top}(u;S)$. 

 The key to obtaining $C^0$ estimates is the is the monotonicity lemma  \cite{Sikorav94} .  For a $J$-holomorphic map $u:\Sigma\to M$ and for a measurable subset $U\subset \Sigma$ write
\begin{equation}
E(u;U):=\int_Uu^*\omega.
\end{equation}
\begin{lm}\label{lmMonEst++}[\textbf{Monotonicity} \cite{Sikorav94} ]
Fix a compatible almost complex structure $J$ on $M$. Let $a$ be a constant and let $p\in M$ be a point at which $|Sec|\leq a^2$ on the ball $B_{1/a}(p)$ and such that at $p$ the injectivity radius is $\geq\frac1{a}$. 
 Let $S$ be a compact Riemann surface with boundary and let $u:S\to M$ be $J$-holomorphic such that $p$ is in the image of $u$ and such that
\[
u(\partial S)\cap B_{1/a}(p)=\emptyset.
\]
Then there is a universal constant $c$ such that
\begin{equation}
E\left(u;u^{-1}(B_{1/a}(p))\right)\geq\frac 1{a^2}.
\end{equation}
If, instead, we only require that that there exists a constant $C>1$ and a Riemannian metric $h$ satisfying the above bounds on the sectional curvature and injectivity radius  $g_J$ and such that 
$$
\frac1{C}g_J(v,v)\leq h(v,v)\leq Cg_J(v,v),
$$
we get the inequality
\begin{equation}
E\left(u;u^{-1}(B_{1/a}(p))\right)\geq\frac{1}{C^3a^2}.
\end{equation}
 \end{lm}
 \begin{rem}\label{rmMonotonicity}
Lemma \ref{lmMonEst++} applies in particular to the case of a $J_H$ holomorphic curve $\tilde{u}$ associated with a monotone Floer datum $(H,J)$ where both $H$ and $J$ are allowed to vary on a compact subset of $\Sigma$. In this case we let $a$ be an estimate of the injectivity radius and sectional curvature of the  almost complex structure on $\Sigma\times M$ defined by $J_{z,x}=j_{\Sigma}\times J_z$ and we let $C=\max_{x\in B_{1/a}(p)}\{\|X_H\|^2,(\partial_sH)^2\}$ \cite[Lemma 5.11]{Groman} 
 \end{rem}
\subsection{The energy distance inequality}
The following proposition is slight refinement of \cite[Propositio 3.5]{Hein} and the proof is  taken from there. 
\begin{prp}\label{PrpDistEn0}
Let $V$ be a possibly time dependent vector field whose flow is complete on a Riemannian manifold $M$. Denoting the time $t$ flow by $\psi_t$ let $\lambda>0$ be a constant such that $\|d\psi_tv\|>\lambda^{-1}|v|$ for all tangent vectors $v$. Let $\gamma:[0,T]\to M$ be a smooth path. Then
\begin{equation}
    d(\gamma(0),\psi_T^{-1}(\gamma(T)))<\lambda\int_0^T\|\gamma'(t)-V_t\circ\gamma(t)\|dt.
\end{equation}
\end{prp}
\begin{proof}
    Let $\eta(t):=\psi_t^{-1}(\gamma(t))$. Then $\dot{\gamma}(t)=d\psi_t\dot{\eta}(t)+V_t\circ\gamma(t)$. So,
    \begin{align*}
       \int_0^T\|\gamma'(t)-V_t\circ\gamma(t)\|dt&= \int_0^T\|d\psi_t\dot{\eta}(t)\|dt\\
       &>\lambda^{-1}\int_0^T\|\dot{\eta}(t)\|dt\\
       &\geq \lambda^{-1}d(\eta(0),\eta(T))\\
       &=\lambda^{-1}d(\gamma(0),\psi_T^{-1}(\gamma(T))).
    \end{align*}
\end{proof}
As a corollary we have
\begin{prp}\label{PrpDistEn}[The energy distance inequality]
On a compact manifold there is a constant $c$ such that, for any loop $\gamma$ we have
\begin{equation}
d(\gamma(0),\psi_1(\gamma(0))<c\|\gamma'-V_t\circ\gamma\|^2_{L^2},
\end{equation}
and, if the flow is time independent, for any path $\gamma$  we have
\begin{equation}\label{eqBoundL2L1}
d(\gamma(T),\psi_T(\gamma(0)))<Te^{cT}\|\gamma'-V_t\circ\gamma\|^2_{L^2}.
\end{equation}
\end{prp}
\begin{proof}
    The first part is a particular case of the second part after renaming constants.  From Cauchy-Schwartz and the previous Lemma we deduce 
    $$
    d(\gamma(0),\psi^{-1}_T(\gamma(T)))<T\|\gamma'-V_t\circ\gamma\|^2_{L^2}.
    $$
    The claim then follows by taking $$e^c=\sup_{t\in[0,1], x,y\in M}\frac{d(\psi_t(x),\psi_t(y))}{d(x,y)}.$$
\end{proof}


% \begin{prp}\label{PrpDistEn}[The energy distance inequality]
%Suppose $H$ is time independent. Let $\lambda> 0$ be a constant bounding $\|\nabla X_H\|$. Let $\gamma:[0,T]\to M$ be a smooth curve and let $p=\gamma(0)$. Write
%$$E_H(\gamma):=\int_{t\in [0,T]}\|\gamma'(t)-X_H\circ\gamma(t)\|^2.$$
%Then
%\begin{equation}\label{eqBoundL2L1}
%d_g(\gamma(T),\phi_T(p))^2\leq \frac{e^{2T\lambda}-1}{2T\lambda} E_H(\gamma),
%\end{equation}
%If we replace the assumption that $H$ is time independent everywhere and that $\|\nabla X_H\|<\lambda$ everywhere by the assumption that these hold on a region $U=H^{-1}([a,b])$ then the same estimate holds if we replace the left hand side by $$\min\{d_g(\gamma(T),\phi_T(p))^2, cd(\gamma(T),\partial U)\}$$ for $c$ depending only on the bounds on $H,J$ in the region $U$. Finally, we can relax the assumption that $H$ is time independent with the assumption that it is sufficiently close to time independent.
%\end{prp}
%\begin{proof}
%See Lemmas 6.16, 6.17, and Remark 6.18 in \cite{Groman}. 
%\end{proof}



\section{Relative symplectic cohomology}\label{SecAcceleration}


 
To set ideas, \emph{an acceleration datum for $K\subset M$} is a family of pairs $\tau\mapsto(H_{\tau},J_{\tau})$ parametrized by $\tau\in\bR$ of time dependent Floer data together with  a smooth function $f:\bR\to\bR$ such that $f(t)=0$ whenever $t\leq 0$ and $f(t)=1$ whenever $t\geq 1$.  These are required to satisfy
\begin{itemize}
\item $H$ is monotone in $\tau$. That is, for all $(t,x)\in  S^1\times M$ and for any pair $\tau\leq\tau'\in\bR$ we have $H_{\tau}(t,x)\leq H_{\tau'}(t,x)$. 
\item For each $i,$ the pair $(H_i,J_i)$ is regular for the definition of Floer cohomology and is dissipative. 
\item For each $i$ the continuation by the family
$$g_i:s\mapsto (H_{i+f(s)},J_{i+f(s)})$$
is regular for the definition of continuation maps and is dissipative.
\end{itemize}
To each $i$  we associate the Floer cohomology group $CF^*(H_i,J_i)$. 
To the family $(H_{g_i(s)},J_{g_i(s)})$ interpolating between $(H_i,J_i)$ and $(H_{i+1},J_{i+1})$ we associate the continuation map $f_i:CF^*(H_i,J_i)\to CF^*(H_{i+1},J_{i+1})$ which commutes with the differential. These are all packaged together into a construction called the \emph{completed telescope} defined as follows.  
 
 
From the acceleration datum $(H_\tau,J_\tau)$ we obtain a $1$-ray of chain complexes over $\Lambda_{\geq 0}$: $$\mathcal{C}(H_\tau):= CF(H_1)\to CF(H_2)\to\ldots. $$
We omit $J$ from the notation whenever there is no fear of confusion.
We define relative symplectic cochain complex by taking the degree-wise completion of the telescope of $\mathcal{C}(H_\tau)$: $$SC_M^*(K,H_\tau):=\widehat{tel}(\mathcal{C}(H_\tau)).$$ Here the telescope is defined as $$tel(\mathcal{C}))=\left(\bigoplus_{i=1}^\infty\oplus C_i[q]\right)$$ with $q$ a degree $1$ variable satisfying $q^2=0$.  The differential is as depicted below \begin{align}\label{teles}
\xymatrix{
C_1\ar@{>}@(ul,ur)^{d }  &C_2\ar@{>}@(ul,ur)^{d} &C_3\ar@{>}@(ul,ur)^{d}\\
C_1[1]\ar@{>}@(dl,dr)_{-d} \ar[u]^{\text{id}}\ar[ur]^{f_1} &C_2[1]\ar@{>}@(dl,dr)_{-d} \ar[u]^{\text{id}}\ar[ur]^{f_2}&\ldots\ar[u]^{\text{id}}_{\ldots} }
\end{align}
and is given by the formula
\begin{equation}\label{eqTelDiff}
\delta qa:=qda+(-1)^{deg(a)}(f_i(a)-a).
\end{equation}

The completion is defined as follows. We assign a non-Archimedean norm to each Floer cohomology $CF^*(H_i,J_i)$ by assigning 
\begin{itemize}
\item norm $1$ to each orbit, 
\item norm $e^{-1}$ to the formal Novikov parameter $T,$ and,
\item norm $1$ to the formal variable $q$. 
\end{itemize}
We consider the Cauchy completion $\widehat{tel}(\mathcal{C}(H_\tau))$ of ${tel}(\mathcal{C}(H_\tau))$ with respect to this norm. For the more algebraically minded reader, completion is a functor $Mod(\Lambda_{\geq 0})\to Mod(\Lambda_{\geq 0})$ defined by \begin{align}
A\mapsto \widehat{A}:\lim_{\xleftarrow[r\geq 0]{}}A\otimes_{\Lambda_{\geq 0}}\Lambda_{\geq 0}/\Lambda_{\geq r}
\end{align} on objects, and by functoriality of inverse limits on the morphisms.

The completion functor automatically extends to a functor $Ch(\Lambda_{\geq 0})\to Ch(\Lambda_{\geq 0})$. Namely, if $(C,d)$ is a chain complex over $\Lambda_{\geq 0}$, then the completion $(\widehat{C},\widehat{d})$ is obtained by applying the completion functor to each graded piece of the underlying graded module, and also to the maps $d_i:C^i\to C^{i+1}$.
\bibliographystyle{plain}
\bibliography{ref}
\end{document}
