\section{Finding large monochromatic rectangles} \label{sec:monochrom-rect}

In this section, we show that the assumption for applying the communication protocol from \cref{sec:comm-protocol} does indeed hold. That is, any bounded integral matrix contains a sufficiently large monochromatic rectangle.

We first reduce the problem of finding large monochromatic rectangles to finding \emph{almost-monochromatic} rectangles. 
We say a rectangle is \emph{$(1-\eps)$-monochromatic} if at least a $(1-\eps)$-fraction of its entries have the same value.

\begin{lemma} \label{lem:almost-monochrom-rec-reduction}
	Suppose $\mm \in \R^{A \times B}$ has rank $r \geq 1$, and is $(1 - \frac{1}{16r})$-monochromatic with color $\alpha$.
	Then $\mm$ contains a monochromatic rectangle of size $\geq \frac{|A||B|}{8}$.
\end{lemma}
\begin{proof}
	Let us call a column of $\mm$ \emph{bad} if it contains at least $\frac{1}{8r} |A|$-many non-$\alpha$ entries.
	By Markov's inequality, at most half the columns are bad.
	Let $B' \subseteq B$ be the remaining good columns, where each contains at least $(1-\frac{1}{8r}) |A|$-many $\alpha$ entries.
	Let $B'' \subseteq B'$ be a maximal set of linearly independent good columns. 
	We know $|B''| \leq r$ as the rank of $\mm$ is $r$.
	
	Let $\mmu = \mm[A,B'']$. 
	Since each column in $B''$ contains at most $\frac{1}{8r} |A|$-many non-$\alpha$ entries, 
	there are at most $r \cdot \frac{1}{8r} |A|$ rows of $\mmu$ that contain non-$\alpha$ entries.
	Let $A'$ denote the $|A| - \frac{1}{8}|A| \geq \frac{1}{2} |A|$ rows of $\mmu$ that contain only $\alpha$ entries.
	So $\mm[A',B'']$ contains only $\alpha$ entries.
	
	Let $\mt = \mm[A',B']$. Since the columns in $B'$ are linear combinations of columns in $B''$, each column of $\mt$ must be of the form $\beta \vone$ for some $\beta$. 
	Finally, we know $|\mt| =  |A'||B'| \geq \frac{1}{4} |A||B|$, and there are at most $\frac{1}{16r}|A||B|$ non-$\alpha$ entries in $\mm$ in total, so at least half the columns of $\mt$ must have value $\alpha$.
\end{proof}

Now, it remains to show that we can find large almost-monochromatic rectangles.
\begin{lemma} \label{lem:almost-monochrom-rects}
	Let $\mm \in \{0,1,\dots, \Delta\}^{A \times B}$ be a bounded integral matrix of rank $r$.
	Then $\mm$ contains a $(1-\frac{1}{16r})$-monochromatic rectangle of size at least
	$|A||B| \cdot \exp(-\Delta^{O(\Delta)} \cdot \sqrt{r} \cdot \log{r})$.
\end{lemma}

To prove \cref{lem:almost-monochrom-rects}, we first define a distribution $\cD$ over the rectangles of $\mm$, and
then use the probabilistic method with respect to $\cD$ to show that there exists a large enough almost-monochromatic rectangle.
We begin with the technical ingredients:

\begin{definition}[Factorization norm]
	For a matrix $\mm \in \R^{A \times B}$, define its \emph{$\gamma_2$-norm} as
	\begin{align*}
		\gamma_2(\mm) \defeq \min \{ R \geq 0 &: \text{for all $a \in A, b \in B$, there exists $\vu_a, \vv_b$} \\
			 &\text{ so that $\mm_{a,b} = \inner{\vu_a, \vv_b}$ and $\norm{\vu_a}_2 \norm{\vv_b}_2 \leq R$}\}.
	\end{align*}
\end{definition}
In other words, $\gamma_2(\mm)$ gives the Euclidean length needed to factor the matrix $\mm$.
Here $\vu_a, \vu_b$ are vectors of any dimension (of course one may choose $\vu_a, \vu_b$ to have dimension $\rank(\mm)$).
The terminology $\gamma_2$-norm or factorization norm is indeed justified as $\gamma_2$ is a norm on the space of real matrices.

The following lemma is well-known:
\begin{lemma}[Lemma 4.2, \cite{linial2007complexity}] \label{lem:mm-gamma2-norm-bound}
	Any matrix $\mm \in \R^{A \times B}$ satisfies $\gamma_2(\mm) \leq \norm{\mm}_\infty \cdot \sqrt{\rank(\mm)}$.
\end{lemma}

It will be convenient for us to factor $\mm$ with vectors of the same Euclidean length, which comes at the expense of the dimension:
\begin{lemma}\label{lem:mm-factorization}
	For any matrix $\mm \in \R^{A \times B}$ and $s \geq \sqrt{\gamma_2(\mm)}$, there are vectors
	$\{\vu_a\}_{a \in A}, \{\vv_b\}_{b \in B}$ such that $\mm_{a,b} = \inner{\vu_a, \vv_b}$ and $\norm{\vu_a}_2 = \norm{\vv_b}_2 = s$ for all $a \in A$ and $b \in B$.
\end{lemma}
\begin{proof}
	Construct vectors $\vu_a, \vv_b$ of length $\norm{\vu_a}_2, \norm{\vv_b}_2 \leq s$ with $\mm_{a,b} = \inner{\vu_a, \vv_b}$ using \cref{lem:mm-gamma2-norm-bound}.
	Then add $|A| + |B|$ new coordinates, where each $\vu_a$ and $\vv_b$ receives a ``private'' coordinate. Set the private coordinate of $\vu_a$ to $\sqrt{s^2 - \norm{\vu_a}_2^2}$, and similarly for $\vv_b$.
\end{proof}

We denote $N^n(0,1)$ as the $n$-dimensional standard Gaussian. 
Let $S^{n-1} \defeq \{ x \in \R^n : \norm{x}_2 = 1\}$ be the $n$-dimensional unit sphere.
The following argument is usually called hyperplane rounding in the context of approximation algorithms:
\begin{lemma}[Sheppard's formula] \label{lem:sheppards}
	Any vectors $\vu, \vv \in S^{n-1}$ with $\langle \vu, \vv \rangle = \alpha$ satisfy
	\[
		\Pr_{\vg \sim N^n(0,1)} \left[ \inner{\vg,\vu} \geq 0 \text{ and } \inner{\vg,\vv} \geq 0 \right] = h(\alpha) \defeq \frac{1}{2} \left( 1 - \frac{\arccos(\alpha)}{\pi} \right).
	\]
	More generally, any vectors $\vu, \vv \in \R^{n} \setminus \{\vzero\}$ satisfy
	\[
	\Pr_{\vg \sim N^n(0,1)} \left[ \inner{\vg, \vu} \geq 0 \text{ and } \inner{\vg,\vv} \geq 0 \right] = h\left(\frac{\inner{\vu,\vv}}{\norm{\vu}_2 \norm{\vv}_2}\right).
	\]
\end{lemma}

We can now define a suitable distribution over the rectangles of $\mm$ using the above tools. In particular, 
we want the probability that a rectangle contains an entry to be a function of the entry value.

\begin{lemma} \label{lem:rect-distribution}
	Let $\mm \in \{0,1, \dots, \Delta\}^{A \times B}$ be a bounded integral matrix of rank $r$. 
	Then for any $k$, there is a distribution $\mathcal{D}_k$ over the rectangles of $\mm$ so that for all $a \in A$ and $b \in B$,
	\[
		\Pr_{\mr \sim \mathcal{D}_k} \left[ (a,b) \in \mr \right] = \left(h \left( \frac{\mm_{a,b}}{\Delta \sqrt{r}} \right) \right)^k,
	\]
	where $h$ is the function defined in \cref{lem:sheppards}.
\end{lemma}
\begin{proof}
	We use \cref{lem:mm-factorization} to factor $\mm$, which gives vectors $\vu_a, \vv_b$ for all $a \in A$ and $b \in B$, such that $\mm_{a,b} = \inner{\vu_a, \vv_b}$, and $\norm{\vu_a}_2 = \norm{\vv_b}_2 = \Delta^{1/2} r^{1/4}$.
	Then we sample independent Gaussians $\vg_1, \dots, \vg_k \sim N^n(0,1)$ and set
	\[
		\mr_i \defeq \{ a \in A : \inner{\vu_a, \vg_i} \geq 0 \} \times \{ b \in B : \inner{\vv_b, \vg_i} \geq 0 \},
	\]
	and then $\mr \defeq \mr_1 \cap \dots \cap \mr_k$. Note that $\mr$ is indeed a rectangle.
	For each $i$ and each $(a,b) \in A \times B$ we have
	\[
		\Pr [ (a,b) \in \mr_i ] = h\left( \frac{\inner{\vu_a, \vv_b}}{\norm{\vu_a}_2 \norm{\vv_b}_2} \right) = h \left( \frac{\mm_{a,b}}{\Delta \sqrt{r}}\right).
	\]
	The overall expression follows by independence of the $k$ rectangles.
\end{proof}

Finally, we use the above distribution to show the existence of large almost-monochromatic rectangles.

\begin{proof}[Proof of \cref{lem:almost-monochrom-rects}]
	Let $\cE_0 \;\dot\cup \cdots \dot\cup \; \cE_{\Delta}$ be the partition of the entries $A \times B$ based on entry values, so that $\cE_j \defeq \{(a,b) \in A \times B : \mm_{a,b} = j\}$.
	Let $m_j \defeq (64 r \Delta)^{(8\Delta)^j}$ for each $j = 0, \dots, \Delta$,
	and let $i$ be the index such that $m_i \cdot |\cE_i|$ is maximized.
	From this, we also get
	\begin{equation} \label{eq:Ei-bound}
		|\mm| = |A| \cdot |B| = \sum_{j=0}^\Delta |\cE_j| \leq \sum_{j=0}^\Delta \frac{m_i}{m_j} |\cE_i| \leq m_i |\cE_i|. 
	\end{equation}
	
	For notational convenience, 
	recall $h(\alpha) \defeq  \frac12 \left( 1 - \frac{\arccos(\alpha)}{\pi} \right)$, and let $c(j) \defeq h \left( \frac{j}{\Delta \sqrt{r}} \right)$.
	We observe that on $[0,1]$, the function $h$ is convex, monotone increasing, lowerbounded by $h(0) = 1/4$, upperbounded by $h(1) = 1/2$, and $h'(\alpha) = \frac{1}{2\pi \sqrt{1-\alpha^2}} \geq \frac{1}{2\pi}$.
	Additionally, the following claim about $c$ will be useful for our calculations later:
	\begin{claim} \label{lem:c-bounds}
		$\frac{c(j)}{c(j-1)} \geq 1 + \frac{4}{3 \pi \Delta \sqrt{r}}$ for $1 \leq j \leq \Delta$.
		Also, 
		$\frac{c(\Delta)}{c(0)} \leq 1 + \frac{4}{\pi \sqrt{r}}.$
	\end{claim}
	\begin{proof}
		We may assume $r \geq 2$. We use first order approximations for $c$. 
		For the first inequality, we also use the fact that $c(j-1) \leq 3/8$:
		\begin{align*}
			\frac{c(j)}{c(j-1)} &\geq \frac{c(j-1) + c'(j-1)}{c(j-1)} \geq 1 + \frac{1}{2 \pi \Delta \sqrt{r} \cdot c(j-1)} \geq 1 + \frac{4}{3\pi \Delta \sqrt{r}}.
		\end{align*}
		For the second inequality, we have
		\begin{align*}
			\frac{c(\Delta)}{c(0)} \leq \frac{c (0) + \Delta \cdot c'(\Delta) }{c(0)} = 1 + 4 \Delta \cdot c'(\Delta) =
			1 + 4 \Delta \frac{1}{\Delta \sqrt{r}} \frac{1}{2\pi \sqrt{1 - 1/r}} \leq 1 + \frac{4}{\pi \sqrt{r}}.
		\end{align*}
	\end{proof}
	
	Next, let $\mathcal{D}_k$ be the distribution from \cref{lem:rect-distribution}, and generate $\mr \sim \mathcal{D}_k$
	for some choice of $k$ to be determined.
	We will show that there exists a $k$ such that $\mr$ is expected to be $(1-\frac{1}{16r})$-monochromatic with color $i$, and is sufficiently large.
	Specifically, the number of $i$-entries in $\mr$ is greater than the number all other entries in $\mr$ by a factor of $16r$ in expectation, and moreover, this difference is sufficiently large, which in turn means $\mr$ is sufficiently large.
	\begin{align*}
		&\phantom{{}={}} \E_{\mr \sim \mathcal{D}_k} \left[ |\cE_i \cap \mr| - 16r \sum_{j \neq i} |\cE_j \cap \mr| \right] \\
		&= | \cE_i | \cdot c(i)^k - 16r \sum_{j \neq i} c(j)^k |\cE_j| \tag{by \cref{lem:rect-distribution}}\\
		&\geq  | \cE_i | \cdot c(i)^k \left( 1 -  16r \sum_{j \neq i} \frac{c(j)^k m_i }{c(i)^k m_j} \right)
		\intertext{
			Suppose $\frac{c(j)^k m_i}{ c(i)^k m_j} \leq \frac{1}{64r \Delta}$ for each $j \neq i$, then we can conclude}
		&\geq | \cE_i | \cdot 4^{-k} (1 - \frac14) \tag{Since $c(i) \geq \frac14$}\\
		&\geq \frac{|A| |B|}{m_i} \frac{1}{2 \cdot 4^k}. \tag{by \cref{eq:Ei-bound}}
	\end{align*}

	\begin{claim}
		There exists a choice of $k$ such that $\frac{c(j)^k m_i}{ c(i)^k m_j} \leq \frac{1}{64r \Delta}$ for all $j \neq i$.
	\end{claim}
	\begin{proof}
		We consider two cases:
		\begin{enumerate}[(1)]
			\item $0 \leq j < i$: In this case we can bound
			$\frac{c(j)^k m_i}{ c(i)^k m_j} \leq \left(\frac{c(i-1)}{c(i)} \right)^{k} \frac{m_i}{m_0}$.
			To get our claim, it suffices to choose $k$ to satisfy
			\begin{align*}
				\left(\frac{c(i-1)}{c(i)} \right)^{k} \frac{m_i}{m_0} &\leq \frac{1}{64r\Delta}  \\
				\Leftarrow \hspace{6em} k  &\geq \frac{((8 \Delta)^i+1) \log (64 r \Delta)}{\log \frac{c(i)}{c(i-1)}}.
			\end{align*} 
			Using the lower bound from \cref{lem:c-bounds}, along with $\log(1+x) \geq x/2$ for $x \leq 1$, we conclude it suffices to choose $k$ to satisfy
			\begin{equation}\label{eq:k-lowerbound}
			k \geq ((8 \Delta)^i+1) \log (64 r \Delta) \cdot \frac{3}{2} \pi \Delta \sqrt{r}.
			\end{equation}
			\item $i < j \leq \Delta$: In this case we can bound
			$\frac{c(j)^k m_i}{ c(i)^k m_j} \leq \left(\frac{c(\Delta)}{c(0)} \right)^{k} \frac{m_i}{m_{i+1}}$.
			To get our claim, it suffices to choose $k$ to satisfy
			\begin{align*}
				\left(\frac{c(\Delta)}{c(0)} \right)^{k} \frac{m_i}{m_{i+1}} &\leq \frac{1}{64r \Delta} \\
				\Leftarrow \hspace{6em} k &\leq \frac{((8 \Delta)^{i+1} - (8\Delta)^i -1) \log (64r \Delta)}{ \log \frac{c(\Delta)}{c(0)}}.
			\end{align*}
			Using the upper bound from \cref{lem:c-bounds}, we conclude it suffices to choose $k$ to satisfy
			\begin{equation} \label{eq:k-upperbound}
			k \leq ((8\Delta)^{i+1} - (8\Delta)^{i}-1) \log (64 r \Delta) \cdot \frac{\pi \sqrt{r}}{4}.
			\end{equation}
		\end{enumerate}
		To choose $k$ to simultaneously satisfy the two cases, we first verify that the lower and upper bound for $k$ in \cref{eq:k-lowerbound} and \cref{eq:k-upperbound} are consistent. 
		Indeed when $i \neq 0$, we have
		\[
			\frac{3}{2} ((8 \Delta)^i+1) \Delta \leq \frac14 ((8\Delta)^{i+1} - (8\Delta)^{i}-1),
		\]
		so we may choose $k$ to be equal to the lower bound.
		If $i = 0$, then the lower bound from \cref{eq:k-lowerbound} does not apply, so we choose $k$ to be equal to the upper bound. 
	\end{proof}
	For any established choice of $k$, we always have $k \leq (8 \Delta)^{\Delta} \log(64 r \Delta) \pi \Delta \sqrt{r}$.
	Moreover, we have $\log m_i \leq (8 \Delta)^\Delta \log(64 r \Delta)$.
	Now we can complete the lower bound on the expected size of $\mr$:
	\begin{align*}
	\E_{\mr \sim \mathcal{D}_k} \left[ |\mr| \right] &\geq \E_{\mr \sim \mathcal{D}_k} \left[ |\cE_i \cap \mr| - 16r \sum_{j \neq i} |\cE_j \cap \mr| \right] \\
	&\geq \frac{|A| |B|}{m_i} \frac{1}{2 \cdot 4^k} \\
	&\geq \exp(-\Delta^{O(\Delta)}\cdot  \sqrt{r} \cdot \log r) |A| |B|,
	\end{align*}
	as claimed.
\end{proof}