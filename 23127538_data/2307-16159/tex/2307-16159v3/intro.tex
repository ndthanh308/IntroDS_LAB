\section{Introduction} \label{sec:intro}

In classical communication complexity, two players, Alice and Bob, are given a Boolean function $f : A \times B \mapsto \{0,1\}$, as well as separate inputs $a \in A$ and $b \in B$,
and wish to compute $f(a,b)$ while minimizing the total amount of communication. 
Alice and Bob have unlimited resources for pre-computations and agree on a deterministic \emph{communication protocol} to compute $f$ before receiving their respective inputs.
The \emph{length} of their protocol is defined to be the maximum number of bits exchanged over all possible inputs.
The \emph{deterministic communication complexity} of $f$, denoted by $CC^{\det}(f)$,
is the minimum length of a protocol to compute $f$.

A major open problem in communication complexity is the \emph{log-rank conjecture} proposed by Lov{\'a}sz and Saks \cite{lovasz1988lattices}, which asks if $CC^{\det}(f) \leq (\log {r})^{O(1)}$ for all Boolean functions $f$ of rank $r$,
where the \emph{rank} of a two-party function $f$ on $A \times B$ is defined to be the rank of the matrix $\mm \in \R^{A \times B}$ with $\mm_{a,b} = f(a,b)$ for all $(a, b) \in A \times B$.
The best upper bound currently known is due to Lovett \cite{lovett2016}, who showed $CC^{\det}(f) \leq O(\sqrt{r} \log r)$ using discrepancy theory techniques.

In this work, we obtain similar deterministic communication complexity bounds for a larger class of functions:

\begin{theorem}[Main result, communication complexity] \label{thm:main}
	Let $f : A \times B \mapsto \{0,  1, \dots, \Delta\}$ be a bounded integral function of rank $r$.
	Then there exists a deterministic communication protocol to compute $f$ with length at most $\Delta^{O(\Delta)} \cdot \sqrt{r} \cdot \log r$ bits.
\end{theorem}

The function $f$ can be directly viewed as a non-negative matrix $\mm \in \{0, \dots, \Delta\}^{A \times B}$ of rank $r$.
We use the matrix representation exclusively in the remainder of this paper.
Let us adopt the convention that a \emph{rectangle} in $\mm$ is a (non-contiguous) submatrix $\mm[A',B']$ indexed by some $A' \subseteq A$ and $B' \subseteq B$. A rectangle is \emph{monochromatic} with color $i$ if all entries in the rectangle have value $i$.

The \emph{non-negative rank} of a non-negative matrix $\mm$, denoted by $\nnr(\mm)$, is defined as the minimum $r$ 
%where there exists \emph{non-negative} vectors $\vu_1, \dots, \vu_r \in \mathbb{R}_{\geq 0}^A$ and $\vv_1, \dots, \vv_r \in \mathbb{R}_{\geq 0}^B$  such that $\mm = \sum_{i=1}^r \vu_i \vv_i^\top$; that is, 
such that $\mm$ can be written as the sum of $r$ non-negative rank-1 matrices,
or equivalently, as $\mm = \mmu \mv$ for non-negative matrices $\mmu \in \R^{A \times r}_{\geq 0}$ and $\mv \in \R^{r \times B}_{\geq 0}$.
It is straightforward to see $\nnr(\mm) \leq 2^{CC^{\det}(\mm)}$ (c.f. Rao and Yehudayoff \cite{rao2020communication}, Chapter 1, Theorem 1.6):
The \emph{protocol tree} to compute $\mm$ has at most $2^{CC^{\det}(\mm)}$ leaves,
each corresponding to a monochromatic rectangle of $\mm$. 
These rectangles are disjoint over all leaves, and their union is $\mm$. 
Since a monochromatic rectangle is a non-negative matrix of rank 0 or 1, 
we conclude that $\mm$ can be written as a sum of at most $2^{CC^{\det}(\mm)}$ rank-1 non-negative matrices.
The \emph{positive semidefinite rank} of $\mm$, denoted by $\rank_{\mathrm{psd}}(\mm)$, generalizes non-negative rank and has an analogous relationship to quantum communication complexity~\cite{fawzi2015positive}. It is defined as the minimum $r$ such that there are positive semidefinite matrices $\mmu_1, \dots \mmu_A$ and $\mv_1,\dots, \mv_B$ of dimension $r \times r$ satisfying $\mm_{i,j} = \mathrm{Tr}(\mmu_i \mv_j)$ for all $i, j$; trivially, $\rank_{\mathrm{psd}}(\mm) \leq \rank_+(\mm)$. Barvinok~\cite{barvinok2012approximations} showed that if $\mm$ has at most $k$ distinct entries, as is the setting studied in this paper, then $\rank_{\mathrm{psd}}(\mm) \leq {k-1 + \rank(\mm) \choose k-1}$.

Non-negative rank brings us to a beautiful connection with \emph{extension complexity},
introduced in the seminal work of Yannakakis \cite{yannakakis1988} in the context of writing combinatorial optimization problems as linear programs.
The \emph{extension complexity} of a polytope $\cP$, denoted $\xc(\cP)$, is defined as the minimum number of facets of some higher dimensional polytope $\cQ$ (its \emph{extended formulation}) such that there exists a linear projection of $\cQ$ to $\cP$.
%When one wishes to optimize $\cP$ which has a large number of facets but a smaller extended formulation $\cQ$, it can be useful to optimize more efficiently over $\cQ$ and project the solution back.
%One such example is the spanning tree polytope $\cP_{ST}$ on the complete graph with $n$ vertices,
%which is known to have $2^{\Omega(n)}$ facets \cite{edmonds1971matroids} but admits an extended formulation with $O(n^3)$ facets \cite{martin1991using}.

A foundational theorem from \cite{yannakakis1988} states that $\xc(\cP) = \nnr(\ms)$, where $\ms$ is the \emph{slack matrix} of $\cP$, defined as follows:
Suppose $\cP$ has facets $\cF$ and vertices $\cV$. Then $\ms$ is a non-negative $\cF \times \cV$ matrix, where the $(f,v)$-entry indexed by facet $f \in \cF$ defined by the halfspace $a^\top x \leq b$ and vertex $v \in \cV$ has value $\ms_{f,v} = b - a^\top v$.
(A facet may be defined by many equivalent halfspaces, and therefore the slack matrix is not unique; the result holds for all valid slack matrices.)
In fact, \cite{yannakakis1988} showed that a factorization of $\ms$ with respect to non-negative rank gives an extended formulation of $\cP$ and vice versa. For a comprehensive preliminary survey, see \cite{conforti2010extended}.
A number of breakthrough results in extended complexity in recent years emerged from lower-bounding the non-negative rank of the slack matrix for specific polytopes, such as the \textsc{TSP}, \textsc{Cut}, and \textsc{Stable-Set} polytopes by Fiorini-Massar-Pokutta-Tiwary-De Wolf \cite{fiorini2015exponential}, and the \textsc{Perfect-Matching} polytope by Rothvoss \cite{rothvoss2017matching}.

Connecting extension complexity to deterministic communication complexity via non-negative rank, we have:

\begin{corollary}[Main result, extension complexity] \label{cor:main-xc}
	Let $\cP$ be a $n$-dimensional polytope that admits slack matrix $\ms$. 
	Suppose the entries of $\ms$ are integral and bounded by $\Delta$.
	Then the extension complexity of $\cP$ is at most $\exp( \Delta^{O(\Delta)} \cdot \sqrt{n} \cdot \log n)$.
\end{corollary}
\begin{proof}
	Since $\cP$ is $n$-dimensional, we know $\ms$ has rank at most $n$, as there are at most $n$ linearly independent vertices of $\cP$, and the slack of a vertex with respect to all the facets is a linear function.
	We combine the theorem of \cite{yannakakis1988} and \cref{thm:main} for the overall conclusion.
\end{proof}

Our proof follows the approach discussed in the note of Rothvoss  \cite{rothvoss2014direct} simplifying Lovett's result. 
The following lemma shows that there are indeed concrete polytopes for which our result applies:

\begin{lemma} \label{lem:sandwich}
	Suppose there are polytopes $\cP \subseteq \cQ \subseteq \R^n$ where $\cP = \mathrm{conv}\{\vx_1, \dots, \vx_v\}$
	and $\cQ = \{ \vx \in \R^n : \ma \vx \leq \vb\}$ with $\ma \in \R^{f \times n}$.
	Suppose the partial slack matrix $\ms \in \R^{f \times v}$ with $\ms_{i,j} = \vb_i - \ma_i \vx_j$ for $i \in [f], j \in [v]$ is integral and bounded by $\Delta$.
	Then there exists a polytope $\cK$ with extension complexity at most $\exp(\Delta^{O(\Delta)} \cdot \sqrt{n} \cdot \log n)$ so that $\cP \subseteq \cK \subseteq \cQ$. 
\end{lemma}
\begin{proof}
	From above, we have $\nnr(\ms) \leq \exp(\Delta^{O(\Delta)} \cdot \sqrt{n} \cdot \log n)$.
	Moreover, it is well-known that $s \defeq \nnr(\ms)$ is the extension complexity of some polytope $\cK$ such that $\cP \subseteq \cK \subseteq \cQ$. (It is in fact the minimum extension complexity over all such sandwiched polytopes.) The conclusion follows.
	
	For completeness, we show the latter fact: 
	Suppose $\ms = \mmu \mv$ is a non-negative factorization of $\ms$ with $\mmu \in \R_{\geq 0}^{f \times s}$ and $\mv \in \R^{s \times v}_{\geq 0}$. Let $\cK^{\mathrm{lift}} \defeq \{ (\vx, \vy) \in \R^{n + s} : \ma \vx + \mmu \vy = \vb, \vy \geq \vzero\}$, and let $\cK$ be the projection of $\cK^{\mathrm{lift}}$ onto the first $n$ coordinates. 
	It is immediately clear that $\cK \subseteq \cQ$, and $\xc(\cK) \leq s$ by definition.
	For each $j \in [v]$, the point $\vx_j$ satisfies $\ma \vx_j + \ms^j = \vb$, where $\ms^j$ is the $j$-th column of $\ms$ given by $\mmu \mv \ve_j$. It follows that $(\vx_j, \mv \ve_j) \in \cK^{\mathrm{lift}}$, so $\vx_j \in \cK$. As this holds for each $\vx_1, \dots, \vx_v$, we conclude $\cP \subseteq \cK$. 
	
	%To see that $s$ is the minimum extension complexity, let $\cK$ be any sandwiched polytope with vertex set $\cV$ and facets $\cF$, and let $\ms' \in \R^{\cF \times \cV}$ be its slack matrix, so that $\xc(\cK) = \nnr(\ms')$.
	%Let $\ms''$ denote the resulting slack matrix obtained from $\ms'$ by adding rows indexed by $\ma_i \vx \leq \vb_i$ for each $i \in [f]$, and columns indexed by $\vx_j$ for each $j \in [v]$. 
	%Since $\cK \subseteq \cQ$, each facet $\ma_i \vx \leq \vb_i$ is a redundant constraint for $\cK$, so can be written as conic combinations of facets from $\cF$, and therefore the additional rows do not increase the non-negative rank.
	%Similarly, since $\cP \subseteq \cK$, each point $\vx_j$ can be written as a convex combination of points in $\cV$, so the additional columns also do not increase the non-negative rank. 
	%Then we have $\nnr(\ms') = \nnr(\ms'')$.
	%Since $\ms$ is a submatrix of $\ms''$, we have $s = \nnr(\ms) \leq \nnr(\ms'')$.
\end{proof}

We give a direct example in a combinatorial optimization setting:
Consider the \textsc{$k$-Set-Packing} problem, where we are given a collection of $n$ sets $S_1, \dots, S_n \subseteq [N]$ with $N \gg n$,
and want to find a maximum subcollection such that each element $j \in [N]$ is contained in at most $k$ sets. 
The \textsc{$k$-Set-Packing} polytope $\cP$ is is the convex hull of all feasible subcollections of sets, given by
\[
	\cP = \mathrm{conv}\Big\{ x \in \{0,1\}^n : \sum_{i : j \in S_i} x_i \leq k \;\; \forall j \in [N]\Big\}.
\]
Its natural LP relaxation $\cQ$ is
\[
	\cQ = \Big\{ x \in [0,1]^n : \sum_{i : j \in S_i} x_i \leq k \;\; \forall j \in [N]\Big\}.
\]
In the regime where $N \gg n$, a priori, the extension complexity of $\cP$ and $\cQ$ could be as large as $N$. 
But interestingly, let $\ms$ be the partial slack matrix with respect to $\cP$ and $\cQ$ as defined in \cref{lem:sandwich}.
Then $\ms$ contains integral values in $\{0, \dots, k\}$, and so we conclude there exists a sandwiched polytope $\cP  \subseteq \cK \subseteq \cQ$ with $\xc(\cK) \leq \exp(k^{O(k)} \cdot \sqrt{n} \cdot \log n)$. 
