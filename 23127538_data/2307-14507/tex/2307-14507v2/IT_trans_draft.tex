
%% bare_jrnl.tex
%% V1.4b
%% 2015/08/26
%% by Michael Shell
%% see http://www.michaelshell.org/
%% for current contact information.
%%
%% This is a skeleton file demonstrating the use of IEEEtran.cls
%% (requires IEEEtran.cls version 1.8b or later) with an IEEE
%% journal paper.
%%
%% Support sites:
%% http://www.michaelshell.org/tex/ieeetran/
%% http://www.ctan.org/pkg/ieeetran
%% and
%% http://www.ieee.org/x

%%*************************************************************************
%% Legal Notice:
%% This code is offered as-is without any warranty either expressed or
%% implied; without even the implied warranty of MERCHANTABILITY or
%% FITNESS FOR A PARTICULAR PURPOSE! 
%% User assumes all risk.
%% In no event shall the IEEE or any contributor to this code be liable for
%% any damages or losses, including, but not limited to, incidental,
%% consequential, or any other damages, resulting from the use or misuse
%% of any information contained here.
%%
%% All comments are the opinions of their respective authors and are not
%% necessarily endorsed by the IEEE.
%%
%% This work is distributed under the LaTeX Project Public License (LPPL)
%% ( http://www.latex-project.org/ ) version 1.3, and may be freely used,
%% distributed and modified. A copy of the LPPL, version 1.3, is included
%% in the base LaTeX documentation of all distributions of LaTeX released
%% 2003/12/01 or later.
%% Retain all contribution notices and credits.
%% ** Modified files should be clearly indicated as such, including  **
%% ** renaming them and changing author support contact information. **
%%*************************************************************************


% *** Authors should verify (and, if needed, correct) their LaTeX system  ***
% *** with the testflow diagnostic prior to trusting their LaTeX platform ***
% *** with production work. The IEEE's font choices and paper sizes can   ***
% *** trigger bugs that do not appear when using other class files.       ***                          ***
% The testflow support page is at:
% http://www.michaelshell.org/tex/testflow/



\documentclass[journal, romanappendices]{IEEEtran}
%
% If IEEEtran.cls has not been installed into the LaTeX system files,
% manually specify the path to it like:
% \documentclass[journal]{../sty/IEEEtran}





% Some very useful LaTeX packages include:
% (uncomment the ones you want to load)


% *** MISC UTILITY PACKAGES ***
%
%\usepackage{ifpdf}
% Heiko Oberdiek's ifpdf.sty is very useful if you need conditional
% compilation based on whether the output is pdf or dvi.
% usage:
% \ifpdf
%   % pdf code
% \else
%   % dvi code
% \fi
% The latest version of ifpdf.sty can be obtained from:
% http://www.ctan.org/pkg/ifpdf
% Also, note that IEEEtran.cls V1.7 and later provides a builtin
% \ifCLASSINFOpdf conditional that works the same way.
% When switching from latex to pdflatex and vice-versa, the compiler may
% have to be run twice to clear warning/error messages.






% *** CITATION PACKAGES ***
%
%\usepackage{cite}
% cite.sty was written by Donald Arseneau
% V1.6 and later of IEEEtran pre-defines the format of the cite.sty package
% \cite{} output to follow that of the IEEE. Loading the cite package will
% result in citation numbers being automatically sorted and properly
% "compressed/ranged". e.g., [1], [9], [2], [7], [5], [6] without using
% cite.sty will become [1], [2], [5]--[7], [9] using cite.sty. cite.sty's
% \cite will automatically add leading space, if needed. Use cite.sty's
% noadjust option (cite.sty V3.8 and later) if you want to turn this off
% such as if a citation ever needs to be enclosed in parenthesis.
% cite.sty is already installed on most LaTeX systems. Be sure and use
% version 5.0 (2009-03-20) and later if using hyperref.sty.
% The latest version can be obtained at:
% http://www.ctan.org/pkg/cite
% The documentation is contained in the cite.sty file itself.






% *** GRAPHICS RELATED PACKAGES ***
%
\ifCLASSINFOpdf
  % \usepackage[pdftex]{graphicx}
  % declare the path(s) where your graphic files are
  % \graphicspath{{../pdf/}{../jpeg/}}
  % and their extensions so you won't have to specify these with
  % every instance of \includegraphics
  % \DeclareGraphicsExtensions{.pdf,.jpeg,.png}
\else
  % or other class option (dvipsone, dvipdf, if not using dvips). graphicx
  % will default to the driver specified in the system graphics.cfg if no
  % driver is specified.
  % \usepackage[dvips]{graphicx}
  % declare the path(s) where your graphic files are
  % \graphicspath{{../eps/}}
  % and their extensions so you won't have to specify these with
  % every instance of \includegraphics
  % \DeclareGraphicsExtensions{.eps}
\fi
% graphicx was written by David Carlisle and Sebastian Rahtz. It is
% required if you want graphics, photos, etc. graphicx.sty is already
% installed on most LaTeX systems. The latest version and documentation
% can be obtained at: 
% http://www.ctan.org/pkg/graphicx
% Another good source of documentation is "Using Imported Graphics in
% LaTeX2e" by Keith Reckdahl which can be found at:
% http://www.ctan.org/pkg/epslatex
%
% latex, and pdflatex in dvi mode, support graphics in encapsulated
% postscript (.eps) format. pdflatex in pdf mode supports graphics
% in .pdf, .jpeg, .png and .mps (metapost) formats. Users should ensure
% that all non-photo figures use a vector format (.eps, .pdf, .mps) and
% not a bitmapped formats (.jpeg, .png). The IEEE frowns on bitmapped formats
% which can result in "jaggedy"/blurry rendering of lines and letters as
% well as large increases in file sizes.
%
% You can find documentation about the pdfTeX application at:
% http://www.tug.org/applications/pdftex





% *** MATH PACKAGES ***
%
%\usepackage{amsmath}
% A popular package from the American Mathematical Society that provides
% many useful and powerful commands for dealing with mathematics.
%
% Note that the amsmath package sets \interdisplaylinepenalty to 10000
% thus preventing page breaks from occurring within multiline equations. Use:
%\interdisplaylinepenalty=2500
% after loading amsmath to restore such page breaks as IEEEtran.cls normally
% does. amsmath.sty is already installed on most LaTeX systems. The latest
% version and documentation can be obtained at:
% http://www.ctan.org/pkg/amsmath





% *** SPECIALIZED LIST PACKAGES ***
%
%\usepackage{algorithmic}
% algorithmic.sty was written by Peter Williams and Rogerio Brito.
% This package provides an algorithmic environment fo describing algorithms.
% You can use the algorithmic environment in-text or within a figure
% environment to provide for a floating algorithm. Do NOT use the algorithm
% floating environment provided by algorithm.sty (by the same authors) or
% algorithm2e.sty (by Christophe Fiorio) as the IEEE does not use dedicated
% algorithm float types and packages that provide these will not provide
% correct IEEE style captions. The latest version and documentation of
% algorithmic.sty can be obtained at:
% http://www.ctan.org/pkg/algorithms
% Also of interest may be the (relatively newer and more customizable)
% algorithmicx.sty package by Szasz Janos:
% http://www.ctan.org/pkg/algorithmicx




% *** ALIGNMENT PACKAGES ***
%
%\usepackage{array}
% Frank Mittelbach's and David Carlisle's array.sty patches and improves
% the standard LaTeX2e array and tabular environments to provide better
% appearance and additional user controls. As the default LaTeX2e table
% generation code is lacking to the point of almost being broken with
% respect to the quality of the end results, all users are strongly
% advised to use an enhanced (at the very least that provided by array.sty)
% set of table tools. array.sty is already installed on most systems. The
% latest version and documentation can be obtained at:
% http://www.ctan.org/pkg/array


% IEEEtran contains the IEEEeqnarray family of commands that can be used to
% generate multiline equations as well as matrices, tables, etc., of high
% quality.




% *** SUBFIGURE PACKAGES ***
%\ifCLASSOPTIONcompsoc
%  \usepackage[caption=false,font=normalsize,labelfont=sf,textfont=sf]{subfig}
%\else
%  \usepackage[caption=false,font=footnotesize]{subfig}
%\fi
% subfig.sty, written by Steven Douglas Cochran, is the modern replacement
% for subfigure.sty, the latter of which is no longer maintained and is
% incompatible with some LaTeX packages including fixltx2e. However,
% subfig.sty requires and automatically loads Axel Sommerfeldt's caption.sty
% which will override IEEEtran.cls' handling of captions and this will result
% in non-IEEE style figure/table captions. To prevent this problem, be sure
% and invoke subfig.sty's "caption=false" package option (available since
% subfig.sty version 1.3, 2005/06/28) as this is will preserve IEEEtran.cls
% handling of captions.
% Note that the Computer Society format requires a larger sans serif font
% than the serif footnote size font used in traditional IEEE formatting
% and thus the need to invoke different subfig.sty package options depending
% on whether compsoc mode has been enabled.
%
% The latest version and documentation of subfig.sty can be obtained at:
% http://www.ctan.org/pkg/subfig




% *** FLOAT PACKAGES ***
%
%\usepackage{fixltx2e}
% fixltx2e, the successor to the earlier fix2col.sty, was written by
% Frank Mittelbach and David Carlisle. This package corrects a few problems
% in the LaTeX2e kernel, the most notable of which is that in current
% LaTeX2e releases, the ordering of single and double column floats is not
% guaranteed to be preserved. Thus, an unpatched LaTeX2e can allow a
% single column figure to be placed prior to an earlier double column
% figure.
% Be aware that LaTeX2e kernels dated 2015 and later have fixltx2e.sty's
% corrections already built into the system in which case a warning will
% be issued if an attempt is made to load fixltx2e.sty as it is no longer
% needed.
% The latest version and documentation can be found at:
% http://www.ctan.org/pkg/fixltx2e


%\usepackage{stfloats}
% stfloats.sty was written by Sigitas Tolusis. This package gives LaTeX2e
% the ability to do double column floats at the bottom of the page as well
% as the top. (e.g., "\begin{figure*}[!b]" is not normally possible in
% LaTeX2e). It also provides a command:
%\fnbelowfloat
% to enable the placement of footnotes below bottom floats (the standard
% LaTeX2e kernel puts them above bottom floats). This is an invasive package
% which rewrites many portions of the LaTeX2e float routines. It may not work
% with other packages that modify the LaTeX2e float routines. The latest
% version and documentation can be obtained at:
% http://www.ctan.org/pkg/stfloats
% Do not use the stfloats baselinefloat ability as the IEEE does not allow
% \baselineskip to stretch. Authors submitting work to the IEEE should note
% that the IEEE rarely uses double column equations and that authors should try
% to avoid such use. Do not be tempted to use the cuted.sty or midfloat.sty
% packages (also by Sigitas Tolusis) as the IEEE does not format its papers in
% such ways.
% Do not attempt to use stfloats with fixltx2e as they are incompatible.
% Instead, use Morten Hogholm'a dblfloatfix which combines the features
% of both fixltx2e and stfloats:
%
% \usepackage{dblfloatfix}
% The latest version can be found at:
% http://www.ctan.org/pkg/dblfloatfix




%\ifCLASSOPTIONcaptionsoff
%  \usepackage[nomarkers]{endfloat}
% \let\MYoriglatexcaption\caption
% \renewcommand{\caption}[2][\relax]{\MYoriglatexcaption[#2]{#2}}
%\fi
% endfloat.sty was written by James Darrell McCauley, Jeff Goldberg and 
% Axel Sommerfeldt. This package may be useful when used in conjunction with 
% IEEEtran.cls'  captionsoff option. Some IEEE journals/societies require that
% submissions have lists of figures/tables at the end of the paper and that
% figures/tables without any captions are placed on a page by themselves at
% the end of the document. If needed, the draftcls IEEEtran class option or
% \CLASSINPUTbaselinestretch interface can be used to increase the line
% spacing as well. Be sure and use the nomarkers option of endfloat to
% prevent endfloat from "marking" where the figures would have been placed
% in the text. The two hack lines of code above are a slight modification of
% that suggested by in the endfloat docs (section 8.4.1) to ensure that
% the full captions always appear in the list of figures/tables - even if
% the user used the short optional argument of \caption[]{}.
% IEEE papers do not typically make use of \caption[]'s optional argument,
% so this should not be an issue. A similar trick can be used to disable
% captions of packages such as subfig.sty that lack options to turn off
% the subcaptions:
% For subfig.sty:
% \let\MYorigsubfloat\subfloat
% \renewcommand{\subfloat}[2][\relax]{\MYorigsubfloat[]{#2}}
% However, the above trick will not work if both optional arguments of
% the \subfloat command are used. Furthermore, there needs to be a
% description of each subfigure *somewhere* and endfloat does not add
% subfigure captions to its list of figures. Thus, the best approach is to
% avoid the use of subfigure captions (many IEEE journals avoid them anyway)
% and instead reference/explain all the subfigures within the main caption.
% The latest version of endfloat.sty and its documentation can obtained at:
% http://www.ctan.org/pkg/endfloat
%
% The IEEEtran \ifCLASSOPTIONcaptionsoff conditional can also be used
% later in the document, say, to conditionally put the References on a 
% page by themselves.




% *** PDF, URL AND HYPERLINK PACKAGES ***
%
%\usepackage{url}
% url.sty was written by Donald Arseneau. It provides better support for
% handling and breaking URLs. url.sty is already installed on most LaTeX
% systems. The latest version and documentation can be obtained at:
% http://www.ctan.org/pkg/url
% Basically, \url{my_url_here}.




% *** Do not adjust lengths that control margins, column widths, etc. ***
% *** Do not use packages that alter fonts (such as pslatex).         ***
% There should be no need to do such things with IEEEtran.cls V1.6 and later.
% (Unless specifically asked to do so by the journal or conference you plan
% to submit to, of course. )


% correct bad hyphenation here
\hyphenation{op-tical net-works semi-conduc-tor}

\usepackage[utf8]{inputenc} 
\usepackage[T1]{fontenc}
\usepackage{url}
\usepackage{ifthen}
\usepackage{cite}
\usepackage{graphicx}
\usepackage{hyperref}
\usepackage{amsmath}
\usepackage{amssymb}
\usepackage{bm}
\usepackage{bbm}
\usepackage{algorithmicx}
\usepackage{algorithm}
\usepackage{algpseudocode}
\usepackage{array}
\usepackage{booktabs}
\usepackage{multirow}
\usepackage{makecell}
\usepackage{mathtools}
\usepackage{mathrsfs}
% \usepackage{amsthm}
\usepackage{xcolor}
\usepackage{balance}



\newtheorem{definition}{Definition}
\newtheorem{theorem}{Theorem}
\newtheorem{remark}{Remark}
\newtheorem{lemma}{Lemma}
\newtheorem{approximation}{Approximation}
\newtheorem{corollary}{Corollary}
\newtheorem{proposition}{Proposition}
\newtheorem{conjecture}{Conjecture}
\newtheorem{fact}{Fact}
\newtheorem{example}{Example}

\newcommand{\N}{\mathbb{N}}
\newcommand{\NN}{\mathcal{N}}
\newcommand{\R}{\mathbb{R}}
\newcommand{\M}{\mathcal{M}}
\newcommand{\PP}{\mathsf{P}}
\newcommand{\X}{\mathcal{X}}
\newcommand{\Y}{\mathcal{Y}}
\newcommand{\U}{\mathcal{U}}
\newcommand{\A}{\mathcal{A}}
\newcommand{\B}{\mathcal{B}}
\newcommand{\G}{\mathcal{G}}
\newcommand{\E}{\mathbb{E}}
\newcommand{\Z}{\mathbb{Z}}
\newcommand{\F}{\mathcal{F}}
\newcommand{\LL}{\mathcal{L}}
\newcommand{\Prob}{\mathbb{P}}
\newcommand{\indicator}{\mathbf{1}}
\newcommand{\setS}{\mathcal{S}}
\newcommand*\diff{\mathop{}\!\mathrm{d}}
\newcommand\norm[1]{\Vert{#1}\Vert}
\newcommand{\Brace}[1]{\left\{#1\right\}}
\newcommand{\Paren}[1]{\left(#1\right)}
\newcommand{\Bracket}[1]{\left[#1\right]}
\newcommand{\bmn}{\bm{n}}
\newcommand{\ssf}{\textrm{sf}}
\newcommand{\ii}{\mathrm{i}}
\newcommand{\bmw}{\bm{w}}
\newcommand{\bmg}{\bm{g}}
\newcommand{\stress}[1]{\textcolor{red}{#1}}



\DeclareMathOperator{\var}{var}
\DeclareMathOperator{\st}{s. t.}
\DeclareMathOperator*{\argmin}{arg\,min}
\DeclareMathOperator{\Bern}{Bern}
\DeclareMathOperator{\Unif}{Unif}
\DeclareMathOperator{\diag}{diag}
\DeclareMathOperator{\tspan}{span}
\allowdisplaybreaks


\begin{document}
%
% paper title
% Titles are generally capitalized except for words such as a, an, and, as,
% at, but, by, for, in, nor, of, on, or, the, to and up, which are usually
% not capitalized unless they are the first or last word of the title.
% Linebreaks \\ can be used within to get better formatting as desired.
% Do not put math or special symbols in the title.
\title{Systematic Transmission With Fountain Parity Checks for Erasure Channels With Stop Feedback}
%
%
% author names and IEEE memberships
% note positions of commas and nonbreaking spaces ( ~ ) LaTeX will not break
% a structure at a ~ so this keeps an author's name from being broken across
% two lines.
% use \thanks{} to gain access to the first footnote area
% a separate \thanks must be used for each paragraph as LaTeX2e's \thanks
% was not built to handle multiple paragraphs
%

\author{Hengjie~Yang,~\IEEEmembership{Member,~IEEE}
  and~Richard~D.~Wesel,~\IEEEmembership{Fellow,~IEEE}
\thanks{This work was supported by the National Science Foundation (NSF) under Grant CCF-1955660. Any opinions, findings, and conclusions or recommendations expressed in this material are those of the authors and do not necessarily reflect views of NSF.

Hengjie Yang is with Qualcomm Technologies, Inc., San Diego, CA 92121 USA (e-mail: hengjie.yang@ucla.edu).

Richard D. Wesel is with the Department of Electrical and Computer Engineering, University of California at Los Angeles, Los Angeles, CA 90095 USA (e-mail: wesel@ucla.edu).
}}

% note the % following the last \IEEEmembership and also \thanks - 
% these prevent an unwanted space from occurring between the last author name
% and the end of the author line. i.e., if you had this:
% 
% \author{....lastname \thanks{...} \thanks{...} }
%                     ^------------^------------^----Do not want these spaces!
%
% a space would be appended to the last name and could cause every name on that
% line to be shifted left slightly. This is one of those "LaTeX things". For
% instance, "\textbf{A} \textbf{B}" will typeset as "A B" not "AB". To get
% "AB" then you have to do: "\textbf{A}\textbf{B}"
% \thanks is no different in this regard, so shield the last } of each \thanks
% that ends a line with a % and do not let a space in before the next \thanks.
% Spaces after \IEEEmembership other than the last one are OK (and needed) as
% you are supposed to have spaces between the names. For what it is worth,
% this is a minor point as most people would not even notice if the said evil
% space somehow managed to creep in.



% The paper headers
% \markboth{IEEE Transactions on Information Theory}%
% {Yang \MakeLowercase{\textit{et al.}}: A Small-Enough-Difference Encoder for Binary-Input Channels with Feedback}
% The only time the second header will appear is for the odd numbered pages
% after the title page when using the twoside option.
% 
% *** Note that you probably will NOT want to include the author's ***
% *** name in the headers of peer review papers.                   ***
% You can use \ifCLASSOPTIONpeerreview for conditional compilation here if
% you desire.




% If you want to put a publisher's ID mark on the page you can do it like
% this:
%\IEEEpubid{0000--0000/00\$00.00~\copyright~2015 IEEE}
% Remember, if you use this you must call \IEEEpubidadjcol in the second
% column for its text to clear the IEEEpubid mark.



% use for special paper notices
%\IEEEspecialpapernotice{(Invited Paper)}




% make the title area
\maketitle

% As a general rule, do not put math, special symbols or citations
% in the abstract or keywords.
\begin{abstract}
In this paper, we present new achievability bounds on the maximal achievable rate of variable-length stop-feedback (VLSF) codes operating over a binary erasure channel (BEC) at a fixed message size $M = 2^k$. We provide new bounds for VLSF codes with zero error, infinite decoding times and with nonzero error, finite decoding times. Both new achievability bounds are proved by constructing a new VLSF code that employs systematic transmission of the first $k$ bits followed by random linear fountain parity bits decoded with a rank decoder. For VLSF codes with infinite decoding times, our new bound outperforms the state-of-the-art result for BEC by Devassy \emph{et al.} in 2016. We also give a negative answer to the open question Devassy \emph{et al.} put forward on whether the $23.4\%$ backoff to capacity at $k = 3$ is fundamental. For VLSF codes with finite decoding times, numerical evaluations show that the achievable rate for VLSF codes with a moderate number of decoding times closely approaches that for VLSF codes with infinite decoding times.
\end{abstract}

% Note that keywords are not normally used for peerreview papers.
\begin{IEEEkeywords}
Binary erasure channel, random linear fountain coding, systematic transmission.
\end{IEEEkeywords}






% For peer review papers, you can put extra information on the cover
% page as needed:
% \ifCLASSOPTIONpeerreview
% \begin{center} \bfseries EDICS Category: 3-BBND \end{center}
% \fi
%
% For peerreview papers, this IEEEtran command inserts a page break and
% creates the second title. It will be ignored for other modes.
\IEEEpeerreviewmaketitle



\section{Introduction}
% The very first letter is a 2 line initial drop letter followed
% by the rest of the first word in caps.
% 
% form to use if the first word consists of a single letter:
% \IEEEPARstart{A}{demo} file is ....
% 
% form to use if you need the single drop letter followed by
% normal text (unknown if ever used by the IEEE):
% \IEEEPARstart{A}{}demo file is ....
% 
% Some journals put the first two words in caps:
% \IEEEPARstart{T}{his demo} file is ....
% 
% Here we have the typical use of a "T" for an initial drop letter
% and "HIS" in caps to complete the first word.
\IEEEPARstart{I}{n} a point-to-point communication system with stop feedback, the decoder decides on the fly when to stop transmission and sends a $1$-bit acknowledgement (ACK) or negative acknowledgement (NACK) symbol via the noiseless feedback channel informing the transmitter whether to stop or continue transmission. Meanwhile, the transmitter cannot utilize the stop-feedback symbol to design the next code symbol. Polyanskiy \emph{et al.} \cite{Polyanskiy2011} formalized this type of code as the \emph{variable-length stop-feedback (VLSF)} code. The VLSF code is of practical interest since it includes the hybrid automatic repeat request and incremental redundancy. Polyanskiy \emph{et al.} showed that even with such a limited use of feedback, the maximal achievable rate of VLSF code is significantly better than that of the fixed-length code in the nonasymptotic regime. Earlier works have also studied various aspects of VLSF codes, including the performance in error-exponent regime \cite{Forney1968}, performance for random linear codes over the binary erasure channel (BEC) \cite{Heidarzadeh2019}, and performance when noisy stop feedback is present \cite{Ostman2019}.

This paper focuses on the BEC with stop feedback and seeks a nonasymptotic achievability bound for the VLSF setup. To the best of our knowledge, Devassy \emph{et al.} obtained the state-of-the-art achievability \cite[Theorem 9]{Devassy2016} and converse bounds \cite[Corollary 6]{Devassy2016} for VLSF codes operating over a BEC. Note that constructing VLSF codes for the BEC is equivalent to constructing rateless erasure codes. Motivated by this observation, Devassy \emph{et al.} derived the achievability bound by analyzing a family of random linear fountain codes \cite[Chapter 50]{MacKay2005} of message size $2^k$ and by using a rank decoder. The rank decoder keeps track of the rank of the generator matrix associated with unerased received symbols. As soon as the rank equals $k$, the decoder stops transmission by sending an ACK symbol and reproduces the $k$-bit message with zero error using the inverse of the generator matrix. However, their achievability bound implies that the ratio of maximal achievable rate to capacity is only a function of message length $k$ and the ratio attains the maximum of $76.6\%$ at $k = 3$. Hence, they posed the question whether the $23.4\%$ backoff percentage to capacity at $k = 3$ is fundamental.

Unlike Devassy \emph{et al.}'s approach, we adopt a new coding scheme called \emph{systematic transmission followed by random linear fountain coding (ST-RLFC)}. Namely, the transmitter simply transmits the first $k$ message bits in the first $k$ time instants. After that, the transmitter employs a random linear fountain code to generate parity bits. Specifically, starting the $(k+1)$th time instant, both the encoder and decoder select the same nonzero base vector in $\{0, 1\}^k$ according to the common randomness. The encoder produces the code symbol by linearly combining the message bits using the selected base vector. The decoder is still the same rank decoder.

Our contributions in this paper are as follows.
\begin{itemize}
  \item By analyzing the ST-RLFC scheme, we present a new achievability bound for VLSF codes of message size $M = 2^k$ and zero error probability. Our new bound outperforms Devassy \emph{et al.}'s result \cite[Theorem 9]{Devassy2016}. In addition, the new bound implies that the $23.4\%$ backoff percentage to capacity reported by Devassy \emph{et al.} is \emph{not} fundamental. On the contrary, the new bound indicates that the backoff percentage is proportional to the erasure probability at any given message length $k$. If the erasure probability is zero, there is no backoff from capacity.
  \item The new VLSF achievability bound for zero-error VLSF codes implies a new achievability bound for VLSF codes constrained to have finite decoding times and a nonzero error probability. Numerical computations show that when number of decoding times $m$ is small, a slight increase in $m$ can dramatically improve the achievable rate. However, when $m$ is moderately large (for instance, $m = 16$ for erasure probability $0.5$), the achievable rate closely approaches that for $m = \infty$.
\end{itemize}

The remainder of this paper is organized as follows. Section \ref{sec: preliminary} introduces the notation and the VLSF code, and presents previously known bounds for VLSF codes operating over a BEC. Section \ref{sec: main results} provides the new VLSF achievability bound and its implications. Section \ref{sec: proofs} includes proofs of the main results. Section \ref{sec: conclusion} concludes the paper.








\section{Preliminaries}\label{sec: preliminary}

\subsection{Notation}

Let $\N = \{0, 1,\dots\}$, $\N_+ = \N\setminus\{0\}$, $\N_{\infty} = \N\cup\{\infty\}$ be the set of natural numbers, positive integers, and extended natural numbers, respectively. For $i\in\N$, $[i]\triangleq \{1,2,\dots, i\}$. We use $x_i^j$ to denote a sequence $(x_i, x_{i+1}, \dots, x_j)$, $1\le i\le j$. We denote by $\bm{e}_i\in\R^{k\times 1}$ the $k$-dimensional natural base vector with $1$ at index $i$ and $0$ everywhere else, $1\le i\le k$. We denote the distribution of a random variable $X$ by $\PP_X$.



\subsection{VLSF Codes}
We consider a BEC with input alphabet $\X = \{0, 1\}$, output alphabet $\Y = \{0, ?, 1\}$, and erasure probability $p\in[0, 1)$. A VLSF code for BEC with finite decoding times is defined as follows.
\begin{definition}\label{def: VLSF code}
An $(l, n_1^m, M, \epsilon)$ VLSF code, where $l > 0$, $m\in\N_{\infty}$, $n_1^m\in\N^m$ satisfying $n_1 < n_2 < \cdots < n_m$, $M\in\N_+$, and $\epsilon\in(0, 1)$, is defined by:
\begin{itemize}
  \item [1)] A finite alphabet $\U$ and a probability distribution $\PP_U$ on $\U$ defining the common randomness random variable $U$ that is revealed to both the transmitter and the receiver before the start of the transmission.
  \item [2)] A sequence of encoders $f_n: \U\times [M] \to \X$, $n = 1,2,\dots, n_m$, defining the channel inputs
    \begin{align}
      X_n = f_n(U, W),
    \end{align}
    where $W\in[M]$ is the equiprobable message.
  \item [3)] A non-negative integer-valued random stopping time $\tau\in\{n_1, n_2, \dots, n_m\}$ of the filtration generated by $\{U, Y^{n_i}\}_{i=1}^m$ that satisfies the average decoding time constraint
    \begin{align}
      \E[\tau] \le l.
    \end{align}
  \item [4)] $m$ decoding functions $g_{n_i}: \U\times \Y^{n_i}\to [M]$, providing the best estimate of $W$ at time $n_i$, $i\in[m]$. The final decision $\hat{W}$ is computed at time instant $\tau$, i.e., $\hat{W} = g_{\tau}(U, Y^\tau)$ and must satisfy the average error probability constraint
    \begin{align}
      P_e \triangleq \Prob[\hat{W}\ne W] \le \epsilon.
    \end{align}
\end{itemize}
\end{definition}
Comparing to Polyanskiy \emph{et al.}'s VLSF code definition \cite{Polyanskiy2011}, the primary distinctions are two-fold. First, the VLSF code is allowed to have finite decoding times rather than infinite decoding times. As a result, the stopping time is constrained within these decoding times. Second, both the expected blocklength and error probability constraints correspond to the given sequence of decoding times rather than $\N$.

In this paper, we focus on upper bounding the average blocklength $\E[\tau]$ of $(l, \N, 2^k, 0)$ VLSF code with $m = \infty$ and $(l, n_1^m, 2^k, \epsilon)$ VLSF code with $m < \infty$. The rate of an $(l, n_1^m, M, \epsilon)$ VLSF code is defined by 
\begin{align}
    R \triangleq \frac{\log M}{\E[\tau]}.
\end{align}


\subsection{Previous Results for VLSF Codes over BECs}

For the BEC, the decoder has the ability to identify the correct message whenever only a single codeword is compatible with the unerased channel outputs up to that point. By exploiting this fact and utilizing the RLFC, Devassy \emph{et al.} \cite{Devassy2016} obtained state-of-the-art achievability bound for zero-error VLSF codes with message size $M$ that is a power of $2$.
\begin{theorem}[Theorem 9, \cite{Devassy2016}]\label{theorem: BEC achievability}
  For each integer $k\ge 1$, there exists an $(l, \N, 2^k, 0)$ VLSF code for a BEC$(p)$ with
    \begin{align}
        l \le \frac{1}{C}\Paren{k + \sum_{i = 1}^{k-1} \frac{2^i - 1}{2^k - 2^i} }, \label{eq: Devassy bound}
    \end{align}
\end{theorem}

Note that the second term in parentheses of \eqref{eq: Devassy bound} is bounded by the Erd\"os-Borwein constant $c = 1.60669515...$ (OEIS: A065442),
\begin{align}
  \sum_{i = 1}^{k-1} \frac{2^i - 1}{2^k - 2^i} \le \sum_{j = 1}^{k-1} \frac{1}{2^{j} - 1}\le \sum_{j = 1}^{\infty} \frac{1}{2^{j} - 1} = c.
\end{align}
In \cite[Theorem 2]{Heidarzadeh2019}, Heidarzadeh \emph{et al.} showed that by constructing random linear codes for which column vectors of the parity check matrix for erased symbols are linearly independent, the average blocklength of the corresponding $(l, \N, 2^k, 0)$ VLSF code is given by $\frac{k+c}{C}$. This indicates that Heidarzadeh's random linear coding scheme performs as good as Devassy's RLFC scheme for sufficiently large message length $k$.



The state-of-the-art converse bound for $(l, \N, 2^k, 0)$ VLSF codes over a BEC is also obtained by Devassy \emph{et al.} using the method of binary sequential hypothesis testing.
\begin{theorem}[Corollary 6, \cite{Devassy2016}]\label{theorem: BEC converse}
  The minimum average blocklength $l_f^*(M, 0)$ of an $(l, \N, M, 0)$ VLSF code over a BEC$(p)$ is given by
  \begin{align}
    l^*_f(M, 0) = \frac{\lfloor\log_2M\rfloor + 2\left(1 - 2^{\lfloor \log_2 M\rfloor - \log_2M}\right)}{C}.
  \end{align}
\end{theorem}
Note that when $M$ is a power of $2$, Theorem \ref{theorem: BEC converse} implies that the converse bound on maximal achievable rate is simply the capacity of the BEC.






















\section{Achievable Rates of VLSF codes over BECs}\label{sec: main results}

In this section, we present a new coding scheme for a BEC called the ST-RLFC, a new achievability bound for zero-error VLSF codes of infinite decoding times, and the comparison with Devassy \emph{et al.}'s result. Finally, we present a new achievability bound for VLSF code with finite decoding times and a comparison of achievability bounds for various numbers of decoding times.

\subsection{ST-RLFC Scheme}

Consider transmitting a $k$-bit message
\begin{align}
  \bm{b} = (b_1, b_2, \dots, b_k)\in\{0, 1\}^k. \label{eq: k_bit msg}
\end{align}
Let us define the set of nonzero base vectors in $\{0, 1\}^k$ by
\begin{align}
  \G_k \triangleq \{\bm{v}\in\{0, 1\}^k:\bm{v}\ne\bm{0} \}.
\end{align}
Using ST-RLFC scheme, the channel input at time instant $n\in\N_+$ for message $\bm{b}$ is given by
\begin{align}
  X_n = \begin{cases}
    b_n, & \text{if } 1\le n\le k\\
    \bigoplus_{i=1}^k g_{n,i}b_i & \text{if } n > k,
  \end{cases} \label{eq: ST-RLFC encoder}
\end{align}
where $\oplus$ denotes bit-wise exclusive-or (XOR) operator, and $\bmg_n = (g_{n,1}, g_{n,2}, \dots, g_{n,k})^\top\in\G_k$ is generated at time instant $n$ according to a uniformly distributed random variable $\tilde{U}\in \G_k$. Note that the encoder and decoder share the same common random variable $\tilde{U}$ at time instant $n > k$ so that the decoder can produce the same $\bmg_n$ at time $n$. For $1\le n \le k$, both the encoder and decoder simply use the natural base vector $\bm{e}_n\in\R^{k\times 1}$. For all $\bm{b}\in\{0,1\}^k$, the procedure \eqref{eq: ST-RLFC encoder} specifies the common codebook before the start of transmission, i.e., the common randomness random variable $U$ in Definition \ref{def: VLSF code}.

Let $Y_n$ be the received symbol after transmitting $X_n$ over a BEC$(p)$, $p\in[0, 1)$. We consider the \emph{rank decoder} \cite{Devassy2016} that keeps track of the rank of generator matrix $G$ associated with received symbols $Y^n = (Y_1, Y_2, \dots, Y_n)$. Let $G(n)$ denote the $n$th column of $G$. If $Y_n = ?$, $G(n) = \bm{0}$; otherwise, $G(n) = \bmg_n$. Define the stopping time
\begin{align}
  \tau \triangleq \inf\{n\in\N : \text{$G(1:n)$ has rank $k$}\}, \label{eq: ST-RLFC stopping time}
\end{align}
where $G(i:j)$ denotes the matrix formed by column vectors from time instants $i$ to $j$, $1\le i\le j$. Thus, the rank decoder stops transmission at time instant $\tau$ and reproduces the $k$-bit message $\bm{b}$ using $Y^\tau$ and the inverse of $G(1:\tau)$ (namely, by solving $k$ message bits from $k$ linearly independent equations). Clearly, the error probability associated with the ST-RLFC scheme is zero.


\subsection{Achievability of Zero-Error VLSF Codes}

The ST-RLFC scheme implies the following achievability bound for $(l, \N, 2^k, 0)$ VLSF codes operating over a BEC.
\begin{theorem}\label{theorem: new BEC achiev bound}
  For a given integer $k\ge 1$, there exists an $(l, \N, 2^k, 0)$ VLSF code for BEC$(p)$, $p\in[0, 1)$, with
  \begin{align}
    l \le k + \frac{1}{C}\sum_{i=0}^{k-1}\frac{2^k-1}{2^k-2^i}F(i; k, 1-p). \label{eq: new BEC bound}
  \end{align}
  where $C = 1 - p$ and
  \begin{align}
    F(i; k, 1-p) \triangleq \sum_{j=0}^{i}\binom{k}{j}(1-p)^{j}p^{k-j} \label{eq: th13_CDF}
  \end{align}
  denotes the CDF evaluated at $i$, $0\le i\le k$, of a binomial distribution with $k$ trials and success probability $1-p$.
\end{theorem}

\begin{IEEEproof}
  See Section \ref{subsec: proof of main theorem}.
\end{IEEEproof}


For non-vanishing error probability $\epsilon > 0$, using Polyanskiy's early termination scheme in \cite[Section III-D]{Polyanskiy2011} by stopping the zero-error VLSF code at $\tau = 0$ with probability $\epsilon$, the corresponding achievability bound can be readily obtained by multiplying the right-hand side (RHS) of \eqref{eq: new BEC bound} by a factor $(1 - \epsilon)$.


We remark that the new achievability bound \eqref{eq: new BEC bound} is tighter than Devassy's bound \eqref{eq: Devassy bound} and two bounds are equal if $p = 1$ or $k = 1$. This is stated in the following corollary.
\begin{corollary}\label{corollary: tighter bound}
  For a given $k\in\N_+$ and BEC$(p)$, $p\in[0, 1]$, it holds that
  \begin{align}
    kC + \sum_{i=0}^{k-1}\frac{2^k-1}{2^k-2^i}F(i; k, 1-p) \le k + \sum_{i=1}^{k-1}\frac{2^i-1}{2^k-2^i},
  \end{align}
  where $C = 1 - p$ and $F(i; k, 1-p)$ is given by \eqref{eq: th13_CDF}. Equality holds if $p = 1$ or $k = 1$.
\end{corollary}

\begin{IEEEproof}
  Fix $k\in\N_+$ and $p\in[0, 1]$. First, note that
  \begin{align}
    k + \sum_{i=1}^{k-1}\frac{2^i-1}{2^k-2^i} = \sum_{i=0}^{k-1}\frac{2^k-1}{2^k - 2^i}.
  \end{align}
  Hence,
  \begin{align}
    &\sum_{i=0}^{k-1}\frac{2^k-1}{2^k - 2^i} - \sum_{i=0}^{k-1}\frac{2^k-1}{2^k-2^i}F(i; k, 1-p) - k(1 - p) \notag\\
    &= \sum_{i=0}^{k-1}\frac{2^k-1}{2^k-2^i}F^c(i; k, 1-p) - k(1 - p) \label{eq: cor2_eq0} \\
    &\ge \sum_{i=0}^{k-1}F^c(i; k, 1-p) - k(1 - p) \label{eq: cor2_eq1} \\
    &= 0, \notag
  \end{align}
  where in \eqref{eq: cor2_eq1}, $F^c(\cdot) \triangleq 1 - F(\cdot)$ denotes the tail probability, and the sum of tail probability equals the expectation $k(1-p)$. Note that \eqref{eq: cor2_eq0} equals $0$ if $p = 1$ or $k = 1$. This completes the proof of Corollary \ref{corollary: tighter bound}.
\end{IEEEproof}

% Figure environment removed


% Figure environment removed

For BEC$(0)$ and $k\ge 2$, our new bound \eqref{eq: new BEC bound} reduces to $k$, whereas Devassy \emph{et al}'s bound \eqref{eq: Devassy bound} is strictly larger than $k$ and approaches $k + c$ for sufficiently large $k$, where $c$ denotes the Erd\"os-Borwein constant. Moreover, \eqref{eq: Devassy bound} also implies an upper bound independent of $p$ on the backoff percentage to capacity,
\begin{align}
  1 - \frac{R}{C}\le 1 - \frac{k}{k + \sum_{i=1}^{k-1}\frac{2^i-1}{2^k-2^i} }. \label{eq: old backoff percentage}
\end{align}
Devassy \emph{et al.} \cite{Devassy2016} reported that this upper bound attains its maximum $23.4\%$ at $k = 3$ and that the maximum is independent of erasure probability, thus raising the question whether this backoff percentage is fundamental. In contrast, our result in \eqref{eq: new BEC bound} implies a refined upper bound on backoff percentage that is dependent on $p$,
\begin{align}
  1 - \frac{R}{C} &\le 1 - \frac{k}{\sum_{i=0}^{k-1}\frac{2^k-1}{2^k-2^i}F(i; k, 1-p) + k(1-p)}. \label{eq: new backoff percentage}
\end{align}
Fig. \ref{fig: backoff from capacity} shows the comparison of these two upper bounds at $k  = 3$. We see that for $k = 3$, the upper bound in \eqref{eq: new backoff percentage} is a strictly increasing function of $p$. As $p\to 0$, this upper bound converges to $0$, which closes the backoff from capacity at $k = 3$. As $p \to 1$, the upper bound in \eqref{eq: new backoff percentage} converges to the backoff percentage in \eqref{eq: old backoff percentage}, as shown in Corollary \ref{corollary: tighter bound}.



% Figure environment removed



Fig. \ref{fig: rate_vs_blocklength} shows the comparison between the new achievability bound (Theorem \ref{theorem: new BEC achiev bound}) and Devassy \emph{et al.}'s bounds (Theorems \ref{theorem: BEC achievability} and \ref{theorem: BEC converse}) for BEC$(0.1)$. As can be seen, for a small message length, the new achievability bound is closer to capacity than Devassy \emph{et al.}'s bound. This is because when $k$ is small, systematic transmission of the uncoded message symbol is more likely to increase rank than transmitting a fountain code symbol. However, as $k$ gets larger, the advantage of systematic transmission over RLFC gradually diminishes. To see this more clearly, let random variables $S^{\text{ST}}_k$ and $S^{\text{RLFC}}_k$ denote the rank of generator matrix $G(1:k)$ for ST and RLFC, respectively. We use $\E[S^{\text{ST}}_k] - \E[S^{\text{RLFC}}_k]$ as the metric to measure the difference of rank increase rate over time range $1$ to $k$. Note that $\E[S^{\text{ST}}_k] = k(1-p)$ since the rank distribution at time instant $k$ is binomial. $\E[S^{\text{RLFC}}_k]$ can be numerically computed using the one-step transfer matrix $P$ in \eqref{eq: ap2_one_step_transfer_matrix} and initial distribution $[1, \bm{0}^\top]\in\R^{1\times (k+1)}$. Fig. \ref{fig: gap_expected_rank_vs_k} shows $\E[S^{\text{ST}}_k] - \E[S^{\text{RLFC}}_k]$ as a function of message length $k$ for BEC$(0.1)$. We see that this gap constantly remains nonnegative, implying that the rank increase rate for ST is always faster than that for RLFC. For sufficiently large $k$, this gap becomes small, indicating the diminishing advantage of ST over RLFC.



\subsection{Achievability of VLSF Codes with Finite Decoding Times}


The ST-RLFC scheme also facilitates an $(l, n_1^m, 2^k, \epsilon)$ VLSF code for a BEC. This code is constructed by using the same ST-RLFC scheme in \eqref{eq: ST-RLFC encoder} but a rank decoder that only considers a finite set of decoding times. Specifically, fix $n_1^m\in\N_+$ satisfying $n_1 < n_2 < \cdots < n_m$. For a given $k\in\N_+$ and $\epsilon\in(0, 1)$, the rank decoder still shares the same common randomness with the encoder in selecting the base vector $\bmg_n$, except that it now adopts the following stopping time:
\begin{align}
  \tau^* \triangleq \inf\{n\in\{n_i\}_{i=1}^m: G(1:n) \text{ has rank $k$ or $n = n_m$} \}. \label{eq: new stopping time}
\end{align}
If $\tau \le n_m$ and $G(1:\tau)$ is full rank, the rank decoder reproduces the transmitted message using $Y^\tau$ and the inverse of $G(1:\tau)$. If $\tau = n_m$ and $G(1:n_m)$ is rank deficient, then the rank decoder outputs an arbitrary message.


The ST-RLFC scheme and the modified rank decoder imply the following achievability bound for an $(l, n_1^m, 2^k, \epsilon)$ VLSF code.
\begin{theorem}\label{theorem: new nonasymptotic achiev bound for BEC}
  Fix $n_1^m\in\N_+^m$ satisfying $n_1 < n_2 < \cdots < n_m$. For any positive integer $k\in\N_+$ and $\epsilon\in(0, 1)$, there exists an $(l, n_1^m, 2^k, \epsilon)$ VLSF code for the BEC$(p)$ with
    \begin{align}
      l &\le n_m - \sum_{i=1}^{m-1}(n_{i+1} - n_i)\Prob[S_{n_i} = k], \label{eq: nonasymptotic bound on l} \\
      \epsilon &\le 1 - \Prob[S_{n_m} = k],  \label{eq: nonasymptotic bound on epsilon}
    \end{align}
  where the random variable $S_n$ denote the rank of the generator matrix $G(1:n)$ observed by the rank decoder. Specifically, $\Prob[S_n = k]$ is given by
  \begin{align}
    \Prob[S_n = k] = \begin{cases}
      0, & \text{if } n < k \\
      1 - \bm{\alpha}^\top T^{n-k}\bm{1}, &\text{if } n\ge k,
    \end{cases} \label{eq: expression for S_n}
  \end{align}
where $\bm{\alpha} = [\alpha_1,\alpha_2,\dots, \alpha_k]^\top\in\R^{k\times 1}$ with $\alpha_i = F(i; k, 1-p)$, $0\le i\le k-1$, and $F(i;k, 1-p)$ is given by \eqref{eq: th13_CDF}, $T\in\R^{k\times k}$ with entries given by
  \begin{align}
    T_{i, i} &= p + \frac{(1-p)(2^{i-1} - 1)}{2^k - 1}, \\
  T_{i,i+1} &= \frac{(1-p)(2^k - 2^{i-1})}{2^k - 1}, \\
  T_{i,j} &= 0, \text{ for } j\ne i \text{ and } j\ne i+1.
  \end{align}
\end{theorem}

\begin{IEEEproof}
  See Section \ref{subsec: proof of second result}.
\end{IEEEproof}


Theorem \ref{theorem: new nonasymptotic achiev bound for BEC} facilitates an integer program that can be used to compute the achievability bound on rate for all zero-error VLSF codes of message size $M = 2^k$ and $m$ decoding times. Define
\begin{align}
  N(n_1^m) &\triangleq n_m - \sum_{i=1}^{m-1}(n_{i+1} - n_{i})\Prob[S_{n_i} = k].
\end{align}
For a given number of decoding times $m\in\N_+$, message length $k\in\N_+$, and a target error probability $\delta\in(0,1)$, 
\begin{align}
  \begin{split}
    \min_{n_1^m}&\quad N(n_1^m) \\
    \st&\quad 1 - \Prob[S_{n_m} = k]\le \delta
  \end{split} \label{eq: ST-RLFC integer program}
\end{align}
Assume $N(a_1^m)$ is the minimum value after solving \eqref{eq: ST-RLFC integer program}, where $a_1^m$ is the minimizer. Then the achievability bound on rate for a VLSF code of message size $2^k$, target error probability $\delta$, and $m$ decoding times is given by $\frac{k}{N(a_1^m)}$. For reasonably small values of $m$, one can use brute-force method to obtain $a_1^m$ and $N(a_1^m)$.


For BEC$(0.5)$, Fig. \ref{fig: rate_vs_blocklength_finite_m} shows achievability bounds for $(l, n_1^m, 2^k, \delta)$ VLSF codes, where $m\in\{1,2,4,8, 16\}$ and target error probability $\delta = 10^{-3}$. The adjusted achievability bound boosted by $\frac{1}{1-\delta}$ for $m = \infty$ using Theorem \ref{theorem: new BEC achiev bound} and Polyanskiy's early termination scheme is also shown. We see that when $m$ is small, increasing $m$ can dramatically improve the achievable rate. However, when $m = 16$, the achievable rate closely approaches that for $m = \infty$. We remark that similar effect on achievable rate by the varying number of decoding times has also been observed in several previous works, e.g., \cite{Heidarzadeh2019,Yavas2021,Yang_ISIT2022}.



















\section{Proofs}\label{sec: proofs}

In this section, we prove our main results.

% Figure environment removed

\subsection{Proof of Theorem \ref{theorem: new BEC achiev bound}} \label{subsec: proof of main theorem}

Let random variable $S_n$ denote the rank of generator matrix $G(1:n)$. According to the ST-RLFC scheme, the probability mass function (PMF) of $S_k$ at time $k$ is given by
\begin{align}
  \Prob[S_k = r] = \binom{k}{r}(1-p)^rp^{k-r},\quad 0\le r\le k. \label{eq: ap2_eq1}
\end{align}
For $n\ge k$, due to the BEC$(p)$ and our RLFC scheme, $S_{n+1} = S_n = r$ occurs if $Y_{n+1} = ?$ or if $Y_{n+1}\ne ?$ and $\bmg_{n+1}$ is a linear combination of previous $r$ independent base vectors. Otherwise, $S_{n+1} = r + 1$. Hence, the behavior of $S_n$, $n\ge k$, is characterized by the following discrete-time homogeneous Markov chain with $k+1$ states.
\begin{align}
  &\Prob[S_{n+1} = r|S_n = r] = p + \frac{(1-p)(2^r - 1)}{2^k - 1}, \\
  & \Prob[S_{n+1} = r+1|S_n = r] = \frac{(1-p)(2^k - 2^r)}{2^k - 1}, \label{eq: rank increase prob}
\end{align}
where $0\le r\le k-1$, and $\Prob[S_{n+1} = k|S_n = k] = 1$. Note that this Markov chain has a single absorbing state $S_n = k$. The time to absorption for this Markov chain follows a discrete phase-type distribution \cite[Chapter 2]{Neuts_book}. More specifically, the one-step transfer matrix $P\in\R^{(k+1)\times (k+1)}$ of this Markov chain can be written as
\begin{align}
  P = \begin{bmatrix}
    T & \bm{t} \\
    \bm{0}^\top & 1
  \end{bmatrix}, \label{eq: ap2_one_step_transfer_matrix}
\end{align}
where the entries of $T\in\R^{k\times k}$ are given by
\begin{align}
  T_{i, i} &= p + \frac{(1-p)(2^{i-1} - 1)}{2^k - 1}, \\
  T_{i,i+1} &= \frac{(1-p)(2^k - 2^{i-1})}{2^k - 1},
\end{align}
and $T_{i,j} = 0$ for any other pair $(i, j)$, $1\le i,j\le k$. Since $P$ is a stochastic matrix, it follows that
\begin{align}
  \bm{t} = (I -T)\bm{1}.
\end{align}
The initial probability distribution is given by $[\bm{\alpha}^\top, \alpha_{k}]$, where
\begin{align}
  \bm{\alpha}^\top \triangleq \begin{bmatrix}
    \Prob[S_k=0] & \Prob[S_k=1] & \cdots & \Prob[S_k=k-1]
  \end{bmatrix}, \label{eq: initial distribution}
\end{align}
with $\Prob[S_k = r]$ given by \eqref{eq: ap2_eq1}, and $\alpha_k = 1 - \bm{\alpha}^\top\bm{1}$. Let random variable $X\in\N$ denote the time to absorbing state $k$ with initial distribution $[\bm{\alpha}^\top, \alpha_{k}]$. Hence, it follows that $X$ has PMF
\begin{align}
  \Prob[X = n] = \bm{\alpha}^\top T^{n-1}\bm{t},\quad n\in\N_+,
\end{align}
and $\Prob[X = 0] = \alpha_{k}$. Define the generating function of $X$ by
\begin{align}
  H_X(z)&\triangleq \E[z^X] = \sum_{n=0}^\infty z^n\Prob[X = n] \notag\\
    &= \alpha_k + \sum_{n=1}^{\infty}z^n\bm{\alpha}^\top T^{n-1}\bm{t} \notag\\
    &= \alpha_k + z\bm{\alpha}^\top\Paren{\sum_{n=0}^\infty (zT)^n }\bm{t} \label{eq: ap2_eq2} \\
    &= \alpha_k + z\bm{\alpha}^\top(I - zT)^{-1}(I - T)\bm{1},
\end{align}
where in \eqref{eq: ap2_eq2}, we have used $\sum_{n=0}^\infty A^n = (I - A)^{-1}$ whenever $|\lambda_i| < 1$ for all $i\in [k]$, where $\{\lambda_i\}_{i=1}^k$ denotes the eigenvalues of a square matrix $A\in\R^{k\times k}$. Hence, the expected time to absorbing state $k$ is given by
\begin{align}
  \E[X] &= \frac{\diff H_X(z)}{\diff z}\Big|_{z = 1} = \bm{\alpha}^\top(I - T)^{-1}\bm{1}.
\end{align}
Therefore, the expected stopping time $\E[\tau]$, with $\tau$ defined in \eqref{eq: ST-RLFC stopping time}, is given by
\begin{align}
  \E[\tau] &= k + \E[X] \notag\\
    &= k + \bm{\alpha}^\top(I - T)^{-1}\bm{1}  \label{eq: ap2_eq3}
\end{align}
Note that 
\begin{align}
  I - T &= (1 - p)\diag\Paren{1, \frac{2^k-2^1}{2^k-1},\frac{2^k-2^2}{2^k-1},\cdots, \frac{2^k-2^{k-1}}{2^k-1} } \notag\\
    &\phantom{==}\cdot\begin{bmatrix}
      1 & -1 & 0 & \cdots & 0 \\
      0 & 1 & -1 & \cdots & 0 \\
      0 & 0 & 1 & \cdots & 0 \\
      \vdots & \vdots & \vdots & \ddots & \vdots \\
      0 & 0 & 0 & \cdots & 1
    \end{bmatrix}
\end{align}
Hence, 
\begin{align}
  &(I - T)^{-1} = (1-p)^{-1}\begin{bmatrix}
      1 & 1 & 1 & \cdots & 1 \\
      0 & 1 & 1 & \cdots & 1 \\
      0 & 0 & 1 & \cdots & 1 \\
      \vdots & \vdots & \vdots & \ddots & \vdots \\
      0 & 0 & 0 & \cdots & 1
    \end{bmatrix}\notag\\
    &\phantom{==}\cdot\diag\Paren{1, \frac{2^k-1}{2^k-2^1}, \frac{2^k-1}{2^k-2^2},\cdots \frac{2^k-1}{2^k - 2^{k-1}} } \notag\\
  &= (1-p)^{-1}\begin{bmatrix}
      1 & \frac{2^k-1}{2^k-2^1} & \frac{2^k-1}{2^k-2^2} & \cdots & \frac{2^k-1}{2^k - 2^{k-1}} \\
      0 & \frac{2^k-1}{2^k-2^1} & \frac{2^k-1}{2^k-2^2} & \cdots & \frac{2^k-1}{2^k - 2^{k-1}} \\
      0 & 0 & \frac{2^k-1}{2^k-2^2}  & \cdots & \frac{2^k-1}{2^k - 2^{k-1}} \\
      \vdots & \vdots & \vdots & \ddots & \vdots \\
      0 & 0 & 0 & \cdots & \frac{2^k-1}{2^k - 2^{k-1}}
    \end{bmatrix}. \label{eq: inverse matrix}
\end{align}
Substituting \eqref{eq: initial distribution} and \eqref{eq: inverse matrix} into \eqref{eq: ap2_eq3}, we finally obtain
\begin{align}
  \E[\tau] &= k + (1-p)^{-1}\sum_{i=0}^{k-1}\frac{2^k-1}{2^k - 2^i}\sum_{j=0}^i\Prob[S_k = j] \\
    &= k + \frac{1}{C}\sum_{i=0}^{k-1}\frac{2^k-1}{2^k - 2^i}F(i; k, 1-p), \label{eq: ap2_eq4}
\end{align}
where $C = 1 - p$ and $F(i;k,1-p) \triangleq \sum_{j=0}^i\Prob[S_k = j]$ denotes the CDF evaluated at $i$ of a binomial distribution with $k$ trials and success probability $1 - p$. Since \eqref{eq: ap2_eq4} is the expected stopping time for an ensemble of zero-error VLSF codes, there exists an $(l, \N, 2^k, 0)$ VLSF code with
\begin{align}
  l \le k + \frac{1}{C}\sum_{i=0}^{k-1}\frac{2^k-1}{2^k - 2^i}F(i; k, 1-p).
\end{align}
This concludes the proof of Theorem \ref{theorem: new BEC achiev bound}.



\subsection{Proof of Theorem \ref{theorem: new nonasymptotic achiev bound for BEC}} \label{subsec: proof of second result}

The proof essentially builds upon that of Theorem \ref{theorem: new BEC achiev bound} with the distinction that the rank decoder adopts a new stopping time given by \eqref{eq: new stopping time}.


Let $S_n$ denote the rank of the generator matrix $G(1:n)$ observed at the rank decoder. The expected stopping time $\E[\tau^*]$ is written as
\begin{align}
  \E[\tau^*] &= \sum_{n=0}^\infty\Prob[\tau^* > n] \notag\\
    &= n_1 + \sum_{i=1}^{m-1}(n_{i+1} - n_i)\Prob[\tau^* > n_i] \\
    &= n_1 + \sum_{i=1}^{m-1}(n_{i+1} - n_i)\Prob[S_{n_i} < k] \\
    &= n_m - \sum_{i=1}^{m-1}(n_{i+1} - n_{i})\Prob[S_{n_i} = k],
\end{align}
thus proving the upper bound in \eqref{eq: nonasymptotic bound on l}.

Note that at finite blocklength, the error only occurs when the rank of generator matrix $G(1:n_m)$ is still less than $k$. Hence,
\begin{align}
  \epsilon &\le \Prob[S_{n_m} < k] \\
    &= 1 - \Prob[S_{n_m} = k],
\end{align}
which is equal to the upper bound in \eqref{eq: nonasymptotic bound on epsilon}. 

At time $n < k$, due to the systematic transmission, $\Prob[S_n = k] = 0$. At time $n\ge k$, as discussed in Section \ref{subsec: proof of main theorem}, the behavior of $S_n$ is characterized by a discrete-time homogeneous Markov chain with $k+1$ states whose one-step transfer matrix is given by \eqref{eq: ap2_one_step_transfer_matrix}, and whose initial probability distribution is $[\bm{\alpha}^\top, \alpha_k]$, where $\bm{\alpha}^\top$ is given by \eqref{eq: initial distribution}. Hence, for $n \ge k$,
\begin{align}
  \Prob[S_n = k] &= 1 - \Prob[S_n < k] \\
    &= 1 - \bm{\alpha}^\top T^{n-k}\bm{1}.
\end{align}
This completes the proof of Theorem \ref{theorem: new nonasymptotic achiev bound for BEC}.



















\section{Conclusion}\label{sec: conclusion}

Using the ST-RLFC scheme and the rank decoder, we have shown an improved achievability bound for zero-error VLSF codes of message size $M = 2^k$. The improvement leverages the fact that when $k$ is small, initially transmitting systematic message symbols is more likely to increase the rank of the generator matrix than transmitting fountain code symbols. However, as demonstrated in Fig. \ref{fig: rate_vs_blocklength}, there is still a significant gap between the converse and the achievability bounds. It remains to be seen how to close this gap. In addition, the extension of Theorem \ref{theorem: new BEC achiev bound} to arbitrary message size $M$ still remains open.

The ST-RLFC scheme combined with a modified rank decoder facilitates a VLSF code of finite decoding times and bounded error probability. Fig. \ref{fig: rate_vs_blocklength_finite_m} shows that when $m$ is small, a slight increase in $m$ can dramatically improve the achievable rate. On the other hand, when $m$ is moderately large (for instance, $m = 16$ for BEC$(0.5)$ shown in Fig. \ref{fig: rate_vs_blocklength_finite_m}), the achievable rate closely approaches that for $m = \infty$. However, a proof that shows this trend still remains elusive.


% Appendix one text goes here.

% you can choose not to have a title for an appendix
% if you want by leaving the argument blank
% \section{}
% Appendix two text goes here.


% use section* for acknowledgment

% \section*{Acknowledgment}


% The authors would like to thank Yury Polyanskiy for providing numerical computations of the VLF achievability and converse bounds for the BSC, and two anonymous reviewers for their constructive comments that improved this paper. 


% Can use something like this to put references on a page
% by themselves when using endfloat and the captionsoff option.
\ifCLASSOPTIONcaptionsoff
  \newpage
\fi



% trigger a \newpage just before the given reference
% number - used to balance the columns on the last page
% adjust value as needed - may need to be readjusted if
% the document is modified later
%\IEEEtriggeratref{8}
% The "triggered" command can be changed if desired:
%\IEEEtriggercmd{\enlargethispage{-5in}}

% references section

% can use a bibliography generated by BibTeX as a .bbl file
% BibTeX documentation can be easily obtained at:
% http://mirror.ctan.org/biblio/bibtex/contrib/doc/
% The IEEEtran BibTeX style support page is at:
% http://www.michaelshell.org/tex/ieeetran/bibtex/
%\bibliographystyle{IEEEtran}
% argument is your BibTeX string definitions and bibliography database(s)
%\bibliography{IEEEabrv,../bib/paper}
%
% <OR> manually copy in the resultant .bbl file
% set second argument of \begin to the number of references
% (used to reserve space for the reference number labels box)
% \begin{thebibliography}{1}

% \bibitem{IEEEhowto:kopka}
% H.~Kopka and P.~W. Daly, \emph{A Guide to \LaTeX}, 3rd~ed.\hskip 1em plus
%   0.5em minus 0.4em\relax Harlow, England: Addison-Wesley, 1999.

% \end{thebibliography}

% biography section
% 
% If you have an EPS/PDF photo (graphicx package needed) extra braces are
% needed around the contents of the optional argument to biography to prevent
% the LaTeX parser from getting confused when it sees the complicated
% \includegraphics command within an optional argument. (You could create
% your own custom macro containing the \includegraphics command to make things
% simpler here.)
%\begin{IEEEbiography}[{% Figure removed}]{Michael Shell}
% or if you just want to reserve a space for a photo:

% \begin{IEEEbiography}{Michael Shell}
% Biography text here.
% \end{IEEEbiography}

% if you will not have a photo at all:
% \begin{IEEEbiographynophoto}{John Doe}
% Biography text here.
% \end{IEEEbiographynophoto}

% insert where needed to balance the two columns on the last page with
% biographies
%\newpage

% \begin{IEEEbiographynophoto}{Jane Doe}
% Biography text here.
% \end{IEEEbiographynophoto}

% You can push biographies down or up by placing
% a \vfill before or after them. The appropriate
% use of \vfill depends on what kind of text is
% on the last page and whether or not the columns
% are being equalized.

%\vfill

% Can be used to pull up biographies so that the bottom of the last one
% is flush with the other column.
%\enlargethispage{-5in}
% \balance
\bibliographystyle{IEEEtran}
\bibliography{IEEEabrv,reference}


% \begin{IEEEbiographynophoto}{Hengjie Yang}
% (Graduate Student Member, IEEE) received the B.Eng. degree in telecommunications engineering from Xidian University, Xi’an, China, in 2017, and the M.S. degree in electrical and computer engineering from the University of California at Los Angeles (UCLA) in 2018, where he is currently pursuing the Ph.D. degree with the Department of Electrical and Computer Engineering. His research interests include channel coding theory and information theory. He was a recipient of the 2021 UCLA Dissertation Year Fellowship and a finalist of the 2022 Distinguished Ph.D. Dissertation Research Award by the Department of Electrical and Computer Engineering at UCLA.
% \end{IEEEbiographynophoto}











% \begin{IEEEbiographynophoto}{Richard D. Wesel}
% (Fellow, IEEE) received the B.S. and M.S. degrees in electrical engineering from the Massachusetts Institute of Technology in 1989 and 1989, respectively, and the Ph.D. degree in electrical engineering from Stanford University in 1996. He is currently a Professor with the Electrical and Computer Engineering Department, UCLA, where he is also an Associate Dean of academic and student affairs with the Henry Samueli School of Engineering and Applied Science. His research interests include communication theory with particular interest in list decoding, short-blocklength communication with and without feedback, low-density parity-check codes, and optimal modulation design. He has received the National Science Foundation CAREER Award, the Okawa Foundation Award for research in information theory and telecommunications, and the Excellence in Teaching Award from the Samueli School of Engineering. He has served as an Associate Editor of Coding and Coded Modulation for the IEEE TRANSACTIONS ON COMMUNICATIONS and currently serves as an Associate Editor of Coding and Decoding for the IEEE TRANSACTIONS ON INFORMATION THEORY.
% \end{IEEEbiographynophoto}











% that's all folks
\end{document}


