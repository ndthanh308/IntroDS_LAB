\section{Problem Formulation}
\label{sec:formulation}
\begin{definition}
Let $f: \cR^d \rightarrow \cR$. If $f$ satisfies:
\[
\abs{f(x) - f(y)} \le L\norm{x - y}
\]
for any $x, y$, then $f$ is $L$-Lipschitz with respect to the norm $\norm{\cdot}$. If the norm is the $\norm{\cdot}_\infty$ norm, then $f$ is simply $L$-Lipschitz, or that $f$ has Lipschitz constant $L$.  
\end{definition}



An adversary selects an $L$-Lipschitz function $f:[0,1]^d \rightarrow [0,L]$. The function $f$ is initially unknown to the learner. At each round $t$, the interaction protocol proceeds as follows:
\begin{enumerate}
    \item The adversary selects a \emph{context} $x_t$ in the input space $\cX = [0,1]^d$.
    \item The learner observes $x_t$ and submits a guess $q_t$ to the value $f(x_t)$.
    \item The adversary observes $q_t$ and computes $\sigma(q_t - f(x_t))$. Here $\sigma(x)$ is the step function that takes value 1 if $x > 0$ and takes value 0 if $x \le 0$. 
    \item The adversary decides whether to corrupt the signal and sends the (possibly corrupted) signal $\sigma_t$ to the learner. 
    \item The learner observes the signal $\sigma_t$, and suffers an unobservable loss $\ell_t := \ell(q_t, f(x_t))$. 
\end{enumerate}


\szdelete{ the adversary selects a \textit{context} $x_t$ in the input space $\cX$, and the learner submits a guess $q_t$ to the value $f(x_t)$. }

\szdelete{After the learner submits a guess, she observes a binary signal $\sigma_t \in \set{0,1}$. The binary signal $\sigma_t$ at round $t$ may be uncorrupted or corrupted. If the signal is uncorrupted, then
\[
\sigma_t := \sigma(q_t - f(x_t)). 
\]
If the signal is corrupted, then
\[
\sigma_t := 1 - \sigma(q_t - f(x_t))
\]
Here, $\sigma(u)$ is the step function that takes the value 1 if $u > 0$ and 0 if $u \le 0$. Hence an uncorrupted signal of $\sigma_t = 1$ indicates a guess too high, and a signal of $\sigma_t = 0$ indicates a guess too low. 
}
%If $f(x_t) = q_t$, then the feedback can be arbitrary in $\set{0,1}$. However, the learner can add infinitesimal perturbation to $q_t$, so that $f(x_t) = q_t$ happens with 0 probability, thus subsequent analysis will not consider the case $f(x_t) = q_t$. 

In step 4, it is assumed that at most $C$ rounds are corrupted. In uncorrupted rounds, the signal $\sigma_t = \sigma(q_t - f(x_t))$. In corrupted rounds, the signal $\sigma_t = 1 - \sigma(q_t - f(x_t))$. In other words, for uncorrupted rounds, a  signal of $\sigma_t = 0$ indicates a guess too low and a signal of $\sigma_t = 1$ indicates a guess too high, and the adversary flips this signal in corrupted rounds. Critically, the value $C$ is \emph{unknown} to the learner. The algorithms presented in this work are agnostic to the quantity $C$, and the performance degrades gracefully as $C$ increases. \szcomment{TODO. add more detail. Done. }

In step 5, the learner suffers a loss $\ell_t = \ell(q_t, f(x_t))$. Note that the functional form of the loss function is known to the learner but the loss value is never revealed to the learner. The goal of the learner will be to incur as little cumulative loss as possible, where the cumulative loss is defined as
\[ 
\sum_{t=1}^T \ell_t = \sum_{t=1}^T \ell(q_t, f(x_t)). 
\]

Two loss functions are considered in this work, the symmetric loss
\[
\ell(q, f(x)) = \abs{f(x) - q}
\]
and the pricing loss
\[
\ell(q, f(x)) = f(x) - q\cdot \ind{q \le f(x)}. 
\]
Separate algorithms are presented for each loss function in the following sections. 

\begin{remark}
    The adversary model in this work is much stronger compared with previous work on corrupion-robust linear contextual search~\cite{krishnamurthy2022contextual, leme2022contextual}. In previous works, the adversary has to commit to a `corruption level' $z_t$ \emph{before} seeing the guess $q_t$ submitted by the learner. In uncorrupted rounds $z_t = 0$, and in corrupted rounds $z_t$ will be a bounded quantity (e.g. $z_t\in [-1,1]$). Then the learner observes the signal $\sigma_t = \sigma(q_t - (f(x_t) + z_t))$. In other words, the binary signal is generated according to the corrupted function value $f(x_t) + z_t$ instead of the true function value $f(x_t)$. 

    In this work, the adversary's power is significantly increased. The adversary has the power to directly corrupt the binary signal and does so \emph{after} the learner has submitted her guess. 
    
    \szcomment{TODO. DONE}
\end{remark}

\begin{remark}
It should be noted there is a slight difference in the range of $f$ compared with the formulation in~\cite{mao2018contextual}. In~\cite{mao2018contextual}, the range of $f$ is assumed to be $[0,1]$ regardless the value of $L$, where this work assume the range of $f$ to be $[0,L]$. The author believes the assumption in this work is the more natural one, as the range of an $L$-Lipschitz function on $[0,1]^d$ is $L$. Assuming the range to be $[0,L]$ also avoids some of the unnecessary complications (in the author's opinion) that arise from the scaling of $L$ (see~\cite{mao2018contextual} appendix A). If one replaces the range $[0,L]$ to $[0,1]$ for $L\ge 1$, the proposed algorithms in this work lead to sharper regret bounds (see details in following sections). 
\end{remark}

\lbcomment{I like these remarks as they compare and contrast with previous works to illustrate your novelty again and allow the readers easily follow}
