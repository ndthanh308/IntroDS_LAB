\documentclass{article}


\usepackage{PRIMEarxiv}

\usepackage[utf8]{inputenc} % allow utf-8 input
\usepackage[T1]{fontenc}    % use 8-bit T1 fonts
\usepackage{hyperref}       % hyperlinks
\usepackage{url}            % simple URL typesetting
\usepackage{booktabs}       % professional-quality tables
\usepackage{amsfonts}       % blackboard math symbols
\usepackage{nicefrac}       % compact symbols for 1/2, etc.
\usepackage{microtype}      % microtypography
\usepackage{lipsum}
\usepackage{fancyhdr}       % header
\usepackage{graphicx}       % graphics
\graphicspath{{media/}}     % organize your images and other figures under media/ folder

\show\abs

\usepackage{fullpage,amssymb,graphicx}
\usepackage{amsmath}
\usepackage{amsthm}
\usepackage{bm}
\usepackage{bbm}
%\usepackage{commath}
\usepackage{physics}
\usepackage{hyperref}
\usepackage{cleveref}
\usepackage{nicefrac}
\usepackage{mathtools}

\usepackage{algorithm}
\usepackage{algpseudocode}
\usepackage{appendix}

\usepackage{thmtools,thm-restate}


\usepackage[suppress]{color-edits}

\addauthor[sz]{sz}{red}
\addauthor[sz]{zz}{red}

\addauthor[draft]{draft}{blue}


\usepackage{amsmath}
\usepackage{amssymb}
\usepackage{amsthm}
\usepackage{mathtools}
\usepackage{comment}
\usepackage{todonotes}
\usepackage{float}
\usepackage[algo2e,ruled,linesnumbered]{algorithm2e}

% for restatable
\usepackage{thmtools,thm-restate}
\usepackage{physics}
% %% >> for restatable links
% \usepackage{xpatch}
% \usepackage{xcolor}
% \usepackage{scalerel}

% % a flag to turn on and off
% \newif\ifmarginprooflinks
%     \marginprooflinkstrue
%     % \marginprooflinksfalse


% %% STEP 1: patch restatable so there are backward links on recall
% \makeatletter
% \xpatchcmd{\thmt@restatable}% Edit \thmt@restatable
%    {\csname #2\@xa\endcsname\ifx\@nx#1\@nx\else[{#1}]\fi}% Replace this code
%    {\ifthmt@thisistheone%
%     \csname #2\@xa\endcsname\ifx\@nx#1\@nx\else[{#1}]\fi% same as before
%     %except with also marginparbox
%    \else\fi} {}{\typeout{FIRST PATCH TO THM RESTATE FAILED}}
% \xpatchcmd{\thmt@restatable}% A second edit to \thmt@restatable
%    {\csname end#2\endcsname}
%    {\ifthmt@thisistheone\csname end#2\endcsname\else\fi}
%    {}{\typeout{FAILED SECOND THMT RESTATE PATCH}}


% \newcommand{\recall}[1]{\medskip\par\noindent{\bf \Cref{thmt@@#1}.} \begingroup\em \noindent
%    \expandafter\csname#1\endcsname* \endgroup\par\smallskip}

% %% STEP 2: make forward links to restatable.
% \setlength\marginparwidth{1.55cm}
% \let\oldmarginpar\marginpar
% \renewcommand{\marginpar}[1]{%
%     \leavevmode%
%     \oldmarginpar{#1}%
%     \ignorespacesafterend\ignorespaces}
% \newsavebox\marginprooflinkbox
% \newenvironment{linked}[3][]{%
%     \def\linkedproof{#3}%
%     \def\linkedtype{#2}%
%     \ifmarginprooflinks%
%     \sbox\marginprooflinkbox{%
%         \centering%
%         \hyperref[proof:\linkedproof]{%
%         \color{blue!30!white}%
%         \scaleleftright{$\Big[$}{\,\mbox{\footnotesize\centering\tt\begin{tabular}{@{}c@{}}
%             link to\\[-0.15em]
%             proof
%         \end{tabular}}\,}{$\Big]$}}~}
%     \fi
%         \restatable[#1]{#2}{#2:#3}\label{#2:#3}%
%     \reversemarginpar	\ifmarginprooflinks\marginpar{\vspace{-1ex}\usebox\marginprooflinkbox}\fi
%     }%
%     {\sbox\marginprooflinkbox{}\endrestatable}
% \newcounter{proofcntr}
% \newenvironment{lproof}{\begin{proof}\refstepcounter{proofcntr}}{\end{proof}}

\newcommand{\vect}[1]{\ensuremath{\mathbf{#1}}}

%% Useful
\newcommand{\p}[1]{\left( #1 \right)}
\newcommand{\br}[1]{\left[ #1 \right]}


%\newcommand{\ev}[1]{\mathbb{E}\left[{#1}\right]}
\newcommand{\evd}[2]{\mathbb{E}_{#1}\left[{#2}\right]}

%% Algortihm notations
\newcommand{\bigO}[1]{O \left( #1 \right )}

%% Calibration
\newcommand{\I}[1]{\mathbb{I}\left[#1\right]}       % Indicator
\newcommand{\calerr}{\mathrm{calerr}}   % Calibration error

\newcommand{\A}{\mathcal{A}}    % Algorithm
\newcommand{\Ber}{\mathrm{Ber}}
\newcommand{\Ecover}{\event^{\textrm{cover}}}   % Event that covered epochs exist
\newcommand{\Enegl}{\event^{\textrm{negl}}}     % Event that negligible epochs exist
\newcommand{\Epoch}{\mathsf{Epoch}}
\newcommand{\eps}{\epsilon}     % epsilon
\newcommand{\Etruth}{\event^{\textrm{truth}}}   % Event that all epochs are truthful
\newcommand{\event}{\mathcal{E}}    % Events
\newcommand{\Ex}[2]{\operatorname*{\mathbb{E}}_{#1}\left[#2\right]}  % Expectation
\newcommand{\Int}{\mathcal{I}}      % Interval
\newcommand{\poly}{\operatorname*{poly}}    % Polynomial
\newcommand{\pr}[1]{\Pr\left[#1\right]}     % Probability
%\newcommand{\red}[1]{{\color{red} #1}}

\newcommand{\red}[1]{\textcolor{red}{#1}}
\newcommand{\blue}[1]{\textcolor{blue}{#1}}

\newcommand{\SPinner}{\mathsf{SP}^{\textrm{inner}}}
\newcommand{\SPouter}{\mathsf{SP}^{\textrm{outer}}}
\newcommand{\Tact}{T^{\mathrm{actual}}}     % Actual stopping time
\newcommand{\prodspace}{\mathcal{X}\times A \times \mathcal{Y}}


%% Fair ERM Notation
\newcommand{\error}[1]{ \left| \mathbb{E}_{(x,y) \sim \mathcal{D}} \ [\one (#1(x) \neq y)] - \ \mathbb{E}_{(x,y) \sim \mathcal{D}} \ [\one (h^*(x) \neq y)] \right|}
\newcommand{\htilde}{\tilde{h}}
\newcommand{\hhat}{\hat{h}}
\newcommand{\hstar}{h^*}
\newcommand{\hclass}{\mathcal{H}}
\newcommand{\posrate}[1]{ P_{(x,y) \sim \DA} [#1 (x)=1]}

\newcommand{\DAC}{\widetilde{\mathcal{D}}_A}
\newcommand{\DBC}{\widetilde{\mathcal{D}}_B}
\newcommand{\RA}{P_{(x,y) \sim \dist } [x \in A]}
\newcommand{\RB}{P_{(x,y) \sim \dist} [x \in B]}
\newcommand{\normalF}{F}
\newcommand{\corruptF}{\widetilde{F}}


%Header
\pagestyle{fancy}
\thispagestyle{empty}
\rhead{ \textit{ }} 

% Update your Headers here
\fancyhead[LO]{Corruption-robust Lipschitz contextual pricing}
% \fancyhead[RE]{Firstauthor and Secondauthor} % Firstauthor et al. if more than 2 - must use \documentclass[twoside]{article}



  
%% Title
\title{Corruption-Robust Lipschitz Contextual Pricing
%%%% Cite as
%%%% Update your official citation here when published 
%\thanks{\textit{\underline{Citation}}: 
%\textbf{Authors. Title. Pages.... DOI:000000/11111.}} 
}

\author{
  Shiliang Zuo \\
  University of Illinois Urbana-Champaign \\
  \texttt{szuo3@illinois.edu} \\
  %% \AND
  %% Coauthor \\
  %% Affiliation \\
  %% Address \\
  %% \texttt{email} \\
  %% \And
  %% Coauthor \\
  %% Affiliation \\
  %% Address \\
  %% \texttt{email} \\
  %% \And
  %% Coauthor \\
  %% Affiliation \\
  %% Address \\
  %% \texttt{email} \\
}


\begin{document}
\maketitle


\begin{abstract}
I consider the problem of learning a Lipschitz function with corrupted binary feedback. An adversary selects a $L$-Lipschitz function $f$ at the beginning of the game. Then each round, the adversary selects a context vector $x_t$ in the input space, then the learner makes a guess to the true function value $f(x_t)$ and receives a binary signal indicating whether the guess was high or low. In a total of $C$ rounds the signal may be corrupted, though the value of $C$ is unknown to the learner. I present a natural yet powerful technique `sanity check', which proves useful in designing corruption-robust algorithms. I design algorithms which: for the symmetric loss, the learner achieves $O(C\log T)$ loss with $d = 1$ and $O_d(C\log T + T^{(d-1)/d})$ loss with $d > 1$; for the pricing loss with a learner's choice of parameter $u \in (0, 1)$, the learner achieves $\Tilde{O}(T^{\frac{(1-u)d + u}{(1-u)d + 1} } + T^u C^{1-u} + C^{1/u} )$ loss. 
\end{abstract}


% keywords can be removed
\keywords{Online Learning \and Dynamic Pricing \and Robust Algorithm}


\section{Introduction}

Consider a seller (she) who at each day attempts to sell a product to a buyer. The seller does not know the maximum price the buyer is willing to pay (i.e. the buyer's private valuation for the item), and must learn this price as she observes the purchasing behavior of the buyer. To this end, the seller sets a price each day, and observes a binary signal from the buyer: the buyer either purchased or did not purchase the item. An unpurchased item indicates an overprice on the seller's side and results in possible loss of revenue; a purchased item indicates an underprice and results in possible loss of surplus. The seller does not directly observe how much revenue she actually lost, but only observes the binary purchasing behavior of the buyer. The seller adjusts the posted price according to the signal, and gradually converges the the optimal price. Such is the dynamic pricing problem in its most basic form.  An early important result appeared in~\cite{kleinberg2003value}, characterizing the optimal regret of the seller. 

The contextual search problem~\cite{liu2021optimal, leme2022contextual, mao2018contextual} comes from the motivation that products may come differentiated, and the buyer's valuation will be a function of the features of the products. Henceforth the features of products will be referred to as context vectors. A common assumption is that the buyer's valuation is a linear function on the context vector. In this setting, Liu et al.~\cite{liu2021optimal} almost fully characterizes the optimal regret. Another assumption is lipschitzness: the valuation function does not assume explicit parametric form, but only requires lipschitzness in the context vector~\cite{mao2018contextual}. 

In practice, it is unreasonable to assume the signals the sellers receives are perfect. For example, agents can act irrational, or malicious entities may interfere with buyer's behaviors, resulting in faulty signals. This motivates the study of contextual pricing with corrupted binary signals. Leme et al. studies the linear contextual search problem with corrupted binary signals~\cite{leme2022corruption}, and proposes a low-regret algorithm based on density updates. 

In this work, I design corruption-robust algorithms for learning Lipschitz functions with binary signals under two loss functions, the symmetric loss and pricing loss. The symmetric loss is defined as
\[
\ell(y, f(x)) = \abs{y - f(x)}. 
\]
A potential application where symmetric loss applies is personalized medicine. Assume the optimal dosage for a patient with context feature $x$ is $f(x)$, and the injected dosage is $y$. The symmetric loss then measures the distance of injected dosage to optimal dosage. The learner (healthcare provider) is not able to directly observe this loss, but can observe a binary signal informing whether she overdosed or underdosed. I give a corruption-robust algorithm for symmetric loss in~\cref{sec:symmetric}, the main result is as follows. 
\begin{theorem*}[\Cref{thm:symm1D} restated]
For $d=1$, there exists an algorithm that achieves cumulative symmetric loss $L\cdot O(C\log T)$. 
\end{theorem*}

\begin{theorem*}[\Cref{thm:symmMD} restated]
For $d>1$, there exists an algorithm that achieves cumulative symmetric loss $L\cdot O_d(C\log T + T^{(d-1)/d})$. 
\end{theorem*}

The pricing loss is defined as
\[
\ell(y, f(x)) = f(x) - y\cdot \ind{y \le f(x)}. 
\]
The natural application is dynamic pricing, as introduced previously. If the seller overprice ($y > f(x)$), she loses the entire revenue $f(x)$; if the seller underprice ($y \le f(x)$), she loses the potential surplus $f(x) - y$. I give corruption-robust algorithms for pricing loss in~\cref{sec:pricing}, a simplified version of the main result is as follows. 
\begin{theorem*}[\Cref{thm:pricing} simplified]
    Fix $u\in(0,1)$. There exists an algorithm that achieves cumulative pricing loss $L\cdot \Tilde{O}(T^{\frac{(1-u)d + u}{(1-u)d + 1} } + T^u C^{1-u} + C^{1/u})$. 
\end{theorem*}
\szcomment{TODO}

\subsection{Related Work}
\szcomment{TODO}


% The problem of learning a lipschitz function through binary feedback was initiated by~\cite{mao2018contextual}. I consider the problem when the binary signals are corrupted by an adaptive adversary, and give corruption robust algorithms for this problem. For $d=1$ under symmetric loss, I give a algorithm with loss bound $L\cdot O(C\log T)$. Here $C$ is the number of rounds with corrupted signals. Critically, the value of $C$ is \emph{unknown} to the learner. I also extend the algorithm to higher dimensions with $d \ge 2$. 

\section{Problem Formulation}
\label{sec:formulation}
\begin{definition}
Let $f: \cR^d \rightarrow \cR$. If $f$ satisfies:
\[
\abs{f(x) - f(y)} \le L\norm{x - y}
\]
for any $x, y$, then $f$ is $L$-Lipschitz with respect to the norm $\norm{\cdot}$. If the norm is the $\norm{\cdot}_\infty$ norm, then $f$ is simply $L$-Lipschitz, or that $f$ has Lipschitz constant $L$.  
\end{definition}

An adversary selects an $L$-Lipschitz function $f:\cR^d \rightarrow [0,L]$. The function $f$ is initially unknown to the learner. At each round $t$, the adversary selects a \textit{context} $x_t$ in the input space $\cX$, and the learner submits a guess $q_t$ to the value $f(x_t)$. 

After the learner submits a guess, she observe a binary signal $\sigma_t \in \set{0,1}$. The binary signal $\sigma_t$ at round $t$ may be uncorrupted or corrupted. If the signal is uncorrupted, then
\[
\sigma_t := \sigma(q_t - f(x_t)). 
\]
If the signal is corrupted, then
\[
\sigma_t := 1 - \sigma(q_t - f(x_t))
\]
Here, $\sigma(u)$ is the step function that takes value 1 if $u > 0$ and 0 if $u \le 0$. Hence an uncorrupted signal of $\sigma_t = 1$ indicates a guess too high, and a signal of $\sigma_t = 0$ indicates a guess too low. If $f(x_t) = q_t$, then the feedback can be arbitrary in $\set{0,1}$. However, the learner can add infinitesimal perturbation to $q_t$, so that $f(x_t) = q_t$ happens with 0 probability, thus subsequent analysis will not consider the case $f(x_t) = q_t$. It is assumed that at most $C$ rounds are corrupted. Critically, the value $C$ is \emph{unknown} to the learner, though it is assumed $C$ is sublinear in $T$. 

After the guess is submitted each round, the learner suffers a loss $\ell_t = \ell(q_t, f(x_t))$. Note that the loss function is known to the learner but the loss value is never revealed to the learner. The goal of the learner will be to incur as little cumulative loss as possible, where the cumulative loss is defined as
\[ 
\sum_{t=1}^T \ell_t = \sum_{t=1}^T \ell(q_t, f(x_t)). 
\]

\begin{remark}
It should be noted there is a slight difference in the range of $f$ compared with the formulation in~\cite{mao2018contextual}. In~\cite{mao2018contextual}, the authors assumed the range of $f$ to be $[0,1]$ regardless the value of $L$, where this work assume the range of $f$ to be $[0,L]$. The author believe the assumption in this paper is the more natural one, as the range of an $L$-Lipschitz function on $[0,1]^d$ is $L$. Assuming the range to be $[0,L]$ also avoids some of the unnecessary complications (in the author's opinion) that arises from the scaling of $L$ (see~\cite{mao2018contextual} appendix A). If one replaces the range $[0,L]$ to $[0,1]$, the proposed algorithms in this work leads to sharper regret bounds (see details in following sections). 
\end{remark}

\section{Algorithm for Absolute Loss}
\label{sec:symmetric}

This section gives a corruption-robust algorithm for the absolute loss, defined by
\[
\ell(q_t, f(x_t)) = \abs{f(x_t) - q_t}. 
\] 
I will first design an algorithm for $d = 1$ in this section (\cref{algo:1dabsolute}), then extend the algorithm to $d \ge 2$. 

\begin{algorithm2e}
\caption{Learning with corrupted binary signals under absolute loss for $d=1$}
\label{algo:1dabsolute}
Learner maintains a partition of intervals $I_j$ of the input space throughout the learning process\;
For each interval $I_j$ in the partition, learner stores a checking interval $S_j$, and maintains an associated range $Y_j$\;
The partition is initialized as $8$ intervals $I_j$ with length $1/8$ each, and each $I_j$ has associated range $Y_j$ and checking interval $S_j$ set to $[0,L]$\;
\For {$t = 1, 2, ..., T$} {
    Learner receives context $x_t$\;
    Learner finds interval $I_j$ such that $x_t \in I_j$\;
    Let $Y_j$ be the associated range of $I_j$\;
    \If {Exists an endpoint of $S_j$ not yet queried} { %\label{algo:queryendpt_beg}
        Learner selects an unqueried endpoint of $S_j$ as guess \;
        \If {Learner guessed $\min(S_j)$ and $\sigma_t = 1$, or guessed $\max(S_j)$ and $\sigma_t = 0$} {
            Mark $I_j$ as unsafe\;
        }
        %\Comment{Contamination found in the current interval}
        \If {Both endpoints of checking interval $S_j$ have been queried} {
            \If {$I_j$ marked unsafe} {
                Set range $Y_j := [0,L]$\;\label{algoline:marked_rangeset}
            }
            \Else {
                Set range $Y_j = [\min(S_j) - L\cdot \len(I_j)), \max(S_j) + L\cdot \len(I_j))] \cap [0,L]$\;\label{algoline:unmarked_rangeset}
            }
        } %\label{algo:queryendpt_end}
    % \Else { %{$I_t$ not marked unsafe} \label{algo:notmarked_beg}
    %     $Y_j := \algmq(I_j, Y_j)$\;
    %     \If { $\len(Y_j) < \max( 4 L\cdot \len(I_j), 4 L / T )$ }  {
    %         Bisect $I_j$ into $I_{j1}, I_{j2}$, set $S_{j1} = Y_j, S_{j2} = Y_j$\;
    %     }%\label{algo:notmarked_end}
    % }
    %\Else \Comment{$I_t$ is marked dishonest} \label{algo:dishonest_begin}
        %\State Start from scratch until converge, takes $\log T$ steps \label{algo:test}
        
        %\State Let $\set{[], [], []}$ be the sequence of feasible region (endpoints? ) on the path to $I_t$. 
        %\State Perform binary search on this path and find the first interval [] that contains $f(I_t)$. \Comment{Takes $\log\log T$ steps}, set feasible region $I_t = []$
        %\State From this point perform binary search and shrink feasible region \Comment{Takes $c$ steps, where $c$ is number of corruptions before reaching a correction interval? }
    }
    \Else { %{$I_t$ not marked unsafe} \label{algo:notmarked_beg}
        $Y_j := \algmq(I_j, Y_j)$\;
        \If { $\len(Y_j) < \max( 4 L\cdot \len(I_j), 4 L / T )$ }  {
            Bisect $I_j$ into $I_{j1}, I_{j2}$, set $S_{j1} = Y_j, S_{j2} = Y_j$\;
        }%\label{algo:notmarked_end}
    }
}
\end{algorithm2e}

\begin{algorithm2e}
\caption{Midpoint query procedure: $\algmq(I_j, Y_j)$}
\label{algo:midpt_query}
    Input: Interval $I_j$, associated range $Y_j$\;
    Let $q$ be midpoint of $Y_j$\;
    Learner queries $q$\;
    \If {$\sigma_t$ = 1}{
        % \State Set $Y'_j := [\min(Y_j), q_t + L\cdot\len(I_j)]$
        Set $Y'_j := [0, q_t + L\cdot\len(I_j)] \cap Y_j$\;
        \szcomment{This and above should be equivalent}
    }
    \Else {
        % \State Set $Y'_j := [q_t - L\cdot\len(I_j), \max(Y_j)]$
        Set $Y'_j := [q_t - L\cdot\len(I_j), 1] \cap Y_j$\;
        \szcomment{This and above should be equivalent}
    }
    Return $Y'_j$\;
\end{algorithm2e}

\subsection{Algorithm for $C=0$}

It will be helpful to first give a brief description of an algorithm that appeared in~\cite{mao2018contextual}. This algorithm works for $d=1$ and when there are no adversarial corruptions. At each point in time, the learner maintains a partition of the input space into intervals. For each interval $I_j$ in the partition, the learner also maintains an associated range $Y_j$, which serves as an estimate of the image of $I_j$. In particular, the algorithm ensures the following is true: $f(I_j)\subset Y_j, \len(Y_j) = L\cdot O(\len(I_j))$. When a context appears in $I_j$, the learner selects the midpoint of $Y_j$ as the query point, and the associated range $Y_j$ shrinks and gets refined over time. When $Y_j$ reaches a point where significant refinement is no longer possible, the learner zooms in on $I_j$ by bisecting it. This corresponds roughly to the subprocedure summarized in~\cref{algo:midpt_query}, which is termed the midpoint query procedure $\algmq$. The midpoint query procedure shall be used as a subprocedure in the corruption-robust algorithms that this work proposes. 

\subsection{Corruption-Robust Search with Agnostic Checks}

The main algorithm for absolute loss with $d = 1$ is summarized in~\cref{algo:1dabsolute}. This algorithm is corruption-robust and agnostic to the corruption level $C$. Below I give the key ideas in this algorithm. 

When there are adversarial corruptions, it will generally be impossible to tell for certain whether the associated range $Y_j$ contains the image of the interval $I_j$. That is, the learner will not know with complete certainty whether $f(I_j) \subset Y_j$ holds. The analysis divides intervals into three types: \textit{safe intervals}, \textit{correcting intervals}, and \textit{corrupted intervals}. 

\begin{definition} [Corrupted, Correcting, and Safe Intervals]
Intervals are divided into three types. Consider some interval $I_j$. 
\begin{itemize}
\item $I_j$ is a \emph{corrupted interval} if, there exists some $t$ where $x_t\in I_j$ and the signal $\sigma_t$ is corrupted. %for any context that appears in the interval, the signal is corrupted gives a corrupted signal. 
\item $I_j$ is a \emph{correcting interval} if its parent interval is a \emph{corrupted interval} and for every round $t$ where $x_t \in I_j$, the signal $\sigma_t$ is uncorrupted. 
\item An interval is a \emph{safe interval} if its parent interval is safe or correcting (or itself is a root interval), and for any round $t$ where $x_t\in Y_j$, the signal $\sigma_t$ is uncorrupted. %A root interval with no corruption is also \emph{safe}. 
\end{itemize}
\end{definition}

I introduce the \textit{agnostic checking} measure to combat adversarial corruptions. Consider a new interval $I_j$ that has been formed by bisecting its parent interval, and that a context $x_t$ appears in this interval $I_j$. The learner will first query the two endpoints of $Y_j$ and test whether $f(I_j)\subset Y_j$ holds (according to the possibly corrupted signals). The interval passes the agnostic check if according to the (possibly corrupted) signals $f(I_j) \subset Y_j$. If the interval does not pass the agnostic check, it is marked as \emph{unsafe}, and the learner resets $Y_j$ to $[0, L]$ and effectively searches from scratch for the associated range. 

The main theorem is stated below. 

\begin{theorem}
\label{thm:symm1D}
\Cref{algo:1dabsolute} incurs $L\cdot O(C \cdot \log T)$ cumulative absolute loss for $d = 1$. 
\end{theorem}

\begin{proof}[Proof Sketch]
The proof consists of several steps. 
\paragraph{Step 1.} An interval not marked unsafe bisects in $O(1)$ rounds. This is because the associated range is shrinking by a constant after each query and after $O(1)$ rounds the associated range will have shrunk enough and meet the criteria for bisecting the interval. By a similar logic, any interval marked unsafe bisects in $O(\log T)$ rounds. 
\paragraph{Step 2.} Consider any \emph{correcting} interval. After the agnostic checking steps, the associated range must contain the true range of the function on the interval. The associated range will then be accurate when the interval is bisected. Then consider any \emph{safe} interval. Its parent must be safe or correcting, and by induction, a \emph{safe} interval will not be marked \emph{unsafe}. 
\paragraph{Step 3.} There can be at most $O(C)$ \emph{corrupted} intervals. Since any \emph{correcting} interval has a \emph{corrupted} interval as a parent, there can be at most $O(C)$ correcting intervals. A total of $O(\log T)$ queries can occur on corrupted and correcting intervals, contributing loss $O(C\log T)$ (actually one can show correcting intervals contribute loss at most $O(C)$). For \emph{safe} intervals, a loss of magnitude $O(2^{-h})$ can occur at most $O(2^h)$ times (since there are $O(2^h)$ intervals at depth $h$), thus the total loss from safe intervals can be bounded by $O(\log T)$. 

Putting everything together gives the total $L\cdot O(C\cdot \log T)$ regret bound. 
\end{proof}

\begin{remark}
The regret is tight up to $\log T$ factors. To see this, consider the first $C$ rounds, where the adversary corrupts the signal with probability $1/2$ each round. The learner essentially receives no information during this period, and each round incurs regret $\Omega(L)$. 
\end{remark}

% A detailed analysis appears in~\cref{app:proof1d}. The main components of the proof include the following. 
% \begin{enumerate}
%     \item Since any corrupted interval is bisected in $O(\log T)$ rounds, the total loss from corrupted interval can be bounded as $O(C\log T)$. 
%     \item There are at most $O(C)$ amending intervals, and each amending interval contribute $O(1)$ loss. 
%     \item A safe internval at depth $h$ contribute loss $O(2^{-h})$. There are at most $O(2^h)$ intervals at depth $h$, hence a loss with $O(2^{-h})$ can be charged at most $O(2^h)$ times. 
% \end{enumerate}


\begin{remark}
    An interesting aspect of this algorithm is that the learner remains agnostic to the type of each interval. In other words, during the run of the algorithm, the learner will not know for certain which type an interval belongs to. The analysis makes use of the three types of the interval, not the algorithm itself. 
\end{remark}


% If the signal at these two endpoints tells the learner $f(I_j) \notin Y_j$, then the learner will know either the signal at these two endpoints are corrupted, or that corruptions in its parent interval lead to an inaccurate $Y_j$ that failed to contain the image of $I_j$. In either case, the learner marks the interval $I_j$ as an unsafe interval, resets the associated range $Y_j$ to $[0,1]$, and thus effectively start from scratch and searches for the associated range $Y_j$ of the interval $I_j$. 

% If the signal at these two endpoints tells the learner that $Y_j$ is indeed an accurate upper bound for $f(I_j)$, then the interval is not marked unsafe. However, it should be noted that, even if the interval is not marked unsafe, the learner still cannot necessarily ensure that $Y_j$ contains the image $f(I_j)$, since the signal at these two endpoints may be corrupted. The learner has no way of verifying whether the feedback was accurate, and will act as if $Y_j$ is indeed an upper bound on the image of $f(I_j)$, and apply the midpoint querying strategy to refine the estimate $Y_j$. If $Y_j$ was indeed an accurate upper bound on the $f(I_j)$ and signals on the interval $I_j$ were accurate, then the learner will incur small loss on this interval. Otherwise if $Y_j$ was not an accurate upper bound on $f(I_j)$, then it must mean that the signal when querying the endpoints were corrupted, and hence the learner will incur regret as a consequence of the corrupted signals. However we shall see that the total loss incured as a result of the corrupted signals can be bounded above by $\tilde{O}(C)$, where recall $C$ is the number of corruptions. 

%The learner starts with a coarse partition of the input space into intervals, and for each interval $I_j$ maintains a interval $Y_j$ that contains the image of this interval. As the feasible interval $Y_j$ shrinks enough, the interval $I_j$ is split in half, thus allowing for finer estimates of the image of the child intervals. 


% \subsection{Analysis}
% A detailed analysis of algorithm~\cref{algo:1dabsolute} is given. Some definitions are first introduced that will be helpful in the analysis. 

% \begin{definition}[Depth]
%     The intervals at initialization has depth 1, and each interval has depth increased by 1 when split. 
% \end{definition}
% The depth of an interval measures how many splits happened before reaching the current interval. 

% Next, three types of intervals are introduced. 
% \begin{definition}[Corrupted Interval]
%     An interval is called a corrupted interval if any round inside the interval is corrupted. 
% \end{definition}

% \begin{definition}[Correcting Interval]
%     An interval is called a correcting interval if any round inside the interval is uncorrupted and the parent interval is corrupted. 
% \end{definition}

% \begin{definition}[Honest Interval]
%     An interval is called a honest interval if any round inside the interval is uncorrupted and the parent interval is either a correction interval or a honest interval. As a special case, intervals at depth 1 are honest if they are not a corrupted interval. 
% \end{definition}

% It can be seen that these three types of intervals are disjoint, and their union is the set of all intervals reached by the learner in the learning process. 

% We begin with the following lemmas. 

% \begin{restatable}{lemma}{markedAreNotHonest}
%     If interval marked dishonest, then must be correcting or corrupted. 
% \end{restatable}

% \begin{restatable}{lemma}{markedSplits}
%     If an interval marked dishonest, then splits in $O(\log T)$ rounds. If interval not marked dishonest, then splits in $O(1)$ rounds. 
% \end{restatable}

% \begin{restatable}{lemma}{correctionCorrects}
%     Let $I_j$ be a correction interval and $Y_j$ be its feasible interval before $I_j$ is split. Then $f(I_j) \in Y_j$. 
% \end{restatable}

% \begin{restatable}{lemma}{corruptedLoss}
%     Corrupted interval contribute $O(C \log T)$ loss. 
% \end{restatable}

% \begin{restatable}{lemma}{correctingLoss}
%     Correcting interval contribute $O(C)$ loss. 
% \end{restatable}


% \begin{restatable}{lemma}{honestLoss}
% Consider an honest interval in depth $h$. This interval is split within $O(1)$ rounds, and each round incur $O(2^{-h})$ regret. 
% \end{restatable}


\szcomment{Optimal 1d lipschitz? }
\szcomment{1/2, 1/2, 3/4, 3/4, 15/16, 15/16, 35/32, 35/32, 315/256}

\subsection{Extending to $d > 1$}
\Cref{algo:1dabsolute} can be extended in a straightforward way to accommodate the case $d > 1$. The full algorithm and analysis is given in~\cref{app:sym-highd}. The only difference is that the learner maintains $d$-dimensional hypercubes instead of intervals. The main theoretical result is as follows. 
\begin{theorem}
\label{thm:symmMD}
    There exists an algorithm that incurs $L\cdot O_d(C\log T +  T^{(d-1) / d)})$ cumulative absolute loss for $d\ge 2$. 
\end{theorem}

\begin{remark}
In~\cite{mao2018contextual}, it was shown the optimal regret when $C=0$ is $\Omega(T^{(d-1)/(d)})$. Hence, the dependence on $C$ and $T$ are both optimal in the above theorem (up to $\log T$ factors). 
\end{remark}



\section{Algorithm for Pricing Loss}
\label{sec:pricing}

\begin{algorithm2e}[htb]
\caption{Learning with corrupted binary signal under pricing loss with uniform discretization}
\label{algo:pricing}
Set parameter $\eta_0 = T^{-1/(d+1)}$, agnostic check schedule $\tau_0$\;
Learner uniformly discretizes input space into hypercubes with lengths no larger than $\eta_0$ each\;
For each hypercube $I_j$, learner maintains an associated range $Y_j$ (initialized as $[0, L]$), query count $c_j$ (initialized as $0$)\;
% \State Set $\eta = 10 L \cdot T^{1/(d+1)}$
\For {$t = 1, 2, ..., T$} {
    Learner receives context $x_t$\;
    Learner finds hypercube $I_j$ such that $x_t \in I_j$\;
    Let $Y_j$ be the associated range of $I_j$\;
    \If {$\len (Y_j) < 10 L \cdot \eta$} { %\Comment{The interval is pricing-ready}
        $c_j := c_j + 1$\;
        \If { $c_j > \tau_0$ } {
            Query $\max(Y_j)$\;
            Set $c_j := 0$\;
        }
        \Else { 
            Query $\min(Y_j)$\;
        }
        \If {Learner is surprised}{  %\LineComment{Learner is surprised if queried $\max(Y_j)$ and $\sigma_t = 0$, or queried $\min(Y_j)$ and $\sigma_t = 1$}
            Set $Y_j := [0,L]$\;%\Comment{Reset the range of $Y_j$}\;
            Set $c_j := 0$\;
        }
    }
    % \ElsIf {Exists an endpoint in $S_j$ not queried} %\label{algo:queryendpt_beg}
    %     \State Query an unqueried endpoint of $S_j$
    %     \If {Queried $\min(S_j)$ and $\sigma_t = 1$, or queried $\max(S_j)$ and $\sigma_t = 0$} 
    %     \State Mark $I_j$ as unsafe \Comment{Contamination found in the current interval}
    %     \EndIf
    %     \If {Both endpoints of $S_j$ have been queried}
    %         \If {$I_j$ marked unsafe}
    %             \State Set range $Y_j := [0,1]$ \label{algoline:marked_rangeset}
    %         \Else
    %             \State Set range $Y_j = [\min(S_j) - L\cdot \len(I_j)), \max(S_j) + L\cdot \len(I_j))] \cap [0,1]$ \label{algoline:unmarked_rangeset}
    %         \EndIf
    %     \EndIf %\label{algo:queryendpt_end}
    \Else {%{$I_t$ not marked unsafe} \label{algo:notmarked_beg}
        $Y_j := \algmq(I_j, Y_j)$\;
        % \If { $\len(Y_j) < 4 L\cdot \len(I_j)$}  \Comment{associated range $Y_j$ has shrunk enough}
        %     \State Bisect each side of $I_j$ to form $2^d$ new hypercubes, each with length $...$
        % \EndIf %\label{algo:notmarked_end}
    }
    %\Else \Comment{$I_t$ is marked dishonest} \label{algo:dishonest_begin}
        %\State Start from scratch until converge, takes $\log T$ steps \label{algo:test}
        
        %\State Let $\set{[], [], []}$ be the sequence of feasible region (endpoints? ) on the path to $I_t$. 
        %\State Perform binary search on this path and find the first interval [] that contains $f(I_t)$. \Comment{Takes $\log\log T$ steps}, set feasible region $I_t = []$
        %\State From this point perform binary search and shrink feasible region \Comment{Takes $c$ steps, where $c$ is number of corruptions before reaching a correction interval? }
    %\EndIf %\label{algo:dishonest_end}
}
\end{algorithm2e}


This section discusses the new ideas needed to design corruption-robust algorithm for the pricing loss. The description shall be given in the context of dynamic pricing, and the learner shall be referred to as the seller in this section. The main algorithm is summarized in~\cref{algo:pricing}. Note that for the pricing loss, the case with $d =1$ and $d > 1$ are treated together. 

Extending~\cref{algo:1dabsolute} for the symmetric loss to pricing loss is not straightforward, since agnostic checks will overprice and the seller necessarily incurs a large loss whenever she overprices. The learner did not have this problem with symmetric loss, since the symmetric loss is continuous. 

\subsection{Algorithm for $C = 0$}
I first give the description of an algorithm for $C = 0$. The algorithm starts with a uniform discretization of the input space into hypercubes with length $\eta_0 = T^{-1/(d+1)}$. When a context appears in some hypercube $I_j$, one of two things can happen: 
\begin{enumerate}
    \item If the associated range is larger than $10L\cdot \eta_0$, the learner performs $\algmq$ and updates the associated range. There can be at most ${O}(\eta_0^d \log T)$ rounds of this type, since for each hypercube the associated range will shrink below $L\cdot O(\eta_0)$ after $O(\log T)$ queries. 
    \item Otherwise, the associated range is less than $10L\cdot \eta_0$, and the learner can directly set the lower end of the interval as the price. The total loss from these rounds can be bounded as $L\cdot O(T\eta_0)$. 
\end{enumerate}
The total loss is then $L (\eta_0^d \log T) + L(T\eta_0) = L\cdot {O}(T^{d/(d+1)} \log T)$. 

\szdelete{
The learner searches for the associated range $Y_j$ of a hypercube using the midpoint query procedure $\algmq$ (these shall be termed \emph{searching rounds} or \emph{searching queries}) and stops the search process when the associated range $Y_j$ becomes small enough (specifically, when the length of $Y_j$ drops below $10L\cdot \eta_0$). The hypercube is said to become \emph{pricing-ready} when this happens. When there are no adversarial corruptions, the learner now has a good estimate of the optimal price and sets the lower end of $Y_j$ as the price for any context in this hypercube. These shall be termed \emph{pricing rounds}. At \emph{pricing rounds}, the seller expects the buyer to purchase the item, and assuming the range $Y_j$ is accurate (which is the case when $C = 0$), the revenue loss should be no larger than $10 L \cdot \eta$. The total loss from \emph{pricing rounds} can be bounded by $L\cdot O(T\cdot \eta_0)$. The total loss from searching rounds can be bounded by $L\cdot  \widetilde{O}(\eta_0^{-d})$ since there are $O(\eta_0^{-d})$ hypercubes, and for each hypercube, there can be at most $O(\log T)$ queries before the hypercube becomes \emph{pricing-ready}. By the choice of parameter $\eta_0 = T^{-1/(d+1)}$, the regret bound is $L\cdot \widetilde{O} (T^{d/(d+1)})$. }

\szdelete{
It should be noted that~\cite{mao2018contextual} proposed an algorithm based on adaptive discretization whereas the above algorithm is based on uniform discretization. Using adaptive discretization improves the regret bound by a $O(\log T)$ factor, the presentation for the corruption-robust algorithm for pricing loss is based on uniform discretization, as it highlights the new algorithmic ideas more clearly. Nevertheless, the algorithm can be combined with adaptive discretization by using ideas from the previous section. }

\szcomment{TODO. Compare with adaptive discretization. }\szcomment{DONE}

\subsection{Corruption-Robust Pricing with Scheduled Agnostic Checks}

The seller could potentially run into issues when there are adversarial corruptions. The adversary can manipulate the seller into underpricing by a large margin by only corrupting a small number of signals. Consider the following example. The buyer is willing to pay $0.5$ for an item with context $x$. The adversary can manipulate the signals in the $\algmq$ procedure so that the seller's associated range for $x$ is $[0, 10L\cdot \eta]$, which does not contain and is well below the optimal price $0.5$ for this item. The seller then posts a price of $0$ for item $x$. Even though the buyer purchases the item at a price of $0$, the seller is losing 0.5 revenue per round, and this happens without the adversary corrupting only $O(\log T)$ rounds. \szcomment{TODO, rephrase}

\szcomment{ $Y_j$ is not a valid pricing range, and even though she prices the item at the lower end of $Y_j$ and the buyer is purchasing the item, the true valuation could be much higher and the seller is losing a large amount of revenue. }

To combat adversarial corruptions, the learner performs agnostic check queries. These queries differ from agnostic checks for the symmetric loss in the following two aspects. First, whereas for the symmetric loss, the learner performs agnostic check queries on both ends of the associated range $Y_j$, for the pricing loss the seller only performs agnostic checks on the upper end of $Y_j$. This ensures the seller is not underpricing the product by a large margin. Second, the seller only performs agnostic check queries when the associated range of the hypercube is sufficiently small (below $L\cdot O(\eta_0)$). The seller expects the buyer not the purchase the item during agnostic check queries. 

The learner will set a schedule for performing agnostic checks. At a high level, performing agnostic checks too often incurs large regret from overpricing, while performing agnostic checks too few may not detect corruptions effectively. The schedule for agnostic checks serves as a mean to balance the two. This schedule informs the learner how often to perform agnostic check so as not to incur large regret when overpricing, and at the same time control the loss incurred from corrupted signals. Specifically, the learner keeps track of how many rounds the context vectors arrived in each hypercube. The learner then performs an agnostic check every $\tau_0$ rounds. Here, $\tau_0$ will be a paramter to be chosen later. 

Some notions are introduced. 
\begin{definition} [Pricing-ready hypercubes]
For hypercube $I_j$, if the length of the associated range is below $10L\cdot \eta_0$, then the hypercube is \emph{pricing-ready}. 
\end{definition}

\begin{definition} [Pricing round, checking round, searching round]
If a hypercube is \emph{pricing-ready}, then setting the price as the lower end of the associated range is termed \emph{pricing round}, and setting the price as the upper end of the associated range is termed \emph{checking round}. If a hypercube is not \emph{pricing-ready}, then performing a $\algmq$ is termed \emph{searching round}. 
\end{definition}

\begin{definition}[Surprises]
    The seller is said to become \emph{surprised} when the signal she receives is inconsistent with her current knowledge. Specifically, the seller becomes surprised when either: she performs a checking round but observes an underprice signal $(\sigma_t = 0)$; or she performs a pricing round but observes an overprice signal $(\sigma_t = 1)$. 
\end{definition}

\begin{definition} [Runs]
Fix some hypercube $I_j$, the set of queries that occurs on $I_j$ before the learner becomes \emph{surprised} (or before the algorithm terminates) called a \emph{run}. 
\end{definition}
When a run ends, the learner resets the associated range and a new run begins. Note there can be multiple runs on the same hypercube. 


The algorithm gives the following cumulative loss bound. A detailed analysis can be found in~\cref{app:pricing}. 
\begin{theorem}
\label{thm:pricing}
Using~\cref{algo:pricing}, the learner incurs a total loss
\[
L\cdot O(C\log T + T^{d/(d+1)}\log T + C\tau_0 + T/\tau_0). 
\]
Specifically, setting $\tau = T^{1/(d+1)}$, the learner incurs total loss
\[
L\cdot O(T^{d/(d+1)}\log T + C\log T + C\cdot T^{1/(d+1)}) = L\cdot \widetilde{O} (T^{d/(d+1)} + C\cdot T^{1/(d+1)} ). 
\]
\end{theorem}

\begin{proof}(Proof Sketch. )
The proof consists of several steps. 
\paragraph{Step 1.} The seller can become only become surprised when at least one corruption occurs on the current `run' of some hypercube. Thus the seller can become surprised at most $C$ times. 
\paragraph{Step 2. }For searching rounds, the total loss can be bounded as $L\cdot O( (C + \eta_0^{-d}) \log T)$. There can be at most $O(T/\tau_0)$ checking rounds, contributing loss $L\cdot O(T /\tau_0)$. 
\paragraph{Step 3. } For pricing rounds, the total loss can be bounded as $L\cdot O(C \tau_0 + T \eta_0)$. At a very high level, if the pricing interval is accurate (meaning the lower endpoint is just below the true price by a margin of $L\cdot O(\tau_0)$), then these rounds contribute loss at most $O(T \eta_0)$. Otherwise, the seller can detect if the pricing interval is inaccurate by performing agnostic checks, and the loss from these rounds can be bound by $O(C \tau_0)$. 

Putting everything together completes the proof. 
\end{proof}

\begin{remark}
The loss is sublinear as long as $C = o(T^{d/(d+1)})$. In~\cite{mao2018contextual}, it was shown the optimal regret is $\Omega(T^{d/(d+1)})$ for $C = 0$. Hence, the above theorem is optimal when $C = O(T^{ (d-1)/(d+1) })$. 
\end{remark}


If the asymptotic order of $C$ is known, the learner can set $\tau_0$ to balance the terms and achieve a sharper regret bound. 
\begin{corollary}
Assume the asymptotic order of $C$ satisfies $C = O(g(T))$ and is known to the learner. Here, $g$ is some sublinear function. Using~\cref{algo:pricing} with $\tau_0 = \sqrt{T/g(T)}$, the learner incurs total loss
\[
L\cdot \widetilde{O} (T^{d/(d+1)} + \sqrt{TC}). 
\]
\end{corollary}
\begin{remark}
When $C = O(T^{(d-1)/(d+1)})$, the loss simplifies to $L\cdot \widetilde{O} (T^{d/(d+1)})$. 
When $C = \Omega(T^{(d-1)/(d+1)})$, the loss simplifies to $L\cdot \widetilde{O} (\sqrt{TC})$. 
\end{remark}



%\section{Conclusion}
% I studied the problem of learning a Lipschitz function with corrupted binary feedback, and proposed corruption-robust algorithms. 
%\section{Covering Dimension Based Bounds}
This sections gives a corrupted robust algorithm for learning general hypothesis classes with finite covering dimension. The problem setup remains the same with the Lipschitz function class being replaced by a general hypothesis class. It is shown here that the density update algorithm in~\cite{liu2021optimal} achieves loss $O(Cd\log T)$. 

Let $\cF$ be a hypothesis class and $\cX$ be the input space. For any function $f\in\cF$, $f$ maps elements in $\cX$ to $\cR$. An adversary selects a ground truth hypothesis $f^*$. At each round, the adversary selects an input $x_t$, and the learner submits a guess $q_t$ upon observing the input. The adversary then gives the learner a binary feedback $\sigma_t$, and the learner suffers the symmetric loss $\ell_t = \abs{f^*(x_t) - q_t}$. 

The algorithm works as follows. The learner takes a $\frac{1}{T^2}$-covering of the hypothesis space $\cF$.  Denote the covering set by $C_\cF$, then there exists a element $f_0$ in the covering set such that $\norm{f_0 - f^*}_\infty < 1/T^2$. 

At initialization, the learner sets a uniform density in the covering set. Subsequently in each round, the learner finds the number $m_t$ such that $y_t$ is the median and queries $q_t$, a perturbed version of $m_t$. 

The analysis focuses on the density of $f_0$. Assume in some round, the query point is far from $f^*(x_t)$ with a distance at least $2/T$ and that the signal was uncorrupted, then density of $f_0$ will be multiplied by a constant. For any corrupted rounds, the density of $f_0$ is shrinks by at most a constant factor. 

\begin{theorem}
    Main theorem. 
\end{theorem}

\szcomment{Moved the proof to appendix. }




% \section{Tight Bound for 1d loss}
% Idea. Express loss in recursive fashion. A large loss implies small loss in future rounds. While a small loss implies potential larger loss in future rounds. Though the loss is never revealed, so this adds difficulty. 

% The midpoint algorithm by ??[][] have $\log T$ loss when querying dyadic numbers. Short proof: assume the function was actually constant. Each dyadic number presented twice... 



%Bibliography
\bibliographystyle{unsrt}  
\bibliography{references}  



\newpage
\begin{appendices}

%\section{Effect of $L$}
This sections gives a brief description on how $L$ affects the loss guarantees. If $L < 1$, then the function is assumed to take the range $[0, L]$. 

If $L > 1$, then the function is assumed to take the range $[0,1]$. 

\szcomment{Other option is to assume function take range $[0, L]$. }



\section{Proof Details for Symmetric Loss}
\label[appendix]{app:proof1d}

%In the analysis, the assumption is made that $L\ge 1$. If $L < 1$, a linear transformation can be applied on the function $f$ so that the range of $f$ becomes $[0,1]$ and the Lipschitz constant becomes $L = 1$ after the transformation. See the auxiliary results (lemma 14, lemma 15) in~\cite{mao2018contextual} for more details. 

%Recall the definition of corrupted, correcting and honest interval. 
The definition of corrupted interval, correcting interval, and safe interval will be restated for convenience. An interval is a corrupted interval if any signal of a query within the interval was corrupted. An interval is an correcting interval if its parent interval is corrupted and any signal of queries within the interval was uncorrupted. An interval is safe if its parent interval is safe or correcting. A root interval with no corrupted signals is also safe. 
\szcomment{rename to safe / unsafe / corrupted / amending / contaminated}

An interval has depth $r$ if it was bisected from an interval at depth $r-1$. Initial root intervals have depth 0. Hence, an interval $I_j$ at depth $r$ has length $\len(I_j) = 2^{-r}$. 


The below lemma shows that each call to $\algmq$ preserves the property $f(I_j)\in Y_j$. 
\begin{lemma}
\label{lemma:updateValid}
    Consider a call to the $\algmq(I_j, Y_j)$ procedure. Let $Y_j$ be the associated range before the query, and let $Y'_j = \algmq(I_j, Y_j)$ be the range after the query. If the signal was uncorrupted and $f(I_j) \in Y_j$, then $f(I_j) \in Y'_j$. 
    % Let $I_j$ be any interval and $Y_j$ the associated range that appears at any point in the learning process. Let $Y'_j = \algmq(I_j, Y_j)$. If $f(I_j) \in Y_j$, then $f(I_j) \in Y'_j$. 
\end{lemma}

\begin{proof}
Let $q_t$ be the query point (i.e. midpoint of $Y_j$). Assume the signal $\sigma_t = 0$, so that $f(x_t) \ge q_t$. The case where $\sigma_t = 1$ will be similar. 

Since $f$ is $L$-Lipschitz, it must be that 
\[
f(I_j) \ge f(x_t) - L\cdot \len(I_j) \ge q_t - L\cdot \len(I_j), 
\]
hence the update 
\[
Y'_j = Y_j\cap [q_t - L\cdot \len(I_j) , L]
\]
still guarantees that $f(I_j) \in Y'_j$. 
\end{proof}

The next lemma shows the length of $Y_j$ shrinks by a constant factor after each call to $\algmq$, regardless of whether the signal was corrupted or not. 
\begin{lemma}
\label{lemma:rangeShrinks}
Consider a call to the $\algmq(I_j, Y_j)$ procedure. Let $Y_j$ be the associated range before the query, and let $Y'_j = \algmq(I_j, Y_j)$ be the range after the query. Suppose $\len(Y_j) \ge 4L\cdot \len(I_j)$, then $\frac{1}{2} \len{(Y_j)} \le \len{(Y'_j)} \le \frac{3}{4} \len{(Y_j)}$ regardless of whether the signal is corrupted or not. 
\end{lemma}

\begin{proof}
    Let $q_t$ be the query point (i.e. midpoint of $Y_j$). Assume the signal $\sigma_t = 0$, the case with $\sigma_t = 1$ will be similar. 
    
    Before the call to $\algmq(I_j, Y_j)$, the learner has $\len(Y_j) \ge 4L\cdot \len(I_j)$. This implies 
    \[
    q_t - \min(Y_j) = \frac{\len(Y_j)}{2} \ge 2L\cdot \len(I_j)
    \]
    so that
    \[
    q_t - L\cdot \len(I_j) > \min(Y_j). 
    \] 


    The update can be written as
    \begin{align*}
        Y'_j &= [q_t - L\cdot \len(I_j), L] \cap Y_j\\
        &= [q_t - L\cdot \len(I_j), q_t] \cup [q_t, \max(Y_j)]
    \end{align*}
    The lemma then follows from $\len{( [q_t, \max(Y_j)] )} = \frac{1}{2} \len{ ( Y_j ) }$, $\len{ ( [q_t - L\cdot \len(I_j), q_t] ) } = L\cdot \len(I_j) \le \frac{1}{4} \len{ ( Y_j ) }$. 
\end{proof}


\begin{lemma}
\label{lemma:bisectFast}
    If an interval is marked unsafe, then the interval is bisected after $O(\log T)$ queries. If an interval is not marked unsafe, then the interval is bisected after $O(1)$ queries. 
\end{lemma}

\begin{proof}
    If the interval is marked unsafe, then the range of the interval is reset to $[0,L]$ in line~\ref{algoline:marked_rangeset} of algorithm~\ref{algo:1dabsolute}. By Lemma~\ref{lemma:rangeShrinks}, the range $Y_j$ shrinks by at least a factor of $\frac{3}{4}$ each query, hence after $O(\log T)$ queries, the learner has $\len(Y_j) < \max(4L\cdot \len(I_j), 4 L / T) $. 

    If the interval were not marked unsafe, then the range $Y_j$ in line~\ref{algoline:unmarked_rangeset} is updated as
    \[
    Y_j = [\min(S_j) - L\cdot \len(I_j), \max(S_j) + L\cdot \len(I_j)] \cap [0,L]. 
    \]

    If the interval has depth less than $\log T$, then $\len(S_j) \le 8 L \cdot \len(I_j) $, and after the update the learner has $\len(Y_j) \le 10 L \cdot \len(I_j)$. If the interval has a depth larger than $\log T$, then $\len(S_j) \le 8 L / T$,  and after the update $\len(Y_j) \le 10L / T$. In either case, Invoking Lemma~\ref{lemma:rangeShrinks} again, the range shrinks by constant factor each query and it takes $O(1)$ rounds for the range to shrink below $\max ( 4L \cdot \len(I_j), 4 L / T)$. 
\end{proof}

\begin{lemma}
\label{lemma:amendingFixes}
    Let $I_j$ be an correcting interval. When $I_j$ is bisected, it holds that $f(I_j) \in Y_j$. 
\end{lemma}

\begin{proof}
    It is first shown that $f(I_j) \in Y_j$ after the update following the agnostic check procedure. There are two cases to consider, whether $I_j$ has been marked unsafe or not after the agnostic check. 
    
    If $I_j$ has been marked unsafe, then the update in line~\ref{algoline:marked_rangeset} resets $Y_j$ to $[0,L]$, hence the learner has $f(I_j) \in Y_j$ trivally. 
    
    % If $I_j$ has been not marked unsafe after the two endpoints had been queried, then the associated range is reset to $[0,1]$, hence the learner effectively searches from scratch. 

    If $I_j$ has not been marked unsafe, the update in line~\ref{algoline:unmarked_rangeset} takes place. Since $I_j$ is an correcting interval, the signals are uncorrupted, and by Lipschitzness of $f$, the learner has
    \begin{align*}
        f(I_j) \ge \min(S_j) - L\cdot\len(I_j) \\
        f(I_j) \le \max(S_j) + L\cdot\len(I_j). 
    \end{align*}
    Hence the learner has $f(I_j) \in Y_j$ after the update. 
        
    Putting these two cases together, after the update on the associated range based on the results of the agnostic checking steps, the learner has $f(I_j) \in Y_j$. By inducting on Lemma~\ref{lemma:updateValid}, repeated calls to $\algmq$ guarantees $f(I_j) \in Y_j$ when $I_j$ is bisected. 
\end{proof}

\begin{lemma}
\label{lemma:safeNotMarked}
    Safe intervals are not marked unsafe. 
\end{lemma}

\begin{proof}
Induction is used to show the following for any safe interval $I_j$: 
\begin{enumerate}
    \item $f(I_j) \in S_j$
    \item $f(I_j) \in Y_j$ when $I_j$ is bisected
\end{enumerate}


Let $I_j$ be a safe interval. If $I_j$ is a root interval, then trivially $f(I_j)\in S_j$. Further by Lemma~\ref{lemma:updateValid}, $f(I_j) \in Y_j$ when $I_j$ is split. 

If $I_j$ has an correcting interval as its parent interval, then by Lemma~\ref{lemma:amendingFixes}, $f(I_j) \in S_j$, and consequently by Lemma~\ref{lemma:updateValid} $f(I_j) \in Y_j$ when $I_j$ is split. 

If $I_j$ has a safe interval as its parent interval, then a simple induction shows the desired result. 
% \begin{enumerate}
%     \item $f(I_j) \in S_j$
%     \item $f(I_j) \in Y_j$ when $I_j$ is bisected
% \end{enumerate}
% Hence safe intervals are never marked unsafe. 
    % If $f(I_j) \in [\lend(I_j), \rend(I_j)]$ and $I_j$ is a safe interval, then $f(I_j) \in Y_j$ when $I_j$ is bisected. The lemma then follows by inducting on the depth of intervals. 
\end{proof}

The following corollary follows directly from Lemma~\ref{lemma:bisectFast} and Lemma~\ref{lemma:safeNotMarked}. 
\begin{corollary}
\label{cor:safeBisectFast}
Safe intervals bisect in $O(1)$ rounds. 
\end{corollary}


% \begin{restatable}{lemma}{correctionCorrects}
%     Let $I_j$ be a correction interval and $Y_j$ be its feasible interval before $I_j$ is split. Then $f(I_j) \in Y_j$. 
% \end{restatable}
% \begin{proof}
% There are two cases to consider, whether $I_j$ had been marked dishonest after its two endpoints are queried. 

%     First consider the case that $I_j$ was not marked dishonest, then it must be the case that the feasible region is contained in $Y_j$ after the update in line~\ref{algo:queryendpt_end}. 

%     Otherwise $I_j$ is marked dishonest. Then the learner searches from scratch, thus he still finds the correct feasible region. 
% \end{proof}

\begin{lemma}
\label{lemma:corruptedLoss}
    Corrupted interval contribute $L\cdot O(C \log T)$ loss. 
\end{lemma}
\begin{proof}
    There are at most $C$ corrupted intervals, and each interval splits in $O(\log T)$ rounds by Lemma~\ref{lemma:bisectFast}, with each round incurring $O(L)$ regret (trivially). Hence corrupted interval contribute total $L \cdot O(C\log T)$ regret. 
\end{proof}

\begin{lemma}
\label{lemma:amendingLoss}
    Correcting intervals contribute $L\cdot O(C)$ loss. 
\end{lemma}
\begin{proof}
    There are at most $2C$ correcting intervals. For each correcting interval $I_j$, querying the two endpoints of $S_j$ incur $O(L)$ regret. Each call to $\algmq(I_j, Y_j)$ incur $O(\len(Y_j))$ regret. By Lemma~\ref{lemma:rangeShrinks}, $\len(Y_j)$ shrinks by a factor of $\frac{3}{4}$, hence the loss is geometrically decreasing with each query, and the total loss incurred within $I_j$ is $O(L)$. 
\end{proof}

\begin{lemma}
\label{lemma:safeLoss}
Consider a safe interval $I_j$ at depth $h < \log T$. Each query in $I_j$ incur $L\cdot O(2^{-h})$ regret. 
\end{lemma}

\begin{proof}
    Consider a safe interval $I_j$. Suppose the depth $h < \log T$. The two queries on the endpoints incur $O(2^{-h})$ loss, since 
    \[
    \len(S_j) < 8 L \cdot \len(I_j) = O(2^{-h}). 
    \]
    Each call to $\algmq(I_j, Y_j)$ also incurs $O(2^{-h})$ loss since $\len(Y_j) = O(L\cdot \len(I_j)) = O(2^{-h})$. By Corollary~\ref{cor:safeBisectFast}, safe intervals splits in $O(1)$ rounds, hence the loss incurred at interval $I_j$ is $O(2^{-h})$. \szcomment{TODO}
\end{proof}

\begin{theorem}[\Cref{thm:symm1D} restated]
\Cref{algo:1dabsolute} achieves cumulative symmetric loss $L\cdot O(C\log T)$. 
\end{theorem}
\begin{proof}
    By Lemma~\ref{lemma:corruptedLoss} and Lemma~\ref{lemma:amendingLoss}, corrupted interval and correcting interval contribute total $L\cdot O(C\log T)$ loss. Hence it is only needed to bound the loss of all safe intervals. 

    Note that there can be at most $O(2^h)$ intervals at depth $h$, thus considering all safe intervals at depth $h$, their total loss is at most $L\cdot 2^h \cdot O(2^{-h}) = L\cdot O(1)$ by Lemma~\ref{lemma:safeLoss}. For any safe interval that has a depth larger than $\log T$, each query incurs loss $L\cdot O(\nicefrac{1}{T})$, hence all queries with a depth larger than $\log T$ incur a total loss at most $L\cdot O(1)$. Consequently, all safe intervals incur loss $L\cdot O(\log T)$. 

    Putting these together, the algorithm incurs loss $L\cdot O(C\log T)$. 
\end{proof}

\subsection{Symmetric Loss with $d > 1$}
\label[appendix]{app:sym-highd}



\begin{algorithm}[h]
\caption{Learning with corrupted binary signal under symmetric loss with $d > 1$}
\label{algo:highDsymm}
\begin{algorithmic}[1]
\State Learner maintains a partition of hypercubes $I_j$ of the input space $[0,1]^d$ throughout learning process
\State For each hypercube $I_j$, learner stores a sanity check interval $S_j$ and maintains an associated range $Y_j$
\State The partition is initialized as a single hypercube with sanity check interval $S_j$ set to $[0, 8L]$
\For {round $t = 1, 2, ..., T$}
    \State Learner receives context $x_t$
    \State Learner finds hypercube $I_j$ such that $x_t \in I_j$
    \State Let $Y_j$ be the feasible interval of $I_j$
    \If {Exists an endpoint of $S_j$ not queried} %\label{algo:queryendpt_beg}
        \State Query an unqueried endpoint of $S_j$
        \If {Queried $\min(S_j)$ and $\sigma_t = 1$, or queried $\max(S_j)$ and $\sigma_t = 0$} 
        \State Mark $I_j$ as unsafe \Comment{Contamination found in the current interval}
        \EndIf
        \If {Both endpoints of $S_j$ have been queried}
            \If {$I_j$ marked unsafe}
                \State Set range $Y_j := [0, 8L]$ \label{algoline:highd_marked_rangeset}
            \Else
                \State Set range $Y_j = [\min(S_j) - L\cdot \len(I_j)), \max(S_j) + L\cdot \len(I_j))] \cap [0,1]$ \label{algoline:highd_unmarked_rangeset}
            \EndIf
        \EndIf %\label{algo:queryendpt_end}
    \Else %{$I_t$ not marked unsafe} \label{algo:notmarked_beg}
        \State $Y_j := \algmq(I_j, Y_j)$
        \If { $\len(Y_j) < \max (4 L\cdot \len(I_j), 4L / T )$ }  \Comment{associated range $Y_j$ has shrunk enough}
            \State Bisect each side of $I_j$ to form $2^d$ new hypercubes each with length $\len(I_j) / 2$
            \State For each new hypercube $I_{ji}$ ($1\le i \le 2^d$) set $S_{ji} = Y_j$
        \EndIf %\label{algo:notmarked_end}
    %\Else \Comment{$I_t$ is marked dishonest} \label{algo:dishonest_begin}
        %\State Start from scratch until converge, takes $\log T$ steps \label{algo:test}
        
        %\State Let $\set{[], [], []}$ be the sequence of feasible region (endpoints? ) on the path to $I_t$. 
        %\State Perform binary search on this path and find the first interval [] that contains $f(I_t)$. \Comment{Takes $\log\log T$ steps}, set feasible region $I_t = []$
        %\State From this point perform binary search and shrink feasible region \Comment{Takes $c$ steps, where $c$ is number of corruptions before reaching a correction interval? }
    \EndIf %\label{algo:dishonest_end}
\EndFor
\end{algorithmic}
\end{algorithm}

The corruption-robust algorithm for learning a Lipschitz function from $\cR^d \rightarrow \cR (d > 1)$ under symmetric loss is summarized in~\cref{algo:highDsymm}. The proposed algorithm initializes the input space as a single hypercube with $S_j$ initialized to $[0, 8L]$ (instead of $8^d$ hypercubes with $S_j$ initialized as $[0,L]$). This is purely for ease of exposition: the number of hypercubes at depth $h$ is simplified from $8^d \cdot O(2^{hd})$ to $O(2^{hd})$. \szcomment{TODO here}
%The analysis makes the assumption that $L = 1/8$, so that there is exactly one root interval (line?? in algorithm ??). If $L < 1/8$, the learner can simply scale range of 

\begin{theorem}[\Cref{thm:symmMD} restated]
\Cref{algo:highDsymm} achieves cumulative symmetric loss $L\cdot O(T^{\nicefrac{(d-1)}{d}} + C\cdot \log T + C\cdot 2^d)$ for $d \ge 2$. 
\end{theorem}
The analysis will be similar to that of~\cref{algo:1dabsolute} and~\cref{thm:symm1D}. 

\begin{proof}
There are $O(C)$ corrupted intervals, contributing $L\cdot O(C\log T)$ loss. There are $O(C\cdot 2^d)$ amending intervals, contributing $ L\cdot O( C\cdot 2^d)$ loss. For safe intervals, there are at most $O(2^{hd})$ intervals at depth $h$. Throughout $T$ rounds, a loss with magnitude $O(2^{-h})$ can be charged at most $O(2^{hd})$ times, and the total loss of all safe intervals is then at most $L\cdot O(T^{(d-1) / d})$, since the loss coming from safe intervals can be upper bounded by
\begin{align*}
    L\cdot \sum_{h=0}^{\frac{\log T}{d}} O( 2^{(d-1)h} ) = L\cdot O(T^{\nicefrac{(d-1)}{d}}). 
\end{align*}
Putting the loss of all three types of intervals together completes the proof. 
\end{proof}




\section{Proof Details for Pricing Loss}
\label{app:pricing}
\szcomment{Don't use adaptive zooming / adaptive discretization, use uniform discretization}


\begin{definition}
\label{def:validPair}
Call a hypercube $I_j$ a pricing-ready hypercube if $\len(Y_j) \le 10 L \cdot \eta$. Call the associated range $Y_j$ the pricing-ready range, and together $I_j, Y_j$ form a pricing-ready pair. A pricing-ready pair is valid if $f(I_j) \in Y_j$, otherwise, it is invalid. 
\end{definition}

\begin{definition}
\label{def:queryType}
Let $I_j, Y_j$ be a pricing-ready pair. A query of $\min(Y_j)$ is said to be a pricing query. A query of $\max(Y_j)$ is said to be a sanity check query. A query on a pair that is not pricing-ready (i.e. when $Y_j > 10L\cdot \eta$) is called a search query. 
\end{definition}


\begin{definition}
\label{def:run}
Let $I_j, Y_j$ be a pricing-ready pair. The set of queries that occurred on the pair before the learner becomes surprised or the algorithm terminates is called a run. 
\end{definition}


% The proof outline is as follows. 
% \begin{enumerate}
%     \item For intervals that are not pricing-ready, use safe/amending/corrupted to analyze. 
%     \item For intervals that are pricing-ready, use valid/invalid to analyze. 
% \end{enumerate}

% The following lemma follow directly from the analogue for symmetric loss, with minor changes. 
% \begin{lemma}
%     Consider all intervals with depth less than $\log 1/\eta$. Safe intervals contribute $O(1)$ loss each, total $O(T^{d/(d+1)})$ loss. Corrupted intervals contribute $O(\log T)$ loss each, total $O(C\log T)$ loss. Amending intervals contribute $O(\log T)$ loss each, total $O(2^d \cdot C\cdot \log T)$ loss. 
% \end{lemma}

On a pricing-ready hypercube $I_j$, the learner is said to become \emph{surprised} when the signal is inconsistent with $Y_j$. In other words, the learner becomes surprised when she queried $\min(Y_j)$ and receives an overprice signal ($\sigma_t = 1$), or queried $\max(Y_j)$ and receives an underprice signal ($\sigma_t = 0$). 

\szdelete{
The sanity check schedule ensures the following holds. 
\begin{claim}
Let a pricing interval be run for $t_0$ rounds before the learner becomes surprised or the algorithm terminates, then the number of sanity checks is $\Theta(g^{-1}(t_0))$. Conversely, if there are $s_0$ sanity checks in this run, then the interval has been run for $\Theta(g(s_0))$ rounds. 
\end{claim}
}

\begin{lemma}
\label{lemma:surprised}
    %There are at most $C$ invalid pricing intervals. 
    The learner becomes surprised at most $C$ times. 
\end{lemma}

\begin{proof}
    The learner can become surprised for the following two reasons: the signal itself is corrupted, or the signal is uncorrupted, but the associated range $Y_j$ is actually invalid (as in~\cref{def:validPair}). If the associated range $Y_j$ is invalid, then corruption must have occurred in the search queries before the hypercube became pricing ready. Since there are at most $C$ corruptions, the learner can become surprised at most $C$ times. 
\end{proof}


\begin{lemma}
\label{lemma:pricingLoss}
    Let $I_j, Y_j$ be a pricing-ready pair that has been queried for $\tau$ rounds before the learner becomes surprised or the algorithm terminates. Further, assume a total of $\xi$ signals were corrupted during this period. Then pricing queries in this pair pick up a loss of $L\cdot O(\tau \eta_0 + \xi \tau_0)$. 
\end{lemma}

\begin{proof}
If the pair were valid, i.e. $f(I_j) \in Y_j$, then each pricing round incurs loss $L\cdot O(\eta_0)$, hence a run with length $\tau$ incurs loss $L\cdot O(\tau \eta_0)$. 

For an invalid pricing interval-range pair, there are four cases. 
\begin{enumerate}
    \item $f(I_j) \ge \max(Y_j)$
    \item $f(I_j)$ has some overlap with $Y_j$ and $f(I_j) \ge \min(Y_j)$, $\max(Y_j) \in f(I_j)$
    \item $f(I_j) \le \min(Y_j)$
    \item $f(I_j)$ has some overlap with $Y_j$ and $f(I_j) \le \max(Y_j)$, $\min(Y_j) \in f(I_j)$
\end{enumerate} 
\szcomment{Insert a illustration here. }

Consider the total loss collected from pricing rounds before learner becomes surprised or algorithm terminates. 

% In all cases, the loss collected from sanity check queries is at most $O(f(t_0))$. 

For case 1, any uncorrupted signal in sanity check queries makes the learner surprised. Hence assuming the adversary spent corruption budge $\xi$, then the run must terminate in $O( \xi \tau_0 )$ rounds since the adversary must be corrupting every sanity check query. The learner then trivially incurs loss $L\cdot O(\xi \tau_0)$. 

For case 2, the loss collected is at most $L\cdot O(\eta_0)$ per pricing round, hence the total loss is $L\cdot O(\tau \eta_0)$. 
%At most $f(c_0)$ corruption. 

For case 3, any uncorrupted signal in pricing rounds makes the learner surprised. Hence assuming the adversary spent a corruption budget $\xi$, the run terminates in $\Theta(\xi)$ rounds (since the adversary must be corrupting all pricing rounds), and the learner incurs regret $L\cdot O(\xi)$. 

For case 4, the adversary does not corrupt sanity check rounds, or the learner will become immediately surprised. If the learner does not become surprised during pricing rounds, this could be due to either: 1. the learner did not overprice, or 2. the learner overpriced but the adversary corrupted the signal. Consequently in pricing rounds, uncorrupted queries accumulate $L\cdot O(\tau \eta_0)$ loss, and corrupted queries accumulate $L\cdot O(\xi)$ loss. 
%Can terminate very long or terminate in $O(c_0)$ rounds. 

Putting all cases together completes the proof. 
\end{proof}

In the following, let $\cT$ be a multi-set with elements denoting the length of runs of pricing-ready pairs. Let $\cC$ be a multi-set with elements denoting the number of corruptions that occurred in runs of pricing-ready pairs. 
%The learner starts with $2^{d \floor{ \log(1/\eta)} }  \le \eta^{-d}$ hypercubes, and becomes surprised at most $C$ times. 
\iffalse
\begin{lemma}
\label{lemma:sanityLoss}
The learner incurs loss $L\cdot O( T / \tau_0 )$ in sanity check queries. 
\end{lemma}
\begin{proof}
The learner performs sanity check every $\tau_0$ rounds, hence there can be $O(T /\tau_0)$ sanity check queries. Each query contribute loss at most $L$, hence the lemma follows. 
\end{proof}
\fi

\szdelete{
    The loss from sanity check queries can be bounded as
    \begin{align*}
        L\cdot \sum_{\tau \in \cT} g^{-1}(\tau) \le L\cdot N g^{-1}(T/N)
    \end{align*}
    by concavity of $g^{-1}$ and the fact that $\cT$ has at most $N$ elements. 
}


\begin{theorem}[\Cref{thm:pricing} restated]
\Cref{algo:pricing} incurs cumulative pricing loss $L\cdot \widetilde{O} \left( T/\tau_0 + C\tau_0 + T^{d/(d+1)}\right) $. 
\end{theorem}
\begin{proof}
    First, an upper bound is obtained on the total loss incurred from search queries (i.e. $\len(Y_j) > 10L\cdot \eta_0$). The learner searches from scratch when starting from initialization ($Y_j$ is set to $[0, L]$ at initialization) or when she becomes surprised ($Y_j$ is reset to $[0, L]$). This can happen at most $C + \eta_0^{-d}$ times, since the learner becomes surprised at most $C$ times (\cref{lemma:surprised}) and there are $\eta_0^{-d}$ hypercubes at initialization.  Whenever the learner searches from scratch, it takes $O(\log T)$ queries before the interval becomes pricing-ready. Thus the total loss incurred from search queries is 
    \begin{align}
    L \cdot \widetilde{O}(C + \eta_0^{-d}). 
    \end{align}

    Next, the loss incurred when range becomes pricing-ready can be separated into two parts, loss from sanity check queries and loss from pricing rounds. %that records the number of queries occurred in pricing-ready intervals before the learner becomes surprised or the algorithm terminates. 
    The total loss from sanity check queries can be bounded by
    \begin{align}
    L\cdot O ( T / \tau_0 ), 
    \end{align}
    since there can be $O(T/\tau_0)$ sanity check queries, and each query contribute loss at most $L$. 
    
    From~\cref{lemma:pricingLoss}, the total loss from pricing rounds can be bounded by
    \begin{align}
    L\cdot O\left( \sum_{\xi \in \cC} \xi \cdot \tau_0 + \sum_{\tau\in \cT} \tau \cdot \eta_0 \right) &\le L\cdot O( C \tau_0 +  T \eta_0 ). 
    \end{align}
    % Hence adding the loss from sanity check queries and pricing rounds, the total loss from pricing-ready intervals is bounded by
    % \begin{align*}
    % L\cdot \left( Ng^{-1}(T/N) + g(C) + T\eta \right)
    % \end{align*}

    Putting everything together, the total loss can be bounded by the sum of loss incurred during sanity check rounds, pricing rounds, and searching rounds:
    \[
    L\cdot \widetilde{O}\left( T / \tau_0 + C\tau_0 + T\eta_0 + \eta_0^{-d} \right). 
    \]
    Plugging the choice of parameter $\eta_0 = T^{-1/(d+1)}$ finishes the proof. 
\end{proof}


% \begin{theorem}
% Pricing loss $O(LT^{d/(d+1)} + C\log T)$. 
% \end{theorem}
% \begin{proof}
% Corrupted interval contribute $O(C\log T)$ loss. Amending interval contribute $O(C\log T)$ loss. 

% In safe intervals, the learner queries the lower end of $Y_j$ once the length of $Y_j$ drops below the threshold $\eta = ...$. Such queries contribute loss at most $O(T\eta) = ...$. 

% In safe intervals, the learner incurs regret $O(1)$ for all queries incurring in each interval (as opposed to $O(2^{-h})$ f or the symmetric loss) when the length has not reached the threshold $\eta$. Such intervals have depth $O(\log(1/\eta))$, and there are at most $O((8L)^d 2^{d\log(1/\eta)}) = O((LT)^{d/(d+1)})$. 
% \end{proof}

\iffalse
\subsection{Sanity Check with Pricing Loss}
Treat $L$ as constant. Recall in pricing loss, we no longer shrinks intervals when associated range falls below $\eta = T^{1/(d+1)}$. The total loss coming from these small boxes are $T\eta$. The total number of intervals required to reach this level of refinement is $2^{d\log(1/\eta)} = \eta^{-d}$. 

Consider a hypercube $I_j$ and associated range $Y_j$ with its length no larger than $\eta$. It could be valid or not. We say it is valid if $f(I_j) \in Y_j$. Otherwise it is invalid. 

Use the following sanity check schedule. When $I_j$ has been queried $t$ times, we sanity check $\max(Y_j)$ for $f(t)$ rounds. Suppose $c_j$ rounds are corrupted in this interval. Another way to think about this is to give a count to all small hypercubes below $\eta$. Let $z_1 = 1^{d+1/d}, z=2 = 2^{d+1/d}, ..., z_k = k^{d+1/d}, ...$ be the critical values. When $I_j$ has been queried for $z_i$-th time, query the upper bound on $Y_j$. 

Bound number of sanity checks on safe intervals $...$

If it is valid, then small loss when signal is uncorrupted. 

If it is invalid, there could be two cases. 

Case 1. $f(x) < Y_j$ for some $x \in I_j$. In this case, it is possible to overprice. 

Case 2. $f(x) > Y_j$ for some $x \in I_j$. In this case, it is possible to underprice by more than $\eta$, since the learner is willing to pay $f(x)$ and only $\min(Y_j)$ is charged. 

\subsection{Pricing Loss}
Use $d=1$ first, the case with $d>1$ will be similar. 

\begin{lemma}
The learner can become surprised $C\cdot 2^d$ times. 
\end{lemma}

Whenever the learner becomes surprised, she restarts the search from scratch by setting the associated range to $[0, L]$. Thus the learner starts from scratch at most $C \cdot 2^d$ times. 

Consider runs until the learner becomes surprised or until the whole algorithm terminates, denote the total number of rounds by $t_j$. Note that the learner can become surprised multiple times on the same interval $I_j$. Note $\sum t_j \le T$. 

\szcomment{From here, there are several ways to go about. First way is continue analyzing safe, corrupted, amending intervals. But this may not be fruitful. It would be better to focus on pricing intervals. }

Call a pricing interval valid if $f(I_j)\in Y_j$. It is invalid otherwise. There are essentially four cases when the interval is invalid. 

\szcomment{This has been moved earlier. }

Hence for any invalid interval, the learner collects loss $(f(t_0) + g(c_0))$ loss, here $g$ is the inverse function of $f$. 


Hence all loss for all invalid interval is $\sum_j f(t_j) + \sum_j g(c_j)$. 

Next, consider a valid interval, it is run for $t_j$ rounds before surprised. The total loss is $O(f(t_j))$ from sanity check rounds plus a pricing loss. 

\subsection{Adaptive Zooming}

% \begin{algorithm}[htbp]
% \caption{Learning with Corrupted Binary Feedback for $d > 1$ under Pricing Loss with Adaptive Zooming}
% \label{algo:1dabsolute}
% \begin{algorithmic}[1]
% \State Learner maintains a partition of hypercubes $I_j$ of the input space throughout learning process
% \State For each interval $I_j$ in the partition, learner maintains an associated range $Y_j$, and two endpoints $\lend(I), \rend(I)$
% \State The partition is initialized as $\ceil{8L}$ hypercubes $I_j$ with length less than $1/8L$ each, and each has feasible interval $Y_j$ set to $[0,1]$, with $\lend(I_j) = 0, \rend(I_j) = 1$
% \For {round $t = 1, 2, ..., T$}
%     \State Learner receives context $x_t$
%     \State Learner finds interval $I_j$ such that $x_t \in I_j$
%     \State Let $Y_j$ be the associated range of $I_j$
%     \If {$\len (Y_j) < \eta$}
%         \If {\textit{Reached sanity check schedule} }
%             \State Query $\max(Y_j)$
%         \EndIf
%         \State Query $\min(Y_j)$
%         \If {$\sigma_t = 1$}
%             \State Set $Y_j = [0,L]$ \Comment{Reset the range of $Y_j$}
%         \EndIf
%     \ElsIf {Exists an endpoint in $S_j$ not queried} %\label{algo:queryendpt_beg}
%         \State Query an unqueried endpoint of $S_j$
%         \If {Queried $\min(S_j)$ and $\sigma_t = 1$, or queried $\max(S_j)$ and $\sigma_t = 0$} 
%         \State Mark $I_j$ as unsafe \Comment{Contamination found in the current interval}
%         \EndIf
%         \If {Both endpoints of $S_j$ have been queried}
%             \If {$I_j$ marked unsafe}
%                 \State Set range $Y_j := [0,1]$ \label{algoline:marked_rangeset}
%             \Else
%                 \State Set range $Y_j = [\min(S_j) - L\cdot \len(I_j)), \max(S_j) + L\cdot \len(I_j))] \cap [0,1]$ \label{algoline:unmarked_rangeset}
%             \EndIf
%         \EndIf %\label{algo:queryendpt_end}
%     \Else %{$I_t$ not marked unsafe} \label{algo:notmarked_beg}
%         \State $Y_j := \algmq(I_j, Y_j)$
%         \If { $\len(Y_j) < 4 L\cdot \len(I_j)$}  \Comment{associated range $Y_j$ has shrunk enough}
%             \State Bisect each side of $I_j$ to form $2^d$ new hypercubes, each with length $...$
%         \EndIf %\label{algo:notmarked_end}
%     \EndIf
%     %\Else \Comment{$I_t$ is marked dishonest} \label{algo:dishonest_begin}
%         %\State Start from scratch until converge, takes $\log T$ steps \label{algo:test}
        
%         %\State Let $\set{[], [], []}$ be the sequence of feasible region (endpoints? ) on the path to $I_t$. 
%         %\State Perform binary search on this path and find the first interval [] that contains $f(I_t)$. \Comment{Takes $\log\log T$ steps}, set feasible region $I_t = []$
%         %\State From this point perform binary search and shrink feasible region \Comment{Takes $c$ steps, where $c$ is number of corruptions before reaching a correction interval? }
%     %\EndIf %\label{algo:dishonest_end}
% \EndFor
% \end{algorithmic}
% \end{algorithm}

\fi


%\section{Covering Dimension Based}

\begin{algorithm}[htbp]
\caption{Covering Dimension Based}
\label{algo:covering}
\begin{algorithmic}
\State Learner start with uniform density over $\cF(1/T^2)$, the $1/T^2$ covering of $\cF$
\For{$t = 1, 2, ..., T$}
    \State Learner finds $m_t$ such that $\Pr[f(x_t) \ge m_t] \ge 1/2$, $\Pr[f(x_t) \le m_t] \ge 1/2$
    \State Learner guesses $q_t$ uniformly chosen from $[m_t - 1/T, m_t + 1/T]$
    \State $w(f) = w(f) * (1 + c)$ if $\sigma(q_t - f(x_t)) = \sigma_t$
    \State $w(f) = w(f) * (1 - c)$ if $\sigma(q_t - f(x_t)) = 1-\sigma_t$
    \State Normalize
\EndFor
\end{algorithmic}
\end{algorithm}

%In addition, $f(x_t) = q_t$ happens with $0$ probability for any $f\in \cF(1/T^2)$. 
\begin{lemma}
    With probability $1 - \frac{1}{2T}$, $f^*(x_t)$ and $f_0(x_t)$ are on the same side of $q_t$.
\end{lemma}
\begin{proof}
    Fix any $x_t, m_t$. Note that $\abs{f_0(x_t) - f^*(x_t)} < \frac{1}{T^2}$. With probability $1 - \frac{1}{2T}$, $f_0(x_t)$ and $f^*(x_t)$ will be on the same side of a randomly drawn point in the interval $[m_t - 1/T, m_t + 1/T]$. 
\end{proof}

In accordance with the possibly corrupted signal $\sigma_t$, let $W_t^-$ be the total density of $f(x_t)$ not on the same side of $q_t$, and let $W_t^+$ be the total density of $f(x_t)$ on the same side of $q_t$. 
\begin{align*}
    W_t^- &= w ( \set{ f: \sigma( q_t - f(x_t)) = \sigma_t} ) \\
    W_t^+ &= w ( \set{ f: \sigma( q_t - f(x_t)) = 1 - \sigma_t} ) 
\end{align*}
Further define $\Delta^+ = W_t^+ - 1/2$, then $\Delta^+ \in [-1/2, 1/2]$. Intuitively, the algorithm makes more progress when $\Delta^+$ is closer to 0. 

Assume the signal was uncorrupted at round $t$, then for any $f(x_t)$ on the same side as $q_t$, the density update follows 
\begin{align*}
w(f) &= w(f) \cdot (1 + c) / [(1+c) W^+ + (1-c) W^-] \\
&= w(f) \cdot (1 + c) / (1 + 2c\Delta^+)
\end{align*}
If $f(x_t)$ is on different side as $q_t$,
\begin{align*}
w(f) &= w(f) \cdot (1 - c) / [(1+c) W^+ + (1-c) W^-] \\
&= w(f) \cdot (1 - c) / (1 + 2c\Delta^+). 
\end{align*}

\begin{lemma}
If the signal is uncorrupted and $\abs{f^*(x_t) - m_t} > 2/T$, then weight of $f_0$ grows. 
\end{lemma}
\begin{proof}
Recall $m_t$ is the median, and $q_t$ is the perturbed version. If $f^*$ and $m_t$ are at least $2/T$ apart, then $f^*$ and $f_0$ will be on same side of $q_t$. With probability $1/2$, the learner has $W^- \ge 1/2$, and consequently $w_{t+1}(f_0) \ge w_t(f_0) \cdot (1 + c)$. With the remaining probability $w_{t+1}(f_0) \ge w_t(f_0)$. 

Hence the learner has the following:
\begin{align*}
    \Ex[w_{t+1}(f_0)] \ge \frac{2+c}{2} w_t(f_0)
\end{align*}
\begin{align*}
    \Ex \left[ \frac{1} {w_{t+1}(f_0)} \right] \le \frac{2+c}{2 + 2c} \cdot \frac{1}{w_t(f_0)}
\end{align*}
\end{proof}

\begin{lemma}
    If the signal is corrupted, then
    \begin{align*}
        w_{t+1}(f_0) \ge w_t(f_0)\cdot (1 - c) / (1 + c)
    \end{align*}
    \begin{align*}
        \frac{1}{w_{t+1}(f_0)} \le \frac{1+c}{1-c} \cdot \frac{1}{w_t(f_0)}
    \end{align*}
\end{lemma}
\begin{proof}
Follows from update rule. 
\end{proof}

\begin{lemma}
If the signal is uncorrupted, then
\begin{align*}
    ...
\end{align*}
\end{lemma}
\begin{proof}
    With $1 - \nicefrac{1}{2T}$ probability, $f^*$ and $f_0$ are on same side of $q_t$ and $w_{t+1}(f_0) \ge w_t(f_0)$. Otherwise with the remaining $\nicefrac{1}{2T}$ probability, 
    \begin{align*}
        w_{t+1}(f_0) \ge w_t(f)\cdot \frac{1-c}{1+c}
    \end{align*}
    Hence
    \begin{align*}
        \Ex[w_{t+1}(f_0)] &\ge (1 - \frac{1}{2T}) w_t(f) + \frac{1}{2T} \cdot \frac{1-c}{1+c} \cdot w_t(f) \\
        &\ge w_t(f) [ 1 - \frac{1}{2T}\cdot \frac{2c}{1 + c} ]
    \end{align*}
    \begin{align*}
        \Ex[\frac{1}{w_{t+1}(f_0)}] &\le (1 - \frac{1}{2T}) \cdot \frac{1}{w_t(f)} + \frac{1}{2T} \cdot \frac{1+c}{1-c} \cdot \frac{1}{w_t(f)} \\
        &\le \frac{1}{w_t(f)} \left( 1 + \frac{1}{2T}\cdot \frac{2c}{1 - c} \right)
    \end{align*}
\end{proof}

\begin{theorem}
The main theorem ...
\end{theorem}
\begin{proof}
    Define the variables:
    $s_1 = 1/\abs{N}$. 
    \begin{align*}
        s_t &= \frac{1+c}{1-c} \text{ if } \text{corrupted} \\
        s_t &= \frac{2+c}{2 + 2c} \cdot s_{t-1}\text{ if } \abs{} > 2/T, \text{uncorrupted} \\
        s_t &= 1 + \frac{1}{2T}\cdot\frac{2c}{1-c} \text{if } \abs{} < 2/T, \text{uncorrupted}
    \end{align*}

    Define the stochastic process
    \[
    Y_t = \frac{1}{s_t\cdot w_t(f_0)}
    \]
    Then $\Ex[Y_{t+1}] \le Y_{t}$. 

    Let $C$ be the total number of corrupted rounds, let $C'$ be the total number of uncorrupted rounds with $\abs{...} < 2/T$. Then $s_T$ can be upper bounded: 
    \begin{align*}
        s_T \le \frac{1}{N} \cdot  (\frac{1+c}{1-c})^{C} \cdot (\frac{2+c}{2+2c})^{C'} \cdot (1 + \frac{1}{2T})^T \le O(1/T)
    \end{align*}

    Hence
    \[
    Y_T = \frac{1}{s_T w_T} \ge \frac{1}{s_T} \ge \Omega(T)
    \]
\end{proof}

\szcomment{TODO remove log T factor??}
\subsection{Removing the $\log T$ factor}
Try using a multi-scale argument. 
Show when guess is $\epsilon$-far, then all $\epsilon$-close hypothesis gets multiplied by a constant factor. This is because any $\epsilon$-close hypothesis must be on the same side of the true hypothesis $f^*$. 

Existing results. 
\begin{enumerate}
    \item $O(d^2)$ loss using multiscale discretization. Deterministic signals, no corruption. No density. Select the midpoint from suitable layer. Then keep all hypothesis in layer $i$ that is $z_i$-consistent to make sure true hypothesis never eliminated. 
    \item $O(d\log T)$ loss using single scale discretization. Noisy signals. Use density. Compute the midpoint. Then the weight of $f_0$ multiplied by constant factor whenever guess is $2/T$ far. Also need to make sure when guess is $2/T$ close, the weight of $f_0$ does not decrease too much, to achieve this add perturbation to guess so that they are on same side. 
\end{enumerate}

To recap, the argument for $\log T$ version with density update goes as follows. When guess $2/T$-far, then $f_0$ must be on same side as $f$, and will get multiplied by constant factor. This can happen at most $O(d\log T)$ times. 

The argument for regret constant in $T$ uses a multiscale argument. 

Let $N_z(\cF)$ be any $z$-covering of the hypothesis class $\cF$. Define the $\gamma$-window median to be the point $q$ such that
\[
w( > q + \gamma) > 1/4, w( < q - \gamma) > 1/4
\]
If there is no such $q$, then the density does not have a $\gamma$-window median. 

The learner keeps a density $w_t$ for each layer of discretization $F^{z_i}$, here $z_i = 3^-d$ so that layer $i$ provides a $z_i$-covering. The density $w^i_1$ is initialized as the uniform density. 

Ideally the algorithm should satisfy the following:
\begin{enumerate}
    \item If the guess was $2z_i$-far, then $f_i \ge f_i \cdot \frac{1.5}{1.25}$. 
    \item If uncorrupted, the weight does not decrese. 
    \item If corrupted, weight decrease by at most $1/3$. 
\end{enumerate}

The analysis focuses on the density of $f_1, f_2, ...$, the density of $z_i$-close approximations. 

The algorithm is summarized as follows. 

\begin{algorithm}
\caption{Covering Dimension based}

\begin{algorithmic}[1]

\State Learner maintains density for all layers

\For {$t = 1, 2, ..., T$} 
    \State Find the smallest index $i$ such that there exists a $z_i$-window median, let the median be $q$. 
    \State Learner queries $q$ and receives possibly corrupted signal $\sigma_t$
    \For {$i = 1, 2, \dots$} 
        \State Update the density of all $f$ in layer $i$ as follows: 
        \begin{align*}
            w(f) &= w(f) * 1.5 \text{ if } f(x) > q - z_i \\
            w(f) &= w(f) * 0.5 \text{ otherwise }
        \end{align*}
    \EndFor
\EndFor
\end{algorithmic}
\end{algorithm}

Let $f_i$ be the function in layer $i$ such that 
\[
\norm{f_i - f^*}_\infty < z_i
\]

\begin{lemma}
    Fix a round $t$. Assume the guess was $2z_i$-far. Then $f_{i}$ is multiplied by a constant factor for any $i$. 
\end{lemma}
\begin{proof}
    We show that there exists a point $q$ such that $q$ is the $\gamma$-window median for any $\gamma \le z_i$. 
\end{proof}



%\section{Improvement of loss}

Can we improve the dependency on $C$ to $\log\log T$ or $\log C$? 

If interval not marked dishonest, then $O(1)$ regret if the interval was corrupted. 

When an interval has been marked dishonest, algorithm 1 searches from scratch for the current interval, thus it takes $O(\log T)$ rounds for the interval to split. If the interval were corrupted, this leades to $O(C\log T)$ regret. 




Idea 1: When an interval marked dishonest, do not start from scratch. Instead, follow the path up and search. 

Idea 2: Perform Binary search on the path. 

\iffalse
I give some instantiations of the theorem. 
\begin{example}
Let $g(x) = x^{1/u}$ where $u \in (0,1)$. The loss can be bounded as
\begin{align*}
    L\cdot \Tilde{O}(Ng^{-1}(T/N) + g(C) + T\eta ) &\le L\cdot \Tilde{O} (T^u N^{1-u} + C^{1/u} + T\eta) \\
    &\le L \cdot \Tilde{O}\left( T^u (\eta^{-d})^{1-u} + T^u C^{1-u} + C^{1/u} + T\eta \right)
\end{align*}
Choose $\eta = T^{-\frac{1-u}{d(1-u) + 1}}$. Then the total pricing loss is
\[
L \cdot \Tilde{O} (T^{\frac{(1-u)d + u}{(1-u)d + 1}} + T^u C^{1-u} + C^{1/u}). 
\]
\begin{itemize}
\item If $C$ has the order $ C = \Theta(T^{\varepsilon})$ for some $\varepsilon = (0,1)$, the learner can then subsequently choose $u$ to balance the terms. 
\item When the corruption level $C$ is unknown, this work recommend setting $u = \frac{d}{d+1}$, this achieves a regret bound of
\[
L\cdot \tilde{O} \left(   \right)
\]
\szcomment{TODO}
\end{itemize}
\end{example}

\begin{example}
If $C$ can treated as a constant, then the learner can choose $g(x) = 2^x - 1$ and set $\eta = T^{-\frac{1}{d+1}}$. Then the total pricing loss is
\[
L \cdot \Tilde{O} (T^{\frac{d}{d+1}} + \exp(C)). 
\]
\end{example}
\fi

\end{appendices}



\end{document}


