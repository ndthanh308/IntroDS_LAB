\subsection{Lower Bound}
\szcomment{Comment: Different approach to state the upper bound. }

It is natural to wonder whether the dependence on $C$ in the above theorem is tight. That is, whether one can achieve a bound of the form:
\[
\widetilde{O}(T^{d/(d+1)} + C)
\]

The below lower bound shows this is not possible. 

\begin{theorem}
Assume $L = 1$. Let $A$ be any algorithm to which the corruption budget $C$ is unknown. Suppose $A$ achieves a cumulative pricing loss $R(T) = o(T)$ when $C = 0$. Then, there exists some $C = 4R(T)$, such that the algorithm suffers $\Omega(T)$ regret. 
\end{theorem}

\begin{proof}
Let us consider two environments. In the first environment, the adversary chooses $f(x)=0.5$. In the second environment, the adversary chooses $f(x)=1$. 
	
The adversary adopts the following strategy. In the first environment where $f(x) = 0.5$, the adversary never corrupts the signal. In the second environment where $f(x) = 1$, the adversary corrupts the signal whenever the seller queries a point above $0.5$. Hence, the adversary manipulates the seller into thinking the true price is 0.5. 

Now we know the algorithm achieves regret $R(T)$ when $C=0$. Then, when $f(x) = 0.5$, the seller can only query values above $0.5$ for at most $2R(T)$ times. However, by choosing $C = 2R(T)$, the adversary can make the two environments indistinguishable to the seller. Hence the seller necessarily incurs $\Omega(T)$ regret. 
\end{proof}

The next theorem shows a lower bound when the corruption budget $C$ is known. We will prove our lower bound against a randomized adversary. Define an instance as the tuple $(f, C, \cS)$, where $f$ is the Lipschitz function, $C$ is the corruption budget, and $\cS$ is the corruption strategy of the adversary. The randomized adversary draws a problem instance from some probability distribution $\cD$, and the corruption budget is defined as the expected value of $C$ under distribution $\cD$. 
\begin{theorem}
Let $0 < C < T$ be the corruption budget and is known to the learner. Then for any algorithm, there exists some corruption strategy such that the learner incurs $\Omega(\sqrt{CT})$ regret. 
\end{theorem}
\begin{proof}
Consider the following corruption strategy. The space is discretized into $\sqrt{T/C}$ hypercubes of equal length. The context that the adversary selects will be the center of each hypercube, and the value at each context will be 0.5 except for one context which has value $0.5 + \eps$. Each hypercube will be selected by the adversary for $\sqrt{CT}$ rounds. The adversary will corrupt the first $C$ queries in the hypercube which has value $0.5+\eps$. At a high level, the learner can only detect the hypercube with value $0.5 + \eps$ if he performs checking rounds in every hypercube for at least $C$ rounds. Alternatively, if the learner does not perform checking query, he will incur regret $\eps \sqrt{CT}$. 

Can we choose there to be $1/\eps = \sqrt{T/C}^{1/d}$ hypercubes that has value $0.5+\eps$? 
\end{proof}

Difficulty should be the learner does not know how many times a interval will be queried. 