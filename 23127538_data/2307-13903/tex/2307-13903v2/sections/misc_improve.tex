\section{Improvement of loss}

Can we improve the dependency on $C$ to $\log\log T$ or $\log C$? 

If interval not marked dishonest, then $O(1)$ regret if the interval was corrupted. 

When an interval has been marked dishonest, algorithm 1 searches from scratch for the current interval, thus it takes $O(\log T)$ rounds for the interval to split. If the interval were corrupted, this leades to $O(C\log T)$ regret. 




Idea 1: When an interval marked dishonest, do not start from scratch. Instead, follow the path up and search. 

Idea 2: Perform Binary search on the path. 

\iffalse
I give some instantiations of the theorem. 
\begin{example}
Let $g(x) = x^{1/u}$ where $u \in (0,1)$. The loss can be bounded as
\begin{align*}
    L\cdot \Tilde{O}(Ng^{-1}(T/N) + g(C) + T\eta ) &\le L\cdot \Tilde{O} (T^u N^{1-u} + C^{1/u} + T\eta) \\
    &\le L \cdot \Tilde{O}\left( T^u (\eta^{-d})^{1-u} + T^u C^{1-u} + C^{1/u} + T\eta \right)
\end{align*}
Choose $\eta = T^{-\frac{1-u}{d(1-u) + 1}}$. Then the total pricing loss is
\[
L \cdot \Tilde{O} (T^{\frac{(1-u)d + u}{(1-u)d + 1}} + T^u C^{1-u} + C^{1/u}). 
\]
\begin{itemize}
\item If $C$ has the order $ C = \Theta(T^{\varepsilon})$ for some $\varepsilon = (0,1)$, the learner can then subsequently choose $u$ to balance the terms. 
\item When the corruption level $C$ is unknown, this work recommend setting $u = \frac{d}{d+1}$, this achieves a regret bound of
\[
L\cdot \tilde{O} \left(   \right)
\]
\szcomment{TODO}
\end{itemize}
\end{example}

\begin{example}
If $C$ can treated as a constant, then the learner can choose $g(x) = 2^x - 1$ and set $\eta = T^{-\frac{1}{d+1}}$. Then the total pricing loss is
\[
L \cdot \Tilde{O} (T^{\frac{d}{d+1}} + \exp(C)). 
\]
\end{example}
\fi