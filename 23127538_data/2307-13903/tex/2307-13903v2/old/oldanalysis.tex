\subsection{Analysis for Weight Update}
\section{Algorithm for the Absolute Loss}
\begin{algorithm}
\begin{algorithmic}
\caption{Learning with Weight Update}
\State Maintain a partition of intervals of the input space throughout learning process
\State Partition initialized as $\ceil{8L}$ intervals with length less than $1/8L$ each
\For {round $t = 1, 2, \dots, T$}
    \State Receive context $x_t$
    \State Find interval $I_t$ such that $x_t \in I_t$
    \If {The concentration endpoint of $I_t$ not queries}
        \State Query concentration endpoint of $I_t$ 
        \State If anomaly detected, mark $I_t$ as untruthful. 
        \State Else, mark the concentration region of $I_t$. 
    \EndIf
    \If {$I_t$ queried less than $c_0$ times}
        \State Find midpoint $q$ in concentration region, specifically find point $q$ such that $f([q + L\cdot\len(I_t), >]; I_t) = f([<, q - L\cdot\len(I_t)]; I_t)$
        % \State Find midpoint $q$ in $[0,1]$
    \Else \Comment{This interval is dishonest}
        \State Find midpoint $q$ in $[0,1]$
    \EndIf
    \State Submit guess $q$, receive binary feedback $\sigma_t$
    \If {$\sigma_t = 1$} \Comment{If guess is high}
        \State update $f[>q + \len] = 0.5 \cdot f[>q + \len] $
        \State update $f[<q + \len] = 1.5 \cdot f[<q + \len] $
        \State Normalize
    \Else \Comment{If guess is low}
        \State update $f[>q - \len] = 1.5 \cdot f[>q - \len] $
        \State update $f[<q - \len] = 0.5 \cdot f[<q - \len] $
        \State Normalize
    \EndIf

    \If{$f(I_t) < 0.1$}
    \State Split and reset weight
    \EndIf
\EndFor
\State \dots
\end{algorithmic}
\end{algorithm}

Some definitions are first provided that will aid the analysis. 

\begin{definition}[Corrupted Interval]
    An interval is called a corrupted interval if any round inside the interval is corrupted. 
\end{definition}

\begin{definition}[Correcting Interval]
    An interval is called a correcting interval if any round inside the interval is uncorrupted and the parent interval is corrupted. 
\end{definition}

\begin{definition}[Truthful Interval]
    An interval is called a truthful interval if any round inside the interval is uncorrupted and the parent interval is either a correction interval or a truthful interval. 
\end{definition}

\begin{definition}[Consistent Region]
    The consistent region of an interval $I$ is defined as $[M - L\cdot \len, m + L\cdot \len]$, where $m = \dots, M = \dots$
\end{definition}

\begin{definition}[Concentrated region]
    Consider an interval $I$. If there exists an interval $[a, b]$ such that $f([a,b]; I) > u$ we say the interval $I$ is concentrated with level $u$ at $[a,b]$. 
\end{definition}

\begin{claim}
For an interval $I$, it is concentrated with level $0.8$ at some interval with length $2\cdot \len$ right before splitting. Equivalently, for any interval $I'$, it is concentrated with level $0.8$ at some interval with length $4\cdot \len$ at the beginning. 
\end{claim}

\begin{lemma}
A correction interval at depth $h$ takes at most $O(c_0 + h)$ rounds before splitting. Further, the correction interval will be concentrated around the consistent region right before splitting. 
\end{lemma}
\begin{proof}
    Let $I$ be a correction interval. Consider the consistent region of interval $I$. 
    At the start the consistent region will have volume $2^{-h}$. 
    Suppose for sake of contradiction $I$ is not splitted after $c_0$ queries. Then for every query in $I$, we must have $p > 0.1$, and hence the volume of the consistent region gets multiplied by a constant factor. After $O(h)$ rounds we will have the volume of the consistent region greater than 1, contradiction. 

    \szcomment{TODO concentrated around consistent region? }
\end{proof}

\begin{lemma}
    At the beginning of a truthful interval, the interval is focused on the consistent region. 
\end{lemma}
\begin{proof}
    If the parent interval is truthful, then yes. If the parent interval is correction interval? \szcomment{TODO}\dots
\end{proof}

\begin{lemma}
    A truthful interval at depth $h$ takes at most $O(c_0)$ queries before splitting, and each query accumulates loss at most $O(2^{-h})$. 
\end{lemma}
\begin{proof}
Consider the consistent region. Suppose the interval has received $c_0$ queries. Then for each query, the consistent region is getting its volume multiplied by some constant. After $c_0$ rounds, its volume will be greater than 1. Hence the interval splits in at most $c_0$ queries. \szcomment{TODO Each query accumulates loss $2^{-h}$. }
\end{proof}

\begin{lemma}
    Consider a corrupted interval and let there be $c_1$ rounds corrupted, then the corrupted interval splits in at most $O(c_1)$ queries. 
\end{lemma}
\begin{proof}
    
\end{proof}

\begin{theorem}
Total loss bounded by $O(C\log T)$. 
\end{theorem}
