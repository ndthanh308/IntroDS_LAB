\section{Conclusion and Future Work}
In this work, I design corruption-robust algorithms for the Lipschitz contextual search problem. I present the \emph{agnostic checking} technique and demonstrate its effectiveness in designing corruption-robust algorithms. An open problem is closing the gap between upper bounds and lower bounds, in particular for the absolute loss when $d = 1$. Specifically, can one actually achieve $O(C + \log T)$ regret in this setting? Another interesting future direction is to relax the Lipschitzness assumption. For example, this work assumes the learner knows the Lipschitz constant $L$. Can the learner design efficient no-regret algorithms without knowledge of $L$? 

\acks{The author would like to thank Jialu Li for her help in preparation of this manuscript. The anonymous reviewers at ALT gave helpful suggestions on improving the presentation of this paper. }