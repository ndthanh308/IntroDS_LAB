I study the problem of learning a Lipschitz function with corrupted binary signals. The learner tries to learn a $L$-Lipschitz function $f: [0,1]^d \rightarrow [0, L]$ that the adversary chooses. There is a total of $T$ rounds. In each round $t$, the adversary selects a context vector $x_t$ in the input space, and the learner makes a guess to the true function value $f(x_t)$ and receives a binary signal indicating whether the guess is high or low. In a total of $C$ rounds, the signal may be corrupted, though the value of $C$ is \emph{unknown} to the learner. The learner's goal is to incur a small cumulative loss. This work introduces the new algorithmic technique \emph{agnostic checking} as well as new analysis techniques. I design algorithms which: for the absolute loss, the learner achieves regret $L\cdot O(C\log T)$ when $d = 1$ and $L\cdot O_d(C\log T + T^{(d-1)/d})$ when $d > 1$; for the pricing loss, the learner achieves regret $L\cdot \widetilde{O} (T^{d/(d+1)} + C\cdot T^{1/(d+1)})$. 

\szdelete{for the pricing loss with a learner's choice of parameter $u \in (0, 1)$, the learner achieves $\widetilde{O}(T^{\frac{(1-u)d + u}{(1-u)d + 1} } + T^u C^{1-u} + C^{1/u} )$ loss. }