
\documentclass[sigconf]{acmart}
%%
%% \BibTeX command to typeset BibTeX logo in the docs
\AtBeginDocument{%
  \providecommand\BibTeX{{%
    Bib\TeX}}}


\copyrightyear{2023}
\acmYear{2023}
\setcopyright{rightsretained}
\acmConference[RecSys '23]{Seventeenth ACM Conference on Recommender
Systems}{September 18--22, 2023}{Singapore, Singapore}
\acmBooktitle{Seventeenth ACM Conference on Recommender Systems (RecSys
'23), September 18--22, 2023, Singapore,
Singapore}\acmDOI{10.1145/3604915.3608792}
\acmISBN{979-8-4007-0241-9/23/09}



\input{math-and-symbols.tex}

\usepackage[utf8]{inputenc} % allow utf-8 input
\usepackage[T1]{fontenc}    % use 8-bit T1 fonts
\usepackage{amsfonts}       % blackboard math symbols



\usepackage{multirow}
\usepackage{bm}
\usepackage{algorithm,algorithmic}
\usepackage{subcaption}
\usepackage{wrapfig}

\pagenumbering{gobble}



\begin{document}


\title{Online Matching: A Real-time Bandit System for Large-scale Recommendations}



\author{Xinyang Yi}
\email{xinyang@google.com}
\affiliation{
  \institution{Google Deepmind}
  \city{Mountain View}
  \state{California}
  \country{USA}
}

\author{Shao-Chuan Wang}
\email{scwang@google.com}
\affiliation{
  \institution{Google Inc}
  \city{Mountain View}
  \state{California}
  \country{USA}
}

\author{Ruining He}
\email{ruininghe@google.com}
\affiliation{
  \institution{Google Deepmind}
  \city{Mountain View}
  \state{California}
  \country{USA}
}

\author{Hariharan Chandrasekaran}
\email{hariharan@google.com}
\affiliation{
  \institution{Google Inc}
  \city{Mountain View}
  \state{California}
  \country{USA}
}

\author{Charles Wu}
\email{charleswu@google.com}
\affiliation{
  \institution{Google Inc}
  \city{Mountain View}
  \state{California}
  \country{USA}
}

\author{Lukasz Heldt}
\email{heldt@google.com}
\affiliation{
  \institution{Google Inc}
  \city{Mountain View}
  \state{California}
  \country{USA}
}

\author{Lichan Hong}
\email{lichan@google.com}
\affiliation{
  \institution{Google Deepmind}
  \city{Mountain View}
  \state{California}
  \country{USA}
}

\author{Minmin Chen}
\email{minminc@google.com}
\affiliation{
  \institution{Google Deepmind}
  \city{Mountain View}
  \state{California}
  \country{USA}
}

\author{Ed H. Chi}
\email{edchi@google.com}
\affiliation{
  \institution{Google Deepmind}
  \city{Mountain View}
  \state{California}
  \country{USA}
}



%%
%% By default, the full list of authors will be used in the page
%% headers. Often, this list is too long, and will overlap
%% other information printed in the page headers. This command allows
%% the author to define a more concise list
%% of authors' names for this purpose.
\renewcommand{\shortauthors}{Yi et al.}
%%
%% Article type: Research, Review, Discussion, Invited or position
% \acmArticleType{Review}
%%
%% Links to code and data
% \acmCodeLink{https://github.com/borisveytsman/acmart}
% \acmDataLink{htps://zenodo.org/link}
%%
%%
%% Sometimes the addresses are too long to fit on the page.  In this
%% case uncomment the lines below and fill them accodingly.
%%
%% \authorsaddresses{Corresponding author: Ben Trovato,
%% \href{mailto:trovato@corporation.com}{trovato@corporation.com};
%% Institute for Clarity in Documentation, P.O. Box 1212, Dublin,
%% Ohio, USA, 43017-6221}
%%
%%
%% Keywords. The author(s) should pick words that accurately describe
%% the work being presented. Separate the keywords with commas.
\keywords{recommender systems, bandits algorithms, neural networks, information retrieval, real-time recommenders}

\begin{abstract}
The last decade has witnessed many successes of deep learning-based models for industry-scale recommender systems. These models are typically trained offline in a batch manner. While being effective in capturing users' past interactions with recommendation platforms, batch learning suffers from long model-update latency and is vulnerable to system biases, making it hard to adapt to distribution shift and explore new items or user interests. Although online learning-based approaches (e.g., multi-armed bandits) have demonstrated promising theoretical results in tackling these challenges, their practical real-time implementation in large-scale recommender systems remains limited. First, the scalability of online approaches in servicing a massive online traffic while ensuring timely updates of bandit parameters poses a significant challenge. Additionally, exploring uncertainty in recommender systems can easily result in unfavorable user experience, highlighting the need for devising intricate strategies that effectively balance the trade-off between exploitation and exploration. In this paper, we introduce \textsl{Online Matching}: a scalable closed-loop bandit system learning from users' direct feedback on items in real time. We present a hybrid \textsl{offline + online} approach for constructing this system, accompanied by a comprehensive exposition of the end-to-end system architecture. We propose Diag-LinUCB -- a novel extension of the LinUCB algorithm -- to enable distributed updates of bandits parameter in a scalable and timely manner. We conduct live experiments in YouTube and show that Online Matching is able to enhance the capabilities of fresh content discovery and item exploration in the present platform.
\end{abstract}

\maketitle

\section{Introduction}
Current quantum hardware is unable to carry out universal quantum computations due to the buildup of errors that occur during the computation. 
The magnitude of the individual error is currently above the value that the Threshold Theorem requires in order to kick-start quantum error correction and fault-tolerant quantum computation~\cite[Section 10.6]{nielsen_chuang_2010}. 
Although the experimentally achieved fidelity rates are promising and the error bounds are inching closer to the required threshold, we will have to work for the foreseeable future with quantum hardware with errors that build-up during the computation.  This implies that we can only do a limited number of steps before the output of the computation has become completely uncorrelated with the intended one.

For fault-tolerant quantum computing, we repeat four steps: 
1) We apply a number of single and two-qubit quantum gates, in parallel whenever possible; 
2) We perform a syndrome measurement on a subset of the qubits; 
3) We perform fast classical computations to determine which errors have occurred and how to correct them; 
and, 4) We apply correction terms based on the classical computations.
We then repeat these four steps with a next sequence of gates. 
These four steps are essential to fault-tolerant quantum computing. 


The starting point of this work is to use the four steps outlined above, not to carry out error correction and fault-tolerant computation, but to enhance short, constant-depth, {\em uncorrected} quantum circuits that perform single qubit gates and {\em nearest-neighbor} two qubit gates. 
Since in the long run we will have to implement error-correction and fault-tolerant computation anyhow, and this is done by such a four-step process, why not make other use of this architecture? Moreover, on some of the quantum hardware platforms, these operations are already in place.
Embracing this idea we naturally arrive at the question: what is the computational power of \textit{low-depth} quantum-classical circuits organized as in the four steps outlined above? 
We thus investigate circuits that execute a small, ideally constant, number of stages, where at each stage we may apply, in parallel, single qubit gates and {\em nearest-neighbor} two qubit gates, followed by measurements, followed by low-depth classical computations of which the outcome can control quantum gates in later stages. 
It is not clear, at first, whether such circuits, especially with constant depth, can do anything remotely useful. 
But we will see that this is indeed the case: many quantum computations can be done by such circuits in constant depth. 
By parallelizing quantum computations in this way, we improve the overall computational capabilities of these circuits, as we do not incur errors on qubits that are idle, simply because qubits are not idle for a very long time. 
Furthermore, reducing the depth of quantum circuits, at the cost of increasing width, allows the circuit to be run faster even if errors occur.

The first usage of such a four-step layout, not to do error correction, but to perform computations, can be found in the paradigm of measurement-based quantum computing~\cite{gottesman1999demonstrating,raussendorf2001one,jozsa2006introduction,clark2007generalised}: 
A universal form of quantum computing where a quantum state is prepared and operations are performed by measuring qubits in different bases, depending on previous measurements and intermediate measurements.

\citeauthor{PhamSvore2013} were the first to formalize the four-step protocol for performing computations~\cite{PhamSvore2013}. They included specific hardware topologies by considering two-dimensional graphs for imposing constraints on qubit interactions. In their model, they develop circuits for particularly useful multi-qubit gates, including specifying costs in the width, number of qubits, depth, number of concurrent time steps, size, and total number of non-Identity operations.
As a result, they find an algorithm that factors integers in polylogarithmic depth.
\citeauthor{Browne:2011} showed that the main tool in the work by \citeauthor{PhamSvore2013}, the fan-out gate, can also be replaced by additional log-depth classical computations in the measurement-based quantum computing setting~\cite{Browne:2011}.

More recently, \citeauthor{Cirac:2021} introduced a scheme to implement unitary operations involving quantum circuits combined with Local Operations and Classical Communication ($\mathsf{LOCC}$) channels: $\mathsf{LOCC}$-assisted quantum circuits~\cite{Cirac:2021}. Similarly to the four-step scheme we just described, they allow for a short depth circuit to be run on the qubits, followed by one round of $\mathsf{LOCC}$, in which ancilla qubits are measured and local unitaries are applied based on the measurement outcomes. They show that in this model any 1D transitionally invariant matrix-product state (MPS) with fixed bond dimension is in the same phase of matter as the trivial state. Similar ideas can be found in~\cite{TVV_NonAbelianTopologicalOrder_2022, tantivasadakarn2021long}.

In this work, we introduce a new model, called \textit{Local Alternating Quantum-Classical Computations} ($\LAQCC$). In this model we alternate between running quantum circuits (constrained by locality), ending in the measurement of a subset of qubits, and fast classical computations based on the measurement results. The outcome of the classical computations are then used to control future quantum circuits. We allow for flexibility in this model, by giving different constraints to the power of both the quantum circuits and the classical circuits as well as the number of alternations between them. 
Most attention will be given to $\LAQCC$ containing quantum circuits of constant depth, classical circuits of logarithmic depth and at most a constant number of alternations between them. 
Any circuit constructed in this model is considered to be of constant depth. 
We restrict ourselves to logarithmic depth classical computations, as this is the first natural and non-trivial extension beyond constant-depth classical computations. 
Constant-depth classical computations do however also have an equivalent constant-depth quantum implementation.

The definition of $\LAQCC$ sharpens the original definition of \citeauthor{PhamSvore2013} by adding constraints to the intermediate classical computations. This allows us to bound the power of $\LAQCC$ from above. 

The main result of \citeauthor{Cirac:2021}, that 1D translational invariant MPS with fixed bond dimension can be prepared by $\mathsf{LOCC}$-assisted circuits, relies on local symmetries of the MPS. These symmetries allow them to prepare local states (on a constant number of qubits) and glue them together by doing one round of the appropriate entangling measurement and corrections, after which they run a round of local unitaries to get the desired result. This general scheme for preparing states that exhibit an MPS description with the appropriate local symmetries requires only geometrically local unitaries and one round of measurement and corrections an therefore is accessible in $\LAQCC$. Studying different local symmetries, known as Symmetry Protected Topological (SPT) phases of matter, to find measurement-based constant depth circuits for states is a broad ongoing field of research~\cite{TVV_NonAbelianTopologicalOrder_2022, tantivasadakarn2021long, smith2023deterministic}. 
All these schemes have a $\LAQCC$ implementation.

%$\LAQCC$-circuits also exist for general schemes of preparing local states, based on the local tensors, and gluing them together using one round of entangled measurement and corrections, based on the local symmetry. 
%The main result of \citeauthor{Cirac:2021}, that 1D translational invariant MPS with fixed bond dimension can be prepared by $\mathsf{LOCC}$-assisted circuits, relies heavily on local symmetries of the MPS and as a result also has an equivalent $\LAQCC$ implementation. 
%The corrections applied after the measurement round are local unitaries depending on the local symmetries of the MPS. 

 

%This general scheme of preparing local states, based on the local tensors, and gluing it together by doing one round of entangled measurement and corrections, based on the local symmetry, is accessible in $\LAQCC$.
Note however that \citeauthor{Cirac:2021} also suggest a circuit for the $W$-state.
This circuit uses sequentially and dependent measurement-based corrections of the ancilla qubits. 
These dependent measurements translate to sequential alternations between the quantum and classical circuits and therefore increase the total depth to linear depth, exceeding the constant-depth constraints imposed by $\LAQCC$-circuits. 

We study the power of the $\LAQCC$ model with respect to state preparation, showing that even with only constant quantum-depth and logarithmic classical depth it remains possible to prepare states with long-range entanglement.
Another surprising result is that it is unlikely that $\LAQCC$ circuits are classically simulatable. We show that any instantaneous quantum polynomial-time (IQP) circuit~\cite{Bremner2010,Shepherd2009} has an $\LAQCC$ implementation.
Classical simulation of IQP circuits implies the collapse of the polynomial hierarchy to the third level, which is not believed to be true~\cite{Bremner2017}. Therefore, we expect that $\LAQCC$ circuits are unlikely to be classically simulatable. We bound the power of $\LAQCC$ by showing that it is contained in $\QNC^1$, the class of polynomial-size, log-depth circuits.

Next, we also study the power that intermediate classical calculations can add to quantum computations, by considering a new model that alternates between polynomially many polynomial-depth quantum circuits and unbounded classical computations
We study this model by doing a complexity theoretical analysis, where we draw inspiration from the notions of complexity given by \citeauthor{RosenthalYuen:2022}, \citeauthor{MetgerYuen:2023}, and \citeauthor{Aaronson:2004}.
All three complexity notions are based on the notion of state preparation, instead of more traditional definition of complexity such as the decidability of a computational problem. 
The first two consider classes based on sequences of quantum states preparable by a polynomial-sized quantum circuit, where the circuits are uniformly generated by a computational class, for instance, the class $\mathsf{PSPACE}$, which results in the complexity class $\mathsf{StatePSPACE}$~\cite{RosenthalYuen:2022,MetgerYuen:2023}.
The third notion considers a relative complexity, where the complexity is measured between two given states, and is measured by the number of gates, from a given gate-set, required to transform one state in another state~\cite{Aaronson:2004}. 
For our definition of state preparation complexity, we drop the uniformity constraint from~\cite{RosenthalYuen:2022,MetgerYuen:2023} and define a class as $\mathsf{StateX}$, which refers to states preparable by circuits of type $\mathsf{X}$. 
As an example, if $\mathsf{X} = \QNC^0$, this results in the class $\mathsf{StateQNC^0}$, which is the set of states preparable from the $\ket{0}^n$ state by poly-size constant-depth circuits. 
This notion is similar to the relative complexity from~\cite{Aaronson:2004}, where one state is the  $\ket{0}^n$ state and instead of counting the number of gates we consider the set of states preparable by a fixed number of gates. Using this notion of complexity we show that any state preparable by an $\LAQCC^*$ circuit is also preparable by a $\mathsf{PostQPoly}$ circuit, the class of circuits of polynomial depth with an additional post-selection gate. 

All Clifford circuits have a constant-depth $\LAQCC$ implementation, implying that any stabilizer state can be implemented by a constant-depth $\LAQCC$ circuit, see Section~\ref{sec:clifford_circuits} for a proof of this statement. 
Efficient circuits for stabilizer states have been known already through measurement-based quantum computing. Therefore this paper focuses on the preparation of non-stabilizer states, and as a surprising result we find novel constant-depth protocols for four very natural classes of non-stabilizer states.
Despite the extensive research into these four classes of non-stabilizer states and the many applications of them, no efficient constant- or low-depth state preparation protocols are known yet. We specifically consider these four classes as they are all often used as initial states in other algorithms.

The first state is a uniform superposition over an arbitrary number of states. 
This state finds applications in many quantum algorithms, as they often start with a uniform superposition over multiple states. 
This superposition is often achieved by applying Hadamard gates to every qubit due to its simplicity to prepare. 
Yet, the analysis of many algorithms, such as Shor's algorithm~\cite{Shor:1997}, would benefit from a different initial superposition. 
The circuit to prepare the uniform superposition over an arbitrary number of states uses an exact version of Grover search as a subroutine, that turns a probabilistic circuit, with a known constant probability of success, into a deterministic circuit. 
We use the circuit for preparing a uniform superposition over an arbitrary number of states as a subroutine in the next two quantum state preparation protocols. 

The second state is the $W$-state, the uniform superposition over all computational basis states of Hamming-weight~$1$, a natural long-ranged entangled state that displays a fundamentally nonequivalent type of entanglement from the Greenberger–Horne–Zeilinger state~\cite{WState:2000}, for which $\LAQCC$-type constant-depth circuits were previously known~\cite{PhamSvore2013, Cirac:2021}. 
The $W$-state is often used as benchmark for new quantum hardware~\cite{Haffner2005,Neeley2010,GarciaPerez:2021}. 
A novel way to prepare the $W$-state therefore gives a new way to benchmark different quantum devices with each other. 
A circuit for preparing the $W$-state was given in~\cite{Cirac:2021}, but this implementation requires sequentially alternating measurements followed by local unitaries, which in the $\LAQCC$ model is not considered to be of constant depth. 
We improve this protocol by giving an $\LAQCC$ implementation of the $W$-state, based on a compress-uncompress method that links the one-hot and binary encoding of integers.

The third state considered is the Dicke state, a generalization of the $W$-state, a superposition over all computational basis states with Hamming-weight $k$~\cite{Dicke:1954}. 
Dicke states have relevance in various practical settings.
For instance, for quantum game theory~\cite{zdemir2007}, quantum storage~\cite{Bacon_Compress:2006,Plesch:2010}, quantum error correction~\cite{ouyang2014permutation}, quantum metrology~\cite{toth2012multipartite}, and quantum networking~\cite{prevedel2009experimental}. 
Dicke states have been used as a starting state for variational optimization algorithms, most notably Quantum Alternating Operator Ansatz (QAOA)~\cite{Hadfield2019}, to find solutions to problems such as Maximum k-vertex Cover~\cite{Brandhofer2022,cook2020quantum}.
The ground states of physical Hamiltonians describing one-dimensional chains tend to show a resemblance to Dicke states such as states resulting from the Bethe ansatz, making them an ideal starting state when investigating the ground state behavior of these Hamiltonians~\cite{TDL_BetheAnsatzDerivation:2010,B_ExcitedStateQuantumPhaseTransitions:2013,DickeTransitions:2021}. 
For instance, the algorithm by \citeauthor{van2021preparing}, who give an algorithm to prepare the Bethe ansatz eigenstates of the spin-1/2 XXZ spin chain, starts by first preparing a Dicke state~\cite{van2021preparing}. 
A Dicke-state preparation protocol based on the compress-uncompress methodology used in the $W$-state furthermore finds applications in entanglement distillation, where the entanglement of a large state is concentrated on only a few qubits. 
Efficient deterministic circuits for preparing Dicke states have been proposed by \citeauthor{bartschi2019deterministic}~\cite{bartschi2019deterministic, bartschi2022deterministic_short_depth}. 
They provide a quantum circuit of depth $\mathO(k \log(\frac{n}{k}))$, allowing arbitrary connectivity, to prepare a Dicke state, which they conjecture to be optimal when $k$ is constant. 
In this work, we provide a constant-depth $\LAQCC$ circuit below their conjectured bound already for constant $k$. 
However, this does not directly disprove their conjecture, as we allow for intermediate measurements and classical computations. 
More significantly, we even construct constant-depth $\LAQCC$ circuits for $k = \mathO(\sqrt{n})$ greatly improving their bound.
This construction extends the compress-uncompress method for the $W$-state combined with additional subroutines. 

We continue with a log-depth state preparation protocol for the Dicke-state for arbitrary $k$. 
This protocol implements an efficient transformation between the factoradic number representation and the combinatorial number representation of a positive integer. 
The combinatorial number representation relates directly to the Dicke state. 
The provided efficient transformation between number representation systems might be of independent interest. 

We conclude by modifying our protocol for preparing a Dicke-state to a protocol that prepares quantum many-body scar states in constant-depth. 
These states have low entanglement and longer coherence times than states with similar energy density.
These characteristics make many-body scar states interesting to analyze and relevant within physics.
Many-body scar states appear for instance in the AKLT model~\cite{AKLT:1987,MRBAR:2018,MRB:2018} and different spin models~\cite{SI:2019,MOBFR:2020}.
Known methods for preparing these states have polynomial-depth~\cite{Gustafson:2023}, whereas our circuit has constant depth. 

% We conclude by studying the power that intermediate classical calculations can add to quantum computations. 
% In this study, we define a new model that relaxes constant-depth quantum circuits to polynomial depth quantum circuits, log-depth classical calculations to unbounded classical computations and a constant number of alternations to a polynomial number of alternations. 
% We call this model $\LAQCC^*$. 
% We study this model by doing a complexity theoretical analysis, where we draw inspiration from the notions of complexity given by \citeauthor{RosenthalYuen:2022}, \citeauthor{MetgerYuen:2023}, and \citeauthor{Aaronson:2004}.
% All three complexity notions are based on the notion of state preparation, instead of more traditional definition of complexity such as the decidability of a computational problem. 
% The first two consider classes based on sequences of quantum states preparable by a polynomial-sized quantum circuit, where the circuits are uniformly generated by a computational class, for instance, the class $\mathsf{PSPACE}$, which results in the complexity class $\mathsf{StatePSPACE}$~\cite{RosenthalYuen:2022,MetgerYuen:2023}.
% The third notion considers a relative complexity, where the complexity is measured between two given states, and is measured by the number of gates, from a given gate-set, required to transform one state in another state~\cite{Aaronson:2004}. 
% For our definition of state preparation complexity, we drop the uniformity constraint from~\cite{RosenthalYuen:2022,MetgerYuen:2023} and define a class as $\mathsf{StateX}$, which refers to states preparable by circuits of type $\mathsf{X}$. 
% As an example, if $\mathsf{X} = \QNC^0$, this results in the class $\mathsf{StateQNC^0}$, which is the set of states preparable from the $\ket{0}^n$ state by poly-size constant-depth circuits. 
% This notion is similar to the relative complexity from~\cite{Aaronson:2004}, where one state is the  $\ket{0}^n$ state and instead of counting the number of gates we consider the set of states preparable by a fixed number of gates. Using this notion of complexity we show that any state preparable by an $\LAQCC^*$ circuit is also preparable by a $\mathsf{PostQPoly}$ circuit, the class of circuits of polynomial depth with an additional post-selection gate. 

\paragraph{Summary of results}
\begin{itemize}
    \item We give a new definition of a computational model that captures the power of the four step process: applying a constant number of layers of one- and two-qubit gates; performing a syndrome measurement; perform a fast classical computation determining corrections; apply corrections. We call this model \emph{Local Alternating Quantum Classical Computations}, or $\LAQCC$ for short. In this model we bound the allowed quantum operations, intermediate classical calculations, and number of rounds separately. In Section~\ref{sec:LAQCC_model} we define this model and give a list of operations based on results from literature contained in this computational model. In some of these operations we explicitly use that we allow for multiple, but at most constant, rounds  of corrections.
    \item  We show show that there exist $\LAQCC$ circuits that can not be weakly simulated in Section~\ref{sec:IQP_in_LAQCC}. We further show that for every $\LAQCC$ circuit there exists a $\QNC^1$ circuit simulating it perfectly, in Section~\ref{sec:LAQCC_in_QNC1}.
    \item We introduce a new type computational complexity for preparing states and show that the extension of $\LAQCC$ where we allow a polynomial number of rounds and unbounded classical computation, is contained in $\mathsf{PostQPoly}$, the class of polynomial circuits with post-selection, in Section~\ref{sec:Complexity results}.
    \item We show a protocol to prepare the uniform superposition state of size $q$ in $\LAQCC$ using $\mathO(\ceil{\log_2(q)}^2)$ qubits in Section~\ref{sec:superposition_modulo_q}. 
    \item We show a protocol to prepare the $W_n$ state in $\LAQCC$ using $\mathO(n\log(n))$ qubits in Section~\ref{sec:W_state_in_LAQCC}.
    \item We show two ways of preparing the Dicke-$(n,k)$ state. The first method is in $\LAQCC$, works up to $k = \mathO(\sqrt{n})$, uses $\mathO(n^2\log(n))$ qubits, and is found in Section~\ref{sec:dicke:small_k}. The second method is in $\LAQCC\text{-}\mathsf{LOG}$ (an extension of $\LAQCC$ allowing for logarithmic number of alterations instead of constant), works for any $k$, uses $\mathO(\text{poly}(n))$ qubits, and is found in Section~\ref{sec:Dicke_in_LAQCC_LOG}. 
    \item We extend on our $\LAQCC$ method of generating Dicke-$(n,k)$ states for $k = \mathO(\sqrt{n})$ and show a protocol to generate many-body scar states for a particular Hamiltonian in $\LAQCC$ (Section~\ref{sec:many_body_scar}). 
\end{itemize}
Summarized in a table, we provide the following state generation protocols:
\begin{table}[htb]
\centering
\begin{tabular}{l|l|l|l}
\textbf{State description} & \textbf{Width} & \textbf{Depth} & \textbf{Implementation}\\
\hline 
Uniform superposition mod $q$: $\frac{1}{\sqrt{q}} \sum_{i = 0}^{q-1}\ket{i}$ & $\mathO(\ceil{\log^2 q})$ & $\mathO(1)$ & Section~\ref{sec:superposition_modulo_q}\\

$W$-state: $\frac{1}{\sqrt{n}}\sum_{i = 0}^{n-1}\ket{e_i}$ & $\mathO(n \log n)$ & $\mathO(1)$ & Section~\ref{sec:W_state_in_LAQCC}\\

Dicke-$(n,k)$, $k = \mathO(\sqrt{n})$: $\binom{n}{k}^{-1/2}\sum_{x \in \{0,1\}^n: |x| = k} \ket{x}$ &  $\mathO(n^2\log n)$ & $\mathO(1)$ 
&Section~\ref{sec:dicke:small_k}\\

Dicke-$(n,k)$: $\binom{n}{k}^{-1/2}\sum_{x \in \{0,1\}^n: |x| = k} \ket{x}$ & $\mathO(\text{poly}(n))$ & $\mathO(\log n)$ &Section~\ref{sec:Dicke_in_LAQCC_LOG}\\

QMBS: $\ket{S_k} = \frac{1}{k! \sqrt{\mathcal N(n,k)}}(Q^\dagger)^k \ket{\Omega}$ &  $\mathO(n^2\log n)$ & $\mathO(1)$  &  Section~\ref{sec:many_body_scar}
\end{tabular}
\caption{Summary of state preparation protocols given in this paper.}
\label{tab:sate_prep}
\end{table}
In the entry for the quantum many-body scar state $Q$ denotes the raising operator and $\mathcal N(n,k)=\binom{n-k-1}{k}$. 
Section~\ref{sec:many_body_scar} will provide more details on the variables and the implementation. 

\paragraph{Organization of the paper}
\noindent We first introduce relevant preliminaries in Section~\ref{sec:preliminaries}. 
In Section~\ref{sec:LAQCC_model} we formally define the class of Local Alternating Quantum-Classical Computations ($\LAQCC$). We also show that any Clifford circuit can be implemented in constant depth $\LAQCC$ (a result based on a result from measurement-based quantum computing~\cite{jozsa2006introduction}). 
This result allows us to give many useful multi-qubit gates and routines in Section~\ref{sec:gates_created_in_LAQCC}. 
Beyond that we show that constant depth $\LAQCC$ circuits are contained in $\QNC^1$ and that any $\mathsf{IQP}$ circuit has an $\LAQCC$ implementation.
We conclude this section with an analysis of a more powerful instantiation of $\LAQCC$ and show an inclusion with respect to the class $\mathsf{PostQPoly}$, which is the class of circuits of polynomial depth with one additional post-selection gate. 
In Section~\ref{sec:state_prep_in_LAQCC} we give $\LAQCC$ circuit implementations for preparing the uniform superposition over an arbitrary number of states, the $W$-state and the Dicke state up to $k = \mathO(\sqrt{n})$. We furthermore give a log-depth circuit implementation for preparing the Dicke state for any $k$. We conclude by showing a $\LAQCC$ circuit for generating many body scar states of a particular type of Hamiltonian.


\section{Related Work}
%\subsection{Cost Volume based Deep Stereo Matching}
%Stereo matching is a typical problem that has been studied for decades and a well-known four-step pipeline \cite{scharstein2002taxonomy} has been established, where cost volume construction is an indispensable step. Current state-of-the-art stereo matching methods are all cost volume based methods and they can be categorized into two types. Typically, a cost volume is a 4D tensor of height, width, disparity, and features. The first category just uses a full correlation to generate a single-feature cost volume. Such methods are usually efficient but lose much information because of the decimation of feature channels. Many previous work, including Dispnet \cite{dispnet}, MADNet \cite{madnet}, IResNet \cite{iresnet} and AANet \cite{aanet}, belong to this category. The second category usually uses concatenation \cite{gcnet} or group-wise correlation \cite{gwcnet} to generate a multi-feature 4D cost volume. Such a method can achieve better performance while requiring higher computational complexity and memory consumption. Actually, a majority of the top-performing networks in public leaderboards belong to this category, such as GANet \cite{ganet}, CSPN \cite{cspn} and ACFNet \cite{acfnet}. These methods generally employ multiple 3D convolution layers to constantly regularize the 4D cost volume and then apply softmax over the disparity dimension to produce a discrete disparity probability distribution. The final predicted disparity is obtained by softly weighting indices according to their probability, which is also called soft argmin in GCNet \cite{gcnet}. However, soft argmin leaves the output susceptible to multi-modal disparity probability distributions. ACFNet \cite{acfnet} observes this problem and proposes to directly supervise the cost volume with unimodal ground truth distributions. In contrast, we define an uncertainty estimation to quantify the degree to which the cost volume tends to be multi-modal distribution, higher implies the higher possibility of estimation error.

\subsection{Multi-scale Cost Volume based Stereo Matching}
Cost volume construction is an indispensable step in the well-known four-step pipeline for stereo matching \cite{scharstein2002taxonomy, pamisurvey1, pamisurvey2}. Typically, current state-of-the-art stereo matching methods can be categorized into two types of cost volume-based methods, where the cost volume is a 4D tensor of height, width, disparity, and features. The first category usually uses the single-feature 3D cost volume generated by full correlation, which is efficient while losing much information due to the decimation of feature channels. Many real-time methods, such as Dispnet \cite{dispnet}, MADNet \cite{madnet, madnet_pami} and AANet \cite{aanet}, belongs to the category. Moreover, two-stage refinement \cite{mcvmfc} and pyramidal towers \cite{madnet} are commonly applied in the single-feature cost volume based network to construct multi-scale cost volume. The second category usually uses the multi-feature 4D cost volume generated by concatenation \cite{gcnet} or group-wise correlation \cite{gwcnet}, which can achieve better performance with higher computational complexity and memory consumption. Most top-performing networks, including GANet \cite{ganet}, CSPN \cite{cspn} and ACFNet \cite{acfnet} belong to this category. 
% In these methods, the 4D cost volume is constantly regularized by multiple 3D convolution layers and then a discrete disparity probability distribution can be produced by softmax. Next, the final predicted disparity can be obtained by softly weighting indices according to their probability \cite{gcnet}. However, such output is susceptible to multimodal disparity probability distributions and ACFNet \cite{acfnet} gives a solution by directly supervising the cost volume with unimodal ground truth distributions to alleviate this problem. 
Recently, to alleviate the high computational complexity and memory consumption when employing multi-feature 4D cost volumes, \cite{cvpmvsnet, cascade, uscnet} propose to use cascade cost volume representation in multi-view stereo. These methods usually first predict an initial disparity at the coarsest resolution of the image and then gradually refine the disparity by narrowing down the disparity search space. More closely related to our approach is Casstereo \cite{cascade}, which first extended such representation to stereo matching. It selected to uniform sample a pre-defined range to generate the next stage’s disparity search range. Instead, we employ pixel-level uncertainty estimation to adaptively adjust the next stage disparity searching range and generate pseudo-labels for subsequent domain adaptation. Our method also shares similarities with UCSNet \cite{uscnet}, which constructs uncertainty-aware cost volume in multi-view stereo while it doesn’t employ uncertainty estimation to generate pseudo-labels.

%\subsection{Multi-scale Cost Volume based Deep Stereo Matching} 
% \subsection{Multi-scale Cost Volume based Stereo Matching} 
%Multi-scale cost volume firstly was applied in the single-feature cost volume based network with the form of two-stage refinement \cite{mcvmfc} and pyramidal towers \cite{madnet}. Recently, cascade cost volume representation \cite{cvpmvsnet, cascade, uscnet} was proposed in multi-view stereo to alleviate the high computational complexity and memory consumption when employing multi-feature 4D cost volumes. These methods generally predict an initial disparity at the coarsest resolution of the image. Then, they will narrow down the disparity search space and gradually refine the disparity. More closely related to our approach is Casstereo \cite{cascade}, which first extended such representation to stereo matching. It selected to uniform sample a pre-defined range to generate the next stage’s disparity search range. Instead, we employ uncertainty estimation to adaptively adjust the next stage pixel-level disparity searching range and push the next stage's cost volume to be predominantly unimodal.

% The single-feature cost volume based network with the form of two-stage refinement \cite{mcvmfc} and pyramidal towers \cite{madnet} first employ multi-scale cost volume for stereo matching. Recently, to alleviate the high computational complexity and memory consumption when employing multi-feature 4D cost volumes, \cite{cvpmvsnet, cascade, uscnet} propose to use cascade cost volume representation in multi-view stereo, which generally predict an initial disparity at the coarsest resolution of the image. Then, the disparity search space is narrowed down and the disparity is gradually refined. More closely related to our approach is Casstereo \cite{cascade}, which first extended such representation to stereo matching. It selected to uniform sample a pre-defined range to generate the next stage’s disparity search range. Instead, we employ uncertainty estimation to adaptively adjust the next stage pixel-level disparity searching range and push the next stage's cost volume to be predominantly unimodal.

% Figure environment removed

\subsection{Robust Stereo Matching} 
There exist three categories of generalization definitions for robust stereo matching. 1) Cross-domain Generalization: the network’s ability to perform well on unseen scenes (cannot see the image pairs of the target domain in advance). Towards this end, Jia et al \cite{sungeneralizaiton} propose to incorporate scene geometry priors into an end-to-end network. Zhang et al \cite{dsmnet} introduce a domain normalization and a trainable non-local graph-based filter to construct a domain-invariant stereo matching network. 2) Adapt Generalization: the network’s ability to adapt pre-trained models to the new domain with unlabeled target data. Previous work usually pre-trains the models on synthetic data and then adapts it to new target domains with Graph Laplacian regularization \cite{zoom}, non-adversarial progressive color transfer \cite{adastereo}, and Knowledge Reverse Distillation \cite{aohnet}. More closely related to our approach are \cite{aohnet, unsuperviseddomainadaptation} in stereo matching and Monoresmatch \cite{monoresmatch} in monocular depth estimation, which also proposes to generate a pseudo-label for domain adaptation. However, these methods all select to employ classical stereo matching methods \cite{sgm} alongside with confidence estimators, e.g., left-right consistency check to generate pseudo-labels. That is all these methods need an independent method to generate corresponding pseudo-labels. Instead, the proposed method is an end-to-end network that can generate the predicted disparity map, corresponding uncertainty map and pseudo-labels jointly, which is a more simple, yet efficient way. 
% Instead, our proposed method can employ pixel-level and area-level uncertainty estimation to self-distill the predicted disparity maps of our pre-training model and generate sparse while reliable pseudo-labels to align the domain gap, which is a more simple, yet efficient way. 
3) Joint Generalization: the network’s ability to perform well on a variety of datasets with the same model parameters. MCV-MFC \cite{mcvmfc} introduces a two-stage finetuning scheme to achieve a good trade-off between generalization and fitting capability on multiple datasets. However, it doesn’t touch the inner difference between diverse datasets, e.g, the unbalanced disparity distribution. To further address this problem, we propose a cascade cost volume to adaptively the next stage disparity searching space, where the pixel-level uncertainty estimation is at the core.

% \subsection{Monocular Depth Estimation}
% Monocular depth estimation aims to estimate depth values from a single image, instead of stereo images or multiple frames in a video. This problem is ill-posed because of the ambiguity of object sizes. However, humans could estimate the depth from a single image with prior knowledge of the scenes. Recently, learning based methods were explored to learn depth values by supervised or unsupervised learning. Eigen et al. first employed Convolutional Neural Networks (CNN) to predict depth in a coarse-to-fine manner and further improved its performance by multi-task learning. Liu et al. presented deep convolutional neural fields model by combining deep model with continuous CRF. Li et al. [22] refined deep CNN outputs with a hierarchical CRF. Multi-scale continuous CRF was formulated into a deep sequential network by Xu et al. [45] to refine depth estimation. Unsupervised methods tried to train monocular depth estimation with stereo
% image pairs or image sequences and test on single images. Garg et al. [9] used novel image view synthesis loss to train a depth estimation network in an unsupervised way. Godard et al. [11] introduced left-right consistency regularization to improve the performance of view synthesis loss. Recently, some work also propose to use the stereo matching network as a proxy to learn depth from synthetic data or directly employ traditional stereo matching methods to distill proxies labels from the target domain, which proves the feasibility of distilling stereo matching networks to learn monocular depth estimation.



\section{METHODS}
\label{sec:methods}
\subsection{Problem Definition and Proposed Framework}
The objective is to reconstruct a dense point cloud that precisely represents the shape of unknown transparent objects from sparse point clouds extracted with active tactile interactive perception. To this end, we propose a novel framework termed ACTOR shown in Fig.~\ref{fig:framework}. In Fig.~\ref{fig:framework}(a) we propose a self-surpervised learning approach with an autoencoder network that is trained on subsampled pointclouds from synthetic objects belonging to the same category but not identical as the real objects. In Fig.~\ref{fig:framework}(b), we propose a novel active tactile-based unknown transparent object exploration strategy which is used for inference with our trained model to reconstruct a dense point cloud. We demonstrate downstream tasks such as tactile-based pose estimation.
% and tactile-based object recognition. 

\subsection{Deep Self-Supervised Learning for 3D Object Reconstruction}
\label{ssec:deep_reconstruction}
We generate a dataset $\mathcal{D}$\footnote{\url{https://www.robotact.de/tactile-reconstruction}} of synthetic object models from the ShapeNet repository~\cite{chang2015shapenet} in order to leverage the open-source datasets and avoid expensive real tactile-data collection. The synthetic object models belong to the same category but are different from the real unknown transparent objects. 
We uniformly sample $N_{in} = 2048$ points from the synthetic object meshes. These pointclouds are normalized and scaled to fit into a $[0,1]^3$ cube and added to the dataset, $\mathcal{P}_{in} \in \mathcal{D}$. 
% The generated dataset is provided in the project page\footnote{\url{https://robotac-bmw.github.io/tactile_reconstruction/}}.
In order to generate the input point clouds $\mathcal{P}^{\bullet}_{in}$ to the network, we randomly subsample the $\mathcal{P}_{in}$ by voxel-grid subsampling by the factor $k$ i.e., $\mathcal{P}^{\bullet}_{in} \in \mathbb{R}^{\lceil \frac{1}{k}N_{in} \rceil \times 3}$.  This creates a challenging task for reconstruction with higher values for $k$ as simpler techniques based on interpolation with neighborhood points cannot be used. 

\subsubsection*{Feature-Extraction Encoder}
The network architecture shown in Figure~\ref{fig:framework}(a) is proposed as an autoencoder (AE) that uses a self-supervised approach to reconstruct the original point cloud from a subsampled point cloud. 
The encoder takes subsampled point clouds as inputs and generates a high dimensional feature vector. The feature vector captures the global geometric shape information of the input point cloud. 
In general, any deep network that works on raw input point clouds to provide a high dimensional feature vector can be used as an encoder. In particular,
we use a modified PointNet architecture~\cite{qi2017pointnet} for the encoder. PointNet takes unordered point clouds and generates a global feature descriptor vector of size 1024. The network learns a set of optimization functions that select interesting or informative points of the point cloud. The encoder consists of $[1\times1]$ convolutions with output channels size $(64, 64, 128, 1024)$ with the first convolutional layer with kernel size $[1\times3]$ to encode the input pointcloud of $N\times3$ dimension. The convolution layers are aggregated by a max-pooling layer. We introduce a self-attention layer~\cite{zhang2019self} whose outputs are aggregated with the max-pooled features to provide the global feature vector.  
We have summarized the encoder in Figure~\ref{fig:framework}(a).
% As the encoder provides a high-dimensional global feature vector, we term it as feature-extraction encoder.

\textbf{Self-Attention (SA) Layer:} The SA layer is introduced as it can encode meaningful spatial relationships between features and focus on important local features. From the input layer ($\mathtt{conv2d-1024}$), two separate multi-layer perceptrons (MLPs) are used to get features $\mathbf{G}$ and $\mathbf{H}$ which are subsequently used to get the weights as $\mathbf{W} = softmax(\mathbf{G}^T\mathbf{H})$. The input features are transformed using another MLP to obtain $\mathbf{K}$ and multiplied with the weights as $\mathbf{W}^T\mathbf{K}$.
These vectors are summed with the input vector to produce the output features.
% The SA layer description is shown in Fig.~\ref{fig:self_atten}.  
% \setlength{\columnsep}{0pt}
% \begin{wrapfigure}[12]{r}{0.8\linewidth}
%   \centering
%     % \vspace{-0.5cm}
%     % Figure removed
%   \caption{The self-attention unit.}
%     % \vspace{-0.5cm}
%   \label{fig:self_atten}
% \end{wrapfigure}
% % Figure environment removed

\subsubsection*{Upsampling Decoder}
We design an upsampling decoder that upsamples the input global feature vector to provide the reconstructed dense output point cloud $\mathcal{P}_{out}$. The upsampling decoder is composed by a fully connected layer with output dimension of 1024 and five deconvolutional layers with kernel sizes and output channels shown in Fig.~\ref{fig:framework}(a).  
The decoder produces the output point cloud with point size set to 2048 while training as this is sufficiently dense for reconstruction purposes. 

\subsubsection*{Loss Function}
In order to encourage the upsampled point cloud to be in proximity to the original input point cloud and follow the underlying geometrical surface of the object, we use the Chamfer distance metric~\cite{borgefors1986distance} as the loss. Given the input point cloud prior to subsampling, $\mathcal{P}_{in}$ and the reconstructed output point cloud $\mathcal{P}_{out}$, the loss is defined as:
\begin{align}
    \mathcal{L}_{CD}(\mathcal{P}_{in}, \mathcal{P}_{out}) &= \frac{1}{|\mathcal{P}_{in}|}\sum_{p_1 \in \mathcal{P}_{in}} \min_{p_2 \in \mathcal{P}_{out}} ||p_1 - p_2||_{2} + \\ & \frac{1}{|\mathcal{P}_{out}|}\sum_{p_2 \in \mathcal{P}_{out}} \min_{p_1 \in \mathcal{P}_{in}} ||p_2 - p_1||_{2} \nonumber,
    \label{eq:chamfer_dist}
\end{align}
where $|\bullet|$ refers to the number of points in the point cloud and $||\bullet||_2$ refers to the L2 norm. The loss $\mathcal{L}_{CD}$ represents the average distance between the \textit{closest} points in the two point clouds. We use the weighted loss for learning stability as the reconstruction loss $\mathcal{L}_{rec} = \alpha\mathcal{L}_{CD}$ with $\alpha = 100$ set empirically.
For surface reconstruction from the dense reconstructed point cloud, we use the ball-pivoting algorithm~\cite{bernardini1999ball}.

% \subsubsection*{Recognition Network}
% \label{ssec:recog_net}
% The pretrained encoder layers for reconstruction task are frozen for category-level classification. We employ three fully-connected layers with parameters 512, 256, and $n$ respectively where $n$ represents the number of categories of the objects.
% The softmax cross-entropy loss is used for training the recognition network. The recognition head is shown in Fig.~\ref{fig:framework}(a.I). The subsampled sparse point clouds from our synthetic dataset with different subsampling ratios and data augmentation with random rotations are used. Network implementation details are provided in Sec.~\ref{ssec:setup}.



%%%%%%%%%%%%%%%%%%%%%%%%%%%%%%%%%%%%%%%%%%%%%%%%%%%%%%%%%%%%%%%%%%%%%%%
%%%%%%%%%%%%%%%%%%%%%%%%%%%%%%%%%%%%%%%%%%%%%%%%%%%%%%%%%%%%%%%%%%%%%%%
\subsection{Active Deep Tactile-based Unknown Transparent Object Reconstruction and Pose Estimation}
\subsubsection{Active Tactile-based Transparent Object Reconstruction}
The model trained with only \textit{synthetic data} as described in Sec.~\ref{ssec:deep_reconstruction} is used during the inference with \textit{real-world} transparent objects. The sparse tactile point cloud data is collected autonomously by the robot using an information gain-based active strategy. We define two types of tactile actions for data acquisition: touch and pinch actions as shown in Figure~\ref{fig:occupancy_grid}.
% The action nomenclature is derived from human grasp taxonomy studies~\cite{feix2015grasp}.
The touch action is executed as a guarded horizontal straight-line motion wherein the object is not moved upon contact. The touch action is defined by a tuple $\mathbf{a}^{t} = \{\mathbf{s}^t, \overrightarrow{\mathbf{d}^t} \}$ where $\mathbf{s}^t \in \mathbb{R}^3$ is the start point of the tactile-sensorised gripper and $\overrightarrow{\mathbf{d}^t} \in \mathbb{R}^3$ is the direction of the gripper-motion defined in the world-coordinate frame $\mathcal{W}$. During the pinch action the robot approaches the object in a vertical straight-line motion with a completely open gripper and performs an antipodal enclosure grasp on the object. The fingers of the gripper are closed until the force on the tactile sensors exceeds a predefined threshold.
The pinch action is characterized by $\mathbf{a}^{p} = \{\mathbf{s}^p \}$ where $\mathbf{s}^p \in \mathbb{R}^3 $ is the start position of the gripper motion vertically above the object at a predefined height as shown in Figure~\ref{fig:occupancy_grid}. Given the 2D bounding box of the object (a priori known or through a RGB camera), a probabilistic occupancy grid $\mathcal{OG}_i$ of preset height and resolution $og_{res}$ is defined. Each cell of the occupancy grid $c_i$ is represented by an occupancy probability $p(c_i)$ which is initially set to 0.5. During exploration, if a cell is discovered to belong to the object, the probability is set to 1 and similarly, if the cell belongs to free space, the probability is set to 0. The probabilities are updated through ray intersections based on the virtual sensor model. We define a virtual sensor model of the tactile sensor which casts a set of rays $\mathcal{R} = \{r_1, r_2, \dots, r_{n_{taxel}} \}$ where ${n_{taxel}} $ refers to the number of taxels in the sensor array. The independence assumption of the probability of each grid cell with one another allows us to calculate the overall entropy of the $\mathcal{OG}$ as the summation of the entropy of each cell. The Shannon entropy of the overall occupancy grid is calculated as:
\begin{equation}
    \mathbb{H}(\mathcal{OG}) = \sum_{c_i \in \mathcal{OG}} p(c_i)log(p(c_i)) + (1 - p(c_i))(1 - log(p(c_i))).
    \label{eq:entropy}
\end{equation}
Monte-Carlo sampling of possible tactile actions $N_{nbt}$ are performed for computing the next best tactile (NBT) action. The actions space $\mathcal{A}_{nbt}$ is comprised of an equal number of touch and pinch respectively as $\mathcal{A}_{nbt} = \{a^p, a^t\}_{N_{nbt}}$. The expected measurements $\hat{\mathbf{z}}_t$ for each action $a_t \in \mathcal{A}$ is computed using ray-traversal algorithms~\cite{hornung2013octomap}. 
Given the observed grid cell $c$ and the measurement from sensor observation $z$, the log-odds is updated as $L(c|z) = L(c) + l(z)$ wherein $L(c) = log\frac{p(c)}{1-p(c)}$ and  
\begin{equation}
    l(z) = \left\{
                \begin{array}{ll}
                  log\frac{p_h}{1-p_h}  \quad \mathrm{if} \ z \widehat{=} \textit{ hit} \\
                  log\frac{p_m}{1-p_m} \quad \mathrm{if} \ z \widehat{=} \textit{ miss} 
                \end{array}
              \right.
    \label{eq:log-odds}
\end{equation}
where $p_h$ and $p_m$ are the probabilities of hit and miss which are user-defined values set to 0.7 and 0.4 respectively as in~\cite{hornung2013octomap}. The posterior probability $p(c|z)$ can be computed by inverting $L(c|z)$. The expected information gain by taking an action $a_t \in \mathcal{A}_{nbt}$ with expected measurement $\hat{\mathbf{z}}_t$ is provided by the Kullback-Liebler divergence of the posterior entropy and the prior entropy as: 
\begin{equation}
    E[\mathbb{I}(p(c_i | \mathbf{a}_t,  \hat{z}_t))] = \mathbb{H}(p(c_i)) - \mathbb{H}(p(c_i | \mathbf{a}_t,  \hat{z}_t))
    \label{eq:kl_view}
\end{equation}
Therefore, the action that maximizes the expected information gain is considered as the NBT action:
\begin{equation}
    \mathbf{a}^{nbt*}_t = \argmax_{\mathbf{a} \in \mathcal{A}}(E[\mathbb{I}(p(c_i | \mathbf{a}_t,  \hat{z}_t))])
    \label{eq:kl_view_max}
\end{equation}
Each tactile action extracts contact positions in 3D space and contact forces. The direction of the normal force is used to extract the normal direction $\hat{n}$ of the object surface. The contact points are aggregated into the tactile point cloud $\mathcal{P}^t$. In order to initialize the NBT action calculation, an initial point cloud (with $N_{\mathcal{P}^t} = 20$) is required, which is extracted by randomised touch actions. Further points are collected in an active manner using the NBT criteria. A minimum number of points in the tactile point cloud is required to perform model inference $N_{\mathcal{P}^t} > N_{min}$ which is tuned empirically. The tactile point cloud is provided as input to the trained network and the reconstructed point cloud $\mathcal{P}_{out}$ is obtained . 
% This is used for downstream task Section~\ref{ssec:pose_estimation}. 
% For acceptable reconstruction accuracy around 100 tactile points is required.  

% [TODO:] check for action taxonomy if its correct

%%%%%%%%%%%%%%%%%%%%%%%%%%%%%%%%%%%%%%%%%%%%%%%%%%%%%%%%%%%%%%%%%%%%%%%
% Figure environment removed
%%%%%%%%%%%%%%%%%%%%%%%%%%%%%%%%%%%%%%%%%%%%%%%%%%%%%%%%%%%%%%%%%%%%%%%


%%%%%%%%%%%%%%%%%%%%%%%%%%%%%%%%%%%%%%%%%%%%%%%%%%%%%%%%%%%%%%%%%%%%%%%
%%%%%%%%%%%%%%%%%%%%%%%%%%%%%%%%%%%%%%%%%%%%%%%%%%%%%%%%%%%%%%%%%%%%%%%
\subsubsection{Tactile-Based Object Pose Estimation}
\label{ssec:pose_estimation}

We perform the 6D pose estimation through dense to sparse point cloud registration. The sparse scene point cloud $\mathbf{s}_i \in \mathcal{S}$ is represented by the tactile points and the dense object point cloud $\mathbf{o}_i \in \mathcal{O}$ is represented by the reconstructed point cloud in~\ref{ssec:deep_reconstruction} without the need for the object model. Point cloud registration problem with $M$ known correspondences can be formulated as:
\begin{equation}
     \mathbf{s}_i = \mathbf{S}\cdot(\mathbf{R}\mathbf{o}_i) + \mathbf{t} \quad i = 1, \dots M,
     \label{eq:generativemodel}
 \end{equation}
where $\mathbf{S} \in \mathbb{R}^3$ represents scale, $\mathbf{R} \in SO(3)$ represents rotation and $\mathbf{t} \in \mathbb{R}^3$ represents translation which are unknown and to be estimated and $\cdot$ is the element-wise product. 
%% [TODO] : check derivation

We perform the point cloud registration using our novel translation-invariant Quaternion filter (TIQF) presented in~\cite{murali2022active} to determine $\mathbf{R}$, $\mathbf{S}$ and $\mathbf{t}$. 
The scale, rotation and translation are decoupled by finding the relative vectors between corresponding points, i.e., $\forall o_i, o_j \in \mathcal{O}, s_i, s_j \in \mathcal{S}$ the relative vectors are $\mathbf{s}_{ji} = \mathbf{s}_j - \mathbf{s}_i$ and $\mathbf{o}_{ji} = \mathbf{o}_j - \mathbf{o}_i$. Equation~\eqref{eq:generativemodel} is reformulated as:
\begin{align}
    \mathbf{s}_j - \mathbf{s}_i &= (\mathbf{S}\cdot\mathbf{R}\mathbf{o}_j + \mathbf{t}) - (\mathbf{S}\cdot\mathbf{R}\mathbf{o}_i + \mathbf{t}) ,\\
    \mathbf{s}_{ji} &= \mathbf{S}\cdot\mathbf{R}\mathbf{o}_{ji} \quad .
    \label{eq:trans_invariance}
\end{align}

We note that equation~\eqref{eq:trans_invariance} is independent of translation. Taking the L2-norm on both sides for Eq.~\eqref{eq:trans_invariance} and recalling that norm is rotation invariant we get:
\begin{equation}
    \mathbf{||s||}_{ji} = \mathbf{||S||}\cdot\mathbf{||o||}_{ji} \quad .
    \label{eq:rot_invariance}
\end{equation}
The scale $\mathbf{S}$ is estimated by taking the ratio of the axis aligned bounding box (AABB) of the scene and object point clouds, i.e., if $\mathcal{X}_{AABB} = \{ (x_{min}, x_{max}), (y_{min}, y_{max}), (z_{min}, z_{max}) \}$ represents the AABB for a point cloud $\mathcal{X}$, then:
\begin{align}
     \mathbf{S} &= \{ \frac{|x_{max} - x_{min}|_{\mathcal{S}}}{|x_{max} - x_{min}|_{\mathcal{O}}}, \frac{|y_{max} - y_{min}|_{\mathcal{S}}}{|y_{max} - y_{min}|_{\mathcal{O}}} , \frac{|z_{max} - z_{min}|_{\mathcal{S}}}{|z_{max} - z_{min}|_{\mathcal{O}}}    \}
     \label{eq:scale}
 \end{align}
Using the estimated scale and using $\tilde{\mathbf{o}}_{ji} = \mathbf{S}\mathbf{o}_{ji}$ for convenience we are left with a pure rotation to estimate:  
\begin{align}
    \tilde{\mathbf{s}}_{ji} &= \mathbf{R}\tilde{\mathbf{o}}_{ji} \quad .
    \label{eq:trans_scale_invariance}
\end{align}
 We cast the rotation estimation problem into a recursive Bayesian estimation framework and derive a linear state and measurement model. Reformulating Eq.\eqref{eq:trans_scale_invariance} using quaternions we get: 
 \begin{equation}
    \overline{\mathbf{s}}_{ji} = \mathbf{x} \odot \overline{\mathbf{o}}_{ji} \odot \mathbf{x}^{*}, 
    \label{eq:quat_objective}
\end{equation}
where $\mathbf{x}$ is the quaternion form of $\mathbf{R}$, $\odot$ is the quaternion product, ${\mathbf{x}}^{*}$ is the conjugate of $\mathbf{x}$, and $\overline{\mathbf{s}}_{ji}=\{0,\tilde{\mathbf{s}}_{ji}\}$ and $\overline{\mathbf{o}}_{ji}=\{0,\tilde{\mathbf{o}}_{ji}\}$.
Using the matrix form of quaternion product, we can rewrite Eq.\eqref{eq:quat_objective} as:
\begin{align}
    \begin{bmatrix}
        0 & -\tilde{\mathbf{s}}_{ji}^T \\
        \tilde{\mathbf{s}}_{ji} & \tilde{\mathbf{s}}_{ji}^{\times}
    \end{bmatrix}\mathbf{x} -  \begin{bmatrix}
        0 & -\tilde{\mathbf{o}}_{ji}^T \\
        \tilde{\mathbf{o}}_{ji} & -\tilde{\mathbf{o}}_{ji}^{\times}
    \end{bmatrix} \mathbf{x} = \mathbf{0} \\
    \underbrace{\begin{bmatrix}
        0 & -(\tilde{\mathbf{s}}_{ji} - \tilde{\mathbf{o}}_{ij})^T \\
        (\tilde{\mathbf{s}}_{ji} - \tilde{\mathbf{o}}_{ji}) & (\tilde{\mathbf{s}}_j + \tilde{\mathbf{s}}_i + \tilde{\mathbf{o}}_j + \tilde{\mathbf{o}}_i)^{\times}
        \end{bmatrix}_{4 \times 4}}_{\mathbf{H}_t} \mathbf{x} &= \mathbf{0} \quad ,
        \label{eq:expected_measurement}
\end{align}
where $(\ )^\times$ denotes the skew-symmetric matrix formulation. Equation~\eqref{eq:expected_measurement} is of the form $\mathbf{H}_t\mathbf{x} = 0$ where $\mathbf{H}_t$ is the pseudo-measurement matrix~\cite{choukroun2006novel}. We note that Eq.~\eqref{eq:expected_measurement} represents a noise-free state estimation where $\mathbf{H}_t$ depends only on sparse and dense point correspondences which are $\tilde{\mathbf{s}}_{ji}$ and $\tilde{\mathbf{o}}_{ji}$. We design a pseudo-measurement model as $ \mathbf{H}_t \mathbf{x} = \mathbf{z}^h$
% \begin{align}
%     \mathbf{H}_t \mathbf{x} &= \mathbf{z}^h,
%     \label{eq:measurement_model}
% \end{align}
and set $\mathbf{z}^h = 0$. Since we have a static process model, the object does not move and $\mathbf{x}$ and $\mathbf{z}_t$ are Gaussian distributed, 
the state $\mathbf{x}_t$ and covariance matrix $\Sigma^{\mathbf{x}}_{t}$ at each timestep $t$ are computed through a linear Kalman filter. The Kalman filter equations are skipped for brevity and a in-depth derivation is provided in our prior work~\cite{murali2022active}.
As the Kalman filter does not implicitly ensure the constraints on the quaternion as $||\mathbf{x}|| = 1$, we normalise the state and uncertainty after each update step as $\bar{\mathbf{x}}_{t} = \frac{\mathbf{x}_{t}}{||\mathbf{x}_{t}||_2} \quad, \bar{\Sigma}^{\mathbf{x}}_{t} = \frac{\Sigma^{\mathbf{x}}_{t}}{||\mathbf{x}_{t}||_2^2}$. We convert the estimated rotation $\Bar{\mathbf{x}}_t$ to its equivalent rotation matrix $\mathbf{R}$. It used to estimate the translation using the following relation: $\mathbf{t} = \frac{1}{N} \sum_{i=0}^{N} (\Bar{\mathbf{s}}_i - \mathbf{R} \Bar{\mathbf{o}}_i).$
% \begin{equation}
%     \mathbf{t} = \frac{1}{N} \sum_{i=0}^{N} (\Bar{\mathbf{s}}_i - \mathbf{R} \Bar{\mathbf{o}}_i).
%     \label{eq:translation_solution}
% \end{equation}
% \setlength{\columnsep}{1pt}
% \begin{wrapfigure}[18]{r}{0.6\linewidth}
%   \centering
%     \vspace{-0.5cm}
%     % Figure removed
%   \caption{Translation-invariant measurements}
%     % \vspace{-0.5cm}
%   \label{fig:TIMS}
% \end{wrapfigure}
At each iteration, a rotation and translation estimate is found which is used to transform the object point cloud and the process is repeated by re-estimating the correspondence points. The convergence criteria are set by (a) maximum number of iterations or (b) the relative change in estimated pose parameters is less than a predefined threshold ($0.1mm$ and $0.1^o$). 

% the linear Kalman filter equations are given as:
% \begin{align}
%     \mathbf{x}_{t} &= \bar{\mathbf{x}}_{t-1} - \mathbf{K}_t \left( \mathbf{H}_t \bar{\mathbf{x}}_{t-1} \right) \\
%     \Sigma^{\mathbf{x}}_{t} &= \left( \mathbf{I} - \mathbf{K}_t \mathbf{H}_t \right) \bar{\Sigma}^{\mathbf{x}}_{t-1} \\
%     \mathbf{K}_t &= \bar{\Sigma}^\mathbf{x}_{t-1} \mathbf{H}_t^T \left( \mathbf{H}_t\bar{\Sigma}^\mathbf{x}_{t-1} \mathbf{H}_t^T + \Sigma_t^{\mathbf{h}}\right)^{-1}, 
%     \label{eq:kalman_equations}
% \end{align}
% where $\bar{\mathbf{x}}_{t-1}$ refers to the normalized mean of the state at $t-1$, Kalman gain $\mathbf{K}_t$ and $\bar{\Sigma}^{\mathbf{x}}_{t-1}$ is the covariance matrix of the state at $t-1$. 
% The parameter $\Sigma_t^{\mathbf{h}}$ is referred as the measurement uncertainty during time $t$. It is dependent on the state and is provided by~\cite{choukroun2006novel}:
% \begin{align}
%     \Sigma_t^{\mathbf{h}} = \frac{1}{4}\rho\left[ tr(\bar{\mathbf{x}}_{t-1}\bar{\mathbf{x}}_{t-1}^T + \bar{\Sigma}^{x}_{t-1})\mathbb{I}_4 - (\bar{\mathbf{x}}_{t-1}\bar{\mathbf{x}}_{t-1}^T + \bar{\Sigma}^{x}_{t-1} )\right], 
%     \label{eq:choukron}
% \end{align}
% wherein the constant $\rho$ corresponds to the uncertainty of the correspondence measurements and $tr$ refers to trace.


%%%%%%%%%%%%%%%%%%%%%%%%%%%%%%%%%%%%%%%%%%%%%%%%%%%%%%%%%%%%%%%%%%%%%%%
%%%%%%%%%%%%%%%%%%%%%%%%%%%%%%%%%%%%%%%%%%%%%%%%%%%%%%%%%%%%%%%%%%%%%%%
% \subsubsection{Transparent Object Manipulation}
% \label{ssec:tactile_manipulation}
% With the computed 6D pose and estimated CAD model, we design a simple grasping technique in order to grasp and lift the transparent objects. For each \textit{category} of objects, we generated several grasp plans using GraspIt~\cite{miller2004graspit}. Each grasp plan includes the grasp position, orientation and approach vector relative to the model of the object and a grasp quality score. With the pose of the object, the grasp plans are filtered based on kinematic constraints of the robot, workspace limitations and possible collisions with other objects in the scene. Among the remaining grasp plans, the plan with the highest score is chosen and executed. The robot lifts the transparent object and places it in a pre-defined position.
% An online grasp planning and collision avoidance framework is out of the scope of this current work but can be readily integrated into the current framework.

%%%%%%%%%%%%%%%%%%%%%%%%%%%%%%%%%%%%%%%%%%%%%%%%%%%%%%%%%%%%%%%%%%%%%%%
%%%%%%%%%%%%%%%%%%%%%%%%%%%%%%%%%%%%%%%%%%%%%%%%%%%%%%%%%%%%%%%%%%%%%%%

% \subsubsection{Tactile-based Transparent Object Recognition}
% \label{ssec:classification}
% % Figure environment removed
% We use the pretrained encoder model with fixed weights for category-level classification. We employ three fully-connected layers with parameters 512, 256 and $n$ respectively where $n$ represents the number of categories of the objects. Transfer learning is employed to fine-tune the classification network shown in Figure~\ref{fig:framework}(a) on the sparse pointclouds from ShapeNet database.
% During inference, the real sparse tactile pointclouds are used as input to the network for recognition network described in Sec.~\ref{ssec:recog_net}. While the task is challenging, the real-world tactile data are not used during fine-tuning intentionally as collection of large-scale datasets is prohibitively time consuming. The input pointcloud is pre-processed prior to inference by normalising and scaling to fit in $[0,1]^3$ cube to be uniform with the training dataset.
\section{System Overview}
In this section, we provide an overview of the Online Matching system. We introduce the end-to-end workflow, followed by a detailed overview of the online agent in Section \ref{sec:online_agent}.

\subsection{End-to-end workflow}
\label{sec:workflow}
The entire Online Matching workflow consists of an offline pipeline and an online agent responsible for the closed-loop learning, as shown in Fig. \ref{fig:e2e}. The offline pipeline produces a sparse bipartite graph, as introduced in Section \ref{sec:sparse_bipartite_graph}, which is adopted by the online agent. It has the following components:
% Figure environment removed

\begin{itemize}

\item \textbf{Two-tower model trainer.}
We train an offline two-tower model by sequentially consuming a large amount of logged user feedback over time. The sequential training ensures that the model can adapt to the distribution change in the latest batch of data.
As mentioned earlier, this model encodes item features, allowing it to generate meaningful embeddings even for newly added items. The two-tower model is exported on a daily basis, with both towers used by the downstream components to create the sparse bipartite graph. The user tower is also used by the online system to generate user embedding $\rvu$ and context vector $\rvw_u$.

\item \textbf{Candidate selection.} 
This component creates a corpus of items eligible for exploration. Multiple filters are applied to ensure the selected candidates satisfy our strict trust-and-safety criteria.
In this paper, our system is mainly focused on exploring fresh videos, therefore a rolling time window that covers a few days is used for item selection. We also apply various quality thresholds to balance the quality and size of the corpus. 

\item \textbf{Clustering and graph building.}
Once clustering is finished based on the two-tower model exported most recently, graph builder is triggered to build the sparse graph according to Algorithm ~\ref{alg:graph-construction}. 
The graph building process is executed in both batch and real-time modes concurrently. In batch mode, graph builder takes the output of the candidate selection step, and exports a new graph every few hours. Real-time mode complements batch mode by incrementally updating the sparse graph with newly eligible items to ensure a small latency for items to enter the exploration pool. 
\end{itemize}

Online agent represents the system that conducts the bandit algorithm and aggregates user feedback in real-time. It takes the sparse graph produced from the above pipeline as input. Whenever the sparse graph is updated, bandit parameters are synchronized in online agent with low latency: 
% new edges are added with infinite confidence bound 
new edges are added with infinite confidence bound (so that they will be prioritized for future exploration),
and old edges that are only in the previous graph version are removed. In the next section, we will delve into the specifics of online agent.

\subsection{Online Agent} \label{sec:online_agent}

% Figure environment removed

As shown in Fig. \ref{fig:feedback}, the key components of online agent are chained as a closed loop. 
Explored items from Online Matching are allowed to be shown at a fixed position in the UI, so that users' direct feedback on them can be measured without being affected by 
% ranking policies.
existing ranking policies. 
The log processor is used to incrementally generate various kinds of engagement signals on the explored items in the format compatible with the downstream jobs. The feedback aggregation processor is built on top of Bigtable ~\cite{bigtable-osdi06} and is responsible for aggregating pair-wise (cluster and video) bandit parameters according to Eq. \eqref{eq:diag_linucb_update}. As illustrated in Table ~\ref{tbl:toy-example}, each row in the Bigtable represents one cluster, and each column represents the corresponding items of that cluster in the sparse graph. Conceptually, Bigtable is a sparsely populated table that can scale to the billions of rows and columns, and is compatible with the proposed Diag-LinUCB algorithm.   


\begin{table}[h]
  \begin{tabular}{|c|c| }
    \hline
     \textbf{Column} & \textbf{Cell value} \\
    \hline
     feedback:$item_1$ & \{ $d_{1,1}$: $54.4$, $b_{1,1}$: $624.2$, $w_{1,1}^{2}$: $1.5$ \} \\
    \hline
     feedback:$item_2$ & \{ $d_{2,1}$: 57.6, $b_{2,1}$: 144.6, $w_{2,1}^{2}$: 1.8 \} \\
    \hline
     feedback:$item_3$ & \{ $d_{3,1}$: 76, $b_{3,1}$: 547.1, $w_{3,1}^{2}$: 2.6 \} \\
    \hline
  \end{tabular}
  \caption{Illustration of how the aggregated bandit parameters $d_{j,c}$, $b_{j,c}$, $w_{j,c}^{2}$ in Eq. \eqref{eq:diag_linucb_update} are stored in Bigtable.  Row keys correspond to hashed cluster IDs. Here we demonstrate one row corresponding to cluster 1 and its three columns.} 
  \label{tbl:toy-example}
\end{table}

Bandit parameters in Bigtable are frequently pushed to the cluster-to-candidates lookup service. The recommender service is used for obtaining user clusters, and look up the candidates and their bandit parameters from the lookup service. The recommender service is further used to rank all the candidates according to the UCB in Eq. \eqref{eq:ucbj} or Eq. \eqref{eq:diag-exploit} if the goal is to exploit high quality candidates.


It is worth noting that, compared to the classic bandits setup with a fixed set of arms, we always have fresh items added as new arms in Online Matching. In particular, fresh items are \textsl{continuously} injected into the bipartite graph through the graph building pipeline. New items that have never been explored would have an infinite confidence bound in Eq. ~\eqref{eq:ucbj} and are therefore prioritized for exploration in the recommender service. Due to the batch addition of new items, we can observe spikes of infinite UCB scores as shown in Fig. \ref{fig:inf-score}. The spikes usually disappear quickly, demonstrating that users' feedback to new items are quickly incorporated in bandit parameters.


% Figure environment removed

\subsection{System Performance}
To demonstrate the real-time performance of Online Matching, we measure the following two types of update latency:
\begin{itemize}
    \item \textsl{Policy update latency}: This is the period of time from the point user sees the explored item, to the point when user's feedback on the item is incorporated in bandit parameters contained in the lookup service in Fig. \ref{fig:feedback}. 
    Since our primary application is video recommendation, this latency also includes user's watch time on video capped at a certain value, as a major part of the latency in the log processor. Actually sessionizing user feedback in the log processor contributes to most of the policy update latency.
    \item \textsl{Corpus update latency}: This is the period of time from the point a fresh item is eligible for exploration to the point when the item is added to the sparse graph. 
\end{itemize}
The median and the 95-th percentile of both latency are summarized in Table \ref{tbl:latency}. Besides latency, our system achieves high throughput, e.g., it can handle millions of bandit updates per second, allowing it to scale to billions of users.

\begin{table*}
  \begin{tabular}{ |c|c|c|c| }
    \hline
     & P50 (minutes) & P95 (minutes) & Throughput (updates/second)\\
    \hline
    Policy update latency & 45 & 74 & O(1M) \\
    \hline
    Corpus update latency & 41.1 & 60.1 & O(1K) \\
    \hline
  \end{tabular}
  \caption{Policy update latency, corpus update latency and the system throughput.}
  \label{tbl:latency}
\end{table*}

Particularly, the low latency of policy update is critical for recommendation quality since the expected regret grows as the feedback delay increases \cite{delayfeedback-JoulaniGS13}. To empirically verify the expected regrets, we add artificial latency into the aggregation processor in Fig. \ref{fig:feedback}. As shown in Table ~\ref{tbl:latency-inject}, as latency was introduced, the agent became less capable of identifying low-performing bandits, resulting in a decrease in CTR and total rewards on explored items.

\begin{table}[h]
  \begin{tabular}{ |c|c|c| }
    \hline
     &  CTR & Total Rewards\\
    \hline
    Baseline with no artificial delay injected & - & - \\
    \hline
    20 min delay added & -2.82\% & -11.82\% \\
    \hline
    40 min delay added & -4.4\% & -22.84\% \\
    \hline
  \end{tabular}
\caption{A study of artificial latency injection in policy updates' impact on CTR (click-through rate) and total rewards (measured by multiple user satisfaction and engagement metrics).}
  \label{tbl:latency-inject}
\end{table}



\section{Experimental Results}\label{sec:results}
    \subsection{General Results}
        The basic ResSAN model is used to determine reference results which our expanded model can be compared to as it is structurally similar to ResLAN but does not possess the Lidar adaptive components of it. Further, we compare with the full-size PackNet-SAN and the unmodified NLSPN architecture. 
        As it can be seen from Tab.\,\ref{tab:sota-results}, our LiDAR-adaptive ResLAN achieves competitive performance compared to state-of-the-art standard depth completion methods, which are specialized to the unfiltered 64-beam-LiDAR. The performance differences are in the range of a few centimetres in terms of MAE, which is acceptable given the practical advantage that ResLAN can generalize to different beam patterns as will be shown below.

        Furthermore, we compared the architectures for a set of three different input types that contained 64, 32 or 16 LiDAR channels using both filter types on the metrics from the KITTI benchmark. The NLSPN model was trained for the standard depth completion task and then evaluated with different input data. As for the ResSAN models, we trained one model for each input type and tested it for the corresponding one which serve serve as the \emph{Baseline} in Tab.\,\ref{tab:overall-results}. Our ResLAN model was jointly trained for all three settings. As listed in Tab.\,\ref{tab:overall-results}, the ResLAN models outperform the challenging baseline in all metrics for FOV filtering and all but one for sparse filtering. This implies that our LiDAR adaptive model is able to outperform dedicated models in case of very sparse input depth. Fig.\,\ref{fig:comp-plot} shows this is indeed the case for 32 and even more for 16 channels. For FOV-filtered inputs with 16 channels, the ResLAN exhibits approx. $10\%$ smaller MAE than the baseline. As for the NLSPN, it becomes apparent that it is not capable of generalizing to other input types since it shows clearly worse results. The difference is especially pronounced for the FOV filtering where on average more than every fourth predicted pixel is more than $25 \%$ deviating from the ground truth\,($\delta_{1.25}$). Therefore, using a weight-adapting network in combination with differently filtered input depths allows us to train models that outperform their non-adaptive counterparts.

        \begin{table}[]
            \centering
    	    \small
            \vspace{0.4cm}
            \caption{\textbf{Depth estimation result for standard depth completion} when the ResSAN model was only trained for 64 channels and the ResLAN model for multiple tasks. The PackNet-SAN and NLSPN models were trained with the setup that was also used for our model architecture.}
            \footnotesize
            \setlength{\tabcolsep}{5pt}
            \begin{tabular}{@{}lrrrrl@{}}
            \toprule
            \multicolumn{6}{c}{\textbf{Standard LiDAR Depth Completion}}                                                                                                                         \\ \midrule
            \multicolumn{1}{l|}{Method}          & RMSE $\downarrow$            & MAE  $\downarrow$            & iRMSE $\downarrow$             & iMAE $\downarrow$ & $\delta_{1.25}$ $\uparrow$ \\
            \multicolumn{1}{l|}{}                & \multicolumn{1}{l}{{[}mm{]}} & \multicolumn{1}{l}{{[}mm{]}} & \multicolumn{1}{l}{{[}1/km{]}} & {[}1/km{]}        &                            \\ \midrule
            \multicolumn{1}{l|}{PackNet-SAN}     &  914                            &  298                            &  2.78                              &  1.4                 &  99.65 \%                          \\
            \multicolumn{1}{l|}{NLSPN}           &  \textbf{889}                            &   \textbf{263}                           &  \textbf{2.62}                              &   \textbf{1.3}                &   \textbf{99.61} \%                         \\ \midrule
            \multicolumn{1}{l|}{ResSAN (Ours)}   & 948                             &  275                            &  2.75                              &    1.4               &   99.58 \%                         \\
            \multicolumn{1}{l|}{ResLAN (Ours)} &   969                           &  283                            &   2.83                             &   1.4                &  99.56 \%                          \\ \bottomrule
            \end{tabular}
            \vspace{0.2cm}
            \label{tab:sota-results}
        \end{table}

        \begin{table}[]
    	    \centering
    	    \small
    	    \caption{\textbf{Depth estimation results of the two baseline setups and the explicit and implicit ResSAN} when evaluated on a combination of 16, 32 and 64 channel depth inputs. Please note that Specialist Methods need to train three specialized networks, one for each of the three types of inputs while our method only uses one network.}
            \footnotesize
            \setlength{\tabcolsep}{4.8pt}
            \begin{tabular}{@{}lrrrrl@{}}
                \toprule
                \multicolumn{6}{c}{\textbf{Sparse Channel Filter}}                                                                                                                                  \\ \midrule
                \multicolumn{1}{l|}{Method}        & RMSE $\downarrow$            & MAE  $\downarrow$            & iRMSE $\downarrow$             & iMAE $\downarrow$ & $\delta_{1.25}$ $\uparrow$  \\
                \multicolumn{1}{l|}{}              & \multicolumn{1}{l}{{[}mm{]}} & \multicolumn{1}{l}{{[}mm{]}} & \multicolumn{1}{l}{{[}1/km{]}} & {[}1/km{]}        &                             \\ \midrule
                \multicolumn{1}{l|}{NLSPN}         &  1396                            &  437                            & 5.54                               &  2.2                 &  98.82 \%                           \\
                \multicolumn{1}{l|}{Baseline}      & \textbf{1207}                             &  381                            & 4.41                               &  1.8                 &  \textbf{99.37} \%                           \\
                \multicolumn{1}{l|}{ResLAN (Ours)} &  1215                            &  \textbf{378}                            &  \textbf{4.27}                              &  \textbf{1.7}                 &  99.31 \%                           \\ \toprule
                \multicolumn{6}{c}{\textbf{Field-of-View Filter}}                                                                                                                                   \\ \midrule
                \multicolumn{1}{l|}{Method}        & RMSE $\downarrow$            & MAE  $\downarrow$            & iRMSE $\downarrow$             & iMAE $\downarrow$ & $\delta_{1.25}$ $\uparrow$ \\
                \multicolumn{1}{l|}{}              & \multicolumn{1}{l}{{[}mm{]}} & \multicolumn{1}{l}{{[}mm{]}} & \multicolumn{1}{l}{{[}1/km{]}} & {[}1/km{]}        &                             \\ \midrule
                \multicolumn{1}{l|}{NLSPN}         &  2738                            &  1702                            & 12.3                              &  4.3                 &  74.69 \%                           \\
                \multicolumn{1}{l|}{Baseline}      &  1556                            &  525                            &  6.8                              &  3.0                 & 98.14 \%                            \\
                \multicolumn{1}{l|}{ResLAN (Ours)} &  \textbf{1548}                            &  \textbf{519}                            &  \textbf{6.44}                              &  \textbf{2.8}                 & \textbf{98.52 \%}                            \\ \bottomrule
            \end{tabular}
            \label{tab:overall-results}
        \end{table}

        
        
        % Figure environment removed
        
        % Figure environment removed

    \subsection{Filter Effects}
        Comparing the effect of the two different types of depth input filters on the model performance, it becomes apparent that FOV filtering is the more challenging task. In that setting, reducing LiDAR channels is more detrimental to the performance than sparse filtering as it creates regions where no depth information is available. Effectively, the model is forced to perform depth prediction in these regions. These effects are highlighted in the depth images in Fig.\,\ref{fig:dense-maps} where the effect of a 16-channel sparse depth filter and a 16-channel FOV can be compared.

    \subsection{Generalization Capabilities}
        We trained three models for both filter types eaach, so the combinations and number of filtered depth inputs they receive are different. This serves the purpose of testing the generalization capabilities of the ResLAN architecture as well as the robustness to different filter settings. After training, the models were evaluated for the depth input settings they were trained for, as well as for ones they weren't exposed to. Overall, ResLAN shows good generalization capabilities. As one can gather from Fig.\,\ref{fig:explicit-comp} and Fig.\,\ref{fig:implicit-comp}, the consequences of slightly varying sets of input depth settings are limited. The most considerable deviations can be seen when the model is tasked to extrapolate. For instance, the model $\{64, 32, 16\}$ shows a noticeably higher MAE for eight-channel depth inputs than the model that was trained for it. Similar behaviour can be seen for the FOV filtering case as well for the model $\{64, 48, 32\}$ when tasked to generalize for a 16-channel input. There is no such pronounced effect for generalization tasks that lie between two filter settings the model was trained for. At most, it can be observed that models that were trained for a smaller range of filter values perform slightly better than ones that have to cover a wider range. The number of filter settings used in a fixed range does not relevantly influence the model performance, as can be seen, when comparing the two models in Fig.\,\ref{fig:implicit-comp}, which are both trained for a range of 64 to 32 channels but one with three filter settings and the other one with five.
    
    % Figure environment removed
    
    
    % Figure environment removed
\section{Conclusion and Future Work}
In this work, I design corruption-robust algorithms for the Lipschitz contextual search problem. I present the \emph{agnostic checking} technique and demonstrate its effectiveness in designing corruption-robust algorithms. There are several open problems for future research. First, in the algorithm I propose for pricing loss, the schedule for agnostic checks is fixed upfront. Can the learner design an adaptive checking schedule for the pricing loss? Second, this work assumes the learner has knowledge of the Lipschitz constant $L$. Can the learner design efficient no-regret algorithms without knowledge of $L$? 

\bibliographystyle{ACM-Reference-Format}
\bibliography{main.bib}

\end{document}
\endinput
%%
%% End of file `sample-acmcp.tex'.
