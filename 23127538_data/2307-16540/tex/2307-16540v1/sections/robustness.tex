

% In this section, we evaluate the robustness of different systems. We vary selectivity of different queries and  

% Figure environment removed

\subsection{Robustness}
\label{sub:robustness}

ADOPT does not rely on query optimization to pick the best attribute order. This can be a significant advantage for queries with user-defined functions or selection predicates, for which there are no available selectivity estimates. Mainstream systems pick a query plan that may be arbitrarily off from a good one. In contrast, ADOPT may quickly realize that such a plan is subpar and switch to a different one. To benchmark this observation, we consider experiments to assess the {\em  robustness} of ADOPT and System-X, which are the two systems we use that rely on attribute orders, when adding to the join queries very simple (unary) yet arbitrary selection conditions that can throw off standard query optimizers.

\begin{hyp}
ADOPT outperforms System-X consistently when varying the selectivity of unary predicates. 
\end{hyp}

% \vspace{-1em}
Figure~\ref{fig:robustness} shows the relative speedup of ADOPT over System-X as we vary the selectivity of unary predicates (selections with constants) on three randomly chosen attributes: we choose the five selectivities 0.2, 0.4, 0.6, 0.8, and 1 for the three attributes along the x-axis,  y-axis, and the circles for an (x,y)-point.
The color of each 3D point in the plot varies from blue to red: The more intense the red is, the higher is the speedup of ADOPT over System-X. System-X mostly outperforms ADOPT for 3-cliques. ADOPT is up to three times faster than System-X for all other cliques and cycles. This is due to the difficulty of optimizers to pick a good query plan in the absence of selectivity estimates, here even for unary predicates.

