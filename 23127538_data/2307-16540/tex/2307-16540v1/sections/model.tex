\section{Problem Model}

We introduce our problem model, related terminology and notations. We consider the following type of queries ... (TODO). 

ADOPT uses the leapfrog triejoin, a worst-case optimal join algorithm, as basis. This algorithm operates on combinations of join attribute values. The set of all possible combinations can be represented as a multi-dimensional cube. By the term ``cube'', we generally refer to cubes of join attribute values in the following. Given a cube $c$, we denote by $c.l$ and $c.u$ the vector of lower and upper cube bounds respectively. Furthermore, we use the subscript to refer to bounds in specific dimensions (e.g., $c.l_i$ refers to the lower bound for the $i$-th attribute within cube $c$). 

TODO: what else should we introduce here?