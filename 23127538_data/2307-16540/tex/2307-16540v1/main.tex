\documentclass[10pt]{article}
\usepackage{pgfplots}
\pgfplotsset{compat=1.14}
\usepackage{xspace}
\usepackage{tikz}
\usepackage{multirow}
\usepackage{marvosym}
\usepackage{enumitem}
\usepackage{fullpage}
\usepackage{amsthm}
\usepackage{amsmath}
\usepackage{amssymb}
\usepackage{mathtools}
\usepackage{amsfonts}
\usepackage{booktabs}
%\usepackage{subcaption}
\usepackage{relsize}
\usepackage[T1]{fontenc}
%\usepackage{thmtools}
\usepackage{adjustbox}


\usepackage{hyperref}
\pgfplotsset{compat=1.16}
\usepackage{tikz-3dplot}
\usepackage{xspace}
\usepackage{tikz}
\pgfdeclarelayer{background}
\pgfsetlayers{background,main}


\usepackage{xcolor,colortbl}

\usepackage{thmtools}

\usepackage{verbatim}
\usepackage{graphicx}
\usepackage{algorithmicx}
\usepackage[ruled]{algorithm}
\usepackage{algpseudocode}
\usepackage{xcolor}
\usepackage[center]{subfigure}
\usepackage{caption}
\usepackage{url}


\algnewcommand{\algorithmicand}{\textbf{ and }}
\algnewcommand{\algorithmicor}{\textbf{ or }}
\algnewcommand{\OR}{\algorithmicor}
\algnewcommand{\AND}{\algorithmicand}

\usepackage{xifthen}
\usepgfplotslibrary{ternary, units}
\usetikzlibrary{patterns,shapes,arrows,positioning,fit,decorations.pathreplacing}
\usetikzlibrary{decorations.pathmorphing, pgfplots.ternary, pgfplots.units}
\usetikzlibrary{shadows.blur}
\usepackage{pgfplotstable}
\usepgfplotslibrary{groupplots}
\usepackage{multicol}
\usepackage[normalem]{ulem}

\definecolor{goodgreen}{rgb}{0.1, 0.5, 0.1}
\definecolor{Purple}{HTML}{911146}

\newcommand{\nop}[1]{}
\theoremstyle{plain}                  
\newtheorem{hyp}{Hypothesis}
\newtheorem{assumption}{Assumption}
\newtheorem{theorem}{Theorem}
\newtheorem{lemma}[theorem]{Lemma}
\newtheorem{proposition}[theorem]{Proposition}
\newtheorem{conjecture}[theorem]{Conjecture}
\newtheorem{corollary}[theorem]{Corollary}         
\newtheorem{definition}[theorem]{Definition}
\newtheorem{example}[theorem]{Example}
\newtheorem{remark}[theorem]{Remark}

\newcommand{\revision}[1]{\textcolor{black}{#1}}




\date{}
\title{ADOPT: Adaptively Optimizing Attribute Orders for \\ 
Worst-Case
Optimal Join Algorithms via Reinforcement Learning}


\author{
\begin{tabular}{cccc}
Junxiong Wang$^1$  & Immanuel Trummer$^1$ & Ahmet Kara$^2$ &  Dan Olteanu$^2$ \\
junxiong@cs.cornell.edu &  itrummer@cornell.edu & kara@ifi.uzh.ch &  olteanu@ifi.uzh.ch 
\end{tabular}\\ \\
$^1$Cornell University  \enspace\enspace $^2$University of Zurich
}

\begin{document}
\maketitle

\begin{abstract}
The performance of worst-case optimal join algorithms depends on the order in which the join attributes are processed. Selecting good orders before query execution is hard, due to the large space of possible orders and unreliable execution cost estimates in case of data skew or data correlation. We propose ADOPT, a query engine that combines adaptive query processing with a worst-case optimal join algorithm, which uses an order on the join attributes instead of a join order on relations. ADOPT divides query execution  into episodes in which different attribute orders are tried. Based on run time feedback on attribute order performance, ADOPT converges quickly to near-optimal orders. It avoids redundant work across different orders via a novel data structure, keeping track of parts of the join input that have been successfully processed. It selects attribute orders to try via reinforcement learning, balancing the need for exploring new orders with the desire to exploit promising orders. 
In experiments with various data sets and queries, it outperforms baselines, including commercial and open-source systems using worst-case optimal join algorithms, whenever queries become complex and therefore difficult to optimize.
\end{abstract}

\section{Introduction}
Current quantum hardware is unable to carry out universal quantum computations due to the buildup of errors that occur during the computation. 
The magnitude of the individual error is currently above the value that the Threshold Theorem requires in order to kick-start quantum error correction and fault-tolerant quantum computation~\cite[Section 10.6]{nielsen_chuang_2010}. 
Although the experimentally achieved fidelity rates are promising and the error bounds are inching closer to the required threshold, we will have to work for the foreseeable future with quantum hardware with errors that build-up during the computation.  This implies that we can only do a limited number of steps before the output of the computation has become completely uncorrelated with the intended one.

For fault-tolerant quantum computing, we repeat four steps: 
1) We apply a number of single and two-qubit quantum gates, in parallel whenever possible; 
2) We perform a syndrome measurement on a subset of the qubits; 
3) We perform fast classical computations to determine which errors have occurred and how to correct them; 
and, 4) We apply correction terms based on the classical computations.
We then repeat these four steps with a next sequence of gates. 
These four steps are essential to fault-tolerant quantum computing. 


The starting point of this work is to use the four steps outlined above, not to carry out error correction and fault-tolerant computation, but to enhance short, constant-depth, {\em uncorrected} quantum circuits that perform single qubit gates and {\em nearest-neighbor} two qubit gates. 
Since in the long run we will have to implement error-correction and fault-tolerant computation anyhow, and this is done by such a four-step process, why not make other use of this architecture? Moreover, on some of the quantum hardware platforms, these operations are already in place.
Embracing this idea we naturally arrive at the question: what is the computational power of \textit{low-depth} quantum-classical circuits organized as in the four steps outlined above? 
We thus investigate circuits that execute a small, ideally constant, number of stages, where at each stage we may apply, in parallel, single qubit gates and {\em nearest-neighbor} two qubit gates, followed by measurements, followed by low-depth classical computations of which the outcome can control quantum gates in later stages. 
It is not clear, at first, whether such circuits, especially with constant depth, can do anything remotely useful. 
But we will see that this is indeed the case: many quantum computations can be done by such circuits in constant depth. 
By parallelizing quantum computations in this way, we improve the overall computational capabilities of these circuits, as we do not incur errors on qubits that are idle, simply because qubits are not idle for a very long time. 
Furthermore, reducing the depth of quantum circuits, at the cost of increasing width, allows the circuit to be run faster even if errors occur.

The first usage of such a four-step layout, not to do error correction, but to perform computations, can be found in the paradigm of measurement-based quantum computing~\cite{gottesman1999demonstrating,raussendorf2001one,jozsa2006introduction,clark2007generalised}: 
A universal form of quantum computing where a quantum state is prepared and operations are performed by measuring qubits in different bases, depending on previous measurements and intermediate measurements.

\citeauthor{PhamSvore2013} were the first to formalize the four-step protocol for performing computations~\cite{PhamSvore2013}. They included specific hardware topologies by considering two-dimensional graphs for imposing constraints on qubit interactions. In their model, they develop circuits for particularly useful multi-qubit gates, including specifying costs in the width, number of qubits, depth, number of concurrent time steps, size, and total number of non-Identity operations.
As a result, they find an algorithm that factors integers in polylogarithmic depth.
\citeauthor{Browne:2011} showed that the main tool in the work by \citeauthor{PhamSvore2013}, the fan-out gate, can also be replaced by additional log-depth classical computations in the measurement-based quantum computing setting~\cite{Browne:2011}.

More recently, \citeauthor{Cirac:2021} introduced a scheme to implement unitary operations involving quantum circuits combined with Local Operations and Classical Communication ($\mathsf{LOCC}$) channels: $\mathsf{LOCC}$-assisted quantum circuits~\cite{Cirac:2021}. Similarly to the four-step scheme we just described, they allow for a short depth circuit to be run on the qubits, followed by one round of $\mathsf{LOCC}$, in which ancilla qubits are measured and local unitaries are applied based on the measurement outcomes. They show that in this model any 1D transitionally invariant matrix-product state (MPS) with fixed bond dimension is in the same phase of matter as the trivial state. Similar ideas can be found in~\cite{TVV_NonAbelianTopologicalOrder_2022, tantivasadakarn2021long}.

In this work, we introduce a new model, called \textit{Local Alternating Quantum-Classical Computations} ($\LAQCC$). In this model we alternate between running quantum circuits (constrained by locality), ending in the measurement of a subset of qubits, and fast classical computations based on the measurement results. The outcome of the classical computations are then used to control future quantum circuits. We allow for flexibility in this model, by giving different constraints to the power of both the quantum circuits and the classical circuits as well as the number of alternations between them. 
Most attention will be given to $\LAQCC$ containing quantum circuits of constant depth, classical circuits of logarithmic depth and at most a constant number of alternations between them. 
Any circuit constructed in this model is considered to be of constant depth. 
We restrict ourselves to logarithmic depth classical computations, as this is the first natural and non-trivial extension beyond constant-depth classical computations. 
Constant-depth classical computations do however also have an equivalent constant-depth quantum implementation.

The definition of $\LAQCC$ sharpens the original definition of \citeauthor{PhamSvore2013} by adding constraints to the intermediate classical computations. This allows us to bound the power of $\LAQCC$ from above. 

The main result of \citeauthor{Cirac:2021}, that 1D translational invariant MPS with fixed bond dimension can be prepared by $\mathsf{LOCC}$-assisted circuits, relies on local symmetries of the MPS. These symmetries allow them to prepare local states (on a constant number of qubits) and glue them together by doing one round of the appropriate entangling measurement and corrections, after which they run a round of local unitaries to get the desired result. This general scheme for preparing states that exhibit an MPS description with the appropriate local symmetries requires only geometrically local unitaries and one round of measurement and corrections an therefore is accessible in $\LAQCC$. Studying different local symmetries, known as Symmetry Protected Topological (SPT) phases of matter, to find measurement-based constant depth circuits for states is a broad ongoing field of research~\cite{TVV_NonAbelianTopologicalOrder_2022, tantivasadakarn2021long, smith2023deterministic}. 
All these schemes have a $\LAQCC$ implementation.

%$\LAQCC$-circuits also exist for general schemes of preparing local states, based on the local tensors, and gluing them together using one round of entangled measurement and corrections, based on the local symmetry. 
%The main result of \citeauthor{Cirac:2021}, that 1D translational invariant MPS with fixed bond dimension can be prepared by $\mathsf{LOCC}$-assisted circuits, relies heavily on local symmetries of the MPS and as a result also has an equivalent $\LAQCC$ implementation. 
%The corrections applied after the measurement round are local unitaries depending on the local symmetries of the MPS. 

 

%This general scheme of preparing local states, based on the local tensors, and gluing it together by doing one round of entangled measurement and corrections, based on the local symmetry, is accessible in $\LAQCC$.
Note however that \citeauthor{Cirac:2021} also suggest a circuit for the $W$-state.
This circuit uses sequentially and dependent measurement-based corrections of the ancilla qubits. 
These dependent measurements translate to sequential alternations between the quantum and classical circuits and therefore increase the total depth to linear depth, exceeding the constant-depth constraints imposed by $\LAQCC$-circuits. 

We study the power of the $\LAQCC$ model with respect to state preparation, showing that even with only constant quantum-depth and logarithmic classical depth it remains possible to prepare states with long-range entanglement.
Another surprising result is that it is unlikely that $\LAQCC$ circuits are classically simulatable. We show that any instantaneous quantum polynomial-time (IQP) circuit~\cite{Bremner2010,Shepherd2009} has an $\LAQCC$ implementation.
Classical simulation of IQP circuits implies the collapse of the polynomial hierarchy to the third level, which is not believed to be true~\cite{Bremner2017}. Therefore, we expect that $\LAQCC$ circuits are unlikely to be classically simulatable. We bound the power of $\LAQCC$ by showing that it is contained in $\QNC^1$, the class of polynomial-size, log-depth circuits.

Next, we also study the power that intermediate classical calculations can add to quantum computations, by considering a new model that alternates between polynomially many polynomial-depth quantum circuits and unbounded classical computations
We study this model by doing a complexity theoretical analysis, where we draw inspiration from the notions of complexity given by \citeauthor{RosenthalYuen:2022}, \citeauthor{MetgerYuen:2023}, and \citeauthor{Aaronson:2004}.
All three complexity notions are based on the notion of state preparation, instead of more traditional definition of complexity such as the decidability of a computational problem. 
The first two consider classes based on sequences of quantum states preparable by a polynomial-sized quantum circuit, where the circuits are uniformly generated by a computational class, for instance, the class $\mathsf{PSPACE}$, which results in the complexity class $\mathsf{StatePSPACE}$~\cite{RosenthalYuen:2022,MetgerYuen:2023}.
The third notion considers a relative complexity, where the complexity is measured between two given states, and is measured by the number of gates, from a given gate-set, required to transform one state in another state~\cite{Aaronson:2004}. 
For our definition of state preparation complexity, we drop the uniformity constraint from~\cite{RosenthalYuen:2022,MetgerYuen:2023} and define a class as $\mathsf{StateX}$, which refers to states preparable by circuits of type $\mathsf{X}$. 
As an example, if $\mathsf{X} = \QNC^0$, this results in the class $\mathsf{StateQNC^0}$, which is the set of states preparable from the $\ket{0}^n$ state by poly-size constant-depth circuits. 
This notion is similar to the relative complexity from~\cite{Aaronson:2004}, where one state is the  $\ket{0}^n$ state and instead of counting the number of gates we consider the set of states preparable by a fixed number of gates. Using this notion of complexity we show that any state preparable by an $\LAQCC^*$ circuit is also preparable by a $\mathsf{PostQPoly}$ circuit, the class of circuits of polynomial depth with an additional post-selection gate. 

All Clifford circuits have a constant-depth $\LAQCC$ implementation, implying that any stabilizer state can be implemented by a constant-depth $\LAQCC$ circuit, see Section~\ref{sec:clifford_circuits} for a proof of this statement. 
Efficient circuits for stabilizer states have been known already through measurement-based quantum computing. Therefore this paper focuses on the preparation of non-stabilizer states, and as a surprising result we find novel constant-depth protocols for four very natural classes of non-stabilizer states.
Despite the extensive research into these four classes of non-stabilizer states and the many applications of them, no efficient constant- or low-depth state preparation protocols are known yet. We specifically consider these four classes as they are all often used as initial states in other algorithms.

The first state is a uniform superposition over an arbitrary number of states. 
This state finds applications in many quantum algorithms, as they often start with a uniform superposition over multiple states. 
This superposition is often achieved by applying Hadamard gates to every qubit due to its simplicity to prepare. 
Yet, the analysis of many algorithms, such as Shor's algorithm~\cite{Shor:1997}, would benefit from a different initial superposition. 
The circuit to prepare the uniform superposition over an arbitrary number of states uses an exact version of Grover search as a subroutine, that turns a probabilistic circuit, with a known constant probability of success, into a deterministic circuit. 
We use the circuit for preparing a uniform superposition over an arbitrary number of states as a subroutine in the next two quantum state preparation protocols. 

The second state is the $W$-state, the uniform superposition over all computational basis states of Hamming-weight~$1$, a natural long-ranged entangled state that displays a fundamentally nonequivalent type of entanglement from the Greenberger–Horne–Zeilinger state~\cite{WState:2000}, for which $\LAQCC$-type constant-depth circuits were previously known~\cite{PhamSvore2013, Cirac:2021}. 
The $W$-state is often used as benchmark for new quantum hardware~\cite{Haffner2005,Neeley2010,GarciaPerez:2021}. 
A novel way to prepare the $W$-state therefore gives a new way to benchmark different quantum devices with each other. 
A circuit for preparing the $W$-state was given in~\cite{Cirac:2021}, but this implementation requires sequentially alternating measurements followed by local unitaries, which in the $\LAQCC$ model is not considered to be of constant depth. 
We improve this protocol by giving an $\LAQCC$ implementation of the $W$-state, based on a compress-uncompress method that links the one-hot and binary encoding of integers.

The third state considered is the Dicke state, a generalization of the $W$-state, a superposition over all computational basis states with Hamming-weight $k$~\cite{Dicke:1954}. 
Dicke states have relevance in various practical settings.
For instance, for quantum game theory~\cite{zdemir2007}, quantum storage~\cite{Bacon_Compress:2006,Plesch:2010}, quantum error correction~\cite{ouyang2014permutation}, quantum metrology~\cite{toth2012multipartite}, and quantum networking~\cite{prevedel2009experimental}. 
Dicke states have been used as a starting state for variational optimization algorithms, most notably Quantum Alternating Operator Ansatz (QAOA)~\cite{Hadfield2019}, to find solutions to problems such as Maximum k-vertex Cover~\cite{Brandhofer2022,cook2020quantum}.
The ground states of physical Hamiltonians describing one-dimensional chains tend to show a resemblance to Dicke states such as states resulting from the Bethe ansatz, making them an ideal starting state when investigating the ground state behavior of these Hamiltonians~\cite{TDL_BetheAnsatzDerivation:2010,B_ExcitedStateQuantumPhaseTransitions:2013,DickeTransitions:2021}. 
For instance, the algorithm by \citeauthor{van2021preparing}, who give an algorithm to prepare the Bethe ansatz eigenstates of the spin-1/2 XXZ spin chain, starts by first preparing a Dicke state~\cite{van2021preparing}. 
A Dicke-state preparation protocol based on the compress-uncompress methodology used in the $W$-state furthermore finds applications in entanglement distillation, where the entanglement of a large state is concentrated on only a few qubits. 
Efficient deterministic circuits for preparing Dicke states have been proposed by \citeauthor{bartschi2019deterministic}~\cite{bartschi2019deterministic, bartschi2022deterministic_short_depth}. 
They provide a quantum circuit of depth $\mathO(k \log(\frac{n}{k}))$, allowing arbitrary connectivity, to prepare a Dicke state, which they conjecture to be optimal when $k$ is constant. 
In this work, we provide a constant-depth $\LAQCC$ circuit below their conjectured bound already for constant $k$. 
However, this does not directly disprove their conjecture, as we allow for intermediate measurements and classical computations. 
More significantly, we even construct constant-depth $\LAQCC$ circuits for $k = \mathO(\sqrt{n})$ greatly improving their bound.
This construction extends the compress-uncompress method for the $W$-state combined with additional subroutines. 

We continue with a log-depth state preparation protocol for the Dicke-state for arbitrary $k$. 
This protocol implements an efficient transformation between the factoradic number representation and the combinatorial number representation of a positive integer. 
The combinatorial number representation relates directly to the Dicke state. 
The provided efficient transformation between number representation systems might be of independent interest. 

We conclude by modifying our protocol for preparing a Dicke-state to a protocol that prepares quantum many-body scar states in constant-depth. 
These states have low entanglement and longer coherence times than states with similar energy density.
These characteristics make many-body scar states interesting to analyze and relevant within physics.
Many-body scar states appear for instance in the AKLT model~\cite{AKLT:1987,MRBAR:2018,MRB:2018} and different spin models~\cite{SI:2019,MOBFR:2020}.
Known methods for preparing these states have polynomial-depth~\cite{Gustafson:2023}, whereas our circuit has constant depth. 

% We conclude by studying the power that intermediate classical calculations can add to quantum computations. 
% In this study, we define a new model that relaxes constant-depth quantum circuits to polynomial depth quantum circuits, log-depth classical calculations to unbounded classical computations and a constant number of alternations to a polynomial number of alternations. 
% We call this model $\LAQCC^*$. 
% We study this model by doing a complexity theoretical analysis, where we draw inspiration from the notions of complexity given by \citeauthor{RosenthalYuen:2022}, \citeauthor{MetgerYuen:2023}, and \citeauthor{Aaronson:2004}.
% All three complexity notions are based on the notion of state preparation, instead of more traditional definition of complexity such as the decidability of a computational problem. 
% The first two consider classes based on sequences of quantum states preparable by a polynomial-sized quantum circuit, where the circuits are uniformly generated by a computational class, for instance, the class $\mathsf{PSPACE}$, which results in the complexity class $\mathsf{StatePSPACE}$~\cite{RosenthalYuen:2022,MetgerYuen:2023}.
% The third notion considers a relative complexity, where the complexity is measured between two given states, and is measured by the number of gates, from a given gate-set, required to transform one state in another state~\cite{Aaronson:2004}. 
% For our definition of state preparation complexity, we drop the uniformity constraint from~\cite{RosenthalYuen:2022,MetgerYuen:2023} and define a class as $\mathsf{StateX}$, which refers to states preparable by circuits of type $\mathsf{X}$. 
% As an example, if $\mathsf{X} = \QNC^0$, this results in the class $\mathsf{StateQNC^0}$, which is the set of states preparable from the $\ket{0}^n$ state by poly-size constant-depth circuits. 
% This notion is similar to the relative complexity from~\cite{Aaronson:2004}, where one state is the  $\ket{0}^n$ state and instead of counting the number of gates we consider the set of states preparable by a fixed number of gates. Using this notion of complexity we show that any state preparable by an $\LAQCC^*$ circuit is also preparable by a $\mathsf{PostQPoly}$ circuit, the class of circuits of polynomial depth with an additional post-selection gate. 

\paragraph{Summary of results}
\begin{itemize}
    \item We give a new definition of a computational model that captures the power of the four step process: applying a constant number of layers of one- and two-qubit gates; performing a syndrome measurement; perform a fast classical computation determining corrections; apply corrections. We call this model \emph{Local Alternating Quantum Classical Computations}, or $\LAQCC$ for short. In this model we bound the allowed quantum operations, intermediate classical calculations, and number of rounds separately. In Section~\ref{sec:LAQCC_model} we define this model and give a list of operations based on results from literature contained in this computational model. In some of these operations we explicitly use that we allow for multiple, but at most constant, rounds  of corrections.
    \item  We show show that there exist $\LAQCC$ circuits that can not be weakly simulated in Section~\ref{sec:IQP_in_LAQCC}. We further show that for every $\LAQCC$ circuit there exists a $\QNC^1$ circuit simulating it perfectly, in Section~\ref{sec:LAQCC_in_QNC1}.
    \item We introduce a new type computational complexity for preparing states and show that the extension of $\LAQCC$ where we allow a polynomial number of rounds and unbounded classical computation, is contained in $\mathsf{PostQPoly}$, the class of polynomial circuits with post-selection, in Section~\ref{sec:Complexity results}.
    \item We show a protocol to prepare the uniform superposition state of size $q$ in $\LAQCC$ using $\mathO(\ceil{\log_2(q)}^2)$ qubits in Section~\ref{sec:superposition_modulo_q}. 
    \item We show a protocol to prepare the $W_n$ state in $\LAQCC$ using $\mathO(n\log(n))$ qubits in Section~\ref{sec:W_state_in_LAQCC}.
    \item We show two ways of preparing the Dicke-$(n,k)$ state. The first method is in $\LAQCC$, works up to $k = \mathO(\sqrt{n})$, uses $\mathO(n^2\log(n))$ qubits, and is found in Section~\ref{sec:dicke:small_k}. The second method is in $\LAQCC\text{-}\mathsf{LOG}$ (an extension of $\LAQCC$ allowing for logarithmic number of alterations instead of constant), works for any $k$, uses $\mathO(\text{poly}(n))$ qubits, and is found in Section~\ref{sec:Dicke_in_LAQCC_LOG}. 
    \item We extend on our $\LAQCC$ method of generating Dicke-$(n,k)$ states for $k = \mathO(\sqrt{n})$ and show a protocol to generate many-body scar states for a particular Hamiltonian in $\LAQCC$ (Section~\ref{sec:many_body_scar}). 
\end{itemize}
Summarized in a table, we provide the following state generation protocols:
\begin{table}[htb]
\centering
\begin{tabular}{l|l|l|l}
\textbf{State description} & \textbf{Width} & \textbf{Depth} & \textbf{Implementation}\\
\hline 
Uniform superposition mod $q$: $\frac{1}{\sqrt{q}} \sum_{i = 0}^{q-1}\ket{i}$ & $\mathO(\ceil{\log^2 q})$ & $\mathO(1)$ & Section~\ref{sec:superposition_modulo_q}\\

$W$-state: $\frac{1}{\sqrt{n}}\sum_{i = 0}^{n-1}\ket{e_i}$ & $\mathO(n \log n)$ & $\mathO(1)$ & Section~\ref{sec:W_state_in_LAQCC}\\

Dicke-$(n,k)$, $k = \mathO(\sqrt{n})$: $\binom{n}{k}^{-1/2}\sum_{x \in \{0,1\}^n: |x| = k} \ket{x}$ &  $\mathO(n^2\log n)$ & $\mathO(1)$ 
&Section~\ref{sec:dicke:small_k}\\

Dicke-$(n,k)$: $\binom{n}{k}^{-1/2}\sum_{x \in \{0,1\}^n: |x| = k} \ket{x}$ & $\mathO(\text{poly}(n))$ & $\mathO(\log n)$ &Section~\ref{sec:Dicke_in_LAQCC_LOG}\\

QMBS: $\ket{S_k} = \frac{1}{k! \sqrt{\mathcal N(n,k)}}(Q^\dagger)^k \ket{\Omega}$ &  $\mathO(n^2\log n)$ & $\mathO(1)$  &  Section~\ref{sec:many_body_scar}
\end{tabular}
\caption{Summary of state preparation protocols given in this paper.}
\label{tab:sate_prep}
\end{table}
In the entry for the quantum many-body scar state $Q$ denotes the raising operator and $\mathcal N(n,k)=\binom{n-k-1}{k}$. 
Section~\ref{sec:many_body_scar} will provide more details on the variables and the implementation. 

\paragraph{Organization of the paper}
\noindent We first introduce relevant preliminaries in Section~\ref{sec:preliminaries}. 
In Section~\ref{sec:LAQCC_model} we formally define the class of Local Alternating Quantum-Classical Computations ($\LAQCC$). We also show that any Clifford circuit can be implemented in constant depth $\LAQCC$ (a result based on a result from measurement-based quantum computing~\cite{jozsa2006introduction}). 
This result allows us to give many useful multi-qubit gates and routines in Section~\ref{sec:gates_created_in_LAQCC}. 
Beyond that we show that constant depth $\LAQCC$ circuits are contained in $\QNC^1$ and that any $\mathsf{IQP}$ circuit has an $\LAQCC$ implementation.
We conclude this section with an analysis of a more powerful instantiation of $\LAQCC$ and show an inclusion with respect to the class $\mathsf{PostQPoly}$, which is the class of circuits of polynomial depth with one additional post-selection gate. 
In Section~\ref{sec:state_prep_in_LAQCC} we give $\LAQCC$ circuit implementations for preparing the uniform superposition over an arbitrary number of states, the $W$-state and the Dicke state up to $k = \mathO(\sqrt{n})$. We furthermore give a log-depth circuit implementation for preparing the Dicke state for any $k$. We conclude by showing a $\LAQCC$ circuit for generating many body scar states of a particular type of Hamiltonian.


\section{Secure Design of \puma}\label{sec:design}
In this section, we first present an overview of \puma, and present the protocols for secure $\gelu$ , $\softmax$, embedding, and $\layernorm$ used by \puma. Note that the linear layers such as matrix multiplication are straightforward in replicated secret sharing, so we mainly describe our protocols for non-linear layers in this manuscript.

\subsection{Overview of \puma}\label{sec:overview}
To achieve secure inference of Transformer models, \puma\ defines three kinds of roles: one model owner, one client, and three computing parties. The model owner and the client  provide their models or inputs to the computing parties (i.e., $P_0$, $P_1$, and $P_2$) in a secret-shared form, then the computing parties execute the MPC protocols and send the results back to the client. Note that the model owner and client can also act as one of the computing party, we describe them separately for generality. \eg, when the model owner acts as $P_0$, the client acts as  $P_1$, a third-party dealer acts as $P_2$, the system model becomes the same with \mpcformer~\citep{li2023mpcformer}.

During the secure inference process, a key invariant is maintained: For any layer, the computing parties always start with 2-out-of-3 replicated secret shares of the previous layer's output and the model weights, and end with 2-out-of-3 replicated secret shares of this layer's output. As the shares do not leak any information to each party, this ensures that the layers can be sequentially combined for arbitrary depths to obtain a secure computation scheme for any Transformer-based model.
%The main focus of \puma\ is to reduce the computation and communication costs between the computing parties while maintaining the desired level of security. 



\iffalse
\textbf{Threat Model.}
Following previous works~\citep{aby3,li2023mpcformer},
\puma\ resists a semi-honest (a.k.a., honest-but-curious) adversary in honest-majority~\citep{lindell2009proof}, where the adversary passively corrupts no more than one computing party. Such an adversary follows the protocol specification exactly, but may try to learn more information than permitted. Please note that \puma\ cannot protect against the extraction of information from the inference results, and the examination of mitigating solutions (\eg, differential privacy~\citep{abadi2016deep}) falls outside the scope of this study.
\fi 

\subsection{Protocol for Secure GeLU}\label{sec:gelu}
Most of the current approaches view the $\gelu$ function as a composition of smaller functions and try to optimize each piece of them, making them to miss the
chance of optimizing the private $\gelu$ as a whole. Given the $\gelu$ function:
\begin{equation}\label{eq:gelu}
\begin{split}
    \gelu(x) &= \frac{x}{2} \cdot \left(1 + \tanh \left( \sqrt{\frac{2}{\pi}} \cdot \left(x + 0.044715 \cdot x^3 \right) \right) \right)\\
    &\approx x\cdot \mathsf{sigmoid}(0.071355\cdot x^3 + 1.595769\cdot x) 
\end{split},
\end{equation}
these approaches~\citep{hao2022iron,characmpctranformer} focus either on designing efficient protocols for function $\tanh$
or using the existing MPC protocols of exponentiation and reciprocal for $\mathsf{sigmoid}$. 

However, none of current approaches have utilized the fact that $\gelu$ function is almost linear on the two sides (\ie, $\gelu(x)\approx 0$ for $x<-4$ and $\gelu(x)\approx x$ for $x>3$). 
Within the short interval $[-4,3]$ of $\gelu$,
we suggest a piece-wise approximation of low-degree polynomials is a more efficient and easy-to-implement choice for its secure protocol. Concretely, our piece-wise low-degree polynomials are shown as equation~(\ref{eq:geluapprox}):
\begin{equation}\label{eq:geluapprox}
\gelu(x)=
\begin{cases}
0, & x<-4 \\
F_0(x), & -4 \le x < -1.95 \\
F_1(x), & -1.95 \le x \le 3 \\
x, & x >3
\end{cases},
\end{equation}
where polynomials $F_0()$ and $F_1()$ are computed by library $\mathsf{numpy.ployfit}$\footnote{\url{https://numpy.org/doc/stable/reference/generated/numpy.polyfit.html}} as equation~(\ref{eq:f0f1}). Surprsingly, the above simple poly fit works very well and our $\mathsf{max\ error}< 0.01403$, $\mathsf{median\ error}< 4.41e-05$, and $\mathsf{mean\ error}< 0.00168$.
\begin{equation}\label{eq:f0f1}
\begin{cases}
F_0(x) &= -0.011034134030615728 x^3 -0.11807612951181953 x^2 \\
&- 0.42226581151983866 x -0.5054031199708174\\
F_1(x) &= 0.0018067462606141187x^6 -0.037688200365904236 x^4 \\
&+ 0.3603292692789629x^2 + 0.5x + 0.008526321541038084
\end{cases}
\end{equation}

Formally, given secret input $\share{x}$, our secure $\gelu$ protocol $\Pi_{\gelu}$ is constructed as algorithm~\ref{protocol:gelu}. 
\iffalse
\begin{itemize}
    \item The parties jointly compute
$\share{b_0}^2 = \Pi_{\mathsf{LT}}(\share{x}, 4)$,
$\share{b_1}^2 = \Pi_{\mathsf{LT}}(\share{x}, -1.95)$, and
$\share{b_2}^2 = \Pi_{\mathsf{LT}}(3, \share{x})$.

\item  Then, each $P_i$ locally compute
$\share{b_3}^2 = \share{b_1}^2 \oplus \share{b_2}^ \oplus 1$ and
$\share{b_4}^2 = \share{b_0}^2 \oplus \share{b_1}^2$

\item Finally, the parties compute and return 
$\share{b_2}^2 \cdot \share{x} + \share{b_4}^2 \cdot F_0(\share{x}) + \share{b_3}^2 \cdot F_1(\share{x})$, where polynomials $(F_0, F_1)$ can be computed easily using secure addition and multiplication (and its variants, \eg, secure square)~\citep{spu}. 
\end{itemize}
\fi 

\begin{algorithm}[tp]
\caption{Secure $\gelu$ Protocol $\Pi_{\mathsf{GeLU}}$}\label{protocol:gelu}
\begin{algorithmic}[1]
\REQUIRE
$P_i$ holds the 2-out-of-3 replicate secret share $\share{x}_i$ for $i\in \{0,1,2\}$ 
\ENSURE
$P_i$ gets the 2-out-of-3 replicate secret share $\share{y}_i$ for $i\in \{0,1,2\}$, where $y=\gelu(x)$.

\STATE $P_0$, $P_1$, and $P_2$ jointly compute
\begin{equation*}
\begin{split}
&\shareb{b_0} = \Pi_{\mathsf{LT}}(\share{x}, -4),~~~\vartriangleright b_0 = 1\{x<-4\}\\
&\shareb{b_1} = \Pi_{\mathsf{LT}}(\share{x}, -1.95),~~~\vartriangleright b_1 = 1\{x<-1.95\} \\
&\shareb{b_2} = \Pi_{\mathsf{LT}}(3, \share{x}),~~~~~~\vartriangleright b_2 = 1\{3<x\}
\end{split}
\end{equation*}
and compute 
$\shareb{z_0} = \shareb{b_0} \oplus \shareb{b_1}$,
$\shareb{z_1} = \shareb{b_1} \oplus \shareb{b_2} \oplus 1$, and $\shareb{z_2}=\shareb{b_2}$. Note that $z_0 = 1\{-4\le x < -1.95\}$, $z_1 = 1\{-1.95\le x\le 3\}$, and $z_2 = 1\{x>3\}$.

\STATE Jointly compute $\share{x^2} = \Pi_{\mathsf{Square}}(\share{x})$, $\share{x^3} = \Pi_{\mathsf{Mul}}(\share{x}, \share{x^2})$, $\share{x^4} = \Pi_{\mathsf{Square}}(\share{x^2})$, and $\share{x^6} = \Pi_{\mathsf{Square}}(\share{x^3})$.

\STATE Computing polynomials $\share{F_0(x)}$ and $\share{F_1(x)}$ based on $\{\share{x}, \share{x^2}, \share{x^3}, \share{x^4}, \share{x^6}\}$ as equation~(\ref{eq:geluapprox}) securely.


\RETURN$\share{y} = \Pi_{\mathsf{Mul_{BA}}}(\shareb{z_0}, \share{F_0(x)}) + \Pi_{\mathsf{Mul_{BA}}}(\shareb{z_1}, \share{F_1(x)})+\Pi_{\mathsf{Mul_{BA}}}(\shareb{z_2}, \share{x})$.

\end{algorithmic}
\end{algorithm}



\subsection{Protocol for Secure Softmax}\label{sec:secureatten}

In the function $\attention(\Q,\K,\V)=
\softmax(\Q \cdot \K^\mathsf{T} + \M) \cdot \V$, where $\M$ can be viewed as a bias matrix, the key challenge is computing function $\softmax$. For the sake of numerical stability, the $\softmax$ function is computed as
\begin{equation}\label{eq:softmax}
    \softmax(\x)[i]=\frac{\exp(\x[i] - \bar{x} - \epsilon)}{\sum_i \exp(\x[i] - \bar{x} - \epsilon)},
\end{equation}
where $\bar{x}$ is the maximum element of the input vector $\x$. 
For the normal plaintext softmax, $\epsilon=0$. For a two-dimension matrix, we apply equation~(\ref{eq:softmax}) to each of its row vector.

Formally, our detailed secure protocol  $\Pi_{\softmax}$ is illustrated in algorithm~\ref{protocol:softmax}, where we propose two optimizations:
\begin{itemize}
\item 
For the first optimization, we set $\epsilon$ in equation~\ref{eq:softmax} to a tiny and positive
value, e.g., $\epsilon =
10^{-6}$, so that the inputs to exponentiation
in equation~\ref{eq:softmax} are all negative. We exploit the negative operands
for acceleration. Particularly, we compute the exponentiation using the Taylor series~\citep{tan2021cryptgpu} with a simple clipping
\begin{equation}\label{eq:negexp}
\mathsf{negExp}(x) = \begin{cases}
    0, &x < T_{\exp} \\
    (1+\frac{x}{2^t})^{2^t}, &x\in [T_{\exp},0].
\end{cases}
\end{equation}
Indeed, we apply the less-than for the branch $x < T_{\exp}$
The division by $2^t$ can be achieved using
$\Pi_{\mathsf{Trunc}}^t$ since the input is already negative. Also, we can
compute the power-of-$2^t$ using $t$-step sequences of square function $\Pi_{\mathsf{square}}$ and $\Pi_{\mathsf{Trunc}}^f$. Suppose our MPC program uses
$18$-bit fixed-point precision. Then we set $T_{\exp}=-14$ given $\exp(-14) < 2^{-18}$, and empirically set $t = 5$.


\item 
Our second optimization is to reduce the number of divisions, which ultimately saves computation and communication costs.
To achieve this, for a vector $\x$ of size $n$, we have replaced the operation $\mathsf{Div}(\x, \mathsf{Broadcast}(y))$ with $\x \cdot  \mathsf{Broadcast}(\frac{1}{y})$, where $y=\sum_{i=1}^n\x[i]$. By making this replacement, we effectively reduce $n$ divisions to just one reciprocal operation and $n$ multiplications.
This optimization is particularly beneficial in the case of the $\softmax$ operation. The $\frac{1}{y}$ in the $\softmax$ operation is still large enough to maintain sufficient accuracy under fixed-point values. As a result, this optimization can significantly reduce the computational and communication costs while still providing accurate results.
\end{itemize}

\begin{algorithm}[tp]
\caption{Secure $\softmax$ Protocol $\Pi_{\softmax}$}\label{protocol:softmax}
\begin{algorithmic}[1]
\REQUIRE
$P_i$ holds the 2-out-of-3 replicate secret share $\share{\x}_i$ for $i\in \{0,1,2\}$, and $\x$ is a vector of size $n$. 
\ENSURE
$P_i$ gets the 2-out-of-3 replicate secret share $\share{\y}_i$ for $i\in \{0,1,2\}$, where $\y=\softmax(\x)$.

\STATE $P_0$, $P_1$, and $P_2$ jointly compute
$\shareb{\mathbf{b}} = \Pi_{\mathsf{LT}}(T_{\exp}, \share{\x})$ and the maximum $\share{\bar{x}} = \Pi_{\mathsf{Max}}(\share{\x})$.

\STATE Parties locally computes $\share{\hat{\x}} = \share{\x} - \share{\bar{x}} - \epsilon$, and jointly compute $\share{\z_0} = 1+  \Pi_{\mathsf{Trunc}}^t(\share{\hat{\x}})$.

\FOR{$j=1,2,\dots, t$}
\STATE $\share{\z_j} = \Pi_{\mathsf{Square}}(\share{\z_{j-1}})$.
\ENDFOR

\STATE Parties locally compute $\share{z} = \sum_{i=1}^n \share{\z[i]}$ and jointly compute $\share{1/z} = \Pi_{\mathsf{Recip}}(\share{z})$.

\STATE Parties jointly compute $\share{\z / z} = \Pi_{\mathsf{Mul}}(\share{\z}, \share{1/z})$

\RETURN $\share{\y} = \Pi_{\mathsf{Mul}_{\mathsf{BA}}}( \shareb{\mathbf{b}}, \share{\z / z})$.

\end{algorithmic}
\end{algorithm}

\subsection{Protocol for Secure Embedding}\label{sec:embed}


The current secure embedding procedure described in~\citep{li2023mpcformer} necessitates the client to  generate a one-hot vector using the token $\tokenid$ locally. This deviates from a plaintext Transformer workflow where the one-hot vector is generated inside the model. As a result, they have to carefully strip off the one-hot step from the pre-trained models, and add the step to the client side, which could be an obstacle for deployment. 



To address this issue, we propose a secure embedding design as follows. Assuming that the token $\tokenid\in [n]$ and all embedding vectors are denoted by $\E= (\e_1^T, \e_2^T, \dots, \e_n^T)$, the embedding can be formulated as $\e_{\tokenid} = \mathbf{E}[\tokenid]$. Given $(\tokenid, \E)$ are in secret-shared fashion, our secure embedding protocol $\Pi_{\mathsf{Embed}}$ works as follows:
\begin{itemize}
    \item The computing parties securely compute the one-hot vector $\shareb{\mathbf{o}}$ after receiving $\share{\tokenid}$ from the client. Specifically, $\shareb{\mathbf{o}[i]}=\Pi_{\mathsf{Eq}}(i,\share{\tokenid})$ for $i\in [n]$.
    \item The parties can compute the embedded vector via $\share{\e_{\tokenid}} = \Pi_{\mathsf{Mul_{BA}}}(\share{\E}, \shareb{\mathbf{o}})$, where  does not require secure truncation.
\end{itemize}
In this way, our $\Pi_{\mathsf{Embed}}$ does not require explicit modification of the workflow of plaintext Transformer models, at the cost of more $\Pi_{\mathsf{Eq}}$ and $\Pi_{\mathsf{Mul_{BA}}}$ operations. 



\subsection{Protocol for Secure LayerNorm}\label{sec:seclayernorm}
Recall that given a vector $\x$ of size $n$, $\layernorm(\x)[i] =  \gamma \cdot \frac{\x[i]-\mu}{\sqrt{\sigma}} + \beta$, where $(\gamma, \beta)$ are trained parameters, $\mu = \frac{\sum_{i=1}^n \x[i]}{n}$, and $\sigma = \sum_{i=1}^n (\x[i] - \mu)^2$. In MPC, the key challenge is the evaluation of the divide-square-root $\frac{\x[i]-\mu}{\sqrt{\sigma}}$ formula. To securely evaluate this formula, CrypTen sequentially executes the MPC protocols of square-root, reciprocal, and multiplication. However, we observe that $\frac{\x[i]-\mu}{\sqrt{\sigma}}$ is equal to $(\x[i]-\mu)\cdot \sigma^{-1/2}$. And in the MPC side, the costs of computing the inverse-square-root $\sigma^{-1/2}$ is similar to that of the square-root operation~\citep{rSqrt}. Besides, inspired by the second optimization of \S~\ref{sec:secureatten}, we can first compute $\sigma^{-1/2}$ and then $\mathsf{Broadcast}(\sigma^{-1/2})$ to support fast and secure $\layernorm(\x)$. And our formal protocol $\Pi_{\layernorm}$ is shown in algorithm~\ref{protocol:layernorm}.

\begin{algorithm}[tp]
\caption{Secure $\mathsf{LayerNorm}$ Protocol $\Pi_{\mathsf{LayerNorm}}$}\label{protocol:layernorm}
\begin{algorithmic}[1]
\REQUIRE
$P_i$ holds the 2-out-of-3 replicate secret share $\share{\x}_i$ for $i\in \{0,1,2\}$, and $\x$ is a vector of size $n$. 
\ENSURE
$P_i$ gets the 2-out-of-3 replicate secret share $\share{\y}_i$ for $i\in \{0,1,2\}$, where $\y=\mathsf{LayerNorm}(\x)$.

\STATE $P_0$, $P_1$, and $P_2$ compute $\share{\mu} = \frac{1}{n}\cdot \sum_{i=1}^n\share{\x[i]}$ and $\share{\sigma} = \sum_{i=1}^n \Pi_{\mathsf{Square}}(\share{\x} - \share{\mu})[i]$.

\STATE Parties jointly compute $\share{\sigma^{-1/2}} = \Pi_{\mathsf{rSqrt}}(\share{\sigma})$.

\STATE Parties jointly compute $\share{\mathbf{c}} = \Pi_{\mathsf{Mul}}((\share{\x} - \share{\mu}), \share{\sigma^{-1/2}})$

\RETURN $\share{\y} = \Pi_{\mathsf{Mul}}(\share{\gamma}, \share{\mathbf{c}}) + \share{\beta}$.

\end{algorithmic}
\end{algorithm}
\section{Algorithm}
\label{sec:algorithm}

We discuss the algorithm used by ADOPT in detail. Section~\ref{sub:main} discusses the top-level function, used to process queries. Section~\ref{sub:join} introduces ADOPT's parallel anytime join algorithm with worst-case optimality guarantees. Section~\ref{sub:cubes} discusses the mechanism by which ADOPT avoids redundant work across different attribute orders. Section~\ref{sub:rl} describes how ADOPT selects attribute orders via reinforcement learning. Finally, Section~\ref{sub:estimation} describes the reward metric used to guide the learning algorithm.

\subsection{Main Function}
\label{sub:main}

\begin{algorithm}[t!]
\caption{Main function of ADOPT, processing queries.\label{alg:main}}
\renewcommand{\algorithmiccomment}[1]{// #1}
\begin{small}
\begin{algorithmic}[1]
\State \textbf{Input:} Query $q$, number of threads $n$, per-episode budget $b$
\State \textbf{Output:} Query result
\Function{ADOPT}{$q, n, b$}
\State \Comment{Filter input tables via unary predicates}
\State $\{R_1,\ldots,R_m\}\gets$\Call{Prep.UnaryFilter}{$q$}
\State \Comment{Initialize join result set}
\State $R\gets\emptyset$
\State \Comment{Initialize reinforcement learning}
\State \Call{RL.Init}{$q$}
\State \Comment{Initialize constraint store}
\State \Call{TM.Init}{$q,n$}
\State \Comment{Iterate until result is complete}
\While {$\lnot$ \Call{TM.Finished}{}}
\State \Comment{Select attribute order via UCT algorithm}
\State $o \gets$ \Call{RL.Select}{}
\State \Comment{Use order for limited join steps}
\State $reward\gets$\Call{AnytimeWCOJ}{$q,o,n,b,R$}
\State \Comment{Update UCT statistics with reward}
\State \Call{RL.Update}{$o,reward$}
\EndWhile
\State \Comment{\revision{Return result after post-processing}}
\State \Return{\revision{\textproc{Post}$(q,R)$}}
\EndFunction
\end{algorithmic}
\end{small}
\end{algorithm}

ADOPT uses Algorithm~\ref{alg:main} to process \revision{simple SPJAG queries (i.e., without sub-queries). In addition to the query,} the algorithm also takes as input a number of data processing threads and a number of computational steps spent to evaluate a selected attribute order.

First, ADOPT filters the tables with the unary predicates \revision{(Line~5)}. \revision{ADOPT supports hash indexes on single columns and uses them, if available, to retrieve rows satisfying unary equality predicates. Without indexes, it scans and filters data, exploiting multi-threading. After that,} the only remaining predicates are then join predicates \revision{(including equality and other join predicates)}. Next, the algorithm initializes the set of join result tuples, the reinforcement learning algorithm by specifying the search space of attribute orders (which depends on the query), and the task manager with the input query and the number of processing threads \revision{(Lines~6 to 11)}. 
Internally, the task manager initializes the hypercube representing the total amount of work for each thread. More precisely, it divides the cube, representing the Cartesian product of all join attribute ranges, into equal shares for each thread. 

The task manager keeps track of cubes processed by the worker threads. Hence, query processing finishes once all processed cubes, in aggregate, cover the full input space. Iterations continue \revision{(Lines~13 to 20)} until that termination condition is satisfied. In each iteration, ADOPT first selects an attribute order via reinforcement learning \revision{(Line~15)}. Then, it executes that order, in parallel, for a fixed number of steps \revision{(Line~17)}. By executing the attribute order, \revision{the result set ($R$) may get updated. Note that $R$ only contains complete result tuples (mapping each attribute to a value) or partial values for aggregates. However, it does not contain any intermediate result tuples. Besides updating results}, executing an attribute order yields reward values, representing execution progress per time unit. Those reward values are used to update statistics \revision{(Line~19)}, maintained internally by the reinforcement learning optimizer, to guide attribute order selections in future iterations. Once \revision{the join finishes}, the algorithm \revision{performs post-processing (e.g., calculating per-group aggregates for group-by queries, based on join results in $R$) and returns the result (Line~22).}

%result tuples may get added into the result set ($R$). \revision{Note that only complete result tuples (i.e., combinations of values for all attributes) are added to that set but no intermediate result tuples.} 
\subsection{Anytime Join Algorithm}
\label{sub:join}

\begin{algorithm}[t!]
\caption{Parallel anytime version of worst-case optimal join algorithm.\label{alg:anytimeLFTJ}}
\renewcommand{\algorithmiccomment}[1]{// #1}
\begin{small}
\begin{algorithmic}[1]
\State \textbf{Input:} Query $q$, attribute order $o$, number of threads $n$, per-episode budget $b$, Result set $R$
\State \textbf{Output:} Reward $r$
\Function{AnytimeWCOJ}{$q,o,n,b,R$}
\State \Comment{Initialize accumulated reward}
\State $r\gets 0$
\State \Comment{Execute in parallel for all threads}
\For{$1\leq t\leq n$ in parallel}
\State \Comment{Initialize remaining cost budget}
\State $l_t\gets b$
\State \Comment{Iterate until per-episode budget spent}
\While{$l_t>0$}
\State \Comment{Retrieve unprocessed target cube}
\State $c_t\gets$\Call{TM.Retrieve}{}
\State \Comment{Process cube until timeout, add results}
\State $\langle P_t,s_t\rangle\gets$\Call{JoinOneCube}{$q,l_t,o,c_t,R$}
\State \Comment{Update constraints via processed cube}
\State \Call{TM.Remove}{$c_t,P_t$}
\State \Comment{Update accumulated reward (see Section~\ref{sub:estimation})}
\State $r\gets r+Reward(P_t,q)$
\State \Comment{Update remaining budget}
\State $l_t\gets l_t-s_t$
\EndWhile
\EndFor
\State \Comment{Return accumulated reward}
\State \Return{$r$}
\EndFunction
\end{algorithmic}
\end{small}
\end{algorithm}

Algorithm~\ref{alg:anytimeLFTJ} is the (worst-case optimal) join algorithm, used to execute a given attribute order for a fixed number of steps. Execution proceeds in parallel: different worker threads operate on non-overlapping cubes. Each worker thread iterates the following steps until its computational budget is depleted \revision{(Lines~11 to 22)}. First, it retrieves an unprocessed cube, the target cube, from the task manager \revision{(Line~13)}. Then, it uses a sub-function (an anytime version of the LFTJ) to process the retrieved target cube \revision{(Line~15)}. In practice, it is often not possible to process the entire target cube under the remaining computation budget. Hence, the result of the triejoin invocation (Function~\textproc{JoinOneCube}) reports the set of cubes, contained within the target cube, that were successfully processed. In addition, it returns the number of computation steps spent. The task manager is notified of successfully processed cubes which will be excluded from further consideration \revision{(Line~17)}. Also, a reward value is calculated that represents progress towards generating a full join result \revision{(Line~19)}. We postpone a detailed discussion of the reward function to Section~\ref{sub:rl}. Finally, Algorithm~\ref{alg:anytimeLFTJ} returns the reward value, accumulated over all threads and iterations \revision{(Line~25)}.


% \begin{algorithm}[t!]
% \caption{Worst-case optimal join algorithm with timeout, joining a single cube.\label{alg:joinOneCube}}
% \renewcommand{\algorithmiccomment}[1]{// #1}
% \begin{small}
% \begin{algorithmic}[1]
% \State \textbf{Input:} Query $q$, remaining budget $b$, attribute order $o$, cube to process $c$, result set $R$, attribute counter $a$
% \State \textbf{Output:} Processed cube $p$, computational steps performed $s$
% \Function{JoinOneCube}{$q,b,o,c,R,a$}
% \If{$a\geq|q.A|$} \Comment{Check for completed result tuples}
% \State Insert tuple with current attribute values into $R$
% \Else
% \State \Comment{Initialize value iterator (do not evaluate it!)}
% \State $V\gets$ iterator over matching values for $o_a$ in $[c.l_{o_{a}},c.u_{o_{a}}]$
% \State \Comment{Iterate over values until timeout}
% \For{$v\in V$}
% \State \Comment{Select values for remaining attributes}
% \State \Call{JoinOneCube}{$q,l,o,c,R,a+1$}
% \State \Comment{Check for timeouts}
% \If{Total computational steps $>b$}
% \State \textbf{Break}
% \EndIf
% \EndFor
% \EndIf
% \State \Comment{Return processing statistics only for top-level instance}
% \If{$a=0$}
% \State $s\gets$ Total number of computational steps performed
% \State $P\gets\emptyset$
% \For{$0\leq a< |q.A|$}
% \State $p_{1\leq i<a}\gets[v_i,v_i]$ s.t.\ $v_i$ is current value for attribute $o_i$
% \State $p_{a}\gets[c.l_{o_{a}},v_a)$ s.t.\ $v_a$ is current value for attribute $o_a$
% \State $p_{a<i}\gets[c.l_{o_{i}},c.u_{o_{i}}]$
% \State $P\gets P\cup\{p\}$
% \EndFor
% \State \Return{$\langle P,s\rangle$}
% \Else
% \State \Return{$\langle -,-\rangle$}
% \EndIf
% \EndFunction
% \end{algorithmic}
% \end{small}
% \end{algorithm}

\begin{algorithm}[t!]
\caption{Worst-case optimal join algorithm with timeout, joining a single cube.\label{alg:joinOneCube}}
\renewcommand{\algorithmiccomment}[1]{// #1}
\begin{small}
\begin{algorithmic}[1]
\State \textbf{Input:} Query $q$, remaining budget $b$, attribute order $o$, target cube to process $c$, result set $R$, attribute counter $a$, value mappings $M$
\State \textbf{Effect:} Iterates over attribute values and possibly adds results to $R$
\Procedure{JoinOneCubeRec}{$q,b,o,c,R,a,M$}
\If{$a\geq|q.A|$} \Comment{Check for completed result tuples}
\State Insert tuple with current attribute values $M$ into $R$
\Else
\State \Comment{Initialize value iterator (do not evaluate it!)}
\State $V\gets$ iterator over values for $o_a$ in $[c.l_{o_{a}},c.u_{o_{a}}]$ \revision{that satisfy}
\Statex $\quad\quad\quad\quad$ \revision{all applicable join predicates in $q$.}
\State \Comment{Iterate over values until timeout}
\For{$v\in V$}
\State \Comment{Select values for remaining attributes}
\State \Call{JoinOneCubeRec}{$q,l,o,c,R,a+1,M\cup\{\langle o_a,v\rangle\}$}
\State \Comment{Check for timeouts}
\If{Total computational steps $>b$}
\State \textbf{Break}
\EndIf
\EndFor
\EndIf
\EndProcedure
\vspace{0.25cm}
\State \textbf{Input:} Query $q$, remaining budget $b$, attribute order $o$, target cube to process $c$, result set $R$
\State \textbf{Output:} Processed cube $p$, computational steps performed $s$
\Function{JoinOneCube}{$q,b,o,c,R$}
\State \Comment{Resume join for fixed number of steps}
\State \Call{JoinOneCubeRec}{$q,b,o,c,R,0,\emptyset$}
\State \Comment{Retrieve state from \textproc{JoinOneCubeRec} invocation}
\State $s\gets$ Number of computational steps spent
\State $v\gets$ Vector s.t.\ $v_a$ is last value considered for attribute $o_a$
\State \Comment{Calculate processed cubes}
\State $P\gets\emptyset$
\For{$0\leq a< |q.A|$}
\State Create new cube $p$ s.t.\
\State $\quad$ $\forall i<a:p_i=[v_i,v_i];$
\State $\quad$ $\quad$ $p_a=[c.l_{o_a},v_a);$
\State $\quad$ $\quad$ $\quad$ $\forall a<i:p_i=[c.l_{o_i},c.u_{0_i}]$
\State $P\gets P\cup\{p\}$
\EndFor
\State \Return{$\langle P,s\rangle$}
\EndFunction
\end{algorithmic}
\end{small}
\end{algorithm}

Algorithm~\ref{alg:joinOneCube} describes the sub-function, used to process a single cube, at a high level of abstraction. The actual join is performed by Procedure~\textproc{JoinOneCubeRec}. This procedure is based on the leapfrog triejoin~\cite{DBLP:conf/icdt/Veldhuizen14}, a classical, worst-case optimal join algorithm\footnote{\revision{A detailed example of the LFTJ execution is given in Appendix \ref{sec:illustrating_LFTJ}.}}. For conciseness, the pseudo-code describes the algorithm as a recursive function (whereas the actual implementation does not use recursion). The input to the algorithm is the join query, the remaining computational budget, an attribute order, a target cube to process, the result set, and the index of the current attribute. The algorithm considers query attributes sequentially, in the given attribute order. The attribute index marks the currently considered attribute. Once the attribute index reaches the total number of attributes (represented as $q.A$), the algorithm has selected one value for each attribute. Furthermore, at that point, it is clear that the combination of attribute values satisfies all applicable join conditions. Hence, the algorithm adds the corresponding result tuple into the result set \revision{(Line~5)}. \revision{As a variant (not shown in Algorithm~\ref{alg:joinOneCube}), for queries with simple aggregates without grouping, ADOPT does not store result tuples but merely updates partial aggregate values for each aggregate.} If the attribute index is below the total number of attributes, the algorithm iterates over values for that attribute (i.e., attribute $o_a$ where $o$ is the order and $a$ the attribute index) \revision{in the loop from Line~10 to 17}. 

In Line~8, Algorithm~\ref{alg:joinOneCube} creates an iterator over values for the current attribute that \revision{satisfy all \textit{applicable} join predicates and are within the target cube,} i.e., values contained in the interval $[c.l_{o_a},c.u_{o_a}]$ for attribute number $a$ within order $o$ ($c.l$ and $c.u$ designate vectors, indexed by attribute, that represent lower and upper target cube bounds respectively). The algorithm does not assemble the full set of matching values before iterating (as that would create significant overheads when switching attribute orders before being able to try all collected values). Instead, Line~8 is meant to represent the initialization of data structures that allow iterating over matching values efficiently. \revision{Join predicates are applicable if, beyond the current attribute $o_a$, they only refer to attributes whose values have been fixed previously (i.e., a corresponding value assignment is contained in $M$). For equality join predicates, ADOPT uses the same mechanism as LFTJ~\cite{DBLP:conf/icdt/Veldhuizen14} to efficiently iterate over satisfying values. This mechanism is described in detail in Appendix~\ref{sec:illustrating_LFTJ}. It is based on data structures that support fast seek operations on query relations. Whenever required data structures are not available, ADOPT dynamically creates them at run time. For base relations, but not for relations filtered via unary predicates, ADOPT caches and reuses those data structures across queries.}

% As those data structures depend on the order in which attributes are considered, ADOPT dynamically creates them if required. At the same time, ADOPT caches  

% For all other join predicates, ADOPT simply evaluates the predicate and skips values where it evaluates to false.} It should be well understood that the algorithm does not assemble the full set of matching values before iterating (as that would create significant overheads when switching attribute orders before being able to try all collected values). Instead, Line~8 is meant to represent the initialization of data structures that allow iterating over matching values efficiently.


%appear in all input relations (as indicated by the expression ``matching values''). \revision{At the same time, values are filtered to the ones that satisfy all predicates of query $q$ that can already be evaluated, given the } It focuses only on attribute values within the target cube, i.e.\ values contained in the interval $[c.l_{o_a},c.u_{o_a}]$ for attribute number $a$ within order $o$ ($c.l$ and $c.u$ designate vectors, indexed by attribute, that represent lower and upper target cube bounds respectively). It should be well understood that the algorithm does not assemble the full set of matching values before iterating (as that would create significant overheads when switching attribute orders before being able to try all collected values). Instead, Line~8 is meant to represent the initialization of data structures that allow iterating over matching values efficiently. More precisely, those data structures allow ADOPT to efficiently intersect values from different relations for the same join attribute. We refer to the original publication for a detailed discussion~\cite{DBLP:conf/icdt/Veldhuizen14}. Note that ADOPT, instead of creating all potentially required data structures a-priori (which depend on the attribute order), it creates them on-demand, only if required for executing a new order.

Join processing via Procedure~\textproc{JoinOneCube} terminates once the computational budget is depleted \revision{(check in Line~14)}, or if the current cube is entirely processed. Function~\textproc{JoinOneCube} retrieves the number of computational steps, spent during join processing, as well as the last selected value for each attribute. It uses the latter to calculate the set of processed cubes (to be removed from the set of unprocessed cubes). Procedure~\textproc{JoinOneCubeRec} does not advance from one value of an attribute to the next, unless all value combinations for the remaining attributes have been fully considered. Hence, if value range  $c.l_{o_{a}}$ to $v_a$ was covered for the current attribute $a$, the cube representing processed value combinations reaches the full cube dimensions for all attributes that appear later than $a$ in the order $o$, and is fixed to the currently selected value for all attributes appearing before $a$ in $o$. Note that the pseudo-code uses a shortcut to assign both cube bounds at once (e.g., $p_i=[v_i,v_i]$ is equivalent to $[p.l_i,p.u_i]=[v_i,v_i]$) \revision{in Lines~32 to 34}.

\tikzstyle{cube}=[draw=black, thick]

% Figure environment removed

\begin{example}
\rm
Figure~\ref{fig:cubes} illustrates the containment relationships between different cubes when processing a query with two attributes. Processed cubes are contained within target cubes and target cubes are contained within the entire query cube. The figure represents target cubes that were processed, in different episodes, according to both possible attribute orders. The first one (left) was processed using order $A_1,A_2$. Hence, values for the first attribute change only after trying all values for the second attribute. Therefore, processed cubes fill the target cube ``column by column''. The other target was processed using the order $A_2,A_1$. Hence, processed cubes fill the target cube ``row by row''.
\end{example}

% The cube representing the scope of a single thread is located within the overall cube representing the entire query. The target cube is contained within the thread scope and represents the (maximal) goal for one invocation of Algorithm~\ref{alg:joinOneCube}. Processed cubes are contained within the target cube and may not entirely cover it, in case of a timeout. The figure indicates that a timeout occurred while processing the target cube. Also, it indicates that Attribute~1 comes, indeed, first in the processed attribute order. Hence, the cube associated with the value range considered for the first attribute covers the entire value range (of the target cube) for the second attribute.
\subsection{Avoiding Redundant Work}
\label{sub:cubes}

\begin{algorithm}[t!]
\caption{Managing cubes representing unprocessed join input.\label{alg:cubes}}
\renewcommand{\algorithmiccomment}[1]{// #1}
\begin{small}
\begin{algorithmic}[1]
\State $U\gets\emptyset$ \Comment{Global variable representing unprocessed cubes}
\vspace{0.25cm}
\State \textbf{Input:} Query $q$, number of threads $n$.
\State \textbf{Effect:} Initialize set of unprocessed cubes.
\Procedure{TM.Init}{$q, n$}
\State $A\gets$ attributes that appear in $q$ in equality join conditions
\State $[l_a,u_a]\gets$ attribute value ranges for all attributes $a\in A$
\State \Comment{Identify attribute with largest value domain}
\State $a^*\gets\arg\max_{a\in A}(u_a-l_a)$
\State \Comment{Use full value range for all but that attribute}
\State $f\gets\times_{a\in A:a\neq a^*}[l_a,u_a]$
\State \Comment{Divide largest value domain into per-thread ranges}
\State $\delta\gets (u_{a^*}-l_{a^*})/n$
\State \Comment{Form one unprocessed cube per thread}
\State $U\gets\{f\times [l_{a^*}+i\cdot\delta,l_{a^*}+(i+1)\cdot\delta|0\leq i<n]\}$
\EndProcedure
\vspace{0.25cm}
\State \textbf{Output:} Returns an unprocessed hypercube.
\Function{TM.Retrieve}{}
\State \Return Randomly selected cube from $U$
\EndFunction
\vspace{0.25cm}
\State \textbf{Input:} Target cube $c$ to subtract, processed cube set $P$.
\State \textbf{Effect:} Updates set of unprocessed cubes.
\Procedure{TM.Remove}{$c, P$}
\State \Comment{Subtract target cube from unprocessed cubes}
\State $U \gets U \setminus c$
\State \Comment{Add complement of processed cubes as unprocessed}
\For{$p \in P$}
\State \Comment{Get dimensions where $p$ fully covers $c$}
\State $F\gets$ indexes $i$ s.t.\ $p.l_i=c.l_i$ and $p.u_i=c.u_i$
\State \Comment{Get dimensions where $p$'s bounds collapse}
\State $S\gets$ indexes $i$ s.t.\ $p.l_i=p.u_i$
\State \Comment{Get single remaining dimension}
\State $d\gets$ single remaining dimension not in $F$ or $S$
\State Create new cube $u$ s.t.\
\State $\quad$ $u_{d}=(p.u_d,c.u_d]$; $\forall f\in F:u_f=p_f$; $\forall s\in S:u_s=p_s$
\State \Comment{Add newly created cube to unprocessed cubes}
\If{$u$ is not empty}
\State $U\gets U\cup\{u\}$
\EndIf
\EndFor
\EndProcedure
\vspace{0.25cm}
\State \textbf{Output:} True iff no unprocessed cubes are left.
\Function{TM.Finished}{}
\State \Return{\textbf{true} iff $U=\emptyset$}
\EndFunction
\end{algorithmic}
\end{small}
\end{algorithm}

ADOPT changes between different attribute orders over the course of query processing. This creates the risk of redundant work across different orders. ADOPT avoids redundant work by keeping track of cubes, in the space of join attribute values, that have not been considered yet. More precisely, ADOPT keeps track, at any point in time, of remaining, i.e.\ unprocessed, cubes. Whenever one of the processing threads requests a new cube to work on, ADOPT returns an unprocessed cube, thereby avoiding redundant work.



Algorithm~\ref{alg:cubes} gives functions used to manipulate cubes. At the beginning (Procedure~\textproc{TM.Init}), it initializes the set of unprocessed cubes to cover the entire attribute space. To do so, ADOPT first retrieves all join attributes \revision{(Line~5)}, then their value ranges \revision{(Line~6)}. Forming one single cube (i.e., the Cartesian product of all value ranges) diminishes chances for parallelization, at least at the start of query processing. Hence, ADOPT divides the attribute value space into equal-sized cubes with one cube per thread \revision{(Lines~7 to 14)}. To do so, it uses the attribute with maximal value domain, dividing its range equally across threads \revision{(Line~12)}. \revision{Note that, as discussed in the following, threads are not restricted to processing cubes initially assigned to them over the entire course of query evaluation. Instead, at the end of each episode, unprocessed parts of cubes assigned to a specific thread may get re-assigned to other threads.}

Whenever a worker threads requests a cube to work on \revision{(Line~13 in Algorithm~\ref{alg:anytimeLFTJ})}, a randomly selected cube from the set of unprocessed cubes is returned \revision{(Line~18 in Algorithm~\ref{alg:cubes})}. Note that the pseudo-code is slightly simplified, compared to the implementation, by omitting checks used to avoid concurrent changes to the set of unprocessed cubes (by multiple threads). 

Whenever a worker threads finished processing, it registers a set of cubes that was processed. It calls Procedure~\textproc{TM.Remove} to update the set of unprocessed cubes. This function takes two parameters, representing the set of processed cubes as well as the target cube, as input. All processed cubes are contained within the target cube and have a special structure, explained in the following. As a first step, ADOPT removes the target cube from the set of unprocessed cubes \revision{in Line24} (the target cube was selected by an invocation of the \textproc{TM.Retrieve} function and is therefore contained in the set $U$). If the set of processed cubes, in aggregate, do not cover the target cube (in general, that is the case), the set of unprocessed cubes is now missing all cubes contained in the target cube but not covered by the processed cubes. Hence, ADOPT adds more unprocessed cubes to reflect the difference.

Each processed cube has a special form, due to the structure of the join algorithm generating it (Lines~23 to 28 in Algorithm~\ref{alg:joinOneCube}). All processed cubes are generated according to the same attribute order and based on the same, final values selected for each attribute. Consider one single processed cube, using the selected attribute values $v_s$ for a prefix $S$ of the attribute order, the range of values up to the selected value $v_d$ for a single attribute $d$, and the full target cube range for the remaining attributes $F$. Clearly, given the selected values for attributes $S$, none of the values greater than $v_d$ for attribute $d$ has been considered by the join algorithm (instead, such value combinations would have been considered later by the join algorithm). Hence, the corresponding cube is added to the set of unprocessed cubes \revision{(Line~37)}. Also note that these unprocessed cubes cannot overlap (as, for each pair of unprocessed cubes, there is at least one attribute $a$ for which one cube fixes a value $v_a$, the other cube covers only values greater than $v_a$). This preserves the invariant that elements of $U$, representing unprocessed cubes, do not overlap. It also means that work done by different threads does not overlap. The processing finishes (Procedure~\textproc{TM.Finished}) whenever no unprocessed cubes are left.

\tikzstyle{values}=[only marks, mark=x, draw=black, mark size=6, ultra thick]
\tikzstyle{processedcube}=[fill=blue!20]
\tikzstyle{unprocessed}=[ultra thick, draw=red]

% Figure environment removed

% % Figure environment removed

\begin{example}
\label{ex:cube-removal}
\rm
Figure~\ref{fig:removal} illustrates the processing of a target cube $([1,5], [1,5], [1,5])$ for an attribute order ($A_0$,$A_1$,$A_2$). In each sub-plot, the x-axis represents attributes while the y-axis represents attribute values.  Assume the timeout for this episode occurs after considering the values $(5,3,4)$ (marked by X). This means that we managed to process the following sub-cubes, left: $([1-4], [1-5], [1-5])$, middle: $(5, [1-2], [1-5])$, right: $(5, 3, [1-4])$. We infer the remaining unprocessed sub-cubes that complement these processed sub-cubes with respect to the target cube, left: $(5, [1-5], [1-5])$, middle: $(5, [4-5], [1-5])$, right:$(5, 3, 5)$.

\nop{
A timeout occurs during join processing after considering values (5,3,4) for the three attributes (in attribute order, selected values are marked by X). 
Each of the three sub-plots, from left to right, represents one of the processed cubes generated in that order via the loop from Lines~30 to 36 in Algorithm~\ref{alg:joinOneCube}. The extent of processed cubes, for each attribute, is marked up in blue. E.g., for the left-most plot, the processed cube is defined as $[1,4]\times[1,5]\times[1,5]$. For each processed cube, an unprocessed cube is added in Lines~26 to 39 in Algorithm~\ref{alg:cubes}, such that the unprocessed cubes together cover all unprocessed values. The value ranges of unprocessed cubes are marked as red rectangles. 
}

% For instance, in the left-most plot, the associated, unprocessed cubes is empty.
\end{example}

% \begin{example}
% Figure~\ref{fig:removal} illustrates removal of processed cubes. We consider a query with three attributes. In each sub-plot, the x-axis represents the attribute while the y-axis represents attribute value ranges. Hence, each sub-plot describes a cube which corresponds to the Cartesian product of the value ranges colored in blue. E.g., the plot in the left-upper corner represents the cube $[1,4]\times[1,5]\times[1,5]$.
% A timeout occurs during join processing after selecting values (5,3,4) for the three attributes (in attribute order). The upper row in Figure~\ref{fig:removal} shows the three processed cubes that result from this final state. From left to right, processed cubes are generated in that order via the loop from Lines~30 to 36 in Algorithm~\ref{alg:joinOneCube}. The cubes on the bottom represent the corresponding complement of unprocessed cubes. Note that the left-most cube is empty (it is therefore not added to the unprocessed cubes). The right-most unprocessed cube, for instance, captures, for fixed values in the first two attributes, the fact that value five for the last attribute was not considered yet.
% \end{example}
\subsection{Learning Attribute Orders}
\label{sub:rl}

ADOPT uses reinforcement learning to learn near-optimal attribute orders, over the course of a single query execution. At the beginning of each time slice, ADOPT selects an attribute order that maximizes the tradeoff between exploration and exploitation. It uses the Upper Confidence Bounds on Trees (UCT) algorithm~\cite{Kocsis2006} to choose an attribute order. This requires mapping the scenario (of attribute order selection) into a Markov-Decision Problem. Next, we discuss the algorithm as well as the problem model.

An episodic Markov Decision Process (MDP) is generally defined by a tuple $\langle s_0,S,A,T,R\rangle$ where $S$ is a set of states, $s_0\in S$ the initial state in each episode, $A$ a set of actions, and $T:S\times A\rightarrow S$ a transition function, linking states and action pairs to target states. Component $R$ represents a reward function, assigning states to a reward value. In our scenario, the transition function is deterministic while the reward function is probabilistic (i.e., states are associated with a probability distribution over possible rewards, rather than a constant reward that is achieved, every time the state is visited). The transition function is partial, meaning that certain actions are not available in certain states. Implicitly, we assume that all states without available actions are end states of an episode. After reaching and end state, the current episode ends and the next episode starts (from the initial state $s_0$ again). Given an MDP, the goal in reinforcement learning~\cite{Sutton2018} is to find a policy, describing behavior that results in maximal (expected) reward. In order to leverage reinforcement learning algorithms for our scenario, we must therefore map attribute order selection into the MDP formalism.

Our goal is to learn a policy that describes an attribute order. The policy generally recommends actions to take in a specific state. Here, we introduce one action for each query attribute. States are associated with attribute order prefixes (i.e., each state represents an order for a subset of attributes). To simplify the notation, we will refer to states by the prefix they represent, to actions by the attribute they correspond to. The transition function connects a first state $s_1$ to a second state $s_2$ via action $a$, if the second state can be reached by appending the attribute, represented by the action, to the prefix represented by the first state. More precisely, using the notation introduced before, the transition function links the state-action pair $\langle s_1,a\rangle$ to state $s_2=s_1\circ a$ (where $\circ$ represents concatenation). Each state represents a prefix of an attribute order in which each attribute appears at most once. Hence, the actions available in a state correspond to attributes that do not appear in the prefix represented by the state. This means that all states representing a complete attribute order are end states, implicitly. As a further restriction, we do not allow actions representing attributes that do not connect to any attributes in the prefix represented by the current state. This is similar to the heuristic of avoiding Cartesian product joins, used almost uniformly in traditional query optimizers. The reward function is set to zero for all states, except for end states. States of the latter category represent complete attribute orders. Upon reaching such a state, ADOPT executes the corresponding attribute order for a limited number of steps, measuring execution progress. The process by which execution process is measured is described in the following subsections. 


ADOPT applies the UCT algorithm to solve the resulting MDP. As the MDP represents the problem of attribute ordering, linking rewards to execution progress, solving the MDP (i.e., finding a policy with maximal expected reward) yields a near-optimal attribute order. The UCT algorithm represents the state space as a search tree. Nodes represent states while tree edges represent transitions. Tree nodes are associated with statistics, establishing confidence bounds on the average reward associated with the sub-tree rooted at that node. Confidence bounds are updated as new reward samples become available. In each episode, the UCT algorithm selects a path from the search tree root to one of the leaf nodes. At each step, the UCT algorithm selects the child node with maximal upper confidence bound (hence the name of the algorithm). This approach converges to optimal policies~\cite{Kocsis2006}. After selecting a path to a leaf and calculating the associated reward, the UCT algorithm updates confidence bounds for each node on that path.

\revision{ADOPT grows the UCT search tree gradually over the course of query execution. At the start of execution, the tree only contains the root node. Then, in each episode, the tree is expanded by at most one node. Which nodes are added depends on the selected attribute orders. Each attribute order corresponds to a sequence of states in the MDP (a state represents an attribute order, each state appending one attribute, compared to its predecessor). In the fully grown search tree, each state is associated with one node. If, for the currently selected attribute order, some of the states do not have associated nodes in the tree yet, ADOPT expands the tree by adding a node for the first such state. ADOPT uses the partial tree to select attribute orders as follows. Given a state for which all possible successor states have associated nodes in the tree (i.e., reward statistics are available), ADOPT uses the aforementioned principle and selects the attribute that maximizes the upper confidence bound on reward values. If some of the successor states do not have associated nodes yet, ADOPT transitions to a randomly selected state among them (which will create a corresponding node). As a special case, if no nodes are available for any of the successor states, ADOPT selects the next attribute with uniform random distribution.}

%In our scenario, it is crucial to avoid generating the entire search tree at once, as this may cause non-negligible overheads. The number of possible attribute orders for a given query can be very large. Instead, we use a UCT variant that builds the search tree gradually. Specifically, starting from a tree containing only the root node, the algorithm expands the search tree by at most one node per sample. If the algorithm reaches an MDP state that has no associated node in the search tree, it performs random transitions until reaching an end state. In our scenario, this means that the remaining attributes are selected in random order. The resulting reward value corresponds to a uniform random sample for rewards in the corresponding sub-tree. If the algorithm encounters states with no associated tree nodes, it creates the first ``missing'' nodes and adds it to the tree. As time progresses, the tree becomes most refined along paths associated with interesting attribute orders.

\tikzstyle{uctnode}=[draw, circle, shade, top color=gray!10, bottom color=gray!20, blur shadow={shadow blur steps=5}, minimum width=1cm]
\tikzstyle{episode}=[color=red]

% Figure environment removed

\begin{example}
\label{ex:expansion}
\rm
\revision{Given a query with three attributes (A, B, and C), assume that ADOPT selects the following attribute orders in the first episodes (some orders are selected in multiple episodes): ABC, BCA, CBA, ABC, ACB, CBA, CAB, CBA. Figure~\ref{fig:uct} shows the UCT search tree after those episodes. Nodes represent partial attribute orders and edges represent the addition of one attribute. Next to each node, in red, the figure shows the number of the episode in which the node was added. Initially (episode zero), the tree contains only the root node. In the first episode, ADOPT selects order ABC, adding a node for the first prefix (A) without corresponding node in the tree. Later, in episode four, ADOPT selects order ABC and, again, adds a node for the first prefix (AB) for which no node has been created. Once nodes are added, ADOPT starts collecting reward statistics for all attribute orders extending the corresponding prefix. These statistics are used to select attribute orders in future episodes.}
\end{example}

% \begin{algorithm}[t!]
% \caption{Reward Function.\label{alg:reward}}
% \renewcommand{\algorithmiccomment}[1]{// #1}
% \begin{small}
% \begin{algorithmic}[1]
% \State \textbf{Input:} Selected hypercube $c$, processed hypercube $p$
% \State \textbf{Output:} Reward $r$
% \Function{Reward}{$c, p$}
% % \State $c_i \gets c.u_i - c.l_i, \forall i = 1 \to n$
% % \State $\delta_i \gets p.u_i - p.l_i, \forall i = 1 \to n$
% \For{$i \gets 1$ to $m$}
%     \State $c_i \gets c.u_i - c.l_i$
%     \State $\delta_i \gets p.u_i - p.l_i$
% \EndFor
% \State $r \gets \sum_{1\leq i \leq m} (\delta_i / \Pi_{1 \leq k \leq m} (c_k) ) \frac{c.volume}{join\_space.volume}$
% % \State \Comment{Scale reward with the relative hypercube volume ratio}
% % \State $r \gets \frac{c.volume}{join\_space.volume}$ 
% \State \Return $r$
% \EndFunction
% \end{algorithmic}
% \end{small}
% \end{algorithm}


\subsection{Estimating Order Quality}
\label{sub:estimation}

%Those values serve as quality samples, judging the quality of an attribute order when processing one specific partition of the data. While quality estimates may vary for the same order, across different invocations, ADOPT converges to the order with maximal average quality over time. 



The reinforcement learning, described in Section~\ref{sub:rl}, is guided by reward values. Next, we discuss the definition of the reward function. Before that, we introduce an auxiliary function, measuring the volume of a cube as the product of range sizes over all dimensions:

\begin{equation}
    Volume(c)=\prod_{i}(c.u_i-c.l_i)
\end{equation}

With a slight abuse of notation, we  write $Volume(q)$ to denote the volume of the cube, spanned by all join attributes of a query $q$. 

In order to fully process a query, ADOPT must cover the cube representing the entire space of attribute value combinations. Hence, the more volume of that cube we cover per time unit, the faster query processing is. \revision{Even for a fixed attribute order, the volume processed per time unit may vary across different parts of the data (e.g., since the number of result tuples per volume varies). However, the fastest order processes most volume in average, averaging over the entire data set, and the UCT algorithm converges to decisions with highest average reward, even if the reward function is noisy~\cite{Kocsis2006}.} This implies that volume covered is a useful measure of progress. The reward function, presented next, follows that intuition. Given a set of processed cubes $P$ for query $q$, it uses the aggregate volume covered, scaled to the total volume to process (scaling ensures reward values between zero and one, consistent with the requirements of the UCT algorithm):

\begin{equation}
    Reward(P,\revision{q})=(\sum_{p\in P}Volume(p))/Volume(q)
\end{equation}

%\junxiong{However, $Reward(P, c)$ is not a precious measurement of the execution progress. During the execution, ADOPT splits cubers into smaller unbalanced cubes and small cubes are more likely to finish during the timeout. Thus, the progress of a target cube $c$ can not reflect the quality of selected join order. To tackle this issue, we scale this reward using the fraction between the volume of target cube and the volume of entire cube. The final reward $r$ is calculate as }

% \begin{equation}
%     r =  Reward(P,c) Volume(c) / Volume(Entire Cube)
% \end{equation}

% \junxiong{Directly using $Reward(P, c)$ as final reward $r$ to update the statistics of UCT leads  }
\subsection{Performance Indicators}\label{subsection:performance_indicators_definition}
We use three categories of indicators to analyze the performance of a scenario definition:

\textbf{Inefficiency Rate}: it is the difference in fuel spent between a traffic scenario in which the aircraft comply to DAA resolution advisories (RA), and the analogous scenario in which each vehicle follows its optimal path, without observation of traffic separation rules. In case of vertical deviations, a higher fuel rate is required for climb maneuvers, a lower fuel rate in descent maneuver, which increase the net total for the mission. The resulting value of this indicator is the average value over all traffic configurations in a scenario set. 

\textbf{Loss of Separation (LoS) Rate}: despite there being several ways of defining traffic separation, we examine just the simplest one, which is checking whether or not the vehicles are separated by at least a fixed \emph{minimum distance}. In order to include both DAIDALUS and ACAS sXu in the same tables, we use one separation distance from each one, respectively: 4,000 ft, which is the Horizontal Miss Distance, or HMD, used in DO-365B \cite{DO_365} to define the so-called \emph{Hazard Alert Zone} (HAZ), associated to the DAA Well-Clear (DWC) concept of separation; and 2,000 ft, used in DO-396 \cite{DO_396}, that defines the Loss of Well-Clear (LoWC) event in relation to large UAVs or manned aircraft. These indicators will denote the rate of scenarios, in a scenario set, where the distance between any aircraft pair fell below the afore mentioned threshold values. 

\textbf{Timeout Rate}: this indicates, in a scenario set, the rate of scenarios where any aircraft exceeded a maximum time without reaching its destination point. As pointed out in section \ref{section:scenario_definitions}, this phenomenon occurs because DAA Resolution Advisories (RAs) cause long chains of maneuvers that extend beyond the energy/fuel allowance of the vehicle, due to shortcomings in coordination. We use a time threshold of 1,000 seconds but, in practice, the timeout would be determined by the energy/fuel capacity of the vehicle.

\textbf{Scenario Computing Time}: the time needed to simulate a single scenario instance, in a single core of an Intel Xeon CPU, discounted the fact that multiple scenario instances can be run in parallel in a multi-core CPU. In our simulated scenarios, the DAA algorithm is called at least each 2 seconds, for each aircraft, but when the aircraft is in avoidance mode, that can happen more often. In the case of ACAS sXu, the requirement of receiving various messages to update a single track contribute to result in multiple calls per simulated second.


\subsection{Scenario Specifications and Labels}
In this study, a scenario specification is defined by features such as: the DAA algorithm used, the dimensionality (2-D or 3-D), if it uses extrinsic priorities or not, the target separation parameter, and possibly other features. The scenario labels used in this section encode these attributes:
\begin{itemize}
\item \texttt{dai\_ip\_2d\_4k}: DAIDALUS without extrinsic priorities, 2-D maneuvering and regular 4 kft Horizontal Miss Distance (HMD);
\item \texttt{dai\_ep\_2d\_4k}: similar to the above, with extrinsic priorities;
\item \texttt{sxu\_ip\_2d\_2k}: ACAS sXu with intrinsic priorities only, 2-D maneuvering and regular 2 kft LoWC threshold;
\item \texttt{sxu\_ep\_2d\_2k}: similar to the above, with extrinsic priorities;
\item \texttt{dai\_ep\_3d\_4k}: similar to \texttt{dai\_ep\_2d\_4k}, with 3-D maneuvering;
\item \texttt{dai\_ep\_2d\_2k}: similar to \texttt{dai\_ep\_2d\_4k}, with HMD reduced to 2 kft.
\end{itemize}

And there are other features and labels that will be mentioned below as needed.

\subsection{Performance Analysis}
The analysis of selected scenario specifications is summarized in table~\ref{table:performance_analysis}. The first notorious observation in this table is the effect of extrinsic priorities to decrease inefficiency. With regards to safety indicators, their effect is mixed, and we have to observe each case separately. In the case of DAIDALUS, priorities decreased the 2 kft LoS rate and, most drastically, the timeout rate, while increased the 4 kft LoS rate. In the case of ACAS sXu, priorities increased both LoS rate indicators, but decreased the timeout rate drastically. Based on our rule of thumb assessment, it can be said that DAIDALUS works better with extrinsic priorities, while ACAS sXu works better without them. We conjecture that the following reasons explain this fact: i) that DAIDALUS has more built-in symmetries than ACAS sXu; ii) that ACAS sXu has already built-in priority rules for multi-aircraft encounters, and extrinsic priorities may contradict with them.

\begin{table}[h]
    \caption{Summary of closed-loop performance indicators per scenario.}
    \label{table:performance_analysis}
    \begin{center}
    \begin{tabularx}{\columnwidth}{|p{0.20\columnwidth}|p{0.11\columnwidth}|p{0.09\columnwidth}|p{0.09\columnwidth}|p{0.1\columnwidth}|p{0.105\columnwidth}|}
        \hline
    Scenario spec. & Ineffici-ency rate & LoS rate 4~kft & LoS rate 2~kft & Timeout rate & Scenario comp. time (s)\\
    \hline
    \texttt{dai\_ip\_2d\_4k} & 9.71\% & 1.4E-2 & 6.5E-5 & 1.6E-2 & 6.5E-2\\
    \texttt{dai\_ep\_2d\_4k} & 4.83\% & 2.4E-2 & 4.1E-5 & 0 & 5.1E-2\\
    \texttt{sxu\_ip\_2d\_2k} & 20.7\% & 8.7E-1 & 4.1E-2 & 1.6E-3 & 7.8E+1\\
    \texttt{sxu\_ep\_2d\_2k} & 10.9\% & 9.0E-1 & 2.0E-1 & 1.8E-5 & 7.8E+1\\
    \texttt{dai\_ep\_3d\_4k} & 4.38\% & 8.9E-6 & 0 & 0 & 7.4E-2\\
    \texttt{dai\_ep\_2d\_2k} & 1.3\% & 9.0E-1 & 1.1E-1 & 0 & 3.9E-2\\
    \hline
    \end{tabularx}
    \end{center}
\end{table}

According to a line of reasoning, it would be expected, that, in the more efficient scenarios, the aircraft fly closer to each other and, therefore, there should be a higher probability of losing separation. But this is not the only principle at play, because, if the aircraft perform deviations with the least extra distance, while keeping separation, they stay less in the air and decrease the total number of conflicts. This becomes more understandable when we compare \texttt{dai\_ep\_2d\_4k} with \texttt{dai\_ep\_3d\_4k}, where the latter achieved a small advantage in efficiency, but a huge one in safety. \texttt{dai\_ep\_3d\_4k} is capable of shortening the total distances, but has a residual cost associated to vertical maneuvers, where the climb maneuvers spend fuel at higher rates. 

It cannot escape from observation that DAIDALUS performed much better than ACAS sXu in almost all indicators. In our opinion, it would be reasonable to expect that ACAS sXu would not excel in the LoS rates, especially that of 4 kft, because its first protection criterion is 2 kft, as a built-in feature. However, with smaller protection volumes, the deviations should be smaller and, by this reasoning, its expected inefficiency would be lower than that of DAIDALUS. But our results show otherwise when we compare the cases of DAIDALUS with those of ACAS sXu. The only case in which ACAS sXu obtained an advantage was for the LoS rates comparison between \texttt{sxu\_ip\_2d\_2k} and \texttt{dai\_ep\_2d\_2k}, which have the same separation target. In any case, the Los rate obtained for ACAS sXu meets the performance requirement of ASTM F3442 \cite{ASTM}, which uses the definition of LoWC Ratio (LR), which is the ratio between the LoS rate of 2 kft shown in table~\ref{table:performance_analysis}, with DAA active, and corresponding LoS rate with DAA inactive, the latter being 0.785 according  to our simulations. Thus, the resulting LR scores for ACAS sXu here are 0.052 and 0.253, respectively for the two ACAS sXu specs, which are well below the value of 0.4 from \cite{ASTM}, and consistent with the performance analysis of \cite{DO_396}. We conjecture that these scores would be lower in a future 3-D scenario spec of a ACAS sXu, by following the same improvement obtained with DAIDALUS. 

\subsection{Possible approximations to the closed-loop behavior}
Trying to alleviate the heavy computational load to simulate large numbers of different traffic configurations, in this multi-aircraft, closed-loop setup, we considered some approximated solutions, such as the use of Deep Neural Networks to emulate the ACAS Xu/sXu behavior, in the lines followed by \cite{Julian2018,Bak2022}. However, the existing solutions that we found available were developed for just one intruder aircraft, so they were not suitable for our study. Another possibility would be the exploitation of symmetry transformations \cite{Sibai2020}, however the history-dependent nature of the ACAS Xu/sXu algorithms, associated to the present closed-loop setup, make this possibility unpractical. Thus, we started exploring simpler ways of deducing closed-loop behavior without having to perform the full simulation of a scenario. So far, we tried to analyze correlations between measures of open-loop maneuvers and the closed loop performance. The features that we explored are:
\begin{itemize}
    \item \textbf{Distance flown until the end of the first deviation maneuver} ($\overline{M/D}$): we consider the total distance flown until a ``Clear-of-Conflict'' (CoC) event happens, that is, after one or more divergent maneuvers start in a scenario instance, we stop the scenario when the first divergent maneuver of any aircraft finishes and that individual aircraft is clear of conflict, the moment from which some decision must be made on how to continue the mission. We count the total number of maneuvers started, and divide it by the sum of the flown distances, across all scenario instances in an execution set associated to a scenario spec.
    \item \textbf{Average angle deviation maneuver} ($\overline{\alpha}$): using the same stopping rule of above, we account the angle difference between the heading angles of the aircraft at the beginning of the divergent maneuver and at the stopping moment. 
\end{itemize}
For each of these measures, we ran the 122,416 traffic configuration instances with the open-loop stopping rule. Here, all the scenario specifications are 2-dimensional, and we use abbreviated lables to achieve a better display in the graph legends. Namely, the scenario specifications in this subsection are defined as:
\begin{itemize}
    \item \texttt{D1}: DAIDALUS without extrinsic priorities and with deterministic sensor data;
    \item \texttt{D2}: DAIDALUS with extrinsic priorities and deterministic sensor data;
    \item \texttt{D3}: DAIDALUS with extrinsic priorities and Sensor Uncertainty Mitigation (SUM). This is a design feature \cite{Narkawicz2018} to mitigate uncertainty in sensor data, as for example, to determine the position of an intruder aircraft;
    \item \texttt{D4}: DAIDALUS with its Horizontal Miss Distance (HMD) set to 200 ft (the standard is 4,000 ft) and uncertain sensor data;
    \item \texttt{X1}: ACAS sXu without extrinsic priorities;
    \item \texttt{X2}: ACAS sXu with extrinsic priorities;
    \item \texttt{X3}: ACAS sXu with scenario downscaled to speed of 43 knots and cell radius of 1 km.
\end{itemize}
We used the values of $\overline{M/D}$ and $\overline{\alpha}$ obtained for each of the specs above as inputs to a linear regressor of inefficiency, as defined in section~\ref{subsection:performance_indicators_definition}, and generated a plot with the pairs of (true, predicted) values from this regression, as shown in fig.~\ref{fig:inefficiency_regression}. The trend line in the figure, which depicts the regressor, seems to represent a strong correlation, which is confirmed by the value of $R^2$. When considering each of the regression inputs separately, we obtain $R^2=0.73$ for $\overline{M/D}$ and $R^2=0.77$ for $\overline{\alpha}$, which show that they contribute with approximately equal predicting power. 
% Figure environment removed

We performed a similar analysis for the LoS indicators, but we found very little correlation, as $R^2=0.11$ for the 2~kft LoS indicator. Nevertheless, it can be concluded that these open-loop measurements are a good proxy for the closed-loop inefficiency, with the advantage that the predictor discounts the bias that may have been introduced by the closed-loop mission management system, which is not part of the DAA specification. A rough estimate for the computing time saved is of 68\% in the inefficiency case.      

\section{Experiments}
% \haizhou{Follow the same way of introduction as we did in Section2.}
% \noindent In this section, we will introduce datasets and experimental setups that we used. Then we evaluate our method, other self-supervised methods, and supervised methods under different distribution shifts (\ie, concept shifts and covariate shifts) under common settings (\ie, transductive, inductive settings). It has to note that we focus on node-level tasks (\eg, node classification) in this work. As for graph-level tasks, we leave it as our future work and some simple experiments can be found in Appendix~\ref{app:graph_classification}. 
In this section, we first introduce the experimental setup including datasets, training, and evaluation protocol in Section~\ref{sec:dataset}~and~\ref{sec:unsupervised}. 
% Next, we present our experimental setup and conduct extensive experiments to evaluate our method in Section~\ref{sec:unsupervised}. 
We then perform an ablation study to demonstrate the effectiveness of each proposed component in Section~\ref{sec:ablation}. 
Additionally, we analyze the impact of important hyper-parameters in Section~\ref{sec:sensitivity}. 
Subsequently, we integrate our method with various encoding models, showcasing the model-agnostic nature of our recipe in Section~\ref{sec:other_models}. 
Finally, we provide some qualitative results such as feature visualization in Section~\ref{sec:vis}.
It is important to note that we focus on node-level tasks (\eg, node classification) in this work. As for graph-level tasks, we leave it as our future work, while some simple experiments are also provided in Appendix~\ref{app:graph_classification}.

\subsection{Datasets}\label{sec:dataset}
There exist some benchmarks for evaluating graph out-of-distribution generalization~\cite{good,ji2022drugood,gds}. 
Among them, GOOD~\cite{good} is the most representative and comprehensive benchmark that curates more diverse graph datasets with diverse tasks, including single/multi-task graph classification, graph regression, and node classification involving more distribution shifts (\ie, concept shifts and covariate shifts). Hence in this work, we follow the evaluation protocol proposed in \cite{good}. Furthermore, we validate the effectiveness of our method in the datasets (\ie, Amazon-Photo, Elliptic) that are used in EERM~\cite{eerm}. The statistics and detailed introduction to these datasets can be found in Table~\ref{tab:dataset} and Appendix~\ref{app:datasets}.

\begin{table*}[htp]
\caption{The descriptions of datasets. ``Domain-Level'' means splitting by graphs, ``Time-Aware'' denotes splitting according to chronological order.``Word'' and ``Degree'' represent splitting according to word diversity and node degree respectively. ``Language'' means splitting by user language, suggesting the prediction should not be impacted by the language the user use. ``University'' denotes splitting according to the domain university, implying that the prediction of webpages should be based on word contents and link connections rather than university features. ``Color'' means that nodes are split according to node differences in covariate shift and color-label correlations in concept shift.}
\label{tab:dataset}
\centering
\begin{tabular}{cccccccc}
\toprule
Datasets     & Network Type        & \#Nodes & \#Edges & \#Attributes &\#Classes& Train/Val/Test Split     & Metric   \\
% Cora         & Artificial Transformation & 2,703   &         &              &         &                      & Accuracy \\
Amazon-Photo\footnotemark
             & Co-purchasing network      & 7,650   & 119,081   & 755          & 10      & Domain-Level         & Accuracy \\
Elliptic\footnotemark  
             & Bitcoin transactions       & 203,769 & 234,355   & 165          & 2       & Time-Aware           & F1-Score \\
GOOD-Cora    & Scientific publications    & 19,793  & 126,842   & 8,710         & 70      & Word/Degree          & Accuracy \\
% GOOD-Arxiv   & arXiv papers               & 169,343 & 2,315,598 & 128          & 40      & Time/Degree          & Accuracy \\
GOOD-Twitch  & Gamer network              & 34,120  & 892,346   & 128          & 2       & Language             & ROC-AUC  \\
GOOD-CBAS    & A BA-house graph           & 700     & 3,962     & 4             & 4       & Color                & Accuracy \\
GOOD-WebKB   & Webpage network            & 617     & 1,138     & 1,703         & 5       & University           & Accuracy \\
\bottomrule
\end{tabular}
\end{table*}
\footnotetext[5]{This dataset is adopted from~\cite{yang2016revisiting}. \cite{eerm} constructs ten graphs with different environment id’s for each graph.} 
\footnotetext[6]{The original is available on \hyperlink{https://www.kaggle.com/ellipticco/elliptic-data-set}{https://www.kaggle.com/ellipticco/elliptic-data-set}}

\subsection{Unsupervised Representation Learning}\label{sec:unsupervised}
\subsubsection{Transductive Setting}~\label{sec:trans}
% \noindent\textbf{Baselines.}\quad We conduct experiments with 12 baselines which consist of three categories: supervised methods and self-supervised generative methods, self-supervised contrastive methods. Specifically, we compare with three supervised baselines: empirical risk minimization~(ERM)~\cite{erm}, invariant risk minimization (IRM)~\cite{irm}, and a recent proposed graph OOD method dubbed EERM~\cite{eerm}. We also compare various unsupervised node-level representation learning methods: three self-supervised generative methods including GAE~\cite{gae}, VGAE~\cite{gae}, GraphMAE~\cite{gmae} and seven self-supervised contrastive methods: DGI~\cite{dgi}, MVGRL~\cite{mvgrl}, GRACE~\cite{grace}, RoSA~\cite{rosa}, BGRL~\cite{bgrl}, COSTA~\cite{costa}, SwAV~\cite{swav}. The descriptions of these methods can be found in Appendix~\ref{app:baselines}.
In this subsection, we focus on validating our proposed algorithm under the transductive setting, where the test nodes will participate in message passing~\cite{gilmer2017neural} during training following~\cite{good}. 

\noindent\textbf{Baselines.} We conduct experiments with 12 baselines from three categories: (i)~supervised methods, including empirical risk minimization~(\textbf{ERM})~\cite{erm}, invariant risk minimization (\textbf{IRM})~\cite{irm}, and a recent proposed graph OOD method \textbf{EERM}~\cite{eerm}; (ii)~self-supervised generative methods including Graph Autoencoder (\textbf{GAE})~\cite{gae}, Variational Graph Autoencoder (\textbf{VGAE})~\cite{gae}, Self-Supervised Masked Graph Autoencoders (\textbf{GraphMAE})~\cite{gmae}; (iii)~self-supervised contrastive methods including Deep Graph Infomax (\textbf{DGI})~\cite{dgi}, Contrastive Multi-View Representation Learning on Graphs (\textbf{MVGRL})~\cite{mvgrl}, Deep Graph Contrastive Representation Learning (\textbf{GRACE})~\cite{grace}, A Robust Self-Aligned Framework for Node-Node Graph Contrastive Learning (\textbf{RoSA})~\cite{rosa}, Bootstrapped Representation Learning on Graphs (\textbf{BGRL})~\cite{bgrl}, Covariance-Preserving Feature Augmentation for Graph Contrastive Learning (\textbf{COSTA})~\cite{costa}, Unsupervised Learning of Visual Features by Contrasting Cluster Assignments (\textbf{SwAV})~\cite{swav}. The detailed descriptions of these baselines can be found in Appendix~\ref{app:baselines}.

\noindent\textbf{Experimental setup.} We use the same graph encoder across different datasets for a fair comparison following~\cite{good}. We use grid search to find other hyper-parameters (\eg, learning rate, epochs) for different methods. For all experiments, we select the best checkpoints for ID and OOD tests according to results on ID and OOD validation sets following~\cite{good}, respectively. Experimental details and hyper-parameter selections are provided in Appendix~\ref{app:hyper}. For evaluating unsupervised methods, a linear classifier will be built on the frozen trained encoder after finishing pre-training. The reported results are the mean performance with standard deviation after 10 runs following~\cite{good}.

\noindent\textbf{Analysis.}\quad Based on the experimental results listed in Table~\ref{tab:trans_concept} and \ref{tab:trans_covariate}, we can draw the following conclusions: firstly, we find strong self-supervised methods (\eg, GRACE, BGRL, COSTA) are more robust to distribution shifts (concept shift in Table~\ref{tab:trans_concept} and covariate shift in Table~\ref{tab:trans_covariate}) compared to supervised methods. For instance, on GOOD-CBAS and GOOD-WebKB datasets, GRACE surpasses the best supervised method by large margins (over 6\% absolute improvement). Interestingly, we find the methods designed for OOD generalization (\ie, IRM) and graph OOD generalization (\ie, EERM) do not attain superior performance than the standard ERM on most of the datasets. For example, EERM shows superior OOD performance compared to ERM in only one experiment, and IRM outperforms ERM in four out of ten experiments across the conducted evaluations. This phenomenon is also observed in \cite{good,ahuja2020empirical,rosenfeld2021risks}, showcasing the challenge of achieving invariant prediction in non-Euclidean graph settings. 

Furthermore, our method surpasses other SOTA self-supervised methods on the OOD test set of all datasets by a considerable margin while achieving comparable performance in the in-distribution test set. For instance, on small datasets such as GOOD-CBAS and GOOD-WebKB, our method outperforms GRACE\footnote{MARIO is built up on GRACE according to our recipe. So, we make a comparison with GRACE here.} by over 2\% absolute accuracy on the OOD test set. On larger datasets such as GOOD-Cora and GOOD-Twitch, our method still outperforms other methods which shows its superiority. For instance, under covariate shift, MARIO surpasses other methods by over 7\% absolute accuracy on the GOOD-Twitch OOD test set. These statistics prove the effectiveness of our design.


\begin{table*}[htp]
\caption{Experimental results of all methods under concept shift. The bold font means the top-1 performance and the underline represents the second performance across the unsupervised methods. 'ID' represents in-distribution test performance and 'OOD' means out-of-distribution test performance. (OOM: out-of-memory on a GPU with 24GB memory)}
\label{tab:trans_concept}
\centering
\scalebox{0.95}{
\begin{tabular}{l|cc|cc|cc|cc|cc}
\toprule
\toprule
\multirow{3}{*}{concept shift} & \multicolumn{4}{c|}{GOOD-Cora}                   & \multicolumn{2}{c|}{GOOD-CBAS} & \multicolumn{2}{c|}{GOOD-Twitch} & \multicolumn{2}{c}{GOOD-WebKB} \\
                           & \multicolumn{2}{c}{word} & \multicolumn{2}{c|}{degree}& \multicolumn{2}{c|}{color}    & \multicolumn{2}{c|}{language}   & \multicolumn{2}{c}{university} \\
                           & ID         & OOD         & ID          & OOD          & ID            & OOD           & ID             & OOD            & ID            & OOD            \\
\midrule
ERM                        & 66.38±0.45 & 64.44±0.18  & 68.60±0.40  & 60.76±0.34   & 89.79±1.39    & 83.43±1.19    & 80.80±1.00     & 56.92±0.92     & 62.67±1.53    & 26.33±1.09     \\
IRM                        & 66.42±0.41 & 64.29±0.31  & 68.57±0.35  & 61.45±0.24   & 89.64±1.21    & 82.29±1.14    & 78.87±1.04     & 59.30±1.79     & 62.67±1.10    & 26.88±1.42     \\
EERM                       & 65.10±0.44 & 62.45±0.19  & 66.95±0.44  & 56.58±0.25   & 79.07±2.12    & 64.50±1.01    & OOM            & OOM            & 62.50±2.01    & 28.07±3.23      \\
\midrule
% Random-Init                & 37.53±1.74 & 32.12±1.24  & 37.82±1.71  & 27.74±1.14   &               &               &                &                & 60.33±2.21    & 27.07±1.70     \\
GAE                        & 60.65±0.89 & 58.00±0.55  & 62.59±1.11  & 53.44±0.80   & 75.28±1.36    & 68.07±2.05    & 81.25±0.81     & 51.51±1.05     & 62.17±3.34    & 25.78±1.85     \\
VGAE                       & 63.19±0.53 & 60.35±0.47  & 61.65±0.66  & 54.28±0.28   & 76.50±0.50    & 59.07±0.56    & 80.46±0.53     & 55.56±4.53     & 62.50±2.38    & 24.40±2.57     \\
GraphMAE                   & \underline{66.44±0.46} & \underline{64.87±0.30}  & 67.95±0.46  & 59.41±0.39   & 89.14±0.89    & 82.93±0.93    & 80.05±0.64     & 59.38±1.49     & 61.83±3.37    & 29.27±2.15     \\
DGI                        & 63.33±0.56 & 60.71±0.49  & 65.93±1.02  & 55.83±0.53   & 91.22±1.47    & 85.00±1.66    & 80.05±0.87     & 59.16±1.88     & 61.83±2.83    & 28.63±1.92      \\
MVGRL                      & OOM        & OOM         & OOM         & OOM          & 88.57±1.15    & 76.50±1.17    & OOM            & OOM            & 62.00±3.79    & 28.26±4.20     \\
GRACE                      & 65.61±0.61 & 63.92±0.44  & \textbf{68.59±0.35}  & 60.15±0.45   & 92.00±1.39    & 88.64±0.67    & \textbf{83.43±0.63}     & \underline{60.45±1.46}     & 64.00±3.43    & \underline{34.86±3.43}  \\
RoSA                       & 64.06±0.67 & 62.44±0.39  & 67.07±0.65  & 57.68±0.44   & 90.78±2.27    & 85.93±2.14    & 82.39±0.42     & 57.45±2.16     & 64.17±4.10    & 32.20±2.15     \\
BGRL                       & 65.18±0.43 & 63.43±0.45  & 66.83±0.80  & 59.63±0.38   & 92.36±1.16    & 87.14±1.60    & 82.52±0.60     & 55.48±1.48     & 63.67±2.33    & 31.47±3.43     \\
COSTA                      & 65.05±0.80 & 62.37±0.45  & 66.76±0.87  & 55.73±0.36   & \underline{93.50±2.62}    & \underline{89.29±3.11}    & 83.15±0.30 & 55.03±3.22     & 61.66±2.58    & 32.39±2.13 \\
% ArCL                       &            &             & 67.64±0.57  & 59.71±0.44   &               &               &                &                & 65.00±3.94    & 35.41±1.97 \\      
SwAV                       & 62.22±0.53 & 59.79±0.53  & 64.65±0.94  & 55.06±0.39   & 89.00±0.79    & 81.72±0.66    & \underline{83.32±0.15}     & 59.69±1.97     & \underline{65.17±3.76}    & 29.36±2.01    \\
\midrule
MARIO                       & \textbf{67.11±0.46} & \textbf{65.28±0.34}  & \underline{68.46±0.40}  & \textbf{61.30±0.28}   & \textbf{94.36±1.21}    & \textbf{91.28±1.10}    & 82.31±0.54     & \textbf{63.33±1.72}     & \textbf{65.67±2.81}    & \textbf{37.15±2.37}     \\
\bottomrule
\end{tabular}}
\end{table*}

\begin{table*}[htp]
\caption{Experimental results of all methods under covariate shift. The bold font means the top-1 performance and the underline represents the second performance across the unsupervised methods. 'ID' represents in-distribution test performance and 'OOD' means out-of-distribution test performance. (OOM: out-of-memory on a GPU with 24GB memory)}
\label{tab:trans_covariate}
\centering
\scalebox{0.95}{
\begin{tabular}{l|cc|cc|cc|cc|cc}
\toprule
\toprule
\multirow{3}{*}{covariate shift} & \multicolumn{4}{c|}{GOOD-Cora}                                   & \multicolumn{2}{c|}{GOOD-CBAS} & \multicolumn{2}{c|}{GOOD-Twitch} & \multicolumn{2}{c}{GOOD-WebKB} \\
                           & \multicolumn{2}{c}{word} & \multicolumn{2}{c|}{degree}& \multicolumn{2}{c|}{color}    & \multicolumn{2}{c|}{language}   & \multicolumn{2}{c}{university} \\
                           & ID         & OOD         & ID          & OOD          & ID            & OOD           & ID             & OOD            & ID            & OOD            \\
\midrule
ERM                        & 70.50±0.41 & 64.69±0.33  & 72.46±0.49  & 55.53±0.50   & 92.00±3.08    & 77.57±1.29    & 70.98±0.41     & 49.35±5.09     & 39.34±1.79    & 14.52±3.14   \\
IRM                        & 70.48±0.26 & 64.53±0.57  & 71.98±0.34  & 53.72±0.46   & 90.86±2.41    & 78.86±1.67    & 69.81±0.95     & 49.11±2.82     & 38.52±3.30    & 13.97±2.80     \\
EERM                       & OOM        & OOM         & OOM         & OOM          & 65.00±2.57    & 57.43±3.60    & OOM            & OOM            & 46.07±4.55    & 27.40±7.65     \\
\midrule
GAE                        & 56.63±0.79 & 48.93±0.93  & 66.30±0.88  & 34.01±0.87   & 73.00±2.16    & 60.86±3.01    & 67.24±1.23     & 47.65±2.49     & 45.08±6.32    & 28.02±6.29    \\
VGAE                       & 62.02±0.66 & 54.12±0.86  & 69.41±0.57  & 44.20±1.29   & 62.29±2.04    & 63.29±1.11    & 66.99±1.43     & \underline{50.48±4.58}     & 48.85±4.68    & 20.87±6.69     \\
GraphMAE                   & 68.14±0.43 & 64.00±0.33  & \textbf{73.36±0.56}  & 53.75±0.55   & 67.28±3.03    & 67.28±1.49    & 68.84±1.20     & 48.02±2.79     & 48.03±4.34    & 30.00±8.09     \\
DGI                        & 60.85±0.75 & 57.03±0.67  & 68.97±0.41  & 41.75±0.88   & 69.57±4.09    & 59.71±3.43    & 68.43±1.05     & 44.83±1.61     & 48.52±5.04    & 21.11±7.50     \\
MVGRL                      & OOM        & OOM         & OOM         & OOM          & 65.00±1.94    & 64.15±0.77    & OOM            & OOM           & \textbf{54.10±5.39}    & 16.59±6.51     \\
GRACE                      & \underline{68.77±0.33} & \underline{64.21±0.41}  & 72.69±0.34  & \underline{56.10±0.63}   & \underline{93.57±1.83}    & \underline{89.29±3.40}    & \underline{71.12±0.87} & 46.21±1.54 & 49.67±5.82    & 28.10±4.68    \\
RoSA                       & 68.19±0.56 & 62.48±0.61  & 71.04±0.62  & 52.72±0.79   & 84.71±4.14    &79.14±3.51     & 70.58±0.36     & 45.83±1.72     & 52.30±4.24    & \underline{34.24±7.92}     \\
BGRL                       & 67.23±0.43 & 61.33±0.36  & 72.11±0.39  & 49.15±0.73   & 89.00±2.56    & 79.86±3.29    & \textbf{71.43±0.53}     & 43.86±0.94     & 51.80±5.55    & 30.32±7.61    \\
COSTA                      & 65.28±0.60 & 60.33±0.53  & 70.65±0.62  & 54.03±0.28   & 92.29±1.59    & 82.71±2.74    & 69.29±1.37     & 49.07±2.13     & 50.49±3.01    & 29.84±4.75   \\
SwAV                       & 63.29±1.01 & 56.98±0.94  & 70.27±0.73  & 43.00±0.52   & 89.57±1.12    & 81.43±1.69    & 69.19±0.93     & 49.37±2.96     & 49.84±4.82    & 30.55±6.72   \\
\midrule
MARIO                       & \textbf{69.99±0.54} & \textbf{65.06±0.34}  & \underline{72.73±0.43}  & \textbf{57.73±0.45}  & \textbf{94.57±2.46}    & \textbf{91.00±2.48}     & 68.31±0.78 & \textbf{57.37±1.37}     & \underline{53.94±3.23}    & \textbf{35.24±4.98}   \\
\bottomrule
\end{tabular}}

\end{table*}

\subsubsection{Inductive Setting}
In this subsection, we conduct experiments under the inductive settings, where the test nodes are kept unseen during training. This setting is more suitable for domain generalization.
% But we think it is more convincing that conduct experiments under inductive settings which means test nodes are unseen during training. This setting is more appropriate for domain generalization.

\noindent\textbf{Baselines:} For GOOD-WebKB and GOOD-CBAS datasets, we adopt ERM, IRM, GraphMAE, and GRACE as our baselines. And for Amazon-Photo and Elliptic datasets, we select ERM, EERM, and GRACE as our baselines.

\noindent\textbf{Experimental setup:} For GOOD-WebKB and GOOD-CBAS datasets, we use the same model configuration in Section~\ref{sec:trans}.
% Besides, we add experiments on Amazon-Photo dataset~\cite{yang2016revisiting} and Elliptic~\cite{elliptic} dataset in this subsection. 
For Amazon-Photo dataset~\cite{yang2016revisiting} and Elliptic~\cite{elliptic} dataset, they consist of many snapshots (training data and testing data use different snapshots) which are naturally inductive. For Amazon-Photo dataset, we use 2-layer GCN~\cite{gcn} as the encoder and for elliptic dataset, we use 5-layer GraphSAGE~\cite{sage} as encoder following~\cite{eerm}.

% Figure environment removed

\noindent\textbf{Analysis:}
According to Figure~\ref{fig:amazon},\ref{fig:elliptic},\ref{fig:ind_con},\ref{fig:ind_cov}, we can draw following conclusions:
firstly, based on Figure~\ref{fig:amazon}, it is evident that our method outperforms other representative supervised and self-supervised methods on all test graphs (T1$\sim$T8). This superiority is reflected in the larger median value of our method compared to others. For instance, MARIO achieves over a 3\% absolute improvement compared to ERM in terms of the mean value of eight median values. Additionally, our method demonstrates higher stability across different random initializations, as indicated by the closer proximity of the first and third quartile values to the median value~(\eg, the difference of first and third quartile values of ERM, EERM, GRACE and MARIO are 4.2, 3.3, 6.7 and 1.0 on T8 respectively which indicates MARIO is much more stable than other methods). Furthermore, our method exhibits consistent performance across different graphs (\eg, The standard deviation of median values on T1$\sim$T8 for ERM, EERM, GRACE, and MARIO are 0.4, 1.1, 1.2, and 0.3, respectively.), indicating its robustness to environmental variations and its ability to extract invariant features: $g(G^e) \approx g(G^{e'})$ for all $e, e' \in \mathcal{E}^\text{train}$. In summary, our method showcases enhanced OOD generalization capabilities.
% $g(G^e)g(G^e^\prime)$ where $any e, e^\prime in \mathcal{E}^{train}$

Secondly, from the results presented in Figure~\ref{fig:elliptic}, we can observe that our method averagely harvests 10.9\% absolute improvement over GRACE and 12.5\% absolute improvement over EERM in terms of F1 scores on Elliptic dataset. This demonstrates the effectiveness of our method in handling distribution shifts and improving performance compared to existing approaches. It is worth noting that GRACE's performance worsens over time, indicating its inability to handle distribution shifts effectively. In contrast, our method consistently achieves better F1 scores, except for T9, which is caused by the dark market shutdown occurred after T7~\cite{elliptic}. The emergence of such an event introduces significant variations in data distributions, which subsequently results in performance degradation for all methods. Indeed, this event serves as an unpredictable external factor that introduces significant challenges for models trained on limited training data. The results indicate that the performance heavily depends on available training data. Nonetheless, our approach outperforms other methods even in such an extreme case. This highlights the effectiveness of our method in addressing distribution shifts and improving generalization performance.

Finally, based on the observations from Figure~\ref{fig:ind_con} and Figure~\ref{fig:ind_cov} MARIO demonstrates the best performances on both ID and OOD test sets for GOOD-WebKB and GOOD-CBAS datasets, under both concept shift and covariate shift. Notably, MARIO outperforms other methods by more than 3\% and 10\% absolute improvement on GOOD-WebKB and GOOD-CBAS, respectively, under covariate shift. We can draw similar conclusions as discussed in Section~\ref{sec:trans}. Even under the inductive setting, our method continues to demonstrate excellent OOD generalization capabilities and achieves comparable or even improved in-distribution test performance. These statistical results further validate the effectiveness of our method in handling distribution shifts and enhancing generalization performance.

Overall, the observations we have made provide strong evidence of the great capacity of our method for handling distribution shifts, validating its effectiveness and potential for real-world applications.



% Figure environment removed

% Figure environment removed


% Figure environment removed


\subsection{Ablation Studies}\label{sec:ablation}
\noindent Table~\ref{tab:aba} provides a detailed analysis of the effect of each component according to our proposed recipe for improving OOD generalization in graph contrastive learning. Let's examine the different variants of our method and their impact on performance.
Specifically, MARIO~(w/o ad) represents MARIO without  adversarial augmentation. MARIO~(w/o cmi) denotes we only maximize the mutual information between positive pairs without considering conditional mutual information. MARIO~(w/o cmi, ad) means a vanilla graph contrastive method that is similar to GRACE. 

From Table~\ref{tab:aba}, we can find MARIO~(w/o cmi) lags far behind MARIO on OOD test set which demonstrates appropriately minimizing the redundant information (\ie, conditional mutual information) is essential to improve OOD generalization of GCL methods. And adversarial augmentation can also boost OOD generalization because it can approximately serve as a supermum operator to learn more invariant features  discussed in Section~\ref{sec:aug}. Based on the analysis of these variants, it is evident that the proposed improvements on data augmentation and contrastive loss in the recipe are both effective in enhancing graph OOD generalization. Each component contributes to the overall performance improvement, and their combination leads to a stronger self-supervised graph learner in terms of graph OOD generalization. 

In short, the findings from Table~\ref{tab:aba} support the rationale behind your proposed recipe and provide empirical evidence of the effectiveness of each proposed component. By incorporating these enhancements, our method achieves superior performance in handling distribution shifts and improving graph OOD generalization in graph contrastive learning.
\begin{table*}[htp]
\caption{Ablation studies for MARIO by masking each component.}
\label{tab:aba}
\centering
\scalebox{0.9}{
\begin{tabular}{l|cc|cc|cc|cc|cc}
\toprule
\toprule
\multirow{3}{*}{concept shift} & \multicolumn{4}{c|}{GOOD-Cora}                       & \multicolumn{2}{c|}{GOOD-CBAS} & \multicolumn{2}{c|}{GOOD-Twitch} & \multicolumn{2}{c}{GOOD-WebKB} \\
                           & \multicolumn{2}{c}{word} & \multicolumn{2}{c|}{degree}& \multicolumn{2}{c|}{color}    & \multicolumn{2}{c|}{language}   & \multicolumn{2}{c}{university} \\
                           & ID         & OOD         & ID          & OOD          & ID            & OOD           & ID             & OOD            & ID            & OOD            \\
\midrule
MARIO                      & \textbf{67.11±0.46} & \textbf{65.28±0.34}  & \textbf{68.46±0.40}  & \textbf{61.30±0.28}      & \textbf{94.36±1.21}  & \textbf{91.28±1.10}    & 82.31±0.54     & \textbf{63.33±1.72}     & \textbf{65.67±2.81}    & \textbf{37.15±2.37}     \\
MARIO(w/o ad)              & 66.23±0.53 & 64.02±0.18  & 67.88±0.38  & 60.46±0.29   & 93.21±1.25    & 90.29±0.91    & 82.42±0.73     & 60.50±1.02     & 64.83±2.83    & 36.51±3.25    \\
MARIO(w/o cmi)             & 65.32±0.60 & 63.51±0.32  & 68.14±0.32  & 61.19±0.34   & 94.15±1.23    & 90.57±1.96    & \textbf{82.51±0.56}     & 61.41±2.63     & 64.50±4.35    & 35.78±2.53     \\
MARIO(w/o cmi, ad)         & 64.67±0.55 & 63.11±0.32  & 67.95±0.65  & 60.01±0.57   & 93.36±1.66    & 89.64±1.73    & 81.90±0.75     & 60.12±1.60     & 64.17±3.67    & 34.13±2.38     \\
\bottomrule
\end{tabular}}
\end{table*}
% & 65.32±0.60 & 63.51±0.32 exchange 64.67±0.55 & 63.11±0.32
% 68.14±0.32       id ood test: 60.95±0.43       ood ood test: 61.19±0.34


\subsection{Sensitivity Analysis}\label{sec:sensitivity}
\noindent In this subsection, we will analyze some important hyper-parameters of our method. We conduct sensitivity analysis on GOOD-WebKB dataset with concept shift, we chose two sensitive hyper-parameters (\ie, the coefficient $\gamma$ of condition mutual information in Equation~\ref{equ:cmi} and the number of prototypes $|C|$ in Equation~\ref{equ:pq}). The coefficient of CMI range in $[0.001, 0.01, 0.1, 0.5, 1]$ and the number of prototypes $|C|$ ranges in $[10, 50, 100, 200, 300]$. From Figure~\ref{fig:sensitivity}, we can observe that $\gamma$ reaches 0.1 and $|C|$ reaches 100 or 200 can achieve the best OOD test accuracy. Both higher and lower values of $\gamma$ result in suboptimal performance. This finding aligns with previous research such as DIB~\cite{dib}, indicating that an appropriate compression level is crucial for achieving optimal performance. Extremely high or low compression values are not ideal. 

Regarding the number of prototypes $|C|$, based on the results shown in Figure~\ref{fig:sensitivity}, it is found that setting $|C|=100$ leads to the best performance in terms of OOD test accuracy. This choice provides a moderate number of pseudo labels, which is beneficial for the learning process. 

Based on the sensitivity analysis, we determined that setting $\gamma=0.1$ and $|C|=100$ on most datasets. These hyperparameter values strike a balance between compression level and the number of prototypes, resulting in improved graph OOD generalization.
% Figure environment removed


\subsection{Integrated with Other Models}\label{sec:other_models}
% Figure environment removed

\begin{table}[htp]
\caption{Results of different learning approaches with different encoding models (\ie, GCN, GraphSAGE, GAT).}
\label{tab:others}
\centering
\scalebox{0.9}{
\begin{tabular}{cc|cc|cc}
\toprule
\toprule
\multirow{3}{*}{Model}& \multirow{3}{*}{Method} & \multicolumn{2}{c|}{GOOD-CBAS} & \multicolumn{2}{c}{GOOD-WebKB} \\
                & & \multicolumn{2}{c|}{color}    & \multicolumn{2}{c}{university} \\
                &   & ID          & OOD         & ID          & OOD            \\
\midrule
\multirow{3}{*}{GCN} 
&ERM               & 89.79±1.39 & 83.43±1.19  &  62.67±1.53 & 26.33±1.09         \\
&GRACE             & 92.00±1.39 & 88.64±0.67  &  64.00±3.43 & 34.86±3.43        \\
&MARIO             & 94.36±1.21 & 91.28±1.10  &  65.67±2.81 & 37.15±2.37        \\ \bottomrule
\multirow{3}{*}{SAGE} 
&ERM               & 95.07±1.51 & 75.14±1.19  & 73.67±2.08  & 46.33±3.42       \\
&GRACE             & 95.29±1.11 & 74.43±2.36  & 70.50±5.06  & 49.54±3.83        \\
&MARIO             & 96.00±1.07 & 76.29±3.01  & 71.00±3.82  & 51.74±4.63        \\ \bottomrule
\multirow{3}{*}{GAT} 
&ERM               & 78.64±3.63 & 72.93±2.64  & 61.33±3.71  & 28.99±2.63        \\
&GRACE             & 84.57±1.79 & 78.36±1.60  & 59.50±2.36  & 35.78±3.26        \\
&MARIO             & 84.93±1.95 & 80.43±1.89  & 62.17±4.78  & 38.17±3.10        \\
\bottomrule
\end{tabular}}
\end{table}



\noindent In the subsection, we demonstrate the model-agnostic nature of the recipe by integrating it with various graph neural network (GNN) models, including GCN, GraphSAGE, and GAT.

From Table~\ref{tab:others}, it can be observed that regardless of the specific GNN model used as the encoder, our method consistently achieves the best performance on the OOD test set. This indicates the effectiveness and robustness of our method across different GNN models.
By achieving superior performance across different GNN models, MARIO demonstrates its versatility and ability to improve the OOD generalization of various graph neural models. This highlights the broad applicability and effectiveness of our recipe in enhancing the performance of different GNN encoders.

Furthermore, we integrate our recipe with other GCL methods in Appendix~\ref{app:other_methods}. The results demonstrate our recipe can boost the OOD generalization ability of various GCL methods which means our recipe can serve as a plug-in for many current classical GCL methods.

% Figure environment removed

\subsection{Visualization}\label{sec:vis}
\subsubsection{Metric Score Curves}
We present metric score curves for ERM and MARIO, including training, ID validation, ID testing, OOD validation, and OOD testing accuracy, in Figure~\ref{fig:curve2}. Notably, MARIO demonstrates superior convergence with approximately 10\% absolute improvement on the OOD test set compared to ERM. Furthermore, MARIO effectively narrows the performance gap between in-distribution and out-of-distribution performance, showcasing its efficacy in enhancing OOD generalization for graph data. More metric score curves can be found in Appendix~\ref{app:curves}.


\subsubsection{Feature Visualization}
In order to assess the quality of learned embeddings, we adopt t-SNE~\cite{tsne} to visualize the node embedding on GOOD-Cora dataset (concept shift in word domain) using random-init of GCN, EERM, GRACE, and MARIO, where different classes have different colors in Figure~\ref{fig:vis}. For clarity, we select eight classes with the largest number of nodes to enhance the informativeness and interpretability of the visualization. We can observe that the 2D projection of node embeddings learned by MARIO has a better separation of clusters, which indicates the model can help learn representative features for downstream tasks. It has to note that we depict both ID nodes and OOD nodes in the same figure. 

Besides, we also separately visualize ID nodes and OOD nodes in the different figures in the Appendix~\ref{app:feature}. And we can find MARIO performs a clearer separation of clusters whether on ID nodes or OOD nodes compared to other methods.







% column for runtime table


% % Figure environment removed

% % Figure environment removed

% % Figure environment removed






\subsection{Runtime Performance}
\label{sub:baselines}

\nop{We verified experimentally the following two hypotheses.}

%%%%%%%%%%%%%%%%%%%%
\begin{table*}[t]
\centering 
\caption{Overall runtime (in seconds) to compute all queries for each benchmark. For the JOB benchmark, ">" indicates the time is only for some of the 113 queries. For the four graph datasets, ">" indicates the time exceeded the six-hour (21,600 seconds) timeout for some of the cyclic queries. The multiplicative factors in parentheses after the runtimes of systems are the speedups of ADOPT over these systems.}
\resizebox{1\textwidth}{!}{
\begin{tabular}{l|rrrrrrr}
\toprule[1pt] Systems & JOB & ego-Facebook & ego-Twitter & soc-Pokec & soc-Livejournal1 & \revision{TPC-H} & \revision{JCC-H} \\
\midrule[1pt]
ADOPT       & 45   & \textbf{4,414}     & \textbf{3,931}   & \textbf{9,268}     & \textbf{26,350} & 141 & \textbf{194} \\
System-X    & > 287 (6.38x)    & > 22,459 (5.09x)  & 11,384 (2.90x) & > 23,623 (2.55x)  & > 63,878 (2.42x) & -- & --  \\
EmptyHeaded & --               & 6,783  (1.54x)    & 10,381 (2.64x) & > 43,444 (4.69x)  & > 55,144 (2.09x) & -- & -- \\
PostgreSQL  & 285 (6.33x)      & > 67,774 (15.35x) & > 70,515 (17.94x)  & > 67,016 (7.23x)    & > 101,193 (3.84x) & 182 (1.53x)  & > 216,122 \\
&&&&&&& (1,114x) \\
MonetDB     & \textbf{41} (0.91x) & > 66,165 (14.99x) & > 86,596 (22.03x) & > 59,131 (7.23x)  & > 96,222 (3.84x)  & \textbf{17} (0.12x) & > 216,035 \\
&&&&&&& (1,114x) \\
SkinnerDB   & 65 (1.44x)       & > 69,366 (15.71x) & > 129,741 (33.00x) & > 95,374  (10.29x)  & > 101,392 (3.85x) & 173 (1.23x) & 320 (1.6x) \\
\bottomrule[1pt]
\end{tabular}
}
\label{tab:overall}
\end{table*}
%%%%%%%%%%%%%%%%%%%%%


\revision{ADOPT puts together worst-case optimal join algorithm, which is primarily motivated by cyclic queries, and adaptive processing, which is motivated by scenarios in which size and cost prediction for query planning is difficult (e.g., due to data skew or complex queries). This motivates the following hypotheses.}

\begin{hyp}
    \revision{ADOPT outperforms baselines without worst-case optimal join algorithms on cyclic queries.}
\end{hyp}

\begin{hyp}
    \revision{ADOPT outperforms non-adaptive baselines for complex queries on skewed data.}
\end{hyp}

\begin{hyp}\label{hyp:bad}
    \revision{ADOPT performs worse, compared to baselines, if queries are simple, acyclic, and are executed on uniform data.}
\end{hyp}

% \begin{hyp}
% ADOPT demonstrates comparable or better performance than its competitors for both cyclic and acyclic queries.
% \end{hyp}


\pgfplotstableread[col sep=comma,]{data/facebook_twitter_clique_queries.csv}\fbtwittercliquequery
\pgfplotstableread[col sep=comma,]{data/pokec_livejournal_clique_queries.csv}\pokeclivejournalcliquequery
\pgfplotstableread[col sep=comma,]{data/cycle_queries.csv}\cyclequery


% Figure environment removed



Table~\ref{tab:overall} reports the total time in seconds for different systems and benchmarks. \revision{ADOPT performs best for the four benchmarks on graphs, featuring cyclic queries. Figure~\ref{fig:clique_cycle_results} breaks those results down by query size and query type. Compared to other baselines using worst-case optimal joins, ADOPT's gains derive from larger queries with more predicates, creating the potential for inter-predicate correlations that are hard to predict. This makes it difficult to select optimal attribute orders before execution. PostgreSQL, MonetDB, and SkinnerDB suffer from over-proportionally large intermediate results when processing cyclic queries as they do not implement worst-case optimal joins.}

\revision{The join order benchmark (JOB) features acyclic queries but non-uniform data (i.e., it contains some elements that should benefit ADOPT in the comparison and some that have the opposite effect). Here, ADOPT performs comparably but slightly worse to the best baseline: MonetDB. For System-X, Table~\ref{tab:overall} only reports time for executing a subset of the queries (39 out of 113). The remaining queries have IS/NOT NULL and IN predicates that are not supported by System-X.} EmptyHeaded needs more than five days to construct the data indices (tries) it needs for the non-binary JOB tables so we were not able to report its runtime on the JOB queries.

\revision{TPC-H and JCC-H share the same query templates and database schema but differ in the database content: TPC-H uses uniform data whereas JCC-H uses highly correlated data. On TPC-H, MonetDB performs best and outperforms ADOPT significantly. This is consistent with prior work~\cite{Aberger2018}, showing that systems with worst-case optimal joins (specifically: the LFTJ that ADOPT uses internally) perform significantly worse than MonetDB on TPC-H. Given those prior results and limited support for TPC-H queries in System~X and EmptyHeaded, we compare only to MonetDB as the strongest baseline. Besides drawbacks due to the join algorithm, ADOPT incurs overheads due to adaptive processing which is unnecessary on TPC-H: predicting sizes of intermediate results and plan execution cost is relatively easy due to uniform data.}

\revision{On the other hand, ADOPT outperforms all other systems on JCC-H. Despite sharing the same query templates with TPC-H, JCC-H makes query  optimization hard due to highly correlated data. Here, both adaptive baselines (SkinnerDB and ADOPT) benefit, with ADOPT being significantly faster, whereas all other systems reach the timeout of six hours. This means, even on acyclic queries, traditionally not considered the sweet spot for LFTJ-based joins~\cite{Aberger2018}, ADOPT is preferable if data is sufficiently correlated.}

%In the join order benchmark (JOB), MonetDB is the fastest, followed very closely by ADOPT. This shows that ADOPT is competitive with the fastest competitor that employs traditional query plans for acyclic queries. For System-X, Table ~\ref{tab:overall} only reports the overall runtime for 39 out of the 113 queries.

% These queries do not have  IS/NOT NULL and IN predicates as they are not supported by System-X. 
%\revision{The remaining queries have IS/NOT NULL and IN predicates that are not supported by System-X.}
%ADOPT thus outperforms System-X by at least six times. EmptyHeaded needs more than five days to construct the data indices (tries) it needs for the non-binary JOB tables so we were not able to report its runtime on the JOB queries.




%\revision{In the TPC-H benchmark, MonetDB outperforms other systems, since cardinality estimation is easy due to uniform data, whereas in JCC-H benchmark,  ADOPT is the most effective system. Particularly, PostgreSQL and MonteDB cannot finish Query 21 within six hours due to large cardinality.}

%For the graph benchmarks, ADOPT outperforms its competitors. MonetDB, PostgreSQL, and SkinnerDB, which use traditional query plans, perform consistently worse on cyclic queries than ADOPT, EmptyHeaded, and System-X, which use worst-case optimal join algorithms. This follows the asymptotic complexity gap and confirms the reported runtime gap between traditional query plans and worst-case optimal plans for cyclic queries~\cite{nguyen2015join}. Yet ADOPT outperforms EmptyHeaded and System-X by at least 1.5x on ego-Facebook and 2.5x on the other graphs. We also benchmarked the BAO learned query optimization for  PostgreSQL~\cite{Marcus} on ego-Twitter graph. There was no noteworthy runtime improvement.
% For the graph benchmarks, there was no noteworthy runtime improvement.

%\begin{hyp}ADOPT outperforms competitors for large queries. \end{hyp}

%Figure~\ref{fig:clique_cycle_results} reports the runtime for: $n$-cliques on the ego-Facebook and ego-Twitter graphs (top) and respectively on the soc-Pokec  and soc-Livejournal1 graphs (middle), and for $n$-cycles on the four graphs (bottom). The red top line in each figure represents the timeout (six hours = 21,600 seconds). For small  cliques ($n=3$), System-X is  the best; it uses sensitivity indices to speed up the leapfrogging that intersects sorted lists in LFTJ. For larger cliques, ADOPT significantly outperforms its competitors. The reason is the much larger search space for a good (attribute or join) order. It is therefore much less likely to find a good order for the competitors in contrast to ADOPT, which tries many orders during query execution. For instance, for 6- and 7-clique queries, ADOPT is at least 2x faster than the runner-up system for all graphs. A similar behavior can be observed for $n$-cycle queries in Figure~\ref{fig:clique_cycle_results}. Notably, ADOPT is the only system to complete 6-cycle on soc-Pockec and 5-cycle on Livejournal1 before timeout.

%EmptyHeaded and System-X are the two competitors that use worst-case optimal join algorithms and outperform the other systems on the graph benchmarks. In the following experiments, we give a more detailed comparison of ADOPT against System-X on the graph benchmarks. One reason for choosing System-X over EmptyHeaded is that both ADOPT and System-X use LFTJ as the underlying join algorithm. Furthermore, the performance of these two competitors on the graph benchmark is very close (within 4.8\% of each other). Also, the query language of System-X supports unary predicates, which are needed for the next robustness experiment.


% \pgfplotstableread[col sep=comma,]{data/relative_facebook_twitter_clique_queries.csv}\relativefbtwittercliquequery
% \pgfplotstableread[col sep=comma,]{data/relative_pokec_livejournal_clique_queries.csv}\relativepokeclivejournalcliquequery
% \pgfplotstableread[col sep=comma,]{data/relative_cycle_queries.csv}\relativecyclequery


% % Figure environment removed






% In this section, we evaluate the robustness of different systems. We vary selectivity of different queries and  

% Figure environment removed

\subsection{Robustness}
\label{sub:robustness}

ADOPT does not rely on query optimization to pick the best attribute order. This can be a significant advantage for queries with user-defined functions or selection predicates, for which there are no available selectivity estimates. Mainstream systems pick a query plan that may be arbitrarily off from a good one. In contrast, ADOPT may quickly realize that such a plan is subpar and switch to a different one. To benchmark this observation, we consider experiments to assess the {\em  robustness} of ADOPT and System-X, which are the two systems we use that rely on attribute orders, when adding to the join queries very simple (unary) yet arbitrary selection conditions that can throw off standard query optimizers.

\begin{hyp}
ADOPT outperforms System-X consistently when varying the selectivity of unary predicates. 
\end{hyp}

% \vspace{-1em}
Figure~\ref{fig:robustness} shows the relative speedup of ADOPT over System-X as we vary the selectivity of unary predicates (selections with constants) on three randomly chosen attributes: we choose the five selectivities 0.2, 0.4, 0.6, 0.8, and 1 for the three attributes along the x-axis,  y-axis, and the circles for an (x,y)-point.
The color of each 3D point in the plot varies from blue to red: The more intense the red is, the higher is the speedup of ADOPT over System-X. System-X mostly outperforms ADOPT for 3-cliques. ADOPT is up to three times faster than System-X for all other cliques and cycles. This is due to the difficulty of optimizers to pick a good query plan in the absence of selectivity estimates, here even for unary predicates.


% Join orders with the same prefix is shared. In skinerDB.
% Hypercube approach divide executions into hypercubes. 

\pgfplotstableread[col sep=space,]{data/share_progress_clique_ratio.csv}\shareprogessclique
\pgfplotstableread[col sep=space,]{data/share_progress_cycle_ratio.csv}\shareprogesscycle

\captionsetup[figure]{font=small}

% Figure environment removed



%\subsection{Comparison of ADOPT Variants}
\vspace{-0.5em}
\subsection{Hypercube Data Partitioning}
\label{sub:variants}

We next benchmark the effect of our hypercube partitioning scheme and verify that it indeed leads to faster execution time than  SkinnerDB's alternatives called shared prefix+offset progress tracker~\cite{trummer2019skinnerdb}. 

\begin{hyp}
Hypercube partitioning leads to faster execution than shared prefix progress tracker and offset progress tracker.
\end{hyp}

SkinnnerDB shares progress between all join orders with the same prefix (iterating over all possible prefix lengths). Given a join order, it restores a state by comparing execution progress between the current join order and all other orders with the same prefix and by selecting the most advanced state. Offset progress tracker keeps the last tuples of each table that have been joined with all other tuples already. Using hypercube partitioning, ADOPT executes the episodes on disjoint parts of the input data so it can trivially compute distributive aggregates such as count. This is not the case for SkinnerDB's partitioning: To avoid recomputation of the same result in different episodes, it has to maintain a data structure (concurrent hash map). In the multi-thread environment, the prefix share progress tracker blocks the concurrent execution and causes significant synchronization overhead. 






Figure~\ref{fig:relative2} shows the speedup of using the hypercube  partitioning over using the prefix+offset share progress tracker in ADOPT. The hypercube approach consistently has significant smaller overhead than prefix+offset share progress tracker. For larger (above 4) clique and cycle queries, the speedup is 10x to 100x.


\subsection{Time Breakdown by Attribute Order}
\label{sub:optimalattributeorder}

\nop{We run experiments to verify the following hypothesis.}

\begin{hyp}
ADOPT spends most time on executing near-optimal attribute orders.
\end{hyp}

\pgfplotstableread[col
sep=space,]{data/five_clique_order.csv}\fivecliqueorder

\pgfplotstableread[col
sep=space,]{data/five_cycle_order.csv}\fivecycleorder

\pgfplotstableread[col
sep=space,]{data/four_clique_order.csv}\fourcliqueorder

\pgfplotstableread[col
sep=space,]{data/four_cycle_order.csv}\fourcycleorder

% Figure environment removed

We verified this hypothesis for $n$-clique and $n$-cycle queries with $n\in\{4,5\}$, since for these queries it was feasible to generate and execute all possible attribute orders. This was necessary to understand which orders are better than others and assess whether ADOPT uses predominantly good or poor orders.
We plot the orders that we select and their quality relative to the optimal orders (i.e., with lowest execution time) in Figure~\ref{fig:orderdetail}. The x-axis is the number of time slices  that use an order: the larger the x-value, the more we use an order. The y-axis is execution time of an order relative to the optimal one: The smaller the y-value, the closer to the optimal the order is. For $4$-clique and $4$-cycle, ADOPT spends more than $10^6$ (over 95\% frequency) times on executing an order with near-optimal performance. For $5$-cycle and $5$-clique,  ADOPT picks a near-optimal order more than $10^8$ times (over 98\% frequency). ADOPT thus quickly converges to a near-optimal order and then uses it for most of the processing, which confirms our hypothesis.

% We show orders that we select and its quality in Figure~\ref{fig:orderdetail}.

\begin{table}[t]
\centering 
\caption{Execution times (sec) for clique and cycle queries on ego-Twitter of: ADOPT, LFTJ with optimal attribute order (OPT), average runtime of LFTJ over all attribute orders (AVG).  Relative speedup of OPT over ADOPT (last column).}
\begin{tabular}{l|rrrr}
\toprule[1pt]  & ADOPT & OPT & AVG & ADOPT/OPT \\
\midrule[1pt]
3 clique & 4.1	 & 1.6	  & 3.7	   & 2.52 \\
4 clique & 10.5  & 6.8	  & 23.9   & 1.54 \\
5 clique & 77.9	 & 52.5	  & 275.6  & 1.48 \\
3 cycle	 & 4.1	 & 1.6	  & 3.5	   & 2.52 \\
4 cycle	 & 20.1  & 17.4	  & 58.9   & 1.16 \\
5 cycle	 & 377.9 & 328.8  &	3618.1 & 1.14 \\
\bottomrule[1pt]
\end{tabular}
\label{tab:ratio_with_opt}
\end{table}

Table~\ref{tab:ratio_with_opt} compares ADOPT and LFTJ with an optimal attribute order: The runtime gap decreases from 2.52x for 3-clique/cycle to 1.48x (1.14x) for 5-clique (5-cycle). This is remarkable, given that ADOPT tries out several attribute orders and switches between them, whereas LFTJ only uses one attribute order, which is optimal. Table~\ref{tab:ratio_with_opt} also shows that ADOPT takes significantly less time than the average runtime of LFTJ over all attribute orders\nop{ the larger the query gets}.



% R_1(a, b), R_2(b, c), R_3(c, a)
% R_1(a, b, c), R_2(b, c, d), R_3(c, d, a), R_4(d, a, b)
% R_1(a, b, c, d), R_2(b, c, d, e), R_3(c, d, e, a), R_4(d, e, a, b), R_5(e, a, b, c)
% R_1(a, b, c, d, e), R_2(b, c, d, e, f), R_3(c, d, e, f, a), R_4(d, e, f, a, b), R_5(e, f, a, b, c), R_6(f, a, b, c, d)

% E(a, b, a), E(b, a, b), E(a, b, a), E(b, a, b)
% R_1(a, b, a, b), R_2(b, a, b, a), R_3(a, b, a, a), R_4(b, a, a, b), R_5(a, a, b, a)



\subsection{Parallelization}
\label{sub:parallelization}


\begin{hyp}
    ADOPT achieves almost linear speedup for large cyclic queries. 
\end{hyp}

\pgfplotstableread[col
sep=comma,]{data/speedup_twitter_clique.csv}\parallelspeedupclique
\pgfplotstableread[col
sep=comma,]{data/speedup_twitter_cycle.csv}\parallelspeedupcycle

\pgfplotsset{
legend image code/.code={
\draw[mark repeat=2,mark phase=2]
plot coordinates {
(0cm,0cm)
(0.05cm,0cm)        %% default is (0.3cm,0cm)
(0.05cm,0cm)         %% default is (0.6cm,0cm)
};%
}
}

% Figure environment removed

Figure~\ref{fig:speedup} plots the speedup of ADOPT as a function of the number of threads. ADOPT achieves significant speedups for large clique and cycle queries. In particular, it achieves nearly 30x speedup on 5- and 6-clique, and 40x speedup on 5-cycle (with 48 threads). The main reason is that the hypercube approach partitions disjointly the workload across threads, minimizing synchronization overheads. %and makes the computation synchronization-free and therefore fully parallelizable.


%avoids concurrent data structures to remove duplicate tuples from different orders and thus can be fully parallelized.

% For complex queries (e.g., 5 cycle and 6 clique), ADOPT 

\section{Related Work}
%\subsection{Cost Volume based Deep Stereo Matching}
%Stereo matching is a typical problem that has been studied for decades and a well-known four-step pipeline \cite{scharstein2002taxonomy} has been established, where cost volume construction is an indispensable step. Current state-of-the-art stereo matching methods are all cost volume based methods and they can be categorized into two types. Typically, a cost volume is a 4D tensor of height, width, disparity, and features. The first category just uses a full correlation to generate a single-feature cost volume. Such methods are usually efficient but lose much information because of the decimation of feature channels. Many previous work, including Dispnet \cite{dispnet}, MADNet \cite{madnet}, IResNet \cite{iresnet} and AANet \cite{aanet}, belong to this category. The second category usually uses concatenation \cite{gcnet} or group-wise correlation \cite{gwcnet} to generate a multi-feature 4D cost volume. Such a method can achieve better performance while requiring higher computational complexity and memory consumption. Actually, a majority of the top-performing networks in public leaderboards belong to this category, such as GANet \cite{ganet}, CSPN \cite{cspn} and ACFNet \cite{acfnet}. These methods generally employ multiple 3D convolution layers to constantly regularize the 4D cost volume and then apply softmax over the disparity dimension to produce a discrete disparity probability distribution. The final predicted disparity is obtained by softly weighting indices according to their probability, which is also called soft argmin in GCNet \cite{gcnet}. However, soft argmin leaves the output susceptible to multi-modal disparity probability distributions. ACFNet \cite{acfnet} observes this problem and proposes to directly supervise the cost volume with unimodal ground truth distributions. In contrast, we define an uncertainty estimation to quantify the degree to which the cost volume tends to be multi-modal distribution, higher implies the higher possibility of estimation error.

\subsection{Multi-scale Cost Volume based Stereo Matching}
Cost volume construction is an indispensable step in the well-known four-step pipeline for stereo matching \cite{scharstein2002taxonomy, pamisurvey1, pamisurvey2}. Typically, current state-of-the-art stereo matching methods can be categorized into two types of cost volume-based methods, where the cost volume is a 4D tensor of height, width, disparity, and features. The first category usually uses the single-feature 3D cost volume generated by full correlation, which is efficient while losing much information due to the decimation of feature channels. Many real-time methods, such as Dispnet \cite{dispnet}, MADNet \cite{madnet, madnet_pami} and AANet \cite{aanet}, belongs to the category. Moreover, two-stage refinement \cite{mcvmfc} and pyramidal towers \cite{madnet} are commonly applied in the single-feature cost volume based network to construct multi-scale cost volume. The second category usually uses the multi-feature 4D cost volume generated by concatenation \cite{gcnet} or group-wise correlation \cite{gwcnet}, which can achieve better performance with higher computational complexity and memory consumption. Most top-performing networks, including GANet \cite{ganet}, CSPN \cite{cspn} and ACFNet \cite{acfnet} belong to this category. 
% In these methods, the 4D cost volume is constantly regularized by multiple 3D convolution layers and then a discrete disparity probability distribution can be produced by softmax. Next, the final predicted disparity can be obtained by softly weighting indices according to their probability \cite{gcnet}. However, such output is susceptible to multimodal disparity probability distributions and ACFNet \cite{acfnet} gives a solution by directly supervising the cost volume with unimodal ground truth distributions to alleviate this problem. 
Recently, to alleviate the high computational complexity and memory consumption when employing multi-feature 4D cost volumes, \cite{cvpmvsnet, cascade, uscnet} propose to use cascade cost volume representation in multi-view stereo. These methods usually first predict an initial disparity at the coarsest resolution of the image and then gradually refine the disparity by narrowing down the disparity search space. More closely related to our approach is Casstereo \cite{cascade}, which first extended such representation to stereo matching. It selected to uniform sample a pre-defined range to generate the next stage’s disparity search range. Instead, we employ pixel-level uncertainty estimation to adaptively adjust the next stage disparity searching range and generate pseudo-labels for subsequent domain adaptation. Our method also shares similarities with UCSNet \cite{uscnet}, which constructs uncertainty-aware cost volume in multi-view stereo while it doesn’t employ uncertainty estimation to generate pseudo-labels.

%\subsection{Multi-scale Cost Volume based Deep Stereo Matching} 
% \subsection{Multi-scale Cost Volume based Stereo Matching} 
%Multi-scale cost volume firstly was applied in the single-feature cost volume based network with the form of two-stage refinement \cite{mcvmfc} and pyramidal towers \cite{madnet}. Recently, cascade cost volume representation \cite{cvpmvsnet, cascade, uscnet} was proposed in multi-view stereo to alleviate the high computational complexity and memory consumption when employing multi-feature 4D cost volumes. These methods generally predict an initial disparity at the coarsest resolution of the image. Then, they will narrow down the disparity search space and gradually refine the disparity. More closely related to our approach is Casstereo \cite{cascade}, which first extended such representation to stereo matching. It selected to uniform sample a pre-defined range to generate the next stage’s disparity search range. Instead, we employ uncertainty estimation to adaptively adjust the next stage pixel-level disparity searching range and push the next stage's cost volume to be predominantly unimodal.

% The single-feature cost volume based network with the form of two-stage refinement \cite{mcvmfc} and pyramidal towers \cite{madnet} first employ multi-scale cost volume for stereo matching. Recently, to alleviate the high computational complexity and memory consumption when employing multi-feature 4D cost volumes, \cite{cvpmvsnet, cascade, uscnet} propose to use cascade cost volume representation in multi-view stereo, which generally predict an initial disparity at the coarsest resolution of the image. Then, the disparity search space is narrowed down and the disparity is gradually refined. More closely related to our approach is Casstereo \cite{cascade}, which first extended such representation to stereo matching. It selected to uniform sample a pre-defined range to generate the next stage’s disparity search range. Instead, we employ uncertainty estimation to adaptively adjust the next stage pixel-level disparity searching range and push the next stage's cost volume to be predominantly unimodal.

% Figure environment removed

\subsection{Robust Stereo Matching} 
There exist three categories of generalization definitions for robust stereo matching. 1) Cross-domain Generalization: the network’s ability to perform well on unseen scenes (cannot see the image pairs of the target domain in advance). Towards this end, Jia et al \cite{sungeneralizaiton} propose to incorporate scene geometry priors into an end-to-end network. Zhang et al \cite{dsmnet} introduce a domain normalization and a trainable non-local graph-based filter to construct a domain-invariant stereo matching network. 2) Adapt Generalization: the network’s ability to adapt pre-trained models to the new domain with unlabeled target data. Previous work usually pre-trains the models on synthetic data and then adapts it to new target domains with Graph Laplacian regularization \cite{zoom}, non-adversarial progressive color transfer \cite{adastereo}, and Knowledge Reverse Distillation \cite{aohnet}. More closely related to our approach are \cite{aohnet, unsuperviseddomainadaptation} in stereo matching and Monoresmatch \cite{monoresmatch} in monocular depth estimation, which also proposes to generate a pseudo-label for domain adaptation. However, these methods all select to employ classical stereo matching methods \cite{sgm} alongside with confidence estimators, e.g., left-right consistency check to generate pseudo-labels. That is all these methods need an independent method to generate corresponding pseudo-labels. Instead, the proposed method is an end-to-end network that can generate the predicted disparity map, corresponding uncertainty map and pseudo-labels jointly, which is a more simple, yet efficient way. 
% Instead, our proposed method can employ pixel-level and area-level uncertainty estimation to self-distill the predicted disparity maps of our pre-training model and generate sparse while reliable pseudo-labels to align the domain gap, which is a more simple, yet efficient way. 
3) Joint Generalization: the network’s ability to perform well on a variety of datasets with the same model parameters. MCV-MFC \cite{mcvmfc} introduces a two-stage finetuning scheme to achieve a good trade-off between generalization and fitting capability on multiple datasets. However, it doesn’t touch the inner difference between diverse datasets, e.g, the unbalanced disparity distribution. To further address this problem, we propose a cascade cost volume to adaptively the next stage disparity searching space, where the pixel-level uncertainty estimation is at the core.

% \subsection{Monocular Depth Estimation}
% Monocular depth estimation aims to estimate depth values from a single image, instead of stereo images or multiple frames in a video. This problem is ill-posed because of the ambiguity of object sizes. However, humans could estimate the depth from a single image with prior knowledge of the scenes. Recently, learning based methods were explored to learn depth values by supervised or unsupervised learning. Eigen et al. first employed Convolutional Neural Networks (CNN) to predict depth in a coarse-to-fine manner and further improved its performance by multi-task learning. Liu et al. presented deep convolutional neural fields model by combining deep model with continuous CRF. Li et al. [22] refined deep CNN outputs with a hierarchical CRF. Multi-scale continuous CRF was formulated into a deep sequential network by Xu et al. [45] to refine depth estimation. Unsupervised methods tried to train monocular depth estimation with stereo
% image pairs or image sequences and test on single images. Garg et al. [9] used novel image view synthesis loss to train a depth estimation network in an unsupervised way. Godard et al. [11] introduced left-right consistency regularization to improve the performance of view synthesis loss. Recently, some work also propose to use the stereo matching network as a proxy to learn depth from synthetic data or directly employ traditional stereo matching methods to distill proxies labels from the target domain, which proves the feasibility of distilling stereo matching networks to learn monocular depth estimation.



\section{Conclusion and Future Work}
In this work, I design corruption-robust algorithms for the Lipschitz contextual search problem. I present the \emph{agnostic checking} technique and demonstrate its effectiveness in designing corruption-robust algorithms. There are several open problems for future research. First, in the algorithm I propose for pricing loss, the schedule for agnostic checks is fixed upfront. Can the learner design an adaptive checking schedule for the pricing loss? Second, this work assumes the learner has knowledge of the Lipschitz constant $L$. Can the learner design efficient no-regret algorithms without knowledge of $L$? 

\bibliographystyle{abbrv}
\bibliography{library}

\appendix
\begin{comment}
\section{System Architecture}
\label{appendix:architecture}
\system has a novel modularized system architecture with three key components: 
\emph{StreamManager}, 
\emph{TxnManager} and \emph{TxnScheduler}. 
These components are instantiated in each thread locally.
The execution outline of \system is presented in Algorithm~\ref{alg:algo}.
Transactional stream processing is continuous and potentially never ends (Line 1$\sim$8).
The dependency resolution and execution of state transactions are separated into two non-overlapping phases by punctuations~\cite{Tucker:2003:EPS:776752.776780} (Line 2 and 5), which guarantees that no subsequent input event will have a smaller timestamp. 
Effectively, a batch of state transactions is collected during the first phase, and processed during the second phase.

In the first phase (i.e., stream processing phase), 
the \emph{StreamManager} conducts preprocessing for every input event ($e$). Similar to some prior works~\cite{tstream}, state transactions may be issued but not immediately processed during preprocessing (Line 3).
The \emph{pre\_processing} and \emph{post\_processing} functions are exposed as APIs to users.
The \emph{TxnManager} handles dependency resolution (Line 4) among state transactions and insert decomposed operations to construct a \tpg. We discuss the detailed two-phase \tpg construction process in Section~\ref{subsec:construction}.

In the second phase  (i.e., transaction processing phase), 
the \emph{TxnManager} is first involved again to refine (Line 6) the constructed \tpg with further dependency resolution.
The \emph{TxnScheduler} 
schedules operations for concurrent execution based on the constructed \tpg according to the three dimensions of scheduling decisions (Line 7). 
In particular, a scheduling decision model $M$ is instantiated based on the constructed \tpg (Line 14).
\textbf{\circled{1}} Guided by $M$, execution threads adopt an exploration strategy (Section~\ref{subsec:explore}) to explore the constructed \tpg for operations available to be scheduled constrained by dependencies. 
\textbf{\circled{2}} 
During exploration, one or multiple operations may be treated as the 
% basic 
unit of scheduling (Section~\ref{subsec:granularity}). 
Subsequently, \textbf{\circled{3}} every thread executes operation(s) in the unit of scheduling with various abort handling mechanisms (Section~\ref{subsec:abort_handling}).
Only when state transactions are processed (i.e., committed or aborted) can the associated input events be postprocessed (Line 8) by the \emph{StreamManager} based on transaction processing results.
\end{comment}

\begin{comment}
\begin{algorithm}
\footnotesize
    \KwData{$e$ \tcp{Input event}}
    \KwData{$txn_{ts}$ \tcp{State transaction}}
    \KwData{$G$ \tcp{The currently constructed TPG}}
    \While{!finish processing of input streams}{
        \eIf(\tcp*[h]{Phase 1}){\text{$e$ is not a $punctuation$}}{
                $txn_{ts}$ $\gets$ PRE\_Processing($e$)\;
                \textbf{TPG\_Construction}($G$, $txn_{ts}$)\; 
          }(\tcp*[h]{Phase 2}){
                \textbf{TPG\_Refinement}($G$)\; 
                \textbf{TXN\_Scheduling}($G$)\; 
                POST\_Processing()\;
          }
    }
    
    \SetKwFunction{FMain}{TPG\_Construction}
    \SetKwProg{Fn}{Function}{:}{}
    \Fn{\FMain{$G$, $txn_{ts}$}}{
        $O_{1..k}$ $\gets$ \textbf{Partition} $txn_{ts}$\;
        \ForEach{\text{operation $O_{i}$ $\in$ $O_{1..k}$}}{
            \textbf{Identify} its \ld\;
            $G$ $\gets$ $G$ + $O_{i}$ \;
        }
    }
    \SetKwFunction{FMain}{TPG\_Refinement}
    \SetKwProg{Fn}{Function}{:}{}
    \Fn{\FMain{$G$}}{
        \ForEach{\text{vertex $e_{i}$ $\in$ $G$}}{
            \textbf{Identify} its \td, \pd\;
        }
    }
    
    \SetKwFunction{FMain}{TXN\_Scheduling}
    \SetKwProg{Fn}{Function}{:}{}
    \Fn{\FMain{$G$}}{
        $M$ $\gets$ Instantiated with $G$;\tcp{A decision model}
        \While{!finish scheduling of $G$
        }{
          \textbf{\circled{2}} $Scheduling Unit$ $\gets$ \textbf{\circled{1}} \emph{Explore}($G$, $M$)\; 
            \textbf{\circled{3}} \emph{Execute with Abort Handling} ($Scheduling Unit$)\; 
        }
    }
  \caption{Execution Outline of \system}
  \label{alg:algo}
\end{algorithm}
\end{comment}
\end{document}
