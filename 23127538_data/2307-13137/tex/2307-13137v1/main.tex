% \documentclass[fleqn,10pt,usenames,dvipsnames]{wlscirep}
\documentclass[11pt,a4paper]{article}

\usepackage[usenames,dvipsnames]{xcolor}
\usepackage{colortbl}
\usepackage{multirow}
%

\usepackage{titling}

\usepackage[T1]{fontenc}
\usepackage[utf8]{inputenc}
\usepackage{textcomp} % provides euro and other symbols

\usepackage{amssymb,amsfonts,amsmath}
\usepackage{hyperref}
% \hypersetup{
%             pdfborder={0 0 0},
%             breaklinks=true}
% \urlstyle{same}  % don't use monospace font for urls
\usepackage{url}

\usepackage{authblk}

\usepackage{booktabs,tabularx,ltablex,longtable}
\usepackage{eucal} % to render the mathcal font "correctly"
\usepackage{paralist}
\usepackage{enumitem}

% Fix footnotes in tables (requires footnote package)
\usepackage{footnote}
\usepackage{graphicx}
% \usepackage[authoryear,round]{natbib}
\usepackage[left=1.5cm, right=1.5cm]{geometry}

\usepackage[style=mla,sorting=nyt,backend=bibtex,%
            noremoteinfo=false,mincitenames=1,maxcitenames=2]{biblatex}
\addbibresource{biblio.bib}

% increasing bibliography item separation
\setlength\bibitemsep{1.5\itemsep}



\newcommand{\etal}{\emph{et al.}}
\newcommand{\ie}{\emph{i.e.,}}
\newcommand{\eg}{\emph{e.g.,}}


% other macros
\newcommand{\kshell}{$k$-shell}
\newcommand{\kcore}{$k$-core}
\newcommand{\ks}{k_\text{s}}
\newcommand{\nc}{n_\text{C}} % general purpose number of communities
\newcommand{\NC}{N_\text{c}} % total number of communities

\newcommand{\mycb}{\textsc{CB}}
\newcommand{\myev}{\textsc{EV}}
\newcommand{\myrel}{\textsc{REL}}
\newcommand{\mycor}{\textsc{COR}}
\newcommand{\mypub}{\textsc{PUB}}
\newcommand{\myall}{\textsc{ALL}}

% macro biblatex-mla
\DeclareAutoCiteCommand{footnote}[f]{\footcite}{\footcites}



% graphic paths
\graphicspath{{graphics/}}

\title{Quantifying women marginalisation in Ibero-American film culture during the first half of XX\textsuperscript{th} century: a quantitative proposal based on network science\thanks{This research is funded by the ERC StG project ``Social Networks of the Past: Mapping Hispanic and Lusophone Literary Modernity, 1898--1959'' (Grant agreement No. 803860).}}

\author[$\dagger$,$\ddag$,$\star$]{Ainamar Clariana-Rodagut}
\author[$\dagger$,$\star$]{Alessio Cardillo}

\affil[$\dagger$]{Universitat Oberta de Catalunya -- Internet interdisciplinarily Institute (IN3), Barcelona, Spain}
\affil[$\ddag$]{Philipps-Universit\"at Marburg, Germany}
\affil[$\star$]{aclariana@uoc.edu; acardillo@uoc.edu}

\date{}



\begin{document}

\begin{titlingpage}

\maketitle

\begin{abstract}
The research presented here uses the tools of Social Network Analysis to empirically show a socio-cultural phenomenon already addressed by the social sciences and history namely: the historical marginalisation of women in the field of cinema. The novelty of our approach lies in the use of a large amount of heterogeneous historical data. On the one hand, we built a network of interactions between people involved in the film field in Ibero-America during the first half of the twentieth century. On the other hand, we propose a \kcore{} decomposition and a multi-layered analysis, as a quantitative way to study the position of women within the cultural melieu. After conducting our analysis, we concluded that women were mostly situated in the outer \kshell{s} of the empirical network, and their distribution was not uniform across the \kshell{s}. From a qualitative perspective, these results can be interpreted as the consequence of the lack of evidence of the participation of women in the public sphere.
\end{abstract}

\end{titlingpage}

%

% \flushbottom
% 
% \thispagestyle{empty}


\section{Introduction}

Digital Humanities blossomed as a discipline at the beginning of the XXI\textsuperscript{st} century. Since then, a multitude of methods imported from the computational sciences have been employed to conduct research in humanities. Digital tools have been successfully used in a plethora of disciplines, including: history, philosophy, philology, and cultural studies; to analyse large amounts of data which, otherwise, would have been impossible to manage using traditional tools, thus allowing to find new patterns pointing to unidentified phenomena \autocite[2014]{schich-2014},\autocite[2018]{fraiberger-2018}. The research presented here uses the tools of Social Network Analysis to empirically show a
socio-cultural phenomenon already addressed by the social sciences and history namely: the historical marginalisation of women in the field of cinema. The novelty of our approach lies in the use of a large amount of heterogeneous historical data, and in the proposal of a method to analyse the position of women within social networks linked to the cultural field. This approach allows us to highlight unbeknownst nuances of marginalisation, and tie them to the network's structural properties and the nature of connections composing it.

Our analysis has two goals. On the one hand, to detect the social position of women in the cultural medium, specifically regarding film, in Ibero-America over the first half of the twentieth century, as per the data compiled in our database. To do so, we apply social network analysis and use a gender perspective and a relational approach. On the other hand, we propose a \kcore{} decomposition and a multi-layered analysis, as a quantitative way to detect the societal role of women in the cultural \emph{melieu}.

The data we rely on for our analysis comes from a dataset comprising data on women who participated in the film field during the first half of the twentieth century in Ibero-America. This dataset was extracted from the database created by the participants of the ERC project and members of the \href{http://globals.research.uoc.edu/}{{GlobaLS}} research group. Within the main project ``Social Networks of the Past: Mapping Hispanic and Lusophone Literary Modernity, 1898--1959'' this dataset was built as part of the cinema and the women subprojects.

The methodological proposal and the analysis undertaken for this paper are the result of a joint effort between Alessio Cardillo -- a complex systems scientist -- and Ainamar Clariana -- a Humanist with a gender perspective and specialised in Cinema studies -- to work interdisciplinarily with the aim of finding influential women in our database. From a qualitative perspective, it is generally accepted that historiography has invisibilised women who worked in the film field during the silent period and the early cinema period.\footnote{This lack of data is exacerbated in our case by the geographical context on which we focus, considered peripheral by film historians. See for example Richard Abel's \emph{Encyclopedia of Early Cinema} (2004) in which the entries of filmmakers are organised according to the countries where they worked, with the United States prevailing over other countries in terms of number of filmmakers. In addition, these data records contain far fewer women than men. To see the recent work being done to recover the stories of women in silent cinema, see the bibliography proposed in the \href{https://wfpp.columbia.edu/}{{Women Film Pioneers project}}~\autocite{gaines}. A selection of references includes the work of Jane Gaines on women in Hollywood, Clara Auclair for French workers in the US, Monica Dall'Asta,, S. Louisa Wei, Anna Kovalova, Aurore Spiers, Dominique Nasta , and Kerstin Fooken for Italian, Chinese, Russian, French, European, and Japanese cinema, respectively.} It means that the data we gathered for the proposed framework are scarce and sparse. Even so, we believe that quantitative methods can 1) contribute to the discussion on the reasons why women suffered marginalisation, and 2) network science can also propose different methods to tackle the historical roles played by women in the cultural field. This paper does not aim at changing the historical knowledge we already have on the diversity of positions women occupied in the film industry, or their contribution to the building of film cultures during the first half of the twentieth century, but aims -- instead, -- at empirically showing in the referred framework how the marginalisation effect conditioned them.

The research questions driving our proposal are: 1) we wanted to visualise the roles and positions women had in the networks extracted from our data, posing the question: were they peripheral? After
realising that women were effectively situated at the margins of our network -- despite the gender perspective and the positive discrimination driving the data gathering process; 2) we wanted to quantify the marginalisation effect we were detecting visually and qualitatively. To accomplish our goals, we choose the \kcore{} decomposition as a suitable method for analysing quantitatively the
women's peripheral position in our networks.

After conducting our analysis, we concluded that women were mostly situated in the outer \kshell{s} of the empirical network, and their distribution was not uniform across the \kshell{s}. In contrast, the randomised counterpart networks displayed a more even distribution of women across the \kshell{s}. Therefore, we postulate the existence of a \emph{social force} pushing women to the outskirts of the network. In light of our results, we consider that the marginalisation effect is not due to mere chance but, instead, it must be due to the way the cultural field is structured and behaves when it
is analysed as a network. From a qualitative perspective, these results can be interpreted as the consequence of the lack of data on women. Such lack of data does not mean in any way that women did not occupy key roles in the institutionalisation process taking place in the film field during the first decades of the twentieth century. A plausible argument to this scarcity of data is related to the disproportion between the few positions women occupied in the public sphere, and the dense fabric of relations and collaborations that, conversely, involved them in the private spheres. As the majority of data we rely on has been gathered from publications and official documents, they fail in
grasping/recording the exchanges taking place in private spheres \autocite[1998]{arendt-1998}. To address data's scarcity, we have developed certain strategies to gather them, such as building ego-networks around certain key women mediators, and then paying special attention to their female collaborators. Alternatively, we can create a category for relationships in our database to add women for whom we do not have other information than their relationship (including marriage) with a man occupying a relevant position in the cultural field of that time. Taking into account our knowledge on the historical period we are working on, we also followed certain steps in the data structuring process in order to highlight the names of women without further meta-information available. An example of a strategy we followed when considering distinct interactions' type separately was to convert the relationship category into a separate network, to recover all the women who had personal relationships with other well-known people, who had a great symbolic power in the film and cultural field. We took that decision keeping in mind that almost all women who were married to a male cultural mediator would have probably carried out relevant activities within the same field. We will provide further explanation for these strategies in the following sections.



\section{Methodology}

\subsection*{Perspectives}

As this paper includes the word ``women'' in its title, there is a need to reflect upon what we mean when referring to it. First of all, it is crucial to state that most of the women we are referring to were white and came from upper and middle class backgrounds. These characteristics were among the reasons why these women survived in the intellectual and cultural fields they were part of. Although, on the one hand, their condition as white and upper-middle class women inevitably limited their social role, their position was, on the other hand, very privileged compared to that of other women. Silvia Federici~\autocite[2004]{federici-2004} gave us the key for thinking about our framework: as long as the sexual division of labour in the framework we were analysing was as powerful as it was since the emergence of capitalism, the concept ``women'' was still a useful category for analysis. Federici's work led us to realise the importance of the strategies adopted by the women from our framework to legitimise their voices and their work in the cinema field they sought to belong. One of the features that characterised most of the women we had in our database was their fight against those roles they were imposed for being women in the societies they were born in. Such a struggle for emancipation indicates that for all of them it existed a clear definition of what a woman was, a category they often referred to and described in their works. Given the cultural environment they were raised in, they had the idea that their primary duty as women was to get married and care for the other members of their family. In one way or another, most of the women we had data on fought against this pre-imposed role. The most important part of such a conflict for them was to avoid the reproductive labour they were aimed at. The result of such a rebellion was the social rejection they suffered. Despite that, the very privileged environment they were part of helped these women to boost their productivity in the artistic and cultural fields, as a direct consequence of their rejection. In this sense, these women could occupy social positions that were previously precluded to women, due to the freedom (\eg{} free time) they acquired.

The comparison among men and women is useful as long as we use it to describe the differences themselves and their potentialities, but the comparison proves useless as long as we measure, assess, or aim at the equal conditions of men and women expecting equal, or even comparable, results. In no way the public roles the women we worked on could be tackled using the same methods we use to analyse men's contribution within the same context. Although men's contributions have always been historically acknowledged, women's contributions did not receive the same praise, because of their limited access to the public sphere. Therefore, if women's contributions were to be compared to their male counterparts using the same tools to measure them, women would always appear as less relevant -- or gifted -- towards the history of culture. The above reasoning led us to work on certain compensation strategies which, taken together, make up the method of analysis we propose in this paper.


\subsection*{Data}

Our analysis leverages data collected by our research group stored in the \href{https://nodegoat.net/}{{Nodegoat}}'s Virtual Research Environment. The database has been filled and enriched since 2019 and,
therefore, is the result of the collaboration between current and past members of our research group. We believe it is important to be clear about the work involved in compiling data, given that such a labour is often invisibilised, as noted by Catherine D'Ignazio and Lauren Klein in their book entitled \emph{Data Feminism} (2019)~\autocite{ignazio-2020}. This invisibility implies that the \emph{places} from which datasets are built are rarely acknowledged. Thus, little responsibility tends to be assumed around the biases affecting data-collection, cleaning, and systematisation processes. While knowledge should always be situated, as advocated by \autocite[1998]{haraway-1988}, it is imperative that we specify our places of enunciation as researchers, \ie{} our standpoints. This means that we must take responsibility for our biases, recognize what is being left out and what we are taking for granted, and clarify where the data were gathered from and what procedure was used to extract them. The goal of such a process is to avoid what Haraway calls ``the god trick,'' that is: the deployment of a perspective that is situated nowhere, presuming that what we are showing reflects reality, instead of reflecting just one of its parts, \emph{our version of it}. In this vein, Haraway's statements are very appropriate to reflect upon works in which data are used, since data visualisations and the ways in which quantitative analysis are conducted and communicated usually do not leave room for alternative interpretations. We aim to steer clear of the idea that the data cannot lie.

Thus, it is worth mentioning that the data-compiling team operates from the south of the Global North and that their salaries are paid by a prestigious funding institution, the European Research Council. These features entail that the means we count on to conduct our research are broader and above the average. Most of the people who took part in the gathering process for our data are \emph{cis} and \emph{white}, and come from middle-class backgrounds. As a consequence of what has been exposed
hitherto, the data are biased by our own standpoint. The data are trustworthy nonetheless, given that they were curated first and introduced then manually, and thus have little to no errors because they
are periodically revised, relying on information obtained from different sources.

The project impulsing data's collection is entitled: ``Social Networks of the Past: Mapping Hispanic and Lusophone Literary Modernity (1989-1959),'' and consists of four research lines feeding the data
collection process. The common framework shared by all the lines is cultural dissemination in Ibero-America during the first half of the twentieth century. We aim at finding the mediators who impulsed the exchange of material (\eg{} books) and immaterial (\eg{} ideas) goods, but also the circulation of people, within the Ibero-American region. We state that these relationships and exchanges are the best ways to understand the contribution that Ibero-America made for the building of artistic modernity. Our main research objects are: translations in cultural and literary historical journals, the International Committee of Intellectual Cooperation of the League of Nations, film criticism in specialised and cultural historical journals, and women who took part in the first Iberoamerican film clubs. The information we gathered about these topics includes human and non-human agents and agencies related to the aforementioned research objects \autocite[2005]{latour-2005}. Following the Actor-Network's tenet named ``\emph{follow the actor},'' all the actors found during our source's harvesting phase were uploaded into our database to retrace the network around our objects. Therefore, in our database we can find data on associations (for example linked to historical journals), on cultural events, and on people (meaning men and women who took part in the first Ibero-American film clubs, collaborated in historical journals, or took part in the activities of the Committee of Intellectual Cooperation, and who participated in events organised by the mentioned associations). We also have data on people with whom the participants of our main objects exchanged letters, and who collaborated in books and articles published by our human actors. Given our data's intrinsic heterogeneity, our sources are varied, as they come from primary sources and secondary bibliography. Primary sources can be located in personal, municipal, and national archives, or historical publications, thus including correspondence, historical documents, pictures, catalogues of exhibitions, lists of members, etc. Secondary sources are also very diverse, including collective volumes or monographs related to the topics of our research project. 

An example of the consequences of using heterogeneous sources is how we fed the records on the relationships based on participation to events. In particular, we do not possess all the data regarding the events organised by all the associations but, in most of the cases, we rely on some events organised by a subset of them, when the information was available. We did not rely on the information concerning all the actors associated with a specific event or association, but included only those we found in the sources we had access to. In other cases, instead, we introduced the data automatically, -- \eg{} for the project on the International Committee of Intellectual Cooperation of the League of Nations, and for the project on translations in cultural journals. Summing up, most of the data we have used in this paper have been manually gathered. During the data harvest, and upload onto our database, each member of the team adopted well established concepts drawn from their own expertise fields to define relevant descriptors. For instance, we have included the descriptor ``cinema'' for describing event's topics, specialisation of a journal, or the main topic of a specific text. These descriptors are written in English -- as this is our group's research communication language -- and in Spanish -- as it is the main vehicular language used by the agents and agencies we work with.

Data heterogeneity reverberates also on the type of relationships that may be established between actors. The plurality of sources and people who have extracted/curated the data may induce some inconsistencies that we tried to mitigate through a constant verification of the information process. The metadata available on each human actor also varies, depending on the information available. In this sense, it is worth mentioning the \emph{invisibilisation} of women in cultural history, and the consequent paucity and sparsity of data on them. Therefore, for the human actors we have considered meta-information on gender, name (first and last), and professional or participation-related ties to magazines, events, publications, and institutions, when available. We have also included personal relationships for the most prominent people in our database. These personal relationships include family relatives and friends, as we consider this information relevant in our framework.

As mentioned above, to address the lack of data on women, we have created a category in our database on relationships, being it personal or professional. Although in some cases we do not have the empirical evidence (\eg{} an article, photo, or the list of collaborators of a magazine) of the professional relationship between two actors, when working with women it is essential to take into account the self-reflexive and personal texts, in which these women often declare their professional or personal relationship with a certain relevant actor in the field, be it cultural or cinematographic.\footnote{See \autocite[2023]{anselmo-2023} and \autocite[2007]{hastie-2007}.} In our case, within the category relationships, we have selected only personal romantic relationships (\eg{} wife-husband, or unmarried couple) and those defined as professional relationships. Such a selection is based on our knowledge of the historical context in which we are working. Our expertise, in fact, leads us to believe that women married to "prominent" men almost always worked in their shadows, which constitutes
an asset of their work. See, for instance, the case of Lola \'Alvarez Bravo (a case study of the ``film club and women'' research line), or many other women who worked as translators (research line on translation). Still, their roles have often not been emphasised as they have been considered secondary, like that of translators. Concerning professional relationships, as we have already mentioned, we have traced them thanks to primary sources (\eg{} self-reflexive texts) or other sources (\eg{} photographs), providing empirical evidence of the existence of a relationship between two (or more) people.

Summing up, the common thread to establish the main criterion to select the relevant data within the whole dataset stored in Nodegoat was that the agents had to have some relationship with the film field. Such a criterion entails the selection of those institutions which played a role in the film field (journals, film clubs, federations, etc.), events and people related to the film field, with the latter having worked or just participated somehow in any of the events, institutions, and journals mentioned. Our decision is based upon the paucity of data and the need to work with indirect data. This means that, if a person attended an event related to the film field, there is a good chance that
she/he took part also in other events, or activities, even if we do not have the data confirming that. Thus, even if for some people we have data only related to a single event they took part in, we will still consider them as actors tied to the film field. In the following, we provide a detailed description of how each channel of interaction has been defined, and subsequently used to generate a network (graph) of people (vertices) interacting with each other (edges). In mathematical terms, a graph (or network) is a discrete mathematical object made of points (called nodes or vertices) connected by edges (or links) \autocite[2017]{latora-2017}.
% 
\newpage
%
\begin{description}
    \item[Collective Bodies (\mycb{})]{Given the list of all the collective bodies available in our database, we:
    % 
    \begin{enumerate}
        \item Select those collective bodies for which the field ``Kind of organization'' contains one of the following terms: '\emph{Cine}', '\emph{cine}', '\emph{Cin\'e}', '\emph{cin\'e}', '\emph{Film}', and '\emph{film}'. This operation returns us all the collective bodies
        classified as '\emph{Film Club}', '\emph{Film Archive}', '\emph{Venue}', and 'Film Distributor'.
        %
        \item Select those collective bodies whose field ``Name'' contains one of the following terms: '\emph{Cine}', '\emph{cine}', '\emph{Cin\'e}', '\emph{cin\'e}', '\emph{Film}', and '\emph{film}'.
    \end{enumerate}
    %
    These steps allow us to select, overall, 422 collective bodies. Then, for each collective body, $i$, we extract the set, $P_{\text{\mycb{}}} (i) = \{ p_1, p_2, \ldots, p_l \}$, of people involved with their activities but classified as prominent figures (\eg{} founders, organisers, managers, treasurers, etc.) or playing a relevant role (for instance, presenters, guests, authors, etc.) within them. Then, given the set $P_{\text{\mycb{}}} (i)$, we add an edge between each pair of its elements.
    } % end item
    %
    \item[Events (\myev{})]{Given the list of all the events available in our database, we:
    %
    \begin{enumerate}
        \item Select those whose field ``Subject'' contains the word `Cinema'.
        %
        \item Select those whose field ``Name'' contains one of the following terms: 'cine', 'film', 'pelicula', and 'proyecci'\footnote{The match is case insensitive.}.
    \end{enumerate}
    %
    These steps allow us to select 99 events. Then, for each event, $i$, we extract the set, $P_{\text{\myev{}}}(i) = \{ p_1, p_2, \ldots, p_l \}$, of its participants, and add a connection between each possible pair of them. It is worth noting that some events' participant list is made of only one -- or any -- person. In such a case, the node corresponding to the sole participant is added to the network but without any edge connecting it.
    } % end of item
    %
    \item[Publications (\mypub{})]{Given the complete list of all publication available in our database, we:
    %
    \begin{enumerate}
        \item Select those whose field ``Genre'' corresponds to `Film'.
        %
        \item Select those whose field ``Title'' contains either the word `film' or the word `cine'\footnote{The match is case insensitive.}.
        %
        \item Select those whose field ``Journal'' contains one of the following terms: `film', and `cine'\footnote{The match is case insensitive.}.
    \end{enumerate}
    %
    Using these criteria, we select 64 publications. From these publications we extract the set of journals where they were published (a total of 16), and create a set $P_{\text{\mypub{}}} (i) = \{ p_1, p_2, \ldots, p_l \}$, of people associated with each journal, $i$, either due to authorship or, for those related to cinema, due to the role they played within them. For the latter, we select among their key characters those whose role is one of the following: `Director', `Editor', `Co-director', `Editor in chief', `Co-founder', and `Founder'.
    } % end of item
    %
    \item[Relationships (\myrel{})]{Given the list of people involved in \mycb{}, \myev{}, or \mypub{} relationships, we extract from our database the list of (pairwise) personal relationships of the type: `Spouse', `Unmarried partner', and `Professional'. This step lead us to include (eventually) people not belonging to the \mycb{}, \myev{}, and \mypub{} relationships. Such a selection criterion translates into 601 connections.
    } % end of item
    %
    \item[Epistolary correspondence (\mycor{})]{Given the list of people involved in \mycb{}, \myev{}, \mypub{}, and \myrel{} relationships, we iterate through our epistolary database and add epistolary interaction only if both the sender and the receiver belong to the aforementioned list of people.
    } % end of item
    %
\end{description}
%

We built one graph for each of the ``\emph{channels}'' of interaction listed above. Each of these networks can be thought either as an independent entity or, alternatively, as a \emph{layer} of a multiplex/multilayer network \autocite[2020]{bianconi-2020}. Table~\ref{tab:net_props} summarises the main structural features of these networks, comprising: the number of nodes, $N$, of edges, $E$, the maximum number of connections per
node, $k_{\max}$, the number of connected components, $N_{\text{comp}}$, (\ie{} pieces of the network in which it is possible to go from every node to every other node within it), the number of isolated nodes, $N_{\text{isol}}$, (\ie{} nodes with no connections), the relative size of the giant component, $S$, (\ie{} the biggest connected component in the graph), and the graph's degeneracy (\ie{} the total number of \kshell{s} available), $D$ \autocite[2017]{latora-2017}. We compute the same indicators also for the network (\myall{}) obtained merging together all the layers.

%
%  TAB DATASETS
%
%
\begin{table}[h!]
    \centering
    \resizebox{0.7\linewidth}{!}{%
%         \begin{tabular}{@{}llllllll@{}}
        \begin{tabular}{rlllllrl}
        \toprule
        \multicolumn{1}{c}{\textbf{Network}} & \multicolumn{1}{c}{$N$} & \multicolumn{1}{c}{$E$} & \multicolumn{1}{c}{$k_{\max}$} & \multicolumn{1}{c}{$N_{\text{comp}}$} & \multicolumn{1}{c}{$N_{\text{isol}}$} & \multicolumn{1}{c}{$S (\%)$} & \multicolumn{1}{c}{$D$}\\\toprule
        \myrel{} & \multirow{5}*{1367} & 6011 & 75 & 783 & 412 & 6.95 & 4\\
        \mycor{} &  & 101 & 91 & 1269 & 1268 & 7.24 & 2\\
        \mycb{} &  & 1315 & 23 & 1136 & 1101 & 1.76 & 20\\
        \mypub{} &  & 244 & 18 & 1323 & 1317 & 1.39 & 16\\
        \myev{} &  & 1898 & 95 & 1258 & 1250 & 7.02 & 58\\
        \rowcolor{black!30!white} \myall{} & 1367 & 4090 & 117 & 373 & 37 & 36.72 & 58\\
        \bottomrule
        \end{tabular}
        %
    } % end resizebox
    \caption{Structural features of the networks used in our study. For each network, we report its number of nodes $N$, of edges $E$, the maximum number of connections per node $k_{\max}$, the number of connected components $N_{\text{comp}}$, the number of isolated nodes $N_{\text{isol}}$, the relative size of the giant component $S$, and the degeneracy $D$. We have highlighted the entries of the \myall{} network.}
    \label{tab:net_props}
\end{table}
%

One way to quantify the redundancy of the information encoded in two layers of a multiplex network is to compute the Jaccard index of their edges' sets \autocite[2020]{bianconi-2020}. Given two sets $\alpha$ and $\beta$, their Jaccard index \autocite[1901]{jaccard-1901}, $J$, can be computed as:
%
\begin{equation}
\label{eq:jaccard}
%
J(\alpha,\beta) = \frac{ \bigl| \alpha \cap \beta \bigr|}{\bigl| \alpha \cup \beta \bigr|}\, .
% 
\end{equation}
%
The numerator $\bigl| \alpha \cap \beta \bigr|$ indicates the size of the set made by the elements belonging both to sets $\alpha$ and $\beta$. The denominator $\bigl| \alpha \cup \beta \bigr|$, instead, denotes the size of the set made by the elements belonging to either set $\alpha$ or $\beta$. The structure of Eq.~\eqref{eq:jaccard} makes that the values of $J(\alpha,\beta)$ can span from zero to one. A value of $J(\alpha,\beta) = 1$ indicates that the two sets have exactly the same elements (\ie{} one is a perfect copy of the other), whereas the case $J(\alpha,\beta) = 0$ indicates that the two sets do not have any element in common (\ie{} they are completely different). A value of $0 < J(\alpha,\beta) < 1$ indicates, instead, a non-zero level of similarity between the two sets; with values of $J(\alpha,\beta)$ close to one (zero) denoting high (low) similarity level.

Apart from studying dyadic interactions, we can leverage the network formalism to identify also mesoscopic structures -- \ie{} coherent structures made by groups of nodes. Amidst the plethora of possible structures \autocite[2019]{lambiotte-2019}, we decided to focus on a type of core-periphery structure called \emph{\kcore{} decomposition} \autocites[2000]{borgatti-2000}[1983]{seidman-1983}[2019]{kong-2019}. Such a technique has proven to be useful in a variety of domains: from identifying and ranking the most influential spreaders in a network \autocites[2007]{carmi-2007}[2010]{kitsak-2010}, to assessing the robustness of mutualistic ecosystem and protein interactions \autocites[2018]{morone-2018}[2019]{kong-2019}.

The \kcore{} decomposition of a network is an iterative pruning process decomposing the system into a set of concentric \kshell{s}. It corresponds to the maximal set of nodes having at least $k$ neighbours within the ­set. In its essence, the algorithm to obtain the \kcore{} decomposition consists in recursively removing the nodes having less than $k$ connections.\footnote{These nodes could have, in principle, more than $k$ neighbours but the removal of other edges decreases de-facto their effective number of connections.} Under this assumption, a \kshell{} is defined as the set of nodes belonging to the $k$-th core but not to the ($k + 1$)-th ­one (\ie{} the partition obtained has a structure akin to an onion). The \kshell{} index, $\ks{}$, of the deepest \kshell{} available for graph $G$ (\ie{} $\max_{G} (\ks)$ is called its degeneracy $D$.

The ability of the \kcore{} decomposition to pinpoint the most influential nodes in the network \autocites[2010]{kitsak-2010}[2018]{morone-2018}, makes it a suitable technique to study the role played by women within our networks. To this aim, we study the fraction of nodes grouped by gender belonging to each \kshell{}. To ensure that the features observed are not the result of mere chance, we decided to compare them with their analogue measured in a randomised version of the network.

Given a network $G$ with $N$ nodes and $E$ edges, $G(N,E)$, the randomisation process generates a network $G^\prime(N,E)$ which, beside the number of nodes and edges, also preserves the number of connections (the so-called degree) of each node. Such a random counterpart can be obtained from network $G$ using the edge swapping rewiring mechanism proposed by \autocite[2018]{fosdick-2018}. The latter constitutes an evolution of the so-called \emph{configuration method} introduced by Molloy and Reed \autocite[1995]{molloy-1995}. In a nutshell, the edge swapping method selects uniformly at random two edges $(a,b)$ and $(c,d)$ of the network $G$, and replaces them -- if they do not exist already, or generate self-loops -- with either the pair or, eventually, with the pair $(a,d),(b,c)$. Both replacements ensure the conservation of the degree of each node. We repeat the edge swap step several times to ensure that the final network $G^\prime(N,E)$ is sufficiently distinct from the original one. We repeat the whole randomisation procedure several times (in our case, 50) to ensure the statistical significance of our results.



\section{Results}

\subsection{Quantifying women marginalisation via \kcore{} decomposition}

One way to quantify the marginalisation of women in the film field is to study their presence -- and abundance -- across the network's \kshell{s}. To this aim, we have built a network -- labelled as \myall{}, -- obtained by collapsing together all the layers (\ie{} the \myrel{}, \mycor{}, \mycb{}, \mypub{}, and \myev{} networks) of the multiplex network. The merge of distinct layers onto a single network implies that the existence of an edge $(i,j)$ in the \myall{} network stems from the existence of the same connection in, at least, one of the layers. Table~\ref{tab:net_props} summarises the structural features of the \myall{} network and; their quick inspection highlights that the number of edges, $E$, and the size of the biggest connected component, $S$, are considerably larger than those of the single layers. These differences, combined together with the smaller number of components, $N_{\text{comp}}$, and of isolated nodes, $N_{\text{isol}}$, tell us that the \myall{} network is considerably more cohesive than the layers' networks taken separately. With this picture in mind, we are now ready to study the \kcore{} decomposition of the \myall{} network. Specifically, for each \kshell{}, we count its number of nodes classified either as men, women or ``unknown,''\footnote{We use this category to indicate the lack in our dataset of meta-information about the gender of a person.} and divide such numbers for the total number of nodes in the \kshell{}.\footnote{Such a procedure ensures that the sum of the fractions of men, women, and unknown nodes of a given \kshell{} is always equal to one.}

The left panel of Fig.~\ref{fig:kcore-comparison-emp_rand-all} displays the fraction of women present in each \kshell{} of the empiric network. We notice how women tend to concentrate mainly in a few \kshell{s}, especially in the outer  -- and shallower -- ones. However, the same analysis performed on the randomised networks delivers a distinct picture. By glancing at the right panel of Fig.~\ref{fig:kcore-comparison-emp_rand-all}, in fact, we notice how women's presence tends to be more widespread across all the \kshell{s}. Also, with the exception of the \kshell{s} corresponding to $\ks{} = \{ 0, 1\}$, the fraction of women appearing in a \kshell{} appears more homogeneous in the random counterpart than in the empiric network. These features make us suspect that there could be some kind of ``force'' exacerbating women's marginalisation.

%
%  FIG 1
%
%
% Figure environment removed
%


\subsection{Decoupling the source of marginalisation}

The marginalisation of women within the \kcore{} decomposition observed in the \myall{} network (displayed in Fig.~\ref{fig:kcore-comparison-emp_rand-all}) raises the question whether such a phenomenon stems directly from a single -- or a few -- channel of interaction or, instead, is due to their interplay. It has been proved, in fact, that the structural features of networks obtained by merging together the layers of a multiplex network can be quite different from those of their layers considered singularly \autocite[2013]{cardillo-2013}. To find whether marginalisation is directly related to the multilayered nature of our system, we leverage the full structure of the multiplex network and repeat our analysis on each of the layers separately.

To understand how overlapped is the information encoded in each channel (layer), we use Eq. (1) to compute the pairwise Jaccard index $J(\alpha,\beta)$ between the set of connections (edges) of layers $\alpha$ and $\beta$. Figure~\ref{fig:jaccard-matrix-layers} shows the heatmap matrix of the values of $J$ computed for all the pairs of layers. Except for the elements of the main diagonal -- corresponding to comparing a layer with itself, thus translating to a value of $J = 1$ -- all the remaining values of $J$ are quite small. This means that the amount of redundant (\ie{} overlapping) information encoded in our layer networks is quite small, ranging between 0.0 and 0.031. Said in other terms: our layer networks are quite distinct from one another. Such differences justify the use of a multiplex network formalism \autocite[2020]{bianconi-2020}.

%
%  FIG 2
%
%
% Figure environment removed
% 
    
    
To delve more into finding the roots of women's marginalisation, we decided to compute for each of the layer networks, as well as for the \myall{} one, the difference between the empiric fraction of women belonging to a given \kshell{}, $f_{W}^{e}$, and the same quantity computed in the randomised networks, $f_{W}^{r}$ (see Figure~\ref{fig:kcore-comparison-emp_rand-layers}). More specifically, for each network $G$ we compute the difference $\Delta f_{W} = f_{W}^{e} - f_{W}^{r}$ for each \kshell{}, $\ks{}$, available. Given a \kshell{} with index $\ks{} = \gamma$, a positive difference (\ie{} $\Delta f_{W} (\ks{} = \gamma) > 0$) indicates that for that \kshell{} there are (in proportion) more women in the empiric network than in the random counterpart. Conversely, a negative difference (\ie{} $\Delta f_{W} (\ks{} = \gamma) < 0$) indicates that -- on average -- the random counterpart has in proportion more women than the empiric case. Finally, a difference equal to zero (\ie{} $\Delta f_{W} ( \ks{} = \gamma) = 0$) indicates that the fraction of women in that \kshell{} is the same in both the empiric and the random networks.

%
%  FIG 3
%
%
% Figure environment removed
% 


The visual inspection of Figure~\ref{fig:kcore-comparison-emp_rand-layers} highlights some remarkable features. The first is that random networks tend to have a shallower \kshell{} structure than their empiric counterparts. In Fig.~\ref{fig:kcore-comparison-emp_rand-layers} the value of the degeneracy for random networks, $D^{r}$, is indicated by the position of the red star. We notice how for all networks -- except the \mycor{} one -- $D^{r}$ is smaller than the empiric case (see also Table~\ref{tab:net_props}). Such a shrinkage has been observed already in empirical and synthetic networks \autocite[2020]{malvestio-2020}, and is due to the break up of mesoscale structures (\eg{} communities) operated by the random rewiring mechanism. Another feature displayed by Fig.~\ref{fig:kcore-comparison-emp_rand-layers} is that marginalisation occurs mainly in the events' (\myev{}) layer, as opposed to its weaker presence in those layers corresponding to ``private'' interaction's channels such as: personal relationships (\myrel{}) or epistolary correspondence (\mycor{}). This dichotomy suggests that public interactions were keener to foster women's marginalisation than those occurring in the private sphere. Finally, we observe how the collapse of the available interactions' channels onto a single one gives rise to a richer marginalisation scenario than that shown by the phenomenology observed when the layers' networks are studied independently.


\section{Conclusions}

As we stated in the introduction, the method proposed in this manuscript can help to understand quantitatively the marginalisation of women in the film field in Ibero-America during the first half of the twentieth century. The novel and valuable nature of this joint research effort does not lie in a discovery but, rather, in a methodological proposal combining quantitative tools of Social Network Analysis and a feminist perspective on historical events. Notwithstanding, there exist other academic texts reflecting on the use of data from a feminist perspective, like the book \emph{Data Feminism} (2020), which constitutes a pillar of our work. At the same time, there exist an increasing bunch of studies analysing data of several cultural and social fields through the lenses of gender and social network analysis \autocites[2010]{smith_doerr-2010}[2015]{lutter-2015}[2020]{verhoeven-2020}[2021]{morgan-2021}[2022]{macedo-2022}[2022]{wapman-2022}[2023]{herrera_guzman-2023}. In this sense, and in comparison with the works cited above, the exceptionality of our research lies in the use of historical data.

One of the main results of our work is the empirical confirmation of existing qualitative knowledge, namely: that during the first decades of the twentieth century, women were rather marginalised from the cultural public sphere. This phenomenon is easily traceable in the literary and publishing world \autocites[1995]{king-1995}[2009]{kowaleski_wallace-2009}. However, the same conclusion holds also for history of cinema in the same period, as publications and the creation of theoretical film knowledge through publications were particularly relevant both in the emergence of film cultures \autocite[2017]{navitski-2017}, as well as in the process of institutionalisation of the film field \autocite[1979]{bourdieu-1979}. The theoretical and historical knowledge generated in spaces of collective participation, such as film clubs, ended up being signed by individual authors. Likewise, the events organised around the medium -- like competitions and festivals -- and their ``visible figures'' would play a cornerstone role in the institutionalisation of a field that was not yet professionalised. It is in this context that our analysis yields the expected results, what Mary Beard \autocite[2017]{beard-2017} pointed out in her classic text, \ie{} that certain mechanisms in Western culture, from its foundation, have hindered and continue to hinder women's access to the spheres of power. This means that women played more central roles in activities happening within the private spheres.

We have reached this conclusion after using affirmative action strategies in the data collection process, as well as in the data's systematisation and analysis. The first strategy was to use the Latourian method for building networks and start the relationships' tracing from different historical female agents. After that, we followed the Latourian method defined as "follow the actor" consisting in: we chose some case studies around which we built networks in the sociological-relational sense of
the term. The selection of the case studies was the result of qualitative historical research and the certainty, after qualitative analysis, that those cultural mediators were relevant within the historical period of analysis. Despite the differences between the actor-network and the networks from graph theory, there are certain similarities allowing the use of graphs to study social phenomena, as
proposed by Venturini \etal{} \autocite[2019]{venturini-2019}. Here we go one step further and consider that, taking into account the differences in the conceptualisation of what a network is, and being aware of the steps involved in the assimilation of methods, we can leverage tools from both approaches to reach our goal: to analyse the marginalisation of women in a specific historical-cultural context.

Another strategy we used to trace women in the data structuring process was to add one category named ``relationships.'' The main purpose of this strategy was to introduce information on personal and work relationships mentioned in underexplored historical sources, whose legitimacy has been neglected until now. After conducting qualitative research on our framework, we discovered that most of the women married to men playing a prominent role in the cultural field carried out relevant cultural activities as well.\footnote{For instance, the work of Lola \'Alvarez Bravo was considered irrelevant for Mexican culture until the 80s, \ie{} the decade of her death.} Furthermore, some evidence, such as photographs from the period, showed us that the participation of women in spaces considered by historiography as eminently masculine was actually quite high.\footnote{For instance, see the picture of the second documented film club in Montevideo (Uruguay) published on page 16 of the number 17 of the journal \emph{Radio actualidad} (Uruguay)~\autocite{radio_actualidad-1937}.} However, as a consequence of the structural chauvinism affecting historiography, this evidence had not been taken into account when writing histories about Ibero-American cinema.

Despite the adoption of all the aforementioned compensation strategies, the data used to carry out our study remain still limited and very heterogeneous. The former, as expected, stems from the little attention these women had received from historiography. The latter, instead, is relevant precisely because only by scraping data from different fields and disciplines we can prove the relevance of the cultural mediators we are studying \autocite[2018]{roig_sanz-2018}. It was such heterogeneity that led us to decide to use the \kcore{} decomposition method. It is known, in fact, that the number of connections involving a node (\ie{} its \emph{degree}) is not a good proxy of its ability to diffuse
information efficiently \autocite[2010]{kitsak-2010}.

Our choice to work with a multi-layered network has been based on two factors: first, preserving all the channels of interaction we deemed relevant to avoid losing any of the facets or roles that these women played in the cultural field as mediators. Second, to see -- a posteriori, -- the differences in the way each layer of interaction was structured. We speak of \emph{cultural field} as the non-institutionalisation of the film field often implied that cultural activities related to film were organised alongside with other literary, artistic, and intellectual activities. Therefore, although the data we used concerned the field of cinema, their relationship with the latter could often be indirect. For instance, when including the data about a piece on a film critique (namely: the author/s' name/s), we added also the names of people involved not only with the volume where the critique was published, but also with the journal where the piece was published. In this sense, the use of a multi-layer network formalism allowed us to preserve information about the relationships that women established through different channels, such as (among others): publications, their involvement in institutions, or their participation in events.

Then, we performed the \kcore{} decomposition on all the layers separately, as well as on their projection onto a single layer. Our intuition was that, given the nature of our data and of our object of study, different interaction layers would probably display different behaviours, as we saw indeed in Figure~\ref{fig:kcore-comparison-emp_rand-layers}. In addition, the intensity of the marginalisation effect was amplified in the projection network, albeit this was not the case when each layer has been analysed individually. In fact, there are layers where this effect did not occur (\eg{} the publications -- \mypub{} -- network). When running the \kcore{} decomposition on separate layers (networks), we observed that the marginalisation effect is stronger in the \myev{} (events) network. In such a layer, women fall into the outer \kshell{s} more often than in the random network counterpart. The strength of the marginalisation effect is stronger in the \myev{} layer because its \kshell{} decomposition has more shells (\ie{} its degeneracy $D$ is higher) than those of the other layers.

One possible explanation behind the stronger marginalisation observed in the \myev{} network is that, by definition, the network corresponding to one event is a \emph{complete graph} or a \emph{clique} (\ie{} a graph where each node is connected to every other node). The edge replacement step adopted by our randomisation process destroys this type of cliques, thus making the whole network's \kcore{} structure shallower \autocite[2020]{malvestio-2020}.

Another possible explanation of the marginalisation phenomenon concerns the nature of the data. Despite the mitigation strategies adopted, our data for public activities like events come from publicly available information (mostly newspapers and magazines). The press releases advertising or reviewing events (our information's source) were less keen to include women in the list of participants, thus making their appearance more marginal than men's. Finally, another explanation of the marginalisation effect has to do with the characteristics of the people's collaboration network. As the majority of unacknowledged women of our dataset come from the lists of participants to associations, they form a sort of ``private club'' inflating the amount of woman-woman connections \autocite[2002]{newman-2002}.\footnote{This phenomenon goes under the name of \emph{assortative mixing}, and has been observed in several types of networks, including social ones (\autocite[2020]{verhoeven-2020} and references therein).} However, such a correlation gets weaker in the randomised network, thus pushing women towards the outer \kshell{s}.

Despite the appeal and advantages of using a quantitative approach to complement qualitative and historical insight, there are some caveats. For instance, the \kcore{} decomposition assumes that the network under scrutiny could be made of several components, but their structure should be complex enough to ensure a moderate value of degeneracy. This means that the insight provided by the \kcore{} decomposition is directly proportional to the total number of \kshell{s} available. For example, the \kcore{} decomposition of a network made by a set of disjoint complete graphs (\eg{} when many events do not have any participant in common) returns nothing else than the members of these graphs (\ie{} the event's participants) grouped by their size (\ie{} their number of attendants). This implies that nodes less entwined with the ``bulk'' of the system, albeit interacting a lot with smaller groups, will be relegated anyway to the outer \kshell{s} of the network. Working with marginalised minorities -- like women in film culture -- which have only scant bonds with the rest of the system, might decrease the information extractable via the \kcore{} decomposition. Another shortcoming stems from the resilience of cliques to the \kcore{'s} iterative pruning process. Such a resistance makes that nodes belonging to these cliques (events) fall into inner \kshell{s} thus placing nodes (people) belonging to bigger cliques (\ie{} attending events with many participants) into deeper \kshell{s}, even though these cliques do not share any (or only a few) nodes.

Given these limitations, one possible workaround could be to extract the so-called \emph{community structure} of the network \autocite[2019]{lambiotte-2019}, and study how the latter overlaps with the innermost \kshell{s} \autocite[2020]{malvestio-2020}. For instance, some of the events organised by the Lyceum club might translate into specific communities falling in the innermost \kshell{s}.

Despite the limitations of the methodology proposed in this manuscript, and the need to continue pushing along this direction to reach a suitable method of analysis, we believe that our attempt is still valuable to study quantitatively minorities, like women, and the social spaces they occupy. Our methodological proposal, albeit it can be used in a contemporary context, has been developed specifically for the study of historical data on a minority group. In a nutshell, we believe that the collaboration between specialists with different expertise and perspective hides a great potential that has yet to be discovered, which could revolutionise the field of humanities and generate spill overs paving the way for new and exciting challenges in network science.


%%%% ACK AND OTHER STUFF %%%%%

\subsection*{Acknowledgements}

The authors thank Ventsislav Ikoff for his support in extracting and curating the data, and the help of Diana Roig-Sanz, and Malte Hagener for their comments during the preliminary stages of this work.\newline\newline
%
Numerical analysis has been carried out using the NumPy and NetworkX Python packages \autocites[2011]{van_der_walt-2011}[2008]{hagberg-2008}. Graphics have been prepared using the Matplotlib Python package \autocite[2017]{hunter-2007}.


%%%%% BIBLIOGRAPHY %%%%%%


% \bibliographystyle{plainnat}
% \bibliographystyle{apalike}
% \autocite{*}
% \bibliography{biblio}

\nocite{*}
\printbibliography


\end{document}
# end of file
