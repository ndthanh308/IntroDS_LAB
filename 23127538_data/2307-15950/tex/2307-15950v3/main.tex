
% % \usepackage[numbers]{natbib}
% %\usepackage[authoryear]{natbib}
% \usepackage[authoryear,longnamesfirst]{natbib}
% \usepackage{siunitx}
% \usepackage{color}
% \usepackage{tabularx}
% \usepackage{booktabs}
% %%%Author definitions
% \def\tsc#1{\csdef{#1}{\textsc{\lowercase{#1}}\xspace}}
% \tsc{WGM}
% \tsc{QE}
% \tsc{EP}
% \tsc{PMS}
% \tsc{BEC}
% \tsc{DE}
%%%

% Uncomment and use as if needed
%\newtheorem{theorem}{Theorem}
%\newtheorem{lemma}[theorem]{Lemma}
%\newdefinition{rmk}{Remark}
%\newproof{pf}{Proof}
%\newproof{pot}{Proof of Theorem \ref{thm}}

\documentclass[lettersize,journal]{IEEEtran}
% \usepackage[utf8]{inputenc}
\usepackage{amsmath,amsfonts,amssymb}
% \usepackage{algorithmic}
% \usepackage{algorithm}
\usepackage[ruled]{algorithm2e}
\usepackage{array}
\usepackage{bbm}
\usepackage[caption=false,font=normalsize,labelfont=sf,textfont=sf]{subfig}
\usepackage{textcomp}
\usepackage{stfloats}
\usepackage{url}
\usepackage{verbatim}
\usepackage{graphicx}
\usepackage{booktabs}
\usepackage{multirow} 
% \usepackage{dblfnote}
\usepackage{cite}
\hyphenation{op-tical net-works semi-conduc-tor IEEE-Xplore}
\usepackage{hyperref} 

\usepackage{siunitx}
\usepackage{color}
\usepackage{tabularx}
% \usepackage{booktabs}

\makeatletter
\newcommand{\rmnum}[1]{\romannumeral #1}
\newcommand{\Rmnum}[1]{\expandafter\@slowromancap\romannumeral #1@}
\makeatother

\begin{document}
% 

% Title of the document
\title{
Human-Like Implicit Intention Expression for Autonomous Driving Motion Planning: A Method Based on Learning Human Intention Priors
% Teaching Autonomous Vehicles to Express Interaction Intent at Unsignalized Intersections: A Human-Prior-Based Trajectory Planning Approach
% In
}

\author{Jiaqi Liu,~\IEEEmembership{Student Member,~IEEE,} Xiao Qi, Ying Ni, Jian Sun, and Peng Hang,~\IEEEmembership{Member,~IEEE}
        % <-this % stops a space
\thanks{This work was supported in part by the National Key R\&D Program of China (2022YFB2502901), the National Natural Science Foundation of China (52125208, 52232015, 52272313), the Young Elite Scientists Sponsorship Program by CAST (2022QNRC001) and the Fundamental Research Funds for the Central Universities.}% <-this % stops a space
\thanks{Jiaqi Liu,  Xiao Qi, Ying Ni, Jian Sun and Peng Hang  are with the Department of
Traffic Engineering and Key Laboratory of Road and Traffic Engineering,
Ministry of Education, Tongji University, Shanghai 201804, China. (e-mail: \{liujiaqi13, qixaio, ying\_ni, sunjian, hangpeng \}@tongji.edu.cn)}

\thanks{Corresponding author: Peng Hang}
}

% The paper headers
\markboth{}%
{Shell \MakeLowercase{\textit{et al.}}: A Sample Article Using IEEEtran.cls for IEEE Journals}



\maketitle 

% Abstract
\begin{abstract}
One of the key factors determining whether autonomous vehicles (AVs) can be seamlessly integrated into existing traffic systems is their ability to interact smoothly and efficiently with human drivers and communicate their intentions. While many studies have focused on enhancing AVs' human-like interaction and communication capabilities at the behavioral decision-making level, a significant gap remains between the actual motion trajectories of AVs and the psychological expectations of human drivers. This discrepancy can seriously affect the safety and efficiency of AV-HV (Autonomous Vehicle-Human Vehicle) interactions. To address these challenges, we propose a motion planning method for AVs that incorporates implicit intention expression. First, we construct a trajectory space constraint based on human implicit intention priors, compressing and pruning the trajectory space to generate candidate motion trajectories that consider intention expression. We then apply maximum entropy inverse reinforcement learning to learn and estimate human trajectory preferences, constructing a reward function that represents the cognitive characteristics of drivers. Finally, using a Boltzmann distribution, we establish a probabilistic distribution of candidate trajectories based on the reward obtained, selecting human-like trajectory actions. We validated our approach on a real trajectory dataset and compared it with several baseline methods. The results demonstrate that our method excels in human-likeness, intention expression capability, and computational efficiency.
\end{abstract}

\begin{IEEEkeywords}
Autonomous Vehicles, Human-like Trajectory Planning, Intent Expression, Inverse Reinforcement Learning
\end{IEEEkeywords}


\section{Introduction}
The emergence of autonomous vehicles (AVs) marks a revolutionary era in transportation, promising widespread transformation across the entire system. In the dynamic landscape of mixed traffic scenarios, ensuring the harmonious coexistence of autonomous vehicles and human-driven vehicles (HVs) is crucial~\cite{toghi2021cooperative,wang2022social}.
A key factor in integrating AVs into existing traffic systems is their ability to interact smoothly and intelligibly with human drivers. Unlike human drivers, who can engage in flexible and diverse social interactions, current AVs still lack effective communication and interaction capabilities\cite{toghi2022social,liu2024enhancing}. In recent years, enhancing the social interaction abilities of AVs has become a focal point for many researchers\cite{schwarting2019social,crosato2022interaction}. However, most studies have concentrated on the behavioral decision-making level. While AVs can exhibit good interactive abilities in terms of decision-making intentions, there remains a significant gap between the actual motion trajectories of AVs and the expectations of human drivers.

In real-world traffic scenarios, motion trajectories serve as a critical means for human drivers to interact and communicate. For instance, drivers can subtly express their intentions during interactions through slight trajectory shifts and speed adjustments\cite{de2013road,lee2021road}. However, existing AV motion planning algorithms often fail to account for these factors. For example, traditional discrete optimization-based algorithms can satisfy requirements such as ease of construction, smooth trajectories, and continuous differentiability. These algorithms are widely used in AV motion planning, enabling simple trajectory actions like collision avoidance and lane changes in high-speed following and lane-changing scenarios. Nevertheless, they struggle to accurately mimic the real driving trajectories of human drivers, leading to significant discrepancies between planned and actual human trajectories. When humans interact with AVs controlled by such planning algorithms, these discrepancies can result in mismatched psychological expectations, thereby affecting the efficiency and safety of interactions \cite{wang2022social,valiente2023prediction}.


% The emergence of Autonomous Vehicles (AVs) marks a revolutionary era in transportation, promising extensive transformations throughout the entire system. In the dynamic landscape of mixed-traffic scenarios, ensuring the harmonious coexistence of AVs and Human-Driven Vehicles (HVs) stands as a matter of paramount significance~\cite{toghi2021cooperative,wang2022social}. 
% The ability of AVs to engage in fluid, understandable interactions with humans is a key factor in their integration into the existing traffic system. Unlike human drivers who can engage in flexible and varied forms of social interaction to communicate, current AVs still lack capabilities in interactive communication\cite{toghi2022social}. In recent years, enhancing the social interaction abilities of autonomous driving has become a focal point for many researchers\cite{schwarting2019social,crosato2022interaction}. Most studies, however, have concentrated on high-level decision-making capabilities. 
% In the real-world traffic scenario, movement trajectories also serve as a vital means for drivers to interact and communicate (e.g., trajectory deviation, speed adjustment) \cite{de2013road,lee2021road}. Present-day AVs lack the capacity to express social intentions at the level of trajectory behavior. Their driving actions often appear ambiguous and difficult for humans to understand \cite{wang2022social}, posing greater challenges for interaction safety in mixed driving environments \cite{wang2022social,valiente2023prediction}.

In response to these issues, some researchers, such as Huang\cite{huang2021driving} and Song\cite{Song1797}, have introduced learning-based trajectory planning techniques aimed at mimicking human-like behavior by drawing inspiration from and imitating expert human trajectories. While these methods have the potential to utilize real HV trajectories to guide AV trajectory planning, they still leave several critical challenges unresolved. One key challenge lies in the fact that high-level decision intentions are crucial in constraining the trajectory planning space. Existing learning methods typically derive AV decisions from trajectory outcomes, which contradicts the fundamental principle of 'decision before execution'. Moreover, these methods often overlook the intention expression embedded in trajectory movements influenced by high-level decisions. This oversight is particularly evident in complex intersection scenarios where different trajectories are involved.

The intention information embedded within trajectory movements plays a crucial role in interactions, leading to significant differences in trajectory strategies and expected spaces under varying decisions. This complexity and importance are further heightened when AVs navigate intersections, which are among the most demanding interaction scenarios. Such complexity often hinders AVs from executing unprotected left turns with the finesse of a human driver\cite{zhao2023unprotected}.

To address these challenges, we propose an AV motion planning framework that incorporates implicit intention expression. This framework learns prior intentions from human driving data and constructs expected trajectory space constraints based on these learned intentions, generating candidate motion trajectories that account for intention expression. This process enables the compression and pruning of the trajectory space.
We then design a reward function that considers efficiency, comfort, and safety to represent the cognitive characteristics of drivers. Maximum Entropy Inverse Reinforcement Learning (ME-IRL) is employed to learn and evaluate human trajectory preferences, and we establish a probability distribution for trajectory selection influenced by the Boltzmann distribution to simulate the human decision-making process.
We applied and tested our method using the most representative unprotected left-turn task at unsignalized intersections. Using a real-world trajectory dataset, we validated our algorithm in both a simulation environment and a human-in-the-loop driving simulator. Comparisons with several baseline algorithms demonstrate that our method excels in human-likeness, intention expression capability, and computational efficiency.

In summary, our contributions are shown as follows:
\begin{itemize}
    \item We propose an AV motion planning method with implicit intention expression, capable of generating motion trajectories that consider intention expression by learning prior knowledge from human trajectories.
    \item Maximum Entropy Inverse Reinforcement Learning is used to learn and evaluate human driving trajectory preferences, with a reward function designed to account for driver cognitive characteristics.
    \item Our method has been validated in both simulation and human-in-the-loop driving environments, with experimental results showing superior overall performance.
\end{itemize}

The rest of the paper is organized as follows: Section~\ref{sec:2} summarizes the recent related works. 
In section~\ref{sec:3}, the trajectory planning approach we proposed is described. In section \ref{sec:4}, the  experiments are introduced and the results are analyzed. Finally, this paper is concluded in section \ref{sec:5}.

\section{Related Works}
\label{sec:2}
\subsection{Intention Expression Between Vehicles}
Research has demonstrated that intention expression and communication between drivers can significantly enhance traffic efficiency and safety, especially in complex driving scenarios such as unsignalized intersections without clear right-of-way\cite{lee2021road,de2013road}. These forms of intention interaction typically include both explicit and implicit communication methods\cite{de2013road}. Explicit methods involve drivers actively sending or displaying specific signals or information to other vehicles, such as gestures, flashing lights, honking, or eye contact. Implicit methods, on the other hand, involve vehicles subtly conveying their intentions or status to others through their driving behavior, such as speed changes, trajectory shifts, or following distance. Human drivers generally prefer and trust implicit communication methods for completing interactive driving tasks.

Studies have also found that AVs can improve interaction efficiency and safety with human drivers by implicitly expressing intentions through adjustments in approach speed, distance, lateral displacement, and braking position. Rettenmaier et al. \cite{rettenmaier2021communication} investigated the impact of AV speed and lateral displacement on vehicle interaction in bottleneck road sections, and their results showed that when AVs included lateral displacement in their movements, participants experienced better traffic efficiency and higher safety. Similarly, left-turning vehicles at intersections can use implicit communication methods such as early lateral displacement\cite{YangLan2022Human}, delayed lateral displacement\cite{ma2017two2}, or early yaw angle deviation\cite{rettenmaier2021communication} in their trajectories to signal whether they intend to proceed or yield, making their intentions clearer and easier for the other party to understand.
While the above studies have explored the effects of implicit intention expression, few have considered incorporating intention expression into trajectory planning. This oversight means that AVs' planned motion trajectories might confuse human drivers, potentially compromising safety and efficiency.


\subsection{Trajectory Planning in Autonomous Vehicle Systems}
The development of trajectory planning algorithms is fundamental in the hierarchical structure of autonomous driving systems, especially post the establishment of decision intent during motion. These algorithms are essential for generating paths that are not only collision-free but also executable \cite{hang2023brain}.

A prevalent approach in trajectory planning is the use of discrete optimization methods, exemplified by the Frenet trajectory planning method \cite{werling2010optimal}. This method relies on the Frenet coordinate framework, simplifying the trajectory planning into two-dimensional spatial-temporal (S-T) and lateral-temporal (L-T) problems. Building upon this, various researchers \cite{zhang2020optimal,hu2022probabilistic,hu2018dynamic} have sought to refine and optimize the Frenet method. For instance, the team at Apollo \cite{zhang2020optimal} adapted the Frenet approach to devise a quadratic optimization strategy, accommodating the non-holonomic constraints of vehicles. Hu et al. \cite{hu2018dynamic} developed a dynamic path planning method, incorporating the Gaussian convolution algorithm to evaluate collision risks with both static and dynamic obstacles.

Alternative trajectory characterization techniques, such as the Archimedean spiral \cite{ma2017two} and the Bezier curve \cite{zhou2022autonomous}, are also employed, particularly useful in scenarios like intersection turns or high-speed lane changes. These methods, however, demand additional parameters, including control points and parameters for trajectory. Unlike polynomial methods, which inherently integrate linear speed and acceleration outcomes, these techniques require further resolution of motion planning parameters.

Compared to direct motion planning methods that yield discrete kinematic control parameters instantaneously, discrete optimization-based trajectory planning algorithms offer the advantage of generating and planning rational operational trajectories within a controlled generation space. This approach effectively addresses challenges such as the absence of comprehensive planning, trajectory smoothness, and the overlooking of trajectory diversity and the crucial aspect of intent expression.

While discrete optimization-based trajectory planning methods can satisfy basic motion planning needs, significant disparities persist between its planned trajectories and human trajectories\cite{huang2021driving}. Moreover, its inability to express intent impedes effective interaction between Avs and HVs\cite{wang2022social}. In our work, the intention-expressing ability will be added into the trajectory planning framework to enhance the social interaction performance of AVs.


\subsection{Inverse Reinforcement Learning for Motion Planning}

In numerous existing studies \cite{hu2022probabilistic,werling2010optimal}, feature weights in the reward function of trajectory planning are assigned manually or optimized through numerous experiments, resulting in a lack of alignment with driver cognitive characteristics. Inverse Reinforcement Learning (IRL) offers a solution to this issue by reconstructing the reward function through learning from expert example trajectories \cite{arora2021survey}. IRL has been extensively applied to problems such as route planning \cite{ziebart2008maximum,wulfmeier2017large} and the learning of human behavior trajectory features \cite{abbeel2004apprenticeship}. Notable IRL methods include Maximum Entropy IRL \cite{ziebart2008maximum,wulfmeier2017large}, Apprentice IRL \cite{abbeel2004apprenticeship}, and Bayesian IRL \cite{ramachandran2007bayesian}. Among these, ME-IRL holds substantial benefits in studying human expert trajectory behavior, including uncertainty modeling, learning behavior diversity, and robust generalization ability.

Some studies have combined discrete optimization-based trajectory planning methods with ME-IRL to learn human driving behavior and trajectory selection mechanisms. Wu et al. \cite{wu2020efficient} proposed a sampling-based continuous-domain ME-IRL algorithm that improves sampling efficiency through spatiotemporal decoupling and adaptive sampling, effectively learning human driving behavior. Huang et al.\cite{hang2020human} further incorporated vehicle-to-vehicle interactions by describing trajectory distribution using Boltzmann noise theory, assuming an exponential relationship between trajectory probability and reward, and ultimately developed a highway driving trajectory planning model based on ME-IRL.

However, simply using IRL to learn the reward function weights is not sufficient to fully express implicit intention. In our research, we first extract the priors of human drivers and utilize them to shape the expected trajectory space, thereby formulating a social trajectory set. Then, ME-IRL is employed to learn the human driving behavior and trajectory selection mechanism.

% Figure environment removed

\section{Methodology}
\label{sec:3}
In this section, we present our framework for social trajectory planning. We begin by providing an overview of the entire framework, followed by an analysis of the left-turn trajectory characteristics of human drivers. Finally, our trajectory generation method is detailed.


\subsection{Overview of Framework}
The overall trajectory planning framework is illustrated in Fig.~\ref{fig:framework}. The specific steps of the trajectory planning method are as follows: (1) Based on the SinD dataset \cite{xu2022drone}, we analyze the intention expression characteristics of human trajectories in left-turn scenarios to provide prior knowledge from human drivers for trajectory generation; (2) Using decision intentions, such as proceeding or yielding, and combining them with the prior knowledge of trajectory intention expression, multiple candidate trajectories are generated to form the expected decision trajectory space; (3) Next, ME-IRL is employed to learn and model human driving behavior in evaluating and selecting trajectories, and design a reward function that expresses trajectory characteristics in terms of traffic efficiency, driving comfort, and interaction safety; (4) We then establish a probability distribution for the candidate trajectories based on the Boltzmann distribution, using a Boltzmann noise rationality model to simulate driver behavior in trajectory selection and complete the trajectory planning process.

After completing the algorithm training, the method will be tested and validated through simulation experiments and human-in-the-loop driving simulation experiments.
% The comprehensive left-turn trajectory planning framework proposed in this study is illustrated in Fig.~\ref{fig:framework}. The specific process and underlying philosophy of our trajectory planning method are as follows:

% We scrutinize the characteristics of human trajectory intent expression in left-turn scenarios based on the human driving dataset, SIND  \cite{xu2022drone}, providing prior knowledge of human drivers for trajectory generation.
% Given the decision-making intent of proceeding or yielding, coupled with prior human knowledge of trajectory intent expression, we generate multiple candidate trajectories to form an expected trajectory space for decision-making.
% We then formulate a trajectory reward function to evaluate the features of all generated candidate trajectories considering aspects such as traffic efficiency, driving comfort, and dynamic interactive safety.
% Subsequently, we establish a probability distribution of candidate trajectories based on the Boltzmann distribution, employing the Boltzmann noise rational model to emulate driver trajectory selection behavior.
% Lastly, we learn the criteria for trajectory evaluation and selection during human driving via the method of ME-IRL.
% Upon completing the algorithm training, the algorithm will be tested and validated through simulated experiments and human-in-the-loop driving simulation experiments.

\subsection{Human Intent Expression Characteristics Analysis}
Unlike existing autonomous driving algorithms, human drivers often proactively express their intentions implicitly to other drivers through speed changes and trajectory shifts in certain interaction scenarios. At intersections, where movement constraints are minimal and flexible, left-turning HVs can express their decision intentions clearly through distinct trajectory maneuvers during interactions. We use the unprotected left-turn interaction as a typical scenario to analyze the intention expression characteristics of human trajectories using the SinD dataset. The distribution of unprotected left-turn interaction trajectories in the SinD dataset is shown in Fig.~\ref{fig:Traj_HV_Prior}. The orange lines represent yield trajectories, while the blue lines represent proceed trajectories. It can be observed that human drivers adopt different intention expression methods for left-turn trajectories based on different decisions:
\begin{itemize}
    \item The proceed trajectories, as shown in Fig.~\ref{fig:Traj_HV_Prior}(a),  involve pre-turn behaviors, such as early steering and lateral displacement on the upstream segment, allowing drivers to complete the turn quickly through direct steering within the intersection.
    \item In Fig.~\ref{fig:Traj_HV_Prior}(b), yield trajectories maintain a straight path without significant displacement on the upstream segment and complete the turn within the intersection using a two-stage maneuver: first continuing straight and then making the turn.
\end{itemize}

Based on these intention expression methods corresponding to different decisions, we will establish an expected trajectory space constraint that allows for implicit intention expression, ensuring that the trajectory operates within the human-expected space and possesses the ability to convey intention.

% Differing from existing autonomous driving algorithms, human drivers often subtly indicate their intentions in interactive scenarios through adjustments in speed and trajectory deviations. In the less constrained intersection driving space, HVs can convey their intent through varying trajectory methods during left turns. Our analysis focuses on unprotected left-turn interactions, using the SIND dataset to examine human trajectory intent expression. We observe that:
% \begin{itemize}
%     \item The preceding trajectory, as shown in the left part of Fig.~\ref{fig:Traj_HV_Prior}, is characterized by a pre-turn maneuver and lateral displacement in the upstream section, leading to a swift, direct turn within the intersection.
%     \item The yielding trajectory, in contrast, follows a straight path without significant deviation upstream, completing the turn in a two-stage manner within the intersection.
% \end{itemize}

% By synthesizing these intent expression strategies, we aim to establish a trajectory space that not only conforms to but also enhances implicit intent expression, aligning the trajectory with human expectational space and endowing it with the capability to express intent.



\subsection{Left-turn Candidate Trajectories Generation}

To efficiently generate smooth and flexible trajectories at intersections, we assume that drivers typically make short-term plans based on the current interaction motion state, considering both lateral and longitudinal expected target points. Since the polynomial trajectory generation method can independently establish expressions for longitudinal and lateral displacement, and it has advantages like smooth trajectories, curvature continuity, and derivability of mathematical expressions, we employ the polynomial method to generate left-turn candidate trajectories based on the Frenet coordinate system.

We use polynomials to fit the expected trajectory, decomposing the three-dimensional optimization problem into two separate optimization problems in the longitudinal direction \(s\) and the lateral direction \(l\), represented as \(s(t)\) and \(l(t)\), respectively. The longitudinal and lateral trajectories are expressed using a quartic polynomial and a quintic polynomial, respectively, as follows:

\begin{equation}
    \begin{aligned}
    s(t) &= a_0 + a_1 t + a_2 t^2 + a_3 t^3 + a_4 t^4, \\
    l(t) &= b_0 + b_1 t + b_2 t^2 + b_3 t^3 + b_4 t^4 + b_5 t^5.
    \end{aligned}
\end{equation}



The boundary conditions for trajectory generation using the polynomial method can be represented as follows:

\begin{align}
\begin{cases}
s\left(t_0\right)=s_0, \dot{s}\left(t_0\right)=v_{s0}, \ddot{s}\left(t_0\right)=a_{s0}, \\
\dot{s}\left(T\right)=v_{sT}, \ddot{s}\left(T\right)=a_{sT}\\
l\left(t_0\right)=l_0, \dot{l}\left(t_0\right)=v_{l0}, \ddot{l}\left(t_0\right)=a_{l0}, l\left(T\right)=l_T,\\
\dot{l}\left(T\right)=v_{lT}, \ddot{l}\left(T\right)=a_{lT}
\end{cases}
\end{align}
where $s_0$, $v_{s_0}$, and $a_{s_0}$ represent the longitudinal displacement, velocity, and acceleration at the initial state,  respectively. $l_0$, $v_{l0}$, and $a_{l_0}$ denote the lateral displacement, velocity, and acceleration at the initial state. $v_{s_T}$, $a_{s_T}$ represent the longitudinal velocity and acceleration at the terminal state. Similarly, $l_T$, $v_{l_T}$, and $a_{l_T}$ denote the lateral displacement, velocity, and acceleration at the terminal state.

Meanwhile, left-turning usually requires determining a reasonable sampling space to generate a set of trajectories, from which the optimal trajectory is selected. The sampling space for generating the trajectories is determined by the range of the terminal state parameters in the boundary conditions. To avoid the dimension explosion problem caused by too many parameters, we set the terminal moment longitudinal acceleration $a_{s_T}=0$ and lateral acceleration $a_{l_T}=0$. Then, we sample the trajectories over the time length $T$. The state variables that actually affect the trajectory sampling space are as follows:

\begin{align}
\begin{cases}
v_{sT}\in\left[v_{s0}-\Delta v_s,v_{s0}+\Delta v_s\right]\\
v_{lT}\in\left[v_{l0}-\Delta v_l,v_{l0}+\Delta v_l\right]\\
l_T=f_l\left(s_T\right)\\
T=f_T\left(s_T\right)
\end{cases}
\end{align}

The change range of the longitudinal and lateral velocities, $v_{s_T}$ and $v_{l_T}$, under the trajectory terminal state is constrained based on the longitudinal and lateral velocities, $v_{s_0}$ and $v_{l_0}$, under the initial state of the trajectory. The lateral deviation $l_T$ under the terminal state is constrained based on the feature distribution of the expected trajectory space in subsequent research on human left-turn trajectory decisions. In determining the terminal position, we generate trajectories using an indefinite time length $T$. When the terminal position is uncertain, we generate trajectories using a fixed time length $T=5s$. These two methods are applied to upstream road sections and internal intersection trajectory generation, respectively.

% Figure environment removed

\subsection{Desired Trajectory Space Constraint Based on Intention Expression}
\subsubsection{Constraint for Upstream Sections}
It was analyzed in the previous sections that for left turn trajectories in the upstream sections, the preceding trajectories tend to make a pre-turn while the yielding trajectories continue to maintain straight-line motion along the lane. As shown in Fig.~\ref{fig:pre_turning}, the position, speed, and acceleration $(s_0,l_0,v_{s0},v_{l0},a_{s0},a_{l0})$ are determined as the initial states of the scene. Based on the analysis of pre-turning behavior characteristics, it is determined that the trajectory will experience a lateral deviation and a change in heading angle when reaching the stop line. Hence, the pre-turning state of the trajectory at the stop line can be represented as $(s_{pre},l_{pre},\theta_{pre})$.
Here, $s_{pre}=s_{stop}$, the longitudinal position of the stop line on the reference line, is known. The trajectory's lateral deviation $l_{pre}$ and heading angle $\theta_{pre}$ at the stop line are determined as follows:
\begin{equation}
    \left\{\begin{matrix}\theta_0=\frac{v_{l0}}{v_{s0}}\\\theta_{pre}=\frac{v_{lpre}}{v_{spre}}\\l_{pre}\in\left[l_0,l_{stop}^{max}\right]\\\theta_{pre}\in\left[\theta_0,\theta_{stop}^{max}\right]\\\end{matrix}\right.
\end{equation}
where $l_{stop}^{max}$ is the maximum trajectory deviation at that point. $\theta_{pre}$ determines the ratio of the lateral speed $v_{lpre}$ to the longitudinal speed $v_{spre}$ of the trajectory at the stop line. $\theta_{stop}^{max}$ is the maximum heading angle of the trajectory at that point. Lateral deviation and heading angle are determined through a uniformly distributed random sampling method.

Therefore, assuming the initial trajectory state at time $(s_t,l_t,v_{st},v_{lt},a_{st},a_{lt})$, the trajectory sampling time length is set as $T_{pre}$. The pre-turning behavior characteristic state $(s_{pre},l_{pre},\theta_{pre})$ corresponds to the trajectory generation control terminal state $(s_{pre},l_{pre},v_{spre},v_{lpre},a_{spre},a_{lpre})$ at this moment, where acceleration items are simplified to default zero. Then, when controlling the trajectory generation space variable duration $T_{pre}$, longitudinal speed $v_{spre}$, and lateral speed $v_{lpre}$, they can be determined as the following calculation results:
\begin{equation}
    \left\{\begin{matrix}\mathrm{\ }\mathrm{T}_{pre}\in\left[\frac{s_{pre}-s_t}{v_{st}}-\Delta T_{pre},\frac{s_{pre}-s_t}{v_{st}}+\Delta T_{pre}\right]\\
    \sqrt{{v_{\mathrm{spre\ }}}^2+{v_{\mathrm{lpre\ }}}^2}\in\\
    \left[\sqrt{{v_{st}}^2+{v_{lt}}^2}-\Delta v_{pre},\sqrt{{v_{st}}^2+{v_{lt}}^2}+\Delta v_{pre}\right]\\\frac{v_{\mathrm{lpre\ }}}{v_{\mathrm{spre\ }}}=\theta_{pre}\\\end{matrix}\right.
\end{equation}
where $\Delta T_{pre}$ and $\Delta v_{pre}$ are control variables of trajectory duration and speed variation interval, respectively. The space variables of the yielding trajectory generation are consistent with the preceding trajectory, both controlling the trajectory sampling duration $T_{pre}$ , longitudinal speed $v_{spre}$ , and lateral speed $v_{lpre}$. The calculation method is consistent with the aforementioned one.
% Figure environment removed

\subsubsection{Constraint within Intersections}
Analysis from previous sections determined that, within an intersection, left-turning vehicles following a trajectory that has the right of way will complete a rapid turn directly, while those yielding will adopt a two-phase turning method, maintaining a straight path before completing the turn.

As shown in Fig.~\ref{fig:Traj_HV_Prior}, different decision trajectories lead to distinctly different trajectory distribution spaces within the intersection due to the different turning methods employed.
In this study, we model the relationship between the longitudinal displacement, s, and lateral displacement, l, of a trajectory using a continuously varying normal distribution. The mean and standard deviation of the preemptive trajectory's normal distribution are denoted as $\mu_{\text{preempt}}(s)$ and $\sigma_{\text{preempt}}(s)$ respectively. The mean and standard deviation of the yielding trajectory's normal distribution are denoted as $\mu_{\text{yield}}(s)$ and $\sigma_{\text{yield}}(s)$ respectively.

By determining the longitudinal displacement at time t, $s(t)$, we can deduce that $l_t \sim N(\mu(s(t)),\sigma(s(t)))$. Further, based on the 3-sigma rule of normal distribution, we can determine the range of the trajectory's lateral displacement at the end, $l_t \in [\mu(s(t)) - 2\sigma(s(t)), \mu(s(t)) + 2\sigma(s(t))]$.

Simultaneously, the longitudinal speed, $v_{sT}$, and the lateral speed, $v_{lT}$, at the end of the trajectory are determined within a changing interval based on the initial speeds $v_{s0}$ and $v_{l0}$. These speeds are uniformly sampled within the interval to determine $v_{sT}$ and $v_{lT}$.
Assuming an initial trajectory state within the intersection at a particular time, $(s_0, l_0, v_{s0}, v_{l0}, a_{s0}, a_{l0})$, the longitudinal displacement at the end of the trajectory, $s_T$, can be computed given $v_{sT}$. With $s_T$ determined, the lateral displacement at the end of the trajectory, $l_T$, can be decided based on the established relationship between $s$ and $l$.

Therefore, the expected trajectory space within the intersection can be controlled and generated by the following variables:
\begin{equation}
    \left\{
\begin{matrix}
v_{sT} \in [v_{s0} - \Delta v_s, v_{s0} + \Delta v_s]\\
v_{lT} \in [v_{l0} - \Delta v_l, v_{l0} + \Delta v_l]\\
l_T \in [\mu(s_T) - 2\sigma(s_T), \mu(s_T) + 2\sigma(s_T)]
\end{matrix}
\right.
\end{equation}
where $\Delta v_s$ and $\Delta v_l$ are the control variables for the changing interval of longitudinal and lateral speeds respectively.

In order to ensure that all generated trajectories reflect normal kinematic characteristics exhibited by human driving and avoid collisions, we apply the following constraints to the dynamics of the trajectory:

\begin{equation}
    \left\{
\begin{matrix}
v_{min} \le v_t \le v_{max}\\
a_{min} \le a_t \le a_{max}\\
c_{min} \le c_t \le c_{max}
\end{matrix}
\right.
\end{equation}
where $v_t$, $a_t$, and $c_t$ represent the speed, acceleration, and curvature of a generated trajectory $\zeta_I^i \in \zeta_I$ at time $t \in [0, T]$. The $(v_{\text{max}}, v_{\text{min}})$, $(a_{\text{max}}, a_{\text{min}})$, and $(c_{\text{max}}, c_{\text{min}})$ are the minimum and maximum speeds, accelerations, and curvatures respectively, as determined from the statistical analysis of real human trajectories.

Furthermore, we conduct collision checks to ensure that the generated trajectories do not result in collisions with interactive objects at any time. This study includes the consideration of a safety margin, determining the safe bounding box $Boundingbox_{\text{safe}}$. The collision constraint can be expressed as:

\begin{equation}
Boundingbox_{\text{safe}}^t\left(\text{left}\right) \cap Boundingbox_{\text{safe}}^t\left(\text{straight}\right) = \emptyset \quad 
\end{equation}
where $Boundingbox_{\text{safe}}^t$ is the safe bounding box at time t, which is calculated by adding 0.5m to the front and rear spaces and 0.3m to the left and right spaces of the vehicle's boundary rectangle. If, at any point, a trajectory $\zeta_{I,K}^i \in \zeta_{I,K}$ fails to meet the collision constraint, that trajectory is removed from the set $\zeta_{I,K}$. The final set of desired trajectories that meet the collision constraints is $\zeta_{I,K,C}$.

\subsection{Human Prior Learning}
In traditional trajectory planning research, after defining the trajectory feature function, the optimal trajectory selection is achieved by manually setting or experimentally determining the trajectory reward function feature weights. However, in real-world interaction scenarios, it's challenging to accurately specify a reward function that captures all aspects of safe and efficient driving. IRL can solve this problem. In this subsection, the reward function is first designed and ME-IRL is utilized to learn the human expert's trajectory behavior selection strategy under different decisions.

During learning process, we assume that the total reward of a trajectory is a linear expression of the trajectory reward function, which is the weighted sum of selected features. Furthermore, we postulate that human drivers' preferences or behaviors under the same decision do not exhibit noticeable time variability and individual heterogeneity. Therefore, the total reward $R\left(\zeta\right)$ of a trajectory $\zeta$ can be expressed as:
\begin{equation}
R\left(\zeta\right)=\theta^T\cdot\mathbf{f}\zeta
\end{equation}
where $\theta=\left[\theta_1,\theta_2,\cdots,\theta_K\right]$ is the weight vector, $\mathbf{f}\zeta$ is the reward function vector of trajectory $\zeta$, and $K$ depends on the number of trajectory reward functions $\mathbf{f}\zeta$.

\subsubsection{Reward Function Design}
When designing the reward function, we considered three aspects: traffic efficiency, driving comfort, and interactive safety.

\textbf{Traffic Efficiency.}
We set the reward function for traffic efficiency as the loss in speed, which is the difference between the trajectory speed and the expected speed. We determine the traffic efficiency feature $f_{\text{efficiency}}\left(\zeta\right)$ of the trajectory $\zeta$ to be the mean speed loss at all times, represented as follows:
\begin{equation}
f_{\text{efficiency}}\left(\zeta\right)=-\frac{\sqrt{\sum_{t=0}^{T}\left(v_t-v_{\text{target}}\right)^2}}{T}
\end{equation}
where $v_t$ is the scalar speed of the trajectory $\zeta$ at time $t$, $T$ is the total duration of trajectory sampling, and $v_{\text{target}}$ is the expected speed of the vehicle turning left.

\textbf{Driving Comfort.}
We select the jerk (rate of change of acceleration) to establish the comfort feature function $f_{\text{comfort}}\left(\zeta\right)$ for assessing whether the trajectory $\zeta$ is comfortable. This feature is determined as the mean of the jerk vector sum longitudinally and laterally at all times for the trajectory $\zeta$, calculated as follows:

\begin{equation}
    \left\{\begin{matrix}
    f_{\text{comfort}}\left(\zeta\right)= - \frac{\sum_{t=0}^{T}\sqrt{((Jerk_s^t)^2 + (Jerk_l^t)^2)}}{T}\\
    Jerk_s^t = s'''(t)\\
    Jerk_l^t = l'''(t)\\
    \end{matrix}
    \right.
\end{equation}
where $Jerk_s^t$ and $Jerk_l^t$ are the longitudinal and lateral jerks of the trajectory $\zeta$ at time $t$ respectively. The reward function $f_{\text{comfort}}\left(\zeta\right)$ takes into account the smoothness in both the longitudinal and lateral directions.

\textbf{Interaction Safety.}
We quantify interaction safety by calculating the time difference $\Delta TTCP$ between the interacting parties reaching the conflict point. $\Delta TTCP$ takes into account the relative relationships of the positions and speeds of the interacting parties, characterizing the time interval for both parties to leave the conflict point under the current state.

The interaction safety is deconstructed into longitudinal progress and lateral deviation. The impact on interaction safety from the longitudinal progress at time $t$ is determined as the time difference for the interacting parties to reach the conflict point without considering the lateral deviation of the left-turning vehicle, denoted as $\Delta TTCP_{st}$. Its calculation is as follows:

\begin{equation}
\Delta TTCP_{st}=TTCP_{st}^{left}-TTCP_{st}^{straight}
\end{equation}
\begin{align}
\begin{cases}
TTCP_{st}^{left}=\frac{s_t^{left}-s_{cp}^{left}}{v_{st}^{left}}\\
TTCP_{st}^{straight}=\frac{s_t^{straight}-s_{cp}^{straight}}{v_{st}^{straight}}
\end{cases}
\end{align}
where $TTCP_{st}^{left}$ and $TTCP_{st}^{straight}$ are the times for the left-turning and straight-driving vehicles to pass the conflict point at the longitudinal level at time $t$ respectively.
$s_t^{left}$ and $s_t^{straight}$ are the longitudinal positions of the left-turning and straight-driving vehicles at time $t$, $s_{cp}^{left}$ and $s_{cp}^{straight}$ are the longitudinal positions of the reference conflict points on the reference lines for the left-turning and straight-driving vehicles respectively, $v_{st}^{left}$ and $v_{st}^{straight}$ are the longitudinal velocities of the left-turning and straight-driving vehicles at time $t$.

% Figure environment removed

After calculating $\Delta TTCP_{st}$, we further determine the feature function $f_{safe,s}\left(\zeta\right)$ on the longitudinal level of interaction safety as the mean of $\Delta TTCP_{st}$ at all times on trajectory $\zeta$, represented as follows:
\begin{equation}
f_{safe,s}\left(\zeta\right)=\frac{\sum_{t=0}^{T}\left|\Delta TTCP_{st}\right|}{T}
\end{equation}

On the other hand, the impact of lateral deviation at time $t$ on interaction safety, as analyzed above, is determined by the impact of the left-turning vehicle's lateral deviation on the time difference for both interacting parties to reach the conflict point, denoted as $\Delta TTCP_{lt}$. Its calculation is as follows:

\begin{equation}
\Delta TTCP_{lt}=\Delta TTCP_l^{left}+\Delta TTCP_l^{straight}
\end{equation}
where $\Delta TTCP_l^{left}$ and $\Delta TTCP_l^{straight}$ are the impacts caused by the left-turn vehicle's lateral deviation on the times for itself and the oncoming straight-driving vehicle to pass through the conflict point, respectively. Their calculations consider the variables as shown in Fig.~\ref{fig:IRL_Safety_Term} and are as follows:

\begin{align}
\begin{cases}
\Delta TTCP_l^{left}=\frac{l_t \cdot tan{\theta_t}}{v_{st}^{left}}\\
\Delta TTCP_l^{straight}=\frac{l_t \cdot cos{\theta_t}}{v_{st}^{straight}}
\end{cases}
\end{align}
where $l_t$ is the lateral deviation of the left-turn vehicle at time $t$, and $\theta_t$ is the angle produced by the expected trajectory of the oncoming straight-driving vehicle and the projection point of the expected conflict point on the reference line along the $l$ axis at time $t$.

After calculating $\Delta TTCP_{lt}$, we further determine the feature function $f_{safe,l}\left(\zeta\right)$ on the longitudinal level of interaction safety as the mean of $\Delta TTCP_{lt}$ at all times on trajectory $\zeta$, represented as follows:

\begin{equation}
f_{safe,l}\left(\zeta\right)=\frac{\sum_{t=0}^{T}\left|\Delta TTCP_{lt}\right|}{T}
\end{equation}

The interaction safety feature function of trajectory $\zeta$ is characterized from both longitudinal and lateral perspectives, represented as $f_{safe,s}\left(\zeta\right)$ and $f_{safe,l}\left(\zeta\right)$.


\subsubsection{Maximum Entropy IRL Learning}

Given a human driving demonstration dataset $D=\{ \zeta_1,\zeta_2,\cdots,\zeta_N \}$ composed of $N$ trajectories, the ME-IRL algorithm is used to infer the reward weight $\theta$, which can then be used to generate driving strategies that match the human expert demonstration trajectories.

Simultaneously, to simulate the randomness of human driver's trajectory selection, we employ the Boltzmann noise theory model to construct the candidate trajectory distribution. Under the Boltzmann distribution, all features expected to match the expert's demonstration have the maximum entropy principle, corresponding to the ME-IRL. Therefore, the probability of a trajectory is proportional to the return of that trajectory,
\begin{equation}
P\left(\zeta\left|\mathbf{\theta}\right.\right)=\frac{e^{R\left(\zeta\right)}}{Z\left(\theta\right)}=\frac{e^{\mathbf{\theta}^T\mathbf{f}\zeta}}{Z\left(\theta\right)}
\end{equation}
where $P\left(\zeta\left|\mathbf{\theta}\right.\right)$ is the probability of trajectory $\zeta$, and $Z\left(\theta\right)$ is the partition function. As the partition function $Z\left(\theta\right)$ represents the integral sum of the rewards of all possible trajectories, it is challenging to directly calculate in continuous and high-dimensional spaces. We reduce the trajectory space to the previously researched and mined human prior expected trajectory space $\zeta_{I,K,C}$ and use a finite number of discretely generated feasible trajectories to approximate the partition function. Therefore, the probability expression of a trajectory yields that
\begin{equation}
P\left(\zeta\left|\mathbf{\theta}\right.\right)\approx\frac{e^{\mathbf{\theta}^T\mathbf{f}\zeta}}{\sum_{i=1}^{N}e^{\mathbf{\theta}^T\mathbf{f}{{\widetilde{\zeta}}^i}}}
\end{equation}
where ${\widetilde{\zeta}}^i\in\zeta{I,K,C}$ is a generated trajectory with the same initial state as trajectory $\zeta$, $\mathbf{f}_{{\widetilde{\zeta}}^i}$ is the feature vector of trajectory ${\widetilde{\zeta}}^i$, and $N$ is the total number of generated trajectories.

The aim of ME-IRL is to maximize the likelihood of expert demonstration trajectories by adjusting the feature weight $\theta$. The optimization objective function can be expressed as:
\begin{equation}
\max_{\theta} \mathcal{J}(\mathbf{\theta})= \max_{\theta} 
\sum_{\zeta \in \mathcal{D}} 
 \log P(\zeta | \theta)
\end{equation}
where $\mathcal{D}=\{ \zeta_i \}{i=1}^N$ represents the set of human expert demonstration trajectories. By substituting $P\left(\zeta\left|\mathbf{\theta}\right.\right)$ into the above equation, we obtain the optimized objective function $\mathcal{J}\left(\theta\right)$ as follows:
\begin{equation}
\mathcal{J}(\mathbf{\theta})=\sum_{\zeta\in\mathcal{D}}\left[\mathbf{\theta}^T\mathbf{f}_\zeta-log{\sum_{i=1}^{M}e^{\mathbf{\theta}^T\mathbf{f}_{{\widetilde{\zeta}}^i}}} \right]
\end{equation}

The above equation can be optimized using gradient-based methods. 
The gradient of the optimization objective function $\mathcal{J}\left(\theta\right)$, $\nabla\theta\mathcal{J}\left(\theta\right)$, can be expressed as follows:
\begin{equation}
\nabla_\mathbf{\theta}\mathcal{J}(\mathbf{\theta})=\sum_{\zeta\in\mathcal{D}}\left[\mathbf{f}_\zeta-\sum_{i=1}^{M}\frac{e^{\mathbf{\theta}^T\mathbf{f}_{{\widetilde{\zeta}}^i}}}{\sum_{i=1}^{M}e^{\mathbf{\theta}^T\mathbf{f}_{{\widetilde{\zeta}}^i}}}\mathbf{f}_{{\widetilde{\zeta}}^i}\right]
\end{equation}
The gradient can be viewed as the difference in feature expectations between human demonstration trajectories and generated trajectories:
\begin{equation}
\nabla_\mathbf{\theta}\mathcal{J}(\mathbf{\theta})=\sum_{\zeta\in\mathcal{D}}\left[\mathbf{f}_\zeta-\sum_{i=1}^{M}P\left({\widetilde{\zeta}}^i\left|\mathbf{\theta}\right.\right)\mathbf{f}_{{\widetilde{\zeta}}^i}\right]
\end{equation}

Following the process outlined by  \cite{huang2021driving}, we use a gradient ascent method to iteratively update the trajectory feature weights and compute the optimization objective until the loss converges. To prevent overfitting, we incorporate L2 regularization into the objective function $\mathcal{J}\left(\theta\right)$. Consequently, the gradient $\nabla\theta\mathcal{J}\left(\theta\right)$ includes the difference in feature expectations plus a regularization term, as shown below:
\begin{equation}
\nabla_\mathbf{\theta}\mathcal{J}(\mathbf{\theta})=\sum_{\zeta\in\mathcal{D}}\left[\mathbf{f}_\zeta-\sum_{i=1}^{M}P\left({\widetilde{\zeta}}^i\left|\mathbf{\theta}\right.\right)\mathbf{f}_{{\widetilde{\zeta}}^i}\right]-2\lambda\theta
\end{equation}
where $\lambda > 0$ is the regularization parameter.

The whole process is summarized in Algorithm \ref{algo:irl_weight_learning}. After learning, the learned reward function weight $\theta$ will be employed to assign scores to the candidate trajectories. The trajectory with the highest score will then be selected as the planned output.

\begin{algorithm}
\SetAlFnt{\small}
\SetKwInOut{Parameter}{Inputs}
\SetKwInOut{Output}{Output}
\caption{ME-IRL Trajectory Weight Learning Process}
\label{algo:irl_weight_learning}
\LinesNumbered % show line number
\SetAlgoLined
\Parameter{Left-turn trajectory set \(D = \{\zeta_1, \zeta_2, \dots, \zeta_N\}\), going straight trajectory set \(P\), learning rate \(\alpha\), regularization parameter \(\lambda\), number of iterations \(E\)}
\Output{Reward function weights \(\theta^*\)}
\vspace{0.2em}
\hrule
\vspace{0.2em}

Initialize reward function weights \(\theta\) randomly;\\
Compute human trajectory features \(\bar{f} = \sum_{i=1}^{N} f(\zeta_i)\);\\
Define sampling space control variables \(\{v_{sT}, v_{lT}, l_T\}\), determine the trajectory planning time window \(T\), and generate the trajectory set \(\tilde{D}_i = \{\tilde{\zeta}_i^j\}\) with the same initial state as \(\zeta_i\);\\
Compute trajectory features \(f(\tilde{\zeta}_i^j)\) for all generated trajectories interacting with the real oncoming straight trajectory set \(P\);\\
\Repeat{iteration count \(E\) is reached}{
    Compute the feature expectation \(\tilde{f}\) for the generated trajectory samples;\\
    Calculate the gradient \(\nabla_\theta J(\theta) = \bar{f} - \tilde{f} - 2\lambda\theta\);\\
    Update the reward weights \(\theta = \theta + \nabla_\theta J(\theta)\);\\
}

\end{algorithm}

% \section{Experiment and Analysis on the Simulation Platform}
\section{Experiments on the Simulation Platform}
\label{sec:4}
In this section, we detail the dataset used and the implementation of our experiment, followed by an analytical review of the experiments conducted.

\subsection{Dataset}
Our analysis utilizes the SIND dataset \cite{xu2022drone}, a drone-sourced dataset, to study human driver interaction at intersections. Compiled by Tsinghua University's SOTIF research team, the SIND dataset comprises intersection data from Tianjin, China, characterized by a two-phase traffic signal system. This system, allowing simultaneous movements of left-turning and straight-moving vehicles, results in significant interactions and conflicts. We meticulously selected 268 one-on-one interaction events between left-turning and straight-moving vehicles from this dataset, including 132 yielding and 136 proceeding left-turning instances. More description can be found here\footnote{See \url{https://drive.google.com/file/d/1FChn8wm8rZQ-OzOUCdbmSBGx41OtdI7_/view?usp=drive_link}}.

\subsection{Implementation Details}
For inverse reinforcement learning, we first extracted expert human left-turning trajectories from the SIND dataset, lasting between 8 to 15 seconds. From the onset of the left-turning and straight-moving interaction, we selected a 5-second trajectory segment at 0.5-second intervals as the expert demonstration, denoted as $\zeta$. Trajectories incorporating intent expression, denoted as $\zeta_{I,K,C}$, are generated from these segments. We use 80\% of these generated trajectories and their corresponding expert demonstrations for training the reward function, reserving the remaining 20\% for testing.

The trajectory generation sample space is defined by three control variables: end-time longitudinal speed $v_{sT}$, lateral speed $v_{lT}$, and lateral displacement $l_T$. We uniformly selected values for $v_{sT}$ and $v_{lT}$ within their respective ranges, and 10 values for $l_T$ within its range. The planning duration is set to 5 seconds, with a 0.1-second interval between trajectory points. For the ME-IRL learning process, we set the number of iterations E to 1000, with parameters $\alpha = 0.05$ and $\lambda = 0.01$.

For comparative analysis, we chose three other trajectory planning methods: Frenet, decision-based ME-IRL without intent expression space consideration, and potential field-based motion planning. These methods were compared using real interactive event decisions as high-level decision inputs. Further details and results are available here.\footnote{See \url{https://drive.google.com/drive/folders/1hMP1QvL8jq-jti_rTme5aG1GQ7POm6K1?usp=drive_link}}

% Figure environment removed

\subsection{Analysis of Inverse Reinforcement Learning Results}
The Maximum Entropy IRL model's training progression is shown in Fig.~\ref{fig:IRL_Result}. Panel (a) of the figure demonstrates the iterative process, highlighting the average feature discrepancy and the change in average log likelihood value of the human expert demonstration trajectory under yield and priority decisions. These correspond to the gradient $\nabla_\theta\mathcal{J}\left(\theta\right)$ and the optimization target function $\mathcal{J}\left(\theta\right)$. The figure shows a steady increase in the average log likelihood value, indicating convergence over iterations for different decision trajectories.

We assess the trajectory planning's closeness to human driving using Average Human Trajectory Similarity (AHL). AHL is defined as the average final displacement error of the n highest probability trajectories in the generated distribution:
\begin{equation}
\text{AHL} = \frac{1}{n} \min_{i=1}^{n} |\hat{\zeta}_i(T) - \zeta(T)|_2
\end{equation}
Here, $\hat{\zeta}_i(T)$ are the top n trajectories in the generated distribution, and $\zeta(T)$ is the actual human driver trajectory. Fig.~\ref{fig:IRL_Result} (b) depicts the ME-IRL accuracy for both preceding and yielding decisions. The priority and yield decision trajectories exhibit average errors of only 0.39m and 0.26m, respectively, compared to actual human trajectories, validating the efficacy of our method.

\subsection{Evaluation of Planning Results}
Our evaluation of the proposed trajectory planning method encompasses four dimensions: trajectory distribution, intent expression capability, safety and efficiency of interactions, and computational efficiency along with learning efficacy.

\subsubsection{Overall Trajectory Distribution}
We assessed the social aptitude of trajectory planning through its overall distribution and spatial extent. Methods with limited sociality, showing minimal consideration for intent expression, often result in a restricted range of trajectory choices. This limitation hinders the clear differentiation of trajectories under various decision-making scenarios, thus impeding intent recognition by other vehicles.

To visually and quantitatively compare the trajectory spaces of different methods, we juxtaposed the distributions of four planning methods with actual human trajectory data, as shown in  Fig.~\ref{fig:planning_results_compare}). Qualitative analysis reveals our method's trajectory space closely aligns with actual human trajectories, while the potential field and DB-ME-IRL methods are somewhat less accurate, and the Frenet method displays the most significant deviation.

For a quantitative comparison, we employed trajectory coverage analysis. This involved dividing the upstream lane and intersection area into 0.5m subregions and checking for trajectory presence within these areas. Our method achieved a 77\% match with actual human trajectory coverage, significantly outperforming the other methods, with the Frenet method only covering 54\% of the actual trajectories.
\begin{table*} %[!htbp]
    \centering
    \caption{Comparison of trajectory coverage of different methods.}
    \label{table:coverage compare}
    \resizebox{\textwidth}{!}{
    \begin{tabular}{*{6}{c}}
    \toprule
        \multirow{2}{*}{Distribution characteristic index} & \multicolumn{5}{c}{Trajectory planning method}\\ 
        \cmidrule(lr){2-6}
        {} & Real trajectory & Our method & Frenet Planning Method & DB-ME-IRL & Potential field method \\
    \midrule
        Meta area coverage & 1066 & 824 & 576 & 710 & 737 \\ \hline
        Ratio to true trajectory & - & 77\% & 54\% & 67\% & 69\% \\
    \bottomrule
    \end{tabular}
    }
\end{table*}

% Figure environment removed


\subsubsection{Capability of Intent Expression}
In interactions with oncoming vehicles, left-turning vehicles typically employ lateral offset as a means to convey decision intent. The capability of a trajectory plan to express intent is thus evaluated by examining the extent of lateral offset.

Fig.\ref{fig:SL_traj_distribution} displays the normalized Straight Line (SL) trajectories under various planning methods. The red (blue) line indicates the center of the virtual left-turn lane, with gray lines marking its boundaries. Offsets to the left are negative, while those to the right are positive. Our method's SL trajectory offset closely mirrors actual driver behavior, outperforming other methods which display notable deviations.

The potential field method, although capable of mimicking priority trajectories with significant leftward offsets, fails to replicate the longer straight-line path indicative of yielding behavior seen in human trajectories. The DB-ME-IRL and Frenet methods show less proficiency in intent expression. 

% Figure environment removed

The total lateral offset of a trajectory, represented by the enclosed area $S_{SL}$ between the normalized SL trajectory and the virtual lane centerline, was calculated and visualized in a violin plot (Fig. \ref{fig:SL_Trajectory_Offset}). Our method achieved an average offset of 0.64, closely following the real trajectory average of 0.75 and showing an 8\% improvement over comparative methods. In terms of maximum offset, our method reached 1.7 compared to the real trajectory's 2.2, marking a 42\% improvement over other methods.


% Figure environment removed

\subsubsection{Evaluation of Motion Interaction Features}
The safety and efficiency of the trajectory planning methods were assessed using Post-Encroachment Time (PET) and Travel Time metrics.

Travel time within the intersection was analyzed (Fig. \ref{fig:travel_time_distri}). Our method's average travel time of 7.4s with a standard deviation of 2.4s closely approximates the real trajectory average of 7.2s. Comparative analysis across different methods indicated our method aligns most closely with real trajectory travel times.

PET, a critical measure of conflict severity during left turns, was also evaluated. The average PET for our method was 4.3s, compared to the real trajectory average of 4.5s (Fig. \ref{fig:result_PET_distri}). Our method's PET performance was the most favorable, with the DB-ME-IRL and Frenet methods showing a marked decrease in PET, implying a higher risk in trajectory interaction safety.

% Figure environment removed

% Figure environment removed

\subsubsection{Assessment of Computational Efficiency and Learning Efficacy}
The study also aimed to evaluate the enhanced computational efficiency of our trajectory planning algorithm, particularly focusing on how the decision expectation space constraint influences this efficiency. Key parameters for this assessment included the average number of candidate trajectories, average computation time (incorporating both feature computation and trajectory search), and the overall learning performance. Comparative experiments were conducted against the DB-ME-IRL method, and the outcomes are tabulated in Tab. \ref{table:calculate_efficiency_compare}. The experiments utilized a computing setup with an i7-8700 processor and 16GB of memory. Our method showed a significant reduction in both the average number of candidate trajectories (by 52.5\%) and computation time (by 41.1\%) after applying the expected trajectory space constraint.

Furthermore, the trajectory learning performance was assessed using the Average Human Likeness (AHL) index. Both our method and the DB-ME-IRL method underwent training over 1000 iterations. The resultant AHL values for our method in priority and yielding decision scenarios were 0.39 and 0.26, respectively, demonstrating a marked improvement of 13\% and 54\% in learning performance compared to the DB-ME-IRL method. These results are comprehensively presented in Tab. \ref{table:calculate_efficiency_compare} and Fig. \ref{fig:IRL_effect_compare}.


\begin{table*}%[!htbp]
    \centering
    \caption{Trajectory Planning Efficiency and Effect Comparison.}
    \label{table:calculate_efficiency_compare}
    % \resizebox{\textwidth}{!}{
    % \begin{tabular}{*{5}{c}}
    \begin{tabular}{>{\centering\arraybackslash}p{3cm} >{\centering\arraybackslash}p{3cm} >{\centering\arraybackslash}p{3cm} >{\centering\arraybackslash}p{3cm} >{\centering\arraybackslash}p{3cm}}
    \toprule
        Method & Average Number of Candidate Trajectories & Average Calculation Time & Learning Effect (Preceding) & Learning Effect (Yielding)\\ 
        % \cmidrule(lr){4-5}
    \midrule
        Our Method & 273 & 0.079s & 0.39 & 0.26 \\ \hline
        Decision-based ME-IRL & 581 & 0.134s & 0.45 & 0.57 \\
    \bottomrule
    \end{tabular}
    % }
\end{table*}

% Figure environment removed



% \subsection{Case Analysis}

% To ascertain the degree to which our method aligns with human characteristics in conveying intentions during the interaction process, we conducted tests on our algorithm within a simulation platform.
% Initially, a 2D intersection simulator is constructed, mirroring the size and shape of the intersection found in the SIND dataset from the real world. Within this simulator, the trajectory of the straight-going High-Velocity vehicle is log-replayed based on real-world data, while the left-turn vehicle can be controlled either by our algorithm or by human-expert data.
% Our algorithm is rigorously tested within the simulator, focusing on two selected scenarios: a left-turn vehicle yielding case and a proceeding-first case. Subsequently, we compare the actual human trajectories with those planned by our method and trajectories devised by the ME-IRL methods \cite{werling2010optimal}.
% For a visual representation of the two interaction cases, animations are available on the provided website.\footnote{Refer to \url{https://shorturl.at/jqu35}}
% % Figure environment removed
% % Figure environment removed

% % Figure environment removed
% \subsubsection{Case 1: Left-turn Vehicle Proceeding}
% Fig.~\ref{fig:case_preceding_trajectory_our} illustrates a case where the left-turn vehicle is proceeding. At T=3s, the trajectory planned by our method begins to laterally offset prior to entering the intersection. Upon entering the intersection, the trajectory maintains a turning approach that passes through the conflict point early by significantly shifting left relative to the reference line, closely resembling the actual human trajectory. In contrast, the comparison method as shown in Fig. \ref{fig:case_preceding_trajectory_baseline} plans trajectories on both sides of the reference line, striving to minimize any lateral offset relative to the reference line. 
% By comparison, it can be found that this method differs greatly from the actual human trajectory and fails to achieve intention communication at the trajectory behavior level.

% Furthermore, our method reduces the trajectory generation search space significantly by adding the constraint of the expected space, whereas the comparison method uniformly generates candidate trajectories based on both sides of the reference line. Throughout the planning process, the maximum number of trajectories planned by our method is 300, while the comparison method needs to generate and search for up to 800 candidate trajectories. Our method reduces the search space by 62.5\%.

% We further quantitatively analyze the behavioral performance difference between the two methods by examining the changes in motion states during the interaction process, as demonstrated in Fig. \ref{fig:case_preceding_analysis}. Fig. \ref{fig:case_preceding_analysis} (a) depicts the relationship between the heading angle of the left-turning vehicle and X. It reveals that our method exhibits a noticeable pre-turning behavior before the stop line of the lane, whereas the comparison method only commences turning after entering the intersection. Fig. \ref{fig:case_preceding_analysis} (b) displays the relationship between the lateral offset of the left-turning vehicle and X. It shows that our method passes through the conflict point first by significantly and swiftly shifting laterally during the turning process, whereas the comparison method adheres to the reference line, resulting in a trajectory with negligible lateral offset.

% \subsubsection{Case 2: Left-turn Vehicle Yielding}
% % Figure environment removed
% % Figure environment removed

% % Figure environment removed
% In the scenario where the left-turn vehicle yields, the trajectories planned by our method and the comparison method are illustrated as the red trajectory lines in Fig.~\ref{fig:case_yielding_trajectory_our} and Fig.~\ref{fig:case_yielding_trajectory_baseline}, respectively.
% Our method, as depicted in Fig. \ref{fig:case_yielding_trajectory_our}, can implement a two-stage turning approach of 'proceed straight then turn' under the yielding scenario when making a left turn. This both communicates its decision to yield to the interacting object and prevents the efficiency loss of stopping and restarting. 

% Fig.~\ref{fig:case_yielding_analysis}(a) illustrates the relationship between the heading angle and X. If we consider a heading angle of 15° as a significant deviation, the real trajectory can maintain a straight phase up to x=10.7m, and our method can maintain a straight phase up to x=7.0m. After entering the intersection, it clearly yields and maintains a straight path. However, the comparison method can maintain a straight phase up to x=8.2m under the judgment standard of 15°. Nonetheless, the comparison method exhibits decision inconsistency; it decides to proceed first upon entering the intersection but fails, and then corrects its decision through the steering angle, resulting in a later turning time for the comparison method. 

% Fig. \ref{fig:case_yielding_analysis}(b) shows the relationship between the lateral offset and X. To quantify the difference between the planned trajectory and the actual trajectory during the straight phase, the area composed of the X-axis coordinates and the lateral offset is described as the degree of offset while going straight. During the straight phase of the real trajectory, our method's process offset degree is greater than that of the comparison method, thus appearing closer to the actual trajectory during the straight phase. The trajectory straight offset degree of our method is 15.6\% different from the real trajectory, and the trajectory straight offset degree of the traditional method decreases by 22.1\% compared to our method.

\section{Evaluation on the Human-in-loop Driving Platform}
Although low-cost, efficient, and rapid experiments can be conducted in simulated environments, vehicles in simulators typically exhibit fixed driving styles and behavior patterns, lacking the ability to perceive and understand socio-interactional cues. To validate the algorithm's socio-interactional capabilities, in this section, we establish a Human-in-Loop driving simulation platform and design simulation experiments involving human drivers to assess the algorithm's interpretability of intent and timing of intent clarification.

\subsection{Platform Construction and Experimental Design}
% We initially constructed the "Human-in-Loop" driving simulation platform, as depicted in Fig.~\ref{fig:human-in-loop driving platform}. The platform consists of three main components: hardware environment, software environment, and interaction strategies.
% Figure environment removed

The Human-in-Loop driving simulation platform was established for our study, as showcased in Fig.~\ref{fig:human-in-loop driving platform}. This platform comprises three integral components: the hardware setup, software framework, and a series of interaction strategies. We recruited fourteen participants, all holding valid driver's licenses with a minimum of one year's driving experience, to partake in our experiments. The simulated environment replicated an intersection with east-west directions featuring bidirectional four-lane roads and north-south directions having bidirectional two-lane roads. The experimental setup involved left-turning vehicles (operated by our algorithm) entering from the west and straight-going vehicles (managed by participants) from the east.

The ME-IRL algorithm was utilized as the baseline for comparison, with high-level decision-making in both our and the baseline algorithm derived from a decision tree algorithm~\cite{ma2017two}. Each participant underwent 20 different scenarios, resulting in 280 sets of interaction data.

A series of subjective perception surveys were conducted before, during, and after the experiments to assess participants' trust in current AVs, feedback on AV interaction strategies, and post-experiment reflections on strategy acceptability and trust alterations. These surveys, detailed in Tab.~\ref{tab:subjective_evaluation}, employed a 5-point Likert scale, apart from decision consistency measured in binary terms during the experiments. Further details on the simulation platform and experiment design are available at the specified link.\footnote{See \url{https://drive.google.com/drive/folders/1hMP1QvL8jq-jti_rTme5aG1GQ7POm6K1?usp=drive_link}}

\begin{table*}%[h]
    \centering
    \caption{Subjective Evaluation using a 5-Point Likert Scale at Different Experimental Stages.}
    \label{tab:subjective_evaluation}
    \begin{tabularx}{1.0\textwidth}{lXl}
        \toprule
        \textbf{Experimental Phase} & \textbf{Subjective Perception Question} & \textbf{5-Point Likert Scale} \\
        \midrule
        Before Experiment & \textbf{Question about Trust Level} & Not at all Trusting — Completely Trusting \\
        & During the current stage of driving and interaction with the AV, do you trust the decision-making behavior of the AV? & \\
        \midrule
        During Experiment & \textbf{Question about Interpretability} & Not at all Clear — Completely Clear \\
        & During this driving process, do you think the left-turning vehicle (AV) expresses its intent clearly? & \\
        & \textbf{Question about Timing of Intent Clarification} & Very Late — Very Early \\
        & During the interaction process, at which stage can you fully understand the intent of the left-turning vehicle? & \\
        \midrule
        After Experiment & \textbf{Question about Approval Level} & Not Necessary — Very Necessary \\
        & Do you think it is necessary for the AV to showcase its intent through different actions? & \\
        & \textbf{Question about Trust Level} & Not at all Helpful — Extremely Helpful \\
        & Does this socio-interaction strategy help improve your trust in the AV? & \\
        \bottomrule
    \end{tabularx}
\end{table*}

\subsection{Experiment Result}
The main focus of our analysis was on the social performance of the algorithm, particularly in terms of interpretability of intent and the timing of intent clarification.

\subsubsection{Interpretability of Intent}
The interpretability of our strategy and comparative strategies was subjected to statistical analysis across various priority scenarios, as depicted in Fig.~\ref{fig:human interpretability evaluation}. The Wilcoxon non-parametric tests were used to compare subjective perception scores for interpretability, followed by Pearson correlation coefficient analysis to assess the impact of different scenarios on participants' perceptions. The results are summarized in Tab.\ref{tab:interpretability_results}. In the table,the values outside and inside the brackets in the Median and Mean columns represent the respective values for our strategy and the comparative strategy.

% Subjective evaluations of interpretability of intent for the proposed strategy and comparative strategies were statistically compared across different priority scenarios. Box plots depicting these evaluations are shown in Fig.~\ref{fig:human interpretability evaluation}. Furthermore, Wilcoxon non-parametric tests were employed to analyze the differences in subjective perception scores for interpretability of intent. Pearson correlation coefficient analysis was then conducted to measure the impact of different conditions on participants' subjective ratings, categorized as no impact, minor impact, moderate impact, and significant impact~\cite{rettenmaier2021communication}. Descriptive statistical results are presented in Table\ref{tab:interpretability_results}.
% Figure environment removed

Our strategy consistently achieved higher interpretability scores than comparative strategies across all scenarios, as shown in Fig.\ref{tab:interpretability_results}, significant differences were noted in interpretability between our strategy and the comparative one, particularly in scenarios involving straight-going and left-turn priorities. Our strategy was found to have a more substantial impact on interpretability, especially in left-turn priority and advantage scenarios. The overall analysis indicated a median interpretability score of 5.0 ("very clear") and a mean score indicating a clarity level between "somewhat clear" and "very clear."

% From the comprehensive results in Fig.~\ref{fig:human interpretability evaluation}, it is evident that the average interpretability of intent for our strategy surpasses that of the comparative strategies in all five priority scenarios. Except for the fuzzy scenario, where interpretability ranges from "general" to "somewhat clear," the interpretability for the remaining four scenarios is above the "somewhat clear" level. This indicates a significant enhancement in the ability of our interactive strategy to express implicit intent. In contrast, the comparative strategy achieves "somewhat clear" interpretability only in the straight-going priority scenario, while in other scenarios, it remains at the "general" level.

% Analyzing the results in Tab.~\ref{tab:interpretability_results}, Wilcoxon tests reveal significant differences in the interpretability of intent between our strategy and the comparative strategy in scenarios with straight-going priority, left-turn priority, and left-turn advantage. These findings suggest that our strategy better communicates its intent to interacting entities. Pearson correlation coefficient calculations indicate that different strategies have a considerable impact on interpretability of intent in scenarios with left-turn priority and left-turn advantage, a moderate impact in scenarios with straight-going priority, and a minor impact in scenarios with fuzzy boundaries and straight-going advantage. This demonstrates that our strategy has the greatest impact on improving interpretability of intent in scenarios with left-turn priority and left-turn advantage, followed by straight-going priority, and lastly, scenarios with fuzzy boundaries and straight-going advantage. In a comprehensive analysis of all scenarios, our strategy's median interpretability score is 5.0, classified as "very clear," with a mean of 4.23 falling between "somewhat clear" and "very clear." The mean improvement over the comparative strategy is 24\%, showing a significant difference ($p=0.000$). Our strategy achieves a moderate level of improvement ($r=0.36$) in interpretability of intent compared to the comparative strategy.
\begin{table*}%[h]
    \centering
    \caption{Descriptive Statistical Results of Subjective Evaluation on the Interpretability of Intent for Different Strategies. }
    \label{tab:interpretability_results}
    \begin{tabular}{lcccccc}
        \toprule
        \textbf{Scenario} & \textbf{Median (Mdn)} & \textbf{Mean} & \textbf{Test Statistic (W)} & \textbf{p-value} & \textbf{Effect Size (r)} \\
        \midrule
        Fuzzy Boundary (n=28) & 4.0 (3.0) & 3.64 & 85.0 & 0.057 & 0.24 \\
        Straight Advantage (n=28) & 4.0 (3.0) & 4.11 & 37.5 & 0.010 & 0.35 \\
        Straight Priority (n=28) & 4.5 (4.0) & 4.43 & 24.0 & 0.058 & 0.22 \\
        Left Turn Advantage (n=28) & 4.0 (3.0) & 4.32 & 22.5 & 0.001 & 0.51 \\
        Left Turn Priority (n=28) & 5.0 (3.5) & 4.64 & 22.5 & 0.000 & 0.52 \\
        All Scenarios (n=140) & 5.0 (3.0) & 4.23 & 847.5 & 0.000 & 0.36 \\
        \bottomrule
    \end{tabular}
    \smallskip
\end{table*}

\subsubsection{Timing of Intent Clarification}
Alongside interpretability, the timing of intent clarification was also a key focus. Earlier clarification of intent by participants suggested a more explicit and timely expression of social intent by our interaction strategy.

Statistical comparisons of our strategy and the comparative strategy regarding intent clarification timing were conducted across different scenarios, as illustrated in Fig.\ref{tab:timing_results} and Fig.~\ref{fig:human_time_clarification_evaluation} indicates that our strategy generally allowed for earlier intent clarification across all scenarios. Wilcoxon tests showed significant differences in all scenarios, confirming that our strategy facilitated earlier and more explicit expression of intent. Pearson correlation analysis further suggested that our strategy most significantly impacted scenarios with left-turn advantage, followed by left-turn priority. In sum, our strategy achieved a median score of 4.0 ("somewhat early") in intent clarification timing, with an average indicating a range between "somewhat early" and "very early," marking a noteworthy improvement over the comparative strategy.

% In addition to subjective evaluations of interpretability, the timing of intent clarification is another dimension to measure the expressive capability of intent. If participants clarify the AV's intent earlier, it indicates that the social intent expression timing of the interaction strategy is earlier and more explicit.

% Subjective evaluations of the timing of intent clarification for our strategy and the comparative strategy were statistically compared across different priority scenarios. Box plots depicting these evaluations are shown in Fig.~\ref{fig:human_time_clarification_evaluation}. Similarly, Wilcoxon tests were applied to analyze the differences in subjective perception scores for the timing of intent clarification. Pearson correlation coefficient analysis was then conducted to measure the impact of different conditions on participants' subjective ratings. Descriptive statistical results are presented in Tab.~\ref{tab:timing_results}.
% Figure environment removed

% Analyzing the comprehensive results in Fig.~\ref{fig:human_time_clarification_evaluation}, it is observed that the average timing of intent clarification for our strategy is earlier than that of the comparative strategy in all five priority scenarios. In scenarios with fuzzy boundaries and straight-going advantage, the clarification timing is within the "general" to "somewhat early" range, while in straight-going priority, left-turn priority, and left-turn advantage scenarios, it is above the "somewhat early" level. In contrast, the comparative strategy's timing of intent clarification remains below the "somewhat early" level in all scenarios. This indicates that our interactive strategy achieves earlier and more explicit intent expression.

% Analyzing the results in Tab.~\ref{tab:timing_results}, Wilcoxon tests reveal significant differences in the timing of intent clarification between our strategy and the comparative strategy in all five scenarios, suggesting that our strategy can express its intent to interacting entities earlier and more explicitly. Pearson correlation coefficient calculations indicate that different strategies have a considerable impact on the timing of intent clarification in scenarios with left-turn advantage, a moderate impact in scenarios with left-turn priority, and a minor impact in scenarios with fuzzy boundaries, straight-going advantage, and straight-going priority. This demonstrates that our strategy has the greatest impact on improving the timing of intent clarification in scenarios with left-turn advantage, followed by left-turn priority, and lastly, scenarios with fuzzy boundaries, straight-going advantage, and straight-going priority. In a comprehensive analysis of all scenarios, our strategy's median timing of intent clarification is 4.0, classified as "somewhat early," with a mean of 4.00 falling between "somewhat early" and "very early." The mean improvement over the comparative strategy is 23\%, showing a significant difference ($p=0.000$). Our strategy achieves a moderate level of improvement ($r=0.33$) in the timing of intent clarification compared to the comparative strategy.

\begin{table*}%[h]
    \centering
    \caption{Descriptive Statistical Results of Timing of Intent Clarification for Different Strategies. }
    \label{tab:timing_results}
    \begin{tabular}{lcccccc}
        \toprule
        \textbf{Scenario} & \textbf{Median (Mdn)} & \textbf{Mean} & \textbf{Test Statistic (W)} & \textbf{p-value} & \textbf{Effect Size (r)} \\
        \midrule
        Fuzzy Boundary (n=28) & 4.0 (3.0) & 3.50 & 59.0 & 0.044 & 0.23 \\
        Straight Advantage (n=28) & 4.0 (3.0) & 3.79 & 52.5 & 0.022 & 0.25 \\
        Straight Priority (n=28) & 4.0 (4.0) & 4.14 & 21.0 & 0.009 & 0.25 \\
        Left Turn Advantage (n=28) & 4.0 (3.0) & 4.04 & 6.0 & 0.000 & 0.48 \\
        Left Turn Priority (n=28) & 5.0 (3.0) & 4.50 & 19.5 & 0.000 & 0.52 \\
        All Scenarios (n=140) & 4.0 (3.0) & 4.00 & 736.0 & 0.000 & 0.33 \\
        \bottomrule
    \end{tabular}
    \smallskip
\end{table*}

\section{Conclusion}
\label{sec:5}
In an endeavor to narrow the divide between AVs and HVs and ensure that AVs can implicitly convey social intent in a manner understandable to HVs in mixed-traffic scenarios, we have introduced an innovative framework for socially-compliant trajectory planning that is robust in implicit intent expression at the unprotected left-turn scenarios.

Our proposed framework is organized into three components: trajectory generation, trajectory evaluation, and trajectory selection. The experimental results substantiate the efficacy of our framework, demonstrating a strong resemblance to actual human trajectories, considerable enhancements in intent expression, safety, and efficiency, along with improved computational efficiency and learning outcomes. Our method shows a 77\% match with the actual trajectory distribution, an average offset of 85\% from the real trajectory, an average travel time of 7.4 seconds within the intersection, and a decrease in the average computation time by 41.1\%.
Concurrently, experiments conducted within the simulation platform and human-in-loop driving platform unequivocally showcase the effectiveness and precedence of the algorithm.

For future research, our research focus will expand the applicability of our trajectory planning methodology to include a broader range of interactive scenarios. Furthermore, we aim to confirm the effectiveness and scalability of our methodology through driving simulation experiments and real-vehicle interactions.

% \newpage

\bibliographystyle{IEEEtran} 
\bibliography{reference}

\vfill



\end{document}


