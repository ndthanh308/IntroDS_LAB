\documentclass[12pt]{amsart}

\usepackage{xy, graphicx, color,hyperref,verbatim,cleveref}
\pdfoutput=1
\usepackage{amsfonts, ytableau,array}
\usepackage[utf8]{inputenc}
\usepackage{amssymb,amsmath,amsthm,bm,mathrsfs,enumerate,bbm,enumitem}
\usepackage[a4paper, total={6in, 8in}]{geometry}
\usepackage[english]{babel}
\usepackage{fancyhdr}
\usepackage[utf8]{inputenc}
\usepackage{mathtools,cite,mathdots}
\usepackage{tikz}
\usepackage{tikz-cd}
\usetikzlibrary{matrix}
%\usepackage{siunitx}
\usepackage{booktabs}
\usepackage{blindtext}
\usepackage{array}
\newcolumntype{P}[1]{>{\centering\arraybackslash}p{#1}}
\PassOptionsToPackage{linktocpage}{hyperref}
   \hypersetup{
           breaklinks=true,  
           colorlinks=true,  
           pdfusetitle=true, 
           citecolor={blue!80!black}
        }

\xyoption{all}

 

\newenvironment{roster}
 {\begin{enumerate}[font=\upshape,label=(\roman*)]}
 {\end{enumerate}}

\renewcommand{\familydefault}{\rmdefault}
 \newtheorem{thm}{Theorem}[section]
\newtheorem{cor}[thm]{Corollary}
\newtheorem{lemma}[thm]{Lemma}
\newtheorem{prop}[thm]{Proposition}
%\newtheorem{cor}{Corollary}[thm]
\newtheorem{problem}[thm]{Problem}
\newtheorem{ex}[thm]{Example}
\theoremstyle{definition}
\newtheorem{remark}[thm]{Remark}
\theoremstyle{definition}
\newtheorem{defn}[thm]{Definition}





%\theoremstyle{definition}
%\newtheorem{defn}{Definition}
%\theoremstyle{theorem}
%\newtheorem{c.intro}{Corollary}

%\newtheorem{cor}{Corollary}[thm]
%\newtheorem{coly}{Corollary}[prop] 
%\theoremstyle{definition}
%\newtheorem{remark}{Remark}
%\newtheorem{ex}{Example}
%\newtheorem{result}{Result}

\newcommand{\nc}{\newcommand}

\newcommand{\overbar}[1]{\mkern 1.5mu\overline{\mkern-1.5mu#1\mkern-1.5mu}\mkern 1.5mu}

\nc{\mc}{\mathcal}
\nc{\mb}{\mathbb}
\nc{\mf}{\mathfrak}
\nc{\ul}{\underline}
\nc{\ol}{\overline}
\nc{\N}{\mb N}
\nc{\R}{\mb R}
\nc{\Z}{\mb Z}
\nc{\Q}{\mb Q}
\nc{\C}{\mb C}
\nc{\F}{\mb F}
\nc{\dmo}{\DeclareMathOperator}

\nc{\mat}[4]{
    \begin{pmatrix}
      #1 & #2 \\
      #3 & #4
    \end{pmatrix}
}
\dmo{\Ker}{Ker} \dmo{\val}{val} \dmo{\ord}{ord}
\dmo{\I}{I}
\dmo{\II}{II}
\dmo{\odd}{odd}
\dmo{\sgn}{sgn}
\dmo{\dett}{det}
\dmo{\Span}{Span}
\dmo{\Syl}{Syl}
\dmo{\diag}{diag}
\dmo{\Sq}{Sq}
\dmo{\irr}{Irr}
\nc{\beq}{\begin{equation*}}
\nc{\eeq}{\end{equation*}}
\nc{\half}{\frac{1}{2}}
\dmo{\Ima}{Im}
\dmo{\too}{top}
\dmo{\mult}{mult}
\dmo{\Mod}{mod}

\dmo{\core}{core}
\dmo{\res}{res}
\dmo{\lin}{lin}
 
 \dmo{\St}{St}
 \dmo{\st}{st}
 
 \dmo{\Tr}{Tr}
 \dmo{\RO}{RO}
\dmo{\Sp}{Sp}
\dmo{\SO}{SO}
\dmo{\SL}{SL}
\dmo{\GL}{GL}
 \dmo{\Spin}{Spin}
\dmo{\GSp}{GSp}
\nc{\la}{\lambda}
  \nc{\eps}{\varepsilon}
  
 \nc{\lip}{\langle}
 \nc{\rip}{\rangle}
\nc{\gm}{\gamma}

\dmo{\Perm}{Perm}
\dmo{\Res}{Res}
\dmo{\Ind}{Ind}
\dmo{\ind}{ind}
\dmo{\tr}{tr}
\dmo{\Sym}{Sym}
\dmo{\reg}{reg}
\dmo{\End}{End}
\dmo{\Hom}{Hom}
\dmo{\Pin}{Pin}
\dmo{\Or}{O}
\dmo{\SW}{SW}
\dmo{\orb}{orb}
\setcounter{tocdepth}{1}



\title[Stiefel-Whitney Classes]{Stiefel-Whitney Classes of representations of Dihedral Groups} 
\author{Sujeet Bhalerao}
\author{Rohit Joshi}
\author{Neha Malik}

\address{Department of Mathematics, University of Illinois Urbana-Champaign} \email{sujeetbhalerao@gmail.com}


\address{Indian Institute of Science Education and Research, Pune, India} \email{rohitsj@students.iiserpune.ac.in}

\address{Indian Institute of Science Education and Research, Mohali, India} \email{51nehamalik94@gmail.com}  


\DeclareMathOperator{\ann}{ann}
\DeclareMathOperator{\O_2}{O_2}
\DeclareMathOperator{\GU}{GU}
\DeclareMathOperator{\Id}{Id}
\DeclareMathOperator{\Aut}{Aut}
\DeclareMathOperator{\FG}{FG}
\DeclareMathOperator{\Gal}{Gal}
\DeclareMathOperator{\Nm}{Nm}
\DeclareMathOperator{\SU}{SU}
\DeclareMathOperator{\pr}{pr}

\dmo{\ORep}{ORep}
\dmo{\Rep}{Rep}

\usepackage{blindtext,graphicx}
\usepackage[absolute]{textpos}
\setlength{\TPHorizModule}{1cm}
\setlength{\TPVertModule}{1cm}

\newcommand{\s}{\sigma}
\newcommand{\as}{\alpha}
\newcommand{\bb}{\beta}
\newcommand{\La}{\Lambda}
\newcommand{\tht}{\chi}
\newcommand{\e}{\equiv}
\newcommand{\g}{\gamma}
\newcommand{\ep}{\varepsilon}
\newcommand{\f}{\mathbb{F}}
\newcommand{\z}{\mathbb{Z}}
\newcommand{\rr}{\mathbb{R}}
\newcommand{\cc}{\mathbb{C}}
\newcommand{\om}{\Omega}
\newcommand{\rta}{\rightarrow}
\newcommand{\ph}{\varphi}
\newcommand{\ze}{\zeta}
\newcommand{\ps}{\psi}
\newcommand{\hh}{\mathbb{H}}

\newcommand{\fe}{\mathfrak{e}}






\begin{document}


\maketitle
\tableofcontents
\begin{abstract}
We compute the total Stiefel-Whitney Classes (SWCs) for representations of dihedral groups $D_{m}$ in terms of character values of order two elements. We also provide criteria to identify spinorial representations and those with non-trivial mod $2$ Euler class.
\end{abstract}

\section{Introduction}
Let $G$ be a finite group, and $\pi$ be an orthogonal representation of $G$. To $\pi$, one can associate cohomology classes
$w_i(\pi)$, living in $H^i(G,\z/2\z)$, called \textit{Stiefel-Whitney Classes}(SWCs) of $\pi$. Their sum 
$w(\pi)=w_0(\pi)+w_1(\pi)+\hdots$
is known as the \textit{total SWC} of $\pi$. These well known classes, with nice properties, are of importance in studying group cohomology through the representations. We refer to \cite{GKT}, \cite[Section 2.3]{NSSL2} for more details.








The paper \cite{Ganguly} of Ganguly and Spallone computed the second SWCs to classify the spinorial representations of symmetric groups. This led to a program of calculating the total SWCs of representations in terms of character values for important groups. Joshi-Ganguly completed the case of $\GL(n,q)$ for $q$ odd in \cite{GJgln}.  The SWCs for $\SL(2,q)$ were computed in \cite{NSSL2} by Malik-Spallone. In this paper, we describe SWCs of representations of dihedral groups in terms of character values. It fits into the program since $O(2,q)$ is a dihedral group when $q$ is odd. An important tool in our calculations is the cohomological \emph{detection} of a group by its subgroups.

We write $H^*(G)$ for $H^*(G,\z/2\z).$ The subgroups $K_1,K_2$  detect the mod $2$ cohomology of $G$ if the restriction map
$$H^*(G)\to H^*(K_1)\oplus H^*(K_2)$$
is an injection. For instance, it is well known \cite[Chapter II, Corollary 5.2]{milgram} that a Sylow $2$-subgroup $K$ detects the mod $2$ cohomology of $G$, meaning the restriction $H^*(G)$ to $H^*(K)$ is injective. 

Let $D_m$ be the dihedral group of order $2m$ with $`r$' the rotation by angle $2\pi/m$ and a reflection $`s$' as its generators.  All representations of $D_m$ are orthogonal. The calculation of SWCs for abelian dihedral groups $D_1$, $D_2$ is straightforward. We review them in Section \ref{small groups}. When $m$ is a multiple of $4$, from \cite{VS}, the mod $2$ cohomology of $D_m$ is

$$H^*(D_{m}, \Z/2\Z)=\frac{\Z/2\Z[x,y,w]}{(y^2+xy)},  $$
where $x,y$ are certain elements of degree $1$ and $w$ is of degree $2$. In this case, the detection of $H^*(D_m)$ by two Klein-$4$ groups is known (see \cite[Proposition 3.3]{FP} for instance or Section \ref{detect} below). We use this detection to get:








%$$D_{2m}=\langle r,s\mid r^{m}=1, s^2=1, rs=sr^{-1}\rangle.$$


%When $4$ divides $m$,  we have (See for instance \cite{VS}. )

\begin{thm}\label{main}
	Let $m$ be a multiple of $4$, and $G=D_m$. Let $\pi$ be a representation of $G$. Then, 
	$$w(\pi)=(1+y)^{a_\pi}(1+x+y)^{b_\pi}(1+x+w)^{c_\pi},$$
	where 
	\begin{align*}
		a_\pi&=\frac14\big(\chi_\pi(1)-2\chi_\pi(rs)+\chi_\pi(r_c)\big),\\
		b_\pi&=\frac14\big(\chi_\pi(1)-2\chi_\pi(s)+\chi_\pi(r_c)\big), \text{ and }\\
		c_\pi&=\frac14\big(\chi_\pi(1)-\chi_\pi(r_c)\big).
	\end{align*}
\end{thm}
\noindent Here $r_c=r^{m/2}$, and $\chi_\pi(g)$ is the character value of representation $\pi$ at $g\in G$.   \\

There are several corollaries of Theorem \ref{main}. First, we have:
\begin{cor}\label{wtriv}
A representation $\pi$ of $D_m$ is trivial if and only if $w(\pi)=1$.
\end{cor}


 Let $W(G)$ be the subgroup generated by total SWCs of orthogonal representations (see \cite[Section 2.6]{NSSL2}). For $G=D_m$, we obtain:

\begin{cor}\label{W(G)iso} 
When $m$ is a multiple of $4$, the group $W(D_m)$ is free abelian with generators $1+y,1+x+y, 1+x+w.$
\end{cor}



For a representation $\pi$ of degree $d$, we define the top SWC $w_{\too}(\pi)=w_d(\pi)$.  When $\det\pi=1$, there is another important characteristic class $e(\pi)\in H^d(G,\z)$, called the \emph{Euler class} of $\pi$. From \cite[Property 9.5]{Milnor}, the top SWC is the reduction of $e(\pi)$ mod $2$. Here, we describe $\pi$ with $w_{\too}(\pi)\neq 0$:

\begin{cor}\label{topswc} 
The top SWC of $\pi$ is non-zero if and only if either $e_\pi=0$ or $f_\pi=0$ where
\begin{align*}
e_\pi&=\frac14\big(\chi_\pi(1)+2\chi_\pi(s)+\chi_\pi(r_c)\big),\\
f_\pi&=\frac14\big(\chi_\pi(1)+2\chi_\pi(rs)+\chi_\pi(r_c)\big).
\end{align*}
\end{cor}
%\noindent(Here, $m_1(\pi)$ is the multiplicity of $1$ in $\pi$, which has the character formula $\cfrac{1}{|G|}\sum\limits_{g\in G}\chi_{\pi}(g).$)\\

For a complex vector space $V$, there is a double cover of the orthogonal group $\Or(V)$, called $\Pin(V)$. Let $\pi$ be an orthogonal representation of a group $G$. We say $\pi$ is \emph{spinorial} provided it lifts to $\Pin(V)$ (see \cite{JSP} for more details). %There is a cohomological criterion to find whether a given $\pi$ is spinorial or not. 
We have $\pi$ is spinorial if and only if $w_2(\pi)+w_1(\pi)\cup w_1(\pi)=0$ (see Section \ref{spindef} below). 

We write $a=_2b$ for $a=b\pmod 2$. We have the following criterion for spinorial representations of $D_m$:

\begin{cor}\label{spin} 
	A representation $\pi$ of $D_{m}$ is spinorial if and only if $c_\pi$ is even, and
	$$
	\binom{a_\pi+1}{2}=_2 \binom{b_\pi+1}{2}=_2  \binom{c_\pi + 1}{2}.
	$$
\end{cor}
We also give the spinoriality criterion for certain representations of $D_m\times D_{m'}$. Let $\pi, \pi'$ be representations of $D_m$ and $D_{m'}$ respectively with $\deg \pi=d$, $\deg\pi'=d'$. Then $\Pi=\pi\boxtimes\pi'$, the external tensor product, is an orthogonal representation of $D_m\times D_{m'}$. We have:









\begin{thm}\label{prodDm}
Let $G=D_{m}$ and $G'= D_{m'}$. The representation $\Pi$ is spinorial if and only if both of the following hold:
\begin{roster}
\item The restriction of $\Pi$ to $G\times 1$ and $1\times G'$ are both spinorial, and
\item $(dd'+1)$ is even or $\det\pi=1$ or $\det\pi'=1$.
\end{roster}
\end{thm}

This paper is organized as follows. In section \ref{notpre} we set up notation and review the group cohomology of $D_{m}$. In section \ref{small groups} we review the calculation of SWCs for $C_2$ and obtain SWCs for representations of $C_2 \times C_2$. In section \ref{D2m} we prove Theorem \ref{main} and Corollary \ref{wtriv}. Section \ref{sectW(G)} is dedicated to prove Corollary \ref{W(G)iso},  Corollary \ref{topswc} and Corollary \ref{spin}. In the last section we give a general spinoriality result about products of groups.   

{\bf Acknowledgments:} The authors would like to thank Steven Spallone for helpful conversations. The contribution of the first author comes from his MS Thesis, at IISER Pune, India. The second author of this paper is a post doctoral fellow at IISER Pune, India and was supported by National Board of Higher Mathematics fellowship, India. The third author of this paper was supported by the Institute fellowship from the Indian Institute of Science Education and Research(IISER), Mohali, India.

\section{Notations and Preliminaries}\label{notpre} 



 Let $D_{m}$ be the group of order $2m $ with  presentation
$$D_{m}=\langle r,s\mid r^{m}=1, s^2=1, rs=sr^{-1}\rangle.$$
%Write $m=2^{n}k$ with $n\in \mathbb{N}$ and $k$ odd. %Then $D_{2^n}$, with usual inclusion, is a Sylow $2$-subgroup of $D_{2m}$. Due to detection \ref{}, we only consider the dihedral groups of order $2^n$.


 We now recall the representation theory of $D_{m}$ (see \cite[Section 5.3]{serre1977linear} for instance).

\subsection{Irreducible Representations of $D_{m}$} 
Write $`\sgn$' for the non-trivial linear character of $D_1$.  

For $m $ even, the group $D_{m}$ has four linear characters, namely $1$, $\chi_s$, $\chi_r$, $\chi_{rs}$:
\begin{align*}
1:(r,s)&\mapsto (1,1)\\
\chi_s:(r,s)&\mapsto (1,-1)\\
\chi_r:(r,s)&\mapsto (-1,1)\\
\chi_{rs}:(r,s)&\mapsto (-1,-1).
\end{align*}
%$$\sgn_{ij}:=\sgn^i\boxtimes \sgn^j\quad;\quad i,j\in \{0,1\}.$$ 
%$$1\boxtimes 1,\sgn\boxtimes 1, 1\boxtimes \sgn, \sgn\boxtimes\sgn.$$
%The group $D_2=$ has only two linear characters, $1$ and $\sgn$.


Set $\theta_k=\frac{2\pi k}{m}$. 
There are also irreducible $2$-dimensional representations enumerated by $k=1,2,\hdots,m/2-1$:
$$\sigma_k:D_{m}\to \GL(2,\cc)$$ is given by $$\sigma_k(r)=\begin{pmatrix}
	\cos\theta_k&&-\sin \theta_k\\[2mm]
	\sin\theta_k&&\cos\theta_k
\end{pmatrix}\quad,\quad\sigma_k(s)=\begin{pmatrix}
	0&&1\\
	1&&0
\end{pmatrix}.$$

Write $\sigma=\sigma_1$, called the standard $2$-dimensional representation of $D_{m}$.

Whereas when $m$ is odd, $D_{m}$ has two linear characters $1$, and $\chi_s$. Also, the representations $\sigma_k$ for $k=1,\hdots,(m-1)/2$ defined above are irreducible.

Note that all the above representations are orthogonal.

\subsection{Restriction to a Sylow-$2$ Subgroup}\label{ressyl}

Set $\chi_0=1$. For this section, write $\chi_{i, m}$ for the linear characters $\chi_i$ for $D_{m}$ and $\sigma_{k,m}$ for $\sigma_k$ for $D_{m}$. 



 Assume $m=2^{n}\ell$ for some $n\in \mathbb{N}$ and $\ell$ odd. $D_{2^n}$, with usual inclusion, is a Sylow $2$-subgroup of $D_{m}$. It is known \cite[Chapter 1, Section 4]{VS} that 

\begin{equation}\label{cohomD2n} H^*(D_{2^n})=\begin{cases}
\z/2\z[v]& n=0,\\
\z/2\z[v_1,v_2]& n=1,\\
\z/2\z[x,y,w]/(y^2+xy)& n\geq 2.
\end{cases}\end{equation}
where $v=w_1(\chi_{s,1})$, $v_1=w_1(\chi_{r,2})$, $v_2=w_1(\chi_{s,2})$, and $x= w_1(\chi_{s,2^n}), y=w_1(\chi_{r,2^n})$ and $w=w_2(\sigma_{1,2^n})$ for $n\geq 2$.
%where $x= w_1(\chi_{s,2^n}), y=w_1(\chi_{r,2^n})$ and $w=w_2(\sigma_{1,2^n})$.


Consider the inclusion $\iota: D_{2^n} \to D_{m}$. Then, the restriction map\begin{equation}\label{incl}\iota^*:H^*(D_{m})\to H^*(D_{2^n})\end{equation}
is an isomorphism \cite[Theorem 4.6]{VS}. To understand $\iota^*$, the restrictions of representations of $D_{m}$ to $D_{2^n}$ are of interest. Below we define the elements of $H^*(D_{m})$ which map to the generators of $H^*(D_{2^n})$ under $\iota^*$.
% Here, we also understand the generators of $H^*(D_{2m})$ as the SWCs of representations of $D_{2m}$.
This is done in three cases.

For $m$ odd, the linear character $\chi_{s,m}$ restricted to $D_{1}$ is the $\sgn$ representation. This gives,
\begin{align*}
\iota^*(w_1(\chi_{s,m}))&=w_1(\sgn)\\
&=v.
\end{align*}
Without any confusion, we simply write \begin{equation}\label{v}w_1(\chi_{s,m})=v.\end{equation}


Now, consider $m \equiv 2 \pmod 4$. The restriction of $\chi_{i,m}$ to $D_2$ is $\chi_{i,2}$ for each $i=0,r,s,rs$. Therefore,
$\iota^*(w_1(\chi_{r,m}))=v_1,$ and
$\iota^*(w_1(\chi_{s,m}))=v_2$
and we write \begin{equation}\label{v1v2} 
\begin{split}w_1(\chi_{r,m})&=v_1, \text{ and }\\
w_1(\chi_{s,m})&=v_2.
\end{split}
\end{equation}% in $H^*(D_{2m})$.
%\begin{align*}
%\chi_{i,2m}|_{D_{4}}=\chi_{i,4}
%\end{align*}

% it is easy to see that
%%\begin{itemize}
  	%\item if $k$ is odd then $\sigma_k\mid_{D_4} \equiv \chi_r  \oplus \chi_{rs}$
    %\item if $k$ is even then $\sigma_k\mid_{D_4} \equiv \chi_0 \oplus \chi_s$
%\end{itemize}
Finally, take $m \equiv 0 \pmod 4$. In this case, $\chi_{i,m}$ restricts to $ \chi_{i, 2^n}$, and $\sigma_{1,m}$ restricts to $\sigma_{1,2^n}$. Again, since $\iota^*$ is an isomorphism, we put
%Write `res' for the restriction from $D_{2m}$ to $D_{2^n}$. Observe that
%	\res\chi_{i,2m} = \chi_{i, 2^n},\\
%	\res\sigma_{1,2m} = \sigma_{1,2^n}.
%\end{align*}
\begin{equation}\label{xyw} 
\begin{split}
w_1(\chi_{s,m})&=x\\
w_1(\chi_{r,m})&=y\\
w_1(\sigma_{1,m})&=w.
\end{split}
\end{equation}

With the notations given by \eqref{v},\eqref{v1v2} and \eqref{xyw}, an equality
similar to \eqref{cohomD2n} holds for $H^*(D_{m})$. Also, we now will simply use $\chi_r$, $\chi_s$, $\sigma_1$ for both $D_m$ and $D_{2^n}$.

\subsection{Spinoriality} \label{spindef}
Consider an orthogonal complex representation $\pi:G\to \Or(V)$. Denote the double cover of $\Or(V)$ defined in \cite{JSP}, by Pin$(V)$. We say $\pi$ is \textit{spinorial} if it lifts to Pin$(V)$. By \cite[Page 97, Proposition(6.4)]{BrokerDieck} we see that a complex orthogonal representation can be realized as a real representation ($\pi_0, V_0$). The fact that, for a finite group $G$, $(\pi,V)$ is spinorial if and only if $(\pi_0, V_0)$ is spinorial comes from the commutative diagram

\begin{equation}
	\xymatrix{\Pin(V_0) \ar[r] \ar[d] & \Pin(V) \ar[d]\\
	\Or(V_0) \ar[r] & \Or(V).\\
}
\end{equation}
Here-on-wards we will write $w_i(\pi)(\text{ resp. }w(\pi))$  for $w_i(\pi_0) (\text{ resp.  }w(\pi_0))$. 
To say $\pi$ is spinorial, there is a cohomological criterion asking:
$$ w_2(\pi)+w_1(\pi)\cup w_1(\pi)=0.$$
A proof of this can be found in \cite{Ganguly} when $V$ is a real vector space.


%Therefore, the group cohomology of $H^*{D_{2m}}$, isomorphic to $H^*{D_{2^n}}$, is given by \eqref{HD2n} with 

%Let $D_{2m}$ be the dihedral group of order $2m$. Assume $m=2^{n-1}k$ for some $n\in \mathbb{N}$ and $k$ odd. Clearly, $D_{2^n}$ is a Sylow $2$-subgroup of $D_{2m}$, and detects the mod $2$ cohomology of $D_{2m}$. Let $\iota: D_{2^n} \to D_{2m}$. The restriction map

%$$\iota^*:H^*(D_{2m})\to H^*(D_{2^n})$$
%is an isomorphism \cite{VS} as:
%${\iota}^*(w_1(\chi_{i,2m}))= w_1(\chi_{i,2^n}) $ and
%${\iota}^*(w_1(\sigma_{1,2m}))= w_1(\sigma_{1,2^n})$.
%Since $x= w_1(\chi_s), y=w_1(\chi_r)$ and $w=w_2(\sigma_1)$, $\iota$ is a isomorphism.




%$\chi_{i}$ and $\sigma_1$ for $D_{2m}$ where $i \in \{0,r,s,rs\}$ restricts to $\chi_{i}$ and $\sigma_1$ for $D_{2^n}$. 




\section{The case of $C_2$ and $C_2 \times C_2$}\label{small groups}


 Let $C_2=\{\pm 1\}$, the cyclic group of order $2$. Then, $D_1$ is simply $C_2$ and $D_2$ is the Klein-$4$ group $C_2\times C_2$.
%Let $G$ be a finite group, and $\pi$ be an orthogonal representation of $G$. To $\pi$, one can associate 
%$$w_i(\pi)\in H^i(G,\z/2\z)\quad,\quad i=0,1,2,\hdots$$
%called the $i$th \textit{Stiefel-Whitney Class}(SWC) of $\pi$. Their sum 
%$$w(\pi)=w_0(\pi)+w_1(\pi)+\hdots$$
%is known as the \textit{total SWC} of $\pi$. Please see \cite[Section 2.3]{NSSL2} for more details. 
%From now on, we write $H^*(G)=H^*(G,\z/2\z).$
%Let $C_2$ be the cyclic group of order $2$. Write $`\sgn$' for the non-trivial linear character of $C_2$. 
%It is well known that
%$$H^*(C_2)=\z/2\z[v],$$
%with $v=w_1(\sgn)$. 
\subsection{The Cyclic group $C_2$}
Let $\pi$ be a representation of $C_2$. Its total SWC is known (see, for instance, in \cite[Lemma 2.5]{NSSL2}). With $v=w_1(\sgn)$, we have
%The total SWC of a representation $\pi$ of $C_2$ (or $D_1$) is
\begin{equation}\label{swcC2}w(\pi)=(1+v)^{b_\pi},\end{equation}
where $b_\pi=\frac 12(\chi_\pi(1)-\chi_\pi(-1))$.

\subsection{The Klein-$4$ Group}\label{C2C2}
Consider the group $C_2\times C_2$ and the projection maps $\pr_i:C_2\times C_2\to C_2$ for $i=1,2$.
 By K\"unneth, we have
$$H^*(C_2\times C_2)\cong \z/2\z[v_1,v_2],$$
where $v_1=\pr^*_1(v)=w_1(\sgn\boxtimes 1)$ and $v_2=\pr^*_2(v)=w_1(1\boxtimes \sgn)$. Here $\boxtimes$ means the external tensor product.

\begin{prop}
Let $\pi$ be a representation of $C_2\times C_2$. Then, the total SWC of $\pi$ is
$$w(\pi)=(1+v_1)^{m_1}(1+v_2)^{m_2}(1+v_1+v_2)^{m_3},$$
where %m_0&=\frac14\left(\deg\pi+\chi_\pi(-1,1)+\chi_\pi(1,-1)+\chi_\pi(-1,-1)\right)\\
\begin{align*}
m_1&=\frac14\left(\deg\pi-\chi_\pi(-1,1)+\chi_\pi(1,-1)-\chi_\pi(-1,-1)\right)\\
m_2&=\frac14\left(\deg\pi+\chi_\pi(-1,1)-\chi_\pi(1,-1)-\chi_\pi(-1,-1)\right)\\
m_3&=\frac14\left(\deg\pi-\chi_\pi(-1,1)-\chi_\pi(1,-1)-\chi_\pi(-1,-1)\right).
\end{align*}

\end{prop}

\begin{proof}


 %The group $G$ has 4 irreducible representations, $$\sgn_{ij}:=\sgn^i\boxtimes \sgn^j\quad;\quad i,j\in \{0,1\}.$$ 
 Any representation $\pi$ of $C_2\times C_2$ has the form
$$\pi=m_01\oplus m_1(\sgn\boxtimes 1)\oplus m_2(1\boxtimes \sgn)\oplus m_3(\sgn\boxtimes\sgn).$$
Note all representations of $G$ are orthogonal.
By multiplicativity of SWCs, we have
$$w(\pi)=(1+v_1)^{m_1}(1+v_2)^{m_2}(1+v_1+v_2)^{m_3}.$$
To express $m_i$ in terms of character values, we use the following equations:
\begin{align*}
\chi_\pi(1,1)&=m_0+m_1+m_2+m_3\\
\chi_\pi(-1,1)&=m_0-m_1+m_2-m_3\\
\chi_\pi(1,-1)&=m_0+m_1-m_2-m_3\\
\chi_\pi(-1,-1)&=m_0-m_1-m_2+m_3.
\end{align*}
Solving these for $m_i$ completes the proof.
\begin{comment}
As a matrix equation:
\begin{align*}\begin{pmatrix}
m_0\\
m_1\\
m_2\\
m_3
\end{pmatrix}&=\begin{pmatrix}
1&1&1&1\\
1&-1&1&-1\\
1&1&-1&-1\\
1&-1&-1&1
\end{pmatrix}^{-1}\begin{pmatrix}
\chi_\pi(1,1)\\
\chi_\pi(-1,1)\\
\chi_\pi(1,-1)\\
\chi_\pi(-1,-1)
\end{pmatrix}\\[2mm]
&=\cfrac14\begin{pmatrix}
1&1&1&1\\
1&-1&1&-1\\
1&1&-1&-1\\
1&-1&-1&1
\end{pmatrix}\begin{pmatrix}
\chi_\pi(1,1)\\
\chi_\pi(-1,1)\\
\chi_\pi(1,-1)\\
\chi_\pi(-1,-1).
\end{pmatrix}
\end{align*}
\end{comment}
\end{proof}
%From above, we have the SWCs for $D_1$, $D_2$. The next section deals with $m\geq 2.$
%Identifying $C_2\times C_2$ with $D_4$ as $(-1,1)\leftrightarrow r$, $(1,-1)\leftrightarrow s$, we have:


%In general, let $G=C_2^n$, and $\pi$ be a representation of $G$. This looks like
%$$\pi=\bigoplus_{\vec x\in B_n}m_{\vec x}\sgn_{\vec x},$$
%where $B_n$ is the set of all binary vectors of length $n$. By multiplicativity of SWCs, we get
%$$w(\pi)=$$

\section{Dihedral Groups}\label{D2m} 

%Let $D_{m}$ be the dihedral group of order $2m$. Assume $m=2^{n-1}k$ for some $n\in \mathbb{N}$ and $k$ odd. Clearly, $D_{2^n}$ is a Sylow $2$-subgroup of $D_{2m}$, and detects the mod $2$ cohomology of $D_{2m}$
Recall $D_{2^n}$ sits in $D_m$ under the inclusion $\iota$, assuming $m=2^nl$ with $n\in\mathbb{N}$ and $l$ odd. Since $\iota$, as  in \eqref{incl}, is an isomorphism, it suffices to work with $D_{2^n}$ to find the SWCs for $D_m$.


The total SWCs for $D_1$ are given in Equation \eqref{swcC2}, and the same formula holds for $D_m$ when $m$ is odd, due to the isomorphism \eqref{incl}. Also, by identifying $C_2\times C_2$ with $D_2$ as $(-1,1)\leftrightarrow r$, $(1,-1)\leftrightarrow s$, we have the SWC formula for $D_2$ from Section \ref{C2C2}, which more generally holds for all dihedral groups $D_m$ when $m\equiv 2\pmod 4$, again due to \eqref{incl}.







In this section, let $4$ divides $m$, and $G=D_{m}$. Write $r_c=r^{m/2}$, the non-trivial central element of $G$.







 %From \cite{VS}, the mod $2$ cohomology of a dihedral group is,
%$$H^*(D_{2^n})\cong \z/2\z[x,y,w]/(y^2+xy),$$
%where $x=w_1(\chi_s)$, $y=w_1(\chi_r)$ and $w=w_2(\sigma)$.
\subsection{Detection}\label{detect} 
 We consider the following subgroups of $G$:
\begin{align*}
E_1&=\{1,s,r_c,sr_c\},\\
E_2&=\{1,rs,r_c,rsr_c\}.
\end{align*}
Both $E_1,E_2$ are isomorphic to the Klein $4$-group. Let $\as_1,\beta_1$ be the linear characters of $E_1$ given by,
\begin{align*}
\as_1:(s,r_c)&\mapsto(-1,1)\\
 \beta_1:(s,r_c)&\mapsto(1,-1).
\end{align*}Then, $H^*(E_1)\cong \z/2\z[v_1,v_2],$ with $v_1=w_1(\as_1),$
$v_2=w_1(\beta_1)$. 
Similarly, we consider the linear characters $\as_2,\beta_2$ of $E_2$ defined by,
\begin{align*} 
\as_2:(rs,r_c)&\mapsto(-1,1)\\
  \beta_2:(rs,r_c)&\mapsto(1,-1)\end{align*}
such that $H^*(E_2)\cong \z/2\z[u_1,u_2]$ with $u_1=w_1(\as_2)$, $u_2=w_1(\beta_2)$.\\


% \begin{tikzcd}
%  & (-1,1)  \\
 % (s,r_c)
 % \arrow[ur, "\as_1"] \arrow[dr, "\beta_1"'] & & (rs,r_c)\arrow[dl, "\beta_2"']\arrow[ul, "\as_2"] \\
 % &(1,-1)
%\end{tikzcd}

The following detection can be found in \cite[Proposition 3.3, Page 322]{FP}.  Although the result itself is accurate, an error is found within the accompanying proof provided in this book. We rectify the issue in the proof below. %The result is true but there is an error in the proof in this book, so we are giving the following correct proof.


\begin{prop}\label{e1e2} When $4$ divides $m$, the subgroups $E_1$, $E_2$ together detect the mod $2$ cohomology of $D_{m}$. %That is, the restriction map
%$$\res^*:H^*(G)\to H^*(E_1)\oplus H^*(E_2)$$
%is an injection.
\end{prop}
Proving this requires a lemma:

\begin{lemma}\label{detlem}
For a non-negative integer $d$, let $\mathcal{P}_d=\{(i,j) \in \Z_{\geq 0}\times\Z_{\geq0} \mid i + 2j = d \}$. Then, the set $S_d = \{a^ib^j(a+b)^j : (i,j) \in \mathcal P_d\}$ in $\Z/2\z[a,b]$ is linearly independent.	
\end{lemma}	

\begin{proof}
	The set $\mathcal P_d$ can be enumerated as follows:
	\begin{equation*} 
		(d,0), (d-2,1),\ldots,(d-2k,k), \ldots, 
		\begin{cases}
			(0,d/2), &\quad\text{ when } d \text{ is even}\\
			(1,(d-1)/2),&\quad\text{ when } d \text{ is odd}.
		\end{cases}
	\end{equation*}
For a polynomial $p\in\z/2\z[a,b]$, let $\deg_a(p)$ be the highest degree of $`a$' appearing in $p$. If we put $q_{ij}(a,b) = a^ib^j(a+b)^j$, then $\deg_a(q_{ij})=i+j$.

 Now, suppose $S_d$ is linearly dependent. That is there are $c_k$, not all zero, such that
\begin{equation}\label{ck}\sum\limits_{k=0}^{\lfloor d/2 \rfloor} c_k q_{d-2k,k}= 0. \end{equation}

Let $\ell$ be the least integer with $c_{\ell}=1$. For $k>\ell$, we have $$\deg_a(q_{d-2k,k})=d-k<d-\ell.$$
But then, the condition \eqref{ck} forces $c_\ell$ to be zero, which is a contradiction.

 %thus the degree goes on decreasing by 1 with increasing $j$. Since $\deg_a(q_{d,0})=d$ is highest among degrees, we have  $c_0=0$ and the proof follows by induction.
\end{proof}

\begin{proof}[Proof of Proposition \ref{e1e2}]
Consider the restriction map
$$\res^*:H^*(G)\to H^*(E_1)\oplus H^*(E_2).$$

We first find the images of $x,y,w$ under $\res^*$. For that, we see 
\begin{equation}\label{chires}
\res^G_{E_1}\chi_r=1,\;\res^G_{E_2}\chi_r=\as_2, \text{ and }\res^G_{E_i}\chi_s=\as_i \end{equation} for $i=1,2$. This, together with \eqref{cohomD2n}, implies
 $$\res^*(x)=(v_1,u_1)\quad,\quad\res^*(y)=(0,u_1).
$$

For $\res^*(w)$, consider the standard representation $\sigma$ of $G$. A simple eigenvalue calculation gives $\res^G_{E_1}\sigma$ is equivalent to the representation mapping
$$r_c\mapsto \begin{pmatrix}
-1&&0\\
0&&-1
\end{pmatrix}\quad,\quad s\mapsto \begin{pmatrix}
-1&&0\\
0&&1
\end{pmatrix}.$$
This gives $\res^G_{E_1}\sigma=\beta_1\oplus(\as_1\otimes\beta_1)$, implying
\begin{align*}
w(\res^G_{E_1}\sigma)&=w(\beta_1\oplus(\as_1\otimes\beta_1))\\
&=(1+v_2)(1+v_1+v_2)\\
&=1+v_1+v_2(v_1+v_2).
\end{align*}
The above equality uses $w(\as_1\otimes\beta_1)=1+w_1(\as_1)+w_1(\beta_1).$
Similarly, we have $w(\res^G_{E_2}\sigma)=1+u_1+u_2(u_1+u_2).$ Therefore, $\res^*$ maps
\begin{equation}\label{resxyw}
\begin{split}
x&\mapsto(v_1,u_1)\\
y&\mapsto (0,u_1)\\
w&\mapsto (v_2^2+v_1v_2,u_2^2+u_1u_2).
\end{split}
\end{equation}
To prove $\res^*$ is injective, consider an arbitrary element $g=\sum\limits_{i,j,k\geq 0} a_{ijk}x^iy^jw^k\in H^*(G)$. The summand might have terms like $y^jw^k$ for $i=0$ and $x^iw^k$ for $j=0$. But for the terms with $i,j>0$, we use $y^2+xy=0$ in $H^*(G)$ to simplify as follows:
\begin{align*}
x^iy^jw^k&=x^{i-1}(xy)y^{j-1}w^k\\
&=x^{i-1}(y^2)y^{j-1}w^k\\
&=x^{i-1}y^{j+1}w^k\\
&=y^{i+j}w^k.
\end{align*}
Thus, we can write
$$g=\sum_{i>0,\;j\geq 0}a_{ij}x^iw^j+\sum_{k>0,\;l\geq 0}b_{k,l}y^kw^l+\sum_{t\geq 0}c_tw^t.$$


Set $\delta_d=\begin{cases}
0& d\text{ is odd}\\
1& d\text{ is even}
\end{cases},$ and  $P_0=\emptyset$. For $d$ a positive integer, let $P_d=\{(i,j)\in \z_{>0}\times\z_{\geq0}:i+2j=d\},$ a subset of $\mathcal P_d$. Then,
\begin{align*}
\res^*(g)&=\res^*\left(\sum_{d> 0}\sum_{(i,j)\in P_d}a_{ij}x^iw^j+\sum_{d> 0}\sum_{(k,l)\in P_d}b_{k,l}y^kw^l+\sum_{t\geq 0}c_tw^t\right)\\
&=\sum_{d\geq0}\left(\sum_{(i,j)\in P_d}a_{ij}\res^*(x^iw^j)+\sum_{(k,l)\in P_d}b_{k,l}\res^*(y^kw^l)+\delta_dc_{d/2}w^{d/2}\right),
\end{align*}
where for each $d$, the term in the bracket belongs to $H^d(G)$. Suppose $\res^*(g)=0$. Since the ring $H^*(G)$ is graded, it is equivalent to saying for each $d\geq 0$,
$$\sum_{(i,j)\in P_d}a_{ij}\res^*(x^iw^j)+\sum_{(k,l)\in P_d}b_{k,l}\res^*(y^kw^l)+\delta_dc_{d/2}\res^*(w^{d/2})=0.$$

Thus, for the injectivity of $\res^*$, it is enough to prove that the set  $$\{\res^*(x^iw^j), \res^*(y^kw^l),\delta_d\res^*(w^{d/2}):(i,j),(k,l)\in P_d\}$$ is linearly independent for all $d\in \z_{\geq 0}.$ We have
\begin{align*}
\res^*(x^iw^j)&=(v_1^iv_2^j(v_1+v_2)^j,u_1^iu_2^j(u_1+u_2)^j)\text{ for } i,j\geq 0\\
\res^*(y^kw^l)&=(0,u_1^ku_2^l(u_1+u_2)^l)\text{  for } k>0, \:l\geq 0.
\end{align*}
This further reduces our problem to showing
$$\{(v_1^iv_2^j(v_1+v_2)^j,0), (0,u_1^ku_2^l(u_1+u_2)^l), \delta_d(v_2^{d/2}(v_1+v_2)^{d/2},u_2^{d/2}(u_1+u_2)^{d/2}):(i,j),(k,l)\in P_d\}$$
is linearly independent for all $d\in \z_{\geq 0}$. But this follows from Lemma \ref{detlem}. 


%Thus,  $\res^*$ is injective, 
%\begin{align*}
%\text{iff } &\{\res^*((x^i+y^i)w^j), \res^*(y^kw^l),\delta_d\res^*(w^{d/2}):(i,j),(k,l)\in P_d\}\text{ for each }d\geq 0\text{ is}\\
%\,\, & \text {linearly independent,}\\


%\text{iff } & \text{, in the ring  } \Z/2\Z[a,b],\text{ we have }\{a^ib^j(a+b)^j:(i,j)\in P_d\}\text{ for each }d\geq 0 \\
%\,\, &\text{ is linearly independent, which we will prove in the following lemma}.
%\end{align*}

\end{proof}
 We now use this detection to give SWCs for $G=D_m$.
% and consider
%\begin{align*}
%S_{x,d}&=\{\res^*(x^iw^j):(i,j)\in P_d\}\\
%S_{y,d}&=\{\res^*(y^kw^l):(k,l)\in P_d\}.
%\end{align*}

\subsection{Formula for SWCs}
  Let $\pi$ be a representation of $G$. It has the form\begin{equation}\label{fpi}
\pi=n_01\oplus n_s\chi_s\oplus n_r \chi_r\oplus n_{rs}\chi_{rs}\oplus\bigoplus_{i=1}^{m/2-1}m_i\sigma_i,
\end{equation}
where $n_0$, $n_s$, $n_r$, $n_{rs}$, $m_i$ are non-negative integers. We put $$m_e=\sum\limits_{i\text{ even }}m_i\quad, \quad m_o=\sum\limits_{i\text{ odd}}m_i.$$
 

For $w(\pi)$, we first find the SWCs for $\sigma_k$. Consider the subgroups $E_1,E_2$ with linear characters $\as_i,\beta_i$ as defined in Section \ref{detect}. From an eigenvalue calculation, we observe that
\begin{equation}\label{sigres}\res^G_{E_i}\sigma_k=\begin{cases}
\beta_i\oplus (\as_i\otimes\beta_i),& \text{when $k$ is odd}\\
1\oplus\as_i,& \text{when $k$ is even}.
\end{cases}\end{equation}
In the case of $k$ odd, we then have
\begin{align*}
\res^*(w(\sigma_k))&=(w(\res^G_{E_1}\sigma_k), w(\res^G_{E_2}\sigma_k))\\
&=((1+v_2)(1+v_1+v_2),(1+u_2)(1+u_1+u_2))\\
&=(1+v_1+v_2(v_1+v_2),1+u_1+u_2(u_1+u_2))\\
&=(1,1)+(v_1,u_1)+(v_2^2+v_1v_2,u_2^2+u_1u_2).
\end{align*}
Thus, $w(\sigma_k)=1+x+w$, due to \eqref{resxyw}. Similarly, when $k$ is even, \begin{align*}
\res^*(w(\sigma_k))&=(1+v_1,1+u_1)\\
&=(1,1)+(v_1,u_1)
\end{align*}
which gives $w(\sigma_k)=1+x$ in this case.\\

Now, we prove our main result giving $w(\pi)$:









%\begin{equation}\label{fpi}
%\pi=n_01\oplus n_1\chi_s\oplus n_2 \chi_r\oplus n_3\chi_{rs}\oplus\bigoplus_{i=1}^{2^{n-2}-1}m_i\sigma_i,
%\end{equation}
%where $n_i,m_i$ are non-negative integers.\\






\begin{proof}[Proof of Theorem \ref{main}]
Let $\pi$ be as in \eqref{fpi}. % $S_e =\{i:1<i<m/2-1,\; i \text{ even}\},
%S_o=\{i:1\leq i\leq m/2-1,\; i \text{ odd}\}$
%\begin{align*}S_e &=\{i:1<i<m/2-1,\; i \text{ even}\},\\
%S_o&=\{i:1\leq i\leq m/2-1,\; i \text{ odd}\}.\end{align*}
By the multiplicativity of SWCs, we have
\begin{align*}
w(\pi)&=(1+x)^{n_s+m_e}(1+y)^{n_r}(1+x+y)^{n_{rs}}(1+x+w)^{m_o}.
\end{align*}
Since $(1+y)(1+x+y)=1+x$ in $H^*(G)$, the above formula becomes
$$w(\pi)=(1+y)^{a_\pi}(1+x+y)^{b_\pi}(1+x+w)^{c_\pi}.$$
with  $a_\pi=n_s+n_r+m_e$, $b_\pi=n_s+n_{rs}+m_e$, $c_\pi=m_o$.


To find the character formulas for $a_\pi,b_\pi$ and $c_\pi$, we restrict $\pi$ to $E_i$:
\begin{equation}\label{rese1e2}
\begin{split}
\res^G_{E_1}\pi&=(n_0+n_r+m_e)1\oplus (n_s+n_{rs}+m_e)\as_1\oplus m_o\beta_1\oplus m_o(\as_1\otimes\beta_1),\\
\res^G_{E_2}\pi&=(n_0+n_{rs}+m_e)1\oplus (n_s+n_r+m_e)\as_2\oplus m_o\beta_2\oplus m_o(\as_2\otimes\beta_2).
\end{split}
\end{equation} 
This comes from \eqref{chires} and \eqref{sigres}. For $\psi$  irreducible, let mult$(\psi,\varphi)$ denote the multiplicity of $\psi$ in $\varphi$. Here, we note that 
\begin{align*}
a_\pi&=\mult(\as_2,\res^G_{E_2}\pi)\\
b_\pi&=\mult(\as_1,\res^G_{E_1}\pi)\\
c_\pi&=\mult(\as_i\otimes\beta_i,\res^G_{E_i}\pi).
\end{align*}
We identify $E_1$ with Klein-$4$ group by $s\leftrightarrow (-1,1)$ and $r_c\leftrightarrow (1,-1)$. Similarly, for $E_2$, we identify $rs$ with $(-1,1)$ and $r_c$ again with $(1,-1)$.
%$a_\pi$ is the multiplicity of $\as_2$ in $\res^G_{E_2}$, $b_\pi$ is the 
%Here, $b_\pi$ is the coefficient of $\as_1$. Also, both $\beta_1$ and $\as_1\otimes\beta_1$ appear $c_\pi$ times in $\res^G_{E_1}\pi$.\\
This forces the identifications $\as_i\leftrightarrow\sgn\boxtimes 1$, $\beta_i\leftrightarrow 1\boxtimes \sgn$, and so on.


Now, we use the character formulas from Section \ref{small groups} to obtain

\begin{align*}
a_\pi&=\frac14(\chi_\pi(1)-\chi_\pi(rs)+\chi_\pi(r_c)-\chi_\pi(rsr_c)),\\
b_\pi&=\frac14(\chi_\pi(1)-\chi_\pi(s)+\chi_\pi(r_c)-\chi_\pi(sr_c)),\\
c_\pi&=\frac14(\chi_\pi(1)-\chi_\pi(s)-\chi_\pi(r_c)+\chi_\pi(sr_c)).
\end{align*}

The elements $s,sr_c$ are conjugates in $G$, and so are $rs, rsr_c$. Therefore, we have the desired formulas by putting $\chi_\pi(s)=\chi_\pi(sr_c)$, and  $\chi_\pi(rs)=\chi_\pi(rsr_c)$. 



\begin{comment}

that $s,sr_c$ belong the same conjugacy class to have
\begin{align*}
b_\pi&=\frac14(\chi_\pi(1)-\chi_\pi(s)+\chi_\pi(r_c)-\chi_\pi(sr_c))\\
&=\frac14(\chi_\pi(1)-2\chi_\pi(s)+\chi_\pi(r_c))
\end{align*}
Similarly, \begin{align*}
c_\pi&=\frac14(\chi_\pi(1)-\chi_\pi(s)-\chi_\pi(r_c)+\chi_\pi(sr_c))\\
&=\frac14(\chi_\pi(1)+\chi_\pi(r_c))
\end{align*}

Again by character formulas for Klein-$4$ group and the fact that $rs,rsr_c$ belong the same conjugacy class, we have
\begin{align*}
a_\pi&=\frac14(\chi_\pi(1)-\chi_\pi(rs)+\chi_\pi(r_c)-\chi_\pi(rsr_c))\\
&=\frac14(\chi_\pi(1)-2\chi_\pi(rs)+\chi_\pi(r_c))
\end{align*}
completing the proof.
\end{comment}

\end{proof}
\begin{ex} Let $\reg(G)$ be the regular representation of $G$. Here, we have
$$w(\reg(D_m))=\left(1+x^2+w+xw\right)^{m/2}.$$
\end{ex}
We now prove the corollaries to Theorem \ref{main}.


\section{Some Corollaries}\label{sectW(G)}


%To find the generators of $W(D_m)$, we investigate the least positive integers that appear as the exponents in $w(\pi)$ as $\pi$ varies over the orthogonal representations of $G$. 
Since all representations are orthogonal for $G=D_m$, we have 
$$W(G)=
\langle w(\pi):\pi\text{ irreducible }\rangle\leqslant H^\bullet(G).$$


When $m$ is odd, the character $\chi_s$ has its total SWC $(1+v)$. The multiplicativity of SWCs, then, gives
$$W(D_m)=\{(1+v)^n:n\in \z\}.$$
Similarly, from Section \ref{C2C2}, it is straightforward that when $m\equiv 2\pmod 4$, 
$$W(D_m)=\{(1+v_1)^a(1+v_2)^b(1+v_1+v_2)^c:a,b,c\in \z\}.$$

Now, we give this subgroup for $G=D_m$ for $m\equiv 0\pmod 4$:


\begin{proof}[Proof of Corollary \ref{W(G)iso}]
We define $\phi: \Z^3 \to W(G)$ by, $$\phi(a,b,c)=  (1+y)^a(1+x+y)^b(1+x+w)^c.$$ 
%Clearly, $\phi$ is a group homomorphism. 
Recall there are representations $\chi_r$, $\chi_{rs}$, $\sigma$ of $G$ such that 
\begin{align*}
w(\chi_r)&=1+y,\\
w(\chi_{rs})&=1+x+y,\\
w(\sigma)&=1+x+w.
\end{align*}
The multiplicativity of SWCs then implies that $\phi$ is surjective.

 Since $\phi$ is a group homomorphism and one can express $v\in \z^3$ as a difference $v=v_1-v_2$, where $v_1,v_2\in \z^3_{\geq 0}$,  it is enough to prove the injectivity on $\z^3_{\geq 0}$.



 %that $\phi(v)=\phi(v')$ implies $v=0$ for all $v=(a,b,c)$ with non-negative integers $a,b,c$.








We first do a change of variables in $H^*(D_m)$ (from Section \ref{ressyl}) by putting $z=x+y$ so that
$$H^*(D_{m})= \Z/2\Z[z,y,w]/(yz). $$ 
With this, $\phi$ becomes
$$  \phi(a,b,c)=(1+y)^a(1+z)^b(1+z+y+w)^c.$$

%\begin{prop}\label{W(G)-iso}
%	The map $\phi: \Z^3 \to W(G)$ defined by $\phi(a,b,c)=  (1+y)^a(1+x+y)^b(1+x+w)^c$ is a group isomorphism.
%\end{prop}

 %First we make a change of variable by putting $z=x+y$ and keeping $y$ the same. Then the cohomology ring becomes
Assuming $a,b,c$ are non-negative, the degrees of $y,z,w$ in $\phi(a,b,c)$ are $a+c,\: b+c,\:c$ respectively.	\begin{comment}Then, we use $yz=0$ to have
\begin{align*}
(1+y)^a(1+z)^b&=\left(\sum_{i=0}^a \binom{a}{i}y^i\right)\left(\sum_{j=0}^b \binom{b}{j} z^j\right)\\
&=(1+y)^a + (1+z)^b -1,
\end{align*}
and
\begin{align*}
(1+z+y+w)^c&=\sum_{\substack{0 \leq i,j,k \leq c\\i+j+k+l=c }}\binom{c}{i,j,k,l}z^{i}y^{j}w^{k}\\
&=\sum_{i=0,j,k}\binom{c}{0,j,k,l}y^jw^k+
		+\sum_{i,j=0,k}\binom{c}{i,0,k,l}z^iw^k- \sum_{i=j=0,k}\binom{c}{k}w^k\\
&=(1+y+w)^c+(1+z+w)^c-(1+w)^c.
\end{align*}

From this, we express $\phi(a,b,c)$ as a polynomial that can not be further reduced in $H^*(D_m)$:
\begin{align*}
		\phi(a,b,c)
		&=\left((1+y)^a + (1+z)^b-1\right)\left((1+y+w)^c+(1+z+w)^c-(1+w)^c\right)\\
		&= ((1+y)^a-1)(1+y+w)^c +(1+z)^b(1+y+w)^c\\
		&\quad\quad+((1+z)^b-1)(1+z+w)^c +(1+y)^a(1+z+w)^c\\
		&\quad\quad\quad+ ((1+y)^a + (1+z)^b-1)(1+w)^c\\
	&=((1+y)^a-1)(1+y+w)^c + ((1+z)^b-1)(1+w)^c + (1+y+w)^c\\
	&\quad\quad+((1+z)^b-1)(1+z+w)^c + ((1+y)^a-1)(1+w)^c+(1+z+w)^c\\
	&\quad\quad\quad+ ((1+y)^a + (1+z)^b-1)(1+w)^c\\
	&= (1+y)^a(1+y+w)^c + (1+z)^b(1+z+w)^c +(1+w)^c.
   \end{align*}

Observe that $\phi(a,b,c)$ has monomials only of the form $1,y^{i}w^{j}, z^{k}w^{l}$ with $i,j,k,l$ non-negative integeers. So, no further reduction of the degree is possible.
\end{comment} 	 
Therefore, if $\phi(a,b,c)= \phi(a',b',c')$ for non-negative triplets $(a,b,c)$, $(a',b',c')$, then by the comparison of degrees of $y,z,w$, the triplets must be the same. %More generally, since $\phi$ is a homomorphism, if it is not an isomorphism then let $v=(a,b,c) \in \ker \phi \subset \Z^3$ be non-zero. We can always choose $v_1=(a_1,b_1, c_1), v_2=(a_2,b_2,c_2) \in \Z_{\geq 0}^3$ such that $v=v_1-v_2 \in \Z^3$, contradicting, $\phi(v_1)= \phi(v_2)$ but $v_1 \neq v_2$.
\end{proof} 
As a consequence, we have:
\begin{proof}[Proof of Corollary \ref{wtriv}]
Let $m$ be a multiple of $4$.  Recall from the proof of Theorem \ref{main} that $a_\pi=n_s+n_r+m_e$, $b_\pi=n_s+n_{rs}+m_e$, $c_\pi=m_o$.  In the proof of Corollary \ref{W(G)iso} above, since $\phi$ is an isomorphism, we have $w(\pi)=1$ if and only if $a_\pi=b_\pi=c_\pi=0$. Thus, the multiplicities $m_e,m_o,n_s,n_r,n_{rs}$ in Equation \eqref{fpi} are all zero, completing the proof.
\end{proof}
\noindent A similar argument holds when $m$ is odd or $m\equiv 2\pmod 4$.

%\section{The Top Stiefel Whitney Class}\label{tswc}

%Let $\pi$ be an orthogonal representation of degree $k$. Its top SWC is $w_k(\pi)\in H^k(G)$. If $w_1(\pi)=0$, then it is also called the Euler class of $\pi$ mod $2$.

%Recall $n_0$ is the multipilcity of $1$ in $\pi$, which is also $\cfrac{1}{|G|}\sum\limits_{g\in G}\chi_{\pi}(g).$
%\begin{cor} 
%The top SWC of $\pi$ is non-zero if and only if either $e_\pi=0$ or $f_\pi=0$ where
%\begin{align*}
%e_\pi&=\frac14\left(\chi_\pi(1)+2\chi_\pi(s)+\chi_\pi(r_c)\right)-n_0\\
%f_\pi&=\frac14\left(\chi_\pi(1)+2\chi_\pi(rs)+\chi_\pi(r_c)\right)-n_0.
%\end{align*}


% $m_e=0$ and $(n_r=0 $ or $ n_{rs}=0)$. 
%\end{cor}
Moving on, we give $\pi$ with non-trivial $w_{\too}(\pi)$:
\begin{proof}[Proof of Corollary \ref{topswc}]
 First, we suppose $w_{\too}(\pi)\neq 0.$ Recall $\pi$ has the form \eqref{fpi}. Clearly, irreducible representations with trivial top SWC must not appear in this decomposition. It implies $n_0=0$ as $w_{\too}(1)=0$, and $m_e=0$, as $w_{\too}(\sigma_{k})=0$ for all even $k$. Then,
%As $w_{\too}(\sigma_{k})=0$ for all $k$ even, it reduces the top degree of $w(\pi)$ if it has positive multiplicity. Thus  $m_{2l}=0$ for $0<l<2^{n-3}$.  Now,
we have
$$w(\pi)=(1+y)^{n_r}(1+x)^{n_s}(1+x+y)^{n_{rs}}(1+x+w)^{m_o}.$$


Suppose $n_r,n_{rs}$ are both non-zero, and WLOG $n_r<n_{rs}$. The relation $(1+y)(1+x+y)=1+x$ in $H^*(G)$ simplies $w(\pi)$ to
$$w(\pi)=(1+x)^{n_r+n_s}(1+x+y)^{n_{rs}-n_r}(1+x+w)^{m_o}.$$

Here, the highest non-zero SWC is $w_h(\pi)=x^{n_r+n_s}(x+y)^{n_{rs}-n_r}w^{m_o}$ for $h=n_s+n_{rs}+2m_o$ which is not equal to $\deg \pi$. This is a contradiction. Therefore, either $n_r=0$ or $n_{rs}=0$. %By a similar argument for the case $0< n_{rs} < n_r$, we get  $n_{rs}=0$, which is also a contradiction. Hence either $n_r=0$ or $n_{rs}=0$.

On the contrary, if $m_e=n_r=0$, then
$$w(\pi)=(1+x)^{n_s}(1+x+y)^{n_{rs}}(1+x+w)^{m_o},$$
giving $w_{\too}(\pi)=x^{n_s}(x+y)^{n_{rs}}w^{m_o}\neq 0.$ Similar holds when $m_e=n_{rs}=0$.


 From Equation \eqref{rese1e2} and the character formula for the multiplicity of $1$ in a representation, we obtain
\begin{align*}
m_1(\pi)+n_r+m_e&=\sum_{g\in E_1}\chi_\pi(g),\\
m_1(\pi)+n_{rs}+m_e&=\sum_{g\in E_2}\chi_\pi(g).
\end{align*}
It follows from these equations that the conditions $m_e=n_r=0$ or $m_e=n_{rs}=0$ are equivalent to having  $e_\pi=0$ or $f_\pi=0$.

\end{proof}

%\section{Spinoriality}\label{spinoriality}
%\begin{cor}
%A representation of $G$ is spinorial if and only if $c_\pi$ is even, and
%$$
%\binom{a_\pi+1}{2}=_2 \binom{b_\pi+1}{2}=_2  \binom{c_\pi+1}{2}.
%$$
%\end{cor}
We now prove the spinoriality criterion:
\begin{proof}[Proof of Corollary \ref{spin}]

From Theorem \ref{main}, we deduce
\begin{align*}
w_1(\pi)&=(b_\pi+c_\pi)x+(a_\pi+b_\pi)y,\\
w_2(\pi)&=\left[\binom{b_\pi}{2}+\binom{c_\pi}{2}+b_\pi c_\pi\right]x^2+\left[\binom{a_\pi}{2}+\binom{b_\pi}{2}\right]y^2+c_\pi w+c_\pi(a_\pi+b_\pi)xy.
\end{align*}
Recall a representation $\pi$ is spinorial if and only if \begin{equation}\label{w2w1}w_2(\pi)=w_1(\pi)\cup w_1(\pi).\end{equation} Clearly,
$c_\pi$ must be $0\pmod 2$. This simplifies $w_2(\pi)$ to give
\begin{align*}
%w_1^2(\pi)&=(b_\pi^2+c_\pi^2) x^2+(a_\pi+b_\pi)y^2,\\
w_2(\pi)&=\left[\binom{b_\pi}{2}+\binom{c_\pi}{2}\right]x^2+\left[\binom{a_\pi}{2}+\binom{b_\pi}{2}\right]y^2.
\end{align*}
By comparing the coefficients in \eqref{w2w1}, we obtain
\begin{align*}
\frac{a_\pi^2+a_\pi}{2}&=_2\frac{b_\pi^2+b_\pi}{2}=_2\frac{c_\pi^2+c_\pi}{2},
\end{align*}
completing the proof.
\end{proof}
\section{Spinoriality for Products of Dihedral Groups}
Let $G, G'$ be finite groups. We consider the product $G\times G'$, and let $\Pi=\pi\boxtimes\pi'$ be an orthogonal representation of $G\times G'$ with $\deg \pi=d$, $\deg\pi'=d'$. From \cite[Section 7.1]{Ganguly}, such a representation is spinorial if and only if the following elements in $H^2(G\times G')$ vanish: 
\begin{enumerate}
\item\label{1} $d'w_2(\pi)+\binom{d'+1}{2}w_1(\pi)\cup w_1(\pi)$,
\item\label{2} $(dd'+1)w_1(\pi)\otimes w_1(\pi')$,
\item\label{3} $dw_2(\pi')+\binom{d+1}{2}w_1(\pi')\cup w_1(\pi')$.
\end{enumerate}
\begin{lemma}\label{cond}
The condition (\ref{1}) above is equivalent to the restriction of $\Pi$ to  $G\times 1$ being spinorial.
\end{lemma}
\begin{proof}
We have the restriction $\Pi|_{G\times 1}=d'\pi$, and its total SWC is
\begin{align*}
w(d'\pi)&=w(\pi)^{d'}\\
&=(1+w_1(\pi)+w_2(\pi)+\hdots)^{d'}\\
&=1+d'w_1(\pi)+d'w_2(\pi)+\binom{d'}{2}w_1(\pi)\cup w_1(\pi)+\hdots
\end{align*}
Now, $d'\pi$ is spinorial if and only if 
$$d'w_2(\pi)+\left(\binom{d'}{2}+d'^2\right) w_1(\pi)\cup w_1(\pi)\in H^2(G) $$
vanishes. Moreover, $$\binom{d'}{2}+d'^2=_2\binom{d'}{2}+d'=_2\binom{d'+1}{2},$$
completing the proof.

\end{proof}
Similarly, condition (\ref{3}) is same as the spinorialty of $\Pi|_{1\times G'}.$ Thus, the conditions (\ref{1}), (\ref{2}), (\ref{3}) along with Lemma \ref{cond} prove Theorem \ref{prodDm}.


%\begin{prop}
%Let $G=D_{2^n}$. $\Pi$ is spinorial iff its restriction to $G\times 1$ and $1\times G$ are spinorial and $(nn'+1)w_1(\pi) w_1(\pi')=0.$


%\end{prop}
\bibliographystyle{alpha}
\bibliography{mybib}
\vspace{10mm}








\end{document}
