\documentclass[%
 reprint,
superscriptaddress,
%groupedaddress,
%unsortedaddress,
%runinaddress,
%frontmatterverbose, 
%preprint,
%showpacs,preprintnumbers,
%nofootinbib,
%nobibnotes,
%bibnotes,
 amsmath,amssymb,
 aps,
 pre,
%pra,
%prb,
%rmp,
%prstab,
%prstper,
%floatfix,
]{revtex4-1}


\usepackage{graphicx}% Include figure files
\usepackage{dcolumn}% Align table columns on decimal point
\usepackage{bm}% bold math
\usepackage[colorlinks= true, linkcolor=blue, citecolor=blue, urlcolor=blue]{hyperref}% add hypertext capabilities
%\usepackage[mathlines]{lineno}% Enable numbering of text and display math
%\linenumbers\relax % Commence numbering lines
\usepackage{lipsum}
\usepackage{bbold}
\usepackage{mathtools}
\usepackage[normalem]{ulem}
\usepackage{float}
\usepackage{adjustbox}
\usepackage{cancel}

%\usepackage[showframe,%Uncomment any one of the following lines to test 
%%scale=0.7, marginratio={1:1, 2:3}, ignoreall,% default settings
%%text={7in,10in},centering,
%%margin=1.5in,
%%total={6.5in,8.75in}, top=1.2in, left=0.9in, includefoot,
%%height=10in,a5paper,hmargin={3cm,0.8in},
%]{geometry}

\def\hM{{\bm{M}}}
\def\hK{{\bm{K}}}
\def\hg{{\boldsymbol{\gamma}}}
\def\hG{{\boldsymbol{\Gamma}}}
\def\bS{{\boldsymbol{\Sigma}}}
\def\bG{{\bm{G}}}
\def\bg{{\bm{g}}}
\def\bea{\begin{eqnarray}}
\def\eea{\end{eqnarray}}
\def\etal{et al.}
\def\nn{\nonumber}
\def\om{\omega}
\def\f{\frac}
\def\bn{{\bm{n}}}
\def\be{{\bm{\hat{e}}}}
\def\bell{\bm \ell}
%https://www.overleaf.com/project/
\def\br{\bm r}
\def\bZ{\bm Z}
\def\p{\partial}

\usepackage{xcolor}
\newcommand{\AP}[1]{\textcolor{blue}    {\bf AP: #1}}
\newcommand{\eref}[1]{Eq.~(\ref{#1})}
\newcommand{\fref}[1]{Fig.~\ref{#1}} 
\begin{document}

%\preprint{APS/123-QED}


\title{Stochasticity in returns can expedite classical first passage under resetting}


\author{Arup Biswas}
\email{arupb@imsc.res.in}
\affiliation{The Institute of Mathematical Sciences, CIT Campus, Taramani, Chennai 600113, India}
\affiliation{Homi Bhabha National Institute, Training School Complex, Anushakti Nagar, Mumbai 400094, India}
\author{Anupam Kundu}
\email{anupam.kundu@icts.res.in}
\affiliation{International Centre for Theoretical Sciences, TIFR, Bangalore, India}

\author{Arnab Pal}
\email{arnabpal@imsc.res.in}
\affiliation{The Institute of Mathematical Sciences, CIT Campus, Taramani, Chennai 600113, India}
\affiliation{Homi Bhabha National Institute, Training School Complex, Anushakti Nagar, Mumbai 400094, India}

%\date{\today}

\begin{abstract}
Classical first passage under resetting is a paradigm in the search process. Despite its multitude applications across interdisciplinary sciences, experimental realizations of such resetting processes posit practical challenges in calibrating these zero time irreversible transitions. We consider a strategy in which resetting is performed using a finite time return control protocol in lieu of instantaneous returns. These controls could also be accompanied with random fluctuations or errors allowing target detection even during the return phase. To better understand the phenomena, we develop a unified renewal approach that can encapsulate arbitrary search processes in a fairly general topography containing targets, various resetting times and return mechanisms in arbitrary dimensions. While such finite-time protocols would apparently seem to 
%impede the overall search efficiency, 
prolong the overall search time in comparison to the instantaneous resetting process, we show \textit{on the contrary} that a significant speed-up can be gained by leveraging the stochasticity in returns. The formalism is then explored to reveal a universal criterion distilling the benefits of this strategy. We demonstrate how this general principle can be utilized to improve overall performance of a one-dimensional diffusive search process. We believe that such strategies designed with inherent randomness can be made optimal with precise controllability in complex search processes. 
\end{abstract}

\pacs{Valid PACS appear here}% PACS, the Physics and Astronomy
                             % Classification Scheme.
%\keywords{Suggested keywords}%Use showkeys class option if keyword
\begin{titlepage}
\maketitle
\end{titlepage}
%\onecolumngrid

Recently, a class of non-equilibrium systems namely resetting or restart has gained a lot of attention due to their multidisciplinary applications in statistical physics \cite{evans_diffusion_2011,kusmierz2014first,gupta2014fluctuating,eule2016non,pal2015diffusion,majumdar2015dynamical,gupta2020work,biroli2023extreme,evans_stochastic_2020,gupta2022stochastic,pogorzelec2023resetting,sokolov2023linear,ghosh2023autonomous,mendez2016characterization,mori2023entropy}, chemical \& biological physics \cite{reuveni2014role,budnar2019anillin,rotbart2015michaelis,biswas2023rate,roldan2016stochastic}, quantum physics \cite{mukherjee2018quantum,yin2023restart,magoni2022emergent}, stochastic processes \cite{kumar2023universal,de2020optimization,de2022optimal,stojkoski2021geometric,wang2022entropy,bressloff2021accumulation,huang2021random}, economics \cite{jolakoski2022fate,stojkoski2022income} and operation research \cite{bonomo2021mitigating}. There has been a spate of excitement in developing theory and application of such non-equilibrium systems, and in parallel, extracting some of the rich physics that lies embedded in such systems using optical trap experiments \cite{tal2020experimental,besga2020optimal,goerlich2023experimental} and programmable robots \cite{altshuler2023environmental,paramanick2023programming}. One of the hallmark results emanating from the seminal work of Evans and Majumdar is the ability of resetting to expedite completion of a first passage process \cite{evans_diffusion_2011}. This is noteworthy since designing optimal search processes has been a central goal of the first passage processes spanning from physics \cite{redner2001,bray2013persistence,metzler2014first}, chemistry \cite{benichou2010geometry,loverdo2008enhanced,lomholt2005optimal}, biology \cite{vergassola2007infotaxis,sheinman2012classes}, economics \cite{gabaix2016dynamics} and ecology \cite{viswanathan1999optimizing,pal_search_2020}. 

There has been a myriad of works showcasing universal dominance of resetting in diffusion alike and arbitrary stochastic processes \cite{evans_stochastic_2020,pal2022inspection,reuveni_optimal_2016,nagar2016diffusion,pal_first_2017,chechkin2018random,evans2018run,sar2023resetting,int-target-2,chen2022first,ray2021resetting,ahmad2019first,pal2019landau,belan2018restart}. Notably, a key assumption here is that resetting occurs instantaneously i.e., the system (say, a Brownian particle) can be reset (or brought back) in zero time. This is, however, a major hindrance to practical realisation or experimental verification in the field since `resetting' should be considered as a finite time process. Moreover, resetting a particle from far should require more time than to reset it from a nearby location. These presumptions seek for a more realistic viewpoint where resetting should be considered as a spatio-temporally coupled process and not just a mere teleportation. 

The first important step to this direction was taken recently by considering the fact that there is a finite time return process that brings the searcher/particle back to a preferred location/home upon resetting \cite{pal_search_2020,tal2020experimental}. It was, however, assumed that the return process is deterministic with absolute precision and moreover, it can not conduct a search (i.e., locate a target) during the return (see also \cite{pal2019invariants,pal2019time,bodrova2020resetting,bodrova2020brownian,radice2021one,stanislavsky2022subdiffusive,bressloff2020search,bressloff2020target,maso2019transport,radice2022diffusion,mercado2020intermittent}). 


Notwithstanding the advancements, deterministic drivings are never perfect and will always be accompanied by uncontrollable random fluctuations. Moreover, precise return controls with regard to the energetics maybe costly as the thermodynamic trade-off relations have taught us \cite{barato2015thermodynamic,horowitz2020thermodynamic,pal2021thermodynamic,pal2023thermodynamic}. It is therefore important to understand how the errors or fluctuations in return process can be incorporated (see e.g., \cite{gupta2021resetting,gupta2020stochastic,gupta2022work,mercado2020intermittent}). Crucially, introducing stochasticity can also render the targets accessible to the searcher during the return phase. In this work, we exactly underpin these effects on the overall search process. Quite remarkably, we find that %`accessible' during the return phase which may lower the global search times and improve overall performance. Taken together, this work aims to underpin the effect of innate stochasticity during returns on the overall search process. Remarkably, we find that 
such innate randomness or errors in return times, albeit bearing overhead time penalties in general, can \textit{often} expedite the classical first passage process under instantaneous resetting. To this end, we unravel a universal condition that rationalizes the underlying physics behind this intriguing observation -- key features are then highlighted for the paradigmatic diffusive process. 




% Figure environment removed


\textbf{\emph{General framework.---}} Consider a prototypical random search process where a searcher starts at the origin $O$, its home, of a $d$-dimensional arena at time zero. The arena may contain one or multiple targets. The searcher can locate one of these targets following a random $T$ which we denote as the first passage time. However, during the search phase, if the target is not found upto a random time $R$ (which we denote as the resetting time), the searcher decides to return to the origin. Crucially, the return process can be stochastic. Owing to this, we assume that the searcher has the ability to detect the target(s) even during the return phase. If the  searcher returns to $O$, failing to detect the target, the search process restarts afresh to the next attempt. Overall, the process is said to be completed once the target is detected irrespective of the searcher's phase (see Fig. \ref{fig1}). 


The time it takes for the searcher to either find the target or the origin during return will typically depend on the searcher’s position $\Vec{x}$ at the time of resetting $R$. Furthermore, various demographical or physical constraints may compel the searcher to follow different routes hence there can be more nontrivial dependence between the position of the searcher and the stochastic return time either to the origin (denoted by $T_{ret}^O(\Vec{x})$) or to any of the targets $A$ (denoted by $T_{ret}^A(\Vec{x})$). In the former case, the searcher resumes its search phase upon returning to the origin while in the latter case the process ends. Denoting the overall completion time by a random variable $T_R$ and considering the above-mentioned possibilities, we can write a renewal equation as follows
\begin{align}
\begin{array}{l}
T_{R}=\left\{ \begin{array}{lll}
T \hspace{3cm} &\text{if }T<R\\
R+T_{ret}^A(\Vec{x})\hspace{1.5cm} & \text{if }R \leq T~ \& ~T_{ret}^A(\Vec{x})<T_{ret}^O(\Vec{x})\\
 R+T^O_{ret}(\Vec{x})+T_R'\hspace{0.6cm} &\text{if }R \leq T ~
\&~ T_{ret}^O(\Vec{x})\leq T_{ret}^A(\Vec{x}) \end{array}\right.\text{ }\end{array}
\label{renewal}
\end{align}
where $T_R'$ is a statistically independent and identically distributed copy of $T_R$. \eref{renewal} can be written in a more concise form as
\begin{align}
    T_R&=min(T,R)+I(R\le T) min(T_{ret}^A(\Vec{x}), T_{ret}^O(\Vec{x})) \nonumber \\
    &+I(R\le T)I(T_{ret}^O(\Vec{x})\le T_{ret}^A(\Vec{x}))T_R',
        \label{S2}
\end{align}
where $min(u,v)$ is the minimum of two random variables $u$ \& $v$ and $I(u  \leq v)$ is an indicator function that takes value unity when $u \leq v$, and zero otherwise. Thus, $ \langle I(u  \leq v) \rangle=Pr(u \leq v)$ i.e, the probability that $u \leq v$.  Crucially, the indicator function $I(T_{ret}^O(\Vec{x})\le T_{ret}^A(\Vec{x}))$ also depends on the coordinate $\Vec{x}$ of the searcher at the time of resetting. This implies that the expectations on \eref{S2} need to be performed over the underlying stochastic process, the resetting time density $f_R(t)$ and the return protocols. For instance, the expectation $\mathcal{E} \equiv \langle I(R\le T)I(T_{ret}^O(\Vec{x})\le T_{ret}^A(\Vec{x})) T_R'\rangle$ can be computed as follows (see Sec S1A in \cite{SI})

{\footnotesize
\begin{align}
    \mathcal{E}&= \int_0^\infty dt f_R(t)  \frac{Pr(T\ge t) \int_{\mathcal{D}}d\Vec{x}G(\Vec{x},t)\langle I(T_{ret}^O(\Vec{x})\le T_{ret}^A(\Vec{x})) \rangle}{Q(t)} 
 \langle T_R' \rangle \nonumber \\
 &= \langle T_R \rangle \int_0^\infty dt f_R(t) \int_{\mathcal{D}}d\Vec{x}G(\Vec{x},t) Pr(T_{ret}^O(\Vec{x})\le T_{ret}^A(\Vec{x})),
 \label{expt-1}
\end{align}}where $\mathcal{D}$ is the domain of search in arbitrary dimension that can contain one or multiple targets and $G(\Vec{x},t)$ is the underlying time-dependent propagator in the presence of targets so that $Q(t)=Pr(T\geq t) =\int_{\mathcal{D}}d\Vec{x}~G(\Vec{x},t)$ becomes the survival probability \cite{redner2001}. In the first line of Eq. (\ref{expt-1}), we have taken averages over $f_R(t)$ and the underlying search propagator. Computing similar expectations from Eq. (\ref{S2}), we obtain the mean first passage time (MFPT) to be \cite{SI}
\begin{align}
\langle T_R \rangle =\frac{\langle min(T,R) \rangle +\langle min\left(T_{ret}^A,T_{ret}^O\right) \rangle}{1-\int_{\mathcal{D}}d\Vec{x}~\widetilde{G}_R(\Vec{x}) Pr(T_{ret}^O(\Vec{x})\le T_{ret}^A(\Vec{x}))} ,
     % \nonumber \\
     % &=\frac{\langle min(T,R) \rangle + \langle min\left(T_{ret}^O,T_{ret}^A\right)\rangle}{Pr(R>T)+ Pr(T_{ret}^A\le T_{ret}^O)}
     \label{mfpt}
\end{align}
where $\widetilde{G}_R({\Vec{x}}) \equiv \int_0^\infty G(\Vec{x},t)f_R(t)dt $ is the time-integrated propagator and $\langle min\left(T_{ret}^A,T_{ret}^O\right) \rangle=\int_{\mathcal{D}}d\Vec{x}\widetilde{G}_R(\Vec{x})\langle min\left(T_{ret}^A(\Vec{x}),T_{ret}^O(\Vec{x})\right)\rangle$. On the other hand,  $Pr(T_{ret}^O(\Vec{x})\le T_{ret}^A(\Vec{x}))$ is the splitting probability that the searcher, starting from $\Vec{x}$, reaches the origin before hitting any of the targets during the return phase. 

\eref{mfpt} is central to this study and is remarkably general since it holds for a) any kind of underlying first passage process  (beyond diffusion) conducted in any dimension in the presence of arbitrary target(s) regardless of their variation in size, shape or nature (purely absorbing or partial), b) arbitrary resetting time distributions, and c) generic returning motion such as instantaneous, deterministic or stochastic and their various modes of return. 






\textbf{\emph{Diffusive search with resetting and stochastic return via controlled potential trap.---}} To illustrate how the framework developed above can be
utilized in practice, we examine the paradigm of a 1d diffusive search process (designated by the diffusion constant $D$) in which
a particle starts at the origin $O$ and continues to diffuse until it hits a
stationary target at a location $L$. In addition, assume that the process is reset at a constant rate $r$ (i.e., resetting time density $f_R(t)=re^{-rt}$) upon which a potential $U(x)$ centered at the origin is turned on. The particle diffuses through the potential and it is switched off when the particle makes a first return to the origin. Subsequently, the particle resumes its diffusive search phase (see \cite{gupta2021resetting,gupta2020stochastic} where non-equilibrium steady state and the relaxation properties were studied under this protocol). However, since the return is not purely deterministic, there is always a chance to find the target during the return phase (Fig. \ref{fig1}). In below, we demonstrate how this could expedite the overall search time.



We start by recalling the diffusive propagator of this
process given by $ G(x,t)=\frac{1}{\sqrt{4\pi Dt}}\left(e^{-\frac{x^2}{4Dt}} -e^{-\frac{(2L-x)^2}{4Dt}}\right)$ \cite{redner2001}. For diffusive search process, the underlying first passage times $(T)$ are sampled from the L\'evy Smirnov distribution so that $\langle min(T,R) \rangle=\frac{1}{r} \left(1- e^{-\sqrt{rL^2/D}} \right)$ \cite{pal_first_2017}. During the return phase, the particle has the possibility to hit either $O$ or $L$ starting from position $x$ which is the 
coordinate of the particle exactly at the time of resetting. 
Considering that the return phase is modulated by a linear potential $U(x)=\lambda |x|$, we can compute the average time 
for the particle
to reach either of the boundaries namely $\langle t_{2}(x) \rangle = \frac{L(1-e^{\lambda x/D}) +x(e^{\lambda L/D }-1)}{\lambda(e^{\lambda L/D}-1)} $ for $x>0$ and $\langle t_{1}(x) \rangle =|x|/\lambda$ for $x<0$ \cite{redner2001}. Using this one can write the following expectation \cite{SI}
\begin{eqnarray}
\langle min\left(T_{ret}^L(x),T_{ret}^O(x)\right)\rangle=   \theta(-x)\langle t_1(x) \rangle + \theta(x)\langle t_2(x) \rangle,
\end{eqnarray}
where $\theta(x)$ is the step-function. In addition, the splitting probability is given by given by $Pr(T_{ret}^O(x)\le T_{ret}^L(x))=\left[ \theta(x)\frac{e^{\lambda x/D}-e^{\lambda L/D}}{1-e^{\lambda L/D}}+\theta(-x) \right]$ \cite{redner2001}. Substituting these expressions into \eref{mfpt}, we arrive at the following expression for the MFPT  $\langle T_R \rangle=D/L^2 \langle \tau(z,\text{Pe}) \rangle$, where
\begin{small}
\begin{eqnarray}
     &&\langle \tau (z, \text{Pe}) \rangle = \frac{1}{z \text{Pe}^2}\bigg(2 \text{Pe}^2 +2 e^{\text{Pe}} z -2 e^{\text{Pe}} \text{Pe}^2-z \text{Pe} -2 z \nonumber \\
     &&\hspace{0.4cm} +\frac{2 \left(e^{\text{Pe}}-1\right) \text{Pe} \left(\left(2 e^{\text{Pe}}-1\right) e^{\sqrt{z}}-1\right) \left(\text{Pe}^2-z\right)}{2 e^{\text{Pe}+\sqrt{z}}\text{Pe} -e^{2 \sqrt{z}} \left(\text{Pe}+\sqrt{z}\right)-\text{Pe}+\sqrt{z}}\Bigg),
     \label{mfpt1}    
 \end{eqnarray}
 \end{small}% \begin{align}
% \label{mfpt1} 
%     \langle \tau (z, \text{Pe}) \rangle=\frac{2\alpha\beta\gamma}{z\text{Pe}^2}\left(1 -\frac{z\text{Pe}}{2\alpha\beta\gamma}+\frac{ \text{Pe}(\gamma e^{\sqrt{z}}-1)}{\alpha e^{2\sqrt{z}} -2e^{\alpha}\text{Pe}-\beta}\right)
% \end{align}
and we have introduced a scaled resetting rate  $z=\frac{rL^2}{D}$ and the Peclet number $\text{Pe}=\frac{\lambda L}{D}$ which is a ratio between drift and diffusive time scales  \cite{redner2001}. \fref{fig2} shows corroborated plots (theory \& simulations) of $\langle \tau (z, \text{Pe}) \rangle$ as a function of $\text{Pe}$. \eref{mfpt1} reproduces $ \langle \tau_{inst}(z) \rangle=\langle \tau (z,\text{Pe} \to \infty) \rangle =  \frac{e^{\sqrt{z}}-1}{z}$ which is the MFPT for instantaneous resetting \cite{evans_diffusion_2011} -- shown in Fig. \ref{fig2} by the horizontal dashed lines for different resetting rates. 


Intriguingly, Fig. \ref{fig2} shows a key feature that
%exhibits an intriguing feature that
the MFPT for the stochastic return can be reduced further than the instantaneous resetting (see e.g., the $z=10$ curve). In fact, \eref{mfpt1} reveals the existence of a critical resetting rate $z^*$, which is a function of Pe, beyond which stochastic return always supersedes the instantaneous return in optimizing the search time (see later for the exact evaluation of $z^*$). However, for $z<z^*$ (see e.g., the $z=5$ curve), the MFPT always stays above the instantaneous (dashed) line indicating that no advantage can be gained from the stochastic return for any resetting rate or Pe. 



% Figure environment removed

Several comments can be made on the shape of the MFPT curve in Fig. \ref{fig2} (e.g., $z=10$). In the limit of large $\text{Pe}$ (i.e., for the strongly attractive trap), the particle returns to the origin almost deterministically with constant speed $\lambda$ and has negligible probability to find the target during the return phase. Therefore, one can approximate the right envelope with $ \langle \tau_{det}(z,\text{Pe}) \rangle=\frac{e^{\sqrt{z}}-1}{z}+\frac{1}{\text{Pe}} \left(\frac{2 \sinh \left(\sqrt{z}\right)}{\sqrt{z}}-1 \right)$ \cite{pal_search_2020} -- plotted with dotted line in Fig \ref{fig2}. On the other hand, for smaller $\text{Pe}$ values, the potential is rather shallow and it takes time for the particle either to return to the origin or to the target. However, if $z$ is large, the resetting events are more frequent and thus the contribution to the global MFPT comes predominantly from the post-resetting phase which can be computed from Eq. \eqref{mfpt1} as $ \langle \tau(z\to\infty,\text{Pe})\rangle=\frac{2 e^{\text{Pe}}-\text{Pe}-2}{\text{Pe}^2}$ -- plotted with dot-dashed line in Fig \ref{fig2} showing an excellent agreement with the left envelope of $\langle \tau(z=10,\text{Pe}) \rangle$ (also see \cite{SI}). 

%The bulk features naturally can not be captured by these extreme limits. 

%In the bulk, however, there is an intricate interplay between finite $z$ and $\text{Pe}$ that can not be captured by these extreme limits. 



\textbf{\emph{A universal criterion for the trade-off between instantaneous and stochastic return---} }
It is evident from the above discussion that stochastic returns can over- or under-perform search process compared to the instantaneous returns. While this observation is made for simple diffusive search, naturally we are intrigued by the question whether this is generically true for any search process. If so, what is the governing criterion for the stochastic returns to be beneficial?

To delve deeper into this question, we first take the limit of instantaneous return \cite{pal_first_2017} where the searcher always returns to the origin in zero time so that $Pr(T_{ret}^O\le T_{ret}^A)=1$ resulting in $\langle T_R^{inst} \rangle=\frac{\langle min(T,R) \rangle}{Pr(T<R)}$ from \eref{mfpt} \cite{SI}. Evidently, stochastic return will be beneficial only if 
$    \langle T_R \rangle < \langle T_R^{inst} \rangle$ which, in turn, implies \cite{SI}
\begin{align}
\mathcal{T} \equiv 
\frac{
\langle min\left(T_{ret}^A,T_{ret}^O\right) \rangle}{Pr(T_{ret}^A<T_{ret}^O)} < \langle T_R^{inst} \rangle ,
  \label{srcond}
\end{align}
where $Pr(T_{ret}^A<T_{ret}^O)=\int_{\mathcal{D}}d\Vec{x}~\widetilde{G}_R(\Vec{x}) Pr(T_{ret}^A(\Vec{x})<T_{ret}^O(\Vec{x})) $. 
Clearly, to understand this trade-off, one needs to consider many realizations of such a process and compare between the time $\mathcal{T}$ it takes, on an average, for the process to either return to the origin or to complete the search, starting from the location at the time of resetting given by $\langle min\left(T_{ret}^A,T_{ret}^O\right) \rangle$ (rescaled with the splitting probability $Pr(T_{ret}^A<T_{ret}^O)$) and the time $ \langle T_R^{inst} \rangle$ that it would have taken for the instantaneous return. This is rather intriguing since the former accumulates time only during the post resetting return phase while latter does so only during the pre-resetting phase. Thus, the relation (\ref{srcond}) puts a strong constraint on the average return time $\mathcal{T}$ regardless of the final destination. Notably, this relation is quite universal since it does not depend on the particular choice of the underlying first passage process, resetting time density 
or the return protocol. 



 % Figure environment removed


Furthermore, this inequality allows us to construct a universal phase diagram, spanned by the system parameters, that govern the dominance of stochastic return over the classical instantaneous return. Such a phase diagram is graphically illustrated in \fref{fig5} for the 1d diffusion. The red dashed line, obtained by setting $\mathcal{T}=\langle T_R^{inst} \rangle$, separates the two phases namely stochastic return -`beneficial' and -`detrimental' than the instantaneous return. The separatrix is the locus of the set of critical resetting rate $z^*$ for each Pe obtained from the above equality (see \cite{SI} for further elaborations). 


 
To gain further insights into the trade-off between two phases, we study $\mathcal{T}$ and discuss some of the limiting cases. For \textit{very low resetting rate} ($z\to0$) and finite Pe, one has  $\mathcal{T} \sim \frac{1}{z} \left( \frac{2 \text{Pe} \left(1-e^{\text{Pe}}\right)  }{\text{Pe} (\text{Pe}+2)-2 e^{\text{Pe}}+2  } \right) $ whereas $\langle\tau_{inst}(z) \rangle \propto 1/\sqrt{z}$. Clearly, the LHS diverges faster than RHS nullifying the criterion (\ref{srcond}) for any $\text{Pe}$. 
Indeed, the trajectories that go away from the target accumulate more time during the stochastic return than the instantaneous return making the former strategy detrimental.
 In contrast for $z\gg\mathcal{O}(1)$ - \textit{frequent resets} - and finite $\text{Pe}$, one has $\mathcal{T}=\frac{2 e^{\text{Pe}}-\text{Pe}-2}{\text{Pe}^2}$ making the LHS finite however the RHS diverges as $\propto e^{\sqrt{z}}/z$. In this case, the pre-resetting phase is very short and the particle is effectively in the return phase. Here, it has the possibility to find the target during return phase while the instantaneous return almost always keeps the particle close to the origin eliminating the target-detection. Clearly, this is a favourable situation for the stochastic returns.
 
 
 %When the potential is too steep ($Pe\to\infty)$ i.e. the instantaneous return case, the particle can not probe the target as such. But when $Pe$ is finite, it has a finite chance of reaching the target thus MFPT is lowered a lot compared to instantaneous return and stochastic return renders a win-win situation. 

For the intermediate case of \textit{finite $z$}, there are two limiting cases of $\text{Pe}$. For large $\text{Pe}$, one finds $\mathcal{T} \sim \text{Pe}\left(\frac{2  \sinh \left(\sqrt{z}\right)}{z^{3/2}}-\frac{1}{z}\right)$ to be divergent while the RHS is finite. Note that in this limit the return probability to the origin (having return time $\approx \sqrt{D/r}/\lambda$) is almost close to unity and the return phase only adds time penalties. Naturally, instantaneous returns are more efficient. For small $\text{Pe}$, however, the average return time is large (specifically from the trajectories in the $x<0$ region) as can be seen from $\mathcal{T}\sim\frac{1}{\text{Pe}}\frac{1}{1-\sqrt{z} \text{csch}\left(\sqrt{z}\right)}\propto\frac{1}{\text{Pe}}$ which diverges invalidating the condition (\ref{srcond}). This ensures that instantaneous returns are more beneficial (see Sec S10 and Table S1 in \cite{SI} for a detailed summary). 


The phase-diagram constructed from Eq. (\ref{srcond}) thus allows us to elucidate the effect of stochastic returns in the parameter space both quantitatively and physically. 




\textbf{\emph{Conclusions.---}}
In this letter, we have developed a unified first passage time framework of a stochastic search process under finite time resetting or returns. Notable distinction between this and the existing body of works \cite{evans_diffusion_2011,bodrova2020resetting,pal_search_2020,gupta2021resetting} lies on the fact that the home-returns can be accompanied with stochasticity and thus the searchers can be fortuitous to find targets during the return. Naively, one expects that finite time returns can only incur delay to the overall completion time. However, we argue that the element of randomness in return process
can expedite the overall completion even in comparison to the classical instantaneous resetting process which takes no time to return. This is the most intriguing observation of this work to which we attribute various physical scenarios. Elucidating further the scope of such observation to arbitrary stochastic process with generic returns, we derive a universal and physically amenable criterion that unveils the superiority of the finite time stochastic returns above the zero-time returns. We 
emphasize that the universal framework of this problem is also a powerful tool to predict the fluctuations and possibly the full distribution. 

We believe that resetting with stochastic returns can turn out to be a universal optimization strategy owing to its dominance over classical first passage resetting with applications to biochemical search \cite{benichou2011intermittent,iyer2016first} and molecular transport \cite{jain2023fick,metzler2014first}. Whether stochastic return can be beneficial than instantaneous return for an optimally restarted process is a challenging question that will be discussed elsewhere.
Importantly, our approach is potentially useful for conducting experiments from the practical implementation of resetting especially since one need not drag or track the agents (e.g., colloids or programmable robots) all the way during the return process. Finally, it is intuitive that stochastic return could be energetically optimal while considering the search completion than the deterministic return protocols. These frontiers remain to be explored further in future.






 




 
 
\emph{\textbf{Acknowledgements.}---}  The numerical calculations reported in
this work were carried out on the Nandadevi cluster, which is maintained and supported by the Institute of Mathematical Science’s High-Performance Computing Center. AK acknowledges the support of the core research grant CRG/2021/002455
and the MATRICS grant MTR/2021/000350 from the SERB, DST, Government of India. AK also acknowledges support of the Department of Atomic Energy, Government of India, under Project No. 19P1112RD. AP gratefully acknowledges research support from the Department of Science and Technology, India, SERB Start-up Research Grant Number SRG/2022/000080 and Department of Atomic Energy, Government of India.


\bibliography{fpusr}
%\input{fpusr.bib}


\begin{titlepage}
\title{Supplemental Material for \\``Stochasticity in returns can expedite classical first passage under resetting''}
\maketitle
\end{titlepage}

\onecolumngrid
\setcounter{page}{1}
\renewcommand{\thepage}{S\arabic{page}}
\setcounter{equation}{0}
\renewcommand{\theequation}{S\arabic{equation}}
\setcounter{figure}{0}
\renewcommand{\thefigure}{S\arabic{figure}}
\setcounter{section}{0}
\renewcommand{\thesection}{S\arabic{section}}
\setcounter{table}{0}
\renewcommand{\thetable}{S\arabic{table}}

This Supplemental Material provides detailed mathematical derivations and additional discussions which support the results described in the Letter. Moreover, it also provides details of the examples used in the main text to demonstrate the validity and applicability of the formalism.

\tableofcontents

\section{Derivation of Eq. (4) in the main text}\label{sc1}
In this section, we will provide a detailed derivation of Eq. (4) -- the mean first passage time (MFPT). We start from the stochastic renewal Eq. (1) 
\begin{align}
\begin{array}{l}
T_{R}=\left\{ \begin{array}{lll}
T \hspace{3cm} &\text{if }T<R\\
R+T_{ret}^A(\Vec{x})\hspace{1.5cm} & \text{if }R \leq T~ \& ~T_{ret}^A(\Vec{x})<T_{ret}^O(\Vec{x})\\
 R+T^O_{ret}(\Vec{x})+T_R'\hspace{0.6cm} &\text{if }R \leq T ~
\&~ T_{ret}^O(\Vec{x})\leq T_{ret}^A(\Vec{x}) \end{array},\right.\text{ }\end{array}
\end{align}
which can be written in a concise form as
\begin{align}
    T_R&=min(T,R)+I(R\le T) min(T_{ret}^A(\Vec{x}), T_{ret}^O(\Vec{x}))
    +I(R\le T)I(T_{ret}^O(\Vec{x})\le T_{ret}^A(\Vec{x}))T_R',
        \label{renewal-2}
\end{align}
where $min(u,v)$ is the minimum of two random variables $u$ \& $v$ and $I(u  \leq v)$ is an indicator function that takes value unity when $u \leq v$, and zero otherwise. Thus, $ \langle I(u  \leq v) \rangle=Pr(u \leq v)$ i.e, the probability that $u \leq v$. Using the definition of the random variable $min(u,v)$, we can write 
\begin{align}
    min(T_{ret}^A(\Vec{x}), T_{ret}^O(\Vec{x}))=I(T_{ret}^A(\Vec{x})< T_{ret}^O(\Vec{x}))T_{ret}^A(\Vec{x})+I(T_{ret}^O(\Vec{x})\le T_{ret}^A(\Vec{x}))T_{ret}^O(\Vec{x}).
    \label{min}
\end{align} 
We now take expectations on the both sides of \eref{renewal-2} to have
\begin{align}
    \langle T_R \rangle&= \langle min(T,R) \rangle + \langle I(R\le T) min(T_{ret}^A(\Vec{x}), T_{ret}^O(\Vec{x})) \rangle + \langle I(R\le T)I(T_{ret}^O(\Vec{x})\le T_{ret}^A(\Vec{x}))T_R' \rangle.
    \label{s1}
\end{align}
Noting that $T_R'$ is an independent and identically distributed copy of $T_R$, we can simplify the last expression from \eref{renewal-2} as $ \langle I(R\le T)I(T_{ret}^O(\Vec{x})\le T_{ret}^A(\Vec{x})) \rangle \langle T_R' \rangle=\langle I(R\le T)I(T_{ret}^O(\Vec{x})\le T_{ret}^A(\Vec{x})) \rangle \langle T_R \rangle $.  Rearranging the terms in \eref{renewal-2} then gives us
\begin{align}
    \langle T_R \rangle &= \underbrace{\frac{\langle min(T,R) \rangle}{1-\langle I(R\le T)I(T_{ret}^O(\Vec{x})\le T_{ret}^A(\Vec{x})) \rangle}}_{\text{search/exploration phase}} +\underbrace{\frac{\langle I(R\le T)I(T_{ret}^O(\Vec{x})\le T_{ret}^A(\Vec{x}))T_{ret}^O(\Vec{x}) \rangle}{1-\langle I(R\le T)I(T_{ret}^O(\Vec{x})\le T_{ret}^A(\Vec{x})) \rangle}}_{\text{return phase: return to origin}} \nonumber \\
    &\hspace{7cm}+ \underbrace{\frac{\langle I(R\le T)I(T_{ret}^A(\Vec{x})< T_{ret}^O(\Vec{x}))T_{ret}^A(\Vec{x}) \rangle}{1-\langle I(R\le T)I(T_{ret}^O(\Vec{x})\le T_{ret}^A(\Vec{x})) \rangle}}_{\text{return phase: finding the target}}
    \label{mfpt3}.
\end{align}
The first term in \eref{mfpt3} accounts for the time taken due to the exploration of the searcher.  The second term contributes to the time that the searcher takes to return to the starting position. Finally, the third term accounts for the time spent while finding the target during the return phase. It should be noted that the expectations in \eref{mfpt3} are taken over three different random components: the underlying stochastic process, the resetting time, and the stochastic return phase trajectories. In what follows, we show how to evaluate these expectations.

\subsection{The denominator}
First we focus on the term in the denominator that is common to all the components. Conditioned on the resetting time $R$ over the density $f_R(t)$, we can write
\begin{align}
   \langle I(R\le T)I(T_{ret}^O(\Vec{x})\le T_{ret}^A(\Vec{x})) \rangle  &=\int_0^\infty dt f_R(t) Pr(T\ge t) \langle I(T_{ret}^O(\Vec{x}(t))\le T_{ret}^A(\Vec{x}(t))) \rangle_{\Vec{x},return},
   %_{\Vec{x},return}
   \label{s3}
\end{align}
where further averaging needs to be done over the underlying process $\Vec{x}$ and then the return process (denoted by the subscripts). The former is done by noting that at the time of resetting, the searcher could be anywhere in the domain $\mathcal{D}$ and thus its position is sampled over $G(\Vec{x},t)$ -- the propagator in the presence of targets. Evidently, the search upto resetting time is possible only if the searcher was not absorbed by then -- this occurs essentially with the survival probability $Q(t)$
given by 
\begin{align}
    Q(t)=Pr(T\ge t)=\int_{\mathcal{D}}d\Vec{x}~G(\Vec{x},t).
    \label{surv}
\end{align}
Thus the averaging should be over the normalized PDF $G(\Vec{x},t)/Q(t)$ so that the expectation inside the integral in \eref{s3} reads
% While a resetting clock hits at $t$, the searcher's position $\Vec{x}$ can be anywhere in the search domain $\mathcal{D}$ and we need to average it out through the propagator $G(\Vec{x},t)$. Note that, as $G(\Vec{x},t)$ is the free range propagator in presence of a target, thus un-normalized by definition. To normalize the propagator we should divide it by the normalization factor $\int_{\mathcal{D}}d\Vec{x}G(\Vec{x},t)$ or the survival probability $Q(t)$. To put it in another way, the search is continued only when the searcher has not found the target up to the resetting time $t$, which is actually the survival probability given as $Q(t)=Pr(T\ge t)=\int_{\mathcal{D}}d\Vec{x}G(\Vec{x},t)$. Now we can take the second expectation of the quantity under the integral over $\Vec{x}$ in Eq. \ref{s3} to write
\begin{align}
     &\int_0^\infty dt f_R(t) Pr(T\ge t) \langle I(T_{ret}^O(\Vec{x}(t))\le T_{ret}^A(\Vec{x}(t))) \rangle_{\Vec{x},return} \nonumber\\
     =&\int_0^\infty dt f_R(t) \cancel{Pr(T\ge t) }\frac{\int_{\mathcal{D}}d\Vec{x}G(\Vec{x},t)\langle I(T_{ret}^O(\Vec{x})\le T_{ret}^A(\Vec{x})) \rangle_{return}}{\cancel{\int_{\mathcal{D}}d\Vec{x}G(\Vec{x},t)}} \nonumber \\
     =& \int_0^\infty dt f_R(t) \int_{\mathcal{D}}d\Vec{x}G(\Vec{x},t)\langle I(T_{ret}^O(\Vec{x})\le T_{ret}^A(\Vec{x})) \rangle.
     \label{s4}
\end{align}
Here we have omitted the subscript denoting average over return motion as this is obvious. We shall not use the subscripts in the following derivation as these same notations follow there. We are now left with the last expectation on the return process which is straightforward such that
 \begin{align}
 \langle I(R\le T)I(T_{ret}^O(\Vec{x})\le T_{ret}^A(\Vec{x})) \rangle & = \int_0^\infty dt f_R(t) \int_{\mathcal{D}}d\Vec{x}~G(\Vec{x},t)\langle I(T_{ret}^O(\Vec{x})\le T_{ret}^A(\Vec{x})) \rangle \nonumber \\
& = \int_0^\infty dt f_R(t) \int_{\mathcal{D}}d\Vec{x}~G(\Vec{x},t)Pr(T_{ret}^O(\Vec{x})\le T_{ret}^A(\Vec{x})) .
\label{s5}
\end{align}
The above-mentioned steps were used to derive the expectation $\mathcal{E} \equiv \langle I(R\le T)I(T_{ret}^O(\Vec{x})\le T_{ret}^A(\Vec{x})) T_R'\rangle$  [Eq. (3)]
in the main text. Furthermore, introducing the time-integrated propagator 
\begin{align}
  \widetilde{G}_R({\Vec{x}})= \int_0^\infty G(\Vec{x},t)f_R(t)dt   ,
 \end{align}
 we can rewrite \eref{s5} as
\begin{align}
     \langle I(R\le T)I(T_{ret}^O(\Vec{x})\le T_{ret}^A(\Vec{x})) \rangle  = \int_{\mathcal{D}}d\Vec{x}~\widetilde{G}_R(\Vec{x}) Pr(T_{ret}^O(\Vec{x})\le T_{ret}^A(\Vec{x})).
\end{align}
%  using this in \eref{s4} we finally have 
%  \begin{align}
% \langle I(R\le T)I(T_{ret}^O\le T_{ret}^A) \rangle& = \int_{\mathcal{D}}d\Vec{x}\widetilde{G}_R(\Vec{x})\langle I(T_{ret}^O(\Vec{x})\le T_{ret}^A(\Vec{x})) \rangle_{return} \nonumber \\
% &= \int_{\mathcal{D}}d\Vec{x}\widetilde{G}_R(\Vec{x}) Pr(T_{ret}^O(\Vec{x})\le T_{ret}^A(\Vec{x}))
% \label{s5}
% \end{align}
In \eref{s5}, the term $Pr(T_{ret}^O(\Vec{x})\le T_{ret}^A(\Vec{x}))$ can be interpreted as the splitting probability of the searcher to $O$. Simply put, this is the probability that the particle has first reached $O$ without touching any of the other boundaries.
%$\{ A_i \}$ \textcolor{magenta}{We have not used multiples boundaries as $A_1,A_2...$ rather treating $A$ as THE boundary. Thus using $A_i$ here is not of much help and also confusing to me}. 

\subsection{The second term} 
For the numerator of the second term in \eref{mfpt3} we proceed with similar arguments as in the previous subsection. This boils down to
\begin{align}
    &\langle I(R\le T)I(T_{ret}^O(\Vec{x})\le T_{ret}^A(\Vec{x}))T_{ret}^O(\Vec{x}) \rangle \\
    =& \int_0^{\infty}dt f_R(t) Pr(T\ge t) \langle I(T_{ret}^O(\Vec{x}(t))\le T_{ret}^A(\Vec{x}))T_{ret}^O(\Vec{x}(t)) \rangle\nonumber\\
    =& \int_0^\infty dt f_R(t) \cancel{Pr(T\ge t) }\frac{\int_{\mathcal{D}}d\Vec{x}G(\Vec{x},t)\langle I(T_{ret}^O(\Vec{x})\le T_{ret}^A(\Vec{x})) T_{ret}^O(\Vec{x})\rangle}{\cancel{\int_{\mathcal{D}}d\Vec{x}G(\Vec{x},t)}}\nonumber\\
    =&\int_{\mathcal{D}}d\Vec{x}\widetilde{G}_R(\Vec{x})\langle I(T_{ret}^O(\Vec{x})\le T_{ret}^A(\Vec{x}))T_{ret}^O(\Vec{x})\rangle \nonumber \\
    =&\int_{\mathcal{D}}d\Vec{x}\widetilde{G}_R(\Vec{x}) Pr(T_{ret}^O(\Vec{x})\le T_{ret}^A(\Vec{x}))\langle T_{ret}^O(\Vec{x})|T_{ret}^O(\Vec{x})\le T_{ret}^A(\Vec{x})\rangle,
    \label{s6}
\end{align}
where $\langle T_{ret}^O(\Vec{x})|T_{ret}^O(\Vec{x})\le T_{ret}^A(\Vec{x})\rangle$ is the conditional mean first passage time of the searcher to reach the origin $O$ before finding any of the targets during the return phase. This will mark the completion of the return phase following which the search/exploration phase will resume.

\subsection{The third term}
Following the same procedure as before, we have 
\begin{align}
& \langle I(R\le T)I(T_{ret}^A(\Vec{x})< T_{ret}^O(\Vec{x}))T_{ret}^A(\Vec{x}) \rangle \nonumber\\
=&\int_0^{\infty} dt f_R(t) Pr(T\ge t) \langle I(T^A_{ret}(\Vec{x}(t))<T^O_{ret}(\Vec{x}(t))) T^A_{ret}(\Vec{x}(t)) \rangle \nonumber\\
 =& \int_0^\infty dt f_R(t) \cancel{Pr(T\ge t) }\frac{\int_{\mathcal{D}}d\Vec{x}G(\Vec{x},t)\langle I(T_{ret}^A(\Vec{x})< T_{ret}^O(\Vec{x})) T_{ret}^A(\Vec{x})\rangle}{\cancel{\int_{\mathcal{D}}d\Vec{x}G(\Vec{x},t)}} \nonumber\\
 =&\int_{\mathcal{D}}d\Vec{x}\widetilde{G}_R(\Vec{x})\langle I(T_{ret}^A(\Vec{x})< T_{ret}^O(\Vec{x}))T_{ret}^A(\Vec{x})\rangle \nonumber \\
 =&\int_{\mathcal{D}}d\Vec{x}\widetilde{G}_R(\Vec{x}) Pr(T_{ret}^A(\Vec{x})< T_{ret}^O(\Vec{x}))\langle T_{ret}^A(\Vec{x})|T_{ret}^A(\Vec{x})< T_{ret}^O(\Vec{x})\rangle,
 \label{s7}
 \end{align}
where $\langle T_{ret}^A(\Vec{x})|T_{ret}^A(\Vec{x})< T_{ret}^O(\Vec{x})\rangle$ is the conditional mean first passage time of the searcher to find any of the targets before reaching the origin during the return phase. This will mark the completion of this search process.



 \subsection{Back to \eref{mfpt3}}
We first note that the sum of two terms in
  \eref{s6} and \eref{s7} give us
 \begin{align}
     &\langle I(R\le T)I(T_{ret}^O(\Vec{x})\le T_{ret}^A(\Vec{x}))T_{ret}^O(\Vec{x}) + I(R\le T)I(T_{ret}^A(\Vec{x})< T_{ret}^O(\Vec{x}))T_{ret}^A (\Vec{x})\rangle \nonumber \\
     =&\int_{\mathcal{D}}d\Vec{x}\widetilde{G}_R(\Vec{x})\Big[ Pr(T_{ret}^O(\Vec{x})\le T_{ret}^A(\Vec{x}))\langle T_{ret}^O(\Vec{x})|T_{ret}^O(\Vec{x})\le T_{ret}^A(\Vec{x})\rangle& \nonumber \\
         & \hspace{5cm}+ Pr(T_{ret}^A(\Vec{x})< T_{ret}^O(\Vec{x}))\langle T_{ret}^A(\Vec{x})|T_{ret}^A(\Vec{x})< T_{ret}^O(\Vec{x})\rangle\Big] \nonumber \\
         =&\int_{\mathcal{D}}d\Vec{x}\widetilde{G}_R(\Vec{x}) \langle min\left(T_{ret}^A(\Vec{x}),T_{ret}^O(\Vec{x})\right)\rangle,
         \label{s8}
 \end{align}
 where we have used the definition from \eref{min}. For brevity, we have also omitted all the subscripts from the averages. 
 
 Finally, combining together \eref{s5} and \eref{s8} into \eref{mfpt3}, we arrive at the Eq. (4) of the main text
  \begin{align}
\langle T_R \rangle =\frac{\langle min(T,R) \rangle +\int_{\mathcal{D}}d\Vec{x}\widetilde{G}_R(\Vec{x})\langle min\left(T_{ret}^A(\Vec{x}),T_{ret}^O(\Vec{x})\right)\rangle}{1-\int_{\mathcal{D}}d\Vec{x}~\widetilde{G}_R(\Vec{x}) Pr(T_{ret}^O(\Vec{x})\le T_{ret}^A(\Vec{x}))} .
     % \nonumber \\
     % &=\frac{\langle min(T,R) \rangle + \langle min\left(T_{ret}^O,T_{ret}^A\right)\rangle}{Pr(R>T)+ Pr(T_{ret}^A\le T_{ret}^O)}
     \label{s10}
\end{align}


\section{Derivation of $\langle min(T,R) \rangle$ for exponential resetting times}\label{S0}\label{S3}
% We define a random variable $Z$ as
% \begin{align}
%     Z=min(T,R).
% \end{align}
% The probability that $Z$ is less than some certain value $t$ is the same as saying either one of $T$ or $R$ is less than $t$, to put mathematically,
% \begin{align}
%     &Pr(Z < t)=1-Pr(Z\ge t) \nonumber\\
%     \implies &Pr(min(T,R)< t)=1-Pr(T\ge t)Pr(R\ge t)
% \end{align}
%  where probability that both of $T,R\ge t$ is thus given by
%  \begin{align}
%      1-Pr(Z< t)=Pr(T\ge t)Pr(R\ge t).
%      \label{s21}
%  \end{align}
% Using the cumulative distribution function in \eref{s21} we obtain
By definition, the expectation of the random variable $ min(T,R) $ can be written as
\begin{align}
% &\langle Z \rangle=\int_0^{\infty} dt [1-Pr(Z<t)] \nonumber\\
      \langle min(T,R) \rangle =\int_0^\infty dt Pr(T\ge t)Pr(R\ge t).
       \label{prz}
\end{align}
In the case of exponential resetting rate, we have $f_R(t)=re^{-rt}$ where $r$ is the resetting rate. Recalling $f_T(t)=-\frac{dPr(T\ge t)}{dt}$ to be the first passage time distribution of the underlying parent process, we obtain from \eref{prz}
\begin{align}
     \langle min(T,R) \rangle &= \int_0^\infty dt e^{-rt} \int_t^\infty dt'f_T(t') \nonumber \\
     &=\frac{1}{r}-\frac{1}{r}\int_0^\infty dt e^{-rt}f_T(t) \nonumber \\
     &=\frac{1-\widetilde{T}(r)}{r},
     \label{mintr}
\end{align}
where $\widetilde{T}(r)=\int_0^\infty dt e^{-rt}f_T(t)$
is the Laplace transform of $f_T(t)$. For the case of 1d diffusion $f_T(t)$ is given by L\'evy Smirnov distribution as follows
\begin{align}
    f_T(t)= \frac{L}{\sqrt{4\pi D t^3}}e^{-L^2/4Dt},
\end{align}
with the following Laplace transform 
\begin{align}
 \widetilde{T}(r)=e^{-\sqrt{rL^2/D}} \label{lsfpt}  .
\end{align}
Combining the above with \eref{mintr} gives the result mentioned in the main text
\begin{align}
    \langle min(T,R) \rangle=\frac{1}{r} \left(1- e^{-\sqrt{rL^2/D}} \right).
\end{align}
At this point, we also find $Pr(T<R)$ which will be useful later on. 
\begin{align}
    & Pr(T<R)=\int_0^{\infty}dt f_T(t) Pr(R>t).
\end{align}
For exponential resetting times $Pr(R>t)=e^{-rt}$ and we have
\begin{align}
    Pr(T<R)&=\int_0^{\infty}dt f_T(t) e^{- r t} =\widetilde{T}(r). \label{ptlr}
\end{align}

\section{First passage statistics of the diffusing particle during the return phase} \label{sec3}
In the main text, we have examined the paradigm of a 1d diffusive search process (designated by the diffusion constant $D$) in which
a particle starts at the origin $O$ and continues to diffuse until it hits a
stationary target at a location $L$. In addition, we assumed that the process is reset at a constant rate $r$ (i.e., resetting time density $f_R(t)=re^{-rt}$) upon which a potential $U(x)=\lambda |x|$ centered at the origin is turned on. The particle diffuses through the potential and it is switched off when the particle makes a first return to the origin. Subsequently, the particle resumes its diffusive search phase. Notably, during the return phase the particle also has a finite probability to get absorbed at $L$. Thus, the return phase set-up is akin to a diffusing particle in a confining interval. The interval has two boundaries both of which serve as absorbing boundaries. 
Being absorbed at $L$ marks the completion of the full process while absorption at the origin renews a trial where the diffusive search phase restarts. For this set-up, several quantities such as the mean first time to any of the boundaries, conditional times and splitting probabilities to to each of the boundaries were used in the main text. The aim of this section is to derive these quantities from scratch. Some of these results can also be found in \cite{redner2001}.

% In this section, we derive the splitting probabilities and the unconditional mean first passage times of a Brownian particle under $U(x)=\lambda |x|$ potential for completeness and can also be found in [1].

Let us denote the propagator in the return phase by $G_{ret}(x,t)$ in the presence of two absorbing boundaries at $x=0$ and $x=L$. We start by recalling that the Fokker-Planck equation for the probability distribution function $G_{ret}(x,t)$ which can be written as \cite{redner2001}
\begin{align}
    \frac{\partial G_{ret}(x,t)}{\partial t}-\lambda \frac{\partial G_{ret}(x,t)}{\partial x}=D\frac{\partial^2 G_{ret}(x,t)}{\partial x^2}
    \label{eq29},
\end{align}
with the initial condition 
\begin{align}
    G_{ret}(x,t=0)=\delta (x-x_0),
\end{align} 
where $x_0$ is the position of the particle at the time of resetting. The boundary conditions read
\begin{align}
    G_{ret}(x=0,t)=G_{ret}(x=L,t)=0.
\end{align} 
In Laplace space, \eref{eq29} reads
\begin{align}
 %   G_{ret}(x,t\to \infty)-G_{ret}(x,t=0)+s\widetilde{G}_{ret}(x,s)-\lambda \frac{\widetilde{G}_{ret}(x,s)}{\partial x}=D\frac{\partial^2 \widetilde{G}_{ret}(x,s)}{\partial x^2} \nonumber \\
    D\frac{\partial^2 \widetilde{G}_{ret}(x,s)}{\partial x^2}+\lambda \frac{\widetilde{G}_{ret}(x,s)}{\partial x}-s\widetilde{G}_{ret}(x,s)=-\delta (x-x_0), \label{GLS}
\end{align}
where $\widetilde{G}_{ret}(x,s)=\int_0^\infty G_{ret}(x,t)e^{-st}dt$ is the Laplace transform of the propagator $G_{ret}(x,t)$. 
We can solve \eref{GLS} by employing the standard Green's function method which gives \begin{align}
   \widetilde{G}_{ret}(x,s)&= -\frac{e^{\frac{\lambda  (x_0-x)}{2 D}} \text{csch}\left(mL\right)}{2mD} \Bigg[\theta (x-x_0) \Big(\cosh \left(m(L+x-x_0)\right)-\cosh \left(m(L-x+x_0)\right)\Big)\nonumber \\
   & \hspace{4cm}+ \cosh \left(m(L-x-x_0)\right)-\cosh \left(m(L+x-x_0)\right)\Bigg],
   \label{gls}
\end{align}
where, $m=\frac{\sqrt{\lambda^2 + 4Ds}}{2D}$ and $\theta(x)$ is the Heaviside step function. 


\subsection{The mean first passage time}
The survival probability $Q(t)$ of the particle in this set up is given as
\begin{align}
   Q(t)=\int_{0}^L~dx~G_{ret}(x,t),
\end{align}
which in Laplace space reads
\begin{align}
    \widetilde{Q}(s)=\int_0^{\infty}dt e^{-st} Q(t)=\int_{0}^L \widetilde{G}_{ret}(x,s) dx.
\end{align}
The unconditional mean first passage time that the particle can exit through any of the boundaries is given by
\begin{align}
   \langle t_2(x_0) \rangle&= \lim_{s\to 0}  \widetilde{Q}(s)
   =\frac{L(1-e^{\lambda x_0/D}) +x_0(e^{\lambda L/D }-1)}{\lambda(e^{\lambda L/D}-1)},
   \label{ucmfpt}
\end{align}
which was used in the main text. When the particle is at the negative side quadrant, it only experiences the boundary at the origin. The mean time for this particle to the origin can simply be obtained by setting $L\to \infty$ so that we have
% \begin{align}
%    \langle t_1(x_0) \rangle= \langle t_2(x_0,L\to \infty) \rangle=\frac{x_0}{\lambda}
% \end{align}
% and when the particle starts its motion from $x_0<0$ then the MFPT is given as
% \begin{align}
%    \langle t_1(x_0) \rangle=\frac{-x_0}{\lambda}.
% \end{align}
% Together, we have
\begin{align}
    \langle t_1(x_0) \rangle=\frac{|x_0|}{\lambda}, \label{ucmfpt1}
\end{align}
which was used in the main text.

\subsection{Splitting probabilities}
The probability flux through each of the boundaries in Laplace space is given by \cite{redner2001}
\begin{align}
    j_L(x_0,s)&=-D\frac{\partial  \widetilde{G}_{ret}(x,s)}{\partial x}\bigg|_{x=L}, \label{jl}\\
        j_O(x_0,t)&=D\frac{\partial  \widetilde{G}_{ret}(x,s)}{\partial x}\bigg|_{x=0}.
        \label{jo}
\end{align}
Using \eref{gls}, (\ref{jl}), (\ref{jo}), one can immediately write the corresponding exit/splitting probabilities through each of the boundaries. These read
\begin{align}
   \epsilon_L(x_0)=j_L(x_0,s\to 0)&=\frac{1-e^{\lambda x_0/D}}{1-e^{\lambda L/D}}, \label{el}\\
    \epsilon_O(x_0)=j_O(x_0,s\to 0)&=1-\epsilon_L(x_0) =\frac{e^{\lambda x_0/D}-e^{\lambda L/D}}{1-e^{\lambda L/D}}. \label{eo}
\end{align}



\section{Derivation of Eq. (5) in the main text}\label{Sec2}
In this section, we derive Eq. (5) in the main text for one-dimensional (1d) search process. However, before doing so we make a general discussion on the splitting probabilities and conditional times for a generic stochastic process. 


\subsection{Splitting probabilities -- general definition}
For any arbitrary stochastic process in the presence of multiple targets (and not necessarily a diffusive process), a generic definition for the splitting probabilities can be made in the following way
\begin{align}
    \epsilon_A(x)=Pr(T_{ret}^A(x)< T_{ret}^O(x)),
\end{align}
where $T_{ret}^i(x)$ is the random time to reach  $i \in \{A,O\}$, starting from a coordinate $x$. Thus, splitting probability to the target $A$ is conditioned on the fact that $T_{ret}^A(x)< T_{ret}^O(x)$. 
Similarly, 
\begin{align}
\epsilon_O(x)=Pr(T_{ret}^O(x)\le T_{ret}^A(x))
\end{align}
is the splitting probability to reach the target $O$ before it hits the other boundaries. For diffusive process, these were computed in the previous section.


\subsection{Conditional and unconditional mean first-passage/exit times -- general definition}
Similar to the above, one can also consider the average of the conditional or first passage times either to the origin or any of the boundaries. For example, the conditional mean time to reach the origin before it reaches the boundary $A$, starting from $x$, can be written as
\begin{align}
    \langle T_{ret}^O(x)|T_{ret}^O(x)\le T_{ret}^A(x) \rangle= \langle t(x) \rangle_O.
\end{align}
Similarly, 
the conditional time to reach the boundary  $\{ A \}$ before it hits the origin $O$ is given by
\begin{align}
    \langle T_{ret}^A(x)|T_{ret}^A(x)<T_{ret}^O(x) \rangle= \langle t(x) \rangle_A.
\end{align}
Thus, the unconditional mean first passage time to exit through any of the boundaries without any preference, starting from $x$, is given by $\epsilon_A(x) \langle t(x) \rangle_A+ \epsilon_O(x)
    \langle t(x) \rangle_O$.



\subsection{Back to the derivation of Eq. (5)}
Before we derive Eq. (5), let us recall the one-dimensional set-up again from Sec. \ref{sec3} and in particular, note that the target ($A$) is located at $L>0$. Thus, if the searcher starts from $x<0$ in the return phase, it will not find $L$ and thus % We let the target be placed at $L>0$.  Note that, the searcher has to cross the origin first before reaching the target when it starts returning from below the origin $O$ ($x=0$), so that for {$x<0$}, 
$Pr(T_{ret}^O(x)\le T_{ret}^L(x))=1$ \text{~and~} $Pr(T_{ret}^L(x)< T_{ret}^O(x))=0$ so that
\begin{align}
\langle min\left(T_{ret}^L(x),T_{ret}^O(x)\right)\rangle=\langle T_{ret}^O(x) \rangle_{return}= \langle t_1(x) \rangle, \hspace{1cm} \text{for } x<0,
\end{align}
where  $\langle t_1(x) \rangle$, as defined before, denotes the MFPT of the searcher with only one virtual absorbing boundary at the origin $O$ during the return phase. 

%The origin acts as a virtual absorbing boundary due to the fact that when the searcher reaches the origin, its return phase is over and the search phase begins.

If the searcher starts from $x>0$, the situation is subtle since it has both the possibilities of reaching the target ($L$) or the origin ($O$) during the return. 
In this case, 
\begin{align}
&\langle min\left(T_{ret}^L(x),T_{ret}^O(x)\right)\rangle\nonumber \\
    =&~Pr(T_{ret}^O(x)\le T_{ret}^L(x)) \langle T_{ret}^O(x)|T_{ret}^O(x)\le T_{ret}^L(x) \rangle +Pr(T_{ret}^L(x)< T_{ret}^O(x)) \langle T_{ret}^L(x)|T_{ret}^L(x)< T_{ret}^O(x) \rangle\nonumber \\
    =&~ \epsilon_O(x) \langle t(x) \rangle_O + \epsilon_L(x) \langle t(x) \rangle_L \nonumber \\
    =&~\langle t_2(x) \rangle, \hspace{1cm} \text{for } x>0,
\end{align}
where $\langle t_2(x) \rangle$, as defined earlier, denotes the unconditional MFPT of a searcher to exit through any of the two boundaries at $x=0$ and $x=L$ during the return phase.
% For $x>0$
% \begin{align}
%     Pr(T_{ret}^L(x)< T_{ret}^O(x))=\epsilon_+(x)
% \end{align}
% which is simply the conditional/splitting probability of the particle getting out of the boundary at $x=L$ during return motion starting from $x$, where there is also a virtual absorbing boundary present at the origin. Also, we have
% \begin{align}
%     \langle T_{ret}^L(x)|T_{ret}^L(x)< T_{ret}^O(x) \rangle_{return}= \langle t(x) \rangle_+
% \end{align}
% which is the conditional first passage time of the searcher to exit through the boundary at $x=L$ starting from $x$ while returning. In a similar fashion we have
% \begin{align}
% Pr(T_{ret}^O(x)\le T_{ret}^L(x))=\epsilon_-(x)
% \end{align}
% which is the conditional/splitting probability of the searcher reaching the line at $x=0$ starting at $x$ and
% \begin{align}
%     \langle T_{ret}^O(x)|T_{ret}^O(x)\le T_{ret}^L(x) \rangle_{return}= \langle t(x) \rangle_-
% \end{align}
% is the conditional mean first passage time of the searcher to exit through origin starting from $x$. Then for $x>0$ 
% \begin{align}
% &\langle min\left(T_{ret}^L(\Vec{x}),T_{ret}^O(\Vec{x})\right)\rangle\nonumber \\
%     =&Pr(T_{ret}^O(x)\le T_{ret}^L(x)) \langle T_{ret}^O(x)|T_{ret}^O(x)\le T_{ret}^L(x) \rangle\nonumber \\
%     & \hspace{6cm}+Pr(T_{ret}^L(x)< T_{ret}^O(x)) \langle T_{ret}^L(x)|T_{ret}^L(x)< T_{ret}^O(x) \rangle\nonumber \\
%     =& \epsilon_-(x) \langle t(x) \rangle_- + \epsilon_+(x) \langle t(x) \rangle_+ \nonumber \\
%     =&\langle t_2(x) \rangle, \hspace{6cm} \text{for } x>0. \nonumber
% \end{align}
% Here $\langle t_2(x) \rangle$ denotes the unconditional MFPT of a searcher to exit through any of the two boundaries at $x=0$ and $x=L$ during the return phase.
Combining the results for $x<0$ and $x>0$, we can write the second term in the numerator of \eref{s10} as
\begin{align}
   \langle min\left(T_{ret}^L(x),T_{ret}^O(x)\right)\rangle=\theta(-x)\langle t_1(x) \rangle + \theta(x)\langle t_2(x) \rangle ,
    \label{mintlto}
\end{align}
which is Eq. (5) in the main text. Similarly, the second term in the denominator of Eq. (4) (or \eref{s10}) can be written as
\begin{align}
    Pr(T_{ret}^O(x)\le T_{ret}^L(x))=\theta (-x)+ \theta(x) \epsilon_O(x).
    \label{prtlto}
\end{align}
\eref{mintlto} and \eref{prtlto} combindedly give the one-dimensional form of the MFPT in \eref{s10} as 
\begin{align}
    \langle T_R \rangle =\frac{\langle min(T,R) \rangle +\int_{-\infty}^Ldx\widetilde{G}_R(x)\left[\theta(-x)\langle t_1(x) \rangle + \theta(x)\langle t_2(x) \rangle\right]}{1-\int_{-\infty}^Ldx~\widetilde{G}_R(x) \left[\theta (-x)+ \theta(x) \epsilon_O(x)\right]}.
    \label{s24}
\end{align}
In case of exponential resetting times $\widetilde{G}_R(x)=r \widetilde{G}(x,r)$ with $\widetilde{G}(x,r)$ being the Laplace transform of $G(x,t)$ -- the propagator for the reset free process. Eq. (\ref{s24}) was used to derive Eq. (6) in the main text.

\section{General expression for MFPT with instantaneous return}
In the case when the searcher returns to the starting position instantaneously, the return time is exactly zero. That implies the second term in the numerator of \eref{s10} is zero i.e. $\langle min\left(T_{ret}^A(\Vec{x}),T_{ret}^O(\Vec{x})\right)\rangle=0$. Alongside that as the return to origin is guaranteed we also have $Pr(T_{ret}^O(\Vec{x})\le T_{ret}^A(\Vec{x}))=1$. Applying these conditions in \eref{s10}, we get
\begin{align}
    \langle T_R^{inst} \rangle =\frac{\langle min(T,R) \rangle }{1-\int_{\mathcal{D}}d\Vec{x}~\widetilde{G}_R(\Vec{x}) }. \label{inst1}
\end{align}
The second term in the denominator can be simplified as the following
\begin{align}
    \int_{\mathcal{D}}d\Vec{x}~\widetilde{G}_R(\Vec{x})&=\int_{\mathcal{D}}d\Vec{x} \int_{0}^{\infty}dt f_R(t) G(\Vec{x},t) \nonumber\\
    &=\int_{0}^{\infty}dt f_R(t) \int_{\mathcal{D}}d\Vec{x} ~G(\Vec{x},t). 
\end{align}
Note that $\int_{\mathcal{D}}d\Vec{x} ~G(\Vec{x},t)$ is nothing but the survival probability $Q(t)$ defined in \eref{surv}. Thus, we have
\begin{align}
    \int_{\mathcal{D}}d\Vec{x}~\widetilde{G}_R(\Vec{x})&=\int_{0}^{\infty}dt f_R(t) Q(t) \nonumber\\
    &=\int_{0}^{\infty}dt f_R(t) Pr(T\ge t) \nonumber\\
    &=Pr(T\ge R) \nonumber\\
    &=1-Pr(T<R).
    \label{intgr}
\end{align}
Substituting the above  into \eref{inst1}, we recover the MFPT for resetting under instantaneous return
\begin{align}
     \langle T_R^{inst} \rangle =\frac{\langle min(T,R) \rangle }{Pr(T<R) }.
     \label{inst}
\end{align}
Further, for exponential waiting time between resetting events, we can use \eref{mintr} and \eref{ptlr} to recover
\begin{align}
     \langle T_R^{inst} \rangle=\frac{1-\widetilde{T}(r)}{r \widetilde{T}(r)}. \label{tinst}
\end{align}




\section{Variation of MFPT $\langle \tau (z, \text{Pe}) \rangle$ with  resetting rate $z$}
In this section, we show how MFPT $\langle \tau (z, \text{Pe}) \rangle$ as in Eq. (6) of the main text, varies with respect to the resetting rate $z=\frac{r L^2}{D}$ for different values of $\text{Pe}=\frac{\lambda L}{D}$.
% Figure environment removed
Let us first discuss the limiting cases. For low enough values of $z$ the MFPT diverges as the particle can disperse much distance away from the target. For very large values of $z$ the potential is effectively  always turned on and the corresponding MFPT saturates to a finite value given by (also shown in main text)
\begin{align}
     \langle \tau(z\to\infty,\text{Pe})\rangle=\frac{2 e^{\text{Pe}}-\text{Pe}-2}{\text{Pe}^2}.
\end{align}
The MFPT shows a non-monotonic behaviour with respect to Pe. For low Pe numbers, the system is effectively diffusive and thus, in the absence of resetting, MFPT is infinite. Even with resetting (and low Pe number), the system effectively remains diffusive since the particle experiences a very weak attraction to the origin. In other words, both the search and return phase remain diffusive resulting in a high MFPT. For very high Pe, the return is instantaneous and we recover the result of the classical first passage under resetting \cite{evans_diffusion_2011}. 

% For a finite value of Pe however both of these phenomenon is supressed and we one obtain a optimum MFPT. This non-monotonicity is also prominent in Fig. S4 when we look at the curves for different values of Pe. 

% Now we note that as $Pr(T>t)$ represents the survival probability of the particle up to time $t$, thus we may write,
% \begin{align}
%    f_T(t)&=-\frac{dPr(T>t)}{dt} \nonumber \\
%    &=-\frac{d}{dt}\int_0^\infty dx G(\Vec{x},t)
% \end{align}
% Thus we obtain with slight simplification from \eref{ttr},
% \begin{align}
%     \tilde{T}(r)&=\int_0^\infty dt e^{-rt}f_T(t) \nonumber \\
%     &=1- r\int_0^\infty dx \tilde{G}_0(\Vec{x},r)
% \end{align}

% where, $ \tilde{G}_0(\Vec{x},r)$ is the Laplace transform of the free range propagator $G(\Vec{x},t)$ . This gives us the value of $\tilde{T}(r)$ as, given in \eref{eq20}.




\section{MFPT for resetting with deterministic return in 1d}
For the deterministic return i.e., when the particle returns to the origin following some deterministic dynamics, there is zero probability of target finding resulting in
\begin{align}
    \epsilon_L(x)=1-\epsilon_O(x)=0.
\end{align}
Henceforth, for the diffusing Brownian particle that returns to origin with a constant velocity $\lambda$ from $x$, the return time is given by
\begin{align}
    \langle t_1(x)\rangle=\langle t_2(x)\rangle=\frac{|x|}{\lambda}.
\end{align}
Note that unlike stochastic return, cases for the $x>0$ and $x<0$ for the deterministic return are identical. Substituting the above expressions into \eref{mintr} and \eref{s24} we obtain the MFPT as
\begin{align}
     \langle T_{det}\rangle =  \frac{\frac{1-\widetilde{T}(r)}{r}  +r\int_{{-\infty}}^L dx \widetilde{G}(x,r)\frac{|x|}{\lambda}}{1-\int_{-\infty}^Ldx~\widetilde{G}(x,r)},\label{s27}
\end{align}
where $\widetilde{G}(x,r)$ and $\widetilde{T}(r)$ are the Laplace transform of the reset free process propagator and the first passage time distribution respectively. Here invoking \eref{intgr} and \eref{ptlr} we can replace the denominator with simply $\widetilde{T}(r)$ to have
\begin{align}
     \langle T_{det}\rangle =  \frac{\frac{1-\widetilde{T}(r)}{r}  +r\int_{{-\infty}}^L dx \widetilde{G}(x,r)\frac{|x|}{\lambda}}{\widetilde{T}(r)} \label{tdet}.
\end{align}
For the diffusive dynamics with propagator
\begin{align}
    G(x,t)&=\frac{1}{\sqrt{4\pi Dt}}\left(e^{-\frac{x^2}{4Dt}} -e^{-\frac{(2L-x)^2}{4Dt}}\right), \nonumber\\
   \implies \widetilde{G}(x,r)&=\frac{1}{\sqrt{4Dr}}\left(e^{-\sqrt{r/D}|x|}-e^{-\sqrt{r/D}(2L-x)}\right), \label{frp}
\end{align}
and \eref{lsfpt}, we can solve the intergrals in \eref{tdet} to arrive at the following expression for MFPT with deterministic return in dimensionless form
\begin{align}
    \langle \tau_{det}(z,\text{Pe}) \rangle=\frac{D \langle T_{det} \rangle }{L^2}=\frac{e^{\sqrt{z}}-1}{z}+\frac{1}{\text{Pe}} \left(\frac{2 \sinh \left(\sqrt{z}\right)}{\sqrt{z}}-1 \right),
    \label{mfptdet}
    \end{align}
where recall $z=\frac{rL^2}{D}$ and Pe$=\frac{\lambda L}{D}$ (also obtained in \cite{pal_search_2020}).

\section{Limiting behaviours of $\langle \tau(z,\text{Pe}) \rangle$ with respect to $\text{Pe}$ for different $z$}
In this section, we illustrate further detail on the limiting behaviour of MFPT of stochastic return for the 1d diffusive system with respect to the Pe number. Each panel in Fig. (\ref{comp}) shows i) MFPT with stochastic return i.e. $\langle \tau(z,\text{Pe}) \rangle$ as obtained in Eq. 6 of the main text  (solid red line), ii) MFPT for deterministic return $\langle \tau_{det}(z,\text{Pe}) \rangle$ (dotted line) as in \eref{mfptdet} and iii) MFPT when the resetting rate is very high (i.e., in effect the potential is always turned on) given by $ \langle \tau(z\to\infty,\text{Pe})\rangle=\frac{2 e^{\text{Pe}}-\text{Pe}-2}{\text{Pe}^2}$ (dotted dashed line). Note that the curve for MFPT (with potential always ON) overlaps (the left envelope of the solid curve) more and more to the MFPT under stochastic return as one increases the resetting rate $z$ (going from the left panel to the right). When Pe is sufficiently strong there is almost a negligible probability for the target detection during the return phase and the particle reaches the origin with constant speed (due to the linear nature of the potential). Thus, this limit (i.e, the right envelope of the solid curve) resonates with the deterministic return problem \cite{pal_search_2020}.


% Figure environment removed






% \section{Derivation of Eq. (5) in the main text}\label{S3}
% To find the points $z_t$ and $ z_c$, we define the function 
% \begin{align}
% f(z,Pe)=\frac{\langle \tau(z,Pe) \rangle}{\langle \tau_{inst}(z) \rangle}-1.
% \end{align}
% Note that at the transition point $z_t$ this function satisfies the following two conditions,
% \begin{align}
%     f(z,Pe)=0 \label{cond1} \\
%     \frac{\partial f(z,Pe)}{\partial Pe}=0 \label{cond2}
% \end{align}

% The first condition is due to the fact that at $z_t$ MFPT in the case of stochastic return becomes equal to the case of instantaneous return. The second condition is because there exists a local minimum at some finite value of $Pe$. \eref{cond1} and \eref{cond2} gives two curves in the $z-Pe$ plane. The intersection of both these curves is where both the above conditions are satisfied (red dot in Fig. \ref{fig3}(a)). The intersection point is found to be $(z_t,Pe_t)\approx (7.41425,2.26073)$.
% % Figure environment removed

% To find $z_c$ we note that $z=z_c$ is the inflection point, so the second derivative has to vanish here(in addition to condition \ref{cond2}) and we have,
% \begin{align}
%     \frac{\partial^2f(z,Pe)}{\partial Pe^2}=0
%     \label{cond3}
% \end{align}
% This gives a new curve in the $z-Pe$ plane (green one). The intersection of this curve with that due to \eref{cond2} gives the point $(z_c,Pe_c) \approx (3.64328,4.57685)$ (blue star in \fref{fig3}(a)).
\section{Derivation of the criterion (Eq. (7) in the main text)}
In this section, we derive the criterion presented in Eq. (7). The criterion states that for stochastic return to expedite the  instantaneous return one should have
\begin{align}
     &\langle T_R \rangle < \langle T^{inst}_R \rangle.
     \label{eq:criterion}
\end{align}
Rewriting \eref{s10} as 
\begin{align}
    \langle T_R \rangle=\frac{\langle min (T,R) \rangle + \langle min\left(T_{ret}^A,T_{ret}^O\right) \rangle}{1-Pr(T_{ret}^O\le T_{ret}^A)}
    \label{s33},\end{align} 
where we have used the following
\begin{align}
    \langle min\left(T_{ret}^A,T_{ret}^O\right) \rangle=\int_{\mathcal{D}}d\Vec{x}\widetilde{G}_R(\Vec{x})\langle min\left(T_{ret}^A(\Vec{x}),T_{ret}^O(\Vec{x})\right)\rangle,
\end{align}
and 
\begin{align}
Pr(T_{ret}^O\le T_{ret}^A)=\int_{\mathcal{D}}d\Vec{x}~\widetilde{G}_R(\Vec{x}) Pr(T_{ret}^O(\Vec{x})\le T_{ret}^A(\Vec{x})) .
\end{align}
Using \eref{inst} and \eref{s33} in \eref{eq:criterion}, we have
\begin{align}
   \frac{\langle min (T,R) \rangle + \langle min\left(T_{ret}^A,T_{ret}^O\right) \rangle}{1-Pr(T_{ret}^O\le T_{ret}^A)}< \frac{\langle min(T,R) \rangle }{Pr(T<R) }. \label{s35}
\end{align}
Note 
\begin{align}
   Pr(T_{ret}^O\le T_{ret}^A)&=\int_{\mathcal{D}}d\Vec{x} ~\widetilde{G}_R(\Vec{x}) Pr(T_{ret}^O(\Vec{x})\le T_{ret}^A(\Vec{x})) \nonumber\\
   &=\int_{\mathcal{D}}d\Vec{x} ~\widetilde{G}_R(\Vec{x}) [1-Pr(T_{ret}^A(\Vec{x})<T_{ret}^O(\Vec{x}))] \nonumber\\
   &= \int_{\mathcal{D}}d\Vec{x}~\widetilde{G}_R(\Vec{x})-\int_{\mathcal{D}}d\Vec{x} ~\widetilde{G}_R(\Vec{x}) Pr(T_{ret}^A(\Vec{x})<T_{ret}^O(\Vec{x})) \nonumber\\
   &=1-Pr(T<R)-Pr(T_{ret}^A<T_{ret}^O), \hspace{0.5cm} \text{using \eref{intgr}} .
\end{align}
One can now rearrange \eref{s35} to find
\begin{align}
     &\frac{\langle min (T,R) \rangle + \langle min\left(T_{ret}^A,T_{ret}^O\right) \rangle}{Pr(T<R)+Pr(T_{ret}^A<T_{ret}^O)}< \frac{\langle min(T,R) \rangle }{Pr(T<R) } \nonumber\\
     % \implies & Pr(T<R) [\langle min (T,R) \rangle + \langle min\left(T_{ret}^A,T_{ret}^O\right) \rangle]< \langle min(T,R) \rangle [Pr(T<R)+Pr(T_{ret}^A<_{ret}T_{ret}^O)] \nonumber\\
     % \implies & Pr(T<R) \langle min\left(T_{ret}^A,T_{ret}^O\right) \rangle < \langle min(T,R) \rangle Pr(T_{ret}^A<T_{ret}^O)\nonumber\\
     \implies & \mathcal{T} \equiv \frac{
\langle min\left(T_{ret}^A,T_{ret}^O\right) \rangle}{Pr(T_{ret}^A<T_{ret}^O)} <  \frac{\langle min(T,R) \rangle }{Pr(T<R) } = \langle T^{inst}_R \rangle,
\label{final-cri}
\end{align}
which is the general criterion Eq. (7) announced in the main text.



\section{Illustrating the criterion in Eq. (7) for diffusive search}
In this section, we illustrate the criterion derived in the previous section for the 1d diffusive search. To this end, we compute each term in \eqref{final-cri} which is also Eq. (7) in the main text. For instance,
\begin{align}
    \mathcal{T}=\frac{
\langle min\left(T_{ret}^L,T_{ret}^O\right) \rangle}{Pr(T_{ret}^L<T_{ret}^O)} =\frac{\int_{-\infty}^L dx \widetilde{G}(x,r)\Big[\theta(-x)\langle t_1(x) \rangle + \theta(x)\langle t_2(x) \rangle \Big]}{\int_{0}^{L} dx~\widetilde{G}(x,r) \epsilon_L(x)},
    \label{sr1d}
\end{align}
where we have used \eref{mintlto} and 
\begin{align}
Pr(T_{ret}^L<T_{ret}^O)&=\int_{-\infty}^Ldx~\widetilde{G}(x,r) Pr(T_{ret}^L(x)<T_{ret}^O(x)) \nonumber\\
&=\int_{0}^{L} dx~\widetilde{G}(x,r) \epsilon_L(x).
\end{align}
Note the negative part of the integration vanishes as the probability of reaching $L$ is simply zero there during the return phase. 


One can now plugin the propagator for diffusion (in Laplace space) i.e., $\widetilde{G}(x,r)$ from \eref{frp}, the unconditional MFPTs $\langle t_1(x) \rangle, \langle t_2(x) \rangle$ from \eref{ucmfpt1}, \eref{ucmfpt}, the splitting probability to target at $L$ i.e. $\epsilon_L(x)$ from \eref{el} and $\widetilde{T}(r)$ from \eref{lsfpt}. The quantity $\mathcal{T}$ then takes the dimensionless form (scaled by $L^2/D$)\begin{small}
\begin{align}
   \mathcal{T}(z,\text{Pe})= \frac{\text{Pe}^2 \left(2 \sqrt{z} e^{\text{Pe}+\sqrt{z}}-e^{2 \sqrt{z}} \left(2 e^{\text{Pe}}+\sqrt{z}-2\right)+2 e^{\text{Pe}}-\sqrt{z}-2\right)-\text{Pe} \left(e^{2 \sqrt{z}}-1\right) z+2 \left(e^{\text{Pe}}-1\right) \left(e^{2 \sqrt{z}}-1\right) z}{\text{Pe} \sqrt{z} \left(\text{Pe}^2 \left(e^{\sqrt{z}}-1\right)^2+\text{Pe} \left(e^{2 \sqrt{z}}-1\right) \sqrt{z}-2 \left(e^{\text{Pe}}-1\right) e^{\sqrt{z}} z\right)},
\end{align}
\end{small}whereas the RHS which is the MFPT of the instantaneous return can be found by inserting \eref{lsfpt} in \eref{tinst} (or taking the limit Pe$\to\infty$ in Eq. (6) of  main text)  as
\begin{align}
   \langle\tau_{inst}(z) \rangle=\frac{D}{L^2}\langle T^{inst}_R \rangle =\frac{e^{\sqrt{z}}-1}{z}.
\end{align}
The criterion in Eq. (7) (or \eref{final-cri}) holds when $\frac{\mathcal{T}(z,\text{Pe})}{\langle\tau_{inst}(z) \rangle}<1$ and stochastic return facilitates first passage over instantaneous return. The equality 
\begin{align}
\mathcal{T}(z,\text{Pe})= \langle\tau_{inst}(z) \rangle,    
\end{align}
gives a solution for $z^*$ for each Pe. Thus, one can generate a separatrix (the red dashed line in Fig. (3) of main text which is essentially the line $z^*(\text{Pe})$) that distinguishes between two regions -- either stochastic return (SR) or instantaneous return (IR) better. In Table S1,  we show different limits of this ratio and conclude where stochastic return is better than instantaneous return for the diffusive case.
\renewcommand{\arraystretch}{3}
\begin{table}[H]
\centering
\caption{ Demonstration  of criterion in Eq. (7) for 1d diffusion in different limits of $z$ and Pe}
\begin{adjustbox}{width=15cm}
\begin{tabular}{||c|c|c|c|c||}
\hline
Quantity &$z\to 0$, Finite Pe & $z\to \infty$, Finite Pe & Pe$\to 0$, Finite $z$ &  Pe$\to \infty$, Finite $z$ \\
\hline
$\mathcal{T}(z,\text{Pe})$ & $\frac{1}{z}\left(\frac{2 \text{Pe} \left(1-e^{\text{Pe}}\right)  }{\text{Pe} (\text{Pe}+2)-2 e^{\text{Pe}}+2  }\right)$ & $\frac{2 e^{\text{Pe}}-\text{Pe}-2}{\text{Pe}^2}$ & $\frac{1}{\text{Pe}}\frac{1}{1-\sqrt{z} \text{csch}\left(\sqrt{z}\right)}$ & $\text{Pe}\left(\frac{2  \sinh \left(\sqrt{z}\right)}{z^{3/2}}-\frac{1}{z}\right)$ \\
\hline
$\langle\tau_{inst}(z) \rangle$ & $ \frac{1}{\sqrt{z}}$ & $\frac{e^{\sqrt{z}}}{z}$ & $\frac{e^{\sqrt{z}}-1}{z}$ & $\frac{e^{\sqrt{z}}-1}{z}$  \\
\hline
$\frac{\mathcal{T}(z,\text{Pe})}{\langle\tau_{inst}(z) \rangle}$ & $>1$, IR better & $<1$, SR better & $>1$, IR better & $>1$, IR better\\
\hline
\end{tabular}
\end{adjustbox}
\end{table} 


% \begin{thebibliography}{2776}
% \bibitem{red} S. Redner, A Guide to First-Passage Processes (Cam-
% bridge University Press, 2001).

% \bibitem{palprr} A. Pal, L. Kusmierz, and S. Reuveni, Physical Review
% Research 2, 043174 (2020).

% \bibitem{bodrovapre} A. S. Bodrova and I. M. Sokolov, Physical Review E 101,
% 052130 (2020).

% \bibitem{majumdarjphys}M. R. Evans, S. N. Majumdar, and G. Schehr, Journal
% of Physics A: Mathematical and Theoretical 53, 193001
% (2020).

%  \bibitem{majumdarprl}M. R. Evans, S. N. Majumdar, Physical Review Letters 106, 160601 (2011)
% \end{thebibliography}
 \end{document}
