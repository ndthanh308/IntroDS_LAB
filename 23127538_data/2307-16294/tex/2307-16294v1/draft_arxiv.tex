% ****** Start of file apssamp.tex ******
%
%   This file is part of the APS files in the REVTeX 4.1 distribution.
%   Version 4.1r of REVTeX, August 2010
%
%   Copyright (c) 2009, 2010 The American Physical Society.
%
%   See the REVTeX 4 README file for restrictions and more information.
%
% TeX'ing this file requires that you have AMS-LaTeX 2.0 installed
% as well as the rest of the prerequisites for REVTeX 4.1
%
% See the REVTeX 4 README file
% It also requires running BibTeX. The commands are as follows:
%
%  1)  latex apssamp.tex
%  2)  bibtex apssamp
%  3)  latex apssamp.tex
%  4)  latex apssamp.tex
%

\documentclass[%
 reprint,
superscriptaddress,
%groupedaddress,
%unsortedaddress,
%runinaddress,
%frontmatterverbose, 
%preprint,
%showpacs,preprintnumbers,
%nofootinbib,
%nobibnotes,
%bibnotes,
 amsmath,amssymb,
 aps,
 prl,
%pra,
%prb,
%rmp,
%prstab,
%prstper,
%floatfix,
]{revtex4-1}


\usepackage{graphicx}% Include figure files
\usepackage{dcolumn}% Align table columns on decimal point
\usepackage{bm}% bold math
\usepackage[colorlinks= true, linkcolor=blue, citecolor=blue, urlcolor=blue]{hyperref}% add hypertext capabilities
%\usepackage[mathlines]{lineno}% Enable numbering of text and display math
%\linenumbers\relax % Commence numbering lines
\usepackage{lipsum}
\usepackage{bbold}
\usepackage{mathtools}
\usepackage[normalem]{ulem}
\usepackage{hyperref}
\usepackage{soul}

%\usepackage[showframe,%Uncomment any one of the following lines to test 
%%scale=0.7, marginratio={1:1, 2:3}, ignoreall,% default settings
%%text={7in,10in},centering,
%%margin=1.5in,
%%total={6.5in,8.75in}, top=1.2in, left=0.9in, includefoot,
%%height=10in,a5paper,hmargin={3cm,0.8in},
%]{geometry}

\def\hM{{\bm{M}}}
\def\hK{{\bm{K}}}
\def\hg{{\boldsymbol{\gamma}}}
\def\hG{{\boldsymbol{\Gamma}}}
\def\bS{{\boldsymbol{\Sigma}}}
\def\bG{{\bm{G}}}
\def\bg{{\bm{g}}}
\def\bea{\begin{eqnarray}}
\def\eea{\end{eqnarray}}
\def\etal{et al.}
\def\nn{\nonumber}
\def\om{\omega}
\def\f{\frac}
\def\bn{{\bm{n}}}
\def\be{{\bm{\hat{e}}}}
\def\bell{\bm \ell}
%https://www.overleaf.com/project/
\def\br{\bm r}
\def\bZ{\bm Z}
\def\p{\partial}

\usepackage{xcolor}
\newcommand{\AP}[1]{\textcolor{blue}    {\bf AP: #1}}
\newcommand{\eref}[1]{Eq.~(\ref{#1})}
\newcommand{\fref}[1]{Fig.~\ref{#1}} 
\newcommand{\tc}[1]{\textcolor{magenta}{#1}} 

\begin{document}


\title{Stochasticity in returns can expedite classical first passage under resetting}


\author{Arup Biswas}
\email{arupb@imsc.res.in}
\affiliation{The Institute of Mathematical Sciences, CIT Campus, Taramani, Chennai 600113, India}
\affiliation{Homi Bhabha National Institute, Training School Complex, Anushakti Nagar, Mumbai 400094, India}
\author{Anupam Kundu}
\email{anupam.kundu@icts.res.in}
\affiliation{International Centre for Theoretical Sciences, TIFR, Bangalore, India}

\author{Arnab Pal}
\email{arnabpal@imsc.res.in}
\affiliation{The Institute of Mathematical Sciences, CIT Campus, Taramani, Chennai 600113, India}
\affiliation{Homi Bhabha National Institute, Training School Complex, Anushakti Nagar, Mumbai 400094, India}

%\date{\today}

\begin{abstract}
% 
Classical first passage under resetting is a paradigm in the search process. Despite its multitude applications across interdisciplinary sciences, experimental realizations of such resetting processes posit practical challenges in calibrating these zero time irreversible transitions. We consider a strategy in which resetting is performed using a finite time return control protocol in lieu of instantaneous returns. These controls could also be accompanied with random fluctuations or errors allowing target detection even during the return phase. To better understand the phenomena, we develop a unified renewal approach that can encapsulate arbitrary search processes in a fairly general topography containing targets, various resetting times and return mechanisms in arbitrary dimensions. While such time-costly protocols would apparently seem to impede the overall search efficiency, we show that by leveraging the stochasticity in returns one can expedite the completion gained even under instantaneous resetting. The formalism is then explored to reveal a universal criterion distilling the benefits of this strategy. We demonstrate how this general principle can be utilized to improve overall performance of a one-dimensional diffusive search process. We believe that such strategies designed with inherent randomness can be made optimal with precise controllability in complex search processes. 
\end{abstract}

\pacs{Valid PACS appear here}% PACS, the Physics and Astronomy
                             % Classification Scheme.
%\keywords{Suggested keywords}%Use showkeys class option if keyword
\maketitle
%\onecolumngrid

%\textbf{\emph{Introduction.---}}
Recently, a class of non-equilibrium systems namely resetting or restart has gained a lot of attention due to their multidisciplinary applications in statistical physics \cite{evans_diffusion_2011,kusmierz2014first,gupta2014fluctuating,eule2016non,pal2015diffusion,majumdar2015dynamical,gupta2020work,biroli2023extreme,evans_stochastic_2020,gupta2022stochastic,pogorzelec2023resetting,sokolov2023linear,ghosh2023autonomous,mendez2016characterization,mori2023entropy}, chemical \& biological physics \cite{reuveni2014role,budnar2019anillin,rotbart2015michaelis,biswas2023rate,roldan2016stochastic}, quantum physics \cite{mukherjee2018quantum,yin2023restart,magoni2022emergent}, stochastic processes \cite{kumar2023universal,de2020optimization,de2022optimal,stojkoski2021geometric,wang2022entropy,bressloff2021accumulation,huang2021random}, economics \cite{jolakoski2022fate,stojkoski2022income} and operation research \cite{bonomo2021mitigating}. There has been a spate of excitement in developing theory and application of such non-equilibrium systems, and in parallel, extracting some of the rich physics that lies embedded in such systems using optical trap experiments \cite{tal2020experimental,besga2020optimal,goerlich2023experimental} and programmable robots \cite{altshuler2023environmental,paramanick2023programming}. One of the hallmark results emanating from the seminal work of Evans and Majumdar is the ability of resetting to expedite completion of a first passage process \cite{evans_diffusion_2011}. This is noteworthy since designing optimal search processes has been a central goal of the first passage processes spanning from physics \cite{redner2001,bray2013persistence,metzler2014first}, chemistry \cite{benichou2010geometry,loverdo2008enhanced,lomholt2005optimal}, biology \cite{vergassola2007infotaxis,sheinman2012classes}, economics \cite{gabaix2016dynamics} and ecology \cite{viswanathan1999optimizing,pal_search_2020}. 

There has been a myriad of works showcasing universal dominance of resetting in diffusion alike and arbitrary stochastic processes \cite{evans_stochastic_2020,pal2022inspection,reuveni_optimal_2016,nagar2016diffusion,pal_first_2017,chechkin2018random,evans2018run,sar2023resetting,int-target-2,chen2022first,ray2021resetting,ahmad2019first,pal2019landau,belan2018restart}. Notably, a key assumption here is that resetting occurs instantaneously i.e., the system (say, a Brownian particle) can be reset (or brought back) in zero time. This is, however, a major hindrance to practical realisation or experimental verification in the field since `resetting' should be considered as a finite time process. Moreover, resetting a particle from far should require more time than to reset it from a nearby location. These presumptions seek for a more realistic viewpoint where resetting should be considered as a spatio-temporally coupled process and not just a mere teleportation. 

The first important step to this direction was taken recently by considering the fact that there is a finite time return process that brings the searcher/particle back to a preferred location/home upon resetting \cite{pal_search_2020,tal2020experimental}. It was, however, assumed that the return process is deterministic with absolute precision and moreover, it can not conduct a search (i.e., locate a target) during the return (see also \cite{pal2019invariants,pal2019time,bodrova2020resetting,bodrova2020brownian,radice2021one,stanislavsky2022subdiffusive,bressloff2020search,bressloff2020target,maso2019transport,radice2022diffusion,mercado2020intermittent}). 


Notwithstanding the advancements, deterministic drivings are never perfect and will always be accompanied by uncontrollable random fluctuations. Moreover, precise return controls with regard to the energetics maybe costly as the thermodynamic trade-off relations have taught us \cite{barato2015thermodynamic,horowitz2020thermodynamic,pal2021thermodynamic,pal2023thermodynamic}. It is therefore important to understand how the errors or fluctuations in return process can be incorporated (see e.g., \cite{gupta2021resetting,gupta2020stochastic,gupta2022work,mercado2020intermittent}). More crucially, introducing stochasticity can also render the target-detection `accessible' during the return phase which may lower the global search times and improve overall performance. Taken together, this work aims to underpin the effect of innate stochasticity during returns on the overall search process. Remarkably, we find that such randomness or errors in return times, albeit bearing overhead time penalties, can \textit{often} expedite the classical first passage process under instantaneous resetting. To this end, we unravel a universal condition that rationalizes the underlying physics behind this intriguing observation -- key features are then highlighted for the paradigmatic diffusive process. 




% Figure environment removed


\textbf{\emph{General framework.---}} Consider a prototypical random search process where a searcher starts at the origin $O$, its home, of a $d$-dimensional arena at time zero. The arena may contain one or multiple targets. The searcher can locate one of these targets following a random $T$ which we denote as the first passage time. However, during the search phase, if the target is not found upto a random time $R$ (which we denote as the resetting time), the searcher decides to return to the origin. Crucially, the return process can be stochastic. Owing to this, we assume that the searcher has the ability to detect the target(s) even during the return phase. If the  searcher returns to $O$, failing to detect the target, the search process restarts afresh to the next attempt. Overall, the process is said to be completed once the target is detected irrespective of the searcher's phase (see Fig. \ref{fig1}). 


The time it takes for the searcher to either find the target or the origin during return will typically depend on the searcher’s position $\Vec{x}$ at the time of resetting. Furthermore, various demographical or physical constraints may compel the searcher to follow different routes hence there can be more nontrivial dependence between the position of the searcher and the return time either to the origin (denoted by $T_{ret}^O(\Vec{x})$) or to any of the targets $A$ (denoted by $T_{ret}^A(\Vec{x})$). In the former case, the searcher resumes its search phase upon returning to the origin while in the latter case the process ends. Denoting the overall completion time by a random variable $T_R$ and considering the above-mentioned possibilities, we can write a renewal equation as follows

{
\begin{align}
\begin{array}{l}
T_{R}=\left\{ \begin{array}{lll}
T \hspace{3cm} &\text{if }T<R\\
R+T_{ret}^A(\Vec{x})\hspace{1.5cm} & \text{if }R \leq T~ \& ~T_{ret}^A(\Vec{x})<T_{ret}^O(\Vec{x})\\
 R+T^O_{ret}(\Vec{x})+T_R'\hspace{0.6cm} &\text{if }R \leq T ~
\&~ T_{ret}^O(\Vec{x})\leq T_{ret}^A(\Vec{x}) \end{array}\right.\text{ }\end{array}
\label{renewal}
\end{align}}where $T_R'$ is a statistically independent and identically distributed copy of $T_R$. \eref{renewal} can be written in a more concise form as
\begin{align}
    T_R&=min(T,R)+I(R\le T) min(T_{ret}^A(\Vec{x}), T_{ret}^O(\Vec{x})) \nonumber \\
    &+I(R\le T)I(T_{ret}^O(\Vec{x})\le T_{ret}^A(\Vec{x}))T_R',
        \label{S2}
\end{align}
where $min(u,v)$ is the minimum of two random variables $u$ \& $v$ and $I(u  \leq v)$ is an indicator function that takes value unity when $u \leq v$, and zero otherwise. Thus, $ \langle I(u  \leq v) \rangle=Pr(u \leq v)$ i.e, the probability that $u \leq v$.  Crucially, the indicator function $I(T_{ret}^O(\Vec{x})\le T_{ret}^A(\Vec{x}))$ also depends on the coordinate $\Vec{x}$ of the searcher at the time of resetting. This implies that the expectations on \eref{S2} need to be performed over the underlying stochastic process, the resetting time density $f_R(t)$ and the return protocols. For instance, the expectation $\mathcal{E}_1 \equiv \langle I(R\le T)I(T_{ret}^O(\Vec{x})\le T_{ret}^A(\Vec{x})) T_R'\rangle$ can be computed as follows

{\footnotesize
\begin{align}
    \mathcal{E}_1&= \int_0^\infty dt f_R(t)  \frac{Pr(T\ge t) \int_{\mathcal{D}}d\Vec{x}G_0(\Vec{x},t)\langle I(T_{ret}^O(\Vec{x})\le T_{ret}^A(\Vec{x})) \rangle}{\mathcal{Q}(t)} 
 \langle T_R' \rangle \nonumber \\
 &= \langle T_R \rangle \int_0^\infty dt f_R(t) \int_{\mathcal{D}}d\Vec{x}G_0(\Vec{x},t) Pr(T_{ret}^O(\Vec{x})\le T_{ret}^A(\Vec{x})),
 \label{expt-1}
\end{align}}where $\mathcal{D}$ is the domain of search in arbitrary dimension that can contain one or multiple targets and $G_0(\Vec{x},t)$ is the underlying time-dependent propagator in the presence of targets so that $\mathcal{Q}(t)=Pr(T\geq t) =\int_{\mathcal{D}}d\Vec{x}~G_0(\Vec{x},t)$ becomes the survival probability \cite{redner2001}. In the first line of Eq. (\ref{expt-1}), we have taken averages over $f_R(t)$ and the underlying search propagator. Computing similar expectations from Eq. (\ref{S2}), we obtain the mean first passage time (MFPT) to be \cite{SI}
\begin{align}
\langle T_R \rangle =\frac{\langle min(T,R) \rangle +\langle min\left(T_{ret}^A,T_{ret}^O\right) \rangle}{1-\int_{\mathcal{D}}d\Vec{x}~\Tilde{G}_R(\Vec{x}) Pr(T_{ret}^O(\Vec{x})\le T_{ret}^A(\Vec{x}))} ,
     % \nonumber \\
     % &=\frac{\langle min(T,R) \rangle + \langle min\left(T_{ret}^O,T_{ret}^A\right)\rangle}{Pr(R>T)+ Pr(T_{ret}^A\le T_{ret}^O)}
     \label{mfpt}
\end{align}
where $\Tilde{G}_R({\Vec{x}}) \equiv \int_0^\infty G_0(\Vec{x},t)f_R(t)dt $ is the time-integrated propagator and $\langle min\left(T_{ret}^A,T_{ret}^O\right) \rangle=\int_{\mathcal{D}}d\Vec{x}\Tilde{G}_R(\Vec{x})\langle min\left(T_{ret}^A(\Vec{x}),T_{ret}^O(\Vec{x})\right)\rangle$. On the other hand,  $Pr(T_{ret}^O(\Vec{x})\le T_{ret}^A(\Vec{x}))$ is the splitting probability that the searcher, starting from $\Vec{x}$, reaches the origin before hitting any of the targets during the return phase. 

\eref{mfpt} is central to this study and is remarkably general since it holds for a) any kind of underlying first passage process  (beyond diffusion) conducted in any dimension in the presence of arbitrary target(s) regardless of their variation in size, shape or nature (purely absorbing or partial), b) arbitrary resetting time distributions, and c) generic returning motion such as instantaneous, deterministic or stochastic and their various modes of return. 






\textbf{\emph{Diffusive search with resetting and stochastic return via controlled potential trap.---}} To illustrate how the framework developed above can be
utilized in practice, we examine the paradigm of a 1d diffusive search process (designated by the diffusion constant $D$) in which
a particle starts at the origin $O$ and continues to diffuse until it hits a
stationary target at a location $L$. In addition, assume that the process is reset at a constant rate $r$ (i.e., resetting time density $f_R(t)=re^{-rt}$) upon which a potential $U(x)$ centered at the origin is turned on. The particle diffuses through the potential and it is switched off when the particle makes a first return to the origin. Subsequently, the particle resumes its diffusive search phase (see \cite{gupta2021resetting,gupta2020stochastic} where non-equilibrium steady state and the relaxation properties were studied under this protocol). However, since the return is not purely deterministic, there is always a chance to find the target during the return phase (Fig. \ref{fig1}). In below, we demonstrate how this could expedite the overall search time.



We start by recalling the diffusive propagator of this
process given by $ G_0(x,t)=\frac{1}{\sqrt{4\pi Dt}}\left(e^{-\frac{x^2}{4Dt}} -e^{-\frac{(2L-x)^2}{4Dt}}\right)$ \cite{redner2001}. For diffusive search process, the underlying first passage times $(T)$ are sampled from the L\'evy Smirnov distribution so that $\langle min(T,R) \rangle=\frac{1}{r} \left(1- e^{-\sqrt{rL^2/D}} \right)$ \cite{pal_first_2017}. During the return phase, the particle has the possibility to hit either $O$ or $L$ starting from position $x$ which is the 
coordinate of the particle exactly at the time of resetting. 
Considering that the return phase is modulated by a linear potential $U(x)=\lambda |x|$, we can compute the average time 
for the particle
to reach either of the boundaries namely $\langle t_{2}(x) \rangle = \frac{L(1-e^{\lambda x/D}) +x(e^{\lambda L/D }-1)}{\lambda(e^{\lambda L/D}-1)} $ for $x>0$ and $\langle t_{1}(x) \rangle =|x|/\lambda$ for $x<0$ \cite{redner2001}. Using this one can write the following expectation \cite{SI}
\begin{eqnarray}
\langle min\left(T_{ret}^L(x),T_{ret}^O(x)\right)\rangle=   \theta(-x)\langle t_1(x) \rangle + \theta(x)\langle t_2(x) \rangle,
\end{eqnarray}
where $\theta(x)$ is the step-function. In addition, the splitting probability is given by given by $Pr(T_{ret}^O(x)\le T_{ret}^L(x))=\left[ \theta(x)\frac{e^{\lambda x/D}-e^{\lambda L/D}}{1-e^{\lambda L/D}}+\theta(-x) \right]$ \cite{redner2001}. Substituting these expressions into \eref{mfpt}, we arrive at the following expression for the MFPT  $\langle T_R \rangle=D/L^2 \langle \tau(z,\text{Pe}) \rangle$, where
\begin{small}
\begin{eqnarray}
     &&\langle \tau (z, \text{Pe}) \rangle = \frac{1}{z \text{Pe}^2}\bigg(  -2 e^{\text{Pe}} \text{Pe}^2+2 \text{Pe}^2-z \text{Pe}+2 e^{\text{Pe}} z -2 z \nonumber \\
     &&\hspace{0.4cm} +\frac{2 \left(e^{\text{Pe}}-1\right) \text{Pe} \left(\left(2 e^{\text{Pe}}-1\right) e^{\sqrt{z}}-1\right) \left(\text{Pe}^2-z\right)}{2 e^{\text{Pe}+\sqrt{z}}\text{Pe} -e^{2 \sqrt{z}} \left(\text{Pe}+\sqrt{z}\right)-\text{Pe}+\sqrt{z}}\Bigg),
     \label{mfpt1}    
 \end{eqnarray}
 \end{small}% \begin{align}
% \label{mfpt1} 
%     \langle \tau (z, \text{Pe}) \rangle=\frac{2\alpha\beta\gamma}{z\text{Pe}^2}\left(1 -\frac{z\text{Pe}}{2\alpha\beta\gamma}+\frac{ \text{Pe}(\gamma e^{\sqrt{z}}-1)}{\alpha e^{2\sqrt{z}} -2e^{\alpha}\text{Pe}-\beta}\right)
% \end{align}
and we have introduced a scaled resetting rate  $z=\frac{rL^2}{D}$ and the Peclet number $\text{Pe}=\frac{\lambda L}{D}$ which is a ratio between drift and diffusive time scales  \cite{redner2001}. \fref{fig2} shows corroborated plots (theory \& simulations) of $\langle \tau (z, \text{Pe}) \rangle$ as a function of $\text{Pe}$. \eref{mfpt1} reproduces $ \langle \tau_{inst}(z) \rangle=\langle \tau (z,\text{Pe} \to \infty) \rangle =  \frac{e^{\sqrt{z}}-1}{z}$ which is the MFPT for instantaneous resetting \cite{evans_diffusion_2011} -- shown in Fig. \ref{fig2} by the horizontal dashed lines for different resetting rates. 


Intriguingly, Fig. \ref{fig2} shows a key feature that
%exhibits an intriguing feature that
the MFPT for the stochastic return can be reduced further than the instantaneous resetting (see e.g., the $z=10$ curve). In fact, \eref{mfpt1} reveals the existence of a critical resetting rate $z^*$, which is a function of Pe, beyond which stochastic return always supersedes the instantaneous return in optimizing the search time (see later for the exact evaluation of $z^*$). However, for $z<z^*$ (see e.g., the $z=5$ curve), the MFPT always stays above the instantaneous (dashed) line indicating that no advantage can be gained from the stochastic return for any resetting rate or Pe. 



% Figure environment removed

Several comments can be made on the shape of the MFPT curve in Fig. \ref{fig2} (e.g., $z=10$). In the limit of large $\text{Pe}$ (i.e., for the strongly attractive trap), the particle returns to the origin almost deterministically with constant speed $\lambda$ and has negligible probability to find the target during the return phase. Therefore, one can approximate the right envelope with $ \langle \tau_{det}(z,\text{Pe}) \rangle=\frac{e^{\sqrt{z}}-1}{z}+\frac{1}{\text{Pe}} \left(\frac{2 \sinh \left(\sqrt{z}\right)}{\sqrt{z}}-1 \right)$ \cite{SI,pal_search_2020} -- plotted with dotted line in Fig \ref{fig2}. On the other hand, for smaller $\text{Pe}$ values, the potential is rather shallow and it takes time for the particle either to return to the origin or to the target. However, since $z$ is large, the resetting events are more frequent and thus the contribution to the global MFPT comes predominantly from the post-resetting phase which can be computed from Eq. \eqref{mfpt1} as $ \langle \tau(z\to\infty,\text{Pe})\rangle=\frac{2 e^{\text{Pe}}-\text{Pe}-2}{\text{Pe}^2}$ \cite{redner2001} -- plotted with dot-dashed line in Fig \ref{fig2} showing an excellent agreement with the left envelope of $\langle \tau(z=10,\text{Pe}) \rangle$. 

%The bulk features naturally can not be captured by these extreme limits. 

%In the bulk, however, there is an intricate interplay between finite $z$ and $\text{Pe}$ that can not be captured by these extreme limits. 



\textbf{\emph{A universal criterion for the trade-off between instantaneous and stochastic return---} }
It is evident from the above discussion that stochastic returns can over- or under-perform search process compared to the instantaneous returns. While this observation is made for simple diffusive search, naturally we are intrigued by the question whether this is generically true for any search process. If so, what is the governing criterion for the stochastic returns to be beneficial?

To delve deeper into this question, we first take the limit of instantaneous return \cite{pal_first_2017} where the searcher always returns to the origin in zero time so that $Pr(T_{ret}^O\le T_{ret}^A)=1$ resulting in $\langle T_R^{inst} \rangle=\frac{\langle min(T,R) \rangle}{Pr(R>T)}$ from \eref{mfpt} \cite{SI}. Evidently, stochastic return will be beneficial only if 
$    \langle T_R \rangle < \langle T_R^{inst} \rangle$ which, in turn, implies \cite{SI}
\begin{align}
\mathcal{T} \equiv 
\frac{
\langle min\left(T_{ret}^A,T_{ret}^O\right) \rangle}{Pr(T_{ret}^A<T_{ret}^O)} < \langle T_R^{inst} \rangle ,
  \label{srcond}
\end{align}
where $Pr(T_{ret}^A<T_{ret}^O)=\int_{\mathcal{D}}d\Vec{x}~\Tilde{G}_R(\Vec{x}) Pr(T_{ret}^A(\Vec{x})<T_{ret}^O(\Vec{x})) $. 
Clearly, to understand this trade-off, one needs to consider many realizations of such a process and compare between the time $\mathcal{T}$ it takes, on an average, for the process to either return to the origin or to complete the search, starting from the location at the time of resetting given by $\langle min\left(T_{ret}^A,T_{ret}^O\right) \rangle$ (rescaled with the splitting probability $Pr(T_{ret}^A<T_{ret}^O)$) and the time $ \langle T_R^{inst} \rangle$ that it would have taken for the instantaneous return. This is rather intriguing since the former accumulates time only during the post resetting return phase while latter does so only during the pre-resetting phase. Thus, the relation (\ref{srcond}) puts a strong constraint on the average return time $\mathcal{T}$ regardless of the final destination. Notably, this relation is quite universal since it does not depend on the particular choice of the underlying first passage process, resetting time density 
or the return protocol. 



 % Figure environment removed


Furthermore, this inequality allows us to construct a universal phase diagram, spanned by the system parameters, that govern the dominance of stochastic return over the classical instantaneous return. Such a phase diagram is graphically illustrated in \fref{fig5} for the 1d diffusion. The red dashed line, obtained by setting $\mathcal{T}=\langle T_R^{inst} \rangle$, separates the two phases namely stochastic return -`beneficial' and -`detrimental'. The separatrix is the locus of the set of critical resetting rate $z^*$ for each Pe obtained from the above equality (see \cite{SI} for further elaborations). 


 
To gain further insights into the trade-off between two phases, we study $\mathcal{T}$ and discuss some of the limiting cases. For \textit{very low resetting rate} ($z\to0$), one has  $\mathcal{T}=\frac{1}{z}\frac{2 \text{Pe} \left(1-e^{\text{Pe}}\right)  }{\text{Pe} (\text{Pe}+2)-2 e^{\text{Pe}}+2  }\propto 1/z$ whereas $\langle\tau_{inst}(z) \rangle \propto 1/\sqrt{z}$. Clearly, the LHS diverges faster than RHS nullifying the criterion (\ref{srcond}) for any $\text{Pe}$. Indeed, for low resetting rate the particle explores for longer time to diffuse to the target covering a typical 
distance $\sqrt{D/r}>L$. However, the trajectories that disperse far from the origin and take longer time to return also accumulate large overhead time. Such scenarios make the stochastic returns detrimental compared to the instantaneous return. In contrast for $z\gg\mathcal{O}(1)$ - \textit{frequent resets} - and finite $\text{Pe}$, one has $\mathcal{T}=\frac{2 e^{\text{Pe}}-\text{Pe}-2}{\text{Pe}^2}$ making the LHS finite however the RHS diverges as $\propto e^{\sqrt{z}}/z$. In this case, the pre-resetting phase is very short and the particle is effectively in the return phase. Here, it has the possibility to find the target during return phase while the instantaneous return almost always keeps the particle close to the origin eliminating the target-detection. Clearly, this is a favourable situation for the stochastic returns.
 
 
 %When the potential is too steep ($Pe\to\infty)$ i.e. the instantaneous return case, the particle can not probe the target as such. But when $Pe$ is finite, it has a finite chance of reaching the target thus MFPT is lowered a lot compared to instantaneous return and stochastic return renders a win-win situation. 

For the intermediate case of \textit{finite $z$}, there are two limiting cases of $\text{Pe}$. For large $\text{Pe}$, one finds $\mathcal{T}=\text{Pe}\left(\frac{2  \sinh \left(\sqrt{z}\right)}{z^{3/2}}-\frac{1}{z}\right)\propto \text{Pe}$ to be divergent while the RHS is finite. Note that in this limit the return probability to the origin (having return time $\approx \sqrt{D/r}/\lambda$) is almost close to unity and the return phase only adds time penalties. Naturally, instantaneous returns are more efficient. For small $\text{Pe}$, however, the average return time is large (specifically from the trajectories in the $x<0$ region) as can be seen from $\mathcal{T}=\frac{1}{\text{Pe}}\frac{1}{1-\sqrt{z} \text{csch}\left(\sqrt{z}\right)}\propto\frac{1}{\text{Pe}}$ which diverges invalidating the condition (\ref{srcond}). This ensures that instantaneous returns are more beneficial (see Sec S5 and Table 1 in \cite{SI} for a detailed summary). 


The phase-diagram constructed from Eq. (\ref{srcond}) thus allows us to elucidate the effect of stochastic returns in the parameter space both quantitatively and physically. 




\textbf{\emph{Conclusions.---}}
In this letter, we have developed a unified first passage time framework of a stochastic search process under finite time resetting or returns. Notable distinction between this and the existing body of works \cite{evans_diffusion_2011,bodrova2020resetting,pal_search_2020,gupta2021resetting} lies on the fact that the home-returns can be accompanied with stochasticity and thus the searchers can be fortuitous to find targets during the return. Naively, one expects that finite time returns can only incur delay to the overall completion time.
However, we argue that such randomness or element of chance can expedite the overall completion in comparison to the classical instantaneous resetting process which takes no time to return. This is the most intriguing observation of this work to which we attribute various physical scenarios. Elucidating further the scope of such observation to arbitrary stochastic process with generic returns, we derive a universal and physically amenable criterion that unveils the unequivocal superiority of the finite time stochastic returns above the zero-time returns. We 
emphasize that the universal framework of this problem is also a powerful tool to predict the fluctuations and possibly the full distribution. 

We believe that resetting with stochastic returns can turn out to be a universal optimization strategy owing to its dominance over classical first passage resetting with applications to biochemical search \cite{benichou2011intermittent,iyer2016first} and molecular transport \cite{jain2023fick,metzler2014first}. Importantly, our approach is potentially useful for conducting experiments from the practical implementation of resetting especially since one need not drag or track the agents (e.g., colloids or programmable robots) all the way during the return process. Finally, it is intuitive that stochastic return could be energetically optimal while considering the search completion than the deterministic return protocols. These frontiers remain to be explored further in future.






 




 
 
\emph{\textbf{Acknowledgements.}---}  The numerical calculations reported in
this work were carried out on the Nandadevi cluster, which is maintained and supported by the Institute of Mathematical Science’s High-Performance Computing Center. AK acknowledges the support of the core research grant CRG/2021/002455
and the MATRICS grant MTR/2021/000350 from the SERB, DST, Government of India. AK also acknowledges support of the Department of Atomic Energy, Government of India, under Project No. 19P1112RD. AP gratefully acknowledges research support from the Department of Science and Technology, India, SERB Start-up Research Grant Number SRG/2022/000080 and Department of Atomic Energy, Government of India.


\bibliography{fpusr}


\end{document}
