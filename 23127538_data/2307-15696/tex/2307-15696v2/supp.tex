\documentclass[twocolumn, superscriptaddress, amsmath, amssymb, aps, prapplied, floatfix]{revtex4-2}

\usepackage{graphicx}
\usepackage{dcolumn}
\usepackage{bm}
\usepackage{siunitx}
\usepackage{textcomp}
\usepackage{xcolor}
\usepackage{float}
\usepackage{array}
\usepackage{multirow}
%\usepackage{hyperref}% add hypertext capabilities
%\linenumbers\relax % Commence numbering lines
\bibliographystyle{apsrev4-2}
\renewcommand{\appendixname}{APPENDIX}

\newcommand{\ket}[1]{\left| #1 \right>} % for Dirac kets
\newcommand{\bra}[1]{\left< #1 \right|} % for Dirac bras
\newcommand{\ketbra}[2]{{\left|#1\right>\hspace{-3pt}\left<#2\right|}} % for Dirac kets

%for centering fixed width columns
\newcolumntype{L}[1]{>{\raggedright\let\newline\\\arraybackslash\hspace{0pt}}m{#1}}
\newcolumntype{C}[1]{>{\centering\let\newline\\\arraybackslash\hspace{0pt}}m{#1}}
\newcolumntype{R}[1]{>{\raggedleft\let\newline\\\arraybackslash\hspace{0pt}}m{#1}}

\newcommand{\titlename}{Development of a Boston-area 50-km fiber quantum network testbed}

\begin{document}
%%%%%%% Begin Supplemental materials %%%%%%%

\clearpage
\widetext
\begin{center}
\textbf{\large Supplementary Information for ``\titlename''}
\end{center}
%%%%%%% Prefix a "S" to all equations, figures, tables and reset the counter %%%%%%%
\setcounter{equation}{0}
\setcounter{figure}{0}
\setcounter{table}{0}
\setcounter{section}{0}
\setcounter{page}{1}
\makeatletter
\renewcommand{\theequation}{S\arabic{equation}}
\renewcommand{\thefigure}{S\arabic{figure}}
\renewcommand{\thetable}{S\arabic{table}}
\renewcommand{\thesection}{S\arabic{section}}
\renewcommand{\bibnumfmt}[1]{[1]}
\renewcommand{\citenumfont}[1]{#1}
%%%%%%%%%% Prefix a "S" to all equations, figures, tables and reset the counter %%%%%%%%%%

\section{Data Collection Dates}

Table~\ref{tab:dates} provides the dates and times corresponding to $t=0$ for each fiber characterization dataset. All times are given in Eastern Time (ET). Exact times were unavailable for the phase measurements, but are known to have been during daytime work hours.

\begin{table}[h]
    \centering
    \begin{tabular}{ |c|c|c| }
    \hline
    Noise Process & Configuration & Datetime at $t=0$ \\ \hline
    \multirow{2}{*}{Phase} & Differential & 2019-08-28, Daytime \\
     & Round-Trip & 2017-04-06, Daytime \\ \hline
    \multirow{2}{*}{Polarization} & Differential & 2023-02-22, 13:36 \\
     & Round-Trip & 2023-03-03, 18:02 \\ \hline
    \multirow{2}{*}{Optical Path Length} & Differential & 2017-10-30, 12:44 \\
     & Round-Trip & 2016-03-09, 06:00 \\
    \hline
    \end{tabular}
    \caption{Starting dates and times for each characterization dataset, formatted \mbox{''Year-Month-Day, Hours:Minutes''}.}
    \label{tab:dates}
\end{table}

\section{Polarization Drift Model}
We model the fiber-induced polarization drift as resulting from a Brownian process. Here, the two-dimensional nature of polarization results in an {angular} drift rate $\dot{\Theta}$ that follows a Rayleigh probability distribution:

\begin{equation}
\rho_p(\dot{\Theta}) = \frac{\dot{\Theta}}{\sigma_p^2}e^{-\dot{\Theta}^2/2\sigma_p^2},
\end{equation}

which has non-zero mean $\langle\bar{\dot{\Theta}}\rangle=\sigma_p\sqrt{\pi/2}$. A possible mechanism for this drift is stress-induced birefringence, where the stress stems from force applied by the wind. In this case, the standard deviation of the drift $\sigma_p$ would stem from variations in the wind force, which have a standard deviation $\sigma_F$. As the force applied by a fluid to a surface is proportional to the fluid velocity squared, this implies proportionality to the standard deviation of the wind speed squared, $\sigma_F\propto \sigma_W^2$. Finally, $\sigma_W$ is reported to be linearly proportional to the wind speed $W$
%\cite{Joffre_1988}.
[47].
Thus, a model of Brownian motion caused by variations in the wind speed should exhibit a relationship $\langle\bar{\dot{\Theta}}\rangle=\kappa\times W^2$ with a scaling factor $\kappa$, which depends on physical parameters of the system such as the fiber stress responsivity, the area of the fiber bundle facing the wind direction, etc. 

Figure~\ref{fig:pol_psd} shows the power spectral density of the polarization drift rate $\dot{\Theta}$ for the (a) Differential Configuration and (b) Round-Trip Configuration, {along with a 20-dB-per-decade reference line (red dashed), the expected profile for noise caused by a Brownian process.} In both cases, this is taken over the entire time trace of the data, and thus does not capture the differing dynamics observed under different wind speed conditions. {Indeed, while the floor of the noise appears to follow the expected Brownian profile, there are large fluctuations in the spectrum at higher frequencies due to the variation in the spectrum across the time trace.} To explore the time-varying nature of this power spectral density, we also plot spectrograms for the (c) Differential and (d) Round-Trip Configurations, where in each case the signal has been binned into 10-minute windows. {Note that this binning limits the low-frequency resolution of the spectrograms to $\sim2$~mHz.}

% Figure environment removed

% Figure environment removed
% Figure environment removed
 {
\section{Power Spectral Densities}
In Figures~\ref{fig:phase_psd}--\ref{fig:time_psd} we show the power spectral densities of the phase and time-of-flight drifts, respectively. These plots also provide a 20-dB-per-decade reference line (red dashed), the expected profile for noise caused by a Brownian process. The good agreement between this model and our spectra indicates that our drifts are well-described by primarily Brownian processes.}

\section{Transmitter and Receiver Sequences}

The experimental sequences at the transmitter (Tx) and receiver (Rx) are controlled each by a local arbitrary waveform generator (HD-AWG, Zurich Instruments). Figures~\ref{fig:sequence_tx} and Figures~\ref{fig:sequence_rx} depict the logic flowchart followed by each of these sequencers.

% Figure environment removed

% Figure environment removed

\bibliography{references}
\noindent [47] S. M. Joffre and T. Laurila, Standard deviations of wind speed and direction from observations over a smooth surface, Journal of Applied Meteorology and Climatology 27, 550 (1988).
\end{document}