% ----- new thm environments ------
\newtheorem{theorem}{Theorem}
\newtheorem{corollary}[theorem]{Corollary}
\newtheorem{lemma}[theorem]{Lemma}
\newtheorem{proposition}[theorem]{Proposition}
\newtheorem{definition}[theorem]{Definition}
%\newtheorem{observation}[theorem]{Observation}
%\newtheorem{problem}[theorem]{Problem}


%%%================================================
%%%==============  new commands  =======================
%%%================================================
%------- smaller space in,e.g., $\exp\left(x\right)$
% \let\originalleft\left
% \let\originalright\right
% \renewcommand{\left}{\mathopen{}\mathclose\bgroup\originalleft}
% \renewcommand{\right}{\aftergroup\egroup\originalright}
\newcommand{\myleft}{\mathopen{}\mathclose\bgroup\left}
\newcommand{\myright}{\aftergroup\egroup\right}
%%% ------ rm ---------
\newcommand{\e}{\ensuremath\mathrm{e}}
\renewcommand{\i}{\ensuremath\mathrm{i}}
\newcommand{\id}{\ensuremath\mathrm{id}}
\newcommand{\rmd}{\ensuremath\mathrm{d}}
\DeclareMathOperator{\Tr}{Tr}
\renewcommand{\Re}{\operatorname{Re}}
\renewcommand{\Im}{\operatorname{Im}}
\DeclareMathOperator{\ran}{ran}
\DeclareMathOperator{\rank}{rank}
\DeclareMathOperator{\vecmap}{vec}
\DeclareMathOperator{\Id}{Id}
\DeclareMathOperator*{\argmin}{arg\,min}
\DeclareMathOperator{\supp}{supp}
\DeclareMathOperator{\diag}{diag}
\DeclareMathOperator{\spec}{spec}
\DeclareMathOperator{\dist}{dist}
\newcommand{\fro}{\mathrm{F}}
\newcommand{\conv}{\operatorname{conv}}
\DeclareMathOperator{\cone}{\operatorname{cone}}
\DeclareMathOperator{\lin}{\operatorname{lin}}
\DeclareMathOperator{\Comm}{Comm}
\DeclareMathOperator{\End}{End}
%\DeclareMathOperator{\End}{L}
\DeclareMathOperator{\Lin}{L}

\DeclareMathOperator{\Pos}{Pos}
\DeclareMathOperator{\CPT}{CPT}
\DeclareMathOperator{\CP}{CP}
\DeclareMathOperator{\Herm}{Herm}
\DeclareMathOperator{\HT}{HT}%hermiticity and trace preserving
\DeclareMathOperator{\TP}{TP}
%\DeclareMathOperator{\HP}{HP}
\newcommand{\Sym}{\mathrm{Sym}}
\DeclareMathOperator{\sign}{sign}

\DeclareMathOperator{\Ad}{Ad}% Adjoint rep.: Ad_U[X] = U X U^\dagger

\DeclareMathOperator{\U}{U}%unitary group U(n)
\DeclareMathOperator{\Cl}{Cl}% Clifford group Cl(n) \subset U(n)
\renewcommand{\O}{\operatorname{O}}%orthogonal group U(n)
\DeclareMathOperator{\SU}{SU}%special unitary group SU(n)

\renewcommand{\fro}{2}
\newcommand{\utpn}{\text{\rm u,tp,0}}

\DeclareMathOperator\tr{Tr}
\DeclareMathOperator\Cliff{Cl}
\DeclareMathOperator\GL{GL}

%%% -------- Landau symbols ---------
\DeclareMathOperator{\LandauO}{\mathrm{O}}
\DeclareMathOperator{\tLandauO}{\tilde{\mathrm{O}}}
\DeclareMathOperator{\LandauOmega}{\Omega}

\DeclareMathOperator{\gend}{g_{\mathrm{end}}}


% %%% ------- complexity -----------
% \makeatletter
% %from complexity.sty
% \newcommand\complexity@possiblymakesmaller[1]{#1} % default: do nothing
% \newcommand\complexity@fontcommand{\mathsf} % default: do noting
% \newcommand{\ComplexityFont}[1]{%
% {\ensuremath{\complexity@possiblymakesmaller{\complexity@fontcommand{#1}}}}%extra {} makes everyone happy.
% }
% \makeatother
% \newcommand{\NP}{\ComplexityFont{NP}}
% \newcommand{\sharpP}{\#\ComplexityFont{P}}


%%% ------ mathbb --------
\newcommand{\CC}{\mathbb{C}}
\newcommand{\RR}{\mathbb{R}}
\newcommand{\KK}{\mathbb{K}}% for \RR or \CC
\newcommand{\QQ}{\mathbb{Q}}
\newcommand{\ZZ}{\mathbb{Z}}
\newcommand{\NN}{\mathbb{N}}
\newcommand{\FF}{\mathbb{F}}
\newcommand{\1}{\mathds{1}}
\newcommand{\EE}{\mathbb{E}}
\newcommand{\PP}{\mathbb{P}}
\newcommand{\md}[1]{\mathbb{#1}}

%%% ------ mathcal ---------
\newcommand{\mc}[1]{\mathcal{#1}}
\newcommand{\X}{\mc{X}}
\newcommand{\Y}{\mc{Y}}
\newcommand{\V}{\mc{V}}
\newcommand{\W}{\mc{W}}
\newcommand{\F}{\mc{F}}
\newcommand{\K}{\mc{K}}
\newcommand{\mcM}{\mc{M}}
\newcommand{\mcL}{\mc{L}}
\newcommand{\mcR}{\mc{R}}
\renewcommand{\H}{\mc{H}}
\newcommand{\landauO}{\mc{O}}
\newcommand{\DM}{\mc{S}}
\newcommand{\meas}{\mc{M}}
\newcommand{\ct}{{}^\dagger}

%%% ------ other ----------
\newcommand{\kw}[1]{\frac{1}{#1}}
\newcommand{\tkw}[1]{\tfrac{1}{#1}}
\newcommand{\argdot}{{\,\cdot\,}}
%\renewcommand{\vec}[1]{\mathbf{#1}}
\providecommand{\DC}{\operatorname{\mathscr{D}}} %descent cone
\newcommand{\ad}{^\dagger}%symbol for conjugate transpose
\renewcommand{\L}{\operatorname{\mathrm{L}}}%linear operators
\newcommand{\LL}{\operatorname{\mathbb{L}}}%linear maps on operators
\newcommand{\M}{\operatorname{\mathbb{L}}}
% \newcommand{\DM}{\operatorname{\mc{D}}} %density operators
\newcommand{\qqquad}{\qquad\qquad}
\newcommand{\dunion}{\,\dot{\cup}\,}


%%% ------ norms, inner product ----------
\newcommand{\norm}[1]{\left\Vert #1 \right\Vert} %norm with variable height
\newcommand{\normn}[1]{\lVert #1 \rVert} %norm with normalsize height
\newcommand{\normb}[1]{\bigl\Vert #1 \bigr\Vert} %norm with big height
\newcommand{\normB}[1]{\Bigl\Vert #1 \Bigr\Vert} %norm with Big height
\newcommand{\snorm}[1]{\norm{#1}_\infty} %spectral norm  =  (2->2)-norm
\newcommand{\snormn}[1]{\normn{#1}_\infty}
\newcommand{\snormb}[1]{\normb{#1}_\infty}
\newcommand{\snormB}[1]{\normB{#1}_\infty}
\newcommand{\tnorm}[1]{\norm{#1}_{1}} %trace norm
\newcommand{\tnormn}[1]{\normn{#1}_{1}}
\newcommand{\tnormb}[1]{\normb{#1}_{1}}
\newcommand{\TrNorm}[1]{\norm{#1}_{1}} %trace norm
\newcommand{\TrNormn}[1]{\normn{#1}_{1}}
\newcommand{\TrNormb}[1]{\normb{#1}_{1}}
\newcommand{\pNorm}[1]{\norm{#1}_p} %diamond norm
\newcommand{\pNormn}[1]{\normn{#1}_p}
\newcommand{\pNormb}[1]{\normb{#1}_p}
\newcommand{\fnorm}[1]{\norm{#1}_\fro} %Frobenius norm
\newcommand{\fnormn}[1]{\normn{#1}_\fro}
\newcommand{\fnormb}[1]{\normb{#1}_\fro}
\newcommand{\fnormB}[1]{\normB{#1}_\fro}
%\newcommand{\nnorm}[1]{\norm{#1}_1} %nuclear norm
%\newcommand{\nnormn}[1]{\normn{#1}_1}
%\newcommand{\nnormb}[1]{\normb{#1}_1}
\newcommand{\dnorm}[1]{\norm{#1}_\diamond} %diamond norm
\newcommand{\dnormn}[1]{\normn{#1}_\diamond}
\newcommand{\dnormb}[1]{\normb{#1}_\diamond}
\newcommand{\jnorm}[1]{\norm{#1}_\mysquare} % \jnorm{J(M)}=dim(V)*\dnorm{M}
\newcommand{\jnormn}[1]{\normn{#1}_\mysquare}
\newcommand{\jnormb}[1]{\normb{#1}_\mysquare}
\newcommand{\threenorm}[1]{\lvert\lvert\lvert#1\rvert\rvert\rvert} % ||| . |||
\newcommand{\lTwoNorm}[1]{\norm{#1}_{\ell_2}} %\ell_2 norm
\newcommand{\lTwonormn}[1]{\normn{#1}_{\ell_2}}
\newcommand{\lTwonormb}[1]{\normb{#1}_{\ell_2}}
\newcommand{\lqNorm}[1]{\norm{#1}_{\ell_q}} %\ell_q norm
\newcommand{\lqnormn}[1]{\normn{#1}_{\ell_q}}
\newcommand{\lqnormb}[1]{\normb{#1}_{\ell_q}}
\newcommand{\TwoNorm}[1]{\norm{#1}_{2}} % 2 norm
\newcommand{\TwoNormn}[1]{\normn{#1}_{2}}
\newcommand{\TwoNormb}[1]{\normb{#1}_{2}}

%%% ---- Kets -----
\newcommand{\ket}[1]{\left.\left|{#1}\right.\right\rangle}
\newcommand{\ketn}[1]{| #1 \rangle}
\newcommand{\ketb}[1]{\bigl| #1 \bigr\rangle}
\newcommand{\bra}[1]{\left.\left\langle{#1}\right.\right|}
\newcommand{\bran}[1]{\langle #1 |}
\newcommand{\brab}[1]{\bigl\langle #1 \bigr|}
\newcommand{\braket}[2]{\left\langle #1 \middle| #2 \right\rangle}
\newcommand{\braketn}[2]{\langle #1 \middle| #2 \rangle}
\newcommand{\ketbra}[2]{\ket{#1} \!\! \bra{#2}}
\newcommand{\proj}[1]{\ket{#1} \!\! \bra{#1}}
\newcommand{\ketibra}[3]{\ket{#1}_{#2} \!\! \bra{#3}}
\newcommand{\ketbran}[2]{\ketn{#1} \! \bran{#2}}
\newcommand{\ketbrab}[2]{\ketb{#1} \! \brab{#2}}
\newcommand{\sandwich}[3]
  {\left\langle  #1 \right| #2 \left| #3 \right\rangle}
\newcommand{\sandwichb}[3]
  {\bigl\langle  #1 \bigr| #2 \bigl| #3 \bigr\rangle}

\newcommand{\kett}[1]{|{#1}{\rangle\!\rangle}}
\newcommand{\braa}[1]{{\langle\!\langle}{#1}|}


%%% ------ specific to this project ----------
\newcommand{\Var}{\operatorname{Var}}
\renewcommand{\Pr}{\operatorname{\PP}}
\newcommand{\Ev}{\operatorname{\EE}} %expectation value
\newcommand{\A}{\mc{A}} %measurement map
\newcommand{\ev}{e} %error vector
\providecommand{\DC}{\operatorname{\mathscr{D}}} %descent cone
\newcommand{\st}{\mathrm{subject\ to}}
\newcommand{\Haar}{\mathrm{Haar}}
\newcommand{\minimize}{\mathrm{minimize }}
\newcommand{\maximize}{\mathrm{maximize }}
\newcommand{\Span}[1]{\langle #1 \rangle}
\newcommand{\mysquare}{{\protect\scalebox{0.5}{$\square$}}}
\renewcommand{\r}{_{|r}}
\renewcommand{\c}{_{\neg r}}
\newcommand{\rec}{\mathrm{rec}}
\newcommand{\Toff}{\mathrm{Toff}}
\newcommand{\dep}{\mathrm{dep}}
\newcommand{\eps}{\mathrm{eps}}
\newcommand{\utp}{\text{\rm u,tp}}
\newcommand{\Favg}{F_\text{avg}}
\newcommand{\tn}[1]{^{\otimes#1}} % Command for tensor power notation
\newcommand{\dens}[1]{\ket{#1}\!\!\bra{#1}}


% Commands allowing to switch-on AGF abbreviation for average gate fidelity
%\newcommand{\AGF}{average gate fidelity}
\newcommand{\AGF}{AGF}
%\newcommand{\AGFs}{average gate fidelities}
\newcommand{\AGFs}{AGFs}
%Commands for consistent notation of reconstruction errors
\newcommand{\recerror}{\ensuremath\varepsilon_{\text{\rm rec}}}
\newcommand{\recerrorTr}{\ensuremath\varepsilon}
\newcommand{\recerrorF}{\ensuremath\varepsilon_{\text{\rm F}}}


\newcommand{\gr}{\ensuremath{\md{G}}}
\newcommand{\gsum}[1]{\ensuremath{\frac{1}{|\gr |}\!\sum_{#1\in \gr}}}
\newcommand{\Irr}{\ensuremath{\mathrm{Irr}(\gr)}}


\newcounter{example}[section]
\newenvironment{example}[1][]{\refstepcounter{example}\par\medskip
   \noindent \textbf{Example~\theexample. #1} \rmfamily}{\medskip}


%-------------Emilio's new commands-----
\newcommand{\Fsigma}{\mc{F}[\phi](\sigma)}
\newcommand{\wh}[1]{\widehat{#1}}
\newcommand{ \whmcE}{\widehat{\mathcal{E}}}
\newcommand{\dsum}{\frac {1}{d^2} \sum_{\lambda \in \Lambda} d_{\sigma_\lambda} \, }
% \newcommand{\set}[1]{\left\{ #1 \right\}}
\newcommand{\Fnorm}[1]{\normn{#1}_\mathrm{F}}
\newcommand{\nnorm}[1]{\normn{#1}_\ast}
\newcommand{\infnorm}[1]{\normn{#1}_2}
\newcommand{\vecOL}{\lvert \braket{z (\sigma_\lambda) }{ r_{\max} (\sigma_\lambda) } \braket{ \ell_{\max} (\sigma_\lambda)}{z (\sigma_\lambda)} \rvert}
\newcommand{\onevecOL}{\big\lvert 1-\braket{z (\sigma_\lambda) }{ r_{\max} (\sigma_\lambda) } \braket{ \ell_{\max} (\sigma_\lambda)}{z (\sigma_\lambda)} \rvert}
\newcommand{\onevecOLcases}{\big\lvert 1-\braket{z (\sigma_\lambda) }{ r_{\max} (\sigma_\lambda) } \braket{ \ell_{\max} (\sigma_\lambda)}{z (\sigma_\lambda)} \rvert}
%\newcommand{\EE}{\mathbb{E}}
%%% -----------------------------------------
%%% ------ new theorem environments -----
%%% -----------------------------------------
\newtheorem{thm}{Theorem}
\newtheorem{defn}[theorem]{Definition}
\DeclarePairedDelimiter{\floor}{\lfloor}{\rfloor}
\DeclarePairedDelimiter{\ceil}{\lceil}{\rceil}
\DeclarePairedDelimiter{\abs}{\lvert}{\rvert}
\newcommand{\ot}{\otimes}
\allowdisplaybreaks[4]


% ---some Anshu-style notations & other (added by Yaroslav)---

\newtheorem{fact}[theorem]{Fact}
\newtheorem{property}{Property}

\newcommand{\bits}{\{0,1\}} 
\newcommand{\defeq}{\mathrel{\overset{\makebox[0pt]{\mbox{\normalfont\tiny\sffamily def}}}{=}}}


\newcommand{\II}{\mathbb{I}}
\newcommand{\Xf}{X^{\scriptscriptstyle{\rm{(f)}}} }
\newcommand{\Zf}{Z^{\scriptscriptstyle{\rm{(f)}}} }
\newcommand{\tXf}{\tilde{X}^{\scriptscriptstyle{\rm{(f)}}} }
\newcommand{\tZf}{\tilde{Z}^{\scriptscriptstyle{\rm{(f)}}} }
\newcommand{\Hf}{\mathbf{H}^{\scriptscriptstyle{\rm{(f)}}}}

\newcommand{\x}{\mathsf{x}}
\newcommand{\z}{ \mathsf{z}}
\newcommand{\g}{ {\mathsf{g}} }
\renewcommand{\H}{\mathbf{H}}
\renewcommand{\ketbra}[1]{\ket{#1}\bra{#1}}
\newcommand{\fm}[1]{#1^{\mathrm{(f)}}}
\newcommand{\qb}[1]{#1^{\mathrm{(q)}}}
\newcommand{\qbqb}[1]{#1^{\mathrm{(q-q)}}}
\newcommand{\qbfm}[1]{#1^{\mathrm{(q-f)}}}
\newcommand{\fmfm}[1]{#1^{\mathrm{(f-f)}}}
\newcommand{\qf}[1]{#1^{\mathrm{(qf)}}}
\newcommand{\cf}[1]{\mathrm{C4(}#1{\mathrm{)}}}
\newcommand{\bs}{\backslash}


\newenvironment{proof_of}[1]{\noindent {\bf Proof of #1:}
    \hspace*{1mm}}{\hspace*{\fill} $\Box$ }

