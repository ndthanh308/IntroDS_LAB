\documentclass[journal,twoside,web]{IEEEtran}
\makeatletter
\let\NAT@parse\undefined
\makeatother
% \documentclass[journal,twoside,web]{ieeecolor}
\usepackage{tmi}
\usepackage[numbers]{natbib}
\usepackage{cite}
\usepackage{amsmath,amssymb,amsfonts}
\usepackage{layouts}
\usepackage{algorithmic}
\usepackage{graphicx, caption}
\usepackage{bm}
\usepackage{textcomp}
\usepackage{diagbox}
\usepackage{makecell}
\usepackage{hyperref}
% \usepackage[english]{babel}
\usepackage{multirow}
\usepackage{hhline}
\usepackage{booktabs}
\usepackage{array}
\newcolumntype{?}{!{\vrule width 1.25pt}}

\def\BibTeX{{\rm B\kern-.05em{\sc i\kern-.025em b}\kern-.08em
    T\kern-.1667em\lower.7ex\hbox{E}\kern-.125emX}}
% (JB changed for arxiv) 
% \markboth{\journalname, VOL. XX, NO. XX, XXXX 2020}
% (JB changed for arxiv)
% {Author \MakeLowercase{\textit{et al.}}: Preparation of Papers for IEEE TRANSACTIONS ON MEDICAL IMAGING}

\begin{document}
\title{Recovering high-quality FODs from a reduced number of diffusion-weighted images using a model-driven deep learning architecture}
\author{Joseph Bartlett, Catherine E. Davey, Leigh A. Johnston, \IEEEmembership{Senior Member, IEEE}, and Jinming Duan
% \thanks{This paragraph of the first footnote will contain the date on which
% you submitted your paper for review. It will also contain support information,
% including sponsor and financial support acknowledgment. For example, 
% ``This work was supported in part by the U.S. Department of Commerce under Grant BS123456.'' }
\thanks{J. Bartlett, and J. Duan are with the School of Computer Science, the University of Birmingham, Birmingham, UK.}
\thanks{J. Duan is with the Alan Turing Institute, London, UK.}
\thanks{J. Bartlett, C. E. Davey and L. A. Johnston are with the Department of Biomedical Engineering, the Melbourne Brain Centre Imaging Unit and the Graeme Clark Institute, the University of Melbourne, Melbourne, Australia.}
\thanks{The corresponding author is J. Duan (j.duan@bham.ac.uk).}}
\maketitle

\begin{abstract}
Fibre orientation distribution (FOD) reconstruction using deep learning has the potential to produce accurate FODs from a reduced number of diffusion-weighted images (DWIs), decreasing total imaging time. Diffusion acquisition invariant representations of the DWI signals are typically used as input to these methods to ensure that they can be applied flexibly to data with different b-vectors and b-values; however, this means the network cannot condition its output directly on the DWI signal. In this work, we propose a spherical deconvolution network, a model-driven deep learning FOD reconstruction architecture, that ensures intermediate and output FODs produced by the network are consistent with the input DWI signals. Furthermore, we implement a fixel classification penalty within our loss function, encouraging the network to produce FODs that can subsequently be segmented into the correct number of fixels and improve downstream fixel-based analysis. Our results show that the model-based deep learning architecture achieves competitive performance compared to a state-of-the-art FOD super-resolution network, FOD-Net. Moreover, we show that the fixel classification penalty can be tuned to offer improved performance with respect to metrics that rely on accurately segmented of FODs. Our code is publicly available at \href{https://github.com/Jbartlett6/SDNet}{https://github.com/Jbartlett6/SDNet}.
\end{abstract}

\begin{IEEEkeywords}
Diffusion MRI, model-based deep learning, FOD reconstruction
\end{IEEEkeywords}

\section{Introduction}
\label{sec:introduction}
\section{Introduction}
%the Introduction section need to be improved (writting & arguments, and reduce non-sense words)
Air quality forecasting using data-driven models has gained significant attention in recent years, thanks to the proliferation of data collection infrastructures such as sensor stations and advancements of telecommunication technologies. These infrastructures are typically managed by national institutes (e.g., AirParif\footnote{https://www.airparif.asso.fr/}, EPA\footnote{https://www.epa.gov/air-quality}) or large companies (e.g., PurpleAir\footnote{https://www2.purpleair.com/}) that specialize in air quality monitoring or forecasting services and products. Leveraging existing data collection infrastructures proves beneficial for initial research exploration or validating product prototypes.
However, reliance on fixed infrastructures presents practical constraints when customization is required for specific tasks. For instance, certain monitoring areas may be inadequately covered or completely absent from the existing infrastructures, or the density of coverage may not be sufficient. This issue particularly affects small or mid-sized industrial and academic players who face budget limitations that prevent them from investing in their own infrastructure from scratch, but have specific customization needs.

% give the motivation from another aspect: incrementally built infrastructure, no need to re-train the model
In addition to data collection, air quality forecasting models trained solely with data from public fixed infrastructures may not perform well for users' specific scenarios, such as forecasting at a higher spatial resolution. Deploying additional sensors as a cost-effective solution can enrich the data and improve forecasting performance without the need to build infrastructures from scratch. 
Subsequently, this targeted solution leads us to consider the practical question: \textit{how we can make use of the data collected from existing infrastructures, when integrating new sensor infrastructures?} 
%which can be equipped on fixed sensor stations or moving objects (e.g., drones) with a higher flexibility.

% Figure environment removed

As depicted in Figure \ref{fig:research_background}, the topological sensor network may change as the urban infrastructure evolves, resulting in varying network structures of air quality sensors. The data collected from the network $G_{\tau}$ needs to be augmented with enriched data from newly installed sensors $\Delta G_{\tau'}$ and $\Delta G_{\tau''}$. Training a model solely on recent data with $G_{\tau''}$ would overlook valuable information contained in the historical data with $G_{\tau}$ and $G_{\tau'}$.

In this paper, we propose an expandable graph attention network (EGAT) that effectively integrates data with various graph structures. This approach is versatile and can be seamlessly embedded into any existing air quality forecasting model. Furthermore, it applies to scenarios where sensors are not installed, enabling accurate forecasting in such areas.
We summarize our approach's main advantages as follows:
\begin{itemize}
    %\item \textbf{Air quality forecasting in real scenarios:} we consider the complex data quality issues, e.g., missing values

    \item \textbf{Less is more:} With fewer installed sensors, we can directly predict the air quality of other unknown area where sensors are not installed and achieve comparable performance to models relying on extensive data collection infrastructures with more sensors.
    \item \textbf{Continual learning with self-adaptation:} The proposed model enables continuous learning from newly collected data with expanded sensor networks, demonstrating self-adaptability to different topological sensor networks.
    \item \textbf{Embeddable module with scalability:} The proposed module can be seamlessly integrated into any air quality forecasting model, enhancing its ability to forecast in real-world scenarios.

\end{itemize}

The rest of this paper starts with a review of the most related work. Then, we formulate the problems of the paper. Later, we present in detail our proposal, which is followed by the experiments on real-life datasets and the conclusion.



\section{Method}
\label{sec:method}
\subsection{Network Architecture}
%Explaining the motivation for spherical deconvolution and how it is formulated. (explanation)
Constrained spherical deconvolution is used to fit FODs to DWI signal by optimising the following objective function:
\begin{equation}    
\label{eq:SDGen}
\mathop {\min\limits_{\bf{c}}} { {\frac{1}{2m}\| {{\cal A} {\cal Q}{\bf{c}} - {\bf{b}}} \|_2^2}  + {\cal R}\left( {\bf{c}} \right)}
\end{equation}

%SDNet architecture figure:
% Figure environment removed


%Specifying what each componenet of the above formula represents and what ehir sizes are.(explanation)
\noindent where ${\bf{c}} \in \mathbb{R}^{n}$ are the SH coefficients of the FOD, ${\bf{b}}\in\mathbb{R}^{m}$ are the DWI signals, and ${\cal AQ} \in\mathbb{R}^{m \times n}$ spherically convolves the FOD with the response functions of the tissue types being modelled. To facilitate a data-driven regularisation term, optimised for FOD reconstruction, we consider an arbitrary regularisation term, ${\cal R}(\cdot)$, in place of the ubiquitous non-negativity constraint. In the following we outline how the variable splitting methods used in \citet{jia2021learning, duan2019vs} can be adapted to solve \eqref{eq:SDGen}.

First, we introduce an auxiliary splitting variable ${\bf{w}} \in \mathbb{R}^{n}$, converting \eqref{eq:SDGen} into the following equivalent form:
\begin{equation}
\label{eq:SDdecouplecon}
\mathop {\min\limits_{{\bf{c,w}}}} { {\frac{1}{2m}\| {{\cal A} {\cal Q}{\bf{c}} - {\bf{b}}} \|_2^2}  + {\cal R}\left( {\bf{w}} \right)} \; s.t. \; {\bf{c=w}}
\end{equation}

Using the penalty function method, we add these constraints back into the model and minimise the joint objective:
\begin{equation}
\label{eq:SDdecouple}
 \mathop { \min\limits_{\bf{c,w}}} { {\frac{1}{2m}\| {{\cal A} {\cal Q}{\bf{c}} - {\bf{b}}} \|_2^2}  + {\cal R}\left( {\bf{w}} \right)} + \frac{\lambda}{2}\|{\bf{c-w}}\|^{2}_{2}
\end{equation}

Eq.~(\ref{eq:SDdecouple}) can  be solved for ${\bf{c}}$ and ${\bf{w}}$ using an alternating optimisation scheme:
\begin{equation}
\label{SDalt}
\left\{ \begin{array}{l}{{\bf{c}}^{k + 1}} = \mathop {\arg \min\limits_{\bf{c}}} \frac{1}{2m} \| {{\cal A} {\cal Q}{\bf{c}} - {\bf{b}}} \|_2^2 + \frac{\lambda }{2} \| {\bf{c-w}}^{k} \|_2^2\\{\bf{w}}^{k + 1} = \mathop {\arg \min\limits_{{\bf{w}}}}  {\frac{\lambda}{2}\| {\bf{c}}^{k+1}-{\bf{w}} \|_2^2}  + {\cal R}\left( {\bf{c}}^{k+1} \right) \\\end{array} \right..
\end{equation}

% Explanation as to how solve these two equations:
The first convex optimisation can be solved using matrix inversion. The second equation is a denoising problem with arbitrary regularisation, the optimal form of which is unknown. In order to learn the regularisation to improve FOD reconstruction performance, the iterative process can be unrolled and the denoising step solved using a neural network, ${\cal{NN}}(\cdot)$:
\begin{equation}
\label{eq:SDunroll}
\left\{ \begin{array}{l}{{\bf{c}}^{k + 1}} = \left(\frac{1}{m} {\cal{Q}^{T}}{\cal{A}^{T}}{\cal{A}}{\cal{Q}} + \lambda {\cal{I}}\right)^{-1}\left(\frac{1}{m}{\cal{Q}^{T}}{\cal{A}^{T}}{\bf{b}} + \lambda {\bf{w}}^{k}\right)\\
{\bf{w}}^{k + 1} = {\cal{NN}}\left({\bf{c}}^{k+1}\right)\end{array} \right..
\end{equation}


 %Network Architecture  (explanation/repeatability). 
The network architecture (Fig.~\ref{fig:SDNet}) takes nine voxels in each spatial dimension for 30 different diffusion gradients, resulting in a $9 \times 9 \times 9 \times 30$ volume of DWI signals as input, and passes them through alternating DWI consistency and deep regularisation blocks. The network outputs a vector ${\hat{\bf{c}}} \in \mathbb{R}^{n}$, a high-fidelity prediction of the FOD from the central voxel of the $9\times9\times9$ input patch. 

 \subsubsection{DWI Consistency}
%DWI consistency:
Each DWI consistency block solves the matrix inversion in (\ref{eq:SDunroll}) independently for each voxel, maintaining spatial resolution. The initial DWI consistency block optimises only for the first three even orders of spherical harmonic coefficients $(l_{max}=4)$ to ensure robustness to aggressive DWI undersampling. 

\subsubsection{Deep Regularisation}
%Deep Regularisation:
Each deep regularisation block is applied to a concatenation of the previous two DWI consistency blocks, meaning the block is conditioned on  both earlier representations. Validation tests showed these connections improve network performance (data not shown). The initial $3 \times 3 \times 3$ convolution kernels are applied with one layer of zero padding in each dimension, as to maintain spatial resolution, and are followed by 3D batch normalisation layers and parametric rectified linear unit (PReLU) activation functions. The number of channels is increased in this manner until it has reached 448 (Fig.~\ref{fig:SDNet}). No padding is applied in the final $3 \times 3 \times 3$ convolution kernel followed by a PReLU function, reducing the resolution in each spatial dimension by two. Finally, a $1 \times 1 \times 1$ convolution kernel is then applied to the 512-channel feature maps to obtain a 94-channel input to a gated linear unit (GLU) activation function,which is the output of the block. Residual connections, referencing the output of the previous DWI consistency block, are used to improve gradient flow through the network. The deep regularisation block reduces each spatial dimension of its input by two.   





\subsection{Loss Functions}
%Classification Loss function Motivation. 
In addition to the customary MSE loss, a fixel classification penalty is proposed to give greater control over the angular separation of the reconstructed FODs. The mechanics of this method can be considered similar to the microstructure sensitive loss proposed for DWI signal reconstruction in \citep{chen2023deep}.
To overcome the inherent non-differentiable nature of the fast marching level set FOD segmentation algorithm \citep{smith2013sift}, a fixel classification network is applied to predict the number of fixels each voxel contains. The output is passed into a cross-entropy component of the loss function. Since we are concerned with the white matter components of the FODs, the loss function and performance metrics are not functions of the grey matter and cerebrospinal fluid components of the FOD. For notational simplicity, from this point onwards ${\bf{c}}$ refers only to the white matter component of the FOD.  The overall loss function is as follows: 
\begin{equation}
{\cal{L}}(\hat{{\bf{c}}}) =\frac{1}{N_{batch}} \sum^{N_{batch}}_{i = 1} \left(\|\hat{{\bf{c}}}_{i} - {\bf{c}}_{i}\|^{2}_{2} + \kappa {\cal{E}}(\hat{{\bf{f}}}(\hat{{\bf{c}}}_{i}),{\bf{f}}_{i})\right)
\label{eq:Loss}
\end{equation}
where $N_{batch}$ is the number of data points in the mini-batch, $\hat{{\bf{c}}}_{i},\;{\bf{c}}_{i}\in\mathbb{R}^{n-2}$ are the reconstructed and fully sampled white matter FODs, ${\cal{E}}(\cdot,\cdot)$ is the cross-entropy, $\hat{{\bf{f}}}(\hat{{\bf{c}}}_{i}), \;{\bf{f}}_{i} \in \mathbb{R}^{5}$ are the predicted logits and the one-hot encoding of the number of fixels respectively and $\kappa$ is a hyperparameter to balance the two components of the loss function.

\begin{table}[h]
\caption{Count and percentage values of fixels in white matter voxels of an individual from the HCP dataset before and after thresholding at 4 fixels. Before thresholding there is a severe class imbalance.}
\label{tab:fixdist}
    \centering
    \begin{tabular}{?c?c?c?c?}
    \specialrule{1.25pt}{0pt}{0pt}
         \makecell{Number \\ of Fixels} & Count & \makecell{Percentage \\ before thresholding} & \makecell{Percentage \\ after thresholding} \\ \specialrule{1.25pt}{0pt}{0pt}
         1 & 310994 & 49\% & 49\% \\
         2 & 200673 & 32\% & 32\% \\
         3 & 76672 & 12\% & 12\% \\
         4 & 24095 & 4\% & 6.7\%\\
         5 & 10975 & 2\% & - \\
         6 & 4979 & 0.8\% & - \\
         7 & 1800 & 0.3\% & - \\
         \specialrule{1.25pt}{0pt}{0pt}
    \end{tabular}
\vspace{-0pt} % \vspace{-10pt} (JB) for arxiv
\end{table}


%One example of downstream analysis of FODs is fixel based analysis, which relies on segmenting the FOD into individual fibres. We propose by using the ground truth fixel maps as supervision signal, we can learn FOD reconstructions which predict the correct number of fibres. 

%Explaining why we thresholded fixel value
 When training the fixel classification network, the number of fixels in each voxel were thresholded to four (Tab.~\ref{tab:fixdist}), reducing the inclusion of spurious peaks and class imbalance. A simple, fully-connected architecture was used, with layers containing $\{$45, 1000, 800, 600, 400, 200, 100, 5$\}$ neurons. Between each layer there are ReLU activation and 1D batch normalisation functions, other than between the penultimate and final layer where the batch normalisation is omitted. A softmax activation function, followed by cross-entropy loss, were then applied to the output of the network. The classification network was trained using the same training set as SDNet. Fully sampled FODs were used as the input, and the ground truth targets were calculated using the fast level set marching algorithm \citep{smith2013sift}. 
 
 % Figure environment removed

\subsection{Implementation Details}
%Training the SDNEt 
To demonstrate the impact of the fixel classification penalty, experiments were carried out with $\kappa = 0$ and $ \kappa = 1.6 \times 10^{-4}$. The ADAM optimiser \citep{kingma2014adam}, with learning rate warm-up, was used for parameter optimisation, with an initial learning rate of $10^{-6}$, increasing to $10^{-4}$ after $10,000$ iterations. To minimise hyperparameter tuning, $\lambda$ was optimised simultaneously with the network weights. From validation experiments (data not included), we found that the most effective way to utilise the classification loss to train SDNet was to initially train the model with $\kappa=0$ and then to increase $\kappa$ to its final value after this initial training stage. To do so we trained SDNet with only MSE loss until convergence, then trained the network until convergence with $ \kappa = 1.6 \times 10^{-4}$. 



\section{Experiments}
\label{sec:experiments}
\subsection{Dataset}
A subset of the WU-Minn Human Connectome Project (HCP) dataset \citep{van2013wu}, consisting of 30 subjects, was split $20/3/7$ and used for training, validation, and testing, respectively. The HCP images have $1.25\text{mm}$ isotropic resolution with 90 gradient directions for $b = 1000, 2000 \text{ and } 3000 \text{ s/mm}^{2}$ and 18 $b_{0}$ images. The HCP dataset was minimally pre-processed in accordance with \citep{sotiropoulos2013advances}. 

%Data Preprocessing:
Additionally, prior to applying SDNet, each subject's data was normalised using MRtrix3's \textit{dwinormalise} function. The fully sampled FODs were fit to all 288 DWIs; first, the response functions were calculated using the method proposed in \citep{dhollander2019improved}, then the FODs calculated using MSMT-CSD \citep{jeurissen2014multi}. White matter response functions and FODs were modelled with $l_{max} = 8$ and the grey matter and cerebrospinal fluid component response functions and FODs were modelled with $l_{max} = 0$, resulting in a total of 47 SH coefficients. 

%HCP intro for how it is specifcally used in this study.
The sampling pattern \citet{caruyer2013design} utilised in the HCP is such that for any $k$, selection of the first $k$ DWI volumes results in evenly spread b-vectors. To prepare the input data, the first 9 DWIs from each non-zero shell were selected with an additional 3 $b_{0}$ images, resulting in a total of 30 DWI signals. 

% Figure environment removed

%How patches were selected for training.
Only patches in which the central voxel is classified as grey matter or white matter are used for training. The grey matter voxels were included to improve performance near the boundary of the two tissue types, as highlighted in \citep{zeng2022fod}. The grey and white matter masks were calculated using the method outlined in \citep{zhang2001segmentation}, which is implemented using the FSL software package \citep{jenkinson2012fsl}. 

From this point onwards, for notational convenience, SDNet ($\kappa = 0$) will be referred to as SDNet and SDNet ($\kappa = 1.6\times 10^{-4}$) will be referred to as ${\text{SDNet}}_{\kappa}$. To evaluate the performance of the introduced methods, SDNet, ${\text{SDNet}}_{\kappa}$, FOD-Net \citep{zeng2022fod}, and super-resolved MSMT CSD, referred to as MSMT CSD for notational simplicity, were all compared. In the original implementation, FOD-Net maps FODs fit using the single shell three tissue CSD algorithm \citep{dhollander2019improved} to 32 DWIs (4 $b_{0} \text{ and } 28 \text{ } b = 1000/2000/3000 \text{ s/mm}^{2}$) to the desired MSMT CSD obtained FODs. To allow a fair comparison between FOD-Net and the proposed networks, FOD-Net was trained using the same training set as SDNet. Since the final block in the SDNet architecture is a DWI consistency block, it cannot map to normalised FODs, therefore the target training data is not normalised. It should be noted that the normalisation can still be performed as a post-processing step. Otherwise, the same configuration settings found in the Github repository released by the FOD-Net authors were used.

\subsection{Performance Metrics}

%Performance Metrics;
%Explain how we aggregated each of the perfromance metrics into one score - including how the white matter masks were obtained.
To evaluate the performance of the FOD reconstruction algorithms, performance metrics were calculated voxel-wise then averaged over regions of interest.  The regions considered were the white matter and intersections of individual tracts within the white matter. The tracts considered were: the corpus callosum (CC), the middle cerebellar peduncle (MCP), the corticospinal tract (CST), and the superior longitudinal fascicle (SLF). To understand how the algorithm performs in voxels containing different numbers of fibres, we considered the intersections of these tracts as in \citep{zeng2022fod}. For voxels containing a single fibre, we considered voxels in the CC containing a single fixel, which we refer to as ROI-1-CC. For two crossing fibres, we considered voxels in the intersection of the MCP and CST containing two fixels, which we refer to as ROI-2-MCP. For three crossing fibres, we considered voxels in the intersection of the SLF, CST and CC containing three fixels, which we refer to as ROI-3-SLF. The white matter mask was calculated using the FSL five tissue type segmentation algorithm in MRtrix3. The segmentation masks for the white matter fibre tracts were obtained using TractSeg \citep{wasserthal2018tractseg}. 

%The MSE Performance metric.
The SSE between the reconstructed FODs, $\hat{{\bf{c}}}$, and the fully sampled FODs, ${\bf{c}}$, was computed as follows:  
\begin{equation}
\label{eq:SSE}
\text{SSE}\left({\bf{c}}, \hat{{\bf{c}}}\right) = \left\|{\bf{c}} - \hat{{\bf{c}}}\right\|^{2}_{2}
\end{equation}

%The ACC performance metric
The angular correlation coefficient (ACC) \citep{anderson2005measurement} was computed as follows:

\begin{equation}
\text{ACC}({\bf{c}}, \hat{{\bf{c}}}) = 
\frac{\sum\limits^{4}_{i=1}\sum\limits_{j=-2i}^{2i}{\bf{c}}_{2i,j}\hat{{\bf{c}}}_{2i,j}}{\sqrt{\left(\sum\limits^{4}_{i=1}\sum\limits^{2i}_{j=-2i}{\bf{c}}_{2i,j}^{2}\right)
\left(\sum\limits^{4}_{i=1}\sum\limits^{2i}_{j=-2i}\hat{{\bf{c}}}_{2i,j}^{2}\right)}}
\end{equation}


%fixel based analysis explanation and introduction. 
We refer to SSE and ACC as \textit{FOD-based performance} metrics, since they compare the SH representation of the FODs prior to any further processing. 

Fixel-based analysis requires each FOD to be segmented into fixels, each of which has associated apparent fibre density and peak amplitude \citep{smith2013sift}. To calculate the associated error metrics, peak amplitude and apparent fibre density vectors must be assembled. Each vector consists of the respective scalar for each fixel ordered according to the peak amplitude and are padded to a fixed length. The remaining metrics are referred to as \textit{fixel-based performance} metrics since they require the FOD to be segmented into fixels prior to evaluation. 

%Introduce the fixel number accuracy metric. 
Fixel accuracy was defined for a region of interest as the proportion of voxels in which the FOD is segmented into the correct number of fixels.  

%The PAE:
The peak amplitude error (PAE) was calculated between the reconstructed, $\hat{{\boldsymbol{f}}}^{P}$, and fully sampled FOD's, ${\boldsymbol{f}}^{P}$, peak amplitude vectors:
\begin{equation}
\label{eq:PAE}
\text{PAE}\left({\boldsymbol{f}}^{P}, \hat{{\boldsymbol{f}}}^{P}\right) = \sum\limits_{i}\left|f^{P}_{i}-\hat{f}^{P}_{i}\right|
\end{equation}

%Apparent fibre density explanation and definition 
The apparent fibre density error (AFDE) was calculated between the reconstructed, $\hat{{\boldsymbol{f}}}^{A}$, and fully sampled FOD's, ${\boldsymbol{f}}^{A}$, apparent fibre density vectors:
\begin{equation}
\label{eq:AFDE}
\text{AFDE}\left({\boldsymbol{f}}^{A}, \hat{{\boldsymbol{f}}}^{A}\right) = \sum\limits_{i}\left|f^{A}_{i}-\hat{f}^{A}_{i}\right|
\end{equation}
% %Mean Angular Error:
% We also considered the mean angular error for the fixels calculated in each voxel. To do so we considered the unit vector which characterises the direction of each of the fixels in a voxel. We then calcuate the angular difference between that and the direction of the corresponding ground truth vector:
% \[MAE = \frac{1}{n}\Sigma^{n}_{i=0}\hat{f}^{angle}\cdot f^{angle}_{gt}\]

%Include a note on how tractography was carried out.

%Which methods we compared: (MSMT CSD, FOD-Net, SDNet (w\ and w\o fixel classification loss))
%FOD-Net



%HCP dataset introduction - should also include that the 3 of the 4 test images are the images containing anomalies.


\subsection{Ablation Study}
To investigate the impact of the DWI consistency block on the performance of the network, an ablation study was conducted. The network was trained without the DWI consistency blocks, and all other aspects of the architecture and network training remained the same. We compared this model to SDNet with the DWI consistency blocks included. 



%Implementation: Include Hardware and software spec here.
%Any other experimental conditions we compared the methods under (e.g. fewer DWIs etc). May turn out to be better to only considered the SDNEt results in this case

% Figure environment removed

%Ablation Study.
\subsection{Statistical Analysis}
Shapiro-Wilk tests for normality ($\alpha=0.05$) were applied for each performance metric and method; unless otherwise stated there is insufficient evidence to reject the null hypothesis that the groups are normally distributed.

Since the data was normally distributed, and each method was applied to the same set of test subjects, a repeated measures one-way ANOVA ($\alpha=0.05$) was applied to each performance metric to determine whether there was a main effect between the conditions. 
% bonferoni corrected t-tests
Finally, to determine which methods contributed to the main effect, post-hoc t-tests with Bonferroni correction (adjusted for $\alpha = 0.05$) were used to identify effects between the FOD reconstruction algorithms. 








\section{Results}
\label{sec:results}
%Comment on the qualitative FOD results (i.e. images):
%SDNet architecture figure:

%Discuss qualitative results


\subsection{Qualitative Results}
The qualitative results comparing all methods (Fig.~\ref{fig:FODQual}) show that the deep learning methods reconstructed FODs that more closely resembled the ground truth when compared to MSMT CSD. The primary difference is the presence of spurious peaks produced by MSMT CSD, whereas the deep learning based algorithms coherently captured the major tracts in this region due to their denoising effect. 

The highlighted region in Fig.~\ref{fig:FODQual} (panels \textbf{f.}-\textbf{j.}) shows an area where FOD-Net produced distorted FODs compared to SDNet and ${\text{SDNet}}_{\kappa}$. MSMT CSD reconstructed particularly noisy FODs in this area, which the results obtained by FOD-Net resembled some similarities to. The FODs produced by SDNet underestimated the amplitude in this region but more accurately distinguished between fibre populations and captured their directions. In this region, which contains dominant fibre populations with large angular separation, the impact of increasing $\kappa$ on the reconstructed FODs is minimal; only a small change in the direction of the fibres is observed. In the larger tracts in panels Fig.~\ref{fig:FODQual} \textbf{a.}-\textbf{e.}, such as the green fibre population going upwards in the bottom left corner, all deep learning methods performed similarly. 

The qualitative results comparing SDNet with ${\text{SDNet}}_{\kappa}$ (Fig.~\ref{fig:FixQual}) illustrate that ${\text{SDNet}}_{\kappa}$ better separated fibre populations. The presence of fibre populations going from the lower left to upper right of panels Fig.~\ref{fig:FixQual} \textbf{d.}-\textbf{f.} are separated from the larger fibre population by ${\text{SDNet}}_{\kappa}$ but not by SDNet without the fixel classification penalty. The FODs reconstructed in the broader region, captured in panels Fig.~\ref{fig:FixQual} \textbf{a.}-\textbf{c.}, show that  larger fibre populations are reconstructed similarly for both SDNet and  ${\text{SDNet}}_{\kappa}$.

\subsection{FOD-based Results}
% Figure environment removed

%  % Figure environment removed

% The SSE and fixel number error maps for a single axial slice (Fig.~\ref{fig:ERRmaps}) show that large errors in the three deep learning based methods are larger in the same locations suggesting that it is the same FODs which all methods find difficult to reconstruct. 
The SSE error maps (Fig.~\ref{fig:ERRmaps}) show that lower SSE is achieved throughout the brain by all deep learning methods compared to MSMT CSD. SDNet generally achieved smaller errors than the other deep learning methods. This is particularly evident in, but not restricted to, the areas highlighted by the red arrows. The error maps produced by ${\text{SDNet}}_{\kappa}$ and FOD-Net are similar.

 The average FOD-based performance results (Fig.~\ref{fig:FODperf} and Tab.~\ref{tab:QuantRes}) show that SDNet reconstructed FODs with significantly lower SSE and higher ACC than the compared methods in all regions of interest considered. The training curves (Fig.~\ref{fig:TrainCurve}) show that increasing $\kappa$ caused the validation ACC to decrease over the validation set. 

 In the white matter voxels, SDNet achieved the lowest SSE by a statistically significant margin over all compared methods, followed by ${\text{SDNet}}_{\kappa}$ and FOD-Net, between which there was no statistically significant difference in SSE. SDNet also achieved the strongest ACC performance in the white matter, where it improved over all other methods by a statistically significant margin. There was no statistically significant difference between ${\text{SDNet}}_{\kappa}$ and FOD-Net with respect to ACC in the white matter. 

%I think this is interesting since it shows that the even the FOD-based performance of SDNet with classification loss improves relatively.
In all of ROI-1-CC, ROI-2-MCP, and ROI-3-SLF, SDNet achieved the strongest SSE and ACC results (Fig.~\ref{fig:FODperf} and Tab.~\ref{tab:QuantRes}) by a statistically significant margin. FOD-Net and ${\text{SDNet}}_{\kappa}$ showed no statistically significant differences with respect to SSE and ACC in ROI-1-CC and ROI-2-MCP but in ROI-3-SLF ${\text{SDNet}}_{\kappa}$ achieved a statistically significant improvement over FOD-Net with respect to both SSE and ACC. In all regions, all deep learning based FOD reconstruction methods outperformed MSMT CSD with respect to SSE and ACC by a statistically significant margin. 

 %The statistical analysis for these regions of interest can be found in the supplementary material.
 
 %With respect to SSE in ROI-3-SLF SDNet with classification loss $0.015$, FOD-Net $0.017$. The ANOVA test showed a significant main effect (p = $6.88e-17$) over all 4 methods, and the Bonferroni adjusted t-test identifies a significant difference between SDNet w/ classification loss and FOD-Net (p = $1.27e-4$). Similarly with respect to ACC in ROI-3-SLF SDNet with classification loss achieve $94.233$ and outperformed FOD-Net $93.291$ The ANOVA test showed a significant main effect (p = $8.44e-19$) over all 4 methods and the Bonferroni adjusted t-tests show a significant difference between SDNet w/ classification loss and FOD-Net (p = $8.43e-19$).  


\subsection{Fixel-based Results}
 The fixel-based performance results (Fig.~\ref{fig:Fixperf} and Tab.~\ref{tab:QuantRes}) show greater variation between regions and an increased dependence on $\kappa$. The training curves (Fig.~\ref{fig:TrainCurve}) show that increasing $\kappa$ caused the validation fixel accuracy to increase over the validation set. In the white matter, ${\text{SDNet}}_{\kappa}$ achieved the strongest fixel accuracy by a significant margin, followed by SDNet and FOD-Net between which there was no statistically significant difference. 



% \begin{table*}
%   \resizebox{1\textwidth}{!}{
%   \begin{tabular}{|l||l|l|l|l||l|l|l|l||l|l|l|l||}
%     \hline
%       &
%       \multicolumn{4}{c||}{Fixel Accuracy} &
%       \multicolumn{4}{c||}{AFDE} &
%       \multicolumn{4}{c||}{PAE} \\
%       \hline
%     \diagbox[innerwidth=6cm]{Method}{Region} & WM & ROI-1-CC & ROI-2-MCP & ROI-3-SLF& WM & ROI-1-CC & ROI-2-MCP & ROI-3-SLF& WM & ROI-1-CC & ROI-2-MCP & ROI-3-SLF\\
%     \hline
%     SDNet & 2.1\% & 2.1\% & 2.1\% & 2.1\% & 2.1\% & 2.1\% \\
%     \hline
%     ${\text{SDNet}}_{\kappa}$ (SDNet p-value)& 11.6\% & 11.6\% & 11.6\% & 11.6\% & 11.6\% & 11.6\% \\
%     \hline
%     FOD-Net (SDNet p-value) (${\text{SDNet}}_{\kappa}$ p-value) & 5.5\% & 5.5\% & 5.5\% & 5.5\% & 5.5\% & 5.5\% \\
%     \hline
%     MSMT CSD (SDNet p-value) (${\text{SDNet}}_{\kappa}$ p-value) & \\
%     \hline
%   \end{tabular}}
%   \label{tab:FODresults}
% \end{table*}

%ROI-1-CC

In ROI-1-CC, ROI-2-MCP, and ROI-3-SLF, we see that the fixel accuracy of the deep learning FOD reconstruction methods decreased as the number of fixels increased. In ROI-1-CC, SDNet achieved the strongest performance by a statistically significant margin, followed by FOD-Net and ${\text{SDNet}}_{\kappa}$, between which there is no statistically significant difference in fixel accuracy in the same region.

%ROIs 2 and 3
As the number of fixels in the ROIs increased, the fixel accuracy of ${\text{SDNet}}_{\kappa}$ increased relative to other methods. In ROI-2-MCP, ${\text{SDNet}}_{\kappa}$ achieved the highest fixel accuracy but not by a statistically significant margin over FOD-Net. Both methods outperformed SDNet by a statistically significant margin.  In ROI-3-SLF this pattern continued as ${\text{SDNet}}_{\kappa}$'s performance further improved, and it achieved a statistically significant fixel accuracy increase over the other deep learning methods. There was no statistically significant difference in fixel accuracy between FOD-Net and SDNet in ROI-3-SLF. In all regions other than ROI-3-SLF, MSMT performed worse than all other methods by a statistically significant margin. 

%Group the AFDE and PAE sections together:
For AFDE in the white matter, ${\text{SDNet}}_{\kappa}$ achieved the lowest error by a statistically significant margin, followed by FOD-Net and SDNet between which there is no statistically significant difference in AFDE in the white matter. For PAE in the white matter, ${\text{SDNet}}_{\kappa}$ achieved the lowest error, which was a statistically significant improvement over SDNet but not FOD-Net. For both AFDE and PAE in the white matter, MSMT CSD achieved a higher error than all compared methods by a statistically significant margin.

\begin{table*}
\caption{Quantitative results for all performance metrics, methods and regions of interest. The arrows below each metric represent the direction of improved performance. The p-values for the paired t-test results between the respective methods and SDNet and $\text{SDNet}_{\kappa}$ are indicated by $\text{p}_{SD}$ and $\text{p}_{SD \kappa}$, respectively. The performance metric of the strongest method in each row is bold, significant p-values (Bonferroni corrected, adjusted for $\alpha=0.05$) are also bold.}
\label{tab:QuantRes}
  \centering
  \resizebox{0.9\textwidth}{!}{%
  \begin{tabular}{?c?l?llll?}
    \specialrule{1.25pt}{0pt}{0pt}
    \textbf{Metric} &\diagbox[innerwidth=1.5cm]{\textbf{Region}}{\textbf{Method}}& 
    \textbf{SDNet}&
    \textbf{SDNet\textsubscript{\boldmath$\kappa$} (p\textsubscript{SD})}&
    \textbf{FOD-Net (p\textsubscript{SD}, p\textsubscript{SD\boldmath${\kappa}$})} & \textbf{MSMT CSD (p\textsubscript{SD}, {p}\textsubscript{SD\boldmath$\kappa$})}\\
    \specialrule{1.1pt}{0pt}{0pt}
    \multirow{4}{*}{\textbf{\shortstack{SSE\\($\downarrow$)}}} & 
    White Matter &
    \textbf{0.011$\pm$0.001} &
    0.012$\pm$0.001 ($<$\textbf{0.001})&
    0.013$\pm$0.001 ($<$\textbf{0.001}, 0.034)& 
    0.041$\pm$0.002 ($<$\textbf{0.001}, $<$\textbf{0.001}) \\
    &
    ROI-1-CC&
    \textbf{0.007$\pm$0.001}&
    0.008$\pm$0.001 ($<$\textbf{0.001}) &
    0.008$\pm$0.001 ($<$\textbf{0.001}, 0.295) &
    0.028$\pm$0.002 ($<$\textbf{0.001}, $<$\textbf{0.001}) \\
    &
    ROI-2-MCP &
     \textbf{0.016$\pm$0.001}&
    0.018$\pm$0.001 ($<$\textbf{0.001}) &
    0.017$\pm$0.001 (\textbf{0.001}, 0.175)&
    0.045$\pm$0.002 ($<$\textbf{0.001}, $<$\textbf{0.001})\\
    
    &
    ROI-3-SLF & 
    \textbf{0.014$\pm$0.001}&
    0.015$\pm$0.001 (\textbf{0.005})& 
    0.017$\pm$0.001 ($<$\textbf{0.001}, $<$\textbf{0.001})& 
    0.063$\pm$0.002 ($<$\textbf{0.001}, $<$\textbf{0.003})\\
    
    \specialrule{1.25pt}{0pt}{0pt}
    \multirow{4}{*}{\textbf{\shortstack{ACC\\($\uparrow$)}}} &
    White Matter &
    \textbf{92.209$\pm$0.003}&
    91.152$\pm$0.003 ($<$\textbf{0.001}) &
    91.184$\pm$0.003 ($<$\textbf{0.001}, 0.484)&
    79.268$\pm$0.005 ($<$\textbf{0.001}, $<$\textbf{0.001})\\
    
    &
    ROI-1-CC&
    \textbf{95.090$\pm$0.002}&
    93.994$\pm$0.002 ($<$\textbf{0.001})&
    94.297$\pm$0.002 ($<$\textbf{0.001}, 0.009)&
    84.662$\pm$0.004 ($<$\textbf{0.001}, $<$\textbf{0.001})\\
    
    &
    ROI-2-MCP &
    \textbf{92.762$\pm$0.003}&
    91.746$\pm$0.003 ($<$\textbf{0.001})& 
    92.046$\pm$0.003 ($<$\textbf{0.001}, 0.032)& 
    79.796$\pm$0.006 ($<$\textbf{0.001}, $<$\textbf{0.001})\\
    
    &
    ROI-3-SLF &
    \textbf{94.577$\pm$0.005}& 
    94.233$\pm$0.005 (\textbf{0.001})&
    93.291$\pm$0.005 ($<$\textbf{0.001}, $<$\textbf{0.001})&
    74.844$\pm$0.011 ($<$\textbf{0.001}, $<$\textbf{0.001})\\
    \specialrule{1.25pt}{0pt}{0pt}
    \multirow{4}{*}{\shortstack{\textbf{Fix}\\
    \textbf{Acc}\\
    ($\uparrow$)}} &
    White Matter &
    0.640$\pm$0.011& 
    \textbf{0.664$\pm$0.008} ($<$\textbf{0.001})&
    0.645$\pm$0.009 (0.037, $<$\textbf{0.001})&
    0.536$\pm$0.006 ($<$\textbf{0.001}, $<$\textbf{0.001})\\
    
    &
    ROI-1-CC&
    \textbf{0.901$\pm$0.002}& 
    0.851$\pm$0.005 ($<$\textbf{0.001})&
    0.867$\pm$0.003 ($<$\textbf{0.001}, 0.036)&
    0.469$\pm$0.010 ($<$\textbf{0.001}, $<$\textbf{0.001})\\
    
    &
    ROI-2-MCP &
    0.754$\pm$0.011& 
    \textbf{0.791$\pm$0.010} ($<$\textbf{0.001})&
    0.772$\pm$0.009 (\textbf{0.001}, 0.018)&
    0.548$\pm$0.009 ($<$\textbf{0.001}, $<$\textbf{0.001})\\
    
    &
    ROI-3-SLF &
    0.606$\pm$0.032& 
    \textbf{0.648$\pm$0.031} (\textbf{0.001})&
    0.588$\pm$0.029 (0.023, $<$\textbf{0.001})&
    0.548$\pm$0.009 (0.076, 0.163)\\
    \specialrule{1.25pt}{0pt}{0pt}
    \multirow{4}{*}{\textbf{\shortstack{PAE \\  ($\downarrow$)}}} &
    White Matter &
    0.155$\pm$0.006& 
    \textbf{0.147$\pm$0.005} (\textbf{0.001})&
    0.152$\pm$0.005 (0.065, 0.011)&
    0.244$\pm$0.007 ($<$\textbf{0.001}, $<$\textbf{0.001})\\
    
    &
    ROI-1-CC&
    \textbf{0.062$\pm$0.002}& 
    0.072$\pm$0.002 ($<$\textbf{0.001})&
    0.069$\pm$0.002 ($<$\textbf{0.001}, 0.053)&
    0.210$\pm$0.007 ($<$\textbf{0.001}, $<$\textbf{0.001})\\
    
    &
    ROI-2-MCP &
     \textbf{0.135$\pm$0.003}& 
     0.136$\pm$0.003 (0.393)&
     0.136$\pm$0.002 (0.843, 0.973)&
     0.219$\pm$0.004 ($<$\textbf{0.001}, $<$\textbf{0.001})\\
    &
    ROI-3-SLF &
     0.179$\pm$0.009& 
     \textbf{0.178$\pm$0.007} (0.779)&
     0.194$\pm$0.010 (\textbf{0.001}, \textbf{0.002})&
     0.278$\pm$0.006 ($<$\textbf{0.001}, $<$\textbf{0.001})\\
    \specialrule{1.25pt}{0pt}{0pt}
    \multirow{4}{*}{\textbf{\shortstack{AFDE \\ ($\downarrow$)}}} &
    White Matter&
    0.164$\pm$0.005& 
    \textbf{0.151$\pm$0.004} ($<$\textbf{0.001})&
    0.160$\pm$0.005 (0.012, \textbf{0.002})&
    0.208$\pm$0.006 ($<$\textbf{0.001}, $<$\textbf{0.001})\\
    
    &
    ROI-1-CC&
    \textbf{0.065$\pm$0.001}& 
    0.074$\pm$0.001 ($<$\textbf{0.001})&
    0.073$\pm$0.002 (\textbf{0.002}, 0.711)&
    0.187$\pm$0.007 ($<$\textbf{0.001}, $<$\textbf{0.001})\\
    
    &
    ROI-2-MCP &
     0.107$\pm$0.002& 
     \textbf{0.105$\pm$0.001} (0.526)&
     0.106$\pm$0.001 (0.713, 0.489)&
     0.171$\pm$0.003 ($<$\textbf{0.001}, $<$\textbf{0.001})\\
    
    &
    ROI-3-SLF&
     0.151$\pm$0.007& 
     \textbf{0.149$\pm$0.006} (0.462)&
     0.165$\pm$0.007 ($<$\textbf{0.001}, $<$\textbf{0.001})&
     0.230$\pm$0.006 ($<$\textbf{0.001}, $<$\textbf{0.001})\\
    \specialrule{1.25pt}{0pt}{0pt}
\end{tabular}%

}
\end{table*}
 
In ROI-1-CC, ROI-2-MCP and ROI-3-SLF, both AFDE and PAE generally increased as the number of fixels increased. In ROI-1-CC, SDNet achieved strongest results with respect to both AFDE and PAE and in ROI-2-MCP all three deep learning methods performed similarly with respect to both AFDE and PAE. In ROI-3-SLF, SDNet and ${\text{SDNet}}_{\kappa}$ achieved similar AFDE and PAE, with no statistically significant difference between them, but both achieved a statistically significant improvement compared to FOD-Net.

\subsection{Ablation Study}
\begin{table} 
    \caption{The results of all five performance metrics (mean $\pm$ standard error), averaged over all white matter voxels in all 7 test subjects. \textbf{${\bf 2^{nd}}$ column:}  SDNet, \textbf{${\bf3^{rd}}$ column:} SDNet without the DWI consistency block, \textbf{${\bf 4^{th}}$ column:} percentage difference between SDNet with and without the DWI consistency blocks, \textbf{${\bf 5^{th}}$ column:} pairwise t-test p-values. Bold p-values indicate a significant ($\alpha = 0.05$) effect.}
\label{tab:ablation}
\resizebox{1\columnwidth}{!}{
    \begin{tabular}{?c?c|c|c|c?}
         \specialrule{1.25pt}{0pt}{0pt}
         Metric & SDNet & \makecell{SDNet \\ w/o DC} & \makecell{Percentage \\ Change} & \makecell{p\\value}\\
         \specialrule{1.25pt}{0pt}{0pt}
         SSE ($\downarrow$)& ${\bf{0.011 \pm 0.001}}$  & $0.012 \pm 0.001$ & 9.1 \%  & {$<$ \textbf{0.05}} \\
         \hline
         ACC ($\uparrow$) & ${\bf{92.209 \pm 0.003}}$ & $91.679 \pm 0.003$ & 0.57\% & {$<$ \textbf{0.05}}\\
         \hline
         Fix Acc ($\uparrow$)& ${\bf{0.640 \pm 0.011}}$ & $0.625 \pm 0.010$ & 2.3\% & {$<$ \textbf{0.05}}\\
         \hline
         AFDE ($\downarrow$) & ${\bf{0.164 \pm 0.005}}$ & $0.177 \pm 0.005$ & 1.3 \% & {$<$ \textbf{0.05}}\\
         \hline
         PAE ($\downarrow$)& ${\bf{0.155 \pm 0.006}}$ & $0.163 \pm 0.006$ & 0.8\% & {$<$ \textbf{0.05}}\\
         \specialrule{1.25pt}{0pt}{0pt}
    \end{tabular}
    }
    \vspace{-10pt}
\end{table}

The results of the ablation study (Tab.~\ref{tab:ablation}) clearly demonstrate that removing the DWI consistency blocks from the SDNet architecture caused the performance of the network to degrade significantly with respect to all metrics. The greatest relative degradation of performance occurred with respect to SSE, however consistent reductions in the performance of all other metrics was also observed. 

% Figure environment removed

\section{Discussion}
\label{sec:discussion}
SDNet is a model-based deep learning architecture that employs DWI consistency blocks to ensure intermediate FODs are consistent with the DWI signal, whilst making use of spatial information and multi-shell DWI data to reconstruct FODs. We compared our network to FOD-Net \citep{zeng2022fod}, a FOD super-resolution network, which fits FODs to the DWI signal prior to the network's forward pass. Our results show that SDNet improved over FOD-Net in terms of FOD-based performance, and performed similarly with respect to most fixel-based metrics. We conjecture that FOD-Net loses some details of the DWI signal in the FOD fitting stage. Our qualitative results (Fig.~\ref{fig:FODQual}) support this since the FODs reconstructed by FOD-Net more closely resembled the unstable input MSMT-CSD FODs, whereas by ensuring consistency with the DWI signal, SDNet more robustly reconstructed FODs which closely resembled the ground truth. The quantitative results collected from our comparison and ablation studies highlighted the improvement in FOD-based performance enabled by including DWI consistency blocks.

The ultimate goal of deep learning based FOD reconstruction is to produce FODs that are useful for quantitative analysis. FOD registration \citep{raffelt2011symmetric}, a key component of longitudinal and group FOD analyses, relies on $L_{2}$ distance between SH coefficients to captures FOD similarity. By achieving a low SSE, the SH representations will bear increased similarity to the ground truth FODs. We therefore anticipate that SDNet will help ensure that FOD registration is minimally impacted by DWI undersampling, and so too the subsequent analysis. 

% Figure environment removed

Another factor that may impact such analyses is data containing abnormalities, such as pathologies. Such data will likely not be abundant in the datasets used for training deep learning based FOD reconstruction networks, and as a consequence, reduced performance caused by overfitting becomes probable. Since the DWI consistency blocks ensure that solutions will be consistent with the measured DWI data, we expect that SDNet will be less likely to overfit therefore performing comparatively well compared to networks without DWI consistency blocks. However, further investigation is beyond the scope of the current work. 


The outcome of such quantitative analysis is also dependent on the post-registration steps in the pipeline, which, in the case of a fixel-based analysis \citep{raffelt2012apparent}, will be predominantly impacted by the fixel-based performance. Comparing multiple FOD reconstruction algorithms revealed that strong FOD-based performance doesn't directly translate to strong fixel-based performance. The disconnect between FOD and fixel-based performance is evident in the statistically significant difference in SSE over the white matter between SDNet and FOD-Net, but the absence of a statistically significant effect in fixel accuracy over the same set of voxels. This effect can be attributed to FOD segmentation's dependence on the angular separation of the FOD lobes, which is dependent on the higher order SH coefficients, which only contribute a small amount to the SSE. This highlights that SSE loss alone may not be optimal for reconstructing FODs that are to be used in a fixel-based analysis pipeline. 

% In this work focus we primarily consider performance metrics related to the fixel-based analysis pipeline and how they can be improved. Tractography and connectome reconstruction is another potential form of quantitative analysis which can be applied to FODs. Whilst beyond the scope of this work we expect that improved fixel accuracy would improve tractography results, and therefore connectome reconstruction results. We In future research we would plan to investigate how FODs can be optimally reconstructed for different stage of a tractography-based analysis pipeline. 


By introducing an additional loss component, which penalises reconstructed FODs judged to be made up of the incorrect number of fixels, we have demonstrated that fixel-based performance can be improved. The impact of the proposed loss function is illustrated by the statistically significant increase in fixel accuracy in the white matter achieved by $\text{SDNet}_{\kappa}$ compared to SDNet and FOD-Net. The qualitative results (Fig.~\ref{fig:FixQual}) highlighted the improved angular separation of fibres with low angular separation.  It is also evident that the overall shape of the FOD is captured, as opposed to discrete, or Dirac-like FODs \citep{elaldi2021equivariant,koppers2016direct,karimi2021learning}. Furthermore, statistically significant improvements were recorded in fixel accuracy, PAE and AFDE by $\text{SDNet}_{\kappa}$ across the white matter.

However, the introduction of fixel classification penalty in ROI-1-CC led to a reduction in fixel-based performance. This highlighted a potential bias of SDNet towards over-estimating the number of fixels in each voxel. The input of FOD reconstruction networks are necessarily derived from a DWI acquisition with low angular resolution, so do not have sufficient information to reconstruct FODs that contain all fixels, as observed in Fig.~\ref{fig:ERRmaps}. Therefore, the effect of the fixel classification penalty will generally be to correct these underestimations by encouraging the network to increase the number of fixels. Since ROI-1-CC contains only single fixel voxels, the fixel-classification penalty may have increased the number of over-estimations in this region, which, when combined with the already strong performance of SDNet and FOD-Net, led to the observed decrease in performance. On the other hand, in ROI-3-SLF, a region containing 3 crossing fibres, the use of fixel classification penalty improved performance compared to the other two deep learning methods, and despite worse performance in ROI-1-CC, $\text{SDNet}_{\kappa}$ resulted in an improvement in performance over the white matter voxels for all fixel-based performance metrics.




In the current work, the fixel classification network is trained on the ground truth data alone, which, depending on the efficacy of the FOD reconstruction algorithm, will have a different distribution to the reconstructed FODs. One possible approach to further improving performance is to devise an algorithm to jointly train the FOD reconstruction network and the fixel classification network, similar to the method used to train generative adversarial networks \citep{Goodfellow2014}.

 The fixel classification penalty component of the loss function appears to share some characteristics with regularisation terms that are ubiquitous in model-based methods for solving ill-posed inverse problems. In particular, to minimise a combination of SSE loss and the fixel classification penalty, a decrease in SSE was incurred, and we have identified in our validation experiments that the extent of such a sacrifice can be controlled by the adjustment of $\kappa$ (data not included). This suggests that the solution that obtains the lowest SSE may fail to capture certain desirable features of the FOD. In this work, we have highlighted this impact on the separation of fibre populations with similar orientations, but it is possible other features such as the continuity of fibre populations through space could also be improved using similar methods.




\section{Conclusion}
\label{sec:conclusion}
In this work we have proposed SDNet, a model-based deep learning architecture optimised for FOD reconstruction. In addition to the learned regularisation blocks, are trained directly in an end-to-end fashion and therefore optimised for the task of FOD reconstruction, the network also takes a neighbourhood of multi-shell DWI signals as input to an architecture containing multiple cascades. We further show that there is a trade-off between FOD-based and fixel-based performance, and propose a fixel classification penalty term in our loss function, as implemented in $\text{SDNet}_{\kappa}$, as a method of controlling the the trade-off between these performance metrics. We show that, when compared to a state-of-the-art FOD super-resolution network, FOD-Net, gains in FOD-based and fixel-based performance were achieved by  SDNet and $\text{SDNet}_{\kappa}$, respectively.  

\section*{Acknowledgment}
We would like to thank Xi Jia from University of Birmingham for the fruitful discussion on network architecture and parameter tuning in this research. The computations described in this research were performed using the Baskerville Tier 2 HPC service (https://www.baskerville.ac.uk/). Baskerville was funded by the EPSRC and UKRI through the World Class Labs scheme (EP/T022221/1) and the Digital Research Infrastructure programme (EP/W032244/1) and is operated by Advanced Research Computing at the University of Birmingham.

\bibliographystyle{IEEEtranN}
% \bibliographystyle{IEEEtranN}
\small\bibliography{bibliography}


% \newpage
% \begin{table*}
  \centering
  \resizebox{1\textwidth}{!}{
  \begin{tabular}{?c?l?llll?}
     \specialrule{1.25pt}{0pt}{0pt}
    \textbf{Metric} &\diagbox[innerwidth=1.5cm]{\textbf{Region}}{\textbf{Method}}& 
    \textbf{SDNet}&
    \textbf{SDNet\textsubscript{\boldmath$\kappa$} (p\textsubscript{SD})}&
    \textbf{FODNet (p\textsubscript{SD}, p\textsubscript{SD\boldmath${\kappa}$})} & \textbf{MSMT CSD (p\textsubscript{SD}, {p}\textsubscript{SD\boldmath$\kappa$})}\\
    \specialrule{1.1pt}{0pt}{0pt}
    \multirow{4}{*}{\textbf{\shortstack{SSE\\(\downarrow)}}} & 
    White Matter &
    \textbf{0.011$\pm$0.001} &
    0.012$\pm$0.001 ($<$\textbf{0.001})&
    0.013$\pm$0.001 ($<$\textbf{0.001}, 0.034)& 
    0.041$\pm$0.002 ($<$\textbf{0.001}, $<$\textbf{0.001}) \\
    &
    ROI-1-CC&
    \textbf{0.009$\pm$0.001}&
    0.010$\pm$0.001 ($<$\textbf{0.001}) &
    0.011$\pm$0.001 ($<$\textbf{0.001}, 0.295) &
    0.037$\pm$0.002 ($<$\textbf{0.001}, $<$\textbf{0.001}) \\
    
    &
    ROI-2-MCP &
     \textbf{0.020$\pm$0.001}&
    0.022$\pm$0.001 ($<$\textbf{0.001}) &
    0.022$\pm$0.001 (\textbf{0.001}, 0.175)&
    0.057$\pm$0.003 ($<$\textbf{0.001}, $<$\textbf{0.001})\\
    
    &
    ROI-3-SLF & 
    \textbf{0.012$\pm$0.001}&
    0.013$\pm$0.001 (\textbf{0.005})& 
    0.014$\pm$0.001 ($<$\textbf{0.001}, $<$\textbf{0.001})& 
    0.043$\pm$0.002 ($<$\textbf{0.001}, $<$\textbf{0.001})\\
    \specialrule{1.25pt}{0pt}{0pt}
    \multirow{4}{*}{\textbf{\shortstack{ACC\\(\uparrow)}}} &
    White Matter &
    \textbf{92.209$\pm$0.003}&
    91.152$\pm$0.003 ($<$\textbf{0.001}) &
    91.184$\pm$0.003 ($<$\textbf{0.001}, 0.484)&
    79.268$\pm$0.005 ($<$\textbf{0.001}, $<$\textbf{0.001})\\
    
    &
    ROI-1-CC&
    \textbf{92.386$\pm$0.003}&
    91.205$\pm$0.003 ($<$\textbf{0.001})&
    91.260$\pm$0.003 ($<$\textbf{0.001}, 0.009)&
    79.017$\pm$0.006 ($<$\textbf{0.001}, $<$\textbf{0.001})\\
    
    &
    ROI-2-MCP &
    \textbf{84.892$\pm$0.006}&
    82.907$\pm$0.007 ($<$\textbf{0.001})& 
    83.053$\pm$0.006 ($<$\textbf{0.001}, 0.032)& 
    68.396$\pm$0.009 ($<$\textbf{0.001}, $<$\textbf{0.001})\\
    
    &
    ROI-3-SLF &
    \textbf{93.290$\pm$0.003}& 
    92.372$\pm$0.003 (0.001)&
    92.375$\pm$0.003 ($<$\textbf{0.001}, $<$\textbf{0.001})&
    81.210$\pm$0.006 ($<$\textbf{0.001}, $<$\textbf{0.001})\\
    \specialrule{1.25pt}{0pt}{0pt}
    \multirow{4}{*}{\shortstack{\textbf{Fix}\\
    \textbf{Acc}\\
    (\uparrow)}} &
    White Matter &
    0.640$\pm$0.011& 
    \textbf{0.664$\pm$0.008} ($<$\textbf{0.001})&
    0.645$\pm$0.009 (0.037, $<$\textbf{0.001})&
    0.536$\pm$0.006 ($<$\textbf{0.001}, $<$\textbf{0.001})\\
    
    &
    ROI-1-CC&
    \textbf{0.901$\pm$0.002}& 
    0.851$\pm$0.005 ($<$\textbf{0.001})&
    0.867$\pm$0.003 ($<$\textbf{0.001}, 0.036)&
    0.469$\pm$0.010 ($<$\textbf{0.001}, $<$\textbf{0.001})\\
    
    &
    ROI-2-MCP &
    0.754$\pm$0.011& 
    \textbf{0.791$\pm$0.010} ($<$\textbf{0.001})&
    0.772$\pm$0.009 (0.001, 0.018)&
    0.548$\pm$0.009 ($<$\textbf{0.001}, $<$\textbf{0.001})\\
    
    &
    ROI-3-SLF &
    0.606$\pm$0.032& 
    \textbf{0.648$\pm$0.031} (0.001)&
    0.588$\pm$0.029 (0.023, $<$\textbf{0.001})&
    0.548$\pm$0.009 (0.076, 0.163)\\
    \specialrule{1.25pt}{0pt}{0pt}
    \multirow{4}{*}{\textbf{\shortstack{PAE \\  ($\downarrow$)}}} &
    White Matter &
    0.155$\pm$0.006& 
    \textbf{0.147$\pm$0.005} (\textbf{0.001})&
    0.152$\pm$0.005 (0.065, 0.011)&
    0.244$\pm$0.007 ($<$\textbf{0.001}, $<$\textbf{0.001})\\
    
    &
    ROI-1-CC&
    \textbf{0.062$\pm$0.002}& 
    0.072$\pm$0.002 ($<$\textbf{0.001})&
    0.069$\pm$0.002 (0.065, 0.011)&
    0.210$\pm$0.007 ($<$\textbf{0.001}, $<$\textbf{0.001})\\
    
    &
    ROI-2-MCP &
     \textbf{0.135$\pm$0.003}& 
     0.136$\pm$0.003 (0.393)&
     0.136$\pm$0.002 (0.065, 0.011)&
     0.219$\pm$0.004 ($<$\textbf{0.001}, $<$\textbf{0.001})\\
    &
    ROI-3-SLF &
     0.179$\pm$0.009& 
     \textbf{0.178$\pm$0.007} (0.779)&
     0.194$\pm$0.010 ($<$\textbf{0.001}, 0.053)&
     0.278$\pm$0.006 ($<$\textbf{0.001}, $<$\textbf{0.001})\\
    \specialrule{1.25pt}{0pt}{0pt}
    \multirow{4}{*}{\textbf{\shortstack{AFDE \\ (\downarrow)}}} &
    White Matter&
    0.164$\pm$0.005& 
    \textbf{0.151$\pm$0.004} ($<$\textbf{0.001})&
    0.160$\pm$0.005 (0.012, \textbf{0.002})&
    0.208$\pm$0.006 ($<$\textbf{0.001}, $<$\textbf{0.001})\\
    
    &
    ROI-1-CC&
    \textbf{0.065$\pm$0.001}& 
    0.074$\pm$0.001 ($<$\textbf{0.001})&
    0.073$\pm$0.002 (0.002, 0.711)&
    0.187$\pm$0.007 ($<$\textbf{0.001}, $<$\textbf{0.001})\\
    
    &
    ROI-2-MCP &
     0.107$\pm$0.002& 
     \textbf{0.105$\pm$0.001} (0.526)&
     0.106$\pm$0.001 (0.713, 0.489)&
     0.171$\pm$0.003 ($<$\textbf{0.001}, $<$\textbf{0.001})\\
    
    &
    ROI-3-SLF&
     0.151$\pm$0.007& 
     \textbf{0.149$\pm$0.006} (0.462)&
     0.165$\pm$0.007 ($<$\textbf{0.001}, $<$\textbf{0.001})&
     0.230$\pm$0.006 ($<$\textbf{0.001}, $<$\textbf{0.001})\\
    \specialrule{1.25pt}{0pt}{0pt}
  \end{tabular}}
  \caption{Quantitative results for all performance metrics, methods and regions of interest. $\text{p}_{SD}$ is the p-value for the paired t-test between the respective method and SDNet, $\text{p}_{SD \kappa}$ is the p-value for the paired t-test between the respective method and $\text{SDNet}_{\kappa}$.}
  \label{tab:QuantRes}
\end{table*}

 \begin{table*}
  \centering
  \resizebox{1\textwidth}{!}{
  \begin{tabular}{?l?l?llll?}
     \specialrule{1.25pt}{0pt}{0pt}
    \textbf{Metric} &\diagbox[innerwidth=2cm]{\textbf{Method}}{\textbf{Region}}& \textbf{White Matter} & \textbf{ROI-1-CC} & \textbf{ROI-2-MCP} & \textbf{ROI-3-SLF}\\
    \specialrule{1.1pt}{0pt}{0pt}
    \multirow{4}{*}{\textbf{SSE}} & 
    White Matter &
    0.011$\pm$0.001 &
    0.009$\pm$0.001 &
    0.020$\pm$0.001 & 
    0.012$\pm$0.001 \\
    \cline{2-6}
    &
    ${\text{SDNet}}_{\kappa}$ ($\text{p}_{\text{SD}}$)&
    0.012$\pm$0.001 ($$<$$\textbf{0.001}) &
    0.010$\pm$0.001 ($$<$$\textbf{0.001}) &
    0.022$\pm$0.001 ($$<$$\textbf{0.001}) &
    0.013$\pm$0.001 (\textbf{0.005}$) \\
    \cline{2-6}
    &
    FODNet ($\text{p}_{\text{SD}}, {\text{p}}_{\text{SD}\kappa}$) &
    0.013$\pm$0.001 $ {\textbf{($$<$$0.001}}, 0.034)$ &
    0.011$\pm$0.001 $ {\textbf{($$<$$0.001}}, 0.295)$&
    0.022$\pm$0.001 $ {\textbf{(0.001}}, 0.175)$ &
    0.014$\pm$0.001 $ {\textbf{($$<$$0.001}}, {\textbf{$$<$$0.001)}}$ \\
    \cline{2-6}
    &
    MSMT CSD ($\text{p}_{\text{SD}}, {\text{p}}_{\text{SD}\kappa}$) & 
    0.041$\pm$0.002 $ \textbf{($$<$$0.001}, \textbf{$$<$$0.001)}$ &
    0.037$\pm$0.002 $ \textbf{($$<$$0.001}, \textbf{$$<$$0.001)}$ & 
    0.057$\pm$0.003 $ \textbf{($$<$$0.001}, \textbf{$$<$$0.001)}$ & 
    0.043$\pm$0.002 $ \textbf{($$<$$0.001}, \textbf{$$<$$0.001)}$ \\
    \specialrule{1.25pt}{0pt}{0pt}
    \multirow{4}{*}{\textbf{ACC}} &
    SDNet &
    92.209$\pm$0.003 &
    92.386$\pm$0.003 &
    84.892$\pm$0.006 &
    93.290$\pm$0.003 \\
    \cline{2-6}
    &
    ${\text{SDNet}}_{\kappa}$ ($\text{p}_{\text{SD}}$)&
    91.152$\pm$0.003 $ \textbf{($$<$$0.001)}$&
    91.205$\pm$0.003  $\textbf{($$<$$0.001)}$ &
    82.907$\pm$0.007 $ \textbf{($$<$$0.001)}$ &
    92.372$\pm$0.003 $ \textbf{(0.001)}$ \\
    \cline{2-6}
    &
    FODNet ($\text{p}_{\text{SD}}$ (${\text{p}}_{\text{SD}\kappa}$) &
    91.184$\pm$0.003 ${\textbf{($$<$$0.001}}, 0.484)$&
    91.260$\pm$0.003 $ {\textbf{($$<$$0.001}}, 0.009)$ & 
    83.053$\pm$0.006 $ {\textbf{($$<$$0.001}}, 0.032)$ & 
    92.375$\pm$0.003 $ {\textbf{($$<$$0.001}}, {\textbf{$$<$$0.001)}}$\\
    \cline{2-6}
    &
    MSMT CSD ($\text{p}_{\text{SD}}, {\text{p}}_{\text{SD}\kappa}$) &
    79.268$\pm$0.005 $ \textbf{($$<$$0.001}, \textbf{$$<$$0.001)}$& 
    79.017$\pm$0.006 $ \textbf{($$<$$0.001}, \textbf{$$<$$0.001)}$ &
    68.396$\pm$0.009 $ \textbf{($$<$$0.001}, \textbf{$$<$$0.001)}$ &
    81.210$\pm$0.006 $ \textbf{($$<$$0.001}, \textbf{$$<$$0.001)}$ \\
    \specialrule{1.25pt}{0pt}{0pt}
    \multirow{4}{*}{\textbf{Fix Acc}} &
    SDNet &
    0.640$\pm$0.011& 
    0.901$\pm$0.002&
    0.754$\pm$0.011&
    0.606$\pm$0.032\\
    \cline{2-6}
    &
    ${\text{SDNet}}_{\kappa}$ ($\text{p}_{\text{SD}}$)&
    0.664$\pm$0.008 $ \textbf{($$<$$0.001)}$& 
    0.851$\pm$0.005 $\textbf{($$<$$0.001)}$&
    0.791$\pm$0.010 $ \textbf{($$<$$0.001)}$&
    0.648$\pm$0.031 $ \textbf{(0.001)}$\\
    \cline{2-6}
    &
    FODNet ($\text{p}_{\text{SD}}, {\text{p}}_{\text{SD}\kappa}$) &
    0.645$\pm$0.009 $ (0.037, \textbf{$$<$$0.001)}$& 
    0.867$\pm$0.003 $\textbf{($$<$$0.001}, 0.036)$&
    0.772$\pm$0.009 $\textbf{(0.001}, 0.018)$&
    0.588$\pm$0.029 $(0.023, \textbf{$$<$$0.001})$\\
    \cline{2-6}
    &
    MSMT CSD ($\text{p}_{\text{SD}}, {\text{p}}_{\text{SD}\kappa}$) &
    0.536$\pm$0.006 $\textbf{($$<$$0.001}, \textbf{$$<$$0.001})$& 
    0.469$\pm$0.010 $\textbf{($$<$$0.001}, \textbf{$$<$$0.001})$&
    0.548$\pm$0.009 $\textbf{($$<$$0.001}, \textbf{$$<$$0.001})$&
    0.548$\pm$0.009 $(0.076, 0.163)$\\
    \specialrule{1.25pt}{0pt}{0pt}
    \multirow{4}{*}{\textbf{PAE}} &
    SDNet &
    0.155$\pm$0.006 & 
    0.062$\pm$0.002&
    0.135$\pm$0.003&
    0.179$\pm$0.009\\
    \cline{2-6}
    &
    ${\text{SDNet}}_{\kappa} (\text{p}_{\text{SD}}$)&
    0.147$\pm$0.005 $ \textbf{(0.001)}$& 
    0.072$\pm$0.002 $ \textbf{($$<$$0.001)}$&
    0.136$\pm$0.003 $(0.393)$&
    0.178$\pm$0.007 $(0.779)$\\
    \cline{2-6}
    &
    FODNet ($\text{p}_{\text{SD}}, {\text{p}}_{\text{SD}\kappa}$) &
    0.152$\pm$0.005 $ \textbf{(0.065, 0.011)}$& 
    0.069$\pm$0.002 $\textbf{(0.065, 0.011)}$&
    0.136$\pm$0.002 $(0.065, 0.011)$&
    0.194$\pm$0.010 $ (\textbf{$$<$$0.001}, 0.053)$\\
    \cline{2-6}
    &
    MSMT CSD ($\text{p}_{\text{SD}}, {\text{p}}_{\text{SD}\kappa}$) &
    0.244$\pm$0.007 $\textbf{($$<$$0.001}, \textbf{$$<$$0.001)}$& 
    0.210$\pm$0.007 $\textbf{($$<$$0.001}, \textbf{$$<$$0.001)}$&
    0.219$\pm$0.004 $\textbf{($$<$$0.001}, \textbf{$$<$$0.001)}$&
    0.278$\pm$0.006 $\textbf{($$<$$0.001}, \textbf{$$<$$0.001)}$\\
    \specialrule{1.25pt}{0pt}{0pt}
    \multirow{4}{*}{\textbf{AFDE}} &
    SDNet&
    0.164$\pm$0.005& 
    0.065$\pm$0.001&
    0.107$\pm$0.002&
    0.151$\pm$0.007\\
    \cline{2-6}
    &
    ${\text{SDNet}}_{\kappa}$ ($\text{p}_{\text{SD}}$)&
     0.151$\pm$0.004 $ \textbf{($$<$$0.001)}$& 
     0.074$\pm$0.001 $ \textbf{($$<$$0.001)}$&
     0.105$\pm$0.001 $(0.526)$&
     0.149$\pm$0.006 $(0.462)$\\
    \cline{2-6}
    &
    FODNet ($\text{p}_{\text{SD}}, {\text{p}}_{\text{SD}\kappa}$) &
    0.160$\pm$0.005 $(0.012, \textbf{0.002)}$& 
    0.073$\pm$0.002 $ \textbf{(0.002}, 0.711)$&
    0.106$\pm$0.001 $ (0.713, 0.489)$&
    0.165$\pm$0.007 $ \textbf{($$<$$0.001}, \textbf{$$<$$0.001)}$\\
    \cline{2-6}
    &
    MSMT CSD (p_{\text{SD}}, {\text{p}}_{\text{SD}\kappa}) &
    0.208$\pm$0.006 $\textbf{($$<$$0.001}, \textbf{$$<$$0.001)}$& 
    0.187$\pm$0.007 $\textbf{($$<$$0.001}, \textbf{$$<$$0.001)}$&
    0.171$\pm$0.003 ($$<$$\textbf{0.001}, $$<$$\textbf{0.001})$&
    0.230$\pm$0.006 ($$<$$\textbf{0.001}, $$<$$\textbf{0.001})\\
    \specialrule{1.25pt}{0pt}{0pt}
  \end{tabular}}
  \caption{Quantitative results for all performance metrics, methods and regions of interest. $\text{p}_{SD}$ is the p-value for the paired t-test between the respective method and SDNet, $\text{p}_{SD \kappa}$ is the p-value for the paired t-test between the respective method and $\text{SDNet}_{\kappa}$.}
  \label{tab:QuantRes}
\end{table*}


\begin{table*}[ht]
\caption{Comparing different state-of-the-arts on 2D OASIS. The first, second, and third ranks are highlighted in  \textcolor{red}{red}, \textcolor{blue}{blue}, and \textcolor{brown}{brown} colors, respectively.}
\centering
\label{tab:oasis2d}
% \resizebox{\columnwidth}{!}{
\begin{small}
\begin{tabular}{lcccccccc}\hline
Methods     & Patch & Dice        & $|J|_{ < 0}\%$  & Parameters   & Mult-Adds (M) & Memory (MB)   & CPU (s) & GPU (s) \\\hline
Initial       & - & 0.544$\pm$0.089 &-           &-&-& - & -      \\
Flash       & 16$\times$16      & 0.702$\pm$0.051 & 0.033$\pm$.126  & - & - & - & 13.699 & - \\
Flash       & 20$\times$24      & 0.727$\pm$0.046 & 0.205$\pm$.279 & - & - & - &  22.575 & -    \\
Flash       & 40$\times$48      & 0.734$\pm$0.045 & 0.049$\pm$.080 & - & -& - & 85.773 & -    \\
DeepFlash   & 16$\times$16      & 0.615$\pm$0.055 & 0.0$\pm$0.0            & - & - & -& 0.487      &-   \\
DeepFlash   & 20$\times$24      & 0.597$\pm$0.066            & 0.0$\pm$0.0 & - & - & -&0.617    &-     \\
% DeepFlash   & 40x48      &             &             &     &     \\
B-Spline-Diff   & 20$\times$24      & 0.710$\pm$0.041 &  0.014$\pm$0.072 & 70,226 & 86.16 & 6.58   &0.012     &0.015     \\
B-Spline-Diff   & 40$\times$48      & 0.737$\pm$0.038  & 0.015$\pm$0.069 & 88,690    & 139.40 & 7.49    &0.012 &0.015 \\\hline
Fourier-Net$_{\text{Small}}$ &40$\times$48     & 0.748$\pm$0.039  & 0.672$\pm$0.391   &357,408          & 224.55 & 17.96    & \textbf{0.007}     & \textbf{0.014} \\
Fourier-Net-Diff$_{\text{Small}}$ &40$\times$48     & 0.750$\pm$0.038  & $<$0.0001         &357,408    & 224.55 & 17.96    & 0.010     & 0.014 \\  
Fourier-Net &40$\times$48     & 0.756$\pm$0.039  & 0.753$\pm$0.408       &1,427,376     & 888.25 & 35.89     & 0.011     & 0.015    \\
Fourier-Net-Diff &  40$\times$48 & 0.756$\pm$0.037  & $<$0.0001 &1,427,376 & 888.25 & 35.89  & 0.015     & 0.015   \\
% Fourier-Net${_{Large}}$ &40$\times$48     & 0.759$\pm$0.040  & 0.781$\pm$0.405             & 0.131     & 0.023    \\
% Fourier-Net-Diff${_{Large}}$ &  40$\times$48 & \textbf{0.761$\pm$0.037}  & 0.0$\pm$0.0   & 0.040     & 0.024   \\\hline
1$\times$Fourier-Net++       & 40$\times$48   & 0.738$\pm$0.041  & 0.674$\pm$0.377 &300,477 & 142.66         & 9.52 & 0.006 & 0.014\\
1$\times$Fourier-Net++-Diff  & 40$\times$48   & 0.740$\pm$0.039  & 0.0$\pm$0.0    &300,477 & 142.66         & 9.52 & 0.009 & 0.014\\
4$\times$Fourier-Net++$_{\text{Small}}$ &  40$\times$48 & 0.757$\pm$0.039  & 0.252$\pm$0.197 &301,716 & 145.44         & 19.12 & 0.018 & 0.017\\
4$\times$Fourier-Net++-Diff$_{\text{Small}}$ &  40$\times$48 &0.746$\pm$0.040  & 0.0$\pm$0.0      &301,716 & 145.44         & 19.12 & 0.021 & 0.017\\
4$\times$Fourier-Net++ &  40$\times$48 & 0.761$\pm$0.039  & 0.278$\pm$0.232 &1,201,908 & 570.64        & 38.08 & 0.022 & 0.017\\
4$\times$Fourier-Net++-Diff &  40$\times$48 &0.755$\pm$0.039  & $<$0.0001     &1,201,908  & 570.64         & 38.08 & 0.025 & 0.017\\\hline
\end{tabular}
\end{small}
% }
\end{table*}









\end{document}
