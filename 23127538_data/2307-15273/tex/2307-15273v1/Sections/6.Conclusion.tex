In this work we have proposed SDNet, a model-based deep learning architecture optimised for FOD reconstruction. In addition to the learned regularisation blocks, are trained directly in an end-to-end fashion and therefore optimised for the task of FOD reconstruction, the network also takes a neighbourhood of multi-shell DWI signals as input to an architecture containing multiple cascades. We further show that there is a trade-off between FOD-based and fixel-based performance, and propose a fixel classification penalty term in our loss function, as implemented in $\text{SDNet}_{\kappa}$, as a method of controlling the the trade-off between these performance metrics. We show that, when compared to a state-of-the-art FOD super-resolution network, FOD-Net, gains in FOD-based and fixel-based performance were achieved by  SDNet and $\text{SDNet}_{\kappa}$, respectively.  

\section*{Acknowledgment}
We would like to thank Xi Jia from University of Birmingham for the fruitful discussion on network architecture and parameter tuning in this research. The computations described in this research were performed using the Baskerville Tier 2 HPC service (https://www.baskerville.ac.uk/). Baskerville was funded by the EPSRC and UKRI through the World Class Labs scheme (EP/T022221/1) and the Digital Research Infrastructure programme (EP/W032244/1) and is operated by Advanced Research Computing at the University of Birmingham.