\subsection{Dataset}
A subset of the WU-Minn Human Connectome Project (HCP) dataset \citep{van2013wu}, consisting of 30 subjects, was split $20/3/7$ and used for training, validation, and testing, respectively. The HCP images have $1.25\text{mm}$ isotropic resolution with 90 gradient directions for $b = 1000, 2000 \text{ and } 3000 \text{ s/mm}^{2}$ and 18 $b_{0}$ images. The HCP dataset was minimally pre-processed in accordance with \citep{sotiropoulos2013advances}. 

%Data Preprocessing:
Additionally, prior to applying SDNet, each subject's data was normalised using MRtrix3's \textit{dwinormalise} function. The fully sampled FODs were fit to all 288 DWIs; first, the response functions were calculated using the method proposed in \citep{dhollander2019improved}, then the FODs calculated using MSMT-CSD \citep{jeurissen2014multi}. White matter response functions and FODs were modelled with $l_{max} = 8$ and the grey matter and cerebrospinal fluid component response functions and FODs were modelled with $l_{max} = 0$, resulting in a total of 47 SH coefficients. 

%HCP intro for how it is specifcally used in this study.
The sampling pattern \citet{caruyer2013design} utilised in the HCP is such that for any $k$, selection of the first $k$ DWI volumes results in evenly spread b-vectors. To prepare the input data, the first 9 DWIs from each non-zero shell were selected with an additional 3 $b_{0}$ images, resulting in a total of 30 DWI signals. 

% Figure environment removed

%How patches were selected for training.
Only patches in which the central voxel is classified as grey matter or white matter are used for training. The grey matter voxels were included to improve performance near the boundary of the two tissue types, as highlighted in \citep{zeng2022fod}. The grey and white matter masks were calculated using the method outlined in \citep{zhang2001segmentation}, which is implemented using the FSL software package \citep{jenkinson2012fsl}. 

From this point onwards, for notational convenience, SDNet ($\kappa = 0$) will be referred to as SDNet and SDNet ($\kappa = 1.6\times 10^{-4}$) will be referred to as ${\text{SDNet}}_{\kappa}$. To evaluate the performance of the introduced methods, SDNet, ${\text{SDNet}}_{\kappa}$, FOD-Net \citep{zeng2022fod}, and super-resolved MSMT CSD, referred to as MSMT CSD for notational simplicity, were all compared. In the original implementation, FOD-Net maps FODs fit using the single shell three tissue CSD algorithm \citep{dhollander2019improved} to 32 DWIs (4 $b_{0} \text{ and } 28 \text{ } b = 1000/2000/3000 \text{ s/mm}^{2}$) to the desired MSMT CSD obtained FODs. To allow a fair comparison between FOD-Net and the proposed networks, FOD-Net was trained using the same training set as SDNet. Since the final block in the SDNet architecture is a DWI consistency block, it cannot map to normalised FODs, therefore the target training data is not normalised. It should be noted that the normalisation can still be performed as a post-processing step. Otherwise, the same configuration settings found in the Github repository released by the FOD-Net authors were used.

\subsection{Performance Metrics}

%Performance Metrics;
%Explain how we aggregated each of the perfromance metrics into one score - including how the white matter masks were obtained.
To evaluate the performance of the FOD reconstruction algorithms, performance metrics were calculated voxel-wise then averaged over regions of interest.  The regions considered were the white matter and intersections of individual tracts within the white matter. The tracts considered were: the corpus callosum (CC), the middle cerebellar peduncle (MCP), the corticospinal tract (CST), and the superior longitudinal fascicle (SLF). To understand how the algorithm performs in voxels containing different numbers of fibres, we considered the intersections of these tracts as in \citep{zeng2022fod}. For voxels containing a single fibre, we considered voxels in the CC containing a single fixel, which we refer to as ROI-1-CC. For two crossing fibres, we considered voxels in the intersection of the MCP and CST containing two fixels, which we refer to as ROI-2-MCP. For three crossing fibres, we considered voxels in the intersection of the SLF, CST and CC containing three fixels, which we refer to as ROI-3-SLF. The white matter mask was calculated using the FSL five tissue type segmentation algorithm in MRtrix3. The segmentation masks for the white matter fibre tracts were obtained using TractSeg \citep{wasserthal2018tractseg}. 

%The MSE Performance metric.
The SSE between the reconstructed FODs, $\hat{{\bf{c}}}$, and the fully sampled FODs, ${\bf{c}}$, was computed as follows:  
\begin{equation}
\label{eq:SSE}
\text{SSE}\left({\bf{c}}, \hat{{\bf{c}}}\right) = \left\|{\bf{c}} - \hat{{\bf{c}}}\right\|^{2}_{2}
\end{equation}

%The ACC performance metric
The angular correlation coefficient (ACC) \citep{anderson2005measurement} was computed as follows:

\begin{equation}
\text{ACC}({\bf{c}}, \hat{{\bf{c}}}) = 
\frac{\sum\limits^{4}_{i=1}\sum\limits_{j=-2i}^{2i}{\bf{c}}_{2i,j}\hat{{\bf{c}}}_{2i,j}}{\sqrt{\left(\sum\limits^{4}_{i=1}\sum\limits^{2i}_{j=-2i}{\bf{c}}_{2i,j}^{2}\right)
\left(\sum\limits^{4}_{i=1}\sum\limits^{2i}_{j=-2i}\hat{{\bf{c}}}_{2i,j}^{2}\right)}}
\end{equation}


%fixel based analysis explanation and introduction. 
We refer to SSE and ACC as \textit{FOD-based performance} metrics, since they compare the SH representation of the FODs prior to any further processing. 

Fixel-based analysis requires each FOD to be segmented into fixels, each of which has associated apparent fibre density and peak amplitude \citep{smith2013sift}. To calculate the associated error metrics, peak amplitude and apparent fibre density vectors must be assembled. Each vector consists of the respective scalar for each fixel ordered according to the peak amplitude and are padded to a fixed length. The remaining metrics are referred to as \textit{fixel-based performance} metrics since they require the FOD to be segmented into fixels prior to evaluation. 

%Introduce the fixel number accuracy metric. 
Fixel accuracy was defined for a region of interest as the proportion of voxels in which the FOD is segmented into the correct number of fixels.  

%The PAE:
The peak amplitude error (PAE) was calculated between the reconstructed, $\hat{{\boldsymbol{f}}}^{P}$, and fully sampled FOD's, ${\boldsymbol{f}}^{P}$, peak amplitude vectors:
\begin{equation}
\label{eq:PAE}
\text{PAE}\left({\boldsymbol{f}}^{P}, \hat{{\boldsymbol{f}}}^{P}\right) = \sum\limits_{i}\left|f^{P}_{i}-\hat{f}^{P}_{i}\right|
\end{equation}

%Apparent fibre density explanation and definition 
The apparent fibre density error (AFDE) was calculated between the reconstructed, $\hat{{\boldsymbol{f}}}^{A}$, and fully sampled FOD's, ${\boldsymbol{f}}^{A}$, apparent fibre density vectors:
\begin{equation}
\label{eq:AFDE}
\text{AFDE}\left({\boldsymbol{f}}^{A}, \hat{{\boldsymbol{f}}}^{A}\right) = \sum\limits_{i}\left|f^{A}_{i}-\hat{f}^{A}_{i}\right|
\end{equation}
% %Mean Angular Error:
% We also considered the mean angular error for the fixels calculated in each voxel. To do so we considered the unit vector which characterises the direction of each of the fixels in a voxel. We then calcuate the angular difference between that and the direction of the corresponding ground truth vector:
% \[MAE = \frac{1}{n}\Sigma^{n}_{i=0}\hat{f}^{angle}\cdot f^{angle}_{gt}\]

%Include a note on how tractography was carried out.

%Which methods we compared: (MSMT CSD, FOD-Net, SDNet (w\ and w\o fixel classification loss))
%FOD-Net



%HCP dataset introduction - should also include that the 3 of the 4 test images are the images containing anomalies.


\subsection{Ablation Study}
To investigate the impact of the DWI consistency block on the performance of the network, an ablation study was conducted. The network was trained without the DWI consistency blocks, and all other aspects of the architecture and network training remained the same. We compared this model to SDNet with the DWI consistency blocks included. 



%Implementation: Include Hardware and software spec here.
%Any other experimental conditions we compared the methods under (e.g. fewer DWIs etc). May turn out to be better to only considered the SDNEt results in this case

% Figure environment removed

%Ablation Study.
\subsection{Statistical Analysis}
Shapiro-Wilk tests for normality ($\alpha=0.05$) were applied for each performance metric and method; unless otherwise stated there is insufficient evidence to reject the null hypothesis that the groups are normally distributed.

Since the data was normally distributed, and each method was applied to the same set of test subjects, a repeated measures one-way ANOVA ($\alpha=0.05$) was applied to each performance metric to determine whether there was a main effect between the conditions. 
% bonferoni corrected t-tests
Finally, to determine which methods contributed to the main effect, post-hoc t-tests with Bonferroni correction (adjusted for $\alpha = 0.05$) were used to identify effects between the FOD reconstruction algorithms. 






