%!TEX root = convex-paper.tex

% Theorem styles
\newtheorem{theorem}{Theorem}[section]
\newtheorem{lemma}[theorem]{Lemma}
\newtheorem{proposition}[theorem]{Proposition}
\newtheorem{corollary}[theorem]{Corollary}
\theoremstyle{definition}
\newtheorem{definition}[theorem]{Definition}
\theoremstyle{remark}
\newtheorem{remark}[theorem]{Remark}


% Tikz styles
\tikzset{
	world/.style={draw,circle,outer sep=0pt,inner sep=0,minimum size=15},
	point/.style={draw=black,fill=black,opacity=1,circle,outer sep=0pt,inner sep=0,minimum size=2},
	facefill/.style={fill=blue!20,fill opacity=0.4}
}
\tikzgraphsset{
	poset/.style={grow up=#1, branch right=#1, empty nodes, simple, nodes=point},
	poset/.default=1cm
}

% New Tikz coordinate system
% Takes the average of any number of points
% Usage: `bc cs:node1,node2,node3,...`
% Modified from the Tikz source tikz.code.tex
\makeatletter
\tikzdeclarecoordinatesystem{bc}
{%
  {%
    \pgf@xa=0pt% point
    \pgf@ya=0pt%
    \pgf@xb=0pt% sum
    \tikz@bc@dolist#1,,%
    \pgfmathparse{1/\the\pgf@xb}%
    \global\pgf@x=\pgfmathresult\pgf@xa%
    \global\pgf@y=\pgfmathresult\pgf@ya%
  }%
}%

\def\tikz@bc@dolist#1,{%
  \def\tikz@temp{#1}%
  \ifx\tikz@temp\pgfutil@empty%
  \else
    \pgf@process{\pgfpointanchor{#1}{center}}%
    \pgfmathparse{1}%
    \advance\pgf@xa by\pgfmathresult\pgf@x%
    \advance\pgf@ya by\pgfmathresult\pgf@y%
    \advance\pgf@xb by\pgfmathresult pt%
    \expandafter\tikz@bc@dolist%
  \fi%
}
\makeatother