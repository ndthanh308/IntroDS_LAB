\documentclass[a4paper]{article}
\usepackage{authblk}

\renewcommand\Affilfont{\small}
\title{The Intermediate Logic of Convex Polyhedra}
\author[1]{Sam Adam-Day}
\author[2]{Nick Bezhanishvili}
\author[3]{David Gabelaia}
\author[4]{\authorcr Vincenzo Marra}
\affil[1]{Mathematical Institute,\vspace{-0.4em}\authorcr\Affilfont University of Oxford, United Kingdom}
\affil[2]{Institute for Logic, Language and Computation,\vspace{-0.4em}\authorcr\Affilfont University of Amsterdam, The Netherlands}
\affil[3]{A. Razmadze Mathematical Institute,\vspace{-0.4em}\authorcr\Affilfont I. Javakhishvili Tbilisi State University, Georgia}
\affil[4]{Dipartimento di Matematica ``Federigo Enriques'',\vspace{-0.4em}\authorcr\Affilfont Universit\`a degli Studi di Milano, Italy}
\date{\today}

\RequirePackage{silence}
\WarningFilter{remreset}{The remreset package}
\WarningFilter{latex}{Font shape declaration has incorrect series value `mc'.}

\usepackage[utf8]{inputenc}
\usepackage[hidelinks]{hyperref}
\usepackage[style=alphabetic,maxbibnames=99]{biblatex}
\usepackage{amsmath}
\usepackage{mathtools}
\usepackage{amsthm}
\usepackage{thmtools}
\usepackage{titling}
\usepackage{csquotes}
\usepackage[inline]{enumitem}
\usepackage{nicefrac}
\usepackage[capitalize,noabbrev]{cleveref}
\usepackage{fonttable}
\usepackage{tikz}
\usepackage{tikz-3dplot}
\usepackage[final]{microtype}
\usepackage{setspace}

\usepackage[charter]{mathdesign}
\usepackage[T1]{fontenc}

\usetikzlibrary{positioning,arrows,graphs,arrows.meta,decorations.pathreplacing,calligraphy,cd}

% \usepackage[utf8]{inputenc}
% \usepackage[margin=1in]{geometry}

% \usepackage{footmisc}
% \usepackage{xr}
% \usepackage[T1]{fontenc}    % use 8-bit T1 fonts
% \usepackage[colorlinks=true, allcolors=blue]{hyperref}
% \usepackage{hyperref}       % hyperlinks
% \usepackage{url}            % simple URL typesetting
% \usepackage{booktabs}       % professional-quality tables
% \usepackage{amsfonts}       % blackboard math symbols
% \usepackage{nicefrac}       % compact symbols for 1/2, etc.
% \usepackage{microtype}      % microtypography
% \usepackage[dvipsnames]{xcolor}         % colors
% \usepackage[normalem]{ulem}
% \usepackage[toc,page]{appendix}
% \usepackage[margin=1in]{geometry}
% \usepackage{empheq}
% \usepackage{amsthm,amssymb} 
\usepackage{bm,verbatim,dsfont,mathtools}
% \usepackage{etoolbox}
\usepackage{tikz}
\usepackage{refcount}
\usetikzlibrary{arrows.meta,patterns,decorations.pathreplacing,decorations.text}
\usepackage{graphicx}
\usepackage{multirow}
% \usepackage{ntheorem}

% \usepackage{subcaption}
% \usepackage{xr,xspace}
% \usepackage{todonotes}
% \usepackage{paralist}
% \usepackage{caption,subcaption,soul}
% \usepackage{comment}
% \usepackage{appendix}
% \setlength\parindent{0pt}
% \theoremstyle{plain}
\newtheorem{theorem}{Theorem}
% \newreptheorem{theorem}{Theorem}
\newtheorem{lemma}{Lemma}
\newtheorem{proposition}{Proposition}
\newtheorem{corollary}{Corollary}
% \theoremstyle{definition}
\newtheorem{definition}{Definition}
\newtheorem{hypothesis}{Hypothesis}
\newtheorem{conjecture}{Conjecture}
\newtheorem{claim}{Claim}
\newtheorem{question}{Question}
\newtheorem{remark}{Remark}
\newtheorem{assumption}{Assumption}
\newtheorem{fact}{Fact}
\newtheorem{exercise}{Exercise}
\newtheorem{example}{Example}
% \makeatletter
%\newtheoremstyle{TheoremNum}
%        {\topsep}{\topsep}              %%% space between body and thm
%        {\itshape}                      %%% Thm body font
        {}                              %%% Indent amount (empty = no indent)
%        {\bfseries}                     %%% Thm head font
%        {.}                             %%% Punctuation after thm head
%        { }                             %%% Space after thm head
 %       {\thmname{#1}\thmnote{ \bfseries #3}}%%% Thm head spec
%    \theoremstyle{TheoremNum}
\newtheorem{thmn}{Theorem}
\newtheorem{crln}{Corollary}
\newtheorem{propn}{Proposition}


% \theoremstyle{remark}
\newtheorem{proof}{Proof}
% \newtheorem{definition}{Definition}


% {\begin{itemize}}
% {\end{itemize}}


\renewcommand{\check}[1]{\widetilde{#1}}

\renewcommand{\implies}{\Rightarrow}
% \newcommand{\1}[1]{{\mathbf{1}_{\left\{{#1}\right\}}}}
\newcommand{\post}[2]{\begin{center} \includegraphics[width=#2]{#1} \end{center} }
% \newcommand \E[1]{\mathbb{E}[#1]}
%\newcommand{\Perp}{\perp \! \! \! \perp}
\newcommand{\Perp}{\perp}
\newcommand{\Hyper}{\text{Hypergeometric}}
% \newcommand{\argmax}{\mathop{\arg\max}}
% \newcommand{\argmin}{\mathop{\arg\min}}
\newcommand{\mathd}{\mathrm{d}}
%\newcommand \P[1]{\mathbb{P}[#1]}


%% Wu
\usepackage{xspace,prettyref}
\newcommand{\CML}{\widehat{C}_{\rm ML}}
\newcommand{\diverge}{\to\infty}
\newcommand{\eqdistr}{{\stackrel{\rm (d)}{=}}}
\newcommand{\iiddistr}{{\stackrel{\text{\iid}}{\sim}}}
\newcommand{\ones}{\mathbf 1}
\newcommand{\zeros}{\mathbf 0}
\newcommand{\reals}{{\mathbb{R}}}
\newcommand{\complex}{{\mathbb{C}}}
\newcommand{\integers}{{\mathbb{Z}}}
\newcommand{\naturals}{{\mathbb{N}}}
\newcommand{\rationals}{{\mathbb{Q}}}
\newcommand{\naturalsex}{\overline{\mathbb{N}}}
\newcommand{\symm}{{\mbox{\bf S}}}  % symmetric matrices
\newcommand{\supp}{{\rm supp}}
\newcommand{\eexp}{{\rm e}}
\newcommand{\rexp}[1]{{\rm e}^{#1}}
\newcommand{\identity}{\mathbf I}
\newcommand{\allones}{\mathbf J}
%\newcommand{\zeros}{\mathbf 0}

\newcommand{\diff}{{\rm d}}
\newcommand{\red}{\color{red}}
\newcommand{\blue}{\color{blue}}
\newcommand{\green}{\color{teal}}
\newcommand{\algcomment}{\color{OliveGreen}}
\newcommand{\nbm}[1]{{\sf\color{blue}[#1]}}
\newcommand{\mx}[1]{\nbm{MX: #1}}
\newcommand{\nbr}[1]{{\sf\red[#1]}}
\newcommand{\Expect}{\mathbb{E}}
\newcommand{\expect}[1]{\mathbb{E}\left[ #1 \right]}
\newcommand{\eexpect}[1]{\mathbb{E}[ #1 ]}
\newcommand{\expects}[2]{\mathbb{E}_{#2}\left[ #1 \right]}
\newcommand{\tExpect}{{\tilde{\mathbb{E}}}}
%\newcommand{\Prob}{\mathop{\mathbb{P}}}
\newcommand{\Prob}{\mathbb{P}}
\newcommand{\pprob}[1]{ \mathbb{P}\{ #1 \} }
\newcommand{\prob}[1]{ \mathbb{P}\left\{ #1 \right\} }
\newcommand{\probp}[1]{ \mathbb{P}'\left\{ #1 \right\} }
\newcommand{\tProb}{{\tilde{\mathbb{P}}}}
\newcommand{\tprob}[1]{{ \tProb\left\{ #1 \right\} }}
\newcommand{\hProb}{\widehat{\mathbb{P}}}
\newcommand{\toprob}{\xrightarrow{\Prob}}
\newcommand{\tolp}[1]{\xrightarrow{L^{#1}}}
\newcommand{\toas}{\xrightarrow{{\rm a.s.}}}
\newcommand{\toae}{\xrightarrow{{\rm a.e.}}}
\newcommand{\todistr}{\xrightarrow{{\rm D}}}
\newcommand{\toweak}{\rightharpoonup}
\newcommand{\var}{\mathsf{var}}
% \newcommand{\Cov}{\text{Cov}}
\newcommand\indep{\protect\mathpalette{\protect\independenT}{\perp}}
\def\independenT#1#2{\mathrel{\rlap{$#1#2$}\mkern2mu{#1#2}}}
\newcommand{\Bern}{{\rm Bern}}
\newcommand{\Binom}{{\rm Binom}}
\newcommand{\Pois}{{\rm Pois}}
\newcommand{\Hyp}{{\rm Hyp}}
\newcommand{\eg}{e.g.\xspace}
\newcommand{\ie}{i.e.\xspace}
\newcommand{\iid}{i.i.d.\xspace}
% for prettyref.sty
\newrefformat{eq}{(\ref{#1})}
\newrefformat{chap}{Chapter~\ref{#1}}
\newrefformat{sec}{Section~\ref{#1}}
\newrefformat{alg}{Algorithm~\ref{#1}}
\newrefformat{fig}{Fig.~\ref{#1}}
\newrefformat{tab}{Table~\ref{#1}}
\newrefformat{rmk}{Remark~\ref{#1}}
\newrefformat{clm}{Claim~\ref{#1}}
\newrefformat{def}{Definition~\ref{#1}}
\newrefformat{cor}{Corollary~\ref{#1}}
\newrefformat{lmm}{Lemma~\ref{#1}}
\newrefformat{prop}{Proposition~\ref{#1}}
\newrefformat{app}{Appendix~\ref{#1}}
\newrefformat{hyp}{Hypothesis~\ref{#1}}
\newrefformat{thm}{Theorem~\ref{#1}}
\newrefformat{ass}{Assumption~\ref{#1}}
\newcommand{\ntok}[2]{{#1,\ldots,#2}}
\newcommand{\pth}[1]{\left( #1 \right)}
\newcommand{\qth}[1]{\left[ #1 \right]}
\newcommand{\sth}[1]{\left\{ #1 \right\}}
\newcommand{\bpth}[1]{\Bigg( #1 \Bigg)}
\newcommand{\bqth}[1]{\Bigg[ #1 \Bigg]}
\newcommand{\abth}[1]{\left | #1 \right |}
\newcommand{\bsth}[1]{\Bigg\{ #1 \Bigg\}}
\newcommand{\norm}[1]{\left\|{#1} \right\|_2}
\newcommand{\Norm}[1]{\|{#1} \|_2}
\newcommand{\lpnorm}[1]{\left\|{#1} \right\|_{p}}
\newcommand{\linf}[1]{\left\|{#1} \right\|_{\infty}}
\newcommand{\lnorm}[2]{\left\|{#1} \right\|_{{#2}}}
\newcommand{\Lploc}[1]{L^{#1}_{\rm loc}}
\newcommand{\hellinger}{d_{\rm H}}
\newcommand{\Fnorm}[1]{\lnorm{#1}{\rm F}}
\newcommand{\fnorm}[1]{\|#1\|_{\rm F}}
\newcommand\defeq{\stackrel{\mathclap{\normalfont\mbox{def}}}{=}}
\newcommand\iidsim{\stackrel{\mathclap{\normalfont\mbox{i.i.d.}}}{$\sim$}}
% \newcommand{\opnorm}[1]{\norm{#1}{\rm op}}
\newcommand{\opnorm}[1]{\left\| #1 \right\|_2}
% inner product
\newcommand{\iprod}[2]{\left \langle #1, #2 \right\rangle}
\newcommand{\Iprod}[2]{\langle #1, #2 \rangle}
% 12/02/2007
\newcommand{\indc}[1]{{\mathbf{1}_{\left\{{#1}\right\}}}}
\newcommand{\Indc}{\mathbf{1}}
\newcommand{\diag}[1]{\mathsf{diag} \left\{ {#1} \right\} }
\newcommand{\degr}{\mathsf{deg} }

\newcommand{\dTV}{d_{\rm TV}}
% \newcommand{\tb}{\widetilde{b}}
% \newcommand{\tr}{\widetilde{r}}
% \newcommand{\tc}{\widetilde{c}}
% \newcommand{\tu}{{\widetilde{u}}}
% \newcommand{\tv}{{\widetilde{v}}}
% \newcommand{\tx}{{\widetilde{x}}}
% \newcommand{\ty}{{\widetilde{y}}}
% \newcommand{\tz}{{\widetilde{z}}}
% \newcommand{\tA}{{\widetilde{A}}}
% \newcommand{\tB}{{\widetilde{B}}}
% \newcommand{\tC}{{\widetilde{C}}}
% \newcommand{\tD}{{\widetilde{D}}}
% \newcommand{\tE}{{\widetilde{E}}}
% \newcommand{\tF}{{\widetilde{F}}}
% \newcommand{\tG}{{\widetilde{G}}}
% \newcommand{\tH}{{\widetilde{H}}}
% \newcommand{\tI}{{\widetilde{I}}}
% \newcommand{\tJ}{{\widetilde{J}}}
% \newcommand{\tK}{{\widetilde{K}}}
% \newcommand{\tL}{{\widetilde{L}}}
% \newcommand{\tM}{{\widetilde{M}}}
% \newcommand{\tN}{{\widetilde{N}}}
% \newcommand{\tO}{{\widetilde{O}}}
% \newcommand{\tP}{{\widetilde{P}}}
% \newcommand{\tQ}{{\widetilde{Q}}}
% \newcommand{\tR}{{\widetilde{R}}}
% \newcommand{\tS}{{\widetilde{S}}}
% \newcommand{\tT}{{\widetilde{T}}}
% \newcommand{\tU}{{\widetilde{U}}}
% \newcommand{\tV}{{\widetilde{V}}}
% \newcommand{\tW}{{\widetilde{W}}}
% \newcommand{\tX}{{\widetilde{X}}}
% \newcommand{\tY}{{\widetilde{Y}}}
% \newcommand{\tZ}{{\widetilde{Z}}}


\newcommand{\sfa}{{\mathsf{a}}}
\newcommand{\sfb}{{\mathsf{b}}}
\newcommand{\sfc}{{\mathsf{c}}}
\newcommand{\sfd}{{\mathsf{d}}}
\newcommand{\sfe}{{\mathsf{e}}}
\newcommand{\sff}{{\mathsf{f}}}
\newcommand{\sfg}{{\mathsf{g}}}
\newcommand{\sfh}{{\mathsf{h}}}
\newcommand{\sfi}{{\mathsf{i}}}
\newcommand{\sfj}{{\mathsf{j}}}
\newcommand{\sfk}{{\mathsf{k}}}
\newcommand{\sfl}{{\mathsf{l}}}
\newcommand{\sfm}{{\mathsf{m}}}
\newcommand{\sfn}{{\mathsf{n}}}
\newcommand{\sfo}{{\mathsf{o}}}
\newcommand{\sfp}{{\mathsf{p}}}
\newcommand{\sfq}{{\mathsf{q}}}
\newcommand{\sfr}{{\mathsf{r}}}
\newcommand{\sfs}{{\mathsf{s}}}
\newcommand{\sft}{{\mathsf{t}}}
\newcommand{\sfu}{{\mathsf{u}}}
\newcommand{\sfv}{{\mathsf{v}}}
\newcommand{\sfw}{{\mathsf{w}}}
\newcommand{\sfx}{{\mathsf{x}}}
\newcommand{\sfy}{{\mathsf{y}}}
\newcommand{\sfz}{{\mathsf{z}}}
\newcommand{\sfA}{{\mathsf{A}}}
\newcommand{\sfB}{{\mathsf{B}}}
\newcommand{\sfC}{{\mathsf{C}}}
\newcommand{\sfD}{{\mathsf{D}}}
\newcommand{\sfE}{{\mathsf{E}}}
\newcommand{\sfF}{{\mathsf{F}}}
\newcommand{\sfG}{{\mathsf{G}}}
\newcommand{\sfH}{{\mathsf{H}}}
\newcommand{\sfI}{{\mathsf{I}}}
\newcommand{\sfJ}{{\mathsf{J}}}
\newcommand{\sfK}{{\mathsf{K}}}
\newcommand{\sfL}{{\mathsf{L}}}
\newcommand{\sfM}{{\mathsf{M}}}
\newcommand{\sfN}{{\mathsf{N}}}
\newcommand{\sfO}{{\mathsf{O}}}
\newcommand{\sfP}{{\mathsf{P}}}
\newcommand{\sfQ}{{\mathsf{Q}}}
\newcommand{\sfR}{{\mathsf{R}}}
\newcommand{\sfS}{{\mathsf{S}}}
\newcommand{\sfT}{{\mathsf{T}}}
\newcommand{\sfU}{{\mathsf{U}}}
\newcommand{\sfV}{{\mathsf{V}}}
\newcommand{\sfW}{{\mathsf{W}}}
\newcommand{\sfX}{{\mathsf{X}}}
\newcommand{\sfY}{{\mathsf{Y}}}
\newcommand{\sfZ}{{\mathsf{Z}}}


\newcommand{\calA}{{\mathcal{A}}}
\newcommand{\calB}{{\mathcal{B}}}
\newcommand{\calC}{{\mathcal{C}}}
\newcommand{\calD}{{\mathcal{D}}}
\newcommand{\calE}{{\mathcal{E}}}
\newcommand{\calF}{{\mathcal{F}}}
\newcommand{\calG}{{\mathcal{G}}}
\newcommand{\calH}{{\mathcal{H}}}
\newcommand{\calI}{{\mathcal{I}}}
\newcommand{\calJ}{{\mathcal{J}}}
\newcommand{\calK}{{\mathcal{K}}}
\newcommand{\calL}{{\mathcal{L}}}
\newcommand{\calM}{{\mathcal{M}}}
\newcommand{\calN}{{\mathcal{N}}}
\newcommand{\calO}{{\mathcal{O}}}
\newcommand{\calP}{{\mathcal{P}}}
\newcommand{\calQ}{{\mathcal{Q}}}
\newcommand{\calR}{{\mathcal{R}}}
\newcommand{\calS}{{\mathcal{S}}}
\newcommand{\calT}{{\mathcal{T}}}
\newcommand{\calU}{{\mathcal{U}}}
\newcommand{\calV}{{\mathcal{V}}}
\newcommand{\calW}{{\mathcal{W}}}
\newcommand{\calX}{{\mathcal{X}}}
\newcommand{\calY}{{\mathcal{Y}}}
\newcommand{\calZ}{{\mathcal{Z}}}
\newcommand{\indstate}{\State\hskip 1.5em}

\newcommand{\comp}[1]{{#1^{\rm c}}}
\newcommand{\Leb}{{\rm Leb}}
\newcommand{\Th}{{\rm th}}

\newcommand{\PDS}{{\sf PDS}\xspace}
\newcommand{\PC}{{\sf PC}\xspace}
\newcommand{\BPDS}{{\sf BPDS}\xspace}
\newcommand{\BPC}{{\sf BPC}\xspace}
\newcommand{\DKS}{{\sf DKS}\xspace}
\newcommand{\ML}{{\rm ML}\xspace}
\newcommand{\SDP}{{\rm SDP}\xspace}
\newcommand{\SBM}{{\sf SBM}\xspace}

\newcommand{\ER}{Erd\H{o}s-R\'enyi\xspace}

% \newcommand{\Tr}{\mathsf{Tr}}
\newcommand{\median}{\mathsf{med}}
\newcommand{\vol}{\mathsf{vol}}
\newcommand{\stab}{\mathsf{stab}}
\newcommand{\im}{\mathsf{i}}
% \newcommand{\sign}{\mathsf{sign}}
\newcommand{\agg}{\mathsf{agg}}
\newcommand{\clip}{\mathsf{clip}}
\newcommand{\dist}{\mathsf{dist}}


\renewcommand{\hat}{\widehat}
\renewcommand{\tilde}{\widetilde}

\makeatletter
\newcommand*{\addFileDependency}[1]{% argument=file name and extension
  \typeout{(#1)}
  \@addtofilelist{#1}
  \IfFileExists{#1}{}{\typeout{No file #1.}}
}
\makeatother

\newcommand*{\myexternaldocument}[1]{%
    \externaldocument{#1}%
    \addFileDependency{#1.tex}%
    \addFileDependency{#1.aux}%
}
% New commands
\newcommand{\incr}{\,\mathrm{d}}
\newcommand{\set}[1]{\{#1\}}
\newcommand{\diff}[2]{\frac{\mathrm{d}{#1}}{\mathrm{d}{#2}}}
\newcommand{\pdiff}[2]{\frac{\partial{#1}}{\partial{#2}}}
\newcommand{\ndiff}[3][]{\frac{\mathrm{d}^{#1}{#2}}{\mathrm{d}{#3}^{#1}}}
\newcommand{\npdiff}[3][]{\frac{\partial^{#1}{#2}}{\partial{#3}^{#1}}}
\newcommand{\R}{\mathbb R}

% New environments
\newtheorem{remark}{Remark}

% Bibliography
\addbibresource{references.bib}

% \doublespacing

\begin{document}

	\maketitle

	\abstract{
		We investigate a recent semantics for intermediate (and modal) logics in terms of polyhedra. The main result is a finite axiomatisation of the intermediate logic of the class of all polytopes --- i.e.,  compact convex polyhedra --- denoted $\PL$. This logic is defined in terms of the Jankov-Fine formulas of two simple frames. Soundness of this axiomatisation requires extracting the geometric constraints imposed on  polyhedra by the two formulas, and then using substantial classical results from polyhedral geometry to show that convex polyhedra satisfy those constraints.  To establish completeness of the axiomatisation, we first define the notion of the geometric realisation of a frame into a polyhedron. We then show that any $\PL$ frame is a p-morphic image of one which has a special form: it is a `sawed tree'. Any sawed tree has a geometric realisation into a convex polyhedron, which completes the proof.
	}

	% Keywords and classification codes
	\renewcommand{\thefootnote}{}
	\footnote{\emph{Keywords}: polyhedral semantics, convex polyhedron, intermediate logic, modal logic, polyhedral completeness, Kripke frame, PL homeomorphism, polyhedral map, convex geometric realisation, Heyting algebra}
	\footnote{\emph{2020 Mathematics Subject Classification}: 03B55 (Primary), 52B05, 06A07, 03B45, 06D20 (Secondary)}
	\renewcommand{\thefootnote}{\arabic{footnote}}
	\addtocounter{footnote}{-2}

	% Figure environment removed

\section{Introduction}
Automatic 3D reconstruction of clothed humans using image inputs has gained increasing significance due to its potential applications in a wide array of AR/VR scenarios. High-fidelity reconstructions typically depend on sophisticated capture systems, which are developed with dense camera arrays~\cite{collet2015high,joo2015panoptic,joo2018total}, programmable light-stages~\cite{Vlasic2009, guo2019relightables}, and depth sensors~\cite{newcombe2011kinectfusion,DoubleFusion,BodyFusion,dou2016fusion4d,newcombe2015dynamicfusion}. However, stringent capture environments equipped with complex hardware pose significant challenges for consumer-level applications.


In this context, considerable research effort has been dedicated to developing methods that allow for more flexible capture configurations, such as utilizing a few RGB inputs. Among these works, learning implicit functions \cite{iccv2020PIFu, saito2020pifuhd, hong2021stereopifu} has proven effective in achieving highly detailed reconstructions by integrating the advancements of deep neural networks. These methods employ large multi-layer perceptrons (MLPs) to predict the occupancy probability or truncated signed distance function (TSDF) value of every queried 3D point based on its associated local feature, which is extracted from images. They can recover a continuous surface at arbitrary resolutions without topology restrictions.


However, in typical MLP-based implicit networks, the occupancy or TSDF value at each location is solved independently with planar image features, rendering them less capable of addressing challenging cases such as occlusions. Consequently, these methods suffer from generalization and robustness issues, particularly when tackling strong occlusions caused by large motion or multiple interacting humans. 
Some follow-up studies  \cite{zheng2021deepmulticap,zheng2021pamir,huang2020arch} utilize an extra geometric model, SMPL~\cite{Loper2015}, to improve robustness by introducing strong shape priors. 
Their success typically relies on the assumption of geometrical similarity \cite{huang2020arch} between the shape prior and target reconstruction, making them intractable for handling complex cases with loose clothes and sensitive to errors in SMPL model fitting.



%\ping{this paragraph sounds like `TSDF is better than MLP/SMPL, and we use TSDF to solve the problem'. But in Sec 3, we are telling a different story, saying `MLP needs a 3D convolutional encoder'. We need to make these two sections consistent.}\sicong{I think in this paragraph we claim that the TSDF}


%We opt for Trucated Signed Distance Funtion (TSDF) volumetric representations as they are naturally suitable for convolution operations, which have shown remarkable performance for learning hierarchical features on 2D visual perception tasks \cite{SunXLW19}. 
%Meanwhile, TSDF also describes the gradual geometry change around shape surface, which is not reflected by occupancy volume. 

We instead revisit the 3D volumetric representation and resort to 3D convolutional neural networks (CNNs) for feature learning, due to their impressive performance in feature learning and the ability to incorporate spatial context. However, volumetric methods and 3D convolution involve discretization, which might raise concerns regarding whether a discretized volume can preserve subtle geometric details as continuous representations learned in implicit functions. We investigate the relationship between volume resolution and quantization error on synthetic data by converting target mesh objects to TSDF volumes, as shown in Figure~\ref{fig:quantization_error}. We observe that the quantization errors are significantly reduced by increasing volume resolution and become nearly negligible when reaching a relatively high resolution (e.g., 512 or higher). In other words, achieving fine-detailed reconstruction is not supposed to be restricted by the use of volume representations as long as a proper volume resolution is utilized. Therefore, we present a method with high-resolution feature volumes, e.g., 256 and 512, while traditional volumetric methods \cite{varol18_bodynet,gilbert2018volumetric} are often limited to much lower resolutions, such as 32 or 128.



On the other hand, an increase in volume resolution may lead to a cubic growth of memory overhead \cite{8100085}. Reducing memory costs while guaranteeing the granularity of volumetric representations is necessary for pursuing high-quality reconstruction. Thus, we adopt a coarse-to-fine approach and cull away irrelevant voxels to build a sparse high-resolution feature volume. At the coarse level, the network computes an initial TSDF by applying a U-Net with sparse 3D CNN \cite{3DSemanticSegmentationWithSubmanifoldSparseConvNet} on the sparse feature volume, which is carved by a visual hull. Through our experiments, it turns out that more than 95\% of the volume grids are discarded by the visual hull culling, making the sparse 3D CNN efficient. At the fine level, the network focuses on a narrow band near the zero-level set of the initial TSDF and discretizes the narrow band with smaller voxels. By employing this narrow-band culling, we further shrink the sampling space, resulting in a relatively small range of grid numbers (usually 300K--500K in our experiments) even with a high volume resolution of 512. The remaining voxels in the narrow band are associated with features that fuse high-frequency information from the computed normal maps upon the low-frequency shape from the coarse level to compute the TSDF at high resolution. The final mesh is then extracted from the TSDF using the Marching-Cube algorithm ~\cite{Lorensen87marchingcubes}.
% Different from the u-net sturcture to preserve global topology context, we then apply a shallow 3dcnn to compute the final TSDF $D_{final}$ which contain more local geometry detail.




% \ping{this paragraph can be expanded. It is an important contribution and often ignored by other works. stress on the novel idea of regressing blending weights instead of colors}

In addition to geometry, high-quality mesh texture is also a crucial factor contributing to visual appearance. Directly computing a color field in 3D space, as in \cite{iccv2020PIFu}, struggles to capture high-frequency texture details, while the neural radiance field (NeRF) \cite{yu2020pixelnerf} or the DoubleField~\cite{shao2022doublefield} require expensive per-instance optimization and are often unstable for sparse input images. In contrast, we adopt an image-based rendering approach to compute a texture atlas map, which is efficient and widely supported in existing computer graphics tools. 
Specifically, we compute a blending weight at each 3D point on the mesh surface to determine its color as a weighted average of the colors at its image projections. The blending weights can be computed at a relatively coarse resolution, e.g., 512 volume resolution in our case, and leave texture details to the high-resolution images, such as 1K or 2K. Unlike previous methods that generate blurry texturing results under sparse input, our method generalizes well on both synthetic and real data with just a few input views. 
Figure~\ref{fig:teaser} shows two examples reconstructed by our method. Despite the challenging garment, pose, and occlusion, our method recovers faithful shape, normal, and texture on the right.

%with a wide variety of poses and clothing styles, and it is also adaptive to handle input image with arbitrary resolutions.
%\sicong{For this concern we claim that when the resolution of dicretized volume meets certain threshold (which is 256 in our experiment), the quantization error can be neglected.} 



In summary, the main contributions of this paper are as follows:
\begin{itemize}
\vspace{-0.1in}
  \item 
  We revisit the 3D volumetric representation and demonstrate that it can support clothed human reconstruction with equal or even better performance compared to implicit representation. 
  \item 
  We develop a memory and computation-efficient method for high-resolution volumetric reconstruction using sophisticated sparse 3D CNN, coarse-to-fine estimation, and voxel culling by visual hull and narrow bands. 
  \item 
  We introduce a novel method to compute a texture atlas map, which captures rich appearance details from high-resolution input images.
  \item 
  We achieve impressive results on standard benchmark datasets Twindom and MultiHuman, significantly reducing the point-2-surface (P2S) precision to approximately 0.2cm from just six input views, with more than $50\%$ error reduction compared to the state-of-the-art methods, including DoubleField~\cite{shao2022doublefield} and PIFuHD~\cite{saito2020pifuhd}.
\end{itemize}
	\section{Preliminaries}

\subsection{\texorpdfstring{$(n, m,\ell, \beta)$}{(n, m, l, beta)} Set-System}
The constructions of the integrality gap instance in \Cref{sec:int-gap} and the reduction from label cover to GMP presented in \Cref{sec: reduction} both use the following set system as a building block. 
\begin{definition}[$(n, \m,  \ell, \bt)$ Set-system] \label{def: mlb-set-system}Let $n, m,\ell$ be positive integers, $\beta \in (0,1)$, $U$ be a set with $|U| = n$, and $A_1, \ldots,  A_{\m}$ be subsets of $U$. The sets $(U; A_1, \ldots, A_m)$ form an $(n, \m,  \ell, \bt)$ set-system if for every set $I$ of at most $ \ell$ indices from $[\m]$, $\left\lvert\cup_{i \in I} B_i\right\rvert \leq (1 - \bt) |U|$, where $B_i$ is either $A_i$ or $\overline{A}_i$. 
\end{definition}

Intuitively, an $(n, m, \ell, \beta)$ set-system has the property that any set cover which uses at most $\ell$ subsets necessarily uses a complementary pair of subsets $A_i$ and $\overline{A}_i$. Moreover, any collection of at most $\ell$ subsets that do not contain any complementary pair can cover at most a $(1-\beta)$ fraction of the elements in $U$. The following lemma shows that for a particular choice of parameters $n, m,\ell, $ and $\beta$, there is a simple and efficient randomized construction of an $(n, m, \ell, \beta)$ set-system.

\begin{lemma}\label{lem: mlb-set-system}
    For a sufficiently large positive integer $n$ and a positive integer $m \in [\sqrt{\log n}, 2\sqrt{\log n}]$ there exists an $(n, \m, \ell,\bt)$ set-system $(U; A_1, \ldots, A_m)$
 with $\,\,\ell = m/10$ and $\bt = \exp(-m) = \exp(-O(\sqrt{\log n}))$. There is a polynomial-time algorithm that constructs such a set system with high probability.
\end{lemma}
\begin{proof}
    Let $U$ be a set of $n$ elements, and initialize $m$ empty sets $A_1, \ldots, A_m$. For each element $e \in U$, sample a random index set $J \subset [\m]$ of size exactly $m/2$ and add $e$ to sets $A_j$ for $j \in J$.

We show that this construction gives an $(n, m,  \ell, \bt)$ set-system with high probability.  
    Consider an index set $I \subset[m]$ with $|I| = \ell$ and a collection of sets $B_i$ for $i \in I$ such that each $B_i$ is either $A_i$ or $\overline{A}_i$. For a fixed $e \in U$, let $p$ denote the probability that $e$ is not contained in $\cup_{i \in I} B_i$, i.e., $p = \Pr[e \notin \cup_{i \in I} B_i]$. We have,
    \[
         p \geq \binom{m-\ell}{m/2}/\binom{m}{m/2}
        \geq \left(\frac{m-\ell}{m/2}\right)^{m/2}/\left(\frac{em}{m/2}\right)^{m/2} 
        \geq \left(\frac{m-\ell}{em} \right)^{m/2}
        \geq \exp(-0.6 m),
    \]
where the second inequality uses that  $\left(\frac{n}{k}\right)^k \leq \binom{n}{k} \leq \left(\frac{en}{k}\right)^k$.

The probability that  $\cup_{i \in I}B_i$  contains a fixed subset of cardinality greater than or equal to $(1-\beta)n$ is at most $(1 - p)^{(1-\beta)n}$. By a union bound over at most $ \binom{n}{\beta n}n$ possible subsets with cardinality at least $(1-\beta)n$,
    \begin{align*}\Pr\left[\lvert \cup_{i \in I}B_i \rvert \geq (1 - \bt) n\right] &\leq \binom{n}{\bt n} n \cdot (1 - p)^{(1-\bt)n}
    \leq \left(e/\bt\right)^{\bt n}n \cdot e^{-(1-\bt)np}\\
    &\leq \exp\left(3n\bt  \log(1/\bt) - np/2\right) \leq \exp(-np/4) \leq \exp(-n^{0.9}),
    \end{align*}
where in the second to last inequality we use that $3\bt \log(1/\bt) < p/4$.

A union bound over the at most $2^\ell \cdot \binom{m}{\ell} \leq \exp( \sqrt{\log n})$ possible choices to pick the $\ell$ sets  $B_i$, gives that with high probability, the union of any $\ell$ sets $B_i$ has cardinality less than $(1-\bt)n$. 
\end{proof}

\subsection{Label Cover}
In \Cref{sec: reduction}, we prove the hardness of approximation of GMP via a reduction from the standard label cover problem as defined below. 
\begin{definition} \label{def: label-cover}
A \emph{label cover} instance $\mathcal{L}$ is defined by a tuple $((U,V, E), \labels, \Pi)$. Here $(U,V,E)$ is a bipartite graph with vertices $U \cup V$ and edges $E \subseteq U \times V$; $\labels$ is a positive integer and $\Pi$ is a set of functions one for each edge $e \in E$ i.e., $\Pi = \{\pi_e: [\labels] \rightarrow [\labels] \,\, |\,\, e \in E \}$. A labeling of the vertices $\sigma: U \cup V \rightarrow [\labels]$ is said to satisfy an edge $e = (u,v)$ if $\pi_e(\sigma(u)) = \sigma(v)$. Given $\mathcal{L}$, the goal of the label cover problem is to find a labeling $\sigma^*$ that satisfies the maximum number of edges in $E$. We use $OPT(\mathcal{L})$ to denote the fraction of the edges in $E$ satisfied by $\sigma^*$.
\end{definition}

As we will need the explicit dependence between the number of labels and the soundness, for completeness we 
sketch below the precise gap version of the label cover problem that we will use. 
\begin{lemma}[Hardness of  Gap  Label Cover] \label{lem: hardness-of-label-cover}
Given a label cover instance $\mathcal{L} = ((U,V, E), \labels, \Pi)$ satisfying:
\begin{enumerate}
  \item[(i)] $|U| = |V| = \lsize$
  \item[(ii)]  The degree of every vertex in $U \cup V$ is $d = O((\log \lsize)^{c_1})$ for some constant $c_1$.
  \item[(iii)]
  $\labels = \sqrt{\log \lsize}$
\end{enumerate}
There is some constant $\razconstant > 0$, for which there is no polynomial-time algorithm to decide if $OPT(\mathcal{L}) = 1$ or $OPT(\mathcal{L}) \leq (\log \lsize)^{-\razconstant}$ provided that $\emph{\textsf{NP}} \not\subseteq \emph{\textsf{DTIME}}(\lsize^{O(\log \log \lsize)})$.
\end{lemma}
\begin{proof}
Using a standard argument (see for ex., \cite{feige1996threshold, arora1996hardness}) one can obtain a reduction from a 3SAT-5 instance $\phi$ with $\satvars$ variables to a Label Cover instance $\mathcal{L}_1 = ((U_1, V_1, E_1), 8, \Pi)$, where $|U_1| = |V_1|=O(\satvars)$ and the graph $(U_1, V_1, E)$ is $15$-regular. The  instance $\mathcal{L}_1$ has the following property: if $\phi$ has a satisfying assignment then $OPT(\mathcal L_1) = 1$; else if any assignment satisfies at most $(1-\epsilon)$ fraction of clauses in $\phi$, then $OPT(\mathcal{L}_1) \leq (1 - \Theta(\epsilon))$. By the PCP-theorem \cite{10.1145/278298.278306} it follows that, for some constant $\epsilon_0 > 0$, deciding if $OPT(\mathcal L_1) = 1$ or $OPT(\mathcal L_1) \leq 1 - \epsilon_0$ is \textsf{NP}-hard.


The following well-known construction \cite{arora1996hardness}
gives stronger inapproximability results for label cover. We define the $k$th power of the label cover instance $\mathcal{L}_k = ((U_k,V_k,E_k), 8^k, \Pi^k)$, where $U_k$, $V_k$ are $k$-tuples of vertices in $U_1$, $V_1$ respectively, $E_k$ is the set of all $k$-tuples of edges in $E_1$. The resulting graph has $N = \satvars^{O(k)}$ vertices and is $(15)^k$-regular. The new set of labels\footnote{The labels are essentially numbers from $1$ to $8^k$.} consist of $k$-tuples of $\{1,\ldots,8\}$.  For an edge $e = (e_1, \ldots, e_k) \in E_k$, we define the function $\pi_e^k(a_1, \ldots, a_k) = (\pi_{e_1}(a_1), \ldots, \pi_{e_k}(a_k))$. Raz's Parallel Repetition Theorem \cite{raz1995parallel}, shows that for the label cover instance constructed above, there exists a constant $\alpha$ such that $OPT(\mathcal L_k) \leq (OPT(\mathcal L_1))^{\alpha k}$. 

We now pick $k$ so that $L = \sqrt{\log N}$. Since $L = 8^k$ and $N = \satvars^{O(k)}$, this gives $k = \Theta(\log \log \satvars)$. This choice of $k$ ensures that $d = (15)^k =(\log N)^{c_1}$ for some constant $c_1$. Moreover, if $OPT(\mathcal{L}_1) \leq (1-\epsilon_0)$, then $OPT(\mathcal L_k) \leq (1-\epsilon_0)^{\alpha k} \leq (\log t)^{-c'} \leq (\log N)^{-\razconstant}$ for some positive constants $\razconstant, c'$.
\end{proof}

	%!TEX root = ../convex-paper.tex

\section{Polyhedral semantics}
\label{sec:polyhedral semantics}

With the preliminaries in place, we are in a position to illustrate the link between intuitionistic logic and polyhedra that is the main focus of this paper. Given a polyhedron $P$, let $\Sub P$ denote the collection of its subpolyhedra.

\begin{theorem}\label{thm:subP co-Heyting algebra}
	$\Sub P$ is a co-Heyting algebra, and a subalgebra of $\Closeds(P)$.
\end{theorem}

\begin{proof}
	See \cite[Corollary~3.8]{tarski-polyhedra}. 
\end{proof}

Any subpolyhedron of $P$ is by definition compact, and hence closed. Therefore it is not surprising, once the algebraic nature of $\Sub P$ is established, that it turns out to be a \emph{co-Heyting} algebra. In topology and logic, on the other hand, it is more conventional to work with open sets and \emph{Heyting} algebras. Thus, it is natural at this point to switch to the Heyting algebra dual to $\Sub P$, which has the following concrete realisation.

Given a polyhedron $P$, we will define an \emph{open subpolyhedron} of $P$ as the complement (in $P$) of a  subpolyhedron of $P$; that is, $O\subseteq P$ is an open subpolyhedron of $P$ precisely when the set-theoretic difference $P\setminus O$ is a member of $\Sub P$. 
\begin{remark}Let $P\subseteq \R^n$ be any polyhedron. It is worth pointing out explicitly that while a subpolyhedron of $P$ is a closed (and compact) set both in $P$ and in the ambient space $\R^n$, an open subpolyhedron of $P$ is by definition open in $P$ but may fail to be open in $\R^n$.
\end{remark}
Let us denote by $\Subo P$ the collection of open subpolyhedra in $P$. It is evidently the dual of $\Sub P$, and \cref{thm:subP co-Heyting algebra} yields the following.

\begin{theorem}\label{thm:suboP Heyting algebra}
	$\Subo P$ is a Heyting algebra, and a subalgebra of $\Opens(P)$.
\end{theorem}

The above  provides a sound semantics for intuitionistic logic in terms of polyhedra: for a polyhedron $P$, say that $P \vD \phi$ if and only if $\Subo P \vD \phi$ as a Heyting algebra. One of the features of this polyhedral semantics  is that it is complete for $\IPC$ --- à la Tarski. Moreover, in contrast with topological semantics, polyhedral semantics can detect dimension, via the bounded depth schema. Let $\Poly$ denote the class of all polyhedra, and let $\Poly_n$ denote the subclass consisting of polyhedra of dimension at most $n$, for each $n \in \NN$.

\begin{theorem}\label{thm:IPC logic of P and BDn logic of Pn}
	\begin{enumerate}[label=(\arabic*)]
		\item\label{item:IPC; thm:IPC logic of P and BDn logic of Pn} 
			$\IPC = \Logic(\Poly)$. That is, intuitionistic logic is complete with respect to the class of all polyhedra.
		\item\label{item:BDn; thm:IPC logic of P and BDn logic of Pn} 
			$\BD_n = \Logic(\Poly_n)$, for each $n \in \NN$.
	\end{enumerate}
\end{theorem}

\begin{proof}
	See \cite[Theorem~1.1]{tarski-polyhedra}. The proof works by showing that every finite poset of height $n$ can be `realised geometrically' in an $n$-dimensional polyhedron. The main idea behind this construction is recalled in \cref{ssec:geometric realisation} below.
\end{proof}


	The Triangulation Lemma provides a key piece of information about the polyhedral semantics of Theorem \ref{thm:IPC logic of P and BDn logic of Pn} --- namely, $\Subo P$ is a locally finite Heyting algebra\footnote{An algebraic structure is \emph{locally finite} if every finitely generated substructure is finite.} for any polyhedron $P$. Given any triangulation $\Sig$ of $P$, denote by $\Pc(\Sig)$ the sublattice of $\Closeds(P)$ generated by $\Sig$, and let:
	\begin{equation*}
		\Po(\Sig) \coloneqq \{P \setminus C \mid C \in \Pc(\Sig)\}
	\end{equation*}

	\begin{lemma}\label{lem:Po Sig iso Up Sig}
		$\Po(\Sig)$ is isomorphic as a Heyting algebra to $\Up \Sig$.
	\end{lemma}

	\begin{proof}
		See \cite[Lemma~4.3]{tarski-polyhedra}.
	\end{proof}

	\begin{theorem}\label{thm:Subo P locally-finite}
		Whenever $P \nvD \phi$ there is a triangulation $\Sig$ of $P$ such that $\Po(\Sig) \nvD \phi$. In particular, $\Subo P$ is locally finite.
	\end{theorem}

	\begin{proof}
		See \cite[Corollary~3.7]{tarski-polyhedra}.
	\end{proof}


	%!TEX root = ../convex-paper.tex

\section{Logic, polyhedra and morphisms}
\label{sec:polyhedral maps}

In this section we develop assorted functorial aspects of  polyhedral semantics for intermediate logics which are essential ingredients in the main findings of the present paper.


\subsection{Homomorphisms induced by maps of spaces}

We begin with a  result that  requires some preliminary technical definitions.

For $X$ a topological space, by a \emph{lattice basis} for $X$ we mean a sublattice $L$ of the topology $\Opens(X)$ of $X$ that is a basis for that topology. If $L$ is moreover a Heyting subalgebra of the Heyting algebra $\Opens(X)$, we call $L$ a \emph{Heyting basis}.  

If $X$ is a space with a specified Heyting basis $L$ then we define
\[
\Logic(X)\coloneqq \Logic(L),
\]
where in the left-hand side we assume the basis $L$ is understood from context.

For any set $A$, write $\P(A)$ for the complete Boolean algebra of all subsets of $A$. For any function $f\colon A\to B$ between sets, write $f^{-1}\colon\P(B)\to\P(A)$ for the inverse-image function --- given $S\subseteq B$, $f^{-1}[S]\coloneqq\{a \in A \mid f(a)\in S\}$. Then $f^{-1}$ is a homomorphism of Boolean algebras that moreover preserves arbitrary joins and meets.


Now consider spaces $X$ and $Y$ with prescribed lattice bases $L$ and $M$, respectively. A function $f\colon X\to Y$ is \emph{bases-continuous} if $f^{-1}[S]\in L$ for each $S\in M$. Such functions are, of course, continuous. In general, a  function $f\colon X \to Y$ is \emph{open} if $f[U]\in \Opens(Y)$ for each $U \in \Opens(X)$. When $X$ and $Y$ come with prescribed lattice bases $L$ and $M$, let us say that a function $f$ is \emph{bases-open} if $f[U]\in M$ for each $U \in L$. It is clear that such a bases-open  function is open, because the direct-image function $f[-]$ preserves arbitrary unions.

\begin{lemma}\label{lem:maps duality}
	Let  $f \colon X \to Y$ be a function between spaces $X$ and $Y$ with prescribed lattice bases $L$ and $M$, respectively. Write $f^{-1}[-]\colon \P(Y) \to \P(X)$ for the inverse-image function.
	\begin{enumerate}[label=(\arabic*)]
		\item\label{item:a; lem:maps duality} The function $f$ is bases-continuous if and only if $f^{-1}$ descends to a lattice homomorphism $f^* \coloneqq f^{-1}\colon M \to L$. When one of these two equivalent conditions is satisfied, $f$ being surjective implies that $f^*$ is injective.
		\item\label{item:b; lem:maps duality} Assume  further $L$ and $M$ are Heyting bases. Assume the function $f$ is bases-continuous and bases-open. Then $f^{-1}$ descends to a  homomorphism of Heyting algebras $f^*\colon  M \to L$. Moreover, if $f$ is injective then $f^*$ is surjective, and if $f$ is a bijection then $f^*$ is an isomorphism.
	\end{enumerate}
\end{lemma}

\begin{proof}
	Since $f^*$ is a homomorphism of Boolean algebras, the first assertion in \ref{item:a; lem:maps duality} follows from the definitions. For the second assertion in \ref{item:a; lem:maps duality}, suppose $f$ is surjective. Pick $U, V \in M$ distinct, and suppose without loss of generality there is $p \in U \setminus V$. Since $f$ is surjective, there is $x \in X$ with $f(x)=p$. Then $x \in f^{-1}[U]$ but $x \not \in f^{-1}[V]$, so $f^{-1}=f^*$ is injective.
	
	As for \ref{item:b; lem:maps duality}, let us first assume that $f$ is bases-continuous and bases-open, and take $U,V \in M$ with the aim of showing that $f^*(U \ra V) = f^*(U) \ra f^*(V)$. For the left-to-right inclusion, using the fact that $M$ is a basis and that $f^*=f^{-1}[-]$ commutes with Boolean operations, write (letting $\comp S$ denote the complement of $S$): 
	\[
	U\ra V=\Int(\comp U \cup V)=\bigcup\{O\in M \mid O\sse \comp U \cup V\}
	\]
	and:
	\[
	f^{-1}[U] \ra f^{-1}[V]=\Int\left(\comp{ f^{-1}[U]} \cup f^{-1}[V]\right)=\Int \left(f^{-1}[\comp U \cup V]\right).
	\]
	
	Since $f^{-1}[-]$ preserves arbitrary unions too, we obtain $f^{-1}[U\ra V]=\bigcup f^{-1}[O]$ for $O \in M$ ranging over subsets of  $\comp U \cup V$. Now $O\sse \comp U \cup V$ entails $f^{-1}[O]\sse f^{-1}[\comp U\cup V]$. Since $f^{-1}[O]$ is open because $f$ is continuous,  by the definition of interior $f^{-1}[O]\sse \Int(f^{-1}[\comp U\cup V])$, which shows $f^{-1}[U\ra V]\sse f^{-1}[U]\ra f^{-1}[V]$.
		
	For the right-to-left inclusion we have the following chain of inclusions.
	\begin{align*}
		f[f^{-1}[U] \ra f^{-1}[V]]
			&= f\left[\Int\left(\comp{f^{-1}[U]} \cup f^{-1}[V]\right)\right] \\
			&\sse \Int\left(f\left[\comp{f^{-1}[U]} \cup f^{-1}[V]\right]\right) \tag{$f$ is open} \\
			&= \Int\left(f\left[f^{-1}[\comp{U} \cup V]\right]\right) \\
			&\sse \Int(\comp U \cup V) \\
			&= U \ra V
	\end{align*}
	Applying $f^{-1}$ to both sides, we get that $f^{-1}[U] \ra f^{-1}[V] \sse f^{-1}[U \ra V]$. Summing up, $f^*(U \ra V) = f^*(U) \ra f^*(V)$.  


	Next, assume $f$ is injective. Let $A\in L$, and let us show $A$ has a pre-image along $f^{*}=f^{-1}$. Certainly  $A\subseteq f^{-1}[f[A]]$. Let us prove the converse inclusion. If $f^{-1}[f[A]]$ is empty then the converse inclusion holds; otherwise, pick $x \in f^{-1}[f[A]]$. Then $f(x)\in f[A]$, so there is $a\in A$ with $f(x)=f(a)$. Since $f$ is injective, $x=a\in A$, and thus $f^{-1}[f[A]]\subseteq A$. Hence $A$ has the pre-image $f[A]$ along $f^{-1}$. Since, moreover, $f$ is bases-open, we have $f[A]\in M$, so $f^*$ is indeed surjective.
	
	Finally, if $f$ is a bijection then by \ref{item:a; lem:maps duality} and what we just proved $f^*$ is a bijective isomorphism of Heyting algebras, and hence an isomorphism.
\end{proof}

\begin{lemma}\label{lem:subspace_quotient}
	Let $X$ be a space, let $L\subseteq \Opens(X)$, let $Y\subseteq X$, and set $M\coloneqq\{O\cap Y\mid O\in L\}$.
	\begin{enumerate}
	\item If $L$ is a (lattice) basis for the topology of $X$ then $M$ is a (lattice) basis for the subspace topology of $Y$.
	\item If $Y$ is open and $L$ is a Heyting basis for the topology of $X$ then $M$ is a Heyting basis for the subspace topology of $Y$.
	\end{enumerate}
\end{lemma}

\begin{proof}
	This is a straightforward verification and shall be omitted.
\end{proof}
	
To deploy Lemmas \ref{lem:maps duality} and \ref{lem:subspace_quotient} in our geometric setting we will require the next fact.

\begin{lemma}\label{lem:convex open subpolyhedra basis}
	The (convex) open subpolyhedra of a (convex) polyhedron $P$ form a basis for the topology on $P$. Moreover, for any polyhedron $P$, $\Subo P$ is a Heyting basis of $P$.
\end{lemma}

\begin{proof}
	Assume $P\subseteq \R^n$ is any polyhedron.
	Take any $x \in P$ and let $U$ be an open neighbourhood of $x$ in $P$. Then there is some open ball $B$ in $\R^n$ about $x$ such that $x \in B \cap P \sse U$. An elementary argument in affine geometry produces  a simplex $\sig$ in $\R^n$ such that $x \in \Relint\sig \sse B$. Then (by the Triangulation Lemma~\ref{lem:triangulation lemma}) the set $\Cl(P \setminus \sig)$ is a compact subpolyhedron of $P$. Its complement $P \cap \Relint \sig$ is therefore an open subpolyhedron of $P$. Furthermore,
	\begin{equation*}
		x \in P \cap \Relint \sig \sse U, 
	\end{equation*}
	which shows   $\Subo P$ is a basis. If $P$ is additionally convex, then $P \cap \Relint \sig$ is also convex because $P$ and $\Relint \sig$ are, which shows that the convex open subpolyhedra of a convex polyhedron form a basis.

	The `moreover' statement follows from the fact that  the basis $\Subo P$ is a Heyting subalgebra of $\Opens(P)$ by Theorem \ref{thm:suboP Heyting algebra}.
\end{proof}

\begin{remark}\label{rem:convention_bases}
	From now on, in light of Lemma \ref{lem:convex open subpolyhedra basis}, we always tacitly assume a polyhedron $P$ is equipped with its Heyting basis $\Subo P$. Also, in light of Lemma \ref{lem:subspace_quotient}, if $Q$ is an open polyhedron in $P$ --- that is, a member of $\Subo P$ for some polyhedron $P$ --- we always tacitly assume that $Q$ is equipped with the Heyting basis $\Subo Q\coloneqq\{O\cap Q \mid O \in \Subo P\}$.
\end{remark}

Finally, in the next definition we isolate the specific instance of basis-con\-ti\-nuous map that is crucial to our context.

\begin{definition}\label{d:polyhedralmap} 
	Let $P$ be a polyhedron and $Y$  a space with a lattice basis $M$. 
	\begin{enumerate*}[label=(\roman*)]
		\item A function $f \colon P \to Y$ is a \emph{polyhedral map} if it is bases-continuous with respect to the  bases $\Subo P$ and $M$, respectively.
		\item Further, let $Q$ be an open subpolyhedron of $P$. A function $f \colon Q \to Y$ is again called a \emph{polyhedral map} if the pre-image of any open set in $M$ is in $\Subo Q$ (see Remark \ref{rem:convention_bases}).
		\item In the special case that the co-domain $Y$ of $f$ is a  poset $F$, we always tacitly assume $M$ is the Heyting basis $\Up F$ of all open sets in the Alexandrov topology on $F$.
		\item When we say  a polyhedral map as in the foregoing items is \emph{open} we always mean it is \emph{bases}-open with respect to the indicated bases.
	\end{enumerate*}
\end{definition}


\subsection{Jankov-Fine, for polyhedra}

\Cref{thm:Jankov-Fine up-reductions} shows that Jankov-Fine formulas encode forbidden configurations for frames. The same is true for polyhedra with respect to polyhedral maps, as we now show.


Let $\Sig$ be a simplicial complex and $F$ a poset. Given any function $f \colon \Sig \to F$, define the map $\wh f \colon \abs\Sig \to F$ by:
\begin{equation*}
	\wh f(x) \coloneqq f(\sig^x)
\end{equation*}

\begin{lemma}\label{lem:p-morphism to open polyhedral map}
	When $f \colon \Sig \to F$ is a p-morphism, $\wh f \colon \abs\Sig \to F$ is an open polyhedral map.
\end{lemma}

\begin{proof}
	For any $U \in \Up F$, we have that:
	\begin{equation*}
		\wh f^{-1}[U] = \bigcup \{\Relint\sig \mid \sig \in \Sig\text{ and } \sig \in f^{-1}[U]\}
	\end{equation*}
	Since $f$ is monotonic, $f^{-1}[U]$ is upwards-closed in $\Sig$ and therefore $\wh f^{-1}[U]$ is an open sub-polyhedron of $\abs\Sig$. Now take an open set $W \sse \abs\Sig$, with the aim of showing that $\wh f[W]$ is open. Define:
	\begin{equation*}
		\Sig\#W \coloneqq \{\sig \in \Sig \mid \Relint(\sig) \cap W \neq \es\}
	\end{equation*}
	Then:
	\begin{equation*}
		\wh f[W] = \{f(\sig^x) \mid x \in W\} = f[\Sig\#W]
	\end{equation*}
	If $\sig \in \Sig\#W$ and $\sig \preceq \tau$, then as $\sig \sse \tau = \Cl\Relint\tau$ and $W$ is open, we have $\tau \in \Sig\#W$; i.e. $\Sig\#W$ is upwards-closed. But now, $f$ is open and so $\wh f[W]$ is also upwards-closed.
\end{proof}

\begin{lemma}\label{lem:Jankov-Fine polyhedral maps}
	Let $P$ be a polyhedron and $F$ a finite rooted frame. Then $P\nvD \chi(F)$ if and only if there exists an open subpolyhedron $Q$ of $P$ and a surjective open polyhedral map $f \colon Q\to F$. Moreover, if $P$ is convex, then we can assume without  loss of generality that $Q$ is also convex. 
\end{lemma}

\begin{proof}
	Let $P\nvD \chi(F)$. By \cref{thm:Subo P locally-finite} there is a triangulation $\Sig$ of $P$ such that $\Po(\Sig) \nvD \chi(F)$, which by \cref{lem:Po Sig iso Up Sig} means that $\Sig \nvD \chi(F)$. Hence by \cref{thm:Jankov-Fine up-reductions} there is an up-reduction $h \colon \Sig \cra F$. Note that $h$ is open (with respect to the Alexandrov topologies) by the definition of p-morphism. Let $H$ be the (upwards-closed) domain of $h$. As $F$ is rooted, $H$ can be assumed without  loss of generality to be rooted --- it suffices to take a pre-image $y$ of the root of $F$ and let $H = \us y$. Applying \cref{lem:p-morphism to open polyhedral map} to the identity map $\id \colon \Sig \to \Sig$ we find an open polyhedral map $\wh\id \colon P \to \Sig$. Let $Q$ be the pre-image of $H$ via $\wh\id$. Then $h\circ \wh\id \colon Q \to F$ is a surjective open polyhedral map.
	
	Now assume that $P$ is convex. Let $x$ be any element in the pre-image of the root of $H$, and note that $Q$ is an open neighbourhood of $x$. Hence by \cref{lem:convex open subpolyhedra basis} there is an open convex subpolyhedron $W\subseteq P$ such that $x \in W\subseteq Q$. Since $\wh\id$ is open, $\wh\id[W]$ is an upwards-closed subset of $H$ containing its root, and therefore $H = \wh\id[W]$. We have thus found a convex open subpolyhedron $W$ such that $h\circ \wh\id[W] = F$, as desired. 
	
	For the converse direction, as $F\nvD \chi(F)$ we obtain from \cref{lem:maps duality} that $Q\nvD \chi(F)$. Then  \cref{lem:subspace_quotient} implies that $\Subo Q$ is a quotient of $\Subo P$ (via the map $O\in P\mapsto O\cap Q\in\Subo Q$), and therefore $P\nvD \chi(F)$. 
\end{proof}

	
\subsection{PL maps}

For any $X\sse\R^m$, $Y \sse \R^n$, a function $X \to Y$ is an \emph{affine map} if it lifts to a map $\R^m\to\R^n$ of the form $x \mapsto Mx + b$, where $M$ is a linear transformation and $b \in \R^n$. Now let $P$  and $Q$ be polyhedra in $\R^m$ and $\R^n$, respectively. A function $f\colon P \to Q$ is \emph{piecewise linear}, or a \emph{PL map} for short, if there are triangulations $\Sigma$ and $\Delta$ of $P$ and $Q$ respectively such that
\begin{enumerate}[label=(\arabic*)]
	\item the function $f$  agrees on each $\sigma\in\Sigma$ with an affine map, and
	\item for each $\sigma\in\Sigma$, $f[\sigma]\in \Delta$.
\end{enumerate}
PL maps as just defined are automatically continuous. 

\begin{remark}\label{rem:PL_graph} 
	There are several characterisations, or equivalent definitions, of PL map; we mention one that we shall use, referring to \cite{rourkesanderson1972} for proofs: a function $f\colon P \to Q$ is PL if and only if it is continuous, and its graph  $\{(x,f(x)) \in \R^{m+n}\mid x \in P\}$ is a polyhedron.
\end{remark}

\begin{remark}\label{rem:PL_is_poly}
	A PL map is a polyhedral map because of the standard fact that the inverse image of a polyhedron under a PL-map is a polyhedron, cf. \cite[Corollary~2.5, p.~13]{rourkesanderson1972}. The converse is not true --- the map $[0,1]\to[0,1]$ given by $x\mapsto x^2$ is a polyhedral map that is not PL.
\end{remark}

A \emph{PL homeomorphism} is a PL map that is a homeomorphism.

\begin{lemma}\label{lem:inversePL}
	The inverse of a PL homeomorphism is a PL homeomorphism.
\end{lemma}

\begin{proof}
	See \cite[p.~6]{rourkesanderson1972}.
\end{proof}

\begin{corollary}\label{cor:PL homeomorphism HA isomorphism}
	A PL homeomorphism $f\colon P\to Q$ between polyhedra and its inverse $g\colon Q\to P$ induce mutually inverse isomorphisms of Heyting algebras $f^*\coloneqq f^{-1}\colon \Subo{Q} \to \Subo{P}$ and $g^*\coloneqq g^{-1}\colon \Subo{P} \to \Subo{Q}$.
\end{corollary}

\begin{proof}
	This is an immediate consequence of \cref{lem:maps duality} together with Lemma \ref{lem:inversePL} and \cref{rem:PL_is_poly}.
\end{proof}

\begin{corollary}\label{cor:PL homeomorphic implies logics same}
	If $P$ and $Q$ are PL homeomorphic then $\Logic(P)=\Logic(Q)$.
\end{corollary}


\subsection{Geometric realisation}
\label{ssec:geometric realisation}

The notion of `geometric realisation' can now be made more precise. Given a polyhedron and a space $Y$ with a Heyting basis $M$, a  \emph{realisation} of $Y$ in a polyhedron $P$ is an open surjective polyhedral map $f\colon P \to Y$. By  \cref{lem:maps duality} the dual map $f^*\colon M\to \Subo P$ is an injective homomorphism of Heyting algebras, and this entails $\Logic(P) \sse \Logic(Y)\coloneqq \Logic(M)$,  which is the key ingredient in the completeness proofs. 

Let us emphasise that our usage of the term `geometric realisation'  is specific to our setting. The map $f\colon P\to Y$ `realises' the Heyting algebra $M$ as a subalgebra of $\Subo P$  by pulling back inverse images along $f^*\coloneqq f^{-1}$. This applies in particular to the special case in which  $Y$ is a finite poset $F$, and $M$ is $\Up F$.   We shall next show how this notion of realisation for finite posets  relates to the  standard one of geometric realisation of a simplicial complex.

Let us see how to produce a geometric realisation for an arbitrary finite poset $F$ of height $n$, following \cite{tarski-polyhedra}. For this, we make use of the following construction coming from combinatorial geometry. The \emph{nerve} of $F$, denoted $\N(F)$ is the poset of all non-empty chains in $F$ ordered by inclusion. The nerve comes equipped with a p-morphism $\max \colon \N(F) \to F$ which sends a chain to its maximum element. Note also that $\height(\N(F)) = \height(F)$.

Using the nerve, we then define the geometric realisation of $F$ via a simplicial complex. Enumerate $F = \{x_1, \ldots, x_m\}$, and let $e_1, \ldots, e_m$ be the standard basis vectors of $\R^m$. The \emph{simplicial complex induced by} $F$ is defined:
\begin{equation*}
	\nabla F \coloneqq \{\Conv\{e_{i_1}, \ldots, e_{i_k}\} \mid \{x_{i_1}, \ldots, x_{i_k}\} \in \N(F)\}
\end{equation*}
Noting that $\nabla F \cong \N(F)$ as posets, the p-morphism $\max \colon \N(F) \to F$ then induces an open surjective polyhedral map $\abs{\nabla F} \to F$. Furthermore, by definition:
\begin{equation*}
	\Dim \abs{\nabla F} = \height(\N(F)) = n
\end{equation*}
In other words, we have an $n$-dimensional geometric realisation of the height-$n$ poset $F$, which is the main component in the proof of \cref{thm:IPC logic of P and BDn logic of Pn}.
	%!TEX root = ../convex-paper.tex

\section{The logic of convex polyhedra}
\label{sec:logic of convex polyhedra}

Recall from \cref{sec:preliminaries} that a polyhedron $P$ is \emph{convex} if $\Conv P = P$, in other words, if the segment joining any two points in $P$ lies entirely in $P$. Let $\PolyCon$ be the class of all convex polyhedra. We can now tackle the question: \emph{what is the logic of all convex polyhedra, $\Logic(\PolyCon)$?} The remainder of the paper will be devoted to a proof that $\Logic(\PolyCon) = \PL$, where $\PL$ is axiomatised by the Jankov-Fine formulas of two simple trees as follows.

\begin{equation*}
	\PL = \IPC + \chi(\FThreeFork) + \chi(\FScott)
\end{equation*}

\begin{theorem}\label{thm:PL logic of CP}
	$\PL$ is the logic of all convex polyhedra: $\PL = \Logic(\PolyCon)$.
\end{theorem}

We show this result by first restricting to the bounded dimension and bounded frame-depth situation, and then use the fact that $\PL$ has the finite model property to obtain the full result. Specifically, let $\PolyCon_n$ denote the class of convex polyhedra of dimension at most $n$, and define:
\begin{equation*}
	\PL_n \coloneqq \BD_n + \PL
\end{equation*}
The main job will be to prove the following.

\begin{theorem}\label{thm:PLn logic of CPn}
	$\PL_n = \Logic(\PolyCon_n)$, for each $n$.
\end{theorem}

\noindent This in turn splits into the following two directions, which will be proved in \cref{sec:soundness} and \cref{sec:completeness}, respectively.

\begin{theorem}[Soundness]\label{thm:PLn soundness for CPn}
	$\PL_n$ is valid on every $P \in \PolyCon_n$.
\end{theorem}

\begin{theorem}[Completeness]\label{thm:PLn completeness for CPn}
	If $\PL_n \nvd \phi$ then there is $P \in \PolyCon_n$ such that $P \nvD \phi$.
\end{theorem}

The final ingredient is the following result due to Zakharyaschev.

\begin{lemma}\label{lem:PL fmp}
	$\PL$ has the finite model property.
\end{lemma}

\begin{proof}
	This follows from the more general result \cite[Corollary 0.11, p.~20]{zakharyaschev93}. This result is stated in terms of `canonical formulas', which are a generalisation of Jankov-Fine formulas. Given a frame $Q$ and a set $\mathfrak D$ of antichains in $Q$ (sets of pairwise incompatible elements of $Q$), we can define the \emph{canonical formula} $\beta(Q, \mathfrak D, \bot)$, which satisfies a similar condition to that satisfied by Jankov-Fine formulas. The result states that if an intermediate logic $\Lo$ is axiomatised by a set of canonical formulas $\beta(Q, \mathfrak D, \bot)$ such that in every $A \in \mathfrak D$ there is at least one point not lying below all maximal points in $\uset A$, then $\Lo$ has the finite model property.
	
	Now, given any frame $Q$, the Jankov-Fine formula $\chi(Q)$ is equivalent to $\beta(Q, \mathfrak D^\#, \bot)$, where $\mathfrak D^\#$ is the set of non-singleton antichains in $Q$ \cite[Proposition 9.41 (i), p.~312]{chagrovzakharyaschev1997}. It is then clear to see that $\chi(\FThreeFork)$ and $\chi(\FScott)$ satisfy the requisite conditions, so the result yields that $\PL$ has the finite model property.
\end{proof}

These lemmas then combine to give the ultimate result.

\begin{proof}[Proof of \cref{thm:PL logic of CP}]
	\cref{lem:PL fmp} entails that:
	\begin{equation*}
		\PL = \bigcap_{n \in \NN} \PL_n
	\end{equation*}
	On the other hand, since all our polyhedra have finite dimension:
	\begin{equation*}
		\PolyCon = \bigcup_{n \in \NN} \PolyCon_n
	\end{equation*}
	Therefore:
	\begin{equation*}
		\Logic(\PolyCon) = \bigcap_{n \in \NN} \Logic(\PolyCon_n)
	\end{equation*}
	\cref{thm:PLn logic of CPn} then completes the proof.
\end{proof}


\subsection{The Logic of a single convex polyhedron}\label{ss:opensimplex}

Any two $n$-simplices $\sigma\subseteq\R^d$ and $\tau\subseteq \R^{d'}$ are PL-homeomorphic --- in fact, affinely homeomorphic. Indeed, since affine maps commute with affine combinations, any bijection of the vertex set of $\sigma$ onto the vertex set of $\tau$ lifts to exactly one bijective affine map $\Aff \sigma \to \Aff \tau$.  Let $e_0, \ldots, e_n$ be the standard basis vectors of $\R^{n+1}$. The \emph{standard $n$-simplex} is $\Delta_n \coloneqq \Conv\{e_0, \ldots, e_n\}$. The following is a classical result.

\begin{lemma}\label{lem:convex n-dimensional polyhedra pl homeomorphic to n-simplex}
	Every $n$-dimensional convex polyhedron is PL-homeomorphic to $\Delta_n$.
\end{lemma}

\begin{proof}
	See \cite[Corollary~2.20, p.~21]{rourkesanderson1972}. There it is shown that \emph{$n$-cells} --- which correspond to our $n$-dimensional convex polyhedra --- are \emph{$n$-balls} --- meaning that they are PL-homeomorphic to the $n$-dimensional cube $[0,1]^n$. Since $\Delta_n$ is a convex polyhedron, the result follows.
\end{proof}

Thus, the logic of all convex polyhedra of dimension at most $n$ is just the logic of any given $n$-dimensional such polyhedron, for instance the   $n$-simplex.

\begin{corollary}\label{cor:logic CPn is logic of n-simplex}
	For any $n$-dimensional convex polyhedron $P$,
	$\Logic(\PolyCon_n) = \Logic(\Delta_n)=\Logic(P)$. 
\end{corollary}

\begin{proof}
	This is immediate from \cref{lem:convex n-dimensional polyhedra pl homeomorphic to n-simplex} using \cref{cor:PL homeomorphic implies logics same}.
\end{proof}

Next, given a convex polyhedron $P$, we are interested in determining the logic of its topological interior in $\Aff P$ --- that is, the logic of a convex open polyhedron of dimension $n$. In the special case that $P$ is an $n$-simplex $\sigma$, its topological interior in $\Aff \sigma$ coincides with its relative interior $\Relint \sigma$.

\begin{lemma}\label{lem:map_of_cubes}
	There exists a surjective open polyhedral map $(0,1)^n\to [0,1]^n$.
\end{lemma}

\begin{proof}
	Let us first assume $n=1$. Consider real numbers $a'<x<a<b<y<b'$. We define a function $f\colon [a',b']\to [x,y]$ by prescribing its action on vertices:
	\[
	f(a')=a, f(b')=b, f(x)=x, f(a)=a,f(b)=b,f(y)=y\,,
	\]
	and  by completing the definition of $f$ through affine extension. Then $f$ is a surjective PL map. Its restriction $g$ to $(a',b')$ is  a polyhedral map that is evidently still surjective onto $[x,y]$, and is moreover open. (To verify $f$ is open let $(\alpha,\beta)\subseteq(a',b')$. If $x\leq \alpha$ and $\beta\leq y$ then $f[(\alpha,\beta)]=(\alpha,\beta)$. If $\alpha\leq x$ and $y\leq \beta$ then $f[(\alpha,\beta)]=[x,y]$. If $\alpha\leq x$ and $\beta\leq y$ then $f[(\alpha,\beta)]=[x,\beta)$. Hence $f$ is open.) This shows the existence of a surjective open polyhedral map $g\colon (0,1)\to [0,1]$ that is the restriction to $(0,1)$ of a PL map $[0,1]\to[0,1]$.

	For $n>1$, consider the product of maps $F\coloneqq f\times\cdots \times f\colon [0,1]^n\to [0,1]^n$ and its restriction to $(0,1)^n$, $G\coloneqq g\times \cdots \times g\colon (0,1)^n\to [0,1]^n$. Then $F$ is PL. Indeed, its graph is the $n$-fold product of copies of the graph of $f$, and the latter graph is a polyhedron because $f$ is PL; hence the graph of $F$ is a polyhedron, too, using the standard fact  that a finite product of polyhedra is a polyhedron. Since $F$ is continuous \cite[Proposition 2.3.6 and p.\ 78]{EngelkingRyszard1989Gt}, and its graph is a polyhedron, then  $F$ is PL (\cref{rem:PL_graph}). This  entails  that $G$ is polyhedral: if $O\in \Subo [0,1]^n$, $F^{-1}[O]\in \Subo [0,1]^n$ because $F$ is PL; then $G^{-1}[O]= F^{-1}[O]\cap (0,1)^n \in\Subo (0,1)^n$. Finally, since a finite product of open maps is open \cite[Proposition 2.3.29]{EngelkingRyszard1989Gt}, $G$ is open.
\end{proof}

\begin{lemma}\label{lem:open_polytope_logic}
	Let $P$ be any convex polyhedron, and let $O$ be its topological interior in $\Aff P$. Then $\Logic(P)=\Logic(O)$.
\end{lemma}

\begin{proof}
	Assume $P$ is of dimension $n$. By \cref{lem:convex n-dimensional polyhedra pl homeomorphic to n-simplex} there is a PL-homeomorphism $f\colon P\to \Delta_n$ with inverse $f^{-1}\colon \Delta_n\to P$ which also is PL (\cref{lem:inversePL}). Hence by \cref{cor:PL homeomorphic implies logics same} we have $\Logic(P)=\Logic(\Delta_n)$. By an elementary topological argument, $f$ and $f^{-1}$ descend to mutually inverse homeomorphisms  $g \colon O \to \Relint \Delta_n$ and $g^{-1}\colon \Relint \Delta_n \to O$. These homeomorphisms are polyhedral because $f$ and $f^{-1}$ are PL. Hence, \cref{lem:maps duality} entails $\Logic{O}=\Logic(\Relint \Delta_n)$. Thus it suffices to prove the lemma for $P=\Delta_n$ and $O=\Relint \Delta_n$.


	The inclusion map $\iota\colon\Relint \Delta_n \to \Delta$ is an injective open polyhedral map, so that its dual $\iota^*\colon \Subo \Delta_n\to \Subo \Relint \Delta_n$ is a surjective homomorphism of Heyting algebras by \cref{lem:maps duality}, which entails $\Logic(\Delta_n)\subseteq\Logic(\Relint \Delta_n)$. For the converse inclusion, \cref{lem:map_of_cubes} and \cref{lem:maps duality} entail $\Logic((0,1)^n)\subseteq \Logic([0,1]^n)$. The  argument in the previous paragraph yields $\Logic(\Delta_n)=\Logic([0,1]^n)$ and $\Logic(\Relint \Delta_n)=\Logic((0,1)^n)$, which completes the proof.
\end{proof}


\subsection{The largest logic}

The importance of convex polyhedra is mirrored on the logical side.

\begin{theorem}\label{thm:largest logic}
	\begin{enumerate}[label=(\arabic*)]
		\item\label{item:infty; thm:largest logic}
		$\PL$ is the largest polyhedrally complete logic of height $\infty$.
		\item\label{item:n; thm:largest logic}
		$\PL_n$ is the largest polyhedrally complete logic of height $n$, for each $n \in \NN$.
	\end{enumerate}
\end{theorem}

The starting point to prove the above theorem  is the observations that every $n$-dimensional polyhedron contains a convex polyhedron of that dimension.

\begin{lemma}\label{lem:convex in every poly}
	If $P$ is $n$-dimensional polyhedron and $m \leq n$ then there is $Q$ an $m$-dimensional convex polyhedron with $Q \subseteq P$.
\end{lemma}

\begin{proof}
	Let $\Sig$ be a triangulation of $P$. Since $P$ has dimension $n$, there is a simplex $\sig \in \Sig$ which has height $m$ (when viewing $\Sig$ as a poset). Then $\sig\subseteq P$ is an $m$-simplex, which is by definition convex.
\end{proof}

The remaining part of the proof  rests on the results of   \cref{ss:opensimplex}.

\begin{proof}[Proof of \cref{thm:largest logic}]
	To prove \ref{item:n; thm:largest logic}, let $\Lo$ be a polyhedrally complete logic of height $n$. Then $\Lo = \Logic(\C)$ for some class $\C$ of polyhedra. We claim that $\C$ contains a polyhedron of dimension at least $n$. Indeed, otherwise $\C \sse \Poly_{n-1}$ so that by \cref{thm:IPC logic of P and BDn logic of Pn} we have:
	\begin{equation*}
		\BD_{n-1} = \Logic(\Poly_{n-1}) \sse \Logic(\C) = \Lo
	\end{equation*}
	By \cref{lem:BDn specifies height} this means that $\Lo$ cannot have frames of height $n$, a contradiction. \contradiction

	So take $P \in \C$ of dimension at least $n$. Then by \cref{lem:convex in every poly} there is $Q$ a convex $n$-dimensional polyhedron with $Q \subseteq P$. Let $O$ be the topological interior of $Q$ in $\Aff Q$. The inclusion $O\subseteq P$ is an open injective polyhedral map, so by \cref{lem:maps duality} we have $\Logic(P) \sse\Logic(O)$. But by \cref{lem:open_polytope_logic} we also have $\Logic(O)=\Logic(Q)$, and by \cref{cor:logic CPn is logic of n-simplex} we know $\Logic(Q) = \PL_n$; hence:
\begin{equation*}
		\Lo = \Logic(\C) \sse \Logic(P) \sse \Logic(O)=\Logic(Q) = \Logic(\Delta_n) = \PL_n
	\end{equation*}

	To prove \ref{item:infty; thm:largest logic}, let $\Lo = \Logic(\C)$ be a polyhedrally complete logic of height $\infty$. We can write $\C = \bigcup_{n \in \NN} \C_n$, where $\C_n = \C \cap \Poly_n$. Then:
	\begin{equation*}
		\Lo = \Logic(\C) = \Logic \left(\bigcup_{n \in \NN} \C_n\right) = \bigcap_{n \in \NN} \Logic (\C_n) \sse \bigcap_{n \in \NN} \PL_n = \PL
	\end{equation*}
	where in the penultimate containment we have used \ref{item:n; thm:largest logic}, and for the last equality we have used that $\PL$ has the finite model property.
\end{proof}
	\definecolor{dkgreen}{rgb}{0,0.6,0}
\definecolor{ltblue}{rgb}{0,0.4,0.4}
\definecolor{dkviolet}{rgb}{0.3,0,0.5}

% lstlisting coq style (inspired from a file of Assia Mahboubi)
\lstdefinelanguage{Coq}{ 
    % Anything betweeen $ becomes LaTeX math mode
    mathescape=true,
    % Comments may or not include Latex commands
    texcl=false, 
    % Vernacular commands
    morekeywords=[1]{Section, Module, End, Require, Import, Export,
        Variable, Variables, Parameter, Parameters, Axiom, Hypothesis,
        Hypotheses, Notation, Local, Tactic, Reserved, Scope, Open, Close,
        Bind, Delimit, Definition, Let, Ltac, Fixpoint, CoFixpoint, Add,
        Morphism, Relation, Implicit, Arguments, Unset, Contextual,
        Strict, Prenex, Implicits, Inductive, CoInductive, Record,
        Structure, Canonical, Coercion, Context, Class, Global, Instance,
        Program, Infix, Theorem, Lemma, Corollary, Proposition, Fact,
        Remark, Example, Proof, Goal, Save, Qed, Defined, Hint, Resolve,
        Rewrite, View, Search, Show, Print, Printing, All, Eval, Check,
        Projections, inside, outside, Def},
    % Gallina
    morekeywords=[2]{forall, exists, exists2, fun, fix, cofix, struct,
        match, with, end, as, in, return, let, if, is, then, else, for, of,
        nosimpl, when},
    % Sorts
    morekeywords=[3]{Type, Prop, Set, true, false, option},
    % Various tactics, some are std Coq subsumed by ssr, for the manual purpose
    morekeywords=[4]{pose, set, move, case, elim, apply, clear, hnf,
        intro, intros, generalize, rename, pattern, after, destruct,
        induction, using, refine, inversion, injection, rewrite, congr,
        unlock, compute, ring, field, fourier, replace, fold, unfold,
        change, cutrewrite, simpl, have, suff, wlog, suffices, without,
        loss, nat_norm, assert, cut, trivial, revert, bool_congr, nat_congr,
        symmetry, transitivity, auto, split, left, right, autorewrite},
    % Terminators
    morekeywords=[5]{by, done, exact, reflexivity, tauto, romega, omega,
        assumption, solve, contradiction, discriminate},
    % Control
    morekeywords=[6]{do, last, first, try, idtac, repeat},
    % Comments delimiters, we do turn this off for the manual
    morecomment=[s]{(*}{*)},
    % Spaces are not displayed as a special character
    showstringspaces=false,
    % String delimiters
    morestring=[b]",
    morestring=[d],
    % Size of tabulations
    tabsize=3,
    % Enables ASCII chars 128 to 255
    extendedchars=false,
    % Case sensitivity
    sensitive=true,
    % Automatic breaking of long lines
    breaklines=false,
    % Default style fors listings
    basicstyle=\small,
    % Position of captions is bottom
    captionpos=b,
    % flexible columns
    columns=[l]flexible,
    % Style for (listings') identifiers
    identifierstyle={\ttfamily\color{black}},
    % Style for declaration keywords
    keywordstyle=[1]{\ttfamily\color{dkviolet}},
    % Style for gallina keywords
    keywordstyle=[2]{\ttfamily\color{dkgreen}},
    % Style for sorts keywords
    keywordstyle=[3]{\ttfamily\color{ltblue}},
    % Style for tactics keywords
    keywordstyle=[4]{\ttfamily\color{dkblue}},
    % Style for terminators keywords
    keywordstyle=[5]{\ttfamily\color{dkred}},
    %Style for iterators
    %keywordstyle=[6]{\ttfamily\color{dkpink}},
    % Style for strings
    stringstyle=\ttfamily,
    % Style for comments
    commentstyle={\ttfamily\color{dkgreen}},
    %moredelim=**[is][\ttfamily\color{red}]{/&}{&/},
    literate=
    {\\forall}{{\color{dkgreen}{$\forall\;$}}}1
    {\\exists}{{$\exists\;$}}1
    {<-}{{$\leftarrow\;$}}1
    {=>}{{$\Rightarrow\;$}}1
    {==}{{\code{==}\;}}1
    {==>}{{\code{==>}\;}}1
    %    {:>}{{\code{:>}\;}}1
    {->}{{$\rightarrow\;$}}1
    {<->}{{$\leftrightarrow\;$}}1
    {<==}{{$\leq\;$}}1
    {\#}{{$^\star$}}1 
    {\\o}{{$\circ\;$}}1 
    {\@}{{$\cdot$}}1 
    {\/\\}{{$\wedge\;$}}1
    {\\\/}{{$\vee\;$}}1
    {++}{{\code{++}}}1
    {~}{{\ }}1
    {\@\@}{{$@$}}1
    {\\mapsto}{{$\mapsto\;$}}1
    {\\hline}{{\rule{\linewidth}{0.5pt}}}1
    %
}[keywords,comments,strings]

\section{Implementing Logical Machinery \& Soundness}
We build our program logic, as an instantiation of Iris~\cite{jung2018iris}.
\subsection{Soundness}
\label{sec:soundness}
Our logic, operates on the machine state, which means we do not need to augment the machine state. The invarian that we pick, \textit{central invariant} ($\textsf{x64\_h}$), is just semantic interpretation of stores in the machine state, $\sigma.\mathcal{R}$ and $\sigma.\mathcal{M}$. This semantic interpratation ensures the correct lifting of mappings in the machine state to the assertions that the client of our logic uses, i.e. points-to assertions that are defined as the ownership of a fragment of the logical state.

We prefer to skip explaining the steps used in instantiation of Iris because it is an almost standard procedure, has already been explained for many other logic \ref{}, and we are concerned with the page-count limitation. However, it is worh noting that once you instantiate Iris for your language, it comes with the semantic definition for the weakest-precondition which we can refactor into Hoare-Doubles to specify our  selected \textsf{AMD64} instructions shown in Figure \ref{fig:wpdamd}, and show that these triples are sound.
\subsection{The Soundness Statement}
\label{def:soundness:statement}
The operational semantics of our simple lang just executes sequences of instructions in our x86-64 model. Therefore, our soundness argument is to show that any execution composed of instructions in our machine model (some of which are shown in Figure \sref{sec:semantics}) does not end-up in a invalid state.
% Figure environment removed

\begin{theorem}[Soundness of the Logic]
  \label{th:adequacy}
 Together with the assumptions on the initial state hold,
 the execution of the instruction~$\instrs$, beginning with this initial state, cannot result in a configuration where the execution is stuck.
\end{theorem}
which states that if the program~$\instr$, with the given \textsf{valid\_init} asserting a valid state initialization, satisfies a semantic Hoare
triple, then this program cannot crash: by a direct consequence of Iris's adequacy theorem~\cite[\S6.4]{iris}.

Moreover we need to show the validity of each rules in Figures \fref{fig:reasoning} and \fref{fig:laws}.
\begin{theorem}[Validity of the Reasoning Rules]
\label{th:validity}
  Each of the rules in Figures~\ref{fig:wpdamd}
  and~\ref{fig:structural} is valid.
\end{theorem}
Due to the space limits in this paper, we do not mention each proof for the rules in Figures \ref{fig:wpdamd} and \ref{fig:structural} within this section, but we provide mechanized proofs for all these in Coq as a part of our artifact submission.
However, we would like to give the definitions and constructions used in our proofs, and would like to give an outline of paper proof for \TirNameStyle{WriteToRegFromVirtMem} in Figure \ref{fig:wpdamd} within this section.

Together, Theorems~\ref{th:adequacy} and~\ref{th:validity} guarantee that, if
the Hoare triple $\textsf{valid\_init }\instrs\;\iTrue$ can be obtained by applying
the reasoning rules of our logic, then the program~$\instrs$ is safe.

\subsection{Logical Constructions}
\label{sec:invariant}
We already have the physical and logical stores and a simple invariant between them. Now, we can rely on the following Assumption \ref{assumption} from Iris to utilize its logical constructions.
% The predicate gen_heap_interp.
\newcommand{\genheapinterp}[1]{\mathit{Heap}\;#1}
\newcommand{\genmemheapinterp}[1]{\mathit{MemHeap}\;#1}
% Our predicate pred (defined in ph.v), expanded.
\newcommand{\pred}[1]{\ownGhost\gammaPred{\authfull{(\mapone\predstore)}}}
% A notation for assigning fraction 1 to every element of \predstore.
\newcommand{\mapone}[1]{1.#1}
% The predicate mapsfrom_exact, expanded.
\newcommand{\mapsfromexact}[3]{
  \ownGhost\gammaPred{\authfrag{\singletonMap{#1}{(#2, #3)}}}
}
% A metavariable for a share.
\newcommand{\sh}{L'}
% The predicate mapsfrom, expanded.
\newcommand{\mapsfromdef}[3]{
  \exists\sh.\;
  \mapsfromexact{#1}{#2}{\sh} \star \pure{\sh \subseteq #3}
}

\begin{assumption}
\label{assumption}
Iris defines two pieces of ghost state
\begin{enumerate}
\item  defines a predicate $\textsf{to\_gen\_heap }\store.\mathcal{R}$
  that ties a store~$\store.\mathcal{R}$ to this ghost state,
  and defines the points-to assertion $\ppointsto\rg\rv\qfrac\rpts$
  in terms of this ghost state.
  This is visible in the paper~\cite[\S6.3.2]{iris}
  and in Iris's \texttt{gen\_heap} library~\cite{genheap}.
  %
  We re-use this machinery without change,
  so we do not repeat these definitions.
  We mention the predicate $\genheapinterp\!$
  in our own invariant (Definition~\ref{def:invariant}),
  where it is applied to the \logical store~$\store$.
\item unlike the register points-to relation obtained by direct interpretion of $\textsf{to\_gen\_heap }\store.\mathcal{R}$ using Iris, the existing \textsf{gen\_heap\_heap},
  does not directly helps introducing the ghost we define a new algebra (\textsf{gen\_mem\_UR}) for abstracting the nested maps due to different levels of masking in memory mappings and an interpretation for this algebra. 
\end{enumerate}
\end{assumption}

\begin{definition}[Ghost State - Memory with Nested Mappings]
We define our custom-tailored algebra 
\newcommand\fpfn{\rightarrow_{\textrm{fin}}}
\( \textsf{gen\_memUR} \stackrel{\triangle}{=}
  \authm(\;
  \Locft \;\fpfn\;
  (\Loctw \;\fpfn\;  (\textsc{Frac }, \mathord{+}) \times (\textsc{Agree } \Loc,\mathord{=}) )
  \)
  for our nested memory mapping abstracting two different machine word masking. To interpret this nested ghost map, i.e. obtain points-to assertions out of mappings in an ordinary \textsf{gmap}, we define \textsf{to\_gen\_mem}
  \begin{lstlisting}[language=Coq]
    Definition to_gen_mem : gmap L1 ( gmap L2 V) $\rightarrow$ gen_memUR L1 L2 V := fmap ($\lambda$ m . to_gen_heap m).
    Definition to_gen_heap : gmap L V $\rightarrow$ gen_heapUR L V :=  $\lambda$  v $\ldotp$ (1, to_agree (v :leibnizO V)).
  \end{lstlisting}
 throuh using \textsf{to\_gen\_heap} from previous Iris version. \todo[inline,color=red]{Ismail give exact version commit etc.}
\end{definition}

\begin{definition}[Ghost State - Register Mappings]
We allocate $\theta$ ghost cell which stores an
element of the monoid 
\newcommand\fpfn{\rightarrow_{\textrm{fin}}}
\(
  \authm(\;
    \regset \;\fpfn\;
    (\textsc{Frac}, \mathord{+})
    \times
    (\regvaltype, \mathord{=})
  \;)
\)
% \emph{authoritative camera}
\cite[\S6.3.3]{iris}.
\end{definition}

\begin{definition}[Central Invariant]
\label{def:invariant}
The central invariant of our logic is, due to lack of need for augmenting the machine state, simply the state interpretation: 
\[
\textsf{x64\_h}\;\store \triangleq
\def\arraystretch{1.2}
\begin{array}{l@{\quad\ast\quad}l@{\quad}l}
  \textsf{to\_gen\_heap} \;\store.\mathcal{R} & \textsf{to\_gen\_mem} \; \store.\mathcal{M}
\end{array}
\]
\end{definition}

As a last definition, we give the head step relation required for the Iris instantiation for our simple language in Figure \ref{} to sequence instructions. This relation allows lifting the program expression to enable the application of changes imposed by the operational semantics on the program state (for a certain cpu, a register map, and a selected memory for an address-space) when applied for an insturction (\textsf{i}).
\begin{lstlisting}[language=Coq]
Definition exec_step (i: instr) ($\sigma$:state) : option state :=  exec_instr i $\sigma.\mathcal{C}$ $\sigma.\mathcal{R}$ ((memToPhysMem $\sigma.\mathcal{M}$)  ($\sigma.\mathcal{R}$ !! cr3)).
\end{lstlisting}
\todo[inline,color=yellow]{Colin, in case needed,not proven, could you simply say that we assume these map equalities or a paper proof etc.}. Again, due to the page limits, we can only give an outline of proof for a selected instruction (\TirNameStyle{WriteToRegFromVirtMem}) from our \textsf{AMD64} model. However, mechanized soundness proofs of other instructions can be found as a part of our submission artifact.

After obtaining ownership (points-to) predicates from application of state interpretation, now, we have well-enough definition for giving an outline for the proof of \TirNameStyle{WriteToRegFromVirtMem}.
 \begin{lemma}[\textsc{\TirNameStyle{WriteToRegFromVirtMem}}]
   \label{lemma:unlink}
\begin{align*}
\inferrule{
  \{\mathsf{P} \ast \mathsf{r}_d \mapsto_{r}  \mathsf{v} \ast \mathsf{r}_a \mapsto_{r} \{\mathsf{q}\} \; \mathsf{ vaddr} \ast \mathsf{vaddr} \mapsto_{\mathsf{v}} \mathsf{v} \}_{\mathsf{rtv}}\;\overline{is}
}{
  \{\mathsf{P} \ast \mathsf{r}_d \mapsto_{r}  \mathsf{rvd} \ast \mathsf{r}_a \mapsto_{r} \{\mathsf{q}\} \;\mathsf{ vaddr} \ast \mathsf{vaddr} \mapsto_{\mathsf{v}} \mathsf{v} \}_{\mathsf{rtv}}
\; \mathsf{ mov}~\mathsf{r}_d~\mathsf{r}_a;\;\overline{is}
}
\end{align*}
 \end{lemma}
 
 \begin{proof}
   Assuming the inference rule realized with \textsf{wpd\_def}, we expand the precondition with $\textsf{cr3} \mapsto_{\textsf{r}} \rtv$.
   Then we do two proofs, one for head reducibility of atomic step \textsf{mov\_reg64\_mem64}, the second one for the executing the expression and obtaining the new state. Steps taken in the first one are subset of the second one, so we outline the second portion of the proof, but in case of an interest in details of the proof, Coq artifact can be consulted.

   \begin{itemize}
   \item Step 1: we apply the head step relation and obtain the current valid state $\sigma1$ interpretation : $\textsf{x64\_h}\;\store$
   \item Step 2: we unfold the virtual-pointsto ($\vaddr \mapsto_{\textsf{v,rtv}} \textsf{v}$) definition, and for an existential physical page address $\paddr$, we exchange our fragmental toke ($\sumwalkabs\vaddr\qfrac\paddr$) to obtain physical table-pointsto ($\textsf{L}_{4}\_\textsf{L}_{1}\_\textsf{PointsTo}$ in Figure \ref{fig:strongvirtualpointsto}) relation
   \item Step 3: for each physical pointsto inside $\textsf{L}_{4}\_\textsf{L}_{1}\_\textsf{PointsTo}$, there exists a \textsf{load}, i.e. a concrete memory lookup
   \item Step 4: use these concrete lookups to traverse the page tables to obtain the value \textsf{v}
     \item Step 5: do the map update the $\sigma.\mathcal{R}$ for relevant register mapping ($r_d$) with the value \textsf{v} 
   \end{itemize}
   
   \end{proof}

	%!TEX root = ../convex-paper.tex

\section{Completeness}
\label{sec:completeness}

	The proof that $\PL_n$ is complete with respect to the class of convex polyhedra of dimension at most $n$ consists of two main parts. In the first part, we show that $\PL_n$ can be expressed as the logic of a set of reasonably regular finite frames --- called \emph{sawed trees}. For the second part, we show that any such sawed tree of height $n$ can be realised geometrically as an $n$-dimensional convex polyhedron --- in other words, given a sawed tree $F$, we construct an open polyhedral map from a convex polyhedron onto $F$. This map is constructed using a more elaborate version of the method used to provide a geometric realisation for an arbitrary finite poset in \cref{ssec:geometric realisation}.


\subsection{The meaning of \texorpdfstring{$\PL_n$}{PLn} on frames}

	First of all, it will be convenient to spell out what it means, structurally, for a frame to satisfy $\PL_n$. For this we introduce some additional terminology and notation.

	For any poset $F$ and $x \in F$, the \emph{strict upset} and \emph{strict downset} are defined, respectively, as follows.
	\begin{gather*}
		\Us x \coloneqq \{y \in F \mid y > x\} \\
		\Ds x \coloneqq \{y \in F \mid y < x\}
	\end{gather*}
	The \emph{depth} of $x$ is defined:
	\begin{equation*}
		\depth(x) \coloneqq \height(\us x)
	\end{equation*}
	A \emph{top element} of $F$ is $t \in F$ such that $\depth(t)=0$. The set of top elements in $F$ is denoted by $\Top(F)$.

	A \emph{path} in $F$ is a sequence $p=x_0\cdots x_k$ of elements of $F$ such that for each $i$ we have $x_i < x_{i+1}$ or $x_i > x_{i+1}$. Write $p \colon x_0 \rsa x_k$. The poset $F$ is \emph{path-connected} if between any two points there is a path.

	\begin{lemma}\label{lem:finite poset path-connected iff connected}
		When $F$ is finite, it is path-connected if and only if it is connected as a topological space.
	\end{lemma}

	\begin{proof}
		See \cite[Lemma~3.4]{Bezhanishviligabelaia2011}.
	\end{proof}

	A \emph{connected component} of $F$ is a subframe $U \sse F$ which is connected as a topological subspace and is such that there is no connected $V$ with $U \subset V$.

	\begin{lemma}\label{lem:properties of connectedness}
		\begin{enumerate}[label=(\arabic*)]
			\item The connected components partition $F$.
			\item Connected components are upwards- and downwards-closed.
		\end{enumerate}
	\end{lemma}

	\begin{proof}
		The first is a standard fact in topology, while the second follows straightforwardly from the fact that by \cref{lem:finite poset path-connected iff connected} the connected components are exactly the equivalence classes under the relation `there is a path from $x$ to $y$'.
	\end{proof}

	Finally, for any $x,y \in F$, say that $x$ is an \emph{immediate predecessor} of $y$ and that $y$ is an \emph{immediate successor} of $x$ if $x < y$ and there is no $z \in F$ such that $x < z < y$.

	We can now describe the structural meaning of $\PL_n$ on frames.

	\begin{lemma}\label{lem:meaning of PLn on posets}
		Let $F$ be a poset. Then $F \vD \PL_n$ if and only if the following are satisfied.
		\begin{enumerate}[label=(\roman*)]
			\item\label{item:height; lem:meaning of PLn on posets}
				$F$ has height at most $n$.
			\item\label{item:depth 1; lem:meaning of PLn on posets}
				Whenever $\depth(x) = 1$, we have $\abs{\Us x} \leq 2$.
			\item\label{item:depth gt1; lem:meaning of PLn on posets}
				Whenever $\depth(x) > 1$, the set $\Us x$ is connected.
		\end{enumerate}
	\end{lemma}

	\begin{proof}
		This follows from the definition of $\PL_n$, using the following facts for finite frames $F$.
		\begin{enumerate}[label=(\roman*)]
			\item $F \vD \BD_n$ if and only if $F$ has height at most $n$.
			\item There is an up-reduction $F \cra \FThreeFork$ if and only if there is $x \in F$ such that $\Us x$ has at least three components.
			\item There is an up-reduction $F \cra \FScott$ if and only if there is $x \in F$ such that $\Us x$ has at least two components, with at least one of which having height greater than $0$.\qedhere
		\end{enumerate}
	\end{proof}

	$\PL_n$-frames also satisfy the following specific connectedness property, which will come in handy in the arguments below.

	\begin{lemma}\label{lem:PLn frames specific connectedness property}
		Let $F$ be a finite rooted frame with $\height(F) > 1$, such that $F \vD \PL_n$. Take $s,t \in \Top(F)$. There is a path $p = a_0 \cdots a_m$ from $s$ to $t$ in $\Us{\bot}$ with the property that for each $i$:
		\begin{enumerate}[label=(\Roman*)]
			\item\label{item:a; lem:PLn frames specific connectedness property} 
				$\Us{a_i} = \es$ when $i$ is even, and
			\item\label{item:b; lem:PLn frames specific connectedness property} 
				$\Us{a_i} = \{a_{i-1},a_{i+1}\}$ when $i$ is odd.
		\end{enumerate}
	\end{lemma}

	\begin{proof}
		Since $\height(F) > 1$ we have that $\depth(\bot) > 1$. Hence by \cref{lem:meaning of PLn on posets}, there is a path $p = a_0 \cdots a_m$ from $s$ to $t$ in $\Us{\bot}$. We may assume that:
		\begin{enumerate}[label=(\Alph*)]
			\item\label{item:immediate; proof:PLn frames specific connectedness property} 
				$a_{i+1}$ is either an immediate successor or an immediate predecessor of $a_i$, for each $i$,
			\item\label{item:height-maximal; proof:PLn frames specific connectedness property} 
				$p$ is `height-maximal': if $i < j < k$ and $a_j < a_i, a_k$, then there is no path $a_i \rsa a_k$ in $\Us{a_j}$, and
			\item\label{item:no repeats; proof:PLn frames specific connectedness property} 
				$p$ has no repeats.
		\end{enumerate}
		Indeed, \ref{item:height-maximal; proof:PLn frames specific connectedness property} can be secured by iteratively replacing each offending $a_j$ with the path $a_i \rsa a_k$ in $\Us{a_j}$. Then \ref{item:no repeats; proof:PLn frames specific connectedness property} can be secured by removing all cycles, a process which preserves \ref{item:height-maximal; proof:PLn frames specific connectedness property}.

		We claim that such a $p$ also satisfies \ref{item:a; lem:PLn frames specific connectedness property} and \ref{item:b; lem:PLn frames specific connectedness property}, which we prove by induction. The base $i=0$ is immediate since $a_0=s$ is a top node. So assume that $i>0$. The first case is when $i$ is odd. By induction hypothesis $\Us{a_{i-1}} = \es$; in other words $a_{i-1}$ is a top node. Hence by \ref{item:immediate; proof:PLn frames specific connectedness property}, $a_i$ is an immediate predecessor of $a_{i-1}$. This means that $\{a_{i-1}\}$ is a connected component in $\Us{a_i}$, and hence by \cref{lem:meaning of PLn on posets} \ref{item:depth 1; lem:meaning of PLn on posets} and \ref{item:depth gt1; lem:meaning of PLn on posets}, we must have $\abs{\Us{a_i}} \leq 2$. Note further that by \ref{item:height-maximal; proof:PLn frames specific connectedness property}, $a_{i+1} \neq a_{i-1}$. Therefore, the task is to show that $a_{i+1} \in \Us{a_i}$. Let us suppose for a contradiction that this is not the case; i.e. $a_{i+1} < a_i$. Since $t$ is a top node, there must be $j \geq i+1$ with $a_j \leq a_{i+1}$ such that $a_{j+1} > a_j$ (in other words, the path can not keep going downwards after $a_{i+1}$). Clearly $\depth(a_j)>1$, hence by \cref{lem:meaning of PLn on posets} \ref{item:depth gt1; lem:meaning of PLn on posets} there must be a path $a_i \rsa a_{j+1}$ in $\Us{a_j}$, which contradicts property \ref{item:height-maximal; proof:PLn frames specific connectedness property}. \contradiction Thus $a_{i+1} \in \Us{a_i}$ as required. The second case when $i$ is even follows immediately from property \ref{item:immediate; proof:PLn frames specific connectedness property} and the induction hypothesis.
	\end{proof}


\subsection{Sawed trees}

	Let $T$ be a finite tree in which every top element has the same height. A linear ordering $\prec$ on $\Top(T)$ (or equivalently an enumeration $t_1, \ldots, t_k$ of $\Top(T)$) is a \emph{plane ordering} if for every $x \in T$ we have that $\us x \cap \Top(T)$ is an interval with respect to $\prec$. When $\height(T)>0$, the \emph{sawed tree} based on $(T,\prec)$ consists of $T$ plus new elements $s_1, \ldots, s_{k-1}$ with relations, for each $i$:
	\begin{equation*}
		t_i,t_{i+1} < s_i
	\end{equation*}
	See \cref{fig:sawed tree example} for an example of a sawed tree.

	% Figure environment removed

	The planarity condition on $\prec$ ensures that the Hasse diagram of the resulting sawed tree can be drawn in the plane with no overlapping lines. Formally, let $G$ be a poset and $d \colon G \to \R^2$ be an injection, such that $d = (d_1,d_2)$. Draw an edge $\mathit{xy}$ between $d(x)$ and $d(y)$ whenever $y$ is an immediate successor of $x$. Then $d$ is a \emph{plane drawing} of $G$ if the following conditions hold.
	\begin{enumerate}[label=(\alph*)]
		\item Whenever $x < y$ we have $d_2(x) < d_2(y)$.
		\item Two distinct edges $x_1y_1$ and $x_2y_2$ only ever intersect at their end-points.
	\end{enumerate}
	The notion of a planar poset has been studied somewhat in the literature (see \cite[\S6.8, p.~101]{brandstadtetal1999} for a short survey), but we will not  use any external results here.

	\begin{lemma}\label{lem:trees are planar}
		Let $\prec$ be a plane ordering on $T$. Then $T$ has plane drawing $d$ with the following properties. 
		\begin{enumerate}[label=(\roman*)]
			\item The top nodes in the drawing are ordered left-to-right as per $\prec$.
			\item $d_2(x) = \height(x)$ for every $x \in T$.
		\end{enumerate}
	\end{lemma}

	\begin{proof}
		C.f. \cite[p.~294]{stanley1997}. We proceed by induction on $n = \height(T)$. The base case $n=0$ is immediate, so assume that $n>0$. Enumerate the immediate successors of $\bot$ in $T$ as $\{x_1, \ldots, x_k\}$, according to $\prec$. That is, for each $i,j \leq k$ with $i<j$ ensure that:
		\begin{equation*}
			\forall t_i \in \us{x_i} \cap \Top(T) \colon \forall t_j \in \us{x_j} \cap \Top(T) \colon t_i \prec t_j
		\end{equation*}
		This is possible since $\us{x} \cap \Top(T)$ is an interval for each $x$. By induction hypothesis, for each $i \leq k$ there is a plane drawing $d^i$ of $\us{x_i}$ satisfying the conditions. We can then form a plane drawing $d$ of $T$ by shifting the drawings $d_1, \ldots, d_k$ up by one, lining them up side by side, then letting $d(\bot) \coloneqq (0,0)$. It is clear that $d$ then also satisfies the required conditions.
	\end{proof}

	\begin{corollary}\label{cor:sawed trees are planar}
		Every sawed tree $F$ admits a plane drawing $d$ with the property that $d_2(x) = \height(x)$ for every $x \in F$.
	\end{corollary}

	\begin{proof}
		Let $F$ be based on $(T,\prec)$, and let $s_1, \ldots, s_{k-1}$ be the top elements. By \cref{lem:trees are planar}, there is a plane drawing $d'$ of $T$ satisfying the property. Extend $d'$ to a drawing $d$ of $F$ by letting $d(s_i) \coloneqq (i,\height(F))$.
	\end{proof}

	The reason for considering sawed trees is that they provide a complete class of frames for $\PL$ which is relatively easy to work with.

	\begin{lemma}\label{lem:sawed trees satisfy PLn}
		Let $F$ be a sawed tree of height $n$. Then $F \vD \PL_n$.
	\end{lemma}

	\begin{proof}
		Let $F$ be based on $(T,\prec)$. Let us verify the conditions of \cref{lem:meaning of PLn on posets}. Conditions \ref{item:height; lem:meaning of PLn on posets} and \ref{item:depth 1; lem:meaning of PLn on posets} are immediate. As for \ref{item:depth gt1; lem:meaning of PLn on posets}, take $x \in F$ with $\depth(x)>1$. By construction, $x \in T$. Since $\prec$ is a plane ordering, we have that $\us x \cap \Top(T)$ is an interval with respect to $\prec$. Therefore, the top two layers of $\Us x$ are connected by the saw structure.
	\end{proof}

	\begin{lemma}\label{lem:frames of PL p-morphic images of sawed trees}
		Every rooted frame $F$ of $\PL$ of height $n$ is the p-morphic image of a sawed tree of height $n$, for every $n \geq 2$.
	\end{lemma}

	\begin{proof}
		We prove this by induction on $n$. For the base case $n=2$, note that $F$ consists of the root $\bot$ together with a number of nodes of depths $0$ and $1$. By gluing together paths obtained from \cref{lem:PLn frames specific connectedness property}, we can find a path $p = a_0 \cdots a_m$ satisfying \ref{item:a; lem:PLn frames specific connectedness property} and \ref{item:b; lem:PLn frames specific connectedness property} of that lemma which visits every top node. We would like to extend $p$ so that it visits \emph{every non-root} node. To do this, take $x \in F$ of depth $1$. By \cref{lem:meaning of PLn on posets} \ref{item:depth 1; lem:meaning of PLn on posets}, $\Us{x}=\{s,t\}$ with $s,t$ top nodes and possibly $s=t$. By inserting the sequence $xtxs$ in $p$ after an occurrence of $s$, we obtain a path satisfying \ref{item:a; lem:PLn frames specific connectedness property} and \ref{item:b; lem:PLn frames specific connectedness property}, which also visits $x$.

		Therefore, we may assume that our path $p$ visits every non-root node. Now, construct the sawed tree $F'$ by taking $\bot$ together with new elements:
		\begin{equation*}
			w_{-1},w_0, \ldots, w_m, w_{m-1}
		\end{equation*}
		with relations as in \cref{fig:F prime n2; proof:frames of PL p-morphic images of sawed trees}.

		% Figure environment removed

		Then define the surjective map $f \colon F' \to F$ by:
		\begin{gather*}
			\bot \mapsto \bot, \\
			w_{-1} \mapsto a_0, \\ 
			w_{m+1} \mapsto a_m, \\
			w_i \mapsto a_i \qquad \forall i \in \{0, \ldots, m\}
		\end{gather*}
		That $f$ is a p-morphism amounts to the fact that $p$ satisfies properties \ref{item:a; lem:PLn frames specific connectedness property} and \ref{item:b; lem:PLn frames specific connectedness property} of \cref{lem:PLn frames specific connectedness property}.

		For the induction step, assume that $n > 2$. Let $z_1, \ldots, z_k$ be the immediate successors of $\bot$ in $F$. By induction hypothesis, for each $i$ there is a sawed tree $G_i$ and a p-morphism $g_i \colon G_i \to \us{z_i}$. Let the sawed tree $G_i$ be based on $(S_i,\prec_i)$, and let $u_i, v_i \in \Top(S_i)$ be the least and greatest elements according to $\prec_i$, respectively. Since $\abs{\us{u_i}}, \abs{\us{v_i}} = 2$, we must have: 
		\begin{equation*}
			\abs{\us{g_i(u_i)}}, \abs{\us{g_i(v_i)}} \leq 2
		\end{equation*}
		Let $s_i \in \us{g_i(u_i)}$ and $t_i \in \us{g_i(v_i)}$ be the greatest elements. Now, by \cref{lem:PLn frames specific connectedness property}, for each $i \leq k-1$ there is a path $p_i \colon t_i \rsa s_{i+1}$ satisfying properties \ref{item:a; lem:PLn frames specific connectedness property} and \ref{item:b; lem:PLn frames specific connectedness property}; write $p_i = a_{i,0} \cdots a_{i,m_i}$.

		We will form our new sawed tree by laying the sawed trees $G_1, \ldots, G_k$ in a line and `gluing' them usings the paths $p_1, \ldots, p_{k-1}$ together with some `rope ladders' beneath. In detail, form $F'$ by taking the following ingredients and combining them as in \cref{fig:F prime construction inductive step; proof:frames of PL p-morphic images of sawed trees}.
		\begin{itemize}
			\item Each sawed tree $G_i$.
			\item For each $i \leq k$, new elements $w_{i,0} \cdots w_{i,k_i}$ corresponding to $a_{i,0} \cdots a_{i,k_i}$.
			\item A chain of length $n-2$ (a rope ladder) to hang below each $w_{i,j}$, with $j$ odd.
		\end{itemize}

		% Figure environment removed

		The result is evidently a sawed tree. Finally, construct the p-morphism $f \colon F' \to F$ as follows.
		\begin{enumerate}[label=(\alph*)]
			\item Inside each sawed tree $G_i$, let $f$ act as $g_i$.
			\item For each $w_{i,j}$, let $f(w_{i,j}) \coloneqq a_{i,j}$.
			\item For each $w_{i,j}$ with $j$ odd, send the rope ladder hanging below $w_{i,j}$ to $a_{i,j}$. \qedhere
		\end{enumerate}
	\end{proof}

	\begin{corollary}\label{cor:PLn logic of sawed trees}
		$\PL_n$ is the logic of sawed trees of height at most $n$, for every $n \geq 2$.
	\end{corollary}

	\begin{proof}
		This follows from \cref{lem:sawed trees satisfy PLn} and \cref{lem:PLn frames specific connectedness property}, and the fact that $\PL_n$, like any intermediate logic, is the logic of its rooted frames.
	\end{proof}


\subsection{Convex geometric realisation}

	In the second stage of the completeness proof, we provide a method of constructing a convex realisation of any sawed tree. To provide intuition for the construction, we first examine an instructive example of height $3$. Consider \cref{fig:convex geometric realisation height 3}. 

	% Figure environment removed
	
	The sawed tree $F$, depicted on the left, is realised in the pyramid $P = \mathit{OABEC}$, depicted on the right. The point $\mathit{D}$ lies midway between $\mathit{C}$ and $\mathit{E}$. An open surjective polyhedral map $f \colon P \to F$ is then defined as follows.
	\begin{itemize}
		\item The point $\mathit{O}$ is mapped to $\bot$.
		\item The remainder of the line $\mathit{OA}$ is mapped to $a$ while the remainder of $\mathit{OB}$ is mapped to $b$.
		\item The remainder of the triangle $\mathit{OAC}$ is mapped to $c$, the remainder of $\mathit{OAD}$ is mapped to $d$, and the remainder of $\mathit{OBE}$ is mapped to $e$.
		\item Finally, the remainder of the region $\mathit{OACD}$ is mapped to $s$ and the remainder of the region $\mathit{OABED}$ is mapped to $t$.
	\end{itemize}
	It is clear that such a map is polyhedral. Further the construction ensures that any open neighbourhood in $P$ is mapped to an upwards-closed subset of $F$. For instance, note that any open set intersecting $\mathit{OAD}$ must also intersect $\mathit{OACD}$ and $\mathit{OABED}$. Hence, $f \colon P \to F$ is an open polyhedral map as required.

	Notice that the two middle layers $(a,b)$ and $(c,d,e)$ of $F$ correspond to the edges $\mathit{AB}$ and $\mathit{CDE}$ of the base of the pyramid. Note further that the preimage of the tree part of $F$ --- i.e. the union of the triangles $\mathit{OAC}$, $\mathit{OAD}$ and $\mathit{OBE}$ --- has a natural triangulation. The definition of $f$ on this region then follows just as in the definition of the geometric realisation from \cref{ssec:geometric realisation}, with respect to this triangulation.

	With this intuition in mind we proceed with the proof in full generality. We make use of the following technical lemma on nerves and simplicial complexes.

	\begin{lemma}\label{lem:nerve geometric realisation criterion}
		Let $F$ be a poset and take any function $\alpha \colon F \to \R^n$. The collection:
		\begin{equation*}
			\{\Conv \alpha[X] \mid X \in \N(F)\}
		\end{equation*}
		forms a simplicial complex if and only if $\Conv\alpha[X]$ and $\Conv\alpha[Y]$ are disjoint for any disjoint $X,Y \in \N(F)$.
	\end{lemma}

	\begin{proof}
		This follows from \cite[Theorem~2]{demendez1999}, noting that the nerve $\N(F)$ is in particular an abstract simplicial complex, as defined there, with vertex set $\{\{x\} \mid x \in F\}$. 
	\end{proof}

	\begin{proof}[Proof of \cref{thm:PLn completeness for CPn}]
		The case $n=0$ is immediate. For $n=1$ note that by \cref{lem:meaning of PLn on posets}:
		\begin{equation*}
			\PL_1 = \Logic(\FPoint, \FOneFork, \FTwoFork) = \Logic(\FTwoFork)
		\end{equation*}
		Consider the convex polyhedron given by the interval $[0,1]$. We can define an open polyhedral map $f \colon [0,1] \to \FTwoFork$ by mapping $\half$ to the root, and the intervals $[0,\half)$ and $(\half,1]$ to each top node, respectively. Therefore:
		\begin{equation*}
			\Logic(\PolyCon_1) \sse \Logic([0,1]) \sse \PL_1 
		\end{equation*}
		Hence we may assume that $n \geq 2$. By \cref{cor:PLn logic of sawed trees} and \cref{lem:maps duality}, it suffices to show that every sawed tree of height $n$ can be realised geometrically in a convex polyhedron of dimension $n$. So, let $F$ be a height-$n$ sawed tree based on $(T,\prec)$. Using \cref{cor:sawed trees are planar}, let $d$ be a plane drawing of $F$ such that $d_2(x) = \height(x)$ for each $x \in F$.

		We first construct a simplicial complex corresponding to the tree part $T$ of $F$. Let $e_0, \ldots, e_n$ be the standard basis vectors of $\R^{n+1}$. Define a function $\alpha \colon T \to \R^{n+1}$ by letting, for $x \in T$:
		\begin{equation*}
			\alpha(x) \coloneqq e_{\height(x)} + d_1(x)e_{n}
		\end{equation*}
		It is helpful to consider the $n$th dimension (spanned by $e_n$) as running from left to right. Then nodes which are further to the right in the plane drawing $d$ map to points which are further to the right in $\R^{n+1}$. For each $X \in \N(T)$, let:
		\begin{equation*}
			\sig(X) \coloneqq \Conv\alpha[X]
		\end{equation*}
		Note that each element in $X$ is of a different height, so that $\alpha[X]$ is an affinely independent set of points; hence $\sig(X)$ is a simplex. Then set:
		\begin{equation*}
			\Sig \coloneqq \{\sig(X) \mid X \in \N(X)\}
		\end{equation*}
		Let us use \cref{lem:nerve geometric realisation criterion} to verify that $\Sig$ is a simplicial complex. Take disjoint $X,Y \in \N(F)$, and suppose for a contradiction that $\sig(X) \cap \sig(Y) \neq \es$. Let $X = \{x_1, \ldots, x_k\}$ and $Y = \{y_1, \ldots, y_l\}$, enumerated according to the order $<$ on $T$. Then, using barycentric coordinates inside $\sig(X)$ and $\sig(Y)$, there must be $r_1,\ldots, r_k \geq 0$ and $q_1, \ldots, q_l \geq 0$ with $\sum_{i=1}^k r_i = 1$ and $\sum_{j=1}^l q_j = 1$ such that:
		\begin{equation*}
			\sum_{i=1}^k r_i \alpha(x_i) = \sum_{j=1}^l q_j \alpha(y_j)
		\end{equation*}
		Using the definition of $\alpha$ and the fact that $e_0, \ldots, e_n$ are linearly independent, we see that:
		\begin{itemize}
			\item $r_i = 0$ if there is no $y_j$ with $\height(x_i) = \height(y_j)$,
			\item $q_j = 0$ if there is no $x_i$ with $\height(x_i) = \height(y_j)$,
			\item $r_i = q_j$ whenever $\height(x_i) = \height(y_j)$, and
			\item $\sum_{i=1}^k r_i d_1(x_i) = \sum_{j=1}^l q_j d_1(y_j)$.
		\end{itemize}
		Hence, we may assume that $k=l$ and that $\height(x_i) = \height(y_i)$ for each $i$. Now, for each $i$, since $X$ and $Y$ are disjoint, we must have $d(x_i) \neq d(y_i)$. But, since $d_2(x_i) = \height(x_i) = d_2(y_i)$, we must have either $d_1(x_i) < d_1(y_i)$ or $d_1(x_i) > d_1(y_i)$. Without loss of generality, assume that $d_1(x_1) < d_1(y_1)$. Then, since $T$ is a tree and no edges overlap in the plane drawing $d$, we must have $d_1(x_i) < d_1(y_i)$ for each $i$. Thus:
		\begin{equation*}
			\sum_{i=1}^k r_i d_1(x_i) = \sum_{i=1}^l q_i d_1(x_i) < \sum_{j=1}^l q_j d_1(y_j)
		\end{equation*}
		which is a contradiction. Therefore, $\Sig$ is a simplicial complex. As in \cref{ssec:geometric realisation}, the p-morphism $\max \colon \N(T) \to T$ gives rise to an open polyhedral map $f_T \colon \abs\Sig \to T$.

		Let us turn our attention now towards the top part of $F$. Enumerate $\Top(T)$ according to $\prec$ as $\{t_1, \ldots, t_k\}$, and let $s_1, \ldots, s_{k-1}$ be the top elements of $F$, as in the definition of a sawed tree. For each $i \leq k$, we have the $(n-1)$-simplex $\tau_i \coloneqq \sig(\ds{t_i})$. For $i \leq k-1$, let:
		\begin{equation*}
			\xi_i \coloneqq \Conv(\alpha[\Ds{s_i}]) = \Conv(\tau_{i-1} \cup \tau_i)
		\end{equation*}
		By considering the definition of $\alpha$, and noting that $\Ds{s_i}$ contains two elements which have the same height, we can see that $\Dim(\xi_i) = n$. Note also that:
		\begin{equation*}
			\xi_i \cap \xi_{i+1} = \tau_i
		\end{equation*}

		Define $P \coloneqq \bigcup_{i=1}^k \xi_i$, which will be our convex geometric realisation. By \cref{lem:dimension of union}, $P$ is an $n$-dimensional polyhedron. Furthermore, note that:
		\begin{equation*}
			P = \Conv (\tau_1 \cup \tau_k) = \Conv(\tau_1 \cup \cdots \cup \tau_k) = \Conv(P)
		\end{equation*}
		so that $P$ is a convex polyhedron and thus $P \in \PolyCon_n$. Extend the map $f_T$ to $f \colon P \to F$ by letting $x \in \xi_i \setminus (\tau_{i-1} \cup \tau_i)$ map to $s_i$. This map is clearly polyhedral. To see that it is open, take $x \in P$ and $U \sse P$ a small open neighbourhood of $x$. There are two cases. If $x \in \xi_i \setminus (\tau_{i-1} \cup \tau_i)$ for some $i$, then (as long as $U$ is small enough), $f[U] = \{s_i\}$ which is open. Otherwise, $x \in \tau_i$ for some $i$. Since $f_T$ is open, $V \coloneqq f[U \cap \abs\Sig]$ is an open subset of $T$. To see that $f[U]$ is open then, it suffices to show that whenever $s_i \in \uset^F V \cap \Top(F)$, we have $U \cap \xi_i \neq \es$. So take such an $s_i$. Since $V$ is open in $T$, we must have $t_{i-1} \in V$ or $t_i \in V$. Without loss of generality, assume the former. Hence we must have $U \cap \tau_{i-1} \neq \es$. But then since $U$ is open, it follows that also $U \cap \xi_i \neq \es$.

		Thus $f \colon P \to F$ is an open surjective polyhedral map from a convex $n$-dimensional polyhedron, as required.
	\end{proof}
	\section{Conclusion and Future Work}
In this work, I design corruption-robust algorithms for the Lipschitz contextual search problem. I present the \emph{agnostic checking} technique and demonstrate its effectiveness in designing corruption-robust algorithms. There are several open problems for future research. First, in the algorithm I propose for pricing loss, the schedule for agnostic checks is fixed upfront. Can the learner design an adaptive checking schedule for the pricing loss? Second, this work assumes the learner has knowledge of the Lipschitz constant $L$. Can the learner design efficient no-regret algorithms without knowledge of $L$? 

	\printbibliography


\end{document}