%!TEX root = ../convex-paper.tex

\section{Polyhedral semantics}
\label{sec:polyhedral semantics}

With the preliminaries in place, we are in a position to illustrate the link between intuitionistic logic and polyhedra that is the main focus of this paper. Given a polyhedron $P$, let $\Sub P$ denote the collection of its subpolyhedra.

\begin{theorem}\label{thm:subP co-Heyting algebra}
	$\Sub P$ is a co-Heyting algebra, and a subalgebra of $\Closeds(P)$.
\end{theorem}

\begin{proof}
	See \cite[Corollary~3.8]{tarski-polyhedra}. 
\end{proof}

Any subpolyhedron of $P$ is by definition compact, and hence closed. Therefore it is not surprising, once the algebraic nature of $\Sub P$ is established, that it turns out to be a \emph{co-Heyting} algebra. In topology and logic, on the other hand, it is more conventional to work with open sets and \emph{Heyting} algebras. Thus, it is natural at this point to switch to the Heyting algebra dual to $\Sub P$, which has the following concrete realisation.

Given a polyhedron $P$, we will define an \emph{open subpolyhedron} of $P$ as the complement (in $P$) of a  subpolyhedron of $P$; that is, $O\subseteq P$ is an open subpolyhedron of $P$ precisely when the set-theoretic difference $P\setminus O$ is a member of $\Sub P$. 
\begin{remark}Let $P\subseteq \R^n$ be any polyhedron. It is worth pointing out explicitly that while a subpolyhedron of $P$ is a closed (and compact) set both in $P$ and in the ambient space $\R^n$, an open subpolyhedron of $P$ is by definition open in $P$ but may fail to be open in $\R^n$.
\end{remark}
Let us denote by $\Subo P$ the collection of open subpolyhedra in $P$. It is evidently the dual of $\Sub P$, and \cref{thm:subP co-Heyting algebra} yields the following.

\begin{theorem}\label{thm:suboP Heyting algebra}
	$\Subo P$ is a Heyting algebra, and a subalgebra of $\Opens(P)$.
\end{theorem}

The above  provides a sound semantics for intuitionistic logic in terms of polyhedra: for a polyhedron $P$, say that $P \vD \phi$ if and only if $\Subo P \vD \phi$ as a Heyting algebra. One of the features of this polyhedral semantics  is that it is complete for $\IPC$ --- à la Tarski. Moreover, in contrast with topological semantics, polyhedral semantics can detect dimension, via the bounded depth schema. Let $\Poly$ denote the class of all polyhedra, and let $\Poly_n$ denote the subclass consisting of polyhedra of dimension at most $n$, for each $n \in \NN$.

\begin{theorem}\label{thm:IPC logic of P and BDn logic of Pn}
	\begin{enumerate}[label=(\arabic*)]
		\item\label{item:IPC; thm:IPC logic of P and BDn logic of Pn} 
			$\IPC = \Logic(\Poly)$. That is, intuitionistic logic is complete with respect to the class of all polyhedra.
		\item\label{item:BDn; thm:IPC logic of P and BDn logic of Pn} 
			$\BD_n = \Logic(\Poly_n)$, for each $n \in \NN$.
	\end{enumerate}
\end{theorem}

\begin{proof}
	See \cite[Theorem~1.1]{tarski-polyhedra}. The proof works by showing that every finite poset of height $n$ can be `realised geometrically' in an $n$-dimensional polyhedron. The main idea behind this construction is recalled in \cref{ssec:geometric realisation} below.
\end{proof}


	The Triangulation Lemma provides a key piece of information about the polyhedral semantics of Theorem \ref{thm:IPC logic of P and BDn logic of Pn} --- namely, $\Subo P$ is a locally finite Heyting algebra\footnote{An algebraic structure is \emph{locally finite} if every finitely generated substructure is finite.} for any polyhedron $P$. Given any triangulation $\Sig$ of $P$, denote by $\Pc(\Sig)$ the sublattice of $\Closeds(P)$ generated by $\Sig$, and let:
	\begin{equation*}
		\Po(\Sig) \coloneqq \{P \setminus C \mid C \in \Pc(\Sig)\}
	\end{equation*}

	\begin{lemma}\label{lem:Po Sig iso Up Sig}
		$\Po(\Sig)$ is isomorphic as a Heyting algebra to $\Up \Sig$.
	\end{lemma}

	\begin{proof}
		See \cite[Lemma~4.3]{tarski-polyhedra}.
	\end{proof}

	\begin{theorem}\label{thm:Subo P locally-finite}
		Whenever $P \nvD \phi$ there is a triangulation $\Sig$ of $P$ such that $\Po(\Sig) \nvD \phi$. In particular, $\Subo P$ is locally finite.
	\end{theorem}

	\begin{proof}
		See \cite[Corollary~3.7]{tarski-polyhedra}.
	\end{proof}

