%!TEX root = ../convex-paper.tex

\section{The logic of convex polyhedra}
\label{sec:logic of convex polyhedra}

Recall from \cref{sec:preliminaries} that a polyhedron $P$ is \emph{convex} if $\Conv P = P$, in other words, if the segment joining any two points in $P$ lies entirely in $P$. Let $\PolyCon$ be the class of all convex polyhedra. We can now tackle the question: \emph{what is the logic of all convex polyhedra, $\Logic(\PolyCon)$?} The remainder of the paper will be devoted to a proof that $\Logic(\PolyCon) = \PL$, where $\PL$ is axiomatised by the Jankov-Fine formulas of two simple trees as follows.

\begin{equation*}
	\PL = \IPC + \chi(\FThreeFork) + \chi(\FScott)
\end{equation*}

\begin{theorem}\label{thm:PL logic of CP}
	$\PL$ is the logic of all convex polyhedra: $\PL = \Logic(\PolyCon)$.
\end{theorem}

We show this result by first restricting to the bounded dimension and bounded frame-depth situation, and then use the fact that $\PL$ has the finite model property to obtain the full result. Specifically, let $\PolyCon_n$ denote the class of convex polyhedra of dimension at most $n$, and define:
\begin{equation*}
	\PL_n \coloneqq \BD_n + \PL
\end{equation*}
The main job will be to prove the following.

\begin{theorem}\label{thm:PLn logic of CPn}
	$\PL_n = \Logic(\PolyCon_n)$, for each $n$.
\end{theorem}

\noindent This in turn splits into the following two directions, which will be proved in \cref{sec:soundness} and \cref{sec:completeness}, respectively.

\begin{theorem}[Soundness]\label{thm:PLn soundness for CPn}
	$\PL_n$ is valid on every $P \in \PolyCon_n$.
\end{theorem}

\begin{theorem}[Completeness]\label{thm:PLn completeness for CPn}
	If $\PL_n \nvd \phi$ then there is $P \in \PolyCon_n$ such that $P \nvD \phi$.
\end{theorem}

The final ingredient is the following result due to Zakharyaschev.

\begin{lemma}\label{lem:PL fmp}
	$\PL$ has the finite model property.
\end{lemma}

\begin{proof}
	This follows from the more general result \cite[Corollary 0.11, p.~20]{zakharyaschev93}. This result is stated in terms of `canonical formulas', which are a generalisation of Jankov-Fine formulas. Given a frame $Q$ and a set $\mathfrak D$ of antichains in $Q$ (sets of pairwise incompatible elements of $Q$), we can define the \emph{canonical formula} $\beta(Q, \mathfrak D, \bot)$, which satisfies a similar condition to that satisfied by Jankov-Fine formulas. The result states that if an intermediate logic $\Lo$ is axiomatised by a set of canonical formulas $\beta(Q, \mathfrak D, \bot)$ such that in every $A \in \mathfrak D$ there is at least one point not lying below all maximal points in $\uset A$, then $\Lo$ has the finite model property.
	
	Now, given any frame $Q$, the Jankov-Fine formula $\chi(Q)$ is equivalent to $\beta(Q, \mathfrak D^\#, \bot)$, where $\mathfrak D^\#$ is the set of non-singleton antichains in $Q$ \cite[Proposition 9.41 (i), p.~312]{chagrovzakharyaschev1997}. It is then clear to see that $\chi(\FThreeFork)$ and $\chi(\FScott)$ satisfy the requisite conditions, so the result yields that $\PL$ has the finite model property.
\end{proof}

These lemmas then combine to give the ultimate result.

\begin{proof}[Proof of \cref{thm:PL logic of CP}]
	\cref{lem:PL fmp} entails that:
	\begin{equation*}
		\PL = \bigcap_{n \in \NN} \PL_n
	\end{equation*}
	On the other hand, since all our polyhedra have finite dimension:
	\begin{equation*}
		\PolyCon = \bigcup_{n \in \NN} \PolyCon_n
	\end{equation*}
	Therefore:
	\begin{equation*}
		\Logic(\PolyCon) = \bigcap_{n \in \NN} \Logic(\PolyCon_n)
	\end{equation*}
	\cref{thm:PLn logic of CPn} then completes the proof.
\end{proof}


\subsection{The Logic of a single convex polyhedron}\label{ss:opensimplex}

Any two $n$-simplices $\sigma\subseteq\R^d$ and $\tau\subseteq \R^{d'}$ are PL-homeomorphic --- in fact, affinely homeomorphic. Indeed, since affine maps commute with affine combinations, any bijection of the vertex set of $\sigma$ onto the vertex set of $\tau$ lifts to exactly one bijective affine map $\Aff \sigma \to \Aff \tau$.  Let $e_0, \ldots, e_n$ be the standard basis vectors of $\R^{n+1}$. The \emph{standard $n$-simplex} is $\Delta_n \coloneqq \Conv\{e_0, \ldots, e_n\}$. The following is a classical result.

\begin{lemma}\label{lem:convex n-dimensional polyhedra pl homeomorphic to n-simplex}
	Every $n$-dimensional convex polyhedron is PL-homeomorphic to $\Delta_n$.
\end{lemma}

\begin{proof}
	See \cite[Corollary~2.20, p.~21]{rourkesanderson1972}. There it is shown that \emph{$n$-cells} --- which correspond to our $n$-dimensional convex polyhedra --- are \emph{$n$-balls} --- meaning that they are PL-homeomorphic to the $n$-dimensional cube $[0,1]^n$. Since $\Delta_n$ is a convex polyhedron, the result follows.
\end{proof}

Thus, the logic of all convex polyhedra of dimension at most $n$ is just the logic of any given $n$-dimensional such polyhedron, for instance the   $n$-simplex.

\begin{corollary}\label{cor:logic CPn is logic of n-simplex}
	For any $n$-dimensional convex polyhedron $P$,
	$\Logic(\PolyCon_n) = \Logic(\Delta_n)=\Logic(P)$. 
\end{corollary}

\begin{proof}
	This is immediate from \cref{lem:convex n-dimensional polyhedra pl homeomorphic to n-simplex} using \cref{cor:PL homeomorphic implies logics same}.
\end{proof}

Next, given a convex polyhedron $P$, we are interested in determining the logic of its topological interior in $\Aff P$ --- that is, the logic of a convex open polyhedron of dimension $n$. In the special case that $P$ is an $n$-simplex $\sigma$, its topological interior in $\Aff \sigma$ coincides with its relative interior $\Relint \sigma$.

\begin{lemma}\label{lem:map_of_cubes}
	There exists a surjective open polyhedral map $(0,1)^n\to [0,1]^n$.
\end{lemma}

\begin{proof}
	Let us first assume $n=1$. Consider real numbers $a'<x<a<b<y<b'$. We define a function $f\colon [a',b']\to [x,y]$ by prescribing its action on vertices:
	\[
	f(a')=a, f(b')=b, f(x)=x, f(a)=a,f(b)=b,f(y)=y\,,
	\]
	and  by completing the definition of $f$ through affine extension. Then $f$ is a surjective PL map. Its restriction $g$ to $(a',b')$ is  a polyhedral map that is evidently still surjective onto $[x,y]$, and is moreover open. (To verify $f$ is open let $(\alpha,\beta)\subseteq(a',b')$. If $x\leq \alpha$ and $\beta\leq y$ then $f[(\alpha,\beta)]=(\alpha,\beta)$. If $\alpha\leq x$ and $y\leq \beta$ then $f[(\alpha,\beta)]=[x,y]$. If $\alpha\leq x$ and $\beta\leq y$ then $f[(\alpha,\beta)]=[x,\beta)$. Hence $f$ is open.) This shows the existence of a surjective open polyhedral map $g\colon (0,1)\to [0,1]$ that is the restriction to $(0,1)$ of a PL map $[0,1]\to[0,1]$.

	For $n>1$, consider the product of maps $F\coloneqq f\times\cdots \times f\colon [0,1]^n\to [0,1]^n$ and its restriction to $(0,1)^n$, $G\coloneqq g\times \cdots \times g\colon (0,1)^n\to [0,1]^n$. Then $F$ is PL. Indeed, its graph is the $n$-fold product of copies of the graph of $f$, and the latter graph is a polyhedron because $f$ is PL; hence the graph of $F$ is a polyhedron, too, using the standard fact  that a finite product of polyhedra is a polyhedron. Since $F$ is continuous \cite[Proposition 2.3.6 and p.\ 78]{EngelkingRyszard1989Gt}, and its graph is a polyhedron, then  $F$ is PL (\cref{rem:PL_graph}). This  entails  that $G$ is polyhedral: if $O\in \Subo [0,1]^n$, $F^{-1}[O]\in \Subo [0,1]^n$ because $F$ is PL; then $G^{-1}[O]= F^{-1}[O]\cap (0,1)^n \in\Subo (0,1)^n$. Finally, since a finite product of open maps is open \cite[Proposition 2.3.29]{EngelkingRyszard1989Gt}, $G$ is open.
\end{proof}

\begin{lemma}\label{lem:open_polytope_logic}
	Let $P$ be any convex polyhedron, and let $O$ be its topological interior in $\Aff P$. Then $\Logic(P)=\Logic(O)$.
\end{lemma}

\begin{proof}
	Assume $P$ is of dimension $n$. By \cref{lem:convex n-dimensional polyhedra pl homeomorphic to n-simplex} there is a PL-homeomorphism $f\colon P\to \Delta_n$ with inverse $f^{-1}\colon \Delta_n\to P$ which also is PL (\cref{lem:inversePL}). Hence by \cref{cor:PL homeomorphic implies logics same} we have $\Logic(P)=\Logic(\Delta_n)$. By an elementary topological argument, $f$ and $f^{-1}$ descend to mutually inverse homeomorphisms  $g \colon O \to \Relint \Delta_n$ and $g^{-1}\colon \Relint \Delta_n \to O$. These homeomorphisms are polyhedral because $f$ and $f^{-1}$ are PL. Hence, \cref{lem:maps duality} entails $\Logic{O}=\Logic(\Relint \Delta_n)$. Thus it suffices to prove the lemma for $P=\Delta_n$ and $O=\Relint \Delta_n$.


	The inclusion map $\iota\colon\Relint \Delta_n \to \Delta$ is an injective open polyhedral map, so that its dual $\iota^*\colon \Subo \Delta_n\to \Subo \Relint \Delta_n$ is a surjective homomorphism of Heyting algebras by \cref{lem:maps duality}, which entails $\Logic(\Delta_n)\subseteq\Logic(\Relint \Delta_n)$. For the converse inclusion, \cref{lem:map_of_cubes} and \cref{lem:maps duality} entail $\Logic((0,1)^n)\subseteq \Logic([0,1]^n)$. The  argument in the previous paragraph yields $\Logic(\Delta_n)=\Logic([0,1]^n)$ and $\Logic(\Relint \Delta_n)=\Logic((0,1)^n)$, which completes the proof.
\end{proof}


\subsection{The largest logic}

The importance of convex polyhedra is mirrored on the logical side.

\begin{theorem}\label{thm:largest logic}
	\begin{enumerate}[label=(\arabic*)]
		\item\label{item:infty; thm:largest logic}
		$\PL$ is the largest polyhedrally complete logic of height $\infty$.
		\item\label{item:n; thm:largest logic}
		$\PL_n$ is the largest polyhedrally complete logic of height $n$, for each $n \in \NN$.
	\end{enumerate}
\end{theorem}

The starting point to prove the above theorem  is the observations that every $n$-dimensional polyhedron contains a convex polyhedron of that dimension.

\begin{lemma}\label{lem:convex in every poly}
	If $P$ is $n$-dimensional polyhedron and $m \leq n$ then there is $Q$ an $m$-dimensional convex polyhedron with $Q \subseteq P$.
\end{lemma}

\begin{proof}
	Let $\Sig$ be a triangulation of $P$. Since $P$ has dimension $n$, there is a simplex $\sig \in \Sig$ which has height $m$ (when viewing $\Sig$ as a poset). Then $\sig\subseteq P$ is an $m$-simplex, which is by definition convex.
\end{proof}

The remaining part of the proof  rests on the results of   \cref{ss:opensimplex}.

\begin{proof}[Proof of \cref{thm:largest logic}]
	To prove \ref{item:n; thm:largest logic}, let $\Lo$ be a polyhedrally complete logic of height $n$. Then $\Lo = \Logic(\C)$ for some class $\C$ of polyhedra. We claim that $\C$ contains a polyhedron of dimension at least $n$. Indeed, otherwise $\C \sse \Poly_{n-1}$ so that by \cref{thm:IPC logic of P and BDn logic of Pn} we have:
	\begin{equation*}
		\BD_{n-1} = \Logic(\Poly_{n-1}) \sse \Logic(\C) = \Lo
	\end{equation*}
	By \cref{lem:BDn specifies height} this means that $\Lo$ cannot have frames of height $n$, a contradiction. \contradiction

	So take $P \in \C$ of dimension at least $n$. Then by \cref{lem:convex in every poly} there is $Q$ a convex $n$-dimensional polyhedron with $Q \subseteq P$. Let $O$ be the topological interior of $Q$ in $\Aff Q$. The inclusion $O\subseteq P$ is an open injective polyhedral map, so by \cref{lem:maps duality} we have $\Logic(P) \sse\Logic(O)$. But by \cref{lem:open_polytope_logic} we also have $\Logic(O)=\Logic(Q)$, and by \cref{cor:logic CPn is logic of n-simplex} we know $\Logic(Q) = \PL_n$; hence:
\begin{equation*}
		\Lo = \Logic(\C) \sse \Logic(P) \sse \Logic(O)=\Logic(Q) = \Logic(\Delta_n) = \PL_n
	\end{equation*}

	To prove \ref{item:infty; thm:largest logic}, let $\Lo = \Logic(\C)$ be a polyhedrally complete logic of height $\infty$. We can write $\C = \bigcup_{n \in \NN} \C_n$, where $\C_n = \C \cap \Poly_n$. Then:
	\begin{equation*}
		\Lo = \Logic(\C) = \Logic \left(\bigcup_{n \in \NN} \C_n\right) = \bigcap_{n \in \NN} \Logic (\C_n) \sse \bigcap_{n \in \NN} \PL_n = \PL
	\end{equation*}
	where in the penultimate containment we have used \ref{item:n; thm:largest logic}, and for the last equality we have used that $\PL$ has the finite model property.
\end{proof}