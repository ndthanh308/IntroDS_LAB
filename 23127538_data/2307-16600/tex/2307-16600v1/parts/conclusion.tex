%!TEX root = ../convex-paper.tex

\section{Conclusion}
\label{sec:conclusion}

In this article, we have provided an axiomatisation of the logic of the class of convex polyhedra. This result fits into a natural programme of investigation, initiated in \cite{tarski-polyhedra} and continued in \cite{polycompleteness}, which seeks to map out the landscape of polyhedrally complete logics. 

In \cite{polycompleteness} it is shown that there are infinitely many polyhedrally complete logics of each height, axiomatised by the Jankov-Fine formulas of `starlike trees'. This in particular includes Scott's logic $\SL$. Beyond these results, \cite{gabelaiatacl} investigates the lower-level structure of this landscape in more detail. First, it is shown that every height-$1$ logic is polyhedrally complete: these are $\BD_1$ plus the logic $\LF_k$ of the `$k$-fork' --- the frame consisting of a root with $k$ immediate successors --- for each $k \geq 2$. Second, turning to the height-$2$ case, the focus is on logics of `flat polygons': $2$-dimensional polyhedra which can be embedded in the plane $\R^2$. Any such logic turns out to be axiomatised by a \emph{subframe formula} (see \cite[p.~313]{chagrovzakharyaschev1997}) plus the Jankov-Fine formulas of certain trees. Moreover, there is a smallest such logic: $\Flat_2$.
\Cref{fig:landscape of poly-complete logics} charts out what is currently known about the landscape of polyhedrally complete logics, to the best of our knowledge. 

% Figure environment removed

One long-term goal is the complete classification of all polyhedrally complete logics. This article presented one schema for attacking this problem: starting with a natural class of polyhedra and asking what its logic is. For this it is important to be able to find a geometric realisation of any frame of a candidate logic in the class of polyhedra under consideration. By contrast, in \cite{polycompleteness} another schema is followed. There we start from the logic side and define a class of logics with the aim that they are polyhedrally complete, making use of the Nerve Criterion for polyhedral completeness.

\vspace{3mm}

\noindent
{\bf Acknowledgement} The authors would  like to acknowledge support by the SRNSF Grant \#FR-22-6700.