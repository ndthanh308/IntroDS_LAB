%!TEX root = ../convex-paper.tex

\section{Preliminaries}
\label{sec:preliminaries}

	The present paper deals with intermediate logics. In this section we remind the reader of two standard semantics for such logics, and survey the definitions and results which will play their part in what follows. We also present the basic notions of polyhedral geometry that we need in the paper. 


\subsection{Posets as Kripke frames}

	A \emph{Kripke frame} for intuitionistic logic is simply a poset $(F,\leq)$. The validity relation $\vD$ between frames and formulas is defined in the usual way. Given a class of frames $\C$, its \emph{logic} is:
	\begin{equation*}
		\Logic(\C) \coloneqq \{\phi \text{ a formula } \mid \forall F \in \C \colon F \vD \phi \}
	\end{equation*}
	Conversely, given a logic $\Lo$, define:
	\begin{equation*}
		\Frames(\Lo) \coloneqq \{F\text{ a Kripke frame} \mid F \vD \Lo\}
	\end{equation*}
	A logic $\Lo$ has the \emph{finite model property} (f.m.p.) if it is the logic of a class of finite frames.

	Fix a poset $F$. For any $x \in F$, its \emph{upset} and \emph{downset} are defined, respectively, as follows.
	\begin{gather*}
		\us x \coloneqq \{y \in F \mid y \geq x\} \\
		\ds x \coloneqq \{y \in F \mid y \leq x\}
	\end{gather*}
	For any set $S \sse F$, its \emph{upset} and \emph{downset} are defined, respectively, as follows.
	\begin{gather*}
		\uset U \coloneqq \bigcup_{x \in U} \us x\\
		\dset U \coloneqq \bigcup_{x \in U} \ds x
	\end{gather*}
	A subframe $U \sse F$ is \emph{upwards-closed} if $U = \uset U$. It is \emph{downwards-closed} if $\dset U = U$. The \emph{Alexandrov topology} on $F$ is the set $\Up F$ of its upwards-closed subsets. This constitutes a topology on $F$. In the sequel, we will freely switch between thinking of $F$ as a poset and as a topological space. Note that the closed sets in this topology correspond to downwards-closed sets.

	A \emph{chain} in $F$ is $X \sse F$ which as a subposet is linearly-ordered. The \emph{length} of the chain $X$ is $\abs X$. A chain $X \sse F$ is maximal if there is no chain $Y \sse F$ such that $X \subset Y$ (i.e. such that $X$ is a proper subset of $Y$). The \emph{height} of $F$ is the element of $\NN \cup \{\infty\}$ defined by:
	\begin{equation*}
		\height(F) \coloneqq \sup\{\abs X-1 \mid X \sse F\text{ is a chain}\}
	\end{equation*}
	For any $x \in F$, define its \emph{height} as follows.
	\begin{gather*}
		\height(x) \coloneqq \height(\ds x)
	\end{gather*}

	The poset $F$ is \emph{rooted} if it has a minimum element, which is called the \emph{root}, and is usually denoted by $\bot$. Define:
	\begin{equation*}
		\FramesRoot(\Lo) \coloneqq \{F \in \Frames(\Lo) \mid F\text{ is rooted}\}
	\end{equation*}

	A function $f \colon F \to G$ is a \emph{p-morphism} if for every $x \in F$ we have:
	\begin{equation*}
		f(\us x) = \us{f(x)}
	\end{equation*}
	An \emph{up-reduction} from $F$ to $G$ is a surjective p-morphism $f$ from an upwards-closed set $U \sse F$ to $G$. Write $f \colon F \cra G$.

	\begin{lemma}\label{lem:up-reduction logic containment}
		If there is an up-reduction $F \cra G$ then $\Logic(F) \sse \Logic(G)$. In other words, if $G \nvD \phi$ then $F \nvD \phi$.
	\end{lemma}

	\begin{proof}
		See \cite[Corollary~2.8, p.~30 and Corollary~2.17, p.~32]{chagrovzakharyaschev1997}.
	\end{proof}

	\begin{corollary}\label{cor:logic of frames logic of rooted frames}
		If $\C$ is any collection of frames and $\Lo = \Logic(\C)$, then:
		\begin{equation*}
			\Lo = \Logic(\FramesRoot(\Lo))
		\end{equation*}
	\end{corollary}

	\begin{proof}
		First, $\Lo \sse \Logic(\FramesRoot(\Lo))$. Conversely, suppose $\Lo \nvd \phi$. Then there exists $F \in \C$ such that $F \nvD \phi$, hence there is $x \in F$ such that $x \nvD \phi$ (for some valuation on $F$), meaning that $\us x \nvD \phi$. Now, $\us x$ is upwards-closed in $F$, hence $\id_{\us x}$ is an up-reduction $F \cra \us x$. Then by \cref{lem:up-reduction logic containment}, we get that $\us x \vD \Lo$, so that $\us x \in \FramesRoot(\Lo)$.
	\end{proof}

	Let $\IPC$ be the logic of all finite frames, and let $\BD_n$ be the logic of all finite frames of height at most $n$. 
	
	\begin{lemma}\label{lem:BDn specifies height}
		Let $F$ be a finite frame. Then $F \vD \BD_n$ if and only if $F$ has height at most $n$.
	\end{lemma}

	\begin{proof}
		See \cite[Proposition~2.38]{chagrovzakharyaschev1997}
	\end{proof}


\subsection{Heyting and co-Heyting algebras}

	A \emph{Heyting algebra} is a tuple $(A,\wedge,\vee,\ra,0,1)$ such that $(A,\wedge,\vee,0,1)$ is a bounded lattice and $\ra$, called the \emph{Heyting implication}, satisfies:
	\begin{equation*}
		c \leq a \ra b \quad\Lra\quad c \wedge a \leq b
	\end{equation*}
	The validity relation $\vD$ between Heyting algebras and formulas is defined in the usual way; the $\Logic$ notation is extended appropriately. Topological spaces provide important examples of Heyting algebras: for every topological space $X$, its collection of open sets $\Opens(X)$ forms a Heyting algebra. We recall that for $U,V\in\Opens(X)$ we have \[U\to V=\bigcup\{Z\in\Opens(X)\mid Z\cap U\subseteq V\}=\Int(U^C\cup V),\]
	where $\Int(-)$ denotes the interior operator and $\comp {(-)}$ denotes set-theoretic complement.

	Co-Heyting algebras are the duals of Heyting algebras. Specifically, a \emph{co-Heyting algebra} is a tuple $(C, \wedge, \vee, \la, 0, 1)$ such that $(C,\wedge,\vee,0,1)$ is a bounded lattice, and $\la$, called the \emph{co-Heyting implication}, satisfies:
	\begin{equation*}
		a \la b \leq c \quad\Lra\quad a \leq b \vee c
	\end{equation*}
	For more information on co-Heyting algebras, the reader is referred to \cite[\S1]{mckinseytarski1946} and \cite{rauszer1974}, where they are called `Brouwerian algebras'.

	Any Heyting algebra $A$ may be regarded as a category. Then its dual category $A^\mathrm{op}$ is a co-Heyting algebra. In the case of the Heyting algebra $\Opens(X)$ of open sets in a topological space, such a duality has a concrete realisation: the co-Heyting algebra $\Opens(X)^\mathrm{op}$ is the algebra $\Closeds(X)$ of closed subsets of $X$.


\subsection{Finite Esakia duality}

	The Alexandrov topology allows us to associate to each poset $F$ the Heyting algebra $\Up F$ consisting of its upwards-closed sets. The process forms part of a contravariant equivalence of categories, known as the Esakia Duality. The finite fragment of this duality relates finite posets with finite Heyting algebras.

	The \emph{spectrum} of a Heyting algebra $A$ is defined as follows. 
	\begin{equation*}
		\Spec(A) \coloneqq \{X \sse A \mid X \text{ is a prime filter of }A\text{ as a distributive lattice}\}
	\end{equation*}
	This constitutes a poset under subset inclusion. 

	\begin{theorem}\label{thm:finite Esakia duality}
		The maps $\Up$ and $\Spec$ are the object-level components of a duality between the category of finite Kripke frames with p-morphisms and the category of finite Heyting algebras with homomorphisms.
	\end{theorem}

	\begin{proof}
		For a proof of the full Esakia Duality see \cite[Corollary~3.4.8]{esakia2019}, which is a translation of the original \cite{esakia1985}. The correspondence was first established in \cite{esakia1974}. Further proofs in English can also be found in \cite{celanijansana2014} and \cite[\S 5]{morandi2005}.
		
		For the finite part, see \cite{dejonghtroelstra1966}. Here, we have isomorphisms $A \cong \Up \Spec A$ and $F \cong \Spec \Up F$ for any finite Heyting algebra $A$ and finite poset $F$. The former is part of Brikhoff's Representation Theorem \cite{birkhoff1937}. Both isomorphisms may be found in \cite[pp.~171-172]{daveypriestley1990}.
	\end{proof}

	Importantly, this duality is logic-preserving.

	\begin{lemma}\label{lem:finite esakia duality logic-preserving}
		Let $F$ be a frame and $A$ be a finite Heyting algebra. Then:
		\begin{gather*}
			\Logic(F) = \Logic(\Up F) \\
			\Logic(A) = \Logic(\Spec A)
		\end{gather*}
	\end{lemma}

	\begin{proof}
		For the first equality, see \cite[Corollary~8.5, p.~238]{chagrovzakharyaschev1997}, noting that our Kripke frames are special cases of what are there called `intuitionistic general frames'. The second equality follows from the first and the finite Esakia duality.
	\end{proof}

\subsection{Jankov-Fine formulas as forbidden configurations}

	To every finite rooted frame $Q$, we associate a formula $\chi(Q)$, the \emph{Jankov-Fine} formula of $Q$ (also called its \emph{Jankov-De Jongh formula}). The precise definition of $\chi(Q)$ is somewhat involved, but the exact details of this syntactical form are not relevant for our considerations. What matters to us is its notable semantic property.

	\begin{theorem}\label{thm:Jankov-Fine up-reductions}
		For any frame $F$, we have that $F \vD \chi(Q)$ if and only if $F$ does not up-reduce to $Q$.
	\end{theorem}

	\begin{proof}
		See \cite[\S9.4, p.~310]{chagrovzakharyaschev1997}, for a treatment in which Jankov-Fine formulas are considered as specific instances of more general `canonical formulas'. A more direct proof is found in \cite[\S3.3, p.~56]{bezhanishvili2006}, which gives a complete definition of $\chi(Q)$. See also \cite{bezhbezh2009} for an algebraic version of this result.
	\end{proof}

	Jankov-Fine formulas formalise the intuition of `forbidden configurations'. The formula $\chi(Q)$ `forbids' the configuration $Q$ from its frames.


\subsection{Polyhedra and simplices}

	Every polyhedron considered here lives in some Euclidean space $\R^n$. Take finitely many points $x_0, \ldots, x_d \in \R^n$. An \emph{affine combination} of $x_0, \ldots, x_d$ is a point $r_0 x_0 + \cdots + r_d x_d$, specified by some $r_0, \ldots, r_d \in \R$ such that $r_0 + \cdots + r_d = 1$. Given a set $S \sse \R^n$, its \emph{affine hull} $\Aff S$ is the collection of affine combinations of its elements. A \emph{convex combination} is an affine combination in which additionally each $r_i \geq 0$. Given a set $S \sse \R^n$, its \emph{convex hull} $\Conv S$ is the collection of convex combinations of its elements. A subset $S \sse \R^n$ is \emph{convex} if $\Conv S = S$. A \emph{polytope} is the convex hull of a finite subset of $\R^n$. A \emph{polyhedron} in $\R^n$ is a set which can be expressed as the finite union of polytopes.  
	
\begin{remark}	A  remark on terminology is in order. In our usage of the term `polyhedron' does not imply convexity, and is the standard one in piecewise-linear topology --- c.f.\@ classic textbooks \cite{stallings1967,rourkesanderson1972}) --- with the following additional conventions. A `polyhedron' \textit{tout court}, as defined in PL topology, need not be compact as a subspace of Euclidean space.  Now, it is a standard fact that `compact polyhedra' (in this more general sense) coincide with what we are referring to in this paper as `polyhedra' (see \cite[Theorem~2.2, p.~12]{rourkesanderson1972}). Hence we are effectively using the term `polyhedron' as a shorthand for `compact polyhedron'. Such abbreviated usage is frequent in the literature (see e.g. \cite{maunder1980algebraic}). Finally,  in our terminology, a `convex polyhedron' is the same thing as a `polytope' --- we will  use the former expression from now on.
\end{remark}
	
	A set of points $x_0, \ldots, x_d$ is \emph{affinely independent} if whenever:
	\begin{equation*}
		r_0 x_0 + \cdots + r_d x_d = \mathbf 0 \quad\text{and}\quad r_0 + \cdots + r_d = 0
	\end{equation*}
	we must have that $r_0=\cdots=r_d = 0$. This is equivalent to saying that the vectors
	\begin{equation*}
		x_1 - x_0, \ldots, x_d - x_0
	\end{equation*}
	are linearly independent. A \emph{$d$-simplex} is the convex hull $\sig$ of $d+1$ affinely independent points $x_0, \ldots, x_d$, which we call its \emph{vertices}. Write $\sig = x_0\cdots x_d$; its \emph{dimension} is $\Dim\sig \coloneqq d$.

	\begin{lemma}
		Every simplex determines its vertex set: two simplices coincide if and only if they share the same vertex set.
	\end{lemma}

	\begin{proof}
		See \cite[Proposition 2.3.3, p.~32]{maunder1980algebraic}.
	\end{proof}

	\noindent A \emph{face} of $\sig$ is the convex hull $\tau$ of some non-empty subset of $\{x_0, \ldots, x_d\}$ (note that $\tau$ is then a simplex too). Write $\tau \preceq \sig$, and $\tau \prec \sig$ if $\tau \neq \sig$.

	Since $x_0, \ldots, x_d$ are affinely independent, every point $x \in \sig$ can be expressed uniquely as a convex combination $x = r_0 x_0 + \cdots + r_d x_d$ with $r_0, \ldots, r_d \geq 0$ and $r_0 + \cdots + r_d = 1$. Call the tuple $(r_0, \ldots, r_d)$ the \emph{barycentric coordinates} of $x$ in $\sig$. The \emph{barycentre} $\wh\sig$ of $\sig$ is the special point whose barycentric coordinates are $(\frac{1}{d+1}, \ldots, \frac{1}{d+1})$. The \emph{relative interior} of $\sig$ is defined as follows.
	\begin{equation*}
		\Relint \sig \coloneqq \{r_0 x_0 + \cdots + r_d x_d \in \sig \mid r_0, \ldots, r_d >0\}
	\end{equation*}
	 Then the relative interior of $\sig$ coincides with the topological interior of $\sig$ inside its affine hull. Note that $\Cl\Relint\sig = \sig$, the closure being taken in the ambient space $\R^n$.
	 


\subsection{Triangulations}

	A \emph{simplicial complex} in $\R^n$ is a finite set $\Sig$ of simplices satisfying the following conditions.
	\begin{enumerate}[label=(\alph*)]
		\item \label{item:downwards-closed; defn:simplicial complex}
		$\Sig$ is $\prec$-downwards-closed: whenever $\sig \in \Sig$ and $\tau \prec \sig$ we have $\tau \in \Sig$.
		\item \label{item:intersection; defn:simplicial complex}
		If $\sig,\tau \in \Sig$, then $\sig \cap \tau$ is either empty or a common face of $\sig$ and $\tau$.
	\end{enumerate}
	The \emph{support} of $\Sig$ is the set $\abs\Sig \coloneqq \bigcup \Sig$. Note that by definition this set is automatically a polyhedron. We say that $\Sig$ is a \emph{triangulation} of the polyhedron $\abs\Sig$. See \cref{fig:triangulations of medley of polyhedra} for some examples of triangulations. 

	% Figure environment removed
	
	Notice that $\Sig$ is a poset under $\prec$, called the \emph{face poset}. A \emph{subcomplex} of $\Sig$ is a subset which is itself a simplicial complex. Note that a subcomplex, as a poset, is precisely a downwards-closed set. Given $\sig \in \Sig$, its \emph{open star} is defined:
	\phantomsection\label{defn:open star}
	\begin{equation*}
		\ostar(\sig) \coloneqq \bigcup \{\Relint \tau \mid \tau \in \Sig \text{ and }\sig \sse \tau\}
	\end{equation*}

		
		\begin{lemma}\label{lem:open star open}
			The open star $\ostar(\sig)$ of any simplex $\sig$ is open in $\abs\Sig$.
		\end{lemma}

		\begin{proof}
			See \cite[Proposition~2.4.3, p.~43]{maunder1980algebraic}.
		\end{proof}



	\begin{lemma}\label{lem:relints partition support}
		The relative interiors of the simplices in a simplicial complex $\Sig$ partition $\abs\Sig$. That is, for every $x \in \abs\Sig$, there is exactly one $\sig \in \Sig$ such that $x \in \Relint\sig$.
	\end{lemma}

	\begin{proof}
		See \cite[Proposition~2.3.6, p.~33]{maunder1980algebraic}.
	\end{proof}

	In light of \cref{lem:relints partition support}, for any $x \in \abs\Sig$ let us write $\sig^x$ for the unique $\sig \in \Sig$ such that $x \in \Relint\sig$.

	\begin{lemma}\label{lem:simplex relint excludes proper faces}
		Let $\Sig$ be a simplicial complex, take $\tau \in \Sig$ and $x \in \Relint\tau$. Then no proper face $\sig \prec \tau$ contains $x$. This means that $\tau^x = \Relint\tau$ is the inclusion-smallest simplex containing $x$.
	\end{lemma}

	\begin{proof}
		See \cite[Lemma~3.1]{tarski-polyhedra}.
	\end{proof}

	The next result is a basic fact of polyhedral geometry, and is of fundamental importance in its connection with logic. For $\Sig$ a triangulation and $S$ a subspace of the ambient Euclidean space $\R^n$, define:
	\begin{equation*}
		\Sig_S \coloneqq \{\sig \in \Sig \mid \sig \sse S\}
	\end{equation*}
	This, being a downwards-closed subset of $\Sig$, is a subcomplex of $\Sig$.

	\begin{lemma}[Triangulation Lemma]\label{lem:triangulation lemma}
		Any polyhedron admits a triangulation which simultaneously triangulates each of any fixed finite set of subpolyhedra. That is, for a collection of polyhedra $P, Q_1, \ldots, Q_m$ such that each $Q_i \sse P$, there is a triangulation $\Sig$ of $P$ such that $\Sig_{Q_i}$ triangulates $Q_i$ for each $i$.
	\end{lemma}

	\begin{proof}
		See \cite[Theorem~2.11 and Addendum~2.12, p.~16]{rourkesanderson1972}.
	\end{proof}

	




\subsection{Dimension theory}

	The \emph{dimension} of simplicial complex $\Sig$ is:
	\begin{equation*}
		\Dim\Sig \coloneqq \max\{\Dim\sig \mid \sig \in \Sig\}
	\end{equation*}

	\begin{remark}
		Note that $\Dim\Sig = \height(\Sig)$ as a poset.
	\end{remark}

	\begin{lemma}\label{lem:dimension invariant under triangulation}
		Let $\Sig,\Delta$ be simplicial complexes. If $\abs\Sig = \abs\Delta$ then $\Dim\Sig=\Dim\Delta$.
	\end{lemma}

	\begin{proof}
		See \cite[Proposition~1.6.12, p.~30]{stallings1967}.
	\end{proof}

	\noindent With this in mind, we define the \emph{dimension} $\Dim P$ of a polyhedron $P$ to be the dimension of its triangulations. When $P = \es$, let $\Dim P \coloneqq -1$.

	\begin{lemma}\label{lem:dimension of union}
		$\Dim(P \cup Q) = \max\{\Dim P, \Dim Q\}$.
	\end{lemma}

	\begin{proof}
		By the Triangulation Lemma \ref{lem:triangulation lemma} we can find a triangulation $\Sig$ of $P \cup Q$ such that $\Sig_P$ and $\Sig_Q$ triangulate $P$ and $Q$ respectively. Since $\Sig = \Sig_P \cup \Sig_Q$ and both $\Sig_P$ and $\Sig_Q$ are downwards-closed the result follows.
	\end{proof}


		In the following, it will be necessary to consider the dimensions of sets which are not polyhedra but whose topological closures are. Note that it is possible to define a theory of dimension which applies even more generally \cite{hurewicz-wallmann}, however here we only need to apply it to sets of this form, and the resulting definition is simpler.

		Let $X \sse \R^n$ be such that $\Cl X$ is a polyhedron, where $\Cl X$ denotes the topological closure taken in the ambient space. The \emph{dimension} of $X$ is the dimension of its closure:
		\begin{equation*}
			\Dim X \coloneqq \Dim \Cl X
		\end{equation*}
		\begin{remark}From now on, when we refer to a set $X$ which has dimension, we tacitly assume that its closure is a polyhedron.\end{remark}
		
		Let us  consider the relationship between the dimension operator and the boundary operator. The \emph{boundary} of a set $X$ is $\partial X \coloneqq \Cl^{\Aff} X \setminus \Int^{\Aff} X$, where the closure and interior operations are taken with respect to the affine hull $\Aff X$ (note that $\Cl^{\Aff} X = \Cl X$ in the ambient space, because any affine subspace of $\R^n$ is closed). Then:
		\begin{lemma}\label{lem:dimension of boundary}
			For any set $X$ whose closure is a non-empty polyhedron we have that:
			\begin{equation*}
				\Dim(\partial X) = \Dim(X) - 1
			\end{equation*} 
		\end{lemma}

		\begin{proof}
			See \cite[Corollary~IV.II, p.~46]{hurewicz-wallmann}.
		\end{proof}
