%!TEX root = ../convex-paper.tex

\section{Logic, polyhedra and morphisms}
\label{sec:polyhedral maps}

In this section we develop assorted functorial aspects of  polyhedral semantics for intermediate logics which are essential ingredients in the main findings of the present paper.


\subsection{Homomorphisms induced by maps of spaces}

We begin with a  result that  requires some preliminary technical definitions.

For $X$ a topological space, by a \emph{lattice basis} for $X$ we mean a sublattice $L$ of the topology $\Opens(X)$ of $X$ that is a basis for that topology. If $L$ is moreover a Heyting subalgebra of the Heyting algebra $\Opens(X)$, we call $L$ a \emph{Heyting basis}.  

If $X$ is a space with a specified Heyting basis $L$ then we define
\[
\Logic(X)\coloneqq \Logic(L),
\]
where in the left-hand side we assume the basis $L$ is understood from context.

For any set $A$, write $\P(A)$ for the complete Boolean algebra of all subsets of $A$. For any function $f\colon A\to B$ between sets, write $f^{-1}\colon\P(B)\to\P(A)$ for the inverse-image function --- given $S\subseteq B$, $f^{-1}[S]\coloneqq\{a \in A \mid f(a)\in S\}$. Then $f^{-1}$ is a homomorphism of Boolean algebras that moreover preserves arbitrary joins and meets.


Now consider spaces $X$ and $Y$ with prescribed lattice bases $L$ and $M$, respectively. A function $f\colon X\to Y$ is \emph{bases-continuous} if $f^{-1}[S]\in L$ for each $S\in M$. Such functions are, of course, continuous. In general, a  function $f\colon X \to Y$ is \emph{open} if $f[U]\in \Opens(Y)$ for each $U \in \Opens(X)$. When $X$ and $Y$ come with prescribed lattice bases $L$ and $M$, let us say that a function $f$ is \emph{bases-open} if $f[U]\in M$ for each $U \in L$. It is clear that such a bases-open  function is open, because the direct-image function $f[-]$ preserves arbitrary unions.

\begin{lemma}\label{lem:maps duality}
	Let  $f \colon X \to Y$ be a function between spaces $X$ and $Y$ with prescribed lattice bases $L$ and $M$, respectively. Write $f^{-1}[-]\colon \P(Y) \to \P(X)$ for the inverse-image function.
	\begin{enumerate}[label=(\arabic*)]
		\item\label{item:a; lem:maps duality} The function $f$ is bases-continuous if and only if $f^{-1}$ descends to a lattice homomorphism $f^* \coloneqq f^{-1}\colon M \to L$. When one of these two equivalent conditions is satisfied, $f$ being surjective implies that $f^*$ is injective.
		\item\label{item:b; lem:maps duality} Assume  further $L$ and $M$ are Heyting bases. Assume the function $f$ is bases-continuous and bases-open. Then $f^{-1}$ descends to a  homomorphism of Heyting algebras $f^*\colon  M \to L$. Moreover, if $f$ is injective then $f^*$ is surjective, and if $f$ is a bijection then $f^*$ is an isomorphism.
	\end{enumerate}
\end{lemma}

\begin{proof}
	Since $f^*$ is a homomorphism of Boolean algebras, the first assertion in \ref{item:a; lem:maps duality} follows from the definitions. For the second assertion in \ref{item:a; lem:maps duality}, suppose $f$ is surjective. Pick $U, V \in M$ distinct, and suppose without loss of generality there is $p \in U \setminus V$. Since $f$ is surjective, there is $x \in X$ with $f(x)=p$. Then $x \in f^{-1}[U]$ but $x \not \in f^{-1}[V]$, so $f^{-1}=f^*$ is injective.
	
	As for \ref{item:b; lem:maps duality}, let us first assume that $f$ is bases-continuous and bases-open, and take $U,V \in M$ with the aim of showing that $f^*(U \ra V) = f^*(U) \ra f^*(V)$. For the left-to-right inclusion, using the fact that $M$ is a basis and that $f^*=f^{-1}[-]$ commutes with Boolean operations, write (letting $\comp S$ denote the complement of $S$): 
	\[
	U\ra V=\Int(\comp U \cup V)=\bigcup\{O\in M \mid O\sse \comp U \cup V\}
	\]
	and:
	\[
	f^{-1}[U] \ra f^{-1}[V]=\Int\left(\comp{ f^{-1}[U]} \cup f^{-1}[V]\right)=\Int \left(f^{-1}[\comp U \cup V]\right).
	\]
	
	Since $f^{-1}[-]$ preserves arbitrary unions too, we obtain $f^{-1}[U\ra V]=\bigcup f^{-1}[O]$ for $O \in M$ ranging over subsets of  $\comp U \cup V$. Now $O\sse \comp U \cup V$ entails $f^{-1}[O]\sse f^{-1}[\comp U\cup V]$. Since $f^{-1}[O]$ is open because $f$ is continuous,  by the definition of interior $f^{-1}[O]\sse \Int(f^{-1}[\comp U\cup V])$, which shows $f^{-1}[U\ra V]\sse f^{-1}[U]\ra f^{-1}[V]$.
		
	For the right-to-left inclusion we have the following chain of inclusions.
	\begin{align*}
		f[f^{-1}[U] \ra f^{-1}[V]]
			&= f\left[\Int\left(\comp{f^{-1}[U]} \cup f^{-1}[V]\right)\right] \\
			&\sse \Int\left(f\left[\comp{f^{-1}[U]} \cup f^{-1}[V]\right]\right) \tag{$f$ is open} \\
			&= \Int\left(f\left[f^{-1}[\comp{U} \cup V]\right]\right) \\
			&\sse \Int(\comp U \cup V) \\
			&= U \ra V
	\end{align*}
	Applying $f^{-1}$ to both sides, we get that $f^{-1}[U] \ra f^{-1}[V] \sse f^{-1}[U \ra V]$. Summing up, $f^*(U \ra V) = f^*(U) \ra f^*(V)$.  


	Next, assume $f$ is injective. Let $A\in L$, and let us show $A$ has a pre-image along $f^{*}=f^{-1}$. Certainly  $A\subseteq f^{-1}[f[A]]$. Let us prove the converse inclusion. If $f^{-1}[f[A]]$ is empty then the converse inclusion holds; otherwise, pick $x \in f^{-1}[f[A]]$. Then $f(x)\in f[A]$, so there is $a\in A$ with $f(x)=f(a)$. Since $f$ is injective, $x=a\in A$, and thus $f^{-1}[f[A]]\subseteq A$. Hence $A$ has the pre-image $f[A]$ along $f^{-1}$. Since, moreover, $f$ is bases-open, we have $f[A]\in M$, so $f^*$ is indeed surjective.
	
	Finally, if $f$ is a bijection then by \ref{item:a; lem:maps duality} and what we just proved $f^*$ is a bijective isomorphism of Heyting algebras, and hence an isomorphism.
\end{proof}

\begin{lemma}\label{lem:subspace_quotient}
	Let $X$ be a space, let $L\subseteq \Opens(X)$, let $Y\subseteq X$, and set $M\coloneqq\{O\cap Y\mid O\in L\}$.
	\begin{enumerate}
	\item If $L$ is a (lattice) basis for the topology of $X$ then $M$ is a (lattice) basis for the subspace topology of $Y$.
	\item If $Y$ is open and $L$ is a Heyting basis for the topology of $X$ then $M$ is a Heyting basis for the subspace topology of $Y$.
	\end{enumerate}
\end{lemma}

\begin{proof}
	This is a straightforward verification and shall be omitted.
\end{proof}
	
To deploy Lemmas \ref{lem:maps duality} and \ref{lem:subspace_quotient} in our geometric setting we will require the next fact.

\begin{lemma}\label{lem:convex open subpolyhedra basis}
	The (convex) open subpolyhedra of a (convex) polyhedron $P$ form a basis for the topology on $P$. Moreover, for any polyhedron $P$, $\Subo P$ is a Heyting basis of $P$.
\end{lemma}

\begin{proof}
	Assume $P\subseteq \R^n$ is any polyhedron.
	Take any $x \in P$ and let $U$ be an open neighbourhood of $x$ in $P$. Then there is some open ball $B$ in $\R^n$ about $x$ such that $x \in B \cap P \sse U$. An elementary argument in affine geometry produces  a simplex $\sig$ in $\R^n$ such that $x \in \Relint\sig \sse B$. Then (by the Triangulation Lemma~\ref{lem:triangulation lemma}) the set $\Cl(P \setminus \sig)$ is a compact subpolyhedron of $P$. Its complement $P \cap \Relint \sig$ is therefore an open subpolyhedron of $P$. Furthermore,
	\begin{equation*}
		x \in P \cap \Relint \sig \sse U, 
	\end{equation*}
	which shows   $\Subo P$ is a basis. If $P$ is additionally convex, then $P \cap \Relint \sig$ is also convex because $P$ and $\Relint \sig$ are, which shows that the convex open subpolyhedra of a convex polyhedron form a basis.

	The `moreover' statement follows from the fact that  the basis $\Subo P$ is a Heyting subalgebra of $\Opens(P)$ by Theorem \ref{thm:suboP Heyting algebra}.
\end{proof}

\begin{remark}\label{rem:convention_bases}
	From now on, in light of Lemma \ref{lem:convex open subpolyhedra basis}, we always tacitly assume a polyhedron $P$ is equipped with its Heyting basis $\Subo P$. Also, in light of Lemma \ref{lem:subspace_quotient}, if $Q$ is an open polyhedron in $P$ --- that is, a member of $\Subo P$ for some polyhedron $P$ --- we always tacitly assume that $Q$ is equipped with the Heyting basis $\Subo Q\coloneqq\{O\cap Q \mid O \in \Subo P\}$.
\end{remark}

Finally, in the next definition we isolate the specific instance of basis-con\-ti\-nuous map that is crucial to our context.

\begin{definition}\label{d:polyhedralmap} 
	Let $P$ be a polyhedron and $Y$  a space with a lattice basis $M$. 
	\begin{enumerate*}[label=(\roman*)]
		\item A function $f \colon P \to Y$ is a \emph{polyhedral map} if it is bases-continuous with respect to the  bases $\Subo P$ and $M$, respectively.
		\item Further, let $Q$ be an open subpolyhedron of $P$. A function $f \colon Q \to Y$ is again called a \emph{polyhedral map} if the pre-image of any open set in $M$ is in $\Subo Q$ (see Remark \ref{rem:convention_bases}).
		\item In the special case that the co-domain $Y$ of $f$ is a  poset $F$, we always tacitly assume $M$ is the Heyting basis $\Up F$ of all open sets in the Alexandrov topology on $F$.
		\item When we say  a polyhedral map as in the foregoing items is \emph{open} we always mean it is \emph{bases}-open with respect to the indicated bases.
	\end{enumerate*}
\end{definition}


\subsection{Jankov-Fine, for polyhedra}

\Cref{thm:Jankov-Fine up-reductions} shows that Jankov-Fine formulas encode forbidden configurations for frames. The same is true for polyhedra with respect to polyhedral maps, as we now show.


Let $\Sig$ be a simplicial complex and $F$ a poset. Given any function $f \colon \Sig \to F$, define the map $\wh f \colon \abs\Sig \to F$ by:
\begin{equation*}
	\wh f(x) \coloneqq f(\sig^x)
\end{equation*}

\begin{lemma}\label{lem:p-morphism to open polyhedral map}
	When $f \colon \Sig \to F$ is a p-morphism, $\wh f \colon \abs\Sig \to F$ is an open polyhedral map.
\end{lemma}

\begin{proof}
	For any $U \in \Up F$, we have that:
	\begin{equation*}
		\wh f^{-1}[U] = \bigcup \{\Relint\sig \mid \sig \in \Sig\text{ and } \sig \in f^{-1}[U]\}
	\end{equation*}
	Since $f$ is monotonic, $f^{-1}[U]$ is upwards-closed in $\Sig$ and therefore $\wh f^{-1}[U]$ is an open sub-polyhedron of $\abs\Sig$. Now take an open set $W \sse \abs\Sig$, with the aim of showing that $\wh f[W]$ is open. Define:
	\begin{equation*}
		\Sig\#W \coloneqq \{\sig \in \Sig \mid \Relint(\sig) \cap W \neq \es\}
	\end{equation*}
	Then:
	\begin{equation*}
		\wh f[W] = \{f(\sig^x) \mid x \in W\} = f[\Sig\#W]
	\end{equation*}
	If $\sig \in \Sig\#W$ and $\sig \preceq \tau$, then as $\sig \sse \tau = \Cl\Relint\tau$ and $W$ is open, we have $\tau \in \Sig\#W$; i.e. $\Sig\#W$ is upwards-closed. But now, $f$ is open and so $\wh f[W]$ is also upwards-closed.
\end{proof}

\begin{lemma}\label{lem:Jankov-Fine polyhedral maps}
	Let $P$ be a polyhedron and $F$ a finite rooted frame. Then $P\nvD \chi(F)$ if and only if there exists an open subpolyhedron $Q$ of $P$ and a surjective open polyhedral map $f \colon Q\to F$. Moreover, if $P$ is convex, then we can assume without  loss of generality that $Q$ is also convex. 
\end{lemma}

\begin{proof}
	Let $P\nvD \chi(F)$. By \cref{thm:Subo P locally-finite} there is a triangulation $\Sig$ of $P$ such that $\Po(\Sig) \nvD \chi(F)$, which by \cref{lem:Po Sig iso Up Sig} means that $\Sig \nvD \chi(F)$. Hence by \cref{thm:Jankov-Fine up-reductions} there is an up-reduction $h \colon \Sig \cra F$. Note that $h$ is open (with respect to the Alexandrov topologies) by the definition of p-morphism. Let $H$ be the (upwards-closed) domain of $h$. As $F$ is rooted, $H$ can be assumed without  loss of generality to be rooted --- it suffices to take a pre-image $y$ of the root of $F$ and let $H = \us y$. Applying \cref{lem:p-morphism to open polyhedral map} to the identity map $\id \colon \Sig \to \Sig$ we find an open polyhedral map $\wh\id \colon P \to \Sig$. Let $Q$ be the pre-image of $H$ via $\wh\id$. Then $h\circ \wh\id \colon Q \to F$ is a surjective open polyhedral map.
	
	Now assume that $P$ is convex. Let $x$ be any element in the pre-image of the root of $H$, and note that $Q$ is an open neighbourhood of $x$. Hence by \cref{lem:convex open subpolyhedra basis} there is an open convex subpolyhedron $W\subseteq P$ such that $x \in W\subseteq Q$. Since $\wh\id$ is open, $\wh\id[W]$ is an upwards-closed subset of $H$ containing its root, and therefore $H = \wh\id[W]$. We have thus found a convex open subpolyhedron $W$ such that $h\circ \wh\id[W] = F$, as desired. 
	
	For the converse direction, as $F\nvD \chi(F)$ we obtain from \cref{lem:maps duality} that $Q\nvD \chi(F)$. Then  \cref{lem:subspace_quotient} implies that $\Subo Q$ is a quotient of $\Subo P$ (via the map $O\in P\mapsto O\cap Q\in\Subo Q$), and therefore $P\nvD \chi(F)$. 
\end{proof}

	
\subsection{PL maps}

For any $X\sse\R^m$, $Y \sse \R^n$, a function $X \to Y$ is an \emph{affine map} if it lifts to a map $\R^m\to\R^n$ of the form $x \mapsto Mx + b$, where $M$ is a linear transformation and $b \in \R^n$. Now let $P$  and $Q$ be polyhedra in $\R^m$ and $\R^n$, respectively. A function $f\colon P \to Q$ is \emph{piecewise linear}, or a \emph{PL map} for short, if there are triangulations $\Sigma$ and $\Delta$ of $P$ and $Q$ respectively such that
\begin{enumerate}[label=(\arabic*)]
	\item the function $f$  agrees on each $\sigma\in\Sigma$ with an affine map, and
	\item for each $\sigma\in\Sigma$, $f[\sigma]\in \Delta$.
\end{enumerate}
PL maps as just defined are automatically continuous. 

\begin{remark}\label{rem:PL_graph} 
	There are several characterisations, or equivalent definitions, of PL map; we mention one that we shall use, referring to \cite{rourkesanderson1972} for proofs: a function $f\colon P \to Q$ is PL if and only if it is continuous, and its graph  $\{(x,f(x)) \in \R^{m+n}\mid x \in P\}$ is a polyhedron.
\end{remark}

\begin{remark}\label{rem:PL_is_poly}
	A PL map is a polyhedral map because of the standard fact that the inverse image of a polyhedron under a PL-map is a polyhedron, cf. \cite[Corollary~2.5, p.~13]{rourkesanderson1972}. The converse is not true --- the map $[0,1]\to[0,1]$ given by $x\mapsto x^2$ is a polyhedral map that is not PL.
\end{remark}

A \emph{PL homeomorphism} is a PL map that is a homeomorphism.

\begin{lemma}\label{lem:inversePL}
	The inverse of a PL homeomorphism is a PL homeomorphism.
\end{lemma}

\begin{proof}
	See \cite[p.~6]{rourkesanderson1972}.
\end{proof}

\begin{corollary}\label{cor:PL homeomorphism HA isomorphism}
	A PL homeomorphism $f\colon P\to Q$ between polyhedra and its inverse $g\colon Q\to P$ induce mutually inverse isomorphisms of Heyting algebras $f^*\coloneqq f^{-1}\colon \Subo{Q} \to \Subo{P}$ and $g^*\coloneqq g^{-1}\colon \Subo{P} \to \Subo{Q}$.
\end{corollary}

\begin{proof}
	This is an immediate consequence of \cref{lem:maps duality} together with Lemma \ref{lem:inversePL} and \cref{rem:PL_is_poly}.
\end{proof}

\begin{corollary}\label{cor:PL homeomorphic implies logics same}
	If $P$ and $Q$ are PL homeomorphic then $\Logic(P)=\Logic(Q)$.
\end{corollary}


\subsection{Geometric realisation}
\label{ssec:geometric realisation}

The notion of `geometric realisation' can now be made more precise. Given a polyhedron and a space $Y$ with a Heyting basis $M$, a  \emph{realisation} of $Y$ in a polyhedron $P$ is an open surjective polyhedral map $f\colon P \to Y$. By  \cref{lem:maps duality} the dual map $f^*\colon M\to \Subo P$ is an injective homomorphism of Heyting algebras, and this entails $\Logic(P) \sse \Logic(Y)\coloneqq \Logic(M)$,  which is the key ingredient in the completeness proofs. 

Let us emphasise that our usage of the term `geometric realisation'  is specific to our setting. The map $f\colon P\to Y$ `realises' the Heyting algebra $M$ as a subalgebra of $\Subo P$  by pulling back inverse images along $f^*\coloneqq f^{-1}$. This applies in particular to the special case in which  $Y$ is a finite poset $F$, and $M$ is $\Up F$.   We shall next show how this notion of realisation for finite posets  relates to the  standard one of geometric realisation of a simplicial complex.

Let us see how to produce a geometric realisation for an arbitrary finite poset $F$ of height $n$, following \cite{tarski-polyhedra}. For this, we make use of the following construction coming from combinatorial geometry. The \emph{nerve} of $F$, denoted $\N(F)$ is the poset of all non-empty chains in $F$ ordered by inclusion. The nerve comes equipped with a p-morphism $\max \colon \N(F) \to F$ which sends a chain to its maximum element. Note also that $\height(\N(F)) = \height(F)$.

Using the nerve, we then define the geometric realisation of $F$ via a simplicial complex. Enumerate $F = \{x_1, \ldots, x_m\}$, and let $e_1, \ldots, e_m$ be the standard basis vectors of $\R^m$. The \emph{simplicial complex induced by} $F$ is defined:
\begin{equation*}
	\nabla F \coloneqq \{\Conv\{e_{i_1}, \ldots, e_{i_k}\} \mid \{x_{i_1}, \ldots, x_{i_k}\} \in \N(F)\}
\end{equation*}
Noting that $\nabla F \cong \N(F)$ as posets, the p-morphism $\max \colon \N(F) \to F$ then induces an open surjective polyhedral map $\abs{\nabla F} \to F$. Furthermore, by definition:
\begin{equation*}
	\Dim \abs{\nabla F} = \height(\N(F)) = n
\end{equation*}
In other words, we have an $n$-dimensional geometric realisation of the height-$n$ poset $F$, which is the main component in the proof of \cref{thm:IPC logic of P and BDn logic of Pn}.