%!TEX root = ../convex-paper.tex

\section{Completeness}
\label{sec:completeness}

	The proof that $\PL_n$ is complete with respect to the class of convex polyhedra of dimension at most $n$ consists of two main parts. In the first part, we show that $\PL_n$ can be expressed as the logic of a set of reasonably regular finite frames --- called \emph{sawed trees}. For the second part, we show that any such sawed tree of height $n$ can be realised geometrically as an $n$-dimensional convex polyhedron --- in other words, given a sawed tree $F$, we construct an open polyhedral map from a convex polyhedron onto $F$. This map is constructed using a more elaborate version of the method used to provide a geometric realisation for an arbitrary finite poset in \cref{ssec:geometric realisation}.


\subsection{The meaning of \texorpdfstring{$\PL_n$}{PLn} on frames}

	First of all, it will be convenient to spell out what it means, structurally, for a frame to satisfy $\PL_n$. For this we introduce some additional terminology and notation.

	For any poset $F$ and $x \in F$, the \emph{strict upset} and \emph{strict downset} are defined, respectively, as follows.
	\begin{gather*}
		\Us x \coloneqq \{y \in F \mid y > x\} \\
		\Ds x \coloneqq \{y \in F \mid y < x\}
	\end{gather*}
	The \emph{depth} of $x$ is defined:
	\begin{equation*}
		\depth(x) \coloneqq \height(\us x)
	\end{equation*}
	A \emph{top element} of $F$ is $t \in F$ such that $\depth(t)=0$. The set of top elements in $F$ is denoted by $\Top(F)$.

	A \emph{path} in $F$ is a sequence $p=x_0\cdots x_k$ of elements of $F$ such that for each $i$ we have $x_i < x_{i+1}$ or $x_i > x_{i+1}$. Write $p \colon x_0 \rsa x_k$. The poset $F$ is \emph{path-connected} if between any two points there is a path.

	\begin{lemma}\label{lem:finite poset path-connected iff connected}
		When $F$ is finite, it is path-connected if and only if it is connected as a topological space.
	\end{lemma}

	\begin{proof}
		See \cite[Lemma~3.4]{Bezhanishviligabelaia2011}.
	\end{proof}

	A \emph{connected component} of $F$ is a subframe $U \sse F$ which is connected as a topological subspace and is such that there is no connected $V$ with $U \subset V$.

	\begin{lemma}\label{lem:properties of connectedness}
		\begin{enumerate}[label=(\arabic*)]
			\item The connected components partition $F$.
			\item Connected components are upwards- and downwards-closed.
		\end{enumerate}
	\end{lemma}

	\begin{proof}
		The first is a standard fact in topology, while the second follows straightforwardly from the fact that by \cref{lem:finite poset path-connected iff connected} the connected components are exactly the equivalence classes under the relation `there is a path from $x$ to $y$'.
	\end{proof}

	Finally, for any $x,y \in F$, say that $x$ is an \emph{immediate predecessor} of $y$ and that $y$ is an \emph{immediate successor} of $x$ if $x < y$ and there is no $z \in F$ such that $x < z < y$.

	We can now describe the structural meaning of $\PL_n$ on frames.

	\begin{lemma}\label{lem:meaning of PLn on posets}
		Let $F$ be a poset. Then $F \vD \PL_n$ if and only if the following are satisfied.
		\begin{enumerate}[label=(\roman*)]
			\item\label{item:height; lem:meaning of PLn on posets}
				$F$ has height at most $n$.
			\item\label{item:depth 1; lem:meaning of PLn on posets}
				Whenever $\depth(x) = 1$, we have $\abs{\Us x} \leq 2$.
			\item\label{item:depth gt1; lem:meaning of PLn on posets}
				Whenever $\depth(x) > 1$, the set $\Us x$ is connected.
		\end{enumerate}
	\end{lemma}

	\begin{proof}
		This follows from the definition of $\PL_n$, using the following facts for finite frames $F$.
		\begin{enumerate}[label=(\roman*)]
			\item $F \vD \BD_n$ if and only if $F$ has height at most $n$.
			\item There is an up-reduction $F \cra \FThreeFork$ if and only if there is $x \in F$ such that $\Us x$ has at least three components.
			\item There is an up-reduction $F \cra \FScott$ if and only if there is $x \in F$ such that $\Us x$ has at least two components, with at least one of which having height greater than $0$.\qedhere
		\end{enumerate}
	\end{proof}

	$\PL_n$-frames also satisfy the following specific connectedness property, which will come in handy in the arguments below.

	\begin{lemma}\label{lem:PLn frames specific connectedness property}
		Let $F$ be a finite rooted frame with $\height(F) > 1$, such that $F \vD \PL_n$. Take $s,t \in \Top(F)$. There is a path $p = a_0 \cdots a_m$ from $s$ to $t$ in $\Us{\bot}$ with the property that for each $i$:
		\begin{enumerate}[label=(\Roman*)]
			\item\label{item:a; lem:PLn frames specific connectedness property} 
				$\Us{a_i} = \es$ when $i$ is even, and
			\item\label{item:b; lem:PLn frames specific connectedness property} 
				$\Us{a_i} = \{a_{i-1},a_{i+1}\}$ when $i$ is odd.
		\end{enumerate}
	\end{lemma}

	\begin{proof}
		Since $\height(F) > 1$ we have that $\depth(\bot) > 1$. Hence by \cref{lem:meaning of PLn on posets}, there is a path $p = a_0 \cdots a_m$ from $s$ to $t$ in $\Us{\bot}$. We may assume that:
		\begin{enumerate}[label=(\Alph*)]
			\item\label{item:immediate; proof:PLn frames specific connectedness property} 
				$a_{i+1}$ is either an immediate successor or an immediate predecessor of $a_i$, for each $i$,
			\item\label{item:height-maximal; proof:PLn frames specific connectedness property} 
				$p$ is `height-maximal': if $i < j < k$ and $a_j < a_i, a_k$, then there is no path $a_i \rsa a_k$ in $\Us{a_j}$, and
			\item\label{item:no repeats; proof:PLn frames specific connectedness property} 
				$p$ has no repeats.
		\end{enumerate}
		Indeed, \ref{item:height-maximal; proof:PLn frames specific connectedness property} can be secured by iteratively replacing each offending $a_j$ with the path $a_i \rsa a_k$ in $\Us{a_j}$. Then \ref{item:no repeats; proof:PLn frames specific connectedness property} can be secured by removing all cycles, a process which preserves \ref{item:height-maximal; proof:PLn frames specific connectedness property}.

		We claim that such a $p$ also satisfies \ref{item:a; lem:PLn frames specific connectedness property} and \ref{item:b; lem:PLn frames specific connectedness property}, which we prove by induction. The base $i=0$ is immediate since $a_0=s$ is a top node. So assume that $i>0$. The first case is when $i$ is odd. By induction hypothesis $\Us{a_{i-1}} = \es$; in other words $a_{i-1}$ is a top node. Hence by \ref{item:immediate; proof:PLn frames specific connectedness property}, $a_i$ is an immediate predecessor of $a_{i-1}$. This means that $\{a_{i-1}\}$ is a connected component in $\Us{a_i}$, and hence by \cref{lem:meaning of PLn on posets} \ref{item:depth 1; lem:meaning of PLn on posets} and \ref{item:depth gt1; lem:meaning of PLn on posets}, we must have $\abs{\Us{a_i}} \leq 2$. Note further that by \ref{item:height-maximal; proof:PLn frames specific connectedness property}, $a_{i+1} \neq a_{i-1}$. Therefore, the task is to show that $a_{i+1} \in \Us{a_i}$. Let us suppose for a contradiction that this is not the case; i.e. $a_{i+1} < a_i$. Since $t$ is a top node, there must be $j \geq i+1$ with $a_j \leq a_{i+1}$ such that $a_{j+1} > a_j$ (in other words, the path can not keep going downwards after $a_{i+1}$). Clearly $\depth(a_j)>1$, hence by \cref{lem:meaning of PLn on posets} \ref{item:depth gt1; lem:meaning of PLn on posets} there must be a path $a_i \rsa a_{j+1}$ in $\Us{a_j}$, which contradicts property \ref{item:height-maximal; proof:PLn frames specific connectedness property}. \contradiction Thus $a_{i+1} \in \Us{a_i}$ as required. The second case when $i$ is even follows immediately from property \ref{item:immediate; proof:PLn frames specific connectedness property} and the induction hypothesis.
	\end{proof}


\subsection{Sawed trees}

	Let $T$ be a finite tree in which every top element has the same height. A linear ordering $\prec$ on $\Top(T)$ (or equivalently an enumeration $t_1, \ldots, t_k$ of $\Top(T)$) is a \emph{plane ordering} if for every $x \in T$ we have that $\us x \cap \Top(T)$ is an interval with respect to $\prec$. When $\height(T)>0$, the \emph{sawed tree} based on $(T,\prec)$ consists of $T$ plus new elements $s_1, \ldots, s_{k-1}$ with relations, for each $i$:
	\begin{equation*}
		t_i,t_{i+1} < s_i
	\end{equation*}
	See \cref{fig:sawed tree example} for an example of a sawed tree.

	% Figure environment removed

	The planarity condition on $\prec$ ensures that the Hasse diagram of the resulting sawed tree can be drawn in the plane with no overlapping lines. Formally, let $G$ be a poset and $d \colon G \to \R^2$ be an injection, such that $d = (d_1,d_2)$. Draw an edge $\mathit{xy}$ between $d(x)$ and $d(y)$ whenever $y$ is an immediate successor of $x$. Then $d$ is a \emph{plane drawing} of $G$ if the following conditions hold.
	\begin{enumerate}[label=(\alph*)]
		\item Whenever $x < y$ we have $d_2(x) < d_2(y)$.
		\item Two distinct edges $x_1y_1$ and $x_2y_2$ only ever intersect at their end-points.
	\end{enumerate}
	The notion of a planar poset has been studied somewhat in the literature (see \cite[\S6.8, p.~101]{brandstadtetal1999} for a short survey), but we will not  use any external results here.

	\begin{lemma}\label{lem:trees are planar}
		Let $\prec$ be a plane ordering on $T$. Then $T$ has plane drawing $d$ with the following properties. 
		\begin{enumerate}[label=(\roman*)]
			\item The top nodes in the drawing are ordered left-to-right as per $\prec$.
			\item $d_2(x) = \height(x)$ for every $x \in T$.
		\end{enumerate}
	\end{lemma}

	\begin{proof}
		C.f. \cite[p.~294]{stanley1997}. We proceed by induction on $n = \height(T)$. The base case $n=0$ is immediate, so assume that $n>0$. Enumerate the immediate successors of $\bot$ in $T$ as $\{x_1, \ldots, x_k\}$, according to $\prec$. That is, for each $i,j \leq k$ with $i<j$ ensure that:
		\begin{equation*}
			\forall t_i \in \us{x_i} \cap \Top(T) \colon \forall t_j \in \us{x_j} \cap \Top(T) \colon t_i \prec t_j
		\end{equation*}
		This is possible since $\us{x} \cap \Top(T)$ is an interval for each $x$. By induction hypothesis, for each $i \leq k$ there is a plane drawing $d^i$ of $\us{x_i}$ satisfying the conditions. We can then form a plane drawing $d$ of $T$ by shifting the drawings $d_1, \ldots, d_k$ up by one, lining them up side by side, then letting $d(\bot) \coloneqq (0,0)$. It is clear that $d$ then also satisfies the required conditions.
	\end{proof}

	\begin{corollary}\label{cor:sawed trees are planar}
		Every sawed tree $F$ admits a plane drawing $d$ with the property that $d_2(x) = \height(x)$ for every $x \in F$.
	\end{corollary}

	\begin{proof}
		Let $F$ be based on $(T,\prec)$, and let $s_1, \ldots, s_{k-1}$ be the top elements. By \cref{lem:trees are planar}, there is a plane drawing $d'$ of $T$ satisfying the property. Extend $d'$ to a drawing $d$ of $F$ by letting $d(s_i) \coloneqq (i,\height(F))$.
	\end{proof}

	The reason for considering sawed trees is that they provide a complete class of frames for $\PL$ which is relatively easy to work with.

	\begin{lemma}\label{lem:sawed trees satisfy PLn}
		Let $F$ be a sawed tree of height $n$. Then $F \vD \PL_n$.
	\end{lemma}

	\begin{proof}
		Let $F$ be based on $(T,\prec)$. Let us verify the conditions of \cref{lem:meaning of PLn on posets}. Conditions \ref{item:height; lem:meaning of PLn on posets} and \ref{item:depth 1; lem:meaning of PLn on posets} are immediate. As for \ref{item:depth gt1; lem:meaning of PLn on posets}, take $x \in F$ with $\depth(x)>1$. By construction, $x \in T$. Since $\prec$ is a plane ordering, we have that $\us x \cap \Top(T)$ is an interval with respect to $\prec$. Therefore, the top two layers of $\Us x$ are connected by the saw structure.
	\end{proof}

	\begin{lemma}\label{lem:frames of PL p-morphic images of sawed trees}
		Every rooted frame $F$ of $\PL$ of height $n$ is the p-morphic image of a sawed tree of height $n$, for every $n \geq 2$.
	\end{lemma}

	\begin{proof}
		We prove this by induction on $n$. For the base case $n=2$, note that $F$ consists of the root $\bot$ together with a number of nodes of depths $0$ and $1$. By gluing together paths obtained from \cref{lem:PLn frames specific connectedness property}, we can find a path $p = a_0 \cdots a_m$ satisfying \ref{item:a; lem:PLn frames specific connectedness property} and \ref{item:b; lem:PLn frames specific connectedness property} of that lemma which visits every top node. We would like to extend $p$ so that it visits \emph{every non-root} node. To do this, take $x \in F$ of depth $1$. By \cref{lem:meaning of PLn on posets} \ref{item:depth 1; lem:meaning of PLn on posets}, $\Us{x}=\{s,t\}$ with $s,t$ top nodes and possibly $s=t$. By inserting the sequence $xtxs$ in $p$ after an occurrence of $s$, we obtain a path satisfying \ref{item:a; lem:PLn frames specific connectedness property} and \ref{item:b; lem:PLn frames specific connectedness property}, which also visits $x$.

		Therefore, we may assume that our path $p$ visits every non-root node. Now, construct the sawed tree $F'$ by taking $\bot$ together with new elements:
		\begin{equation*}
			w_{-1},w_0, \ldots, w_m, w_{m-1}
		\end{equation*}
		with relations as in \cref{fig:F prime n2; proof:frames of PL p-morphic images of sawed trees}.

		% Figure environment removed

		Then define the surjective map $f \colon F' \to F$ by:
		\begin{gather*}
			\bot \mapsto \bot, \\
			w_{-1} \mapsto a_0, \\ 
			w_{m+1} \mapsto a_m, \\
			w_i \mapsto a_i \qquad \forall i \in \{0, \ldots, m\}
		\end{gather*}
		That $f$ is a p-morphism amounts to the fact that $p$ satisfies properties \ref{item:a; lem:PLn frames specific connectedness property} and \ref{item:b; lem:PLn frames specific connectedness property} of \cref{lem:PLn frames specific connectedness property}.

		For the induction step, assume that $n > 2$. Let $z_1, \ldots, z_k$ be the immediate successors of $\bot$ in $F$. By induction hypothesis, for each $i$ there is a sawed tree $G_i$ and a p-morphism $g_i \colon G_i \to \us{z_i}$. Let the sawed tree $G_i$ be based on $(S_i,\prec_i)$, and let $u_i, v_i \in \Top(S_i)$ be the least and greatest elements according to $\prec_i$, respectively. Since $\abs{\us{u_i}}, \abs{\us{v_i}} = 2$, we must have: 
		\begin{equation*}
			\abs{\us{g_i(u_i)}}, \abs{\us{g_i(v_i)}} \leq 2
		\end{equation*}
		Let $s_i \in \us{g_i(u_i)}$ and $t_i \in \us{g_i(v_i)}$ be the greatest elements. Now, by \cref{lem:PLn frames specific connectedness property}, for each $i \leq k-1$ there is a path $p_i \colon t_i \rsa s_{i+1}$ satisfying properties \ref{item:a; lem:PLn frames specific connectedness property} and \ref{item:b; lem:PLn frames specific connectedness property}; write $p_i = a_{i,0} \cdots a_{i,m_i}$.

		We will form our new sawed tree by laying the sawed trees $G_1, \ldots, G_k$ in a line and `gluing' them usings the paths $p_1, \ldots, p_{k-1}$ together with some `rope ladders' beneath. In detail, form $F'$ by taking the following ingredients and combining them as in \cref{fig:F prime construction inductive step; proof:frames of PL p-morphic images of sawed trees}.
		\begin{itemize}
			\item Each sawed tree $G_i$.
			\item For each $i \leq k$, new elements $w_{i,0} \cdots w_{i,k_i}$ corresponding to $a_{i,0} \cdots a_{i,k_i}$.
			\item A chain of length $n-2$ (a rope ladder) to hang below each $w_{i,j}$, with $j$ odd.
		\end{itemize}

		% Figure environment removed

		The result is evidently a sawed tree. Finally, construct the p-morphism $f \colon F' \to F$ as follows.
		\begin{enumerate}[label=(\alph*)]
			\item Inside each sawed tree $G_i$, let $f$ act as $g_i$.
			\item For each $w_{i,j}$, let $f(w_{i,j}) \coloneqq a_{i,j}$.
			\item For each $w_{i,j}$ with $j$ odd, send the rope ladder hanging below $w_{i,j}$ to $a_{i,j}$. \qedhere
		\end{enumerate}
	\end{proof}

	\begin{corollary}\label{cor:PLn logic of sawed trees}
		$\PL_n$ is the logic of sawed trees of height at most $n$, for every $n \geq 2$.
	\end{corollary}

	\begin{proof}
		This follows from \cref{lem:sawed trees satisfy PLn} and \cref{lem:PLn frames specific connectedness property}, and the fact that $\PL_n$, like any intermediate logic, is the logic of its rooted frames.
	\end{proof}


\subsection{Convex geometric realisation}

	In the second stage of the completeness proof, we provide a method of constructing a convex realisation of any sawed tree. To provide intuition for the construction, we first examine an instructive example of height $3$. Consider \cref{fig:convex geometric realisation height 3}. 

	% Figure environment removed
	
	The sawed tree $F$, depicted on the left, is realised in the pyramid $P = \mathit{OABEC}$, depicted on the right. The point $\mathit{D}$ lies midway between $\mathit{C}$ and $\mathit{E}$. An open surjective polyhedral map $f \colon P \to F$ is then defined as follows.
	\begin{itemize}
		\item The point $\mathit{O}$ is mapped to $\bot$.
		\item The remainder of the line $\mathit{OA}$ is mapped to $a$ while the remainder of $\mathit{OB}$ is mapped to $b$.
		\item The remainder of the triangle $\mathit{OAC}$ is mapped to $c$, the remainder of $\mathit{OAD}$ is mapped to $d$, and the remainder of $\mathit{OBE}$ is mapped to $e$.
		\item Finally, the remainder of the region $\mathit{OACD}$ is mapped to $s$ and the remainder of the region $\mathit{OABED}$ is mapped to $t$.
	\end{itemize}
	It is clear that such a map is polyhedral. Further the construction ensures that any open neighbourhood in $P$ is mapped to an upwards-closed subset of $F$. For instance, note that any open set intersecting $\mathit{OAD}$ must also intersect $\mathit{OACD}$ and $\mathit{OABED}$. Hence, $f \colon P \to F$ is an open polyhedral map as required.

	Notice that the two middle layers $(a,b)$ and $(c,d,e)$ of $F$ correspond to the edges $\mathit{AB}$ and $\mathit{CDE}$ of the base of the pyramid. Note further that the preimage of the tree part of $F$ --- i.e. the union of the triangles $\mathit{OAC}$, $\mathit{OAD}$ and $\mathit{OBE}$ --- has a natural triangulation. The definition of $f$ on this region then follows just as in the definition of the geometric realisation from \cref{ssec:geometric realisation}, with respect to this triangulation.

	With this intuition in mind we proceed with the proof in full generality. We make use of the following technical lemma on nerves and simplicial complexes.

	\begin{lemma}\label{lem:nerve geometric realisation criterion}
		Let $F$ be a poset and take any function $\alpha \colon F \to \R^n$. The collection:
		\begin{equation*}
			\{\Conv \alpha[X] \mid X \in \N(F)\}
		\end{equation*}
		forms a simplicial complex if and only if $\Conv\alpha[X]$ and $\Conv\alpha[Y]$ are disjoint for any disjoint $X,Y \in \N(F)$.
	\end{lemma}

	\begin{proof}
		This follows from \cite[Theorem~2]{demendez1999}, noting that the nerve $\N(F)$ is in particular an abstract simplicial complex, as defined there, with vertex set $\{\{x\} \mid x \in F\}$. 
	\end{proof}

	\begin{proof}[Proof of \cref{thm:PLn completeness for CPn}]
		The case $n=0$ is immediate. For $n=1$ note that by \cref{lem:meaning of PLn on posets}:
		\begin{equation*}
			\PL_1 = \Logic(\FPoint, \FOneFork, \FTwoFork) = \Logic(\FTwoFork)
		\end{equation*}
		Consider the convex polyhedron given by the interval $[0,1]$. We can define an open polyhedral map $f \colon [0,1] \to \FTwoFork$ by mapping $\half$ to the root, and the intervals $[0,\half)$ and $(\half,1]$ to each top node, respectively. Therefore:
		\begin{equation*}
			\Logic(\PolyCon_1) \sse \Logic([0,1]) \sse \PL_1 
		\end{equation*}
		Hence we may assume that $n \geq 2$. By \cref{cor:PLn logic of sawed trees} and \cref{lem:maps duality}, it suffices to show that every sawed tree of height $n$ can be realised geometrically in a convex polyhedron of dimension $n$. So, let $F$ be a height-$n$ sawed tree based on $(T,\prec)$. Using \cref{cor:sawed trees are planar}, let $d$ be a plane drawing of $F$ such that $d_2(x) = \height(x)$ for each $x \in F$.

		We first construct a simplicial complex corresponding to the tree part $T$ of $F$. Let $e_0, \ldots, e_n$ be the standard basis vectors of $\R^{n+1}$. Define a function $\alpha \colon T \to \R^{n+1}$ by letting, for $x \in T$:
		\begin{equation*}
			\alpha(x) \coloneqq e_{\height(x)} + d_1(x)e_{n}
		\end{equation*}
		It is helpful to consider the $n$th dimension (spanned by $e_n$) as running from left to right. Then nodes which are further to the right in the plane drawing $d$ map to points which are further to the right in $\R^{n+1}$. For each $X \in \N(T)$, let:
		\begin{equation*}
			\sig(X) \coloneqq \Conv\alpha[X]
		\end{equation*}
		Note that each element in $X$ is of a different height, so that $\alpha[X]$ is an affinely independent set of points; hence $\sig(X)$ is a simplex. Then set:
		\begin{equation*}
			\Sig \coloneqq \{\sig(X) \mid X \in \N(X)\}
		\end{equation*}
		Let us use \cref{lem:nerve geometric realisation criterion} to verify that $\Sig$ is a simplicial complex. Take disjoint $X,Y \in \N(F)$, and suppose for a contradiction that $\sig(X) \cap \sig(Y) \neq \es$. Let $X = \{x_1, \ldots, x_k\}$ and $Y = \{y_1, \ldots, y_l\}$, enumerated according to the order $<$ on $T$. Then, using barycentric coordinates inside $\sig(X)$ and $\sig(Y)$, there must be $r_1,\ldots, r_k \geq 0$ and $q_1, \ldots, q_l \geq 0$ with $\sum_{i=1}^k r_i = 1$ and $\sum_{j=1}^l q_j = 1$ such that:
		\begin{equation*}
			\sum_{i=1}^k r_i \alpha(x_i) = \sum_{j=1}^l q_j \alpha(y_j)
		\end{equation*}
		Using the definition of $\alpha$ and the fact that $e_0, \ldots, e_n$ are linearly independent, we see that:
		\begin{itemize}
			\item $r_i = 0$ if there is no $y_j$ with $\height(x_i) = \height(y_j)$,
			\item $q_j = 0$ if there is no $x_i$ with $\height(x_i) = \height(y_j)$,
			\item $r_i = q_j$ whenever $\height(x_i) = \height(y_j)$, and
			\item $\sum_{i=1}^k r_i d_1(x_i) = \sum_{j=1}^l q_j d_1(y_j)$.
		\end{itemize}
		Hence, we may assume that $k=l$ and that $\height(x_i) = \height(y_i)$ for each $i$. Now, for each $i$, since $X$ and $Y$ are disjoint, we must have $d(x_i) \neq d(y_i)$. But, since $d_2(x_i) = \height(x_i) = d_2(y_i)$, we must have either $d_1(x_i) < d_1(y_i)$ or $d_1(x_i) > d_1(y_i)$. Without loss of generality, assume that $d_1(x_1) < d_1(y_1)$. Then, since $T$ is a tree and no edges overlap in the plane drawing $d$, we must have $d_1(x_i) < d_1(y_i)$ for each $i$. Thus:
		\begin{equation*}
			\sum_{i=1}^k r_i d_1(x_i) = \sum_{i=1}^l q_i d_1(x_i) < \sum_{j=1}^l q_j d_1(y_j)
		\end{equation*}
		which is a contradiction. Therefore, $\Sig$ is a simplicial complex. As in \cref{ssec:geometric realisation}, the p-morphism $\max \colon \N(T) \to T$ gives rise to an open polyhedral map $f_T \colon \abs\Sig \to T$.

		Let us turn our attention now towards the top part of $F$. Enumerate $\Top(T)$ according to $\prec$ as $\{t_1, \ldots, t_k\}$, and let $s_1, \ldots, s_{k-1}$ be the top elements of $F$, as in the definition of a sawed tree. For each $i \leq k$, we have the $(n-1)$-simplex $\tau_i \coloneqq \sig(\ds{t_i})$. For $i \leq k-1$, let:
		\begin{equation*}
			\xi_i \coloneqq \Conv(\alpha[\Ds{s_i}]) = \Conv(\tau_{i-1} \cup \tau_i)
		\end{equation*}
		By considering the definition of $\alpha$, and noting that $\Ds{s_i}$ contains two elements which have the same height, we can see that $\Dim(\xi_i) = n$. Note also that:
		\begin{equation*}
			\xi_i \cap \xi_{i+1} = \tau_i
		\end{equation*}

		Define $P \coloneqq \bigcup_{i=1}^k \xi_i$, which will be our convex geometric realisation. By \cref{lem:dimension of union}, $P$ is an $n$-dimensional polyhedron. Furthermore, note that:
		\begin{equation*}
			P = \Conv (\tau_1 \cup \tau_k) = \Conv(\tau_1 \cup \cdots \cup \tau_k) = \Conv(P)
		\end{equation*}
		so that $P$ is a convex polyhedron and thus $P \in \PolyCon_n$. Extend the map $f_T$ to $f \colon P \to F$ by letting $x \in \xi_i \setminus (\tau_{i-1} \cup \tau_i)$ map to $s_i$. This map is clearly polyhedral. To see that it is open, take $x \in P$ and $U \sse P$ a small open neighbourhood of $x$. There are two cases. If $x \in \xi_i \setminus (\tau_{i-1} \cup \tau_i)$ for some $i$, then (as long as $U$ is small enough), $f[U] = \{s_i\}$ which is open. Otherwise, $x \in \tau_i$ for some $i$. Since $f_T$ is open, $V \coloneqq f[U \cap \abs\Sig]$ is an open subset of $T$. To see that $f[U]$ is open then, it suffices to show that whenever $s_i \in \uset^F V \cap \Top(F)$, we have $U \cap \xi_i \neq \es$. So take such an $s_i$. Since $V$ is open in $T$, we must have $t_{i-1} \in V$ or $t_i \in V$. Without loss of generality, assume the former. Hence we must have $U \cap \tau_{i-1} \neq \es$. But then since $U$ is open, it follows that also $U \cap \xi_i \neq \es$.

		Thus $f \colon P \to F$ is an open surjective polyhedral map from a convex $n$-dimensional polyhedron, as required.
	\end{proof}