%!TEX root = ../convex-paper.tex

\section{Introduction}
\label{sec:introduction}

Polyhedral semantics was introduced in \cite{tarski-polyhedra}. The starting point is that the collection of open subpolyhedra\footnote{For the terminology we adopt in polyhedral geometry the reader is referred to Section \ref{sec:preliminaries}.} of a compact polyhedron (of any dimension) forms a Heyting algebra. This then allows for the interpretation of intuitionistic and modal formulas in polyhedra. This semantics is closely related to the well-known topological semantics, as pioneered in \cite{Stone1938, tsao-chen1938, Tarski1939, mckinsey1941, mckinseytarski44, rasiowasikorski1963}. In topological semantics, one takes the Heyting algebra of open sets of a topological space as the basis for the interpretation of formulas. A celebrated result due to Tarski \cite{Tarski1939} shows that this provides a complete semantics for intuitionistic propositional logic ($\IPC$). The paper \cite{tarski-polyhedra} proved an analogous result for polyhedral semantics: the logic of the class of all polyhedra is \IPC. Moreover, this semantics can access the dimension of a polyhedron via the bounded-depth schema, something beyond the capabilities of topological semantics.

Precursors to the work in \cite{tarski-polyhedra} are \cite{AvBB03, vBBG03, vBB07, interpretating-topological-2010}. In \cite{planar-polygons} and \cite{gabelaiatacl} the authors developed the modal logic of the plane $\R^2$ considered as a non-compact polyhedron. The present authors extended the results of \cite{tarski-polyhedra} in \cite{polycompleteness}, where we introduced the notion of polyhedral completeness: a logic $\Lo$ is polyhedrally complete if it is the logic of some class of polyhedra. We developed the `Nerve Criterion', which provides a necessary and sufficient condition for the polyhedral completeness of a logic based on the combinatorial properties of its frames. This criterion was used to provide a wide class of polyhedrally complete logics axiomatised by the Jankov-Fine formulas of `starlike trees'. The first-named author's M.Sc thesis \cite{sam-thesis} investigated the polyhedral semantics defined in \cite{tarski-polyhedra} and is the basis for both \cite{polycompleteness} and the present paper. Recently, this semantics has been applied to the field of model checking. The authors of \cite{model-checking} developed a geometric spatial model checker using polyhedral semantics, introducing the notion of bisimularity for polyhedra along the way.

In the present paper, we investigate convex (compact) polyhedra, also known as polytopes, from a logical perspective. Our main result (\cref{thm:PL logic of CP}) is that the logic of the class of all convex polyhedra is $\PL$, a logic which is axiomatised by the Jankov-Fine formulas of two simple frames: $\FThreeFork$ and $\FScott$. Moreover, we obtain a more fine-grained result by restricting dimension. Letting $\PL_n$ be $\PL$ plus the logic of bounded depth $n$, we see that this is the logic of the class of all convex polyhedra of dimension at most $n$ (\cref{thm:PLn logic of CPn}).

To prove these results, the first step is a development of the logic-polyhedra connection on the level of morphisms. We introduce the notion of a `polyhedral map' from a polyhedron to a Kripke frame, and show that the open polyhedral maps are exactly those which give rise to contravariant  homomorphisms of the Heyting algebras associated with the polyhedron and the frame, respectively. With this, we can define the notion of the geometric realisation of a frame $F$ to be a polyhedron $P$ together with an open surjective polyhedral map $P \to F$. Moreover, we consider PL (for ``piecewise-linear'') homeomorphisms, which is the standard notion of isomorphism in polyhedral geometry. We show that PL homeomorphisms preserve the logics of polyhedra. 

Now, the proof that $\PL$ is the logic of convex polyhedra consists of two parts: soundness and completeness. For the soundness part, we first make use of the standard geometric fact that every $n$-dimensional convex polyhedron is PL homeomorphic to the $n$-simplex: the `simplest' polyhedron of dimension $n$. Given that PL homeomorphisms preserve logic, it suffices to show that $\PL_n$ is valid on the $n$-simplex, for which we give a geometric proof utilising classical results from polyhedral geometry.

The completeness direction splits into three stages. First, using a combinatorial argument, we show that every $\PL_n$ frame is the p-morphic image of a `sawed tree of height $n$'. This is a frame which has the form of a planar tree with a `saw structure' added on top. Once we have a sawed tree, we show how to realise it geometrically as an $n$-dimensional convex polyhedron. This realisation is built recursively on the frame structure, and makes key use of the fact that sawed trees are planar. Finally, we utilise a result due to Zakharyaschev \cite{zakharyaschev93} which entails that $\PL$ is the intersection of each $\PL_n$, and this completes the proof.

\Cref{sec:preliminaries} introduces the background on intermediate logics and polyhedral geometry, and \cref{sec:polyhedral semantics} introduces polyhedral semantics, following \cite{tarski-polyhedra}. While polyhedra can be used to provide a semantics for both intermediate and modal logics, we focus on the former side here.