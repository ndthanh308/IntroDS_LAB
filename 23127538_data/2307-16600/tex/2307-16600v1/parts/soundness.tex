%!TEX root = ../convex-paper.tex

\section{Soundness}
\label{sec:soundness}


The first half of the proof of \cref{thm:PLn logic of CPn} involves showing that:
\begin{equation*}
    \PL_n = \BD_n + \chi(\FThreeFork) + \chi(\FScott)
\end{equation*}
is valid on all of $\PolyCon_n$. The validity of the first summand follows from \cref{thm:IPC logic of P and BDn logic of Pn}, while for the other two we provide geometric arguments utilising classical results about polyhedra and dimension theory.

We first need the following lemma which relates open polyhedral maps to the boundary operation.

\begin{lemma}\label{lem:polyhedral map boundary}
    Let $f$ be a surjective open polyhedral map from $P$ onto a poset $F$. Whenever $x < y$ in $F$ we have  $f^{-1}[x] \sse \partial f^{-1}[y]$.
\end{lemma}

\begin{proof}
    Since $f$ is open and continuous we have:
    \begin{equation*}
        f^{-1}[x]
            \sse f^{-1} [\ds y]
            = f^{-1} [\Cl \{y\}]
            = \Cl f^{-1} [y]
            = \Cl^{\Aff} f^{-1} [y]
    \end{equation*}
    On the other hand $\Int^{\Aff} f^{-1} [y] \sse f^{-1} [y]$ and $f^{-1}[x]$ is disjoint from $f^{-1}[y]$. Hence:
    \begin{equation*}
        f^{-1}[x] \sse \Cl^{\Aff} f^{-1}[y] \setminus \Int^{\Aff} f^{-1}[y] = \partial f^{-1}[y]\qedhere
    \end{equation*}
\end{proof}

Now, the following is a pure dimension-theoretic result, which is essentially the geometric content of the statement that $\PolyCon \vD \chi(\FScott)$.

\begin{lemma}\label{lem:n-2 disconnection}
    Let $X$ be a convex set of dimension\footnote{Recall that whenever we state that a set has a dimension, we implicitly assume that its closure is a polyhedron.} $n$. There is no $Y \sse X$ of dimension $n-2$ or less such that $X \setminus Y$ is disconnected as a subspace of $X$.
\end{lemma}

\begin{proof}
    See \cite[Corollary~IV.1, p.~48]{hurewicz-wallmann}.
\end{proof}

Similarly, the following is essentially the geometric content of $\PolyCon \vD \chi(\FThreeFork)$.

\begin{lemma}\label{lem:n-1 disconnection}
    Let $X$ be a convex set of dimension $n$. There is no $Y \sse X$ of dimension $n-1$ or less such that $X \setminus Y$ can be partitioned into open sets $U$, $V$ and $W$ with $Y \sse \Cl U \cap \Cl V \cap \Cl W$.
\end{lemma}

To prove this we need the following classical result concerning triangulations of convex polyhedra.

\begin{lemma}\label{lem:n-1 simplices in convex polyhedron}
    Let $\Sig$ be a triangulation of a convex $n$-dimensional polyhedron. Then every $(n-1)$-simplex in $\Sig$ is the face of either one or two simplices of $\Sig$. 
\end{lemma}

\begin{proof}
    See \cite[Exercise~II.4, p.~27]{glaser1970geometricalvI}.
\end{proof}

\begin{proof}[Proof of \cref{lem:n-1 disconnection}]
    Assume for a contradiction that $Y$ disconnects $X$ in such a way that $X \setminus Y$ can be partitioned into open sets $U$, $V$ and $W$ with $Y \sse \Cl U \cap \Cl V \cap \Cl W$. By the Triangulation Lemma~\ref{lem:triangulation lemma} take a triangulation $\Sig$ of $\Cl X$ which simultaneously triangulates $\Cl Y$, $\Cl U$, $\Cl V$ and $\Cl W$.

    By \cref{lem:n-2 disconnection} the set $Y$ must have dimension exactly $n-1$. Hence there is an $(n-1)$-simplex $\sig \in \Sig$ such that $\sig \sse \Cl Y$. By \cref{lem:n-1 simplices in convex polyhedron} we have that $\sig$ is the face of either one or two simplices in $\Sig$. Let $\sig$ be the face of $\tau_1$ and $\tau_2$, where we allow that $\tau_1 = \tau_2$. By our choice of $\Sigma$, each $\Relint \tau_i$ is contained in exactly one of $U$, $V$ and $W$. Assume without loss of generality that $\Relint \tau_1 \sse U$. Similarly, assume that either $\Relint \tau_2 \sse U$ or $\Relint \tau_2 \sse V$.

    Now consider the open star of $\sig$:
    \begin{equation*}
        \ostar(\sig) = \Relint \sig \cup \Relint \tau_1 \cup \Relint \tau_2
    \end{equation*}
    By \cref{lem:open star open} this is open in $X$. Since $\ostar(\sig) \cap Y \neq \es$ and $Y \sse \Cl W$ we have that $\ostar(\sig) \cap W \neq \es$. But this is impossible since $\{Y, U, V, W\}$ forms a partition of $X$ and we have $\Relint \sig \sse Y$ and $\Relint \tau_1, \Relint \tau_2 \sse \Cl U \cup \Cl V$. \contradiction
\end{proof}

With all the pieces in place, we are now in a position to prove the desired soundness result.

\begin{proof}[Proof of \cref{thm:PLn soundness for CPn}]
    That $\PolyCon_n \vD \BD_n$ follows by \cref{thm:IPC logic of P and BDn logic of Pn} \ref{item:BDn; thm:IPC logic of P and BDn logic of Pn}.

    To show the validity of $\chi(\FThreeFork)$, suppose for a contradiction that there is a convex polyhedron $P$ such that $P \nvD \chi(\FThreeFork)$. Then by \cref{lem:Jankov-Fine polyhedral maps} there is a convex open subpolyhedron $Q$ of $P$ and a surjective open polyhedral map $f \colon Q \to \FThreeFork$. By \cref{lem:polyhedral map boundary} this partitions $Q$ into subsets $X, U, V, W$ such that $U$, $V$ and $W$ are open subpolyhedra of $P$ and:
    \begin{equation*}
        X \sse \partial U, \quad
        X \sse \partial V, \quad
        X \sse \partial W
    \end{equation*}
    By \cref{lem:dimension of boundary} we have that $\Dim X \leq \Dim Q - 1$ but $Q \setminus X = U \cup V \cup W$ is disconnected with at least three connected components. This contradicts \cref{lem:n-1 disconnection}. \contradiction

    As for the validity of $\chi(\FScott)$, suppose again for a contradiction that there is a convex polyhedron $P$ such that $P \nvD \chi(\FScott)$.  By \cref{lem:Jankov-Fine polyhedral maps} there is a convex open subpolyhedron $Q$ of $P$ and a surjective open polyhedral map $f \colon Q \to \FScott$.
    Then by \cref{lem:polyhedral map boundary} this partitions $Q$ into subsets $X, U_1, U_2, V_1$ such that $U_1$ and $V_1$ are open subpolyhedra of $P$ and:
    \begin{equation*}
        X \sse \partial U_1, \quad
        U_1 \sse \partial U_2, \quad
        X \sse \partial V_1
    \end{equation*}
    By \cref{lem:dimension of boundary} we have that $\Dim X \leq \Dim Q - 2$ but $Q \setminus X = (U_1 \cup U_2) \cup V_1$ is disconnected. This contradicts \cref{lem:n-2 disconnection}. \contradiction
\end{proof}
    
