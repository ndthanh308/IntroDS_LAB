We have presented a full description of PELICAN: a network that respects particle permutation and Lorentz symmetries important in particle physics. PELICAN is a general network which is performant at 4-vector regression and provides state-of-the-art performance in the tasks of top-tagging and quark-gluon tagging. The IRC-safe modification of the network is similarly leading among other such architectures. PELICAN also achieves state-of-the-art performance in the more difficult task of multi-class jet identification. Due to the equivariant architecture, all of this is made possible despite PELICAN's relatively compact model size.

To demonstrate PELICAN's regression capabilities, we chose the reconstruction of the $W$-boson's 4-momentum from a full top quark jet, and to our knowledge PELICAN is the first ML method applied to this problem. Even within these tasks there is room to improve PELICAN's performance by introducing additional scalar information such as particle charges, which would allow the network to account for the simulated collider's magnetic field. 

PELICAN's most unique features, however, have to do with its relatively low complexity (in terms of the number of parameters) and explainability. For example, it provides highly competitive top-tagging performance with only several hundred parameters. At the same time, in regression tasks the ``PELICAN weights'' represent a new kind of output that not only has a direct physical interpretation, but also allows one to perform unprecedented particle-level ML-based analysis of particle jets. PELICAN's architecture, its flexibility, and generalizability may also allow for future applications to charged-particle track reconstruction, pile-up identification, and full-event reconstruction. Being a general architecture, PELICAN is not limited to top quark decays or even jets. This network inherently provides powerful tools for investigating its own behavior due to the combination of Lorentz and permutation equivariance and shows promise as a tool which can be thoroughly investigated if deployed in real world scenarios.
