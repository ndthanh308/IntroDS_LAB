\subsection{Definitions}

Perturbative computations in QCD suffer from a divergence caused by two types of processes: soft emission and collinear splittings. As a consequence, meaningful observables in this theory need to be insensitive to such processes, and this requirement is known as IRC-safety. In this section we provide a precise definition, give a characterization of IRC-safe Lorentz-invariant observables (see details in \appref{appendix_irc}), and describe modifications to the PELICAN architecture that make it IR-safe or IRC-safe.

Infrared safety (IR-safety) guarantees insensitivity to soft emissions, i.e.~particles with relatively low energies and momenta. A family of continuous symmetric observables $f^{(N)}(p_1,\ldots,p_N)$ is said to define an IR-safe observable $f$ if 
\[\lim_{\epsilon\to 0}f^{(N+1)}(p_1,\ldots,p_N,\epsilon p)=f^{(N)}(p_1,\ldots,p_N)\]
for any $N$ and any $p_1,\ldots,p_N,p$, where $\epsilon$ controls how infinitesimally small the considered soft emission $p$ is.

Collinear safety (C-safety) is a restriction on observables in perturbative QCD that arises from the divergent contributions of collinear emissions of gluons. Positive gluon mass would prevent such divergences, which is why C-safety concerns only massless particles. We can define C-safety formally as follows: an observable $f(p_1,\ldots,p_N)$ is C-safe if, whenever two massless 4-momenta $p_1$ and $p_2$  become collinear (which happens for massless particles iff $p_1\cdot p_2=0$), the value of $f$ depends only on the total momentum $p_1+p_2$. Expressed even more explicitly, C-safety says that setting $p_1=\lambda p$ and $p_2=(1-\lambda)p$ with some 4-vector $p$ such that $p^2=0$ must lead to the same output regardless of the value of $\lambda\in (0,1)$, i.e.
\[C_{12}(p)f=\partial_\lambda f(\lambda p,(1-\lambda)p, p_3,\ldots,p_N)=0,\quad \lambda\in (0,1).\label{25773}\]

In \appref{appendix_irc} we characterize a certain class of IRC-safe Lorentz-invariant observables in terms of polynomial bases, but the following summary will suffice for the purpose of designing an IRC-safe version of PELICAN. First, a Lorentz-invariant observable (assumed to be consistently defined for any finite number $N$ of 4-vector inputs) is IR-safe if and only if it has no explicit dependence on the multiplicity $N$. More precisely, adding the zero 4-vector to the list of inputs should leave the output value invariant. Second, an IRC-safe Lorentz-invariant observable is one that is IR-safe and moreover depends on any of its massless inputs in such a way that when any number of them become collinear, the observable ends up depending only on the sum of their energies. E.g.~if $p_1,p_2,p_3$ are fixed to be massless, then $f(p_1,p_2,p_3,p_4,\ldots)$ must depend only on $p_1+p_2+p_3$ when $p_1\parallel p_2\parallel p_3$. Note that such an observable is still completely unrestricted away from the massless manifold.

It is instructive to reflect on the interplay between IR-safety and C-safety. C-safety allows us to exchange momentum between collinear particles without affecting the values of C-safe observables, whereas IR-safety allows us to omit vanishing 4-momenta from the list of inputs. C-safety in itself doesn't change the number of input 4-vectors -- it only requires $f$ to be invariant under a certain transformation that mixes collinear massless inputs. Meanwhile IR-safety effectively requires $f$ to encode an infinite family of observables $f^{(N)}(p_1,\ldots,p_N)$ for any possible number of inputs with the compatibility condition $f^{(N)}(p_1,\ldots,p_{N-1},0)=f^{(N-1)}(p_1,\ldots,p_{N-1})$.

The original PELICAN architecture as introduced above is neither IR- nor C-safe. Below we modify the architecture to make it exactly IRC-safe and evaluate the implications.

%------------------------
% IR-SAFE PELICAN
%------------------------
% \subsection{IR-safe PELICAN} 

% As shown above, IR-safety in Lorentz-invariant networks essentially requires the outputs to be independent of the multiplicity $N$. There are four ways in which the multiplicity shows up in PELICAN:
% %===========================================
% \begin{enumerate}
%     \item Scaling with $N^\alpha/\bar{N}^\alpha$ in the equivariant block. This must be disabled for IR-safety, or alternatively $N$ can be replaced with an IR-safe multiplicity such as the number of constituents whose jet-frame energy exceeds some fixed threshold: $p_i\cdot J>E_{\mathrm{softcut}}$, where $J=\sum_{i=1}^N p_i$.\footnote{We thank Andrew Larkoski for this suggestion.}
%     \item Non-zero bias values in linear layers. Since the network is permutation-equivariant, the bias values are shared across jet constituents, which means that upon aggregation in the following equivariant layer they contribute multiples of $N$. All biases in all linear layers must be disabled for IR-safety. 
%     \item The input embedding must map zero to zero, but our original choice already satisfies this. In addition, the activation function must also have a fixed point at zero. Our default choice, \texttt{LeakyReLU}, also satisfies this.
%     \item Following an application of a PELICAN equivariant block, rows and columns corresponding to soft constituents will  contain a combination of sums over all constituents. Even in the absence of biasing constants, this effectively increases the multiplicity with which these values enter in the later aggregations. This can be resolved by making sure that rows and columns that are soft at the input remain soft throughout the whole network. Therefore we introduce \textit{soft masking}, whereby the last 12 equivariant aggregators (those don't preserve the softness of rows/columns) are followed by a multiplication by the vector of values $J\cdot p_i$, scaled and clipped to be within $[-1,1]$. In $\mathrm{Eq}_{2\to 2}$ this multiplication is applied both row-wise and column-wise, and in $\mathrm{Eq}_{2\to 1}$ it's component-wise. 
% \end{enumerate}
% %===========================================
% With these modifications, PELICAN becomes IR-safe. As we will see, this restriction leads to a modest reduction in the performance of PELICAN's predictions in our tasks of interest.

\subsection{IRC-safe PELICAN}

% Adding C-safety to the architecture is much simpler. As stated above, the necessary requirement is that the output depend on massless inputs only through their sum. In PELICAN this can be achieved by inserting a linear permutation-equivariant layer with a mass-based soft mask immediately at the input (any nonlinear embedding has to be done later). Consider a case where $p_1,p_2$ are massless and the dot product matrix $\{d_{ij}\}$ is fed into such an equivariant layer. Most of the aggregators will compute sums over rows or columns, thus immediately producing C-safe quantities. However, several of the aggregators, including the identity, will preserve individual information about each $p_i$, therefore their output rows and columns corresponding to $p_1$ and $p_2$ need to be thrown out. This can be done by a soft mask that turns to zero as the mass of any input goes to zero. This mask is defined in the same way as the IR mask except using $m_i^2$ instead of $J\cdot p_i$. It needs to be applied only to the first 2 order zero and the first 7 order one aggregators. 

% Coincidentally, this soft mask can also be used in place of an IR mask, which means that we only need the C-safe soft mask to make a fully IRC-safe PELICAN architecture. Altogether it gets applied to all equivariant aggregators except the third one (which extracts the diagonal and is thus IRC-safe by definition).

To enforce IRC-safety in PELICAN, we follow a strategy almost identical to that of ref.~\cite{Spannowsky_IRC_safe}, based on the original idea from ref.~\cite{EFN}, with the correction that our equivariant layers are a generalization of message passing and that we need to take extra care to preserve Lorentz symmetry. The basic idea is that if $F_i$ are a permutation-equivariant set of IRC-safe observables of our inputs $\{p_i\}$, then so is any observable of the form
\[f\left(\sum_{i=1}^N z_i F_i\right),\label{irc-recursion}\]
where $f$ is any scalar function and $z_i$ are appropriately picked energy-dependent weights, commonly chosen to be either fractional energy $E_i/\sum_j E_j$ or fractional transverse energy $p_i^T/\sum_j p_j^T$. IR-safety is guaranteed due to the fact that any soft constituent has $z\to 0$ and the expression above is invariant under the insertion of such terms. C-safety is slightly more involved: if, say, $p_1$ and $p_2$ are massless and collinear, then due to the C-safety and permutation-equivariance of $F_i$, we have $F_1=F_2$, and all $F_j$ depend on these two particles only through $z_1+z_2$ and their common spatial direction. \Equref{irc-recursion} then guarantees that the new observable also depends on the magnitudes of these two vectors only through the sum $z_1+z_2$. All of these arguments obviously also apply to higher numbers of collinear particles. It is also important for us to note that this property applies to permutation-equivariant observables as well.

To apply this idea to PELICAN, we first need a Lorentz-invariant analog of the energy weights $z_i$ and the corresponding normalized vectors $\hat{p}_i$. Since the only permutation-invariant inertial frame that can be defined based on the list of the constituents $p_i$ is the frame of the jet $J=\sum_i p_i$, it is natural to define the jet-frame energies
\[\mathcal{E}_i=\frac{p_i\cdot  J}{m_J}=\frac{\sum_{j=1}^N d_{ij}}{\sqrt{\sum_{j,k=1}^N d_{jk}}},\label{energy-def}\]
where $m_J$ is the invariant jet mass. The energy weights are then simply the fractional jet-frame energies, and the de-dimensionalized 4-momenta are rescaled by these energies:
\[z_i = \frac{\mathcal{E}_i}{\sum_{j=1}^N \mathcal{E}_j},\quad\quad \hat{p}_i=\frac{p_i}{\mathcal{E}_i},\quad\quad \hat{d}_{ij}=\hat{p}_i \cdot \hat{p}_j=\frac{d_{ij}}{\mathcal{E}_i\mathcal{E}_j}.\label{angle-def}\]

Note that $\hat{d}_{ij}$\footnote{If the inputs are massless, $\hat{d}_{ij}=1-\cos\Theta_{ij}$, where $\Theta_{ij}$ are the pairwise spatial angles in the jet frame.} are Lorentz-invariant and constitute the natural input to the IRC-safe version of PELICAN. We are finally ready to modify the architecture to make it IRC-safe. Combining the idea from \equref{irc-recursion} with the definition of equivariant aggregators in \equref{aggregators}, we come to the following prescription for the $\mathrm{Eq}_{2\to 2}$ blocks:
\begin{itemize}
    \item The 5 order zero aggregators don't need to be modified since the application of any nonlinear function to an IRC-safe observable preserves its IRC-safety.
    \item The 8 order one aggregators involve summation over one equivariant input index, $\sum_i \bullet_i$, and all of these need to be modified by including the Lorentz-invariant energy weights: $\sum_i z_i \bullet_i$.
    \item The 2 order two aggregators are similarly adjusted with two energy weights: $\sum_{i,j} z_i z_j \bullet_{i,j}$.
\end{itemize}
Note that the aggregation function \textit{has to be} based on summation, since max/min pooling or any other nonlinear aggregation will immediately fail C-safety. The way the outputs of these aggregations are embedded into the output $N\times N$ array is unchanged since it doesn't affect IRC-safety. Since the inputs $\hat{d}_{ij}$ are IRC-safe and since \equref{irc-recursion} guarantees recursive IRC-safety, \textit{every individual component} of the output array of every IRC-safe equivariant block is IRC-safe.

The same kind of prescription can also be used to make the $\mathrm{Eq}_{2\to 1}$ (for 4-vector regression), $\mathrm{Eq}_{1\to 2}$ (for scalar inputs), and $\mathrm{Eq}_{2\to 0}$ (for classification) layers IRC-safe. As a consequence, the PELICAN weights $c_i$ become IRC-safe. The regression case deserves special attention, since strictly speaking the PELICAN weights $c_i$ themselves don't have to be IRC-safe for the combination $p_{\mathrm{reco}}=\sum_i c_i p_i$ to be IRC-safe. In particular, whereas the IR-safety of $p_{\mathrm{reco}}$ necessarily makes it possible to find IR-safe $c_i$, there is no such criterion for C-safety.

However, under the interpretation of $c_i$'s as ``soft clustering'' coefficients (as discussed in \secref{Weights}) it does make sense to require IRC-safety from them. In that case the values of weights corresponding to collinear inputs necessarily match due to permutation symmetry. Indeed, IRC-safety implies that if, say, $p_1$ and $p_2$ are collinear, then both $c_1$ and $c_2$ are functions of only their spatial direction and of $E_1+E_2$. Since the direction and the total energy are permutation-invariant but $c_1$ and $c_2$ are permutation-\textit{equivariant}, i.e.\ $c_1(p_1,p_2,\ldots)=c_2(p_2,p_1,\ldots)$, we conclude that $c_1=c_2$. This in turn also guarantees the IRC-safety of the reconstructed 4-vector. Nonetheless, whether IRC-safe PELICAN can approximate \textit{any} IRC-safe vector-valued observable, and whether there is a more general way of generating such observables, is an open problem deserving future investigation.

Finally, since the multiplicity $N$ is not IR-safe, the aggregation function cannot be defined using means. However, there exist IRC-safe analogs of $N$ such as the Soft Drop multiplicity defined in ref.~\cite{Larkoski_SoftDrop}.\footnote{The Soft Drop multiplicity actually measures the depth of the Cambridge-Aachen branching tree along the hard core of the jet. Its definition also involves several new hyperparameters denoted $z_{\mathrm{cut}}$, $\theta_{\mathrm{cut}}$ and $\beta$, see ref.~\cite{Larkoski_SoftDrop}.} Nevertheless, even they can't be directly used in PELICAN due to their non-Lorentz-invariance. For PELICAN tests, we have modified the Soft Drop algorithm by executing it \textit{in the jet frame}, making it manifestly Lorentz-invariant, and defined the Lorentz-invariant Soft Drop multiplicity $n_{\mathrm{SD}}$ that way. This is also equivalent to replacing energies and angles in the original definitions of the Cambridge/Aachen \cite{CA-algorithm-1,CA-algorithm-2} and Soft Drop algorithms with their Lorentz-invariant analogs defined in \equref{energy-def} and \equref{angle-def}. With this, it is safe to use aggregators of the form $n_{\mathrm{SD}}^\alpha\sum_{i=1}^N \bullet_i$.

\subsection{Testing IRC-safe PELICAN models} 
%\paragraph{Testing IR/C-safe PELICAN models} 

First we quantify the deviation in PELICAN's outputs that occurs under soft and collinear splittings and observe how training affects them. We define an IR-splitting as adding a zero 4-vector to the list of input constituents. Then PELICAN's output on IR-split data is directly compared to the original output. Defining a C-splitting is more difficult since realistic events never contain any exactly collinear constituents, and we want to avoid changing the number of particles so as to make this test independent of IR-safety. Therefore we prepare the data by inserting two copies of the vector $(1,0,0,1)$ to each event. Then the C-splitting will amount to replacing these two vectors with $(1.5,0,0,1.5)$ and $(0.5,0,0,0.5)$ (recall that C-safety does not apply when $\lambda=1$ in \equref{25773} because the zero vector is ``collinear'' with anything). The outputs on the same event prepared in these two ways can be directly compared.

To compare two outputs $p_{\mathrm{reco}}, p_{\mathrm{reco}}'$ we compute the relative deviation $|(p_{\mathrm{reco}}'-p_{\mathrm{reco}})/p_{\mathrm{reco}}|$, where the division is component-wise. To estimate the effect of an IR- or C-splitting on PELICAN's predictions, we take the median value of this deviation over a batch of events. The same can also be done with PELICAN weights as the outputs. The splittings are applied to 100-event batches of events from one of our datasets and the relative deviations are averaged over 300 batches. We test 5 randomly-initialized models and 5 models trained on the full variable $W$ mass dataset from \secref{Wmass}.

We find that a randomly-initialized PELICAN regression model's output 4-vector deviates by 0.5\%-14\% (measured by the deviation of each of the 4 Cartesian components) under an IR-split, and the PELICAN weights deviate by up to 9\%. After training on one of our datasets the range of these deviations doesn't appear to change. Under the C-safety test defined above, the 4-vector prediction of a randomly initialized model deviates by an absolute amount of up to $15\,\mathrm{GeV}$, and the median absolute deviation of the PELICAN weights can be as high as $0.03$ (depending on the random seed). With IRC-safety enabled these go down to $10^{-4}\,\mathrm{GeV}$ and $10^{-7}$, respectively. The main reason these deviations aren't even smaller under IRC-safety is the accumulation of numerical errors after repeated aggregation with the energy weights $z_i$. The values of these weights on realistic data are highly concentrated near zero, and overall they can span up to 8 orders of magnitude. After several layers of aggregation with weighting by $z_i z_j$ the spread of the values goes way beyond the size of the standard floating point types, which limits the precision of IRC-safety. However, for practical purposes the current precision of PELICAN is likely to be sufficient, seeing as it is still dominated by the uncertainties originating from the random initialization.


The resolutions $\sigma_{p_T}, \sigma_m$, and $\sigma_{\Delta R}$ of the IRC-safe \Delphes models (using  aggregations weighted by Soft Drop multiplicities) trained on the variable $W$ mass dataset targeting $p^W_{\mathrm{true}}$ are, respectively, 6.8\%, 8.5\%, and 3.1 centirad, which is only about 10\%, 3\%, and 8\% worse (higher) than the resolutions of the original non-IRC-safe models as reported in \tabref{btW6m_W_table}.
Here is how these deviations in the values of the outputs reflect on the overall quality of the predictions. Under an IR-splitting (adding a zero vector to every event) the resolutions of trained non-IRC-safe models get worse by between 0.3\% and 2\% depending on the initialization. Under an ``IRC-splitting'' (splitting the first beam $(1,0,0,1)$ into two constituents $(0.5,0,0,0.5)$)  the resolutions get worse by  between 10\% and 25\%. Therefore non-IRC-safe PELICAN models can be quite sensitive to IRC splittings, especially collinear ones.

Aside from regression, we also trained IRC-safe PELICAN classifiers for top-tagging and quark-gluon tagging. These results are included alongside the non-safe models in \tabref{tab1} and \tabref{tab_QG}. In both cases PELICAN provides state-of-the-art performance among IRC-safe models. In quark-gluon tagging, IRC-safe classifiers perform almost as well as the non-IRC-safe analogs, which confirms the conclusions of the recent study in ref.~\cite{Larkoski23}. Notably, the IRC-safe PELICAN top-tagger used the Lorentz-invariant Soft Drop multiplicity in place of $N$ in its aggregators, whereas the quark-gluon tagger only used simple summation aggregators (i.e.~$N$ was replaced with a fixed constant). This makes the comparison in \tabref{tab_QG} fair, but at the same time we expect even better performance from a model that uses Soft Drop multiplicity. 