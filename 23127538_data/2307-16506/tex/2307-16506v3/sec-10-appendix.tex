\newpage
\section{Additional results and plots}\label{appendix_plots}

\subsection{$W$-boson 4-momentum reconstruction}

Below is the list of additional figures for the models trained on the $W$-boson 4-momentum regression dataset from \secref{Wreco}:
\begin{itemize}
    \item \Figref{btW6WW_res}: histograms corresponding to \tabref{btW6_W_table} (models trained targeting $p^W_{\mathrm{true}}$);
    \item \Figref{btW6DW_res}: histograms corresponding to \tabref{btW6_D_table} (models trained targeting $p^W_{\mathrm{cont}}$);
    \item \Figref{btW6DD_res}: histograms for models trained targeting $p^W_{\mathrm{cont}}$.
    \item \Figref{weights_event_6D} distributions of PELICAN weights for models trained targeting $p^W_{\text{cont}}$;
    \item \Figref{m_corr_WW}: target vs.~reconstructed mass correlation histograms for models trained targeting $p^W_{\mathrm{true}}$;
    \item \Figref{m_corr_DW}: true $W$-boson vs.~reconstructed mass correlation histograms for models trained targeting $p^W_{\mathrm{cont}}$;
    \item \Figref{m_corr_DD}: target vs.~reconstructed mass correlation histograms for models trained targeting $p^W_{\mathrm{cont}}$.
\end{itemize}


% Figure environment removed

% Figure environment removed

% Figure environment removed

% Figure environment removed

% Figure environment removed

% Figure environment removed

% Figure environment removed

\subsection{$W$-boson mass measurement}

Below is the list of additional tables and figures for the models trained on the variable mass dataset from \secref{Wmass}:
\begin{itemize}
    \item \Tabref{btW6m_W_table}: resolutions for models trained targeting $p^W_{\mathrm{true}}$;
    \item \Figref{btW6mWW_res} histograms corresponding to \tabref{btW6m_W_table} (models trained targeting $p^W_{\mathrm{true}}$);
    \item \Figref{btW6mDW_res}: histograms corresponding to \tabref{btW6m_D_table} (models trained targeting $p^W_{\mathrm{cont}}$ and compared to $p^W_{\mathrm{true}}$);
    \item \Figref{btW6mDD_res}: histograms for models trained targeting $p^W_{\mathrm{cont}}$ and compared to $p^W_{\mathrm{cont}}$;
    \item \Figref{m_corr_mDW}: target vs.~reconstructed mass correlation histograms for models trained targeting $p^W_{\mathrm{cont}}$.
\end{itemize}


\begin{table}[h]
    \centering
    \begin{small}
        \begin{tabular}{ccS[table-format=3.2]<{\%}S[table-format=3.2]<{\%}S[table-format=3.3]}
        \toprule
        & Method &  \multicolumn{1}{c}{$\sigma_{p^T}$ (\%)} & \multicolumn{1}{c}{$\sigma_{m}$ (\%)} & \multicolumn{1}{c}{$\sigma_{\Delta R}$ (centirad)}\\
        \midrule
        \multirow{3}{*}{\rotatebox[origin=c]{90}{\parbox{1.3cm}{\centering Without\\ \Delphes}}} 
        & JH               & 7.98    & 4.75     & 22.180   \\
        & PELICAN$\mid$JH  & 0.26    & 0.58     & 0.111   \\
        & PELICAN$\mid$FC  & 0.31    & 0.76     & 0.142   \\
        & PELICAN          & 0.88    & 1.67    & 0.548   \\
        \midrule
        \multirow{3}{*}{\rotatebox[origin=c]{90}{\parbox{1.3cm}{\centering With\\ \Delphes}}} 
        & JH               & 16.0    & 12.0   & 25.4      \\
        & PELICAN$\mid$JH  & 4.1     & 6.3    & 3.2      \\
        & PELICAN$\mid$FC  & 4.7     & 7.3    & 3.5      \\
        & PELICAN          & 6.2     & 8.2    & 5.6      \\
        \bottomrule
        \end{tabular}
    \end{small}
    \caption{PELICAN resolutions for models trained to reconstruct $p^W_{\mathrm{true}}$ with variable $W$ mass.\label{btW6m_W_table}}
    \vspace{-0.5\intextsep}
\end{table}

% Figure environment removed

% Figure environment removed

% Figure environment removed

% Figure environment removed

\subsection{Additional event displays}

Below is the list of additional event displays:
\begin{itemize}
    \item \Figref{event_display_2} an event display showing how PELICAN processes a $qq$ event that JH fails to tag;
    \item \Figref{event_display_4}: an event display showing how PELICAN processes a $bqq$ event that JH fails to tag.
\end{itemize}

% Figure environment removed

% Figure environment removed


\newpage
\section{IRC-safety and Lorentz symmetry}\label{appendix_irc}

Let us now try to characterize the IRC-safe Lorentz invariants of a set of jet constituents. When working with Lorentz-invariant observables, it is convenient to rewrite everything in terms of Lorentz invariant coordinates -- the dot products $d_{ij}=p_i\cdot p_j$ (note that these coordinates are in general not independent since any $5\times 5$ minor of the matrix $\{d_{ij}\}$ must vanish due to the fact that any five $4$-vectors are linearly dependent, but they will suffice for our purposes). For two massless particles, collinearity is equivalent to the vanishing of the dot product. Conversely, the dot product between two time- or light-like vectors is zero only if both are light-like and collinear. Thus the C-safety condition amounts to the equality of the gradients of $f$ with respect to any two of the rows (or columns) of $\{d_{ij}\}$ when evaluated on the corresponding coordinate hyperplane:

\begin{equation}
    \restr{\frac{\partial f}{\partial d_{ki}}}{d_{ij=0}}=\restr{\frac{\partial f}{\partial d_{kj}}}{d_{ij=0}}.\label{2525}
\end{equation}

In general this condition is difficult to solve, however it turns out to be extremely powerful in one special case. If we assume that $f$ is \textit{analytic} in the dot products $d_{ij}$ (i.e.~it can be expanded into a convergent multivariate Taylor series in some vicinity of the origin), then by the Identity theorem from complex analysis \equref{2525} must hold in an entire neighborhood of the origin, even when $d_{ij}\neq 0$. However, if this formula holds for all values of $k$ in an entire domain, then $f$ is necessarily only a function of $p_i+p_j$. Moreover, if all inputs are assumed to be massless, then this applies to all pairs of indices $i,j$, and $f$ becomes a function of only the jet mass $m_J^2$. Thus we arrive at the following rather disappointing result.

\begin{thm}\label{thmB1}
If an IRC-safe Lorentz-invariant observable with a mixture of massless and massive inputs is \textit{real analytic} in the pairwise dot products $d_{ij}$ near the origin, then it can depend on the massless inputs only through their sum. If all inputs are massless, this observable can be reduced to an analytic function of a single scalar -- the jet mass $m_J^2=\sum_{i,j}d_{ij}$.
\end{thm}

In this way, IRC-safety is a significantly more powerful restriction than one could na\"ively expect based on the definitions of IR and C-safety alone. As an example, consider the quantity $\prod_{i<j}d_{ij}=d_{12}d_{23}d_{31}$ on $N=3$ massless inputs. It is in fact C-safe because it simply vanishes whenever any two inputs are massless and collinear. However it is not IR-safe because sending $p_3\to 0$ reduces the expression to zero instead of the expected $d_{12}$. Alternatively, as we will see in the following section, the natural IR-safe extension of $d_{12}d_{23}d_{31}$ to arbitrary $N$ will not be C-safe.

Clearly, by restricting ourselves to analytic functions, i.e.~functions that can be approximated by polynomials in the variables $d_{ij}$, we have arrived at an extremely narrow set of IRC-safe observables which will be of little practical use. This is where our construction of IRC-safe PELICAN comes in handy, seeing as it clearly produces many more IRC-safe observables. In terms of the jet-frame coordinates defined in \equref{energy-def} and \equref{angle-def}, the C-safety condition reads
\[
 \restr{\frac{\partial f}{\partial\mathcal{E}_i}}{\hat{d}_{ij=0}}=\restr{\frac{\partial f}{\partial\mathcal{E}_j}}{\hat{d}_{ij=0}}.
\]
The crucial observation here is that $\mathcal{E}_i$ and $\hat{d}_{ij}$ are \textit{not} analytic functions of the original coordinates $d_{ij}$, as is evident from \equref{energy-def}. Therefore the set of analytic symmetric functions in these coordinates is very different, and in fact much larger, as we already know from our construction of IRC-safe PELICAN.


% We intend to show that any smooth Lorentz-invariant observable $f$ that is both IR-safe and C-safe in fact depends only on the jet mass $m_{\mathrm{jet}}^2=\sum_{k,l} d_{kl}=\left(\sum_k p_k\right)^2$. To that end, assume for a moment that this is not true. Then we can invoke IR-safety to summon a new input 4-vector $p_{N+1}=0$ without changing the initial values of $f$, and then start varying it. Equation (\ref{2525}) must remain true for all values of the dot products $d_{i,N+1}$ and $d_{j,N+1}$. It is certainly impossible for both sides of the equation to be independent of $d_{i,N+1}$ unless $f$ is independent of $p_{N+1}$, which would imply that $f$ is a constant by permutation symmetry. If, as we assumed, $f$ is functionally independent of $m_{\mathrm{jet}}^2$ (i.e.~$\dd f\wedge \dd m_{\mathrm{jet}}^2\neq 0$), then due to permutation symmetry the partial derivatives with respect to \textit{any} two different components of the matrix $\{d_{kl}\}$ do not always match, and the two sides of (\ref{2525}) will necessarily start diverging in value as we vary $p_{N+1}$, making $f$ not C-safe. Therefore in the C-safe case the restriction to the coordinate hyperplane in (\ref{2525}) can be lifted so that the derivatives of $f$ with respect to all dot products coincide, and hence $f$ is only a function of $m_{\mathrm{jet}}^2$.

% The theorem above becomes possible only when both of these constraints are enforced.


\subsection{Lorentz meets EFPs}

\subsubsection{Review of EFPs}

Particle data is often analyzed via IRC-safe observables such as $N$-subjettiness \cite{Thaler:2010tr}. A general polynomial basis for all IRC-safe observables was obtained in ref.~\cite{EFP}. Here we first retrace some of the  steps from the original derivation, and introduce a Lorentz-invariant analog at the end. Massless vectors can be written as $p=(E,E\hat{p})$ with a unit 3-vector $\hat{p}$. Then any (not necessarily Lorentz-invariant) smooth,\footnote{It is important to note that smoothness is generally an excessively powerful restriction, and many useful IRC-safe observables cannot be expanded in Taylor series this way. For instance, staying within the realm of Lorentz-invariant observables, the quantity $\sum_{i,j} \mathcal{E}_i \mathcal{E}_j \left(\hat{d}_{ij}\right)^\beta$ is IRC-safe for any $\beta$, but when written out in terms of $d_{ij}$ it is clearly not differentiable at the origin. We thank Andrew Larkoski for this example, prompting much of our discussion of IRC-safety as presented in this work.} symmetric observable of a set of $N$ massless vectors can be expanded at low energies as a combination of terms (omitting the constant term) of the form
\[\sum_{i_1,\ldots ,i_M=1}^N \sum_{j_1,\ldots, j_L=1}^N E_{i_1}\dots E_{i_M}\cdot f^{(N)}_{i_1,i_2,\ldots,i_M}\left(\hat{p}_1,\ldots,\hat{p}_N\right)\]
with some ``angular functions'' $f^{(N)}_{i_1,i_2,\ldots,i_M}$.  As shown in ref.~\cite{EFP}, IR-safety for such a series amounts to requiring that the angular function $f^{(N)}_{i_1,i_2,\ldots,i_M}$ in fact depends only on the particles whose indices are listed in the label of the function and is also independent of the total number $N$, i.e. it can be replaced by $f_{i_1,i_2,\ldots,i_M}\left(\hat{p}_{i_1},\ldots,\hat{p}_{i_M}\right)$. Furthermore, permutation symmetry of the entire observable descends to the total permutation symmetry of each angular function. Finally, C-safety requires that equating any two unit vectors $\hat{p}_i\to \hat{p}_j$ should lead to a function of only $E_1+E_2$. At the level of angular functions, this implies that whenever $i$ appears as an index in the subscript of the angular function, it can be replaced with $j$, and this can be done one index at a time. Ultimately this reduces the entire set of angular functions to just one $f_M=f_{1,2,\ldots,M}$, and therefore the order $M$ term in the low-energy expansion of any IRC-safe observable becomes
\[\sum_{i_1,\ldots ,i_M=1}^N E_{i_1}\dots E_{i_M}\cdot f_M\left(\hat{p}_{i_1},\ldots,\hat{p}_{i_M}\right).\]
From here, EFP's are derived by expanding the angular functions into a Taylor series around small angles $\cos \theta_{ij}=\hat{p}_i\cdot\hat{p}_j$. The resulting polynomial expansion can be broken up into a linear basis enumerated by the set of isomorphism classes of multigraphs $G$ with no loops (edges connecting a vertex to itself):
\[\text{EFP}_G=\sum_{i_1,\ldots, i_M=1}^N \prod_{j\in V(G)}E_{i_j} \cdot \prod_{(k,l)\in E(G)}\theta_{i_k i_l},\]
where $V(G)=\{1,2,\ldots,M\}$ is the set of vertices of $G$, and $E(G)$ is the set of edges. The EFP is a homogenous polynomial in the energies of degree $|V(G)|=M$ and in the angles of degree $D=|E(G)|$. Notice that the indices $i_j$ can coincide but the corresponding terms vanish since $\theta_{ii}=0$, which is why multigraphs with loops are excluded from the basis. As a trivial but important example, the total energy $E=\sum_{i=1}^N E_i$ corresponds to $G=\bullet$. If $G$ consists of multiple connected components, the resulting EFP is the product of the EFP's corresponding to the components, so the entire basis is algebraically generated by just the connected multigraphs.

It is instructive to see how IRC-safety works in EFP's. IR-safety is guaranteed by the mere presence of the energy pre-factors: sending $E_N\to 0$ will simply recover the same EFP for the remaining $N-1$ particles.  Meanwhile, C-safety is observed because if two particles, say $1$ and $2$, become collinear, then the angular factor is completely invariant under their permutations. In other words, all terms where the list $(i_1,i_2,\ldots,i_M)$ contains a fixed number, say $L$, of $1$'s \textit{or} $2$'s, can be grouped together, and after summation over those indices the energy pre-factor manifestly depends on $E_1$ and $E_2$ only through an overall factor of $(E_1+E_2)^L$. This is exactly the statement of C-safety.

\subsubsection{MFPs: Lorentz-invariant analytic EFPs}

Now let us task ourselves with identifying the subset of Lorentz-invariant IRC-safe observables. Obviously none of the EFP's are Lorentz-invariant due to the direct dependence on spatial angles, but smooth Lorentz-invariant observables can still be expanded in the EFP basis at small energies and angles. The only Lorentz-invariant of two massless 4-vectors $p_i,p_j$ is $p_i\cdot p_j=E_i E_j(1-\cos\theta_{ij})$. At small angles the approximately boost-invariant combination is then $E_iE_j\theta_{ij}^2$. The expansion of any permutation-symmetric function of such dot products into a series as above will differ from general EFP's as follows: every angle $\theta_{ij}$ must appear in the product an even number of times, and the pre-factor must consist of nothing but one factor of $E_i E_j$ for every factor of $\theta_{ij}^2$ (or $1-\cos\theta_{ij}$). The Lorentz-invariant version of the polynomial can then be obtained by replacing each $\theta_{ij}^2$ with $2\hat{d}_{ij}$. In each EFP only a subset of terms will generally satisfy these conditions, so a better representation of the Lorentz-invariant basis is warranted.

Let us start again with the most general Lorentz-invariant and permutation-invariant observable of $N$ 4-momenta that is \textit{analytic} when expressed in terms of the dot products $d_{ij}=p_i\cdot p_j$. Such a function $f$ can be expanded at small arguments into a linear combination $f\sim \sum_G w_G f_G^{(N)}$ of terms of the form
\[f^{(N)}_G=\sideset{}{'}\sum_{i_1,\ldots, i_M=1}^N \prod_{(k,l)\in E(G)} d_{i_k i_l},\label{1535}\]
where the prime indicates that the terms where any two of the indices coincide are excluded. Including them would also produce a valid basis, but we leave them out to reduce the complexity of each polynomial. Such terms simply correspond to other multigraphs obtained from $G$ by vertex identification, and those already appear in the expansion with independent coefficients. This is still an over-complete basis since the matrix of dot products $d_{ij}$ is highly degenerate for $N\geq 5$. Unlike in EFP's, we also allow multigraphs with loops since we are not restricting ourselves to purely massless inputs. However, if all inputs are massless, all graphs with loops will produce vanishing polynomials.

Now we can easily see how to enforce IRC-safety in these polynomials. IR-safety requires that sending any of the 4-momenta to zero, say $p_N\to 0$, must be equivalent to simply restricting the same expression to only the other $N-1$ particles. We observe that the only case when this is not already satisfied in the expression above is when the multigraph $G$ contains isolated vertices (vertices of degree zero). Isolated vertices lead to summations over dummy indices corresponding to those vertices, which results in an overall factor of some power of $N$. The multiplicity $N$ is not an IR-safe variable, and it is sufficient to ban graphs with isolated vertices to enforce IR-safety. We can thus drop the notation $f^{(N)}_G$ and simply write $f_G$ with the implicit understanding that \equref{1535} defines the full infinite family of observables for all $N$.

% Figure environment removed
Now we address C-safety. Much like in EFP's, C-safety in $f_G$'s means that whenever, say, $p_1$ and $p_2$ are massless, each monomial that contains a certain number of indices equal to $1$ or $2$ must match its coefficient with any other monomial that differs only by flipping any number of the indices from $2$ to $1$ or vice versa. In the language of multigraphs, this means that each monomial that assigns a vertex of degree $n_1$ to $p_1$ and another vertex of degree $n_2$ to $p_2$ must in fact result as just one among all possible monomials obtained via vertex identification from a graph $G'$ where those two vertices have been ``blown up'' into $n_1+n_2$ vertices of degree one. We also notice that this condition applies only to vertices that do not have any loops attached to them because otherwise the corresponding terms would vanish as $p_1$ and $p_2$ become massless, thereby making any additional restrictions unnecessary. To summarize, any IRC-safe polynomial contains with equal coefficients all $f_G$'s with $G$'s obtained by any combination of vertex identifications from some multigraph $G'$ such that all of its loopless vertices are \textit{leaves} (i.e.~of degree one). We shall call such multigraphs \textit{loop-saturated}. In this process it is sufficient to allow identifications of only the leaf vertices because all other vertex identifications will result in a polynomial independently defined by another loop-saturated multigraph.

We thus define an IRC-safe Lorentz-invariant polynomial basis indexed by non-isomorphic loop-saturated multigraphs $G$ with no isolated vertices. Assuming the vertices of $G$ are identified with the set $\{1,2,\ldots,M\}$, we write
\[\text{MFP}_G=\sum_{i_1,\ldots, i_M=1}^N \prod_{(k,l)\in E(G)} d_{i_k i_l} = \sum_{i_1,\ldots, i_M=1}^N \prod_{j\in V(G)}\mathcal{E}_{i_j}^{\delta_j} \prod_{(k,l)\in E(G)} \hat{d}_{i_k i_l},\label{MFP-jet-frame}\]
where the sum is taken over all $M$-tuples and $\delta_1,\ldots,\delta_M$ are the degrees of the vertices of $G$ (each loop increases the degree by $2$). An alternative basis is
\[\text{MFP}_G'=\sideset{}{'}\sum_{i_1,\ldots, i_M=1}^N \prod_{(k,l)\in E(G)} d_{i_k i_l},\]
where the coincidence of two or more indices $i_k=i_l$ is allowed only if their corresponding vertices ($k$ and $l$) are leaves of $G$. $\text{MFP}_G$ is simply $\text{MFP}_G'$ plus a linear combination of certain $\text{MFP}_H'$'s with fewer vertices, namely such that $H$ is a graph obtained from $G$ by one or more vertex identifications each of which involves at least one non-leaf vertex of $G$. Indeed, such identifications preserve the property of being loop-saturated, and all other identifications are already included in the definition of $\text{MFP}_G'$.

Note that just like for EFP's, disconnected multigraphs correspond to products of their connected components (this holds for $\text{MFP}_G$ but not for $\text{MFP}_G'$):
\[\text{MFP}_{G\sqcup H}=\text{MFP}_G \cdot \text{MFP}_H.\]
As an example, if all inputs are known to be massless, then any connected multigraph with loops will produce a vanishing polynomial. And the only loopless loop-saturated connected multigraph is the graph consisting of just one edge, which evaluates to the jet mass: $\text{MFP}_{\bullet - \bullet}=m_J^2=\sum_{i,j} d_{ij}=\left(\sum_{i} \mathcal{E}_i\right)^2$. Therefore in the massless case MFP's generate all analytic functions of $m_J^2$, which is a disappointingly small subset of all IRC-safe Lorentz-invariant observables. As a final example, we can consider the polynomial $d_{12} d_{23}d_{31}$, which nominally seems C-safe. It can only be interpreted as an IR-safe polynomial if it corresponds to the triangle graph. However, such a graph is not loop-saturated, and therefore this polynomial is not IRC-safe despite being C-safe for $N=3$ (which can be confirmed by observing that for $N=4$ the triangle graph generates a non-C-safe polynomial), and the ``correct'' IRC-safe completion is simply $m_J^6$.

What we found is congruent with \equref{thmB1} in that MFP's represent a polynomial expansion basis for all IRC-safe observables that are analytic in the dot products $d_{ij}$, and depend on massless inputs only through their sum. Since we already know how restrictive this result is, we can now move on to a more useful generalization of EFPs.

\subsubsection{JFPs: Lorentz-invariant jet-frame EFPs}

As already discussed above, by transforming our inputs from $d_{ij}$ to the jet-frame coordinates $\mathcal{E}_i$ and $\hat{d}_{ij}$ we get a much larger space of IRC-safe analytic observables. Ideologically, the transformation to the jet-frame coordinates is extremely simple: $\mathcal{E}_i$ is nothing but the regular energy $E_i$ \textit{as measured in the rest frame of the jet}, and the same is true for the angular parts which take the form $\hat{d}_{ij}=1-\cos\Theta_{ij}$ for a pair of massless particles. This coordinate transformation itself is IRC-safe in the sense that IR and collinear splittings that don't involve constituent $p_i$ will not affect its jet-frame energy and angles with the other constituents. Therefore the notion of IRC-safe \textit{continuous} observables is the same in both coordinate systems. On the other hand, as we have seen above, the restriction to analyticity makes the two sets of IRC-safe observables very different.

All of this implies that the set of IRC-safe observables which are invariant under spatial rotations and restricted to the manifold of events whose jet momentum $J$ is purely time-like is in one-to-one correspondence with Lorentz-invariant IRC-safe observables. Indeed, given a rotationally invariant observable $f(p_1,\ldots,p_N)$, consider
\[
f_L(p_1,\ldots,p_N)=f(\Lambda_J p_1,\ldots,\Lambda_J p_N),
\]
where $\Lambda_J$ is a Lorentz transformation that maps the jet momentum $J=\sum_i p_i$ to $(m_J,0,0,0)$. Note that $\Lambda_J$ is not unique, namely $R\Lambda_J$ is also a solution for any spatial rotation $R$. However, since we have assumed $f$ to be rotationally invariant, the resulting observable $f_L$ is independent of the specific choice of $\Lambda_J$ and is \textit{fully Lorentz-invariant}. And since $\Lambda_J$ is invertible, the transformation $f\mapsto f_L$ is also invertible, which proves the one-to-one correspondence.\footnote{This principle, of course, applies not only to observables but to entire algorithms, such as jet clustering. E.g.~we defined our Lorentz-invariant analog of Soft Drop multiplicity using exactly this method.}

We conclude that a polynomial expansion basis for analytic IRC-safe Lorentz-invariant observables in the jet-frame coordinates can be trivially obtained by taking the original EFPs, replacing $\theta_{ij}^2$ with their rotationally invariant analogs $2(1-\cos\theta_{ij})$, and then replacing $E_i\mapsto \mathcal{E}_i$ and $(1-\cos\theta_{ij})\mapsto \hat{d}_{ij}$. We do have to abandon EFPs that include odd powers of any angle due to their non-analyticity, but that doesn't affect the completeness of the basis since all the angular coordinates are non-negative (however it means that every edge in our multigraphs corresponds to two edges in the analogous EFP). The only subtlety here is that the original EFPs were built only for all-massless inputs, so they don't depend on masses whereas $\hat{d}_{ij}$ do. But all the original arguments in the derivation of the EFPs still apply even when some of the inputs are massive (as long as those masses are also perturbatively small), so we still get a complete basis of IRC-safe observables that are analytic in the jet-frame coordinates. We call these polynomials the \textit{Jet Flow Polynomials} (JFPs):
\[
\mathrm{JFP}_G = \sum_{i_1,\ldots, i_M=1}^N \prod_{j\in V(G)}\mathcal{E}_{i_j} \cdot \prod_{(k,l)\in E(G)}\hat{d}_{i_k i_l},
\]
where the multigraphs $G$ have no restrictions on them and can contain loops. Comparing to \equref{MFP-jet-frame}, we see that the powers of the energies are no longer tethered to the structure of the multigraph, which is why C-safety doesn't place any constraints on the multigraph unlike in the case of MFPs. Namely, the symmetry that C-safety forces on the angular parts $\prod_{(k,l)} \hat{d}_{i_k i_l}$ (that one must be able to switch the value of an index $i_k$ from one collinear particle to another and obtain another monomial in the same polynomial) is no longer coupled to the power counting of the energy factors, which makes any multigraph $G$ permissible as long as we sum over all possible values of the indices $i_k$.

\subsection{Universality of PELICAN}

While the Deep Sets theorem \cite{ZaKoRaPoSS17} shows that permutation-invariant networks are universal for set learning problems, it is known that message passing networks are strictly non-universal for \textit{graph} learning problems, i.e.~in the presence of rank 2 data such as an adjacency matrix. Permutation-equivariant networks that take graph data as input (PELICAN's inputs can be interpreted as such) and use layers like $\mathrm{Eq}_{2\to 2}$ are known to have the expressivity of the 2-WL (Weisfeiler-Lehman) graph isomorphism test, matching the expressivity of message passing networks, see ref.~\cite{Azizian_expressivity}. By adding matrix multiplication to the list of equivariant aggregators (thus turning the linear equivariant layer into a quadratic or polynomial one), one can raise the expressivity to the ``folklore'' version of the WL test, i.e.~2-FWL for PELICAN, which is equivalent to 3-WL \cite{Maron19ProvablyPowerful}. However, in our tests we have not been able to see any performance improvements from the addition of an aggregator of this kind.

We can see the limited expressivity of PELICAN in terms of the graphs $G$ in \equref{1535} that can be generated by PELICAN when expanded at small values of the inputs (assuming a smooth activation function). Indeed, the messaging layer combined with single-index aggregation $\sum_i \bullet_{ij}$ can attach new edges ($d_{ij}$), loops ($d_{jj}$), and multi-edges ($d_{ij}^n$) to a given vertex $j$. By applying more of these layers we can only keep joining such ``starfish'' graphs at their central vertex $j$, whereas even something as simple as the triangle graph can never be generated due to the fact that it requires three free indices at an intermediate step. The only graphs with two free indices generated by PELICAN are the multi-edges $d_{ij}^n$, but joining them together will never produce a more complex graph.  At best, PELICAN can only generate multigraphs that can be broken up into subgraphs with at most two severed edges on each of them (since each severed edge requires a free index on the corresponding vertices for us to be able to ``glue'' the vertex back together via aggregation).

Finally, it is curious to compare PELICAN-like architectures to architectures that work with irreducible representations of the continuous symmetry group, like LGN \cite{Bogatskiy:2020tje}. As was shown in ref.~\cite{Maron21SO3Universality}, the latter kind of network is universal, provided that the set of the finite-dimensional irreducible representations that are stored is not limited. In particular, the proof involves showing that a universal network based on the tensor product nonlinearity will be able to generate all symmetric polynomial tensors of the form $P_{\alpha}=\sum_{i_1,\ldots,i_r} p_{i_1}^{\alpha_1}\otimes \cdots \otimes p_{i_r}^{\alpha_r}$ where $\alpha\in\mathbb{Z}^r$. It is easy to see that the Lorentz-invariant part of such a tensor produces all possible multigraphs $G$ with vertices of degrees $\alpha_1,\ldots,\alpha_r$. Therefore, indeed, the ability of a network to generate all such tensors implies its universality on the space of Lorentz-invariant observables.

However, in practice the dimensionality of these tensor representations has to be quite limited. For example, LGN kept only representations up to spin 2, that is, tensors of the form $p_i\otimes p_j$, but not higher than that (of course, these can also be multiplied by invariant scalars without increasing the dimension). This effectively limits the number of free particle indices in all latent values of the network to $2$ (right before aggregation reduces it back to 1 again), just like in PELICAN. Therefore there is ample reason to believe that the expressivity of LGN-like networks that go up to spin $k$ is not universal, but is equivalent to the expressivity of PELICAN-like networks that go up to the permutation-equivariant rank of $k$ (using $\mathrm{Eq}_{k\to k}$ blocks). Moreover, this expressivity can be stated in terms of the class of multigraphs $G$ that any analytic network of this kind can generate. We leave this problem for future work.

\subsection{Quantifying IRC-safety}

This section is an attempt to define some Lorentz-invariant quantities that can measure \textit{whether} and \textit{how} IR/C-safe any given observable is. It is mostly of theoretical interest since it does not yet have a software implementation.

Let $f^{(N)}$ be a Lorentz-invariant permutation-symmetric observable defined for varying number $N$ of inputs, for instance the output of a PELICAN instance. We define the following difference operator that measures the obstruction to $f$ being IR-safe:
%-------------------
\begin{equation}
    \Delta^{\mathrm{IR}} f^{(N)} = f^{(N+1)}\left(p_1,\ldots,p_N,0\right)- f^{(N)}\left(p_1,\ldots,p_N\right).
\end{equation}
%-------------------
IR-safety is equivalent to the vanishing of this operator:
\[\mathrm{IR-safety:}\quad \Delta^{\mathrm{IR}} f=0.\]
Furthermore, we can define \textit{IR-robustness} as the partial derivative corresponding to the injection of an infinitesimally soft but non-zero 4-momentum:
%-------------------
\begin{equation}
    \mathrm{IR-robustness:}\quad R^{\mathrm{IR}}_p f^{(N)}=\restr{\frac{\dd}{\dd\epsilon}}{\epsilon=0} f^{(N+1)}(p_1,\ldots,p_N,\epsilon p).
\end{equation}
%-------------------
All together, we have the following series expansion for a general observable near $p_{N+1}=0$:
\[f^{(N+1)}=f^{(N)}+\Delta^{\mathrm{IR}}f^{(N)}+\left(R^{\mathrm{IR}}_{p_{N+1}} f^{(N)}\right)\cdot p_{N+1}+\mathcal{O}(p_{N+1}^2).\]

An IR-safe observable must always have $\Delta^{\mathrm{IR}}f=0$, and lower values of IR-robustness indicate lower sensitivity to soft particles. In practice, we can evaluate the ratio $(\Delta^{\mathrm{IR}}f^{(N)})/f^{(N)}$ and average it over a testing dataset to quantify the IR-safety of a model.

For C-safety, we pick a massless 4-vector $p$ and define the differential operator
%-------------------
\begin{equation}
    D^{\mathrm{C}}_{p} f=\frac{\dd}{\dd\lambda}{\lambda=0}f(p_1+\lambda p,p_2-\lambda p,p_3,\ldots).
\end{equation}
%-------------------
C-safety then is equivalent to the requirement that $D^{\mathrm{C}}_{p} f=0$ for any massless $p$ whenever $p_1$ and $p_2$ are collinear with $p$:
%-------------------
\begin{equation}
    \mathrm{C-safety: }\quad \restr{D^{\mathrm{C}}_{p} f}{p_1\parallel p_2\parallel p}=0, \quad p^2=0.
\end{equation}
%-------------------

\textit{C-robustness} can be easily defined as the second-order analog of $D^{\mathrm{C}}_{p}$. Namely, we pick a second vector $\hat{p}$ such that $p\cdot \hat{p}=0$ (which means that $\hat{p}$ is tangent to the light cone at $p$, as required by the massless constraint) and measure how quickly $D^{\mathrm{C}}_{p} f$ deviates from zero as we deform $p_1$ and $p_2$ away from collinearity:
%-------------------
\begin{equation}
    \mathrm{C-robustness: }\quad R^{\mathrm{C}}_{\hat{p},p}f=D^{\mathrm{C}}_{\hat{p}} D^{\mathrm{C}}_{p} f, \quad \hat{p}\cdot p=p^2=0, \quad p_1\parallel p_2\parallel p.
\end{equation}
%-------------------

It is instructive to express these quantities in terms of the dot products $d_{ij}$. We have
%-------------------
\begin{equation}
    D^{\mathrm{C}}_{p}f=\sum_{j=1}^N \left(\partial_{1j}f-\partial_{2j}f\right)(p_j\cdot p),
\end{equation}
%-------------------
where $\partial_{kl}$ indicates the partial derivative with respect to $d_{kl}$. Note that individual partial derivatives aren't really well-defined due to the fact that  $\{d_{ij}\}$ is not a set of independent coordinates on the manifold of $N$ 4-vectors for $N\geq 5$. However, combinations such as above are valid since they are nothing but a different way of expressing the original well-defined operator. We also immediately notice that if $N=2$ and $p_1\parallel p_2\parallel p$, then $D^{\mathrm{C}}_{p}f=0$ automatically. Indeed, every Lorentz-invariant observable for two particles is C-safe due to the fact that $d_{12}=\frac12 (p_1+p_2)^2$ for massless inputs.

C-robustness is similarly expressed by
%-------------------
\begin{equation}
    R^{\mathrm{C}}_{\hat{p},p} f=\sum_{i,j=1}^N\left(\partial^2_{1i,1j}+\partial^2_{2i,2j}-\partial^2_{1i,2j}-\partial^2_{1j,2i}\right)f\cdot (p_i \cdot \hat{p})(p_j\cdot p).
\end{equation}
%-------------------





% \section{Symmetry properties of PELICAN aggregators}
% The following alternative basis of the $\mathrm{Eq}_{2\to2}$ linear equivariant aggregators is convenient for studying and interpreting trained PELICAN networks. In the following, $S$ and $A$ will stand for symmetric and antisymmetric output arrays respectively (``output-(anti)symmetric''), and the $+$ and $-$ subscripts will stand for symmetry with respect to input arrays (``input-(anti)symmetric''). Namely, denoting by $T$ the transposition operator $(T F)_{ij}=F_{ji}$, we have $TS=S$, $TA=-A$, $S_+ T=S_+$, $S_- T=-S_-$, $TS_+ T=S_+$, and so on. Finally, $D$ will stand for aggregators where only the diagonal of the output depends on the input (``output-diagonal'', they are also output-symmetric), and subscript $0$ denotes aggregators whose outputs depend only on the diagonal components of the input (``input-diagonal'', they are also input-symmetric). Aggregators with no subscript are bi-symmetric. The superscript shows the aggregation order. With this, we can define a symmetrized linear basis for the 15-dimensional space of aggregators. 

% \begin{align}
%     S^0=&\frac12 (B_1+B_2-2B_3)\\
%     A^0=&\frac12 (B_1-B_2)\\
%     D^0=&B_3\\
%     S_0^0=&\frac12(B_4+B_5-2B_3)=S_+^1 D^0\\
%     A_0^0=&\frac12(B_4-B_5)=A_+^1 D\\
%     D_+^1=&\frac12(B_6+B_9-2B_3)\\
%     D_-^1=&\frac12(B_6-B_9)\\
%     S_+^1=&\frac14(B_7+B_{10}+B_8+B_{11}-4B_3)\\
%     S_-^1=&\frac14(B_7-B_{10}+B_8-B_{11})\\
%     A_+^1=&\frac14(B_7+B_{10}-B_8-B_{11})\\
%     A_-^1=&\frac14(B_7-B_{10}-B_8+B_{11})\\
%     S_0^1=&B_{12}\\
%     D_0^1=&B_{13}\\
%     S^2 = & B_{14}\\
%     D^2 = &B_{15}
% \end{align}