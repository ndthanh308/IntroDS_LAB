In this section we apply PELICAN to another binary classification task: distinguishing jets produced by gluons from those produced by light quarks. We study three variants of PELICAN for this task: both with and without the use of particle ID labels (PID), as well as a third IRC-safe version. We compare the performance of each of these three variants to other published architectures. We also compare PELICAN models of three different widths and find that even a very small PELICAN model achieves state-of-the-art tagging performance.

%------------------------
% DATASET
%------------------------
\subsection{Quark-gluon jet dataset}

We use the public dataset introduced in ref.~\cite{EFN}. It consists of an equal number of jets produced either by gluons or light quarks $(u,d,s)$. The non-neutrino products are clustered using \textsc{FastJet}~\cite{FastJet} with anti-$k_T$ jet~\cite{Cacciari:2008gp} radius $R=0.4$, and the jets are restricted to $p_T \in [500,550]$ GeV and $|y|<2$. There is no detector simulation, and the particle ID is stored. We select the first 200k events for validation, the following 200k for final testing, and the remaining 1.6M for training. The samples can be downloaded from \cite{QG_Zenodo}.


%===========================================
% \WFclear % see here: https://tex.stackexchange.com/a/442369

%------------------------
% RESULTS
%------------------------

\begin{table}[t]
    \vspace{-0.\intextsep}
    \centering
    \begin{small}
    \begin{tabular}{l@{\hspace{2mm}}l@{\hspace{2mm}}l@{\hspace{2mm}}l@{\hspace{2mm}}l@{\hspace{2mm}}r}
    \toprule
    Architecture                 &   Accuracy    &   AUC   &   $1/\epsilon_B\;(\epsilon_S=0.3)$ & $1/\epsilon_B\;(\epsilon_S=0.5)$ &   \# Params \\
    \midrule
    \textbf{Not IRC-safe, w/ PID}  &&&&&\\
    PFN-ID\cite{EFN}              &   --          &   0.9052(7)   &   --       & 37.4 $\pm$ 0.7 &   82k     \\
    ParticleNet-ID\cite{ParticleNet} &   0.840       &   0.9116       &   98.6 $\pm$ 1.3 &  39.8 $\pm$ 0.2 &   498k   \\
    ABCNet\cite{ABCNet} &   0.840       &   0.9126      & 118.2 $\pm$ 1.5  & -- &   230k    \\
    LorentzNet\cite{LorentzNet22} &   0.844       &   0.9156      & 110.2 $\pm$ 1.3  & 42.4 $\pm$ 0.4&   220k    \\
    $\text{ParT}_{\text{full}}$ \cite{ParT}    &   0.849    &   0.9203   & 129.5 $\pm$ 0.9  &  47.9 $\pm$ 0.5&   2.1M    \\
    % $\text{PELICAN-ID}_{5,25/15}$   &    0.8519(5)  &   0.9212(7) &  131.1  $\pm$ 2.5  &  47.8 $\pm$ 0.5  & 12k  \\   %n=100, qg5
    % $\text{PELICAN-ID}_{5,60/35}$   & 0.8547(2)  &  0.9236(5)  & 133.4 $\pm$ 0.9   &  50.3 $\pm$ 0.8   &   50k  \\  %n=80, qg3
    $\text{PELICAN}_{\text{PID}}$  &  \textbf{0.8551(2)}    &   \textbf{0.9252(1)}   &  \textbf{149.8 $\pm$ 2.4} &  \textbf{52.3 $\pm$ 0.3}  &   209k \\ %n=100, gq17
    \midrule
    \textbf{Not IRC-safe, w/o PID}  &&&&&\\
    PFN\cite{EFN}                 &   --          &   0.8911(8)   &   --       & 30.8 $\pm$ 0.4 &   82k     \\
    ParticleNet\cite{ParticleNet} &   0.828       &   0.9014       &   85.4      &  33.7 &   498k   \\
    $\text{PELICAN}$  &    \textbf{0.8349(4)}  &   \textbf{0.9076(1)}   &  \textbf{94.5 $\pm$ 0.5} &  \textbf{37.5 $\pm$ 0.1}  &   208k     \\    % qg17
    \midrule
    \textbf{IRC-safe}  &&&&&\\
    EFN\cite{EFN}                 &   --          &   0.8824(5)   &   --       & 28.6 $\pm$ 0.3 & 82k        \\
    EFP\cite{EFP}                 &   --          &   0.8919      &   --       & 29.7 & 1k        \\
    EMPN\cite{Spannowsky_IRC_safe}&   --          &   0.8932(6)   &   --       & 30.8 $\pm$ 0.2 & $\sim$110k \\
    $\text{PELICAN}_{\text{IRC}}$  &  \textbf{0.8321(9)}    &   \textbf{0.9056(7)}   &  \textbf{91.7 $\pm$ 0.4}  &  \textbf{36.5 $\pm$ 0.3}  &   9k     \\   % n=100, (qg4)
    \bottomrule
    \end{tabular}
    \end{small}
    \caption{Comparison of different quark-gluon classifiers trained on the quark-gluon tagging dataset \cite{QG_Zenodo}.\label{tab_QG}}
\end{table}

\begin{table}[t]
    \vspace{-0.\intextsep}
    \centering
    \begin{small}
    \begin{tabular}{l@{\hspace{2mm}}l@{\hspace{2mm}}l@{\hspace{2mm}}l@{\hspace{2mm}}l@{\hspace{2mm}}l@{\hspace{2mm}}l@{\hspace{2mm}}r}
    \toprule
    PELICAN variant & $L$ & Width &    Accuracy    &   AUC   &   $1/\epsilon_B\;(\epsilon_S=0.3)$ & $1/\epsilon_B\;(\epsilon_S=0.5)$ &   \# Params \\
    \midrule
    % Not IRC-safe, w/ PID  & 132/78  &  0.8555(2)    &   0.9247(3)   &  134.8 $\pm$ 1.8 &  51.3 $\pm$ 0.7  &   209k \\ %n=100, gq4
    Not IRC-safe, w/ PID  & 5& 132/78  &  \textbf{0.8551(2)}    &   \textbf{0.9252(1)}   &  \textbf{149.8 $\pm$ 2.4} &  \textbf{52.3 $\pm$ 0.3}  &   209k \\ %n=100, gq17
     % & 60/35   & 0.8547(2)  &  0.9236(5)  & 133.4 $\pm$ 0.9   &  50.3 $\pm$ 0.8   &   50k  \\  %n=80, qg3
     % & 60/35   & 0.8538(3)  &  0.9242(2)  & 147.6 $\pm$ 1.4   &  50.7 $\pm$ 0.4   &   50k  \\  %qg12
     &5& 60/35   & 0.8544(5)  &  0.9245(2)  & 148.7 $\pm$ 2.2   &  51.5 $\pm$ 0.2   &   49k  \\  %qg16
     % & 25/15   &    0.8519(5)  &   0.9212(7) &  131.1  $\pm$ 2.5  &  47.8 $\pm$ 0.5  & 12k  \\ %n=100, qg5
     &5& 25/15   &    0.8506(3)  &   0.9216(2) &  134.8  $\pm$ 0.7  &  48.5 $\pm$ 0.4  & 12k  \\ % qg14
    \midrule
%    Not IRC-safe, w/o PID  & 132/78 &    0.8342(2)  &   0.9059(8)   &  88.9 $\pm$ 0.5 &  36.0 $\pm$ 0.2  &   209k     \\    %n=100, qg9
    % Not IRC-safe, w/o PID  &5& 132/78 &   \textbf{0.8347(3)}  &   \textbf{0.9075(1)}   &  \textbf{95.3 $\pm$ 1.0} &  \textbf{37.7 $\pm$ 0.2}  &   208k     \\    %n=100, qg9
    Not IRC-safe, w/o PID  &5& 132/78 &   \textbf{0.8349(4)}  &   \textbf{0.9076(1)}   &  \textbf{94.5 $\pm$ 0.5} &  \textbf{37.5 $\pm$ 0.1}  &   208k     \\    % qg17
     % & 60/35   &    0.8344(2)  &   0.9059(9)   &  89.2  $\pm$ 0.5  &  35.6 $\pm$ 0.4  &   48k \\ %n=100, qg2 (qg2-5-9)
     %& 60/35   &    0.8343(3)  &   0.9073(1)   &  93.9  $\pm$ 0.6  &  37.4 $\pm$ 0.2  &   48k   \\ % qg12
     &5& 60/35   &    0.8344(1)  &   0.9073(1)   &  93.6  $\pm$ 1.1  &  37.4 $\pm$ 0.1  &   48k   \\ % qg16
     % & 25/15   &    0.8324(3)  &   0.9046(5)   &  87.5  $\pm$ 0.7  &  35.0 $\pm$ 0.2  &   11k  \\  %n=120, qg5
     &5& 25/15   &    0.8330(5)  &   0.9063(3)   &  92.8  $\pm$ 0.5  &  36.8 $\pm$ 0.3  &   11k  \\  % qg14
    \midrule
    % IRC-safe  & 132/78  &  0.8299(3)    &   0.8955(18)   &  85.7 $\pm$ 1.2  &  33.8 $\pm$ 0.2  &   204k  \\% n=100, (qg4)
    %IRC-safe  & 132/78  &  0.8292(7)    &   0.9034(3)   &  89.9 $\pm$ 0.8  &  35.7 $\pm$ 0.2  &   204k  \\%qg11
    IRC-safe  & 5&132/78  &  0.8294(3)    &   0.9034(1)   &  90.5 $\pm$ 1.3  &  35.7 $\pm$ 0.2  &   204k  \\%qg17
     % & 60/35   &  0.8293(2)  &  0.8949(24)  &   85.2 $\pm$ 0.7  &  33.5 $\pm$ 0.2   &   48k  \\   %n=100, qg8
     %& 60/35   &  0.8287(4)  &  0.9028(1)  &   89.2 $\pm$ 1.2  &  35.2 $\pm$ 0.2   &   48k  \\   % qg12
     &5& 60/35   &  0.8291(4)  &  0.9030(1)  &   90.5 $\pm$ 1.2  &  35.2 $\pm$ 0.2   &   46k  \\   % qg16
     % & 25/15   &  0.8282(4)  &  0.9005(3)  &   83.4 $\pm$ 0.4 &  33.3 $\pm$ 0.3   &   11k  \\  % n=100, qg5
     &4& 25/15   &  \textbf{0.8321(9)}  &  \textbf{0.9056(7)}  &   \textbf{91.7 $\pm$ 0.4} &  \textbf{36.5 $\pm$ 0.3}   &   9k  \\ % qg13
    \bottomrule
    \end{tabular}
    \end{small}
    \caption{Comparison of PELICAN $q/g$ classifiers of varying widths. The smallest IRC-safe model was found to have a lower optimal depth of $L=4$.\label{tab_QG_size}}
\end{table}

\subsection{Quark-gluon jet tagging results}

The training procedure for the PELICAN quark-gluon tagger is identical to that of the top-tagger described above. The hyperparameters were also chosen to be the same, except here we take up to 100 highest-$p_T$ constituents. As before, we train models of three different shapes and compare their performance. In addition, we train these models both with and without PID information supplied as a set of scalar inputs. Finally, we train IRC-safe PELICAN models so that they can be directly compared to other IRC-safe architectures. For the details of the IRC-safe implementation, see \secref{irc}.

In \tabref{tab_QG} we compare PELICAN to other architectures trained on the quark-gluon dataset. We group the architectures according to whether they use PID inputs, and whether they are IRC-safe, to enable a more direct and fair comparison. Moreover, in \tabref{tab_QG_size} we also compare the performance of PELICAN models of different sizes, analogous to the comparison in \secref{toptagging}. Interestingly, among the IRC-safe models the one with the lowest number of parameters performed the best. This can potentially be explained by the accumulation of floating point errors that we discuss below in \secref{irc}. PELICAN achieves impressive state-of-the-art classification performance with as few as 11k parameters, surpassing architectures of up to 190 times larger sizes. Finally, while \tabref{tab_QG} includes only the non-pre-trained version of ParT to facilitate a direct and fair comparison, it is worth noting that the PELICAN quark-gluon tagger surpasses even the pre-trained ParT tagger as reported in Ref.~\cite{ParT}.