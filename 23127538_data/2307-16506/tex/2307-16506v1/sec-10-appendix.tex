\newpage
\section{Additional results and plots}\label{appendix_plots}

\subsection{$W$-boson 4-momentum reconstruction}
% Figure environment removed

% Figure environment removed

% Figure environment removed

% Figure environment removed

% Figure environment removed

% Figure environment removed

\subsection{$W$-boson mass measurement}

\begin{table}[h]
    \centering
    \begin{small}
        \begin{tabular}{ccS[table-format=3.2]<{\%}S[table-format=3.2]<{\%}S[table-format=3.3]}
        \toprule
        & Method &  \multicolumn{1}{c}{$\sigma_{p^T}$ (\%)} & \multicolumn{1}{c}{$\sigma_{m}$ (\%)} & \multicolumn{1}{c}{$\sigma_{\Delta R}$ (centirad)}\\
        \midrule
        \multirow{3}{*}{\rotatebox[origin=c]{90}{\parbox{1.3cm}{\centering Without\\ \Delphes}}} 
        & JH               & 7.98    & 4.75     & 22.180   \\
        & PELICAN$\mid$JH  & 0.26    & 0.58     & 0.111   \\
        & PELICAN$\mid$FC  & 0.31    & 0.76     & 0.142   \\
        & PELICAN          & 0.88    & 1.67    & 0.548   \\
        \midrule
        \multirow{3}{*}{\rotatebox[origin=c]{90}{\parbox{1.3cm}{\centering With\\ \Delphes}}} 
        & JH               & 16.0    & 12.0   & 25.4      \\
        & PELICAN$\mid$JH  & 4.1     & 6.3    & 3.2      \\
        & PELICAN$\mid$FC  & 4.7     & 7.3    & 3.5      \\
        & PELICAN          & 6.2     & 8.2    & 5.6      \\
        \bottomrule
        \end{tabular}
    \end{small}
    \caption{PELICAN resolutions for models trained to reconstruct $p^W_{\mathrm{true}}$ with variable $W$ mass.\label{btW6m_W_table}}
    \vspace{-0.5\intextsep}
\end{table}

% Figure environment removed

% Figure environment removed

% Figure environment removed

% Figure environment removed

\newpage
\section{IRC-safety and Lorentz symmetry}\label{appendix_irc}

Let us now characterize the IRC-safe Lorentz invariants in the case of events consisting entirely of massless inputs (which is often artificially enforced in simulated data that includes detector simulations). When working with Lorentz-invariant observables, it is convenient to rewrite everything in terms of Lorentz invariant coordinates -- the dot products $d_{ij}=p_i\cdot p_j$ (note that these coordinates are in general not independent since any $5\times 5$ minor of the matrix $\{d_{ij}\}$ must vanish due to the fact that any five $4$-vectors are linearly dependent, but they will suffice for our purposes). For two massless particles, collinearity is equivalent to the vanishing of the dot product. Conversely, the dot product between two time- or light-like vectors is zero only if both are light-like and collinear. Thus the C-safety condition amounts to the equality of the gradients of $f$ with respect to any two of the rows (or columns) of $\{d_{ij}\}$ when evaluated on the corresponding coordinate hyperplane:

\begin{equation}
    \restr{\frac{\partial f}{\partial d_{ki}}}{d_{ij=0}}=\restr{\frac{\partial f}{\partial d_{kj}}}{d_{ij=0}}.\label{2525}
\end{equation}

We intend to show that any smooth Lorentz-invariant observable $f$ that is both IR-safe and C-safe in fact depends only on the jet mass $m_{\mathrm{jet}}^2=\sum_{k,l} d_{kl}=\left(\sum_k p_k\right)^2$. To that end, assume for a moment that this is not true. Then we can invoke IR-safety to summon a new input 4-vector $p_{N+1}=0$ without changing the initial values of $f$, and then start varying it. Equation (\ref{2525}) must remain true for all values of the dot products $d_{i,N+1}$ and $d_{j,N+1}$. It is certainly impossible for both sides of the equation to be independent of $d_{i,N+1}$ unless $f$ is independent of $p_{N+1}$, which would imply that $f$ is a constant by permutation symmetry. If, as we assumed, $f$ is functionally independent of $m_{\mathrm{jet}}^2$ (i.e.~$\dd f\wedge \dd m_{\mathrm{jet}}^2\neq 0$), then due to permutation symmetry the partial derivatives with respect to \textit{any} two different components of the matrix $\{d_{kl}\}$ do not always match, and the two sides of (\ref{2525}) will necessarily start diverging in value as we vary $p_{N+1}$, making $f$ not C-safe. Therefore in the C-safe case the restriction to the coordinate hyperplane in (\ref{2525}) can be lifted so that the derivatives of $f$ with respect to all dot products coincide, and hence $f$ is only a function of $m_{\mathrm{jet}}^2$.

It is instructive to reflect on the interplay between IR-safety and C-safety. C-safety allows us to exchange momentum between collinear particles without affecting the values of C-safe observables, whereas IR-safety allows us to omit vanishing 4-momenta from the list of inputs. C-safety in itself doesn't change the number of input 4-vectors -- it only requires $f$ to be invariant under a certain transformation that mixes collinear massless inputs. Meanwhile IR-safety effectively requires $f$ to encode an infinite family of observables $f^{(N)}(p_1,\ldots,p_N)$ for any possible number of inputs with the compatibility condition $f^{(N)}(p_1,\ldots,p_{N-1},0)=f^{(N-1)}(p_1,\ldots,p_{N-1})$.  Only after combining C-safety with IR-safety do we arrive at the following result.

\begin{thm}
An IRC-safe Lorentz-invariant observable with a mixture of massless and massive inputs can depend on the massless inputs only through their sum (which is generically a massive vector).
\end{thm}

In this way, IRC-safety is a significantly more powerful restriction than one could na\"ively expect based on the definitions of IR and C-safety alone. As an example, consider the quantity $\prod_{i<j}d_{ij}=d_{12}d_{23}d_{31}$ on $N=3$ massless inputs. It is in fact C-safe because it simply vanishes whenever any two inputs are collinear. However it is not IR-safe because sending $p_3\to 0$ reduces the expression to zero instead of the expected $d_{12}$. Alternatively, as we will see in the following section, the natural IR-safe extension of $d_{12}d_{23}d_{31}$ to arbitrary $N$ will not be C-safe.

\subsection{Lorentz meets EFPs}

Particle data is often analyzed via IRC-safe observables such as $N$-subjettiness \cite{Thaler:2010tr}. A general polynomial basis for all IRC-safe observables was obtained in Ref.~\cite{EFP}. Here we first retrace some of the  steps from the original derivation, and introduce a Lorentz-invariant analog at the end. Massless vectors can be written as $p=(E,E\hat{p})$ with a unit 3-vector $\hat{p}$. Then any (not necessarily Lorentz-invariant) smooth, symmetric observable of a set of $N$ massless vectors can be expanded at low energies as a combination of terms (omitting the constant term) of the form
\[\sum_{i_1,\ldots ,i_M=1}^N \sum_{j_1,\ldots, j_L=1}^N E_{i_1}\dots E_{i_M}\cdot f^{(N)}_{i_1,i_2,\ldots,i_M}\left(\hat{p}_1,\ldots,\hat{p}_N\right)\]
with some ``angular functions'' $f^{(N)}_{i_1,i_2,\ldots,i_M}$.  As shown in Ref.~\cite{EFP}, IR-safety for such a series amounts to requiring that the angular function $f^{(N)}_{i_1,i_2,\ldots,i_M}$ in fact depends only on the particles whose indices are listed in the label of the function and is also independent of the total number $N$, i.e. it can be replaced by $f_{i_1,i_2,\ldots,i_M}\left(\hat{p}_{i_1},\ldots,\hat{p}_{i_M}\right)$. Furthermore, permutation symmetry of the entire observable descends to the total permutation symmetry of each angular function. Finally, C-safety requires that equating any two unit vectors $\hat{p}_i\to \hat{p}_j$ should lead to a function of only $E_1+E_2$. At the level of angular functions, this implies that whenever $i$ appears as an index in the subscript of the angular function, it can be replaced with $j$, and this can be done one index at a time. Ultimately this reduces the entire set of angular functions to just one $f_M=f_{1,2,\ldots,M}$, and therefore the order $M$ term in the low-energy expansion of any IRC-safe observable becomes
\[\sum_{i_1,\ldots ,i_M=1}^N E_{i_1}\dots E_{i_M}\cdot f_M\left(\hat{p}_{i_1},\ldots,\hat{p}_{i_M}\right).\]
From here, EFP's are derived by expanding the angular functions into a Taylor series around small angles $\cos \theta_{ij}=\hat{p}_i\cdot\hat{p}_j$. The resulting polynomial expansion can be broken up into a linear basis enumerated by the set of isomorphism classes of multigraphs $G$ with no loops (edges connecting a vertex to itself):
\[\text{EFP}_G=\sum_{i_1,\ldots, i_M=1}^N \prod_{j\in V(G)}E_{i_j} \cdot \prod_{(k,l)\in E(G)}\theta_{i_k i_l},\]
where $V(G)=\{1,2,\ldots,M\}$ is the set of vertices of $G$, and $E(G)$ is the set of edges. The EFP is a homogenous polynomial in the energies of degree $|V(G)|=M$ and in the angles of degree $D=|E(G)|$. Notice that the indices $i_j$ can coincide but the corresponding terms vanish since $\theta_{ii}=0$, which is why multigraphs with loops are excluded from the basis. As a trivial but important example, the total energy $E=\sum_{i=1}^N E_i$ corresponds to $G=\bullet$. If $G$ consists of multiple connected components, the resulting EFP is the product of the EFP's corresponding to the components, so the entire basis is algebraically generated by just the connected multigraphs.

It is instructive to see how IRC-safety works in EFP's. IR-safety is guaranteed by the mere presence of the energy pre-factors: sending $E_N\to 0$ will simply recover the same EFP for the remaining $N-1$ particles.  Meanwhile, C-safety is observed because if two particles, say $1$ and $2$, become collinear, then the angular factor is completely invariant under permutations of their two indices. In other words, all terms where the list $(i_1,i_2,\ldots,i_M)$ contains a fixed number, say $L$, of $1$'s \textit{or} $2$'s, can be grouped together, and after summation over those indices the energy pre-factor manifestly depends on $E_1$ and $E_2$ only through an overall factor of $(E_1+E_2)^L$. This is exactly the statement of C-safety.

Now let us task ourselves with identifying the subset of Lorentz-invariant IRC-safe observables. Obviously none of the EFP's are Lorentz-invariant due to the explicit dependence on angles, but smooth Lorentz-invariant observables can still be expanded in the EFP basis at small energies and angles. The only Lorentz-invariant of two massless 4-vectors $p_i,p_j$ is $p_i\cdot p_j=E_i E_j(1-\cos\theta_{ij})$. At small angles the approximately boost-invariant combination is then $E_iE_j\theta_{ij}^2$. The expansion of any permutation-symmetric function of such dot products into a series as above will differ from general EFP's as follows: every angle $\theta_{ij}$ must appear in the product an even number of times, and the pre-factor must consist of nothing but one factor of $E_i E_j$ for every factor of $\theta_{ij}^2$. In each EFP only a subset of terms will generally satisfy this condition, so a better representation of the Lorentz-invariant basis is warranted.

Let us start again with the most general Lorentz-invariant and permutation-invariant observable of $N$ 4-momenta. As we know, it is a function of only the dot products $d_{ij}=p_i\cdot p_j$. Such a function can be expanded at small arguments into terms like
\[f^{(N)}_G=\sideset{}{'}\sum_{i_1,\ldots, i_M=1}^N \prod_{(k,l)\in E(G)} d_{i_k i_l},\label{1535}\]
where the prime indicates that the terms where any two of the indices coincide are excluded. Including them would also produce a valid basis, but we leave them out to reduce the complexity of each polynomial. Such terms simply correspond to other multigraphs obtained from $G$ by vertex identification, and those already appear in the expansion with independent coefficients. Unlike in EFP's, we also allow multigraphs with loops since we are not restricting ourselves to purely massless inputs. However, if all inputs are massless, all graphs with loops will produce vanishing polynomials.

Now we can easily see how to enforce IRC-safety in these polynomials. IR-safety requires that sending any of the 4-momenta to zero, say $p_N\to 0$, must be equivalent to simply restricting the same expression to only the other $N-1$ particles. We observe that the only case when this is not already satisfied in the expression above is when the multigraph $G$ contains isolated vertices (vertices of degree zero). Isolated vertices lead to summations over dummy indices corresponding to those vertices, which results in an overall factor of some power of $N$. The multiplicity $N$ is not an IR-safe variable, and it is sufficient to ban graphs with isolated vertices to enforce IR-safety. We can thus drop the notation $f^{(N)}_G$ and simply write $f_G$ with the implicit understanding that (\ref{1535}) defines the full infinite family of observables for all $N$.

\begin{wrapfigure}{r}{0.3\textwidth}
    \caption{The addition of the dashed loops makes this multigraph \textit{loop-saturated}: all non-leaf vertices come with at least one loop. Loop-saturated multigraphs enumerate all IRC-safe polynomials.}
    \begin{center}
        \adjustbox{trim=0cm 0.6cm 0cm 0.4cm}{
        % Figure removed
        }
    \end{center}
\end{wrapfigure}
Now we address C-safety. Much like in EFP's, C-safety in $f_G$'s means that whenever, say, $p_1$ and $p_2$ are massless, each monomial that contains a certain number of indices equal to $1$ or $2$ must match its coefficient with any other monomial that differs only by flipping any number of the indices from $2$ to $1$ or vice versa. In the language of multigraphs, this means that each monomial that assigns a vertex of degree $n_1$ to $p_1$ and another vertex of degree $n_2$ to $p_2$ must in fact result as just one among all possible monomials obtained via vertex identification from a graph $G'$ where those two vertices have been ``blown up'' into $n_1+n_2$ vertices of degree one. We also notice that this condition applies only to vertices that do not have any loops attached to them because otherwise the corresponding terms would vanish as $p_1$ and $p_2$ become massless, thereby making any additional restrictions unnecessary. To summarize, any IRC-safe polynomial contains with equal coefficients all $f_G$'s with $G$'s obtained by any combination of vertex identifications from some multigraph $G'$ such that all of its loopless vertices are \textit{leaves} (i.e.~of degree one). We shall call such multigraphs \textit{loop-saturated}. In this process it is sufficient to allow identifications of only the leaf vertices because all other vertex identifications will result in a polynomial independently defined by another loop-saturated multigraph.

We thus define an IRC-safe Lorentz-invariant polynomial basis indexed by non-isomorphic loop-saturated multigraphs $G$ with no isolated vertices. Assuming the vertices of $G$ are identified with the set $\{1,2,\ldots,M\}$, we write
\[\text{MFP}_G=\sum_{i_1,\ldots, i_M=1}^N \prod_{(k,l)\in E(G)} d_{i_k i_l}\]
where the sum is taken over all $M$-tuples, and an alternative basis 
\[\text{MFP}_G'=\sideset{}{'}\sum_{i_1,\ldots, i_M=1}^N \prod_{(k,l)\in E(G)} d_{i_k i_l},\]
where the coincidence of two or more indices $i_k=i_l$ is allowed only if their corresponding vertices ($k$ and $l$) are leaves of $G$. $\text{MFP}_G$ is simply $\text{MFP}_G'$ plus a linear combination of certain $\text{MFP}_H'$'s with fewer vertices, namely such that $H$ is a graph obtained from $G$ by one or more vertex identifications each of which involves at least one non-leaf vertex of $G$. Indeed, such identifications preserve the property of being loop-saturated, and all other identifications are already included in the definition of $\text{MFP}_G'$.

Note that just like for EFP's, disconnected multigraphs correspond to products of their connected components (this holds for $\text{MFP}_G$ but not for $\text{MFP}_G'$):
\[\text{MFP}_{G\sqcup H}=\text{MFP}_G \cdot \text{MFP}_H.\]
As an example, if all inputs are known to be massless, then any connected multigraph with loops will produce a vanishing polynomial. And the only loopless loop-saturated connected multigraph is the graph consisting of just one edge, which evaluates to the jet mass: $\text{MFP}_{\bullet - \bullet}=\sum_{i,j} d_{ij}=m_{\mathrm{jet}}^2$. Therefore in the massless case we generate all analytic functions of $m_{\mathrm{jet}}^2$, which is exactly what we expect based on the results of the previous section. Lastly, returning to the example from the last section, $d_{12} d_{23}d_{31}$, it can only be interpreted as an IR-safe polynomial if it corresponds to the triangle graph. However, such a graph is not loop-saturated, and therefore this polynomial is not IRC-safe despite being C-safe for $N=3$ (which can be confirmed by observing that for $N=4$ the triangle graph generates a non-C-safe polynomial), and the correct IRC-safe completion is simply $m_{\mathrm{jet}}^6$.


\subsection{Quantifying IRC-safety}

Let $f^{(N)}$ be a Lorentz-invariant permutation-symmetric observable defined for varying number $N$ of inputs, for instance the output of a PELICAN instance. We define the following difference operator that measures the obstruction to $f$ being IR-safe:
%-------------------
\begin{equation}
    \Delta^{\mathrm{IR}} f^{(N)} = f^{(N+1)}\left(p_1,\ldots,p_N,0\right)- f^{(N)}\left(p_1,\ldots,p_N\right).
\end{equation}
%-------------------
IR-safety is equivalent to the vanishing of this operator:
\[\mathrm{IR-safety:}\quad \Delta^{\mathrm{IR}} f=0.\]
Furthermore, we can define \textit{IR-robustness} as the partial derivative corresponding to the injection of an infinitesimally soft but non-zero 4-momentum:
%-------------------
\begin{equation}
    \mathrm{IR-robustness:}\quad R^{\mathrm{IR}}_p f^{(N)}=\frac{\dd}{\dd\epsilon}{\epsilon=0} f^{(N+1)}(p_1,\ldots,p_N,\epsilon p).
\end{equation}
%-------------------
All together, we have the following series expansion for a general observable near $p_{N+1}=0$:
\[f^{(N+1)}=f^{(N)}+\Delta^{\mathrm{IR}}f^{(N)}+\left(R^{\mathrm{IR}}_{p_{N+1}} f^{(N)}\right)\cdot p_{N+1}+\mathcal{O}(p_{N+1}^2).\]

An IR-safe observable must always have $\Delta^{\mathrm{IR}}f=0$, and lower values of IR-robustness indicate lower sensitivity to soft particles. In practice, we can evaluate the ratio $(\Delta^{\mathrm{IR}}f^{(N)})/f^{(N)}$ and average it over a testing dataset to quantify the IR-safety of PELICAN.

For C-safety, we pick a massless 4-vector $p$ and define the differential operator
%-------------------
\begin{equation}
    D^{\mathrm{C}}_{p} f=\frac{\dd}{\dd\lambda}{\lambda=0}f(p_1+\lambda p,p_2-\lambda p,p_3,\ldots).
\end{equation}
%-------------------
C-safety then is equivalent to the requirement that $D^{\mathrm{C}}_{p} f=0$ for any massless $p$ whenever $p_1$ and $p_2$ are collinear with $p$:
%-------------------
\begin{equation}
    \mathrm{C-safety: }\quad \restr{D^{\mathrm{C}}_{p} f}{p_1\parallel p_2\parallel p}=0, \quad p^2=0.
\end{equation}
%-------------------

\textit{C-robustness} can be easily defined as the second-order analog of $D^{\mathrm{C}}_{p}$. Namely, we pick a second vector $\hat{p}$ such that $p\cdot \hat{p}=0$ (which means that $\hat{p}$ is tangent to the light cone at $p$) and measure how quickly $D^{\mathrm{C}}_{p} f$ deviates from zero as we deform $p_1$ and $p_2$ away from collinearity:
%-------------------
\begin{equation}
    \mathrm{C-robustness: }\quad R^{\mathrm{C}}_{\hat{p},p}f=D^{\mathrm{C}}_{\hat{p}} D^{\mathrm{C}}_{p} f, \quad \hat{p}\cdot p=p^2=0, \quad p_1\parallel p_2\parallel p.
\end{equation}
%-------------------

It is instructive to express these quantities in terms of the dot products $d_{ij}$. We have
%-------------------
\begin{equation}
    D^{\mathrm{C}}_{p}f=\sum_{j=1}^N \left(\partial_{1j}f-\partial_{2j}f\right)(p_j\cdot p),
\end{equation}
%-------------------
where $\partial_{kl}$ indicates the partial derivative with respect to $d_{kl}$. Note that individual partial derivatives aren't really well-defined due to the fact that  $\{d_{ij}\}$ is not a set of independent coordinates on the manifold of $N$ 4-vectors for $N\geq 5$. However, combinations such as above are valid since they are nothing but a different way of expressing the original well-defined operator. We also immediately notice that if $N=2$ and $p_1\parallel p_2\parallel p$, then $D^{\mathrm{C}}_{p}f=0$ automatically. Indeed, every Lorentz-invariant observable for two particles is C-safe due to the fact that $d_{12}=\frac12 (p_1+p_2)^2$ for massless inputs.

C-robustness is similarly expressed by
%-------------------
\begin{equation}
    R^{\mathrm{C}}_{\hat{p},p} f=\sum_{i,j=1}^N\left(\partial^2_{1i,1j}+\partial^2_{2i,2j}-\partial^2_{1i,2j}-\partial^2_{1j,2i}\right)f\cdot (p_i \cdot \hat{p})(p_j\cdot p).
\end{equation}
%-------------------

