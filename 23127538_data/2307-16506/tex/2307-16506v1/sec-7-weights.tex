Despite the output of the PELICAN regression model ostensibly being a 4-vector (or multiple 4-vectors), the richer and more natural object to treat as the output are the PELICAN weights $\{c_i\}$ introduced in (Eq.~\ref{def pelican weights}). Each $c_i$ is attached to its corresponding input constituent $p_i$ due to permutation equivariance and therefore encodes \textit{a scalar feature of that particle within the event}. As we will show in this section, the behavior of these weights is key to the unique explainability and visualization features of the PELICAN architecture.  

In essence, PELICAN is able to take a set of $N$ input 4-vectors and assign $N$ scalar features to them (of course there can be several features per input as well) in a Lorentz-invariant way. This can be powerful in a variety of applications, but in the context of particle reconstruction the problem of finding the right values of the weights is similar to a soft clustering problem. Assuming an idealized dataset with perfect information about the decay products, the model should identify the decay products of the $W$-boson, assign $c_i=1$ to them, and zero to all other constituents. This is analogous to what taggers like the Johns Hopkins top-tagger aim to do via jet clustering. However, since any five 4-vectors are linearly dependent, there is a continuum family of solutions $\{c_i\}$ and it is not clear that PELICAN will prefer the clustering solution.

%===========================================
% Figure environment removed
%===========================================

%------------------------
% MODEL COMPLEXITY
%------------------------
\subsection{Distributions of PELICAN weights} 
%\paragraph{Distributions of PELICAN weights} 

In Fig.~\ref{weights_parent_6W} we display the distributions of all PELICAN weights for models from Section~\ref{Wreco} trained targeting $p^W_{\mathrm{true}}$. We also mark each constituent as either a $W$- or a $b$-daughter. This yields several observations. 

Firstly, nearly all weights are either non-negative or very slightly negative (e.g.~above $-0.1$) with a very sharp peak at zero (the peak is entirely to the left of zero to very high precision\footnote{The bin $[-10^{-6},0)$ contains about 100 times more constituents than the bin $[0,10^{-6})$.}). This is the first feature that justifies the interpretation of PELICAN as a \textit{soft clustering} method. Since our inputs represent realistic events, all input 4-vectors in them are causally related, and in particular they belong to the future light cone, as does the target vector. This implies that no linear combination of these vectors with positive coefficients can produce a zero vector. The distributions, therefore, show that PELICAN weights assigned to $b$-daughters are not ``contaminated'' with these degenerate combinations.

Secondly, the truth-level distribution is highly concentrated at $0$ and $1$ and very closely matches the binary clustering solution. That is, almost all constituents assigned weight $0$ are $b$-daughters, and almost all of those assigned $1$ are $W$-daughters. Nevertheless, $30\%$ of $b$-daughters are assigned positive weights, prompting further investigation. Moreover, the distribution of $W$-daughter weights in the \Delphes case is so spread out that it becomes difficult to explain it by a mere analogy with clustering.

%===========================================
% Figure environment removed
%===========================================

%===========================================
% Figure environment removed
%===========================================

%===========================================
% Figure environment removed
%===========================================

We can delve more deeply into the weight distribution by evaluating the sub-populations of weights based on jet containment. Fig.~\ref{weights_event_6W} shows the distributions of weights for $bqq$, $qq$, and non-FC events. The majority of constituents at the high end of the weight scale belong to non-FC events. Similarly, the weights produced by the models trained targeting $p^W_{\mathrm{cont}}$, shown in Fig.~\ref{weights_parent_6D}, are more highly concentrated at $0$ and $1$, and have much lower and shorter ``tails'' on the right, especially among $b$-daughters. This is the first indication that PELICAN tends to up-weight some constituents in events where it doesn't have enough information for an accurate reconstruction.

This approach allows to characterize the constituents that are being up-weighted. Fig.~\ref{weights_pT_6W} shows the constituent weights as a function of the constituent's $p_T$. The main observation here is that among high-energy (``hard'') constituents with $p_T>100\text{ GeV}$ the weight distribution is much more binary, and the vast majority of constituents with weights falling away from the two peaks are soft, below $20\text{ GeV}$. In the \Delphes case PELICAN appears to down-weight high-energy $W$-daughters and up-weight soft constituents. Once again, loss of information in the form of detector effects appears to lead to PELICAN up-weighting soft constituents.


%------------------------
% DETECTOR EFFECTS
%------------------------
\subsection{Detector effects on PELICAN weights} 
%\paragraph{Detector effects on PELICAN weights} 

While the truth-level PELICAN models reliably converge to a binary clustering solution, the weights in the \Delphes case do not permit such a straightforward interpretation. To better understand their behavior, we ran additional experiments using custom datasets that exclude different components of the \Delphes detector simulation one by one. \Delphes performs the following steps: simulate the effect of the magnetic field $B_z$ on charged final-state particles; aggregate truth-level particle energies within each electromagnetic calorimeter (ECAL) and hadronic calorimeter (HCAL) detector cell; apply energy smearing by sampling a lognormal distribution; unify the ECAL and HCAL cells; apply spatial smearing by picking a uniformly random point within the detector cell; construct the spatial momentum so that the resulting 4-vector, which represents a detector cell, is massless. We found that while each of these steps contributes to smearing out the truth-level distribution of PELICAN weights and shifting the peak downwards, the magnetic field is responsible for almost all of the differences between truth and \Delphes results.

%===========================================
% Figure environment removed
%===========================================

The simulated magnetic field is able to deflect charged particles very significantly, enough to account for most of the error in PELICAN's reconstruction relative to the truth-level reconstruction. Our hypothesis for why this leads to lower PELICAN weights for hard constituents is the following. Deflected hard particles produce large errors in the direction but not the energy of the reconstruction, therefore one can down-weight them and compensate for the energy deficit using softer constituents. Moreover, by up-weighting softer constituents PELICAN can in fact partially correct the error in the direction since the deflections of positively charged particles can be partially cancelled out by those of negatively charged particles.

%===========================================
% Figure environment removed
%===========================================

An extra piece of evidence in support of this hypothesis can be found by modifying the loss function. If we re-train the model on the same \Delphes dataset using a loss function consisting of a single energy term $|E_{\mathrm{reco}}-E_{\mathrm{true}}|$, we find a distribution of weights (see Fig.~\ref{lossE_weights}) nearly as bimodal as the original one trained on truth-level data (see Fig.~\ref{weights_parent_6D}). This indicates that the source of the error in PELICAN's reconstruction on \Delphes data is overwhelmingly \textit{spatial}. Out of all the steps that \Delphes performs, only two are purely spatial: momentum smearing within one cell, and the simulated magnetic field. However, the detector cells (approximately $0.02\times 0.02$ in $(\eta,\phi)$) are much smaller than the magnitude of PELICAN's typical angular error, and thus smearing cannot explain the error. 

%{\textcolor{red}{might add concrete results for a model trained on data simulated without the magnetic field}}

%===========================================
% Figure environment removed
%===========================================

%------------------------
% EVENT VISUALIZATION
%------------------------
\subsection{Event visualization} 
%\paragraph{Event visualization} 

As we discussed above, despite being a single-vector regression model, PELICAN produces one feature \textit{per input constituent} (namely the weight $c_i$), and these features become interpretable by virtue of Eq.~\ref{def pelican weights}. This gives us a unique opportunity to make event-level visualizations that provide insight into how PELICAN treats jet topology and how it compares to conventional methods such as the JH tagger's jet clustering.

In Fig.~\ref{multi_display} we show an amalgamation of 200 events from the \Delphes dataset from Section~\ref{Wreco} projected onto the unit sphere. Each event was spatially rotated so that the position of the true $W$ within the image is fixed and the true $b$-quark is located in the negative $\phi$ direction. In one display the constituents are colored according to their parent being either the $W$ boson or the $b$-quark, and in the other they're colored based on their assigned PELICAN weight. The correlation between the two images is clear: $b$-daughters tend to be correctly assigned zero weight, whereas $W$-daughters have positive weights with the hardest constituents having weights between $0.4$ and $0.8$.

In Fig.~\ref{display_energy} we show a single event in the $(\eta,\phi)$ plane, with dot color and size dependent on the constituent energy. Note the reduced number of constituents in the \Delphes display, and how some of the constituents get strongly deflected by the simulated magnetic field. The same event can be visualized in three more helpful ways. In addition to parent type and PELICAN visualizations introduced in Fig.~\ref{multi_display}, we can also extract the list of constituents that the JH tagger identifies as belonging to the $W$ boson and highlight them. Fig.~\ref{event_display} displays the same single event in all three ways. In addition, we add a special marker for the direction of the reconstructed $W$ boson. In the parent type pane, this reconstruction is defined as $\sum_{i=1}^N r_i p_i$ where $r_i$ is the energy of the true $W$-daughters within that constituent divided by the actual energy of the constituent. In the JH and PELICAN panes, the marker corresponds to the corresponding reconstructions obtained by those methods.

%===========================================
% Figure environment removed
%===========================================