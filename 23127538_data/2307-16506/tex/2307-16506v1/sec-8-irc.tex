Perturbative computations in QCD suffer from a divergence caused by two types of processes: soft emission and collinear splittings. As a consequence, meaningful observables in this theory need to be insensitive to such processes, and this requirement is known as IRC-safety. In this section we provide a precise definition, give a characterization of IRC-safe Lorentz-invariant observables (see details in Appendix~\ref{appendix_irc}), and describe modifications to the PELICAN architecture that make it IR-safe or IRC-safe.

Infrared safety (IR-safety) guarantees insensitivity to soft emissions, i.e.~particles with relatively low energies and momenta. A family of continuous symmetric observables $f^{(N)}(p_1,\ldots,p_N)$ is said to define an IR-safe observable $f$ if 
\[\lim_{\epsilon\to 0}f^{(N+1)}(p_1,\ldots,p_N,\epsilon p)=f^{(N)}(p_1,\ldots,p_N)\]
for any $N$ and any $p_1,\ldots,p_N,p$, where $\epsilon$ controls how infinitesimally small the considered soft emission $p$ is.

Collinear safety (C-safety) is a restriction on observables in perturbative QCD that arises from the divergent contributions of collinear emissions of gluons. Positive gluon mass would prevent such divergences, which is why C-safety concerns only massless particles. We can define C-safety formally as follows: an observable $f(p_1,\ldots,p_N)$ is C-safe if, whenever two massless 4-momenta $p_1$ and $p_2$  become collinear (which happens for massless particles iff $p_1\cdot p_2=0$), the value of $f$ depends only on the total momentum $p_1+p_2$. Expressed even more explicitly, C-safety says that setting $p_1=\lambda p$ and $p_2=(1-\lambda)p$ with some 4-vector $p$ such that $p^2=0$ must lead to the same output regardless of the value of $\lambda$, i.e.
\[C_{12}(p)f=\partial_\lambda f(\lambda p,(1-\lambda)p, p_3,\ldots,p_N)=0.\label{25773}\]

In Appendix~\ref{appendix_irc} we characterize IRC-safe Lorentz-invariant observables, but the following summary will suffice for the purpose of designing an IRC-safe version of PELICAN. First, a Lorentz-invariant observable (assumed to be consistently defined for any finite number $N$ of 4-vector inputs) is IR-safe if and only if it has no explicit dependence on the multiplicity $N$. More precisely, adding the zero 4-vector to the list of inputs should leave the output value invariant. Second, an IRC-safe Lorentz-invariant observable is one that is IR-safe and moreover depends on any of its massless inputs only through their total. E.g.~if $p_1,p_2,p_3$ are fixed to be massless, then $f(p_1,p_2,p_3,p_4,\ldots)$ must depend only on $p_1+p_2+p_3$. In particular, if all inputs are massless, then all IRC-safe invariant observables are functions of only the jet mass $M^2=\left(\sum_{i} p_i\right)^2$. Note, however, that such an observable can still depend on these vectors in an arbitrarily complex fashion away from the massless manifold.

The original PELICAN architecture as introduced above is neither IR- nor C-safe. Below we modify the architecture to make it exactly IR-safe or IRC-safe and evaluate the implications.

%------------------------
% IR-SAFE PELICAN
%------------------------
\subsection{IR-safe PELICAN} 
%\paragraph{IR-safe PELICAN}

As shown above, IR-safety in Lorentz-invariant networks essentially requires the outputs to be independent of the multiplicity $N$. There are four ways in which the multiplicity shows up in PELICAN:
%===========================================
\begin{enumerate}
    \item Scaling with $N^\alpha/\bar{N}^\alpha$ in the equivariant block. This must be disabled for IR-safety.
    \item Non-zero bias values in linear layers. Since the network is permutation-equivariant, the bias values are shared across jet constituents, which means that upon aggregation in the following equivariant layer they contribute multiples of $N$. All biases in all linear layers must be disabled for IR-safety. 
    \item The input embedding must map zero to zero, but our original choice already satisfies this. In addition, the activation function must also have a fixed point at zero. Our default choice, \texttt{LeakyReLU}, also satisfies this.
    \item Following an application of a PELICAN equivariant block, rows and columns corresponding to soft constituents will  contain a combination of sums over all constituents. Even in the absence of biasing constants, this effectively increases the multiplicity with which these values enter in the later aggregations. This can be resolved by making sure that rows and columns that are soft at the input remain soft throughout the whole network. Therefore we introduce \textit{soft masking}, whereby the last 12 equivariant aggregators (those don't preserve the softness of rows/columns) are followed by a multiplication by the vector of values $J\cdot p_i$, where $J=\sum_{i=1}^N p_i$, scaled and clipped to be within $[-1,1]$. In $\mathrm{Eq}_{2\to 2}$ this multiplication is applied both row-wise and column-wise, and in $\mathrm{Eq}_{2\to 1}$ it's component-wise. 
\end{enumerate}
%===========================================
With these modifications, PELICAN becomes IR-safe. As we will see, this restriction leads to a modest reduction in the performance of PELICAN's predictions in our tasks of interest.

%------------------------
% IR-SAFE PELICAN
%------------------------
\subsection{IRC-safe PELICAN}
%\paragraph{IRC-safe PELICAN} 

Adding C-safety to the architecture is much simpler. As stated above, the necessary requirement is that the output depend on massless inputs only through their sum. In PELICAN this can be achieved by inserting a linear permutation-equivariant layer with a mass-based soft mask immediately at the input (any nonlinear embedding has to be done later). Consider a case where $p_1,p_2$ are massless and the dot product matrix $\{d_{ij}\}$ is fed into such an equivariant layer. Most of the aggregators will compute sums over rows or columns, thus immediately producing C-safe quantities. However, several of the aggregators, including the identity, will preserve individual information about each $p_i$, therefore their output rows and columns corresponding to $p_1$ and $p_2$ need to be thrown out. This can be done by a soft mask that turns to zero as the mass of any input goes to zero. This mask is defined in the same way as the IR mask except using $m_i^2$ instead of $J\cdot p_i$. It needs to be applied only to the first 2 order zero and the first 7 order one aggregators. 

Coincidentally, this soft mask can also be used in place of an IR mask, which means that we only need the C-safe soft mask to make a fully IRC-safe PELICAN architecture. Altogether it gets applied to all equivariant aggregators except the third one (which extracts the diagonal and is thus IRC-safe by definition).

%------------------------
% IR-SAFE PELICAN
%------------------------
\subsection{Testing IR/C-safe PELICAN models} 
%\paragraph{Testing IR/C-safe PELICAN models} 

First we quantify the deviation in PELICAN's outputs that occurs under soft and collinear splittings and observe how training affects them. We define an IR-splitting as adding a zero 4-vector to the list of input constituents. Then PELICAN's output on IR-split data is directly compared to the original output. Defining a C-splitting is more difficult since realistic events never contain any exactly collinear constituents, and we want to avoid changing the number of particles so as to make this test independent of IR-safety. Therefore we prepare the data by inserting two copies of the vector $(1,0,0,1)$ to each event. Then the C-splitting will amount to replacing these two vectors with $(2,0,0,2)$ and $(0,0,0,0)$. The outputs on the same event prepared in these two ways can be directly compared.

To compare two outputs $p_{\mathrm{reco}}, p_{\mathrm{reco}}'$ we compute the relative deviation $|(p_{\mathrm{reco}}'-p_{\mathrm{reco}})/p_{\mathrm{reco}}|$, where the division is component-wise. To estimate the effect of an IR- or C-splitting on PELICAN's predictions, we take the median value of this deviation over a batch of events. The same can also be done with PELICAN weights as the outputs. The splittings are applied to 100-event batches of events from one of our datasets and the median deviations are averaged over 300 batches. We test 5 randomly-initialized models and 5 models trained on the full variable $W$ mass dataset from Section~\ref{Wmass}.

We find that a randomly-initialized PELICAN regression model's output 4-vector deviates by 0.5\%-7\% under an IR-split, and the PELICAN weights deviate by up to 5\%. However, regression models trained to reconstruct $p^W_{\mathrm{cont}}$ have both deviations under 5\%, potentially indicating a slight improvement in IR-safety due to training. The resolutions $\sigma_{p_T}, \sigma_m$, and $\sigma_{\Delta R}$ of the trained IR-safe truth-level models are about 20\%-35\% worse (larger) than the original models, and similarly the \Delphes resolutions get 40\%-50\% worse. We also note that IR-safe PELICAN models appear to be slightly more rigid under C-splits, showing 5\%-16\% deviations that enhance slightly to at most 11\% after training.

Under a C-splitting, the randomly-initialized regression model's outputs (both the 4-vector and the weights) consistently deviate by 3\%-8\%, and the same range of deviations was observed on fully trained models as well. The resolutions of trained IRC-safe truth-level models suffer significantly in comparison to the regular models, exhibiting 5-6 times higher values of $\sigma_{p_T}, \sigma_m$, and $\sigma_{\Delta R}$. We do not perform this comparison for \Delphes models since the jet constituents coming out of \Delphes are massless, so only functions of the jet mass are expressible by IRC-safe PELICAN on that data.