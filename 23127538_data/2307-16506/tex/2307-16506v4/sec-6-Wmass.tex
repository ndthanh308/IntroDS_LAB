As we saw above, PELICAN is able to reconstruct the mass of the $W$-boson, $m_W$, found within the dense environment of the complete decay products of a top quark jet. For truth-level datasets, the resolution of this reconstruction is below the natural width ~\cite{ParticleDataGroup:2022pth} of the mass spectrum, $\Gamma_W/m_W\approx 2.59\%$. In the \Delphes case, the resolution is too wide to produce any substantial correlation between the true and reconstructed masses (see \appref{appendix_plots} for figures that demonstrate this). We would like to eliminate the possibility that the reason that the true masses are highly concentrated around $80$ GeV is due in part to the potential for PELICAN to effectively \textit{memorize} a single number: the $W$ mass. In this section we examine a more realistic reconstruction task, where the true mass of the target particle is unknown, and the dataset uniformly covers a wide range of its masses.


The reconstruction task is still identical to that of \secref{Wreco}. Even though we could use an invariant scalar-valued version of PELICAN to target the mass of the $W$-boson, the accuracy of that reconstruction would in fact suffer in comparison with the equivariant 4-vector-valued model. This is simply due to the fact that the 4-momentum contains more relevant information than the mass alone, since the direction and the energy of the particle are, in part, correlated with the mass. Thus the only new element in this experiment will be the dataset, which will now involve $W$-bosons of varying masses uniformly covering the range $m_W\in [64,96]$~GeV.
%
%===========================================
% Figure environment removed
%===========================================
%
%------------------------
% DATASET
%------------------------
%\subsection{Regression dataset} 
%
The dataset is also identical to that used in \secref{Wreco}, including the number of events, except that the $W$ mass is set to be variable. This is achieved by combining multiple independently-produced datasets where the generator-level value of $m_W$ was modified from its default value. \Figref{m_targetm} shows the resulting distribution of $W$ masses, as well as that of the sum of $W$-daughters contained within each jet.

%===========================================
\begin{table}[t]
    % \vspace{-1\intextsep}
    \centering
    \begin{small}
        \begin{tabular}{ccS[table-format=3.2]<{\%}S[table-format=3.2]<{\%}S[table-format=3.3]}
        \toprule
        & Method &  \multicolumn{1}{c}{$\sigma_{p^T}$ (\%)} & \multicolumn{1}{c}{$\sigma_{m}$ (\%)} & \multicolumn{1}{c}{$\sigma_{\Delta R}$ (centirad)}\\
        \midrule
        \multirow{3}{*}{\rotatebox[origin=c]{90}{\parbox{1.3cm}{\centering Without\\ \Delphes}}} 
        & JH               & 7.98    & 4.75     & 22.180   \\
        & PELICAN$\mid$JH  & 0.27    & 0.63     & 0.111   \\
        & PELICAN$\mid$FC  & 0.35    & 0.89     & 0.143   \\
        & PELICAN          & 2.64    & 39.00    & 0.744   \\
        \midrule
        \multirow{3}{*}{\rotatebox[origin=c]{90}{\parbox{1.3cm}{\centering With\\ \Delphes}}} 
        & JH               & 16.0    & 12.0   & 25.4      \\
        & PELICAN$\mid$JH  & 4.2     & 6.5  & 3.4      \\
        & PELICAN$\mid$FC  & 4.9     & 8.0   & 3.8      \\
        & PELICAN          & 7.3     & 40.7   & 6.7      \\
        \bottomrule
        \end{tabular}
    \end{small}
    \caption{PELICAN resolutions for models trained to reconstruct $p^W_{\mathrm{cont}}$ with variable $m_W$. Resolutions are still obtained by comparing the model predictions to $p^W_{\mathrm{true}}$.\label{btW6m_D_table}}
    \vspace{-0.5\intextsep}
\end{table}
%===========================================

%------------------------
% RESULTS
%------------------------
\subsection{Regression results for $m_W$ reconstruction}
%\paragraph{Results} 

The hyperparameters and the training regime used here are the same as in \secref{Wreco}. Here we focus on the model trained to reconstruct the contained momentum $p^W_{\mathrm{cont}}$ (see \appref{appendix_plots} to find the results for the model targeting $p^W_{\mathrm{true}}$). The outputs are then compared to the true $W$-boson $p^W_{\mathrm{true}}$. The accuracies for the full 4-vector reconstruction are presented in \tabref{btW6m_D_table}. The largest loss of accuracy relative to \secref{Wreco} is, unsurprisingly, in the mass column. However, since the true mass now covers a much wider range, while the number of training samples remained the same, this still presents a significant improvement in the mass reconstruction capability. To demonstrate this better, we show the 2D correlations between target and reconstructed masses in \figsref{m_corr_mWW}{m_corr_mDD} for the models trained targeting $p^W_{\mathrm{true}}$ and $p^W_{\mathrm{cont}}$, respectively. We also differentiate between non-FC (left) and FC (right) events in the two sides of each of the panels in each figure. 

%===========================================
% Figure environment removed
%===========================================

%===========================================
% Figure environment removed
%===========================================



%------------------------
% MODEL COMPLEXITY
%------------------------
\subsection{Model complexity}

The model examined above has 210k trainable parameters, however even significantly smaller models achieve good accuracy. As an illustration, we compare the resolutions of three PELICAN models trained on the variable mass dataset targeting $p^W_{\mathrm{true}}$. They are obtained from the original model by a proportional rescaling of the widths of all layers. The first model is the 210k parameter one, with 132/78 channels, i.e.\ each messaging layer has 132 input and 78 output channels. The second model has 60/35 channels and 49k parameters. The third model has 25/15 channels and 11k parameters. The resolutions over the \Delphes test dataset are reported in \tabref{model_size_table}, and we observe that even the 11k-parameter model handily beats the JH method.

%===========================================
\begin{table}[t]
    % \vspace{-1\intextsep}
    \centering
    \begin{small}
        \begin{tabular}{c *{3}{S[table-format=3.1]<{\%}} c}
        \toprule
        PELICAN width &  \multicolumn{1}{c}{$\sigma_{p_T}$ (\%)} & \multicolumn{1}{c}{$\sigma_{m}$ (\%)} & \multicolumn{1}{c}{$\sigma_{\Delta R}$ (centirad)} & \# Params\\
        \midrule
        132/78 & 6.1    & 8.2   & 2.8  & 210k    \\
        60/35 & 6.5     & 8.6  & 3.2   & 49k     \\
        25/15 & 7.4     & 9.5   & 3.8  & 11k    \\       
        \bottomrule
        \end{tabular}
    \end{small}
    \caption{Comparison of PELICAN models of three different widths trained to reconstruct $p^W_{\mathrm{true}}$ with variable $W$ mass. Width is defined as in \tabref{tab_pelican_model_size}. Trained and tested on \Delphes data.\label{model_size_table}}
    \vspace{-0.5\intextsep}
\end{table}
%===========================================


%------------------------
% DISCUSSION
%------------------------
\subsection{Discussion}

In the \Delphes dataset, we observe that for non-FC events (bottom left pane of \figref{m_corr_mDD}), the reconstructed contained mass is only weakly correlated with the true contained mass (or with the true $W$ mass, as shown in \figref{m_corr_DW} in \appref{appendix_plots}). However, in the quadrant where both masses exceed $55$ GeV, we find a $65\%$ correlation on FC events in the \Delphes case. The most important type of error PELICAN makes here is when a non-FC event gets assigned a high reconstructed mass, that is a mass near that of the true $W$-boson was assigned to a jet with few of the $W$ decay products in it. Among all events with $m_{\mathrm{reco}}>55\text{ GeV}$, $3.6\%$ are non-FC, and they bring the correlation among that population down to $51\%$ ($p_T$, mass, and angular resolutions on this population closely track those of PELICAN$\mid$FC above). But since in practice we're interested in $m^W_{\mathrm{true}}$, the correlation between that and $m_{\mathrm{reco}}$ is higher, at $59\%$ among events with $m_{\mathrm{reco}}>55\text{ GeV}$. This is a significant improvement over the model trained on the original $m^W_{\mathrm{true}}\sim 80\text{ GeV}$ \Delphes dataset, and especially over non-ML methods such as the JH tagger (see \figref{m_corr_mJH}). 

Therefore a workflow that guarantees both high background rejection and high reconstruction quality would involve first using a model trained on $p^W_{\text{cont}}$ as a classifier to filter out well-contained events, and then using a model trained on $p^W_{\text{true}}$ to obtain a high-precision reconstruction on that population. However, even a model trained on \Delphes data to reconstruct $p^W_{\mathrm{true}}$, in fact, achieves a $40\%$ correlation with $m^W_{\mathrm{true}}$ on non-FC events (see \figref{m_corr_mWW}), so FC-tagging may not always be necessary. Overall, PELICAN provides a viable method for estimating Lorentz-invariant particle masses.


%===========================================
% Figure environment removed
%===========================================