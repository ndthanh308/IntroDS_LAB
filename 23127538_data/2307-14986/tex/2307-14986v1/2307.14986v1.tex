\documentclass[11pt,twoside]{article}

%%%%%%% Packete von Kussmaul fuer Deutsche Zeichen %%%%%%%%%%%%%%%
%\usepackage[utf8]{inputenc}
%\usepackage[T1]{fontenc}
%\usepackage[english]{babel}
%\usepackage{lmodern}
%\usepackage[pdftex]{graphicx}
%\usepackage{latexsym}
%%\numberwithin{equation}{section}
%\usepackage{graphicx}
%\graphicspath{ {./images/} }
%\usepackage{tikz}
%\usepackage{subcaption}
%\usepackage{float}
%\usepackage{comment}
\usepackage{mathtools} %Für := (\coloneqq)
%\usepackage{hyperref}

\usepackage{mathrsfs} % used for mathscr

\newcommand{\FF}{\mathcal{F}}
\newcommand{\HH}{\mathscr H}
\newcommand{\LL}{\mathscr L}

%%%%%%%%%%%%%%%%%%%%%%%%%%%%%%%%%%%%%%%%%%%%%%%%%%%


%\usepackage[ansinew]{inputenc} %%Siehe UTF-8
\usepackage{amsmath,amssymb,amsthm}
\usepackage{pstricks,pst-node,pst-coil,pst-plot,pstricks-add}
\usepackage{geometry,epsfig}
\usepackage{bbm}
\usepackage{cite}
%\usepackage{graphicx}
%\usepackage{thmbox}

\usepackage{cite}
\usepackage{paralist}
\usepackage{cleveref}
\usepackage{enumitem}
%\usepackage{menukeys}
\usepackage{emptypage}


\DeclareGraphicsExtensions{.pdf}
%\usepackage{floatflt}
%\setlength{\parindent}{0mm}

\setlength{\oddsidemargin}{7mm} \setlength{\evensidemargin}{7mm}
\setlength{\topmargin}{-10mm} %\setlength{\topmargin}{10mm}
\setlength{\textheight}{9in} \setlength{\textwidth}{6in} % originally 6.2in
\setlength{\topsep}{0.2in}
%\setlength{\parskip}{1ex}


%%%%%%%%%%%%%%% Schriften %%%%%%%%%%%%%%%%%%%%%%%%%%%%%%
%\usepackage[T1]{fontenc}
%\newcommand{\changefont}[3]{
%\fontfamily{#1} \fontseries{#2} \fontshape{#3} \selectfont}


%%%%%%%%%%%%%%%%   KUSSMAUL  %%%%%%%%%%%%%%%%%%%%%%%%%


\newtheorem{Lemma}{Lemma}[section]
\newtheorem{Theorem}[Lemma]{Theorem}
\newtheorem{Corollary}[Lemma]{Corollary}
\newtheorem{Proposition}[Lemma]{Proposition}


\newcommand{\C}{\mathbb{C}} % komplexe
\newcommand{\K}{\mathbb{K}} % komplexe
\newcommand{\R}{\mathbb{R}} % reelle
\newcommand{\Q}{\mathbb{Q}} % rationale
\newcommand{\Z}{\mathbb{Z}} % ganze
\newcommand{\N}{\mathbb{N}} % natuerliche
\DeclareMathOperator*{\esssup}{ess\,sup}

%\newcommand{\begin{proof}}{\textit{Proof. }}	%für Beweise	
%\newcommand{\end{proof}inalign}{\tag*{$\blacksquare$}} %eproof in align Umgebung nach rechts
%\newcommand{\end{proof}}{\hfill $\blacksquare$}	

\DeclareRobustCommand{\rchi}{{\mathpalette\irchi\relax}}
\newcommand{\irchi}[2]{\raisebox{\depth}{$#1\chi$}} %richtiges Chi
\newcommand{\intd}{\int \hspace{-1.5mm} d} %\int dx richtig plaziert
\newcommand{\all}{\quad \text{for all} \: \,} %Abkürzung von für alle
\newcommand{\Ima}{\operatorname{Im}}
\newcommand{\Rea}{\operatorname{Re}}

\renewcommand{\baselinestretch}{1.1}
\renewcommand{\qedsymbol}{\rule{1.3mm}{2.6mm}}

%---------------------------------------------------------------------------------------------------------------------
%\font\notefont=cmsl8 \pagestyle{myheadings}
%\markright{\notefont No contact interactions for fermions, M. Griesemer, M. Hofacker -- November 16, 2021.\hfill}
%---------------------------------------------------------------------------------------------------------------------

\title{\textbf{Pointwise bounds on eigenstates in non-relativistic quantum field theory}}

\author{M.~Griesemer\footnote{marcel.griesemer@mathematik.uni-stuttgart.de}\,\ and V.~Ku{\ss}maul\footnote{valentin.kussmaul@mathematik.uni-stuttgart.de}\\  
\small Fachbereich Mathematik, Universit\"at Stuttgart, D-70569 Stuttgart, Germany}  
\date{}

\begin{document}
\maketitle

\begin{abstract}
In this paper we establish subsolution estimates for vector-valued Sobolev functions obeying a very mild subharmonicity condition. 
Our results generalize and improve a well known subsolution estimate in the scalar-valued case, and, most importantly, they apply to models from non-relativistic quantum field theory: for eigenstates of the Nelson and Pauli-Fierz models we show that an $L^2$-exponential bound in terms of a Lipschitz function implies the corresponding pointwise exponential bound.      
\end{abstract}

%\tableofcontents

\section{Introduction}

States of an atom or molecule with energy distribution strictly below the ionization threshold are well localized in a neighborhood of the nuclei: both in models of non-relativistic quantum mechanics and non-relativistic quantum field theory (QFT), the wave function decays exponentially as the distance $|x|$ of the electronic configuration $x=(x_1, \ldots,x_N)\in \R^{3N}$ from the nuclear positions grows. This decay implies an effective screening of the positive nuclear charges and it plays an important role in the mathematical analysis of many-particle quantum systems \cite{LT1986, AnaSig,AnaLew}. While this exponential decay is well understood and well established in non-relativistic quantum mechanics \cite{A}, in atomic models from non-relativistic quantum field theory we know little more than an averaged decay of the form 
\begin{equation}\label{L2-decay}
 \int e^{2 (1-\varepsilon) \beta |x|}|\psi(x)|^2\, dx <\infty,\qquad \text{all}\ \varepsilon>0,
\end{equation}
for the wave function $\psi:\R^{3N}\to \HH'$ with $|\psi(x)|$ the norm of $\psi(x)\in\HH'$, $\HH'$ being the tensor product of spin and Fock space \cite{G}. The rate of decay, $\beta>0$, is explicit and depends on the difference between ionization threshold and upper bound on the energy distribution of $\psi$. One expects that \eqref{L2-decay} implies the corresponding \emph{pointwise} decay at the same rate, at least if $\psi$ is an eigenvector, but a general result of this type is not yet known, see \cite{Hiro2019,HiHi2010}. In this paper we provide such a result.

It is well known, that solutions $\psi$ to a Schr\"odinger equation $(-\Delta+V)\psi = E\psi$, or a more general elliptic PDE, under very mild assumptions on the potential $V:\R^n\to \R$ satisfy a Harnack-inequality, or some subsolution estimate
\begin{equation}\label{Agmon}
    \esssup_{y\in B(x,1/2)}|\psi(y)| \leq C\bigg(\int_{B(x,1)}|\psi(y)|^2\, dy\bigg)^{1/2}
\end{equation}
for all $x\in \R^n$ \cite{A,AiSi}. It follows that $L^2$ exponential decay, such as \eqref{L2-decay}, with a Lipschitz function $\beta|x|$ implies the corresponding pointwise decay,
\begin{equation}\label{L-infty-bound}
     \esssup_{x\in \R^{n}} e^{(1-\varepsilon) \beta |x|}|\psi(x)| <\infty, \qquad \text{all}\ \varepsilon>0.
\end{equation}
Here, ``$\esssup$'' could be replaced by a ``$\sup$'' because eigenfunctions to many-particle Schr\"odinger equations are continuous, in fact uniformly H\"older continuous \cite{Kato57}. This property alone has been used  to derive \emph{some} pointwise exponential decay from \eqref{L2-decay}, but the rate so obtained is worse than in \eqref{L2-decay}, see \cite{Ahlrichs}. Our goal is to generalize \eqref{Agmon} to a vector-valued setting with $|\psi(y)|$ the norm of $\psi(y)$ in a Hilbert space $\HH'$.

To explain our work we first review the proof of \eqref{Agmon} in Agmon's book \cite{A}, which is done in a fairly general setting 
for second order elliptic equations. The first step in that proof is to show that $|\psi|$ is a subsolution to the Schr\"odinger equation solved by $\psi$: if 
 $(-\Delta+V)\psi = E\psi$, then, by Kato's distributional inequality,
\begin{equation}\label{mod-sol}
    -\Delta|\psi | \le (V_{-}+E)|\psi|
\end{equation}
where $-V_{+}$ has been dropped. The second step is to derive a functional inequality for $|\psi|$ depending on some exponent $p\geq 2$. Upon iteration of this inequality, in combination with a standard Sobolev-embedding, a bound on the local $L^p$-norm of  $|\psi|$ is obtained that is uniform in $p$ and implies \eqref{Agmon} (de Giorgio-Nash-Moser iteration).
In simple models of QFT, like the Nelson model, the electron-photon interaction acts pointwise in the electronic configuration $x\in \R^n$, and it is small as a form perturbation. This implies that an eigenstate $\psi\in L^2(\R^{n};\HH')$ of the Hamiltonian satisfies 
 \begin{equation} \label{sub-sol}
    \Rea(\psi,-\Delta\psi) \leq q_{-}|\psi|^2,
\end{equation}
where $(\cdot,\cdot)$ and $|\cdot|$ denote the inner product and the norm of the Hilbert space $\HH'$,  
$q_{-} = V_{-} + K$ and $K \in \R$ is some constant. Combining \eqref{sub-sol} with a generalization of Kato's distributional inequality to elements $\psi\in H^2_\mathrm{loc}(\R^n; \HH')$, see \Cref{loc lemma} below, we arrive at an inequality of the type \eqref{mod-sol} and hence \eqref{Agmon} follows from Agmon's work. In more realistic models of QFT, like the Pauli-Fierz model of quantum electrodynamics, we don't have \eqref{sub-sol} for an eigenstate, but we have the weaker bound 
\begin{equation} \label{sub-sol-grad}
    \Rea(\psi,-\Delta\psi) \leq |\nabla\psi |^2 + q_{-}|\psi|^2.
\end{equation}
Our main abstract result shows that \eqref{sub-sol-grad} implies \eqref{Agmon} with the $L^2$-norm on the right hand side of \eqref{Agmon} replaced by an $L^p$-norm and any $p>2$. The remaining step from this $L^p$ norm to an $L^2$ norm is done by an independent argument based on another Sobolev embedding and special properties of the Pauli-Fierz Hamiltonian.
 
The aforementioned results apply to eigenstates $\psi$ of the Nelson and Pauli-Fierz models of atoms and molecules with static nuclei. Concerning the existence of such eigenstates we refer to \cite{BFS,GLL,LL} and the references therein.
We show that $L^2$ exponential decay of the form \eqref{L2-decay} entails the analog pointwise decay \eqref{L-infty-bound}. The same is true if  $\beta |x|$ in \eqref{L2-decay} and \eqref{L-infty-bound} is replaced by an arbitrary Lipschitz function.
If $\psi$ is not an eigenvector but a more general state with energy distribution strictly below the ionization threshold, then the strategy outlined above brakes down and we do not have a subsolution estimate such as \eqref{Agmon}. But we 
can still show, by exploiting the Sobolev embedding $H^2(\R^3)\to L^{\infty}(\R^3)$, that the one-particle density of $\psi$ exhibits pointwise exponential decay at the rate expected from the $L^2$ decay of $\psi$. A fortiori the same remark applies to $|\psi|$ if $N=1$.

Pointwise bounds similar to \eqref{L-infty-bound} (not related to an $L^2$ bound) with some $\beta>0$ were previously obtained by Hiroshima by means of path integral representations for the semi-group generated by the Hamiltonian \cite{Hiro2019,HiHi2010}. 

%------------------------------------------------------------------------------------------------------------------

This paper is organized as follows. In \Cref{Vector-valued functions} we recall the notion of (local) Sobolev spaces of vector-valued functions and we collect generalizations - to this setting - of important rules for weak differentiation such as Kato's distributional inequality. Details are deferred to an appendix of the paper. \Cref{sec:main} contains our main abstract results, the new subsolution estimates derived from  \eqref{sub-sol} and \eqref{sub-sol-grad} for $\psi\in H^2_\mathrm{loc}(\R^n; \HH')$.
The assumption $\psi\in H^2_\mathrm{loc}$ - rather than $\psi\in H^1_\mathrm{loc}$ - simplifies our presentation compared to \cite{A}. \Cref{sec:qft} introduces the Nelson and Pauli-Fierz models and \Cref{sec:qft-exp} contains our main applied results, the pointwise exponential decay of eigenvectors and one-particle densities in these models.

%---------------------------------------------------------------------------------------------------------------------------------------------------------------------

\section{Preparations on vector-valued Sobolev functions}
\label{Vector-valued functions}


In this section we introduce Sobolev spaces of functions with values in a separable Hilbert space $\HH'$. 
There is a more general theory of Sobolev functions with values in a general Banach space \cite{E,K}, but for $\HH'$ a separable Hilbert space, this theory simplifies a lot. \newline

Let $\HH'$ be a separable Hilbert space with inner product $(\cdot, \cdot)$ and norm $| \cdot |$. Let $\Omega\subset \R^n$ be open. We say that $\psi:\Omega \to\HH'$ is measurable, if $x \mapsto (v, \psi(x) )$ is measurable for all $v \in \HH'$. Other common notions of measurability of vector-valued functions are equivalent in our setting, see Theorem IV.22 in \cite{RSI}. If $\varphi, \psi$ are measurable, then so is $x \mapsto (\varphi(x), \psi(x))$. This follows from the Fourier decomposition
\begin{align} 
        \psi(x) = \sum_\ell e_\ell \, (e_\ell, \psi(x))
\label{FS}
\end{align} 
with respect to an orthonormal basis $(e_\ell)$ for $\HH'$. Let $L^2(\Omega; \, \HH')$ denote the set of (equivalence classes of) measurable functions $\psi: \Omega \rightarrow \HH'$ with $|\psi|\in L^2(\Omega)$.
$L^2(\Omega; \, \HH')$ is a Hilbert space with inner product
\begin{align*} 
\langle \varphi, \psi \rangle_2 = \int\displaylimits_\Omega \hspace{-1.6mm} dx \, (\varphi(x), \psi(x)).
\end{align*}
For functions $\varphi, \psi \in L^2(\Omega; \,\HH')$ and $\alpha \in \N_0^n$ we write $\partial^\alpha \varphi = \psi$ if 
\begin{align} 
    (-1)^{| \alpha |}\intd x \, \partial^\alpha \zeta(x) \: (v, \varphi(x)) = \intd x \, \zeta(x) \: (v, \psi(x))
\label{def: weak partial}
\end{align}
for all $\zeta \in C_0^\infty(\Omega)$ and $v \in \HH'$. Then $\partial^\alpha (v, \varphi(x)) = (v,\partial^\alpha\varphi(x))$ and the weak derivative $\psi=\partial^\alpha \varphi$ is unique because $(v, \psi(x))$ in \eqref{def: weak partial} is unique, up to a set of measure zero, and $\HH'$ is separable. As in the case $\HH'=\C$, $\partial^\alpha$ is a closed operator in $L^2(\Omega;\, \HH')$ on its natural domain
\begin{align*} 
D(\partial^\alpha) =   \{ \varphi \in L^2(\Omega; \, \HH') \, | \, \partial^\alpha \varphi \in L^2(\Omega; \, \HH')\}.
\end{align*}
For every $m\in\N$ we introduce the Hilbert space
\begin{align*} 
H^m(\Omega; \, \HH') = \bigcap_{| \alpha | \leq m} D(\partial^\alpha)
\end{align*}
with the inner product
\begin{align*}  
\langle \varphi, \psi \rangle_{H^m} = \sum_{|\alpha| \leq m} \langle \partial^\alpha \varphi, \partial^\alpha \psi \rangle_2.
\end{align*} 
Completeness of $H^m(\Omega; \, \HH')$ follows from the completeness of $L^2(\Omega; \, \HH')$ and the fact that $\partial^\alpha$ is closed. We write $\psi \in H_\mathrm{loc}^m(\R^n; \, \HH')$ if the restriction of $\psi:\R^n\to \HH'$ to $\Omega$ belongs to $H^m(\Omega; \, \HH')$ for every open and bounded $\Omega \subset \R^n$. \newline

If $\HH' = \C$ then our new $H^m(\Omega; \, \C)$ agrees with the well-known Sobolev space $H^m(\Omega) = W^{m, 2}(\Omega)$. By the theorem of Meyers-Serrin
\begin{align*} 
\mathcal{C}_\Omega \coloneqq \begin{cases}
C^\infty(\Omega) \cap H^m(\Omega) \quad &\Omega \neq \R^n \\
C_0^\infty(\R^n) \quad &\Omega = \R^n
\end{cases}
\end{align*}
is dense in $H^m(\Omega)$. 
From this we obtain the following generalization of the Meyers-Serrin theorem:

%---------------------------------------------------------------------------------------------------------------------------
\begin{Theorem}
\label{meyers serrin}
The subspace
\begin{align*}  
\mathcal{C}_\Omega' \coloneqq \left\{ \left. \sum_{\mathrm{finite}}   f_k v_k \, \right| \,  f_k \in \mathcal{C}_\Omega, \, v_k \in \HH' \right \}
\end{align*} 
is dense in $H^m(\Omega; \, \HH')$. 
\end{Theorem} 

\begin{proof}
Let $(e_\ell)$ be an orthonormal basis of $\HH'$. On the one hand,
\begin{align*}  
H^m(\Omega; \, \HH') &\rightarrow \bigoplus_\ell H^m(\Omega) \\
\psi &\mapsto \psi_\ell(x) = ( e_\ell, \psi(x))
\end{align*} 
is a unitary map. On the other hand, by Meyers-Serrin, the set 
$\{ (\psi_\ell) \: | \: \psi_\ell \in \mathcal{C}_\Omega, \, \psi_\ell = 0 \: \, \mathrm{for} \: \, \mathrm{sufficiently} \: \, \mathrm{large} \: \, \ell \}$
is dense in $\bigoplus_\ell H^m(\Omega)$. This proves the theorem.
\end{proof}
%---------------------------------------------------------------------------------------------------------------------------------

We will need the following lemmas about $H_\mathrm{loc}^m(\R^n; \, \HH')$. 

\begin{Lemma} 
\label{Produktregel (vektorwertig)}
If $\varphi, \psi \in H_\mathrm{loc}^1(\R^n; \, \HH')$, then $(\varphi, \psi) \in L_\mathrm{loc}^1(\R^n)$ is weakly differentiable and 
\begin{align*}
\nabla (\varphi, \psi) = (\nabla \varphi, \psi) + (\varphi, \nabla \psi).
\end{align*}
\end{Lemma}

\begin{proof} 
Let $\Omega \subset \R^n$ be open and bounded. By \Cref{meyers serrin} we find sequences $(\varphi_k), (\psi_k)$ in $\mathcal{C}_\Omega'$ with $\varphi_k \rightarrow \varphi$ and $\psi_k \rightarrow \psi$ in $H^1(\Omega; \, \HH')$. It follows that $(\varphi_k, \psi_k) \rightarrow (\varphi, \psi)$ in $L^1(\Omega)$ and by the product rule for $C^\infty$ functions
\begin{align*}
\nabla (\varphi_k, \psi_k) = (\nabla \varphi_k, \psi_k) + (\varphi_k, \nabla \psi_k) 
\rightarrow  (\nabla \varphi, \psi) + (\varphi, \nabla \psi) \quad \text{in} \: L^1(\Omega).
\end{align*}
Hence $\nabla (\varphi, \psi) = (\nabla \varphi, \psi) + (\varphi, \nabla \psi)$.
\end{proof}

%----------------------------------------------------------------------------------------------------------------------------------------------------------------------
\begin{Lemma}
\label{loc lemma}
For $\psi : \R^n \rightarrow \HH'$ define
\begin{align*}
\mathrm{sgn} \, \psi =
\begin{dcases}
\dfrac{\psi}{|\psi|} \quad & \psi \neq 0 \\
\, \, \, 0 \quad & \psi = 0.
\end{dcases}
\end{align*}

\begin{enumerate}
\item[(i)] \textbf{Diamagnetic inequality}: If $\psi \in H_\mathrm{loc}^1(\R^n; \, \HH')$, then $| \psi |$ is weakly differentiable and $\nabla | \psi | = \mathrm{Re} ( \mathrm{sgn} \,  \psi, \nabla \psi)$.
In particular $| \psi | \in H_\mathrm{loc}^1(\R^n)$ and 
\begin{align*}  
| \nabla | \psi | | \leq | \nabla \psi |.
\end{align*}
\item[(ii)] If $\psi \in H_\mathrm{loc}^2(\R^n; \, \HH')$, then $| \psi | \nabla | \psi |$ is weakly differentiable and 
\begin{align*}  
\nabla\left( | \psi | \nabla | \psi | \right) = \mathrm{Re} (\psi, \Delta \psi) + | \nabla \psi |^2.
\end{align*} 
\item[(iii)] \textbf{Kato's distributional inequality}:  If $\psi \in H_\mathrm{loc}^2(\R^n; \, \HH')$, then
\begin{align*} 
\intd x \nabla |\psi| \nabla \zeta \leq \intd x \, \mathrm{Re} \, (\mathrm{sgn} \, \psi, - \Delta \psi) \zeta
\end{align*} 
for all $\zeta \in C_0^\infty(\R^n; \, \R_{\geq 0})$.
\end{enumerate}
\end{Lemma}

\begin{proof} 
The following proof is inspired by the proof of Lemma 5.4 in \cite{A}, which concernes the case $\HH'=\C$. 

(i) Let $\psi_\varepsilon = ( | \psi |^2 + \varepsilon^2)^{1/2}$,  $\varepsilon > 0$. From Lemma \ref{Produktregel (vektorwertig)} it follows that $\nabla | \psi |^2 = 2 \mathrm{Re}( \psi, \nabla \psi)$. By the chain rule (Lemma \ref{Kettenregel}) we see that
\begin{align}
\nabla \psi_\varepsilon &= \dfrac{1}{2} ( | \psi |^2 + \varepsilon^2)^{-1/2} \nabla | \psi |^2 = \mathrm{Re} \left( \dfrac{\psi}{\psi_\varepsilon}, \nabla \psi \right).
\label{e1}
\end{align}
Since $\psi / \psi_\varepsilon \rightarrow \mathrm{sgn} \, \psi$ pointwise and $| \psi / \psi_\varepsilon | \leq 1$, it follows that $\nabla \psi_\varepsilon \rightarrow \mathrm{Re} (\mathrm{sgn} \, \psi, \nabla \psi)$ in $L_\mathrm{loc}^2(\R^n)$. In conjunction with $\psi_\varepsilon \rightarrow | \psi |$ in $L_\mathrm{loc}^2(\R^n)$ it follows that $\nabla | \psi | = \mathrm{Re} (\mathrm{sgn} \, \psi, \nabla \psi)$. In particular $| \nabla | \psi | | \leq | \nabla \psi |$ and $| \psi | \in H_\mathrm{loc}^1(\R^n)$. 

(ii) From the product rule (Lemma \ref{Produktregel}) and Lemma \ref{Produktregel (vektorwertig)} it follows that
\begin{align}
| \psi | \nabla |\psi|  = \dfrac{1}{2} \nabla | \psi |^2 = \mathrm{Re} ( \psi, \nabla \psi).
\label{e2}
\end{align}
Hence $\nabla ( | \psi | \nabla | \psi |) = | \nabla \psi |^2 + \mathrm{Re}(\psi, \Delta \psi). $ 

(iii) From \eqref{e1} and \eqref{e2} it follows that
\begin{align*} 
\nabla \psi_\varepsilon = \frac{| \psi | \nabla | \psi |}{\psi_\varepsilon}.
\end{align*} 
From the chain rule (Lemma \ref{Kettenregel}) it follows that
\begin{align}  
\nabla \frac{1}{\psi_\varepsilon} = - \frac{| \psi | \nabla | \psi |}{\psi_\varepsilon^3}.
\label{e3}
\end{align} 
Using the product rule (Lemma \ref{Produktregel}), \eqref{e3} and (ii) it follows that
\begin{align}  
- \nabla \left( \frac{1}{\psi_\varepsilon} | \psi | \nabla | \psi| \right) &= -\nabla \left( \frac{1}{\psi_\varepsilon} \right) | \psi|\nabla | \psi | - \frac{1}{\psi_\varepsilon} \nabla ( | \psi| \nabla | \psi|) \nonumber \\
&= \frac{| \psi |^2 |\nabla | \psi ||^2}{\psi_\varepsilon^3} +  \frac{1}{\psi_\varepsilon}  \left( \mathrm{Re} (\psi, -\Delta \psi) - | \nabla \psi |^2 \right) \nonumber \\
&\leq \frac{1}{\psi_\varepsilon} \mathrm{Re} \, ( \psi, - \Delta \psi). 
\label{e4}
\end{align} 
In the last line we used $| \psi|^2 / \psi_\varepsilon^2 \leq 1$ and $|\nabla |\psi||^2 \leq |\nabla \psi|^2$. From \eqref{e4} it follows that
\begin{align*}  
\intd x \frac{|\psi|}{\psi_\varepsilon} \nabla |\psi| \nabla \zeta \leq \intd x \,  \mathrm{Re} \, \left( \frac{\psi}{\psi_\varepsilon}, - \Delta \psi \right) \zeta
\end{align*} 
for all  $\zeta \in C_0^\infty(\R^n; \, \R_{\geq 0})$. (iii) now follows in the limit $\varepsilon \rightarrow 0$ by dominated convergence.
\end{proof}


%----------------------------------------------------------------------------------------------------------------------------------------------------

\section{Pointwise bounds on vector-valued Sobolev functions}
\label{sec:main}

In this section we present a general theorem for obtaining local pointwise bounds for functions in the Sobolev space
$H_\mathrm{loc}^2(\R^n;\,\HH')$, with $\HH'$ a separable Hilbert space. This section is motivated by the applications to non-relativistic quantum field theory in \Cref{sec:qft-exp}.


%------------------------------------------------------------------------------------------------
\begin{Theorem}
\label{local pointwise bounds}
Let $q_{-} : \R^n \rightarrow [0, \infty)$ be measurable and suppose for some $c > 0, \mu > 0, s \in (2,4)$ that
\begin{align}
\| q_{-}^{1/2} u \|_2 \leq  & \, \varepsilon 
\| \nabla u \|_2 + c \,  \varepsilon^{- \mu} \| u \|_2,
\label{q2} \\ 
\intd x \, q_{-} &|u|^{s} < \infty,
\label{q1} 
\end{align}
for all  $u \in H^1(\R^n)$ and sufficiently small $\varepsilon > 0$. If $\psi \in H^2_\mathrm{loc}(\R^n; \, \HH')$ satisfies the pointwise bound
\begin{align} 
\mathrm{Re} \, (\psi, - \Delta \psi) \leq | \nabla \psi |^2 + q_{-}| \psi |^2,
\label{p2}
\end{align} 
then $| \psi | \in L_\mathrm{loc}^\infty(\R^n)$. Moreover, for every $\delta_0  \in (0,1)$, $r_0 > 0$ and $p_0 > 2$ there exists a constant $C_0$ such that
\begin{align}  
\| | \psi | \|_{\infty, B(x, \delta_0 r_0)} \leq C_0 \| | \psi | \|_{p_0, B(x, r_0)}
\label{Linfty bound}
\end{align} 
for all $x \in \R^n$. If \eqref{p2} holds without the gradient term, then \eqref{Linfty bound} holds with $p_0 = 2$.
\end{Theorem}

\noindent
\emph{Remark:} The assumptions on $q_{-}$ are satisfied, for example, if $q_{-} = V_{-} + K$, with $K > 0$ some constant and $V_{-} = \mathrm{max}(0, -V)$ the negative part of an $N$-particle potential
\begin{align*}
V(x) = \sum_{j = 1}^N v_j(x_j) + \sum_{j < k} w_{j k}(x_j - x_k), \quad x = (x_1, ..., x_N) \in \R^{3 N},
\end{align*}
with $(v_j)_{-},(w_{j k})_{-} \in L^p(\R^3) + L^\infty(\R^3)$ for some $p > 3/2$. For the proof see \Cref{Potential Epsilon Lemma}.
\newline
%------------------------------------------------------------------------------------------------------------------------------------------------------------

In the case of $\HH' = \C$ and
\begin{align} 
   \mathrm{Re} \, (\psi, - \Delta \psi) \leq q_{-}| \psi |^2
\label{p1}
\end{align} 
the theorem is due to Agmon (see the proof of Theorem 5.1 in \cite{A}). Under the assumption \eqref{p1} Agmon's argument easily generalizes to the present vector valued case. The only new input compared to Agmon's proof is Kato's distributional inequality for $\HH'$-valued functions, \Cref{loc lemma}. From this lemma and assumption \eqref{p1} we find that
\begin{align} 
\intd x \nabla |\psi| \nabla \zeta \leq \intd x \, \mathrm{Re} \, (\mathrm{sgn} \, \psi, - \Delta \psi) \zeta \leq \intd x \, q_{-} | \psi | \zeta
\label{p3}
\end{align} 
for all $\zeta \in C_0^\infty(\R^n; \, \R_{\geq 0})$. Note that \eqref{p3} only involves scalar-valued functions. Agmon's proof is solely based on \eqref{p3}. A modified version of his argument is given in the proof of \Cref{local pointwise bounds} below. We now give an overview of the remaining part of Agmon's proof. By replacing $\zeta$ with (a regularization of) $\zeta^2 | \psi |^{p - 1}$ in \eqref{p3}  it follows that
\begin{align}
(p - 1) \intd x \, \zeta^2  |\psi|^{p - 2} | \nabla |\psi| |^2 \leq 2  \intd x \, \zeta | \nabla \zeta | |\psi|^{p - 1} | \nabla | \psi | | + \intd x \, \zeta^2 \, q_{-}| \psi |^p
\label{p4}
\end{align}
for all $\zeta \in C_0^\infty(\R^n; \, \R_{\geq 0})$ and $p \geq 2$. It is well-known, see Theorem 4.12 in \cite{Adams}, that there exists a constant $S$ such that
\begin{align}
\| u \|_{2^{*}} \leq S (\| \nabla u \|_2 + \|  u \|_2) 
\label{GN}
\end{align}
for all $u \in H^1(\R^{n})$ with $2^{*} > 2$ defined as
\begin{align*}
2^{*} \coloneqq \begin{dcases}
\: \: \: 3 \quad &n \leq 2 \\ 
\frac{2 n}{n - 2} \quad &n \geq 3.
\end{dcases}
\end{align*} 
By an iteration argument based on \eqref{p4} and the Sobolev embedding \eqref{GN}, Agmon arrives at a uniform bound of the form
\begin{align}  
\| | \psi | \|_{p_i, B(x, \delta_0 r_0)} \leq C_0 \| | \psi | \|_{2, B(x, r_0)}
\label{Agmon bound}
\end{align} 
for all $x \in \R^{n}$ and $i \in \N$. Here, the sequence $(p_i)$ is defined as 
\begin{align*}  
p_i = \left( \frac{2^{*}}{2} \right)^i 2, \quad i \in \N.
\end{align*} 
It follows that
\begin{align*}  
\| | \psi |\|_{\infty, B(x, \delta_0 r_0)} =  \lim_{i \rightarrow \infty} \| | \psi | \|_{p_i, B(x, \delta_0 r_0)} \leq C_0 \| | \psi | \|_{2, B(x, r_0)}
\end{align*} 
for all $x \in \R^{n}$. This concludes Agmon's proof. \\ 
%----------------------------------------------------------------------------------------------------------------------------

The proof of Theorem \ref{local pointwise bounds}  is based on the following three lemmas, the first one being the heart of it:
%---------------------------------------------------------------------------------------------------------------------------
\begin{Lemma}
Under the hypothesis of Theorem \ref{local pointwise bounds} it follows that
\begin{align}
(p - 2) \intd x \, \zeta^2  |\psi|^{p - 2} | \nabla |\psi| |^2 \leq  2 \intd x \, \zeta | \nabla \zeta | |\psi|^{p - 1} | \nabla | \psi | | + \intd x \, \zeta^2 \, q_{-} | \psi |^p
\label{P1}
\end{align}
for all $p > 2$ and $\zeta \in C_0^\infty(\R^{n}; \, \R_{\geq 0})$.
\label{Hauptlemma pkt Schranken}
\end{Lemma}
\begin{proof}
Note that \eqref{P1} is a weaker version of Agmon's inequality \eqref{p4} with the factor $(p - 1)$ replaced by $(p - 2)$. Contrary to Agmon, we won't use Kato's distributional inequality. Instead, using Lemma \ref{loc lemma} (ii) we rewrite \eqref{p2} as
\begin{align}  
- \nabla( |\psi| \nabla | \psi|) = \mathrm{Re} \, (\psi, - \Delta \psi) - | \nabla \psi |^2 \leq q_{-} | \psi |^2
\label{P2}
\end{align} 
so that
\begin{align}  
\intd x |\psi|\nabla |\psi| \nabla \zeta  \leq \intd x \, q_{-} |\psi|^2 \zeta
\label{P3}
\end{align} 
for all $\zeta \in C_0^\infty(\R^{n}; \, \R_{\geq 0})$. \eqref{P3} should be compared to \eqref{p3}. We will now replace $\zeta$ with (a regularization of) $\zeta^2 |\psi|^{p - 2}$ to obtain the analog \eqref{P1} to Agmon's inequality \eqref{p4}. For $\varepsilon > 0$ and $R > 1$ define
\begin{align*}
\psi_{\varepsilon, R} = 
\begin{cases}
\varepsilon \quad &| \psi | \in [0, \varepsilon] \\
| \psi | \quad  &| \psi | \in [\varepsilon, R] \\
R \quad &| \psi | \in  [R, \infty).
\end{cases}
\end{align*}
By the chain rule (Lemma \ref{Kettenregel}) it follows that
\begin{align*}
\nabla \psi_{\varepsilon, R}^{p - 2} = (p - 2) | \psi |^{p - 3} \nabla | \psi | \rchi_{\{\varepsilon < | \psi | < R\}}.
\end{align*}
Using the product rule (Lemma \ref{Produktregel}) we find that
\begin{align}
(p - 2)  | \psi |^{p - 2} | \nabla | \psi ||^2  \rchi_{\{\varepsilon < | \psi | < R\}} = \nabla(\psi_{\varepsilon, R}^{p - 2} | \psi | \nabla | \psi |) - \psi_{\varepsilon, R}^{p - 2} \nabla ( | \psi | \nabla | \psi |).
\label{P4}
\end{align}
We now multiply \eqref{P4} with $\zeta^2$, use \eqref{P2} and integrate to obtain
\begin{align*}
(p - 2) \hspace{-3mm} \int\displaylimits_{\varepsilon < | \psi | < R} \hspace{-5mm} dx \;  \zeta^2  | \psi |^{p - 2} | \nabla | \psi ||^2 \leq 2 \intd x \, \zeta | \nabla \zeta | \psi_{\varepsilon, R}^{p - 2} | \psi | |\nabla | \psi || + \intd x \, \zeta^2 \psi_{\varepsilon, R}^{p - 2} q_{-} | \psi |^2
\end{align*}
for all $\zeta \in C_0^\infty(\R^{n}; \, \R_{\geq 0})$. Note that $\zeta^2 |\nabla|\psi||^2$, $\zeta |\nabla \zeta| |\psi| |\nabla|\psi||$ and $\zeta^2 q_{-} |\psi|^2$ are in $L^1(\R^{n})$. The latter follows from \eqref{q2}. 
(\ref{P1}) now follows by first passing to the limit $\varepsilon \rightarrow 0$ (dominated convergence) and then $R \rightarrow \infty$ (monotone convergence).
\end{proof}


%--------------------------------------------------------------------------------------------------------------------------------------------------
\begin{Lemma}\label{H1loc Lemma}
Under the hypothesis of Theorem \ref{local pointwise bounds} it follows that
\begin{align}
| \psi |^{p/2} \in H_\mathrm{loc}^1(\R^{n}) \; \,  \text{and} \; \,  \nabla | \psi |^{p/2} = \dfrac{p}{2} \, | \psi |^{p/2 - 1} \nabla |\psi|
\label{ugl 0}
\end{align}
for all $p \in [2, \infty)$.
\end{Lemma} 

\begin{proof} 
The proof of this lemma is based on \eqref{P1} and follows the line of arguments in the proof of Theorem 5.1 in \cite{A}.  
From Lemma \ref{loc lemma} it follows that $| \psi | \in H_{\mathrm{loc}}^1(\R^{n})$.  It therefore suffices to show that there exists an $\eta > 1$ such that if \eqref{ugl 0} holds for $p_0 \in [2, \infty)$, then \eqref{ugl 0} holds for all $p \in (p_0, \eta p_0]$. From \eqref{P1} and H\"older's inequality it follows that
\begin{align}
(p - 2) \intd x \, \zeta^2  |\psi|^{p - 2} | \nabla |\psi| |^2 \leq  2  \left( \intd x \, \zeta^2 | \psi |^{p_0 - 2} | \nabla | \psi | |^2 \right)^{1/2} & \left( \intd x \, | \nabla \zeta |^2 | \psi |^{2p - p_0} \right)^{1/2} \nonumber \\ 
 + & \intd x \, \zeta^2 q_{-} | \psi |^p
\label{ugl 1}
\end{align}
for all $\zeta \in C_0^\infty(\R^{n}; \, \R_{\geq 0})$. The first integral on the right-hand side is finite since (\ref{ugl 0}) is assumed to be true for $p_0$. From \eqref{GN} it follows that
\begin{align}  
H_\mathrm{loc}^1(\R^{n}) \subset L_\mathrm{loc}^{2^*}(\R^{n}).
\label{H1loc Einbettung}
\end{align} 
We conclude that $| \psi | \in L_\mathrm{loc}^{2^* p_0/2}(\R^{n})$ and the second integral on the right-hand side of \eqref{ugl 1} is finite, provided $2p - p_0 \leq 2^* p_0/2 \iff p \leq (2^* + 2)/4 \, p_0$. We also note that $| \psi |^{p/2} \in L_\mathrm{loc}^2(\R^n)$ for $p \leq 2^*/2 \, p_0$. We now impose the restriction that $\zeta = \gamma^2$ for some $\gamma \in C_0^\infty(\R^n; \, \R_{\geq 0})$. Then, with $s \in (2,4)$ given by \Cref{local pointwise bounds}$,\zeta^{2/s} = \gamma^{4/s} \in C_0^1(\R^n)$. Inserting $u = \zeta^{2/s} | \psi |^{p_0/2} \in H^1(\R^n)$ in \eqref{q1} we obtain 
\begin{align*} 
\intd x \, \zeta^2 q_{-} | \psi |^{s p_0/2} < \infty.
\end{align*}
We define $\eta = \mathrm{min} \left( \frac{2^* + 2}{4}, \frac{s}{2} \right) > 1$, for then the right-hand side of (\ref{ugl 1}) is finite for $p \leq \eta p_0$. By our freedom of choice for $\gamma$ it follows that
\begin{align*} 
|\psi|^{p/2} \in L_\mathrm{loc}^2(\R^{n}) \; \mathrm{and} \; | \psi |^{p/2 - 1} \nabla |\psi | \in L_\mathrm{loc}^2(\R^{n})
\end{align*}
for all $p \in (p_0, \eta p_0]$. In view of Lemma \ref{u^p Kettenregel} this concludes the proof. 
\end{proof} 
%----------------------------------------------------------------------------------



%----------------------------------------------------------------------------------------------
\begin{Lemma}
\label{p geq p0}
Under the hypothesis of Theorem \ref{local pointwise bounds}, for every $p_0 > 2$ there exists a constant $C_1$ such that for all $p \geq p_0$ and $\zeta \in C_0^\infty(\R^{n}; \, \R_{\geq 0})$ 
\begin{align}
\| \nabla( \zeta | \psi |^{p/2}) \|_2 \leq C_1 \, p^\alpha \| ( | \nabla \zeta | + \zeta ) | \psi |^{p/2} \|_2,
\label{p geq p0 ineq}
\end{align}
where $\alpha = (\mu + 1)/2$.
\end{Lemma}

\begin{proof} 
The proof of this lemma is based on \eqref{P1} and it follows the proof of (Theorem 5.1 in \cite{A}). Note that \eqref{P1} becomes useless for $p = 2$. Agmon works with the stronger inequality \eqref{p4}, which doesn't degenerate at $p = 2$. By fixing a number $p_0 > 2$ and looking at numbers $p \geq p_0$ we can reuse Agmon's proof. From \eqref{P1} and H\"older's inequality it follows that
\begin{align*}
(p - 2) \intd x \, \zeta^2  |\psi|^{p - 2} | \nabla |\psi| |^2 \leq  2  \left( \intd x \, \zeta^2 | \psi |^{p - 2} | \nabla | \psi | |^2 \right)^{1/2} & \left( \intd x \, | \nabla \zeta |^2 | \psi |^{p} \right)^{1/2} \nonumber \\ 
 + & \intd x \, \zeta^2 q_{-} | \psi |^p
\end{align*}
Using  (\ref{ugl 0}) this can be rewritten as
\begin{align*}
(p - 2) \dfrac{4}{p^2} \underbrace{\intd x \, \zeta^2 | \nabla |\psi|^{p/2} |^2}_{= A^2} \leq 2 \dfrac{2}{p} \underbrace{\left(\intd x \, \zeta^2 | \nabla |\psi|^{p/2}|^2 \right)^{1/2}}_{= A} & \underbrace{\left( \intd x \, | \nabla \zeta |^2 (| \psi |^{p/2})^2 \right)^{1/2}}_{= B} \\ 
+ & \underbrace{\intd x \, \zeta ^2 q_{-} (|\psi |^{p/2})^2}_{= C}.
\end{align*}
Note that by Lemma \ref{H1loc Lemma} and assumption \eqref{q2} all integrals are finite. The above inequality is equivalent to
\begin{align*}
A^2 \leq \dfrac{p}{p - 2} A B + \dfrac{p^2}{4(p - 2)} C.
\end{align*}
Next, we use that $p/(p - 2) \leq p_0/(p_0 - 2) \eqqcolon D_0$ for $p \geq p_0$ and $A B \leq \delta A^2 + B^2/\delta$ for all $\delta > 0$:
\begin{align*}
A^2 &\leq D_0 \left( \delta A^2 + \dfrac{B^2}{\delta} \right) + \dfrac{p}{4} D_0 C \\
\iff (1 - D_0 \delta) A^2 &\leq D_0 \dfrac{B^2}{\delta} + \dfrac{p}{4} D_0 C.
\end{align*}
We choose $\delta = \dfrac{1}{2 D_0}$, so that
\begin{align*}
\dfrac{1}{2} A^2 \leq 2 D_0^2 B^2 + \dfrac{p}{4} D_0 C.
\end{align*}
Using $\sqrt{a + b} \leq \sqrt{a} + \sqrt{b}$ it follows that
\begin{align*}
A \leq 2 D_0 B + \sqrt{\dfrac{D_0}{2}} \sqrt{p} \, \sqrt{C}
\end{align*}
or equivalently
\begin{align}
\| \zeta \nabla |\psi|^{p/2} \|_2 \leq 2 D_0 \| \nabla \zeta | \psi |^{p/2} \|_2 + \sqrt{\dfrac{D_0}{2}} \sqrt{p} \, \| \zeta q_{-}^{1/2} |\psi|^{p/2} \|_2
\end{align}
for all $p \geq p_0$ and $\zeta \in C_0^\infty(\R^{n}; \, \R_{\geq 0})$. By the product rule it follows that
\begin{align}
\| \nabla( \zeta | \psi |^{p/2}) \|_2 \leq (1 + 2 D_0) \| \nabla \zeta | \psi |^{p/2} \|_2 + \sqrt{\dfrac{D_0}{2}} \sqrt{p} \, \| \zeta q_{-}^{1/2} |\psi|^{p/2} \|_2.
\label{UGL 1}
\end{align}
By \eqref{q2} we have 
\begin{align}
\| q_{-}^{1/2} u \|_2 \leq  \varepsilon \left(\| \nabla u \|_2 + c \, \varepsilon^{-(\mu + 1)}  \| u \|_2 \right)
\label{UGL 2}
\end{align}
for all $u \in H^1(\R^{n})$ and sufficiently small $\varepsilon > 0$. We now enlarge $D_0$ if necessary and choose $\varepsilon = \varepsilon(p)$ such that $\sqrt{\dfrac{D_0}{2}} \sqrt{p} \, \varepsilon = \dfrac{1}{2}$. It then follows from (\ref{UGL 1}) and (\ref{UGL 2}) that
\begin{align*}
\| \nabla( \zeta | \psi |^{p/2}) \|_2 \leq (1 + 2 D_0) \| \nabla \zeta | \psi |^{p/2} \|_2 + \dfrac{1}{2} \| \nabla( \zeta | \psi |^{p/2}) \|_2 + \dfrac{1}{2} c \, \varepsilon^{-(\mu + 1)}  \| \zeta | \psi |^{p/2} \|_2.
\end{align*}
Since $\varepsilon$ is proportional to $p^{-1/2}$, we conclude that 
\begin{align*}
\| \nabla( \zeta | \psi |^{p/2}) \|_2 \leq C_1 \, p^{(\mu + 1)/2} \| (\zeta + | \nabla \zeta |) | \psi |^{p/2} \|_2
\end{align*}
for some constant $C_1$ and all $p \geq p_0$, $\zeta \in C_0^\infty(\R^{n}; \, \R_{\geq 0})$. 
\end{proof} 
%--------------------------------------------------------------------------------------------------------------------------




%---------------------------------------------------------------------------
\begin{proof}[Proof of Theorem \ref{local pointwise bounds}:] 
Fix $\delta_0 \in (0,1)$, $r_0 > 0$ and $p_0 > 2$. Following Agmon we combine \eqref{p geq p0 ineq} with the Sobolev embedding \eqref{GN} and obtain
\begin{align}
\| \zeta | \psi |^{p/2} \|_{2^{*}} &\leq S(C_1 \, p^\alpha + 1))  \| ( | \nabla \zeta | + \zeta ) | \psi |^{p/2} \|_2 \nonumber \\ 
&\leq C_2 \, p^\alpha  \| ( | \nabla \zeta | + \zeta ) | \psi |^{p/2} \|_2
\label{Ugl 2}
\end{align}
for all $p \geq p_0$ and $\zeta \in C_0^\infty(\R^{n}; \, \R_{\geq 0})$. This will give us a uniform bound similar to \eqref{Agmon bound}. Note that (\ref{Ugl 2}) extends to non-negative, compactly supported Lipschitz functions. We define the sequence 
\begin{align*}
r_i = R - \varepsilon \sum_{j = 1}^{i + 1} \dfrac{1}{j^2}, \quad i \in \N_0,
\end{align*}
with $R, \varepsilon > 0$ chosen such that $r_i $ agrees for $i = 0$ with the above $r_0$ and $r_i \searrow \delta_0 r_0$. Furthermore we define, for $i \in \N_0$, continuous characteristic functions of the balls $B(x, r_i)$,
\begin{align*}
\zeta_i (y) = 
\begin{cases}
1 \quad &| y | \leq r_{i + 1} \\
\dfrac{r_i - |y|}{r_i - r_{i + 1}} \quad &r_{i + 1} \leq | y | \leq r_{i} \\
0 \quad &| y | \geq r_{i}.
\end{cases}
\end{align*}
Notice that $\| \nabla \zeta_i \|_\infty \leq 1/(r_i - r_{i + 1}) =  \frac{1}{\varepsilon}(i + 2)^2 $. The bound  \eqref{Ugl 2}
for $\zeta(y) =\zeta_{i - 1}( y - x)$ gives 
\begin{align}
\| | \psi |^{p/2} \|_{2^{*}, B(x, r_i)} &\leq C_2 \, p^\alpha \left(\dfrac{1}{\varepsilon} (i + 1)^2 + 1 \right) \| |\psi|^{p/2} \|_{2, B(x, r_{i - 1})} \nonumber \\
&\leq C_3 \, p^\alpha \, i^2 \| |\psi|^{p/2} \|_{2, B(x, r_{i - 1})}
\label{Ugl 3}
\end{align}
for all $x \in \R^n$, $i \in \N$ and $p \geq p_0$. Let $p_i = (2^{*}/2)^{i} p_0$, $i \in \N_0$.
From \eqref{Ugl 3} with  $p = p_{i - 1}$ we obtain
\begin{align}
\| | \psi | \|_{p_i, B(x, r_i)} \leq \left(C_3 \, p_{i - 1}^\alpha \, i^2\right)^{2^{*}/p_i} \| | \psi | \|_{p_{i - 1}, B(x, r_{i - 1})} 
\label{Ugl 4}
\end{align}
for all $i \in \N$ and $x \in \R^n$. By iteration of  \eqref{Ugl 4} we find
\begin{align}
\| | \psi | \|_{p_i, B(x, r_i)} \leq \prod_{j = 1}^{i} \left(C_3 \, p_{j - 1}^\alpha \, j^2\right)^{2^{*}/p_j} \| | \psi | \|_{p_0, B(x, r_0)}. 
\label{prod}
\end{align}
Since
\begin{align*}  
\mathrm{log} \left( \prod_{j = 1}^{i} \left(C_3 \, p_{j - 1}^\alpha \, j^2\right)^{2^{*}/p_j} \right) = \sum_{j = 1}^i \left(\frac{2}{2^{*}} \right)^j \, O(j)
\end{align*} 
we can bound the product in \eqref{prod} uniformly in $i$. Hence
\begin{align}
\| | \psi | \|_{\infty, B(x, \delta_0 r_0)} = \lim_{i \rightarrow \infty} \| | \psi | \|_{p_i, B(x, \delta_0 r_0)} \leq C_4 \| | \psi | \|_{p_0, B(x, r_0)} 
\label{Ugl 5}
\end{align}
for all $x \in \R^n$. 
\end{proof}
%----------------------------------------------------------------------------------------------------------------


\section{The Nelson and Pauli-Fierz models}
\label{sec:qft}

\subsection{Fock space and second quantization}
\label{fock space}

The one particle Hilbert space associated with the quantized boson field will be $\mathfrak{h} = L^2(\R^3 \times \{1, 2\})$  for the Pauli-Fierz model and $\mathfrak{h} = L^2(\R^3)$ for the Nelson model. The symmetric Fock space over $\mathfrak{h}$ will be denoted by
\begin{align*} 
     \Gamma(\mathfrak{h}) =\bigoplus_{n = 0}^\infty \otimes_\mathrm{sym}^n \,  \mathfrak{h},\qquad \otimes_\mathrm{sym}^0 \,  \mathfrak{h} := \C.
%\mathcal{F}_n^{+}, \quad \mathcal{F}_0^{+} = \C, \quad \mathcal{F}_n^{+} = \quad n \geq 1.
\end{align*}
For $h \in \mathfrak{h}$ let $a(h)$ and $a^*(h)$ denote the usual bosonic annihilation and creation operators in $\Gamma(\mathfrak{h})$, and let
\begin{align*} 
\phi(h) = a(h) + a^*(h),
\end{align*} 
which is defined and symmetric on the subspace of finite particle vectors from $\Gamma(\mathfrak{h})$.
The closure of this operator is self-adjoint and denoted by the same symbol.
For $\mu \geq 0$ let 
\begin{align*} 
       \omega(k) = \sqrt{|k|^2 + \mu^2}, \quad k \in \R^3,
\end{align*} 
and let $H_f = d\Gamma(\omega)$ be the second quantization of multiplication with $\omega$ in $\mathfrak{h}$.
If $h/\sqrt{\omega} \in \mathfrak{h}$, then $D(H_f^{1/2}) \subset  D(\phi(h))$ and
\begin{align} 
\| \phi(h) (H_f + 1)^{-1/2} \| \leq 2 (\| h \|^2 + \| h/\sqrt{\omega}  \|^2)^{1/2}.
\label{phi relative bounded}
\end{align} 
For the proofs of these statements see, e.g., \cite{BFS, HH}. 

%-----------------------------------------------------------------------------------------------------------------------------
\subsection{The Pauli-Fierz model}

We now introduce the Pauli-Fierz model for $N$ identical spin-$1/2$ particles, called electrons, in $\R^3$. 
More elaborate descriptions may be found in \cite{BFS,HH}.

The Hilbert space of the model (without statistics yet) is the tensor product
\begin{align}
     \HH = L^2(\R^{3 N}) \otimes \bigg(\bigotimes_{j = 1}^N \C^2\bigg) \otimes \mathcal{F}^{+}.
\label{full Hilbert space}
\end{align}
of particle and Fock space $\FF^{+} =\Gamma(L^2(\R^3\times\{1,2\}))$, with the $N$ factors of $\C^2$ accounting for the spin degrees of the particles. The inner product and norm of $\HH$ will be denoted by $\langle \cdot, \cdot \rangle$ and $\| \cdot \|$, respectively. 

The Pauli-Fierz Hamiltonian is composed of operators acting on the various factors of  \eqref{full Hilbert space}. The operator $H_f = d\Gamma(\omega)$ in $\FF^{+}$ accounts for the energy of massless photons, that is, $\omega(k)= |k|$.
The potential $V:\R^{3 N} \rightarrow \R$ acts by multiplication on the first factor of \eqref{full Hilbert space}. We
assume that
\begin{align}
V(x) &= \sum_{j = 1}^N v(x_j) + \sum_{j < k} w(x_j - x_k), \quad x = (x_1, ..., x_N) \in \R^{3 N}, \label{potential}
\\
&v,w \in L^2(\R^3) + L^{\infty}(\R^3), \qquad w(x) = w(-x). \nonumber
\end{align}
It follows that $V$ is infinitesimally bounded with respect to the Laplacian $-\Delta$. As usual we shall not distinguish in notation between the operator $V$ in $L^2(\R^{3 N})$ and the operator $V \otimes 1$ in $\HH$. The same remark applies to $-\Delta$ and the components of the momentum operator $- i \nabla_j$ of the $j$th electron.  

To define quantized vector potential and magnetic field we introduce, for $x  \in \R^3$ and $\ell = 1,2,3$, the elements
$G_{\ell}(x), F_{\ell}(x) \in \mathfrak{h}$ by the functions
\begin{align}
  [G_{\ell}(x)](k, \lambda) &= \frac{\rchi_\Lambda(k)}{\sqrt{\omega(k)}} e^{-i k \cdot x} \varepsilon_\ell(k, \lambda),  \label{def G}\\
[F_{\ell}(x)](k, \lambda) &= -i  \frac{\rchi_\Lambda(k)}{\sqrt{\omega(k)}} e^{-i k \cdot x} (k \wedge \varepsilon(k, \lambda))_\ell, \quad k \in \R^3, \lambda = 1,2. \nonumber
\end{align}
Here $\Lambda < \infty$ is an arbitrary ultraviolet cutoff and $\rchi_\Lambda$ denotes the characteristic function of the set $\{|k| \leq \Lambda \}$. The polarization vectors  $\varepsilon(k, 1), \varepsilon(k, 2)\in \R^3$, for $k\neq 0$, are normalized and orthogonal to $k \in \R^3$. With the identification $L^2(\R^{3 N}) \bigotimes \FF^+ \, \widetilde{=} \, L^2(\R^{3 N} ; \FF^+)$ we can define
\begin{align}
(A_{j, \ell} \, \psi)(x) &= \phi(G_\ell(x_j)) \psi(x), \nonumber \\ 
(B_{j, \ell} \, \psi)(x) &= \phi(F_\ell(x_j)) \psi(x).
\label{vector potential}
\end{align}
By $A_j$ and $B_j$ we denote the operator-valued vectors with components
$A_{j, \ell}$ and $B_{j, \ell}$, $\ell = 1,2,3$. With the above notations and conventions the Pauli-Fierz Hamiltonian reads
\begin{equation}
   H=\sum_{j = 1}^N \left[ (-i \nabla_j + \sqrt{\alpha}  A_j)^2 + \sqrt{\alpha} \, \sigma_j \cdot B_j \right]  + V + H_f,
\label{HAMILTONIAN}
\end{equation}
where $\sigma_j$ denotes the triple of Pauli matrices acting on the $j$th factor in $\bigotimes_{j = 1}^N \C^2$. 
It is well known that for all values of the coupling constant $\alpha > 0$ the Hamiltonian is self-adjoint on $D(H) = D(-\Delta + H_f) = D(-\Delta) \cap D(H_f)$ and bounded from below \cite{Hi2002,HH}. Moreover, $H$ is essentially self-adjoint on any core for $-\Delta + H_f$ \cite{HH}.

\medskip
To fit the Pauli-Fierz model into the setting of the previous sections, we now define
\begin{align*}  
    \HH'= \bigg(\bigotimes_{j = 1}^N \C^2 \bigg) \otimes \mathcal{F}^{+}
\end{align*} 
with inner product $( \cdot , \cdot )$ and norm $| \cdot |$. Then $\HH$ can be identified with $L^2(\R^{3 N};\, \HH')$ via the unitary map that is determined by $\psi = f \otimes v \mapsto \psi(x) = f(x) \, v$. 
The domain of the Laplacian $-\Delta$ now becomes the Sobolev space
\begin{align*}  
       D(-\Delta) = H^2(\R^{3 N}; \,\HH'),
\end{align*} 
and the operators $-i \nabla_j$ and $- \Delta$ correspond  to weak derivatives in $H^2(\R^{3 N};\,\HH')$. The field energy in $L^2(\R^{3 N}; \, \HH')$ is given by $(H_f \psi)(x) = H_f \psi(x)$ and its domain reads
\begin{align*}  
D(H_f) = \bigg\{ \psi \in L^2(\R^{3 N}; \, \HH')\bigg| \, \psi(x) \in D(H_f) \: \mathrm{for} \: \mathrm{a.e.} \,  x, \int | H_f \psi(x) |^2\, dx < \infty \bigg\}.
\end{align*} 
Notice that $\HH'$ contains the spin-space and hence all Fock-space operators are extended to $\HH'$ in the usual way. 
To further simply notation we introduce operator-valued vectors $A,B$ and $\sigma$ with $3N$ components
such that \eqref{HAMILTONIAN} becomes
\begin{equation}\label{Pauli-Fierz} 
     H = (-i \nabla + \sqrt{\alpha} A)^2 + \sqrt{\alpha} \, \sigma \cdot B + V + H_f.
\end{equation} 

We now restrict to a closed subspace of $\HH$ to account for the fermionic nature of the electrons. We write $\psi \in \HH^{-}$  if $\psi \in \HH$ is anti-symmetric with respect to permutations in $L^2(\R^{3 N}) \otimes \bigotimes_{j = 1}^N \C^2$. Since $H$ is symmetric under such permutations, it follows that the orthogonal projector onto $\HH^{-}$ commutes with $H$. Thus we can we restrict $H$ onto $\HH^{-}$ and obtain a self-adjoint operator
\begin{align*}
H^{-} = H: D(H) \cap \HH^{-} \rightarrow \HH^{-}
\end{align*} 
called the \emph{Pauli-Fierz Hamiltonian}. The \emph{ionization threshold} of $H^-$ is defined by 
\begin{equation}
\label{def:sigma}
\Sigma = \lim_{R \rightarrow \infty} \left( \inf_{\psi \in D_R} \langle \psi, H^- \psi \rangle \right)
\end{equation}
where $D_R\subset D(H^-)$ is the subset of normalized states supported outside the ball $B(0,R) \subset \R^{3N}$ \cite{G}. 

%The \textbf{ionization energy} is the energy difference $\Sigma - \inf \sigma(H^{-})$. In the case of many-particle potentials of the form \eqref{potential}, it is well known that, see Theorem 3 in \cite{G}, 
%\begin{align*}
 %\Sigma = \min_{N' = \, 1, ..., N} \left( E_{N - N'} + E_{N'}^0 \right).
%\end{align*}
%Here $E_N = \inf \sigma(H^-)$ is the lowest energy of the $N$-electron system and $E_{N  = \, 0} = 0$. $E_N^0$ is the lowest energy of the $N$-electron system, if the external potentials are dropped in (\ref{potential}).  

States with energy distribution strictly below $\Sigma$ decay exponentially in the following sense (Theorem 1 in \cite{G}):
For $\lambda \in \R$ let
 \begin{align*}  
 P_\lambda^{-} \coloneqq \rchi_{(- \infty, \lambda]}(H^{-}).
 \end{align*} 
 If $\lambda < \Sigma$ and $\psi \in \mathrm{Ran} \, P_\lambda^{-}$, then
\begin{align}
        \int e^{2 (1 - \varepsilon) \sqrt{\Sigma - \lambda} |x|} \, | \psi(x) |^2\, dx < \infty, \qquad \text{all}\ \varepsilon > 0.
\label{L 2 exponential decay}
\end{align}
This includes eigenvectors of $H^{-}$ with eigenvalue $\lambda$ below $\Sigma$ in which case the result is from Lemma 6.2 in \cite{GLL} and the proof is easier. We conclude this section with a lemma needed for applying \Cref{local pointwise bounds} to eigenvectors of $H^{-}$.
%-------------------------------------------------------------------------------------------------------------------------------------
\begin{Lemma}
\label{coulomb gauge}
There exists a constant $\nu > 0$ such that for all $\psi \in D(H_f)$ and all $x \in \R^{3 N}$,
\begin{align} 
(\psi(x), H_f \psi(x) + \sqrt{\alpha}(\sigma \cdot B \psi)(x)) \geq -\nu |\psi(x)|^2.
\label{lower bound}
\end{align} 
If $\psi \in D(H)$, then $\nabla \cdot (A \psi) = A \cdot (\nabla \psi)$.
\end{Lemma}

\begin{proof}
The bound \eqref{lower bound} may be seen either as a consequence of \eqref{phi relative bounded} and Kato-Rellich, or else the fact that $\sigma \cdot B$ acts point-wise in $x\in \R^{3N}$ by a sum of field operators. By a well-known argument (completion of squares) an explicit finite expression for $\nu$ is obtained.

For $\psi \in D(H)$ we claim that the product rule
\begin{align*}  
 \nabla \cdot (A \, \psi) =  (\nabla \cdot A) \psi + A \cdot (\nabla \psi)
\end{align*} 
holds, where $(\nabla \cdot  A)$ is defined by differentiating the form-factors \eqref{def G}. For the proof see Lemma 13 and the proof of Lemma 11 (c) in \cite{HH}. From our definition \eqref{def G} of $G_\ell(x)$ in combination with $k  \perp \varepsilon(k, 1), \varepsilon(k, 2)$ it follows that $(\nabla \cdot A) = 0$ (Coulomb gauge). Hence $\nabla \cdot (A \psi) = A \cdot (\nabla \psi)$. 
\end{proof}
%------------------------------------------------------------------------------------------------------------------------------------------------

\subsection{The Nelson model}

We now describe the Nelson model for $N$ distinguishable quantum particles interacting with a quantized scalar field of bosons. We assume that $V:\R^{3N}\to\R$ is an $N$-particle potential as in \Cref{Potential Epsilon Lemma} with $p=2$, or slightly less general, as in \eqref{potential}. Then $V$ is infinitesimally bounded with respect to the Laplacian $-\Delta$. 
With $\HH' = \Gamma(L^2(\R^3))$, the Hilbert space of the system reads again
\begin{align*}  
    \HH= L^2(\R^{3 N}) \otimes  \HH' \, \widetilde{=} \, L^2(\R^{3 N}; \, \HH').
\end{align*} 
The operator $H_f$ accounting for the field energy is defined as in Section \ref{fock space} with the boson mass $\mu\geq 0$ being arbitrary. 


Let $u \in L^2(\R^3)$ be a function with $u/\sqrt{\omega} \in L^2(\R^3)$. For $x \in \R^3$ we define $u(x)\in L^2(\R^3)$ by 
\begin{align*}  
       [u(x)](k) =  u(k) e^{-i k \cdot x}, \quad k \in \R^3.
\end{align*} 
For $\psi\in\HH$ in the domain of $H_f$ and $x=(x_1,\ldots,x_N)\in\R^{3N}$, we set 
\begin{align*} 
     (\Phi_j \psi)(x) = \phi(u(x_j)) \psi(x).
\end{align*} 
Let $\Phi = \sum_{j = 1}^N \Phi_j$ and let the closure of this symmetric operator be denoted by the same symbol.
From \eqref{phi relative bounded} it follows that $\phi$ is infinitesimally bounded with respect to $H_f$ and hence $H_f+\Phi$ is bounded below. In fact, there is some explicit $\nu > 0$ such that for all $\psi \in D(H_f)$ and all
$x \in \R^{3 N}$,  
\begin{align} 
     (\psi(x), H_f \psi(x)+ (\Phi \psi)(x)) \geq -\nu | \psi(x) |^2. 
\label{lower bound nelson}
\end{align} 

By the theorem of Kato-Rellich, the \emph{Nelson Hamiltonian}
\begin{equation}\label{Nelson}  
      H = - \Delta + \Phi + V + H_f
\end{equation} 
is self-adjoint on $D(H) = D(- \Delta + H_f) = D( - \Delta ) \cap D(H_f)$ and bounded from below. 
Its ionization threshold is defined as in the previous section on the Pauli-Fierz model,
\begin{equation*}
\Sigma = \lim_{R \rightarrow \infty} \left( \inf_{\psi \in D_R} \langle \psi, H \psi \rangle \right)
\end{equation*}
with $D_R \subset\HH$ the set of normalized states in $D(H)$ supported outside of the ball $B(0,R)\subset \R^{3N}$. We set $P_\lambda \coloneqq \rchi_{(- \infty, \lambda]}(H)$. It then follows for $\psi \in \mathrm{Ran} \, P_\lambda$ with $\lambda < \Sigma$ that
\begin{align}
\int e^{2 (1 - \varepsilon) \sqrt{\Sigma - \lambda} |x|} \, | \psi(x) |^2\, dx < \infty, \qquad \text{all}\ \varepsilon > 0.
\label{exponential bound nelson}
\end{align}
For the proof see Theorem 1 in \cite{G}. 


%-------------------------------------------------------------------------------------------------------------------------
\section{Estimating commutators} 

The main purpose of this section is to estimate  - for eigenvectors of the Pauli-Fierz Hamiltonian $H^{-}$ \eqref{Pauli-Fierz} - the local $L^{p_0}$-norm in \Cref{local pointwise bounds} in terms of a local $L^2$-norm, see \Cref{p0 leq 2} below. To this end, it is convenient to introduce the operator $\Pi$ with components 
\begin{align*} 
\Pi_m = - i \partial_m + \sqrt{\alpha} A_m, \quad m = 1, ..., 3N.
\end{align*} 
Then $H$ can be written as
\begin{align*} 
H = \Pi^2 + \sqrt{\alpha} \, \sigma \cdot B + V + H_f.
\end{align*} 
As a preparation we need the following lemma.
%-------------------------------------------------------------------------------------------------------------------------------------------------------------------
\begin{Lemma}\label{Kommutator Lemma}
Let $\gamma \in C^2(\R^{3 N})$ be a bounded function with bounded derivatives up to order $2$. Then
\begin{align}
[\Pi_m, \gamma ] &= -i (\partial_m \gamma) \quad \hspace{19mm} \text{on} \: D(\partial_m) \cap D(A_m)
\label{Kommutator Pi}  \\
[H, \gamma] &= -(\Delta \gamma) - 2 i (\nabla \gamma) \cdot \Pi \quad \text{on} \: D(H).
\label{Kommutator von H}
\end{align}
\end{Lemma}

\begin{proof} 
By the product rule $\gamma D(\partial_m) \subset D(\partial_m)$ and $[\partial_m, \gamma ] = (\partial_m \gamma)$ on $D(\partial_m)$. Since $\gamma$ commutes with $A_m$, it follows that $\gamma D(A_m) \subset D(A_m)$ and $[A_m, \gamma] = 0$ on $D(A_m)$. This proves (\ref{Kommutator Pi}).  We have $\gamma D(- \Delta) \subset D(- \Delta)$ and $\gamma$ commutes with $\sigma, B, V$ and $H_f$. Therefore $D(H) = D(-\Delta) \cap D(H_f)$ is invariant under $\gamma$. 
By \eqref{Kommutator Pi} it follows that
\begin{align*}
[H,\gamma] &= [\Pi^2, \gamma ] =  \Pi \, [\Pi, \gamma] + [\Pi, \gamma] \, \Pi  \\
&= [\Pi, -i(\nabla \gamma)] - 2 i (\nabla \gamma) \cdot \Pi  = - (\Delta \gamma) - 2 i (\nabla \gamma) \cdot \Pi. \qedhere
\end{align*} 
\end{proof}
%-------------------------------------------------------------------------------------------------------------------------------------------------------------

Before proving the next lemma we note that if $C$ is a closed operator and $H$ a self-adjoint operator with $D(C) \supset D(H)$, then $C$ is bounded relative to $H$. Indeed, $C (H + i)^{-1}$ is a closed (hence bounded) operator on $\HH$ by the closed graph theorem. 



%-------------------------------------------------------------------------------------------------------------------------------------------------------------
\begin{Lemma}
\label{lokalisations lemma}
Let $H^{-}$ be the Pauli-Fierz Hamiltonian. Let $\gamma \in C^\infty(\R^{3 N}; \R)$ be a bounded function with bounded derivatives of all orders. Suppose that $\mathrm{supp} \, \gamma$ is contained in some measurable  set $B \subset \R^{3 N}$. For $x \in \R^{3 N}$ define $\gamma_x(y) = \gamma( y - x)$ and $B_x = B + x$.
\begin{enumerate}
\item[(i)] If $\lambda < \infty$, then for every $\tau < 1$ there exists a constant $C_\tau$ such that
\begin{align*}
\| H \gamma_x P_\lambda^{-} \|  \leq C_\tau \| \rchi_{B_x} P_\lambda^{-}  \|^\tau \all x \in \R^{3 N}.
\end{align*}
Here $\| \cdot \|$ is the operator norm in $\mathcal{L}(\HH^{-}, \HH)$.
\item[(ii)] If $\psi$ is a normalized eigenvector of $H^{-}$, then for every $\tau < 1$ there exists a constant $C_\tau$ such that
\begin{align*}
\| H \gamma_x \psi \| \leq C_\tau \| \rchi_{B_x} \psi \|^\tau \all x \in \R^{3 N}.
\end{align*}
\end{enumerate}
For the Nelson model \eqref{Nelson} the same bounds hold with $\HH^{-}, H^{-}, P_\lambda^{-}$ replaced by $\HH, H, P_\lambda$.
\end{Lemma}

\begin{proof} We only prove (i) for the Pauli-Fierz Hamiltonian $H^{-}$. The proof of (ii) and the proofs for the Nelson model are similar and easier.  Let $\psi \in \mathrm{Ran} \, P_\lambda^-$ be normalized. From Lemma \ref{Kommutator Lemma} it follows that
\begin{align}
\| H \gamma_x \psi \| &= \| [H, \gamma_x] \psi + \gamma_x H \psi\| \nonumber \\
&\leq \| -(\Delta \gamma_x) \psi \| + 2  \| (\nabla \gamma_x)\cdot \Pi  \psi \| + \| \gamma_x H  \psi \| \nonumber \\
&= \| -(\Delta \gamma_x) P_\lambda^- \psi \| + 2  \| (\nabla \gamma_x)\cdot \Pi  \psi \| + \| \gamma_x P_\lambda^- H^- P_\lambda^-  \psi \| \nonumber \\
&\leq C_1 \| \rchi_{B_x} P_\lambda^- \| + 2  \| (\nabla \gamma_x)\cdot \Pi  \psi \|
\label{H gamma_x psi}
\end{align}
for some constant $C_1$. We used that $\mathrm{supp} \, \gamma_x \subset B_x$ and $H^- P_\lambda^-$ is bounded. By the Cauchy-Schwarz inequality in $\C^{3 N}$, a commutator identity and (\ref{Kommutator Pi}) we have
\begin{align}
\| (\nabla \gamma_x)\cdot \Pi  \psi \|^2 &\leq \| |\nabla \gamma_x| \, \Pi \psi \|^2 \nonumber \\
   &=  \langle \psi, \Pi |\nabla \gamma_x|^2 \Pi \psi \rangle \nonumber \\
&= \frac{1}{2} \langle \psi, (\Pi^2 |\nabla \gamma_x|^2 + |\nabla \gamma_x|^2 \Pi^2) \psi \rangle - \frac{1}{2}  \langle \psi, [\Pi, [\Pi, |\nabla \gamma_x|^2 ]] \psi \rangle \nonumber \\
&=  \mathrm{Re}  \langle \psi, |\nabla \gamma_x|^2 \Pi^2  \psi \rangle + \frac{1}{2} \langle \psi, \Delta (|\nabla \gamma_x|^2) \psi \rangle \nonumber\\
&\leq  \mathrm{Re} \langle \psi, | \nabla \gamma_x |^2 (\Pi^2 - H) \psi \rangle + C_2 \| \rchi_{B_x} P_\lambda^- \|^2.
\label{nabla term}
\end{align}
In the last line we used $\Pi^2 = (\Pi^2 - H) + H$ and the same arguments as in (\ref{H gamma_x psi}). Next, we decompose the potential $V = V_{+} - V_{-}$ into positive and negative part so that $-V \leq V_{-}$. In combination with the lower bound \eqref{lower bound} we find
\begin{align*}  
\Pi^2 - H = -\left(\sqrt{\alpha} \, \sigma \cdot B + H_f \right) - V \leq \nu + V_{-}.
\end{align*} 
Since $\partial_m \gamma_x$ commutes with $\sigma, B$, $H_f$ it follows that
\begin{align} 
\mathrm{Re} \langle \psi, | \nabla \gamma_x |^2 (\Pi^2 - H) \psi \rangle &\leq \sum_m \langle (\partial_m \gamma_x) \psi, (\nu + V_{-}) (\partial_m \gamma_x) \psi \rangle \nonumber \\
&\leq \sum_m \| (\partial_m \gamma_x) P_\lambda^{-} \| \| (\nu + V_{-}) (\partial_m \gamma_x) P_\lambda^{-} \|.
\label{Pi^2 - H}
\end{align} 
Since $\psi\in \mathrm{Ran} \, P_\lambda^{-}$ was an arbitrary normalized vector, it follows from \eqref{H gamma_x psi}, \eqref{nabla term} and \eqref{Pi^2 - H} that
\begin{align}
\| H \gamma_x P_\lambda^{-} \| \leq C_3 \left( \| \rchi_{B_x} P_\lambda^- \| +  \| \rchi_{B_x} P_\lambda^- \|^{1/2} \sum_m \| V_{-} (\partial_m \gamma_x) P_\lambda^- \|^{1/2}  \right). 
\label{rekursions ugl}
\end{align}
If $V_{-}$ is bounded, then the lemma follows with $\tau = 1$. In the general case, where $V_{-}$ may be unbounded, we use that $D(V_{-}) \supset D(H)$. Therefore $V_{-}$ is bounded relative to $H$ and by enlarging the constant $C_3$, it follows that
\eqref{rekursions ugl} holds with $V_{-}$ replaced by $H$. Iterating this inequality $n$ times, using $\left( \sum_m a_m \right)^{1/2} \leq \sum_m a_m^{1/2}$ in each step, we arrive at
\begin{equation*}
\| H \gamma_x P_\lambda^{-} \| \leq C_4 \left( \| \rchi_{B_x} P_\lambda^- \| +  \| \rchi_{B_x} P_\lambda^- \|^{1 - 1/{2^n}} \sum_{m_1, ..., m_n} \| H (\partial_{m_n} ... \partial_{m_1} \gamma_x) P_\lambda^- \|^{1/{2^n}} \right).
\end{equation*}
By \eqref{rekursions ugl},  the factor $\|H (\partial_{m_n} ... \partial_{m_1} \gamma_x) P_\lambda^- \|$ is bounded uniformly in $x$. Hence
\begin{align*}
\| H \gamma_x P_\lambda^{-} \| &\leq C_n \| \rchi_{B_x} P_\lambda^- \|^{1 - 1/{2^n}} \all x \in \R^{3 N}.  \quad  \qedhere
\end{align*}
\end{proof}
%----------------------------------------------------------------------------------------------------------------------------




%---------------------------------------------------------------------------------------------------------------------------
\begin{Corollary}
\label{p0 leq 2}
Let $\psi$ be a normalized eigenvector for the Pauli-Fierz Hamiltonian $H^{-}$. Then there exists a number $p_0 > 2$ such that the following holds: For all $0 < r_0 < R$, $\tau < 1$ there exists a constant $C$ such that
\begin{align*} 
\| | \psi | \|_{p_0, B(x, r_0)} \leq C \| \rchi_{B(x, R)} \psi \|^\tau \all x \in \R^{3 N}.
\end{align*} 
\end{Corollary}
\begin{proof}
Let $p_0 = 6 N / (3 N - 2) > 2$. We choose a function $\gamma \in C_0^\infty(\R^{3 N}; \, [0, 1])$ with $\gamma = 1$ in $B(0, r_0)$ and $\mathrm{supp} \, \gamma \subset B(0,R)$. We use the notation $\gamma_x = \gamma( \cdot - x)$ introduced in \Cref{lokalisations lemma}. From the Galiardo-Nirenberg Sobolev inequality (see Theorem 2.44 in \cite{FLW}),
\begin{align*}
\| u \|_{p_0} \leq C_1 \| \nabla u \|_2, \quad u \in H^1(\R^{3 N}),
\end{align*}
in conjunction with the diamagnetic inequality (Lemma \ref{loc lemma}) it follows that
\begin{align*}  
 \| | \psi | \|_{p_0, B(x, r_0)} &\leq \| | \gamma_x \psi| \|_{p_0} \leq C_1 \| \nabla | \gamma_x \psi | \|_2 \leq C_1 \| \nabla( \gamma_x \psi ) \| 
\end{align*} 
for all $x \in \R^{3 N}$.  From $D(- \Delta) \supset D(H)$ it follows that $\nabla$ is bounded relative to $H$. Hence
\begin{align*}  
 \| | \psi | \|_{p_0, B(x, r_0)} \leq C_2 (\| H \gamma_x \psi \| + \| \gamma_x \psi \|)
\end{align*} 
for all $x \in \R^{3 N}$. In view of \Cref{lokalisations lemma} (ii) this concludes the proof.
\end{proof}
%------------------------------------------------------------------------------------------------------------------



%-------------------------------------------------------------------------------------------------------------------------------------------

\section{Pointwise bounds on eigenvectors}
\label{sec:qft-exp}

In this section we prove a local pointwise bound for eigenstates of the Nelson and Pauli-Fierz models. When combined with the known $L^2$-exponential bounds \eqref{exponential bound nelson} and \eqref{L 2 exponential decay}, a pointwise exponential bound with the same decay rate follows. 

%--------------------------------------------------------------------------------
\begin{Theorem}
\label{thm:ptw-bound}
Let $\psi$ be a normalized eigenvector for the Pauli-Fierz Hamiltonian \eqref{Pauli-Fierz}.  Then for every $\tau < 1$  there exists a constant $C_\tau$ such that
\begin{align}
\esssup_{y \, \in B(x, 1/2)} | \psi(y) | \leq  C_\tau \| \rchi_{B(x, 1)} \psi \|^\tau
\label{main bound}
\end{align}
for all $x \in \R^{3 N}$. For the Nelson model \eqref{Nelson} the same bound holds with $\tau=1$.
\end{Theorem}

\noindent
\emph{Remark:} For Schr\"odinger operators there is an analog result with $\tau = 1$ due to Agmon
(see Theorem 5.1 in \cite{A}). 

\begin{proof} 
Let us begin with the Nelson model \eqref{Nelson}. Suppose $H \psi = \lambda \psi$. Then 
$\psi \in D(H) \subset  D(- \Delta) = H^2(\R^{3 N}; \, \HH')$ and
\begin{align*}  
   - \Delta \psi = (- \Phi - H_f - V + \lambda) \psi.
\end{align*} 
We decompose the potential $V = V_{+} - V_{-}$ into positive and negative part so that $-V \leq V_{-}$. When combined with the lower bound \eqref{lower bound nelson} it follows that
\begin{align*} 
\mathrm{Re} ( \psi, - \Delta \psi) &= - (\psi, \Phi \psi + H_f \psi) + (- V + \lambda ) |\psi|^2 \\
&\leq (\nu + V_{-} + \lambda_{+}) | \psi |^2
\end{align*}
pointwise on $\R^{3 N}$. By \Cref{local pointwise bounds} and the subsequent remark, the bound \eqref{main bound} with $\tau = 1$ now follows. 

Next, consider the Pauli-Fierz model \eqref{Pauli-Fierz} and suppose $H^{-} \psi = \lambda \psi$. In view of \Cref{p0 leq 2} it suffices to verify that $\psi$ satisfies the assumptions of \Cref{local pointwise bounds}. From $D(H^{-}) \subset  D(- \Delta)$ it follows that $\psi \in H^2(\R^{3 N}; \, \HH')$. Using \Cref{coulomb gauge} we can rewrite the eigenvalue equation
\begin{align*} 
\left((-i \nabla + \sqrt{\alpha} A)^2 + \sqrt{\alpha} \sigma \cdot B + V + H_f\right) \psi = \lambda \psi
\end{align*} 
in terms of 
\begin{align*} 
- \Delta \psi =  (2 i \sqrt{\alpha} A \cdot \nabla - \alpha A^2 - H_f - \sqrt{\alpha} \, \sigma \cdot B  - V + \lambda) \psi.
\end{align*} 
We now use $2 a b - a^2 \leq b^2$ in combination with the lower bound \eqref{lower bound} to obtain
\begin{align*}
\mathrm{Re} \, ( \psi, - \Delta \psi) &= 2 \mathrm{Re} \, (\sqrt{\alpha} A \psi, i \nabla \psi) - | \sqrt{\alpha} A \psi |^2 - (\psi, (H_f + \sqrt{\alpha} \, \sigma \cdot B) \psi ) + (- V + \lambda) |\psi|^2 \\
&\leq |\nabla \psi|^2 + (\nu + V_{-} + \lambda_{+}) \, |\psi|^2
\end{align*}
pointwise on $\R^{3 N}$. The theorem now follows from \Cref{local pointwise bounds} and \Cref{p0 leq 2}.
\end{proof}
%---------------------------------------------------------------------------------------------------------------------------------

%---------------------------------------------------------------------------------------------------------------------------------
Theorem \ref{thm:ptw-bound} allows us to pass from $L^2$ exponential bounds to $L^\infty$ exponential bounds:

\begin{Corollary}
\label{anisotrope Schranken}
Let $\psi$ be a normalized eigenvector for the Pauli-Fierz or Nelson Hamiltonian and suppose that
\begin{align*} 
\intd x \, e^{2 (1-\varepsilon)f(x)}  |\psi(x)|^2 < \infty, \qquad \text{all}\ \varepsilon>0,
\end{align*} 
with some Lipschitz continuous function $f : \R^{3 N} \rightarrow [0, \infty)$. Then for every $\varepsilon>0$ there exists a constant $C_\varepsilon$ such that for a.e. $x \in \R^{3 N}$,
\begin{align*} 
| \psi(x) | \leq C_\varepsilon e^{ -(1-\varepsilon)f(x)}. 
\end{align*} 
\end{Corollary}

\begin{proof} 
Let $\kappa:=(1-\varepsilon) < 1$. Choose $\tau < 1$ such that $\kappa/\tau < 1$ and let $C_\tau$ be given by Theorem \ref{thm:ptw-bound}. Denoting with $L$ the Lipschitz constant of $f$, it follows that
\begin{align*} 
\esssup_{y \, \in B(x, 1/2)} e^{\kappa f(y)} | \psi(y) | &\leq C_\tau e^{\kappa L/2} e^{\kappa f(x)}  \| \rchi_{B(x, 1)} \psi \|^\tau \\
&= C_\tau e^{\kappa L/2} \| e^{\kappa/\tau f(x)} \rchi_{B(x, 1)} \psi \|^\tau \\
&\leq C_\tau e^{\kappa L/2} \| e^{\kappa/\tau (L + f)} \psi \|^\tau < \infty
\end{align*} 
for all $x \in \R^{3 N}$. Note that the right-hand side does not depend on $x$.
\end{proof}
%---------------------------------------------------------------------------------------------------------------------------------------


\Cref{anisotrope Schranken} with $f(x) = \sqrt{\Sigma - \lambda}\,|x|$ in conjunction with the $L^2$ exponential bounds \eqref{L 2 exponential decay}, \eqref{exponential bound nelson} yields the following theorem:

%--------------------------------------------------------------------------------------------------------------------------------------
\begin{Theorem}
\label{pointwise bounds on bound states}
Let $\psi$ be a normalized eigenvector for the Pauli-Fierz or Nelson Hamiltonian with eigenvalue $\lambda$ below the ionization threshold $\Sigma$. Then for every $\varepsilon >0$ there exists a constant $C_\varepsilon$ such that for a.e. $x \in \R^{3 N}$, 
\begin{align*}
| \psi(x) | \leq C_\varepsilon \, e^{-(1-\varepsilon)\sqrt{\Sigma - \lambda}\, | x | }. 
\end{align*}
\end{Theorem}



%------------------------------------------------- Sec : Pointwise bounds on one-particle densities---------------------------------------------------------------------------------

\section{Pointwise bounds on one-particle densities}
\label{particle density section}

We consider the Nelson and Pauli-Fierz models \eqref{Nelson}, \eqref{Pauli-Fierz}. For $\psi \in \HH \, \widetilde{=} \, L^2(\R^{3 N}; \, \HH')$ we define $\rho_\psi \in L^1(\R^3)$ by $\rho_\psi(x_1) = | \psi(x_1) |^2$ if $N = 1$ and otherwise by
$$
  \rho_\psi (x_1) = \int \hspace{-1.5mm} \, dx_2\ldots dx_N |\psi(x_1, x_2, \ldots, x_N)|^2, \quad x_1 \in \R^3.
$$
If $\psi$ is normalized, then $\rho_\psi$ is said to be the \emph{one-particle density} associated to $\psi$. In this section we establish pointwise exponential decay for one-particle densities associated to states $\psi$ with energy distribution strictly below the ionization threshold. This result is based on a combination of the $L^2$-exponential bounds \eqref{L 2 exponential decay}, \eqref{exponential bound nelson} with the Sobolev-type estimate of \Cref{sobolev lemma}; it does not require $\psi$ to be an eigenstate.

%--------------------------------------------------------------------------------------------------------------------
\begin{Lemma}
There exists a constant $B$ such that
\begin{align} 
 \rho_\psi \in C(\R^3) \quad \mathrm{and} \quad \|\rho_\psi \|_\infty \leq B \, \| \psi \|_{H^2}^2 
 \label{density embedding}
 \end{align} 
for all $\psi \in H^2(\R^{3 N}; \, \HH')$. 
\label{sobolev lemma}
\end{Lemma}

\emph{Remark: } If $N = 1$, then $H^2(\R^3)$ and hence $H^2(\R^3; \, \HH')$ is embedded in the space of H\"older continuous functions of order $1/2$.
\begin{proof}
Notice that it suffices to prove \eqref{density embedding} on a dense subspace of $H^2(\R^{3 N}; \, \HH')$ and 
recall that vectors of the form
\begin{align*} 
\psi(x) = \sum_{\mathrm{finite}} f_k(x) v_k, \quad f_k \in C_0^\infty(\R^{3 N}), \, v_k \in \HH',
\end{align*} 
are dense in $H^2(\R^{3 N}; \, \HH')$, by \Cref{meyers serrin}. Without loss of generality we assume that the $v_k$'s are orthogonal. Then
\begin{equation}\label{dem1} 
   \rho_{\psi}(x_1) = \sum_\mathrm{finite} \rho_{f_k}(x_1) | v_k |^2.
\end{equation} 
From the assumption on $f_k$ it follows that $\rho_{f_k} \in C(\R^3)$ and hence $\rho_{\psi} \in C(\R^3)$. By the embedding $H^2(\R^3) \rightarrow L^{\infty}(\R^3)$ there exists a constant $B$ such that
\begin{equation}\label{dem2}  
\| \rho_{f_k} \|_\infty \leq B \| f_k \|_{H^2(\R^{3 N})}^2.
\end{equation} 
The bound \eqref{density embedding} now follows from \eqref{dem1}, \eqref{dem2} and the orthogonality of the $v_k$'s.
\end{proof}
%----------------------------------------------------------------------------------------


%-------------------------------------------------------------------------------------------------------------------------
\begin{Proposition}
\label{pointwise bounds one particle}
Let $H^{-}$ be the Pauli-Fierz Hamiltonian. If $\lambda < \infty$ and $\psi \in \mathrm{Ran} \, P_\lambda^{-}$ is normalized, then the one-particle density $\rho_\psi$ is continuous and for every $\tau<1$ there exists a constant $C_\tau$ such that 
$$
    \sup_{y_1 \in B(x_1,1/2)} \rho_{\psi}(y_1) \leq C_\tau \| \rchi_{B(x_1, 1)} \otimes 1 \,  P_\lambda^{-}  \|^{2 \tau}.
$$
for all $x_1 \in \R^3$. The same bound holds for the Nelson model.
\end{Proposition}
\begin{proof}
Continuity follows from $\psi \in D(H^-)\subset D(-\Delta) = H^2(\R^{3 N}; \, \HH')$ and \Cref{sobolev lemma}. Let $\gamma \in C_0^\infty(\R^3; [0,1])$ with $\mathrm{supp} \, \gamma \subset B(0, 1/2)$ and $\gamma = 1$ in $B(0,1)$. We define for $x \in \R^{3 N}$ the function $\Gamma_x(y) = \gamma( y_1 - x_1), \, y \in \R^{3 N}$. By Lemma \ref{sobolev lemma} and $D(\partial^\alpha) \supset D(H)$ for all $| \alpha | \leq 2$ there exists a constant $B'$ such that 
\begin{equation}\label{ptw-dichte1} 
  \sup_{y_1 \in B(x_1,1/2)} \rho_{\psi}(y_1) \leq \sup_{y_1 \in \R^3} \rho_{(\Gamma_x \psi)} (y_1) \leq B'(\|H \Gamma_x \psi \|^2 + \| \Gamma_x \psi \|^2),
 \end{equation} 
for all $ x \in \R^{3 N}$. By Lemma \ref{lokalisations lemma} (i) we find a constant $C$ such that
\begin{equation}\label{ptw-dichte2} 
     \| H \Gamma_x \psi \|^2 + \| \Gamma_x \psi \|^2 \leq C \| \rchi_{B(x_1,1)} \otimes 1 \, P_\lambda^{-}  \|^{2 \tau}
\end{equation}
for all $x \in \R^{3 N}$. Notice that $\| \Gamma_x P_\lambda^{-} \| \leq \| \gamma \|_\infty \| \rchi_{B(x_1,1)} \otimes 1 \, P_\lambda^{-}  \|^{\tau}$ because $\mathrm{supp} \, \Gamma_x \subset B(x_1, 1) \times \R^{3(N - 1)}$ and $\tau < 1$. Combining \eqref{ptw-dichte1} and \eqref{ptw-dichte2} the proposition follows.
\end{proof}
%--------------------------------------------------------------------------------------------------------------------------------


%----------------------------------------------------------------------------------------------------------------------------
\begin{Theorem} 
Consider the Pauli-Fierz model \eqref{Pauli-Fierz}. Let $\psi \in \mathrm{Ran} \, P_\lambda^{-}$ be normalized with total energy $\lambda$ below the ionization threshold $\Sigma$. Then the one-particle density $\rho_\psi$ 
is continuous and for every $\varepsilon>0$ there exists a constant $C_\varepsilon$ such that for all $x_1 \in \R^3$,
\begin{align*}
 \rho_\psi(x_1) \leq C_\varepsilon \,  e^{- 2(1-\varepsilon)\sqrt{\Sigma - \lambda} \, | x_1 |} 
\end{align*}
If $N = 1$, then $\psi$ is H\"older continuous of order $1/2$. Analogous statements apply to the Nelson model \eqref{Nelson}.
\label{Punktweiser Abfall Teilchenzahldichte}
\end{Theorem}

\begin{proof}
Let $L \coloneqq 2 (1 - \varepsilon) \sqrt{\Sigma - \lambda}$. Choose $\tau < 1$ such that $(1 - \varepsilon)/\tau < 1$ and let $C_\tau$ be given by \Cref{pointwise bounds one particle}. Then
\begin{align*}
\sup_{y_1 \in B(x_1, 1/2)} e^{L |y_1|} \rho_\psi(y_1) &\leq e^{L (|x_1| + 1/2)} C_\tau \| \rchi_{B(x_1, 1)} \otimes 1 \, P_\lambda^{-}  \|^{2 \tau} \\
&= e^{L/2}  C_\tau \, \| e^{L/2\tau |x_1|} \rchi_{B(x_1,1)} \otimes 1 \, P_\lambda^{-} \|^{2 \tau} \\
&\leq e^{L/2} C_\tau \| e^{L/2\tau(1 + | \, \cdot \, |)}  P_\lambda^{-} \|^{2 \tau}.
\end{align*}
Note that the right-hand side is independent of $x_1$. By \eqref{L 2 exponential decay} and \eqref{exponential bound nelson} the operator inside the norm is well defined on $\mathcal{H}^{-}$ and closed.  Therefore the operator has finite norm. 
\end{proof}
%-----------------------------------------------------------------------------------------------------------------------------------------



\appendix
\section{Appendix}
\label{sec:appendix}

In this appendix we collect technical results and estimates to which we referred in the main part of the paper.



\begin{Lemma}
$\mathrm{(Product \, rule)}$ Let $\Omega \subset \R^n$ be open. Suppose that $u, v \in L_\mathrm{loc}^1(\Omega)$ are weakly differentiable. If $u v \in L_\mathrm{loc}^1(\Omega)$ and $u (\nabla v) + (\nabla u) v \in L_\mathrm{loc}^1(\Omega)$, then $u v$ is weakly differentiable and
\begin{align*}
\nabla ( u v ) = u (\nabla v) + (\nabla u) v.
\end{align*} 
\label{Produktregel}
\end{Lemma}
\vspace{-5mm}
For the proof see Lemma 2.14 in  \cite{FLW}. In our applications of this result, one of the two factors $u,v$ is bounded, which is the easier case established first in \cite{FLW}.


\begin{Lemma}
$\mathrm{(Chain \, rule)}$   Let $f_1, f_2, f_3$ be $C^1$-functions on the intervals $(- \infty, a], [a, b], [b, \infty)$ and suppose that $f_1', f_2', f_3'$ are bounded. Suppose that $f_1(a) = f_2(a)$ and $f_2(b) = f_3(b)$. Let the function $f$ be of the form
\begin{align*}
f(t) = 
\begin{cases}
f_1(t) \quad t \leq a \\
f_2(t) \quad a \leq t \leq b \\
f_3(t) \quad t \geq b.
\end{cases}
\end{align*}
Let $\Omega \subset \R^n$ be open. Suppose that $u \in L_\mathrm{loc}^1(\Omega)$ is real-valued and weakly differentiable. Then the composition $f(u)$ is weakly differentiable and
\begin{align*}
\nabla f(u) = 
\begin{cases}
f_1'(u) \nabla u \quad &u < a \\
f_2'(u) \nabla u \quad &a < u < b \\
f_3'(u) \nabla u \quad &u > b \\
0 \quad & u \in \{a, b\}.
\end{cases}
\end{align*}
\label{Kettenregel}
\end{Lemma}
For the proof see Theorem 7.8 in \cite{GT}.


\begin{Lemma}
Let $\Omega \subset \R^n$ be open and $p > 1$. Suppose that $0 \leq u \in L_\mathrm{loc}^1(\Omega)$ is weakly differentiable. If $u^p \in L_\mathrm{loc}^1(\Omega)$ and $u^{p - 1} \nabla u \in L_\mathrm{loc}^1(\Omega)$, then $u^{p}$ is weakly differentiable and
\begin{align*}
\nabla u^p = p u^{p - 1} \nabla u.
\end{align*}
\label{u^p Kettenregel}
\end{Lemma}
\vspace{-5mm}
\begin{proof}  Taken from Corollary 5.6 in \cite{A}: For $R > 0$ define
\begin{align*}
f_R(t) = \begin{cases}
t^p \quad & 0 \leq t \leq R \\
R^p \quad & t \geq R.
\end{cases}
\end{align*}
From Lemma \ref{Kettenregel} it follows that $\nabla f_R(u) = p u^{p - 1} \nabla u \rchi_{\{u < R \}}$. By dominated convergence it follows that
\begin{align*}
- \intd x \nabla \gamma u^p &= \lim_{R \rightarrow \infty} - \intd x \, \nabla \gamma f_R(u) \\
&= \lim_{R \rightarrow \infty} \, \intd x \, \gamma \, p u^{p - 1} \nabla u \rchi_{\{u < R \}} \\
&= \intd x \, \gamma \, p u^{p - 1} \nabla u \all \gamma \in C_0^\infty(\Omega). \qedhere
\end{align*}
\end{proof}


%--------------------------------------------------------------------------------------------------------------------------
\begin{Lemma}
Let $V : \R^{3 N} \rightarrow \R$ be of the form
\begin{align*}
V(x) = \sum_{j = 1}^N v_j(x_j) + \sum_{j < k} w_{j k}(x_j - x_k), \quad x = (x_1, ..., x_N) \in \R^{3 N},
\end{align*}
with $v_j,w_{j k} \in L^p(\R^3) + L^\infty(\R^3)$ for some $p > 3/2$.
\begin{enumerate}
\item[(i)]  There exist $c > 0, \mu > 0$ such that
\begin{align*}
\| | V |^{1/2} u \|_2 \leq \varepsilon  
\| \nabla u \|_2 + c \, \varepsilon^{-\mu} \| u \|_2
\end{align*}
for all $u \in H^1(\R^{3 N})$ and sufficiently small $\varepsilon > 0$.
\item[(ii)] There exists a number $s \in (2, 4)$ such that for all $u \in H^1(\R^{3 N})$,
\begin{align*}
\intd x |V| |u|^{s} < \infty. 
\end{align*} 
\end{enumerate}
Statements (i) and (ii) remain valid if $V$ is changed by an additive constant. 
\label{Potential Epsilon Lemma}
\end{Lemma}

\begin{proof} 
(i) We write $V \in M_\delta(\R^{3 N})$ for some $\delta \in (0,1)$ if $V \in L_\mathrm{loc}^1(\R^{3 N})$ and
\begin{align*} 
\sup_{x \in \R^{3 N}} \int\displaylimits_{|y - x| \leq 1} \hspace{-5mm} d^{3 N}y \, |V(y)| |y - x|^{2 - 3 N - \delta} < \infty.
\end{align*}
In Lemma 4.7 in \cite{A} it is shown, that $V \in M_\delta(\R^{3 N})$, if $v_j,w_{j k} \in M_\delta(\R^3)$. We now verify the latter for small $\delta$: We define $q$ by $1/p + 1/q = 1$. Then $q < 3$. From H\"older's inequality we find, writing $v_j = v_p + v_\infty \in L^p(\R^3) + L^\infty(\R^3)$, that
 \begin{align*}  
 \sup_{x \in \R^3} \int\displaylimits_{|y - x| \leq 1} \hspace{-5mm} d^3y \,  |v_j(y)| |y - x|^{2 - 3 - \delta} \leq \| v_\infty \|_\infty &\int\displaylimits_{|y| \leq 1} \hspace{-3mm} d^3y \, |y|^{-(1 + \delta)} \\ 
 &+ \|v_p\|_p \left(\int\displaylimits_{\, | y | \leq 1}\hspace{-3mm} d^3y \, |y|^{-q(1 + \delta)} \right)^{1/q} \hspace{-3mm}.
 \end{align*}
Therefore $V \in M_\delta(\R^{3 N})$ for some $\delta \in (0, 1)$. It then follows from Theorem 7.3 in \cite{S} that there exists a constant $C_1$ such that
\begin{align} 
\| | V |^{1/2} \gamma \|_2 &\leq C_1 (\| (1 - \Delta)^{\theta/2} \gamma \|_2 \all \gamma \in C_0^\infty(\R^{3 N}), \label{U1}
\end{align} 
where $\theta = 1 - \delta/2 \in (0,1)$. By Lemma 5.7 in \cite{A} there exists for every $\theta \in (0,1)$ a constant $C_2$ and a number $r \in (1,2)$ such that
\begin{align} 
\| (1 - \Delta)^{\theta/2} \gamma \|_2 &\leq C_2 (\| \nabla \gamma \|_r + \| \gamma \|_r ), \label{U2}\\ 
\| (1 - \Delta)^{\theta/2} \gamma \|_2 &\leq \varepsilon \| \nabla\gamma \|_2 + C_2 (\varepsilon^{-\theta/(1 - \theta)} + \varepsilon) \| \gamma \|_2 \label{U3}
\end{align} 
for all $\gamma \in C_0^\infty(\R^{3 N})$ and $\varepsilon > 0$. Combining \eqref{U1} and \eqref{U3} (i) follows. To verify (ii) we follow Agmon's argument in the proof of Theorem 5.1 in \cite{A}:  From \eqref{U1}, \eqref{U2} we find that
\begin{align} 
\| | V |^{1/2} u \|_2 \leq C_3 ( \| \nabla u \|_r + \| u \|_r ) \all u \in W^{1,r}(\R^{3 N}).
\label{U4}
\end{align} 
For $\gamma \in C_0^\infty(\R^{3 N})$ we set $u = |\gamma|^{2/r}$. Then $u \in C_0^1(\R^{3 N})$ and $|\nabla  u  |\leq 2/r  \, |\gamma |^{2/r - 1} | \nabla \gamma|$. From \eqref{U4} and H\"older's inequality it follows that
\begin{align*}
\| |V|^{1/2} | \gamma |^{2/r} \|_2 &\leq C_3 \left(  \, \frac{2}{r} \| |\gamma|^{2/r - 1}  |\nabla \gamma | \|_r + + \| |\gamma|^{2/r} \|_r \right) \\ 
&\leq C_3 \left( \, \frac{2}{r} \| \gamma \|_2^{(2 - r)/r} \| \nabla \gamma \|_2 + \| \gamma \|_2^{2/r} \right).
\end{align*}
The inequality extends to all $\gamma \in H^1(\R^{3 N})$, so that
\begin{align*}
\intd x | V | | \gamma |^{4/r} &< \infty \all \gamma \in H^1(\R^{3 N}). \qedhere
\end{align*}
\end{proof} 
%----------------------------------------------------------------------------------------------------------------------------------------------
 

\begin{thebibliography}{10}

\bibitem{Adams}
Robert~A. Adams and John J.~F. Fournier.
\newblock {\em Sobolev spaces}, volume 140 of {\em Pure and Applied Mathematics
  (Amsterdam)}.
\newblock Elsevier/Academic Press, Amsterdam, second edition, 2003.

\bibitem{A}
Shmuel Agmon.
\newblock {\em Lectures on exponential decay of solutions of second-order
  elliptic equations: bounds on eigenfunctions of {$N$}-body {S}chr\"{o}dinger
  operators}, volume~29 of {\em Mathematical Notes}.
\newblock Princeton University Press, Princeton, NJ; University of Tokyo Press,
  Tokyo, 1982.

\bibitem{Ahlrichs}
Reinhart Ahlrichs.
\newblock Asymptotic behavior of atomic bound state wave functions.
\newblock {\em J. Mathematical Phys.}, 14:1860--1863, 1973.

\bibitem{AiSi}
M.~Aizenman and B.~Simon.
\newblock Brownian motion and {H}arnack inequality for {S}chr\"{o}dinger
  operators.
\newblock {\em Comm. Pure Appl. Math.}, 35(2):209--273, 1982.

\bibitem{AnaLew}
Ioannis Anapolitanos and Mathieu Lewin.
\newblock Compactness of molecular reaction paths in quantum mechanics.
\newblock {\em Arch. Ration. Mech. Anal.}, 236(2):505--576, 2020.

\bibitem{AnaSig}
Ioannis Anapolitanos and Israel~Michael Sigal.
\newblock Long-range behavior of the van der {W}aals force.
\newblock {\em Comm. Pure Appl. Math.}, 70(9):1633--1671, 2017.

\bibitem{BFS}
Volker Bach, J\"{u}rg Fr\"{o}hlich, and Israel~Michael Sigal.
\newblock Spectral analysis for systems of atoms and molecules coupled to the
  quantized radiation field.
\newblock {\em Comm. Math. Phys.}, 207(2):249--290, 1999.

\bibitem{E}
Lawrence~C. Evans.
\newblock {\em Partial differential equations}, volume~19 of {\em Graduate
  Studies in Mathematics}.
\newblock American Mathematical Society, Providence, RI, second edition, 2010.

\bibitem{FLW}
Rupert~L. Frank, Ari Laptev, and Timo Weidl.
\newblock {\em Schr\"{o}dinger operators: eigenvalues and {L}ieb-{T}hirring
  inequalities}, volume 200 of {\em Cambridge Studies in Advanced Mathematics}.
\newblock Cambridge University Press, Cambridge, 2023.

\bibitem{GT}
David Gilbarg and Neil~S. Trudinger.
\newblock {\em Elliptic partial differential equations of second order}.
\newblock Classics in Mathematics. Springer-Verlag, Berlin, 2001.
\newblock Reprint of the 1998 edition.

\bibitem{G}
M.~Griesemer.
\newblock Exponential decay and ionization thresholds in non-relativistic
  quantum electrodynamics.
\newblock {\em J. Funct. Anal.}, 210(2):321--340, 2004.

\bibitem{GLL}
Marcel Griesemer, Elliott~H. Lieb, and Michael Loss.
\newblock Ground states in non-relativistic quantum electrodynamics.
\newblock {\em Invent. Math.}, 145(3):557--595, 2001.

\bibitem{HH}
D.~Hasler and I.~Herbst.
\newblock On the self-adjointness and domain of {P}auli-{F}ierz type
  {H}amiltonians.
\newblock {\em Rev. Math. Phys.}, 20(7):787--800, 2008.

\bibitem{HiHi2010}
Takeru Hidaka and Fumio Hiroshima.
\newblock Pauli-{F}ierz model with {K}ato-class potentials and exponential
  decays.
\newblock {\em Rev. Math. Phys.}, 22(10):1181--1208, 2010.

\bibitem{Hi2002}
F.~Hiroshima.
\newblock Self-adjointness of the {P}auli-{F}ierz {H}amiltonian for arbitrary
  values of coupling constants.
\newblock {\em Ann. Henri Poincar\'{e}}, 3(1):171--201, 2002.

\bibitem{Hiro2019}
Fumio Hiroshima.
\newblock Pointwise exponential decay of bound states of the {N}elson model
  with {K}ato-class potentials.
\newblock In {\em Analysis and operator theory}, volume 146 of {\em Springer
  Optim. Appl.}, pages 225--250. Springer, Cham, 2019.

\bibitem{Kato57}
Tosio Kato.
\newblock On the eigenfunctions of many-particle systems in quantum mechanics.
\newblock {\em Comm. Pure Appl. Math.}, 10:151--177, 1957.

\bibitem{K}
Marcel Kreuter.
\newblock Sobolev spaces of vector-valued functions.
\newblock Master's thesis, Ulm University, 2015.

\bibitem{LL}
Elliott~H. Lieb and Michael Loss.
\newblock Existence of atoms and molecules in non-relativistic quantum
  electrodynamics.
\newblock {\em Adv. Theor. Math. Phys.}, 7(4):667--710, 2003.

\bibitem{LT1986}
Elliott~H. Lieb and Walter~E. Thirring.
\newblock Universal nature of van der waals forces for coulomb systems.
\newblock {\em Phys. Rev. A}, 34:40--46, Jul 1986.

\bibitem{RSI}
Michael Reed and Barry Simon.
\newblock {\em Methods of modern mathematical physics. {I}. {F}unctional
  analysis}.
\newblock Academic Press, New York-London, 1972.

\bibitem{S}
Martin Schechter.
\newblock {\em Spectra of partial differential operators}.
\newblock North-Holland Series in Applied Mathematics and Mechanics, Vol. 14.
  North-Holland Publishing Co., Amsterdam-London; American Elsevier Publishing
  Co., Inc., New York, 1971.

\end{thebibliography}



%\bibliography{pointwise_refrences}
%\bibliographystyle{plain}

\end{document}



 
      




