%
%.  Version of June 7, 2024 with unnecessary (commented out) parts removed
%
%%%%%%%%%%%%%%%%%%%%%%%%%%%%%%%%%%%%%%%%%%%%%%%


\documentclass[11pt,twoside]{article}

\usepackage{mathtools} %Für := (\coloneqq)

\usepackage{mathrsfs} % used for mathscr

\newcommand{\FF}{\mathcal{F}}
\newcommand{\HH}{\mathscr H}
\newcommand{\LL}{\mathscr L}

%%%%%%%%%%%%%%%%%%%%%%%%%%%%%%%%%%%%%%%%%%%%%%%%%%%


%\usepackage[ansinew]{inputenc} %%Siehe UTF-8
\usepackage{amsmath,amssymb,amsthm}
\usepackage{pstricks,pst-node,pst-coil,pst-plot,pstricks-add}
\usepackage{geometry,epsfig}
\usepackage{bbm}
\usepackage{cite}
%\usepackage{graphicx}
%\usepackage{thmbox}

\usepackage{cite}
\usepackage{paralist}
\usepackage{cleveref}
\usepackage{enumitem}
%\usepackage{menukeys}
\usepackage{emptypage}


\DeclareGraphicsExtensions{.pdf}
%\usepackage{floatflt}
%\setlength{\parindent}{0mm}

\setlength{\oddsidemargin}{7mm} \setlength{\evensidemargin}{7mm}
\setlength{\topmargin}{-10mm} %\setlength{\topmargin}{10mm}
\setlength{\textheight}{9in} \setlength{\textwidth}{6in} % originally 6.2in
\setlength{\topsep}{0.2in}
%\setlength{\parskip}{1ex}


%%%%%%%%%%%%%%%%%%%%%%%%%%%%%%%%%%%%%%%%%


\newtheorem{Lemma}{Lemma}[section]
\newtheorem{Theorem}[Lemma]{Theorem}
\newtheorem{Corollary}[Lemma]{Corollary}
\newtheorem{Proposition}[Lemma]{Proposition}


\newcommand{\C}{\mathbb{C}} % komplexe
\newcommand{\K}{\mathbb{K}} % komplexe
\newcommand{\R}{\mathbb{R}} % reelle
\newcommand{\Q}{\mathbb{Q}} % rationale
\newcommand{\Z}{\mathbb{Z}} % ganze
\newcommand{\N}{\mathbb{N}} % natuerliche
\DeclareMathOperator*{\esssup}{ess\,sup}
	

\DeclareRobustCommand{\rchi}{{\mathpalette\irchi\relax}}
\newcommand{\irchi}[2]{\raisebox{\depth}{$#1\chi$}} %richtiges Chi
\newcommand{\intd}{\int \hspace{-1.5mm} d} %\int dx richtig plaziert
\newcommand{\all}{\quad \text{for all} \: \,} %Abkürzung von für alle
\newcommand{\Ima}{\operatorname{Im}}
\newcommand{\Rea}{\operatorname{Re}}

\renewcommand{\baselinestretch}{1.1}
\renewcommand{\qedsymbol}{\rule{1.3mm}{2.6mm}}

%---------------------------------------------------------------------------------------------------------------------
%\font\notefont=cmsl8 \pagestyle{myheadings}
%\markright{\notefont  Pointwise bounds -- 6 June, 2024.\hfill}
%---------------------------------------------------------------------------------------------------------------------

\title{\textbf{Pointwise Bounds on Eigenstates\\ in Non-relativistic QED}}

\author{M.~Griesemer\footnote{marcel.griesemer@mathematik.uni-stuttgart.de}\,\ and V.~Ku{\ss}maul\footnote{valentin.kussmaul@mathematik.uni-stuttgart.de}\\  
\small Fachbereich Mathematik, Universit\"at Stuttgart, D-70569 Stuttgart, Germany}  
\date{}

\begin{document}
\maketitle

\begin{abstract}
In the present paper, Kato's distributional inequality with magnetic field is generalized to vector-valued functions and operator-valued vector potentials.
This result is then used in non-relativistic quantum electrodynamics (QED) to show that eigenstates of the Pauli-Fierz Hamiltonian satisfy a subsolution estimate, and hence that any $L^2$-exponential bound in terms of a Lipschitz function implies the corresponding pointwise exponential bound. Similar pointwise bounds are also established  for the one-particle density of states that are not eigenstates.       
\end{abstract}

\section{Introduction}

States of an atom or molecule with energy distribution strictly below the ionization threshold are well localized in a neighborhood of the nuclei: both in models of non-relativistic quantum mechanics and non-relativistic quantum field theory (QFT), the wave function decays exponentially as the distance $|x|$ of the electronic configuration $x=(x_1, \ldots,x_N)\in \R^{3N}$ from the nuclear positions grows. This decay implies an effective screening of the positive nuclear charges and it plays an important role in the mathematical analysis of many-particle quantum systems \cite{LT1986, AnaSig,AnaLew}. While this exponential decay is well understood and well established in non-relativistic quantum mechanics \cite{A}, in the standard model of non-relativistic QED (the Pauli-Fierz model) we know little more than an \emph{averaged} decay of the form 
\begin{equation}\label{L2-decay}
 \int e^{2 \beta |x|}|\psi(x)|^2\, dx <\infty
\end{equation}
for the wave function $\psi:\R^{3N}\to \HH'$ with $|\psi(x)|$ the norm of $\psi(x)\in\HH'$, $\HH'$ being the tensor product of spin and Fock space \cite{BFS98,G}. The rate of decay, $\beta>0$, is explicit and depends on the difference between ionization threshold and upper bound on the energy distribution of $\psi$. One expects that \eqref{L2-decay} implies the corresponding \emph{pointwise} decay at the same rate, but a general result of this type is not yet known, see \cite{HiHi2010}. In this paper we provide such a result for the case of $N = 1$ electron and for arbitrary $N\geq 1$ under the additional assumption that $\psi$ is an eigenstate. 

In non-relativistic quantum mechanics, pointwise exponential decay is derived from \eqref{L2-decay} by the following line of arguments \cite{A}:
Suppose $\psi$ is a solution to a Schr\"odinger equation $(-\Delta+V)\psi = 0$, where a possible eigenvalue $E\neq 0$ has been absorbed in $V$ and the Laplacian $-\Delta$ may contain a magnetic field. Then, by Kato's distributional inequality, $|\psi |$ is a subsolution to a simplified equation, 
\begin{equation} 
  -\Delta | \psi | \leq V_{-}|\psi|,
\label{def:subsolution}
\end{equation} 
where the positive part of $V_{+}$ of $V$ has been dropped and the magnetic field is gone. Under weak assumptions on $V_{-}$, \eqref{def:subsolution} implies a Harnack-inequality, or some subsolution estimate
\begin{equation}\label{Agmon}
    \esssup_{y\in B(x,1/2)}|\psi(y)| \leq C\bigg(\int_{B(x,1)}|\psi(y)|^2\, dy\bigg)^{1/2}
\end{equation}
for all $x\in \R^n$ \cite{A,AiSi}. It follows that $L^2$ exponential decay, such as \eqref{L2-decay}, with a Lipschitz function $x \mapsto \beta|x|$ implies the corresponding pointwise decay,
\begin{equation}\label{L-infty-bound}
     \esssup_{x\in \R^{n}} e^{\beta |x|}|\psi(x)| <\infty.
\end{equation}
The only missing ingredient to extend this line of arguments to the Pauli-Fierz model of non-relativistic QED is a suitable generalization of Kato's distributional inequality \cite{Kato72}, 
\begin{equation}\label{Kato}
   \Delta | \psi | \geq \mathrm{Re}\big[ \overline{\mathrm{sgn}\psi} (\nabla + i A)^2 \psi\big]
\end{equation}
to vector-valued wave functions $\psi$ and quantized vector-potentials $A$. In Section 2, below, we establish such a generalization. If $\psi$ is an eigenstate of the Pauli-Fierz Hamiltonian, the subsolution estimate \eqref{Agmon} then follows as a corollary, and hence \eqref{L2-decay} will imply \eqref{L-infty-bound}. A fortiori, analog results hold for the Nelson model \cite{pointwise23}. If $\psi$ is not an eigenvector but a more general state with energy distribution strictly below the ionization threshold, then we do not have a subsolution estimate such as \eqref{Agmon}, but we can still show, by exploiting the Sobolev embedding $H^2(\R^3)\to L^{\infty}(\R^3)$, that the one-particle density of $\psi$ exhibits pointwise exponential decay at the rate expected from the $L^2$ decay of $\psi$. The same remark applies to $|\psi|$ if $N=1$.

Eigenfunctions to many-particle Schr\"odinger equations are continuous, in fact uniformly H\"older continuous \cite{Kato57}, which means that  ``$\esssup$'' in \eqref{L-infty-bound} may be replaced by ``$\sup$''. H\"older continuity
alone can be used  to derive \emph{some} pointwise exponential decay from \eqref{L2-decay}, but the rate so obtained is worse than in \eqref{L2-decay}, see \cite{Ahlrichs}. We expect - but we do not have - a similar regularity result for eigenstates of the Pauli-Fierz model.  

In models of QFT, pointwise bounds similar to \eqref{L-infty-bound} (not related to an $L^2$ bound) with some $\beta>0$ were previously obtained by Hiroshima and Hidaka by means of path integral representations for the semi-group generated by the Hamiltonian \cite{Hiro2019,HiHi2010, Hi14}. Pointwise exponential decay for spectral subspaces in the renormalized Nelson model was established by Hiroshima and Matte \cite[Theorem 1.1]{HiMa22}. They consider the $N = 1$ case and obtain an explicit decay rate in terms of the ionization threshold. For results on $L^2$ exponential decay in pseudo-relativistic QFT models we refer to \cite{MS10}. For the existence of eigenstates in the Pauli-Fierz model we refer to \cite{BFS,GLL,LL} and the references therein. Our results concerning pointwise exponential decay have previously been established in the preprint \cite{pointwise23}. In the present paper, the proofs are simplified with the help of the new version of Kato's distributional inequality.  Some diamagnetic inequalities for the Pauli-Fierz Hamiltonian with and without dipole approximation were previously derived by Hiroshima \cite{Hiro1996,Hiro1997}. These are weaker than ours because they contain the free field energy on one side.  

Kato's distributional inequality was originally conceived to prove essential self-adjointness for (magnetic) Schr\"odinger operators with singular potentials \cite{Kato72}. For the inequality to be useful to that end, it must be established under minimal assumptions \cite[Lemma A]{Kato72}. In the context of QED, this seems difficult. Another important consequence of Kato's inequality is the fact that, for spinless bosons and boltzons, the ground state energy with (classic) magnetic field is bounded below by the ground state energy without magnetic field \cite{Si76}. An analog result in QED now easily follows from our generalized version of Kato's inequality, see \Cref{lower bound on ground state energy}.

%------------------------------------------------------------------------------------------------------------------

This paper is organized as follows. In \Cref{Kato sec} we establish Kato's distributional inequality in the generalized form mentioned above, see \Cref{Kato ineq}. The Pauli-Fierz model is introduced in \Cref{sec:qft}, and \Cref{sec:qft-exp} contains our results on eigenstates, \Cref{thm:subsolution} and \Cref{pointwise bounds on bound states}. \Cref{sec:commutators} serves as a preparation for \Cref{particle density section} and can be omitted at first reading. Pointwise exponential decay for one-particle densities, \Cref{Punktweiser Abfall Teilchenzahldichte}, is proven in \Cref{particle density section}.
 
%---------------------------------------------------------------------------------------------------------------------------------------------------------------------

\section{Kato's distributional inequality}
\label{Kato sec}

Let $\HH'$ be a separable Hilbert space with inner product $(\cdot, \cdot)$ and norm $| \cdot |$. We denote by  $\HH$ the Hilbert space $L^2(\R^n; \, \HH')$. For each $m = 1, ..., n$ let the operator $A_m$ be given as a direct integral of (possibly unbounded) symmetric operators $(A_m(x))_{x \in \R^n}$ acting on $\HH'$, that is $A_m$ acts on $\HH$ according to
\begin{align*} 
(A_m \psi)(x) = A_m(x) \psi(x), \quad x \in \R^n. 
\end{align*} 
We denote by $p_m = - i \partial_{m}$ the $m$-th component of the momentum operator, defined on it's natural domain of self-adjointness. Moreover, we set
\begin{align*}
\psi \in D(p + A) &\overset{Def.}{\iff} \psi \in D(p_m) \cap D(A_m)  \: \,  \forall m = 1, ..., n.\\
\psi \in D((p + A)^2) &\overset{Def.}{\iff} \psi \in D(p + A) \, \, \text{and} \, \, p_m \psi + A_m \psi \in D(p_m) \cap D(A_m) \: \,  \forall  m = 1, ..., n. 
\end{align*}

\begin{Theorem}
\label{Kato ineq}
For $\psi \in \HH$ define 
\begin{align*}
(\mathrm{sgn} \, \psi)(x) =
\begin{dcases}
\dfrac{\psi(x)}{|\psi(x)|} \quad & \psi(x) \neq 0, \\
\, \, \, 0 \quad & \psi(x) = 0.
\end{dcases}
\end{align*}

\begin{enumerate}
\item[(i)] \textbf{Diamagnetic inequality}: If $\psi \in D(p + A)$, then $| \psi |$ is weakly differentiable and 
$$\nabla | \psi | = \mathrm{Re} \, ( \mathrm{sgn} \,  \psi, (\nabla + i A) \psi). $$
In particular $| \nabla | \psi | | \leq | (\nabla + i A) \psi |$ and $| \psi | \in H^1(\R^n)$. 
\item[(ii)] \textbf{Kato's distributional inequality}:  If $\psi \in D((p + A)^2)$, then
\begin{align*} 
\Delta | \psi | \geq \mathrm{Re} \, (\mathrm{sgn} \, \psi, (\nabla + i A)^2 \psi)
\end{align*} 
in the sense of distributions on $C_0^\infty(\R^n)$. 
\end{enumerate}
\end{Theorem}

\begin{proof} 
The following proof is inspired by Kato's original proof \cite[Lemma A]{Kato72}, which concerns the case $\HH'=\C$. 

(i) From $\psi \in D(p + A) \subset D(p)$, it follows that $|\psi|^2 = (\psi, \psi)$ is weakly differentiable and 
\begin{align}  
\nabla | \psi |^2 = (\nabla \psi, \psi) + (\psi, \nabla \psi) = 2 \mathrm{Re} \, (\psi, \nabla \psi).
\label{e0}
\end{align} 
For $\varepsilon > 0$ we define $\psi_\varepsilon = ( | \psi |^2 + \varepsilon^2)^{1/2}$.  By the chain rule (Lemma \ref{Kettenregel}), we see that
\begin{align}
\nabla \psi_\varepsilon &= \dfrac{1}{2} ( | \psi |^2 + \varepsilon^2)^{-1/2} \nabla | \psi |^2 = \mathrm{Re} \left( \dfrac{\psi}{\psi_\varepsilon}, \nabla \psi \right).
\label{e1}
\end{align}
Since $\psi / \psi_\varepsilon \rightarrow \mathrm{sgn} \, \psi$ pointwise and $| \psi / \psi_\varepsilon | \leq 1$, it follows that $\nabla \psi_\varepsilon \rightarrow \mathrm{Re} (\mathrm{sgn} \, \psi, \nabla \psi)$ in $L^2(\R^n)$. In conjunction with $\psi_\varepsilon \rightarrow | \psi |$ in $L_\mathrm{loc}^2(\R^n)$, it follows that $| \psi |$ is weakly differentiable and 
$$\nabla | \psi | = \mathrm{Re} (\mathrm{sgn} \, \psi, \nabla \psi) = \mathrm{Re} (\mathrm{sgn} \, \psi, (\nabla + i A) \psi),$$ 
where the symmetry of the operators $A_{m}(x)$ was used in the last equation.

(ii) From part (i) it follows that $|\psi| \in H^1(\R^n)$ and hence $\nabla |\psi|^2 = 2 |\psi|  \nabla | \psi |$. Therefore Equation \eqref{e1} combined with \eqref{e0} now becomes 
\begin{align*}  
\nabla \psi_\varepsilon = \frac{| \psi | \nabla | \psi|}{\psi_\varepsilon},
\end{align*} 
and, by the chain rule (\Cref{Kettenregel}), we arrive at
\begin{align}  
\nabla \frac{1}{\psi_\varepsilon} = - \frac{| \psi | \nabla | \psi |}{\psi_\varepsilon^3}.
\label{e3}
\end{align} 
The assumption $\psi \in D((p + A)^2)$ implies that $(\psi, (\nabla + i A)\psi)$ is weakly differentiable, and, by the symmetry of $A(x)$, that 
\begin{align}  
\nabla \cdot (\psi, (\nabla + i A) \psi) &= (\nabla \psi, ( \nabla + i A) \psi ) + ( \psi, \nabla \cdot (\nabla + i A) \psi)\nonumber \\ 
&= ( (\nabla + i A) \psi , (\nabla + i A) \psi)  + ( \psi, (\nabla + i A) \cdot (\nabla + i A) \psi)\nonumber \\
&= | (\nabla + i A) \psi |^2 + ( \psi, (\nabla + i A)^2 \psi).\label{p-and-A}
\end{align} 
After these preparations we are now ready to prove part (ii): from part (i) - or \eqref{e0} and the symmetry of $A(x)$ - we know that
\begin{align*}  
| \psi | \nabla |\psi| = \mathrm{Re} \, (\psi, (\nabla + i A) \psi).
\end{align*} 
Differentiating this with the help of \eqref{p-and-A} we arrive at the key identity
\begin{align}  
\nabla \cdot ( | \psi | \nabla |\psi| ) =  | (\nabla + i A) \psi |^2 + \mathrm{Re} \, ( \psi, (\nabla + i A)^2 \psi).
\label{key}
\end{align} 
From  \eqref{e3}, \eqref{key} and the product rule (\Cref{Produktregel}), it follows that
\begin{align}  
\nabla \cdot \left( \frac{1}{\psi_\varepsilon} | \psi | \nabla | \psi| \right) &= \nabla \left( \frac{1}{\psi_\varepsilon} \right) \cdot | \psi|\nabla | \psi | +  \frac{1}{\psi_\varepsilon} \nabla \cdot ( | \psi| \nabla | \psi|) \nonumber \\
&= - \frac{| \psi |^2 |\nabla | \psi ||^2}{\psi_\varepsilon^3} +  \frac{1}{\psi_\varepsilon}  \Bigl( | (\nabla + i A) \psi |^2 + \mathrm{Re} \, ( \psi, (\nabla + i A)^2 \psi) \Bigr) \nonumber \\
&\geq \frac{1}{\psi_\varepsilon} \mathrm{Re} \, ( \psi, (\nabla + i A)^2 \psi).
\label{e4}
\end{align} 
In the last line we used $| \psi|^2 / \psi_\varepsilon^2 \leq 1$ and $|\nabla |\psi||^2 \leq | (\nabla + i A) \psi|^2$. We now integrate \eqref{e4} against a non-negative function $\zeta \in C_0^\infty(\R^n)$ and obtain
\begin{align*}  
- \intd x \,  \frac{|\psi|}{\psi_\varepsilon} \nabla |\psi| \cdot \nabla \zeta \geq \intd x \,   \mathrm{Re}  \left( \frac{\psi}{\psi_\varepsilon}, (\nabla + i A)^2 \psi \right) \zeta.
\end{align*} 
In the limit $\varepsilon \rightarrow 0$ we obtain, by dominated convergence, 
\begin{align*}  
- \intd x \, \nabla |\psi| \cdot \nabla \zeta \geq \intd x \,   \mathrm{Re}  \left( \mathrm{sgn} \, \psi, (\nabla + i A)^2 \psi \right) \zeta.
\end{align*} 
 This concludes the proof of (ii). 
\end{proof}

%%%%%%%%%%%%%%%%%%%%%%%%%%%%%%%%%%%%%PAULI-FIERZ%%%%%%%%%%%%%%%%%%%%%%%%%%%%%%%%%%%%%%%%%%%%%%%%%%%%%%%%%%%%


\section{The Pauli-Fierz model}
\label{sec:qft}

We introduce the Pauli-Fierz model for $N$ identical spin-$1/2$ particles, called electrons, in $\R^3$. 
More elaborate descriptions may be found in \cite{BFS,HH}.

The one-particle Hilbert space associated with the photon field is $\mathfrak{h} = L^2(\R^3 \times \{1, 2\})$. The symmetric Fock space over $\mathfrak{h}$ is denoted by
\begin{align*} 
     \Gamma(\mathfrak{h}) =\bigoplus_{n = 0}^\infty \otimes_\mathrm{sym}^n \,  \mathfrak{h},\qquad \otimes_\mathrm{sym}^0 \,  \mathfrak{h} := \C.
%\mathcal{F}_n^{+}, \quad \mathcal{F}_0^{+} = \C, \quad \mathcal{F}_n^{+} = \quad n \geq 1.
\end{align*}
For $h \in \mathfrak{h}$ let $a(h)$ and $a^*(h)$ denote the usual bosonic annihilation and creation operators in $\Gamma(\mathfrak{h})$, and let
\begin{align*} 
\phi(h) = a(h) + a^*(h),
\end{align*} 
which is defined and symmetric on the subspace of finite particle vectors from $\Gamma(\mathfrak{h})$. The closure of this operator is self-adjoint and denoted by the same symbol.
Let
\begin{align*} 
       \omega(k) = | k |, \quad k \in \R^3,
\end{align*} 
be the photon dispersion relation and let $H_f = d\Gamma(\omega)$ be the second quantization of multiplication with $\omega$ in $\mathfrak{h}$. 

The Hilbert space of the model (without statistics yet) is the tensor product
\begin{align}
     \HH = L^2(\R^{3 N}) \otimes \bigg(\bigotimes_{j = 1}^N \C^2\bigg) \otimes \mathcal{F}^{+}
\label{full Hilbert space}
\end{align}
of particle and Fock space $\FF^{+} =\Gamma(L^2(\R^3\times\{1,2\}))$, with the $N$ factors of $\C^2$ accounting for the spin degrees of the particles. The inner product and norm of $\HH$ will be denoted by $\langle \cdot, \cdot \rangle$ and $\| \cdot \|$, respectively. 

The Pauli-Fierz Hamiltonian is composed of operators acting on the various factors of  \eqref{full Hilbert space}. The operator $H_f = d\Gamma(\omega)$ in $\FF^{+}$ accounts for the energy of massless photons. The potential $V:\R^{3 N} \rightarrow \R$ acts by multiplication on the first factor of \eqref{full Hilbert space}. We
assume that
\begin{align}
V(x) &= \sum_{j = 1}^N v(x_j) + \sum_{j < k} w(x_j - x_k), \quad x = (x_1, ..., x_N) \in \R^{3 N}, \label{potential}
\\
&v,w \in L^2(\R^3) + L^{\infty}(\R^3), \qquad w(x) = w(-x). \nonumber
\end{align}
It follows that $V$ is infinitesimally bounded with respect to the Laplacian $-\Delta$. As usual we shall not distinguish in notation between the operator $V$ in $L^2(\R^{3 N})$ and the operator $V \otimes 1$ in $\HH$. The same remark applies to $-\Delta$ and the components of the momentum operator $- i \nabla_j$ of the $j$th electron.  

To define quantized vector potential and magnetic field we introduce, for $x  \in \R^3$ and $\ell = 1,2,3$, the elements
$G_{\ell}(x), F_{\ell}(x) \in \mathfrak{h}$ by the functions
\begin{align}
  [G_{\ell}(x)](k, \lambda) &= \frac{\rchi_\Lambda(k)}{\sqrt{\omega(k)}} e^{-i k \cdot x} \varepsilon_\ell(k, \lambda),  \label{def G}\\
[F_{\ell}(x)](k, \lambda) &= -i  \frac{\rchi_\Lambda(k)}{\sqrt{\omega(k)}} e^{-i k \cdot x} (k \wedge \varepsilon(k, \lambda))_\ell, \quad k \in \R^3, \lambda = 1,2. \label{def F}
\end{align}
Here $\Lambda < \infty$ is an arbitrary ultraviolet cutoff and $\rchi_\Lambda$ denotes the characteristic function of the set $\{|k| \leq \Lambda \}$. The polarization vectors  $\varepsilon(k, 1), \varepsilon(k, 2)\in \R^3$, for $k\neq 0$, are normalized and orthogonal to $k \in \R^3$. With the identification $L^2(\R^{3 N}) \otimes \FF^+ \, \widetilde{=} \, L^2(\R^{3 N} ; \FF^+)$ we can define
\begin{align}
(A_{j, \ell} \, \psi)(x) &= \phi(G_\ell(x_j)) \psi(x), \nonumber \\ 
(B_{j, \ell} \, \psi)(x) &= \phi(F_\ell(x_j)) \psi(x).
\label{vector potential}
\end{align}
By $A_j$ and $B_j$ we denote the operator-valued vectors with components
$A_{j, \ell}$ and $B_{j, \ell}$, $\ell = 1,2,3$. With the above notations and conventions the Hamiltonian of the system reads
\begin{equation}
   H=\sum_{j = 1}^N \left[ (-i \nabla_j + \sqrt{\alpha}  A_j)^2 + \sqrt{\alpha} \, \sigma_j \cdot B_j \right]  + V + H_f,
\label{HAMILTONIAN}
\end{equation}
where $\sigma_j$ denotes the triple of Pauli matrices acting on the $j$th factor in $\bigotimes_{j = 1}^N \C^2$. 
It is well known that for all values of the coupling constant $\alpha > 0$ the Hamiltonian is self-adjoint on $D(H) = D(-\Delta + H_f) = D(-\Delta) \cap D(H_f)$ and bounded from below \cite{Hi2002,HH}. Moreover, $H$ is essentially self-adjoint on any core for $-\Delta + H_f$ \cite{HH}.

\medskip
To fit the Pauli-Fierz model into the setting of the previous section, we now define
\begin{align*}  
    \HH'= \bigg(\bigotimes_{j = 1}^N \C^2 \bigg) \otimes \mathcal{F}^{+}
\end{align*} 
with inner product $( \cdot , \cdot )$ and norm $| \cdot |$. Then $\HH$ can be identified with $L^2(\R^{3 N};\, \HH')$ via the unitary map that is determined by $\psi = f \otimes v \mapsto \psi(x) = f(x) \, v$. 
To further simplify notation we introduce operator-valued vectors $A,B$ and $\sigma$ with $3N$ components
such that \eqref{HAMILTONIAN} becomes
\begin{equation}\label{Pauli-Fierz} 
     H = (-i \nabla + \sqrt{\alpha} A)^2 + \sqrt{\alpha} \, \sigma \cdot B + V + H_f.
\end{equation} 

To account for the fermionic nature of electrons, we now reduce the Hilbert space to the closed subspace $\HH^{-}\subset \HH$ consisting of all vectors $\psi$ that are antisymmetric with respect to permutations of the $N$ factors of $L^2(\R^3)\otimes \C^2$ in \eqref{full Hilbert space}. By the above mentioned isomorphism, $\HH^{-}$ may be identified with a closed subspace of $L^2(\R^{3 N};\, \HH')$.  
Since $H$ is symmetric w.r.t.~permutations of the $N$ electrons, it follows that the orthogonal projection onto $\HH^{-}$ commutes with $H$. We thus can restrict $H$ onto $\HH^{-}$ and obtain a self-adjoint operator
\begin{align*}
H^{-} = H: D(H) \cap \HH^{-} \rightarrow \HH^{-}
\end{align*} 
called the \emph{Pauli-Fierz Hamiltonian}. The \emph{ionization threshold} of $H^-$ is defined by 
\begin{equation}
\label{def:sigma}
\Sigma = \lim_{R \rightarrow \infty} \left( \inf_{\psi \in D_R} \langle \psi, H^- \psi \rangle \right)
\end{equation}
where $D_R\subset D(H^-)$ is the subset of normalized states supported outside the ball $B(0,R) \subset \R^{3N}$ \cite{G}. 

%The \textbf{ionization energy} is the energy difference $\Sigma - \inf \sigma(H^{-})$. In the case of many-particle potentials of the form \eqref{potential}, it is well known that, see Theorem 3 in \cite{G}, 
%\begin{align*}
 %\Sigma = \min_{N' = \, 1, ..., N} \left( E_{N - N'} + E_{N'}^0 \right).
%\end{align*}
%Here $E_N = \inf \sigma(H^-)$ is the lowest energy of the $N$-electron system and $E_{N  = \, 0} = 0$. $E_N^0$ is the lowest energy of the $N$-electron system, if the external potentials are dropped in (\ref{potential}).  

States with energy distribution strictly below $\Sigma$ decay exponentially in the following sense \cite[Theorem 1]{G}:
For $\lambda \in \R$ let
 \begin{align*}  
 P_\lambda^{-} \coloneqq \rchi_{(- \infty, \lambda]}(H^{-}).
 \end{align*} 
 If $\lambda < \Sigma$ and $\psi \in \mathrm{Ran} \, P_\lambda^{-}$, then for all $\beta < \sqrt{\Sigma - \lambda}$
\begin{align}
        \intd  x \, e^{2 \beta |x|} \, | \psi(x) |^2  < \infty.
\label{L 2 exponential decay}
\end{align}
This includes eigenvectors of $H^{-}$ with eigenvalue $\lambda$ below $\Sigma$ in which case the result is from \cite[Lemma 6.2]{GLL} and the proof is easier. 


%------------------------------------------------------------------------------------------------------------------------------POINTWISE BOUNDS ON EIGENSTATES---------------------------------------------

\section{Pointwise bounds on eigenstates}
\label{sec:qft-exp}

In this section we establish pointwise bounds on eigenstates $\psi$ of the Pauli-Fierz Hamiltonian \eqref{Pauli-Fierz}. Using the new version of Kato's distributional inequality, \Cref{Kato ineq}, we show that $| \psi |$ is a subsolution to 
a Schr\"odinger equation, and hence that $| \psi |$ satisfies a subsolution estimate \cite{A}. When combined with the known $L^2$-exponential bound \eqref{L 2 exponential decay}, a pointwise exponential bound with the same rate of decay follows. 

%--------------------------------------------------------------------------------
\begin{Theorem}
\label{thm:subsolution}
If $\psi$ is a normalized eigenvector of the Hamiltonian \eqref{Pauli-Fierz}, then $| \psi | \in H^1(\R^{3 N})$ and there exists a constant $c > 0$, such that
\begin{align} 
- \Delta | \psi | \leq (V_{-} + c) | \psi |
\label{subsolution}
\end{align} 
in the sense of distributions on $C_0^\infty(\R^{3 N})$. Here $V_{-} = \mathrm{max} \, (-V, 0)$ denotes the negative part of the potential $V$. 
\end{Theorem}

\begin{proof} 
Let $\psi \in D(H)$ be an eigenstate for $H$ with eigenvalue $\lambda$. Since $D(H)$ is contained in  $D( (p + \sqrt{\alpha} \, A)^2)$, see \cite[Theorem 7]{HH}, Theorem \ref{Kato ineq} implies that
\begin{align}  
- \Delta | \psi | \leq \mathrm{Re} \, (\mathrm{sgn} \, \psi, (-i \nabla + \sqrt{\alpha}A)^2 \psi) 
\label{I}
\end{align} 
in the sense of distributions on $C_0^\infty(\R^{3 N})$. On the other hand, using $H \psi = \lambda \psi$, it follows that pointwise on $\R^{3 N}$
\begin{align}  
 \mathrm{Re} \, (\mathrm{sgn} \, \psi, (-i \nabla + \sqrt{\alpha}A)^2 \psi) &= \mathrm{Re} \, ( \mathrm{sgn} \, \psi, (\lambda - \sqrt{\alpha} \, \sigma \cdot B - V - H_f) \psi) \nonumber \\
 &\leq (\lambda+V_{-}) | \psi | - (\mathrm{sgn} \, \psi, (\sqrt{\alpha} \, \sigma \cdot B + H_f) \psi). 
 \label{II}
\end{align} 
By Lemma \ref{lower bound on spin} we have 
\begin{align} 
 - (\mathrm{sgn} \, \psi, (\sqrt{\alpha} \, \sigma \cdot B + H_f) \psi) \leq  \frac{8 \pi}{3} \alpha N^2 \Lambda^3 | \psi |.
 \label{III}
\end{align} 
Combining \eqref{I}, \eqref{II} and \eqref{III} the proof is complete. 
\end{proof}

As a corollary to Theorem \ref{thm:subsolution} we obtain the following theorem:

\begin{Theorem}
\label{thm:ptw-bound}
If $\psi$ is a normalized eigenvector of the Hamiltonian \eqref{Pauli-Fierz}, then $| \psi |$ satisfies a subsolution estimate: For any positive real numbers $r,R \in \R$, with $r < R$, there exists a constant $C$ such that
\begin{align}
  \esssup_{y\in B(x,r)}|\psi(y)| \leq C\bigg(\int_{B(x,R)}|\psi(y)|^2\, dy\bigg)^{1/2}
\label{main bound}
\end{align}
for all $x \in \R^{3 N}$.
\end{Theorem}

\begin{proof}
By \Cref{thm:subsolution}, $|\psi|$ satisfies \eqref{subsolution} and hence, by Theorem 5.1 in Agmon's book \cite{A}, the subsolution estimate \eqref{main bound} follows. Actually, Theorem 5.1 in Agmon's book is about solutions rather than subsolutions to  Schr\"odinger equations, but the first step in the proof is to derive an inequality of the form \eqref{subsolution} and the rest follows from this inequality. The class of potentials considered by Agmon is more general than  \eqref{potential}.
\end{proof}

Subsolution estimates of the form  \eqref{main bound} for ($\C$-valued) subsolutions to Schr\"odinger equations were known previous to Agmon's work, see \cite{A, AiSi} and the references therein, but for the 
proof of \Cref{thm:ptw-bound} the version of Agmon appears most convenient. %In \cite[Theorem 6.1]{AiSi} local boundedness is assumed of a subsolution. 

%---------------------------------------------------------------------------------------------------------------------------------

%---------------------------------------------------------------------------------------------------------------------------------
Theorem \ref{thm:ptw-bound} allows us to pass from $L^2$ exponential bounds to $L^\infty$ exponential bounds:

\begin{Corollary}
\label{anisotrope Schranken}
Let $\psi$ be a normalized eigenvector of the Hamiltonian~\eqref{Pauli-Fierz} and suppose that
\begin{align*} 
\intd x \, e^{2 f(x)}  |\psi(x)|^2 < \infty, 
\end{align*} 
with some Lipschitz function $f : \R^{3 N} \rightarrow [0, \infty)$. Then 
\begin{align*} 
\esssup_{x \in \R^{3 N}}  \, e^{f(x)}  |\psi(x)| < \infty.
\end{align*} 
\end{Corollary}

\begin{proof} 
Let  $r = 1/2$, $R = 1$ and let $C$  be given by Theorem \ref{thm:ptw-bound}. Denoting by $L$ the Lipschitz constant of $f$, it follows that
\begin{align*} 
\esssup_{y \, \in B(x, 1/2)} e^{f(y)} | \psi(y) | &\leq C e^{L/2} e^{f(x)} \bigg(\int_{B(x,1)}|\psi(y)|^2\, dy\bigg)^{1/2} \\ 
&= C e^{L/2}  \bigg(\int_{B(x,1)}e^{2 f(x)} |\psi(y)|^2\, dy\bigg)^{1/2} \\ 
&\leq C e^{L/2} e^L \bigg(\int e^{2 f(y)} |\psi(y)|^2\, dy\bigg)^{1/2}
\end{align*} 
for all $x \in \R^{3 N}$. Note that the right-hand side does not depend on $x$.
\end{proof}
%---------------------------------------------------------------------------------------------------------------------------------------


\Cref{anisotrope Schranken} with $f(x) = \beta |x|$ in conjunction with the $L^2$ exponential bound \eqref{L 2 exponential decay} yields the following theorem:

%--------------------------------------------------------------------------------------------------------------------------------------
\begin{Theorem}
\label{pointwise bounds on bound states}
Let $\psi$ be a normalized eigenvector of the Pauli-Fierz Hamiltonian $H^{-}$ with eigenvalue $\lambda$ below the ionization threshold $\Sigma$. Then for all $\beta < \sqrt{\Sigma - \lambda}$
\begin{align*}
\esssup_{x \in \R^{3 N}}  \, e^{\beta | x |}  |\psi(x)| < \infty.
\end{align*}
\end{Theorem}



%-------------------------------Sec: Estimating commutators----------------------------------------------------------------------


\section{Estimating commutators} 
\label{sec:commutators}

The purpose of this section is to establish \Cref{lokalisations lemma} below, a
technical result needed in \Cref{particle density section}. For notational convenience we introduce the operator $\Pi$ with components 
\begin{align*} 
\Pi_m = - i \partial_m + \sqrt{\alpha} A_m, \quad m = 1, ..., 3N.
\end{align*} 
Then the Hamiltonian \eqref{Pauli-Fierz} becomes
\begin{align*} 
H = \Pi^2 + \sqrt{\alpha} \, \sigma \cdot B + V + H_f.
\end{align*} 
As a preparation we need the following lemma.
%-------------------------------------------------------------------------------------------------------------------------------------------------------------------
\begin{Lemma}\label{Kommutator Lemma}
Let $\gamma \in C^2(\R^{3 N})$ be a bounded function with bounded derivatives up to 2nd order. Then
\begin{align}
[\Pi_m, \gamma ] &= -i (\partial_m \gamma) \quad \hspace{19mm} \text{on} \: D(-i \partial_m) \cap D(A_m)
\label{Kommutator Pi}  \\
[H, \gamma] &= -(\Delta \gamma) - 2 i (\nabla \gamma) \cdot \Pi \quad \text{on} \: D(H).
\label{Kommutator von H}
\end{align}
\end{Lemma}

\begin{proof} 
By the product rule $\gamma D(\partial_m) \subset D(\partial_m)$ and $[\partial_m, \gamma ] = (\partial_m \gamma)$ on $D(\partial_m)$. Since $\gamma$ commutes with $A_m$, it follows that $\gamma D(A_m) \subset D(A_m)$ and $[A_m, \gamma] = 0$ on $D(A_m)$. This proves (\ref{Kommutator Pi}).  We have $\gamma D(- \Delta) \subset D(- \Delta)$ and $\gamma$ commutes with $\sigma, B, V$ and $H_f$. Therefore $D(H) = D(-\Delta) \cap D(H_f)$ is invariant under $\gamma$. 
By \eqref{Kommutator Pi} it follows that
\begin{align*}
[H,\gamma] &= [\Pi^2, \gamma ] =  \Pi  \cdot [\Pi, \gamma] + [\Pi, \gamma] \cdot \Pi  \\
&= [\Pi, -i(\nabla \gamma)] - 2 i (\nabla \gamma) \cdot \Pi  = - (\Delta \gamma) - 2 i (\nabla \gamma) \cdot \Pi. \qedhere
\end{align*} 
\end{proof}
%-------------------------------------------------------------------------------------------------------------------------------------------------------------


%-------------------------------------------------------------------------------------------------------------------------------------------------------------
\begin{Lemma}
\label{lokalisations lemma}
Let $H,\ H^{-}$ and $P^{-}_{\lambda}$ be defined as in \Cref{sec:qft}. Let $\gamma \in C^\infty(\R^{3 N}; \R)$ be a bounded function with bounded derivatives of all orders. Suppose 
$B \subset \R^{3 N}$ is a measurable  set containing $\mathrm{supp} \, \gamma$. For $x \in \R^{3 N}$ let $\gamma_x(y) := \gamma( y - x)$ and $B_x := B + x$. If $\lambda < \infty$, then for every $\tau < 1$ there exists a constant $C_\tau$ such that
\begin{align*}
\| H \gamma_x P_\lambda^{-} \|  \leq C_\tau \| \rchi_{B_x} P_\lambda^{-}  \|^\tau
\end{align*}
for all $x \in \R^{3 N}$. Here $\| \cdot \|$ is the operator norm in $\mathcal{L}(\HH^{-}, \HH)$ and $P_\lambda^{-} = \rchi_{(- \infty, \lambda]}(H^{-})$.
\end{Lemma}

Before beginning the proof we recall the general fact, that a closed operator $C$ is bounded relative to a self-adjoint operator $H$ with $D(H)\subset D(C)$. Indeed $C (H + i)^{-1}$ is closed and hence bounded. 

\begin{proof} Let $\psi \in \mathrm{Ran} \, P_\lambda^-$ be normalized. From Lemma \ref{Kommutator Lemma} it follows that
\begin{align}
\| H \gamma_x \psi \| &= \| [H, \gamma_x] \psi + \gamma_x H \psi\| \nonumber \\
&\leq \| -(\Delta \gamma_x) \psi \| + 2  \| (\nabla \gamma_x)\cdot \Pi  \psi \| + \| \gamma_x H  \psi \| \nonumber \\
&= \| (\Delta \gamma_x) P_\lambda^- \psi \| + 2  \| (\nabla \gamma_x)\cdot \Pi  \psi \| + \| \gamma_x P_\lambda^- H^- P_\lambda^-  \psi \| \nonumber \\
&\leq C_1 \| \rchi_{B_x} P_\lambda^- \| + 2  \| (\nabla \gamma_x)\cdot \Pi  \psi \|
\label{H gamma_x psi}
\end{align}
for some constant $C_1$. We used that $\mathrm{supp} \, \gamma_x \subset B_x$ and $H^- P_\lambda^-$ is bounded. By the Cauchy-Schwarz inequality in $\C^{3 N}$, a commutator identity and (\ref{Kommutator Pi}) we have
\begin{align}
\| (\nabla \gamma_x)\cdot \Pi  \psi \|^2 &\leq \| |\nabla \gamma_x| \, \Pi \psi \|^2 \nonumber \\
   &=  \langle \psi, \Pi |\nabla \gamma_x|^2 \Pi \psi \rangle \nonumber \\
&= \frac{1}{2} \langle \psi, (\Pi^2 |\nabla \gamma_x|^2 + |\nabla \gamma_x|^2 \Pi^2) \psi \rangle - \frac{1}{2}  \langle \psi, [\Pi, [\Pi, |\nabla \gamma_x|^2 ]] \psi \rangle \nonumber \\
&=  \mathrm{Re}  \langle \psi, |\nabla \gamma_x|^2 \Pi^2  \psi \rangle + \frac{1}{2} \langle \psi, \Delta (|\nabla \gamma_x|^2) \psi \rangle \nonumber\\
&\leq  \mathrm{Re} \langle \psi, | \nabla \gamma_x |^2 (\Pi^2 - H) \psi \rangle + C_2 \| \rchi_{B_x} P_\lambda^- \|^2.
\label{nabla term}
\end{align}
In the last line we used $\Pi^2 = (\Pi^2 - H) + H$ and the same arguments as in (\ref{H gamma_x psi}). Next, we decompose the potential $V = V_{+} - V_{-}$ into positive and negative part so that $-V \leq V_{-}$. By Lemma \ref{lower bound on spin}, there is a constant $\nu$ such that
\begin{align*}  
\Pi^2 - H = -\left(\sqrt{\alpha} \, \sigma \cdot B + H_f \right) - V \leq \nu + V_{-}.
\end{align*} 
Since $\partial_m \gamma_x$ commutes with $\sigma, B$, $H_f$ it follows that
\begin{align} 
\mathrm{Re} \langle \psi, | \nabla \gamma_x |^2 (\Pi^2 - H) \psi \rangle &\leq \sum_m \langle (\partial_m \gamma_x) \psi, (\nu + V_{-}) (\partial_m \gamma_x) \psi \rangle \nonumber \\
&\leq \sum_m \| (\partial_m \gamma_x) P_\lambda^{-} \| \| (\nu + V_{-}) (\partial_m \gamma_x) P_\lambda^{-} \|.
\label{Pi^2 - H}
\end{align} 
Since $\psi\in \mathrm{Ran} \, P_\lambda^{-}$ was an arbitrary normalized vector, it follows from \eqref{H gamma_x psi}, \eqref{nabla term} and \eqref{Pi^2 - H} that
\begin{align}
\| H \gamma_x P_\lambda^{-} \| \leq C_3 \left( \| \rchi_{B_x} P_\lambda^- \| +  \| \rchi_{B_x} P_\lambda^- \|^{1/2} \sum_m \| V_{-} (\partial_m \gamma_x) P_\lambda^- \|^{1/2}  \right). 
\label{rekursions ugl}
\end{align}
If $V_{-}$ is bounded, then the lemma follows with $\tau = 1$. In the general case, where $V_{-}$ may be unbounded, we use that $D(V_{-}) \supset D(H)$. Therefore $V_{-}$ is bounded relative to $H$ and by enlarging the constant $C_3$, it follows that
\eqref{rekursions ugl} holds with $V_{-}$ replaced by $H$. Iterating this inequality $n$ times, using $\left( \sum_m a_m \right)^{1/2} \leq \sum_m a_m^{1/2}$ in each step, we arrive at
\begin{equation*}
\| H \gamma_x P_\lambda^{-} \| \leq C_4 \left( \| \rchi_{B_x} P_\lambda^- \| +  \| \rchi_{B_x} P_\lambda^- \|^{1 - 1/{2^n}} \sum_{m_1, ..., m_n} \| H (\partial_{m_n} ... \partial_{m_1} \gamma_x) P_\lambda^- \|^{1/{2^n}} \right).
\end{equation*}
By \eqref{rekursions ugl},  the factor $\|H (\partial_{m_n} ... \partial_{m_1} \gamma_x) P_\lambda^- \|$ is bounded uniformly in $x$. Hence
\begin{align*}
\| H \gamma_x P_\lambda^{-} \| &\leq C_n \| \rchi_{B_x} P_\lambda^- \|^{1 - 1/{2^n}} \all x \in \R^{3 N}.  \quad  \qedhere
\end{align*}
\end{proof}
%----------------------------------------------------------------------------------------------------------------------------





%------------------------------------------------- Sec : Pointwise bounds on one-particle densities---------------------------------------------------------------------------------

\section{Pointwise bounds on one-particle densities}
\label{particle density section}

We now turn to one-particle densities. For $\psi \in \HH \, \widetilde{=} \, L^2(\R^{3 N}; \, \HH')$ we define $\rho_\psi \in L^1(\R^3)$ by $\rho_\psi(x_1) = | \psi(x_1) |^2$ if $N = 1$ and otherwise by
$$
  \rho_\psi (x_1) = \int \hspace{-1.5mm} \, dx_2\ldots dx_N |\psi(x_1, x_2, \ldots, x_N)|^2, \quad x_1 \in \R^3.
$$
If $\psi$ is normalized, then $\rho_\psi$ is called the \emph{one-particle density} associated to $\psi$. In this section we establish pointwise exponential decay for one-particle densities associated to states $\psi$ with energy distribution strictly below the ionization threshold. This result is based on a combination of the $L^2$-exponential bound \eqref{L 2 exponential decay} with the Sobolev-type estimate of \Cref{sobolev lemma}; it does not require $\psi$ to be an eigenstate.

%--------------------------------------------------------------------------------------------------------------------
\begin{Lemma}
There exists a constant $B$ such that
\begin{align} 
 \rho_\psi \in C(\R^3) \quad \mathrm{and} \quad \|\rho_\psi \|_\infty \leq B \,( \| \psi \|^2 + \| \Delta \psi \|^2)
 \label{density embedding}
 \end{align} 
for all $\psi \in D(-\Delta)$. 
\label{sobolev lemma}
\end{Lemma}
\begin{proof}

Notice that it suffices to prove \eqref{density embedding} on an operator core for $-\Delta$. Therefore we may assume that $\psi$ is of the form
\begin{align*} 
\psi(x) = \sum_{\mathrm{finite}} f_k(x) v_k, \quad f_k \in C_0^\infty(\R^{3 N}), \, v_k \in \HH'.
\end{align*} 
Without loss of generality we assume that the $v_k$'s are orthogonal. Then
\begin{equation}\label{dem1} 
   \rho_{\psi}(x_1) = \sum_\mathrm{finite} \rho_{f_k}(x_1) | v_k |^2.
\end{equation} 
By the embedding $H^2(\R^3) \rightarrow L^{\infty}(\R^3)$, there exists a constant $B$ such that
\begin{equation}\label{dem2}  
\| \rho_{f_k} \|_\infty \leq B ( \|f_k \|^2 + \| \Delta_{x_1} f_k \|^2) \leq B ( \|f_k \|^2 + \| \Delta f_k \|^2).
\end{equation} 
The bound \eqref{density embedding} now follows from \eqref{dem1}, \eqref{dem2} and the orthogonality of the $v_k$'s.
\end{proof}
%------------------------------------------------------------------------------------------------------------------------


%-------------------------------------------------------------------------------------------------------------------------
\begin{Proposition}
\label{pointwise bounds one particle}
Let $P_\lambda^{-} = \chi_{(-\infty,\lambda]}(H^{-})$ with $H^{-}$ defined as in \Cref{sec:qft}.
If $\lambda < \infty$ and $\psi \in \mathrm{Ran} \, P_\lambda^{-}$ is normalized, then the one-particle density $\rho_\psi$ is continuous and for every 
$\tau<1$ there exists a constant $C_\tau$ such that for all $x_1 \in \R^3$,
$$
    \sup_{y_1 \in B(x_1,1/2)} \rho_{\psi}(y_1) \leq C_\tau \| (\rchi_{B(x_1, 1)} \otimes 1) \,  P_\lambda^{-}  \|^{2 \tau}.
$$
\end{Proposition}

\begin{proof}
Continuity follows from $\psi \in D(H^-)\subset D(-\Delta)$ and \Cref{sobolev lemma}. Let $\gamma \in C_0^\infty(\R^3; [0,1])$ with $\mathrm{supp} \, \gamma \subset B(0, 1)$ and $\gamma = 1$ in $B(0,1/2)$. For $x \in \R^{3 N}$ we define the function $\Gamma_x(y) = \gamma( y_1 - x_1), \, y \in \R^{3 N}$. We now use Lemma \ref{sobolev lemma} and the fact that  $-\Delta$ is $H$-bounded (a consequence of $D(-\Delta) \supset D(H)$), to find a constant $B'$ such that 
\begin{equation}\label{ptw-dichte1} 
  \sup_{y_1 \in B(x_1,1/2)} \rho_{\psi}(y_1) \leq \sup_{y_1 \in \R^3} \rho_{(\Gamma_x \psi)} (y_1) \leq B'(\|H \Gamma_x \psi \|^2 + \| \Gamma_x \psi \|^2),
 \end{equation} 
for all $ x \in \R^{3 N}$. By Lemma \ref{lokalisations lemma} we find a constant $C$ such that
\begin{equation}\label{ptw-dichte2} 
     \| H \Gamma_x \psi \|^2 + \| \Gamma_x \psi \|^2 \leq C \| (\rchi_{B(x_1,1)} \otimes 1) \, P_\lambda^{-}  \|^{2 \tau}
\end{equation}
for all $x \in \R^{3 N}$. Notice that $\| \Gamma_x P_\lambda^{-} \| \leq \| \gamma \|_\infty \| (\rchi_{B(x_1,1)} \otimes 1) \, P_\lambda^{-}  \|^{\tau}$ because $\mathrm{supp} \, \Gamma_x \subset B(x_1, 1) \times \R^{3(N - 1)}$ and $\tau < 1$. Combining \eqref{ptw-dichte1} and \eqref{ptw-dichte2} the proposition follows.
\end{proof}
%--------------------------------------------------------------------------------------------------------------------------------


%----------------------------------------------------------------------------------------------------------------------------
\begin{Theorem} 
Let $P_\lambda^{-} = \chi_{(-\infty,\lambda]}(H^{-})$ with $H^{-}$ defined as in \Cref{sec:qft}.
Let $\psi \in \mathrm{Ran} \, P_\lambda^{-}$ be normalized with total energy $\lambda$ below the ionization threshold $\Sigma$. Then the one-particle density $\rho_\psi$ 
is continuous and for all $\beta < \sqrt{\Sigma - \lambda}$
\begin{align*}
\sup_{x_1 \in \R^3}  e^{2 \beta | x_1 |}   \rho_\psi(x_1) < \infty.
\end{align*}
\label{Punktweiser Abfall Teilchenzahldichte}
\end{Theorem}

\begin{proof}
Choose $\tau < 1$ such that $\beta/\tau < \sqrt{\Sigma - \lambda}$ and let $C_\tau$ be given by \Cref{pointwise bounds one particle}. Then
\begin{align*}
\sup_{y_1 \in B(x_1, 1/2)} e^{2 \beta |y_1|} \rho_\psi(y_1) &\leq e^\beta e^{2 \beta | x_1 |} C_\tau \| (\rchi_{B(x_1, 1)} \otimes 1) \, P_\lambda^{-}  \|^{2 \tau} \\
&= e^\beta  C_\tau \, \| e^{\beta/\tau  |x_1|} (\rchi_{B(x_1,1)} \otimes 1) \, P_\lambda^{-} \|^{2 \tau} \\
&\leq e^{\beta} C_\tau \| e^{\beta/\tau(1 + | \, \cdot \, |)}  P_\lambda^{-} \|^{2 \tau}.
\end{align*}
Note that the right-hand side is independent of $x_1$. By \eqref{L 2 exponential decay}, the right-hand side is finite and independent of $x_1$. 
\end{proof}
%-----------------------------------------------------------------------------------------------------------------------------------------

\noindent
\textit{Acknowledgement.} V.K. thanks Heinz Siedentop and Simone Rademacher for the opportunity to present an earlier version of this paper at the DMV Meeting 2023 at TU Ilmenau. This work is supported by the 
Deutsche Forschungsgemeinschaft (DFG, German Research Foundation) - Projektnummer 531147062.


\appendix
\section{Appendix}
\label{sec:appendix}

\begin{Lemma}
$\mathrm{(Product \, rule)}$ Let $\Omega \subset \R^n$ be open. Suppose that $u, v \in L_\mathrm{loc}^1(\Omega)$ are weakly differentiable. If $u v \in L_\mathrm{loc}^1(\Omega)$ and $u (\nabla v) + (\nabla u) v \in L_\mathrm{loc}^1(\Omega)$, then $u v$ is weakly differentiable and
\begin{align*}
\nabla ( u v ) = u (\nabla v) + (\nabla u) v.
\end{align*} 
\label{Produktregel}
\end{Lemma}
\vspace{-7mm}
For the proof see \cite[Lemma 2.14]{FLW}. In our applications of this result, one of the two factors $u,v$ is bounded, which is the easier case established first in \cite{FLW}.


\begin{Lemma}
$\mathrm{(Chain \, rule)}$   
Let $f \in C^1(\R)$ and suppose that $f'$ is bounded. Let $\Omega \subset \R^n$ be open. If $u \in L_\mathrm{loc}^1(\Omega)$ is real-valued and weakly differentiable, then the composition $f \circ u$ is weakly differentiable and 
\begin{align*}
\nabla (f \circ u) = (f' \circ u) \, \nabla u. 
\end{align*} 
\label{Kettenregel}
\end{Lemma}
\vspace{-7mm}
For the proof see \cite[Theorem 7.8]{GT}.


\begin{Lemma}
\label{lower bound on spin}
For all $\psi \in D(H_f)$ we have that pointwise on $\R^{3 N}$
\begin{align*} 
(\psi, ( H_f + \sqrt{\alpha} \, \sigma \cdot B) \psi) \geq - \frac{8 \pi}{3} \alpha N^2 \Lambda^3 | \psi |^2.
\end{align*} 
\end{Lemma}

See, e.g.,  \cite[Lemma A.1]{LL} for the proof, but notice that the magnetic field operator in \cite{LL} differs by a factor of $(2 \pi)^{-1}$ from our definition. 

As a corollary of \Cref{lower bound on spin} in combination with \Cref{Kato ineq} we immediately obtain the following known results. It also follows from \Cref{lower bound on spin}  and from Kato's distributional inequality in its original form after passing to the Schr\"odinger representation of Fock space, see \cite{FFG}.

\begin{Corollary}
\label{lower bound on ground state energy}
Let $H$ be the Pauli-Fierz Hamiltonian \eqref{Pauli-Fierz} for $N = 1$ electron. Then the ground state energy is bounded from below by
\begin{align*} 
\min \sigma(H) \geq \min \sigma(-\Delta + V) - \frac{8 \pi}{3} \alpha \, \Lambda^3.
\end{align*} 
\end{Corollary}

\noindent
\emph{Remark:} For $N \geq 2$ there is no such lower bound, since $\psi \mapsto | \psi |$ destroys Fermi statistics. 
\begin{proof}
Let $\psi \in D(H)$. From the diamagnetic inequality or from Kato's distributional inequality (\Cref{Kato ineq}) it follows that
\begin{align*}  
\langle \psi, (-i \nabla + \sqrt{\alpha} A)^2 \psi  \rangle \geq \langle \nabla | \psi |, \nabla | \psi | \rangle.
\end{align*} 
In combination with $\langle \psi, V \psi \rangle = \langle | \psi|, V | \psi | \rangle$ and \Cref{lower bound on spin} we find
\begin{align*} 
\langle \psi, H \psi \rangle \geq \left[ \min \sigma(-\Delta + V) - \frac{8 \pi}{3} \alpha \, \Lambda^3 \right] \| \psi \|^2,
\end{align*} 
which completes the proof.
\end{proof}


%----------------------------------------------------------------------------------------------------------------------------------------------

%\bibliography{pointwise_refrences}
%\bibliographystyle{plain}

\begin{thebibliography}{10}

\bibitem{A}
Shmuel Agmon.
\newblock {\em Lectures on exponential decay of solutions of second-order
  elliptic equations: bounds on eigenfunctions of {$N$}-body {S}chr\"{o}dinger
  operators}, volume~29 of {\em Mathematical Notes}.
\newblock Princeton University Press, Princeton, NJ; University of Tokyo Press,
  Tokyo, 1982.

\bibitem{Ahlrichs}
Reinhart Ahlrichs.
\newblock Asymptotic behavior of atomic bound state wave functions.
\newblock {\em J. Mathematical Phys.}, 14:1860--1863, 1973.

\bibitem{AiSi}
M.~Aizenman and B.~Simon.
\newblock Brownian motion and {H}arnack inequality for {S}chr\"{o}dinger
  operators.
\newblock {\em Comm. Pure Appl. Math.}, 35(2):209--273, 1982.

\bibitem{AnaLew}
Ioannis Anapolitanos and Mathieu Lewin.
\newblock Compactness of molecular reaction paths in quantum mechanics.
\newblock {\em Arch. Ration. Mech. Anal.}, 236(2):505--576, 2020.

\bibitem{AnaSig}
Ioannis Anapolitanos and Israel~Michael Sigal.
\newblock Long-range behavior of the van der {W}aals force.
\newblock {\em Comm. Pure Appl. Math.}, 70(9):1633--1671, 2017.

\bibitem{BFS98}
Volker Bach, J\"{u}rg Fr\"{o}hlich, and Israel~Michael Sigal.
\newblock Quantum electrodynamics of confined nonrelativistic particles.
\newblock {\em Adv. Math.}, 137(2):299--395, 1998.

\bibitem{BFS}
Volker Bach, J\"{u}rg Fr\"{o}hlich, and Israel~Michael Sigal.
\newblock Spectral analysis for systems of atoms and molecules coupled to the
  quantized radiation field.
\newblock {\em Comm. Math. Phys.}, 207(2):249--290, 1999.

\bibitem{FFG}
Charles Fefferman, J\"{u}rg Fr\"{o}hlich, and Gian~Michele Graf.
\newblock Stability of ultraviolet-cutoff quantum electrodynamics with
  non-relativistic matter.
\newblock {\em Comm. Math. Phys.}, 190(2):309--330, 1997.

\bibitem{FLW}
Rupert~L. Frank, Ari Laptev, and Timo Weidl.
\newblock {\em Schr\"{o}dinger operators: eigenvalues and {L}ieb-{T}hirring
  inequalities}, volume 200 of {\em Cambridge Studies in Advanced Mathematics}.
\newblock Cambridge University Press, Cambridge, 2023.

\bibitem{GT}
David Gilbarg and Neil~S. Trudinger.
\newblock {\em Elliptic partial differential equations of second order}.
\newblock Classics in Mathematics. Springer-Verlag, Berlin, 2001.
\newblock Reprint of the 1998 edition.

\bibitem{G}
M.~Griesemer.
\newblock Exponential decay and ionization thresholds in non-relativistic
  quantum electrodynamics.
\newblock {\em J. Funct. Anal.}, 210(2):321--340, 2004.

\bibitem{pointwise23}
M~Griesemer and V~Ku{\ss}maul.
\newblock Pointwise bounds on eigenstates in non-relativistic quantum field
  theory.
\newblock {\em arXiv:2307.14986}, 2023.

\bibitem{GLL}
Marcel Griesemer, Elliott~H. Lieb, and Michael Loss.
\newblock Ground states in non-relativistic quantum electrodynamics.
\newblock {\em Invent. Math.}, 145(3):557--595, 2001.

\bibitem{HH}
D.~Hasler and I.~Herbst.
\newblock On the self-adjointness and domain of {P}auli-{F}ierz type
  {H}amiltonians.
\newblock {\em Rev. Math. Phys.}, 20(7):787--800, 2008.

\bibitem{HiHi2010}
Takeru Hidaka and Fumio Hiroshima.
\newblock Pauli-{F}ierz model with {K}ato-class potentials and exponential
  decays.
\newblock {\em Rev. Math. Phys.}, 22(10):1181--1208, 2010.

\bibitem{Hi2002}
F.~Hiroshima.
\newblock Self-adjointness of the {P}auli-{F}ierz {H}amiltonian for arbitrary
  values of coupling constants.
\newblock {\em Ann. Henri Poincar\'{e}}, 3(1):171--201, 2002.

\bibitem{Hiro1996}
Fumio Hiroshima.
\newblock Diamagnetic inequalities for systems of nonrelativistic particles
  with a quantized field.
\newblock {\em Rev. Math. Phys.}, 8(2):185--203, 1996.

\bibitem{Hiro1997}
Fumio Hiroshima.
\newblock Functional integral representation of a model in quantum
  electrodynamics.
\newblock {\em Rev. Math. Phys.}, 9(4):489--530, 1997.

\bibitem{Hi14}
Fumio Hiroshima.
\newblock Functional integral approach to semi-relativistic {P}auli-{F}ierz
  models.
\newblock {\em Adv. Math.}, 259:784--840, 2014.

\bibitem{Hiro2019}
Fumio Hiroshima.
\newblock Pointwise exponential decay of bound states of the {N}elson model
  with {K}ato-class potentials.
\newblock In {\em Analysis and operator theory}, volume 146 of {\em Springer
  Optim. Appl.}, pages 225--250. Springer, Cham, 2019.

\bibitem{HiMa22}
Fumio Hiroshima and Oliver Matte.
\newblock Ground states and associated path measures in the renormalized
  {N}elson model.
\newblock {\em Rev. Math. Phys.}, 34(2):Paper No. 2250002, 84, 2022.

\bibitem{Kato57}
Tosio Kato.
\newblock On the eigenfunctions of many-particle systems in quantum mechanics.
\newblock {\em Comm. Pure Appl. Math.}, 10:151--177, 1957.

\bibitem{Kato72}
Tosio Kato.
\newblock Schr\"{o}dinger operators with singular potentials.
\newblock {\em Israel J. Math.}, 13:135--148 (1973), 1972.

\bibitem{LL}
Elliott~H. Lieb and Michael Loss.
\newblock Existence of atoms and molecules in non-relativistic quantum
  electrodynamics.
\newblock {\em Adv. Theor. Math. Phys.}, 7(4):667--710, 2003.

\bibitem{LT1986}
Elliott~H. Lieb and Walter~E. Thirring.
\newblock Universal nature of van der waals forces for coulomb systems.
\newblock {\em Phys. Rev. A}, 34:40--46, Jul 1986.

\bibitem{MS10}
Oliver Matte and Edgardo Stockmeyer.
\newblock Exponential localization of hydrogen-like atoms in relativistic
  quantum electrodynamics.
\newblock {\em Comm. Math. Phys.}, 295(2):551--583, 2010.

\bibitem{Si76}
Barry Simon.
\newblock Universal diamagnetism of spinless bose systems.
\newblock {\em Phys. Rev. Lett.}, 36:1083--1084, May 1976.

\end{thebibliography}



\end{document}



 
      




