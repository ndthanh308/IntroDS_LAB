% Please make sure you insert your data according to the instructions in PoSauthmanual.pdf
\documentclass[a4paper,11pt]{article}
\usepackage{pos}
\usepackage{subcaption}
\usepackage{booktabs}

\newcommand{\atel}[1]{\href{http://www.astronomerstelegram.org/?read=#1}{ATel {#1}}}

\title{IceCube search for neutrinos from novae}

\ShortTitle{IceCube search for neutrinos from novae}

% Don't change:
\author{The IceCube Collaboration \\{\normalsize \normalfont(a complete list of authors can be found at the end of the proceedings)}\\}

% Your emails:
\emailAdd{jessie.thwaites@icecube.wisc.edu}
\emailAdd{justin.vandenbroucke@wisc.edu}

\abstract{
Despite being one of the longest known classes of astrophysical transients, novae continue to present modern surprises. The \textit{Fermi}-LAT discovered that many if not all novae are GeV gamma ray sources, even though theoretical models had not even considered them as a possible source class. More recently, MAGIC and H.E.S.S. detected TeV gamma rays from a nova. Moreover, there is strong evidence that the gamma rays are produced hadronically, and that the long-studied optical emission by novae is also shock-powered. If this is true, novae should emit a neutrino signal correlated with their gamma-ray and optical signals. We present the first search for neutrinos from novae. Because the neutrino energy spectrum is expected to match the gamma-ray spectrum, we use an IceCube DeepCore event selection focused on GeV-TeV neutrinos. We present results from two searches, one for neutrinos correlated with gamma-ray emission and one for neutrinos correlated with optical emission. The event selection presented here is promising for additional astrophysical transients including gamma-ray bursts and gravitational wave sources.


\vspace{4mm}
{\bfseries Corresponding authors:}
Jessie Thwaites$^{1}$, Justin Vandenbroucke$^{1*}$\\
{$^{1}$ \itshape University of Wisconsin-Madison and Wisconsin IceCube Particle Astrophysics Center}\\[4mm]
$^*$ Presenter

\ConferenceLogo{PoS_ICRC2023_logo.pdf}

\FullConference{The 38th International Cosmic Ray Conference (ICRC2023)\\ 26 July -- 3 August, 2023\\ Nagoya, Japan}
}

\begin{document}

\maketitle

\section{Introduction}

Novae, one of the longest known, historical classes of astrophysical transients, continue to surprise us.  In addition to the GeV gamma rays discovered by Fermi LAT, MAGIC and H.E.S.S. have recently discovered nova emission approaching the TeV scale.  Those novae that are brightest in the optical band have been detected in gamma rays, indicating that many or potentially all novae emit gamma rays and that those that have not been detected are simply below instrument detection thresholds.  Furthermore, strong correlation between the optical and gamma-ray light curves indicates that shock acceleration powers not only the gamma-ray emission, but also the optical emission and therefore the total bolometric power of novae.  If the gamma-ray emission mechanism is hadronic rather than leptonic, neutrinos are expected to accompany the gamma-ray signal.  Novae could even be brighter in neutrinos than in gamma rays if the gamma rays are partially absorbed within the source.  We present the first search for neutrinos from novae, using the IceCube Neutrino Observatory. We use a new event selection, GRECO Astronomy \cite[Appendix B]{IceCube:2022lnv}, which uses IceCube-DeepCore, the densely instrumented inner subarray of IceCube. 
%Specifically, we use a new event selection, GRECO Astronomy, which uses DeepCore, the densely instrumented inner region of IceCube.

\subsection{Nova Sample}
The catalog of novae used in this search is described in full in \cite[Appendix A]{IceCube:2022lnv}. At the time of that study, we only had processed GRECO Astronomy data spanning April 2012 to May 2020. With the addition of two more years of processed data (up to October 2022), it is now possible to include nova RS Ophiuchi in the list of neutrino source candiates. All of these novae, including RS Oph, are shown in Figure \ref{fig:skymap}.

% Figure environment removed

\subsection{Analysis techniques}
The analysis uses an unbinned maximum likelihood technique described in \cite{IceCube:2022lnv}. We assume neutrino emission with a power-law spectrum $dN/dE \propto E^{-\gamma}$, with $\gamma$ the spectral index. We maximize the likelihood and define our test statistic (TS) as the log-likelihood ratio between the best-fit signal and the background hypothesis
\begin{equation}
    \mathrm{TS} = - 2 \ln \left[\frac{\mathcal{L}(\hat{n}_{s}, \hat{\gamma})}{\mathcal{L}(n_{s}=0)}\right],
\end{equation}
with $\hat{n}_s$ the best-fit number of signal events and $\hat{\gamma}$ the spectral index of a given source. In the case where $\hat{n}_{s}=0$, our observed $\mathrm{TS}=0$, which represents an underfluctuation with respect to the expected number of background events in the short time window considered.

\section{Individual gamma-ray correlation analysis} 
\label{sec:individual}
We search for neutrino emission coincident with each of the gamma-ray detected novae, as described in \cite{IceCube:2022lnv}. For each nova, we define our on-time window to be the detection time of the nova by \textit{Fermi}-LAT. The time windows are chosen to match those given by \cite{Gordon:2020fqv} for most novae, with a few exceptions. Two novae, V745 Sco and V1535 Sco, were identified as candidate gamma-ray sources  because they did not reach a detection at the 3$\sigma$ level in \cite{Franckowiak:2017iwj}, so we use the detection times reported in that paper. Two other novae, V3890 Sgr and V1707 Sco, had more recent outbursts reported by \textit{Fermi}-LAT in real-time via the Astronomer's Telegram (ATel), so we use their detection times as reported in their ATels (\atel{13114} and \atel{13116}, respectively). The time window for each nova is given in Table \ref{tab:gamma_results}.

We do not detect any significant neutrino emission from any of the novae, so we set 90\% confidence level (CL) upper limits. All of these limits are plotted in Fig. \ref{fig:ul_panel}. 

% Figure environment removed

\subsection{Search for neutrinos from RS Ophiuchi}
For the analysis searching for neutrino emission from nova RS Oph, we use two additional processed years of GRECO Astronomy data, including data up to October 2022. We use the same analysis as that of the other 16 gamma-ray detected novae. To chose the time window considered for RS Oph, we again use the detection time window by \textit{Fermi}-LAT given in \cite{MAGIC:2022rmr}, with a total duration of 30 days. This also covers the TeV detections of RS Ophiuchi reported by H.E.S.S. \cite{HESS:2022qap} and MAGIC \cite{MAGIC:2022rmr}. 

% Figure environment removed

% Figure environment removed

% Figure environment removed

No significant neutrino emission is found from nova RS Oph. We compare the observed test statistic (TS) to a distribution of 10,000 pseudo-experiments, generated from scrambling our data in right ascension. The observed test statistic, compared to the background TS distribution for this source can be seen in Fig \ref{fig:tsd}. We then place 90\% CL upper limits using the one-sided Neyman-Pearson construction. The upper limit assuming a spectrum $dN/dE \propto E^{-2}$ is plotted in Fig. \ref{fig:magic_comp}, compared to the upper limit for high-energy neutrino emission computed in real-time using the Fast Response Analysis \cite{IceCube:2020mzw}, which uses the Gamma-ray follow-up (GFU, \cite{IceCube:2016cqr}) data sample, with results reported in \atel{14851}. These are compared to a joint fit for gamma-ray observations from \textit{Fermi}-LAT and MAGIC and neutrino expectations calculated in \cite{MAGIC:2022rmr}. 

We set upper limits for a power-law spectrum with 3 different spectral indices, as well as a physical model for neutrino emission in novae \cite{Bednarek:2022vey}, shown in Fig. \ref{fig:ul_panel_rs_oph}. We also show a joint fit to the gamma-ray observations from \textit{Fermi}-LAT and H.E.S.S. at the peak of the gamma-ray emission (August 9th, 2021), and observations five days later (August 13th, 2021). An updated table of results, including analysis results for nova RS Oph, is provided in Table \ref{tab:gamma_results}. 

\begin{table*}
\centering
    %make table fit in margins
    \tabcolsep=4.5pt\relax
    \small
\begin{tabular}{ccccccccccc}
\toprule
Name &                 $\alpha$ & $\delta$ & MJD$_{\mathrm{start}}$ & MJD$_{\mathrm{stop}}$ &  $\Delta$T & TS & $\hat{n}_{s}$ &$\hat{\gamma}$ &   pre-trial & $E^2F$ @ 1~TeV \\
&  &  & &  &(days)  &  & & &   $p$-value & (GeV cm$^{-2}$)\\
\midrule
V1324 Sco & 267.7$^{\circ}$ & -32.6$^{\circ}$ &                56093.0 &               56110.0 & 17.0 &         0.00 &                     0.0& -- & 1.000 & 26.4\\
 V959 Mon &   99.9$^{\circ}$ & +5.9$^{\circ}$ &                56097.0 &               56119.0 & 22.0 &         1.14 &                  14.8& 2.75 & 0.197 & 9.76\\
 V339 Del & 305.9$^{\circ}$& +20.8$^{\circ}$ &                56520.0 &               56547.0 & 27.0 &         0.00 &                     0.0& -- & 1.000 & 2.10\\
V1369 Cen & 208.7$^{\circ}$& -59.2$^{\circ}$ &                56634.0 &               56672.0 & 38.0 &          0.01 &                   3.8& 4.00 & 0.070 & 62.9\\
 V745 Sco & 268.8$^{\circ}$& -33.2$^{\circ}$ &                56694.0 &               56695.0 & 1.0 &         0.00 &                     0.0& -- & 1.000 & 7.45\\
V1535 Sco & 255.9$^{\circ}$& -35.1$^{\circ}$ &                57064.0 &               57071.0 & 7.0 &         6.95 &                  59.9& 3.02 & 0.002 & 42.6\\
V5668 Sgr & 279.2$^{\circ}$& -28.9$^{\circ}$ &                57105.0 &               57158.0 &  53.0 &        1.30 &                  62.0& 3.16 & 0.112 & 69.6\\
 V407 Lup & 232.3$^{\circ}$& -44.8$^{\circ}$ &                57657.0 &               57660.0 & 3.0 &         0.00 &                     0.0& -- & 1.000 & 11.2\\
V5855 Sgr & 272.6$^{\circ}$& -27.5$^{\circ}$ &                57686.0 &               57712.0 & 26.0 &           0.00 &                     0.0& -- & 1.000 & 28.3\\
V5856 Sgr & 275.2$^{\circ}$& -28.4$^{\circ}$ &                57700.0 &               57715.0 & 15.0 &          5.13 &                  40.3& 2.81 & 0.015 & 47.0\\
 V549 Vel & 132.6$^{\circ}$& -47.8$^{\circ}$ &                58037.0 &               58070.0 & 33.0 &          0.22 &                  22.4& 4.00 & 0.062 & 51.6\\
 V357 Mus & 171.6$^{\circ}$& -65.5$^{\circ}$ &                58129.0 &               58156.0 & 27.0 &          0.01 &                   5.0& 4.00 & 0.115 & 52.3\\
 V906 Car & 159.1$^{\circ}$& -59.6$^{\circ}$ &                58216.0 &               58239.0 & 23.0 &         0.00 &                     0.0& -- & 1.000 & 36.5\\
 V392 Per &  70.8$^{\circ}$& +47.4$^{\circ}$ &                58238.0 &               58246.0 & 8.0 &          0.88 &                  15.4& 3.64 & 0.373 & 1.92\\
V3890 Sgr & 277.7$^{\circ}$& -24.0$^{\circ}$ &                58718.0 &               58739.0 & 21.0 &          0.00 &                     0.0& -- & 1.000 & 22.6\\
V1707 Sco & 264.3$^{\circ}$& -35.2$^{\circ}$ &                58740.0 &               58744.0 & 4.0 &          0.00 &                     0.0& -- & 1.000 & 10.6\\
RS Oph & 267.6$^{\circ}$ & -6.7$^{\circ}$ & 59434.8 & 59464.8 & 30.0 & 0.28 & 3.8 & 2.4 & 0.255 & 16.1 \\
\bottomrule
\end{tabular}
\caption{Results from the gamma-ray correlation analysis. MJD$_{\mathrm{start}}$ and MJD$_{\mathrm{stop}}$ represent the beginning and end of when a nova was detected by \textit{Fermi}-LAT, respectively. This is the same time window used for the neutrino search. The observed test statistic (TS), best-fit number of signal events ($\hat{n}_{s}$) and best-fit spectral index ($\hat{\gamma}$) for the neutrino search from each nova are given here (Note that $\hat{\gamma}$ is undefinded in the case of $\hat{n}_{s}=0$). The $p$-values shown are before accounting for the factor accrued from performing multiple searches. The final column gives the time-integrated flux upper limit for each nova, assuming a power-law spectrum with a spectral index of 2.0 ($dN/dE \propto E^{-2}$) and including the systematic uncertainty as discussed in \cite[Section 4.3]{IceCube:2022lnv}.}
\label{tab:gamma_results}
\end{table*}

\section{Stacking analysis}

In addition to searching individual novae for neutrino emission, we performed two stacking analyses.  One stacked 16 gamma-ray detected novae, excluding RS Oph, because the analysis was designed prior to its 2021 outburst, to test the hypothesis that neutrino emission is proportional to the gamma-ray flux.  The other stacked $N$ optically-detected novae, to test the hypothesis that neutrino emission is proportional to the optical flux.  While gamma-ray and neutrino emission are tightly coupled at production in hadronic models, there may be substantial absorption of the gamma-ray signal, in which case the optical flux is expected to better trace the neutrino emission.  Under the particular proportionality models tested by these stacking analyses, they achieve sensitivity superior to the single-nova analyses presented in Section \ref{sec:individual}.  Results from the stacking analyses are shown in Fig.~\ref{fig:llh_spaces}.  No significant signal is found in either analysis, and we therefore place upper limits on the total number of signal neutrinos summed over all novae as a function of the power law index, for each of the two stacking hypotheses.

\section{Discussion/Conclusion}

We have presented several searches for neutrinos from novae using a new IceCube DeepCore event selection named GRECO Astronomy.  In~\cite{IceCube:2022lnv}, we described searches for neutrinos from 16 individual gamma-ray-detected novae.  That analysis was developed before the 2021 outburst of RS Oph, which was detected by MAGIC and H.E.S.S.  In addition to the real-time search for TeV-PeV neutrinos from that outburst reported in~\cite{2021ATel14851....1P}, we have now added an archival search for GeV-TeV neutrinos from RS Oph using GRECO Astronomy.  In addition to the search for neutrinos from individual novae, we performed two stacking searches, one based on gamma-ray fluxes and one based on optical fluxes.  There is no evidence for neutrino emission from novae in any of the analyses, and the upper limits we present are an order of magnitude above the measured gamma-ray fluxes and corresponding neutrino predictions.  Nevertheless, these upper limits can already reject scenarios in which a large fraction of the gamma-ray emission is absorbed within the source.

This analysis uses a new event selection, GRECO Astronomy, which is well suited for astrophysical transients and has already been used for other source classes including gravitational wave sources and gamma-ray bursts.  %Although the atmospheric muon and neutrino backgrounds are substantial at these relatively low energies, they pose less of a challenge for transient analyses than for long-duration (mulit-year) analyses.
The IceCube Upgrade is a fully funded project now underway to install an additional seven strings within DeepCore, with even denser instrumentation than DeepCore both horizontally and vertically.  It will directly improve the performance of IceCube in the GeV-TeV band.  In particular, event direction reconstruction will improve over that demonstrated with GRECO Astronomy, lowering the effective background rate in analyses of astrophysical sources.  In addition to the neutrino paticle physics for which it was designed, the IceCube Upgrade will enable IceCube to continue to advance neutrino astronomy in the GeV-TeV band, where gamma-ray telescopes have demonstrated the sky to be full of a variety of surprising sources including novae.

%\clearpage

% Figure environment removed

% Bibtex references:
\bibliographystyle{ICRC}
\bibliography{references}

\clearpage

%The following list of authors, affiliations and funding agencies will be updated at the day of submission. The following template is a placeholder generated via https://authorlist.icecube.wisc.edu/icecube on March 25, 2023 and will be updated.
\section*{Full Author List: IceCube Collaboration}

\scriptsize
\noindent
R. Abbasi$^{17}$,
M. Ackermann$^{63}$,
J. Adams$^{18}$,
S. K. Agarwalla$^{40,\: 64}$,
J. A. Aguilar$^{12}$,
M. Ahlers$^{22}$,
J.M. Alameddine$^{23}$,
N. M. Amin$^{44}$,
K. Andeen$^{42}$,
G. Anton$^{26}$,
C. Arg{\"u}elles$^{14}$,
Y. Ashida$^{53}$,
S. Athanasiadou$^{63}$,
S. N. Axani$^{44}$,
X. Bai$^{50}$,
A. Balagopal V.$^{40}$,
M. Baricevic$^{40}$,
S. W. Barwick$^{30}$,
V. Basu$^{40}$,
R. Bay$^{8}$,
J. J. Beatty$^{20,\: 21}$,
J. Becker Tjus$^{11,\: 65}$,
J. Beise$^{61}$,
C. Bellenghi$^{27}$,
C. Benning$^{1}$,
S. BenZvi$^{52}$,
D. Berley$^{19}$,
E. Bernardini$^{48}$,
D. Z. Besson$^{36}$,
E. Blaufuss$^{19}$,
S. Blot$^{63}$,
F. Bontempo$^{31}$,
J. Y. Book$^{14}$,
C. Boscolo Meneguolo$^{48}$,
S. B{\"o}ser$^{41}$,
O. Botner$^{61}$,
J. B{\"o}ttcher$^{1}$,
E. Bourbeau$^{22}$,
J. Braun$^{40}$,
B. Brinson$^{6}$,
J. Brostean-Kaiser$^{63}$,
R. T. Burley$^{2}$,
R. S. Busse$^{43}$,
D. Butterfield$^{40}$,
M. A. Campana$^{49}$,
K. Carloni$^{14}$,
E. G. Carnie-Bronca$^{2}$,
S. Chattopadhyay$^{40,\: 64}$,
N. Chau$^{12}$,
C. Chen$^{6}$,
Z. Chen$^{55}$,
D. Chirkin$^{40}$,
S. Choi$^{56}$,
B. A. Clark$^{19}$,
L. Classen$^{43}$,
A. Coleman$^{61}$,
G. H. Collin$^{15}$,
A. Connolly$^{20,\: 21}$,
J. M. Conrad$^{15}$,
P. Coppin$^{13}$,
P. Correa$^{13}$,
D. F. Cowen$^{59,\: 60}$,
P. Dave$^{6}$,
C. De Clercq$^{13}$,
J. J. DeLaunay$^{58}$,
D. Delgado$^{14}$,
S. Deng$^{1}$,
K. Deoskar$^{54}$,
A. Desai$^{40}$,
P. Desiati$^{40}$,
K. D. de Vries$^{13}$,
G. de Wasseige$^{37}$,
T. DeYoung$^{24}$,
A. Diaz$^{15}$,
J. C. D{\'\i}az-V{\'e}lez$^{40}$,
M. Dittmer$^{43}$,
A. Domi$^{26}$,
H. Dujmovic$^{40}$,
M. A. DuVernois$^{40}$,
T. Ehrhardt$^{41}$,
P. Eller$^{27}$,
E. Ellinger$^{62}$,
S. El Mentawi$^{1}$,
D. Els{\"a}sser$^{23}$,
R. Engel$^{31,\: 32}$,
H. Erpenbeck$^{40}$,
J. Evans$^{19}$,
P. A. Evenson$^{44}$,
K. L. Fan$^{19}$,
K. Fang$^{40}$,
K. Farrag$^{16}$,
A. R. Fazely$^{7}$,
A. Fedynitch$^{57}$,
N. Feigl$^{10}$,
S. Fiedlschuster$^{26}$,
C. Finley$^{54}$,
L. Fischer$^{63}$,
D. Fox$^{59}$,
A. Franckowiak$^{11}$,
A. Fritz$^{41}$,
P. F{\"u}rst$^{1}$,
J. Gallagher$^{39}$,
E. Ganster$^{1}$,
A. Garcia$^{14}$,
L. Gerhardt$^{9}$,
A. Ghadimi$^{58}$,
C. Glaser$^{61}$,
T. Glauch$^{27}$,
T. Gl{\"u}senkamp$^{26,\: 61}$,
N. Goehlke$^{32}$,
J. G. Gonzalez$^{44}$,
S. Goswami$^{58}$,
D. Grant$^{24}$,
S. J. Gray$^{19}$,
O. Gries$^{1}$,
S. Griffin$^{40}$,
S. Griswold$^{52}$,
K. M. Groth$^{22}$,
C. G{\"u}nther$^{1}$,
P. Gutjahr$^{23}$,
C. Haack$^{26}$,
A. Hallgren$^{61}$,
R. Halliday$^{24}$,
L. Halve$^{1}$,
F. Halzen$^{40}$,
H. Hamdaoui$^{55}$,
M. Ha Minh$^{27}$,
K. Hanson$^{40}$,
J. Hardin$^{15}$,
A. A. Harnisch$^{24}$,
P. Hatch$^{33}$,
A. Haungs$^{31}$,
K. Helbing$^{62}$,
J. Hellrung$^{11}$,
F. Henningsen$^{27}$,
L. Heuermann$^{1}$,
N. Heyer$^{61}$,
S. Hickford$^{62}$,
A. Hidvegi$^{54}$,
C. Hill$^{16}$,
G. C. Hill$^{2}$,
K. D. Hoffman$^{19}$,
S. Hori$^{40}$,
K. Hoshina$^{40,\: 66}$,
W. Hou$^{31}$,
T. Huber$^{31}$,
K. Hultqvist$^{54}$,
M. H{\"u}nnefeld$^{23}$,
R. Hussain$^{40}$,
K. Hymon$^{23}$,
S. In$^{56}$,
A. Ishihara$^{16}$,
M. Jacquart$^{40}$,
O. Janik$^{1}$,
M. Jansson$^{54}$,
G. S. Japaridze$^{5}$,
M. Jeong$^{56}$,
M. Jin$^{14}$,
B. J. P. Jones$^{4}$,
D. Kang$^{31}$,
W. Kang$^{56}$,
X. Kang$^{49}$,
A. Kappes$^{43}$,
D. Kappesser$^{41}$,
L. Kardum$^{23}$,
T. Karg$^{63}$,
M. Karl$^{27}$,
A. Karle$^{40}$,
U. Katz$^{26}$,
M. Kauer$^{40}$,
J. L. Kelley$^{40}$,
A. Khatee Zathul$^{40}$,
A. Kheirandish$^{34,\: 35}$,
J. Kiryluk$^{55}$,
S. R. Klein$^{8,\: 9}$,
A. Kochocki$^{24}$,
R. Koirala$^{44}$,
H. Kolanoski$^{10}$,
T. Kontrimas$^{27}$,
L. K{\"o}pke$^{41}$,
C. Kopper$^{26}$,
D. J. Koskinen$^{22}$,
P. Koundal$^{31}$,
M. Kovacevich$^{49}$,
M. Kowalski$^{10,\: 63}$,
T. Kozynets$^{22}$,
J. Krishnamoorthi$^{40,\: 64}$,
K. Kruiswijk$^{37}$,
E. Krupczak$^{24}$,
A. Kumar$^{63}$,
E. Kun$^{11}$,
N. Kurahashi$^{49}$,
N. Lad$^{63}$,
C. Lagunas Gualda$^{63}$,
M. Lamoureux$^{37}$,
M. J. Larson$^{19}$,
S. Latseva$^{1}$,
F. Lauber$^{62}$,
J. P. Lazar$^{14,\: 40}$,
J. W. Lee$^{56}$,
K. Leonard DeHolton$^{60}$,
A. Leszczy{\'n}ska$^{44}$,
M. Lincetto$^{11}$,
Q. R. Liu$^{40}$,
M. Liubarska$^{25}$,
E. Lohfink$^{41}$,
C. Love$^{49}$,
C. J. Lozano Mariscal$^{43}$,
L. Lu$^{40}$,
F. Lucarelli$^{28}$,
W. Luszczak$^{20,\: 21}$,
Y. Lyu$^{8,\: 9}$,
J. Madsen$^{40}$,
K. B. M. Mahn$^{24}$,
Y. Makino$^{40}$,
E. Manao$^{27}$,
S. Mancina$^{40,\: 48}$,
W. Marie Sainte$^{40}$,
I. C. Mari{\c{s}}$^{12}$,
S. Marka$^{46}$,
Z. Marka$^{46}$,
M. Marsee$^{58}$,
I. Martinez-Soler$^{14}$,
R. Maruyama$^{45}$,
F. Mayhew$^{24}$,
T. McElroy$^{25}$,
F. McNally$^{38}$,
J. V. Mead$^{22}$,
K. Meagher$^{40}$,
S. Mechbal$^{63}$,
A. Medina$^{21}$,
M. Meier$^{16}$,
Y. Merckx$^{13}$,
L. Merten$^{11}$,
J. Micallef$^{24}$,
J. Mitchell$^{7}$,
T. Montaruli$^{28}$,
R. W. Moore$^{25}$,
Y. Morii$^{16}$,
R. Morse$^{40}$,
M. Moulai$^{40}$,
T. Mukherjee$^{31}$,
R. Naab$^{63}$,
R. Nagai$^{16}$,
M. Nakos$^{40}$,
U. Naumann$^{62}$,
J. Necker$^{63}$,
A. Negi$^{4}$,
M. Neumann$^{43}$,
H. Niederhausen$^{24}$,
M. U. Nisa$^{24}$,
A. Noell$^{1}$,
A. Novikov$^{44}$,
S. C. Nowicki$^{24}$,
A. Obertacke Pollmann$^{16}$,
V. O'Dell$^{40}$,
M. Oehler$^{31}$,
B. Oeyen$^{29}$,
A. Olivas$^{19}$,
R. {\O}rs{\o}e$^{27}$,
J. Osborn$^{40}$,
E. O'Sullivan$^{61}$,
H. Pandya$^{44}$,
N. Park$^{33}$,
G. K. Parker$^{4}$,
E. N. Paudel$^{44}$,
L. Paul$^{42,\: 50}$,
C. P{\'e}rez de los Heros$^{61}$,
J. Peterson$^{40}$,
S. Philippen$^{1}$,
A. Pizzuto$^{40}$,
M. Plum$^{50}$,
A. Pont{\'e}n$^{61}$,
Y. Popovych$^{41}$,
M. Prado Rodriguez$^{40}$,
B. Pries$^{24}$,
R. Procter-Murphy$^{19}$,
G. T. Przybylski$^{9}$,
C. Raab$^{37}$,
J. Rack-Helleis$^{41}$,
K. Rawlins$^{3}$,
Z. Rechav$^{40}$,
A. Rehman$^{44}$,
P. Reichherzer$^{11}$,
G. Renzi$^{12}$,
E. Resconi$^{27}$,
S. Reusch$^{63}$,
W. Rhode$^{23}$,
B. Riedel$^{40}$,
A. Rifaie$^{1}$,
E. J. Roberts$^{2}$,
S. Robertson$^{8,\: 9}$,
S. Rodan$^{56}$,
G. Roellinghoff$^{56}$,
M. Rongen$^{26}$,
C. Rott$^{53,\: 56}$,
T. Ruhe$^{23}$,
L. Ruohan$^{27}$,
D. Ryckbosch$^{29}$,
I. Safa$^{14,\: 40}$,
J. Saffer$^{32}$,
D. Salazar-Gallegos$^{24}$,
P. Sampathkumar$^{31}$,
S. E. Sanchez Herrera$^{24}$,
A. Sandrock$^{62}$,
M. Santander$^{58}$,
S. Sarkar$^{25}$,
S. Sarkar$^{47}$,
J. Savelberg$^{1}$,
P. Savina$^{40}$,
M. Schaufel$^{1}$,
H. Schieler$^{31}$,
S. Schindler$^{26}$,
L. Schlickmann$^{1}$,
B. Schl{\"u}ter$^{43}$,
F. Schl{\"u}ter$^{12}$,
N. Schmeisser$^{62}$,
T. Schmidt$^{19}$,
J. Schneider$^{26}$,
F. G. Schr{\"o}der$^{31,\: 44}$,
L. Schumacher$^{26}$,
G. Schwefer$^{1}$,
S. Sclafani$^{19}$,
D. Seckel$^{44}$,
M. Seikh$^{36}$,
S. Seunarine$^{51}$,
R. Shah$^{49}$,
A. Sharma$^{61}$,
S. Shefali$^{32}$,
N. Shimizu$^{16}$,
M. Silva$^{40}$,
B. Skrzypek$^{14}$,
B. Smithers$^{4}$,
R. Snihur$^{40}$,
J. Soedingrekso$^{23}$,
A. S{\o}gaard$^{22}$,
D. Soldin$^{32}$,
P. Soldin$^{1}$,
G. Sommani$^{11}$,
C. Spannfellner$^{27}$,
G. M. Spiczak$^{51}$,
C. Spiering$^{63}$,
M. Stamatikos$^{21}$,
T. Stanev$^{44}$,
T. Stezelberger$^{9}$,
T. St{\"u}rwald$^{62}$,
T. Stuttard$^{22}$,
G. W. Sullivan$^{19}$,
I. Taboada$^{6}$,
S. Ter-Antonyan$^{7}$,
M. Thiesmeyer$^{1}$,
W. G. Thompson$^{14}$,
J. Thwaites$^{40}$,
S. Tilav$^{44}$,
K. Tollefson$^{24}$,
C. T{\"o}nnis$^{56}$,
S. Toscano$^{12}$,
D. Tosi$^{40}$,
A. Trettin$^{63}$,
C. F. Tung$^{6}$,
R. Turcotte$^{31}$,
J. P. Twagirayezu$^{24}$,
B. Ty$^{40}$,
M. A. Unland Elorrieta$^{43}$,
A. K. Upadhyay$^{40,\: 64}$,
K. Upshaw$^{7}$,
N. Valtonen-Mattila$^{61}$,
J. Vandenbroucke$^{40}$,
N. van Eijndhoven$^{13}$,
D. Vannerom$^{15}$,
J. van Santen$^{63}$,
J. Vara$^{43}$,
J. Veitch-Michaelis$^{40}$,
M. Venugopal$^{31}$,
M. Vereecken$^{37}$,
S. Verpoest$^{44}$,
D. Veske$^{46}$,
A. Vijai$^{19}$,
C. Walck$^{54}$,
C. Weaver$^{24}$,
P. Weigel$^{15}$,
A. Weindl$^{31}$,
J. Weldert$^{60}$,
C. Wendt$^{40}$,
J. Werthebach$^{23}$,
M. Weyrauch$^{31}$,
N. Whitehorn$^{24}$,
C. H. Wiebusch$^{1}$,
N. Willey$^{24}$,
D. R. Williams$^{58}$,
L. Witthaus$^{23}$,
A. Wolf$^{1}$,
M. Wolf$^{27}$,
G. Wrede$^{26}$,
X. W. Xu$^{7}$,
J. P. Yanez$^{25}$,
E. Yildizci$^{40}$,
S. Yoshida$^{16}$,
R. Young$^{36}$,
F. Yu$^{14}$,
S. Yu$^{24}$,
T. Yuan$^{40}$,
Z. Zhang$^{55}$,
P. Zhelnin$^{14}$,
M. Zimmerman$^{40}$\\
\\
$^{1}$ III. Physikalisches Institut, RWTH Aachen University, D-52056 Aachen, Germany \\
$^{2}$ Department of Physics, University of Adelaide, Adelaide, 5005, Australia \\
$^{3}$ Dept. of Physics and Astronomy, University of Alaska Anchorage, 3211 Providence Dr., Anchorage, AK 99508, USA \\
$^{4}$ Dept. of Physics, University of Texas at Arlington, 502 Yates St., Science Hall Rm 108, Box 19059, Arlington, TX 76019, USA \\
$^{5}$ CTSPS, Clark-Atlanta University, Atlanta, GA 30314, USA \\
$^{6}$ School of Physics and Center for Relativistic Astrophysics, Georgia Institute of Technology, Atlanta, GA 30332, USA \\
$^{7}$ Dept. of Physics, Southern University, Baton Rouge, LA 70813, USA \\
$^{8}$ Dept. of Physics, University of California, Berkeley, CA 94720, USA \\
$^{9}$ Lawrence Berkeley National Laboratory, Berkeley, CA 94720, USA \\
$^{10}$ Institut f{\"u}r Physik, Humboldt-Universit{\"a}t zu Berlin, D-12489 Berlin, Germany \\
$^{11}$ Fakult{\"a}t f{\"u}r Physik {\&} Astronomie, Ruhr-Universit{\"a}t Bochum, D-44780 Bochum, Germany \\
$^{12}$ Universit{\'e} Libre de Bruxelles, Science Faculty CP230, B-1050 Brussels, Belgium \\
$^{13}$ Vrije Universiteit Brussel (VUB), Dienst ELEM, B-1050 Brussels, Belgium \\
$^{14}$ Department of Physics and Laboratory for Particle Physics and Cosmology, Harvard University, Cambridge, MA 02138, USA \\
$^{15}$ Dept. of Physics, Massachusetts Institute of Technology, Cambridge, MA 02139, USA \\
$^{16}$ Dept. of Physics and The International Center for Hadron Astrophysics, Chiba University, Chiba 263-8522, Japan \\
$^{17}$ Department of Physics, Loyola University Chicago, Chicago, IL 60660, USA \\
$^{18}$ Dept. of Physics and Astronomy, University of Canterbury, Private Bag 4800, Christchurch, New Zealand \\
$^{19}$ Dept. of Physics, University of Maryland, College Park, MD 20742, USA \\
$^{20}$ Dept. of Astronomy, Ohio State University, Columbus, OH 43210, USA \\
$^{21}$ Dept. of Physics and Center for Cosmology and Astro-Particle Physics, Ohio State University, Columbus, OH 43210, USA \\
$^{22}$ Niels Bohr Institute, University of Copenhagen, DK-2100 Copenhagen, Denmark \\
$^{23}$ Dept. of Physics, TU Dortmund University, D-44221 Dortmund, Germany \\
$^{24}$ Dept. of Physics and Astronomy, Michigan State University, East Lansing, MI 48824, USA \\
$^{25}$ Dept. of Physics, University of Alberta, Edmonton, Alberta, Canada T6G 2E1 \\
$^{26}$ Erlangen Centre for Astroparticle Physics, Friedrich-Alexander-Universit{\"a}t Erlangen-N{\"u}rnberg, D-91058 Erlangen, Germany \\
$^{27}$ Technical University of Munich, TUM School of Natural Sciences, Department of Physics, D-85748 Garching bei M{\"u}nchen, Germany \\
$^{28}$ D{\'e}partement de physique nucl{\'e}aire et corpusculaire, Universit{\'e} de Gen{\`e}ve, CH-1211 Gen{\`e}ve, Switzerland \\
$^{29}$ Dept. of Physics and Astronomy, University of Gent, B-9000 Gent, Belgium \\
$^{30}$ Dept. of Physics and Astronomy, University of California, Irvine, CA 92697, USA \\
$^{31}$ Karlsruhe Institute of Technology, Institute for Astroparticle Physics, D-76021 Karlsruhe, Germany  \\
$^{32}$ Karlsruhe Institute of Technology, Institute of Experimental Particle Physics, D-76021 Karlsruhe, Germany  \\
$^{33}$ Dept. of Physics, Engineering Physics, and Astronomy, Queen's University, Kingston, ON K7L 3N6, Canada \\
$^{34}$ Department of Physics {\&} Astronomy, University of Nevada, Las Vegas, NV, 89154, USA \\
$^{35}$ Nevada Center for Astrophysics, University of Nevada, Las Vegas, NV 89154, USA \\
$^{36}$ Dept. of Physics and Astronomy, University of Kansas, Lawrence, KS 66045, USA \\
$^{37}$ Centre for Cosmology, Particle Physics and Phenomenology - CP3, Universit{\'e} catholique de Louvain, Louvain-la-Neuve, Belgium \\
$^{38}$ Department of Physics, Mercer University, Macon, GA 31207-0001, USA \\
$^{39}$ Dept. of Astronomy, University of Wisconsin{\textendash}Madison, Madison, WI 53706, USA \\
$^{40}$ Dept. of Physics and Wisconsin IceCube Particle Astrophysics Center, University of Wisconsin{\textendash}Madison, Madison, WI 53706, USA \\
$^{41}$ Institute of Physics, University of Mainz, Staudinger Weg 7, D-55099 Mainz, Germany \\
$^{42}$ Department of Physics, Marquette University, Milwaukee, WI, 53201, USA \\
$^{43}$ Institut f{\"u}r Kernphysik, Westf{\"a}lische Wilhelms-Universit{\"a}t M{\"u}nster, D-48149 M{\"u}nster, Germany \\
$^{44}$ Bartol Research Institute and Dept. of Physics and Astronomy, University of Delaware, Newark, DE 19716, USA \\
$^{45}$ Dept. of Physics, Yale University, New Haven, CT 06520, USA \\
$^{46}$ Columbia Astrophysics and Nevis Laboratories, Columbia University, New York, NY 10027, USA \\
$^{47}$ Dept. of Physics, University of Oxford, Parks Road, Oxford OX1 3PU, United Kingdom\\
$^{48}$ Dipartimento di Fisica e Astronomia Galileo Galilei, Universit{\`a} Degli Studi di Padova, 35122 Padova PD, Italy \\
$^{49}$ Dept. of Physics, Drexel University, 3141 Chestnut Street, Philadelphia, PA 19104, USA \\
$^{50}$ Physics Department, South Dakota School of Mines and Technology, Rapid City, SD 57701, USA \\
$^{51}$ Dept. of Physics, University of Wisconsin, River Falls, WI 54022, USA \\
$^{52}$ Dept. of Physics and Astronomy, University of Rochester, Rochester, NY 14627, USA \\
$^{53}$ Department of Physics and Astronomy, University of Utah, Salt Lake City, UT 84112, USA \\
$^{54}$ Oskar Klein Centre and Dept. of Physics, Stockholm University, SE-10691 Stockholm, Sweden \\
$^{55}$ Dept. of Physics and Astronomy, Stony Brook University, Stony Brook, NY 11794-3800, USA \\
$^{56}$ Dept. of Physics, Sungkyunkwan University, Suwon 16419, Korea \\
$^{57}$ Institute of Physics, Academia Sinica, Taipei, 11529, Taiwan \\
$^{58}$ Dept. of Physics and Astronomy, University of Alabama, Tuscaloosa, AL 35487, USA \\
$^{59}$ Dept. of Astronomy and Astrophysics, Pennsylvania State University, University Park, PA 16802, USA \\
$^{60}$ Dept. of Physics, Pennsylvania State University, University Park, PA 16802, USA \\
$^{61}$ Dept. of Physics and Astronomy, Uppsala University, Box 516, S-75120 Uppsala, Sweden \\
$^{62}$ Dept. of Physics, University of Wuppertal, D-42119 Wuppertal, Germany \\
$^{63}$ Deutsches Elektronen-Synchrotron DESY, Platanenallee 6, 15738 Zeuthen, Germany  \\
$^{64}$ Institute of Physics, Sachivalaya Marg, Sainik School Post, Bhubaneswar 751005, India \\
$^{65}$ Department of Space, Earth and Environment, Chalmers University of Technology, 412 96 Gothenburg, Sweden \\
$^{66}$ Earthquake Research Institute, University of Tokyo, Bunkyo, Tokyo 113-0032, Japan \\

\subsection*{Acknowledgements}

\noindent
The authors gratefully acknowledge the support from the following agencies and institutions:
USA {\textendash} U.S. National Science Foundation-Office of Polar Programs,
U.S. National Science Foundation-Physics Division,
U.S. National Science Foundation-EPSCoR,
Wisconsin Alumni Research Foundation,
Center for High Throughput Computing (CHTC) at the University of Wisconsin{\textendash}Madison,
Open Science Grid (OSG),
Advanced Cyberinfrastructure Coordination Ecosystem: Services {\&} Support (ACCESS),
Frontera computing project at the Texas Advanced Computing Center,
U.S. Department of Energy-National Energy Research Scientific Computing Center,
Particle astrophysics research computing center at the University of Maryland,
Institute for Cyber-Enabled Research at Michigan State University,
and Astroparticle physics computational facility at Marquette University;
Belgium {\textendash} Funds for Scientific Research (FRS-FNRS and FWO),
FWO Odysseus and Big Science programmes,
and Belgian Federal Science Policy Office (Belspo);
Germany {\textendash} Bundesministerium f{\"u}r Bildung und Forschung (BMBF),
Deutsche Forschungsgemeinschaft (DFG),
Helmholtz Alliance for Astroparticle Physics (HAP),
Initiative and Networking Fund of the Helmholtz Association,
Deutsches Elektronen Synchrotron (DESY),
and High Performance Computing cluster of the RWTH Aachen;
Sweden {\textendash} Swedish Research Council,
Swedish Polar Research Secretariat,
Swedish National Infrastructure for Computing (SNIC),
and Knut and Alice Wallenberg Foundation;
European Union {\textendash} EGI Advanced Computing for research;
Australia {\textendash} Australian Research Council;
Canada {\textendash} Natural Sciences and Engineering Research Council of Canada,
Calcul Qu{\'e}bec, Compute Ontario, Canada Foundation for Innovation, WestGrid, and Compute Canada;
Denmark {\textendash} Villum Fonden, Carlsberg Foundation, and European Commission;
New Zealand {\textendash} Marsden Fund;
Japan {\textendash} Japan Society for Promotion of Science (JSPS)
and Institute for Global Prominent Research (IGPR) of Chiba University;
Korea {\textendash} National Research Foundation of Korea (NRF);
Switzerland {\textendash} Swiss National Science Foundation (SNSF);
United Kingdom {\textendash} Department of Physics, University of Oxford.

\end{document}