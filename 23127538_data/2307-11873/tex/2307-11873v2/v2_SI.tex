%\documentclass{article}
%\usepackage[utf8]{inputenc}
%\usepackage{amsmath}
%\usepackage{amsfonts}
%\usepackage{amssymb}
%\usepackage{mathtools}
%\DeclarePairedDelimiter{\ceil}{\lceil}{\rceil}
%\DeclarePairedDelimiter{\floor}{\lfloor}{\rfloor}
%\usepackage{xcolor,colortbl}
%\usepackage{adjustbox}
%
%% \renewcommand{\thepage}{S\arabic{page}} 
%\renewcommand{\thesection}{S\arabic{section}}  
%\renewcommand{\thetable}{S\arabic{table}}  
%\renewcommand{\thefigure}{S\arabic{figure}}
%\renewcommand{\theequation}{S\arabic{equation}}
%
%
%\bibliographystyle{unsrt}
%
%\title{It Is Easy For Multi-Issue Bundles To Advance Anti-Democratic Agendas: Supplementary Information}
%\author{Matthew I Jones, Matthew Chervenak, Nicholas Christakis}
%\date{}
%
%\begin{document}
%
%\maketitle

\section{Details of the spatial model}

There have been many variations on the spatial model of voting throughout the years. Here, we use one of the simplest models, in which voters are assigned ideal points in the $d$-dimensional hypercube, $[0,1]^d$. Issues are assigned two points, a ``yes'' and a ``no'', and each voter's utility on the issue is determined by the relative distance between her ideal point and the ``yes''/``no'' points. 

In most of the paper, these points are uniformly distributed throughout $[0,1]^n$, as in Figure \ref{fig:spatial_skew}a. When studying the effect of a shifting majority, however, we allow our population to drift to one corner of the hypercube by adjusting a parameter $q$. Each coordinate of a voter's ideal point is uniform between 0 and $\frac{1}{2}$ with probability $q$ and uniform between $\frac{1}{2}$ and 1 with probability $1-q$. Each issue's ``yes'' position is sampled the same way, while the ``no'' positions are reversed, being larger than $\frac{1}{2}$ with probability $q$ and less than $\frac{1}{2}$ with probability $1-q$. Therefore, when $q$ is much greater than $\frac{1}{2}$, the population tends to be in one corner of the hypercube with all the ``yes'' points, while all the ``no'' points are in the opposite corner, and most voters support most issues, as in Figure \ref{fig:spatial_skew}b.

% Figure environment removed

The use of a spatial model requires choosing the dimension of the underlying space. The natural choice for $d$ is application-specific. A bill deciding whether or not to give extra funding to schools may be fairly one-dimensional if each voter prefers to only fund schools who meet some standardized testing threshold. On the other hand, a group deciding which snacks to get for a party may have many different dimensions, such as sweet vs salty, healthy vs unhealthy, etc. The results in the main paper used $d=2$, and fortunately, this choice has very little impact on the value of the bundle, as can be seen in Figure \ref{fig:vary_dim}.

% Figure environment removed


\section{Derivation of bounds}

Here we derive the upper and lower bounds for the number of number of issues supported in a single bill with $\overline{n}$ supporters, as well as the limits on the number of supporters of the majority and minority supported subbundles.

\subsection{Single Bundle}

Let $n$ be the number of voters considering a bundle of $m$ issues. To avoid ties, assume that $n$ and $m$ are both odd. Suppose the number of voters who vote for the bundle (and therefore support over half the issues) is $\overline{n}$. We would like bounds on the number of issues that are supported, denoted $\overline{m}$. 

\subsubsection{Lower Bound}

In the worst-case scenario, none of the $n-\overline{n}$ voters who did not support the bundle support any of the issues and each of the $\overline{n}$ supporters support the fewest issues possible while still supporting the bundle, $\frac{m+1}{2}$ issues. Therefore, the number of positive utilities in the utility profile can be as low as $\overline{n} \frac{m+1}{2}$.

As far as issues are concerned, each issue can receive $\frac{n-1}{2}$ votes and still be opposed by a majority. Therefore, if the number of positive utilities is less than $m \frac{n-1}{2}$, all issues could have only minority support. In the worst case scenario (for the lower bound), any excess utility past this point is all given to the same issue until that issue has unanimous support to avoid passing more issues than necessary.

\begin{equation}
    \text{Fewest possible total positive utility:  } \overline{n}\frac{m+1}{2}
\end{equation}

\begin{equation}
    \text{Maximum positive utility before any issues pass:  } m \frac{n-1}{2}
\end{equation}

\begin{equation}
    \text{Minimum excess positive utility:  } \overline{n}\frac{m+1}{2} - m \frac{n-1}{2}
\end{equation}

\begin{equation}
    \text{Min num issues passed:  } \ceil*{ \frac{\overline{n}\frac{m+1}{2} - m \frac{n-1}{2}}{\frac{n+1}{2}}
    } = \ceil*{
    \frac{\overline{n}(m+1) - m (n-1)}{n+1}
    }
\end{equation}

Strictly speaking, these final two quantities could be negative (and in fact are for most values of $\overline{n}$), in which case no issues need to be passed.

\subsubsection{Upper Bound}

The computation for the upper bound is similar, except there is no longer any concern about excess positive utility. Instead, ``yes'' votes are all cast for the same issue until it reaches the necessary $\frac{n+1}{2}$ vote threshold to be passed. 

For the upper bound, suppose that the $\overline{n}$ supporters of the bundle support every single issue, and the $n - \overline{n}$ voters that did not support the bundle support $\frac{m-1}{2}$ issues, the maximum possible. Therefore, we have $(n-\overline{n}) \frac{m-1}{2}$ votes from non-supporters. Each issue needs $\frac{n+1}{2} - \overline{n}$ votes from non-bundle-supporters to gain a majority. 

\begin{equation}
    \text{Maximum positive utility:  } \overline{n}m + (n-\overline{n}) \frac{m-1}{2}
\end{equation}

\begin{equation}
    \text{Maximum non-supporter positive utility:  } (n-\overline{n}) \frac{m-1}{2}
\end{equation}

\begin{equation}
    \text{Minimum non-supporter votes needed to pass an issue:  } \frac{n+1}{2}-\overline{n}
\end{equation}

\begin{equation}
    \text{Max num issues passed:  } \floor*{
    \frac{(n-\overline{n})\frac{m-1}{2}}{\frac{n+1}{2} - \overline{n}}
    } = \floor*{
    \frac{(n-\overline{n})(m-1)}{n+1-2\overline{n}}
    }
\end{equation}


\subsubsection{Upper-Lower Bound Duality}

As an interesting note, the placement of positive and negative utility to minimize the number of issues passed with $\overline{n}$ bundle supporters is exactly the opposite of the placement when trying to maximize the number of issues passed with $n - \overline{n}$ supporters. This gives us the clean relationship

\begin{equation}
    \text{lower}(\overline{n}) = \text{upper}(n-\overline{n})
\end{equation}


\subsection{Majority/Minority Supported Subbundles}

Suppose we have $n$ voters and $m$ issues, $m_1$ of which have majority support and $m_2 = m-m_1$ of which do not. We assume $n$ and $m$ are odd to avoid ties, but it is unavoidable that either $m_1$ or $m_2$ will be even.

\textbf{Majority supported issue subbundle}

The majority support subbundle has $m_1$ issues, each with majority support. We want a lower bound on the number of supporters. To begin, we count the number of positive utilities in the subbundle: 

\begin{equation}
    \text{Amount of positive utility} \geq m_1 \left( \frac{n+1}{2} \right).
\end{equation}

Depending on if $m_1$ is even or odd, each voter can support just under half the issues before any voters support the subbundle.

\begin{equation}
    \text{positive utility before any supporters of subbundle} \leq n \left( \frac{m-c}{2} \right)
\end{equation}

where $c = \begin{cases}
1 & m \text{ is odd} \\
2 & m \text{ is even}
\end{cases}.$ 

Therefore, the number of excess votes is bounded by

\begin{equation}
    \text{Amount of excess positive utility} \geq m_1 \left( \frac{n+1}{2} \right) - n \left( \frac{m-c}{2} \right).
\end{equation}

By dividing by the amount of positive utility each voter can take before supporting every issue, we get our final bound:

\begin{equation}
    \text{Number of supporters} \geq \ceil*{
    \frac{m_1 \left( \frac{n+1}{2} \right) - n \left( \frac{m-c}{2} \right)}{\frac{m+c}{2}}
    } = \ceil*{
    \frac{m_1(n+1)-n(m-c)}{m+c}
    }
\end{equation}

\textbf{Minority supported issue subbundle}

This subbundle is similar. Because each issue is supported by a minority, we have the following bound on the number of positive utilities:

\begin{equation}
    \text{Amount of positive utility} \leq m_2\frac{n-1}{2}.
\end{equation}

We also have a lower bound on how much positive utility a supporter needs:

\begin{equation}
    \text{Amount of positive utility per supporter} \geq \frac{m_2+c}{2}
\end{equation}

where $c$ is as above. Dividing these two, we get

\begin{equation}
    \text{Number of supporters} \leq \floor*{
    \frac{m_2 \frac{n-1}{2}}{\frac{m_2+c}{2}}
    } = \floor*{
    \frac{m_2(n-1)}{m_2+c}
    }
\end{equation}

\section{Derivation of bundle value}

In the basic IID model, utility is $+1$ with probability $\frac{1}{2}$ and $-1$ with probability $\frac{1}{2}$, for an expected value of zero. In fact, the expected sum of all the utilities in a utility profile is zero. Despite this, the expected utility score is not zero. To see this, imagine that instead of voting for a bundle, we accept it if it has positive net utility and reject it if it has negative net utility. Both decisions result in positive value, so the expected value from this decision process is positive. While bundled voting may not be quite as effective as this, it still functions the same way, and manages to get a positive expected score by rejecting negative-utility bundles and accepting those with positive utility.

Approximately half of voters will support the bundle and half will reject it. However, by random chance and because $n$ is odd, the number of supporters will be slightly different than the number of detractors. Consider the difference between the number of supports and detractors. If these marginal voters are supporters, the bill passes and they provide positive value. If they are detractors, the bill fails and they \textbf{also} provide positive value (by rejecting a bundle with negative utility). Either way, the non-marginal voters cancel out in expectation (Figure \ref{fig:marginals} blocks A and B) and the marginal voters provide positive value to the bundle.

% Figure environment removed

We can estimate the number of marginal voters by first observing that the number of supporters is binomially distributed, where there are $n$ trials and the probability of success is $\frac{1}{2}$. For large $n$, the binomial distribution $B(n,\frac{1}{2})$ can be approximated by a normal distribution, so the number of supporters is proportional to $\mathcal{N}(n\frac{1}{2}, n\frac{1}{2}\frac{1}{2}) = \mathcal{N}(\frac{n}{2}, \frac{n}{4})$. The normal distribution has expected absolute deviation from the mean of $\sqrt{\frac{2}{\pi}}\sigma$, so the number of expected supporters varies from the mean by $\sqrt{\frac{2}{\pi}}\sqrt{\frac{n}{4}}$. Doubling this gives the expected difference between supporters and detractors, aka the number of marginal voters:

\begin{equation}
    E(\text{Number of marginal voters}) = \sqrt{\frac{2n}{\pi}}.
\end{equation}

We know that each marginal voter votes the same way, and we can use the exact same trick to determine how much value each of these voters adds to the bundle. The number of positive utilities is binomially distributed ($B(m,\frac{1}{2})$) which can be approximated by a normal distribution ($\mathcal{N}(\frac{m}{2}, \frac{m}{4})$) which has an expected absolute deviation from the mean of $\sqrt{\frac{2}{\pi}}\sqrt{\frac{m}{4}}$ which we double to get the marginal utility of voter:

\begin{equation}
    E(\text{Marginal utility of each voter}) = \sqrt{\frac{2m}{\pi}}.
\end{equation}

The rest of the utility for each voter cancels out (Figure \ref{fig:marginals} blocks C and D), so in the end the final utility of the bundle is just the number of marginal voters times the marginal utility per voter:

\begin{equation}
    E(\text{Expected utility of bundle}) = \frac{2}{\pi} \sqrt{mn}.
\end{equation}

The utility score of a bundle is normalized by the maximum possible utility, so the expected utility score is the expected utility divided by $nm$:

\begin{equation}
    E(\text{Expected utility score of bundle}) = \frac{2}{\pi \sqrt{mn}}.
\end{equation}


\section{ANES Data details}

The 11 final issues from the survey data (with their ANES variable name in parentheses) were:

\begin{itemize}
    \item Limits on foreign imports (V202231x)
    \item Change in immigration levels (V202232)
    \item Preferential hiring for blacks (V202252x)
    \item Level of government regulation (V202256)
    \item Government efforts to reduce income inequality (V202259x)
    \item Higher taxes on millionaires (V202325)
    \item Regulation of greenhouse gases (V202336x)
    \item Ban on assault-style weapons (V202344x)
    \item Free trade agreements (V202361x)
    \item Universal basic income of 12k/year (V202376x)
    \item Government spending on healthcare (V202380x)
\end{itemize}

For completeness, the four issues that we considered but discarded were:

\begin{itemize}
    \item Limits on campaign spending (V202225)
    \item Vaccine requirements in schools (V202331x)
    \item Background checks for guns (V202341x)
    \item Government action on opioid addiction (V202350x)
\end{itemize}

The subbundling scheme shown in Figure 5 of the main paper was the most extreme case found. All other subbundle schemes had a lesser degree of manipulation. Figure \ref{fig:anes_distribution} shows the distribution of minimum issue score for each sample of 31 voters. A more representative example of a manipulated subbundle scheme is shown in Figure \ref{fig:representative_gerry}, in which each subbundle has an issue with majority approval.

% Figure environment removed

% Figure environment removed


% \bibliography{refs}



% \end{document}
