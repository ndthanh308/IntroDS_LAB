\documentclass[10pt, letter, onecolumn]{arxiv}


\usepackage{kantlipsum, lipsum}
\usepackage{dm-colors}
\usepackage{amsmath}
\usepackage{pstricks, pst-node}
\usepackage{verbatim}
\usepackage{multirow}
\usepackage{scalerel}
\usepackage{booktabs}
\usepackage{enumitem}
\usepackage{xspace}
\usepackage{bm}
\usepackage{bbm}
\usepackage{mathtools}
\usepackage{soul}
\usepackage{epsfig}
\usepackage{graphicx}
\usepackage{amssymb}
\usepackage{colortbl}
\usepackage{csquotes}
\usepackage{setspace}
\usepackage{colortbl}
\usepackage{tabularx,ragged2e}
\usepackage{placeins}
\usepackage[symbol]{footmisc}
\usepackage[bibstyle=nature,citestyle=numeric-comp,%
            natbib=true,backend=biber,maxbibnames=99,%
            giveninits=false,sorting=none]{biblatex}
\usepackage{nameref}
\usepackage{varioref}
\usepackage[pagebackref=false,breaklinks=false,%
            colorlinks=true,bookmarks=true,citecolor=ourdarkblue,%
            urlcolor=ourdarkblue,linkcolor=ourdarkblue]{hyperref}
\usepackage[noabbrev,capitalize]{cleveref}
\usepackage{etoc}

\addbibresource{references.bib}
\renewcommand*{\bibfont}{\linespread{0.8}\footnotesize}

\graphicspath{{figures/}}


\newcommand{\pete}[1]{\textcolor{cyan}{[pete: #1]}}
\newcommand{\shek}[1]{\textcolor{red}{[shek: #1]}}
\newcommand{\tao}[1]{\textcolor{red}{[tao: #1]}}


\newcommand{\todo}[1]{\textcolor{blue}{\textbf{TODO:} (#1)}}
\newcommand{\mn}[1]{\textcolor{blue}{{(MN: #1)}}}
\newcommand{\one}[1]{\mathbbm{1}_{[#1]}}
\newcommand{\figref}[1]{Fig.~\ref{#1}}
\newcommand{\tabref}[1]{Tab.~\ref{#1}}
\newcommand{\secref}[1]{Sec.~\ref{#1}}
\newcommand{\AlgRef}[1]{Algorithm~\ref{#1}}
\newcommand{\equref}[1]{Equ.~(\ref{#1})}

\newenvironment{rednote}{\par\color{red}}{\par}


\def\eg{{\em e.g.,}}
\def\ie{{\em i.e.,}}
\def\vs{{\em vs.~}}
\def\etc{{\em etc}\xspace}
\def\etal{{\em et al.}\xspace}


% -------------  Title  ---------------------- 
\title{{\Huge{Towards Generalist Biomedical AI}

}}

% -------------  Authors ----------------------
% \author[$\ast$,1]{Anonymous Author List}

% % -------------  Authors ----------------------
\author[$\ast$, $\ddagger$, 1]{Tao Tu} %#1
\author[$\ast$, $\ddagger$, 2]{Shekoofeh Azizi}
\author[2]{\\Danny Driess}
\author[1]{Mike Schaekermann}
\author[1]{Mohamed Amin}
\author[1]{Pi-Chuan Chang}
\author[1]{Andrew Carroll}
\author[1]{\\Chuck Lau}
\author[2]{Ryutaro Tanno}
\author[2]{Ira Ktena}  %#10
\author[2]{Basil Mustafa}
\author[2]{Aakanksha Chowdhery}
\author[1]{Yun Liu}
\author[2]{\\Simon Kornblith} 
\author[2]{David Fleet}
\author[1]{Philip Mansfield} %#16
\author[1]{Sushant Prakash}
\author[1]{Renee Wong}
\author[1]{Sunny Virmani}
\author[1]{Christopher Semturs}  %#20
\author[2]{S Sara Mahdavi}
\author[1]{Bradley Green}
\author[1]{Ewa Dominowska}
\author[1]{Blaise Aguera y Arcas}
\author[2]{Joelle Barral}
\author[1]{Dale Webster}
\author[1]{Greg S. Corrado}
\author[1]{Yossi Matias}
\author[1]{Karan Singhal}
\author[2]{Pete Florence}  %#30
\author[$\dagger$, $\ddagger$,1]{\\Alan Karthikesalingam}
\author[$\dagger$, $\ddagger$,1]{Vivek Natarajan}  %#32

\affil[1]{Google Research, }
\affil[2]{Google DeepMind }


\renewcommand{\correspondingauthor}[1]{$\ast$~Equal contributions. %
                                       $\dagger$~Equal leadership. \\%
                                       $\ddagger$~Corresponding authors: \{taotu, shekazizi, alankarthi, natviv\}@google.com }



\begin{document}
\begin{refsection}


\begin{abstract}
Medicine is inherently multimodal, with rich data modalities spanning text, imaging, genomics, and more. Generalist biomedical artificial intelligence (AI) systems that flexibly encode, integrate, and interpret this data at scale can potentially enable impactful applications ranging from scientific discovery to care delivery.
To enable the development of these models, we first curate MultiMedBench, a new multimodal biomedical benchmark. MultiMedBench encompasses 14 diverse tasks such as medical question answering, mammography and dermatology image interpretation, radiology report generation and summarization, and genomic variant calling.
%
We then introduce Med-PaLM Multimodal (Med-PaLM M), our proof of concept for a generalist biomedical AI system. Med-PaLM M is a large multimodal generative model that flexibly encodes and interprets biomedical data including clinical language, imaging, and genomics with the \textit{same set of model weights}. Med-PaLM M reaches performance competitive with or exceeding the state of the art on all MultiMedBench tasks, often surpassing specialist models by a wide margin. We also report examples of zero-shot generalization to novel medical concepts and tasks, positive transfer learning across tasks, and emergent zero-shot medical reasoning.
%
To further probe the capabilities and limitations of Med-PaLM M, we conduct a radiologist evaluation of model-generated (and human) chest X-ray reports and observe encouraging performance across model scales. In a side-by-side ranking on 246 retrospective chest X-rays, clinicians express a pairwise preference for Med-PaLM M reports over those produced by radiologists in up to 40.50\% of cases, suggesting potential clinical utility.
%
While considerable work is needed to validate these models in real-world use cases, our results represent a milestone towards the development of generalist biomedical AI systems.

\end{abstract}



\maketitle

% ------------ SECTIONS ---------------------

\ProvidesPackage{article}
\usepackage{graphicx}
\usepackage{epsfig}


\usepackage{amssymb} %if documentclass is amsart, amsmath package isn't needed
%\usepackage[varg]{pxfonts}   %Nice fonts! The pxfonts package must be loaded AFTER amssymb, amsmath etc
%\usepackage{mathabx} % loaded this pacakge to use \square; it makes many symbols (as $\leq$) more beautiful
%\usepackage{epic}  % improvements to LaTex picture enviroment. DO NOT USE eepic (destroys diagonal lines)
%\usepackage{xypic}  % fancy matrices
%\usepackage[vcentermath]{youngtab} % Young diagrams
\usepackage[backref=page, colorlinks=true]{hyperref} % Links in the pdf. Backref option: each bibliographical entry denotes where it was cited
\usepackage{amsthm}
%\usepackage{showlabels} %{showkeys}        %must come after hyperref
%\usepackage[capitalise]{cleveref} % must be loaded after hyperref
\usepackage{color}
%\usepackage{subfigure}
\hypersetup{urlcolor=blue, citecolor=red}
%\renewcommand{\rmdefault}{pplx} 
%% A theorem-style with smaller font
\newtheoremstyle{note}% name
  {5pt}%              Space above
  {5pt}%              Space below
  {\footnotesize}%    Body font
  {}%                 Indent amount (empty = no indent, \parindent = para indent)
  {\bfseries}%        Thm head font: \itshape, \bfseries
  {.}%                Punctuation after thm head
  {.5em}%             Space after thm head: " " = normal interword space;  \newline = linebreak
  {}%                 Thm head spec (can be left empty, meaning `normal')



% Annoying conflict between amsthm and hyperref generates lots of warnings when two enviroments (e.g. Lemma and Proposition) have the same counter.
% http://tex.stackexchange.com/questions/25047/how-do-i-get-rid-of-particular-pdftex-warning-message
% But the guy doesn't explain how to solve it! 
% Also "silence" package doesn't work!

\theoremstyle{plain}
    \newtheorem{thm}{Theorem}
    %\renewcommand{\thethm}{\Alph{thm}}
    \newtheorem{prop}{Proposition}[section]
    \newtheorem{corol}[prop]{Corollary}
    \newtheorem{otherthm}[prop]{Theorem}
    \newtheorem{lemma}[prop]{Lemma}
    \newtheorem{scholium}[prop]{Scholium}
    %\newtheorem{problem}[prop]{Problem}
    %\newtheorem{scho}[prop]{Scholium}
	\newtheorem{claim}[prop]{Claim}
\theoremstyle{definition}
    \newtheorem{defn}[prop]{Definition}
    \newtheorem{question}[prop]{Question}
\theoremstyle{remark}
	%\newtheorem{example}[prop]{Example}
    \newtheorem*{ack}{Acknowledgements}
    \newtheorem{conj}[prop]{Conjecture}
\theoremstyle{note}
    \newtheorem{rem}[prop]{Remark}
	\newtheorem{example}[prop]{Example}

%  got confused because both "thm" and "otherthm" are Theorems. This fixes the problem:
%\crefname{thm}{Theorem}{Theorems} 
%\Crefname{thm}{Theorem}{Theorems} 
%\crefname{otherthm}{Theorem}{Theorems}  
%\Crefname{otherthm}{Theorem}{Theorems} 
% For some reason it also got confused with the prop's
%\crefname{prop}{Proposition}{Propositions} 
%\Crefname{prop}{Proposition}{Propositions}


%\newcommand{\closeremark}{\hfill$\triangleleft$}  % with THMTOOLS, I woudn't need this.


\numberwithin{equation}{section}         %makes labeled equations easier to find.

\setcounter{secnumdepth}{3}              % 2 => no subsubsection numbering

\setcounter{tocdepth}{2}                % 1 => only sections on the table of contents
\hypersetup{bookmarksdepth=subsection} % without this, the PDF bookmarks depth becomes the same as the tocdepth

\renewcommand{\theenumi}{\arabic{enumi}}  %changing enumerate's counter (can be r(R)oman, a(A)lph, arabic)
\renewcommand{\labelenumi}{\theenumi.}



% changing backref options (each bibliographical entry denotes where it was cited)
\renewcommand*{\backref}[1]{}
\renewcommand*{\backrefalt}[4]{\quad \tiny 
    \ifcase #1 (Not cited.)%
    \or        (Cited on page~#2.)%
    \else      (Cited on pages~#2.)%
    \fi}

\newcommand{\margem}[1]{\marginpar{{\scriptsize $\triangleright${#1}}}}
%\newcommand{\arxiv}[1]{Preprint \href{http://arxiv.org/abs/#1}{arXiv:{#1}}}
\newcommand{\doi}[1]{\href{http://dx.doi.org/#1}{doi:{#1}}}
\newcommand{\mr}[1]{\href{http://www.ams.org/mathscinet-getitem?mr=#1}{MR{#1}}}
\newcommand{\externallinkicon}{{\footnotesize $\Box\!\!\!^\nearrow$}}



% Customising cref: change the name for subsections. 
%\crefname{subsection}{\S}{\S\S} 
%\Crefname{subsection}{\S}{\S\S} 

%%%%%%%%%%%%%%%

\newcommand{\comment}[1]{}

\newcommand{\lS}{\mathbb{S}}
\newcommand{\C}{\mathbb{C}}
\newcommand{\R}{\mathbb{R}}
\newcommand{\Q}{\mathbb{Q}}
\newcommand{\Z}{\mathbb{Z}}
\newcommand{\N}{\mathbb{N}}
\newcommand{\T}{\mathbb{T}}
\newcommand{\I}{\mathbb{I}}
\newcommand{\G}{\mathbb{G}}
\newcommand{\D}{\mathbb{D}}
\newcommand{\K}{\mathbb{K}}
\renewcommand{\P}{\mathbb{P}}


\newcommand{\Si}{\Sigma}
\newcommand{\cA}{\mathcal{A}}\newcommand{\cB}{\mathcal{B}}\newcommand{\cC}{\mathcal{C}}
%\renewcommand{\cD}{\mathcal{D}}
\newcommand{\cE}{\mathcal{E}}\newcommand{\cF}{\mathcal{F}}
\newcommand{\cG}{\mathcal{G}}
%\newcommand{\cH}{\mathcal{H}}
\newcommand{\cI}{\mathcal{I}}
\newcommand{\cJ}{\mathcal{J}}\newcommand{\cK}{\mathcal{K}}
%\newcommand{\cL}{\mathcal{L}}  
\newcommand{\cM}{\mathcal{M}}\newcommand{\cN}{\mathcal{N}}\newcommand{\cO}{\mathcal{O}}
\newcommand{\cP}{\mathcal{P}}\newcommand{\cQ}{\mathcal{Q}}
%\newcommand{\cR}{\mathcal{R}}
\newcommand{\cS}{\mathcal{S}}\newcommand{\cT}{\mathcal{T}}\newcommand{\cU}{\mathcal{U}}
\newcommand{\cV}{\mathcal{V}}\newcommand{\cW}{\mathcal{W}}\newcommand{\cX}{\mathcal{X}}
\newcommand{\cY}{\mathcal{Y}}\newcommand{\cZ}{\mathcal{Z}}
\newcommand{\bA}{\mathbf{A}}\newcommand{\bB}{\mathbf{B}}\newcommand{\bC}{\mathbf{C}}
\newcommand{\bD}{\mathbf{D}}\newcommand{\bE}{\mathbf{E}}\newcommand{\bF}{\mathbf{F}}
\newcommand{\bG}{\mathbf{G}}\newcommand{\bH}{\mathbf{H}}\newcommand{\bI}{\mathbf{I}}
\newcommand{\bJ}{\mathbf{J}}\newcommand{\bK}{\mathbf{K}}\newcommand{\bL}{\mathbf{L}}
\newcommand{\bM}{\mathbf{M}}\newcommand{\bN}{\mathbf{N}}\newcommand{\bO}{\mathbf{O}}
\newcommand{\bP}{\mathbf{P}}\newcommand{\bQ}{\mathbf{Q}}\newcommand{\bR}{\mathbf{R}}
\newcommand{\bS}{\mathbf{S}}\newcommand{\bT}{\mathbf{T}}\newcommand{\bU}{\mathbf{U}}
\newcommand{\bV}{\mathbf{V}}\newcommand{\bW}{\mathbf{W}}\newcommand{\bX}{\mathbf{X}}
\newcommand{\bY}{\mathbf{Y}}\newcommand{\bZ}{\mathbf{Z}}
\newcommand{\tA}{\mathtt{A}}\newcommand{\tB}{\mathtt{B}}\newcommand{\tC}{\mathtt{C}}
\newcommand{\tD}{\mathtt{D}}\newcommand{\tE}{\mathtt{E}}\newcommand{\tF}{\mathtt{F}}
\newcommand{\tG}{\mathtt{G}}\newcommand{\tH}{\mathtt{H}}\newcommand{\tI}{\mathtt{I}}
\newcommand{\tJ}{\mathtt{J}}\newcommand{\tK}{\mathtt{K}}\newcommand{\tL}{\mathtt{L}}
\newcommand{\tM}{\mathtt{M}}\newcommand{\tN}{\mathtt{N}}\newcommand{\tO}{\mathtt{O}}
\newcommand{\tP}{\mathtt{P}}\newcommand{\tQ}{\mathtt{Q}}\newcommand{\tR}{\mathtt{R}}
\newcommand{\tS}{\mathtt{S}}\newcommand{\tT}{\mathtt{T}}\newcommand{\tU}{\mathtt{U}}
\newcommand{\tV}{\mathtt{V}}\newcommand{\tW}{\mathtt{W}}\newcommand{\tX}{\mathtt{X}}
\newcommand{\tY}{\mathtt{Y}}\newcommand{\tZ}{\mathtt{Z}}
\newcommand{\eps}{\varepsilon}
\renewcommand{\setminus}{\smallsetminus}
\renewcommand{\emptyset}{\varnothing}
\renewcommand{\angle}{\measuredangle}
\newcommand{\SL}{\mathrm{SL}}
\newcommand{\GL}{\mathrm{GL}}
\newcommand{\SO}{\mathrm{SO}}
\newcommand{\gl}{\frak{gl}}
\newcommand{\LE}{\mathit{LE}}
\newcommand{\Id}{\mathrm{Id}}
\newcommand{\Mat}{\mathrm{Mat}}
\newcommand{\Diag}{\mathrm{Diag}}
\newcommand{\Ad}{\mathrm{Ad}}
\newcommand{\id}{\mathrm{id}}
\newcommand{\Aut}{\mathit{Aut}}
\newcommand{\Homeo}{\mathit{Homeo}}
\newcommand{\Diff}{\mathit{Diff}}
\DeclareMathOperator{\diam}{diam}
\DeclareMathOperator{\trace}{tr}
\DeclareMathOperator*{\supp}{supp}
\DeclareMathOperator{\interior}{int}
\DeclareMathOperator{\esssup}{ess\;sup}
\newcommand{\wed}{{\mathord{\wedge}}}
\newcommand{\subN}{{}_N \negthinspace}
\newcommand{\subp}{{}_p \negthinspace}

\newcommand{\RP}{\mathbb{R}\mathrm{P}}
\newcommand{\CP}{\mathbb{C}\mathrm{P}}
\newcommand{\KP}{\mathbb{K}\mathrm{P}}
\renewcommand{\Im}{\mathrm{Im}\;}
\DeclareMathOperator{\codim}{codim}
\DeclareMathOperator{\Ker}{Ker}
%\DeclareMathOperator{\evil}{evil}
\DeclareMathOperator{\orb}{orb}
\DeclareMathOperator{\spa}{span}
\DeclareMathOperator{\sorb}{sorb}
\DeclareMathOperator{\pop}{pop}
\DeclareMathOperator{\area}{area}
%\DeclareMathOperator{\nst}{nst}
%\DeclareMathOperator{\naug}{naug}
\DeclareMathOperator{\dpop}{dpop}
\DeclareMathOperator{\rig}{rig}
\DeclareMathOperator{\rank}{rank}
\DeclareMathOperator{\acyc}{acyc}
\DeclareMathOperator{\Sat}{Sat}
\DeclareMathOperator{\B}{B}



\DeclareMathOperator{\Sl}{SL}
\DeclareMathOperator{\Psl}{PSL}

\DeclareMathOperator{\pr}{pr}
\DeclareMathOperator{\Jac}{Jac}


\DeclareMathOperator{\Hol}{Hol}

\DeclareMathOperator{\vol}{vol}
\DeclareMathOperator{\Gl}{GL}
\DeclareMathOperator{\const}{const}
\DeclareMathOperator{\Cr}{CR}
\DeclareMathOperator{\dist}{dist}
\DeclareMathOperator{\diag}{diag}

\newcommand{\prow}{\pi_\mathrm{r}}
\newcommand{\pcol}{\pi_\mathrm{c}}
%\DeclareMathOperator{\joco}{joco}
%\newcommand{\district}{\tD}
%\newcommand{\city}{\tC}
%\newcommand{\island}{\tI}
%\newcommand{\arch}{\tA}
%\newcommand{\world}{\tW}
%\newcommand{\unit}{\tU}
%\DeclareMathOperator{\iseq}{iseq}
\newcommand{\gra}[2]{G_{#1}(\mathbb{C}^{#2})}
\renewcommand{\pitchfork}{\;\;\makebox[0pt]{$\top$}\makebox[0pt]{\small $\cap$}\;\;} % transversality symbol
\newcommand{\cupro}{\smallsmile}
\newcommand{\brickpattern}{\drawline(.25,0)(.25,.25)\drawline(.75,.25)(.75,.5)\drawline(.25,.5)(.25,.75)\drawline(.75,.75)(.75,1)\put(0,0){\grid(1,1)(1,.25)}}
\newcommand{\diagonalpattern}{\drawline(.75,0)(1,.25)\drawline(.5,0)(1,.5)\drawline(.25,0)(1,.75)\drawline(0,0)(1,1)\drawline(0,.25)(.75,1)\drawline(0,.5)(.5,1)\drawline(0,.75)(.25,1)}
\newcommand{\lightdiagonalpattern}{\drawline(.5,0)(1,.5)\drawline(0,0)(1,1)\drawline(0,.5)(.5,1)}





\newpage
\begin{comment}
\section{System Architecture}
\label{appendix:architecture}
\system has a novel modularized system architecture with three key components: 
\emph{StreamManager}, 
\emph{TxnManager} and \emph{TxnScheduler}. 
These components are instantiated in each thread locally.
The execution outline of \system is presented in Algorithm~\ref{alg:algo}.
Transactional stream processing is continuous and potentially never ends (Line 1$\sim$8).
The dependency resolution and execution of state transactions are separated into two non-overlapping phases by punctuations~\cite{Tucker:2003:EPS:776752.776780} (Line 2 and 5), which guarantees that no subsequent input event will have a smaller timestamp. 
Effectively, a batch of state transactions is collected during the first phase, and processed during the second phase.

In the first phase (i.e., stream processing phase), 
the \emph{StreamManager} conducts preprocessing for every input event ($e$). Similar to some prior works~\cite{tstream}, state transactions may be issued but not immediately processed during preprocessing (Line 3).
The \emph{pre\_processing} and \emph{post\_processing} functions are exposed as APIs to users.
The \emph{TxnManager} handles dependency resolution (Line 4) among state transactions and insert decomposed operations to construct a \tpg. We discuss the detailed two-phase \tpg construction process in Section~\ref{subsec:construction}.

In the second phase  (i.e., transaction processing phase), 
the \emph{TxnManager} is first involved again to refine (Line 6) the constructed \tpg with further dependency resolution.
The \emph{TxnScheduler} 
schedules operations for concurrent execution based on the constructed \tpg according to the three dimensions of scheduling decisions (Line 7). 
In particular, a scheduling decision model $M$ is instantiated based on the constructed \tpg (Line 14).
\textbf{\circled{1}} Guided by $M$, execution threads adopt an exploration strategy (Section~\ref{subsec:explore}) to explore the constructed \tpg for operations available to be scheduled constrained by dependencies. 
\textbf{\circled{2}} 
During exploration, one or multiple operations may be treated as the 
% basic 
unit of scheduling (Section~\ref{subsec:granularity}). 
Subsequently, \textbf{\circled{3}} every thread executes operation(s) in the unit of scheduling with various abort handling mechanisms (Section~\ref{subsec:abort_handling}).
Only when state transactions are processed (i.e., committed or aborted) can the associated input events be postprocessed (Line 8) by the \emph{StreamManager} based on transaction processing results.
\end{comment}

\begin{comment}
\begin{algorithm}
\footnotesize
    \KwData{$e$ \tcp{Input event}}
    \KwData{$txn_{ts}$ \tcp{State transaction}}
    \KwData{$G$ \tcp{The currently constructed TPG}}
    \While{!finish processing of input streams}{
        \eIf(\tcp*[h]{Phase 1}){\text{$e$ is not a $punctuation$}}{
                $txn_{ts}$ $\gets$ PRE\_Processing($e$)\;
                \textbf{TPG\_Construction}($G$, $txn_{ts}$)\; 
          }(\tcp*[h]{Phase 2}){
                \textbf{TPG\_Refinement}($G$)\; 
                \textbf{TXN\_Scheduling}($G$)\; 
                POST\_Processing()\;
          }
    }
    
    \SetKwFunction{FMain}{TPG\_Construction}
    \SetKwProg{Fn}{Function}{:}{}
    \Fn{\FMain{$G$, $txn_{ts}$}}{
        $O_{1..k}$ $\gets$ \textbf{Partition} $txn_{ts}$\;
        \ForEach{\text{operation $O_{i}$ $\in$ $O_{1..k}$}}{
            \textbf{Identify} its \ld\;
            $G$ $\gets$ $G$ + $O_{i}$ \;
        }
    }
    \SetKwFunction{FMain}{TPG\_Refinement}
    \SetKwProg{Fn}{Function}{:}{}
    \Fn{\FMain{$G$}}{
        \ForEach{\text{vertex $e_{i}$ $\in$ $G$}}{
            \textbf{Identify} its \td, \pd\;
        }
    }
    
    \SetKwFunction{FMain}{TXN\_Scheduling}
    \SetKwProg{Fn}{Function}{:}{}
    \Fn{\FMain{$G$}}{
        $M$ $\gets$ Instantiated with $G$;\tcp{A decision model}
        \While{!finish scheduling of $G$
        }{
          \textbf{\circled{2}} $Scheduling Unit$ $\gets$ \textbf{\circled{1}} \emph{Explore}($G$, $M$)\; 
            \textbf{\circled{3}} \emph{Execute with Abort Handling} ($Scheduling Unit$)\; 
        }
    }
  \caption{Execution Outline of \system}
  \label{alg:algo}
\end{algorithm}
\end{comment}

\end{refsection}
\end{document}
