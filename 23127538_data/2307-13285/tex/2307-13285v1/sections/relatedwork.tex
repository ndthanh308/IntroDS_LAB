\section{Related work\label{relatedwork}}
We briefly discuss existing works related to \sys in terms of request cloning, server-level solutions, and in-network computing solutions.

\textbf{Request cloning.}
Vulimiri \etal~\cite{vulimiri13} investigates the tradeoff of client-based request cloning.
They identify the threshold load and the client-side overhead.
Gardner \etal~\cite{powerofd,redundancy} provide rigorous theoretical analysis for cloning.
Dolly~\cite{dolly} and RepFlow~\cite{xu14a} utilize the cloning technique for mitigating stragglers in MapReduce clusters and multi-path routing in data center networks, respectively.
\sota~\cite{laedge} performs dynamic cloning using the coordinator but lacks low latency overhead and scalability.
\sys is the first dynamic request cloning system for microsecond-scale RPCs.

\textbf{Server-level solutions for microsecond-scale RPCs.}
There are line of works that reduce the latency of RPCs at the server level in hardware and software.
ALTOCUMULUS~\cite{alto} avoids the scheduling overhead using direct register-level messaging.
RPCValet~\cite{rpcvalet} bypasses slow PCIe buses using shared caches when dispatching RPCs to CPU cores.
nanoPU~\cite{nanopu} bypasses the cache and memory hierarchy to provide a fast path from NIC to applications using a hardware accelerator.
eRPC~\cite{erpc} improves the performance of small messages by optimizing common cases with various software techniques.
IX~\cite{ix}, ZygOS~\cite{zygos}, and Shinjuku~\cite{shinjuku} are data plane OSes that provide efficient CPU scheduling for microsecond-scale RPCs.
For example, Shinjuku~\cite{shinjuku} implements a preemptive scheduling algorithm by re-queueing long-lasting RPCs if the runtime exceeds a given threshold.
The above works address the RPC latency at the server level, whereas \sys tries to optimize RPC latency at the cluster level.
Since \sys does not restrict the server-side mechanism to a specific solution, \sys is orthogonal to the existing works.

\textbf{In-network computing for microsecond-scale RPCs.}
The capability and flexibility of programmable switch ASICs trigger the emergence of in-network computing.
%In-network computing is an emerging paradigm that utilizes the network switch to accelerate applications.
NetCache~\cite{jin17}, Pegasus~\cite{pegasus}, DistCache~\cite{liu19}, Harmonia~\cite{harmonia}, NetLR~\cite{netlr}, P4DB~\cite{p4db}, and Transaction Triaging~\cite{triaging} are solutions to accelerate distributed storage.
\sys can improve the latency of GET queries and is a more generic solution.
RackSched~\cite{racksched} is an in-network request scheduler for microsecond-scale RPCs that performs the JSQ load balancing.
\sys is orthogonal to RackSched since \sys does not specify its load balancing algorithm.


