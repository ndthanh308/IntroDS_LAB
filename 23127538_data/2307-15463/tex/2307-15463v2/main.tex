\pdfoutput=1
\newif\iffull \fullfalse

\documentclass[acmsmall, screen, authorversion]{acmart}

%%% The following is specific to POPL '24 and the paper
%%% 'Answer Refinement Modification: Refinement Type System for Algebraic Effects and Handlers'
%%% by Fuga Kawamata, Hiroshi Unno, Taro Sekiyama, and Tachio Terauchi.
%%%
\setcopyright{rightsretained}
\acmDOI{10.1145/3633280}
\acmYear{2024}
\copyrightyear{2024}
\acmSubmissionID{popl24main-p20-p}
\acmJournal{PACMPL}
\acmVolume{8}
\acmNumber{POPL}
\acmArticle{5}
\acmMonth{1}
\received{2023-07-11}
\received[accepted]{2023-11-07}

\citestyle{acmauthoryear}
% \citestyle{acmnumeric}

\usepackage{proof}
\usepackage{mathtools}
\mathtoolsset{showonlyrefs=true}

\usepackage{multirow}
\usepackage{wrapfig}

\usepackage{bussproofs}
\usepackage{stackengine}

\usepackage{tikz}
\usetikzlibrary{automata,positioning,arrows.meta}

\usepackage{thelanguage}
% \newcommand*{\defeq}{\stackrel{\text{def}}{=}}
\newcommand*{\defeq}{\triangleq}
\newcommand{\rulename}[1]{(\textsc{#1})}
\newcommand{\infersc}[3][]{\infer[\!\!\text{\rulename{#1}}]{#2}{#3}}


\begin{document}

\title{Answer Refinement Modification: Refinement Type System for Algebraic Effects and Handlers}

\author{Fuga Kawamata}
\orcid{0009-0003-4147-9572}
\affiliation{%
  \institution{Waseda University}
  \city{Tokyo}
  \country{Japan}
}
\email{maple-river@fuji.waseda.jp}

\author{Hiroshi Unno}
\orcid{0000-0002-4225-8195}
\affiliation{%
  \institution{University of Tsukuba}
  \city{Tsukuba}
  \country{Japan}
}
\email{uhiro@cs.tsukuba.ac.jp}

\author{Taro Sekiyama}
\orcid{0000-0001-9286-230X}
\affiliation{%
  \institution{National Institute of Informatics}
  \city{Tokyo}
  \country{Japan}
}
\email{ryukilon@gmail.com}

\author{Tachio Terauchi}
\orcid{0000-0001-5305-4916}
\affiliation{%
  \institution{Waseda University}
  \city{Tokyo}
  \country{Japan}
}
\email{terauchi@waseda.jp}

% \renewcommand{\shortauthors}{Kawamata et al.}

\begin{abstract}
Algebraic effects and handlers are a mechanism to structure
programs with computational effects in a modular way. They
are recently gaining popularity and being adopted in practical languages,
such as OCaml.
%
Meanwhile, there has been substantial progress in program verification via {\em
refinement type systems}.  While a variety of
refinement type systems have been proposed, thus far there has not been a
satisfactory refinement type system for
algebraic effects and handlers.  In this paper, we fill the void by proposing a
novel refinement type system for languages with algebraic effects and handlers.
%
The expressivity and usefulness of algebraic effects and handlers come
from their ability to manipulate \emph{delimited continuations}, but delimited continuations also complicate programs'
control flow and make their verification harder.
%
To address the complexity,
we introduce a novel concept that we call {\em answer refinement modification} (ARM for
short), which allows the refinement type system to precisely track what effects occur and in what order when a program is executed, and reflect such information as modifications to the refinements in the types of delimited continuations.
%
We formalize our type system that supports ARM (as well as answer \emph{type} modification, or ATM) and prove its soundness. Additionally, as a proof of concept, we have
extended the refinement type system to a subset of OCaml 5 which comes with a built-in support for effect handlers,
implemented a type checking and inference algorithm for the extension,
and evaluated it on a number of benchmark programs that use algebraic effects and handlers.
The evaluation demonstrates that ARM is conceptually simple and practically useful.

Finally, a natural alternative to directly reasoning about a program with delimited continuations is to apply a {\em continuation passing style} (CPS) transformation that transforms the program to a pure program without delimited continuations.  We investigate this alternative in the paper, and show that the approach is indeed possible by proposing a novel CPS transformation for algebraic effects and handlers that enjoys bidirectional (refinement-)type-preservation.  We show that there are pros and cons with this approach, namely, while one can use an existing refinement type checking and inference algorithm that can only (directly) handle pure programs, there are issues such as
needing type annotations in source programs and
making the inferred types less informative to a user.
\end{abstract}

%% > The code below is generated by the tool at http://dl.acm.org/ccs.cfm.
%% > Please copy and paste the code instead of the example below.
\begin{CCSXML}
<ccs2012>
   <concept>
       <concept_id>10003752.10003790.10011740</concept_id>
       <concept_desc>Theory of computation~Type theory</concept_desc>
       <concept_significance>500</concept_significance>
       </concept>
   <concept>
       <concept_id>10011007.10011006.10011008.10011009.10011012</concept_id>
       <concept_desc>Software and its engineering~Functional languages</concept_desc>
       <concept_significance>500</concept_significance>
       </concept>
   <concept>
       <concept_id>10011007.10011006.10011008.10011024.10011027</concept_id>
       <concept_desc>Software and its engineering~Control structures</concept_desc>
       <concept_significance>500</concept_significance>
       </concept>
   <concept>
       <concept_id>10003752.10010124.10010138.10010142</concept_id>
       <concept_desc>Theory of computation~Program verification</concept_desc>
       <concept_significance>500</concept_significance>
       </concept>
   <concept>
       <concept_id>10003752.10010124.10010125.10010126</concept_id>
       <concept_desc>Theory of computation~Control primitives</concept_desc>
       <concept_significance>500</concept_significance>
       </concept>
 </ccs2012>
\end{CCSXML}

\ccsdesc[500]{Theory of computation~Type theory}
\ccsdesc[500]{Software and its engineering~Functional languages}
\ccsdesc[500]{Software and its engineering~Control structures}
\ccsdesc[500]{Theory of computation~Program verification}
\ccsdesc[500]{Theory of computation~Control primitives}

%% > Keywords. The author(s) should pick words that accurately describe
%% > the work being presented. Separate the keywords with commas.
\keywords{algebraic effects and handlers, type-and-effect system, refinement type system, answer type modification, answer refinement modification, CPS transformation}


\maketitle

% Figure environment removed

\section{Introduction}
Automatic 3D reconstruction of clothed humans using image inputs has gained increasing significance due to its potential applications in a wide array of AR/VR scenarios. High-fidelity reconstructions typically depend on sophisticated capture systems, which are developed with dense camera arrays~\cite{collet2015high,joo2015panoptic,joo2018total}, programmable light-stages~\cite{Vlasic2009, guo2019relightables}, and depth sensors~\cite{newcombe2011kinectfusion,DoubleFusion,BodyFusion,dou2016fusion4d,newcombe2015dynamicfusion}. However, stringent capture environments equipped with complex hardware pose significant challenges for consumer-level applications.


In this context, considerable research effort has been dedicated to developing methods that allow for more flexible capture configurations, such as utilizing a few RGB inputs. Among these works, learning implicit functions \cite{iccv2020PIFu, saito2020pifuhd, hong2021stereopifu} has proven effective in achieving highly detailed reconstructions by integrating the advancements of deep neural networks. These methods employ large multi-layer perceptrons (MLPs) to predict the occupancy probability or truncated signed distance function (TSDF) value of every queried 3D point based on its associated local feature, which is extracted from images. They can recover a continuous surface at arbitrary resolutions without topology restrictions.


However, in typical MLP-based implicit networks, the occupancy or TSDF value at each location is solved independently with planar image features, rendering them less capable of addressing challenging cases such as occlusions. Consequently, these methods suffer from generalization and robustness issues, particularly when tackling strong occlusions caused by large motion or multiple interacting humans. 
Some follow-up studies  \cite{zheng2021deepmulticap,zheng2021pamir,huang2020arch} utilize an extra geometric model, SMPL~\cite{Loper2015}, to improve robustness by introducing strong shape priors. 
Their success typically relies on the assumption of geometrical similarity \cite{huang2020arch} between the shape prior and target reconstruction, making them intractable for handling complex cases with loose clothes and sensitive to errors in SMPL model fitting.



%\ping{this paragraph sounds like `TSDF is better than MLP/SMPL, and we use TSDF to solve the problem'. But in Sec 3, we are telling a different story, saying `MLP needs a 3D convolutional encoder'. We need to make these two sections consistent.}\sicong{I think in this paragraph we claim that the TSDF}


%We opt for Trucated Signed Distance Funtion (TSDF) volumetric representations as they are naturally suitable for convolution operations, which have shown remarkable performance for learning hierarchical features on 2D visual perception tasks \cite{SunXLW19}. 
%Meanwhile, TSDF also describes the gradual geometry change around shape surface, which is not reflected by occupancy volume. 

We instead revisit the 3D volumetric representation and resort to 3D convolutional neural networks (CNNs) for feature learning, due to their impressive performance in feature learning and the ability to incorporate spatial context. However, volumetric methods and 3D convolution involve discretization, which might raise concerns regarding whether a discretized volume can preserve subtle geometric details as continuous representations learned in implicit functions. We investigate the relationship between volume resolution and quantization error on synthetic data by converting target mesh objects to TSDF volumes, as shown in Figure~\ref{fig:quantization_error}. We observe that the quantization errors are significantly reduced by increasing volume resolution and become nearly negligible when reaching a relatively high resolution (e.g., 512 or higher). In other words, achieving fine-detailed reconstruction is not supposed to be restricted by the use of volume representations as long as a proper volume resolution is utilized. Therefore, we present a method with high-resolution feature volumes, e.g., 256 and 512, while traditional volumetric methods \cite{varol18_bodynet,gilbert2018volumetric} are often limited to much lower resolutions, such as 32 or 128.



On the other hand, an increase in volume resolution may lead to a cubic growth of memory overhead \cite{8100085}. Reducing memory costs while guaranteeing the granularity of volumetric representations is necessary for pursuing high-quality reconstruction. Thus, we adopt a coarse-to-fine approach and cull away irrelevant voxels to build a sparse high-resolution feature volume. At the coarse level, the network computes an initial TSDF by applying a U-Net with sparse 3D CNN \cite{3DSemanticSegmentationWithSubmanifoldSparseConvNet} on the sparse feature volume, which is carved by a visual hull. Through our experiments, it turns out that more than 95\% of the volume grids are discarded by the visual hull culling, making the sparse 3D CNN efficient. At the fine level, the network focuses on a narrow band near the zero-level set of the initial TSDF and discretizes the narrow band with smaller voxels. By employing this narrow-band culling, we further shrink the sampling space, resulting in a relatively small range of grid numbers (usually 300K--500K in our experiments) even with a high volume resolution of 512. The remaining voxels in the narrow band are associated with features that fuse high-frequency information from the computed normal maps upon the low-frequency shape from the coarse level to compute the TSDF at high resolution. The final mesh is then extracted from the TSDF using the Marching-Cube algorithm ~\cite{Lorensen87marchingcubes}.
% Different from the u-net sturcture to preserve global topology context, we then apply a shallow 3dcnn to compute the final TSDF $D_{final}$ which contain more local geometry detail.




% \ping{this paragraph can be expanded. It is an important contribution and often ignored by other works. stress on the novel idea of regressing blending weights instead of colors}

In addition to geometry, high-quality mesh texture is also a crucial factor contributing to visual appearance. Directly computing a color field in 3D space, as in \cite{iccv2020PIFu}, struggles to capture high-frequency texture details, while the neural radiance field (NeRF) \cite{yu2020pixelnerf} or the DoubleField~\cite{shao2022doublefield} require expensive per-instance optimization and are often unstable for sparse input images. In contrast, we adopt an image-based rendering approach to compute a texture atlas map, which is efficient and widely supported in existing computer graphics tools. 
Specifically, we compute a blending weight at each 3D point on the mesh surface to determine its color as a weighted average of the colors at its image projections. The blending weights can be computed at a relatively coarse resolution, e.g., 512 volume resolution in our case, and leave texture details to the high-resolution images, such as 1K or 2K. Unlike previous methods that generate blurry texturing results under sparse input, our method generalizes well on both synthetic and real data with just a few input views. 
Figure~\ref{fig:teaser} shows two examples reconstructed by our method. Despite the challenging garment, pose, and occlusion, our method recovers faithful shape, normal, and texture on the right.

%with a wide variety of poses and clothing styles, and it is also adaptive to handle input image with arbitrary resolutions.
%\sicong{For this concern we claim that when the resolution of dicretized volume meets certain threshold (which is 256 in our experiment), the quantization error can be neglected.} 



In summary, the main contributions of this paper are as follows:
\begin{itemize}
\vspace{-0.1in}
  \item 
  We revisit the 3D volumetric representation and demonstrate that it can support clothed human reconstruction with equal or even better performance compared to implicit representation. 
  \item 
  We develop a memory and computation-efficient method for high-resolution volumetric reconstruction using sophisticated sparse 3D CNN, coarse-to-fine estimation, and voxel culling by visual hull and narrow bands. 
  \item 
  We introduce a novel method to compute a texture atlas map, which captures rich appearance details from high-resolution input images.
  \item 
  We achieve impressive results on standard benchmark datasets Twindom and MultiHuman, significantly reducing the point-2-surface (P2S) precision to approximately 0.2cm from just six input views, with more than $50\%$ error reduction compared to the state-of-the-art methods, including DoubleField~\cite{shao2022doublefield} and PIFuHD~\cite{saito2020pifuhd}.
\end{itemize}

\section{Secure Design of \puma}\label{sec:design}
In this section, we first present an overview of \puma, and present the protocols for secure $\gelu$ , $\softmax$, embedding, and $\layernorm$ used by \puma. Note that the linear layers such as matrix multiplication are straightforward in replicated secret sharing, so we mainly describe our protocols for non-linear layers in this manuscript.

\subsection{Overview of \puma}\label{sec:overview}
To achieve secure inference of Transformer models, \puma\ defines three kinds of roles: one model owner, one client, and three computing parties. The model owner and the client  provide their models or inputs to the computing parties (i.e., $P_0$, $P_1$, and $P_2$) in a secret-shared form, then the computing parties execute the MPC protocols and send the results back to the client. Note that the model owner and client can also act as one of the computing party, we describe them separately for generality. \eg, when the model owner acts as $P_0$, the client acts as  $P_1$, a third-party dealer acts as $P_2$, the system model becomes the same with \mpcformer~\citep{li2023mpcformer}.

During the secure inference process, a key invariant is maintained: For any layer, the computing parties always start with 2-out-of-3 replicated secret shares of the previous layer's output and the model weights, and end with 2-out-of-3 replicated secret shares of this layer's output. As the shares do not leak any information to each party, this ensures that the layers can be sequentially combined for arbitrary depths to obtain a secure computation scheme for any Transformer-based model.
%The main focus of \puma\ is to reduce the computation and communication costs between the computing parties while maintaining the desired level of security. 



\iffalse
\textbf{Threat Model.}
Following previous works~\citep{aby3,li2023mpcformer},
\puma\ resists a semi-honest (a.k.a., honest-but-curious) adversary in honest-majority~\citep{lindell2009proof}, where the adversary passively corrupts no more than one computing party. Such an adversary follows the protocol specification exactly, but may try to learn more information than permitted. Please note that \puma\ cannot protect against the extraction of information from the inference results, and the examination of mitigating solutions (\eg, differential privacy~\citep{abadi2016deep}) falls outside the scope of this study.
\fi 

\subsection{Protocol for Secure GeLU}\label{sec:gelu}
Most of the current approaches view the $\gelu$ function as a composition of smaller functions and try to optimize each piece of them, making them to miss the
chance of optimizing the private $\gelu$ as a whole. Given the $\gelu$ function:
\begin{equation}\label{eq:gelu}
\begin{split}
    \gelu(x) &= \frac{x}{2} \cdot \left(1 + \tanh \left( \sqrt{\frac{2}{\pi}} \cdot \left(x + 0.044715 \cdot x^3 \right) \right) \right)\\
    &\approx x\cdot \mathsf{sigmoid}(0.071355\cdot x^3 + 1.595769\cdot x) 
\end{split},
\end{equation}
these approaches~\citep{hao2022iron,characmpctranformer} focus either on designing efficient protocols for function $\tanh$
or using the existing MPC protocols of exponentiation and reciprocal for $\mathsf{sigmoid}$. 

However, none of current approaches have utilized the fact that $\gelu$ function is almost linear on the two sides (\ie, $\gelu(x)\approx 0$ for $x<-4$ and $\gelu(x)\approx x$ for $x>3$). 
Within the short interval $[-4,3]$ of $\gelu$,
we suggest a piece-wise approximation of low-degree polynomials is a more efficient and easy-to-implement choice for its secure protocol. Concretely, our piece-wise low-degree polynomials are shown as equation~(\ref{eq:geluapprox}):
\begin{equation}\label{eq:geluapprox}
\gelu(x)=
\begin{cases}
0, & x<-4 \\
F_0(x), & -4 \le x < -1.95 \\
F_1(x), & -1.95 \le x \le 3 \\
x, & x >3
\end{cases},
\end{equation}
where polynomials $F_0()$ and $F_1()$ are computed by library $\mathsf{numpy.ployfit}$\footnote{\url{https://numpy.org/doc/stable/reference/generated/numpy.polyfit.html}} as equation~(\ref{eq:f0f1}). Surprsingly, the above simple poly fit works very well and our $\mathsf{max\ error}< 0.01403$, $\mathsf{median\ error}< 4.41e-05$, and $\mathsf{mean\ error}< 0.00168$.
\begin{equation}\label{eq:f0f1}
\begin{cases}
F_0(x) &= -0.011034134030615728 x^3 -0.11807612951181953 x^2 \\
&- 0.42226581151983866 x -0.5054031199708174\\
F_1(x) &= 0.0018067462606141187x^6 -0.037688200365904236 x^4 \\
&+ 0.3603292692789629x^2 + 0.5x + 0.008526321541038084
\end{cases}
\end{equation}

Formally, given secret input $\share{x}$, our secure $\gelu$ protocol $\Pi_{\gelu}$ is constructed as algorithm~\ref{protocol:gelu}. 
\iffalse
\begin{itemize}
    \item The parties jointly compute
$\share{b_0}^2 = \Pi_{\mathsf{LT}}(\share{x}, 4)$,
$\share{b_1}^2 = \Pi_{\mathsf{LT}}(\share{x}, -1.95)$, and
$\share{b_2}^2 = \Pi_{\mathsf{LT}}(3, \share{x})$.

\item  Then, each $P_i$ locally compute
$\share{b_3}^2 = \share{b_1}^2 \oplus \share{b_2}^ \oplus 1$ and
$\share{b_4}^2 = \share{b_0}^2 \oplus \share{b_1}^2$

\item Finally, the parties compute and return 
$\share{b_2}^2 \cdot \share{x} + \share{b_4}^2 \cdot F_0(\share{x}) + \share{b_3}^2 \cdot F_1(\share{x})$, where polynomials $(F_0, F_1)$ can be computed easily using secure addition and multiplication (and its variants, \eg, secure square)~\citep{spu}. 
\end{itemize}
\fi 

\begin{algorithm}[tp]
\caption{Secure $\gelu$ Protocol $\Pi_{\mathsf{GeLU}}$}\label{protocol:gelu}
\begin{algorithmic}[1]
\REQUIRE
$P_i$ holds the 2-out-of-3 replicate secret share $\share{x}_i$ for $i\in \{0,1,2\}$ 
\ENSURE
$P_i$ gets the 2-out-of-3 replicate secret share $\share{y}_i$ for $i\in \{0,1,2\}$, where $y=\gelu(x)$.

\STATE $P_0$, $P_1$, and $P_2$ jointly compute
\begin{equation*}
\begin{split}
&\shareb{b_0} = \Pi_{\mathsf{LT}}(\share{x}, -4),~~~\vartriangleright b_0 = 1\{x<-4\}\\
&\shareb{b_1} = \Pi_{\mathsf{LT}}(\share{x}, -1.95),~~~\vartriangleright b_1 = 1\{x<-1.95\} \\
&\shareb{b_2} = \Pi_{\mathsf{LT}}(3, \share{x}),~~~~~~\vartriangleright b_2 = 1\{3<x\}
\end{split}
\end{equation*}
and compute 
$\shareb{z_0} = \shareb{b_0} \oplus \shareb{b_1}$,
$\shareb{z_1} = \shareb{b_1} \oplus \shareb{b_2} \oplus 1$, and $\shareb{z_2}=\shareb{b_2}$. Note that $z_0 = 1\{-4\le x < -1.95\}$, $z_1 = 1\{-1.95\le x\le 3\}$, and $z_2 = 1\{x>3\}$.

\STATE Jointly compute $\share{x^2} = \Pi_{\mathsf{Square}}(\share{x})$, $\share{x^3} = \Pi_{\mathsf{Mul}}(\share{x}, \share{x^2})$, $\share{x^4} = \Pi_{\mathsf{Square}}(\share{x^2})$, and $\share{x^6} = \Pi_{\mathsf{Square}}(\share{x^3})$.

\STATE Computing polynomials $\share{F_0(x)}$ and $\share{F_1(x)}$ based on $\{\share{x}, \share{x^2}, \share{x^3}, \share{x^4}, \share{x^6}\}$ as equation~(\ref{eq:geluapprox}) securely.


\RETURN$\share{y} = \Pi_{\mathsf{Mul_{BA}}}(\shareb{z_0}, \share{F_0(x)}) + \Pi_{\mathsf{Mul_{BA}}}(\shareb{z_1}, \share{F_1(x)})+\Pi_{\mathsf{Mul_{BA}}}(\shareb{z_2}, \share{x})$.

\end{algorithmic}
\end{algorithm}



\subsection{Protocol for Secure Softmax}\label{sec:secureatten}

In the function $\attention(\Q,\K,\V)=
\softmax(\Q \cdot \K^\mathsf{T} + \M) \cdot \V$, where $\M$ can be viewed as a bias matrix, the key challenge is computing function $\softmax$. For the sake of numerical stability, the $\softmax$ function is computed as
\begin{equation}\label{eq:softmax}
    \softmax(\x)[i]=\frac{\exp(\x[i] - \bar{x} - \epsilon)}{\sum_i \exp(\x[i] - \bar{x} - \epsilon)},
\end{equation}
where $\bar{x}$ is the maximum element of the input vector $\x$. 
For the normal plaintext softmax, $\epsilon=0$. For a two-dimension matrix, we apply equation~(\ref{eq:softmax}) to each of its row vector.

Formally, our detailed secure protocol  $\Pi_{\softmax}$ is illustrated in algorithm~\ref{protocol:softmax}, where we propose two optimizations:
\begin{itemize}
\item 
For the first optimization, we set $\epsilon$ in equation~\ref{eq:softmax} to a tiny and positive
value, e.g., $\epsilon =
10^{-6}$, so that the inputs to exponentiation
in equation~\ref{eq:softmax} are all negative. We exploit the negative operands
for acceleration. Particularly, we compute the exponentiation using the Taylor series~\citep{tan2021cryptgpu} with a simple clipping
\begin{equation}\label{eq:negexp}
\mathsf{negExp}(x) = \begin{cases}
    0, &x < T_{\exp} \\
    (1+\frac{x}{2^t})^{2^t}, &x\in [T_{\exp},0].
\end{cases}
\end{equation}
Indeed, we apply the less-than for the branch $x < T_{\exp}$
The division by $2^t$ can be achieved using
$\Pi_{\mathsf{Trunc}}^t$ since the input is already negative. Also, we can
compute the power-of-$2^t$ using $t$-step sequences of square function $\Pi_{\mathsf{square}}$ and $\Pi_{\mathsf{Trunc}}^f$. Suppose our MPC program uses
$18$-bit fixed-point precision. Then we set $T_{\exp}=-14$ given $\exp(-14) < 2^{-18}$, and empirically set $t = 5$.


\item 
Our second optimization is to reduce the number of divisions, which ultimately saves computation and communication costs.
To achieve this, for a vector $\x$ of size $n$, we have replaced the operation $\mathsf{Div}(\x, \mathsf{Broadcast}(y))$ with $\x \cdot  \mathsf{Broadcast}(\frac{1}{y})$, where $y=\sum_{i=1}^n\x[i]$. By making this replacement, we effectively reduce $n$ divisions to just one reciprocal operation and $n$ multiplications.
This optimization is particularly beneficial in the case of the $\softmax$ operation. The $\frac{1}{y}$ in the $\softmax$ operation is still large enough to maintain sufficient accuracy under fixed-point values. As a result, this optimization can significantly reduce the computational and communication costs while still providing accurate results.
\end{itemize}

\begin{algorithm}[tp]
\caption{Secure $\softmax$ Protocol $\Pi_{\softmax}$}\label{protocol:softmax}
\begin{algorithmic}[1]
\REQUIRE
$P_i$ holds the 2-out-of-3 replicate secret share $\share{\x}_i$ for $i\in \{0,1,2\}$, and $\x$ is a vector of size $n$. 
\ENSURE
$P_i$ gets the 2-out-of-3 replicate secret share $\share{\y}_i$ for $i\in \{0,1,2\}$, where $\y=\softmax(\x)$.

\STATE $P_0$, $P_1$, and $P_2$ jointly compute
$\shareb{\mathbf{b}} = \Pi_{\mathsf{LT}}(T_{\exp}, \share{\x})$ and the maximum $\share{\bar{x}} = \Pi_{\mathsf{Max}}(\share{\x})$.

\STATE Parties locally computes $\share{\hat{\x}} = \share{\x} - \share{\bar{x}} - \epsilon$, and jointly compute $\share{\z_0} = 1+  \Pi_{\mathsf{Trunc}}^t(\share{\hat{\x}})$.

\FOR{$j=1,2,\dots, t$}
\STATE $\share{\z_j} = \Pi_{\mathsf{Square}}(\share{\z_{j-1}})$.
\ENDFOR

\STATE Parties locally compute $\share{z} = \sum_{i=1}^n \share{\z[i]}$ and jointly compute $\share{1/z} = \Pi_{\mathsf{Recip}}(\share{z})$.

\STATE Parties jointly compute $\share{\z / z} = \Pi_{\mathsf{Mul}}(\share{\z}, \share{1/z})$

\RETURN $\share{\y} = \Pi_{\mathsf{Mul}_{\mathsf{BA}}}( \shareb{\mathbf{b}}, \share{\z / z})$.

\end{algorithmic}
\end{algorithm}

\subsection{Protocol for Secure Embedding}\label{sec:embed}


The current secure embedding procedure described in~\citep{li2023mpcformer} necessitates the client to  generate a one-hot vector using the token $\tokenid$ locally. This deviates from a plaintext Transformer workflow where the one-hot vector is generated inside the model. As a result, they have to carefully strip off the one-hot step from the pre-trained models, and add the step to the client side, which could be an obstacle for deployment. 



To address this issue, we propose a secure embedding design as follows. Assuming that the token $\tokenid\in [n]$ and all embedding vectors are denoted by $\E= (\e_1^T, \e_2^T, \dots, \e_n^T)$, the embedding can be formulated as $\e_{\tokenid} = \mathbf{E}[\tokenid]$. Given $(\tokenid, \E)$ are in secret-shared fashion, our secure embedding protocol $\Pi_{\mathsf{Embed}}$ works as follows:
\begin{itemize}
    \item The computing parties securely compute the one-hot vector $\shareb{\mathbf{o}}$ after receiving $\share{\tokenid}$ from the client. Specifically, $\shareb{\mathbf{o}[i]}=\Pi_{\mathsf{Eq}}(i,\share{\tokenid})$ for $i\in [n]$.
    \item The parties can compute the embedded vector via $\share{\e_{\tokenid}} = \Pi_{\mathsf{Mul_{BA}}}(\share{\E}, \shareb{\mathbf{o}})$, where  does not require secure truncation.
\end{itemize}
In this way, our $\Pi_{\mathsf{Embed}}$ does not require explicit modification of the workflow of plaintext Transformer models, at the cost of more $\Pi_{\mathsf{Eq}}$ and $\Pi_{\mathsf{Mul_{BA}}}$ operations. 



\subsection{Protocol for Secure LayerNorm}\label{sec:seclayernorm}
Recall that given a vector $\x$ of size $n$, $\layernorm(\x)[i] =  \gamma \cdot \frac{\x[i]-\mu}{\sqrt{\sigma}} + \beta$, where $(\gamma, \beta)$ are trained parameters, $\mu = \frac{\sum_{i=1}^n \x[i]}{n}$, and $\sigma = \sum_{i=1}^n (\x[i] - \mu)^2$. In MPC, the key challenge is the evaluation of the divide-square-root $\frac{\x[i]-\mu}{\sqrt{\sigma}}$ formula. To securely evaluate this formula, CrypTen sequentially executes the MPC protocols of square-root, reciprocal, and multiplication. However, we observe that $\frac{\x[i]-\mu}{\sqrt{\sigma}}$ is equal to $(\x[i]-\mu)\cdot \sigma^{-1/2}$. And in the MPC side, the costs of computing the inverse-square-root $\sigma^{-1/2}$ is similar to that of the square-root operation~\citep{rSqrt}. Besides, inspired by the second optimization of \S~\ref{sec:secureatten}, we can first compute $\sigma^{-1/2}$ and then $\mathsf{Broadcast}(\sigma^{-1/2})$ to support fast and secure $\layernorm(\x)$. And our formal protocol $\Pi_{\layernorm}$ is shown in algorithm~\ref{protocol:layernorm}.

\begin{algorithm}[tp]
\caption{Secure $\mathsf{LayerNorm}$ Protocol $\Pi_{\mathsf{LayerNorm}}$}\label{protocol:layernorm}
\begin{algorithmic}[1]
\REQUIRE
$P_i$ holds the 2-out-of-3 replicate secret share $\share{\x}_i$ for $i\in \{0,1,2\}$, and $\x$ is a vector of size $n$. 
\ENSURE
$P_i$ gets the 2-out-of-3 replicate secret share $\share{\y}_i$ for $i\in \{0,1,2\}$, where $\y=\mathsf{LayerNorm}(\x)$.

\STATE $P_0$, $P_1$, and $P_2$ compute $\share{\mu} = \frac{1}{n}\cdot \sum_{i=1}^n\share{\x[i]}$ and $\share{\sigma} = \sum_{i=1}^n \Pi_{\mathsf{Square}}(\share{\x} - \share{\mu})[i]$.

\STATE Parties jointly compute $\share{\sigma^{-1/2}} = \Pi_{\mathsf{rSqrt}}(\share{\sigma})$.

\STATE Parties jointly compute $\share{\mathbf{c}} = \Pi_{\mathsf{Mul}}((\share{\x} - \share{\mu}), \share{\sigma^{-1/2}})$

\RETURN $\share{\y} = \Pi_{\mathsf{Mul}}(\share{\gamma}, \share{\mathbf{c}}) + \share{\beta}$.

\end{algorithmic}
\end{algorithm}

\section{Language} \label{sec:language}

This section presents our language with algebraic effects and handlers.
The semantics is formalized using evaluation contexts like in \citet{Leijen17},
and the type system is a novel refinement type system with ARM (and ATM).

\subsection{Syntax and Semantics} \label{sec:language/syntax-semantics}

\begin{wrapfigure}[10]{R}{0.5\textwidth}
    $\begin{array}{rcl}
        p &::=& \exptrue \mid \expfalse \mid \ldots \\
        v &::=& x \mid p \mid \exprec{f}{x}{c} \\
        c &::=& \expret{v} \mid \expop{v}{y}{c} \mid v_1~v_2 \\
            &&\mid \expif{v}{c_1}{c_2} \mid \explet{x}{c_1}{c_2} \\
            &&\mid \expwith{h}{c} \\
        h &::=& \{ \expret{x_r} \mapsto c_r, \repi{\op_i(x_i, k_i) \mapsto c_i} \} \qquad
    \end{array}$
    \caption{Program syntax.}
    \label{fig:syntax}
\end{wrapfigure}
%
Figure~\ref{fig:syntax} shows the syntax of our language.
It indicates that expressions are split into values, ranged over by $v$, and computations, ranged over by $c$,
as in the fine-grained call-by-value style of \citet{Levy03}.
Values, which are effect-free expressions in a canonical form,
consist of variables $x$, primitive values $p$,
and (recursive) functions $\exprec{f}{x}{c}$
where variable $f$ denotes the function itself for recursive calls in the body $c$.
If $f$ does not occur freely in $c$, we simply write $\lambda x. c$.
Computations, which are possibly effectful expressions,
consist of six kinds of constructs.
A value-return $\expret{v}$ lifts a value $v$ to a computation.
An operation call $\expop{v}{y}{c}$ performs the operation $\op$ with the
argument $v$.  Once the operation call returns a value, the continuation $c$
will be executed by binding variable $y$ to the return value.  Note that $y$ is
bound in $c$.  The notation $\op~v$ used in Section~\ref{sec:overview/algeff} is
an abbreviation of the expression $\expop{v}{y}{\expret{y}}$ in this language.
A function application $v_1~v_2$, conditional branch $\expif{v}{c_1}{c_2}$,
and let-expression $\explet{x}{c_1}{c_2}$ are standard.
Note that functions, arguments, and conditional expressions are
restricted to values, but this does not reduce expressivity because, e.g.,
a conditional branch $\expif{c}{c_1}{c_2}$ can be expressed as
$\explet{x}{c}{\expif{x}{c_1}{c_2}}$ using a fresh variable $x$.
A handling construct $\expwith{h}{c}$ handles operations performed during the
evaluation of the handled computation $c$ using the clauses in the handler $h$.
A handler $\{ \expret{x_r} \mapsto c_r, \repi{\op_i(x_i, k_i) \mapsto c_i} \}$
has a return clause $\expret{x_r} \mapsto c_r$ where the variable $x_r$ denotes
the value of the handled computation $c$, and an operation clause
$\op_i(x_i, k_i) \mapsto c_i$ for each operation $\op_i$ where the variables
$x_i$ and $k_i$ denote the argument to $\op_i$ and the continuation from the
invocation of $\op_i$, respectively.
The notions of free variables and substitution are defined as usual.
We write $c[v/x]$ for the computation obtained by substituting the value $v$ for
the variable $x$ in the computation $c$.  We use similar notation to substitute
values for variables in types and substitute types for type variables.

\iffull{
%
% Figure environment removed

The semantics of the language is defined by the evaluation relation $\eval$,
which is the smallest binary relation over computations satisfying the
evaluation rules in Figure~\ref{fig:eval}.
The evaluation of a let-expression $\explet{x}{c_1}{c_2}$ begins by
evaluating the computation $c_1$.
When $c_1$ returns a value, the computation $c_2$ evaluates after substituting
the return value for $x$. If $c_1$ evaluates to an operation call
$\expop{v}{y}{c_1'}$, the outer context $\explet{x}{\hole}{c_2}$ is absorbed
into the continuation involved in the operation call.
The evaluation rules for conditional branching and function application are
standard.
The result of applying a primitive value relies on the metafunction $\zeta$,
which maps pairs of a primitive value and an argument value to computations.
For a handling-expression $\expwith{h}{c}$, the handled computation $c$
evaluates first.
When $c$ returns a value, the body of the return clause in the handler $h$
evaluates with the return value.
If $c$ evaluates to an operation call $\expop[\op_i]{v}{y}{c'}$, the body $c_i$
of the operation clause $\op_i(x_i, k_i) \mapsto c_i$ for $\op_i$ in the handler
$h$ evaluates after substituting the argument $v$ and the function
$\lambda y. \expwith{h}{c'}$ for variables $x_i$ and $k_i$, respectively.
%
Note that the function $\lambda y. \expwith{h}{c'}$ substituted for $k_i$ wraps
the delimited continuation $c'$ in the operation call by the handling construct
with the handler $h$.
It means that the operation calls in $c'$ are handled by the handler $h$.
For simplifying the typing rule for handlers, we assume that the handler $h$
provides operation clauses for all the operations performed by the handled
computation $c$.
%
The type system ensures that this assumption holds on any well-typed expression.
An alternative formalism for handlers is allowing handlers to omit operation
clauses for some operations and automatically forwarding calls to the operations
towards outer handlers.
This can be implemented in our formalism: given a handler that does not contain
an operation clause for $\op$, we can add to the handler an operation clause
$\op(x,k) \mapsto \expop{x}{y}{k~y}$, which is an implementation of operation
call forwarding.
}
\else{
%
The semantics of the language is defined by the evaluation relation $\eval$,
which is a binary relation over computations.
%
It is mostly identical to the semantics given by \citet{Pretnar15}; we refer to the supplementary material for its definition.
%
The only difference is that Pretnar's semantics allows handlers
that do not involve clauses for operations performed by a handled
computation---a call to such an operation is forwarded to outer
handlers---whereas our semantics assumes handlers to contain clauses for all such operations.
%
Our type system ensures that this assumption holds on any well-typed expression.
However, our language can also implement the forwarding semantics by encoding: given a handler
that does not contain an operation clause for $\op$, we add to the handler
an operation clause $\op(x,k) \mapsto \expop{x}{y}{k~y}$.\footnote{We use the semantics without forwarding in the body of the paper to simplify the typing rules.  Appendix~\ref{sec:language/forwarding} shows an extended typing rule for handling constructs that natively supports forwarding.
%
}\fi


\subsection{Type System} \label{sec:language/type-system}

\begin{wrapfigure}[11]{R}{0.65\textwidth}
    \vspace*{-4ex}
    $\begin{array}{rc@{\ \ }c@{\ }l@{\qquad}rc@{\ \ }c@{\ \ }l}
        \text{term} &
        t &::=& x \mid \ldots
        &
        \text{formula} &
        \phi &::=& A(\rep{t}) \mid \ldots
        \\
        \text{predicate} &
        A &::=& X \mid \ldots
        &
        \text{base type} &
        B &::=& \tybool \mid \ldots
    \end{array}$
    $\begin{array}{rc@{\ \ }c@{\ }l}
        \text{value type} &
        T &::=& \tyrfn{x}{B}{\phi} \mid (x: T) \rarr C
        \\
        \text{computation type} &
        C &::=& \tycomp{\Sigma}{T}{S}
        \\
        \text{operation signature} &
        \Sigma &::=& \{ \repi{\op_i : \forall \rep{X_i: \rep{B}_i}. F_i} \}
        \\
        &
        F &::=& (x: T_1) \rarr ((y: T_2) \rarr C_1) \rarr C_2
        \\
        \text{control effect} &
        S &::=& \square \mid \tyctl{x}{C_1}{C_2}
        \\
        \text{typing context} &
        \Gamma &::=& \emptyset \mid \Gamma, x: T \mid \Gamma, X: \rep{B}
    \end{array}$
    \caption{Type syntax.}
    \label{fig:type-syntax}
\end{wrapfigure}

Figure~\ref{fig:type-syntax} shows the syntax of types.
%
As in prior refinement type systems \cite{Bengston11, Rondon08, Unno09}, our type system allows a type
specification for values of base types, ranged over by $B$, such as $\tybool$
and $\tyint$, to be refined using logic formulas, ranged over by $\phi$.
%
Our type system is parameterized over a logic. We assume that the logic is a
predicate logic where: terms, denoted by $t$, include variables $x$;
predicates, denoted by $A$, include predicate variables $X$; and each primitive
value $p$ can be represented as a term.
%
Throughout the paper, we use the over-tilde notation to denote a sequence of
entities.  For example, $\rep{t}$ represents a sequence $t_1, \cdots, t_n$ of
some terms $t_1, \ldots, t_n$, and then $A(\rep{t})$ represents a formula
$A(t_1, \cdots, t_n)$.
%
We also assume that base types include at least the Boolean type $\tybool$.

Types consist of value and computation types, which are assigned to values and
computations, respectively.
%
A value type, denoted by $T$, is either a refinement type $\tyrfn{x}{B}{\phi}$,
which is assigned to a value $v$ of base type $B$ such that the formula
$\phi[v/x]$ is true, or a dependent function type $(x: T) \rarr C$, which is
assigned to a function that, given an argument $v$ of the type $T$, performs
the computation specified by the type $C[v/x]$.
We abbreviate $\tyfun{x}{T}{C}$ as $\tyfunshort{T}{C}$ if $x$ does not occur in $C$, and $\tyrfn{z}{B}{\exptrue}$ as $B$.

A computation type is formed by three components: an operation signature,
which specifies operations that a computation may perform;
a value type, which specifies the value that the computation returns
if any; and a control effect, which specifies how the computation modifies the
answer type via operation call.

Control effects, denoted by $S$, are inspired by the formalism of \citet{Sekiyama23} who extended control effects in simple typing~\cite{Materzok11} to dependent typing.
%
A control effect is either pure or impure.
%
The pure control effect $\square$ means that a computation calls no operation.
%
An impure control effect is given to a computation that may perform operations,
specifying how the execution of the computation modifies its answer type.
%
Impure control effects take the form $\tyctl{x}{C_1}{C_2}$ where variable $x$ is
bound in computation type $C_1$.
%
We write $\tyctlMB{C_1}{C_2}$ when $x$ does not occur in $C_1$.
%
In what follows, we first illustrate impure control effects in the simple,
nondependent form $\tyctlMB{C_1}{C_2}$ and then extend to the fully dependent
form $\tyctl{x}{C_1}{C_2}$ that can specify the behavior of captured
continuations using the input (denoted by $x$) to the continuations.

A control effect $\tyctlMB{C_1}{C_2}$ represents the answer type of a program
changes from type $C_1$ to type $C_2$.
%
When it is assigned to a computation $c$, the initial answer type $C_1$
specifies how the continuation of the computation $c$ up to the closest handing
construct behaves, and the final answer type $C_2$ specifies what can be
guaranteed for the \emph{meta-context}, i.e., the context of the closest
handling construct.
%
To see the idea more concretely, revisit the first example in
Section~\ref{sec:overview/atm}:
\begin{align}
	&\explet{x}{
		\expwith{
			\{ x_r \mapsto x_r, \ 
				\op((), k) \mapsto k~0 < k~1 \}
		}{1 + \op~()}
	}{c}
  ~.
\end{align}
%
Let $h$ be the handler in the example.
%
Focusing on the operation call $\op~()$, we can find that it captures the
continuation $\expwith{h}{1 + \hole}$.
%
Because the continuation behaves as if it is a pure function returning integers,
the initial answer type of $\op~()$ turns out to be the computation type
$\tycompMB{\tyint}{\square}$ (we omit $\Sigma$ for a while; it will be explained shortly).
%
Furthermore, by the operation call, the handling construct
$\expwith{h}{1+\op~()}$ is replaced with the body $k~0 < k~1$ of $\op$'s clause
in $h$ and the functional representation $v$ of the continuation is substituted
for $k$.
%
It means that the meta-context $\explet{x}{\hole}{c}$ of the operation call
takes the computation $v~0 < v~1$, which is of type
$\tycompMB{\tybool}{\square}$ (note that $v~0 < v~1$ is pure because $v$ is a
pure function).
%
Therefore, the final answer type of $\op~()$ is $\tycompMB{\tybool}{\square}$.
%
As a result, the impure control effect of $\op~()$ is
$\tyctlMB{\tycompMB{\tyint}{\square}}{\tycompMB{\tybool}{\square}}$.

\citet{Sekiyama23} extended the simple form of impure control effects to a
dependent form $\tyctl{x}{C_1}{C_2}$, where the initial answer type $C_1$ can
depend on inputs, denoted by variable $x$, to continuations.
%
For instance, consider the continuation $\expwith{h}{1 + \hole}$ captured in the
above example.
%
When passed an integer $n$, it returns $1+n$.
%
Using the dependent form of impure control effects, we can describe such
behavior by the control effect
$\tyctl{x}{\tycompMB{\tyrfn{y}{\tyint}{y=x+1}}{\square}}{\tycompMB{\tybool}{\square}}$,
where $x$ represents the input to the continuation and the refinement type
$\tyrfn{y}{\tyint}{y=x+1}$ precisely specifies the return value of the
continuation for input $x$.
%
The type of $x$ is matched with the continuation's input type.
%
Since the continuation of $\op~()$ takes integers, the type assigned to $x$ is $\tyint$.
%
In general, given a computation type $\tycompMB{T}{\tyctl{x}{C_1}{C_2}}$, the
type $T$ is assigned to the variable $x$ because it corresponds to the input type of
the continuations of computations given that computation type.
%
The type information refined by dependent impure control effects is exploited in
typechecking operation clauses.
%
In the example, our type system typechecks the body of $\op$'s clause by
assigning the function type
$\tyfun{x}{\tyint}{\tycompMB{\tyrfn{y}{\tyint}{y=x+1}}{\square}}$
to the continuation variable $k$.
%
Then, since the body is $k~0 < k~1$, its type---i.e., the final answer
type---can be refined to $\tycompMB{\tyrfn{z}{\tybool}{z=\exptrue}}{\square}$.
%
Hence, the type system can assign  control effect
$\tyctl{x}{\tycompMB{\tyrfn{y}{\tyint}{y=x+1}}{\square}}{\tycompMB{\tyrfn{z}{\tybool}{z=\exptrue}}{\square}}$
to the operation call and ensure that the meta-context takes $\exptrue$ finally
(if the handling construct terminates).
%
We will demonstrate the expressivity and usefulness of dependent control effects
in more detail in Section~\ref{sec:language/exmaples}.
%

Operation signatures, denoted by $\Sigma$, are sets of pairs of an operation
name and a type scheme.
%
We write $\repi{\cdot}$ to denote a sequence of entities indexed by $i$.
%
The type scheme associated with an operation $\op$ is in the form
$\forall \rep{X: \rep{B}}. (x: T_1) \rarr ((y: T_2) \rarr C_1) \rarr C_2$,
where the types $T_1$ and $T_2$ are the input and output types, respectively, of the operation and the types $C_1$ and $C_2$ are
the initial and final answer types, respectively, of the operation call for $\op$.
%
Recall that the initial answer type $C_1$ corresponds to the return type
of delimited continuations captured by the call to $\op$, and that the
continuations take the return values of the operation call.
%
Therefore, the function type $(y: T_2) \rarr C_1$ represents the type of the
captured delimited continuations. Note that the variable $y$ denotes values
passed to the continuations.
%
Furthermore, the final answer type $C_2$ corresponds to the type of
the operation clause for $\op$ in the closest enclosing handler.
%
Therefore, the operation clause $\op(x, k) \mapsto c$ in the
handler is typed by checking that the body $c$ is of the type $C_2$ with the assumption that
argument variable $x$ is of the type $T_1$ and the continuation variable $k$ is
of the type $(y: T_2) \rarr C_1$.
%
A notable point of the type scheme is that it can be parameterized over
predicates. The predicate variables $\rep{X}$ abstract over the predicates, and the annotations $\rep{B}$ represent
the (base) types of the arguments to the predicates.
%
This allows calls to the same
operation in different contexts to have different control effects, which is
crucial for precisely verifying programs with algebraic effects and handlers as
we will show in Section~\ref{sec:language/exmaples}.
%
It is also noteworthy that operation signatures include not only operation names
but also type schemes as in \citet{Kammar13} and \citet{Kammar17}.
%
It allows an operation to have different types depending on the contexts
where it is used.
%
Another approach is to include only operation names and assumes that unique types
are assigned to them globally as in, e.g., \citet{Bauer13} and \citet{Leijen17}.
%
We decided to assign types to operations locally because it makes the type
system more flexible in that the types of operations can be refined depending on
contexts if needed.

Typing contexts $\Gamma$ are lists of variable bindings $x: T$
and predicate variable bindings $X: \rep{B}$.
We write $\Gamma, \phi$ for $\Gamma, x: \tyrfn{z}{B}{\phi}$
where $x$ and $z$ are fresh.
The notions of free variables, free predicate variables,
and predicate substitution are defined as usual.

\fullfalse
\iffull{
% Figure environment removed

Well-formedness of typing contexts, value types, and computation types, whose
judgments are in the forms $\jdwf{}{\Gamma}$, $\jdwf{\Gamma}{T}$, and
$\jdwf{\Gamma}{C}$, respectively, are defined straightforwardly as shown in
Figure~\ref{fig:wf}.
%
We write $\dom(\Gamma)$ for the set of variables
and predicate variables bound in the typing context $\Gamma$.
%
The well-formedness of a computation type $\tycomp{\Sigma}{T}{S}$ rests on the
well-formedness of its components, that is, the operation signature $\Sigma$,
the value type $T$, and the control effect $S$.
%
An operation signature $\Sigma$ is well-formed under a typing context $\Gamma$,
written as $\jdwf{\Gamma}{\Sigma}$, if the type schemes associated with
operations in $\Sigma$ are well-formed under $\Gamma$.
%
Note that, for every type scheme $\forall \rep{X_i: \rep{B}_i}. F$, the qualified type $F$ is a function type.
%
The well-formedness of a control effect $S$ in terms of value type $T$, written as $\jdwf{\Gamma \mid T}{S}$ with a typing context $\Gamma$,
is derived by either of the last two rules in Figure~\ref{fig:wf}.
%
When the control effect $S$ is an impure effect $\tyctl{x}{C_1}{C_2}$, the type
$T$ is assigned to the variable $x$ because $x$ denotes values passed to
continuations of computations of the type $\tycomp{\Sigma}{T}{S}$ and the passed
values are the return values of the computations.
%
Additionally, we assume that the logic for refinements is equipped with
well-formedness judgments of formulas $\jdwf{\Gamma}{\phi}$ and of predicates
$\jdwf{\Gamma}{A : \rep{B}}$.
%
The properties assumed on these judgments are stated in the supplementary
material.
%
}\else{
%
Well-formedness of typing contexts, value types, and computation types, whose
judgments are in the forms $\jdwf{}{\Gamma}$, $\jdwf{\Gamma}{T}$, and
$\jdwf{\Gamma}{C}$, respectively, are defined straightforwardly by following \citet{Sekiyama23}.
We refer to the supplementary material for detail.
%
}\fi

% Figure environment removed

Typing judgements for values and computations are in the forms $\jdty{\Gamma}{v}{T}$ and $\jdty{\Gamma}{c}{C}$,
respectively.
Figure~\ref{fig:typing} shows the typing rules.
By \rulename{T-CVar}, a variable $x$ of a refinement type is assigned a type
which states that the value of this type is exactly $x$.
For a variable of a non-refinement type (i.e., a function type in our language),
the rule \rulename{T-Var} assigns the type associated with the variable in the
typing context.
%
The rule \rulename{T-Prim} uses the mapping $\ty$ to type primitive values $p$.
We assume that $\ty$ assigns an appropriate value type to every primitive value.
%
We refer to the supplementary material for the formalization of the assumption.
%
The rule \rulename{T-Fun} for functions, \rulename{T-App} for function
applications, and \rulename{T-If} for conditional branches are standard in
refinement type systems (with support for value-dependent refinements).
%
The rules \rulename{T-VSub} and \rulename{T-CSub} allow values and computations,
respectively, to be typed at supertypes of their types.
%
We will define subtyping shortly.
%
By \rulename{T-Ret}, a value-return $\expret{v}$ has a computation type
where the operation signature is empty, the return value type is the type of $v$,
and the control effect is pure.

To type a let-expression $\explet{x}{c_1}{c_2}$, either the rule \rulename{T-LetP} or \rulename{T-LetIp} is used. Both of them
require that the types of the sub-expressions $c_1$ and $c_2$ have the same
operation signature $\Sigma$ and then assign $\Sigma$ to the type of the entire
let-expression.
%
The typing context for $c_2$ is extended by $x: T_1$ with the value
type $T_1$ of $c_1$, but $x$ cannot occur in $\Sigma$ and $T_2$ (as well as $C_{21}$ in \rulename{T-LetIp}) to prevent
the leakage of $x$ from its scope.
%
On the other hand, the two rules differ in how they treat control effects.
%
When both of the control effects of $c_1$ and $c_2$ are pure, the rule \rulename{T-LetP} is used. It states that the control effect of the entire let-expression is also pure.
%
When both are impure, the rule \rulename{T-LetIp} is used. It states that the control effect of the let-expression results in an impure control effect that is composed of the control effects of $c_1$ and $c_2$.
%
Note that, even when one of the control effects of $c_1$ and $c_2$ is pure and the other is
impure, we can view both of them as impure effects via subtyping
because it allows converting a pure control effect to an impure control
effect, as shown later.
%
We first explain how the composition works in the non-dependent form. Let
the control effect of $c_1$ be $\tyctlMB{C_{11}}{C_{12}}$ and that of $c_2$ be $\tyctlMB{C_{21}}{C_{22}}$,
and assume that a control effect $\tyctlMB{C_1}{C_2}$ is assigned to the
let-expression.
%
First, recall that the type $C_1$ expresses the return type of the continuation
of the let-expression up to the closest handling construct and that the closest
handling construct is replaced by a computation of the type $C_2$.
%
Based on this idea, the types $C_1$ and $C_2$ can be determined as follows.
%
First, because the delimited continuation of the let-expression is matched with
that of the computation $c_2$, the initial answer type $C_{21}$ of $c_2$
expresses the return type of the delimited continuation of the let-expression.
Therefore, the type $C_1$ should be matched with the type $C_{21}$.
%
Second, because the closest handling construct enclosing the let-expression is the same
as the one enclosing the sub-computation $c_1$, the type $C_2$ should be matched
with the final answer type $C_{12}$ of $c_1$.
%
Therefore, the control effect $\tyctlMB{C_1}{C_2}$ should be matched with
$\tyctlMB{C_{21}}{C_{12}}$, as stated in \rulename{T-LetIp}.
%
Furthermore, the rule \rulename{T-LetIp} requires that the
initial answer type $C_{11}$ of $c_1$ to be the same as the final answer type
$C_{22}$ of $c_2$.
%
This requirement is explained as follows.
%
First, the computation $c_1$ expects its delimited continuation to behave as
specified by the type $C_{11}$.
%
The delimited continuation of $c_1$ first evaluates the succeeding computation
$c_2$.
%
The final answer type $C_{22}$ of $c_2$ expresses that the closest
handling construct enclosing $c_2$ behaves as specified by the type $C_{22}$.
%
Because the closest handling construct enclosing $c_2$ corresponds to the
top-level handling construct in the delimited continuation of $c_1$, the
type $C_{11}$ should be matched with the type $C_{22}$.
%
We now extend to the fully dependent form.
%
From the discussion thus far, we can let
the control effects of $c_1$, $c_2$, and the let-expression be
$\tyctl{x_1}{C}{C_{12}}$, $\tyctl{x_2}{C_{21}}{C}$, and
$\tyctl{y}{C_{21}}{C_{12}}$ respectively, for some variables $x_1$, $x_2$, and $y$.
%
Then, the constraints on the names of these variables are determined as follows.
%
First, the input to the delimited continuation of $c_1$, which is denoted by the variable $x_1$,
should be matched with the evaluation result of $c_1$.
%
Then, since the let-expression binds the variable $x$ to the evaluation result of $c_1$,
the variable $x_1$ is matched with $x$.
%
Second, because the delimited continuation of $c_2$ is matched with
that of the let-expression, the inputs to them should be matched with each other.
%
They are denoted by the variables $x_2$ and $y$ respectively, and hence the variable $x_2$ is matched with $y$.

The rule \rulename{T-Hndl} for handling constructs $\expwith{h}{c}$
is one of the most important rules of our system.
%
It assumes that the handled computation $c$ is of a type
$\tycomp{\Sigma}{T}{\tyctl{x_r}{C_1}{C_2}}$, where the control effect is impure.
%
Even when $c$ is pure (i.e., performs no operation), it can have an impure control effect via subtyping.
%
Because the type of the handling construct represents how the expression is
viewed from the context, it should be matched with the final answer type $C_2$
of the handled computation $c$.
%
The premises in the second line define typing disciplines that the clauses in the installed handler $h$ have to satisfy.
%
First, let us consider the return clause $\expret{x_r} \mapsto c_r$.
%
Because the variable $x_r$ denotes the return value of the handled
computation $c$, the value type $T$ of $c$ is assigned to $x_r$.
%
Moreover, since the return clause is executed after evaluating $c$, the body
$c_r$ is the delimited continuation of $c$.
%
Therefore, the type of $c_r$ should be matched with the initial answer type $C_1$
of $c$.
%
Because the variable $x_r$ bound in the return clause can be viewed as the input
to the delimited continuation $c_r$, it should be matched with the
variable $x_r$ bound in the impure control effect $\tyctl{x_r}{C_1}{C_2}$.
%
Operation clauses are typed using the corresponding type schemes in the
operation signature $\Sigma$, as explained above.
%
Note that the rule also requires the installed handler $h$ to include a clause
for each of the operations in $\Sigma$, i.e., those that $c$ may perform.

The rule \rulename{T-Op} for operation calls is
another important rule.
%
Consider an operation call $\op~v$.
%
The rule assumes that an enclosing handler addresses the operation $\op$
by requiring that an operation signature $\Sigma$ assigned to the operation call
include the operation $\op$ with a type scheme
$\forall \rep{X: \rep{B}}. (x: T_1) \rarr ((y: T_2) \rarr C_1) \rarr C_2$,
and instantiates the predicate
variables $\rep{X}$ in the type scheme with well-formed predicates $\rep{A}$ to
reflect the contextual information of the operation call.
%
Then, it checks that the argument $v$ has the input type $T_1[\rep{A/X}]$ of the
operation.
%
Finally, the rule assigns the output type $T_2[\rep{A/X}][v/x]$ of the operation
as the value type of the operation call, and $C_1[\rep{A/X}][v/x]$ and
$C_2[\rep{A/X}][v/x]$ as the initial and final answer types of the operation call, respectively
(note that the types $T_2$, $C_1$, and $C_2$ are parameterized over predicates and arguments).

% Figure environment removed

The type system defines four kinds of subtyping judgments:
$\jdsub{\Gamma}{T_1}{T_2}$ for value types,
$\jdsub{\Gamma}{C_1}{C_2}$ for computations types,
$\jdsub{\Gamma}{\Sigma_1}{\Sigma_2}$ for operation signatures, and
$\jdsub{\Gamma \mid T}{S_1}{S_2}$ for control effects.
%
Figure~\ref{fig:subty} shows the subtyping rules.
%
The subtyping rules for control effects are adopted from the work of \citet{Sekiyama23},
which extends subtyping for control effects given by \citet{Materzok11} to
dependent typing.
%
The rules \rulename{S-Rfn} and \rulename{S-Fun} for value types are standard.
%
The judgement $\valid{\Gamma}{\phi}$ in \rulename{S-Rfn} means
the semantic validity of the formula $\phi$ under the assumption $\Gamma$.
%
Subtyping between operation signatures is determined by \rulename{S-Sig}.
%
This rule is based on the observation that an operation signature $\Sigma$
represents the types of operation clauses in handlers, as seen in
\rulename{T-Hndl}.
%
Then, the rule \rulename{S-Sig} can be viewed as defining a subtyping relation between
the types of handlers (except for return clauses): a handler for operations in
$\Sigma_1$ can be used as one for operations in $\Sigma_2$ if every operation
$\op$ in $\Sigma_2$ is included in $\Sigma_1$ (i.e., the handler has an
operation clause for every $\op$ in $\Sigma_2$) and the type scheme of $\op$ in
$\Sigma_1$ is a subtype of the type scheme of $\op$ in $\Sigma_2$ (i.e., the
operation clause for $\op$ in the handler works as one for $\op$ in $\Sigma_2$).
%
Given a computation type $C_1 \defeq \tycomp{\Sigma_1}{T_1}{S_1}$ and its
supertype $C_2 \defeq \tycomp{\Sigma_2}{T_2}{S_2}$, a handler for operations
performed by the computations of the type $C_2$ (i.e., the operations in
$\Sigma_2$) is required to be able to handle operations performed by the
computations of the type $C_1$ (i.e., the operations in $\Sigma_1$) because the
subtyping allows deeming the computations of $C_1$ to be of $C_2$.
%
The safety of such handling is ensured by requiring $\Sigma_2 <: \Sigma_1$.
%
In the rule \rulename{S-Comp}, the first premise represents this requirement.
%
The second premise $\jdsub{\Gamma}{T_1}{T_2}$ in \rulename{S-Comp} allows
viewing the return values of the computations of the type $C_1$ as those of the
type $C_2$.
%
The third premise $\jdsub{\Gamma \mid T_1}{S_1}{S_2}$ expresses that the use of
effects by the computations of the type $C_1$ is subsumed by the use of effects
allowed by the type $C_2$.
%
It is derived by the last three rules: \rulename{S-Pure},
\rulename{S-ATM}, and \rulename{S-Embed}.
%
The rule \rulename{S-Pure} just states reflexivity of the pure control effect.
%
If both $S_1$ and $S_2$ are impure, the rule \rulename{S-ATM} is applied.
%
Because initial answer types represent the assumptions of computations
on their contexts, \rulename{S-ATM} allows strengthening the assumptions by
being contravariant in them.
%
By contrast, because final answer types represent the guarantees of how
enclosing handling constructs behave, \rulename{S-ATM} allows weakening the
guarantees by being covariant in them.
%
Note that the typing context for the initial answer types is extended
with the binding $x : T_1$ because they may reference the inputs to the
continuations via the variable $x$ and the inputs are of the type
$T_1$.
%
Finally, the rule \rulename{S-Embed} allows converting the pure control effect
to an impure control effect $\tyctl{x}{C_1}{C_2}$.
%
Because a computation $c$ with the pure control effect performs no operation,
what is guaranteed for the behavior of the handling construct enclosing $c$
coincides with what is assumed on $c$'s delimited continuation.
%
Because the guarantee and assumption are specified by the types $C_2$ and
$C_1$, respectively, if $C_1$ is matched with $C_2$---more generally, the
``assumption'' $C_1$ implies the ``guarantee'' $C_2$---the pure computation
$c$ can be viewed as the computation with the impure control effect
$\tyctl{x}{C_1}{C_2}$.
%
The first premise in \rulename{S-Embed} formalizes this idea.
%
Note that, because the variable $x$ is bound in the type $C_1$, the rule
\rulename{S-Embed} disallows $x$ to occur in the type $C_2$.

Finally, we state the type safety of our system. Its proof, via progress and subject reduction, is given in the supplementary material.
We define $\eval^*$ as the reflexive, transitive closure of the one-step evaluation relation $\eval$.
\begin{theorem}[type safety] \label{thm:safety}
    If\, $\jdty{\emptyset}{c}{\tycomp{\Sigma}{T}{S}}$ and $c \eval^* c'$, then one of the following holds:
    % \begin{itemize}
    (1) $c' = \expret{v}$ for some $v$ such that $\jdty{\emptyset}{v}{T}$;
    (2) $c' = K[\op~v]$ for some $K$, $\op$, and $v$ such that $\op \in \dom(\Sigma)$; or
    (3) $c' \eval c''$ for some $c''$ such that $\jdty{\emptyset}{c''}{\tycomp{\Sigma}{T}{S}}$~.
    % \end{itemize}
\end{theorem}


\section{\add{Assembly-level Verification for Page Table Traversal and Mapping}}
\label{sec:experiment_appendix}
%To both validate and demonstrate the value of the modal approach to reasoning about virtual memory management, 
% we study several
% We validate our logic by studying
% distillations of key VMM functionality.
% real concerns of virtual memory managers.
% Recall from Section \ref{sec:logic} that virtual points-to assertions work just like regular points-to assertions, by design.

\replace{
In this section we verify several critical and challenging pieces of VMM code.
First, in several stages, we work up to mapping a new page in the current address space.
This requires a number of independently challenging substeps: dynamically traversing a page table to find
the appropriate L1 entry to update; inserting additional levels of the page table if necessary (updating
the VMM invariants along the way);
converting the physical addresses found in intermediate entries into the corresponding virtual addresses
that can be used for memory access;
installing the new mapping;
and collecting sufficient resources to form a virtual points-to assertion.
Of these, only the second-to-last step (installing the correct mapping into the
current address space) has previously been formally verified with respect to a machine model with address translation.
Second, we formally verify a switch into a new address space as part of a task switch,
the first such verification handling both old and new processes' assertions (in different address spaces) at the time of the switch.
}{
While our logic was developed and proven sound for x86-64 assembly,
Section \ref{sec:traversingC} described verification of software page table walking code (\lstinline|pte_get_next_table| and \lstinline|walkpgdir|)
as if at the level of C for improved readability.
This appendix describes the actual assembly-level verification carried out in Rocq.
Careful readers of both Section \ref{sec:traversingC} and this appendix will notice
strong similarities in the assertions and and reasoning, for good reason:
The C code in Section \ref{sec:traversingC} was the original kernel code that was compiled
(with no optimizations) to x86-64 assembly and verified with our logic, and the proof outlines
in that section largely back-port the assembly proofs back to C.
\looseness=-1
}

\add{
 This section describes the assembly proofs without reference to the C outlines given in Section \ref{sec:traversingC}.
 The main additional details of note at the assembly level are:
 \begin{itemize}
 \item Accurate treatment of register management (particularly the AMD64 System V calling convention) leading to more direct correspondence
       with our logic
 \item The assembly is naturally more verbose than the C, so the proof outlines are relatively more sparse, with assertions written
       only for key updates.
 \item Bitwise manipulations of page table entries are harder to follow than C's bitfield access support.
       Multiple manipulations which are each explicit in C become adjacent (or sometimes non-adjacent) bitwise operations.
       The critical ones are commented in the assembly figures.
 \item And compared to the C-based presentation earlier, there are differences in logical variable names. For example,
       the assembly proofs use \textsf{entry} as the name for the \emph{physical} address of the entry modified by
       \lstinline|pte_get_next_table| in Figure \ref{fig:calltopteinitialize}, whereas to make sense of the C code
       in Figure \ref{fig:calltopteinitializeC} we used \textsf{entry} consistently with the C variable name and introduced
       separate logical names for physical addresses. This propagates to figures presenting larger invariants separately,
       as they also refer to the logical names from the proofs.
 \end{itemize}
}

%\begin{comment}
%\todo[inline]{Identity mappings are difficult, and our current approach won't quite work. Consider trying to have a virtual pointsto for an actual page table entry (i.e., that one could use to update a page table mapping), while also having a virtual pointsto for an address that entry mapped. With the current (let's call it v1) solution, we can't actually have both of those simultaneously!  That's because the PTE pointsto will assert full ownership of the physical memory cell holding the PTE as its data value, while the virtual pointsto for the data mapped by that entry will \emph{also} assert (fractional) ownership of all entries a page table walk would traverse.
%}
%\todo[inline,color=violet]{This doesn't seem to cause issues with the mapping/unmapping examples, only with changing intermediate page table pointers. The mapping example requires a virtual pointsto for the blank PTE, and once filled in that ownership can be immediately split to create the 512 new virtual pointsto assertions for the newly mapped page. Conversely, for unmapping we'd assume ownership of all the relevant virtual pointsto assertions for the page we're unmapping, at which point we can (with a bit of work) show that they all correspond to the same L1 PTE, and extract the 512 fractional shares of that entry from the pointsto assertions.  But changing intermediate page tables, as one would do for coallescing or splitting a superpage while preserving the virtual-to-physical mappings, couldn't be done without some really complicated separating implication tricks.}
%\todo[inline,color=green]{One possible approach to resolving this, which we came up with in our Tuesday meeting, is to recognize that the current (v1) virtual points-to is too strong, because it really doesn't care about \emph{owning} those fractional resources, it only cares that \emph{something} ensures the correct page table walk exists. Iris has a ghost map resource where authoritative ownership of an individual key-value pair can be handled as a resource.  (Colin was using this in the filesystem cache.)
%We can use that mechanism to separate the virtual-to-physical translation from the physical memory involved (Kolanski and Klein may have done something similar for different reasons): (fractional) virtual points-to assertions can be defined in terms of (fractional) ownership of these authoritative ghost map entry assertions, plus sharing an invariant that the current installed page table respects all entries of the mapping. Unmapping collects the authoritative map kvpairs from collecting the assertions, and then can remove them from the ghost map and update the page tables. Critically, physical ownership of the page tables then lives in the invariant on the current page table, so some virtual pointsto assertions can refer to memory in those page tables.
%This still works with the modality, since that invariant is also semantically a predicate on a page table root.
%Let's call this v2.
%}
%\end{comment}
\subsection{Traversing Live Page Tables}
\label{sec:traversing}
We build up to the main task of mapping a new page after traversing page tables in software.
The mapping operation of Figure \ref{fig:mapping_code} assumes an operation \textsf{walkpgdir} which must traverse the page tables
in order to locate the address of the L1 entry to update --- 
% possibly allocating tables for levels 3, 2, and 1 in the process,
% installing them into levels 4, 3, and 2, along the way.
possibly allocating new L3, L2, and L1 tables as necessary.
Traversing the page tables is itself challenging functionality to verify: loading the current table root from \lstinline|cr3| is straightforward
(a \lstinline|mov| instruction), however this produces the physical address of \lstinline|cr3|, not the virtual address the kernel code would use to access that memory.
This problem repeats at each level of the page table: assuming the code has \emph{somehow} read the appropriate L4 (or L3, or L2) entry, those entries again
yield physical addresses, not virtual.

\subsubsection{Loading Page-Table Address Value}
We will discuss access to the level 4 table later (Section \ref{wlkpgdir}). But for subsequent levels, the base address of level $n$ must be
fetched from the appropriate entry in the level $n+1$ table.
This is the role of \lstinline|pte_get_next_table| (Figures \ref{fig:calltopteinitialize} and \ref{fig:p2v}):
it is passed the virtual address of the page table entry in level $n+1$, and should return the \emph{virtual} 
address of the \emph{base} of the level $n$ table
indicated by that entry.
If the entry is empty (i.e., this is a sparse part of the page table representation),
the code also allocates a page for the level $n$ table, installs it in the level $n+1$ entry, and establishes appropriate invariants.
Figure \ref{fig:calltopteinitialize} presents the initial part of the function, which performs the allocation if necessary.
Figure \ref{fig:p2v} (discussed in Section \ref{sec:p2v}) deals with the cases where no allocation is necessary \emph{or} the allocation has already
been performed by the code in this figure.
\looseness=-1

Note that the specification does \emph{not} assume a specific page table level --- logical parameter \textsf{v} represents the level
of the entry passed as an argument, and this code
is used for all three level transitions when traversing page tables (4 to 3, 3 to 2, 2 to 1).
This comes into play with a subtlety of the specification of \lstinline|pte_get_next_table| that we will
revisit several times: \lstinline|pte_get_next_table|'s specification
assumes it is given a virtual \emph{vpte-pointsto}
(a virtual points-to exposing the underlying physical address instead of existentially quantifying it;
 see Section \ref{sec:mapnew}) granting access to the specified entry,
but its postcondition does not yield new virtual points-to assertions!
Instead it merely computes the base virtual address of the next table, and returns adequate capabilities (discussed in Section \ref{subsec:identitymappings})
for the \emph{caller} to construct a vpte-pointsto for any entry of the next table level (if this is not an L1 entry ---
the caller knows which level of the table this is for).
\looseness=-1

Within \textsf{get\_next\_table}, after a standard function prologue, the code 
loads the entry pointed to by the argument (logical variable \textsf{entry} in the proof outline).
This is a page table entry: a 64-bit word divided into bit-fields for
the physical address of the next table, and control bits like the valid bit, as discussed in 
Section \ref{sec:backgroundonmachinemodel}.



\ifPLDI
Line \ref{line:mask_present} checks % In the condensed figure, it's all on one line
\else
Lines \ref{line:mask_present}--\ref{line:check_entry_present} check
\fi
if the entry's ``present'' bit is set.
If it is zero, a new page must be allocated for the next level of the table --- which is done by the fall-through
from Line \ref{line:check_entry_present_jump}'s conditional jump. Otherwise the code jumps ahead to
the case for the next level already existing, which is discussed in Section \ref{sec:p2v} and Figure \ref{fig:p2v}.
First, we must discuss another refinement of the address space invariant, establishing
enough structure on the page tables themselves to allow the traversal.
The code for allocating a new level of the page table must establish this extended invariant.

%wshiftll (wshiftll (natToWord 64 entry) (WordImpl.concat (WordImpl.zero 56) (WordImpl.from_nat 8 12 ^& WordImpl.concat (WordImpl.zero 2) WO~1~1~1~1~1~1)) ^& constf)
%(WordImpl.concat (WordImpl.zero 56) (natToWord 8 12 ^& WordImpl.concat (WordImpl.zero 2) WO~1~1~1~1~1~1))
%
%wshiftll
 %      (wshiftll
%          ((((natToWord 64 entry ^& WordImpl.concat (WordImpl.zero 32) consta ^| WordImpl.concat (WordImpl.zero 32) (natToWord 32 2))
%             ^& WordImpl.concat (WordImpl.zero 32) constb ^| WordImpl.concat (WordImpl.zero 32) (natToWord 32 4)) ^& constd
%            ^| wshiftll
%                 (wshiftll (nextpaddr ^+ ^~ (natToWord 64 KERNBASE))
%                    (WordImpl.concat (WordImpl.zero 56) (WordImpl.from_nat 8 12 ^& WordImpl.concat (WordImpl.zero 2) WO~1~1~1~1~1~1))
%                  ^& constf)
%                 (WordImpl.concat (WordImpl.zero 56) (WordImpl.from_nat 8 12 ^& WordImpl.concat (WordImpl.zero 2) WO~1~1~1~1~1~1)))
%           ^& WordImpl.concat (WordImpl.zero 32) conste ^| wone 64)
%          (WordImpl.concat (WordImpl.zero 56) (WordImpl.from_nat 8 12 ^& WordImpl.concat (WordImpl.zero 2) WO~1~1~1~1~1~1)) ^& constf)
%       (WordImpl.concat (WordImpl.zero 56) (natToWord 8 12 ^& WordImpl.concat (WordImpl.zero 2) WO~1~1~1~1~1~1)) 
% Figure environment removed

\subsubsection{Identity Mappings}
\label{subsec:identitymappings}
Kernels need to convert between physical and virtual addresses, in both directions.
Traversing the page tables in software is the simplest way to convert a virtual address to a physical address; this is the context we are working up to.
However, implementing this virtual-to-physical (V2P) translation in this way ironically requires physical-to-virtual (P2V) translation,
because the addresses stored in page table entries are physical, but memory accesses issued by the OS code use virtual addresses.
% There is no universal way to convert physical addresses to virtual --- doing so relies on the kernel maintaining careful invariants or
% additional data structures to enable P2V translation.
\looseness=-1

Because VMM operations are performance-critical for many workloads, most kernels 
maintain invariants that enable very fast P2V conversions (rather than adding another data structure).
Most kernels maintain an invariant on their page tables that the virtual address of any page used for a page table 
% lives at a virtual address whose value 
is \emph{a constant offset from the physical address} --- a practice sometimes referred to as \emph{identity mapping} 
(even though the physical-to-virtual translation
is typically not literally the identity function, but adding a non-zero constant offset).\footnote{Some kernels do this for all physical memory on the machine, simplifying interaction
with DMA devices.
On newer platforms like RISC-V, this sometimes truly is an identity mapping ---
x86-64 machines are forced into offsets by backwards compatibility with bootloaders that cannot access the full memory space of the
machine.
}

For this reason we extend the per-address-space invariant as in Figure \ref{fig:peraspaceinvariant_with_p2v_extension}, to also track which
addresses we can perform a P2V conversion on by a adding a constant offset.
$\Xi$ is another ghost map, from physical addresses to the level of the page table they represent (1--4).
\emph{Only} physical addresses in $\Xi$ can undergo P2V conversion. 
Section \ref{sec:p2v} describes the actual conversion,
but we describe the invariant here 
because adding new level 3/2/1 tables must maintain the invariant.

% Figure environment removed

For each $\paddr\mapsto \textsf{v} \in\Xi$, the invariant tracks a virtual points-to justifying that virtual address $\paddr+\textsf{KERNBASE}$ maps to physical address $\paddr$
(the ``Ghost translation'' in Figure \ref{fig:peraspaceinvariant_with_p2v_extension});
fractional ownership of the physical memory for that page table entry;
and for valid entries (with the present bit set) above L1, ghost map tokens for every entry in the table pointed to by the entry, which can be used
to repeat the process one level down. 
% (L1 entries point to data pages, whose physical memory ownership resides in some virtual points-to).
The assertion on Line \ref{line:conditional_children} of Figure \ref{fig:calltopteinitialize} comes from the invariant one level up; 
if the valid bit is set,
the code can return those child tokens without the conditional guard.
\looseness=-1

The fractional ownership of the entry's physical memory is subtle. Recall that $\textsf{L}_{4}\_\textsf{L}_{1}\_\textsf{PointsTo}$ retains some physical
ownership of each page table entry that is traversed (proportional to how many virtual addresses share the entry).
So in general the invariant cannot keep full permission to the memory in this part of the invariant, or it would overlap the page table walk for virtual points-to
assertions. But in the case where the entry is invalid, we may need to write to it (e.g., to install a reference to a next-level table, as we do in Figure \ref{fig:calltopteinitialize}),
which requires full permission. Fortunately, the entry can only be in use if its valid bit is set; if the valid bit is not set we know
that no virtual points-to entry in $\delta$/$\theta$ holds any partial ownership.
Thus we use the invariant portion annotated as ``Entry validity'' in Figure \ref{fig:peraspaceinvariant_with_p2v_extension} to capture this:
if the entry is invalid the invariant holds full ownership of the entry, so it can be updated; while if the entry is valid,
the invariant owns only a constant non-zero fragment sufficient to read the entry, but not modify it (which would invalidate some virtual points-to assertions):
\begin{equation*}
 \ulcorner \textsf{qfrac} = 1 \leftrightarrow \; \lnot\textsf{entry\_present }(\vale) \urcorner \tag{*}
\end{equation*}
Thus the fractional ownership of the physical location is enough for Line \ref{line:read_entry_contents} in Figure \ref{fig:calltopteinitialize} to access the entry, though in \lstinline|get_next_table|
the caller has pulled that piece of information out of the invariant and passed it for the entry at hand.
This removal appears explicitly in assertions,
as the argument to the invariant is $\Xi\setminus\{\mathsf{entry}\}$ (indexing by the set $\Xi$ allows us to borrow the physical resources
for a specific page table entry out of the invariant, and later put them back).
Line \ref{line:check_entry_present_jump}'s conditional then determines in the fall-through case that the bit is not set, which 
together with other facts entails $\textsf{qfrac} = 1$ at Line \ref{line:after_concluding_qfrac1},
and permits storing a new entry (in ellided code around Line \ref{line:install_new_entry}).
\looseness=-1

This seemingly-simple piece of code has a highly non-trivial correctness argument, which depends critically on detailed invariants on how access to page table
entries is shared between parts of the kernel. No prior work has engaged with this problem.

% Concretely speaking, going back to Line 15 in Figure \ref{fig:calltopteinitialize}, to read the value referenced by physical address \textsf{entry} while preserving the soundness of memory mappings, our extended invariant introduces the side condition (*)
% \begin{equation*}
%  \ulcorner \textsf{qfrac} = 1 \leftrightarrow \; \lnot\textsf{entry\_present }(\vale) \urcorner \tag{*}
% \end{equation*}
% assuring that looking the identity mapping for \textsf{entry} is safe under the subtle justification which equates the full ownership to the non/presence of the entry which can only be known when investigated in Line 21 in Figure \ref{fig:calltopteinitialize}.
\begin{comment}
 % Figure environment removed
\end{comment}


 \subsubsection{Installing a New Table}
 After obtaining the identity mapping for \textsf{entry}, we are able to load the \textsf{entry\_val} into \textsf{rdi}, and check the presence bit through
\ifPLDI
Line \ref{line:mask_present} % in condensed version, all on same line
\else
Lines \ref{line:mask_present}--\ref{line:check_entry_present} 
\fi
in Figure \ref{fig:calltopteinitialize}.
Accessing the presence bit and checking the value allows us to exploit the condition (*) that was just discussed when verifying the allocation
path (i.e., when the entry is invalid  and Lines \ref{line:alloc_path_start}--\ref{line:alloc_path_end} in Figure \ref{fig:calltopteinitialize}
must allocate the next level of tables).
This operation is subtle. To reiterate: the operation requires that the relevant table entry is readable, but the exact portion of ownership 
returned must be determined by inspecting the valid bit of the value in memory --- so full ownership is returned only for unused entries.
When the bit is not set, that entails full ownership of the entry's memory ($\textsf{qfrac} = 1$) and justifies writing to that memory.
Otherwise, the code jumps past the end of this listing, to the following code at the top of Figure \ref{fig:p2v} (which is also the
continuation of this code).

% Figure environment removed

If the entry is not set, \textsf{pte\_initialize} (Line \ref{line:call_to_pte_initialize} in Figure \ref{fig:calltopteinitialize}) 
allocates a physical page (internally utilizing the only unverified (trusted) code in our case studies, the page-allocator's \textsf{kalloc},\footnote{
  This is an allocator for regions of pre-zeroed physical memory that is mapped, but not accessed by the allocator itself,
  as is typical for slab allocators~\cite{bonwick1994slab}.
  Its verification would be similar to verifying a usermode \textsf{malloc} verifications~\cite{Chlipala2013Bedrock,wickerson2010explicit},
  just with additional invariants on the memory pool.
} 
on Line \ref{line:call_to_kalloc} in Figure \ref{pteinitializespec}). 
Since we are using \textsf{pte\_initialize} for page-table address allocation, we must relate this newly
allocated physical address to the identity mapping map $\Xi$ --- 
see Line \ref{line:page_of_caps} in Figure \ref{fig:calltopteinitialize}, where
\texttt{kalloc}'s specification guarantees it has returned memory from a designated memory
pool that is already mapped
\ifPLDI
\else
\footnote{A reasonable reader might wonder where this pool
initially comes from, and how it might grow when needed. Typically an initial mapping subject to this identity mapping
constraint is set up prior to transition to 64-bit kernel code (notably,
a page table must exist \emph{before} virtual memory is enabled during boot, as part of enabling it is setting
a page table root).
Growing this pool later requires cooperation of physical memory range allocation and virtual memory range allocation,
typically by starting general virtual address allocation at the highest physical memory address plus the identity mapping offset.
This reserves the virtual addresses corresponding to all physical addresses plus the offset for later use in this pool,
as needed.
} 
\fi
and satisfies the offset invariants.
% \todo[inline,color=blue]{colin frontier.
% Stuck with line 31 onwards in Figure 7. rax holds nextpaddr, but I think that should be entrypfn, and 
% the explicit entrypfn id token assertion should go away, as its covered by the forall assertion.
% then the postcondition for pte-initialize should have a specific level now for the entries,
% like 0, which can be updated in the view shift on line 42.
% }
% Focusing on the specification of \textsf{pte\_initialize} separately in Figure \ref{fig:pteinitializespec}, 
% we right immediately realize that instead of seeing see a physical pointsto for the fresly page-table address 
% (e.g. $\mathsf{nextpaddr} \mapsto_{\mathsf{p}} \mathsf{w64\_0}$) deliberately in the post-conditoin in Lines 15-16,
%  we observe a full-ownership token representing the knowledge that a frame and all the entries indexed from this 
% frame are freshly allocated with full-ownership to be a part of the identity map, $\Xi$. 
The soundness argument of this specification relies on the fact that these freshly allocated resources are part 
of an entry construction that has not been completed yet: the presence bit is set 
(Line \ref{line:install_new_entry} in Figure \ref{fig:calltopteinitialize}) after these freshly allocated resources are incorporated to the 
entry construction via the page-frame portion of the PTE. In other words, the side condition, (*),
 formalizes that any access to the entry with these resources is \textit{invalid} (in the sense of not necessarily
having accompanying resources) until the entry is marked present (and thus the memory returned from \textsf{kalloc}
moves into the page table invariant.

\add{Note that the C presentation in Figure \ref{fig:calltopteinitializeC}
omitted the precondition on the implication of Figure \ref{fig:calltopteinitialize}'s Line \ref{line:page_of_caps},
which is logically equivalent to \textsf{True} since \textsf{entry\_present} checks if the present bit is set in an entry,
and \textsf{pte\_initialize} sets that bit. The actual invariant has this form here, and in the postcondition
of \lstinline|pte_initialize| (Figure \ref{pteinitializespec}), to match the conditional form from earlier in
\lstinline|pte_get_next_table| (which is also provably true when the check of the present bit
determines that the entry was already valid/present).
Our proof discharges the conditional at the join point, rather than eagerly in each branch.
}

\subsubsection{Physical-to-Virtual Conversion with \textsf{P2V}}
\label{sec:p2v}
Once we know the entry refers to a physical address in the identity mapping range ($\Xi$)
(via the branch at Line \ref{line:check_entry_present_jump}, or  by allocating and installing a new entry
as just discussed for Lines \ref{line:check_entry_present_jump}--\ref{line:end_of_allocation_path}), 
we can convert this frame address to a corresponding virtual address via the identity mappings
discussed in Section \ref{subsec:identitymappings} and Figure \ref{fig:peraspaceinvariant_with_p2v_extension}.
in the last lines of \lstinline|pte_get_next_table| shown in Figure \ref{fig:p2v} (the continuation of Figure \ref{fig:calltopteinitialize}).
This is a critical piece of the full page table walk verification.
In our small kernel (Line \ref{line:p2v} in Figure \ref{fig:p2v}), as in larger kernels, the C macro \texttt{P2V} common to many kernels
is actually just addition by the constant offset mentioned in Section \ref{subsec:identitymappings}.
But the correctness of this simple instruction is quite subtle.
%  and cannot be proven 
% without the extended invariant (Figure \ref{fig:peraspaceinvariant_with_p2v_extension})
% worked out Section \ref{subsec:identitymappings}.

% Figure environment removed
Figure \ref{fig:p2v} shows the verification of the end of \lstinline|pte_get_next_table| specialized to the case where 
where no allocation was necessary (i.e., the conditional on Line \ref{line:check_entry_present} of Figure \ref{fig:calltopteinitialize} was taken).
In this case, the true present bit allows access to the child tokens from Line \ref{line:conditional_children} of Figure \ref{fig:calltopteinitialize},
which is then refined to the assertion on Line \ref{line:children} of Figure \ref{fig:p2v}.
The code loads \lstinline|rcx| with the offset value \textsf{KERNBASE}, which gives us the value of the virtual address ($\textsf{entry}_{\textsf{pfn}}$ \textsf{+KERNBASE})
of the \emph{base} of the next level of the page table.
% \todo[inline]{the next sentence depends on having figure 10 updated to reflect the page-worth of tokens}
While we could now convert this address to a virtual points-to, this is not necessarily the correct thing to do.
The caller \lstinline|walkpgdir| (discussed next) uses \lstinline|pte_get_next_table| to retrieve just the base address,
because only the caller knows which entry in the subsequent table will be accessed (it depends on the corresponding bits from the virtual
address being translated). So instead we pass back the per-address-space invariant with the identity mapping resources for \lstinline|entry|
pulled out. The caller determines which entry in that table must actually
be accessed --- by selecting the appropriate index into the 512 ghost map tokens returned in the postcondition,
and using the ghost translation and physical location portions of the invariant to assemble a vpte-pointsto
that justifies the caller's subsequent access to a particular entry of the returned table.
% in the identity map ($\Xi\setminus\{entry\}$) of the kernel invariant.
% the logical update in Specification  Lines 5-10 to 10-14 for obtaining virtual-pointsto resource for the frame 
% ($\textsf{entry}_{\textsf{pfn}}$) by removing it from the ghost map ($\Xi\setminus\{entry\}\cup \{\textsf{entry}_{\textsf{pfn}}) \}$) 
% in Line 5 and compute the identity mapping for this physical frame address in Line 13 in Figure \ref{fig:p2v}).

\subsubsection{Walking Page-Table Tree: Calling \textsf{pte\_get\_next\_table} for Each Level}
\label{wlkpgdir}
% Figure environment removed

% Figure environment removed
Implementing a software page-table walk amounts to calling \textsf{pte\_get\_next\_table} for each level as shown in Figure \ref{walkpgdir}. 
The key part of the specification and proof for a page table walk is accumulation of memory mappings for the page-table entries 
visited and frame addresses for page-tables. 
For example, Lines \ref{line:ex_l4_vpte} and \ref{line:ex_l3_vpte} in Figure \ref{walkpgdir} show the virtual pte-pointsto assertions for L4 and L3 entries.
In the final post-condition, we expect the accumulation of these resources from each level -- $\textsf{R}_{\textsf{walk}}$ -- 
which allows us to construct and return the path to the L1 entry in the tree to insert a new page.  

This is the code which performs most actual physical-to-virtual conversions using the identity mapping portion of the per-address-space invariant.
\lstinline|walkpgdir| accepts a \emph{virtual} pointer to the base of the L4 table, and the address to translate.
The precondition provides knowledge that the virtual base of the L4 is at the appropriate offset from the current \lstinline|cr3| value,
but does not provide a virtual points-to assertion --- because the function must calculate (Lines \ref{line:start_pml4_calc}--\ref{line:end_pml4_calc})
which entry it needs access to.
Instead the precondition has 512 identity map tokens, guaranteeing that every entry on the page is subject to the identity mapping invariant.
Line \ref{line:end_pml4_calc} calculates the virtual address of the relevant entry, and the subsequent view shift
pulls that entry out of the identity mapping ($\Xi$) and fetches its corresponding resources as
described by Figure \ref{fig:peraspaceinvariant_with_p2v_extension} and Section \ref{subsec:identitymappings}.
The ghost translation and physical location are used to form the virtual pte-pointsto for the L4 entry
(Line \ref{line:first_pte_pointsto}), with the entry validity and next-level indexing
satisfying the rest of the precondition for \lstinline|pte_get_next_table|.
\lstinline|pte_get_next_table| then, as described earlier, checks the valid bit in the indicated
entry and either returns the (unconditional) tokens for the L3 entry physical addresses (if valid), or
allocates into the entry and returns new (also unconditional) tokens for the L3 entry physical addresses.
\lstinline|pte_get_next_table|'s first call (Line \ref{line:first_getnext_call}) returns
the virtual address of the base of the L3 table (a \emph{page directory pointer}, so PDP, in official
x86-64 terminology). Then the situation to move from that pointer to the base of the L2
is just like the process just followed: the proof calculates the address of the relevant
L3 entry, uses the appropriate L3 identity mapping token to construct a virtual pte-pointsto to that entry,
and passes that along with additional resources pulled out of the invariant to another call to
\lstinline|pte_get_next_table|. That call then returns the base of an L2 table, and the process
repeats until the function returns the virtual address of the relevant L1 entry.
That will then be used in the next section by the caller of \lstinline|walkpgdir|
to install a new mapping.


% \textsf{walkpgdir}, as a client, holds the knowledge that there exists an identity mapping for the physical entry address (\textsf{entry})
%  in the root page table ($\textsf{L}_{4}$):  $\mathsf{entry} \mapsto_{\textsf{id}} \textsf{\_}$ in Specification Line 3 is a partially owned
%  token for accessing and looking up the resources in the identity map, $\Xi$, to construct the \textit{virtual-to-physical} pointsto relation 
% $\textsf{entry+KERNBASE} \mapsto_{\textsf{vpte,qfrac}} \textsf{entry \entry\_val}$ with the virtual address (\textsf{entry+KERNBASE}) obtained 
% by offsetting the physical address (\textsf{entry}). With this knowledge on the root-page-table-entry, we can start traversing the page-table 
% tree which requires locating the address of the next table -- a call to \textsf{pte\_get\_next\_table} shown in Figure \ref{fig:calltopteinitialize}. 
% Beyond a frame, the precondition before Line 15 requires the current address space invariant, and knowledge that \textsf{entry} is mapped to a 
% random entry value, subtly, 
% the operation also, at least, requires that the relevant table entry is readable, but the exact portion of ownership 
% returned must be determined by inspecting the valid bit
% of the value in memory --- so full ownership is returned only for unused entries.
% This is a simple piece of code whose functionality is critical and whose correctness is highly non-trivial. No prior work engages with this problem.



%% Figure environment removed



%\caption{Traversing page-tables, and allocating entries as needed while mapping-a-page in Figure \ref{fig:mappingcode}.}
% \citet{kolanski08vstte,kolanski09tphols} verified a single code block with their logic which was roughly Figure \ref{fig:mapping_code} for a 2-level ARM
% page table, but several critical complexities our work deals with were not addressed.
% First, beyond the limitations discussed in Section \ref{sec:overly-restrictive}, Kolanski and Klein assumed that virtual addresses
% for page tables at each level were given as parameters rather than verifying any conversion from physical addresses to virtual addresses (or even axiomatizing their lookup).
% In contrast, our verification articulates the address space invariant from which the physical-to-virtual translation can be implemented.
% Second, our proof deals with the construction of a valid virtual points-to \emph{to the PTE to update} in mapping, which Kolanski and Klein also
% assumed was given.
% \todo{some of this is really an argument for our verification being more thorough, rather than being about our logic}

% Reasoning about the page table walk in their logic would have required 
% could reason about the walk, but would need to explicitly prove that all other invariants
% of the kernel, the current address space, and all other address spaces of interest were preserved by each update, because their model
% only supports separation within a single address space. In our model, this follows for free from making
% our separation logic directly aware of address translation and internalizing assumptions about other address spaces as further separable assertions.
% Kolanski and Klein did address part of the walk information for a 2-level page table (a possible ARM configuration), but 

% \textsc{seL4} currently still trusts address translations; it models page tables as a data structure in regular memory, thus not capturing the possibility that even
% temporarily destroying the mappings and restoring them can actually crash the OS. \textsc{CertiKOS} papers share little in the way of precise details about
% their virtual memory management, but because their core technology is based on a fork of \textsc{CompCert}, whose model of memory is
% a set of unordered block allocations, we can infer their proofs must also trust these translations.


\subsection{Mapping a New Page}
\label{sec:mapnew}
One of the key tasks of a page fault handler in a general-purpose OS kernel is
to map new pages into an address space by writing into an existing page table via a call\\
\centerline{\textsf{vaspace\_mappage(pte\_t *pml4, void *va,uintptr\_t fpaddr)}}\\
in Figure \ref{fig:mapping_code}.
To do so, with a given allocated a fresh page (\textsf{fpaddr}), then calculate the appropriate
known-valid page table walks (via \textsf{walkpgdir} Line \ref{line:call_walkpgdir} in Figure \ref{fig:mapping_code})  and update 
the appropriate L1 page table entry (Line 35 in Figure \ref{fig:mapping_code});
unmapping is the reverse of the logic we discuss here.
\looseness=-1
%\lstset{
%  columns=fullflexible,
%  numbers=left,
%  basicstyle=\ttfamily,
%  keywordstyle=\color{blue}\bfseries,
%  morekeywords={mov,add,call},
%  emph={rsp,rdx,rax,rbx,rbp,rsi,rdi,rcx,r8,r9,r10,r11,r12,r13,r14,r15},
%  emphstyle=\color{green},
%  emph={[2]cr3},
%  emphstyle={[2]\color{violet}},
%  morecomment=[l]{;;},
%  mathescape
%}
% Figure environment removed

In Figure \ref{fig:mapping_code}, we see an address ($\vaddr$) currently not
mapped to a page ($\theta \; !!\; \vaddr = \texttt{None}$). Mapping a fresh
physical page to back the desired virtual page first requires ensuring
the existence of a memory location for an appropriate L1 table entry.
The code uses a helper function \lstinline{walkpgdir} (discussed again in Section \ref{sec:traversing}).
\textsf{walkpgdir}'s postcondition contains virtual \emph{PTE} pointsto assertions ($\mapsto_{\textsf{vpte}}$)
both for ensuring partial page table walk reaching the
L1 entry (l1e) by asserting that higher levels of the page table exist (R$_{\textsf{walk}}$ in Figure \ref{fig:rwalk}), 
and for allowing access to the memory of the L1 entry via virtual address (R$_{\textsf{l1e}}$ in Figure \ref{fig:rwalk}).

% After obtaining a virtual address \textsf{pte\_addr} in \textsf{rax} backed 
% by the physical memory for the L1 entry that will be used to translate the virtual addresses
% we are mapping, we save it to \textsf{r14} to be updated later in Line 9.

%In the precondition, we see Line 12 allocates a fresh page-aligned, zero-initialized page  (at \textsf{fpaddr}),
%returning a pre-filled PTE entry in \textsf{rax} ($+3$ sets the lower 2 bits).

% , to hold the freshly
% allocated physical page address (\textsf{fpaddr}) in Line X.

We already discussed for the upper level page-tables how the entry-present checks are handled.
However, for L1 entries this check is left to the caller of the 
traversal function \textsf{walkpgdir}. In other words, unlike what we see in R$_{\textsf{walk}}$ for the upper levels where all entry-present
checks have already been performed, the specification in R$_{\textsf{l1e}}$ ensures that page table entry for L1 needs to be checked at the caller site. 
By doing so, as we see in Figure \ref{fig:mapping_code}, the page reference \textsf{fpaddr} is linked to back the virtual address \textsf{va} 
only if it is not already referring to a physical resource (Lines \ref{line:mappage_pte_present_start}--\ref{line:mappage_pte_present_end} in Figure \ref{fig:mapping_code}). 

The crucial step in addition to traversing the page table in Figure \ref{walkpgdir} is actually updating the L1 entry (Line \ref{line:updatepfn} in Figure \ref{fig:mapping_code}),
via the virtual address (\textsf{pt\_entry+KERNBASE}) known to translate to the appropriate physical address, in our example the L1
table entry address ($\textsf{PTE\_ADDR\_TO\_PFN(fpaddr)}$).

Unlike the only prior work verifying analogous code for mapping a new page~\cite{kolanski08vstte,kolanski09tphols}, our proof above
does \emph{not} need to reason directly over the operational semantics,
making this the first verification we know of for mapping a virtual memory page that 
stays entirely at the program logic level.
\looseness=-1
% By incorporating verification of the
% \lstinline|ensure_L1| function (see Section \ref{sec:traversing}), our verification also directly handles several subtle aspects which
% were axiomatized in prior work.
\ifPLDI
\else
\subsection{Unmapping a Page}
\todo[inline]{update (esp. line refs) for new mapping code}
The reverse operation, unmapping a designated page that is currently mapped,
would essentially be the reverse of
the reasoning around line 22 above: given the virtual points-to assertions for all 512
machine words of memory that the L1 entry would map,
and information about the physical location, 
full permission on the L1 entry could be obtained, allowing the construction of a
full virtual PTE pointer for it, setting to 0, and reclaiming the now-unmapped physical memory.
\fi


% % Figure environment removed

% \subsection{Change of Address Space}
% A critical piece of \emph{trusted} code in current verified OS kernels is the assembly code to change the current address space; current verified OS kernels currently 
% lack effective ways to specify and reason about this low-level operation, for reasons outlined in Section \ref{sec:relwork}.

% Figure \ref{fig:swtch} gives simplified code for a basic task switch, the heart of an OS scheduler implementation. This is code that saves the context (registers and stack)
% of the running thread (here in a structure pointed to by \lstinline|rdi|'s value shown in Lines \ref{line:start_save}--\ref{line:end_save} of Figure \ref{fig:swtch}) and restores the context of 
% an existing thread (from \lstinline|rsi| shown in abbreviated Lines \ref{line:start_restore}--\ref{line:end_restore}), including the corresponding change of address space for a target thread in another process.
% This code assumes the System V AMD64 ABI calling convention, where the normal registers not mentioned are caller-save, and therefore saved on the stack of the thread
% that calls this code, as well as on the new stack of the thread that is restored, thus only the callee-save registers and \texttt{cr3} must be 
% restored.\footnote{We are simplifying by ignoring non-integer registers (e.g., floating point, vector registers),
% and the caller-save registers should be initialized to 0 to avoid leaking information across processes, but this captures the key challenges.}
% With the addition of a return instruction, this code would satisfy the C function signature\footnote{This is the function in UNIX 6th Edition 
% with the infamous ``You are not expected to understand this'' comment~\cite{lions1996lions}.}
% \centerline{\lstinline[language=C]|void swtch(context_t* save, context_t* restore);|}\\
% A call to this code begins executing one thread (until just before Line \ref{line:end_save}) in one address space ($\rtv$), whose information will be saved in a structure at address $old$,
% and finishes execution executing a different thread in a different address space (Line \ref{line:end_restore} on) whose information is initially in $new$.

% Because this code does not directly update the instruction pointer, it is worth explaining \emph{how} this switches threads: by switching address spaces and stacks. 
% This is meant to be called with a return address for the current thread stored on the current stack when called. 
% The precondition of the return address on the initial stack requires the callee-save register values at the time of the call: those stored in the first 
% half of the code.
% Likewise, part of the invariant of the stack of the second thread, the one being restored, is that the return address on \emph{that} stack requires the saved 
% callee-save registers stored in that context to be in registers as its precondition.

% The wrinkle, and the importance of the modal treatment of assertions, is that the target thread's precondition is \emph{relative to its address space}, 
% not the address space of the calling thread, which is reflected by
% the other-space modality 
% $[\rtv']( I\texttt{ASpace}(\theta,\Xi,m) \ast \texttt{Pother})$
% in the specfication. 
% The precondition of this code,
% in context, would include that the initial stack pointer (before \lstinline|rsp| is updated)
% has a return address expecting the then-current callee-save register values and 
% suitably updated (i.e., post-return) stack in the \emph{current} (initial) address space;
% this would be part of \textsf{P} in the precondition.
% The specification also requires that
% the stack pointer saved in the context to restore expects the same of the saved registers and stack 
% \emph{in the other address space}. 
% The other-space modality plays a critical role here; \textsf{Pother} would contain these assumptions in the other
% address space.
% \looseness=-1

% % Lines 10--16 save the current context into memory (in the current address space).
% % Line 22 saves the initial page table root.
% % Lines 33--38 begin restoring the target context, including the stack pointer (line 33),
% % which may not be mapped in the address space at that time: it is the stack for the context being
% % loaded into the CPU.
% % The actual address switch occurs on line 45, which is verified with our modal rule for updating \lstinline|cr3|,
% % and thus shifts resources in and out of other-space modalities as appropriate.

% The postcondition is analagous to the precondition, but interpreted \emph{in the new address space}: the then-current (updated) stack would have a return address expecting the new (restored) register values (again, in \textsf{Pother}),
% and the saved context's invariant captures the precondition for restoring its execution \emph{in the previous address space} (as part of \textsf{P}). 

% Immediately after the page table switch, assertions about the saved and restored contexts are
% guarded by a modality for the retiring
% address space \rtv{} (Line \ref{line:modality_switch}), per
% \textsc{WriteToRegCtlFromRegModal} (Figure \ref{fig:wpdamd}),
% because
% there is no guarantee that the data structures of the previous address space are mapped in the new address space.
% The ability to transfer that points-to information out of that modality is specific to a given kernel's design. 
% Kernels that map kernel memory into all address spaces would need invariants
% that justified moving those assertions out of the other-space modality.
% % Following Spectre and Meltdown, this kernel design became less prevalent because speculative execution of accesses to kernel addresses could leak information even if the access did eventually cause a fault (the user/kernel mode permission check was done after fetching data from memory). Thus many modern kernels have reverted to the older kernel design where the kernel inhabits its own unique address space, and user processes have only enough extra material mapped in their address spaces to switch into the kernel (CPUs do not speculate past updates to \texttt{cr3}).
% \looseness=-1

% While prior work has verified context switches within a single address space~\cite{ni2007contexts}, and context switches
% without any code before or after~\cite{syeda2020formal} (i.e., not reasoning about the impact of address space change
% on what data was accessible), this is the first verification handling both.
% \looseness=-1

% \begin{comment}
% \[  
% $\specline{\exists (\entryf ,\;\entrytr,\; \entrytw,\; \entryo,\;\textsf{pte\_addr },\paddr) \; \ldotp\textsf{P} \ast  I\texttt{ASpace}(\theta,m) \ast  \texttt{r14}\mapsto_{\textsf{r}} \_ \ast \texttt{rdi}\mapsto_{r} \vaddr \ast \texttt{rax}\mapsto_{\textsf{r}} \textsf{ pte\_addr} \; \ast }_{\rtv}$
% $\specline{ \ulcorner  \texttt{addr\_L1 }(\vaddr, \entryo) = \paddr \urcorner \ast \ulcorner \texttt{entry\_present } \entryf \land \texttt{entry\_present } \entrytr \land  \texttt{entry\_present } \entrytw \urcorner \; \ast}_{\rtv}$
% $\specline{\nfpointsto{\mask\vaddr\maskfour\rtv}{\mask\vaddr\maskfouroff\rtv}\entryf\qone\naddr \; \ast \nfpointsto{\mask\vaddr\maskthree\entryf}{\mask\vaddr\maskthreeoff\entryf}\entrytr\qtwo\naddr \ast}_{\rtv}$ 
% $\specline{  \nfpointsto{\mask\vaddr\masktwo\entrytr}{\mask\vaddr\masktwooff\entrytr}\paddr\qthree\entryo \;\ast \texttt{pte\_addr} \mapsto_{\texttt{vpte}} \paddr \;(\texttt{wzero 64}) \ast \texttt{rax}\mapsto_{\textsf{r}} \texttt{pte\_addr}  }_{\rtv}$
% mov r14, rax ;; Save that before another call
% $\specline{\textsf{P} \ast  I\texttt{ASpace}(\theta,m) \ast  \texttt{r14}\mapsto_{\textsf{r}} \texttt{pte\_addr} \ast \texttt{rdi}\mapsto_{\textsf{r}} \vaddr \ast \texttt{rax}\mapsto_{\textsf{r}} \textsf{ pte\_addr} \; \ast }_{\rtv}$
% $\specline{ \nfpointsto{\mask\vaddr\maskfour\rtv}{\mask\vaddr\maskfouroff\rtv}\entryf\qone\naddr \ast \ulcorner \texttt{entry\_present } \entryf \land \texttt{entry\_present } \entrytr \land  \texttt{entry\_present } \entrytw \urcorner \ast}_{\rtv}$ 
% $\specline{  \nfpointsto{\mask\vaddr\maskthree\entryf}{\mask\vaddr\maskthreeoff\entryf}\entrytr\qtwo\naddr \ast \nfpointsto{\mask\vaddr\masktwo\entrytr}{\mask\vaddr\masktwooff\entrytr}\paddr\qthree\entryo \;\ast}_{\rtv}$
% $\specline{\texttt{pte\_addr} \mapsto_{\texttt{vpte}} \paddr \;(\texttt{wzero 64}) \ast \texttt{rax}\mapsto_{\textsf{r}} \texttt{pte\_addr}  }_{\rtv}$
% call alloc_phys_page_or_panic
% $\specline{\textsf{P} \ast  I\texttt{ASpace}(\theta,m) \ast  \texttt{r14}\mapsto_{\textsf{r}} \texttt{pte\_addr} \ast \texttt{rdi}\mapsto_{\textsf{r}} \vaddr \;\ast \nfpointsto{\mask\vaddr\maskfour\rtv}{\mask\vaddr\maskfouroff\rtv}\entryf\qone\naddr \ast}_{\rtv}$ 
% $\specline{  \nfpointsto{\mask\vaddr\maskthree\entryf}{\mask\vaddr\maskthreeoff\entryf}\entrytr\qtwo\naddr \ast \nfpointsto{\mask\vaddr\masktwo\entrytr}{\mask\vaddr\masktwooff\entrytr}\paddr\qthree\naddr \;\ast}_{\rtv}$
% $\specline{\texttt{pte\_addr} \mapsto_{\texttt{vpte}} \paddr\; (\texttt{wzero 64})  \ast \ulcorner \texttt{entry\_present } \entryf \land \texttt{entry\_present } \entrytr \land  \texttt{entry\_present } \entrytw \urcorner}_{\rtv}$
% $\specline{\exists \texttt{ fpaddr} \ldotp \ulcorner \texttt{aligned fpaddr} \urcorner \ast \texttt{rax}\mapsto_{\textsf{r}} \texttt{fpaddr+3} \ast \texttt{fpaddr} \mapsto_{\textsf{p}} (\texttt{wzero 64}) \ast \ulcorner \texttt{entry\_present (fpaddr+3)}\urcorner}_{\rtv}$
% ;; Calculate new L1 entry
% mov [r14], rax ;; store the page table entry, mapping the page
% $\specline{\textsf{P} \ast  I\texttt{ASpace}(\theta,m) \ast  \texttt{r14}\mapsto_{\textsf{r}} \texttt{pte\_addr} \ast \texttt{rdi}\mapsto_{\textsf{r}} \vaddr \;\ast \nfpointsto{\mask\vaddr\maskfour\rtv}{\mask\vaddr\maskfouroff\rtv}\entryf\qone\naddr \ast}_{\rtv}$ 
% $\specline{  \nfpointsto{\mask\vaddr\maskthree\entryf}{\mask\vaddr\maskthreeoff\entryf}\entrytr\qtwo\naddr \ast \nfpointsto{\mask\vaddr\masktwo\entrytr}{\mask\vaddr\masktwooff\entrytr}\paddr\qthree\entryo \;\ast}_{\rtv}$
% $\specline{\texttt{pte\_addr} \mapsto_{\texttt{vpte}} \paddr \;(\texttt{fpaddr+3}) \; \ast \ulcorner \texttt{entry\_present } \entryf \land \texttt{entry\_present } \entrytr \land  \texttt{entry\_present } \entrytw \urcorner }_{\rtv}$
% $\specline{\ulcorner \texttt{aligned fpaddr} \urcorner \ast \texttt{rax}\mapsto_{\textsf{r}} \texttt{fpaddr+3} \ast \texttt{fpaddr} \mapsto_{\textsf{p}} (\texttt{wzero 64}) \ast \ulcorner \texttt{entry\_present fpaddr+3}\urcorner}_{\rtv}$
% $\;\;\;\;\;\;\;\;\;\;\;\;\;\;\;\;\;\;\;\;\;\;\;\;\;\;\;\;\;\;\;\;\;\;\;\;\;\;\;\;\;\;\;\; \sqsubseteq $
% $\specline{\textsf{P} \ast  I\texttt{ASpace}(\theta,m) \ast  \texttt{r14}\mapsto_{\textsf{r}} \texttt{pte\_addr} \ast \texttt{rdi}\mapsto_{\textsf{r}} \vaddr \ast }_{\rtv}$
% $\specline{\textsf{L}_{4}\_\textsf{L}_{1}\_\textsf{PointsTo}(\vaddr,\entryf,\entrytr,\entrytw,\fpaddr+3) \ast \ulcorner \theta \;!!\;\vaddr = \texttt{None}\urcorner \; \ast}_{\rtv}$
% $\specline{\ulcorner \texttt{aligned fpaddr} \urcorner \ast \texttt{rax}\mapsto_{\textsf{r}} \texttt{fpaddr+3} \ast \texttt{fpaddr} \mapsto_{\textsf{p}} (\texttt{wzero 64}) }_{\rtv}$
% $\;\;\;\;\;\;\;\;\;\;\;\;\;\;\;\;\;\;\;\;\;\;\;\;\;\;\;\;\;\;\;\;\;\;\;\;\;\;\;\;\;\;\;\; \sqsubseteq $
% $\specline{\textsf{P} \ast  I\texttt{ASpace} (<[\vaddr:=\texttt{fpaddr}]> \theta,m) \ast}_{\rtv}$
% $\specline{\ulcorner \texttt{aligned fpaddr} \urcorner \ast \texttt{fpaddr} \mapsto_{\textsf{p}} \textsf{ wzero 64} \ast \ghostmaptoken{\delta{}s}{\rtv}{\delta}  \ast\sumwalkabs\vaddr\qfrac\fpaddr}_{\rtv}$
% $\;\;\;\;\;\;\;\;\;\;\;\;\;\;\;\;\;\;\;\;\;\;\;\;\;\;\;\;\;\;\;\;\;\;\;\;\;\;\;\;\;\;\;\; \sqsubseteq $
%   $\specline{\textsf{P} \ast  I\texttt{ASpace} (<[\vaddr:=\texttt{fpaddr}]> \theta,m) \ast \vaddr \mapsto_{\textsf{vpte}}\; \{\qfrac\} \;\fpaddr \textsf{ wzero 64}}_{\rtv}$
% $\;\;\;\;\;\;\;\;\;\;\;\;\;\;\;\;\;\;\;\;\;\;\;\;\;\;\;\;\;\;\;\;\;\;\;\;\;\;\;\;\;\;\;\; \sqsubseteq $
% $\specline{\textsf{P} \ast  I\texttt{ASpace} (<[\vaddr:=\texttt{fpaddr}]> \theta,m) \ast \vaddr \mapsto_{\textsf{v}}\; \{\qfrac\} \textsf{wzero 64}}_{\rtv}$
% \end{comment}


% % \subsection{Typing Rule for Operation Forwarding} \label{sec:language/forwarding}
\section{Typing Rule for Operation Forwarding} \label{sec:language/forwarding}
The typing rule for handling constructs presented in Section 3.2 of the main paper %\ref{sec:language/type-system}
assumes that a handler covers all the operations performed
by the handled expression.
In this section, we present another typing rule for handling constructs
to allow \emph{operation forwarding}, that is, allow unhandled operations to be forwarded to outer handlers automatically.
The idea of the typing rule is simple: we derive it from an implementation of operation forwarding.
As mentioned in Section~3.1 of the main paper, %\ref{sec:language/syntax-semantics},
operation forwarding can be implemented in a calculus without forwarding by adding to a handler
an operation clause $\op(x, k) \mapsto \explet{y}{\op~x}{k~y}$ for each forwarded operation $\op$.
Therefore, we can derive the new typing rule
from the typing of the added clauses. The following is the thus derived new typing rule for handling constructs which natively supports operation forwarding:
\[
\infer{\jdty{\Gamma}{\expwith{h}{c}}{C_2}}
{\begin{gathered}
    h = \{ \expret{x_r} \mapsto c_r, \repi{\op_i(x_i, k_i) \mapsto c_i} \} \quad
    \jdty{\Gamma}{c}{\tycomp{\Sigma}{T}{\tyctl{x_r}{C_1}{C_2}}} \\[-.5ex]
    \jdty{\Gamma, x_r: T}{c_r}{C_1} \quad
    \bigrepi{\jdty{\Gamma, \rep{X_i: \rep{B}_i}, x_i: T_{1i}, k_i: (y_i: T_{2i}) \rarr C_{1i}}{c_i}{C_{2i}}} \\[-.5ex]
    \bigrepi{ \Sigma \ni \op_i: \forall \rep{X_i: \rep{B}_i}. (x_i: T_{1i}) \rarr ((y_i: T_{2i}) \rarr C_{1i}) \rarr C_{2i} }
    \qquad \mathit{Ops}_{\mathrm{fwd}} = \dom(\Sigma) \setminus \dom(h) \\[-.5ex]
    \bigrepi[\op \in \mathit{Ops}_{\mathrm{fwd}}]{\begin{gathered}
        \begin{multlined}
            \Sigma \ni \op : \forall \rep{X^{\op}: \rep{B^{\op}}}. (x^{\op}: T_1^{\op}) \rarr \\
                ((y^{\op}: T_2^{\op}) \rarr
                    \tycomp{\Sigma'}{T_0^{\op}}{\tyctl{z^{\op}}{C_0^{\op}}{C_1^{\op}}}) \rarr
                \tycomp{\Sigma'}{T_0^{\op}}{\tyctl{z^{\op}}{C_0^{\op}}{C_2^{\op}}}
        \end{multlined} \\
        \Sigma' \ni \op : \forall \rep{X^{\op}: \rep{B^{\op}}}. (x^{\op}: T_1^{\op}) \rarr ((y^{\op}: T_2^{\op}) \rarr C_1^{\op}) \rarr C_2^{\op} \quad
        y^{\op} \notin C_0^{\op} \setminus \{ z^{\op} \}
    \end{gathered}
    }
\end{gathered}}
\]
where $\dom(\Sigma)$ denotes the set of the operations associated by $\Sigma$
and $\dom(h)$ denotes the set of the operations handled by $h$,
that is, the set $\{ \repi{\op_i} \}$.
The first two lines are the same as \rulename{T-Hndl}.
The third line is also similar to the last premise of \rulename{T-Hndl},
but here $\Sigma$ is allowed to contain operations other than those handled by $h$.
$\mathit{Ops}_{\mathrm{fwd}}$ is exactly the set of the unhandled (i.e., forwarded) operations.
The last part is the requirement for the forwarded operations,
which can be obtained from the typing derivations of $\op(x, k) \mapsto \explet{y}{\op~x}{k~y}$ as follows.
When we simulate the operation forwarding with the explicit clause,
the operation call $\op~x$ in the clause is handled
by an immediate outer handler (we denote it by $h'$ in what follows).
Therefore, its operation signature is different from $\Sigma$;
in fact, it corresponds to $\Sigma'$ in the rule.
Also, the answer types of the original operation calls of $\op$
(i.e., the answer types of the operation calls of $\op$ in the handled computation $c$)
should have $\Sigma'$ as their operation signatures,
because the final answer type corresponds to the type of the handling construct,
which is handled by the immediate outer handler $h'$.
Therefore, the types of the forwarded operations in $\Sigma$ contains $\Sigma'$
in their answer types.
In addition, the types
$T_0^{\op}$, $T_1^{\op}$, $T_2^{\op}$, $C_0^{\op}$, $C_1^{\op}$, and $C_2^{\op}$
appear multiple times in $\Sigma$ and $\Sigma'$, restricting the type schemes of the operations in $\mathit{Ops}_{\mathrm{fwd}}$.
This restriction can be understood as follows.
First, assume that the original operation call of $\op$ in $c$
has the operation signature $\Sigma$ such that
\[
    \Sigma \ni \op : T_1^{\op} \rarr
        (T_2^{\op} \rarr
            \tycomp{\Sigma'}{T_0^{\op}}{\tyctlMB{C_0^{\op}}{C_1^{\op}}}) \rarr
        \tycomp{\Sigma'}{T_{0A}^{\op}}{\tyctlMB{C_{0A}^{\op}}{C_2^{\op}}}
\]
for some $T_1^{\op}$, $T_2^{\op}$, $T_0^{\op}$, $C_0^{\op}$, $C_1^{\op}$,
$T_{0A}^{\op}$, $C_{0A}^{\op}$, $C_2^{\op}$, and $\Sigma'$, under a context $\Gamma$.
Here we consider only simple types for simplicity,
but a similar argument can be made for dependent and refinement types
by appropriately naming the variables like in the rule above.
Note that its answer types have $\Sigma'$ as described earlier,
and that we do not impose the restriction yet.
From the assumption, the clause $\explet{y}{\op~x}{k~y}$ should be typed
under the context $\Gamma, x: T_1^{\op},
k: T_2^{\op} \rarr \tycomp{\Sigma'}{T_0^{\op}}{\tyctlMB{C_0^{\op}}{C_1^{\op}}}$~.
Then, the input type of $\op$ in the clause should be the type of $x$, namely, $T_1^{\op}$,
and the output type of $\op$ should be the type of the variable $y$,
which turns out to be $T_2^{\op}$ from the type of $k$.
Therefore, the operation signature $\Sigma'$ for $\op~x$ should contain
$\op : T_1^{\op} \rarr (T_2^{\op} \rarr C_{1A}^{\op}) \rarr C_{2A}^{\op}$
for some $C_{1A}^{\op}$ and $C_{2A}^{\op}$~.
Then, according to the typing rules for operation calls and let-expressions,
it is required that $C_{1A}^{\op} = C_1^{\op}$,
and the type of $\explet{y}{\op~x}{k~y}$ is
$\tycomp{\Sigma'}{T_{0}^{\op}}{\tyctlMB{C_{0}^{\op}}{C_{2A}^{\op}}}$~.
Finally, since the type of the clause corresponds to
the final answer type of the operation $\op$ in $\Sigma$
(which is $\tycomp{\Sigma'}{T_{0A}^{\op}}{\tyctlMB{C_{0A}^{\op}}{C_2^{\op}}}$ from the assumption),
it should satisfy
$T_{0}^{\op} = T_{0A}^{\op}$, $C_{0}^{\op} = C_{0A}^{\op}$, and $C_{2A}^{\op} = C_{2}^{\op}$~.



%!TEX root = ../Schur indices and line operators.tex


\section{Discussion}














\section{Implementation} \label{sec:impl}

In this section, we describe our prototype implementation of
a refinement type checking and inference system, \textsc{RCaml}\footnote{available at \url{https://github.com/hiroshi-unno/coar}}.
It takes a program written in a subset of the OCaml 5 language
(including algebraic data types, pattern matching, recursive functions, exceptions, references,
let-polymorphism, and effect handlers)
and a specification of its main function represented as a refinement type.
It first (1) obtains an ML-typed AST of the program
using OCaml's compiler library,
(2) infers refinement-free operation signatures and control effects,
(3) generates refinement constraints for the program and its specification as Constrained Horn Clauses (CHCs) (see e.g., \cite{Bjorner2015a}),
and finally (4) solves these constraints
to verify whether the program satisfies the specifications.
The steps (3) and (4), where the refinement type checking is reduced to CHC solving,
follow existing standard approach such as \cite{Unno09,Rondon08}.
The inference of (refinement-free) operation signatures is similar to
that of record types using row variables,
and is mutually recursive with the inference of control effects.
It is based on the type inference system of control effects for shift0/reset0
proposed by \citet{Materzok11}.
%
As we split the steps of CHC generation and CHC solving,
we can use different kinds of solvers as the backend CHC solver depending on benchmarks.
In this experiment, we used two kinds of CHC solvers:
\textsc{Spacer}~\cite{Komuravelli13} that is based on Property Directed Reachability (PDR)\cite{Bradley11,Een11} and CEGAR\cite{Clarke00},
and \textsc{PCSat}~\cite{Unno2021} that is based on template-based CEGIS\cite{Solar-Lezama06,Unno2021} with Z3~\cite{Moura2008} as an SMT solver.

Because inputs to the implementation are OCaml programs
that are type-checked by OCaml's type checker which does not allow ATM,
the underlying OCaml types of the answer types cannot be modified.
However, as remarked before in Section~\ref{sec:intro}, our aim is to verify \emph{existing} programs with algebraic effects and handlers,
and, as remarked before, our ARM, that is only modification in the refinements, is useful for that purpose.

Our implementation supports several kinds of polymorphism.
In addition to the standard let-polymorphism on types,
it supports refinement predicate polymorphism.
The implementation extends the formal system by allowing {\em bounded} predicate polymorphism in which abstracted predicates can be bounded by constraints on them, and further allows predicate-polymorphic types to be assigned to let-bound terms.
However, because the implementation can infer predicate-polymorphic types only at let-bindings,
we used a different approach, which we will discuss in Section~\ref{sec:impl/eval},
to simulate predicate polymorphism in operation signatures.

%
Another notable point is that our implementation deals with operations and exceptions uniformly.
That is, exception raising is treated as an operation invocation
and it can be handled by a certain kind of effect handlers which have clauses for exceptions
(the exception clauses are included in the effect handlers of OCaml by default).

\subsection{Evaluation} \label{sec:impl/eval}

We performed a preliminary experiments to evaluate our method
on some benchmark programs that use algebraic effect handlers.
The benchmarks are based on example programs
from \citet{Bauer15} and the repository of the Eff language~\cite{Effrepo}.
We gathered the effect handlers in those examples
and created benchmark programs each of which uses one of the effect handlers.
We also added a refinement type specification of the main function to each benchmark.
(Other functions are not given such extra information,
and so their types are \emph{inferred automatically} even for recursive functions.)
Also, we added small amount of  annotations to the benchmarks.
Most benchmarks could be solved automatically without the annotations,
but some needs them as hints.
We discuss the details at the end of this section.
We refer to the supplementary material
for the concrete source codes and the specifications of our benchmarks.
Particularly, an interesting one (\texttt{queue-2-SAT.ml}) is explained in detail in Appendix~\ref{sec:benchmark-details}.
All the experiments were conducted on
Intel Xeon Platinum8360Y, 256GB RAM.

\begin{table}
    \caption{Evaluation results}
    \label{tab:eval}
    \footnotesize
    \begin{tabular}{lcrcr}
        \toprule
        \multirow[c]{2}{*}{file name} & \multicolumn{2}{c}{\textsc{Spacer}} & \multicolumn{2}{c}{\textsc{PCSat}} \\
        & result correct? & time (sec.) & result correct? & time (sec.) \\
        \midrule
        \texttt{amb-1-SAT.ml} & Yes & 0.55 & Yes & 15.30 \\
        \texttt{amb-1-UNSAT.ml} & Yes & 0.72 & Yes & 63.62 \\
        \texttt{amb-2-SAT.ml} & Yes & 2.31 & Yes & 31.48 \\
        \texttt{amb-2-UNSAT.ml} & Yes & 2.26 & - & timeout$^\dagger$ \\
        \texttt{amb-3-SAT.ml} & Yes & 3.20 & Yes & 182.41 \\
        \texttt{amb-3-simpl-SAT.ml} & Yes & 1.71 & Yes & 16.79 \\
        \texttt{bfs-SAT.ml} & No$^{*1}$ & 1.67 & - & timeout$^{*1}$ \\
        \texttt{bfs-UNSAT.ml} & Yes & 2.00 & - & timeout$^\dagger$ \\
        \texttt{bfs-simpl-SAT.ml} & No$^{*1}$ & 2.22 & - & timeout$^{*1}$ \\
        \texttt{choose-all-SAT.ml} & Yes & 16.23 & - & timeout$^\dagger$ \\
        \texttt{choose-all-UNSAT.ml} & Yes & 12.56 & - & timeout$^\dagger$ \\
        \texttt{choose-max-SAT.ml} & Yes & 23.08 & - & timeout$^\dagger$ \\
        \texttt{choose-max-UNSAT.ml} & Yes & 15.97 & - & timeout$^\dagger$ \\
        \texttt{choose-sum-SAT.ml} & Yes & 1.54 & - & timeout$^\dagger$ \\
        \texttt{choose-sum-UNSAT.ml} & Yes & 7.99 & Yes & 15.00 \\
        \texttt{deferred-1-SAT.ml} & Yes & 0.46 & Yes & 4.49 \\
        \texttt{deferred-1-UNSAT.ml} & Yes & 0.27 & Yes & 4.09 \\
        \texttt{deferred-2-SAT.ml} & Yes & 0.43 & Yes & 4.38 \\
        \texttt{distribution-SAT.ml} & Abort$^\div$ & - & - & timeout$^{*2}$ \\
        \texttt{distribution-UNSAT.ml} & Abort$^\div$ & - & - & timeout$^{*2}$ \\
        \texttt{expectation-SAT.ml} & Yes & 0.51 & Yes & 7.25 \\
        \texttt{expectation-UNSAT.ml} & Yes & 1.45 & Yes & 7.33 \\
        \texttt{io-read-1-SAT.ml} & Yes & 0.43 & Yes & 13.90 \\
        \texttt{io-read-1-UNSAT.ml} & Yes & 0.41 & Yes & 12.21 \\
        \texttt{io-read-2-SAT.ml} & Yes & 0.56 & Yes & 21.10 \\
        \texttt{io-read-3-SAT.ml} & Yes & 0.54 & Yes & 14.88 \\
        \texttt{io-write-1-SAT.ml} & Yes & 0.32 & Yes & 8.48 \\
        \texttt{io-write-1-UNSAT.ml} & Yes & 0.32 & Yes & 8.76 \\
        \texttt{io-write-2-SAT.ml} & Yes & 0.46 & Yes & 11.33 \\
        \texttt{io-write-2-UNSAT.ml} & Yes & 0.68 & Yes & 11.65 \\
        \texttt{modulus-SAT.ml} & Yes & 14.23 & Yes & 11.89 \\
        \texttt{modulus-UNSAT.ml} & Yes & 26.56 & Yes & 11.91 \\
        \texttt{queue-1-SAT.ml} & Yes & 0.78 & Yes & 19.22 \\
        \texttt{queue-1-UNSAT.ml} & Yes & 0.52 & Yes & 16.93 \\
        \texttt{queue-2-SAT.ml} & Yes & 0.89 & Yes & 22.63 \\
        \texttt{round-robin-SAT.ml} & Yes & 0.96 & - & timeout$^\dagger$ \\
        \texttt{round-robin-UNSAT.ml} & Yes & 0.73 & - & timeout$^\dagger$ \\
        \texttt{safe-div-1-SAT.ml} & Abort$^\div$ & - & Yes & 2.71 \\
        \texttt{safe-div-1-UNSAT.ml} & Abort$^\div$ & - & Yes & 2.73 \\
        \texttt{safe-div-2-SAT.ml} & Abort$^\div$ & - & Yes & 2.55 \\
        \texttt{safe-div-2-UNSAT.ml} & Abort$^\div$ & - & Yes & 3.58 \\
        \texttt{select-SAT.ml} & - & timeout$^{\ddagger}$ & Yes & 13.28 \\
        \texttt{select-UNSAT.ml} & - & timeout$^{\ddagger}$ & Yes & 13.26 \\
        \texttt{shift-SAT.ml} & Yes & 0.28 & Yes & 2.92 \\
        \texttt{shift-UNSAT.ml} & Yes & 1.25 & Yes & 3.93 \\
        \texttt{state-SAT.ml} & - & timeout$^{\ddagger}$ & Yes & 33.69 \\
        \texttt{state-UNSAT.ml} & Yes & 0.63 & Yes & 13.56 \\
        \texttt{state-easy-SAT.ml} & Yes & 0.90 & Yes & 35.54 \\
        \texttt{transaction-SAT.ml} & - & timeout$^{\ddagger}$ & Yes & 15.36 \\
        \texttt{transaction-UNSAT.ml} & - & timeout$^{\ddagger}$ & Yes & 15.77 \\
        \texttt{yield-SAT.ml} & Yes & 1.51 & Yes & 17.57 \\
        \texttt{yield-UNSAT.ml} & Yes & 1.52 & - & timeout$^\dagger$ \\
        \bottomrule
    \end{tabular}
\end{table}

Table~\ref{tab:eval} shows the results of the evaluation.
The files that are suffixed with \texttt{-SAT} are expected to result in ``SAT'',
that is, the programs are expected to be typed
with the refinement types given as their specification.
The other files (suffixed with \texttt{-UNSAT}) are expected to result in ``UNSAT'',
that is, the programs are expected not to be typed
with the refinement types given as their specification.
For each program, we conducted verification in two configurations
((1) \textsc{Spacer}, and (2) \textsc{PCSat}).
The field ``time'' indicates the time spent in the whole process of the verification.
We set the timeout to 600 seconds.
%
Our implementation successfully answered correct result for most programs.
The ones that could not be verified in both configurations are
\texttt{bfs(-simpl)-SAT.ml} (marked with $*1$)
and \texttt{distribution-(UN)SAT.ml} (marked with $*2$).
They need some specific features which the implementation does not support.
The formers need
an invariant which states that there exists an element of a list
that satisfies a certain property.
The latter needs recursive predicates
in the type of an integer list, which states a property about the sum of the elements of the list.
These issues are orthogonal to the main contributions of this paper;
they are about the expressiveness of the background theory used for refinement predicates, to which our novel refinement type system is agnostic.


We discuss pros and cons between the two configurations.
First, \textsc{Spacer} does not support division operator,
and so it cannot verify some programs that use division (marked with $\div$,
aborting with the message ``\texttt{Z3 Error: Uninterpreted 'div' in <null>}'').
Also, there are some programs which can be solved by \textsc{Spacer}
but not solved by \textsc{PCSat} in time, and vice versa.
The formers (marked with $\dagger$) seem due to the huge size of generated constraints,
which can be solved by \textsc{Spacer} but not by \textsc{PCSat}.
The latters (marked with $\ddagger$) can be solved
by \textsc{PCSat} but not by \textsc{Spacer}.

It is worth noting that
our benchmark programs do not rely on refinement type annotation in most places,
even for recursive functions and recursive ADTs.
However, a few kinds of annotations are still needed.
First, as mentioned in Section~\ref{sec:language/discussions},
our type system does not support effect polymorphism.
Therefore, we added effect annotations to function-type arguments
which may perform operations when executed.
These annotations are written in the underlying OCaml types,
that is, we did not specify concrete refinements in the annotations.
Second,
we provided refinement type annotations for two small parts of \texttt{state-SAT.ml},
because otherwise it could not be verified within the timeout period in both configurations.
Third, because our implementation
infers predicate-polymorphic types only at let-bindings,
we added \emph{ghost parameters} to some operations and functions
to
infer precise refinement types of them which are not let-bound
but need some abstraction of refinements.
Ghost parameters are parameters which are used to express dependencies in dependent type checking,
but have no impact on the dynamic execution of the program so they can be removed at runtime.
In automated verification, completely inferring predicate variables requires
higher-order predicate constraints, which are not expressible with CHC.
Therefore, we provided ghost parameter annotations
to make it possible to reduce the verification to CHC constraint solving.
For example, the following code is a part of \texttt{state-SAT.ml}:
\begin{verbatim}
let rec counter c =
    let i = perform (Lookup c) in
    if i = 0 then c else (perform (Update (c, i - 1)); counter (c + 1))
in counter 0
\end{verbatim}
which is handled by a handler that simulates a mutable reference
similar to that of Example 2 in Section~\ref{sec:language/examples/state}.
Here, we pass the variable $c$ to the operation \texttt{Lookup} and \texttt{Update}
as the ghost parameter.
In the formal system presented in Section~\ref{sec:language/type-system}
where predicate polymorphism is available in operation signatures,
we can give \texttt{Update} the type
\begin{align}
    \forall X: (\tyint, \tyint).\,
        (x: \tyint) \rarr (\tyunit &\rarr ((s: \tyint) \rarr \tyrfn{z}{\tyint}{X(z, s)})) \\
    &\rarr ((s: \tyint) \rarr \tyrfn{z}{\tyint}{X(z, x)})
\end{align}
in the same way as Example 2 in Section~\ref{sec:language/examples/state},
and instantiate the predicate variable $X$ with $\lambda (z, s). z = c + 1 + s$
to correctly verify \texttt{state-SAT.ml}.
On the other hand, in the implementation, since predicate polymorphism is not available in operation signatures,
the handler needs to know the concrete predicate which replaces $X$.
However, the predicate contains $c$, which the handler cannot know
without receiving some additional information.
Therefore, we need to add the ghost parameter $c$ to \texttt{Update}
(and the same for \texttt{Lookup}).
This time we added them manually,
but one possible approach for automating insertion of ghost parameters is
to adopt the technique proposed by \citet{Unno13}.
We conjecture that a similar technique can be used
for our purpose.


\section{CPS Transformation} \label{sec:cps}

\subsection{Definitions and Properties} \label{sec:cps/def}

This section presents the crux of our CPS transformation that translate the
language defined in Section~\ref{sec:language} to a $\lambda$-calculus without
effect handlers.
%
Readers interested in the complete definitions of the target language and the
CPS transformation are referred to the supplementary material.

% Figure environment removed

The target language of the CPS transformation is a polymorphic
$\lambda$-calculus with records and recursion.
Its program and type syntax are defined as follows:
%
\[\begin{array}{rcl}
 v &::=& x \mid p \mid \exprec{f:T_1}{x:T_2}{c} \mid \lambda x. c \mid \Lambda \rep{X: \rep{B}}. c \mid \{ \repi{\op_i = v_i} \} \mid \Lambda \alpha. c \\
 c &::=& v \mid c~v \mid \expif{v}{c_1}{c_2} \mid v~\rep{A} \mid v\#\op \mid v~\nmbullet \mid (c : \tau) \\
 \tau &::=& \tyrfn{x}{B}{\phi} \mid (x: \tau_1) \rarr \tau_2 \mid \forall \rep{X: \rep{B}}. \tau
        \mid \{ \repi{\op_i : \tau_i} \} \mid \alpha \mid \forall \alpha. \tau
  \end{array}
\]
%
In the target language, values are not strictly separated from expressions as
those in the source language; for example, functions in function applications
can be expressions.
%
The metavariables $\alpha$ and $\beta$ range over type variables.  Expressions
$\Lambda \alpha. c$ and $v~\nmbullet$ are a type abstraction and application,
respectively. Type arguments in type applications are not given explicitly; they
are assumed to be inferred during the typechecking.
%
Type polymorphism is introduced to express the pure control effect in the target
language using \emph{answer type polymorphism}~\cite{Thielecke03}.
%
Expressions $\{ \repi{\op_i = v_i} \}$ and $v\#\op$ are a record literal and
projection, respectively.
%
We use operation names as record labels for the target language to encode
handlers using records.
%
Our CPS transformation produces programs with type annotations for proving
bidirectional type-preservation. Recursive functions with type annotations and
type ascriptions $(c : \tau)$ are used to annotate programs.
%
We abbreviate $\exprec{f:T_1}{x:T_2}{c}$ to $\lambda x: T_2. c$ if $f$ does not
occur in $c$.
%
Types are defined in a standard manner.
%
Typing contexts $\Gamma$ are extended to include type variables.  The type
system is also standard, except for the typing rule for type applications, which
is presented in Figure~\ref{fig:subty-cps-excerpt}.  We write $\jdwf{\Gamma}{\tau}$ to
state that all the free variables (including type and predicate variables) in
the type $\tau$ are bound in the typing context $\Gamma$.
%
The subtyping for record types allows supertypes to forget some fields in
subtypes, and the types of each corresponding field in two record types to be in
the subtyping relation (we deem record types, as well as records, to be
equivalent up to permutation of fields).
%
The subtyping rule for type polymorphism is a weaker variant of the containment
rule for polymorphic types~\cite{Mitchell88}.
%
It is introduced to emulate \rulename{S-Embed} in the target language.

% Figure environment removed

We show the key part of the CPS transformation in
Figure~\ref{fig:cps-trans-excerpt}.
%
The upper half presents the transformation of types.
%
The transformation of value types is straightforward.  Operation signatures are
transformed into record types, which means that operation clauses in a handler
are transformed into a record.
%
The transformations of computation types indicate that computations are
transformed into functions that receive two value parameters: handlers and
continuations.
%
If the control effect is pure, the answer types of computations become
polymorphic in CPS.
%
This treatment of control effects is different from that of \citet{Materzok11},
who define CPS transformation for control effects in the
simply typed setting.
%
Their CPS transformation transforms, in our notation,
%
a computation type $\tycompMB{T}{\square}$ (note that their computation types do
not involve operation signatures) into the type $\cps{T}$, and
%
a type $\tycompMB{T}{\tyctlMB{C_1}{C_2}}$ (again note that they do not consider
dependent typing) into the type $\tyfunshort{(\tyfunshort{\cps{T}}{\cps{C_1}})}{\cps{C_2}}$.
%
Because the latter takes continuations whereas the former does not, CPS
transformation needs to know where pure computations are converted into impure
ones (via subtyping).
%
To address this issue, Materzok and Biernacki's CPS transformation focuses on
typing derivations in the source language rather than expressions.
%
However, because our aim is at reducing the typing of programs with
algebraic effects and handlers to that of programs without them,
we cannot assume typing derivations in the source language to be available.
%
By treating two kinds of control effects uniformly using answer type
polymorphism, our CPS transform can focus only on expressions (with type
annotations).

The lower half of Figure~\ref{fig:cps-trans-excerpt} shows the key cases of the
transformation of expressions.
%
For backward type-preservation (Theorem~\ref{thm:cps-backward-excerpt}), we
extend the source language with type annotations.
%
For example, in an operation call $\expop[\op^{\rep{\mathit{A}}}]{v}{y^{T_y}}{c^{\tycomp{\Sigma}{T}{\tyctl{z}{C_1}{C_2}}}}$,
$\rep{\mathit{A}}$ are predicates used to instantiate the type scheme of the operation $\op$,
$T_y$ is the output type of $\op$, and
$\tycomp{\Sigma}{T}{\tyctl{z}{C_1}{C_2}}$ is the type of the continuation $c$.
%
Without type annotations, CPS-transformed expressions may have a type that
cannot be transformed back to a type in the source language.
%
An operation call $\expop[\op^{\rep{\mathit{A}}}]{v}{y^{T_y}}{c^{\tycomp{\Sigma}{T}{\tyctl{z}{C_1}{C_2}}}}$
is transformed into a function that seeks the corresponding
operation clause in a given handler and then applies it to a given sequence of
predicates, argument, and continuation.
%
An expression $\expwith{h}{c}$ is transformed into a function that applies the
CPS-transformed handled computation to the record of the CPS-transformed
operation clauses and the CPS-transformed return clause (because the return
clause works as the continuation of $c$).

Now, we state forward and backward type-preservation of the CPS transformation.
%
\begin{theorem}[Forward type-preservation] \label{thm:cps-forward-excerpt}
    \hspace*{10pt}
    \begin{itemize}
        \item If\, $\jdty{\Gamma}{v}{T}$ then $\jdty{\cps{\Gamma}}{\cps{v}}{\cps{T}}$.
        \item If\, $\jdty{\Gamma}{c}{C}$ then $\jdty{\cps{\Gamma}}{\cps{c}}{\cps{C}}$.
    \end{itemize}
\end{theorem}

\begin{theorem}[Backward type-preservation] \label{thm:cps-backward-excerpt}
    \hspace*{10pt}
    \begin{itemize}
     \item If\, $\jdty{\emptyset}{\cps{v}}{\tau}$, then
           there exists some $T$ such that
           $\jdty{\emptyset}{v}{T}$ and
           $\jdsub{\emptyset}{\cps{T}}{\tau}$.
     \item If\, $\jdty{\emptyset}{\cps{c}}{\tau}$, then
           there exists some $C$ such that
           $\jdty{\emptyset}{c}{C}$ and
           $\jdsub{\emptyset}{\cps{C}}{\tau}$.
    \end{itemize}
\end{theorem}
%
\noindent
Theorem~\ref{thm:cps-backward-excerpt} is implied immediately by backward type
preservation of the CPS transformation for \emph{open} expressions.
See the supplementary material for the statement for open expressions.
%
Theorem~\ref{thm:cps-backward-excerpt} indicates that it is possible to reduce
typechecking in our source language to that in a language without effect
handlers.
%
That is, if ones want to verify whether an expression $c$ has type $C$,
they can obtain the same result as the direct verification
by first applying CPS transformation to $c$ and $C$, and
then checking whether $\cps{c}$ has type $\cps{C}$
with a refinement type verification tool that does not support algebraic effect handlers.

Type annotations in the source language are necessary to restrict the image of the transformation.
Without them, a CPS-transformed program may be of a type $\tau$ that cannot be transformed to a type in the source language inversely (i.e., there exists no type $C$ in the source language satisfying $\cps{C} = \tau$).
For example, consider $\Lambda \alpha. \lambda h. \lambda k. k~0$,
the CPS form (without annotations) of expression $\expret{0}$.
Without annotations, we can pick arbitrary types as the type of $h$.
Therefore, it can have type
$\forall \alpha. \tybool \rarr (\tyint \rarr \alpha) \rarr \alpha$.
%
However, there is no type $C$ in the source language such that $\cps{C} = \forall \alpha. \tybool \rarr (\tyint \rarr \alpha) \rarr \alpha$.
Even worse, the source language has no type that is a \emph{subtype} of the type of the CPS form
since $\tybool$ and record types are incomparable with each other.
Another example is $\lambda x. \Lambda \alpha. \lambda h. \lambda k. k~x$,
the CPS form (again, without annotations) of expression $\lambda x. \expret{x}$.
Its type can be
$(\tyint \rarr \tyint) \rarr \forall \alpha. \{\} \rarr ((\tyint \rarr \tyint) \rarr \alpha) \rarr \alpha$,
that is, $x$ can be of type $\tyint \rarr \tyint$.
However, there is no value type $T$ in the source language
such that $\cps{T}$ is a subtype of $\tyint \rarr \tyint$.
Note that since a function type in the source language is in the form $(x: T_x) \rarr C$,
the right hand side of the arrow in the CPS-transformed function type must be in the form
$\forall \alpha. \{ \cdots \} \rarr \cdots$, which does not match with $\tyint$.
%
Therefore, without type annotations, Theorem~\ref{thm:cps-backward-excerpt} does not hold.

\newcommand{\er}{\mathit{er}}

While our formalization requires concrete refinement type annotations in the source language,
actually we can relax this restriction
by using predicate variables as placeholders instead of concrete refinements in type annotations.
This is because type annotations are only
for prohibiting occurrences of types with unintended \emph{structures}, not for restricting refinements.
Those predicate variables are instantiated after CPS transformation
with concrete predicates inferred by generating and solving CHC constraints
that contains these predicate variables from the CPS-transformed expression.
%
Formally, by allowing occurrences of predicate variables
in type annotations of both the source and target language,
and introducing predicate variable substitution $\sigma$,
we can state that $\cps{\sigma(c)} = \sigma(\cps{c})$.
This means that,
for an expression $c$ that is annotated with types containing predicate variables,
both of the followings result in the same expression:
(1) first instantiating the predicate variables in $c$ with concrete refinements,
and then CPS-transforming it (i.e., CPS-transforming the concretely-annotated expression),
and
(2) first CPS-transforming $c$,
and then instantiating the predicate variables in the CPS-transformed expression
with the concrete refinements.
In other words, concrete refinements are irrelevant to the CPS transformation.
This irrelevance is ensured by the fact that
refinements can depend only on first-order values because it means that handler variables $h$ and continuation variables $k$,
which occur only in CPS-transformed expressions, cannot be used in instantiated refinements.
% }
The reason why we have defined the CPS transformation with concrete refinements
is just to state Theorem~\ref{thm:cps-forward-excerpt} and Theorem~\ref{thm:cps-backward-excerpt}.

We do not address whether our CPS transformation is semantics-preserving in this paper
since our CPS transformation is intended to be used only for type checking.
However, the transformation is inspired by that of \citet{Hillerstrom17},
which satisfies the preservation of the dynamic semantics.
The differences between our and their CPS transformation are not so radical,
and therefore we believe that the preservation of the dynamic semantics holds too in our CPS transformation.


\subsection{Comparison between the direct verification and the indirect verification} \label{sec:cps/comparison}

In this section, We compare the direct verification using our refinement system
presented in Section~\ref{sec:language}
with the indirect verification via CPS transformation presented above.
One of the differences is that the direct verification requires
special support of verification tools for algebraic effect handlers,
while the indirect one can be done by existing tools without such support.
On the other hand, the indirect verification has some disadvantages.
%
First, in most cases, CPS-transformed programs tend to be complicated and be in
the forms quite different from the source programs.  This complexity incurred in
the indirect typechecking may lead to confusing error messages when the
typechecking fails.
Transforming the inferred complex types back to the types of the source language would be helpful,
but it is unclear whether we can do this
because the inferred types of the CPS-transformed expressions do not necessarily correspond to
the CPS-transformed types of the source expressions,
as stated in Section~\ref{sec:cps/def}.
By contrast, because the direct typechecking deals with the
structures of the source programs as they are, error messages can be made more
user-friendly.
%
Second, our CPS transformation needs a non-negligible amount of type
annotations---type annotations are necessary in let-expressions, conditional
branches, and recursive functions as well as operation calls and handling
constructs.  In practice, it is desired to infer as many types as possible.
However, it seems quite challenging to define a CPS transformation that
enjoys backward type-preservation and needs no, or few, type annotations.
One of the possible approaches for addressing type annotations in more automated way is
to use the underlying simple type system of our refinement type system
for algebraic effect handlers.
As mentioned in Section~\ref{sec:cps/def},
concrete refinements are not necessary for type annotations.
Therefore, we can generate type annotations for an expression
using its simple type inferred by the underlying type system.

We also compare these two approaches based on an experiment.
We used some direct style (DS) programs
(i.e., programs using algebraic effect handlers),
and for each program, we applied our CPS transformation manually,
and run the verification on both DS one and CPS one.
Additionally, we also compared them with optimized CPS programs
where administrative redexes were reduced.
We used the same implementation as the one in Section~\ref{sec:impl} with the configuration of \textsc{Spacer}.
We added annotations of source programs to only top-level expressions,
but the correctness of the verification can be justified by the preservation of dynamic semantics
which we believe to hold as described in Section~\ref{sec:cps/def}.

\begin{table}
    \caption{Evaluation results of CPS transformation}
    \label{tab:cps}
    \begin{tabular}{lcrcrcr}
        \toprule
        \multirow[c]{2}{*}{program} & \multicolumn{2}{c}{DS} & \multicolumn{2}{c}{CPS} & \multicolumn{2}{c}{CPS (optimized)} \\
        & verified? & time (sec.) & verified? & time (sec.) & verified? & time (sec.) \\
        \midrule
        \texttt{amb-2} &Yes & 1.30 & Yes & 1.32 & Yes & 0.91 \\
        \texttt{choose-easy} &Yes & 0.26 & Yes & 0.27 & Yes & 0.22 \\
        \texttt{choose-sum} &Yes & 2.18 & Yes & 1.79 & Yes & 12.87 \\
        \texttt{io-read-2} &Yes & 0.66 & No & 1.29 & No & 0.62 \\
        \texttt{simple} &Yes & 0.11 & Yes & 0.16 & Yes & 0.14 \\
        \bottomrule
    \end{tabular}
\end{table}

Table~\ref{tab:cps} shows the results of the experiment.
Some programs have no big difference in execution time
among the three variants,
but there are two notable things.
First, optimized CPS version of \texttt{choose-sum} took more time than the other versions.
This seems because the size of the program became larger by the optimization.
The DS \texttt{choose-sum} program contains some branching expressions
and each branch uses variables representing its continuation and the outer handler.
By reducing administrative redexes in the program, these variables are instantiated
with concrete continuation and handler,
that is, the continuation and handler are copied to each branch,
which results in larger size of the program and its constraints generated during the verification.
Second, CPS version of \texttt{io-read-2} could not be verified correctly.
One possible reason is lack of support for higher-order predicate polymorphism.
Since CPS programs explicitly pass around continuations,
their types tend to be higher-order.
Then, in some cases, higher-order predicate polymorphism becomes necessary by CPS transformation.

\endinput


\section{Related Work}
\label{appsec: related work}
Bayesian causal discovery literature has primarily focused on inference in linear models with closed-form posteriors or marginalized parameters. Early works considered sampling directed acyclic graphs (DAGs) for discrete~\cite{cooper1992bayesian, madigan1995bayesian, heckerman2006bayesian} and Gaussian random variables~\cite{friedman2003being, tong2001active} using Markov chain Monte Carlo (MCMC) in the DAG space. However, these approaches exhibit slow mixing and convergence~\cite{eaton2012bayesian,grzegorczyk2008improving}, often requiring restrictions on number of parents~\cite{kuipers2017partition}. %Alternative exact dynamic programming methods are limited to small settings~\cite{koivisto2012advances}. 

Recent advances in variational inference~\cite{zhang2018advances} have facilitated graph inference in DAG space, with gradient-based methods employing the NOTEARS DAG penalty \cite{zheng2018dags}.\cite{annadani2021variational} samples DAGs from autoregressive adjacency matrix distributions, while \cite{lorch2021dibs} utilizes Stein variational approach \cite{liu2016stein} for DAGs and causal model parameters. \cite{cundy2021bcd} proposed a variational inference framework on node orderings using the gumbel-sinkhorn gradient estimator \cite{mena2018learning}. \cite{deleu2022bayesian,nishikawa2022bayesian} employ the GFlowNet framework \cite{bengio2021gflownet} for inferring the DAG posterior. Most methods, except\cite{lorch2021dibs} are restricted to linear models, while \cite{lorch2021dibs} has high computational costs and lacks DAG generation guarantees compared to our method.
% at least quadratic scaling complexity, both with respect to the number of nodes (due to the DAG penalty) as well as number of posterior samples. Our proposed approach instead has linear complexity with respect to number of posterior samples and does not require any additional DAG penalty.     

In contrast, \emph{quasi-Bayesian} methods, such as DAG bootstrap \cite{friedman2013data}, demonstrate competitive performance. DAG bootstrap resamples data and estimates a single DAG using PC \cite{spirtes2000causation}, GES \cite{chickering2002optimal}, or similar algorithms, weighting the obtained DAGs by their unnormalized posterior probabilities. Recent neural network-based works employ variational inference to learn DAG distributions and point estimates for nonlinear model parameters \cite{charpentier2022differentiable,geffner2022deep}.

\section{Conclusion and Future Work}
In this work, I design corruption-robust algorithms for the Lipschitz contextual search problem. I present the \emph{agnostic checking} technique and demonstrate its effectiveness in designing corruption-robust algorithms. There are several open problems for future research. First, in the algorithm I propose for pricing loss, the schedule for agnostic checks is fixed upfront. Can the learner design an adaptive checking schedule for the pricing loss? Second, this work assumes the learner has knowledge of the Lipschitz constant $L$. Can the learner design efficient no-regret algorithms without knowledge of $L$? 

\section*{Data-Availability Statement}
Our artifact is available in the GitHub repository, at \url{https://github.com/hiroshi-unno/coar}.
The experimental results shown in Table~\ref{tab:eval} and Table~\ref{tab:cps}
can be reproduced by following the instructions in \texttt{popl24ae/README.md} of the repository.

\begin{acks}
We are grateful to anonymous reviewers for their helpful and useful comments on the paper,
especially regarding its presentation.
We also thank Yiyang Guo and Kanaru Isoda for their contribution to our implementation.
%
This work was supported by \grantsponsor{JSPS}{JSPS}{} KAKENHI Grant Numbers
\grantnum{JSPS}{JP19K20247}, % Sekiyama Kakenhi 1
\grantnum{JSPS}{JP22K17875}, % Sekiyama Kakenhi 2
\grantnum{JSPS}{JP20H00582}, % Igarashi Kakenhi
\grantnum{JSPS}{JP20H04162}, % Unno Kaken
\grantnum{JSPS}{JP22H03564}, % Tsukada Kaken
\grantnum{JSPS}{JP20H05703}, % Kobayashi Kaken
\grantnum{JSPS}{JP20K20625}, % Terauchi Kaken
and
\grantnum{JSPS}{JP22H03570}  % Terauchi Kaken
as well as
\grantsponsor{JST}{JST}{} CREST Grant Number \grantnum{JST}{JPMJCR21M3}.
\end{acks}

\bibliographystyle{ACM-Reference-Format}
\bibliography{main}

% \begin{comment}
\section{System Architecture}
\label{appendix:architecture}
\system has a novel modularized system architecture with three key components: 
\emph{StreamManager}, 
\emph{TxnManager} and \emph{TxnScheduler}. 
These components are instantiated in each thread locally.
The execution outline of \system is presented in Algorithm~\ref{alg:algo}.
Transactional stream processing is continuous and potentially never ends (Line 1$\sim$8).
The dependency resolution and execution of state transactions are separated into two non-overlapping phases by punctuations~\cite{Tucker:2003:EPS:776752.776780} (Line 2 and 5), which guarantees that no subsequent input event will have a smaller timestamp. 
Effectively, a batch of state transactions is collected during the first phase, and processed during the second phase.

In the first phase (i.e., stream processing phase), 
the \emph{StreamManager} conducts preprocessing for every input event ($e$). Similar to some prior works~\cite{tstream}, state transactions may be issued but not immediately processed during preprocessing (Line 3).
The \emph{pre\_processing} and \emph{post\_processing} functions are exposed as APIs to users.
The \emph{TxnManager} handles dependency resolution (Line 4) among state transactions and insert decomposed operations to construct a \tpg. We discuss the detailed two-phase \tpg construction process in Section~\ref{subsec:construction}.

In the second phase  (i.e., transaction processing phase), 
the \emph{TxnManager} is first involved again to refine (Line 6) the constructed \tpg with further dependency resolution.
The \emph{TxnScheduler} 
schedules operations for concurrent execution based on the constructed \tpg according to the three dimensions of scheduling decisions (Line 7). 
In particular, a scheduling decision model $M$ is instantiated based on the constructed \tpg (Line 14).
\textbf{\circled{1}} Guided by $M$, execution threads adopt an exploration strategy (Section~\ref{subsec:explore}) to explore the constructed \tpg for operations available to be scheduled constrained by dependencies. 
\textbf{\circled{2}} 
During exploration, one or multiple operations may be treated as the 
% basic 
unit of scheduling (Section~\ref{subsec:granularity}). 
Subsequently, \textbf{\circled{3}} every thread executes operation(s) in the unit of scheduling with various abort handling mechanisms (Section~\ref{subsec:abort_handling}).
Only when state transactions are processed (i.e., committed or aborted) can the associated input events be postprocessed (Line 8) by the \emph{StreamManager} based on transaction processing results.
\end{comment}

\begin{comment}
\begin{algorithm}
\footnotesize
    \KwData{$e$ \tcp{Input event}}
    \KwData{$txn_{ts}$ \tcp{State transaction}}
    \KwData{$G$ \tcp{The currently constructed TPG}}
    \While{!finish processing of input streams}{
        \eIf(\tcp*[h]{Phase 1}){\text{$e$ is not a $punctuation$}}{
                $txn_{ts}$ $\gets$ PRE\_Processing($e$)\;
                \textbf{TPG\_Construction}($G$, $txn_{ts}$)\; 
          }(\tcp*[h]{Phase 2}){
                \textbf{TPG\_Refinement}($G$)\; 
                \textbf{TXN\_Scheduling}($G$)\; 
                POST\_Processing()\;
          }
    }
    
    \SetKwFunction{FMain}{TPG\_Construction}
    \SetKwProg{Fn}{Function}{:}{}
    \Fn{\FMain{$G$, $txn_{ts}$}}{
        $O_{1..k}$ $\gets$ \textbf{Partition} $txn_{ts}$\;
        \ForEach{\text{operation $O_{i}$ $\in$ $O_{1..k}$}}{
            \textbf{Identify} its \ld\;
            $G$ $\gets$ $G$ + $O_{i}$ \;
        }
    }
    \SetKwFunction{FMain}{TPG\_Refinement}
    \SetKwProg{Fn}{Function}{:}{}
    \Fn{\FMain{$G$}}{
        \ForEach{\text{vertex $e_{i}$ $\in$ $G$}}{
            \textbf{Identify} its \td, \pd\;
        }
    }
    
    \SetKwFunction{FMain}{TXN\_Scheduling}
    \SetKwProg{Fn}{Function}{:}{}
    \Fn{\FMain{$G$}}{
        $M$ $\gets$ Instantiated with $G$;\tcp{A decision model}
        \While{!finish scheduling of $G$
        }{
          \textbf{\circled{2}} $Scheduling Unit$ $\gets$ \textbf{\circled{1}} \emph{Explore}($G$, $M$)\; 
            \textbf{\circled{3}} \emph{Execute with Abort Handling} ($Scheduling Unit$)\; 
        }
    }
  \caption{Execution Outline of \system}
  \label{alg:algo}
\end{algorithm}
\end{comment}

\end{document}
\endinput
