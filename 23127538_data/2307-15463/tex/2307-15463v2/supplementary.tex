\pdfoutput=1
\documentclass[a4paper]{article}

\usepackage[margin=20truemm]{geometry}

\usepackage{amsmath,amssymb,stmaryrd}
\usepackage[inline]{enumitem}

\renewcommand*{\labelitemii}{$\circ$}

\usepackage{natbib}
\usepackage{hyperref}

\usepackage{proof}
\usepackage{mathtools}
\mathtoolsset{showonlyrefs=true}

\usepackage{thelanguage}
\newcommand*{\defeq}{\stackrel{\text{def}}{=}}
\newcommand{\rulename}[1]{({\mdseries\textsc{#1}})}
\newcommand{\infersc}[3][]{\infer[\!\!\text{\rulename{#1}}]{#2}{#3}}

\usepackage{amsthm, thmtools}
\declaretheorem[style=plain, numberwithin=section]{theorem}
\declaretheorem[style=plain, numberwithin=section, sibling=theorem]{lemma, corollary}
\declaretheorem[style=definition, numberwithin=section, sibling=theorem]{definition, assumption}

\usepackage{color, xcolor}
\newcommand{\todo}[1][]{\textcolor{red}{TODO: #1}}


\newcommand{\llabel}[1]{\label{\currentprefix:#1}}
\newcommand{\lref}[2][\currentprefix]{\ref{#1:#2}}

\newlist{enumrm}{enumerate}{3}
\setlist[enumrm]{label=(\roman*)}
\newlist{enumit}{enumerate}{3}
\setlist[enumit]{label=\arabic*., font=\itshape}

\begin{document}

\tableofcontents

\listoftheorems[swapnumber]

\begin{comment}
\section{System Architecture}
\label{appendix:architecture}
\system has a novel modularized system architecture with three key components: 
\emph{StreamManager}, 
\emph{TxnManager} and \emph{TxnScheduler}. 
These components are instantiated in each thread locally.
The execution outline of \system is presented in Algorithm~\ref{alg:algo}.
Transactional stream processing is continuous and potentially never ends (Line 1$\sim$8).
The dependency resolution and execution of state transactions are separated into two non-overlapping phases by punctuations~\cite{Tucker:2003:EPS:776752.776780} (Line 2 and 5), which guarantees that no subsequent input event will have a smaller timestamp. 
Effectively, a batch of state transactions is collected during the first phase, and processed during the second phase.

In the first phase (i.e., stream processing phase), 
the \emph{StreamManager} conducts preprocessing for every input event ($e$). Similar to some prior works~\cite{tstream}, state transactions may be issued but not immediately processed during preprocessing (Line 3).
The \emph{pre\_processing} and \emph{post\_processing} functions are exposed as APIs to users.
The \emph{TxnManager} handles dependency resolution (Line 4) among state transactions and insert decomposed operations to construct a \tpg. We discuss the detailed two-phase \tpg construction process in Section~\ref{subsec:construction}.

In the second phase  (i.e., transaction processing phase), 
the \emph{TxnManager} is first involved again to refine (Line 6) the constructed \tpg with further dependency resolution.
The \emph{TxnScheduler} 
schedules operations for concurrent execution based on the constructed \tpg according to the three dimensions of scheduling decisions (Line 7). 
In particular, a scheduling decision model $M$ is instantiated based on the constructed \tpg (Line 14).
\textbf{\circled{1}} Guided by $M$, execution threads adopt an exploration strategy (Section~\ref{subsec:explore}) to explore the constructed \tpg for operations available to be scheduled constrained by dependencies. 
\textbf{\circled{2}} 
During exploration, one or multiple operations may be treated as the 
% basic 
unit of scheduling (Section~\ref{subsec:granularity}). 
Subsequently, \textbf{\circled{3}} every thread executes operation(s) in the unit of scheduling with various abort handling mechanisms (Section~\ref{subsec:abort_handling}).
Only when state transactions are processed (i.e., committed or aborted) can the associated input events be postprocessed (Line 8) by the \emph{StreamManager} based on transaction processing results.
\end{comment}

\begin{comment}
\begin{algorithm}
\footnotesize
    \KwData{$e$ \tcp{Input event}}
    \KwData{$txn_{ts}$ \tcp{State transaction}}
    \KwData{$G$ \tcp{The currently constructed TPG}}
    \While{!finish processing of input streams}{
        \eIf(\tcp*[h]{Phase 1}){\text{$e$ is not a $punctuation$}}{
                $txn_{ts}$ $\gets$ PRE\_Processing($e$)\;
                \textbf{TPG\_Construction}($G$, $txn_{ts}$)\; 
          }(\tcp*[h]{Phase 2}){
                \textbf{TPG\_Refinement}($G$)\; 
                \textbf{TXN\_Scheduling}($G$)\; 
                POST\_Processing()\;
          }
    }
    
    \SetKwFunction{FMain}{TPG\_Construction}
    \SetKwProg{Fn}{Function}{:}{}
    \Fn{\FMain{$G$, $txn_{ts}$}}{
        $O_{1..k}$ $\gets$ \textbf{Partition} $txn_{ts}$\;
        \ForEach{\text{operation $O_{i}$ $\in$ $O_{1..k}$}}{
            \textbf{Identify} its \ld\;
            $G$ $\gets$ $G$ + $O_{i}$ \;
        }
    }
    \SetKwFunction{FMain}{TPG\_Refinement}
    \SetKwProg{Fn}{Function}{:}{}
    \Fn{\FMain{$G$}}{
        \ForEach{\text{vertex $e_{i}$ $\in$ $G$}}{
            \textbf{Identify} its \td, \pd\;
        }
    }
    
    \SetKwFunction{FMain}{TXN\_Scheduling}
    \SetKwProg{Fn}{Function}{:}{}
    \Fn{\FMain{$G$}}{
        $M$ $\gets$ Instantiated with $G$;\tcp{A decision model}
        \While{!finish scheduling of $G$
        }{
          \textbf{\circled{2}} $Scheduling Unit$ $\gets$ \textbf{\circled{1}} \emph{Explore}($G$, $M$)\; 
            \textbf{\circled{3}} \emph{Execute with Abort Handling} ($Scheduling Unit$)\; 
        }
    }
  \caption{Execution Outline of \system}
  \label{alg:algo}
\end{algorithm}
\end{comment}

\section{Definitions (other than those shown in the main paper) and Assumptions}

\subsection{Well-formedness of typing contexts, value types, and computation types}

\fbox{$\jdwf{}{\Gamma}$} \quad
\fbox{$\jdwf{\Gamma}{T}$} \quad \fbox{$\jdwf{\Gamma}{C}$} \quad
\fbox{$\jdwf{\Gamma}{\Sigma}$} \quad \fbox{$\jdwf{\Gamma \mid T}{S}$}
\begin{gather}
    \infersc[WE-Empty]{\jdwf{}{\emptyset}}
    {}
    \quad
    \infersc[WE-Var]{\jdwf{}{\Gamma, x: T}}
    {
        \jdwf{}{\Gamma} &
        x \notin \dom(\Gamma) &
        \jdwf{\Gamma}{T}
    }
    \quad
    \infersc[WE-PVar]{\jdwf{}{\Gamma, X: \rep{B}}}
    {
        \jdwf{}{\Gamma} &
        X \notin \dom(\Gamma)
    }
    \\
    \infersc[WT-Rfn]{\jdwf{\Gamma}{\tyrfn{x}{B}{\phi}}}
    {\Gamma, x: B \vdash \phi}
    \quad
    \infersc[WT-Fun]{\jdwf{\Gamma}{(x: T) \rarr C}}
    {
        \jdwf{\Gamma, x: T}{C}
    }
    \\
    \infersc[WT-Comp]{\jdwf{\Gamma}{\tycomp{\Sigma}{T}{S}}}
    {
        \jdwf{\Gamma}{\Sigma} &
        \jdwf{\Gamma}{T} &
        \jdwf{\Gamma \mid T}{S}
    }
    \qquad
    \infersc[WT-Sig]{\jdwf{\Gamma}{\{ \repi{\op_i : \forall \rep{X_i: \rep{B}_i}. F_i} \}}}
    {\repi{\jdwf{\Gamma, \rep{X_i: \rep{B}_i}}{F_i}}}
    \\
    \infersc[WT-Pure]{\jdwf{\Gamma \mid T}{\square}}
    {\jdwf{}{\Gamma}}
    \quad
    \infersc[WT-ATM]{\jdwf{\Gamma \mid T}{\tyctl{x}{C_1}{C_2}}}
    {
        \jdwf{\Gamma, x: T}{C_1} &
        \jdwf{\Gamma}{C_2}
    }
\end{gather}


\subsection{Assumptions on well-formedness judgments of formulas, well-formedness judgments of predicates, and semantic validity judgements of formulas}


\begin{assumption} \label{asm:formla} \quad
    \begin{itemize}
        \item If $\jdwf{\Gamma}{\phi}$, then $\jdwf{}{\Gamma}$.
        \item If $\jdwf{}{\Gamma}$, $z \notin \dom(\Gamma)$ and $\dom(\Gamma, z: B) \supseteq \fv(\phi)$, then $\jdwf{\Gamma, z: B}{\phi}$.
        % \item If $\dom(\Gamma) \supseteq \fv(t)$, then $\jdwf{\Gamma}{t = t}$.
        \item If $\jdwf{}{\Gamma, x: T, \Gamma'}$ and $\jdty{\Gamma, \Gamma'}{A}{\rep{B}}$, then $\jdty{\Gamma, x: T, \Gamma'}{A}{\rep{B}}$.
        \item If $\jdwf{}{\Gamma, x: T, \Gamma'}$ and $\jdwf{\Gamma, \Gamma'}{\phi}$, then $\jdwf{\Gamma, x: T, \Gamma'}{\phi}$.
        \item If $\valid{\Gamma, \Gamma'}{\phi}$, then $\valid{\Gamma, x: T, \Gamma'}{\phi}$.
        \item If $\jdty{\Gamma}{v}{T}$ and $\jdty{\Gamma, x: T, \Gamma'}{A}{\rep{B}}$, then $\jdty{\Gamma, \Gamma'[v/x]}{A[v/x]}{\rep{B}}$.
        \item If $\jdty{\Gamma}{v}{T}$ and $\jdwf{\Gamma, x: T, \Gamma'}{\phi}$, then $\jdwf{\Gamma, \Gamma'[v/x]}{\phi[v/x]}$.
        \item If $\jdty{\Gamma}{v}{T}$ and $\valid{\Gamma, x: T, \Gamma'}{\phi}$, then $\valid{\Gamma, \Gamma'[v/x]}{\phi[v/x]}$.
        \item If $\jdty{\Gamma}{A}{\rep{B}}$ and $\jdty{\Gamma, X: \rep{B}, \Gamma'}{A'}{\rep{B'}}$, then $\jdty{\Gamma, \Gamma'[A/X]}{A'[A/X]}{\rep{B'}}$.
        \item If $\jdty{\Gamma}{A}{\rep{B}}$ and $\jdwf{\Gamma, X: \rep{B}, \Gamma'}{\phi}$, then $\jdwf{\Gamma, \Gamma'[A/X]}{\phi[A/X]}$.
        \item If $\jdty{\Gamma}{A}{\rep{B}}$ and $\valid{\Gamma, X: \rep{B}, \Gamma'}{\phi}$, then $\valid{\Gamma, \Gamma'[A/X]}{\phi[A/X]}$.
        \item If $\jdsub{\Gamma}{T_1}{T_2}$, $\jdwf{}{\Gamma, x:T_1, \Gamma'}$ and $\jdty{\Gamma, x:T_2, \Gamma'}{A}{\rep{B}}$, then $\jdty{\Gamma, x:T_1, \Gamma'}{A}{\rep{B}}$.
        \item If $\jdsub{\Gamma}{T_1}{T_2}$, $\jdwf{}{\Gamma, x:T_1, \Gamma'}$ and $\jdwf{\Gamma, x:T_2, \Gamma'}{\phi}$, then $\jdwf{\Gamma, x:T_1, \Gamma'}{\phi}$.
        \item If $\jdsub{\Gamma}{T_1}{T_2}$ and $\valid{\Gamma, x:T_2, \Gamma'}{\phi}$, then $\valid{\Gamma, x:T_1, \Gamma'}{\phi}$.
        \item If $x \notin \fv(\Gamma', \phi)$ and $\jdwf{\Gamma, x: T_0, \Gamma'}{\phi}$, then $\jdwf{\Gamma, \Gamma'}{\phi}$.
        \item If $\jdwf{\Gamma, x: (y: T_1) \rarr C_1, \Gamma'}{\phi}$, then $x \notin \fv(\Gamma', \phi)$.
        % \item If $\vDash \phi$ and $\jdty{\Gamma, \phi, \Gamma'}{A}{\rep{B}}$, then $\jdty{\Gamma, \Gamma'}{A}{\rep{B}}$.
        % \item If $\vDash \phi$ and $\jdwf{\Gamma, \phi, \Gamma'}{\phi'}$, then $\jdwf{\Gamma, \Gamma'}{\phi'}$.
        \item If $\vDash \phi$ and $\valid{\Gamma, \phi, \Gamma'}{\phi'}$, then $\valid{\Gamma, \Gamma'}{\phi'}$.
        \item If $\jdwf{\Gamma}{\phi}$, then $\Gamma \vDash \phi \Rarr \phi$.
        \item If $\Gamma \vDash \phi_1 \Rarr \phi_2$ and $\Gamma \vDash \phi_2 \Rarr \phi_3$, then $\Gamma \vDash \phi_1 \Rarr \phi_3$.
        \item If $\jdwf{\Gamma, x: \tyrfn{z}{B}{z = y}, \Gamma'}{\phi}$, then $\Gamma, x: \tyrfn{z}{B}{z = y}, \Gamma' \vDash \phi \implies \phi[y/x]$~.
        \item If $\jdwf{\Gamma, x: \tyrfn{z}{B}{z = y}, \Gamma'}{\phi}$, then $\Gamma, x: \tyrfn{z}{B}{z = y}, \Gamma' \vDash \phi[y/x] \implies \phi$~.
    \end{itemize}
\end{assumption}

\subsection{Assumptions on primitives}

\begin{assumption} \label{asm:prim} \quad
    \begin{itemize}
        \item $\jdwf{}{\ty(p)}$ for all $p$.
        \item If $\ty(p) = (x: T) \rarr C$, then
            $\zeta(p, v)$ is defined and $\jdty{}{\zeta(p, v)}{C[v/x]}$
            for all $v$ such that $\jdty{}{v}{T}$.
        \item If $\ty(p) = \tyrfn{z}{\tybool}{\phi}$, then $p = \exptrue$ or $p = \expfalse$.
    \end{itemize}
\end{assumption}





\section{Proof of Type Safety}

\subsection{Progress}

\begin{lemma}[Weakening] \label{lem:weaken} \quad
    \begin{enumerate}
        \item Assume that $\jdwf{}{\Gamma, x: T_0, \Gamma'}$.
            \begin{itemize}
                \item If $\jdwf{\Gamma, \Gamma'}{T}$, then $\jdwf{\Gamma, x: T_0, \Gamma'}{T}$.
                \item If $\jdwf{\Gamma, \Gamma'}{C}$, then $\jdwf{\Gamma, x: T_0, \Gamma'}{C}$.
                \item If $\jdwf{\Gamma, \Gamma'}{\Sigma}$, then $\jdwf{\Gamma, x: T_0, \Gamma'}{\Sigma}$.
                \item If $\jdwf{\Gamma, \Gamma' \mid T}{S}$, then $\jdwf{\Gamma, x: T_0, \Gamma' \mid T}{S}$.
            \end{itemize}
        \item Assume that $\jdwf{}{\Gamma, x: T_0, \Gamma'}$.
            \begin{itemize}
                \item If $\jdty{\Gamma, \Gamma'}{v}{T}$, then $\jdty{\Gamma, x: T_0, \Gamma'}{v}{T}$.
                \item If $\jdty{\Gamma, \Gamma'}{c}{C}$, then $\jdty{\Gamma, x: T_0, \Gamma'}{c}{C}$.
            \end{itemize}
        \item
            \begin{itemize}
                \item If $\jdsub{\Gamma, \Gamma'}{T_1}{T_2}$, then $\jdsub{\Gamma, x: T_0, \Gamma'}{T_1}{T_2}$.
                \item If $\jdsub{\Gamma, \Gamma'}{C_1}{C_2}$, then $\jdsub{\Gamma, x: T_0, \Gamma'}{C_1}{C_2}$.
                \item If $\jdsub{\Gamma, \Gamma'}{\Sigma_1}{\Sigma_2}$, then $\jdsub{\Gamma, x: T_0, \Gamma'}{\Sigma_1}{\Sigma_2}$.
                \item If $\jdsub{\Gamma, \Gamma' \mid T}{S_1}{S_2}$, then $\jdsub{\Gamma, x: T_0, \Gamma' \mid T}{S_1}{S_2}$.
            \end{itemize}
    \end{enumerate}
\end{lemma}
\begin{proof}
    % depends on:: asm:formula
    By simultaneous induction on the derivations.
    The cases for \rulename{WT-Rfn}, \rulename{T-Op} and \rulename{S-Rfn} use Assumption \ref{asm:formla}.
\end{proof}

\begin{lemma}[Narrowing] \label{lem:narrow} \quad
    \begin{enumerate}
        \item Assume that $\jdsub{\Gamma}{T_1}{T_2}$ and $\jdwf{}{\Gamma, x:T_1, \Gamma'}$.
            \begin{itemize}
                \item If $\jdwf{\Gamma, x:T_2, \Gamma'}{T}$, then $\jdwf{\Gamma, x: T_1, \Gamma'}{T}$.
                \item If $\jdwf{\Gamma, x:T_2, \Gamma'}{C}$, then $\jdwf{\Gamma, x: T_1, \Gamma'}{C}$.
                \item If $\jdwf{\Gamma, x:T_2, \Gamma'}{\Sigma}$, then $\jdwf{\Gamma, x: T_1, \Gamma'}{\Sigma}$.
                \item If $\jdwf{\Gamma, x:T_2, \Gamma' \mid T}{S}$, then $\jdwf{\Gamma, x: T_1, \Gamma' \mid T}{S}$.
            \end{itemize}
        \item Assume that $\jdsub{\Gamma}{T_1}{T_2}$ and $\jdwf{}{\Gamma, x:T_1, \Gamma'}$.
            \begin{itemize}
                \item If $\jdty{\Gamma, x:T_2, \Gamma'}{v}{T}$, then $\jdty{\Gamma, x: T_1, \Gamma'}{v}{T}$.
                \item If $\jdty{\Gamma, x:T_2, \Gamma'}{c}{C}$, then $\jdty{\Gamma, x: T_1, \Gamma'}{c}{C}$.
            \end{itemize}
        \item Assume that $\jdsub{\Gamma}{T_1}{T_2}$.
            \begin{itemize}
                \item If $\jdsub{\Gamma, x:T_2, \Gamma'}{T_1'}{T_2'}$, then $\jdsub{\Gamma, x: T_1, \Gamma'}{T_1'}{T_2'}$.
                \item If $\jdsub{\Gamma, x:T_2, \Gamma'}{C_1}{C_2}$, then $\jdsub{\Gamma, x: T_1, \Gamma'}{C_1}{C_2}$.
                \item If $\jdsub{\Gamma, x:T_2, \Gamma'}{\Sigma_1}{\Sigma_2}$, then $\jdsub{\Gamma, x: T_1, \Gamma'}{\Sigma_1}{\Sigma_2}$.
                \item If $\jdsub{\Gamma, x:T_2, \Gamma' \mid T}{S_1}{S_2}$, then $\jdsub{\Gamma, x: T_1, \Gamma' \mid T}{S_1}{S_2}$.
            \end{itemize}
        \item If $\jdsub{\Gamma}{T_1}{T_2}$ and $\jdsub{\Gamma \mid T_2}{S_1}{S_2}$, then $\jdsub{\Gamma \mid T_1}{S_1}{S_2}$.
    \end{enumerate}
\end{lemma}
\begin{proof}
    % depends on:: asm:formula
    By simultaneous induction on the derivations.
    The cases for \rulename{WT-Rfn}, \rulename{T-Op} and \rulename{S-Rfn} use Assumption \ref{asm:formla}.
\end{proof}

\begin{lemma}[Substitution] \label{lem:subst} \quad
    \begin{enumerate}
        \item Assume that $\jdty{\Gamma}{v}{T_0}$.
            \begin{itemize}
                \item If $\jdwf{}{\Gamma, x:T_0, \Gamma'}$, then $\jdwf{}{\Gamma, \Gamma'[v/x]}$.
                \item If $\jdwf{\Gamma, x:T_0, \Gamma'}{T}$, then $\jdwf{\Gamma, \Gamma'[v/x]}{T[v/x]}$.
                \item If $\jdwf{\Gamma, x:T_0, \Gamma'}{C}$, then $\jdwf{\Gamma, \Gamma'[v/x]}{C[v/x]}$.
                \item If $\jdwf{\Gamma, x:T_0, \Gamma'}{\Sigma}$, then $\jdwf{\Gamma, \Gamma'[v/x]}{\Sigma[v/x]}$.
                \item If $\jdwf{\Gamma, x:T_0, \Gamma' \mid T}{S}$, then $\jdwf{\Gamma, \Gamma'[v/x] \mid T[v/x]}{S[v/x]}$.
            \end{itemize}
        \item Assume that $\jdty{\Gamma}{v}{T_0}$.
            \begin{itemize}
                \item If $\jdty{\Gamma, x:T_0, \Gamma'}{v}{T}$, then $\jdty{\Gamma, \Gamma'[v/x]}{v[v/x]}{T[v/x]}$.
                \item If $\jdty{\Gamma, x:T_0, \Gamma'}{c}{C}$, then $\jdty{\Gamma, \Gamma'[v/x]}{c[v/x]}{C[v/x]}$.
            \end{itemize}
        \item Assume that $\jdty{\Gamma}{v}{T_0}$.
            \begin{itemize}
                \item If $\jdsub{\Gamma, x:T_0, \Gamma'}{T_1}{T_2}$, then $\jdsub{\Gamma, \Gamma'[v/x]}{T_1[v/x]}{T_2[v/x]}$.
                \item If $\jdsub{\Gamma, x:T_0, \Gamma'}{C_1}{C_2}$, then $\jdsub{\Gamma, \Gamma'[v/x]}{C_1[v/x]}{C_2[v/x]}$.
                \item If $\jdsub{\Gamma, x:T_0, \Gamma'}{\Sigma_1}{\Sigma_2}$, then $\jdsub{\Gamma, \Gamma'[v/x]}{\Sigma_1[v/x]}{\Sigma_2[v/x]}$.
                \item If $\jdsub{\Gamma, x:T_0, \Gamma' \mid T}{S_1}{S_2}$, then $\jdsub{\Gamma, \Gamma'[v/x] \mid T}{S_1[v/x]}{S_2[v/x]}$.
            \end{itemize}
    \end{enumerate}
\end{lemma}
\begin{proof}
    % depends on:: asm:formula
    By simultaneous induction on the derivations.
    The cases for \rulename{WT-Rfn}, \rulename{T-Op} and \rulename{S-Rfn} use Assumption \ref{asm:formla}.
\end{proof}

\begin{lemma}[Predicate Substitution] \label{lem:subst-pred} \quad
    \begin{enumerate}
        \item Assume that $\jdty{\Gamma}{A}{\rep{B}}$.
            \begin{itemize}
                \item If $\jdwf{}{\Gamma, X: \rep{B}, \Gamma'}$, then $\jdwf{}{\Gamma, \Gamma'[A/X]}$.
                \item If $\jdwf{\Gamma, X: \rep{B}, \Gamma'}{T}$, then $\jdwf{\Gamma, \Gamma'[A/X]}{T[A/X]}$.
                \item If $\jdwf{\Gamma, X: \rep{B}, \Gamma'}{C}$, then $\jdwf{\Gamma, \Gamma'[A/X]}{C[A/X]}$.
                \item If $\jdwf{\Gamma, X: \rep{B}, \Gamma'}{\Sigma}$, then $\jdwf{\Gamma, \Gamma'[A/X]}{\Sigma[A/X]}$.
                \item If $\jdwf{\Gamma, X: \rep{B}, \Gamma' \mid T}{S}$, then $\jdwf{\Gamma, \Gamma'[A/X] \mid T[A/X]}{S[A/X]}$.
            \end{itemize}
        \item Assume that $\jdty{\Gamma}{A}{\rep{B}}$.
            \begin{itemize}
                \item If $\jdty{\Gamma, X: \rep{B}, \Gamma'}{v}{T}$, then $\jdty{\Gamma, \Gamma'[A/X]}{v[A/X]}{T[A/X]}$.
                \item If $\jdty{\Gamma, X: \rep{B}, \Gamma'}{c}{C}$, then $\jdty{\Gamma, \Gamma'[A/X]}{c[A/X]}{C[A/X]}$.
            \end{itemize}
        \item Assume that $\jdty{\Gamma}{A}{\rep{B}}$.
            \begin{itemize}
                \item If $\jdsub{\Gamma, X: \rep{B}, \Gamma'}{T_1}{T_2}$, then $\jdsub{\Gamma, \Gamma'[A/X]}{T_1[A/X]}{T_2[A/X]}$.
                \item If $\jdsub{\Gamma, X: \rep{B}, \Gamma'}{C_1}{C_2}$, then $\jdsub{\Gamma, \Gamma'[A/X]}{C_1[A/X]}{C_2[A/X]}$.
                \item If $\jdsub{\Gamma, X: \rep{B}, \Gamma'}{\Sigma_1}{\Sigma_2}$, then $\jdsub{\Gamma, \Gamma'[A/X]}{\Sigma_1[A/X]}{\Sigma_2[A/X]}$.
                \item If $\jdsub{\Gamma, X: \rep{B}, \Gamma' \mid T}{S_1}{S_2}$, then $\jdsub{\Gamma, \Gamma'[A/X] \mid T}{S_1[A/X]}{S_2[A/X]}$.
            \end{itemize}
    \end{enumerate}
\end{lemma}
\begin{proof}
    % depends on:: asm:formula
    By simultaneous induction on the derivations.
    The cases for \rulename{WT-Rfn}, \rulename{T-Op} and \rulename{S-Rfn} use Assumption \ref{asm:formla}.
\end{proof}

\begin{lemma}[Remove unused type bindings] \label{lem:rm-unused} \quad
    \begin{itemize}
        \item If $x \notin \fv(\Gamma')$ and $\jdwf{}{\Gamma, x:T_0, \Gamma'}$, then $\jdwf{}{\Gamma, \Gamma'}$.
        \item If $x \notin \fv(\Gamma', T)$ and $\jdwf{\Gamma, x:T_0, \Gamma'}{T}$, then $\jdwf{\Gamma, \Gamma'}{T}$.
        \item If $x \notin \fv(\Gamma', C)$ and $\jdwf{\Gamma, x:T_0, \Gamma'}{C}$, then $\jdwf{\Gamma, \Gamma'}{C}$.
        \item If $x \notin \fv(\Gamma', \Sigma)$ and $\jdwf{\Gamma, x:T_0, \Gamma'}{\Sigma}$, then $\jdwf{\Gamma, \Gamma'}{\Sigma}$.
        \item If $x \notin \fv(\Gamma', T, S)$ and $\jdwf{\Gamma, x:T_0, \Gamma' \mid T}{S}$, then $\jdwf{\Gamma, \Gamma' \mid T}{S}$.
    \end{itemize}
\end{lemma}
\begin{proof}
    % depends on:: asm:formula
    By simultaneous induction on the derivations.
    The case for \rulename{WT-Rfn} uses Assumption \ref{asm:formla}.
\end{proof}

\begin{lemma}[Variables of non-refinement types do not occur in types] \label{lem:notin-nonrfn} \quad
    \begin{itemize}
        \item If $\jdwf{}{\Gamma, x:(y:T_1) \rarr C_1, \Gamma'}$, then $x \notin \fv(\Gamma')$.
        \item If $\jdwf{\Gamma, x:(y:T_1) \rarr C_1, \Gamma'}{T}$, then $x \notin \fv(\Gamma', T)$.
        \item If $\jdwf{\Gamma, x:(y:T_1) \rarr C_1, \Gamma'}{C}$, then $x \notin \fv(\Gamma', C)$.
        \item If $\jdwf{\Gamma, x:(y:T_1) \rarr C_1, \Gamma'}{\Sigma}$, then $x \notin \fv(\Gamma', \Sigma)$.
        \item If $\jdwf{\Gamma, x:(y:T_1) \rarr C_1, \Gamma' \mid T}{S}$, then $x \notin \fv(\Gamma', T, S)$.
    \end{itemize}
\end{lemma}
\begin{proof}
    % depends on:: asm:formula
    By simultaneous induction on the derivations.
    The case for \rulename{WT-Rfn} uses Assumption \ref{asm:formla}.
\end{proof}

\begin{lemma}[Remove non-refinement type bindings] \label{lem:rm-nonrfn} \quad
    \begin{itemize}
        \item If $\jdwf{}{\Gamma, x:(y:T_1) \rarr C_1, \Gamma'}$, then $\jdwf{}{\Gamma, \Gamma'}$.
        \item If $\jdwf{\Gamma, x:(y:T_1) \rarr C_1, \Gamma'}{T}$, then $\jdwf{\Gamma, \Gamma'}{T}$.
        \item If $\jdwf{\Gamma, x:(y:T_1) \rarr C_1, \Gamma'}{C}$, then $\jdwf{\Gamma, \Gamma'}{C}$.
        \item If $\jdwf{\Gamma, x:(y:T_1) \rarr C_1, \Gamma'}{\Sigma}$, then $\jdwf{\Gamma, \Gamma'}{\Sigma}$.
        \item If $\jdwf{\Gamma, x:(y:T_1) \rarr C_1, \Gamma' \mid T}{S}$, then $\jdwf{\Gamma, \Gamma' \mid T}{S}$.
    \end{itemize}
\end{lemma}
\begin{proof}
    Immediate by Lemma \ref{lem:notin-nonrfn} and \ref{lem:rm-unused}.
\end{proof}

\begin{lemma}[Well-formedness of typing contexts from other judgements] \label{lem:wfg} \quad
    \begin{enumerate}
        \item If $\jdwf{\Gamma}{T}$, then $\jdwf{}{\Gamma}$.
        \item If $\jdwf{\Gamma}{C}$, then $\jdwf{}{\Gamma}$.
        \item If $\jdwf{\Gamma}{\Sigma}$, then $\jdwf{}{\Gamma}$.
        \item If $\jdwf{\Gamma \mid T}{S}$, then $\jdwf{}{\Gamma}$.
    \end{enumerate}
\end{lemma}
\begin{proof}
    By simultaneous induction on the derivations.
\end{proof}

\begin{lemma}[Well-formedness of types from other judgements] \label{lem:wft} \quad
    \begin{enumerate}
        \item If $\jdty{\Gamma}{v}{T}$, then $\jdwf{\Gamma}{T}$.
        \item If $\jdty{\Gamma}{c}{C}$, then $\jdwf{\Gamma}{C}$.
    \end{enumerate}
\end{lemma}
% depends on:: asm:formla asm:prim
%              lem:weaken lem:rm-nonrfn lem:rm-unused lem:subst lem:subst-pred lem:wfg
\begin{proof}
    \def\currentprefix{wft}
    By simultaneous induction on the derivations.
    \begin{enumit}
        \item 
        \begin{description}
            \item[Case \rulename{T-CVar}:] We have
                \def\currentprefix{wft:cvar}
                \begin{enumrm}
                    \item\llabel{eq-v} $v = x$,
                    \item\llabel{eq-T} $T = \tyrfn{z}{B}{z = x}$,
                    \item\llabel{wf-G} $\jdwf{}{\Gamma}$, and
                    \item\llabel{eq-Gx} $\Gamma(x) = \tyrfn{z}{B}{\phi}$
                \end{enumrm}
                for some $z, x$, and $B$.
                W.l.o.g., we can assume that $z \notin \dom(\Gamma)$.
                Then, By \rulename{WE-BVar} with \lref{wf-G}, we have
                % \begin{enumrm}[resume]
                    % \item\llabel{wf-Gx} 
                    $\jdwf{}{\Gamma, x: B}$.
                % \end{enumrm}
                Also, since \lref{eq-Gx} implies $x \in \dom(\Gamma)$,
                it holds that $\dom(\Gamma, x: B) \supseteq \fv(z = x)$.
                By the Assumption \ref{asm:formla} with these two fact,
                we have $\jdwf{\Gamma, x: B}{z = x}$.
                By \rulename{WT-Rfn}, we have the conclusion.
            \item[Case \rulename{T-Var}:] We have
                \def\currentprefix{wft:var}
                \begin{enumrm}
                    \item\llabel{eq-v} $v = x$,
                    \item\llabel{eq-T} $T = \Gamma(x)$,
                    \item\llabel{wf-G} $\jdwf{}{\Gamma}$, and
                    \item\llabel{eq-Gx} $\Gamma(x) \neq \tyrfn{z}{B}{\phi}$ for all $z, B$, and $\phi$
                \end{enumrm}
                for some $x$.
                \lref{eq-T} implies that $\Gamma$ is of the form $\Gamma_1, x: T, \Gamma_2$
                for some $\Gamma_1$ and $\Gamma_2$.
                Therefore, by inverting \lref{wf-G} repeatedly, we have $\jdwf{\Gamma_1}{T}$.
                By Lemma \ref{lem:weaken} with \lref{wf-G}, we have the conclusion.
            \item[Case \rulename{T-Prim}:] We have
                \def\currentprefix{wft:prim}
                \begin{enumrm}
                    \item\llabel{eq-v} $v = p$,
                    \item\llabel{eq-T} $T = \ty(p)$, and
                    \item\llabel{wf-G} $\jdwf{}{\Gamma}$
                \end{enumrm}
                for some $p$.
                By Assumption \ref{asm:prim}, we have $\jdwf{}{\ty(p)}$.
                By Lemma \ref{lem:weaken} with \lref{wf-G}, we have the conclusion.
            \item[Case \rulename{T-Fun}:] We have
                \def\currentprefix{wft:fun}
                \begin{enumrm}
                    \item\llabel{eq-v} $v = \exprec{f}{x}{c}$,
                    \item\llabel{eq-T} $T = (x: T_0) \rarr C$, and
                    \item\llabel{ty-c} $\jdty{\Gamma, x: T_0}{c}{C}$
                \end{enumrm}
                for some $f, x, c, T_0$, and $C$.
                By the IH of \lref{ty-c}, we have $\jdwf{\Gamma, f:(x: T_0) \rarr C, x: T_0}{C}$.
                By Lemma \ref{lem:rm-nonrfn}, we have $\jdwf{\Gamma, x: T_0}{C}$.
                By \rulename{WT-Fun}, we have the conclusion.
            \item[Case \rulename{T-VSub}:] Immediate by inversion.
        \end{description}
        \item 
        \begin{description}
            \item[Case \rulename{T-Ret}:] We have
                \def\currentprefix{wft:ret}
                \begin{enumrm}
                    \item\llabel{eq-c} $c = \expret{v}$,
                    \item\llabel{eq-C} $C = \tycomp{\emptyset}{T}{\square}$, and
                    \item\llabel{ty-v} $\jdty{\Gamma}{v}{T}$
                \end{enumrm}
                for some $v$ and $T$.
                By the IH of \lref{ty-v}, we have $\jdwf{\Gamma}{T}$.
                By Lemma \ref{lem:wfg}, we have $\jdwf{}{\Gamma}$.
                Then, we have the conclusion by the following derivation:
                \[
                    \infer{\jdwf{\Gamma}{\tycomp{\emptyset}{T}{\square}}}{
                        \infer{\jdwf{\Gamma}{\emptyset}}{}
                        &
                        \jdwf{\Gamma}{T}
                        &
                        \infer{\jdwf{\Gamma \mid T}{\square}}
                        {\jdwf{}{\Gamma}}
                    }
                \]
            \item[Case \rulename{T-App}:] We have
                \def\currentprefix{wft:app}
                \begin{enumrm}
                    \item\llabel{eq-c} $c = v_1~v_2$,
                    \item\llabel{eq-C} $C = C_0[v_2/x]$,
                    \item\llabel{ty-v1} $\jdty{\Gamma}{v_1}{(x:T_0) \rarr C_0}$, and
                    \item\llabel{ty-v2} $\jdty{\Gamma}{v_2}{T_0}$
                \end{enumrm}
                for some $x, v_1, v_2, T_0$ and $C_0$.
                By the IH of \lref{ty-v1}, we have $\jdwf{\Gamma}{(x:T_0) \rarr C_0}$.
                By inversion, we have $\jdwf{\Gamma, x:T_0}{C_0}$.
                By Lemma \ref{lem:subst}, we have the conclusion.
            \item[Case \rulename{T-If}:] We have
                \def\currentprefix{wft:if}
                \begin{enumrm}
                    \item\llabel{eq-c} $c = \expif{v}{c_1}{c_2}$,
                    \item\llabel{ty-v} $\jdty{\Gamma}{v}{\tyrfn{x}{\tybool}{\phi}}$,
                    \item\llabel{ty-c1} $\jdty{\Gamma, v = \exptrue}{c_1}{C}$, and
                    \item\llabel{ty-c2} $\jdty{\Gamma, v = \expfalse}{c_2}{C}$
                \end{enumrm}
                for some $x, v, c_1, c_2$, and $\phi$.
                By the IH of \lref{ty-c1}, we have $\jdwf{\Gamma, v = \exptrue}{C}$.
                By Lemma \ref{lem:rm-unused}, we have the conclusion.
            \item[Case \rulename{T-CSub}:] Immediate by inversion.
            \item[Case \rulename{T-Let}:] We have
                \def\currentprefix{wft:let}
                \begin{enumrm}
                    \item\llabel{eq-c} $c = \explet{x}{c_1}{c_2}$,
                    \item\llabel{eq-C} $C = \tycomp{\Sigma}{T_2}{\bind{S_1}{x}{S_2}}$,
                    \item\llabel{ty-c1} $\jdty{\Gamma}{c_1}{\tycomp{\Sigma}{T_1}{S_1}}$,
                    \item\llabel{ty-c2} $\jdty{\Gamma, x: T_1}{c_2}{\tycomp{\Sigma}{T_2}{S_2}}$, and
                    \item\llabel{in-x} $x \notin \fv(T_2) \cup \fv(\Sigma)$
                \end{enumrm}
                for some $x, c_1, c_2, \Sigma, T_1, T_2, S_1$ and $S_2$.
                By the IHs of \lref{ty-c1} and \lref{ty-c2} respectively, we have
                \begin{itemize}
                    \item $\jdwf{\Gamma}{\tycomp{\Sigma}{T_1}{S_1}}$ and
                    \item $\jdwf{\Gamma, x: T_1}{\tycomp{\Sigma}{T_2}{S_2}}$.
                \end{itemize}
                By inversion, we have
                \begin{enumrm}[resume]
                    \item\llabel{wf-sig} $\jdwf{\Gamma}{\Sigma}$,
                    \item\llabel{wf-S1} $\jdwf{\Gamma \mid T_1}{S_1}$,
                    \item\llabel{wf-T2} $\jdwf{\Gamma, x:T_1}{T_2}$, and
                    \item\llabel{wf-S2} $\jdwf{\Gamma, x:T_1 \mid T_2}{S_2}$.
                \end{enumrm}
                By Lemma \ref{lem:rm-unused} with \lref{in-x} \lref{wf-T2}, we have
                \begin{enumrm}[resume]
                    \item\llabel{wf-T2-2} $\jdwf{\Gamma}{T_2}$~.
                \end{enumrm}
                Case analysis on $\bind{S_1}{x}{S_2}$.
                (Note that the definition of $\ggeq$ has only two cases.)
                \begin{description}
                    \item[Case $S_1 = S_2 = \square$:]
                        By Lemma \ref{lem:wfg} with \lref{wf-sig}, we have $\jdwf{}{\Gamma}$.
                        From this fact and \lref{wf-sig} and \lref{wf-T2-2},
                        we have the conclusion by the following derivation:
                        \[
                            \infer{\jdwf{\Gamma}{\tycomp{\Sigma}{T_2}{\square}}}{
                                \jdwf{\Gamma}{\Sigma}
                                &
                                \jdwf{\Gamma}{T_2}
                                &
                                \infer{\jdwf{\Gamma \mid T_2}{\square}}
                                {\jdwf{}{\Gamma}}
                            }
                        \]
                    \item[Case $S_1 = \tyctl{x}{C_0}{C_1}$ and $S_2 = \tyctl{z}{C_2}{C_0}$
                        for some $z, C_0, C_1$, and $C_2$:]
                        We have
                        \begin{enumrm}[resume]
                            \item\llabel{in-x-2} $x \notin \fv(C_2) \setminus \{z\}$
                        \end{enumrm}
                        from the side condition of the definition of $\ggeq$.
                        By inversion with \lref{wf-S1} and \lref{wf-S2} respectively, we have
                        \begin{enumrm}[resume]
                            \item\llabel{wf-C0} $\jdwf{\Gamma, x:T_1}{C_0}$,
                            \item\llabel{wf-C1} $\jdwf{\Gamma}{C_1}$, and
                            \item\llabel{wf-C2} $\jdwf{\Gamma, x:T_1, z:T_2}{C_2}$~.
                        \end{enumrm}
                        W.l.o.g., we can assume $x \neq z$.
                        Then, \lref{in-x-2} implies $x \notin \fv(C_2)$.
                        Therefore, by Lemma \ref{lem:rm-unused} with \lref{wf-C2} and \lref{in-x},
                        we have $\jdwf{\Gamma, z:T_2}{C_2}$.
                        From this and \lref{wf-sig}, \lref{wf-T2-2}, and \lref{wf-C1},
                        we have the conclusion by the following derivation
                        (note that $\bind{S_1}{x}{S_2} = \tyctl{z}{C_2}{C_1}$):
                        \[
                            \infer{\jdwf{\Gamma}{\tycomp{\Sigma}{T_2}{\tyctl{z}{C_2}{C_1}}}}{
                                \jdwf{\Gamma}{\Sigma}
                                &
                                \jdwf{\Gamma}{T_2}
                                &
                                \infer{\jdwf{\Gamma \mid T_2}{\tyctl{z}{C_2}{C_1}}}{
                                    \jdwf{\Gamma, z:T_2}{C_2}
                                    &
                                    \jdwf{\Gamma}{C_1}
                                }
                            }
                        \]
                \end{description}
            \item[Case \rulename{T-Op}:] We have
                \def\currentprefix{wft:op}
                \begin{enumrm}
                    \item\llabel{eq-c} $c = \expop{v}{y}{c}$,
                    \item\llabel{eq-C} $C = \tycomp{\Sigma}{T_3}{\tyctl{z}{C_0}{C_2[\rep{A/X}][v/x]}}$,
                    \item\llabel{in-sig} $\Sigma \ni \op: \forall \rep{X: \rep{B}}. (x: T_1) \rarr ((y: T_2) \rarr C_1) \rarr C_2$,
                    \item\llabel{in-y} $y \notin \fv(T_3) \cup \fv(\Sigma) \cup (\fv(C_0) \setminus \{ z \})$,
                    \item\llabel{wf-A} $\rep{\jdty{\Gamma}{A}{\rep{B}}}$,
                    \item\llabel{ty-v} $\jdty{\Gamma}{v}{T_1[\rep{A/X}]}$, and
                    \item\llabel{ty-c} $\jdty{\Gamma, y: T_2[\rep{A/X}][v/x]}{c}{\tycomp{\Sigma}{T_3}{\tyctl{z}{C_0}{C_1[\rep{A/X}][v/x]}}}$,
                \end{enumrm}
                for some $x, y, z, v, c, \rep{X}, \rep{A}, \rep{\rep{B}}, \Sigma, T_1, T_2, T_3, C_0, C_1$ and $C_2$.
                By the IH of \lref{ty-c}, we have
                \[
                    \jdwf{\Gamma, y: T_2[\rep{A/X}][v/x]}{\tycomp{\Sigma}{T_3}{\tyctl{z}{C_0}{C_1[\rep{A/X}][v/x]}}}~.
                \]
                By inversion and Lemma \ref{lem:rm-unused} with \lref{in-y}, we have
                \begin{enumrm}[resume]
                    \item\llabel{wf-sig} $\jdwf{\Gamma}{\Sigma}$,
                    \item\llabel{wf-T3} $\jdwf{\Gamma}{T_3}$, and
                    \item\llabel{wf-C0} $\jdwf{\Gamma, z:T_3}{C_0}$~.
                \end{enumrm}
                By inversion of \lref{wf-sig} with \lref{in-sig}, we have
                $\jdwf{\Gamma, \rep{X: \rep{B}}}{(x: T_1) \rarr ((y: T_2) \rarr C_1) \rarr C_2}$~.
                By more inversion and Lemma \ref{lem:rm-nonrfn},
                we have $\jdwf{\Gamma, \rep{X: \rep{B}}, x: T_1}{C_2}$.
                By Lemma \ref{lem:subst-pred} with \lref{wf-A} and Lemma \ref{lem:subst} with \lref{ty-v},
                we have $\jdwf{\Gamma}{C_2[\rep{A/X}][v/x]}$.
                From this and \lref{wf-sig}, \lref{wf-T3}, \lref{wf-C0},
                we have the conclusion by the following derivation:
                \[
                    \infer{\jdwf{\Gamma}{\tycomp{\Sigma}{T_3}{\tyctl{z}{C_0}{C_2[\rep{A/X}][v/x]}}}}{
                        \jdwf{\Gamma}{\Sigma}
                        &
                        \jdwf{\Gamma}{T_3}
                        &
                        \infer{\jdwf{\Gamma \mid T_3}{\tyctl{z}{C_0}{C_2[\rep{A/X}][v/x]}}}{
                            \jdwf{\Gamma, z:T_3}{C_0}
                            &
                            \jdwf{\Gamma}{C_2[\rep{A/X}][v/x]}
                        }
                    }
                \]
            \item[Case \rulename{T-Hndl}:] We have
                \def\currentprefix{wft:hndl}
                \begin{enumrm}
                    \item\llabel{eq-c} $c = \expwith{h}{c_0}$,
                    \item\llabel{ty-c0} $\jdty{\Gamma}{c_0}{\tycomp{\Sigma}{T}{\tyctl{x_r}{C_1}{C}}}$
                \end{enumrm}
                for some $x_r, h, c_0, \Sigma, T$ and $C_1$.
                We have the conclusion by applying inversion twice to \lref{ty-c0}.
        \end{description}
    \end{enumit}
\end{proof}

\begin{lemma}[Canonical forms] \label{lem:cano} \quad
    \begin{enumerate}
        \item If $\jdty{}{v}{(x: T) \rarr C}$, then (i) $v = \exprec{f}{x}{c}$ for some $f, c$,
            or (ii) $v = p$ for some $p$ and $\zeta(p, v)$ is defined for all $v$ such that $\jdty{}{v}{T}$.
        \item If $\jdty{}{v}{\tyrfn{x}{\tybool}{\phi}}$, then $v = \exptrue$ or $v = \expfalse$.
    \end{enumerate}
\end{lemma}
\begin{proof}
    % depends on:: asm:prim lem:wft
    By induction on the derivations.
    \begin{enumit}
        \item 
        \begin{description}
            \item[Case \rulename{T-Fun}:] Obvious.
            \item[Case \rulename{T-Prim}:] Immediate from Assumption \ref{asm:prim}.
            \item[Case \rulename{T-VSub}:] By the IH and inversion of the subtyping judgment.
                The case for (ii) uses Lemma \ref{lem:wft}.
            \item[Otherwise:] Contradictory.
        \end{description}
        \item 
        \begin{description}
            \item[Case \rulename{T-Prim}:] Immediate from Assumption \ref{asm:prim}.
            \item[Case \rulename{T-VSub}:] By the IH and inversion of the subtyping judgment.
            \item[Otherwise:] Contradictory.
        \end{description}

    \end{enumit}
\end{proof}

\begin{theorem}[Progress] \label{thm:progress}
    If $\jdty{\emptyset}{c}{\tycomp{\Sigma}{T}{S}}$, then either
    \begin{itemize}
        \item $c = \expret{v}$ for some $v$ such that $\jdty{\emptyset}{v}{T}$,
        \item $c = K[\op~v]$ for some $K, \op$ and $v$ such that $\op \in \dom(\Sigma)$, or
        \item $c \eval c'$ for some $c'$.
    \end{itemize}
\end{theorem}

% 
Building upon the notion of well-formed types (\cref{def:types_well_formed}), a well-formed configuration exhibits behaviour without deadlocks, where there is always some action able to be performed in the future, until the termination type is reached.
This behaviour is referred to as \emph{liveness}, and is given in~\cref{def:configs_iso_live}, and depends on \emph{future enabled} actions.
\begin{definition}[Future-enabled Configurations (\isFE*)]\label{def:configs_fe}
   A configuration \VIso*\ is \emph{future-enabled (\isFE*)} if\ %
   $\exists\ValTime$ such that \Trans*{\VIso}:{\ValTime,\CommMsg}[\VIso'].
   The same applies for \VSoc*.
 \end{definition}
\begin{definition}[Live Configurations]\label{def:configs_live}
    \CIso*\ is \emph{live} if
    $\TypeS=\TypeEnd$ or if \VIso*\ is \isFE*.
\end{definition}
The liveness of a configuration indicates whether a deadlock has been reached. There must either be an action immediately viable, some future enabled actions, or the type must have reached termination.

Building upon the notion of well-formed types (\cref{def:types_well_formed}), a well-formed configuration exhibits behaviour without deadlocks, where there is always some action able to be performed in the future, until the termination type is reached.
This behaviour is referred to as \emph{liveness}, and is given in~\cref{def:configs_iso_live}, and depends on \emph{future enabled} actions.
\input{tex/defs/configs/iso/future_en.tex}
\input{tex/defs/configs/iso/live.tex}
The liveness of a configuration indicates whether a deadlock has been reached. There must either be an action immediately viable, some future enabled actions, or the type must have reached termination.
\input{tex/defs/configs/iso/well_formed.tex}
Given a type \TypeS*\ well-formed against \ValClocks*, it holds that a configuration of \IsoCfg*\ will have future enabled actions, and therefore be live.
\begin{note}
    For a \emph{well-formed} \TypeS*, \CfgIso*1{\ValClocks_{0}} is \emph{live}.
\end{note}

Given a type \TypeS*\ well-formed against \ValClocks*, it holds that a configuration of \IsoCfg*\ will have future enabled actions, and therefore be live.
\begin{note}
    For a \emph{well-formed} \TypeS*, \CfgIso*1{\ValClocks_{0}} is \emph{live}.
\end{note}


\newcommand{\FullChoice}[0]{\left\{\TypComm_i \,\MsgLabel_i\left\langle T_i\right\rangle \left(\delta_i,\lambda_i\right).\TypeS_i\right\}_{i\in I}}

\begin{lemma}\label{lem:cfg_wf_neq_alpha}
   %
   If \CIso*\ is \emph{well-formed} then $\TypeS\neq\alpha$.
   %
\end{lemma}
\begin{proof}
   %
   By the hypothesis \CIso*\ is \emph{well-formed} meaning $\exists\Const$ such that $\ValClocks\models\Const$ and  $\emptyset;\Const~\Entails\TypeS$.

   Consider by contradiction of the hypothesis that $\TypeS=\alpha$.
   If \CIso*;[\alpha]\ is \emph{well-formed} then, by the judgement of rule \LblTypVar*\ $\alpha:\Const;\Const~\Entails\alpha$ and the evaluation shown in~\Cref{eq:cfg_wf_neq_alpha_eval} must hold, where there must be some recursive definition earlier in the derivation tree $\mu\alpha.\TypeS'$ corresponding to $\alpha$.
   %
   \begin{minieq}\label{eq:cfg_wf_neq_alpha_eval}
   \begin{array}[c]{l}
      \infer[\LblTypRec]{%
         \emptyset;\Const~\Entails\mu\alpha.\TypeS'%
      }{%   
         \infer{\alpha:\Const;\Const~\Entails\TypeS'}{%
            \infer[\vdots]{}{%
               \infer[\LblTypVar]{%
                  \alpha:\Const;\Const~\Entails\alpha%
               }{}%
            }%
         }%
      }
    \end{array}
   \end{minieq}
   
   \noindent If $\exists\ValClocks',\mu\alpha.\TypeS'$ such that \CIso*[\ValClocks'];[\mu\alpha.\TypeS']\ is \emph{well-formed} and \Trans*{\CIso[\ValClocks'];[\mu\alpha.\TypeS']}*[\CIso;[\alpha]], then the immediate transition must be via rule \LblCfgIsoUnfold*\ as shown in~\Cref{eq:cfg_wf_neq_alpha_unfold}; where either $\CIso[{\ValClocks}''];[{\TypeS}'']=\CIso;[\alpha]$ or \Trans*{\CIso[{\ValClocks}''];[{\TypeS}'']}*[\CIso;[\alpha]].
   %
   By the premise of rule \LblCfgIsoUnfold*\ any recursive calls following their definition are replaced by the definition of their next unfolding denoted by $\TypeS'\Subst[\mu\alpha.\TypeS'][\alpha]$.
   %
   \begin{minieq}\label{eq:cfg_wf_neq_alpha_unfold}
   \begin{array}[c]{l}
      \infer[\LblCfgIsoUnfold]{%
      \Trans{\CIso[\ValClocks'];[\mu\alpha.\TypeS']}:{\ProgAction}[\CIso[{\ValClocks}''];[{\TypeS}'']]
      }{%
      \Trans{\CIso[\ValClocks'];[{\TypeS}'\Subst[\mu\alpha.\TypeS'][\alpha]]}:{\ProgAction}[\CIso[{\ValClocks}''];[{\TypeS}'']]
      }
    \end{array}
   \end{minieq}
   
   \noindent Therefore, if \CIso*\ is \emph{well-formed} then $\TypeS\neq\alpha$.
   %
\end{proof}

% \begin{definition}[Live Configurations]\label{def:configs_live}
    \CIso*\ is \emph{live} if
    $\TypeS=\TypeEnd$ or if \VIso*\ is \isFE*.
\end{definition}

\begin{lemma}\label{lem:cfg_wf_end}
   %
   \CIso*;[\TypeEnd]\ is always \emph{well-formed}.
   %
\end{lemma}
\begin{proof}
   %
   By the hypothesis $\exists\Const$ such that $\ValClocks\models\Const$ and $\emptyset;\Const~\Entails\TypeS$, and by rule \LblTypEnd*\ $\emptyset;\TypeTrue~\Entails\TypeEnd$.
%   
   Therefore the hypothesis holds as $\ValClocks\models\TypeTrue$ always holds.
   %
\end{proof}

%
\begin{lemma}\label{lem:cfg_wf_then_live}
    %
    If \CIso*\ is \emph{well-formed}, then \CIso*\ is \emph{live}.
    %
\end{lemma}
\begin{proof}
    %
    By the hypothesis, there must $\exists\Const$ such that $\emptyset;\Const~\Entails\TypeS$ and $\ValClocks\models\Const$.
    %
    We proceed by induction on each case of \TypeS*:
    \begin{inductivecase}
        %
        %
        %
        % ~ choice
        \item\NewCase[$\TypeS=\simplechoice$]\label{itm:cfg_wf_then_live_choice} 
        By the hypothesis and the judgement of rule \LblTypChoice*\ $\exists\Const_i$ such that $\ValClocks\models\Past[\Const_i]$ and $\emptyset;\Past[\Const_i]~\Entails\FullChoice$.
        %
        By~\Cref{def:configs_wf,def:configs_fe} \CIso*\ is \isFE*, and therefore, by~\Cref{def:configs_live} \CIso*\ is \emph{live}.
        %
        %
        %
        % ~ rec def
        \item\NewCase[$\TypeS=\mu\alpha.{\TypeS}'$]\label{itm:wf_then_live_recdef} 
        By the hypothesis $\exists\Const$ such that $\ValClocks\models\Const$ and $\emptyset;\Const~\Entails\TypRecDef$, and by the premise of rule \LblTypRec*\ $\alpha:\Const;\Const~\Entails{\TypeS}'$ and \CIso*;[\TypeS']\ is \emph{well-formed}.
        
        We proceed by inner induction on each case of \TypeS*':
        \begin{inductivecase}
            %
            %
            % ~ rec def -> choice
            \item\NewCase[$\TypeS'=\simplechoice$] 
            By the judgement of rule \LblTypChoice*\ and the premise of rule \LblTypRec*, $\exists\Const_i$ such that $\ValClocks\models\Const_i$ and $\emptyset;\Const_i~\Entails\TypRecDef$ and $\alpha:\Const_i;\Past[\Const_i]~\Entails\FullChoice$.
            %
            It follows~\Cref{itm:cfg_wf_then_live_choice} of~\Cref{lem:cfg_wf_then_live} that, by~\Cref{def:configs_wf,def:configs_fe,def:configs_live}, \CIso*;[\simplechoice]\ is \isFE*, \emph{well-formed} and \emph{live}.
            Therefore, it holds that \CIso*\ is \emph{live}.
            % Then by the judgement of \LblTypChoice*\ and the premise of \LblTypRec*\ $\exists\Const_i$ such that $\ValClocks\models\Const_i$ and $\emptyset;\Const_i~\Entails\TypRecDef$ and $\alpha:\Const_i;\Past[\Const_i]~\Entails\simplechoice$ and \CIso*;[\simplechoice]\ is \emph{well-formed} and \emph{future-enabled} by~\cref{def:configs_fe}.
            %
            % Therefore \CIso*;[\simplechoice]\ is \emph{live} by definition.
            %
            %
            % ~ rec def -> rec def
            \item\NewCase[$\TypeS'=\mu\alpha'.{\TypeS}''$] 
            Then $\alpha:\Const;\Const~\Entails\mu\alpha'.{\TypeS}''$ and $\alpha:\Const,\alpha':\Const;\Const~\Entails{\TypeS}''$ and \CIso*;[{\TypeS}'']\ is \emph{well-formed}.
            %
            Therefore, \CIso*\ is \emph{live} if \CIso*;[{\TypeS}'']\ is \emph{live} (see other cases).
            %
            %
            % ~ rec def -> end
            \item $\TypeS'=\TypeEnd$ is \emph{live} by~\Cref{def:configs_live}.
            %
            %
            % ~ rec def -> alpha
            \item ${\TypeS'}$ cannot equal $\alpha$ by~\Cref{lem:cfg_wf_neq_alpha}.
            %
        \end{inductivecase}
        %
        %
        %
        % ~ end
        \item\NewCase[$\TypeS=\TypeEnd$]\label{itm:wf_then_live_end} 
        By~\Cref{def:configs_live} it holds that \CIso*;[\TypeEnd]\ is live.
        %
        %
        %
        % ~ alpha
        \item\NewCase[$\TypeS\neq\alpha$] By~\Cref{lem:cfg_wf_neq_alpha}.
        %
    \end{inductivecase}

    \noindent Therefore, it holds that for any $S$ that \CIso*\ is \emph{well-formed}, \CIso*\ is \emph{live}.
    %
\end{proof}
%
%

%

\begin{lemma}\label{lem:init_wf_then_live}
   %
   Given a \emph{well-formed} \TypeS*, \CIso*[\ValClocks_0]\ is \emph{live}.
   %
\end{lemma}
\begin{proof}
   %
   By~\Cref{def:types_wf}, \Sat*[\ValClocks_0]:[\Past]\ holds for any valid \Const*.
   %
\end{proof}

% ~ configuration transitions
% ! 
%
% ! (lemma 12) : iso cfg trans
\begin{lemma}\label{lem:configs_iso_trans}
   %
   The following holds for transitions of configurations:
   \begin{minieq}*
      \begin{array}[c]{c c l}
         %
         \Trans{\CIso}:{\ValTime}[\CIso']%
         & \implies & %
         \begin{array}[c]{lcl}
            %
            \ValClocks'=\ValClocks+\ValTime & %
            \land & %
            \TypeS'=\TypeS%
            %
         \end{array}
         %
         \\[-1ex]\\%
         \Trans{\CIso}:{\CommAction}[\CIso']%
         & \implies & %
         \begin{array}[t]{lcl c lcl c lcl}
            %
            \ValClocks & \models & \Const & %
            \land & %
            \ValClocks' & = & \ReSet[\ValClocks] & %
            \land & %
            \Const' & = & \ReSet[\Const]%
            %
            \\[-1ex]\\%
            \mathllap{\land\;}\ValClocks' & \models & \Const' & %
            \land & %
            \Const' & \subseteq & \TypEnvCond[\TypeS'] %
            %
         \end{array}%
         %
      \end{array}%
   \end{minieq}
   %
\end{lemma}
\begin{proof}
   %
   By inspection of the formation and transition rules in~\Cref{fig:types_rule,fig:typesemantics_tuple}.
   %
\end{proof}
%
%
% ! (lemma 11) : soc cfg trans
\begin{lemma}\label{lem:configs_soc_trans}
   %
   The following holds for transitions of configurations with queues:
   \begin{minieq}*
      \begin{array}[c]{c c lcl c lcl c l}
         %
         \Trans{\CSoc}:{\ValTime}[\CSoc']%
         & \implies & %
            %
            \TypeS' & = & \TypeS & \land & %
            \Queue' & = & \Queue & \land & %
            \ValClocks'=\ValClocks+\ValTime%
            %
         \\[1ex]%
         \Trans{\CSoc}:{\RecvMsg}[\CSoc']%
         & \implies & %
            %
            \TypeS' & = & \TypeS & \land & %
            \Queue' & = & \Queue;\Msg & \land & %
            \ValClocks'=\ValClocks
            %
         \\[1ex]%
         \Trans{\CSoc}:{\SendMsg}[\CSoc']%
         & \implies & %
            %
            \MsgType & = & \Msg & \land & %
            \Queue' & = & \Queue & \land & %
            \Trans{\CIso}:{\SendAction}[\CIso'] %
            %
         \\[1ex]%
         \Trans{\CSoc}:{\SiltAction}[\CSoc']%
         & \implies & %
            % 
            \MsgType & = & \Msg & \land & %
            \Queue' & = & \Msg;\Queue & \land & %
            \Trans{\CIso}:{\RecvAction}[\CIso']%
            %
      \end{array}
   \end{minieq}
   %
\end{lemma}
\begin{proof}
   %
   By inspection of the transitions in~\Cref{fig:typesemantics_triple}, supported by~\Cref{lem:configs_iso_trans}.
   %
\end{proof}
%

% ~ preservation of wf and compat
% \begin{definition}[Compatible Systems]\label{def:configs_compat}
    %
    Let $\VSoc_1 = \CSoc_1$ and $\VSoc_2 = \CSoc_2$. 
    System \VSys*\ is \emph{compatible} (written $\VSoc_1\bot\, \VSoc_2$) if:
    \begin{enumerate}
    \item\label{itm:configs_compat_non_empty_queues} $\Queue_1=\emptyset%
            ~\lor~%
            \Queue_2=\emptyset$%
        %
        \\
       \item\label{itm:configs_compat_dual_types} $\Queue_1=\Queue_2=\emptyset%
            ~\implies~%
            \ValClocks_1=\ValClocks_2%
            ~\land~%
            \TypeS_1=\Dual_2$
        %
        \\
  \item
    \label{itm:configs_compat_expected_receive} 
    $\Queue_1=\Msg;\Queue'_1
            ~\Implies~%
            \exists\ValClocks'_1,\TypeS'_1:
            \Trans{\CIso_1}:{\RecvMsg}[\CIso'_1]%
            ~\land~%
            \CSoc'_1 \bot\, \VSoc_2$
         %   \Compat[\VSoc_1'][\VSoc_2]$%
            \\
\item
    %\label{itm:configs_compat_expected_receive} 
    $\Queue_2=\Msg;\Queue'_2
            ~\Implies~%
            \exists\ValClocks'_2,\TypeS'_2:
            \Trans{\CIso_2}:{\RecvMsg}[\CIso'_2]%
            ~\land~%
            \VSoc_1 \bot\, \CSoc'_2$
         %   \Compat[\VSoc_1'][\VSoc_2]$%
            \\
        % $\forall i, j\in\mkSet[1,2]:%
        % i\neq j%
        % \quad%
        % \Queue_i=\Msg;\Queue_i'%
        % ~\Implies~%
        % \exists\ValClocks'_i,\TypeS'_i:%
        % \Trans{\CIso_i}:{\RecvMsg}[\CIso'_i]%
        % ~\land~%
        % \Compat[\VSoc'_i][\VSoc_j]$%
        %
    \end{enumerate}
    %
    % \noindent We write \Compat*\ if system \VSys*\ is compatible.
    % \Cref{itm:configs_compat_expected_receive} is symmetric.
\end{definition}
% \begin{definition}[Latest-enabled Action (\isLE*)]\label{def:configs_le}
    %
    A configuration \CIso*, that is future-enabled, has a \emph{latest-enabled send} (resp. \emph{latest-enabled receive}), or \isLE*!\ for short (resp. \isLE*?), 
    if $\forall t$ such that $\CIso+{+t}\isFE~\Implies~\exists t'\geq t: \Trans{\VIso}:{\ValTime',\SendMsg}$.
\end{definition}

% ! 
%
% ! (lemma 17) : compat wf, single transition -> wf
\begin{lemma}\label{lem:cfgs_trans_wf_pres}
    %
    If \VIso*_1\ and \VIso*_2\ are both \emph{well-formed} 
    and \Compat*[\VSoc_1][\VSoc_2]\ 
    and \Trans*{\Parl{\VSoc_1,\VSoc_2}}[\Parl{\VSoc'_1,\VSoc'_2}], 
    then both \VIso*'_1\ and \VIso*'_2\ are \emph{well-formed}.
    %
\end{lemma}
\begin{proof}
    %
    We proceed by induction on the depth of the derivation tree, analysing each case of the last rule applied for the transition \Trans*{\Parl{\VSoc_1,\VSoc_2}}[\Parl{\VSoc'_1,\VSoc'_2}].
    % \begin{inline}+
    %     \item \LblCfgSysWait*
    %     \item \LblCfgSysLComm*
    %     \item \LblCfgSysLPar*
    % \end{inline}
    %
    \begin{inductivecase}
        %
        %
        %
        %
        %
        % ~ wait
        \item\NewCase[\LblCfgSysWait*]\label{case:cfgs_trans_wf_pres_wait} 
        As shown in~\Cref{eq:cfgs_trans_wf_pres_wait_trans}, both \VSoc*_1\ and \VSoc*_2\ make a rule \LblCfgSocTime*\ transition with the same valuation of \ValTime*.
        If $\ValTime=0$ then by~\Cref{lem:configs_iso_trans,lem:configs_soc_trans} $\VIso_1=\VIso'_1$ and $\VIso_2=\VIso'_2$ and both \VIso*'_1\ and \VIso*'_2\ are \emph{well-formed}.
        %
        \begin{minieq}\label{eq:cfgs_trans_wf_pres_wait_trans}
        \begin{array}[c]{l}
            \infer[\LblCfgSysWait]{%
                \Trans{\Parl{\VSoc_1,\VSoc_2}}:{\ValTime}[\Parl{\VSoc'_1,\VSoc'_2}]
            }{%
                \infer[\LblCfgSocTime]{%
                    \Trans{\VSoc_1}:{\ValTime}[\VSoc'_1]
                }{\dots}
                & %
                \infer[\LblCfgSocTime]{%
                    \Trans{\VSoc_2}:{\ValTime}[\VSoc'_2]
                }{\dots}
            }
            \end{array}
        \end{minieq}
            
        \noindent We proceed with only \VSoc*_1\ as the analysis covers \VSoc*_2.
        If $\ValTime>0$ then by~\Cref{lem:configs_iso_trans,lem:configs_soc_trans} $\ValClocks'_1=\ValClocks_1+\ValTime$ and $\TypeS'_1=\TypeS_1$.
        %
        By induction on the depth of the derivation tree, analysing the last rule applied for the transition \Trans*{\VIso_1}:{\ValTime}[\VIso'_1]:
        \begin{inductivecase}
            %
            %
            % ~ wait -> tick
            \item\NewCase[\LblCfgIsoTick*]\label{case:cfgs_trans_wf_pres_wait_tgtz_tick} 
            Then \Trans*{\CIso_1}:{\ValTime}[\CIso+{+\ValTime}_1].
            We proceed by inner induction on the different cases of \TypeS*_1:
            \begin{inductivecase}
                %
                % ~ wait -> tick -> choice
                \item\NewCase[$\TypeS_1=\simplechoice$]\label{case:cfgs_trans_wf_pres_wait_tgtz_tick_choice} 
                By rule \LblCfgIsoInteract*\ $\exists\Const_i$ such that $\ValClocks_1\models\Past[\Const_i]$ and $\emptyset;\Past[\Const_i]~\Entails\FullChoice$, as in~\Cref{itm:cfg_wf_then_live_choice} of~\Cref{lem:cfg_wf_then_live}.
                %
                By induction hypothesis $\exists\Const'_i$ such that $\ValClocks_1+\ValTime\models\Past[\Const'_i]$ and $\emptyset;\Past[\Const'_i]~\Entails\FullChoice$, which is assured by the (persistency) premise of rule \LblCfgSocTime*.
                %
                Therefore, it holds that \CIso*[\ValClocks_1]+{+\ValTime};[\simplechoice]\ is \emph{well-formed}.
                % By~\Cref{def:configs_fe} \CIso*_1\ is \emph{future-enabled} and by rule \LblCfgSocTime*\ $\text{(persistency)}$ it holds that \CIso*+{+\ValTime}_1\ is also \emph{future-enabled}.
                % %
                % Therefore the hypothesis holds, \CIso*[\ValClocks_1]+{+\ValTime};[\simplechoice]\ is \emph{well-formed}; by~\Cref{def:types_wf} and rule \LblTypChoice*\ $\exists\Const'_i$ such that $\ValClocks_1+\ValTime\models\Past[\Const'_i]$ and $\emptyset;\Past[\Const'_i]~\Entails\simplechoice$.
                %
                %
                % ~ wait -> tick -> recursion
                \item\NewCase[$\TypeS_1=\mu\alpha.{\TypeS}''_1$]\label{case:cfgs_trans_wf_pres_wait_tgtz_tick_recursion} 
                By~\Cref{def:types_wf} $\exists\Const$ such that $\ValClocks_1\models\Const$ and $\emptyset;\Const~\Entails\mu\alpha.{\TypeS}''_1$, and (by rule \LblTypRec*) $\alpha:\Const;\Const~\Entails{\TypeS}''_1$ and \CIso*[\ValClocks_1];[{\TypeS}''_1]\ is \emph{well-formed}.
                %
                Therefore, the well-formedness of \CIso*[\ValClocks_1]+{+\ValTime};[\mu\alpha.{\TypeS}''_1]\ is dependant on the well-formedness of \CIso*[\ValClocks_1]+{+\ValTime};[{\TypeS}''_1]. (See other cases, as in~\Cref{itm:wf_then_live_recdef} of~\Cref{lem:cfg_wf_then_live}.)
                %
                %
                % ~ wait -> tick -> end
                \item\NewCase[$\TypeS_1=\TypeEnd$]\label{case:cfgs_trans_wf_pres_wait_tgtz_tick_end} 
                By~\Cref{lem:cfg_wf_end} \CIso*+{+\ValTime};[\TypeEnd]\ is \emph{well-formed}.
                %
                %
                % ~ wait -> tick -> rec call
                \item\label{case:cfgs_trans_wf_pres_wait_tgtz_tick_reccal} ${\TypeS}_1$ cannot equal $\alpha$ by~\Cref{lem:cfg_wf_neq_alpha}.
                %
            \end{inductivecase}
            %
            %
            % ~ wait -> unfold
            \item\NewCase[\LblCfgIsoUnfold*]\label{case:cfgs_trans_wf_pres_wait_tgtz_unfold} 
            Then $\TypeS_1=\mu\alpha.{\TypeS}''_1$ and by the hypothesis $\exists\Const$ such that $\ValClocks_1\models\Const$ and $\emptyset;\Const~\Entails\TypRecDef$, and by rule \LblTypRec*\ $\alpha:\Const;\Const~\Entails{\TypeS}''_1$.
            The transition is as shown below:
            %
            \begin{minieq}*%\label{eq:cfgs_trans_wf_pres_wait_tgtz_unfold_trans}
        \begin{array}[c]{l}
                \infer[\LblCfgIsoUnfold]{%
                    \Trans{\CIso[\ValClocks_1];[\mu\alpha.{\TypeS}''_1]}:{\ProgAction}[\CIso'_1]
                }{%
                    \Trans{\CIso[\ValClocks_1];[{\TypeS}''_1\Subst[\mu\alpha.{\TypeS}''_1][\alpha]]}:{\ValTime}[\CIso'_1]
                }
                \end{array}
            \end{minieq}
            
            \noindent By inner induction on the different cases of ${\TypeS}''_1$:
            \begin{inductivecase}
                %
                % ~ wait -> unfold -> choice
                \item\NewCase[${\TypeS}''_1=\simplechoice$]\label{case:cfgs_trans_wf_pres_wait_tgtz_unfold_choice} 
                Then, by rule \LblCfgIsoTick*:
                \[\Trans{\CIso[\ValClocks_1];[{\simplechoice}\Subst[\mu\alpha.{\simplechoice}][\alpha]]}:{\ValTime}[\CIso[\ValClocks_1]+{+\ValTime};[\simplechoice]]\]
                
                \noindent By the rules \LblTypRec*\ and \LblTypChoice*\ the following holds:
                \[
                \infer[\LblTypRec]{%
                \emptyset;\Const_i~\Entails\mu\alpha.{\FullChoice}
                }{%
                \infer[\LblTypChoice]{%
                    \alpha:\Const_i;\Past_i~\Entails\FullChoice
                }{%
                \dots
                }
                }
                \]
                
                % \noindent It holds that \CIso*[\ValClocks_1];[{\simplechoice}\Subst[\mu\alpha.{\simplechoice}][\alpha]]\ is \emph{well-formed} and \emph{future-enabled}.
                %
                \noindent By induction hypothesis: \[\exists\Const'_i:\ValClocks_1+\ValTime\models\Past[\Const'_i] ~\land~ \emptyset;\Past[\Const'_i]~\Entails\FullChoice\] which is assured by the (persistency) premise of rule \LblCfgSocTime*, as in~\Cref{case:cfgs_trans_wf_pres_wait_tgtz_tick_choice} of~\Cref{lem:cfgs_trans_wf_pres}.
% 
                Therefore, \CIso*[\ValClocks_1]+{+\ValTime};[\mu\alpha.{\simplechoice}]\ is \emph{well-formed} as \CIso*[\ValClocks_1]+{+\ValTime};[{\simplechoice}\Subst[\mu\alpha.{\simplechoice}][\alpha]]\ the following is \emph{well-formed}.
                %
                %
                % ~ wait -> unfold -> recursion
                \item\NewCase[${\TypeS}''_1=\mu\alpha'.{\TypeS}'''_1$]\label{case:cfgs_trans_wf_pres_wait_tgtz_unfold_recursion} 
                % By the hypothesis $\exists\Const$ such that $\ValClocks_1\models\Const$ and $\emptyset;\Const~\Entails\mu\alpha.\mu\alpha'.{\TypeS}'''_1$, by the premise of rule \LblTypRec*\ $\alpha:\Const;\Const~\Entails\mu\alpha'.{\TypeS}'''_1$ and $\alpha:\Const,\alpha':\Const;\Const~\Entails{\TypeS}'''_1$, and \CIso*[\ValClocks_1];[{\TypeS}'''_1]\ is \emph{well-formed}.
                % 
                The well-formedness of \CIso*[\ValClocks_1]+{+\ValTime};[\mu\alpha'.{\TypeS}'''_1]\ depends on the well-formedness of \CIso*[\ValClocks_1]+{+\ValTime};[{\TypeS}'''_1]. (See other cases of $S$.) %, as in~\Cref{itm:wf_then_live_recdef} of~\Cref{lem:cfg_wf_then_live}.)
                %
                %
                % ~ wait -> unfold -> end
                \item\NewCase[${\TypeS}''_1=\TypeEnd$]\label{case:cfgs_trans_wf_pres_wait_tgtz_unfold_end} 
                By~\Cref{lem:cfg_wf_end} \CIso*+{+\ValTime};[\TypeEnd]\ is \emph{well-formed}.
                %
                %
                % ~ wait -> unfold -> rec call
                \item\label{case:cfgs_trans_wf_pres_wait_tgtz_unfold_reccall} ${\TypeS}''_1$ cannot equal $\alpha$ by~\Cref{lem:cfg_wf_neq_alpha}.
                %
            \end{inductivecase}
            %
        \end{inductivecase}

        \noindent Therefore, it holds that well-formedness is preserved by transition made by \emph{well-formed} configurations via rule \LblCfgSysWait*.
        %
        %
        %
        %
        %
        % ~ comm
        \item\NewCase[\LblCfgSysLComm*]\label{case:cfgs_trans_wf_pres_comm} 
        % Then by~\cref{case:configs_trans_compat_pres_comm} of~\cref{lem:configs_trans_compat_pres} $\Queue_1=\Queue'_1=\emptyset$.
        The transition is as shown below:
        %
        \begin{minieq}*%\label{eq:cfgs_trans_wf_pres_comm_trans}
        \begin{array}[c]{l}
            \infer[\LblCfgSysLComm]{%
                \Trans{\Parl{\VSoc_1,\VSoc_2}}:{\SiltAction}[\Parl{\VSoc'_1,\VSoc'_2}]
            }{%
                \infer[\LblCfgSocSend]{%
                    \Trans{\VSoc_1}:{\SendMsg}[\VSoc'_1]
                }{\dots}
                & %
                % \infer[\LblCfgSocEnqu]{%
                    \Trans{\VSoc_2}:{\RecvMsg}[\VSoc'_2]
					\quad \LblCfgSocEnqu
                % }{\dots}
            }
            \end{array}
        \end{minieq}
        
        % ~ comm-l s1
        \noindent Focusing first on \VSoc*_1, we proceed by induction on the depth of the derivation tree, analysing the last rule applied for the transition \Trans*{\VIso_1}:{\SendMsg}[\VIso'_1]:
        \begin{inductivecase}
            %
            %
            % ~ comm-l s1 -> act
            \item\NewCase[\LblCfgIsoInteract*]\label{case:cfgs_trans_wf_pres_comm_act} 
            Then $\TypeS_1=\simplechoice$, and the evaluation is shown below:
            %
            \begin{minieq}*\label{eq:cfgs_trans_wf_pres_comm_act_trans}
        \begin{array}[c]{l}
                \infer[\LblCfgSocSend]{%
                    \Trans{\CSoc[\ValClocks_1];[\simplechoice]:{\Queue_1}}:{\SendMsg}[\CSoc[\ValClocks_1]+{\ReSet[]_j};[\TypeS_j]:{\Queue_1}]
                }{%
                    \infer[\LblCfgIsoInteract]{%
                        \Trans{\CIso[\ValClocks_1];[\TypInteract]}:{\SendMsg}[\CIso[\ValClocks_1]+{\ReSet[]_j};[\TypeS_j]]
                    }{%
                        \ValClocks_1\models\Const_j
                        & %
                        {m}={l_j\left\langle T_j \right\rangle}
                        & % 
                        {\TypSend=\TypComm_j}
                        & % 
                        j\in I
                    }
                }
                \end{array}
            \end{minieq}
            
            \noindent By rule \LblTypChoice*\ $\exists\Const_i:\ValClocks_1\models\Past[\Const_i]$ and $\emptyset;\Past[\Const_i]~\Entails\FullChoice$, and by the premise of rule \LblTypChoice*\ it holds that $\Const_i\ReSet[]_i\subseteq\gamma$ and $\emptyset;\gamma~\Entails\TypeS_i$.
            %
            Combined with~\Cref{lem:configs_iso_trans} it holds that $\ValClocks_1\models\Const_j$ and $\emptyset;\Const_j\ReSet[]_j~\Entails\TypeS_j$ and $\ValClocks_1\ReSet[]_j\models\Const_j\ReSet[]_j$.
            
            Therefore, \CIso*[\ValClocks_1]+{\ReSet[]_j};[\TypeS_j]\ is \emph{well-formed}.
            %
            %
            % ~ comm-l s1 -> unfold
            \item\NewCase[\LblCfgIsoUnfold*]\label{case:cfgs_trans_wf_pres_comm_unfold} 
            Then $\TypeS_1=\mu\alpha.{\TypeS}''_1$.
            The transition is as shown below:
            %
            \begin{minieq}*\label{eq:cfgs_trans_wf_pres_comm_unfold_trans}
        \begin{array}[c]{l}
                \infer[\LblCfgIsoUnfold]{%
                    \Trans{\CIso[\ValClocks_1];[\mu\alpha.{\TypeS}''_1]}:{\ProgAction}[\CIso'_1]
                }{%
                    \Trans{\CIso[\ValClocks_1];[{\TypeS}''_1\Subst[\mu\alpha.{\TypeS}''_1][\alpha]]}:{\SendMsg}[\CIso'_1]
                }
                \end{array}
            \end{minieq}
            
            \noindent By the hypothesis $\exists\Const$ such that $\ValClocks_1\models\Const$ and $\emptyset;\Const~\Entails\TypRecDef$, and by the premise of rule \LblTypRec*\ $\alpha:\Const;\Const~\Entails{\TypeS}''_1$, and \CIso*[\ValClocks_1];[{\TypeS}''_1\Subst[\mu\alpha.{\TypeS}''_1][\alpha]]\ is \emph{well-formed}.
            %
            The well-formendess of \CIso*'_1\ is dependant on the state of ${\TypeS}''_1$, which for the transition \Trans*{\CIso[\ValClocks_1];[{\TypeS}''_1\Subst[\mu\alpha.{\TypeS}''_1][\alpha]]}:{\SendMsg}[\CIso'_1]\ must be either $\simplechoice$ or $\mu\alpha'.{\TypeS}'''_1$ (see other cases, as in~\Cref{lem:cfg_wf_then_live}).
            %
        \end{inductivecase}
        %
        % ~ comm-l s1
        Now, focusing on \VSoc*_2, the transition \Trans*{\CSoc_2}:{\RecvMsg}[\CSoc'_2]\ via rule \LblCfgSocEnqu*\ yields $\ValClocks_2'=\ValClocks_2$ and $\TypeS_2'=\TypeS_2$ and $\Queue_2'=\Queue_2;\Msg$ by~\Cref{lem:configs_soc_trans}.
        %
        Therefore \CIso*'_2\ is \emph{well-formed} as $\VIso_2=\VIso'_2$.
        %
        Transitions via rule \LblCfgSysRComm*\ are symmetric and omitted.
        %
        %
        %
        %
        %
        % ~ par-l
        \item\NewCase[\LblCfgSysLPar*]\label{case:cfgs_trans_wf_pres_par} 
        By~\Cref{lem:configs_soc_trans} $\Queue'_1=\Msg;\Queue_1$.
        The transition is as shown below: %in~\Cref{eq:cfgs_trans_wf_pres_par_trans}.
        %
        \begin{minieq}*\label{eq:cfgs_trans_wf_pres_par_trans}
        \begin{array}[c]{l}
            \infer[\LblCfgSysLPar]{%
                \Trans{\Parl{\VSoc_1,\VSoc_2}}:{\SiltAction}[\Parl{\VSoc'_1,\VSoc'_2}]
            }{%
                \infer[\LblCfgSocRecv]{%
                    \Trans{\VSoc_1}:{\SiltAction}[\VSoc'_1]
                }{\dots}
            }
            \end{array}
        \end{minieq}
        
        \noindent We proceed by induction on the depth of the derivation tree, analysing the last rule applied for the transition \Trans*{\VIso_1}:{\RecvMsg}[\VIso'_1]\ via the premise of rule \LblCfgSocRecv*:
        \begin{inductivecase}
            %
            %
            % ~ par-l -> act
            \item\NewCase[\LblCfgIsoInteract*]\label{case:cfgs_trans_wf_pres_par_act} 
            Then $\TypeS_1=\simplechoice$.
            The transition is as shown below: %in~\Cref{eq:cfgs_trans_wf_pres_par_act_trans}.
            %
            \begin{minieq}*\label{eq:cfgs_trans_wf_pres_par_act_trans}
        \begin{array}[c]{l}
                \infer[\LblCfgSocRecv]{%
                    \Trans{\CSoc[\ValClocks_1];[\simplechoice]:{\Msg;\Queue_1}}:{\SiltAction}[\CSoc[\ValClocks_1]+{\ReSet[]_j};[\TypeS_j]:{\Queue_1}]
                }{%
                    \infer[\LblCfgIsoInteract]{%
                        \Trans{\CIso[\ValClocks_1];[\TypInteract]}:{\RecvMsg}[\CIso[\ValClocks_1]+{\ReSet[]_j};[\TypeS_j]]
                    }{%
                        \ValClocks_1\models\Const_j
                        & %
                        {m}={l_j\left\langle T_j \right\rangle}
                        & % 
                        {\TypRecv=\TypComm_j}
                        & % 
                        j\in I
                    }
                }
                \end{array}
            \end{minieq}
            
            \noindent By the hypothesis and the judgement of rule \LblTypChoice*\ $\exists\Const_i$ such that $\emptyset;\Past[\Const_i]~\Entails\simplechoice$ and $\ValClocks_1\models\Past[\Const_i]$, and by the premise of rule \LblTypChoice*\ $\Const_i\ReSet[]_i\subseteq\gamma$ and $\emptyset;\gamma~\Entails\TypeS_i$.
            
            It follows~\Cref{case:cfgs_trans_wf_pres_comm_act} of~\Cref{lem:cfgs_trans_wf_pres} that \CIso*[\ValClocks_1]+{\ReSet[]_j};[\TypeS_j]\ is \emph{well-formed}.
            % Therefore \CIso*[\ValClocks_1]+{\ReSet[]_j};[\TypeS_j]\ is \emph{well-formed} as $\ValClocks_1\models\Const_j$ and $\emptyset;\Const_j\ReSet[]_j~\Entails\TypeS_j$ and $\ValClocks_1\ReSet[]_j\models\Const_j\ReSet[]_j$ (as in~\Cref{case:cfgs_trans_wf_pres_comm_act} of~\Cref{lem:cfgs_trans_wf_pres}).
            %
            %
            % ~ par-l -> unfold
            \item\NewCase[\LblCfgIsoUnfold*]\label{case:cfgs_trans_wf_pres_par_unfold} 
            Then $\TypeS_1=\mu\alpha.{\TypeS}''_1$.
            The transition shown below, and is analogous to the one in~\Cref{eq:cfgs_trans_wf_pres_comm_unfold_trans} of~\Cref{lem:cfgs_trans_wf_pres}:
            % in~\Cref{eq:cfgs_trans_wf_pres_par_unfold_trans} and is analogous to the one in~\Cref{eq:cfgs_trans_wf_pres_comm_unfold_trans}.
            %
            \begin{minieq}\label{eq:cfgs_trans_wf_pres_par_unfold_trans}
        \begin{array}[c]{l}
                \infer[\LblCfgIsoUnfold]{%
                    \Trans{\CIso[\ValClocks_1];[\mu\alpha.{\TypeS}''_1]}:{\ProgAction}[\CIso'_1]
                }{%
                    \Trans{\CIso[\ValClocks_1];[{\TypeS}''_1\Subst[\mu\alpha.{\TypeS}''_1][\alpha]]}:{\RecvMsg}[\CIso'_1]
                }
                \end{array}
            \end{minieq}
            
            \noindent By the hypothesis $\exists\Const$ such that $\ValClocks_1\models\Const$ and $\emptyset;\Const~\Entails\TypRecDef$, and by the premise of rule \LblTypRec*\ $\alpha:\Const;\Const~\Entails{\TypeS}''_1$, and \CIso*[\ValClocks_1];[{\TypeS}''_1\Subst[\mu\alpha.{\TypeS}''_1][\alpha]]\ is \emph{well-formed}.
            %
            The well-formendess of \CIso*'_1\ is dependant on ${\TypeS}''_1$, which for the transition by the premise of rule \LblCfgIsoUnfold*\ must be either $\simplechoice$ or $\mu\alpha'.{\TypeS}'''_1$ (see other cases, as in~\Cref{lem:cfg_wf_then_live}).
            %
        \end{inductivecase}
        %
    \end{inductivecase}

    \noindent Therefore, it holds that any transition made by a system composed of compatible and \emph{well-formed} configurations will result in configurations that are \emph{well-formed}.
    %
\end{proof}
% 

% ~ time passing
% ! 
% \newpage
%
% ! (lemma 13) : time passing implies empty queues
\begin{lemma}\label{lem:sys_compat_time_trans}
   %
   If \Compat*[\VSoc_1][\VSoc_2]\ and \Trans*{\Parl{\VSoc_1,\VSoc_2}}:{\ValTime}\ and $\ValTime>0$ then $\Queue_1=\emptyset=\Queue_2$.
   %
\end{lemma}
\begin{proof}
   %
   Such a transition is only specified by \LblCfgSysWait*, which by its premise requires a \LblCfgSocTime*\ transition of \ValTime*\ for each \VSoc*_1\ and \VSoc*_2.
   %
   By contradiction, if one queue were \emph{non-empty}, say $\Queue_1=\Msg;\Queue_1$, then by~\Cref{itm:configs_compat_expected_receive} of~\Cref{def:configs_compat} message \Msg*\ must be able to be received immediately.
   %
   The premise of \LblCfgSocTime*\ (urgency) ensures that \ValTime*\ must be valued such that no time passes while a message is able to be received.
   %
   % It holds that $t$ must equal 0 when there is a non-empty queue.
   
   Therefore the hypothesis holds.
   %if a system makes a $t$ transition where $t>0$ then all queues in the system must be empty.
   % Therefore, \ValTime*\ must equal $0$ when there is a message in any queue in a system composed of compatible configurations.
   %
 \end{proof}
 % 

%
% ! (lemma 15) : compat, single transition -> compat
\begin{lemma}\label{lem:configs_trans_compat_pres}
	%
	If \VIso*_1\ and \VIso*_2\ are both \emph{well-formed} 
    and \Compat*[\VSoc_1][\VSoc_2]\ 
    and \Trans*{\Parl{\VSoc_1,\VSoc_2}}[\Parl{\VSoc'_1,\VSoc'_2}], 
	then \Compat*[\VSoc'_1][\VSoc'_2].
	%
\end{lemma}
\begin{proof}
	%
	We proceed by induction on the depth of the derivation tree, analysing each case of the last rule applied for the transition \Trans*{\Parl{\VSoc_1,\VSoc_2}}[\Parl{\VSoc'_1,\VSoc'_2}]:
	% \begin{inline}+
	% 	\item \LblCfgSysWait*
	% 	\item \LblCfgSysLComm*
	% 	\item \LblCfgSysLPar*
	% \end{inline}
	%
	\begin{inductivecase}
		%
		%
		%
		%
		%
		% ~ wait
		\item\NewCase[\LblCfgSysWait*]\label{case:configs_trans_compat_pres_wait}
		Then both \VSoc*_1\ and \VSoc*_2\ make a \ValTime*\ transition via \LblCfgSocTime*\ as shown in~\Cref{eq:configs_trans_compat_pres_wait_trans}.
		%
        If $\ValTime=0$ then by~\Cref{lem:configs_iso_trans,lem:configs_soc_trans} $\VSoc_1=\VSoc'_1$ and $\VSoc_2=\VSoc'_2$ and the hypothesis holds; \Compat*[\VSoc'_1][\VSoc'_2].
        %
        \begin{minieq}\label{eq:configs_trans_compat_pres_wait_trans}
            \infer[\LblCfgSysWait]{%
                \Trans{\Parl{\VSoc_1,\VSoc_2}}:{\ValTime}[\Parl{\VSoc'_1,\VSoc'_2}]
            }{%
                \infer[\LblCfgSocTime]{%
                    \Trans{\VSoc_1}:{\ValTime}[\VSoc'_1]
                }{\dots}
                & %
                \infer[\LblCfgSocTime]{%
                    \Trans{\VSoc_2}:{\ValTime}[\VSoc'_2]
                }{\dots}
            }
        \end{minieq}
        
        \noindent If $\ValTime>0$ then by~\Cref{lem:configs_iso_trans,lem:configs_soc_trans} $\ValClocks'_1=\ValClocks_1+\ValTime$ and $\TypeS'_1=\TypeS_1$ and $\Queue'_1=\Queue_1$ (and the same for \ValClocks*'_2, \TypeS*'_2\ and \Queue*'_2).
		%
        By~\Cref{lem:cfgs_trans_wf_pres} \VIso*'_1\ and \VIso*'_2\ are both \emph{well-formed}.
		%
		By~\Cref{lem:sys_compat_time_trans} $\Queue_1=\emptyset=\Queue_2$ and by~\Cref{itm:configs_compat_dual_types} of~\Cref{def:configs_compat} $\ValClocks_1=\ValClocks_2$ and $\TypeS_1=\Dual[\TypeS_2]$.
		%
		Therefore \Compat*[\CSoc[\ValClocks_1]+{+\ValTime};[\TypeS_1]:{\emptyset}][\CSoc[\ValClocks_2]+{+\ValTime};[\Dual[\TypeS_2]]:{\emptyset}].
		%
		%
		%
		%
		%
		% ~ comm
		\item\NewCase[\LblCfgSysLComm*]\label{case:configs_trans_compat_pres_comm}
		By~\cref{lem:cfgs_trans_wf_pres} both \VIso*'_1\ and \VIso*'_2\ are \emph{well-formed}.
        The transition is as shown below: %in~\cref{eq:configs_trans_compat_pres_comm_trans}.
		%
		\begin{minieq}*\label{eq:configs_trans_compat_pres_comm_trans}
			% \begin{array}{c}%\mathllap{%
			\resizebox{\linewidth}{!}{$%
				\infer[\LblCfgSysLComm]{%
					\Trans{\Parl{\CSoc[\ValClocks_1];[\TypeS_1]:{\emptyset},\CSoc[\ValClocks_2];[\TypeS_2]:{\Queue_2}}}:{\SiltAction}[\Parl{\CSoc[\ValClocks'_1];[\TypeS'_1]:{\emptyset},\CSoc[\ValClocks_2];[\TypeS_2]:{\Queue_2;\Msg}}]
				}{%
					\infer[\LblCfgSocSend]{%
						\Trans{\CSoc[\ValClocks_1];[\TypeS_1]:{\emptyset}}:{\SendMsg}[\CSoc[\ValClocks_1]+{\ReSet[]_j};[\TypeS_j]:{\emptyset}]
					}{%
						\infer[\LblCfgIsoInteract]{%
							\Trans{\CIso[\ValClocks_1];[\TypInteract]}:{\SendMsg}[\CIso[\ValClocks_1]+{\ReSet[]_j};[\TypeS_j]]
						}{%
                        \ValClocks_1\models\Const_j
                        & %
                        {m}={l_j\left\langle T_j \right\rangle}
                        & % 
                        {\TypSend=\TypComm_j}
                        & % 
                        j\in I
						}
					}
					& %
					% \infer[\LblCfgSocEnqu]{%
						\Trans{\VSoc_2}:{\RecvMsg}[\CSoc[\ValClocks_2];[\TypeS_2]:{\Queue_2;\Msg}]
						\quad \LblCfgSocEnqu
					% }{\dots}
				}
			$}%
	%	}\end{array}
		\end{minieq}
		
        \noindent We proceed by inner induction on each combination of the contents of queues:
		\begin{inductivecase}
			%
			%
			% ~ comm -> e e
			\item\NewCase[$\Queue_1=\emptyset$, $\Queue_2=\emptyset$]\label{case:configs_trans_compat_pres_comm_ee}
			By~\Cref{itm:configs_compat_dual_types} of~\Cref{def:configs_compat} $\ValClocks_1=\ValClocks_2$ and $\TypeS_1=\Dual[\TypeS_2]$.
			%
			The resulting system is no longer \emph{dual}.
			%
			By~\cref{lem:cfgs_trans_wf_pres,lem:sys_compat_time_trans} time cannot pass if $\Queue_2\neq\emptyset$.
			%
			By~\Cref{def:types_dual} the message \Msg*\ sent by \VSoc*_1\ must have a corresponding receiving action in \VSoc*_2\ as in~\Cref{itm:configs_compat_expected_receive} of~\Cref{def:configs_compat}.
			%
			Therefore \Compat*[\CSoc[\ValClocks'_1];[{\TypeS}'_1]:{\emptyset}][\CSoc;+{\Msg}_2].
			%
			%
			%
			% ~ comm -> e m
			\item\NewCase[$\Queue_1=\emptyset$, $\Queue_2\neq\emptyset$]\label{case:configs_trans_compat_pres_comm_en}
			By~\Cref{itm:configs_compat_expected_receive} of~\Cref{def:configs_compat} $\exists\Msg',{\ValClocks}'',{\TypeS}''$ such that $\Queue_2=\Msg';\Queue_2$ and \Trans*{\CIso_2}:{\TypRecv,\Msg'}[\CIso[{\ValClocks}''_2];[{\TypeS}''_2]]\ and \Compat*[\CSoc[\ValClocks'_1];[{\TypeS}'_1]:{\emptyset}][\CSoc[{\ValClocks}''_2];[{\TypeS}''_2]:{{\Queue}_2}]\ and by \LblCfgSocTime*\ (urgency) time cannot pass.
			%
			If a system has a configuration with sequence of outgoing sending actions and each has constraints that are satisfiable immediately after the other, then the system can both receive the messages as they arrive, or accumulate the messages and instantly receive each in succession and become \emph{dual} again (by inspection of~\Cref{def:types_dual,def:configs_compat} and~\Cref{fig:types_rule,fig:typesemantics_tuple,fig:typesemantics_triple}).
			%
			Therefore \Compat*[\CSoc[\ValClocks'_1];[{\TypeS}'_1]:{\emptyset}][\CSoc[{\ValClocks}''_2];[{\TypeS}''_2]:{\Msg';{\Queue}_2;\Msg}].
			%
			%
			% ~ comm -> m e
			\item\NewCase[$\Queue_1\neq\emptyset$, $\Queue_2=\emptyset$]\label{case:configs_trans_compat_pres_comm_ne}
			Contradicts the hypothesis by~\Cref{itm:configs_compat_expected_receive} of~\Cref{def:configs_compat} as by \LblCfgSocTime*\ (urgency) messages must be removed from a queue immediately, and by \LblTypChoice*\ of~\Cref{fig:types_rule} sending and receiving actions cannot be performed at the same time.
			%
			%
			% ~ comm -> m m
			\item\NewCase[$\Queue_1\neq\emptyset$, $\Queue_2\neq\emptyset$]\label{case:configs_trans_compat_pres_comm_nn}
			Contradicts the hypothesis by~\Cref{itm:configs_compat_non_empty_queues} of~\Cref{def:configs_compat}.
			%
		\end{inductivecase}

		\noindent Therefore, compatibility is preserved across \LblCfgSysLComm*\ transitions.
		%
		%
		%
		%
		%
		% ~ dequ
		\item\NewCase[\LblCfgSysLPar*]\label{case:configs_trans_compat_pres_dequ}
		By~\Cref{lem:configs_soc_trans} $\Queue_2=\Msg;\Queue_2$ and by~\Cref{itm:configs_compat_expected_receive} of~\Cref{def:configs_compat} \Compat*[\VSoc'_1][\VSoc'_2], the hypothesis holds.
		%
	\end{inductivecase}

	\noindent Therefore, it holds that any transition made by a compatible system composed of well-formed types will result in configurations that are \emph{compatible}.
	%
\end{proof}
% 

% ~ preservation is preserved 
% ! 
%
% ! (lemma 16) : compat wf, end or future enabled
\begin{lemma}\label{lem:configs_compat_wf_fe}
	%
	If both \VIso*_1\ and \VIso*_2\ are \emph{well-formed} 
    and \Compat*[\VSoc_1][\VSoc_2],
	then both \VSoc*_1\ and \VSoc*_2\ are \emph{final} 
	or $\exists\ValTime$ such that \Trans*{\VSys}:{\ValTime,\SiltAction}[\Parl{\VSoc'_1,\VSoc'_2}].%
	%
\end{lemma}
\begin{proof}
	%
	By~\Cref{lem:cfgs_trans_wf_pres} \VIso*'_1\ and \VIso*'_2\ are \emph{well-formed} and by~\Cref{lem:cfg_wf_then_live} are \emph{live}.
	%
	By~\Cref{def:types_progress} if \VSoc*_2\ is \emph{final} then $\VSoc_2=\CSoc[\ValClocks_2];[\TypeEnd]:{\emptyset}$.
    %
    We proceed with the assumption that at least one participant is \emph{not final}, and hereafter only consider \VSoc*'_1.
	%
	The transition is given below:
 %in~\Cref{eq:configs_compat_wf_fe_trans}.
	%
	\begin{minieq}*\label{eq:configs_compat_wf_fe_trans}
		\Trans{\VSys}:{\ValTime}[\Trans{\Parl{\CSoc+{+\ValTime}_1,\CSoc+{+\ValTime}_2}}:{\SiltAction}[\Parl{\CSoc'_1,\CSoc'_2}]]
	\end{minieq}
	%
	We proceed only considering each case of \VSoc*_1\ not being \emph{final}.
	By induction on the cases of \TypeS*_1:
	%
	\begin{inductivecase}
		%
		%
		% ~ choice
		\item\NewCase[$\TypeS=\simplechoice$] 
		% Then by judgement of \LblTypChoice*\ $\exists\Const_i$ such that $\ValClocks\models\Past[\Const_i]$ and $\emptyset;\Past[\Const_i]~\Entails\simplechoice$.
		%
		As described in~\Cref{sec:types} we write $\Past$ if $\exists\ValTime$ such that $\ValClocks+\ValTime\models\Const$.
		Therefore, if $\ValClocks\not\models\Const$ and $\emptyset;\Const~\Entails\TypeS$ for a \emph{well-formed} \CIso*\ then $\ValClocks\models\Past[\Const]$.
		%
		By~\Cref{lem:cfgs_trans_wf_pres} the only possible values of \ValTime*\ ensure that the latest system-wide sending action is never missed and messages are received as soon as they arrive in a queue by rule \LblCfgSocTime*.
		
		Therefore, the hypothesis holds for systems composed where one participant is known to be a \emph{non-final} choice type.
		%
		%
		% ~ rec def
		\item\NewCase[$\TypeS=\mu\alpha.\TypeS'$]
		It follows~\Cref{lem:cfgs_trans_wf_pres} that \TypeS*'\ is \emph{well-formed} against $\ValClocks+\ValTime$.
		%
	\end{inductivecase}
	
	\noindent Therefore, if a \emph{well-formed} and compatible system \Parl*{\VSoc_1,\VSoc_2}\ that is not \emph{final}, then there is some value of time $\ValTime\geq 0$ that will enable a future action, which will result in a \emph{well-formed} and compatible system \Parl*{\VSoc'_1,\VSoc'_2}, which may or may not be \emph{final}, and to which this behaviour still applies.
	%
\end{proof}
% 
%
% ! (lemma 18) : compat wf, any amount of transitions -> compat wf
\begin{lemma}\label{lem:configs_trans_compat_wf_pres}
    %
    If both \VIso*_1 and \VIso*_2 are \emph{well-formed}
    and \Compat*[\VSoc_1][\VSoc_2]\ 
    and $\Trans{\Parl{\VSoc_1\!,\VSoc_2}}*[\Parl{\VSoc'_1\!,\VSoc'_2}]$, 
    then \Compat*[\VSoc'_1][\VSoc'_2]\ 
    and both \VIso*'_1\ and \VIso*'_2\ are \emph{well-formed}.%
    %
\end{lemma}
\begin{proof}
    %
    By~\Cref{lem:cfgs_trans_wf_pres,lem:configs_trans_compat_pres} the hypothesis holds for single transitions and that the resulting configurations are \emph{live}, and either \emph{final} or \emph{satisfies progress} by~\Cref{lem:cfg_wf_then_live,lem:configs_compat_wf_fe}.
    %
    Therefore it holds that \Compat*[\VSoc'_1][\VSoc'_2]\ 
    and both \VIso*'_1\ and \VIso*'_2\ are \emph{well-formed} across an arbitrary number of transitions, as each single transition preserves compatibility and well-formedness.
    %
\end{proof}
% 

% ~ system progress
% ! 
%
% ! (lemma 20) : sys cfg progress
\begin{lemma}\label{lem:configs_sys_progress}
    %
    For all \TypeS*, \ValClocks* such that \ITJudgement*[\emptyset];{\Const}[\TypeS]\ and \Sat*\ :\par\noindent
    \hfill\ \Parl*{\CSoc:{\emptyset},\CSoc;[\Dual]:{\emptyset}}\ satisfies progress.\hfill\ \ %
    %
\end{lemma}
\begin{proof}
    %
    By the hypothesis the system is compatible and composed of well-formed dual types. 
    %
    By~\Cref{lem:configs_trans_compat_wf_pres} any configurations reachable by such a system will be compatible and \emph{well-formed}.
    %
    By~\Cref{lem:configs_compat_wf_fe} such a system adheres to~\Cref{def:types_progress} and \emph{satisfies progress}.
    %
\end{proof}
% 

% ! thm proof
\ThmProgress*
\begin{proof}
   %
   By~\Cref{def:types_wf}, types \TypeS*\ and \Dual*\ are always \emph{well-formed} against \ValClocks*_0.
   %
   By \Cref{lem:init_wf_then_live} and~\Cref{itm:configs_compat_dual_types} of~\Cref{def:configs_compat}, both \CSoc*[\ValClocks_0]:{\emptyset}\ and \CSoc*[\ValClocks_0];[\Dual]:{\emptyset} are \emph{live} and \emph{compatible}.
   %
   It follows~\Cref{lem:configs_compat_wf_fe,lem:configs_trans_compat_wf_pres} that such as system will always perform actions when possible, waiting if necessary (never missing the \emph{latest-enabled} action), until reaching a \emph{final} configuration, and any \emph{non-final} configuration is guaranteed to be \emph{well-formed}, \emph{compatible} and \emph{live}.

   Therefore, it holds that an initial system composed of dual types that are well-formed is compatible, and guaranteed to \emph{satisfy progress}.
   %
\end{proof}



\subsection{Subject Reduction}

\begin{lemma}[Remove tautology] \label{lem:rm-tauto} \quad
    \begin{enumerate}
        \item If $\jdwf{}{\Gamma, \phi, \Gamma'}$, then $\jdwf{}{\Gamma, \Gamma'}$.
        \item 
            \begin{itemize}
                \item If $\jdwf{\Gamma, \phi, \Gamma'}{T}$, then $\jdwf{\Gamma, \Gamma'}{T}$.
                \item If $\jdwf{\Gamma, \phi, \Gamma'}{C}$, then $\jdwf{\Gamma, \Gamma'}{C}$.
                \item If $\jdwf{\Gamma, \phi, \Gamma'}{\Sigma}$, then $\jdwf{\Gamma, \Gamma'}{\Sigma}$.
                \item If $\jdwf{\Gamma, \phi, \Gamma' \mid T}{S}$, then $\jdwf{\Gamma, \Gamma' \mid T}{S}$.
            \end{itemize}
        \item Assume that $\vDash \phi$.
            \begin{itemize}
                \item If $\jdty{\Gamma, \phi, \Gamma'}{v}{T}$, then $\jdty{\Gamma, \Gamma'}{v}{T}$.
                \item If $\jdty{\Gamma, \phi, \Gamma'}{c}{C}$, then $\jdty{\Gamma, \Gamma'}{c}{C}$.
            \end{itemize}
        \item Assume that $\vDash \phi$.
            \begin{itemize}
                \item If $\jdsub{\Gamma, \phi, \Gamma'}{T_1'}{T_2'}$, then $\jdsub{\Gamma, \Gamma'}{T_1'}{T_2'}$.
                \item If $\jdsub{\Gamma, \phi, \Gamma'}{C_1}{C_2}$, then $\jdsub{\Gamma, \Gamma'}{C_1}{C_2}$.
                \item If $\jdsub{\Gamma, \phi, \Gamma'}{\Sigma_1}{\Sigma_2}$, then $\jdsub{\Gamma, \Gamma'}{\Sigma_1}{\Sigma_2}$.
                \item If $\jdsub{\Gamma, \phi, \Gamma' \mid T}{S_1}{S_2}$, then $\jdsub{\Gamma, \Gamma' \mid T}{S_1}{S_2}$.
            \end{itemize}
    \end{enumerate}
\end{lemma}
\begin{proof} \quad
    % depends on:: asm:formula lem:rm-unused
    \begin{enumit}
        \item Immediate by Lemma \ref{lem:rm-unused}.
        \item Immediate by Lemma \ref{lem:rm-unused}.
        \item By simultaneous induction on the derivations.
        \item By simultaneous induction on the derivations.
            The case for \rulename{S-Rfn} uses Assumption \ref{asm:formla}.
    \end{enumit}
\end{proof}

\begin{lemma}[Reflexivity of subtyping] \label{lem:refl} \quad
    \begin{enumerate}
        \item If $\jdwf{\Gamma}{T}$, then $\jdsub{\Gamma}{T}{T}$.
        \item If $\jdwf{\Gamma}{C}$, then $\jdsub{\Gamma}{C}{C}$.
        \item If $\jdwf{\Gamma}{\Sigma}$, then $\jdsub{\Gamma}{\Sigma}{\Sigma}$.
        \item If $\jdwf{\Gamma \mid T}{S}$, then $\jdsub{\Gamma \mid T}{S}{S}$.
    \end{enumerate}
\end{lemma}
\begin{proof}
    % depend on:: asm:formula
    By simultaneous induction on the derivations.
    The case for \rulename{WT-Rfn} uses Assumption \ref{asm:formla}.
\end{proof}

\begin{lemma}[Transitivity of subtyping] \label{lem:trans} \quad
    \begin{enumerate}
        \item If $\jdsub{\Gamma}{T_1}{T_2}$ and $\jdsub{\Gamma}{T_2}{T_3}$, then $\jdsub{\Gamma}{T_1}{T_3}$.
        \item If $\jdsub{\Gamma}{C_1}{C_2}$ and $\jdsub{\Gamma}{C_2}{C_3}$, then $\jdsub{\Gamma}{C_1}{C_3}$.
        \item If $\jdsub{\Gamma}{\Sigma_1}{\Sigma_2}$ and $\jdsub{\Gamma}{\Sigma_2}{\Sigma_3}$, then $\jdsub{\Gamma}{\Sigma_1}{\Sigma_3}$.
        \item If $\jdsub{\Gamma \mid T}{S_1}{S_2}$ and $\jdsub{\Gamma \mid T}{S_2}{S_3}$, then $\jdsub{\Gamma \mid T}{S_1}{S_3}$.
    \end{enumerate}
\end{lemma}
% depends on:: asm:formla lem:narrow lem:weaken
\begin{proof}
    By simultaneous induction on the structure of $T_2, C_2, \Sigma_2$ and $S_2$.
    \begin{enumit}
        \item Case analysis on $\jdsub{\Gamma}{T_1}{T_2}$.
            \begin{description}
                \item[Case \rulename{S-Rfn}:] By inversion, Assumption \ref{asm:formla} and \rulename{S-Rfn}.
                \item[Case \rulename{S-Fun}:] By inversion, the IHs, Lemma \ref{lem:narrow}, and \rulename{S-Fun}.
            \end{description}
        \item By inversion of the both derivations, we have
            \def\currentprefix{trans:opsig}
            \begin{enumrm}
                \item\llabel{eq-sig1} $\Sigma_1 = \{ \repi{\op_i: \forall \rep{X_i: \rep{B}_i}. F_{1i}},
                    \repi{\op'_i: \forall \rep{X'_i: \rep{B'}_i}. F'_{1i}},
                    \repi{\op''_i: \forall \rep{X''_i: \rep{B''}_i}. F''_{1i}} \}$,
                \item\llabel{eq-sig2} $\Sigma_2 = \{ \repi{\op_i: \forall \rep{X_i: \rep{B}_i}. F_{2i}},
                    \repi{\op'_i: \forall \rep{X'_i: \rep{B'}_i}. F'_{2i}} \}$,
                \item\llabel{eq-sig3} $\Sigma_3 = \{ \repi{\op_i: \forall \rep{X_i: \rep{B}_i}. F_{2i}} \}$,
                \item\llabel{sub-f1} $\repi{\jdsub{\Gamma, \rep{X_i: \rep{B}_i}}{F_{1i}}{F_{2i}}}$,
                \item\llabel{sub-f2} $\repi{\jdsub{\Gamma, \rep{X_i: \rep{B}_i}}{F_{2i}}{F_{3i}}}$, and
                \item\llabel{sub-f1'} $\repi{\jdsub{\Gamma, \rep{X'_i: \rep{B'}_i}}{F'_{1i}}{F'_{2i}}}$~.
            \end{enumrm}
            By the IH with \lref{sub-f1} and \lref{sub-f2}, we have
            $\repi{\jdsub{\Gamma, \rep{X_i: \rep{B}_i}}{F_{1i}}{F_{3i}}}$~.
            By \rulename{S-Sig}, we have the conclusion.
        \item By inversion, the IHs, Lemma \ref{lem:narrow}, and \rulename{S-Comp}.
        \item Case analysis on $\jdsub{\Gamma}{S_1}{S_2}$.
        \begin{description}
            \item[Case \rulename{S-Pure}:]
                Since we have $S_1 = \square = S_2$,
                we have the conclusion immediately from $\jdsub{\Gamma}{S_2}{S_3}$.
            \item[Case \rulename{S-ATM}:] We have
                \def\currentprefix{trans:atm}
                \begin{enumrm}
                    \item\llabel{eq-S1} $S_1 = \tyctl{x}{C_{11}}{C_{12}}$,
                    \item\llabel{eq-S2} $S_2 = \tyctl{x}{C_{21}}{C_{22}}$,
                    \item\llabel{sub-C21} $\jdsub{\Gamma, x:T}{C_{21}}{C_{11}}$, and
                    \item\llabel{sub-C12} $\jdsub{\Gamma}{C_{12}}{C_{22}}$
                \end{enumrm}
                for some $x, C_{11}, C_{12}, C_{21}$, and $C_{22}$.
                Since \lref{eq-S2}, the only rule applicable to $\jdsub{\Gamma}{S_2}{S_3}$ is \rulename{S-ATM}.
                Therefore, by inversion we have
                \begin{enumrm}[resume]
                    \item\llabel{eq-S3} $S_3 = \tyctl{x}{C_{31}}{C_{32}}$,
                    \item\llabel{sub-C31} $\jdsub{\Gamma, x:T}{C_{31}}{C_{21}}$, and
                    \item\llabel{sub-C22} $\jdsub{\Gamma}{C_{22}}{C_{32}}$
                \end{enumrm}
                for some $C_{31}$ and $C_{32}$.
                By the IHs, we have
                \begin{itemize}
                    \item $\jdsub{\Gamma, x:T}{C_{31}}{C_{11}}$ and
                    \item $\jdsub{\Gamma}{C_{12}}{C_{32}}$~.
                \end{itemize}
                We have the conclusion by \rulename{S-ATM}.
            \item[Case \rulename{S-Embed}:] We have
            \def\currentprefix{trans:emb}
            \begin{enumrm}
                \item\llabel{eq-S1} $S_1 = \square$,
                \item\llabel{eq-S2} $S_2 = \tyctl{x}{C_{21}}{C_{22}}$,
                \item\llabel{sub-C21} $\jdsub{\Gamma, x:T}{C_{21}}{C_{22}}$, and
                \item\llabel{in-x} $x \notin \fv(C_{22})$
            \end{enumrm}
            for some $x, C_{21}$, and $C_{22}$.
            Since \lref{eq-S2}, the only rule applicable to $\jdsub{\Gamma}{S_2}{S_3}$ is \rulename{S-ATM}.
            Therefore, by inversion we have
            \begin{enumrm}[resume]
                \item\llabel{eq-S3} $S_3 = \tyctl{x}{C_{31}}{C_{32}}$,
                \item\llabel{sub-C31} $\jdsub{\Gamma, x:T}{C_{31}}{C_{21}}$, and
                \item\llabel{sub-C22} $\jdsub{\Gamma}{C_{22}}{C_{32}}$
            \end{enumrm}
            for some $C_{31}$ and $C_{32}$.
            W.l.o.g., we can assume that $x \notin \fv(C_{32})$.
            Then, by Lemma \ref{lem:weaken} with \lref{sub-C22}, we have
            \begin{enumrm}[resume]
                \item\llabel{sub-C22-2} $\jdsub{\Gamma, x:T}{C_{22}}{C_{32}}$~.
            \end{enumrm}
            By the IHs with \lref{sub-C21}, \lref{sub-C21} and \lref{sub-C22-2},
            we have $\jdsub{\Gamma, x:T}{C_{31}}{C_{32}}$.
            Then we have the conclusion by \rulename{S-Embed}.
        \end{description}
    \end{enumit}
    
\end{proof}

\begin{lemma}[Subtyping with equal variables] \label{lem:sub-eq}
    \quad
    \begin{itemize}
        \item If $\jdwf{\Gamma, x: \tyrfn{z}{B}{z = y}, \Gamma'}{T}$, then
            $\jdsub{\Gamma, x: \tyrfn{z}{B}{z = y}, \Gamma'}{T}{T[y/x]}$~.
        \item If $\jdwf{\Gamma, x: \tyrfn{z}{B}{z = y}, \Gamma'}{T}$, then
            $\jdsub{\Gamma, x: \tyrfn{z}{B}{z = y}, \Gamma'}{T[y/x]}{T}$~.
        \item If $\jdwf{\Gamma, x: \tyrfn{z}{B}{z = y}, \Gamma'}{C}$, then
            $\jdsub{\Gamma, x: \tyrfn{z}{B}{z = y}, \Gamma'}{C}{C[y/x]}$~.
        \item If $\jdwf{\Gamma, x: \tyrfn{z}{B}{z = y}, \Gamma'}{C}$, then
            $\jdsub{\Gamma, x: \tyrfn{z}{B}{z = y}, \Gamma'}{C[y/x]}{C}$~.
        \item If $\jdwf{\Gamma, x: \tyrfn{z}{B}{z = y}, \Gamma'}{\Sigma}$, then
            $\jdsub{\Gamma, x: \tyrfn{z}{B}{z = y}, \Gamma'}{\Sigma}{\Sigma[y/x]}$~.
        \item If $\jdwf{\Gamma, x: \tyrfn{z}{B}{z = y}, \Gamma'}{\Sigma}$, then
            $\jdsub{\Gamma, x: \tyrfn{z}{B}{z = y}, \Gamma'}{\Sigma[y/x]}{\Sigma}$~.
        \item If $\jdwf{\Gamma, x: \tyrfn{z}{B}{z = y}, \Gamma' \mid T}{S}$, then
            $\jdsub{\Gamma, x: \tyrfn{z}{B}{z = y}, \Gamma' \mid T}{S}{S[y/x]}$~.
        \item If $\jdwf{\Gamma, x: \tyrfn{z}{B}{z = y}, \Gamma' \mid T}{S}$, then
            $\jdsub{\Gamma, x: \tyrfn{z}{B}{z = y}, \Gamma' \mid T}{S[y/x]}{S}$~.
    \end{itemize}
\end{lemma}
\begin{proof}
    % depend on:: asm:formula lem:narrow
    By simultaneous induction on the derivations.
    The case for \rulename{WT-Rfn} uses Assumption~\ref{asm:formla}.
    The cases for \rulename{WT-Fun} and \rulename{WT-Comp} uses Lemma~\ref{lem:narrow}.
\end{proof}

\begin{lemma}[Inversion] \label{lem:inv} \quad
    \begin{enumerate}
        \item If $\jdty{\Gamma}{p}{T}$, then
            \begin{itemize}
                \item $\jdsub{\Gamma}{\ty(p)}{T}$, and
                \item $\jdty{\Gamma}{p}{\ty(p)}$.
            \end{itemize}
        \item If $\jdty{\Gamma}{\exprec{f}{x}{c}}{(x: T) \rarr C}$, then
            there exist some $T_0$ and $C_0$ such that
            \begin{itemize}
                \item $\jdty{\Gamma}{\exprec{f}{x}{c}}{(x: T_0) \rarr C_0}$,
                \item $\jdsub{\Gamma}{(x: T_0) \rarr C_0}{(x: T) \rarr C}$, and
                \item $\jdty{\Gamma, f: (x: T_0) \rarr C_0, x: T_0}{c}{C_0}$.
            \end{itemize}
        \item If $\jdty{\Gamma}{\expret{v}}{\tycomp{\Sigma}{T}{S}}$, then
            there exist some $T'$ such that
            \begin{itemize}
                \item $\jdsub{\Gamma}{T'}{T}$,
                \item $\jdty{\Gamma}{v}{T'}$, and
                \item $\jdsub{\Gamma \mid T'}{\square}{S}$.
            \end{itemize}
        % \item If $\jdty{\Gamma}{\op~v}{\tycomp{\Sigma}{T}{S}}$, then
            % there exist some $\rep{X}, \rep{\rep{B}}, \rep{A}, x, y, \Sigma', T_1, T_2, C_1$, and $C_2$ such that
            % \begin{itemize}
            %     \item $\Sigma' \ni \op: \forall \rep{X: \rep{B}}. (x: T_1) \rarr ((y: T_2) \rarr C_1) \rarr C_2$,
            %     \item $\rep{\jdty{\Gamma}{A}{\rep{B}}}$,
            %     \item $\jdty{\Gamma}{v}{T_1[\rep{A/X}]}$, and
            %     \item $\jdsub{\Gamma}{\tycomp{\Sigma'}{T_2[\rep{A/X}][v/x]}{\tyctl{y}{C_1[\rep{A/X}][v/x]}{C_2[\rep{A/X}][v/x]}}}{\tycomp{\Sigma}{T}{S}}$~.
            % \end{itemize}
        \item If $\jdty{\Gamma}{\op~v}{\tycomp{\Sigma}{T}{S}}$, then
            there exist some $\rep{X}, \rep{\rep{B}}, \rep{A}, x, y, T_1, T_2, C_1, C_2, C_{01}$, and $C_{02}$ such that
            \begin{itemize}
                \item $S = \tyctl{y}{C_{01}}{C_{02}}$,
                \item $\Sigma \ni \op: \forall \rep{X: \rep{B}}. (x: T_1) \rarr ((y: T_2) \rarr C_1) \rarr C_2$,
                \item $\rep{\jdty{\Gamma}{A}{\rep{B}}}$,
                \item $\jdty{\Gamma}{v}{T_1[\rep{A/X}]}$,
                \item $\jdsub{\Gamma}{T_2[\rep{A/X}][v/x]}{T}$,
                \item $\jdsub{\Gamma, y: T_2[\rep{A/X}][v/x]}{C_{01}}{C_1[\rep{A/X}][v/x]}$, and
                \item $\jdsub{\Gamma}{C_2[\rep{A/X}][v/x]}{C_{02}}$~.
            \end{itemize}
        % \item If $\jdty{\Gamma}{\explet{x}{c_1}{c_2}}{\tycomp{\Sigma}{T}{S}}$, then
            % there exist some $\Sigma', T_1, T_2, C, C_{12}$ and $C_{21}$ such that
            % \begin{itemize}
            %     \item $\jdty{\Gamma}{c_1}{\tycomp{\Sigma'}{T_1}{\tyctl{x}{C}{C_{12}}}}$,
            %     \item $\jdty{\Gamma, x: T_1}{c_2}{\tycomp{\Sigma'}{T_2}{\tyctl{z}{C_{21}}{C}}}$,
            %     \item $x \notin \fv(T_2) \cup \fv(\Sigma') \cup (\fv(C_{21}) \setminus \{z\})$, and
            %     \item $\jdsub{\Gamma}{\tycomp{\Sigma'}{T_2}{\tyctl{z}{C_{21}}{C_{12}}}}{\tycomp{\Sigma}{T}{\tyctl{z}{C_1}{C_2}}}$~.
            % \end{itemize}
        \item If $\jdty{\Gamma}{\explet{x}{c_1}{c_2}}{\tycomp{\Sigma}{T}{\square}}$, then
            there exists some $T_1$ such that
            \begin{itemize}
                \item $\jdty{\Gamma}{c_1}{\tycomp{\Sigma}{T_1}{\square}}$,
                \item $\jdty{\Gamma, x: T_1}{c_2}{\tycomp{\Sigma}{T}{\square}}$, and
                \item $x \notin \fv(T) \cup \fv(\Sigma)$~.
            \end{itemize}
        \item If $\jdty{\Gamma}{\explet{x}{c_1}{c_2}}{\tycomp{\Sigma}{T}{\tyctl{z}{C_1}{C_2}}}$, then
            there exist some $T_1$ and $C_0$ such that
            \begin{itemize}
                \item $\jdty{\Gamma}{c_1}{\tycomp{\Sigma}{T_1}{\tyctl{x}{C_0}{C_2}}}$,
                \item $\jdty{\Gamma, x: T_1}{c_2}{\tycomp{\Sigma}{T}{\tyctl{z}{C_1}{C_0}}}$, and
                \item $x \notin \fv(T) \cup \fv(\Sigma) \cup (\fv(C_1) \setminus \{z\})$~.
            \end{itemize}
    \end{enumerate}
\end{lemma}
% depends on:: lem:refl lem:trans lem:narrow
\begin{proof}
    \def\currentprefix{inv}
    By induction on the derivations.
    \begin{enumit}
        \item Straightforward with Lemma \ref{lem:refl} and \ref{lem:trans}.
        \item Straightforward with Lemma \ref{lem:refl} and \ref{lem:trans}.
        \item Straightforward with Lemma \ref{lem:refl} and \ref{lem:trans}.
        \item 
        \begin{description}
            \item[Case \rulename{T-Op}:] Obvious with Lemma \ref{lem:refl}.
            \item[Case \rulename{T-CSub}:] We have
                \begin{enumrm}
                    \item\llabel{ty-op} $\jdty{\Gamma}{\expop{v}{y}{c}}{\tycomp{\Sigma'}{T'}{S'}}$,
                    \item\llabel{sub-C} $\jdsub{\Gamma}{\tycomp{\Sigma'}{T'}{S'}}{\tycomp{\Sigma}{T}{S}}$, and
                    \item $\jdwf{\Gamma}{\tycomp{\Sigma}{T}{S}}$
                \end{enumrm}
                for some $\Sigma', T'$, and $C'$.
                W.l.o.g., we can assume that $y \notin \fv(\Sigma) \cup \fv(T)$.
                By inversion of \lref{sub-C}, we have
                \begin{enumrm}[resume]
                    \item\llabel{sub-sig} $\jdsub{\Gamma}{\Sigma}{\Sigma'}$,
                    \item\llabel{sub-T} $\jdsub{\Gamma}{T'}{T}$, and
                    \item\llabel{sub-S} $\jdsub{\Gamma \mid T'}{S'}{S}$~.
                \end{enumrm}
                By the IH on \lref{ty-op}, we have
                \begin{enumrm}[resume]
                    \item\llabel{in-sig} $\Sigma' \ni \op \forall \rep{X: \rep{B}}. \op: (x: T_1) \rarr ((y: T_2) \rarr C_1) \rarr C_2$,
                    \item\llabel{wf-A} $\rep{\jdty{\Gamma}{A}{\rep{B}}}$,
                    \item\llabel{ty-v} $\jdty{\Gamma}{v}{T_1[\rep{A/X}]}$,
                    \item\llabel{sub-T'} $\jdsub{\Gamma}{T''}{T'}$,
                    \item\llabel{ty-c} $\jdty{\Gamma, y: T_2[\rep{A/X}][v/x]}{c}{\tycomp{\Sigma}{T''}{\tyctl{z}{C_0}{C_1[\rep{A/X}][v/x]}}}$,
                    \item\llabel{sub-S'} $\jdsub{\Gamma \mid T''}{\tyctl{z}{C_0}{C_2[\rep{A/X}][v/x]}}{S'}$, and
                    \item\llabel{in-y} $y \notin \fv(\Sigma') \cup \fv(T') \cup \fv(T'') \cup (\fv(C_0) \setminus \{ z \})$
                \end{enumrm}
                for some $\rep{X}, \rep{\rep{B}}, \rep{A}, x, z, T'', T_1, T_2, C_0, C_1$, and $C_2$.
                From the assumption on $y$ and \lref{in-y}, we have
                \begin{enumrm}[resume]
                    \item \llabel{in-y'} $y \notin \fv(\Sigma) \cup \fv(T) \cup \fv(T'') \cup (\fv(C_0) \setminus \{ z \})$~.
                \end{enumrm}
                Inversion on \lref{sub-sig} implies that all field in $\Sigma'$ is also in $\Sigma$,
                and therefore we have
                \begin{enumrm}[resume]
                    \item\llabel{in-sig'} $\Sigma \ni \op: \forall \rep{X: \rep{B}}. (x: T_1) \rarr ((y: T_2) \rarr C_1) \rarr C_2$
                \end{enumrm}
                from \lref{in-sig}.
                Lemma \ref{lem:narrow} with \lref{sub-T'} and \lref{sub-S} implies
                \begin{enumrm}[resume]
                    \item\llabel{sub-S-str} $\jdsub{\Gamma \mid T''}{S'}{S}$~.
                \end{enumrm}
                Lemma \ref{lem:trans} with \lref{sub-T}, \lref{sub-T'}, \lref{sub-S-str} and \lref{sub-S'} implies
                $\jdsub{\Gamma}{T''}{T}$ and $\jdsub{\Gamma \mid T''}{\tyctl{z}{C_0}{C_2[\rep{A/X}][v/x]}}{S}$~.
                From these two with \lref{wf-A}, \lref{ty-v}, \lref{ty-c}, \lref{in-y'} and \lref{in-sig'},
                we have the conclusion.
        \end{description}
    \end{enumit}
\end{proof}

\begin{lemma}[Inversion with pure evaluation contexts] \label{lem:inv-ctx}
    If $\jdty{\Gamma}{K[c]}{\tycomp{\Sigma}{T}{\tyctl{z}{C_1}{C_2}}}$, then
    there exist some $y, T_1$, and $C_0$ such that
    \begin{itemize}
        \item $\jdty{\Gamma}{c}{\tycomp{\Sigma}{T_1}{\tyctl{y}{C_0}{C_2}}}$ and
        \item $\jdty{\Gamma, y: T_1}{K[\expret{y}]}{\tycomp{\Sigma}{T}{\tyctl{z}{C_1}{C_0}}}$~.
    \end{itemize}
\end{lemma}
% depends on :: lem:wft lem:wfg lem:sub-eq
%               lem:notin-nonrfn lem:refl lem:inv lem:weaken
\begin{proof}
    By induction on the structure of $K$.
    \begin{description}
        \item[{Case $K = [\ ]$:}]
            \def\currentprefix{inv-ctx:empty}
            We have $\jdty{\Gamma}{c}{\tycomp{\Sigma}{T}{\tyctl{z}{C_1}{C_2}}}$.
            By $\alpha$-renaming, we have
            \begin{enumrm}
                \item\llabel{ty-c} $\jdty{\Gamma}{c}{\tycomp{\Sigma}{T}{\tyctl{y}{C_1[y/z]}{C_2}}}$~.
            \end{enumrm}
            Therefore, we have the first half of the conclusion with $T_1 = T$ and $C_0 = C_1[y/z]$.

            On the other hand, from \lref{ty-c}, it holds that
            \begin{enumrm}[resume]
                \item\llabel{wf-gy} $\jdwf{}{\Gamma, y: T}$
            \end{enumrm}
            by Lemma~\ref{lem:wft}, Lemma~\ref{lem:wfg}, and inversion.
            We show the second half of the conclusion by case analysis on $T$.
            \begin{description}
                \item[Case that $T$ is a refinement type $\tyrfn{z_0}{B}{\phi}$:]
                    By \rulename{T-CVar} and \rulename{T-Ret} with \lref{wf-gy}, it holds that
                    \begin{enumrm}[resume]
                        \item\llabel{ty-rety} $\jdty{\Gamma, y: T}{\expret{y}}{\tycomp{\emptyset}{\tyrfn{z_0}{B}{z_0 = y}}{\square}}$~.
                    \end{enumrm}
                    % Also, we have
                    % $\jdsub{\Gamma, y: T}{\tycomp{\emptyset}{\tyrfn{z_0}{B}{z_0 = y}}{\square}}{\tycomp{\Sigma}{T}{\tyctl{z}{C_1}{C_1[y/z]}}}$
                    % by the following derivation:
                    % \[
                    %     \infer{\jdsub{\Gamma, y: T}{\tycomp{\emptyset}{\tyrfn{z_0}{B}{z_0 = y}}{\square}}{\tycomp{\Sigma}{T}{\tyctl{z}{C_1}{C_1[y/z]}}}}
                    %     {
                    %         \infer{\jdsub{\Gamma, y: T}{\Sigma}{\emptyset}}
                    %         {}
                    %         &
                    %         \infer{\jdsub{\Gamma, y: \tyrfn{z_0}{B}{\phi}}{\tyrfn{z_0}{B}{z_0 = y}}{\tyrfn{z_0}{B}{\phi}}}
                    %         {\Gamma, y: \tyrfn{z_0}{B}{\phi}, z_0: B \vDash z_0 = y \implies \phi}
                    %         &
                    %         \infer{\jdsub{\Gamma, y: T \mid \tyrfn{z_0}{B}{z_0 = y}}{\square}{\tyctl{z}{C_1}{C_1[y/z]}}}
                    %         {
                    %             \infer{\jdsub{\Gamma, y: T, z: \tyrfn{z_0}{B}{z_0 = y}}{C_1}{C_1[y/z]}}{}
                    %         }
                    %     }
                    % \]
                    Also, we have the following subtyping with Lemma~\ref{lem:sub-eq}:
                    \[
                        \infer{\jdsub{\Gamma, y: T \mid \tyrfn{z_0}{B}{z_0 = y}}{\square}{\tyctl{z}{C_1}{C_1[y/z]}}}
                        {
                            \infer{\jdsub{\Gamma, y: T, z: \tyrfn{z_0}{B}{z_0 = y}}{C_1}{C_1[y/z]}}
                            {}
                        }
                    \]
                    Then it holds that
                    \begin{enumrm}[resume]
                        \item\llabel{sub-comp} $\jdsub{\Gamma, y: T}{\tycomp{\emptyset}{\tyrfn{z_0}{B}{z_0 = y}}{\square}}{\tycomp{\Sigma}{T}{\tyctl{z}{C_1}{C_1[y/z]}}}$
                    \end{enumrm}
                    by subtyping.
                    Therefore, by subsumption with \lref{ty-rety} and \lref{sub-comp}, we have the conclusion.
                \item[Case that $T$ is not a refinement type:]
                    By \rulename{T-Var} and \rulename{T-Ret} with \lref{wf-gy}, it holds that
                    \begin{enumrm}[resume]
                        \item\llabel{ty-rety2} $\jdty{\Gamma, y: T}{\expret{y}}{\tycomp{\emptyset}{T}{\square}}$~.
                    \end{enumrm}
                    Also, since $T$ is not a refinement type, by Lemma~\ref{lem:notin-nonrfn}
                    we have $z \notin C_1$ and so $C_1[y/z] = C_1$.
                    Then, we have the following subtyping with Lemma~\ref{lem:refl}:
                    \[
                        \infer{\jdsub{\Gamma, y: T \mid T}{\square}{\tyctl{z}{C_1}{C_1[y/z]}}}
                        {
                            \infer{\jdsub{\Gamma, y: T, z: T}{C_1}{C_1[y/z]}}
                            {}
                        }
                    \]
                    Then it holds that
                    \begin{enumrm}[resume]
                        \item\llabel{sub-comp2} $\jdsub{\Gamma, y: T}{\tycomp{\emptyset}{T}{\square}}{\tycomp{\Sigma}{T}{\tyctl{z}{C_1}{C_1[y/z]}}}$
                    \end{enumrm}
                    by subtyping.
                    Therefore, by subsumption with \lref{ty-rety2} and \lref{sub-comp2}, we have the conclusion.
            \end{description}
        \item[Case $K = \explet{x}{K_1}{c_2}$:]
            \def\currentprefix{inv-ctx:let}
            We have $\jdty{\Gamma}{\explet{x}{K_1[c]}{c_2}}{\tycomp{\Sigma}{T}{\tyctl{z}{C_1}{C_2}}}$.
            By Lemma~\ref{lem:inv}, we have
            \begin{enumrm}
                \item\llabel{ty-k1c} $\jdty{\Gamma}{K_1[c]}{\tycomp{\Sigma}{T'}{\tyctl{x}{C'}{C_2}}}$,
                \item\llabel{ty-c2} $\jdty{\Gamma, x: T'}{c_2}{\tycomp{\Sigma}{T}{\tyctl{z}{C_1}{C'}}}$, and
                \item\llabel{in-x} $x \notin \fv(T) \cup \fv(\Sigma) \cup (\fv(C_1) \setminus \{z\})$
            \end{enumrm}
            for some $T'$ and $C'$.
            By the IH of \lref{ty-k1c}, we have
            \begin{enumrm}[resume]
                \item\llabel{ty-c} $\jdty{\Gamma}{c}{\tycomp{\Sigma}{T_1}{\tyctl{y}{C_0}{C_2}}}$ and
                \item\llabel{ty-k1y} $\jdty{\Gamma, y: T_1}{K_1[\expret{y}]}{\tycomp{\Sigma}{T'}{\tyctl{x}{C'}{C_0}}}$
            \end{enumrm}
            for some $y, T_1$ and $C_0$.

            By Lemma~\ref{lem:wfg} with \lref{ty-c2}, we have $\jdwf{}{\Gamma, x: T'}$.
            By inversion, we have $x \notin \dom(\Gamma)$ and $\jdwf{\Gamma}{T'}$.
            Also, By Lemma~\ref{lem:wfg} with \lref{ty-k1y}, we have $\jdwf{}{\Gamma, y: T_1}$.
            Then, by Lemma~\ref{lem:weaken} we have $\jdwf{\Gamma, y: T_1}{T'}$.
            Moreover, w.l.o.g, we can assume $x \ne y$, and so
            $x \notin \dom(\Gamma) \cup \{y\} = \dom(\Gamma, y: T_1)$.
            Then we have $\jdwf{}{\Gamma, y: T_1, x: T'}$.

            Therefore, by Lemma~\ref{lem:weaken} with \lref{ty-c2}, we have
            \[
                \jdty{\Gamma, y: T_1, x: T'}{c_2}{\tycomp{\Sigma}{T}{\tyctl{z}{C_1}{C'}}}~.
            \]
            Then, by \rulename{T-LetIp} with \lref{ty-k1y} and \lref{in-x}, we have
            \[
                \jdty{\Gamma, y: T_1}{\explet{x}{K_1[\expret{y}]}{c_2}}{\tycomp{\Sigma}{T}{\tyctl{z}{C_1}{C_0}}}~,
            \]
            that is,
            \begin{enumrm}[resume]
                \item\llabel{ty-ky} $\jdty{\Gamma, y: T_1}{K[\expret{y}]}{\tycomp{\Sigma}{T}{\tyctl{z}{C_1}{C_0}}}$~.
            \end{enumrm}
            Therefore, from \lref{ty-c} and \lref{ty-ky} we have the conclusion.
    \end{description}
\end{proof}

\begin{theorem}[Subject reduction] \label{thm:subjred}
    If $\jdty{\emptyset}{c}{C}$ and $c \eval c'$, then $\jdty{\emptyset}{c'}{C}$.
\end{theorem}
% depend on:: asm:prim lem:inv lem:rm-nonrfn lem:rm-tauto lem:rm-unused lem:notin-nonrfn
%             lem:weaken lem:narrow lem:subst lem:subst-pred
%             lem:refl lem:wfg lem:wft
\begin{proof}
    \def\currentprefix{subred}
    By induction on the typing derivation.
    \begin{description}
        \item[Case \rulename{T-Ret} and \rulename{T-Op}:]
            Contradictory because there is no evaluation rule for $c$.
        \item[Case \rulename{T-App}:] We have
            \def\currentprefix{subred:app}
            \begin{enumrm}
                \item\llabel{eq-c} $c = v_1~v_2$,
                \item\llabel{eq-C} $C = C_1[v_2/x]$,
                \item\llabel{ty-v1} $\jdty{}{v_1}{(x:T_1) \rarr C_1}$, and
                \item\llabel{ty-v2} $\jdty{}{v_2}{T_1}$
            \end{enumrm}
            for some $x, v_1, v_2, T_1$ and $C_1$.
            Case analysis on the evaluation derivation.
            \begin{description}
                \item[Case \rulename{E-App}:] We have
                    \def\currentprefix{subred:app:app}
                    \begin{enumrm}[resume]
                        \item\llabel{eq-v1} $v_1 = \exprec{f}{x}{c_1}$, and
                        \item\llabel{eq-c'} $c' = c_1[v_2/x][(\exprec{f}{x}{c_1})/f]$
                    \end{enumrm}
                    for some $f, x$ and $c_1$.
                    By Lemma \ref{lem:inv} with \lref[subred:app]{ty-v1}, we have
                    \begin{enumrm}[resume]
                        \item\llabel{ty-v1-2} $\jdty{}{v_1}{(x: T_0) \rarr C_0}$,
                        \item\llabel{sub-fun} $\jdsub{}{(x: T_0) \rarr C_0}{(x: T_1) \rarr C_1}$, and
                        \item\llabel{ty-c} $\jdty{f: (x: T_0) \rarr C_0, x: T_0}{c}{C_0}$
                    \end{enumrm}
                    for some $T_0$ and $C_0$.
                    By Lemma \ref{lem:wfg} with \lref{ty-c}, inversion, and Lemma \ref{lem:rm-nonrfn},
                    we have $\jdwf{}{T_0}$.
                    Also, by inversion of \lref{sub-fun}, we have $\jdsub{}{T_1}{T_0}$.
                    Then, By \rulename{T-VSub} with \lref[subred:app]{ty-v2},
                    we have $\jdty{}{v_2}{T_0}$.
                    Using this and \lref{ty-v1-2}, we have the conclusion
                    by Lemma \ref{lem:subst} with \lref{ty-c}.
                \item[Case \rulename{E-Prim}:] We have
                    \def\currentprefix{subred:app:prim}
                    \begin{enumrm}[resume]
                        \item\llabel{eq-v1} $v_1 = p$, and
                        \item\llabel{eq-c'} $c' = \zeta(p, v_2)$
                    \end{enumrm}
                    for some $p$.
                    By Lemma \ref{lem:inv} with \lref[subred:app]{ty-v1}, we have
                    \begin{enumrm}[resume]
                        \item\llabel{ty-p} $\jdty{}{p}{\ty(p)}$, and
                        \item\llabel{sub-typ} $\jdsub{}{\ty(p)}{(x:T_1) \rarr C_1}$~.
                    \end{enumrm}
                    By inversion of \lref{sub-typ}, we have
                    \begin{enumrm}[resume]
                        \item\llabel{eq-typ} $\ty(p) = (x:T_0) \rarr C_0$,
                        \item\llabel{sub-t1} $\jdsub{}{T_1}{T_0}$, and
                        \item\llabel{sub-c0} $\jdsub{x: T_1}{C_0}{C_1}$
                    \end{enumrm}
                    for some $T_0$ and $C_0$.
                    By Lemma \ref{lem:wfg} with \lref{ty-p} and \lref{eq-typ} and inversion,
                    we have $\jdwf{}{T_0}$.
                    Then, by \rulename{T-VSub} with \lref[subred:app]{ty-v2} and \lref{sub-t1},
                    we have $\jdty{}{v_2}{T_0}$.
                    Therefore, by Assumption \ref{asm:prim} with \lref{eq-typ}, we have
                    \begin{enumrm}[resume]
                        \item\llabel{ty-z} $\jdty{}{\zeta(p, v_2)}{C_0[v_2/x]}$~.
                    \end{enumrm}
                    Also, by Lemma \ref{lem:wft} with \lref[subred:app]{ty-v1} and inversion,
                    we have
                    \begin{enumrm}[resume]
                        \item\llabel{wf-C1} $\jdwf{x: T_1}{C_1}$~.
                    \end{enumrm}
                    Using \lref[subred:app]{ty-v2},
                    by Lemma \ref{lem:subst} with \lref{sub-c0} and \lref{wf-C1} respectively,
                    we have
                    \begin{itemize}
                        \item $\jdsub{}{C_0[v_2/x]}{C_1[v_2/x]}$ and
                        \item $\jdwf{}{C_1[v_2/x]}$~.
                    \end{itemize}
                    Therefore, by \rulename{T-CSub} with \lref{ty-z}, we have the conclusion.
            \end{description}
        \item[Case \rulename{T-If}:] We have
            \def\currentprefix{subred:if}
            \begin{enumrm}
                \item\llabel{eq-c} $c = \expif{v}{c_1}{c_2}$,
                \item\llabel{ty-v} $\jdty{}{v}{\tyrfn{x}{\tybool}{\phi}}$,
                \item\llabel{ty-c1} $\jdty{v = \exptrue}{c_1}{C}$, and
                \item\llabel{ty-c2} $\jdty{v = \expfalse}{c_2}{C}$
            \end{enumrm}
            for some $x, v, c_1, c_2$, and $\phi$.
            Case analysis on the evaluation derivation.
            \begin{description}
                \item[Case \rulename{E-IfT}: ] We have
                    \begin{enumrm}[resume]
                        \item\llabel{eq-v} $v = \exptrue$, and
                        \item\llabel{eq-c'} $c' = c_1$~.
                    \end{enumrm}
                    We have the conclusion by Lemma \ref{lem:rm-tauto} with \lref{ty-c1}.
                \item[Case \rulename{E-IfF}: ] Similar.
            \end{description}
        \item[Case \rulename{T-CSub}:] By the IH and \rulename{T-CSub}.
        \item[Case \rulename{T-Let}:] We have
            \def\currentprefix{subred:let}
            \begin{enumrm}
                \item\llabel{eq-c} $c = \explet{x}{c_1}{c_2}$,
                \item\llabel{eq-C} $C = \tycomp{\Sigma}{T_2}{\bind{S_1}{x}{S_2}}$,
                \item\llabel{ty-c1} $\jdty{}{c_1}{\tycomp{\Sigma}{T_1}{S_1}}$,
                \item\llabel{ty-c2} $\jdty{x: T_1}{c_2}{\tycomp{\Sigma}{T_2}{S_2}}$, and
                \item\llabel{in-x} $x \notin \fv(T_2) \cup \fv(\Sigma)$
            \end{enumrm}
            for some $x, c_1, c_2, \Sigma, T_1, T_2, S_1$ and $S_2$.
            Case analysis on the evaluation derivation.
            \begin{description}
                \item[Case \rulename{E-Let}:] %We have
                    By the IH and \rulename{T-Let}.
                    % \def\currentprefix{subred:let:let}
                    % \begin{enumrm}[resume]
                    %     \item\llabel{eq-c'} $c' = \explet{x}{c_1'}{c_2}$, and
                    %     \item\llabel{ev-c1} $c_1 \eval c_1'$
                    % \end{enumrm}
                    % for some $c_1'$.
                    % By the IH with \lref[subred:let]{ty-c1} and \lref{ev-c1}, we have
                    % $\jdty{}{c_1'}{\tycomp{\Sigma}{T_1}{S_1}}$.
                    % Then we have the conclusion by \rulename{T-Let}.
                \item[Case \rulename{E-LetRet}:] We have
                    \def\currentprefix{subred:let:ret}
                    \begin{enumrm}[resume]
                        \item\llabel{eq-c1} $c1 = \expret{v}$, and
                        \item\llabel{eq-c'} $c' = c_2[v/x]$
                    \end{enumrm}
                    for some $v$.
                    By Lemma \ref{lem:inv} with \lref[subred:let]{ty-c1}, we have
                    \begin{enumrm}[resume]
                        \item\llabel{sub-T0} $\jdsub{}{T_0}{T_1}$,
                        \item\llabel{ty-v} $\jdty{}{v}{T_0}$, and
                        \item\llabel{sub-S1} $\jdsub{ \mid T_0}{\square}{S_1}$
                    \end{enumrm}
                    for some $T_0$.
                    By Lemma \ref{lem:wft} with \lref[subred:let]{ty-c1} and inversion,
                    we have $\jdwf{}{T_1}$.
                    Then, by \rulename{T-VSub} with \lref{ty-v} and \lref{sub-T0},
                    we have $\jdty{}{v}{T_1}$.
                    Therefore, by Lemma \ref{lem:subst} with \lref[subred:let]{ty-c2},
                    we have
                    \begin{enumrm}[resume]
                        \item\llabel{ty-c2-2} $\jdty{}{c_2[v/x]}{\tycomp{\Sigma}{T_2}{(S_2[v/x])}}$~.
                    \end{enumrm}
                    (Note that since \lref[subred:let]{in-x},
                    it holds that $\Sigma[v/x] = \Sigma$ and $T_2[v/x] = T_2$.)
                    Case analysis on $\bind{S_1}{x}{S_2}$.
                    (Note that the definition of $\ggeq$ has only two cases.)
                    \begin{description}
                        \item[Case $S_1 = S_2 = \square$:]
                            Since $\bind{S_1}{x}{S_2} = \square = S_2[v/x]$,
                            we have the conclusion from \lref{ty-c2-2}.
                        \item[Case $S_1 = \tyctl{x}{C_0}{C_1}$ and $S_2 = \tyctl{z}{C_2}{C_0}$
                            for some $z, C_0, C_1$, and $C_2$:]
                            % We have
                            % \begin{enumrm}[resume]
                            %     \item\llabel{in-x-2} $x \notin \fv(C_2) \setminus \{z\}$
                            % \end{enumrm}
                            % from the side condition of the definition of $\ggeq$.
                            By inversion of \lref{sub-S1}, we have
                            \begin{enumrm}[resume]
                                \item\llabel{sub-C0} $\jdsub{x:T_0}{C_0}{C_1}$ and
                                \item\llabel{in-x-3} $x \notin \fv(C_2)$~.
                            \end{enumrm}
                            By Lemma \ref{lem:wft} with \lref[subred:let]{ty-c1} and inversion,
                            we have $\jdwf{}{C_1}$, which means $x \notin \fv(C_1)$.
                            Therefore, by Lemma \ref{lem:subst} with \lref{sub-C0},
                            we have
                            \begin{enumrm}[resume]
                                \item\llabel{sub-C0-2} $\jdsub{}{C_0[v/x]}{C_1}$~.
                            \end{enumrm}
                            On the other hand, by Lemma \ref{lem:wft} with \lref[subred:let]{ty-c2} and inversion,
                            we have $\jdwf{x:T_1, z:T_2}{C_2}$.
                            By Lemma \ref{lem:rm-unused} with \lref[subred:let]{in-x} and \lref{in-x-3},
                            we have $\jdwf{z:T_2}{C_2}$.
                            \lref{in-x-3} also implies $C_2 = C_2[v/x]$.
                            Therefore, by Lemma \ref{lem:refl}, we have
                            \begin{enumrm}[resume]
                                \item\llabel{sub-C2} $\jdsub{z:T_2}{C_2}{C_2[v/x]}$~.
                            \end{enumrm}
                            Hence, by \rulename{S-ATM} with \lref{sub-C0-2} and \lref{sub-C2},
                            we have
                            $\jdsub{\mid T_2}{\tyctl{z}{C_2[v/x]}{C_0[v/x]}}{\tyctl{z}{C_2}{C_1}}$,
                            that is, $\jdsub{\mid T_2}{S_2[v/x]}{\bind{S_1}{x}{S_2}}$~.
                            Now we have the conclusion by subsumption of \lref{ty-c2-2}.
                    \end{description}
                \item[Case \rulename{E-LetOp}:] We have
                    \def\currentprefix{subred:let:op}
                    \begin{enumrm}[resume]
                        \item\llabel{eq-c1} $c_1 = \expop{v}{y}{c_0}$, and
                        \item\llabel{eq-c'} $c' = \expop{v}{y}{\explet{x}{c_0}{c_2}}$
                    \end{enumrm}
                    for some $y, v$ and $c_0$.
                    W.l.o.g., we can assume that $y \notin \fv(c_2, \Sigma, T_2, S_2)$.
                    By Lemma \ref{lem:inv} with \lref[subred:let]{ty-c1}, we have
                    \begin{enumrm}[resume]
                        \item\llabel{in-sig} $\Sigma \ni \op: \forall \rep{X: \rep{B}}. (x_0: T_{01}) \rarr ((y: T_{02}) \rarr C_{01}) \rarr C_{02}$,
                        \item\llabel{wf-A} $\rep{\jdty{\Gamma}{A}{\rep{B}}}$,
                        \item\llabel{ty-v} $\jdty{}{v}{T_{01}[\rep{A/X}]}$,
                        \item\llabel{sub-T1'} $\jdsub{}{T_1'}{T_1}$,
                        \item\llabel{ty-c0} $\jdty{y: T_{02}[\rep{A/X}][v/x_0]}{c_0}{\tycomp{\Sigma}{T_1'}{\tyctl{z}{C_0}{C_{01}[\rep{A/X}][v/x_0]}}}$,
                        \item\llabel{sub-S1} $\jdsub{ \mid T_1'}{\tyctl{z}{C_0}{C_{02}[\rep{A/X}][v/x_0]}}{S_1}$, and
                        \item\llabel{in-y} $y \notin \fv(\Sigma) \cup \fv(T_1) \cup \fv(T_1') \cup (\fv(C_0) \setminus \{ z \})$
                    \end{enumrm}
                    for some $\rep{X}, \rep{\rep{B}}, \rep{A}, x_0, z, T_1', T_{01}, T_{02}, C_0, C_{01}$, and $C_{02}$.
                    By inversion of \lref{sub-S1}, we have
                    \begin{enumrm}[resume]
                        \item\llabel{eq-S1} $S_1 = \tyctl{z}{C_{11}}{C_{12}}$,
                        \item\llabel{sub-C11} $\jdsub{x:T_1'}{C_{11}}{C_0}$, and
                        \item\llabel{sub-C02} $\jdsub{}{C_{02}[\rep{A/X}][v/x_0]}{C_{12}}$
                    \end{enumrm}
                    for some $C_{11}$ and $C_{12}$.
                    By definition of $\bind{S_1}{x}{S_2}$, we have
                    \begin{enumrm}[resume]
                        \item\llabel{eq-x} $z = x$,
                        \item\llabel{eq-S2} $S_2 = \tyctl{z_0}{C_{21}}{C_{11}}$, and
                        \item\llabel{in-x-2} $x \notin \fv(C_{21}) \setminus \{z_0\}$
                    \end{enumrm}
                    for some $z_0$ and $C_{21}$.
                    By subsumption of \lref{ty-c0} using \lref{sub-C11}, we have
                    \[
                        \jdty{y: T_{02}[\rep{A/X}][v/x_0]}{c_0}
                        {\tycomp{\Sigma}{T_1'}{\tyctl{x}{C_{11}}{C_{01}[\rep{A/X}][v/x_0]}}}~.
                    \]
                    Also, by Lemma \ref{lem:weaken} and Lemma \ref{lem:narrow} with \lref{sub-T1'}
                    applied to \lref[subred:let]{ty-c2}, we have
                    \[
                        \jdty{y: T_{02}[\rep{A/X}][v/x_0], x: T_1'}{c_2}
                        {\tycomp{\Sigma}{T_2}{\tyctl{z_0}{C_{21}}{C_{11}}}}~.
                    \]
                    % From these, we have the following derivation:
                    % \[
                    %     \infersc[T-Op]{\jdty{}{\expop{v}{y}{\explet{x}{c_0}{c_2}}}
                    %         {\tycomp{\Sigma}{T_2}{\tyctl{z_0}{C_{21}}{C_{02}[\rep{A/X}][v/x_0]}}}}
                    %     {
                    %         \lref{in-sig}, \lref{wf-A}, \lref{ty-v}, \lref{in-y}
                    %         &
                    %         \infersc[T-Let]{\jdty{y: T_{02}[\rep{A/X}][v/x_0]}{\explet{x}{c_0}{c_2}}
                    %             {\tycomp{\Sigma}{T_2}{\tyctl{z_0}{C_{21}}{C_{01}[\rep{A/X}][v/x_0]}}}}
                    %         {\begin{gathered}
                    %             \jdty{y: T_{02}[\rep{A/X}][v/x_0]}{c_0}
                    %                 {\tycomp{\Sigma}{T_1'}{\tyctl{x}{C_{11}}{C_{01}[\rep{A/X}][v/x_0]}}} \\
                    %             \jdty{y: T_{02}[\rep{A/X}][v/x_0], x: T_1'}{c_2}
                    %                 {\tycomp{\Sigma}{T_2}{\tyctl{z_0}{C_{21}}{C_{11}}}}
                    %         \end{gathered}
                    %         }
                    %     }
                    % \]
                    From these two, by \rulename{T-Let}, we have
                    \[
                        \jdty{y: T_{02}[\rep{A/X}][v/x_0]}{\explet{x}{c_0}{c_2}}
                        {\tycomp{\Sigma}{T_2}{\tyctl{z_0}{C_{21}}{C_{01}[\rep{A/X}][v/x_0]}}}~.
                    \]
                    Moreover, by \rulename{T-Op} with \lref{in-sig}, \lref{wf-A}, \lref{ty-v}, and \lref{in-y},
                    we have
                    \[
                        \jdty{}{\expop{v}{y}{\explet{x}{c_0}{c_2}}}
                            {\tycomp{\Sigma}{T_2}{\tyctl{z_0}{C_{21}}{C_{02}[\rep{A/X}][v/x_0]}}}~.
                    \]
                    Now we have the conclusion by subsumption with \lref{sub-C02}.
            \end{description}
        \item[Case \rulename{T-Hndl}:] We have
            \def\currentprefix{subred:hndl}
            \begin{enumrm}
                \item\llabel{eq-c} $c = \expwith{h}{c_0}$,
                \item\llabel{eq-h} $h = \{ \expret{x_r} \mapsto c_r, \repi{\op_i(x_i, k_i) \mapsto c_i} \}$,
                \item\llabel{ty-c0} $\jdty{}{c_0}{\tycomp{\Sigma_0}{T_0}{\tyctl{x_r}{C_1}{C}}}$,
                \item\llabel{ty-cr} $\jdty{x_r: T_0}{c_r}{C_1}$,
                \item\llabel{ty-ci} $\bigrepi{\jdty{\rep{X_i: \rep{B}_i}, x_i: T_{i1}, k_i: (y_i: T_{i2}) \rarr C_{i1}}{c_i}{C_{i2}}}$, and
                \item\llabel{eq-sig} $\Sigma_0 = \{ \repi{\op_i: \forall \rep{X_i: \rep{B}_i}. (x_i: T_{i1}) \rarr ((y_i: T_{i2}) \rarr C_{i1}) \rarr C_{2i}} \}$
            \end{enumrm}
            Case analysis on the evaluation derivation.
            \begin{description}
                \item[Case \rulename{E-Hndl}:]
                    By the IH and \rulename{T-Hndl}.
                \item[Case \rulename{E-HndlRet}:] We have
                    \def\currentprefix{subred:hndl:ret}
                    \begin{enumrm}
                        \item\llabel{eq-c0} $c_0 = \expret{v}$ and
                        \item\llabel{eq-c'} $c' = c_r[v/x_r]$
                    \end{enumrm}
                    for some $v$.
                    By Lemma \ref{lem:inv} with \lref[subred:hndl]{ty-c0}, we have
                    \begin{enumrm}[resume]
                        \item\llabel{sub-T0} $\jdsub{}{T_0'}{T_0}$,
                        \item\llabel{ty-v} $\jdty{}{v}{T_0'}$, and
                        \item\llabel{sub-S0} $\jdsub{ \mid T_0'}{\square}{\tyctl{x_r}{C_1}{C}}$
                    \end{enumrm}
                    for some $T_0'$.
                    By inversion of \lref{sub-S0}, we have
                    \begin{enumrm}[resume]
                        \item\llabel{sub-C1} $\jdsub{x_r: T_0'}{C_1}{C}$ and
                        \item\llabel{in-xr} $x_r \notin \fv(C)$~.
                    \end{enumrm}
                    By Lemma \ref{lem:narrow} with \lref[subred:hndl]{ty-cr} and \lref{sub-T0},
                    we have
                    \begin{enumrm}[resume]
                        \item\llabel{ty-cr-2} $\jdty{x_r: T_0'}{c_r}{C_1}$~.
                    \end{enumrm}
                    By Lemma \ref{lem:subst} with \lref{ty-v}
                    applied to \lref{sub-C1} and \lref{ty-cr-2}, we have
                    \begin{enumrm}[resume]
                        \item\llabel{sub-C1-2} $\jdsub{}{C_1[v/x_r]}{C}$ and
                        \item\llabel{ty-cr-3} $\jdty{}{c_r[v/x_r]}{C_1[v/x_r]}$
                    \end{enumrm}
                    respectively.
                    (Note that $C[v/x_r] = C$ since \lref{in-xr}.)
                    By Lemma \ref{lem:wft} with \lref[subred:hndl]{ty-c0} and inversion,
                    we have $\jdwf{}{C}$.
                    From this and \lref{sub-C1-2} and \lref{ty-cr-3},
                    we have the conclusion by \rulename{T-CSub}.
                \item[Case \rulename{E-HndlOp}:] We have
                    \def\currentprefix{subred:hndl:op}
                    \begin{enumrm}
                        \item\llabel{eq-c0} $c_0 = \expop[\op_i]{v}{y}{c_{00}}$ and
                        \item\llabel{eq-c'} $c' = c_i[v/x_i][(\lambda y. \expwith{h}{c_{00}})/k_i]$
                    \end{enumrm}
                    for some $y, v$ and $c_{00}$.
                    W.l.o.g., we can assume that $y$ is disjoint from
                    the variables of $h$ and the types related to $h$.
                    By Lemma \ref{lem:inv} with \lref[subred:hndl]{ty-c0}, we have
                    \begin{enumrm}[resume]
                        \item\llabel{in-sig} $\Sigma_0 \ni \op_i: \forall \rep{X_i: \rep{B}_i}. (x_i: T_{i1}) \rarr ((y: T_{i2}) \rarr C_{i1}) \rarr C_{i2}$,
                        \item\llabel{wf-A} $\rep{\jdty{\Gamma}{A}{\rep{B}_i}}$,
                        \item\llabel{ty-v} $\jdty{}{v}{T_{i1}[\rep{A/X_i}]}$,
                        \item\llabel{sub-T0'} $\jdsub{}{T_0'}{T_0}$,
                        \item\llabel{ty-c00} $\jdty{y: T_{i2}[\rep{A/X_i}][v/x_i]}{c_{00}}{\tycomp{\Sigma_0}{T_0'}{\tyctl{z}{C_0}{C_{i1}[\rep{A/X_i}][v/x_i]}}}$,
                        \item\llabel{sub-S0} $\jdsub{ \mid T_0'}{\tyctl{z}{C_0}{C_{i2}[\rep{A/X_i}][v/x_i]}}{\tyctl{x_r}{C_1}{C}}$, and
                        \item\llabel{in-y} $y \notin \fv(\Sigma_0) \cup \fv(T_0) \cup \fv(T_0') \cup (\fv(C_0) \setminus \{ z \})$
                    \end{enumrm}
                    for some $\rep{A}$, and $T_0'$.
                    Note that since \lref[subred:hndl]{eq-sig} holds, it holds that $y = y_i$ and we use
                    $\rep{X_i}, \rep{\rep{B}_i}, x_i, T_{i1}, T_{i2}, C_{i1}$, and $C_{i2}$ here
                    instead of introducing new ones.
                    % From \lref[subred:hndl]{eq-sig} and \lref{in-sig}, it holds that
                    % $x_i = x_0$, $T_{i1} = T_{01}$, $T_{i2} = T_{02}$, $C_{i1} = C_{01}$, and $C_{i2} = C_{02}$.
                    By inversion of \lref{sub-S0}, we have
                    \begin{enumrm}[resume]
                        \item\llabel{eq-z} $z = x_r$,
                        \item\llabel{sub-C1} $\jdsub{x_r: T_0'}{C_1}{C_0}$, and
                        \item\llabel{sub-Ci2} $\jdsub{}{C_{i2}[\rep{A/X_i}][v/x_i]}{C}$~.
                    \end{enumrm}
                    By Lemma \ref{lem:weaken} applied to \lref{sub-T0'} and \lref{sub-C1},
                    we have
                    \begin{itemize}
                        \item $\jdsub{y: T_{i2}[\rep{A/X_i}][v/x_i]}{T_0'}{T_0}$ and
                        \item $\jdsub{y: T_{i2}[\rep{A/X_i}][v/x_i], x_r: T_0'}{C_1}{C_0}$~.
                    \end{itemize}
                    Using these subtyping, by subsumption of \lref{ty-c00}, we have
                    \[
                        \jdty{y: T_{i2}[\rep{A/X_i}][v/x_i]}{c_{00}}{\tycomp{\Sigma_0}{T_0}{\tyctl{x_r}{C_1}{C_{i1}[\rep{A/X_i}][v/x_i]}}}~.
                    \]
                    Then, by \rulename{T-Hndl} with \lref[subred:hndl]{ty-cr},
                    \lref[subred:hndl]{ty-ci} and \lref[subred:hndl]{eq-sig} with Lemma \ref{lem:weaken},
                    we have
                    \[
                        \jdty{y: T_{i2}[\rep{A/X_i}][v/x_i]}{\expwith{h}{c_{00}}}{C_{i1}[\rep{A/X_i}][v/x_i]}~.
                    \]
                    By \rulename{T-Fun}, we have
                    \begin{enumrm}[resume]
                        \item\llabel{ty-fun} $\jdty{}{\lambda y. \expwith{h}{c_{00}}}
                            {(y: T_{i2}[\rep{A/X_i}][v/x_i]) \rarr C_{i1}[\rep{A/X_i}][v/x_i]}$~.
                    \end{enumrm}
                    On the other hand, by Lemma \ref{lem:subst-pred} with \lref{wf-A}
                    applied to \lref[subred:hndl]{ty-ci}, we have
                    \[
                        \jdty{x_i: T_{i1}[\rep{A/X_i}], k_i: (y_i: T_{i2}[\rep{A/X_i}]) \rarr C_{i1}[\rep{A/X_i}]}
                            {c_i}{C_{i2}[\rep{A/X_i}]}~.
                    \]
                    By applying \ref{lem:subst} twice with \lref{ty-v} and \lref{ty-fun} in a row, we have
                    \[
                        \jdty{}{c_i[v/x_i][(\lambda y. \expwith{h}{c_{00}})/k_i]}{C_{i2}[\rep{A/X_i}][v/x_i]}~.
                    \]
                    Note that
                    $C_{i2}[\rep{A/X_i}][v/x_i][(\lambda y. \expwith{h}{c_{00}})/k_i] = C_{i2}[\rep{A/X_i}][v/x_i]$
                    since $k_i \notin \fv(C_{i2}[\rep{A/X_i}][v/x_i])$ by Lemma \ref{lem:notin-nonrfn}.
                    Now we have the conclusion by subsumption with \lref{sub-Ci2}.
            \end{description}
    \end{description}
\end{proof}

\subsection{Type Safety}

\begin{theorem}[Type safety] \label{thm:safety}
    If $\jdty{\emptyset}{c}{\tycomp{\Sigma}{T}{S}}$ and $c \eval^* c'$, then either:
    \begin{itemize}
        \item $c' = \expret{v}$ for some $v$ such that $\jdty{\emptyset}{v}{T}$,
        \item $c' = K[\op~v]$ for some $K, \op$ and $v$ such that $\op \in \dom(\Sigma)$, or
        \item $c' \eval c''$ for some $c''$ such that $\jdty{\emptyset}{c''}{\tycomp{\Sigma}{T}{S}}$.
    \end{itemize}
\end{theorem}
\begin{proof}
    By induction on the length of $\eval^*$
    with Theorem~\ref{thm:progress} and Theorem~\ref{thm:subjred}.
\end{proof}

\section{Definitions for the CPS transformation}

\subsection{Evaluation rules for the target language of the CPS transformation}

\begin{align}
    \text{evaluation context} \quad
    E ::= [\ ] \mid E~v \mid E~\rep{A} \mid E~\tau
\end{align}
\fbox{$c \eval c'$}
\begin{gather}
    \infersc[Ec-Ctx]{E[c] \eval E[c']}
    {c \eval c'}
    \quad
    \infersc[Ec-IfT]{\expif{\exptrue}{c_1}{c_2} \eval c_1}
    {}
    \quad
    \infersc[Ec-IfF]{\expif{\expfalse}{c_1}{c_2} \eval c_2}
    {}
    \\
    \infersc[Ec-App]{(\exprec{f:\tau_1}{x:\tau_2}{c})~v \eval c[v/x][(\exprec{f:\tau_1}{x:\tau_2}{c})/f]}
    {}
    \\[1.5ex]
    \infersc[Ec-Prim]{p~v \eval \zeta_{\textit{cps}}(p, v)}
    {}
    \quad
    \infersc[Ec-PApp]{(\Lambda \rep{X: \rep{B}}. c)~\rep{A} \eval c[\rep{A/X}]}
    {}
    \\[1.5ex]
    \infersc[Ec-Proj]{\{ \repi{\op_i = v_i} \}\#\op_i \eval v_i}
    {}
    \quad
    \infersc[Ec-TApp]{(\Lambda \alpha. c)~\tau \eval c[\tau/\alpha]}
    {}
    \quad
    \infersc[Ec-Acsr]{(c: \tau) \eval c}
    {}
\end{gather}

\subsection{Syntax of typing contexts of the target language of the CPS transformation}

\begin{gather}
    \Gamma ::= \emptyset \mid \Gamma, x: \tau
        \mid \Gamma, X:\rep{B} \mid \Gamma, \alpha
\end{gather}

\subsection{Well-formedness rules of the target language of the CPS transformation}

\fbox{$\jdwf{}{\Gamma}$} \quad \fbox{$\jdwf{\Gamma}{\tau}$}
\begin{gather}
    \infersc[WEc-Empty]{\jdwf{}{\emptyset}}
    {}
    \quad
    \infersc[WEc-Var]{\jdwf{}{\Gamma, x: \tau}}
    {
        \jdwf{}{\Gamma} &
        x \notin \dom(\Gamma) &
        \jdwf{\Gamma}{\tau}
    }
    \\
    \infersc[WEc-PVar]{\jdwf{}{\Gamma, X: \rep{B}}}
    {
        \jdwf{}{\Gamma} &
        X \notin \dom(\Gamma)
    }
    \quad
    \infersc[WEc-TVar]{\jdwf{}{\Gamma, \alpha}}
    {
        \jdwf{}{\Gamma} &
        \alpha \notin \dom(\Gamma)
    }
    \\
    \infersc[WTc-Rfn]{\jdwf{\Gamma}{\tyrfn{x}{B}{\phi}}}
    {\Gamma, x: B \vdash \phi}
    \quad
    \infersc[WTc-Fun]{\jdwf{\Gamma}{(x: \tau_1) \rarr \tau_2}}
    {
        \jdwf{\Gamma, x: \tau_1}{\tau_2}
    }
    \quad
    \infersc[WTc-PPoly]{\jdwf{\Gamma}{\forall \rep{X: \rep{B}}. \tau}}
    {
        \jdwf{\Gamma, \rep{X: \rep{B}}}{\tau}
    }
    \\
    \infersc[WTc-Rcd]{\jdwf{\Gamma}{\{\repi{\op_i: \tau_i}\}}}
    {
        \bigrepi{\jdwf{\Gamma}{\tau_i}}
    }
    \quad
    \infersc[WTc-TVar]{\jdwf{\Gamma}{\alpha}}
    {
        \alpha \in \Gamma
    }
    \quad
    \infersc[WTc-TPoly]{\jdwf{\Gamma}{\forall \alpha. \tau}}
    {
        \jdwf{\Gamma, \alpha}{\tau}
    }
    \quad
\end{gather}


\subsection{Typing rules of the target language of the CPS transformation}

\fbox{$\jdty{\Gamma}{c}{\tau}$}
\begin{gather}
    \infersc[Tc-CVar]{\jdty{\Gamma}{x}{\tyrfn{y}{B}{x = y}}}
    {
        \jdwf{}{\Gamma} &
        \Gamma(x) = \tyrfn{y}{B}{\phi}
    }
    \quad
    \infersc[Tc-Var]{\jdty{\Gamma}{x}{\Gamma(x)}}
    {
        \jdwf{}{\Gamma} &
        \forall y, B, \phi. \Gamma(x) \neq \tyrfn{y}{B}{\phi}
    }
    \quad
    \infersc[Tc-Prim]{\jdty{\Gamma}{p}{\tycps(p)}}
    {\jdwf{}{\Gamma}}
    \\
    \infersc[Tc-Fun]{\jdty{\Gamma}{\exprec{f:(x: \tau_1) \rarr \tau_2}{x:\tau_1}{c}}{(x: \tau_1) \rarr \tau_2}}
    {
        \jdty{\Gamma, f: (x: \tau_1) \rarr \tau_2, x: \tau_1}{c}{\tau_2}
    }
    \quad
    % \infersc[Tc-Lam]{\jdty{\Gamma}{\lambda x. c}{(x: \tau_1) \rarr \tau_2}}
    % {
    %     \jdty{x: \tau_1}{c}{\tau_2}
    % }
    % \\
    \infersc[Tc-App]{\jdty{\Gamma}{c~v}{\tau_2[v/x]}}
    {
        \jdty{\Gamma}{c}{(x: \tau_1) \rarr \tau_2} &
        \jdty{\Gamma}{v}{\tau_1}
    }
    \\
    \infersc[Tc-TAbs]{\jdty{\Gamma}{\Lambda \alpha. c}{\forall \alpha. \tau}}
    {\jdty{\Gamma, \alpha}{c}{\tau}}
    \quad
    \infersc[Tc-TApp]{\jdty{\Gamma}{c~\tau}{\tau'[\tau/\alpha]}}
    {
        \jdty{\Gamma}{c}{\forall \alpha. \tau'} &
        \jdwf{\Gamma}{\tau}
    }
    \\
    \infersc[Tc-PAbs]{\jdty{\Gamma}{\Lambda \rep{X: \rep{B}}. c}{\forall \rep{X: \rep{B}}. \tau}}
    {
        \jdty{\Gamma, \rep{X: \rep{B}}}{c}{\tau}
    }
    \quad
    \infersc[Tc-PApp]{\jdty{\Gamma}{c~\rep{A}}{\tau[\rep{A/X}]}}
    {
        \jdty{\Gamma}{c}{\forall \rep{X: \rep{B}}. \tau} &
        \rep{\jdty{\Gamma}{A}{\rep{B}}}
    }
    \\
    \infersc[Tc-Rcd]{\jdty{\Gamma}{\{ \repi{\op_i = v_i} \}}{\{ \repi{\op_i : \tau_i} \}}}
    {
        \repi{\jdty{\Gamma}{v_i}{\tau_i}}
    }
    \quad
    \infersc[Tc-Proj]{\jdty{\Gamma}{v\#\op_i}{\tau_i}}
    {
        \jdty{\Gamma}{v}{\{ \repi{\op_i : \tau_i} \}}
    }
    \\
    \infersc[Tc-If]{\jdty{\Gamma}{\expif{v}{c_1}{c_2}}{\tau}}
    {
        \jdty{\Gamma}{v}{\tyrfn{x}{\tybool}{\phi}} &
        \jdty{\Gamma, v = \exptrue}{c_1}{\tau} &
        \jdty{\Gamma, v = \expfalse}{c_2}{\tau}
    }
    \\
    \infersc[Tc-Ascr]{\jdty{\Gamma}{(c : \tau)}{\tau}}
    {
        \jdty{\Gamma}{c}{\tau'} &
        \jdsub{\Gamma}{\tau'}{\tau} &
        \jdwf{\Gamma}{\tau}
    }
    \quad
    \infersc[Tc-Sub]{\jdty{\Gamma}{c}{\tau_2}}
    {
        \jdty{\Gamma}{c}{\tau_1} &
        \jdsub{\Gamma}{\tau_1}{\tau_2} &
        \jdwf{\Gamma}{\tau_2}
    }
\end{gather}

\subsection{Subtyping rules of the target language of the CPS transformation}

\fbox{$\jdsub{\Gamma}{\tau_1}{\tau_2}$}
\begin{gather}
    \infersc[Sc-Rfn]{\jdsub{\Gamma}{\tyrfn{x}{B}{\phi_1}}{\tyrfn{x}{B}{\phi_2}}}
    {\Gamma, x: B \vDash \phi_1 \implies \phi_2}
    \quad
    \infersc[Sc-Fun]{\jdsub{\Gamma}{(x: \tau_{11}) \rarr \tau_{12}}{(x: \tau_{21}) \rarr \tau_{22}}}
    {
        \jdsub{\Gamma}{\tau_{21}}{\tau_{11}} &
        \jdsub{\Gamma, x: \tau_{21}}{\tau_{12}}{\tau_{22}}
    }
    \\
    \infersc[Sc-PPoly]{\jdsub{\Gamma}{\forall \rep{X: \rep{B}}. \tau_1}{\forall \rep{X: \rep{B}}. \tau_2}}
    {
        \jdsub{\Gamma, \rep{X: \rep{B}}}{\tau_1}{\tau_2}
    }
    \quad
    \infersc[Sc-Rcd]{\jdsub{\Gamma}{\{ \repi{\op_i: \tau_{1i}}, \repi{\op'_i: \tau'_i} \}}{\{ \repi{\op_i:  \tau_{2i}} \}}}
    {\repi{\jdsub{\Gamma}{\tau_{1i}}{\tau_{2i}}}}
    \\
    \infersc[Sc-TVar]{\jdsub{\Gamma}{\alpha}{\alpha}}
    {\alpha \in \Gamma}
    \quad
    \infersc[Sc-Poly]{\jdsub{\Gamma}{\forall \alpha. \tau_1}{\forall \beta. \tau_2}}
    {
        \jdsub{\Gamma, \beta}{\tau_1[\tau/\alpha]}{\tau_2} &
        \jdwf{\Gamma, \beta}{\tau}&
        \beta \notin \fv(\forall \alpha. \tau_1)
    }
\end{gather}

\subsection{CPS transformation of expressions}

\begingroup
\allowdisplaybreaks
\begin{align}
    \cps{x} &\defeq x \\
    \cps{p} &\defeq \mathit{cps}(p) \\
    \cps{\exprec{f^{(x:T_1) \rarr C_1}}{x^{T_2}}{c}} &\defeq \exprec{f:\cps{(x:T_1) \rarr C_1}}{x:\cps{T_2}}{\cps{c}} \\
    \cps{\expret{v^T}} &\defeq \stLambda \alpha. \stlambda h:\{\}. \stlambda k:\cps{T} \rarr \alpha. k~\cps{v} \\
    \cps{\explet{x}{c_1^{\tycomp{\Sigma}{T_1}{\square}}}{c_2^{\tycomp{\Sigma}{T_2}{\square}}}}
        &\defeq \stLambda \alpha. \stlambda h: \cps{\Sigma}. \stlambda k:\cps{T_2} \rarr \alpha.
        \cps{c_1} \stapp \alpha \stapp h \stapp (\lambda x:\cps{T_1}. \cps{c_2} \stapp \alpha \stapp h \stapp k) \\
    & \hspace*{-120pt} \cps{\explet{x}{c_1^{\tycomp{\Sigma}{T_1}{\tyctl{x}{C_1}{C_2}}}}{c_2^{\tycomp{\Sigma}{T_2}{\tyctl{z}{C_0}{C_1}}}}} \defeq \\
        & \hspace*{-90pt} \stLambda \alpha. \stlambda h: \cps{\Sigma}. \stlambda k:(z:\cps{T_2}) \rarr \cps{C_0}.
        \cps{c_1} \stapp \cps{C_2} \stapp h \stapp (\lambda x:\cps{T_1}. \cps{c_2} \stapp \cps{C_1} \stapp h \stapp k) \\
    \cps{v_1~v_2} &\defeq \cps{v_1}~\cps{v_2} \\
    \cps{(\expif{v}{c_1}{c_2})^{C}} &\defeq (\expif{\cps{v}}{\cps{c_1}}{\cps{c_2}} : \cps{C}) \\
    % \cps{\expop[\op^{\rep{\mathit{A}}}]{v}{y^{T_y}}{c^{\tycomp{\Sigma}{T}{\tyctl{z}{C_1}{C_2}}}}} &\defeq \\[-1ex]
    %     & \hspace*{-20pt} \Lambda \alpha. \lambda h:\cps{\Sigma}. \lambda k:(z: \cps{T} \rarr \cps{C_1}).
    %     h\#\op~\rep{A}~\cps{v}~(\lambda y:\cps{T_y}. \cps{c}~\nmbullet~h~k) \\
    \cps{(\op^{\rep{\mathit{A}}}~v)^{\tycomp{\Sigma}{T}{\tyctl{y}{C_1}{C_2}}}} &\defeq
        \stLambda \alpha. \stlambda h:\cps{\Sigma}. \stlambda k:(y: \cps{T} \rarr \cps{C_1}).
        h\#\op~\rep{A}~\cps{v}~(\lambda y': \cps{T}. k~y') \\
    \cps{(\expwith{h}{c})^C} &\defeq \cps{c} \stapp \cps{C} \stapp \cps{h^{\mathit{ops}}} \stapp \cps{h^{\mathit{ret}}} \\[-.5ex]
        \text{where} \hspace{-25pt} & \hspace{25pt} \left\{ \begin{aligned}
        h &= \{ \expret{x_r^{T_r}} \mapsto c_r, \repi{\op_i^{\rep{X_i: \rep{B_i}}}(x_i^{T_{x_i}}, k_i^{T_{k_i}}) \mapsto c_i} \} \\[-.5ex]
        \cps{h^{\mathit{ops}}} &\defeq
            \{ \repi{\op_i = \Lambda \rep{X_i: \rep{B_i}}. \lambda x_i:\cps{T_{x_i}}. \lambda k_i:\cps{T_{k_i}}. \cps{c_i}} \} \\[-.5ex]
        \cps{h^{\mathit{ret}}} &\defeq
            \lambda x_r:\cps{T_r}. \cps{c_r}
    \end{aligned} \right.
\end{align}
\endgroup

\subsection{CPS transformation of types and typing contexts}

\begin{align}
    \cps{\tyrfn{x}{B}{\phi}} &\defeq \tyrfn{x}{B}{\phi} \\
    \cps{(x: T) \rarr C} &\defeq (x: \cps{T}) \rarr \cps{C} \\
    \cps{\tycomp{\Sigma}{T}{\tyctl{x}{C_1}{C_2}}} &\defeq
        \forall \_. \cps{\Sigma} \rarr ((x: \cps{T}) \rarr \cps{C_1}) \rarr \cps{C_2} \\
    \cps{\tycomp{\Sigma}{T}{\square}} &\defeq
        \forall \alpha. \cps{\Sigma} \rarr (\cps{T} \rarr \alpha) \rarr \alpha \\
    \cps{\{ \repi{\op_i : \forall \rep{X_i: \rep{B}_i}. F_i} \}} &\defeq
        \{ \repi{\op_i : \forall \rep{X_i: \rep{B}_i}. \cps{F_i}^\mathcal{F}} \} \\
    \cps{(x: T_1) \rarr ((y: T_2) \rarr C_1) \rarr C_2}^\mathcal{F} &\defeq
        (x: \cps{T_1}) \rarr \cps{((y: T_2) \rarr C_1)} \rarr \cps{C_2}
\end{align}
\begin{align}
    \cps{\emptyset} &\defeq \emptyset \\
    \cps{\Gamma, x: T} &\defeq \cps{\Gamma}, x: \cps{T} \\
    \cps{\Gamma, X:\rep{B}} &\defeq \cps{\Gamma}, X:\rep{B}
\end{align}


\section{Proof of dynamic semantics preservation of the CPS transformation}

Regarding dynamic semantics,
we identify values and computations modulo types and predicates
since they are irrelevant to the dynamic semantics.
That is, the following equations hold, for example:
\begin{align}
    \lambda x: \tau_1. c &= \lambda x: \tau_2. c \\
    c[\tau/\alpha] &= c \\
    c[A/X] &= c
\end{align}
% Note that because of this, \rulename{Ec-TApp} can be treated as a deterministic rule,
% and hence, the evaluation of the target language is deterministic too by definition.
Also, We often omit type annotations when they are unnecessary.

Moreover, We also identify values and computations
modulo $\beta$ equivalence of the (static) meta language
(this is admissible because the meta language is pure).
Formally, we define a relation $\equiv_\beta$
as the smallest congruence relation over expressions in the target language
that satisfies the following equations:
\begin{align}
    (\stlambda x: \tau. c) \stapp v &\equiv_\beta c[v/x] \\
    (\stLambda \alpha. c) \stapp \tau &\equiv_\beta c[\tau/\alpha]
\end{align}
and we admit the $\equiv_\beta$-equivalence.

\newcommand{\stappBHK}{\stapp \tau \stapp v_h \stapp v_k}

\begin{assumption} \label{asm:cps:prim-dyn}
    \quad
    \begin{itemize}
        \item $\mathit{cps}(p)~\cps{v} \stappBHK \eval^* \cps{\zeta(p, v)} \stappBHK$
        \item If $\zeta(p, v)$ is undefined, then $\mathit{cps}(p)~\cps{v}$ gets stuck.
        \item $p = \exptrue \iff \mathit{cps}(p) = \exptrue$
        \item $p = \expfalse \iff \mathit{cps}(p) = \expfalse$
    \end{itemize}
\end{assumption}

\begin{lemma}[CPS transformation is homomorphic for substitution] \label{lem:cps:homo-subst-term}
    \quad
    \begin{itemize}
        \item $\cps{v[v_0/x]} = \cps{v}[\cps{v_0}/x]$
        \item $\cps{c[v_0/x]} = \cps{c}[\cps{v_0}/x]$
    \end{itemize}
\end{lemma}
\begin{proof}
    By simultaneous induction on the structure of $v$ and $c$.
\end{proof}

\begin{lemma}[Evaluation with pure evaluation context] \label{lem:cps:pure-eval-ctx}
    \begin{align}
        \cps{K[\op~v]} \stappBHK \eval^*
        v_h\#\op~\rep{A}~\cps{v}~(\lambda y. \cps{K[\expret{y}]} \stappBHK)~.
    \end{align}
\end{lemma}
\begin{proof}
    By induction on the structure of $K$.
    \begin{description}
        \item[{Case $K = [\ ]$:}]
            \begin{align}
                \text{LHS} &= \cps{\op~v} \stappBHK \\
                &= (\stLambda \alpha. \stlambda h. \stlambda k.
                    h\#\op~\rep{A}~\cps{v}~(\lambda y. k~y)) \stappBHK \\
                &\eval^* v_h\#\op~\rep{A}~\cps{v}~(\lambda y. v_k~y) \\
                % &\text{(by equality of the meta language)} \\
                &\equiv_\beta v_h\#\op~\rep{A}~\cps{v}~(\lambda y. (\stLambda \alpha. \stlambda h. \stlambda k. k~y) \stappBHK) \\
                &= v_h\#\op~\rep{A}~\cps{v}~(\lambda y. \cps{\expret{y}} \stappBHK) \\
                &= \text{RHS}
            \end{align}
        \item[Case $K = \explet{x}{K_1}{c_2}$:]
            \begingroup
            \allowdisplaybreaks
            \begin{align}
                \text{LHS} &= \cps{\explet{x}{K_1[\op~v]}{c_2}} \stappBHK \\
                &= (\stLambda \alpha. \stlambda h. \stlambda k.
                    \cps{K_1[\op~v]} \stapp \tau \stapp h \stapp (\lambda x. \cps{c_2} \stapp \tau \stapp h \stapp k))
                    \stappBHK \\
                &\eval^* \cps{K_1[\op~v]} \stapp \tau \stapp v_h \stapp (\lambda x. \cps{c_2} \stappBHK) \\
                &\text{(by the IH)} \\
                &\eval^* v_h\#\op~\rep{A}~\cps{v}
                    ~(\lambda y. \cps{K_1[\expret{y}]} \stapp \tau \stapp v_h \stapp (\lambda x. \cps{c_2} \stappBHK)) \\
                % &\text{(by equality of the meta language)} \\
                &\equiv_\beta v_h\#\op~\rep{A}~\cps{v}
                    ~(\lambda y. (\stLambda \alpha. \stlambda h. \stlambda k.
                    \cps{K_1[\expret{y}]} \stapp \tau \stapp h \stapp (\lambda x. \cps{c_2} \stapp \tau \stapp h \stapp k)
                    ) \stappBHK) \\
                &= v_h\#\op~\rep{A}~\cps{v}~(\lambda y. \cps{\explet{x}{K_1[\expret{y}]}{c_2}} \stappBHK) \\
                &= v_h\#\op~\rep{A}~\cps{v}~(\lambda y. \cps{K[\expret{y}]} \stappBHK) \\
                &= \text{RHS}
            \end{align}
            \endgroup
    \end{description}
\end{proof}

\begin{lemma}[One-step simulation] \label{lem:cps:sim-onestep}
    If $c \eval c'$, then
    $\cps{c} \stappBHK \eval^* \cps{c'} \stappBHK$~.
\end{lemma}
% depends on:: asm:cps:prim-dyn
%              lem:cps:homo-subst-term lem:cps:pure-eval-ctx
\begin{proof}
    By induction on the derivation of $c \eval c'$.
    In the following, we implicitly use Lemma~\ref{lem:cps:homo-subst-term} and
    the equality $c[\tau/\alpha] = c$ and $c[A/X] = c$
    (note that we identify computations modulo types and predicates regarding dynamic semantics).
    \begin{description}
        \item[Case \rulename{E-Let}:]
            \begin{align}
                \text{LHS} &= \cps{\explet{x}{c_1}{c_2}} \stappBHK \\
                &= (\stLambda \alpha. \stlambda h. \stlambda k.
                    \cps{c_1} \stapp \tau \stapp h \stapp (\lambda x. \cps{c_2} \stapp \tau \stapp h \stapp k))
                    \stappBHK \\
                &\eval^* \cps{c_1} \stapp \tau \stapp v_h \stapp (\lambda x. \cps{c_2} \stapp \tau \stapp v_h \stapp v_k) \\
                &\text{(by the IH)} \\
                &\eval^* \cps{c_1'} \stapp \tau \stapp v_h \stapp (\lambda x. \cps{c_2} \stapp \tau \stapp v_h \stapp v_k) \\
                % &\text{(by equality of the meta language)} \\
                &\equiv_\beta (\stLambda \alpha. \stlambda h. \stlambda k.
                    \cps{c_1'} \stapp \tau \stapp h \stapp (\lambda x. \cps{c_2} \stapp \tau \stapp h \stapp k))
                    \stappBHK \\
                &= \cps{\explet{x}{c_1'}{c_2}} \stappBHK \\
                &= \text{RHS}
            \end{align}
        \item[Case \rulename{E-LetRet}:]
            First, w.l.o.g., we can assume that $x \notin \fv(v_h) \cup \fv(v_k)$. Then,
            \begin{align}
                \text{LHS} &= \cps{\explet{x}{\expret{v}}{c_2}} \stappBHK \\
                &= (\stLambda \alpha. \stlambda h. \stlambda k.
                    \cps{\expret{v}} \stapp \tau \stapp h \stapp (\lambda x. \cps{c_2} \stapp \tau \stapp h \stapp k))
                    \stappBHK \\
                &\eval^* \cps{\expret{v}} \stapp \tau \stapp v_h \stapp (\lambda x. \cps{c_2} \stapp \tau \stapp v_h \stapp v_k) \\
                &= (\stLambda \alpha. \stlambda h. \stlambda k. k~\cps{v})
                    \stapp \tau \stapp v_h \stapp (\lambda x. \cps{c_2} \stapp \tau \stapp v_h \stapp v_k) \\
                &\eval^* (\lambda x. \cps{c_2} \stapp \tau \stapp v_h \stapp v_k)~\cps{v} \\
                &\eval (\cps{c_2} \stapp \tau \stapp v_h \stapp v_k)[\cps{v}/x] \\
                &= (\cps{c_2} \stapp \tau \stapp v_h \stapp v_k)[\cps{v}/x] \\
                &= \cps{c_2}[\cps{v}/x] \stapp \tau \stapp v_h \stapp v_k \\
                &= \cps{c_2[v/x]} \stapp \tau \stapp v_h \stapp v_k \\
                &= \text{RHS}
            \end{align}
        \item[Case \rulename{E-IfT}:]
            \begin{align}
                \text{LHS} &= \cps{(\expif{\exptrue}{c_1}{c_2})^C} \stappBHK \\
                &= (\expif{\exptrue}{\cps{c_1}}{\cps{c_2}} : \cps{C}) \stappBHK \\
                &\eval^* \cps{c_1} \stappBHK \\
                &= \text{RHS}
            \end{align}
        \item[Case \rulename{E-IfF}:] similar.
        \item[Case \rulename{E-App}:]
            \begin{align}
                \text{LHS} &= \cps{(\exprec{f}{x}{c})~v} \stappBHK \\
                &= (\exprec{f}{x}{\cps{c}})~\cps{v} \stappBHK \\
                &\eval \cps{c}[\exprec{f}{x}{\cps{c}}/f, \cps{v}/x] \stappBHK \\
                &= \cps{c[\exprec{f}{x}{c}/f, v/x]} \stappBHK \\
                &= \text{RHS}
            \end{align}
        \item[Case \rulename{E-Prim}:] By Assumption~\ref{asm:cps:prim-dyn}.
        \item[Case \rulename{E-Hndl}:]
            \begin{align}
                \text{LHS} &= \cps{\expwith{h}{c}} \stappBHK \\
                &= (\cps{c} \stapp \tau \stapp \cps{h^{\mathit{ops}}} \stapp \cps{h^{\mathit{ret}}})
                    \stappBHK \\
                &\text{(by the IH)} \\
                &= (\cps{c'} \stapp \tau \stapp \cps{h^{\mathit{ops}}} \stapp \cps{h^{\mathit{ret}}})
                    \stappBHK \\
                &= \cps{\expwith{h}{c'}} \stappBHK \\
                &= \text{RHS}
            \end{align}
        \item[Case \rulename{E-HndlRet}:]
            \begin{align}
                \text{LHS} &= \cps{\expwith{h}{\expret{v}}} \stappBHK \\
                &= (\cps{\expret{v}} \stapp \tau \stapp \cps{h^{\mathit{ops}}} \stapp \cps{h^{\mathit{ret}}})
                    \stappBHK \\
                &= (((\stLambda \alpha. \stlambda h. \stlambda k. k~\cps{v}))
                    \stapp \tau \stapp \cps{h^{\mathit{ops}}} \stapp \cps{h^{\mathit{ret}}})
                    \stappBHK \\
                &\eval^* (\cps{h^{\mathit{ret}}}~\cps{v}) \stappBHK \\
                &= ((\lambda x_r. \cps{c_r})~\cps{v}) \stappBHK \\
                &= \cps{c_r}[\cps{v}/x_r] \stappBHK \\
                &= \cps{c_r[v/x_r]} \stappBHK \\
                &= \text{RHS}
            \end{align}
        \item[Case \rulename{E-HndlOp}:]
            \begin{align}
                \text{LHS} &= \cps{\expwith{h}{K[\op_i~v]}} \stappBHK \\
                &= (\cps{K[\op_i~v]} \stapp \tau \stapp \cps{h^{\mathit{ops}}} \stapp \cps{h^{\mathit{ret}}})
                    \stappBHK \\
                &\text{(by Lemma~\ref{lem:cps:pure-eval-ctx})} \\
                &\eval^* \cps{h^{\mathit{ops}}}\#\op_i~\rep{A}~\cps{v}
                    ~(\lambda y. \cps{K[\expret{y}]} \stapp \tau \stapp \cps{h^{\mathit{ops}}} \stapp \cps{h^{\mathit{ret}}}) \\
                &= \cps{h^{\mathit{ops}}}\#\op_i~\rep{A}~\cps{v}~(\lambda y. \cps{\expwith{h}{K[\expret{y}]}}) \\
                &\eval (\Lambda \rep{X_i}. \lambda x_i. \lambda k_i. \cps{c_i})
                    ~\rep{A}~\cps{v}~(\lambda y. \cps{\expwith{h}{K[\expret{y}]}}) \\
                &\eval^* \cps{c_i}[\cps{v}/x_i][\lambda y. \cps{\expwith{h}{K[\expret{y}]}}/k_i] \\
                &= \cps{c_i[v/x_i][\lambda y. \expwith{h}{K[\expret{y}]}/k_i]} \\
                &= \text{RHS}
            \end{align}
    \end{description}
\end{proof}

\newcommand{\stappTop}{\stapp \tau \stapp \{\} \stapp (\lambda x: \tau. x)}

\begin{theorem}[Forward (multi-step) simulation] \label{thm:cps:sim-forward}
    If $c \eval^* \expret{v}$, then
    $\cps{c} \stappTop \eval^+ \cps{v}$~.
\end{theorem}
\begin{proof}
    By applying Lemma~\ref{lem:cps:sim-onestep} repeatedly, we have
    \begin{align}
        \cps{c} \stappTop \eval^* \cps{\expret{v}} \stappTop~.
    \end{align}
    Then,
    \begin{align}
        &\cps{\expret{v}} \stappTop \\
        &= (\stLambda \alpha. \stlambda h. \stlambda k. k~\cps{v}) \stappTop \\
        &\eval^* (\lambda x. x)~\cps{v} \\
        &\eval \cps{v}
    \end{align}
    and therefore we have the conclusion.
\end{proof}

\begin{definition}
    We define evaluation contexts $E$ as follows:
    \begin{align}
        E ::= [\ ] \mid \explet{x}{E}{c} \mid \expwith{h}{E}
    \end{align}
\end{definition}

\newcommand{\bop}{\mathit{bop}}

\begin{definition}
    We define a function $\bop$ as follows:
    \begin{align}
        \bop([\ ]) &\defeq \emptyset \\
        \bop(\explet{x}{E}{c}) &\defeq \bop(E) \\
        \bop(\expwith{h}{E}) &\defeq \dom(h) \cup \bop(E)
    \end{align}
    That is, $\bop(E)$ is a set of operations that are handled by a handler in $E$.
\end{definition}

We say \emph{$c$ is stuck} if $c$ is irreducible and $c \ne \expret{v}$.
We proceed the proof of the backward simulation following \citet{Danvy03}.

\newcommand{\hopsz}{h_0^{\mathtt{ops}}}
\newcommand{\stappBZK}{\stapp \tau \stapp \cps{\hopsz} \stapp v_k}

\begin{lemma}[Preservation of the specific forms of stuck computations] \label{lem:cps:sim-stuck-cases}
    \quad
    \begin{enumerate}
        \item If $c = E[\expif{v}{c_1}{c_2}]$
            where $v$ is not $\exptrue$ nor $\expfalse$,
            then $\cps{c} \stappBHK$ gets stuck.
        \item If $c = E[v_1~v_2]$
            where $v_1$ is not $\exprec{f}{x}{c}$ nor $p$ such that $\zeta(p, v_2)$ is defined,
            then $\cps{c} \stappBHK$ gets stuck.
        \item Let $h_0$ be a handler.
            If $c = E[\op~v]$
            where $\op \notin \bop(E) \cup \dom(h_0)$,
            then $\cps{c} \stapp \tau \stapp \cps{h_0^{\mathtt{ops}}} \stapp v_k$ gets stuck.
    \end{enumerate}
\end{lemma}
\begin{proof}
    \quad
    \begin{enumerate}
        \item \label{enum:stuck-1} By induction on the structure of $E$.
            \begin{description}
                \item[{Case $E = [\ ]$:}]
                    \begin{align}
                        \cps{c} \stappBHK &= \cps{\expif{v}{c_1}{c_2}} \stappBHK \\
                        &= \expif{\cps{v}}{\cps{c_1}}{\cps{c_2}} \stappBHK \\
                    \end{align}
                    From Assumption~\ref{asm:cps:prim-dyn},
                    $\cps{v}$ is neither $\exptrue$ nor $\expfalse$.
                    Therefore, there is no applicable evaluation rule,
                    and hence this computation is stuck.
                \item[Case $E = \explet{x}{E_1}{c}$:]
                    \begin{align}
                        \cps{c} \stappBHK &= \cps{\explet{x}{E_1[\expif{v}{c_1}{c_2}]}{c}} \stappBHK \\
                        &= (\stLambda \alpha. \stlambda h. \stlambda k.
                            \cps{E_1[\expif{v}{c_1}{c_2}]} \stapp \tau \stapp h \stapp (\lambda x. \cps{c} \stapp \tau \stapp h \stapp k)) \stappBHK \hspace*{-2ex} \\
                        &\eval^* \cps{E_1[\expif{v}{c_1}{c_2}]} \stapp \tau \stapp v_h \stapp (\lambda x. \cps{c} \stappBHK)
                    \end{align}
                    By the IH, this computation gets stuck.
                \item[Case $E = \expwith{h}{E_1}$:]
                    \begin{align}
                        \cps{c} \stappBHK &= \cps{\expwith{h}{E_1[\expif{v}{c_1}{c_2}]}} \stappBHK \\
                        &= (\cps{E_1[\expif{v}{c_1}{c_2}]} \stapp \tau \stapp \cps{h^{\mathit{ops}}} \stapp \cps{h^{\mathit{ret}}}) \stappBHK
                    \end{align}
                    By the IH, this computation gets stuck.
            \end{description}
        \item Similar to the case \ref{enum:stuck-1}.
        \item By induction on the structure of $E$.
            \begin{description}
                \item[{Case $E = [\ ]$:}]
                    \begin{align}
                        \cps{c} \stappBZK &= \cps{\op~v} \stappBZK \\
                        &= (\stLambda \alpha. \stlambda h. \stlambda k.
                            h\#\op~\rep{A}~\cps{v}~(\lambda y. \cps{\expret{y}} \stapp \tau \stapp h \stapp k))
                            \stappBZK \\
                        &\eval^* \hopsz\#\op~\rep{A}~\cps{v}~(\lambda y. \cps{\expret{y}} \stappBZK)
                    \end{align}
                    Here, $\hopsz$ does not have a field with $\op$ since $\op \notin \dom(h_0)$.
                    Therefore, there is no applicable evaluation rule,
                    and hence this computation is stuck.
                \item[Case $E = \explet{x}{E_1}{c}$:]
                    \begin{align}
                        \cps{c} \stappBZK &= \cps{\explet{x}{E_1[\op~v]}{c}} \stappBZK \\
                        &= (\stLambda \alpha. \stlambda h. \stlambda k.
                            \cps{E_1[\op~v]} \stapp \tau \stapp h \stapp (\lambda x. \cps{c} \stapp \tau \stapp h \stapp k)) \stappBZK \hspace*{-2ex} \\
                        &\eval^* \cps{E_1[\op~v]} \stapp \tau \stapp \cps{\hopsz} \stapp (\lambda x. \cps{c} \stappBZK)
                    \end{align}
                    Since $\op \notin \bop(E) \cup \dom(h_0)$ and $\bop(E) = \bop(\explet{x}{E_1}{c}) = \bop(E_1)$,
                    it holds that $\op \notin \bop(E_1) \cup \dom(h_0)$.
                    Then, by the IH, this computation gets stuck.
                \item[Case $E = \expwith{h}{E_1}$:]
                    \begin{align}
                        \cps{c} \stappBZK &= \cps{\expwith{h}{E_1[\op~v]}} \stappBZK \\
                        &= (\cps{E_1[\op~v]} \stapp \tau \stapp \cps{h^{\mathit{ops}}} \stapp \cps{h^{\mathit{ret}}}) \stappBZK
                    \end{align}
                    Here, $\op \notin \bop(E) = \bop(\expwith{h}{E_1}) = \bop(E_1) \cup \dom(h)$.
                    Therefore, by the IH, this computation gets stuck.
            \end{description}
    \end{enumerate}
\end{proof}

\begin{lemma}[Preservation of stuck computations] \label{lem:cps:sim-stuck}
    If $c$ is a stuck computation,
    then $\cps{c} \stappTop$ also gets stuck.
\end{lemma}
\begin{proof}
    A stuck computation $c$ is either:
    \begin{itemize}
        \item $E[\expif{v}{c_1}{c_2}]$ where $v$ is not $\exptrue$ nor $\expfalse$,
        \item $E[v_1~v_2]$ where $v_1$ is not $\exprec{f}{x}{c}$ nor $p$ such that $\zeta(p, v_2)$ is defined, or
        \item $E[\op~v]$ where $\op \notin \bop(E)$.
    \end{itemize}
    Therefore, it is immediate from Lemma~\ref{lem:cps:sim-stuck-cases}.
\end{proof}

\begin{theorem}[Backward simulation] \label{thm:cps:sim-backward}
    If $\cps{c} \stappTop \eval^+ v'$, then
    $c \eval^* \expret{v}$ and $\cps{v} = v'$~.
\end{theorem}
\begin{proof}
    We show this theorem by proving its contraposition:
    If ``$c \eval^* \expret{v}$ and $\cps{v} = v'$'' does not hold, then
    $\cps{c} \stappTop \eval^+ v'$
    also does not hold.
    We can divide the situation into two cases:
    \begin{description}
        \item[Case that $c \eval^* \expret{v}$ does not hold:]
            There are two possibilities where $c$ does not evaluate to a value-return.
            \begin{description}
                \item[Case that $c$ diverges:]
                    Since $c$ diverges, for all natural numbers $n$,
                    there exists a sequence
                    \begin{align}
                        c \eval c_1 \eval \cdots \eval c_n~.
                    \end{align}
                    Then, by Lemma~\ref{lem:cps:sim-onestep},
                    we have a sequence
                    \begin{align}
                        \cps{c} \stappTop \eval^+ \cps{c_1} \stappTop \eval^+ \cdots \eval^+ \cps{c_n} \stappTop
                    \end{align}
                    for all $n$.
                    The length of the sequence is at least $n$,
                    and therefore $\cps{c} \stappTop$ has evaluation sequences of arbitrary length,
                    which means it cannot be evaluated to a value.
                \item[Case that $c \eval^* c'$ and $c'$ is stuck:]
                    By applying Lemma~\ref{lem:cps:sim-onestep} repeatedly,
                    we have
                    \begin{align}
                        \cps{c} \stappTop \eval^* \cps{c'} \stappTop~.
                    \end{align}
                    Also, by Lemma~\ref{lem:cps:sim-stuck},
                    it holds that $\cps{c'} \stappTop$ gets stuck.
                    Therefore, $\cps{c} \stappTop$ cannot be evaluated to a value.
            \end{description}
        \item[Case that $c \eval^* \expret{v}$ holds but $\cps{v} = v'$ does not:]
            By Theorem~\ref{thm:cps:sim-forward}, we have
            \begin{align}
                \cps{c} \stappTop \eval^+ \cps{v}~.
            \end{align}
            Then, from the premise $\cps{v} \ne v'$
            and the fact that the evaluation of the target language is deterministic,
            it cannot be the case that $\cps{c} \stappTop \eval^+ v'$~.
    \end{description}
\end{proof}

\begin{corollary}[Simulation] \label{cor:cps:sim}
    If $c \eval^* \expret{v}$, then
    $\cps{c} \stappTop \eval^+ \cps{v}$~.
    Also, if $\cps{c} \stappTop \eval^+ v'$, then
    $c \eval^* \expret{v}$ and $\cps{v} = v'$~.
\end{corollary}
\begin{proof}
    Immediate from Theorem~\ref{thm:cps:sim-forward} and \ref{thm:cps:sim-backward}.
\end{proof}

\section{Proof of type preservation of the CPS transformation}

In the following, we consider static expressions and dynamic ones as identical
since the distinction is irrelevant to the discussion on the type preservation.
In other words, we write $\cps{c} \stappBHK$ as $\cps{c}~\tau~v_h~v_k$ below, for example.

\subsection{Basic properties for the target language of the CPS transformation}

\begin{assumption} \label{asm:cps:formula} \quad
    \begin{itemize}
        \item If $\jdwf{}{\Gamma}$ and $\dom(\Gamma) \supseteq \fv(\phi)$, then $\jdwf{\Gamma}{\phi}$.
        \item If $\jdwf{\Gamma}{\phi}$, then $\jdwf{}{\Gamma}$.
        \item If $\jdwf{\Gamma}{\phi}$, then $\Gamma \vDash \phi \Rarr \phi$.
        \item If $\Gamma \vDash \phi_1 \Rarr \phi_2$ and $\Gamma \vDash \phi_2 \Rarr \phi_3$, then $\Gamma \vDash \phi_1 \Rarr \phi_3$.
        \item If $\jdty{\Gamma}{v}{\tau}$ and $\jdty{\Gamma, x: \tau, \Gamma'}{A}{\rep{B}}$, then $\jdty{\Gamma, \Gamma'[v/x]}{A[v/x]}{\rep{B}}$.
        \item If $\jdty{\Gamma}{v}{\tau}$ and $\jdwf{\Gamma, x: \tau, \Gamma'}{\phi}$, then $\jdwf{\Gamma, \Gamma'[v/x]}{\phi[v/x]}$.
        \item If $\jdty{\Gamma}{v}{\tau}$ and $\valid{\Gamma, x: \tau, \Gamma'}{\phi}$, then $\valid{\Gamma, \Gamma'[v/x]}{\phi[v/x]}$.
        \item If $\jdty{\Gamma}{A}{\rep{B}}$ and $\jdty{\Gamma, X: \rep{B}, \Gamma'}{A'}{\rep{B'}}$, then $\jdty{\Gamma, \Gamma'[A/X]}{A'[A/X]}{\rep{B'}}$.
        \item If $\jdty{\Gamma}{A}{\rep{B}}$ and $\jdwf{\Gamma, X: \rep{B}, \Gamma'}{\phi}$, then $\jdwf{\Gamma, \Gamma'[A/X]}{\phi[A/X]}$.
        \item If $\jdty{\Gamma}{A}{\rep{B}}$ and $\valid{\Gamma, X: \rep{B}, \Gamma'}{\phi}$, then $\valid{\Gamma, \Gamma'[A/X]}{\phi[A/X]}$.
        \item If $\jdwf{}{\Gamma_1, \Gamma_2, \Gamma_3}$ and $\jdwf{\Gamma_1, \Gamma_2}{\phi}$, then $\jdwf{\Gamma_1, \Gamma_2, \Gamma_3}{\phi}$.
        \item If $\jdwf{}{\Gamma_1, \Gamma_2, \Gamma_3}$ and $\jdty{\Gamma_1, \Gamma_2}{A}{\rep{B}}$, then $\jdty{\Gamma_1, \Gamma_2, \Gamma_3}{A}{\rep{B}}$.
        \item If $\jdwf{}{\Gamma_1, \Gamma_2, \Gamma_3}$ and $\Gamma_1, \Gamma_2 \vDash \phi$, then $\Gamma_1, \Gamma_2, \Gamma_3 \vDash \phi$.
        \item If $\jdsub{\Gamma}{\tau_1}{\tau_2}$, $\jdwf{}{\Gamma, x:\tau_1, \Gamma'}$ and $\jdty{\Gamma, x:\tau_2, \Gamma'}{A}{\rep{B}}$, then $\jdty{\Gamma, x:\tau_1, \Gamma'}{A}{\rep{B}}$.
        \item If $\jdsub{\Gamma}{\tau_1}{\tau_2}$, $\jdwf{}{\Gamma, x:\tau_1, \Gamma'}$ and $\jdwf{\Gamma, x:\tau_2, \Gamma'}{\phi}$, then $\jdwf{\Gamma, x:\tau_1, \Gamma'}{\phi}$.
        \item If $\jdsub{\Gamma}{\tau_1}{\tau_2}$ and $\valid{\Gamma, x:\tau_2, \Gamma'}{\phi}$, then $\valid{\Gamma, x:\tau_1, \Gamma'}{\phi}$.
        \item If $x \notin \fv(\Gamma', \phi)$ and $\jdwf{\Gamma, x: \tau_0, \Gamma'}{\phi}$, then $\jdwf{\Gamma, \Gamma'}{\phi}$.
        \item If $x \notin \fv(\Gamma', A)$ and $\jdty{\Gamma, x: \tau_0, \Gamma'}{A}{\rep{B}}$, then $\jdty{\Gamma, \Gamma'}{A}{\rep{B}}$.
        \item If $\jdwf{\Gamma, x: \tau, \Gamma'}{\phi}$ and $\tau$ is not a refinement type, then $x \notin \fv(\Gamma', \phi)$.
        \item If $\jdty{\Gamma, x: \tau, \Gamma'}{A}{\rep{B}}$ and $\tau$ is not a refinement type, then $x \notin \fv(\Gamma', A)$.
        \item If $\Gamma, x: \tau, \Gamma' \vDash \phi$ and $\tau$ is not a refinement type, then $x \notin \fv(\Gamma', \phi)$ and $\Gamma, \Gamma' \vDash \phi$.
        \item If $\alpha \notin \fv(\Gamma', \phi)$ and $\jdwf{\Gamma, \alpha, \Gamma'}{\phi}$, then $\jdwf{\Gamma, \Gamma'}{\phi}$.
        \item If $\alpha \notin \fv(\Gamma', \phi)$ and $\Gamma, \alpha, \Gamma' \vDash \phi$, then $\Gamma, \Gamma' \vDash \phi$.
    \end{itemize}
\end{assumption}

\begin{assumption} \label{asm:cps:prim} \quad
    \begin{itemize}
        \item $\jdwf{}{\tycps(p)}$ for all $p$.
    \end{itemize}
\end{assumption}

\begin{lemma}[Weakening] \label{lem:cps:weaken} \quad
    Assume that $\jdwf{}{\Gamma_1, \Gamma_2, \Gamma_3}$.
    \begin{itemize}
        \item If $\jdwf{\Gamma_1, \Gamma_3}{\tau}$, then $\jdwf{\Gamma_1, \Gamma_2, \Gamma_3}{\tau}$.
        \item If $\jdty{\Gamma_1, \Gamma_3}{c}{\tau}$, then $\jdty{\Gamma_1, \Gamma_2, \Gamma_3}{c}{\tau}$.
        \item If $\jdsub{\Gamma_1, \Gamma_3}{\tau_1}{\tau_2}$, then $\jdsub{\Gamma_1, \Gamma_2, \Gamma_3}{\tau_1}{\tau_2}$.
    \end{itemize}
\end{lemma}
\begin{proof}
    By induction on the derivation. Assumption \ref{asm:cps:formula} is used.
\end{proof}

\begin{lemma}[Narrowing] \label{lem:cps:narrow}
    Assume that $\jdsub{\Gamma}{\tau_1}{\tau_2}$.
    \begin{itemize}
        \item If $\jdwf{}{\Gamma, x: \tau_1, \Gamma'}$ and $\jdwf{\Gamma, x: \tau_2, \Gamma'}{\tau}$,
            then $\jdwf{\Gamma, x: \tau_1, \Gamma'}{\tau}$.
        \item If $\jdwf{}{\Gamma, x: \tau_1, \Gamma'}$ and $\jdty{\Gamma, x: \tau_2, \Gamma'}{c}{\tau}$,
            then $\jdty{\Gamma, x: \tau_1, \Gamma'}{c}{\tau}$.
        \item If $\jdsub{\Gamma, x: \tau_2, \Gamma'}{\tau_1}{\tau_2}$,
            then $\jdsub{\Gamma, x: \tau_1, \Gamma'}{\tau_1}{\tau_2}$.
    \end{itemize}
\end{lemma}
\begin{proof}
    By induction on the derivation. Assumption \ref{asm:cps:formula} is used.
\end{proof}

\begin{lemma}[Remove unused type bindings] \label{lem:cps:rm-unused} \quad
    \begin{itemize}
        \item If $x \notin \fv(\Gamma')$ and $\jdwf{}{\Gamma, x: \tau_0, \Gamma'}$,
            then $\jdwf{}{\Gamma, \Gamma'}$.
        \item If $x \notin \fv(\Gamma', \tau)$ and $\jdwf{\Gamma, x: \tau_0, \Gamma'}{\tau}$,
            then $\jdwf{\Gamma, \Gamma'}{\tau}$.
    \end{itemize}
\end{lemma}
\begin{proof}
    By induction on the derivation.
    The case for \rulename{WTc-Rfn} uses Assumption \ref{asm:cps:formula}.
\end{proof}

\begin{lemma}[Variables of non-refinement types do not apper in types] \label{lem:cps:notin-nonrfn}
    Assume that $\tau_0$ is not a refinement type.
    \begin{itemize}
        \item If $\jdwf{}{\Gamma, x: \tau_0, \Gamma'}$, then  $x \notin \fv(\Gamma')$.
        \item If $\jdwf{\Gamma, x: \tau_0, \Gamma'}{\tau}$, then $x \notin \fv(\Gamma', \tau)$.
    \end{itemize}
\end{lemma}
\begin{proof}
    By induction on the derivation.
    The case for \rulename{WTc-Rfn} uses Assumption \ref{asm:cps:formula}.
\end{proof}

\begin{lemma}[Remove non-refinement type bindings] \label{lem:cps:rm-nonrfn}
    Assume that $\tau_0$ is not a refinement type.
    \begin{enumerate}
        \item If $\jdwf{}{\Gamma, x: \tau_0, \Gamma'}$,
            then $\jdwf{}{\Gamma, \Gamma'}$.
        \item If $\jdwf{\Gamma, x: \tau_0, \Gamma'}{\tau}$,
            then $\jdwf{\Gamma, \Gamma'}{\tau}$.
        \item If $x \notin \fv(c)$ and $\jdty{\Gamma, x: \tau_0, \Gamma'}{c}{\tau}$,
            then $\jdty{\Gamma, \Gamma'}{c}{\tau}$.
        \item If $\jdsub{\Gamma, x: \tau_0, \Gamma'}{\tau_1}{\tau_2}$,
            then $\jdsub{\Gamma, \Gamma'}{\tau_1}{\tau_2}$.
    \end{enumerate}
\end{lemma}
\begin{proof} \quad
    \begin{enumit}
        \item Immediate by Lemma \ref{lem:cps:notin-nonrfn} and \ref{lem:cps:rm-unused}.
        \item Immediate by Lemma \ref{lem:cps:notin-nonrfn} and \ref{lem:cps:rm-unused}.
        \item By induction on the derivation.
            The case for \rulename{Tc-PApp} uses Assumption \ref{asm:cps:formula}.
        \item By induction on the derivation.
            The case for \rulename{Sc-Rfn} uses Assumption \ref{asm:cps:formula}.
        \end{enumit}
\end{proof}

\begin{lemma}[Remove unused type variable bindings] \label{lem:cps:rm-unused-tvar} \quad
    \begin{itemize}
        \item If $\alpha \notin \fv(\Gamma')$
            and $\jdwf{}{\Gamma, \alpha, \Gamma'}$,
            then $\jdwf{}{\Gamma, \Gamma'}$.
        \item If $\alpha \notin \fv(\Gamma', \tau)$
            and $\jdwf{\Gamma, \alpha, \Gamma'}{\tau}$,
            then $\jdwf{\Gamma, \Gamma'}{\tau}$.
        \item If $\alpha \notin \fv(\Gamma', \tau_1, \tau_2)$
            and $\jdsub{\Gamma, \alpha, \Gamma'}{\tau_1}{\tau_2}$,
            then $\jdsub{\Gamma, \Gamma'}{\tau_1}{\tau_2}$.
    \end{itemize}
\end{lemma}
\begin{proof}
    By induction on the derivation.
    The case for \rulename{WTSc-Rfn} and \rulename{Sc-Rfn} uses Assumption \ref{asm:cps:formula}.
\end{proof}

\begin{lemma}[Substitution] \label{lem:cps:subst}
    Assume that $\jdty{\Gamma}{v}{\tau_0}$.
    \begin{itemize}
        \item If $\jdwf{}{\Gamma, x: \tau_0, \Gamma'}$,
            then $\jdwf{}{\Gamma, \Gamma'[v/x]}$.
        \item If $\jdwf{\Gamma, x: \tau_0, \Gamma'}{\tau}$,
            then $\jdwf{\Gamma, \Gamma'[v/x]}{\tau[v/x]}$.
        \item If $\jdty{\Gamma, x: \tau_0, \Gamma'}{c}{\tau}$,
            then $\jdty{\Gamma, \Gamma'[v/x]}{c[v/x]}{\tau[v/x]}$.
        \item If $\jdsub{\Gamma, x: \tau_0, \Gamma'}{\tau_1}{\tau_2}$,
            then $\jdsub{\Gamma, \Gamma'[v/x]}{\tau_1[v/x]}{\tau_2[v/x]}$.
    \end{itemize}
\end{lemma}
\begin{proof}
    By induction on the derivation. Assumption \ref{asm:cps:formula} is used.
\end{proof}

\begin{lemma}[Predicate substitution] \label{lem:cps:subst-pred}
    Assume that $\jdty{\Gamma}{A}{\rep{B}}$.
    \begin{itemize}
        \item If $\jdwf{}{\Gamma, X: \rep{B}, \Gamma'}$,
            then $\jdwf{}{\Gamma, \Gamma'[A/X]}$.
        \item If $\jdwf{\Gamma, X: \rep{B}, \Gamma'}{\tau}$,
            then $\jdwf{\Gamma, \Gamma'[A/X]}{\tau[A/X]}$.
        \item If $\jdty{\Gamma, X: \rep{B}, \Gamma'}{c}{\tau}$,
            then $\jdty{\Gamma, \Gamma'[A/X]}{c[A/X]}{\tau[A/X]}$.
        \item If $\jdsub{\Gamma, X: \rep{B}, \Gamma'}{\tau_1}{\tau_2}$,
            then $\jdsub{\Gamma, \Gamma'[A/X]}{\tau_1[A/X]}{\tau_2[A/X]}$.
    \end{itemize}
\end{lemma}
\begin{proof}
    By induction on the derivation. Assumption \ref{asm:cps:formula} is used.
\end{proof}

\begin{lemma}[Type substitution] \label{lem:cps:subst-type}
    Assume that $\jdwf{\Gamma}{\tau_0}$.
    \begin{itemize}
        \item If $\jdwf{}{\Gamma, \alpha, \Gamma'}$,
            then $\jdwf{}{\Gamma, \Gamma'[\tau_0/\alpha]}$.
        \item If $\jdwf{\Gamma, \alpha, \Gamma'}{\tau}$,
            then $\jdwf{\Gamma, \Gamma'[\tau_0/\alpha]}{\tau[\tau_0/\alpha]}$.
        \item If $\jdty{\Gamma, \alpha, \Gamma'}{c}{\tau}$,
            then $\jdty{\Gamma, \Gamma'[\tau_0/\alpha]}{c[\tau_0/\alpha]}{\tau[\tau_0/\alpha]}$.
        \item If $\jdsub{\Gamma, \alpha, \Gamma'}{\tau_1}{\tau_2}$,
            then $\jdsub{\Gamma, \Gamma'[\tau_0/\alpha]}{\tau_1[\tau_0/\alpha]}{\tau_2[\tau_0/\alpha]}$.
    \end{itemize}
\end{lemma}
\begin{proof}
    By induction on the derivation. Assumption \ref{asm:cps:formula} is used.
\end{proof}

\begin{lemma}[Well-formedness of typing contexts from that of types] \label{lem:cps:wfg}
    If $\jdwf{\Gamma}{\tau}$, then $\jdwf{}{\Gamma}$.
\end{lemma}
\begin{proof}
    By induction on the derivation.
    The case for \rulename{WTc-Rfn} uses Assumption \ref{asm:cps:formula}.
\end{proof}

\begin{lemma}[Well-formedness of types from typings] \label{lem:cps:wft}
    If $\jdty{\Gamma}{c}{\tau}$, then $\jdwf{\Gamma}{\tau}$.
\end{lemma}
\begin{proof}
    By induction on the derivation.
    \begin{description}
        \item[Case \rulename{Tc-CVar}:] By Assumption \ref{asm:cps:formula}.
        \item[Case \rulename{Tc-Var}:] By Lemma \ref{lem:cps:weaken}.
        \item[Case \rulename{Tc-Prim}:]
            By Assumption \ref{asm:cps:prim} and Lemma \ref{lem:cps:weaken}.
        \item[Case \rulename{Tc-Fun}:]
            By the IH, Lemma \ref{lem:cps:rm-nonrfn}, and \rulename{WTc-Fun}.
        \item[Case \rulename{Tc-App}:]
            By the IH, inversion, and Lemma \ref{lem:cps:subst}.
        \item[Case \rulename{Tc-TAbs}:]
            By the IH and \rulename{WTc-TPoly}.
        \item[Case \rulename{Tc-TApp}:]
            By the IH, inversion, and Lemma \ref{lem:cps:subst-type}.
        \item[Case \rulename{Tc-PAbs}:]
            By the IH and \rulename{WTc-PPoly}.
        \item[Case \rulename{Tc-PApp}:]
            By the IH, inversion, and Lemma \ref{lem:cps:subst-pred}.
        \item[Case \rulename{Tc-If}:]
            By the IH and Lemma \ref{lem:cps:rm-unused}.
        \item[Case \rulename{Tc-Ascr} and \rulename{Tc-Sub}:]
            Immediate.
    \end{description}
\end{proof}

\begin{lemma}[Reflexivity] \label{lem:cps:refl}
    If $\jdwf{\Gamma}{\tau}$, then $\jdsub{\Gamma}{\tau}{\tau}$.
\end{lemma}
\begin{proof}
    By induction on the derivation.
    The case for \rulename{WTc-Rfn} uses Assumption \ref{asm:cps:formula}.
\end{proof}

\begin{lemma}[Transitivity] \label{lem:cps:trans}
    If $\jdsub{\Gamma}{\tau_1}{\tau_2}$ and $\jdsub{\Gamma}{\tau_2}{\tau_3}$, then $\jdsub{\Gamma}{\tau_1}{\tau_3}$.
\end{lemma}
\begin{proof}
    By induction on the structure of $\tau_2$.
    Assumption~\ref{asm:cps:formula}, Lemma~\ref{lem:cps:narrow}, and \ref{lem:cps:weaken} are used.
\end{proof}

\begin{lemma}[Inversion] \label{lem:cps:inv} \quad
    \begin{itemize}
        \item If $\jdty{\Gamma}{x}{\tau}$, then either
            \begin{itemize}
                \item $\jdwf{}{\Gamma}$ and  $\jdsub{\Gamma}{\tyrfn{z}{B}{z = x}}{\tau}$
                    (if $\Gamma(x) = \tyrfn{z}{B}{\phi}$ for some $z, B$ and $\phi$)
                \item $\jdwf{}{\Gamma}$ and $\jdsub{\Gamma}{\Gamma(x)}{\tau}$ (otherwise)
            \end{itemize}
        \item If $\jdty{\Gamma}{p}{\tau}$, then
            $\jdwf{}{\Gamma}$ and  $\jdsub{\Gamma}{\tycps(p)}{\tau}$.
        \item If $\jdty{\Gamma}{\exprec{f:(x:\tau_1) \rarr \tau_2}{x:\tau_1}{c}}{\tau}$, then
            $\jdty{\Gamma, f:(x:\tau_1) \rarr \tau_2, x:\tau_1}{c}{\tau_2}$ and
            $\jdsub{\Gamma}{(x:\tau_1) \rarr \tau_2}{\tau}$.
        \item If $\jdty{\Gamma}{\Lambda \alpha. c}{\tau}$, then
            $\jdty{\Gamma, \alpha}{c}{\tau'}$ and $\jdsub{\Gamma}{\forall \alpha. \tau'}{\tau}$
            for some $\tau'$.
        \item If $\jdty{\Gamma}{\{\repi{\op_i = v_i}\}}{\tau}$, then
            $\bigrepi{\jdty{\Gamma}{v_i}{\tau_i}}$ and $\jdsub{\Gamma}{\{\op_i: \tau_i\}}{\tau}$
            for some $\repi{\tau_i}$.
        \item If $\jdty{\Gamma}{c~v}{\tau}$, then
            $\jdty{\Gamma}{c}{(x:\tau_1) \rarr \tau_2}$, $\jdty{\Gamma}{v}{\tau_1}$
            and $\jdsub{\Gamma}{\tau_2[v/x]}{\tau}$
            for some $x, \tau_1$ and $\tau_2$.
        \item If $\jdty{\Gamma}{c~\rep{A}}{\tau}$, then
            $\jdty{\Gamma}{c}{\forall \rep{X:\rep{B}}. \tau'}$, $\rep{\jdty{\Gamma}{A}{\rep{B}}}$
            and $\jdsub{\Gamma}{\tau'[\rep{A/X}]}{\tau}$
            for some $\rep{X}, \rep{\rep{B}}$ and $\tau'$.
        \item If $\jdty{\Gamma}{c~\tau'}{\tau}$, then
            $\jdty{\Gamma}{c}{\forall \alpha. \tau_1}$, $\jdwf{\Gamma}{\tau'}$
            and $\jdsub{\Gamma}{\tau_1[\tau'/\alpha]}{\tau}$
            for some $\alpha$ and $\tau_1$.
        \item If $\jdty{\Gamma}{v\#\op}{\tau}$, then
            $\jdty{\Gamma}{v}{\{\ldots, \op:\tau, \ldots\}}$.
        \item If $\jdty{\Gamma}{(c : \tau')}{\tau}$, then
            $\jdty{\Gamma}{c}{\tau'}$ and $\jdsub{\Gamma}{\tau'}{\tau}$.
        \item If $\jdty{\Gamma}{\expif{v}{c_1}{c_2}}{\tau}$, then
            $\jdty{\Gamma}{v}{\tyrfn{z}{\tybool}{\phi}}$,
            $\jdty{\Gamma, v = \exptrue}{c_1}{\tau'}$, $\jdty{\Gamma, v = \expfalse}{c_2}{\tau'}$, 
            and $\jdsub{\Gamma}{\tau'}{\tau}$
            for some $z, \phi$ and $\tau'$.
    \end{itemize}
\end{lemma}
\begin{proof}
    By induction on the derivation. Lemma \ref{lem:cps:refl} and \ref{lem:cps:trans} are used.
\end{proof}

\begin{lemma}[Inversion for CPS-transformed computations] \label{lem:cps:inv-c}
    If $\jdty{\Gamma}{\Lambda \alpha. \lambda h:\tau_h. \lambda k:\tau_k. c}{\tau}$
    and neither $\tau_h$ nor $\tau_k$ is a refinement type,
    then there exists some $\tau'$ such that
    \begin{itemize}
        \item $\jdty{\Gamma, \alpha, h:\tau_h, k:\tau_k}{c}{\tau'}$ and
        \item $\jdsub{\Gamma}{\forall \alpha. \tau_h \rarr \tau_k \rarr \tau'}{\tau}$~.
    \end{itemize}
\end{lemma}
\begin{proof}
    By Lemma \ref{lem:cps:inv}, \rulename{Sc-Poly}, \rulename{Sc-Fun}, and Lemma \ref{lem:cps:trans}.
\end{proof}

\begin{lemma}[Inversion for the specific form of application] \label{lem:cps:inv-c-app}
    If $\jdty{\Gamma}{c~\tau_0~v_1~v_2}{\tau}$, then
    there exist some $\tau', \tau_1$, and $\tau_2$ such that
    \begin{itemize}
        \item $\jdty{\Gamma}{c}{\tau'}$,
        \item $\jdty{\Gamma}{v_1}{\tau_1}$, and
        \item $\jdty{\Gamma}{v_2}{\tau_2}$~.
    \end{itemize}
    In addition, if $\jdsub{\Gamma}{\tau_1'}{\tau_1}$ and $\jdsub{\Gamma}{\tau_2'}{\tau_2}$
    for some $\tau_1'$ and $\tau_2'$
    and neither $\tau_1'$ nor $\tau_2'$ is a refinement type,
    then $\jdsub{\Gamma}{\tau'}{\forall \alpha. \tau_1' \rarr \tau_2' \rarr \tau}$
    where $\alpha$ is fresh.
\end{lemma}
\begin{proof}
    The first half is by Lemma \ref{lem:cps:inv}.
    The second half is by Lemma \ref{lem:cps:wft}, \ref{lem:cps:weaken} and \ref{lem:cps:trans}
    with the results of the first half.
\end{proof}



\subsection{Forward type preservation}

\begin{assumption} \label{asm:cps:formula-cps} \quad
    \begin{itemize}
        \item If $\jdwf{\Gamma}{\phi}$, then $\jdwf{\cps{\Gamma}}{\phi}$.
        \item If $\jdty{\Gamma}{A}{\rep{B}}$, then $\jdty{\cps{\Gamma}}{A}{\rep{B}}$.
        \item If $\Gamma \vDash \phi$, then $\cps{\Gamma} \vDash \phi$.
    \end{itemize}
\end{assumption}

\begin{assumption} \label{asm:cps:prim-cps} \quad
    \begin{itemize}
        \item $\cps{\ty(p)} = \tycps(\cps{p})$.
        \item If $\ty(p) = \tyrfn{x}{B}{\phi}$ for some $x, B$ and $\phi$, then $\cps{p} = p$.
    \end{itemize}
\end{assumption}

\begin{lemma}[CPS transformation preserves free variables in types] \label{lem:cps:presv-fv} \quad
    \begin{itemize}
        \item $\fv(\cps{T}) = \fv(T)$.
        \item $\fv(\cps{C}) = \fv(C)$.
        \item $\fv(\cps{\Sigma}) = \fv(\Sigma)$.
    \end{itemize}
\end{lemma}
\begin{proof}
    By simultaneous induction on the structure of types.
\end{proof}

\begin{lemma}[CPS transformation is homomorphic for substitution] \label{lem:cps:homo-subst} \quad
    \begin{itemize}
        \item $\cps{T[v/x]} = \cps{T}[\cps{v}/x]$.
        \item $\cps{C[v/x]} = \cps{C}[\cps{v}/x]$.
        \item $\cps{\Sigma[v/x]} = \cps{\Sigma}[\cps{v}/x]$.
        \item $\cps{T[A/X]} = \cps{T}[A/X]$.
        \item $\cps{C[A/X]} = \cps{C}[A/X]$.
        \item $\cps{\Sigma[A/X]} = \cps{\Sigma}[A/X]$.
    \end{itemize}
\end{lemma}
\begin{proof}
    By simultaneous induction on the structure of types.
    The case for $T = \tyrfn{x}{B}{\phi}$ uses Assumption \ref{asm:cps:prim-cps}.
\end{proof}

\begin{lemma}[CPS transformation preserves well-formedness] \label{lem:cps:presv-wf} \quad
    \begin{itemize}
        \item If $\jdwf{}{\Gamma}$, then $\jdwf{}{\cps{\Gamma}}$.
        \item If $\jdwf{\Gamma}{T}$, then $\jdwf{\cps{\Gamma}}{\cps{T}}$.
        \item If $\jdwf{\Gamma}{C}$, then $\jdwf{\cps{\Gamma}}{\cps{C}}$.
        \item If $\jdwf{\Gamma}{\Sigma}$, then $\jdwf{\cps{\Gamma}}{\cps{\Sigma}}$.
    \end{itemize}
\end{lemma}
\begin{proof}
    By simultaneous induction on the derivations. Lemma \ref{lem:cps:weaken} is used.
    The case for \rulename{WT-Rfn} uses Assumption \ref{asm:cps:formula-cps}.
\end{proof}

\begin{lemma}[CPS transformation preserves subtyping] \label{lem:cps:presv-sub} \quad
    \begin{itemize}
        \item If $\jdsub{\Gamma}{T_1}{T_2}$, then $\jdsub{\cps{\Gamma}}{\cps{T_1}}{\cps{T_2}}$.
        \item If $\jdsub{\Gamma}{C_1}{C_2}$, then $\jdsub{\cps{\Gamma}}{\cps{C_1}}{\cps{C_2}}$.
        \item If $\jdsub{\Gamma}{\Sigma_1}{\Sigma_2}$, then $\jdsub{\cps{\Gamma}}{\cps{\Sigma_1}}{\cps{\Sigma_2}}$.
    \end{itemize}
\end{lemma}
\begin{proof}
    By simultaneous induction on the derivations. Lemma \ref{lem:cps:weaken} is used.
    The case for \rulename{S-Rfn} uses Assumption \ref{asm:cps:formula-cps}.
\end{proof}

\begin{theorem}[Forward type preservation] \quad
    \begin{enumerate}
        \item If\, $\jdty{\Gamma}{v}{T}$, then $\jdty{\cps{\Gamma}}{\cps{v}}{\cps{T}}$.
        \item If\, $\jdty{\Gamma}{c}{C}$, then $\jdty{\cps{\Gamma}}{\cps{c}}{\cps{C}}$.
    \end{enumerate}
\end{theorem}
% depend on:: asm:cps:prim asm:cps:formula
%             lem:cps:presv-wf lem:cps:presv-wf lem:cps:presv-sub
%             lem:cps:weaken lem:cps:homo-subst
\begin{proof}
    By simultaneous induction on the typing derivation of the source language.
    \begin{enumit}
        \item 
        \begin{description}
            \item[Case \rulename{T-CVar}:] By Lemma \ref{lem:cps:presv-wf},
                definition of CPS transformation of typing contexts, and \rulename{Tc-CVar}.
            \item[Case \rulename{T-Var}:] By Lemma \ref{lem:cps:presv-wf},
                definition of CPS transformation of typing contexts, and \rulename{Tc-Var}.
            \item[Case \rulename{T-Prim}:] By Lemma \ref{lem:cps:presv-wf},
                Assumption \ref{asm:cps:prim-cps}, and \rulename{Tc-Prim}.
            \item[Case \rulename{T-Fun}:] By the IH and \rulename{Tc-Fun}.
            \item[Case \rulename{T-VSub}:] By the IH, Lemma \ref{lem:cps:presv-sub},
                Lemma \ref{lem:cps:presv-wf} and \rulename{Tc-Sub}.
        \end{description}
        \item 
        \begin{description}
            \item[Case \rulename{T-Ret}:] we have
                \begin{itemize}
                    \item $c = \expret{v}$,
                    \item $C = \tycomp{\{\}}{T}{\square}$, and
                    \item $\jdty{\Gamma}{v}{T}$
                \end{itemize}
                for some $v$ and $T$.
                Then, we have
                \begin{itemize}
                    \item $\cps{c} = \Lambda \alpha. \lambda h:\{\}. \lambda k:\cps{T} \rarr \alpha. k~\cps{v}$ and
                    \item $\cps{C} = \forall \alpha. \{\} \rarr (\cps{T} \rarr \alpha) \rarr \alpha$~.
                \end{itemize}
                By the IH, we have
                \begin{itemize}
                    \item $\jdty{\cps{\Gamma}}{\cps{v}}{\cps{T}}$~.
                \end{itemize}
                We have the conclusion by the following derivation with Lemma \ref{lem:cps:weaken}:
                \[
                    \infersc[Tc-TAbs]{\jdty{\cps{\Gamma}}{\Lambda \alpha. \lambda h:\{\}. \lambda k:\cps{T} \rarr \alpha. k~\cps{v}}
                        {\forall \alpha. \{\} \rarr (\cps{T} \rarr \alpha) \rarr \alpha}}
                    {
                        \infersc[Tc-Lam]{\jdty{\cps{\Gamma}, \alpha}{\lambda h:\{\}. \lambda k:\cps{T} \rarr \alpha. k~\cps{v}}
                            {\{\} \rarr (\cps{T} \rarr \alpha) \rarr \alpha}}
                        {
                            \infersc[Tc-Lam]{\jdty{\cps{\Gamma}, \alpha, h: \{\}}{\lambda k\cps{T} \rarr \alpha. k~\cps{v}}
                                {(\cps{T} \rarr \alpha) \rarr \alpha}}
                            {
                                \infersc[Tc-App]{\jdty{\cps{\Gamma}, \alpha, h: \{\}, k:\cps{T} \rarr \alpha}
                                    {k~\cps{v}}{\alpha}}
                                {
                                    \infersc[Tc-Var]{\jdty{\Gamma_{\alpha,h,k}}
                                        {k}{\cps{T} \rarr \alpha}}
                                    {}
                                    &
                                    \jdty{\Gamma_{\alpha,h,k}}
                                        {\cps{v}}{\cps{T}}
                                }
                            }
                        }
                    }
                \]
                where $\Gamma_{\alpha,h,k} \defeq \cps{\Gamma}, \alpha, h: \{\}, k:\cps{T} \rarr \alpha$~.
            \item[Case \rulename{T-App}:] By the IH, Lemma \ref{lem:cps:homo-subst}
                and \rulename{Tc-App}.
            \item[Case \rulename{T-If}:] By the IH, \rulename{Tc-If} and \rulename{Tc-Ascr}.
            \item[Case \rulename{T-CSub}:] similar to the case for \rulename{T-VSub}.
            \item[Case \rulename{T-Let}:] We have
                \begin{itemize}
                    \item $c = \explet{x}{c_1}{c_2}$,
                    \item $C = \tycomp{\Sigma}{T_2}{\bind{S_1}{x}{S_2}}$,
                    \item $\jdty{\Gamma}{c_1}{\tycomp{\Sigma}{T_1}{S_1}}$,
                    \item $\jdty{\Gamma, x: T_1}{c_2}{\tycomp{\Sigma}{T_2}{S_2}}$, and
                    \item $x \notin \fv(T_2) \cup \fv(\Sigma)$
                \end{itemize}
                for some $x, c_1, c_2, \Sigma, T_1, T_2, S_1$ and $S_2$.
                By Lemma \ref{lem:cps:presv-fv}, we have
                \begin{itemize}
                    \item $x \notin \fv(\cps{T_2}) \cup \fv(\cps{\Sigma})$~.
                \end{itemize}
                Case analysis on the definition of $\bind{S_1}{x}{S_2}$.
                \begin{description}
                    \item[Case $S_1 = S_2 = \square$:] We have
                        \begin{itemize}
                            \item $\cps{c} = \Lambda \alpha. \lambda h: \cps{\Sigma}. \lambda k:\cps{T_2} \rarr \alpha.
                                \cps{c_1}~\nmbullet~h~(\lambda x:\cps{T_1}. \cps{c_2}~\nmbullet~h~k)$ and
                            \item $\cps{C} = \forall \alpha. \cps{\Sigma} \rarr (\cps{T_2} \rarr \alpha) \rarr \alpha$~.
                        \end{itemize}
                        Also, by the IHs, we have
                        \begin{itemize}
                            \item $\jdty{\cps{\Gamma}}{\cps{c_1}}
                                {\forall \beta. \cps{\Sigma} \rarr (\cps{T_1} \rarr \beta) \rarr \beta}$ and
                            \item $\jdty{\cps{\Gamma}, x:\cps{T_1}}{\cps{c_2}}
                                {\forall \gamma. \cps{\Sigma} \rarr (\cps{T_2} \rarr \gamma) \rarr \gamma}$~.
                        \end{itemize}
                        We have the conclusion by the following derivations with Lemma \ref{lem:cps:weaken}:
                        \[  (A):
                            \infersc[Tc-App]{
                                \jdty{\cps{\Gamma}_{\alpha, h, k}}
                                    {\cps{c_1}~\nmbullet~h}{(\cps{T_1} \rarr \alpha) \rarr \alpha}
                            }{
                                \infersc[Tc-App]{
                                    \jdty{\Gamma_{\alpha, h, k}}
                                        {\cps{c_1}~\nmbullet~h}{(\cps{T_1} \rarr \alpha) \rarr \alpha}
                                }{
                                    \infersc[Tc-TApp]{
                                        \jdty{\Gamma_{\alpha, h, k}}
                                            {\cps{c_1}~\nmbullet}{\cps{\Sigma} \rarr (\cps{T_1} \rarr \alpha) \rarr \alpha}
                                    }{
                                        \jdty{\Gamma_{\alpha, h, k}}
                                            {\cps{c_1}}{\forall \beta. \cps{\Sigma} \rarr (\cps{T_1} \rarr \beta) \rarr \beta}
                                        &
                                        \jdwf{\Gamma_{\alpha, h, k}}{\alpha}
                                    }
                                    &
                                    \infersc[Tc-Var]{
                                        \jdty{\Gamma_{\alpha, h, k}}
                                            {h}{\cps{\Sigma}}
                                    }{}
                                }
                            }
                        \]
                        \[  (B):
                            \infersc[Tc-App]{
                                \jdty{\Gamma_{\alpha, h, k, x}}
                                    {\cps{c_2}~\nmbullet~h}{(\cps{T_2} \rarr \alpha) \rarr \alpha}
                            }{
                                \infersc[Tc-TApp]{
                                    \jdty{\Gamma_{\alpha, h, k, x}}
                                        {\cps{c_2}~\nmbullet}{\cps{\Sigma} \rarr (\cps{T_2} \rarr \alpha) \rarr \alpha}
                                }{
                                    \jdty{\Gamma_{\alpha, h, k, x}}
                                        {\cps{c_2}}{\forall \gamma. \cps{\Sigma} \rarr (\cps{T_2} \rarr \gamma) \rarr \gamma}
                                    &
                                    \jdwf{\Gamma_{\alpha, h, k, x}}{\alpha}
                                }
                                &
                                \infersc[Tc-Var]{
                                    \jdty{\Gamma_{\alpha, h, k, x}}
                                        {h}{\cps{\Sigma}}
                                }{}
                            }
                        \]
                        \[
                            \infersc[Tc-TAbs]{
                                \jdty{\cps{\Gamma}}{\Lambda \alpha. \lambda h: \cps{\Sigma}. \lambda k:\cps{T_2} \rarr \alpha.
                                    \cps{c_1}~\nmbullet~h~(\lambda x:\cps{T_1}. \cps{c_2}~\nmbullet~h~k)}
                                    {\forall \alpha. \cps{\Sigma} \rarr (\cps{T_2} \rarr \alpha) \rarr \alpha}
                            }{
                                \infersc[Tc-Fun]{
                                    \jdty{\cps{\Gamma}, \alpha}{\lambda h: \cps{\Sigma}. \lambda k:\cps{T_2} \rarr \alpha.
                                        \cps{c_1}~\nmbullet~h~(\lambda x:\cps{T_1}. \cps{c_2}~\nmbullet~h~k)}
                                        {\cps{\Sigma} \rarr (\cps{T_2} \rarr \alpha) \rarr \alpha}
                                }{
                                    \infersc[Tc-Fun]{
                                        \jdty{\cps{\Gamma}, \alpha, h: \cps{\Sigma}}{\lambda k:\cps{T_2} \rarr \alpha.
                                            \cps{c_1}~\nmbullet~h~(\lambda x:\cps{T_1}. \cps{c_2}~\nmbullet~h~k)}
                                            {(\cps{T_2} \rarr \alpha) \rarr \alpha}
                                    }{
                                        \infersc[Tc-App]{
                                            \jdty{\cps{\Gamma}, \alpha, h: \cps{\Sigma}, k:\cps{T_2} \rarr \alpha}{
                                                \cps{c_1}~\nmbullet~h~(\lambda x:\cps{T_1}. \cps{c_2}~\nmbullet~h~k)}
                                                {\alpha}
                                        }{
                                            (A)
                                            &
                                            \infersc[Tc-Fun]{
                                                \jdty{\Gamma_{\alpha, h, k}}
                                                    {\lambda x:\cps{T_1}. \cps{c_2}~\nmbullet~h~k}{\cps{T_1} \rarr \alpha}
                                            }{
                                                \infersc[Tc-App]{
                                                    \jdty{\Gamma_{\alpha, h, k}, x:\cps{T_1}}
                                                        {\cps{c_2}~\nmbullet~h~k}{\alpha}
                                                }{
                                                    (B)
                                                    &
                                                    \infersc[Tc-Var]{
                                                        \jdty{\Gamma_{\alpha, h, k, x}}
                                                            {k}{\cps{T_2} \rarr \alpha}
                                                    }{}
                                                }
                                            }
                                        }
                                    }
                                }
                            }
                        \]
                        where $\Gamma_{\alpha, h, k} \defeq \cps{\Gamma}, \alpha, h: \cps{\Sigma}, k:\cps{T_2} \rarr \alpha$
                        and $\Gamma_{\alpha, h, k, x} \defeq \Gamma_{\alpha, h, k}, x:\cps{T_1}$~.
                    \item[Case $S_1 = \tyctl{x}{C_1}{C_2}$ and  $S_2 = \tyctl{x}{C_0}{C_1}$:] We have
                        \begin{itemize}
                            \item $\cps{c} = \Lambda \alpha. \lambda h: \cps{\Sigma}. \lambda k:(z:\cps{T_2}) \rarr \cps{C_0}.
                                (\cps{c_1}~\nmbullet~h~(\lambda x:\cps{T_1}. (\cps{c_2}~\nmbullet~h~k : \cps{C_1})) : \cps{C_2})$ and
                            \item $\cps{C} = \forall \alpha. \cps{\Sigma} \rarr ((z:\cps{T_2}) \rarr \cps{C_0}) \rarr \cps{C_2}$~.
                        \end{itemize}
                        Also, by the IHs, we have
                        \begin{itemize}
                            \item $\jdty{\cps{\Gamma}}{\cps{c_1}}
                                {\forall \alpha. \cps{\Sigma} \rarr ((x:\cps{T_1}) \rarr \cps{C_1}) \rarr \cps{C_2}}$ and
                            \item $\jdty{\cps{\Gamma}, x:\cps{T_1}}{\cps{c_2}}
                                {\forall \alpha. \cps{\Sigma} \rarr ((z:\cps{T_2}) \rarr \cps{C_0}) \rarr \cps{C_1}}$~.
                        \end{itemize}
                        We have the conclusion by a straightforward derivation
                        like the case for $S_1 = S_2 = \square$
                        using those judgements shown so far and Lemma \ref{lem:cps:weaken}.
                        % We have the conclusion by the following derivations with Lemma \ref{lem:cps:weaken}:
                        % \todo[derivation]
                \end{description}
            \item[Case \rulename{T-Op}:]
                (In this case, we use Lemma \ref{lem:cps:homo-subst} frequently and implicitly.) \\
                We have
                \begin{itemize}
                    \item $c = \expop{v}{y}{c}$,
                    \item $C = \tycomp{\Sigma}{T_3}{\tyctl{z}{C_0}{C_2[\rep{A/X}][v/x]}}$,
                    \item $\Sigma \ni \op: \forall \rep{X: \rep{B}}. (x: T_1) \rarr ((y: T_2) \rarr C_1) \rarr C_2$,
                    \item $y \notin \fv(T_3) \cup \fv(\Sigma) \cup (\fv(C_0) \setminus \{ z \})$,
                    \item $\rep{\jdty{\Gamma}{A}{\rep{B}}}$,
                    \item $\jdty{\Gamma}{v}{T_1[\rep{A/X}]}$, and
                    \item $\jdty{\Gamma, y: T_2[\rep{A/X}][v/x]}{c}{\tycomp{\Sigma}{T_3}{\tyctl{z}{C_0}{C_1[\rep{A/X}][v/x]}}}$,
                \end{itemize}
                for some $x, y, z, v, c, \rep{X}, \rep{A}, \rep{\rep{B}}, \Sigma, T_1, T_2, T_3, C_0, C_1$ and $C_2$.
                Then, we have
                \begin{itemize}
                    \item $\cps{c} = \Lambda \alpha. \lambda h:\cps{\Sigma}. \lambda k:(z: \cps{T} \rarr \cps{C_1}).
                        h\#\op~\rep{A}~\cps{v}~(\lambda y:\cps{T_y}. \cps{c}~\nmbullet~h~k)$,
                    \item $\cps{C} = \forall \alpha. \cps{\Sigma} \rarr ((z: \cps{T_3}) \rarr \cps{C_0}) \rarr \cps{C_2}[\rep{A/X}][\cps{v}/x]$, and
                    \item $\cps{\Sigma} \ni \op: \forall \rep{X: \rep{B}}. (x: \cps{T_1}) \rarr ((y: \cps{T_2}) \rarr \cps{C_1}) \rarr \cps{C_2}$~.
                \end{itemize}
                Also, by the IHs, we have
                \begin{itemize}
                    \item $\jdty{\cps{\Gamma}}{\cps{v}}{\cps{T_1}[\rep{A/X}]}$ and
                    \item $\jdty{\cps{\Gamma}, y: \cps{T_2}[\rep{A/X}][\cps{v}/x]}{\cps{c}}
                        {\forall \alpha. \cps{\Sigma} \rarr ((z:\cps{T_3}) \rarr \cps{C_0}) \rarr \cps{C_1}[\rep{A/X}][\cps{v}/x]}$~.
                \end{itemize}
                By Lemma \ref{lem:cps:presv-fv}, we have
                \begin{itemize}
                    \item $y \notin \fv(\cps{T_3}) \cup \fv(\cps{\Sigma}) \cup (\fv(\cps{C_0}) \setminus \{ z \})$~.
                \end{itemize}
                By Lemma \ref{asm:cps:formula-cps}, we have
                \begin{itemize}
                    \item $\rep{\jdty{\cps{\Gamma}}{A}{\rep{B}}}$~.
                \end{itemize}
                We have the conclusion by a straightforward derivation
                like the cases for \rulename{T-Ret} and \rulename{T-Let}
                using those judgements shown so far and Lemma \ref{lem:cps:weaken}.
                % We have the conclusion by the following derivations with Lemma \ref{lem:cps:weaken}:
                % \todo[derivation]
            \item[Case \rulename{T-Hndl}:] We have
                \begin{itemize}
                    \item $c = \expwith{h}{c_0}$,
                    \item $h = \{ \expret{x_r} \mapsto c_r, \repi{\op_i(x_i, k_i) \mapsto c_i} \}$,
                    \item $\jdty{\Gamma}{c_0}{\tycomp{\Sigma_0}{T_r}{\tyctl{x_r}{C_1}{C}}}$,
                    \item $\jdty{\Gamma, x_r: T_r}{c_r}{C_1}$,
                    \item $\bigrepi{\jdty{\Gamma, \rep{X_i: \rep{B}_i}, x_i: T_{i1}, k_i: (y_i: T_{i2}) \rarr C_{i1}}{c_i}{C_{i2}}}$, and
                    \item $\Sigma_0 = \{ \repi{\op_i: \forall \rep{X_i: \rep{B}_i}. (x_i: T_{i1}) \rarr ((y_i: T_{i2}) \rarr C_{i1}) \rarr C_{2i}} \}$
                \end{itemize}
                Then, we have
                \begin{itemize}
                    \item $\cps{c} = \cps{c}~\nmbullet~\cps{h^{\mathit{ops}}}~\cps{h^{\mathit{ret}}}$,
                    \item $\cps{h^{\mathit{ret}}} = \lambda x_r:\cps{T_r}. \cps{c_r}$,
                    \item $\cps{h^{\mathit{ops}}} = \{ \repi{\op_i = \Lambda \rep{X_i: \rep{B_i}}. \lambda x_i:\cps{T_{i1}}. \lambda k_i:(y_i:\cps{T_{i2}}) \rarr \cps{C_{i1}}. \cps{c_i}} \}$, and
                    \item $\cps{\Sigma_0} = \{ \repi{\op_i: \forall \rep{X_i: \rep{B}_i}. (x_i: \cps{T_{i1}}) \rarr ((y_i: \cps{T_{i2}}) \rarr \cps{C_{i1}}) \rarr \cps{C_{i2}}} \}$~.
                \end{itemize}
                Also, by the IHs, we have
                \begin{itemize}
                    \item $\jdty{\cps{\Gamma}}{\cps{c_0}}{\forall \alpha. \cps{\Sigma_0} \rarr ((x_r: \cps{T_r}) \rarr \cps{C_1}) \rarr \cps{C}}$,
                    \item $\jdty{\cps{\Gamma}, x_r: \cps{T_r}}{\cps{c_r}}{\cps{C_1}}$, and
                    \item $\bigrepi{\jdty{\cps{\Gamma}, \rep{X_i: \rep{B}_i}, x_i: \cps{T_{i1}}, k_i: (y_i: \cps{T_{i2}}) \rarr \cps{C_{i1}}}{\cps{c_i}}{\cps{C_{i2}}}}$~.
                \end{itemize}
                We have the conclusion by a straightforward derivation
                like the cases for \rulename{T-Ret} and \rulename{T-Let}
                using those judgements shown so far and Lemma \ref{lem:cps:weaken}.
                % We have the conclusion by the following derivations with Lemma \ref{lem:cps:weaken}:
                % \todo[derivation]
        \end{description}
    \end{enumit}
\end{proof}


\subsection{Backward type preservation}

For the backward direction, we define some notations.
\begin{definition}
    $\Gamma$ is \emph{cps-wellformed}
    if for all $(x:\tau) \in \Gamma$, it holds that $\tau = \cps{T}$ for some $T$.
\end{definition}
\begin{definition}
    $\rmtv$ is a function which removes all bindings of type variables
    from a typing context. Formally, it is defined as follows:
    \begin{align}
        \rmtv(\emptyset) &\defeq \emptyset &
        \rmtv(\Gamma, x: \tau) &\defeq \rmtv(\Gamma), x: \tau \\
        \rmtv(\Gamma, X: \rep{B}) &\defeq \rmtv(\Gamma), X: \rep{B} &
        \rmtv(\Gamma, \alpha) &\defeq \rmtv(\Gamma)
    \end{align}
\end{definition}

\begin{lemma}[CPS-wellformed target typing contexts have corresponding source ones] \label{lem:cps:cpswf-rmtv}
    If $\Gamma$ is cps-wellformed,
    then there exists some $\Gamma'$ such that $\cps{\Gamma'} = \rmtv(\Gamma)$.
\end{lemma}
\begin{proof}
    By induction on the structure of $\Gamma$.
\end{proof}

Since the CPS transformation is injective,
there is only one $\Gamma'$ which satisfies the equation in Lemma \ref{lem:cps:cpswf-rmtv}.
Therefore, we define a function $\cpsinv{-}$ that maps $\Gamma$ to $\Gamma'$:
\begin{definition}
    Let $\Gamma$ be a cps-wellformed typing context in the target language.
    We define $\cpsinv{\Gamma}$ to be the typing context in the source language
    such that $\cps{\cpsinv{\Gamma}} = \rmtv(\Gamma)$.
\end{definition}

\begin{assumption} \label{asm:cps:formula-cpsinv}
    Assume that $\Gamma$ is cps-wellformed.
    \begin{itemize}
        \item If $\jdwf{\Gamma}{\phi}$, then $\jdwf{\cpsinv{\Gamma}}{\phi}$.
        \item If $\jdty{\Gamma}{A}{\rep{B}}$, then $\jdty{\cpsinv{\Gamma}}{A}{\rep{B}}$.
        \item If $\Gamma \vDash \phi$, then $\cpsinv{\Gamma} \vDash \phi$.
    \end{itemize}
\end{assumption}

\begin{lemma}[Computation types in the specific form of subtyping are pure] \label{lem:cps:sub-pure}
    If $\jdsub{\Gamma}{\cps{C}}{\forall \alpha. \tau_1 \rarr (\tau_2 \rarr \beta) \rarr \tau_4}$
    and $\beta \in \Gamma$,
    then $C = \tycomp{\Sigma}{T}{\square}$ (for some $\Sigma$ and $T$),
    and $\tau_4 = \beta$.
\end{lemma}
\begin{proof}
    Assume that $C = \tycomp{\Sigma}{T}{\tyctl{x}{C_1}{C_2}}$ for some $\Sigma, T, x, C_1$ and $C_2$.
    Then, we have
    \[
        \jdsub{\Gamma}
            {\forall \gamma. \cps{\Sigma} \rarr ((x: \cps{T}) \rarr \cps{C_1}) \rarr \cps{C_2}}
            {\forall \alpha. \tau_1 \rarr (\tau_2 \rarr \beta) \rarr \tau_4}
    \]
    where $\gamma$ is fresh.
    By inversion, we have $\jdsub{\Gamma, \alpha, h:\tau_1, x:\cps{T}}{\beta}{\cps{C_1}}$,
    that is, $\jdsub{\Gamma, \alpha, h:\tau_1, x:\cps{T}}{\beta}{\forall \delta. \tau_5}$
    for some $\tau_5$ and $\delta$.
    This is contradictory since there is no subtyping rule for such a judgment.

    Therefore, $C = \tycomp{\Sigma}{T}{\square}$ for some $\Sigma$ and $T$.
    In this case, we have
    \[
        \jdsub{\Gamma}
            {\forall \gamma. \cps{\Sigma} \rarr (\cps{T} \rarr \gamma) \rarr \gamma}
            {\forall \alpha. \tau_1 \rarr (\tau_2 \rarr \beta) \rarr \tau_4}
    \]
    where $\gamma$ is fresh.
    By inversion, we have
    \begin{itemize}
        \item $\jdwf{\Gamma, \alpha}{\tau_6}$,
        \item $\jdsub{\Gamma, \alpha, h:\tau_1, x:\cps{T}[\tau_6/\gamma]}{\beta}{\gamma[\tau_6/\gamma]}$, and
        \item $\jdsub{\Gamma, \alpha, h:\tau_1, x:\cps{T}[\tau_6/\gamma]}{\gamma[\tau_6/\gamma]}{\tau_4}$
    \end{itemize}
    for some $\tau_6$.
    The second judgment can be derived by only \rulename{Sc-TVar} where $\gamma[\tau_6/\gamma] = \beta$.
    Therefore, the third judgment becomes
    $\jdsub{\Gamma, \alpha, h:\tau_1, x:\cps{T}[\tau_6/\gamma]}{\beta}{\tau_4}$,
    which can be derived by only \rulename{Sc-TVar} where $\tau_4 = \beta$.
\end{proof}

\begin{lemma}[Computation types can be assumed to be impure] \label{lem:assume-atm}
    If $\jdty{\Gamma}{c}{C}$, then w.l.o.g.,
    we can assume that $C = \tycomp{\Sigma}{T}{\tyctl{x}{C_1}{C_2}}$
    for some $\Sigma, T, x, C_1$ and $C_2$.
\end{lemma}
\begin{proof} \quad
    \begin{description}
        \item[Case $C = \tycomp{\Sigma}{T}{\tyctl{x}{C_1}{C_2}}$:] Immediate.
        \item[Case $C = \tycomp{\Sigma}{T}{\square}$:]
            It holds that
            $\jdsub{\Gamma}{\tycomp{\Sigma}{T}{\square}}{\tycomp{\Sigma}{T}{\tyctl{x}{C_0}{C_0}}}$
            for any $C_0$ such that $\jdwf{\Gamma}{C_0}$.
            Therefore, by subsumption
            we have $\jdty{\Gamma}{c}{\tycomp{\Sigma}{T}{\tyctl{x}{C_0}{C_0}}}$~.
    \end{description}
\end{proof}

\begin{lemma}[Backward preservation on well-formedness] \label{lem:cps:presv-b-wf}
    Assume that $\Gamma$ is cps-wellformed.
    \begin{enumerate}
        \item If $\jdwf{}{\Gamma}$, then $\jdwf{}{\cpsinv{\Gamma}}$.
        \item If $\jdwf{\Gamma}{\cps{T}}$, then $\jdwf{\cpsinv{\Gamma}}{T}$.
        \item If $\jdwf{\Gamma}{\cps{C}}$, then $\jdwf{\cpsinv{\Gamma}}{C}$.
        \item If $\jdwf{\Gamma}{\cps{\Sigma}}$, then $\jdwf{\cpsinv{\Gamma}}{\Sigma}$.
    \end{enumerate}
\end{lemma}

\begin{proof}
    By simultaneous induction on the derivation.
    \begin{enumit}
        \item
        \begin{description}
            \item[Case \rulename{WEc-Empty}:] Obvious since $\cpsinv{\emptyset} = \emptyset$.
            \item[Case \rulename{WEc-Var}:] We have
                \begin{itemize}
                    \item $\Gamma = \Gamma', x:\tau$,
                    \item $\jdwf{}{\Gamma'}$,
                    \item $x \notin \dom(\Gamma')$, and
                    \item $\jdwf{\Gamma'}{\tau}$
                \end{itemize}
                for some $\Gamma', x$, and $\tau$.
                By the IH, we have $\jdwf{}{\cpsinv{\Gamma'}}$.
                Also, we have $x \notin \dom(\cpsinv{\Gamma'})$
                since $\dom(\Gamma') \supseteq \dom(\cpsinv{\Gamma'})$.
                Moreover, since $\Gamma$ is cps-wellformed, $\tau = \cps{T}$ for some $T$.
                Then, by the IH, we have $\jdwf{\cpsinv{\Gamma'}}{T}$.
                We have the conclusion by \rulename{WE-Var}.
                (Note that $\cpsinv{\Gamma} = \cpsinv{\Gamma', x:\cps{T}} = \cpsinv{\Gamma'}, x: T$.)
            \item[Case \rulename{WEc-BVar}:] By the IH and \rulename{WE-BVar}.
            \item[Case \rulename{WEc-PVar}:] By the IH and \rulename{WE-PVar}.
            \item[Case \rulename{WEc-TVar}:] By the IH.
                Note that $\cpsinv{\Gamma', \alpha} = \cpsinv{\Gamma'}$.
        \end{description}
        \item Case analysis on $T$.
        \begin{description}
            \item[Case $T = \tyrfn{z}{B}{\phi}$:]
                By Assumption \ref{asm:cps:formula-cpsinv} and \rulename{WT-Rfn}.
            \item[Case $T = (x:T_1) \rarr C_1$:] By the IHs and \rulename{WT-Fun}.
        \end{description}
        \item Case analysis on $C$.
        \begin{description}
            \item[Case $C = \tycomp{\Sigma}{T}{\square}$:] We have
                $\cps{C} = \forall \alpha. \cps{\Sigma} \rarr (\cps{T} \rarr \alpha) \rarr \alpha$
                for some $\alpha$.
                By inversion, we have
                \begin{itemize}
                    \item $\jdwf{\Gamma, \alpha}{\cps{\Sigma}}$ and
                    \item $\jdwf{\Gamma, \alpha, h: \cps{\Sigma}}{\cps{T}}$~.
                \end{itemize}
                By \ref{lem:rm-nonrfn}, we have
                \begin{itemize}
                    \item $\jdwf{\Gamma, \alpha}{\cps{T}}$~.
                \end{itemize}
                By the IHs, we have
                \begin{itemize}
                    \item $\jdwf{\cpsinv{\Gamma}}{\Sigma}$ and
                    \item $\jdwf{\cpsinv{\Gamma}}{T}$~.
                \end{itemize}
                Also, by \rulename{WT-Pure}, we have $\jdwf{\cpsinv{\Gamma} \mid T}{\square}$.
                Then we have the conclusion by \rulename{WT-Comp}.
            \item[Case $C = \tycomp{\Sigma}{T}{\tyctl{x}{C_1}{C_2}}$:] We have
                $\cps{C} = \forall \alpha. \cps{\Sigma} \rarr ((x:\cps{T}) \rarr \cps{C_1}) \rarr \cps{C_2}$~.
                By inversion, we have
                \begin{itemize}
                    \item $\jdwf{\Gamma, \alpha}{\cps{\Sigma}}$,
                    \item $\jdwf{\Gamma, \alpha, h: \cps{\Sigma}}{\cps{T}}$,
                    \item $\jdwf{\Gamma, \alpha, h: \cps{\Sigma}, x:\cps{T}}{\cps{C_1}}$, and
                    \item $\jdwf{\Gamma, \alpha, h: \cps{\Sigma}, k: (x:\cps{T}) \rarr \cps{C_1}}{\cps{C_2}}$~.
                \end{itemize}
                By \ref{lem:rm-nonrfn}, we have
                \begin{itemize}
                    \item $\jdwf{\Gamma, \alpha}{\cps{T}}$,
                    \item $\jdwf{\Gamma, \alpha, x:\cps{T}}{\cps{C_1}}$, and
                    \item $\jdwf{\Gamma, \alpha}{\cps{C_2}}$~.
                \end{itemize}
                By the IHs, we have
                \begin{itemize}
                    \item $\jdwf{\cpsinv{\Gamma}}{\Sigma}$,
                    \item $\jdwf{\cpsinv{\Gamma}}{T}$,
                    \item $\jdwf{\cpsinv{\Gamma}, x:T}{C_1}$, and
                    \item $\jdwf{\cpsinv{\Gamma}}{C_2}$~.
                \end{itemize}
                Then we have the conclusion by \rulename{WT-ATM} and \rulename{WT-Comp}.
        \end{description}
        \item By the IHs and \rulename{WT-Sig}.
    \end{enumit}
\end{proof}

\begin{lemma}[Backward preservation on subtyping] \label{lem:cps:cpsinv-sub}
    Assume that $\Gamma$ is cps-wellformed.
    \begin{enumerate}
        \item If $\jdsub{\Gamma}{\cps{T_1}}{\cps{T_2}}$, then $\jdsub{\cpsinv{\Gamma}}{T_1}{T_2}$.
        \item If $\jdsub{\Gamma}{\cps{C_1}}{\cps{C_2}}$, then $\jdsub{\cpsinv{\Gamma}}{C_1}{C_2}$.
        \item If $\jdsub{\Gamma}{\cps{\Sigma_1}}{\cps{\Sigma_2}}$, then $\jdsub{\cpsinv{\Gamma}}{\Sigma_1}{\Sigma_2}$.
    \end{enumerate}
\end{lemma}


\begin{proof}
    By simultaneous induction on the derivation.
    \begin{enumit}
        \item Case analysis on $T_1$ and $T_2$.
        \begin{description}
            \item[Case $T_1 = \tyrfn{z}{B}{\phi_1}$ and $T_2 = \tyrfn{z}{B}{\phi_2}$:] We have
                \begin{itemize}
                    \item $\cps{T_1} = \tyrfn{z}{B}{\phi_1}$ and
                    \item $\cps{T_2} = \tyrfn{z}{B}{\phi_2}$~.
                \end{itemize}
                We have the conclusion
                by Assumption \ref{asm:cps:formula-cpsinv} and \rulename{S-Rfn}.
            \item[Case $T_1 = (x:T_{10}) \rarr C_1$ and $T_1 = (x:T_{10}) \rarr C_1$:] We have
                \begin{itemize}
                    \item $\cps{T_1} = (x:\cps{T_{10}}) \rarr \cps{C_1}$ and
                    \item $\cps{T_2} = (x:\cps{T_{20}}) \rarr \cps{C_2}$.
                \end{itemize}
                We have the conclusion by the IHs and \rulename{S-Fun}.
            \item[Otherwise:] Contradictory since there is no applicable rule.
        \end{description}
        \item Case analysis on $C_1$ and $C_2$.
        \begin{description}
            \item[Case $C_1 = \tycomp{\Sigma_1}{T_1}{\square}$
                and $C_2 = \tycomp{\Sigma_2}{T_2}{\square}$:] We have
                \begin{itemize}
                    \item $\cps{C_1} = \forall \alpha. \cps{\Sigma_1} \rarr (\cps{T_1} \rarr \alpha) \rarr \alpha$ and
                    \item $\cps{C_2} = \forall \beta. \cps{\Sigma_2} \rarr (\cps{T_2} \rarr \beta) \rarr \beta$
                \end{itemize}
                for some $\alpha$ and $\beta$.
                By inversion, we have
                \begin{itemize}
                    \item $\jdwf{\Gamma, \beta}{\tau}$,
                    \item $\jdsub{\Gamma, \beta}{\cps{\Sigma_2}}{\cps{\Sigma_1}[\tau/\alpha]}$ and
                    \item $\jdsub{\Gamma, \beta, h: \cps{\Sigma_2}}{\cps{T_1}}{{\cps{T_2}[\tau/\alpha]}}$
                \end{itemize}
                for some $\tau$.
                Since CPS-transformed types do not contain type variables, we have
                \begin{itemize}
                    \item $\jdsub{\Gamma, \beta}{\cps{\Sigma_2}}{\cps{\Sigma_1}}$ and
                    \item $\jdsub{\Gamma, \beta, h: \cps{\Sigma_2}}{\cps{T_1}}{{\cps{T_2}}}$~.
                \end{itemize}
                By \ref{lem:rm-nonrfn}, we have
                \begin{itemize}
                    \item $\jdsub{\Gamma, \beta}{\cps{\Sigma_2}}{\cps{\Sigma_1}}$ and
                    \item $\jdsub{\Gamma, \beta}{\cps{T_1}}{{\cps{T_2}}}$~.
                \end{itemize}
                By the IHs, we have
                \begin{itemize}
                    \item $\jdsub{\cpsinv{\Gamma}}{\Sigma_2}{\Sigma_1}$ and
                    \item $\jdsub{\cpsinv{\Gamma}}{T_1}{T_2}$~.
                \end{itemize}
                Also, by \rulename{S-Pure}, we have $\jdsub{\cpsinv{\Gamma} \mid T_1}{\square}{\square}$.
                Then we have the conclusion by \rulename{S-Comp}.
            \item[Case $C_1 = \tycomp{\Sigma_1}{T_1}{\tyctl{x}{C_{11}}{C_{12}}}$
                and $C_2 = \tycomp{\Sigma_2}{T_2}{\tyctl{x}{C_{21}}{C_{22}}}$:] We have
                \begin{itemize}
                    \item $\cps{C_1} = \forall \alpha. \cps{\Sigma_1} \rarr ((x:\cps{T_1}) \rarr \cps{C_{11}}) \rarr \cps{C_{12}}$ and
                    \item $\cps{C_2} = \forall \beta. \cps{\Sigma_2} \rarr ((x:\cps{T_2}) \rarr \cps{C_{21}}) \rarr \cps{C_{22}}$~.
                \end{itemize}
                By inversion, we have
                \begin{itemize}
                    \item $\jdwf{\Gamma, \beta}{\tau}$,
                    \item $\jdsub{\Gamma, \beta}{\cps{\Sigma_2}}{\cps{\Sigma_1}[\tau/\alpha]}$,
                    \item $\jdsub{\Gamma, \beta, h: \cps{\Sigma_2}}{\cps{T_1}[\tau/\alpha]}{{\cps{T_2}}}$,
                    \item $\jdsub{\Gamma, \beta, h: \cps{\Sigma_2}, x:\cps{T_1}[\tau/\alpha]}{\cps{C_{21}}}{\cps{C_{11}}[\tau/\alpha]}$, and
                    \item $\jdsub{\Gamma, \beta, h: \cps{\Sigma_2}, k: (x:\cps{T_2}) \rarr \cps{C_{21}}}{\cps{C_{12}}[\tau/\alpha]}{\cps{C_22}}$~.
                \end{itemize}
                for some $\tau$.
                Since CPS-transformed types do not contain type variables, we have
                \begin{itemize}
                    \item $\jdsub{\Gamma, \beta}{\cps{\Sigma_2}}{\cps{\Sigma_1}}$,
                    \item $\jdsub{\Gamma, \beta, h: \cps{\Sigma_2}}{\cps{T_1}}{{\cps{T_2}}}$,
                    \item $\jdsub{\Gamma, \beta, h: \cps{\Sigma_2}, x:\cps{T_1}}{\cps{C_{21}}}{\cps{C_{11}}}$, and
                    \item $\jdsub{\Gamma, \beta, h: \cps{\Sigma_2}, k: (x:\cps{T_2}) \rarr \cps{C_{21}}}{\cps{C_{12}}}{\cps{C_22}}$~.
                \end{itemize}
                By \ref{lem:rm-nonrfn}, we have
                \begin{itemize}
                    \item $\jdsub{\Gamma, \beta}{\cps{\Sigma_2}}{\cps{\Sigma_1}}$,
                    \item $\jdsub{\Gamma, \beta}{\cps{T_1}}{{\cps{T_2}}}$,
                    \item $\jdsub{\Gamma, \beta, x:\cps{T_1}}{\cps{C_{21}}}{\cps{C_{11}}}$, and
                    \item $\jdsub{\Gamma, \beta}{\cps{C_{12}}}{\cps{C_22}}$~.
                \end{itemize}
                By the IHs, we have
                \begin{itemize}
                    \item $\jdsub{\cpsinv{\Gamma}}{\Sigma_2}{\Sigma_1}$,
                    \item $\jdsub{\cpsinv{\Gamma}}{T_1}{T_2}$,
                    \item $\jdsub{\cpsinv{\Gamma}, x:T_1}{C_{21}}{C_{11}}$, and
                    \item $\jdsub{\cpsinv{\Gamma}}{C_{12}}{C_{22}}$~.
                \end{itemize}
                Then we have the conclusion by \rulename{S-ATM} and \rulename{S-Comp}.
            \item[Case $C_1 = \tycomp{\Sigma_1}{T_1}{\square}$
                and $C_2 = \tycomp{\Sigma_2}{T_2}{\tyctl{x}{C_{21}}{C_{22}}}$:] We have
                \begin{itemize}
                    \item $\cps{C_1} = \forall \alpha. \cps{\Sigma_1} \rarr (\cps{T_1} \rarr \alpha) \rarr \alpha$ and
                    \item $\cps{C_2} = \forall \beta. \cps{\Sigma_2} \rarr ((x:\cps{T_2}) \rarr \cps{C_{21}}) \rarr \cps{C_{22}}$~.
                \end{itemize}
                W.l.o.g., we can assume that $x \notin \cps{C_{22}}$.
                By inversion, we have
                \begin{itemize}
                    \item $\jdwf{\Gamma, \beta}{\tau}$,
                    \item $\jdsub{\Gamma, \beta}{\cps{\Sigma_2}}{\cps{\Sigma_1}[\tau/\alpha]}$,
                    \item $\jdsub{\Gamma, \beta, h: \cps{\Sigma_2}}{\cps{T_1}[\tau/\alpha]}{{\cps{T_2}}}$,
                    \item $\jdsub{\Gamma, \beta, h: \cps{\Sigma_2}, x:\cps{T_1}[\tau/\alpha]}{\cps{C_{21}}}{\alpha[\tau/\alpha]}$, and
                    \item $\jdsub{\Gamma, \beta, h: \cps{\Sigma_2}, k: (x:\cps{T_2}) \rarr \cps{C_{21}}}{\alpha[\tau/\alpha]}{\cps{C_22}}$~.
                \end{itemize}
                for some $\tau$.
                Since CPS-transformed types do not contain type variables, we have
                \begin{itemize}
                    \item $\jdsub{\Gamma, \beta}{\cps{\Sigma_2}}{\cps{\Sigma_1}}$,
                    \item $\jdsub{\Gamma, \beta, h: \cps{\Sigma_2}}{\cps{T_1}}{{\cps{T_2}}}$,
                    \item $\jdsub{\Gamma, \beta, h: \cps{\Sigma_2}, x:\cps{T_1}}{\cps{C_{21}}}{\tau}$, and
                    \item $\jdsub{\Gamma, \beta, h: \cps{\Sigma_2}, k: (x:\cps{T_2}) \rarr \cps{C_{21}}}{\tau}{\cps{C_22}}$~.
                \end{itemize}
                By \ref{lem:rm-nonrfn}, we have
                \begin{itemize}
                    \item $\jdsub{\Gamma, \beta}{\cps{\Sigma_2}}{\cps{\Sigma_1}}$,
                    \item $\jdsub{\Gamma, \beta}{\cps{T_1}}{{\cps{T_2}}}$,
                    \item $\jdsub{\Gamma, \beta, x:\cps{T_1}}{\cps{C_{21}}}{\tau}$, and
                    \item $\jdsub{\Gamma, \beta}{\tau}{\cps{C_22}}$~.
                \end{itemize}
                By Lemma \ref{lem:cps:weaken} and \ref{lem:cps:trans}, we have
                \begin{itemize}
                    \item $\jdsub{\Gamma, \beta}{\cps{\Sigma_2}}{\cps{\Sigma_1}}$,
                    \item $\jdsub{\Gamma, \beta}{\cps{T_1}}{{\cps{T_2}}}$, and
                    \item $\jdsub{\Gamma, \beta, x:\cps{T_1}}{\cps{C_{21}}}{\cps{C_22}}$~.
                \end{itemize}
                By the IHs, we have
                \begin{itemize}
                    \item $\jdsub{\cpsinv{\Gamma}}{\Sigma_2}{\Sigma_1}$,
                    \item $\jdsub{\cpsinv{\Gamma}}{T_1}{T_2}$, and
                    \item $\jdsub{\cpsinv{\Gamma}, x:T_1}{C_{21}}{C_{22}}$~.
                \end{itemize}
                Then we have the conclusion by \rulename{S-Embed} and \rulename{S-Comp}.
            \item[Case $C_1 = \tycomp{\Sigma_1}{T_1}{\tyctl{x}{C_{11}}{C_{12}}}$
                and $C_2 = \tycomp{\Sigma_2}{T_2}{\square}$:] We have
                \begin{itemize}
                    \item $\cps{C_1} = \forall \alpha. \cps{\Sigma_1} \rarr ((x:\cps{T_1}) \rarr \cps{C_{11}}) \rarr \cps{C_{12}}$ and
                    \item $\cps{C_2} = \forall \beta. \cps{\Sigma_2} \rarr (\cps{T_2} \rarr \beta) \rarr \beta$~.
                \end{itemize}
                By inversion, we have
                \begin{itemize}
                    \item $\jdwf{\Gamma, \beta}{\tau}$, and
                    % \item $\jdsub{\Gamma, \beta}{\cps{\Sigma_2}}{\cps{\Sigma_1}[\tau/\alpha]}$,
                    % \item $\jdsub{\Gamma, \beta, h: \cps{\Sigma_2}}{\cps{T_1}[\tau/\alpha]}{{\cps{T_2}}}$,
                    % \item $\jdsub{\Gamma, \beta, h: \cps{\Sigma_2}, x:\cps{T_1}[\tau/\alpha]}{\beta}{\cps{C_{11}}[\tau/\alpha]}$, and
                    \item $\jdsub{\Gamma, \beta, h: \cps{\Sigma_2}, k: (x:\cps{T_2}) \rarr \beta}{\cps{C_{12}}[\tau/\alpha]}{\beta}$~.
                \end{itemize}
                for some $\tau$.
                Since CPS-transformed types do not contain type variables, we have
                \begin{itemize}
                    % \item $\jdsub{\Gamma, \beta}{\cps{\Sigma_2}}{\cps{\Sigma_1}}$,
                    % \item $\jdsub{\Gamma, \beta, h: \cps{\Sigma_2}}{\cps{T_1}}{{\cps{T_2}}}$,
                    % \item $\jdsub{\Gamma, \beta, h: \cps{\Sigma_2}, x:\cps{T_1}}{\beta}{\cps{C_{11}}}$, and
                    \item $\jdsub{\Gamma, \beta, h: \cps{\Sigma_2}, k: (x:\cps{T_2}) \rarr \beta}{\cps{C_{12}}}{\beta}$~.
                \end{itemize}
                This is contradictory since $\cps{C_{12}}$ cannot be a type variable
                and thus there is no applicable rule.
        \end{description}
        \item By the IHs, and \rulename{S-Sig}.
    \end{enumit}
\end{proof}

% Using these notations and lemmas, the backward direction is stated as follows.
\begin{theorem}[Backward type preservation (for open expressions)] \label{thm:cps-backward}
    Assume that $\Gamma$ is cps-wellformed.
    \begin{enumerate}
        \item If $\jdty{\Gamma}{\cps{v}}{\tau}$, then
            there exists $T$ such that
            $\jdty{\cpsinv{\Gamma}}{v}{T}$ and $\jdsub{\Gamma}{\cps{T}}{\tau}$.
        \item If $\jdty{\Gamma}{\cps{c}}{\tau}$, then
            there exists $C$ such that
            $\jdty{\cpsinv{\Gamma}}{c}{C}$ and $\jdsub{\Gamma}{\cps{C}}{\tau}$.
    \end{enumerate}
\end{theorem}
% depends on:: 
\begin{proof}
    By simultaneous induction on the structure of $v$ and $c$.
    \begin{enumit}
        \item
        \begin{description}
            \item[Case $v = x$:] We have $\cps{v} = x$.
            By Lemma \ref{lem:cps:inv}, we have \emph{either}
            \begin{enumerate}
                \item $\jdwf{}{\Gamma}$ and  $\jdsub{\Gamma}{\tyrfn{z}{B}{z = x}}{\tau}$
                    (if $\Gamma(x) = \tyrfn{z}{B}{\phi}$ for some $z, B$ and $\phi$)
                \item $\jdwf{}{\Gamma}$ and $\jdsub{\Gamma}{\Gamma(x)}{\tau}$ (otherwise)
            \end{enumerate}
            \begin{description}
                \item[Case 1:]
                    By Lemma \ref{lem:cps:presv-b-wf}, we have $\jdwf{}{\cpsinv{\Gamma}}$.
                    Also, since $\cps{\tyrfn{z}{B}{\phi'}} = \tyrfn{z}{B}{\phi'}$ for any $\phi'$, we have
                    \begin{itemize}
                        \item $\cpsinv{\Gamma}(x) = \tyrfn{z}{B}{\phi}$ and
                        \item $\jdsub{\Gamma}{\cps{\tyrfn{z}{B}{z = x}}}{\tau}$~.
                    \end{itemize}
                    Then, by \rulename{T-CVar}, we have $\jdty{\cpsinv{\Gamma}}{x}{\tyrfn{z}{B}{z = x}}$.
                    Now we have the conclusion with $T = \tyrfn{z}{B}{z = x}$.
                \item[Case 2:]
                    By Lemma \ref{lem:cps:presv-b-wf}, we have $\jdwf{}{\cpsinv{\Gamma}}$.
                    Then, since
                    $\Gamma(x) = \rmtv(\Gamma)(x) = \cps{\cpsinv{\Gamma}}(x) = \cps{\cpsinv{\Gamma}(x)}$
                    holds by Lemma \ref{lem:cps:cpswf-rmtv},
                    we have $\jdsub{\Gamma}{\cps{\cpsinv{\Gamma}(x)}}{\tau}$.
                    Also, by \rulename{T-Var}, we have $\jdty{\cpsinv{\Gamma}}{x}{\cpsinv{\Gamma}(x)}$.
                    Now we have the conclusion with $T = \cpsinv{\Gamma}(x)$.
            \end{description}
            \item[Case $v = p$:] We have $\cps{v} = \mathit{cps}(p)$.
                By Lemma \ref{lem:cps:inv}, we have
                \begin{itemize}
                    \item $\jdwf{}{\Gamma}$ and
                    \item $\jdsub{\Gamma}{\tycps(\cps{p})}{\tau}$.
                \end{itemize}
                By Lemma \ref{lem:cps:presv-b-wf}, we have $\jdwf{}{\cpsinv{\Gamma}}$.
                Then, by \rulename{T-Prim}, we have $\jdty{\cpsinv{\Gamma}}{p}{\ty(p)}$.
                Also, by Assumption \ref{asm:cps:prim-cps},
                we have $\jdsub{\Gamma}{\cps{\ty(p)}}{\tau}$.
                Now we have the conclusion with $T = \ty(p)$.
            \item[Case $v = \exprec{f^{(x:T_1) \rarr C_1}}{x^{T_2}}{c}$:]
                We have $\cps{v} = \exprec{f:(x:\cps{T_1}) \rarr \cps{C_1}}{x:\cps{T_2}}{\cps{c}}$.
                By Lemma \ref{lem:cps:inv}, we have
                \begin{itemize}
                    \item $\jdty{\Gamma, f:(x:\cps{T_1}) \rarr \cps{C_1}, x: \cps{T_1}}{\cps{c}}{\cps{C_1}}$ and
                    \item $\jdsub{\Gamma}{(x:\cps{T_1}) \rarr \cps{C_1}}{\tau}$~.
                \end{itemize}
                By the IH, we have
                \begin{itemize}
                    \item $\jdty{\cpsinv{\Gamma}, f:(x:T_1) \rarr C_1, x: T_1}{c}{C_1'}$ and
                    \item $\jdsub{\Gamma, f:(x:\cps{T_1}) \rarr \cps{C_1}, x: \cps{T_1}}{\cps{C_1'}}{\cps{C_1}}$
                \end{itemize}
                for some $C_1'$.
                By Lemma \ref{lem:cps:cpsinv-sub}, we have
                \[
                    \jdsub{\cpsinv{\Gamma}, f:(x:T_1) \rarr C_1, x: T_1}{C_1'}{C_1}~.
                \]
                Then, by \rulename{T-CSub} and \rulename{T-Fun}, we have
                \[
                    \jdty{\Gamma}{\exprec{f^{(x:T_1) \rarr C_1}}{x^{T_2}}{c}}{(x:T_1) \rarr C_1}~.
                \]
                Now we have the conclusion with $T = (x:T_1) \rarr C_1$.
        \end{description}
        \item
        \begin{description}
            \item[Case $c = \expret{v^T}$:]
                We have $\cps{c} = \Lambda \alpha. \lambda h:\{\}. \lambda k:\cps{T} \rarr \alpha. k~\cps{v}$~.
                By Lemma \ref{lem:cps:inv-c}, we have
                \begin{itemize}
                    \item $\jdty{\Gamma, \alpha, h:\{\}, k:\cps{T} \rarr \alpha}{k~\cps{v}}{\tau'}$ and
                    \item $\jdsub{\Gamma}{\forall \alpha. \{\} \rarr (\cps{T} \rarr \alpha) \rarr \tau'}{\tau}$
                \end{itemize}
                for some $\tau'$.
                By Lemma \ref{lem:cps:inv}, we have
                \begin{itemize}
                    \item $\jdty{\Gamma, \alpha, h:\{\}, k:\cps{T} \rarr \alpha}{k}{(y:\tau_1) \rarr \tau_2}$,
                    \item $\jdty{\Gamma, \alpha, h:\{\}, k:\cps{T} \rarr \alpha}{\cps{v}}{\tau_1}$, and
                    \item $\jdsub{\Gamma, \alpha, h:\{\}, k:\cps{T} \rarr \alpha}{\tau_2[\cps{v}/y]}{\tau'}$
                \end{itemize}
                for some $y, \tau_1$, and $\tau_2$.
                By Lemma \ref{lem:cps:rm-nonrfn}, we have
                \begin{itemize}
                    \item $\jdty{\Gamma, \alpha}{\cps{v}}{\tau_1}$~.
                \end{itemize}
                Then, by the IH, we have
                \begin{itemize}
                    \item $\jdty{\cpsinv{\Gamma}}{v}{T}$ and
                    \item $\jdsub{\Gamma}{\cps{T}}{\tau_1}$~.
                \end{itemize}
                Therefore, by \rulename{T-Ret}, we have
                \begin{itemize}
                    \item $\jdty{\cpsinv{\Gamma}}{\expret{v}}{\tycomp{\{\}}{T}{\square}}$~.
                \end{itemize}
                On the other hand, by Lemma \ref{lem:cps:inv}, we have
                \begin{itemize}
                    \item $\jdsub{\Gamma, \alpha, h:\{\}, k:\cps{T} \rarr \alpha}{\cps{T} \rarr \alpha}{(y:\tau_1) \rarr \tau_2}$~.
                \end{itemize}
                Then, By inversion, we have $\tau_2 = \alpha$.
                By inversion again, we have $\tau' = \alpha$.
                Therefore, we have
                \begin{itemize}
                    \item $\jdsub{\Gamma}{\forall \alpha. \{\} \rarr (\cps{T} \rarr \alpha) \rarr \alpha}{\tau}$,
                \end{itemize}
                that is,
                \begin{itemize}
                    \item $\jdsub{\Gamma}{\cps{\tycomp{\{\}}{T}{\square}}}{\tau}$~.
                \end{itemize}
                Now we have the conclusion with $C = \tycomp{\{\}}{T}{\square}$.
            \item[Case $c = \explet{x}{c_1^{\tycomp{\Sigma}{T_1}{\square}}}{c_2^{\tycomp{\Sigma}{T_2}{\square}}}$:]
                \def\currentprefix{cps-bw:let-pure}
                We have $\cps{c} = \Lambda \alpha. \lambda h: \cps{\Sigma}. \lambda k:\cps{T_2} \rarr \alpha.
                \cps{c_1}~\alpha~h~(\lambda x:\cps{T_1}. \cps{c_2}~\alpha~h~k)$~.
                By Lemma \ref{lem:cps:inv-c}, we have
                \begin{enumrm}
                    \item\llabel{ty-body} $\jdty{\Gamma, \alpha, h:\cps{\Sigma}, k:\cps{T_2} \rarr \alpha}
                        {\cps{c_1}~\alpha~h~(\lambda x:\cps{T_1}. \cps{c_2}~\alpha~h~k)}{\tau'}$ and
                    \item\llabel{sub-t} $\jdsub{\Gamma}{\forall \alpha. \cps{\Sigma} \rarr (\cps{T_2} \rarr \alpha) \rarr \tau'}{\tau}$
                \end{enumrm}
                for some $\tau'$.
                By Lemma \ref{lem:cps:inv-c-app} with \lref{ty-body}, we have
                \begin{enumrm}[resume]
                    \item\llabel{ty-cpsc1} $\jdty{\Gamma, \alpha, h:\cps{\Sigma}, k:\cps{T_2} \rarr \alpha}
                        {\cps{c_1}}{\tau''}$,
                    \item\llabel{ty-h} $\jdty{\Gamma, \alpha, h:\cps{\Sigma}, k:\cps{T_2} \rarr \alpha}{h}{\tau_1}$, and
                    \item\llabel{ty-fun} $\jdty{\Gamma, \alpha, h:\cps{\Sigma}, k:\cps{T_2} \rarr \alpha}
                        {\lambda x:\cps{T_1}. \cps{c_2}~\alpha~h~k}{\tau_2}$
                \end{enumrm}
                for some $\tau'', \tau_1$ and $\tau_2$.
                By Lemma \ref{lem:cps:inv} with \lref{ty-h} and \lref{ty-fun} respectively, we have
                \begin{enumrm}[resume]
                    \item $\jdsub{\Gamma, \alpha, h:\cps{\Sigma}, k:\cps{T_2} \rarr \alpha}{\cps{\Sigma}}{\tau_1}$,
                    \item\llabel{ty-body-2} $\jdty{\Gamma, \alpha, h:\cps{\Sigma}, k:\cps{T_2} \rarr \alpha, x:\cps{T_1}}
                        {\cps{c_2}~\alpha~h~k}{\tau_3}$, and
                    \item $\jdsub{\Gamma, \alpha, h:\cps{\Sigma}, k:\cps{T_2} \rarr \alpha}
                        {(x:\cps{T_1}) \rarr \tau_3}{\tau_2}$
                \end{enumrm}
                for some $\tau_3$.
                Then, by the second half of Lemma \ref{lem:cps:inv-c-app}, we have
                \begin{enumrm}[resume]
                    \item\llabel{sub-t''} $\jdsub{\Gamma, \alpha, h:\cps{\Sigma}, k:\cps{T_2} \rarr \alpha}
                        {\tau''}{\forall \beta. \cps{\Sigma} \rarr ((x:\cps{T_1}) \rarr \tau_3) \rarr \tau'}$
                \end{enumrm}
                where $\beta$ is fresh.

                On the other hand, by Lemma \ref{lem:cps:inv-c-app} with \lref{ty-body-2}, we have
                \begin{enumrm}[resume]
                    \item\llabel{ty-cpsc2} $\jdty{\Gamma, \alpha, h:\cps{\Sigma}, k:\cps{T_2} \rarr \alpha, x:\cps{T_1}}
                        {\cps{c_2}}{\tau_3'}$,
                    \item\llabel{ty-h-2} $\jdty{\Gamma, \alpha, h:\cps{\Sigma}, k:\cps{T_2} \rarr \alpha, x:\cps{T_1}}
                        {h}{\tau_4}$, and
                    \item\llabel{ty-k} $\jdty{\Gamma, \alpha, h:\cps{\Sigma}, k:\cps{T_2} \rarr \alpha, x:\cps{T_1}}
                        {k}{\tau_5}$
                \end{enumrm}
                for some $\tau_3', \tau_4$ and $\tau_5$.
                By Lemma \ref{lem:cps:inv} with \lref{ty-h-2} and \lref{ty-k} respectively, we have
                \begin{itemize}
                    \item $\jdsub{\Gamma, \alpha, h:\cps{\Sigma}, k:\cps{T_2} \rarr \alpha, x:\cps{T_1}}
                        {\cps{\Sigma}}{\tau_4}$ and
                    \item $\jdsub{\Gamma, \alpha, h:\cps{\Sigma}, k:\cps{T_2} \rarr \alpha, x:\cps{T_1}}
                        {\cps{T_2} \rarr \alpha}{\tau_5}$~.
                \end{itemize}
                Then, by the second half of Lemma \ref{lem:cps:inv-c-app}, we have
                \begin{enumrm}[resume]
                    \item\llabel{sub-t3'} $\jdsub{\Gamma, \alpha, h:\cps{\Sigma}, k:\cps{T_2} \rarr \alpha, x:\cps{T_1}}
                        {\tau_3'}{\forall \gamma. \cps{\Sigma} \rarr (\cps{T_2} \rarr \alpha) \rarr \tau_3}$
                \end{enumrm}
                where $\gamma$ is fresh.

                By Lemma \ref{lem:cps:rm-nonrfn}
                with \lref{ty-cpsc1}, \lref{sub-t''}, \lref{ty-cpsc2} and \lref{sub-t3'}, we have
                \begin{enumrm}[resume]
                    \item\llabel{ty-cpsc1-2} $\jdty{\Gamma, \alpha}{\cps{c_1}}{\tau''}$,
                    \item\llabel{sub-t''-2} $\jdsub{\Gamma, \alpha}
                        {\tau''}{\forall \beta. \cps{\Sigma} \rarr ((x:\cps{T_1}) \rarr \tau_3) \rarr \tau'}$
                    \item\llabel{ty-cpsc2-2} $\jdty{\Gamma, \alpha, x:\cps{T_1}}{\cps{c_2}}{\tau_3'}$
                    \item\llabel{sub-t3'-2} $\jdsub{\Gamma, \alpha, x:\cps{T_1}}
                        {\tau_3'}{\forall \gamma. \cps{\Sigma} \rarr (\cps{T_2} \rarr \alpha) \rarr \tau_3}$~.
                \end{enumrm}
                Then, by the IHs of \lref{ty-cpsc1-2} and \lref{ty-cpsc2-2} respectively, we have
                \begin{enumrm}[resume]
                    \item\llabel{ty-c1} $\jdty{\cpsinv{\Gamma}}{c_1}{C_1}$,
                    \item\llabel{sub-cpsC1} $\jdsub{\Gamma, \alpha}{\cps{C_1}}{\tau''}$,
                    \item\llabel{ty-c2} $\jdty{\cpsinv{\Gamma}, x:T_1}{c_2}{C_2}$, and
                    \item\llabel{sub-cpsC2} $\jdsub{\Gamma, \alpha, x:\cps{T_1}}{\cps{C_2}}{\tau_3'}$
                \end{enumrm}
                for some $C_1$ and $C_2$.
                By Lemma \ref{lem:cps:trans} with ``\lref{sub-t''-2} and \lref{sub-cpsC1}''
                and ``\lref{sub-t3'-2} and \lref{sub-cpsC2}'' respectively, we have
                \begin{enumrm}[resume]
                    \item\llabel{sub-cpsC1-2} $\jdsub{\Gamma, \alpha}
                        {\cps{C_1}}{\forall \beta. \cps{\Sigma} \rarr ((x:\cps{T_1}) \rarr \tau_3) \rarr \tau'}$ and
                    \item\llabel{sub-cpsC2-2} $\jdsub{\Gamma, \alpha, x:\cps{T_1}}
                        {\cps{C_2}}{\forall \gamma. \cps{\Sigma} \rarr (\cps{T_2} \rarr \alpha) \rarr \tau_3}$~.
                \end{enumrm}
                By Lemma \ref{lem:cps:sub-pure} with \lref{sub-cpsC2-2}, we have
                \begin{itemize}
                    \item $C_1 = \tycomp{\Sigma_{11}}{T_{11}}{\square}$ and
                    \item $\tau_3 = \alpha$
                \end{itemize}
                for some $\Sigma_{11}$ and $T_{11}$.
                Then, by Lemma \ref{lem:cps:sub-pure} again with \lref{sub-cpsC1-2}, we have
                \begin{itemize}
                    \item $C_2 = \tycomp{\Sigma_{22}}{T_{22}}{\square}$ and
                    \item $\tau' = \alpha$
                \end{itemize}
                for some $\Sigma_{22}$ and $T_{22}$.
                By inversion of \lref{sub-cpsC1-2}, we have
                \begin{itemize}
                    \item $\jdsub{\Gamma, \alpha, \beta}{\cps{\Sigma}}{\cps{\Sigma_{11}}}$ and
                    \item $\jdsub{\Gamma, \alpha, h:\cps{\Sigma}, \beta}{\cps{T_{11}}}{\cps{T_1}}$~.
                \end{itemize}
                By Lemma \ref{lem:cps:rm-nonrfn}, \ref{lem:cps:rm-unused-tvar} and \ref{lem:cps:cpsinv-sub}, we have
                \begin{itemize}
                    \item $\jdsub{\cpsinv{\Gamma}}{\Sigma}{\Sigma_{11}}$ and
                    \item $\jdsub{\cpsinv{\Gamma}}{T_{11}}{T_1}$~.
                \end{itemize}
                Then, by subsumption on \lref{ty-c1}, we have
                \begin{enumrm}[resume]
                    \item\llabel{ty-c1-2} $\jdty{\cpsinv{\Gamma}}{c_1}{\tycomp{\Sigma}{T_1}{\square}}$~.
                \end{enumrm}
                In the same way, from \lref{sub-cpsC2-2}, we have
                \begin{enumrm}[resume]
                    \item\llabel{ty-c2-2} $\jdty{\cpsinv{\Gamma}, x:T_1}{c_2}{\tycomp{\Sigma}{T_2}{\square}}$~.
                \end{enumrm}
                Therefore, by \rulename{T-LetP} with \lref{ty-c1-2} and \lref{ty-c2-2}, we have
                \[
                    \jdty{\cpsinv{\Gamma}}{\explet{x}{c_1}{c_2}}{\tycomp{\Sigma}{T_2}{\square}}~.
                \]
                Also, since $\tau' = \alpha$, \lref{sub-t} implies
                \[
                    \jdsub{\Gamma}{\cps{\tycomp{\Sigma}{T_2}{\square}}}{\tau}~.
                \]
                Now we have the conclusion with $C = \tycomp{\Sigma}{T_2}{\square}$.
            \item[Case $c = \explet{x}{c_1^{\tycomp{\Sigma}{T_1}{\tyctl{x}{C_1}{C_2}}}}{c_2^{\tycomp{\Sigma}{T_2}{\tyctl{z}{C_0}{C_1}}}}$:]
                \def\currentprefix{cps-bw:let}
                We have $\cps{c} = \Lambda \alpha. \lambda h: \cps{\Sigma}. \lambda k:(z:\cps{T_2}) \rarr \cps{C_0}.
                    \cps{c_1}~\cps{C_2}~h~(\lambda x:\cps{T_1}. \cps{c_2}~\cps{C_1}~h~k)$~.
                In the similar way to the previous case, we have
                \begin{enumrm}
                    \item\llabel{sub-t} $\jdsub{\Gamma}
                        {\forall \alpha. \cps{\Sigma} \rarr ((z:\cps{T_2}) \rarr \cps{C_0}) \rarr \tau'}{\tau}$,
                    \item\llabel{ty-c1} $\jdty{\cpsinv{\Gamma}}{c_1}{C_1}$,
                    \item\llabel{sub-cpsC1} $\jdsub{\Gamma, \alpha}{\cps{C_1}}
                        {\forall \beta. \cps{\Sigma} \rarr ((x:\cps{T_1}) \rarr \tau_3) \rarr \tau'}$,
                    \item\llabel{ty-c2} $\jdty{\cpsinv{\Gamma}, x:T_1}{c_2}{C_2}$, and
                    \item\llabel{sub-cpsC2} $\jdsub{\Gamma, \alpha, x:\cps{T_1}}{\cps{C_2}}
                        {\forall \gamma. \cps{\Sigma} \rarr ((z:\cps{T_2}) \rarr \cps{C_0}) \rarr \tau_3}$
                \end{enumrm}
                for some $\tau', \tau_3, C_1$ and $C_2$.
                By Lemma \ref{lem:assume-atm}, we can assume that
                \begin{itemize}
                    \item $C_1 = \tycomp{\Sigma_1}{T_{10}}{\tyctl{x_1}{C_{11}}{C_{12}}}$ and
                    \item $C_2 = \tycomp{\Sigma_2}{T_{20}}{\tyctl{x_2}{C_{21}}{C_{22}}}$
                \end{itemize}
                for some $\Sigma_1, T_{10}, x_1, C_{11}, C_{12}, \Sigma_2, T_{20}, x_2, C_{21}$ and $C_{22}$.
                Then, by inversion of \lref{sub-cpsC1}, we have
                \begin{enumrm}[resume]
                    \item $x_1 = x$
                    \item\llabel{sub-cpssig} $\jdsub{\Gamma, \alpha, \beta}{\cps{\Sigma}}{\cps{\Sigma_1}}$,
                    \item\llabel{sub-cpsT10} $\jdsub{\Gamma, \alpha, \beta, h:\cps{\Sigma}}
                        {\cps{T_{10}}}{\cps{T_1}}$,
                    \item\llabel{sub-t3} $\jdsub{\Gamma, \alpha, \beta, h:\cps{\Sigma}, x:\cps{T_{10}}}
                        {\tau_3}{\cps{C_{11}}}$, and
                    \item\llabel{sub-cpsC12} $\jdsub{\Gamma, \alpha, \beta, h:\cps{\Sigma}, k:(x:\cps{T_1}) \rarr \tau_3}
                        {\cps{C_{12}}}{\tau'}$~.
                \end{enumrm}
                By Lemma \ref{lem:cps:rm-nonrfn}, \ref{lem:cps:rm-unused-tvar} and \ref{lem:cps:cpsinv-sub}
                with \lref{sub-cpssig} and \lref{sub-cpsT10} respectively, we have
                \begin{enumrm}[resume]
                    \item\llabel{sub-sig} $\jdsub{\cpsinv{\Gamma}}{\Sigma}{\Sigma_1}$ and
                    \item\llabel{sub-T10} $\jdsub{\cpsinv{\Gamma}}{T_{10}}{T_1}$~.
                \end{enumrm}
                By subsumption on \lref{ty-c1} with \lref{sub-sig}, we have
                \begin{enumrm}[resume]
                    \item\llabel{ty-c1-2} $\jdty{\cpsinv{\Gamma}}{c_1}{\tycomp{\Sigma}{T_{10}}{\tyctl{x_1}{C_{11}}{C_{12}}}}$~.
                \end{enumrm}
                On the other hand, by inversion of \lref{sub-cpsC2}, we have
                \begin{itemize}
                    \item $x_2 = z$
                    \item $\jdsub{\Gamma, \alpha, x: \cps{T_1}, \gamma}
                        {\cps{\Sigma}}{\cps{\Sigma_2}}$,
                    \item $\jdsub{\Gamma, \alpha, x: \cps{T_1}, \gamma, h:\cps{\Sigma}}
                        {\cps{T_{20}}}{\cps{T_2}}$,
                    \item $\jdsub{\Gamma, \alpha, x: \cps{T_1}, \gamma, h:\cps{\Sigma}, z:\cps{T_{20}}}
                        {\cps{C_0}}{\cps{C_{21}}}$, and
                    \item $\jdsub{\Gamma, \alpha, x: \cps{T_1}, \gamma, h:\cps{\Sigma}, k:(z:\cps{T_2}) \rarr \cps{C_0}}
                        {\cps{C_{22}}}{\tau_3}$~.
                \end{itemize}
                By Lemma \ref{lem:cps:narrow} with \lref{sub-cpsT10}, we have
                \begin{enumrm}[resume]
                    \item\llabel{sub-cpssig-2} $\jdsub{\Gamma, \alpha, x: \cps{T_{10}}, \gamma}
                        {\cps{\Sigma}}{\cps{\Sigma_2}}$,
                    \item\llabel{sub-cpsT20} $\jdsub{\Gamma, \alpha, x: \cps{T_{10}}, \gamma, h:\cps{\Sigma}}
                        {\cps{T_{20}}}{\cps{T_2}}$,
                    \item\llabel{sub-cpsC0} $\jdsub{\Gamma, \alpha, x: \cps{T_{10}}, \gamma, h:\cps{\Sigma}, z:\cps{T_{20}}}
                        {\cps{C_0}}{\cps{C_{21}}}$, and
                    \item\llabel{sub-cpsC22} $\jdsub{\Gamma, \alpha, x: \cps{T_{10}}, \gamma, h:\cps{\Sigma}, k:(z:\cps{T_2}) \rarr \cps{C_0}}
                        {\cps{C_{22}}}{\tau_3}$~.
                \end{enumrm}
                By Lemma \ref{lem:cps:rm-nonrfn}, \ref{lem:cps:rm-unused-tvar} and \ref{lem:cps:trans}
                with \lref{sub-t3} and \lref{sub-cpsC22}, we have
                \begin{enumrm}[resume]
                    \item\llabel{sub-cpsC22-2} $\jdsub{\Gamma, \alpha, x: \cps{T_{10}}}{\cps{C_{22}}}{\cps{C_{11}}}$~.
                \end{enumrm}
                By Lemma \ref{lem:cps:rm-nonrfn}, \ref{lem:cps:rm-unused-tvar} and \ref{lem:cps:cpsinv-sub}
                with \lref{sub-cpssig-2}, \lref{sub-cpsT20}, \lref{sub-cpsC0} and \lref{sub-cpsC22-2},
                we have
                \begin{itemize}
                    \item $\jdsub{\cpsinv{\Gamma}, x: T_{10}}{\Sigma}{\Sigma_2}$,
                    \item $\jdsub{\cpsinv{\Gamma}, x: T_{10}}{T_{20}}{T_2}$,
                    \item $\jdsub{\cpsinv{\Gamma}, x: T_{10}, z:T_{20}}{C_0}{C_{21}}$, and
                    \item $\jdsub{\cpsinv{\Gamma}, x: T_{10}}{C_{22}}{C_{11}}$~.
                \end{itemize}
                Then, by Lemma \ref{lem:cps:narrow} and subsumption on \lref{ty-c2}, we have
                \begin{enumrm}[resume]
                    \item\llabel{ty-c2-2} $\jdty{\cpsinv{\Gamma}, x:T_{10}}{c_2}
                        {\tycomp{\Sigma}{T_2}{\tyctl{z}{C_0}{C_{11}}}}$~.
                \end{enumrm}
                Therefore, by \rulename{T-LetIp}, we have
                \[
                    \jdty{\cpsinv{\Gamma}}{\explet{x}{c_1}{c_2}}{\tycomp{\Sigma}{T_2}{\tyctl{z}{C_0}{C_{12}}}}~.
                \]

                Also, by Lemma \ref{lem:cps:rm-nonrfn}, \ref{lem:cps:rm-unused-tvar}, \ref{lem:cps:weaken}, and \ref{lem:cps:trans}
                with \lref{sub-t} and \lref{sub-cpsC12}, we have
                \begin{itemize}
                    \item $\jdsub{\Gamma}
                    {\forall \alpha. \cps{\Sigma} \rarr ((z:\cps{T_2}) \rarr \cps{C_0}) \rarr \cps{C_{12}}}{\tau}$,
                \end{itemize}
                that is,
                \[
                    \jdsub{\Gamma}
                    {\cps{\tycomp{\Sigma}{T_2}{\tyctl{z}{C_0}{C_{12}}}}}{\tau}~.
                \]
                Now we have the conclusion with $C = \tycomp{\Sigma}{T_2}{\tyctl{z}{C_0}{C_{12}}}$~.
            \item[Case $c = v_1~v_2$:]
                We have $\cps{c} = \cps{v_1}~\cps{v_2}$~.
                By Lemma \ref{lem:cps:inv}, we have
                \begin{itemize}
                    \item $\jdty{\Gamma}{\cps{v_1}}{(x: \tau_1) \rarr \tau_2}$,
                    \item $\jdty{\Gamma}{\cps{v_2}}{\tau_1}$, and
                    \item $\jdsub{\Gamma}{\tau_2[\cps{v_2}/x]}{\tau}$
                \end{itemize}
                for some $x, \tau_1$ and $\tau_2$.
                By the IHs, we have
                \begin{itemize}
                    \item $\jdty{\cpsinv{\Gamma}}{v_1}{T_1}$,
                    \item $\jdty{\cpsinv{\Gamma}}{v_2}{T_2}$,
                    \item $\jdsub{\Gamma}{\cps{T_1}}{(x: \tau_1) \rarr \tau_2}$, and
                    \item $\jdsub{\Gamma}{\cps{T_2}}{\tau_1}$
                \end{itemize}
                for some $T_1$ and $T_2$.
                By inversion, we have
                \begin{itemize}
                    \item $T_1 = (x: T_{11}) \rarr C_{12}$,
                    \item $\jdsub{\Gamma}{\tau_1}{\cps{T_{11}}}$, and
                    \item $\jdsub{\Gamma, x: \tau_1}{\cps{C_{12}}}{\tau_2}$
                \end{itemize}
                for some $T_{11}$ and $C_{12}$.
                By Lemma \ref{lem:cps:trans}, we have $\jdsub{\Gamma}{\cps{T_2}}{\cps{T_{11}}}$~.
                Then, by Lemma \ref{lem:cps:cpsinv-sub},
                we have $\jdsub{\cpsinv{\Gamma}}{T_2}{T_{11}}$,
                and hence by \rulename{T-VSub} we have $\jdty{\cpsinv{\Gamma}}{v_2}{T_{11}}$~.
                Therefore, by \rulename{T-App},
                we have $\jdty{\cpsinv{\Gamma}}{v_1~v2}{C_{12}[v_2/x]}$~.
                
                On the other hand, by Lemma \ref{lem:cps:subst},
                we have $\jdsub{\Gamma}{\cps{C_{12}}[\cps{v_2}/x]}{\tau_2[\cps{v_2}/x]}$~.
                Then, by Lemma \ref{lem:cps:trans},
                we have $\jdsub{\Gamma}{\cps{C_{12}}[\cps{v_2}/x]}{\tau}$~.

                Now we have the conclusion with $C = C_{12}[v_2/x]$~.
            \item[Case $c = (\expif{v}{c_1}{c_2})^{C'}$:]
                We have $\cps{c} = (\expif{\cps{v}}{\cps{c_1}}{\cps{c_2}} : \cps{C'})$~.
                By Lemma \ref{lem:cps:inv}, we have
                \begin{itemize}
                    \item $\jdty{\Gamma}{\expif{\cps{v}}{\cps{c_1}}{\cps{c_2}}}{\cps{C'}}$ and
                    \item $\jdsub{\Gamma}{\cps{C'}}{\tau}$~.
                \end{itemize}
                By Lemma \ref{lem:cps:inv} again, we have
                \begin{itemize}
                    \item $\jdty{\Gamma}{\cps{v}}{\tyrfn{z}{\tybool}{\phi}}$,
                    \item $\jdty{\Gamma, \cps{v} = \exptrue}{\cps{c_1}}{\tau'}$,
                    \item $\jdty{\Gamma, \cps{v} = \expfalse}{\cps{c_2}}{\tau'}$, and
                    \item $\jdsub{\Gamma}{\tau'}{\cps{C'}}$
                \end{itemize}
                for some $z, \phi$ and $\tau'$.
                By the IHs, we have
                \begin{itemize}
                    \item $\jdty{\cpsinv{\Gamma}}{v}{\tyrfn{z}{\tybool}{\phi}}$,
                    \item $\jdty{\cpsinv{\Gamma}, v = \exptrue}{c_1}{C_1}$,
                    \item $\jdty{\cpsinv{\Gamma}, v = \expfalse}{c_2}{C_2}$,
                    \item $\jdsub{\Gamma, \cps{v} = \exptrue}{\cps{C_1}}{\tau'}$, and
                    \item $\jdsub{\Gamma, \cps{v} = \expfalse}{\cps{C_2}}{\tau'}$
                \end{itemize}
                for some $C_1$ and $C_2$.
                (Note that since $v$ is of a refinement type, it holds that $\cps{v} = v$.)
                By Lemma \ref{lem:cps:weaken} and \ref{lem:cps:trans}, we have
                \begin{itemize}
                    \item $\jdsub{\Gamma, \cps{v} = \exptrue}{\cps{C_1}}{\cps{C'}}$ and
                    \item $\jdsub{\Gamma, \cps{v} = \expfalse}{\cps{C_2}}{\cps{C'}}$~.
                \end{itemize}
                By Lemma \ref{lem:cps:cpsinv-sub}, we have
                \begin{itemize}
                    \item $\jdsub{\cpsinv{\Gamma}, v = \exptrue}{C_1}{C'}$ and
                    \item $\jdsub{\cpsinv{\Gamma}, v = \expfalse}{C_2}{C'}$~.
                \end{itemize}
                Then, by \rulename{T-CSub}, we have
                \begin{itemize}
                    \item $\jdty{\cpsinv{\Gamma}, v = \exptrue}{c_1}{C'}$ and
                    \item $\jdty{\cpsinv{\Gamma}, v = \expfalse}{c_2}{C'}$~.
                \end{itemize}
                Therefore by \rulename{T-If}, we have
                $\jdty{\cpsinv{\Gamma}}{\expif{v}{c_1}{c_2}}{C'}$~.
                Now we have the conclusion with $C = C'$~.
            \item[Case $c = (\op^{\rep{\mathit{A}}}~v)^{\tycomp{\Sigma}{T}{\tyctl{y}{C_1}{C_2}}}$:]
                \def\currentprefix{cpsbw:op}
                We have $\cps{c} = \Lambda \alpha. \lambda h:\cps{\Sigma}. \lambda k:(y: \cps{T}) \rarr \cps{C_1}.
                    h\#\op~\rep{A}~\cps{v}~(\lambda y':\cps{T}. k~y')$~.
                By Lemma \ref{lem:cps:inv-c}, we have
                \begin{enumrm}
                    \item\llabel{ty-body} $\jdty{\Gamma, \alpha, h:\cps{\Sigma}, k:(y: \cps{T}) \rarr \cps{C_1}}
                        {h\#\op~\rep{A}~\cps{v}~(\lambda y':\cps{T}. k~y')}{\tau'}$ and
                    \item\llabel{sub-t} $\jdsub{\Gamma}{\forall \alpha. \cps{\Sigma} \rarr ((y: \cps{T}) \rarr \cps{C_1}) \rarr \tau'}{\tau}$
                \end{enumrm}
                for some $\tau'$.
                (Below, we write $\Gamma_{\alpha,h,k}$ for $\Gamma, \alpha, h:\cps{\Sigma}, k:(y: \cps{T}) \rarr \cps{C_1}$~.)

                By Lemma \ref{lem:cps:inv} with \lref{ty-body}, we have
                \begin{enumrm}[resume]
                    \item\llabel{ty-fun} $\jdty{\Gamma_{\alpha,h,k}}{\lambda y':\cps{T}. k~y'}{\tau_1}$,
                    \item\llabel{ty-cpsv} $\jdty{\Gamma_{\alpha,h,k}}{\cps{v}}{\tau_3}$,
                    \item\llabel{wf-A} $\rep{\jdty{\Gamma_{\alpha,h,k}}{A}{\rep{B}}}$,
                    \item\llabel{sub-cpssig} $\jdsub{\Gamma_{\alpha,h,k}}{\cps{\Sigma}}{\{\ldots, \forall \rep{X:\rep{B}}. \tau_5, \ldots\}}$,
                    \item\llabel{sub-t5} $\jdsub{\Gamma_{\alpha,h,k}}{\tau_5[\rep{A/X}]}{(x: \tau_3) \rarr \tau_4}$,
                    \item\llabel{sub-t4} $\jdsub{\Gamma_{\alpha,h,k}}{\tau_4[\cps{v}/x]}{\tau_1 \rarr \tau_2}$, and
                    \item\llabel{sub-t2} $\jdsub{\Gamma_{\alpha,h,k}}{\tau_2}{\tau'}$~.
                \end{enumrm}
                By Assumption \ref{asm:cps:formula} and \ref{asm:cps:formula-cpsinv} with \lref{wf-A}, we have
                \begin{itemize}
                    \item $\rep{\jdty{\cpsinv{\Gamma}}{A}{\rep{B}}}$~.
                \end{itemize}

                By inversion of \lref{sub-cpssig}, we have
                \begin{itemize}
                    \item $\Sigma = \{\ldots, \op: \forall \rep{X:\rep{B}}.
                        (x_{\op}: T_{\op 1}) \rarr ((y_{\op}: T_{\op 2}) \rarr C_{\op 1}) \rarr C_{\op 2}, \ldots\}$ and
                    \item $\jdsub{\Gamma_{\alpha,h,k}, \rep{X:\rep{B}}}
                        {(x_{\op}: \cps{T_{\op 1}}) \rarr ((y_{\op}: \cps{T_{\op 2}}) \rarr \cps{C_{\op 1}}) \rarr \cps{C_{\op 2}}}{\tau_5}$~.
                \end{itemize}
                By repeatedly inverting this subtyping judgment
                with applying Lemma \ref{lem:cps:subst-pred} with \lref{wf-A},
                Lemma \ref{lem:cps:subst} with \lref{ty-cpsv},
                and Lemma \ref{lem:cps:trans} with \lref{sub-t5}, \lref{sub-t4} and \lref{sub-t2},
                we have
                \begin{enumrm}[resume]
                    \item $x = x_{\op}$,
                    \item\llabel{sub-t3} $\jdsub{\Gamma_{\alpha,h,k}}{\tau_3}{\cps{T_{\op 1}}[\rep{A/X}]}$,
                    \item\llabel{sub-t1} $\jdsub{\Gamma_{\alpha,h,k}}
                        {\tau_1}{(y_{\op}:\cps{T_{\op 2}}[\rep{A/X}][\cps{v}/x]) \rarr \cps{C_{\op 1}}[\rep{A/X}][\cps{v}/x]}$, and
                    \item\llabel{sub-cpsCop2} $\jdsub{\Gamma_{\alpha,h,k}}
                        {\cps{C_{\op 2}}[\rep{A/X}][\cps{v}/x]}{\tau'}$~.
                \end{enumrm}

                By Lemma \ref{lem:cps:rm-nonrfn} with \lref{ty-cpsv}, we have
                \begin{itemize}
                    \item $\jdty{\Gamma, \alpha}{\cps{v}}{\tau_3}$~.
                \end{itemize}
                Then, by the IH, we have
                \begin{enumrm}[resume]
                    \item\llabel{ty-v} $\jdty{\cpsinv{\Gamma}}{v}{T_v}$ and
                    \item\llabel{sub-cpsTv} $\jdsub{\Gamma, \alpha}{\cps{T_v}}{\tau_3}$
                \end{enumrm}
                for some $T_v$.
                By Lemma \ref{lem:cps:trans} with \lref{sub-t3} and \lref{sub-cpsTv}
                (using Lemma \ref{lem:cps:rm-nonrfn}), we have
                \begin{itemize}
                    \item $\jdsub{\Gamma, \alpha}{\cps{T_v}}{\cps{T_{\op 1}}[\rep{A/X}]}$~.
                \end{itemize}
                Then, by Lemma \ref{lem:cps:cpsinv-sub}, we have
                \begin{itemize}
                    \item $\jdsub{\cpsinv{\Gamma}}{T_v}{T_{\op 1}[\rep{A/X}]}$
                \end{itemize}
                and hence, by \rulename{T-VSub} with \lref{ty-v}, we have
                \begin{itemize}
                    \item $\jdty{\cpsinv{\Gamma}}{v}{T_{\op 1}[\rep{A/X}]}$~.
                \end{itemize}

                Also, by Lemma~\ref{lem:cps:wfg} and inversion, we have
                $\jdwf{\Gamma, \alpha}{\cps{\Sigma}}$.
                Then by Lemma~\ref{lem:cps:presv-b-wf}, we have
                $\jdwf{\cpsinv{\Gamma}}{\Sigma}$.

                Therefore, by \rulename{T-Op}, we have
                \[
                    \jdty{\cpsinv{\Gamma}}{\op~v}{\tycomp{\Sigma}{T_{\op 2}[\rep{A/X}][v/x]}{\tyctl{y_{\op}}{C_{\op 1}[\rep{A/X}][v/x]}{C_{\op 2}[\rep{A/X}][v/x]}}}~.
                \]

                On the other hand, by Lemma~\ref{lem:cps:inv} with \lref{ty-fun}, we have
                \begin{enumrm}[resume]
                    \item\llabel{sub-cpsT-t6} $\jdsub{\Gamma_{\alpha,h,k}}{(y':\cps{T}) \rarr \tau_6}{\tau_1}$,
                    \item\llabel{sub-t8} $\jdsub{\Gamma_{\alpha,h,k}, y':\cps{T}}{\tau_8[y'/y_0]}{\tau_6}$,
                    \item\llabel{sub-cpsT-cpsC1} $\jdsub{\Gamma_{\alpha,h,k}, y':\cps{T}}{(y:\cps{T}) \rarr \cps{C_1}}{(y_0:\tau_7) \rarr \tau_8}$, and
                    \item\llabel{ty-y'} $\jdty{\Gamma_{\alpha,h,k}, y':\cps{T}}{y'}{\tau_7}$~.
                \end{enumrm}
                By inversion of \lref{sub-cpsT-cpsC1}, we have
                \begin{itemize}
                    \item $y = y_0$ and
                    \item $\jdsub{\Gamma_{\alpha,h,k}, y':\cps{T}, y:\tau_7}{\cps{C_1}}{\tau_8}$~.
                \end{itemize}
                Then, by Lemma~\ref{lem:cps:subst} with \lref{ty-y'}, we have
                \begin{itemize}
                    \item $\jdsub{\Gamma_{\alpha,h,k}, y':\cps{T}}{\cps{C_1}[y'/y]}{\tau_8[y'/y]}$~.
                \end{itemize}
                Then, by Lemma~\ref{lem:cps:trans} with \lref{sub-t8}, we have
                \begin{itemize}
                    \item $\jdsub{\Gamma_{\alpha,h,k}, y':\cps{T}}{\cps{C_1}[y'/y]}{\tau_6}$
                \end{itemize}
                (Note that $y = y_0$).
                Then by \rulename{Sc-Fun}, we have
                \begin{itemize}
                    \item $\jdsub{\Gamma_{\alpha,h,k}}{(y':\cps{T}) \rarr \cps{C_1}[y'/y]}{(y':\cps{T}) \rarr \tau_6}$
                \end{itemize}
                and by $\alpha$-renaming we have
                \begin{itemize}
                    \item $\jdsub{\Gamma_{\alpha,h,k}}{(y:\cps{T}) \rarr \cps{C_1}}{(y':\cps{T}) \rarr \tau_6}$~.
                \end{itemize}
                Then, by Lemma~\ref{lem:cps:trans} with \lref{sub-cpsT-t6} and \lref{sub-t1}, we have
                \begin{enumrm}[resume]
                    \item\llabel{sub-cpsT-cpsC1-2} $\jdsub{\Gamma_{\alpha,h,k}}{(y:\cps{T}) \rarr \cps{C_1}}{(y_{\op}:\cps{T_{\op 2}}[\rep{A/X}][\cps{v}/x]) \rarr \cps{C_{\op 1}}[\rep{A/X}][\cps{v}/x]}$~.
                \end{enumrm}

                Therefore, by some subtyping rules with \lref{sub-cpsT-cpsC1-2} and \lref{sub-cpsCop2}, we have
                \begin{itemize}
                    \item $\jdsub{\Gamma}
                        {\forall \alpha. \cps{\Sigma} \rarr ((y_{\op}:\cps{T_{\op 2}}[\rep{A/X}][\cps{v}/x]) \rarr \cps{C_{\op 1}}[\rep{A/X}][\cps{v}/x]) \rarr \cps{C_{\op 2}}[\rep{A/X}][\cps{v}/x]}
                        {\forall \alpha. \cps{\Sigma} \rarr ((y: \cps{T}) \rarr \cps{C_1}) \rarr \tau'}$~.
                \end{itemize}
                Then by Lemma \ref{lem:cps:trans} with \lref{sub-t}, we have
                \begin{itemize}
                    \item $\jdsub{\Gamma}
                        {\forall \alpha. \cps{\Sigma} \rarr ((y_{\op}:\cps{T_{\op 2}}[\rep{A/X}][\cps{v}/x]) \rarr \cps{C_{\op 1}}[\rep{A/X}][\cps{v}/x]) \rarr \cps{C_{\op 2}}[\rep{A/X}][\cps{v}/x]}
                        {\tau}$~,
                \end{itemize}
                that is,
                \[
                    \jdsub{\Gamma}{\cps{\tycomp{\Sigma}{T_{\op 2}[\rep{A/X}][v/x]}{\tyctl{y_{\op}}{C_{\op 1}[\rep{A/X}][v/x]}{C_{\op 2}[\rep{A/X}][v/x]}}}}{\tau}~.
                \]
                Now we have the conclusion with $C = \tycomp{\Sigma}{T_{\op 2}[\rep{A/X}][v/x]}{\tyctl{y_{\op}}{C_{\op 1}[\rep{A/X}][v/x]}{C_{\op 2}[\rep{A/X}][v/x]}}$~.
            \item[Case $c = (\expwith{h}{c})^C$:]
                \def\currentprefix{cpsbw:hndl}
                We have $\cps{c} = \cps{c}~\cps{C}~\cps{h^{\mathit{ops}}}~\cps{h^{\mathit{ret}}}$
                where
                \[
                    \left\{ \begin{aligned}
                        h &= \{ \expret{x_r^{T_r}} \mapsto c_r, \repi{\op_i^{\rep{X_i: \rep{B_i}}}(x_i^{T_{i1}}, k_i^{(y_i:T_{i2}) \rarr C_{i1}}) \mapsto c_i} \} \\
                        \cps{h^{\mathit{ret}}} &= \lambda x_r:\cps{T_r}. \cps{c_r} \\
                        \cps{h^{\mathit{ops}}} &=
                            \{ \repi{\op_i = \Lambda \rep{X_i: \rep{B_i}}. \lambda x_i:\cps{T_{i1}}. \lambda k_i:(y_i:\cps{T_{i2}}) \rarr \cps{C_{i1}}. \cps{c_i}} \}
                    \end{aligned} \right.
                \]
                By Lemma \ref{lem:cps:inv-c-app}, we have
                \begin{enumrm}
                    \item\llabel{ty-cpsc} $\jdty{\Gamma}{\cps{c}}{\tau'}$,
                    \item\llabel{ty-cpshops} $\jdty{\Gamma}{\cps{h^{\mathit{ops}}}}{\tau_1}$, and
                    \item\llabel{ty-cpshret} $\jdty{\Gamma}{\cps{h^{\mathit{ret}}}}{\tau_2}$
                \end{enumrm}
                for some $\tau', \tau_1$ and $\tau_2$.

                By Lemma \ref{lem:cps:inv} with \lref{ty-cpshops} and \lref{ty-cpshret}, we have
                \begin{enumrm}[resume]
                    \item\llabel{ty-cpsopc} $\bigrepi{\jdty{\Gamma}
                        {\Lambda \rep{X_i: \rep{B_i}}. \lambda x_i:\cps{T_{i1}}. \lambda k_i:(y_i:\cps{T_{i2}}) \rarr \cps{C_{i1}}. \cps{c_i}}{\tau_i}}$,
                    \item\llabel{sub-t1} $\jdsub{\Gamma}{\{\repi{\op_i: \tau_i}\}}{\tau_1}$,
                    \item\llabel{ty-cpscr} $\jdty{\Gamma, x_r:\cps{T_r}}{\cps{c_r}}{\tau_3}$, and
                    \item\llabel{sub-t2} $\jdsub{\Gamma}{(x_r:\cps{T_r}) \rarr \tau_3}{\tau_2}$~.
                \end{enumrm}
                Then, by the second half of Lemma \ref{lem:cps:inv-c-app}, we have
                \begin{enumrm}[resume]
                    \item\llabel{sub-t'} $\jdsub{\Gamma}{\tau'}
                        {\forall \alpha. \{\repi{\op_i: \tau_i}\} \rarr ((x_r:\cps{T_r}) \rarr \tau_3) \rarr \tau}$
                \end{enumrm}
                where $\alpha$ is fresh.

                By the IH of \lref{ty-cpscr}, we have
                \begin{enumrm}[resume]
                    \item\llabel{ty-cr} $\jdty{\cpsinv{\Gamma}, x_r:T_r}{c_r}{C_r}$ and
                    \item\llabel{sub-cpsCr} $\jdsub{\Gamma, x_r:\cps{T_r}}{\cps{C_r}}{\tau_3}$
                \end{enumrm}
                for some $C_r$.

                By repeatedly inverting \lref{ty-cpsopc} and by Lemma \ref{lem:cps:trans}, we have
                \begin{enumrm}[resume]
                    \item\llabel{ty-cpsci} $\bigrepi{\jdty{\Gamma, \rep{X_i: \rep{B_i}}, x_i:\cps{T_{i1}}, k_i:(y_i:\cps{T_{i2}}) \rarr \cps{C_{i1}}}
                        {\cps{c_i}}{\tau_i'}}$ and
                    \item\llabel{sub-ti} $\bigrepi{\jdsub{\Gamma}
                        {\forall \rep{X_i: \rep{B_i}}. (x_i:\cps{T_{i1}}) \rarr ((y_i:\cps{T_{i2}}) \rarr \cps{C_{i1}}) \rarr \tau_i'}{\tau_i}}$
                \end{enumrm}
                for some $\tau_i'$.
                By the IH of \lref{ty-cpsci}, we have
                \begin{enumrm}[resume]
                    \item\llabel{ty-ci} $\bigrepi{\jdty{\cpsinv{\Gamma}, \rep{X_i: \rep{B_i}}, x_i:T_{i1}, k_i:(y_i:T_{i2}) \rarr C_{i1}}
                    {c_i}{C_i}}$ and
                    \item\llabel{sub-Ci} $\bigrepi{\jdsub{\Gamma, \rep{X_i: \rep{B_i}}, x_i:\cps{T_{i1}}, k_i:(y_i:\cps{T_{i2}}) \rarr \cps{C_{i1}}}
                        {\cps{C_i}}{\tau_i'}}$
                \end{enumrm}
                for some $C_i$'s.
                By \rulename{Sc-Fun} and \rulename{Sc-PPoly} with \lref{sub-Ci}, we have
                \begin{itemize}
                    \item {\small $\bigrepi{\jdsub{\Gamma}
                        {\forall \rep{X_i: \rep{B_i}}. (x_i:\cps{T_{i1}}) \rarr ((y_i:\cps{T_{i2}}) \rarr \cps{C_{i1}}) \rarr \cps{C_i}}
                        {\forall \rep{X_i: \rep{B_i}}. (x_i:\cps{T_{i1}}) \rarr ((y_i:\cps{T_{i2}}) \rarr \cps{C_{i1}}) \rarr \tau_i'}}$~.
                    }
                \end{itemize}
                Then, by Lemma \ref{lem:cps:trans} with \lref{sub-ti}, we have
                \begin{enumrm}[resume]
                    \item\llabel{sub-ti-2} $\bigrepi{\jdsub{\Gamma}
                        {\forall \rep{X_i: \rep{B_i}}. (x_i:\cps{T_{i1}}) \rarr ((y_i:\cps{T_{i2}}) \rarr \cps{C_{i1}}) \rarr \cps{C_i}}
                        {\tau_i}}$~.
                \end{enumrm}
                Thus, by Lemma \ref{lem:cps:trans}
                and subtyping with \lref{sub-t'}, \lref{sub-cpsCr} and \lref{sub-ti-2}, we have
                \begin{itemize}
                    \item $\jdsub{\Gamma}{\tau'}
                    {\forall \alpha. \tau_s \rarr ((x_r:\cps{T_r}) \rarr \cps{C_r}) \rarr \tau}$
                \end{itemize}
                where $\tau_s \defeq \{\repi{\op_i: \forall \rep{X_i: \rep{B_i}}. (x_i:\cps{T_{i1}}) \rarr ((y_i:\cps{T_{i2}}) \rarr \cps{C_{i1}}) \rarr \cps{C_i}}\}$~.
                Here, we define $\Sigma$ to be
                $\{\repi{\op_i: \forall \rep{X_i: \rep{B_i}}. (x_i:T_{i1}) \rarr ((y_i:T_{i2}) \rarr C_{i1}) \rarr C_i}\}$,
                Then, it holds that $\tau_s = \cps{\Sigma}$.
                That is, we have
                \begin{enumrm}[resume]
                    \item\llabel{sub-t'-2} $\jdsub{\Gamma}{\tau'}
                    {\forall \alpha. \cps{\Sigma} \rarr ((x_r:\cps{T_r}) \rarr \cps{C_r}) \rarr \tau}$~.
                \end{enumrm}

                On the other hand, by the IH of \lref{ty-cpsc}, we have
                \begin{enumrm}[resume]
                    \item\llabel{ty-c} $\jdty{\cpsinv{\Gamma}}{c}{C_0}$ and
                    \item\llabel{sub-C0} $\jdsub{\Gamma}{\cps{C_0}}{\tau'}$
                \end{enumrm}
                for some $C_0$.
                By Lemma \ref{lem:assume-atm}, w.l.o.g., we can assume that
                $C_0 = \tycomp{\Sigma_0}{T_0}{\tyctl{x_0}{C_{01}}{C_{02}}}$~.
                Then, by Lemma \ref{lem:cps:trans} with \lref{sub-t'-2} and \lref{sub-C0}, we have
                \begin{itemize}
                    \item $\jdsub{\Gamma}{\forall \beta. \cps{\Sigma_0} \rarr ((x_0:\cps{T_0}) \rarr \cps{C_{01}}) \rarr C_{02}}
                    {\forall \alpha. \cps{\Sigma} \rarr ((x_r:\cps{T_r}) \rarr \cps{C_r}) \rarr \tau}$~.
                \end{itemize}
                Then, by inversion, we have
                \begin{itemize}
                    \item $x_0 = x_r$,
                    \item $\jdsub{\Gamma, \alpha}{\cps{\Sigma}}{\cps{\Sigma_0}}$,
                    \item $\jdsub{\Gamma, \alpha, h:\cps{\Sigma}}{\cps{T_0}}{\cps{T_r}}$,
                    \item $\jdsub{\Gamma, \alpha, h:\cps{\Sigma}, x_r:\cps{T_0}}{\cps{C_r}}{\cps{C_{01}}}$,
                \end{itemize}
                and
                \begin{enumrm}[resume]
                    \item\llabel{sub-t} $\jdsub{\Gamma, \alpha, h:\cps{\Sigma}, k:(x_r:\cps{T_0}) \rarr \cps{C_{01}}}{\cps{C_{02}}}{\tau}$~.
                \end{enumrm}
                By Lemma \ref{lem:cps:rm-nonrfn}, we have
                \begin{itemize}
                    \item $\jdsub{\Gamma, \alpha}{\cps{\Sigma}}{\cps{\Sigma_0}}$,
                    \item $\jdsub{\Gamma, \alpha}{\cps{T_0}}{\cps{T_r}}$, and
                    \item $\jdsub{\Gamma, \alpha, x_r:\cps{T_0}}{\cps{C_r}}{\cps{C_{01}}}$~.
                \end{itemize}
                Then, by \ref{lem:cps:cpsinv-sub}, we have
                \begin{itemize}
                    \item $\jdsub{\cpsinv{\Gamma}}{\Sigma}{\Sigma_0}$,
                    \item $\jdsub{\cpsinv{\Gamma}}{T_0}{T_r}$, and
                    \item $\jdsub{\cpsinv{\Gamma}, x_r:T_0}{C_r}{C_{01}}$~.
                \end{itemize}
                Therefore, by subsumption on \lref{ty-c}, we have
                \begin{enumrm}[resume]
                    \item\llabel{ty-c-2} $\jdty{\cpsinv{\Gamma}}{c}{\tycomp{\Sigma}{T_r}{\tyctl{x_r}{C_r}{C_{02}}}}$~.
                \end{enumrm}
                Thus, by \rulename{T-Hndl} with \lref{ty-cr}, \lref{ty-ci} and \lref{ty-c-2}, we have
                \[
                    \jdty{\cpsinv{\Gamma}}{\expwith{h}{c}}{C_{02}}~.
                \]
                Also, by Lemma \ref{lem:cps:rm-nonrfn} and \ref{lem:cps:rm-unused-tvar} with \lref{sub-t}, we have
                \[
                    \jdsub{\Gamma}{\cps{C_{02}}}{\tau}~.
                \]
                Now we have the conclusion with $C = C_{02}$.
        \end{description}
    \end{enumit}
\end{proof}

\begin{corollary}[Backward type preservation (for closed expressions)] \quad
    \begin{itemize}
     \item If\, $\jdty{\emptyset}{\cps{v}}{\tau}$, then
           there exists some $T$ such that
           $\jdty{\emptyset}{v}{T}$ and
           $\jdsub{\emptyset}{\cps{T}}{\tau}$.
     \item If\, $\jdty{\emptyset}{\cps{c}}{\tau}$, then
           there exists some $C$ such that
           $\jdty{\emptyset}{c}{C}$ and
           $\jdsub{\emptyset}{\cps{C}}{\tau}$.
    \end{itemize}
\end{corollary}
\begin{proof}
    Immediate from Theorem \ref{thm:cps-backward}
    since $\emptyset$ is obviously cps-wellformed.
\end{proof}

\bibliographystyle{unsrtnat}
\bibliography{main}

\end{document}
