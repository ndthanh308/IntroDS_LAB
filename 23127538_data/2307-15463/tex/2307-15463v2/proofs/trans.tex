% depends on:: asm:formla lem:narrow lem:weaken
\begin{proof}
    By simultaneous induction on the structure of $T_2, C_2, \Sigma_2$ and $S_2$.
    \begin{enumit}
        \item Case analysis on $\jdsub{\Gamma}{T_1}{T_2}$.
            \begin{description}
                \item[Case \rulename{S-Rfn}:] By inversion, Assumption \ref{asm:formla} and \rulename{S-Rfn}.
                \item[Case \rulename{S-Fun}:] By inversion, the IHs, Lemma \ref{lem:narrow}, and \rulename{S-Fun}.
            \end{description}
        \item By inversion of the both derivations, we have
            \def\currentprefix{trans:opsig}
            \begin{enumrm}
                \item\llabel{eq-sig1} $\Sigma_1 = \{ \repi{\op_i: \forall \rep{X_i: \rep{B}_i}. F_{1i}},
                    \repi{\op'_i: \forall \rep{X'_i: \rep{B'}_i}. F'_{1i}},
                    \repi{\op''_i: \forall \rep{X''_i: \rep{B''}_i}. F''_{1i}} \}$,
                \item\llabel{eq-sig2} $\Sigma_2 = \{ \repi{\op_i: \forall \rep{X_i: \rep{B}_i}. F_{2i}},
                    \repi{\op'_i: \forall \rep{X'_i: \rep{B'}_i}. F'_{2i}} \}$,
                \item\llabel{eq-sig3} $\Sigma_3 = \{ \repi{\op_i: \forall \rep{X_i: \rep{B}_i}. F_{2i}} \}$,
                \item\llabel{sub-f1} $\repi{\jdsub{\Gamma, \rep{X_i: \rep{B}_i}}{F_{1i}}{F_{2i}}}$,
                \item\llabel{sub-f2} $\repi{\jdsub{\Gamma, \rep{X_i: \rep{B}_i}}{F_{2i}}{F_{3i}}}$, and
                \item\llabel{sub-f1'} $\repi{\jdsub{\Gamma, \rep{X'_i: \rep{B'}_i}}{F'_{1i}}{F'_{2i}}}$~.
            \end{enumrm}
            By the IH with \lref{sub-f1} and \lref{sub-f2}, we have
            $\repi{\jdsub{\Gamma, \rep{X_i: \rep{B}_i}}{F_{1i}}{F_{3i}}}$~.
            By \rulename{S-Sig}, we have the conclusion.
        \item By inversion, the IHs, Lemma \ref{lem:narrow}, and \rulename{S-Comp}.
        \item Case analysis on $\jdsub{\Gamma}{S_1}{S_2}$.
        \begin{description}
            \item[Case \rulename{S-Pure}:]
                Since we have $S_1 = \square = S_2$,
                we have the conclusion immediately from $\jdsub{\Gamma}{S_2}{S_3}$.
            \item[Case \rulename{S-ATM}:] We have
                \def\currentprefix{trans:atm}
                \begin{enumrm}
                    \item\llabel{eq-S1} $S_1 = \tyctl{x}{C_{11}}{C_{12}}$,
                    \item\llabel{eq-S2} $S_2 = \tyctl{x}{C_{21}}{C_{22}}$,
                    \item\llabel{sub-C21} $\jdsub{\Gamma, x:T}{C_{21}}{C_{11}}$, and
                    \item\llabel{sub-C12} $\jdsub{\Gamma}{C_{12}}{C_{22}}$
                \end{enumrm}
                for some $x, C_{11}, C_{12}, C_{21}$, and $C_{22}$.
                Since \lref{eq-S2}, the only rule applicable to $\jdsub{\Gamma}{S_2}{S_3}$ is \rulename{S-ATM}.
                Therefore, by inversion we have
                \begin{enumrm}[resume]
                    \item\llabel{eq-S3} $S_3 = \tyctl{x}{C_{31}}{C_{32}}$,
                    \item\llabel{sub-C31} $\jdsub{\Gamma, x:T}{C_{31}}{C_{21}}$, and
                    \item\llabel{sub-C22} $\jdsub{\Gamma}{C_{22}}{C_{32}}$
                \end{enumrm}
                for some $C_{31}$ and $C_{32}$.
                By the IHs, we have
                \begin{itemize}
                    \item $\jdsub{\Gamma, x:T}{C_{31}}{C_{11}}$ and
                    \item $\jdsub{\Gamma}{C_{12}}{C_{32}}$~.
                \end{itemize}
                We have the conclusion by \rulename{S-ATM}.
            \item[Case \rulename{S-Embed}:] We have
            \def\currentprefix{trans:emb}
            \begin{enumrm}
                \item\llabel{eq-S1} $S_1 = \square$,
                \item\llabel{eq-S2} $S_2 = \tyctl{x}{C_{21}}{C_{22}}$,
                \item\llabel{sub-C21} $\jdsub{\Gamma, x:T}{C_{21}}{C_{22}}$, and
                \item\llabel{in-x} $x \notin \fv(C_{22})$
            \end{enumrm}
            for some $x, C_{21}$, and $C_{22}$.
            Since \lref{eq-S2}, the only rule applicable to $\jdsub{\Gamma}{S_2}{S_3}$ is \rulename{S-ATM}.
            Therefore, by inversion we have
            \begin{enumrm}[resume]
                \item\llabel{eq-S3} $S_3 = \tyctl{x}{C_{31}}{C_{32}}$,
                \item\llabel{sub-C31} $\jdsub{\Gamma, x:T}{C_{31}}{C_{21}}$, and
                \item\llabel{sub-C22} $\jdsub{\Gamma}{C_{22}}{C_{32}}$
            \end{enumrm}
            for some $C_{31}$ and $C_{32}$.
            W.l.o.g., we can assume that $x \notin \fv(C_{32})$.
            Then, by Lemma \ref{lem:weaken} with \lref{sub-C22}, we have
            \begin{enumrm}[resume]
                \item\llabel{sub-C22-2} $\jdsub{\Gamma, x:T}{C_{22}}{C_{32}}$~.
            \end{enumrm}
            By the IHs with \lref{sub-C21}, \lref{sub-C21} and \lref{sub-C22-2},
            we have $\jdsub{\Gamma, x:T}{C_{31}}{C_{32}}$.
            Then we have the conclusion by \rulename{S-Embed}.
        \end{description}
    \end{enumit}
    
\end{proof}