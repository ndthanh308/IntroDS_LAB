% depend on:: asm:prim lem:inv lem:rm-nonrfn lem:rm-tauto lem:rm-unused lem:notin-nonrfn
%             lem:weaken lem:narrow lem:subst lem:subst-pred
%             lem:refl lem:wfg lem:wft
\begin{proof}
    \def\currentprefix{subred}
    By induction on the typing derivation.
    \begin{description}
        \item[Case \rulename{T-Ret} and \rulename{T-Op}:]
            Contradictory because there is no evaluation rule for $c$.
        \item[Case \rulename{T-App}:] We have
            \def\currentprefix{subred:app}
            \begin{enumrm}
                \item\llabel{eq-c} $c = v_1~v_2$,
                \item\llabel{eq-C} $C = C_1[v_2/x]$,
                \item\llabel{ty-v1} $\jdty{}{v_1}{(x:T_1) \rarr C_1}$, and
                \item\llabel{ty-v2} $\jdty{}{v_2}{T_1}$
            \end{enumrm}
            for some $x, v_1, v_2, T_1$ and $C_1$.
            Case analysis on the evaluation derivation.
            \begin{description}
                \item[Case \rulename{E-App}:] We have
                    \def\currentprefix{subred:app:app}
                    \begin{enumrm}[resume]
                        \item\llabel{eq-v1} $v_1 = \exprec{f}{x}{c_1}$, and
                        \item\llabel{eq-c'} $c' = c_1[v_2/x][(\exprec{f}{x}{c_1})/f]$
                    \end{enumrm}
                    for some $f, x$ and $c_1$.
                    By Lemma \ref{lem:inv} with \lref[subred:app]{ty-v1}, we have
                    \begin{enumrm}[resume]
                        \item\llabel{ty-v1-2} $\jdty{}{v_1}{(x: T_0) \rarr C_0}$,
                        \item\llabel{sub-fun} $\jdsub{}{(x: T_0) \rarr C_0}{(x: T_1) \rarr C_1}$, and
                        \item\llabel{ty-c} $\jdty{f: (x: T_0) \rarr C_0, x: T_0}{c}{C_0}$
                    \end{enumrm}
                    for some $T_0$ and $C_0$.
                    By Lemma \ref{lem:wfg} with \lref{ty-c}, inversion, and Lemma \ref{lem:rm-nonrfn},
                    we have $\jdwf{}{T_0}$.
                    Also, by inversion of \lref{sub-fun}, we have $\jdsub{}{T_1}{T_0}$.
                    Then, By \rulename{T-VSub} with \lref[subred:app]{ty-v2},
                    we have $\jdty{}{v_2}{T_0}$.
                    Using this and \lref{ty-v1-2}, we have the conclusion
                    by Lemma \ref{lem:subst} with \lref{ty-c}.
                \item[Case \rulename{E-Prim}:] We have
                    \def\currentprefix{subred:app:prim}
                    \begin{enumrm}[resume]
                        \item\llabel{eq-v1} $v_1 = p$, and
                        \item\llabel{eq-c'} $c' = \zeta(p, v_2)$
                    \end{enumrm}
                    for some $p$.
                    By Lemma \ref{lem:inv} with \lref[subred:app]{ty-v1}, we have
                    \begin{enumrm}[resume]
                        \item\llabel{ty-p} $\jdty{}{p}{\ty(p)}$, and
                        \item\llabel{sub-typ} $\jdsub{}{\ty(p)}{(x:T_1) \rarr C_1}$~.
                    \end{enumrm}
                    By inversion of \lref{sub-typ}, we have
                    \begin{enumrm}[resume]
                        \item\llabel{eq-typ} $\ty(p) = (x:T_0) \rarr C_0$,
                        \item\llabel{sub-t1} $\jdsub{}{T_1}{T_0}$, and
                        \item\llabel{sub-c0} $\jdsub{x: T_1}{C_0}{C_1}$
                    \end{enumrm}
                    for some $T_0$ and $C_0$.
                    By Lemma \ref{lem:wfg} with \lref{ty-p} and \lref{eq-typ} and inversion,
                    we have $\jdwf{}{T_0}$.
                    Then, by \rulename{T-VSub} with \lref[subred:app]{ty-v2} and \lref{sub-t1},
                    we have $\jdty{}{v_2}{T_0}$.
                    Therefore, by Assumption \ref{asm:prim} with \lref{eq-typ}, we have
                    \begin{enumrm}[resume]
                        \item\llabel{ty-z} $\jdty{}{\zeta(p, v_2)}{C_0[v_2/x]}$~.
                    \end{enumrm}
                    Also, by Lemma \ref{lem:wft} with \lref[subred:app]{ty-v1} and inversion,
                    we have
                    \begin{enumrm}[resume]
                        \item\llabel{wf-C1} $\jdwf{x: T_1}{C_1}$~.
                    \end{enumrm}
                    Using \lref[subred:app]{ty-v2},
                    by Lemma \ref{lem:subst} with \lref{sub-c0} and \lref{wf-C1} respectively,
                    we have
                    \begin{itemize}
                        \item $\jdsub{}{C_0[v_2/x]}{C_1[v_2/x]}$ and
                        \item $\jdwf{}{C_1[v_2/x]}$~.
                    \end{itemize}
                    Therefore, by \rulename{T-CSub} with \lref{ty-z}, we have the conclusion.
            \end{description}
        \item[Case \rulename{T-If}:] We have
            \def\currentprefix{subred:if}
            \begin{enumrm}
                \item\llabel{eq-c} $c = \expif{v}{c_1}{c_2}$,
                \item\llabel{ty-v} $\jdty{}{v}{\tyrfn{x}{\tybool}{\phi}}$,
                \item\llabel{ty-c1} $\jdty{v = \exptrue}{c_1}{C}$, and
                \item\llabel{ty-c2} $\jdty{v = \expfalse}{c_2}{C}$
            \end{enumrm}
            for some $x, v, c_1, c_2$, and $\phi$.
            Case analysis on the evaluation derivation.
            \begin{description}
                \item[Case \rulename{E-IfT}: ] We have
                    \begin{enumrm}[resume]
                        \item\llabel{eq-v} $v = \exptrue$, and
                        \item\llabel{eq-c'} $c' = c_1$~.
                    \end{enumrm}
                    We have the conclusion by Lemma \ref{lem:rm-tauto} with \lref{ty-c1}.
                \item[Case \rulename{E-IfF}: ] Similar.
            \end{description}
        \item[Case \rulename{T-CSub}:] By the IH and \rulename{T-CSub}.
        \item[Case \rulename{T-LetP}:] We have
            \def\currentprefix{subred:letp}
            \begin{enumrm}
                \item\llabel{eq-c} $c = \explet{x}{c_1}{c_2}$,
                \item\llabel{eq-C} $C = \tycomp{\Sigma}{T_2}{\square}$,
                \item\llabel{ty-c1} $\jdty{}{c_1}{\tycomp{\Sigma}{T_1}{\square}}$,
                \item\llabel{ty-c2} $\jdty{x: T_1}{c_2}{\tycomp{\Sigma}{T_2}{\square}}$, and
                \item\llabel{in-x} $x \notin \fv(T_2) \cup \fv(\Sigma)$
            \end{enumrm}
            for some $x, c_1, c_2, \Sigma, T_1$ and $T_2$.
            Case analysis on the evaluation derivation.
            \begin{description}
                \item[Case \rulename{E-Let}:] %We have
                    By the IH and \rulename{T-LetP}.
                    % \def\currentprefix{subred:let:let}
                    % \begin{enumrm}[resume]
                    %     \item\llabel{eq-c'} $c' = \explet{x}{c_1'}{c_2}$, and
                    %     \item\llabel{ev-c1} $c_1 \eval c_1'$
                    % \end{enumrm}
                    % for some $c_1'$.
                    % By the IH with \lref[subred:let]{ty-c1} and \lref{ev-c1}, we have
                    % $\jdty{}{c_1'}{\tycomp{\Sigma}{T_1}{S_1}}$.
                    % Then we have the conclusion by \rulename{T-LetP}.
                \item[Case \rulename{E-LetRet}:] We have
                    \def\currentprefix{subred:letp:ret}
                    \begin{enumrm}[resume]
                        \item\llabel{eq-c1} $c_1 = \expret{v}$, and
                        \item\llabel{eq-c'} $c' = c_2[v/x]$
                    \end{enumrm}
                    for some $v$.
                    By Lemma \ref{lem:inv} with \lref[subred:letp]{ty-c1}, we have
                    \begin{enumrm}[resume]
                        \item\llabel{sub-T0} $\jdsub{}{T_0}{T_1}$ and
                        \item\llabel{ty-v} $\jdty{}{v}{T_0}$
                    \end{enumrm}
                    for some $T_0$.
                    By Lemma \ref{lem:wft} with \lref[subred:letp]{ty-c1} and inversion,
                    we have $\jdwf{}{T_1}$.
                    Then, by \rulename{T-VSub} with \lref{ty-v} and \lref{sub-T0},
                    we have $\jdty{}{v}{T_1}$.
                    Therefore, by Lemma \ref{lem:subst} with \lref[subred:letp]{ty-c2},
                    we have
                    \[
                        \jdty{}{c_2[v/x]}{\tycomp{\Sigma}{T_2}{\square}}
                    \]
                    (Note that since \lref[subred:letp]{in-x},
                    it holds that $\Sigma[v/x] = \Sigma$ and $T_2[v/x] = T_2$.)
                    That is, we have the conclusion.
            \end{description}
        \item[Case \rulename{T-LetIp}:] We have
            \def\currentprefix{subred:leti}
            \begin{enumrm}
                \item\llabel{eq-c} $c = \explet{x}{c_1}{c_2}$,
                \item\llabel{eq-C} $C = \tycomp{\Sigma}{T_2}{\tyctl{z}{C_{21}}{C_{12}}}$,
                \item\llabel{ty-c1} $\jdty{}{c_1}{\tycomp{\Sigma}{T_1}{\tyctl{x}{C_0}{C_{12}}}}$,
                \item\llabel{ty-c2} $\jdty{x: T_1}{c_2}{\tycomp{\Sigma}{T_2}{\tyctl{z}{C_{21}}{C_0}}}$, and
                \item\llabel{in-x} $x \notin \fv(T_2) \cup \fv(\Sigma) \cup (\fv(C_{21}) \setminus \{z\})$
            \end{enumrm}
            for some $x, z, c_1, c_2, \Sigma, T_1, T_2, C_0, C_{12}$ and $C_{21}$.
            Case analysis on the evaluation derivation.
            \begin{description}
                \item[Case \rulename{E-Let}:] %We have
                    By the IH and \rulename{T-LetIp}.
                    % \def\currentprefix{subred:let:let}
                    % \begin{enumrm}[resume]
                    %     \item\llabel{eq-c'} $c' = \explet{x}{c_1'}{c_2}$, and
                    %     \item\llabel{ev-c1} $c_1 \eval c_1'$
                    % \end{enumrm}
                    % for some $c_1'$.
                    % By the IH with \lref[subred:let]{ty-c1} and \lref{ev-c1}, we have
                    % $\jdty{}{c_1'}{\tycomp{\Sigma}{T_1}{S_1}}$.
                    % Then we have the conclusion by \rulename{T-LetIp}.
                \item[Case \rulename{E-LetRet}:] We have
                    \def\currentprefix{subred:leti:ret}
                    \begin{enumrm}[resume]
                        \item\llabel{eq-c1} $c_1 = \expret{v}$, and
                        \item\llabel{eq-c'} $c' = c_2[v/x]$
                    \end{enumrm}
                    for some $v$.
                    By Lemma \ref{lem:inv} with \lref[subred:leti]{ty-c1}, we have
                    \begin{enumrm}[resume]
                        \item\llabel{sub-T0} $\jdsub{}{T_0}{T_1}$,
                        \item\llabel{ty-v} $\jdty{}{v}{T_0}$, and
                        \item\llabel{sub-S1} $\jdsub{ \mid T_0}{\square}{\tyctl{x}{C_0}{C_{12}}}$
                    \end{enumrm}
                    for some $T_0$.
                    By Lemma \ref{lem:wft} with \lref[subred:leti]{ty-c1} and inversion,
                    we have $\jdwf{}{T_1}$.
                    Then, by \rulename{T-VSub} with \lref{ty-v} and \lref{sub-T0},
                    we have $\jdty{}{v}{T_1}$.
                    Therefore, by Lemma \ref{lem:subst} with \lref[subred:leti]{ty-c2},
                    we have
                    \begin{enumrm}[resume]
                        \item\llabel{ty-c2-2} $\jdty{}{c_2[v/x]}{\tycomp{\Sigma}{T_2}{\tyctl{z}{C_{21}}{C_0[v/x]}}}$~.
                    \end{enumrm}
                    (Note that since \lref[subred:leti]{in-x},
                    it holds that $\Sigma[v/x] = \Sigma$, $T_2[v/x] = T_2$ and $C_{21}[v/x] = C_{21}$.)
                    % We have
                    % \begin{enumrm}[resume]
                    %     \item\llabel{in-x-2} $x \notin \fv(C_{21}) \setminus \{z\}$
                    % \end{enumrm}
                    % from the side condition of the definition of $\ggeq$.
                    By inversion of \lref{sub-S1}, we have
                    \begin{enumrm}[resume]
                        \item\llabel{sub-C0} $\jdsub{x:T_0}{C_0}{C_{12}}$~.
                    \end{enumrm}
                    By Lemma \ref{lem:wft} with \lref[subred:leti]{ty-c1} and inversion,
                    we have $\jdwf{}{C_{12}}$, which means $x \notin \fv(C_{12})$.
                    Therefore, by Lemma \ref{lem:subst} with \lref{sub-C0},
                    we have
                    \begin{enumrm}[resume]
                        \item\llabel{sub-C0-2} $\jdsub{}{C_0[v/x]}{C_{12}}$~.
                    \end{enumrm}
                    On the other hand, by Lemma \ref{lem:wft} with \lref[subred:leti]{ty-c2} and inversion,
                    we have $\jdwf{x:T_1, z:T_2}{C_{21}}$.
                    By Lemma \ref{lem:rm-unused} with \lref[subred:leti]{in-x},
                    we have $\jdwf{z:T_2}{C_{21}}$.
                    Then, by Lemma \ref{lem:refl} we have
                    \begin{enumrm}[resume]
                        \item\llabel{sub-C2} $\jdsub{z:T_2}{C_{21}}{C_{21}}$~.
                    \end{enumrm}
                    Hence, by \rulename{S-ATM} with \lref{sub-C0-2} and \lref{sub-C2},
                    we have
                    $\jdsub{\mid T_2}{\tyctl{z}{C_{21}}{C_0[v/x]}}{\tyctl{z}{C_{21}}{C_{12}}}$~.
                    Now we have the conclusion by subsumption of \lref{ty-c2-2}.
            \end{description}
        \item[Case \rulename{T-Hndl}:] We have
            \def\currentprefix{subred:hndl}
            \begin{enumrm}
                \item\llabel{eq-c} $c = \expwith{h}{c_0}$,
                \item\llabel{eq-h} $h = \{ \expret{x_r} \mapsto c_r, \repi{\op_i(x_i, k_i) \mapsto c_i} \}$,
                \item\llabel{ty-c0} $\jdty{}{c_0}{\tycomp{\Sigma_0}{T_0}{\tyctl{x_r}{C_1}{C}}}$,
                \item\llabel{ty-cr} $\jdty{x_r: T_0}{c_r}{C_1}$,
                \item\llabel{ty-ci} $\bigrepi{\jdty{\rep{X_i: \rep{B}_i}, x_i: T_{i1}, k_i: (y_i: T_{i2}) \rarr C_{i1}}{c_i}{C_{i2}}}$, and
                \item\llabel{eq-sig} $\Sigma_0 = \{ \repi{\op_i: \forall \rep{X_i: \rep{B}_i}. (x_i: T_{i1}) \rarr ((y_i: T_{i2}) \rarr C_{i1}) \rarr C_{2i}} \}$
            \end{enumrm}
            Case analysis on the evaluation derivation.
            \begin{description}
                \item[Case \rulename{E-Hndl}:]
                    By the IH and \rulename{T-Hndl}.
                \item[Case \rulename{E-HndlRet}:] We have
                    \def\currentprefix{subred:hndl:ret}
                    \begin{enumrm}[resume]
                        \item\llabel{eq-c0} $c_0 = \expret{v}$ and
                        \item\llabel{eq-c'} $c' = c_r[v/x_r]$
                    \end{enumrm}
                    for some $v$.
                    By Lemma \ref{lem:inv} with \lref[subred:hndl]{ty-c0}, we have
                    \begin{enumrm}[resume]
                        \item\llabel{sub-T0} $\jdsub{}{T_0'}{T_0}$,
                        \item\llabel{ty-v} $\jdty{}{v}{T_0'}$, and
                        \item\llabel{sub-S0} $\jdsub{ \mid T_0'}{\square}{\tyctl{x_r}{C_1}{C}}$
                    \end{enumrm}
                    for some $T_0'$.
                    By inversion of \lref{sub-S0}, we have
                    \begin{enumrm}[resume]
                        \item\llabel{sub-C1} $\jdsub{x_r: T_0'}{C_1}{C}$ and
                        \item\llabel{in-xr} $x_r \notin \fv(C)$~.
                    \end{enumrm}
                    By Lemma \ref{lem:narrow} with \lref[subred:hndl]{ty-cr} and \lref{sub-T0},
                    we have
                    \begin{enumrm}[resume]
                        \item\llabel{ty-cr-2} $\jdty{x_r: T_0'}{c_r}{C_1}$~.
                    \end{enumrm}
                    By Lemma \ref{lem:subst} with \lref{ty-v}
                    applied to \lref{sub-C1} and \lref{ty-cr-2}, we have
                    \begin{enumrm}[resume]
                        \item\llabel{sub-C1-2} $\jdsub{}{C_1[v/x_r]}{C}$ and
                        \item\llabel{ty-cr-3} $\jdty{}{c_r[v/x_r]}{C_1[v/x_r]}$
                    \end{enumrm}
                    respectively.
                    (Note that $C[v/x_r] = C$ since \lref{in-xr}.)
                    By Lemma \ref{lem:wft} with \lref[subred:hndl]{ty-c0} and inversion,
                    we have $\jdwf{}{C}$.
                    From this and \lref{sub-C1-2} and \lref{ty-cr-3},
                    we have the conclusion by \rulename{T-CSub}.
                \item[Case \rulename{E-HndlOp}:] We have
                    \def\currentprefix{subred:hndl:op}
                    \begin{enumrm}[resume]
                        \item\llabel{eq-c0} $c_0 = K[\op_i~v]$ and
                        \item\llabel{eq-c'} $c' = c_i[v/x_i][(\lambda y. \expwith{h}{K[\expret{y}]})/k_i]$
                    \end{enumrm}
                    for some $K$ and $v$.
                    W.l.o.g., we can assume that $y$ is disjoint from
                    any other existing variables.
                    By Lemma \ref{lem:inv-ctx} with \lref[subred:hndl]{ty-c0}, we have
                    \begin{enumrm}[resume]
                        \item\llabel{ty-opv} $\jdty{}{\op~v}{\tycomp{\Sigma_0}{T_1}{\tyctl{y}{C_0}{C}}}$ and
                        \item\llabel{ty-Ky} $\jdty{y: T_1}{K[\expret{y}]}{\tycomp{\Sigma_0}{T_0}{\tyctl{x_r}{C_1}{C_0}}}$
                    \end{enumrm}
                    for some $y, T_1$ and $C_0$.
                    By Lemma \ref{lem:inv} with \lref{ty-opv}, we have
                    \begin{enumrm}[resume]
                        \item\llabel{in-sig} $\Sigma_0 \ni \op_i: \forall \rep{X_i: \rep{B}_i}. (x_i: T_{i1}) \rarr ((y: T_{i2}) \rarr C_{i1}) \rarr C_{i2}$,
                        \item\llabel{wf-A} $\rep{\jdty{}{A}{\rep{B}_i}}$,
                        \item\llabel{ty-v} $\jdty{}{v}{T_{i1}[\rep{A/X_i}]}$,
                        \item\llabel{sub-Ti2} $\jdsub{}{T_{i2}[\rep{A/X_i}][v/x_i]}{T_1}$,
                        \item\llabel{sub-C0} $\jdsub{y: T_{i2}[\rep{A/X_i}][v/x_i]}{C_0}{C_{i1}[\rep{A/X_i}][v/x_i]}$, and
                        \item\llabel{sub-Ci2} $\jdsub{}{C_{i2}[\rep{A/X_i}][v/x_i]}{C}$
                    \end{enumrm}
                    for some $\rep{A}$.
                    Note that since \lref[subred:hndl]{eq-sig} holds, it holds that $y = y_i$ and we use
                    $\rep{X_i}, \rep{\rep{B}_i}, x_i, T_{i1}, T_{i2}, C_{i1}$, and $C_{i2}$ here
                    instead of introducing new ones.
                    % From \lref[subred:hndl]{eq-sig} and \lref{in-sig}, it holds that
                    % $x_i = x_0$, $T_{i1} = T_{01}$, $T_{i2} = T_{02}$, $C_{i1} = C_{01}$, and $C_{i2} = C_{02}$.

                    Also, by Lemma~\ref{lem:narrow} with \lref{ty-Ky} and \lref{sub-Ti2}, we have
                    \[
                        \jdty{y: T_{i2}[\rep{A/X_i}][v/x_i]}{K[\expret{y}]}{\tycomp{\Sigma_0}{T_0}{\tyctl{x_r}{C_1}{C_0}}}~.
                    \]
                    Then, by subsumption with \lref{sub-C0}, we have
                    \begin{enumrm}[resume]
                        \item\llabel{ty-Ky-2} $\jdty{y: T_{i2}[\rep{A/X_i}][v/x_i]}{K[\expret{y}]}{\tycomp{\Sigma_0}{T_0}{\tyctl{x_r}{C_1}{C_{i1}[\rep{A/X_i}][v/x_i]}}}$~.
                    \end{enumrm}
                    On the other hand, by Lemma~\ref{lem:wfg} with \lref{ty-Ky-2} we have
                    $\jdwf{}{y: T_{i2}[\rep{A/X_i}][v/x_i]}$,
                    and hence by Lemma~\ref{lem:weaken} with \lref[subred:hndl]{ty-cr} and \lref[subred:hndl]{ty-ci}, we have
                    \begin{enumrm}[resume]
                        \item\llabel{ty-cr-2} $\jdty{y: T_{i2}[\rep{A/X_i}][v/x_i], x_r: T_0}{c_r}{C_1}$ and
                        \item\llabel{ty-ci-2} $\bigrepi{\jdty{y: T_{i2}[\rep{A/X_i}][v/x_i], \rep{X_i: \rep{B}_i}, x_i: T_{i1}, k_i: (y_i: T_{i2}) \rarr C_{i1}}{c_i}{C_{i2}}}$~.
                    \end{enumrm}
                    Therefore, by \rulename{T-Hndl} with
                    \lref[subred:hndl]{eq-h}, \lref[subred:hndl]{eq-sig},
                    \lref{ty-Ky-2}, \lref{ty-cr-2}, and \lref{ty-ci-2}, we have
                    \[
                        \jdty{y: T_{i2}[\rep{A/X_i}][v/x_i]}{\expwith{h}{K[\expret{y}]}}{C_{i1}[\rep{A/X_i}][v/x_i]}~.
                    \]
                    Then by \rulename{T-Fun} we have
                    \begin{enumrm}[resume]
                        \item\llabel{ty-yhKy} $\jdty{}{\lambda y. \expwith{h}{K[\expret{y}]}}{(y: y: T_{i2}[\rep{A/X_i}][v/x_i]) \rarr C_{i1}[\rep{A/X_i}][v/x_i]}$~.
                    \end{enumrm}

                    Now, by Lemma \ref{lem:subst-pred} with \lref{wf-A}
                    applied to \lref[subred:hndl]{ty-ci}, we have
                    \[
                        \jdty{x_i: T_{i1}[\rep{A/X_i}], k_i: (y_i: T_{i2}[\rep{A/X_i}]) \rarr C_{i1}[\rep{A/X_i}]}
                            {c_i}{C_{i2}[\rep{A/X_i}]}~.
                    \]
                    By applying \ref{lem:subst} twice with \lref{ty-v} and \lref{ty-yhKy} in a row, we have
                    \[
                        \jdty{}{c_i[v/x_i][(\lambda y. \expwith{h}{K[\expret{y}]})/k_i]}{C_{i2}[\rep{A/X_i}][v/x_i]}~.
                    \]
                    Note that
                    $C_{i2}[\rep{A/X_i}][v/x_i][(\lambda y. \expwith{h}{K[\expret{y}]})/k_i] = C_{i2}[\rep{A/X_i}][v/x_i]$
                    since $k_i \notin \fv(C_{i2}[\rep{A/X_i}][v/x_i])$ by Lemma \ref{lem:notin-nonrfn}.
                    Now we have the conclusion by subsumption with \lref{sub-Ci2}.
            \end{description}
    \end{description}
\end{proof}