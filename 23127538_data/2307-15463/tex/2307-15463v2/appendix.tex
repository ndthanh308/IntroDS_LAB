% \appendix

% \subsection{Typing Rule for Operation Forwarding} \label{sec:language/forwarding}
\section{Typing Rule for Operation Forwarding} \label{sec:language/forwarding}
The typing rule for handling constructs presented in Section 3.2 of the main paper %\ref{sec:language/type-system}
assumes that a handler covers all the operations performed
by the handled expression.
In this section, we present another typing rule for handling constructs
to allow \emph{operation forwarding}, that is, allow unhandled operations to be forwarded to outer handlers automatically.
The idea of the typing rule is simple: we derive it from an implementation of operation forwarding.
As mentioned in Section~3.1 of the main paper, %\ref{sec:language/syntax-semantics},
operation forwarding can be implemented in a calculus without forwarding by adding to a handler
an operation clause $\op(x, k) \mapsto \explet{y}{\op~x}{k~y}$ for each forwarded operation $\op$.
Therefore, we can derive the new typing rule
from the typing of the added clauses. The following is the thus derived new typing rule for handling constructs which natively supports operation forwarding:
\[
\infer{\jdty{\Gamma}{\expwith{h}{c}}{C_2}}
{\begin{gathered}
    h = \{ \expret{x_r} \mapsto c_r, \repi{\op_i(x_i, k_i) \mapsto c_i} \} \quad
    \jdty{\Gamma}{c}{\tycomp{\Sigma}{T}{\tyctl{x_r}{C_1}{C_2}}} \\[-.5ex]
    \jdty{\Gamma, x_r: T}{c_r}{C_1} \quad
    \bigrepi{\jdty{\Gamma, \rep{X_i: \rep{B}_i}, x_i: T_{1i}, k_i: (y_i: T_{2i}) \rarr C_{1i}}{c_i}{C_{2i}}} \\[-.5ex]
    \bigrepi{ \Sigma \ni \op_i: \forall \rep{X_i: \rep{B}_i}. (x_i: T_{1i}) \rarr ((y_i: T_{2i}) \rarr C_{1i}) \rarr C_{2i} }
    \qquad \mathit{Ops}_{\mathrm{fwd}} = \dom(\Sigma) \setminus \dom(h) \\[-.5ex]
    \bigrepi[\op \in \mathit{Ops}_{\mathrm{fwd}}]{\begin{gathered}
        \begin{multlined}
            \Sigma \ni \op : \forall \rep{X^{\op}: \rep{B^{\op}}}. (x^{\op}: T_1^{\op}) \rarr \\
                ((y^{\op}: T_2^{\op}) \rarr
                    \tycomp{\Sigma'}{T_0^{\op}}{\tyctl{z^{\op}}{C_0^{\op}}{C_1^{\op}}}) \rarr
                \tycomp{\Sigma'}{T_0^{\op}}{\tyctl{z^{\op}}{C_0^{\op}}{C_2^{\op}}}
        \end{multlined} \\
        \Sigma' \ni \op : \forall \rep{X^{\op}: \rep{B^{\op}}}. (x^{\op}: T_1^{\op}) \rarr ((y^{\op}: T_2^{\op}) \rarr C_1^{\op}) \rarr C_2^{\op} \quad
        y^{\op} \notin C_0^{\op} \setminus \{ z^{\op} \}
    \end{gathered}
    }
\end{gathered}}
\]
where $\dom(\Sigma)$ denotes the set of the operations associated by $\Sigma$
and $\dom(h)$ denotes the set of the operations handled by $h$,
that is, the set $\{ \repi{\op_i} \}$.
The first two lines are the same as \rulename{T-Hndl}.
The third line is also similar to the last premise of \rulename{T-Hndl},
but here $\Sigma$ is allowed to contain operations other than those handled by $h$.
$\mathit{Ops}_{\mathrm{fwd}}$ is exactly the set of the unhandled (i.e., forwarded) operations.
The last part is the requirement for the forwarded operations,
which can be obtained from the typing derivations of $\op(x, k) \mapsto \explet{y}{\op~x}{k~y}$ as follows.
When we simulate the operation forwarding with the explicit clause,
the operation call $\op~x$ in the clause is handled
by an immediate outer handler (we denote it by $h'$ in what follows).
Therefore, its operation signature is different from $\Sigma$;
in fact, it corresponds to $\Sigma'$ in the rule.
Also, the answer types of the original operation calls of $\op$
(i.e., the answer types of the operation calls of $\op$ in the handled computation $c$)
should have $\Sigma'$ as their operation signatures,
because the final answer type corresponds to the type of the handling construct,
which is handled by the immediate outer handler $h'$.
Therefore, the types of the forwarded operations in $\Sigma$ contains $\Sigma'$
in their answer types.
In addition, the types
$T_0^{\op}$, $T_1^{\op}$, $T_2^{\op}$, $C_0^{\op}$, $C_1^{\op}$, and $C_2^{\op}$
appear multiple times in $\Sigma$ and $\Sigma'$, restricting the type schemes of the operations in $\mathit{Ops}_{\mathrm{fwd}}$.
This restriction can be understood as follows.
First, assume that the original operation call of $\op$ in $c$
has the operation signature $\Sigma$ such that
\[
    \Sigma \ni \op : T_1^{\op} \rarr
        (T_2^{\op} \rarr
            \tycomp{\Sigma'}{T_0^{\op}}{\tyctlMB{C_0^{\op}}{C_1^{\op}}}) \rarr
        \tycomp{\Sigma'}{T_{0A}^{\op}}{\tyctlMB{C_{0A}^{\op}}{C_2^{\op}}}
\]
for some $T_1^{\op}$, $T_2^{\op}$, $T_0^{\op}$, $C_0^{\op}$, $C_1^{\op}$,
$T_{0A}^{\op}$, $C_{0A}^{\op}$, $C_2^{\op}$, and $\Sigma'$, under a context $\Gamma$.
Here we consider only simple types for simplicity,
but a similar argument can be made for dependent and refinement types
by appropriately naming the variables like in the rule above.
Note that its answer types have $\Sigma'$ as described earlier,
and that we do not impose the restriction yet.
From the assumption, the clause $\explet{y}{\op~x}{k~y}$ should be typed
under the context $\Gamma, x: T_1^{\op},
k: T_2^{\op} \rarr \tycomp{\Sigma'}{T_0^{\op}}{\tyctlMB{C_0^{\op}}{C_1^{\op}}}$~.
Then, the input type of $\op$ in the clause should be the type of $x$, namely, $T_1^{\op}$,
and the output type of $\op$ should be the type of the variable $y$,
which turns out to be $T_2^{\op}$ from the type of $k$.
Therefore, the operation signature $\Sigma'$ for $\op~x$ should contain
$\op : T_1^{\op} \rarr (T_2^{\op} \rarr C_{1A}^{\op}) \rarr C_{2A}^{\op}$
for some $C_{1A}^{\op}$ and $C_{2A}^{\op}$~.
Then, according to the typing rules for operation calls and let-expressions,
it is required that $C_{1A}^{\op} = C_1^{\op}$,
and the type of $\explet{y}{\op~x}{k~y}$ is
$\tycomp{\Sigma'}{T_{0}^{\op}}{\tyctlMB{C_{0}^{\op}}{C_{2A}^{\op}}}$~.
Finally, since the type of the clause corresponds to
the final answer type of the operation $\op$ in $\Sigma$
(which is $\tycomp{\Sigma'}{T_{0A}^{\op}}{\tyctlMB{C_{0A}^{\op}}{C_2^{\op}}}$ from the assumption),
it should satisfy
$T_{0}^{\op} = T_{0A}^{\op}$, $C_{0}^{\op} = C_{0A}^{\op}$, and $C_{2A}^{\op} = C_{2}^{\op}$~.



\section{Detailed explanation of the benchmark} \label{sec:benchmark-details}

In this section, we present the result of the verification of
the benchmark \texttt{queue-2-SAT.ml} as an example.
The following is the main part of the program of \texttt{queue-2-SAT.ml}:
\begin{verbatim}
let[@annot_MB "int list ->
    (unit -> ({Get_next: s1, Add_to_queue: s2} |> int option / s => s)) ->
    int option"]
  queue initial (body :unit -> int option) =
    match_with body () {
      retc = (fun x -> (fun _ -> x));
      exnc = raise;
      effc = fun (type a) (e: a eff) -> match e with
        | Get_next _ctx -> Some (fun (k: (a, _) continuation) ->
            (fun queue -> match queue with
              | [] -> continue k None []
              | hd::tl -> continue k (Some hd) tl) )
        | Add_to_queue v -> Some (fun (k: (a, _) continuation) ->
            (fun queue -> continue k () (queue @ [v])) )
    } initial

let main init =
  queue init (fun () ->
    perform (Add_to_queue 42);
    let _ = perform (Get_next 1(*dummy*)) in
    perform (Get_next 2(*dummy*)) )
\end{verbatim}
This program uses two operations \texttt{Get\_next} and \texttt{Add\_to\_queue},
which are used to dequeue and enqueue elements respectively.
The function \texttt{queue} manages the queue.
It receives an initial queue \texttt{initial} and the function \texttt{body},
handling the operations performed in \texttt{body} in the state-passing manner
to simulate the behavior of the queue.
The first three lines of the program are the underlying simple type annotation,
which tells the function \texttt{queue}
that the argument \texttt{body} may perform the operations \texttt{Get\_next} and \texttt{Add\_to\_queue}
and that its control effect is impure.
This annotaion is necessary because our implementation does not support effect polymorphism
as mentioned in Section~4 of the main paper. %\ref{sec:impl}.
The main function \texttt{main} first enqueue one element,
and then try to dequeue twice
(\texttt{Get\_next} returns \texttt{None} when the queue is empty).
Note that we added a ghost parameter \texttt{\_ctx} to \texttt{Get\_next},
which is used to distinguish its two occurrences.
We give \texttt{1} to the first occurrence of \texttt{Get\_next},
and \texttt{2} to the second.
This ghost parameter is crucial for the precise verification of this program,
described later in this section.

\newcommand{\ilist}{\kwty{ilist}}
% \newcommand{\ilist}{\kwty{int\ list}}
\newcommand{\iopt}{\kwty{iopt}}
% \newcommand{\iopt}{\kwty{int\ option}}
\newcommand{\expNil}{\mathtt{[]}}
\newcommand{\expNone}{\mathtt{None}}
\newcommand{\isCons}{\mathrm{isCons}}
\newcommand{\isSome}{\mathrm{isSome}}
\newcommand{\tail}{\mathrm{tail}}

We defined the following refinement type
as the specification for the main function \texttt{main}
(here after, we abbreviate the type $\kwty{int\ list}$ and $\kwty{int\ option}$
as $\ilist$ and $\iopt$ respectively):
\[
  \tyrfn{z}{\ilist}{z \ne \expNil} \rarr \tyrfn{z}{\iopt}{z \ne \expNone}
\]
That is, if the queue is initially not empty,
the last dequeue should return some value.

By running the verification of the program with the specification,
our implementation returns ``SAT'' as shown in Table~1 in the main paper, %\ref{tab:eval},
that is, the function \texttt{main} certainly has the type given as the specification.
Let us investigate more detail by seeing the inferred type of the function \texttt{queue}:
{\allowdisplaybreaks
\begin{align}
  &(init:\tyrfn{z}{\ilist}{z \ne \expNil}) \\
  &\rarr (\tyunit \rarr \tycomp{\Sigma}{\iopt}{\tyctl{x}{
    (\ilist \rarr \tyrfn{z}{\iopt}{\phi_1})
  }{
    (\tyrfn{z}{\ilist}{\phi_2} \rarr \tyrfn{z}{\iopt}{z \ne \expNone})
  }}) \!\!\! \\
  &\rarr \tyrfn{z}{\iopt}{z \ne \expNone}
\end{align}
\[
  \text{where} \begin{array}[t]{r@{}l}
    \Sigma \defeq \{
      &\mathtt{Add\_to\_queue}: \tyint \rarr (\tyunit \rarr \\
        &\quad ((q:\ilist) \rarr \tyrfn{z}{\iopt}{\phi_{41}}))
          \rarr (\tyrfn{z}{\ilist}{\phi_2} \rarr \tyrfn{z}{\iopt}{z \ne \expNone}), \\
      &\mathtt{Get\_next}: (ctx: \tyint) \rarr ((y: \iopt) \rarr \\
        &\quad ((q:\ilist) \rarr \tyrfn{z}{\iopt}{\phi_{31} \land \phi_{32}}))
          \rarr ((q:\ilist) \rarr \tyrfn{z}{\iopt}{\phi_{41} \land \phi_{42}})
    \}\end{array}
\]
\[\begin{array}[t]{r@{}l@{\quad}r@{}l}
  \phi_1 &\defeq \isSome(x) \Rarr z \ne \expNone &
  \phi_2 &\defeq init \ne \expNil \Rarr z \ne \expNil \\
  \phi_{31} &\defeq \isCons(q) \land \isSome(y) \Rarr z \ne \expNone &
  \phi_{32} &\defeq \isSome(y) \land ctx \ge 2 \Rarr z \ne \expNone \\
  \phi_{41} &\defeq \isCons(q) \land \isCons(\tail(q)) \Rarr z \ne \expNone &
  \phi_{42} &\defeq \isCons(q) \land ctx \ge 2 \Rarr z \ne \expNone
\end{array}\]
}
where $\isSome(x)$ holds if $x = \mathtt{Some}~v$ for some $v$,
$\isCons(x)$ holds if $x = v\mathtt{::}w$ for some $v$ and $w$,
and $\tail(x)$ returns the tail of the list $x$.
In the operation sigunature, we can find that
\texttt{Add\_to\_queue} changes the answer type from
$(q:\ilist) \rarr \tyrfn{z}{\iopt}{\phi_{41}}$
to
$\tyrfn{z}{\ilist}{\phi_2} \rarr \tyrfn{z}{\iopt}{z \ne \expNone}$.
Therefore, \verb|perform (Add_to_queue 42)| can be given the control effect
\[
  \tyctl{\_}{
    ((q:\ilist) \rarr \tyrfn{z}{\iopt}{\phi_{41}})
  }{
    (\tyrfn{z}{\ilist}{\phi_2} \rarr \tyrfn{z}{\iopt}{z \ne \expNone})
  }~.
\]
Similarly, in the operation sigunature,
\texttt{Get\_next} changes the answer type from
$(q:\ilist) \rarr \tyrfn{z}{\iopt}{\phi_{31} \land \phi_{32}}$
to
$(q:\ilist) \rarr \tyrfn{z}{\iopt}{\phi_{41} \land \phi_{42}}$~.
Here, since the refinements of these answer types contain a condition on $ctx$,
their truth depend on whether $ctx = 1\ (< 2)$ or $ctx = 2\ (\ge 2)$.
This enables assigning different control effects (i.e., different ARM)
to each occurrence of \texttt{Get\_next} depending on the context.
Namely, \verb|perform (Get_next 1)| can be given the control effect
\[
  \tyctl{y}{
    (q:\ilist) \rarr \tyrfn{z}{\iopt}{\phi_{31}}
  }{
    (q:\ilist) \rarr \tyrfn{z}{\iopt}{\phi_{41}}
  }
\]
since $ctx = 1$,
while \verb|perform (Get_next 2)| can be given the control effect
\[
  \tyctl{y}{
    \ilist \rarr \tyrfn{z}{\iopt}{\isSome(y) \Rarr z \ne \expNone}
  }{
    (q:\ilist) \rarr \tyrfn{z}{\iopt}{\isCons(q) \Rarr z \ne \expNone}
  }
\]
since $ctx = 2$.
Now, the control effect of the argument \texttt{body} can be obtained from
the composition of these three control effects,
which results in
\[
  \tyctl{x}{
    (\ilist \rarr \tyrfn{z}{\iopt}{\phi_1})
  }{
    (\tyrfn{z}{\ilist}{\phi_2} \rarr \tyrfn{z}{\iopt}{z \ne \expNone})
  }~.
\]
Then, the handling construct is assigned the final answer type of \texttt{body},
i.e., $\tyrfn{z}{\ilist}{\phi_2} \rarr \tyrfn{z}{\iopt}{z \ne \expNone}$,
and finally applying the non-empty initial queue to the handling construct returns
a value of type $\tyrfn{z}{\iopt}{z \ne \expNone}$ as expected.