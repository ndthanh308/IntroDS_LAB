In this work, we investigated functional mimicry as a smooth and effective telemanipulation paradigm. Functional mimicry allows a robot to exploit flexibility in task tolerances to generate more accurate, smooth, and feasible motions. Our user evaluation demonstrated that the autonomous adjustments within task tolerances led to equal or better performance without sacrificing perceived control of the robot. Moreover, the functional mimicry manipulator that exploited task tolerances was perceived to be more under control, predictable, fluent, and trustworthy than the manipulator that exactly mimicked its human operator. Below, we discuss additional findings, limitations of this work, and implications for future teleoperation systems.

\subsection{Subgroup Analysis}

Figure \ref{fig: subgroup_results} shows our subgroup analysis results. As mentioned in \cref{sec:perception_shared_control}, prior works have found that people with a high internal Locus of Control (LoC) have \replaced{trouble}{problems} giving up control to an autonomous system. We observed a similar phenomenon in our study. Although not statistically significant, the autonomous adjustments provided in functional mimicry led to less performance and perception improvement in high internal LoC participants than in participants with an average LoC. Moreover, we also classified participants' expertise according to their reported familiarity in operating robots, avatars in video games, and virtual objects in Computer-Aided Design software or virtual reality. We identified 9 participants with low expertise and 11 with high expertise. Although not statistically significant,  functional mimicry brought a larger performance and perception improvement in expert users than in non-expert users. We note that the result is not directly comparable with what is found in prior shared control work \cite{milliken2017modeling}, where non-expert users benefit more from autonomous assistance. As described in \cref{sec:shared_control}, in prior work the shared control system \textit{explicitly}  assists users in avoiding obstacles and directly contributes to task completion, so non-expert users \replaced{with insufficient}{of low} obstacle avoidance skills gain more benefits from the explicit assistance. Meanwhile, our system \textit{implicitly} assists them by generating high-quality robot motions. While both expert and non-expert users benefit from the high-quality motions, we speculate that expert users can take more advantage of the high-quality motions to complete the manipulation tasks. 

% Figure environment removed

\subsection{Limitations}
While our results demonstrate the potential for functional mimicry to provide smooth and effective telemanipulation, the limitations of the present work suggest directions for future research.
In our empirical study, participants and the robot were located in the same room, which gives the participants perfect situational awareness of the robot and its surroundings and allows the robot to be controlled with low latency. With limited situational awareness (\textit{e.g.}, viewing the robot's workspace through a camera) and high latency, human operators may be more sensitive and disturbed by autonomous robot adjustments, even if they are within task tolerances. Future work should evaluate functional mimicry when the human operator and the robot are in separate physical spaces. While this work demonstrates the benefits of exploiting task tolerances, the task tolerances in our experiment were manually specified. Future work should investigate algorithms to automatically detect task tolerances. 

\subsection{Implications}
Instead of forcing a robot manipulator to exactly mimic its human operator, a mimicry-based teleoperation system can allow some task-specific inaccuracy. Such flexibility in task tolerance can be exploited by a robot to generate accurate, smooth, and feasible motions, leading to user perception and performance improvements. We believe that \replaced{our interaction paradigm is beneficial to teleoperation of welding, sanding, painting, wiping, pouring, and many other tasks that allow some positional or rotational inaccuracy.}{many high-precision teleoperation scenarios would benefit from our interaction paradigm.} Additionally, our user study results suggest that autonomous robot adjustments do not impede perceived control of the robot as long as the end-effector poses are within task tolerances. This finding provides a robot more freedom to enhance motion generation. Aside from the improved accuracy, smoothness, and manipulability demonstrated in this work, we believe that other aspects of robot motions, such as legibility, predictability \cite{dragan2013legibility}, and expressiveness \cite{venture2019robot}, can be enhanced with the flexibility in task tolerances.

\vspace{-0.5mm}
\section{Acknowledgement}
We would like to thank Seth Peterson for 3D printing tool mounts for the robot used in the study\added{ and anonymous reviewers for their suggestions and comments}. 

