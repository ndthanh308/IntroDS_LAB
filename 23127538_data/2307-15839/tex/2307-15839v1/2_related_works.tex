We review relevant prior research from three areas: shared control methods, user perception of shared control systems, and semi-constrained tasks.

\subsection{Shared Control} \label{sec:shared_control}
Niemeyer et al. \cite{niemeyer2016telerobotics} organize teleoperation control methods in a spectrum that involves direct and shared control. Direct control allows the user to unambiguously control the robot, while shared control seeks to provide \added{some} autonomous assistance, such as obstacle avoidance \cite{you2012assisted, kang2021rcik, takayama2011assisted, acharya2018inference, milliken2017modeling} or guidance to an effective motion or strategy \cite{dragan2013policy, nikolaidis2017human}.
The teleoperation paradigm explored in this paper aims to gain benefits from both direct and shared control: it makes autonomous robot adjustments for more accurate, smooth, and feasible motions while preserving directness in the users' control. Below, we further articulate the distinctions between our method and prior shared control works.

Both our system and some shared control works autonomously control a subset of an end-effector's degrees of freedom (DoFs). However, our system exploits the DoFs that are \textit{task-irrelevant} to improve robot motion quality, while shared control systems often take over \textit{task-relevant} DoFs to reduce a user's workload, \textit{e.g.}, autonomously rotating a manipulator's gripper for grasping \cite{stoyanov2018assisted, abi2016visual}. 
While most shared control systems \textit{explicitly} assist task completion, the teleoperation paradigm explored in this paper provides \textit{implicit} assistance by generating high-quality robot motions.

\subsection{User Perception in Shared Control}
\label{sec:perception_shared_control}
While improving safety and performance, shared control may make the user's control less direct, suggesting a trade-off between performance and perception \cite{javdani2015shared}. While some prior works have shown that user satisfaction of a shared control system strongly correlates with task performance \cite{dragan2013policy, hauser2013recognition}, You et al. \cite{you2012assisted} found that users are willing to tolerate loss of control and less predictable motions \textit{only} for significant performance improvements. Moreover, Nikolaidis et al. \cite{nikolaidis2017human} found trust to be inconsistent with performance in some shared control systems.
In our teleoperation paradigm, although users lose some control of the robot by allowing autonomous adjustments within task tolerances, we anticipate that users will still feel in control because the adjustments do not affect task completion and are natural to users because humans also utilize task tolerances in manipulation.

To further understand the perception of shared control systems, prior research has studied intrinsic user qualities, \textit{e.g.}, Locus of Control (LoC). LoC describes the degree to which people believe that their behaviors affect the outcome of events in their life \cite{rotter1966generalized}. People with a more \textit{internal} LoC believe they have more control over the events in their life whereas people with a more \textit{external} LoC believe the outcome of events is more affected by external forces such as luck or fate. 
Prior works have found that people with a high internal LoC have trouble giving up control to an autonomous system\replaced{ \cite{takayama2011assisted, acharya2018inference}.}{. Takayama et al. \cite{takayama2011assisted} found that a collision avoidance system significantly increased task completion time for operators with a high internal LoC. Similarly, Acharya et al. \cite{acharya2018inference} found that people with high internal LoC were more likely to issue conflicting commands that diverge from an obstacle avoidance algorithm.} \replaced{Therefore, we anticipate that users with a high internal LoC are more sensitive to the autonomous robot adjustments in our \textit{functional} mimicry paradigm, leading to less improvements in task performance and user perception.}{In our study, we also anticipate users with a high internal LoC to have trouble giving up control to our autonomous system.}

\subsection{Utilizing Flexibility in Semi-Constrained Tasks}

A task with tolerances is semi-constrained, as it does not impose \added{strict} constraints to all six degrees of freedom of the end effector\added{, constructing a null space in which the robot can freely move}. Prior works have exploited the flexibility in semi-constrained tasks by projecting a secondary task into the null space of a semi-constrained task. Common secondary tasks include self-collision avoidance \cite{petrivc2011smooth}, singularity avoidance \cite{nemec2000null}, compliant behavior \cite{sadeghian2013task}, balancing a humanoid robot \cite{henze2016passivity, abi2018humanoid}, or conveying emotions \cite{claret2017exploiting}. In contrast to previous work, our robot exploits the flexibility in a semi-constrained task to generate accurate and smooth motions and we explore its effects in mimicry-based telemanipulation.

To generate motions that utilize flexibility in semi-constrained tasks, several semi-constrained motion planners have been presented \cite{Descartes, de2017cartesian, malhan2022generation, berenson2011task, cefalo2020opportunistic}. 
While semi-constrained motion planners can effectively generate motions for path following, they are not appropriate in time-sensitive scenarios such as teleoperation. In this work, we employ \textit{RangedIK}\cite{wang2023rangedik}, which is a per-instant pose optimization method, to exploit task tolerances and generate robot motions in real time.