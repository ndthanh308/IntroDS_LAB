Our evaluation was focused on assessing how automatic robot adjustments within task tolerances may influence robot telemanipulation. 
We pre-registered\footnote{\url{https://osf.io/2ryng/?view\_only=a04dac33e23c4448883f79302de0d0b1}} our hypothesis, study design, measures, sample size, and statistical analyses before collecting data.
Our central hypothesis is that allowing a robot to exploit task tolerances in mimicry-based telemanipulation will lead to better task performance and user experience than requiring the robot to exactly mimic its operator.

\subsection{Experimental Design \& Conditions} \label{sec:desgin}

Our user evaluation followed a within-participants design, with each participant working with the robots in both conditions. We counterbalanced the order in which the two conditions were presented.

\emph{Exact Mimicry} (control condition): as described in \cref{sec:mimicry}, the robot tried to exactly mimic all six degrees-of-freedom of the participant's hand movements. 
% mimicry control \cite{rakita2017motion}

\emph{Functional Mimicry} (experimental condition): as described in \cref{sec:mimicry}, the robot mimicked the participant's hand movements but was allowed to autonomously adjust within task tolerances. 

\subsection{Experimental Tasks} \label{sec:tasks}
% Figure environment removed

\begin{table}[!tb]
\caption{Task Tolerances}
\label{tab:task_ranges}
\vspace{-3mm}
\begin{center}
\begin{tabular}{lllllll}
\hline
\rule{0pt}{1.1\normalbaselineskip}%
Task & \multicolumn{6}{c}{Tolerances} \\
& \makecell[l]{$x$ \\ (m)} & \makecell[l]{$y$ \\ (m)} & \makecell[l]{$z$ \\ (m)} & \makecell[l]{rx \\ (rad)} & \makecell[l]{ry \\ (rad)} & \makecell[l]{rz \\ (rad)} \\[0.8mm] 
\hline
\rule{0pt}{1.1\normalbaselineskip}Writing Letters & 0 & 0 & 0 & $\pm\pi/6$ & $\pm\pi/6$ & $\pm\infty$ \\
Erasing Lines & 0 & 0 & 0 & 0 & 0 & $\pm\infty$  \\
Dropping Envelope & $\pm0.05$ & 0 & 0 & 0 & 0 & 0  \\
Carrying Full Cup & 0 & 0 & 0 & 0 & $\pm\infty$ & 0  \\
Pouring Objects & 0 & 0 & 0 & 0 & $\pm\infty$ & 0  \\[0.8mm] 
\hline
\vspace{-8mm}
\end{tabular}
\end{center}
\end{table}

We constructed five manipulation tasks that have tolerances (Figure \ref{fig: tasks} and Table \ref{tab:task_ranges}). The first two tasks involved manipulation on a whiteboard and a haptic stylus provided haptic feedback to assist participants in keeping the end-effector on the whiteboard plane. The other three tasks involved large robot movements in free space and we used a VR Controller as the input device. 

1) \emph{Writing Letters}: This task required participants to write letters on a virtual whiteboard. To ensure fair comparisons, we instructed participants to trace the same pattern as quickly as possible while maintaining accuracy. This task required accurate pen tip positions but \replaced{allowed}{allows} the pen to tilt (rotational tolerances about the $x$ and $y$ axes) or freely rotate about the pen’s principal $z$ axis. This task was challenging because it required the pen tip to accurately trace a long curve.

2) \emph{Erasing Lines}: With a round eraser as the robot's end-effector, participants wiped down a pre-defined, lawnmower trace on the whiteboard. Participants were instructed to finish the task as quickly as possible. 
The round eraser had rotational symmetry and the robot was free to rotate about the axis of the eraser.
This task required dexterously rotating the eraser to keep it parallel to the whiteboard while moving the eraser along a long, continuous path.

3) \emph{Dropping an Envelope}: This task required dropping an envelope into a mailbox. Participants were instructed to finish the task as quickly as possible but avoid collisions with the mailbox which bends the envelope. The aperture of the mailbox is 0.22 m long, enabling positional tolerance along the $x$ axis.  

4) \emph{Carrying a Full Cup}: Like holding a full cup of water, participants were instructed to keep a cup upright and carry the cup to a target position as quickly as possible. A warning sound was played if the cup was tilted more than 10 degrees. Keeping the cup upright required dexterous manipulation of the rotation about the $x$ and $z$ axes, while the rotational symmetry of the cup allowed unbounded rotational tolerance about the $y$ axis.

5) \emph{Pouring Objects}: This task involved pouring 10 binder clips from one cup to another cup on the tabletop. Participants were instructed to pour all the binder clips as quickly as possible. The pouring task required smooth rotation to prevent spilling. 

\subsection{Implementation Details}

A participant's hand motions were captured using a 3D Systems Touch Stylus at approximately 100 Hz for the \textit{Writing} and \textit{Erasing} tasks and an HTC Vive Controller at approximately 60 Hz for the \textit{Dropping}, \textit{Carrying}, and \textit{Pouring} tasks. The hand motions were mapped to the end-effector of a six degree-of-freedom Universal Robots UR5 manipulator. 
We used the open-source implementation of \textit{RangedIK}\footnote{\url{https://github.com/uwgraphics/relaxed\_ik\_core/tree/ranged-ik}} to generate robot motions \added{at 200 Hz} in both conditions. As described in \cref{sec:exploiting}, the autonomous robot adjustments in functional mimicry were enabled by handling the degrees of freedom (DoFs) with tolerances differently. Meanwhile, the exact mimicry condition required the robot to accurately match all DoFs of the end-effector.
A button on the stylus and the trigger button on the Vive controller served as clutching buttons to \replaced{connect or disconnect}{build or demolish connections} to the robot. To efficiently display curves for participants to trace or erase, we implemented a virtual whiteboard using the \texttt{pygame} Python library on a TV with a resolution of 1920 $\times$ 1080 and a screen size of 1.05m $\times$ 0.59m. 

\subsection{Experimental Procedure}
Upon receiving consent, an experimenter introduced the goal of the study and the usage of both input devices (a haptic stylus and a VR controller) to participants. Then participants were presented the first condition, in which participants performed practice tasks and then the experimental tasks described in \cref{sec:tasks}. The practice tasks were simplified from the experimental tasks: writing a straight line, erasing a small area, dropping an envelope into a mailbox that is close to the robot, carrying a full cup for a short time, and pouring binder clips into a cup that is close to the robot. 
Upon finishing all the practice and experimental tasks, participants filled out a questionnaire regarding their experience in the condition. 
This procedure, including performing the practice and experimental tasks and filling out the questionnaire, was repeated for the other condition. Upon finishing both conditions, participants completed a demographic questionnaire followed by an  Internal Control Index questionnaire \cite{duttweiler1984internal} to measure their Locus of Control. 
The experiment ended with a semi-structured interview.
% in which participants were asked three questions: \emph{``Which section do you prefer?'', ``Can you tell the differences between sections?'', ``Throughout the experiment, have you ever felt that the robot is no longer under control?''}.
Participants received \$10 compensation for about 40 minutes in the experiment.

\subsection{Measures}
We employed a combination of objective and subjective measures to assess participants’ performance and user experience.
\subsubsection{Objective measures}
We employed completion time and an error metric to assess the performance of each task. The maximum time limit to complete each task was 60 seconds. Because the time and the error metric were possibly associated, \textit{e.g.}, a participant who hurriedly finished the task in a short time might have a large error, we formulated a \textit{combined metric} for each task to aggregate the data.  We combined task time $T$ and error metrics $E$ by normalizing them\replaced{ over all participants}{ to $[0,1]$} and summing them together. 
%
\begin{equation}
    \textit{Combined Metric} = \frac{T-T_{min}}{T_{max}-T_{min}} + \frac{E-E_{min}}{E_{max}-E_{min}}
\end{equation}
%
The resulting range is $[0,2]$, where a lower value indicates better performance. 

For the \textit{Writing} task, we formulated a trajectory error metric to assess how well participants trace the target curve. The trajectory error metric is the sum of an accuracy score and a completeness score. The accuracy score is the average error distance between the pen tip and its closest point on the target curve. The closest point on the target curve is marked as reached. To compute the completeness score, we first associate an arc-length parameter value in $[0,1]$ to all points on the target curve. The completeness score is the maximum arc-length parameter value of the points that are marked as reached. We normalize the accuracy score \replaced{over all participants. The resulting}{to $[0,1]$ and the} range of trajectory error is $[0,2]$, where a lower value indicates a better trajectory.

To measure the performance of the \emph{Erasing} task, we measured the area of non-erased marks and reported them in $m^2$. For the \emph{Dropping} task, we counted the number of collisions between the envelope and the drop box, which lead to bends in the envelope. We counted the amount of time when the cup was tilted more than 10 degrees to measure errors in the \emph{Carrying} task. For the \emph{Pouring} task, we counted the number of clips that fell outside of the target cup or stayed in the robot's cup.

In addition, we measured the \emph{accuracy}, \emph{smoothness}, and \emph{manipulability} of robot motions using 6 metrics. Motion \emph{accuracy} was measured using mean position error (m) and mean rotation error (rad) between an end-effector's pose and its goal pose specified by the user. The errors were measured only in the task-relevant degrees of freedom. We used mean joint velocity (rad/s), mean joint acceleration (rad/s$^2$), and mean joint jerk (rad/s$^3$) to assess motion \emph{smoothness}. Motion \emph{manipulability} was measured by mean Yoshikawa manipulability \cite{yoshikawa1985manipulability}, where a higher value indicates better manipulability.

\subsubsection{Subjective measures}
We administered a questionnaire based on prior research in mimicry-based telemanipulation \cite{rakita2020effects} and shared control \cite{javdani2015shared, qiao2021learning} to measure perceived \textit{control}, \textit{predictability},  \textit{fluency}, and \textit{trust}. Additionally, we employed NASA TLX \cite{hart1988development} to assess perceived workload.

\begin{table}[tb]
% \begin{tabularx}{\textwidth}{X}
\caption{Items in Subjective Questionnaire}
\label{tab:questions}
\vspace{-3mm}
\begin{center}
\begin{tabular}{l}
\hline
\rule{0pt}{1.1\normalbaselineskip}%
\textbf{Control} \\
\hspace{5mm} I felt in control. \\
\hspace{5mm} I felt I could control the robot. \\
%\hline
\textbf{Predictability} \\
\hspace{5mm} The robot consistently moved in a way that I expected. \\
\hspace{5mm} The robot’s motion was not surprising. \\
\hspace{5mm} The robot responded to my motion inputs in a predictable way. \\
\hspace{5mm} I was often confused about where to move the robot. \\
%\hline
\textbf{Fluency} \\
\hspace{5mm} The robot contributed to the fluency of the interaction. \\
\hspace{5mm} The robot and I worked fluently together as a team. \\
%\hline 
\textbf{Trust} \\
\hspace{5mm} I trusted the robot to do the right thing at the right time. \\
\hspace{5mm} The robot was trustworthy. \\
\hline
\end{tabular}
\end{center}
\vspace{-4mm}
\end{table}
    % \end{tabularx}
% \fi

\subsection{Participants} \label{sec:participants}

We recruited 20 participants from a university campus (10 females, 9 males, and 1 non-binary). Participants, aged 18 to 39 (M = 23.40, SD = 5.03), had a variety of education backgrounds, including engineering, business, biology, and communication arts. Through 5-point Likert scale, participants reported low-to-moderate familiarity with robots (M = 2.35, SD = 1.06), 3D video games (M = 2.65, SD = 1.59), Computer-Aided Design (CAD) software (M = 2.15, SD = 1.06), and VR controllers (M = 2.40, SD = 1.20). One of 20 participants reported themselves as left-handed, with others being right-handed. Participants used their dominant hands to operate the robot.

Participants' Locus of Control (LoC) was measured using the Internal Control Index (ICI) \cite{duttweiler1984internal}, which involves twenty-eight 5-point Likert scales and generates an ICI score ranging from 28 to 140. Following prior work \cite{chiou2021trust}, we equally divided ICI scores into three categories: \replaced{high \textit{external} LoC ($<65$),  \textit{average} LoC ($65-102$), and high \textit{internal} LoC ($>102$)}{scores lower than 65 have high \textit{external} LoC, scores between 66 and 102 have \textit{average} LoC, and scores larger than 102 have high \textit{internal} LoC}. In our study, 10 participants had average LoC and the remaining 10 participants had high internal LoC.


% Figure environment removed

\begin{table*}[tb]
\caption{Motion Qualities$^{\dag}$}
\label{tab:motion_results}
\vspace{-3mm}
\begin{center}
\begin{tabular}{llllllll}
\hline
\rule{0pt}{1.1\normalbaselineskip}%
Task & Method & \makecell{Mean Pos. \\ Error (m)} & \makecell{Mean Rot. \\ Error (rad)} & \makecell{Mean Joint \\ Vel. (rad/s)} & \makecell{Mean Joint \\ Acc. (rad/s$^2$)} & \makecell{Mean Joint \\Jerk (rad/s$^3$)} & \makecell{Mean Mani-\\pulability} \\
\hline

\rule{0pt}{1.1\normalbaselineskip}%
%%%%%%%%%%%%%%%%%%%% paste data from python %%%%%%%%%%%%%%%%%%%% 
\multirow{2}{*}{Writing Letters}& \emph{Exact} Mimicry & 0.091$\pm$0.085 & N/A$^{\ddag}$ & 0.133$\pm$0.05 & 1.80$\pm$0.7 & 53.9$\pm$19.5 & 0.067$\pm$0.02 \\   
& \emph{Functional} Mimicry & \textbf{0.006}$\pm$0.009 & N/A$^{\ddag}$ & \textbf{0.076}$\pm$0.04 & \textbf{0.45}$\pm$0.2 & \textbf{10.5}$\pm$5.1 & \textbf{0.085}$\pm$0.02 \\   
\hline 
\rule{0pt}{1.1\normalbaselineskip}%
\multirow{2}{*}{Erasing Lines} & \emph{Exact} Mimicry & 0.093$\pm$0.074 & 0.0204$\pm$0.007 & 0.240$\pm$0.11 & 3.11$\pm$1.8 & 91.1$\pm$56.3 & 0.060$\pm$0.02 \\   
& \emph{Functional} Mimicry & \textbf{0.025}$\pm$0.020 & \textbf{0.0107}$\pm$0.007 & \textbf{0.227}$\pm$0.08 & \textbf{2.23}$\pm$0.9 & \textbf{62.9}$\pm$26.1 & \textbf{0.081}$\pm$0.02 \\   
\hline 
\rule{0pt}{1.1\normalbaselineskip}%
\multirow{2}{*}{Dropping an Envelope}  & \emph{Exact} Mimicry & 0.090$\pm$0.148 & 0.0073$\pm$0.007 & 0.150$\pm$0.05 & 1.42$\pm$0.5 & 40.0$\pm$15.0 & 0.053$\pm$0.03 \\   
& \emph{Functional} Mimicry & \textbf{0.028}$\pm$0.042 & \textbf{0.0070}$\pm$0.004 & \textbf{0.149}$\pm$0.06 & \textbf{1.18}$\pm$0.5 & \textbf{32.3}$\pm$15.7 & \textbf{0.069}$\pm$0.03 \\  
\hline 
\rule{0pt}{1.1\normalbaselineskip}%
\multirow{2}{*}{Carrying a Full Cup}  & \emph{Exact} Mimicry & 0.053$\pm$0.093 & 0.0059$\pm$0.009 & 0.179$\pm$0.09 & 1.63$\pm$0.8 & 45.8$\pm$19.6 & 0.048$\pm$0.02 \\   
& \emph{Functional} Mimicry & \textbf{0.007}$\pm$0.004 & \textbf{0.0021}$\pm$0.001 & \textbf{0.116}$\pm$0.05 & \textbf{0.90}$\pm$0.4 & \textbf{24.3}$\pm$9.3 & \textbf{0.071}$\pm$0.02 \\   
\hline 
\rule{0pt}{1.1\normalbaselineskip}%
\multirow{2}{*}{Pouring Objects} & \emph{Exact} Mimicry & 0.133$\pm$0.144 & 0.0149$\pm$0.008 & 0.218$\pm$0.10 & 2.42$\pm$1.9 & 71.1$\pm$71.5 & 0.035$\pm$0.02 \\   
& \emph{Functional} Mimicry & \textbf{0.064}$\pm$0.065 & \textbf{0.0102}$\pm$0.005 & \textbf{0.158}$\pm$0.06 & \textbf{1.35}$\pm$0.7 & \textbf{35.9}$\pm$20.8 & \textbf{0.068}$\pm$0.02 \\    
\hline 
%%%%%%%%%%%%%%%%%%%%%%%%%%%%%%%%%%%%%%%%%%%%%%%%%%%%%%%%%%%% 
\multicolumn{8}{l}{\rule{0pt}{1\normalbaselineskip}%
$^{\dag}$ The range values are standard deviations. The better value between the two telemanipulation paradigms for each measure is highlighted in bold.} \\
\multicolumn{8}{p{0.95\linewidth}}{\rule{0pt}{1\normalbaselineskip}% 
$^{\ddag}$ The position and rotation errors were measured in the task-relevant degrees of freedom that do not have tolerances. In the writing task, all three rotational degrees of freedom had tolerances, so no rotation errors were measured.} 
\vspace{-5mm}
\end{tabular}
\end{center}
\end{table*}



\begin{table}[tb]
\caption{Statistical results of our measures}
\label{tab:main_results}
\vspace{-3mm}
\begin{center}
\begin{tabular}{l@{\hspace{7pt}}lll@{\hspace{10pt}}l@{\hspace{10pt}}l}
\hline
\rule{0pt}{1.4\normalbaselineskip}%
Metrics & \makecell{\textit{Exact}\\ Mimicry}  & \makecell{\textit{Functional}\\ Mimicry} &  \multicolumn{3}{l}{Statistical Test Results}  \\ 
 & Mean {\scriptsize(SD)} & Mean {\scriptsize(SD)} & $t(19)$ & \makecell[c]{$p$} & \makecell[c]{$d$} \\ 
\hline
\rule{0pt}{1\normalbaselineskip}%
%%%%%%%%%%%%%%%%%%%%%%%%%  Paste here %%%%%%%%%%%%%%%%%%%%%%%%%%%%
Control&4.08 {\scriptsize (1.27)}&\textbf{5.52} {\scriptsize (0.97)}& -4.68&$<.001$&1.29\\
Predictability&3.88 {\scriptsize (1.04)}&\textbf{5.29} {\scriptsize (0.96)}& -4.84&$<.001$&1.41\\
Fluency&3.75 {\scriptsize (1.38)}&\textbf{5.10} {\scriptsize (1.18)}& -3.97&$<.001$&1.05\\
Trust&4.22 {\scriptsize (1.21)}&\textbf{5.25} {\scriptsize (1.09)}& -3.47&$.003$&0.89\\
TLX Overall &48.3 {\scriptsize (18.9)}&\textbf{33.6} {\scriptsize (16.0)}& 4.10&$<.001$&0.84\\
Writing Metric&1.21 {\scriptsize (0.40)}&\textbf{0.75} {\scriptsize (0.38)}& 4.55& $<.001$&1.16 \\
Writing Error&0.72 {\scriptsize (0.46)}&\textbf{0.32} {\scriptsize (0.32)}& 3.29&$.004$&1.01 \\
Erasing Error&0.03 {\scriptsize (0.02)}&\textbf{0.01} {\scriptsize (0.02)}& 3.82&$.001$&1.01 \\
Dropping Metric&\textbf{0.47} {\scriptsize (0.30)}&0.48 {\scriptsize (0.44)}& -0.06&.952&0.02 \\
Pouring Metric&0.99 {\scriptsize (0.77)}&\textbf{0.69} {\scriptsize (0.62)}& 1.24&.230&0.43 \\
Pouring Time&40.1 {\scriptsize (17.1)}&\textbf{35.3} {\scriptsize (13.2)}& 0.88&.391&0.31 \\
\hline
\rule{0pt}{1\normalbaselineskip}%
& Mdn {\scriptsize(IQR)} & Mdn {\scriptsize(IQR)} & W & \makecell[c]{$p$} & \makecell[c]{$r$} \\ 
\hline
\rule{0pt}{1\normalbaselineskip}%
Writing Time&59.2 {\scriptsize (0.88)}&\textbf{48.0}  {\scriptsize (0.10)}& 176&$<.001$&0.84\\
Erasing Metric&1.68 {\scriptsize (0.77)}&\textbf{1.08}  {\scriptsize (0.21)}& 174&$<.001$&0.83\\
Erasing Time&59.5 {\scriptsize (0.71)}&\textbf{59.2}  {\scriptsize (0.4)}& 116&.020&0.61\\
Dropping Time&21.5 {\scriptsize (9.10)}&\textbf{20.8}  {\scriptsize (11.6)}& 28&.622&0.13\\
Dropping Error& 0.00  {\scriptsize (1.00)}& 0.00  {\scriptsize (0.25)}& 4&.809&0.09\\
Carrying Metric&0.32 {\scriptsize (0.37)}&\textbf{0.27}  {\scriptsize (0.26)}& 16&.784&0.08\\
Carrying Time&19.9 {\scriptsize (14.1)}&\textbf{19.8}  {\scriptsize (6.18)}& 64&.245&0.30\\
Carrying Error&3.15 {\scriptsize (3.68)}&\textbf{2.70}  {\scriptsize (5.48)}& 41&.332&0.27\\
Pouring Error&1.50 {\scriptsize (10.0)}&\textbf{0.00}  {\scriptsize (0.25)}& 30&.164&0.45\\
%%%%%%%%%%%%%%%%%%%%%%%%%%%%%%%%%%%%%%%%%%%%%%%%%%%%%%%%%%%%%%%%%%%
\hline

\multicolumn{6}{p{3.15in}}{\rule{0pt}{1\normalbaselineskip}For normally distributed data, we report the mean value with standard deviations, $t$-test results, and Cohen's $d$. For non-normally distributed data, we report the median with interquartile ranges, Wilcoxon signed rank test results (W is the signed-rank sum), and rank biserial correlation coefficient $r$ \cite{king2018statistical}.}
\vspace{-8mm}
\end{tabular}
\end{center}
\end{table}

\subsection{Results}
We first determined whether our data had normal distributions using the Shapiro–Wilk normality test. For each measure, if the result of the Shapiro-Wilk test suggested that the differences between conditions were normally distributed ($p>.05$), we employed the two-tailed paired $t$-test to evaluate the difference between conditions. 
If the \textit{p}-value of the Shapiro-Wilk test was smaller than $.05$, we could not assume normality and used the two-tailed Wilcoxon signed rank test. 
Additionally, we calculated Cohen's $d$ or matched-pairs rank biserial correlation coefficient \cite{king2018statistical} to assess the effect size of normally or non-normally distributed data, respectively. 
Figure \ref{fig: main_results} and Table \ref{tab:main_results} summarize our results. 

\emph{Task Performance}
Our results indicate that in the \textit{Writing} and \textit{Erasing} tasks, the participants performed significantly better (both $p<.001$) in functional mimicry than in exact mimicry with large effect sizes ($d=1.16$ for \textit{Writing} and $r=.83$ for \textit{Erasing}). The effect sizes of the other three tasks were not large enough to lead to statistically significant differences with the sample size in our study. Our results show that the autonomous adjustments in task tolerances do not lead to the detriment of task performance and can improve the performance of some tasks.

\emph{User Perception}
Our results revealed that participants still perceive the robot to be under control even though the robot did not exactly mimic their movements.  
In particular, the robot that exploits task tolerances was perceived to be significantly more under control, predictable, fluent, and trustworthy, and to require significantly lower workload (all $p$-values $<.01$) with large effect sizes (all Cohen's $d$s $>.8$). 

In the post-experiment interview, 18 out of 20 participants stated that they preferred the functional mimicry robot. Participants described functional mimicry in different ways, for example,
P3 commented that ``\textit{I thought that [the functional mimicry robot] compensated more for my movements, which at first I didn't like, but then once I seemed to get a better feel for the compensation, it went more smoothly.}'' 
We speculate that the compensation refers to the autonomous robot adjustments within task tolerances.
Moreover, the subtle adjustments could increase perceived fluency, as described by P11: ``\textit{I felt that [the functional mimicry robot] was more forgiving when I made a mistake. Like it was able to recover and continue with the task without making big adjustments.}'' 

\emph{Motion Qualities} As shown in Table \ref{tab:motion_results}, when functionally mimicking users, the robot generated more accurate (fewer position and rotation errors) and smoother (lower velocities, accelerations, and jerks in the joint space) motions with better manipulability compared to the robot that exactly mimicked users, across all the five tasks. The numerical motion accuracy and smoothness matched the perceived predictability and fluency reported by the participants. Although prior work \cite{wang2023rangedik} has already shown that task tolerances can be exploited to generate high-quality motions, in this experiment, the robot was controlled by a user in real time, in contrast to following a pre-defined trajectory in prior work. Our results indicate that, in mimicry-based telemanipulation, autonomous robot adjustments within task tolerances enabled higher-quality motions, leading to user experience and performance improvements. 



