In this section, we introduce \textit{functional} mimicry in contrast to the commonly-used \textit{exact} mimicry, discuss task tolerances that are exploited by functional mimicry, and describe how \textit{RangedIK} enables functional mimicry. 

\subsection{Mimicry-based Telemanipulation}
\label{sec:mimicry}
A mimicry-based telemanipulation system maps the six degree-of-freedom (DoF) movement of a user’s hand to the robot’s end effector in real time. Such mimicry-control methods have been shown to be intuitive and effective in many applications \cite{rakita2017motion, rolley2018bi}. In this paper, we call these methods \emph{exact} mimicry because the robot \emph{tries} to match \textit{all} six DoFs. In reality, 6-DoF matching is often challenging because of the kinematic and dynamic differences between a robot and its human operator. Moreover, some DoFs are actually task-irrelevant and do not require high accuracy. 

While exact mimicry treats both task-relevant and task-irrelevant DoFs uniformly,  \textit{functional} mimicry deliberately sacrifices the accuracy in task-irrelevant DoFs to improve the accuracy in task-relevant DoFs. Specifically, functional mimicry allows a robot to autonomously move in task-irrelevant DoFs to generate smooth and feasible motions. 

\subsection{Task Tolerance}
Throughout \replaced{this}{the} paper, task tolerances are specified as the \deleted{allowable }amount of position or rotation inaccuracy \added{allowed} to complete a manipulation task. Task tolerances can facilitate task completion, enabling humans or robots to conduct the task without accurately manipulating all six DoFs. We describe task tolerances using the allowable inaccuracy in each DoF (examples are in Table \ref{tab:task_ranges}).

A task with tolerances often involves objects that have rotational symmetry or large effective regions. Rotational symmetry occurs when an object is equivalent under any rotation about a certain axis. Common objects with rotational symmetry include pens, bottles, and bowls. Rotational symmetry simplifies not only the manufacturing process (\textit{e.g.}, using a lathe) but also the usage of an object, \textit{i.e.}, users can manipulate one fewer rotational DoF. The rotational DoF about the rotational symmetry axis has unbounded tolerances. 

Aside from rotational symmetry, large effective areas also create task tolerances. 
The effective parts, or working areas, of many daily objects are designed to be larger than needed, such as the edge of a knife or the aperture of a letter box. Even a pen has a large spherical effective area on the pen tip to ensure enough contact with the writing surface when the pen is tilted. The position or rotational DoFs in the effective areas have bounded tolerances. 

\subsection{Exploiting Task Tolerance}
\label{sec:exploiting}

In this section, we describe how we use \textit{RangedIK} \cite{wang2023rangedik} to exploit task tolerances to enable functional mimicry. To perform a task with tolerances, the DoFs of a robot's end-effector can be classified into three categories: (1) a DoF with zero tolerance requires the robot to accurately match a \textit{specific} goal, \textit{e.g.}, a writing task requires accurate pen tip positions; (2) a DoF with unbounded tolerance provides the robot a \textit{range} of equally valid goals, such as allowing a pen to rotate about its principal axis; and (3) a DoF with bounded tolerance gives the robot a \emph{range} of acceptable goals with a preference toward a \emph{specific} goal, \textit{e.g.}, allowing a tilting pen but preferring it to follow user's commands. 
\textit{RangedIK} is a  real-time optimization-based motion synthesis method that is able to accommodate these three categories of requirements within a single, unified framework. 
With other objectives to avoid self-collision, maintain manipulability, keep joint positions within limits, and minimize joint velocities, accelerations, and jerks, \textit{RangedIK} enables a robot to exploit task tolerances to generate accurate, smooth, and feasible motions. 