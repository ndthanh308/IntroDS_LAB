%%%%%%%%%%%%%%%%%%%%%%%%%%%%%%%%%%%%%%%%%%%%%%%%%%%%%%%%%%%%%%%%%%%%%%%%%%%%%%%%
%2345678901234567890123456789012345678901234567890123456789012345678901234567890
%        1         2         3         4         5         6         7         8

\documentclass[letterpaper, 10 pt, conference]{ieeeconf}  % Comment this line out if you need a4paper

%\documentclass[a4paper, 10pt, conference]{ieeeconf}      % Use this line for a4 paper

\IEEEoverridecommandlockouts                              % This command is only needed if 
                                                          % you want to use the \thanks command

\overrideIEEEmargins                                      % Needed to meet printer requirements.

%In case you encounter the following error:
%Error 1010 The PDF file may be corrupt (unable to open PDF file) OR
%Error 1000 An error occurred while parsing a contents stream. Unable to analyze the PDF file.
%This is a known problem with pdfLaTeX conversion filter. The file cannot be opened with acrobat reader
%Please use one of the alternatives below to circumvent this error by uncommenting one or the other
%\pdfobjcompresslevel=0
%\pdfminorversion=4

% See the \addtolength command later in the file to balance the column lengths
% on the last page of the document

% The following packages can be found on http:\\www.ctan.org
\usepackage{graphics} % for pdf, bitmapped graphics files
%\usepackage{epsfig} % for postscript graphics files
%\usepackage{mathptmx} % assumes new font selection scheme installed
%\usepackage{times} % assumes new font selection scheme installed
\usepackage{amsmath} % assumes amsmath package installed
\usepackage{amssymb}  % assumes amsmath package installed
\usepackage{graphicx}
\usepackage{subfiles}
\usepackage{multirow}
\usepackage{makecell}
\usepackage{tabularx}
\usepackage{soul}
\usepackage{color}
%\usepackage{changes}
\usepackage[final]{changes}
\usepackage{caption}
\usepackage{hyperref}
\usepackage{cleveref}
\usepackage[T1]{fontenc}

\urlstyle{same}

\captionsetup[table]{textfont={sc,footnotesize}, labelfont=footnotesize, labelsep=newline}
\captionsetup[figure]{textfont={footnotesize}, labelfont=footnotesize}

\crefformat{section}{\S#2#1#3} % see manual of cleveref, section 8.2.1
\crefformat{subsection}{\S#2#1#3}
\crefformat{subsubsection}{\S#2#1#3}

\setlength{\skip\footins}{5pt}

\title{\LARGE \bf
Exploiting Task Tolerances in Mimicry-based Telemanipulation
}

\author{Yeping Wang$^{1}$, Carter Sifferman$^{1}$, and Michael Gleicher$^{1}$% <-this % stops a space
\thanks{$^{1}$Yeping Wang, Carter Sifferman, and Michael Gleicher are with the Department of Computer Sciences, University of Wisconsin-Madison, Madison 53706, USA
$\quad \quad \quad \quad \quad \quad \quad \quad \quad \quad \quad \quad \quad \quad \quad \quad $ 
{\tt\small [yeping|sifferman|gleicher]@cs.wisc.edu}}%
\thanks{This work was supported by Los Alamos National Laboratory and \added{the} Department of Energy, a University of Wisconsin Vilas Associates Award, and National Science Foundation award 1830242.}% <-this % stops a space
}

\makeatletter
\let\@oldmaketitle\@maketitle% Store \@maketitle
\renewcommand{\@maketitle}{\@oldmaketitle% Update \@maketitle to insert...
   \vspace{5mm}
    % Figure removed
    \captionof{figure}{We explore how autonomous robot adjustments within task tolerances shape task performance and user experience in mimicry-based telemanipulation. For example, (A) in a teleoperated writing task, our human-subject experiment results indicate that (B) if the robot autonomously tilts or rotates the pen within task tolerances to improve pen tip accuracy and motion smoothness,  telemanipulation is enhanced by the high-quality robot motions enabled by task tolerances, despite users lacking full control of the robot.}
    \label{fig: teaser}
    }
\makeatother

\begin{document}

 \vspace{-8mm}
\maketitle

\addtocounter{figure}{-1} % hack, otherwise following figures start with Fig 3

%%%%%%%%%%%%%%%%%%%%%%%%%%%%%%%%%%%%%%%%%%%%%%%%%%%%%%%%%%%%%%%%%%%%%%%%%%%%%%%%
\begin{abstract}
We explore task tolerances, \textit{i.e.}, allowable position or rotation inaccuracy, as an important resource to facilitate smooth and effective telemanipulation. Task tolerances provide a robot flexibility to generate smooth and feasible motions; however, in teleoperation, this flexibility may make the user's control less direct. In this work, we implemented a telemanipulation system that allows a robot to autonomously adjust its configuration within task tolerances. We conducted a user study comparing a telemanipulation paradigm that exploits task tolerances (\emph{functional} mimicry) to a paradigm that requires the robot to exactly mimic its human operator (\emph{exact} mimicry), and assess how the choice in paradigm shapes user experience and task performance. Our results show that autonomous adjustments within task tolerances can lead to performance improvements without sacrificing perceived control of the robot. Additionally, we find that users perceive the robot to be more under control, predictable, fluent, and trustworthy in functional mimicry than in exact mimicry. 
\end{abstract}

\section{Introduction}
\subfile{1_introduction}

\section{Related Works}
\subfile{2_related_works}

\section{Functional Mimicry}
\subfile{3_task_tolerances}

\section{User Study}
\subfile{4_evaluation}

\section{Discussion}
\subfile{5_discussion}

\bibliography{reference}
\bibliographystyle{IEEEtran}
\end{document}
