Mimicry-based telemanipulation maps a human operator’s hand movement to a robot’s end effector in real time \cite{rakita2017motion, rolley2018bi}. 
The robot is often required to mimic the operator's movement as exactly as possible, so we call this paradigm \textit{exact} mimicry. \replaced{However, this approach may cause the robot to lose manipulability or generate jerky motions b}{B}ecause of the kinematic and dynamic differences between the robot and the operator\deleted{, exact mimicry may cause the robot to lose manipulability or generate jerky motions}. Exact mimicry may be overly \textit{restrictive} for some tasks, imposing unnecessary requirements on the robot.

Many \deleted{everyday} tasks are designed to be accomplished while allowing position or rotation inaccuracy, \textit{i.e.}, tolerances.
For instance, a writing task requires accurate pen tip positions, but does not require the pen to be strictly perpendicular to the surface\replaced{. Similarly, the rotation of a welding torch around its axis does not negatively impact the quality of a welding task \cite{de2017cartesian}. T}{, so wrist movements in writing can naturally rotate or tilt the pen.
Similarly, t}he aperture of a mailbox is generally wider than an envelope, avoiding the need for accurate alignment to drop a letter. 
These task-specific tolerances are naturally utilized by humans. 
In principle, these tolerances also give a robot extra freedom that it can exploit to better execute tasks, \textit{e.g.}, while writing, a robot can autonomously tilt the pen to avoid joint limits or singular configurations. However, in teleoperation, such autonomous adjustments \replaced{mean that the user lacks}{make the user lack} full control of the robot. The less direct control could harm user experience and task performance.

In this paper, we explore task tolerances as an important resource to facilitate \textit{functional} mimicry in telemanipulation. 
The functional mimicry paradigm allows a robot to autonomously adjust within tolerances to generate more accurate, smooth, and feasible motions. 
In our previous work, we presented \textit{RangedIK} \cite{wang2023rangedik} as a real-time motion generation method that exploits flexibility afforded by task tolerances. In this paper, we apply \textit{RangedIK} in a mimicry-based telemanipulation system and investigate whether task performance and user experience will be improved given autonomous robot adjustments within task tolerances. 

We conducted a human-subject experiment in which participants perform tasks using a teleoperation system with or without task tolerance exploitation. Our results indicate that autonomous robot adjustments within task tolerances lead to equal or better performance without sacrificing perceived control of the robot. Moreover, the participants perceived the robot was more under control, predictable, fluent, and trustworthy when it exploited task tolerances than when it exactly followed the users' commands. 

The central contribution of this paper is empirical evidence showing that exploiting the flexibility in task tolerances enables a robot to generate more accurate, smooth, and feasible motions, leading to better task performance and user experience in mimicry-based telemanipulation. 

