
%% format
\usepackage[utf8]{inputenc}
\usepackage{todonotes}
\ifx\preambleEN\undefined
\usepackage[french]{babel}
\else
\usepackage[english]{babel}
\fi
%\uselanguage{French}
%\languagepath{French}
\usepackage{array} % tableaux
\usepackage{xspace} % espace après guillemet français fermant
\usepackage{xcolor} % coloriser les cellules des tableaux (cf https://latex-tutorial.com/tables-in-latex/#color)

% assombrir une zone (texte temporaire/ en construction)
\newenvironment{temp}
    {\color{blue}}
    {}

% commentaires
\newenvironment{commentaire}
    {\color{red}}
    {}

%% maths de base
\usepackage{amsmath,amsthm,amsfonts,amssymb,mathrsfs} % maths
\usepackage{mathtools}
\usepackage{faktor} % groupes quotient etc.
\usepackage{stmaryrd} % composition dans le bon sens (\fatsemi)

%% Théorèmes

% traductions anglaises
\ifx\preambleEN\undefined
\theoremstyle{plain}
\newtheorem{thm}{Théorème}[section]
\newtheorem*{thm*}{Théorème}
\newcommand{\thmautorefname}{théorème}
\newtheorem{prop}[thm]{Proposition}
\newtheorem*{prop*}{Proposition}
\newcommand{\propautorefname}{proposition}
\newtheorem{lemma}[thm]{Lemme} % don't work with beamer
\newtheorem*{lemma*}{Lemme}
\newcommand{\lemmaautorefname}{lemme}
\newtheorem{coro}[thm]{Corollaire}
\newtheorem*{coro*}{Corollaire}
\newcommand{\coroautorefname}{corollaire}

\theoremstyle{definition}
\newtheorem{defi}[thm]{Définition}
\newtheorem*{defi*}{Définition}
\newcommand{\defiautorefname}{définition}

\theoremstyle{remark}
\newtheorem{rmk}[thm]{Remarque}
\newtheorem*{rmk*}{Remarque}
\newcommand{\rmkautorefname}{Remarque}
\newtheorem{example}[thm]{Exemple} % don't work with beamer
\newtheorem*{example*}{Exemple}
\newcommand{\exampleautorefname}{exemple}

\else

\theoremstyle{plain}
\newtheorem{thm}{Theorem}[section]
\newtheorem*{thm*}{Theorem}

\newtheorem{prop}[thm]{Proposition}
\newtheorem*{prop*}{Proposition}

\newtheorem{lemma}[thm]{Lemma} % don't work with beamer
\newtheorem*{lemma*}{Lemma}

\newtheorem{coro}[thm]{Corollary}
\newtheorem*{coro*}{Corollary}


\theoremstyle{definition}
\newtheorem{defi}[thm]{Definition}
\newtheorem*{defi*}{Definition}


\theoremstyle{remark}
\newtheorem{rmk}[thm]{Remark}
\newtheorem*{rmk*}{Remark}

\newtheorem{example}[thm]{Example} % don't work with beamer
\newtheorem*{example*}{Example}

 \newcommand{\thmautorefname}{Theorem}%
 \newcommand{\propautorefname}{Proposition}%
 \newcommand{\lemmaautorefname}{Lemma}%
 \newcommand{\coroautorefname}{Corollary}%
 \newcommand{\defiautorefname}{Definition}%
 \newcommand{\rmkautorefname}{Remark}%
 \newcommand{\exampleautorefname}{Example}%
 \addto\captionsenglish{
  \renewcommand*{\sectionautorefname}{Section}%
  \renewcommand*{\subsectionautorefname}{Subsection}%
  \renewcommand*{\subsubsectionautorefname}{Subsubsection}%
 }
\fi

\makeatletter
\newcommand\RedeclareMathOperator{%
  \@ifstar{\def\rmo@s{m}\rmo@redeclare}{\def\rmo@s{o}\rmo@redeclare}%
}
% this is taken from \renew@command
\newcommand\rmo@redeclare[2]{%
  \begingroup \escapechar\m@ne\xdef\@gtempa{{\string#1}}\endgroup
  \expandafter\@ifundefined\@gtempa
     {\@latex@error{\noexpand#1undefined}\@ehc}%
     \relax
  \expandafter\rmo@declmathop\rmo@s{#1}{#2}}
% This is just \@declmathop without \@ifdefinable
\newcommand\rmo@declmathop[3]{%
  \DeclareRobustCommand{#2}{\qopname\newmcodes@#1{#3}}%
}
\@onlypreamble\RedeclareMathOperator
\makeatother


%% refs et url
%\usepackage[style=alphabetic,sorting=none]{biblatex}
%\addbibresource{src.bib}
%\nocite{*} % affiche la biblio sans citations
\usepackage{hyperref}% déjà chargé par beamer


%% diagrammes et automates
%\usepackage{tikz-cd} % for diagrams
\usepackage{tikz}
\usetikzlibrary{automata,positioning,arrows.meta,cd,babel,shapes}
\usepackage{graphicx} %to rotate \oslash...



%%%%%%%%%%%%%%%%%%%%%%%%%%%
%% Macros et raccourcis %%%
%%%%%%%%%%%%%%%%%%%%%%%%%%%

% Problème avec label + autoref (hyperref) dans amsart
% https://tex.stackexchange.com/questions/372922/hyperlinks-theorems-enumerate-and-autoref
\makeatletter
\newcommand{\amslabel}[1]{%
  \Hy@MakeCurrentHref{\@currenvir.\the\Hy@linkcounter}
  \Hy@raisedlink{\hyper@anchorstart{\@currentHref}\hyper@anchorend}
  \label{#1}% 
}
\makeatother

%% renommage de symboles
\renewcommand{\phi}{\varphi}
\renewcommand{\epsilon}{\varepsilon}

%% théorie des ensembles
\newcommand{\ens}[2]{\left\{#1\middle|#2\right\}} % ensemble avec axiome de compréhension
\newcommand{\nN}{\mathbb{N}} % entiers naturels
\DeclareMathOperator{\zZ}{\mathbb{Z}}
\newcommand{\rR}{\mathbb{R}} % réels
\DeclareMathOperator{\suc}{succ} % successeurs dans les entiers naturels
\newcommand{\point}[1]{\left\{#1\right\}} % singleton ensembliste, "point"
\newcommand\restr[2]{{% restriction de fonction (https://tex.stackexchange.com/questions/22252/how-to-typeset-function-restrictions)
  \left.\kern-\nulldelimiterspace % automatically resize the bar with \right
  #1 % the function
  \vphantom{\big|} % pretend it's a little taller at normal size
  \right|_{#2} % this is the delimiter
  }}

%% géométrie
\newcommand{\cercle}{\mathbb{S}^1}

%% groupes et monoïdes
\DeclareMathOperator{\Grp}{Grp}
\newcommand{\permut}{\mathfrak{S}}
\DeclareMathOperator{\Bij}{Bij}
\newcommand{\noms}{\mathbb{A}}
\DeclareMathOperator{\Supp}{Supp}
\DeclareMathOperator{\Nom}{Nom}
\DeclareMathOperator{\supp}{supp}
\DeclareMathOperator{\Mon}{Mon}
\DeclareMathOperator{\stab}{stab}


%% catégories
\renewcommand{\;}{\fatsemi}
\DeclareMathOperator{\op}{^{op}}
\DeclareMathOperator{\emor}{End}
\DeclareMathOperator{\Aut}{Aut}
\DeclareMathOperator{\imor}{Iso}
\DeclareMathOperator{\hmor}{Hom}
\DeclareMathOperator{\mor}{Mor}
\DeclareMathOperator{\Ima}{Im}
\DeclareMathOperator{\id}{id}
\DeclareMathOperator{\cofree}{Cofree}
\DeclareMathOperator{\free}{Free}
\newcommand{\cC}{\mathcal{C}}
\newcommand{\dD}{\mathcal{D}}
\newcommand{\Bb}{\mathsf{B}}
\newcommand{\Cc}{\mathscr{C}}
\newcommand{\Dd}{\mathscr{D}}
\newcommand{\Ee}{\mathscr{E}}
\newcommand{\Ss}{\mathscr{S}}
\RedeclareMathOperator{\Im}{Im}
\newcommand{\rightarrowdbl}{\rightarrow\mathrel{\mkern-14mu}\rightarrow} % epis
\newcommand{\xrightarrowdbl}[1]{\xrightarrow{}\mathrel{\mkern-14mu}\xrightarrow{#1}}
\newcommand{\leftarrowdbl}{\leftarrow\mathrel{\mkern-14mu}\leftarrow} % monos
\newcommand{\xleftarrowdbl}[1]{\xleftarrow{#1}\mathrel{\mkern-14mu}\xleftarrow{}}
\newcommand{\coslice}[2]{(#1 \downarrow #2)}
\DeclareMathOperator{\colim}{colim}
\newcommand{\textofs}[2]{$(\text{#1},\text{#2})$}
\DeclareMathOperator{\comp}{comp} % morphisme composition
\DeclareMathOperator{\Disc}{Disc} % catégorie discrète
\DeclareMathOperator{\Eq}{Eq}
\DeclareMathOperator{\Top}{Top}
\DeclareMathOperator{\Ind}{Ind}
\DeclareMathOperator{\Et}{\acute{E}t}

%% Objets terminal et initial
\usepackage{dsfont} % for terminal object 1
%\usepackage{bbold} % objets initial et terminal stylisés
\newcommand{\terminal}{\mathds{1}}
\newcommand{\initial}{\emptyset}
\DeclareMathOperator{\Min}{Min}


% Extensions de Kan
\DeclareMathOperator{\Lan}{Lan}
\DeclareMathOperator{\Ran}{Ran}


%% Topos
\DeclareMathOperator{\Ens}{Ens}
\DeclareMathOperator{\EnsFin}{EnsFin}
\DeclareMathOperator{\Set}{Set}
\DeclareMathOperator{\FinSet}{FinSet}
\DeclareMathOperator{\sets}{Set} % topos des ensembles
\DeclareMathOperator{\Const}{LConst} % image inverse du mor géom terminal
\newcommand{\eE}{\mathcal{E}}
\newcommand{\fF}{\mathcal{F}}
\newcommand{\gG}{\mathcal{G}}
\newcommand{\sS}{\mathcal{S}}
\newcommand{\classif}{\Omega}
\newcommand{\complclassif}{2}
\DeclareMathOperator{\Sub}{Sub} % sous-objets
\newcommand{\singleton}[1]{\{\cdot\}_{#1}} % unité de l'adjunction avec l'allégorie / singleton
\DeclareMathOperator{\Rel}{Rel} % catégorie des relations
\DeclareMathOperator{\Par}{Par} % catégorie des ensembles et fonctions partielles
\DeclareMathOperator{\dom}{dom} % domaine d'un morphisme partiel
\DeclareMathOperator{\Psh}{Psh} % préfaisceaux
\DeclareMathOperator{\Sh}{Sh} % faisceaux
\newcommand{\khi}{\chi}
\DeclareMathOperator{\power}{\mathcal{P}}
\DeclareMathOperator{\class}{cl}
\DeclareMathOperator{\ev}{ev} % cartésianité
\DeclareMathOperator{\coev}{coev}
\DeclareMathOperator{\Constr}{Constr} % cat des faisceaux constructibles
\DeclareMathOperator{\Plat}{Plat} % foncteurs plats
\DeclareMathOperator{\Pt}{Pt} % points 
\DeclareMathOperator{\Geom}{Geom}
\DeclareMathOperator{\espcl}{\mathbb{B}}
\newcommand{\nom}[1]{\left\ulcorner #1\right\urcorner}

%% Inclusion de Yoneda
\newcommand{\jo}{\text{\usefont{U}{min}{m}{n}\symbol{'210}}}
\DeclareFontFamily{U}{min}{}
\DeclareFontShape{U}{min}{m}{n}{<-> udmj30}{}
\DeclareMathOperator{\yoneda}{\jo}
\DeclareMathOperator{\adenoy}{\reflectbox{\jo}}


%% théorie de Colcombet-Petrisan
\DeclareMathOperator{\objin}{in}
\DeclareMathOperator{\objout}{out}
\DeclareMathOperator{\objstates}{st}
\DeclareMathOperator{\iword}{\mathcal{I}}
\DeclareMathOperator{\oword}{\mathcal{O}}



%% théories internes
\DeclareMathOperator{\Cat}{\mathcal{C}at}
\DeclareMathOperator{\Cocomp}{\mathcal{C}ocomp}


%% automates
\newcommand{\aA}{\mathcal{A}}
\newcommand{\bB}{\mathcal{B}}
\newcommand{\uaA}{\underline{\mathcal{A}}}
\newcommand{\ubB}{\underline{\mathcal{B}}}
\DeclareMathOperator{\autocat}{Auto}
\DeclareMathOperator{\nautocat}{NAuto}
\DeclareMathOperator{\Syn}{Syn}

%% langages
\newcommand{\lL}{\mathcal{L}}
\newcommand{\kleene}[1]{{#1}^\ast}


%% kan extensions
\DeclareMathOperator{\rightkan}{Ran}
\DeclareMathOperator{\leftkan}{Lan}


%% catégories enrichies
\newcommand{\vV}{\mathcal{V}}
\DeclareMathOperator{\Quiv}{Quiv}


%% ensembles partiellement ordonnés (posets)
\DeclareMathOperator{\Pos}{Pos}


%% Logique
%\DeclareMathOperator{\Const}{Const}
\DeclareMathOperator{\Typ}{Types}
\DeclareMathOperator{\Equ}{\acute{E}qu}
\DeclareMathOperator{\Term}{Termes}
%% plus utilisés
%\usepackage{mathabx} %bigtimes
%\usepackage{mathtools}