
%\subsection{Free enriched categories}

\section{Pointwise enriched Kan extensions}
\label{sec:pw-enrich-kan-ext}

When $\Cc'$ has ``enough'' co/powers, there exists a description of Kan extensions using ends and coends:

$$ \Ran_I(F)(-) = \int_{x:\Cc'}\Cc'(-,I(x))\pitchfork F(x) $$
and
$$ \Lan_I(F)(-) = \int^{x:\Cc'}\Cc'(I(x),-)\odot F(x) $$
and in turn those ends can be computed using co/products, co/equalizers and co/powers as it will be shown in the next lemma.

We introduce some notation. When a $\vV$-category $\Dd$ has powers, then it means we have a $\vV$-natural isomorphism
$$\vV(v,\Dd(d,d')) \cong \Dd(d,v\pitchfork d')$$
in $v$, $d$ and $d'$, which restricts to a natural isomorphism
$$\vV_0(v,\Dd(d,d'))\cong \Dd_0(d,v\pitchfork d')$$
and we have the dual facts for copowers; by definition we have a $\vV$-natural isomorphism
$$\vV(v,\Dd(d,d')) \cong \Dd(v\odot d,d')$$
in $v$, $d$ and $d'$, which restricts to a natural isomorphism
$$\vV_0(v,\Dd(d,d'))\cong \Dd_0(v\odot d,d')$$
As an important remark, if $\vV$ can be considered as a $\vV$-category (meaning that it is monoidal closed) then
it has powers given by $v\pitchfork w = [v,w]$ and copowers given by $v \odot w = v \otimes w$.

This Kelly \cite[Lemma 3.68]{kellyBasicConceptsEnriched1982} gives us a way to explicitly compute pointwise Kan extensions.
\begin{lemma}\amslabel{lem:end-existence}
  Let $\Cc$ and $\Dd$ be $\vV$-categories for $\vV$ a symmetric monoidal closed category
  and let $P:\Cc\op\otimes\Cc\xrightarrow{}\Dd$ be a $\Dd$-valued $\vV$-distributor on $\Cc$ (i.e.\ a $\vV$-functor).

  If the following powers of $\Dd$, conical products and conical equalizer exist,
  then the end of $P$ exists, and is the equalizer of the diagram:
  \begin{center}
    \begin{tikzcd}[ampersand replacement=\&]
      {} \& {} \& {\prod_{c\in\Cc_0}P(c,c)} \&\&\& {\prod_{(c,c')\in(\Cc_0)^2}\Cc(c,c')\pitchfork P(c,c')}
      \arrow["{\phi}", shift left=2, from=1-3, to=1-6]
      %(\prod_{c\in\Cc_0}P(c,c)\xrightarrow{\pi_c}P(c,c)\xrightarrow{(P(c,-)_{c,c'})^\dashv}\cC(c,c')\pitchfork P(c,c'))_{(c,c')\in(\cC_0)^2}
      \arrow["{\psi}"', shift right=2, from=1-3, to=1-6]
      %(\prod_{c\in\Cc_0}P(c,c)\xrightarrow[\pi_{c'}]{}P(c',c')\xrightarrow[(P(-,c')_{c,c'})^\dashv]{}\cC(c,c')\pitchfork P(c,c'))_{(c,c')\in(\cC_0)^2}
    \end{tikzcd}
  \end{center}
  where the top arrow (of $\Dd_0$) is defined, at component $(c,c')\in(\Cc_0)^2$, as the composite
  $$\prod_{c\in\Cc_0}P(c,c)\xrightarrow{\pi_c}P(c,c)\xrightarrow{(P(c,-)_{c,c'})^\dashv}\Cc(c,c')\pitchfork P(c,c')$$
  where $(P(c,-)_{c,c'})^\dashv$ is the adjunct of the following morphism of $\vV_0$:
  $$\Cc(c,c')\cong^{\lambda^{-1}}I\otimes\Cc(c,c')\xrightarrow{\id^\Cc_c\otimes\Cc(c,c')}\Cc(c,c)\otimes\Cc(c,c')
  \xrightarrow{P_{(c,c),(c,c')}}\Dd(P(c,c),P(c,c'))$$
  that we denote $P(c,-)_{c,c'}$. The bottom arrow of $\Dd_0$ is given at component $(c,c')\in(\Cc_0)^2$ by the adjunct
  $$P(c',c')\xrightarrow{(P(-,c')_{c,c'})^\dashv}\Cc(c,c')\pitchfork P(c,c')$$
  of the morphism of $\vV_0$
  $$\Cc(c,c')\cong^{\rho^{-1}}\Cc(c,c')\otimes I\xrightarrow{\Cc(c,c')\otimes\id^\Cc_{c'}}\Cc(c,c')\otimes\Cc(c',c')
  \xrightarrow{P_{(c',c'),(c,c')}}\Dd(P(c',c'),P(c,c'))$$
  that we denote $P(-,c')_{c,c'}$.

  Dually, if the following copowers of $\Dd$, (conical) coproducts and (conical) coequalizer exist,
  then the coend of $P$ exists, and is the coequalizer of the diagram:
  \begin{center}
    \begin{tikzcd}[ampersand replacement=\&, outer sep=3pt]
      {} \& {} \& {\sum_{c\in\Cc_0}P(c,c)} \&\&\& {\sum_{(c,c')\in(\Cc_0)^2}\Cc(c,c')\odot P(c,c')}
      \arrow["{(\sum_{c\in\Cc_0}P(c,c)\xleftarrow{\i_c}P(c,c)\xleftarrow{(P(c,-)_{c,c'})^\vdash}\Cc(c,c')\odot P(c,c'))_{(c,c')\in(\Cc_0)^2}}"', shift right=2, from=1-6, to=1-3]
      \arrow["{(\sum_{c\in\Cc_0}P(c,c)\xleftarrow[\i_{c'}]{}P(c',c')\xleftarrow[(P(-,c')_{c,c'})^\vdash]{}\Cc(c,c')\odot P(c,c'))_{(c,c')\in(\Cc_0)^2}}", shift left=2, from=1-6, to=1-3]
    \end{tikzcd}
  \end{center}
  where this time, the adjuncts of the morphisms $P(c,-)_{c,c'}$ and $P(-,c')_{c,c'}$ of $\Dd_0$ have to be understood with respect
  to the copowers.
\end{lemma}

\begin{proof}
  Sketch of proof: the formulas can be shown to be true for $\Dd=\vV$ using the definition of an end.
  Now for $\Dd$-valued distributors, $e\in\Dd_0$ is the end of $P$ if there is an isomorphism
  $$\Dd(d,\int_{c:\Cc}P(c,c))\cong\int_{c:\Cc}\Dd(d,P(c,c))$$
  natural in $d$. Now
  \begin{align*}
    \Dd(d,\Eq(\phi,\psi))
    &\cong \Eq(\Dd(d,\phi),\Dd(d,\psi))
  \end{align*}
  where the equalizer on the right is the one of the lemma for the $\vV$-distributor $\Dd(d,P(-,=))$.
  Indeed, by continuity of $\Dd(d,-)$,
  $$\prod_{c\in\Cc_0}\Dd(d,P(c,c))\xrightarrow{\Dd(d,\phi)}
  \prod_{c,c'\in\Cc_0}\Dd(d,\Cc(c,c')\pitchfork P(c,c'))$$
  is given at component $(c,c')$ by
  $$\prod_{c\in\Cc_0}\Dd(d,P(c,c))\xrightarrow{\pi_c}\Dd(d,P(c,c))
  \xrightarrow{\Dd(d,(P(c,-)_{c,c'})^\dashv)}\Dd(\Cc(c,c')\pitchfork P(c,c'))$$
  which because of powering is in fact
  $$\prod_{c\in\Cc_0}\Dd(d,P(c,c))\xrightarrow{\pi_c}\Dd(d,P(c,c))
  \xrightarrow{\Dd(d,(P(c,-)_{c,c'}))^\dashv}\Dd(\Cc(c,c')\pitchfork P(c,c'))$$
  therefore $\Eq(\Dd(d,\phi),\Dd(d,\psi))$ is the end of $\Dd(d,P(-,=))$.
\end{proof}

% [Pointwise Kan extensions are ``local'' Kan extensions]
\begin{lemma}\amslabel{lem:pt-kan-are-loc-kan}
  Consider a span of $\vV$-functors $\Cc'\xleftarrow{H}\Cc\xrightarrow{F}\Dd$.
  If the pointwise left (resp. right) Kan extension $(\Lan_H F,\lambda:F\xRightarrow{}H^\ast\Lan_H F)$ (resp. $(\Ran_H F,\rho:H^\ast\Ran_H F\xRightarrow{}F)$) exists
  then we have a $\vV$-natural isomorphism (in $G$)
  $$[\Lan_H F,G]_\vV\cong[F,HG]_\vV$$
  and on the unenriched side, the unenriched natural isomorphism is given by
  $$(\alpha : \Lan_H F \xRightarrow{} G) \xmapsto{} (F\xRightarrow{\lambda}H\Lan_H F\xRightarrow{H\ast\alpha}HG)$$
  Respectively, for the pointwise right Kan extension,
  $$[G,R]_\vV\cong[HG,F]_\vV$$
\end{lemma}

\begin{proof}
  We do it for the left Kan extension, the same arguments apply dually for the right Kan extension.
  \begin{align*}
    [\Lan_H F,G]_\vV
    &\cong [\int^{c:\Cc}\Cc'(H(c),-)\odot F(c),G]_\vV \quad \text{by definition of L} \\
    &\cong \int_{c:\Cc}[\Cc'(H(c),-)\odot F(c),G]_\vV \quad \text{by cocontinuity of the }\hom \\
    &\cong \int_{c:\Cc}\int_{c'\in\Cc'}\Dd(\Cc'(H(c),c')\odot F(c),G(c')) \quad \text{by definition of $\vV\Cat$ as a $\vV$-category} \\
    &\cong \int_{c:\Cc}\int_{c'\in\Cc'}[\Cc'(H(c),c'),\Dd(F(c),G(c'))] \quad \text{by definition of the copower} \\
    &\cong \int_{c:\Cc}\Dd(F(c),G(H(c))) \quad \text{by ninja Yoneda lemma on } \Dd(F(c),G(-)) \\
    &\cong [F,HG]_\vV\quad \text{by definition}\\
  \end{align*}
\end{proof}

This following crucial lemma shows that in the case we consider a Kan extension along a full subcategory inclusion, then the Kan extension is a ``real''
extension.

\begin{lemma}\amslabel{lem:enriched-kan-ext-along-fully-faithful}
  Consider a span of $\vV$-functors $\Cc'\xhookleftarrow{H}\Cc\xrightarrow{F}\Dd$, such that $H$ is fully faithful\footnote{recall that is the enriched sense, it means that $H_{c,d}:\Cc(c,d)\xrightarrow{}\Cc'(H(c),H(d))$
    are isomorphisms in $\vV$ for all pairs of objects $(c,d)$ of $\Cc$.}.
  If the pointwise left (resp. right) Kan extension $\lambda:F\xRightarrow{}HL$ (resp. $\rho:HR\xRightarrow{}F$) exists,
  then $\lambda$ (resp. $\rho$) is in fact an isomorphism.
\end{lemma}

\begin{proof} %https://math.stackexchange.com/questions/892326/nice-proof-that-the-unit-of-the-left-kan-extension-of-f-is-an-isomorphism-i
  We give a sketch of proof based on the proof of Kelly \cite[Proposition 4.23]{kellyBasicConceptsEnriched1982}, but for the right Kan extension,
  and using only ends.

  First consider the Yoneda $\vV$-functor $\Cc(-_1,-_2):\Cc\op\xrightarrow{}[\Cc,\vV]_\vV$, and the composite $\vV$-functor
  $\Cc'(H(-_1),H(-_2)):\Cc\op\xrightarrow{}[\Cc,\vV]_\vV$. Then
  $$H_{-_1,-_2}:\Cc(-_1,-_2)\xRightarrow{}\Cc'(H(-_1),H(-_2))$$
  defined componentwise by
  $$H_{c,c'}:\Cc(c,c')\xrightarrow{}\Cc'(H(c),H(c'))$$
  is $\vV$-natural, and is an isomorphism if and only if $H$ is fully faithful, because by definition $H$ is fully faithful
  if and only if $H_{c,c'}:\Cc(c,c')\xrightarrow{}\Cc'(H(c),H(c'))$ is an isomorphism for all $(c,c')\in(\Cc_0)^2$.

  Now by Yoneda's lemma, there is a $\vV$-natural isomorphism
  $$\int_{c:\Cc}\Cc(-,c)\pitchfork F(c)\xRightarrow{y} F(-)$$
  and one may check that the counit $\rho$ of the right Kan extension is in fact
  % (remember that powering $(-_1)\pitchfork(-_2)$ is contravariant in the first variable)
  \begin{center}
    \begin{tikzcd}[ampersand replacement=\&]
      {(H\Ran_HF)(c)=\int_{c':\Cc}\Cc'(H(c),H(c'))\pitchfork F(c')} \&\& {F(c)} \\
      {\int_{c':\Cc}\Cc(c,c')\pitchfork F(c')}
      \arrow["{\rho_c}", from=1-1, to=1-3]
      \arrow["{\int_{c':\Cc}H_{c,c'}\pitchfork F(c')}"', from=1-1, to=2-1]
      \arrow["{y_c}"', from=2-1, to=1-3]
    \end{tikzcd}
  \end{center}
  Now because $y$ is a $\vV$-natural isomorphism, if $H$ is fully faithful, then $H_{-_1,-_2}$ is a $\vV$-natural isomorphism and
  therefore so is $\int_{c':\Cc}H_{c,c'}\pitchfork F(c')$ and finally so is $\rho$.
\end{proof}

\section{Free $\eE$-categories for a bicomplete elementary topos $\eE$}
\label{sec:free-e-categories}

\begin{defi}
  Let $\vV=(\vV_0,\otimes, I)$ be a closed symmetric monoidal category, a (small) $\vV$-quiver $Q=(Q_0,Q(-,=))$
  is given by a set \emph{of vertices} $Q_0$ and for all $(x,y)\in Q^2$, an object $Q(x,y)$ \emph{of edges} of $\vV_0$.
  % Equivalently, it is a functor from the square of a (small) discrete category to $\vV_0$.
  % With this last description we define the category $\vV\Quiv$ of $\vV$-quivers : the morphisms
  % are collections $f_0,(f_{a,b})$ such that
  % % https://q.uiver.app/?q=WzAsNCxbMCwwLCJcXERpc2MoVilee1xcdGltZXMgMn0iXSxbMSwwLCJcXERpc2MoVylee1xcdGltZXMgMn0iXSxbMCwxLCJcXHZWXzAiXSxbMSwxLCJcXHZWXzAiXSxbMCwyLCJRIiwyXSxbMSwzLCJSIl0sWzIsMywiXFxpZF97XFx2Vl8wfSIsMl0sWzAsMSwiKGZfMCxmXzApIl0sWzQsNSwiZl97LSw9fSIsMix7InNob3J0ZW4iOnsic291cmNlIjoyMCwidGFyZ2V0IjoyMH19XV0=
  % \[\begin{tikzcd}
  %     {\Disc(V)^{\times 2}} & {\Disc(W)^{\times 2}} \\
  %     {\vV_0} & {\vV_0}
  %     \arrow[""{name=0, anchor=center, inner sep=0}, "Q"', from=1-1, to=2-1]
  %     \arrow[""{name=1, anchor=center, inner sep=0}, "R", from=1-2, to=2-2]
  %     \arrow["{\id_{\vV_0}}"', from=2-1, to=2-2]
  %     \arrow["{(f_0,f_0)}", from=1-1, to=1-2]
  %     \arrow["{f_{-,=}}"', shorten <=11pt, shorten >=11pt, Rightarrow, from=0, to=1]
  %   \end{tikzcd}
  % \]
  % and composition is done by gluing the squares, so that
  % $$f;g=(f_0;g_0,f_{-,=};\Disc(V)^{\times 2}(g_0)\ast g_{-,=})$$
  
  A \emph{$\vV$-quiver morphism} $f:Q\xrightarrow{} R$ is a collection $(f,(f_{x,y})_{(x,y)\in Q^2})$ where $f$ is
  a function from $Q_0$ to $R_0$ and $f_{x,y}$ a morphism of $\vV_0$ from $Q(x,y)$ to $R(f(x),f(y))$.
\end{defi}

\begin{defi}
  Let $\vV=(\vV_0,\otimes, I)$ be a symmetric closed monoidal category, we say $\vV$ admits free $\vV$-categories if the \emph{forgetful} (unenriched) functor $$\vV\Cat \ni \Cc \xmapsto{} (\Cc_0,\Cc(-,=)) \in \vV\Quiv$$ has a left adjoint.
\end{defi}

%\subsubsection{Free $\eE$-categories for a co/complete elementary topos $\eE$}

\begin{prop}
  A bicomplete elementary topos $\eE$ admits free $\eE$-categories.
\end{prop}

\begin{proof}
  The free $\eE$-category $\Cc$ generated by the $\eE$-quiver $Q$ has for set of objects $Q_0$ and for all $x,y\in Q_0$, $$\Cc(x,y)=\sum_{n\in \nN}\sum_{x_1,x_2,\dots,x_{n-1}\in V_0}Q_0(x,x_1)\times Q_0(x_1,x_2)\times Q_0(x_2,x_3)\times \cdots \times Q_0(x_{n-1},y)$$ so that the identity of $x$ is given by the coproduct injection (for $n=0$, and path $(x)$) $\terminal \xhookrightarrow{}Q(x,x)$ and composition works because sums distribute over products in this setting (a topos being an extensive category).
  
  Now the adjunction itself. Let $f$ be an $\eE$-quivers morphism between a quiver $Q$ and an $\eE$-category $\Dd$ seen as an $\eE$-quiver, and denote $\Cc$ the free $\eE$-category generated by $Q$. Then the adjunct of $f$ is an $\eE$-functor $F$ such that $F_0=f_0:\Cc_0\xrightarrow{}\Dd_0$ where $\Cc_0=Q_0$ by definition. Now for all $x,y\in\cC_0$, $$\Cc(x,y)=\sum_{n\in \nN}\sum_{x_1,x_2,\dots,x_{n-1}\in \cC_0}Q(x,x_1)\times Q(x_1,x_2)\times Q(x_2,x_3)\times \cdots \times Q(x_{n-1},y)$$ so $F_{x,y}:\Cc(x,y)\xrightarrow{}\Dd(F_0(x),F_0(y))$ is defined using the universal property of the coproduct: for all $n\in\nN, x_1,x_2,\dots,x_{n-1}\in\cC_0$, $\restr{F_{x,y}}{\prod_{i=0}^{n-1}Q(x_i,x_{i-1})}$ is defined by
  \begin{center}
    % https://q.uiver.app/?q=WzAsMyxbMCwwLCJcXHByb2Rfe2k9MH1ee24tMX1RKHhfaSx4X3tpLTF9KSJdLFsxLDEsIlxccHJvZF97aT0wfV57bi0xfVxcZEQoRl8wKHhfaSksRl8wKHhfe2ktMX0pKSJdLFsyLDAsIlxcZEQoRl8wKHgpLEZfMCh5KSkiXSxbMCwyLCJcXHJlc3Rye0Zfe3gseX19e1xccHJvZF97aT0wfV57bi0xfVEoeF9pLHhfe2ktMX0pfSJdLFswLDEsIlxccHJvZF97aT0wfV57bi0xfWZfe3hfaSx4X3tpLTF9fSIsMl0sWzEsMiwiXFxjb21wX3t4LHhfMSx4XzIsXFxkb3RzLHhfe24tMX0seX0iLDJdXQ==
\[\begin{tikzcd}
	{\prod_{i=0}^{n-1}Q(x_i,x_{i-1})} && {\dD(F_0(x),F_0(y))} \\
	& {\prod_{i=0}^{n-1}\dD(F_0(x_i),F_0(x_{i-1}))}
	\arrow["{\restr{F_{x,y}}{\prod_{i=0}^{n-1}Q(x_i,x_{i-1})}}", from=1-1, to=1-3]
	\arrow["{\prod_{i=0}^{n-1}f_{x_i,x_{i-1}}}"', from=1-1, to=2-2]
	\arrow["{\comp_{F_0(x),F_0(x_1),\dots,F_0(x_{n-1}),F_0(y)}}"', from=2-2, to=1-3]
      \end{tikzcd}\]
  \end{center}
  where $\comp_{x,x_1,x_2,\dots,x_{n-1},y}$ is the iteration of composition of the $\eE$-category $\Dd$ (it can be defined in different manners thanks to associativity), with the convention that for $n=0$, $\comp_y:\terminal\xrightarrow{}\Dd(y,y)$ is the identity of $y$. The newly defined $F$ preserves identities because by definition, the identity of $x\in\Cc_0$ is the global element corresponding to the coproduct injection for $n=0,x=x$. $F$ preserves it because of the convention that $\comp_{F_0(x)}$ is the identity of $F_0(x)$. It preserves composition of morphisms by definition of composition in the free category $\Cc$.
\end{proof}

% \begin{temp}
%   \subsubsection{Free $\vV$-categories for a Bénabou cosmos $\vV$}

%   Recall that a Bénabou cosmos (see Street) is a monoidal, symmetric closed category $\vV=(\vV_0,\otimes,I,\alpha,\lambda,\rho)$
%   such that $\vV_0$ the underlying category, is both complete and cocomplete.

%   We should be able to have free $\vV$-categories with way less hypothesises on the enriching monoidal category $\vV$, but in our case,
%   a co/complete elementary topos $\eE$ can be seen as a Bénabou cosmos, so this is sufficent.

%   Free categories are generated by quivers, free $\vV$-categories are generated by $\vV$-quivers.
%   \begin{defi}
%     The discrete $\vV$-category $D(V)$ on a set (or even class) $V$ has set (or class) of objects $D(V)_0=V$ and morphisms objects
%     $D(V)(u,v)=I$ if $u=v$ and $D(V)(u,v)=\emptyset$ if $u\neq v$. For any triple $(u,v,w)\in V^3$ of objects, composition morphisms
%     $\comp_{u,v,w}:D(V)(u,v)\otimes D(V)(v,w)\xrightarrow{} D(V)(u,w)$ can only be of one of the following four forms :
%     \begin{center}
%       \begin{tabular}{|c|c|c|}
%         \hline
%         Relation between $u$ and $w$, & $v=w$ & $v\neq w$ \\
%         and definition of $\comp_{u,v,w}$ & & \\
%         \hline
%         $u=v$ & then $u=w$,                                       & then $u\neq w$, \\
%                                       & $I\otimes I \xrightarrow{\lambda_{I}^{-1}} I$ & $I\otimes \initial \cong \initial$ \\
%         \hline
%         $u\neq v$ & then $u\neq w$,                        & if $u=w$, \\
%                                       & $\initial\otimes I\cong \initial$ & $\initial\otimes \initial\cong \initial \xrightarrow{!}I$\\
%                                       &                                   & if $u\neq w$, \\
%                                       &                                   & $\initial\otimes \initial\cong \initial \xrightarrow{\id_{\initial}}I$\\
%         \hline
%       \end{tabular}
%     \end{center}
%   \end{defi}

%   Here, $\initial$ denotes the initial object of $\vV_0$, and the isomorphisms of the antidiagonal of the table are given by the fact that,
%   because $\vV$ is symmetric closed, $-\otimes I$ and $I\otimes -$ have left adjoints, therefore preserves colimits, and in particular the initial
%   object. Same argument for the down, right corner.
%   The fact that it defines a $\vV$-category is because of the coherence diagrams satisfied by the left unitor and because of the universal property
%   of the initial object.

%   \begin{defi}
%     The $\vV$-category of (small) $\vV$-quivers is the coproduct $\sum_{V\in\sets}[D(V)^2,\vV]$ of $\vV$-categories of $\vV$-functors.
%     Spelt out, a $\vV$-quiver is the data of a set $Q_0$ of \emph{vertices} and for every pair $(u,v)$ of vertices, an object of \emph{edges} $Q(u,v)$
%     of $\vV$. A $\vV$-quiver morphism 
%   \end{defi}
% \end{temp}

% \subsection{Enriched structure of an elementary topos}
% \label{sec:enrich-struct-of-topos}

% Fix an elementary topos $\eE$. We precisely describe the $\eE$-category structure of $\eE$.
% \begin{description}
% \item[Objects]
% the objects of $\eE$.

% \item[Objects of morphisms]
% between $A$ and $B$, $\hom(A,B)=B^A$

% \item[Identities]
% the identity of $A$ of $\eE$ is the adjunct of the identity of $\eE$ underlying : $\id_A^\eE:\terminal\xrightarrow{}A^A$.

% \item[Compositions]
% the composition
% $$\comp_{A,B,C} : B^A \times C^B \xrightarrow{} C^A$$
% is the $(A\times- \dashv (-)^A)$-adjunct of
% $$A\times B^A \times C^B \xrightarrow{\ev^A_B\times C^B} B \times C^B \xrightarrow{\ev^B_C} C$$

% \end{description}

% We check the following axioms are satisfied:
% \begin{description}

% \item[Associativity of composition]

% We want to show

% \begin{center}
% % https://q.uiver.app/?q=WzAsNSxbMCwwLCIoQl5BXFx0aW1lcyBDXkIpXFx0aW1lcyBEXkMiXSxbMCwyLCJCXkFcXHRpbWVzIChDXkJcXHRpbWVzIEReQykiXSxbMywwLCJDXkFcXHRpbWVzIEReQyJdLFszLDIsIkJeQVxcdGltZXMgRF5CIl0sWzQsMSwiRF5BIl0sWzAsMSwiXFxhbHBoYV97Ql5BLENeQixEXkN9Il0sWzAsMiwiXFxjb21wX3tBLEIsQ31cXHRpbWVzIEReQyJdLFsxLDMsIkJeQVxcdGltZXMgXFxjb21wX3tCLEMsRH0iXSxbMyw0LCJcXGNvbXBfe0EsQixEfSIsMl0sWzIsNCwiXFxjb21wX3tBLEMsRH0iXV0=
%   \begin{tikzcd}[ampersand replacement=\&]
%     {(B^A\times C^B)\times D^C} \&\&\& {C^A\times D^C} \\
%     \&\&\&\& {D^A} \\
%     {B^A\times (C^B\times D^C)} \&\&\& {B^A\times D^B}
%     \arrow["{\alpha_{B^A,C^B,D^C}}", from=1-1, to=3-1]
%     \arrow["{\comp_{A,B,C}\times D^C}", from=1-1, to=1-4]
%     \arrow["{B^A\times \comp_{B,C,D}}", from=3-1, to=3-4]
%     \arrow["{\comp_{A,B,D}}"', from=3-4, to=2-5]
%     \arrow["{\comp_{A,C,D}}", from=1-4, to=2-5]
%   \end{tikzcd}
% \end{center}
% commute, where $\alpha$ is the associator of the cartesian monoidal structure.
% \end{description}