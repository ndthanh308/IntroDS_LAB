%!TEX root = ../Schur indices and line operators.tex
\subsection{\texorpdfstring{$\mathcal{N}=4$ $SO(7)$ SYM}{}}


The Schur index of the $\mathcal{N} = 4$ $SO(7)$ theory is a contour integral of the following integrand,
\begin{align}
\mathcal{Z}\left(a_1,a_2,a_3\right)=\left(\frac{\vartheta_1^{\prime}(0)}{\vartheta_4(0)}\right)^3\prod_{\alpha,\beta}\prod_{i<j}\frac{\vartheta_1(\alpha\mathfrak{a}_i+\beta\mathfrak{a}_j,q)}{\vartheta_4(\alpha\mathfrak{a}_i+\beta\mathfrak{a}_j+\mathfrak{b},q)}\prod_{\alpha}\prod_{i=1}^3 \frac{\vartheta_1(\alpha\mathfrak{a}_i,q)}{\vartheta_4(\alpha\mathfrak{a}_i+\mathfrak{b},q)} \ ,
\end{align}
which is separately elliptic with respect to all three variables $\mathfrak{a}_{1,2,3}$.

The integral can be performed analytically by integrating $a_1, a_2, a_3$ one after another using the integration formula collected in the appendix \ref{app:integration-formula}. The $a_1$ integration involves the following simple poles which are all imaginary,
\begin{align}
\alpha\mathfrak{b}+\frac{\tau}{2},
\quad \alpha\mathfrak{a}_2+\beta\mathfrak{b}+\frac{\tau}{2},
\quad \alpha\mathfrak{a}_3+\beta\mathfrak{b}+\frac{\tau}{2} \ .
\end{align}
The residues of these simple poles are denoted by $\mathcal{P}_{\alpha}$, $\mathcal{Q}_{\alpha\beta}$ and $\tilde{\mathcal{Q}}_{\alpha\beta}$.
%\begin{align}
%\mathcal{P}_{\alpha}= & \ \frac{\alpha\left(\vartheta_1^\prime(0)\right)^2}{\vartheta_1(2\mathfrak{b})\vartheta_4(\mathfrak{b})}\prod_{\gamma}\frac{\vartheta_1^2\left(\mathfrak{a}_2+\gamma\mathfrak{a}_3\right)\vartheta_1\left(\mathfrak{a}_2+\gamma\mathfrak{b}\right)\vartheta_1\left(\mathfrak{a}_3+\gamma\mathfrak{b}\right)}{\vartheta_1\left(\mathfrak{a}_2+2\gamma\mathfrak{b}\right)\vartheta_1\left(2\mathfrak{b}+\gamma\mathfrak{a}_3\right)\vartheta_4\left(\mathfrak{a}_2-\mathfrak{a}_3+\gamma\mathfrak{b}\right)\vartheta_4\left(\mathfrak{a}_2+\mathfrak{a}_3+\gamma\mathfrak{b}\right)}  \ , \nonumber\\
%\mathcal{Q}_{\alpha\beta}= & \ -\frac{\vartheta_1^\prime(0)^2}{\vartheta_4(\mathfrak{b})}\frac{\vartheta_1(\alpha\mathfrak{a}_2)\vartheta_4(\beta\mathfrak{b}+\alpha\mathfrak{a}_2)\vartheta_4^2(\beta\mathfrak{b}+2\alpha\mathfrak{a}_2)}{\vartheta_1(2\beta\mathfrak{b})\vartheta_1(2\alpha\mathfrak{a}_2)\vartheta_1(2\beta\mathfrak{b}+\alpha\mathfrak{a}_2)\vartheta_1(2\beta\mathfrak{b}+2\alpha\mathfrak{a}_2)\vartheta_4(-\beta\mathfrak{b}+\alpha\mathfrak{a}_2)}\notag\\
%&\times\prod_{\gamma=\pm}\frac{\vartheta_1(\mathfrak{a}_2)\vartheta_1(\mathfrak{a}_2+\gamma\mathfrak{a}_3)\vartheta_4(\beta\mathfrak{b}+\alpha\mathfrak{a}_3+\gamma\mathfrak{a}_2)}{\vartheta_1(2\beta\mathfrak{b}+\alpha\mathfrak{a}_3+\gamma\mathfrak{a}_2)\vartheta_4(\beta\mathfrak{b}-\alpha\mathfrak{a}_3+\gamma\mathfrak{a}_2)\vartheta_4(\mathfrak{a}_2+\gamma\mathfrak{b})} \ , \nonumber\\
%\tilde{\mathcal{Q}}_{\alpha\beta} = & \ -\frac{\vartheta_1^\prime(0)^2}{\vartheta_4(\mathfrak{b})}\frac{\vartheta_1(\alpha\mathfrak{a}_3)\vartheta_4(\beta\mathfrak{b}+\alpha\mathfrak{a}_3)\vartheta_4^2(\beta\mathfrak{b}+2\alpha\mathfrak{a}_3)}{\vartheta_1(2\beta\mathfrak{b})\vartheta_1(2\alpha\mathfrak{a}_3)\vartheta_1(2\beta\mathfrak{b}+\alpha\mathfrak{a}_3)\vartheta_1(2\beta\mathfrak{b}+2\alpha\mathfrak{a}_3)\vartheta_4(-\beta\mathfrak{b}+\alpha\mathfrak{a}_3)} \nonumber\\
%& \times\prod_{\gamma=\pm}\frac{\vartheta_1(\mathfrak{a}_3)\vartheta_1(\mathfrak{a}_3+\gamma\mathfrak{a}_2)\vartheta_4(\beta\mathfrak{b}+\alpha\mathfrak{a}_2+\gamma\mathfrak{a}_3)}{\vartheta_1(2\beta\mathfrak{b}+\alpha\mathfrak{a}_2+\gamma\mathfrak{a}_3)\vartheta_4(\beta\mathfrak{b}-\alpha\mathfrak{a}_2+\gamma\mathfrak{a}_3)\vartheta_4(\mathfrak{a}_3+\gamma\mathfrak{b})}  \ .
%\end{align}
Using the integration formula (\ref{integration-formula-f}), the $a_1$ integration leaves an integrand
\begin{align}
\mathcal{Z}_1(a_2,a_3)
= & \ \oint_{|a_1|=1}\frac{da_1}{2\pi i a_1}\mathcal{Z}(a_1,a_2,a_3) \nonumber\\
= & \ \sum_{\alpha}\mathcal{P}_\alpha E_1\begin{bmatrix}
-1\\
b^\alpha
\end{bmatrix}+\sum_{\alpha,\beta}\mathcal{Q}_{\alpha\beta}E_1\begin{bmatrix}
-1\\
a_2^{\alpha}b^{\beta}
\end{bmatrix}+\sum_{\alpha,\beta}\tilde{\mathcal{Q}}_{\alpha\beta}E_1\begin{bmatrix}
-1\\
a_3^{\alpha}b^\beta \
\end{bmatrix}.
\end{align}
The poles and residues of $\mathcal{P}, \mathcal{Q}, \tilde {\mathcal{Q}}$ are listed in Table \ref{poles-residues-SO(7)}, which are used in the $a_2$-integration.
{
	\renewcommand{\arraystretch}{1.8}
	\begin{table}[h!]
		\centering
		\begin{tabular}{c|c|c}
			& poles  & residues  \\
			\hline
			$\mathcal{P}_{\alpha}$ & $2\gamma\mathfrak{b}$ & $\mathcal{P}_{\alpha\gamma}$\\
			&  $\frac{1}{2}\left(2\gamma\mathfrak{a}_3+2\beta\mathfrak{b}+\tau\right)$    &  $\mathcal{P}_{\alpha\beta\gamma}$   \\
			\hline
			 $\mathcal{Q}_{\alpha\beta}$  & $\frac{k}{2}+\frac{\ell}{2}\tau$,$\quad$  $(k,\ell) = \{(0,1),(1,0),(1,1)\}$ & $ \mathcal{Q}_{\alpha\beta}^{(k,\ell)} $\\
			                              & $-\alpha\beta\mathfrak{b}+\frac{k}{2}+\frac{\ell}{2}\tau$, $\quad $ $\{(k,\ell)\}=\{(0,0),(1,0),(1,1)\}$  & $-\mathcal{Q}_{\alpha\beta}^{(k,\ell+1)}$ \\
			                              & $\alpha\beta\mathfrak{b}+\frac{\tau}{2}$  & $\gamma\mathcal{P}_{\alpha\gamma}$ \\
			                              & $\gamma\mathfrak{a}_3+\alpha\beta\mathfrak{b}+\frac{\tau}{2}$ &  $\mathcal{Q}_{\alpha\beta\gamma}$   \\
			                              &   $\gamma\mathfrak{a}_3-2\alpha\beta\mathfrak{b}$           &  $\mathcal{Q}_{\alpha\beta -\gamma}$  \\
			\hline
			$\tilde{\mathcal{Q}}_{\alpha\beta}$ &    $\gamma\mathfrak{b}+\frac{\tau}{2}$    &    $-\mathcal{P}_{\gamma\beta\alpha}$   \\
			&  $-\gamma\mathfrak{a}_3+\alpha\beta\gamma\mathfrak{b}+\frac{\tau}{2}$  &   $\tilde{\mathcal{Q}}_{\alpha\beta\gamma}$   \\
			&  $\gamma(-\mathfrak{a}_3-2\alpha\beta\mathfrak{b})$        &  $-\gamma\mathcal{Q}_{\alpha\beta 1}$   
		\end{tabular}
		\caption{Poles and residues of $\mathcal{P}_{\alpha}$, $\mathcal{Q}_{\alpha\beta}$ and $\tilde{\mathcal{Q}}_{\alpha\beta}$ with respect to the variable $\mathfrak{a}_2$. Here $\alpha, \beta, \gamma=\pm 1$.\label{poles-residues-SO(7)}}
	\end{table}
}

 %The function $\mathcal{P}_{\alpha}$ has the following simple poles as an elliptic function of $\mathfrak{a}_2$,
%\begin{align}
%2\gamma\mathfrak{b} , \quad \frac{1}{2}\left(2\gamma\mathfrak{a}_3+2\beta\mathfrak{b}+\tau\right) \ ,\qquad \gamma,\beta=\pm 1 \ .
%\end{align}
%The corresponding residues are listed below
%\begin{align}
%&\mathcal{P}_{\alpha\gamma}=-\gamma\alpha\frac{\vartheta_1^\prime(0)\vartheta_4(3\mathfrak{b})\vartheta_1(\mathfrak{a}_3-2\mathfrak{b})\vartheta_1(\mathfrak{a}_3+2\mathfrak{b})}{\vartheta_1(2\mathfrak{b})\vartheta_1(4\mathfrak{b})\vartheta_4(\mathfrak{a}_3-3\mathfrak{b})\vartheta_4(\mathfrak{a}_3+3\mathfrak{b})} \ ,\\
%&\mathcal{P}_{\alpha\beta\gamma}=\alpha\beta\frac{\vartheta_1(\gamma\mathfrak{a}_3)\vartheta_4(\mathfrak{b})\vartheta_4(\beta\mathfrak{b}+\gamma \mathfrak{a}_3)\vartheta_4(\beta\mathfrak{b}+2\gamma\mathfrak{a}_3)^2\vartheta_1^\prime(0)}{\vartheta_1(2\mathfrak{b})^2\vartheta_1(2\gamma \mathfrak{a}_3)\vartheta_1(-2\beta\mathfrak{b}+\gamma\mathfrak{a}_3)\vartheta_1(2\beta\mathfrak{b}+2\gamma\mathfrak{a}_3)\vartheta_4(3\beta\mathfrak{b}+\gamma\mathfrak{a}_3)} \ .
%\end{align}

 Using the integration formula (\ref{integration-formula-fE-1}), the $a_2$ integration leaves a final integrand
 \begin{align}
\mathcal{Z}_2(a_3)=\oint \frac{da_2}{2\pi i a_2}\mathcal{Z}_1(a_2,a_3)= I_1+ I_2 +I_3,
 \end{align}
where 
%For the function $\mathcal{Q}_{\alpha\beta}$, as an elliptic function for $\mathfrak{a}_2$, it has the following poles, 
%\begin{align}
%&\frac{k}{2}+\frac{\ell}{2}\tau,  & (k,\ell) = & \{(0,1),(1,0),(1,1)\} \ ,\\
%&-\alpha\beta\mathfrak{b}+\frac{k}{2}+\frac{\ell}{2}\tau,\quad & \ \{(k,\ell)\}= & \ \{(0,0),(1,0),(1,1)\}, \\
%& \ \alpha\beta \mathfrak{b}+\frac{\tau}{2} \ ,\\
%&\gamma\mathfrak{a}_3+\alpha\beta\mathfrak{b}+\frac{\tau}{2},\quad \gamma\mathfrak{a}_3-2\alpha\beta\mathfrak{b}\quad & \gamma= & \ \pm 1 \ .
%\end{align}
%At the poles $\mathfrak{a}_2=\frac{k}{2}+\frac{\ell}{2}\tau$, the corresponding residues are:
%\begin{align}
%\mathcal{Q}_{\alpha\beta}^{(k,l)} = \frac{
%	-2\alpha (k-\frac{1}{2})b^{-\ell}\psi \vartheta_1(\frac{k}{2}+\frac{\ell}{2}\tau)\vartheta_1(\mathfrak{a}_3+\frac{k}{2}+\frac{\ell}{2}\tau)^2
%}{
%	\vartheta_1\left(2\mathfrak{b}+\frac{k}{2}+\frac{\ell}{2}\tau\right)\vartheta_1\left(\mathfrak{a}_3+2\mathfrak{b}+\frac{k}{2}+\frac{\ell}{2}\tau\right)\vartheta_1\left(\mathfrak{a}_3-2\mathfrak{b}+\frac{k}{2}+\frac{\ell}{2}\tau\right)
%} \ , \nonumber
%\end{align}
%where
%\begin{align}
%	\psi \coloneqq \frac{\vartheta_1^\prime(0)\vartheta_1(\mathfrak{a}_3)^2\vartheta_4(\mathfrak{b})}{2\vartheta_1(2\mathfrak{b})^2\vartheta_4(\mathfrak{a}_3+\mathfrak{b})\vartheta_4(\mathfrak{a}_3-\mathfrak{b})} \ .
%\end{align} 
%At the poles $\mathfrak{a}_2=-\alpha\beta\mathfrak{b}+\frac{k}{2}+\frac{\ell}{2}\tau$, the residues are:
%\begin{align}
% -\mathcal{Q}_{\alpha\beta}^{(k,l+1)}=2\alpha \left(k-\frac{1}{2}\right)b^{-\ell}\psi\frac{\vartheta_4\left(\frac{k}{2}+\frac{\ell}{2}\tau\right)\vartheta_4\left(\mathfrak{a}_3+\frac{k}{2}+\frac{\ell}{2}\tau\right)^2}{\vartheta_4\left(2\mathfrak{b}+\frac{k}{2}+\frac{\ell}{2}\tau\right)\vartheta_4\left(\mathfrak{a}_3+2\mathfrak{b}+\frac{k}{2}+\frac{\ell}{2}\tau\right)\vartheta_4\left(\mathfrak{a}_3-2\mathfrak{b}+\frac{k}{2}+\frac{\ell}{2}\tau\right)}
%\end{align} 
%At $\mathfrak{a}_2=\alpha\beta\mathfrak{b}+\frac{\tau}{2}$, the residue is given by:
%\begin{align}
%\gamma\mathcal{P}_{\alpha\gamma}=-\alpha\frac{\vartheta^\prime_1(0)\vartheta_4(3\mathfrak{b})}{\vartheta_1(2\mathfrak{b})\vartheta_1(4\mathfrak{b})}\prod_{\gamma=\pm}\frac{\vartheta_1\left(\mathfrak{a}_3+2\gamma\mathfrak{b}\right)}{\vartheta_4\left(\mathfrak{a}_3+3\gamma\mathfrak{b}\right)}
%\end{align}
%At $\mathfrak{a}_2=\gamma\mathfrak{a}_3+\alpha\beta\mathfrak{b}+\frac{\tau}{2}$, the residue is:
%\begin{align}
%\mathcal{Q}_{\alpha\beta\gamma}=-\alpha\frac{\vartheta_1^\prime(0)}{\vartheta_4(3\mathfrak{b})}\frac{\vartheta_1\left(\mathfrak{a}_3\right)\vartheta_1\left(\mathfrak{a}_3+2\alpha\beta\gamma\mathfrak{b}\right)\vartheta_4\left(2\mathfrak{a}_3+\alpha\beta\gamma\mathfrak{b}\right)\vartheta_4\left(2\mathfrak{a}_3+3\alpha\beta\gamma\mathfrak{b}\right)}{\vartheta_1(2\mathfrak{a}_3)\vartheta_1\left(2\mathfrak{a}_3+4\alpha\beta\gamma\mathfrak{b}\right)\vartheta_4\left(\mathfrak{a}_3-\alpha\beta\gamma\mathfrak{b}\right)\vartheta_4\left(\mathfrak{a}_3+3\alpha\beta\gamma\mathfrak{b}\right)}
%\end{align}
%At $\mathfrak{a}_2=\gamma\mathfrak{a}_3-2\alpha\beta\mathfrak{b}$:
%\begin{align}
%\mathcal{Q}_{\alpha\beta -\gamma}=\alpha\frac{\vartheta_1^\prime(0)}{\vartheta_4(3\mathfrak{b})}\frac{\vartheta_1\left(\mathfrak{a}_3\right)\vartheta_1\left(\mathfrak{a}_3-2\alpha\beta\gamma\mathfrak{b}\right)\vartheta_4\left(2\mathfrak{a}_3-\alpha\beta\gamma\mathfrak{b}\right)\vartheta_4\left(2\mathfrak{a}_3-3\alpha\beta\gamma\mathfrak{b}\right)}{\vartheta_1(2\mathfrak{a}_3)\vartheta_1\left(2\mathfrak{a}_3-4\alpha\beta\gamma\mathfrak{b}\right)\vartheta_4\left(\mathfrak{a}_3+\alpha\beta\gamma\mathfrak{b}\right)\vartheta_4\left(\mathfrak{a}_3-3\alpha\beta\gamma\mathfrak{b}\right)}
%\end{align}


%For $\tilde{\mathcal{Q}}_{\alpha\beta}$, as an elliptic function for $a_2$, it has the following poles:
%\begin{align}
%\gamma\mathfrak{b}+\frac{\tau}{2}\quad -\gamma\mathfrak{a}_3+\alpha\beta\gamma\mathfrak{b}+\frac{\tau}{2}\quad \gamma(-\mathfrak{a}_3-2\alpha\beta\mathfrak{b})\quad \gamma=\pm 1
%\end{align}
%The residues are:
%\begin{align}
%&-\mathcal{P}_{\gamma\beta\alpha}=-\beta\gamma\frac{\vartheta_4(\mathfrak{b})\vartheta_1^\prime(0)}{\vartheta_1(2\mathfrak{b})^2}\frac{\vartheta_1(\mathfrak{a}_3)\vartheta_4(\mathfrak{a}_3+\alpha\beta\mathfrak{b})\vartheta_4(2\mathfrak{a}_3+\alpha\beta\mathfrak{b})^2}{\vartheta_1(2\mathfrak{a}_3)\vartheta_1(\mathfrak{a}_3-2\alpha\beta\mathfrak{b})\vartheta_1(2\mathfrak{a}_3+2\alpha\beta\mathfrak{b})\vartheta_4(\mathfrak{a}_3+3\alpha\beta\mathfrak{b})}\\
%&\tilde{\mathcal{Q}}_{\alpha\beta\gamma}=-\alpha\gamma\frac{\vartheta_1^\prime(0)}{\vartheta_4(3\mathfrak{b})}\prod_{\delta=\pm 1}\frac{\vartheta_4(\mathfrak{a}_3+\delta\mathfrak{b})\vartheta_4(2\mathfrak{a}_3+\delta\mathfrak{b})}{\vartheta_1(2\mathfrak{a}_3+2\delta\mathfrak{b})\vartheta_1(\mathfrak{a}_3+2\delta\mathfrak{b})}\\
%&-\gamma\mathcal{Q}_{\alpha\beta 1}=\alpha\gamma\frac{\vartheta_1^\prime(0)}{\vartheta_4(3\mathfrak{b})}\frac{\vartheta_1(\mathfrak{a}_3)\vartheta_1(\mathfrak{a}_3+2\alpha\beta\mathfrak{b})\vartheta_4(2\mathfrak{a}_3+\alpha\beta\mathfrak{b})\vartheta_4(2\mathfrak{a}_3+3\alpha\beta\mathfrak{b})}{\vartheta_1(2\mathfrak{a}_3)\vartheta_1(2\mathfrak{a}_3+4\alpha\beta\mathfrak{b})\vartheta_4(\mathfrak{a}_3-\alpha\beta\mathfrak{b})\vartheta_4(\mathfrak{a}_3+3\alpha\beta\mathfrak{b})}
%\end{align}
%Therefore, according to the integration formula (\ref{integration-formula-f}):(take $\mathfrak{z}_0=\mathfrak{b}$ here)
\begin{align}
I_1=&\ \sum_{\alpha=\pm 1}\oint\frac{da_2}{2\pi i a_2}\mathcal{P}_{\alpha}E_1\begin{bmatrix}
-1\\
b^\alpha
\end{bmatrix}\nonumber\\
= & \sum_{\alpha, \gamma=\pm 1}\mathcal{P}_{\alpha\gamma}E_1\begin{bmatrix}
-1\\
b^{2\gamma-1}q^{\frac{1}{2}}
\end{bmatrix}E_1\begin{bmatrix}
-1\\
b^\alpha
\end{bmatrix}+\sum_{\alpha, \beta,\gamma=\pm 1}\mathcal{P}_{\alpha\beta\gamma}E_1\begin{bmatrix}
-1\\
a_3^\gamma b^{\beta-1}
\end{bmatrix}E_1\begin{bmatrix}
-1\\
b^\alpha
\end{bmatrix}.
\end{align}
%According to (\ref{integration-formula-fE-1}), (here we take $\mathfrak{z}_0=0$)
%\begin{align}
%&\oint \frac{da_2}{2\pi i a_2}\alpha\mathcal{Q}_{\alpha\beta}E_1\begin{bmatrix}
%-1\\
%a_2 b^{\alpha\beta}
%\end{bmatrix}=-\alpha\sum_{j \operatorname{Real}/\operatorname{Imag.}}R_j\left(\mathcal{S}_1 E_1\begin{bmatrix}
%-1\\
%z_j q^{\pm\frac{1}{2}}
%\end{bmatrix}+\mathcal{S}_0 E_2\begin{bmatrix}
%+1\\
%z_j b^{\alpha\beta}q^{\pm\frac{1}{2}}
%\end{bmatrix}\right)\\
%&=-\alpha\left(\mathcal{Q}^{(1,0)}_{\alpha\beta}\left(\mathcal{S}_1E_1\begin{bmatrix}
%-1\\
%-q^{\frac{1}{2}}
%\end{bmatrix}+\mathcal{S}_0 E_2\begin{bmatrix}
%+1\\
%-b^{\alpha\beta}q^{\frac{1}{2}}
%\end{bmatrix}\right)\right.\\
%&-\mathcal{Q}^{(0,1)}_{\alpha\beta}\left(\mathcal{S}_1 E_1\begin{bmatrix}
%-1\\
%b^{-\alpha\beta}q^{\frac{1}{2}}
%\end{bmatrix}+\mathcal{S}_0 E_2\begin{bmatrix}
%+1\\
%q^{\frac{1}{2}}
%\end{bmatrix}\right)\\
%&-\mathcal{Q}^{(1,1)}_{\alpha\beta}\left(\mathcal{S}_1 E_1\begin{bmatrix}
%-1\\
%-b^{-\alpha\beta}q^{\frac{1}{2}}
%\end{bmatrix}+\mathcal{S}_0 E_2\begin{bmatrix}
%+1\\
%-q^{\frac{1}{2}}
%\end{bmatrix}\right)\\
%&\left. +\sum_{\gamma}\mathcal{Q}_{\alpha\beta-\gamma}\left(\mathcal{S}_1 E_1\begin{bmatrix}
%-1\\
%a_3^{\gamma}b^{-2\alpha\beta}q^{\frac{1}{2}}
%\end{bmatrix}+\mathcal{S}_0 E_2\begin{bmatrix}
%+1\\
%a_3^{\gamma}b^{-\alpha\beta}q^{\frac{1}{2}}
%\end{bmatrix}\right)\right)\\
%&-\alpha\left(\mathcal{Q}^{(0,1)}_{\alpha\beta}\left(\mathcal{S}_1 E_1\begin{bmatrix}
%-1\\
%+1
%\end{bmatrix}+\mathcal{S}_0 E_2\begin{bmatrix}
%+1\\
%b^{\alpha\beta}
%\end{bmatrix}\right)+\mathcal{Q}^{(1,1)}_{\alpha\beta}\left(\mathcal{S}_1 E_1\begin{bmatrix}
%-1\\
%-1
%\end{bmatrix}+\mathcal{S}_0 E_2\begin{bmatrix}
%+1\\
%-b^{\alpha\beta}
%\end{bmatrix}\right)\right.\\
%&.-\mathcal{Q}^{(1,2)}_{\alpha\beta}\left(\mathcal{S}_1 E_1\begin{bmatrix}
%	-1\\
%	-b^{-\alpha\beta}
%\end{bmatrix}+\mathcal{S}_0 E_2\begin{bmatrix}
%+1\\
%-1
%\end{bmatrix}\right)+\gamma\mathcal{P}_{\alpha\gamma}\left(\mathcal{S}_1 E_1\begin{bmatrix}
%	-1\\
%	b^{\alpha\beta}
%\end{bmatrix}+\mathcal{S}_0 E_2\begin{bmatrix}
%+1\\
%b^{2\alpha\beta}
%\end{bmatrix}\right)\\
%&\left.+\sum_{\gamma}\mathcal{Q}_{\alpha\beta\gamma}\left(\mathcal{S}_1 E_1\begin{bmatrix}
%-1\\
%a_3^{\gamma}b^{\alpha\beta}
%\end{bmatrix}+\mathcal{S}_0 E_2\begin{bmatrix}
%+1\\
%a_3^{\gamma}b^{2\alpha\beta}
%\end{bmatrix}\right)\right)
%\end{align}
and 
\begin{align}
I_2= & \ \sum_{\alpha=\pm 1}\oint \frac{da_2}{2\pi i a_2}\mathcal{Q}_{\alpha\beta}E_1\begin{bmatrix}
	-1\\
	a_2^{\alpha} b^{\beta}
\end{bmatrix}=\sum_{\alpha=\pm 1}\sum_{k,\ell,m,n=0}^{1}(-1)^n \alpha \mathcal{Q}^{(k,\ell)}_{\alpha\beta}\mathcal{S}_{m}E_1\begin{bmatrix}
(-1)^m\\
(-1)^k b^{(n-m)\alpha\beta}q^{\frac{n+(-1)^n \ell}{2}}
\end{bmatrix}\nonumber\\
& \ -\sum_{\alpha,\beta,\gamma=\pm 1}\sum_{k=0}^1 \alpha\mathcal{Q}_{\alpha\beta\gamma}\mathcal{S}_{k}\left(E_{2-k}\begin{bmatrix}
(-1)^k\\
a_3^{-\gamma}b^{-(k+1)\alpha\beta}q^{\frac{1}{2}}
\end{bmatrix}+E_{2-k}\begin{bmatrix}
(-1)^k\\
a_3^\gamma b^{(2-k)\alpha\beta}
\end{bmatrix}\right)\nonumber\\
& \ -\sum_{\alpha,\beta=\pm 1}\sum_{k=0}^1 \alpha\gamma \mathcal{P}_{\alpha\gamma}\mathcal{S}_k E_{2-k}\begin{bmatrix}
(-1)^k\\
b^{(2-k)\alpha\beta}
\end{bmatrix},
\end{align}
and
\begin{align}
I_3= & \ \sum_{\alpha,\beta=\pm 1}\oint \frac{da_2}{2\pi i a_2}\tilde{Q}_{\alpha\beta}E_1\begin{bmatrix}
-1\\
a_3^{\alpha}b^\beta
\end{bmatrix}\nonumber\\
= & \ \sum_{\alpha,\beta=\pm 1}\left.\tilde{Q}_{\alpha\beta}\right|_{\mathfrak{a}_2=0} E_1\begin{bmatrix}
-1\\
a_3^{\alpha}b^\beta
\end{bmatrix}-\sum_{\alpha,\beta,\gamma=\pm 1}\gamma \mathcal{Q}_{\alpha\beta 1}E_1\begin{bmatrix}
-1\\
a_3^{-\gamma}b^{-2\alpha\beta\gamma}q^{\frac{1}{2}}
\end{bmatrix}E_1\begin{bmatrix}
-1\\
a_3^{\alpha}b^\beta
\end{bmatrix}\notag\\
& \ -\sum_{\alpha,\beta,\gamma=\pm 1}\mathcal{P}_{\gamma\beta\alpha}E_1\begin{bmatrix}
-1\\
b^\gamma
\end{bmatrix}E_1\begin{bmatrix}
-1\\
a_3^{\alpha}b^\beta
\end{bmatrix}+\sum_{\alpha,\beta,\gamma=\pm 1}\tilde{\mathcal{Q}}_{\alpha\beta\gamma}E_1\begin{bmatrix}
-1\\
a_3^{-\gamma}b^{\alpha\beta\gamma}
\end{bmatrix}E_1\begin{bmatrix}
-1\\
a_3^{\alpha}b^\beta
\end{bmatrix}.
\end{align}
The closed-form of $\mathcal{N}=4$ $SO(7)$ Schur index is then given by
\begin{align}
\mathcal{I}=\oint_{|a_3|=1}\frac{da_3}{2\pi i a_3}\mathcal{Z}_2(a_3).
\end{align}
At this stage we encounter the following types of integral
\begin{align}
\oint_{|z|=1}\frac{dz}{2\pi i z}f(z)E_k\begin{bmatrix}
\pm 1\\
z a
\end{bmatrix}, \quad \oint_{|z|=1}\frac{dz}{2\pi i z}f(z)E_1\begin{bmatrix}
-1\\
z a
\end{bmatrix}E_1\begin{bmatrix}
-1\\
z b
\end{bmatrix} \ ,
\end{align}
which can be computed using the integration formula (\ref{integration-formula-fE-1}), (\ref{integration-formula-fE-2}), and (\ref{integration-formula-fEE-1}). Since the computation of integrand is somewhat technical and tedious, we will only present the final result without the details. To do so, wee define $\mathbf{I}_{1,2,3}$ as some intricate combinations of Eisenstein series,
\begin{align}
(\mathbf{I}_1)_{\alpha\beta} \coloneqq & \ E_1\begin{bmatrix}
-1\\
b
\end{bmatrix}\left(-\sum_{k=0}^2 (-1)^{\lceil\frac{k}{2}\rceil}(3-k)E_2\begin{bmatrix}
(-1)^{\beta+k+1}\\
(-1)^\alpha b^k
\end{bmatrix}\right. \nonumber\\
&\ \left. +(-1)^\beta\left(\sum_{\pm 1}\mp E_1\begin{bmatrix}
-1\\
(-1)^\alpha b^{\frac{1}{2}} q^{\pm\frac{1}{4}}
\end{bmatrix}+\frac{1}{4}\right)\right),
\end{align}
and
\begin{align}
(\mathbf{I}_2)_{\beta\gamma} \coloneqq & \ 4(-1)^\beta \left(3E_3\begin{bmatrix}
(-1)^{\beta+\gamma+1}\\
(-1)^\beta b^2
\end{bmatrix}-3 E_3\begin{bmatrix}
(-1)^{\beta+\gamma}\\
(-1)^\beta b
\end{bmatrix}+E_3\begin{bmatrix}
(-1)^{\beta+\gamma}\\
(-1)^\beta b^3
\end{bmatrix}\right) \nonumber\\
& \ -2 \sum_{k=0}^1 (-1)^{\beta+k} E_1\begin{bmatrix}
-1\\
b
\end{bmatrix}E_2\begin{bmatrix}
(-1)^{\beta+\gamma}\\
(-1)^\beta b^{2k+1}
\end{bmatrix}-4(-1)^\beta E_2\begin{bmatrix}
1\\
b^2
\end{bmatrix} E_1\begin{bmatrix}
(-1)^{\beta+\gamma+1}\\
(-1)^\beta b^2
\end{bmatrix} \nonumber \\
& \ +2(-1)^\beta E_1\begin{bmatrix}
(-1)^{\beta+\gamma+1}\\
(-1)^\beta b^2
\end{bmatrix}\sum_{k\in\{0,1,3\}}\left(4-k\right)(-1)^{\lceil\frac{k}{2}\rceil}E_2\begin{bmatrix}
(-1)^{\beta+\gamma+k}\\
b^k
\end{bmatrix}\\
& \ +\sum_{\pm}\sum_{k=0}^1 E_1\begin{bmatrix}
	1\\
	(-1)^\beta b^{\frac{1}{2}+k} q^{\pm\frac{(-1)^k}{4}}
\end{bmatrix}\left( 2(-1)^\beta  \left(E_2\begin{bmatrix}
1\\
b^2
\end{bmatrix}-E_2\begin{bmatrix}
-1\\
b
\end{bmatrix}\right)\pm (-1)^\gamma E_1\begin{bmatrix}
-1\\
b
\end{bmatrix}\right) \nonumber\\
& \ -\frac{5+4\gamma -4\beta}{2}E_1\begin{bmatrix}
(-1)^{\beta+\gamma+1}\\
(-1)^\beta b^2
\end{bmatrix}+5\beta\gamma E_1\begin{bmatrix}
-1\\
-b^2
\end{bmatrix} \nonumber\\
& \ +\sum_{\pm}\sum_{k=0}^1 \frac{(-1)^\beta}{4}E_1\begin{bmatrix}
1\\
(-1)^\beta b^{\frac{1}{2}+k} q^{\pm\frac{(-1)^k}{4}}
\end{bmatrix}, \nonumber
\end{align}
and
\begin{align}
\mathbf{I}_3 \coloneqq & \ -4\left(-\sum_{k=0}^1\sum_{\alpha=\pm 1}E_1\begin{bmatrix}
1\\
b^{k+\frac{3}{2}}q^{\frac{\alpha}{4}}
\end{bmatrix}+E_1\begin{bmatrix}
-1\\
b^5
\end{bmatrix}\right)\left(\sum_{i=1}^2 (-1)^{i+1}E_2\begin{bmatrix}
(-1)^i\\
b^i
\end{bmatrix}-\frac{1}{8}\right) \nonumber\\
& \ -2\sum_{k=1}^2 E_1\begin{bmatrix}
-1\\
b^{2k+1}
\end{bmatrix}E_1\begin{bmatrix}
-1\\
b
\end{bmatrix}^2+E_1\begin{bmatrix}
-1\\
b^3
\end{bmatrix}\left(\sum_{k\in\{1,2,4\}}B_k E_2\begin{bmatrix}
(-1)^k\\
b^k
\end{bmatrix}+4\right) \nonumber\\
& \ +E_1\begin{bmatrix}
-1\\
b
\end{bmatrix}\left(\sum_{\alpha=\pm 1}2\alpha \left(\sum_{k=1}^2 k E_1\begin{bmatrix}
(-1)^{k+1}\\
b^{\frac{5}{2}}q^{\frac{\alpha}{4}}
\end{bmatrix}- E_1\begin{bmatrix}
1\\
b^{\frac{3}{2}}q^{\frac{\alpha}{4}}
\end{bmatrix}\right)+\sum_{k=1}^3 C_k E_2\begin{bmatrix}
(-1)^k\\
b^k
\end{bmatrix}+\frac{9}{2} \right) \nonumber\\
& \ -2\prod_{k=0}^2 E_1\begin{bmatrix}
-1\\
b^{2k+1}
\end{bmatrix}-2 E_1\begin{bmatrix}
-1\\
b
\end{bmatrix}^3+\sum_{n=1}^4 A_n E_3\begin{bmatrix}
(-1)^n\\
b^n
\end{bmatrix}.
\end{align}
With the above three definitions, we can express the $\mathcal{N} = 4$ $SO(7)$ Schur index as
\begin{align}
\mathcal{I}_{\mathcal{N} = 4 \ SO(7)}
= \frac{i}{48}\sum_{\alpha,\beta}\left(\mathbf{R}_{\alpha\beta}(\mathbf{I}_1)_{\alpha\beta}+\mathbf{T}_{\alpha\beta}(\mathbf{I}_2)_{\alpha\beta}\right)+\frac{i}{48}\mathbf{W}\mathbf{I}_3.
\end{align}
In this formula, the greek indices $(\alpha,\beta)$ sum over the set $\{(0,0),(1,0),(1,1)\}$.
Note that
\begin{align}
&\mathbf{R}_{\alpha\beta} \coloneqq b^{-2\beta}\frac{\vartheta_1\left(\mathfrak{b}+\frac{\alpha+\beta\tau}{2}\right)\vartheta_4\left(\frac{\alpha+\beta\tau}{2}\right)\vartheta_4(\mathfrak{b})^3}{\vartheta_1\left(3\mathfrak{b}+\frac{\alpha+\beta\tau}{2}\right)\vartheta_4\left(2\mathfrak{b}+\frac{\alpha+\beta\tau}{2}\right)\vartheta_1(2\mathfrak{b})^3},\\
&\mathbf{T}_{\beta\gamma} \coloneqq  b^{\gamma-\beta}\frac{\vartheta_4\left(\frac{\beta+(\beta-\gamma)\tau}{2}\right)\vartheta_4\left(2\mathfrak{b}+\frac{\beta+(\beta-\gamma)\tau}{2}\right)\vartheta_4(\mathfrak{b})}{\vartheta_1\left(3\mathfrak{b}+\frac{\beta+(\beta-\gamma)\tau}{2}\right)\vartheta_1\left(\mathfrak{b}+\frac{\beta+(\beta-\gamma)\tau}{2}\right)\vartheta_1(4\mathfrak{b})}
\quad
\mathbf{W} \coloneqq  \prod_{k=0}^2\frac{\vartheta_4\left((2k+1)\mathfrak{b}\right)}{\vartheta_1\left((2k+2)\mathfrak{b}\right)}.
\end{align}

Finally, it is straightforward to check that the unflavored limit of $\mathcal{I}_{\mathcal{N} = 4 \ SO(7)}$ satisfies a monic $\Gamma^0(2)$ modular differential equation at order 10,
\begin{footnotesize}
\begin{align}
	&\mathcal{D}^{\mathcal{N}=4}_{\mathfrak{so}(7)}=\mathcal{D}_q^{(10)}+\left(\frac{64169}{45888}\Theta_{1,1}-\frac{116269}{45888}\Theta_{0,2}\right)\mathcal{D}_q^{(8)}+\left(\frac{99455}{68832}\Theta_{0,3}-\frac{51397}{22944}\Theta_{1,2}\right)\mathcal{D}_q^{(7)}+\left(\frac{4531009 \Theta_{0,4}}{13215744}\right.\nonumber\\
	&\left.-\frac{3779273\Theta_{1,3}}{3303936}+\frac{5245697\Theta_{2,2}}{4405248}\right)\mathcal{D}^{(6)}_q+\left(-\frac{1133653\Theta_{0,5}}{4405248}+\frac{2557903\Theta_{1,4}}{4405248}-\frac{55973\Theta_{2,3}}{2202624}\right)\mathcal{D}^{(5)}_q\nonumber\\
	&+\left(\frac{1190885473\Theta_{0,6}}{22836805632}-\frac{924970757\Theta_{1,5}}{3806134272}+\frac{3937715525\Theta_{2,4}}{7612268544}-\frac{2505775369\Theta_{3,3}}{11418402816}\right)\mathcal{D}_q^{(4)}+\left(-\frac{117336059\Theta_{0,7}}{22836805632}\right.\nonumber\\
	&\left.+\frac{2991097351\Theta_{1,6}}{22836805632}-\frac{380366011\Theta_{2,5}}{2537422848}+\frac{930902663\Theta_{3,4}}{22836805632}\right)\mathcal{D}_q^{(3)}+\left(-\frac{274137107749\Theta_{0,8}}{26308000088064}-\frac{3736889371\Theta_{1,7}}{3288500011008}\right.\nonumber\\
	&\left.-\frac{330134662435\Theta_{2,6}}{6577000022016}+\frac{199201642115\Theta_{3,5}}{3288500011008}+\frac{41820786289\Theta_{4,4}}{26308000088064}\right)\mathcal{D}_q^{(2)}+\left(\frac{240693275531\Theta_{0,9}}{39462000132096}\right.\nonumber\\
	&\left.-\frac{88992212869\Theta_{1,8}}{4384666681344}-\frac{8099874757\Theta_{2,7}}{1096166670336}+\frac{330064570085\Theta_{3,6}}{3288500011008}-\frac{174451260571\Theta_{4,5}}{2192333340672}\right)\mathcal{D}_q^{(1)}+\left(-\frac{256921875\Theta_{0,10}}{256624295936}\right.\nonumber\\
	&\left.+\frac{477416835\Theta_{1,9}}{128312147968}-\frac{59821335\Theta_{2,8}}{256624295936}+\frac{559460601\Theta_{3,7}}{32078036992}-\frac{13109319531\Theta_{4,6}}{128312147968}+\frac{10552431897\Theta_{5,5}}{128312147968}\right),
\end{align}
\end{footnotesize}
and non-monic $\Gamma^0(2)$ equation at order 9,
\begin{footnotesize}
\begin{align}
&\mathcal{D}_{\mathfrak{so}(7)}^{\mathcal{N}=4}=\Theta_{0,1}\mathcal{D}_q^{(9)}+\left(\frac{2477\Theta_{1,1}}{1912}-\frac{3407\Theta_{0,2}}{1912}\right)\mathcal{D}_q^{(8)}+\left(\frac{6971\Theta_{1,2}}{11472}-\frac{10377\Theta_{0,3}}{3824}\right)\mathcal{D}_q^{(7)}+\left(\frac{1339781\Theta_{0,4}}{275328} \right.\nonumber\\
&\left.-\frac{15625\Theta_{1,3}}{1434}+\frac{1767811\Theta_{2,2}}{275328}\right)\mathcal{D}_q^{(6)}+\left(\frac{17516635\Theta_{0,5}}{13215744}-\frac{37436353\Theta_{1,4}}{13215744}+\frac{12423443\Theta_{2,3}}{6607872}\right)\mathcal{D}_q^{(5)}+\nonumber\\
&\left(-\frac{44438921\Theta_{0,6}}{317177856}+\frac{24643855\Theta_{1,5}}{52862976}-\frac{177147775\Theta_{2,4}}{35241984}+\frac{767384147\Theta_{3,3}}{158588928}\right)\mathcal{D}_q^{(4)}+\left(\frac{436377635\Theta_{0,7}}{11418402816}\right.\nonumber\\
&\left.-\frac{2631122143\Theta_{1,6}}{11418402816}+\frac{6486893273\Theta_{2,5}}{3806134272}-\frac{17028452303\Theta_{3,4}}{11418402816}\right)\mathcal{D}_q^{(3)}+\left(\frac{287431763\Theta_{0,8}}{30449074176}+\frac{4632486095\Theta_{1,7}}{22836805632}\right.\nonumber\\
&\left.-\frac{3490714387\Theta_{2,6}}{2537422848}+\frac{19973363051\Theta_{3,5}}{7612268544}-\frac{133319985805\Theta_{4,4}}{91347222528}\right)\mathcal{D}_q^{(2)}+\left(-\frac{247722449497\Theta_{0,9}}{26308000088064}\right.\nonumber\\
&\left.+\frac{663592423\Theta_{1,8}}{36087791616}-\frac{105995132111\Theta_{2,7}}{243592593408}+\frac{2910273174797\Theta_{3,6}}{2192333340672}-\frac{439516599949\Theta_{4,5}}{487185186816}\right)\mathcal{D}_q^{(1)}+\left(\frac{49415625\Theta_{0,10}}{32078036992}\right.\nonumber\\
&\left.-\frac{52688223\Theta_{1,9}}{16039018496}-\frac{2541812427\Theta_{2,8}}{32078036992}+\frac{3001266891\Theta_{3,7}}{4009754624}-\frac{36356798079\Theta_{4,6}}{16039018496}+\frac{25650617139\Theta_{5,5}}{16039018496}\right).
\end{align}
\end{footnotesize}

We also note that each row vector $(\mathbf{R}_{00},\mathbf{R}_{10},\mathbf{R}_{11})$, and $(\mathbf{T}_{00},\mathbf{T}_{10},\mathbf{T}_{11})$ forms the same 3-dimensional representation $\rho$ of $\Gamma^0(2)$, following from the modularity of Jacobi-theta function. In particular, the representation matrix of $STS$ is given by
\begin{align}
\rho(STS)=\left(\begin{array}{ccc}
-1 & 0 & 0\\
0  & 0 & -1\\
0  & -1 & 0
\end{array}\right),
\end{align}
acting to these two row vectors. The factor $\mathbf{W}$ form a one-dimensional representation of $\Gamma^0(2)$. It would be interesting to further investigate the relation between the many ingredients in the above closed-form of $\mathcal{I}_{\mathcal{N} = 4 \ SO(7)}$ and the highest weight characters of the associated chiral algebra $\mathbb{V}_{\mathcal{N} = 4 \ SO(7)}$, which we leave for future work.
%To continue, we should know the simple poles distribution of the following elliptic function of $a_3$:
%\begin{align}
%\mathcal{P}_{\alpha\beta}\quad \mathcal{P}_{\alpha\beta\gamma}\quad \mathcal{Q}^{(k,\ell)}_{\alpha\beta}\quad \tilde{\mathcal{Q}}_{\alpha\beta\gamma}\quad \mathcal{Q}_{\alpha\beta\gamma}\quad \tilde{\mathcal{Q}}_{\alpha\beta}(\mathfrak{a}_2=0)
%\end{align}
%The elliptic function $\mathcal{P}_{\alpha\beta}$ has two poles:
%\begin{align}
%3\gamma\mathfrak{b}+\frac{\tau}{2}\quad \gamma=\pm 1
%\end{align}
%The residues are:
%\begin{align}
%\alpha\beta\gamma\frac{\vartheta_4(\mathfrak{b})\vartheta_4(3\mathfrak{b})\vartheta_4(5\mathfrak{b})}{\vartheta_1(2\mathfrak{b})\vartheta_1(4\mathfrak{b})\vartheta_1(6\mathfrak{b})}
%\end{align}
%For the function $\mathcal{P}_{\alpha\beta\gamma}$, the poles are:
%\begin{align}
%&\frac{k}{2}+\frac{\ell}{2}\tau\quad \text{where} (k+\ell\neq 0, k,\ell=0,1)\\
%&-\beta\gamma\mathfrak{b}+\frac{k}{2}+\frac{\ell}{2}\tau\quad \text{where}\quad k\geq \ell,\quad k,\ell=0,1\\
%&2\beta\gamma\mathfrak{b}
%\end{align}
%The residues are:
%\begin{align}
%&\alpha\beta b^{-2\beta\gamma \ell}\frac{\vartheta_4\left(\beta\gamma\mathfrak{b}\right)^3 \vartheta_1\left(\frac{k}{2}+\frac{\ell}{2}\tau\right)\vartheta_4\left(\beta\gamma\mathfrak{b}+\frac{k}{2}+\frac{\ell}{2}\tau\right)}{2\vartheta_1\left(2\beta\gamma\mathfrak{b}\right)^3\vartheta_1\left(2\beta\gamma\mathfrak{b}+\frac{k}{2}+\frac{\ell}{2}\tau\right)\vartheta_4\left(\frac{k}{2}+\frac{\ell}{2}\tau+3\beta\gamma\mathfrak{b}\right)}\\
%& -\alpha\beta b^{-2\beta\gamma\ell}\frac{\vartheta_4\left(\beta\gamma\mathfrak{b}\right)^3 \vartheta_1\left(\beta\gamma\mathfrak{b}+\frac{k}{2}+\frac{\ell}{2}\tau\right)\vartheta_4\left(\frac{k}{2}+\frac{\ell}{2}\tau\right)}{2\vartheta_1\left(2\beta\gamma\mathfrak{b}\right)^3\vartheta_1\left(3\beta\gamma\mathfrak{b}+\frac{k}{2}+\frac{\ell}{2}\tau\right)\vartheta_4\left(2\beta\gamma\mathfrak{b}+\frac{k}{2}+\frac{\ell}{2}\tau\right)}\\
%&\alpha\gamma\frac{\vartheta_4\left(\mathfrak{b}\right)\vartheta_4\left(3\mathfrak{b}\right)\vartheta_4\left(5\mathfrak{b}\right)}{\vartheta_1\left(2\mathfrak{b}\right)\vartheta_1\left(4\mathfrak{b}\right)\vartheta_1\left(6\mathfrak{b}\right)}
%\end{align}
%For $\mathcal{Q}^{(k,\ell)}_{\alpha\beta}$, it has the following poles:
%\begin{align}
%\gamma\mathfrak{b}+\frac{\tau}{2} \quad 2\gamma\mathfrak{b}+\frac{k}{2}+\frac{\ell}{2}\tau\quad \gamma=\pm 1 
%\end{align}
%The residues are listed below:
%\begin{align}
%&-\alpha\gamma b^{-2\ell}\frac{\vartheta_4\left(\mathfrak{b}\right)^3\vartheta_1\left(\frac{k}{2}+\frac{\ell}{2}\tau\right)\vartheta_4\left(\mathfrak{b}+\frac{k}{2}+\frac{\ell}{2}\tau\right)}{2\vartheta_1\left(2\mathfrak{b}\right)^3\vartheta_1\left(2\mathfrak{b}+\frac{k}{2}+\frac{\ell}{2}\tau\right)\vartheta_4\left(3\mathfrak{b}+\frac{k}{2}+\frac{\ell}{2}\tau\right)}\\
%&2\alpha\gamma\left(k-\frac{1}{2}\right)b^{-\ell}\frac{\vartheta_4\left(\mathfrak{b}\right)\vartheta_1\left(\frac{k}{2}+\frac{\ell}{2}\tau\right)\vartheta_1\left(2\mathfrak{b}+\frac{k}{2}+\frac{\ell}{2}\tau\right)}{2\vartheta_1\left(4\mathfrak{b}\right)\vartheta_4\left(\mathfrak{b}+\frac{k}{2}+\frac{\ell}{2}\tau\right)\vartheta_4\left(3\mathfrak{b}+\frac{k}{2}+\frac{\ell}{2}\tau\right)}
%\end{align}
%For $\tilde{\mathcal{Q}}_{\alpha\beta\gamma}$:
%\begin{align}
%2\delta\mathfrak{b}\quad \delta\mathfrak{b}+\frac{k}{2}+\frac{\ell}{2}\tau\quad \delta=\pm 1
%\end{align}
%The residues:
%\begin{align}
%&\alpha\gamma\delta\frac{\vartheta_4\left(\mathfrak{b}\right)\vartheta_4\left(3\mathfrak{b}\right)\vartheta_4\left(5\mathfrak{b}\right)}{\vartheta_1\left(2\mathfrak{b}\right)\vartheta_1\left(4\mathfrak{b}\right)\vartheta_1\left(6\mathfrak{b}\right)}
%\end{align}