%!TEX root = ../Schur indices and line operators.tex

\section{More on Schur index}

Several integration formula were proposed in \cite{Pan:2021mrw}, which can be used to analytically compute some multivariate contour integral of elliptic functions. Those formula were enough to compute exactly the Schur index of $A_1$ class-$\mathcal{S}$ theories and some low rank $\mathcal{N} = 4$ theories. However, they were insufficient for more general $A_N$ class-$\mathcal{S}$ theories. In this section, with the help from some new integration formula, we explore the exact computation the Schur index of a series of $A_2$ theories and the $\mathcal{N} = 4$ $SO(7)$ theory, generalizing the results in \cite{Pan:2021mrw}. The computation in this section is relatively technical, and uninterested readers may skip to section \ref{section:Wilson-index-A1-theories} for the computation of line operator index.


\subsection{\texorpdfstring{$A_2$ theories of class-$\mathcal{S}$}{}}



First we recall the Schur index of the $SU(3)$ SQCD. It can be computed as a contour integral
\begin{equation}
	\mathcal{I}_{\text{SQCD}} = - \frac{1}{3!} \eta(\tau)^{16} \oint \prod_{A = 1}^2 \frac{da_A}{2\pi i a_A}
	\frac{\prod_{A \ne B} \vartheta_1(\mathfrak{a}_A - \mathfrak{a}_B)}{\prod_{A = 1}^3 \prod_{i = 1}^{6} \vartheta_4(\mathfrak{a}_A - \mathfrak{m}_i)}
	\equiv \oint \prod_{A = 1}^2 \frac{da_A}{2\pi i a_A}\mathcal{Z}(\mathfrak{a})\ ,
\end{equation}
where $\mathfrak{a}_3 = -\mathfrak{a}_1 - \mathfrak{a}_2$, $a_3 = (a_1 a_2)^{-1}$, and $a_i = e^{2\pi i \mathfrak{a}_i}$, $m_i = e^{2\pi i \mathfrak{m}_i}$. See also Appendix \ref{app:special-functions} for the definitions and properties of the Eisenstein series $E_k\big[\substack{\phi\\\theta}\big]$ and the Jocobi theta functions. The integral can be performed by applying the integration formula, which yields the exact (albeit slightly complicated) result,
\begin{align}\label{index-SQCD}
	\mathcal{I}_{\text{SQCD}} = & \ \sum_{j_2 = 1}^{6}2R_{0j_2}E_1\left[\begin{matrix}
		-1 \\ m_{j_2}
	\end{matrix}\right] \nonumber \\
	& \ + \sum_{j_1}^{6}
	  \left(
	  R_{j_1}(\mathfrak{a}_2 = 0) + \sum_{j_2 = 1}^6
	  \Big(  
	  R_{j_1 j_2}E_1\left[\begin{matrix}
	  	-1 \\ m_{j_2}
	  \end{matrix}\right]
	  + R_{j_1 j_2} E_1\left[\begin{matrix}
	  	-1 \\ m_{j_1}m_{j_2}q^{ - \frac{1}{2}}
	  \end{matrix}\right]  \Big)
	\right)E_1\left[\begin{matrix}
		-1 \\ m_{j_1}
	\end{matrix}\right]\nonumber\\
	& \ + \sum_{j_1, j_2 = 1}^{6} R_{j_1 j_2}\left(
			E_2\left[\begin{matrix}
				1 \\ m_{j_1} m_{j_2}
			\end{matrix}\right]
			- E_2\left[
			\begin{matrix}
				1 \\ m_{j_2}q^{ - \frac{1}{2}}
			\end{matrix}\right]
		\right)  \ .
\end{align}
From the computation of $SU(3)$ SQCD index, we already see that the complexity is far above the $A_1$ type. Therefore, we shall focus on arguing that the index can be computed using the existing integration formula. The complexity could decrease once more optimized integration formula is found, which we leave to future work.


As a class-$\mathcal{S}$ theory, the SQCD has manifest flavor symmetry $SU(3)^{(1)} \times SU(3)^{(2)} \times U(1)^{(1)} \times U(1)^{(2)}$. We shall denote the fugacities of $SU(3)^{(\alpha)}$ as $c^{(\alpha)}$, and those of $U(1)^{(\alpha)}$ as $d^{(\alpha)}$. They are related to $m_j$ by
\begin{align}
  c^{(1)}_1 = & \ m_1/d^{(1)}, \qquad c^{(1)}_2 = m_2 / d^{(1)}, \qquad
  d^{(1)} = (m_1 m_2 m_3)^{1/3} \ ,\\
  c^{(2)}_1 = & \ m_4/d^{(2)}, \qquad c^{(2)}_2 = m_5/d^{(2)}, \qquad
  d^{(2)} =  (m_4 m_5 m_6)^{1/3} \ .
\end{align}

Starting from $SU(3)$ SQCD, one can build $SU(3)$ linear quiver theories by successively gauging in $9$ hypermultiplets one after another. Let us perform one such computation. The gauging procedure multiplies to $\mathcal{I}_\text{SQCD}$ factors
\begin{align}
	\mathcal{I}_\text{VM} \sim \prod_{\substack{A,B = 1 \\ A\ne B}}^{3} \vartheta_1 (\mathfrak{a}_A - \mathfrak{a}_B) \ ,\qquad
	\mathcal{I}_\text{HM} = \prod_{A, B = 1}^{3}\frac{\eta(\tau)}{\vartheta_4(- \mathfrak{a}_A + \mathfrak{c}^{(3)}_B + \mathfrak{d}^{(3)})} \ ,
\end{align}
where again $\mathfrak{a}_3 = - \mathfrak{a}_1 - \mathfrak{a}_2$, $\mathfrak{c}^{(3)}_3 = - \mathfrak{c}^{(3)}_ 1 - \mathfrak{c}^{(3)}_2$. The gauging also identifies $\mathfrak{c}^{(2)}_A$ with $\mathfrak{a}_A$, and a contour integral of $a_1, a_2$ should be performed, 
\begin{align}
	\mathcal{I} = \oint \frac{da_1}{2\pi i a_1}
	\frac{da_2}{2\pi i a_2} \mathcal{I}_\text{SQCD}(c^{(1)}, a, d^{(1)}, d^{(2)}) \mathcal{I}_\text{VM}(a) \mathcal{I}_\text{HM}(a, c^{(3)}, d^{(3)}) \ .
\end{align}

Let us look at the various terms in this integral. First of all, we have an integral of
\begin{align}
	\sum_{j_2 = 1}^6
	R_{0 j_2} \mathcal{I}_\text{VM} \mathcal{I}_\text{HM} E_1 \begin{bmatrix}
		-1 \\ m_{j_2}
	\end{bmatrix} \ .
\end{align}
It is straightforward to verify that, as a function of $\mathfrak{a}_{1,2}$, the factor $R_{0j_2}\mathcal{I}_\text{VM}\mathcal{I}_\text{HM}$ is elliptic with respect to both $\mathfrak{a}_{1,2}$. Moreover, after the replacing $m$ with the $c, d$ fugacities and $a$,
\begin{align}
	(m_1, \ldots, m_6) = (
	c^{(1)}_1 d^{(1)},
	c^{(1)}_2 d^{(1)},
	\frac{d^{(1)}}{c^{(1)}_1 c^{(1)}_2},
	a_1d^{(2)},
	a_2 d^{(2)},
	\frac{d^{(2)}}{a_1 a_2}
	) \ ,
\end{align}
and similarly $m_{j_1} m_{j_2 \ne j_1} \sim ((\ldots),  a_1^{\pm 1} (\ldots), a_2^{\pm 1}(\ldots), (a_1 a_2)^{\pm 1} (\ldots) )$ where $(\ldots)$ denotes combinations of $c^{(1)}, d^{(1)}, d^{(2)}$. Therefore, one can perform the $a_1$ integral using (\ref{integration-formula-fE-1}) or (\ref{integration-formula-fE-2}). For all $j_2$, there are several types of poles from $R_{0j_2}\mathcal{I}_\text{VM}\mathcal{I}_\text{HM}$,
\begin{align}
	\mathfrak{a}_1 = & \ [1,2], \quad  [3],
	\quad - \mathfrak{a}_2 + [1,2], \quad
	 - \mathfrak{a}_2 + [3] \ .
\end{align}
Here $[1,2]$ and $[3]$ denote respectively linear combinations of $\mathfrak{c}^{(1)}, d^{(1,2)}$ and $\mathfrak{c}^{(3)}, \mathfrak{d}^{(3)}$. The Eisenstein series $E_1 \big[\substack{-1\\m_{j_2 = 1,2,3}}\big]$ are independent of $a_1, a_2$, and will never participate in subsequent integrations or gauging. The variables $a_1, a_2$ in $E_1 \big[ \substack{-1\\ m_{j_2 = 4,5,6}} \big]$ appear in the form $a_1$, or $a_2$, or a product $a_1 a_2$. The $a_1$ integral using the integration formula will produce $E_1 \big[ \substack{\pm 1\\ [1,2]} \big], E_1 \big[ \substack{\pm 1\\ [3]} \big]$, $E_2\big[ \substack{\pm1\\ a_2[1,2]} \big]$ or $E_2\big[ \substack{\pm1\\ a_2[3]} \big]$, where $[1,2]$ and $[3]$ denote respectively combinations of the flavor fugacities $c^{(1)}, d^{(1)}, d^{(2)}$, and of $c^{(3)}, d^{(3)}$. The $a_2$-integration of these terms can be further carried out, and we have Eisenstein structure, 
\begin{equation}
	E_{1} \begin{bmatrix}
    \pm 1\\ [1,2]
\end{bmatrix}E_{1} \begin{bmatrix}
    \pm 1\\ [1,2]
\end{bmatrix}, \quad
	E_{1} \begin{bmatrix}
    \pm 1\\ [1,2]
\end{bmatrix}E_{1} \begin{bmatrix}
    \pm 1\\ [3]
\end{bmatrix},  \quad
	E_{1,2} \begin{bmatrix}
    \pm 1\\ [1,2,3]
\end{bmatrix} \ ,
\end{equation}
where $[1,2,3]$ denotes products of $c^{(1)}, c^{(3)}, d^{(1,2,3)}$.

Next we have $R_{j_1}(\mathfrak{a}_2 = 0)\mathcal{I}_\text{VM}\mathcal{I}_\text{HM}E_1 \big[\substack{-1 \\ m_{j_1}}\big]$ integral. Again, the prefactor $R_{j_1}(\mathfrak{a}_2 = 0)\mathcal{I}_\text{VM}\mathcal{I}_\text{HM}$ is separately elliptic with respect to both $\mathfrak{a}_{1,2}$. This factor again has $\mathfrak{a}_1$-poles of the form
\begin{align}
	\mathfrak{a}_1 = & \ [1,2], \quad  [3],
	\quad - \mathfrak{a}_2 + [1,2], \quad
	 - \mathfrak{a}_2 + [3] \ .
\end{align}
Therefore, the $a_1$, $a_2$ can also be straightforwardly performed with a reference point $\mathfrak{a}_1 = 0$. The Eisenstein structure is the same as the that of the previous term.

Let us also look at the last two terms in (\ref{index-SQCD}),
\begin{align}
	R_{j_1 j_2}\left(
	E_2 \begin{bmatrix}
		1 \\ m_{j_1}m_{j_2}
	\end{bmatrix}
	- E_2 \begin{bmatrix}
		1 \\ m_{j_2}
	\end{bmatrix}
	\right) \ .
\end{align}
We note that $R_{j_1 j_2} = 0$ when $j_1 = j_2$. One can also directly verify that $R_{j_1j_2} \mathcal{I}_\text{VM} \mathcal{I}_\text{HM}$ is elliptic, with $\mathfrak{a}_1$ poles of the same simple form as the above. Hence, one can also proceed with both $a_1, a_2$ integral using (\ref{integration-formula-fE-1}), (\ref{integration-formula-fE-2}). The Eisenstein structure of the result involves
\begin{align}
	E_1 \begin{bmatrix}
  	\pm 1 \\ [1,2]  
	\end{bmatrix}
	E_2 \begin{bmatrix}
  	\pm 1 \\ [1,2]  
	\end{bmatrix}
	E_1 \begin{bmatrix}
  	\pm 1 \\ [1,2] \text{ or } [3]
	\end{bmatrix} \ ,\quad
	E_1 \begin{bmatrix}
  	\pm 1 \\ [3]  
	\end{bmatrix}
	E_2 \begin{bmatrix}
  	\pm 1 \\ [1,2]  
	\end{bmatrix}
	E_1 \begin{bmatrix}
  	\pm 1 \\ [1,2] \text{ or } [3]
	\end{bmatrix} \ ,
\end{align}

We are now ready to deal with the middle two terms in (\ref{index-SQCD}). Again, the factor in front of the Eisenstein series is suitably elliptic. But now this elliptic function is multiplying with
\begin{align}
	E_1 \begin{bmatrix}
		-1 \\ m_{j_2}
	\end{bmatrix}E_1 \begin{bmatrix}
  	-1 \\ m_{j_1}  
	\end{bmatrix}, \qquad
	E_1 \begin{bmatrix}
		-1 \\ m_{j_1} m_{j_2}q^{-1/2}
	\end{bmatrix}E_1 \begin{bmatrix}
  	-1 \\ m_{j_1}  
	\end{bmatrix} \ .
\end{align}
When substituting in the $a, c, d$ fugacities we will need to integrate
\begin{align}
	f(a_1, a_2)E_1 \begin{bmatrix}
		-1 \\ a_A[1,2]
	\end{bmatrix}
	E_1 \begin{bmatrix}
		-1 \\ a_B [1,2]
	\end{bmatrix}, \quad
	f(a_1, a_2)
	E_1 \begin{bmatrix}
		-1 \\ a_A [1,2]
	\end{bmatrix}
	E_1 \begin{bmatrix}
		-1 \\ a_1 a_2 [1,2]
	\end{bmatrix} \ .
\end{align}
We can carry out the $a_1$ integral which involves poles of the same form as the above, $\mathfrak{a}_1 = \text{expressions of } \mathfrak{c}, \mathfrak{d}$ and $\mathfrak{a}_1 = - \mathfrak{a}_2 + \text{expressions of } \mathfrak{c}, \mathfrak{d}$. The integral can be performed using the integration formula \ref{integration-formula-fE-1}. After the $a_1$ integral, we will have the following type of integrand left to integrate (factors independent of $a_2$ are omitted),
\begin{align}
	& \ f(a_2)E_{k = 1,2} \begin{bmatrix}
			\pm 1 \\ a_2(\ldots)
		\end{bmatrix}, \qquad \text{ or },\qquad
	f(a_2)E_{k = 1,2} \begin{bmatrix}
			\pm 1 \\ a_2(\ldots)
		\end{bmatrix}
		E_1 \begin{bmatrix}
				\pm 1 \\ (\ldots)
			\end{bmatrix}\ .
\end{align}
To illustrate this, we can look at a term in the sum, for example,
\begin{align}
	f(a_1, a_2) E_1 \begin{bmatrix}
		-1 \\ a_1 a_2 (\ldots)
	\end{bmatrix}
	E_1 \begin{bmatrix}
		-1 \\ a_1 a_2 (\ldots)'
	\end{bmatrix} \ .
\end{align}
Since $f(a_1, a_2)$ has poles only of the form $a_1 = (\ldots)$ and $a_1 = a_2^{-1} (\ldots)$, the integral of the above will produce Eisenstein series with arguments
\begin{align}
	\frac{a_2(\ldots)}{a_2(\ldots)'}, \quad e^{2\pi i 0} a_2^{-1}(\ldots), \quad e^{2\pi i 0}(\ldots), \quad a_2(\ldots) a_2^{-1}(\ldots), \quad
	a_2(\ldots) (\ldots) \ , \quad \text{etc.}
\end{align}
Here we have chosen the reference point as $\mathfrak{a}_1 = 0$. Therefore, although tedious, the leftover $a_2$ integral can be dealt with, and it produces the exact Schur index for the $SU(3) \times SU(3)$ linear quiver theory. In the end, the exact index contains Eisenstein structures
\begin{align}
	E_3 \begin{bmatrix}
  	\pm 1\\ (\ldots)  
	\end{bmatrix}
	, \quad
	E_1 \begin{bmatrix}
  	\pm 1\\ (\ldots)  
	\end{bmatrix}E_2 \begin{bmatrix}
  	\pm 1\\ (\ldots)  
	\end{bmatrix} \ .
\end{align}

The above analysis can be repeated for longer linear $SU(3)$ quiver theories, where we will encounter integrals in the presence of
\begin{align}
	E_n \begin{bmatrix}
  	\pm 1 \\ z(\ldots)
	\end{bmatrix}, \qquad
	E_1 \begin{bmatrix}
  	\pm 1 \\ z(\ldots)  
	\end{bmatrix} E_n \begin{bmatrix}
  	\pm 1 \\ z(\ldots)  
	\end{bmatrix} \ .
\end{align}
These integrals can be treated using the integration formula in the appendix, and therefore Schur index of all linear $SU(3)$-quiver are computable, though rather tedious, with the current method.



Now that gauging a $SU(3)$ symmetry with fugacities $c^{(2)}$ can be carried out using the integration formula, we are able to also compute Schur index of some non-Lagrangian theories. Consider the $E_6$ superconformal field theory of Minahan and Nemeschansky \cite{Minahan:1996fg}, whose index can be computed by exploiting the Argyres-Seiberg duality \cite{Argyres:2007cn} and an inversion formula \cite{Gadde:2010te,Razamat:2012uv},
\begin{align}
  & \ \mathcal{I}_{E_6}(\mathbf{c}^{(1)}, \mathbf{c}^{(2)}, (wr, w^{-1}r, r^{-2})) \nonumber\\
  = & \ \frac{\mathcal{I}_{\text{SQCD}}(\mathbf{c}^{(1)}, \mathbf{c}^{(2)}, \frac{w^{\frac{1}{3}}}{r}, \frac{w^{- \frac{1}{3}}}{r})_{w \to q^{\frac{1}{2}}w}}{\theta(w^2)} + \frac{\mathcal{I}_{\text{SQCD}}(\mathbf{c}^{(1)}, \mathbf{c}^{(2)}, \frac{w^{\frac{1}{3}}}{r}, \frac{w^{- \frac{1}{3}}}{r})_{w \to q^{ - \frac{1}{2}}w}}{\theta(w^{ - 2})} \ ,
\end{align}
where the denominator is related to $\vartheta_1$ by
\begin{equation}
	\theta(z) \equiv \frac{\vartheta_1( \mathfrak{z})}{i z^{\frac{1}{2}} q^{\frac{1}{8}} (q;q)} \ .
\end{equation}
Note that two of the $SU(3)$ flavor symmetries of the $E_6$ theory share identical fugacities $\mathbf{c}^{(1)}, \mathbf{c}^{(2)}$ with those of the $SU(3)$ SQCD. The above formula allows one to directly compute the Schur index of, for instance, a theory of class-$\mathcal{S}$ with three maximal and one minimal punctures (see Figure \ref{fig:T3-HM}),
\begin{align}
	\mathcal{I}
	= \oint \frac{d\mathbf{a}}{2\pi i \mathbf{a}} \sum_{\pm} 
	\frac{\mathcal{I}_{\text{SQCD}}(\mathbf{c}^{(1)}, \mathbf{a}^{-1}, \frac{w^{\frac{1}{3}}}{r}, \frac{w^{- \frac{1}{3}}}{r})_{w \to q^{\pm \frac{1}{2}}w}}{\theta(w^{\pm 2})} \mathcal{I}_\text{VM}(\mathbf{a}) \mathcal{I}_\text{HM}(\mathbf{a}, c^{(3)}, d^{(3)}) \ .
\end{align}
As we have argued, this can be computed exactly with the currently available formula in the appendix.
% Figure environment removed









