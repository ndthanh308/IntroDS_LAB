%!TEX root = ../Schur indices and line operators.tex

\section{\texorpdfstring{Line operator index of $A_1$-theories of class-$\mathcal{S}$}{}\label{section:Wilson-index-A1-theories}}

In this and the following section we discuss the Schur index in the presence of a line operator. For a Lagrangian 4d $\mathcal{N} = 2$ SCFT, the Schur index in the absence of operator insersion can be computed by a multivariate contour integral \cite{Gadde:2011uv,Beem:2013sza}
\begin{align}
  \mathcal{I} = \oint \left[\frac{da}{2\pi i a}\right] \mathcal{Z}(a) \ ,
\end{align}
where the integrand $\mathcal{Z}(a)$ is elliptic with respect to the ``exponent variables'' $\mathfrak{a}_i$ separately, and captures contributions from the vector and hypermultiplets in a gauge theory description.

One can introduce half line operators in the 4d theory that extend from the origin to infinity while preserving certain amount of supercharges \cite{Cordova:2016uwk}. In particular, there are line operators that preserve the supercharges used to construct the Schur index. In the presence of such a BPS half Wilson line operator in the representation $\mathcal{R}$ of the gauge group, the half Wilson line index can be computed simply by\footnote{For simplicity we omit the normalization factor $\mathcal{I}^{-1}$.} \cite{Gang:2012yr,Cordova:2016uwk}
\begin{align}
	\langle W_{\mathcal{R}}\rangle = \oint \left[\frac{da}{2\pi i a}\right]
	\chi_\mathcal{R}(a) \mathcal{Z}(a) \ ,
\end{align}
where $\chi_\mathcal{R}(a)$ denotes the character of representation $\mathcal{R}$ of $G$. The Wilson index counts the local Schur operators (in the free limit) that are gauge-variant and can absorb the charge at the end of the half line. A full Wilson line operator in representation $\mathcal{R}$ can be thought of as a junction at the origin of two half Wilson line operators in complex-conjugating representation $\mathcal{R}, \overline{\mathcal{R}}$, and hence the full Wilson line index can be computed by
\begin{align}
  \langle W_{\mathcal{R}}^\text{full}\rangle = \oint \left[\frac{da}{2\pi i a}\right]
  \chi_\mathcal{R}(a)\chi_{\overline {\mathcal{R}}}(a) \mathcal{Z}(a) \ .
\end{align}
In our notation, we will only add the superscript ``full'' when dealing with a full Wilson line operator.

One can also consider correlators of half Wilson line operators, which take the form
\begin{align}
  \langle W_{\mathcal{R}_1} \cdots W_{\mathcal{R}_n}\rangle
  = \oint \bigg[\frac{da}{2\pi i a}\bigg]
  \bigg[\prod_{i=1}^n\chi_{\mathcal{R}_i}(a)\bigg]
  \mathcal{Z}(a)\ .
\end{align}
One can consider applying the tensor product decomposition $\otimes_{i = 1}^n \mathcal{R}_i = \sum_{j} m_j \mathcal{R}^{(j)}$ and reduce the product of characters on the right to a sum of characters of the irreducible representations $\mathcal{R}^{(j)}$ of the gauge group,
\begin{align}
  \langle W_{\mathcal{R}_1} \cdots W_{\mathcal{R}_n}\rangle = \sum_{j} m_j \langle W_{\mathcal{R}^{(j)}} \rangle \ .
\end{align}
In this sense, half Wilson line index in irreducible representations are the basic building blocks for correlators of half/full Wilson line, which will be our main focus.

In the following we will study line operator index for $A_1$ theories of class-$\mathcal{S}$. We will start with some simple examples where we are able to compute both the Wilson line index and the $S$-dual `t Hooft line index. Eventually we will analyze in detail half Wilson line index for general $A_1$ theories of class-$\mathcal{S}$. When possible, we also comment on the relation between the index and associated chiral algebra characters.


\subsection{\texorpdfstring{$\mathcal{N} = 4 $  $ SU(2)$ theory}{}\label{section:N4SU(2)}}

\subsubsection{Half Wilson line index}

The associated chiral algebra $\mathbb{V}_{\mathcal{N} = 4}$ of the $\mathcal{N} = 4$ theory with an $SU(2)$ gauge group is given by the 2d small $\mathcal{N} = 4$ superconformal algebra. The Schur index, which is identified with the vacuum character of $\mathbb{V}_{\mathcal{N} = 4}$, can be computed by the contour integral
\begin{align}
  \mathcal{I}_{\mathcal{N} = 4} 
  = & \ - \frac{1}{2}\frac{\eta(\tau)^3}{\vartheta_4(\mathfrak{b})}
  \oint_{|a| = 1} \frac{da}{2\pi i a} 
  \frac{
    \vartheta_1(2\mathfrak{a})\vartheta_1(- 2\mathfrak{a})
  }{
    \vartheta_4(2\mathfrak{a} + \mathfrak{b})
    \vartheta_4(-2\mathfrak{a} + \mathfrak{b})
  }
  \coloneqq \oint \frac{da}{2\pi i a} \mathcal{Z}(a)\\
  = & \ \frac{i\vartheta_4(\mathfrak{b})}{\vartheta_1(2 \mathfrak{b})} E_1 \begin{bmatrix}
    -1 \\ b  
  \end{bmatrix} \ . \nonumber
\end{align}
In the following we consider the index in the presence of a half Wilson line operator in the spin-$j$ representation. The index is then given by the integral
\begin{align}
	\langle W_j\rangle =
  \oint_{|a| = 1} \frac{da}{2\pi i a} \left[\sum_{m = - j}^{j} a^{2m}\right]
  \mathcal{Z}(a)\ .
\end{align}
Here the spin-$j$ character is given by $\chi_j(a) = \sum_{m = -j}^j a^{2m}$.

To proceed, we note that there are a collection of poles from the elliptic integrand,
\begin{align}
	\mathfrak{a}_{k\ell}^\pm = \pm \frac{\mathfrak{b}}{2} + \frac{(2k + 1)\tau}{4} + \frac{\ell}{2}, \qquad
  k, \ell = 0, 1 \ .
\end{align}
Due to the presence of $\tau/4$, all these poles are imaginary, with essentially the same residues
\begin{align}
  R^\pm_{k\ell} = \mp \frac{i}{4} \frac{\vartheta_4(\mathfrak{b})}{\vartheta_1(2 \mathfrak{b})} \ .
\end{align}
Applying the integral formula (\ref{integration-formula-monomial}), the index reads
\begin{align}
  \langle W_j\rangle = \mathcal{I}_{\mathcal{N} = 4}\delta_{j \in \mathbb{Z}} - \frac{i}{4} \frac{\vartheta_4(\mathfrak{b})}{\vartheta_1(2 \mathfrak{b})}\sum_{\substack{m = -j \\ m \ne 0}}^{j}\sum_{k, \ell = 0, 1} \frac{(-1)^{2\ell m} (b^m - b^{-m})q^{( - \frac{1}{2} + k )m}}{q^{m} - q^{-m}} \ .
\end{align}
Note that for $j \in \mathbb{Z}$, the character $\chi_j(a)$ contains a constant term $1$, which lead to the original Schur index $\mathcal{I}_{\mathcal{N} = 4}$. In fact, when $j \in \mathbb{Z} + \frac{1}{2}$, the entire expression vanishes identically thanks to the summation over $\ell = 0, 1$. Therefore, we have
\begin{align}
  \langle W_{j \in \mathbb{Z}}\rangle
  = + \mathcal{I}_{\mathcal{N} = 4} - \frac{i}{2} \frac{\vartheta_4(\mathfrak{b})}{\vartheta_1(2 \mathfrak{b})}\sum_{\substack{m = -j \\ m \ne 0}}^{j} \frac{b^m - b^{-m}}{q^{m/2} - q^{-m/2}} \ ,
  \qquad
  \langle W_{j \in \mathbb{Z} + \frac{1}{2}}\rangle = 0 \ .
\end{align}
The first term $\mathcal{I}_{\mathcal{N} = 4} = \operatorname{ch}_0$ is identified with the vacuum character of the associated chiral algebra $\mathbb{V}_{\mathcal{N} = 4}$. The factor $\frac{i\vartheta_4(\mathfrak{b})}{\vartheta_1(2 \mathfrak{b})}$ in the second term is the residue of the integrand $\mathcal{Z}$ which is related to the Schur index of Gukov-Witten type surface defect in the $\mathcal{N} = 4$ theory \cite{Pan:2021ulr}. It can be shown to satisfy $\frac{i\vartheta_4(\mathfrak{b})}{\vartheta_1(2 \mathfrak{b})} = \operatorname{ch}_0 + \operatorname{ch}_M$ where $M$ is another irreducible module $M$ of $\mathbb{V}_{\mathcal{N} = 4}$ \cite{Adamovic:2014lra,Bonetti:2018fqz,Pan:2021ulr}. As module characters of $\mathbb{V}_{\mathcal{N} = 4}$, both $\operatorname{ch}_0$ and $\operatorname{ch}_M$ satisfy the flaovred modular differential equations arising from null states in $\mathbb{V}_{\mathcal{N} = 4}$ \cite{Gaberdiel:2008pr,Gaberdiel:2009vs,Beem:2017ooy,Pan:2021ulr,Zheng:2022zkm}. Therefore, the line index can be written as a combination of the two irreducible characters,
\begin{align}
  \langle W_{j \in \mathbb{Z}}\rangle
  = \bigg(1 - \frac{1}{2}\sum_{\substack{m = -j \\ m \ne 0}}^{+j}\frac{b^m - b^{-m}}{q^{m/2} - q^{- m /2}}\bigg)\operatorname{ch}_0
  - \frac{1}{2}\Big(\sum_{\substack{m = -j \\ m \ne 0}}^{+j}\frac{b^m - b^{-m}}{q^{m/2} - q^{- m /2}}\Big) \operatorname{ch}_M \ .
\end{align}
Note however that the coefficients of the linear combination are rational functions of $b$ and $q$.





\subsubsection{`t Hooft line index}

In the 4d $\mathcal{N} = 4$ SYM (and in general $\mathcal{N} = 2$ superconformal gauge theories), one can define `t Hooft line operators by specifying certain singular profile for the gauge field and scalars in the path integral. By the Dirac quantization condition, the magnetic charge $B$ of a `t Hooft operator is valued in the cocharacter lattice $\Lambda_\text{cochar}$ inside the Cartan $\mathfrak{h}$ of the gauge group $G$. This lattice $\Lambda_\text{cochar}$ corresponds to the weights of the Langland dual group $G^\vee$, and therefore a dominant integral element $B$ corresponds to a $G^\vee$-representation $\mathcal{R}^\vee_B$. The cocharacters as weights in $\mathcal{R}^\vee_B$ are obtained from $B$ by subtracting suitable coroot element $\alpha^\vee$, and weights related by the Weyl group $W$ of the gauge group $G$ are identified. A weight $v$ in $\mathcal{R}^\vee_B$ that is not Weyl-related to $B$ can screen the `t Hooft operator and signals monopole bubbling effect \cite{Lee:1996vz,Gomis:2009ir,Ito:2011ea,Brennan:2018yuj}.

Under S-duality, a full Wilson line in a $\mathcal{N} = 4$ SYM is mapped to a `t Hooft line. If the magnetic charge of a 't Hooft operator corresponds to a minuscule representation of $G^\vee$, then its index is safe from monopole bubbling effect, and the index can be computed by a relatively simple contour integral \cite{Gang:2012yr}. In particular, For the $\mathcal{N} = 4$ $U(2)$ theory, the 't Hooft line with minimal magnetic charge $(1,0)$ corresponds to a minuscule representation, and is dual to the a full Wilson operator in the fundamental representation. The `t Hooft index can be written as a contour integral \cite{Gang:2012yr},
\begin{align}\label{U2-t-hooft}
  \langle H_{(1,0)}^\text{full} \rangle
  = - \oint \frac{da}{2\pi i a} \frac{(a - b)(-1 + a b)}{(\sqrt{q} - a)(-1 + \sqrt{q}a)b}
  \frac{\eta(\tau)^6 \vartheta_4(\mathfrak{a})^2}{
    \vartheta_1(\mathfrak{a} - \mathfrak{b})
    \vartheta_1(\mathfrak{a} + \mathfrak{b})
    \vartheta_4(\mathfrak{b})^2
  } \ .
\end{align}
Note that the parameters and integration variables have been renamed and reorganized compared to the double contour integral in \cite{Gang:2012yr}. In series expansion,
\begin{align}
  \langle H_{(1,0)}^\text{full}\rangle = 1 + 2(b + b^{-1})\sqrt{q}
  + (1 + 3b^2 + 3b^{-2}) q
  + 4(b^3 + b^{-3})q^{3/2} + \cdots \ .
\end{align}
The ratio of $\vartheta$ functions in $\langle H^\text{full}\rangle$ are essentially identical to the original integrand that computes $\mathcal{I}_{\mathcal{N} = 4}$, up to a shift from $\vartheta_{1, 4} \to \vartheta_{4,1}$. It is therefore elliptic in $\mathfrak{a}$, with real poles $\mathfrak{a} = \pm \mathfrak{b}$. The rational factor in the integrand can also be expanded in the $SU(2)$ characters,
\begin{align}
  - \frac{(a-b)(-1 + ab)}{(\sqrt{q} - a)(-1 + \sqrt{q}a)}
  = (1 + b^2) \sum_{n = 0}^{+\infty}q^{\frac{n}{2}}\chi_{j = \frac{n}{2}}(a)
  - b \sum_{n = 0}^{+\infty}q^{n/2}
  \chi_{j = \frac{1}{2}}(a)\chi_{j = \frac{n}{2}}(a) \\
  = (1 + b^2) \sum_{n = 0}^{+\infty}q^{n/2}\chi_{j = \frac{n}{2}}(a)
  - b \sum_{n = 0}^{+\infty}q^{n/2}
  \chi_{j = \frac{n}{2} + \frac{1}{2}}(a)
  - b \sum_{n = 0}^{+\infty}q^{n/2}
  \chi_{j = \frac{n}{2} - \frac{1}{2}}(a) \ . 
\end{align}
Therefore, the integral $\langle H^\text{full}\rangle$ can be computed directly and exactly using (\ref{integration-formula-χf}). In this case, the residues of two real poles $a = b^{\pm}$ are given by
\begin{align}
  R_\pm = \pm \frac{i \eta(\tau)^3}{\vartheta_1(2\mathfrak{b})} \ .
\end{align}
After some algebra, we have
\begin{align}
  \langle H_{(1,0)}^\text{full}\rangle
  = \frac{i \eta(\tau)^3}{\vartheta_1(2\mathfrak{b})}
  (q^{\frac{1}{2}} & \ + q^{-\frac{1}{2}} - b - b^{-1})
  \sum_{n = 0}^{+\infty}
  \sum_{\substack{m = - n/2 \\ m \ne 0}}^{+ n/2}
  q^{\frac{n}{2}} \frac{b^{2m} - b^{-2m}}{1 - q^{-2m}}
  \nonumber\\
  & \ + \frac{2(b + b^{-1} - 2q^{\frac{1}{2}})}{1-q} \frac{i \eta(\tau)^3}{\vartheta_1(2\mathfrak{b})} E_1 \begin{bmatrix}
    -1 \\ b  
  \end{bmatrix} \ ,
\end{align}
where in the second line we applied
\begin{align}
  \oint \frac{da}{2\pi i a}\frac{\eta(\tau)^6 \vartheta_4(\mathfrak{a})^2}{
    \vartheta_1(\mathfrak{a} - \mathfrak{b})
    \vartheta_1(\mathfrak{a} + \mathfrak{b})
    \vartheta_4(\mathfrak{b})^2
  } = \frac{2i \eta(\tau)^3}{\vartheta_1(2 \mathfrak{b})} E_1 \begin{bmatrix}
    -1 \\ b  
  \end{bmatrix} \ .
\end{align}

The dual Wilson operator index can be computed a lot more easily with (\ref{integration-formula-monomial}),
\begin{align}
  \langle W^\text{full}_{j = 1/2}\rangle
  = & \ - \frac{1}{2}\frac{\eta(\tau)^6}{\vartheta_4(\mathfrak{b})^2}
  \oint_{|a| = 1} \frac{da}{2\pi i a}
  (a + \frac{1}{a})^2 \frac{
    \vartheta_1(2\mathfrak{a})\vartheta_1(- 2\mathfrak{a})
  }{
    \vartheta_4(2\mathfrak{a} + \mathfrak{b})
    \vartheta_4(-2\mathfrak{a} + \mathfrak{b})
  } \\
  = & \ \langle W_{j = 1}\rangle_{U(2)} + \mathcal{I}_{\mathcal{N} = 4 \ U(2)}
  = q^{-\frac{1}{2}}
  \frac{i\eta(\tau)^3}{\vartheta_4(\mathfrak{b})}
  \frac{\vartheta_4(\mathfrak{b})}{\vartheta_1(2\mathfrak{b})}\left(
  2E_1\begin{bmatrix}
    -1 \\b  
  \end{bmatrix} -  \frac{b -b^{-1}}{q^{1/2} - q^{-1/2}}
  \right) \ . \nonumber
\end{align}
As required by S-duality, $\langle W^\text{full}_{j = 1/2}\rangle = \langle H_{(1,0)}^\text{full}\rangle$. This equality indeed follows analytically from the identity (\ref{E1-expansions}). Stripping off the $U(1)$ vector multiplet and the free hypermultiplet contribution $\eta(\tau)^3/\vartheta_4(\mathfrak{b})$, both the full Wilson index and the `t Hooft index are linear combinations of two $\mathcal{V}_{\mathcal{N} = 4}$ characters with rational coefficients, so schematically
\begin{align}
  \langle W_{j = 1}^\text{full} \rangle = A \operatorname{ch}_0 + B \operatorname{ch}_M, \qquad
  \langle H_{(1,0)}^\text{full}\rangle = C \operatorname{ch}_0 + D \operatorname{ch}_M \ .
\end{align}
However, the S-duality $\langle W^\text{full}_{j = 1/2}\rangle = \langle H_{(1,0)}^\text{full}\rangle$ is not because $A = C, B = D$; instead, the $S$-duality induces some highly nontrivial mixing between the vacuum and the $M$ module contributions.



Let us also consider `t-Hooft operators with non-minimal charge $B = (2,0)$. In this case, the index receives contribution from monopole bubbling with $v = (1,1)$, and is expected to equal the $U(2)$ Wilson index in the tensor product of fundamental representation. The `t Hooft index reads
\begin{align}
  \langle H^\text{full}_{(2,0)}\rangle
  = q^{-1/2} \oint \frac{da}{2\pi i a} \mathcal{Z}(a) \frac{\eta(\tau)^6}{\vartheta_4(\mathfrak{b})^2} \frac{\vartheta_1(\mathfrak{a})^2}{\vartheta_4(\pm \mathfrak{a} + \mathfrak{b})} \ ,
\end{align}
where
\begin{align}
  \mathcal{Z}(a)
  = \frac{(1 - \frac{\sqrt{q}}{ab}) (1 - \frac{a\sqrt{q}}{b})}{(1 - \frac{1}{a})(1 - a)}& \ \frac{(1 - \frac{b \sqrt{q}}{a})(1 - a b \sqrt{q})}{(1 - \frac{q}{a})(1 - aq)} \nonumber \\
  & \ + \frac{1}{2} \left[\frac{(q - 1)^2 + (b + \frac{1}{b})\sqrt{q}(1 + q) - 2q (a + \frac{1}{a}) }{(1 - \frac{q}{a})(1 - a q)}\right]^2 \ .
\end{align}
Note that
\begin{align}
  \frac{1}{(1 - \frac{q}{a})(1 - aq)} = \sum_{j \in \frac{1}{2}\mathbb{N}}q^{2j} \chi_j(a) \ ,\quad
  (1 - \frac{b^\pm \sqrt{q}}{a})(1 - a b^\pm \sqrt{q}) = (1 + b^{\pm2} q) - b^\pm q^{\frac{1}{2}} \chi_{\frac{1}{2}}(a) \ . \nonumber
\end{align}
Inserting these expansion, we have
\begin{align}
  \mathcal{Z}
  = & \ \frac{1}{(1-z)(1-1/z)} \left[A - B\chi_{1/2}(a) + q \chi_1(a)
  \right]\sum_{j\in \frac{1}{2}\mathbb{N}} q^{2j}\chi_{j}(a) \nonumber \\
  & \ + \Big[
  4q^2(1 + \chi_1 (a)) - C^2 - 2C q \chi_{\frac{1}{2}}(a)
  \Big]\sum_{j, j', j'' \in \frac{1}{2}\mathbb{N}} q^{2(j + j')}N_{j j'}^{j''} \chi_{j''}(a) \\
  \coloneqq & \ \frac{1}{(1-a)(1-1/a)} \sum_{j \in \frac{1}{2}\mathbb{N}} \mathcal{Z}_j \chi_j(a) + \sum_{j \in \frac{1}{2}\mathbb{N}}\mathcal{Z}_j'\chi_j(a),
\end{align}
where
\begin{align}
  A & \ \coloneqq (1 + b^2q)(1 + \frac{q}{b^2}) + q, 
  & B \coloneqq & \ (b + b^{-1})\sqrt{q}(1+q)\\
  C & \ \coloneqq (q-1)^2 + (b + \frac{1}{b}) \sqrt{q}(1 + q), 
  & \chi_{J}(a)\chi_{J'}(a) = & \ \sum_{J''}N_{JJ'}^{J''} \chi_{J''}(a) \ ,
\end{align}
and $\mathcal{Z}_j$, $\mathcal{Z}'_j$ are polynomials of $b, q$ from applying the tensor product rule for the $SU(2)$ characters,
\begin{align}
   \sum_{j \in \frac{1}{2}\mathbb{N}}\mathcal{Z}_j \chi_j(a)  = & \ [A - B \chi_{1/2}(a) + q \chi_1(a)] \sum_{j \in \frac{1}{2}\mathbb{N}}q^{2j} \chi_j(a)\\
   \sum_{j \in \frac{1}{2}\mathbb{N}}\mathcal{Z}'_j \chi_j(a) = & \ \Big[
     4q^2(1 + \chi_1 (a)) - C^2 - 2C q \chi_{\frac{1}{2}}(a)
     \Big]\sum_{j, j', j'' \in \frac{1}{2}\mathbb{N}} q^{2(j + j')}N_{j j'}^{j''} \chi_{j''}(a) \ ,
\end{align}
while their explicit expressions will be left implicit. Plugging this expansion into the integral, we have
\begin{align}
  \langle H^\text{full}_{(2,0)} \rangle
  = & \ \frac{i \eta(\tau)^3}{\vartheta_1(2 \mathfrak{b})}\sum_{j \in \frac{1}{2}\mathbb{N}} \mathcal{Z}_j \left(
  - \lfloor (j + \frac{1}{2})^2 \rfloor 2E_1 \begin{bmatrix}
    -1 \\ b  
  \end{bmatrix}
  + \sum_{m = -j}^{+j}\sum^{+\infty}_{\substack{k = 0 \\ k+2m \ne 0}}
  \frac{k(b^{k + 2m} - b^{-k -2m})}{q^{\frac{k}{2} + m} - q^{- \frac{k}{2} - m}}
  \right) \nonumber \\
  & \ + \frac{i \eta(\tau)^3}{\vartheta_1(2\mathfrak{b})}\sum_{j \in \frac{1}{2} \mathbb{N}} \mathcal{Z}'_j \left(
    \delta_{j \in \mathbb{Z}} 2E_1 \begin{bmatrix}
      -1 \\ b  
    \end{bmatrix}
    - \sum_{\substack{m = -j \\ m \ne 0}}^{+j} \frac{b^{2m} - b^{-2m}}{q^m - q^{-m}}
  \right) \ .
\end{align}
Unfortunately, we are unable to recast the expression to a more elegant form, therefore we do not prove $\langle W_{\mathbf{2} \otimes \mathbf{2}}^\text{full}\rangle_{U(2)} = \langle H_{(2,0)}^\text{full}\rangle$ analytically. Still, once the free contribution $\eta(\tau)^3/\vartheta_4(\mathfrak{b})$ is removed, the index $\langle H^\text{full}_{(2,0)}\rangle$ remains a linear combination of $\mathbb{V}_{\mathcal{N} = 4}$ characters.



\subsection{\texorpdfstring{$SU(2)$ theory with four flavors}{}}

Next we consider the $\mathcal{N} = 2$ $SU(2)$ gauge theory with four fundamental flavors. In terms of the class-$\mathcal{S}$ description, the theory is associated to the four puncture sphere $\Sigma_{0,4}$ and it admits three weak coupling limits corresponding to three different pants-decompositions. For any such limit, we can insert a half or full Wilson line operator of the $SU(2)$ gauge group in the spin-$j$ representation. The half Wilson index can be computed by the following integral,
% Figure environment removed
\begin{align}
  \langle W_j\rangle_{0,4} = - \frac{1}{2} \oint \frac{da}{2\pi i z} \left[\sum_{m = -j}^{j} a^{2m}\right] \frac{da}{2\pi i a}
  \vartheta_1(2\mathfrak{a}) \vartheta_1(-2\mathfrak{a})
  \prod_{j = 1}^{4} \frac{\eta(\tau)^2}{\vartheta_1(\mathfrak{a} + \mathfrak{m}_j)
  \vartheta_1(- \mathfrak{a} + \mathfrak{m}_j)} \ .
\end{align}
The poles of the integrand are all imaginary, given by $\mathfrak{a}_i^\pm = \pm \mathfrak{m}_i + \frac{\tau}{2}$ with residues
\begin{align}
  R_{i, \pm} = \pm \frac{i}{2} \frac{\vartheta_1(2 \mathfrak{m}_i)}{\eta(\tau)}
  \prod_{\ell \ne i} \frac{\eta(\tau)}{\vartheta_1(\mathfrak{m}_i + \mathfrak{m}_\ell) \vartheta_1(\mathfrak{m}_i - \mathfrak{m}_\ell)}
  \coloneqq \pm R_i
\end{align}
Applying the integration formula (\ref{integration-formula-monomial}), we have
\begin{align}\label{Wilson-index-SQCD}
  \langle W_j \rangle_{0,4} =  & \ \mathcal{I}_{0,4}\delta_{j \in \mathbb{Z}} - \sum_{\substack{m = - j \\ m \ne 0}}^{+ j} \sum_{\pm} \sum_{i = 1}^4 R_{i, \pm} \frac{1}{q^{2m} - 1} (b_i^\pm q^{\frac{1}{2}})^{2m} \nonumber\\
  = & \ \mathcal{I}_{0,4}\delta_{j \in \mathbb{Z}}
  - \sum_{i = 1}^{4} \left(\sum_{\substack{m = - j \\ m \ne 0}}^{+j} \frac{M_i^{2m} - M_i^{-2m}}{q^{m} - q^{-m}}\right)R_i \ , 
\end{align}
where $M_i \coloneqq e^{2\pi i \mathfrak{m}_i}$. The theory is of class-$\mathcal{S}$ associated to the four-punctured sphere. The $SU(2)^4$ fugacities $b_i$ are related to the $m_i$ by
\begin{align}
  M_1 = b_1 b_2, \quad
  M_2 = b_1/b_2, \quad
  M_3 = b_3 b_4, \quad
  M_4 = b_3/b_4 \ .
\end{align}

In \cite{Cordova:2016uwk}, several Wilson line index in $SU(2)$ SQCD were computed, and the results can be organized as linear combinations of the infinitely many highest weight characters $\chi_{[m, n, 0,0,0]}$ of $\widehat{\mathfrak{so}}(8)_{-2}$ which were obtained from the Kazhdan-Lusztig formula \cite{Lusztig1979}. Our new computation improves the result and relates all $\langle W_j\rangle_{0,4}$ to just five highest weight characters, with respect to finite weights $\lambda = 0, -2 \omega_1, - \omega_2, - 2 \omega_3, -2 \omega_4$, of the simple vertex operator algebra $\widehat{\mathfrak{so}}(8)_{-2}$ \cite{Arakawa:2015jya,Arakawa:2016hkg}. Indeed, the four residues $R_i$ in the above are related to the Schur index of Gukov-Witten type surface defects, and also to the the module characters \cite{Peelaers,Pan:2021mrw,2023arXiv230409681L,Pan:2023jjw,Arai:2020qaj},
\begin{align}
  \operatorname{ch}_{-2\widehat \omega_1} = & \ \operatorname{ch}_0 - 2R_1\\
  \operatorname{ch}_{-\widehat \omega_2} = & \ -2 \operatorname{ch}_0 + 2R_1 + 2R_2\\
  \operatorname{ch}_{-2\widehat \omega_3} = & \ \operatorname{ch}_0 - R_1 - R_2 - R_3 - R_4\\
  \operatorname{ch}_{-2\widehat \omega_4} = & \ \operatorname{ch}_0 - R_1 - R_2 - R_3 + R_4 \ ,
\end{align}
where $\operatorname{ch}_0$ is the vacuum character of $\widehat{\mathfrak{so}}(8)_{-2}$, identified with the Schur index $\mathcal{I}_{0,4}$. Therefore, one may write the half Wilson line index as a linear combination of the five module characters,
\begin{align}
  \langle W_j\rangle_{0,4}
  = (\delta_{j \in \mathbb{Z}} - \frac{1}{2}\mathcal{M}_{1j} - \frac{1}{2}\mathcal{M}_{2j})\operatorname{ch}_0
  + & \ \frac{1}{2}(\mathcal{M}_{1j} - \mathcal{M}_{2j})\operatorname{ch}_1
  + \frac{1}{2}(\mathcal{M}_{3j} - \mathcal{M}_{2j})\operatorname{ch}_2 \nonumber\\
  & \ + \frac{1}{2}(\mathcal{M}_{3j} + \mathcal{M}_{4j})\operatorname{ch}_3
  + \frac{1}{2}(\mathcal{M}_{3j} - \mathcal{M}_{4j})\operatorname{ch}_4 \ , \nonumber
\end{align}
where we define the rational functions
\begin{align}
  \mathcal{M}_{ij} \coloneqq \sum_{\substack{m = - j\\m \ne 0}}^{+j}\frac{M_i^{2m} - M_i^{-2m}}{q^m - q^{-m}} \ .
\end{align}



With the half-Wilson index, the index of a full Wilson line operator in the fundamental representation is then given by
\begin{align}
  \langle W_{j = \frac{1}{2}}^\text{full}\rangle_{0,4} = \langle W_{j = \frac{1}{2}} W_{j = \frac{1}{2}}\rangle_{0,4}
  = \mathcal{I}_{0,4} + \langle W_{j = 1}\rangle_{0,4} \ .
\end{align}
By S-duality, this Wilson operator is mapped to the `t Hooft operator with a minimal magnetic charge $B = (-1, 1)$ which receives contribution from monopole bubbling \cite{Gang:2012yr}. The `t Hooft index is given by a slightly more involved contour integral,
\begin{align}
  \langle H_{1,-1}\rangle_{0,4}
  = & \ \oint \frac{da}{2\pi i a} \frac{2q^{\frac{5}{12}}\prod_{i =1}^{4}(a - M_i)(-1 + aM_i)}{(-1 + a^2)^2 (a^2 - q)(-1 + a^2 q) \prod_{i = 1}^{4}M_i}
  \left(- \frac{1}{2}\vartheta_1(\pm 2 \mathfrak{a})\right) \prod_{i = 1}^{4}\frac{\eta(\tau)^2}{\vartheta_1(\pm \mathfrak{a} + M_i)} \nonumber \\
  & \ + q^{-\frac{7}{12}} \oint \frac{da}{2\pi i a} Z_\text{mono}\left(- \frac{1}{2}\vartheta_1(\pm 2 \mathfrak{a})\right) \prod_{i = 1}^{4}\frac{\eta(\tau)^2}{\vartheta_4(\pm \mathfrak{a} + M_i)} \ ,
\end{align}
where
\begin{align}
  Z_\text{mono} = \frac{1}{q \prod_{i =1}^{4}M_i}\left[
  - \left( q + \prod_{i = 1}^{4}M_i\right)
  + \sum_{\pm}\frac{\prod_{i = 1}^{4}(q^{\frac{1}{2}}a^\pm - M_i)}{(1 - a^{\pm 2}) (1 - q a^{\pm 2})}
  \right]^2 \ .
\end{align}
We can rewrite
\begin{align}
  & \ \frac{2q^{\frac{5}{12}}\prod_{i =1}^{4}(a - M_i)(-1 + aM_i)}{(-1 + a^2)^2 (a^2 - q)(-1 + a^2 q) \prod_{i = 1}^{4}M_i} \nonumber \\
  = & \ \frac{2q^{\frac{5}{12}}}{(1-a^2)(1-a^{- 2})}
  \left[\sum_{J \in \mathbb{N}}q^{J}\sum_{j = 0}^{J}(-1)^j\chi_{J - j}(a)\right]
  \prod_{i =1}^{4} \Big(  \chi_{1/2}(a) - \chi_{1/2}(M_i)  \Big) \nonumber\\
  \coloneqq & \ \frac{2q^{\frac{5}{12}}}{(1-a^2)(1-a^{- 2})} \sum_{J \in \frac{1}{2}\mathbb{N}} \mathcal{Z}_J\chi_J(a) \ ,  \\
  Z_\text{mono} = & \ \frac{1}{q \prod_{i = 1}^{4}M_i}
  \left[
  \sum_{j \in \frac{1}{2}\mathbb{N}}(-1)^{2j + 1}\chi_j(a) g_j(M) q^{1 + j}
  \right]^2 \coloneqq \sum_{J \in\frac{1}{2} \mathbb{N}} \mathcal{Z}'_J \chi_J(a) \ .
\end{align}
where
\begin{align}
  g_{J \in \mathbb{N}}(M) \coloneqq 1 + \sum_{\substack{i, j = 1 \\ i < j}}^4M_i M_j + \prod_{i = 1}^{4}M_i \ ,\quad
  g_{J \in \mathbb{N} + \frac{1}{2}}(M) \coloneqq \sum_{i = 1}^{4}M_i + \sum_{\substack{i,j,k = 1\\i < j < k}}^{4}M_i M_j M_k \ ,
\end{align}
and $\mathcal{Z}_J$ and $\mathcal{Z}'_J$ are rational function of $q$ and fugacities $M$ which simply follow from expanding tensor product of $SU(2)$ irreps; their explicit form will be left implicit. Therefore, we have the exact formula for the `t-Hooft index,
\begin{align}
  \langle H_{1, -1}\rangle_{0,4}
  = & \ \sum_{J \in \frac{1}{2}\mathbb{N}}\mathcal{Z}_J
  \sum_{m = -J}^{+J}\sum_{\substack{k = 1 \\ 2k + 2m \ne 0}}^{+\infty}\left[\sum_{i, \pm}R_{i} \frac{k(M_i^{2k + 2m} - M_i^{ - 2k - 2m})}{q^{\frac{2k + 2m}{2}} - q^{- \frac{2k + 2m}{2}}}
      + \frac{2m}{2}\delta_{\frac{2m}{2} \in \mathbb{Z}_{< 0}} \mathcal{I}_{0,4}\right] \nonumber \\
  & \ + \sum_{J \in \frac{1}{2}  \mathbb{N}} \mathcal{Z}'_J\sum_{m = -J}^{+J} 
  \sum_{i = 1}^4 R_{i} \frac{M_i^{2m} - M_i^{-2m}}{q^m - q^{-m}} \ .
\end{align}
Unfortunately we are unable to reorganize the expression into a more elegant form. Therefore we do not further compare analytically between this `t-Hooft index with the corresponding Wilson index. Although fairly complicated, the expression $\langle H_{1, -1}\rangle_{0,4}$ remain explicitly a linear combination of $\widehat{\mathfrak{so}}(8)_{-2}$ characters, with rational functions in $b_i, q$ as the coefficients.




\subsection{Genus-one theory with two punctures \label{section:genus-one-two-punctures}}

Let us consider a higher rank theory with $g = 1$ and $n = 2$, which can be obtained by gauging a diagonal $SU(2) \times SU(2)$ subgroup of the flavor symmetry of two copies of trinion theories $\mathcal{T}_{0,3}$. There are essentially two different weak-coupling frames one can consider, and here we focus on the frame illustrated in Figure \ref{fig:genus-one-type-1}. In this frame, the original Schur index is given as a contour integral
\begin{align}
  \mathcal{I}_{1,2} 
  = \oint \prod_{i = 1}^{2}\frac{da_i}{2\pi i a_i}
  \prod_{j = 1}^{2}\prod_{\pm \pm} \frac{\eta(\tau)}{\vartheta_4(\mathfrak{b}_j \pm \mathfrak{a}_1 \pm \mathfrak{a}_2)}
  \prod_{i = 1}^{2}\left(- \frac{1}{2}\vartheta(\pm 2 \mathfrak{a}_i)\right)
  \coloneqq \oint \left[\frac{da}{2\pi i a}\right]\mathcal{Z}_{1,2}(a) \ .
\end{align}
Let us consider a half Wilson line operator associated to one of the $SU(2)$ gauge group, whose index is given by the integral
\begin{align}
  \langle W_j\rangle_{1, 2}^{(1)} = \oint \prod_{i = 1}^{2}\frac{da_i}{2\pi i a_i}
  \left(\sum_{m = - j}^{j}a_1^{2m}\right)
  \prod_{j = 1}^{2}\prod_{\pm \pm} \frac{\eta(\tau)}{\vartheta_4(\mathfrak{b}_j \pm \mathfrak{a}_1 \pm \mathfrak{a}_2)}
  \prod_{i = 1}^{2}\left(- \frac{1}{2}\vartheta(\pm 2 \mathfrak{a}_i)\right)\ .
\end{align}

% Figure environment removed

The integral can be evaluated in two different orders: first $a_1$  or first $a_2$. We choose to integrate over $a_1$ first, where the relevant poles are $ \mathfrak{a}_1 = \alpha \mathfrak{b}_j + \beta \mathfrak{a}_2 + \frac{\tau}{2}$ with residues (where $\alpha, \beta = \pm 1$)
\begin{align}
  R_{i \alpha \beta} = \frac{
    i \eta(\tau)^5 \vartheta_1(2 \beta \mathfrak{a}_2) \vartheta_1( 2 \beta \mathfrak{a}_2 + 2 \alpha \mathfrak{b}_i)
  }
  {
  4\vartheta_1(2 \alpha \mathfrak{b}_i)
  \vartheta_1(\alpha \mathfrak{b}_i - \beta \mathfrak{b}_{3-i})
  \vartheta_1(\alpha \mathfrak{b}_i + \beta \mathfrak{b}_{3-i})
  \vartheta_1(2 \mathfrak{a}_2 + \alpha \beta \mathfrak{b}_i - \mathfrak{b}_{3-i})
  \vartheta_1(2 \mathfrak{a}_2 + \alpha \beta \mathfrak{b}_i + \mathfrak{b}_{3-i}) 
  } \ , \nonumber
\end{align}
The $a_1$ integral leaves integrals of the form
\begin{align}
  \oint \frac{da_2}{2\pi i a_2} f(\mathfrak{a}_2) a_2^{n} \ ,
\end{align}
which can be carried out using formula \eqref{integration-formula-monomial}. Finally, the index in the presence of the Wilson line operator gives
\begin{align}
  \langle W_{j \in \mathbb{Z}}\rangle_{1, 2}
  = \mathcal{I}_{1,2}
     + \frac{\eta(\tau)^2}{2 \prod_{i = 1}^2 \vartheta_1(2\mathfrak{b}_i)}
     \sum_{\substack{m = - j \\ m \ne 0}}^{+j}
     \frac{\prod_{i = 1}^{2}(b_i^m - b_i^{-m})}{(q^{m/2} - q^{-m/2})^2}
     \ ,
  \quad
  \langle W_{j \in \mathbb{Z} + \frac{1}{2}}\rangle = 0 \ .
\end{align}
The result is symmetric in $b_1, b_2$ as expected. Note that the first term is clearly the vacuum character of the associated chiral algebra of $\mathcal{T}[\Sigma_{1,2}]$. The factor $\eta(\tau^2)/\prod_{i = 1}^2 \vartheta_1(2 \mathfrak{b}_i)$ arises as the unique\footnote{One can try different nested residues, but they are either zero or proportional to $\eta(\tau^2)/\prod_{i = 1}^2 \vartheta_1(2 \mathfrak{b}_i)$. } nested residue of $\mathcal{Z}_{1,2}(a)$,
\begin{align}
  \operatorname{Res}_{\mathfrak{a}_2 = - \frac{\mathfrak{b}_1 - \mathfrak{b}_2}{2}}\operatorname{Res}_{\mathfrak{a}_1 = \mathfrak{a}_2 + \mathfrak{b}_1 + \frac{\tau}{2}} \mathcal{Z}_{1,2}(\mathfrak{a}_{1,2}) = \frac{\eta(\tau)^2}{8 \vartheta_1(2 \mathfrak{b}_1)\vartheta_1(2 \mathfrak{b}_2)} \ ,
\end{align}
and is also expected to be a linear combination of non-vacuum module character, since it has been shown to satisfy a set of flavored modular differential equations that should annihilate all module characters \cite{zhu1996modular,Zheng:2022zkm}. For example, at weight-two there are two equations
\begin{align}
  0 = \Bigg[
  D_q^{(1)}
  - \frac{1}{4} \sum_{i = 1,2} D_{b_i}^2
  -\frac{1}{4}& \ \sum_{\alpha_i = \pm} E_1 \begin{bmatrix}
    1 \\ b_1^{\alpha_1}b_2^{\alpha_2}
  \end{bmatrix}
  \sum_{i = 1,2}\alpha_i D_{b_i}
  - \sum_{i = 1,2} E_1 \begin{bmatrix}
    1 \\ b_i^2
  \end{bmatrix}D_{b_i} \\
  & \ + 2 \bigg(
  E_2 + \frac{1}{2} \sum_{\alpha_i = \pm}E_2 \begin{bmatrix}
    1 \\ b_1^{\alpha_1}b_2^{\alpha_2}
  \end{bmatrix}
  + \sum_{i = 1,2} E_2 \begin{bmatrix}
    1 \\ b_i^2
  \end{bmatrix}
  \bigg) \Bigg] \mathcal{I}_{1,2} \ ,
\end{align}
and
\begin{align}
  0 = \left(D_{b_1}^2 + 4 E_1 \begin{bmatrix}
    1 \\ b_1^2 
  \end{bmatrix}
  - 8 E_2 \begin{bmatrix}
    1 \\ b_1^2
  \end{bmatrix}\right) \mathcal{I}_{1,2}
  = \left(D_{b_2}^2 + 4 E_1 \begin{bmatrix}
    1 \\ b_2^2 
  \end{bmatrix}
  - 8 E_2 \begin{bmatrix}
    1 \\ b_2^2
  \end{bmatrix}\right)\mathcal{I}_{1,2} \ .
\end{align}








\subsection{\texorpdfstring{Type-1 half Wilson line index in $\mathcal{T}[\Sigma_{g,n}]$}{}}


% Figure environment removed

Now we are ready to consider more general type $A_1$ class-$\mathcal{S}$ theories $\mathcal{T}[\Sigma_{g,n}]$. Any such theory usually admits several weak-coupling limits as different supersymmetric gauge theories. With respect to each gauge theory description, we can introduce a half Wilson operator associated to one of the $SU(2)$ gauge group. In general one can introduce Wilson line charged under multiple $SU(2)$ gauge groups in the weak-coupling description, however, we leave the study of their index and correlation functions to future work.

Let us build on top of the previous $\langle W_j\rangle_{1,2}$ by extending the corresponding Riemann surface to the left and right, while maintaining the location of the Wilson line operator. We simply refer to such construction of Wilson line operator as type-$1$. The resulting configuration is shown in Figure \ref{Wilson-loop-type-1}, and it is clear from the figure that type-1 Wilson line operator encircles a tube that when cut the Riemann surface $\Sigma_{g, n}$ remain connected. Put differently, the type-1 Wilson operator can be constructed from a single connected Riemann surface $\Sigma_{g, n + 2}$ where one glues two punctures and simultaneous inserts a Wilson operator at the tube. In this subsection we will prove that the index os type-1 Wilson line operator in the spin-$j$ representation is given by
\begin{align}\label{Wilson-index-1-general}
  \langle W_{j \in \mathbb{Z}}\rangle^{(1)}_{g \ge 1, n}
  = & \ \mathcal{I}_{g,n}
  - \frac{1}{2}\left[
    \prod_{i = 1}^{n} \frac{i \eta(\tau)}{\vartheta_1(2 \mathfrak{b}_i)}
  \right]
    \sum_{\substack{m = - j\\m \ne 0}}^{+ j}
    \left[\frac{\eta(\tau)}{q^{m/2} - q^{-m /2}}\right]^{2g - 2}
    \prod_{i = 1}^{n} \frac{b_i^m - b_i^{-m}}{q^{m/2} - q^{- m /2}} \ ,\\
  \langle W_{j \in \mathbb{Z} + \frac{1}{2}}\rangle^{(1)}_{g \ge 1, n} = & \ 0 \ .
\end{align}
Although in any given gauge theory description of $\mathcal{T}[\Sigma_{g,n}]$ there may be different choices of $SU(2)$ gauge groups to support a half Wilson line, the final index is actually independent of the choice, as long as they are all type-1. Also we emphasize that type-1 Wilson line exists only for genus $g \ge 1$.

The factor $\eta(\tau)^{2g - 2}\prod_{i = 1}^n \frac{\eta(\tau)}{\vartheta_1(2 \mathfrak{b}_i)}$ can be shown to be the unique\footnote{Up to some numerical factors.} nested residue of the integrand $\mathcal{Z}_{g,n}$ that computes the original Schur index. The uniqueness is only true for $g \ge 1$, as we have already encountered four different residues $R_i$ in the $\mathcal{T}[\Sigma_{0,4}]$ computation; in this sense, class-$\mathcal{S}$ theories at $g \ge 1$ seem to enjoy some nicer properties than the $g = 0$ counterparts \footnote{See also \cite{Satoshi:2023}, where Landau-Ginzburg description can be found for $g \ge 1$ $\mathcal{N} = (0,2)$ and $(0,4)$ class-$\mathcal{S}$ theories in two dimensions. It might suggest some subtle difference in the representation theory of associated chiral algebras of the $g = 0$ and $g \ge 1$ cases. It will be interesting to clarify this issue in the future.}. Extrapolating from the discussions in \cite{Zheng:2022zkm,Pan:2021ulr}, it is natural to expect that this factor is a solution to the set of flavored modular differential equations that annihilate the Schur index, namely, the vacuum character of the associated chiral algebra $\chi(\mathcal{T}[\Sigma_{g,n}])$ of $\mathcal{T}[\Sigma_{g,n}]$, and therefore a linear combination (with constant coefficients) of non-vacuum module characters. This implies that the Wilson line index is also a linear combination of $\chi(\mathcal{T}[\Sigma_{g,n}])$ characters, with rational coefficients
\begin{align}
  \sum_{\substack{m = - j\\m \ne 0}}^{+ j}
  \left[\frac{1}{q^{m/2} - q^{-m /2}}\right]^{2g - 2}
  \prod_{j = 1}^{n} \frac{b_j^m - b_j^{-m}}{q^{m/2} - q^{- m /2}}\  .
\end{align}
The closed-form expression is essentially a sum of products of contributions from the punctures and a contribution and a ``three point function'' contribution, which closely resembles that of the $q$-deformed Yang-Mills partition function on $\Sigma_{g, n}$. It would be interesting to match our result in detail with results by punctured network \cite{Watanabe:2016bwr,Watanabe:2017bmi}.


The proof of the index formula (\ref{Wilson-index-1-general}) can be done recursively by assuming at $g \ge 1, n\ge 0$ $\langle W_j\rangle^{(1)}_{g,n}$ is given by the anzatz (\ref{Wilson-index-1-general}). We already know that the above anzatz works for the $g = 1, n = 2$ case. We can compute $\langle W_j\rangle_{g, n + 1}^{(1)}$ by gluing $\mathcal{T}[\Sigma_{0,3}]$ to that associated to $\mathcal{T}[\Sigma_{g, n}]$,
\begin{align}
  \langle W_j \rangle_{g, n + 1}^{(1)} = \oint \frac{da}{2\pi i a} \langle W_j\rangle^{(1)}_{g, n}(\mathfrak{b}_1, \ldots, \mathfrak{b}_{n - 2}, \mathfrak{a}) \mathcal{I}_\text{VM}(\mathfrak{a}) \mathcal{I}_{0,3}( - \mathfrak{a}, \mathfrak{b}_{n - 1}, \mathfrak{b}_n) \ .
\end{align}
Let us compute
\begin{align}
  & \ \langle W_{j \in \mathbb{Z}}\rangle_{g, n + 1} \nonumber\\
  = & \ \oint \frac{da}{2\pi i a}\bigg[\mathcal{I}_{g, n}(\mathfrak{b}_1, \ldots, \mathfrak{b}_{n - 1}, \mathfrak{a}) \mathcal{I}_\text{VM}(\mathfrak{a}) \mathcal{I}_{0,3}(-\mathfrak{a}, \mathfrak{b}_n, \mathfrak{b}_{n + 1}) \nonumber \\
  & \ - \frac{i^n \eta(\tau)^n}{2
    \vartheta_1(2 \mathfrak{a})\prod_{j = 1}^{n - 1} \vartheta_1(2 \mathfrak{b}_j)
  }
    \\
  & \ \qquad \times \sum_{\substack{m = - j \\ m \ne 0}}^{+j}
  \left[\frac{\eta(\tau)}{q^{m/2} - q^{-m/2}}\right]^{2g - 2}
  \frac{\prod_{j = 1}^{n - 1}(b_j^m - b_j^{-m})}{(q^{m/2} - q^{-m/2})^n}
  (a^m - a^{-m})\mathcal{I}_\text{VM}(\mathfrak{a}) \mathcal{I}_{0,3}(-\mathfrak{a}, \mathfrak{b}_n, \mathfrak{b}_{n + 1})  \bigg]\ . \nonumber
\end{align}
The first term clearly gives $\mathcal{I}_{g, n + 1}$. The second integral is of the form (up to irrelevant factors pulled out of the integral)
\begin{align}
  \oint \frac{da}{2\pi i a} \frac{a^m - a^{-m}}{\vartheta_1(2\mathfrak{a})}
  \mathcal{I}_\text{VM}(a)\mathcal{I}_{0,3}(-\mathfrak{a}, \mathfrak{b}_n, \mathfrak{b}_{n + 1})\ .
\end{align}
It is easy to check that 
\begin{align}
  \frac{\mathcal{I}_\text{VM}(a)}{\vartheta_1(2\mathfrak{a})}\mathcal{I}_{0,3}(- \mathfrak{a}, \mathfrak{b}_n, \mathfrak{b}_{n + 1})
\end{align}
is elliptic in $\mathfrak{a}$. Therefore \eqref{integration-formula-monomial} implies that
\begin{align}
  & \ \oint \frac{da}{2\pi i a}(a^m - a^{-m})\frac{\mathcal{I}_\text{VM}(a)}{\vartheta_1(2\mathfrak{a})}\mathcal{I}_{0,3}(- \mathfrak{a}, \mathfrak{b}_1, \mathfrak{b}_2)
  =
  \frac{i \eta(\tau)}{\prod_{j = 1}^2\vartheta_1(2 \mathfrak{b}_j)} \frac{\prod_{j = 1}^{2}(b_j^m - b_j^{-m})}{(q^{m/2} - q^{-m/2})} \ .
\end{align}
In other words we have verified $\langle W_j\rangle_{g, n + 1}^{(1)}$ also satisfies (\ref{Wilson-index-1-general}),
\begin{align}
  \langle W_{j \in \mathbb{Z}}\rangle _{g, n + 1}
  = \mathcal{I}_{g, n + 1}
    - \frac{i^{n + 1}\eta(\tau)^{n + 1}}{2\prod_{j = 1}^{n + 1}\vartheta_1(2 \mathfrak{b}_j)} \sum_{\substack{m = -j\\m \ne 0}}^{+j}
    \left[\frac{\eta(\tau)}{q^{m/2} - q^{-m/2}}\right]^{2g - 2}
    \frac{\prod_{j = 1}^{n+1}(b_j^m - b_j^{-m})}{(q^{m/2} - q^{-m/2})^{n + 1}} \ .
\end{align}


In the direction of increasing genus $g$, one can glue pairs of punctures to obtain Wilson line operator index $\langle W_j\rangle_{g + 1, n}^{(1)}$ for theories of higher genus $g + 1$,
\begin{align}
  \langle W_j \rangle^{(1)}_{g + 1, n} = \oint \frac{da}{2\pi i a} \mathcal{I}_\text{VM}(a) \langle W_j \rangle^{(1)}_{g, n + 2}(\mathfrak{b}_1, \ldots, \mathfrak{b}_n, \mathfrak{a}, - \mathfrak{a}) \ .
\end{align}
Assuming the anzatz hods at genus $g$, we have
\begin{align}
  & \ \langle W_j\rangle^{(1)}_{g + 1, n} \nonumber \\
   = & \ \mathcal{I}_{g + 1, n}
   - \oint \frac{da}{2\pi i a} \frac{1}{2} \frac{i^n \eta(\tau)^n}{\prod_{j = 1}^{n}\vartheta_1(2\mathfrak{b}_j)}
   \frac{i^2 \eta(\tau)^2}{\vartheta_1(\pm 2 \mathfrak{a})}\left(-\frac{1}{2} \vartheta_1(\pm 2 \mathfrak{a})\right)\\
   & \ \qquad \qquad\qquad\times \sum_{\substack{m = - j\\ m \ne 0}}^{+ j}
   \left[\frac{\eta(\tau)}{q^{m/2} - q^{-m /2}}\right]^{2g - 2}
   \frac{(a^{m} - a^{-m})(a^{- m} - a^{+m})}{(q^{m/2} - q^{-m/2})^2}
   \prod_{j = 1}^{n}\frac{b_j^{m} - b_j^{-m}}{q^{m/2} - q^{-m/2}} \ .\nonumber
\end{align}
The two $\vartheta_1(\pm 2 \mathfrak{a})$ factors are cancelled, while
\begin{align}
  {(a^{m} - a^{-m})(a^{- m} - a^{+m})} = - a^{2m} - a^{-2m} + 2 \ .
\end{align}
Only the $+2$ survives the $a$-integration since $m \ne 0$. Hence,
\begin{align}
  \langle W_j\rangle^{(1)}_{g + 1, n}
  = \mathcal{I}_{g + 1, n}
  - \frac{1}{2}\prod_{j = 1}^{n}\frac{i \eta(\tau)}{\vartheta_1(2 \mathfrak{b}_j)}
    \sum_{\substack{m = - j\\ m \ne 0}}^{+ j}
     \left[\frac{\eta(\tau)}{q^{m/2} - q^{- m /2}}\right]^{2(g+1) - 2}
     \prod_{j = 1}^{n}\frac{b_j^{m} - b_j^{-m}}{q^{m/2} - q^{-m/2}} \ ,
\end{align}
proving the index formula (\ref{Wilson-index-1-general}).

\vspace{2em}

The Type-1 Wilson index $\langle W_j\rangle_{g \ge 1,n}^{(1)}$ can be computed in a different approach, by gluing two existing punctures and simultaneously insert a half Wilson operator,
\begin{align}
  \langle W_j\rangle_{g \ge 1, n}
  = \oint \frac{da}{2\pi i a} \chi_j(a) \mathcal{I}_{g - 1, n + 2}(\mathfrak{b}_1, \ldots, \mathfrak{b}_n, \mathfrak{a}, - \mathfrak{a}) \mathcal{I}_\text{VM}(\mathfrak{a}) \ .
\end{align}
Recall that for $g \ge 0, n > 0$, the $A_1$ Schur index is given by
\begin{align}
  \mathcal{I}_{g, n} = & \ \frac{i^n}{2} \frac{\eta(\tau)^{n + 2g - 2}}{\prod_{j = 1}^{n}\vartheta_1(2 \mathfrak{b}_j)}
  \sum_{\vec\alpha = \pm}\Big(  \prod_{j = 1}^{n}\alpha_j  \Big)\sum_{k = 1}^{n + 2g - 2}\lambda_k^{(n + 2g - 2)} E_k\left[\begin{matrix}
    (-1)^n \\ \prod_{j = 1}^{n}b_j^{\alpha_j}
  \end{matrix}\right] \ .
\end{align}
After identifying $\mathfrak{b}_{n + 1} = \mathfrak{a}$, $\mathfrak{b}_{n + 2} = - \mathfrak{a}$ and multiplying the vector multiplet contribution $\mathcal{I}_\text{VM}(\mathfrak{a})$, all the $\vartheta_1(2\mathfrak{a})$ factors cancel out, and the integration variable $a$ is only present inside the Eisenstein series. When $j \in \mathbb{Z}$, the constant term in $\chi_j(a)$ leads to a additive term $\mathcal{I}_{g, n}$. For the terms in $\chi_j(a)$ with non-zero $m$, we proceed with the integration,
\begin{align}
  \oint \frac{da}{2\pi i a} a^{2m} \frac{i^{n + 2}}{2} \frac{\eta(\tau)^{2g - 2 + n}}{\prod_{j = 1}^{n}\vartheta_1(2\mathfrak{b}_j)}
  \sum_{\vec \alpha = \pm 1} \left(\prod_{j = 1}^{n + 2}\alpha_j\right)
  \sum_{k = 1}^{2g - 2 + n}
  \lambda_k^{(2g - 2 + n)}
  E_k \begin{bmatrix}
    (-1)^n \\
    \prod_{j = 1}^{n + 2}b_j^{\alpha_j}
  \end{bmatrix}_{\substack{b_{n + 1} = a\\b_{n + 2} = 1/a}} \ .
\end{align}
Only the terms with $\alpha_{n + 1} = - \alpha_{n + 2} \coloneqq \beta$, such that $b_{n + 1}^{\alpha_{n + 1}}b_{n + 2}^{\alpha_{n + 2}} = a^{2\beta}$, survives the integration since $2m \ne 0$. 

Let us look at cases with even $n$, where the integral becomes
\begin{align}
  % & \ \frac{i^{n}}{2} \frac{\eta(\tau)^{2g - 2 + n}}{\prod_{j = 1}^{n}\vartheta_1(2\mathfrak{b}_j)}
  % \oint \frac{da}{2\pi i a} a^{2m}
  % \sum_{\beta = \pm}\sum_{\vec \alpha = \pm 1} \left(
  % \prod_{j = 1}^{n}\alpha_j\right)
  % \sum_{k = 1}^{2g - 2 + n}
  % \lambda_{k}^{(2g - 2 + n)}E_k \begin{bmatrix}
  %   (-1)^n \\
  %   a^{2\beta}\prod_{j = 1}^{n}b_j^{\alpha_j}
  % \end{bmatrix}\\
  = & \ - \frac{i^{n}}{2} \frac{\eta(\tau)^{2g - 2 + n}}{\prod_{j = 1}^{n}\vartheta_1(2\mathfrak{b}_j)}
  \sum_{k = 1}^{2g - 2 + n}
    \lambda_k^{(2g - 2 + n)}
    \frac{q^m}{(k-1)!}
    \frac{\text{Eu}_{k - 1}(q^m)}{(1 - q^m)^k}
  \prod_{j = 1}^{n}(b_j^{m} - b_j^{-m})\ . \nonumber
\end{align}
where we applied integration formula. Note also that $k$ is even in order for the rational numbers $\lambda$ to be non-vanishing, and
\begin{align}
  \sum_{\vec \alpha = \pm}\left(\prod_{j = 1}^{n}\alpha_j\right)
  \left(
  \frac{1}{\prod_{j = 1}^n b_j^{m\alpha_j}}
  + \prod_{j = 1}^{n}b_j^{m \alpha_j}
  \right)
  = 2 \prod_{j = 1}^{n}(b_j^{m} - b_j^{-m}) \ .
\end{align}
Therefore,
\begin{align}
  \langle W_j\rangle_{g, n}^{(1)}
  = & \ \mathcal{I}_{g, n}\delta_{j \in \mathbb{Z}}
  - \sum_{\substack{m = - j\\m \ne 0}}^{+j} \frac{i^{n}}{2} \frac{\eta(\tau)^{2g - 2 + n}}{\prod_{j = 1}^{n}\vartheta_1(2\mathfrak{b}_j)}\prod_{j = 1}^{n}(b_j^{m} - b_j^{-m})
    \sum_{k = 1}^{2g - 2 + n}
      \lambda_k^{(2g - 2 + n)}
      \frac{q^m}{(k-1)!}
      \frac{\text{Eu}_{k - 1}(q^m)}{(1 - q^m)^k} \nonumber \\
  = & \ \mathcal{I}_{g, n}\delta_{j \in \mathbb{Z}}
  - \frac{1}{2}
    \prod_{i = 1}^{n} \frac{i \eta(\tau)^n}{\vartheta_2(\mathfrak{b}_j)}
    \sum_{\substack{m = - j\\m \ne 0}}^{+j} 
    \frac{\eta(\tau)^{2g - 2}}{(q^{m/2} - q^{-m/2})^{2g - 2}}\prod_{j = 1}^{n}\frac{b_j^{m} - b_j^{-m}}{q^{m/2} - q^{-m/2}} \ ,
\end{align}
where in the second equality we apply the identity (for even $n$)
\begin{align}
  \sum_{k = 1}^{2g -2 + n}\lambda_k^{(2g - 2 + n)} \frac{q^m}{(k-1)!} \frac{\text{Eu}_{k - 1}(q^m)}{(1 - q^m)^k}
  = \frac{1}{(q^{m/2} - q^{-m/2})^{2g - 2 + n}} \ .
\end{align}


A similar computation can be carried out with odd $n$. Again, an $m \ne 0$ term integrates to
\begin{align}
  = + \frac{i^n}{2} \frac{\eta(\tau)^{2g - 2 + n}}{\prod_{j = 1}^{n} \vartheta_1(2 \mathfrak{b}_j)}
  \sum_{k = 1}^{2g - 2 + n} 
  \lambda_k^{(2g - 2 + n)} \frac{q^{m/2}}{(k - 1)!}\Phi(q^m, 1 - k, \frac{1}{2})\prod_{j - 1}^{n}(b_j^m - b_j^{-m}) \ ,
\end{align}
where we used for odd $n$,
\begin{align}
  \sum_{\vec \alpha = \pm } \left(\prod_{j = 1}^{n}\alpha_j\right)
  \left(\prod_{j = 1}^{n}b_j^{-m} - \prod_{j = 1}^{n} b_j^m\right)
  = -2 \prod_{j = 1}^{n}(b_j^m - b_j^{-m})\ .
\end{align}
For odd $n$ we continue to have the same formula as the even $n$ case,
\begin{align}
  \langle W_j\rangle^{(1)}_{g,n}
  = & \ \mathcal{I}_{g, n}\delta_{j \in \mathbb{Z}}
  - \frac{1}{2}
    \prod_{i = 1}^{n} \frac{i \eta(\tau)^n}{\vartheta_1(\mathfrak{b}_j)}
    \sum_{\substack{m = - j\\m \ne 0}}^{+j} 
    \frac{\eta(\tau)^{2g - 2}}{(q^{m/2} - q^{-m/2})^{2g - 2}}\prod_{j = 1}^{n}\frac{b_j^{m} - b_j^{-m}}{q^{m/2} - q^{-m/2}} \ ,
\end{align}
thanks to the curious identity for odd $n$,
\begin{align}
  \sum_{k = 1}^{2g - 2 + n}\lambda_k^{(2g - 2 + n)} \frac{q^{m/2}}{(k - 1)!}
  \Phi(q^m, 1 - k, \frac{1}{2}) = - \frac{1}{(q^{m/2} - q^{-m/2})^{2 g - 2 + n}} \ .
\end{align}






\subsection{\texorpdfstring{Type-2 half Wilson line index in $\mathcal{T}[\Sigma_{g,n}]$}{}}

Next we consider another type of half Wilson operator index, which can be built on top of that of the $SU(2)$ SQCD by extending the relevant Riemann surface on either sides (but not further connecting the two sides). Put differently, we consider a half Wilson operator sitting at a tube that separates the Riemann surface into two disconnected pieces $\Sigma_{g_1, n_1}$ and $\Sigma_{g_2, n_2}$. See Figure \ref{fig:type-2-Wilson-line}. Let us denote such a Wilson index by $\langle W_j\rangle^{(2)}_{g_1, n_1; g_2, n_2}$. In this notation, the previous Wilson index $\langle W_j\rangle_{0,4}$ of the $SU(2)$ SQCD can be written as $\langle W\rangle_{0,3;0;3}^{(2)}$.
% Figure environment removed


% Figure environment removed

% Figure environment removed



\subsubsection{Simple type-2 examples}

We begin our analysis by looking at a simple genus-one configuration in Figure \ref{fig:genus-one-type-2}. It can be constructed from the $SU(2)$ SQCD by gauging the diagonal of the $SU(2)_{b_1} \times SU(2)_{b_2}$. The Wilson index can be computed by
\begin{align}
  \langle W_j \rangle^{(2)}_{1, 2}
  = \oint \frac{da}{2\pi i a} \langle W\rangle_{0,3;0;3}^{(2)}\Big|_{\substack{b_1 = a\\b_2 = 1/a}} \left(-\frac{1}{2}\right)\vartheta_1(\pm 2 \mathfrak{a}) \ .
\end{align}
Recall that (\ref{Wilson-index-SQCD})
\begin{align}
  \langle W\rangle_{0,3;0;3}^{(2)}
  = & \ \mathcal{I}_{0,4}\delta_{j \in \mathbb{Z}}
  - \sum_{i = 1}^{4} \left(\sum_{\substack{m = - j \\ m \ne 0}}^{+j} \frac{M_i^{2m} - M_i^{-2m}}{q^{m} - q^{-m}}\right)R_i \ ,
\end{align}
where $M_1 = b_1b_2$, $M_2 = b_1/b_2$, $M_3 = b_3 b_4$ and $M_4 = b_3 / b_4$. Obviously as $b_1 = a, b_1 = 1/a$, the $i = 1$ term does not contribute. Therefore, the Wilson index reads (where we have renamed $b_3, b_4 \to b_1, b_2$),
\begin{align}
  \langle W_j\rangle^{(2)}_{1,1; 0,3}
  = & \ \delta_{j \in \mathbb{Z}}\mathcal{I}_{1,2}
  - \frac{\eta(\tau)^2}{\prod_{i = 1}^2\vartheta_1(2 \mathfrak{b}_i)}
    \sum_{\substack{m = j \\ m\ne 0}}^{+j} (q^m + q^{-m})\prod_{i =1,2}\frac{b_i^{2m} - b_i^{-2m}}{q^m - q^{-m}} \nonumber \\
  & \ - \frac{\eta(\tau)^2}{2 \prod_{i=3,4} \vartheta_1(2 \mathfrak{b}_i)}
  \sum_{\alpha = \pm} \bigg(\alpha
  E_1 \begin{bmatrix}
    1 \\ b_1 b_2^\alpha  
  \end{bmatrix}
  \sum_{\substack{m = -j\\m \ne 0}}^{+j}
  \frac{(b_1b_2^\alpha)^{2m} - (b_1b_2^\alpha)^{- 2m}}{q^m - q^{-m}}
  \bigg) \ .
\end{align}
There are four major terms in this half Wilson index, which are proportional respectively to four linear independent expressions,
\begin{align}
  \mathcal{I}_{1,2}, \qquad
  \frac{\eta(\tau)^2}{\prod_{i = 1}^2 \vartheta_1(2 \mathfrak{b}_i)} , \qquad
  \frac{\eta(\tau)^2}{\prod_{i = 1}^2 \vartheta_1(2 \mathfrak{b}_i)}
    E_1 \begin{bmatrix}
      1 \\ b_1 b_2^\pm
    \end{bmatrix} \ ,
\end{align}
with rational coefficients in $b_i, q$. The first two factors have appeared previously in section \ref{section:genus-one-two-punctures}, both being solutions to the flavored modular differential equations \cite{Zheng:2022zkm}. It turns out that the two new factors containing $E_1$ are also additional solutions to the same set of equations, and therefore the type-2 index $\langle W_j\rangle^{(2)}_{1,1; 0,3}$ is also a linear combinations of $\chi(\mathcal{T}[\Sigma_{1,2}])$ characters with rational coefficients.



Next we consider a Wilson operator as demonstrated in Figure \ref{Wilson-type-2-example-1}. There are different ways to compute the index, and the most straightforward way is through the contour integral
\begin{align}
  \langle W_j\rangle_{1,2; 0,3}^{(2)} = & \ \oint \prod_{i = }^{3}\frac{da_i}{2\pi i a_i}\left[\sum_{m = -j}^{+j}a_3^{2m}\right]
  \prod_{\pm\pm}\frac{\eta(\tau)}{
    \vartheta_4(\mathfrak{b}_1 \pm \mathfrak{a}_1 \pm \mathfrak{a}_2)
  }
  \prod_{\pm\pm}\frac{\eta(\tau)}{
    \vartheta_4(\mathfrak{a}_3 \mp \mathfrak{a}_1 \mp \mathfrak{a}_2)
  }\nonumber\\
  & \ \qquad \times \prod_{\pm \pm}\frac{\eta(\tau)}{\vartheta_4(- \mathfrak{a}_3 \pm \mathfrak{b}_2 \pm \mathfrak{b}_3)}
  \prod_{i = 1}^{3}\left(- \frac{1}{2}\vartheta_1(\pm 2 \mathfrak{a}_i)\right) \ .
\end{align}
We choose to evaluate first the $a_3$-integral, and then $a_1, a_2$-integral. The computation is fairly tedious, and we only show the end result,
\begin{align}
  & \ \langle W_j\rangle_{1,2; 0,3}^{(2)} = \mathcal{I} \delta_{j \in \mathbb{Z}} \nonumber \\
  & \ + \sum_{\alpha, \beta = \pm}\sum_{\substack{m = -j \\ m \ne 0}}^j \frac{i \eta(\tau)^3}{8 \prod_{i = }^{3}\vartheta_1(2\mathfrak{b}_i)} \bigg[
  - \frac{4\alpha \beta b_2^{2m \alpha}b_3^{2m \beta}}{q^m - q^{-m}}
  \sum_{\gamma, \delta = \pm} \delta E_2 \begin{bmatrix}
    1 \\ q^{\frac{\gamma}{2}}  b_1^\delta b_2^\alpha b_3^\beta
  \end{bmatrix}(2\tau) \nonumber \\
  & \ \qquad\qquad\qquad\qquad\qquad + \frac{\alpha \beta b_2^{2m \alpha}b_3^{2m \beta}}{q^m - q^{-m}}
  \sum_{\gamma, \delta = \pm} \delta \gamma E_1 \begin{bmatrix}
    1 \\ q^{\frac{\gamma}{2}} b_1^\delta b_2^\alpha b_3^\beta
  \end{bmatrix}(2\tau)\\
  & \ \qquad\qquad\qquad\qquad\qquad - \frac{2 \alpha \beta b_2^{2m \alpha}b_3^{2m \beta}}{q^m - q^{-m}}
  \sum_{\delta = \pm} \delta E_2 \begin{bmatrix}
    -1 \\ b_1^\delta b_2^\alpha b_3^\beta  
  \end{bmatrix} \nonumber\\
  & \ \qquad\qquad\qquad\qquad\qquad + \frac{1}{q^m - q^{-m}}\frac{1}{1 - q^{-2m\alpha}} \sum_{\kappa, \gamma, \delta = \pm}b_2^{2m\gamma \alpha}b_3^{2m \delta \alpha} \alpha \gamma \delta \kappa E_1 \begin{bmatrix}
    -1 \\ b_1^\kappa b_2^{\gamma \alpha \beta}  b_3^{\delta \alpha \beta}
  \end{bmatrix}
  \bigg] \nonumber \\
  & \ + \sum_{\alpha, \beta = \pm} \sum_m'
    \frac{b_1^{2m \alpha} + b_1^{-2m\alpha}}{(q^m - q^{-m})(q^{m \alpha} - q^{-m \alpha})}
    \frac{\alpha \eta(\tau)^6}{8 \prod_{\pm \pm}\vartheta_4(\mathfrak{b}_1 \pm \mathfrak{b}_2 \pm \mathfrak{b}_3)} \ . \nonumber
\end{align}
Note that the Eisenstein series in the first two lines depend on $2\tau$ instead of just $\tau$, a price to pay for simplifying the result using the following identities,
\begin{align}
  \sum_{\pm}E_k\left[\begin{matrix}
    \phi \\ \pm z
  \end{matrix}\right](\tau) = & \ 2 E_k\left[\begin{matrix}
    \phi \\ z^2
  \end{matrix}\right](2\tau) \ , \nonumber \\
  \sum_{\pm} \pm E_k\left[\begin{matrix}
    \phi \\ \pm z
  \end{matrix}\right](\tau)
  = & \ -2 E_k\left[\begin{matrix}
    \phi \\ z^2
  \end{matrix}\right](2\tau)
   + 2 E_k\left[\begin{matrix}
    \phi \\ z
   \end{matrix}\right](\tau)\ , \nonumber
  \\
  E_k\left[\begin{matrix}
    + 1\\z
  \end{matrix}\right](2\tau)
  + E_k\left[\begin{matrix}
    - 1\\z
  \end{matrix}\right](2\tau) = & \ 
  \frac{2}{2^k}E_k\left[\begin{matrix}
    + 1 \\ z
  \end{matrix}\right] \ ,\\
  \sum_{\pm \pm} E_k\left[\begin{matrix}
    \pm 1 \\ \pm z
  \end{matrix}\right](\tau) = & \ \frac{4}{2^k}E_k\left[
  \begin{matrix}
    + 1 \\ z^2
  \end{matrix}\right](\tau)\ . \nonumber
\end{align}



\subsubsection{General type-2 Wilson index}


From the above two examples, it is somewhat clear that the Wilson index of type-2 are significantly more complex than the type-1 index. Moreover, unlike that in type-1, the Wilson index with spin $j \in \mathbb{Z}+\frac{1}{2}$ is nontrivial. Let us compute the type-2 index from another perspective. We consider gluing two Schur index $\mathcal{I}_{g_i, n_i}$ and insert a Wilson operator at the connecting tube,
\begin{align}
  \langle W\rangle_{g_1, n_1; g_2, n_2}^{(2)}
  = \oint \frac{da}{2\pi i a} \chi_j(a) \mathcal{I}_{g_1, n_1}(\mathfrak{b}_1, \ldots, \mathfrak{b}_{n_1 - 1}, \mathfrak{a}) \mathcal{I}_\text{VM}(\mathfrak{a})
  \mathcal{I}_{g_2, n_2}( - \mathfrak{a}, \tilde{\mathfrak{b}}_1, \ldots, \tilde{\mathfrak{b}}_{n_2 - 1}) \ .
\end{align}
For this we can apply the closed-form expressions for $\mathcal{I}_{g, n}$ \cite{Pan:2021mrw}, and the above becomes
\begin{align}
  - \frac{1}{2}\oint\frac{da}{2\pi i a}
  & \ \chi_j(a)
  \frac{i^{n_1}}{2}
  \frac{\eta(\tau)^{n_1 + 2g_1 - 2}}{\prod_{j = 1}^{n_1 - 1}\vartheta_1(2 \mathfrak{b}_j)}
  \frac{\eta(\tau)^{n_2 + 2g_2 - 2}}{\prod_{j = 1}^{n_2 - 1}\vartheta_1(2 \tilde{\mathfrak{b}}_j)}\\
  & \ \times \sum_{\vec\alpha,\vec \beta} \left(\prod_{j = 1}^{n_1}\alpha_j\right)
  \left(\prod_{j = 1}^{n_2}\beta_j\right)
  \sum_{k = 1}^{n_1 + 2g_1 - 2}\sum_{\ell = 1}^{n_2 + 2g_2 - 2}
  \lambda_k^{(n_1 + 2g_1 - 2)}
  \lambda_\ell^{(n_2 + 2g_2 - 2)}\\
  & \ \qquad\qquad E_k \begin{bmatrix}
    (-1)^{n_1}  \\ a^{\alpha_{n_1}} \prod_{j = 1}^{n_1 - 1}b_j^{\alpha_j}
  \end{bmatrix}
  E_\ell \begin{bmatrix}
      (-1)^{n_2}  \\ a^{ - \beta_{n_2}} \prod_{j = 1}^{n_2 - 1}\tilde b_j^{\beta_j}
    \end{bmatrix} \ .
\end{align}
Note that the vector multiplet factor has cancelled the $\vartheta_1(2 \mathfrak{a})\vartheta_1( - 2 \mathfrak{a})$ in the denominator. Therefore, the integration boils down to computing
\begin{align}
  \oint \frac{da}{2\pi i z}\chi_j(z) E_k \begin{bmatrix}
    \pm 1 \\
    z a
  \end{bmatrix}
  E_\ell \begin{bmatrix}
    \pm 1 \\
    z b
  \end{bmatrix} \ .
\end{align}

% Figure environment removed

For the special case of $n_1 = n_2 = 1$, $g_1 = g_2 = 1$ corresponding to a Wilson line in the genus-two theory as illustrated in Figure \ref{fig:type-2-genus-two}, we can easily compute the type-2 Wilson index by applying the two identities
\begin{align}
  \oint \frac{dz}{2\pi i z} E_1 \begin{bmatrix}
      + 1 \\ z
  \end{bmatrix}^2 = \frac{q^n( (n - 2) - n q^n )}{(1 - q^n)^2}, \quad
  \oint \frac{dz}{2\pi i z} E_1 \begin{bmatrix}
      - 1 \\ z
  \end{bmatrix}^2 = \frac{q^{n/2}( (n - 1) - (n + 1) q^n )}{(1 - q^n)^2} \ . \nonumber
\end{align}
The index then reads,
\begin{align}
  \langle W_j\rangle^{(2)}_{1,1;1,1}
  = & \ \oint \frac{da}{2\pi i a}
  \chi_j(a)
  \frac{i \eta(\tau)}{\vartheta_1(2 \mathfrak{a})}
  \frac{i \eta(\tau)}{\vartheta_1(-2 \mathfrak{a})}
  \left(- \frac{1}{2}\vartheta_1(\pm 2 \mathfrak{a})\right)
  E_1 \begin{bmatrix}
    -1 \\ a  
  \end{bmatrix}
  E_1 \begin{bmatrix}
    -1 \\ a^{-1}
  \end{bmatrix} \nonumber\\
  = & \ \frac{1}{2} \bigg(
     \delta_{j \in \mathbb{Z}}\eta(\tau)^2\left(E_2(\tau) + \frac{1}{12}\right)
    - \eta(\tau)^2\sum_{\substack{m = - j \\ m\ne 0 }}^{+j}
    \frac{ (2m - 1)q^{-m} - (2m + 1)q^{m}}{(q^m - q^{-m})^2}
    \bigg) \ . \nonumber
\end{align}
We note that both $\eta(\tau)^2$ and $\eta(\tau)^2 (E_2 + \frac{1}{12})$ are solutions to the modular differential equation that annihilates the genus two Schur index $\mathcal{I}_{2,0}$ \cite{Beem:2017ooy,Zheng:2022zkm},
\begin{align}
  0 = \Big[D_q^{(6)} - 305 E_4 D_q^{(4)} - 4060E_6 D_q^{(3)}
      + 20275E_4^2 & \ D_q^{(2)} + 2100E_4 E_6 D_q^{(1)} \nonumber \\
      & \ - 68600(E_6^2 - 49125E_4^3) \Big]\mathcal{I}_{2,0} \ ,
\end{align}
and therefore the above Wilson index $\langle W_j\rangle^{(2)}_{1,1;1,1}$ is also expected to be a linear combination of characters of the chiral algebra $\chi(\mathcal{T}[\Sigma_{2,0}])$.

The same structure of linear combination actually holds true for all type-2 index $\langle W_j\rangle^{(2)}_{g_1, 1; g_2, 1}$ illustrated in Figure \ref{fig:genus-g-type-2}. Indeed, the relevant integrals are of the form ($k_i \le 2g_i - 1$)
\begin{align}
  \oint \frac{da}{2\pi i a} E_{k_1} \begin{bmatrix}
      -1 \\ a
  \end{bmatrix}E_{k_2} \begin{bmatrix}
      -1 \\ a
  \end{bmatrix}
  \sim \text{linear combination of } E_{2}, E_4, \cdots, E_{2g - 2} \ ,
\end{align}
where we have used (\ref{integration-formula-zEE-1}), (\ref{integration-formula-zEE-2}). In the end, the Wilson index $\langle W_j\rangle^{(2)}_{g_1, 1; g_2, 1}$ is a linear combination of $\eta(\tau)^{2g - 2}, \eta(\tau)^{2g - 2} E_2(\tau), \cdots, \eta(\tau)^{2g - 2}E_{2g - 2}(\tau)$ with the coefficients being rational functions of $q$. This series of functions are indeed solutions to the modular differential equations annihilating the Schur index $\mathcal{I}_{g, 0}$, as they are simply the Schur index of the vortex surface defects in the 4d theory $\mathcal{T}[\Sigma_{g,0}]$ \cite{Gaiotto:2012xa,Zheng:2022zkm}.



% Figure environment removed


For more general $n_i, g_i$, we need to apply the integration formula \eqref{integration-formula-zEE-1}, \eqref{integration-formula-zEE-2} and their variants. For example, with both $n_1, n_2$ even, we have
\begin{align}
  & \ \langle W_j\rangle_{g_1, n_1; g_2, n_2}^{(2)} \nonumber \\
  = & \ \mathcal{I}_{g_1 + g_2, n_1 + n_2 - 2}\delta_{j \in \mathbb{Z}} \nonumber\\
  & \ + \frac{\eta(\tau)^{2(g_1 + g_2) + (n_1 + n_2 - 2) - 2}}{
    2\prod_{i = 1}^{n_1 + n_2 - 2}\vartheta_1(2 \mathfrak{b}_i)
  }\\
  & \ \qquad \times \sum_{\substack{m = - j \\ m\ne 0}}^j \sum_{\vec \alpha} \left(\prod_{i=1}^{n_1 + n_2 - 2}\alpha_i\right)
  \sum_{\ell = 0}^{\operatorname{max}(n_i + 2g_i - 2)}
    \Lambda_\ell^{(g_1, n_1; g_2, n_2)}(\mathbf{b}^{2m}, q^{2m})
    E_\ell \begin{bmatrix}
    1 \\ \prod_{i}^{n_1 + n_2 -2} b_i^{\alpha_i}
  \end{bmatrix} \ . \nonumber
\end{align}
Here we have merged the two sets of flavor fugacities $(b_1, \ldots, b_{n_1 - 1})$ and $(\tilde b_1, \ldots, \tilde b_{n_2 - 1})$ into a larger set $\mathbf{b} = (b_{i}, \ldots, b_{n_1 + n_2 - 2})$, and the corresponding signs $(\alpha_1, \ldots, \alpha_{n_1 - 1}, \beta_1, \ldots, \beta_{n_2 - 1})$ into $(\alpha_1, \ldots, \alpha_{n_1 + n_2 - 2})$. Finally, the $\Lambda$ are a set of rational functions of $b_i$ and $q$ coming from applying the integration formula (\ref{integration-formula-zEE-3}),
\begin{align}
  \Lambda_\ell^{(g_1, n_1; g_2, n_2)}(\mathbf{b}^{2m}, q^{2m})
  = \sum_{k_i = 0}^{n_i + 2g_i - 2}\frac{1}{\ell!} \frac{(-1)^{k_2 + 1} q^{2m}}{\prod_{i = 1}^{n_2 - 1}\tilde b_i^{2m \beta_i}} \lambda_{k_1}^{(n_1 + 2g_1 - 2)}\lambda_{k_2}^{(n_2 + 2g_2 - 2)} \mathcal{E}_{k_1, k_2; \ell}(\mathbf{b}^{2m \alpha}, q^{2m}) \ . \nonumber
\end{align}
Although it is a finite sum, unlike the beautiful result for the type-1 index formula, we are unable to reorganize the above type-2 result into a more elegant form. It would be interesting to further explore the relation between the type-2 Wilson line index and the characters of the associated chiral algebra $\chi(\mathcal{T}[\Sigma_{g,n}])$, and it is likely that the Wilson line index has access to new characters besides those from the surface defects index \cite{Zheng:2022zkm}.



\section{Line operator index in other gauge theories}


\subsection{\texorpdfstring{$\mathcal{N} = 4$ $SU(3)$ theory}{}}

The flavored $\mathcal{N} = 4$ $SU(N)$ Schur index in the presence of Wilson line operators is studied in \cite{Hatsuda:2023iwi} using the Fermi-gas formalism. In the following we also compute some simple examples using our integration formula. The relevant integral is of the form
\begin{align}
  \langle W_\mathcal{R}\rangle
  = - \frac{1}{N!} \frac{\eta(\tau)^{3N - 3}}{\vartheta_4(\mathfrak{b})^{N - 1}}\oint \prod_{A = 1}^{N - 1}  \frac{da_A}{2\pi i a_A}
  \chi_\mathcal{R}(a)
  \prod_{\substack{A, B = 1 \\ A\ne B}}^N \frac{\vartheta_1(\mathfrak{a}_A - \mathfrak{a}_B)}{\vartheta_4(\mathfrak{b} + \mathfrak{a}_A - \mathfrak{a}_B)} \ .
\end{align}

We will focus on $N = 3$. The $SU(3)$ character $\chi_\mathcal{R}(a)$ is a sum of monomials $a_1^{n_1} a_2^{n_2}$. Note that the ratio of the Jacobi theta functions is symmetric in $a_1 \leftrightarrow a_2$ and $\mathfrak{a}_A \to -\mathfrak{a}_A$, and therefore we can focus on monomials of the form $a_1^{n_1 > 0} a_2^{n_2}$; trivial monomial $a_1^0 a_2^0$ insertion simply integrates to the original $\mathcal{N} = 4$ Schur index. Now we compute
\begin{align}
  - \frac{1}{N!} \frac{\eta(\tau)^{3N - 3}}{\vartheta_4(\mathfrak{b})^{N - 1}}\oint \prod_{A = 1}^{N - 1}  \frac{da_A}{2\pi i a_A}
  a_1^{n_1} a_2^{n_2}
  \prod_{\substack{A, B = 1 \\ A\ne B}}^N \frac{\vartheta_1(\mathfrak{a}_A - \mathfrak{a}_B)}{\vartheta_4(\mathfrak{b} + \mathfrak{a}_A - \mathfrak{a}_B)} \ ,
\end{align}
by first integrating $a_1$ and then $a_2$. The $a_1$ integration is easy, leaving an $a_2$ integration of
\begin{align}
  - a_2^{n_2} \sum_{\pm}R^{(1)}_{1,\pm} \frac{a_2^{n_1}b^{\pm n_1}}{q^{n_1/2} - q^{-n_1/2}}
  & \ - a_2^{n_2}\sum_{\pm} R^{(1)}_{2, \pm}\frac{a_2^{-2n_1}b^{\pm n_1}}{q^{n_1/2} - q^{- n_1/2}} \nonumber \\
  & \ - a_2^{n_2}\sum_{\pm; k,\ell = 0,1}R^{(1)}_{3, \pm, k\ell} \frac{a_2^{- n_1/2 }b^{\pm n_1/2}q^{n_1/4} q^{\frac{k-1}{2} n_1} (-1)^{\ell n_1}}{q^{n_1/2} - q^{- n_1/2}}\ ,
\end{align}
where the poles are all imaginary with residues given by the following table.
\begin{center}
  \renewcommand{\arraystretch}{2}
  \begin{tabular}{c|c}
    $(\mathfrak{a}_1)^{(1)}_{1, \pm}$ & $\mathfrak{a}_2 \pm \mathfrak{b} + \tau/2$\\
    \hline
    $R_{1,\pm}^{(1)}$ & $\displaystyle \frac{i}{6}\eta(\tau)^3
    \frac{
      \vartheta_4(3 \mathfrak{a}_2 \pm \mathfrak{b})
      \vartheta_1(3 \mathfrak{a}_2 \pm 2 \mathfrak{b})
    }{
      \vartheta_1(\pm 2 \mathfrak{b})
      \vartheta_1(3 \mathfrak{a}_2)
      \vartheta_4(3 \mathfrak{a}_2 \pm 3 \mathfrak{b})
    }$\\
    \hline
    $(\mathfrak{a}_1)^{(1)}_{2, \pm}$ & $\mathfrak{a}_1 = - 2 \mathfrak{a}_2 \pm \mathfrak{b} + \tau/2$\\
    \hline
    $R_{2,\pm}^{(1)}$ & $\displaystyle
    \frac{i}{6} \eta(\tau)^3
    \frac{
      \vartheta_4(3 \mathfrak{a}_2 \mp \mathfrak{b})
      \vartheta_1(3 \mathfrak{a}_2 \mp 2 \mathfrak{b})
    }{
      \vartheta_1(\pm 2 \mathfrak{b})
      \vartheta_1(3 \mathfrak{a}_2)
      \vartheta_4(3 \mathfrak{a}_2 \mp 3 \mathfrak{b})}
    = - R_{1, \mp}^{(1)}
    $\\
    \hline
    $(\mathfrak{a}_1)^{(1)}_{3, \pm, k\ell}$ & $\displaystyle - \frac{\mathfrak{a}_2}{2} \pm \frac{\mathfrak{b}}{2} + \frac{\tau}{4} + \frac{k \tau}{2} + \frac{\ell}{2}$\\
    \hline
    $R^{(1)}_{3,\pm, k\ell}$ & $
    \displaystyle
    \frac{i}{12} \frac{\eta(\tau)^3}{\vartheta_1(\pm 2 \mathfrak{b})} \prod_{\gamma = \pm} \frac{\vartheta_1(\frac{3}{2} \gamma \mathfrak{a}_2 \pm \frac{1}{2}\mathfrak{b} + \frac{1}{4}\tau + \frac{k}{2}\tau + \frac{\ell}{2})^2}{
      \vartheta_4(\frac{3}{2} \gamma \mathfrak{a}_2 \pm \frac{3}{2}\mathfrak{b} + \frac{1}{4} \tau + \frac{k}{2}\tau + \frac{\ell}{2})
      \vartheta_4(\frac{3}{2} \gamma \mathfrak{a}_2 \mp \frac{1}{2}\mathfrak{b} + \frac{1}{4} \tau + \frac{k}{2}\tau + \frac{\ell}{2})
    }
    $
  \end{tabular}
\end{center}

% \begin{align}
%   R_{1,\pm}^{(1)} = & \ \frac{i}{6}\eta(\tau)^3
%   \frac{
%     \vartheta_4(3 \mathfrak{a}_2 \pm \mathfrak{b})
%     \vartheta_1(3 \mathfrak{a}_2 \pm 2 \mathfrak{b})
%   }{
%     \vartheta_1(\pm 2 \mathfrak{b})
%     \vartheta_1(3 \mathfrak{a}_2)
%     \vartheta_4(3 \mathfrak{a}_2 \pm 3 \mathfrak{b})
%   } \ , \\
%   R_{2,\pm}^{(1)} = & \ \frac{i}{6} \eta(\tau)^3
%     \frac{
%       \vartheta_4(3 \mathfrak{a}_2 \mp \mathfrak{b})
%       \vartheta_1(3 \mathfrak{a}_2 \mp 2 \mathfrak{b})
%     }{
%       \vartheta_1(\pm 2 \mathfrak{b})
%       \vartheta_1(3 \mathfrak{a}_2)
%       \vartheta_4(3 \mathfrak{a}_2 \mp 3 \mathfrak{b})}
%   = - R^{(1)}_{1, \mp} \ ,\\ 
%   R^{(1)}_{3,\pm, k\ell} = & \ \frac{i}{12} \frac{\eta(\tau)^3}{\vartheta_1(\pm 2 \mathfrak{b})} \prod_{\gamma = \pm} \frac{\vartheta_1(\frac{3}{2} \gamma \mathfrak{a}_2 \pm \frac{1}{2}\mathfrak{b} + \frac{1}{4}\tau + \frac{k}{2}\tau + \frac{\ell}{2})^2}{
%     \vartheta_4(\frac{3}{2} \gamma \mathfrak{a}_2 \pm \frac{3}{2}\mathfrak{b} + \frac{1}{4} \tau + \frac{k}{2}\tau + \frac{\ell}{2})
%     \vartheta_4(\frac{3}{2} \gamma \mathfrak{a}_2 \mp \frac{1}{2}\mathfrak{b} + \frac{1}{4} \tau + \frac{k}{2}\tau + \frac{\ell}{2})
%   } \ .  \nonumber
% \end{align}

It can be shown that,
\begin{align}
  - \oint \frac{da_2}{2\pi i a_2} a_2^n R^{(1)}_{1,\pm} = 0 \ , \qquad
  \text{if } n \not \in 3 \mathbb{Z} \ .
\end{align}
Therefore, we only focus on $n_1 + n_2 = 3p \ge 0$. Note also that $n_2 - 2n_1 = 3(p - n_1)$ in the second sum is also a multiple of $3$. With this assumption,
\begin{align}
  & \ - \sum_{\pm}\frac{b^{\pm n_1}}{q^{n_1/2} - q^{-n_1/2}} \oint \frac{da_2}{2\pi i a_2} a_2^{3p} R^{(1)}_{1,\pm}
  = - \sum_{\pm}\frac{b^{\pm n_1}}{q^{n_1/2} - q^{-n_1/2}} \oint \frac{da_2}{2\pi i a_2} a_2^{p} \left[R^{(1)}_{1,\pm}\right]_{3\mathfrak{a}_2 \to \mathfrak{a}_2} \nonumber \\
  = & \ - \delta_{n_1 + n_2 = 0} \sum_\pm
  \frac{b^{\pm n_1}}{q^{\frac{n_1}{2}} - q^{- \frac{n_1}{2}}}
  \frac{1}{6}\frac{\vartheta_4(\mathfrak{b})}{\vartheta_4(3 \mathfrak{b})}
  \left(
  E_1 \begin{bmatrix}
    -1 \\ b^{\pm 2} q^{\frac{1}{2}}
  \end{bmatrix}
  - E_1 \begin{bmatrix}
    -1 \\ b^{\mp}  
  \end{bmatrix}
  \right) \\
  & - \delta_{n_1 + n_2 \ne 0} \sum_{\pm}
  \frac{b^{\pm n_1}}{q^{\frac{n_1}{2}} - q^{- \frac{n_1}{2}}}
  \frac{1}{6}\frac{\vartheta_4(\mathfrak{b})}{\vartheta_4(3 \mathfrak{b})}
  \left(
  \frac{q^{\frac{1}{2}p} - b^{\mp 3 p}}{q^{p/2} - q^{-p/2}}
  \right) \ . \nonumber
\end{align}
Similarly
\begin{align}
  & \ - \oint \frac{da_2}{2\pi i a_2} a_2^{n_2}\sum_{\pm} R^{(1)}_{2, \pm}\frac{a_2^{-2n_1}b^{\pm n_1}}{q^{n_1/2} - q^{- n_1/2}}  \\
  = & \ - \delta_{n_2 - 2n_1 = 0}
  \frac{1}{6}\frac{\vartheta_4(\mathfrak{b})}{\vartheta_4(3 \mathfrak{b})}
  \sum_{\pm} \frac{b^{\pm n_1}}{q^{n_1/2} - q^{- n_1/2}}
  \left(
    E_1 \begin{bmatrix}
      -1 \\ b^{\mp 2}q^{\frac{1}{2}}
    \end{bmatrix}
    - E_1 \begin{bmatrix}
      -1 \\ b^{\pm}
    \end{bmatrix}
  \right)\\
  & \ + \delta_{n_2 - 2n_1 \ne 0}
  \frac{1}{6}
  \frac{\vartheta_4(\mathfrak{b})}{\vartheta_4(3 \mathfrak{b})}
  \sum_{\pm} \frac{b^{\pm n_1}}{q^{n_1/2} - q^{- n_1/2}}
  \left(
  \frac{q^{\frac{n_2 - 2n_1}{6}} - b^{\pm (n_2 - 2n_1)}}{q^{\frac{n_2 - 2n_1}{6}} - q^{- \frac{n_2 - 2n_1}{6}}}
  \right) \ .
\end{align}
Lastly, one can also check that
\begin{align}
  \oint \frac{da_2}{2\pi i a_2}a_2^n R^{(1)}_{3, \pm, k\ell} = 0, \qquad
  \text{if } n \not \in \frac{3}{2} \mathbb{Z} \ .
\end{align}
Therefore, since $n_1 + n_2 $ is an integer, we may assume $n_2 - n_1/2 = n_1 + n_2 - \frac{3}{2}n_1 = 3p - \frac{3n_1}{2}$ with $p \in \mathbb{Z}$ in order for the integral to be non-zero,
\begin{align}
  & \ - \sum_{\pm, k,\ell} \frac{b^{\pm n_1/2}q^{n_1/4} q^{\frac{k-1}{2} n_1} (-1)^{\ell n_1}}{q^{n_1/2} - q^{- n_1/2}}
  \oint \frac{da_2}{2\pi i a_2} R^{(1)}_{3, \pm, k\ell} a_2^{n_2 - \frac{n_1}{2}} \nonumber \\
  = & \ \delta_{n_2 \ne \frac{1}{2}n_1} \frac{\vartheta_4(\mathfrak{b})}{12\vartheta_4(3 \mathfrak{b})}
  \sum_{\alpha, \gamma = \pm}\sum_{k,\ell = 0,1}
  \gamma
  \frac{
    b^{\frac{\alpha}{2}( (1 + \gamma) n_1 - 2\gamma n_2)}
    q^{- \frac{1}{12}(2k - 1)( (\gamma - 1)n_1 - 2\gamma n_2 )}
  }{
    (q^{\frac{n_1}{2}} - q^{- \frac{n_1}{2}})
    (q^{\frac{1}{6}(2n_2 - 1)}
        - q^{ - \frac{1}{6}(2n_2 - 1)})
  } \\
  & \ - \delta_{n_2 = \frac{1}{2}n_1}
  \frac{\vartheta_4(\mathfrak{b})}{12\vartheta_4(3 \mathfrak{b})}
  \sum_{\alpha, \gamma = \pm}\sum_{k,\ell = 0}^{1}
  \gamma
  \frac{
    b^{\alpha \frac{n_1}{2}}
    q^{\frac{1}{4}n_1(2k - 1)}
    (-1)^{\ell n_1}
  }{
    q^{n_1/2} - q^{- n_1/2}
  }
  E_1 \begin{bmatrix}
    -1\\
    b^{ - \frac{1}{2}\alpha(3\gamma + 1)}  
    q^{- \frac{1}{4}(2k(\gamma - 1) - (\gamma + 1))}
  \end{bmatrix} \ . \nonumber
\end{align}

In the above we have used the poles and residues of the $R$-factors listed in the following table.
{
\renewcommand{\arraystretch}{1.5}
\begin{table}[h!]
\centering
  \begin{tabular}{c|c|c}
    factor & poles & residues\\
    \hline
    $R^{(1)}_{1,\pm}$ & $\mathfrak{a}_2 = 0$ & $ - \frac{i}{6\eta(\tau)} \frac{\vartheta_4( \mathfrak{b})}{\vartheta_4( 3 \mathfrak{b})}$\\
                      & $\mathfrak{a}_2 = \mp 3 \mathfrak{b} + \frac{\tau}{2}$ & $ + \frac{i}{6\eta(\tau)} \frac{\vartheta_4( \mathfrak{b})}{\vartheta_4( 3 \mathfrak{b})}$\\
    \hline
    $R^{(1)}_{2,\pm}$ & $\mathfrak{a}_2 = 0$ & $ + \frac{i}{6\eta(\tau)} \frac{\vartheta_4( \mathfrak{b})}{\vartheta_4( 3 \mathfrak{b})}$\\
                      & $\mathfrak{a}_2 = \pm 3 \mathfrak{b} + \frac{\tau}{2}$ & $ - \frac{i}{6\eta(\tau)} \frac{\vartheta_4( \mathfrak{b})}{\vartheta_4( 3 \mathfrak{b})}$\\
    \hline
    $R^{(1)}_{3,\pm,k\ell}$ & $\mathfrak{a}_2 = \mp \frac{3}{2} \gamma \mathfrak{b} + \frac{\tau}{2} + \frac{1}{4}(2k - 1)\gamma \tau + \frac{\ell}{2}$, $\gamma = \pm 1$ & $ \gamma \frac{\vartheta_4(\mathfrak{b})}{12 \vartheta_4(3 \mathfrak{b})}$
  \end{tabular}
\end{table}
}

Putting all the above together, we have
\begin{align}
  & \ - \frac{1}{N!} \frac{\eta(\tau)^{3N - 3}}{\vartheta_4(\mathfrak{b})^{N - 1}}\oint \prod_{A = 1}^{N - 1}  \frac{da_A}{2\pi i a_A}
  a_1^{n_1} a_2^{n_2}
  \prod_{\substack{A, B = 1 \\ A\ne B}}^N \frac{\vartheta_1(\mathfrak{a}_A - \mathfrak{a}_B)}{\vartheta_4(\mathfrak{b} + \mathfrak{a}_A - \mathfrak{a}_B)}\nonumber\\
  = & \ 0 \qquad \text{ if } n_1 + n_2 \ne 0 \mod 3 ,  \\
  \text{else} = & \ 
  + \delta_{n_1 + n_2 = 0} 
    \frac{1}{6}\frac{\vartheta_4(\mathfrak{b})}{\vartheta_4(3 \mathfrak{b})}
    \sum_\pm
    \frac{b^{\pm n_1}}{q^{\frac{n_1}{2}} - q^{- \frac{n_1}{2}}}
    \left(
    E_1 \begin{bmatrix}
      -1 \\ b^{\pm 2} q^{\frac{1}{2}}
    \end{bmatrix}
    - E_1 \begin{bmatrix}
      -1 \\ b^{\mp}  
    \end{bmatrix}
    \right) \nonumber\\
    & - \delta_{n_1 + n_2 \ne 0} 
    \frac{1}{6}\frac{\vartheta_4(\mathfrak{b})}{\vartheta_4(3 \mathfrak{b})}
    \sum_{\pm}
    \frac{b^{\pm n_1}}{q^{\frac{n_1}{2}} - q^{- \frac{n_1}{2}}}
    \left(
    \frac{q^{\frac{1}{2}p} - b^{\mp 3 p}}{q^{p/2} - q^{-p/2}}
    \right) \nonumber\\
    & \ - \delta_{n_2 = 2n_1}
      \frac{1}{6}\frac{\vartheta_4(\mathfrak{b})}{\vartheta_4(3 \mathfrak{b})}
      \sum_{\pm} \frac{b^{\pm n_1}}{q^{n_1/2} - q^{- n_1/2}}
      \left(
        E_1 \begin{bmatrix}
          -1 \\ b^{\mp 2}q^{\frac{1}{2}}
        \end{bmatrix}
        - E_1 \begin{bmatrix}
          -1 \\ b^{\pm}
        \end{bmatrix}
      \right) \nonumber\\
      & \ + \delta_{n_2 \ne 2n_1}
      \frac{1}{6}
      \frac{\vartheta_4(\mathfrak{b})}{\vartheta_4(3 \mathfrak{b})}
      \sum_{\pm} \frac{b^{\pm n_1}}{q^{n_1/2} - q^{- n_1/2}}
      \left(
      \frac{q^{\frac{n_2 - 2n_1}{6}} - b^{\pm (n_2 - 2n_1)}}{q^{\frac{n_2 - 2n_1}{6}} - q^{- \frac{n_2 - 2n_1}{6}}}
      \right)\nonumber \\
    & \ - \delta_{n_2 = \frac{1}{2}n_1}
    \frac{\vartheta_4(\mathfrak{b})}{12\vartheta_4(3 \mathfrak{b})}
    \sum_{\alpha, \gamma = \pm}\sum_{k,\ell = 0}^{1}
    \gamma
    \frac{
      b^{\alpha \frac{n_1}{2}}
      q^{\frac{1}{4}n_1(2k - 1)}
      (-1)^{\ell n_1}
    }{
      q^{n_1/2} - q^{- n_1/2}
    }
    E_1 \begin{bmatrix}
      -1\\
      b^{ - \frac{1}{2}\alpha(3\gamma + 1)}  
      q^{- \frac{1}{4}(2k(\gamma - 1) - (\gamma + 1))}
    \end{bmatrix} \nonumber\\
    & \ + \delta_{n_2 \ne \frac{1}{2}n_1} \frac{\vartheta_4(\mathfrak{b})}{12\vartheta_4(3 \mathfrak{b})}
      \sum_{\alpha, \gamma = \pm}\sum_{k,\ell = 0,1}
      \gamma
      \frac{
        b^{\frac{\alpha}{2}( (1 + \gamma) n_1 - 2\gamma n_2)}
        q^{- \frac{1}{12}(2k - 1)( (\gamma - 3)n_1 - 2\gamma n_2 )}
      }{
        (q^{\frac{n_1}{2}} - q^{- \frac{n_1}{2}})
        (q^{\frac{1}{6}(2n_2 - n_1)}
            - q^{ - \frac{1}{6}(2n_2 - n_1)})
      } \nonumber \ .
\end{align}
The formula above implies the following symmetry which can be used to simplify computations,
\begin{align}
  \mathcal{I}(n_1, n_2) = & \ \mathcal{I}(n_2, n_1) = \mathcal{I}(- n_1, - n_2), \\
  \mathcal{I}(n_1, n_2) = & \ \mathcal{I}(n_1, n_1 - n_2) = \mathcal{I}(n_2 - n_1, n_2) \ .
\end{align}
With this formula, one can compute any Wilson line index in any $SU(3)$ representation $\mathcal{R}$ in closed-form. For example,
\begin{align}
  \langle W_{[1,1]} \rangle
  = & \ 2\mathcal{I}_{\mathcal{N} = 4 \ SU(3)} + 6 \mathcal{I}_{1,2} \nonumber \\
  = & \ 2 \mathcal{I}_{\mathcal{N}=4 \ SU(3)}\\
  & \ + \frac{\vartheta_4(\mathfrak{b})}{\vartheta_4(3 \mathfrak{b})}
  \Bigg[
    \frac{b\sqrt{q} - (1+b^4)q + b^3 q^{\frac{3}{2}}}{b^2 (1 - q)^2}
    + \frac{(b^2 -1)\sqrt{q}}{b(q-1)}
    \left(
    E_1 \begin{bmatrix}
      -1 \\ b  
    \end{bmatrix}
    + E_1 \begin{bmatrix}
      -1 \\ b^2 q^{\frac{1}{2}}
    \end{bmatrix}
    \right)
  \Bigg] \ . \nonumber
\end{align}

\begin{align}
  \langle W_{[2,2]} \rangle
  = & \ 3\mathcal{I}_{\mathcal{N} = 4 \ SU(3)} + 12 \mathcal{I}_{1,2}
  + 6 \mathcal{I}_{2,4} + 6 \mathcal{I}_{3,0} \nonumber \\
  = & \ 3 \mathcal{I}_{\mathcal{N}=4 \ SU(3)} \nonumber\\
  & \ + \frac{\vartheta_4(\mathfrak{b})}{\vartheta_4(3 \mathfrak{b})}
  \Bigg[
    \frac{\sqrt{q} (b^3 q^{\frac{1}{2}}-1) (-b^5 q-2 b^4 q^{\frac{1}{2}} (q+1)-b^3 (q (q+4)+2))}{b^4 \left(q^2-1\right)^2}
  \Bigg] \\
  & \ + \frac{\vartheta_4(\mathfrak{b})}{\vartheta_4(3 \mathfrak{b})}
  \Bigg[
    \frac{\sqrt{q} (+b^2 (2 q (q+2)+1) q^{\frac{1}{2}}+2 b (q+1) q+q^{3/2})}{b^4 \left(q^2-1\right)^2}
  \Bigg] \nonumber \\
  & \ + \frac{\vartheta_4(\mathfrak{b})}{\vartheta_4(3 \mathfrak{b})}
  \frac{\sqrt{q}(b^2 -1) \Big[(b^2+1) \sqrt{q}+2 b q+2 b \Big]}{b^2(q^2 - 1)}
  \left(
  E_1 \begin{bmatrix}
    -1 \\ b  
  \end{bmatrix}
  + E_1 \begin{bmatrix}
    -1 \\ b^2\sqrt{q}  
  \end{bmatrix}
  \right) \ . \nonumber
\end{align}



\begin{align}
  \langle W_{[3,3]} \rangle
  = & \ 4\mathcal{I}_{\mathcal{N} = 4 \ SU(3)} + 18 \mathcal{I}_{1,2}
  + 12 \mathcal{I}_{2,4} + 12 \mathcal{I}_{3,0} + 12 \mathcal{I}_{4,5}
  + 6 \mathcal{I}_{3,6} \ .
\end{align}














\subsection{\texorpdfstring{$\mathcal{N} = 2$ $SU(3)$ SQCD}{}}

Let us also consider Wilson operator in the $SU(3)$ SQCD. The relevant integral reads
\begin{align}
  \mathcal{I}_{SU(3) \ \text{SQCD}} = & \ - \frac{1}{3!} \eta(\tau)^{16} \oint \prod_{A = 1}^2 \frac{da_A}{2\pi i a_A}
  \chi_\mathcal{R}(a)
  \frac{\prod_{A \ne B} \vartheta_1(\mathfrak{a}_A - \mathfrak{a}_B)}{\prod_{A = 1}^3 \prod_{i = 1}^{6} \vartheta_4(\mathfrak{a}_A - \mathfrak{m}_i)} \\
  \coloneqq & \ \oint \prod_{A = 1}^2 \frac{da_A}{2\pi i a_A}
  \chi_\mathcal{R}(a) \mathcal{Z}(\mathfrak{a}, \mathfrak{m}) \ .
\end{align}


\subsubsection{Fundamental representation}

As the simplest example, we consider the fundamental representation
\begin{align}
  \chi (a) = a_1 + a_2 + \frac{1}{a_1 a_2} \ .
\end{align}
First we note that
\begin{align}
  \oint \prod_{A = 1}^2 \frac{da_A}{2\pi i a_A} a_1 \mathcal{Z}(\mathfrak{a}, \mathfrak{m})
  = \oint \prod_{A = 1}^2 \frac{da_A}{2\pi i a_A} a_2 \mathcal{Z}(\mathfrak{a}, \mathfrak{m}) \ .
\end{align}
Therefore we simply compute the one with $a_1$ insertion. The relevant poles when performing the $a_1$ integration are all imaginary given by
\begin{align}
  \mathfrak{a}_1 = \mathfrak{m}_{j_1} + \frac{\tau}{2}, \qquad
  \mathfrak{a}_1 = - \mathfrak{a}_2 - \mathfrak{m}_{j_1} + \frac{\tau}{2} \ ,
\end{align}
with the respective residues
\begin{align}
  R_{j_1} \coloneqq - \frac{1}{6}
  \frac{
    \eta(\tau)^{13}q^{\frac{1}{8}}
    \prod_{A\ne B} \vartheta_1(\mathfrak{a}_A - \mathfrak{a}_B)|_{\mathfrak{a}_1 = \mathfrak{m}_{j_1} + \frac{\tau}{2}}
  }{
    \prod_i\vartheta_4(\mathfrak{a}_2 - \mathfrak{m}_i)
    \prod_i\vartheta_4(\mathfrak{a}_2 + \mathfrak{m}_{j_1} + \mathfrak{m}_i + \frac{\tau}{2})
    \prod_{i \ne j_1}\vartheta_4(\mathfrak{m}_i - \mathfrak{m}_{j_1} - \frac{\tau}{2})
  } \ ,  \quad -R_{j_1} \ . \nonumber
\end{align}
Therefore, after the $a_1$ integral we are left with
\begin{align}
  \oint \frac{da_2}{2\pi i a_2} \left[- \sum_{j_1 = 1}^{6}R_{j_1} \frac{1}{q^1 - 1} (m_{j_1}q^{\frac{1}{2}})
      + \sum_{j_1 = 1}^{6}R_{j_1} \frac{1}{q^1 - 1} (a_2^{-1} m^{-1}_{j_1}q^{\frac{1}{2}})\right] \ .
\end{align}

Next we perform the $a_2$ integral. Each residue $R_{j_1}$ is an elliptic function with respect to $\mathfrak{a}_2$, with imaginary and real poles
\begin{align}
  \mathfrak{a}_2 = & \ + \mathfrak{m}_{j_2} + \frac{\tau}{2}, & j_2 \ne & \ j_1 \\
  \text{or}, \qquad = & \ - \mathfrak{m}_{j_1} - \mathfrak{m}_{j_2} \ ,  & j_2 \ne & \ j_1 \ .
\end{align}
The corresponding residues are, respectively,
\begin{align}
  R_{j_1 j_2} = \frac{
      \eta(\tau)^{10}
      \vartheta_4(2 \mathfrak{m}_{j_1} + \mathfrak{m}_{j_2})
      \vartheta_4(\mathfrak{m}_{j_1} + 2\mathfrak{m}_{j_2})}{
    6
    \prod_{i\ne j_1, j_2}\vartheta_1(\mathfrak{m}_{j_1} - \mathfrak{m}_i)\vartheta_1(\mathfrak{m}_{j_2} - \mathfrak{m}_i)
    \prod_{i \ne j_1, j_2} \vartheta_4(\mathfrak{m}_{j_1}+ \mathfrak{m}_{j_2} + \mathfrak{m}_i)
  }, \quad
  - R_{j_1 j_2} \ .
\end{align}
We also set $R_{j_1 j_2} = 0$ when $j_1 = j_2$. With this, we have by applying (\ref{integration-formula-f})
\begin{align}
  \oint \frac{da_2}{2\pi i a_2} R_{j_1} = R_{j_1}(\mathfrak{a}_2 = 0)
  + \sum_{j_2 = 1}^{6} R_{j_1 j_2}E_1 \begin{bmatrix}
    -1 \\ m_{j_2}  
  \end{bmatrix}
  + R_{j_1 j_2}E_1 \begin{bmatrix}
    -1 \\ m_{j_1}m_{j_2}q^{-\frac{1}{2}}
  \end{bmatrix}\ ,
\end{align}
where we have picked $\mathfrak{a}_2 = 0$ as the reference point, and
\begin{align}
  \oint \frac{da_2}{2\pi i a_2}a_2^{-1}R_{j_1}
  = & \ + \sum_{j_2 = 1}^6 R_{j_1 j_2} \frac{1}{1 - q}m_{j_1}m_{j_2}
  - \sum_{j_2 = 1}^{6}R_{j_1 j_2} \frac{1}{q^{-1} - 1} (m_{j_2}q^{\frac{1}{2}})^{-1} \\
  = & \ + \sum_{j_2 = 1}^6 R_{j_1 j_2} \frac{m_{j_1}m_{j_2} - m_{j_2}^{-1}q^{\frac{1}{2}}}{1 - q} \ .
\end{align}
Collecting the results, the integral with $a_1$-insertion reads
\begin{align}
  & \ \oint \prod_{A = }^{2} \frac{da_A}{2\pi i a_A} a_1 \mathcal{Z} \\
  = & \ \frac{q^{\frac{1}{2}}}{1 - q} \sum_{j_1 = 1}^{6}m_{j_1}\left(
  R_{j_1}(\mathfrak{a}_2 = 0)
  + \sum_{j_2 = 1}^{6} R_{j_1 j_2}E_1 \begin{bmatrix}
    -1 \\ m_{j_2}  
  \end{bmatrix}
  + R_{j_1 j_2}E_1 \begin{bmatrix}
    -1 \\ m_{j_1}m_{j_2}q^{-\frac{1}{2}}
  \end{bmatrix}
  \right) \\
  & \ - \frac{1}{(1-q)^2} \sum_{j_1 = 1}^{6}
  \sum_{j_2 = 1}^{6}R_{j_1 j_2}(m_{j_2}q^{\frac{1}{2}} - m_{j_1}^{-1}m_{j_2}^{-1}q) \ .
\end{align}

Next we compute the integral
\begin{align}
  \oint \prod_{A = 1}^{2} \frac{da_A}{2\pi i a_A} \frac{1}{a_1 a_2} \mathcal{Z} \ .
\end{align}
Similar to the previous computation, we first integrate $a_1$ with poles $\mathfrak{a}_1 = \mathfrak{m}_{j_1} + \frac{\tau}{2}$ and $\mathfrak{a}_1 = - \mathfrak{a}_2 - \mathfrak{m}_{j_2} + \frac{\tau}{2}$,
\begin{align}
  & \ \oint \frac{da_2}{2\pi i a_2} \frac{1}{a_2} \left[- \sum_{j_1 = 1}^{6}R_{j_1} \frac{1}{q^{ - 1} - 1} (m_{j_1}q^{\frac{1}{2}})^{-1}
      + \sum_{j_1 = 1}^{6}R_{j_1} \frac{1}{q^{-1} - 1} (a_2^{-1} m^{-1}_{j_1}q^{\frac{1}{2}})^{-1}\right]\\
  = & \ \oint \frac{da_2}{2\pi i a_2} \left[- \sum_{j_1 = 1}^{6}R_{j_1} \frac{1}{q^{ - 1} - 1} (a_2^{-1} m_{j_1}^{-1}q^{-\frac{1}{2}})
      + \sum_{j_1 = 1}^{6}R_{j_1} \frac{1}{q^{-1} - 1} m_{j_1}q^{ - \frac{1}{2}}\right] \ .
\end{align}
Carrying out the $a_2$ integration, we have
\begin{align}
  & \ \oint \frac{da_1}{2\pi i a_1}\frac{da_2}{2\pi i a_2} \frac{1}{a_1 a_2} \mathcal{Z}\\
  = & \ + \frac{q^{\frac{1}{2}}}{1 - q} \sum_{j_1 = 1}^{6} m_{j_1} \left(
  R_{j_1}(\mathfrak{a}_2 = 0)
  + \sum_{j_2 = 1}^{6}R_{j_1 j_2} E_1 \begin{bmatrix}
    -1 \\ m_{j_2}  
  \end{bmatrix}
  + R_{j_1 j_2} E_1 \begin{bmatrix}
    -1 \\ m_{j_1} m_{j_2} q^{-1/2}  
  \end{bmatrix}
  \right) \\
  & \ - \frac{1}{(1 - q)^2} \sum_{j_1 = 1}^{6}
  \sum_{j_2 = 1}^{6} R_{j_1 j_2} (m_{j_2}q^{+ \frac{1}{2}} - m_{j_1}^{-1}m_{j_2}^{-1}q) \ .
\end{align}
Actually, this is identical to the previous result,
\begin{align}
  \oint \frac{da_1}{2\pi i a_1}\frac{da_2}{2\pi i a_2} \frac{1}{a_1 a_2} \mathcal{Z}
  = \oint \frac{da_1}{2\pi i a_1}\frac{da_2}{2\pi i a_2} a_1 \mathcal{Z}
  = \oint \frac{da_1}{2\pi i a_1}\frac{da_2}{2\pi i a_2} a_2 \mathcal{Z} \ .
\end{align}



Combining the integration of all three terms in the fundamental characters, we therefore have a fairly simple result,
\begin{align}
  \langle W_{\mathbf{3}} \rangle_{SU(3) \ \text{SQCD}}
  = & \ \frac{3q^{\frac{1}{2}}}{1 - q}\sum_{j_1 = 1}^{6}\left(
  R_{j_10} + \sum_{j_2 = 1}^{6}R_{j_1 j_2} \left(E_1 \begin{bmatrix}
    -1 \\ m_{j_2}  
  \end{bmatrix}
  + E_1 \begin{bmatrix}
    -1 \\ m_{j_1} m_{j_2}q^{-\frac{1}{2}}  
  \end{bmatrix}
  \right)
  \right) \\
  & \ - \frac{3}{(1-q)^2} \sum_{j_1, j_2 = 1}^{6}R_{j_1 j_2} (m_{j_2} q^{\frac{1}{2}} - m_{j_1}^{-1}m_{j_2}^{-1}q) \ .
\end{align}
where we abbreviate
\begin{align}
  R_{j_1 0} \coloneqq R_{j_1}(\mathfrak{a}_2 = 0) \ .
\end{align}


\subsubsection{General representation}

The above computation can be generalized to insertion of all half Wilson operator in any representation of the gauge group $SU(3)$. The basic building block is of course a monomial $a_1^{n_1} a_2^{n_2}$. Let us therefore compute the basic integral
\begin{align}
  \oint \frac{da_1}{2\pi i a_1}\frac{da_2}{2\pi i a_2} a_1^{n_1} a_2^{n_2} \mathcal{Z} \ .
\end{align}
Note that
\begin{align}
  \oint \frac{da_1}{2\pi i a_1}\frac{da_2}{2\pi i a_2}a_2^{n_2}\mathcal{Z}
  = \oint \frac{da_1}{2\pi i a_1}\frac{da_2}{2\pi i a_2}a_1^{n_2}\mathcal{Z} \ .
\end{align}
Therefore, without loss of generality we assume $n_1 \in \mathbb{Z}_{\ne0}$, and we first perform $a_1$ and then $a_2$ integration. The first step picks up the imaginary poles $\mathfrak{a}_1 = \mathfrak{m}_{j_1} + \frac{\tau}{2}$ and $- \mathfrak{a}_2 - \mathfrak{m}_{j_1} + \frac{\tau}{2}$, which produces
\begin{align}
  & \ \oint \frac{da_2}{2\pi i a_2} a_2^{n_2} \left[- \sum_{j_1}^{6}R_{j_1} \frac{1}{q^{n_1} - 1}(m_{j_1} q^{\frac{1}{2}})^{n_1}
  - \sum_{j_1 = 1}^{6}(-R_{j_1}) \frac{1}{q^{n_1} - 1}(a_2^{-1} m_{j_1}^{-1}q^{\frac{1}{2}})^{n_1}
  \right] \\
  = & \ \oint \frac{da_2}{2\pi i a_2} \left[- \sum_{j_1}^{6} a_2^{n_2} R_{j_1} \frac{1}{q^{n_1} - 1}(m_{j_1} q^{\frac{1}{2}})^{n_1}
  - \sum_{j_1 = 1}^{6}(-R_{j_1})a_2^{n_2 - n_1} \frac{1}{q^{n_1} - 1}(m_{j_1}^{-1}q^{\frac{1}{2}})^{n_1}
  \right] \ . \nonumber
\end{align}
Depending on whether $n_2 = 0$ or $n_2 - n_1 = 0$ or a generic $n_2$, the $a_2$-integration of the two terms take different form.

For the first term, if $n_2 = 0$, then the integral picks up contributions from the imaginary poles $\mathfrak{m}_{j_2} + \frac{\tau}{2}$ and the real poles $- \mathfrak{m}_{j_1} - \mathfrak{m}_{j_2}$, which reads
\begin{align}
  & \ \oint \frac{da_2}{2\pi i a_2} \left[- \sum_{j_1}^{6} a_2^{n_2 = 0} R_{j_1} \frac{1}{q^{n_1} - 1}(m_{j_1} q^{\frac{1}{2}})^{n_1}\right] \\
  = & \ - \sum_{j_1 = 1}^{6}\frac{(m_{j_1}q^{1/2})^{n_1}}{q^{n_1} - 1}\left(
  R_{j_10} + R_{j_1 j_2} E_1 \begin{bmatrix}
    -1 \\ m_{j_2}  
  \end{bmatrix}
  + R_{j_1 j_2} E_1 \begin{bmatrix}
    -1 \\ m_{j_1} m_{j_2}q^{-1/2}  
  \end{bmatrix}
  \right) \ .
\end{align}
However, if $n_2 \ne 0$, then
\begin{align}
  & \ \oint \frac{da_2}{2\pi i a_2} \left[- \sum_{j_1}^{6} a_2^{n_2} R_{j_1} \frac{1}{q^{n_1} - 1}(m_{j_1} q^{\frac{1}{2}})^{n_1}\right] \\
  = & \ - \sum_{j_1 = 1}^{6} \frac{(m_{j_1}q^{\frac{1}{2}})^{n_1}}{q^{n_1} - 1}
  \left(
  - \sum_{j_2 = 1}^{6}R_{j_1 j_2} \frac{1}{q^{n_2} - 1}(m_{j_2}q^{\frac{1}{2}})^{n_2}
  - \sum_{j_2 = 1}^{6} (- R_{j_1 j_2}) \frac{1}{1 - q^{-m_2}} (m_{j_1}^{-1} m_{j_2}^{-1})^{n_2}
  \right) \ . \nonumber
\end{align}



For the second term, if $n_2 - n_1 = 0$, then
\begin{align}
  & \ \oint \frac{da_2}{2\pi i a_2}\left[
    \sum_{j_1 = 1}^{6}R_{j_1} a_2^{n_2 - n_1} \frac{1}{q^{n_1} - 1} (m_{j_1}^{-1} q^{\frac{1}{2}})^{n_1}
  \right]\\
  = & \ \sum_{j_1 = 1}^{6} \frac{m_{j_1}^{-n_1}q^{\frac{n_1}{2}}}{q^{n_1} - 1} \left(
  R_{j_10} + R_{j_1 j_2} E_1 \begin{bmatrix}
    -1 \\ m_{j_2}  
  \end{bmatrix}
  + R_{j_1 j_2} E_1 \begin{bmatrix}
    -1 \\ m_{j_1} m_{j_2}q^{-1/2}  
  \end{bmatrix}
  \right) \ .
\end{align}
On the other hand, if $n_2 - n_1 \ne 0$ then 
\begin{align}
  & \ \oint \frac{da_2}{2\pi i a_2}\left[
    \sum_{j_1 = 1}^{6}R_{j_1} a_2^{n_2 - n_1} \frac{1}{q^{n_1} - 1} (m_{j_1}^{-1} q^{\frac{1}{2}})^{n_1}
  \right]\\
  = & \ \sum_{j_1 = 1}^{6} \frac{m_{j_1}^{-n_1} q^{\frac{n_1}{2}}}{q^{n_1} - 1}
  \left(
  - \sum_{j_2 = 1}^{6} R_{j_1j_2} \frac{m_{j_2}^{n_2 - n_1}q^{\frac{1}{2}(n_2 - n_1)}}{q^{n_1 - n_2} - 1}
  + \sum_{j_2 = 1}^{6} R_{j_1 j_2} \frac{(m_{j_1}^{-1} m_{j_2}^{-1})^{n_2 - n_1}}{1 - q^{-(n_2 - n_1)}}
  \right) \ .
\end{align}

Putting all terms together, we have
\begin{align}
  & \ \oint \frac{da_1}{2\pi i a_1}\frac{da_2}{2\pi i a_2}a_1^{n_1 \ne 0} a_2^{n_2} \mathcal{Z} \nonumber\\
  = & \ - \delta_{n_2 = 0} \sum_{j_1 = 1}^{6}\frac{m_{j_1}^{n_1}q^{\frac{1}{2} n_1}}{q^{n_1} - 1}\left(
  R_{j_10} + R_{j_1 j_2} E_1 \begin{bmatrix}
    -1 \\ m_{j_2}  
  \end{bmatrix}
  + R_{j_1 j_2} E_1 \begin{bmatrix}
    -1 \\ m_{j_1} m_{j_2}q^{-1/2}  
  \end{bmatrix}
  \right) \nonumber\\
  & \ + \delta_{n_2 \ne 0}\sum_{j_1, j_2 = 1}^{6}
    R_{j_1j_2}\frac{m_{j_1}^{n_1}q^{\frac{1}{2} n_1}}{q^{n_1} - 1}
    \frac{(m_{j_2}q^{\frac{1}{2}})^{n_2}-(m_{j_1}^{-1}m_{j_2}^{-1})^{n_2}q^{n_2}}{q^{n_2}-1} \nonumber \\
  & \ + \delta_{n_2 = n_1}\sum_{j_1 = 1}^{6} \frac{m_{j_1}^{-n_1}q^{\frac{n_1}{2}}}{q^{n_1} - 1} \left(
  R_{j_10} + R_{j_1 j_2} E_1 \begin{bmatrix}
    -1 \\ m_{j_2}  
  \end{bmatrix}
  + R_{j_1 j_2} E_1 \begin{bmatrix}
    -1 \\ m_{j_1} m_{j_2}q^{-1/2}  
  \end{bmatrix}
  \right) \\
  & \ - \delta_{n_2 \ne n_1}\sum_{j_1, j_2 = 1}^{6}
    R_{j_1j_2}
    \frac{m_{j_1}^{-n_1} q^{\frac{n_1}{2}}}{q^{n_1} - 1}
    \frac{m_{j_2}^{n_2-n_1}q^{\frac{1}{2}(n_2-n_1)}-(m_{j_1}^{-1}m_{j_2}^{-1})^{n_2-n_1}q^{n_2-n_1}}{q^{n_2-n_1}-1} \ . \nonumber
\end{align}

For example, for the Wilson operator in the anti-fundamental representation,
\begin{align}
  \langle W_{\overline {\mathbf{3}}}\rangle_{SU(3) \ \text{SQCD}}
  = & \ 3 \oint \frac{da_1}{2\pi i a_1}\frac{da_2}{2\pi i a_2} a_1^{-1} \mathcal{Z} \nonumber \\
  = & \  - 3 \frac{q^{\frac{1}{2}}}{1 - q} \sum_{j_1 = 1}^{6}m_{j_1}^{-1}\left(
  R_{j_10} + R_{j_1 j_2} E_1 \begin{bmatrix}
    -1 \\ m_{j_2}  
  \end{bmatrix}
  + R_{j_1 j_2} E_1 \begin{bmatrix}
    -1 \\ m_{j_1} m_{j_2}q^{-1/2}  
  \end{bmatrix}
  \right) \nonumber \\
  & \  + 3 \frac{1}{1-q} \sum_{j_1, j_2 = 1}^{6}
    R_{j_1j_2}
    \frac{m_{j_2}^{-1}q^{\frac{1}{2}} - m_{j_1} m_{j_2}q}{q-1}
\end{align}




\subsection{\texorpdfstring{ $\mathcal{N} = 4$ $SO(4)$ SYM}{}}

The Lie algebra $\mathfrak{so}(4)$ is isomorphic to $\mathfrak{su}(2)^2$. The Schur index of a Lagrangian theory is only sensitive to the gauge Lie algebra, and therefore the $\mathcal{N} = 4$ $SO(4)$ and $SU(2)^2$ gauge theory share an identical Schur index,
\begin{align}
  \mathcal{I}_{SU(2)^2} = \mathcal{I}_{SO(4)}
  = \frac{1}{4}\eta(\tau)^{4} \frac{\eta(\tau)^2}{\vartheta_4(\mathfrak{b})^2} & \ \oint \prod_{A = 1}^{2} \frac{da_A}{2\pi i a_A} \prod_{\alpha, \beta = \pm}\prod_{A < B} \frac{\vartheta_1(\alpha \mathfrak{a}_A + \beta \mathfrak{a}_B)}{\vartheta_4 (\alpha \mathfrak{a}_A + \beta\mathfrak{a}_B + \mathfrak{b})} \nonumber \\
  \coloneqq & \ \oint \prod_{A = 1}^{2} \frac{da_A}{2\pi i a_A} \mathcal{Z}(\mathfrak{a}_1, \mathfrak{a}_2) \ .
\end{align}
In the following we will compute a few full Wilson operator index and compare it with the $S$-dual `t Hooft operator index using the formula in \cite{Gang:2012yr}. 


We first analyze the index of a full Wilson operator associated to the vector representation $\mathbf{4}$ and its $S$-dual. The full Wilson index reads
\begin{align}
  \langle W^\text{full}_{\mathbf{4}}\rangle_{SO(4) \ \mathcal{N} = 4} = \oint \prod_{A = 1}^{2} \frac{da_A}{2\pi i a_A} (a_1 + \frac{1}{a_1} + a_2 + \frac{1}{a_2})^2 \mathcal{Z} \ .
\end{align}
By a change of variables $\mathfrak{a}_1' \coloneqq \mathfrak{a}_1 + \mathfrak{a}_2$ and $\mathfrak{a}'_2 \coloneqq \mathfrak{a}_1 - \mathfrak{a}_2$,  the Wilson index can be rewritten as a product
\begin{align}
  \langle W^\text{full}_{\mathbf{4}}\rangle_{SO(4) \ \mathcal{N} = 4}
  = \left[- \frac{1}{2} \oint \frac{da}{2\pi i a} \frac{(a + 1)^2}{a} \frac{\vartheta_1(\pm \mathfrak{a})}{\vartheta_4(\pm \mathfrak{a} + \mathfrak{b})} \frac{\eta(\tau)^3}{\vartheta_4(\mathfrak{b})}\right]^2 \ ,
\end{align}
which is identical to
\begin{align}
  (\langle W_{j = 1/2}^\text{full} \rangle_{SU(2) \ \mathcal{N} = 4})^2 \ .
\end{align}
The vector representation of $SO(4)$ is minuscule, and the S-dual `t Hooft index is safe from monopole bulling, given by
\begin{align}
  \langle H\rangle_{SO(4) \ \mathcal{N} = 4}
  = \oint \prod_{A = 1}^{2} \frac{da_A}{2\pi i a_A}
  \frac{4q^{\frac{1}{4}} (ba_1 - a_2)(-a_1 + ba_2)(b - a_1 a_2)(-1 + b a_1 a_2)}{b^2(\sqrt{q}a_1 - a_2)(\sqrt{q}a_2 - a_1)(\sqrt{q} - a_1 a_2)(-1 + \sqrt{q}a_1 a_2 ) } \mathcal{Z}' \ , \nonumber
\end{align}
where
\begin{align}
  \mathcal{Z}' = \frac{1}{4} \eta(\tau)^4 \frac{\eta(\tau)^2}{\vartheta_4(\mathfrak{b})^2}\prod_{\alpha, \beta = \pm} \frac{\vartheta_4(\alpha \mathfrak{a}_1 + \beta \mathfrak{a}_2)}{\vartheta_1(\alpha \mathfrak{a}_1 + \beta \mathfrak{a}_2 + \mathfrak{b})} \ .
\end{align}
In terms of the $a'$ variables, the above factorizes into
\begin{align}
  \langle H\rangle_{SO(4) \ \mathcal{N} = 4} = \left[\oint \frac{da'}{2\pi i a'_1}\frac{q^{\frac{1}{8}}(b - a'_1)(-1 + b a'_1)}{b(\sqrt{q} - a'_1)(-1 + \sqrt{q}a'_1)}\frac{\eta(\tau)^3}{\vartheta_4(\mathfrak{b})} \frac{\vartheta_4(\pm \mathfrak{a}')}{\vartheta_1(\pm\mathfrak{a}' + \mathfrak{b})}\right]^2 \ .
\end{align}
Up to the square and some simple factors, the result is identical to that of the $U(2)$ minimal `t Hooft operator index (\ref{U2-t-hooft}) in section \ref{section:N4SU(2)}, and naturally
\begin{align}
  \langle H\rangle_{SO(4) \ \mathcal{N} = 4} = \langle W^\text{full}_{\mathbf{4}}\rangle_{SO(4) \ \mathcal{N} = 4} \ .
\end{align}




Next we consider the index of a full Wilson operator in chiral spinor representation $\mathbf{2}$. The corresponding character is
\begin{align}
  \chi_\mathbf{2}(a) = \frac{1}{\sqrt{a_1 a_2}} + \sqrt{a_1 a_2} \ ,
\end{align}
and the relevant index is given by
\begin{align}
  \langle W_\mathbf{2}^\text{f}\rangle_{SO(4) \ \mathcal{N} = 4} = & \ \oint \prod_{A = 1}^{2} \frac{da_A}{2\pi i a_A}
  \chi_{\mathbf{2}}(a)^2 
  \mathcal{Z}(\mathfrak{a}_1, \mathfrak{a}_2) \\
  = & \ \oint \prod_{A = 1}^{2} \frac{da_A}{2\pi i a_A}
  (1 + 1 + a_1 a_2 + \frac{1}{a_1 a_2})
  \mathcal{Z}(\mathfrak{a}_1, \mathfrak{a}_2) \ .
\end{align}
In terms of the $a'$ variable, the above factorizes
\begin{align}
  \langle W_\mathbf{2}^\text{f}\rangle_{SO(4) \ \mathcal{N} = 4}
  = & \ \left[\oint \frac{da'_1}{2\pi i a'_1}(\chi_{j = 0} + \chi_{j = 1})(a'_1) \left(- \frac{1}{2}\right)\frac{\eta(\tau)^3}{\vartheta_4(\mathfrak{b})} \frac{\vartheta_1(\pm\mathfrak{a}'_1)}{\vartheta_1(\pm\mathfrak{a}'_1 + \mathfrak{b})}\right] \mathcal{I}_{SU(2) \ \mathcal{N} = 4} \nonumber \\
  = & \ \left( \mathcal{I}_{\mathcal{N} = 4 \ SU(2)} + \langle W^\text{full}_{j = 1}\rangle_{\mathcal{N} = 4 \ SU(2)} \right) \mathcal{I}_{SU(2) \ \mathcal{N} = 4} \ .
\end{align}

The S-dual `t Hooft line index is given by
\begin{align}
  \langle H\rangle
  = \oint \prod_{A = 1}^{2} \frac{da_A}{2\pi i a_A}
  \frac{2(b - a_1 a_2)(-1 + b a_1a_2)}{bq^{\frac{1}{4}}(\sqrt{q} - a_1 a_2)(-1 + \sqrt{q} a_1 a_2)}
  \mathcal{Z}' \ , \nonumber
\end{align}
where
\begin{equation}
  \mathcal{Z}' = \frac{1}{4}\eta(\tau)^4 \frac{\eta(\tau)^2}{\vartheta_4(\mathfrak{b})^2}
  \frac{\vartheta_4(\pm (\mathfrak{a}_1 + \mathfrak{a}_2))}{\vartheta_1(\pm (\mathfrak{a}_1 + \mathfrak{a}_2) + \mathfrak{b})}
  \frac{\vartheta_1(\pm (\mathfrak{a}_1 - \mathfrak{a}_2))}{\vartheta_4(\pm (\mathfrak{a}_1 - \mathfrak{a}_2) + \mathfrak{b})}  \ .
\end{equation}
In terms of the $a'$ variables,
\begin{align}
  \langle H\rangle
  = \left[- \oint \frac{da'_1}{2\pi i a'_1} \frac{(b - a'_1)(-1 + b a_1' )}{bq^{1/4}(\sqrt{q} - a'_1)(-1 + \sqrt{q}a'_1)} \frac{\vartheta_4(\pm \mathfrak{a}_1)}{\vartheta_1(\pm \mathfrak{a}_1 + \mathfrak{b})} \frac{\eta(\tau)^3}{\vartheta_4(\mathfrak{b})}\right] \mathcal{I}_{\mathcal{N} = 4 \ SU(2)} \ .
\end{align}
The equality from S-duality also follows from the discussion in section \ref{section:N4SU(2)}.




\subsection{\texorpdfstring{$\mathcal{N} = 4$ $SO(5)$ SYM}{}}



Let us now consider $\mathcal{N} = 4$ $SO(5)$ SYM with insertion of a half Wilson operator in the fundamental representation
\begin{align}
  \oint \frac{da_1}{2\pi i a_1}\frac{da_2}{2\pi i a_2}\chi_{\mathbf{5}}(a)
  \mathcal{Z}(\mathfrak{a}_1, \mathfrak{a}_2) \ ,
\end{align}
where
\begin{equation}
\mathcal{Z}(\mathfrak{a}_1, \mathfrak{a}_2) = \frac{1}{8} \frac{\eta(\tau)^6}{\vartheta_4(\mathfrak{b})^2} \frac{
    - \vartheta_1(\mathfrak{a}_1)^2
    \vartheta_1(\mathfrak{a}_2)^2
    \vartheta_1(\mathfrak{a}_1 + \mathfrak{a}_2)^2
    \vartheta_1(\mathfrak{a}_1 - \mathfrak{a}_2)^2
  }{
    \vartheta_4(\mathfrak{a}_1 \pm \mathfrak{b})
    \vartheta_4(\mathfrak{a}_2 \pm \mathfrak{b})
    \vartheta_4(\mathfrak{a}_1 + \mathfrak{a}_2\pm \mathfrak{b})
    \vartheta_4(\mathfrak{a}_1 - \mathfrak{a}_2\pm \mathfrak{b})
  }\ ,
\end{equation}
and
\begin{align}
  \chi_{\mathbf{5}}(a) = a_1 + \frac{1}{a_1} + a_2 + \frac{1}{a_2} + 1 \ .
\end{align}

From the symmetry between $\mathcal{Z}(\mathfrak{a}_1, \mathfrak{a}_2) = \mathcal{Z}(\mathfrak{a}_2, \mathfrak{a}_1)$, we only need to compute
\begin{align}
  \oint \frac{da_1}{2\pi i a_1}\frac{da_2}{2\pi i a_2}a_1^{\pm 1}
  \mathcal{Z}(\mathfrak{a}_1, \mathfrak{a}_2) \ .
\end{align}
Moreover, the symmetry $\mathcal{Z}(\mathfrak{a}_1, \mathfrak{a}_2) = \mathcal{Z}( - \mathfrak{a}_1, \mathfrak{a}_2)$ also implies
\begin{align}
  \oint \frac{da_1}{2\pi i a_1}\frac{da_2}{2\pi i a_2}a_1
  \mathcal{Z}(\mathfrak{a}_1, \mathfrak{a}_2)
  = \oint \frac{da_1}{2\pi i a_1}\frac{da_2}{2\pi i a_2}a_1^{-1}
  \mathcal{Z}(\mathfrak{a}_1, \mathfrak{a}_2) \ .
\end{align}



The $a_1$-integration picks up imaginary poles
\begin{align}
  \mathfrak{a}_1 = \alpha \mathfrak{b} + \frac{\tau}{2}, \qquad
  \mathfrak{a}_1 = \beta \mathfrak{a}_2 + \gamma \mathfrak{b} + \frac{\tau}{2} \ , \qquad \alpha, \beta, \gamma = \pm \ ,
\end{align}
with residues respectively
\begin{align}
  R_\alpha \coloneqq \frac{i}{8}\eta(\tau)^3 \frac{\vartheta_4(\mathfrak{a}_2 + \alpha \mathfrak{b})\vartheta_4(\mathfrak{a}_2 - \alpha \mathfrak{b})}{
        \vartheta_1(2 \alpha \mathfrak{b})
        \vartheta_1(\mathfrak{a}_2 + 2 \alpha \mathfrak{b})
        \vartheta_1(\mathfrak{a}_2 - 2 \alpha \mathfrak{b})} \ ,
\end{align}
and
\begin{align}
  R_{\beta \gamma} \coloneqq\frac{i}{8} \eta(\tau)^3 \frac{
    \vartheta_4(\mathfrak{a}_2 + \beta \gamma \mathfrak{b})
    \vartheta_1(\mathfrak{a}_2)
    \vartheta_4(2 \mathfrak{a}_2 + \beta \gamma \mathfrak{b})^2
  }{
    \vartheta_1(\mathfrak{a}_2 + 2 \beta \gamma \mathfrak{b})
    \vartheta_4(\mathfrak{a}_2 - \beta \gamma \mathfrak{b})
    \vartheta_1(2\mathfrak{a}_2 )
    \vartheta_1(2 \gamma \mathfrak{b}) \vartheta_1(2 \mathfrak{a}_2 + 2 \beta \gamma \mathfrak{b})
  } \ .
\end{align}
The $a_1$-integration leaves us with
\begin{align}
  \oint \frac{da_2}{2\pi i a_2} \left[
  - \sum_{\alpha = \pm} R_\alpha \frac{1}{q^\pm - 1} (b^\alpha q^{\frac{1}{2}})^\pm
  - \sum_{\beta \gamma = \pm} R_{\beta \gamma} \frac{1}{q^\pm - 1} (a_2^\beta b^\gamma q^{\frac{1}{2}})^\pm
  \right] \ .
\end{align}
The residues $R_\alpha$ and $R_{\beta \gamma}$ are all elliptic with respect to $\mathfrak{a}_2$, and therefore the $a_2$-integration of both terms can be carried out. In $R_\alpha$, there are poles and residues
\begin{align}
  \mathfrak{a}_2 = 2 \alpha \delta \mathfrak{b}, \qquad
  \mathop{\operatorname{Res}}_{\mathfrak{a}_2 = 2\alpha \delta \mathfrak{b}}R_\alpha = - \frac{\delta}{8}\frac{\vartheta_4(3 \mathfrak{b}) \vartheta_4(\mathfrak{b})}{\vartheta_1(2 \mathfrak{b}) \vartheta_1(4 \mathfrak{b})} \ .
\end{align}
Hence
\begin{align}
  & \ - \oint \frac{da_2}{2\pi i a_2} \sum_{\alpha = \pm}R_\alpha \frac{1}{q^{\pm} - 1} (b^\alpha q^{\frac{1}{2}})^\pm \\
  = & \ - \sum_{\alpha = \pm} \frac{b^{\pm\alpha} q^{\pm \frac{1}{2}}}{q^{\pm} - 1}
  \left(
  R_\alpha(\mathfrak{a} = 0) + \sum_{\delta = \pm} \frac{- \delta}{8} \frac{\vartheta_4(3 \mathfrak{b}) \vartheta_4(\mathfrak{b})}{\vartheta_1(2 \mathfrak{b}) \vartheta_1(4 \mathfrak{b})} E_1
  \begin{bmatrix}
    -1 \\ b^{2\alpha \delta}q^{\frac{1}{2}}  
  \end{bmatrix}
  \right) \ .
\end{align}
By direct computation, one sees that the above is actually independent of $\pm$ sign in the $a_1^\pm$ insertion, consistent with the symmetry $\mathcal{Z}(\mathfrak{a}_1, \mathfrak{a}_2) = \mathcal{Z}( - \mathfrak{a}_1, \mathfrak{a}_2)$.


The term with $R_{\beta \gamma}$ can be carried using (\ref{integration-formula-monomial}),
\begin{align}
  \oint \frac{da_2}{2\pi i a_2}R_{\beta \gamma} a_2^{\pm \beta}
  = - \sum_{\operatorname{real} \ j} R_{\beta \gamma j} \frac{(a^{(\beta \gamma j)}_{2}q)^{\pm \beta}}{q^{\pm \beta} - 1}
  - \sum_{\operatorname{img} \ j} R_{\beta \gamma j}\frac{(a^{(\beta \gamma j)}_{2})^{\pm \beta}}{q^{\pm \beta} - 1} \ .
\end{align}
Here $a^{(\beta \gamma j)}_{2}$ denotes the simple poles of $R_{\beta \gamma}$ with respect to $a_2$, with the corresponding residue $R_{\beta \gamma j}$. We list the poles and their residues in Table \ref{poles-residues-SO(5)}.
{
\renewcommand{\arraystretch}{1.8}
\begin{table}[h!]
\centering
  \begin{tabular}{c|c|c}
    & poles $a_2^{(\beta \gamma j)}$ & residues $R_{\beta \gamma j}$ \\
    \hline
    Real & $\mathfrak{a}_2 = - 2\beta \gamma \mathfrak{b}$ & $ + \frac{\beta}{8} \frac{\vartheta_4( \mathfrak{b})\vartheta_4(3 \mathfrak{b})}{\vartheta_1(2  \mathfrak{b})\vartheta_1(4  \mathfrak{b})}$\\
    & $\mathfrak{a}_2 = - \beta \gamma \mathfrak{b} + \frac{1}{2}$ & $ + \frac{\beta \vartheta_4(\mathfrak{b})^2 \vartheta_3(0)}{16 \vartheta_1(2 \mathfrak{b})^2 \vartheta_3(2 \mathfrak{b})}$\\
    & $\mathfrak{a}_2 = \frac{1}{2}$ & $ - \frac{\beta \vartheta_4(\mathfrak{b})
    ^2 \vartheta_2(0)}{16 \vartheta_1(2 \mathfrak{b})^2 \vartheta_2(2 \mathfrak{b})}$\\
    & $\mathfrak{a}_2 = - \beta \gamma \mathfrak{b}$ & $ - \frac{\beta \vartheta_4(\mathfrak{b})^2 \vartheta_4(0)}{16 \vartheta_1(2 \mathfrak{b})^2 \vartheta_4(2 \mathfrak{b})}$\\
    \hline
    Imaginary & $\mathfrak{a}_2 = \beta \gamma \mathfrak{b} + \frac{1}{2}$ & $- \frac{\beta}{8} \frac{\vartheta_4( \mathfrak{b})\vartheta_4(3  \mathfrak{b})}{\vartheta_1(2 \mathfrak{b})\vartheta_1(4 \mathfrak{b})}$\\
    & $\mathfrak{a}_2 = \frac{\tau}{2}$ & $\frac{\beta \vartheta_4(\mathfrak{b})^2 \vartheta_4(0)}{16 \vartheta_1(2 \mathfrak{b})^2 \vartheta_4(2 \mathfrak{b})}$\\
    & $\mathfrak{a}_2 = \frac{1}{2} + \frac{\tau}{2}$ & $ - \frac{\beta \vartheta_4(\mathfrak{b})^2 \vartheta_3(0)}{16 \vartheta_1(2 \mathfrak{b})^2 \vartheta_3(2 \mathfrak{b})}$\\
    & $\mathfrak{a}_2 = - \beta \gamma \mathfrak{b} + \frac{1}{2} + \frac{\tau}{2}$ & $ + \frac{\beta \vartheta_4(\mathfrak{b})^2 \vartheta_2(0)}{16 \vartheta_1(2 \mathfrak{b})^2 \vartheta_2(2 \mathfrak{b})}$
  \end{tabular}
  \caption{Poles and residues of the elliptic functions $R_{\beta \gamma}$.\label{poles-residues-SO(5)}}
\end{table}
}

Performing the sum over $\beta, \gamma$,
\begin{align}
  & \ - \oint \frac{da_2}{2\pi i a_2}\sum_{\beta \gamma = \pm 1} R_{\beta \gamma} \frac{1}{q^{\pm} - 1} (a_2^\beta b^\gamma q^{\frac{1}{2}})^\pm \nonumber \\
  = & \ \frac{\sqrt{q}\left( b^2(q+1)-4b\sqrt{q}+q+1 \right)}{2}\frac{\vartheta _4(\mathfrak{b} )^2}{8b(q-1)^2\vartheta _1(2\mathfrak{b} )^2}\frac{\vartheta _2(0)}{\vartheta _2(2\mathfrak{b} )} \nonumber
  \\
  & \ + ( b^2q-b\sqrt{q}(q+1)+q ) \frac{\vartheta _4(\mathfrak{b} )^2}{8b(q-1)^2\vartheta _1(2\mathfrak{b} )^2}\left[ \frac{\vartheta _3(0)}{\vartheta _3(2\mathfrak{b} )}+\frac{\vartheta _4(0)}{\vartheta _4(2\mathfrak{b} )} \right]  \nonumber\\
  & \ +\frac{\sqrt{q} \left(-2 b^4 \sqrt{q}+b^3 (q+1)+b (q+1)-2 \sqrt{q}\right) }{8 b^2 (q-1)^2} \frac{\vartheta_4(3 \mathfrak{b}) \vartheta_4(\mathfrak{b})}{ \vartheta_1(2 \mathfrak{b}) \vartheta_1(4 \mathfrak{b})} \ ,
\end{align}
which is independent of the $\pm$ in $a_2^\pm$, consistent with the symmetry $\mathcal{Z}(\mathfrak{a}_1, \mathfrak{a}_2) = \mathcal{Z}( - \mathfrak{a}_1, \mathfrak{a}_2)$.

To summarize,
\begin{align}
  & \ \oint \frac{da_1}{2\pi i a_1}\frac{da_2}{2\pi i a_2}a_1
  \mathcal{Z}(\mathfrak{a}_1, \mathfrak{a}_2) \\
  = & \ \frac{i(b^2 - 1)\sqrt{q}}{8b(q-1)} \frac{\eta(\tau)^3 \vartheta_4(\mathfrak{b})^2}{\vartheta_1(2 \mathfrak{b})^3}\\
  & \ + \frac{\sqrt{q}(1 - 4b \sqrt{q} + q + b^2(1 + q))}{16 b(q -1)^2}
  \frac{\vartheta_4(\mathfrak{b})^2}{\vartheta_1(2 \mathfrak{b})^2}
  \frac{\vartheta_2(0)}{\vartheta_2(2 \mathfrak{b})}\\
  & \ + \frac{\sqrt{q}(b - \sqrt{q})(b\sqrt{q} - 1)}{8b (q-1)^2} \frac{\vartheta_4(\mathfrak{b})^2}{\vartheta_1(2 \mathfrak{b})^2}
  \left(
  \frac{\vartheta_3(0)}{\vartheta_3(2 \mathfrak{b})}
  + \frac{\vartheta_4(0)}{\vartheta_4(2 \mathfrak{b})}
  \right)\\
  & \ + \frac{\vartheta_4(\mathfrak{b})\vartheta_4(3 \mathfrak{b})}{\vartheta_1(2 \mathfrak{b}) \vartheta_1(4 \mathfrak{b})}
  \left(
  \frac{\sqrt{q}(b - \sqrt{q})(1 - b^3 \sqrt{q})}{4b^2 (q - 1)^2}
  + \frac{(b^2 -1)\sqrt{q}}{4b(q - 1)} E_1 \begin{bmatrix}
    -1 \\ b^2 q^{\frac{1}{2}}  
  \end{bmatrix}
  \right) \ .
\end{align}
Therefore,
\begin{align}
  \langle W_\mathbf{5}\rangle_{\mathcal{N} = 4 \ SO(5)}
  = & \ \mathcal{I}_{\mathcal{N} = 4 \ SO(5)}
  + \frac{i(b^2 - 1)\sqrt{q}}{2b(q-1)} \frac{\eta(\tau)^3 \vartheta_4(\mathfrak{b})^2}{\vartheta_1(2 \mathfrak{b})^3} \nonumber\\
  & \ + \frac{\sqrt{q}(1 - 4b \sqrt{q} + q + b^2(1 + q))}{4 b(q -1)^2}
  \frac{\vartheta_4(\mathfrak{b})^2}{\vartheta_1(2 \mathfrak{b})^2}
  \frac{\vartheta_2(0)}{\vartheta_2(2 \mathfrak{b})} \nonumber\\
  & \ + \frac{\sqrt{q}(b - \sqrt{q})(b\sqrt{q} - 1)}{2b (q-1)^2} \frac{\vartheta_4(\mathfrak{b})^2}{\vartheta_1(2 \mathfrak{b})^2}
  \left(
  \frac{\vartheta_3(0)}{\vartheta_3(2 \mathfrak{b})}
  + \frac{\vartheta_4(0)}{\vartheta_4(2 \mathfrak{b})}
  \right)\\
  & \ + \frac{\vartheta_4(\mathfrak{b})\vartheta_4(3 \mathfrak{b})}{\vartheta_1(2 \mathfrak{b}) \vartheta_1(4 \mathfrak{b})}
  \left(
  \frac{\sqrt{q}(b - \sqrt{q})(1 - b^3 \sqrt{q})}{b^2 (q - 1)^2}
  + \frac{(b^2 -1)\sqrt{q}}{b(q - 1)} E_1 \begin{bmatrix}
    -1 \\ b^2 q^{\frac{1}{2}}  
  \end{bmatrix}
  \right) \ , \nonumber
\end{align}
where the $\mathcal{I}_{\mathcal{N} = 4 \ SO(5)}$ is the original Schur index of the $SO(5)$ $\mathcal{N} = 4$ SYM.



\subsubsection{General representation}

Let us consider the $\mathfrak{so}(5)$ representations whose characters can be written as polynomials of $a_1, a_2$ with integral powers,
\begin{align}
  \chi_\mathcal{R}(a) = \sum_{n_1, n_2} c_{n_1 n_2} a_1^{n_1} a_2^{n_2} \ .
\end{align}
In particular, using the symmetry $a_1 \leftrightarrow a_2$, $a_i \leftrightarrow a_i^{-1}$, we can focus on the integrals of the following form
\begin{align}
  \oint \frac{da_1}{2\pi i a_1} \frac{da_2}{2\pi i a_2}
  a_1^{n_1 > 0}a_2^{n_2 \ge 0} \mathcal{Z} \ .
\end{align}
The $a_1$-integration leaves (recall that the $a_1$-integral picks up $6$ imaginary poles)
\begin{align}\label{a1integral}
  \oint \frac{da_2}{2\pi i a_2} a_2^{n_2} \left[
  - \sum_{\alpha = \pm}R_\alpha \frac{b^{\alpha n_1} q^{\frac{1}{2} n_1}}{q^{n_1} - 1}
  - \sum_{\beta \gamma = \pm}R_{\beta \gamma} \frac{a_2^{n_1\beta} b^{n_1\gamma} q^{\frac{1}{2} n_1}}{q^{n_1} - 1}
  \right] \ .
\end{align}
Depending on whether $n_2 = 0$ or $n_2 \ne 0$ in the first term, and whether $n_2 \pm n_1 = 0$ in the second term, the integral leads to different closed-form result. When $n_2 = 0$, the first term integrates to
\begin{align}\label{Rintegrate}
  = \delta_{n_2 = 0} \frac{1}{4}\frac{\vartheta_4(\mathfrak{b})}{\vartheta_1(2 \mathfrak{b})}
  \left(
  \frac{i \eta(\tau)^3 \vartheta_4(\mathfrak{b})}{2 \vartheta_1(2 \mathfrak{b})^2} + \frac{\vartheta_4(3 \mathfrak{b})}{\vartheta_1(4 \mathfrak{b})}
  E_1 \begin{bmatrix}
    1 \\ b^2  
  \end{bmatrix}
  \right)
  \frac{b^{n_1} - b^{-n_1}}{q^{n_1/2} - q^{- n_1/2}} \ ,
\end{align}
while when $n_2 > 0$, it integrates to
\begin{align}
  & \ - \delta_{n_2 > 0} \sum_{\alpha = \pm} \frac{b^{n_1 \alpha} q^{\frac{1}{2}n_1}}{q^{n_1} - 1}
  \sum_{\delta = \pm 1}
  \left(- \frac{\delta}{8} \frac{\vartheta_4(3 \mathfrak{b}) \vartheta_4( \mathfrak{b})}{\vartheta_1(2 \mathfrak{b})\vartheta_1(4 \mathfrak{b})}\right)
  \frac{(b^{2\alpha \delta} q^{\frac{1}{2}})^{n_2}}{q^{n_2/2} - q^{- n_2/2}} \ \nonumber\\
  = & \ - \delta_{n_2 > 0}\frac{\vartheta_4(\mathfrak{b}) \vartheta_4(3 \mathfrak{b})}{8 \vartheta_1(2 \mathfrak{b}) \vartheta_1(4 \mathfrak{b})} \frac{(b^{n_1} - b^{-n_1})(b^{2n_2} - b^{-2n_2})}{(q^{n_1/2} - q^{- n_1/2})(1 - q^{-n_2})} \ .
\end{align}

In the second term, when $0 < n_1 \ne n_2$, we have $n_2 + n_1 \beta \ne 0$ for either $\beta = \pm 1$. In this case,
\begin{align}\label{RalphabetaIntegrate}
  & \ - \delta_{n_1 \ne n_2} \oint \frac{da_2}{2\pi i a_2}\sum_{\beta \gamma = \pm} \frac{b^{n_1 \gamma} q^{\frac{1}{2}n_1}}{q^{n_1} - 1} R_{\beta \gamma}
  a_2^{n_2 + n_1 \beta} \\
  = & \ + \delta_{n_1 \ne n_2}\sum_{\beta \gamma = \pm}\frac{b^{n_1 \gamma} q^{\frac{1}{2}n_1}}{q^{n_1} - 1}  \sum_{\text{real/img} \ j} R_{\beta \gamma j} \frac{(a_2^{(\beta \gamma j)} q^{\pm \frac{1}{2}})^{n_2 + n_1 \beta}}{q^{\frac{1}{2}(n_2 + n_1 \beta)} - q^{-\frac{1}{2}(n_2 + n_1 \beta)} } \ .
\end{align}
On the other hand, when $n_1 = n_2 > 0$, we have $n_2 + n_1 \beta = 0$ for $\beta = -1$, and $n_2 + n_1 \beta = 2n_1 \ne 0$ for $\beta = 1$. In this situation,
\begin{align}
  & \ - \oint \frac{da_2}{2\pi i a_2}\sum_{\beta \gamma = \pm}R_{\beta \gamma} \frac{a_2^{n_2 + n_1 \beta} b^{n_1 \gamma} q^{\frac{1}{2}n_1}}{q^{n_1} - 1}
  \nonumber\\
  = & \ \delta _{n_1=n_2}\sum_{\gamma =\pm}{\left[ \sum_{\text{real}/\text{img}\ j}{R_{+\gamma j}\frac{( a_{2}^{\left( +\gamma j \right)})^{2n_1} q^{\pm \frac{1}{2}2n_1}}{q^{n_1}-q^{-n_1}}} \right]}\frac{b^{n_1\gamma}q^{\frac{1}{2}n_1}}{q^{n_1}-1}\\
  & \ -\delta _{n_1=n_2}\sum_{\gamma =\pm}
  \frac{b^{n_1\gamma}q^{\frac{1}{2}n_1}}{q^{n_1}-1}
  \left( R_{-\gamma}\left( \mathfrak{a} _2=\mathfrak{a}_2^{(0)} \right) +\sum_{\text{real}/\text{img} \ j}{R_{-\gamma j}}E_1\begin{bmatrix}
    -1\\
    \frac{a_{2}^{\left( -\gamma j \right)}}{a_2^{(0)}}q^{\pm \frac{1}{2}}  
  \end{bmatrix} \right) \nonumber \ ,
\end{align}
where $a_2^{(0)}$ is a generic reference value, for example, $a_2^{(0)} = b^3$. In the above, we have used the poles and residues in Table \ref{poles-residues-SO(5)}. Putting all the contributions together, we deduce that for $n_1 > 0, n_2 \ge 0$,
\begin{align}
  & \ \oint\prod_{i = }^{2}\frac{da_i}{2\pi i a_i}a_1^{n_1}a_2^{n_2} \mathcal{Z}\\
  = & \ \delta_{n_2 = 0}\frac{1}{4}\frac{\vartheta_4(\mathfrak{b})}{\vartheta_1(2 \mathfrak{b})}
  \left(
  \frac{i \eta(\tau)^3 \vartheta_4(\mathfrak{b})}{2 \vartheta_1(2 \mathfrak{b})^2} + \frac{\vartheta_4(3 \mathfrak{b})}{\vartheta_1(4 \mathfrak{b})}
  E_1 \begin{bmatrix}
    1 \\ b^2  
  \end{bmatrix}
  \right)
  \frac{b^{n_1} - b^{-n_1}}{q^{n_1/2} - q^{- n_1/2}} \\
  & \ - \delta_{n_2 > 0}\frac{\vartheta_4(\mathfrak{b}) \vartheta_4(3 \mathfrak{b})}{8 \vartheta_1(2 \mathfrak{b}) \vartheta_1(4 \mathfrak{b})} \frac{(b^{n_1} - b^{-n_1})(b^{2n_2} - b^{-2n_2})}{(q^{n_1/2} - q^{- n_1/2})(1 - q^{-n_2})} \\
  & \ + \delta_{n_1 \ne n_2}\sum_{\beta \gamma = \pm}\frac{b^{n_1 \gamma} q^{\frac{1}{2}n_1}}{q^{n_1} - 1}  \sum_{\text{real/img} \ j} R_{\beta \gamma j} \frac{(a_2^{(\beta \gamma j)} q^{\pm \frac{1}{2}})^{n_2 + n_1 \beta}}{q^{\frac{1}{2}(n_2 + n_1 \beta)} - q^{-\frac{1}{2}(n_2 + n_1 \beta)} }\\
  & \ + \delta _{n_2=n_1}\sum_{\gamma =\pm}{\left[ \sum_{\text{real}/\text{img}\ j}{R_{+\gamma j}\frac{( a_{2}^{\left( +\gamma j \right)})^{2n_1} q^{\pm \frac{1}{2}2n_1}}{q^{n_1}-q^{-n_1}}} \right]}\frac{b^{n_1\gamma}q^{\frac{1}{2}n_1}}{q^{n_1}-1}\\
    & \ -\delta _{n_2=n_1}\sum_{\gamma =\pm}
    \frac{b^{n_1\gamma}q^{\frac{1}{2}n_1}}{q^{n_1}-1}
    \left( R_{-\gamma}\left( \mathfrak{a} _2=\mathfrak{a}_2^{(0)} \right) +\sum_{\text{real}/\text{img} \ j}{R_{-\gamma j}}E_1\begin{bmatrix}
      -1\\
      \frac{a_{2}^{\left( -\gamma j \right)}}{a_2^{(0)}}q^{\pm \frac{1}{2}}  
    \end{bmatrix} \right) \ .
\end{align}
The Wilson index corresponding to the $SO(5)$ representations with Dynkin labels $[n, 0]$ can be computed using the above integration formula by simple substitution, sine the corresponding character can be written as a sum of simple monomials,
\begin{align}
  \chi_{[n,0]}\left(a_1,a_2\right)
  = & \ \sum_{m=0}^n \sum_{j=0}^{m} \sum_{i=0}^{m}
  a_1^{j-i} a_2^{i+j-m} \\
  = & \ \lceil\frac{n+1}{2}\rceil+\sum_{m=0}^n\sum_{\substack{i=0\\i\neq m/2}}^m a_2^{2i-m}+\sum_{m=1}^n \sum_{\substack{i,j = 0 \\ i\ne j}}^n a_2^{i+j-m}a_1^{j-i} \\
  \sim & \ \lceil\frac{n+1}{2}\rceil+\sum_{m=0}^n\sum_{\substack{i=0\\i\neq m/2}}^m a_1^{|2i-m|}+\sum_{m=1}^n \sum_{\substack{i,j = 0 \\ i\ne j}}^n a_2^{|i+j-m|}a_1^{|j-i|} \ .
\end{align}
Here in the last line we have rewritten the expression using the symmetries $a_1 \leftrightarrow a_2$, $a_i \leftrightarrow a_i^{-1}$ of the integral, so that each term can be easily computed with the above integration formula. Although the Wilson line index can be computed straightforwardly simply by substitution, we are unfortunately unable to reorganize the final result in an elegant form, so we will refrain from presenting the final expression of $\langle W_{[n,0]}\rangle_{\mathcal{N} = 4 \ SO(5)}$ here.


% \YP{to be continued}
% \YW{For an $SO(5)$ irreducible representation with Dynkin label $[n_1,n_2]$, the character can be written as
% \begin{align}
% 	\chi_{[n_1,n_2]}(a_1,a_2)=\frac{\left|
% 		\begin{array}{cc}
% 			a_1^{n_1+\frac{n_2}{2}+\frac{3}{2}}-a_1^{-n_1-\frac{n_2}{2}-\frac{3}{2}} & a_2^{n_1+\frac{n_2}{2}+\frac{3}{2}}-a_2^{-n_1-\frac{n_2}{2}-\frac{3}{2}} \\
% 			a_1^{\frac{n_2}{2}+\frac{1}{2}}-a_1^{-\frac{n_2}{2}-\frac{1}{2}} & a_2^{\frac{n_2}{2}+\frac{1}{2}}-a_2^{-\frac{n_2}{2}-\frac{1}{2}} \\
% 		\end{array}
% 		\right|}{\left|
% 		\begin{array}{cc}
% 			a_1^{\frac{3}{2}}-a_1^{-\frac{3}{2}} & a_2^{\frac{3}{2}}-a_2^{-\frac{3}{2}} \\
% 			a_1^{\frac{1}{2}}-a_1^{-\frac{1}{2}} & a_2^{\frac{1}{2}}-a_2^{-\frac{1}{2}} \\
% 		\end{array}
% 		\right|}.
% \end{align}
% We shall concentrate on the case when $n_1=n$, $n_2=0$ first. In this case the character can be recast into:
% \begin{align}\label{character resummation}
% &\chi_{[n,0]}\left(a_1,a_2\right)=\sum_{m=0}^n \sum_{j=0}^{m} \sum_{i=0}^{m}a_2^{i+j-m}a_1^{j-i}\notag\\
% &=\sum_{m=0}^n\sum_{i=0}^m a_2^{2i-m}+\sum_{m=1}^n \sum_{m\geq i>j\geq 0}a_2^{i+j-m}a_1^{j-i}+\sum_{m=1}^n\sum_{m\geq j>i\geq 0}a_2^{i+j-m}a_1^{j-i}\notag\\
% &=\lceil\frac{n+1}{2}\rceil+\sum_{m=0}^n\sum_{\substack{i=0\\i\neq m/2}}^m a_2^{2i-m}+\sum_{m=1}^n \sum_{m\geq i>j\geq 0}a_2^{i+j-m}a_1^{j-i}+\sum_{m=1}^n\sum_{m\geq j>i\geq 0}a_2^{i+j-m}a_1^{j-i}
% \end{align}
% For the second term, we have to deal with the following integral:
% \begin{align}
% \sum_{m=0}^{n}\sum_{\substack{i=0\\ i\neq m/2}}^m \oint_{|a_1|=1}\oint_{|a_2|=1}\frac{da_1}{2\pi i a_1}\frac{da_2}{2\pi i a_2}a_2^{2i-m}\mathcal{Z}(a_1,a_2).
% \end{align}
% Performing the $a_2$ integral by using (\ref{a1integral}) it gives:
% \begin{align}
% \sum_{m=0}^{n}\sum_{\substack{i=0\\ i\neq m/2}}^m \oint_{|a_1|=1}\frac{da_1}{2\pi i a_1}\left(-\sum_{\alpha=\pm}R_{\alpha}\frac{b^{\alpha(2i-m)}q^{(2i-m)/2}}{q^{2i-m}-1}-\sum_{\beta\gamma=\pm}R_{\beta\gamma}\frac{a_1^{\beta(2i-m)}b^{\gamma(2i-m)}q^{(2i-m)/2}}{q^{2i-m}-1}\right)
% \end{align}
% Note that
% \begin{align}
% \sum_{m=0}^{n}\sum_{\substack{i=0\\ i\neq m/2}}^m f(2i-m)=\sum_{i=1}^n\left(f(i)+f(-i)\right)\lceil\frac{n-i+1}{2}\rceil,
% \end{align}
% the result can be recast into:
% \begin{align}
% -\sum_{i=1}^n \sum_{\alpha=\pm}2\lceil\frac{n-i+1}{2}\rceil\frac{b^{\alpha i}}{q^{i/2}-q^{-i/2}}\oint \frac{da_1}{2\pi ia_1}R_\alpha -\sum_{i=1}^n \sum_{\beta,\gamma=\pm}2\lceil\frac{n-i+1}{2}\rceil\frac{b^{\gamma i}}{q^{i/2}-q^{-i/2}}\oint \frac{da_1}{2\pi i a_1}R_{\beta\gamma}a_1^{\beta i}
% \end{align}
% Recall the integral formula (\ref{Rintegrate}) and (\ref{RalphabetaIntegrate}), we can obtain:
% \begin{align}\label{part1SO5SYM}
% &= \sum_{i=1}^n\sum_{\alpha,\delta}\lceil\frac{n-i+1}{2}\rceil\frac{b^{\alpha i}\delta}{4(q^{i/2}-q^{-i/2})}Q E_1\begin{bmatrix}
% -1\\
% b^{2\alpha\delta}q^{1/2}
% \end{bmatrix}\notag\\
% &-\sum_{\beta\gamma}\sum_{i=1}^n \sum_{\text{real/Imag}j}2\lceil\frac{n-i+1}{2}\rceil \frac{\beta b^{\gamma i}}{\left(q^{i/2}-q^{-i/2}\right)^2} R_{\beta\gamma j}\left(a_2^{(\beta\gamma j)}q^{\pm 1/2}\right)^{\beta i},
% \end{align}
% where $Q=\frac{\vartheta_4\left(3\mathfrak{b},q\right)\vartheta_4\left(\mathfrak{b},q\right)}{\vartheta_1\left(2\mathfrak{b},q\right)\vartheta_1\left(4\mathfrak{b},q\right)}$.

% Since the last two terms in (\ref{character resummation}) gives the same result after integration, we only need to compute
% \begin{align}
% \sum_{m=1}^n \sum_{0\leq j<i\leq m}\oint \frac{da_1}{2\pi ia_1}\frac{da_2}{2\pi ia_2}a_2^{i+j-m}a_1^{j-i}\mathcal{Z}(a_1,a_2).
% \end{align}
% Using (\ref{a1integral}), we can perform the $a_1$ integral:
% \begin{align}
% &=-\sum_{m=1}^n \sum_{0\leq j<i\leq m}\sum_{\alpha}\frac{b^{\alpha(j-i)}q^{(j-i)/2}}{q^{j-i}-1}\oint\frac{da_2}{2\pi i a_2}a_2^{i+j-m}R_\alpha\notag\\
% &-\sum_{m=1}^n \sum_{0\leq j<i\leq m} \sum_{\beta\gamma}\frac{b^{\gamma(j-i)}q^{(j-i)/2}}{q^{j-i}-1}\oint\frac{da_2}{2\pi i a_2}a_2^{i+j+\beta(j-i)-m}R_{\beta\gamma}
% \end{align}
% The first term from above equals to:
% \begin{align}\label{part2SO5SYM}
% &=\frac{1}{8}Q\sum_{m=1}^n \sum_{\substack{0\leq j<i\leq m\\i+j=m}}\sum_{\alpha\delta}\frac{\delta b^{\alpha(j-i)}}{q^{(j-i)/2}-q^{-(j-i)/2}}E_1\begin{bmatrix}
% -1\\
% b^{2\alpha\delta}q^{1/2}
% \end{bmatrix}\\
% &+\frac{1}{8}Q\sum_{m=1}^n \sum_{\substack{0\leq j<i\leq m\\i+j\neq m}}\sum_{\alpha\delta}\frac{\delta b^{\alpha(j-i)}}{q^{(j-i)/2}-q^{-(j-i)/2}}\frac{(b^{2\alpha\delta}q^{1/2})^{i+j-m}}{q^{(i+j-m)/2}-q^{-(i+j-m)/2}}
% \end{align}
% The second term gives:
% \begin{align}\label{part3SO5SYM}
% &-\sum_{m=1}^n \sum_{0\leq j<i\leq m} \sum_{\gamma}\frac{b^{\gamma(j-i)}q^{(j-i)/2}}{q^{j-i}-1}\left(\oint\frac{da_2}{2\pi i a_2}a_2^{2j-m}R_{+1\gamma}+\oint\frac{da_2}{2\pi i a_2}a_2^{2i-m}R_{-1\gamma}\right)\\
% &=-\sum_{m=1}^n \sum_{\substack{0\leq j<i\leq m\\j= m/2}}\sum_{\gamma}\frac{b^{\gamma(j-i)}q^{(j-i)/2}}{q^{j-i}-1}\sum_{\substack{\text{Real/Imag}\\J}}R_{+1\gamma J}E_1\begin{bmatrix}
% -1\\
% a_2^{(+1\gamma J)}q^{\pm\frac{1}{2}}
% \end{bmatrix}\\
% &-\sum_{m=1}^n \sum_{\substack{0\leq j<i\leq m\\j\neq m/2}}\sum_{\gamma}\frac{b^{\gamma(j-i)}q^{(j-i)/2}}{q^{j-i}-1}\sum_{\substack{\text{Real/Imag}\\J}}R_{+1\gamma J}\frac{\left(a_2^{(+1\gamma J)}q^{\pm 1/2}\right)^{2j-m}}{q^{(2j-m)/2}-q^{-(2j-m)/2}}\\
% &-\sum_{m=1}^n \sum_{\substack{0\leq j<i\leq m\\i= m/2}}\sum_{\gamma}\frac{b^{\gamma(j-i)}q^{(j-i)/2}}{q^{j-i}-1}\sum_{\substack{\text{Real/Imag}\\J}}R_{-1\gamma J}E_1\begin{bmatrix}
% 	-1\\
% 	a_2^{(-1\gamma J)}q^{\pm\frac{1}{2}}
% \end{bmatrix}\\
% &-\sum_{m=1}^n \sum_{\substack{0\leq j<i\leq m\\i\neq m/2}}\sum_{\gamma}\frac{b^{\gamma(j-i)}q^{(j-i)/2}}{q^{j-i}-1}\sum_{\substack{\text{Real/Imag}\\J}}R_{-1\gamma J}\frac{\left(a_2^{(-1\gamma J)}q^{\pm 1/2}\right)^{2i-m}}{q^{(2i-m)/2}-q^{-(2i-m)/2}}
% \end{align}
% Combine three parts (\ref{part1SO5SYM}), (\ref{part2SO5SYM}), and (\ref{part3SO5SYM}) together with the orginal Schur index, we can get the final result.

% Observing that only when $n_1=n\in \mathbb{N}_{>0}$ and $n_2=2m$, $m\in \mathbb{N}_{>0}$, the insertion of $\chi_{[n_1,n_2]}$ gives non-zero result after integration. The result above can certainly cover this case by replacing $n$ with $n+m$.}
% According to the integral formulas as follows
% \begin{align}
% & \oint_{|a_2|=1}\frac{da_2}{2\pi i a_2}a_2^n \mathcal{Z}(a_1,a_2)=-R(a_1,b,q)\frac{b^{n}-b^{-n}}{q^{n/2}-q^{-n/2}}-\sum_{\alpha,\beta=\pm}R_{\alpha\beta}(a_1,b,q)\frac{a_1^{\alpha n}b^{\beta % n}}{q^{n/2}-q^{-n/2}}\\
% & \oint_{|a_2|=1}\frac{da_2}{2\pi i a_2}R(a_2,b,q)=A\left(E_1\begin{bmatrix}
%	-1\\
%	b
%\end{bmatrix}+E_1\begin{bmatrix}
%	-1\\
%	b^3
% \end{bmatrix}\right)\\
% & \oint_{|a_2|=1}\frac{da_2}{2\pi i a_2}a_2^{n}R_{\alpha\beta}(a_2,b,q)=(-1)^n\alpha B\frac{\left( b^{-n\alpha\beta}q^{n/2}-1\right)}{2\left(q^{n/2}-q^{-n/2}\right)}q^{n/2}-(-1)^n \alpha % C\frac{q^{n/2}-b^{-n\alpha\beta}}{2\left(q^{n/2}-q^{-n/2}\right)}q^{n/2}\notag\\
% &-\alpha A\frac{b^{n\alpha\beta}q^{n/2}-b^{-2n\alpha\beta}}{q^{n/2}-q^{-n/2}}q^{n/2}+\alpha D\frac{q^{n/2}-b^{-n\alpha\beta}}{2\left(q^{n/2}-q^{-n/2}\right)}q^{n/2},
% \end{align}
% We have
% \begin{align}
% &\sum_{m=0}^n\sum_{\substack{i=0\\i\neq m/2}}^m\oint_{|a_1|=1}\frac{da_1}{2\pi ia_1}\oint_{|a_2|=1}\frac{da_2}{2\pi i a_2} a_2^{2i-m}\mathcal{Z}(a_1,a_2)\notag\\
% &=\sum_{m=0}^n\sum_{\substack{i=0\\i\neq m/2}}^m\oint_{|a_1|=1}\frac{da_1}{2\pi % ia_1}\left(-R(a_1,b,q)\frac{b^{2i-m}-b^{m-2i}}{q^{(2i-m)/2}-q^{-(2i-m)/2}}-\sum_{\alpha,\beta=\pm}R_{\alpha\beta}(a_1,b,q)\frac{a_1^{\alpha (2i-m)}b^{\beta % (2i-m)}}{q^{(2i-m)/2}-q^{-(2i-m)/2}}\right)\notag\\
% &=-\sum_{m=0}^n\sum_{\substack{i=0\\i\neq m/2}}^m\frac{b^{2i-m}-b^{m-2i}}{q^{(2i-m)/2}-q^{-(2i-m)/2}}A\left(E_1\begin{bmatrix}
%	-1\\
%	b
% \end{bmatrix}+E_1\begin{bmatrix}
%	-1\\
% 	b^3
% \end{bmatrix}\right)\notag\\
% &-\sum_{m=0}^n\sum_{\substack{i=0\\i\neq m/2}}^m \sum_{\alpha,\beta=\pm}\frac{b^{\beta (2i-m)}}{q^{(2i-m)/2}-q^{-(2i-m)/2}}\left((-1)^m\alpha B\frac{\left( % b^{-(2i-m)\beta}q^{\alpha(2i-m)/2}-1\right)}{2\left(q^{\alpha(2i-m)/2}-q^{-\alpha(2i-m)/2}\right)}q^{\alpha(2i-m)/2}\right.\notag\\
% &-(-1)^m \alpha C\frac{q^{\alpha(2i-m)/2}-b^{-(2i-m)\beta}}{2\left(q^{\alpha(2i-m)/2}-q^{-\alpha(2i-m)/2}\right)}q^{\alpha(2i-m)/2}-\alpha %A\frac{b^{(2i-m)\beta}q^{\alpha(2i-m)/2}-b^{-2(2i-m)\beta}}{q^{\alpha(2i-m)/2}-q^{-\alpha(2i-m)/2}}q^{\alpha(2i-m)/2}\notag\\
% &\left.+\alpha D\frac{q^{\alpha(2i-m)/2}-b^{-(2i-m)\beta}}{2\left(q^{\alpha(2i-m)/2}-q^{-\alpha(2i-m)/2}\right)}q^{\alpha(2i-m)/2}\right)
%\end{align}
%where:
%\begin{align}
%	A=\frac{\vartheta_4\left(3\mathfrak{b},q\right)\vartheta_4\left(\mathfrak{b},q\right)}{\vartheta_1\left(2\mathfrak{b},q\right)\vartheta_1\left(4\mathfrak{b},q\right)}\quad %B=\frac{\vartheta_2\left(0,q\right)\vartheta_4^2\left(\mathfrak{b},q\right)}{\vartheta_1^2\left(2\mathfrak{b},q\right)\vartheta_2\left(2\mathfrak{b},q\right)}\quad %C=\frac{\vartheta_3\left(0,q\right)\vartheta_4^2\left(\mathfrak{b},q\right)}{\vartheta_1^2\left(2\mathfrak{b},q\right)\vartheta_3\left(2\mathfrak{b},q\right)}\quad       %D=\frac{\vartheta_4\left(0,q\right)\vartheta_4^2\left(\mathfrak{b},q\right)}{\vartheta_1^2\left(2\mathfrak{b},q\right)\vartheta_4\left(2\mathfrak{b},q\right)}
%\end{align}
%To simplify a little bit, note that
%\begin{align}
%\sum_{m=0}^n \sum_{\substack{i=0\\ i\neq m/2}}^m f(2i-m)=\sum_{i=1}^{n}\left(f(i)+f(-i)\right)\lceil\frac{-i+n+1}{2}\rceil.
%\end{align}
%In this way,
% \begin{align}
% &\sum_{m=0}^n\sum_{\substack{i=0\\i\neq m/2}}^m\oint_{|a_1|=1}\frac{da_1}{2\pi ia_1}\oint_{|a_2|=1}\frac{da_2}{2\pi i a_2} a_2^{2i-m}\mathcal{Z}(a_1,a_2)\notag\\
% &=-2A\left(E_1\begin{bmatrix}
% -1\\
% b
% \end{bmatrix}+E_1\begin{bmatrix}
% -1\\
% b^3
% \end{bmatrix}\right)\sum_{i=1}^n \lceil\frac{n-i+1}{2}\rceil\frac{b^i-b^{-i}}{q^i-q^{-i}}\notag\\
% &-2\sum_{i=1}^n\lceil\frac{n-i+1}{2}\rceil\left(A\left(\frac{-(q^i+q^{-i})(b^{2i}+b^{-2i})+(b^{-i}+b^{i})(q^{i/2}+q^{-i/2})}{(q^{-i/2}-q^{i/2})^2}\right)\right.\notag\\
% &-(-1)^i B \left(\frac{-2(q^i+q^{-i})+(b^i+b^{-i})(q^{i/2}+q^{-i/2})}{2(-q^{i/2}+q^{-i/2})^2}\right)\notag\\
% &-(-1)^i C\frac{-2(q^{i/2}+q^{-i/2})+(q^i+q^{-i})(b^i+b^{-i}) }{2(-q^{i/2}+q^{-i/2})^2}\notag\\
% &\left. +D\frac{-2(q^{i/2}+q^{-i/2})+(b^i+b^{-i})(q^i+q^{-i})}{2(-q^{i/2}+q^{-i/2})^2}\right)\notag\\
% &=-2A\left(E_1\begin{bmatrix}
%	-1\\
%	b
%\end{bmatrix}+E_1\begin{bmatrix}
%	-1\\
%	b^3
% \end{bmatrix}\right)\sum_{i=1}^n \lceil\frac{n-i+1}{2}\rceil\frac{b^i-b^{-i}}{q^i-q^{-i}}\notag\\
% &+16\sum_{i=1}^n \sum_{\beta,\gamma=\pm}\sum_{\text{Real}j}\lceil\frac{n-i+1}{2}\rceil\beta R_{\beta\gamma j}\left(a_2^{(\beta\gamma j)}q\right)^{\beta i}\notag\\
% &+16\sum_{i=1}^n \sum_{\beta,\gamma=\pm}\sum_{\text{Imag}j}\lceil\frac{n-i+1}{2}\rceil\beta R_{\beta\gamma j}\left(a_2^{(\beta\gamma j)}\right)^{\beta i}.
%\end{align}
%Since the last two terms in (\ref{character resummation}) have the same contribution to the integral, we only need to deal with the integral as follows:
%\begin{align}
%\sum_{m=1}^{n}\sum_{m\geq i> j\geq 0}\oint \frac{da}{2\pi i a}a_2^{i+j-m}a_1^{j-i}\mathcal{Z}(a_1,a_2)
%\end{align}}














