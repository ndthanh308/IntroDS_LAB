%!TEX root = ../Schur indices and line operators.tex

\section{Introduction}



Any 4d $\mathcal{N} = 2$ superconformal field theory contains a nontrivial protected subsector of Schur operators that form an associated two-dimensional chiral algebra \cite{Beem:2013sza}, providing an important invariants of $\mathcal{N} = 2$ SCFTs. These operators are defined as the cohomology class of some well-chosen supercharge of the $\mathcal{N} = 2$ superconformal algebra. The index that counts these BPS operators is called the Schur index $\mathcal{I}(q)$, which happens to be a special limit $t\to q$ of the full $\mathcal{N} = 2$ superconformal index $\mathcal{I}(p,q,t)$ \cite{Gadde:2011uv}. The Schur index plays a central role in the SCFT/VOA correspondence, as it coincides with the vacuum character of the associated chiral algebra.

Similar to the $S^4$ supersymmetric partition functions \cite{Pestun:2007rz}, the superconformal index \cite{Kinney:2005ej,Romelsberger:2005eg}, and in particular, the Schur index \cite{Gadde:2011uv} is an exactly computable quantity. For theories with a Lagrangian, the Schur index can be computed as a $S^3 \times S^1$ partition function and localizes to a multivariate contour integral along the the unit circles \cite{Nawata:2011un,Pan:2019bor,Jeong:2019pzg,Dedushenko:2019yiw,Oh:2019mcg}. Alternatively for theories of class-$\mathcal{S}$ \cite{Gaiotto:2009we}, the index can be identified as the partition function the 2d $q$-deformed Yang-Mills theory \cite{Gadde:2011ik,Mekareeya:2012tn,Lemos:2012ph,Lemos:2014lua,Buican:2015ina,Xie:2016evu}. There are also instances where the associated chiral algebras are known from other methods, whose module characters have already existed in the literature \cite{2016arXiv161207423K}. There are also other methods to compute the Schur index in different scenarios \cite{Zafrir:2020epd,Gadde:2010te,Agarwal:2018ejn,Kang:2021lic}. Many of these results, although exact, are not given in closed-form in terms of finite combinations of special functions with well-controlled periodic and modular properties. This problem is tackled in several recent works. The unflavored and later the flavored Schur index for the $\mathcal{N} = 4$ $SU(N)$ theories are computed in closed-form using the Fermi-gas formalism in \cite{Bourdier:2015wda,Bourdier:2015sga,Hatsuda:2022xdv} and modular anomaly equation \cite{Huang:2022bry}. In \cite{Beemetal}, the unflavored Schur index for many class-$\mathcal{S}$ theories are computed in terms of quasi-modular forms. In \cite{Pan:2021mrw}, several integration formula are proposed to compute analytically the index for a wide range of Lagrangian theories (and some non-Lagrangian ones) in terms of finite combination of twisted Eisenstein series and Jacobi theta functions.

Line operators can be introduced into 4d $\mathcal{N} = 2$ SCFTs that preserve some amount of supersymmetry \cite{Lee:1996vz,Gomis:2009ir,Gang:2012yr,Cordova:2016uwk}, and the corresponding $S^4$ partition function and superconformal index in their presence have been computed exactly \cite{Pestun:2007rz,Giombi:2009ek,Gomis:2011pf,Dimofte:2011py,Drukker:2015spa,Neitzke:2017cxz}. In the context of the AGT-correspondence, the line operators correspond to the Verlinde network operators in the Liouville/Toda CFT \cite{Drukker:2009id,Drukker:2009tz,Gomis:2010kv,Drukker:2010jp,Bullimore:2013xsa}. In \cite{Gang:2012yr}, the Schur index of supersymmetric Wilson lines and the S-dual `t Hooft lines in different gauge theories are studied, incorporating the monopole bubbling effects. In \cite{Cordova:2016uwk}, the authors propose an infrared computation method using the infrared Seiberg-Witten description and compute the line operator index for Argyres-Douglas theories and $SU(2)$ SQCD. The papers \cite{Watanabe:2016bwr,Watanabe:2017bmi} further studied the Schur index of Wilson-`t Hooft line operators in terms the punctured networks.

In this work, we focus on computing analytically the Schur index with or without line defects in 4d $\mathcal{N} = 2$ SCFTs, generalizing the work in \cite{Pan:2021mrw}. See also \cite{Hatsuda:2023iwi} for extensive analytic results on Wilson line index for the $\mathcal{N} = 2^*$ $U(N)$ theories. The key tools for our purpose are new a set of integration formula that can be applied to a wide range of 4d $\mathcal{N} = 2$ Lagrangian SCFTs, expressing multivariate integrals of elliptic functions in terms of Eisenstein series and rational functions of flavor fugacities. Concretely, we propose integration formula for integrals with the following types (and some variants) of integrand,
\begin{align}
	f(\mathfrak{z}) E_1 \begin{bmatrix}
  	\pm 1 \\ za  
	\end{bmatrix}
	E_k \begin{bmatrix}
  	\pm 1 \\ zb  
	\end{bmatrix} , \quad
	z^m f(\mathfrak{z}), \quad
	z^m E_k \begin{bmatrix}
  	\pm 1 \\ za  
	\end{bmatrix}E_\ell \begin{bmatrix}
  	\pm 1 \\ zb  
	\end{bmatrix} \ ,
	\quad
	z^m f(\mathfrak{z}) E_k \begin{bmatrix}
  	\mp 1 \\ z a  
	\end{bmatrix}
\end{align}
where $f(\mathfrak{z})$ denotes an elliptic function in $\mathfrak{z}$, and $z = e^{2\pi i \mathfrak{z}}$. These formula can be used to compute the standard Schur index of some $A_2$-type theories of class-$\mathcal{S}$, Wilson line index in $A_1$-type theories of class-$\mathcal{S}$, $SU(N)$ SQCD and $\mathcal{N} = 4$ theories of gauge group $SO(N)$. We also compute the `t Hooft line index in some simplest cases. 

Let us explain the $A_1$ case in a bit more detail. Recall that an $SU(2)$ Wilson line operator\footnote{We consider the simplest Wilson lines charged under one $SU(2)$ gauge group, and leave more general Wilson lines and their correlators to future work.} in a $A_1$ class-$\mathcal{S}$ theories $\mathcal{T}[\Sigma_{g, n}]$ is dual to a line operator on the $\Sigma_{g, n}$, and in particular, such a line operator resides at some long tube in the pants-decomposition of $\Sigma_{g,n}$ which provides a gauge theory description for $\mathcal{T}[\Sigma_{g,n}]$. We find that for $A_1$-type theories $\mathcal{T}[\Sigma_{g, n}]$, there are two major types of Wilson line operators: if the relevant tube separates $\Sigma_{g,n}$ into two disconnected parts, the index is called type-2, and otherwise type-1. Note that type-1 line operators exist only when genus $g \ge 1$. It turns out that the type-1 Wilson index is easy to compute and we are able to obtain an elegant compact closed-form, 
\begin{align}
  \langle W_{j \in \mathbb{Z}}\rangle^{(1)}_{g \ge 1, n}
  = & \ \mathcal{I}_{g,n}
  - \left[\prod_{i = 1}^{n} \frac{i \eta(\tau)}{\vartheta_1(2 \mathfrak{b}_i)}
  \right]
        \sum_{\substack{m = - j\\m \ne 0}}^{+ j}
    \left[\frac{\eta(\tau)}{q^{m/2} - q^{-m /2}}\right]^{2g - 2}
    \prod_{i = 1}^{n} \frac{b_i^m - b_i^{-m}}{q^{m/2} - q^{- m /2}} \ ,\\
  \langle W_{j \in \mathbb{Z} + \frac{1}{2}}\rangle^{(1)}_{g \ge 1, n} = & \ 0 \ .
\end{align}
Note that given a $\Sigma_{g, n}$, as long as the Wilson operator is of type-1, the index is independent of the specific tube it resides. Generalizing the observation in \cite{Cordova:2016uwk}, the type-1 index can be viewed as a linear combination of the vacuum character $\mathcal{I}_{g,n}$ of the associated chiral algebra of $\mathcal{T}[\Sigma_{g,n}]$ and a non-vacuum module character $\eta(\tau)^{2 g - 2}\prod_{i = 1}^{n}\frac{i \eta(\tau)}{\vartheta_i(2 \mathfrak{b}_i)}$, where the coefficient is a rational function
\begin{align}
	\sum_{\substack{m = -j \\ m \ne 0}}^{+j}\frac{1}{(q^{m/2} - q^{-m/2})^{2 g - 2}} \prod_{i = 1}^n \frac{b_i^m - b_i^{-m}}{q^{m/2} - q^{-m/2}} \ .
\end{align}
For the type-2 Wilson line, the closed-form index take a less elegant form. Still, we are able to identify a similar structure of finite linear combination of characters for the $SU(2)$ SQCD, $\mathcal{T}[\Sigma_{2,1}]$ and all $\mathcal{T}[\Sigma_{g, 0}]$. In particular, the type-2 Wilson index in $\mathcal{T}[\Sigma_{2,1}]$ provides two new linear independent (combinations of) characters of the associated chiral algebra, which were previously not visible from analyzing surface defects in $\mathcal{T}[\Sigma_{1,2}]$.



This paper is organized as follows. In section 2, we demonstrate that the generalized integration formula can be used to compute analytically the Schur index of a series of $A_2$-type class-$\mathcal{S}$ theories with or without Lagrangian, and of $SO(7)$ $\mathcal{N} = 4$ SYM. In section 3, we compute both type-1 and type-2 line operator index for $A_1$-type class-$\mathcal{S}$ theories. In section 4, we further compute line operator index for some other higher rank gauge theories. The appendix contains a quick review of the relevant special functions, as well as a series of new integration formula that help compute Schur index or and without line operators.








