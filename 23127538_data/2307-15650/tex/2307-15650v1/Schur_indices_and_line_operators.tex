% \pdfoutput=1
\documentclass[12pt,letterpaper]{article}
\usepackage{putex}
\usepackage{graphicx}
\usepackage{latexsym,amsmath,amsfonts,amssymb,amsthm}
\usepackage{empheq}
\usepackage{bbm}
\usepackage{cite}
\usepackage{indentfirst}
\usepackage{booktabs,array}
\usepackage{placeins}
\usepackage{mathtools}
\DeclarePairedDelimiter\ceil{\lceil}{\rceil}
\DeclarePairedDelimiter\floor{\lfloor}{\rfloor}
\usepackage{colonequals}
\usepackage{cancel}
\usepackage[mode=image|tex]{standalone}
\usepackage[dvipsnames]{xcolor}
\usepackage{tikz}
\usepackage{tikz-cd}
\usepackage{adjustbox}
%\usepackage{genyoungtabtikz}
\usepackage{enumitem}
\usetikzlibrary{decorations.markings}
\usetikzlibrary{decorations.text}
\usepackage{graphicx}
\usepackage[margin=10pt,font=small,labelfont=bf]{caption}
\usepackage{subcaption}
\usepackage{xcolor}
\usepackage{float}
\theoremstyle{definition}
\newtheorem{claim}{Claim}
\newtheorem{question}{Question}
\usepackage{microtype}
\usepackage{hyperref}
\hypersetup{unicode}
\hypersetup{linktoc = all}
\hypersetup{pdfborderstyle={/S/U/W 0.5}}
\hypersetup{linkbordercolor = gray}
\hypersetup{bookmarksnumbered}
\pdfstringdefDisableCommands{%
  \def\({}%
  \def\){}%
  \def\\{}%
  \def\infty{\042\036}%
  \def\Tr{Tr }%
}


\usepackage[T1]{fontenc}
\usepackage[utf8]{inputenc}
\usepackage{lmodern}

\newcommand{\subtitle}[1]{\textbf{#1}}
%%%%%%%%%%%%%%%%%%%%%%%%%%%%%%%%%%%%%%%%%%



\addtolength{\textheight}{.1truein}
\addtolength{\voffset}{-.1truein}
\setlist{itemsep=2pt plus 1pt minus 1pt, topsep=2pt plus 1pt minus 1pt}

%%%%%%%%%%%%%%%%%%%%%%%%%%%%%%%%%%%%%%%%%%%%%%%%%%%%%%%%%%%%%%%%%%%%
% Code to compress the references

\let\oldthebibliography=\thebibliography
\let\endoldthebibliography=\endthebibliography
\renewenvironment{thebibliography}[1]{%
\begin{oldthebibliography}{#1}%
\setlength\itemsep{-2mm}%
}%
{%
\end{oldthebibliography}%
}
%%%%%%%%%%%%%%%%%%%%%%%%%%%%%%%%%%%%%%%%%%%%%%%%%%%%%
%%
%%MACROs
\newcommand\udl[1]{
	\vspace{10pt}
	\underline{#1}
}


% \newcommand{\vev}[1]{\langle {#1} \rangle}
\newcommand{\trans}{\ensuremath{\mathsf T}}
\newcommand{\dvol}{d\mathrm{vol}}
\newcommand{\ib}{\bar \imath}
\newcommand{\jb}{\bar \jmath}
\newcommand{\parfrac}[2]{\frac{\partial #1}{\partial #2}}
\newcommand{\delfrac}[2]{\frac{\delta #1}{\delta #2}}
\newcommand{\ud}[2]{^{#1}_{\phantom{#1}#2}}
\newcommand{\du}[2]{_{#1}^{\phantom{#1}#2}}
\newcommand{\rep}[1]{\ensuremath{\mathbf{#1}}}
\newcommand{\tf}[2]{\theta\left[\begin{smallmatrix}#1 \\#2 \end{smallmatrix}\right]}

\newcommand{\eg}{\textsl{e.g.\@}}
\newcommand{\Eg}{\textsl{E.g.\@}}
\newcommand{\etal}{\textsl{et al.\@}}
\newcommand{\ie}{\textsl{i.e.\@}}
\newcommand{\ala}{\textsl{\`a la}}
\newcommand{\etc}{\textsl{etc.\@}}



\numberwithin{equation}{section}
\newcommand{\Dslash}{D \!\!\!\!\slash\,}
\newcommand{\nn}{\nonumber}
\newcommand{\mat}[1]{\begin{pmatrix} #1 \end{pmatrix}}
\newcommand{\smat}[1]{\big( \begin{smallmatrix} #1 \end{smallmatrix} \big)}



\newcommand{\tabs}{\rule[-1ex]{0pt}{3.5ex}}
\newcommand{\Tabs}{\rule[-2.2ex]{0pt}{6.4ex}}

\newcommand{\unit}{\mathbbm{1}}

\DeclareMathOperator{\sign}{sign}
\DeclareMathOperator{\Tr}{Tr}
\DeclareMathOperator{\Det}{Det}
\DeclareMathOperator{\rank}{rank}
\DeclareMathOperator*{\Res}{Res}
\DeclareMathOperator{\re}{\mathbb{R}e}
\DeclareMathOperator{\im}{\mathbb{I}m}
\DeclareMathOperator{\coker}{coker}
\DeclareMathOperator{\ind}{ind}
\DeclarePairedDelimiter{\abs}{\lvert}{\rvert}
\DeclarePairedDelimiter{\vev}{\langle}{\rangle}
\DeclareMathOperator*{\SumInt}{%
\mathchoice%
  {\ooalign{$\displaystyle\sum$\cr\hidewidth$\displaystyle\int$\hidewidth\cr}}
  {\ooalign{\raisebox{.14\height}{\scalebox{.7}{$\textstyle\sum$}}\cr\hidewidth$\textstyle\int$\hidewidth\cr}}
  {\ooalign{\raisebox{.2\height}{\scalebox{.6}{$\scriptstyle\sum$}}\cr$\scriptstyle\int$\cr}}
  {\ooalign{\raisebox{.2\height}{\scalebox{.6}{$\scriptstyle\sum$}}\cr$\scriptstyle\int$\cr}}
}


\newcommand\qq{\mathbbmtt{Q}}



\newcommand{\YW}[1]{{\color{Cerulean!80!black}\bf[WP: #1]}}
\newcommand{\YP}[1]{{\color{orange!90!black}\bf[YP: #1]}}

%% end of macros
%%
%%%%%%%%%%%%%%%%%%%%%%%%%%%%%%%%%%%%%%%%%%%%%%%%%%%%%%%%%%%%%%%%

\begin{document}

%%%%%%% title page %%%%%%%%%

\title{\begin{LARGE}
$\mathcal{N} = 2$ Schur index and line operators
\end{LARGE}}


\authors{Zhaoting Guo, Yutong Li, Yiwen Pan, Yufan Wang
\medskip\medskip\medskip\medskip
 }



\institution{UU}{${}^1$
School of Physics, Sun Yat-Sen University, \cr
$\;\,$ Guangzhou, Guangdong, China}

\abstract{\begin{onehalfspace}{
4d $\mathcal{N} = 2$ SCFTs and their invariants can be often enriched by non-local BPS operators. In this paper we study the flavored Schur index of several types of $\mathcal{N} = 2$ SCFTs with and without line operators, using a series of new integration formula of elliptic functions and Eisenstein series. We demonstrate how to evaluate analytically the Schur index for a series of $A_2$ class-$\mathcal{S}$ theories and the $\mathcal{N} = 4$ $SO(7)$ theory. For all $A_1$ class-$\mathcal{S}$ theories we obtain closed-form expressions for $SU(2)$ Wilson line index, and 't Hooft line index in some simple cases. We also observe the relation between the line operator index with the characters of the associated chiral algebras. Wilson line index for some other low rank gauge theories are also studied.
}\end{onehalfspace}}




\preprint{}
\setcounter{page}{0}
\maketitle


%%%%%% end of title page %%%%%%

%%%%%% table of contents %%%%%%
{%\small
\setcounter{tocdepth}{2}
\setlength\parskip{-0.7mm}
\tableofcontents
}
%%%%%%%%%%%%%%%%%%%%%


\section{Introduction}
Current quantum hardware is unable to carry out universal quantum computations due to the buildup of errors that occur during the computation. 
The magnitude of the individual error is currently above the value that the Threshold Theorem requires in order to kick-start quantum error correction and fault-tolerant quantum computation~\cite[Section 10.6]{nielsen_chuang_2010}. 
Although the experimentally achieved fidelity rates are promising and the error bounds are inching closer to the required threshold, we will have to work for the foreseeable future with quantum hardware with errors that build-up during the computation.  This implies that we can only do a limited number of steps before the output of the computation has become completely uncorrelated with the intended one.

For fault-tolerant quantum computing, we repeat four steps: 
1) We apply a number of single and two-qubit quantum gates, in parallel whenever possible; 
2) We perform a syndrome measurement on a subset of the qubits; 
3) We perform fast classical computations to determine which errors have occurred and how to correct them; 
and, 4) We apply correction terms based on the classical computations.
We then repeat these four steps with a next sequence of gates. 
These four steps are essential to fault-tolerant quantum computing. 


The starting point of this work is to use the four steps outlined above, not to carry out error correction and fault-tolerant computation, but to enhance short, constant-depth, {\em uncorrected} quantum circuits that perform single qubit gates and {\em nearest-neighbor} two qubit gates. 
Since in the long run we will have to implement error-correction and fault-tolerant computation anyhow, and this is done by such a four-step process, why not make other use of this architecture? Moreover, on some of the quantum hardware platforms, these operations are already in place.
Embracing this idea we naturally arrive at the question: what is the computational power of \textit{low-depth} quantum-classical circuits organized as in the four steps outlined above? 
We thus investigate circuits that execute a small, ideally constant, number of stages, where at each stage we may apply, in parallel, single qubit gates and {\em nearest-neighbor} two qubit gates, followed by measurements, followed by low-depth classical computations of which the outcome can control quantum gates in later stages. 
It is not clear, at first, whether such circuits, especially with constant depth, can do anything remotely useful. 
But we will see that this is indeed the case: many quantum computations can be done by such circuits in constant depth. 
By parallelizing quantum computations in this way, we improve the overall computational capabilities of these circuits, as we do not incur errors on qubits that are idle, simply because qubits are not idle for a very long time. 
Furthermore, reducing the depth of quantum circuits, at the cost of increasing width, allows the circuit to be run faster even if errors occur.

The first usage of such a four-step layout, not to do error correction, but to perform computations, can be found in the paradigm of measurement-based quantum computing~\cite{gottesman1999demonstrating,raussendorf2001one,jozsa2006introduction,clark2007generalised}: 
A universal form of quantum computing where a quantum state is prepared and operations are performed by measuring qubits in different bases, depending on previous measurements and intermediate measurements.

\citeauthor{PhamSvore2013} were the first to formalize the four-step protocol for performing computations~\cite{PhamSvore2013}. They included specific hardware topologies by considering two-dimensional graphs for imposing constraints on qubit interactions. In their model, they develop circuits for particularly useful multi-qubit gates, including specifying costs in the width, number of qubits, depth, number of concurrent time steps, size, and total number of non-Identity operations.
As a result, they find an algorithm that factors integers in polylogarithmic depth.
\citeauthor{Browne:2011} showed that the main tool in the work by \citeauthor{PhamSvore2013}, the fan-out gate, can also be replaced by additional log-depth classical computations in the measurement-based quantum computing setting~\cite{Browne:2011}.

More recently, \citeauthor{Cirac:2021} introduced a scheme to implement unitary operations involving quantum circuits combined with Local Operations and Classical Communication ($\mathsf{LOCC}$) channels: $\mathsf{LOCC}$-assisted quantum circuits~\cite{Cirac:2021}. Similarly to the four-step scheme we just described, they allow for a short depth circuit to be run on the qubits, followed by one round of $\mathsf{LOCC}$, in which ancilla qubits are measured and local unitaries are applied based on the measurement outcomes. They show that in this model any 1D transitionally invariant matrix-product state (MPS) with fixed bond dimension is in the same phase of matter as the trivial state. Similar ideas can be found in~\cite{TVV_NonAbelianTopologicalOrder_2022, tantivasadakarn2021long}.

In this work, we introduce a new model, called \textit{Local Alternating Quantum-Classical Computations} ($\LAQCC$). In this model we alternate between running quantum circuits (constrained by locality), ending in the measurement of a subset of qubits, and fast classical computations based on the measurement results. The outcome of the classical computations are then used to control future quantum circuits. We allow for flexibility in this model, by giving different constraints to the power of both the quantum circuits and the classical circuits as well as the number of alternations between them. 
Most attention will be given to $\LAQCC$ containing quantum circuits of constant depth, classical circuits of logarithmic depth and at most a constant number of alternations between them. 
Any circuit constructed in this model is considered to be of constant depth. 
We restrict ourselves to logarithmic depth classical computations, as this is the first natural and non-trivial extension beyond constant-depth classical computations. 
Constant-depth classical computations do however also have an equivalent constant-depth quantum implementation.

The definition of $\LAQCC$ sharpens the original definition of \citeauthor{PhamSvore2013} by adding constraints to the intermediate classical computations. This allows us to bound the power of $\LAQCC$ from above. 

The main result of \citeauthor{Cirac:2021}, that 1D translational invariant MPS with fixed bond dimension can be prepared by $\mathsf{LOCC}$-assisted circuits, relies on local symmetries of the MPS. These symmetries allow them to prepare local states (on a constant number of qubits) and glue them together by doing one round of the appropriate entangling measurement and corrections, after which they run a round of local unitaries to get the desired result. This general scheme for preparing states that exhibit an MPS description with the appropriate local symmetries requires only geometrically local unitaries and one round of measurement and corrections an therefore is accessible in $\LAQCC$. Studying different local symmetries, known as Symmetry Protected Topological (SPT) phases of matter, to find measurement-based constant depth circuits for states is a broad ongoing field of research~\cite{TVV_NonAbelianTopologicalOrder_2022, tantivasadakarn2021long, smith2023deterministic}. 
All these schemes have a $\LAQCC$ implementation.

%$\LAQCC$-circuits also exist for general schemes of preparing local states, based on the local tensors, and gluing them together using one round of entangled measurement and corrections, based on the local symmetry. 
%The main result of \citeauthor{Cirac:2021}, that 1D translational invariant MPS with fixed bond dimension can be prepared by $\mathsf{LOCC}$-assisted circuits, relies heavily on local symmetries of the MPS and as a result also has an equivalent $\LAQCC$ implementation. 
%The corrections applied after the measurement round are local unitaries depending on the local symmetries of the MPS. 

 

%This general scheme of preparing local states, based on the local tensors, and gluing it together by doing one round of entangled measurement and corrections, based on the local symmetry, is accessible in $\LAQCC$.
Note however that \citeauthor{Cirac:2021} also suggest a circuit for the $W$-state.
This circuit uses sequentially and dependent measurement-based corrections of the ancilla qubits. 
These dependent measurements translate to sequential alternations between the quantum and classical circuits and therefore increase the total depth to linear depth, exceeding the constant-depth constraints imposed by $\LAQCC$-circuits. 

We study the power of the $\LAQCC$ model with respect to state preparation, showing that even with only constant quantum-depth and logarithmic classical depth it remains possible to prepare states with long-range entanglement.
Another surprising result is that it is unlikely that $\LAQCC$ circuits are classically simulatable. We show that any instantaneous quantum polynomial-time (IQP) circuit~\cite{Bremner2010,Shepherd2009} has an $\LAQCC$ implementation.
Classical simulation of IQP circuits implies the collapse of the polynomial hierarchy to the third level, which is not believed to be true~\cite{Bremner2017}. Therefore, we expect that $\LAQCC$ circuits are unlikely to be classically simulatable. We bound the power of $\LAQCC$ by showing that it is contained in $\QNC^1$, the class of polynomial-size, log-depth circuits.

Next, we also study the power that intermediate classical calculations can add to quantum computations, by considering a new model that alternates between polynomially many polynomial-depth quantum circuits and unbounded classical computations
We study this model by doing a complexity theoretical analysis, where we draw inspiration from the notions of complexity given by \citeauthor{RosenthalYuen:2022}, \citeauthor{MetgerYuen:2023}, and \citeauthor{Aaronson:2004}.
All three complexity notions are based on the notion of state preparation, instead of more traditional definition of complexity such as the decidability of a computational problem. 
The first two consider classes based on sequences of quantum states preparable by a polynomial-sized quantum circuit, where the circuits are uniformly generated by a computational class, for instance, the class $\mathsf{PSPACE}$, which results in the complexity class $\mathsf{StatePSPACE}$~\cite{RosenthalYuen:2022,MetgerYuen:2023}.
The third notion considers a relative complexity, where the complexity is measured between two given states, and is measured by the number of gates, from a given gate-set, required to transform one state in another state~\cite{Aaronson:2004}. 
For our definition of state preparation complexity, we drop the uniformity constraint from~\cite{RosenthalYuen:2022,MetgerYuen:2023} and define a class as $\mathsf{StateX}$, which refers to states preparable by circuits of type $\mathsf{X}$. 
As an example, if $\mathsf{X} = \QNC^0$, this results in the class $\mathsf{StateQNC^0}$, which is the set of states preparable from the $\ket{0}^n$ state by poly-size constant-depth circuits. 
This notion is similar to the relative complexity from~\cite{Aaronson:2004}, where one state is the  $\ket{0}^n$ state and instead of counting the number of gates we consider the set of states preparable by a fixed number of gates. Using this notion of complexity we show that any state preparable by an $\LAQCC^*$ circuit is also preparable by a $\mathsf{PostQPoly}$ circuit, the class of circuits of polynomial depth with an additional post-selection gate. 

All Clifford circuits have a constant-depth $\LAQCC$ implementation, implying that any stabilizer state can be implemented by a constant-depth $\LAQCC$ circuit, see Section~\ref{sec:clifford_circuits} for a proof of this statement. 
Efficient circuits for stabilizer states have been known already through measurement-based quantum computing. Therefore this paper focuses on the preparation of non-stabilizer states, and as a surprising result we find novel constant-depth protocols for four very natural classes of non-stabilizer states.
Despite the extensive research into these four classes of non-stabilizer states and the many applications of them, no efficient constant- or low-depth state preparation protocols are known yet. We specifically consider these four classes as they are all often used as initial states in other algorithms.

The first state is a uniform superposition over an arbitrary number of states. 
This state finds applications in many quantum algorithms, as they often start with a uniform superposition over multiple states. 
This superposition is often achieved by applying Hadamard gates to every qubit due to its simplicity to prepare. 
Yet, the analysis of many algorithms, such as Shor's algorithm~\cite{Shor:1997}, would benefit from a different initial superposition. 
The circuit to prepare the uniform superposition over an arbitrary number of states uses an exact version of Grover search as a subroutine, that turns a probabilistic circuit, with a known constant probability of success, into a deterministic circuit. 
We use the circuit for preparing a uniform superposition over an arbitrary number of states as a subroutine in the next two quantum state preparation protocols. 

The second state is the $W$-state, the uniform superposition over all computational basis states of Hamming-weight~$1$, a natural long-ranged entangled state that displays a fundamentally nonequivalent type of entanglement from the Greenberger–Horne–Zeilinger state~\cite{WState:2000}, for which $\LAQCC$-type constant-depth circuits were previously known~\cite{PhamSvore2013, Cirac:2021}. 
The $W$-state is often used as benchmark for new quantum hardware~\cite{Haffner2005,Neeley2010,GarciaPerez:2021}. 
A novel way to prepare the $W$-state therefore gives a new way to benchmark different quantum devices with each other. 
A circuit for preparing the $W$-state was given in~\cite{Cirac:2021}, but this implementation requires sequentially alternating measurements followed by local unitaries, which in the $\LAQCC$ model is not considered to be of constant depth. 
We improve this protocol by giving an $\LAQCC$ implementation of the $W$-state, based on a compress-uncompress method that links the one-hot and binary encoding of integers.

The third state considered is the Dicke state, a generalization of the $W$-state, a superposition over all computational basis states with Hamming-weight $k$~\cite{Dicke:1954}. 
Dicke states have relevance in various practical settings.
For instance, for quantum game theory~\cite{zdemir2007}, quantum storage~\cite{Bacon_Compress:2006,Plesch:2010}, quantum error correction~\cite{ouyang2014permutation}, quantum metrology~\cite{toth2012multipartite}, and quantum networking~\cite{prevedel2009experimental}. 
Dicke states have been used as a starting state for variational optimization algorithms, most notably Quantum Alternating Operator Ansatz (QAOA)~\cite{Hadfield2019}, to find solutions to problems such as Maximum k-vertex Cover~\cite{Brandhofer2022,cook2020quantum}.
The ground states of physical Hamiltonians describing one-dimensional chains tend to show a resemblance to Dicke states such as states resulting from the Bethe ansatz, making them an ideal starting state when investigating the ground state behavior of these Hamiltonians~\cite{TDL_BetheAnsatzDerivation:2010,B_ExcitedStateQuantumPhaseTransitions:2013,DickeTransitions:2021}. 
For instance, the algorithm by \citeauthor{van2021preparing}, who give an algorithm to prepare the Bethe ansatz eigenstates of the spin-1/2 XXZ spin chain, starts by first preparing a Dicke state~\cite{van2021preparing}. 
A Dicke-state preparation protocol based on the compress-uncompress methodology used in the $W$-state furthermore finds applications in entanglement distillation, where the entanglement of a large state is concentrated on only a few qubits. 
Efficient deterministic circuits for preparing Dicke states have been proposed by \citeauthor{bartschi2019deterministic}~\cite{bartschi2019deterministic, bartschi2022deterministic_short_depth}. 
They provide a quantum circuit of depth $\mathO(k \log(\frac{n}{k}))$, allowing arbitrary connectivity, to prepare a Dicke state, which they conjecture to be optimal when $k$ is constant. 
In this work, we provide a constant-depth $\LAQCC$ circuit below their conjectured bound already for constant $k$. 
However, this does not directly disprove their conjecture, as we allow for intermediate measurements and classical computations. 
More significantly, we even construct constant-depth $\LAQCC$ circuits for $k = \mathO(\sqrt{n})$ greatly improving their bound.
This construction extends the compress-uncompress method for the $W$-state combined with additional subroutines. 

We continue with a log-depth state preparation protocol for the Dicke-state for arbitrary $k$. 
This protocol implements an efficient transformation between the factoradic number representation and the combinatorial number representation of a positive integer. 
The combinatorial number representation relates directly to the Dicke state. 
The provided efficient transformation between number representation systems might be of independent interest. 

We conclude by modifying our protocol for preparing a Dicke-state to a protocol that prepares quantum many-body scar states in constant-depth. 
These states have low entanglement and longer coherence times than states with similar energy density.
These characteristics make many-body scar states interesting to analyze and relevant within physics.
Many-body scar states appear for instance in the AKLT model~\cite{AKLT:1987,MRBAR:2018,MRB:2018} and different spin models~\cite{SI:2019,MOBFR:2020}.
Known methods for preparing these states have polynomial-depth~\cite{Gustafson:2023}, whereas our circuit has constant depth. 

% We conclude by studying the power that intermediate classical calculations can add to quantum computations. 
% In this study, we define a new model that relaxes constant-depth quantum circuits to polynomial depth quantum circuits, log-depth classical calculations to unbounded classical computations and a constant number of alternations to a polynomial number of alternations. 
% We call this model $\LAQCC^*$. 
% We study this model by doing a complexity theoretical analysis, where we draw inspiration from the notions of complexity given by \citeauthor{RosenthalYuen:2022}, \citeauthor{MetgerYuen:2023}, and \citeauthor{Aaronson:2004}.
% All three complexity notions are based on the notion of state preparation, instead of more traditional definition of complexity such as the decidability of a computational problem. 
% The first two consider classes based on sequences of quantum states preparable by a polynomial-sized quantum circuit, where the circuits are uniformly generated by a computational class, for instance, the class $\mathsf{PSPACE}$, which results in the complexity class $\mathsf{StatePSPACE}$~\cite{RosenthalYuen:2022,MetgerYuen:2023}.
% The third notion considers a relative complexity, where the complexity is measured between two given states, and is measured by the number of gates, from a given gate-set, required to transform one state in another state~\cite{Aaronson:2004}. 
% For our definition of state preparation complexity, we drop the uniformity constraint from~\cite{RosenthalYuen:2022,MetgerYuen:2023} and define a class as $\mathsf{StateX}$, which refers to states preparable by circuits of type $\mathsf{X}$. 
% As an example, if $\mathsf{X} = \QNC^0$, this results in the class $\mathsf{StateQNC^0}$, which is the set of states preparable from the $\ket{0}^n$ state by poly-size constant-depth circuits. 
% This notion is similar to the relative complexity from~\cite{Aaronson:2004}, where one state is the  $\ket{0}^n$ state and instead of counting the number of gates we consider the set of states preparable by a fixed number of gates. Using this notion of complexity we show that any state preparable by an $\LAQCC^*$ circuit is also preparable by a $\mathsf{PostQPoly}$ circuit, the class of circuits of polynomial depth with an additional post-selection gate. 

\paragraph{Summary of results}
\begin{itemize}
    \item We give a new definition of a computational model that captures the power of the four step process: applying a constant number of layers of one- and two-qubit gates; performing a syndrome measurement; perform a fast classical computation determining corrections; apply corrections. We call this model \emph{Local Alternating Quantum Classical Computations}, or $\LAQCC$ for short. In this model we bound the allowed quantum operations, intermediate classical calculations, and number of rounds separately. In Section~\ref{sec:LAQCC_model} we define this model and give a list of operations based on results from literature contained in this computational model. In some of these operations we explicitly use that we allow for multiple, but at most constant, rounds  of corrections.
    \item  We show show that there exist $\LAQCC$ circuits that can not be weakly simulated in Section~\ref{sec:IQP_in_LAQCC}. We further show that for every $\LAQCC$ circuit there exists a $\QNC^1$ circuit simulating it perfectly, in Section~\ref{sec:LAQCC_in_QNC1}.
    \item We introduce a new type computational complexity for preparing states and show that the extension of $\LAQCC$ where we allow a polynomial number of rounds and unbounded classical computation, is contained in $\mathsf{PostQPoly}$, the class of polynomial circuits with post-selection, in Section~\ref{sec:Complexity results}.
    \item We show a protocol to prepare the uniform superposition state of size $q$ in $\LAQCC$ using $\mathO(\ceil{\log_2(q)}^2)$ qubits in Section~\ref{sec:superposition_modulo_q}. 
    \item We show a protocol to prepare the $W_n$ state in $\LAQCC$ using $\mathO(n\log(n))$ qubits in Section~\ref{sec:W_state_in_LAQCC}.
    \item We show two ways of preparing the Dicke-$(n,k)$ state. The first method is in $\LAQCC$, works up to $k = \mathO(\sqrt{n})$, uses $\mathO(n^2\log(n))$ qubits, and is found in Section~\ref{sec:dicke:small_k}. The second method is in $\LAQCC\text{-}\mathsf{LOG}$ (an extension of $\LAQCC$ allowing for logarithmic number of alterations instead of constant), works for any $k$, uses $\mathO(\text{poly}(n))$ qubits, and is found in Section~\ref{sec:Dicke_in_LAQCC_LOG}. 
    \item We extend on our $\LAQCC$ method of generating Dicke-$(n,k)$ states for $k = \mathO(\sqrt{n})$ and show a protocol to generate many-body scar states for a particular Hamiltonian in $\LAQCC$ (Section~\ref{sec:many_body_scar}). 
\end{itemize}
Summarized in a table, we provide the following state generation protocols:
\begin{table}[htb]
\centering
\begin{tabular}{l|l|l|l}
\textbf{State description} & \textbf{Width} & \textbf{Depth} & \textbf{Implementation}\\
\hline 
Uniform superposition mod $q$: $\frac{1}{\sqrt{q}} \sum_{i = 0}^{q-1}\ket{i}$ & $\mathO(\ceil{\log^2 q})$ & $\mathO(1)$ & Section~\ref{sec:superposition_modulo_q}\\

$W$-state: $\frac{1}{\sqrt{n}}\sum_{i = 0}^{n-1}\ket{e_i}$ & $\mathO(n \log n)$ & $\mathO(1)$ & Section~\ref{sec:W_state_in_LAQCC}\\

Dicke-$(n,k)$, $k = \mathO(\sqrt{n})$: $\binom{n}{k}^{-1/2}\sum_{x \in \{0,1\}^n: |x| = k} \ket{x}$ &  $\mathO(n^2\log n)$ & $\mathO(1)$ 
&Section~\ref{sec:dicke:small_k}\\

Dicke-$(n,k)$: $\binom{n}{k}^{-1/2}\sum_{x \in \{0,1\}^n: |x| = k} \ket{x}$ & $\mathO(\text{poly}(n))$ & $\mathO(\log n)$ &Section~\ref{sec:Dicke_in_LAQCC_LOG}\\

QMBS: $\ket{S_k} = \frac{1}{k! \sqrt{\mathcal N(n,k)}}(Q^\dagger)^k \ket{\Omega}$ &  $\mathO(n^2\log n)$ & $\mathO(1)$  &  Section~\ref{sec:many_body_scar}
\end{tabular}
\caption{Summary of state preparation protocols given in this paper.}
\label{tab:sate_prep}
\end{table}
In the entry for the quantum many-body scar state $Q$ denotes the raising operator and $\mathcal N(n,k)=\binom{n-k-1}{k}$. 
Section~\ref{sec:many_body_scar} will provide more details on the variables and the implementation. 

\paragraph{Organization of the paper}
\noindent We first introduce relevant preliminaries in Section~\ref{sec:preliminaries}. 
In Section~\ref{sec:LAQCC_model} we formally define the class of Local Alternating Quantum-Classical Computations ($\LAQCC$). We also show that any Clifford circuit can be implemented in constant depth $\LAQCC$ (a result based on a result from measurement-based quantum computing~\cite{jozsa2006introduction}). 
This result allows us to give many useful multi-qubit gates and routines in Section~\ref{sec:gates_created_in_LAQCC}. 
Beyond that we show that constant depth $\LAQCC$ circuits are contained in $\QNC^1$ and that any $\mathsf{IQP}$ circuit has an $\LAQCC$ implementation.
We conclude this section with an analysis of a more powerful instantiation of $\LAQCC$ and show an inclusion with respect to the class $\mathsf{PostQPoly}$, which is the class of circuits of polynomial depth with one additional post-selection gate. 
In Section~\ref{sec:state_prep_in_LAQCC} we give $\LAQCC$ circuit implementations for preparing the uniform superposition over an arbitrary number of states, the $W$-state and the Dicke state up to $k = \mathO(\sqrt{n})$. We furthermore give a log-depth circuit implementation for preparing the Dicke state for any $k$. We conclude by showing a $\LAQCC$ circuit for generating many body scar states of a particular type of Hamiltonian.


%!TEX root = ../Schur indices and line operators.tex

\section{More on Schur index}

Several integration formula were proposed in \cite{Pan:2021mrw}, which can be used to analytically compute some multivariate contour integral of elliptic functions. Those formula were enough to compute exactly the Schur index of $A_1$ class-$\mathcal{S}$ theories and some low rank $\mathcal{N} = 4$ theories. However, they were insufficient for more general $A_N$ class-$\mathcal{S}$ theories. In this section, with the help from some new integration formula, we explore the exact computation the Schur index of a series of $A_2$ theories and the $\mathcal{N} = 4$ $SO(7)$ theory, generalizing the results in \cite{Pan:2021mrw}. The computation in this section is relatively technical, and uninterested readers may skip to section \ref{section:Wilson-index-A1-theories} for the computation of line operator index.


\subsection{\texorpdfstring{$A_2$ theories of class-$\mathcal{S}$}{}}



First we recall the Schur index of the $SU(3)$ SQCD. It can be computed as a contour integral
\begin{equation}
	\mathcal{I}_{\text{SQCD}} = - \frac{1}{3!} \eta(\tau)^{16} \oint \prod_{A = 1}^2 \frac{da_A}{2\pi i a_A}
	\frac{\prod_{A \ne B} \vartheta_1(\mathfrak{a}_A - \mathfrak{a}_B)}{\prod_{A = 1}^3 \prod_{i = 1}^{6} \vartheta_4(\mathfrak{a}_A - \mathfrak{m}_i)}
	\equiv \oint \prod_{A = 1}^2 \frac{da_A}{2\pi i a_A}\mathcal{Z}(\mathfrak{a})\ ,
\end{equation}
where $\mathfrak{a}_3 = -\mathfrak{a}_1 - \mathfrak{a}_2$, $a_3 = (a_1 a_2)^{-1}$, and $a_i = e^{2\pi i \mathfrak{a}_i}$, $m_i = e^{2\pi i \mathfrak{m}_i}$. See also Appendix \ref{app:special-functions} for the definitions and properties of the Eisenstein series $E_k\big[\substack{\phi\\\theta}\big]$ and the Jocobi theta functions. The integral can be performed by applying the integration formula, which yields the exact (albeit slightly complicated) result,
\begin{align}\label{index-SQCD}
	\mathcal{I}_{\text{SQCD}} = & \ \sum_{j_2 = 1}^{6}2R_{0j_2}E_1\left[\begin{matrix}
		-1 \\ m_{j_2}
	\end{matrix}\right] \nonumber \\
	& \ + \sum_{j_1}^{6}
	  \left(
	  R_{j_1}(\mathfrak{a}_2 = 0) + \sum_{j_2 = 1}^6
	  \Big(  
	  R_{j_1 j_2}E_1\left[\begin{matrix}
	  	-1 \\ m_{j_2}
	  \end{matrix}\right]
	  + R_{j_1 j_2} E_1\left[\begin{matrix}
	  	-1 \\ m_{j_1}m_{j_2}q^{ - \frac{1}{2}}
	  \end{matrix}\right]  \Big)
	\right)E_1\left[\begin{matrix}
		-1 \\ m_{j_1}
	\end{matrix}\right]\nonumber\\
	& \ + \sum_{j_1, j_2 = 1}^{6} R_{j_1 j_2}\left(
			E_2\left[\begin{matrix}
				1 \\ m_{j_1} m_{j_2}
			\end{matrix}\right]
			- E_2\left[
			\begin{matrix}
				1 \\ m_{j_2}q^{ - \frac{1}{2}}
			\end{matrix}\right]
		\right)  \ .
\end{align}
From the computation of $SU(3)$ SQCD index, we already see that the complexity is far above the $A_1$ type. Therefore, we shall focus on arguing that the index can be computed using the existing integration formula. The complexity could decrease once more optimized integration formula is found, which we leave to future work.


As a class-$\mathcal{S}$ theory, the SQCD has manifest flavor symmetry $SU(3)^{(1)} \times SU(3)^{(2)} \times U(1)^{(1)} \times U(1)^{(2)}$. We shall denote the fugacities of $SU(3)^{(\alpha)}$ as $c^{(\alpha)}$, and those of $U(1)^{(\alpha)}$ as $d^{(\alpha)}$. They are related to $m_j$ by
\begin{align}
  c^{(1)}_1 = & \ m_1/d^{(1)}, \qquad c^{(1)}_2 = m_2 / d^{(1)}, \qquad
  d^{(1)} = (m_1 m_2 m_3)^{1/3} \ ,\\
  c^{(2)}_1 = & \ m_4/d^{(2)}, \qquad c^{(2)}_2 = m_5/d^{(2)}, \qquad
  d^{(2)} =  (m_4 m_5 m_6)^{1/3} \ .
\end{align}

Starting from $SU(3)$ SQCD, one can build $SU(3)$ linear quiver theories by successively gauging in $9$ hypermultiplets one after another. Let us perform one such computation. The gauging procedure multiplies to $\mathcal{I}_\text{SQCD}$ factors
\begin{align}
	\mathcal{I}_\text{VM} \sim \prod_{\substack{A,B = 1 \\ A\ne B}}^{3} \vartheta_1 (\mathfrak{a}_A - \mathfrak{a}_B) \ ,\qquad
	\mathcal{I}_\text{HM} = \prod_{A, B = 1}^{3}\frac{\eta(\tau)}{\vartheta_4(- \mathfrak{a}_A + \mathfrak{c}^{(3)}_B + \mathfrak{d}^{(3)})} \ ,
\end{align}
where again $\mathfrak{a}_3 = - \mathfrak{a}_1 - \mathfrak{a}_2$, $\mathfrak{c}^{(3)}_3 = - \mathfrak{c}^{(3)}_ 1 - \mathfrak{c}^{(3)}_2$. The gauging also identifies $\mathfrak{c}^{(2)}_A$ with $\mathfrak{a}_A$, and a contour integral of $a_1, a_2$ should be performed, 
\begin{align}
	\mathcal{I} = \oint \frac{da_1}{2\pi i a_1}
	\frac{da_2}{2\pi i a_2} \mathcal{I}_\text{SQCD}(c^{(1)}, a, d^{(1)}, d^{(2)}) \mathcal{I}_\text{VM}(a) \mathcal{I}_\text{HM}(a, c^{(3)}, d^{(3)}) \ .
\end{align}

Let us look at the various terms in this integral. First of all, we have an integral of
\begin{align}
	\sum_{j_2 = 1}^6
	R_{0 j_2} \mathcal{I}_\text{VM} \mathcal{I}_\text{HM} E_1 \begin{bmatrix}
		-1 \\ m_{j_2}
	\end{bmatrix} \ .
\end{align}
It is straightforward to verify that, as a function of $\mathfrak{a}_{1,2}$, the factor $R_{0j_2}\mathcal{I}_\text{VM}\mathcal{I}_\text{HM}$ is elliptic with respect to both $\mathfrak{a}_{1,2}$. Moreover, after the replacing $m$ with the $c, d$ fugacities and $a$,
\begin{align}
	(m_1, \ldots, m_6) = (
	c^{(1)}_1 d^{(1)},
	c^{(1)}_2 d^{(1)},
	\frac{d^{(1)}}{c^{(1)}_1 c^{(1)}_2},
	a_1d^{(2)},
	a_2 d^{(2)},
	\frac{d^{(2)}}{a_1 a_2}
	) \ ,
\end{align}
and similarly $m_{j_1} m_{j_2 \ne j_1} \sim ((\ldots),  a_1^{\pm 1} (\ldots), a_2^{\pm 1}(\ldots), (a_1 a_2)^{\pm 1} (\ldots) )$ where $(\ldots)$ denotes combinations of $c^{(1)}, d^{(1)}, d^{(2)}$. Therefore, one can perform the $a_1$ integral using (\ref{integration-formula-fE-1}) or (\ref{integration-formula-fE-2}). For all $j_2$, there are several types of poles from $R_{0j_2}\mathcal{I}_\text{VM}\mathcal{I}_\text{HM}$,
\begin{align}
	\mathfrak{a}_1 = & \ [1,2], \quad  [3],
	\quad - \mathfrak{a}_2 + [1,2], \quad
	 - \mathfrak{a}_2 + [3] \ .
\end{align}
Here $[1,2]$ and $[3]$ denote respectively linear combinations of $\mathfrak{c}^{(1)}, d^{(1,2)}$ and $\mathfrak{c}^{(3)}, \mathfrak{d}^{(3)}$. The Eisenstein series $E_1 \big[\substack{-1\\m_{j_2 = 1,2,3}}\big]$ are independent of $a_1, a_2$, and will never participate in subsequent integrations or gauging. The variables $a_1, a_2$ in $E_1 \big[ \substack{-1\\ m_{j_2 = 4,5,6}} \big]$ appear in the form $a_1$, or $a_2$, or a product $a_1 a_2$. The $a_1$ integral using the integration formula will produce $E_1 \big[ \substack{\pm 1\\ [1,2]} \big], E_1 \big[ \substack{\pm 1\\ [3]} \big]$, $E_2\big[ \substack{\pm1\\ a_2[1,2]} \big]$ or $E_2\big[ \substack{\pm1\\ a_2[3]} \big]$, where $[1,2]$ and $[3]$ denote respectively combinations of the flavor fugacities $c^{(1)}, d^{(1)}, d^{(2)}$, and of $c^{(3)}, d^{(3)}$. The $a_2$-integration of these terms can be further carried out, and we have Eisenstein structure, 
\begin{equation}
	E_{1} \begin{bmatrix}
    \pm 1\\ [1,2]
\end{bmatrix}E_{1} \begin{bmatrix}
    \pm 1\\ [1,2]
\end{bmatrix}, \quad
	E_{1} \begin{bmatrix}
    \pm 1\\ [1,2]
\end{bmatrix}E_{1} \begin{bmatrix}
    \pm 1\\ [3]
\end{bmatrix},  \quad
	E_{1,2} \begin{bmatrix}
    \pm 1\\ [1,2,3]
\end{bmatrix} \ ,
\end{equation}
where $[1,2,3]$ denotes products of $c^{(1)}, c^{(3)}, d^{(1,2,3)}$.

Next we have $R_{j_1}(\mathfrak{a}_2 = 0)\mathcal{I}_\text{VM}\mathcal{I}_\text{HM}E_1 \big[\substack{-1 \\ m_{j_1}}\big]$ integral. Again, the prefactor $R_{j_1}(\mathfrak{a}_2 = 0)\mathcal{I}_\text{VM}\mathcal{I}_\text{HM}$ is separately elliptic with respect to both $\mathfrak{a}_{1,2}$. This factor again has $\mathfrak{a}_1$-poles of the form
\begin{align}
	\mathfrak{a}_1 = & \ [1,2], \quad  [3],
	\quad - \mathfrak{a}_2 + [1,2], \quad
	 - \mathfrak{a}_2 + [3] \ .
\end{align}
Therefore, the $a_1$, $a_2$ can also be straightforwardly performed with a reference point $\mathfrak{a}_1 = 0$. The Eisenstein structure is the same as the that of the previous term.

Let us also look at the last two terms in (\ref{index-SQCD}),
\begin{align}
	R_{j_1 j_2}\left(
	E_2 \begin{bmatrix}
		1 \\ m_{j_1}m_{j_2}
	\end{bmatrix}
	- E_2 \begin{bmatrix}
		1 \\ m_{j_2}
	\end{bmatrix}
	\right) \ .
\end{align}
We note that $R_{j_1 j_2} = 0$ when $j_1 = j_2$. One can also directly verify that $R_{j_1j_2} \mathcal{I}_\text{VM} \mathcal{I}_\text{HM}$ is elliptic, with $\mathfrak{a}_1$ poles of the same simple form as the above. Hence, one can also proceed with both $a_1, a_2$ integral using (\ref{integration-formula-fE-1}), (\ref{integration-formula-fE-2}). The Eisenstein structure of the result involves
\begin{align}
	E_1 \begin{bmatrix}
  	\pm 1 \\ [1,2]  
	\end{bmatrix}
	E_2 \begin{bmatrix}
  	\pm 1 \\ [1,2]  
	\end{bmatrix}
	E_1 \begin{bmatrix}
  	\pm 1 \\ [1,2] \text{ or } [3]
	\end{bmatrix} \ ,\quad
	E_1 \begin{bmatrix}
  	\pm 1 \\ [3]  
	\end{bmatrix}
	E_2 \begin{bmatrix}
  	\pm 1 \\ [1,2]  
	\end{bmatrix}
	E_1 \begin{bmatrix}
  	\pm 1 \\ [1,2] \text{ or } [3]
	\end{bmatrix} \ ,
\end{align}

We are now ready to deal with the middle two terms in (\ref{index-SQCD}). Again, the factor in front of the Eisenstein series is suitably elliptic. But now this elliptic function is multiplying with
\begin{align}
	E_1 \begin{bmatrix}
		-1 \\ m_{j_2}
	\end{bmatrix}E_1 \begin{bmatrix}
  	-1 \\ m_{j_1}  
	\end{bmatrix}, \qquad
	E_1 \begin{bmatrix}
		-1 \\ m_{j_1} m_{j_2}q^{-1/2}
	\end{bmatrix}E_1 \begin{bmatrix}
  	-1 \\ m_{j_1}  
	\end{bmatrix} \ .
\end{align}
When substituting in the $a, c, d$ fugacities we will need to integrate
\begin{align}
	f(a_1, a_2)E_1 \begin{bmatrix}
		-1 \\ a_A[1,2]
	\end{bmatrix}
	E_1 \begin{bmatrix}
		-1 \\ a_B [1,2]
	\end{bmatrix}, \quad
	f(a_1, a_2)
	E_1 \begin{bmatrix}
		-1 \\ a_A [1,2]
	\end{bmatrix}
	E_1 \begin{bmatrix}
		-1 \\ a_1 a_2 [1,2]
	\end{bmatrix} \ .
\end{align}
We can carry out the $a_1$ integral which involves poles of the same form as the above, $\mathfrak{a}_1 = \text{expressions of } \mathfrak{c}, \mathfrak{d}$ and $\mathfrak{a}_1 = - \mathfrak{a}_2 + \text{expressions of } \mathfrak{c}, \mathfrak{d}$. The integral can be performed using the integration formula \ref{integration-formula-fE-1}. After the $a_1$ integral, we will have the following type of integrand left to integrate (factors independent of $a_2$ are omitted),
\begin{align}
	& \ f(a_2)E_{k = 1,2} \begin{bmatrix}
			\pm 1 \\ a_2(\ldots)
		\end{bmatrix}, \qquad \text{ or },\qquad
	f(a_2)E_{k = 1,2} \begin{bmatrix}
			\pm 1 \\ a_2(\ldots)
		\end{bmatrix}
		E_1 \begin{bmatrix}
				\pm 1 \\ (\ldots)
			\end{bmatrix}\ .
\end{align}
To illustrate this, we can look at a term in the sum, for example,
\begin{align}
	f(a_1, a_2) E_1 \begin{bmatrix}
		-1 \\ a_1 a_2 (\ldots)
	\end{bmatrix}
	E_1 \begin{bmatrix}
		-1 \\ a_1 a_2 (\ldots)'
	\end{bmatrix} \ .
\end{align}
Since $f(a_1, a_2)$ has poles only of the form $a_1 = (\ldots)$ and $a_1 = a_2^{-1} (\ldots)$, the integral of the above will produce Eisenstein series with arguments
\begin{align}
	\frac{a_2(\ldots)}{a_2(\ldots)'}, \quad e^{2\pi i 0} a_2^{-1}(\ldots), \quad e^{2\pi i 0}(\ldots), \quad a_2(\ldots) a_2^{-1}(\ldots), \quad
	a_2(\ldots) (\ldots) \ , \quad \text{etc.}
\end{align}
Here we have chosen the reference point as $\mathfrak{a}_1 = 0$. Therefore, although tedious, the leftover $a_2$ integral can be dealt with, and it produces the exact Schur index for the $SU(3) \times SU(3)$ linear quiver theory. In the end, the exact index contains Eisenstein structures
\begin{align}
	E_3 \begin{bmatrix}
  	\pm 1\\ (\ldots)  
	\end{bmatrix}
	, \quad
	E_1 \begin{bmatrix}
  	\pm 1\\ (\ldots)  
	\end{bmatrix}E_2 \begin{bmatrix}
  	\pm 1\\ (\ldots)  
	\end{bmatrix} \ .
\end{align}

The above analysis can be repeated for longer linear $SU(3)$ quiver theories, where we will encounter integrals in the presence of
\begin{align}
	E_n \begin{bmatrix}
  	\pm 1 \\ z(\ldots)
	\end{bmatrix}, \qquad
	E_1 \begin{bmatrix}
  	\pm 1 \\ z(\ldots)  
	\end{bmatrix} E_n \begin{bmatrix}
  	\pm 1 \\ z(\ldots)  
	\end{bmatrix} \ .
\end{align}
These integrals can be treated using the integration formula in the appendix, and therefore Schur index of all linear $SU(3)$-quiver are computable, though rather tedious, with the current method.



Now that gauging a $SU(3)$ symmetry with fugacities $c^{(2)}$ can be carried out using the integration formula, we are able to also compute Schur index of some non-Lagrangian theories. Consider the $E_6$ superconformal field theory of Minahan and Nemeschansky \cite{Minahan:1996fg}, whose index can be computed by exploiting the Argyres-Seiberg duality \cite{Argyres:2007cn} and an inversion formula \cite{Gadde:2010te,Razamat:2012uv},
\begin{align}
  & \ \mathcal{I}_{E_6}(\mathbf{c}^{(1)}, \mathbf{c}^{(2)}, (wr, w^{-1}r, r^{-2})) \nonumber\\
  = & \ \frac{\mathcal{I}_{\text{SQCD}}(\mathbf{c}^{(1)}, \mathbf{c}^{(2)}, \frac{w^{\frac{1}{3}}}{r}, \frac{w^{- \frac{1}{3}}}{r})_{w \to q^{\frac{1}{2}}w}}{\theta(w^2)} + \frac{\mathcal{I}_{\text{SQCD}}(\mathbf{c}^{(1)}, \mathbf{c}^{(2)}, \frac{w^{\frac{1}{3}}}{r}, \frac{w^{- \frac{1}{3}}}{r})_{w \to q^{ - \frac{1}{2}}w}}{\theta(w^{ - 2})} \ ,
\end{align}
where the denominator is related to $\vartheta_1$ by
\begin{equation}
	\theta(z) \equiv \frac{\vartheta_1( \mathfrak{z})}{i z^{\frac{1}{2}} q^{\frac{1}{8}} (q;q)} \ .
\end{equation}
Note that two of the $SU(3)$ flavor symmetries of the $E_6$ theory share identical fugacities $\mathbf{c}^{(1)}, \mathbf{c}^{(2)}$ with those of the $SU(3)$ SQCD. The above formula allows one to directly compute the Schur index of, for instance, a theory of class-$\mathcal{S}$ with three maximal and one minimal punctures (see Figure \ref{fig:T3-HM}),
\begin{align}
	\mathcal{I}
	= \oint \frac{d\mathbf{a}}{2\pi i \mathbf{a}} \sum_{\pm} 
	\frac{\mathcal{I}_{\text{SQCD}}(\mathbf{c}^{(1)}, \mathbf{a}^{-1}, \frac{w^{\frac{1}{3}}}{r}, \frac{w^{- \frac{1}{3}}}{r})_{w \to q^{\pm \frac{1}{2}}w}}{\theta(w^{\pm 2})} \mathcal{I}_\text{VM}(\mathbf{a}) \mathcal{I}_\text{HM}(\mathbf{a}, c^{(3)}, d^{(3)}) \ .
\end{align}
As we have argued, this can be computed exactly with the currently available formula in the appendix.
% Figure environment removed










%!TEX root = ../Schur indices and line operators.tex
\subsection{\texorpdfstring{$\mathcal{N}=4$ $SO(7)$ SYM}{}}


The Schur index of the $\mathcal{N} = 4$ $SO(7)$ theory is a contour integral of the following integrand,
\begin{align}
\mathcal{Z}\left(a_1,a_2,a_3\right)=\left(\frac{\vartheta_1^{\prime}(0)}{\vartheta_4(0)}\right)^3\prod_{\alpha,\beta}\prod_{i<j}\frac{\vartheta_1(\alpha\mathfrak{a}_i+\beta\mathfrak{a}_j,q)}{\vartheta_4(\alpha\mathfrak{a}_i+\beta\mathfrak{a}_j+\mathfrak{b},q)}\prod_{\alpha}\prod_{i=1}^3 \frac{\vartheta_1(\alpha\mathfrak{a}_i,q)}{\vartheta_4(\alpha\mathfrak{a}_i+\mathfrak{b},q)} \ ,
\end{align}
which is separately elliptic with respect to all three variables $\mathfrak{a}_{1,2,3}$.

The integral can be performed analytically by integrating $a_1, a_2, a_3$ one after another using the integration formula collected in the appendix \ref{app:integration-formula}. The $a_1$ integration involves the following simple poles which are all imaginary,
\begin{align}
\alpha\mathfrak{b}+\frac{\tau}{2},
\quad \alpha\mathfrak{a}_2+\beta\mathfrak{b}+\frac{\tau}{2},
\quad \alpha\mathfrak{a}_3+\beta\mathfrak{b}+\frac{\tau}{2} \ .
\end{align}
The residues of these simple poles are denoted by $\mathcal{P}_{\alpha}$, $\mathcal{Q}_{\alpha\beta}$ and $\tilde{\mathcal{Q}}_{\alpha\beta}$.
%\begin{align}
%\mathcal{P}_{\alpha}= & \ \frac{\alpha\left(\vartheta_1^\prime(0)\right)^2}{\vartheta_1(2\mathfrak{b})\vartheta_4(\mathfrak{b})}\prod_{\gamma}\frac{\vartheta_1^2\left(\mathfrak{a}_2+\gamma\mathfrak{a}_3\right)\vartheta_1\left(\mathfrak{a}_2+\gamma\mathfrak{b}\right)\vartheta_1\left(\mathfrak{a}_3+\gamma\mathfrak{b}\right)}{\vartheta_1\left(\mathfrak{a}_2+2\gamma\mathfrak{b}\right)\vartheta_1\left(2\mathfrak{b}+\gamma\mathfrak{a}_3\right)\vartheta_4\left(\mathfrak{a}_2-\mathfrak{a}_3+\gamma\mathfrak{b}\right)\vartheta_4\left(\mathfrak{a}_2+\mathfrak{a}_3+\gamma\mathfrak{b}\right)}  \ , \nonumber\\
%\mathcal{Q}_{\alpha\beta}= & \ -\frac{\vartheta_1^\prime(0)^2}{\vartheta_4(\mathfrak{b})}\frac{\vartheta_1(\alpha\mathfrak{a}_2)\vartheta_4(\beta\mathfrak{b}+\alpha\mathfrak{a}_2)\vartheta_4^2(\beta\mathfrak{b}+2\alpha\mathfrak{a}_2)}{\vartheta_1(2\beta\mathfrak{b})\vartheta_1(2\alpha\mathfrak{a}_2)\vartheta_1(2\beta\mathfrak{b}+\alpha\mathfrak{a}_2)\vartheta_1(2\beta\mathfrak{b}+2\alpha\mathfrak{a}_2)\vartheta_4(-\beta\mathfrak{b}+\alpha\mathfrak{a}_2)}\notag\\
%&\times\prod_{\gamma=\pm}\frac{\vartheta_1(\mathfrak{a}_2)\vartheta_1(\mathfrak{a}_2+\gamma\mathfrak{a}_3)\vartheta_4(\beta\mathfrak{b}+\alpha\mathfrak{a}_3+\gamma\mathfrak{a}_2)}{\vartheta_1(2\beta\mathfrak{b}+\alpha\mathfrak{a}_3+\gamma\mathfrak{a}_2)\vartheta_4(\beta\mathfrak{b}-\alpha\mathfrak{a}_3+\gamma\mathfrak{a}_2)\vartheta_4(\mathfrak{a}_2+\gamma\mathfrak{b})} \ , \nonumber\\
%\tilde{\mathcal{Q}}_{\alpha\beta} = & \ -\frac{\vartheta_1^\prime(0)^2}{\vartheta_4(\mathfrak{b})}\frac{\vartheta_1(\alpha\mathfrak{a}_3)\vartheta_4(\beta\mathfrak{b}+\alpha\mathfrak{a}_3)\vartheta_4^2(\beta\mathfrak{b}+2\alpha\mathfrak{a}_3)}{\vartheta_1(2\beta\mathfrak{b})\vartheta_1(2\alpha\mathfrak{a}_3)\vartheta_1(2\beta\mathfrak{b}+\alpha\mathfrak{a}_3)\vartheta_1(2\beta\mathfrak{b}+2\alpha\mathfrak{a}_3)\vartheta_4(-\beta\mathfrak{b}+\alpha\mathfrak{a}_3)} \nonumber\\
%& \times\prod_{\gamma=\pm}\frac{\vartheta_1(\mathfrak{a}_3)\vartheta_1(\mathfrak{a}_3+\gamma\mathfrak{a}_2)\vartheta_4(\beta\mathfrak{b}+\alpha\mathfrak{a}_2+\gamma\mathfrak{a}_3)}{\vartheta_1(2\beta\mathfrak{b}+\alpha\mathfrak{a}_2+\gamma\mathfrak{a}_3)\vartheta_4(\beta\mathfrak{b}-\alpha\mathfrak{a}_2+\gamma\mathfrak{a}_3)\vartheta_4(\mathfrak{a}_3+\gamma\mathfrak{b})}  \ .
%\end{align}
Using the integration formula (\ref{integration-formula-f}), the $a_1$ integration leaves an integrand
\begin{align}
\mathcal{Z}_1(a_2,a_3)
= & \ \oint_{|a_1|=1}\frac{da_1}{2\pi i a_1}\mathcal{Z}(a_1,a_2,a_3) \nonumber\\
= & \ \sum_{\alpha}\mathcal{P}_\alpha E_1\begin{bmatrix}
-1\\
b^\alpha
\end{bmatrix}+\sum_{\alpha,\beta}\mathcal{Q}_{\alpha\beta}E_1\begin{bmatrix}
-1\\
a_2^{\alpha}b^{\beta}
\end{bmatrix}+\sum_{\alpha,\beta}\tilde{\mathcal{Q}}_{\alpha\beta}E_1\begin{bmatrix}
-1\\
a_3^{\alpha}b^\beta \
\end{bmatrix}.
\end{align}
The poles and residues of $\mathcal{P}, \mathcal{Q}, \tilde {\mathcal{Q}}$ are listed in Table \ref{poles-residues-SO(7)}, which are used in the $a_2$-integration.
{
	\renewcommand{\arraystretch}{1.8}
	\begin{table}[h!]
		\centering
		\begin{tabular}{c|c|c}
			& poles  & residues  \\
			\hline
			$\mathcal{P}_{\alpha}$ & $2\gamma\mathfrak{b}$ & $\mathcal{P}_{\alpha\gamma}$\\
			&  $\frac{1}{2}\left(2\gamma\mathfrak{a}_3+2\beta\mathfrak{b}+\tau\right)$    &  $\mathcal{P}_{\alpha\beta\gamma}$   \\
			\hline
			 $\mathcal{Q}_{\alpha\beta}$  & $\frac{k}{2}+\frac{\ell}{2}\tau$,$\quad$  $(k,\ell) = \{(0,1),(1,0),(1,1)\}$ & $ \mathcal{Q}_{\alpha\beta}^{(k,\ell)} $\\
			                              & $-\alpha\beta\mathfrak{b}+\frac{k}{2}+\frac{\ell}{2}\tau$, $\quad $ $\{(k,\ell)\}=\{(0,0),(1,0),(1,1)\}$  & $-\mathcal{Q}_{\alpha\beta}^{(k,\ell+1)}$ \\
			                              & $\alpha\beta\mathfrak{b}+\frac{\tau}{2}$  & $\gamma\mathcal{P}_{\alpha\gamma}$ \\
			                              & $\gamma\mathfrak{a}_3+\alpha\beta\mathfrak{b}+\frac{\tau}{2}$ &  $\mathcal{Q}_{\alpha\beta\gamma}$   \\
			                              &   $\gamma\mathfrak{a}_3-2\alpha\beta\mathfrak{b}$           &  $\mathcal{Q}_{\alpha\beta -\gamma}$  \\
			\hline
			$\tilde{\mathcal{Q}}_{\alpha\beta}$ &    $\gamma\mathfrak{b}+\frac{\tau}{2}$    &    $-\mathcal{P}_{\gamma\beta\alpha}$   \\
			&  $-\gamma\mathfrak{a}_3+\alpha\beta\gamma\mathfrak{b}+\frac{\tau}{2}$  &   $\tilde{\mathcal{Q}}_{\alpha\beta\gamma}$   \\
			&  $\gamma(-\mathfrak{a}_3-2\alpha\beta\mathfrak{b})$        &  $-\gamma\mathcal{Q}_{\alpha\beta 1}$   
		\end{tabular}
		\caption{Poles and residues of $\mathcal{P}_{\alpha}$, $\mathcal{Q}_{\alpha\beta}$ and $\tilde{\mathcal{Q}}_{\alpha\beta}$ with respect to the variable $\mathfrak{a}_2$. Here $\alpha, \beta, \gamma=\pm 1$.\label{poles-residues-SO(7)}}
	\end{table}
}

 %The function $\mathcal{P}_{\alpha}$ has the following simple poles as an elliptic function of $\mathfrak{a}_2$,
%\begin{align}
%2\gamma\mathfrak{b} , \quad \frac{1}{2}\left(2\gamma\mathfrak{a}_3+2\beta\mathfrak{b}+\tau\right) \ ,\qquad \gamma,\beta=\pm 1 \ .
%\end{align}
%The corresponding residues are listed below
%\begin{align}
%&\mathcal{P}_{\alpha\gamma}=-\gamma\alpha\frac{\vartheta_1^\prime(0)\vartheta_4(3\mathfrak{b})\vartheta_1(\mathfrak{a}_3-2\mathfrak{b})\vartheta_1(\mathfrak{a}_3+2\mathfrak{b})}{\vartheta_1(2\mathfrak{b})\vartheta_1(4\mathfrak{b})\vartheta_4(\mathfrak{a}_3-3\mathfrak{b})\vartheta_4(\mathfrak{a}_3+3\mathfrak{b})} \ ,\\
%&\mathcal{P}_{\alpha\beta\gamma}=\alpha\beta\frac{\vartheta_1(\gamma\mathfrak{a}_3)\vartheta_4(\mathfrak{b})\vartheta_4(\beta\mathfrak{b}+\gamma \mathfrak{a}_3)\vartheta_4(\beta\mathfrak{b}+2\gamma\mathfrak{a}_3)^2\vartheta_1^\prime(0)}{\vartheta_1(2\mathfrak{b})^2\vartheta_1(2\gamma \mathfrak{a}_3)\vartheta_1(-2\beta\mathfrak{b}+\gamma\mathfrak{a}_3)\vartheta_1(2\beta\mathfrak{b}+2\gamma\mathfrak{a}_3)\vartheta_4(3\beta\mathfrak{b}+\gamma\mathfrak{a}_3)} \ .
%\end{align}

 Using the integration formula (\ref{integration-formula-fE-1}), the $a_2$ integration leaves a final integrand
 \begin{align}
\mathcal{Z}_2(a_3)=\oint \frac{da_2}{2\pi i a_2}\mathcal{Z}_1(a_2,a_3)= I_1+ I_2 +I_3,
 \end{align}
where 
%For the function $\mathcal{Q}_{\alpha\beta}$, as an elliptic function for $\mathfrak{a}_2$, it has the following poles, 
%\begin{align}
%&\frac{k}{2}+\frac{\ell}{2}\tau,  & (k,\ell) = & \{(0,1),(1,0),(1,1)\} \ ,\\
%&-\alpha\beta\mathfrak{b}+\frac{k}{2}+\frac{\ell}{2}\tau,\quad & \ \{(k,\ell)\}= & \ \{(0,0),(1,0),(1,1)\}, \\
%& \ \alpha\beta \mathfrak{b}+\frac{\tau}{2} \ ,\\
%&\gamma\mathfrak{a}_3+\alpha\beta\mathfrak{b}+\frac{\tau}{2},\quad \gamma\mathfrak{a}_3-2\alpha\beta\mathfrak{b}\quad & \gamma= & \ \pm 1 \ .
%\end{align}
%At the poles $\mathfrak{a}_2=\frac{k}{2}+\frac{\ell}{2}\tau$, the corresponding residues are:
%\begin{align}
%\mathcal{Q}_{\alpha\beta}^{(k,l)} = \frac{
%	-2\alpha (k-\frac{1}{2})b^{-\ell}\psi \vartheta_1(\frac{k}{2}+\frac{\ell}{2}\tau)\vartheta_1(\mathfrak{a}_3+\frac{k}{2}+\frac{\ell}{2}\tau)^2
%}{
%	\vartheta_1\left(2\mathfrak{b}+\frac{k}{2}+\frac{\ell}{2}\tau\right)\vartheta_1\left(\mathfrak{a}_3+2\mathfrak{b}+\frac{k}{2}+\frac{\ell}{2}\tau\right)\vartheta_1\left(\mathfrak{a}_3-2\mathfrak{b}+\frac{k}{2}+\frac{\ell}{2}\tau\right)
%} \ , \nonumber
%\end{align}
%where
%\begin{align}
%	\psi \coloneqq \frac{\vartheta_1^\prime(0)\vartheta_1(\mathfrak{a}_3)^2\vartheta_4(\mathfrak{b})}{2\vartheta_1(2\mathfrak{b})^2\vartheta_4(\mathfrak{a}_3+\mathfrak{b})\vartheta_4(\mathfrak{a}_3-\mathfrak{b})} \ .
%\end{align} 
%At the poles $\mathfrak{a}_2=-\alpha\beta\mathfrak{b}+\frac{k}{2}+\frac{\ell}{2}\tau$, the residues are:
%\begin{align}
% -\mathcal{Q}_{\alpha\beta}^{(k,l+1)}=2\alpha \left(k-\frac{1}{2}\right)b^{-\ell}\psi\frac{\vartheta_4\left(\frac{k}{2}+\frac{\ell}{2}\tau\right)\vartheta_4\left(\mathfrak{a}_3+\frac{k}{2}+\frac{\ell}{2}\tau\right)^2}{\vartheta_4\left(2\mathfrak{b}+\frac{k}{2}+\frac{\ell}{2}\tau\right)\vartheta_4\left(\mathfrak{a}_3+2\mathfrak{b}+\frac{k}{2}+\frac{\ell}{2}\tau\right)\vartheta_4\left(\mathfrak{a}_3-2\mathfrak{b}+\frac{k}{2}+\frac{\ell}{2}\tau\right)}
%\end{align} 
%At $\mathfrak{a}_2=\alpha\beta\mathfrak{b}+\frac{\tau}{2}$, the residue is given by:
%\begin{align}
%\gamma\mathcal{P}_{\alpha\gamma}=-\alpha\frac{\vartheta^\prime_1(0)\vartheta_4(3\mathfrak{b})}{\vartheta_1(2\mathfrak{b})\vartheta_1(4\mathfrak{b})}\prod_{\gamma=\pm}\frac{\vartheta_1\left(\mathfrak{a}_3+2\gamma\mathfrak{b}\right)}{\vartheta_4\left(\mathfrak{a}_3+3\gamma\mathfrak{b}\right)}
%\end{align}
%At $\mathfrak{a}_2=\gamma\mathfrak{a}_3+\alpha\beta\mathfrak{b}+\frac{\tau}{2}$, the residue is:
%\begin{align}
%\mathcal{Q}_{\alpha\beta\gamma}=-\alpha\frac{\vartheta_1^\prime(0)}{\vartheta_4(3\mathfrak{b})}\frac{\vartheta_1\left(\mathfrak{a}_3\right)\vartheta_1\left(\mathfrak{a}_3+2\alpha\beta\gamma\mathfrak{b}\right)\vartheta_4\left(2\mathfrak{a}_3+\alpha\beta\gamma\mathfrak{b}\right)\vartheta_4\left(2\mathfrak{a}_3+3\alpha\beta\gamma\mathfrak{b}\right)}{\vartheta_1(2\mathfrak{a}_3)\vartheta_1\left(2\mathfrak{a}_3+4\alpha\beta\gamma\mathfrak{b}\right)\vartheta_4\left(\mathfrak{a}_3-\alpha\beta\gamma\mathfrak{b}\right)\vartheta_4\left(\mathfrak{a}_3+3\alpha\beta\gamma\mathfrak{b}\right)}
%\end{align}
%At $\mathfrak{a}_2=\gamma\mathfrak{a}_3-2\alpha\beta\mathfrak{b}$:
%\begin{align}
%\mathcal{Q}_{\alpha\beta -\gamma}=\alpha\frac{\vartheta_1^\prime(0)}{\vartheta_4(3\mathfrak{b})}\frac{\vartheta_1\left(\mathfrak{a}_3\right)\vartheta_1\left(\mathfrak{a}_3-2\alpha\beta\gamma\mathfrak{b}\right)\vartheta_4\left(2\mathfrak{a}_3-\alpha\beta\gamma\mathfrak{b}\right)\vartheta_4\left(2\mathfrak{a}_3-3\alpha\beta\gamma\mathfrak{b}\right)}{\vartheta_1(2\mathfrak{a}_3)\vartheta_1\left(2\mathfrak{a}_3-4\alpha\beta\gamma\mathfrak{b}\right)\vartheta_4\left(\mathfrak{a}_3+\alpha\beta\gamma\mathfrak{b}\right)\vartheta_4\left(\mathfrak{a}_3-3\alpha\beta\gamma\mathfrak{b}\right)}
%\end{align}


%For $\tilde{\mathcal{Q}}_{\alpha\beta}$, as an elliptic function for $a_2$, it has the following poles:
%\begin{align}
%\gamma\mathfrak{b}+\frac{\tau}{2}\quad -\gamma\mathfrak{a}_3+\alpha\beta\gamma\mathfrak{b}+\frac{\tau}{2}\quad \gamma(-\mathfrak{a}_3-2\alpha\beta\mathfrak{b})\quad \gamma=\pm 1
%\end{align}
%The residues are:
%\begin{align}
%&-\mathcal{P}_{\gamma\beta\alpha}=-\beta\gamma\frac{\vartheta_4(\mathfrak{b})\vartheta_1^\prime(0)}{\vartheta_1(2\mathfrak{b})^2}\frac{\vartheta_1(\mathfrak{a}_3)\vartheta_4(\mathfrak{a}_3+\alpha\beta\mathfrak{b})\vartheta_4(2\mathfrak{a}_3+\alpha\beta\mathfrak{b})^2}{\vartheta_1(2\mathfrak{a}_3)\vartheta_1(\mathfrak{a}_3-2\alpha\beta\mathfrak{b})\vartheta_1(2\mathfrak{a}_3+2\alpha\beta\mathfrak{b})\vartheta_4(\mathfrak{a}_3+3\alpha\beta\mathfrak{b})}\\
%&\tilde{\mathcal{Q}}_{\alpha\beta\gamma}=-\alpha\gamma\frac{\vartheta_1^\prime(0)}{\vartheta_4(3\mathfrak{b})}\prod_{\delta=\pm 1}\frac{\vartheta_4(\mathfrak{a}_3+\delta\mathfrak{b})\vartheta_4(2\mathfrak{a}_3+\delta\mathfrak{b})}{\vartheta_1(2\mathfrak{a}_3+2\delta\mathfrak{b})\vartheta_1(\mathfrak{a}_3+2\delta\mathfrak{b})}\\
%&-\gamma\mathcal{Q}_{\alpha\beta 1}=\alpha\gamma\frac{\vartheta_1^\prime(0)}{\vartheta_4(3\mathfrak{b})}\frac{\vartheta_1(\mathfrak{a}_3)\vartheta_1(\mathfrak{a}_3+2\alpha\beta\mathfrak{b})\vartheta_4(2\mathfrak{a}_3+\alpha\beta\mathfrak{b})\vartheta_4(2\mathfrak{a}_3+3\alpha\beta\mathfrak{b})}{\vartheta_1(2\mathfrak{a}_3)\vartheta_1(2\mathfrak{a}_3+4\alpha\beta\mathfrak{b})\vartheta_4(\mathfrak{a}_3-\alpha\beta\mathfrak{b})\vartheta_4(\mathfrak{a}_3+3\alpha\beta\mathfrak{b})}
%\end{align}
%Therefore, according to the integration formula (\ref{integration-formula-f}):(take $\mathfrak{z}_0=\mathfrak{b}$ here)
\begin{align}
I_1=&\ \sum_{\alpha=\pm 1}\oint\frac{da_2}{2\pi i a_2}\mathcal{P}_{\alpha}E_1\begin{bmatrix}
-1\\
b^\alpha
\end{bmatrix}\nonumber\\
= & \sum_{\alpha, \gamma=\pm 1}\mathcal{P}_{\alpha\gamma}E_1\begin{bmatrix}
-1\\
b^{2\gamma-1}q^{\frac{1}{2}}
\end{bmatrix}E_1\begin{bmatrix}
-1\\
b^\alpha
\end{bmatrix}+\sum_{\alpha, \beta,\gamma=\pm 1}\mathcal{P}_{\alpha\beta\gamma}E_1\begin{bmatrix}
-1\\
a_3^\gamma b^{\beta-1}
\end{bmatrix}E_1\begin{bmatrix}
-1\\
b^\alpha
\end{bmatrix}.
\end{align}
%According to (\ref{integration-formula-fE-1}), (here we take $\mathfrak{z}_0=0$)
%\begin{align}
%&\oint \frac{da_2}{2\pi i a_2}\alpha\mathcal{Q}_{\alpha\beta}E_1\begin{bmatrix}
%-1\\
%a_2 b^{\alpha\beta}
%\end{bmatrix}=-\alpha\sum_{j \operatorname{Real}/\operatorname{Imag.}}R_j\left(\mathcal{S}_1 E_1\begin{bmatrix}
%-1\\
%z_j q^{\pm\frac{1}{2}}
%\end{bmatrix}+\mathcal{S}_0 E_2\begin{bmatrix}
%+1\\
%z_j b^{\alpha\beta}q^{\pm\frac{1}{2}}
%\end{bmatrix}\right)\\
%&=-\alpha\left(\mathcal{Q}^{(1,0)}_{\alpha\beta}\left(\mathcal{S}_1E_1\begin{bmatrix}
%-1\\
%-q^{\frac{1}{2}}
%\end{bmatrix}+\mathcal{S}_0 E_2\begin{bmatrix}
%+1\\
%-b^{\alpha\beta}q^{\frac{1}{2}}
%\end{bmatrix}\right)\right.\\
%&-\mathcal{Q}^{(0,1)}_{\alpha\beta}\left(\mathcal{S}_1 E_1\begin{bmatrix}
%-1\\
%b^{-\alpha\beta}q^{\frac{1}{2}}
%\end{bmatrix}+\mathcal{S}_0 E_2\begin{bmatrix}
%+1\\
%q^{\frac{1}{2}}
%\end{bmatrix}\right)\\
%&-\mathcal{Q}^{(1,1)}_{\alpha\beta}\left(\mathcal{S}_1 E_1\begin{bmatrix}
%-1\\
%-b^{-\alpha\beta}q^{\frac{1}{2}}
%\end{bmatrix}+\mathcal{S}_0 E_2\begin{bmatrix}
%+1\\
%-q^{\frac{1}{2}}
%\end{bmatrix}\right)\\
%&\left. +\sum_{\gamma}\mathcal{Q}_{\alpha\beta-\gamma}\left(\mathcal{S}_1 E_1\begin{bmatrix}
%-1\\
%a_3^{\gamma}b^{-2\alpha\beta}q^{\frac{1}{2}}
%\end{bmatrix}+\mathcal{S}_0 E_2\begin{bmatrix}
%+1\\
%a_3^{\gamma}b^{-\alpha\beta}q^{\frac{1}{2}}
%\end{bmatrix}\right)\right)\\
%&-\alpha\left(\mathcal{Q}^{(0,1)}_{\alpha\beta}\left(\mathcal{S}_1 E_1\begin{bmatrix}
%-1\\
%+1
%\end{bmatrix}+\mathcal{S}_0 E_2\begin{bmatrix}
%+1\\
%b^{\alpha\beta}
%\end{bmatrix}\right)+\mathcal{Q}^{(1,1)}_{\alpha\beta}\left(\mathcal{S}_1 E_1\begin{bmatrix}
%-1\\
%-1
%\end{bmatrix}+\mathcal{S}_0 E_2\begin{bmatrix}
%+1\\
%-b^{\alpha\beta}
%\end{bmatrix}\right)\right.\\
%&.-\mathcal{Q}^{(1,2)}_{\alpha\beta}\left(\mathcal{S}_1 E_1\begin{bmatrix}
%	-1\\
%	-b^{-\alpha\beta}
%\end{bmatrix}+\mathcal{S}_0 E_2\begin{bmatrix}
%+1\\
%-1
%\end{bmatrix}\right)+\gamma\mathcal{P}_{\alpha\gamma}\left(\mathcal{S}_1 E_1\begin{bmatrix}
%	-1\\
%	b^{\alpha\beta}
%\end{bmatrix}+\mathcal{S}_0 E_2\begin{bmatrix}
%+1\\
%b^{2\alpha\beta}
%\end{bmatrix}\right)\\
%&\left.+\sum_{\gamma}\mathcal{Q}_{\alpha\beta\gamma}\left(\mathcal{S}_1 E_1\begin{bmatrix}
%-1\\
%a_3^{\gamma}b^{\alpha\beta}
%\end{bmatrix}+\mathcal{S}_0 E_2\begin{bmatrix}
%+1\\
%a_3^{\gamma}b^{2\alpha\beta}
%\end{bmatrix}\right)\right)
%\end{align}
and 
\begin{align}
I_2= & \ \sum_{\alpha=\pm 1}\oint \frac{da_2}{2\pi i a_2}\mathcal{Q}_{\alpha\beta}E_1\begin{bmatrix}
	-1\\
	a_2^{\alpha} b^{\beta}
\end{bmatrix}=\sum_{\alpha=\pm 1}\sum_{k,\ell,m,n=0}^{1}(-1)^n \alpha \mathcal{Q}^{(k,\ell)}_{\alpha\beta}\mathcal{S}_{m}E_1\begin{bmatrix}
(-1)^m\\
(-1)^k b^{(n-m)\alpha\beta}q^{\frac{n+(-1)^n \ell}{2}}
\end{bmatrix}\nonumber\\
& \ -\sum_{\alpha,\beta,\gamma=\pm 1}\sum_{k=0}^1 \alpha\mathcal{Q}_{\alpha\beta\gamma}\mathcal{S}_{k}\left(E_{2-k}\begin{bmatrix}
(-1)^k\\
a_3^{-\gamma}b^{-(k+1)\alpha\beta}q^{\frac{1}{2}}
\end{bmatrix}+E_{2-k}\begin{bmatrix}
(-1)^k\\
a_3^\gamma b^{(2-k)\alpha\beta}
\end{bmatrix}\right)\nonumber\\
& \ -\sum_{\alpha,\beta=\pm 1}\sum_{k=0}^1 \alpha\gamma \mathcal{P}_{\alpha\gamma}\mathcal{S}_k E_{2-k}\begin{bmatrix}
(-1)^k\\
b^{(2-k)\alpha\beta}
\end{bmatrix},
\end{align}
and
\begin{align}
I_3= & \ \sum_{\alpha,\beta=\pm 1}\oint \frac{da_2}{2\pi i a_2}\tilde{Q}_{\alpha\beta}E_1\begin{bmatrix}
-1\\
a_3^{\alpha}b^\beta
\end{bmatrix}\nonumber\\
= & \ \sum_{\alpha,\beta=\pm 1}\left.\tilde{Q}_{\alpha\beta}\right|_{\mathfrak{a}_2=0} E_1\begin{bmatrix}
-1\\
a_3^{\alpha}b^\beta
\end{bmatrix}-\sum_{\alpha,\beta,\gamma=\pm 1}\gamma \mathcal{Q}_{\alpha\beta 1}E_1\begin{bmatrix}
-1\\
a_3^{-\gamma}b^{-2\alpha\beta\gamma}q^{\frac{1}{2}}
\end{bmatrix}E_1\begin{bmatrix}
-1\\
a_3^{\alpha}b^\beta
\end{bmatrix}\notag\\
& \ -\sum_{\alpha,\beta,\gamma=\pm 1}\mathcal{P}_{\gamma\beta\alpha}E_1\begin{bmatrix}
-1\\
b^\gamma
\end{bmatrix}E_1\begin{bmatrix}
-1\\
a_3^{\alpha}b^\beta
\end{bmatrix}+\sum_{\alpha,\beta,\gamma=\pm 1}\tilde{\mathcal{Q}}_{\alpha\beta\gamma}E_1\begin{bmatrix}
-1\\
a_3^{-\gamma}b^{\alpha\beta\gamma}
\end{bmatrix}E_1\begin{bmatrix}
-1\\
a_3^{\alpha}b^\beta
\end{bmatrix}.
\end{align}
The closed-form of $\mathcal{N}=4$ $SO(7)$ Schur index is then given by
\begin{align}
\mathcal{I}=\oint_{|a_3|=1}\frac{da_3}{2\pi i a_3}\mathcal{Z}_2(a_3).
\end{align}
At this stage we encounter the following types of integral
\begin{align}
\oint_{|z|=1}\frac{dz}{2\pi i z}f(z)E_k\begin{bmatrix}
\pm 1\\
z a
\end{bmatrix}, \quad \oint_{|z|=1}\frac{dz}{2\pi i z}f(z)E_1\begin{bmatrix}
-1\\
z a
\end{bmatrix}E_1\begin{bmatrix}
-1\\
z b
\end{bmatrix} \ ,
\end{align}
which can be computed using the integration formula (\ref{integration-formula-fE-1}), (\ref{integration-formula-fE-2}), and (\ref{integration-formula-fEE-1}). Since the computation of integrand is somewhat technical and tedious, we will only present the final result without the details. To do so, wee define $\mathbf{I}_{1,2,3}$ as some intricate combinations of Eisenstein series,
\begin{align}
(\mathbf{I}_1)_{\alpha\beta} \coloneqq & \ E_1\begin{bmatrix}
-1\\
b
\end{bmatrix}\left(-\sum_{k=0}^2 (-1)^{\lceil\frac{k}{2}\rceil}(3-k)E_2\begin{bmatrix}
(-1)^{\beta+k+1}\\
(-1)^\alpha b^k
\end{bmatrix}\right. \nonumber\\
&\ \left. +(-1)^\beta\left(\sum_{\pm 1}\mp E_1\begin{bmatrix}
-1\\
(-1)^\alpha b^{\frac{1}{2}} q^{\pm\frac{1}{4}}
\end{bmatrix}+\frac{1}{4}\right)\right),
\end{align}
and
\begin{align}
(\mathbf{I}_2)_{\beta\gamma} \coloneqq & \ 4(-1)^\beta \left(3E_3\begin{bmatrix}
(-1)^{\beta+\gamma+1}\\
(-1)^\beta b^2
\end{bmatrix}-3 E_3\begin{bmatrix}
(-1)^{\beta+\gamma}\\
(-1)^\beta b
\end{bmatrix}+E_3\begin{bmatrix}
(-1)^{\beta+\gamma}\\
(-1)^\beta b^3
\end{bmatrix}\right) \nonumber\\
& \ -2 \sum_{k=0}^1 (-1)^{\beta+k} E_1\begin{bmatrix}
-1\\
b
\end{bmatrix}E_2\begin{bmatrix}
(-1)^{\beta+\gamma}\\
(-1)^\beta b^{2k+1}
\end{bmatrix}-4(-1)^\beta E_2\begin{bmatrix}
1\\
b^2
\end{bmatrix} E_1\begin{bmatrix}
(-1)^{\beta+\gamma+1}\\
(-1)^\beta b^2
\end{bmatrix} \nonumber \\
& \ +2(-1)^\beta E_1\begin{bmatrix}
(-1)^{\beta+\gamma+1}\\
(-1)^\beta b^2
\end{bmatrix}\sum_{k\in\{0,1,3\}}\left(4-k\right)(-1)^{\lceil\frac{k}{2}\rceil}E_2\begin{bmatrix}
(-1)^{\beta+\gamma+k}\\
b^k
\end{bmatrix}\\
& \ +\sum_{\pm}\sum_{k=0}^1 E_1\begin{bmatrix}
	1\\
	(-1)^\beta b^{\frac{1}{2}+k} q^{\pm\frac{(-1)^k}{4}}
\end{bmatrix}\left( 2(-1)^\beta  \left(E_2\begin{bmatrix}
1\\
b^2
\end{bmatrix}-E_2\begin{bmatrix}
-1\\
b
\end{bmatrix}\right)\pm (-1)^\gamma E_1\begin{bmatrix}
-1\\
b
\end{bmatrix}\right) \nonumber\\
& \ -\frac{5+4\gamma -4\beta}{2}E_1\begin{bmatrix}
(-1)^{\beta+\gamma+1}\\
(-1)^\beta b^2
\end{bmatrix}+5\beta\gamma E_1\begin{bmatrix}
-1\\
-b^2
\end{bmatrix} \nonumber\\
& \ +\sum_{\pm}\sum_{k=0}^1 \frac{(-1)^\beta}{4}E_1\begin{bmatrix}
1\\
(-1)^\beta b^{\frac{1}{2}+k} q^{\pm\frac{(-1)^k}{4}}
\end{bmatrix}, \nonumber
\end{align}
and
\begin{align}
\mathbf{I}_3 \coloneqq & \ -4\left(-\sum_{k=0}^1\sum_{\alpha=\pm 1}E_1\begin{bmatrix}
1\\
b^{k+\frac{3}{2}}q^{\frac{\alpha}{4}}
\end{bmatrix}+E_1\begin{bmatrix}
-1\\
b^5
\end{bmatrix}\right)\left(\sum_{i=1}^2 (-1)^{i+1}E_2\begin{bmatrix}
(-1)^i\\
b^i
\end{bmatrix}-\frac{1}{8}\right) \nonumber\\
& \ -2\sum_{k=1}^2 E_1\begin{bmatrix}
-1\\
b^{2k+1}
\end{bmatrix}E_1\begin{bmatrix}
-1\\
b
\end{bmatrix}^2+E_1\begin{bmatrix}
-1\\
b^3
\end{bmatrix}\left(\sum_{k\in\{1,2,4\}}B_k E_2\begin{bmatrix}
(-1)^k\\
b^k
\end{bmatrix}+4\right) \nonumber\\
& \ +E_1\begin{bmatrix}
-1\\
b
\end{bmatrix}\left(\sum_{\alpha=\pm 1}2\alpha \left(\sum_{k=1}^2 k E_1\begin{bmatrix}
(-1)^{k+1}\\
b^{\frac{5}{2}}q^{\frac{\alpha}{4}}
\end{bmatrix}- E_1\begin{bmatrix}
1\\
b^{\frac{3}{2}}q^{\frac{\alpha}{4}}
\end{bmatrix}\right)+\sum_{k=1}^3 C_k E_2\begin{bmatrix}
(-1)^k\\
b^k
\end{bmatrix}+\frac{9}{2} \right) \nonumber\\
& \ -2\prod_{k=0}^2 E_1\begin{bmatrix}
-1\\
b^{2k+1}
\end{bmatrix}-2 E_1\begin{bmatrix}
-1\\
b
\end{bmatrix}^3+\sum_{n=1}^4 A_n E_3\begin{bmatrix}
(-1)^n\\
b^n
\end{bmatrix}.
\end{align}
With the above three definitions, we can express the $\mathcal{N} = 4$ $SO(7)$ Schur index as
\begin{align}
\mathcal{I}_{\mathcal{N} = 4 \ SO(7)}
= \frac{i}{48}\sum_{\alpha,\beta}\left(\mathbf{R}_{\alpha\beta}(\mathbf{I}_1)_{\alpha\beta}+\mathbf{T}_{\alpha\beta}(\mathbf{I}_2)_{\alpha\beta}\right)+\frac{i}{48}\mathbf{W}\mathbf{I}_3.
\end{align}
In this formula, the greek indices $(\alpha,\beta)$ sum over the set $\{(0,0),(1,0),(1,1)\}$.
Note that
\begin{align}
&\mathbf{R}_{\alpha\beta} \coloneqq b^{-2\beta}\frac{\vartheta_1\left(\mathfrak{b}+\frac{\alpha+\beta\tau}{2}\right)\vartheta_4\left(\frac{\alpha+\beta\tau}{2}\right)\vartheta_4(\mathfrak{b})^3}{\vartheta_1\left(3\mathfrak{b}+\frac{\alpha+\beta\tau}{2}\right)\vartheta_4\left(2\mathfrak{b}+\frac{\alpha+\beta\tau}{2}\right)\vartheta_1(2\mathfrak{b})^3},\\
&\mathbf{T}_{\beta\gamma} \coloneqq  b^{\gamma-\beta}\frac{\vartheta_4\left(\frac{\beta+(\beta-\gamma)\tau}{2}\right)\vartheta_4\left(2\mathfrak{b}+\frac{\beta+(\beta-\gamma)\tau}{2}\right)\vartheta_4(\mathfrak{b})}{\vartheta_1\left(3\mathfrak{b}+\frac{\beta+(\beta-\gamma)\tau}{2}\right)\vartheta_1\left(\mathfrak{b}+\frac{\beta+(\beta-\gamma)\tau}{2}\right)\vartheta_1(4\mathfrak{b})}
\quad
\mathbf{W} \coloneqq  \prod_{k=0}^2\frac{\vartheta_4\left((2k+1)\mathfrak{b}\right)}{\vartheta_1\left((2k+2)\mathfrak{b}\right)}.
\end{align}

Finally, it is straightforward to check that the unflavored limit of $\mathcal{I}_{\mathcal{N} = 4 \ SO(7)}$ satisfies a monic $\Gamma^0(2)$ modular differential equation at order 10,
\begin{footnotesize}
\begin{align}
	&\mathcal{D}^{\mathcal{N}=4}_{\mathfrak{so}(7)}=\mathcal{D}_q^{(10)}+\left(\frac{64169}{45888}\Theta_{1,1}-\frac{116269}{45888}\Theta_{0,2}\right)\mathcal{D}_q^{(8)}+\left(\frac{99455}{68832}\Theta_{0,3}-\frac{51397}{22944}\Theta_{1,2}\right)\mathcal{D}_q^{(7)}+\left(\frac{4531009 \Theta_{0,4}}{13215744}\right.\nonumber\\
	&\left.-\frac{3779273\Theta_{1,3}}{3303936}+\frac{5245697\Theta_{2,2}}{4405248}\right)\mathcal{D}^{(6)}_q+\left(-\frac{1133653\Theta_{0,5}}{4405248}+\frac{2557903\Theta_{1,4}}{4405248}-\frac{55973\Theta_{2,3}}{2202624}\right)\mathcal{D}^{(5)}_q\nonumber\\
	&+\left(\frac{1190885473\Theta_{0,6}}{22836805632}-\frac{924970757\Theta_{1,5}}{3806134272}+\frac{3937715525\Theta_{2,4}}{7612268544}-\frac{2505775369\Theta_{3,3}}{11418402816}\right)\mathcal{D}_q^{(4)}+\left(-\frac{117336059\Theta_{0,7}}{22836805632}\right.\nonumber\\
	&\left.+\frac{2991097351\Theta_{1,6}}{22836805632}-\frac{380366011\Theta_{2,5}}{2537422848}+\frac{930902663\Theta_{3,4}}{22836805632}\right)\mathcal{D}_q^{(3)}+\left(-\frac{274137107749\Theta_{0,8}}{26308000088064}-\frac{3736889371\Theta_{1,7}}{3288500011008}\right.\nonumber\\
	&\left.-\frac{330134662435\Theta_{2,6}}{6577000022016}+\frac{199201642115\Theta_{3,5}}{3288500011008}+\frac{41820786289\Theta_{4,4}}{26308000088064}\right)\mathcal{D}_q^{(2)}+\left(\frac{240693275531\Theta_{0,9}}{39462000132096}\right.\nonumber\\
	&\left.-\frac{88992212869\Theta_{1,8}}{4384666681344}-\frac{8099874757\Theta_{2,7}}{1096166670336}+\frac{330064570085\Theta_{3,6}}{3288500011008}-\frac{174451260571\Theta_{4,5}}{2192333340672}\right)\mathcal{D}_q^{(1)}+\left(-\frac{256921875\Theta_{0,10}}{256624295936}\right.\nonumber\\
	&\left.+\frac{477416835\Theta_{1,9}}{128312147968}-\frac{59821335\Theta_{2,8}}{256624295936}+\frac{559460601\Theta_{3,7}}{32078036992}-\frac{13109319531\Theta_{4,6}}{128312147968}+\frac{10552431897\Theta_{5,5}}{128312147968}\right),
\end{align}
\end{footnotesize}
and non-monic $\Gamma^0(2)$ equation at order 9,
\begin{footnotesize}
\begin{align}
&\mathcal{D}_{\mathfrak{so}(7)}^{\mathcal{N}=4}=\Theta_{0,1}\mathcal{D}_q^{(9)}+\left(\frac{2477\Theta_{1,1}}{1912}-\frac{3407\Theta_{0,2}}{1912}\right)\mathcal{D}_q^{(8)}+\left(\frac{6971\Theta_{1,2}}{11472}-\frac{10377\Theta_{0,3}}{3824}\right)\mathcal{D}_q^{(7)}+\left(\frac{1339781\Theta_{0,4}}{275328} \right.\nonumber\\
&\left.-\frac{15625\Theta_{1,3}}{1434}+\frac{1767811\Theta_{2,2}}{275328}\right)\mathcal{D}_q^{(6)}+\left(\frac{17516635\Theta_{0,5}}{13215744}-\frac{37436353\Theta_{1,4}}{13215744}+\frac{12423443\Theta_{2,3}}{6607872}\right)\mathcal{D}_q^{(5)}+\nonumber\\
&\left(-\frac{44438921\Theta_{0,6}}{317177856}+\frac{24643855\Theta_{1,5}}{52862976}-\frac{177147775\Theta_{2,4}}{35241984}+\frac{767384147\Theta_{3,3}}{158588928}\right)\mathcal{D}_q^{(4)}+\left(\frac{436377635\Theta_{0,7}}{11418402816}\right.\nonumber\\
&\left.-\frac{2631122143\Theta_{1,6}}{11418402816}+\frac{6486893273\Theta_{2,5}}{3806134272}-\frac{17028452303\Theta_{3,4}}{11418402816}\right)\mathcal{D}_q^{(3)}+\left(\frac{287431763\Theta_{0,8}}{30449074176}+\frac{4632486095\Theta_{1,7}}{22836805632}\right.\nonumber\\
&\left.-\frac{3490714387\Theta_{2,6}}{2537422848}+\frac{19973363051\Theta_{3,5}}{7612268544}-\frac{133319985805\Theta_{4,4}}{91347222528}\right)\mathcal{D}_q^{(2)}+\left(-\frac{247722449497\Theta_{0,9}}{26308000088064}\right.\nonumber\\
&\left.+\frac{663592423\Theta_{1,8}}{36087791616}-\frac{105995132111\Theta_{2,7}}{243592593408}+\frac{2910273174797\Theta_{3,6}}{2192333340672}-\frac{439516599949\Theta_{4,5}}{487185186816}\right)\mathcal{D}_q^{(1)}+\left(\frac{49415625\Theta_{0,10}}{32078036992}\right.\nonumber\\
&\left.-\frac{52688223\Theta_{1,9}}{16039018496}-\frac{2541812427\Theta_{2,8}}{32078036992}+\frac{3001266891\Theta_{3,7}}{4009754624}-\frac{36356798079\Theta_{4,6}}{16039018496}+\frac{25650617139\Theta_{5,5}}{16039018496}\right).
\end{align}
\end{footnotesize}

We also note that each row vector $(\mathbf{R}_{00},\mathbf{R}_{10},\mathbf{R}_{11})$, and $(\mathbf{T}_{00},\mathbf{T}_{10},\mathbf{T}_{11})$ forms the same 3-dimensional representation $\rho$ of $\Gamma^0(2)$, following from the modularity of Jacobi-theta function. In particular, the representation matrix of $STS$ is given by
\begin{align}
\rho(STS)=\left(\begin{array}{ccc}
-1 & 0 & 0\\
0  & 0 & -1\\
0  & -1 & 0
\end{array}\right),
\end{align}
acting to these two row vectors. The factor $\mathbf{W}$ form a one-dimensional representation of $\Gamma^0(2)$. It would be interesting to further investigate the relation between the many ingredients in the above closed-form of $\mathcal{I}_{\mathcal{N} = 4 \ SO(7)}$ and the highest weight characters of the associated chiral algebra $\mathbb{V}_{\mathcal{N} = 4 \ SO(7)}$, which we leave for future work.
%To continue, we should know the simple poles distribution of the following elliptic function of $a_3$:
%\begin{align}
%\mathcal{P}_{\alpha\beta}\quad \mathcal{P}_{\alpha\beta\gamma}\quad \mathcal{Q}^{(k,\ell)}_{\alpha\beta}\quad \tilde{\mathcal{Q}}_{\alpha\beta\gamma}\quad \mathcal{Q}_{\alpha\beta\gamma}\quad \tilde{\mathcal{Q}}_{\alpha\beta}(\mathfrak{a}_2=0)
%\end{align}
%The elliptic function $\mathcal{P}_{\alpha\beta}$ has two poles:
%\begin{align}
%3\gamma\mathfrak{b}+\frac{\tau}{2}\quad \gamma=\pm 1
%\end{align}
%The residues are:
%\begin{align}
%\alpha\beta\gamma\frac{\vartheta_4(\mathfrak{b})\vartheta_4(3\mathfrak{b})\vartheta_4(5\mathfrak{b})}{\vartheta_1(2\mathfrak{b})\vartheta_1(4\mathfrak{b})\vartheta_1(6\mathfrak{b})}
%\end{align}
%For the function $\mathcal{P}_{\alpha\beta\gamma}$, the poles are:
%\begin{align}
%&\frac{k}{2}+\frac{\ell}{2}\tau\quad \text{where} (k+\ell\neq 0, k,\ell=0,1)\\
%&-\beta\gamma\mathfrak{b}+\frac{k}{2}+\frac{\ell}{2}\tau\quad \text{where}\quad k\geq \ell,\quad k,\ell=0,1\\
%&2\beta\gamma\mathfrak{b}
%\end{align}
%The residues are:
%\begin{align}
%&\alpha\beta b^{-2\beta\gamma \ell}\frac{\vartheta_4\left(\beta\gamma\mathfrak{b}\right)^3 \vartheta_1\left(\frac{k}{2}+\frac{\ell}{2}\tau\right)\vartheta_4\left(\beta\gamma\mathfrak{b}+\frac{k}{2}+\frac{\ell}{2}\tau\right)}{2\vartheta_1\left(2\beta\gamma\mathfrak{b}\right)^3\vartheta_1\left(2\beta\gamma\mathfrak{b}+\frac{k}{2}+\frac{\ell}{2}\tau\right)\vartheta_4\left(\frac{k}{2}+\frac{\ell}{2}\tau+3\beta\gamma\mathfrak{b}\right)}\\
%& -\alpha\beta b^{-2\beta\gamma\ell}\frac{\vartheta_4\left(\beta\gamma\mathfrak{b}\right)^3 \vartheta_1\left(\beta\gamma\mathfrak{b}+\frac{k}{2}+\frac{\ell}{2}\tau\right)\vartheta_4\left(\frac{k}{2}+\frac{\ell}{2}\tau\right)}{2\vartheta_1\left(2\beta\gamma\mathfrak{b}\right)^3\vartheta_1\left(3\beta\gamma\mathfrak{b}+\frac{k}{2}+\frac{\ell}{2}\tau\right)\vartheta_4\left(2\beta\gamma\mathfrak{b}+\frac{k}{2}+\frac{\ell}{2}\tau\right)}\\
%&\alpha\gamma\frac{\vartheta_4\left(\mathfrak{b}\right)\vartheta_4\left(3\mathfrak{b}\right)\vartheta_4\left(5\mathfrak{b}\right)}{\vartheta_1\left(2\mathfrak{b}\right)\vartheta_1\left(4\mathfrak{b}\right)\vartheta_1\left(6\mathfrak{b}\right)}
%\end{align}
%For $\mathcal{Q}^{(k,\ell)}_{\alpha\beta}$, it has the following poles:
%\begin{align}
%\gamma\mathfrak{b}+\frac{\tau}{2} \quad 2\gamma\mathfrak{b}+\frac{k}{2}+\frac{\ell}{2}\tau\quad \gamma=\pm 1 
%\end{align}
%The residues are listed below:
%\begin{align}
%&-\alpha\gamma b^{-2\ell}\frac{\vartheta_4\left(\mathfrak{b}\right)^3\vartheta_1\left(\frac{k}{2}+\frac{\ell}{2}\tau\right)\vartheta_4\left(\mathfrak{b}+\frac{k}{2}+\frac{\ell}{2}\tau\right)}{2\vartheta_1\left(2\mathfrak{b}\right)^3\vartheta_1\left(2\mathfrak{b}+\frac{k}{2}+\frac{\ell}{2}\tau\right)\vartheta_4\left(3\mathfrak{b}+\frac{k}{2}+\frac{\ell}{2}\tau\right)}\\
%&2\alpha\gamma\left(k-\frac{1}{2}\right)b^{-\ell}\frac{\vartheta_4\left(\mathfrak{b}\right)\vartheta_1\left(\frac{k}{2}+\frac{\ell}{2}\tau\right)\vartheta_1\left(2\mathfrak{b}+\frac{k}{2}+\frac{\ell}{2}\tau\right)}{2\vartheta_1\left(4\mathfrak{b}\right)\vartheta_4\left(\mathfrak{b}+\frac{k}{2}+\frac{\ell}{2}\tau\right)\vartheta_4\left(3\mathfrak{b}+\frac{k}{2}+\frac{\ell}{2}\tau\right)}
%\end{align}
%For $\tilde{\mathcal{Q}}_{\alpha\beta\gamma}$:
%\begin{align}
%2\delta\mathfrak{b}\quad \delta\mathfrak{b}+\frac{k}{2}+\frac{\ell}{2}\tau\quad \delta=\pm 1
%\end{align}
%The residues:
%\begin{align}
%&\alpha\gamma\delta\frac{\vartheta_4\left(\mathfrak{b}\right)\vartheta_4\left(3\mathfrak{b}\right)\vartheta_4\left(5\mathfrak{b}\right)}{\vartheta_1\left(2\mathfrak{b}\right)\vartheta_1\left(4\mathfrak{b}\right)\vartheta_1\left(6\mathfrak{b}\right)}
%\end{align}
%!TEX root = ../Schur indices and line operators.tex

\section{\texorpdfstring{Line operator index of $A_1$-theories of class-$\mathcal{S}$}{}\label{section:Wilson-index-A1-theories}}

In this and the following section we discuss the Schur index in the presence of a line operator. For a Lagrangian 4d $\mathcal{N} = 2$ SCFT, the Schur index in the absence of operator insersion can be computed by a multivariate contour integral \cite{Gadde:2011uv,Beem:2013sza}
\begin{align}
  \mathcal{I} = \oint \left[\frac{da}{2\pi i a}\right] \mathcal{Z}(a) \ ,
\end{align}
where the integrand $\mathcal{Z}(a)$ is elliptic with respect to the ``exponent variables'' $\mathfrak{a}_i$ separately, and captures contributions from the vector and hypermultiplets in a gauge theory description.

One can introduce half line operators in the 4d theory that extend from the origin to infinity while preserving certain amount of supercharges \cite{Cordova:2016uwk}. In particular, there are line operators that preserve the supercharges used to construct the Schur index. In the presence of such a BPS half Wilson line operator in the representation $\mathcal{R}$ of the gauge group, the half Wilson line index can be computed simply by\footnote{For simplicity we omit the normalization factor $\mathcal{I}^{-1}$.} \cite{Gang:2012yr,Cordova:2016uwk}
\begin{align}
	\langle W_{\mathcal{R}}\rangle = \oint \left[\frac{da}{2\pi i a}\right]
	\chi_\mathcal{R}(a) \mathcal{Z}(a) \ ,
\end{align}
where $\chi_\mathcal{R}(a)$ denotes the character of representation $\mathcal{R}$ of $G$. The Wilson index counts the local Schur operators (in the free limit) that are gauge-variant and can absorb the charge at the end of the half line. A full Wilson line operator in representation $\mathcal{R}$ can be thought of as a junction at the origin of two half Wilson line operators in complex-conjugating representation $\mathcal{R}, \overline{\mathcal{R}}$, and hence the full Wilson line index can be computed by
\begin{align}
  \langle W_{\mathcal{R}}^\text{full}\rangle = \oint \left[\frac{da}{2\pi i a}\right]
  \chi_\mathcal{R}(a)\chi_{\overline {\mathcal{R}}}(a) \mathcal{Z}(a) \ .
\end{align}
In our notation, we will only add the superscript ``full'' when dealing with a full Wilson line operator.

One can also consider correlators of half Wilson line operators, which take the form
\begin{align}
  \langle W_{\mathcal{R}_1} \cdots W_{\mathcal{R}_n}\rangle
  = \oint \bigg[\frac{da}{2\pi i a}\bigg]
  \bigg[\prod_{i=1}^n\chi_{\mathcal{R}_i}(a)\bigg]
  \mathcal{Z}(a)\ .
\end{align}
One can consider applying the tensor product decomposition $\otimes_{i = 1}^n \mathcal{R}_i = \sum_{j} m_j \mathcal{R}^{(j)}$ and reduce the product of characters on the right to a sum of characters of the irreducible representations $\mathcal{R}^{(j)}$ of the gauge group,
\begin{align}
  \langle W_{\mathcal{R}_1} \cdots W_{\mathcal{R}_n}\rangle = \sum_{j} m_j \langle W_{\mathcal{R}^{(j)}} \rangle \ .
\end{align}
In this sense, half Wilson line index in irreducible representations are the basic building blocks for correlators of half/full Wilson line, which will be our main focus.

In the following we will study line operator index for $A_1$ theories of class-$\mathcal{S}$. We will start with some simple examples where we are able to compute both the Wilson line index and the $S$-dual `t Hooft line index. Eventually we will analyze in detail half Wilson line index for general $A_1$ theories of class-$\mathcal{S}$. When possible, we also comment on the relation between the index and associated chiral algebra characters.


\subsection{\texorpdfstring{$\mathcal{N} = 4 $  $ SU(2)$ theory}{}\label{section:N4SU(2)}}

\subsubsection{Half Wilson line index}

The associated chiral algebra $\mathbb{V}_{\mathcal{N} = 4}$ of the $\mathcal{N} = 4$ theory with an $SU(2)$ gauge group is given by the 2d small $\mathcal{N} = 4$ superconformal algebra. The Schur index, which is identified with the vacuum character of $\mathbb{V}_{\mathcal{N} = 4}$, can be computed by the contour integral
\begin{align}
  \mathcal{I}_{\mathcal{N} = 4} 
  = & \ - \frac{1}{2}\frac{\eta(\tau)^3}{\vartheta_4(\mathfrak{b})}
  \oint_{|a| = 1} \frac{da}{2\pi i a} 
  \frac{
    \vartheta_1(2\mathfrak{a})\vartheta_1(- 2\mathfrak{a})
  }{
    \vartheta_4(2\mathfrak{a} + \mathfrak{b})
    \vartheta_4(-2\mathfrak{a} + \mathfrak{b})
  }
  \coloneqq \oint \frac{da}{2\pi i a} \mathcal{Z}(a)\\
  = & \ \frac{i\vartheta_4(\mathfrak{b})}{\vartheta_1(2 \mathfrak{b})} E_1 \begin{bmatrix}
    -1 \\ b  
  \end{bmatrix} \ . \nonumber
\end{align}
In the following we consider the index in the presence of a half Wilson line operator in the spin-$j$ representation. The index is then given by the integral
\begin{align}
	\langle W_j\rangle =
  \oint_{|a| = 1} \frac{da}{2\pi i a} \left[\sum_{m = - j}^{j} a^{2m}\right]
  \mathcal{Z}(a)\ .
\end{align}
Here the spin-$j$ character is given by $\chi_j(a) = \sum_{m = -j}^j a^{2m}$.

To proceed, we note that there are a collection of poles from the elliptic integrand,
\begin{align}
	\mathfrak{a}_{k\ell}^\pm = \pm \frac{\mathfrak{b}}{2} + \frac{(2k + 1)\tau}{4} + \frac{\ell}{2}, \qquad
  k, \ell = 0, 1 \ .
\end{align}
Due to the presence of $\tau/4$, all these poles are imaginary, with essentially the same residues
\begin{align}
  R^\pm_{k\ell} = \mp \frac{i}{4} \frac{\vartheta_4(\mathfrak{b})}{\vartheta_1(2 \mathfrak{b})} \ .
\end{align}
Applying the integral formula (\ref{integration-formula-monomial}), the index reads
\begin{align}
  \langle W_j\rangle = \mathcal{I}_{\mathcal{N} = 4}\delta_{j \in \mathbb{Z}} - \frac{i}{4} \frac{\vartheta_4(\mathfrak{b})}{\vartheta_1(2 \mathfrak{b})}\sum_{\substack{m = -j \\ m \ne 0}}^{j}\sum_{k, \ell = 0, 1} \frac{(-1)^{2\ell m} (b^m - b^{-m})q^{( - \frac{1}{2} + k )m}}{q^{m} - q^{-m}} \ .
\end{align}
Note that for $j \in \mathbb{Z}$, the character $\chi_j(a)$ contains a constant term $1$, which lead to the original Schur index $\mathcal{I}_{\mathcal{N} = 4}$. In fact, when $j \in \mathbb{Z} + \frac{1}{2}$, the entire expression vanishes identically thanks to the summation over $\ell = 0, 1$. Therefore, we have
\begin{align}
  \langle W_{j \in \mathbb{Z}}\rangle
  = + \mathcal{I}_{\mathcal{N} = 4} - \frac{i}{2} \frac{\vartheta_4(\mathfrak{b})}{\vartheta_1(2 \mathfrak{b})}\sum_{\substack{m = -j \\ m \ne 0}}^{j} \frac{b^m - b^{-m}}{q^{m/2} - q^{-m/2}} \ ,
  \qquad
  \langle W_{j \in \mathbb{Z} + \frac{1}{2}}\rangle = 0 \ .
\end{align}
The first term $\mathcal{I}_{\mathcal{N} = 4} = \operatorname{ch}_0$ is identified with the vacuum character of the associated chiral algebra $\mathbb{V}_{\mathcal{N} = 4}$. The factor $\frac{i\vartheta_4(\mathfrak{b})}{\vartheta_1(2 \mathfrak{b})}$ in the second term is the residue of the integrand $\mathcal{Z}$ which is related to the Schur index of Gukov-Witten type surface defect in the $\mathcal{N} = 4$ theory \cite{Pan:2021ulr}. It can be shown to satisfy $\frac{i\vartheta_4(\mathfrak{b})}{\vartheta_1(2 \mathfrak{b})} = \operatorname{ch}_0 + \operatorname{ch}_M$ where $M$ is another irreducible module $M$ of $\mathbb{V}_{\mathcal{N} = 4}$ \cite{Adamovic:2014lra,Bonetti:2018fqz,Pan:2021ulr}. As module characters of $\mathbb{V}_{\mathcal{N} = 4}$, both $\operatorname{ch}_0$ and $\operatorname{ch}_M$ satisfy the flaovred modular differential equations arising from null states in $\mathbb{V}_{\mathcal{N} = 4}$ \cite{Gaberdiel:2008pr,Gaberdiel:2009vs,Beem:2017ooy,Pan:2021ulr,Zheng:2022zkm}. Therefore, the line index can be written as a combination of the two irreducible characters,
\begin{align}
  \langle W_{j \in \mathbb{Z}}\rangle
  = \bigg(1 - \frac{1}{2}\sum_{\substack{m = -j \\ m \ne 0}}^{+j}\frac{b^m - b^{-m}}{q^{m/2} - q^{- m /2}}\bigg)\operatorname{ch}_0
  - \frac{1}{2}\Big(\sum_{\substack{m = -j \\ m \ne 0}}^{+j}\frac{b^m - b^{-m}}{q^{m/2} - q^{- m /2}}\Big) \operatorname{ch}_M \ .
\end{align}
Note however that the coefficients of the linear combination are rational functions of $b$ and $q$.





\subsubsection{`t Hooft line index}

In the 4d $\mathcal{N} = 4$ SYM (and in general $\mathcal{N} = 2$ superconformal gauge theories), one can define `t Hooft line operators by specifying certain singular profile for the gauge field and scalars in the path integral. By the Dirac quantization condition, the magnetic charge $B$ of a `t Hooft operator is valued in the cocharacter lattice $\Lambda_\text{cochar}$ inside the Cartan $\mathfrak{h}$ of the gauge group $G$. This lattice $\Lambda_\text{cochar}$ corresponds to the weights of the Langland dual group $G^\vee$, and therefore a dominant integral element $B$ corresponds to a $G^\vee$-representation $\mathcal{R}^\vee_B$. The cocharacters as weights in $\mathcal{R}^\vee_B$ are obtained from $B$ by subtracting suitable coroot element $\alpha^\vee$, and weights related by the Weyl group $W$ of the gauge group $G$ are identified. A weight $v$ in $\mathcal{R}^\vee_B$ that is not Weyl-related to $B$ can screen the `t Hooft operator and signals monopole bubbling effect \cite{Lee:1996vz,Gomis:2009ir,Ito:2011ea,Brennan:2018yuj}.

Under S-duality, a full Wilson line in a $\mathcal{N} = 4$ SYM is mapped to a `t Hooft line. If the magnetic charge of a 't Hooft operator corresponds to a minuscule representation of $G^\vee$, then its index is safe from monopole bubbling effect, and the index can be computed by a relatively simple contour integral \cite{Gang:2012yr}. In particular, For the $\mathcal{N} = 4$ $U(2)$ theory, the 't Hooft line with minimal magnetic charge $(1,0)$ corresponds to a minuscule representation, and is dual to the a full Wilson operator in the fundamental representation. The `t Hooft index can be written as a contour integral \cite{Gang:2012yr},
\begin{align}\label{U2-t-hooft}
  \langle H_{(1,0)}^\text{full} \rangle
  = - \oint \frac{da}{2\pi i a} \frac{(a - b)(-1 + a b)}{(\sqrt{q} - a)(-1 + \sqrt{q}a)b}
  \frac{\eta(\tau)^6 \vartheta_4(\mathfrak{a})^2}{
    \vartheta_1(\mathfrak{a} - \mathfrak{b})
    \vartheta_1(\mathfrak{a} + \mathfrak{b})
    \vartheta_4(\mathfrak{b})^2
  } \ .
\end{align}
Note that the parameters and integration variables have been renamed and reorganized compared to the double contour integral in \cite{Gang:2012yr}. In series expansion,
\begin{align}
  \langle H_{(1,0)}^\text{full}\rangle = 1 + 2(b + b^{-1})\sqrt{q}
  + (1 + 3b^2 + 3b^{-2}) q
  + 4(b^3 + b^{-3})q^{3/2} + \cdots \ .
\end{align}
The ratio of $\vartheta$ functions in $\langle H^\text{full}\rangle$ are essentially identical to the original integrand that computes $\mathcal{I}_{\mathcal{N} = 4}$, up to a shift from $\vartheta_{1, 4} \to \vartheta_{4,1}$. It is therefore elliptic in $\mathfrak{a}$, with real poles $\mathfrak{a} = \pm \mathfrak{b}$. The rational factor in the integrand can also be expanded in the $SU(2)$ characters,
\begin{align}
  - \frac{(a-b)(-1 + ab)}{(\sqrt{q} - a)(-1 + \sqrt{q}a)}
  = (1 + b^2) \sum_{n = 0}^{+\infty}q^{\frac{n}{2}}\chi_{j = \frac{n}{2}}(a)
  - b \sum_{n = 0}^{+\infty}q^{n/2}
  \chi_{j = \frac{1}{2}}(a)\chi_{j = \frac{n}{2}}(a) \\
  = (1 + b^2) \sum_{n = 0}^{+\infty}q^{n/2}\chi_{j = \frac{n}{2}}(a)
  - b \sum_{n = 0}^{+\infty}q^{n/2}
  \chi_{j = \frac{n}{2} + \frac{1}{2}}(a)
  - b \sum_{n = 0}^{+\infty}q^{n/2}
  \chi_{j = \frac{n}{2} - \frac{1}{2}}(a) \ . 
\end{align}
Therefore, the integral $\langle H^\text{full}\rangle$ can be computed directly and exactly using (\ref{integration-formula-χf}). In this case, the residues of two real poles $a = b^{\pm}$ are given by
\begin{align}
  R_\pm = \pm \frac{i \eta(\tau)^3}{\vartheta_1(2\mathfrak{b})} \ .
\end{align}
After some algebra, we have
\begin{align}
  \langle H_{(1,0)}^\text{full}\rangle
  = \frac{i \eta(\tau)^3}{\vartheta_1(2\mathfrak{b})}
  (q^{\frac{1}{2}} & \ + q^{-\frac{1}{2}} - b - b^{-1})
  \sum_{n = 0}^{+\infty}
  \sum_{\substack{m = - n/2 \\ m \ne 0}}^{+ n/2}
  q^{\frac{n}{2}} \frac{b^{2m} - b^{-2m}}{1 - q^{-2m}}
  \nonumber\\
  & \ + \frac{2(b + b^{-1} - 2q^{\frac{1}{2}})}{1-q} \frac{i \eta(\tau)^3}{\vartheta_1(2\mathfrak{b})} E_1 \begin{bmatrix}
    -1 \\ b  
  \end{bmatrix} \ ,
\end{align}
where in the second line we applied
\begin{align}
  \oint \frac{da}{2\pi i a}\frac{\eta(\tau)^6 \vartheta_4(\mathfrak{a})^2}{
    \vartheta_1(\mathfrak{a} - \mathfrak{b})
    \vartheta_1(\mathfrak{a} + \mathfrak{b})
    \vartheta_4(\mathfrak{b})^2
  } = \frac{2i \eta(\tau)^3}{\vartheta_1(2 \mathfrak{b})} E_1 \begin{bmatrix}
    -1 \\ b  
  \end{bmatrix} \ .
\end{align}

The dual Wilson operator index can be computed a lot more easily with (\ref{integration-formula-monomial}),
\begin{align}
  \langle W^\text{full}_{j = 1/2}\rangle
  = & \ - \frac{1}{2}\frac{\eta(\tau)^6}{\vartheta_4(\mathfrak{b})^2}
  \oint_{|a| = 1} \frac{da}{2\pi i a}
  (a + \frac{1}{a})^2 \frac{
    \vartheta_1(2\mathfrak{a})\vartheta_1(- 2\mathfrak{a})
  }{
    \vartheta_4(2\mathfrak{a} + \mathfrak{b})
    \vartheta_4(-2\mathfrak{a} + \mathfrak{b})
  } \\
  = & \ \langle W_{j = 1}\rangle_{U(2)} + \mathcal{I}_{\mathcal{N} = 4 \ U(2)}
  = q^{-\frac{1}{2}}
  \frac{i\eta(\tau)^3}{\vartheta_4(\mathfrak{b})}
  \frac{\vartheta_4(\mathfrak{b})}{\vartheta_1(2\mathfrak{b})}\left(
  2E_1\begin{bmatrix}
    -1 \\b  
  \end{bmatrix} -  \frac{b -b^{-1}}{q^{1/2} - q^{-1/2}}
  \right) \ . \nonumber
\end{align}
As required by S-duality, $\langle W^\text{full}_{j = 1/2}\rangle = \langle H_{(1,0)}^\text{full}\rangle$. This equality indeed follows analytically from the identity (\ref{E1-expansions}). Stripping off the $U(1)$ vector multiplet and the free hypermultiplet contribution $\eta(\tau)^3/\vartheta_4(\mathfrak{b})$, both the full Wilson index and the `t Hooft index are linear combinations of two $\mathcal{V}_{\mathcal{N} = 4}$ characters with rational coefficients, so schematically
\begin{align}
  \langle W_{j = 1}^\text{full} \rangle = A \operatorname{ch}_0 + B \operatorname{ch}_M, \qquad
  \langle H_{(1,0)}^\text{full}\rangle = C \operatorname{ch}_0 + D \operatorname{ch}_M \ .
\end{align}
However, the S-duality $\langle W^\text{full}_{j = 1/2}\rangle = \langle H_{(1,0)}^\text{full}\rangle$ is not because $A = C, B = D$; instead, the $S$-duality induces some highly nontrivial mixing between the vacuum and the $M$ module contributions.



Let us also consider `t-Hooft operators with non-minimal charge $B = (2,0)$. In this case, the index receives contribution from monopole bubbling with $v = (1,1)$, and is expected to equal the $U(2)$ Wilson index in the tensor product of fundamental representation. The `t Hooft index reads
\begin{align}
  \langle H^\text{full}_{(2,0)}\rangle
  = q^{-1/2} \oint \frac{da}{2\pi i a} \mathcal{Z}(a) \frac{\eta(\tau)^6}{\vartheta_4(\mathfrak{b})^2} \frac{\vartheta_1(\mathfrak{a})^2}{\vartheta_4(\pm \mathfrak{a} + \mathfrak{b})} \ ,
\end{align}
where
\begin{align}
  \mathcal{Z}(a)
  = \frac{(1 - \frac{\sqrt{q}}{ab}) (1 - \frac{a\sqrt{q}}{b})}{(1 - \frac{1}{a})(1 - a)}& \ \frac{(1 - \frac{b \sqrt{q}}{a})(1 - a b \sqrt{q})}{(1 - \frac{q}{a})(1 - aq)} \nonumber \\
  & \ + \frac{1}{2} \left[\frac{(q - 1)^2 + (b + \frac{1}{b})\sqrt{q}(1 + q) - 2q (a + \frac{1}{a}) }{(1 - \frac{q}{a})(1 - a q)}\right]^2 \ .
\end{align}
Note that
\begin{align}
  \frac{1}{(1 - \frac{q}{a})(1 - aq)} = \sum_{j \in \frac{1}{2}\mathbb{N}}q^{2j} \chi_j(a) \ ,\quad
  (1 - \frac{b^\pm \sqrt{q}}{a})(1 - a b^\pm \sqrt{q}) = (1 + b^{\pm2} q) - b^\pm q^{\frac{1}{2}} \chi_{\frac{1}{2}}(a) \ . \nonumber
\end{align}
Inserting these expansion, we have
\begin{align}
  \mathcal{Z}
  = & \ \frac{1}{(1-z)(1-1/z)} \left[A - B\chi_{1/2}(a) + q \chi_1(a)
  \right]\sum_{j\in \frac{1}{2}\mathbb{N}} q^{2j}\chi_{j}(a) \nonumber \\
  & \ + \Big[
  4q^2(1 + \chi_1 (a)) - C^2 - 2C q \chi_{\frac{1}{2}}(a)
  \Big]\sum_{j, j', j'' \in \frac{1}{2}\mathbb{N}} q^{2(j + j')}N_{j j'}^{j''} \chi_{j''}(a) \\
  \coloneqq & \ \frac{1}{(1-a)(1-1/a)} \sum_{j \in \frac{1}{2}\mathbb{N}} \mathcal{Z}_j \chi_j(a) + \sum_{j \in \frac{1}{2}\mathbb{N}}\mathcal{Z}_j'\chi_j(a),
\end{align}
where
\begin{align}
  A & \ \coloneqq (1 + b^2q)(1 + \frac{q}{b^2}) + q, 
  & B \coloneqq & \ (b + b^{-1})\sqrt{q}(1+q)\\
  C & \ \coloneqq (q-1)^2 + (b + \frac{1}{b}) \sqrt{q}(1 + q), 
  & \chi_{J}(a)\chi_{J'}(a) = & \ \sum_{J''}N_{JJ'}^{J''} \chi_{J''}(a) \ ,
\end{align}
and $\mathcal{Z}_j$, $\mathcal{Z}'_j$ are polynomials of $b, q$ from applying the tensor product rule for the $SU(2)$ characters,
\begin{align}
   \sum_{j \in \frac{1}{2}\mathbb{N}}\mathcal{Z}_j \chi_j(a)  = & \ [A - B \chi_{1/2}(a) + q \chi_1(a)] \sum_{j \in \frac{1}{2}\mathbb{N}}q^{2j} \chi_j(a)\\
   \sum_{j \in \frac{1}{2}\mathbb{N}}\mathcal{Z}'_j \chi_j(a) = & \ \Big[
     4q^2(1 + \chi_1 (a)) - C^2 - 2C q \chi_{\frac{1}{2}}(a)
     \Big]\sum_{j, j', j'' \in \frac{1}{2}\mathbb{N}} q^{2(j + j')}N_{j j'}^{j''} \chi_{j''}(a) \ ,
\end{align}
while their explicit expressions will be left implicit. Plugging this expansion into the integral, we have
\begin{align}
  \langle H^\text{full}_{(2,0)} \rangle
  = & \ \frac{i \eta(\tau)^3}{\vartheta_1(2 \mathfrak{b})}\sum_{j \in \frac{1}{2}\mathbb{N}} \mathcal{Z}_j \left(
  - \lfloor (j + \frac{1}{2})^2 \rfloor 2E_1 \begin{bmatrix}
    -1 \\ b  
  \end{bmatrix}
  + \sum_{m = -j}^{+j}\sum^{+\infty}_{\substack{k = 0 \\ k+2m \ne 0}}
  \frac{k(b^{k + 2m} - b^{-k -2m})}{q^{\frac{k}{2} + m} - q^{- \frac{k}{2} - m}}
  \right) \nonumber \\
  & \ + \frac{i \eta(\tau)^3}{\vartheta_1(2\mathfrak{b})}\sum_{j \in \frac{1}{2} \mathbb{N}} \mathcal{Z}'_j \left(
    \delta_{j \in \mathbb{Z}} 2E_1 \begin{bmatrix}
      -1 \\ b  
    \end{bmatrix}
    - \sum_{\substack{m = -j \\ m \ne 0}}^{+j} \frac{b^{2m} - b^{-2m}}{q^m - q^{-m}}
  \right) \ .
\end{align}
Unfortunately, we are unable to recast the expression to a more elegant form, therefore we do not prove $\langle W_{\mathbf{2} \otimes \mathbf{2}}^\text{full}\rangle_{U(2)} = \langle H_{(2,0)}^\text{full}\rangle$ analytically. Still, once the free contribution $\eta(\tau)^3/\vartheta_4(\mathfrak{b})$ is removed, the index $\langle H^\text{full}_{(2,0)}\rangle$ remains a linear combination of $\mathbb{V}_{\mathcal{N} = 4}$ characters.



\subsection{\texorpdfstring{$SU(2)$ theory with four flavors}{}}

Next we consider the $\mathcal{N} = 2$ $SU(2)$ gauge theory with four fundamental flavors. In terms of the class-$\mathcal{S}$ description, the theory is associated to the four puncture sphere $\Sigma_{0,4}$ and it admits three weak coupling limits corresponding to three different pants-decompositions. For any such limit, we can insert a half or full Wilson line operator of the $SU(2)$ gauge group in the spin-$j$ representation. The half Wilson index can be computed by the following integral,
% Figure environment removed
\begin{align}
  \langle W_j\rangle_{0,4} = - \frac{1}{2} \oint \frac{da}{2\pi i z} \left[\sum_{m = -j}^{j} a^{2m}\right] \frac{da}{2\pi i a}
  \vartheta_1(2\mathfrak{a}) \vartheta_1(-2\mathfrak{a})
  \prod_{j = 1}^{4} \frac{\eta(\tau)^2}{\vartheta_1(\mathfrak{a} + \mathfrak{m}_j)
  \vartheta_1(- \mathfrak{a} + \mathfrak{m}_j)} \ .
\end{align}
The poles of the integrand are all imaginary, given by $\mathfrak{a}_i^\pm = \pm \mathfrak{m}_i + \frac{\tau}{2}$ with residues
\begin{align}
  R_{i, \pm} = \pm \frac{i}{2} \frac{\vartheta_1(2 \mathfrak{m}_i)}{\eta(\tau)}
  \prod_{\ell \ne i} \frac{\eta(\tau)}{\vartheta_1(\mathfrak{m}_i + \mathfrak{m}_\ell) \vartheta_1(\mathfrak{m}_i - \mathfrak{m}_\ell)}
  \coloneqq \pm R_i
\end{align}
Applying the integration formula (\ref{integration-formula-monomial}), we have
\begin{align}\label{Wilson-index-SQCD}
  \langle W_j \rangle_{0,4} =  & \ \mathcal{I}_{0,4}\delta_{j \in \mathbb{Z}} - \sum_{\substack{m = - j \\ m \ne 0}}^{+ j} \sum_{\pm} \sum_{i = 1}^4 R_{i, \pm} \frac{1}{q^{2m} - 1} (b_i^\pm q^{\frac{1}{2}})^{2m} \nonumber\\
  = & \ \mathcal{I}_{0,4}\delta_{j \in \mathbb{Z}}
  - \sum_{i = 1}^{4} \left(\sum_{\substack{m = - j \\ m \ne 0}}^{+j} \frac{M_i^{2m} - M_i^{-2m}}{q^{m} - q^{-m}}\right)R_i \ , 
\end{align}
where $M_i \coloneqq e^{2\pi i \mathfrak{m}_i}$. The theory is of class-$\mathcal{S}$ associated to the four-punctured sphere. The $SU(2)^4$ fugacities $b_i$ are related to the $m_i$ by
\begin{align}
  M_1 = b_1 b_2, \quad
  M_2 = b_1/b_2, \quad
  M_3 = b_3 b_4, \quad
  M_4 = b_3/b_4 \ .
\end{align}

In \cite{Cordova:2016uwk}, several Wilson line index in $SU(2)$ SQCD were computed, and the results can be organized as linear combinations of the infinitely many highest weight characters $\chi_{[m, n, 0,0,0]}$ of $\widehat{\mathfrak{so}}(8)_{-2}$ which were obtained from the Kazhdan-Lusztig formula \cite{Lusztig1979}. Our new computation improves the result and relates all $\langle W_j\rangle_{0,4}$ to just five highest weight characters, with respect to finite weights $\lambda = 0, -2 \omega_1, - \omega_2, - 2 \omega_3, -2 \omega_4$, of the simple vertex operator algebra $\widehat{\mathfrak{so}}(8)_{-2}$ \cite{Arakawa:2015jya,Arakawa:2016hkg}. Indeed, the four residues $R_i$ in the above are related to the Schur index of Gukov-Witten type surface defects, and also to the the module characters \cite{Peelaers,Pan:2021mrw,2023arXiv230409681L,Pan:2023jjw,Arai:2020qaj},
\begin{align}
  \operatorname{ch}_{-2\widehat \omega_1} = & \ \operatorname{ch}_0 - 2R_1\\
  \operatorname{ch}_{-\widehat \omega_2} = & \ -2 \operatorname{ch}_0 + 2R_1 + 2R_2\\
  \operatorname{ch}_{-2\widehat \omega_3} = & \ \operatorname{ch}_0 - R_1 - R_2 - R_3 - R_4\\
  \operatorname{ch}_{-2\widehat \omega_4} = & \ \operatorname{ch}_0 - R_1 - R_2 - R_3 + R_4 \ ,
\end{align}
where $\operatorname{ch}_0$ is the vacuum character of $\widehat{\mathfrak{so}}(8)_{-2}$, identified with the Schur index $\mathcal{I}_{0,4}$. Therefore, one may write the half Wilson line index as a linear combination of the five module characters,
\begin{align}
  \langle W_j\rangle_{0,4}
  = (\delta_{j \in \mathbb{Z}} - \frac{1}{2}\mathcal{M}_{1j} - \frac{1}{2}\mathcal{M}_{2j})\operatorname{ch}_0
  + & \ \frac{1}{2}(\mathcal{M}_{1j} - \mathcal{M}_{2j})\operatorname{ch}_1
  + \frac{1}{2}(\mathcal{M}_{3j} - \mathcal{M}_{2j})\operatorname{ch}_2 \nonumber\\
  & \ + \frac{1}{2}(\mathcal{M}_{3j} + \mathcal{M}_{4j})\operatorname{ch}_3
  + \frac{1}{2}(\mathcal{M}_{3j} - \mathcal{M}_{4j})\operatorname{ch}_4 \ , \nonumber
\end{align}
where we define the rational functions
\begin{align}
  \mathcal{M}_{ij} \coloneqq \sum_{\substack{m = - j\\m \ne 0}}^{+j}\frac{M_i^{2m} - M_i^{-2m}}{q^m - q^{-m}} \ .
\end{align}



With the half-Wilson index, the index of a full Wilson line operator in the fundamental representation is then given by
\begin{align}
  \langle W_{j = \frac{1}{2}}^\text{full}\rangle_{0,4} = \langle W_{j = \frac{1}{2}} W_{j = \frac{1}{2}}\rangle_{0,4}
  = \mathcal{I}_{0,4} + \langle W_{j = 1}\rangle_{0,4} \ .
\end{align}
By S-duality, this Wilson operator is mapped to the `t Hooft operator with a minimal magnetic charge $B = (-1, 1)$ which receives contribution from monopole bubbling \cite{Gang:2012yr}. The `t Hooft index is given by a slightly more involved contour integral,
\begin{align}
  \langle H_{1,-1}\rangle_{0,4}
  = & \ \oint \frac{da}{2\pi i a} \frac{2q^{\frac{5}{12}}\prod_{i =1}^{4}(a - M_i)(-1 + aM_i)}{(-1 + a^2)^2 (a^2 - q)(-1 + a^2 q) \prod_{i = 1}^{4}M_i}
  \left(- \frac{1}{2}\vartheta_1(\pm 2 \mathfrak{a})\right) \prod_{i = 1}^{4}\frac{\eta(\tau)^2}{\vartheta_1(\pm \mathfrak{a} + M_i)} \nonumber \\
  & \ + q^{-\frac{7}{12}} \oint \frac{da}{2\pi i a} Z_\text{mono}\left(- \frac{1}{2}\vartheta_1(\pm 2 \mathfrak{a})\right) \prod_{i = 1}^{4}\frac{\eta(\tau)^2}{\vartheta_4(\pm \mathfrak{a} + M_i)} \ ,
\end{align}
where
\begin{align}
  Z_\text{mono} = \frac{1}{q \prod_{i =1}^{4}M_i}\left[
  - \left( q + \prod_{i = 1}^{4}M_i\right)
  + \sum_{\pm}\frac{\prod_{i = 1}^{4}(q^{\frac{1}{2}}a^\pm - M_i)}{(1 - a^{\pm 2}) (1 - q a^{\pm 2})}
  \right]^2 \ .
\end{align}
We can rewrite
\begin{align}
  & \ \frac{2q^{\frac{5}{12}}\prod_{i =1}^{4}(a - M_i)(-1 + aM_i)}{(-1 + a^2)^2 (a^2 - q)(-1 + a^2 q) \prod_{i = 1}^{4}M_i} \nonumber \\
  = & \ \frac{2q^{\frac{5}{12}}}{(1-a^2)(1-a^{- 2})}
  \left[\sum_{J \in \mathbb{N}}q^{J}\sum_{j = 0}^{J}(-1)^j\chi_{J - j}(a)\right]
  \prod_{i =1}^{4} \Big(  \chi_{1/2}(a) - \chi_{1/2}(M_i)  \Big) \nonumber\\
  \coloneqq & \ \frac{2q^{\frac{5}{12}}}{(1-a^2)(1-a^{- 2})} \sum_{J \in \frac{1}{2}\mathbb{N}} \mathcal{Z}_J\chi_J(a) \ ,  \\
  Z_\text{mono} = & \ \frac{1}{q \prod_{i = 1}^{4}M_i}
  \left[
  \sum_{j \in \frac{1}{2}\mathbb{N}}(-1)^{2j + 1}\chi_j(a) g_j(M) q^{1 + j}
  \right]^2 \coloneqq \sum_{J \in\frac{1}{2} \mathbb{N}} \mathcal{Z}'_J \chi_J(a) \ .
\end{align}
where
\begin{align}
  g_{J \in \mathbb{N}}(M) \coloneqq 1 + \sum_{\substack{i, j = 1 \\ i < j}}^4M_i M_j + \prod_{i = 1}^{4}M_i \ ,\quad
  g_{J \in \mathbb{N} + \frac{1}{2}}(M) \coloneqq \sum_{i = 1}^{4}M_i + \sum_{\substack{i,j,k = 1\\i < j < k}}^{4}M_i M_j M_k \ ,
\end{align}
and $\mathcal{Z}_J$ and $\mathcal{Z}'_J$ are rational function of $q$ and fugacities $M$ which simply follow from expanding tensor product of $SU(2)$ irreps; their explicit form will be left implicit. Therefore, we have the exact formula for the `t-Hooft index,
\begin{align}
  \langle H_{1, -1}\rangle_{0,4}
  = & \ \sum_{J \in \frac{1}{2}\mathbb{N}}\mathcal{Z}_J
  \sum_{m = -J}^{+J}\sum_{\substack{k = 1 \\ 2k + 2m \ne 0}}^{+\infty}\left[\sum_{i, \pm}R_{i} \frac{k(M_i^{2k + 2m} - M_i^{ - 2k - 2m})}{q^{\frac{2k + 2m}{2}} - q^{- \frac{2k + 2m}{2}}}
      + \frac{2m}{2}\delta_{\frac{2m}{2} \in \mathbb{Z}_{< 0}} \mathcal{I}_{0,4}\right] \nonumber \\
  & \ + \sum_{J \in \frac{1}{2}  \mathbb{N}} \mathcal{Z}'_J\sum_{m = -J}^{+J} 
  \sum_{i = 1}^4 R_{i} \frac{M_i^{2m} - M_i^{-2m}}{q^m - q^{-m}} \ .
\end{align}
Unfortunately we are unable to reorganize the expression into a more elegant form. Therefore we do not further compare analytically between this `t-Hooft index with the corresponding Wilson index. Although fairly complicated, the expression $\langle H_{1, -1}\rangle_{0,4}$ remain explicitly a linear combination of $\widehat{\mathfrak{so}}(8)_{-2}$ characters, with rational functions in $b_i, q$ as the coefficients.




\subsection{Genus-one theory with two punctures \label{section:genus-one-two-punctures}}

Let us consider a higher rank theory with $g = 1$ and $n = 2$, which can be obtained by gauging a diagonal $SU(2) \times SU(2)$ subgroup of the flavor symmetry of two copies of trinion theories $\mathcal{T}_{0,3}$. There are essentially two different weak-coupling frames one can consider, and here we focus on the frame illustrated in Figure \ref{fig:genus-one-type-1}. In this frame, the original Schur index is given as a contour integral
\begin{align}
  \mathcal{I}_{1,2} 
  = \oint \prod_{i = 1}^{2}\frac{da_i}{2\pi i a_i}
  \prod_{j = 1}^{2}\prod_{\pm \pm} \frac{\eta(\tau)}{\vartheta_4(\mathfrak{b}_j \pm \mathfrak{a}_1 \pm \mathfrak{a}_2)}
  \prod_{i = 1}^{2}\left(- \frac{1}{2}\vartheta(\pm 2 \mathfrak{a}_i)\right)
  \coloneqq \oint \left[\frac{da}{2\pi i a}\right]\mathcal{Z}_{1,2}(a) \ .
\end{align}
Let us consider a half Wilson line operator associated to one of the $SU(2)$ gauge group, whose index is given by the integral
\begin{align}
  \langle W_j\rangle_{1, 2}^{(1)} = \oint \prod_{i = 1}^{2}\frac{da_i}{2\pi i a_i}
  \left(\sum_{m = - j}^{j}a_1^{2m}\right)
  \prod_{j = 1}^{2}\prod_{\pm \pm} \frac{\eta(\tau)}{\vartheta_4(\mathfrak{b}_j \pm \mathfrak{a}_1 \pm \mathfrak{a}_2)}
  \prod_{i = 1}^{2}\left(- \frac{1}{2}\vartheta(\pm 2 \mathfrak{a}_i)\right)\ .
\end{align}

% Figure environment removed

The integral can be evaluated in two different orders: first $a_1$  or first $a_2$. We choose to integrate over $a_1$ first, where the relevant poles are $ \mathfrak{a}_1 = \alpha \mathfrak{b}_j + \beta \mathfrak{a}_2 + \frac{\tau}{2}$ with residues (where $\alpha, \beta = \pm 1$)
\begin{align}
  R_{i \alpha \beta} = \frac{
    i \eta(\tau)^5 \vartheta_1(2 \beta \mathfrak{a}_2) \vartheta_1( 2 \beta \mathfrak{a}_2 + 2 \alpha \mathfrak{b}_i)
  }
  {
  4\vartheta_1(2 \alpha \mathfrak{b}_i)
  \vartheta_1(\alpha \mathfrak{b}_i - \beta \mathfrak{b}_{3-i})
  \vartheta_1(\alpha \mathfrak{b}_i + \beta \mathfrak{b}_{3-i})
  \vartheta_1(2 \mathfrak{a}_2 + \alpha \beta \mathfrak{b}_i - \mathfrak{b}_{3-i})
  \vartheta_1(2 \mathfrak{a}_2 + \alpha \beta \mathfrak{b}_i + \mathfrak{b}_{3-i}) 
  } \ , \nonumber
\end{align}
The $a_1$ integral leaves integrals of the form
\begin{align}
  \oint \frac{da_2}{2\pi i a_2} f(\mathfrak{a}_2) a_2^{n} \ ,
\end{align}
which can be carried out using formula \eqref{integration-formula-monomial}. Finally, the index in the presence of the Wilson line operator gives
\begin{align}
  \langle W_{j \in \mathbb{Z}}\rangle_{1, 2}
  = \mathcal{I}_{1,2}
     + \frac{\eta(\tau)^2}{2 \prod_{i = 1}^2 \vartheta_1(2\mathfrak{b}_i)}
     \sum_{\substack{m = - j \\ m \ne 0}}^{+j}
     \frac{\prod_{i = 1}^{2}(b_i^m - b_i^{-m})}{(q^{m/2} - q^{-m/2})^2}
     \ ,
  \quad
  \langle W_{j \in \mathbb{Z} + \frac{1}{2}}\rangle = 0 \ .
\end{align}
The result is symmetric in $b_1, b_2$ as expected. Note that the first term is clearly the vacuum character of the associated chiral algebra of $\mathcal{T}[\Sigma_{1,2}]$. The factor $\eta(\tau^2)/\prod_{i = 1}^2 \vartheta_1(2 \mathfrak{b}_i)$ arises as the unique\footnote{One can try different nested residues, but they are either zero or proportional to $\eta(\tau^2)/\prod_{i = 1}^2 \vartheta_1(2 \mathfrak{b}_i)$. } nested residue of $\mathcal{Z}_{1,2}(a)$,
\begin{align}
  \operatorname{Res}_{\mathfrak{a}_2 = - \frac{\mathfrak{b}_1 - \mathfrak{b}_2}{2}}\operatorname{Res}_{\mathfrak{a}_1 = \mathfrak{a}_2 + \mathfrak{b}_1 + \frac{\tau}{2}} \mathcal{Z}_{1,2}(\mathfrak{a}_{1,2}) = \frac{\eta(\tau)^2}{8 \vartheta_1(2 \mathfrak{b}_1)\vartheta_1(2 \mathfrak{b}_2)} \ ,
\end{align}
and is also expected to be a linear combination of non-vacuum module character, since it has been shown to satisfy a set of flavored modular differential equations that should annihilate all module characters \cite{zhu1996modular,Zheng:2022zkm}. For example, at weight-two there are two equations
\begin{align}
  0 = \Bigg[
  D_q^{(1)}
  - \frac{1}{4} \sum_{i = 1,2} D_{b_i}^2
  -\frac{1}{4}& \ \sum_{\alpha_i = \pm} E_1 \begin{bmatrix}
    1 \\ b_1^{\alpha_1}b_2^{\alpha_2}
  \end{bmatrix}
  \sum_{i = 1,2}\alpha_i D_{b_i}
  - \sum_{i = 1,2} E_1 \begin{bmatrix}
    1 \\ b_i^2
  \end{bmatrix}D_{b_i} \\
  & \ + 2 \bigg(
  E_2 + \frac{1}{2} \sum_{\alpha_i = \pm}E_2 \begin{bmatrix}
    1 \\ b_1^{\alpha_1}b_2^{\alpha_2}
  \end{bmatrix}
  + \sum_{i = 1,2} E_2 \begin{bmatrix}
    1 \\ b_i^2
  \end{bmatrix}
  \bigg) \Bigg] \mathcal{I}_{1,2} \ ,
\end{align}
and
\begin{align}
  0 = \left(D_{b_1}^2 + 4 E_1 \begin{bmatrix}
    1 \\ b_1^2 
  \end{bmatrix}
  - 8 E_2 \begin{bmatrix}
    1 \\ b_1^2
  \end{bmatrix}\right) \mathcal{I}_{1,2}
  = \left(D_{b_2}^2 + 4 E_1 \begin{bmatrix}
    1 \\ b_2^2 
  \end{bmatrix}
  - 8 E_2 \begin{bmatrix}
    1 \\ b_2^2
  \end{bmatrix}\right)\mathcal{I}_{1,2} \ .
\end{align}








\subsection{\texorpdfstring{Type-1 half Wilson line index in $\mathcal{T}[\Sigma_{g,n}]$}{}}


% Figure environment removed

Now we are ready to consider more general type $A_1$ class-$\mathcal{S}$ theories $\mathcal{T}[\Sigma_{g,n}]$. Any such theory usually admits several weak-coupling limits as different supersymmetric gauge theories. With respect to each gauge theory description, we can introduce a half Wilson operator associated to one of the $SU(2)$ gauge group. In general one can introduce Wilson line charged under multiple $SU(2)$ gauge groups in the weak-coupling description, however, we leave the study of their index and correlation functions to future work.

Let us build on top of the previous $\langle W_j\rangle_{1,2}$ by extending the corresponding Riemann surface to the left and right, while maintaining the location of the Wilson line operator. We simply refer to such construction of Wilson line operator as type-$1$. The resulting configuration is shown in Figure \ref{Wilson-loop-type-1}, and it is clear from the figure that type-1 Wilson line operator encircles a tube that when cut the Riemann surface $\Sigma_{g, n}$ remain connected. Put differently, the type-1 Wilson operator can be constructed from a single connected Riemann surface $\Sigma_{g, n + 2}$ where one glues two punctures and simultaneous inserts a Wilson operator at the tube. In this subsection we will prove that the index os type-1 Wilson line operator in the spin-$j$ representation is given by
\begin{align}\label{Wilson-index-1-general}
  \langle W_{j \in \mathbb{Z}}\rangle^{(1)}_{g \ge 1, n}
  = & \ \mathcal{I}_{g,n}
  - \frac{1}{2}\left[
    \prod_{i = 1}^{n} \frac{i \eta(\tau)}{\vartheta_1(2 \mathfrak{b}_i)}
  \right]
    \sum_{\substack{m = - j\\m \ne 0}}^{+ j}
    \left[\frac{\eta(\tau)}{q^{m/2} - q^{-m /2}}\right]^{2g - 2}
    \prod_{i = 1}^{n} \frac{b_i^m - b_i^{-m}}{q^{m/2} - q^{- m /2}} \ ,\\
  \langle W_{j \in \mathbb{Z} + \frac{1}{2}}\rangle^{(1)}_{g \ge 1, n} = & \ 0 \ .
\end{align}
Although in any given gauge theory description of $\mathcal{T}[\Sigma_{g,n}]$ there may be different choices of $SU(2)$ gauge groups to support a half Wilson line, the final index is actually independent of the choice, as long as they are all type-1. Also we emphasize that type-1 Wilson line exists only for genus $g \ge 1$.

The factor $\eta(\tau)^{2g - 2}\prod_{i = 1}^n \frac{\eta(\tau)}{\vartheta_1(2 \mathfrak{b}_i)}$ can be shown to be the unique\footnote{Up to some numerical factors.} nested residue of the integrand $\mathcal{Z}_{g,n}$ that computes the original Schur index. The uniqueness is only true for $g \ge 1$, as we have already encountered four different residues $R_i$ in the $\mathcal{T}[\Sigma_{0,4}]$ computation; in this sense, class-$\mathcal{S}$ theories at $g \ge 1$ seem to enjoy some nicer properties than the $g = 0$ counterparts \footnote{See also \cite{Satoshi:2023}, where Landau-Ginzburg description can be found for $g \ge 1$ $\mathcal{N} = (0,2)$ and $(0,4)$ class-$\mathcal{S}$ theories in two dimensions. It might suggest some subtle difference in the representation theory of associated chiral algebras of the $g = 0$ and $g \ge 1$ cases. It will be interesting to clarify this issue in the future.}. Extrapolating from the discussions in \cite{Zheng:2022zkm,Pan:2021ulr}, it is natural to expect that this factor is a solution to the set of flavored modular differential equations that annihilate the Schur index, namely, the vacuum character of the associated chiral algebra $\chi(\mathcal{T}[\Sigma_{g,n}])$ of $\mathcal{T}[\Sigma_{g,n}]$, and therefore a linear combination (with constant coefficients) of non-vacuum module characters. This implies that the Wilson line index is also a linear combination of $\chi(\mathcal{T}[\Sigma_{g,n}])$ characters, with rational coefficients
\begin{align}
  \sum_{\substack{m = - j\\m \ne 0}}^{+ j}
  \left[\frac{1}{q^{m/2} - q^{-m /2}}\right]^{2g - 2}
  \prod_{j = 1}^{n} \frac{b_j^m - b_j^{-m}}{q^{m/2} - q^{- m /2}}\  .
\end{align}
The closed-form expression is essentially a sum of products of contributions from the punctures and a contribution and a ``three point function'' contribution, which closely resembles that of the $q$-deformed Yang-Mills partition function on $\Sigma_{g, n}$. It would be interesting to match our result in detail with results by punctured network \cite{Watanabe:2016bwr,Watanabe:2017bmi}.


The proof of the index formula (\ref{Wilson-index-1-general}) can be done recursively by assuming at $g \ge 1, n\ge 0$ $\langle W_j\rangle^{(1)}_{g,n}$ is given by the anzatz (\ref{Wilson-index-1-general}). We already know that the above anzatz works for the $g = 1, n = 2$ case. We can compute $\langle W_j\rangle_{g, n + 1}^{(1)}$ by gluing $\mathcal{T}[\Sigma_{0,3}]$ to that associated to $\mathcal{T}[\Sigma_{g, n}]$,
\begin{align}
  \langle W_j \rangle_{g, n + 1}^{(1)} = \oint \frac{da}{2\pi i a} \langle W_j\rangle^{(1)}_{g, n}(\mathfrak{b}_1, \ldots, \mathfrak{b}_{n - 2}, \mathfrak{a}) \mathcal{I}_\text{VM}(\mathfrak{a}) \mathcal{I}_{0,3}( - \mathfrak{a}, \mathfrak{b}_{n - 1}, \mathfrak{b}_n) \ .
\end{align}
Let us compute
\begin{align}
  & \ \langle W_{j \in \mathbb{Z}}\rangle_{g, n + 1} \nonumber\\
  = & \ \oint \frac{da}{2\pi i a}\bigg[\mathcal{I}_{g, n}(\mathfrak{b}_1, \ldots, \mathfrak{b}_{n - 1}, \mathfrak{a}) \mathcal{I}_\text{VM}(\mathfrak{a}) \mathcal{I}_{0,3}(-\mathfrak{a}, \mathfrak{b}_n, \mathfrak{b}_{n + 1}) \nonumber \\
  & \ - \frac{i^n \eta(\tau)^n}{2
    \vartheta_1(2 \mathfrak{a})\prod_{j = 1}^{n - 1} \vartheta_1(2 \mathfrak{b}_j)
  }
    \\
  & \ \qquad \times \sum_{\substack{m = - j \\ m \ne 0}}^{+j}
  \left[\frac{\eta(\tau)}{q^{m/2} - q^{-m/2}}\right]^{2g - 2}
  \frac{\prod_{j = 1}^{n - 1}(b_j^m - b_j^{-m})}{(q^{m/2} - q^{-m/2})^n}
  (a^m - a^{-m})\mathcal{I}_\text{VM}(\mathfrak{a}) \mathcal{I}_{0,3}(-\mathfrak{a}, \mathfrak{b}_n, \mathfrak{b}_{n + 1})  \bigg]\ . \nonumber
\end{align}
The first term clearly gives $\mathcal{I}_{g, n + 1}$. The second integral is of the form (up to irrelevant factors pulled out of the integral)
\begin{align}
  \oint \frac{da}{2\pi i a} \frac{a^m - a^{-m}}{\vartheta_1(2\mathfrak{a})}
  \mathcal{I}_\text{VM}(a)\mathcal{I}_{0,3}(-\mathfrak{a}, \mathfrak{b}_n, \mathfrak{b}_{n + 1})\ .
\end{align}
It is easy to check that 
\begin{align}
  \frac{\mathcal{I}_\text{VM}(a)}{\vartheta_1(2\mathfrak{a})}\mathcal{I}_{0,3}(- \mathfrak{a}, \mathfrak{b}_n, \mathfrak{b}_{n + 1})
\end{align}
is elliptic in $\mathfrak{a}$. Therefore \eqref{integration-formula-monomial} implies that
\begin{align}
  & \ \oint \frac{da}{2\pi i a}(a^m - a^{-m})\frac{\mathcal{I}_\text{VM}(a)}{\vartheta_1(2\mathfrak{a})}\mathcal{I}_{0,3}(- \mathfrak{a}, \mathfrak{b}_1, \mathfrak{b}_2)
  =
  \frac{i \eta(\tau)}{\prod_{j = 1}^2\vartheta_1(2 \mathfrak{b}_j)} \frac{\prod_{j = 1}^{2}(b_j^m - b_j^{-m})}{(q^{m/2} - q^{-m/2})} \ .
\end{align}
In other words we have verified $\langle W_j\rangle_{g, n + 1}^{(1)}$ also satisfies (\ref{Wilson-index-1-general}),
\begin{align}
  \langle W_{j \in \mathbb{Z}}\rangle _{g, n + 1}
  = \mathcal{I}_{g, n + 1}
    - \frac{i^{n + 1}\eta(\tau)^{n + 1}}{2\prod_{j = 1}^{n + 1}\vartheta_1(2 \mathfrak{b}_j)} \sum_{\substack{m = -j\\m \ne 0}}^{+j}
    \left[\frac{\eta(\tau)}{q^{m/2} - q^{-m/2}}\right]^{2g - 2}
    \frac{\prod_{j = 1}^{n+1}(b_j^m - b_j^{-m})}{(q^{m/2} - q^{-m/2})^{n + 1}} \ .
\end{align}


In the direction of increasing genus $g$, one can glue pairs of punctures to obtain Wilson line operator index $\langle W_j\rangle_{g + 1, n}^{(1)}$ for theories of higher genus $g + 1$,
\begin{align}
  \langle W_j \rangle^{(1)}_{g + 1, n} = \oint \frac{da}{2\pi i a} \mathcal{I}_\text{VM}(a) \langle W_j \rangle^{(1)}_{g, n + 2}(\mathfrak{b}_1, \ldots, \mathfrak{b}_n, \mathfrak{a}, - \mathfrak{a}) \ .
\end{align}
Assuming the anzatz hods at genus $g$, we have
\begin{align}
  & \ \langle W_j\rangle^{(1)}_{g + 1, n} \nonumber \\
   = & \ \mathcal{I}_{g + 1, n}
   - \oint \frac{da}{2\pi i a} \frac{1}{2} \frac{i^n \eta(\tau)^n}{\prod_{j = 1}^{n}\vartheta_1(2\mathfrak{b}_j)}
   \frac{i^2 \eta(\tau)^2}{\vartheta_1(\pm 2 \mathfrak{a})}\left(-\frac{1}{2} \vartheta_1(\pm 2 \mathfrak{a})\right)\\
   & \ \qquad \qquad\qquad\times \sum_{\substack{m = - j\\ m \ne 0}}^{+ j}
   \left[\frac{\eta(\tau)}{q^{m/2} - q^{-m /2}}\right]^{2g - 2}
   \frac{(a^{m} - a^{-m})(a^{- m} - a^{+m})}{(q^{m/2} - q^{-m/2})^2}
   \prod_{j = 1}^{n}\frac{b_j^{m} - b_j^{-m}}{q^{m/2} - q^{-m/2}} \ .\nonumber
\end{align}
The two $\vartheta_1(\pm 2 \mathfrak{a})$ factors are cancelled, while
\begin{align}
  {(a^{m} - a^{-m})(a^{- m} - a^{+m})} = - a^{2m} - a^{-2m} + 2 \ .
\end{align}
Only the $+2$ survives the $a$-integration since $m \ne 0$. Hence,
\begin{align}
  \langle W_j\rangle^{(1)}_{g + 1, n}
  = \mathcal{I}_{g + 1, n}
  - \frac{1}{2}\prod_{j = 1}^{n}\frac{i \eta(\tau)}{\vartheta_1(2 \mathfrak{b}_j)}
    \sum_{\substack{m = - j\\ m \ne 0}}^{+ j}
     \left[\frac{\eta(\tau)}{q^{m/2} - q^{- m /2}}\right]^{2(g+1) - 2}
     \prod_{j = 1}^{n}\frac{b_j^{m} - b_j^{-m}}{q^{m/2} - q^{-m/2}} \ ,
\end{align}
proving the index formula (\ref{Wilson-index-1-general}).

\vspace{2em}

The Type-1 Wilson index $\langle W_j\rangle_{g \ge 1,n}^{(1)}$ can be computed in a different approach, by gluing two existing punctures and simultaneously insert a half Wilson operator,
\begin{align}
  \langle W_j\rangle_{g \ge 1, n}
  = \oint \frac{da}{2\pi i a} \chi_j(a) \mathcal{I}_{g - 1, n + 2}(\mathfrak{b}_1, \ldots, \mathfrak{b}_n, \mathfrak{a}, - \mathfrak{a}) \mathcal{I}_\text{VM}(\mathfrak{a}) \ .
\end{align}
Recall that for $g \ge 0, n > 0$, the $A_1$ Schur index is given by
\begin{align}
  \mathcal{I}_{g, n} = & \ \frac{i^n}{2} \frac{\eta(\tau)^{n + 2g - 2}}{\prod_{j = 1}^{n}\vartheta_1(2 \mathfrak{b}_j)}
  \sum_{\vec\alpha = \pm}\Big(  \prod_{j = 1}^{n}\alpha_j  \Big)\sum_{k = 1}^{n + 2g - 2}\lambda_k^{(n + 2g - 2)} E_k\left[\begin{matrix}
    (-1)^n \\ \prod_{j = 1}^{n}b_j^{\alpha_j}
  \end{matrix}\right] \ .
\end{align}
After identifying $\mathfrak{b}_{n + 1} = \mathfrak{a}$, $\mathfrak{b}_{n + 2} = - \mathfrak{a}$ and multiplying the vector multiplet contribution $\mathcal{I}_\text{VM}(\mathfrak{a})$, all the $\vartheta_1(2\mathfrak{a})$ factors cancel out, and the integration variable $a$ is only present inside the Eisenstein series. When $j \in \mathbb{Z}$, the constant term in $\chi_j(a)$ leads to a additive term $\mathcal{I}_{g, n}$. For the terms in $\chi_j(a)$ with non-zero $m$, we proceed with the integration,
\begin{align}
  \oint \frac{da}{2\pi i a} a^{2m} \frac{i^{n + 2}}{2} \frac{\eta(\tau)^{2g - 2 + n}}{\prod_{j = 1}^{n}\vartheta_1(2\mathfrak{b}_j)}
  \sum_{\vec \alpha = \pm 1} \left(\prod_{j = 1}^{n + 2}\alpha_j\right)
  \sum_{k = 1}^{2g - 2 + n}
  \lambda_k^{(2g - 2 + n)}
  E_k \begin{bmatrix}
    (-1)^n \\
    \prod_{j = 1}^{n + 2}b_j^{\alpha_j}
  \end{bmatrix}_{\substack{b_{n + 1} = a\\b_{n + 2} = 1/a}} \ .
\end{align}
Only the terms with $\alpha_{n + 1} = - \alpha_{n + 2} \coloneqq \beta$, such that $b_{n + 1}^{\alpha_{n + 1}}b_{n + 2}^{\alpha_{n + 2}} = a^{2\beta}$, survives the integration since $2m \ne 0$. 

Let us look at cases with even $n$, where the integral becomes
\begin{align}
  % & \ \frac{i^{n}}{2} \frac{\eta(\tau)^{2g - 2 + n}}{\prod_{j = 1}^{n}\vartheta_1(2\mathfrak{b}_j)}
  % \oint \frac{da}{2\pi i a} a^{2m}
  % \sum_{\beta = \pm}\sum_{\vec \alpha = \pm 1} \left(
  % \prod_{j = 1}^{n}\alpha_j\right)
  % \sum_{k = 1}^{2g - 2 + n}
  % \lambda_{k}^{(2g - 2 + n)}E_k \begin{bmatrix}
  %   (-1)^n \\
  %   a^{2\beta}\prod_{j = 1}^{n}b_j^{\alpha_j}
  % \end{bmatrix}\\
  = & \ - \frac{i^{n}}{2} \frac{\eta(\tau)^{2g - 2 + n}}{\prod_{j = 1}^{n}\vartheta_1(2\mathfrak{b}_j)}
  \sum_{k = 1}^{2g - 2 + n}
    \lambda_k^{(2g - 2 + n)}
    \frac{q^m}{(k-1)!}
    \frac{\text{Eu}_{k - 1}(q^m)}{(1 - q^m)^k}
  \prod_{j = 1}^{n}(b_j^{m} - b_j^{-m})\ . \nonumber
\end{align}
where we applied integration formula. Note also that $k$ is even in order for the rational numbers $\lambda$ to be non-vanishing, and
\begin{align}
  \sum_{\vec \alpha = \pm}\left(\prod_{j = 1}^{n}\alpha_j\right)
  \left(
  \frac{1}{\prod_{j = 1}^n b_j^{m\alpha_j}}
  + \prod_{j = 1}^{n}b_j^{m \alpha_j}
  \right)
  = 2 \prod_{j = 1}^{n}(b_j^{m} - b_j^{-m}) \ .
\end{align}
Therefore,
\begin{align}
  \langle W_j\rangle_{g, n}^{(1)}
  = & \ \mathcal{I}_{g, n}\delta_{j \in \mathbb{Z}}
  - \sum_{\substack{m = - j\\m \ne 0}}^{+j} \frac{i^{n}}{2} \frac{\eta(\tau)^{2g - 2 + n}}{\prod_{j = 1}^{n}\vartheta_1(2\mathfrak{b}_j)}\prod_{j = 1}^{n}(b_j^{m} - b_j^{-m})
    \sum_{k = 1}^{2g - 2 + n}
      \lambda_k^{(2g - 2 + n)}
      \frac{q^m}{(k-1)!}
      \frac{\text{Eu}_{k - 1}(q^m)}{(1 - q^m)^k} \nonumber \\
  = & \ \mathcal{I}_{g, n}\delta_{j \in \mathbb{Z}}
  - \frac{1}{2}
    \prod_{i = 1}^{n} \frac{i \eta(\tau)^n}{\vartheta_2(\mathfrak{b}_j)}
    \sum_{\substack{m = - j\\m \ne 0}}^{+j} 
    \frac{\eta(\tau)^{2g - 2}}{(q^{m/2} - q^{-m/2})^{2g - 2}}\prod_{j = 1}^{n}\frac{b_j^{m} - b_j^{-m}}{q^{m/2} - q^{-m/2}} \ ,
\end{align}
where in the second equality we apply the identity (for even $n$)
\begin{align}
  \sum_{k = 1}^{2g -2 + n}\lambda_k^{(2g - 2 + n)} \frac{q^m}{(k-1)!} \frac{\text{Eu}_{k - 1}(q^m)}{(1 - q^m)^k}
  = \frac{1}{(q^{m/2} - q^{-m/2})^{2g - 2 + n}} \ .
\end{align}


A similar computation can be carried out with odd $n$. Again, an $m \ne 0$ term integrates to
\begin{align}
  = + \frac{i^n}{2} \frac{\eta(\tau)^{2g - 2 + n}}{\prod_{j = 1}^{n} \vartheta_1(2 \mathfrak{b}_j)}
  \sum_{k = 1}^{2g - 2 + n} 
  \lambda_k^{(2g - 2 + n)} \frac{q^{m/2}}{(k - 1)!}\Phi(q^m, 1 - k, \frac{1}{2})\prod_{j - 1}^{n}(b_j^m - b_j^{-m}) \ ,
\end{align}
where we used for odd $n$,
\begin{align}
  \sum_{\vec \alpha = \pm } \left(\prod_{j = 1}^{n}\alpha_j\right)
  \left(\prod_{j = 1}^{n}b_j^{-m} - \prod_{j = 1}^{n} b_j^m\right)
  = -2 \prod_{j = 1}^{n}(b_j^m - b_j^{-m})\ .
\end{align}
For odd $n$ we continue to have the same formula as the even $n$ case,
\begin{align}
  \langle W_j\rangle^{(1)}_{g,n}
  = & \ \mathcal{I}_{g, n}\delta_{j \in \mathbb{Z}}
  - \frac{1}{2}
    \prod_{i = 1}^{n} \frac{i \eta(\tau)^n}{\vartheta_1(\mathfrak{b}_j)}
    \sum_{\substack{m = - j\\m \ne 0}}^{+j} 
    \frac{\eta(\tau)^{2g - 2}}{(q^{m/2} - q^{-m/2})^{2g - 2}}\prod_{j = 1}^{n}\frac{b_j^{m} - b_j^{-m}}{q^{m/2} - q^{-m/2}} \ ,
\end{align}
thanks to the curious identity for odd $n$,
\begin{align}
  \sum_{k = 1}^{2g - 2 + n}\lambda_k^{(2g - 2 + n)} \frac{q^{m/2}}{(k - 1)!}
  \Phi(q^m, 1 - k, \frac{1}{2}) = - \frac{1}{(q^{m/2} - q^{-m/2})^{2 g - 2 + n}} \ .
\end{align}






\subsection{\texorpdfstring{Type-2 half Wilson line index in $\mathcal{T}[\Sigma_{g,n}]$}{}}

Next we consider another type of half Wilson operator index, which can be built on top of that of the $SU(2)$ SQCD by extending the relevant Riemann surface on either sides (but not further connecting the two sides). Put differently, we consider a half Wilson operator sitting at a tube that separates the Riemann surface into two disconnected pieces $\Sigma_{g_1, n_1}$ and $\Sigma_{g_2, n_2}$. See Figure \ref{fig:type-2-Wilson-line}. Let us denote such a Wilson index by $\langle W_j\rangle^{(2)}_{g_1, n_1; g_2, n_2}$. In this notation, the previous Wilson index $\langle W_j\rangle_{0,4}$ of the $SU(2)$ SQCD can be written as $\langle W\rangle_{0,3;0;3}^{(2)}$.
% Figure environment removed


% Figure environment removed

% Figure environment removed



\subsubsection{Simple type-2 examples}

We begin our analysis by looking at a simple genus-one configuration in Figure \ref{fig:genus-one-type-2}. It can be constructed from the $SU(2)$ SQCD by gauging the diagonal of the $SU(2)_{b_1} \times SU(2)_{b_2}$. The Wilson index can be computed by
\begin{align}
  \langle W_j \rangle^{(2)}_{1, 2}
  = \oint \frac{da}{2\pi i a} \langle W\rangle_{0,3;0;3}^{(2)}\Big|_{\substack{b_1 = a\\b_2 = 1/a}} \left(-\frac{1}{2}\right)\vartheta_1(\pm 2 \mathfrak{a}) \ .
\end{align}
Recall that (\ref{Wilson-index-SQCD})
\begin{align}
  \langle W\rangle_{0,3;0;3}^{(2)}
  = & \ \mathcal{I}_{0,4}\delta_{j \in \mathbb{Z}}
  - \sum_{i = 1}^{4} \left(\sum_{\substack{m = - j \\ m \ne 0}}^{+j} \frac{M_i^{2m} - M_i^{-2m}}{q^{m} - q^{-m}}\right)R_i \ ,
\end{align}
where $M_1 = b_1b_2$, $M_2 = b_1/b_2$, $M_3 = b_3 b_4$ and $M_4 = b_3 / b_4$. Obviously as $b_1 = a, b_1 = 1/a$, the $i = 1$ term does not contribute. Therefore, the Wilson index reads (where we have renamed $b_3, b_4 \to b_1, b_2$),
\begin{align}
  \langle W_j\rangle^{(2)}_{1,1; 0,3}
  = & \ \delta_{j \in \mathbb{Z}}\mathcal{I}_{1,2}
  - \frac{\eta(\tau)^2}{\prod_{i = 1}^2\vartheta_1(2 \mathfrak{b}_i)}
    \sum_{\substack{m = j \\ m\ne 0}}^{+j} (q^m + q^{-m})\prod_{i =1,2}\frac{b_i^{2m} - b_i^{-2m}}{q^m - q^{-m}} \nonumber \\
  & \ - \frac{\eta(\tau)^2}{2 \prod_{i=3,4} \vartheta_1(2 \mathfrak{b}_i)}
  \sum_{\alpha = \pm} \bigg(\alpha
  E_1 \begin{bmatrix}
    1 \\ b_1 b_2^\alpha  
  \end{bmatrix}
  \sum_{\substack{m = -j\\m \ne 0}}^{+j}
  \frac{(b_1b_2^\alpha)^{2m} - (b_1b_2^\alpha)^{- 2m}}{q^m - q^{-m}}
  \bigg) \ .
\end{align}
There are four major terms in this half Wilson index, which are proportional respectively to four linear independent expressions,
\begin{align}
  \mathcal{I}_{1,2}, \qquad
  \frac{\eta(\tau)^2}{\prod_{i = 1}^2 \vartheta_1(2 \mathfrak{b}_i)} , \qquad
  \frac{\eta(\tau)^2}{\prod_{i = 1}^2 \vartheta_1(2 \mathfrak{b}_i)}
    E_1 \begin{bmatrix}
      1 \\ b_1 b_2^\pm
    \end{bmatrix} \ ,
\end{align}
with rational coefficients in $b_i, q$. The first two factors have appeared previously in section \ref{section:genus-one-two-punctures}, both being solutions to the flavored modular differential equations \cite{Zheng:2022zkm}. It turns out that the two new factors containing $E_1$ are also additional solutions to the same set of equations, and therefore the type-2 index $\langle W_j\rangle^{(2)}_{1,1; 0,3}$ is also a linear combinations of $\chi(\mathcal{T}[\Sigma_{1,2}])$ characters with rational coefficients.



Next we consider a Wilson operator as demonstrated in Figure \ref{Wilson-type-2-example-1}. There are different ways to compute the index, and the most straightforward way is through the contour integral
\begin{align}
  \langle W_j\rangle_{1,2; 0,3}^{(2)} = & \ \oint \prod_{i = }^{3}\frac{da_i}{2\pi i a_i}\left[\sum_{m = -j}^{+j}a_3^{2m}\right]
  \prod_{\pm\pm}\frac{\eta(\tau)}{
    \vartheta_4(\mathfrak{b}_1 \pm \mathfrak{a}_1 \pm \mathfrak{a}_2)
  }
  \prod_{\pm\pm}\frac{\eta(\tau)}{
    \vartheta_4(\mathfrak{a}_3 \mp \mathfrak{a}_1 \mp \mathfrak{a}_2)
  }\nonumber\\
  & \ \qquad \times \prod_{\pm \pm}\frac{\eta(\tau)}{\vartheta_4(- \mathfrak{a}_3 \pm \mathfrak{b}_2 \pm \mathfrak{b}_3)}
  \prod_{i = 1}^{3}\left(- \frac{1}{2}\vartheta_1(\pm 2 \mathfrak{a}_i)\right) \ .
\end{align}
We choose to evaluate first the $a_3$-integral, and then $a_1, a_2$-integral. The computation is fairly tedious, and we only show the end result,
\begin{align}
  & \ \langle W_j\rangle_{1,2; 0,3}^{(2)} = \mathcal{I} \delta_{j \in \mathbb{Z}} \nonumber \\
  & \ + \sum_{\alpha, \beta = \pm}\sum_{\substack{m = -j \\ m \ne 0}}^j \frac{i \eta(\tau)^3}{8 \prod_{i = }^{3}\vartheta_1(2\mathfrak{b}_i)} \bigg[
  - \frac{4\alpha \beta b_2^{2m \alpha}b_3^{2m \beta}}{q^m - q^{-m}}
  \sum_{\gamma, \delta = \pm} \delta E_2 \begin{bmatrix}
    1 \\ q^{\frac{\gamma}{2}}  b_1^\delta b_2^\alpha b_3^\beta
  \end{bmatrix}(2\tau) \nonumber \\
  & \ \qquad\qquad\qquad\qquad\qquad + \frac{\alpha \beta b_2^{2m \alpha}b_3^{2m \beta}}{q^m - q^{-m}}
  \sum_{\gamma, \delta = \pm} \delta \gamma E_1 \begin{bmatrix}
    1 \\ q^{\frac{\gamma}{2}} b_1^\delta b_2^\alpha b_3^\beta
  \end{bmatrix}(2\tau)\\
  & \ \qquad\qquad\qquad\qquad\qquad - \frac{2 \alpha \beta b_2^{2m \alpha}b_3^{2m \beta}}{q^m - q^{-m}}
  \sum_{\delta = \pm} \delta E_2 \begin{bmatrix}
    -1 \\ b_1^\delta b_2^\alpha b_3^\beta  
  \end{bmatrix} \nonumber\\
  & \ \qquad\qquad\qquad\qquad\qquad + \frac{1}{q^m - q^{-m}}\frac{1}{1 - q^{-2m\alpha}} \sum_{\kappa, \gamma, \delta = \pm}b_2^{2m\gamma \alpha}b_3^{2m \delta \alpha} \alpha \gamma \delta \kappa E_1 \begin{bmatrix}
    -1 \\ b_1^\kappa b_2^{\gamma \alpha \beta}  b_3^{\delta \alpha \beta}
  \end{bmatrix}
  \bigg] \nonumber \\
  & \ + \sum_{\alpha, \beta = \pm} \sum_m'
    \frac{b_1^{2m \alpha} + b_1^{-2m\alpha}}{(q^m - q^{-m})(q^{m \alpha} - q^{-m \alpha})}
    \frac{\alpha \eta(\tau)^6}{8 \prod_{\pm \pm}\vartheta_4(\mathfrak{b}_1 \pm \mathfrak{b}_2 \pm \mathfrak{b}_3)} \ . \nonumber
\end{align}
Note that the Eisenstein series in the first two lines depend on $2\tau$ instead of just $\tau$, a price to pay for simplifying the result using the following identities,
\begin{align}
  \sum_{\pm}E_k\left[\begin{matrix}
    \phi \\ \pm z
  \end{matrix}\right](\tau) = & \ 2 E_k\left[\begin{matrix}
    \phi \\ z^2
  \end{matrix}\right](2\tau) \ , \nonumber \\
  \sum_{\pm} \pm E_k\left[\begin{matrix}
    \phi \\ \pm z
  \end{matrix}\right](\tau)
  = & \ -2 E_k\left[\begin{matrix}
    \phi \\ z^2
  \end{matrix}\right](2\tau)
   + 2 E_k\left[\begin{matrix}
    \phi \\ z
   \end{matrix}\right](\tau)\ , \nonumber
  \\
  E_k\left[\begin{matrix}
    + 1\\z
  \end{matrix}\right](2\tau)
  + E_k\left[\begin{matrix}
    - 1\\z
  \end{matrix}\right](2\tau) = & \ 
  \frac{2}{2^k}E_k\left[\begin{matrix}
    + 1 \\ z
  \end{matrix}\right] \ ,\\
  \sum_{\pm \pm} E_k\left[\begin{matrix}
    \pm 1 \\ \pm z
  \end{matrix}\right](\tau) = & \ \frac{4}{2^k}E_k\left[
  \begin{matrix}
    + 1 \\ z^2
  \end{matrix}\right](\tau)\ . \nonumber
\end{align}



\subsubsection{General type-2 Wilson index}


From the above two examples, it is somewhat clear that the Wilson index of type-2 are significantly more complex than the type-1 index. Moreover, unlike that in type-1, the Wilson index with spin $j \in \mathbb{Z}+\frac{1}{2}$ is nontrivial. Let us compute the type-2 index from another perspective. We consider gluing two Schur index $\mathcal{I}_{g_i, n_i}$ and insert a Wilson operator at the connecting tube,
\begin{align}
  \langle W\rangle_{g_1, n_1; g_2, n_2}^{(2)}
  = \oint \frac{da}{2\pi i a} \chi_j(a) \mathcal{I}_{g_1, n_1}(\mathfrak{b}_1, \ldots, \mathfrak{b}_{n_1 - 1}, \mathfrak{a}) \mathcal{I}_\text{VM}(\mathfrak{a})
  \mathcal{I}_{g_2, n_2}( - \mathfrak{a}, \tilde{\mathfrak{b}}_1, \ldots, \tilde{\mathfrak{b}}_{n_2 - 1}) \ .
\end{align}
For this we can apply the closed-form expressions for $\mathcal{I}_{g, n}$ \cite{Pan:2021mrw}, and the above becomes
\begin{align}
  - \frac{1}{2}\oint\frac{da}{2\pi i a}
  & \ \chi_j(a)
  \frac{i^{n_1}}{2}
  \frac{\eta(\tau)^{n_1 + 2g_1 - 2}}{\prod_{j = 1}^{n_1 - 1}\vartheta_1(2 \mathfrak{b}_j)}
  \frac{\eta(\tau)^{n_2 + 2g_2 - 2}}{\prod_{j = 1}^{n_2 - 1}\vartheta_1(2 \tilde{\mathfrak{b}}_j)}\\
  & \ \times \sum_{\vec\alpha,\vec \beta} \left(\prod_{j = 1}^{n_1}\alpha_j\right)
  \left(\prod_{j = 1}^{n_2}\beta_j\right)
  \sum_{k = 1}^{n_1 + 2g_1 - 2}\sum_{\ell = 1}^{n_2 + 2g_2 - 2}
  \lambda_k^{(n_1 + 2g_1 - 2)}
  \lambda_\ell^{(n_2 + 2g_2 - 2)}\\
  & \ \qquad\qquad E_k \begin{bmatrix}
    (-1)^{n_1}  \\ a^{\alpha_{n_1}} \prod_{j = 1}^{n_1 - 1}b_j^{\alpha_j}
  \end{bmatrix}
  E_\ell \begin{bmatrix}
      (-1)^{n_2}  \\ a^{ - \beta_{n_2}} \prod_{j = 1}^{n_2 - 1}\tilde b_j^{\beta_j}
    \end{bmatrix} \ .
\end{align}
Note that the vector multiplet factor has cancelled the $\vartheta_1(2 \mathfrak{a})\vartheta_1( - 2 \mathfrak{a})$ in the denominator. Therefore, the integration boils down to computing
\begin{align}
  \oint \frac{da}{2\pi i z}\chi_j(z) E_k \begin{bmatrix}
    \pm 1 \\
    z a
  \end{bmatrix}
  E_\ell \begin{bmatrix}
    \pm 1 \\
    z b
  \end{bmatrix} \ .
\end{align}

% Figure environment removed

For the special case of $n_1 = n_2 = 1$, $g_1 = g_2 = 1$ corresponding to a Wilson line in the genus-two theory as illustrated in Figure \ref{fig:type-2-genus-two}, we can easily compute the type-2 Wilson index by applying the two identities
\begin{align}
  \oint \frac{dz}{2\pi i z} E_1 \begin{bmatrix}
      + 1 \\ z
  \end{bmatrix}^2 = \frac{q^n( (n - 2) - n q^n )}{(1 - q^n)^2}, \quad
  \oint \frac{dz}{2\pi i z} E_1 \begin{bmatrix}
      - 1 \\ z
  \end{bmatrix}^2 = \frac{q^{n/2}( (n - 1) - (n + 1) q^n )}{(1 - q^n)^2} \ . \nonumber
\end{align}
The index then reads,
\begin{align}
  \langle W_j\rangle^{(2)}_{1,1;1,1}
  = & \ \oint \frac{da}{2\pi i a}
  \chi_j(a)
  \frac{i \eta(\tau)}{\vartheta_1(2 \mathfrak{a})}
  \frac{i \eta(\tau)}{\vartheta_1(-2 \mathfrak{a})}
  \left(- \frac{1}{2}\vartheta_1(\pm 2 \mathfrak{a})\right)
  E_1 \begin{bmatrix}
    -1 \\ a  
  \end{bmatrix}
  E_1 \begin{bmatrix}
    -1 \\ a^{-1}
  \end{bmatrix} \nonumber\\
  = & \ \frac{1}{2} \bigg(
     \delta_{j \in \mathbb{Z}}\eta(\tau)^2\left(E_2(\tau) + \frac{1}{12}\right)
    - \eta(\tau)^2\sum_{\substack{m = - j \\ m\ne 0 }}^{+j}
    \frac{ (2m - 1)q^{-m} - (2m + 1)q^{m}}{(q^m - q^{-m})^2}
    \bigg) \ . \nonumber
\end{align}
We note that both $\eta(\tau)^2$ and $\eta(\tau)^2 (E_2 + \frac{1}{12})$ are solutions to the modular differential equation that annihilates the genus two Schur index $\mathcal{I}_{2,0}$ \cite{Beem:2017ooy,Zheng:2022zkm},
\begin{align}
  0 = \Big[D_q^{(6)} - 305 E_4 D_q^{(4)} - 4060E_6 D_q^{(3)}
      + 20275E_4^2 & \ D_q^{(2)} + 2100E_4 E_6 D_q^{(1)} \nonumber \\
      & \ - 68600(E_6^2 - 49125E_4^3) \Big]\mathcal{I}_{2,0} \ ,
\end{align}
and therefore the above Wilson index $\langle W_j\rangle^{(2)}_{1,1;1,1}$ is also expected to be a linear combination of characters of the chiral algebra $\chi(\mathcal{T}[\Sigma_{2,0}])$.

The same structure of linear combination actually holds true for all type-2 index $\langle W_j\rangle^{(2)}_{g_1, 1; g_2, 1}$ illustrated in Figure \ref{fig:genus-g-type-2}. Indeed, the relevant integrals are of the form ($k_i \le 2g_i - 1$)
\begin{align}
  \oint \frac{da}{2\pi i a} E_{k_1} \begin{bmatrix}
      -1 \\ a
  \end{bmatrix}E_{k_2} \begin{bmatrix}
      -1 \\ a
  \end{bmatrix}
  \sim \text{linear combination of } E_{2}, E_4, \cdots, E_{2g - 2} \ ,
\end{align}
where we have used (\ref{integration-formula-zEE-1}), (\ref{integration-formula-zEE-2}). In the end, the Wilson index $\langle W_j\rangle^{(2)}_{g_1, 1; g_2, 1}$ is a linear combination of $\eta(\tau)^{2g - 2}, \eta(\tau)^{2g - 2} E_2(\tau), \cdots, \eta(\tau)^{2g - 2}E_{2g - 2}(\tau)$ with the coefficients being rational functions of $q$. This series of functions are indeed solutions to the modular differential equations annihilating the Schur index $\mathcal{I}_{g, 0}$, as they are simply the Schur index of the vortex surface defects in the 4d theory $\mathcal{T}[\Sigma_{g,0}]$ \cite{Gaiotto:2012xa,Zheng:2022zkm}.



% Figure environment removed


For more general $n_i, g_i$, we need to apply the integration formula \eqref{integration-formula-zEE-1}, \eqref{integration-formula-zEE-2} and their variants. For example, with both $n_1, n_2$ even, we have
\begin{align}
  & \ \langle W_j\rangle_{g_1, n_1; g_2, n_2}^{(2)} \nonumber \\
  = & \ \mathcal{I}_{g_1 + g_2, n_1 + n_2 - 2}\delta_{j \in \mathbb{Z}} \nonumber\\
  & \ + \frac{\eta(\tau)^{2(g_1 + g_2) + (n_1 + n_2 - 2) - 2}}{
    2\prod_{i = 1}^{n_1 + n_2 - 2}\vartheta_1(2 \mathfrak{b}_i)
  }\\
  & \ \qquad \times \sum_{\substack{m = - j \\ m\ne 0}}^j \sum_{\vec \alpha} \left(\prod_{i=1}^{n_1 + n_2 - 2}\alpha_i\right)
  \sum_{\ell = 0}^{\operatorname{max}(n_i + 2g_i - 2)}
    \Lambda_\ell^{(g_1, n_1; g_2, n_2)}(\mathbf{b}^{2m}, q^{2m})
    E_\ell \begin{bmatrix}
    1 \\ \prod_{i}^{n_1 + n_2 -2} b_i^{\alpha_i}
  \end{bmatrix} \ . \nonumber
\end{align}
Here we have merged the two sets of flavor fugacities $(b_1, \ldots, b_{n_1 - 1})$ and $(\tilde b_1, \ldots, \tilde b_{n_2 - 1})$ into a larger set $\mathbf{b} = (b_{i}, \ldots, b_{n_1 + n_2 - 2})$, and the corresponding signs $(\alpha_1, \ldots, \alpha_{n_1 - 1}, \beta_1, \ldots, \beta_{n_2 - 1})$ into $(\alpha_1, \ldots, \alpha_{n_1 + n_2 - 2})$. Finally, the $\Lambda$ are a set of rational functions of $b_i$ and $q$ coming from applying the integration formula (\ref{integration-formula-zEE-3}),
\begin{align}
  \Lambda_\ell^{(g_1, n_1; g_2, n_2)}(\mathbf{b}^{2m}, q^{2m})
  = \sum_{k_i = 0}^{n_i + 2g_i - 2}\frac{1}{\ell!} \frac{(-1)^{k_2 + 1} q^{2m}}{\prod_{i = 1}^{n_2 - 1}\tilde b_i^{2m \beta_i}} \lambda_{k_1}^{(n_1 + 2g_1 - 2)}\lambda_{k_2}^{(n_2 + 2g_2 - 2)} \mathcal{E}_{k_1, k_2; \ell}(\mathbf{b}^{2m \alpha}, q^{2m}) \ . \nonumber
\end{align}
Although it is a finite sum, unlike the beautiful result for the type-1 index formula, we are unable to reorganize the above type-2 result into a more elegant form. It would be interesting to further explore the relation between the type-2 Wilson line index and the characters of the associated chiral algebra $\chi(\mathcal{T}[\Sigma_{g,n}])$, and it is likely that the Wilson line index has access to new characters besides those from the surface defects index \cite{Zheng:2022zkm}.



\section{Line operator index in other gauge theories}


\subsection{\texorpdfstring{$\mathcal{N} = 4$ $SU(3)$ theory}{}}

The flavored $\mathcal{N} = 4$ $SU(N)$ Schur index in the presence of Wilson line operators is studied in \cite{Hatsuda:2023iwi} using the Fermi-gas formalism. In the following we also compute some simple examples using our integration formula. The relevant integral is of the form
\begin{align}
  \langle W_\mathcal{R}\rangle
  = - \frac{1}{N!} \frac{\eta(\tau)^{3N - 3}}{\vartheta_4(\mathfrak{b})^{N - 1}}\oint \prod_{A = 1}^{N - 1}  \frac{da_A}{2\pi i a_A}
  \chi_\mathcal{R}(a)
  \prod_{\substack{A, B = 1 \\ A\ne B}}^N \frac{\vartheta_1(\mathfrak{a}_A - \mathfrak{a}_B)}{\vartheta_4(\mathfrak{b} + \mathfrak{a}_A - \mathfrak{a}_B)} \ .
\end{align}

We will focus on $N = 3$. The $SU(3)$ character $\chi_\mathcal{R}(a)$ is a sum of monomials $a_1^{n_1} a_2^{n_2}$. Note that the ratio of the Jacobi theta functions is symmetric in $a_1 \leftrightarrow a_2$ and $\mathfrak{a}_A \to -\mathfrak{a}_A$, and therefore we can focus on monomials of the form $a_1^{n_1 > 0} a_2^{n_2}$; trivial monomial $a_1^0 a_2^0$ insertion simply integrates to the original $\mathcal{N} = 4$ Schur index. Now we compute
\begin{align}
  - \frac{1}{N!} \frac{\eta(\tau)^{3N - 3}}{\vartheta_4(\mathfrak{b})^{N - 1}}\oint \prod_{A = 1}^{N - 1}  \frac{da_A}{2\pi i a_A}
  a_1^{n_1} a_2^{n_2}
  \prod_{\substack{A, B = 1 \\ A\ne B}}^N \frac{\vartheta_1(\mathfrak{a}_A - \mathfrak{a}_B)}{\vartheta_4(\mathfrak{b} + \mathfrak{a}_A - \mathfrak{a}_B)} \ ,
\end{align}
by first integrating $a_1$ and then $a_2$. The $a_1$ integration is easy, leaving an $a_2$ integration of
\begin{align}
  - a_2^{n_2} \sum_{\pm}R^{(1)}_{1,\pm} \frac{a_2^{n_1}b^{\pm n_1}}{q^{n_1/2} - q^{-n_1/2}}
  & \ - a_2^{n_2}\sum_{\pm} R^{(1)}_{2, \pm}\frac{a_2^{-2n_1}b^{\pm n_1}}{q^{n_1/2} - q^{- n_1/2}} \nonumber \\
  & \ - a_2^{n_2}\sum_{\pm; k,\ell = 0,1}R^{(1)}_{3, \pm, k\ell} \frac{a_2^{- n_1/2 }b^{\pm n_1/2}q^{n_1/4} q^{\frac{k-1}{2} n_1} (-1)^{\ell n_1}}{q^{n_1/2} - q^{- n_1/2}}\ ,
\end{align}
where the poles are all imaginary with residues given by the following table.
\begin{center}
  \renewcommand{\arraystretch}{2}
  \begin{tabular}{c|c}
    $(\mathfrak{a}_1)^{(1)}_{1, \pm}$ & $\mathfrak{a}_2 \pm \mathfrak{b} + \tau/2$\\
    \hline
    $R_{1,\pm}^{(1)}$ & $\displaystyle \frac{i}{6}\eta(\tau)^3
    \frac{
      \vartheta_4(3 \mathfrak{a}_2 \pm \mathfrak{b})
      \vartheta_1(3 \mathfrak{a}_2 \pm 2 \mathfrak{b})
    }{
      \vartheta_1(\pm 2 \mathfrak{b})
      \vartheta_1(3 \mathfrak{a}_2)
      \vartheta_4(3 \mathfrak{a}_2 \pm 3 \mathfrak{b})
    }$\\
    \hline
    $(\mathfrak{a}_1)^{(1)}_{2, \pm}$ & $\mathfrak{a}_1 = - 2 \mathfrak{a}_2 \pm \mathfrak{b} + \tau/2$\\
    \hline
    $R_{2,\pm}^{(1)}$ & $\displaystyle
    \frac{i}{6} \eta(\tau)^3
    \frac{
      \vartheta_4(3 \mathfrak{a}_2 \mp \mathfrak{b})
      \vartheta_1(3 \mathfrak{a}_2 \mp 2 \mathfrak{b})
    }{
      \vartheta_1(\pm 2 \mathfrak{b})
      \vartheta_1(3 \mathfrak{a}_2)
      \vartheta_4(3 \mathfrak{a}_2 \mp 3 \mathfrak{b})}
    = - R_{1, \mp}^{(1)}
    $\\
    \hline
    $(\mathfrak{a}_1)^{(1)}_{3, \pm, k\ell}$ & $\displaystyle - \frac{\mathfrak{a}_2}{2} \pm \frac{\mathfrak{b}}{2} + \frac{\tau}{4} + \frac{k \tau}{2} + \frac{\ell}{2}$\\
    \hline
    $R^{(1)}_{3,\pm, k\ell}$ & $
    \displaystyle
    \frac{i}{12} \frac{\eta(\tau)^3}{\vartheta_1(\pm 2 \mathfrak{b})} \prod_{\gamma = \pm} \frac{\vartheta_1(\frac{3}{2} \gamma \mathfrak{a}_2 \pm \frac{1}{2}\mathfrak{b} + \frac{1}{4}\tau + \frac{k}{2}\tau + \frac{\ell}{2})^2}{
      \vartheta_4(\frac{3}{2} \gamma \mathfrak{a}_2 \pm \frac{3}{2}\mathfrak{b} + \frac{1}{4} \tau + \frac{k}{2}\tau + \frac{\ell}{2})
      \vartheta_4(\frac{3}{2} \gamma \mathfrak{a}_2 \mp \frac{1}{2}\mathfrak{b} + \frac{1}{4} \tau + \frac{k}{2}\tau + \frac{\ell}{2})
    }
    $
  \end{tabular}
\end{center}

% \begin{align}
%   R_{1,\pm}^{(1)} = & \ \frac{i}{6}\eta(\tau)^3
%   \frac{
%     \vartheta_4(3 \mathfrak{a}_2 \pm \mathfrak{b})
%     \vartheta_1(3 \mathfrak{a}_2 \pm 2 \mathfrak{b})
%   }{
%     \vartheta_1(\pm 2 \mathfrak{b})
%     \vartheta_1(3 \mathfrak{a}_2)
%     \vartheta_4(3 \mathfrak{a}_2 \pm 3 \mathfrak{b})
%   } \ , \\
%   R_{2,\pm}^{(1)} = & \ \frac{i}{6} \eta(\tau)^3
%     \frac{
%       \vartheta_4(3 \mathfrak{a}_2 \mp \mathfrak{b})
%       \vartheta_1(3 \mathfrak{a}_2 \mp 2 \mathfrak{b})
%     }{
%       \vartheta_1(\pm 2 \mathfrak{b})
%       \vartheta_1(3 \mathfrak{a}_2)
%       \vartheta_4(3 \mathfrak{a}_2 \mp 3 \mathfrak{b})}
%   = - R^{(1)}_{1, \mp} \ ,\\ 
%   R^{(1)}_{3,\pm, k\ell} = & \ \frac{i}{12} \frac{\eta(\tau)^3}{\vartheta_1(\pm 2 \mathfrak{b})} \prod_{\gamma = \pm} \frac{\vartheta_1(\frac{3}{2} \gamma \mathfrak{a}_2 \pm \frac{1}{2}\mathfrak{b} + \frac{1}{4}\tau + \frac{k}{2}\tau + \frac{\ell}{2})^2}{
%     \vartheta_4(\frac{3}{2} \gamma \mathfrak{a}_2 \pm \frac{3}{2}\mathfrak{b} + \frac{1}{4} \tau + \frac{k}{2}\tau + \frac{\ell}{2})
%     \vartheta_4(\frac{3}{2} \gamma \mathfrak{a}_2 \mp \frac{1}{2}\mathfrak{b} + \frac{1}{4} \tau + \frac{k}{2}\tau + \frac{\ell}{2})
%   } \ .  \nonumber
% \end{align}

It can be shown that,
\begin{align}
  - \oint \frac{da_2}{2\pi i a_2} a_2^n R^{(1)}_{1,\pm} = 0 \ , \qquad
  \text{if } n \not \in 3 \mathbb{Z} \ .
\end{align}
Therefore, we only focus on $n_1 + n_2 = 3p \ge 0$. Note also that $n_2 - 2n_1 = 3(p - n_1)$ in the second sum is also a multiple of $3$. With this assumption,
\begin{align}
  & \ - \sum_{\pm}\frac{b^{\pm n_1}}{q^{n_1/2} - q^{-n_1/2}} \oint \frac{da_2}{2\pi i a_2} a_2^{3p} R^{(1)}_{1,\pm}
  = - \sum_{\pm}\frac{b^{\pm n_1}}{q^{n_1/2} - q^{-n_1/2}} \oint \frac{da_2}{2\pi i a_2} a_2^{p} \left[R^{(1)}_{1,\pm}\right]_{3\mathfrak{a}_2 \to \mathfrak{a}_2} \nonumber \\
  = & \ - \delta_{n_1 + n_2 = 0} \sum_\pm
  \frac{b^{\pm n_1}}{q^{\frac{n_1}{2}} - q^{- \frac{n_1}{2}}}
  \frac{1}{6}\frac{\vartheta_4(\mathfrak{b})}{\vartheta_4(3 \mathfrak{b})}
  \left(
  E_1 \begin{bmatrix}
    -1 \\ b^{\pm 2} q^{\frac{1}{2}}
  \end{bmatrix}
  - E_1 \begin{bmatrix}
    -1 \\ b^{\mp}  
  \end{bmatrix}
  \right) \\
  & - \delta_{n_1 + n_2 \ne 0} \sum_{\pm}
  \frac{b^{\pm n_1}}{q^{\frac{n_1}{2}} - q^{- \frac{n_1}{2}}}
  \frac{1}{6}\frac{\vartheta_4(\mathfrak{b})}{\vartheta_4(3 \mathfrak{b})}
  \left(
  \frac{q^{\frac{1}{2}p} - b^{\mp 3 p}}{q^{p/2} - q^{-p/2}}
  \right) \ . \nonumber
\end{align}
Similarly
\begin{align}
  & \ - \oint \frac{da_2}{2\pi i a_2} a_2^{n_2}\sum_{\pm} R^{(1)}_{2, \pm}\frac{a_2^{-2n_1}b^{\pm n_1}}{q^{n_1/2} - q^{- n_1/2}}  \\
  = & \ - \delta_{n_2 - 2n_1 = 0}
  \frac{1}{6}\frac{\vartheta_4(\mathfrak{b})}{\vartheta_4(3 \mathfrak{b})}
  \sum_{\pm} \frac{b^{\pm n_1}}{q^{n_1/2} - q^{- n_1/2}}
  \left(
    E_1 \begin{bmatrix}
      -1 \\ b^{\mp 2}q^{\frac{1}{2}}
    \end{bmatrix}
    - E_1 \begin{bmatrix}
      -1 \\ b^{\pm}
    \end{bmatrix}
  \right)\\
  & \ + \delta_{n_2 - 2n_1 \ne 0}
  \frac{1}{6}
  \frac{\vartheta_4(\mathfrak{b})}{\vartheta_4(3 \mathfrak{b})}
  \sum_{\pm} \frac{b^{\pm n_1}}{q^{n_1/2} - q^{- n_1/2}}
  \left(
  \frac{q^{\frac{n_2 - 2n_1}{6}} - b^{\pm (n_2 - 2n_1)}}{q^{\frac{n_2 - 2n_1}{6}} - q^{- \frac{n_2 - 2n_1}{6}}}
  \right) \ .
\end{align}
Lastly, one can also check that
\begin{align}
  \oint \frac{da_2}{2\pi i a_2}a_2^n R^{(1)}_{3, \pm, k\ell} = 0, \qquad
  \text{if } n \not \in \frac{3}{2} \mathbb{Z} \ .
\end{align}
Therefore, since $n_1 + n_2 $ is an integer, we may assume $n_2 - n_1/2 = n_1 + n_2 - \frac{3}{2}n_1 = 3p - \frac{3n_1}{2}$ with $p \in \mathbb{Z}$ in order for the integral to be non-zero,
\begin{align}
  & \ - \sum_{\pm, k,\ell} \frac{b^{\pm n_1/2}q^{n_1/4} q^{\frac{k-1}{2} n_1} (-1)^{\ell n_1}}{q^{n_1/2} - q^{- n_1/2}}
  \oint \frac{da_2}{2\pi i a_2} R^{(1)}_{3, \pm, k\ell} a_2^{n_2 - \frac{n_1}{2}} \nonumber \\
  = & \ \delta_{n_2 \ne \frac{1}{2}n_1} \frac{\vartheta_4(\mathfrak{b})}{12\vartheta_4(3 \mathfrak{b})}
  \sum_{\alpha, \gamma = \pm}\sum_{k,\ell = 0,1}
  \gamma
  \frac{
    b^{\frac{\alpha}{2}( (1 + \gamma) n_1 - 2\gamma n_2)}
    q^{- \frac{1}{12}(2k - 1)( (\gamma - 1)n_1 - 2\gamma n_2 )}
  }{
    (q^{\frac{n_1}{2}} - q^{- \frac{n_1}{2}})
    (q^{\frac{1}{6}(2n_2 - 1)}
        - q^{ - \frac{1}{6}(2n_2 - 1)})
  } \\
  & \ - \delta_{n_2 = \frac{1}{2}n_1}
  \frac{\vartheta_4(\mathfrak{b})}{12\vartheta_4(3 \mathfrak{b})}
  \sum_{\alpha, \gamma = \pm}\sum_{k,\ell = 0}^{1}
  \gamma
  \frac{
    b^{\alpha \frac{n_1}{2}}
    q^{\frac{1}{4}n_1(2k - 1)}
    (-1)^{\ell n_1}
  }{
    q^{n_1/2} - q^{- n_1/2}
  }
  E_1 \begin{bmatrix}
    -1\\
    b^{ - \frac{1}{2}\alpha(3\gamma + 1)}  
    q^{- \frac{1}{4}(2k(\gamma - 1) - (\gamma + 1))}
  \end{bmatrix} \ . \nonumber
\end{align}

In the above we have used the poles and residues of the $R$-factors listed in the following table.
{
\renewcommand{\arraystretch}{1.5}
\begin{table}[h!]
\centering
  \begin{tabular}{c|c|c}
    factor & poles & residues\\
    \hline
    $R^{(1)}_{1,\pm}$ & $\mathfrak{a}_2 = 0$ & $ - \frac{i}{6\eta(\tau)} \frac{\vartheta_4( \mathfrak{b})}{\vartheta_4( 3 \mathfrak{b})}$\\
                      & $\mathfrak{a}_2 = \mp 3 \mathfrak{b} + \frac{\tau}{2}$ & $ + \frac{i}{6\eta(\tau)} \frac{\vartheta_4( \mathfrak{b})}{\vartheta_4( 3 \mathfrak{b})}$\\
    \hline
    $R^{(1)}_{2,\pm}$ & $\mathfrak{a}_2 = 0$ & $ + \frac{i}{6\eta(\tau)} \frac{\vartheta_4( \mathfrak{b})}{\vartheta_4( 3 \mathfrak{b})}$\\
                      & $\mathfrak{a}_2 = \pm 3 \mathfrak{b} + \frac{\tau}{2}$ & $ - \frac{i}{6\eta(\tau)} \frac{\vartheta_4( \mathfrak{b})}{\vartheta_4( 3 \mathfrak{b})}$\\
    \hline
    $R^{(1)}_{3,\pm,k\ell}$ & $\mathfrak{a}_2 = \mp \frac{3}{2} \gamma \mathfrak{b} + \frac{\tau}{2} + \frac{1}{4}(2k - 1)\gamma \tau + \frac{\ell}{2}$, $\gamma = \pm 1$ & $ \gamma \frac{\vartheta_4(\mathfrak{b})}{12 \vartheta_4(3 \mathfrak{b})}$
  \end{tabular}
\end{table}
}

Putting all the above together, we have
\begin{align}
  & \ - \frac{1}{N!} \frac{\eta(\tau)^{3N - 3}}{\vartheta_4(\mathfrak{b})^{N - 1}}\oint \prod_{A = 1}^{N - 1}  \frac{da_A}{2\pi i a_A}
  a_1^{n_1} a_2^{n_2}
  \prod_{\substack{A, B = 1 \\ A\ne B}}^N \frac{\vartheta_1(\mathfrak{a}_A - \mathfrak{a}_B)}{\vartheta_4(\mathfrak{b} + \mathfrak{a}_A - \mathfrak{a}_B)}\nonumber\\
  = & \ 0 \qquad \text{ if } n_1 + n_2 \ne 0 \mod 3 ,  \\
  \text{else} = & \ 
  + \delta_{n_1 + n_2 = 0} 
    \frac{1}{6}\frac{\vartheta_4(\mathfrak{b})}{\vartheta_4(3 \mathfrak{b})}
    \sum_\pm
    \frac{b^{\pm n_1}}{q^{\frac{n_1}{2}} - q^{- \frac{n_1}{2}}}
    \left(
    E_1 \begin{bmatrix}
      -1 \\ b^{\pm 2} q^{\frac{1}{2}}
    \end{bmatrix}
    - E_1 \begin{bmatrix}
      -1 \\ b^{\mp}  
    \end{bmatrix}
    \right) \nonumber\\
    & - \delta_{n_1 + n_2 \ne 0} 
    \frac{1}{6}\frac{\vartheta_4(\mathfrak{b})}{\vartheta_4(3 \mathfrak{b})}
    \sum_{\pm}
    \frac{b^{\pm n_1}}{q^{\frac{n_1}{2}} - q^{- \frac{n_1}{2}}}
    \left(
    \frac{q^{\frac{1}{2}p} - b^{\mp 3 p}}{q^{p/2} - q^{-p/2}}
    \right) \nonumber\\
    & \ - \delta_{n_2 = 2n_1}
      \frac{1}{6}\frac{\vartheta_4(\mathfrak{b})}{\vartheta_4(3 \mathfrak{b})}
      \sum_{\pm} \frac{b^{\pm n_1}}{q^{n_1/2} - q^{- n_1/2}}
      \left(
        E_1 \begin{bmatrix}
          -1 \\ b^{\mp 2}q^{\frac{1}{2}}
        \end{bmatrix}
        - E_1 \begin{bmatrix}
          -1 \\ b^{\pm}
        \end{bmatrix}
      \right) \nonumber\\
      & \ + \delta_{n_2 \ne 2n_1}
      \frac{1}{6}
      \frac{\vartheta_4(\mathfrak{b})}{\vartheta_4(3 \mathfrak{b})}
      \sum_{\pm} \frac{b^{\pm n_1}}{q^{n_1/2} - q^{- n_1/2}}
      \left(
      \frac{q^{\frac{n_2 - 2n_1}{6}} - b^{\pm (n_2 - 2n_1)}}{q^{\frac{n_2 - 2n_1}{6}} - q^{- \frac{n_2 - 2n_1}{6}}}
      \right)\nonumber \\
    & \ - \delta_{n_2 = \frac{1}{2}n_1}
    \frac{\vartheta_4(\mathfrak{b})}{12\vartheta_4(3 \mathfrak{b})}
    \sum_{\alpha, \gamma = \pm}\sum_{k,\ell = 0}^{1}
    \gamma
    \frac{
      b^{\alpha \frac{n_1}{2}}
      q^{\frac{1}{4}n_1(2k - 1)}
      (-1)^{\ell n_1}
    }{
      q^{n_1/2} - q^{- n_1/2}
    }
    E_1 \begin{bmatrix}
      -1\\
      b^{ - \frac{1}{2}\alpha(3\gamma + 1)}  
      q^{- \frac{1}{4}(2k(\gamma - 1) - (\gamma + 1))}
    \end{bmatrix} \nonumber\\
    & \ + \delta_{n_2 \ne \frac{1}{2}n_1} \frac{\vartheta_4(\mathfrak{b})}{12\vartheta_4(3 \mathfrak{b})}
      \sum_{\alpha, \gamma = \pm}\sum_{k,\ell = 0,1}
      \gamma
      \frac{
        b^{\frac{\alpha}{2}( (1 + \gamma) n_1 - 2\gamma n_2)}
        q^{- \frac{1}{12}(2k - 1)( (\gamma - 3)n_1 - 2\gamma n_2 )}
      }{
        (q^{\frac{n_1}{2}} - q^{- \frac{n_1}{2}})
        (q^{\frac{1}{6}(2n_2 - n_1)}
            - q^{ - \frac{1}{6}(2n_2 - n_1)})
      } \nonumber \ .
\end{align}
The formula above implies the following symmetry which can be used to simplify computations,
\begin{align}
  \mathcal{I}(n_1, n_2) = & \ \mathcal{I}(n_2, n_1) = \mathcal{I}(- n_1, - n_2), \\
  \mathcal{I}(n_1, n_2) = & \ \mathcal{I}(n_1, n_1 - n_2) = \mathcal{I}(n_2 - n_1, n_2) \ .
\end{align}
With this formula, one can compute any Wilson line index in any $SU(3)$ representation $\mathcal{R}$ in closed-form. For example,
\begin{align}
  \langle W_{[1,1]} \rangle
  = & \ 2\mathcal{I}_{\mathcal{N} = 4 \ SU(3)} + 6 \mathcal{I}_{1,2} \nonumber \\
  = & \ 2 \mathcal{I}_{\mathcal{N}=4 \ SU(3)}\\
  & \ + \frac{\vartheta_4(\mathfrak{b})}{\vartheta_4(3 \mathfrak{b})}
  \Bigg[
    \frac{b\sqrt{q} - (1+b^4)q + b^3 q^{\frac{3}{2}}}{b^2 (1 - q)^2}
    + \frac{(b^2 -1)\sqrt{q}}{b(q-1)}
    \left(
    E_1 \begin{bmatrix}
      -1 \\ b  
    \end{bmatrix}
    + E_1 \begin{bmatrix}
      -1 \\ b^2 q^{\frac{1}{2}}
    \end{bmatrix}
    \right)
  \Bigg] \ . \nonumber
\end{align}

\begin{align}
  \langle W_{[2,2]} \rangle
  = & \ 3\mathcal{I}_{\mathcal{N} = 4 \ SU(3)} + 12 \mathcal{I}_{1,2}
  + 6 \mathcal{I}_{2,4} + 6 \mathcal{I}_{3,0} \nonumber \\
  = & \ 3 \mathcal{I}_{\mathcal{N}=4 \ SU(3)} \nonumber\\
  & \ + \frac{\vartheta_4(\mathfrak{b})}{\vartheta_4(3 \mathfrak{b})}
  \Bigg[
    \frac{\sqrt{q} (b^3 q^{\frac{1}{2}}-1) (-b^5 q-2 b^4 q^{\frac{1}{2}} (q+1)-b^3 (q (q+4)+2))}{b^4 \left(q^2-1\right)^2}
  \Bigg] \\
  & \ + \frac{\vartheta_4(\mathfrak{b})}{\vartheta_4(3 \mathfrak{b})}
  \Bigg[
    \frac{\sqrt{q} (+b^2 (2 q (q+2)+1) q^{\frac{1}{2}}+2 b (q+1) q+q^{3/2})}{b^4 \left(q^2-1\right)^2}
  \Bigg] \nonumber \\
  & \ + \frac{\vartheta_4(\mathfrak{b})}{\vartheta_4(3 \mathfrak{b})}
  \frac{\sqrt{q}(b^2 -1) \Big[(b^2+1) \sqrt{q}+2 b q+2 b \Big]}{b^2(q^2 - 1)}
  \left(
  E_1 \begin{bmatrix}
    -1 \\ b  
  \end{bmatrix}
  + E_1 \begin{bmatrix}
    -1 \\ b^2\sqrt{q}  
  \end{bmatrix}
  \right) \ . \nonumber
\end{align}



\begin{align}
  \langle W_{[3,3]} \rangle
  = & \ 4\mathcal{I}_{\mathcal{N} = 4 \ SU(3)} + 18 \mathcal{I}_{1,2}
  + 12 \mathcal{I}_{2,4} + 12 \mathcal{I}_{3,0} + 12 \mathcal{I}_{4,5}
  + 6 \mathcal{I}_{3,6} \ .
\end{align}














\subsection{\texorpdfstring{$\mathcal{N} = 2$ $SU(3)$ SQCD}{}}

Let us also consider Wilson operator in the $SU(3)$ SQCD. The relevant integral reads
\begin{align}
  \mathcal{I}_{SU(3) \ \text{SQCD}} = & \ - \frac{1}{3!} \eta(\tau)^{16} \oint \prod_{A = 1}^2 \frac{da_A}{2\pi i a_A}
  \chi_\mathcal{R}(a)
  \frac{\prod_{A \ne B} \vartheta_1(\mathfrak{a}_A - \mathfrak{a}_B)}{\prod_{A = 1}^3 \prod_{i = 1}^{6} \vartheta_4(\mathfrak{a}_A - \mathfrak{m}_i)} \\
  \coloneqq & \ \oint \prod_{A = 1}^2 \frac{da_A}{2\pi i a_A}
  \chi_\mathcal{R}(a) \mathcal{Z}(\mathfrak{a}, \mathfrak{m}) \ .
\end{align}


\subsubsection{Fundamental representation}

As the simplest example, we consider the fundamental representation
\begin{align}
  \chi (a) = a_1 + a_2 + \frac{1}{a_1 a_2} \ .
\end{align}
First we note that
\begin{align}
  \oint \prod_{A = 1}^2 \frac{da_A}{2\pi i a_A} a_1 \mathcal{Z}(\mathfrak{a}, \mathfrak{m})
  = \oint \prod_{A = 1}^2 \frac{da_A}{2\pi i a_A} a_2 \mathcal{Z}(\mathfrak{a}, \mathfrak{m}) \ .
\end{align}
Therefore we simply compute the one with $a_1$ insertion. The relevant poles when performing the $a_1$ integration are all imaginary given by
\begin{align}
  \mathfrak{a}_1 = \mathfrak{m}_{j_1} + \frac{\tau}{2}, \qquad
  \mathfrak{a}_1 = - \mathfrak{a}_2 - \mathfrak{m}_{j_1} + \frac{\tau}{2} \ ,
\end{align}
with the respective residues
\begin{align}
  R_{j_1} \coloneqq - \frac{1}{6}
  \frac{
    \eta(\tau)^{13}q^{\frac{1}{8}}
    \prod_{A\ne B} \vartheta_1(\mathfrak{a}_A - \mathfrak{a}_B)|_{\mathfrak{a}_1 = \mathfrak{m}_{j_1} + \frac{\tau}{2}}
  }{
    \prod_i\vartheta_4(\mathfrak{a}_2 - \mathfrak{m}_i)
    \prod_i\vartheta_4(\mathfrak{a}_2 + \mathfrak{m}_{j_1} + \mathfrak{m}_i + \frac{\tau}{2})
    \prod_{i \ne j_1}\vartheta_4(\mathfrak{m}_i - \mathfrak{m}_{j_1} - \frac{\tau}{2})
  } \ ,  \quad -R_{j_1} \ . \nonumber
\end{align}
Therefore, after the $a_1$ integral we are left with
\begin{align}
  \oint \frac{da_2}{2\pi i a_2} \left[- \sum_{j_1 = 1}^{6}R_{j_1} \frac{1}{q^1 - 1} (m_{j_1}q^{\frac{1}{2}})
      + \sum_{j_1 = 1}^{6}R_{j_1} \frac{1}{q^1 - 1} (a_2^{-1} m^{-1}_{j_1}q^{\frac{1}{2}})\right] \ .
\end{align}

Next we perform the $a_2$ integral. Each residue $R_{j_1}$ is an elliptic function with respect to $\mathfrak{a}_2$, with imaginary and real poles
\begin{align}
  \mathfrak{a}_2 = & \ + \mathfrak{m}_{j_2} + \frac{\tau}{2}, & j_2 \ne & \ j_1 \\
  \text{or}, \qquad = & \ - \mathfrak{m}_{j_1} - \mathfrak{m}_{j_2} \ ,  & j_2 \ne & \ j_1 \ .
\end{align}
The corresponding residues are, respectively,
\begin{align}
  R_{j_1 j_2} = \frac{
      \eta(\tau)^{10}
      \vartheta_4(2 \mathfrak{m}_{j_1} + \mathfrak{m}_{j_2})
      \vartheta_4(\mathfrak{m}_{j_1} + 2\mathfrak{m}_{j_2})}{
    6
    \prod_{i\ne j_1, j_2}\vartheta_1(\mathfrak{m}_{j_1} - \mathfrak{m}_i)\vartheta_1(\mathfrak{m}_{j_2} - \mathfrak{m}_i)
    \prod_{i \ne j_1, j_2} \vartheta_4(\mathfrak{m}_{j_1}+ \mathfrak{m}_{j_2} + \mathfrak{m}_i)
  }, \quad
  - R_{j_1 j_2} \ .
\end{align}
We also set $R_{j_1 j_2} = 0$ when $j_1 = j_2$. With this, we have by applying (\ref{integration-formula-f})
\begin{align}
  \oint \frac{da_2}{2\pi i a_2} R_{j_1} = R_{j_1}(\mathfrak{a}_2 = 0)
  + \sum_{j_2 = 1}^{6} R_{j_1 j_2}E_1 \begin{bmatrix}
    -1 \\ m_{j_2}  
  \end{bmatrix}
  + R_{j_1 j_2}E_1 \begin{bmatrix}
    -1 \\ m_{j_1}m_{j_2}q^{-\frac{1}{2}}
  \end{bmatrix}\ ,
\end{align}
where we have picked $\mathfrak{a}_2 = 0$ as the reference point, and
\begin{align}
  \oint \frac{da_2}{2\pi i a_2}a_2^{-1}R_{j_1}
  = & \ + \sum_{j_2 = 1}^6 R_{j_1 j_2} \frac{1}{1 - q}m_{j_1}m_{j_2}
  - \sum_{j_2 = 1}^{6}R_{j_1 j_2} \frac{1}{q^{-1} - 1} (m_{j_2}q^{\frac{1}{2}})^{-1} \\
  = & \ + \sum_{j_2 = 1}^6 R_{j_1 j_2} \frac{m_{j_1}m_{j_2} - m_{j_2}^{-1}q^{\frac{1}{2}}}{1 - q} \ .
\end{align}
Collecting the results, the integral with $a_1$-insertion reads
\begin{align}
  & \ \oint \prod_{A = }^{2} \frac{da_A}{2\pi i a_A} a_1 \mathcal{Z} \\
  = & \ \frac{q^{\frac{1}{2}}}{1 - q} \sum_{j_1 = 1}^{6}m_{j_1}\left(
  R_{j_1}(\mathfrak{a}_2 = 0)
  + \sum_{j_2 = 1}^{6} R_{j_1 j_2}E_1 \begin{bmatrix}
    -1 \\ m_{j_2}  
  \end{bmatrix}
  + R_{j_1 j_2}E_1 \begin{bmatrix}
    -1 \\ m_{j_1}m_{j_2}q^{-\frac{1}{2}}
  \end{bmatrix}
  \right) \\
  & \ - \frac{1}{(1-q)^2} \sum_{j_1 = 1}^{6}
  \sum_{j_2 = 1}^{6}R_{j_1 j_2}(m_{j_2}q^{\frac{1}{2}} - m_{j_1}^{-1}m_{j_2}^{-1}q) \ .
\end{align}

Next we compute the integral
\begin{align}
  \oint \prod_{A = 1}^{2} \frac{da_A}{2\pi i a_A} \frac{1}{a_1 a_2} \mathcal{Z} \ .
\end{align}
Similar to the previous computation, we first integrate $a_1$ with poles $\mathfrak{a}_1 = \mathfrak{m}_{j_1} + \frac{\tau}{2}$ and $\mathfrak{a}_1 = - \mathfrak{a}_2 - \mathfrak{m}_{j_2} + \frac{\tau}{2}$,
\begin{align}
  & \ \oint \frac{da_2}{2\pi i a_2} \frac{1}{a_2} \left[- \sum_{j_1 = 1}^{6}R_{j_1} \frac{1}{q^{ - 1} - 1} (m_{j_1}q^{\frac{1}{2}})^{-1}
      + \sum_{j_1 = 1}^{6}R_{j_1} \frac{1}{q^{-1} - 1} (a_2^{-1} m^{-1}_{j_1}q^{\frac{1}{2}})^{-1}\right]\\
  = & \ \oint \frac{da_2}{2\pi i a_2} \left[- \sum_{j_1 = 1}^{6}R_{j_1} \frac{1}{q^{ - 1} - 1} (a_2^{-1} m_{j_1}^{-1}q^{-\frac{1}{2}})
      + \sum_{j_1 = 1}^{6}R_{j_1} \frac{1}{q^{-1} - 1} m_{j_1}q^{ - \frac{1}{2}}\right] \ .
\end{align}
Carrying out the $a_2$ integration, we have
\begin{align}
  & \ \oint \frac{da_1}{2\pi i a_1}\frac{da_2}{2\pi i a_2} \frac{1}{a_1 a_2} \mathcal{Z}\\
  = & \ + \frac{q^{\frac{1}{2}}}{1 - q} \sum_{j_1 = 1}^{6} m_{j_1} \left(
  R_{j_1}(\mathfrak{a}_2 = 0)
  + \sum_{j_2 = 1}^{6}R_{j_1 j_2} E_1 \begin{bmatrix}
    -1 \\ m_{j_2}  
  \end{bmatrix}
  + R_{j_1 j_2} E_1 \begin{bmatrix}
    -1 \\ m_{j_1} m_{j_2} q^{-1/2}  
  \end{bmatrix}
  \right) \\
  & \ - \frac{1}{(1 - q)^2} \sum_{j_1 = 1}^{6}
  \sum_{j_2 = 1}^{6} R_{j_1 j_2} (m_{j_2}q^{+ \frac{1}{2}} - m_{j_1}^{-1}m_{j_2}^{-1}q) \ .
\end{align}
Actually, this is identical to the previous result,
\begin{align}
  \oint \frac{da_1}{2\pi i a_1}\frac{da_2}{2\pi i a_2} \frac{1}{a_1 a_2} \mathcal{Z}
  = \oint \frac{da_1}{2\pi i a_1}\frac{da_2}{2\pi i a_2} a_1 \mathcal{Z}
  = \oint \frac{da_1}{2\pi i a_1}\frac{da_2}{2\pi i a_2} a_2 \mathcal{Z} \ .
\end{align}



Combining the integration of all three terms in the fundamental characters, we therefore have a fairly simple result,
\begin{align}
  \langle W_{\mathbf{3}} \rangle_{SU(3) \ \text{SQCD}}
  = & \ \frac{3q^{\frac{1}{2}}}{1 - q}\sum_{j_1 = 1}^{6}\left(
  R_{j_10} + \sum_{j_2 = 1}^{6}R_{j_1 j_2} \left(E_1 \begin{bmatrix}
    -1 \\ m_{j_2}  
  \end{bmatrix}
  + E_1 \begin{bmatrix}
    -1 \\ m_{j_1} m_{j_2}q^{-\frac{1}{2}}  
  \end{bmatrix}
  \right)
  \right) \\
  & \ - \frac{3}{(1-q)^2} \sum_{j_1, j_2 = 1}^{6}R_{j_1 j_2} (m_{j_2} q^{\frac{1}{2}} - m_{j_1}^{-1}m_{j_2}^{-1}q) \ .
\end{align}
where we abbreviate
\begin{align}
  R_{j_1 0} \coloneqq R_{j_1}(\mathfrak{a}_2 = 0) \ .
\end{align}


\subsubsection{General representation}

The above computation can be generalized to insertion of all half Wilson operator in any representation of the gauge group $SU(3)$. The basic building block is of course a monomial $a_1^{n_1} a_2^{n_2}$. Let us therefore compute the basic integral
\begin{align}
  \oint \frac{da_1}{2\pi i a_1}\frac{da_2}{2\pi i a_2} a_1^{n_1} a_2^{n_2} \mathcal{Z} \ .
\end{align}
Note that
\begin{align}
  \oint \frac{da_1}{2\pi i a_1}\frac{da_2}{2\pi i a_2}a_2^{n_2}\mathcal{Z}
  = \oint \frac{da_1}{2\pi i a_1}\frac{da_2}{2\pi i a_2}a_1^{n_2}\mathcal{Z} \ .
\end{align}
Therefore, without loss of generality we assume $n_1 \in \mathbb{Z}_{\ne0}$, and we first perform $a_1$ and then $a_2$ integration. The first step picks up the imaginary poles $\mathfrak{a}_1 = \mathfrak{m}_{j_1} + \frac{\tau}{2}$ and $- \mathfrak{a}_2 - \mathfrak{m}_{j_1} + \frac{\tau}{2}$, which produces
\begin{align}
  & \ \oint \frac{da_2}{2\pi i a_2} a_2^{n_2} \left[- \sum_{j_1}^{6}R_{j_1} \frac{1}{q^{n_1} - 1}(m_{j_1} q^{\frac{1}{2}})^{n_1}
  - \sum_{j_1 = 1}^{6}(-R_{j_1}) \frac{1}{q^{n_1} - 1}(a_2^{-1} m_{j_1}^{-1}q^{\frac{1}{2}})^{n_1}
  \right] \\
  = & \ \oint \frac{da_2}{2\pi i a_2} \left[- \sum_{j_1}^{6} a_2^{n_2} R_{j_1} \frac{1}{q^{n_1} - 1}(m_{j_1} q^{\frac{1}{2}})^{n_1}
  - \sum_{j_1 = 1}^{6}(-R_{j_1})a_2^{n_2 - n_1} \frac{1}{q^{n_1} - 1}(m_{j_1}^{-1}q^{\frac{1}{2}})^{n_1}
  \right] \ . \nonumber
\end{align}
Depending on whether $n_2 = 0$ or $n_2 - n_1 = 0$ or a generic $n_2$, the $a_2$-integration of the two terms take different form.

For the first term, if $n_2 = 0$, then the integral picks up contributions from the imaginary poles $\mathfrak{m}_{j_2} + \frac{\tau}{2}$ and the real poles $- \mathfrak{m}_{j_1} - \mathfrak{m}_{j_2}$, which reads
\begin{align}
  & \ \oint \frac{da_2}{2\pi i a_2} \left[- \sum_{j_1}^{6} a_2^{n_2 = 0} R_{j_1} \frac{1}{q^{n_1} - 1}(m_{j_1} q^{\frac{1}{2}})^{n_1}\right] \\
  = & \ - \sum_{j_1 = 1}^{6}\frac{(m_{j_1}q^{1/2})^{n_1}}{q^{n_1} - 1}\left(
  R_{j_10} + R_{j_1 j_2} E_1 \begin{bmatrix}
    -1 \\ m_{j_2}  
  \end{bmatrix}
  + R_{j_1 j_2} E_1 \begin{bmatrix}
    -1 \\ m_{j_1} m_{j_2}q^{-1/2}  
  \end{bmatrix}
  \right) \ .
\end{align}
However, if $n_2 \ne 0$, then
\begin{align}
  & \ \oint \frac{da_2}{2\pi i a_2} \left[- \sum_{j_1}^{6} a_2^{n_2} R_{j_1} \frac{1}{q^{n_1} - 1}(m_{j_1} q^{\frac{1}{2}})^{n_1}\right] \\
  = & \ - \sum_{j_1 = 1}^{6} \frac{(m_{j_1}q^{\frac{1}{2}})^{n_1}}{q^{n_1} - 1}
  \left(
  - \sum_{j_2 = 1}^{6}R_{j_1 j_2} \frac{1}{q^{n_2} - 1}(m_{j_2}q^{\frac{1}{2}})^{n_2}
  - \sum_{j_2 = 1}^{6} (- R_{j_1 j_2}) \frac{1}{1 - q^{-m_2}} (m_{j_1}^{-1} m_{j_2}^{-1})^{n_2}
  \right) \ . \nonumber
\end{align}



For the second term, if $n_2 - n_1 = 0$, then
\begin{align}
  & \ \oint \frac{da_2}{2\pi i a_2}\left[
    \sum_{j_1 = 1}^{6}R_{j_1} a_2^{n_2 - n_1} \frac{1}{q^{n_1} - 1} (m_{j_1}^{-1} q^{\frac{1}{2}})^{n_1}
  \right]\\
  = & \ \sum_{j_1 = 1}^{6} \frac{m_{j_1}^{-n_1}q^{\frac{n_1}{2}}}{q^{n_1} - 1} \left(
  R_{j_10} + R_{j_1 j_2} E_1 \begin{bmatrix}
    -1 \\ m_{j_2}  
  \end{bmatrix}
  + R_{j_1 j_2} E_1 \begin{bmatrix}
    -1 \\ m_{j_1} m_{j_2}q^{-1/2}  
  \end{bmatrix}
  \right) \ .
\end{align}
On the other hand, if $n_2 - n_1 \ne 0$ then 
\begin{align}
  & \ \oint \frac{da_2}{2\pi i a_2}\left[
    \sum_{j_1 = 1}^{6}R_{j_1} a_2^{n_2 - n_1} \frac{1}{q^{n_1} - 1} (m_{j_1}^{-1} q^{\frac{1}{2}})^{n_1}
  \right]\\
  = & \ \sum_{j_1 = 1}^{6} \frac{m_{j_1}^{-n_1} q^{\frac{n_1}{2}}}{q^{n_1} - 1}
  \left(
  - \sum_{j_2 = 1}^{6} R_{j_1j_2} \frac{m_{j_2}^{n_2 - n_1}q^{\frac{1}{2}(n_2 - n_1)}}{q^{n_1 - n_2} - 1}
  + \sum_{j_2 = 1}^{6} R_{j_1 j_2} \frac{(m_{j_1}^{-1} m_{j_2}^{-1})^{n_2 - n_1}}{1 - q^{-(n_2 - n_1)}}
  \right) \ .
\end{align}

Putting all terms together, we have
\begin{align}
  & \ \oint \frac{da_1}{2\pi i a_1}\frac{da_2}{2\pi i a_2}a_1^{n_1 \ne 0} a_2^{n_2} \mathcal{Z} \nonumber\\
  = & \ - \delta_{n_2 = 0} \sum_{j_1 = 1}^{6}\frac{m_{j_1}^{n_1}q^{\frac{1}{2} n_1}}{q^{n_1} - 1}\left(
  R_{j_10} + R_{j_1 j_2} E_1 \begin{bmatrix}
    -1 \\ m_{j_2}  
  \end{bmatrix}
  + R_{j_1 j_2} E_1 \begin{bmatrix}
    -1 \\ m_{j_1} m_{j_2}q^{-1/2}  
  \end{bmatrix}
  \right) \nonumber\\
  & \ + \delta_{n_2 \ne 0}\sum_{j_1, j_2 = 1}^{6}
    R_{j_1j_2}\frac{m_{j_1}^{n_1}q^{\frac{1}{2} n_1}}{q^{n_1} - 1}
    \frac{(m_{j_2}q^{\frac{1}{2}})^{n_2}-(m_{j_1}^{-1}m_{j_2}^{-1})^{n_2}q^{n_2}}{q^{n_2}-1} \nonumber \\
  & \ + \delta_{n_2 = n_1}\sum_{j_1 = 1}^{6} \frac{m_{j_1}^{-n_1}q^{\frac{n_1}{2}}}{q^{n_1} - 1} \left(
  R_{j_10} + R_{j_1 j_2} E_1 \begin{bmatrix}
    -1 \\ m_{j_2}  
  \end{bmatrix}
  + R_{j_1 j_2} E_1 \begin{bmatrix}
    -1 \\ m_{j_1} m_{j_2}q^{-1/2}  
  \end{bmatrix}
  \right) \\
  & \ - \delta_{n_2 \ne n_1}\sum_{j_1, j_2 = 1}^{6}
    R_{j_1j_2}
    \frac{m_{j_1}^{-n_1} q^{\frac{n_1}{2}}}{q^{n_1} - 1}
    \frac{m_{j_2}^{n_2-n_1}q^{\frac{1}{2}(n_2-n_1)}-(m_{j_1}^{-1}m_{j_2}^{-1})^{n_2-n_1}q^{n_2-n_1}}{q^{n_2-n_1}-1} \ . \nonumber
\end{align}

For example, for the Wilson operator in the anti-fundamental representation,
\begin{align}
  \langle W_{\overline {\mathbf{3}}}\rangle_{SU(3) \ \text{SQCD}}
  = & \ 3 \oint \frac{da_1}{2\pi i a_1}\frac{da_2}{2\pi i a_2} a_1^{-1} \mathcal{Z} \nonumber \\
  = & \  - 3 \frac{q^{\frac{1}{2}}}{1 - q} \sum_{j_1 = 1}^{6}m_{j_1}^{-1}\left(
  R_{j_10} + R_{j_1 j_2} E_1 \begin{bmatrix}
    -1 \\ m_{j_2}  
  \end{bmatrix}
  + R_{j_1 j_2} E_1 \begin{bmatrix}
    -1 \\ m_{j_1} m_{j_2}q^{-1/2}  
  \end{bmatrix}
  \right) \nonumber \\
  & \  + 3 \frac{1}{1-q} \sum_{j_1, j_2 = 1}^{6}
    R_{j_1j_2}
    \frac{m_{j_2}^{-1}q^{\frac{1}{2}} - m_{j_1} m_{j_2}q}{q-1}
\end{align}




\subsection{\texorpdfstring{ $\mathcal{N} = 4$ $SO(4)$ SYM}{}}

The Lie algebra $\mathfrak{so}(4)$ is isomorphic to $\mathfrak{su}(2)^2$. The Schur index of a Lagrangian theory is only sensitive to the gauge Lie algebra, and therefore the $\mathcal{N} = 4$ $SO(4)$ and $SU(2)^2$ gauge theory share an identical Schur index,
\begin{align}
  \mathcal{I}_{SU(2)^2} = \mathcal{I}_{SO(4)}
  = \frac{1}{4}\eta(\tau)^{4} \frac{\eta(\tau)^2}{\vartheta_4(\mathfrak{b})^2} & \ \oint \prod_{A = 1}^{2} \frac{da_A}{2\pi i a_A} \prod_{\alpha, \beta = \pm}\prod_{A < B} \frac{\vartheta_1(\alpha \mathfrak{a}_A + \beta \mathfrak{a}_B)}{\vartheta_4 (\alpha \mathfrak{a}_A + \beta\mathfrak{a}_B + \mathfrak{b})} \nonumber \\
  \coloneqq & \ \oint \prod_{A = 1}^{2} \frac{da_A}{2\pi i a_A} \mathcal{Z}(\mathfrak{a}_1, \mathfrak{a}_2) \ .
\end{align}
In the following we will compute a few full Wilson operator index and compare it with the $S$-dual `t Hooft operator index using the formula in \cite{Gang:2012yr}. 


We first analyze the index of a full Wilson operator associated to the vector representation $\mathbf{4}$ and its $S$-dual. The full Wilson index reads
\begin{align}
  \langle W^\text{full}_{\mathbf{4}}\rangle_{SO(4) \ \mathcal{N} = 4} = \oint \prod_{A = 1}^{2} \frac{da_A}{2\pi i a_A} (a_1 + \frac{1}{a_1} + a_2 + \frac{1}{a_2})^2 \mathcal{Z} \ .
\end{align}
By a change of variables $\mathfrak{a}_1' \coloneqq \mathfrak{a}_1 + \mathfrak{a}_2$ and $\mathfrak{a}'_2 \coloneqq \mathfrak{a}_1 - \mathfrak{a}_2$,  the Wilson index can be rewritten as a product
\begin{align}
  \langle W^\text{full}_{\mathbf{4}}\rangle_{SO(4) \ \mathcal{N} = 4}
  = \left[- \frac{1}{2} \oint \frac{da}{2\pi i a} \frac{(a + 1)^2}{a} \frac{\vartheta_1(\pm \mathfrak{a})}{\vartheta_4(\pm \mathfrak{a} + \mathfrak{b})} \frac{\eta(\tau)^3}{\vartheta_4(\mathfrak{b})}\right]^2 \ ,
\end{align}
which is identical to
\begin{align}
  (\langle W_{j = 1/2}^\text{full} \rangle_{SU(2) \ \mathcal{N} = 4})^2 \ .
\end{align}
The vector representation of $SO(4)$ is minuscule, and the S-dual `t Hooft index is safe from monopole bulling, given by
\begin{align}
  \langle H\rangle_{SO(4) \ \mathcal{N} = 4}
  = \oint \prod_{A = 1}^{2} \frac{da_A}{2\pi i a_A}
  \frac{4q^{\frac{1}{4}} (ba_1 - a_2)(-a_1 + ba_2)(b - a_1 a_2)(-1 + b a_1 a_2)}{b^2(\sqrt{q}a_1 - a_2)(\sqrt{q}a_2 - a_1)(\sqrt{q} - a_1 a_2)(-1 + \sqrt{q}a_1 a_2 ) } \mathcal{Z}' \ , \nonumber
\end{align}
where
\begin{align}
  \mathcal{Z}' = \frac{1}{4} \eta(\tau)^4 \frac{\eta(\tau)^2}{\vartheta_4(\mathfrak{b})^2}\prod_{\alpha, \beta = \pm} \frac{\vartheta_4(\alpha \mathfrak{a}_1 + \beta \mathfrak{a}_2)}{\vartheta_1(\alpha \mathfrak{a}_1 + \beta \mathfrak{a}_2 + \mathfrak{b})} \ .
\end{align}
In terms of the $a'$ variables, the above factorizes into
\begin{align}
  \langle H\rangle_{SO(4) \ \mathcal{N} = 4} = \left[\oint \frac{da'}{2\pi i a'_1}\frac{q^{\frac{1}{8}}(b - a'_1)(-1 + b a'_1)}{b(\sqrt{q} - a'_1)(-1 + \sqrt{q}a'_1)}\frac{\eta(\tau)^3}{\vartheta_4(\mathfrak{b})} \frac{\vartheta_4(\pm \mathfrak{a}')}{\vartheta_1(\pm\mathfrak{a}' + \mathfrak{b})}\right]^2 \ .
\end{align}
Up to the square and some simple factors, the result is identical to that of the $U(2)$ minimal `t Hooft operator index (\ref{U2-t-hooft}) in section \ref{section:N4SU(2)}, and naturally
\begin{align}
  \langle H\rangle_{SO(4) \ \mathcal{N} = 4} = \langle W^\text{full}_{\mathbf{4}}\rangle_{SO(4) \ \mathcal{N} = 4} \ .
\end{align}




Next we consider the index of a full Wilson operator in chiral spinor representation $\mathbf{2}$. The corresponding character is
\begin{align}
  \chi_\mathbf{2}(a) = \frac{1}{\sqrt{a_1 a_2}} + \sqrt{a_1 a_2} \ ,
\end{align}
and the relevant index is given by
\begin{align}
  \langle W_\mathbf{2}^\text{f}\rangle_{SO(4) \ \mathcal{N} = 4} = & \ \oint \prod_{A = 1}^{2} \frac{da_A}{2\pi i a_A}
  \chi_{\mathbf{2}}(a)^2 
  \mathcal{Z}(\mathfrak{a}_1, \mathfrak{a}_2) \\
  = & \ \oint \prod_{A = 1}^{2} \frac{da_A}{2\pi i a_A}
  (1 + 1 + a_1 a_2 + \frac{1}{a_1 a_2})
  \mathcal{Z}(\mathfrak{a}_1, \mathfrak{a}_2) \ .
\end{align}
In terms of the $a'$ variable, the above factorizes
\begin{align}
  \langle W_\mathbf{2}^\text{f}\rangle_{SO(4) \ \mathcal{N} = 4}
  = & \ \left[\oint \frac{da'_1}{2\pi i a'_1}(\chi_{j = 0} + \chi_{j = 1})(a'_1) \left(- \frac{1}{2}\right)\frac{\eta(\tau)^3}{\vartheta_4(\mathfrak{b})} \frac{\vartheta_1(\pm\mathfrak{a}'_1)}{\vartheta_1(\pm\mathfrak{a}'_1 + \mathfrak{b})}\right] \mathcal{I}_{SU(2) \ \mathcal{N} = 4} \nonumber \\
  = & \ \left( \mathcal{I}_{\mathcal{N} = 4 \ SU(2)} + \langle W^\text{full}_{j = 1}\rangle_{\mathcal{N} = 4 \ SU(2)} \right) \mathcal{I}_{SU(2) \ \mathcal{N} = 4} \ .
\end{align}

The S-dual `t Hooft line index is given by
\begin{align}
  \langle H\rangle
  = \oint \prod_{A = 1}^{2} \frac{da_A}{2\pi i a_A}
  \frac{2(b - a_1 a_2)(-1 + b a_1a_2)}{bq^{\frac{1}{4}}(\sqrt{q} - a_1 a_2)(-1 + \sqrt{q} a_1 a_2)}
  \mathcal{Z}' \ , \nonumber
\end{align}
where
\begin{equation}
  \mathcal{Z}' = \frac{1}{4}\eta(\tau)^4 \frac{\eta(\tau)^2}{\vartheta_4(\mathfrak{b})^2}
  \frac{\vartheta_4(\pm (\mathfrak{a}_1 + \mathfrak{a}_2))}{\vartheta_1(\pm (\mathfrak{a}_1 + \mathfrak{a}_2) + \mathfrak{b})}
  \frac{\vartheta_1(\pm (\mathfrak{a}_1 - \mathfrak{a}_2))}{\vartheta_4(\pm (\mathfrak{a}_1 - \mathfrak{a}_2) + \mathfrak{b})}  \ .
\end{equation}
In terms of the $a'$ variables,
\begin{align}
  \langle H\rangle
  = \left[- \oint \frac{da'_1}{2\pi i a'_1} \frac{(b - a'_1)(-1 + b a_1' )}{bq^{1/4}(\sqrt{q} - a'_1)(-1 + \sqrt{q}a'_1)} \frac{\vartheta_4(\pm \mathfrak{a}_1)}{\vartheta_1(\pm \mathfrak{a}_1 + \mathfrak{b})} \frac{\eta(\tau)^3}{\vartheta_4(\mathfrak{b})}\right] \mathcal{I}_{\mathcal{N} = 4 \ SU(2)} \ .
\end{align}
The equality from S-duality also follows from the discussion in section \ref{section:N4SU(2)}.




\subsection{\texorpdfstring{$\mathcal{N} = 4$ $SO(5)$ SYM}{}}



Let us now consider $\mathcal{N} = 4$ $SO(5)$ SYM with insertion of a half Wilson operator in the fundamental representation
\begin{align}
  \oint \frac{da_1}{2\pi i a_1}\frac{da_2}{2\pi i a_2}\chi_{\mathbf{5}}(a)
  \mathcal{Z}(\mathfrak{a}_1, \mathfrak{a}_2) \ ,
\end{align}
where
\begin{equation}
\mathcal{Z}(\mathfrak{a}_1, \mathfrak{a}_2) = \frac{1}{8} \frac{\eta(\tau)^6}{\vartheta_4(\mathfrak{b})^2} \frac{
    - \vartheta_1(\mathfrak{a}_1)^2
    \vartheta_1(\mathfrak{a}_2)^2
    \vartheta_1(\mathfrak{a}_1 + \mathfrak{a}_2)^2
    \vartheta_1(\mathfrak{a}_1 - \mathfrak{a}_2)^2
  }{
    \vartheta_4(\mathfrak{a}_1 \pm \mathfrak{b})
    \vartheta_4(\mathfrak{a}_2 \pm \mathfrak{b})
    \vartheta_4(\mathfrak{a}_1 + \mathfrak{a}_2\pm \mathfrak{b})
    \vartheta_4(\mathfrak{a}_1 - \mathfrak{a}_2\pm \mathfrak{b})
  }\ ,
\end{equation}
and
\begin{align}
  \chi_{\mathbf{5}}(a) = a_1 + \frac{1}{a_1} + a_2 + \frac{1}{a_2} + 1 \ .
\end{align}

From the symmetry between $\mathcal{Z}(\mathfrak{a}_1, \mathfrak{a}_2) = \mathcal{Z}(\mathfrak{a}_2, \mathfrak{a}_1)$, we only need to compute
\begin{align}
  \oint \frac{da_1}{2\pi i a_1}\frac{da_2}{2\pi i a_2}a_1^{\pm 1}
  \mathcal{Z}(\mathfrak{a}_1, \mathfrak{a}_2) \ .
\end{align}
Moreover, the symmetry $\mathcal{Z}(\mathfrak{a}_1, \mathfrak{a}_2) = \mathcal{Z}( - \mathfrak{a}_1, \mathfrak{a}_2)$ also implies
\begin{align}
  \oint \frac{da_1}{2\pi i a_1}\frac{da_2}{2\pi i a_2}a_1
  \mathcal{Z}(\mathfrak{a}_1, \mathfrak{a}_2)
  = \oint \frac{da_1}{2\pi i a_1}\frac{da_2}{2\pi i a_2}a_1^{-1}
  \mathcal{Z}(\mathfrak{a}_1, \mathfrak{a}_2) \ .
\end{align}



The $a_1$-integration picks up imaginary poles
\begin{align}
  \mathfrak{a}_1 = \alpha \mathfrak{b} + \frac{\tau}{2}, \qquad
  \mathfrak{a}_1 = \beta \mathfrak{a}_2 + \gamma \mathfrak{b} + \frac{\tau}{2} \ , \qquad \alpha, \beta, \gamma = \pm \ ,
\end{align}
with residues respectively
\begin{align}
  R_\alpha \coloneqq \frac{i}{8}\eta(\tau)^3 \frac{\vartheta_4(\mathfrak{a}_2 + \alpha \mathfrak{b})\vartheta_4(\mathfrak{a}_2 - \alpha \mathfrak{b})}{
        \vartheta_1(2 \alpha \mathfrak{b})
        \vartheta_1(\mathfrak{a}_2 + 2 \alpha \mathfrak{b})
        \vartheta_1(\mathfrak{a}_2 - 2 \alpha \mathfrak{b})} \ ,
\end{align}
and
\begin{align}
  R_{\beta \gamma} \coloneqq\frac{i}{8} \eta(\tau)^3 \frac{
    \vartheta_4(\mathfrak{a}_2 + \beta \gamma \mathfrak{b})
    \vartheta_1(\mathfrak{a}_2)
    \vartheta_4(2 \mathfrak{a}_2 + \beta \gamma \mathfrak{b})^2
  }{
    \vartheta_1(\mathfrak{a}_2 + 2 \beta \gamma \mathfrak{b})
    \vartheta_4(\mathfrak{a}_2 - \beta \gamma \mathfrak{b})
    \vartheta_1(2\mathfrak{a}_2 )
    \vartheta_1(2 \gamma \mathfrak{b}) \vartheta_1(2 \mathfrak{a}_2 + 2 \beta \gamma \mathfrak{b})
  } \ .
\end{align}
The $a_1$-integration leaves us with
\begin{align}
  \oint \frac{da_2}{2\pi i a_2} \left[
  - \sum_{\alpha = \pm} R_\alpha \frac{1}{q^\pm - 1} (b^\alpha q^{\frac{1}{2}})^\pm
  - \sum_{\beta \gamma = \pm} R_{\beta \gamma} \frac{1}{q^\pm - 1} (a_2^\beta b^\gamma q^{\frac{1}{2}})^\pm
  \right] \ .
\end{align}
The residues $R_\alpha$ and $R_{\beta \gamma}$ are all elliptic with respect to $\mathfrak{a}_2$, and therefore the $a_2$-integration of both terms can be carried out. In $R_\alpha$, there are poles and residues
\begin{align}
  \mathfrak{a}_2 = 2 \alpha \delta \mathfrak{b}, \qquad
  \mathop{\operatorname{Res}}_{\mathfrak{a}_2 = 2\alpha \delta \mathfrak{b}}R_\alpha = - \frac{\delta}{8}\frac{\vartheta_4(3 \mathfrak{b}) \vartheta_4(\mathfrak{b})}{\vartheta_1(2 \mathfrak{b}) \vartheta_1(4 \mathfrak{b})} \ .
\end{align}
Hence
\begin{align}
  & \ - \oint \frac{da_2}{2\pi i a_2} \sum_{\alpha = \pm}R_\alpha \frac{1}{q^{\pm} - 1} (b^\alpha q^{\frac{1}{2}})^\pm \\
  = & \ - \sum_{\alpha = \pm} \frac{b^{\pm\alpha} q^{\pm \frac{1}{2}}}{q^{\pm} - 1}
  \left(
  R_\alpha(\mathfrak{a} = 0) + \sum_{\delta = \pm} \frac{- \delta}{8} \frac{\vartheta_4(3 \mathfrak{b}) \vartheta_4(\mathfrak{b})}{\vartheta_1(2 \mathfrak{b}) \vartheta_1(4 \mathfrak{b})} E_1
  \begin{bmatrix}
    -1 \\ b^{2\alpha \delta}q^{\frac{1}{2}}  
  \end{bmatrix}
  \right) \ .
\end{align}
By direct computation, one sees that the above is actually independent of $\pm$ sign in the $a_1^\pm$ insertion, consistent with the symmetry $\mathcal{Z}(\mathfrak{a}_1, \mathfrak{a}_2) = \mathcal{Z}( - \mathfrak{a}_1, \mathfrak{a}_2)$.


The term with $R_{\beta \gamma}$ can be carried using (\ref{integration-formula-monomial}),
\begin{align}
  \oint \frac{da_2}{2\pi i a_2}R_{\beta \gamma} a_2^{\pm \beta}
  = - \sum_{\operatorname{real} \ j} R_{\beta \gamma j} \frac{(a^{(\beta \gamma j)}_{2}q)^{\pm \beta}}{q^{\pm \beta} - 1}
  - \sum_{\operatorname{img} \ j} R_{\beta \gamma j}\frac{(a^{(\beta \gamma j)}_{2})^{\pm \beta}}{q^{\pm \beta} - 1} \ .
\end{align}
Here $a^{(\beta \gamma j)}_{2}$ denotes the simple poles of $R_{\beta \gamma}$ with respect to $a_2$, with the corresponding residue $R_{\beta \gamma j}$. We list the poles and their residues in Table \ref{poles-residues-SO(5)}.
{
\renewcommand{\arraystretch}{1.8}
\begin{table}[h!]
\centering
  \begin{tabular}{c|c|c}
    & poles $a_2^{(\beta \gamma j)}$ & residues $R_{\beta \gamma j}$ \\
    \hline
    Real & $\mathfrak{a}_2 = - 2\beta \gamma \mathfrak{b}$ & $ + \frac{\beta}{8} \frac{\vartheta_4( \mathfrak{b})\vartheta_4(3 \mathfrak{b})}{\vartheta_1(2  \mathfrak{b})\vartheta_1(4  \mathfrak{b})}$\\
    & $\mathfrak{a}_2 = - \beta \gamma \mathfrak{b} + \frac{1}{2}$ & $ + \frac{\beta \vartheta_4(\mathfrak{b})^2 \vartheta_3(0)}{16 \vartheta_1(2 \mathfrak{b})^2 \vartheta_3(2 \mathfrak{b})}$\\
    & $\mathfrak{a}_2 = \frac{1}{2}$ & $ - \frac{\beta \vartheta_4(\mathfrak{b})
    ^2 \vartheta_2(0)}{16 \vartheta_1(2 \mathfrak{b})^2 \vartheta_2(2 \mathfrak{b})}$\\
    & $\mathfrak{a}_2 = - \beta \gamma \mathfrak{b}$ & $ - \frac{\beta \vartheta_4(\mathfrak{b})^2 \vartheta_4(0)}{16 \vartheta_1(2 \mathfrak{b})^2 \vartheta_4(2 \mathfrak{b})}$\\
    \hline
    Imaginary & $\mathfrak{a}_2 = \beta \gamma \mathfrak{b} + \frac{1}{2}$ & $- \frac{\beta}{8} \frac{\vartheta_4( \mathfrak{b})\vartheta_4(3  \mathfrak{b})}{\vartheta_1(2 \mathfrak{b})\vartheta_1(4 \mathfrak{b})}$\\
    & $\mathfrak{a}_2 = \frac{\tau}{2}$ & $\frac{\beta \vartheta_4(\mathfrak{b})^2 \vartheta_4(0)}{16 \vartheta_1(2 \mathfrak{b})^2 \vartheta_4(2 \mathfrak{b})}$\\
    & $\mathfrak{a}_2 = \frac{1}{2} + \frac{\tau}{2}$ & $ - \frac{\beta \vartheta_4(\mathfrak{b})^2 \vartheta_3(0)}{16 \vartheta_1(2 \mathfrak{b})^2 \vartheta_3(2 \mathfrak{b})}$\\
    & $\mathfrak{a}_2 = - \beta \gamma \mathfrak{b} + \frac{1}{2} + \frac{\tau}{2}$ & $ + \frac{\beta \vartheta_4(\mathfrak{b})^2 \vartheta_2(0)}{16 \vartheta_1(2 \mathfrak{b})^2 \vartheta_2(2 \mathfrak{b})}$
  \end{tabular}
  \caption{Poles and residues of the elliptic functions $R_{\beta \gamma}$.\label{poles-residues-SO(5)}}
\end{table}
}

Performing the sum over $\beta, \gamma$,
\begin{align}
  & \ - \oint \frac{da_2}{2\pi i a_2}\sum_{\beta \gamma = \pm 1} R_{\beta \gamma} \frac{1}{q^{\pm} - 1} (a_2^\beta b^\gamma q^{\frac{1}{2}})^\pm \nonumber \\
  = & \ \frac{\sqrt{q}\left( b^2(q+1)-4b\sqrt{q}+q+1 \right)}{2}\frac{\vartheta _4(\mathfrak{b} )^2}{8b(q-1)^2\vartheta _1(2\mathfrak{b} )^2}\frac{\vartheta _2(0)}{\vartheta _2(2\mathfrak{b} )} \nonumber
  \\
  & \ + ( b^2q-b\sqrt{q}(q+1)+q ) \frac{\vartheta _4(\mathfrak{b} )^2}{8b(q-1)^2\vartheta _1(2\mathfrak{b} )^2}\left[ \frac{\vartheta _3(0)}{\vartheta _3(2\mathfrak{b} )}+\frac{\vartheta _4(0)}{\vartheta _4(2\mathfrak{b} )} \right]  \nonumber\\
  & \ +\frac{\sqrt{q} \left(-2 b^4 \sqrt{q}+b^3 (q+1)+b (q+1)-2 \sqrt{q}\right) }{8 b^2 (q-1)^2} \frac{\vartheta_4(3 \mathfrak{b}) \vartheta_4(\mathfrak{b})}{ \vartheta_1(2 \mathfrak{b}) \vartheta_1(4 \mathfrak{b})} \ ,
\end{align}
which is independent of the $\pm$ in $a_2^\pm$, consistent with the symmetry $\mathcal{Z}(\mathfrak{a}_1, \mathfrak{a}_2) = \mathcal{Z}( - \mathfrak{a}_1, \mathfrak{a}_2)$.

To summarize,
\begin{align}
  & \ \oint \frac{da_1}{2\pi i a_1}\frac{da_2}{2\pi i a_2}a_1
  \mathcal{Z}(\mathfrak{a}_1, \mathfrak{a}_2) \\
  = & \ \frac{i(b^2 - 1)\sqrt{q}}{8b(q-1)} \frac{\eta(\tau)^3 \vartheta_4(\mathfrak{b})^2}{\vartheta_1(2 \mathfrak{b})^3}\\
  & \ + \frac{\sqrt{q}(1 - 4b \sqrt{q} + q + b^2(1 + q))}{16 b(q -1)^2}
  \frac{\vartheta_4(\mathfrak{b})^2}{\vartheta_1(2 \mathfrak{b})^2}
  \frac{\vartheta_2(0)}{\vartheta_2(2 \mathfrak{b})}\\
  & \ + \frac{\sqrt{q}(b - \sqrt{q})(b\sqrt{q} - 1)}{8b (q-1)^2} \frac{\vartheta_4(\mathfrak{b})^2}{\vartheta_1(2 \mathfrak{b})^2}
  \left(
  \frac{\vartheta_3(0)}{\vartheta_3(2 \mathfrak{b})}
  + \frac{\vartheta_4(0)}{\vartheta_4(2 \mathfrak{b})}
  \right)\\
  & \ + \frac{\vartheta_4(\mathfrak{b})\vartheta_4(3 \mathfrak{b})}{\vartheta_1(2 \mathfrak{b}) \vartheta_1(4 \mathfrak{b})}
  \left(
  \frac{\sqrt{q}(b - \sqrt{q})(1 - b^3 \sqrt{q})}{4b^2 (q - 1)^2}
  + \frac{(b^2 -1)\sqrt{q}}{4b(q - 1)} E_1 \begin{bmatrix}
    -1 \\ b^2 q^{\frac{1}{2}}  
  \end{bmatrix}
  \right) \ .
\end{align}
Therefore,
\begin{align}
  \langle W_\mathbf{5}\rangle_{\mathcal{N} = 4 \ SO(5)}
  = & \ \mathcal{I}_{\mathcal{N} = 4 \ SO(5)}
  + \frac{i(b^2 - 1)\sqrt{q}}{2b(q-1)} \frac{\eta(\tau)^3 \vartheta_4(\mathfrak{b})^2}{\vartheta_1(2 \mathfrak{b})^3} \nonumber\\
  & \ + \frac{\sqrt{q}(1 - 4b \sqrt{q} + q + b^2(1 + q))}{4 b(q -1)^2}
  \frac{\vartheta_4(\mathfrak{b})^2}{\vartheta_1(2 \mathfrak{b})^2}
  \frac{\vartheta_2(0)}{\vartheta_2(2 \mathfrak{b})} \nonumber\\
  & \ + \frac{\sqrt{q}(b - \sqrt{q})(b\sqrt{q} - 1)}{2b (q-1)^2} \frac{\vartheta_4(\mathfrak{b})^2}{\vartheta_1(2 \mathfrak{b})^2}
  \left(
  \frac{\vartheta_3(0)}{\vartheta_3(2 \mathfrak{b})}
  + \frac{\vartheta_4(0)}{\vartheta_4(2 \mathfrak{b})}
  \right)\\
  & \ + \frac{\vartheta_4(\mathfrak{b})\vartheta_4(3 \mathfrak{b})}{\vartheta_1(2 \mathfrak{b}) \vartheta_1(4 \mathfrak{b})}
  \left(
  \frac{\sqrt{q}(b - \sqrt{q})(1 - b^3 \sqrt{q})}{b^2 (q - 1)^2}
  + \frac{(b^2 -1)\sqrt{q}}{b(q - 1)} E_1 \begin{bmatrix}
    -1 \\ b^2 q^{\frac{1}{2}}  
  \end{bmatrix}
  \right) \ , \nonumber
\end{align}
where the $\mathcal{I}_{\mathcal{N} = 4 \ SO(5)}$ is the original Schur index of the $SO(5)$ $\mathcal{N} = 4$ SYM.



\subsubsection{General representation}

Let us consider the $\mathfrak{so}(5)$ representations whose characters can be written as polynomials of $a_1, a_2$ with integral powers,
\begin{align}
  \chi_\mathcal{R}(a) = \sum_{n_1, n_2} c_{n_1 n_2} a_1^{n_1} a_2^{n_2} \ .
\end{align}
In particular, using the symmetry $a_1 \leftrightarrow a_2$, $a_i \leftrightarrow a_i^{-1}$, we can focus on the integrals of the following form
\begin{align}
  \oint \frac{da_1}{2\pi i a_1} \frac{da_2}{2\pi i a_2}
  a_1^{n_1 > 0}a_2^{n_2 \ge 0} \mathcal{Z} \ .
\end{align}
The $a_1$-integration leaves (recall that the $a_1$-integral picks up $6$ imaginary poles)
\begin{align}\label{a1integral}
  \oint \frac{da_2}{2\pi i a_2} a_2^{n_2} \left[
  - \sum_{\alpha = \pm}R_\alpha \frac{b^{\alpha n_1} q^{\frac{1}{2} n_1}}{q^{n_1} - 1}
  - \sum_{\beta \gamma = \pm}R_{\beta \gamma} \frac{a_2^{n_1\beta} b^{n_1\gamma} q^{\frac{1}{2} n_1}}{q^{n_1} - 1}
  \right] \ .
\end{align}
Depending on whether $n_2 = 0$ or $n_2 \ne 0$ in the first term, and whether $n_2 \pm n_1 = 0$ in the second term, the integral leads to different closed-form result. When $n_2 = 0$, the first term integrates to
\begin{align}\label{Rintegrate}
  = \delta_{n_2 = 0} \frac{1}{4}\frac{\vartheta_4(\mathfrak{b})}{\vartheta_1(2 \mathfrak{b})}
  \left(
  \frac{i \eta(\tau)^3 \vartheta_4(\mathfrak{b})}{2 \vartheta_1(2 \mathfrak{b})^2} + \frac{\vartheta_4(3 \mathfrak{b})}{\vartheta_1(4 \mathfrak{b})}
  E_1 \begin{bmatrix}
    1 \\ b^2  
  \end{bmatrix}
  \right)
  \frac{b^{n_1} - b^{-n_1}}{q^{n_1/2} - q^{- n_1/2}} \ ,
\end{align}
while when $n_2 > 0$, it integrates to
\begin{align}
  & \ - \delta_{n_2 > 0} \sum_{\alpha = \pm} \frac{b^{n_1 \alpha} q^{\frac{1}{2}n_1}}{q^{n_1} - 1}
  \sum_{\delta = \pm 1}
  \left(- \frac{\delta}{8} \frac{\vartheta_4(3 \mathfrak{b}) \vartheta_4( \mathfrak{b})}{\vartheta_1(2 \mathfrak{b})\vartheta_1(4 \mathfrak{b})}\right)
  \frac{(b^{2\alpha \delta} q^{\frac{1}{2}})^{n_2}}{q^{n_2/2} - q^{- n_2/2}} \ \nonumber\\
  = & \ - \delta_{n_2 > 0}\frac{\vartheta_4(\mathfrak{b}) \vartheta_4(3 \mathfrak{b})}{8 \vartheta_1(2 \mathfrak{b}) \vartheta_1(4 \mathfrak{b})} \frac{(b^{n_1} - b^{-n_1})(b^{2n_2} - b^{-2n_2})}{(q^{n_1/2} - q^{- n_1/2})(1 - q^{-n_2})} \ .
\end{align}

In the second term, when $0 < n_1 \ne n_2$, we have $n_2 + n_1 \beta \ne 0$ for either $\beta = \pm 1$. In this case,
\begin{align}\label{RalphabetaIntegrate}
  & \ - \delta_{n_1 \ne n_2} \oint \frac{da_2}{2\pi i a_2}\sum_{\beta \gamma = \pm} \frac{b^{n_1 \gamma} q^{\frac{1}{2}n_1}}{q^{n_1} - 1} R_{\beta \gamma}
  a_2^{n_2 + n_1 \beta} \\
  = & \ + \delta_{n_1 \ne n_2}\sum_{\beta \gamma = \pm}\frac{b^{n_1 \gamma} q^{\frac{1}{2}n_1}}{q^{n_1} - 1}  \sum_{\text{real/img} \ j} R_{\beta \gamma j} \frac{(a_2^{(\beta \gamma j)} q^{\pm \frac{1}{2}})^{n_2 + n_1 \beta}}{q^{\frac{1}{2}(n_2 + n_1 \beta)} - q^{-\frac{1}{2}(n_2 + n_1 \beta)} } \ .
\end{align}
On the other hand, when $n_1 = n_2 > 0$, we have $n_2 + n_1 \beta = 0$ for $\beta = -1$, and $n_2 + n_1 \beta = 2n_1 \ne 0$ for $\beta = 1$. In this situation,
\begin{align}
  & \ - \oint \frac{da_2}{2\pi i a_2}\sum_{\beta \gamma = \pm}R_{\beta \gamma} \frac{a_2^{n_2 + n_1 \beta} b^{n_1 \gamma} q^{\frac{1}{2}n_1}}{q^{n_1} - 1}
  \nonumber\\
  = & \ \delta _{n_1=n_2}\sum_{\gamma =\pm}{\left[ \sum_{\text{real}/\text{img}\ j}{R_{+\gamma j}\frac{( a_{2}^{\left( +\gamma j \right)})^{2n_1} q^{\pm \frac{1}{2}2n_1}}{q^{n_1}-q^{-n_1}}} \right]}\frac{b^{n_1\gamma}q^{\frac{1}{2}n_1}}{q^{n_1}-1}\\
  & \ -\delta _{n_1=n_2}\sum_{\gamma =\pm}
  \frac{b^{n_1\gamma}q^{\frac{1}{2}n_1}}{q^{n_1}-1}
  \left( R_{-\gamma}\left( \mathfrak{a} _2=\mathfrak{a}_2^{(0)} \right) +\sum_{\text{real}/\text{img} \ j}{R_{-\gamma j}}E_1\begin{bmatrix}
    -1\\
    \frac{a_{2}^{\left( -\gamma j \right)}}{a_2^{(0)}}q^{\pm \frac{1}{2}}  
  \end{bmatrix} \right) \nonumber \ ,
\end{align}
where $a_2^{(0)}$ is a generic reference value, for example, $a_2^{(0)} = b^3$. In the above, we have used the poles and residues in Table \ref{poles-residues-SO(5)}. Putting all the contributions together, we deduce that for $n_1 > 0, n_2 \ge 0$,
\begin{align}
  & \ \oint\prod_{i = }^{2}\frac{da_i}{2\pi i a_i}a_1^{n_1}a_2^{n_2} \mathcal{Z}\\
  = & \ \delta_{n_2 = 0}\frac{1}{4}\frac{\vartheta_4(\mathfrak{b})}{\vartheta_1(2 \mathfrak{b})}
  \left(
  \frac{i \eta(\tau)^3 \vartheta_4(\mathfrak{b})}{2 \vartheta_1(2 \mathfrak{b})^2} + \frac{\vartheta_4(3 \mathfrak{b})}{\vartheta_1(4 \mathfrak{b})}
  E_1 \begin{bmatrix}
    1 \\ b^2  
  \end{bmatrix}
  \right)
  \frac{b^{n_1} - b^{-n_1}}{q^{n_1/2} - q^{- n_1/2}} \\
  & \ - \delta_{n_2 > 0}\frac{\vartheta_4(\mathfrak{b}) \vartheta_4(3 \mathfrak{b})}{8 \vartheta_1(2 \mathfrak{b}) \vartheta_1(4 \mathfrak{b})} \frac{(b^{n_1} - b^{-n_1})(b^{2n_2} - b^{-2n_2})}{(q^{n_1/2} - q^{- n_1/2})(1 - q^{-n_2})} \\
  & \ + \delta_{n_1 \ne n_2}\sum_{\beta \gamma = \pm}\frac{b^{n_1 \gamma} q^{\frac{1}{2}n_1}}{q^{n_1} - 1}  \sum_{\text{real/img} \ j} R_{\beta \gamma j} \frac{(a_2^{(\beta \gamma j)} q^{\pm \frac{1}{2}})^{n_2 + n_1 \beta}}{q^{\frac{1}{2}(n_2 + n_1 \beta)} - q^{-\frac{1}{2}(n_2 + n_1 \beta)} }\\
  & \ + \delta _{n_2=n_1}\sum_{\gamma =\pm}{\left[ \sum_{\text{real}/\text{img}\ j}{R_{+\gamma j}\frac{( a_{2}^{\left( +\gamma j \right)})^{2n_1} q^{\pm \frac{1}{2}2n_1}}{q^{n_1}-q^{-n_1}}} \right]}\frac{b^{n_1\gamma}q^{\frac{1}{2}n_1}}{q^{n_1}-1}\\
    & \ -\delta _{n_2=n_1}\sum_{\gamma =\pm}
    \frac{b^{n_1\gamma}q^{\frac{1}{2}n_1}}{q^{n_1}-1}
    \left( R_{-\gamma}\left( \mathfrak{a} _2=\mathfrak{a}_2^{(0)} \right) +\sum_{\text{real}/\text{img} \ j}{R_{-\gamma j}}E_1\begin{bmatrix}
      -1\\
      \frac{a_{2}^{\left( -\gamma j \right)}}{a_2^{(0)}}q^{\pm \frac{1}{2}}  
    \end{bmatrix} \right) \ .
\end{align}
The Wilson index corresponding to the $SO(5)$ representations with Dynkin labels $[n, 0]$ can be computed using the above integration formula by simple substitution, sine the corresponding character can be written as a sum of simple monomials,
\begin{align}
  \chi_{[n,0]}\left(a_1,a_2\right)
  = & \ \sum_{m=0}^n \sum_{j=0}^{m} \sum_{i=0}^{m}
  a_1^{j-i} a_2^{i+j-m} \\
  = & \ \lceil\frac{n+1}{2}\rceil+\sum_{m=0}^n\sum_{\substack{i=0\\i\neq m/2}}^m a_2^{2i-m}+\sum_{m=1}^n \sum_{\substack{i,j = 0 \\ i\ne j}}^n a_2^{i+j-m}a_1^{j-i} \\
  \sim & \ \lceil\frac{n+1}{2}\rceil+\sum_{m=0}^n\sum_{\substack{i=0\\i\neq m/2}}^m a_1^{|2i-m|}+\sum_{m=1}^n \sum_{\substack{i,j = 0 \\ i\ne j}}^n a_2^{|i+j-m|}a_1^{|j-i|} \ .
\end{align}
Here in the last line we have rewritten the expression using the symmetries $a_1 \leftrightarrow a_2$, $a_i \leftrightarrow a_i^{-1}$ of the integral, so that each term can be easily computed with the above integration formula. Although the Wilson line index can be computed straightforwardly simply by substitution, we are unfortunately unable to reorganize the final result in an elegant form, so we will refrain from presenting the final expression of $\langle W_{[n,0]}\rangle_{\mathcal{N} = 4 \ SO(5)}$ here.


% \YP{to be continued}
% \YW{For an $SO(5)$ irreducible representation with Dynkin label $[n_1,n_2]$, the character can be written as
% \begin{align}
% 	\chi_{[n_1,n_2]}(a_1,a_2)=\frac{\left|
% 		\begin{array}{cc}
% 			a_1^{n_1+\frac{n_2}{2}+\frac{3}{2}}-a_1^{-n_1-\frac{n_2}{2}-\frac{3}{2}} & a_2^{n_1+\frac{n_2}{2}+\frac{3}{2}}-a_2^{-n_1-\frac{n_2}{2}-\frac{3}{2}} \\
% 			a_1^{\frac{n_2}{2}+\frac{1}{2}}-a_1^{-\frac{n_2}{2}-\frac{1}{2}} & a_2^{\frac{n_2}{2}+\frac{1}{2}}-a_2^{-\frac{n_2}{2}-\frac{1}{2}} \\
% 		\end{array}
% 		\right|}{\left|
% 		\begin{array}{cc}
% 			a_1^{\frac{3}{2}}-a_1^{-\frac{3}{2}} & a_2^{\frac{3}{2}}-a_2^{-\frac{3}{2}} \\
% 			a_1^{\frac{1}{2}}-a_1^{-\frac{1}{2}} & a_2^{\frac{1}{2}}-a_2^{-\frac{1}{2}} \\
% 		\end{array}
% 		\right|}.
% \end{align}
% We shall concentrate on the case when $n_1=n$, $n_2=0$ first. In this case the character can be recast into:
% \begin{align}\label{character resummation}
% &\chi_{[n,0]}\left(a_1,a_2\right)=\sum_{m=0}^n \sum_{j=0}^{m} \sum_{i=0}^{m}a_2^{i+j-m}a_1^{j-i}\notag\\
% &=\sum_{m=0}^n\sum_{i=0}^m a_2^{2i-m}+\sum_{m=1}^n \sum_{m\geq i>j\geq 0}a_2^{i+j-m}a_1^{j-i}+\sum_{m=1}^n\sum_{m\geq j>i\geq 0}a_2^{i+j-m}a_1^{j-i}\notag\\
% &=\lceil\frac{n+1}{2}\rceil+\sum_{m=0}^n\sum_{\substack{i=0\\i\neq m/2}}^m a_2^{2i-m}+\sum_{m=1}^n \sum_{m\geq i>j\geq 0}a_2^{i+j-m}a_1^{j-i}+\sum_{m=1}^n\sum_{m\geq j>i\geq 0}a_2^{i+j-m}a_1^{j-i}
% \end{align}
% For the second term, we have to deal with the following integral:
% \begin{align}
% \sum_{m=0}^{n}\sum_{\substack{i=0\\ i\neq m/2}}^m \oint_{|a_1|=1}\oint_{|a_2|=1}\frac{da_1}{2\pi i a_1}\frac{da_2}{2\pi i a_2}a_2^{2i-m}\mathcal{Z}(a_1,a_2).
% \end{align}
% Performing the $a_2$ integral by using (\ref{a1integral}) it gives:
% \begin{align}
% \sum_{m=0}^{n}\sum_{\substack{i=0\\ i\neq m/2}}^m \oint_{|a_1|=1}\frac{da_1}{2\pi i a_1}\left(-\sum_{\alpha=\pm}R_{\alpha}\frac{b^{\alpha(2i-m)}q^{(2i-m)/2}}{q^{2i-m}-1}-\sum_{\beta\gamma=\pm}R_{\beta\gamma}\frac{a_1^{\beta(2i-m)}b^{\gamma(2i-m)}q^{(2i-m)/2}}{q^{2i-m}-1}\right)
% \end{align}
% Note that
% \begin{align}
% \sum_{m=0}^{n}\sum_{\substack{i=0\\ i\neq m/2}}^m f(2i-m)=\sum_{i=1}^n\left(f(i)+f(-i)\right)\lceil\frac{n-i+1}{2}\rceil,
% \end{align}
% the result can be recast into:
% \begin{align}
% -\sum_{i=1}^n \sum_{\alpha=\pm}2\lceil\frac{n-i+1}{2}\rceil\frac{b^{\alpha i}}{q^{i/2}-q^{-i/2}}\oint \frac{da_1}{2\pi ia_1}R_\alpha -\sum_{i=1}^n \sum_{\beta,\gamma=\pm}2\lceil\frac{n-i+1}{2}\rceil\frac{b^{\gamma i}}{q^{i/2}-q^{-i/2}}\oint \frac{da_1}{2\pi i a_1}R_{\beta\gamma}a_1^{\beta i}
% \end{align}
% Recall the integral formula (\ref{Rintegrate}) and (\ref{RalphabetaIntegrate}), we can obtain:
% \begin{align}\label{part1SO5SYM}
% &= \sum_{i=1}^n\sum_{\alpha,\delta}\lceil\frac{n-i+1}{2}\rceil\frac{b^{\alpha i}\delta}{4(q^{i/2}-q^{-i/2})}Q E_1\begin{bmatrix}
% -1\\
% b^{2\alpha\delta}q^{1/2}
% \end{bmatrix}\notag\\
% &-\sum_{\beta\gamma}\sum_{i=1}^n \sum_{\text{real/Imag}j}2\lceil\frac{n-i+1}{2}\rceil \frac{\beta b^{\gamma i}}{\left(q^{i/2}-q^{-i/2}\right)^2} R_{\beta\gamma j}\left(a_2^{(\beta\gamma j)}q^{\pm 1/2}\right)^{\beta i},
% \end{align}
% where $Q=\frac{\vartheta_4\left(3\mathfrak{b},q\right)\vartheta_4\left(\mathfrak{b},q\right)}{\vartheta_1\left(2\mathfrak{b},q\right)\vartheta_1\left(4\mathfrak{b},q\right)}$.

% Since the last two terms in (\ref{character resummation}) gives the same result after integration, we only need to compute
% \begin{align}
% \sum_{m=1}^n \sum_{0\leq j<i\leq m}\oint \frac{da_1}{2\pi ia_1}\frac{da_2}{2\pi ia_2}a_2^{i+j-m}a_1^{j-i}\mathcal{Z}(a_1,a_2).
% \end{align}
% Using (\ref{a1integral}), we can perform the $a_1$ integral:
% \begin{align}
% &=-\sum_{m=1}^n \sum_{0\leq j<i\leq m}\sum_{\alpha}\frac{b^{\alpha(j-i)}q^{(j-i)/2}}{q^{j-i}-1}\oint\frac{da_2}{2\pi i a_2}a_2^{i+j-m}R_\alpha\notag\\
% &-\sum_{m=1}^n \sum_{0\leq j<i\leq m} \sum_{\beta\gamma}\frac{b^{\gamma(j-i)}q^{(j-i)/2}}{q^{j-i}-1}\oint\frac{da_2}{2\pi i a_2}a_2^{i+j+\beta(j-i)-m}R_{\beta\gamma}
% \end{align}
% The first term from above equals to:
% \begin{align}\label{part2SO5SYM}
% &=\frac{1}{8}Q\sum_{m=1}^n \sum_{\substack{0\leq j<i\leq m\\i+j=m}}\sum_{\alpha\delta}\frac{\delta b^{\alpha(j-i)}}{q^{(j-i)/2}-q^{-(j-i)/2}}E_1\begin{bmatrix}
% -1\\
% b^{2\alpha\delta}q^{1/2}
% \end{bmatrix}\\
% &+\frac{1}{8}Q\sum_{m=1}^n \sum_{\substack{0\leq j<i\leq m\\i+j\neq m}}\sum_{\alpha\delta}\frac{\delta b^{\alpha(j-i)}}{q^{(j-i)/2}-q^{-(j-i)/2}}\frac{(b^{2\alpha\delta}q^{1/2})^{i+j-m}}{q^{(i+j-m)/2}-q^{-(i+j-m)/2}}
% \end{align}
% The second term gives:
% \begin{align}\label{part3SO5SYM}
% &-\sum_{m=1}^n \sum_{0\leq j<i\leq m} \sum_{\gamma}\frac{b^{\gamma(j-i)}q^{(j-i)/2}}{q^{j-i}-1}\left(\oint\frac{da_2}{2\pi i a_2}a_2^{2j-m}R_{+1\gamma}+\oint\frac{da_2}{2\pi i a_2}a_2^{2i-m}R_{-1\gamma}\right)\\
% &=-\sum_{m=1}^n \sum_{\substack{0\leq j<i\leq m\\j= m/2}}\sum_{\gamma}\frac{b^{\gamma(j-i)}q^{(j-i)/2}}{q^{j-i}-1}\sum_{\substack{\text{Real/Imag}\\J}}R_{+1\gamma J}E_1\begin{bmatrix}
% -1\\
% a_2^{(+1\gamma J)}q^{\pm\frac{1}{2}}
% \end{bmatrix}\\
% &-\sum_{m=1}^n \sum_{\substack{0\leq j<i\leq m\\j\neq m/2}}\sum_{\gamma}\frac{b^{\gamma(j-i)}q^{(j-i)/2}}{q^{j-i}-1}\sum_{\substack{\text{Real/Imag}\\J}}R_{+1\gamma J}\frac{\left(a_2^{(+1\gamma J)}q^{\pm 1/2}\right)^{2j-m}}{q^{(2j-m)/2}-q^{-(2j-m)/2}}\\
% &-\sum_{m=1}^n \sum_{\substack{0\leq j<i\leq m\\i= m/2}}\sum_{\gamma}\frac{b^{\gamma(j-i)}q^{(j-i)/2}}{q^{j-i}-1}\sum_{\substack{\text{Real/Imag}\\J}}R_{-1\gamma J}E_1\begin{bmatrix}
% 	-1\\
% 	a_2^{(-1\gamma J)}q^{\pm\frac{1}{2}}
% \end{bmatrix}\\
% &-\sum_{m=1}^n \sum_{\substack{0\leq j<i\leq m\\i\neq m/2}}\sum_{\gamma}\frac{b^{\gamma(j-i)}q^{(j-i)/2}}{q^{j-i}-1}\sum_{\substack{\text{Real/Imag}\\J}}R_{-1\gamma J}\frac{\left(a_2^{(-1\gamma J)}q^{\pm 1/2}\right)^{2i-m}}{q^{(2i-m)/2}-q^{-(2i-m)/2}}
% \end{align}
% Combine three parts (\ref{part1SO5SYM}), (\ref{part2SO5SYM}), and (\ref{part3SO5SYM}) together with the orginal Schur index, we can get the final result.

% Observing that only when $n_1=n\in \mathbb{N}_{>0}$ and $n_2=2m$, $m\in \mathbb{N}_{>0}$, the insertion of $\chi_{[n_1,n_2]}$ gives non-zero result after integration. The result above can certainly cover this case by replacing $n$ with $n+m$.}
% According to the integral formulas as follows
% \begin{align}
% & \oint_{|a_2|=1}\frac{da_2}{2\pi i a_2}a_2^n \mathcal{Z}(a_1,a_2)=-R(a_1,b,q)\frac{b^{n}-b^{-n}}{q^{n/2}-q^{-n/2}}-\sum_{\alpha,\beta=\pm}R_{\alpha\beta}(a_1,b,q)\frac{a_1^{\alpha n}b^{\beta % n}}{q^{n/2}-q^{-n/2}}\\
% & \oint_{|a_2|=1}\frac{da_2}{2\pi i a_2}R(a_2,b,q)=A\left(E_1\begin{bmatrix}
%	-1\\
%	b
%\end{bmatrix}+E_1\begin{bmatrix}
%	-1\\
%	b^3
% \end{bmatrix}\right)\\
% & \oint_{|a_2|=1}\frac{da_2}{2\pi i a_2}a_2^{n}R_{\alpha\beta}(a_2,b,q)=(-1)^n\alpha B\frac{\left( b^{-n\alpha\beta}q^{n/2}-1\right)}{2\left(q^{n/2}-q^{-n/2}\right)}q^{n/2}-(-1)^n \alpha % C\frac{q^{n/2}-b^{-n\alpha\beta}}{2\left(q^{n/2}-q^{-n/2}\right)}q^{n/2}\notag\\
% &-\alpha A\frac{b^{n\alpha\beta}q^{n/2}-b^{-2n\alpha\beta}}{q^{n/2}-q^{-n/2}}q^{n/2}+\alpha D\frac{q^{n/2}-b^{-n\alpha\beta}}{2\left(q^{n/2}-q^{-n/2}\right)}q^{n/2},
% \end{align}
% We have
% \begin{align}
% &\sum_{m=0}^n\sum_{\substack{i=0\\i\neq m/2}}^m\oint_{|a_1|=1}\frac{da_1}{2\pi ia_1}\oint_{|a_2|=1}\frac{da_2}{2\pi i a_2} a_2^{2i-m}\mathcal{Z}(a_1,a_2)\notag\\
% &=\sum_{m=0}^n\sum_{\substack{i=0\\i\neq m/2}}^m\oint_{|a_1|=1}\frac{da_1}{2\pi % ia_1}\left(-R(a_1,b,q)\frac{b^{2i-m}-b^{m-2i}}{q^{(2i-m)/2}-q^{-(2i-m)/2}}-\sum_{\alpha,\beta=\pm}R_{\alpha\beta}(a_1,b,q)\frac{a_1^{\alpha (2i-m)}b^{\beta % (2i-m)}}{q^{(2i-m)/2}-q^{-(2i-m)/2}}\right)\notag\\
% &=-\sum_{m=0}^n\sum_{\substack{i=0\\i\neq m/2}}^m\frac{b^{2i-m}-b^{m-2i}}{q^{(2i-m)/2}-q^{-(2i-m)/2}}A\left(E_1\begin{bmatrix}
%	-1\\
%	b
% \end{bmatrix}+E_1\begin{bmatrix}
%	-1\\
% 	b^3
% \end{bmatrix}\right)\notag\\
% &-\sum_{m=0}^n\sum_{\substack{i=0\\i\neq m/2}}^m \sum_{\alpha,\beta=\pm}\frac{b^{\beta (2i-m)}}{q^{(2i-m)/2}-q^{-(2i-m)/2}}\left((-1)^m\alpha B\frac{\left( % b^{-(2i-m)\beta}q^{\alpha(2i-m)/2}-1\right)}{2\left(q^{\alpha(2i-m)/2}-q^{-\alpha(2i-m)/2}\right)}q^{\alpha(2i-m)/2}\right.\notag\\
% &-(-1)^m \alpha C\frac{q^{\alpha(2i-m)/2}-b^{-(2i-m)\beta}}{2\left(q^{\alpha(2i-m)/2}-q^{-\alpha(2i-m)/2}\right)}q^{\alpha(2i-m)/2}-\alpha %A\frac{b^{(2i-m)\beta}q^{\alpha(2i-m)/2}-b^{-2(2i-m)\beta}}{q^{\alpha(2i-m)/2}-q^{-\alpha(2i-m)/2}}q^{\alpha(2i-m)/2}\notag\\
% &\left.+\alpha D\frac{q^{\alpha(2i-m)/2}-b^{-(2i-m)\beta}}{2\left(q^{\alpha(2i-m)/2}-q^{-\alpha(2i-m)/2}\right)}q^{\alpha(2i-m)/2}\right)
%\end{align}
%where:
%\begin{align}
%	A=\frac{\vartheta_4\left(3\mathfrak{b},q\right)\vartheta_4\left(\mathfrak{b},q\right)}{\vartheta_1\left(2\mathfrak{b},q\right)\vartheta_1\left(4\mathfrak{b},q\right)}\quad %B=\frac{\vartheta_2\left(0,q\right)\vartheta_4^2\left(\mathfrak{b},q\right)}{\vartheta_1^2\left(2\mathfrak{b},q\right)\vartheta_2\left(2\mathfrak{b},q\right)}\quad %C=\frac{\vartheta_3\left(0,q\right)\vartheta_4^2\left(\mathfrak{b},q\right)}{\vartheta_1^2\left(2\mathfrak{b},q\right)\vartheta_3\left(2\mathfrak{b},q\right)}\quad       %D=\frac{\vartheta_4\left(0,q\right)\vartheta_4^2\left(\mathfrak{b},q\right)}{\vartheta_1^2\left(2\mathfrak{b},q\right)\vartheta_4\left(2\mathfrak{b},q\right)}
%\end{align}
%To simplify a little bit, note that
%\begin{align}
%\sum_{m=0}^n \sum_{\substack{i=0\\ i\neq m/2}}^m f(2i-m)=\sum_{i=1}^{n}\left(f(i)+f(-i)\right)\lceil\frac{-i+n+1}{2}\rceil.
%\end{align}
%In this way,
% \begin{align}
% &\sum_{m=0}^n\sum_{\substack{i=0\\i\neq m/2}}^m\oint_{|a_1|=1}\frac{da_1}{2\pi ia_1}\oint_{|a_2|=1}\frac{da_2}{2\pi i a_2} a_2^{2i-m}\mathcal{Z}(a_1,a_2)\notag\\
% &=-2A\left(E_1\begin{bmatrix}
% -1\\
% b
% \end{bmatrix}+E_1\begin{bmatrix}
% -1\\
% b^3
% \end{bmatrix}\right)\sum_{i=1}^n \lceil\frac{n-i+1}{2}\rceil\frac{b^i-b^{-i}}{q^i-q^{-i}}\notag\\
% &-2\sum_{i=1}^n\lceil\frac{n-i+1}{2}\rceil\left(A\left(\frac{-(q^i+q^{-i})(b^{2i}+b^{-2i})+(b^{-i}+b^{i})(q^{i/2}+q^{-i/2})}{(q^{-i/2}-q^{i/2})^2}\right)\right.\notag\\
% &-(-1)^i B \left(\frac{-2(q^i+q^{-i})+(b^i+b^{-i})(q^{i/2}+q^{-i/2})}{2(-q^{i/2}+q^{-i/2})^2}\right)\notag\\
% &-(-1)^i C\frac{-2(q^{i/2}+q^{-i/2})+(q^i+q^{-i})(b^i+b^{-i}) }{2(-q^{i/2}+q^{-i/2})^2}\notag\\
% &\left. +D\frac{-2(q^{i/2}+q^{-i/2})+(b^i+b^{-i})(q^i+q^{-i})}{2(-q^{i/2}+q^{-i/2})^2}\right)\notag\\
% &=-2A\left(E_1\begin{bmatrix}
%	-1\\
%	b
%\end{bmatrix}+E_1\begin{bmatrix}
%	-1\\
%	b^3
% \end{bmatrix}\right)\sum_{i=1}^n \lceil\frac{n-i+1}{2}\rceil\frac{b^i-b^{-i}}{q^i-q^{-i}}\notag\\
% &+16\sum_{i=1}^n \sum_{\beta,\gamma=\pm}\sum_{\text{Real}j}\lceil\frac{n-i+1}{2}\rceil\beta R_{\beta\gamma j}\left(a_2^{(\beta\gamma j)}q\right)^{\beta i}\notag\\
% &+16\sum_{i=1}^n \sum_{\beta,\gamma=\pm}\sum_{\text{Imag}j}\lceil\frac{n-i+1}{2}\rceil\beta R_{\beta\gamma j}\left(a_2^{(\beta\gamma j)}\right)^{\beta i}.
%\end{align}
%Since the last two terms in (\ref{character resummation}) have the same contribution to the integral, we only need to deal with the integral as follows:
%\begin{align}
%\sum_{m=1}^{n}\sum_{m\geq i> j\geq 0}\oint \frac{da}{2\pi i a}a_2^{i+j-m}a_1^{j-i}\mathcal{Z}(a_1,a_2)
%\end{align}}















% %!TEX root = ../Schur indices and line operators.tex


\section{Discussion}


















\section*{Acknowledgments}
Y.P. is supported by the National Natural Science Foundation of China (NSFC) under Grant No. 11905301, the Fundamental Research Funds for the Central Universities, Sun Yat-sen University under Grant No. 2021qntd27. The work of W.P. was partially supported by grant \#{}494786 from the Simons Foundation.





\appendix
% !TEX root = ../Schur indices and line operators.tex

\section{Special Functions\label{app:special-functions}}

Through out this appendix and the paper, we adopt the convention that fraktur letters $\mathfrak{a}, \mathfrak{b},$ etc., are related to the standard letters by
\begin{align}
	a = e^{2\pi i \mathfrak{a}}, \qquad
	b = e^{2 \pi i \mathfrak{b}}, \qquad \cdots, \qquad
	z = e^{2\pi i \mathfrak{z}} \ .
\end{align}
except for the standard notation $q = e^{2\pi i \tau}$.


\subsection{\texorpdfstring{The Weierstrass $\zeta$-function}{}}

The Weierstrass $\zeta$-function is defined by
\begin{align}\label{weierstrasszetadef}
	\zeta(z) \coloneqq \frac{1}{z} + \sum'_{\substack{(m, n) \in \mathbb{Z}^2\\(m, n) \ne (0, 0)}}
	\left[\frac{1}{z - m - n \tau} + \frac{1}{m + n \tau} + \frac{z }{(m + n \tau)^2} \right]\ .
\end{align}
In the following and in the main text we will often abbreviate
\begin{align}
	\sum'_{\substack{(m, n) \in \mathbb{Z}^2\\(m, n) \ne (0, 0)}} \to \ \ \sum'_{m, n} \ , \qquad \sum_{\substack{m \in \mathbb{Z}\\m \ne 0}} \to \sum_m '\ .
\end{align}
The $\zeta$ function is not elliptic, and under pull period shift of $z$,
\begin{align}\label{shift-formula-zeta}
  \zeta(z + 1 | \tau) - \zeta(z| \tau) = & \ 2\eta_1(\tau)\\
  \zeta(z + \tau |\tau) - \zeta(z|\tau) = & \ 2 \eta_2(\tau) \equiv 2\tau \eta_1(\tau) - 2\pi i\ ,
\end{align}
where $\eta_1$ and $\eta_2$ are independent of $z$ and are both related to the Eisenstein series $E_2$. Note that $\zeta$ has a simple pole at each lattice point $m + n \tau$ with unit residue. The fact that $\zeta$ isn't fully elliptic is due to the fact that meromorphic function on $T^2$ with only one simple pole doesn't exist.

% \begin{itemize}

	% \item The Weierstrass $\wp$-function
	% \begin{align}
	% 	\wp(z) \coloneqq & \ \frac{1}{z^2} + \sum_{(m,n) \ne (0,0)} \left[\frac{1}{(z - m - n \tau)^2} - \frac{1}{(m + n \tau)^2}\right] \ .
	% \end{align}
	% This function is elliptic,
	% \begin{align}
	% 	\wp(z) = \wp(z + 1) = \wp(z + \tau) \ .
	% \end{align}
	% Following from the simple fact that $\partial_z z^{-1} = - z^{-2}$, one has
	% \begin{align}
	% 	\wp(z) = - \partial_z \zeta(z)\ .
	% \end{align}
	% By definition, $\wp$ has only one double pole on $T^2_\tau$.

	% \item The descendants $\partial_z^n \wp(z)$ are all elliptic functions, all with a single $n + 2$-th order pole on $T^2_\tau$.
% \end{itemize}






\subsection{Jacobi theta functions}


The standard Jacobi theta functions can be defined as infinite products of the $q$-Pochhammer symbol $(z;q) \coloneqq \prod_{k = 0}^{+\infty}(1 - zq)$,
\begin{align}\label{def:Jacobi-theta}
	\vartheta_1(\mathfrak{z}) \coloneqq & \ - i z^{\frac{1}{2}}q^{\frac{1}{8}}(q;q)(zq;q)(z^{-1};q),
	& \vartheta_2(\mathfrak{z}) \coloneqq & z^{1/2}q^{\frac{1}{8}}(q;q)(-zq;q)(-z^{-1};q) \ ,\\
	\vartheta_3(\mathfrak{z}) \coloneqq & \ (q;q)(-zq^{1/2};q)(-z^{-1}q^{1/2}),
	& \vartheta_4(\mathfrak{z}) \coloneqq & (q;q)(zq^{1/2};q)(z^{-1}q^{1/2};q) \ .
\end{align}
From the definition is easy to read off their simple zeros, for example,
\begin{align}
	\vartheta_1(m + n \tau) = 0, \qquad
	\vartheta_4(m + n \tau + \frac{\tau}{2}) = 0 \ , \qquad
	m, n \in \mathbb{Z} \ .
\end{align}
The Jacobi theta functions can also be rewritten in infinite series in $q$, or Fourier series in $\mathfrak{z}$,
\begin{align}
	\vartheta_1(\mathfrak{z}) = & \ -i \sum_{r \in \mathbb{Z} + \frac{1}{2}} (-1)^{r-\frac{1}{2}} e^{2\pi i r \mathfrak{z}} q^{\frac{r^2}{2}} ,
	& \vartheta_2(\mathfrak{z}) = & \sum_{r \in \mathbb{Z} + \frac{1}{2}} e^{2\pi i r \mathfrak{z}} q^{\frac{r^2}{2}} \ ,\\
	\vartheta_3(\mathfrak{z}) = & \ \sum_{n \in \mathbb{Z}} e^{2\pi i n \mathfrak{z}} q^{\frac{n^2}{2}},
	& \vartheta_4(\mathfrak{z}) = & \sum_{n \in \mathbb{Z}} (-1)^n e^{2\pi i n \mathfrak{z}} q^{\frac{n^2}{2}} \ .
\end{align}

The functions $\vartheta_i(z)$ share similar shift properties under the full period shifts,
\begin{align}
	\vartheta_{1,2}(\mathfrak{z} + 1) = & - \vartheta_{1,2}(\mathfrak{z}) , & 
	\vartheta_{3,4}(\mathfrak{z} + 1) = & + \vartheta_{3,4}(\mathfrak{z}) , & \\
	\vartheta_{1,4}(\mathfrak{z} + \tau) = & - \lambda \vartheta_{1,4}(\mathfrak{z}), & 
	\vartheta_{2,3}(\mathfrak{z} + \tau) = & + \lambda \vartheta_{2,3}(\mathfrak{z}) , & 
\end{align}
where $\lambda \coloneqq e^{-2\pi i \mathfrak{z}}e^{- \pi i \tau}$, while under half-period shifts which can be summarized as in the following diagram,
\begin{center}
	% Figure removed
\end{center}
where $\mu = e^{- \pi i \mathfrak{z}} e^{- \frac{\pi i}{4}}$, and $f \xrightarrow{a} g$ means
\begin{align}
	\text{either}\qquad  f\left(\mathfrak{z} + \frac{1}{2}\right) = a g(\mathfrak{z}) \qquad \text{or} \qquad
	f\left(\mathfrak{z} + \frac{\tau}{2}\right) = a g(\mathfrak{z}) \ ,
\end{align}
depending on whether the arrow is horizontal or (slanted) vertical respectively.

% The functions $\vartheta_{i = 2,3,4}(z | \tau)$ transform into each other under the modular $S$ and $T$ transformations, which act, as usual, on the nome and flavor fugacity as $(\frac{\mathfrak{z}}{\tau}, - \frac{1}{\tau})\xleftarrow{~~S~~}(\mathfrak{z}, \tau) \xrightarrow{~~T~~} (\mathfrak{z}, \tau + 1).$ In summary, with $\alpha = \sqrt{-i \tau}e^{\frac{\pi i z^2}{\tau}}$,
% \begin{center}
% 	% Figure removed
% \end{center}




Finally, we will frequently encounter residues of the $\vartheta$ functions. In particular,
\begin{align}\label{theta-function-residue}
	\mathop{\operatorname{Res}}\limits_{a \to b^{\frac{1}{n}}q^{\frac{k}{n} + \frac{1}{2n}}e^{2\pi i \frac{\ell}{n}}} \frac{1}{a} \frac{1}{\vartheta_4(n\mathfrak{a} - \mathfrak{b})} = & \ \frac{1}{n} \frac{1}{(q;q)^3} (-1)^k q^{\frac{1}{2} k (k + 1)} \ , \\
	\mathop{\operatorname{Res}}\limits_{a \to b^{\frac{1}{n}}q^{\frac{k}{n}}e^{2\pi i \frac{\ell}{n}}} \frac{1}{a} \frac{1}{\vartheta_1(n\mathfrak{a} - \mathfrak{b})} = & \ \frac{1}{n} \frac{i }{\eta(\tau)^3} (-1)^{k + \ell} q^{\frac{1}{2}k^2}\ .
\end{align}
Note that the $(-1)^\ell$ in the second line is related to the presence of a branch point at $z = 0$ according to \eqref{def:Jacobi-theta}. 





\subsection{Eisenstein series}


The twisted Eisenstein series are defined by the following infinite sum,
\begin{align}
	E_{k \ge 1}\left[\begin{matrix}
		\phi \\ \theta
	\end{matrix}\right] \coloneqq & \ - \frac{B_k(\lambda)}{k!} \\
	& \ + \frac{1}{(k-1)!}\sum_{r \ge 0}' \frac{(r + \lambda)^{k - 1}\theta^{-1} q^{r + \lambda}}{1 - \theta^{-1}q^{r + \lambda}}
	+ \frac{(-1)^k}{(k-1)!}\sum_{r \ge 1} \frac{(r - \lambda)^{k - 1}\theta q^{r - \lambda}}{1 - \theta q^{r - \lambda}} \ ,
\end{align}
where the parameter $\phi \equiv e^{2\pi i \lambda}$ with $\lambda \in [0, 1)$, $B_k(x)$ denotes the $k$-th Bernoulli polynomial, and the prime $^\prime$ in the summation means that the term with $r = 0$ should be omitted whenever $\phi = \theta = 1$. We also define $E_0\big[\substack{\phi \\ \theta}\big] = -1$.

The standard Eisenstein series $E_{2n}$ are the $\theta, \phi \to 1$ limit of the above Eisenstein series. When $k$ is odd, $\theta = \phi = 1$ gives zero except for the special instance with $k = 1$, where there is a simple pole $\mathfrak{z} \to 0$,
\begin{align}
	E_{2n}\left[\begin{matrix}
		+1 \\ +1
	\end{matrix}\right] = E_{2n} \ , \qquad
	E_{2n + 1 \ge 3}\left[\begin{matrix}
		+1 \\ +1
	\end{matrix}\right] = 0 
	\qquad
	E_1\left[\begin{matrix}
		+ 1 \\ z
	\end{matrix}\right] = \frac{1}{2\pi i }\frac{\vartheta'_1(\mathfrak{z})}{\vartheta_1(\mathfrak{z})}
	\ .
\end{align}

The Eisenstein series exhibits several useful properties. For example, the symmetry property
\begin{align}\label{Eisenstein-symmetry}
	E_k\left[\begin{matrix}
	  \pm 1 \\ z^{-1}
	\end{matrix}\right] = (-1)^k E_k\left[\begin{matrix}
	  \pm 1 \\ z
	\end{matrix}\right] \ .
\end{align}
% The twisted Eisenstein series of neighboring weights are related by
% \begin{align}\label{EisensteinDerivative}
% 	q \partial_q E_k\left[\begin{matrix}
% 		\phi \\ b
% 	\end{matrix}
% 	\right] = (- k) b \partial_b E_{k + 1}\left[\begin{matrix}
% 		\phi \\ b
% 	\end{matrix}
% 	\right]\ .
% \end{align}

When shifting the argument $\mathfrak{z}$ of the Eisenstein series by half or full periods of $\tau$, or equivalently, shifting $z$ by $q^{\frac{n}{2}}$, one has
\begin{align}\label{Eisenstein-half-shift}
	E_k\left[\begin{matrix}
		\pm 1\\ z q^{\frac{n}{2}}
	\end{matrix}\right]
	=
	\sum_{\ell = 0}^{k} \left(\frac{n}{2}\right)^\ell \frac{1}{\ell !}
	E_{k - \ell}\left[\begin{matrix}
		(-1)^n(\pm 1) \\ z
	\end{matrix}\right] \ , \qquad n \in \mathbb{Z} \ .
\end{align}
A simple consequence is that\footnote{In fact, these equalities remain true even after replacing $1$ by $e^{2\pi i \lambda}$ and $- 1$ by $e^{2\pi i (\lambda + \frac{1}{2})}$.}
\begin{align}\label{Eisenstein-shift-1}
	E_k\left[\begin{matrix}
		\pm 1 \\ zq^{\frac{1}{2}}
	\end{matrix}\right]
	- E_k\left[\begin{matrix}
		\pm 1 \\ zq^{ - \frac{1}{2}}
	\end{matrix}\right]
	= & \ \sum_{m = 0}^{\floor{\frac{k - 1}{2}}} \frac{1}{2^{2m}(2m+1)!}E_{k - 1 - 2m}\left[\begin{matrix}
		\mp 1\\z
	\end{matrix}\right] \ ,
\end{align}
or more generally
\begin{align}
	E_k\left[\begin{matrix}
		\pm 1 \\ zq^{\frac{1}{2} + n}
	\end{matrix}\right]
	- E_k\left[\begin{matrix}
		\pm 1 \\ zq^{ - \frac{1}{2} - n}
	\end{matrix}\right]
	= & \ 2\sum_{m = 0}^{\floor{\frac{k - 1}{2}}} \left(\frac{2n+1}{2}\right)^{2m + 1}\frac{1}{(2m+1)!}E_{k - 1 - 2m}\left[\begin{matrix}
		\mp 1\\z
	\end{matrix}\right] \ .
\end{align}


% With the Eisenstein series one define the twisted Elliptic-$P$ functions, generalizing the well-known Weierstrass $\wp$-family. In particular \cite{Mason:2008zzb},
% \begin{align}\label{P1}
% 	P_{k = 1}\left[\begin{matrix}
% 		\phi \\ \theta
% 	\end{matrix}\right](y) \coloneqq - \frac{1}{y}\sum_{m \ge 0}E_m\left[\begin{matrix}
% 		\phi \\ \theta
% 	\end{matrix}\right] y^m \ ,
% \end{align}
% while the remaining twisted-$P_k$ with higher $k$ are obtained by taking successively $y$-derivatives.

% With $P$, the difference equations can be further reorganized into the more compact formula
% \begin{align}\label{Delta-Eisenstein}
% 	\Delta_k \left[\begin{matrix}
% 		\pm 1 \\ z
% 	\end{matrix}\right]
% 	= - 2\oint_0 \frac{dy}{2\pi i} \frac{1}{y^k} \sinh \left(\frac{y}{2}\right) P_1\left[\begin{matrix}
% 		\mp 1 \\ z
% 	\end{matrix}\right](y) \ .
% \end{align}
% where the $y$-contour goes around the origin. Conversely, the individual twisted Eisenstein series can be rewritten in terms of the above differences $\Delta_k$. Let us define $\mathcal{S}_\ell$ by
% \begin{align}\label{S2k}
% 	\frac{1}{2}\frac{y}{\sinh \frac{y}{2}}
% 	\equiv \sum_{\ell \ge 0} \mathcal{S}_\ell\, y^\ell .
% \end{align}
% It is straightforward to show that
% \begin{align}\label{Eisenstein-from-Delta}
% 	E_k\left[\begin{matrix}
% 		\pm 1 \\ z
% 	\end{matrix}\right]
% 	= \sum_{\ell = 0}^{k} \mathcal{S}_\ell\, \Delta_{k - \ell + 1}\left[\begin{matrix}
% 		\mp 1 \\ z
% 	\end{matrix}\right]\ .
% \end{align}




\subsection{Elliptic function}
In this paper we frequently encounter elliptic functions with respect to a complex structure $\tau$. They are meromorphic functions on $\mathbb{C}$ satisfying the doubly-periodic condition,
\begin{align}
	f(\mathfrak{z}) = f(\mathfrak{z} + \tau) = f(\mathfrak{z} + 1) \ .
\end{align}
Here $\tau \in \mathbb{C}$ with $\operatorname{Im}\tau > 0$. Exploiting the periodicity, one may restrict the domain of $\mathfrak{z}$ to be the \emph{fundamental parallelogram} in $\mathbb{C}$ with vertices $0$, $1$, $\tau$, $1 + \tau$. Equivalently, an elliptic function $f$ is a meromorphic function on the torus $T^2_\tau$ with complex structure $\tau$. Using $z = e^{2\pi i \mathfrak{z}}$, $f(\mathfrak{z})$ is some times rewritten as $f(z)$.

As a meromorphic function, $f(\mathfrak{z})$ may have poles in the parallelogram. In this paper we mainly focus on elliptic functions $f(\mathfrak{z})$ with only simple poles. We classify the poles $\mathfrak{z}_j$ into two types by the following criteria: we call $\mathfrak{z}_j$ \emph{real} if $\operatorname{Im}\mathfrak{z}_j = 0$, or \emph{imaginary} if $\operatorname{Im}\mathfrak{z}_j > 0$. The residues at the simple poles $\mathfrak{z}_j$ are captured by $R_j$,
\begin{align}
	R_j \coloneqq \mathop{\operatorname{Res}}_{z \to z_j}\frac{1}{z}f(z) \ .
\end{align}
Using the well-known Weierstrass $\zeta$ function and the Eisenstein series, any elliptic function with $f(\mathfrak{z})$ with only simple poles\footnote{For functions with higher order poles, one needs to include derivatives of $\zeta$-function or Eisentein series.} can be expanded in various ways,
\begin{align}\label{elliptic-function-decomposition}
	f(\mathfrak{z}) = C_f + \frac{1}{2\pi i}\sum_{j} R_j \zeta(\mathfrak{z} - \mathfrak{z}_j)
	= & \ f(\mathfrak{z}_0) + \sum_j R_j \left(
		E_1 \begin{bmatrix}
    	+1 \\ \frac{z_j}{z_0}
		\end{bmatrix}
		- E_1 \begin{bmatrix}
    	+1 \\ \frac{z_j}{z}
		\end{bmatrix}
	\right) \nonumber\\
	= & \ f(\mathfrak{z}_0) + \sum_j R_j \left(
		E_1 \begin{bmatrix}
    	- 1 \\ \frac{z_j}{z_0} q^{\frac{1}{2}}
		\end{bmatrix}
		- E_1 \begin{bmatrix}
    	- 1 \\ \frac{z_j}{z} q^{\frac{1}{2}}
		\end{bmatrix}
	\right) \\
	= & \ f(\mathfrak{z}_0) + \sum_j R_j \left(
		E_1 \begin{bmatrix}
    	- 1 \\ \frac{z_j}{z_0} q^{-\frac{1}{2}}
		\end{bmatrix}
		- E_1 \begin{bmatrix}
    	- 1 \\ \frac{z_j}{z} q^{-\frac{1}{2}}
		\end{bmatrix}
	\right) \nonumber \\
		= & \ f(\mathfrak{z}_0) + \sum_{\text{real/img} \ \mathfrak{z}_j} R_j \left(
			E_1 \begin{bmatrix}
	    	- 1 \\ \frac{z_j}{z_0} q^{\pm\frac{1}{2}}
			\end{bmatrix}
			- E_1 \begin{bmatrix}
	    	- 1 \\ \frac{z_j}{z} q^{\pm\frac{1}{2}}
			\end{bmatrix}
		\right) \nonumber \ .
\end{align}
Here $z_0 = e^{2\pi i \mathfrak{z}_0}$ is an arbitrary and generic reference value. Note that the expansion is valid for all types of pole combinations, real or imaginary, where the last line incorporates explicitly the real-ness of the poles to determine the $\pm \frac{1}{2}$. These expansions lead to useful integration formula that we will review later.




\subsection{Useful identities}

% The Jacobi theta functions satisfy a collection of \emph{duplication formulas}, for example,
% \begin{align}\label{duplication}
% 	\vartheta_1(2 \mathfrak{z})\vartheta_1'(0)
% 	= & \ 2\pi\prod_{i = 1}^{4}\vartheta_i(\mathfrak{z})
% 	= \pi \vartheta_1(2 \mathfrak{z}) \prod_{i = 2}^{4}\vartheta_i(0) \ ,\\
%   \vartheta_4(2 \mathfrak{z}) \vartheta_4(0)^3
%   = & \ \vartheta_4(\mathfrak{z})^4 - \vartheta_1(\mathfrak{z})^4
%   = \vartheta_3(\mathfrak{z})^4 - \vartheta_2(\mathfrak{z})^4 \ .
% \end{align}
% The $\mathfrak{z} \to 0$ limit of the first line gives the well-known identity $\vartheta'_1(0) = \pi \vartheta_2(0)\vartheta_3(0)\vartheta_4(0)$. The derivatives of $\vartheta_i$ satisfy, among a few other relations,
% \begin{align}\label{theta-derivative-formula}
% 	\frac{d}{d \mathfrak{z}} \left[\frac{\vartheta_1(\mathfrak{z})}{\vartheta_4(\mathfrak{z})}\right] = \vartheta_4(0)^2 \frac{\vartheta_2(\mathfrak{z})\vartheta_3(\mathfrak{z})}{\vartheta_4(\mathfrak{z})^2} \quad \Rightarrow
% 	\quad
% 	\frac{\vartheta'_4(\mathfrak{z})}{\vartheta_4(\mathfrak{z})}
% 	- \frac{\vartheta'_1(\mathfrak{z})}{\vartheta_1(\mathfrak{z})}
% 	= - \pi \vartheta_4(0)^2 \frac{\vartheta_2(\mathfrak{z})\vartheta_3(\mathfrak{z})}{\vartheta_1(\mathfrak{z})\vartheta_4(\mathfrak{z})}\ .
% \end{align}



% One can express both the Weierstrass family and the Eisenstein series in terms of the Jacobi theta functions. For example,
% \begin{align}\label{zeta-thetap}
% 	\zeta(\mathfrak{z}) = \frac{\vartheta'_1(\mathfrak{z})}{\vartheta_1(\mathfrak{z})} - 4\pi^2 \mathfrak{z} E_2 \ .
% \end{align}
% The quasi-periodicity of $\zeta$ now follows and one can express the $\eta_i(\tau)$ in \eqref{shift-formula-zeta} as
% \begin{align}
% 	\eta_1(\tau) = - 2\pi^2 E_2, \qquad \eta_2(\tau) = \tau \eta_1(\tau) - \pi i \ .
% \end{align}

% The schematic relation between the Eisenstein series and the Jacob-theta functions can be summarized in the diagram
% \begin{center}
% 	% Figure removed
% \end{center}
% In more details, the Eisenstein series can be rewritten in terms of ratios of $\vartheta$ functions and their derivatives,
% \begin{align}\label{EisensteinToTheta}
% 	E_k\left[\begin{matrix}
% 		+ 1 \\ z
% 	\end{matrix}\right] = - \left[e^{ - \frac{y}{2\pi i }\mathcal{D}_\mathfrak{z} - P_2(y) }\right]_k \vartheta_1(\mathfrak{z})
% \end{align}
% where $P_2$ is a Weierstrass elliptic-$P$ function (\ref{P2}), $[f(y)]_k$ denotes the $k$-th coefficient of the Taylor series of $f(y)$ around $y=0$, and we define an abstract differential operators $\mathcal{D}_\mathfrak{z}^n$ by
% \begin{align}
% 	\underbrace{\mathcal{D}_\mathfrak{z} \ldots \mathcal{D}_\mathfrak{z}}_{n \text{ copies}} \vartheta_i(\mathfrak{z}) = \mathcal{D}_\mathfrak{z}^n \vartheta_i(\mathfrak{z}) \equiv \frac{\vartheta^{(n)}_i(\mathfrak{z})}{\vartheta_i(\mathfrak{z})} \ .
% \end{align}
The Eisenstein series are related to the Jacobi theta functions,
\begin{align}\label{EisensteinToTheta-2}
	E_k\left[\begin{matrix}
		+ 1 \\ z
	\end{matrix}\right] = \sum_{\ell = 0}^{\floor{k/2}}  \frac{(-1)^{k + 1}}{(k - 2\ell)!}\left(\frac{1}{2\pi i}\right)^{k - 2\ell} \mathbb{E}_{2\ell} \frac{\vartheta_1^{(k - 2\ell)}(\mathfrak{z})}{\vartheta_1(\mathfrak{z})} \ ,
\end{align}
where we define
\begin{align}\label{Ebold}
	& \mathbb{E}_{2} \coloneqq E_2, \qquad \mathbb{E}_4 \coloneqq E_4 + \frac{1}{2}(E_2)^2, \qquad
	\mathbb{E}_6 \coloneqq E_6 + \frac{3}{4}E_4E_2 + \frac{1}{8}(E_2)^3 \ , \qquad \ldots\\
	& \mathbb{E}_{2\ell} \coloneqq \sum_{\substack{\{n_p\} \\ \sum_{p \ge 1} (2p)n_p = 2\ell}} \prod_{p\ge 1} \frac{1}{n_p !} \left(\frac{1}{2p}E_{2p}\right)^{n_p}\ .
\end{align}
Similar formula for $E_k\left[\substack{- 1 \\ \pm z}\right]$ can be obtained by replacing $\vartheta_1$ with $\vartheta_{2,3,4}$. For the reader's convenience we list the first few conversions here.
\begin{align}\label{Ek-thetap}
	E_1\left[\begin{matrix}
		+1 \\ z
	\end{matrix}
	\right] = & \ \frac{1}{2\pi i} \frac{\vartheta'_1(\mathfrak{z})}{\vartheta_1(\mathfrak{z})}\ ,  \\
	E_2\left[\begin{matrix}
		+1 \\ z
	\end{matrix}
	\right] = & \ \frac{1}{8\pi^2}\frac{\vartheta_1''(\mathfrak{z})}{\vartheta_1(\mathfrak{z})} - \frac{1}{2} E_2 \ , \\
	E_3\left[\begin{matrix}
		+1 \\ z
	\end{matrix}
	\right] = & \ \frac{i}{48\pi^3} \frac{\vartheta'''_1(\mathfrak{z})}{\vartheta_1(\mathfrak{z})}
	  - \frac{i}{4\pi}\frac{\vartheta'_1(\mathfrak{z})}{\vartheta_1(\mathfrak{z})} E_2,  \\
	E_4\left[\begin{matrix}
		+1 \\ z
	\end{matrix}\right] = & \ - \frac{1}{384\pi^4} \frac{\vartheta''''_1(\mathfrak{z})}{\vartheta_1(\mathfrak{z})} + \frac{1}{16\pi^2}E_2 \frac{\vartheta''_1(\mathfrak{z})}{\vartheta_1(\mathfrak{z})} - \frac{1}{4} \left(E_4 + \frac{1}{2}(E_2)^2\right) \\
	E_5\left[\begin{matrix}
		+1 \\ z
	\end{matrix}\right]
	= & \ - \frac{i}{3840 \pi^5} \frac{\vartheta^{(5)}_1(\mathfrak{z})}{\vartheta_1(\mathfrak{z})} + \frac{i}{96\pi^3}E_2 \frac{\vartheta_1^{(3)}(\mathfrak{z})}{\vartheta_1(\mathfrak{z})} - \frac{i}{8\pi}\left(E_4 + \frac{1}{2}(E_2)^2\right)\frac{\vartheta_1'(\mathfrak{z})}{\vartheta_1(\mathfrak{z})} \\
	E_6\left[\begin{matrix}
		+ 1 \\ z
	\end{matrix}\right]
	= & \ \frac{1}{46080\pi^6} \frac{\vartheta^{(6)}_1(\mathfrak{z})}{\vartheta_1(\mathfrak{z})} - \frac{1}{768\pi^4}E_2 \frac{\vartheta_1^{(4)}(\mathfrak{z})}{\vartheta_1(\mathfrak{z})} + \frac{1}{32\pi^2} \left(E_4 + \frac{1}{2}(E_2)^2\right) \frac{\vartheta_1^{(2)}(\mathfrak{z})}{\vartheta_1(\mathfrak{z})} \nonumber\\
	& \ - \frac{1}{6}\left(E_6 + \frac{3}{4}E_4 E_2 + \frac{1}{8}E_2^3\right) \ .
\end{align}


% From the above conversion one computes the residues of Eisenstein series,
% \begin{align}
% 	\mathop{\operatorname{Res}}_{z \to 1}\frac{1}{z}E_k\left[\begin{matrix}
% 		+ 1 \\ z
% 	\end{matrix}\right] = \delta_{k1} \ ,
% 	\qquad
% 	\mathop{\operatorname{Res}}_{z \to q^{\frac{1}{2} + n}}\frac{1}{z}E_k\left[\begin{matrix}
% 		- 1 \\ z
% 	\end{matrix}\right] = \frac{1}{2^{k - 1} (k - 1)!} \ .
% \end{align}



Moreover, the Eisenstein series satisfy the following relations which are generalization of the so-called duplication formula of the Jacobi theta functions,
\begin{align}\label{duplication-Eisenstein}
	\sum_{\pm}E_k\left[\begin{matrix}
		\phi \\ \pm z
	\end{matrix}\right](\tau) = & \ 2 E_k\left[\begin{matrix}
		\phi \\ z^2
	\end{matrix}\right](2\tau) \ , \nonumber \\
	\sum_{\pm} \pm E_k\left[\begin{matrix}
		\phi \\ \pm z
	\end{matrix}\right](\tau)
	= & \ -2 E_k\left[\begin{matrix}
		\phi \\ z^2
	\end{matrix}\right](2\tau)
	 + 2 E_k\left[\begin{matrix}
	 	\phi \\ z
	 \end{matrix}\right](\tau)\ , \nonumber
	\\
	E_k\left[\begin{matrix}
		+ 1\\z
	\end{matrix}\right](2\tau)
	+ E_k\left[\begin{matrix}
		- 1\\z
	\end{matrix}\right](2\tau) = & \ 
	\frac{2}{2^k}E_k\left[\begin{matrix}
		+ 1 \\ z
	\end{matrix}\right] \ ,\\
	\sum_{\pm \pm} E_k\left[\begin{matrix}
		\pm 1 \\ \pm z
	\end{matrix}\right](\tau) = & \ \frac{4}{2^k}E_k\left[
	\begin{matrix}
		+ 1 \\ z^2
	\end{matrix}\right](\tau)\ . \nonumber\\
	E_1 \begin{bmatrix}
  	\phi \\ zq^{-1/2}  
	\end{bmatrix}(2\tau) + E_1 \begin{bmatrix}
  	\phi \\ z q^{\frac{1}{2}}  
	\end{bmatrix}(2\tau) = & \ E_1 \begin{bmatrix}
  	\phi \\ z  
	\end{bmatrix}(\tau), \qquad \phi = \pm 1 \ .
\end{align}
% There are similar identities coming from applying the shift $z \to z q^{\frac{1}{2}}$. Combining the duplication formulas and (\ref{theta-derivative-formula}), one finds a useful identity
% \begin{align}\label{Eisenstein-identity-1}
% 	E_1\left[\begin{matrix}
% 		+ 1 \\ z
% 	\end{matrix}\right]
% 	- E_1\left[\begin{matrix}
% 		- 1 \\ z
% 	\end{matrix}\right]
% 	= \frac{\eta(\tau)^3}{2i} \frac{\vartheta_1(2 \mathfrak{z})\vartheta_4(0)^2}{\vartheta_1(\mathfrak{z})^2 \vartheta_4(\mathfrak{z})^2}\ .
% \end{align}

The $E_1$ function also has some alternative expansions besides its definition, for example,
\begin{align}\label{E1-expansions}
	E_1 \begin{bmatrix}
  	-1 \\ b  
	\end{bmatrix}
	=  & \ \frac{b^{-1} q^{\frac{1}{2}}}{1 - b^{-1} q^{\frac{1}{2}}}
	- \frac{b q^{\frac{1}{2}}}{1 - b q^{\frac{1}{2}}}
	+ \sum_{k = 1}^{+\infty} q^n \frac{b^n - b^{-n}}{q^{\frac{n}{2}} - q^{-\frac{n}{2}}} \\
	= & \ \frac{1}{2}\left(\frac{b^{-1} q^{\frac{1}{2}}}{1 - b^{-1} q^{\frac{1}{2}}}
	- \frac{b q^{\frac{1}{2}}}{1 - b q^{\frac{1}{2}}}\right)
	+ \frac{1-q}{2}\sum_{n = 1}^{+\infty}q^{n/2}\sum_{\substack{m = - n/2 \\ m\ne0 }}^{n/2} \frac{b^{2m} - b^{-2m}}{1 - q^{-2m}} \ .
\end{align}
% To understand these relations, it is useful to also note the following $q$-series identities,
% \begin{align}
% 	\sum_{n = 1}^{+\infty}\frac{x^n}{q^{- n/2} - q^{n/2}} = \sum_{n = 0}^{+\infty} \frac{x q^{n + \frac{1}{2}}}{1 - xq^{n + \frac{1}{2}}} \ ,
% \end{align}
% and the identity
% \begin{align}
% 	& \ \frac{1}{1 - q^2}\sum_{n = 1}^{+\infty}a_n q^n
% 	= \sum_{n = 1}^{+\infty} \sum_{\ell = 0}^{+\infty} a_n q^{2\ell + n}
% 	= \sum_{N = 1}^{+\infty} \sum_{\ell = 0}^{\lfloor\frac{N-1}{2} \rfloor} a_{N - 2\ell} q^N \\
% 	= & \ \sum_{N = 1}^{+\infty}q^N \sum_{\ell = \frac{N}{2} - \lfloor \frac{N-1}{2} \rfloor}^{\frac{N}{2}}a_{2\ell} \ ,
% \end{align}
% combined with
% \begin{align}
% 	& \ \sum_{\substack{m = - n/2 \\ m \ne 0}}^{+n/2} \frac{b^{2m}}{q^m - q^{-m}} = \sum_{\ell = \frac{n}{2} - \lfloor \frac{n}{2} \rfloor}^{\frac{n}{2}} \frac{b^{2\ell} - b^{-2\ell}}{q^\ell - q^{-\ell}} \ ,
% \end{align}
% to imply
% \begin{align}
% 	\sum_{n = 1}^{\infty}\sum_{\substack{m = - n/2 \\ m \ne 0}}^{n/2} q^n \frac{b^{2m}}{q^m - q^{-m}} = \frac{1}{1-q^2} \sum_{n = 1}^{+\infty} \frac{b^n - b^{-n}}{q^{n/2} - q^{-n/2}} \ .
% \end{align}




\section{Integration formula \label{app:integration-formula}}

In this appendix we collect integration formula for contour integrals containing an elliptic function, some products of Eisenstein series and some monomial factors.


\subsection{Integration formula without monomial}

We begin with the simplest formula. Consider an elliptic function $f(\mathfrak{z})$ with only simple poles. Denoting $z = e^{2\pi i \mathfrak{z}}$, then the contour integral of $f$ along the unit circle can be computed analytically,
\begin{align}\label{integration-formula-f}
	\oint_{|z| = 1} \frac{dz}{2\pi i z} f(\mathfrak{z})
	= f(\mathfrak{z}_0)
	+ \sum_{\text{real/img} \ \mathfrak{z}_j} R_j E_1 \begin{bmatrix}
  	-1 \\ \frac{z_j}{z_0}q^{\pm \frac{1}{2}} 
	\end{bmatrix} \ .
\end{align}
Here, $\mathfrak{z}_0$ (and $z_0 = e^{2\pi i \mathfrak{z}_0}$) denotes an arbitrary and generic reference value, and $\mathfrak{z}_j$ (with $z_j = e^{2\pi i \mathfrak{z}_j}$) are the simple poles of $f$. Recall that $\mathfrak{z}_j$ is real if $\operatorname{Im}\mathfrak{z}_j = 0$, or imaginary if $\operatorname{Im}\mathfrak{z}_j > 0$. This formula follows directly from the decomposition
\begin{align}
	f(\mathfrak{z}) = f(\mathfrak{z}_0) + \sum_{\text{real/img} \ \mathfrak{z}_j} R_j \left(
			E_1 \begin{bmatrix}
	    	- 1 \\ \frac{z_j}{z_0} q^{\pm\frac{1}{2}}
			\end{bmatrix}
			- E_1 \begin{bmatrix}
	    	- 1 \\ \frac{z_j}{z} q^{\pm\frac{1}{2}}
			\end{bmatrix}
		\right) \ .
\end{align}
Note that only the last term depends on $z$, and upon integration,
\begin{align}
	\oint \frac{dz}{2\pi i z} E_1 \begin{bmatrix}
  	-1 \\ za  
	\end{bmatrix} = 0 \ .
\end{align}
Only the $z$-independent terms survives the contour integration, yielding \eqref{integration-formula-f}.

In computing Schur index, we often encounter more complicated contour integrals involving the product of an elliptic function and several Eisenstein series. For example, for the class-$\mathcal{S}$ $A_1$ index, we need the following integration formula,
\begin{align}
	\label{integration-formula-fE-1}
		& \ \oint_{|z| = 1} \frac{dz}{2\pi i z}f(\mathfrak{z})E_k\left[
		\begin{matrix}
			-1 \\ za
		\end{matrix}\right] \\
		= & \ - \mathcal{S}_{k} \left(f(\mathfrak{z}_0)
		  + \sum_{\text{real/imag } \mathfrak{z}_j}R_j  E_{1}\left[\begin{matrix}
				-1 \\ \frac{z_j}{z_0}q^{\pm \frac{1}{2}}
			\end{matrix}\right]
		\right)
		- \sum_{\text{real/imag } \mathfrak{z}_j} R_j\,  \sum_{\ell = 0}^{k - 1} \mathcal{S}_{\ell}\, E_{k - \ell + 1}\left[
			\begin{matrix}
				1 \\ z_j a q^{\pm \frac{1}{2}}
			\end{matrix}\right] \ .
\end{align}
Here $\mathcal{S}_k$ are rational numbers defined through the series expansion
\begin{align}
	\frac{1}{2} \frac{y}{\sinh(y/2)} = \sum_{\ell \ge 0} \mathcal{S}_\ell y^\ell \ .
\end{align}
Explicitly, we list a few instances of $\mathcal{S}_\ell$ below.
\begin{center}
	\begin{tabular}{c|c|c|c|c|c|c|c|c|c|c|c|c|c}
		$\ell$ & 0 & 1 & 2 & 3 & 4 & 5 & 6 & 7 & 8 & 9 & 10 & 11 & 12 \\
		\hline
		$\mathcal{S}_\ell$ & 1 & 0 & $- \frac{1}{24}$ & 0 & $\frac{7}{5760}$ & 0 & $ - \frac{31}{967680}$ & 0 & $\frac{127}{154828800}$ & 0 & $- \frac{73}{3503554560}$
		& $0$ & 
		$\frac{1414477}{2678117105664000}$
	\end{tabular}
\end{center}

Similarly, we also have
\begin{align}\label{integration-formula-fE-2}
	& \ \oint_{|z| = 1} \frac{dz}{2\pi i z} f(\mathfrak{z}) E_k\left[\begin{matrix}
		+ 1 \\ za
	\end{matrix}
	\right] \nonumber\\
	= & \ - \mathcal{A}_k\left(f(\mathfrak{z}_0) + \sum_{\text{real/imag } \mathfrak{z}_j}R_j\, E_1\left[\begin{matrix}
			- 1\\ \frac{z_j}{z_0}q^{\pm \frac{1}{2}}
		\end{matrix}\right]\right)  \nonumber\\
	  & \ - \sum_{\text{real/imag } \mathfrak{z}_j} R_j \left(
	    - \mathcal{B}_k\, E_1\left[\begin{matrix}
	  	  -1 \\ z_jaq^{\pm \frac{1}{2}}
	    \end{matrix}\right]
	  + \sum_{\ell = 0}^{k - 1} \mathcal{S}_{\ell}\, E_{k + 1 - \ell} \left[\begin{matrix}
	  	  		-1 \\ z_j a q^{\pm  \frac{1}{2}}
	  	  	\end{matrix}\right]\right) \; ,
\end{align}
where
\begin{align}
	\mathcal{A}_{2n} = \frac{B_{2n}}{(2n)!}, \qquad \mathcal{A}_{2n + 1} = \frac{\delta_{n, 0}}{2}, \qquad \mathcal{B}_{2n} = \frac{B_{2n}}{(2n)!} - \mathcal{S}_{2n}, \qquad \mathcal{B}_{2n + 1} = \frac{\delta_{n,0}}{2} \ .
\end{align}

To compute the Schur index of the some class-$\mathcal{S}$ $A_2$ with 2 or 3 maximal punctures and an arbitrary number of minimal punctures, we need the following type of integrals,
\begin{align}\label{EEtype}
	\oint \frac{dz}{2\pi i z} f(\mathfrak{z}) E_1 \begin{bmatrix}
  	\pm 1 \\ za  
	\end{bmatrix}
	E_k \begin{bmatrix}
  	\pm 1 \\ zb
	\end{bmatrix} \ .
\end{align}

As simplest cases, when $f(\mathfrak{z}) = 1$ we have the following integration formula,
\begin{align}\label{integration-formula-EE}
	& \ \oint \frac{dz}{2\pi i z}E_k \begin{bmatrix}
  	-1 \\ za  
	\end{bmatrix}
	E_\ell \begin{bmatrix}
  	-1 \\ zb
	\end{bmatrix}\\
	= & \ (-1)^{k + \ell + 1}\Bigg(
	- C_{k + \ell}^{k} \mathcal{S}_{k + \ell} 
	+ \sum_{r = 2}^{\ell}\sum_{s = r}^{\ell}
	(-1)^{r + s}C_{k + \ell - s}^{\ell + 1 - r}\mathcal{S}_{k + \ell - s} E_s \begin{bmatrix}
  	1 \\ a/b  
	\end{bmatrix}
	\\
	& \ \qquad
	\qquad\qquad\qquad + \sum_{r = \ell + 1}^{k + \ell}(-1)^r C_{k + \ell - 1 - r}^{\ell - 1}
	E_r \begin{bmatrix}
  	1 \\ a/b  
	\end{bmatrix}
	\Bigg) \ .
\end{align}
One can also utilize this formula to derive or understand the structure of the other ones we introduce in this appendix. For example, to compute \eqref{integration-formula-fE-1}, one can begin by decomposing $f$ as in (\ref{elliptic-function-decomposition}), and apply (\ref{integration-formula-EE}).




When $f(\mathfrak{z})$ is a nontrivial elliptic function, we have the following series of integration formula,
\begin{align}\label{integration-formula-fEE-1}
  & \ \oint \frac{dz}{2\pi i z} f(\mathfrak{z}) E_1 \begin{bmatrix}
    -1 \\ za
  \end{bmatrix}
  E_{2k} \begin{bmatrix}
    -1 \\ zb
  \end{bmatrix} \nonumber \\
  = & \ 
  \left[\oint \frac{dz}{2\pi i z}f(\mathfrak{z})\right]\sum_{\ell = 0}^{k - 1}
  \mathcal{S}_{2\ell} E_{2k + 1 - 2\ell} \begin{bmatrix}
      1 \\ a/b
    \end{bmatrix} \\
  & \ - \sum_{\text{real/img }\mathfrak{z}_j} R_j E_1 \begin{bmatrix}
      1 \\ a z_j q^{\pm \frac{1}{2}}
    \end{bmatrix}\sum_{\ell = 0}^{k - 1}
  \mathcal{S}_{2\ell} 
    E_{2k + 1 - 2\ell} \begin{bmatrix}
      1 \\ b z_j q^{\pm \frac{1}{2}}
    \end{bmatrix} \nonumber\\
   & \ - \sum_{\text{real/img }\mathfrak{z}_j} R_j  E_1 \begin{bmatrix}
      1 \\ a z_j q^{\pm \frac{1}{2}}
    \end{bmatrix}\sum_{\ell = 0}^{k - 1}
    \mathcal{S}_{2\ell} 
    E_{2k + 1 - 2\ell} \begin{bmatrix}
      1 \\ \frac{a}{b}
    \end{bmatrix}
    \nonumber\\
  & \ - \sum_{\text{real/img }\mathfrak{z}_j} R_j
  \sum_{\ell = 0}^{k}(1-2\ell)\mathcal{S}_{2\ell} E_{2k + 2 - 2\ell}\begin{bmatrix}
    1 \\ b z_j q^{\pm \frac{1}{2}}  
  \end{bmatrix}
  + \mathcal{S}_{2k}\left(E_2\begin{bmatrix}
        1 \\ a z_j q^{\pm \frac{1}{2}}  
      \end{bmatrix}
    + E_2\begin{bmatrix}
        1 \\ b z_j q^{\pm \frac{1}{2}}  
      \end{bmatrix}  \right)\ , \nonumber
\end{align}
and
\begin{align}\label{integration-formula-fEE-2}
  & \ \oint \frac{dz}{2\pi i z} f(\mathfrak{z}) E_1 \begin{bmatrix}
    -1 \\ za  
  \end{bmatrix}
  E_{2k + 1} \begin{bmatrix}
    -1 \\ zb
  \end{bmatrix} \nonumber \\
  = & \ \left[\oint \frac{dz}{2\pi i z}f(\mathfrak{z})\right]\left[- (2k + 1)\mathcal{S}_{2k + 2} + \sum_{\ell = 0}^{k + 1} \mathcal{S}_{2\ell} E_{2k+2- 2\ell}\begin{bmatrix}
        1 \\ a/b  
      \end{bmatrix}\right] \nonumber \\
  & \ - \sum_{\text{real/img.}\ \mathfrak{z}_j} R_j E_1 \begin{bmatrix}
    1 \\ a z_j q^{\pm \frac{1}{2}}
  \end{bmatrix} \sum_{\ell = 0}^{k} \mathcal{S}_{2\ell} E_{2k + 2 - 2\ell}\begin{bmatrix}
    1 \\ bz_j q^{\pm \frac{1}{2}}
  \end{bmatrix} \\
  & \ + \sum_{\text{real/img.}\ \mathfrak{z}_j} R_j E_1 \begin{bmatrix}
    1 \\ a z_j q^{\pm \frac{1}{2}}
  \end{bmatrix} \sum_{\ell = 0}^{k} \mathcal{S}_{2\ell} E_{2k + 2 - 2\ell}\begin{bmatrix}
    + 1 \\ a/b
  \end{bmatrix} \nonumber\\
  & \ - \sum_{\text{real/img.}\ \mathfrak{z}_j} R_j \sum_{\ell = 0}^{k} (1-2\ell)\mathcal{S}_{2\ell} E_{2k + 3 - 2\ell}\begin{bmatrix}
    1 \\ b z_j q^{\pm \frac{1}{2}}
  \end{bmatrix} \nonumber \ .
\end{align}
Variants of these integration formula with $E_1 \big[\substack{+ 1 \\ za}\big]$, $E_1 \big[\substack{+ 1 \\ zb}\big]$ can be obtained by applying (\ref{Eisenstein-half-shift}).

\subsection{Integration formula with monomial}

In the previous discussions, we have encountered integrals involving products of elliptic functions and Eisenstein series. In the following, we further enrich the integration formula by including a monomial of the integration variable,
\begin{align}
	\oint \frac{dz}{2\pi i z} z^n f(\mathfrak{z}), \qquad
	\oint \frac{dz}{2\pi i z} z^n f(\mathfrak{z}) E_k \begin{bmatrix}
  	\pm 1 \\
  	z a  
	\end{bmatrix}\ , \qquad
	n \in \mathbb{Z}_{\ne 0} \ .
\end{align}
where $f(\mathfrak{z})$ is again an elliptic function in $\mathfrak{z}$. These formula will be important when dealing with loop operator index.


\subsubsection{One Eisenstein and monomial}
In the presence of a Eisenstein series, we have the following integration formula for a generic $a$ independent of $q$,
\begin{align}
	\oint \frac{dz}{2\pi i z} z^n E_k \begin{bmatrix}
		1 \\ za
	\end{bmatrix}
	= & \ \frac{1}{(k-1)!} \frac{q^n}{a^n} \frac{\text{Eu}_{k - 1}(q^n)}{(1 - q^n)^k} \ ,\\
	\oint \frac{dz}{2\pi i z} z^n E_k \begin{bmatrix}
		1 \\ z^{-1}a
	\end{bmatrix}
	= & \ \frac{(-1)^k}{(k-1)!} (aq)^n \frac{\text{Eu}_{k - 1}(q^n)}{(1 - q^n)^k} \ .
\end{align}
Here $\text{Eu}_n(t)$ denotes the Eulerian polynomial that is defined by the equation
\begin{align}
	\sum_{n = 0}^{+\infty}\text{Eu}_n(t) \frac{x^n}{n!} = \frac{t - 1}{t - e^{(t - 1)x}} \ .
\end{align}
Similarly, we have the following parallel integration formula
\begin{align}
	\oint \frac{dz}{2\pi i z} z^n E_k \begin{bmatrix}
		- 1 \\ za
	\end{bmatrix}
	= & \ \frac{1}{(k-1)!} \frac{q^{n/2}}{a^n} \Phi(q^n, 1 - k, \frac{1}{2}) \ ,\\
	\oint \frac{dz}{2\pi i z} z^n E_k \begin{bmatrix}
		- 1 \\ z^{-1}a
	\end{bmatrix}
	= & \ \frac{(-1)^k}{(k-1)!} a^nq^{n/2} \Phi(q^n, 1 - k, \frac{1}{2}) \ ,
\end{align}
where the $\Phi$ denotes the Lerch transcendent function $\Phi(z, s, a)$ given by
\begin{align}
	\Phi(z, s, a) \coloneqq \sum_{p = 0}^{+\infty} \frac{z^p}{(p + a)^s} \ .
\end{align}
Recall that the Eisenstein series enjoy shift property (\ref{Eisenstein-half-shift}). When inserted into the above integration formula, the shift property translates to
\begin{align}
	\frac{1}{(1 - q^n)^k} \operatorname{Eu}_{k - 1}(q^n)
	= \sum_{\ell = 0}^{k} \left(\frac{1}{2}\right)^\ell \frac{(k-1)!}{\ell!(k - 1 - \ell)!} \Phi(q^n, 1 - k + \ell, \frac{1}{2}) \ .
\end{align}
For readers' convenience, we list a few instances of $\operatorname{Eu}$ and $\Phi$,
\begin{center}
	\begin{tabular}{c|c|c|c|c|c|c}
		$n$ & $1$ & $2$ & $3$ & $4$ & $5$ \\
		\hline
		$\operatorname{Eu}_n(t)$ & $1$ & $1 + q$ & $1 + 4q + q^2$ & $1 + 11q + 11q^2 + q^3$ & $1 + 26 q + 66q^2 + 26q^3 + q^4$ \\
	\end{tabular}
	\begin{tabular}{c|c|c|c|c}
		$n$ & $6$ \\
		\hline
		$\operatorname{Eu}_n(t)$ & $1 + 57q + 302q^2 + 302q^3 + 57q^4 + q^5$
	\end{tabular}
	\begin{tabular}{c|c|c|c|c}
		$n$ & $7$ \\
		\hline
		$\operatorname{Eu}_n(t)$ & $1 + 120q + 1191q^2 + 2416 q^3 + 1191q^4 + 120 q^5 + q^6$
	\end{tabular}
\end{center}

In the presence of certain amount of $q$-shift, the above formula needs some modifications. For example, with generic $a$, $0 < \alpha < 1$, $\ell \in \mathbb{N}_{> 0}$
\begin{align}
	\oint \frac{dz}{2\pi i z}z^n E_1 \begin{bmatrix}
  	1 \\ z^{-1}aq
	\end{bmatrix}
	= & \ \frac{(a)^n}{1 - q^{-n}} = \oint \frac{dz}{2\pi i z} z^n E_1 \begin{bmatrix}
  	1 \\ z^{-1} a  
	\end{bmatrix}, \\
	\oint \frac{dz}{2\pi i z}z^n E_{k} \begin{bmatrix}
  	1 \\ z^{-1}aq^\alpha
	\end{bmatrix}
	= & \ \frac{(-1)^k}{(k-1)!} (a q^\alpha)^n q^{n} \frac{\operatorname{Eu}_{k - 1}(q^n)}{(1 - q^n)^k}- \delta_{k = 1}(aq^\alpha)^n \ ,\\
	\oint \frac{dz}{2\pi i z}z^n E_1 \begin{bmatrix}
  	-1 \\ z^{-1}aq^\ell
	\end{bmatrix}
	= & \ \frac{-2 + 2q^n}{1 + q^n} \frac{(-1)^k}{(k-1)!} a^n q^{n/2} \Phi(q^n, 1 - k, \frac{1}{2})  \ ,\\
	\oint \frac{dz}{2\pi i z}z^n E_2 \begin{bmatrix}
  	-1 \\ z^{-1}aq^\ell
	\end{bmatrix}
	= & \ \frac{-(2\ell - 1) + (2\ell + 1)q^n}{1 + q^n} \frac{(-1)^k}{(k-1)!} a^n q^{n/2} \Phi(q^n, 1 - k, \frac{1}{2})  \ ,\\
	\oint \frac{dz}{2\pi i z}z^n E_3 \begin{bmatrix}
  	-1 \\ z^{-1}aq^\ell
	\end{bmatrix}
	= & \ \frac{(2\ell - 1)^2 - 2(-3 + 4\ell^2)q^n + (2\ell + 1)^2 q^{2n}}{1 + q^n} \\
	& \ \qquad\qquad\qquad \times \frac{(-1)^k}{(k-1)!} a^n q^{n/2} \Phi(q^n, 1 - k, \frac{1}{2})  \ .
\end{align}








\subsubsection{Two Eisensteins}
With two factors of Eisenstein series, the integration formula become much more tedious. For $n \in \mathbb{Z}_{\ne 0}$ and $k_1 \ge k_2$, we have
\begin{align}\label{integration-formula-zEE-1}
	& \ \oint \frac{dz}{2\pi i z} z^n E_{k_1}\begin{bmatrix}
  	+ 1 \\ z  
	\end{bmatrix}E_{k_2} \begin{bmatrix}
  	+ 1 \\ za  
	\end{bmatrix} \\
	= & \ \sum_{\ell = 0}^{k_1} \frac{1}{\ell!} \frac{q^n}{a^n}  \frac{\operatorname{Eu}_{k_2 + \ell - 1}(q^n)}{(1 - q^n)^{k_2 + \ell}}\left[
		\frac{(-1)^{k_1 - \ell}}{(k_2 - 1)!} + \frac{\ell!a^n}{(k_1 - 1)! (k_2 - k_1 + \ell)!}
	\right] E_{k_1 - \ell}\begin{bmatrix}
  	+ 1 \\ a  
	\end{bmatrix} \ , \nonumber
\end{align}
and when $k_1 \le k_2$,
\begin{align}\label{integration-formula-zEE-2}
	& \ \oint \frac{dz}{2\pi i z} z^n E_{k_1}\begin{bmatrix}
  	+ 1 \\ z  
	\end{bmatrix}E_{k_2} \begin{bmatrix}
  	+ 1 \\ za  
	\end{bmatrix} \\
	= & \ \sum_{\ell = 0}^{k_2} \frac{1}{\ell!} \frac{q^n}{a^n}  \frac{\operatorname{Eu}_{k_1 + \ell - 1}(q^n)}{(1 - q^n)^{k_1 + \ell}}\left[
		\frac{a^n}{(k_1 - 1)!} + \frac{(-1)^{k_2 - \ell} \ell!}{(k_2 - 1)! (k_1 - k_2 + \ell)!}
	\right] E_{k_2 - \ell}\begin{bmatrix}
  	+ 1 \\ a
	\end{bmatrix} \ .\nonumber
\end{align}
It may be convenient to merge the two identities into
\begin{align}\label{integration-formula-zEE-3}
	\oint \frac{dz}{2\pi i z} z^n E_{k_1}\begin{bmatrix}
  	+ 1 \\ z  
	\end{bmatrix}
	E_{k_2}\begin{bmatrix}
  	+ 1 \\ z  a
	\end{bmatrix}
	= \sum_{\ell = 1}^{\operatorname{max}(k_1, k_2)}
	\frac{1}{\ell!}\frac{q^n}{a^n}\mathcal{E}_{k_1, k_2; \ell}(a^n, q^n) E_{\max(k_1, k_2) - \ell} \begin{bmatrix}
  	+1 \\ a  
	\end{bmatrix} \ ,
\end{align}
where $\mathcal{E}$ can be read off from (\ref{integration-formula-zEE-1}) and (\ref{integration-formula-zEE-2}). Note also that
\begin{align}
	\oint \frac{dz}{2\pi i z}z^n E_{k_1} \begin{bmatrix}
  	+1 \\ z a  
	\end{bmatrix}
	E_{k_1} \begin{bmatrix}
  	+1 \\ z b  
	\end{bmatrix}
	= & \ b^{-n} \oint \frac{dz}{2\pi i z}z^n E_{k_1} \begin{bmatrix}
  	+1 \\ z a/b
	\end{bmatrix}
	E_{k_2} \begin{bmatrix}
  	+1 \\ z
	\end{bmatrix} \nonumber \\
	= & \ a^{-n} \oint \frac{dz}{2\pi i z}z^n E_{k_1} \begin{bmatrix}
  	+1 \\ z
	\end{bmatrix}
	E_{k_2} \begin{bmatrix}
  	+1 \\ z b/a
	\end{bmatrix} \ .
\end{align}
The other integration formula variants of (\ref{integration-formula-zEE-1}), (\ref{integration-formula-zEE-2}) of involving $E_k \big[\substack{- 1\\za}\big]$, $E_k \big[\substack{- 1\\zb}\big]$ can be straightforwardly derived using the shift property (\ref{Eisenstein-half-shift}). For example, solving the system of equations
\begin{align}
	q^{- \frac{n}{2}}
	\oint \frac{dz}{2\pi i z}z^n E_{k_1} \begin{bmatrix}
  	+ 1 \\ z
	\end{bmatrix}
	E_{k_2} \begin{bmatrix}
  	+ 1 \\ z
	\end{bmatrix}
	= \sum_{\ell_1 = 0}^{k_1}
	\sum_{\ell_2 = 0}^{k_2} \frac{1}{2^{\ell_1 + \ell_2}} \frac{1}{\ell_1! \ell_2!} \oint \frac{dz}{2\pi i z}
	\prod_{i = 1}^2E_{k_i - \ell_i}\begin{bmatrix}
  	-1 \\ z  
	\end{bmatrix} \ ,
\end{align}
produces the integration formula for $z^n E_{1 \le \ell_1 \le k_1}\big[\substack{-1\\ z}\big] E_{1 \le \ell_2 \le k_2}\big[\substack{-1\\ z}\big]$ in terms of a linear combination of the known results for $z^n E_{1 \le \ell_1 \le k_i}\big[\substack{-1\\ z}\big]$ and $z^n E_{1 \le \ell_1 \le k_1}\big[\substack{+1\\ z}\big] E_{1 \le \ell_2 \le k_2}\big[\substack{+1\\ z}\big]$. 



Combining the above results, one can write down integration formula for $m \in \mathbb{Z}_{\ne 0}$
\begin{align}\label{integration-formula-zfE}
	\oint \frac{dz}{2\pi i z} z^m f(\mathfrak{z}) E_k \begin{bmatrix}
		\pm 1 \\ za
	\end{bmatrix}
	= & \ \left[\oint \frac{dz}{2\pi i z} f(\mathfrak{z})\right]\oint \frac{dz}{2\pi i z}z^m E_k \begin{bmatrix}
  	\pm 1 \\ za  
	\end{bmatrix} \nonumber\\
	& \ - \sum_{\operatorname{real/img} \ \mathfrak{z}_j}R_j
		\oint \frac{dz}{2\pi i z}
		z^m
		E_1 \begin{bmatrix}
  		- 1 \\ \frac{z_j}{z} q^{\pm\frac{1}{2}} 
		\end{bmatrix}E_k \begin{bmatrix}
  		\pm 1 \\ za  
		\end{bmatrix} \ ,
\end{align}
which follows easily from the decomposition
\begin{align}
	f(\mathfrak{z}) = f(\mathfrak{z}_0) + \sum_{\text{real/img} \ \mathfrak{z}_j} R_j \left(
			E_1 \begin{bmatrix}
	    	- 1 \\ \frac{z_j}{z_0} q^{\pm\frac{1}{2}}
			\end{bmatrix}
			- E_1 \begin{bmatrix}
	    	- 1 \\ \frac{z_j}{z} q^{\pm\frac{1}{2}}
			\end{bmatrix}
		\right) \ .
\end{align}





\subsubsection{Elliptic functions and monomial}

We proceed with the first integral by recalling that
\begin{align}
	f(\mathfrak{z})
	= C_f(\tau) + \frac{1}{2\pi i}\sum_{j}R_j\zeta (\mathfrak{z} - \mathfrak{z}_j) \ ,
\end{align}
where $\zeta$ can be expanded in Fourier series,
\begin{align}
	\zeta(\mathfrak{z}) = - 4\pi^2 \mathfrak{z} E_2(\tau) - (2m + 1)\pi i
	+ \pi \sum_{n}' \frac{1}{\sin n \pi \tau}q^{- \frac{n}{2}} e^{2\pi i n(\mathfrak{z}_0 + \lambda \tau)},
\end{align}
for $\mathfrak{z} = \mathfrak{z}_0 + \lambda \tau + m \tau$, $\lambda \in [0,1)$ and $m \in \mathbb{Z}$. The integral in the presence of $z^n$ with $n \ne 0$ can be carried out easily, which gives
\begin{align}\label{integration-formula-monomial}
	\oint \frac{dz}{2\pi i z} z^n f(\mathfrak{z})
	= & \ - \sum_{\text{real} \ \mathfrak{z}_j} R_j \frac{1}{1 - q^{-n}} z_j^n - \sum_{\text{imag.}\ \mathfrak{z}_j}R_j \frac{1}{q^n - 1}z_j^n  \nonumber\\
	= & \  - \sum_{\text{real/img.} \ \mathfrak{z}_j} R_j \frac{(z_j q^{\pm\frac{1}{2}})^n}{q^{n/2} - q^{-n/2}} \ .
\end{align}
Summing over $n$ with suitable coefficients, we further obtain some useful formula. For example,
\begin{align}
	\oint \frac{dz}{2\pi i z}\sum_{n \in \mathbb{Z}}' z^nf(\mathfrak{z})
	= - \sum_{\text{real/img } \mathfrak{z}_j} R_j E_1 \begin{bmatrix}
  	-1 \\ z_jq^{\pm \frac{1}{2}}  
	\end{bmatrix} \ .
\end{align}
This is simply a special case of (\ref{integration-formula-f}), since
\begin{align}
	f(z = 1) = \oint \frac{dz}{2\pi i z} \delta(z)f(\mathfrak{z}) 
	= \oint \frac{dz}{2\pi i z} (1 + \sum_{n}' z^n)f(\mathfrak{z}) \ .
\end{align}
Also, for $n \in \mathbb{N}$,
\begin{align}
	\oint{\frac{dz}{2\pi iz}\frac{\left( q-1 \right) z}{\left( 1-z \right) \left( 1-qz \right)}f\left( \mathfrak{z} \right)}
	= & \ - \sum_{\text{real } \mathfrak{z}_j}{R_j\frac{qz_j}{1-qz_j}}
	  -\sum_{\text{img } \mathfrak{z}_j}{R_j\frac{z_j}{1-z_j}} \\
	= & \ - \sum_{\text{real } \mathfrak{z}_j}{R_j\frac{1}{1-qz_j}}
	  -\sum_{\text{img } \mathfrak{z}_j}{R_j\frac{1}{1-z_j}} \ ,\\
	\oint \frac{dz}{2\pi i z} \frac{1}{(1-z^p)(1 - \frac{1}{z^p})} z^n f(\mathfrak{z})
	= & \ \sum_{\text{real/img } \mathfrak{z}_j} R_j (z_jq^{\pm \frac{1}{2}})^n  \sum_{\substack{k \ge 0 \\ k + n \ne 0}} \frac{k(z_j q^{\pm \frac{1}{2}})^{pk}}{q^{\frac{pk + n}{2}} - q^{- \frac{pk + n}{2}}} \\
	& + \frac{n}{p} \delta_{\frac{n}{p} \in \mathbb{Z}_{< 0}}\oint \frac{dz}{2\pi i z} f(\mathfrak{z}) \ .
\end{align}
When $z$ is the $SU(2)$ fugacity, then we have with the insertion of a spin-$J$ character $\chi_J(z) \coloneqq \sum_{m = -J}^{J}z^{2m}$,
\begin{align}\label{integration-formula-χf}
	\oint \frac{dz}{2\pi i z} & \ \chi_{J}(z) f(\mathfrak{z}) \nonumber\\
	= & \ \delta_{J \in \mathbb{Z}} \oint \frac{dz}{2\pi i z}f(\mathfrak{z})
	- \sum_{\substack{m = -J\\ m\ne 0}}^{+J}
	\left(
		\sum_{\text{real} \ \mathfrak{z}_j}R_j  \frac{1}{1 - q^{-2m}}z_j^{2m}
		+ \sum_{\text{img} \ \mathfrak{z}_j}R_j \frac{1}{q^{2m} - 1}z_j^{2m}
	\right) \ ,
\end{align}
and
\begin{align}
	\oint \frac{dz}{2\pi i z}\frac{\chi_J(z)}{(1-z^p)(1-1/z^p)}f(\mathfrak{z})
	= & \ \sum_{m = -J}^{J}
				\sum_{\text{real/img } \mathfrak{z}_j}R_j(z_jq^{\pm \frac{1}{2}})^{2m}
				\sum_{\substack{k \ge 0 \\ pk + 2m \ne 0}} \frac{k(z_j q^{\pm \frac{1}{2}})^{pk}}{q^{\frac{pk+2m}{2}} - q^{- \frac{pk+2m}{2}}} \nonumber \\
	& \ + \left[\sum_{m = - J}^{+J} \frac{2m}{p}\delta_{\frac{2m}{p} \in \mathbb{Z}_{< 0}}\right] \oint \frac{dz}{2\pi i z}f(\mathfrak{z}) \ .
\end{align}
Note that for $p = 1$, $J \in\frac{1}{2} \mathbb{N}$, $\sum_{m = -J}^{+J}2m \delta_{2m < 0} = \lceil J \rceil (\lceil J \rceil - 2J - 1) = - \lfloor(J + \frac{1}{2})^2\rfloor$.














\clearpage



%%%%%%%%%%  Bibliography  %%%%%%%%%%%%
{%\small
% \linepenalty=1000
\bibliographystyle{utphys}
\bibliography{ref}
}



\end{document}
