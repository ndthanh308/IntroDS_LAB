  \documentclass[aps,prx,superscriptaddress,reprint]{revtex4-1}
\usepackage{amsmath}
\usepackage{amssymb}
\usepackage{graphicx}
\usepackage[colorlinks,urlcolor=blue]{hyperref}
\usepackage{url}
\usepackage{hyperref}
\usepackage{color,xcolor}
\usepackage{epstopdf}
\usepackage{float}
\usepackage{ulem}
\usepackage[percent]{overpic}
\usepackage{siunitx}
\usepackage{todonotes}
\usepackage{lineno}
\usepackage[none]{hyphenat}

\begin{document}

\title{Structural phase transition, $s_{\pm}$-wave pairing, and magnetic stripe order in bilayered superconductor La$_3$Ni$_2$O$_7$ under pressure}
\author{Yang Zhang}
\author{Ling-Fang Lin}
\email{lflin@utk.edu}
\affiliation{Department of Physics and Astronomy, University of Tennessee, Knoxville, Tennessee 37996, USA}
\author{Adriana Moreo}
\affiliation{Department of Physics and Astronomy, University of Tennessee, Knoxville, Tennessee 37996, USA}
\affiliation{Materials Science and Technology Division, Oak Ridge National Laboratory, Oak Ridge, Tennessee 37831, USA}
\author{Thomas A. Maier}
\email{maierta@ornl.gov}
\affiliation{Computational Sciences and Engineering Division, Oak Ridge National Laboratory, Oak Ridge, Tennessee 37831, USA}
\author{Elbio Dagotto}
\email{edagotto@utk.edu}
\affiliation{Department of Physics and Astronomy, University of Tennessee, Knoxville, Tennessee 37996, USA}
\affiliation{Materials Science and Technology Division, Oak Ridge National Laboratory, Oak Ridge, Tennessee 37831, USA}


\begin{abstract}
Motivated by the recently discovered high-$T_c$ superconductor La$_3$Ni$_2$O$_7$, we comprehensively study this system using density functional theory and random phase approximation calculations.
At low pressures, the Amam phase is stable, containing the Y$^{2-}$ mode distortion from the Fmmm phase,
while the Fmmm phase is unstable. Because of small differences in enthalpy and a considerable Y$^{2-}$ mode amplitude,
the two phases may coexist in the range between 10.6 and 14 GPa, beyond which the Fmmm phase dominates. In addition, the magnetic
stripe-type spin order with wavevector ($\pi$, 0) was stable at the intermediate region.
Pairing is induced in the $s_{\pm}$-wave channel due to partial nesting between the {\bf M}=$(\pi, \pi)$ centered pockets and portions of the Fermi surface centered at the {\bf X}=$(\pi, 0)$ and {\bf Y}=$(0, \pi)$ points. This resembles results for iron-based superconductors but has a fundamental difference with iron pnictides and selenides. Moreover, our present efforts also suggest La$_3$Ni$_2$O$_7$ is qualitatively different from infinite-layer nickelates and cuprate superconductors.
\end{abstract}

\maketitle

\noindent {\bf \\Introduction\\}
The recently discovered infinite-layer (IL) nickelate superconductors~\cite{Li:Nature} opened the newest branch of the high-temperature superconductors family~\cite{Nomura:Prb,Botana:prx,Zhou:MTP,Nomura:rpp,Zhang:prb20}, including materials such as Sr-doped RNiO$_2$ films (R = Nd or Pr)~\cite{Li:Nature,Zeng:prl20} and quintuple-layer nickelate Nd$_6$Ni$_5$O$_{12}$~\cite{Pan:nm}. Similar to the widely discussed high $T_c$ cuprates superconductors~\cite{Dagotto:rmp94}, the IL nickelates also have a $d^9$ electronic configuration (Ni$^{\rm 1+}$) in the parent phase, as well as a NiO$_2$ two-dimensional (2D) square layer lattice. However,  many theoretical and experimental efforts have revealed that ``Ni$^+$ is not Cu$^{2+}$''~\cite{Lee:prb04}, and the fundamental similarities and differences between individual IL nickelate and cuprates have been extensively discussed~\cite{Zhang:prb20,Sakakibara:prl,Jiang:prl,Wu:prb,Gu:prb,Karp:prx,Fowlie:np,Rossi:np}. One key difference is that
in nickelates two $d$-orbitals ($d_{3z^2-r^2}$ and $d_{x^2-y^2}$) are important, while in cuprates only $d_{x^2-y^2}$ is relevant.



Recently,  La$_3$Ni$_2$O$_7$ (LNO) (with the novel $d^{\rm 7.5}$ configuration) was reported to be superconducting at high pressure, becoming the first non-IL NiO$_2$ layered nickelate superconductor~\cite{Sun:arxiv}, with highest $T_c \sim 80$ K. This conclusion was based on measurements of the resistance using a four-terminal device on a sample with an unknown degree of inhomogeneity. Furthermore, they observed a sharp transition and flat stage in resistance, by using KBr as the pressure-transmitting medium, as well as a diamagnetic response in the susceptibility, which it was interpreted as indication of the two prominent properties of superconductivity, zero resistance and Meissner effect~\cite{Sun:arxiv}. Subsequently, zero resistance has been confirmed by several studies~\cite{Hou:arxiv,Zhang:exp-arxiv,Zhang:arxiv-experiment}. However, the Meissner effect has not been conclusively observed yet. Recently, the potentially ``filamentary'' nature of the superconducting state has been presented by an experimental group caused by inhomogeneities in the sample~\cite{Wang:experiment}, providing a tentative explanation for why the Meissner effect has not been observed yet.

LNO displays the reduced Ruddlesden-Popper (RP) perovskite structure. At ambient conditions, LNO has the Amam structure with space group No. 63~\cite{Ling:jssc}, with an ($a^-$-$a^-$-$c^0$) out-of-phase oxygen octahedral tilting distortion around the [110] axis from the I4/mmm phase~\cite{Zhang:jssc}. By applying pressure of the order of 10 GPa, the NiO$_6$ rotations are suppressed and transform to an Fmmm space group (No. 69)~\cite{Sun:arxiv}. Then, the Fmmm phase becomes superconducting in a broad pressure range from 14 to 43.5 GPa~\cite{Sun:arxiv}.

Density functional theory (DFT) calculations revealed that the many components of the Fermi surface (FS) are contributed by the Ni orbitals
$d_{x^2-y^2}$ and $d_{3z^2-r^2}$. This FS consists of two-electron sheets with mixed $e_g$ orbitals and a hole pocket dominated by the $d_{3z^2-r^2}$ orbital, suggesting a Ni two-orbital minimum model is necessary~\cite{Luo:arxiv,Zhang:arxiv}. While completing our work, recent theoretical studies suggested that $s_{\pm}$-wave pairing superconductivity should dominate, in agreement with our results. This pairing channel is induced by spin fluctuations in the Fmmm phase of LNO~\cite{Yang:arxiv,Sakakibara:arxiv,Gu:arxiv,Shen:arxiv,Liu:arxiv}, indicating also the importance of the $d_{3z^2-r^2}$ orbital~\cite{Luo:arxiv,Zhang:arxiv,Lechermann:arxiv,Christiansson:arxiv,Shilenko:arxiv}. Furthermore, the role of the Hund coupling~\cite{Zhang:arxiv,Cao:arxiv}, electronic correlation effects~\cite{Lechermann:arxiv,Christiansson:arxiv,LiuZhe:arxiv,Wu:arxiv}, and the charge and spin instability~\cite{Luo:arxiv,Shilenko:arxiv,Chen:arxiv,LaBollita:arxiv} were also recently discussed.


% Figure environment removed



Interestingly, the Amam to Fmmm phase transition was found around 10 GPa, but the superconductivity was obtained only above 14 GPa~\cite{Sun:arxiv}. In addition, the values of $T_c$ do not dramatically change in a broad superconducting pressure region in the Fmmm phase of LNO~\cite{Sun:arxiv}. In this case, several questions naturally arise for LNO: what interesting physics occurs between 10 to 14 GPa?
Why is the observed $T_c$ in the superconducting pressure region 14 to 43.5 GPa relatively independent of pressure, as opposed to showing a dome-like dependence? Are the Fmmm structure or pressure itself important for superconductivity? Does the FS topology and $s_{\pm}$-wave pairing symmetry change in the Fmmm phase under high pressure? What are the main differences between LNO and
the previously well-studied bilayered system Bi$_{2}$Sr$_{2}$CaCu$_{2}$O$_{8}$ (BSCCO) of the cuprate family?


In this work, to answer these questions, we studied in detail the LNO system under pressure, using first-principles DFT and random phase approximation (RPA) calculations. In the low-pressure region (0 to 10.5 GPa), the Amam phase -- with the Y$^{2-}$ mode distortion from the Fmmm phase -- is stable, while the Fmmm phase is unstable. Due to small differences in enthalpy and a considerable Y$^{2-}$ mode amplitude, between 10.6 and 14 GPa the Amam phase could potentially coexist with the Fmmm phase in the same sample, or leading to sample-dependent behavior, and resulting in a greatly reduced or vanishing $T_c$ in this pressure region. Furthermore, in the range of pressures studied, two pockets ($\alpha$ and $\beta$) with mixed $d_{3z^2-r^2}$ and $d_{x^2-y^2}$ orbitals, and a $\gamma$ pocket made primarily of the $d_{3z^2-r^2}$ orbital contribute to the FS. Compared to ambient pressure, the $\gamma$ pocket is stretched and the $\beta$ pocket is reduced in size.
Furthermore, the DFT+$U$ and RPA calculations suggest a stripe spin order instability
with wavevector ($\pi$, 0).

Thus, our results highlight that the main fundamental differences with BSCCO are: (i) in the Ni bilayer
with two active Ni orbitals, it is $d_{3z^2-r^2}$ that plays the key role, as compared to $d_{x^2-y^2}$ for BSCCO cuprates
(see sketch Fig.~\ref{sketch}). (ii) This leads to $s_{\pm}$-wave pairing for Ni, while it is $d$-wave for Cu. Or, in other words, the inter-layer hybridization being large induces $s^\pm$ in LNO,
but when this hybridization is small then $d$-wave dominates as in cuprates.
(iii) The FSs of LNO and BSCCO fundamentally differ with regards to the presence of hole pockets at $(\pi, \pi)$ for LNO. These pockets are crucial for the stability of $s_{\pm}$ pairing.


\noindent {\bf \\Results\\}
\noindent {\small \bf \\DFT results\\}
Based on the group analysis obtained from the AMPLIMODES software~\cite{Orobengoa:jac,Perez-Mato:aca}, the distortion mode from the high symmetry phase (Fmmm) to the low symmetry phase (Amam) is the Y$^{2-}$ (see Fig.~\ref{DFT}{\bf a}). At 0 GPa, the distortion amplitude of the Y$^{2-}$ mode is  $\sim 0.7407$ \AA. As shown in Fig.~\ref{DFT}{\bf b}, this Y$^{2-}$ mode amplitude is gradually reduced under pressure, reaching nearly zero value at 15 GPa ($\sim 0.0016$ \AA). At 0 GPa, the Amam phase has an energy lower by about -21.01 meV/Ni than the Fmmm structure.
As shown in Fig.~\ref{DFT}{\bf b}, the difference in enthalpy between the Amam and Fmmm phases also smoothly decreases by increasing pressure. Interestingly, the Fmmm and Amam phases have very close enthalpies in the pressure range from 9 to 14 GPa, while the Y$^{2-}$ mode distortion still exists with sizeable distortion amplitude in this region. To better understand the structural stability of LNO, we calculated the phonon spectrum of the Fmmm and Amam phases with or without pressure, by using the density functional perturbation theory approach~\cite{Baroni:Prl,Gonze:Pra1,Gonze:Pra2} analyzed by the PHONONPY software~\cite{Chaput:prb,Togo:sm}. For the Amam phase of LNO, there is no imaginary frequency obtained in the phonon dispersion spectrum from 0 to 15 GPa (see results in the Supplementary Note I), suggesting that the Amam phase is stable in this pressure range.

% Figure environment removed


The phonon dispersion spectrum displays imaginary frequencies appearing at high symmetry points for the Fmmm structure of LNO below 10.5 GPa (see the results of 0 GPa as example, in Fig.~\ref{DFT}{\bf c}). However, the Fmmm phase becomes stable without any imaginary frequency from 10.6 GPa to 50 GPa, the maximum value we studied (see $P = 11$ GPa as an example in Fig.~\ref{DFT}{\bf d}, while the rest of the results can be found in the Supplementary Note I). Between 10.6 and 14 GPa, Fmmm has an enthalpy slightly lower than that of the Amam state ($\textless$ $\sim 0.3$ meV/Ni). This value ($\sim 10.6$ GPa) is quite close to the experimental observed critical pressure ($\sim 10$ GPa) for the Amam to Fmmm transition~\cite{Sun:arxiv}.

Recent DFT calculations found that the pocket around ($\pi$, $\pi$) vanishes in the Amam phase~\cite{Sun:arxiv,Zhang:arxiv}, which was also confirmed by angle-resolved photoemission spectroscopy experiments~\cite{Yang:arxiv-exp}. This pocket induces the $s_{\pm}$-wave pairing symmetry in the superconducting phase, as discussed below. Due to the small difference in enthalpy and considerable Y$^{2-}$ mode amplitude in this pressure range, the Amam phase could also be obtained experimentally in some portions of the same sample of LNO. Recent experiments also suggest a first-order structural transition from the Amam to Fmmm phases under pressure~\cite{Sun:arxiv}. Our RPA calculation shows the $s_{\pm}$-wave pairing superconductivity seems to be unlikely when this $\gamma$ pocket is absent, as discussed in the ``Pairing symmetry'' section. In this case, the $T_c$ would be greatly reduced or vanish in this pressure region due to the coexistence with the Amam phase.

While finishing the present manuscript, we noted that a very recent experimental effort reported zero resistance below 10~K in some samples above 10 GPa and below 15 GPa~\cite{Hou:arxiv}, supporting our conclusion. This could also qualitatively explain the absence of superconductivity between 10 and 14 GPa in the original high-pressure efforts~\cite{Sun:arxiv}. At 15 GPa, our DFT results found that the Y$^{2-}$ mode amplitude is almost zero ($\sim 0.0016$ \AA), indicating a pure Fmmm symmetry phase and robust superconductivity above 15 GPa.




\noindent {\small \bf \\Tight-binding results\\}
Next, we constructed a four-band $e_g$ orbital tight binding (TB) model in a bilayer lattice~\cite{Nakata:prb17,Maier:prb11,Maier:prb19,Maier:prb22} for the Fmmm phase. It is four orbitals because the unit cell contains
two Ni's, and each Ni contributes two orbitals. As pressure increases, the values of the hopping matrix elements also increase. As shown in Fig.~\ref{bandstructre} {\bf a}, the ratio of $t^{z}_{11}$ ($d_{3z^2-r^2}$ along the inter-layer direction) and $t^{x/y}_{22}$ ($d_{x^2-y^2}$ in plane) slightly decreases from 1.325 (0 GPa) to 1.286 (50 GPa), although with some small oscillations. Furthermore, the in-plane $\lvert$$t_{12}/t_{22}$$\rvert$ increases from 0.457 (0 GPa) to 0.483 (50 GPa), suggesting an enhanced hybridization between $d_{3z^2-r^2}$ and $d_{x^2-y^2}$ orbitals under pressure (see Fig.~\ref{bandstructre} {\bf b}). Moreover, the crystal field splitting $\Delta$ also increases under pressure, with small oscillations, as displayed in Fig.~\ref{bandstructre} {\bf c}. These small oscillations may be caused by a lack of convergence of optimized crystal structures at some pressure values, but do not change the main physical conclusions discussed in our publication.


% Figure environment removed


The TB calculations indicate that the electronic density of the $d_{3z^2-r^2}$ orbital gradually reduces from 1.86 (0 GPa) to 1.78 (50 GPa), as shown in Fig.~\ref{bandstructre} {\bf d}. Note that the electronic population of both orbitals is not an integer, thus this system is ``self-doped''. The band structures indicate that the bandwidth of $e_g$ orbitals increases by about
$\sim 23.1$ $\%$, from 0 ($\sim 3.63$ eV) to 50 ($\sim 4.47$ eV) GPa (see Figs.~\ref{bandstructre} {\bf e} and {\bf f}). Furthermore, the $e_g$ states of Ni display the orbital-selective spin singlet formation behavior~\cite{Zhang:arxiv}, where the energy gap $\Delta E$
between bonding and antibonding states of the $d_{3z^2-r^2}$ orbital increases by about $20\%$ from 0 GPa ($\sim 1.20$ eV) to 50 GPa ($\sim 1.44$ eV).

In addition, a van Hove singularity (vHS) near the Fermi level was also found at the $X$ point ($\pi$, 0) in the BZ (see Figs.~\ref{bandstructre} {\bf e} and {\bf f}), indicating a possible stripe ($\pi$, 0) order instability. As pressure increases, the vHS shifts away from the Fermi level, leading to reduced magnetic scattering near ($\pi$, 0), as discussed in the following section. We wish to remark that having the vHS at exactly the Fermi energy is not necessary for the stability of the magnetic stripe order. It is sufficient to have the vHS close to that Fermi energy so that the associated wavevector ($\pi$, 0) dominates. The FS consists of two electron pockets ($\alpha$ and $\beta$) with a mixture of $d_{3z^2-r^2}$ and $d_{x^2-y^2}$ orbitals, while the $\gamma$ hole-pocket is made up almost exclusively of the $d_{3z^2-r^2}$ orbital at all pressures we studied (see $P = 0$ and 50 GPa as examples in Figs.~\ref{bandstructre} {\bf g} and {\bf h}, and the Supplementary Note II for other pressures). The $\gamma$ pocket increases in size with pressure, while the size of the $\beta$ pocket decreases at high pressure.

\noindent {\small \bf \\Stripe order instability\\}
To better understand the tendency towards a possible magnetic instability in LNO under pressure, several possible in-plane magnetic structures of the Ni bilayer spins were considered here: A-AFM with wavevector (0, 0), G-AFM with ($\pi$, $\pi$), and stripe ($\pi$, 0), as shown in Fig.~\ref{MP} {\bf a}. In all cases, the coupling between the two Ni layers of the bilayer was assumed to be antiferromagnetic (AFM) due to the large interlayer hopping discussed in previous studies~\cite{Luo:arxiv,Zhang:arxiv}. Note that the possible in-plane stripe order ($\pi$, 0) can be understood as induced by the strong competition between intraorbital and interorbital dominated hopping mechanisms~\cite{Lin:prl21}, namely the competition between AFM and ferromagnetic (FM) tendencies that may induce a state with half the bonds AFM and half FM. To discuss the importance of $J$ for the stripe instability, based on the same crystal structures, we used the Liechtenstein formulation within the double-counting item to deal with the onsite Coulomb interactions, where $U$ and $J$ are independent variables~\cite{Liechtenstein:prb}.


% Figure environment removed



As displayed in Figs.~\ref{MP} {\bf b} and {\bf c}, the stripe ($\pi$, 0) magnetic order has the lowest energy among the three magnetic candidates considered in the pressure region that we studied, using robust Hund couplings $J = 0.8$ eV and $J = 1.0$ eV. Furthermore, the energy differences between the stripe and other magnetic configurations decreases as the pressure increases.

Note that the order (from lower to higher energy) is stripe, G-AFM, A-AFM, and finally FM  phases, when working at $U = 4$ eV and $J  = 0.8$ eV, and at 30 GPa, which is exactly the same as in our previous recent work~\cite{Zhang:arxiv}. However,  the energy differences are not the same. The reason is that in our previous work~\cite{Zhang:arxiv} we relaxed the atomic positions and used the experimental lattice constants at 300 K provided by the original discovery publication~\cite{Sun:arxiv}. However, in our present efforts we optimized the atomic positions and lattice constants at 0 K. This leads to energy differences between various magnetic states, but the relative order of those phases reamins the same.

By reducing $J$ to 0.6 eV (see Fig.~\ref{MP} {\bf d}), the stripe order ($\pi$, 0) has the lowest energy below 25 GPa,
while it has a close energy ($\sim 0.3$ meV/Ni) to the G-AFM phase above 25 GPa, suggesting the important role of $J$ to stabilize stripe order. Furthermore, the strongly reduced energy difference indicates the tendency of the strong competition between FM and AFM in the plane direction increasing pressure, suggesting that long-range spin order may not develop in the Fmmm phase under high pressure. Under pressure, the intraorbital hopping of the $e_g$ orbitals increases, enhancing the
canonical AFM Heisenberg interaction. Furthermore, the reduced $J$ would also reduce the FM coupling caused by the interorbital hopping between half-filled and empty orbitals via Hund's coupling $J$~\cite{Zhang:arxiv,Lin:prl21}.
If continuing to reduce $J$, then the AFM Heisenberg interaction induced by the intraorbital hopping will eventually win, leading to
G-type AFM order (see the results for $J = 0.4$~eV in Supplementary Note III).

Considering quantum fluctuations, the system may not develop long-range order due to the in-plane AFM and FM competition
in portions of the vast parameter space involving hoppings, Hund coupling $J$, and Hubbard interaction $U$.
This competition deserves further many-body model studies.
In addition, the calculated magnetic moment of the magnetic stripe phase also decreases under pressure (Fig.~\ref{MP} {\bf e}) because increasing the hoppings (namely, increasing
the bandwidth $W$) ``effectively'' reduces the electronic correlation via $U/W$. Finally, note that the stripe order ($\pi$, 0) is
degenerate with stripe order (0, $\pi$). Thus, there could occur an Ising spontaneous symmetry breaking upon cooling before long-range order is reached. In the context of the study of iron-based superconductors, ``nematicity'' was extensively discussed based on having in-plane stripe ($\pi$, 0) spin order~\cite{nematic1,nematic2}. The essence of this phenomenon is that the stripe states with wavevectors $k_1$=($\pi$, 0) and $k_2$=($\pi$, 0) should be degenerate
by symmetry and, thus, at high temperature their spin structure factors $S(k)$ should be equal .
Then, upon cooling two transitions could potentially exist. At the first one, say $T_{nem}$,
the spin structure factor of these two wave vectors becomes different, signalling a dominance of one stripe over the other.
This is the ``nematic'' state where rotational invariance by $90^o$ degrees is spontaneously broken, but there is yet no long
range order. At a lower temperature, the true N\'eel temperature, long-range order is finally established.
Consequently, our theoretical results for LNO indicate the possible existence of ``nematicity'' in LNO as well, as it occurs in iron-based superconductors~\cite{nematic1,nematic2}. However, this issue certainly merits further investigation
and detailed discussion that is left for future studies.



To assess the DFT extracted TB models for their magnetic and superconducting behavior, we have performed multi-orbital RPA calculations (see Methods section) for the Fmmm phase of LNO. Figure~\ref{fig:RPA_chi} shows the static RPA enhanced spin susceptibility $\chi'({\bf q}, \omega=0)$ for $q_z=\pi$ and $q_x$, $q_y$ along a high-symmetry path in the Brillouin zone. At 0 GPa, $\chi'({\bf q}, \omega=0)$ has a strong peak near the stripe wavevector ${\bf q}=(\pi, 0)$. A closer inspection of the contributions to the spin susceptibility shows that the dominant scattering process giving rise to this peak comes from intra-orbital $d_{3z^2-r^2}$ scattering between the $(0, \pi)$ region on the $\beta$ electron pocket and the $\gamma$ hole pocket at $(\pi, \pi)$ (see the results in Supplementary Note IV). In addition, we also wish to remark that this strong increase of the magnetic scattering at this wavevector is much enhanced by the vHs that happens at $X$ point in our TB calculation. The peak in the spin susceptibility of Fig.~\ref{fig:RPA_chi} at ambient pressure will be much reduced if the vHS shifts further down from the Fermi level. Thus, the vHS is crucial for the sharp peak features in the spin susceptibility. Because the Fmmm phase is not stable at 0 GPa, here we will not elaborate further about the
strong influence of the vHS on the susceptibility.


% Figure environment removed



For larger pressures, this saddle-point moves away from the Fermi level (see Fig.~\ref{bandstructre} {\bf f}), and, as a consequence, the magnetic scattering near $(\pi, 0)$ is reduced, as shown in Fig.~\ref{fig:RPA_chi}. This is also in agreement with the decreasing energy differences between stripe ($\pi$, 0) and other magnetic states under pressures obtained from DFT+$U$ calculations (see Fig.~\ref{MP}). Furthermore, the huge reduction of magnetic scattering under pressure also suggests the long-range spin stripe order may not be stable at high pressure, which may explain the absence of long-range order in the Fmmm phase under pressure. However, the short-range order is still possible.


\noindent {\small \bf \\Pairing symmetry\\}
In the RPA approach, the spin (and also charge) susceptibilities enter directly in the pairing interaction for the states on the FS. Figure~\ref{fig:RPA_gap} displays the leading pairing symmetry $g_{\alpha}({\bf k})$ obtained from solving the eigenvalue problem in Eq.~(\ref{eq:pp}) (Methods section) for the RPA pairing interaction for the model at (a) 0 GPa and (b) 25 GPa. In both cases, the leading superconducting gap has an $s^\pm$ structure, where the gap switches sign between the $\alpha$ and $\beta$ pockets, and between the $\beta$ and $\gamma$ pockets. As for the spin susceptibility, and as one would expect for spin-fluctuation mediated pairing, a detailed analysis of the different contributions to the $s^\pm$ pairing strength reveals that this gap structure is driven by intra-orbital $(\pi, 0)$ scattering between the $(0, \pi)$ region of the $\beta$-sheet with significant $d_{3z^2-r^2}$ character of the Bloch states, to the $d_{3z^2-r^2}$ $\gamma$-pocket at $(\pi, \pi)$. Moreover, the gap amplitude on the $\alpha$ pocket grows relative to that on the $\beta$ and $\gamma$ pockets with increasing pressure. Also, independent of pressure, the gap on the $\beta$ pocket has strong momentum dependence, becoming very small near the zone diagonal where it has accidental nodes. We reserve a detailed analysis of the factors leading to this momentum dependence for a future study.


% Figure environment removed


In addition, we show in Fig.~\ref{fig:RPA_lambda} the pressure dependence of the pairing strength $\lambda$ of the leading $s^\pm$ gap. The $s^\pm$ gap remains the leading instability over the subleading $d_{x^2-y^2}$ gap over the entire pressure range we considered. With increasing pressure, both pairing strengths monotonically decrease. In the RPA approach we use, changes in the pairing strength $\lambda$ translate to changes in the superconducting transition temperature $T_c$ through a Bardeen-Cooper-Schrieffer (BCS) like equation,  $T_c=\omega_0e^{-1/\lambda}$, where $\omega_0$ is a cut-off freqeuncy that is determined from the spin-fluctuation spectrum. We wish to remark that the quite
drastic increase of the pairing strength in the $s^\pm$ channel as we reach ambient pressure is much enhanced by the
vHS. But even if the vHS would not be exactly at the Fermi energy, to the extent that it is simply close to the Fermi
energy would be sufficient for dominance of the $s^\pm$ channel. Furthermore, our RPA shows the pairing strength in the $s^\pm$ channel is considerable in the pressure region we studied, indicating a broad superconducting region, which can explain qualitatively that superconductivity was found in a broad pressure region in the experiment at the region they studied (14 to 43.5 GPa)~\cite{Sun:arxiv}.


% Figure environment removed

Moreover, very recent angle-resolved photoemission spectroscopy experiment reveals that the hole $\gamma$-pocket made of $d_{3z^2-r^2}$ was absent in the Amam phase at ambient pressure~\cite{Yang:arxiv-exp}, while this pocket was found in the DFT studies in the high-pressure Fmmm phase~\cite{Sun:arxiv,Luo:arxiv,Zhang:arxiv}. To understand the importance of this hole $\gamma$-pocket made of $d_{3z^2-r^2}$, here, we also performed additional calculations for a model with artificially large crystal-field splitting $\Delta=0.6$ eV at 0 GPa, for which the hole band sinks below the Fermi level and the $\gamma$-pocket disappears (see the results in Supplementary Note V).  For this case, $\lambda_{s^\pm}$ is suppressed significantly from 1.55 (at $\Delta=0.474$ eV) to 0.040, and $\lambda_{d_{x^2-y^2}}$ becomes the leading solution, albeit with a similarly small $\lambda_{d_{x^2-y^2}} = 0.045$. Consistent with the discussion above, this provides further evidence of the importance of the $(\pi, \pi)$ $\gamma$ hole pocket in mediating superconductivity in this system. This could explain the absence of superconductivity in the low-pressure Amam phase of LNO~\cite{Sun:arxiv}, where the $\gamma$ pocket around ($\pi$, $\pi$) vanishes, indicating the importance of the Fmmm phase for the superconductivity in LNO system.



It may occur that small variations in the Hubbard $U$ may lead to qualitatively different results. For this reason, we varied $U$ in an allowed range before a spin-density-wave state starts dominating at 25 GPa in the RPA context. This requirement
establishes $U=1.05$ as the upper limit that can be studied within our RPA formalism. In Table~\ref{Table1}, we provide the values of $\lambda$ for the $s_{\pm}$ and $d$-wave channels
with increasing $U$ in that allowed range. The results show that $\lambda$ increases smoothly
with increasing $U$ and our study produces a dominant $s_{\pm}$ pairing state in the entire range analyzed~\cite{Zhang:arxiv10}, suggesting that our results are stable under small variations of the Hubbard strength.


\begin{table}[]
\centering\caption{Pairing results in the $s_{\pm}$ and $d$-wave channels corresponding to the range of $U$ allowed by the RPA formalism before a spin-density-wave state starts dominating. In the entire range studied, the $s_{\pm}$ state is the prevailing superconducting channel.}
\begin{tabular}{lll}
\hline
\multicolumn{3}{|l|}{Table I|Pairing strength $\lambda$ vs $U$}                                                                             \\ \hline
\multicolumn{1}{|l|}{$U$}  & \multicolumn{1}{l|}{$\lambda_s$}   & \multicolumn{1}{l|}{$\lambda_d$}          \\ \hline
\multicolumn{1}{|l|}{0.80} & \multicolumn{1}{l|}{0.141}         & \multicolumn{1}{l|}{0.020}             \\ \hline
\multicolumn{1}{|l|}{0.82} & \multicolumn{1}{l|}{0.162}         & \multicolumn{1}{l|}{0.023}             \\ \hline
\multicolumn{1}{|l|}{0.84} & \multicolumn{1}{l|}{0.186}         & \multicolumn{1}{l|}{0.026}             \\ \hline
\multicolumn{1}{|l|}{0.86} & \multicolumn{1}{l|}{0.215}         & \multicolumn{1}{l|}{0.030} \\ \hline
\multicolumn{1}{|l|}{0.88} & \multicolumn{1}{l|}{0.249}         & \multicolumn{1}{l|}{0.035} \\ \hline
\multicolumn{1}{|l|}{0.90} & \multicolumn{1}{l|}{0.290}         & \multicolumn{1}{l|}{0.041} \\ \hline
\multicolumn{1}{|l|}{0.92} & \multicolumn{1}{l|}{0.339}         & \multicolumn{1}{l|}{0.049} \\ \hline
\multicolumn{1}{|l|}{0.94} & \multicolumn{1}{l|}{0.400}         & \multicolumn{1}{l|}{0.060} \\ \hline
\multicolumn{1}{|l|}{1.00} & \multicolumn{1}{l|}{0.691}         & \multicolumn{1}{l|}{0.125} \\ \hline
\multicolumn{1}{|l|}{1.05} & \multicolumn{1}{l|}{1.264}         & \multicolumn{1}{l|}{0.330} \\ \hline
\multicolumn{3}{l}{}
\end{tabular}
\label{Table1}
\end{table}




Note that above $U=1.05$, where magnetic order develops, in principle pairing could still occur. To address this matter, the RPA formalism must be generalized by carrying out the resummation of bubble diagrams now using, e.g.,
dressed propagators corresponding to the dominant magnetic order. This task is demanding and results will be presented in future work.


\noindent {\small \bf \\Charge order instability\\}
In the experimental studies, they excluded the possibility of the presence of charge-density-wave order at low temperatures under pressures from 14.0 to 43.5 GPa, based on the resistance measurements~\cite{Sun:arxiv}. Here, we will also briefly discuss charge order (CO) instabilities in the Fmmm phase under pressure, by carrying out DFT calculations. The checkerboard CO (G-type) configuration was considered, with both AFM coupling in-plane and between the two Ni layers. This checkerboard charge order was proposed in another bilayer nickelate La$_3$Ni$_2$O$_6$~\cite{Botana:prb16}. Here, we studied two specific pressure values (15 GPa and 30 GPa) in the Fmmm phase, with the two values in the ``superconducting'' region of the phase diagram in the experiments~\cite{Sun:arxiv,Zhang:arxiv-experiment}.


\begin{table}[]
\centering\caption{Energy differences (meV/Ni) and calculated magnetic moment ($\mu_B$/Ni) for the various input spin configurations by using the same lattice structure. Here, the stripe configuration is taken as the reference of energy.
In addition, we also considered the possibility of structural distortions, namely lattice relaxations, and all these results are shown in parenthesis.}
\begin{tabular}{lll}
\hline
\multicolumn{3}{|l|}{Table II|Charge order instability}                                                                   \\ \hline
\multicolumn{3}{|l|}{15 GPa ($J = 0.6$ eV)}                                                                             \\ \hline
\multicolumn{1}{|l|}{Magnetism} & \multicolumn{1}{l|}{Energy(Enthalpy}  & \multicolumn{1}{l|}{Magnetic moment}          \\ \hline
\multicolumn{1}{|l|}{Stripe}    & \multicolumn{1}{l|}{0(0)}             & \multicolumn{1}{l|}{0.895(0.933)}             \\ \hline
\multicolumn{1}{|l|}{G-AFM}     & \multicolumn{1}{l|}{8.17(16.10)}      & \multicolumn{1}{l|}{0.630(0.650)}             \\ \hline
\multicolumn{1}{|l|}{A-AFM}     & \multicolumn{1}{l|}{20.77(30.10)}     & \multicolumn{1}{l|}{0.598(0.650)}             \\ \hline
\multicolumn{1}{|l|}{CO}        & \multicolumn{1}{l|}{8.11(13.89)}      & \multicolumn{1}{l|}{0.933/0.377(0.968/0.335)} \\ \hline
\multicolumn{3}{|l|}{30 GPa ($J = 0.6$ eV)}                                                                             \\ \hline
\multicolumn{1}{|l|}{Magnetism} & \multicolumn{1}{l|}{Energy(Enthalpy)} & \multicolumn{1}{l|}{Magnetic moment}          \\ \hline
\multicolumn{1}{|l|}{Stripe}    & \multicolumn{1}{l|}{0(0)}             & \multicolumn{1}{l|}{0.658(0.702)}             \\ \hline
\multicolumn{1}{|l|}{G-AFM}     & \multicolumn{1}{l|}{-0.03(3.82)}       & \multicolumn{1}{l|}{0.560(0.574)}             \\ \hline
\multicolumn{1}{|l|}{A-AFM}     & \multicolumn{1}{l|}{11.91(16.58)}     & \multicolumn{1}{l|}{0.510(0.528)}             \\ \hline
\multicolumn{1}{|l|}{CO}        & \multicolumn{1}{l|}{-0.03(3.80)}      & \multicolumn{1}{l|}{0.561/0.560(0.600/0.549)} \\ \hline
\multicolumn{3}{|l|}{15 GPa ($J = 0.8$ eV)}                                                                             \\ \hline
\multicolumn{1}{|l|}{Magnetism} & \multicolumn{1}{l|}{Energy(Enthalpy)} & \multicolumn{1}{l|}{Magnetic moment}          \\ \hline
\multicolumn{1}{|l|}{Stripe}    & \multicolumn{1}{l|}{0(0)}             & \multicolumn{1}{l|}{0.976(1.009)}             \\ \hline
\multicolumn{1}{|l|}{G-AFM}     & \multicolumn{1}{l|}{27.90(37.63)}     & \multicolumn{1}{l|}{0.725(0.746)}             \\ \hline
\multicolumn{1}{|l|}{A-AFM}     & \multicolumn{1}{l|}{29.18(36.01)}     & \multicolumn{1}{l|}{1.027(1.067)}             \\ \hline
\multicolumn{1}{|l|}{CO}        & \multicolumn{1}{l|}{24.31(17.47)}     & \multicolumn{1}{l|}{1.041/0.368(1.236/0.364)} \\ \hline
\multicolumn{3}{|l|}{30 GPa ($J = 0.8$ eV)}                                                                             \\ \hline
\multicolumn{1}{|l|}{Magnetism} & \multicolumn{1}{l|}{Energy(Enthalpy}  & \multicolumn{1}{l|}{Magnetic moment}          \\ \hline
\multicolumn{1}{|l|}{Stripe}    & \multicolumn{1}{l|}{0(0)}             & \multicolumn{1}{l|}{0.913(0.944)}              \\ \hline
\multicolumn{1}{|l|}{G-AFM}     & \multicolumn{1}{l|}{13.24(22.30)}     & \multicolumn{1}{l|}{0.651(0.667)}             \\ \hline
\multicolumn{1}{|l|}{A-AFM}     & \multicolumn{1}{l|}{27.30(35.98)}     & \multicolumn{1}{l|}{0.726(0.780)}             \\ \hline
\multicolumn{1}{|l|}{CO}        & \multicolumn{1}{l|}{13.71(20.89)}     & \multicolumn{1}{l|}{0.795/0.514(0.953/0.395)} \\ \hline
\multicolumn{3}{l}{}
\end{tabular}
\label{Table2}
\end{table}


First, we used the specific values $U = 4$ eV and $J = 0.6$ eV, very similar to those obtained from constrained density functional calculations ($U \sim 3.8$ eV and $J \sim 0.61$ eV)~\cite{Christiansson:arxiv}, as shown in Table~\ref{Table2}.
Without lattice relaxations, in our study at 15 GPa, we observed a strong charge disproportionation in this charge order state with values 0.933 and 0.377 $\mu_B$/Ni for two different Ni sites, but this state has a higher energy of about ($\sim 8.11$ meV/Ni) than the magnetic stripe ground state. Moreover, we do not obtain any obvious charge disproportionation at 30 GPa, as displayed in Table~\ref{Table2}. In addition, we also considered the possibility of structural distortions for different magnetic configurations, namely lattice relaxations, and this enhances the charge disproportionation in the checkerboard CO state, but our conclusions above did not change. Increasing $J$ to 0.8 eV, by introducing the structural distortions, the charge disproportionation increases to 1.236/0.364 and 0.953/0.395 $\mu_B$/Ni for 15 and 30 GPa, respectively, in this charge order state. Even though the charge disproportionation is enhanced in this charge order state with higher $J$, this state still has higher enthalpy ($\sim$ 17.47 and 20.89 meV/Ni) than the magnetic stripe state at 15 and 30 GPa.  Hence, from our DFT perspective, the charge density wave state is not stable in the Fmmm phase under pressure, at least at low temperatures. This is in agreement with the experimental observations.


\noindent {\small \bf \\Comparison with IL nickelates and cuprates superconductors\\}
The discovery of high $T_c \sim 80$ K in LNO naturally reminds us of the recently discussed IL nickelates as well as the
previous widely studied cuprates superconductors, and show some significant differences. The experimentally observed phase diagram of the bilayer LNO is significantly different from the previously reported phase diagram of IL nickelates and cuprates superconductors, where no obvious sharp and narrow superconducting dome was obtained~\cite{Sun:arxiv,Hou:arxiv,Zhang:exp-arxiv,Zhang:arxiv-experiment}. On the contrary, a very broad pressure superconducting region was observed in LNO. This suggests that LNO system is quite unique, compared with IL nickelates and cuprates superconductors.


Based on our results described here,  the strong inter-layer coupling caused by the $d_{3z^2-r^2}$ orbital leads to possible $s_{\pm}$-wave pairing superconductivity in LNO in a broad pressure region. However, the inter-layer coupling in IL nickelates is weak~\cite{Nomura:Prb,Nomura:rpp}.  These combined results suggest that the $d_{3z^2-r^2}$ orbital plays a quite different role in LNO and in the previously discovered IL nickelates. In addition, a robust interorbital hopping between $e_g$ orbitals was found via DFT in LNO, while this hopping is nearly zero in the IL nickelates, leading to in-plane stripe vs G-AFM spin order in those two systems, as discussed above~\cite{Zhang:prb20,Zhang:arxiv}. These two characteristics are the main differences between the two systems.

In the cuprates, superconductivity is believed to be driven by the in-plane AFM fluctuations of the $d_{x^2-y^2}$ orbital, resulting in $d$-wave superconductivity~\cite{Dagotto:rmp94}. In addition, the inter-layer coupling is weak in the cuprates. Furthermore, oxygen also plays a different role in the bilayer LNO and cuprates superconductors. LNO has a large charge-transfer gap from oxygen $p$ to Ni's $3d$ orbitals~\cite{Zhang:arxiv}, resulting in being more close to a Mott-Hubbard system. But this charge-transfer gap is quite smaller in the cuprate superconductors, leading to a charge-transfer system~\cite{Zhang:prb20}. Moreover, in the context of the study of cuprates, it is well known that $T_c$ increases substantially going from 1 layer to 2 layers, and then slightly increases further in the case of 3 layers~\cite{Wang:prm}. However, the RP Ni-oxide layered materials are not simply following the same trend as the cuprates: the superconductivity was not yet observed in single-layer at the pressure region studied~\cite{Zhang:arxiv-experiment} but LNO is already superconducting in this region. In addition, the superconductivity was also not found in a earlier study~\cite{Zhang:arxiv-experiment} but   signatures of superconductivity were reported in trilayer RP nickelate under pressure~\cite{Li:cpl}. Interestingly, the inter-layer coupling is also weak in the single-layer, suggesting the importance of inter-layer coupling for the superconductivity in Ni-oxide layered materials. All these results combined strongly indicate that LNO is qualitatively different from both IL nickelates and cuprates superconductors.

In addition, for completeness we also considered a single $d_{x^2-y^2}$-orbital model for BSCCO at a filling of $n = 0.85$, based on the hopping obtained from previous work~\cite{Kreisel:prl}. To find the same $\lambda$ as obtained for LNO at 25 GPa,
we need a $\sim 25 \%$ larger $U$ for BSCCO, which seems reasonable because cuprates are widely believed to have stronger electronic correlations than nickelates. However, it should be noted that a similar $\lambda$ does not necessarily translate to the same $T_c$, because $T_c=\omega_0e^{-1/\lambda}$. Then, the cut-off frequency prefactor $\omega_0$ also plays a role, and $\omega_0$ is different in those two systems. To provide a more quantitative comparison in $T_c$ between LNO and a typical cuprate, many additional complex effects must be incorporated, more orbitals are need in the model calculations, as well as a different cut-off frequency prefactor $\omega_0$ if the RPA formalism is still being used.
But at least the qualitative trends appear to be correctly reproduced: cuprates require a larger $U$ than nickelates.

\noindent {\bf \\Discussion\\}
The recently discovered bilayer nickelate superconductor LNO has opened a new platform for the study of the origin of unconventional superconductivity, unveiling several challenging results that theory needs to explain.
Combining first-principles DFT and many-body RPA methods, here we comprehensively studied the LNO system under pressure from $P = 0$ GPa to $P = 50$ GPa. Based on group analysis, the distortion Y$^{2-}$ mode induces
the structural transition from Fmmm (No. 69) to Amam (No. 63). At 0 GPa, the Amam phase has lower energy ($\sim  -21.01$ meV/Ni)
than the Fmmm phase due to a large distortion amplitude of Y$^{2-}$ mode ($\sim 0.7407$ \AA).
By introducing pressure, the Y$^{2-}$ mode amplitude is gradually reduced, reaching nearly zero value at 15 GPa ($\sim 0.0016$ \AA), while the enthalphy difference [$\Delta H$ = H(Amam)-H(Fmmm)] also decreases.
Furthermore, there is no imaginary frequency obtained for the Amam structure from 0 to 15 GPa. The Fmmm phase eventually
becomes stable at $\sim 10.6$ GPa, which is quite close to the experimentally observed critical pressure ($\sim 10$ GPa) of
the Fmmm structure of LNO.


In the pressure range from 10.6 to 14 GPa, we found that the enthalpy differences between the Amam and Fmmm phases are quite small ($\sim$ 0.3 meV/Ni), while the Y$^{2-}$ mode amplitude is still considerable ($\sim 0.1$ ~\AA). Furthermore,
both the Amam and Fmmm phases are stable in this pressure region. In this case, the Amam phase could coexist with the Fmmm phase in samples of LNO, suggesting a first-order pressure-induced
structural phase transition from Amam to Fmmm. Due to the existence of the Amam phase in the Fmmm structure of the LNO sample, as an overall effect the superconducting $T_c$ would be gradually reduced or fully vanish in the intermediate pressure region,
leading to potentially sample-dependent issues, supporting recent experiments where zero resistance was found
below 10~K in some samples above 10 GPa and below 15 GPa~\cite{Hou:arxiv}. This could qualitatively
explain the absence of superconductivity in LNO for the Fmmm
structure in the pressure region from 10 to 14 GPa~\cite{Sun:arxiv}.  More detailed studies are needed
to confirm our results and obtain direct experimental evidence for the coexistence of the Amam and Fmmm
phases in this pressure region.


Furthermore, a vHs near the Fermi level was found at the $X$ ($\pi$, 0) point in the BZ in our TB band structure, indicating a possible stripe ($\pi$, 0) order instability.
Our DFT+$U$+$J$ and RPA results both indicate that the magnetic
stripe phase with wavevector $(\pi, 0)$ [degenerate with
$(0, \pi)$] should dominate once Hubbard and Hund
correlation effects are taken into account, at least in the intermediate
range of Hubbard-Hund couplings. Moreover, due to the particular shape of the FS, and its orbital composition, the $s^\pm$ pairing channel should be the pairing symmetry of LNO. The subleading superconducting instability was the $d_{x^2-y^2}$ state.
Furthermore, we also found that the $d$-wave pairing channel has energy close to the $s$-wave pairing channel. Thus, in experiments, depending on the specific chemical formula and pressure and sample quality, experimentalists may see $s_{\pm}$-wave or $d$-wave superconductivity tendencies in this bilayer family, or even coexisting signals,
which also deserves further experimental and theoretical studies.



The significant suppression of the $s^{\pm}$ pairing strength we find in calculations without the
$\gamma$ pocket provide strong evidence of the importance of the $(\pi, \pi)$ $\gamma$ hole pocket
in mediating superconductivity in this system.
This could also explain why superconductivity is only observed in the high-pressure Fmmm phase since the $\gamma$ hole pocket around ($\pi$, $\pi$) vanishes in the Amam phase, suggesting the importance of the Fmmm phase (stable $\gamma$ pocket) for superconductivity in the LNO system. In addition, from our DFT perspective, the charge density wave state seems to be unlikely in the Fmmm phase under pressure at least at low temperatures. In addition, our efforts in this work also indicate the bilayer nickelate LNO supercondoctor is unique, because it is  qualitatively different from previously discussed IL nickelates and cuprate superconductors. However, to provide a good quantitative comparison in $T_c$ between LNO and cuprates, as well as IL nickelates, many complicated effects and more orbitals need to be considered in the further model calculations.





\noindent {\bf \\Methods\\}

\noindent {\small \bf DFT method\\}
In this work, first-principles DFT calculations were implemented based on the Vienna {\it ab initio} simulation package (VASP) code, by using the projector augmented wave (PAW) method~\cite{Kresse:Prb,Kresse:Prb96,Blochl:Prb}.
The electronic correlations were considered by the generalized gradient approximation and the Perdew-Burke-Ernzerhof exchange potential~\cite{Perdew:Prl}. The plane-wave cutoff energy was set as $550$~eV and a $k$-point grid $12\times12\times3$
was adopted for the conventional structure of LNO of both the Amam and Fmmm phases. Note we also tested that this $k$-point mesh produces converged energies. Moreover, the lattice constants and atomic positions were fully relaxed until the Hellman-Feynman force on each atom was smaller than $0.01$ eV/{\AA}. Moreover, the onsite Coulomb interactions were considered via the Dudarev formulation~\cite{Dudarev:prb}. Here, we used the value $U_{eff} = 4$ eV in the relaxation of crystal structures under pressure, following recent DFT studies of LNO~\cite{Sun:arxiv,Christiansson:arxiv,Chen:arxiv}. Our optimized lattice parameters are $a = 5.434$ \AA, $b = 5.367$ \AA,  and $c = 20.670$ \AA ~for the Amam phase at ambient conditions, which are in good agreement with the experimental values, such as in the neutron data ($a = 5.448$ \AA, $b = 5.393$ \AA, and $c = 20.518$ \AA)~\cite{Ling:jssc} and in recent X-ray diffraction ($a = 5.438$ \AA, $b = 5.400$, and $c = 20.455$ \AA)~\cite{Liu:scpma}. We also notice that a recent linear response study suggests $\sim 6$ eV for LNO~\cite{Sun:arxiv}. However, our optimized lattice parameters of the Amam phase without pressure for $U_{eff} = 6$ eV are $a = 5.366$ \AA, $b = 5.299$ \AA, $c = 21.429$ \AA, in good agreement with experiments. While the calculation of structural parameters does not provide sufficient basis for our specific choice of $U_{eff}$, considering together
all the results mentioned above, it is clear that the choice of $U_{eff} = 4$ eV is better than $U_{eff} = 6$ eV.

We calculated the phonon spectra of the Amam and Fmmm phases for different pressures by using the density functional perturbation theory approach~\cite{Baroni:Prl,Gonze:Pra1,Gonze:Pra2}, analyzed by the PHONONPY software in the primitive unit cell~\cite{Chaput:prb,Togo:sm}. Furthermore, the onsite Coulomb interactions were considered via the Dudarev formulation~\cite{Dudarev:prb} with $U_{eff} = 4$ eV. Here, we considered a pressure grid with an interval of 1 Gpa from 0 to 15 Gpa for the Amam and Fmmm phases, but the interval changed to 0.1 Gpa near the critical pressures. In addition,
a pressure grid with an interval of 5 Gpa was used for the Fmmm phase
from 15 to 50 Gpa. To avoid repeating too many displays, we only show the phonon spectrum of a few key values of pressures in both the main text and in the supplemental materials (see these results in Supplementary Note I). We chose the conventional cell structures, corresponding to the $\sqrt2\times\sqrt2\times1$ supercell of the undistorted parent $I4/mmm$ conventional cell, in order to study the dynamic stability. This is enough to obtain possible unstable modes for the 327-type RP perovskite system. In addition to the standard DFT calculation discussed thus far, the maximally localized Wannier functions (MLWFs) method was employed to fit Ni's $e_g$ bands and obtain the hoppings and crystal-field splitting for model calculations, by using the WANNIER90 package~\cite{Mostofi:cpc}.

To better understand the possible magnetic instabilities and the importance of the Hund coupling $J$ in the LNO system under pressure, we also used the DFT plus $U$ and $J$ within the Liechtenstein formulation using the double-counting item~\cite{Liechtenstein:prb}, where $U$ was fixed to be $4$ eV and $J$ was changed from 0.4 to 1 eV. In addition, we also show the band structures of the magnetic stripe phase from $J = 1$ eV to $J = 0.4$ eV at 25 GPa in the Supplementary Note III.

\noindent {\small \bf TB method\\}
In our TB model, a four-band bilayer lattice with filling $n = 3$ was used, corresponding to 1.5 electrons per site, where the kinetic hopping component is:
\begin{eqnarray}
H_k = \sum_{\substack{i\sigma\\\vec{\alpha}\gamma\gamma'}}t_{\gamma\gamma'}^{\vec{\alpha}}
(c^{\dagger}_{i\sigma\gamma}c^{\phantom\dagger}_{i+\vec{\alpha}\sigma\gamma'}+H.c.)+ \sum_{i\gamma\sigma} \Delta_{\gamma} n_{i\gamma\sigma}.
\end{eqnarray}
The first term represents the hopping of an electron from orbital $\gamma$ at site $i$ to orbital $\gamma'$ at the nearest-neighbor
(NN) site $i+\vec{\alpha}$. $c^{\dagger}_{i\sigma\gamma}$($c^{\phantom\dagger}_{i\sigma\gamma}$) is the standard creation (annihilation) operator, $\gamma$ and $\gamma'$ represent the different orbitals, and $\sigma$ is the $z$-axis spin projection. $\Delta_{\gamma}$ represents the crystal-field splitting of each orbital $\gamma$. The unit vectors $\vec{\alpha}$ are along the three directions.


The NN hopping matrix of different pressures was obtained from MLWFs. The detailed values can be found in Supplementary Note II. The Fermi energy is obtained by integrating the density of states for all $\omega$ to reach the number of electrons $n = 3$. Based on the obtained Fermi energy, a $4000\times4000$ k-mesh was used to calculate the Fermi surface. The main characters of those Fermi surfaces, namely hole pocket $\gamma$ and two electron sheets ($\alpha$ and $\beta$) are qualitatively in agreement with the present DFT and TB studies~\cite{Luo:arxiv,Lechermann:arxiv,Lechermann:arxiv,Gu:arxiv,LaBollita:arxiv}. However, the vHs just ``happen'' at the $X$ point in our TB band with nearest-neighbor hopping for 0 GPa, but this singularity would be softened by adding additional hoppings beyond nearest-neighbors, or by other means such as broadening by disorder. A small shift of the flat band at $X$ from the Fermi level cannot change the physics much by mere continuity.



\noindent {\small \bf RPA method\\}
The RPA method we used to assess the bilayer TB models for their magnetic and superconducting behavior is based on a perturbative weak-coupling expansion in the Coulomb interaction. It has been shown in many studies to capture the essence of the physics (e.g. Ref.~\cite{Romer2020}). The full Hamiltonian for the bilayer Hubbard model discussed here, includes the kinetic energy and interaction terms, and it is  written as $H = H_{\rm k} + H_{\rm int}$.

The electronic interaction portion of the Hamiltonian includes the standard same-orbital Hubbard repulsion $U$, the electronic repulsion  $U'$ between electrons
at different orbitals, the Hund's coupling $J$, and the on-site inter-orbital electron-pair hopping terms ($J'$). Formally, it is given by:
\begin{eqnarray}
H_{\rm int}= U\sum_{i\gamma}n_{i \uparrow\gamma} n_{i \downarrow\gamma} +(U'-\frac{J}{2})\sum_{\substack{i\\\gamma < \gamma'}} n_{i \gamma} n_{i\gamma'} \nonumber \\
-2J \sum_{\substack{i\\\gamma < \gamma'}} {{\bf S}_{i,\gamma}}\cdot{{\bf S}_{i,\gamma'}}+J \sum_{\substack{i\\\gamma < \gamma'}} (P^{\dagger}_{i\gamma} P_{i\gamma'}+H.c.),
\end{eqnarray}
where the standard relation $U'=U-2J$ and  $J' = J$ are assumed, and $P_{i\gamma}$=$c_{i \downarrow \gamma} c_{i \uparrow \gamma}$. Thus, there are only two free parameters.


In the multi-orbital RPA approach \cite{Kubo2007,Graser2009,Altmeyer2016}, the RPA enhanced spin susceptibility shown in Fig.~\ref{fig:RPA_chi} is obtained from the bare susceptibility (Lindhart function) $\chi_0({\bf q})$ as $\chi({\bf q}) = \chi_0({\bf q})[1-{\cal U}\chi_0({\bf q})]^{-1}$. Here, $\chi_0({\bf q})$ is an orbital-dependent susceptibility tensor and ${\cal U}$ is a tensor that contains the intra-orbital $U$ and inter-orbital $U'$ density-density interactions, the Hund's rule coupling $J$, and the pair-hopping $J'$ term. The pairing strength $\lambda_\alpha$ for channel $\alpha$ shown in Fig.~\ref{fig:RPA_lambda} and the corresponding gap structure $g_\alpha({\bf k})$ shown in Fig.~\ref{fig:RPA_gap} are obtained from solving an eigenvalue problem of the form
\begin{eqnarray}\label{eq:pp}
	\int_{FS} d{\bf k'} \, \Gamma({\bf k -  k'}) g_\alpha({\bf k'}) = \lambda_\alpha g_\alpha({\bf k})\,,
\end{eqnarray}
where the momenta ${\bf k}$ and ${\bf k'}$ are on the FS and $\Gamma({\bf k - k'})$ contains the irreducible particle-particle vertex. In the RPA approximation, the dominant term entering $\Gamma({\bf k-k'})$ is the RPA spin susceptibility $\chi({\bf k-k'})$. For the models considered here, we find that the eigenvector $g_\alpha({\bf k})$ corresponding to the largest eigenvalue $\lambda_\alpha$ has $s^\pm$ symmetry as shown in Fig.~\ref{fig:RPA_gap}.

\noindent {\bf {\small \\ Data availability\\}} All data needed to evaluate the conclusions presented in this study have been deposited in Figshare database under the following accession code \href{https://doi.org/10.6084/m9.figshare.25245994}{https://doi.org/10.6084/m9.figshare.25245994}. The data for our TB calculations are available in the main text or the supplementary materials. Any additional data that support the findings of this study are available from the corresponding author upon request.
\noindent {\bf {\small \\ Code availability\\}} The Ab initio calculations are done with the code VASP.  Simulation RPA codes are available from the corresponding author upon reasonable request.



\begin{references}
\bibitem{Li:Nature} Li, D. {\it et al.} {Superconductivity in an infinite-layer nickelate} \href{https://doi.org/10.1038/s41586-019-1496-5}{\it Nature} {\textbf{572}, 624 (2019).}
\bibitem{Nomura:Prb} Nomura, Y. {\it et al.} {Formation of a two-dimensional single-component correlated electron system and band engineering in the nickelate superconductor
NdNiO$_2$} \href{https://doi.org/10.1103/PhysRevB.100.205138}{\it Phys. Rev. B} {\textbf{100}, 205138 (2019).}
\bibitem{Botana:prx} Botana, A. S., \& Norman, M. R. {Similarities and Differences between LaNiO$_2$ and CaCuO$_2$ and Implications for Superconductivity} \href{https://doi.org/10.1103/PhysRevX.10.011024}{\it Phys. Rev. X} {\textbf{10}, 011024  (2020).}
\bibitem{Nomura:rpp} Nomura, Y. \& Arita, R. {Superconductivity in infinite-layer nickelates} \href{https://doi.org/10.1088/1361-6633/ac5a60}{\it Rep. Prog. Phys.} {\textbf{85}, 052501  (2022).}
\bibitem{Zhou:MTP} Zhou, X. {\it et al.} {Experimental progress on the emergent infinite-layer Ni-based superconductor} \href{https://doi.org/10.1016/j.mattod.2022.02.016}{\it Materials Today} {\textbf{55}, 170  (2022).}
\bibitem{Zhang:prb20} Zhang, Y. {\it et al.} {Similarities and differences between nickelate and cuprate films grown on a SrTiO$_3$ substrate} \href{https://doi.org/10.1103/PhysRevB.102.195117}{\it Phys. Rev. B} {\textbf{102}, 195117 (2020).}
\bibitem{Zeng:prl20} Zeng, S. {\it et al.} {Phase Diagram and Superconducting Dome of Infinite-Layer Nd$_{1-x}$Sr$_x$NiO$_2$Thin Films} \href{https://doi.org/10.1103/PhysRevLett.125.147003}{\it Phys. Rev. Lett.} {\textbf{125}, 147003 (2020).}
\bibitem{Pan:nm} Pan, G. A. {\it et al.} {Superconductivity in a quintuple-layer square-planar nickelate} \href{https://doi.org/10.1038/s41563-021-01142-9}{\it Nat. Mater.} {\textbf{21}, 160 (2022).}
\bibitem{Dagotto:rmp94} Dagotto, E. {Correlated electrons in high-temperature superconductors.} \href{https://doi.org/10.1103/RevModPhys.66.763}{\it Rev. Mod. Phys.}{ \textbf{66}, 763 (1994)}.
\bibitem{Lee:prb04} Lee, K.-W. \& Pickett, W. E. {Infinite-layer LaNiO$_2$: Ni$^{1+}$ is not Cu$^{2+}$} \href{https://doi.org/10.1103/PhysRevB.70.165109}{\it Phys. Rev. B} {\textbf{70}, 165109 (2004).}
\bibitem{Sakakibara:prl} Sakakibara, H. {\it et al.} {Model Construction and a Possibility of Cupratelike Pairing in a New $d^9$ Nickelate Superconductor (Nd,Sr)NiO$_2$} \href{https://doi.org/10.1103/PhysRevLett.125.077003}{\it Phys. Rev. Lett.} {\textbf{125}, 077003 (2020).}
\bibitem{Jiang:prl} Jiang,  M., Berciu, M. \& Sawatzky G. A. {Critical Nature of the Ni Spin State in Doped NdNiO$_2$} \href{https://doi.org/10.1103/PhysRevLett.124.207004}{\it Phys. Rev. Lett.} {\textbf{124}, 207004 (2020).}
\bibitem{Wu:prb} Wu, X. {\it et al.} {Robust $d_{x^2-y^2}$-wave superconductivity of infinite-layer nickelates} \href{https://doi.org/10.1103/PhysRevB.101.060504}{\it Phys. Rev. B} {\textbf{101}, 060504(R) (2020).}
\bibitem{Gu:prb} Gu, Y., Zhu, S., Wang, X.,  Hu, J. \& Chen, H. {A substantial hybridization between correlated Ni-d orbital and itinerant electrons in infinite-layer nickelates} \href{https://doi.org/10.1038/s42005-020-0347-x}{\it Commun. Phys.} {\textbf{3}, 84 (2020).}
\bibitem{Karp:prx} Karp, J. {\it et al.} {Many-Body Electronic Structure of NdNiO$_2$ and CaCuO$_2$} \href{https://doi.org/10.1103/PhysRevX.10.021061}{\it Phys. Rev. X} {\textbf{10}, 021061 (2020).}
\bibitem{Fowlie:np} Fowlie, J. {\it et al.} {Intrinsic magnetism in superconducting infinite-layer nickelates} \href{https://doi.org/10.1038/s41567-022-01684-y}{\it Nat. Phys.} {\textbf{18}, 1043 (2022).}
\bibitem{Rossi:np} Rossi, M. {\it et al.} {A broken translational symmetry state in an infinite-layer nickelate} \href{https://doi.org/10.1038/s41567-022-01660-6}{\it Nat. Phys.} {\textbf{18}, 869 (2022).}
\bibitem{Sun:arxiv} Sun, H. {\it et al.} {Signatures of superconductivity near 80 K in a nickelate under high pressure} \href{https://doi.org/10.1038/s41586-023-06408-7}{\it Nature}{\textbf{621}, 493 (2023).}
\bibitem{Hou:arxiv} Hou, J. {\it et al.} {Emergence of high-temperature superconducting phase in the pressurized La$_3$Ni$_2$O$_7$ crystals} \href{https://doi.org/10.1088/0256-307X/40/11/117302}{\it Chin. Phys. Lett.} {\textbf{40}, 117302 (2023).}
\bibitem{Zhang:exp-arxiv} Zhang, Y. {\it et al.} {High-temperature superconductivity with zero-resistance and strange metal behavior in La$_3$Ni$_2$O$_7$.} \href{https://doi.org/10.48550/arXiv.2307.14819}{\it arXiv} {2307.14819 (2023).}
\bibitem{Zhang:arxiv-experiment} Zhang, M. {\it et al.} {Effects of Pressure and Doping on Ruddlesden-Popper phases La$_{n+1}$Ni$_n$O$_{3n+1}$.} \href{https://doi.org/10.1016/j.jmst.2023.11.011}{\it J. Mater. Sci. Technol.} {\textbf{185}, 147 (2024).}
\bibitem{Wang:experiment} Wang, G. {\it et al.} {Pressure-induced superconductivity in polycrystalline La$_3$Ni$_2$O$_{7-\delta}$}
\href{https://doi.org/10.48550/arXiv.2309.17378}{\it arXiv} {2309.17378 (2023)}
\bibitem{Ling:jssc} Ling, C. D., Argyriou, D. N., Wu, G. \& Neumeier,  {Neutron Diffraction Study of La$_3$Ni$_2$O$_7$: Structural Relationships Among n = 1, 2, and 3 Phases La$_{n+1}$Ni$_n$O$_{3n+1}$} \href{https://doi.org/10.1006/jssc.2000.8721}{\it J. Solid State Chem.} {\textbf{152}  517 (2000).}
\bibitem{Zhang:jssc} Zhang, Z., Greenblatt,  M. \& Goodenough, J. B. {Synthesis, Structure, and Properties of the Layered Perovskite La$_3$Ni$_2$O$_{7-\delta}$}\href{https://doi.org/10.1006/jssc.1994.1059}{\it J. Solid State Chem.} {\textbf{108} 402 (1994).}
\bibitem{Luo:arxiv} Luo, Z., Hu, X., Wang, M., Wu, W. \& Yao, D.-X. {Bilayer two-orbital model of La$_3$Ni$_2$O$_7$ under pressure} \href{https://doi.org/10.1103/PhysRevLett.131.126001}{\it Phys. Rev. Lett.} {\textbf{131} 126001 (2023).}
\bibitem{Zhang:arxiv} Zhang, Y., Lin, L.-F., Moreo, A. \& Dagotto, E. {Electronic structure, orbital-selective behavior, and magnetic tendencies in the bilayer nickelate superconductor La$_3$Ni$_2$O$_7$ under pressure} \href{https://doi.org/10.1103/PhysRevB.108.L180510} {\it  Phys. Rev. B} {\textbf{108} L180510 (2023).}
\bibitem{Yang:arxiv} Yang, Q.-G., Wang, D. \& Wang, Q.-H. {Possible $s_{\pm}$-wave superconductivity in La$_3$Ni$_2$O$_7$} \href{https://doi.org/10.1103/PhysRevB.108.L140505}{\it Phys. Rev. B} {\textbf{108} L140505 (2023).}
\bibitem{Sakakibara:arxiv} Sakakibara, H., Kitamine, N., Ochi, M. \& Kuroki, K. {Possible high $T_c$ superconductivity in La$_3$Ni$_2$O$_7$ under high pressure through
manifestation of a nearly-half-filled bilayer Hubbard model} \href{https://doi.org/10.48550/arXiv.2306.06039}{\it arXiv} {2306.06039 (2023).}
\bibitem{Gu:arxiv} Gu, Y., Le, C., Yang, Z., Wu, X. \& Hu, J. {Effective model and pairing tendency in bilayer Ni-based superconductor La$_3$Ni$_2$O$_7$} \href{https://doi.org/10.48550/arXiv.2306.07275}{\it arXiv} {2306.07275 (2023).}
\bibitem{Shen:arxiv} Shen, Y., Qin, M. \& Zhang, G.-M. {Effective bi-layer model Hamiltonian and density-matrix renormalization group study for the high-T$_c$ superconductivity in La$_3$Ni$_2$O$_7$ under high pressure} \href{https://doi.org/10.1088/0256-307X/40/12/127401}{\it Chinese Phys. Lett.} {\textbf{40} 127401 (2023).}
\bibitem{Liu:arxiv} Liu, Y.-B., Mei, J.-W., Ye, F., Chen, W.-Q. \& Yang, F. {The $s^{\pm}$-wave Pairing and the Destructive Role of Apical-Oxygen Deficiencies in La$_3$Ni$_2$O$_7$ Under Pressure} \href{https://doi.org/10.1103/PhysRevLett.131.236002}{\it Phys. Rev. Lett.} {\textbf{131} 236002 (2023).}
\bibitem{Lechermann:arxiv} Lechermann, F., Gondolf, J., B\"otzel, S. \& Eremin, I. M. {Electronic correlations and superconducting instability in La$_3$Ni$_2$O$_7$ under high pressure} \href{https://doi.org/10.1103/PhysRevB.108.L201121}{\it Phys. Rev. B} {\textbf{108}  L201121 (2023).}
\bibitem{Christiansson:arxiv} Christiansson, V., Petocchi F. \& Werner, P. {Correlated electronic structure of La$_3$Ni$_2$O$_7$ under pressure} \href{https://doi.org/10.1103/PhysRevLett.131.206501}{\it Phys. Rev. Lett.} {\textbf{131} 206501 (2023).}
\bibitem{Shilenko:arxiv} Shilenko D. A.\& Leonov, I. V. {Correlated electronic structure, orbital-selective behavior, and magnetic correlations in double-layer La$_3$Ni$_2$O$_7$ under pressure} \href{https://doi.org/10.1103/PhysRevB.108.125105}{\it Phys. Rev. B} {\textbf{108} 125105 (2023).}
\bibitem{Cao:arxiv} Cao, Y. \& Yang, Y.-F. {Flat bands promoted by Hund's rule coupling in the candidate double-layer high-temperature superconductor La$_3$Ni$_2$O$_7$} \href{https://doi.org/10.1103/PhysRevB.109.L081105}{\it Phys. Rev. B} {\textbf{109} L081105 (2024).}
\bibitem{LiuZhe:arxiv} Liu, Z. {\it et al.} {Electronic correlations and energy gap in the bilayer nickelate La$_3$Ni$_2$O$_7$} \href{https://doi.org/10.48550/arXiv.2307.02950}{\it arXiv} {2307.02950 (2023).}
\bibitem{Wu:arxiv} Wu, W., Luo, Z., Yao, D.-X. \& Wang, M. {Charge Transfer and Zhang-Rice Singlet Bands in the Nickelate Superconductor La$_3$Ni$_2$O$_7$ under Pressure} \href{https://doi.org/10.48550/arXiv.2307.05662}{\it arXiv} {2307.05662 (2023).}
\bibitem{Chen:arxiv} Chen, X., Jiang, P., Li, J., Zhong, Z. \& Lu, Y. {Critical charge and spin instabilities in superconducting La$_3$Ni$_2$O$_7$} \href{https://doi.org/10.48550/arXiv.2307.07154}{\it arXiv} {2307.07154 (2023).}
\bibitem{LaBollita:arxiv} H. LaBollita, V. Pardo, M. R. Norman, and A. S. Botana {Electronic structure and magnetic properties of La$_3$Ni$_2$O$_7$ under pressure.} \href{https://doi.org/10.48550/arXiv.2309.17279}{\it arXiv} {2309.17279 (2023).}
\bibitem{Momma:vesta} Momma, K. \& Izumi, F. {Vesta 3 for three-dimensional visualization of crystal, volumetric and morphology data.} \href{https://doi.org/10.1107/S0021889811038970} {\it J. Appl. Crystallogr.}{ \textbf {44}, 1272-1276 (2011)}.
\bibitem{Orobengoa:jac} Orobengoa, D., Capillas, C., Aroyo,  M. I. \&  Perez-Mato, J. M. {AMPLIMODES: symmetry-mode analysis on the Bilbao Crystallographic Server} \href{https://doi.org/10.1107/S0021889809028064}{\it J. Appl. Crystallogr.} {\textbf{42}, 820 (2009).}
\bibitem{Perez-Mato:aca} Perez-Mato, J., Orobengoa, D. \& Aroyo, M. I. {Mode crystallography of distorted structures} \href{https://doi.org/10.1107/S0108767310016247}{\it Acta Crystallogr. A} {\textbf{66}, 558 (2010).}
\bibitem{Baroni:Prl}  Baroni, S., Giannozzi, P. \& Testa, A.  {Green's-function approach to linear response in solids} \href{https://doi.org/10.1103/PhysRevLett.58.1861}{\it Phys. Rev. Lett.} {\textbf{58}, 1861 (1987).}
\bibitem{Gonze:Pra1} Gonze, X. {Perturbation expansion of variational principles at arbitrary order} \href{https://doi.org/10.1103/PhysRevA.52.1086}{Phys. Rev. A} {\textbf{52}, 1086 (1995).}
\bibitem{Gonze:Pra2} Gonze, X. {Adiabatic density-functional perturbation theory} \href{https://doi.org/10.1103/PhysRevA.52.1096}{\it Phys. Rev. A} {\textbf{52}, 1096 (1995).}
\bibitem{Chaput:prb} Chaput, L., Togo, A., Tanaka, I. \& Hug, G. {Phonon-phonon interactions in transition metals} \href{https://doi.org/10.1103/PhysRevB.84.094302}{\it Phys. Rev. B} {\textbf{84}, 094302 (2011).}
\bibitem{Togo:sm} Togo, A. Tanaka, \& I. {First principles phonon calculations in materials science} \href{https://doi.org/10.1016/j.scriptamat.2015.07.021}{\it Scr. Mater.} {\textbf{108}, 1 (2015).}
\bibitem{Yang:arxiv-exp} Yang, J., {\it et al.} {Orbital-Dependent Electron Correlation in Double-Layer Nickelate La$_3$Ni$_2$O$_7$} \href{https://doi.org/10.48550/arXiv.2309.01148}{\it arXiv} {2309.01148 (2023).}
\bibitem{Nakata:prb17} Nakata, M., Ogura, D., Usui, H. \&  Kuroki, K. {Finite-energy spin fluctuations as a pairing glue in systems with coexisting electron and hole bands} \href{https://doi.org/10.1103/PhysRevB.95.214509}{\it Phys. Rev. B} {\textbf{95}, 214509 (2017).}
\bibitem{Maier:prb11} Maier, T. A. \& Scalapino, D. J. {Pair structure and the pairing interaction in a bilayer Hubbard model for unconventional superconductivity} \href{https://doi.org/10.1103/PhysRevB.84.180513}{\it Phys. Rev. B} {\textbf{84}, 180513(R) (2011).}
\bibitem{Maier:prb19} Maier, T. A., Mishra, V., Balduzzi, G. \& Scalapino, {Effective pairing interaction in a system with an incipient band} \href{https://doi.org/10.1103/PhysRevB.99.140504}{\it Phys. Rev. B} {\textbf{99}, 140504(R) (2019).}
\bibitem{Maier:prb22} Dee, P. M., Johnston, S. \& Maier, T. A. {Enhancing $T_c$ in a composite superconductor/metal bilayer system: A dynamical cluster approximation study} \href{https://doi.org/10.1103/PhysRevB.105.214502}{\it Phys. Rev. B} {\textbf{105}, 214502 (2022).}
\bibitem{Mostofi:cpc} Mostofi, A.~A. {\it et al.} {Wannier90: A tool for obtaining maximally-localised wannier functions.} \href {https://doi.org/10.1016/j.cpc.2007.11.016} {\it Comput. Phys. Commun.}{ \textbf {178}, 685-699 (2008)}.
\bibitem{Lin:prl21} Lin, L.-F., Zhang, Y., Alvarez, G., Moreo, A. \& Dagotto, E. {Origin of insulating ferromagnetism in iron oxychalcogenide Ce$_2$O$_2$FeSe$_2$.} \href{https://doi.org/10.1103/PhysRevLett.127.077204} {\it Phys. Rev. Lett.}{ \textbf {127}, 077204 (2021)}.
\bibitem{Liechtenstein:prb} Liechtenstein, A. I., Anisimov, V. I. \& Zaanen, J. {Density-functional theory and strong interactions: Orbital ordering in Mott-Hubbard insulators} \href{https://doi.org/10.1103/PhysRevB.52.R5467}{\it Phys. Rev. B} {\textbf{52}, R5467 (1995).}
\bibitem{nematic1} Fernandes, R. M., {\it et al.} {Unconventional pairing in the iron arsenide superconductors.} \href{https://doi.org/10.1103/PhysRevB.81.140501} {\it Phys. Rev. B}{\textbf {81}, 140501(R) (2010).}
\bibitem{nematic2} Liang, S., Moreo, A., \& Dagotto, E., {Nematic State of Pnictides Stabilized by Interplay between Spin, Orbital, and Lattice Degrees of Freedom.} \href{https://doi.org/10.1103/PhysRevLett.111.047004} {\it Phys. Rev. Lett.}{\textbf {111}, 047004 (2013).}
\bibitem{Zhang:arxiv10} Zhang, Y., Lin, L.-F., Moreo, A., Maier, T. A. \& Dagotto, E. {Electronic structure, magnetic correlations, and superconducting pairing in the reduced Ruddlesden-Popper bilayer La$_3$Ni$_2$O$_6$ under pressure: different role of $d_{3z^-r^2}$ orbital compared with La$_3$Ni$_2$O$_7$} \href{https://doi.org/10.1103/PhysRevB.109.045151} {\it Phys. Rev. B} {\textbf {109}, 045151 (2024).}
\bibitem{Botana:prb16} Botana, A. S., Pardo, V., Pickett, W. E., \& Norman M. R. {Charge ordering in Ni$^{1+}$/Ni$^{2+}$ nickelates: La$_4$Ni$_3$O$_8$ and La$_3$Ni$_2$O$_6$.} \href{https://doi.org/10.1103/PhysRevB.94.081105}{\it Phys. Rev. B} {\textbf{94}, 081105(R) (2016).}
\bibitem{Wang:prm} Wang, L., {\it et al.} {Growth and characterization of HgBa$_2$CaCu$_2$O$_{6+\delta}$ and HgBa$_2$Ca$_2$Cu$_3$O$_{8+\delta}$ crystals.} \href{https://doi.org/10.1103/PhysRevMaterials.2.123401}{\it Phys. Rev. Mater.} {\textbf{2}, 123401 (2018).}
\bibitem{Li:cpl} Q. Li,  {\it et al.} {Signature of Superconductivity in Pressurized La$_3$Ni$_2$O$_{10}$} \href{https://doi.org/10.1088/0256-307X/41/1/017401}{\it Chinese Phys. Lett.} {\textbf{41}, 017401 (2024).}
\bibitem{Kreisel:prl} Kreisel, A., {\it et al.} {Interpretation of Scanning Tunneling Quasiparticle Interference and Impurity States in Cuprates.} \href{https://doi.org/10.1103/PhysRevLett.114.217002}{\it Phys. Rev. Lett.} {\textbf{114}, 217002 (2014).}
\bibitem{Kresse:Prb} Kresse, G., \& Hafner J. {Ab initio molecular dynamics for liquid metals.} \href{https://doi.org/10.1103/PhysRevB.47.558}{\it Phys. Rev. B} {\textbf{47}, 558 (1993).}
\bibitem{Kresse:Prb96} Kresse, G., \& Furthm\"{u}ller, J. {Efficient iterative schemes for ab initio total-energy calculations using a plane-wave basis set.} \href{https://doi.org/10.1103/PhysRevB.54.11169}{\it Phys. Rev. B} {\textbf{54}, 11169 (1996).}
\bibitem{Blochl:Prb} Bl\"{o}chl, P. E. {Projector augmented-wave method.} \href{https://doi.org/10.1103/PhysRevB.50.17953}{Phys. Rev. B} {\textbf{50}, 17953 (1994).}
\bibitem{Perdew:Prl} Perdew, J. P., K. Burke, \& Ernzerhof, M. {Generalized Gradient Approximation Made Simple.} \href{https://doi.org/10.1103/PhysRevLett.77.3865}{\it Phys. Rev. Lett.} {\textbf{77}, 3865 (1996).}
\bibitem{Dudarev:prb} Dudarev, S. L., Botton, G. A., Savrasov, S. Y., Humphreys, C. J. \& Sutton,  A. P. {Electron-energy-loss spectra and the structural stability of nickel oxide: An LSDA+U study} \href{https://doi.org/10.1103/PhysRevB.57.1505}{\it Phys. Rev. B} {\textbf{57}, 1505 (1998).}
\bibitem{Liu:scpma} Liu, Z., {\it et al.} {Evidence for charge and spin density waves in single crystals of La$_3$Ni$_2$O$_7$ and La$_3$Ni$_2$O$_6$.} \href{https://doi.org/10.1007/s11433-022-1962-4}{\it Sci. China Phys. Mech. Astron.} {\textbf{66}, 217411 (2023).}
\bibitem{Romer2020} R{\o}mer, A. T., {\it et al.} {Pairing in the two-dimensional Hubbard model from weak to strong coupling} \href{https://doi.org/10.1103/PhysRevResearch.2.013108}{\it Phys. Rev. Res.} {\textbf{2}, 13108 (2020).}
\bibitem{Kubo2007} Kubo, K., {Pairing symmetry in a two-orbital Hubbard model on a square lattice} \href{https://doi.org/10.1103/PhysRevB.75.224509}{\it Phys. Rev. B} {\textbf{8}, 224509 (2007).}
\bibitem{Graser2009}  Graser, S., Maier, T. A., Hirschfeld, P. J., \&  Scalapino, D. J., {Near-degeneracy of several pairing channels in multiorbital models for the Fe pnictides} \href{https://doi.org/10.1088/1367-2630/11/2/025016}{\it New J. Phys.} {\textbf{11}, 25016 (2009).}
\bibitem{Altmeyer2016} Altmeyer, M., {\it et al.} {Role of vertex corrections in the matrix formulation of the random phase approximation for the multiorbital Hubbard model} \href{https://doi.org/10.1103/PhysRevB.94.214515}{\it Phys. Rev. B} {\textbf{94}, 214515 (2016).}
\end{references}

\noindent {\bf {\\Acknowledgements\\}}
\small {The work was supported by the U.S. Department of Energy (DOE), Office of Science, Basic Energy Sciences (BES), Materials Sciences and Engineering Division.}

\noindent {\bf {\\ Author contributions\\}} \small {Y.Z., L.-F.L., T.A.M., and E.D. designed the project. Y.Z., L.F.L., and T.A.M. carried out numerical calculations for DFT, the TB model, and RPA calculations. Y.Z., A.M., T.A.M., and E.D. wrote the manuscript. All co-authors provided useful comments and discussion on the paper.}

\noindent {\bf {\\Competing interests\\}} \small {The authors declare no competing interest.}

\noindent {\bf {\\ Additional information\\}}
Correspondence should be addressed to Ling-Fang Lin, Thomas A. Maier or Elbio Dagotto.



\newpage
\clearpage


\title{Supplementary Material of ``Structural phase transition, $s_{\pm}$-wave pairing and magnetic stripe order in the bilayered nickelate superconductor La$_3$Ni$_2$O$_7$ under pressure''}


\date{\today}
\maketitle

\newcommand*\mycommand[1]{\texttt{\emph{#1}}}
\renewcommand\thefigure{S\arabic{figure}}
\renewcommand{\thepage}{S\arabic{page}}
\renewcommand{\theequation}{S\arabic{equation}}
\renewcommand{\thetable}{S\arabic{table}}
\setcounter{figure}{0}
\setcounter{page}{1}




\section{Supplementary Note I: PHONON SPECTRUM UNDER PRESSURE}
We calculated the phonon spectrum of the Fmmm and Amam phases of La$_3$Ni$_2$O$_7$ (LNO) for different pressures, by using the density functional perturbation theory approach~\cite{Baroni:Prl,DFPT1,DFPT2} analyzed by the PHONONPY software in the primitive unit cell~\cite{Chaput:prb,Togo:sm}. Below 10.5 GPa, the phonon dispersion spectra clearly display imaginary frequencies at high symmetry points for the Fmmm structure of LNO. This same Fmmm phase becomes stable without any imaginary frequency from 10.6 GPa to 50 GPa, the maximum value we studied, as shown in Supplementary Fig.~\ref{Phonon-Fmmm1} and Supplementary Fig.~\ref{Phonon-Fmmm2}. For the Amam phase of LNO, there is no imaginary frequency obtained in the phonon dispersion spectra, indicating that the Amam phase is stable from 0 to 14 GPa (see Supplementary Fig.~\ref{Phonon-Amam}). To avoid repeating too many displays, we only show the phonon spectrum of several key values of pressures for Fmmm and Amam phases.

% Figure environment removed

% Figure environment removed

% Figure environment removed

\section{Supplementary Note II: HOPPINGS AND FERMI SURFACE UNDER PRESSURE}
Based on the hoppings and crystal-field splitting obtained from the maximally localized Wannier functions~\cite{Mostofi:cpc}, we calculated the Fermi surfaces for different pressures (see Supplementary Fig.~\ref{FS}), by using a bilayer four-band $e_g$-orbital tight binding (TB) model with nearest-neighbor (NN) hopping (see Supplementary Fig.~\ref{bilayer}). The Fermi level is made of two pockets ($\alpha$ and $\beta$) with a mixture of $d_{3z^2-r^2}$ and $d_{x^2-y^2}$ orbitals, while the $\gamma$ pocket is made almost exclusively of the $d_{3z^2-r^2}$ orbital.

% Figure environment removed

% Figure environment removed


% Figure environment removed

For the comparison, the calculated Density-functional theory (DFT) Fermi surfaces of the Fmmm phase under pressure are also shown in Supplementary Fig.~\ref{DFT-FS}.  The main characters of those Fermi surfaces, namely hole pocket $\gamma$ and two electron sheets ($\alpha$ and $\beta$) are qualitatively in agreement with the present DFT and TB calculations.

Below is the list of hopping amplitudes for the Fmmm phase at several pressures, deduced from DFT bands
and the fitting near the Fermi level using a tight-binding two-orbital model.
The values at 0~GPa and 50~GPa can be found in the main text. Varying pressure, the hoppings only weakly
change.

\subsection{Hoppings at 5 GPa}


\begin{equation}
\begin{split}
t_{\vec{x}} =
\begin{bmatrix}
          d_{z^2}   &      d_{x^2-y^2}   \\
         -0.094  	&        0.216	   \\
          0.216     &       -0.468	  	
\end{bmatrix},\\
\end{split}
\label{hopping1}
\end{equation}

\begin{equation}
\begin{split}
t_{\vec{y}} =
\begin{bmatrix}
          d_{z^2}   &      d_{x^2-y^2}   \\
         -0.094  	&       -0.216	   \\
         -0.216	    &       -0.468	  	
\end{bmatrix},\\
\end{split}
\label{hopping1}
\end{equation}


\begin{equation}
\begin{split}
t_{\vec{z}} =
\begin{bmatrix}
          d_{z^2}   &      d_{x^2-y^2}   \\
         -0.611  	&        0.000	   \\
          0.000	    &        0.000	  	
\end{bmatrix}.\\
\end{split}
\label{hopping1}
\end{equation}


\subsection{Hoppings at 10 GPa}

\begin{equation}
\begin{split}
t_{\vec{x}} =
\begin{bmatrix}
          d_{z^2}   &      d_{x^2-y^2}   \\
         -0.097  	&        0.224	   \\
          0.224     &       -0.480	  	
\end{bmatrix},\\
\end{split}
\label{hopping1}
\end{equation}

\begin{equation}
\begin{split}
t_{\vec{y}} =
\begin{bmatrix}
          d_{z^2}   &      d_{x^2-y^2}   \\
         -0.097  	&       -0.224	   \\
         -0.224	    &       -0.480	  	
\end{bmatrix},\\
\end{split}
\label{hopping1}
\end{equation}


\begin{equation}
\begin{split}
t_{\vec{z}} =
\begin{bmatrix}
          d_{z^2}   &      d_{x^2-y^2}   \\
         -0.625  	&        0.000	   \\
          0.000	    &        0.000	  	
\end{bmatrix}.\\
\end{split}
\label{hopping1}
\end{equation}



\subsection{Hoppings at 15 GPa}

\begin{equation}
\begin{split}
t_{\vec{x}} =
\begin{bmatrix}
          d_{z^2}   &      d_{x^2-y^2}   \\
         -0.100  	&        0.230	   \\
          0.230     &       -0.490	  	
\end{bmatrix},\\
\end{split}
\label{hopping1}
\end{equation}

\begin{equation}
\begin{split}
t_{\vec{y}} =
\begin{bmatrix}
          d_{z^2}   &      d_{x^2-y^2}   \\
         -0.100  	&       -0.230	   \\
         -0.230	    &       -0.490	  	
\end{bmatrix},\\
\end{split}
\label{hopping1}
\end{equation}


\begin{equation}
\begin{split}
t_{\vec{z}} =
\begin{bmatrix}
          d_{z^2}   &      d_{x^2-y^2}   \\
         -0.640  	&        0.000	   \\
          0.000	    &        0.000	  	
\end{bmatrix}.\\
\end{split}
\label{hopping1}
\end{equation}


\subsection{Hoppings at 20 GPa}

\begin{equation}
\begin{split}
t_{\vec{x}} =
\begin{bmatrix}
          d_{z^2}   &      d_{x^2-y^2}   \\
         -0.106  	&        0.238	   \\
          0.238     &       -0.505	  	
\end{bmatrix},\\
\end{split}
\label{hopping1}
\end{equation}

\begin{equation}
\begin{split}
t_{\vec{y}} =
\begin{bmatrix}
          d_{z^2}   &      d_{x^2-y^2}   \\
         -0.106  	&       -0.238	   \\
         -0.238	    &       -0.505	  	
\end{bmatrix},\\
\end{split}
\label{hopping1}
\end{equation}


\begin{equation}
\begin{split}
t_{\vec{z}} =
\begin{bmatrix}
          d_{z^2}   &      d_{x^2-y^2}   \\
         -0.654  	&        0.000	   \\
          0.000	    &        0.000	  	
\end{bmatrix}.\\
\end{split}
\label{hopping1}
\end{equation}


\subsection{Hoppings at 25 GPa}

\begin{equation}
\begin{split}
t_{\vec{x}} =
\begin{bmatrix}
          d_{z^2}   &      d_{x^2-y^2}   \\
         -0.110  	&        0.243	   \\
          0.243     &       -0.515	  	
\end{bmatrix},\\
\end{split}
\label{hopping1}
\end{equation}

\begin{equation}
\begin{split}
t_{\vec{y}} =
\begin{bmatrix}
          d_{z^2}   &      d_{x^2-y^2}   \\
         -0.110  	&       -0.243	   \\
         -0.243	    &       -0.515	  	
\end{bmatrix},\\
\end{split}
\label{hopping1}
\end{equation}


\begin{equation}
\begin{split}
t_{\vec{z}} =
\begin{bmatrix}
          d_{z^2}   &      d_{x^2-y^2}   \\
         -0.666  	&        0.000	   \\
          0.000	    &        0.000	  	
\end{bmatrix}.\\
\end{split}
\label{hopping1}
\end{equation}

\subsection{Hoppings at 30 GPa}

\begin{equation}
\begin{split}
t_{\vec{x}} =
\begin{bmatrix}
          d_{z^2}   &      d_{x^2-y^2}   \\
         -0.113  	&        0.250	   \\
          0.250     &       -0.526	  	
\end{bmatrix},\\
\end{split}
\label{hopping1}
\end{equation}

\begin{equation}
\begin{split}
t_{\vec{y}} =
\begin{bmatrix}
          d_{z^2}   &      d_{x^2-y^2}   \\
         -0.113  	&       -0.250	   \\
         -0.250	    &       -0.526	  	
\end{bmatrix},\\
\end{split}
\label{hopping1}
\end{equation}


\begin{equation}
\begin{split}
t_{\vec{z}} =
\begin{bmatrix}
          d_{z^2}   &      d_{x^2-y^2}   \\
         -0.676  	&        0.000	   \\
          0.000	    &        0.000	  	
\end{bmatrix}.\\
\end{split}
\label{hopping1}
\end{equation}

\subsection{Hoppings at 35 GPa}

\begin{equation}
\begin{split}
t_{\vec{x}} =
\begin{bmatrix}
          d_{z^2}   &      d_{x^2-y^2}   \\
         -0.116  	&        0.256	   \\
          0.256     &       -0.534	  	
\end{bmatrix},\\
\end{split}
\label{hopping1}
\end{equation}

\begin{equation}
\begin{split}
t_{\vec{y}} =
\begin{bmatrix}
          d_{z^2}   &      d_{x^2-y^2}   \\
         -0.116  	&       -0.256	   \\
         -0.256	    &       -0.534	  	
\end{bmatrix},\\
\end{split}
\label{hopping1}
\end{equation}


\begin{equation}
\begin{split}
t_{\vec{z}} =
\begin{bmatrix}
          d_{z^2}   &      d_{x^2-y^2}   \\
         -0.686  	&        0.000	   \\
          0.000	    &        0.000	  	
\end{bmatrix}.\\
\end{split}
\label{hopping1}
\end{equation}

\subsection{Hoppings at 40 GPa}

\begin{equation}
\begin{split}
t_{\vec{x}} =
\begin{bmatrix}
          d_{z^2}   &      d_{x^2-y^2}   \\
         -0.118  	&        0.260	   \\
          0.260     &       -0.540	  	
\end{bmatrix},\\
\end{split}
\label{hopping1}
\end{equation}

\begin{equation}
\begin{split}
t_{\vec{y}} =
\begin{bmatrix}
          d_{z^2}   &      d_{x^2-y^2}   \\
         -0.118  	&       -0.260	   \\
         -0.260	    &       -0.540	  	
\end{bmatrix},\\
\end{split}
\label{hopping1}
\end{equation}


\begin{equation}
\begin{split}
t_{\vec{z}} =
\begin{bmatrix}
          d_{z^2}   &      d_{x^2-y^2}   \\
         -0.696  	&        0.000	   \\
          0.000	    &        0.000	  	
\end{bmatrix}.\\
\end{split}
\label{hopping1}
\end{equation}

\subsection{Hoppings at 45 GPa}


\begin{equation}
\begin{split}
t_{\vec{x}} =
\begin{bmatrix}
          d_{z^2}   &      d_{x^2-y^2}   \\
         -0.121  	&        0.263	   \\
          0.263     &       -0.546	  	
\end{bmatrix},\\
\end{split}
\label{hopping1}
\end{equation}

\begin{equation}
\begin{split}
t_{\vec{y}} =
\begin{bmatrix}
          d_{z^2}   &      d_{x^2-y^2}   \\
         -0.121  	&       -0.263	   \\
         -0.263	    &       -0.546	  	
\end{bmatrix},\\
\end{split}
\label{hopping1}
\end{equation}


\begin{equation}
\begin{split}
t_{\vec{z}} =
\begin{bmatrix}
          d_{z^2}   &      d_{x^2-y^2}   \\
         -0.706  	&        0.000	   \\
          0.000	    &        0.000	  	
\end{bmatrix}.\\
\end{split}
\label{hopping1}
\end{equation}

\section{Supplementary Note III: ADDITIONAL DFT MAGNETIC STATES RESULTS}
Here, we considered three possible in-plane spin orders: A-AFM with wavector (0,0), G-AFM with ($\pi$, $\pi$), and stripe with ($\pi$, 0) orders in the plane. The magnetic coupling between layers was considered to be antiferromagnetic (AFM) in all cases. We used the optimized crystal structures at different pressures. Then, the differences in total energy and enthalpy between different magnetic configurations are the same due to having the same crystal structures at different pressures.

As shown in Supplementary Fig.~\ref{J}, by {\it decreasing} the Hund coupling in units of $U$ as compared with values used in the main text, namely by using
$J = 0.4$ eV and $U=4$ eV, the G-type AFM order now becomes the lowest
energy state as pressure increases instead of the stripe magnetic order, due to the enhanced AFM Heisenberg interaction induced by the larger intraorbital hopping under pressure. This highlights that
the stripe order, and associated $s^{\pm}$ pairing, is stable when the Hund coupling is robust.

The calculated magnetic moments of different magnetic phases all decrease when increasing pressure,
as shown in Supplementary Fig.~\ref{M-P}.



% Figure environment removed

% Figure environment removed

In addition, we also showed the band structures of the magnetic stripe phase from $J = 1$ eV to $J = 0.4$ eV at 25 GPa, where $U$ is considered to be 4 eV, as displayed in Supplementary Fig.~\ref{Stripe-band}.

% Figure environment removed

\section{Supplementary Note IV: CONTRIBUTION OF THE SUSCEPTIBILITY}
In Fig. 5 of the main text, we calculated the random phase approximation (RPA) susceptibility. Here, we briefly discuss the
contribution of the intraorbital and interorbital components. For our RPA study, the susceptibility in Fig. 5 of the main text is renormalized and all the contributions are mixed due to the matrix form of the RPA equations. To better understand which contributions are dominant, we look at  0 GPa, where this is the bare, i.e., unrenormalized susceptibility. Below are the results for the orbital contributions at 0 GPa.

% Figure environment removed


As shown in Supplementary Fig.~\ref{chi0}, $\chi_0$ has two peaks, one close to ($\pi$, 0) and another one between ($\pi$, 0) and ($\pi$, $\pi$). Taking into account electronic interactions through the RPA will strongly enhance the ($\pi$, 0) peak
but not the other one. In addition, we also plotted the dominant contributions of the three orbitals (see the blue, red, and green curves). Adding them up gives a result very close to the black curve. Here, those orbitals are indexed as ``0'' for the $d_{x^2-y^2}$ orbital in layer1,``1'' for the $d_{3z^2-r^2}$ orbital in layer1, ``2'' for the $d_{x^2-y^2}$ orbital in layer2, and ``3'' for the $d_{3z^2-r^2}$ orbital in layer2.  Clearly, the scattering between the $d_{3z^2-r^2}$ orbitals dominates. The inter-orbital scattering between the layers (1133) is almost as strong as the intra-orbital $d_{3z^2-r^2}$ scattering within a layer.

In addition, we also show the diagonal and off-diagonal contributions to the RPA spin susceptibility, as displayed in Supplementary Fig.~\ref{Contribution}. Here, the off-diagonal contribution to the peak around ($\pi$, 0) is just as large as the diagonal one due to the $d_{3z^2-r^2}$ scattering between the two layers.

% Figure environment removed

\section{Supplementary Note V: IMPORTANCE OF M POINT POCKET}
To better understand the importance of the $\gamma$ pocket in the pairing process, we calculated the Fermi surface by changing
 $\Delta$ to 0.6 eV, while other parameters remain the same as for 0 GPa.
The hole band sinks below the Fermi level and the $\gamma$-pocket disappears, as shown in Supplementary Fig.~\ref{Delta-FS}. Then, the most notorious aspect to notice is that the pockets at $(\pi,\pi)$, and rotated equivalent points, are absent. This goes together with the suppression of $s^{\pm}$ pairing.

% Figure environment removed



\begin{references}
\bibitem{Baroni:Prl}  Baroni, S., Giannozzi, P. \& Testa, A.  {Green-function approach to linear response in solids}\href{https://doi.org/10.1103/PhysRevLett.58.1861}{\it Phys. Rev. Lett.} {\textbf{58}, 1861 (1987).}
\bibitem{DFPT1} Gonze, X. {Perturbation expansion of variational principles at arbitrary order}\href{https://doi.org/10.1103/PhysRevA.52.1086}{Phys. Rev. A} {\textbf{52}, 1086 (1995).}
\bibitem{DFPT2} Gonze, X. {Adiabatic density-functional perturbation theory}\href{https://doi.org/10.1103/PhysRevA.52.1096}{\it Phys. Rev. A} {\textbf{52}, 1096 (1995).}
\bibitem{Chaput:prb} Chaput, L., Togo, A., Tanaka, I. \& Hug, G. {Phonon-phonon interactions in transition metals}\href{https://doi.org/10.1103/PhysRevB.84.094302}{\it Phys. Rev. B} {\textbf{84}, 094302 (2011).}
\bibitem{Togo:sm} Togo, A. Tanaka, \& I. {First principles phonon calculations in materials science}\href{https://doi.org/10.1016/j.scriptamat.2015.07.021}{\it Scr. Mater.} {\textbf{108}, 1 (2015).}
\bibitem{Mostofi:cpc} Mostofi, A.~A. {\it et al.} {Wannier90: A tool for obtaining maximally-localised wannier functions.} \href {https://doi.org/10.1016/j.cpc.2007.11.016} {\it Comput. Phys. Commun.}{ \textbf {178}, 685-699 (2008)}.
\end{references}


\end{document}

