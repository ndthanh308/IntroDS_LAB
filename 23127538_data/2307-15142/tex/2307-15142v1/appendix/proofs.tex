\newpage
\section{Proofs}\label{sec:proofs}
In this section, we prove each of our results. In \Cref{sec:central-lemma}, we prove a technical lemma that plays a central role in the proofs of each part of \Cref{thm:general} and \Cref{thm:ber-decay}, which we prove in \Cref{sec:proof-thm1} and \Cref{sec:proof-thm2} respectively. In \Cref{sec:proof-rounding-lemma}, we prove another useful lemma---showing that integer solutions are close to real solutions for a class of optimization problems---which will be helpful in the proofs of \Cref{prop:uniform} and \Cref{thm:ber-varying}, which we give in \Cref{sec:proof-of-uniform} and \Cref{sec:proof-thm3} respectively.

\subsection{A central lemma}\label{sec:central-lemma}
Let $\ZZ$ denote the set of non-negative integers and $A_n\subset \ZZ^m$ denote the set of $m$-tuples whose elements sum to $n$. We will say that a function $h: \ZZ\rightarrow \RR$ is \vocab{strictly concave} if $h(a+1)-h(a) < h(a)-h(a-1)$ for all $a$.

\begin{lemma}\label{lem:fennel}
Consider an integer $m$ and $p_1,p_2,\cdots,p_m\ge 0$. Let $h:\ZZ \rightarrow \RR$ be monotonically increasing. For each positive integer $n$, choose $(a_1^{(n)},\cdots,a_m^{(n)})$ such that
 \begin{equation}
        (a_1^{(n)},\cdots,a_m^{(n)}) \in \argmax_{(a_1,\cdots,a_m)\in A_n} \sum_{t=1}^m p_t h(a_t),
    \end{equation}
and define 
\begin{equation}
    r_t^{(n)}:= \frac{a_t^{(n)}}{n}.
\end{equation}
Then the following statements hold.
\begin{enumerate}
        \item[(i)] Suppose there exist constants $A,B>0$ and $\sigma<0$ such that
        \begin{equation}\lim_{a\rightarrow \infty} \frac{\log(A - h(a))}{Ba^\sigma} = 1.\end{equation}
        Then
        \begin{equation}
            \lim_{n\rightarrow \infty} r_t^{(n)} = \frac{1}{m}.
        \end{equation}
        \item[(ii)] Suppose there exist constants $A,B>0$ and $\sigma<0$ such that
        \begin{equation}
            \lim_{a\rightarrow \infty} \frac{A-h(a)}{Ba^{\sigma}} = 1.
        \end{equation}
        Then
        \begin{equation}
            \lim_{n\rightarrow \infty} r_t^{(n)} = \frac{p_t^{\frac{1}{1-\sigma}}}{\sum_{i=1}^m p_i^{\frac{1}{1-\sigma}}}.
        \end{equation}
        \item[(iii)] Suppose $h$ is strictly concave, and that there exist constants $B,C>0$ such that
        \begin{equation}
        \lim_{a\rightarrow\infty} h(a) - B\log a - C = 0.
        \end{equation}
        Then
        \begin{equation}
            \lim_{n\rightarrow \infty} r_t^{(n)} = p_t.
        \end{equation}
        \item[(iv)] Suppose $h$ is strictly concave, and that there exist constants $B>0$ and $0 < \sigma < 1$ such that
        \begin{equation}
            \lim_{a\rightarrow \infty} \frac{h(a)}{Ba^{\sigma}} = 1.
        \end{equation}
        Then
        \begin{equation}
            \lim_{n\rightarrow \infty} r_t^{(n)} = \frac{p_t^{\frac{1}{1-\sigma}}}{\sum_{i=1}^m p_i^{\frac{1}{1-\sigma}}}.
        \end{equation}
    \end{enumerate}
\end{lemma}

We spend the remainder of the section proving \Cref{lem:fennel}. A useful first step is to show that in each of parts (i)-(iv), we have that
\begin{equation}\label{eq:lim-a}
    \lim_{n\rightarrow \infty} a_t^{(n)} = \infty
\end{equation}
for each $t\in [m],$ allowing us to use the asymptotic assumptions in the lemma's statement.

Assume for the sake of contradiction that there exists $t\in [m]$ and an integer $d$ such that for any integer $N$ there exists $n>N$ for which $a_t^{(n)} < d.$ 
Since $h$ is strictly increasing and $d$ is finite, there exists $\delta>0$ such that 
\begin{equation}
    h(a+1)-h(a) > \delta
\end{equation} for all $a < d.$ Also, there exists an integer $N'$ such that for all $a>N'$, 
\begin{equation}\label{eq:sand}
    h(a)-h(a-1) < \delta\cdot \min_{i\in [m]} \frac{p_t}{p_i}. 
\end{equation}
\eqref{eq:sand} holds in parts (i)-(ii) because $h$ is monotonically increasing and is upper bounded by $A$, and in parts (iii)-(iv) because $h$ is strictly concave.

Now consider $N = N'm$. Then there exists $n > N$ such that $a_t^{(n)} < d.$ Since $\sum_{t=1}^m a_t^{(n)} = n > N'm$, there exists $t'\in [m]$ such that $a_{t'}^{(n)} > N'.$ Thus,
\begin{equation}
    p_{t'} h(a_{t'}^{(n)}) - p_{t'} h(a_{t'}^{(n)} - 1) < p_t\delta < p_t h(a_t^{(n)} + 1) - p_t h(a_t^{(n)}),
\end{equation}
which implies that the switch $a_t^{(n)} \rightarrow a_t^{(n)} + 1, a_{t'}^{(n)}\rightarrow a_{t'}^{(n)} - 1$ increases
\begin{equation}
    \sum_{t=1}^m p_t h(a_t^{(n)}),
\end{equation}
contradicting the optimality of $(a_1^{(n)},\cdots, a_m^{(n)}).$


With \eqref{eq:lim-a} in hand, we turn to the bulk of the proof. In each part, we would like to show that
\begin{equation}
\lim_{n\rightarrow \infty} (r_1^{(n)},\cdots,r_m^{(n)}) = (\wh{r}_1, \cdots, \wh{r}_m)    
\end{equation} 
for some specified $(\wh{r}_1, \cdots, \wh{r}_m)$ depending on the part. We assume for the sake of contradiction that $\{(r_1^{(n)},\cdots,r_m^{(n)})\}_{n=1}^\infty$ does not converge to $(\wh{r}_1, \cdots, \wh{r}_m).$ If this is the case, then by the Bolzano-Weierstrass theorem, since $[0,1]^m$ is compact, there is a subsequence
\begin{equation}\{(r_1^{(s_i)},\cdots,r_m^{(s_i)})\}_{i=1}^\infty
\end{equation}
such that
$\lim_{i\rightarrow \infty} (r_1^{(s_i)},\cdots,r_m^{(s_i)}) = (r_1,\cdots,r_m)$
for some $(r_1,\cdots,r_m) \neq (\wh{r}_1, \cdots, \wh{r}_m).$ For notational ease, we will simply assume that
\begin{equation}
\lim_{n\rightarrow \infty} (r_1^{(n)},\cdots,r_m^{(n)}) = (r_1,\cdots,r_m)
\end{equation}
for some $(r_1,\cdots,r_m) \neq (\wh{r}_1, \cdots, \wh{r}_m)$. The proof holds analogously when the subsequence $\{(r_1^{(s_i)},\cdots,r_m^{(s_i)})\}_{i=1}^\infty$ differs from $\{(r_1^{(n)},\cdots,r_m^{(n)})\}_{n=1}^\infty$.

Then consider any sequence $\{(\wh{a}_1^{(n)},\cdots, \wh{a}_m^{(n)})\}_{n=1}^\infty$ such that
\begin{equation}
    \lim_{n\rightarrow \infty} \left(\frac{\wh{a}_1^{(n)}}{n},\cdots, \frac{\wh{a}_m^{(n)}}{n}\right) = (\wh{r}_1, \cdots, \wh{r}_m).
\end{equation}
(Clearly, such a sequence exists.) In each part, we will show that for sufficiently large $n$,
\begin{equation}
    \sum_{t=1}^m p_t h(a_t^{(n)}) < \sum_{t=1}^m p_t h(\wh{a}_t^{(n)}),
\end{equation}
contradicting the optimality of $(a_1^{(n)},\cdots, a_m^{(n)}).$ To complete the proof, we analyze each part separately:

\begin{enumerate}
    \item[(i)] In this part, there exist constants $A,B>0$ and $\sigma<0$ such that
        \begin{equation}\lim_{a\rightarrow \infty} \frac{\log(A - h(a))}{Ba^\sigma} = 1.\end{equation}
        We set $\wh{r}_t := \frac{1}{m}$ for each $t\in [m].$
    Observe that
    \begin{align}
        \lim_{a\rightarrow \infty} \frac{\log (A - h(a))}{Ba^\sigma} = 1
    \end{align}
    implies that for all $\epsilon > 0$, there exists $c$ such that for all $a>c,$
    \begin{equation}
        e^{(1-\epsilon)Ba^\sigma} \le A-h(a)\le e^{(1+\epsilon)Ba^\sigma}.
    \end{equation}
    Then observe that by taking sufficiently small $\epsilon,$ we have
\begin{align}
\lim_{n\rightarrow \infty} \frac{\sum_{t=1}^m p_t (A - h(a_t^{(n)}))}{\sum_{t=1}^m p_t (A - h(\wh{a}_t^{(n)}))} &\ge \lim_{n\rightarrow \infty} \frac{\sum_{t=1}^m p_t \exp[(1-\epsilon)B(a_t^{(n)})^\sigma)]}{\sum_{t=1}^m p_t \exp[(1+\epsilon)B(\wh{a}_t^{(n)})^\sigma)]}\\
&= \lim_{n\rightarrow \infty} \frac{\sum_{t=1}^m p_t \exp[(1-\epsilon)B(nr_t)^\sigma]}{\exp[(1+\epsilon)B(n/m)^\sigma]}\\
&= \lim_{n\rightarrow \infty} \sum_{t=1}^m p_t \exp[Bn^{\sigma}((1-\epsilon)r_t^\sigma - (1+\epsilon)(1/m)^\sigma)] = \infty
\end{align}
where the last limit holds for $\epsilon$ sufficiently small because $r_t - \frac{1}{m} > 0$ for some $t$.

It follows that for $n$ sufficiently large, $\sum_{t=1}^m p_t h(a_t^{(n)}) < \sum_{t=1}^m p_t h(\wh{a}_t^{(n)}),$ as desired.
    
    \item[(ii)] In this part, there exist constants $A,B>0$ and $\sigma<0$ such that
        \begin{equation}
            \lim_{a\rightarrow \infty} \frac{A-h(a)}{Ba^{\sigma}} = 1.
        \end{equation}
        We set
    \begin{equation}
       \wh{r}_t := \frac{p_t^{\frac{1}{1-\sigma}}}{\sum_{i=1}^m p_i^{\frac{1}{1-\sigma}}} 
    \end{equation}
    for each $t\in [m].$
    Then observe that
    \begin{align}
&\lim_{n\rightarrow \infty} \frac{\sum_{t=1}^m p_t  (A - h(a_t^{(n)}))}{\sum_{t=1}^m p_t  (A - h(\wh{a}_t^{(n)}))}\\
= &\lim_{n\rightarrow \infty} \frac{\sum_{t=1}^m p_t  (A - h(a_t^{(n)}))}{\sum_{t=1}^m p_t  (A - h(\wh{a}_t^{(n)}))} \cdot \lim_{n\rightarrow \infty} \frac{\sum_{t=1}^m p_t B(a_t^{(n)})^\sigma}{\sum_{t=1}^m p_t  (A - h(a_t^{(n)}))} \cdot \lim_{n\rightarrow \infty} \frac{\sum_{t=1}^m p_t  (A - h(\wh{a}_t^{(n)}))}{\sum_{t=1}^m p_t B(\wh{a}_t^{(n)})^\sigma} \label{eq:mug-2}\\
= &\lim_{n\rightarrow \infty} \frac{\sum_{t=1}^m p_t  (A - h(a_t^{(n)}))}{\sum_{t=1}^m p_t  (A - h(\wh{a}_t^{(n)}))} \cdot \frac{\sum_{t=1}^m p_t B(a_t^{(n)})^\sigma}{\sum_{t=1}^m p_t  (A - h(a_t^{(n)}))} \cdot \frac{\sum_{t=1}^m p_t  (A - h(\wh{a}_t^{(n)}))}{\sum_{t=1}^m p_t B(\wh{a}_t^{(n)})^\sigma} \label{eq:mug-3}\\
= &\lim_{n\rightarrow \infty} \frac{\sum_{t=1}^m p_t B(a_t^{(n)})^\sigma}{\sum_{t=1}^m p_t B(\wh{a}_t^{(n)})^\sigma} \\
= & \frac{\sum_{t=1}^m p_tr_t^\sigma}{\sum_{t=1}^m p_t\wh{r}_t^\sigma} > 1,\label{eq:mug}
\end{align}
where \eqref{eq:mug-2} follows from the latter two limits being equal to 1, \eqref{eq:mug-3} follows from the product rule for limits, and \eqref{eq:mug} follows from the observation that for $\sigma < 0$
\begin{equation}
    \sum_{t=1}^m p_t x_t^\sigma,
\end{equation}
subject to the constraint $\sum_{t=1}^m x_t = 1$ for $x_t\ge 0$ has a unique minimum at $(x_1,\cdots,x_m)=(\wh{r}_1,\cdots,\wh{r}_m).$ This is direct, for example, by using Lagrange multipliers. \eqref{eq:mug} implies that
\begin{equation}
    \lim_{n\rightarrow \infty} \frac{\sum_{t=1}^m p_t  h(a_t^{(n)})}{\sum_{t=1}^m p_t  h(\wh{a}_t^{(n)})} < 1.
\end{equation}

It follows that for $n$ sufficiently large, $\sum_{t=1}^m p_t h(a_t^{(n)}) < \sum_{t=1}^m p_t h(\wh{a}_t^{(n)}),$ as desired.

    \item[(iii)] In this part, $h$ is strictly concave, and there exist constants $B,C>0$ such that
        \begin{equation}
        \lim_{a\rightarrow\infty} h(a) - B\log a - C = 0.
        \end{equation}
        We set $\wh{r}_t := p_t$ for each $t\in [m].$
    Then observe that
\begin{align}
\lim_{n\rightarrow \infty} \sum_{t=1}^m p_t h(a_t^{(n)}) - \sum_{t=1}^m p_t h(\wh{a}_t^{(n)}) &= \lim_{n\rightarrow \infty} \sum_{t=1}^m p_t B\log a_t^{(n)}-\sum_{t=1}^m p_t B\log \wh{a}_t^{(n)}\\
&= B\log n + B\sum_{t=1}^m p_t \log r_t - B\log n - B\sum_{t=1}^m p_t \log \wh{r}_t\\
&< 0.
\end{align}
The final inequality here follows from the observation that
\begin{equation}
    \sum_{t=1}^m p_t \log x_t,
\end{equation}
subject to the constraint $\sum_{t=1}^m x_t = 1$ for $x_t>0$ has a unique minimum at $(x_1,\cdots,x_m)=(\wh{r}_1,\cdots,\wh{r}_m).$ This is direct, for example, by using Lagrange multipliers.

It follows that for $n$ sufficiently large, $\sum_{t=1}^m p_t h(a_t^{(n)}) < \sum_{t=1}^m p_t h(\wh{a}_t^{(n)}),$ as desired.
    
    \item[(iv)] In this part, $h$ is strictly concave, and there exist constants $B>0$ and $0 < \sigma < 1$ such that
        \begin{equation}
            \lim_{a\rightarrow \infty} \frac{h(a)}{Ba^{\sigma}} = 1.
        \end{equation}
        We set
    \begin{equation}
       \wh{r}_t := \frac{p_t^{\frac{1}{1-\sigma}}}{\sum_{i=1}^m p_i^{\frac{1}{1-\sigma}}} 
    \end{equation}
    for each $t\in [m].$
    Then observe that
\begin{align}
\lim_{n\rightarrow \infty} \frac{\sum_{t=1}^m p_t h(a_t^{(n)})}{\sum_{t=1}^m p_t h(\wh{a}_t^{(n)})} = \lim_{n\rightarrow \infty} \frac{\sum_{t=1}^m p_t B(a_t^{(n)})^\sigma}{\sum_{t=1}^m p_t B(\wh{a}_t^{(n)})^\sigma} = \frac{\sum_{t=1}^m p_tr_t^\sigma}{\sum_{t=1}^m p_t\wh{r}_t^\sigma} < 1.
\end{align}
The first equality is a consequence of the asymptotic assumption on $h$ and the product rule for limits (as in part (ii). The final inequality here follows from the observation that for $\sigma > 0$,
\begin{equation}
    \sum_{t=1}^m p_t x_t^\sigma,
\end{equation}
subject to the constraint $\sum_{t=1}^m x_t = 1$ for $x_t>0$ has a unique maximum at $(x_1,\cdots,x_m)=(\wh{r}_1,\cdots,\wh{r}_m).$ This is direct, for example, by using Lagrange multipliers.

It follows that for $n$ sufficiently large, $\sum_{t=1}^m p_t h(a_t^{(n)}) < \sum_{t=1}^m p_t h(\wh{a}_t^{(n)}),$ as desired.
    
\end{enumerate}


