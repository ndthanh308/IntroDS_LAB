\section{Additional related work}\label{sec:additional-related-work}

Our work sits at the intersection of two broad sets of work. On the one hand are arguments that diversity is key to achieving efficiency. On the other are those that cast diversity as in conflict with efficiency or accuracy, but perhaps that diversity should nevertheless be pursued as an axiomatic good. 

Broadly, our work seeks to understand this tension by sharply characterizing the \textit{amount} of diversity in efficient solutions, as a function of key setting characteristics: user utilities and consumption constraints, and uncertainty in the item quality distribution. In particular, our results characterize \textit{in what settings} the intuition regarding diversity being efficient holds, and in what settings they may be in conflict.

\paragraph{The (efficiency) benefits of diversity} The importance of diversity for efficiency is an old idea present across many fields; \citet{page2008difference} synthesizes the conceptual and empirical arguments in support of this principle. \citet{hong2004groups} develop a model in which a randomly selected team of problem solvers outperforms a team of the individually best-performing agents, due to diversity in problem solving perspective (\citet{kleinberg2018team} show that, in some settings, there exist \textit{tests} under which selecting the best-performing agents again becomes optimal). \citet{kleinberg2018selection} show that constraints promoting diversity can improve efficiency when they work to counteract a decision-maker's biases. \citet{10.1145/1498759.1498766} develop an algorithm to diversify search results, to minimize the risk of user dissatisfaction. We are particularly influenced by the work of \citet{steck2018calibrated}, who presents the intuition that recommendations should be \textit{calibrated}: ``When a user has watched, say, 70 romance movies and 30 action movies, then it is reasonable to expect the personalized list of recommended movies to be comprised of about 70\% romance and 30\% action movies as well.'' \citet{guo2021stereotyping} show that collaborative filtering-based recommendations may not be able to effectively show users such a diverse set of content, harming efficiency.

More broadly, researchers studying various combinatorial optimization problems may find it obvious that homogeneous solutions can be sub-optimal; indeed, in classical problems like \textit{maximum coverage}, redundancy is undesirable. 

Our work particularly is intimately connected to the large literature on assortment optimization \cite{kok2007demand,rusmevichientong2010dynamic,Davis2014AssortmentOU,jagabathula2014assortment,rusmevichientong2014assortment,gallego2014constrained,bertsimas2015data,el2021joint,chen2022fair,el2022joint}. That literature also considers consumption-constrained consumer item selections based on an intermediary's recommendations (e.g., that customers picks one item  according to a multinomial choice model). The literature primarily devises \textit{approximation algorithms} to find the optimal recommendation (``assortment'') as a function of the consumer's choice model, platform objective, and the item distribution. In other words, an implicit premise of this literature is that the naive approach of presenting the items with highest individual expected values is sub-optimal, i.e., that optimal assortments are not completely `homogeneous.' On the other hand, optimal assortments are not necessarily diverse; roughly speaking, the results of \citet{el2022joint} imply that a standard assortment approach (Mixed MNL) might produce solutions that are not ``diverse'' enough to satisfy multiple customer types, and so there is benefit to personalize to each type.\footnote{We thank the authors for highlighting this connection to us.}\footnote{Furthermore, as \citet{chen2022fair} recently characterize, standard assortment optimization approaches may be ``unfair'' to items in other ways.} Our work contributes to this literature by (a) examining the implicit premise that optimal assortments are not homogeneous (i.e., when is the naive\footnote{Note that \textit{naive}
is much simpler than the \textit{greedy} approach studied in the literature, which picks items iteratively potentially as a function of previous items picked.} approach sufficient?); and (b) showing the characteristics under which optimal assortments are not diverse. 

\paragraph{Diversity and fairness as a contrast to efficiency and accuracy} On the other hand, many works start with the premise that---although diversity may conflict with efficiency or accuracy---it is an axiomatic good that should be pursued. For example, diversity is often considered to be inherently desirable from a fairness perspective and user satisfaction perspective. As a result, there is a wide body of work devoted to optimizing for various metrics of diversity. A common approach (taken, for example, in \citet{carbonell1998use} and \citet{gimpel2013systematic}) is to consider an objective function that balances a weighted measure of ``accuracy'' or ``relevance'' with a measure of diversity. More recently, \citet{brown2022diversified} consider set recommendation for an agent with adaptive preferences, to ensure that consumption over time is diverse. Numerous metrics for diversity have been proposed---we refer the reader to \citet{kunaver2017diversity} for a survey. Similarly, the fair ranking and recommendation literature (see \citet{patro2022fair} and \citet{zehlike2021fairness} for recent surveys) considers metrics and methods for fairness in such problems. On the other hand, empirical work has demonstrated that such tradeoffs may be small in practice \cite{rodolfa2021empirical}. Such formulations imply that there is a tension between diversity and measures of accuracy.