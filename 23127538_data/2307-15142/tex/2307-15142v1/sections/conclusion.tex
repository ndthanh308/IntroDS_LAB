\section{Conclusion}
We introduce and analyze a stylized model that reconciles the apparent accuracy-diversity trade-off in recommendations. We characterize the diversity of optimal sets both when ``optimality'' captures and does not capture user consumption constraints. Broadly speaking, we show that the former naturally induces diversity while the latter results in homogeneity. Therefore, the apparent accuracy-diversity trade-off is partially due to traditional accuracy metrics not accounting for consumption constraints.

\paragraph{Limitations and future work.} A particular strength of our model is that we were able to derive precise and interpretable characterizations of diversity in many settings. One limitation of our work is that many of our results are asymptotic (i.e., $n\rightarrow \infty$). We expect that it is possible to obtain further results in our model for finite $n$. We gave \Cref{prop:uniform} as one such example, and outline additional possible directions in \Cref{sec:simulations}.

The purposefully stylized nature of our model---in which users prefer only one type of item and items fall under only one type---allows for a particularly interpretable evaluation of diversity where we compare the representation of each type in comparison to the likelihood that the user prefers that type. This simple model is expressive enough to admit a wide range of results. Still, it would be interesting to generalize our findings in more complex models of users and items. In \Cref{sec:generalizations}, we discuss these possibilities further, providing some basic results both when users can prefer multiple types of items and when users and items are represented by embeddings.

Finally, there are possible model characteristics beyond consumption constraints that implicitly reward diversity. As discussed in \Cref{sec:intuition}, diversity in our model is a consequence of diminishing returns when recommending additional items of a type. Diminishing returns may arise from other assumptions, such as decaying user attention (e.g., \cite{Kleinberg2023CalibratedRF}). We point those interested in considering these alternate assumptions to \Cref{lem:fennel} in the appendix, in which we abstract the technical relationship between diminishing returns and diversity.

\paragraph{Broader impacts.} Recommender systems have broad societal consequences, particularly in high-stakes settings like employment. Our work does not model many additional reasons for diversity in these settings, including those based on fairness and equity. We emphasize again that our results are intended to convey \textit{minimal} assumptions that induce diversity and illustrate the tendency of certain objectives to produce homogeneous recommendations. Designers of recommender systems, and researchers in the area, should take a broader view when making decisions regarding diversity.