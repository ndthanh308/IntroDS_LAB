\subsection{Heterogeneous Bernoulli conditional item values}\label{sec:bernoulli}
Now consider when $X_i^{(t)}$ are independent random variables drawn from $\Ber(q_i^{(t)})$, reflecting a model in which items have binary values (i.e., a user is either satisfied or not satisfied by an item). In this section, we allow $q_i^{(t)}$ to differ across $i$ and $t$, implying that the recommender has knowledge about which items are more likely to be successful conditional on a user's preferred type. Specifically, in \Cref{sec:ber-decay} we allow $q_i^{(t)}$ to vary across $i$ and in \Cref{sec:ber-varying} we allow $q_i^{(t)}$ to vary across $t$.

Our results will focus on $S_{n,1}$ and $S_{n,n}$, which both have natural interpretations in this setting:
\begin{itemize}[leftmargin=*]
    \item $S_{n,1}$ maximizes the probability that the user will be satisfied by at least one recommended item.
    \item $S_{n,n}$ maximizes the the expected number of recommended items the user will be satisfied by, which is equivalent to the standard metric of accuracy.
\end{itemize}
Before proceeding, we note that the basic case $q_i^{(t)} = q$ for all $i,t$ is handled as a direct corollary of \Cref{thm:general}(i).

\begin{corollary}[Conditional item values are i.i.d. Bernoulli]\label{cor:ber}
When $X_i^{(t)}\simiid \Ber(q)$ for $q>0,$ then $S_{n,1}$ is $0$-homogeneous.
\end{corollary}

Therefore, if the success probability is the same for all items, optimal solutions are $0$-homogeneous (each item is equally represented) for large $n$, even as the likelihoods $p_t$ vary across type.

\subsubsection{Decaying success probabilities}\label{sec:ber-decay} We now consider a setting in which among items of the same type, the recommender knows that some items have higher success probability. This maps onto settings where the recommender knows which items of a type are most likely to be satisfactory, e.g., some action movies are more commonly liked than are others. Thus, we assume that the recommender has access to items with decaying success probabilities.

\begin{theorem}[Decaying success probabilities]\label{thm:ber-decay}
Suppose that $X_i^{(t)}\simiid \Ber(q_i^{(t)})$ are i.i.d. Bernoulli random variables such that $q_i^{(t)}= c(i+d)^{-\alpha}$ for all $i\ge 1$ and some $\alpha,c,d\ge 0.$ Then the following statements hold.
\begin{enumerate}
    \item[(i)] $\{S_{n,1}\}_{n=1}^\infty$ is $0$-homogeneous for $\alpha < 1.$
    \item[(ii)] $\{S_{n,1}\}_{n=1}^\infty$ is $\frac{1}{1+c}$-homogeneous for $\alpha = 1.$
    \item[(iii)] $\{S_{n,1}\}_{n=1}^\infty$ is $\frac{1}{\alpha}$-homogeneous for $\alpha > 1.$
\end{enumerate}
Additionally,
\begin{enumerate}
    \item[(iv)] $\{S_{n,n}\}_{n=1}^\infty$ is $\frac{1}{\alpha}$-homogeneous for $\alpha\ge 0.$
\end{enumerate}
\end{theorem}

\input{figures/decay.tex}

When the success probabilities of items have moderate decay ($\alpha<1$), then $0$-homogeneity is maintained in the case $k=1$ (note that $\alpha = 0$ recovers \Cref{cor:ber}). Moreover, for all rates of decay, optimal recommendations reflect at least proportional diversity for large $n$ and $k=1$.

\Cref{thm:ber-decay} also reveals surprising \textit{non-monotonic} behavior. In particular, there is a discontinuity at $\alpha=1$, where homogeneity suddenly increases, but then decreases as $\alpha$ continues to increase. At $\alpha=1,$ the optimal amount of diversity when $\alpha=1$ can range between $0$ and $1$ depending on $c$.

When $k=n$, a larger rate of decay induces more diverse recommendations. Intuitively, when there is a larger rate of decay, the recommender has fewer high-quality options of a given type and is more incentivized to recommend high-quality options of other types. Note that for moderate rates of decay ($\alpha < 1$), $S_{n,n}$ remains less than proportionally diverse for large $n$, unlike $S_{n,1}$.

\subsubsection{Varying success probability across types}\label{sec:ber-varying} We now consider a setting in which the success probability of an item varies across types. This can arise when a users are more picky for some types of items, or when the recommender has more information about items from one type than another.%

\begin{theorem}[Varying success probability across types]\label{thm:ber-varying}
Suppose that for each fixed $t$, $X_i^{(t)} \simiid \Ber(q_t)$ are i.i.d. Bernoulli random variables. Then
\begin{equation}
\lim_{n\rightarrow\infty}r_t(S_{n,1}) \propto \frac{1}{\log \frac{1}{1-q_t}}
\end{equation}
while $S_{n,n}$ contains only items of type $t = \argmax_{t\in [m]} p_tq_t.$
\end{theorem}

The surprising high-level takeaway from \Cref{thm:ber-varying} is that, for large $n$, a \textit{smaller} success probability $q_t$ results in \textit{more} representation of type $t$. The less likely an item of a given type is satisfactory, the more that type is recommended. Moreover, note that the amount of representation in this setting is independent of the popularities $p_1,\cdots,p_m.$

This paradox is illustrated in grocery stores, where more space is devoted to ice cream than milk, despite milk being much more popular than ice cream. Here, $p_1$ (the popularity of milk) is higher than $p_2$ (the popularity of ice cream). However, $q_1$ (the likelihood a given milk product satisfies a shopper looking for milk) is also higher than $q_2$ (the likelihood a given ice cream product satisfies a shopper looking for ice cream), since people tend to have more specific tastes for ice cream. Thus, since $q_2$ is smaller than $q_1$ and the grocery store should ``recommend'' many more ice creams than milks, explaining why more space is devoted to ice cream. Intuitively, while more shoppers want milk, these consumers can be satisfied with a small selection of milk; thus, it is more beneficial to devote more space to ice cream, for which shoppers have more specific tastes.