\begin{abstract}
    In recommendation settings, there is an apparent trade-off between the goals of \textit{accuracy} (to recommend items a user is most likely to want) and \textit{diversity} (to recommend items representing a range of categories). As such, real-world recommender systems often explicitly incorporate diversity separately from accuracy. This approach, however, leaves a basic question unanswered: Why is there a trade-off in the first place?

    We show how the trade-off can be explained via a user's \textit{consumption constraints}---users typically only consume a few of the items they are recommended. In a stylized model we introduce, objectives that account for this constraint induce diverse recommendations, while objectives that do not account for this constraint induce homogeneous recommendations. This suggests that accuracy and diversity appear misaligned because standard accuracy metrics do not consider consumption constraints. Our model yields precise and interpretable characterizations of diversity in different settings, giving practical insights into the design of diverse recommendations. 
\end{abstract}

\section{Introduction}
Current quantum hardware is unable to carry out universal quantum computations due to the buildup of errors that occur during the computation. 
The magnitude of the individual error is currently above the value that the Threshold Theorem requires in order to kick-start quantum error correction and fault-tolerant quantum computation~\cite[Section 10.6]{nielsen_chuang_2010}. 
Although the experimentally achieved fidelity rates are promising and the error bounds are inching closer to the required threshold, we will have to work for the foreseeable future with quantum hardware with errors that build-up during the computation.  This implies that we can only do a limited number of steps before the output of the computation has become completely uncorrelated with the intended one.

For fault-tolerant quantum computing, we repeat four steps: 
1) We apply a number of single and two-qubit quantum gates, in parallel whenever possible; 
2) We perform a syndrome measurement on a subset of the qubits; 
3) We perform fast classical computations to determine which errors have occurred and how to correct them; 
and, 4) We apply correction terms based on the classical computations.
We then repeat these four steps with a next sequence of gates. 
These four steps are essential to fault-tolerant quantum computing. 


The starting point of this work is to use the four steps outlined above, not to carry out error correction and fault-tolerant computation, but to enhance short, constant-depth, {\em uncorrected} quantum circuits that perform single qubit gates and {\em nearest-neighbor} two qubit gates. 
Since in the long run we will have to implement error-correction and fault-tolerant computation anyhow, and this is done by such a four-step process, why not make other use of this architecture? Moreover, on some of the quantum hardware platforms, these operations are already in place.
Embracing this idea we naturally arrive at the question: what is the computational power of \textit{low-depth} quantum-classical circuits organized as in the four steps outlined above? 
We thus investigate circuits that execute a small, ideally constant, number of stages, where at each stage we may apply, in parallel, single qubit gates and {\em nearest-neighbor} two qubit gates, followed by measurements, followed by low-depth classical computations of which the outcome can control quantum gates in later stages. 
It is not clear, at first, whether such circuits, especially with constant depth, can do anything remotely useful. 
But we will see that this is indeed the case: many quantum computations can be done by such circuits in constant depth. 
By parallelizing quantum computations in this way, we improve the overall computational capabilities of these circuits, as we do not incur errors on qubits that are idle, simply because qubits are not idle for a very long time. 
Furthermore, reducing the depth of quantum circuits, at the cost of increasing width, allows the circuit to be run faster even if errors occur.

The first usage of such a four-step layout, not to do error correction, but to perform computations, can be found in the paradigm of measurement-based quantum computing~\cite{gottesman1999demonstrating,raussendorf2001one,jozsa2006introduction,clark2007generalised}: 
A universal form of quantum computing where a quantum state is prepared and operations are performed by measuring qubits in different bases, depending on previous measurements and intermediate measurements.

\citeauthor{PhamSvore2013} were the first to formalize the four-step protocol for performing computations~\cite{PhamSvore2013}. They included specific hardware topologies by considering two-dimensional graphs for imposing constraints on qubit interactions. In their model, they develop circuits for particularly useful multi-qubit gates, including specifying costs in the width, number of qubits, depth, number of concurrent time steps, size, and total number of non-Identity operations.
As a result, they find an algorithm that factors integers in polylogarithmic depth.
\citeauthor{Browne:2011} showed that the main tool in the work by \citeauthor{PhamSvore2013}, the fan-out gate, can also be replaced by additional log-depth classical computations in the measurement-based quantum computing setting~\cite{Browne:2011}.

More recently, \citeauthor{Cirac:2021} introduced a scheme to implement unitary operations involving quantum circuits combined with Local Operations and Classical Communication ($\mathsf{LOCC}$) channels: $\mathsf{LOCC}$-assisted quantum circuits~\cite{Cirac:2021}. Similarly to the four-step scheme we just described, they allow for a short depth circuit to be run on the qubits, followed by one round of $\mathsf{LOCC}$, in which ancilla qubits are measured and local unitaries are applied based on the measurement outcomes. They show that in this model any 1D transitionally invariant matrix-product state (MPS) with fixed bond dimension is in the same phase of matter as the trivial state. Similar ideas can be found in~\cite{TVV_NonAbelianTopologicalOrder_2022, tantivasadakarn2021long}.

In this work, we introduce a new model, called \textit{Local Alternating Quantum-Classical Computations} ($\LAQCC$). In this model we alternate between running quantum circuits (constrained by locality), ending in the measurement of a subset of qubits, and fast classical computations based on the measurement results. The outcome of the classical computations are then used to control future quantum circuits. We allow for flexibility in this model, by giving different constraints to the power of both the quantum circuits and the classical circuits as well as the number of alternations between them. 
Most attention will be given to $\LAQCC$ containing quantum circuits of constant depth, classical circuits of logarithmic depth and at most a constant number of alternations between them. 
Any circuit constructed in this model is considered to be of constant depth. 
We restrict ourselves to logarithmic depth classical computations, as this is the first natural and non-trivial extension beyond constant-depth classical computations. 
Constant-depth classical computations do however also have an equivalent constant-depth quantum implementation.

The definition of $\LAQCC$ sharpens the original definition of \citeauthor{PhamSvore2013} by adding constraints to the intermediate classical computations. This allows us to bound the power of $\LAQCC$ from above. 

The main result of \citeauthor{Cirac:2021}, that 1D translational invariant MPS with fixed bond dimension can be prepared by $\mathsf{LOCC}$-assisted circuits, relies on local symmetries of the MPS. These symmetries allow them to prepare local states (on a constant number of qubits) and glue them together by doing one round of the appropriate entangling measurement and corrections, after which they run a round of local unitaries to get the desired result. This general scheme for preparing states that exhibit an MPS description with the appropriate local symmetries requires only geometrically local unitaries and one round of measurement and corrections an therefore is accessible in $\LAQCC$. Studying different local symmetries, known as Symmetry Protected Topological (SPT) phases of matter, to find measurement-based constant depth circuits for states is a broad ongoing field of research~\cite{TVV_NonAbelianTopologicalOrder_2022, tantivasadakarn2021long, smith2023deterministic}. 
All these schemes have a $\LAQCC$ implementation.

%$\LAQCC$-circuits also exist for general schemes of preparing local states, based on the local tensors, and gluing them together using one round of entangled measurement and corrections, based on the local symmetry. 
%The main result of \citeauthor{Cirac:2021}, that 1D translational invariant MPS with fixed bond dimension can be prepared by $\mathsf{LOCC}$-assisted circuits, relies heavily on local symmetries of the MPS and as a result also has an equivalent $\LAQCC$ implementation. 
%The corrections applied after the measurement round are local unitaries depending on the local symmetries of the MPS. 

 

%This general scheme of preparing local states, based on the local tensors, and gluing it together by doing one round of entangled measurement and corrections, based on the local symmetry, is accessible in $\LAQCC$.
Note however that \citeauthor{Cirac:2021} also suggest a circuit for the $W$-state.
This circuit uses sequentially and dependent measurement-based corrections of the ancilla qubits. 
These dependent measurements translate to sequential alternations between the quantum and classical circuits and therefore increase the total depth to linear depth, exceeding the constant-depth constraints imposed by $\LAQCC$-circuits. 

We study the power of the $\LAQCC$ model with respect to state preparation, showing that even with only constant quantum-depth and logarithmic classical depth it remains possible to prepare states with long-range entanglement.
Another surprising result is that it is unlikely that $\LAQCC$ circuits are classically simulatable. We show that any instantaneous quantum polynomial-time (IQP) circuit~\cite{Bremner2010,Shepherd2009} has an $\LAQCC$ implementation.
Classical simulation of IQP circuits implies the collapse of the polynomial hierarchy to the third level, which is not believed to be true~\cite{Bremner2017}. Therefore, we expect that $\LAQCC$ circuits are unlikely to be classically simulatable. We bound the power of $\LAQCC$ by showing that it is contained in $\QNC^1$, the class of polynomial-size, log-depth circuits.

Next, we also study the power that intermediate classical calculations can add to quantum computations, by considering a new model that alternates between polynomially many polynomial-depth quantum circuits and unbounded classical computations
We study this model by doing a complexity theoretical analysis, where we draw inspiration from the notions of complexity given by \citeauthor{RosenthalYuen:2022}, \citeauthor{MetgerYuen:2023}, and \citeauthor{Aaronson:2004}.
All three complexity notions are based on the notion of state preparation, instead of more traditional definition of complexity such as the decidability of a computational problem. 
The first two consider classes based on sequences of quantum states preparable by a polynomial-sized quantum circuit, where the circuits are uniformly generated by a computational class, for instance, the class $\mathsf{PSPACE}$, which results in the complexity class $\mathsf{StatePSPACE}$~\cite{RosenthalYuen:2022,MetgerYuen:2023}.
The third notion considers a relative complexity, where the complexity is measured between two given states, and is measured by the number of gates, from a given gate-set, required to transform one state in another state~\cite{Aaronson:2004}. 
For our definition of state preparation complexity, we drop the uniformity constraint from~\cite{RosenthalYuen:2022,MetgerYuen:2023} and define a class as $\mathsf{StateX}$, which refers to states preparable by circuits of type $\mathsf{X}$. 
As an example, if $\mathsf{X} = \QNC^0$, this results in the class $\mathsf{StateQNC^0}$, which is the set of states preparable from the $\ket{0}^n$ state by poly-size constant-depth circuits. 
This notion is similar to the relative complexity from~\cite{Aaronson:2004}, where one state is the  $\ket{0}^n$ state and instead of counting the number of gates we consider the set of states preparable by a fixed number of gates. Using this notion of complexity we show that any state preparable by an $\LAQCC^*$ circuit is also preparable by a $\mathsf{PostQPoly}$ circuit, the class of circuits of polynomial depth with an additional post-selection gate. 

All Clifford circuits have a constant-depth $\LAQCC$ implementation, implying that any stabilizer state can be implemented by a constant-depth $\LAQCC$ circuit, see Section~\ref{sec:clifford_circuits} for a proof of this statement. 
Efficient circuits for stabilizer states have been known already through measurement-based quantum computing. Therefore this paper focuses on the preparation of non-stabilizer states, and as a surprising result we find novel constant-depth protocols for four very natural classes of non-stabilizer states.
Despite the extensive research into these four classes of non-stabilizer states and the many applications of them, no efficient constant- or low-depth state preparation protocols are known yet. We specifically consider these four classes as they are all often used as initial states in other algorithms.

The first state is a uniform superposition over an arbitrary number of states. 
This state finds applications in many quantum algorithms, as they often start with a uniform superposition over multiple states. 
This superposition is often achieved by applying Hadamard gates to every qubit due to its simplicity to prepare. 
Yet, the analysis of many algorithms, such as Shor's algorithm~\cite{Shor:1997}, would benefit from a different initial superposition. 
The circuit to prepare the uniform superposition over an arbitrary number of states uses an exact version of Grover search as a subroutine, that turns a probabilistic circuit, with a known constant probability of success, into a deterministic circuit. 
We use the circuit for preparing a uniform superposition over an arbitrary number of states as a subroutine in the next two quantum state preparation protocols. 

The second state is the $W$-state, the uniform superposition over all computational basis states of Hamming-weight~$1$, a natural long-ranged entangled state that displays a fundamentally nonequivalent type of entanglement from the Greenberger–Horne–Zeilinger state~\cite{WState:2000}, for which $\LAQCC$-type constant-depth circuits were previously known~\cite{PhamSvore2013, Cirac:2021}. 
The $W$-state is often used as benchmark for new quantum hardware~\cite{Haffner2005,Neeley2010,GarciaPerez:2021}. 
A novel way to prepare the $W$-state therefore gives a new way to benchmark different quantum devices with each other. 
A circuit for preparing the $W$-state was given in~\cite{Cirac:2021}, but this implementation requires sequentially alternating measurements followed by local unitaries, which in the $\LAQCC$ model is not considered to be of constant depth. 
We improve this protocol by giving an $\LAQCC$ implementation of the $W$-state, based on a compress-uncompress method that links the one-hot and binary encoding of integers.

The third state considered is the Dicke state, a generalization of the $W$-state, a superposition over all computational basis states with Hamming-weight $k$~\cite{Dicke:1954}. 
Dicke states have relevance in various practical settings.
For instance, for quantum game theory~\cite{zdemir2007}, quantum storage~\cite{Bacon_Compress:2006,Plesch:2010}, quantum error correction~\cite{ouyang2014permutation}, quantum metrology~\cite{toth2012multipartite}, and quantum networking~\cite{prevedel2009experimental}. 
Dicke states have been used as a starting state for variational optimization algorithms, most notably Quantum Alternating Operator Ansatz (QAOA)~\cite{Hadfield2019}, to find solutions to problems such as Maximum k-vertex Cover~\cite{Brandhofer2022,cook2020quantum}.
The ground states of physical Hamiltonians describing one-dimensional chains tend to show a resemblance to Dicke states such as states resulting from the Bethe ansatz, making them an ideal starting state when investigating the ground state behavior of these Hamiltonians~\cite{TDL_BetheAnsatzDerivation:2010,B_ExcitedStateQuantumPhaseTransitions:2013,DickeTransitions:2021}. 
For instance, the algorithm by \citeauthor{van2021preparing}, who give an algorithm to prepare the Bethe ansatz eigenstates of the spin-1/2 XXZ spin chain, starts by first preparing a Dicke state~\cite{van2021preparing}. 
A Dicke-state preparation protocol based on the compress-uncompress methodology used in the $W$-state furthermore finds applications in entanglement distillation, where the entanglement of a large state is concentrated on only a few qubits. 
Efficient deterministic circuits for preparing Dicke states have been proposed by \citeauthor{bartschi2019deterministic}~\cite{bartschi2019deterministic, bartschi2022deterministic_short_depth}. 
They provide a quantum circuit of depth $\mathO(k \log(\frac{n}{k}))$, allowing arbitrary connectivity, to prepare a Dicke state, which they conjecture to be optimal when $k$ is constant. 
In this work, we provide a constant-depth $\LAQCC$ circuit below their conjectured bound already for constant $k$. 
However, this does not directly disprove their conjecture, as we allow for intermediate measurements and classical computations. 
More significantly, we even construct constant-depth $\LAQCC$ circuits for $k = \mathO(\sqrt{n})$ greatly improving their bound.
This construction extends the compress-uncompress method for the $W$-state combined with additional subroutines. 

We continue with a log-depth state preparation protocol for the Dicke-state for arbitrary $k$. 
This protocol implements an efficient transformation between the factoradic number representation and the combinatorial number representation of a positive integer. 
The combinatorial number representation relates directly to the Dicke state. 
The provided efficient transformation between number representation systems might be of independent interest. 

We conclude by modifying our protocol for preparing a Dicke-state to a protocol that prepares quantum many-body scar states in constant-depth. 
These states have low entanglement and longer coherence times than states with similar energy density.
These characteristics make many-body scar states interesting to analyze and relevant within physics.
Many-body scar states appear for instance in the AKLT model~\cite{AKLT:1987,MRBAR:2018,MRB:2018} and different spin models~\cite{SI:2019,MOBFR:2020}.
Known methods for preparing these states have polynomial-depth~\cite{Gustafson:2023}, whereas our circuit has constant depth. 

% We conclude by studying the power that intermediate classical calculations can add to quantum computations. 
% In this study, we define a new model that relaxes constant-depth quantum circuits to polynomial depth quantum circuits, log-depth classical calculations to unbounded classical computations and a constant number of alternations to a polynomial number of alternations. 
% We call this model $\LAQCC^*$. 
% We study this model by doing a complexity theoretical analysis, where we draw inspiration from the notions of complexity given by \citeauthor{RosenthalYuen:2022}, \citeauthor{MetgerYuen:2023}, and \citeauthor{Aaronson:2004}.
% All three complexity notions are based on the notion of state preparation, instead of more traditional definition of complexity such as the decidability of a computational problem. 
% The first two consider classes based on sequences of quantum states preparable by a polynomial-sized quantum circuit, where the circuits are uniformly generated by a computational class, for instance, the class $\mathsf{PSPACE}$, which results in the complexity class $\mathsf{StatePSPACE}$~\cite{RosenthalYuen:2022,MetgerYuen:2023}.
% The third notion considers a relative complexity, where the complexity is measured between two given states, and is measured by the number of gates, from a given gate-set, required to transform one state in another state~\cite{Aaronson:2004}. 
% For our definition of state preparation complexity, we drop the uniformity constraint from~\cite{RosenthalYuen:2022,MetgerYuen:2023} and define a class as $\mathsf{StateX}$, which refers to states preparable by circuits of type $\mathsf{X}$. 
% As an example, if $\mathsf{X} = \QNC^0$, this results in the class $\mathsf{StateQNC^0}$, which is the set of states preparable from the $\ket{0}^n$ state by poly-size constant-depth circuits. 
% This notion is similar to the relative complexity from~\cite{Aaronson:2004}, where one state is the  $\ket{0}^n$ state and instead of counting the number of gates we consider the set of states preparable by a fixed number of gates. Using this notion of complexity we show that any state preparable by an $\LAQCC^*$ circuit is also preparable by a $\mathsf{PostQPoly}$ circuit, the class of circuits of polynomial depth with an additional post-selection gate. 

\paragraph{Summary of results}
\begin{itemize}
    \item We give a new definition of a computational model that captures the power of the four step process: applying a constant number of layers of one- and two-qubit gates; performing a syndrome measurement; perform a fast classical computation determining corrections; apply corrections. We call this model \emph{Local Alternating Quantum Classical Computations}, or $\LAQCC$ for short. In this model we bound the allowed quantum operations, intermediate classical calculations, and number of rounds separately. In Section~\ref{sec:LAQCC_model} we define this model and give a list of operations based on results from literature contained in this computational model. In some of these operations we explicitly use that we allow for multiple, but at most constant, rounds  of corrections.
    \item  We show show that there exist $\LAQCC$ circuits that can not be weakly simulated in Section~\ref{sec:IQP_in_LAQCC}. We further show that for every $\LAQCC$ circuit there exists a $\QNC^1$ circuit simulating it perfectly, in Section~\ref{sec:LAQCC_in_QNC1}.
    \item We introduce a new type computational complexity for preparing states and show that the extension of $\LAQCC$ where we allow a polynomial number of rounds and unbounded classical computation, is contained in $\mathsf{PostQPoly}$, the class of polynomial circuits with post-selection, in Section~\ref{sec:Complexity results}.
    \item We show a protocol to prepare the uniform superposition state of size $q$ in $\LAQCC$ using $\mathO(\ceil{\log_2(q)}^2)$ qubits in Section~\ref{sec:superposition_modulo_q}. 
    \item We show a protocol to prepare the $W_n$ state in $\LAQCC$ using $\mathO(n\log(n))$ qubits in Section~\ref{sec:W_state_in_LAQCC}.
    \item We show two ways of preparing the Dicke-$(n,k)$ state. The first method is in $\LAQCC$, works up to $k = \mathO(\sqrt{n})$, uses $\mathO(n^2\log(n))$ qubits, and is found in Section~\ref{sec:dicke:small_k}. The second method is in $\LAQCC\text{-}\mathsf{LOG}$ (an extension of $\LAQCC$ allowing for logarithmic number of alterations instead of constant), works for any $k$, uses $\mathO(\text{poly}(n))$ qubits, and is found in Section~\ref{sec:Dicke_in_LAQCC_LOG}. 
    \item We extend on our $\LAQCC$ method of generating Dicke-$(n,k)$ states for $k = \mathO(\sqrt{n})$ and show a protocol to generate many-body scar states for a particular Hamiltonian in $\LAQCC$ (Section~\ref{sec:many_body_scar}). 
\end{itemize}
Summarized in a table, we provide the following state generation protocols:
\begin{table}[htb]
\centering
\begin{tabular}{l|l|l|l}
\textbf{State description} & \textbf{Width} & \textbf{Depth} & \textbf{Implementation}\\
\hline 
Uniform superposition mod $q$: $\frac{1}{\sqrt{q}} \sum_{i = 0}^{q-1}\ket{i}$ & $\mathO(\ceil{\log^2 q})$ & $\mathO(1)$ & Section~\ref{sec:superposition_modulo_q}\\

$W$-state: $\frac{1}{\sqrt{n}}\sum_{i = 0}^{n-1}\ket{e_i}$ & $\mathO(n \log n)$ & $\mathO(1)$ & Section~\ref{sec:W_state_in_LAQCC}\\

Dicke-$(n,k)$, $k = \mathO(\sqrt{n})$: $\binom{n}{k}^{-1/2}\sum_{x \in \{0,1\}^n: |x| = k} \ket{x}$ &  $\mathO(n^2\log n)$ & $\mathO(1)$ 
&Section~\ref{sec:dicke:small_k}\\

Dicke-$(n,k)$: $\binom{n}{k}^{-1/2}\sum_{x \in \{0,1\}^n: |x| = k} \ket{x}$ & $\mathO(\text{poly}(n))$ & $\mathO(\log n)$ &Section~\ref{sec:Dicke_in_LAQCC_LOG}\\

QMBS: $\ket{S_k} = \frac{1}{k! \sqrt{\mathcal N(n,k)}}(Q^\dagger)^k \ket{\Omega}$ &  $\mathO(n^2\log n)$ & $\mathO(1)$  &  Section~\ref{sec:many_body_scar}
\end{tabular}
\caption{Summary of state preparation protocols given in this paper.}
\label{tab:sate_prep}
\end{table}
In the entry for the quantum many-body scar state $Q$ denotes the raising operator and $\mathcal N(n,k)=\binom{n-k-1}{k}$. 
Section~\ref{sec:many_body_scar} will provide more details on the variables and the implementation. 

\paragraph{Organization of the paper}
\noindent We first introduce relevant preliminaries in Section~\ref{sec:preliminaries}. 
In Section~\ref{sec:LAQCC_model} we formally define the class of Local Alternating Quantum-Classical Computations ($\LAQCC$). We also show that any Clifford circuit can be implemented in constant depth $\LAQCC$ (a result based on a result from measurement-based quantum computing~\cite{jozsa2006introduction}). 
This result allows us to give many useful multi-qubit gates and routines in Section~\ref{sec:gates_created_in_LAQCC}. 
Beyond that we show that constant depth $\LAQCC$ circuits are contained in $\QNC^1$ and that any $\mathsf{IQP}$ circuit has an $\LAQCC$ implementation.
We conclude this section with an analysis of a more powerful instantiation of $\LAQCC$ and show an inclusion with respect to the class $\mathsf{PostQPoly}$, which is the class of circuits of polynomial depth with one additional post-selection gate. 
In Section~\ref{sec:state_prep_in_LAQCC} we give $\LAQCC$ circuit implementations for preparing the uniform superposition over an arbitrary number of states, the $W$-state and the Dicke state up to $k = \mathO(\sqrt{n})$. We furthermore give a log-depth circuit implementation for preparing the Dicke state for any $k$. We conclude by showing a $\LAQCC$ circuit for generating many body scar states of a particular type of Hamiltonian.


\section{Evaluating diversity}\label{sec:diversity}

We now formalize our approach to evaluating diversity. For a set $S$ of items, we define
\begin{equation}
    r_t(S) := \frac{\text{\# of items in }S\text{ of type }t}{|S|},
\end{equation}
the \textbf{representation} of type $t$ in $S$. Intuitively, a set of recommendations is diverse if all types are well represented. We now define an interpretable family of representations that interpolates between maximum diversity and maximum homogeneity, and which arises naturally in many of our results.

\begin{definition}[$\gamma$-homogeneity]
A set $S$ is \textbf{$\bm{\gamma}$-homogeneous} if for all $t\in [m],$
\begin{equation}\label{eq:gamma}
r_t(S) = \frac{p_t^{\gamma}}{\sum_{i=1}^m p_i^{\gamma}}.
\end{equation}
\end{definition}

\noindent $\gamma$-homogeneity captures several intuitive notions of diversity, using $p_1,\cdots,p_m$ as a benchmark:
\begin{itemize}[leftmargin=*]
\item When $\gamma = 0$, $r_t(S)=\frac{1}{m}.$ There is ``equal representation.''
\item When $\gamma = 1$, $r_t(S)=p_t.$ There is ``proportional representation,'' where an item type is represented in proportion to its likelihood.
\item When $\gamma = \infty$, $r_t(S)=1$ for $t=\argmax_{i\in [m]} p_i$ and $r_t(S)=0$ otherwise. There is ``complete homogeneity,'' where only the highest-likelihood item type is represented.
\end{itemize}
A smaller $\gamma$ corresponds to more diversity, with $\gamma\le 1$ indicating \textit{at least proportional} representation. In practice, it is challenging to show that individual sets are $\gamma$-homogeneous; for one, since sets have an integer number of items from each type, it is typically impossible to obtain the exact ratios in \eqref{eq:gamma}. Instead, we will give primarily asymptotic results, showing that as $n$ grows large, the optimal set $S_{n,k}$ approaches $\gamma$-homogeneity. Formally, we define $\gamma$-homogeneity over sequences of sets:

\begin{definition}[$\gamma$-homogeneity for set sequences]
A sequence of sets $\{S_n\}_{n=1}^\infty$ is \textbf{$\bm{\gamma}$-homogeneous} if for all $t\in [m],$
\begin{equation}
\lim_{n\rightarrow \infty} r_t(S_n) = \frac{p_t^{\gamma}}{\sum_{i=1}^m p_i^{\gamma}}.
\end{equation}
\end{definition}

One perhaps surprising aspect of our results is that $\gamma$-homogeneity is sufficient to characterize diversity in a large class of settings, as opposed to requiring more complicated functions of proportions $p_t$. 

\section{Main results}
We now state our main results, which consider several settings reflecting different assumptions about the conditional item values $X_i^{(t)}.$ In \Cref{sec:general}, we assume that $X_i^{(t)}\simiid \DD$ are drawn from a shared distribution, which implies that the recommender has little information about the value of specific items. In \Cref{sec:bernoulli}, $X_i^{(t)}$ are Bernoulli random variables with success probabilities differing depending on $i$ and $t$, meaning that the recommender has information about which items are more likely to satisfy a user.

In each setting, we analyze the diversity of optimal sets $S_{n,k}$ for when $k$ is fixed (i.e., accounting for consumption constraints) and $k=n$ (i.e., not accounting for consumption constraints). We sketch our proof strategy in \Cref{sec:proof-sketch} and defer full proofs to \Cref{sec:proofs}.

\subsection{i.i.d. conditional item values}\label{sec:general}
Consider the setting in which
$X_i^{(t)}\simiid \DD,$
i.e., conditional item values are drawn from a shared distribution. This implies that the values of items behave similarly across types and within types, and the platform cannot easily distinguish between the items in a type. From the perspective of a hiring platform, there may be many candidates with similar backgrounds (e.g., education or work history), none of whom can be distinguished from another by the platform. Conditional on a recruiter preferring this background, candidate values can be modeled as coming from a shared distribution.

We show that for a fixed $k$, as $n$ grows large, the diversity of $S_{n,k}$ theoretically varies between equal representation, proportional representation, and near-complete homogeneity depending on the tail-behavior of $\DD$. In particular, distributions that are bounded or have exponential tails induce at least proportional representation. Meanwhile, $S_{n,n}$ is completely homogeneous.

\begin{theorem}\label{thm:general}
Suppose $X_i^{(t)}\simiid \DD$ where $\DD$ has finite mean. Then the following statements hold.
\begin{enumerate}
\item[(i)] \textbf{[Finite Discrete]} If $\DD$ is a finite discrete distribution, $\{S_{n,k}\}_{n=1}^\infty$ is $0$-homogeneous.
\item[(ii)] \textbf{[Bounded]} If $\DD$ has support bounded from above by $M$ with pdf $f_\DD$ satisfying
\begin{equation}
\lim_{x\rightarrow M} \frac{f_\DD(x)}{(M-x)^{\beta-1}} = c
\end{equation}
for some $\beta, c>0$, then $\{S_{n,k}\}_{n=1}^\infty$ is $\frac{\beta}{\beta+1}$-homogeneous.\\(This pdf class contains beta distributions, including the uniform distribution.)
\item[(iii)] \textbf{[Exponential tail]} If $\DD = \Exp(\lambda)$ for $\lambda > 0,$ then $\{S_{n,k}\}_{n=1}^\infty$ is $1$-homogeneous.
\item[(iv)] \textbf{[Heavy tail]} If $\DD = \Pareto(\alpha)$ for $\alpha>1$, then $\{S_{n,k}\}_{n=1}^\infty$ is $\frac{\alpha}{\alpha-1}$-homogeneous.
\end{enumerate}
Additionally,
\begin{enumerate}
    \item[(v)] $S_{n,n}$ contains only items of type $t = \argmax_{t\in [m]} p_t.$
\end{enumerate}
\end{theorem}

As \Cref{table:thmgeneral} illustrates, the theorem shows how for fixed $k$, the diversity of optimal solutions depends on the tail behavior of $\DD$. In fact, we can obtain $\gamma$-homogeneity for any $\gamma$:
\begin{corollary}
    For any $\gamma\ge 0,$ there exists $\DD$ such that when $X_i^{(t)}\simiid \DD$ and $k$ is fixed, $\{S_{n,k}\}_{n=1}^\infty$ is $\gamma$-homogeneous.
\end{corollary}
Intuitively, heavy-tailed distributions (part (iv)) induce less diverse recommendations since the marginal returns of recommending more items from the same type remains high: drawing more samples from a heavy-tailed distribution produces ever-increasing item values. This contrasts with bounded distributions like the uniform distribution (part (ii)), where once an item has close to the maximum value, additional draws of that type will not further improve the utility significantly. 
\begin{table}\label{table:thmgeneral}
    \begin{center}
    \caption{A summary of Theorem 1. For $X_i^{(t)}\simiid \DD$, distributions $\DD$ with heavier tails induce less diversity.}
    \label{table:thmgeneral}
      \vspace*{3mm}
      \begin{tabular}{l c c c c c}
        \toprule %
        \textbf{} & \multicolumn{3}{c}{\textbf{bounded}} & \textbf{exp. tail} & \textbf{heavy tail}\\
        & \multicolumn{3}{c}{Thm. 1(ii)} & Thm. 1(iii) & Thm. 1(iv)\\
        \cmidrule(r){2-4}
        \cmidrule(r){5-5}
        \cmidrule(r){6-6}
        example $\DD$ &  & $\betaa(\cdot,\beta)$ &  & $\Exp(\lambda)$ & $\Pareto(\alpha)$\\
         & \small{$0<\beta<1$} & \small{$\beta=1$} & \small{$\beta>1$} & \small{$\lambda>0$} & \small{$\alpha>1$}\\        
% Figure removed
&% Figure removed

& % Figure removed

& % Figure removed 

&% Figure removed

&% Figure removed
\\
$\{S_{n,k}\}_{n=1}^\infty$\\$\gamma$-homog.\\for $\gamma\in$ & $(0,1/2)$ & $1/2$ & $(1/2,1)$ & $1$ & $(1,\infty)$\\
&&&&(i.e., proportional)&\\
&&&&&\\
&\multicolumn{5}{c}{$\longleftarrow$ more diverse \qquad \qquad \qquad \qquad \qquad less diverse $\longrightarrow$}\\
\midrule
\end{tabular}
\end{center}
\end{table}

\paragraph{A result for finite $n$ and larger $k$.} One limitation of our main results is that they are asymptotic $(n\rightarrow \infty)$ and are restricted to fixed consumption constraints $k$. Stronger results can be obtained by considering specific distributions. For example, the result below characterizes for any $n, k$ the representation of each type when conditional item values are uniformly distributed on $[0,1].$

\begin{proposition}\label{prop:uniform}
When $X_i^{(t)}\simiid U([0,1])$,
\begin{equation}
    \left|r_t(S_{n,k}) - \frac{\sqrt{p_t}}{\sum_{i=1}^m \sqrt{p_i}}\right|\le \frac{m+1}{n}.
\end{equation}
for all $k\le \frac{\sqrt{p_m}}{\sum_{i=1}^m \sqrt{p_i}}n - m - 1.$
\end{proposition}
Therefore, for any $n$, $S_{n,k}$ is approximately $\frac{1}{2}$-homogeneous. In addition, for any $k$ that is smaller than a constant fraction of $n$, the diversity $S_{n,k}$ does not depend on $k$. Thus, even for small sets of recommendations and large consumption constraints, diversity is optimal in this setting. We further give simulated results for small $n$ and large $k$ in \Cref{sec:simulations} corroborating our theoretical results.
\subsection{Heterogeneous Bernoulli conditional item values}\label{sec:bernoulli}
Now consider when $X_i^{(t)}$ are independent random variables drawn from $\Ber(q_i^{(t)})$, reflecting a model in which items have binary values (i.e., a user is either satisfied or not satisfied by an item). In this section, we allow $q_i^{(t)}$ to differ across $i$ and $t$, implying that the recommender has knowledge about which items are more likely to be successful conditional on a user's preferred type. Specifically, in \Cref{sec:ber-decay} we allow $q_i^{(t)}$ to vary across $i$ and in \Cref{sec:ber-varying} we allow $q_i^{(t)}$ to vary across $t$.

Our results will focus on $S_{n,1}$ and $S_{n,n}$, which both have natural interpretations in this setting:
\begin{itemize}[leftmargin=*]
    \item $S_{n,1}$ maximizes the probability that the user will be satisfied by at least one recommended item.
    \item $S_{n,n}$ maximizes the the expected number of recommended items the user will be satisfied by, which is equivalent to the standard metric of accuracy.
\end{itemize}
Before proceeding, we note that the basic case $q_i^{(t)} = q$ for all $i,t$ is handled as a direct corollary of \Cref{thm:general}(i).

\begin{corollary}[Conditional item values are i.i.d. Bernoulli]\label{cor:ber}
When $X_i^{(t)}\simiid \Ber(q)$ for $q>0,$ then $S_{n,1}$ is $0$-homogeneous.
\end{corollary}

Therefore, if the success probability is the same for all items, optimal solutions are $0$-homogeneous (each item is equally represented) for large $n$, even as the likelihoods $p_t$ vary across type.

\subsubsection{Decaying success probabilities}\label{sec:ber-decay} We now consider a setting in which among items of the same type, the recommender knows that some items have higher success probability. This maps onto settings where the recommender knows which items of a type are most likely to be satisfactory, e.g., some action movies are more commonly liked than are others. Thus, we assume that the recommender has access to items with decaying success probabilities.

\begin{theorem}[Decaying success probabilities]\label{thm:ber-decay}
Suppose that $X_i^{(t)}\simiid \Ber(q_i^{(t)})$ are i.i.d. Bernoulli random variables such that $q_i^{(t)}= c(i+d)^{-\alpha}$ for all $i\ge 1$ and some $\alpha,c,d\ge 0.$ Then the following statements hold.
\begin{enumerate}
    \item[(i)] $\{S_{n,1}\}_{n=1}^\infty$ is $0$-homogeneous for $\alpha < 1.$
    \item[(ii)] $\{S_{n,1}\}_{n=1}^\infty$ is $\frac{1}{1+c}$-homogeneous for $\alpha = 1.$
    \item[(iii)] $\{S_{n,1}\}_{n=1}^\infty$ is $\frac{1}{\alpha}$-homogeneous for $\alpha > 1.$
\end{enumerate}
Additionally,
\begin{enumerate}
    \item[(iv)] $\{S_{n,n}\}_{n=1}^\infty$ is $\frac{1}{\alpha}$-homogeneous for $\alpha\ge 0.$
\end{enumerate}
\end{theorem}

\input{figures/decay.tex}

When the success probabilities of items have moderate decay ($\alpha<1$), then $0$-homogeneity is maintained in the case $k=1$ (note that $\alpha = 0$ recovers \Cref{cor:ber}). Moreover, for all rates of decay, optimal recommendations reflect at least proportional diversity for large $n$ and $k=1$.

\Cref{thm:ber-decay} also reveals surprising \textit{non-monotonic} behavior. In particular, there is a discontinuity at $\alpha=1$, where homogeneity suddenly increases, but then decreases as $\alpha$ continues to increase. At $\alpha=1,$ the optimal amount of diversity when $\alpha=1$ can range between $0$ and $1$ depending on $c$.

When $k=n$, a larger rate of decay induces more diverse recommendations. Intuitively, when there is a larger rate of decay, the recommender has fewer high-quality options of a given type and is more incentivized to recommend high-quality options of other types. Note that for moderate rates of decay ($\alpha < 1$), $S_{n,n}$ remains less than proportionally diverse for large $n$, unlike $S_{n,1}$.

\subsubsection{Varying success probability across types}\label{sec:ber-varying} We now consider a setting in which the success probability of an item varies across types. This can arise when a users are more picky for some types of items, or when the recommender has more information about items from one type than another.%

\begin{theorem}[Varying success probability across types]\label{thm:ber-varying}
Suppose that for each fixed $t$, $X_i^{(t)} \simiid \Ber(q_t)$ are i.i.d. Bernoulli random variables. Then
\begin{equation}
\lim_{n\rightarrow\infty}r_t(S_{n,1}) \propto \frac{1}{\log \frac{1}{1-q_t}}
\end{equation}
while $S_{n,n}$ contains only items of type $t = \argmax_{t\in [m]} p_tq_t.$
\end{theorem}

The surprising high-level takeaway from \Cref{thm:ber-varying} is that, for large $n$, a \textit{smaller} success probability $q_t$ results in \textit{more} representation of type $t$. The less likely an item of a given type is satisfactory, the more that type is recommended. Moreover, note that the amount of representation in this setting is independent of the popularities $p_1,\cdots,p_m.$

This paradox is illustrated in grocery stores, where more space is devoted to ice cream than milk, despite milk being much more popular than ice cream. Here, $p_1$ (the popularity of milk) is higher than $p_2$ (the popularity of ice cream). However, $q_1$ (the likelihood a given milk product satisfies a shopper looking for milk) is also higher than $q_2$ (the likelihood a given ice cream product satisfies a shopper looking for ice cream), since people tend to have more specific tastes for ice cream. Thus, since $q_2$ is smaller than $q_1$ and the grocery store should ``recommend'' many more ice creams than milks, explaining why more space is devoted to ice cream. Intuitively, while more shoppers want milk, these consumers can be satisfied with a small selection of milk; thus, it is more beneficial to devote more space to ice cream, for which shoppers have more specific tastes.
\section{Sketch of Proofs}\label{sec:proof-sketch}
We now sketch the proof of Theorem 1(ii). (The general strategy applies to the rest of Theorem 1, as well as Theorem 2 and Theorem 3.) First, recall the general setup in Theorem 1, where the conditional item values $X_i^{(t)}$ are drawn i.i.d. from a shared distribution $\DD$. In other words, a user prefers type $t\in [m]$ with probability $p_t$ such that the user prefers exactly one type of item, and conditioned on the user preferring type $t$, the value of an item of that type is drawn i.i.d. from $\DD$. We are interested in analyzing, depending on $\DD$, the composition of $S_{n,k}$, the set of $n$ items maximizing the expected sum of the $k$ highest value items in the set. If $S_{n,k}$ contains $a_t^{(n)}$ items of type $t$, we need to analyze
\begin{equation}\label{eq:lim}
\lim_{n\rightarrow \infty} r_t(S_{n,k}) = \lim_{n\rightarrow \infty} \frac{a_t^{(n)}}{n}.
\end{equation}
We first provide an expression for the expected sum of the $k$ highest value items in a set $S$ with $a_t$ items of type $t$. The following definition will be useful.
\begin{definition}
    Define $\mu_\DD(i,a)$ to be the expected value of the $i$-th order statistic\footnote{The $i$-th order statistic of $n$ random variables is the $i$-th smallest of the $n$ realized values.} of $a$ random variables drawn i.i.d. from $\DD$. (Thus, $\mu_\DD(1,a)$ is the expected minimum of $a$ i.i.d. draws from $\DD$ and $\mu_\DD(a,a)$ is the expected maximum.)
\end{definition}
Then conditioned on the user preferring type $t$, the expected sum of the $k$ highest value of items in $S$ is equal to
   $h(a_t) := \sum_{i=1}^{\min\{k, a_t\}} \mu_\DD(a_t-i+1,a_t),$
which follows from the linearity of expectation. Therefore, the expected sum of the $k$ highest value items in $S$ is $\sum_{t=1}^m p_t h(a_t)$. Define $A_n$ to be the set of tuples of non-negative integers whose entries sum to $n$. Then
$(a_1^{(n)}, a_2^{(n)}, \cdots, a_m^{(n)}) = \argmax_{(a_1,\cdots,a_m)\in A_n} \sum_{t=1}^m p_t h(a_t).$

We can then determine the limit in \eqref{eq:lim} given asymptotic information about $h$. In \Cref{lem:fennel} in the appendix, we develop general technical machinery for this task. Below, we state \Cref{lem:fennel}(ii), which can be used to prove Theorem 1(ii).
\begin{mylem}{\ref{lem:fennel}(ii)}
If $h$ is monotonically increasing and there exist constants $A,B>0$ and $\sigma<0$ such that $\lim_{a\rightarrow \infty} \frac{A-h(a)}{Ba^{\sigma}} = 1,$ then
$\lim_{n\rightarrow \infty} \frac{a_t^{(n)}}{n} = \frac{p_t^{\frac{1}{1-\sigma}}}{\sum_{i=1}^m p_i^{\frac{1}{1-\sigma}}}.$
\end{mylem}
Then, considering $\DD$ as in Theorem 1(ii), we can prove the necessary asymptotic result about $h$:
\begin{lemma}\label{lem:bob}
If $\DD$ has support bounded from above by $M$ with pdf $f_\DD$ such that
$\lim_{x\rightarrow M} \frac{f_\DD(x)}{(M-x)^{\beta-1}} = c$
for some $\beta, c>0$, then
$\lim_{a\rightarrow \infty} \frac{Mk - h(a)}{Ba^{-\frac{1}{\beta}}} = 1.$
\end{lemma}
Combining \Cref{lem:bob} with \Cref{lem:fennel}(ii), with $\sigma = -\frac{1}{\beta}$, we show that for $\DD$ as in Theorem 1(ii),
\begin{equation}
    \lim_{n\rightarrow \infty}\frac{a_t^{(n)}}{n} = \frac{p_t^{\frac{\beta}{\beta+1}}}{\sum_{i=1}^m p_i^{\frac{\beta}{\beta+1}}}.
\end{equation}

\section{Conclusion and Future Work}
In this work, I design corruption-robust algorithms for the Lipschitz contextual search problem. I present the \emph{agnostic checking} technique and demonstrate its effectiveness in designing corruption-robust algorithms. There are several open problems for future research. First, in the algorithm I propose for pricing loss, the schedule for agnostic checks is fixed upfront. Can the learner design an adaptive checking schedule for the pricing loss? Second, this work assumes the learner has knowledge of the Lipschitz constant $L$. Can the learner design efficient no-regret algorithms without knowledge of $L$? 

\newpage
{\small
\bibliographystyle{unsrt}
\bibliography{bib}
}

\newpage
\pagebreak
\appendix
Most academic vocabulary lists have been developed in the context of English for Academic Purposes (EAP). On the whole, two categories of lists exist. One list type aims to identify academic words commonly used in EAP across disciplines, which students could be made aware of. The studies aiming to provide cross-disciplinary academic word lists usually use large corpora containing expert academic writing from various disciplines. The widely used lists of this type are the Academic Word List (AWL) \cite{coxhead2000new} and the Academic Vocabulary List (AVL) \cite{gardner2014new}. The second type of list seeks to identify discipline or field-specific words worth teaching. Various specialised lists have been developed for fields such as veterinary medicine \cite{ohashi2020esp} or nursing \cite{yang2015nursing}.

While there is a growing interest in building cross-disciplinary academic word lists for languages other than English, these academic word lists remain few. See, for example studies conducted for French \cite{cobb2004there}, Persian \cite{rezvanifirst}, Portuguese \cite{baptista2010p}, Swedish \cite{carlund2012academic}, and Norwegian \cite{johannessen2016constructing}. An explanation for this scarcity might be that academic language data sets are rare and often not freely available due to copyright. This can be especially true for low-resource languages, such as Romanian. Access to a representative corpus is crucial, as the validity and reliability of an academic word list highly depend on the quality of the data set. 

Apart from the limited availability of academic writing corpora, an additional challenge may be that there is no standard procedure for extracting academic word lists. Scholars are still exploring and testing various methodologies. For example, some studies build on the methods used for the AWL or the AVL \cite{johannessen2016constructing,rezvanifirst}. One study uses the translated version of the AVL in Portuguese as a starting point for its investigation \cite{baptista2010p}. Another study proposes a new word list extraction method different from previous ones \cite{carlund2012academic}.  

In the case of Romanian, no previous studies have compiled specialised or general academic word lists. Although in the last 10-15 years, several research institutions and projects have been involved in developing corpus resources in Romanian, relatively few have focused exclusively on general academic writing. Some of the most significant corpora recently compiled, such as ROMBAC (Romanian Balanced Annotated Corpus, see \citet{ion2012rombac}), with more than 30 million words, CoRoLa (Corpus of Contemporary Romanian Language, see \citet{mititelu2014corola}), or The Balanced Romanian Corpus (BRC, see \citet{midrigan2020resources}) cover only few disciplines or subsets: 5 sections for ROMBAC (journalism, literature, medical texts, legal texts, biographies), uneven and unfiltered distribution of resources in CoRoLa (the collection of academic writing texts is based on agreements with publishing houses and journals, without filtering of the content on quality criteria) and BRC (literary text samples, research articles, news, spoken data). The ROMBAC corpus (excluding the medical subcorpus) was already used to develop the Romanian Word List (RWL, see \citet{szabo2015introducing}), targeted at Romanian L2 learners (e.g. from the Hungarian minority in Romania). The list is a general list of words, not focused on academic language. As far as discipline-specific corpora are concerned, smaller corpora such as SiMoNERo (medical corpus, \citet{mitrofan2019monero}), BioRo \cite{mitrofan2018bioro}, PARSEME-Ro (news articles), LegalNERo (legal, \citet{paiș2021named}), MARCELL (legal, multilingual, see \citet{varadi2020marcell}), CURLICAT (multilingual, containing several domains: Economics, Education, Health, Sciences, etc., see \citet{varadi2022introducing}) have been compiled. However, apart from compiling the datasets and conducting a series of descriptive studies, no special attention is given to the lexical level. 

In this context, the EXPRES corpus (Corpus of Expert Writing in Romanian and English) is the first corpus of discipline-specific academic writing in the Romanian context (academic writing in Romanian L1 and academic writing in English L2 produced by Romanians) \cite{bucur2022expres,chitez2022write}. Covering four disciplines – Linguistics, Economics, Political Sciences, Information Technology –, the Romanian subset contains 200 open-access research articles from each domain, published in the past 5-10 years in peer-reviewed journals (see \citet{chitez2022expres}). The rigorous selection criteria \cite{rogobete2021challenges} contribute to the representativeness of the corpus, making it a suitable candidate for testing a possible Romanian Word List and narrowing it down to an Academic Word List. Furthermore, the EXPRES corpus is the first Romanian expert academic corpus available on an open-access query platform. Unlike other Romanian corpora, which offer limited access to third parties and poor resources for downloading search results or statistics, the EXPRES corpus support platform has been implemented as a cross-platform distributed web application  \cite{chitez2022expres}.

\subsection{Computer Simulated Experiments}

\subsubsection{MNIST Simulated Data Preparation}

% \subsubsection{MNIST Data Pre-processing}

\Xpolish{The MNIST dataset consists of handwritten digits from 0 to 9, with a default size of 28 by 28 pixels \cite{lecun1998MNIST}. In order to align the data with the Hadamard matrices, we padded the images with black pixels at the edges to resize them to 32 by 32 pixels \cite{lecun1998MNIST}. We then transformed the range of all pixel values from $[0,1]$ to $[0.3, 1]$. This operation enhanced the persuasiveness of the dataset for various noise models, as the added black pixels do not generate Poisson noise, whose variance is proportional to the expected photon counts. \bnote{did we ever vary the dark level (0.3)/ have results showing its effect?}}
{The MNIST dataset, which comprises handwritten digits ranging from 0 to 9, has a default size of 28 by 28 pixels \cite{lecun1998MNIST}. To align the data with Hadamard matrices, we added black pixels to the edges of the images and resized them to 32 by 32 pixels \cite{lecun1998MNIST}. Subsequently, we rescaled the pixel values from the original range of $[0,1]$ to $[0.3, 1]$. This rescaling was aimed at improving the dataset's suitability for different noise models, as the initial black backgrounds do not introduce Poisson noise, which is typically proportional to the expected photon counts.}

\subsubsection{Spectral Datasets and Performance Evaluation}


% \rnote{How well this model can work w/o noise}

% \subsubsection{Performance Evaluation}


\Xpolish{The performances of the models were evaluated through their average classification rates. To ensure a thorough assessment, the model was tested five times independently, each time with a randomly split dataset for 70\% training and 30\% validation. During each assessment, the model generated noisy photon counts, which were then used to train the classifier until the validation rate reached a plateau or declined. The best validation rate achieved during the training process was chosen as the representative performance upper bound for each assessment. The overall performance of the model was determined by averaging the results of all five assessments.}
{The model performances were assessed based on their average classification rates, which were calculated after five independent tests. In each test, the dataset was randomly split into 70\% for training and 30\% for validation. The model generated noisy photon counts during each assessment, and the training continued until the validation rate reached a plateau or declined. The highest validation rate obtained during the training process was considered as the representative upper bound for each assessment. The overall model performance was determined by averaging the results from all five assessments. This comprehensive evaluation ensured the reliability and accuracy of the model performance.}


\section{Generalizations}\label{sec:generalizations}
In this section, we consider two directions in which our model can be extended. First, we consider a model in which users can prefer multiple item types at a time. Second, we consider a model in which users and items are both represented by vector embeddings, a common paradigm in recommender systems.

\subsection{Preference for multiple item types}

In this section, we consider an extension of our model in which a user can prefer multiple types of items. In particular, we focus on the case when $m=2$ and when the user prefers only type $1$ with probability $p_1$, only type $2$ with probability $p_2$, and both types with probability $p_{12}.$ We also assume that an item of type $i$ satisfies a user that prefers type $i$ independently with probability $q$. We consider when $k=1$, when the user can only use one item. Then, we would like to minimize the probability of having no satisfying items:
\begin{equation}
    p_1(1-q)^{a_1} + p_2(1-q)^{a_2} + p_{12}(1-q)^n,
\end{equation}
where $a_1 + a_2 = n$.
This is equivalent to minimizing
\begin{equation}
    p_1(1-q)^{a_1} + p_2(1-q)^{a_2},
\end{equation}
which is mathematically equivalent to \Cref{thm:general}(i). Thus, we have that as $n$ grows large, both item types are represented equally, irrespective of $p_1, p_2,$ and $p_{12}$---in other words, allowing for the possibility that the user prefers multiple item types does not change the result in this setting.

The setup here can be naturally generalized to the case in which there are more than two types, and where item values can come from distributions other than Bernoulli. We leave these generalizations to future work.


\subsection{User and item embeddings} In this section, we consider the setting in which users and items each correspond to vector embeddings, and where the likelihood a user $u$ is satisfied by an item $v$ is a function of their cosine distance. Suppose that a user has preference $u$ drawn according to a probability measure $\mu$ supported on the unit $d-$dimensional sphere $S_d\subset \RR^d.$ Further suppose that for a movie $v\in S_d,$ the probability that a user is not satisfied by the movie is $p(u,v) = q(\norm{u-v})$, a function of the cosine distance between $u$ and $v$.

If the goal is to maximize the probability a user is satisfied by at least one item, what is the optimal choice of items to recommend as the number of recommendations we can make grows large? Here, we will show that items should be chosen ``uniformly'' from $S_d$---a result that may seem surprising as it is independent of the distribution of user preferences, and which can be viewed as an analog of \Cref{thm:general}(i).

To make things precise and tractable, rather than consider the recommendation of individual items, we will focus on the recommendation of a ``distribution of items.'' This is analogous to making a continuous relaxation of the discrete item recommendation problem. For a set of items $V = \{v_1,\cdots,v_n\},$ the probability a user with preference drawn according to a probability measure $\mu$ does not like any item in $V$ is
\begin{equation}
    \int_{S_d} \mu(u) \prod_{v\in V} p(u,v) \,du = \int_{S_d} \mu(u) \exp \sum_{v\in V} \log p(u,v) \,du.
\end{equation}
Then for a density function $\alpha: S_d \rightarrow [0,\infty)$, we can consider the expression
\begin{equation}
    \int_{S_d} \mu(u) \exp \left[n\int_{S_d} \alpha(v) \log p(u,v)\,dv\right] \,du,
\end{equation}
which can be thought of splitting the $n$ item recommendations continuously across $S_d$ according to the density function $\alpha$. We can think of $\alpha$ as representing some ``profile'' of items to show. Going forward, we consider optimal distributions rather than optimal discrete sets of items.

\begin{proposition}
Consider a non-constant function $p(u,v): S_d \times S_d \rightarrow (0,1]$, interpreted as the probability that item $v$ does not satisfy a user with preference $u$, such that $p(u,v) = q(\norm{u-v})$ can be expressed as a function of the cosine distance between $u$ and $v$.
Then let $\mu$ be a probability measure with support $S_d$, representing the distribution of user preferences. Then for a probability measure $\alpha$ on $S_d$, define
\begin{equation}
    \Gamma(\alpha) = \lim_{n\rightarrow \infty}\frac{1}{n}\log \int_{S_d} \mu(u) \exp \left[n \int_{S_d} \alpha(v)\log p(u,v)\, dv\right]\,du.
\end{equation}
Then
\begin{equation}
    \Gamma(\pi) \in \inf_{\alpha} \Gamma(\alpha),
\end{equation}
where $\pi$ is the uniform probability measure over $S_d$.
\end{proposition}

\begin{proof}
Let
\begin{equation}
    \rho(u;\alpha) = \int_{S_d} \alpha(v)\log p(u,v)\,dv.
\end{equation}
Then
\begin{equation}
    \Gamma(\alpha) = \lim_{n\rightarrow\infty} \frac{1}{n} \log \int_{S_d} e^{n\rho(u;\alpha)} \mu(u)\, du = \lim_{n\rightarrow\infty} \frac{1}{n} \log \int_{S_d} e^{n\rho(u;\alpha)}\, du = \sup_{u\in S_d} \rho(u;\alpha),
\end{equation}
where the final equality follows from the Laplace principle from large deviations theory. Now note that
\begin{align}
    \int_{S_d} \rho(u;\alpha)\,du &= \int_{S_d} \int_{S_d} \alpha(v) \log p(u,v) \,dv\,du\\
    &= \int_{S_d} \alpha(v) \int_{S_d} \log p(u,v) \,du\,dv \\
    &= \int_{S_d} \alpha(v)C\, dv\\
    &= C,
\end{align}
where the second to last equality follows from the observation that
\begin{equation}
    \int_{S_d} \log p(u,v)\, du = \int_{S_d} \log q(\norm{u-v})\, du = C
\end{equation}
for a constant $C$ independent of $v$.
Therefore, there exists $u$ such that $\rho(u;\alpha) \ge \frac{C}{m(A)},$ so $\sup_{u\in A}\rho(u;\alpha) \ge \frac{C}{m(A)}.$ Now note that when $\pi$ is the uniform probability measure over $A$, we have
\begin{equation}
    \rho(u;\pi) = \int_{S_d} \pi(v) \log p(u,v) \,dv = \frac{C}{m(A)}
\end{equation}
for all $u$, so $\sup_{u\in A}\rho(u;\pi) = \frac{C}{m(A)}.$ So $\Gamma(\pi) = \inf_\alpha \Gamma(\alpha).$
\end{proof}
%!TEX root = ../main.tex

\newcommand{\Paughproj}{\lawP^{\mathrm{proj}}_{\mathrm{aug},h}}
\newcommand{\seqz}{\mathsf{z}}
\newcommand{\zst}{\seqz^\star}
\newcommand{\zstil}{\tilde{\seqz}^\star}

\newcommand{\phiZ}{\phi_{\cZ}}
\newcommand{\phiV}{\phi_{\cV}}
\newcommand{\seqv}{\mathsf{v}}





\section{Imitation in the Composite MDP}\label{sec:imit_composite}
In this section, we prove our imitation guarantees in the composite MDP under the full generality of data augmentation.  The majority of this section is devoted to proving  a more general version of \Cref{thm:smooth_cor} that applies to vectorized notions of distance and helps tighten our bounds when instantiated in the control setting.  In Appendix \ref{app:generalizationsmooth}, we introduce some notation and state our most general result, \Cref{thm:smooth_cor_general}.  We then proceed to show that \Cref{thm:smooth_cor} follows from \Cref{thm:smooth_cor_general} and in Appendix \ref{app:smoothcor_general_proof}, we provide a detailed and rigorous proof of the main result.  In Appendix \ref{app:smoothcor_proof}, we show that the more general \Cref{thm:smooth_cor_general} impiles \Cref{thm:smooth_cor} from the text.

Throughout, we  also assume $\cS$ admits a direct decomposition. This is useful to capture the fact that we only apply smoothing on the $\pathm$ coordinates (memory chunk), not the full trajectory chunk $\pathc$.  
\begin{definition}[Direct Decomposition]\label{defn:direct_decomp} Let $\cS = \cZ \oplus \cV$ is a direct decomposition. We let $\phiZ$ and $\phiV$ denote projections onto the $\cZ$ and $\cV$ components, respectively.  We say that the $\cS = \cZ \oplus \cV$ is \emph{compatible} with the dynamics if  $F_h((\seqz,\seqv),\seqa) = F_h((\seqz,\seqv'),\seqa)$ for all $\seqv, \seqv' \in \cV$ and $\seqz \in \cZ$, and \emph{compatible} with policy $\pi$ if $\pi_h((\seqz,\seqv),\seqa) = \pi_h((\seqz,\seqv'),\seqa)$.; we define compatibility of a kernel $\lawW$ and of a pseudometric $\dist(\cdot,\cdot): \cS \times \cS \to \R_{\ge 0}$ with $\cS = \cZ \oplus \cV$ similarly.
\end{definition}
We emphasize that compatibility of dynamics with a direct decomposition does not make $\seqv$ irrelevant because $\dists$ still depends on $\seqv$.  For the purposes of the instantiation for control in the following appendix, we wish to control the imitation gaps on distances that do depend on $\seqv_h$, even though $\seqv_h$ does not figure directly into the dynamics.  Note that as defined, $\seqv_h$ does depend on the dynamics up until time $h-1$ and thus it is necessary to deal with this component in order to provide guarantees in $\dists$.

\subsection{A generalization of Theorem \ref{thm:smooth_cor}}\label{app:generalizationsmooth}
\newcommand{\epsvec}{\vec{\epsilon}}
\newcommand{\distsvec}{\vec{\dist}_{\cS}}
\newcommand{\distsi}[1][i]{{\dist}_{\cS,#1}}
\newcommand{\distsone}{\distsi[1]}

\newcommand{\distai}[1][i]{{\dist}_{\cA,#1}}
\newcommand{\distavec}{\vec{\dist}_{\cA}}
\newcommand{\gapjointvec}{\vec\Gamma_{\mathrm{joint},\epsvec}}
\newcommand{\gapmargvec}[1][\epsvec]{\vec\Gamma_{\mathrm{marg},#1}}
\newcommand{\drobvec}[1][\epsvec]{\vec{\dist}_{\mathrm{os},#1}}

We now state a generalization of \Cref{thm:smooth_cor}, which replaces a single distance by a vector of distances of dimension $K$; this will be useful for our instantiation of the composite MDP as a chunked control system in our final application (in particular, for deriving a bound on $\Imitfin$). It also showcases the most general structure accomodated by our proof technique. 

We begin by defining some notation:
\begin{itemize}
\item Let $K \in \N$ denote a dimension
\item Let $\epsvec \in \R_{\ge 0}^K$ denote a vector of tolerances
\item Let $\distsvec(\cdot,\cdot)$ denote a vector of pseudometrics $\distsi$ on $\cS$
\item Let $\distavec$ denote a vector of non-negative functions $\distai:\cA^2 \to \R_{\ge 0}$, not necessarily pseuometrics.
\item Let $\preceq$ denote vector wise inequality, and let the symbols $\wedge$ and $\vee$ be generalized to denote entrywise minima and maxima.  Similarly, addition of vectors is coordinate wise with scalars assumed to be broadcast appropriately.
\item We let $\distsi[1] = \disttvc$ denote the metric we consider for evaluating total variation distance. 
\end{itemize} 
We generalize We assume the following measure-theoretic regularity conditions, generalizing \Cref{ass:polishspaces} as follows.
\begin{assumption} \label{ass:polish_spaces_general}
    We assume that $\cS$ and $\cA$ are Polish spaces. This means they are metrizable, but we do not annotate their metrics because, e.g. the metric on $\cS$ may be other than $\dists$. We further assume that 
\begin{itemize}
\item $\distsi$ is a pseudometric and Borel measurable function from $\cS \times \cS \to \R_{\ge 0}$. 
\item For any $\epsilon \ge 0$, the set $\{(\seqa,\seqa') \in \cA \times \cA : \distai(\seqa,\seqa') > \epsilon\}$ is an open subset of $\cA\times \cA$; i.e. $\distai(\cdot,\cdot)$ is lower semicontinuous. In particular, this means $\distai$ is a Borel measurable function. Note that this implies that the 
\begin{align}\{(\seqa,\seqa') \in \cA \times \cA : \distavec(\seqa,\seqa') \not \preceq \epsvec\}.
\end{align}
is closed and thus measurable.
\end{itemize}
\end{assumption}
Note that the above assumption is the natural vectorized generalization of \Cref{ass:polishspaces}.  Next, we define vector versions of our imitation errors.
\begin{definition}[Imitation Errors, vector version]\label{defn:imit_gaps_vec} Given error parameter $\epsvec \in \R_{\ge 0}^K$, define 
\begin{itemize}
\item The \bfemph{vector joint-error} 
\begin{align}
\gapjointvec(\polhat \parallel \pist) := \inf_{\coup_1}\Pr_{\coup_1}\left[\exists h \in [H]: \distsvec(\shat_{h+1},\sstar_{h+1}) \vee \distavec(\seqast_h,\seqahat_h)   \not \preceq \epsvec\right],
\end{align} 
where the infimum is over trajectory couplings $((\shat_{1:H+1},\seqahat_{1:H}),(\sstar_{1:H+1},\seqa^\star_{1:H})) \sim \coup_1 \in \couple(\Dist_{\polhat},\Dist_{\polst})$ satisfying $\Pr_{\coup_1}[\shat_{1} = \sstar_1] = 1$.   
\item The \bfemph{vector marginal error} 
\begin{align}
\gapmargvec(\polhat \parallel \pist) := \max_{h \in [H]}\max\left\{\inf_{\coup_1}\Pr_{\coup_1}\left[\distsvec(\shat_{h+1},\sstar_{h+1})\not \preceq \epsvec\right],\, \inf_{\coup_1}\Pr_{\coup_1}\left[\distavec (\seqast_h,\seqahat_h)\not \preceq \epsvec\right]\right\}
\end{align} the same as the to joint-gap, with the ``$\max$'' outside the probability and infimum over couplings. 
\item The \bfemph{vector-wise one-step error}  
\begin{align}
\drobvec(\polhat_h(\seqs) \parallel \polst_h(\seqs)) := \inf_{\coup_2}\Pr_{\coup_2}\left[\distavec(\seqahat_h,\seqast_h) \not  \preceq \epsvec \right],
\end{align} where the infimum is over $(\seqast_h, \hat \seqa_h) \sim \coup_2 \in \couple( \bpolhat_h(\seqs),\bpol_h^\star(\seqs))$.
\end{itemize} 
\end{definition}

We now describe input stability. 
\begin{definition}[Input-Stability, vector version] \label{defn:fis_vector} A trajectory $(\seqs_{1:H+1},\seqa_{1:H})$ is \bfemph{input-stable} w.r.t. $(\distsvec,\distavec)$ if  all sequences $\seqs_1' = \seqs_1$ and $\seqs_{h+1}' = F_h(\seqs_h',\seqa_h')$ satisfy  
\begin{align}\distsi(\seqs_{h+1}',\seqs_{h+1}) \le  \max_{1 \le j \le h}\distai\left(\seqa_{j}',\seqa_j\right) ,\quad \forall h \in [H], i \in [K]
\end{align}
\end{definition} 


Finally, define input process stability. A slight technicality is that, in our instantiation, $\pist$ is taken to be a suitable regular condition probability of the joint distribution $\Dexp$ of expert trajectories. This means that $\pist$ can only really satisfy desired regularity conditions on  states visited with positive probabiliy by $\Dexp$. We address this subtlety by considering the following definition generalizing \Cref{defn:ips_body} in the body. We also restrict the kernels under consideration to those which produce distributions \emph{absolutely continuous} (\Cref{defn:abs_cont}) with respect to $\Psth$, and denoted with the $\ll$ comparator. More specifically, we only care about absolute continuity under the projections onto the $\cZ$ component of $\cS$. 
\begin{definition}[Input \& Process Stability, vector version]\label{defn:ips_vec}
Let $\pips \in (0,1)$, $\gamipsvec = (\gamipsi)_{1\le i \le K}$ be a collection non-decreasing maps $\gamipsi:\R_{\ge 0} \to  \R_{\ge 0}$, let   $\distips:\cS \times \cS \to \R$ be a pseudometric (possibly other than any of the $\distsi$), and $\rips > 0$.  We say a policy $\pist$ is \emph{$(\gamipsvec,\distips,\rips,\pips)$-(vectorwise-input-\&-process stable (vIPS)} if the following holds for any $r \in [0,\rips]$: 

Consider any sequence of kernels $\lawW_h:\cS \to \laws(\cS)$, $1\le h \le H$, satisfying 
\begin{align}
\forall h, \seqs \in \cS: \quad \Pr_{\tilde \seqs\sim \lawW_h(\seqs)}[\distips(\tilde \seqs,\seqs) \le r] = 1, \quad \phiZ \circ \lawW_h(\seqs) \ll \phiZ \circ \Psth. \label{eq:supp_contained}
\end{align}
Define a process $\seqs_1 \sim \Dinit$, $\tilde\seqs_h \sim \lawW_h(\seqs_h),\seqa_h \sim \pi_h(\tilde \seqs_h)$, and $\seqs_{h+1} := F_h(\seqs_h,\seqa_h)$. Then, with probability at least $1- \pips$,
\begin{itemize}
\item[(a)] the sequence $(\seqs_{1:H+1},\seqa_{1:H})$ is input-stable w.r.t $(\distsvec,\distavec)$ (as defined by \Cref{defn:fis_vector}).
\item[(b)]$\max_{h \in [H]} \distsi(F_h(\tilde\seqs_h,\seqa_h),\seqs_{h+1}) \le \gamipsi(r)$. 
\end{itemize}
\end{definition}
\newcommand{\epsvecmarg}{\epsvec_{\mathrm{marg}}}




We can now state our desired generalization. 



\begin{theorem}\label{thm:smooth_cor_general}   Suppose that there 
\begin{itemize}
\item[(a)]$\pist$ is $(\gamipsvec,\distips,\rips,\pips)$-vector IPS in the sense of \Cref{defn:ips_vec}.
\item[(b)] There is a direct decomposition of $\cS = \cZ \oplus \cV$, which associated projection maps $\phiZ$ and $\phiV$, and which is compatible with the dynamics, and policies $\pist$, $\pihat$, and smoothing kernel $\Wsig$, and $\distips$.
\item[(c)]  $\phiZ \circ \Wsig$ is $\gamma_{\sigma}$-TVC with respect to the pseudometric $\disttvc = \distsone$. 
\end{itemize} 
Let $\pihatsig$ be any policy which is $\gamhat$-TVC, also w.r.t. $\disttvc = \distsone$. Finally, let $\epsvec \in \R_{\ge 0}^K$, $r \in (0,\frac{1}{2}\rips]$, and define 
\begin{align}
p_r &:= \sup_{\seqs}\Pr_{\seqs' \sim \Wsig(\seqs)}[\distips(\seqs',\seqs) >  r], \quad \epsvecmarg := \epsvec + \gamipsvec(2r).
\end{align} Then, 
\begin{itemize}
\item For any policy $\pihat$,  both  $\gapjointvec (\pihatsig  \parallel \pistrep)$ and  $\gapmargvec[\epsvecmarg] (\pihatsig \parallel \pist)$ are upper bounded by%$\gapmarg[\epsilon + 2r](\pihat \circ \Wsig \parallel \pist)$ are both at most
\begin{align}
%\inf_{r > 0}  
\pips + H(2p_r + \gamhat(\epsvec_1) + (\gamhat + \gamtvcsig) \circ \gamipsone(2r))  + \sum_{h=1}^H\Exp_{\sstar_h \sim \Psth}\drobvec\,( \pihatsigh(\stel_h) \parallel \pistreph(\stel_h)) \label{eq:smooth_ub_app_one}
\end{align}
\item In the special case where $\pihatsig = \pihat \circ \Wsig$, we can take $\gamhat = \gamsig$, and obtain that $\gapjointvec(\pihatsig \parallel \pistrep)$ and $\gapmargvec[\epsvecmarg](\pihatsig \parallel \pist)$ are upper bounded by
\begin{align}
\pips + H\left(2p_r +  3\gamma_{\sigma}(\max\{\epsilon,\gamipsone(2r)\}\right)  + \textstyle \sum_{h=1}^H\Exp_{\sstar_h \sim \Psth}\Exp_{\sstartil_h \sim \Wsig(\sstar_h) } \drobvec( \pihat_{h}(\sstartil_h) \parallel \pidec(\sstartil_h)) . \label{eq:smooth_ub_app_two}
\end{align}
\end{itemize}
\end{theorem}
We note that \Cref{thm:smooth_cor} is  a special case of \Cref{thm:smooth_cor_general} and prove the former assuming the latter here at the end of the section.

\subsection{Proof of Theorem \ref{thm:smooth_cor_general} }\label{app:smoothcor_general_proof}



\subsubsection{Proof Overview and Coupling Construction}\label{sec:proof_construction}
We begin with an intuitive overview of the proof and partially construct the relevant intermediate trajectories used to define our coupling, after which we sketch the organization of the rest of Appendix \ref{app:smoothcor_general_proof}.

The proof proceeds by constucting a sophisticated coupling between the law of a trajectory evolving according to $\pihat$ and a trajectory evolving according to $\pistrep$ by introducing several intermediate sequences of composite states and composite actions.  

We partially specify this coupling below and formally construct it in Appendix \ref{app:proof_smooth_cor_general}.  Our construction is recursive and relies on the input and process stability as well as total variation continuity to show that if the trajectories generated by $\pistrep$ and $\pihat$ are close in $\drobvec[\epsvec]$ evaluated on states at step $h$, then they will remain close at step $h+1$.  There are a number of technical subtelties involved, especially those of a measure-theoretic nature, but much of the inuition can be gleaned from the following partial specification of the coupling $\coup$ over composite-state 
$(\shat_{1:H},\srep_{1:H},\stel_{1:H},\ssq_{1:H}) \subset \cS$, composite-actions  $(\arep_{1:H},\seqahat_{1:h},\atel_{1:H}) \subset \cK$ and interpolating composite-actions, $(\arepinter_{1:H},\atelinter_{1:H}) \subset \cA$. 

To define the construction, we define the probability kernels corresponding to the replica and deconvolution policies.  Note that these are slightly different from the definitions in the body due to the use of the direct decomposition; the intuition is the same, however.

\newcommand{\QdechZ}[1][h]{\lawW^{\star}_{\mathrm{dec},\cZ,h}}

\begin{definition}[Replica and Deconvolution Kernels]\label{defn:all_kernels} Let $\Paughproj$denote the joint distribution over $(\zst_h,\sstar_h,\zstil_h,\astar_h)$ under the generative process
\begin{align}
\sstar_h \sim \Psth, \quad \astar_h \sim \pist_h(\sstar_h), \quad 
\zst_h = \phiZ(\sstar_h), \quad \zstil_h \sim \phiZ\circ \Wsig(\sstar_h)
\end{align}
For $\seqz \in \cZ$, let $\QdechZ(\seqz)$ denote the distribution of $\zst_h$ conditioned on $\zstil_h = \seqz$, under $\Paughproj$. Given $\seqs = (\seqz,\seqv)$, define 
\begin{align}
&\Qdech(\seqs) = \QdechZ(\phiZ(\seqs)) \otimes \dirac_{\phiV(\seqs)}, \quad \\
&\Qreph(\seqs) = \Qdech \circ ( \Wsig(\phiZ(\seqs))\otimes \dirac_{\phiV(\seqs)}) =   (\QdechZ \circ \Wsig(\phiZ(\seqs)))\otimes \dirac_{\phiV(\seqs)}.
\end{align}
where we recall the dirac-delta $\dirac$. Equivalently, $\Qdech(\seqs)$ denotes the conditional sequence of $(\tilde \seqz,\seqv)$, where $\seqv = \phiV(\seqs)$, and $\tilde \seqz \sim \QdechZ(\seqs)$; $\Qreph$ can be expressed similarly. 
\end{definition}
We remark that $\Qdech$ and $\Qreph$ are both kernels and by \Cref{thm:durrett}, we may assume that the joint distribution over $(\sstar_h, \ssq_h)$ admits a regular conditional probability and thus these constructions are well-defined. 
\begin{remark}Note that the kernels $\Qdech$ and $\Qreph$ are  compatible with the decomposition $\cS = \cZ \oplus \cV$ by construction. Moreover, note that if $\seqs = (\seqz,\seqv)$, $\phiV \circ \Qdech(\seqs) = \phiV \circ \Qreph(\seqs)$ is the dirac-delta distribution supported on $\seqv$.
\end{remark}
\begin{lemma} Under our the assumption that $\pist$ and $\Wsig$ are compatible with the direct decomposition,  
\begin{align}
\pidech(\seqs) = \pist \circ \Qdech , \quad \pistreph(\seqs) = \pist \circ \Qreph 
\end{align}
\end{lemma}
\begin{proof} This follows imediately because $\pist$ and $\Wsig$ are  compatile with the direct decomposition, and by the definition of \Cref{defn:body_replica}.
\end{proof}


% Figure environment removed
\paragraph{A template for the coupling.} Our couplings are partially specified by the following generative process, and what remains unspecified are couplings between random variables at each each step $h$. In what follows, let $\cF_0$ denote the $\upsigma$-algebra generatived by $\shat_1 = \srep_1 = \stel_1 $. Let $\cF_h$ denote the sigma-algebra generated by  $(\shat_{1:h},\srep_{1:h},\stel_{1:h})$, $(\arep_{1:h},\sreptil_{1:h},\ssq_{1:h},\atel_{1:h},\seqahat_{1:h})$, and $(\arepinter_{1:h},\atelinter_{1:h})$.
\begin{itemize}
    \item The initial states are drawn as
    \begin{align}
    \shat_1 = \srep_1 = \stel_1 \sim \Dinit. 
    \end{align}
    \item The dynamics satisfy
    \begin{align}
    \shat_{h+1} = F_h(\shat_h,\seqahat_h), \quad \srep_{h+1} = F_h(\srep_h,\arep_h), \quad \stel_{h+1} = F_h(\ssq_h,\atel_h)
    \end{align}
    Note that determinism of the dynamics implies that $\stel_{h+1}$, $\srep_{h+1}$ and $\shat_{h+1}$ are $\cF_{h}$-measurable. 
    \item We generate
    \begin{align}
    &\sreptil_h \mid \cF_{h-1} \sim \Qreph(\srep_h), \quad \arep_h \mid \cF_{h-1},\sreptil_h \sim \pisth(\sreptil_h), \qquad \label{eq:trajevolve1} \\
    &\ssq_h \mid \cF_{h-1} \sim \Qreph(\stel_h), \quad \atel_h \mid \cF_{h-1},\ssq_h \sim \pisth(\ssq_h).\label{eq:trajevolve2}\\
    &\seqahat_h \mid \cF_{h-1} \sim \pihatsigh(\shat_h) \label{eq:trajevolve_ahat}
\end{align}
Importantly, we note that, marginalizing over $\ssq_h$ and $\sreptil_h$, respectively, $\atel_h \mid \cF_{h-1} \sim \pistreph(\stel)$ and $\arep_h \mid \cF_{h-1} \sim \pistreph(\srep_h)$.  
\item Lastly, we select interpolating actions via
\begin{align}
    &\arepinter_h \mid \cF_{h-1} \sim \pihatsigh(\srep_h), \qquad \atelinter_h \mid \cF_{h-1} \sim \pihatsigh(\stel_h)\label{eq:trajevolve3}
\end{align}
\end{itemize}
We will say $\coup$ is ``respects the construction'' as shorthand to mean that $\coup$ obeys the above equations.  The coupling is illustrated graphically in \Cref{fig:coupling_illustration}.  We now establish several key properties of the above constructions, separated into a subsection for the sake of clarity.


\paragraph{Organization of the remaining parts of Appendix \ref{app:smoothcor_general_proof}.}   In Appendix \ref{app:prop_of_deconv_replica}, we prove several prerequisite properties of the construction given above, including concentration of the smoothing kernel, and key properties of the replica distribution. Next, Appendix \ref{app:marg_imit_gap} shows that, due to these properties of the replica distribution, we can bound the marginal imitation gap by controlling the tracking of the teleporting sequence constructed above. Finally, in Appendix \ref{app:proof_smooth_cor_general} we formally construct the coupling and rigorously prove \Cref{thm:smooth_cor_general}.
\begin{comment}
\begin{observation} Let $\trajhat = (\shat_{1:H},\seqahat_{1:H})$, $\trajrep_{1:H} = (\sstar_{1:H},\seqast_{1:H})$, $\trajtel = (\stel_{1:H},\atel_{1:H})$.
\begin{itemize}
\item $\coup$ is an interpolating construction for $(\trajrep,\trajhat,\arepinter_{1:H})$ with respect to $(\pistrep,\pihat,(\cF_{h})_{h \ge 0})$.
\item $\coup$ is a teleporting construction for $(\trajtel,\trajrep,\ssq_{1:H})$ with respect to $(\pist,\Wsig,(\cF_h)_{h \ge 0})$. 
\end{itemize}
\end{observation}
\end{comment}




\newcommand{\Ctelh}[1][h]{\cC_{\mathrm{tel} ,#1}}
\newcommand{\Crephath}[1][h]{\cC_{ \hat{\seqs},#1}}
\newcommand{\Binterh}[1][h]{\cB_{ \mathrm{inter},#1}}
\newcommand{\Bhath}[1][h]{\cB_{\hat{\seqa},#1}}
\newcommand{\Btelh}[1][h]{\cB_{\mathrm{tel},#1}}
\newcommand{\Bfsh}[1][h]{\cB_{\mathrm{est},#1}}
\newcommand{\Callh}[1][h]{\cC_{\mathrm{all},#1}}
\newcommand{\Callbarh}[1][h]{\bar\cC_{\mathrm{all},#1}}
\subsubsection{Properties of smoothing, deconvolution, and replicas.}\label{app:prop_of_deconv_replica}

In this section, we establish several useful properties of smoothed and replica policies.  We begin by showing that smoothed policies are TVC.
\begin{lemma}\label{lem:pistrep_tvc}
The following hold
\begin{itemize}
    \item For any $h$, $\phiZ \circ \Qreph$ and $\pistreph$ are $\gamma_{\sigma}$ TVC.
    \item If $\pi$ is any policy compatible with the direct decomposition $\cS = \cZ \oplus \cV$ (in the sense of \Cref{defn:direct_decomp}), then $\pi\circ \Wsig$ is $\gamma_{\sigma}$-TVC.
\end{itemize}
\end{lemma}
\begin{proof} We observe that $\phiZ \circ \Qreph = \phiZ \circ \Qdech \circ \Wsig(\seqs)$. Moreover, we observe $\Qdech$ satisfies  $\phiZ \circ \Qdech(\seqs) =  \QdechZ \circ \phiZ$, so that $\phiZ \circ \Qreph = \QdechZ \circ \phiZ \circ \Wsig(\seqs)$. As $\phiZ \circ \Wsig$ is TVC, the first claim is a consequence of the data-processing inequality \Cref{cor:tv_two}. The second uses the fact that all listed objects involve composition of kernels with $\Wsig$.
\end{proof}
Next, we show that the replica construction preserves marginals. 
\begin{lemma}[Marginal-Preservation]\label{lem:replica_property} 
 There exists a coupling $\Pr$ of $\seqz_h \sim \phiZ \circ \Psth$, $\seqz_h' \sim \phiZ\circ\Wsig(\seqz_h,\cdot)$ (where ($\cdot$) denotes an irrelevant argument due to compatibility of $\Wsig$ with the direct decomposition), and $\tilde \seqz_h \sim \phiZ \circ \Qreph(\seqz_h,\cdot)$ (again, ($\cdot$) denotes an irrelevant argument) such that 
 \begin{align}
 (\seqz_h,\seqz_h') \overset{\mathrm{d}}{=}  (\tilde\seqz_h,\seqz_h').
 \end{align}
 In particular, for $\stel_h$ and $\ssq_h$ as in our construction, the marginal distributions of $\phiZ(\stel_h)$ and $\phiZ(\ssq_h)$ are the same, where $\stel_h \sim \Psth$ and  $\ssq_h \mid \stel_h \sim \Qreph(\stel_h)$.
\end{lemma}
\begin{proof}
    By \Cref{ass:polishspaces} and \Cref{thm:durrett}, we may assume that all joint distributions' conditional probabilities are regular conditional probabilities and thus almost surely equal to a kernel.  Moreover, since all kernels are compatible with the direct decomposition, it suffices to prove the special case of the trivial direct-decomposition where $\cZ = \cS$.  Fix a common measure $\pp$ over which $\stel_h, \ssq_h$, and $\mathsf{s}_h'$ are defined such that $\stel_h \sim \Psth$, $\mathsf{s}_h' \sim \Wsig(\stel_h)$, and $\ssq_h \sim \Wdeconvh(\mathsf{s}_h')$. Then for any measurable sets $A, B$, we have
    \begin{align}
        \pp(\stel_h \in A,\, \seqs_h' \in B) &= \pp(\seqs_h' \in B) \cdot \ee_{\seqs_h'}\left[\I[\seqs_h' \in B] \cdot \pp(\stel_h \in A | \seqs_h' ) \right] \\
        &= \pp(\seqs_h' \in B) \cdot \ee_{\seqs_h'}\left[\I[\seqs_h' \in B] \cdot \pp(\ssq_h \in A | \seqs_h' ) \right]\\
        &= \pp\left( \ssq_h \in A, \, \seqs_h' \in B \right),
    \end{align}
    where the first equality holds by the fact that we are working with regular conditional probabilities and Bayes' rule, the second equality holds by the definition of the deconvolution kernel above, and the last equality holds again by Bayes' rule and the tower rule for conditional expectations.

    To prove the second statement, we apply induction, again assuming that $\cZ = \cS$ as in the proof of the first statement.  Note that $\stel_1 \sim \Psth[1] = \Dinit$, and $\ssq_1 \sim \Qreph[1] \circ \Psth[1]$. Thus, from the first part of the lemma, $\phiZ (\stel_1) \sim \phiZ \circ \Psth[1]$. Now, suppose the induction holds up to step $h$. Then, $\ssq_h \sim \Psth$, as $\atel_h \sim \pist_h(\atel_h)$, then $\stel_{h+1} = F_{h}(\ssq_h,\atel_h) \sim \Psth[h+1]$. Again  $\ssq_{h+1} \sim \Qreph[h+1](\stel_{h+1})$, so that $\ssq_{h+1}$ has marginal $\Qreph[h+1]\circ \Psth[h+1] = \Psth[h+1]$, as needed.  
\end{proof}
We further show that $\Wreph$ can be defined to be absolutely continuous with respect to $\Psth$.
\begin{lemma}\label{lem:absolute_continuity}
    The kernel $\Wreph$ satisfies that $\phiZ \circ \Wreph \ll \phiZ \circ \Psth$ as laws, validating the second condition in \eqref{eq:supp_contained}.  It further holds that $\phiZ \circ \Wdeconvh \ll \phiZ \circ \Psth$.
\end{lemma}
\begin{proof}
    The first statement follows immediately from \Cref{lem:replica_property} because these distributions are the same.  The second statement follows immediately from the tower law of conditional expectation and the definition of $\Wdeconvh$.
\end{proof}

Lastly, we establish that the replica kernel inherits all concentration properties from the smoothing kernel.
\begin{lemma}[Replica Concentration]\label{lem:rep_conc} Recall that 
\begin{align}
p_r := \sup_{\seqs}\Pr_{\seqs' \sim \Wsig(\seqs) }[\distips(\seqs',\seqs) >  r].
\end{align} We then have
\begin{align}
\Pr_{\seqs_h \sim \Psth,\stil_h \sim \Qreph(\seqs_h)}[\distips(\stil_h,\seqs_h) > 2\rsmooth] \le 2p_r \label{eq:concentration_conv_two}
\end{align}
\end{lemma}
\begin{proof} %We fix a common measure $\Pr[\cdot]$ over $\seqs_h \sim \Psth,\stil_h \sim \Qreph(\seqs_h)$
Again, all terms -- $\Wsig,\Qreph,\Qdech$ and $\distips$ -- are compatible with the direct decomposition, it suffices to consider the case of the trivial direct decomposition under whcih $\cZ = \cS$.

Let $\Pr$ denote a distribution over $\seqs_h \sim \Psth$, $\seqs_h' \sim \Wsig(\seqs_h)$, and $\stil_h \sim \Qdech(\seqs_h')$. In this special case,  we see that $\stil_h \mid \seqs_h \sim \Qreph(\seqs_h)$\footnote{Notice that, for general $\cS = \cZ \oplus \cV$, this condition would become $\phiZ(\stil_h) \mid \phiZ(\seqs_h) \sim \phiZ \circ \Qreph(\phiZ(\seqs_h),\cdot)$, where the $\cdot$ argument is irrelevant.}. By a union bound,
\begin{align}\label{eq:dists_conv_bound_two}
\Pr_{\seqs_h \sim \Psth,\stil_h \sim \Qreph(\seqs_h)}[\distips(\seqs_h,\stil_h) > 2\rsmooth] &\le \Pr[\distips(\stil_h,\seqs'_h) > \rsmooth]  + \Pr[\distips(\seqs_h,\seqs'_h) > \rsmooth] \\
&= 2 \Pr[\distips(\seqs_h,\seqs'_h) > \rsmooth] \le 2p_r,
\end{align}
where the equality follows from the first statment of \Cref{lem:replica_property}.
\end{proof}
\begin{remark}Note that, in the previous lemma, it suffices that the following weaker condition holds: $\Pr_{\seqs \sim \Psth,\seqs' \sim \Wsig(\seqs)}[\distips(\seqs',\seqs) >  \rsmooth] \le p_r$, i.e. for concentration to hold only in distribution over $\seqs \sim \Psth$, instead of \emph{uniformly} over states.
\end{remark}
\newcommand{\Qtilreph}[1][h]{\tilde{\lawW}_{\repsymbol,#1}}


\subsubsection{Bounding the marginal imitation gaps in terms of the teleporting sequence error}\label{app:marg_imit_gap}
Before turning to the proof of \Cref{thm:smooth_cor_general}, we verify that closeness to the \emph{teleporting sequences} suffices to control error in marginal gap to $\pist$. The key property here is that the teleporting sequence, as shown in \Cref{lem:replica_property}, has the same marginal distribution over states as does $\pist$.

\begin{lemma}\label{lem:marg_imit_gap_tel} Let $\coup$ be any coupling obeying the construction of the couplings above. Then, 
\begin{align}
\gapmargvec(\pihatsig \parallel \pist) \le 
\Pr_{\coup}\left[\exists h \in [H]: \left\{\distsvec(\stel_{h+1},\shat_{h+1}) \not \preceq \epsvecmarg\right\} \cup \left\{ \distavec(\atel_h,\ahat_h) \not \preceq \epsvecmarg\right\}\right] 
\end{align}
\end{lemma}
\begin{proof}
 We begin with a (reverse) union bound.
\begin{align}
&\Pr_{\coup}\left[\exists h \in [H]: \left\{\distsvec(\stel_{h+1},\shat_{h+1}) \not \preceq \epsvecmarg\right\} \cup \left\{ \distavec(\atel_h,\ahat_h) \not \preceq \epsvecmarg\right\}\right] \\
&\ge\max_h\max\left\{\Pr_{\coup}\left[\distsvec(\stel_{h+1},\shat_{h+1}) \not \preceq \epsvecmarg\right],\, \Pr_{\coup}\left[\distavec(\atel_h,\ahat_h) \not \preceq \epsvecmarg\right]\right\}.
\end{align}
 By \Cref{lem:replica_property} implies that $\stel_h$ has the marginal distribution of $\sstar_h \sim \Psth$. Moreover, by construction, for each $h$, $\atel_h \mid \cF_h \sim \pistreph(\stel_h)$, Thus, for each $h$, $\stel_{h+1}$ and $\atel_h$ have the same \emph{marginals} as the marginals as $\sstar_{h+1}$ and $\astar_h$ under the distribution $\Dist_{\pist}$ induced by $\pist$. Hence, 
 \begin{align}
 \Pr_{\coup}\left[\distsvec(\stel_{h+1},\shat_{h+1}) \not \preceq \epsvecmarg\right] &\ge \inf_{\coup_1} \Pr\left[\distsvec(\sstar_{h+1},\shat_{h+1}) \not \preceq \epsvecmarg\right] \\
 \Pr_{\coup}\left[\distavec(\atel_{h},\ahat_{h}) \not \preceq \epsvecmarg\right] &\ge \inf_{\coup_1} \Pr\left[\distsvec(\astar_{h},\ahat_{h}) \not \preceq \epsvecmarg\right],
 \end{align}
 where the $\inf_{\coup_1}$ is, as  in \Cref{defn:imit_gaps,defn:imit_gaps_vec}, the infinum over couplings between $\Dist_{\pist}$ and $\Dist_{\pihat}$. Thus, 
 \begin{align}
&\Pr_{\coup}\left[\exists h \in [H]: \left\{\distsvec(\stel_{h+1},\shat_{h+1}) \not \preceq \epsvecmarg\right\} \cup \left\{ \distavec(\atel_h,\ahat_h) \not \preceq \epsvecmarg\right\}\right] \\
&\ge\max_h\max\left\{\inf \Pr_{\coup_1}\left[\distsvec(\sstar_{h+1},\shat_{h+1}) \not \preceq \epsvecmarg\right],\, \inf_{\coup}\Pr_{\coup_1}\left[\distavec(\astar_h,\ahat_h) \not \preceq \epsvecmarg\right]\right\}\\
&:= \gapmargvec(\pihatsig \parallel \pist).
\end{align}

\end{proof}

\subsubsection{Formal proof of Theorem \ref{thm:smooth_cor_general}}\label{app:proof_smooth_cor_general}
We now proceed to formally prove \Cref{thm:smooth_cor_general}

\paragraph{Key Events. } For the random variables defined above, we define three groups of events. 
\begin{itemize}
\item The \emph{coupling events}, denoted by $\cB$, which are controlled by carefully selecting a coupling.
\item The \emph{inductive events}, denoted by $\cC$, which we condition on when bounding the probability of the coupling events.
\item The \emph{stability events}, denoted by $\cQ$, which take advantage of the stability properties of the imitation policy. 
\end{itemize}
\newcommand{\Ballbarh}[1][h]{\bar\cB_{\mathrm{all},#1}}
\newcommand{\Ballh}[1][h]{\cB_{\mathrm{all},#1}}
\newcommand{\Qis}{\cQ_{\textsc{is}}}
\newcommand{\Qips}{\cQ_{\textsc{ips}}}
\newcommand{\Qclose}{\cQ_{\mathrm{close}}}
\newcommand{\Qall}{\cQ_{\mathrm{all}}}

\begin{definition}[Coupling Events]\label{defn:all_key_eents} Define the events
\begin{align}
    \Btelh &=  \left\{ \arep_h = \atel_h, ~\phiZ(\sreptil_h) = \phiZ(\ssq_h) \right\}\\
     \Bfsh &= \left\{ \distavec( \atelinter_h,\atel_h) \not \preceq \epsvec \right\} \\
    \Binterh &= \left\{ \atelinter_h = \arepinter_h  \right\} \\
    \Bhath &= \left\{  \arepinter_h = \seqahat_h \right\} \\
     \Ballh &= \Binterh \cap \Btelh \cap \Bfsh \cap \Bhath\\
    \Ballbarh &=  \bigcap_{j=1}^h \Ballh[h]
\end{align}
Notice that each of the events above are $\cF_{h}$-measurable. Moreover, note that on $\Ballbarh$, $\max_{1\le j \le h}\phiis(\seqahat_j,\arep_j) \le \epsilon$.
\end{definition}
\begin{definition}[Inductive Event]\label{def:inductive_event}

Define the events
\begin{align}
\Crephath &= \left\{  \distsvec(\srep_h, \shat_h) \preceq \epsvec \right\}, \\
\Ctelh &= \left\{  \distsvec(\srep_h, \stel_h) \preceq \gamipsvec(2r) \right\} \\
\Callh &:= \Crephath \cap \Ctelh\\
 \Callbarh &=  \bigcap_{j=1}^h \Callh[j]
\end{align}
Notice that all the above events are $\cF_{h-1}$-measurable, due to determinism of the dynamics. Note that also $\Pr_{\coup}[\Callbarh[1]] = 1$ for any $\coup$ that respects the construction (as $\srep_1 = \stel_1 = \shat_1$).
\end{definition} 
\begin{definition}[Stability Events] Define the events 
\begin{align} 
\Qclose &:= \left\{\forall h \in [H]: \distips(\srep_h,\sreptil_h) \le 2r \right\}\\
\Qis &:= \left\{(\srep_{1:H+1},\arep_{1:H}) \text{ is input-stable w.r.t. } (\distsvec,\distavec)\right\}\\
\Qips &:= \left\{\distsvec(F_h(\sreptil_{h},\arep_h),\srep_{h+1}) \le  \gamipsvec\circ \distips\left(\sreptil_h,\srep_{h}\right), \quad 1 \le j \le H\right\} \\
\Qall &:=  \Qips \cap \Qclose .
\end{align}
In words, $\Qclose$ the event on which $\srep_h$ and $\sreptil_h \sim \Qreph(\stel_h)$ are close, and $\Qis$and  $\Qips$ ensure consequencs of  (vector) input-stability and (vector) input process stability holds.
\end{definition}

\paragraph{Steps of the proof.}
First, we use stability to reduce the event $\Callbarh[h+1]$ to $\Callbarh \cap \Ballbarh$:
\begin{claim}[Stability Claim]\label{claim:stability_claim} By construction, 
\begin{align}\Callbarh[h+1] \subset \Qall \cap \Callbarh \cap \Ballbarh.
\end{align}
\end{claim} 
\begin{proof} It suffices to show that on $\Qall \cap \Callbarh \cap \Ballbarh$, $ \distsvec(\srep_{h+1},\shat_{h+1}) \preceq \epsvec$ and $\distsvec(\srep_{h+1},\stel_{h+1}) \preceq \gamipsvec(2r)$. By applying the event $\Qis$ to the sequence $\seqa'_h = \seqahat_h$ and $\seqs'_h = \shat_h$, we have that on $\Qall \subset \Qis$ that
\begin{align}
 \forall h \in [H], i \in [K], \quad \distsi(\srep_{h+1},\shat_{h+1}) \le  \max_{1 \le j \le h}\distai\left(\arep_j,\seqahat_{j}\right) \label{eq:Qis_consequence}
\end{align}


For the next point, note that the compatibility of the dynamics with the direct decomposition $\cS = \cZ \oplus \cV$ implies that there exists a dynamics map $F_h^{\cZ}$  for which 
\begin{align}
F_h(\seqs,\seqa) = F_h^{\cZ}(\phiZ(\seqs),\seqa).
\end{align}
Similarly, by applying $\Qips$ and $\Qclose$ and the event $\{\phiZ(\sreptil_h) = \phiZ(\ssq_h),\atel_h = \arep_h\}$ on $\Btelh$, it holds that on $\Qall \cap \Callbarh \cap \Ballbarh$ that, for all $h \in [H]$,
\begin{align}
\distsvec(\srep_{h+1},F_h(\sreptil_{h},\arep_h))  &= \distsvec(\srep_{h+1},F_h^\cZ(\phiZ(\sreptil_{h}),\arep_h)) \\
&= \distsvec(\srep_{h+1},F_h^\cZ(\phiZ(\ssq_{h}),\atel_h)) \tag{$\Btelh$}\\
&= \distsvec(\srep_{h+1},F_h(\ssq_{h},\atel_h)) \\
&= \distsvec(\srep_{h+1},\stel_{h+1})\\ 
&\le \gamipsvec\circ\distips\left(\stel_j,\ssq_{j}\right) \tag{$\Qips$}\\
&\le \gamipsvec\circ\distips\left(2r\right) \tag{$\Qclose$}.
\end{align}
\end{proof}
From \Cref{claim:stability_claim}, we decompose our error probability as follows:
\begin{lemma}[Key Error Decomposition] \label{lem:putting_couplings_together} Let $\coup$ respect the construction (in the sense of \Cref{sec:proof_construction}). Then, for any coupling $\coup$ which respects the construction,
\begin{align}
&\gapjointvec(\pihatsig \parallel \pirep) \vee \gapmargvec(\pihatsig \parallel \pist) \le \Pr_{\coup}[\Qall^c] + \sum_{h=1}^H\Pr_{\coup}[ \Ballbarh^c \cap \Callbarh \cap \Ballbarh[h-1]]\label{eq:Gamimit_decomp}
\end{align}
\end{lemma}
\begin{proof} In what follows, we use $\vec{v} \vee \vec{w}$ to denote the entrywise maximum of two vectors of the same dimension. Define the events $\cE_h := \Callbarh[h+1] \cap \Ballbarh$. Observe that the events are nested: $\cE_{h} \supset \cE_{h+1}$, and that on $\cE_H$, we have that for all $h \in [H]$
\begin{align}
\distsvec(\srep_{h+1},\shat_{h+1}) \vee \distavec(\arep_h,\ahat_h) &\preceq \epsvec \vee \distavec(\arep_h,\ahat_h) \tag{$\Crephath[h+1] \supset \Callbarh[h+1] \supset \cE_h$}\\
&\preceq \epsvec \tag{$\Ballbarh \supset \cE_h$}.
\end{align}
On $\Qall \cap \cE_H$, we have that
\begin{align}
\max_h \distsvec(\srep_h,\stel_h) \le \gamipsvec(2r), \quad \text{and}\quad \atel_h = \arep_h
\end{align}
Thus, by the triangle inequality and $\epsvecmarg = \epsvec + \gamipsvec(2r)$, on $\Qall \cap \cE_H$,
\begin{align}
\max_h \distsvec(\srep_h,\stel_h) \le \epsvecmarg, \quad \text{and}\quad  \distavec(\atel_h,\ahat_h)  =\distavec(\arep_h,\ahat_h)  \le \epsvec \le \epsvecmarg.
\end{align}
Thus, 
\begin{align}
&\Pr_{\coup}\left[\exists h \in [H]: \left\{\distsvec(\srep_{h+1},\shat_{h+1}) \vee \distavec(\arep_h,\ahat_h) \not \preceq \epsvec\right\} \cup \left\{\distsvec(\stel_{h+1},\shat_{h+1}) \vee \distavec(\atel_h,\ahat_h) \not \preceq \epsvecmarg\right\}\right]\\
&\le \Pr_{\coup}[(\Qall \cap \cE_H)^c] \label{eq:first_pr_coup_big}
\end{align}
In particular, this shows that
\begin{align}
\gapjointvec(\pihatsig \parallel \pirep)  \le \Pr_{\coup}[(\Qall \cap \cE_H)^c], 
\end{align}
and similarly, by \Cref{lem:marg_imit_gap_tel},
\begin{align}
\gapmargvec(\pihatsig \parallel \pist) \le \Pr_{\coup}[(\Qall \cap \cE_H)^c]
\end{align}
As $(\srep_{1:H+1},\arep_{1:H}) \sim \Dist_{\pistrep}$, \eqref{eq:first_pr_coup_big} shows that
\begin{align}\gapjointvec(\pihatsig \parallel \pirep) \vee \gapmargvec(\pihatsig \parallel \pirep)\le \Pr_{\coup}[(\Qall \cap \cE_H)^c].
\end{align} 
Let us conclude by bounding $\Pr_{\coup}[(\Qall \cap \cE_H)^c]$. Using the nesting structure $\cE_h = \bigcap_{j=1}^h \cE_j$, the peeling lemma, \Cref{lem:peeling_lem}, and a union bound, it holds that
\begin{align}
\Pr_{\coup}\left[(\Qall \cap \cE_H)^c\right] &\le \Pr_{\coup}[\Qall^c] + \Pr\left[ \exists h \in [H] \text{ s.t. } \left(\Qall \cap \cE_{h-1} \cap \cE_h^c\right) \text{ holds } \right ]\\
&\le \Pr_{\coup}[\Qall^c] + \sum_{h=1}^H\Pr_{\coup}\left[ \Qall \cap \cE_{h-1} \cap \cE_h^c \text{ holds } \right ]\\
&= \Pr_{\coup}[\Qall^c] + \sum_{h=1}^H\Pr_{\coup}\left[ \Qall \cap \Ballbarh[h-1] \cap \Callbarh[h]  \cap (\Ballbarh \cap \Callbarh[h+1])^c \text{ holds } \right ]\\
&= \Pr_{\coup}[\Qall^c] + \sum_{h=1}^H\Pr_{\coup}\left[ \Qall \cap \Ballbarh[h-1] \cap \Callbarh[h]  \cap \Ballbarh^c\right ]\\
&= \Pr_{\coup}[\Qall^c] + \sum_{h=1}^H\Pr_{\coup}\left[ \Qall \cap \Ballbarh[h-1] \cap \Callbarh[h]  \cap \Ballh^c\right ],
\end{align}
where the last step invokes \Cref{claim:stability_claim}.
\end{proof}
Next, we bound the contribution of $\Pr_{\coup}[\Qall^c]$ in \eqref{eq:Gamimit_decomp}, uniformly over all couplings.
\begin{lemma}\label{lem:Qall_bound} For all $\coup$ which respect the construnction, 
\begin{align}
\Pr_{\coup}[\Qall^c] \le \pips + 2Hp_r.
\end{align}
\end{lemma}
\begin{proof} $\Pr_{\coup}[\Qclose^c] = \Pr_{\coup}[\exists h: \distips(\stel_h,\ssq_h) > 2r] \le 2Hp_r$ by \Cref{lem:rep_conc} and a union bound. 


Let us now bound $\Pr_{\coup}[\Qclose \cap \Qips^c] \le \Pr_{\coup}[\Qips^c \mid \Qclose ]$. Define the kernels $\lawW_h(\seqs)$ to be equal to the kernel $\Wreph(\seqs)$ conditioned on the event $\seqs' \sim \Wreph(\seqs)$ satisfies $\distips(\seqs',\seqs) \le 2r$. Then, conditional on $\Qclose$, we see that the sequence $(\srep_{1:H+1},\sreptil_{1:H},\arep_{1:H})$ obeys the generative process
\begin{align}
\sreptil_{h} \mid \sreptil_{1:h-1},\srep_{1:h},\arep_{1:h-1} \sim \lawW_h(\seqs), \quad \arep_{h} \mid \sreptil_{1:h},\srep_{1:h},\arep_{1:h-1} \sim \pisth(\sreptil_h), \quad \srep_{h+1} = F_h(\srep_h,\arep_h).
\end{align} 
By construction, for each $h$, $\Pr_{\seqs' \sim \Wreph(\seqs)}[\distips(\seqs',\seqs) > 2r] = 0$. Thus, the definition of (vector) input process stability (\Cref{defn:ips_vec}) and assumption $r \le \frac{1}{2}\rips$ implies that $\Pr_{\coup}[\Qips^c \mid \Qclose ] \le \pips$.
\end{proof}
The remaining step of the proof is therefore to bound the second term in \eqref{eq:Gamimit_decomp}.
\begin{lemma}\label{lem:make_coupling}There exists a coupling $\coup$ which respects the construction and satisfies the following for any $h \in [H]$
\begin{align}
&\Pr_{\coup}[\Ballh^c \mid \cF_{h-1}] \\
&\le \gamhat \circ \disttvc(\srep_h,\shat_h) + (\gamhat + \gamtvcsig) \circ \disttvc(\srep_h,\stel_h)  + \drobvec\,( \pihatsigh(\stel_h) \parallel \pistreph(\stel_h)),~ \text{$\coup$-almost surely }
    %&\pp_{\coup}\left[ \Ballbarh^c \cap \Callbarh \cap \Ballbarh[h-1] \right] \leq \\
   % &\quad\delta + \gamtvc(\epsilon) + (\gamtvc + \gamsig)\circ\gamips(2r) + \Exp_{\stel_h \sim \coup}\drobvec[\epsilon](\pihat \parallel \pistrep \mid \stel_{h}, h)
    %2 \cdot p_r + 2 \cdot \gamtvc\left( \gamips(3r) \right) + \drob[\epsilon]\left( \pistrep, \pihat | \stel_{h+1}, h+1 \right) + \gamtvc(\epsilon).
\end{align}
Consequently, for all $h \in [H]$,
\begin{align}
&\Pr_{\coup}[ \Ballh^c \cap \Callbarh \cap \Ballbarh[h-1]] \\
&\le \gamhat(\epsvec_1) + (\gamhat + \gamtvcsig) \circ \gamipsone(2r) + 
\Exp_{\coup}[\drobvec\,( \pihatsigh(\stel_h) \parallel \pistreph(\stel_h))]
\end{align}
Moreover, $\seqs \mapsto \drobvec\,( \pihatsigh(\seqs) \parallel \pistreph(\seqs))$ is measurable. 
%as well as $\pp_{\coup}[\Callbarh[0]] = 1$.
\end{lemma}

\begin{proof}[Proof Sketch]
We begin by giving a high level overview of the construction, which is done recursively.  The key technical tool is \Cref{lem:couplinggluing} above, which allows us to transform any coupling $\coup$ between random variables $(X, Y)$ into a probability kernel $\coup(\cdot| X)$ mapping instances of $X$ to probability distributions on $Y$ such that $(X, Y) \sim \coup$ has the same law as $(X, Y \sim \coup(\cdot | X))$.  For each $h$, we then show that, assuming the coupling has kept the states and controls close together until time $h-1$, this will imply the following chain:
\begin{align}
    \underbrace{(\arep \leftrightarrow \atel)}_{\gamtvc \text{ and induction}} \to \underbrace{(\atel \leftrightarrow \atelinter)}_{\text{learning and sampling}} \to \underbrace{(\atelinter \leftrightarrow \arepinter)}_{\gamtvc \text{ and induction}} \to \underbrace{(\arepinter \leftrightarrow \seqahat)}_{\gamtvc \text{ and induction}}, \label{eq:mainproofoutline}
\end{align}
where the bidirectional arrows indicate individual couplings between the laws of the random variables that are constructed by the method outlined in text below and the single directional arrows denote the probability kernels described above. The full proof of the lemma is given in \Cref{sec:couplingconstruction}.
\end{proof}

\paragraph{Concluding the proof.}  Here, we finish the proof of \Cref{thm:smooth_cor_general}.  Recall that we wish to bound $\gapjointvec\,(\pihatsig \parallel \pirep) \vee \gapmargvec[\epsvecmarg](\pihatsig \parallel \pist)$. We begin by bounding $\gapjointvec\,(\pihatsig \parallel \pirep) \vee \gapmargvec[\epsvecmarg](\pihatsig \parallel \pistrep)$.  In light of \Cref{lem:putting_couplings_together}, it suffices to bound
 \begin{align}
 \Pr_{\coup}[\Qall^c] + \sum_{h=1}^H\Pr_{\coup}[ \Ballbarh^c \cap \Callbarh \cap \Ballbarh[h-1]], 
 \end{align}
 where $\coup$ is the coupling in \Cref{lem:make_coupling}.
Applying \Cref{lem:Qall_bound} and \Cref{lem:make_coupling},
\begin{align}
&\Pr_{\coup}[\Qall^c] + \sum_{h=1}^H\Pr_{\coup}[ \Ballbarh^c \cap \Callbarh \cap \Ballbarh[h-1]] \\
&\le \pips + 2Hp_r +   \sum_{h=1}^H\Pr_{\coup}[ \Ballbarh^c \cap \Callbarh \cap \Ballbarh[h-1]] \\
&\le   \pips + H(2p_r + \gamhat(\epsvec_1) + (\gamhat + \gamtvcsig) \circ \gamipsone(2r))  + \sum_{h=1}^H\Exp_{\stel_h \sim \coup}\drobvec\,( \pihatsigh(\stel_h) \parallel \pistreph(\stel_h))
\end{align}
To conclude, we note that for any $\coup$ which respects the construction, \Cref{lem:replica_property} ensures that $\stel_h$ as the marginal distribution of $\sstar_h \sim \pisth$. Thus, the above is at most
\begin{align}
\pips + H(2p_r + \gamhat(\epsvec_1) + (\gamhat + \gamtvcsig) \circ \gamipsone(2r))  + \sum_{h=1}^H\Exp_{\sstar_h \sim \Psth}\drobvec\,( \pihatsigh(\sstar_h) \parallel \pistreph(\sstar_h)) \label{eq:first_eq_i_showed}
\end{align}
which concludes the proof of \eqref{eq:smooth_ub_app_one} for $\gapjointvec(\pihat \parallel \pistrep)$. 

To prove \eqref{eq:smooth_ub_app_two} for $\gapjointvec(\pihat \parallel \pistrep)$, we consider the special case that $\pihatsig = \pihat \circ \Wsig$. By definition, $\pihatsigh =\pihat \circ \Wsig$. Thus, the  data-processing inequality for optimal transport (\Cref{cor:opt_trans}) 
\begin{align}\drobvec\,( \pihatsigh(\sstar_h) \parallel \pistreph(\sstar_h))  \le \Exp_{\seqs_h' \sim \Wsig(\sstar_h)}\drobvec\,( \pihat(\seqs_h') \parallel \pidech(\seqs_h')),
\end{align}
for all $\sstar_h$. Substituting this into \eqref{eq:first_eq_i_showed}, and setting $\gamhat = \gamsig$ (in view of \Cref{lem:pistrep_tvc}), finishes the argument.













\subsubsection{Proof of Lemma \ref{lem:make_coupling}}\label{sec:couplingconstruction}

Recall that \Cref{ass:polish_spaces_general} ensures all of the general measure-theoretic guarantees of Appendix \ref{app:prob_theory} hold true in our setting. Notably we need the gluing lemma (\Cref{lem:couplinggluing}) and the commuting of optimal transport metrics and conditional probabilities (\Cref{prop:MK_RCP}).

\paragraph{Proof strategy.} Our proof follows along similar lines as that of \Cref{prop:IS_general_body}, although with the added complication of including the smoothing.  We will inductively construct $\coup$.  A useful schematic for the construction at each step is the following diagram:
\begin{align}
    \underbrace{(\sreptil \leftrightarrow \ssq),(\arep \leftrightarrow \atel)}_{\Btelh} \to \underbrace{(\atel \leftrightarrow \atelinter)}_{\Bfsh} \to \underbrace{(\atelinter \leftrightarrow \arepinter)}_{\Binterh}\to \underbrace{(\arepinter \leftrightarrow \seqahat)}_{\Bhath}, \label{eq:mainproofoutline2}
\end{align}
where the events under each bidirectional arrow refer to the event such ensuring that there exists a coupling such that the objects are close.  We then will apply \Cref{lem:couplinggluing} to glue the individual couplings together.  We will then use \Cref{lem:peeling_lem} and a union bound to control the probability under our constructed coupling that any of the relevant events fail to hold, concluding the proof.

\newcommand{\coupfs}{\coup_{\mathrm{est}}}
\newcommand{\Ebarh}{\bar{\cE}_{h}}

\newcommand{\coupstel}{\coup_{\seqs,\mathrm{tel}}}  
\newcommand{\coupinterr}{\coup_{\mathrm{inter}}}  

\newcommand{\couptel}{\coup_{\mathrm{tel}}}  
\newcommand{\coupahat}{\coup_{\seqahat}}  
\paragraph{Recursive construction of $\coup$.} Let $h \ge 1$, and suppose that we have constructed the coupling $\coup^{(1:h-1)}$ for steps $1,\dots,h-1$ which respects the construction. Recall that $\cF_h$ denotes the sigma-algebra generated by  $(\shat_{1:h},\srep_{1:h},\stel_{1:h})$, $(\arep_{1:h},\sreptil_{1:h},\ssq_{1:h},\atel_{1:h},\seqahat_{1:h})$, and $(\arepinter_{1:h},\atelinter_{1:h})$. Notice that $\stel_{h+1},\srep_{h+1},\shat_{h+1}$ are determined by $\cF_h$ as well. Similarly, it can be seen from \Cref{defn:all_kernels} that $\phiV(\ssq_{h+1})$ and $\phiV(\sreptil_{h+1})$ are also determined by $\cF_{h}$ (since the replica kernel preserves the $\cV$-components). We summarize all these aforementioned variables in a random variable $Y_h$. Let $\cF_0$ denote the filtration generated by $\srep_1 = \stel_1 = \shat_1$. We let $Y_0 = (\srep_1,\stel_1,\shat_1)$. 

Correspondingly, let $Z_h$ denote the random variables $(\arep_{h},\phiZ(\sreptil_{h}),\phiZ(\ssq_{h}),\atel_{h},\seqahat_{h})$, and $(\arepinter_{h},\atelinter_{h})$ such that the joint law of these random variables respects the construction.  Our goal is then to specify, for each $h \in [H]$, a joint distribution of $(Y_{h-1},Z_{h})$.
Note that $Z_h,Y_{h-1}$ determines $Y_{h}$, and we call this induced law $\coup^{(h)}$.






We begin by specifying joint distributions conditional on $Y_{h-1}$ and subsets of $Z_h$, then glue them together by the gluing lemma. Below, we use use information-theoretic notation. 
\begin{itemize}
    \item By total variation continuity of $\phiZ \circ \Qreph$ (\Cref{lem:pistrep_tvc}),
    \begin{align}
    \TV(\pp_{\phiZ(\sreptil_{h}) \mid  Y_{h-1}},\pp_{\phiZ(\ssq_{h}) \mid  Y_{h-1}}) \le \gamtvcsig \circ \disttvc(\srep_h,\stel_h). 
    \end{align}
    Because $\arep_{h} \sim \pisth(\sreptil_{h+1})$ and $\atel_{h} \sim \pisth(\ssq_{h})$,  and $\pist$ is compatible with the decomposition $\cS = \cZ \oplus \cV$ (i.e. $\pisth(\seqs)$ is a function of $\phiZ(\seqs)$)
    \Cref{cor:tv_two} implies that (almost surely)
    \begin{align}
    \TV(\pp_{(\arep_h,\phiZ(\sreptil_{h}) \mid Y_{h-1}},\pp_{(\atel_h,\phiZ(\ssq_{h}) \mid  Y_{h-1}}) \le \gamtvcsig \circ \disttvc(\srep_h,\stel_h). 
    \end{align}
    Hence, \Cref{cor:first_TV} implies that there exists a coupling $\couptel^{(h)}$ over $Y_{h-1},(\phiZ(\sreptil_{h}),\arep_h),(\phiZ(\ssq_{h}),\atel_{h})$ respecting the construction such that $Y_h \sim \coup^{(h-1)}$ and such that (almost surely)
    \begin{align}\label{eq:couptel}
    \Exp_{\couptel^{(h)}}[\Btelh \mid Y_{h-1}] = \Pr_{\couptel^{(h)}}[(\phiZ(\sreptil_{h}),\arep_h) \ne (\phiZ(\ssq_{h}),\atel_{h})\mid Y_{h-1}] &\le \disttvc(\srep_h,\stel_h)].
    \end{align}
    \item In our construction, $\atel_h \mid Y_{h-1} \sim \pistreph(\stel_h)$, and $\atelinter_h \mid Y_{h-1} \sim \pihatsigh(\stel_h)$. 
    Thus, by definition of $\drobvec$, and the assumption $\I\{\distavec(\cdot,\cdot) \not\preceq \epsvec\}$ is lower semicontinuous, \Cref{prop:MK_RCP} implies that we may find a coupling $\coupfs^{(h)}$ of $(\atel_{h},\atelinter_{h},Y_{h-1})$ respecting the construction such that, almost surely,
    \begin{align}\label{eq:coupfs}
    \pp_{\coupfs^{(h)}}\left[\Bfsh^c \mid Y_{h-1} \right] &=  \pp_{\coupfs^{(h)}}\left[ \distavec(\atelinter_{h},\atel_{h}) \not \preceq \epsvec \mid Y_{h-1} \right] \\
    &= \drobvec\,( \pihatsigh(\stel_h) \parallel \pistreph(\stel_h))].
\end{align}
Moreover, that same proposition ensures measurability of $\seqs \to \drobvec\,( \pihatsigh(\seqs) \parallel \pistreph(\seqs))$.
\item Since $\atelinter_{h} \mid \cF_h \sim \pihatsigh(\stel_h)$ and $\arepinter_{h+1} \mid \cF_h \sim \pihatsigh(\srep_h)$, and since  $\pihatsigh(\cdot)$ is $\gamhat$-TVC by assumption,  
\begin{align}
\TV(\pp_{\atelinter_{h} \mid  Y_{h-1}},\pp_{ \arepinter_{h} \mid  Y_{h-1}}) \le \gamhat \circ \disttvc(\srep_h,\stel_h). 
\end{align}

\Cref{cor:first_TV}  implies that there is a coupling $\coupinterr^{(h)}$ between $(\atelinter_{h},\arepinter_{h},Y_{h-1})$ such that
\begin{align}\label{eq:coupinterr}
\pp_{\coupinterr^{(h)}}[\Binterh^c \mid Y_{h-1}] = \pp_{\coupinterr^{(h)}}\left[\atelinter_{h} \ne \arepinter_{h}  \mid Y_{h-1}\right] &\le  \gamhat \circ  \disttvc(\stel_h,\srep_h)
\end{align}
\item  Similarly, since $\arepinter_{h} \mid \cF_{h-1} \sim \pihat_h(\srep_h)$ and $\seqahat_{h+1} \mid \cF_{h-1}\sim \pihat_h(\shat_h)$, $\pihat_h(\cdot)$ is $\gamhat$-TVC,  \Cref{cor:first_TV} implies that there is a coupling $\coupahat^{(h)}$ between $(\arepinter_{h},\seqahat_{h},Y_{h-1})$ such that
\begin{align}\label{eq:coupahat}
\pp_{\coupahat^{(h)}}[\Bhath^c \mid Y_{h-1}] = \pp_{\coupahat^{(h)}}\left[\seqahat_{h} \ne \arepinter_{h} \mid Y_{h-1}  \right] \le \gamhat \circ \disttvc(\srep_h,\shat_h)
\end{align}
\end{itemize}

We can then apply the gluing lemma (\Cref{lem:couplinggluing}) to 
\begin{align}
X_{h,1} &= (\phiZ(\ssq_h),\atel_h,Y_{h-1}) \\ 
X_{h,2} &= (\phiZ(\sreptil_h),\arep_h,Y_{h-1}) \\
 X_{h,3} &= (\atel_h,\atelinter_h,Y_{h-1}) \\
  X_{h,4} &= (\atelinter_h,\arepinter_h,Y_{h-1}) \\
   X_{h,5} &= (\arepinter_h,\ahat_h,Y_{h-1})  
\end{align}
with 
\begin{align}
(X_{h,1},X_{h,2}) \sim \couptel^{(h)},\quad (X_{h,2},X_{h,3}) \sim  \coupfs^{(h)}, \quad (X_{h,3},X_{h,4})\sim \coupinterr^{(h)}, \quad (X_{h,4},X_{h,5})\sim\coupahat^{(h)}.
\end{align}
\Cref{lem:couplinggluing} guarantees the existence of a coupling $\mu^{(h)}$ consident with all sub-couplings $\couptel^{(h)}$, $\coupfs^{(h)},\coupinter^{(h)},\coupahat^{(h)}$. Then, $\coup^{(h)}$-almost surely (and using that $\cF_{h-1}$ is precisely the $\upsigma$-algebra generated by $Y_{h-1}$)
\begin{align}
&\Pr_{\coup^{(h)}}[\Ballh^c \mid \cF_{h-1}] \\
&\le \Pr_{\coup^{(h)}}[\Btelh^c \mid \cF_{h-1}] + \Pr_{\coup^{(h)}}[\Bfsh^c \cF_{h-1}] +  \Pr_{\coup^{(h)}}[\Binterh^c \cF_{h-1}]+\Pr_{\coup^{(h)}}[\Bhath^c \cF_{h-1}]\\
&\le \gamhat \circ \disttvc(\srep_h,\shat_h) + (\gamhat + \gamtvcsig) \circ \disttvc(\srep_h,\stel_h)  + \drobvec\,( \pihatsigh(\stel_h) \parallel \pistreph(\stel_h))\\
&= \gamhat \circ \disttvc(\srep_h,\shat_h) + (\gamhat + \gamtvcsig) \circ \disttvc(\srep_h,\stel_h)  + \drobvec\,( \pihatsigh(\stel_h) \parallel \pistreph(\stel_h))
\end{align}
This concludes the inductive construction.


For the second statement, notice that the events $\Callbarh \cap \Ballbarh[h-1]$ are $\cF_h$ measurable (thus determined by $\coup^{(h-1)}$) and, when they hold, $\distsvec(\srep_h,\stel_h) \preceq \gamipsvec(2r)$ and $\dists(\srep_h,\shat_h)  \preceq \epsvec$. For our purposes, we use $\disttvc = \distsi[1](\srep_h,\stel_h) \preceq \gamipsone(2r)$ and $\dists(\srep_h,\shat_h)  \preceq \epsvec_1$. Hence, 
\begin{align}
\max_{h\in [H]}\Pr_{\coup}[ \Ballh^c \cap \Callbarh \cap \Ballbarh[h-1]] &\le \gamhat(\epsvec_1) + (\gamhat + \gamtvcsig) \circ \gamipsone(2r) \\
&\quad+ \drobvec\,( \pihatsigh(\stel_h) \parallel \pistreph(\stel_h)).
\end{align}
The result follows.


\subsection{Proof of Theorem \ref{thm:smooth_cor}, and generalization to direct decompositions}\label{app:smoothcor_proof}

In this subsection, we consider the special case dealt with in \Cref{thm:smooth_cor}.  Note that there always exists a trivial direct decomposition that is compatible with all policies and dynamics simply by letting $\cV = \emptyset$ and $\cS = \cZ$.  We prove here the version of the result that involves a possibly nontrivial direct decomposition, as we will instantiate this in our control setting by letting $\cZ = \left\{ \pathm[h] \right\}$ and $\cS = \left\{ \pathc[h] \right\}$, i.e., projecting $\pathc[h]$ onto the last $\taum$ coordinates gives $\seqz_h$. We further consider a restriction of IPS to consider kernels absolutely continuous with respect to $\Psth$ in their $\cZ$ component. 
\begin{definition}[Restricted IPS]\label{defn:ips_restricted}
For a non-decreasing maps $\gamipsone,\gamipstwo:\R_{\ge 0} \to  \R_{\ge 0}$ a  pseudometric $\distips:\cS \times \cS \to \R$ (possibly other than $\dists$ or $\disttvc$), and $\rips > 0$, we say a policy $\pi$ is \emph{$(\gamipsone,\gamipstwo,\distips,\rips)$-restricted IPS} if the following holds for any $r \in [0,\rips]$. Consider any sequence of kernels $\lawW_1,\dots,\lawW_H:\cS \to \laws(\cS)$ satisfying 
\begin{align}
\max_{h,\seqs \in \cS}\Pr_{\tilde \seqs\sim \lawW_h(\seqs)}[\distips(\tilde \seqs,\seqs) \le r] = 1, \quad \forall s, \quad \phiZ \circ \lawW_h(\seqs_h) \ll \phiZ \circ \Psth. 
\end{align}
 and define a process $\seqs_1 \sim \Dinit$, $\tilde\seqs_h \sim \lawW_h(\seqs_h),\seqa_h \sim \pi_h(\tilde \seqs_h)$, and $\seqs_{h+1} := F_h(\seqs_h,\seqa_h)$. Then, almost surely, (a) the sequence $(\seqs_{1:H+1},\seqa_{1:H})$ is input-stable w.r.t $(\dists,\dista)$ (b) $\max_{h \in [H]} \disttvc(F_h(\tilde\seqs_h,\seqa_h),\seqs_{h+1}) \le \gamipsone(r)$ and (c) $\max_{h \in [H]} \dists(F_h(\tilde\seqs_h,\seqa_h),\seqs_{h+1}) \le \gamipstwo(r)$.
\end{definition}

Note that the above is a  slightly weaker condition than the one in \Cref{defn:ips_body} in the main text and consequently, the following theorem which uses it as an assumption implies \Cref{thm:smooth_cor} in the body.
\begin{theorem}\label{thm:smooth_cor_decomp} Suppose $\cS = \cZ \oplus \cV$ as in \Cref{defn:direct_decomp} and projections $\phiZ,\phiV$, which is compatible with the dynamics and with given policies $\pihat,\pist$, smoothing kernel $\Wsig$, and pseudometric $\distips$.
Suppose $\pist$ satisfies $(\gamipsone,\gamipstwo,\distips,\rips)$-restricted IPS (\Cref{defn:ips_restricted}) and $\phiZ \circ \Wsig$ is $\gamma_{\sigma}$-TVC. Let $\epsilon > 0$, $r \in (0,\frac{1}{2}\rips]$; define $p_r := \sup_{\seqs}\Pr_{\seqs' \sim \Wsig(\seqs)}[\distips(\seqs',\seqs) >  r]$ and $\epsilon' := \epsilon+\gamipstwo(2r)$. Then, for any policy $\pihat$,  both  $\gapjoint (\pihat \circ \Wsig \parallel \pistrep)$ and  $\gapmarg[\epsilon'] (\pihat \circ \Wsig \parallel \pist)$ are upper bounded by
\begin{align}
%\inf_{r > 0}  
H\left(2p_r +  3\gamma_{\sigma}(\max\{\epsilon,\gamipsone(2r)\})\right)  + \textstyle \sum_{h=1}^H\Exp_{\sstar_h \sim \Psth}\Exp_{\sstartil_h \sim \Wsig(\sstar_h) } \drob( \pihat_{h}(\sstartil_h) \parallel \pidec(\sstartil_h))  . \label{eq:smooth_ub}
\end{align}
\end{theorem}


Consider the special case $K = 2$ with $\distsi[1] = \disttvc$, $\distsi[2] = \dists$, $\distai[1] = \distai[2] = \dista$ and $\epsvec = (\epsilon, \epsilon)$.  In this case, applying \eqref{eq:smooth_ub_app_two}, we see that
\begin{align}
    &\gapjointvec(\pihatsig \parallel \pistrep) \vee \gapmargvec[\epsvecmarg](\pihatsig \parallel \pistrep) \\
    &\leq \pips + H\left(2p_r +  3\gamma_{\sigma}(\max\{\epsilon,\gamipsone(2r)\}\right)  + \textstyle \sum_{h=1}^H\Exp_{\sstar_h \sim \Psth}\Exp_{\sstartil_h \sim \Wsig(\sstar_h) } \drobvec( \pihat_{h}(\sstartil_h) \parallel \pidec(\sstartil_h))
\end{align}
We now observe that under this convention,
\begin{align}
    \gapjoint(\pihatsig \parallel \pistrep) &= \inf_{\coup_1} \pp_{\coup_1}[\max_{h \in [H]} \dists(\shat_{h+1}, \sstar_{h+1}) \vee \dista(\ahat_h, \astar_h) > \epsilon] \\
    &\leq \inf_{\coup_1} \pp_{\coup_1}\left[ \max_{h \in [H]} \left( \disttvc(\shat_{h+1}, \sstar_{h+1}), \dists(\shat_{h+1}, \sstar_{h+1}) \right) \vee \left( \dista(\ahat_h, \astar_h), \dista(\ahat_h, \astar_h) \right) \not \preceq \epsvec\right] \\
    &= \gapjointvec(\pihatsig \parallel \pistrep)
\end{align}
and similarly $\gapmarg[\epsilon'](\pihatsig \parallel \pist) \leq \gapmargvec[\epsvec + \gamips(2r)](\pihatsig \parallel \pist)$.  From the construction of $\distavec$, however, we see that $\left\{ \distavec(\seqa, \seqa') \not \preceq \epsvec \right\} = \left\{ \dista(\seqa, \seqa') > \epsilon \right\}$ for all $\seqa, \seqa'$ and thus for all $h \in [H]$,
\begin{align}
    \drobvec(\pihat_{h}(\sstartil_h) \parallel \pist_h(\sstartil_h)) &= \inf_{\coup_2} \pp_{\coup_2}\left[ \distavec(\seqahat_h, \seqast_h) \not \preceq \epsvec \right] \\
    &= \inf_{\coup_2} \pp_{\coup_2}\left[ \dista(\seqahat_h, \seqast_h) \geq \epsilon \right] \\
    &= \drob(\pihat_h(\sstartil_h) \parallel \pist_h(\sstartil_h)).
\end{align}
Plugging in to \eqref{eq:smooth_ub_app_two} concludes the proof.









\subsection{Proof of \Cref{thm:general}}\label{sec:proof-thm1}
We now turn to the proofs of \Cref{thm:general}(i)-(iv). In each of these parts, we consider a set of recommendations with $a_t$ items of type $t$ for each $t\in [m].$ Then observe that the expected total value of the $k$ highest value recommended items is equal to
\begin{equation}
    \sum_{t=1}^m p_t h(a_t),
\end{equation}
for
\begin{equation}
    h:\ZZ \rightarrow \RR, \quad h:a\mapsto \EE\left[\topp_k\{X_1,\cdots,X_a\simiid \DD\}\right],
\end{equation}
where $\topp_k$ evaluates the sum of the $k$ highest values in a set. Intuitively, conditional on a user preferring type $t$, the top $k$ items are just the top $k$ items recommended of type $t$. The sum of their values, conditioned on the user preferring type $t$, is simply the sum of the $k$ highest values among $a$ random draws from $\DD$. Clearly, $h$ here is monotonically increasing.

Then, with \Cref{lem:fennel} in hand, parts (i)-(iv) reduces to showing the following:
\begin{enumerate}
    \item[(i)] If $\DD$ is a finite discrete distribution, there exist constants $A,B>0$ and $\sigma>0$ such that  \begin{equation}\lim_{a\rightarrow \infty} \frac{\log(A - h(a))}{Ba^\sigma} = 1.\end{equation}
    \item[(ii)] If $\DD$ has support bounded from above by $M$ with pdf $f_\DD$ satisfying
    \begin{equation}
    \lim_{x\rightarrow M} \frac{f_\DD(x)}{(M-x)^{\beta-1}} = c
    \end{equation}
    for some $\beta, c>0$, then there exist constants $A,B>0$ such that
        \begin{equation} \lim_{a\rightarrow \infty} \frac{A-h(a)}{Ba^{-\frac{1}{\beta}}} = 1.
        \end{equation}

    \item[(iii)] If $\DD = \Exp(\lambda)$ for $\lambda > 0,$ then $h$ is strictly concave and there exists a constant $B>0$ such that
        \begin{equation}
            \lim_{a\rightarrow \infty} \frac{h(a)}{B\log a} = 1.
        \end{equation}
    \item[(iv)] If $\DD = \Pareto(\alpha)$ for $\alpha>1$, then $h$ is strictly concave and there exists a constant $B>0$ such that
        \begin{equation}
            \lim_{a\rightarrow \infty} \frac{h(a)}{Ba^{\frac{1}{\alpha}}} = 1.
        \end{equation}
\end{enumerate}

The following identity, mentioned in \Cref{sec:proof-sketch}, will be useful for parts (ii)-(iv).
\begin{proposition}\label{prop:mu}
For $X_i^{(t)}\simiid \DD$,
    \begin{equation}
    h(a) = \sum_{i=1}^{\min\{k, a\}} \mu_\DD(a-i+1,a).
    \end{equation}
\end{proposition}
Recall that $\mu_\DD(i,a)$ is the expected value of the $i$-th order statistic of $a$ random variables drawn i.i.d. from $\DD$.

\begin{proof}
    Let $Y_{k,n}$ be the $k$-th order statistic of $n$ random variables distributed i.i.d. from $\DD$. Then
\begin{equation}\label{eq:pond}
\topp_{k}\{X_1^{(t)},\cdots,X_{a}^{(t)}\}
= \sum_{i=1}^{\min\{k,a\}} Y_{a-i+1, a}.
\end{equation}
So, as desired,
\begin{equation}
    \EE\left[\topp_{k}\{X_1^{(t)},\cdots,X_{a}^{(t)}\}\right] = \sum_{i=1}^{\min\{k,a\}} \EE\left[Y_{a-i+1,a}\right] = \sum_{i=1}^{\min\{k, a\}} \mu_\DD(a-i+1,a),
\end{equation}
where the first equality follows from \eqref{eq:pond} and the linearity of expectation.
\end{proof}

\paragraph{Proof of \Cref{thm:general}(i).}
Suppose $\DD$ has support $\{x_1,\cdots,x_r\}$ with $x_1>\cdots>x_r$ such that for $X\sim \DD$, $\Pr[X=x_1] = q.$ Now consider a set of recommendations with $a_t$ items of type $t$ for each $t\in [m].$ Then consider $X_1,\cdots,X_a\simiid \DD$. Let $E$ be the event that at least $k$ of $X_1,\cdots,X_a$ equal $x_1$. Then,
\begin{align}
    h(a) &\ge \EE[\topp_k\{X_1,\cdots,X_a\}|E] \cdot \Pr[E]\\
    &= x_1k\cdot \left(1 - \sum_{j=0}^{k-1}\binom{a}{j}(1-q)^{a-j}q^j\right)\\
    &\ge x_1k(1 - a^k(1-q)^{a-k+1})
\end{align}
for all $a>2.$ Now let $E'$ be the event that at least one of $X_1,\cdots, X_a$ equals $x_1$. Then,
\begin{align}
    h(a) &= \EE[\topp_k\{X_1,\cdots,X_a\}|E'] \cdot \Pr[E'] + \EE[\topp_k\{X_1,\cdots,X_a\}|\overline{E'}] \cdot (1-\Pr[E'])\\
    &\le x_1k(1 - (1-q)^{a}) + x_2k(1-q)^{a}\\
    &= x_1k(1 - (1-\frac{x_2}{x_1})(1-q)^{a}).
\end{align}
Now note that for $A = x_1k,$ we have that
\begin{align}
    x_1k(1 - \frac{x_2}{x_1})(1-q)^{a} &\le A - h(a) \le x_1k a^k (1-q)^{a-k+1}\\
    \log(x_1k(1 - \frac{x_2}{x_1})(1-q)^{a}) &\le \log(A - h(a)) \le \log(x_1k a^k (1-q)^{a-k+1})\\
    \log(x_1k) + \log(1 - \frac{x_2}{x_1}) + a\log(1-q) &\le \log(A - h(a)) \le \log(x_1k) + k\log(a) + (a-k+1)\log(1-q).
\end{align}
It follows that for $B = \log(1-q),$
\begin{equation}
    \lim_{a\rightarrow \infty} \frac{\log(A - h(a))}{Ba} = 1,
\end{equation}
as desired. The result follows from \Cref{lem:fennel}(i).

\paragraph{Proof of \Cref{thm:general}(ii).}

First recall from \Cref{prop:mu} that
\begin{equation}
    h(a) = \sum_{i=1}^{\min\{k, a\}} \mu_\DD(a-i+1,a).
\end{equation}
We will show that \begin{equation}
\lim_{a\rightarrow \infty} \frac{Mk - h(a)}{Ba^{-\frac{1}{\beta}}} = 1
\end{equation}
for a constant $B>0.$ \Cref{thm:general}(ii) then follows immediately by applying \Cref{lem:fennel}(ii) with $\sigma = -\frac{1}{\beta}$.

Consider a probability distribution $\DD'$ with pdf $g_X(x)=f_X(M-x)$ and cdf $G_X(x).$ Then
\begin{equation}
\mu_\DD(a-i+1,a) = M - \mu_{\DD'}(i,a),
\end{equation}
which implies that
\begin{equation}
    Mk - \sum_{i=1}^{k} \mu_\DD(a-i+1,a) = \sum_{i=1}^{k} \mu_{\DD'}(i,a)
\end{equation}
Since
\begin{equation}
    \mu_{\DD'}(i,a) = \sum_{j=0}^{i-1} \int_0^\infty \binom{a}{j}G_X(x)^j (1-G_X(x))^{a-j}\,dx,
\end{equation}
it remains to show that for all fixed $j$,
\begin{equation}\label{eq:orangepeel}
    \lim_{a\rightarrow \infty}\frac{\int_0^\infty \binom{a}{j}G_X(x)^j (1-G_X(x))^{a-j}\,dx}{a^{-\frac{1}{\beta}}} = B
\end{equation}
for some constant $B$ (that can vary depending on $j$). Verifying \eqref{eq:orangepeel} comprises the bulk of the technical work of the proof, and we isolate it in the following lemma.

\begin{lemma}
For $\beta>0$,
\begin{equation}
\int_0^\infty \binom{a}{j} G_X(x)^j (1 - G_X(x))^{a-j}\,dx \propto a^{-\frac{1}{\beta}}.
\end{equation}
\end{lemma}

\begin{proof}
We have that
\begin{equation}
    \lim_{x\rightarrow 0^+} \frac{g_X(x)}{cx^{\beta-1}} = \lim_{x\rightarrow M^{-}} \frac{f_X(x)}{c(M-x)^{\beta-1}} = 1
\end{equation}
for a positive constant $c$. So for all $\epsilon>0$ there exists $\delta>0$ such that
\begin{equation}
(1 - \epsilon)cx^{\beta-1} \le g_X(x) \le (1 + \epsilon)cx^{\beta-1}
\end{equation}
for all $x<\delta.$
Now note that $g_X(x) \le (1 + \epsilon)cx^{\beta-1}$ implies that
\begin{equation}
    G_X(x) = \int_0^x g_X(u)\,du \le (1+\epsilon)\int_0^x cu^{\beta-1}\,du = (1+\epsilon)\frac{c}{\beta}x^\beta.
\end{equation}
Likewise, $g_X(x) \ge (1 - \epsilon)cx^{\beta-1}$ implies that
\begin{equation}
    G_X(x) = \int_0^x g_X(u)\,du \ge (1-\epsilon)\int_0^x cu^{\beta-1}\,du = (1-\epsilon)\frac{c}{\beta}x^\beta.
\end{equation}
Now write
\begin{align}
&a^{\frac{1}{\beta}}\int_0^\infty \binom{a}{j} G_X(x)^j (1 - G_X(x))^{a-j}\,dx\\ &=a^{\frac{1}{\beta}}\int_0^\delta \binom{a}{j} G_X(x)^j (1 - G_X(x))^{a-j}\,dx + a^{\frac{1}{\beta}}\int_\delta^\infty \binom{a}{j} G_X(x)^j (1 - G_X(x))^{a-j}\,dx.
\end{align}
We will analyze these two integral separately. It will turn out that the second integral vanishes as $a$ grows.
\end{proof}
\paragraph{The first integral.} We have that
\begin{align}
&a^{\frac{1}{\beta}}\int_0^\delta \binom{a}{j} G_X(x)^j (1 - G_X(x))^{a-j}\,dx\\
&\le a^{\frac{1}{\beta}}\int_0^\delta \binom{a}{j} (1+\epsilon)^j\left(\frac{c}{\beta}\right)^j x^{\beta j} (1 - (1-\epsilon)\frac{c}{\beta}x^\beta)^{a-j}\,dx\\
&= \int_0^{\delta a^{\frac{1}{\beta}}} \binom{a}{j} (1+\epsilon)^j\left(\frac{c}{\beta}\right)^j \left(\frac{x}{a^{\frac{1}{\beta}}}\right)^{\beta j} \left(1 - (1-\epsilon)\frac{c}{\beta}\left(\frac{x}{a^{\frac{1}{\beta}}}\right)^\beta\right)^{a-j}\,dx\\
&= \int_0^{\delta a^{\frac{1}{\beta}}} \binom{a}{j} (1+\epsilon)^j\left(\frac{c}{\beta}\right)^j \frac{x^\beta j}{a^j} \left(1 - (1-\epsilon)\frac{c}{\beta}\frac{x}{a}\right)^{a-j}\,dx.
\end{align}
Then
\begin{equation}
    \int_0^{\delta a^{\frac{1}{\beta}}} \binom{a}{j} (1+\epsilon)^j\left(\frac{c}{\beta}\right)^j \frac{x^\beta j}{a^j} \left(1 - (1-\epsilon)\frac{c}{\beta}\frac{x}{a}\right)^{a-j}\,dx
    = \int_0^\infty \phi_{a}(x)\,dx,
\end{equation}
where
\begin{equation}
    \phi_{a}(x) := \begin{cases}
    \binom{a}{j} (1+\epsilon)^j\left(\frac{c}{\beta}\right)^j \frac{x^\beta j}{a^j} \left(1 - (1-\epsilon)\frac{c}{\beta}\frac{x}{a}\right)^{a-j}\,dx &\quad \text{for }0\le x\le \delta a^{\frac{1}{\beta}}\\
    0&\quad \text{for }x>\delta a^{\frac{1}{\beta}}.
    \end{cases}
\end{equation}
We have that
\begin{equation}
\lim_{a\rightarrow \infty} \phi_{a}(x) = \frac{1}{j!}(1+\epsilon)^j\left(\frac{c}{\beta}\right)^jx^{\beta j}e^{-(1-\epsilon)\frac{c}{\beta}x^{\beta}}
\end{equation}
and
\begin{equation}
    \phi_{a}(x) \le \frac{1}{j!}(1+\epsilon)^j\left(\frac{c}{\beta}\right)^jx^{\beta j}e^{-(1-\epsilon)\frac{c}{\beta}x^{\beta}} (1 - (1-\epsilon)\frac{c}{\beta}\epsilon^\beta))^{-j} = C(j,\epsilon)x^{\beta j}e^{-(1-\epsilon)\frac{c}{\beta}x^{\beta}}
\end{equation}
for a constant $C(j,\epsilon)$ independent of $a.$ Now note that $\int_0^\infty x^{\beta j}e^{-(1-\epsilon)\frac{c}{\beta}x^{\beta}} < \infty.$ It follows from the dominated convergence theorem that
\begin{equation}
    \lim_{a\rightarrow \infty} \int_0^\infty \phi_{a}(x)\,dx
    = \int_0^\infty \lim_{a\rightarrow \infty} \phi_{a}(x)\,dx
    = \int_0^\infty \frac{1}{j!}(1+\epsilon)^j\left(\frac{c}{\beta}\right)^jx^{\beta j}e^{-(1-\epsilon)\frac{c}{\beta}x^{\beta}}\,dx < \infty.
\end{equation}
Therefore, for $a$ sufficiently large,
\begin{equation}
    \int_0^\delta \binom{a}{j} G_X(x)^j (1 - G_X(x))^{a-j}\,dx \le a^{-\frac{1}{\beta}}(1+\epsilon)\int_0^\infty \frac{1}{j!}(1+\epsilon)^j\left(\frac{c}{\beta}\right)^jx^{\beta j}e^{-(1-\epsilon)\frac{c}{\beta}x^{\beta}}\,dx.
\end{equation}
Analogously, we can show that for $a$ sufficiently large,
\begin{equation}
    \int_0^\delta \binom{a}{j} G_X(x)^j (1 - G_X(x))^{a-j}\,dx \ge a^{-\frac{1}{\beta}}(1-\epsilon)\int_0^\infty \frac{1}{j!}(1-\epsilon)^j\left(\frac{c}{\beta}\right)^jx^{\beta j}e^{-(1+\epsilon)\frac{c}{\beta}x^{\beta}}\,dx.
\end{equation}
Now observe that
\begin{align}
&\lim_{\epsilon\rightarrow 0^+} (1+\epsilon)\int_0^\infty \frac{1}{j!}(1+\epsilon)^j\left(\frac{c}{\beta}\right)^jx^{\beta j}e^{-(1-\epsilon)\frac{c}{\beta}x^{\beta}}\,dx\\
&= \int_0^\infty \frac{1}{j!}\left(\frac{c}{\beta}\right)^j x^{\beta j}e^{-\frac{c}{\beta}x^\beta}\,dx\\
&= \lim_{\epsilon\rightarrow 0^+} (1-\epsilon)\int_0^\infty \frac{1}{j!}(1-\epsilon)^j\left(\frac{c}{\beta}\right)^jx^{\beta j}e^{-(1+\epsilon)\frac{c}{\beta}x^{\beta}}\,dx,
\end{align}
where we once again apply the dominated convergence theorem. It follows that
\begin{equation}\label{eq:onion}
    \lim_{a\rightarrow \infty} \frac{\int_0^\delta \binom{a}{j} G_X(x)^j (1 - G_X(x))^{a-j}\,dx}{a^{-\frac{1}{\beta}}} = \int_0^\infty \frac{1}{j!}\left(\frac{c}{\beta}\right)^j x^{\beta j}e^{-\frac{c}{\beta}x^\beta}\,dx.
\end{equation}

\paragraph{The second integral.}
We now analyze
\begin{equation}
    a^{\frac{1}{\beta}}\int_\delta^\infty \binom{a}{j} G_X(x)^j (1 - G_X(x))^{a-j}\,dx.
\end{equation}
Observe that
\begin{align}
    a^{\frac{1}{\beta}}\int_\delta^\infty \binom{a}{j} G_X(x)^j (1 - G_X(x))^{a-j}\,dx
    &< a^{\frac{1}{\beta}} \binom{a}{j} \int_\delta^\infty (1 - G_X(x))^{a-j}\,dx\\
    &< a^{\frac{1}{\beta}} \binom{a}{j} (1 - G_X(\delta))^{a-j} \int_\delta^\infty 1 - G_X(x)\,dx\\
    &< a^{\frac{1}{\beta}} \binom{a}{j} (1 - G_X(\delta))^{a-j} \EE[X].
\end{align}
Thus,
\begin{equation}\label{eq:carrot}
    \lim_{a\rightarrow \infty} \frac{\int_\delta^\infty \binom{a}{j} G_X(x)^j (1 - G_X(x))^{a-j}\,dx}{a^{\frac{1}{\beta}}} = 0.
\end{equation}

Combining \eqref{eq:onion} and \eqref{eq:carrot} gives us that
\begin{equation}
\int_0^\infty \binom{a}{j} G_X(x)^j (1 - G_X(x))^{a-j}\,dx \propto a^{-\frac{1}{\beta}},
\end{equation}
as desired.

\paragraph{Proof of \Cref{thm:general}(iii).} Recall again that
\begin{equation}
    h(a) := \sum_{i=1}^{\min\{k, a\}} \mu_\DD(a-i+1,a).
\end{equation}
We show that $h$ is strictly concave and
\begin{equation}\label{eq:mouse-2}
\lim_{a\rightarrow \infty} h(a) - B\log a - C = 0
\end{equation}
for constants $B,C>0$. Both of these facts follow directly from the lemma below. \Cref{thm:general}(iii) then follows immediately by applying \Cref{lem:fennel}(iii).

\begin{lemma}\label{lem:mouse}
For $\DD$ an exponential distribution with rate parameter $\lambda$, so that $f_X(x) = \lambda e^{-\lambda x}$ for $\lambda > 0,$
\begin{equation}\label{eq:papaya}
    \lim_{a\rightarrow \infty} \mu_\DD(a-i,a) - \log a - B(j) = 0
\end{equation}
for a constant $B(j)>0.$ Moreover, $\mu_\DD(a-i,a)$ is strictly concave.
\end{lemma}

\begin{proof}
For an exponential distribution with rate parameter $\lambda,$ it is well known that
\begin{equation}\label{eq:bridge}
    \mu_\DD(a-i,a) = \sum_{j=i+1}^{a} \frac{1}{\lambda n}.
\end{equation}
It is clear, then, that $\mu_\DD(a-i,a)$ is strictly concave. \eqref{eq:bridge}  is equal to
\begin{equation}
    \frac{1}{\lambda}\left(\log n + \gamma + \epsilon(a) - \sum_{j=1}^i \frac{1}{j}\right),
\end{equation}
where $\gamma$ is the Euler-Mascheroni constant and $\lim_{a\rightarrow \infty}\epsilon(a) = 0,$ from which \eqref{eq:papaya} follows.
\end{proof}

\paragraph{Proof of \Cref{thm:general}(iv).}

Recall again that
\begin{equation}
    h(a) := \sum_{i=1}^{\min\{k, a\}} \mu_\DD(a-i+1,a).
\end{equation} Then it suffices to show that $h$ is strictly concave and
\begin{equation}
\lim_{a\rightarrow \infty} \frac{h(a)}{Ba^{\frac{1}{\alpha}}} = 1
\end{equation}
for a constant $B>0.$ Both of these facts follow directly from the lemma below. \Cref{thm:general}(iv) then follows immediately by applying \Cref{lem:fennel}(iv).

\begin{lemma}
For $\DD$ a Pareto distribution with pdf $f_X(x) = x^{-\alpha-1}$ for $\alpha > 1,$
\begin{equation}
    \lim_{a_t\rightarrow \infty} \frac{\mu_\DD(a-i,a)}{a^\frac{1}{\alpha}} = C
\end{equation}
for a constant $C>0.$ Moreover, $\mu_\DD(a-i,a)$ is strictly concave.
\end{lemma}

\begin{proof}
The result follows directly from Lemmas D.10 and D.11 in \cite{kleinberg2018selection}, where it is shown (in our notation) that
\begin{align}
    \lim_{a\rightarrow \infty}\frac{\mu_\DD(a,a)}{a^{\frac{1}{\alpha}}} = \Gamma\left(\frac{\alpha-1}{\alpha}\right)
\end{align}
and
\begin{equation}
    \mu_\DD(a-i,a) = \prod_{j=1}^i \left(1 - \frac{1}{j\alpha}\right)\mu_\DD(a,a).
\end{equation}
\end{proof}

Thus,
\begin{equation}
    \lim_{a\rightarrow \infty} \frac{\sum_{i=1}^{k} \mu_\DD(a-i+1,a)}{B \log a} = 1
\end{equation}
for a constant $B$. Also, note that $\mu_\DD(a-i,a)$ is a constant multiple of $\mu_\DD(a,a)$, and that $\mu_\DD(a,a)$ is strictly concave, since the mean of the largest order statistic of a distribution is strictly concave in sample size. Thus, $\mu_\DD(a-i,a)$ is strictly concave.


\subsection{Proof of \Cref{thm:ber-decay}}\label{sec:proof-thm2}

Suppose that for each fixed $i$, $X_i^{(t)}$ are i.i.d. Bernoulli random variables with success probability $q_i$ such that $q_i= c(i+d)^{-\beta}$ for some $\beta,c,d\ge 0.$ We begin by proving parts (i),(ii), and (iii), wherein the user only utilizes the highest value recommended item. In the Bernoulli setting, this amounts to determining the probability that \textit{at least one} movie is satisfactory, i.e.,
\begin{equation}
    \sum_{t=1}^m p_t h(a_t),
\end{equation}
where
\begin{align}\label{eq:grass}
    h(a) = 1 - \prod_{i=1}^{a} (1 - c(i+d)^{-\beta}).
\end{align}
It suffices now to show the desired asymptotic properties for $h$ depending on $\beta$, and applying \Cref{lem:fennel}.

We note the following fact, which will be helpful in our analysis:
\begin{equation}\label{eq:exp-bound}
    1-x>e^{-x-x^2}\quad \text{for $0< x < \frac{1}{2}$}.
\end{equation}

\subsubsection{Proof of \Cref{thm:ber-decay}(i)}
We now consider the case $0\le \beta < 1$. We analyze
\begin{equation}
    1 - h(a) = \prod_{i=1}^{a} (1 - c(i+d)^{-\beta}) = \prod_{i=d+1}^{a+d} (1 - ci^{-\beta}),
\end{equation}
and bound it from above and below. Let $i'$ be the smallest $i'$ such that $c(i')^{-\beta}<\frac{1}{2}$. Bounding from above,
\begin{align}
\prod_{i=i'+1}^{a+d} (1 - ci^{-\beta})
&< \prod_{i=i'+1}^{a+d} e^{-ci^{-\beta}}\\
&= \exp \left[\sum_{i=i'}^{a+d} -ci^{-\beta}\right]\\
&< \exp \left[\int_{i'}^{a+d+1} -cx^{-\beta}\,dx\right]\\
&= \exp \left[ -\left[ \frac{cx^{1-\beta}}{1-\beta} \right]_{i'}^{a+d+1}\right]\\
&= \exp \left[-\frac{c}{1-\beta} \left((a+d+1)^{1-\beta}-(i')^{1-\beta}\right)\right]
\end{align}
Now, bounding from below,
\begin{align}
\prod_{i=i'}^{a+d} (1 - ci^{-\beta})
&> \prod_{i=i'}^{a+d} e^{-ci^{-\beta}-c^2i^{-2\beta}}\\
&= \exp \left[\sum_{i=i'}^{a+d} -ci^{-\beta}-c^2i^{-2\beta}\right]\\
&> \exp \left[\int_{i'-1}^{a+d} -cx^{-\beta}-c^2x^{-2\beta}\,dx\right]\\
&= \exp \left[ -\left[ \frac{cx^{1-\beta}}{1-\beta} + \frac{c^2x^{2-\beta}}{2-\beta} \right]_{i'-1}^{a+d}\right]\\
&= \exp \left[-\frac{c}{1-\beta} \left((a+d)^{1-\beta}-(i'-1)^{1-\beta}\right) -\frac{c^2}{1-2\beta} \left((a+d)^{2-\beta}-(i'-1)^{2-\beta}\right)\right],
\end{align}
where the first inequality follows from \eqref{eq:exp-bound}. Then observing that
\begin{equation}
    1 - h(a) = \prod_{i=d+1}^{i'-1} (1 - ci^{-\beta})\cdot \prod_{i=i'}^{a+d} (1 - ci^{-\beta}),
\end{equation}
where the first product is constant in $a$, it follows that
\begin{equation}
    \lim_{a \rightarrow \infty} \frac{\log (1-h(a))}{-\frac{c}{1-\beta}a^{1-\beta}} = 1,
\end{equation}
as desired. The result follows from \Cref{lem:fennel}(i).

\subsubsection{Proof of \Cref{thm:ber-decay}(ii)}
We turn to the case $\beta=1$. We have that
\begin{align}
    1-h(a) = \prod_{i=d+1}^{a+d} \left(1 - \frac{c}{i}\right) &= \frac{d+1-c}{d}\cdot \frac{d+2-c}{d+1}\cdot \ldots \cdot \frac{a+d-c}{a+d},
\end{align}
which telescopes to
\begin{equation}
    \prod_{i=1}^c \frac{d+i-c}{a+d-i+1}.
\end{equation}
Now note that
\begin{equation}
    \lim_{a\rightarrow \infty} \frac{\prod_{i=1}^c \frac{d+i-c}{a+d-i+1}}{a^{-c}} = \prod_{i=1}^c (d+i-c)
\end{equation}
is a finite constant. Therefore,
\begin{equation}
    \lim_{a\rightarrow \infty} \frac{1 - h(a)}{Ba^{-c}} = 1
\end{equation}
for constant $B$.


\subsubsection{Proof of \Cref{thm:ber-decay}(iii)}
We now consider the case $\beta > 1$. In this case, we will again bound \eqref{eq:grass} from above and below. First note that
\begin{equation}
    1 - h(a) = \prod_{i=d+1}^{\infty} (1 - ci^{-\beta}) = S
\end{equation}
for a finite constant $S$.

We have that
\begin{align}
\prod_{i=a+d+1}^{\infty} (1 - ci^{-\beta})
&< \prod_{i=a+d+1}^{\infty} e^{-ci^{-\beta}}\\
&= \exp \left[\sum_{i=a+d+1}^{\infty} -ci^{-\beta}\right]\\
&< \exp \left[\int_{a+d+1}^{\infty} -cx^{-\beta}\,dx\right]\\
&= \exp \left[ -\left[ \frac{cx^{1-\beta}}{1-\beta} \right]_{a+d+1}^{\infty}\right]\\
&= \exp \left[-\frac{c}{1-\beta} (a+d+1)^{1-\beta}\right]
\end{align}
Therefore,
\begin{equation}
    \prod_{i=1}^{a+d} (1 - ci^{-\beta}) = \frac{S}{\prod_{i=a+d+1}^{\infty}} (1 - ci^{-\beta}) > S / \exp \left[-\frac{c}{1-\beta} (a+d+1)^{1-\beta}\right].
\end{equation}

Also,
\begin{align}
\prod_{i=a+d+1}^{\infty} (1 - ci^{-\beta})
&> \prod_{i=a+d+1}^{\infty} e^{-ci^{-\beta}-c^2i^{-2\beta}}\\
&= \exp \left[\sum_{i=a+d+1}^{\infty} -ci^{-\beta}-c^2i^{-2\beta}\right]\\
&< \exp \left[\int_{a+d}^{\infty} -cx^{-\beta}-c^2x^{-2\beta}\,dx\right]\\
&= \exp \left[ -\left[ \frac{cx^{1-\beta}}{1-\beta} + \frac{c^2x^{2-\beta}}{2-\beta} \right]_{a+d}^{\infty}\right]\\
&= \exp \left[-\frac{c}{1-\beta} (a+d)^{1-\beta} -\frac{c^2}{1-2\beta} (a+d)^{2-\beta}\right],
\end{align}
where the first inequality holds for $a$ sufficiently large due to \eqref{eq:exp-bound}.

Therefore,
\begin{equation}
    \prod_{i=1}^{a+d} (1 - ci^{-\beta}) = \frac{S}{\prod_{i=a+d+1}^{\infty}} (1 - ci^{-\beta}) < S / \exp \left[-\frac{c}{1-\beta} (a+d-1)^{1-\beta} -\frac{c^2}{1-2\beta} (a+d)^{2-\beta}\right].
\end{equation}

Now observe that
\begin{equation}
    \lim_{a\rightarrow \infty} -\frac{c}{1-\beta} (a+d+1)^{1-\beta} = 0
\end{equation}
and
\begin{equation}
    \lim_{a\rightarrow \infty} -\frac{c}{1-\beta} (a+d-1)^{1-\beta} -\frac{c^2}{1-2\beta} (a+d)^{2-\beta} = 0.
\end{equation}

Therefore,
\begin{equation}
    \lim_{a\rightarrow \infty} \frac{\prod_{i=1}^{a+d} (1 - ci^{-\beta})}{\frac{S}{1 - \frac{c}{1-\beta} (a+d+1)^{1-\beta}}} = 1.
\end{equation}
Also note that
\begin{equation}
    \frac{S}{1 - \frac{c}{1-\beta} (a+d+1)^{1-\beta}}
    = S - \frac{S \frac{c}{1-\beta}}{(a+d+1)^{\beta-1} + \frac{c}{1-\beta}}
\end{equation}

Also,
\begin{equation}
    \lim_{a\rightarrow \infty} \frac{\prod_{i=1}^{a+d} (1 - ci^{-\beta})}{ \frac{S}{1 - \frac{c}{1-\beta} (a+d-1)^{1-\beta} -\frac{c^2}{1-2\beta} (a+d)^{2-\beta}}} = 1
\end{equation}
and
\begin{equation}
    \lim_{a\rightarrow \infty} \frac{1 - \frac{c}{1-\beta} (a+d)^{1-\beta} -\frac{c^2}{1-2\beta} (a+d)^{2-\beta}}{1 - \frac{c}{1-\beta} (a+d)^{1-\beta}} = 1.
\end{equation}

It follows that
\begin{equation}
    \lim_{a\rightarrow \infty} \frac{1 - S - h(a)}{-\frac{Sc}{1-\beta}a^{1-\beta}} = 1
\end{equation}
as desired. The result follows from \Cref{lem:fennel}(ii).

\subsubsection{Proof of \Cref{thm:ber-decay}(iv)}
We now prove part (iv), where we consider the expected total value of all recommended items, which is equal to
\begin{equation}
    \sum_{t=1}^{m} p_t h(a_t),
\end{equation}
where
\begin{equation}
    h(a) = \sum_{i=1}^{a} c(i+d)^{-\beta}.
\end{equation}
Again, we consider three cases, $0< \beta < 1, \beta=1, \beta > 1.$

\paragraph{Case 1: $0 < \beta < 1$.} For $0 <\beta < 1,$ observe that
\begin{align}
    \sum_{i=1}^{a} c(i+d)^{-\beta} < \int_{d}^{a+d} cx^{-\beta}\,dx &= \left[\frac{c}{1-\beta}x^{1-\beta}\right]_{d}^{a+d}\\
    &= \frac{c}{1-\beta}(a+d)^{1-\beta} - \frac{c}{1-\beta}d^{1-\beta}
\end{align}
and
\begin{align}
    \sum_{i=1}^{a} c(i+d)^{-\beta} > \int_{d+1}^{a+d+1} cx^{-\beta}\,dx &= \left[\frac{c}{1-\beta}x^{1-\beta}\right]_{d+1}^{a+d+1}\\
    &= \frac{c}{1-\beta}(a+d+1)^{1-\beta} - \frac{c}{1-\beta}(d+1)^{1-\beta}.
\end{align} 
It follows that
\begin{align}
\lim_{a\rightarrow \infty}\frac{h(a)}{a^{1-\beta}} = \frac{c}{1-\beta},
\end{align} 
and the result in this case follows by applying \Cref{lem:fennel}(iv).

\paragraph{Case 2: $\beta = 1$.} Now for $\beta=1,$ we have that
\begin{align}
    \sum_{i=1}^{a} c(i+d)^{-\beta} = c\sum_{i=d+1}^{a+d} \frac{1}{i}.
\end{align}
\begin{align}
\lim_{a\rightarrow \infty}h(a) - c\log a + c\gamma - c\sum_{i=1}^d \frac{1}{i} = 0,
\end{align} 
where $\gamma$ is the Euler-Mascheroni constant The result in this case follows by applying \Cref{lem:fennel}(iii).

\paragraph{Case 3: $\beta > 1$.} Finally, for $\beta>1,$ we have that 
\begin{equation}
     \sum_{i=1}^{\infty} c(i+d)^{-\beta} = S
\end{equation}
for some finite $S$. Then note that
\begin{equation}
    \sum_{i=1}^{a} c(i+d)^{-\beta} = S - \sum_{i=a+1}^{\infty} c(i+d)^{-\beta}.
\end{equation}
Then we have
\begin{align}
    \sum_{i=a+1}^{\infty} c(i+d)^{-\beta} < \int_{a}^{\infty} cx^{-\beta}\,dx &= \left[\frac{c}{1-\beta}x^{1-\beta}\right]_{a}^{\infty}
    = - \frac{c}{1-\beta}a^{1-\beta}
\end{align}
and
\begin{align}
    \sum_{i=a+1}^{\infty} c(i+d)^{-\beta} > \int_{a+1}^{\infty} cx^{-\beta}\,dx &= \left[\frac{c}{1-\beta}x^{1-\beta}\right]_{a+1}^{\infty}
    = - \frac{c}{1-\beta}(a+1)^{1-\beta}
\end{align}
It follows that
\begin{equation}
    \lim_{a\rightarrow \infty}\frac{h(a)}{S - \frac{c}{\beta - 1}a^{1-\beta}} = 1.
\end{equation}
The result in this case follows again by applying \Cref{lem:fennel}(ii).

\subsection{A rounding lemma}\label{sec:proof-rounding-lemma}
The following lemma is useful for showing that---for the class of problems we consider here---optimal integer solutions are well-approximated by optimal real solutions.

\begin{lemma}
\label{lem:roundingloss}
Let $g_1,g_2,\cdots,g_m: [0,\infty)^m \rightarrow \RR$ be strictly convex functions over the non-negative reals. Then define
\begin{equation}
    g(x_1,\cdots,x_m):=\sum_{t=1}^m g_t(x_t).
\end{equation}
Then, under the constraint that $\sum_{t=1}^m x_t = n$, if $(x_1^*,\cdots,x_m^*)$ is the maximum of $g$ over the non-negative reals and $(a_1^*,\cdots,a_m^*)$ is the maximum of $g$ over the non-negative integers, then
\begin{equation}
\lfloor x_t^* \rfloor - m < a_t^* < \lfloor x_t^* \rfloor + m
\end{equation}
for all $t$.
\end{lemma}

\begin{proof}
The key idea is to show that there cannot be $i,j$ such that $a_i^*\ge \lceil x_i \rceil + 1$ and $a_j^*\le \lceil x_j \rceil - 1.$ If such a pair does exist, we show that
\begin{equation}
g(\cdots,a_i^*-1,\cdots,a_j^*+1,\cdots) \ge g(\cdots,a_i^*,\cdots,a_j^*,\cdots),
\end{equation}
contradicting the optimality of $a_1^*,\cdots,a_m^*.$ It suffices to show that
\begin{equation}
g_i(a_i^*-1) + g_j(a_j^*+1)\ge g_i(a_i^*) + g_j(a_j^*),
\end{equation}
or equivalently,
\begin{equation}
g_j(a_j^*+1) - g_j(a_j^*)\ge g_i(a_i^*) - g_i(a_i^*-1).
\end{equation}
This holds, as
\begin{align}
   g_j(a_j^*+1) - g_j(a_j^*) \ge \frac{\partial g_j}{\partial x_j} = \frac{\partial g_i}{\partial x_i} \ge g_i(a_i^*) - g_i(a_i^*-1).
\end{align}
\end{proof}


\subsection{Proof of \Cref{prop:uniform}}\label{sec:proof-of-uniform}
 For ease of exposition, we will begin by proving the result for $k=1$ and then handle the more general case.
\paragraph{First case ($k=1$).} Let us first consider the case $k=1.$ We would like to find an ordered tuple of non-negative integers $(a_1,\cdots,a_m)$ that maximizes
\begin{equation}
\sum_{t=1}^m p_t\mu_\DD(a_t,a_t)=\sum_{t=1}^m p_t\left(1 - \frac{1}{a_t+1}\right)
\end{equation}
subject to the constraint $\sum_{t=1}^m a_t = n.$
Our strategy will be to solve the relaxed optimization problem over non-negative reals, and then show that the optimal integer solution is ``close to'' the optimal real solution. Consider the function $g:[0,\infty)^m \rightarrow \RR$, where
\begin{equation}
g(x_1,\cdots,x_m) = \sum_{t=1}^m p_t\left(1 - \frac{1}{x_t+1}\right).
\end{equation}
Subject to the constraint $\sum_{t=1}^m x_t = n,$ $g(x_1,\cdots,x_m)$ is maximized exactly when 
\begin{equation}
\frac{\partial g}{\partial x_1} = \frac{\partial g}{\partial x_2} = \cdots = \frac{\partial g}{\partial x_m}.
\end{equation}
This is clear after noting that $g$ is convex on its domain and applying Lagrange multipliers. We have that
\begin{equation}
\frac{\partial g}{\partial x_t} = p_t\left(\frac{1}{x_t+1}\right)^2.
\end{equation}
So if $(x_1^*,\cdots,x_m^*)$ is a maximum, then $x_t^*+1 \propto \sqrt{p_t},$ meaning that
\begin{equation}
x_t^* = \left(\frac{\sqrt{p_t}}{\sum_{i=1}^m \sqrt{p_i}}\right)n-1.
\end{equation}
We now apply \Cref{lem:roundingloss}: if $(a_1^*,\cdots,a_m^*)$ is the optimal integer solution, then $|a_t^*-x_t^*|\le m$. Thus,
\begin{equation}
    \left|a_t^* - \frac{\sqrt{p_t}}{\sum_{i=1}^m \sqrt{p_i}}n\right|\le m+1
\end{equation}
for all $n$.

\paragraph{General case $k$.}  More generally, we would like to maximize
\begin{equation}
\sum_{t=1}^m p_t \sum_{i=1}^{\min\{a_t,k(n)\}} \mu_\DD(a_t-i+1, a_t).
\end{equation}
subject to the constraint $\sum_{t=1}^m a_t = n,$ where we let $k=k(n)$ be a function of $n$.
We first analyze each case of the inner summation:
\begin{equation}
\sum_{i=1}^{\min\{a_t,k(n)\}} \mu_\DD(a_t,a_t-i+1) =
\begin{cases}
\frac{a_t}{2} &\quad a_t\le k(n)\\
\sum_{i=1}^{k(n)} 1 - \frac{i}{a_t+1} = k(n) - \frac{k(n)^2 + k(n)}{2(a_t+1)}& \quad a_t> k(n)
\end{cases}.
\end{equation}
The top case $a_t \leq k(n)$ follows from all $a_t$ items of that type contributing to the objective, each with a mean value of $\frac12$.

Then define
\begin{equation}
g:[0,\infty)^m \rightarrow \RR,\quad
(x_1,\cdots,x_m)\mapsto \sum_{t=1}^m p_th(x_t)
\end{equation}
where
\begin{equation}
    h(x):=\begin{cases}
\frac{x}{2} &\quad x\le k(n)\\
\sum_{i=1}^{k(n)} 1 - \frac{i}{x+1} = k(n) - \frac{k(n)^2 + k(n)}{2(x+1)}& \quad x> k(n)
\end{cases}.
\end{equation}

So we have
\begin{equation}
\frac{\partial g}{\partial x_t} = 
\begin{cases}
\frac{p_tx}{2} &\quad x_t\le k(n)\\
p_t(k(n)^2 + k(n))\left(\frac{1}{x_t + 1}\right)^2 &\quad x_t> k(n)
\end{cases}.
\end{equation}
We first consider possible solutions where $x_t > k(n)$ for all $t\in[m].$ In this case,
\begin{equation}
\frac{\partial g}{\partial x_t} = p_t(k(n)^2 + k(n))\left(\frac{1}{x_t + 1}\right)^2
\end{equation}
is equal for all $t$ at the maximum, as before. Remarkably, the new $k(n)^2 + k(n)$ term drops out. So if $(x_1^*,\cdots,x_m^*)$ is the optimal real solution, again,
$x_t^*+1 \propto \sqrt{p_t},$ and the result follows as in the first case. Here, $x_t^* > k(n)$ holds whenever $k\le \frac{\sqrt{p_m}}{\sum_{i=1}^m \sqrt{p_i}}n - m - 1.$



\subsection{Proof of \Cref{thm:ber-varying}}\label{sec:proof-thm3}
\begin{proof}
We would like to find $a_1,\cdots,a_m$ that maximizes
\begin{equation}
\sum_{t=1}^m p_t \left(1 - (1 - q_t)^{a_t}\right) = 1 - \sum_{t=1}^m p_t(1-q_t)^{a_t}.
\end{equation}
subject to the constraint $\sum_{t=1}^m p_t = n.$ This is equivalent to minimizing
\begin{equation}
\sum_{t=1}^m p_t(1-q_t)^{a_t}.
\end{equation}
Now define a function
\begin{equation}
    g:[0,\infty)^m\rightarrow \RR,\quad (x_1,\cdots,x_m)\mapsto \sum_{t=1}^m p_t(1-q_t)^{x_t}.
\end{equation}
Subject to the constraint $\sum_{t=1}^m x_t = n,$ $g(x_1,\cdots,x_m)$ is maximized exactly when 
\begin{equation}
\frac{\partial g}{\partial x_1} = \frac{\partial g}{\partial x_2} = \cdots = \frac{\partial g}{\partial x_m}.
\end{equation}
We have
\begin{equation}
    \frac{\partial g}{\partial x_t} = -p_t(1-q_t)^{x_t}\log(1-q_t).
\end{equation}
Solving $\partial g/\partial x_i = \partial g/\partial x_j$ gives
\begin{align}
    p_i(1-q_i)^{x_i}\log(1-q_i) &= p_j(1-q_j)^{x_j}\log(1-q_j)\\
    \implies \log p_i + x_i\log(1-q_i) + \log\log(1-q_i) &= \log p_j + x_j\log(1-q_j) + \log\log(1-q_j)
\end{align}
It follows that $a_t\propto \frac{1}{\log(1-q_t)}$ for all $t$, where we have once again applied \Cref{lem:roundingloss}.
\end{proof}