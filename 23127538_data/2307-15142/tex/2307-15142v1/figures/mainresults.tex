\begin{table}
    \begin{center}
    \caption{Our main results broadly show: (1) consumption constraints induce diverse recommendations, and (2) without accounting for consumption constraints, recommendations tend to be homogeneous.}
    \label{table:mainresults}
      \vspace*{3mm}
      \small
\begin{tabular}{@{}>{\raggedright}p{4.3cm}p{4.3cm}p{4.3cm}}
\toprule
\multicolumn{1}{c}{\textbf{Setting}} & \multicolumn{1}{c}{\textbf{With consumption constraints}} & \multicolumn{1}{c}{\textbf{Without consumption constraints}} \\
\midrule
\textbf{Thm. 1.} 
\begin{equation*}
    X_i^{(t)}\sim \DD
\end{equation*} & As $n$ grows large for fixed $k$, $S_{n,k}$ exhibits diversity depending on the tail behavior of $\DD$. \vspace{0.1cm} \newline For non-heavy-tailed $\DD$, $S_{n,k}$ is at least proportionally diverse. & $S_{n,n}$ contains only one type of item.\\ & & \\
\textbf{Thm. 2.}
\begin{equation*}
    X_i^{(t)}\sim \Ber(q_i)
\end{equation*}for $q_1,q_2,\cdots$ decaying by a power law (roughly, $q_i\propto i^{-\alpha}$). & For moderate amounts of decay $(\alpha < 1)$, as $n$ grows large, $S_{n,1}$ represents each item type equally. & For moderate amounts of decay $(\alpha < 1)$, as $n$ grows large, $S_{n,n}$ is less than proportionally diverse.\\
& & \\
\textbf{Thm. 3.}
\begin{equation*}
    X_i^{(t)}\sim \Ber(q_t)
\end{equation*} for $q_1,q_2,\cdots,q_m$. & For large $n$, $S_{n,1}$ represents each item type in proportion to \begin{equation*}
    \frac{1}{\log \frac{1}{1-q_t}},
\end{equation*}
so that items of \textit{lower} success probability are recommended \textit{more}. & $S_{n,n}$ contains only one type of item.\\
\bottomrule
\end{tabular}
\end{center}
\end{table}