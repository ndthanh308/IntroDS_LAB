\begin{table}\label{table:thmgeneral}
    \begin{center}
    \caption{A summary of Theorem 1. For $X_i^{(t)}\simiid \DD$, distributions $\DD$ with heavier tails induce less diversity.}
    \label{table:thmgeneral}
      \vspace*{3mm}
      \begin{tabular}{l c c c c c}
        \toprule %
        \textbf{} & \multicolumn{3}{c}{\textbf{bounded}} & \textbf{exp. tail} & \textbf{heavy tail}\\
        & \multicolumn{3}{c}{Thm. 1(ii)} & Thm. 1(iii) & Thm. 1(iv)\\
        \cmidrule(r){2-4}
        \cmidrule(r){5-5}
        \cmidrule(r){6-6}
        example $\DD$ &  & $\betaa(\cdot,\beta)$ &  & $\Exp(\lambda)$ & $\Pareto(\alpha)$\\
         & \small{$0<\beta<1$} & \small{$\beta=1$} & \small{$\beta>1$} & \small{$\lambda>0$} & \small{$\alpha>1$}\\        
% Figure removed
&% Figure removed

& % Figure removed

& % Figure removed 

&% Figure removed

&% Figure removed
\\
$\{S_{n,k}\}_{n=1}^\infty$\\$\gamma$-homog.\\for $\gamma\in$ & $(0,1/2)$ & $1/2$ & $(1/2,1)$ & $1$ & $(1,\infty)$\\
&&&&(i.e., proportional)&\\
&&&&&\\
&\multicolumn{5}{c}{$\longleftarrow$ more diverse \qquad \qquad \qquad \qquad \qquad less diverse $\longrightarrow$}\\
\midrule
\end{tabular}
\end{center}
\end{table}