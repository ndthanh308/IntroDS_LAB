%\documentclass[prd,twocolumn,showpacs,superscriptaddress,groupedaddress,nofootinbib,floatfix]{revtex4-1}
\documentclass[prd,aps,twocolumn,a4paper,showkeys,nofootinbib,floatfix]{revtex4-1}

\usepackage{graphicx,psfrag}
\usepackage{mathrsfs}
\usepackage{amsmath,amsfonts,amssymb}
\usepackage{multirow}
\usepackage{comment}
\usepackage{ulem} % \sout{} remove in final version
\usepackage{hyperref}
\usepackage{enumitem}
\usepackage{morefloats}
\usepackage{bm}		% package for bold fonting greek letters in math mode

\newcommand{\be}{\begin{equation}}
\newcommand{\ee}{\end{equation}}
\newcommand{\bel}{\begin{align}}
\newcommand{\eel}{\end{align}}

\def\lm{\ell m}
\def\p{\partial}
\def\non{\nonumber}                     
\def\e{{\rm e}}
\def\i{{\rm i}}
\def\f{{\rm f,~BH}}
\def\gccm{{\rm g\,cm^{-3}}}
\def\Msun{{M_{\odot}}}
\def\Mpc{{\rm Mpc}}
\def\GMc2{{\rm G M_{\odot} c^{-2}}}
\def\Mpc{{\rm Mpc}}
\def\eps{\epsilon}
\def\eps{\epsilon}
\def\B{\mathcal{B}}
\def\I{\mathcal{I}}
\def\M{\mathcal{M}}
\def\O{\mathcal{O}}
\def\R{\mathcal{R}}
\def\vareps{\varepsilon}
\def\vrho{\varrho}
\def\check{$\checkmark$}
\def\cross{$\times$}
\def\l{\ell}
\def\lm{{\ell m}}
\def\hE{\hat{E}}
\def\mns{M_\text{NS}}
\def\Rns{R_\text{NS}}
\def\kt2{\kappa^\text{T}_2}
\def\Mmax{M_\text{TOV}^\text{max}}
\def\Mo{{\rm M_{\odot}}}
\def\kt2{\kappa^\text{T}_2}
\def\Mmax{M_\text{max}}
\def\Rmax{R_\text{max}}
\def\Rmax{R_\text{max}}

\def\params{{\boldsymbol{\theta}}}
\def\F{\mathcal{F}}
\def\Fbar{\bar{\mathcal{F}}}
\def\MM{\Fbar}

\newcommand{\rdot}{\dot{r}}
\newcommand{\pr}{p_{r}}
\newcommand{\mpd}{\mu_{+,2}}
\newcommand{\mmd}{\mu_{-,2}}
\newcommand{\spd}{\sigma_{+,2}}
\newcommand{\smd}{\sigma_{-,2}}
\newcommand{\rLR}{r_{\rm LR}}


\newcommand{\nrpm}{\texttt{NRPM}}
\newcommand{\TEOB}[1]{\texttt{TEOBResumS{#1}}}
\newcommand\gsftides[1]{{\rm GSF{#1}}$^{(+)}${\rm PN}$^{(-)}$}

\usepackage{color}
\definecolor{cyan}{rgb}{0,0.9,0.9}
\definecolor{orange}{rgb}{0.9,0.5,0}
\definecolor{magenta}{rgb}{1,0,1}
\definecolor{purple}{rgb}{0.8,0.4,0.8}
\definecolor{gray}{rgb}{0.5,0.5,0.5}
\newcommand{\rg}[1]{{\textcolor{orange}{\texttt{RG: #1}} }}
\newcommand{\mb}[1]{{\textcolor{blue}{\texttt{MB: #1}} }}
\newcommand{\bs}[1]{{\textcolor{green}{\texttt{SB: #1}} }}
\newcommand{\an}[1]{{\textcolor{red}{\texttt{AN: #1}} }}
\newcommand{\wc}[1]{{\textcolor{cyan}{\texttt{WC: #1}} }}


\newcommand{\todo}[1]{{\textcolor{red}{TODO: [#1]}}} 
\newcommand{\red}[1]{{\textcolor{red}{#1}}} 
\newcommand{\newtxt}[1]{{\textcolor{red}{#1}}} 
\newcommand{\oldtxt}[1]{{\textcolor{gray}{\sout{#1}}}} 
\newcommand{\oldnewtxt}[2]{{\textcolor{gray}{\sout{#1}}}\red{#2}} 
\newcommand{\timesto}[1]{\times 10^{#1}}

\begin{document}

\title{Analytically improved and numerical-relativity informed effective-one-body model\\ for coalescing binary neutron stars}

\author{Rossella \surname{Gamba}${}^{1}$}
\author{Matteo \surname{Breschi}${}^{1,2}$}
\author{Sebastiano \surname{Bernuzzi}${}^{1}$}
\author{Alessandro \surname{Nagar}${}^{3,4,5}$}
\author{William \surname{Cook}${}^{1}$}
\author{Georgios \surname{Doulis}${}^{1}$}
\author{Francesco \surname{Fabbri}${}^{1}$}
\author{Néstor \surname{Ortiz}${}^{6}$}
\author{Amit \surname{Poudel}${}^{7}$}
\author{Alireza \surname{Rashti}${}^{8,9}$}
\author{Wolfgang \surname{Tichy}${}^{7}$}
\author{Maximiliano \surname{Ujevic}${}^{10}$}

\affiliation{${}^1$Theoretisch-Physikalisches Institut, Friedrich-Schiller-Universit{\"a}t Jena, 07743, Jena, Germany}  
\affiliation{${}^2$Theoretical and Scientific Data Science group, Scuola Internazionale Superiore di Studi Avanzati (SISSA), via Bonomea 265, 34136 Trieste Italy}
\affiliation{${}^3$Dipartimento di Fisica, Universit\`a di Torino, Torino, 10125, Italy}
\affiliation{${}^4$INFN sezione di Torino, Torino, 10125, Italy}
\affiliation{${}^5$Institut des Hautes Etudes Scientifiques, 35 Route de Chartres, Bures-sur-Yvette, 91440, France}
\affiliation{${}^6$Instituto de Ciencias Nucleares, Universidad Nacional Aut\'onoma de México, Circuito Exterior C.U., A.P. 70-543, México D.F. 04510, México}
\affiliation{${}^7$Department of Physics, Florida Atlantic University, Boca Raton, FL 33431, USA}
\affiliation{${}^8$Institute for Gravitation \& the Cosmos, The Pennsylvania State University, University Park, PA 16802, USA}
\affiliation{${}^9$Department of Physics, The Pennsylvania State University, University Park, PA 16802, USA}
\affiliation{${}^{10}$Centro de Ci\^encias Naturais e Humanas, Universidade Federal do ABC, Santo André 09210-170, SP, Brazil}
%%
\date{\today}

\begin{abstract}
Gravitational wave astronomy pipelines rely on template waveform models for searches and parameter estimation purposes.
For coalescing binary neutron stars (BNS), such models need to accurately reproduce numerical relativity (NR) up to merger, 
in order to provide robust estimate of the stars' equation of state - dependent parameters.
%
In this work we present an improved version of the Effective One Body (EOB) model 
\TEOB{} for gravitational waves from BNS systems. Building upon recent post-Newtonian calculations, we include subleading 
order tidal terms in the waveform multipoles and EOB metric potentials, as well as add up to 5.5PN terms in the gyro-gravitomagnetic 
functions entering the spin-orbit sector of the model. In order to further improve the EOB-NR agreement in the last few orbital cycles before merger, 
we introduce next-to-quasicircular corrections in the waveform -- informed by a large number of BNS NR simulations -- and introduce a new NR-informed parameter entering the
tidal sector of our conservative dynamics.
%
The performance of our model is then validated against 14 new eccentricity reduced simulations of unequal mass, spinning
binaries with varying equation of state. A time-domain phasing analysis and mismatch computations demonstrate
that the new model overall improves over the previous version of \TEOB{}.
%
Finally, we present a closed-form frequency domain representation of the (tidal) amplitude and phase of the new \TEOB{} model. 
This representation accounts for mass-ratio, aligned spin and (resummed) spin-quadrupole effects in the tidal phase and 
-- within the calibration region -- it is faithful to the original model.
\end{abstract}

\maketitle



\section{Introduction}

Gravitational Waves (GWs) from coalescing binary neutron stars (BNS)
carry information on the stars' internal structure and composition, i.e. their equation of state (EOS)~\cite{Flanagan:2007ix,Hinderer:2007mb}. Such information is mainly, but not exclusively,
encoded in the tidal parameters of the stars themselves, which describe
the tidal response of a body due to the external gravitational field of its companion~\cite{Binnington:2009bb,Damour:2009vw}.
Precise measurements of the tidal parameters of neutron stars 
are a key science goal for current and next generation detectors~\cite{Radice:2017lry, Raithel:2019uzi, Capano:2019eae, Breschi:2021xrx, Puecher:2022oiz}.
As such, it is fundamental that waveform models -- which are extensively
employed during parameter estimation (PE) to extract the source properties from the data -- give faithful representations of the entire coalescence up to 
merger. Comparisons between current state of the art models (also called approximants, as they provide approximate solutions to the general relativistic two body problem) and Numerical relativity (NR) simulations, 
however, highlight that approximants do not correctly reproduce NR waveforms in the last stages of the inspiral, when matter contributions are best measured and their impact is the largest \cite{Gamba:2020wgg, Dudi:2021wcf, Gamba:2022mgx}. 
Numerous studies have shown that the imperfect modeling of matter effects will have large repercussions on PE with next-generation (XG) detectors, 
with waveform systematics that could potentially already be relevant for signals detected in the next LIGO-Virgo-Kagra observing run, O4.

% discussion on current state of the art models for BNS:
% phenom_Tidal
% SEOB
% TEOB

In this paper we improve the tidal sector of the effective one body (EOB) model \TEOB~\cite{Akcay:2018yyh,Nagar:2018plt,Nagar:2020pcj,Gamba:2021ydi}, and provide a phenomenological representation of it for spinning BNS systems.
The improvements concern (i) the computation and inclusion of additional higher order analytical information in the metric and waveform multipoles, and 
(ii) the inclusion of NR information via next to quasi-circular (NQC) parameters and through an additional NR-calibrated parameter, 
which enters the tidal part of the radial metric potential. For equal mass binaries, this parameter is clearly correlated with the effective tidal parameter of the system, 
and given a large enough number of high-quality simulations can be fit directly to NR. 

The paper is structured as follows. In Sec.~\ref{sec:eobmodel} we describe the structure of our model and the analytical improvements that we consider. 
In Sec.~\ref{sec:nrinfo} we discuss the NR information added and show comparisons between our model and few high resolution simulations, used 
as calibration set. 
In Sec.~\ref{sec:nrcomp} we validate the resulting model via additional comparisons against a set of new eccentricity reduced NR simulations characterized by rather extreme parameters, such as unequal masses and spins. In Sec.~\ref{sec:tidal} we construct a phenomenological representation of our model in the frequency domain, 
which can easily be added to any point mass binary black hole (BBH) model to reproduce the tidal sector of \TEOB{}.
Finally, in Sec.~\ref{sec:conc} we summarize and discuss our results.

Unless otherwise specified, we work in geometric units with $G=c=1$.
We label with $A, B$ the two stars and denote with $M_A, M_B$ the masses of the component stars, $M = M_A+M_B$ is the total mass of the system, 
$q\geq1$ is its mass ratio, $X_A = M_A/M$ and $X_B = M_B/M$ are the mass fractions, $\chi_A$ and $\chi_B$ are the (dimensionless) $z$ components of the 
spin angular momenta $S_A, S_B$, $\nu = X_A X_B$ is the symmetric mass ratio.

We further denote the electric-type, $\ell$-th multipole coefficient of the body $A$ as:
\begin{equation}
\mu_\ell^{A} = 2 \frac{k_{\ell}^A}{(2 \ell - 1)!!} R^{2 \ell + 1}_{A} \, ,
\end{equation}
where $k_{\ell}^A$ is the $\ell$-th electric Love number of body $A$, and $R_A$ is the radius of body $A$. 
The dimensionless tidal parameters $\Lambda_{\ell}$ are defined as
\begin{equation}
\Lambda_{\ell}^A = \frac{\mu^A_\ell}{M_A^{2 \ell +1}}  =  \frac{2}{(2 \ell +1)!!} k_{\ell} \mathcal{C}_A^{-(2 \ell +1)} \, ,
\end{equation}
where $\mathcal{C}_A=R_A/M_A$ is the compactness of body $A$.
Note that the usual NS parameter $\Lambda_A$ employed in GW data analysis refers to the $\ell = 2$ dimensionless tidal parameter above.
The effective tidal parameter $\tilde\Lambda$ is then obtained from the component stars' tidal parameters via
\be
\tilde\Lambda = \frac{16}{13} (X_A + 12 X_B)X_A^4 \Lambda_A + (A \leftrightarrow B) \, .
\ee

The EOB electric tidal coefficients, instead, are usually denoted as $\kappa^{(\ell +)}$. They are related to the tidal Love number $k^A_{\ell}$ through
\begin{eqnarray}
\kappa_{A}^{(\ell +)} &= 2 k_{\ell} \frac{X_B}{X_A} \Bigl(\frac{R_A}{M}  \Bigr)^{2 \ell +1} = 2 k_{\ell} \frac{X_B}{X_A} \mathcal{C}_A^{2 \ell +1} \, .
%G \mu_\ell^{A} &= \kappa^{A}_{\ell, +} \frac{X_A}{X_B} \frac{G^{2 \ell +1}}{(2 \ell +1)!!} \Bigl( \frac{M_A}{c^2} \Bigr)^{2 \ell +1}
\end{eqnarray}
%
Similarly, for the magnetic-type coefficients we denote with $j^A_\ell$ the magnetic-tipe love number, and define
the $\ell$-th multipole coefficient $\sigma_\ell^{A} $, the dimensionless tidal parameter $\Sigma_{\ell}^A$ and the EOB tidal coefficient $\kappa_A^{(\ell -)}$ as:
\begin{eqnarray}
\sigma_\ell^{A}     =& \frac{ (\ell -1) j_{\ell}^A}{4(\ell +2)(2 \ell - 1)!!} R^{2 \ell + 1}_{A} \, ,\\
\Sigma_{\ell}^A     =& \frac{\sigma_\ell^{A}}{M_A^{2 \ell +1}} = \frac{ (\ell -1) j_{\ell}^A}{4(\ell +2)(2 \ell - 1)!!} \mathcal{C}_A^{-(2 \ell +1)} \, ,\\
\kappa^{(\ell -)}_A =& \frac{1}{2} \frac{X_B}{X_A} j^A_\ell \Bigl( \frac{R_A}{M} \Bigr)^{2 \ell +1} = \frac{1}{2} \frac{X_B}{X_A} j^A_\ell \mathcal{C}_A^{2 \ell +1} \, .
\end{eqnarray}
%

\section{EOB Tidal model}
\label{sec:eobmodel}

The \TEOB ~model is a state-of-the-art EOB waveform model for generic-spin coalescing compact binaries \cite{Damour:2014sva, Nagar:2015xqa, Nagar:2018zoe, Nagar:2019wds, Nagar:2020pcj, Riemenschneider:2021ppj, Akcay:2020qrj, Gamba:2021ydi}.
As all EOB models, it is made up of three main building blocks, in principle separate from one another: a Hamiltonian describing the conservative motion, a radiation reaction force which accounts for energy and momentum losses and a prescription for the waveform at infinity. 
Below, we discuss the model used for BNS systems, highlighting improvements
and differences with respect to the previous model of \cite{Akcay:2018yyh}, here dubbed \gsftides{3}, in each of the fundamental blocks mentioned.

\subsection{Hamiltonian}

The EOB Hamiltonian is given by~\cite{Buonanno:1998gg, Damour:2000we, Damour:2015isa}
\be
\label{eq:HEOB}
\hat{H}_{\rm EOB} = \frac{1}{\nu}\sqrt{1+2\nu(\hat{H}_{\rm eff}-1)} \, ,
\ee
with
\be
\label{eq:Heff_orb}
\hat{H}_{\rm eff} = \hat{H}_{\rm eff}^{\rm orb} = \sqrt{p_{r*}^2 + A(r)\Bigl(1+ \frac{p_{\varphi}^2}{r^2}\Bigr)} \,.
\ee
in the case of nonspinning systems.
Here, $p_\varphi$ is the orbital angular momentum, $p_{r*}$ is the tortoise coordinate associated to the radial momentum $p_r$ given by $p_{r*} = \sqrt{A(r)/B(r)}  p_r$, and $A(r)$, $B(r)$ are the EOB metric potentials.
%
Within \TEOB, this Hamiltonian includes point-mass post-Newtonian (PN) analytical information up to 4PN in $A(r)$, and up to 3PN in the auxiliary potential $D(r)$, defined such that $A(r) B(r) = D(r)$.
Most of the analytical information described above is resummed via Padé approximants, to achieve robustness in the strong field, and further augmented by one pseudo-5PN coefficient which enters $A(r)$, $a_6^c$, calibrated to NR simulations of coalescing BBHs~\cite{Nagar:2018zoe}. 

\subsubsection{Spin}
Spins are included in the model in the Damour-Jaranowsky-Schaefer (DJS) gauge \cite{Damour:2008qf, Nagar:2011fx} via a modification of the effective Hamiltonian. Spin-orbit terms are accounted by two gyro-gravitomagnetic coefficients, $\hat{G}_{\hat{S}}$ and $\hat{G}_{\hat{S}_*}$, which are inverse-resummed and added to Eq.~\eqref{eq:Heff_orb}:
\be 
\hat{H}_{\rm eff} = \hat{H}_{\rm eff}^{\rm orb} + p_\varphi(\hat{G}_{S}\hat{S} + \hat{G}_{\hat{S}_*}\hat{S}_*) \, .
\ee 
$\hat{G}_{\hat{S}}$ and $\hat{G}_{\hat{S}_*}$ are currently (partially) known up to $5.5$ PN, 
i.e. to next-to-next-to-next-to-next-to-leading order (N4LO) \cite{Khalil:2021fpm}.
Here, we include and inverse-resum them up to this PN oder, improving on the previous model which only accounted for PN terms up to next-to-next to leading order (N2LO). %%; see App.~\ref{app:GSS} for their explicit expressions. 
Differently from BBH models, we do not employ any NR-informed coefficient in the gyro-gravitomagnetic terms.
%
Spin-squared terms are instead implemented to next-to-next-to-leading order (NNLO) in the model by 
replacing the EOB radial separation $r$ with the centrifugal radius $r_c$~\cite{Damour:2014sva}, defined as
\be
r^2_c = r^2 + \tilde{a}_Q^2\Bigl(1 + \frac{2}{r}\Bigr) + \frac{\delta\hat{a}_{\rm NLO}^2}{r} + \frac{\delta\hat{a}_{\rm NNLO}^2}{r^2}  \, ,
\ee
where $\tilde{a}_Q$ is an effective spin parameter that accounts for EOS-dependent spin-monupole interactions \cite{Nagar:2018plt} and reduces to the effective Kerr spin $\tilde{a}_0$ for BBH systems,
and beyond leading order (LO) spin squared contribution are included in $\delta\hat{a}_{\rm NLO}^2$ and  $\delta\hat{a}_{\rm NNLO}^2$, which can be read e.g. in \cite{Nagar:2018zoe, Nagar:2018plt}.

\subsubsection{Tidal effects}
Following the notation of \cite{Akcay:2018yyh}, electric $(+)$ and magnetic $(-)$ tidal interactions are generally included within the $A(r)$ and $B(r)$ potential as 
\be
A = A_0 + A_T^+ + A_T^- \, ,
\ee
\be
B = B_0 + B_T^+ + B_T^- \, .
\ee
where $A_0, B_0$ are the point mass BBH baselines of the metric potentials.
%
The tidal part of $A(r)$ is then typically further factorized as 
\be
\label{eq:A_tidal}
A_T^\pm(u) = \sum_{\ell \geq 2} A^{(\ell\pm)\rm LO}_A(u)\hat{A}_A^{\ell\pm}(u) + (A\leftrightarrow B)
\ee
where $u = 1/r$ is the inverse of the EOB radial coordinate, and the LO coefficients are straightforwardly given by
\be
A_A^{(\ell +)\rm LO}(u) = -\kappa_A^{(\ell +)}u^{2\ell +2} \, ,
\ee
and
\be
A_A^{(\ell -)\rm LO}(u) = -\kappa_A^{(\ell -)}u^{2\ell +3} \, .
\ee 
Within \TEOB, we consider LO electric contributions up to $\ell=8$ \cite{Godzieba:2021vnz}, extending the previous model of \cite{Akcay:2018yyh} that only included them up to $\ell=4$, and magnetic contributions up to $\ell=3$.

The higher order corrections to Eq.~(\ref{eq:A_tidal}) are known up to 2PN (NNLO)\cite{Bini:2012gu}. %%, and can be read from App.~\ref{app:APN}.
Notably however, rather than relying on the simple PN-expanded expressions above for 
$\hat{A}_A^{(2 +)}$ and $\hat{A}_A^{(3 +)}$, the \TEOB ~model implements a GSF-informed resummation of the potential:
\be
\hat{A}_A^{(\ell \pm)}(u) = \hat{A}_A^{(\ell \pm)\rm 0GSF} + X_A\hat{A}_A^{(\ell \pm) \rm 1GSF} + X_A^2 \hat{A}_A^{(\ell \pm) \rm 2GSF} \, .
\ee
The expressions for the GSF coefficients are collected in App.~\ref{app:APN}. 
Notably, all terms explicitly depend on the light ring radius $\rLR$, which determines the position of the pole in $A(u)$, and a ``GSF exponent" $p$, which appears in the 2GSF terms. The former used to be set by computing the adiabatic light ring of the $A(u)$ potential with 2PN tidal terms; the latter was fixed to $p=4$ independently of the binary parameters. 
Here, we make full use of the flexibility provided by the GSF resummation, fix $p=9/2$ and determine 
$r_{\rm LR}$ via comparisons to NR data (see Sec.~\ref{sec:nrinfo} below).

The $B(u)$ potential includes tidal effects to leading order in the electric and magnetic tidal parameters \cite{Vines:2010ca,Akcay:2018yyh}:
\begin{align}
B^+_T(u) &=  3(3-5\nu)(\kappa_A^{(2+)} + \kappa_B^{(2+)}) u^6 \, , \\
B^-_T(u) &=  5(\kappa_A^{(2-)} + \kappa_B^{(2-)}) u^6 \, .
\end{align}
The magnetic term is -- to the best of our knowledge -- presented here for the first time. 
Its computation used the results of Ref.~\cite{Henry:2019xhg}.
Following standard techniques, we perform a Legendre transformation of the harmonic center of mass 1PN 
Lagrangian of Ref.~\cite{Henry:2019xhg} proceeding order-by-order and
recovering an ``ADM-like'' Hamiltonian, which coincides with the usual ADM Hamiltonian in the point mass sector.
Via a canonical transformation, closely following the procedure outlined in \cite{Damour:2000we}, we express the Hamiltonian $\hat{H}_{\rm ADM-like}$ and $\hat{H}_{\rm eff}$ in the same 
set of coordinates.
We parameterize the 1PN generating function $G_{\rm 1PN}(\mathbf{q}, \mathbf{p}')$, where
$(\mathbf{q}, \mathbf{p})$ are the original (ADM-like) coordinates and $(\mathbf{q}', \mathbf{p}')$
are the desired EOB coordinates.
We then compute:
\begin{eqnarray}
\mathbf{q}' = \mathbf{q} + \frac{\partial G}{\partial \mathbf{p}'} \, ,\\
\mathbf{p} = \mathbf{p}' + \frac{\partial G}{\partial \mathbf{q}} \, ,
\end{eqnarray}
and compare:
\begin{equation}
\hat{H}_{\rm eff}^2(q',p') = \Bigl[ 1 + \hat{H}_{\rm ADM-like}(q,p) + \alpha_1 \hat{H}_{\rm ADM-like}(q,p)^2 \Bigr]^2 \, .
\end{equation}
the $A(u)$ and $B(u)$ potentials within $\hat{H}_{\rm eff}$ are expanded to 1PN, and the LO magnetic and electric terms in $B(u)$ are parameterized via two unknown 
coefficients $\chi_2$ and $\beta_2$, respectively.
We then find
\begin{eqnarray}
\beta_2 =& 3(3 - 5 \nu) \, ,\\
\chi_2 =&5 \, .
\end{eqnarray}
thus confirming the result of Vines et. al.~\cite{Vines:2010ca} and of Bini et. al.~\cite{Bini:2012gu}, and computing $\chi_2$ for the first time.
%
Notably, $B(u)$ is typically computed as the ratio of the EOB $D(u)$ BBH potential
and the $A(u)$ potential 
discussed above. Since $A(u)$ already contains tidal corrections, in order to obtain the correct $B(u)$ PN LO one needs to correct the ratio $D(u)/A(u)$ with an additional term $B'_T(u)$, so that in practice
\begin{align*}
B_T(u) &= \frac{D(u)}{A(u)} + B'_T(u) \, \\
B'_T(u) &= (\kappa_A^{(2+)} + \kappa_B^{(2+)})(8 -15\nu)u^6 + B^{-}_T(u) \, .
\end{align*}

\subsection{Waveform and radiation reaction}
%

The BBH EOB waveform is given by
\begin{equation}
h_{\ell m}^0 = c_{\ell + \epsilon}(\nu) h' {}_{\ell m}^{(N, \epsilon)} S^{(\epsilon)}h_{\ell m}^{\rm tail} f_{\ell m} \, = h' {}_{\ell m}^{(N, \epsilon)}\hat{h}^0_{\ell m} .
\end{equation}
where $h' {}_{\ell m}^{(N, \epsilon)}$ is a Newtonian prefactor,  $S^{(\epsilon)}$ is the source term, $h_{\ell m}^{\rm tail}$ includes tail contribitions and $f_{\ell m}$ resums the residual terms \cite{Damour:2008gu, Pan:2010hz, Damour:2014sva,Nagar:2016ayt,Messina:2018ghh,Nagar:2019wrt}.
%
Tidal contributions are included in the EOB multipoles $h_{\ell m}$ via a simple additive correction to the BBH baseline, 
$h_{\ell m} = h_{\ell m}^0 + h_{\ell m}^{\rm tidal}$.
%
The tidal part of the waveform multipoles does not follow the standard EOB factorization, and is instead simply 
given by:
\begin{eqnarray}
h_{\ell m}^{\rm tidal} =& h' {}_{\ell m}^{(N, \epsilon)}\hat{h}_{\ell m}^{\rm tidal}\, , \\
\hat{h}_{\ell m}^{\rm tidal} =& c_{\ell m}^{\rm LO}\kappa_A (1 + \beta_{\ell m}^1 x + \beta_{\ell m}^2 x^2) + (A \leftrightarrow B) \, .
\end{eqnarray}
where $c_{\ell m}^{\rm LO}$ and $\beta_{\ell m}^{n}$ parameterize the leading order and $n$-th PN corrections to the waveform amplitude.
These terms were previously known up to NLO \cite{Damour:2009wj, Vines:2010ca, Damour:2012yf}; 
here we exploit the results of Ref.~\cite{Henry:2020ski} to compute and include also the leading order correction to $h_{44}$ and $h_{42}$ and the subleading terms $\beta_{22}^2$ and $\beta_{21}^1$.

To obtain the desired terms, we first evaluate the multipolar fluxes $F_{\ell m}$, parameterized by the unknown coefficients, via
\begin{equation}
F_{\ell m} = F_{\ell m}^{(N, \epsilon)} |\hat{h}_{\ell m}|^2 = F_{\ell m}^{(N, \epsilon)} |\hat{h}_{\ell m}^0 + \hat{h}_{\ell m}^{\rm tidal}|^2 \, .
\end{equation}
The Newtonian flux prefactors $F_{\ell m}^{(N, \epsilon)}$ can be found in e.g. App.~A of~\cite{Messina:2018ghh}. 
Note however that, with respect to the definition given in Ref.~\cite{Messina:2018ghh}, 
the term $c_{\ell+\epsilon}(\nu)$ is here factored out and included directly in the definition of $\hat{h}_{\ell m}^0$.
%
We then compare our results to the multipolar fluxes computed in Ref.~\cite{Henry:2020ski}, and thus extract the desired parameters.
We find:

\begin{widetext}
\begin{eqnarray}
\beta_{22}^2 =& \frac{111970 X_A^7-414911 X_A^6+805952 X_A^5-728217 X_A^4+363195
   X_A^3-347284 X_A^2+153638 X_A-16366}{10584 X_A (2 X_A-3)} \, ,\\
\beta_{21}^1 =& \frac{-220 X_A^3-130 X_A^2+203 X_A+15}{126-168 X_A} \, ,\\
c_{44}^{LO} =&  2 (5 - 9 X_A + 6 X_A^2) \, ,\\
c_{42}^{LO} =&  3584 (5 - 9 X_A + 6 X_A^2) \, .
\end{eqnarray}
\end{widetext}

When specifying $\beta_{22}^2$ to the equal mass case ($X_A = X_B = 1/2$) we find $\beta_{22}^2 = 167/256 \sim 0.653$. 
This value is extremely close, but not exactly equal, to the $\beta_{22}^2$ estimated in Ref.~\cite{Henry:2020ski} when comparing their results 
with Ref.~\cite{Damour:2012yf}. We attribute this discrepancy to possible computation errors in Ref.~\cite{Damour:2012yf}, 
which was also found to be incorrect in the $7.5$PN tail term.
Crucially, with respect to previous versions of the model, we do not propagate the tail contribution $h_{\ell m}^{\rm tail}$ to the tidal amplitudes, 
in order to correctly recover the LO 6.5PN tail terms.
Given the new tidal terms contributing to the waveform amplitudes $\hat{h}_{\ell m}$, the radiation reaction force $\hat{\mathcal{F}}_\varphi$ is extended straightforwardly.

\subsection{Effect of the analytical information}

% Figure environment removed

The left panel of Fig.~\ref{fig:analytical} shows the impact of the additional analytical information included in the Hamiltonian and the waveform
for $10^4$ systems with varying mass ratios, tidal parameters and spins. Such a difference is here quantified in terms of the phase difference at merger $\Delta \phi = \phi^{\rm old} - \phi^{\rm new}$, 
where ``new'' and ``old'' indicate the models with and without the extra terms discussed in the previous subsections. 
Note that, in both cases, $\ell > 2$ multipolar tidal parameters are estimated via the fits from Ref.~\cite{Godzieba:2020bbz, Godzieba:2021vnz}.
%
From this comparison it appears that: 
(i) the (absolute) values of the phase differences are smallest for binaries with small tidal parameters or larger mass ratios, 
and maximal for equal mass binaries with large tidal parameters; 
(ii) the phase difference is positive over a considerable portion of the parameter space, implying that the new tidal model 
prescribes weaker matter effects than the previous.
%
The facts observed above are, at first impact, rather puzzling: while it is expected for the models to differ the most for large values of 
$\tilde\Lambda$, it is not immediately clear why they should agree for high mass ratios, or -- conversely -- why they should differ 
the most for equal mass BNSs.

To further understand this picture, then, we consider two representative cases: an equal mass, non-spinning binary with $\tilde{\Lambda}=2500$, 
and a $q=2$ binary with the same effective tidal parameter and spins. 
For each, we also compute the waveform combining the ``new'' and ``old'' model with different fits for the multipolar tidal parameters 
$\Lambda_{\ell}$. The right panel of Fig.~\ref{fig:analytical} shows the $(2,2)$ amplitude evolution for the two systems considered.
Focusing on the equal mass case, the largest effect for this binary is given by choice of the fit for $\Lambda_{3,4}$. 
The inclusion of $\Lambda_{\ell}$ terms with $\ell \geq 5$, too, has a considerable effect on the waveform.
Contrasting models based on the same fits and with identical $\Lambda_{\ell}$ content, it appears that the tidal contributions to the $B(u)$ potential, 
the new tidal contributions to $h_{22}$ itself and the lack of propagation of $h_{\ell m}^{\rm tail}$ to the multipolar waveforms have an overall 
repulsive effect on the dynamics. When the mass ratio is increased the importance of higher modes also grows, 
and differences between models are attenuated. In this scenario, the ``new'' model with $\Lambda_{\ell \leq 8}$ is comparable to the ``old'' model
with  $\Lambda_{\ell \leq 4}$, which in the equal mass case was significantly more attractive. 

Qualitatively, moving from the model of Ref.~\cite{Akcay:2018yyh} (which used the fits of Ref.~\cite{Yagi:2013sva}), to a model which includes more analytical 
information and is based on the fits from Ref.~\cite{Godzieba:2020bbz, Godzieba:2021vnz} represents a step towards the right direction: 
previous EOB-NR comparisons highlighted how the models fail to capture the last few cycles before merger, with tidal effects not being attractive enough.
Quantitatively, however, the new contributions are relatively small: the dephasing at merger varies between $-0.05$ and $0.3$ rad.
The inclusion of the analytical information discussed in the
previous section to the conservative and radiative sectors of the model therefore
does not, unfortunately, provide corrections to the final waveform that are large enough to
fill the gap with NR simulations, as observed in e.g. \cite{Akcay:2018yyh,Gamba:2022mgx}.
%


\section{Numerically informing the model}
\label{sec:nrinfo}
% Figure environment removed

The need of employing some NR information to improve our model in the last few cycles before
merger appears inevitable. 
We proceed on two different fronts.
First, we include NQC corrections in our waveform, in order to ensure that
the expected NR-prescribed values of amplitude and frequency at merger are reached by our waveforms.
%
Then, 
we employ the flexibility of the tidal model described in Sec.~\ref{sec:eobmodel} provided by the light ring radius $r_{\rm LR}$ 
or the GSF-inspired exponent $p_{\rm GSF}$. These parameters effectively determine the strength of tidal interactions by shifting 
the position of the pole in the tidal potential ($r_{\rm LR}$) or the degree of the singularity ($p_{\rm GSF}$).

\subsection{NQC model for BNS systems}
NQCs, or ``next-to-quasicircular" corrections have been first introduced 
within the EOB formalism by Damour and Nagar in Ref.~\cite{Damour:2007xr}, 
and have proven to be fundamental in the creation of faithful inspiral-merger-ringdown BBH waveform models 
\cite{Nagar:2017jdw,Nagar:2018zoe,Riemenschneider:2021ppj,Albertini:2021tbt}. 
Here, we apply NQC corrections to BNS systems for the dominant $\ell=m=2$ mode, making use of the merger fits provided by
\cite{Breschi:2022xnc}.

We remind the reader that NQC corrections enter the factorized EOB waveform as a multiplicative term $\hat{h}_{\ell m}^{\rm NQC}$, which in the general case explicitly reads:
\be
\hat{h}_{\ell m}^{\rm NQC} = (1 + a_1 n_1 + a_2 n_2) e^{i(b_1 n_3 + b_2 n_4)} \, .
\ee
The numerical coefficients $a_1, a_2, b_1, b_2$ are to be determined from the values of NR amplitude and frequency at merger, while $n_1, n_2, n_3, n_4$ are the NQC basis functions. Consistently with the BBH case, we employ:
\begin{eqnarray}
n_1 &=  \frac{p_{r_*}^2}{r^2\Omega^2} ,\\
n_2 &=  \frac{\ddot{r}}{r \Omega^2} ,\\
n_3 &=  \frac{p_{r_*}}{r \Omega} ,\\
n_4 &=  n_3 r^2 \Omega^2 .
\end{eqnarray}
Following standard EOB techniques, the NQC coefficients $a_i, b_i$ are determined by imposing that the EOB amplitude $\hat{A}_{22}$, its time derivative $\dot{\hat{A}}_{22}$, the frequency $\hat{\omega}_{22}$ and the frequency time derivative $\dot{\hat{\omega}}_{22}$
extracted $3 M$ before the peak of the EOB orbital frequency are equal to the same NR quantities extracted at the NR NQC time $t_{\rm NQC}^{\rm NR}$, which is here identified with the merger, $t_{\rm NQC}^{\rm NR} = t_{\rm mrg}^{\rm NR}$.
\begin{eqnarray}
\hat{A}_{22}^{\rm EOB}(t_{\Omega^{\rm peak}}-3) &=  \hat{A}_{22}^{\rm NR}(t_{\rm mrg}) \, ,\\
\dot{\hat{A}}_{22}^{\rm EOB}(t_{\Omega^{\rm peak}}-3) &=  \dot{\hat{A}}_{22}^{\rm NR}(t_{\rm mrg}) = 0 \, ,\\
\hat{\omega}_{22}^{\rm EOB}(t_{\Omega^{\rm peak}}-3) &=  \hat{\omega}_{22}^{\rm NR}(t_{\rm mrg}) \, ,\\
\dot{\hat{\omega}}_{22}^{\rm EOB}(t_{\Omega^{\rm peak}}-3) &=  \dot{\hat{\omega}}_{22}^{\rm NR}(t_{\rm mrg}) \, ,\\
\end{eqnarray}
This choice of the extraction point is slightly different with respect to the BBH case, where $t_{\rm NQC}^{\rm NR} = t_{\rm mrg}^{\rm NR} + 2$, and is motivated by the desire of employing already existing fits to the desired NR quantities when they are specified at merger. In detail, we supplement the fits presented in Ref.~\cite{Breschi:2022xnc} for 
$\hat{A}_{22}^{\rm NR}(t_{\rm mrg})$ and $\hat{\omega}_{22}^{\rm NR}(t_{\rm mrg})$ with new ones for
$\dot{\hat{\omega}}_{22}^{\rm NR}(t_{\rm mrg})$ (see App.~\ref{app:domg_fit}).

\subsection{Light ring singularity}
The GSF-informed resummation introduced in \cite{Akcay:2018yyh} naturally introduced
a \textit{pole} in the denominator of the $\hat{A}^{\rm XGSF}$ terms, with $X=0,1,2$.
The precise location of this pole is however quite uncertain: 
the analytical GSF expressions placed the pole at $r=3$, where the BBH light ring is situated. However, clearly, given that we are not describing coalescing BHs, this value 
can be modified at will. One solution is to employ the light ring radius $r_{\rm LR}$ implied by the $A$ potential augmented by NNLO tidal effects. 
The value of $r_{\rm LR}$ so estimated is generally larger than the BBH light ring, effectively enhancing the tidal terms closer to the end of the evolution, in the last few orbital cycles.
Such an enhancement was shown to improve the agreement between NR and EOB, although
it was still not sufficient to reproduce NR within its error bars in most of the cases studied.

Here, we build on this approach by multiplying the light ring radius estimated from the NNLO $A$ potential by a parameter $\alpha$, 
to be deterimined case-by-case by comparing our model to high resolution NR simulations after alignment.
%
We start from the equal mass, non-spinning case, and consider the public $\tt CoRe$ simulations {\tt BAM:0037}, {\tt BAM:0064}, {\tt BAM:0095}, and {\tt BAM:0097}. 
All simulations show clear convergence properties, with the latter displaying evident $4^{\rm th}$ order convergence thanks to the Enthropy-Flux-Limited (EFL) 
method developed in \cite{GUERMOND2008801, Guercilena:2016fdl} and improved in \cite{Doulis:2022vkx}.
Given the multiple available resolutions and extraction radii associated to each simulation, 
we estimate the total error budget on the waveform phase as the sum in quadrature of the resolution and finite extraction error. 
The former is computed by evaluating the phase difference between the highest and second-highest resolutions; 
the latter by estimating the phase difference between the waveform extrapolated to infinity and extracted 
at the largest finite radius.

Comparisons between our $\alpha$-calibrated and uncalibrated EOB model with the NR
waveforms above are shown in Fig.~\ref{fig:bam_q1}. 
The calibrated model can reproduce all the simulations shown within their NR error, 
except for {\tt BAM:0097}. Notably, even when the phase error at merger is within the NR errorbars, 
the NQC-enhanced and NR-tuned EOB model still does not seem
to be able to fully capture the frequency evolution over the last two cycles. 
Nonetheless, the introduction of the $\alpha$ parameter noticeably improves the EOB/NR agreement 
for binaries with large effective tidal parameters, such as {\tt BAM:0037} and {\tt BAM:0064}. 
%
A trend in $\alpha$ is also easily identifiable: it is clear that to match the simulations with larger tidal parameters, 
larger values of $\alpha$ are necessary. 
Conversely, given that we set $p=9/2$, values of $\alpha < 1$ need to be chosen in order 
to obtain agreement with those simulations with $\kappa^T_2 \leq 100$.
Using the four simulations shown in Fig.~\ref{fig:bam_q1} we fit
\be
\label{eq:alpha_fit}
\alpha = \frac{a_1 \kappa_2^T}{1 + a_2 \kappa_2^T} \, ,
\ee
finding $a_1 = 0.06306$ and $a_2 = 0.05732$. 

Beyond our calibration set, we determine $\alpha$ for $\sim 10$ additional lower-resolution non-spinning 
simulations and compare the extracted values with the $\alpha$ computed via Eq.~\ref{eq:alpha_fit}. 
This investigation indicates that the trend we identified is maintained when considering other equal-mass 
configurations, with the largest $\Delta\alpha/\alpha$ amounting to about $13\%$.
%
Concerning unequal mass and spinning terms, instead, we do not attempt to fit these contributions to $\alpha$ 
due to the large uncertainties affecting the simulations. 
However, we mention that even in this scenario by modifying $\alpha$ it is usually possible to 
reproduce the NR waveforms within their estimated error bands.

%%  % Figure removed

\section{EOB/NR comparisons}
\label{sec:nrcomp}

\subsection{New simulations}

\begin{table}[t]
	% \resizebox{\textwidth}{!}{
	  \centering
	\begin{tabular}{ccccccccccc}
	\hline\hline
	ID & EOS & $M\omega_0$ & $M$ & $q$ & $\Lambda^A_2$ & $\Lambda^B_2$ & $\chi_A$ & $\chi_B$ & $\bar{\mathcal{F}}_{old}$ & $\bar{\mathcal{F}}_{new}$ \\
	\hline 
	ER01 & H4  & $0.0373$ & $2.77$ & $1.02$ & $888$ & $1007$ & $0.03$ & $0.07$ & $0.0119$ & $0.0039$\\
	ER02 & H4  & $0.0339$ & $2.6$ & $1.26$ & $719$ & $2789$ & $0.04$ & $0.13$ & $0.0225$ & $0.0045$\\
	ER03 & H4  & $0.0339$ & $2.6$ & $1.26$ & $718$ & $2794$ & $0.11$ & $0.05$ & $0.0219$ & $0.009$\\
	ER04 & H4  & $0.0373$ & $2.77$ & $1.02$ & $887$ & $1008$ & $0.06$ & $0.03$ & $0.0132$ & $0.0032$\\
	ER05 & H4  & $0.0382$ & $2.82$ & $1.06$ & $718$ & $1008$ & $0.11$ & $0.05$ & $0.008$ & $0.0024$\\
	ER06 & MS1b  & $0.0369$ & $2.77$ & $1.02$ & $1268$ & $1420$ & $0.1$ & $0.03$ & $0.0059$ & $0.0169$\\
	ER10 & MS1b  & $0.0347$ & $2.65$ & $1.21$ & $1048$ & $2854$ & $0.03$ & $0.14$ & $0.0339$ & $0.0072$\\
	ER11 & MS1b  & $0.0379$ & $2.82$ & $1.06$ & $1048$ & $1418$ & $0.03$ & $0.1$ & $0.0045$ & $0.0267$\\
	ER12 & MS1b  & $0.0379$ & $2.82$ & $1.06$ & $1048$ & $1417$ & $0.03$ & $0.13$ & $0.0203$ & $0.0014$\\
	ER13 & MS1b  & $0.0378$ & $2.82$ & $1.06$ & $1047$ & $1420$ & $0.09$ & $0.03$ & $0.018$ & $0.0044$\\
	ER14 & MS1b  & $0.0379$ & $2.82$ & $1.06$ & $1046$ & $1420$ & $0.12$ & $0.03$ & $0.0164$ & $0.0026$\\
	ER15 & H4  & $0.0382$ & $2.82$ & $1.05$ & $719$ & $1006$ & $0.04$ & $0.11$ & $0.0075$ & $0.0019$\\
	ER17 & H4  & $0.0373$ & $2.77$ & $1.02$ & $888$ & $1006$ & $0.05$ & $0.11$ & $0.0109$ & $0.0068$\\
	ER18 & H4  & $0.0373$ & $2.77$ & $1.02$ & $886$ & $1008$ & $0.11$ & $0.05$ & $0.0128$ & $0.0054$\\
	\hline
	\end{tabular}
	% }
	\caption{New spinning, unequal mass NR simulations used to test \TEOB{}. The new model has lower unfaithfulness that the previous one in all but two cases.}
	\label{tab:nr_sims}
\end{table}

We present 14 new simulations of unequal mass, spinning binaries. 
The initial data for these simulations has been computed using the SGRID library~\cite{Tichy:2011gw,Tichy:2012rp,Tichy:2019ouu}, and eccentricity reduction
has been performed in order to minimize residual spurious artifacts in the waveform~\cite{Moldenhauer:2014yaa,Dietrich:2015pxa}.
The constraint-satisfying data is then evolved with the {\tt BAM} code \cite{Bruegmann:1996kz,Brugmann:2008zz,Thierfelder:2011yi}.
All simulations are run at multiple resolutions, with 64, 96, 128 or 72, 108, 144 points per directions in the
finest (moving) mesh refinement level covering the individual NSs and using a high-order hydrodynamics scheme \cite{Bernuzzi:2016pie}. 
More details on the simulations are given in Tab.~\ref{tab:nr_sims} and in Appendix~\ref{app:NR}.

\subsection{Time-domain phasing}

% Figure environment removed

To validate our new model, hereafter dubbed \gsftides{3NR}, in a regime far from the one considered in our calibration set, 
we align in the time domain the \gsftides{3} and the \gsftides{3NR} models 
with the NR simulations described in the previous sub-section over a window spanning the frequency range $\omega_{22}\in [0.4, 0.5]$.
When using the \gsftides{3NR} model, we iterate on the NQCs 5 times, employing them also in the flux, before performing the alignment.
The results of our alignment procedure can be inspected from Fig.~\ref{fig:bam_H4q1} and Fig.~\ref{fig:bam_MS1bq1} for the H4 and MS1b simulations, respectively.
Notably, we do not present results for the ER06 and ER11 simulations, as we were not able to correctly align them to the EOB model. 
%
Overall, we observe that although no unequal mass or spin-dependent corrections have been introduced to the \gsftides{3NR} model, it improves over
the previous one in all the comparisons presented.
%
The EOB/NR phase difference at merger is smaller than the estimated NR error in eight out of twelve cases displayed, with an average $\Delta\phi^{\rm EOBNR} \sim 0.5$ rad 
at merger. 
Unsurprisingly, the two cases for which significant disagreement -- larger than the NR error -- is found are both high $q$, large spins ones with the H4 EOS.
Notably, as was also the case for the simulations of Fig.~\ref{fig:bam_q1}, we observe again a faster increase of the NR frequency over the last two cycles with respect to the 
one predicted by the EOB model. This is especially evident in the $\tt ER05$, $\tt ER13$, $\tt ER14$, $\tt ER15$ and $\tt ER18$ simulations, where 
$\Delta\phi^{\rm EOB/NR}$ is smaller than the NR error in the inspiral and at merger, but larger than this quantity approaching the end of the coalescence.
This indicates that our model is not fully capturing the physics of the system beyond the contact of the two bodies.

\subsection{Unfaithfulness}
%
To further quantify the agreement of our model with NR,
we compute the EOB/NR unfaithfulness (or mismatch) $\bar{\mathcal{F}}$, which describes the global agreement
between our model and the NR data.
The EOB/NR unfaithfulness is defined as
\begin{equation}
\label{eq:fbar}
\bar{\mathcal{F}} = 1 - \max_{\phi_0, t_0} \frac{(h_{\rm EOB}, h_{\rm NR})}{\sqrt{(h_{\rm NR}, h_{\rm NR}) (h_{\rm EOB}, h_{\rm EOB})}}  \, ,
\end{equation}
where $(\cdot, \cdot)$ is the inner product in waveform space and $\phi_0, t_0$ are a reference time and phase. 
The action of the inner product on two generic waveforms $h$, $k$ is given by
\begin{equation}
(h, k) = 4\Re \int_{f_{\rm min}}^{f_{\rm max}} \frac{\tilde{h}(f) \tilde{k}^*(f)}{S_n(f)} df \, ,
\end{equation}
where $S_n(f)$ is the noise curve of the detector. For our comparisons, we employ the Einstein Telescope (ET) noise curve \cite{Hild:2010id},
and choose $f_{\rm min}$, $f_{\rm max}$ respectively as the initial frequency of the NR simulation (typically $> 300$ Hz) and the merger frequency. 

Applying Eq.~\eqref{eq:fbar}, we find that our model is either comparable with or improves the previous one in all 
but two cases, with typical mismatches below $1\%$.
Note that this number provides a conservative limit on the NR-faithfulness of our model, as little to no early inspiral
is included in our simulations. If hybrid waveforms were to be considered, the mismatch of \TEOB{} would decrease
as well.
This result confirms and complements the time domain phasing analysis discussed in the previous paragraph (see Tab.~\ref{tab:nr_sims}).

% Figure environment removed


\section{Closed form representation of tidal sector}
\label{sec:tidal}

% Figure environment removed

\begin{table*}
	\caption{
	The coefficients of $\Psi_{\Lambda}$ and $\Psi_{MQ}$ as defined in equations \ref{eq:phase:phenomodel}.
	}
	\label{tab:fits}
	% \resizebox{\textwidth}{!}{
		\begin{tabular}{c|cccc||cccc}
		\hline\hline
		  & \multicolumn{4}{c||}{$\Psi^{\Lambda}$} & \multicolumn{4}{|c}{$\Psi^{\rm MQ}$} \\
		\hline 
		& $d_1$ & $d_2$ &$n_{5/2}$& $n_3$ & $d_1$ & $d_2$ &$d_3$& $n_4$ \\
		\hline 
		$\nu=1/4$ & $99.80$ & $1560.60$ & $-2191.56$ & $5307.07$ & $-9.61$ & $23.12$ & $3.90$ & $-27.50$ \\
		$\nu\neq 1/4$ & $4.08$&$39.67$&$-86.62$& $158.92$ & $-2.84$&$28.93$& $63.34$&$-58.83$ \\
		\hline \hline
		\end{tabular}
	% }
\end{table*}

Following the ideas proposed in Ref.~\cite{Dietrich:2017aum} (see also
\cite{Kawaguchi:2018gvj,Dietrich:2018uni, Dietrich:2019kaq}), we develop 
a FD {\it closed-form} representation of the tidal sector of \TEOB{}
that can be employed  to augment any point-particle model of choice to
include the effects of tides. This representation is faithful ($\bar{\mathcal{F}} \lesssim \mathcal{O}(10^{-3})$) for $\Lambda\in[10,1000]$, mass
ratios $q\in[1,2.5]$ and spins $|\chi_i|\lesssim 0.05$. When widening the parameter range considered to $\Lambda\in[10,1000]$ and $|\chi_i|\lesssim 0.35$,
we find more that 99.9\% of the mismatches lie below 1\%, and $\sim30\%$ below $1\text{\textperthousand}$.

Phenomenological representations of tidal approximants are usually built from hybrid PN-EOB-NR
waveforms by (i) subtracing the point-mass (binary black hole) and
(ii) fitting the differences in phase and amplitude. The use of
numerical relativity data at high-frequencies potentially improves the
accuracy with respect to the ``exact'' (unknown) waveform, but also implies a 
limitation in the parameter space coverage.
For example, neither the {\tt NRTidal} model nor its improved version {\tt NRTidalv2} \cite{Dietrich:2019kaq} 
incorporate mass-ratio induced corrections, and take into account spin-quadrupole effects
only in the phase difference through the PN expression of~\cite{Nagar:2018plt}.
Moreover, the use of different approximants in the hybrid construction and in the
subtraction step can result in inconsistencies and systematic effects. 
%
The phenomenological representation of \TEOB{} does not have these
drawbacks, although it retains the uncertainties of \TEOB{} in the
merger description when compared to NR waveforms~\cite{Bernuzzi:2012ci,Bernuzzi:2015rla,Akcay:2018yyh}.

The phase of the $(2,2)$ FD waveform is modeled as the
sum of the contributions due to pure orbital interactions (O), 
pure tidal effects  ($\Lambda$), spin orbit and spin-spin effects (S),
and self-spin couplings, also known as monopole-quadrupole 
(MQ) terms. It formally reads
\be
\label{eq:phasetot}
\Psi(f) = \Psi_{\rm O} + \Psi_{\Lambda} + \Psi_{S} + \Psi_{\rm MQ}\,.
\ee
Under the simplifying assumption that these contributions can, indeed,
be clearly separated, the contribution to the phasing due to tidal
effects can be expressed as: 
\be
\label{eq:deltaphase}
\Delta\Psi(f) = \Psi^{\rm BNS}(f) - \Psi^{\rm BBH}(f) \approx \Psi_{\Lambda} + \Psi_{\rm MQ}\,.
\ee
The PN expression valid in the low-frequency, weak-field  regime of
$\Psi^{\Lambda}$ at incomplete 7.5PN accuracy was originally obtained 
in~\cite{Damour:2012yf}, and is of the form
\be
\label{eq:PsiFD}
\Psi_{\Lambda} = c_{\rm LO}^\Lambda x^{5/2}(1 + c_1^\Lambda x + c_{3/2}^\Lambda x^{3/2} + c_2^\Lambda x^2 + c_{5/2}^\Lambda x^{5/2})\, ,
\ee
while the self-spin contributions $\Psi_{\rm MQ}$ at 3.5PN accuracy 
is given by~\cite{Bohe:2015ana, Mishra:2016whh, Nagar:2018plt}
\begin{align}
\label{eq:PsiFDss}
\Psi_{\rm MQ} = \frac{3}{128 \nu} c_{\rm LO}^{\rm MQ} x^{-1/2}(1 + c_{1}^{\rm MQ, NLO} x + c_{3/2}^{\rm MQ, tail} x^{3/2})\,,
\end{align}
with the coefficients listed in Appendix~\ref{app:PNcoefs}. 
The functional form of our representations is obtained by 
the Pad\'e resummation of the PN expression, 
\begin{subequations}\label{eq:phase:phenomodel}
\begin{align}
\Psi_{\Lambda} =&  c_{\rm LO}^\Lambda x^{5/2}\frac{ 1 + \sum_{i=2}^{6}{n_{i/2}^\Lambda x^{i/2}}}{1 + \sum_{i=2}^{4}{d_{i/2}^\Lambda x^{i/2}}} \, ,\\
\Psi_{\rm MQ} =&  \frac{3}{128 \nu} c_{\rm LO}^{\rm MQ} x^{-1/2} \frac{ 1 + \sum_{i=2}^{3}{n_{i/2}^{\rm MQ} x^{i/2}}}{1 + \sum_{i=2}^{4}{d_{i/2}^{\rm MQ} x^{i/2}}}\,.
\end{align} 
\end{subequations}
The PN limit requires the following constraints on the pure tidal and spin-quadrupole coefficients:
\begin{subequations}
\label{eq:PadPhase}
\begin{align}
n_{1}^{\Lambda} =& c_1^{\Lambda} + d_1^{\Lambda} \, , \\
n_{3/2}^{\Lambda} =& (c_1^{\Lambda} c_{3/2}^{\Lambda} -  c_{5/2}^{\Lambda} - c_{3/2}^{\Lambda} d_1^{\Lambda} + n_{5/2}^{\Lambda})/c_1^{\Lambda} \, ,\\
n_{2}^{\Lambda} =& c_2^{\Lambda} + c_1^{\Lambda} d_1^{\Lambda} + d_2^{\Lambda} \, ,\\
d_{3/2}^{\Lambda} =& -(c_{5/2}^{\Lambda} + c_{3/2}^{\Lambda} d_1^{\Lambda} - n_{5/2}^{\Lambda})/c_1^{\Lambda} \, ,\\ 
n_1^{\rm MQ} =& c_1^{\rm MQ} + d_1^{\rm MQ}  \, ,\\
n_{3/2}^{\rm MQ} =& c_{3/2}^{\rm MQ} + d_2^{\rm MQ} \,.
\end{align} 
\end{subequations}
The remaining coefficients are fitted to \TEOB{PA}.
To incorporate corrections due to unequal-mass effects we parameterize
the free coefficients as a $\nu=1/4$ contribution plus a factor proportional to $\sqrt{1-4\nu}$. 
In particular, denoting a generic coefficient $n_i^{\Lambda}, d_i^{\Lambda}$ 
or $n_i^{\rm MQ}, d_i^{\rm MQ}$ as $p_i^{\Lambda}$ 
and $p_i^{\rm MQ}$, we have:
\begin{subequations}
\begin{align}
p_i^{\Lambda} =& p_i^{(\nu=1/4)} + \frac{\kappa_1 - \kappa_2}{\kappa_1 + \kappa_2}p_i^{(\nu \neq 1/4)}\sqrt{1-4\nu} \, , \\
p_i^{\rm MQ}  =& p_i^{(\nu=1/4)} + \frac{(C_{\rm Q1} \tilde{a}_1^2 - C_{\rm Q2}\tilde{a}_2^2)}{(C_{\rm Q1} \tilde{a}_1^2 + C_{\rm Q2}\tilde{a}_2^2)}p_i^{(\nu \neq 1/4)}\sqrt{1-4\nu} \, .
\end{align} 
\end{subequations}
These functional forms are inspired by the Taylor expansions of the
coefficients known from PN theory about $\nu=1/4$. 

We then proceed as follows: (i) we compute the phase difference between a
set of $\nu=1/4$, nonspinning waveforms, and obtain the values of $(d_1^{\Lambda}, d_2^{\Lambda}, n_{5/2}^{\Lambda}, n_3^{\Lambda})^{(\nu=1/4)}$; (ii)
we consider a set of unequal mass, nonspinning waveforms and fit $(d_1^{\Lambda}, d_2^{\Lambda}, n_{5/2}^{\Lambda}, n_3^{\Lambda})^{(\nu \neq 1/4)}$,
setting the $\nu=1/4$ coefficients to the values found in the previous point; (iii) we fit
equal mass, spinning waveforms, and find the values of $(d_1^{\rm MQ}, d_2^{\rm MQ}, n_4^{\rm MQ}, d_3^{\rm MQ})^{(\nu=1/4)}$ using
again the equal-mass coeffcients of (i); (iv) we use all information
found up to now, and fit unequal mass, spinning waveforms to find
$(d_1^{\rm MQ}, d_2^{\rm MQ}, n_4^{\rm MQ}, d_3^{\rm MQ})^{(\nu \neq 1/4)}$.  
The values of all fitted coefficients, obtained from a dataset of $\sim 1000$ waveforms, are summarized in
Table~\ref{tab:fits}. 

We proceed analogously for the amplitude, whose
tidal and self-spin terms are modeled as
\begin{align}
\label{eq:AmpTidal}
\tilde{A}^\Lambda =& \sqrt{\frac{2 \nu}{3}} \pi x^{13/4} c_{5}^\Lambda \frac{1 + \frac{c_{6}^\Lambda}{c_{5}^\Lambda} x + \frac{ 22672}{9} x^{2.89}}{1 + d_4^\Lambda x^4} \, ,\\
\label{eq:AmpSpin}
\tilde{A}_{\rm MQ} =&-\sqrt{\frac{3 \nu}{2}} \pi x^{1/4}\frac{(C_{Q1} \tilde{a}_1^2 + C_{Q2} \tilde{a}_2^2)}{1 + e_4 x^4} \,.
\end{align}
In Appendix~\ref{app:PNcoefs} we compute the monupole-quadrupole interactions to the
FD amplitude, that are used to constrain some of the fitting coefficients.
In Eq.~\eqref{eq:AmpSpin} we incorporate only LO PN information 
for the spin sector, as we find that, while the addition of the NLO term slightly improves
the low frequency behavior of our fits, it also negatively impacts
the overall agreement in the high-frequency regime.
The unknown coefficients are fit by following the same procedure of the previous paragraph. We find:
\begin{subequations}
\begin{align}
d_4^{(\nu=1/4)} =& 5009.8736694 \, ,\\
d_4^{(\nu\neq 1/4)}=& -4017.88863642 \, ,\\
e_4^{(\nu=1/4)} =&  5.98351934 \, ,\\
e_4^{(\nu \neq 1/4)} =& 20.04283392 \,.
\end{align}
\end{subequations}

We compute the unfaithfulness between BNS {\tt TEOBResumSPA} waveform and the BBH
{\tt TEOBResumSPA} model 
augmented by the phenomenological description of tides.
We consider $10^4$ systems with masses uniformly distributed between $[1, 2.5]$ $M_{\odot}$, dimensionless spins uniformly distributed in $[-0.35, 0.35]$ and tidal 
parameters in $[10, 3000]$.
%
When restricting the calculation to non-spinning binaries with $\tilde{\Lambda} < 1000$ and $m_{1,2} \in [1, 2.5] \Msun$ 
we find a maximum unfaithfulness of $2 \times 10^{-3}$. Widening the range of tidal parameters, we find that the 
faithfulness degrades for tidal deformabilities larger than roughly $\sim 2000$.  
The worst matches ($\bar{\mathcal{F}} \sim 2\%$)
are obtained, as expected, for unequal mass configurations with large $\tilde\Lambda$.
When also considering spins, we find that the largest differences are obtained with configurations having at least one rapidly spinning NS.
In this scenario, unfaithfulness values can increase above the nominal $1\%$ threshold.
When the spins of the NS are moderate ($|\chi_{A,B}|< 0.05$), instead, we find mismatch values around $\mathcal{O}(10^{-3})$ or lower.


\section{Conclusions}
\label{sec:conc}


In this paper we presented a new, improved gravitational wave model for coalescing binary neutron stars.
Building on a previous version of \TEOB{}, we included: (i) leading order magnetic tidal corrections to the 
$B$ EOB potential; (ii) 2PN electric tidal terms in the $(\ell,m) = \{(2,2), (2,1), (4,2), (4,4) \}$ waveform multipoles;
(iii) 5.5PN spin-orbit terms in the gyro-gravitomagnetic coefficients entering the EOB Hamiltonian; (iv) NQC corrections to the quadrupolar $(2,2)$ mode; (v) a new NR-informed parameter, $\alpha$, which enters the GSF-resummed $A$ tidal potential and is fit against four high resolution NR simulations from the CoRe database.
%
The model was then benchmarked against 14 new eccentricity reduced BNS simulations for spinning binaries, here presented for the first time. 
The new tidal \TEOB{} improves over the previous model in terms of both mismatches and time-domain phasing, 
representing a robust and efficient alternative to current state of the art tidal models.
%
Finally, we presented a frequency domain closed form representation of the matter contributions to the phase and amplitude of the quadrupolar (2,2) 
mode implied by our full EOB model. This phenomenological representation faithfully approximates \TEOB{} ($\bar{\mathcal{F}} \lesssim 0.001$) 
over a considerable portion of the parameter space, with the worst mismatches obtained for strongly asymmetric systems with large spins.

In spite of the model's performance being overall satisfactory for many of the systems we inspected, 
our comparisons highlighted how improvements are necessary in the last few cycles before merger. 
Comparisons between our model and NR waveforms highlight
that the frequency evolution of the NR waveforms close to merger is steeper with respect to the one provided by 
the models. This indicates the need of further improving the treatment of matter effects beyond contact, which will
require large amounts of sufficiently accurate NR data, spanning large portions of the BNS parameter space. 
This ambitious goal will enable precise inference of the equation of state of cold, dense matter with XG
detectors such as ET \cite{Maggiore:2019uih} and Cosmic Explorer (CE) \cite{Reitze:2019iox}.



\begin{acknowledgments}
  We thank Thibault Damour for fruitful discussions during the preparation of this manuscript
  and for	partly supporting this work via the ``2021 Balzan Prize for Gravitation: 
  Physical and Astrophysical Aspects''.
  %
  RG is supported by the Deutsche Forschungsgemeinschaft (DFG) under Grant No.
  406116891 within the Research Training Group RTG 2522/1. 
  SB acknowledges support by the EU Horizon under ERC Consolidator Grant, no. InspiReM-101043372.
  NO acknowledges support by the UNAM-PAPIIT Grant No. IA101123.
  WT acknowledges support by the National Science Foundation under grant PHY-2136036.
  %
  The authors acknowledge the Gauss Centre for Supercomputing e.V. for
  funding this project by providing computing time on the
  GCS Supercomputer SuperMUC-NG at LRZ (allocations
  $\tt pn36ge$ and $\tt pn36jo$).
  %
  \TEOB{} is publicly available at
 
  {\footnotesize \url{https://bitbucket.org/eob_ihes/teobresums/src/master/}}

\end{acknowledgments}

\documentclass[twocolumn,hyperpdf,amsmath,amssymb,aps,prd,10pt,superscriptaddress,nofootinbib,noeprint,preprintnumbers,floatfix]{revtex4-2}

%% \usepackage[utf8]{inputenc} % allow utf-8 input
%\usepackage[T1]{fontenc}    % use 8-bit T1 fonts
%\usepackage{hyperref}       % hyperlinks
\usepackage{url}            % simple URL typesetting
\usepackage{booktabs}       % professional-quality tables
\usepackage{multirow}    
\usepackage{amsfonts}       % blackboard math symbols
\usepackage{nicefrac}       % compact symbols for 1/2, etc.
\usepackage{microtype}      % microtypogrhy
% \usepackage{natbib}
\usepackage{enumerate}
%\usepackage{enumitem}
\usepackage{hhline}
\usepackage{makecell}
\usepackage{pifont}

% use Times
%\usepackage{times}
% For figures
\usepackage{graphicx} % more modern
%\usepackage{epsfig} % less modern
%\usepackage{subfigure}
\usepackage{caption}
\usepackage{subcaption}
% For citations
\usepackage{amsmath}
\usepackage{amsthm}
\usepackage{amssymb}
\usepackage{tikz}
\usepackage{xcolor}
\usetikzlibrary{arrows}

\allowdisplaybreaks

%for fonts
\usepackage{mathrsfs}

% For algorithms
\usepackage{algorithm}
\usepackage{algorithmic}
% \usepackage{algpseudocode}
% \usepackage[noend]{algpseudocode}
\usepackage{hyperref}
\usepackage{bm}
%\usepackage{todonotes}

%For theorems
\allowdisplaybreaks

%for convinience
\newcommand{\RR}{\mathbb{R}}
\newcommand{\vct}{\boldsymbol }
%\newcommand{\mat}{\mathbf}
\newcommand{\rnd}{\mathsf}
\newcommand{\ud}{\mathrm d}
\newcommand{\nml}{\mathcal{N}}
\newcommand{\loss}{\mathcal{L}}
\newcommand{\hinge}{\mathcal{R}}
\newcommand{\kl}{\mathrm{KL}}
\newcommand{\cov}{\mathrm{cov}}
\newcommand{\dir}{\mathrm{Dir}}
\newcommand{\mult}{\mathrm{Mult}}
\newcommand{\err}{\mathrm{err}}
\newcommand{\sgn}{\mathrm{sgn}}
%\renewcommand{\span}{\mathrm{span}}
% \newcommand{\argmin}{\mathrm{argmin}}
% \newcommand{\argmax}{\mathrm{argmax}}
\newcommand{\poly}{\mathrm{poly}}
% \newcommand{\rank}{\mathrm{rank}}
% \newcommand{\conv}{\mathrm{conv}}
%\newcommand{\E}{\mathbb{E}}
% \newcommand{\diag}{\mat{diag}}
\newcommand{\acc}{\mathrm{acc}}

\newcommand{\labs}{\left\vert}
\newcommand{\rabs}{\right\vert}
\newcommand{\lnorm}{\left\Vert}
\newcommand{\rnorm}{\right\Vert}

\newcommand{\aff}{\mathrm{aff}}
% \newcommand{\range}{\mathrm{Range}}
\newcommand{\Sgn}{\mathrm{sign}}

\newcommand{\hit}{\mathrm{hit}}
\newcommand{\cross}{\mathrm{cross}}
\newcommand{\Left}{\mathrm{left}}
\newcommand{\Right}{\mathrm{right}}
\newcommand{\Mid}{\mathrm{mid}}
\newcommand{\bern}{\mathrm{Bernoulli}}
\newcommand{\ols}{\mathrm{ols}}
\newcommand{\tr}{\operatorname{tr}}
\newcommand{\opt}{\mathrm{opt}}
%\newcommand{\ridge}{\mathrm{ridge}}
\newcommand{\unif}{\mathrm{Unif}}
\newcommand{\Image}{\mathrm{im}}
\newcommand{\Kernel}{\mathrm{ker}}
\newcommand{\supp}{\mathrm{supp}}
\newcommand{\pred}{\mathrm{pred}}
\newcommand{\distequal}{\stackrel{\mathbf{P}}{=}}
%\newcommand{\gege}{\textcircled{1}}
\newcommand{\gege}{{A(\vect{w},\vect{w}_*)}}
\newcommand{\gele}{{A(\vect{w},-\vect{w}_*)}}
\newcommand{\lele}{{A(-\vect{w},-\vect{w}_*)}}
\newcommand{\lege}{{A(-\vect{w},\vect{w}_*)}}
\newcommand{\firstlayer}{\mathbf{W}}
\newcommand{\firstlayerWN}{v}
\newcommand{\secondlayer}{a}
\newcommand{\inputvar}{\vect{x}}
\newcommand{\anglemat}{\mathbf{\Phi}}
\newcommand{\holder}{H\"{o}lder }
\newcommand{\real}{\mathbb{R}}
\newcommand{\approxerr}{\delta}

\def\R{\mathbb{R}}
\def\Z{\mathbb{Z}}
\def\cA{\mathcal{A}}
\def\cB{\mathcal{B}}
\def\cD{\mathcal{D}}
\def\cE{\mathcal{E}}
\def\cF{\mathcal{F}}
\def\cG{\mathcal{G}}
\def\cH{\mathcal{H}}
\def\cS{\mathcal{S}}
\def\cI{\mathcal{I}}
\def\cL{\mathcal{L}}
\def\cM{\mathcal{M}}
\def\cN{\mathcal{N}}
\def\cP{\mathcal{P}}
\def\cS{\mathcal{S}}
\def\cT{\mathcal{T}}
\def\cV{\mathcal{V}}
\def\cW{\mathcal{W}}
\def\cZ{\mathcal{Z}}
\def\SS{\mathbb{S}}
\def\NN{\mathbb{N}}
\def\bP{\mathbf{P}}
\def\TV{\mathrm{TV}}
\def\MSE{\mathrm{MSE}}

\def\vw{\mathbf{w}}
\def\va{\mathbf{a}}
\def\vZ{\mathbf{Z}}

\newcommand{\mat}[1]{#1}
\newcommand{\vect}[1]{#1}
\newcommand{\norm}[1]{\left\|#1\right\|}
\newcommand{\normop}[1]{\left\|#1\right\|_{\mathrm{op}}}
\newcommand{\simplex}{\triangle}
\newcommand{\abs}[1]{\left|#1\right|}
\newcommand{\expect}{\mathbb{E}}
\newcommand{\prob}{\mathbb{P}}
\newcommand{\proj}{\gP}
% \newcommand{\prox}[2]{\textbf{Prox}_{#1}\left\{#2\right\}}
\newcommand{\event}[1]{\mathscr{#1}}
\newcommand{\set}[1]{#1}
\newcommand{\diff}{\text{d}}
\newcommand{\difference}{\triangle}
\newcommand{\inputdist}{\mathcal{Z}}
\newcommand{\indict}{\mathbb{I}}
\newcommand{\rotmat}{\mathbf{R}}
\newcommand{\normalize}[1]{\overline{#1}}
\newcommand{\vectorize}[1]{\text{vec}\left(#1\right)}
\newcommand{\vclass}{\mathcal{G}}
\newcommand{\pclass}{\Pi}
\newcommand{\qclass}{\mathcal{Q}}
\newcommand{\rclass}{\mathcal{R}}
\newcommand{\classComplexity}[2]{N_{class}(#1,#2)}
\newcommand{\cclass}{\mathcal{F}}
\newcommand{\gclass}{\mathcal{G}}
\newcommand{\pthres}{p_{thres}}
\newcommand{\ethres}{\epsilon_{thres}}
\newcommand{\eclass}{\epsilon_{class}}
\newcommand{\states}{\mathcal{S}}
\newcommand{\trans}{P}
\newcommand{\lowprobstate}{\psi}
\newcommand{\actions}{\mathcal{A}}
\newcommand{\contexts}{\mathcal{X}}
\newcommand{\edges}{\mathcal{E}}
\newcommand{\variance}{\text{Var}}
\newcommand{\params}{\vect{w}}

\newcommand{\relu}[1]{\sigma\left(#1\right)}
\newcommand{\reluder}[1]{\sigma'\left(#1\right)}
\newcommand{\act}[1]{\sigma\left(#1\right)}

\newtheorem{thm}{Theorem}
% \newtheorem{thm}{Theorem}
\newtheorem{lem}{Lemma}
% Thm -> corollary 
\newtheorem{cor}{Corollary}
\newtheorem{prop}{Proposition}
\newtheorem{asmp}{Assumption}
\newtheorem{defn}{Definition}
\newtheorem{oracle}{Oracle}
\newtheorem{fact}{Fact}
\newtheorem{conj}{Conjecture}
\newtheorem{rem}{Remark}
\newtheorem{example}{Example}
\newtheorem{condition}{Condition}
\newtheorem{exercise}{Exercise}
\newtheorem{mess}{Message}
\newtheorem{claim}{Claim}
\newtheorem{ec}{Empirical Conclusion}






\usepackage[capitalize,noabbrev]{cleveref}
% \usepackage{cleveref}
\crefname{thm}{Theorem}{Theorems}
\crefname{lem}{Lemma}{Lemmas}
\crefname{cor}{Corollary}{Corollaries}
\crefname{prop}{Proposition}{Propositions}
\crefname{asmp}{Assumption}{Assumptions}
\crefname{defn}{Definition}{Definitions}
\crefname{oracle}{Oracle}{Oracles}
\crefname{fact}{Fact}{Facts}
\crefname{conj}{Conjecture}{Conjectures}
\crefname{rem}{Remark}{Remarks}
\crefname{claim}{Claim}{Claims}
\crefname{ec}{Empirical Observation}{Empirical Observations}


\renewcommand{\algorithmicrequire}{\textbf{Input:}}
\renewcommand{\algorithmicensure}{\textbf{Output:}}


\definecolor{red}{rgb}{1, 0, 0}
\newcommand{\RED}[1]{{\color{red} #1}}

\definecolor{green}{rgb}{0, 1, 0}
\definecolor{darkgreen}{rgb}{0.0, 0.2, 0.13}
\definecolor{darkseagreen}{rgb}{0.56, 0.74, 0.56}
\definecolor{officegreen}{rgb}{0.0, 0.5, 0.0}


\newcommand{\GREEN}[1]{{\color{green} #1}}

\definecolor{blue}{rgb}{0, 0, 1}
\newcommand{\BLUE}[1]{{\color{blue} #1}}

\definecolor{orange}{rgb}{1, 0.4, 0.0}
\newcommand{\ORANGE}[1]{{\color{orange} #1}}


\usepackage{graphicx, color}
\usepackage[dvipsnames]{xcolor}

%% text %%
\usepackage[letterspace=-10]{microtype} 

%% math, tables %%
\usepackage{bm, amsmath, amsfonts, amssymb,xfrac}
\usepackage{multirow, tabularx, dcolumn}
\usepackage{mathtools, leftidx, braket, slashed, cancel, bigdelim}
\usepackage{blkarray}
\usepackage[figures]{rotating}

\usepackage{tikz}
\usetikzlibrary[plotmarks]
\usepackage{anyfontsize}

%% referencing %%
\usepackage[utf8]{inputenc} 
\usepackage{hyperref}
\pdfstringdefDisableCommands{ \renewcommand{\bm}[1]{#1} }

%% colors %%
\definecolor{jlab_red}{RGB}{192,39,45}
\definecolor{jlab_orange}{RGB}{249,102,0}
\definecolor{jlab_blue}{RGB}{47,122,121}
\definecolor{jlab_green}{RGB}{65,125,10}
\definecolor{jlab_gray}{gray}{0.6}
\definecolor{magenta}{rgb}{0.5, 0, 0.5}

\newcommand\bef{% Figure environment removed}
\newcommand\beq{\begin{equation}}
\newcommand\eeq[1]{\label{#1}\end{equation}}
\newcommand\beqa{\begin{eqnarray}}
\newcommand\eeqa[1]{\label{#1}\end{eqnarray}}
\newcommand\bet{\begin{table}}
\newcommand\eet[1]{\label{tb:#1}\end{table}}
\newcommand\fgn[1]{Figure \ref{fg:#1}} 
\newcommand\eqn[1]{Eq.\ (\ref{#1})}
\newcommand\tbn[1]{Table \ref{tb:#1}} 
\newcommand\pmn[1]{\textcolor{red}{#1}} 
\newcommand\pmnc[1]{\textcolor{red}{\it Comment: #1}}
\newcommand\nma[1]{\textcolor{blue}{#1}} 
\newcommand\nmac[1]{\textcolor{blue}{\it NM Comment: #1}}
\newcommand\arad[1]{\textcolor{green}{#1}} 
\newcommand\aradc[1]{\textcolor{green}{\it AR Comment: #1}}

%% editing macros %%
\newcommand{\cm}{\ensuremath{\mathsf{cm}}}


%% pdf hypertext links
\hypersetup{%
pdftitle = {title},
pdfsubject = {},
pdfkeywords = {},
%pdfauthor = {Hadron Spectrum Collaboration},
colorlinks = {true},
filecolor = {black},
linkcolor = {jlab_blue},
menucolor = {black},
citecolor = {jlab_green},
urlcolor = {jlab_green},
}{}

\begin{document}

%%%%%%%%%%%%%%%%%%%%%%%%%%%%%%%%%%%%%%%%%%%%%%%%%%%%%%%%%%%%%%%%%%%%%%
%\preprint{JLAB-THY-22-xxxx}
%
\title{Bound isoscalar axial-vector $bc\bar u\bar d$ tetraquark $T_{bc}$ in QCD}
%
\author{M. Padmanath}
\email{padmanath@imsc.res.in}
\affiliation{The Institute of Mathematical Sciences, a CI of Homi Bhabha National Institute, Chennai, 600113, India}
%
\author{Archana Radhakrishnan}
\email{archana.radhakrishnan@tifr.res.in}
\affiliation{Department of Theoretical Physics, Tata Institute of Fundamental Research, \\ Homi Bhabha Road, Mumbai 400005, India }

%
\author{Nilmani Mathur}
\email{nilmani@theory.tifr.res.in}
\affiliation{Department of Theoretical Physics, Tata Institute of Fundamental Research, \\ Homi Bhabha Road, Mumbai 400005, India }
%
% \collaboration{for the Hadron Spectrum Collaboration}
%

\preprint{IMSc/23/05, TIFR/TH/23-14}

\date{\today}
\begin{abstract}

The Fast Reciprocal Square Root Algorithm is a well-established approximation technique consisting of two stages: first, a coarse approximation is obtained by manipulating the bit pattern of the floating point argument using integer instructions, and second, the coarse result is refined through one or more steps, traditionally using Newtonian iteration but alternatively using improved expressions with carefully chosen numerical constants found by other authors. The algorithm was widely used before microprocessors carried built-in hardware support for computing reciprocal square roots. At the time of writing, however, there is in general no hardware acceleration for computing other fixed fractional powers. This paper generalises the algorithm to cater to all rational powers, and to support any polynomial degree(s) in the refinement step(s), and under the assumption of unlimited floating point precision provides a procedure which automatically constructs provably optimal constants in all of these cases. It is also shown that, under certain assumptions, the use of monic refinement polynomials yields results which are much better placed with respect to the cost/accuracy tradeoff than those obtained using general polynomials. Further extensions are also analysed, and several new best approximations are given.

\end{abstract}

\maketitle
%%%%%%%%%%%%%%%%%%%%%%%%%%%%%%%%%%%%%%%%%%%%%%%%%%%%%%%%%%%%%%%%%%%%%%%%%%%%%%%%%

% Figure environment removed

\section{Introduction}
Automatic 3D reconstruction of clothed humans using image inputs has gained increasing significance due to its potential applications in a wide array of AR/VR scenarios. High-fidelity reconstructions typically depend on sophisticated capture systems, which are developed with dense camera arrays~\cite{collet2015high,joo2015panoptic,joo2018total}, programmable light-stages~\cite{Vlasic2009, guo2019relightables}, and depth sensors~\cite{newcombe2011kinectfusion,DoubleFusion,BodyFusion,dou2016fusion4d,newcombe2015dynamicfusion}. However, stringent capture environments equipped with complex hardware pose significant challenges for consumer-level applications.


In this context, considerable research effort has been dedicated to developing methods that allow for more flexible capture configurations, such as utilizing a few RGB inputs. Among these works, learning implicit functions \cite{iccv2020PIFu, saito2020pifuhd, hong2021stereopifu} has proven effective in achieving highly detailed reconstructions by integrating the advancements of deep neural networks. These methods employ large multi-layer perceptrons (MLPs) to predict the occupancy probability or truncated signed distance function (TSDF) value of every queried 3D point based on its associated local feature, which is extracted from images. They can recover a continuous surface at arbitrary resolutions without topology restrictions.


However, in typical MLP-based implicit networks, the occupancy or TSDF value at each location is solved independently with planar image features, rendering them less capable of addressing challenging cases such as occlusions. Consequently, these methods suffer from generalization and robustness issues, particularly when tackling strong occlusions caused by large motion or multiple interacting humans. 
Some follow-up studies  \cite{zheng2021deepmulticap,zheng2021pamir,huang2020arch} utilize an extra geometric model, SMPL~\cite{Loper2015}, to improve robustness by introducing strong shape priors. 
Their success typically relies on the assumption of geometrical similarity \cite{huang2020arch} between the shape prior and target reconstruction, making them intractable for handling complex cases with loose clothes and sensitive to errors in SMPL model fitting.



%\ping{this paragraph sounds like `TSDF is better than MLP/SMPL, and we use TSDF to solve the problem'. But in Sec 3, we are telling a different story, saying `MLP needs a 3D convolutional encoder'. We need to make these two sections consistent.}\sicong{I think in this paragraph we claim that the TSDF}


%We opt for Trucated Signed Distance Funtion (TSDF) volumetric representations as they are naturally suitable for convolution operations, which have shown remarkable performance for learning hierarchical features on 2D visual perception tasks \cite{SunXLW19}. 
%Meanwhile, TSDF also describes the gradual geometry change around shape surface, which is not reflected by occupancy volume. 

We instead revisit the 3D volumetric representation and resort to 3D convolutional neural networks (CNNs) for feature learning, due to their impressive performance in feature learning and the ability to incorporate spatial context. However, volumetric methods and 3D convolution involve discretization, which might raise concerns regarding whether a discretized volume can preserve subtle geometric details as continuous representations learned in implicit functions. We investigate the relationship between volume resolution and quantization error on synthetic data by converting target mesh objects to TSDF volumes, as shown in Figure~\ref{fig:quantization_error}. We observe that the quantization errors are significantly reduced by increasing volume resolution and become nearly negligible when reaching a relatively high resolution (e.g., 512 or higher). In other words, achieving fine-detailed reconstruction is not supposed to be restricted by the use of volume representations as long as a proper volume resolution is utilized. Therefore, we present a method with high-resolution feature volumes, e.g., 256 and 512, while traditional volumetric methods \cite{varol18_bodynet,gilbert2018volumetric} are often limited to much lower resolutions, such as 32 or 128.



On the other hand, an increase in volume resolution may lead to a cubic growth of memory overhead \cite{8100085}. Reducing memory costs while guaranteeing the granularity of volumetric representations is necessary for pursuing high-quality reconstruction. Thus, we adopt a coarse-to-fine approach and cull away irrelevant voxels to build a sparse high-resolution feature volume. At the coarse level, the network computes an initial TSDF by applying a U-Net with sparse 3D CNN \cite{3DSemanticSegmentationWithSubmanifoldSparseConvNet} on the sparse feature volume, which is carved by a visual hull. Through our experiments, it turns out that more than 95\% of the volume grids are discarded by the visual hull culling, making the sparse 3D CNN efficient. At the fine level, the network focuses on a narrow band near the zero-level set of the initial TSDF and discretizes the narrow band with smaller voxels. By employing this narrow-band culling, we further shrink the sampling space, resulting in a relatively small range of grid numbers (usually 300K--500K in our experiments) even with a high volume resolution of 512. The remaining voxels in the narrow band are associated with features that fuse high-frequency information from the computed normal maps upon the low-frequency shape from the coarse level to compute the TSDF at high resolution. The final mesh is then extracted from the TSDF using the Marching-Cube algorithm ~\cite{Lorensen87marchingcubes}.
% Different from the u-net sturcture to preserve global topology context, we then apply a shallow 3dcnn to compute the final TSDF $D_{final}$ which contain more local geometry detail.




% \ping{this paragraph can be expanded. It is an important contribution and often ignored by other works. stress on the novel idea of regressing blending weights instead of colors}

In addition to geometry, high-quality mesh texture is also a crucial factor contributing to visual appearance. Directly computing a color field in 3D space, as in \cite{iccv2020PIFu}, struggles to capture high-frequency texture details, while the neural radiance field (NeRF) \cite{yu2020pixelnerf} or the DoubleField~\cite{shao2022doublefield} require expensive per-instance optimization and are often unstable for sparse input images. In contrast, we adopt an image-based rendering approach to compute a texture atlas map, which is efficient and widely supported in existing computer graphics tools. 
Specifically, we compute a blending weight at each 3D point on the mesh surface to determine its color as a weighted average of the colors at its image projections. The blending weights can be computed at a relatively coarse resolution, e.g., 512 volume resolution in our case, and leave texture details to the high-resolution images, such as 1K or 2K. Unlike previous methods that generate blurry texturing results under sparse input, our method generalizes well on both synthetic and real data with just a few input views. 
Figure~\ref{fig:teaser} shows two examples reconstructed by our method. Despite the challenging garment, pose, and occlusion, our method recovers faithful shape, normal, and texture on the right.

%with a wide variety of poses and clothing styles, and it is also adaptive to handle input image with arbitrary resolutions.
%\sicong{For this concern we claim that when the resolution of dicretized volume meets certain threshold (which is 256 in our experiment), the quantization error can be neglected.} 



In summary, the main contributions of this paper are as follows:
\begin{itemize}
\vspace{-0.1in}
  \item 
  We revisit the 3D volumetric representation and demonstrate that it can support clothed human reconstruction with equal or even better performance compared to implicit representation. 
  \item 
  We develop a memory and computation-efficient method for high-resolution volumetric reconstruction using sophisticated sparse 3D CNN, coarse-to-fine estimation, and voxel culling by visual hull and narrow bands. 
  \item 
  We introduce a novel method to compute a texture atlas map, which captures rich appearance details from high-resolution input images.
  \item 
  We achieve impressive results on standard benchmark datasets Twindom and MultiHuman, significantly reducing the point-2-surface (P2S) precision to approximately 0.2cm from just six input views, with more than $50\%$ error reduction compared to the state-of-the-art methods, including DoubleField~\cite{shao2022doublefield} and PIFuHD~\cite{saito2020pifuhd}.
\end{itemize}
\section{Ensembles and fermion actions}\label{sec:lattice}

We use the same computational setup as in several of our previous publications \cite{Junnarkar:2019equ,
Junnarkar:2018twb,Basak:2014kma,Padmanath:2017lng,Basak:2012py,Basak:2013oya,Mathur:2016hsm,
Mathur:2018epb,Mathur:2018rwu,Junnarkar:2022yak,Mathur:2022ovu}, which we briefly summarize below for completeness. Four 
$N_f=2+1+1$ lattice QCD ensembles generated by the MILC collaboration are used in this study \cite{MILC:2012znn}, where
the dynamical quark flavors were simulated using Highly Improved Staggered Quark (HISQ) action on gauge fields 
that respect one-loop, tadpole-improved Symanzik gauge action with tuned coefficients through 
$\mathcal{O}(\alpha_sa^2, n_f\alpha_sa^2)$ \cite{Follana:2006rc}. The charm and strange quark masses are 
tuned to their respective physical values, whereas the dynamical light quarks are chosen such 
that $m_s/m_l\sim 5$. We list the relevant details of various lattice QCD ensembles used in \tbn{lattice}.

\bet[tbh]
  \begin{center}
	  \begin{tabular}{p{1.5cm}p{1.5cm}p{1.5cm}>{\hfill\arraybackslash}p{1.5cm}}
      \hline
Label & Symbol & $a~[fm]$     & $N_s^3\times N_t$ \\ \hline
$S_1$ & \pmb{\textcolor{red}{\tikz{\pgfsetplotmarksize{0.8ex}\pgfuseplotmark{diamond}}}} & 0.1207(11)   & $24^3\times64$ \\
$S_2$ & \pmb{\textcolor{magenta}{\tikz{\pgfsetplotmarksize{0.8ex}\pgfuseplotmark{pentagon}}}} & 0.0888(8)    & $32^3\times96$ \\
$S_3$ & \pmb{\textcolor{blue}{\tikz{\pgfsetplotmarksize{0.7ex}\pgfuseplotmark{o}}}} & 0.0582(4)    & $48^3\times144$ \\
$L_1$ & \pmb{\textcolor{OliveGreen}{\pgfsetplotmarksize{0.7ex}\tikz{\pgfuseplotmark{square}}}} & 0.1189(9)    & $40^3\times64$ \\   \hline
  \end{tabular}
  \end{center}
\caption{Relevant details of the lattice QCD ensembles used. The lattice spacing estimates 
are measured using the $r_1$ parameter \cite{MILC:2012znn}. $L$ in $L_1$ refers to large spatial volume, 
and $S$ in $S_1,~S_2$, and $S_3$ refer to small spatial volume. }
\eet{lattice}

The valence quark fields for the light, strange and charm flavors are realized using an overlap 
fermion action that is $\mathcal{O}(am)$ improved. To this end, we utilize the numerical 
implementation of the overlap action following Refs. \cite{Chen:2003im,xQCD:2010pnl}. Following 
the Fermilab prescription \cite{El-Khadra:1996wdx}, the bare charm quark mass on each ensemble was tuned 
using the kinetic mass of spin averaged $1S$ charmonia $\{a\overline M_{kin}^{\bar cc} = 0.75 aM_{kin}(J/\psi) + 0.25 aM_{kin}(\eta_c)\}$
determined for the respective ensembles. Further details on the tuning of charm quark mass, 
the tuned bare quark mass, and resulting discretization effects are discussed in Refs. \cite{Basak:2012py,Basak:2013oya}.
The bare strange quark mass is set by equating the lattice estimate for the fictitious pseudoscalar $\bar ss$ 
meson mass to 688.5 MeV \cite{Chakraborty:2014aca}. Additionally, we perform the quark propagator 
measurements in the valence sector using overlap fermion action for three other quark masses in 
all the ensembles corresponding to pseudoscalar masses of approximately 0.5, 0.6 and 1.0 GeV. 

We employ a nonrelativistic QCD (NRQCD) Hamiltonian \cite{Lepage:1992tx} for the bottom quark. 
We tuned the bottom quark mass using the Fermilab prescription \cite{El-Khadra:1996wdx}, by equating 
the lattice extracted kinetic mass of the spin averaged 1S bottomonia $\{\overline M_{kin}^{\bar bb} = 0.75 M_{kin}(\Upsilon) + 0.25 M_{kin}(\eta_b)\}$
to its experimental value, where the kinetic mass is evaluated from the dispersion relation 
$aM_{kin}^2 = ((ap)^2 - (a\Delta E)^2)/2a\Delta E$. The details of NRQCD Hamiltonian, the improvement 
coefficients, and bottom quark mass tuning on our setup are discussed in Ref. \cite{Mathur:2016hsm}.

\bef[tbh]
% Figure removed
\caption{A landscape plot of the pseudoscalar masses corresponding to the quark mass that we have utilized 
in this work for different lattice ensembles used. The horizontal gray bands indicate a representative 
$M_{ps}$ estimate to guide the eye for a similar pseudoscalar meson mass across all four ensembles.} 
\eef{mpiVslat}

In this work, we assume isospin symmetry ($m_u = m_d$), and then for the channel that study here, 
involves three quark masses: the bottom ($b$), the charm ($c$), and the light ($u/d$) quarks.
For the light quark mass, we investigate five 
different cases: three unphysical quark masses discussed above [referred in terms of the 
corresponding approximate pseudoscalar meson masses $M_{ps}\sim$0.5, 0.6, and 1.0 GeV], the 
strange quark mass [$M_{ps}\sim$0.7 GeV] and the charm quark mass [$M_{ps}\sim$3.0 GeV]. In 
\fgn{mpiVslat}, we present the landscape of the five light quark masses studied in terms of the 
corresponding $M_{ps}$ versus the ensembles used. Using this setup, we evaluate the finite-volume spectrum in the 
isoscalar axialvector channel with $bc\bar u\bar d$ flavor for all these five quark masses on all 
four ensembles, next investigate the scattering of $D$ and $B^*$ mesons in all five scenarios and then 
extract the $m_{u/d}$ (otherwise $M_{ps}$) dependence of the scattering parameters. We utilize a wall-smearing procedure for all 
our quark propagator measurements (see Refs. \cite{Mathur:2018epb,Junnarkar:2018twb,Mathur:2022ovu} 
for details), and our primary focus on the finite-volume spectrum is on the ground state in each case. 






\section{Measurements and interpolators}\label{sec:2ptIO}

Lattice determination of finite-volume spectrum follows through an evaluation or measurement of Euclidean 
two-point correlation functions $\mathcal{C}_{ij}(t)$, of interpolating operators 
$\mathcal{O}_i(\mathbf{x},t)$ with desired quantum numbers, given by
\beq
\mathcal{C}_{ij}(t) = \sum_{\mathbf{x}}\left<\mathcal{O}_i(\mathbf{x},t)\mathcal{O}_j^{\dagger}(\mathbf{0},0)\right> = \sum_n \frac{Z_i^nZ_j^{n\dagger}}{2E^n} e^{-E^nt}.
\eeq{c2pt}
Here the second equality suggests that $\mathcal{C}_{ij}(t)$ can be expressed as a sum of exponentials 
following a spectral decomposition. $Z_i^n = \bra{0}\mathcal{O}_i\ket{n}$ is the operator-state overlap that 
quantifies the efficacy of the interpolator $\mathcal{O}_i$ in determining the time evolution of the state $n$. 
The utilization of wall smearing for the quark sources effectively kills all the high-momentum modes
at the source, whereas a zero momentum projection at the sink time slice ($\sum_{\mathbf{x}}$), as shown
in \eqn{c2pt}, efficiently projects the correlation function to the rest frame.  

Our main focus is on the ground state in the $T_{1}^+$ irreducible representation (irrep) in the rest frame, 
which is the only relevant rest frame finite-volume irrep for studying states in the infinite-volume continuum 
with quantum numbers ($J^P = 1^+$). To this end, we use a similar set of operators in the $T_{1}^+$ irrep as 
was utilized in Ref. \cite{Francis:2018jyb} and we briefly discuss them below for completeness. Assuming isospin symmetry, 
the relevant low-lying two-meson thresholds in the order of increasing energy are $E_{DB^*} = M_{B^*}+M_{D}$, 
$E_{BD^*}=M_B+M_{D^*}$, and $E_{D^*B^*}=M_B^*{}+M_{D^*}$. Hence, we consider the following low-lying two-meson 
interpolators 
\beqa
\mathcal{O}_1(x) &=& [\bar u(x) \gamma_i b(x)][\bar d(x) \gamma_5 c(x)]  \nonumber \\&& - [\bar d(x) \gamma_i b(x)][\bar u(x) \gamma_5 c(x)] \nonumber \\
\mathcal{O}_2(x) &=& [\bar u(x) \gamma_5 b(x)][\bar d(x) \gamma_i c(x)]  \nonumber \\&& - [\bar d(x) \gamma_5 b(x)][\bar u(x) \gamma_i c(x)] \nonumber \\
\mathcal{O}'(x) &=& \epsilon_{ijk} [\bar u(x) \gamma_i b(x)][\bar d(x) \gamma_j c(x)] \nonumber \\&& - [\bar d(x) \gamma_i b(x)][\bar u(x) \gamma_j c(x)].
\eeqa{mmops}
We utilize $\mathcal{O}_1(x)$ and $\mathcal{O}_2(x)$ in the computation of correlation functions. 
$\mathcal{O}'(x)$ has its associated two-meson threshold sufficiently higher up in the energy 
spectrum compared to the other two thresholds and it was found to have no effects in the low-lying 
energy spectrum. Hence we disregard this operator
from the rest of our analysis. Note that the lowest three particle threshold $DB\pi$ is above $E_{BD^*}$
for all the considered heavier-than-physical light quark masses. At $m_{u/d}^{phys}$, the $BD\pi$ threshold is 
immediately below $E_{BD^*}$, yet it remains sufficiently above $E_{DB^*}$ 
to have any significant effects on the ground states that we extract. We also compute two-point 
correlation functions for $B$, $B^*$, $D$, and $D^*$ mesons, using standard local quark 
bilinear interpolators ($\overline Q~\Gamma~q$) with spin structures $\Gamma\sim\gamma_5$ and 
$\gamma_i$ for pseudoscalar and vector quantum numbers, respectively. 

Phenomenologically, doubly bottom tetraquark in the axialvector channel is expected to be deeply 
bound. Such a state is expected to be quite compact owing to its doubly heavy flavor content 
and deeply bound nature \cite{Francis:2016hui,Czarnecki:2017vco}. Consequently, a local 
diquark-antidiquark interpolator is naturally interesting. Such an operator has already 
been utilized in all lattice QCD studies of the doubly bottom as well as bottom charm tetraquarks 
in the past \cite{Bicudo:2015kna,Francis:2016hui,Bicudo:2017szl,Junnarkar:2018twb,Leskovec:2019ioa,
Francis:2018jyb,Hudspith:2020tdf,Meinel:2022lzo,Hudspith:2023loy} and we follow the same strategy. 
Along with operators in \eqn{mmops}, we employ a local diquark-antidiquark interpolator 
\beq
\mathcal{O}_3(x) = (\bar u(x)^T \Gamma_5 \bar d(x) - \bar d(x)^T \Gamma_5 \bar u(x))( b(x) \Gamma_i c(x)),
\eeq{dadops}
where $\Gamma_k = C\gamma_k$ with $C=i\gamma_y\gamma_t$ being the charge conjugation matrix and 
the diquarks (antidiquarks) in the color antitriplet (triplet) representations. 

Our final basis is composed of the above-mentioned three interpolators $\{\mathcal{O}_1(x), \mathcal{O}_2(x), \mathcal{O}_3(x)\}$, 
which is diverse enough to reliably determine the ground state in the energy spectra that we are interested in.. Using this basis we determine 
the correlation matrices, with elements evaluated as prescribed in \eqn{c2pt}. Then the correlation matrices $\mathcal{C}$
are analyzed following a variational approach \cite{Michael:1985ne} to determine the energy estimates for 
low-lying levels in the spectrum. In this procedure, we look for the solutions of the generalized eigenvalue 
problem (GEVP) given by 
\beq
\mathcal{C}(t)v^n(t) = \lambda^n(t) \mathcal{C}(t_0)v^n(t),
\eeq{gevp}
where $t_0$ is a reference timeslice at which the eigenvalues $\lambda^n$s are identically unity. 
\bef[h]
% Figure removed
\caption{Effective energy plot for the eigenvalue correlation function $\lambda^0(t)$ (square) and for the product 
of single-meson correlators (circle) representing the noninteracting two-meson correlation function 
($\mathcal{C}_{D}(t)\mathcal{C}_{B^*}(t)$). The data correspond to $M_{ps} \sim 700$ MeV in the finest ensemble. 
The bands shown are the energy fit estimates for the final chosen time intervals.}
\eef{effmass}
The eigensolutions in the large time limit represent the lowest $N$ eigenstates $E^n$, for which the time 
evolution is dictated by the eigenvalues as $\lim_{t\to\infty}\lambda^n(t) \sim A_ne^{-E^nt}$. 
The corresponding eigenvectors are represented by $v^n(t)$, which are related to the operator-state-overlaps as
\beq
Z_i^{n}=\bra{0}\mathcal{O}_i \ket{n} = \sqrt{2E^n}(V^{-1})_i^n e^{E^{n}(t_0)/2},
\eeq{overlaps}
where $V$ is a matrix built out of $v^n(t)$. $v^n(t)$ is expected to be time independent in the limit, 
where the signal in $\mathcal{C}$ is saturated by the lowest $N$ eigenstates of the system.  

Conventionally the signal in the two point correlator data $C(t)$ is first assessed based on the 
large time plateauing in effective energies defined as $aE_{eff} = [ln(C(t)/C(t+\delta t))]/\delta t$. In 
\fgn{effmass}, we present the effective energies as a function of time for the eigenvalue correlation 
function (squares) and the noninteracting two-meson ($\mathcal{C}_{D}(t)\mathcal{C}_{B^*}(t)$) 
correlation function (circles). These effective energies can be seen to saturate around timeslices 
24 to 28 in the example shown. The results presented correspond to the lowest eigenvalue correlator 
$\lambda^0(t)$ at the strange quark mass ($M_{ps}\sim0.7$ GeV) in the finest ensemble we study. 
Evidently, there is a negative shift in the energies in $\lambda^0(t)$ with respect to the 
noninteracting energies at all times, except at very large times where the signal-to-noise ratio 
degrades substantially. 

Extraction of the energy spectra proceeds via fitting the eigenvalue correlators, $\lambda_{n}(t)$, 
with the expected asymptotic exponential behaviour. Alternatively, one can fit the asymptotic time 
estimates for the ratio of correlators given by 
\beq
R^n(t)=\frac{\lambda^n(t)}{\mathcal{C}_{m_1}(t) \mathcal{C}_{m_2}(t)}, 
\eeq{ratio}
to a single exponential form ($Ae^{-\Delta E^nt}$), where $\Delta E^n$ is expected to saturate to 
$E^n-M_{m_1}-M_{m_2}$ at large times. Here, $\mathcal{C}_{m_i}$ is the correlation function for 
the meson $m_i$, and $M_{m_i}$ is its mass. Being a ratio, $R^n(t)$ is empirically known to efficiently 
mitigate correlated noise between the product of two meson correlators and the interacting correlator 
for the two-meson system \cite{Green:2021qol}. Note that the automatic cancellation of the additive 
mass renormalization, inherent to NRQCD formulation, is an added advantage in using \eqn{ratio} for 
the fits. In \fgn{fitcompare}, we present a representative plot showing the $t_{min}$ dependence of 
the $\Delta E^n$ fit estimates determined from the fits to $\lambda^n(t)$ and $R^n(t)$, respectively, 
where $t_{min}$ is the lower boundary of the time interval used for these fits for a fixed upper boundary 
timeslice for the time interval. The energy differences are evaluated from $\lambda^n(t)$ using the relation 
$\Delta E^n = E^n-M_{m_1}-M_{m_2}$, where $M_{m_1}$ and $M_{m_2}$ are mass estimates for individual 
mesons determined from separate fits to $\mathcal{C}_{m_1}(t)$ and $\mathcal{C}_{m_2}(t)$, respectively. 
The estimates from different procedures can be seen to agree asymptotically in time, based on which 
optimal $t_{min}$ values are chosen. Our final results are based on fitting the ratio correlators 
defined in \eqn{ratio}.

    
\bef[h]
% Figure removed
\caption{$t_{min}$ dependence of the $\Delta E^0$ fit estimates determined from the fits to $\lambda^0$
and $R^0(t)$ for the case $M_{ps} \sim 700$ MeV in the finest ensemble. Here the superscript 0 refers 
to the ground state. }
\eef{fitcompare}



%%%%%%%%%%%%%%%%%%%%%%%%%%%%%%%%%%%%%%%%%%%%%%%%%%%%%%%%%%%%%%
\section{Energy spectra in finite-volume}\label{fvresults}
%%%%%%%%%%%%%%%%%%%%%%%%%%%%%%%%%%%%%%%%%%%%%%%%%%%%%%%%%%%%%%
In this section, we present our results that we obtain from the finite-volume correlators. 
After presenting the energy spectrum extracted using variational techniques, we discuss 
the operator-state-overlaps and the operator basis dependence. In the final subsection, 
we describe our strategy for rebuilding the ground state energies that are corrected for 
the additive NRQCD offset and for using them in further amplitude fits. 

\subsection{Details of energy spectra}
% Figure environment removed
In \fgn{spectrum}, we present the finite-volume energy spectra of the isoscalar 
axialvector $bc{\bar{u}}{\bar{d}}$ channel that we extract on the four ensembles 
listed in \tbn{lattice}, at the five different $m_{u/d}$ values corresponding to 
$M_{ps}\sim$ 0.5, 0.6, 0.7, 1.0, and 3.0 GeV. The energy spectrum is shown in 
lattice units. Note that these levels are shown with unaccounted additive 
renormalization measures related to the NRQCD-based dynamics of the heavy bottom quarks. 
The noninteracting two-meson energy levels corresponding to $DB^*$ and $BD^*$ thresholds 
are indicated as dotted horizontal line segments for each lattice and each $M_{ps}$. 
A clear trend for negative energy shifts can be observed in all the cases, indicating 
a possible attractive interaction between the scattering particles involved \cite{scalarbc}.
The $B^*D^*$ threshold in each case is also shown in the figure by dashed lines. 

From the energy spectra in the lattices $L_1$, $S_2$ and $S_3$, it can be observed 
that a consistent pattern emerges with respect to the two-meson thresholds.
The relative positioning of the ground state energy with the elastic threshold in the 
$S_1$ ensemble is also consistent with the other three ensembles. This is an encouraging 
feature in the finite-volume spectrum, as our main interest is on reliable extraction of the 
ground state energies. It is this ground state energy from each ensemble that we 
later on employ to constrain the $DB^*$ scattering amplitude.  

The excited states in the $S_1$ ensemble for $M_{ps}$, other than at the charm point, 
indicate enhanced negative shifts compared to that on the other ensembles. This could be 
related to a combination of effects arising from various less attractive features of the 
$S_1$ lattice, which includes the coarsest lattice spacing, small spatial volume and 
possible insufficient statistics for the study at lighter $M_{ps}$. To this end, we 
perform two additional checks. First, we make an associated study of the $S_1$ and the 
$L_1$ ensembles at the level of variational analysis and fitting procedures to determine 
the low-lying spectra with emphasis on the ground and the first excited states. We discuss 
this part of the investigation in Appendix \ref{app:S1L1}. Secondly, we perform amplitude 
fits with and without results from the $S_1$ ensemble to verify the robustness in our estimates 
for the scattering length. We discuss this in detail in Section \ref{Ampfits}.

\subsection{Operator-state overlaps}\label{sec:OSO}
\bef[hbt!]
% Figure removed
\caption{Normalized operator-state overlaps $\tilde{Z}_i^n$ for a state indicated by $n={0, 1, 2}$ 
and an operator represented by $\mathcal{O}_i$, where $i={1, 2, 3}$ on the $L_1$ ensemble. 
The errors in the normalized overlap factors are smaller than the symbols and hence are 
suppressed. The five horizontal panes stand for the five different $M_{ps}$ values we 
investigate. The two vertical lines in each horizontal pane separate $\tilde{Z}_i^n$ for 
different operators $\mathcal{O}_i$. }
\eef{Zratiosl40}
Now we investigate the operator-state overlaps $Z_i^n$, as in \eqn{overlaps}, to evaluate 
the efficacy of the interpolators in determining the low-lying spectra. To this end, 
we define normalized operator-state overlaps $\tilde{Z}_i^n$ such that its largest value 
for any given operator $\mathcal{O}_i$ across all the states $\{n\}$ is unity \cite{Dudek:2009qf,Padmanath:2013zfa}.
$\tilde{Z}_i^n$ quantifies the relative relevance of any given operator across all the 
states. In \fgn{Zratiosl40}, we present $\tilde{Z}_i^n$ at all $M_{ps}$ values we have used
on the $L_1$ ensemble. Each square marker corresponds to the $\tilde{Z}_i^n$ for a given operator 
$\mathcal{O}_i$ on to a given state $n$. Each horizontal pane stands for an $M_{ps}$ indicated on the 
right-hand side, whereas the vertical lines in each horizontal pane part $\tilde{Z}_i^n$ 
for different operators indicated on the top pane. The $x$-axis ticks refer to the three low-lying 
states we have extracted. $\mathcal{O}_1$, the two-meson operator related to $DB^*$ threshold, 
can be seen to have the largest overlap with the ground state and has significantly small 
overlaps with the excited states. $\mathcal{O}_2$, the two-meson operator related to $BD^*$ 
threshold, has the largest overlap with the first excited state and a very small overlap with 
the ground state. $\mathcal{O}_2$ also have nonnegligible overlap factors with the second 
excited state indicating $BD^*$-type two-meson Fock component, which decreases with increasing 
$M_{ps}$. On the other hand, $\mathcal{O}_3$, the diquark-antidiquark type operator, have 
substantial overlap factors with all states at the two lightest $M_{ps}$ values, whereas with 
an increased $M_{ps}$ its largest overlap is with the second excited state. Note that 
$\mathcal{O}_3$ is Fierz related to two-meson interpolators \cite{Padmanath:2015era}, and 
the large $\tilde{Z}_3^n$ values of $\mathcal{O}_3$ for all $n$ could be related to this 
underlying connection between two-meson and diquark-antidiquark operators. 


A summary from the above observations on overlap factors is as follows. $\mathcal{O}_1$
predominantly determines the ground state, whereas it has a significantly small coupling with 
the excited states. Similar patterns of overlap factors are also observed for other ensembles, 
all of which indicate that $\mathcal{O}_1$ predominantly determines the ground state. 
The two excited states have strong two-meson and diquark-antidiquark Fock components 
in the two lightest $M_{ps}$ values. The two-meson Fock components in the second excited state 
and the diquark-antidiquark Fock components in the first excited state decreases with 
increasing $M_{ps}$. This is consistent with the phenomenological expectation, which suggests 
that the binding energy in doubly heavy tetraquarks increases with increasing 
relative heaviness for the heavy quarks with respect to the light quarks \cite{Francis:2016hui,
Czarnecki:2017vco,Junnarkar:2018twb}. A deeply bound state could be significantly compact 
and hence could have large Fock components of a compact object such as that of a 
diquark-antidiquark. In other words, the relevance of compact diquark-antiquark operators for 
the low-lying spectrum increases with decreasing light quark mass, as is evident from \fgn{Zratiosl40}. 

\subsection{Operator basis dependence}
\bef[tbh!]
% Figure removed
\caption{Operator basis dependence of the low energy spectra of the $L_1$ ensemble and 
$M_{ps}\sim$700 MeV for all possible operator basis that can be built out of the three operators 
discussed in Section \ref{sec:2ptIO}. The basis is presented in digital notation ($x$-axis tick labels) 
where the operators are arranged in the order $\{\mathcal{O}_1, \mathcal{O}_2, \mathcal{O}_3\}$.
The horizontal lines refer to the $DB^*$, $BD^*$, and $D^*B^*$ thresholds. The bands indicate the 
bounds of the ground and first excited state energy estimates from the full three-operator basis. }
\eef{basisdep}
Next, we look into the basis dependence of the finite-volume energy spectra presented in 
\fgn{spectrum}. In \fgn{basisdep}, we show this basis dependence as determined for $M_{ps}\sim$ 
700 MeV in the $L_1$ ensemble, for various operator basis build out of $\mathcal{O}_1$, 
$\mathcal{O}_2$, and $\mathcal{O}_3$ operators as defined in \eqn{mmops} and \eqn{dadops}. The digital 
indexing on the $x$-axis tick labels refers to various operator basis in the order 
$\{\mathcal{O}_1, \mathcal{O}_2, \mathcal{O}_3\}$, with an overline on the third index 
as a visual aid within the plot to highlight the diquark-antidiqaurk interpolator. 1 (0) 
indicates an operator is included in (excluded from) the basis. The horizontal 
lines refer to the $DB^*$, $BD^*$ and $B^*D^*$ thresholds. The gray horizontal bands 
refer to the two lowest levels in the full basis indicated by $11\overline{1}$. A level 
below the threshold appears only when $\mathcal{O}_1$ is present in the basis. The first 
excited state in the full basis $11\overline{1}$ is faithfully reproduced in those bases 
where $\mathcal{O}_2$ is included. $\mathcal{O}_3$ alone does not precisely determine 
any level in the energy spectrum using full basis. Similar observations are also made 
on other ensembles. 

In summary, the ground state in the full basis $11\overline{1}$ is reliably determined 
with $\mathcal{O}_1$ and is unaffected by the inclusion of $\mathcal{O}_2$ and 
$\mathcal{O}_3$ operators. The excited states have nonnegligible overlap factors with 
$\mathcal{O}_2$ and $\mathcal{O}_3$ operators. Given our setup with only a few energy 
levels, any assumption more complicated than a simple elastic $DB^*$ assumption for 
the amplitude fits is beyond the scope of this work. Such an assumption is justified within the 
isosymmetric limit as the lowest inelastic threshold ($BD^*$ at unphysically heavy 
$m_{u/d}$ or $BD\pi$ for $m_{u/d}^{phys}$) is significantly high. In light of all 
these observations above, we limit ourselves to using only ground states determined 
from all ensembles at various $M_{ps}$ values to constrain the elastic $S$-wave
$DB^*$ scattering amplitude.

\subsection{The ground state energies}

\begin{figure}[h]
% Figure removed
\caption{The ground state energies in units of the elastic threshold ($DB^*$) on all 
ensembles (see \tbn{lattice} for color-symbol conventions) for all $M_{ps}$ values
(different vertical panes).}
\eef{gsspectrum}

In this subsection, we discuss how we obtain ground state energies after adjusting the 
additive correction that is inherent to an NRQCD calculation. The set of numbers that 
we extract from our variational analysis and eigenvalue correlator fitting procedures 
are the single meson masses $M_{B}$, $M_{B^*}$, $M_{D}$, and $M_{D^*}$ and the energy 
splittings $\Delta E_n = E_n - M_{m_1} - M_{m_2}$ (see \eqn{ratio}). %for the interacting system from the reference 
%two meson level $m_1m_2$ determined from the ratio of correlators defined in \eqn{ratio}. 
First, we account for the NRQCD corrections in single meson masses (involving a bottom quark) as 
\beq
\tilde{M}_{B^{(*)}} = M_{B^{(*)}} - 0.5\overline M^{\bar bb}_{lat} + 0.5 \overline M^{\bar bb}_{phys},
\eeq{msnNRcor}
where $\overline M^{\bar bb}_{lat}(\overline M^{\bar bb}_{phys})$ refers to the spin averaged mass 
of the $1S$ bottomonium measured on the lattice (experiments). We follow this procedure as we 
have tuned the bottom quark mass through the spin average bottomonia at each ensemble.

For the interacting energy spectrum, the NRQCD offset is automatically canceled in the energy splittings
$\Delta E^n$. One can then build the energy estimates $\tilde{E}^n$ of interacting spectrum by adding 
the noninteracting level energy ($M_{m_1} + M_{m_2}$) with the energy splittings $\Delta E^n$ as,
\beq 
\tilde{E}^n = \Delta E^n + M_{m_1} + M_{m_2}.
\eeq{intNRcor}
If either $m_1$ or $m_2$ is a bottom meson, we use the corresponding corrected $\tilde{M}_{m_i}$\footnote{From 
the next section, for brevity we suppress the $\tilde{}$ notation indicating corrected masses and energies.} 
determined using \eqn{msnNRcor}, instead of $M_{m_i}$. 

In \fgn{gsspectrum}, we present the corrected ground state energy estimates, 
in units of the energy of elastic threshold $E_{DB^*}$, at various $M_{ps}$ and 
for all the ensembles we have employed. The spectrum clearly shows a trend of decreasing 
energy spitting, hence decreasing interaction strength, with increasing $M_{ps}$. 
Another feature worth noting here is that the lattice spacing dependence of the 
ground state energies on similar volume ensembles ($S_1$, $S_2$, $S_3$) for the 
non-charm $M_{\pi}$ are opposite to that at the charm point. We will revisit this 
point when we discuss extraction of $DB^*$ scattering amplitude using these energy levels. 



%%%%%%%%%%%%%%%%%%%%%%%%%%%%%%%%%%%%%%%%%%%%%%%%%%%%%%%%%%%%%%
\section{$\mathbf{DB^*}$ scattering amplitude}\label{Ampfits}
%%%%%%%%%%%%%%%%%%%%%%%%%%%%%%%%%%%%%%%%%%%%%%%%%%%%%%%%%%%%%%

\subsection{Strategy}

The finite-volume energy splittings determined in the previous section are related to 
the infinite-volume scattering physics via L\"uscher's finite-volume prescription 
\cite{Luscher:1990ux} and its generalizations, e.g. \cite{Briceno:2014oea}. Assuming 
these energy splittings are purely described by an elastic scattering in the $DB^*$ 
system, we utilize them to constrain the associated $S$-wave scattering amplitude. Here 
we consider only the ground states in all ensembles for all quark mass scenarios, as 
the excited states are found to be affected by the inelastic $BD^*$ channel. 

It is interesting that even the excited states are also found to have statistically 
significant shifts with respect to the inelastic $BD^*$ threshold (see \fgn{spectrum}), 
possibly indicating nontrivial interactions between $B$ and $D^*$ mesons. If the $DB^*$ 
and $BD^*$ channels were totally decoupled, such shifts point to equivalent interactions 
in both channels \cite{bdsbc}. However, independent elastic analysis for the excited 
states is not well justified. On the other hand, the inclusion of excited states in our 
analysis demands an inelastic treatment involving more parameters than the available 
degrees of freedom in the amplitude fits, which is beyond the scope of this work. We 
also assume only negligible effects from higher partial waves or any off-shell pion 
exchange interactions that can induce coupling between $DB^*$ and $BD^*$ channels 
\cite{Du:2023hlu}, for the same reason. 

\subsection{Amplitude fits and continuum extrapolations}
For the scattering of a $D$ and a $B^*$ meson in the $S$-wave leading to total angular 
momentum and parity $J^P=1^+$, the scattering phase shifts $\delta_{l=0}(k)$ are related to 
the finite-volume energy spectrum through \cite{Luscher:1990ux}:
\beq
kcot[\delta_0(k)] = \frac{2Z_{00}[1;(\frac{kL}{2\pi})^2)]}{L\sqrt{\pi}},
\eeq{luscher}
where $k$ is the momentum of either mesons in the center of momentum frame corresponding 
to the center of momentum energy $E_{cm}=\sqrt{s}$. $k$ and $E_{cm}$ are related to each other through 
\beq
4sk^2 = (s-(M_{D}+M_{B^*})^2)(s-(M_{D}-M_{B^*})^2).
\eeq{k2cm}
A sub-threshold pole singularity in the $S$-wave scattering amplitude $t = ({\mathrm{cot}}\delta_0 - i)^{-1}$ 
occurs when $k{\mathrm{cot}}\delta_0 = \pm\sqrt{-k^2}$ 
\bet[hb]
  \begin{center}
          \begin{tabular}{p{2.0cm}p{2.0cm}p{2.0cm}>{\hfill\arraybackslash}p{2.cm}}
      \hline
      \hline
$M_{ps}$ [GeV] & $\chi^2/d.o.f$ & $A^{[0]}/E_{DB^*}$ & $A^{[1]}/E_{DB^*}$ \\\hline
\multirow{2}{*}{0.5} & 2.1/2 & $-0.05(1)$ & $~0.17(_{-11}^{+13})$ \\\cline{2-4} 
                     & 1.3/1 & $-0.05(1)$ & $~0.13(_{-12}^{+13})$ \\ \hline
\multirow{2}{*}{0.6} & 0.5/2 & $-0.044(_{-8}^{+9})$ & $~0.10(_{-9}^{+9})$ \\ \cline{2-4} 
                     & 0.3/1 & $-0.043(_{-8}^{+9})$ & $~0.09(_{-10}^{+9})$ \\ \hline
\multirow{2}{*}{0.7} & 3.0/2 & $-0.042(_{-6}^{+8})$ & $~0.09(_{-7}^{+6})$ \\ \cline{2-4} 
                     & 1.5/1 & $-0.040(_{-6}^{+8})$ & $~0.06(_{-8}^{+6})$ \\ \hline
\multirow{2}{*}{1.0} & 2.9/2 & $-0.043(4)$ & $~0.11(_{-5}^{+5})$ \\ \cline{2-4} 
                     & 0.4/1 & $-0.041(4)$ & $~0.14(_{-4}^{+5})$ \\ \hline
\multirow{2}{*}{3.0} & 3.6/2 & $~0.006(_{-5}^{+6})$ & $-0.20(_{-5}^{+4})$ \\ \cline{2-4} 
                     & 1.9/1 & $~0.010(_{-5}^{+6})$ & $-0.25(_{-5}^{+4})$ \\ \hline
      \hline
  \end{tabular}
  \end{center}
\caption{Results from amplitude fits for different light quark mass scenarios indicated 
in terms of $M_{ps}$ in the first column. For each $M_{ps}$, two independent fits are 
performed with (top row) and without (bottom row) the level from $S_1$ ensemble. All fits 
are performed with the parameterization in \eqn{linparam}, where the optimized parameter 
values in the table are presented in units of the $DB^*$ threshold, $E_{DB^*}$. }
\eet{Ampfits1}
for scattering in $S$-wave. We follow the procedure outlined in Appendix B of Ref. 
\cite{Padmanath:2022cvl} in constraining the amplitude, such that the parametrization of 
$k{\mathrm{cot}}\delta_0$ is tuned to satisfy Eq. (\ref{luscher}). The parametrized 
$k{\mathrm{cot}}\delta_0$ is then investigated for poles of $t$ in the complex energy plane.  

\begin{figure}[h]
% Figure removed
\caption{$k{\mathrm{cot}}\delta_0$, in units of the elastic threshold $E_{DB^*}$, versus $a$ 
(lattice spacing) for all $M_{\pi}$ values. We follow the marker/color coding in \tbn{lattice} 
for the data points referring to the simulated data. The colored/gray bands indicate the fit 
results to the continuum extrapolation fit form in \eqn{linparam} with/without the data from 
$S_1$ ensemble.} 
\eef{alatdep}
Since we use only the ground states for amplitude fits, we limit ourselves to a scattering 
amplitude parametrization that is completely described by scattering length $a_0$ in an effective 
range expansion near the threshold. Additionally, we also consider a lattice spacing dependence 
on the parametrization of $k{\mathrm{cot}}\delta_0$. We find that a linear functional form
given by 
\beq
k{\mathrm{cot}}\delta_0 = A^{[0]} + aA^{[1]}
\eeq{linparam}
provides acceptable fits to the scattering amplitudes. Such an $a$ dependence was also found to 
be necessary in our previous investigations using NRQCD framework as well \cite{Mathur:2022ovu}, 
and is consistent with the leading $a$ dependence of observables
involving an NRQCD evolution. In this form, $A^{[0]}=-1/a_0$, where $a_0$ is the scattering 
length in the continuum limit. 
We list the results from different amplitude fits in \tbn{Ampfits1}. In \fgn{alatdep}, we present 
the quality of these fits by comparing the fit results with the data points. The colored/gray 
bands indicate the fit results including/excluding results from the $S_1$ ensemble to the respective 
fits. It can be clearly seen that fit results are less affected by inputs from $S_1$ ensemble, 
which is obvious given the large uncertainties associated with them, in contrast to inputs from 
$L_1$, $S_2$, and $S_3$. In \fgn{pcotdelta_summary}, we present $k{\mathrm{cot}}\delta_0$ versus 
$k^2$ based on the ground state energies presented in \fgn{gsspectrum} following \eqn{luscher}. 
The colored/gray bands indicate continuum extrapolated results including/excluding results from 
the $S_1$ ensemble to the respective fits. Clearly, there are no statistically significant effects
from the inclusion/exclusion of the energy levels from the $S_1$ ensemble observed.
\begin{figure}[h]
% Figure removed
\caption{$k{\mathrm{cot}}\delta_0$ versus $k^2$ for all $M_{\pi}$ values studied in units of 
the elastic threshold $E_{DB^*}$. The data points refer to the simulated data and follow 
the color coding in \tbn{lattice}. The dashed orange (cyan) curve indicates the constraint 
for the existence of a sub-threshold pole in the scattering amplitude. The horizontal bands
are the continuum extrapolated estimates of $k{\mathrm{cot}}\delta_0$ for the respective
$M_{\pi}$ (see \fgn{alatdep}). }
\eef{pcotdelta_summary}

Our main aim is to reliably determine $A^{[0]}=-1/a_0$, the sign of which determines the fate 
of the near threshold pole, if there exists one. A negative (positive) value of $A^{[0]}$($a_0$) 
indicates that the interaction potential is strong enough to form a real bound state\cite{Landau:1991wop}. 
It can be seen from \tbn{Ampfits1} and \fgn{alatdep} that for the non-charm light quark masses, 
$A^{[0]}$, the continuum extrapolated value for $k{\mathrm{cot}}\delta_0$ is negative, which 
indicates a possibly strong attractive interaction sufficient enough to host a real bound state. 
Whereas at the charm point, despite the unambiguous negative energy shifts in the finite-volume 
ground state energies with respect to the elastic threshold, the attraction is weak to host any 
real bound state as suggested by the positive value of $k{\mathrm{cot}}\delta_0$ in the continuum 
limit. This observation goes in line with the phenomenological expectation for doubly heavy four 
quark ($QQ'l_1l_2$) systems with $m_{l_1}=m_{l_2}$ that the binding increases with increased 
relative heaviness of the heavy quarks with respect to its light quark content
\cite{Francis:2016hui,Czarnecki:2017vco,Junnarkar:2017sey}. 

Another interesting observation is related to the lattice spacing dependence of $k{\mathrm{cot}}\delta_0$
values. At the charm point, $A^{[1]}$ (see \eqn{linparam}) acquires a different signature in contrast 
to that for the light quark masses. This suggests that for a doubly heavy four quark ($QQ'l_1l_2$) system 
with $(m_{l_1} = m_{l_2}, ~m_{Q},m_{Q'}>>m_{l})$, the cut off effects weaken the finite-volume energy 
splitting of the ground state with the elastic threshold. On the other hand, at the charm point (where 
$m_{Q},m_{Q'}\sim m_{l}$) such effects enhance this energy splitting in the $QQ'l_1l_2$ system determined 
in a finite-volume. Relatively large errors at the noncharm $M_{ps}$ values partially obscure these effects, 
if any exist, while at the charm point such effects are clearly reflected. 

\subsection{Light quark mass dependence}
Following the individual amplitude fits to different light quark mass cases, now we investigate the light 
quark mass ($m_{u/d}$) or $M_{ps}$ dependence of the parameters $A^{[0]}$ and $A^{[1]}$. Due to leading order 
$M_{ps}^2$ terms in the chiral expansion, we assume the $M_{ps}$ dependence of hadron masses for 
light $m_{u/d}$ values ($m_q\lesssim\Lambda_{QCD}$) to be linear in $M_{ps}^2$. Whereas towards the heavy 
$m_{u/d}$ regime ($m_q>>\Lambda_{QCD}$) heavy hadron masses are expected to be proportional to the quark mass, 
hence to $M_{ps}$ \cite{Neubert:1993mb}. With these assumptions, we work with three following fit forms that 
could be useful. 
\beqa
	f_l(M_{ps}) &=& \alpha_c + \alpha_l M_{ps}, \nonumber \\
	f_s(M_{ps}) &=& \beta_c + \beta_s M_{ps}^2, \mbox{~~~and} \nonumber \\
	f_q(M_{ps}) &=& \theta_c + \theta_l M_{ps} + \theta_s M_{ps}^2.
\eeqa{mqdep}
Fits to determine the $M_{ps}$ dependence were made by minimizing a single cost function 
defined combinedly for $A^{[0]}$ and $A^{[1]}$ as %\cite{FullLuscher}
\beq
	\chi^2 =\sum_{\substack{x, y \\ \in \{A^{[j]}_{i}\}}}\left(f_x-f_{px}(M_{ps})\right)\tilde{\mathcal{C}}^{-1}_{xy}\left(f_y-f_{py}(M_{ps})\right),
\eeq{chi2mqdep}
where the summation runs over all fitted parameters $\{A^{[j]}_{i}\}$ with $j\in\{0, 1\}$ and $i$ 
referring to the five different light quark masses studied. In \tbn{Ampfits1}, we list the fit results 
for $f_{x,y}$. $\tilde{\mathcal{C}}_{ij}$ is the associated data covariance determined following Ref. \cite{Prelovsek:2020eiw}. 
$f_{pn}(M_{ps})$ are the fit forms incorporating the $m_{u/d}$ dependence in parameters $\{A^{[j]}_{i}\}$.  
In \fgn{a0a1_separate}, we show the fit results for $A^{[0]}=-1/a_0$ to the fit forms in \eqn{mqdep}. The large 
circles represent the $A^{[0]}$ values at different $M_{ps}$, the bands represent the fit results 
with different fit forms in \eqn{mqdep}, and the two stars represent $A^{[0]}$ at the physical 
$M_{ps}$ (equivalently the physical scattering length $a_0^{phys}$) and the critical $M_{ps}$ at which 
$A^{[0]}$ changes its sign (positive to negative), in other words, the system becomes unbound. It is 
indeed desired to have more points in the intermediate mass regime between the charm and the strange 
\begin{figure}[h]
% Figure removed
\caption{Continuum extrapolated $k{\mathrm{cot}}\delta_0$ or $A^{[0]}=-1/a_0$ estimates of the $DB^*$ system 
as a function of $M_{ps}^2$ in units of $E_{DB^*}$. The band indicates fit results to the simulated results. 
The legend carries info on the fit forms presented (see also \eqn{mqdep}) and the quality of fits. The dotted 
vertical line close to the $y$-axis indicates the physical $M_{ps}$. The two star symbols represent the 
amplitude at the physical $M_{ps}$ and the critical $M_{ps}$ at which the system becomes unbound.}
\eef{a0a1_separate}
quark masses to further constrain the dependence. Yet, our fits in this work demonstrate near independence in 
the fit forms as can be observed from the consistency between the error bands from different fit 
forms. 

\begin{figure}[h]
% Figure removed
\caption{The landscape of the continuum scattering length $A^{[0]}$ versus $A^{[1]}$ (see \eqn{linparam}) 
for all $M_{ps}$ values (indicated in the legend) studied. The central values are represented by black 
edged circles with color fillings, whereas the scattered points are the bootstrap samples. The band 
represents the correlated $M_{ps}$ dependence of the fitted parameters.} 
\eef{a0a1_combined}
Next we look at the correlated pion mass dependence in the parameters $A^{[0]}$ and $A^{[1]}$ (see 
\eqn{chi2mqdep} for the definition of the cost function) presented in \fgn{a0a1_combined}. The black 
bordered symbols are the central values of parameters determined for each $M_{ps}$, whereas scattered 
small circles indicate the bootstrap sample distribution in the $[A^{[0]},~A^{[1]}]$ landscape. The bands 
in the figure represent the uncertainty in the parameters, with the inner band quantifying the statistical 
errors, while the outer band also incorporates the systematic uncertainty arising from different fit forms 
added in quadrature symmetrically. A negative correlation can clearly be observed between the parameters 
across different quark masses studied, which is accounted in the fits through the data covariance matrix 
entering the cost function. This correlation can also be observed within the distribution of the bootstrap 
samplings at all quark masses. This observation clearly demonstrates the need for a careful treatment of 
cutoff errors, particularly in heavy hadron systems with interesting near threshold features, such as this. 

In the chiral regime ($m_{u/d}\lesssim\Lambda_{QCD}$), leading $m_{u/d}$ dependence in hadronic observables 
is assumed to go as linear in $M_{ps}^2$. Based on the fit form $f_s(M_{ps})$, we find that the scattering length 
of the $DB^*$ system at the physical light quark mass ($m_{u/d}^{phys}$) to be
\beq
a_0^{phys} = 0.57(^{+4}_{-5})(17) \mbox{~fm}.
\eeq{scatlen}
The asymmetric errors indicate the statistical uncertainties, whereas the second parenthesis quotes 
the systematic uncertainties with the most dominant contribution arising from the chiral extrapolation 
fit forms. We elaborate on various systematic uncertainties towards end of this section. The positive 
value of the scattering length is an unambiguous evidence for the ability/strength of the hadron-hadron 
interaction potential to host a real bound state (when $k~{\mathrm{cot}}\delta_0 = -\sqrt{-k^2}$). 
The observed scattering length at physical light quark mass suggests the presence of a real $bc\bar u\bar d$ 
tetraquark bound state $T_{bc}$ with binding energy 
\beq
\delta m_{T_{bc}} = -43(^{+6}_{-7})(^{+14}_{-24}) \mbox{~MeV},
\eeq{betbc}
with respect to $E_{DB^*}$. The systematic effects on the $a_0^{phys}$ and $\delta m_{T_{bc}}$ estimates 
of ignoring the charm point in the fits to the $m_{u/d}$ dependence are found to be very small, 
compared to the number quoted for systematic uncertainties in \eqn{scatlen}. 

Towards the heavy quark regime ($m_{u/d}>>\Lambda_{QCD}$), the heavy hadron masses can have 
leading linear dependence in $M_{ps}$ as $M_{ps}\propto m_{u/d}\sim m_{Q}$ \cite{Neubert:1993mb}. 
Following the fit form $f_l(M_{ps})$, which is linear in quark mass, the critical light quark mass 
$m_{u/d}^*$ at which the scattering length diverges, then changes its signature such that the 
interaction potential is not able host a real bound state, corresponds to the critical pseudoscalar 
meson mass given by 
\beq
M^{*}_{ps} = 2.73(21)(14) \mbox{~GeV}.
\eeq{unitary}
This corresponds to the star symbol at the zero crossing in the $x$-axis ($A^{[0]}=0$) in \fgn{a0a1_separate}. 
Once again the first parenthesis indicates the statistical errors and the second one quantifies various 
systematic uncertainties added in quadrature. 


Now we briefly comment on other possible sources of systematic uncertainties in this calculation. Our lattice setup, 
discussed in Section \ref{sec:lattice}, together with the bare bottom and charm quark mass tuning procedure 
has been demonstrated to reproduce the $1S$ hyperfine splittings in quarkonia with uncertainties less than 6 MeV
\cite{Mathur:2022ovu,Mathur:2016hsm}. Additionally, our strategy of evaluating the energy differences 
and working with mass ratios has also been shown to significantly mitigate the systematic uncertainties related 
heavy quark masses \cite{Mathur:2018epb,Mathur:2022ovu}. Our fitting procedure discussed in Section \ref{sec:2ptIO} 
involves careful and conservative determination of statistical errors, and uncertainties related to the 
excited-state-contamination and fit-window errors. The amplitude determination and followed extrapolations
are performed with results from varying the fit-windows to evaluate the uncertainties propagated to our final 
results. The uncertainties related to the fit forms used in chiral extrapolations are observed to be dominant,  
and the number in the second parenthesis in Eqs. \ref{scatlen}, \ref{betbc}, and \ref{unitary} are the total 
systematic uncertainties added in quadrature. Uncertainty related to scale setting are also found to be negligibly 
small in comparison to the statistical uncertainties \cite{Mathur:2018epb,Mathur:2022ovu}. 


%!TEX root = ../Schur indices and line operators.tex


\section{Discussion}












\begin{table*}[t]
\caption{Summary of the top-performing teams in each track of the RoboDepth Challenge.}
\centering\scalebox{1}{
\begin{tabular}{c|p{5cm}|p{5cm}}
\toprule
\textbf{Rank} & \textbf{\#1: Robust Self-Supervised MDE} & \textbf{\#2: Robust Supervised MDE}
\\\midrule\midrule
\multirow{13}{*}{\textcolor{robo_blue}{\textbf{1st Place}}} & \textbf{Team Name} & \textbf{Team Name}
\\
& \textcolor{robo_blue}{OpenSpaceAI} & \textcolor{robo_blue}{USTCxNetEaseFuxi}
\\
\cmidrule{2-3}
& \textbf{Team Members} & \textbf{Team Members}
\\
& Ruijie Zhu$^1$, Ziyang Song$^1$, Li Liu$^1$, Tianzhu Zhang$^{1,2}$ & Jun Yu$^1$, Mohan Jing$^1$, Pengwei Li$^1$, Xiaohua Qi$^1$, Cheng Jin$^2$, Yingfeng Chen$^2$, Jie Hou$^2$
\\
\cmidrule{2-3}
& \textbf{Affiliations} & \textbf{Affiliations}
\\
& $^1$University of Science and Technology of China, $^2$Deep Space Exploration Lab & $^1$University of Science and Technology of China, $^2$NetEase Fuxi
% \\
% \cmidrule{2-3}
% & \textbf{Approach} & \textbf{Approach}
% \\
% & IRUDepth with MPViT as depth encoder and PoseNet for camera poses and depth maps with AugMix& <...>
\\\cmidrule{2-3}
& \textbf{Contact} $\textrm{\Letter}$ & \textbf{Contact} $\textrm{\Letter}$
\\
& \texttt{ruijiezhu@mail.ustc.edu.cn} & \texttt{USTC\_IAT\_United@163.com}
\\\midrule\midrule
\multirow{17}{*}{\textcolor{robo_red}{\textbf{2nd Place}}} & \textbf{Team Name} & \textbf{Team Name}
\\
& \textcolor{robo_red}{USTC-IAT-United} & \textcolor{robo_red}{OpenSpaceAI}
\\
\cmidrule{2-3}
& \textbf{Team Members} & \textbf{Team Members}
\\
& Jun Yu$^1$, Xiaohua Qi$^1$, Jie Zhang$^2$, Mohan Jing$^1$, Pengwei Li$^1$, Zhen Kan$^1$, Qiang Ling$^1$, Liang Peng$^3$, Minglei Li$^3$, Di Xu$^3$, Changpeng Yang$^3$ & Li Liu$^1$, Ruijie Zhu$^1$, Ziyang Song$^1$, Tianzhu Zhang$^{1,2}$
\\
\cmidrule{2-3}
& \textbf{Affiliations} & \textbf{Affiliations}
\\
& $^1$University of Science and Technology of China, $^2$Central South University, $^3$Huawei Cloud Computing Technology Co., Ltd & $^1$University of Science and Technology of China, $^2$Deep Space Exploration Lab
\\
\cmidrule{2-3}
& \textbf{Contact} $\textrm{\Letter}$ & \textbf{Contact} $\textrm{\Letter}$
\\
& \texttt{USTC\_IAT\_United@163.com} & \texttt{liu\_li@mail.ustc.edu.cn}
\\\midrule\midrule
\multirow{11}{*}{\textcolor{robo_green}{\textbf{3rd Place}}} & \textbf{Team Name} & \textbf{Team Name}
\\
& \textcolor{robo_green}{YYQ} & \textcolor{robo_green}{GANCV}
\\
\cmidrule{2-3}
& \textbf{Team Members} & \textbf{Team Members}
\\
& Yuanqi Yao$^1$, Gang Wu$^1$, Jian Kuai$^1$, Xianming Liu$^1$, Junjun Jiang$^1$ & Jiamian Huang$^1$, Baojun Li$^1$
\\
\cmidrule{2-3}
& \textbf{Affiliations} & \textbf{Affiliations}
\\
& $^1$Harbin Institute of Technology & $^1$Individual Researcher
\\
\cmidrule{2-3}
& \textbf{Contact} $\textrm{\Letter}$ & \textbf{Contact} $\textrm{\Letter}$
\\
& \texttt{yuanqiyao@stu.hit.edu.cn} & \texttt{huang176368745@gmail.com}
\\\bottomrule
\end{tabular}
}
\label{tab:summary}
\end{table*}
%%%%%%%%%%%%%%%%%%%%%%%%%%%%%%%%%%%%%%%%%%%%%%%%%%%%%%%%%%%%%%%%%%%%%%%%%%%%%%%%
%% ACKNOWLEDGMENTS
\begin{acknowledgments}
%
This work is supported by the Department of Atomic Energy, Government of India, under Project Identification Number RTI 4002. We are thankful to the MILC collaboration and in particular to S. Gottlieb for providing us with the HISQ lattice ensembles. We thank Sara Collins for a careful reading of the manuscript. We thank the authors of Ref. \cite{Morningstar:2017spu} for making the {\it TwoHadronsInBox} package utilized in this work. We also thank Gunnar Bali, Parikshit Junnarkar and Sayantan Sharma for discussions. Computations were carried out on the Cray-XC30 of ILGTI, TIFR. Amplitude analyses were performed on Nandadevi computing cluster at IMSc Chennai. N. M. would also like to thank A. Salve and K. Ghadiali for computational support.
\end{acknowledgments}



\begin{comment}
\section{System Architecture}
\label{appendix:architecture}
\system has a novel modularized system architecture with three key components: 
\emph{StreamManager}, 
\emph{TxnManager} and \emph{TxnScheduler}. 
These components are instantiated in each thread locally.
The execution outline of \system is presented in Algorithm~\ref{alg:algo}.
Transactional stream processing is continuous and potentially never ends (Line 1$\sim$8).
The dependency resolution and execution of state transactions are separated into two non-overlapping phases by punctuations~\cite{Tucker:2003:EPS:776752.776780} (Line 2 and 5), which guarantees that no subsequent input event will have a smaller timestamp. 
Effectively, a batch of state transactions is collected during the first phase, and processed during the second phase.

In the first phase (i.e., stream processing phase), 
the \emph{StreamManager} conducts preprocessing for every input event ($e$). Similar to some prior works~\cite{tstream}, state transactions may be issued but not immediately processed during preprocessing (Line 3).
The \emph{pre\_processing} and \emph{post\_processing} functions are exposed as APIs to users.
The \emph{TxnManager} handles dependency resolution (Line 4) among state transactions and insert decomposed operations to construct a \tpg. We discuss the detailed two-phase \tpg construction process in Section~\ref{subsec:construction}.

In the second phase  (i.e., transaction processing phase), 
the \emph{TxnManager} is first involved again to refine (Line 6) the constructed \tpg with further dependency resolution.
The \emph{TxnScheduler} 
schedules operations for concurrent execution based on the constructed \tpg according to the three dimensions of scheduling decisions (Line 7). 
In particular, a scheduling decision model $M$ is instantiated based on the constructed \tpg (Line 14).
\textbf{\circled{1}} Guided by $M$, execution threads adopt an exploration strategy (Section~\ref{subsec:explore}) to explore the constructed \tpg for operations available to be scheduled constrained by dependencies. 
\textbf{\circled{2}} 
During exploration, one or multiple operations may be treated as the 
% basic 
unit of scheduling (Section~\ref{subsec:granularity}). 
Subsequently, \textbf{\circled{3}} every thread executes operation(s) in the unit of scheduling with various abort handling mechanisms (Section~\ref{subsec:abort_handling}).
Only when state transactions are processed (i.e., committed or aborted) can the associated input events be postprocessed (Line 8) by the \emph{StreamManager} based on transaction processing results.
\end{comment}

\begin{comment}
\begin{algorithm}
\footnotesize
    \KwData{$e$ \tcp{Input event}}
    \KwData{$txn_{ts}$ \tcp{State transaction}}
    \KwData{$G$ \tcp{The currently constructed TPG}}
    \While{!finish processing of input streams}{
        \eIf(\tcp*[h]{Phase 1}){\text{$e$ is not a $punctuation$}}{
                $txn_{ts}$ $\gets$ PRE\_Processing($e$)\;
                \textbf{TPG\_Construction}($G$, $txn_{ts}$)\; 
          }(\tcp*[h]{Phase 2}){
                \textbf{TPG\_Refinement}($G$)\; 
                \textbf{TXN\_Scheduling}($G$)\; 
                POST\_Processing()\;
          }
    }
    
    \SetKwFunction{FMain}{TPG\_Construction}
    \SetKwProg{Fn}{Function}{:}{}
    \Fn{\FMain{$G$, $txn_{ts}$}}{
        $O_{1..k}$ $\gets$ \textbf{Partition} $txn_{ts}$\;
        \ForEach{\text{operation $O_{i}$ $\in$ $O_{1..k}$}}{
            \textbf{Identify} its \ld\;
            $G$ $\gets$ $G$ + $O_{i}$ \;
        }
    }
    \SetKwFunction{FMain}{TPG\_Refinement}
    \SetKwProg{Fn}{Function}{:}{}
    \Fn{\FMain{$G$}}{
        \ForEach{\text{vertex $e_{i}$ $\in$ $G$}}{
            \textbf{Identify} its \td, \pd\;
        }
    }
    
    \SetKwFunction{FMain}{TXN\_Scheduling}
    \SetKwProg{Fn}{Function}{:}{}
    \Fn{\FMain{$G$}}{
        $M$ $\gets$ Instantiated with $G$;\tcp{A decision model}
        \While{!finish scheduling of $G$
        }{
          \textbf{\circled{2}} $Scheduling Unit$ $\gets$ \textbf{\circled{1}} \emph{Explore}($G$, $M$)\; 
            \textbf{\circled{3}} \emph{Execute with Abort Handling} ($Scheduling Unit$)\; 
        }
    }
  \caption{Execution Outline of \system}
  \label{alg:algo}
\end{algorithm}
\end{comment}



%%%%%%%%%%%%%%%%%%%%%%%%%%%%%%%%%%%%%%%%%%%%%%%%%%%%%%%%%%%%%%%%%%%%%%%%%%%%%%%%%
%% BIBILOGRAPHY
%\bibliographystyle{apsrev4-1}
%\bibliography{bib}
\bibliography{paper}

%%%%%%%%%%%%%%%%%%%%%%%%%%%%%%%%%%%%%%%%%%%%%%%%%%%%%%%%%%%%%%%%%%%%%%%%%%%%%%%%%


\end{document}


\clearpage
\widetext
\appendix

\section{GSF coefficients of the $A$ EOB potential}
\label{app:APN}

For completeness, we collect here the %PN (up to NNLO) and 
GSF coefficients entering the radial
potential $A(r)$, see Sec.~\ref{sec:eobmodel}.
%\subsection{PN coefficients}

For $\ell=2$, the 0GSF, 1GSF, 2GSF terms explicitly read:
\begin{eqnarray}
\label{eq:A2p_gsf}
\hat{A}_A^{(2+)\rm 0GSF} =& 1 + \frac{3 u^2}{1-\rLR u} \, , \\
\hat{A}_A^{(2+)\rm 1GSF} =& \frac{1}{(1-\rLR u)}\frac{5}{2} u (1-a_1 u)(1-a_2 u) \frac{1+n_1 u}{1 + d_2 u^2} \, , \\
\hat{A}_A^{(2+)\rm 2GSF} =& \frac{337}{28} \frac{u^2}{(1-\rLR u)^p} \, ,
\end{eqnarray}
while for $\ell=3$ we have:
\begin{eqnarray}
\label{eq:A3p_gsf}
\hat{A}_A^{(3+)\rm 0GSF} =& (1-2u)(1+\frac{8}{3}\frac{u^2}{1-\rLR u}) \, , \\
\hat{A}_A^{(3+)\rm 1GSF} =& \frac{1}{(1-\rLR u)^{7/2}}\frac{15}{2} u (1+c_1 u +c_2u^2+c_3 u^3) \\
& \times \frac{1+c_4 u + c_5 u^2}{1 + c_6 u^2} \, , \\
\hat{A}_A^{(3+)\rm 2GSF} =& \frac{110}{3} \frac{u^2}{(1-\rLR u)^p} \, .
\end{eqnarray}
The parameters ($a_1, \dots,a_4$) and ($c_1, \dots,c_6$) can be read from Eq. (17) and Eq. (29) of \cite{Akcay:2018yyh}.

\section{Simulations}
\label{app:NR}

\begin{table}
\begin{tabular}{ccccccccc}
\hline\hline
Sim. & $L$ & $l^{\rm mv}$ & $n$ & $n^{\rm mv}$ & $h_{L-1}$ & $h_0$ \\
\hline 
ER01 & 8 & 2 & [144,216,288] & [72,108,144] & [0.25,0.16667,0.125] & [32.0,21.33376,16.0]\\
ER02 & 7 & 2 & [128,192,256] & [64,96,128] & [0.28125,0.1875,0.140625] & [18.0,12.0,9.0]\\
ER03 & 7 & 2 & [128,192,256] & [64,96,128] & [0.28125,0.1875,0.140625] & [18.0,12.0,9.0]\\
ER04 & 8 & 2 & [144,216,288] & [72,108,144] & [0.25,0.16667,0.125] & [32.0,21.33376,16.0]\\
ER05 & 8 & 2 & [144,216,288] & [72,108,144] & [0.25,0.16667,0.125] & [32.0,21.33376,16.0]\\
ER06 & 7 & 2 & [128,192,256] & [64,96,128] & [0.249,0.166,0.1245] & [15.936,10.624,7.968]\\
ER10 & 7 & 2 & [128,192,256] & [64,96,128] & [0.28125,0.1875,0.140625] & [18.0,12.0,9.0]\\
ER11 & 7 & 2 & [128,192,256] & [64,96,128] & [0.249,0.166,0.1245] & [15.936,10.624,7.968]\\
ER12 & 7 & 2 & [128,192,256] & [64,96,128] & [0.28125,0.1875,0.140625] & [18.0,12.0,9.0]\\
ER13 & 7 & 2 & [128,192,256] & [64,96,128] & [0.28125,0.1875,0.140625] & [18.0,12.0,9.0]\\
ER14 & 7 & 2 & [128,192,256] & [64,96,128] & [0.28125,0.1875,0.140625] & [18.0,12.0,9.0]\\
ER15 & 7 & 2 & [128,192,256] & [64,96,128] & [0.28125,0.1875,0.140625] & [18.0,12.0,9.0]\\
ER17 & 7 & 2 & [128,192,256] & [64,96,128] & [0.28125,0.1875,0.140625] & [18.0,12.0,9.0]\\
ER18 & 7 & 2 & [128,192,256] & [64,96,128] & [0.28125,0.1875,0.140625] & [18.0,12.0,9.0]\\
\hline\hline
\end{tabular}
\caption{Grid parameters for the NR simulations. From the second to the last column: Refinement levels, number of moving levels (finest), number of points per direction in non-moving refinement levels, number of points per directions in moving refinement levels, grid spacing per direction in the finest level, grid spacing in the coarser level.}
\label{tab:nr_grids}
\end{table}

Initial data for the simulations are prepared using the formalism and iterative procedure described in Refs.~\cite{Moldenhauer:2014yaa,Dietrich:2015pxa}. 
Typical initial eccentricities estimates from the coordinate distance are of the order of $10^{-4}$ after four iterations.
%
Simulations are performed with the methods described in Ref.~\cite{Bernuzzi:2016pie}, in particular a high-order scheme for 
hydrodynamics based on characteristic reconstruction and the WENOZ reconstruction. 
The grid setups employed in the simulations are detailed in Tab.~\ref{tab:nr_grids}.
%
Their self-convergence is assessed by computing the phase difference between $h_{22}$ considered at different resolutions.
The convergence rate $p$ is found by rescaling such differences by the scaling factor $\sigma_p$, defined as~\cite{Gonzalez:2022mgo}
\be
\sigma_p = \frac{\Delta^p_{\rm LR} - \Delta^p_{\rm SR}}{\Delta^p_{\rm SR} - \Delta^p_{\rm HR}} \, ,
\ee
where (LR, SR and HR) denote respectively the low, standard and high resolution simulations and $\Delta_x$ is the 
grid spacing at the resolution $x$.
Figure~\ref{fig:convplots} shows the self convergence test for four representative simulations.
Overall, we find that ten of the new simulations behave like $\tt ER01$ or $\tt ER14$ shown, 
which fail to show clear convergence over the full waveform. 
The remaining four simulations, instead, converge either at first or third order, similar to $\tt ER04$ or $\tt ER10$ shown.

Given the lack of clear convergence for most of the simulations considered,
we do not Richardson extrapolate the waveforms. We do, however, extrapolate 
them to null infinity using a polynomial in $1/R$ of order $K=2$.
%
The error budget of the simulation is then estimated as the sum in quadrature of the resolution error,
computed considering the phase difference between the highest and second highest simulations, and the extraction error,
computed as the phase difference between the waveform extrapolated to null infinity and the waveform extracted at the largest $R$
available (typically either 1000M or 1600M).

% Figure environment removed

\section{Fit for $\dot{\omega}_{\rm mrg}^{\rm NR}$}
\label{app:domg_fit}
We fit the time derivative of the NR frequency at merger employing all simulations associated to the 
first release of the {\tt CoRe} database \cite{Dietrich:2018phi} and a handful of simulations from the {\tt SACRA} database \cite{Kawaguchi:2018gvj,Kiuchi:2017pte,Kiuchi:2019kzt}.
Following Ref.~\cite{Breschi:2022xnc}, we assume a functional form of the type:
\begin{equation}
\dot{\omega}_{\rm mrg}^{\rm NR} = a_0 Q^M(X) Q^S(\hat{S}, X) Q^T(\kappa^T_2, X) \, ,
\label{eq:domg_mrg_fit}
\end{equation}
where $X=X_A - X_B$, $\hat{S} = S_A + S_B$, $\kappa_2^T = \kappa_A^{(2+)} + \kappa_B^{(2+)}$, and:
\begin{align*}
Q^M &= 1+ a_1^M X \, ,\\
Q^S &= 1+ a_1^S(1+b_1^S X) \hat{S} \,, \\
Q^T &= \frac{1 + a_1^T(1+b_1^T X)^T \kappa_2 ^T + a_2 ^T(1+b_2 ^T X){\kappa_2^T}^2}{1 + a_3^T(1+b_3^T X)\kappa_2^T + a_4^T(1+b_4^T X) {\kappa_2^T}^2} \, .
\end{align*}
The coefficients of the fit can be read from Tab.~\ref{tab:domg_fits}.

\begin{table}
	\centering
	% \resizebox{\textwidth}{!}{
	  \begin{tabular}{c|cccc||cccc}
	  \hline\hline
	  X  &  $a_1^X$ & $a_2^X$ &$a_3^X$& $a_4^X$ & $b_1^X$ & $b_2^X$ &$b_3^X$& $b_4^X$ \\
	  \hline 
	  $M$ & -1.7988 & -- &--      & --     & --      & --  & --     & -- \\
	  $S$ & 0.3555  & -- &--      & --     & -7.1674 & --  & --     & -- \\
	  $T$ & 0.0314  & 0  & 0.1714 & 0.0006 & -6.8144 & 1.0 & 5.1651 & -1.0333 \\
	  \hline \hline
	  \end{tabular}
	% }
	\caption{  
	\label{tab:domg_fits}
	The coefficients for $\dot{\hat{\omega}}_{\rm mrg}^{\rm NR}$ from Eq.~\ref{eq:domg_mrg_fit}. The fitted value of $a_0$ is $a_0=0.0074$.
	}
\end{table}

\section{Tidal PN coefficients for phase and amplitude}
\label{app:PNcoefs}

We collect the PN coefficients entering the
phenomenological representation of \TEOB{}. We also compute for the
first time - to our knowledge - 
the contribution of spin-quadrupole interactions to the frequency
domain amplitude $\tilde{A}_{22}$. 

\subsection{Phase}
The pure tidal PN coefficients entering Eq. \eqref{eq:PsiFD} and
Eq. \eqref{eq:PadPhase} beyond leading order (LO) can be compactly 
expressed as 
\be 
c_i^{\Lambda} = \Bigl[-\kappa_A^{(2+)} \frac{3}{16 \nu} \frac{(12 - 11 X_A)}{(1 - X_A)} \bar{c}^\Lambda_i(X_A) + (A \leftrightarrow B)\Bigr]/c^{\Lambda}_{\rm LO} \,,
\ee
where $i$ denotes the relative PN order beyond LO, and the $c^{\Lambda}_{\rm LO}, \bar{c}^\Lambda$ coefficients
are given by:

%\begin{widetext}
\begin{subequations}
\begin{align}
c^\Lambda_{\rm LO}    =& -\kappa_A^{(2+)} \frac{3}{16 \nu} \frac{(12 - 11 X_A)}{(1 - X_A)} + (A \leftrightarrow B) \, ,\\
\bar{c}^\Lambda_1     =& -\frac{2}{4} \frac{5(260 X_A^3 - 2286 X_A^2 - 919 X_A + 3179) }{336 (11 X_A - 12)} \, ,\\
\bar{c}^\Lambda_{3/2} =& -\pi \, ,\\
\bar{c}^\Lambda_2     =& \frac{5 (67702048 X_A^5 - 223216640 X_A^4 + 337457524 X_A^3 - 141992280 X_A^2 + 96008668 X_A - 143740242)}{3 \times 3048192 (11 X_A - 12)} \, ,\\
\bar{c}_{5/2} =& -\frac{\pi (10232 X_A^3 - 7022 X_A^2 + 22127 X_A - 27719)}{192 (11 X_A - 12)} \,.
\end{align}
\end{subequations}
%\end{widetext}
The monupole-quadrupole interactions included in Eq.~\eqref{eq:PsiFDss} instead read:
%\begin{widetext}
\begin{subequations}
\begin{align}
c_{\rm LO}^{\rm MQ}  =& -50 (C_{QA} \tilde{a}_A^2 + C_{QB} \tilde{a}_B^2) \, ,\\
c_{1}^{\rm MQ}  =& \frac{5}{84}\Bigl[(9407 + 8218 X_A + 2016 X_A^2) C_{QA}  \tilde{a}_A^2 + (A  \leftrightarrow B)\Bigr]/c_{\rm LO}^{\rm MQ} \, ,\\
c_{3/2}^{\rm MQ} =& \Bigl\{ 10\Bigl[(X_A + 308/3)\tilde{a}_A - (X_A + 86/3)\tilde{a}_B - 40 \pi\Bigr] C_{QA} \tilde{a}_A^2 - 440 C_{\rm OcA} \tilde{a}_A^3 +  (A  \leftrightarrow B) \Bigr\} /c_{\rm LO}^{\rm MQ} \,,
\end{align} 
\end{subequations}
%\end{widetext}
where $\tilde{a}_i = X_i \chi_i$, $i=A,B$.

\subsection{Amplitude}
The SPA allows us to compute the corrections to the frequency domain
amplitude of the waveform beyond leading PN order. Assuming that the
only modes contributing to the waveform are those with $\ell=2, |m|=2$
and denoting the amplitude of the 22 mode as $A_{22}$, one
has that
\begin{subequations}
\begin{align}
\label{eq:af_def}
\tilde{A}_{22} = & A_{22}(t_f)\sqrt{\frac{\pi}{\ddot{\phi}(t_f)}} \, ,\\
\ddot{\phi}(x) = & -\frac{3}{2} x^{1/2} \frac{\mathcal{F}(x)}{E'(x)} \,.
\end{align}
\end{subequations}
Where $E(x)$ and $\mathcal{F}(x)$ denote, respectively, the energy and its flux.
We employ expressions for $A_{22}$, $E(x)$ and $\mathcal{F}(x)$ 
which contain point-particle and spin-orbit corrections known up to 3.5 PN
\cite{Mikoczi:2005dn, Arun:2008kb, Buonanno:2009zt}, spin-quadrupole corrections up to
relative 1PN \cite{Bohe:2015ana, Messina:2018ghh}, and pure tidal corrections up to relative 1PN \cite{Vines:2011ud, Damour:2012yf, Wade:2014vqa}.
Expanding Eq. \eqref{eq:af_def}, we find that the waveform amplitude
is given by 
\be
\label{af_gen}
\tilde{A}_{22}(x) = \sqrt{\frac{2 \nu}{3}}\pi x^{-7/4} \sum_{i=0}^{i=X}{c_{i/2}   x^{i/2}} \,.
\ee
The coefficients of the point mass and spinnning terms are given in
e.g \cite{Khan:2015jqa}. Spin-quadrupole and tidal effects enter the amplitude at
relative 2 and 5 PN orders. The LO and NLO coefficients of both are given by: 
\begin{subequations}
\begin{align}
\label{af_coeff}
c_2^{\rm MQ} =& -\frac{3}{2}(C_{QA} \tilde{a}_A^2 + C_{QB}\tilde{a}_B^2) \, ,\\
c_{5/2}^{\rm MQ} =& -C_{QA} \tilde{a}_A^2 (\frac{12247}{1344}-\frac{2221}{672} \sqrt{1-4\nu}+\frac{53}{336} \nu ) +
      -C_{QB} \tilde{a}_B^2 (\frac{12247}{1344}+\frac{2221}{672} \sqrt{1-4\nu}+\frac{53}{336} \nu ) \, ,\\
c_{5}^{\Lambda} =& \frac{27}{2} \nu \left[(\Lambda_A + \Lambda_B)(-1 + 3\nu) + (\Lambda_A - \Lambda_B)\sqrt{1 - 4 \nu} (\nu -1)\right] \, ,\\
c_{6}^{\Lambda} =& \frac{1}{448}\Bigl[(\Lambda_A+\Lambda_B)(930 -40455 \nu + 131233 \nu^2 - 37748 \nu^3) 
     + (\Lambda_A - \Lambda_B)\sqrt{1-4\nu}(930 - 38595 \nu + 55903 \nu^2 
		        + 588 \nu^3)\Bigr] \,.
\end{align}
\end{subequations}
\end{document}
