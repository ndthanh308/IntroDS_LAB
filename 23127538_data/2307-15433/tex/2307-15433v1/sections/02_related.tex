%!TEX root = ../main.tex

\section{Related work}
\label{sec:related_work}

Besides insect monitoring~\cite{jonason2014lighttrap,svenningsen2020contrasting,bjerge2021automated,Radig2021:AVL,Korsch21_DLP}, there also exist automatic recognition systems for other animals.
They are often used to re-identify individuals of a certain species, e.g., great apes~\cite{Freytag16_CFW,Brust2017AVM,Kaeding18_ALR,yang2019great,sakib2020visual}, elephants~\cite{Koerschens18:Elephants,Koerschens19:ELPephants}, or sharks~\cite{hughes2017automated}, to name a few.
Of course, the field of fine-grained recognition mainly benefited from bird species datasets like CUB~\cite{WahCUB_200_2011} and NA-Birds~\cite{NABirds}.

For moth species identification, datasets exist with insects from Ecuador and Costa Rica containing images of 675 and 331 different species, respectively~\cite{Rodner15:FRD}.
There is only a single individual in each image spanning the whole image area.
In contrast to these datasets, we consider images of moth species from Central Europe recorded by light-based camera traps.
First, the recorded insects are still alive and may take various positions, making it harder to analyze the image.
Second, the captured image contains multiple insects, and individuals need to be localized before inferring the species.
Furthermore, we extend our datasets over time due to continuous monitoring.

There are also similar camera trapping systems developed by other groups, e.g., as reported in~\cite{bjerge2021automated}.
So far, they only consider eight different moth species and do not provide ground truth bounding box annotations for the individual insects.
In contrast, our EU-Moths dataset contains \num{200} different species, and we provide algorithms that can reliably recognize individuals from this larger set of species.

