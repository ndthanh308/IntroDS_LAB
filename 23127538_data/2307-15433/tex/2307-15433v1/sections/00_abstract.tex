%!TEX root = ../main.tex

\begin{abstract}
Automatic camera-assisted monitoring of insects for abundance estimations is crucial to understand and counteract ongoing insect decline.
In this paper, we present two datasets of nocturnal insects, especially moths as a subset of Lepidoptera, photographed in Central Europe.
One of the datasets, the EU-Moths dataset, was captured manually by citizen scientists and contains species annotations for \num{200} different species and bounding box annotations for those.
We used this dataset to develop and evaluate a two-stage pipeline for insect detection and moth species classification in previous work.
We further introduce a prototype for an automated visual monitoring system.
This prototype produced the second dataset consisting of more than \num{27000} images captured on \num{95} nights.
For evaluation and bootstrapping purposes, we annotated a subset of the images with bounding boxes enframing nocturnal insects.
Finally, we present first detection and classification baselines for these datasets and encourage other scientists to use this publicly available data.
\end{abstract}
