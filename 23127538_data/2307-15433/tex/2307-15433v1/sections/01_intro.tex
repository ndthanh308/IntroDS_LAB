%!TEX root = ../main.tex

\section{Introduction}
\label{sec:intro}

Climate change and the decline of species richness are severe challenges that influence the living conditions of humans around the world.
Especially the dramatic loss of insects~\cite{hallmann2017more,wagner2021insect} plays a crucial role in many ecological processes that affect agriculture and others.
Hence, monitoring insect species populations becomes more important nowadays to better understand insect decline and long-term trends in species distributions.
Furthermore, there are about one million named species on our planet~\cite{stork2018many}, making manual counting of individuals unrealistic.
Consequently, automated monitoring of insects is inevitably required to infer abundance estimations across larger regions.
One possible way is to use camera traps to collect images of insects that computer vision algorithms can then process to recognize the depicted species automatically.

In this paper, we focus on nocturnal insects, mainly nocturnal moths (Lepidoptera).
Even for this subset, there exist hundred thousands of different species worldwide and depending on the habitat, species lists can be narrowed down based on the study region.
For example, image datasets containing hundreds of moth species from Ecuador and Costa Rica are publicly available and can directly be used for evaluating fine-grained recognition algorithms~\cite{Rodner15:FRD}.
Here, we are interested in monitoring moth species in Central Europe.
We present datasets of moth images we have collected so far and our analysis of algorithms for insect localization and species classification.

% Figure environment removed

Our work is part of a larger project called AMMOD\footnote{\scriptsize{AMMOD = \textbf{A}utomated \textbf{M}ultisensor Station for \textbf{M}onitoring \textbf{o}f Bio\textbf{d}iversity (\url{https://ammod.de/})}}, which aims at developing self-sustaining multi-sensor stations for monitoring species diversity~\cite{Waegele22:TAM}.
One component of these stations is a light-based camera trap for nocturnal insects, called the \emph{moth~scanner}~\cite{Radig2021:AVL,Korsch21_DLP}.
It is a non-invasive monitoring system for automatically gathering images at nighttime.
A UV-LED lamp illuminates a white planar surface to attract the insects that land on this surface.
A high-resolution camera takes an image of the whole surface every two minutes.
Our prototype is shown in Figure~\ref{fig:prototype}.

With this setup, we can collect large-scale datasets of nocturnal insects over a long period that can then be used to develop and evaluate appropriate fine-grained species recognition algorithms.
The moth scanner takes several hundred images during one night, and within five months, we collected more than \num{27000} images with our prototype.
In this paper, we refer to the resulting dataset as the \emph{nocturnal insects dataset~(NID)}, and more details are given in Section~\ref{sec:dataset}.
Note that this dataset is supposed to be extended over time as our system will be in operation within the following years.
We plan to maintain multiple sensor stations in parallel at different locations.
Hence, it has the potential to become a valuable source for large-scale learning and continuous learning within a fine-grained domain.

Besides its impact on research in fine-grained recognition, our developments for automated visual monitoring of nocturnal insects are beneficial for ecologists.
Until now, insect monitoring is mainly done by hand and supported by citizen scientists who manually take images of individual insects in their gardens. 
Previously, we published an image dataset of nocturnal moths captured manually by citizen scientists, called \emph{\mbox{EU-Moths}} dataset at a local workshop~\cite{Korsch21_DLP}.
This paper also includes a dataset description and our baseline results for insect localization and species classification.
There are two reasons for this.
First, we want to announce this dataset to a broader audience interested in fine-grained recognition because it can directly be used for algorithm development and evaluation.
Second, we want to highlight the challenges for recognition algorithms that arise when processing automatically captured camera trap images compared to manually taken images with hand-held cameras.

In general, our paper aims to promote the application of moth species identification as a fine-grained visual recognition problem.
We underpin this with existing datasets, results of baseline algorithms, and a light-based camera trap setup that will be used during the following years to automatically collect further large-scale image data.
We believe that research on automated visual identification of hundreds to thousands of different nocturnal moth species can have a major impact on developing fine-grained recognition algorithms in general, and we, therefore, want to share our insights and datasets with the community.

%The rest of the paper is structured as follows. 
%After a short review of related work in Section~\ref{sec:related_work}, we describe the two abovementioned datasets containing images of nocturnal moths in Section~\ref{sec:dataset}.
%The algorithms we applied to both datasets are described in Section~\ref{sec:methods} and we present the achieved results in Section~\ref{sec:results}. 
%We discuss challenges of processing automatically captured images with light-based camera traps in Section~\ref{sec:challenges} that are also important to consider for similar projects, followed by conclusions in Section~\ref{sec:conclusions}.

%\todo{REWRITE from here}

%Before the classification can be performed, we need to perform a detection of the insects.
%At this stage, the application of the state-of-the-art detection models like SSD~\cite{liu2016ssd} or YOLO~\cite{redmon2016you} is an obvious step.
%On the other hand, these models are computationally expensive and other light-weight methods like the MCC blob detector~\cite{bjerge2021automated} are more suitable for the application in the field.
%Unfortunately, to evaluate and compare different detection methods suitable benchmark datasets are missing.

%In this paper, we present a new dataset collected with the help of our prototype.
%In the period of five months, we captured over \num{27000} images in suburban area in Middle Germany.
%For bootstrapping and first evaluations we annotated a subset of these images with bounding boxes for the captured insects.
%The image data and the annotations will be soon publicly available.

%As a first baseline for insect detection task, we evaluated two different methods on the data and present these results further in our paper.
%First, we used a well-established Deep Learning detection model capable of identifying multiple objects in an image, namely the single-shot MultiBox detector (SSD)~\cite{liu2016ssd}.
%As a light-weight alternative that can be easily deployed directly at the computationally limited hardware of the moth scanner, we developed and evaluated a multi-step blob detection algorithm.
% \todo{Edge Computing as buzzword? EdgeAI may be wrong here?}
%First, it reduces the power consumption due to the reduction of computations.
%Further, applying the detection directly at the moth scanner, we can drastically reduce the amount of data that needs to be transmitted  when the system will gather data autonomously in the field.
%The algorithm is closely related to the blob detection method proposed by Bjerge~\etal\cite{bjerge2021automated} but mitigates some of the method's limits.
%We present the idea and the improvement in more detail in Sect.~\ref{sec:methods}.
