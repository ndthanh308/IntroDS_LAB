\subsection{Preliminary Knowledge}
\label{sec:background}

\subsubsection{Poisson Surface Reconstruction}
\label{sec:dpsr}
 Poisson surface reconstruction (PSR) \cite{kazhdan2006poisson} aims to recover an indicator function $\chi \in \mathbb{R}^n$ from K sampled points $\mathcal{P}=\left\{p_i \in \mathbb{R}^3\right\}_{i=1}^K$ with normals $\mathcal{N}=\left\{n_i \in \mathbb{R}^3\right\}_{i=1}^K$, by satisfying that $\chi$ changes sharply between positive and negative values at the surface boundary along the direction orthogonal to the surface. We consider the case where $n := r \times r \times r$, and $d=3$,  where $r$ is the resolution of the indicator grid. In practice, PSR first constructs a point normal field $\mathbf{q} \in \mathbb{R}^{n \times d}$ from $\mathcal{P}$ and $\mathcal{N}$.  Then, it formulates the reconstruction of $\chi$ as a Poisson equation: $\nabla^2 \chi:=\nabla \cdot \nabla \chi=\nabla \cdot \mathbf{q}$,
 % \begin{equation}
    % \nabla^2 \chi:=\nabla \cdot \nabla \chi=\nabla \cdot \mathbf{v}
%     \label{eq:psr}
% \end{equation}
 where the Laplacian of $\chi$ is equal to the divergence of the normal vector field $\mathbf{v}$, subject to the boundary condition that $\chi$ is zero at infinity. This is equivalent to minimizing a quadratic energy function such that $\min _\chi\|\nabla \chi-\mathbf{q}\|_2^2$. 

 Unlike \cite{kazhdan2006poisson}, which encodes the indicator function $\chi$ as a linear combination of sparse basis functions and solves the partial differential equation (PDE) using a finite element solver on an octree, Differentiable Poisson Surface Reconstruction (DPSR) \cite{Peng2021SAP} represents $\chi$ on a 3D grid in a discrete Fourier basis and employs a spectral solver \cite{canuto2007spectral}. The unnormalized indicator function $\chi^{\prime}$ is given by
 \begin{equation}
     \tilde{\chi}=\tilde{g}_{\sigma, r}(\mathbf{u}) \odot \frac{i \mathbf{u} \cdot \tilde{\mathbf{q}}}{-2 \pi\|\mathbf{u}\|^2},\quad \chi^{\prime}=\operatorname{IFFT}(\tilde{\chi}) 
    \label{eq:dpsr}
\end{equation}
 % This spectral approach offers the benefits of fast computation and differentiability, albeit with a tradeoff of cubic memory consumption proportional to the grid size. 
where spectral domain signal is denoted as tilde symbol, i.e., $\tilde \chi = FFT(\chi)$, $\mathbf{u} \in \mathbb{R}^{n \times d}$ denotes the spectral frequencies, $\tilde{\mathbf{q}}$ represents the fast Fourier transform (FFT) of $\mathbf{q}$, IFFT($\tilde \chi$) represents the inverse FFT of $\tilde \chi$, and $\tilde{g}_{\sigma, r}(\mathbf{u})$ is a Gaussian smoothing kernel of bandwidth $\sigma$ at grid resolution $r$ in the spectral domain. We denote the element-wise product as $\odot : \mathbb{R}^n \times \mathbb{R}^n \mapsto \mathbb{R}^n$, and the dot product $(\cdot): \mathbb{R}^{n \times d} \times \mathbb{R}^{n \times d} \mapsto \mathbb{R}^n$, and the L2-norm as $\|\cdot\|^2 : \mathbb{R}^{n \times d} \mapsto \mathbb{R}^n$ .
%%
Finally, the normalized indicator function is obtained by subtracting the mean of the unnormalized indicator function at $\mathcal{P}_{up}$ and re-scaling it, written as 
 \begin{equation}
    \chi=\frac{m}{\operatorname{abs}\left(\left.\chi^{\prime}\right|_{\mathbf{x}=0}\right)} 
    \left(
    \chi^{\prime}-\frac{1}{|\{\mathcal{P}\}|} 
    \sum_{\mathbf{c} \in\{\mathcal{P}\}} 
    {\chi^{\prime}|_{\mathbf{x}=\mathbf{c}}}
    \right).
\label{eq:dpsr_nor}
\end{equation}
%%
% This spectral approach offers the benefits of fast computation and differentiability, albeit with a tradeoff of cubic memory consumption proportional to $r$. 
%%

\subsubsection{Diffeomorphic Flow}
\label{sec:cdf}

Diffeomorphic flows can establish dense point correspondences between source and target surfaces, and preserve the desired geometric topology. In our methods, it is used for mesh registration in topology correction pipeline. 
%%
Let $\Phi(\boldsymbol{p}, t): \Omega\subset\mathbb{R}^{3} \times [0, 1] \mapsto \Omega\subset\mathbb{R}^3$ define a continuous, invertible trajectory from the initial position $\boldsymbol{p}=\Phi(\boldsymbol{p}, 0)$ to the final position $\boldsymbol{p}^{\prime}=\Phi(\boldsymbol{p}, 1)$, satisfying such ordinary differential equation (ODE) and the initial condition:
\begin{equation}
    \frac{\partial \Phi(\boldsymbol{p}, t)}{\partial t}=\boldsymbol{v}(\Phi(\boldsymbol{p}, t), t) \quad \text { s.t. } \quad \Phi(\boldsymbol{p}, 0)=\boldsymbol{p},
    \label{eq:forward_ode}
\end{equation}
where $\boldsymbol{v}(\boldsymbol{p}, t): \Omega \times [0, 1] \mapsto \Omega$ indicates the velocity vector of coordinate $\boldsymbol{p}$ at time t. If $\boldsymbol{v}$ is Lipschitz continuous, a solution to Eq.~\ref{eq:forward_ode} exists and is unique in the interval $[0, 1]$, which ensures that any two deformation trajectories do not cross each other \cite{Coddington1984-jh}. 
%%
% In this work, we assume that ${\boldsymbol{v}}$ is stationary and can be modeled via a neural field \cite{xie2022neural, sun2022mirnf}, named as $\mathcal{F}_{\theta}(\boldsymbol{p}) = [\boldsymbol{v}_{p_x}, \boldsymbol{v}_{p_y}, \boldsymbol{v}_{p_z}]^T$, where $\theta$ represents the network parameters. 
% %%
% The initial value problem (IVP) in Eq.~\ref{eq:forward_ode} can be solved with a Differentiable ODE Solver (NODE) \cite{chen2018neural} whose dynamic function is set to be $\mathcal{F}_{\theta}$. In the forward pass, the destination position $\boldsymbol{p}^{\prime}$ starting from $\boldsymbol{p}$ can be estimated by integrating $\mathcal{F}_{\theta}(\boldsymbol{p})$ from $t=0$ to $t=1$ via NODE, such that $\boldsymbol{p}^{\prime} = \Phi(\boldsymbol{p}, 1) = \Phi(\boldsymbol{p}, 0) + \int_{0}^{1} \mathcal{F}_{\theta}({\Phi(\boldsymbol{p}, t)}) \mathrm{d}t$.
% %%
% For backpropagation, NODE adopts the adjoint sensitivity method \cite{pontryagin1987mathematical}, which retrieves the gradient by solving the adjoint ODE backwards in time and allows solving with O(1) memory usage no matter how many steps the ODE solver takes.