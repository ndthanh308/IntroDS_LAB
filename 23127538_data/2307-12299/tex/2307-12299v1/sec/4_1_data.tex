\subsection{Datasets}
\label{exp:data}
To evaluate the performance of our method on the reconstruction of cortical surfaces from MRI images, we use three publicly available datasets: the Alzheimer's Disease Neuroimaging Initiative (ADNI) dataset \cite{jack2008alzheimer}, the OASIS-1 dataset \cite{marcus2007open}, and the test-retest (TRT) dataset \cite{maclaren2014reliability}. We obtain the pseudo-ground truth surfaces generated from Freesurfer v5.3 \cite{fischl2012freesurfer} for all three datasets. We strictly follow pre-processing pipeline from \cite{bongratz2022vox2cortex}. Specifically, we first register the MRIs to the MNI152 scan. After padding the input images to have shape $192 \times 208 \times 192$, we resize them to $128 \times 144 \times 128$. The intensity values are min-max-normalized to the range $[0,1]$.

\vspace{-0.8em}
\paragraph{ADNI} We use a subset of the ADNI dataset \cite{jack2008alzheimer} containing a total of $419$ T1-weighted (T1w) brain MRI from subjects aged from $55$ to $90$ years old. We stratify the dataset into $299$ scans for training ($\approx 70\%$), $40$ scans for validation($\approx 10\%$), and $80$ scans for testing ($\approx 20\%$). We report all of our experiment results on the test set.

\vspace{-0.8em}
\paragraph{OASIS} For the OASIS dataset \cite{marcus2007open}, we use all of $416$ T1-weighted (T1w) brain MRI images. We stratify the dataset into $292$ scans for training ($\approx 70\%$), $44$ scans for validation ($\approx 10\%$), and $80$ scans for testing ($\approx 20\%$). We report all of our experiment results on the test set.

\vspace{-0.8em}
\paragraph{Test-retest} To analyze the consistency of our approach, we evaluate all $120$ scans from three different subjects, where each subject is scanned twice in $20$ days.
