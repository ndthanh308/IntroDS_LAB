\section{Related Work}
Traditional cortical surface reconstruction was accomplished through a sequence of image-processing steps, with FreeSurfer \cite{fischl2012freesurfer}, being a widely used approach. While accurate, these methods are constrained by time limitations, with each case taking up to 7-8 hours to complete, thus hindering practical application in clinical settings. Deep learning methods, therefore, are proposed to address the time limitation of traditional approaches and show potential improvement for cortical surface reconstruction task.

\vspace{-0.8em}
\paragraph{Implicit representation} 
FastSurfer\cite{henschel2020fastsurfer} utilizes eigenfunctions of the Laplace-Beltrami operator to parametrize the surface and generate the final spherical map by scaling the 3D spectral embedding vector to unit length.
Furthermore, recent research in 3D computer vision has focused on deep implicit representations \cite{mescheder2019occupancy, sitzmann2020implicit, shipp2007structure, park2019deepsdf}, which have shown great potential in improving the accuracy and efficiency of surface reconstruction. For instance, SegRecon \cite{gopinath2021segrecon} employs a 3D CNN for simultaneously learning segmentation and surface reconstruction by utilizing 3D signed distance function. Additionally, DeepCSR\cite{cruz2021deepcsr} and CortexODE \cite{ma2022cortexode} leverage deep implicit functions to represent the surface, which is then extracted using Marching Cubes \cite{lorensen1987marching, lewiner2003efficient} to produce the output mesh.

\vspace{-0.8em}
\paragraph{Explicit representation} 
Explicit representation methods provide an alternative approach to cortical surface reconstruction, which often learns deformation networks to directly transform a source mesh into a target mesh. For example, CorticalFlow \cite{lebrat2022corticalflow} implements a flow Ordinary Differential Equation (ODE) framework to learn to deform a reference template towards a targeted object. Other methods, such as Voxel2Mesh \cite{wickramasinghe2020voxel2mesh} and Vox2Cortex \cite{bongratz2022vox2cortex}, learn deformable mesh models that take as input a template mesh or sphere initialization and iteratively deform the mesh by learning deformation field of the vertices. 

\vspace{-0.8em}
\paragraph{Diffeomorphic Transformation}
A diffeomorphism is an invertible mapping where the forward and backward transformations are smooth. It is widely used in medical registration problems. Many works \cite{dalca2018unsupervised, balakrishnan2019voxelmorph, krebs2019learning, dalca2019unsupervised, avants2008symmetric} usually assume the velocity field is stationary and defined in the grid space \cite{ashburner2007fast} so that they can apply scaling and squaring method \cite{arsigny2006log} to do fast integration. CorticalFlow and CorticalFlow++ \cite{santa2022corticalflow++} learn a discrete stationary velocity field and integrate the deformations associated with the template meshes using interpolation and traditional ODE solvers.
Recently, with the power of neural ordinary differential equation solver \cite{chen2018neural, chen2020learning}, optimizing a neural diffeomorphic flow efficiently became possible. NDF \cite{sun2022topology} addresses organ shape representation and registration simultaneously by decomposing the implicit shape representation into continuous diffeomorphic transformations and template shape representation. Neural Mesh Flow\cite{gupta2020neural} focuses on generating manifold mesh from images or point clouds via conditional continuous diffeomorphic flow. 