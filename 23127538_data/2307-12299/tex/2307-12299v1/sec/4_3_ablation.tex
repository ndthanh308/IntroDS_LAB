\subsection{Ablation Study}
\label{sec:ablation}

We conduct two ablation experiments of WM surface reconstruction on OASIS validation set. For fair comparisons, all experiments below are based on the same coarse mesh deformation module pre-trained for 50 iterations, then the oriented point cloud module will be finetuned for another 50 iterations using different settings. The evaluation metrics are ASSD and 90-percentile Hausdorff distance (HD90).

\vspace{-0.8em}
\begin{table}
%\setlength{\tabcolsep}{4.6pt} % Reduce white-space between columns, SHOULD BE SET AFTER TABLE IS FINISHED
\centering
\caption{
\textbf{Ablation Study on Upsample Ratio} $\boldsymbol{S}$. 
% An asterick (*) indicate training with large template meshes ($\approx 168,000$ vertices per surface). Both Vox2Cortex and Vox2Cortex* use large template meshes for inference, while our methods use small template meshes for inference.
}
\resizebox{\columnwidth}{!}{
\begin{tabular}{lcccccc}
\toprule
& & \multicolumn{2}{c}{Left WM} & \multicolumn{2}{c}{Right WM} \\
\cmidrule(lr){3-4}\cmidrule(lr){5-6}
Method  & \#~params   & ASSD (mm)         & HD90 (mm)     & ASSD (mm)      & HD90 (mm)     \\ 
\midrule
    % Vox2Cortex         & .344 $\pm .033$  & .792 $\pm .099$ & .344 $\pm .041$ & .768 $\pm .114$  \\
    % Vox2Cortex*        & .315 $\pm .037$  & .718 $\pm .107$ & .309 $\pm .043$ & .704 $\pm .115$  \\
    % \midrule
    $S=1$              & 6.49M & .357 $\pm .034$  & .881 $\pm .109$ & .352 $\pm .039$ & .867 $\pm .114$  \\
    $S=4$              & 6.54M & .323 $\pm .036$ & .727 $\pm .105$ & .317 $\pm .042$ & .712 $\pm .110$  \\
    $S=7$              & 6.59M & .314 $\pm .036$ & .692 $\pm .107$ & .309 $\pm .044$ & .678 $\pm .116$ \\
    $S=10$             & 6.64M & \textbf{.312} $\pm .037$ & \textbf{.688} $\pm .107$ & \textbf{.308} $\pm .043$ & \textbf{.669} $\pm .113$ \\
\bottomrule
\vspace{-0.8em}
\end{tabular}
}

\label{tab:upsample}
\end{table}
\begin{table}
%\setlength{\tabcolsep}{4.6pt} % Reduce white-space between columns, SHOULD BE SET AFTER TABLE IS FINISHED
\centering
\caption{
\textbf{Ablation Study on Network Type} 
}
\resizebox{\columnwidth}{!}{
\begin{tabular}{ccccc}
\toprule
Network & \multicolumn{2}{c}{Left WM} & \multicolumn{2}{c}{Right WM} \\
\cmidrule(lr){2-3}\cmidrule(lr){4-5}
Type     & ASSD          & HD90      & ASSD      & HD90      \\ 
\midrule
    GCN        & .314 $\pm .036$ & .692 $\pm .107$ & .309 $\pm .044$ & .678 $\pm .116$ \\
    % Linear     & .309 $\pm .038$ & .674 $\pm .110$ & .304 $\pm .044$ & .662 $\pm .118$  \\
    % \midrule
    GLU        & \textbf{.304} $\pm .036$ & \textbf{.661} $\pm .108$ & \textbf{.298} $\pm .044$ & \textbf{.649} $\pm .114$ \\
\bottomrule
\end{tabular}
}

\label{tab:glu}
\end{table}
\paragraph{Hybrid Representation and Upsample Ratio}  In this study, we employed GCN to estimate point positions and normals for Hybrid-CSR with varying upsample ratios denoted by $S$. As shown in Table~\ref{tab:upsample}, increasing the upsample ratio results in more accurate reconstruction of the white matter surface, while not significantly increasing the number of parameters that need optimization. We have chosen the value of $S=7$ in our experiments to strike a balance between training efficiency and accuracy of the reconstructed surfaces.

\vspace{-0.8em}
\paragraph{GLU Estimation} In this study, we apply two different networks, i.e., GCN and GLU, to estimate the oriented point clouds. As can be observed from Tab~\ref{tab:glu}, GLU can generate significantly more accurate reconstruction results on WM surfaces. 
GCN tends to generate over-smoothing predictions, while GLU can produce more acute deformations. Since PSR is robust to outlier points, GLU might be an advantageous choice. More discussions on GLU-based point cloud estimation are included in the supplementary.
