\section{Introduction}

\input{fig/toy_example}

Cortical surface reconstruction is the task to extract both inner and outer surfaces of the cerebral cortex from brain MRI scans, with the inner surface situated between the cortical gray matter (GM) and white matter (WM) while the outer surface between the cerebrospinal fluid and the cortical gray matter.
%Cortical surface reconstruction aims to extract the inner and outer surfaces of the cerebral cortex from brain MRI scans, allowing for visualization and analysis of the 3D structure of cortex. The inner cortical surface, situated between the cortical gray matter (GM) and white matter (WM), and the outer surface, situated between the cerebrospinal fluid and the cortical gray matter, are both extracted. 
Accurate and detailed reconstruction of the cortical surface can be used to facilitate brain mapping \cite{destrieux2010automatic, glasser2016multi, toga2003mapping, fischl1999high}, identify biomarkers for neurological disorders\cite{han2006reliability, desikan2010automated, apostolova2010surface}, and enable pre-surgical planning \cite{de2011brain}. However, extracting accurate cortical surface with genus of 0 is still very challenging due to structural complexity of the cerebral cortex and partial volume effect (PVE) \cite{ballester2002estimation} in medical imaging. 

Traditional methods~\cite{dahnke2013cortical, fischl2012freesurfer, fischl1999cortical, kim2005automated} typically first segment the volumetric structures and then extract the cortical meshes via Marching Cubes~\cite{lorensen1987marching,lewiner2003efficient} or level set-based approaches~\cite{osher2004level,li2010distance}.
% One approach to reconstruct cortical surface from MRI involves segmenting volumetric structures and then extracting cortical meshes via Marching Cubes \cite{lorensen1987marching,lewiner2003efficient} or level set-based methods \cite{osher2004level,li2010distance}, which is commonly used in traditional methods such as \cite{dahnke2013cortical, fischl2012freesurfer, fischl1999cortical, kim2005automated}. 
Specifically, FreeSurfer \cite{fischl2012freesurfer}, the current standard for cortical surface reconstruction, obtains WM surfaces by applying mesh tessellation to segmented WM. 
With the help of convolutional neural network and a novel spherical embedding, FastSurfer \cite{henschel2020fastsurfer} achieves faster and better brain segmentation. While deep learning-based volumetric segmentation models have demonstrated remarkable performance in medical imaging \cite{ronneberger2015u,isensee2018nnu,hatamizadeh2022unetr,cao2023swin,zhou2021nnformer}, the reconstructed cortical meshes may not accurately delineate tissue boundaries due to PVE problem. Furthermore, the predicted brain segmentation may contain topological defects, necessitating time-consuming topology correction algorithms to be applied for genus-0 reconstructed meshes.


One way to overcome the partial volume effect is to utilize deep implicit functions, which can represent complex shapes with fine-grained details by representing the surface implicitly as the zero level-set of continuous functions~\cite{park2019deepsdf, saito2020pifuhd, mescheder2019occupancy, sitzmann2020implicit, tancik2020fourier}. 
Recently, DeepCSR has been proposed to reconstruct the cortical surface leveraging the deep implicit functions in \cite{cruz2021deepcsr}.
As deep implicit functions can be trained efficiently using simple L1 or L2 loss functions, the optimization process is straightforward and robust in terms of convergence.
However, DeepCSR still requires a time-consuming spherical level-set evolution algorithm \cite{pham2010digital} to remove holes and handles from the reconstructed meshes, due to the absence of topology awareness.

% To overcome the partial volume effect, DeepCSR \cite{cruz2021deepcsr} proposes to leverage deep implicit functions for cortical surface reconstruction. By representing the surface implicitly as the zero level-set of continuous functions, deep implicit function-based methods \cite{park2019deepsdf, saito2020pifuhd, mescheder2019occupancy, sitzmann2020implicit, tancik2020fourier} can represent complex shapes with fine-grained details. 
% Moreover, deep implicit functions can be trained efficiently using regression-based loss functions like L1 or L2 loss, making optimization more straightforward and less prone to convergence issues. Thus, deep implicit functions seem to be a promising approach to address the challenges of accurately reconstructing the highly-folded structure of the cerebral cortex. Despite these advantages, DeepCSR still requires a time-consuming spherical level-set evolution algorithm \cite{pham2010digital} to remove holes and handles from the reconstructed meshes owing to the absence of topology awareness.

Alternatively, there is another branch of work that exploit explicit topology to resolve topology defects and partial volume effects simultaneously~\cite{bongratz2022vox2cortex, lebrat2021corticalflow, santa2022corticalflow++, hoopes2021topofit}.
By progressively deforming explicit meshes from template meshes, explicit cortical reconstruction methods can guarantee reconstructed meshes inherit the desired topology, thus avoiding time-consuming post-processing.
At the same time,  sub-voxel variations of cerebral cortex surfaces can also be captured.
Nonetheless, these methods suffer from two main problems.
Firstly, the often used Chamfer distance loss to align predicted and target meshes tends to get trapped in local minima easily, failing to distinguish bad samples from the true one~\cite{achlioptas2018learning, nguyen2021point}.
Although a weighted Chamfer distance has been proposed to prioritize fitting local regions with high curvatures in Vox2Cortex~\cite{bongratz2022vox2cortex}, the issue is only alleviated but not resolved completely, still resulting in suboptimal assignments between two sets of points \cite{pomerleau2015review, kolouri2018sliced}.
Secondly, while explicit regularizations can reduce self-intersections, they also lead to lower geometric accuracy. 
Neural Mesh Flow~\cite{gupta2020neural} tries to avoid explicit regularization by deforming template meshes through continuous diffeomorphic flow, however, 
it does not perform well in modeling sharp and large deformations. 
% Recently, explicit cortical reconstruction methods \cite{bongratz2022vox2cortex, lebrat2021corticalflow, santa2022corticalflow++, hoopes2021topofit} have been proposed to resolve topology defects and partial volume effects simultaneously. By progressively deforming explicit meshes from template meshes, reconstructed meshes can inherit the desired topology, avoiding post-processing. Learning continuous deformation with respect to template meshes can capture sub-voxel variations of cerebral cortex surfaces. But explicit surface reconstruction encounters two major challenges. First, chamfer distances, which are often used to align predicted meshes and targets, tend to be trapped in local minima, failing to distinguish bad samples from the true one \cite{achlioptas2018learning, nguyen2021point}. To address this, Vox2Cortex \cite{bongratz2022vox2cortex} proposes a weighted chamfer distance that prioritizes fitting local regions with high curvatures. However, it does not solve the inherent issue of chamfer distance, namely suboptimal assignments between two sets of points \cite{pomerleau2015review, kolouri2018sliced}. Second, explicit regularizations to reduce self-intersection can lead to lower geometric accuracy. To this end, Neural Mesh Flow (NMF) \cite{gupta2020neural} proposes to deform template meshes using continuous diffeomorphic flow, without explicit regularization, but they are not good at modeling sharp and large deformations.

In this work, we propose a novel approach for cortical surface reconstruction called Hybrid-CSR, which integrates both implicit and explicit shape representations. 
%%
The method involves an initial step to deform template meshes into a coarsely reconstructed cortical surface, based on which oriented point clouds are estimated for the subsequent implicit cortical surface reconstruction.
%%
Specifically, we apply differentiable Poisson surface reconstruction \cite{Peng2021SAP} to bridge oriented point clouds (explicit) to indicator grids (implicit), from which watertight cortical meshes can be extracted via Marching Cubes. 
%%
In addition, we propose to apply optimization-based diffeomorphic surface registration to realize topology correction.
%%
Our Hybrid-CSR offers several advantages over existing techniques. Unlike voxel-based methods, our approach is not susceptible to partial volume effects. Compared to deep implicit function-based method, Hybrid-CSR produces fewer topology defects and demonstrates superior efficiency in inference. Finally, compared to mesh-based methods, our hybrid approach offers a higher level of expressiveness and flexibility.
%but comes at the expense of potentially compromising the desired topology.

The efficacy of our proposed method can be demonstrated by using a simple contour representation toy example, as shown in Fig.~\ref{fig:toy_example}.
%As part of our study, we present a contour representation toy example in order to demonstrate the efficacy of our proposed method. 
The target contour (Fig.~\ref{fig:toy_example}(a)) is a polygon that can be either reconstructed through the deformation of a source contour (circle) or the extraction of the zero level set from an indicator field. 
%%
As can be seen in Fig.~\ref{fig:toy_example}(d), the results obtained from explicit contour deformations optimization are not optimal due to the ``regularizer's dilemma" \cite{gupta2020neural}. 
%%
Instead, our proposed hybrid method optimizes the positions and normals of the oriented points (initialized by Fig.~\ref{fig:toy_example}(e)) and minimizes the difference between the indicator map reconstructed from the ground truth and the optimized oriented point clouds (Fig.~\ref{fig:toy_example}(c)(i)). Hybrid shape representation can produce a well-captured contour from the predicted indicator field, as demonstrated in Fig.~\ref{fig:toy_example}(g). More details about this toy example can be found in the supplementary.
% To model mesh deformation, we encoder poistions with random Fourier mapping \cite{tancik2020fourier} and approximate the deformation field with a SIREN \cite{sitzmann2020implicit} model. 
% We model mesh deformations via neural fields \cite{xie2022neural}. 
% During each iteration, the chamfer distance between two sets of 1000 sampled oriented points are minimized while also edge length and Laplacian regularizations \cite{wang2018pixel2mesh} are carefully imposed to prevent self-intersections. 
% Nevertheless, the results obtained from explicit contour deformations are not optimal due to the "regularizer's dilemma" \cite{gupta2020neural} as demonstrated in Fig.~\ref{fig:toy_example}(d). 
% Instead, Our proposed hybrid method optimize the positions and normals of the oriented points using the results from Fig.~\ref{fig:toy_example}(e) as initializations. To do so, we minimize the L2 loss between the indicator map reconstructed from the ground truth and the optimized oriented point clouds. Rapidly, hybrid shape representation can produce a well-captured contour with the predicted indicator field, as demonstrated in Figure \ref{fig:toy_example}(g) and (i). 
%In addition, this approach demonstrates robustness against outlier points, as illustrated in Figure \ref{fig:toy_example}(h).

In summary, our main contributions are:

\begin{itemize}
    \item We propose Hybrid-CSR, the first cortical surface reconstruction framework coupling the explicit and implicit shape representation based on differentiable Poisson surface reconstruction.
    
    \item We propose a new topology correction pipeline based on optimization-based diffeomorphic surface registration.
    
    \item We demonstrate on multiple brain datasets that Hybrid-CSR surpasses implicit and explicit reconstruction methods in terms of accuracy, regularity as well as consistency.
\end{itemize}


