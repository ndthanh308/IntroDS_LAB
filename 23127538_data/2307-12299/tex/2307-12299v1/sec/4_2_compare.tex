\subsection{Performance Comparison}
\label{sec:exp_compare}

\subsubsection{Competing Methods}

For all competing methods, we train their models with their official implementations using their suggested experimental settings and pick the best checkpoints on the validation set for comparisons.

\vspace{-0.8em}
\paragraph{DeepCSR \cite{cruz2021deepcsr}} is an implicit surface-based that directly predicts implicit surface representations for the coordinates in the MRI images. The surface can be reconstructed using either occupancy field or signed distance function. We reproduce the method in both ways and observe that signed distance function yields better performance. 

\vspace{-0.8em}
\paragraph{Vox2Cortex \cite{bongratz2022vox2cortex}} is a deformation-based model proposed to retrieve cortical surfaces by deforming a generic template and employing a joint graph neural network and U-Net. The results shown below are generated based on the higher-resolution templates with $\approx 168,000$ vertices for each structure. 
Our framework is built upon Vox2Cortex \cite{bongratz2022vox2cortex}, and therefore it serves as our baseline model.

\vspace{-0.8em}
\paragraph{CorticalFlow++ \cite{lebrat2022corticalflow}} is a diffeomorphic-based method that successively deforms template meshes by integrating a discrete stationary velocity grid using traditional ODE solvers. 
% They have provided model checkpoints trained on a large ADNI dataset (over $3,876$ scans) and we will present the performance of their pre-trained models in the supplementary as well.

\vspace{-0.8em}
\paragraph{CortexODE \cite{ma2022cortexode}} is a multi-stage deformation approach that obtains the white matter initial surface from a volumetric segmentation model, then leverages neural ordinary differential equations to deform an initial WM surface into refined WM and pial surfaces by learning a diffeomorphic flow.

\subsubsection{Evaluation Metrics}
We evaluate our method as well as other approaches using three commonly used metrics including average symmetric surface distance (ASSD), normal consistency (NC), and self-intersection faces ratio (SI). To calculate ASSD and NC, we sample $100$K points uniformly from both predicted and target meshes. For measuring regularity, we determine SI faces using PyMesh \cite{zhou2019pymesh} library. 
\subsubsection{Results Discussion}

\vspace{-0.8em}
\paragraph{Accuracy}
\begin{table*}
\centering
\caption{
% 
\textbf{Cortical Surface Reconstruction Performance Comparison} in terms of average symmetric surface distance (ASSD), normal consistency (NC), and self-intersection face ratio (SI) on ADNI and OASIS datasets. Best values are highlighted. ASSD results are in mm. All results are listed in the format ``mean value $\pm$ standard deviation". \textbf{Hybrid-CSR} represents the cortical reconstruction results $\hat{M}$ without any topology correction operation. \textbf{\qq{+TC}} and \textbf{\qq{+TC+Refine}} indicate the performance of reconstructed cortical surfaces $\hat{M}_{tc}$ and $\hat{M}_{f}$ separately. While $\downarrow$ means smaller metric value is better, $\uparrow$ indicates larger metric value is better.
% 
} % \caption

(a) ANDI dataset

\vspace{0.3em}
\resizebox{\linewidth}{!}{%
\begin{tabular}{lcccccccccccc} 
\toprule
               & \multicolumn{3}{c}{Left Pial} 
               & \multicolumn{3}{c}{Left WM} 
               & \multicolumn{3}{c}{Right Pial} 
               & \multicolumn{3}{c}{Right WM}  \\ 
               
\cmidrule(lr){2-4}\cmidrule(lr){5-7}\cmidrule(lr){8-10}\cmidrule(lr){11-13}
          Method & ASSD (mm) $\downarrow$ & NC $\uparrow$    & SI (\%) $\downarrow$    & ASSD (mm) $\downarrow$ & NC $\uparrow$    & SI (\%) $\downarrow$ & ASSD (mm) $\downarrow$ & NC $\uparrow$   & SI (\%) $\downarrow$ & ASSD (mm) $\downarrow$ & NC $\uparrow$  & SI (\%) $\downarrow$             \\ 
\midrule
DeepCSR~\cite{cruz2021deepcsr}              & .368 $\pm .082$   & .908 $\pm .015$  & \textbf{0}   & .390 $\pm .162$  & .934 $\pm .016$  & \textbf{0}    & .394 $\pm .083$  & .914 $\pm .012$  & \textbf{0}    & .388 $\pm .172$  & .936 $\pm .014$ & \textbf{0}  \\
Vox2Cortex~\cite{bongratz2022vox2cortex}    & .339 $\pm .055$   & .918 $\pm .010$  & .741 $\pm .221$  & .346 $\pm .073$  & .926 $\pm .011$  & .719 $\pm .214$    & .350 $\pm .037$  & .915 $\pm .009$  & 1.025 $\pm .237$ & .335 $\pm .061$  & .927 $\pm .010$ & .745 $\pm .199$         \\
CorticalFlow++~\cite{lebrat2021corticalflow}& .296 $\pm .079$   & .925 $\pm .011$  & .164 $\pm .093$  & .271 $\pm .071$  & .936 $\pm .009$  & .058 $\pm .032$    & .270 $\pm .044$  & .924 $\pm .010$  & .187 $\pm .104$    & .268 $\pm .073$  & .933 $\pm .009$ & .067 $\pm .032$            \\
CortexODE~\cite{ma2022cortexode}            & .258 $\pm .073$   & .929 $\pm .010$  & .112 $\pm .072$  & \textbf{.234} $\pm .064$  & .938 $\pm .010$  & .013 $\pm .011$   & .214 $\pm .035$  & .927 $\pm .009$  & .173 $\pm .091$  & \textbf{.231} $\pm .052$  & .939 $\pm .009$  & .004 $\pm .005$          \\
\cmidrule(lr){1-13}
Hybrid-CSR (ours)                           & .254 $\pm .054$   & .886 $\pm .012$  & \textbf{0}    & .264 $\pm .055$  & .899 $\pm .010$  &  \textbf{0}    & .250 $\pm .041$  & .886 $\pm .012$  &  \textbf{0}    & .257 $\pm .045$  & .901 $\pm .010$ & \textbf{0}              \\
+ TC                                        & .267 $\pm .056$   & .926 $\pm .010$  & \textbf{0}  & .268 $\pm .056$  & \textbf{.939} $\pm .010$  & \textbf{0}    & .262 $\pm .042$  & .926 $\pm .010$  & \textbf{0}     & .260 $\pm .046$  & .940 $\pm .010$ & \textbf{0}           \\
+ TC + Refine                               & \textbf{.203} $\pm .049$   & \textbf{.935} $\pm .009$  & .090 $\pm .075$  & .244 $\pm .056$  & \textbf{.939} $\pm .010$  & .042 $\pm .025$   & \textbf{.200} $\pm .034$  & \textbf{.935} $\pm .008$ & .080 $\pm .071$    & .240 $\pm .043$  & \textbf{.941} $\pm .010$ & .011 $\pm .010$    \\ 
\bottomrule
% \vspace{-0.8em}
\end{tabular}
}

\vspace{1em}

(b) OASIS dataset

\vspace{0.3em}

\resizebox{\linewidth}{!}{%
\begin{tabular}{lcccccccccccc} 
\toprule
               & \multicolumn{3}{c}{Left Pial} 
               & \multicolumn{3}{c}{Left WM} 
               & \multicolumn{3}{c}{Right Pial} 
               & \multicolumn{3}{c}{Right WM}  \\ 
               
\cmidrule(lr){2-4}\cmidrule(lr){5-7}\cmidrule(lr){8-10}\cmidrule(lr){11-13}
          Method & ASSD (mm) $\downarrow$ & NC $\uparrow$    & SI (\%) $\downarrow$    & ASSD (mm) $\downarrow$ & NC $\uparrow$    & SI (\%) $\downarrow$ & ASSD (mm) $\downarrow$ & NC $\uparrow$   & SI (\%) $\downarrow$ & ASSD (mm) $\downarrow$ & NC $\uparrow$  & SI (\%) $\downarrow$             \\ 
\midrule
DeepCSR                     & .424 $\pm .075$   & .898 $\pm .016$   & \textbf{0} & .312 $\pm .124$  & .941 $\pm .010$ & \textbf{0} & .444 $\pm .087$  & .895 $\pm .018$ & \textbf{0} & .344 $\pm .158$  & .941 $\pm .011$  & \textbf{0}          \\
Vox2Cortex                  & .401 $\pm .040$   & .900 $\pm .012$   & 1.110 $\pm .270$ & .302 $\pm .037$  & .928 $\pm .008$ & .994 $\pm .193$ & .405 $\pm .044$  & .898 $\pm .012$ & 1.321 $\pm .252$ & .303 $\pm .042$  & .929 $\pm .009$ & 1.022 $\pm .186$           \\
CorticalFlow++ ~~~~~~~~     & .326 $\pm .058$   & .913 $\pm .011$   & .147 $\pm .100$ & .225 $\pm .038$  & .937 $\pm .007$ & .054 $\pm .060$ & .318 $\pm .057$  & .913 $\pm .011$ & .192 $\pm .123$ & .227 $\pm .046$  & .935 $\pm .008$ & .076 $\pm .068$           \\
CortexODE                   & .279 $\pm .052$   & .919 $\pm .009$   & .277 $\pm .096$ & \textbf{.183} $\pm .036$  & \textbf{.943} $\pm .007$   & .032 $\pm .025$ & .280 $\pm .052$   & .918 $\pm .010$    & .151 $\pm .060$ & \textbf{.182} $\pm .052$  & \textbf{.943} $\pm .008$ & .022 $\pm .020$           \\
\cmidrule(lr){1-13}
Hybrid-CSR (ours)           & .298 $\pm .045$   & .879 $\pm .013$   & \textbf{0} & .220 $\pm .039$  & .905 $\pm .008$ & \textbf{0} & .301 $\pm .049$  & .880 $\pm .013$    & \textbf{0} & .218 $\pm .047$  & .907 $\pm .009$ & \textbf{0}             \\
+ TC                        & .311 $\pm .045$   & .915 $\pm .011$   & \textbf{0} & .220 $\pm .039$  & .941 $\pm .008$ & \textbf{0} & .313 $\pm .049$  & .915 $\pm .012$    & \textbf{0} & .219 $\pm .046$  & .942 $\pm .009$ & \textbf{0}           \\
+ TC + Refine               & \textbf{.274} $\pm .043$   & \textbf{.921} $\pm .010$ & .034 $\pm .022$ & .198 $\pm .035$  & \textbf{.943} $\pm .008$  & .040 $\pm .022$ & \textbf{.275} $\pm .052$  & \textbf{.920} $\pm .010$  & .029 $\pm .018$ & .199 $\pm .033$  & \textbf{.943} $\pm .008$  & .037 $\pm .026$   \\ 
\bottomrule
\vspace{-0.8em}
\end{tabular}
}

\label{tab:benchmark}
\end{table*}
As shown in Tab.~\ref{tab:benchmark}, our proposed method surpasses other competing methods in terms of surface reconstruction accuracy (ASSD and NC). 
%%
Compared with any implicit-based and explicit-based methods including DeepCSR, Vox2Cortex, and CorticalFlow++, our hybrid approach without refinement can generate significantly more accurate results on all cortical surfaces.
%%
Furthermore, we got comparable performance with CortexODE on both ADNI and OASIS datasets. While CortexODE performs slightly better than us on in terms of ASSD on WM surfaces, we excel them in terms of ASSD on pial surfaces and NC on all surfaces. We observe that WM surfaces in CortexODE have a higher performance than our Hybrid-CSR because it is based on volumetric segmentation models which are typically suitable for extracting structures with details but few ambiguities in topology, i.e., white matter. Our method, on the other hand, performs better in more diverse shapes. 

\vspace{-0.8em}
\paragraph{Regularity}
% \begin{table}
%\setlength{\tabcolsep}{4.6pt} % Reduce white-space between columns, SHOULD BE SET AFTER TABLE IS FINISHED
\centering
\caption{
\textbf{Cortical Surface Reconstruction Regularity Comparison} in terms of self-intersection face ratio (SI).
}
\label{tab:self-intersection}
(a) ANDI dataset

\vspace{0.3em}
\resizebox{\linewidth}{!}{%
\begin{tabular}{lcccc}
    \toprule
        Method & Left Pial & Left WM & Right Pial & Right WM  \\
    \midrule
        DeepCSR & 0   & 0  & 0  & 0 \\
        Vox2Cortex & .741 $\pm .221$   & .719 $\pm .214$  & 1.025 $\pm .237$  & .745 $\pm .199$ \\
        CorticalFlow++ & .164 $\pm .093$   & .058 $\pm .032$  & .187 $\pm .104$  & .067 $\pm .032$ \\
        CortexODE & .112 $\pm .072$   & .013 $\pm .011$  & .173 $\pm .091$  & .004 $\pm .005$ \\
    \midrule
        Hybrid-CSR (ours) & 0   & 0  & 0  & 0 \\
        + TC  & 0   & 0  & 0  & 0 \\
        + TC + Refine & .090 $\pm .075$   & .042 $\pm .025$  & .080 $\pm .071$  & .011 $\pm .010$ \\
    \bottomrule
    \vspace{-0.8em}
\end{tabular}
}

\vspace{0.5em}
(b) OASIS dataset

\vspace{0.3em}
\resizebox{\linewidth}{!}{%
\begin{tabular}{lcccc}
    \toprule
        Method & Left Pial & Left WM & Right Pial & Right WM  \\
    \midrule
        DeepCSR & 0   & 0  & 0  & 0 \\
        Vox2Cortex & 1.110 $\pm .270$   & .994 $\pm .193$  & 1.321 $\pm .252$  & 1.022 $\pm .186$ \\
        CorticalFlow++ & .147 $\pm .100$   & .054 $\pm .060$  & .192 $\pm .123$  & .076 $\pm .068$ \\
        CortexODE & .277 $\pm .096$   & .032 $\pm .025$  & .151 $\pm .060$  & .022 $\pm .020$ \\
    \midrule
        Hybrid-CSR (ours) & 0   & 0  & 0  & 0 \\
        + TC  & 0   & 0  & 0  & 0 \\
        + TC + Refine & .034 $\pm .022$   & .040 $\pm .022$  & .029 $\pm .018$  & .037 $\pm .026$ \\
    \bottomrule
    \vspace{-0.8em}
\end{tabular}
}

% \vspace{0.5em}


\end{table}
To assess the regularity of Hybrid-CSR, we calculate the percentage of self-intersecting faces and present them in Tab.~\ref{tab:benchmark}. 
%%
Thanks to the non-self-intersection property of Marching Cubes, we guarantee to achieve $0$ self-intersection, same as DeepCSR \cite{cruz2021deepcsr}. 
%%
Although a small number of self-intersecting faces are introduced after surface refinement, we still significantly surpass Vox2Cortex, CorticalFlow++, and CortexODE on all surfaces, except for CortexODE \cite{ma2022cortexode} on WM surfaces.

\vspace{-0.8em}
\paragraph{Consistency}
\label{exp:consistency}

\begin{table}
%\setlength{\tabcolsep}{4.6pt} % Reduce white-space between columns, SHOULD BE SET AFTER TABLE IS FINISHED
\centering
\caption{
\textbf{Cortical Surface Reconstruction Consistency Comparison} in terms of ASSD on TRT dataset. Hybrid-CSR here indicates the result going through the whole post processing procedures.
}
\resizebox{\columnwidth}{!}{
\begin{tabular}{lccr}
    \toprule
        Method & ASSD (mm) & $>1$mm & $>2$mm  \\
    \midrule
        Vox2Cortex & .301 $\pm .171$ & 3.12\% & .69\% \\
        CortexODE & .280 $\pm .164$ &  2.21\%   &  .38\%\\
        FreeSurfer & .301 $\pm .176$ & 3.35\%   & .94\% \\
    \midrule
        Hybrid-CSR + TC + Refine & \textbf{.258} $\pm .051$  & \textbf{1.61}\%  & \textbf{.18}\%  \\
    \bottomrule
\vspace{-0.8em}
\end{tabular}
}

\label{tab:consistency}
\end{table}
We evaluate the consistency of Hybrid-CSR (with topology correction and surface refinement), Vox2Cortex \cite{bongratz2022vox2cortex}, and CortexODE \cite{ma2022cortexode} (which are all trained on OASIS), and Freesurfer on TRT dataset. 
%%
We generate cortical surfaces from MRI images of the same subject on the same day and measure the ASSD of the resulting reconstructions. The brain morphology of two consecutive scans taken on the same day should be similar to each other, except for the variations caused by the imaging process. 
%%
The result from Table \ref{tab:consistency} shows that we outperform Vox2Cortex \cite{bongratz2022vox2cortex}, CortexODE \cite{ma2022cortexode}, and Freesurfer on the consistency aspect. 
 

% \subsubsection{Generability}
% \label{exp:consistency}