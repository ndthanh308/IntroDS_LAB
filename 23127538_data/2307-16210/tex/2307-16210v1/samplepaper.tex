% This is samplepaper.tex, a sample chapter demonstrating the
% LLNCS macro package for Springer Computer Science proceedings;
% Version 2.21 of 2022/01/12
%
\documentclass[runningheads]{llncs}
%
\usepackage[T1]{fontenc}
\usepackage{graphicx}
\usepackage{soul}
\usepackage{url}
\usepackage[hidelinks]{hyperref}
\usepackage[utf8]{inputenc}
%\usepackage[small]{caption}
\usepackage{graphicx}
\usepackage{amsmath}
%\usepackage{amsthm}
\usepackage{booktabs}
\usepackage{algorithm}
\usepackage{algorithmic}
\urlstyle{same}
\usepackage{multirow}
\usepackage{multicol}
\usepackage{bm}
\usepackage{underscore}
\usepackage{amsfonts}
\usepackage{xcolor}
\usepackage{color, colortbl}
\usepackage{makecell}
\usepackage{enumitem}
\let\Bbbk\relax
\usepackage{amssymb}
\usepackage{colortbl}

% \usepackage[dvipsnames,svgnames]{xcolor}
%\usepackage[table, svgnames, dvipsnames]{xcolor}
\usepackage[switch]{lineno}
% Comment out this line in the camera-ready submission
%\linenumbers

\definecolor{mygray}{gray}{.9}
\definecolor{gray2}{gray}{.78}
\definecolor{gray3}{gray}{.7}
\definecolor{gray4}{gray}{.6}
\definecolor{gray5}{gray}{.5}
% \newcommand{\CC}[1]{\cellcolor{blue!#1}}
% \newcommand{\CC}[1]{\cellcolor{blue!#1}}
\newcommand{\CC}{\cellcolor{mygray}}
\newcommand{\CK}{\cellcolor{gray2}}
\newcommand{\rmnum}[1]{\romannumeral #1}
\newcommand{\Rmnum}[1]{\expandafter\@slowromancap\romannumeral #1@}
\newcommand{\todo}[1]{{\color{red}[#1]}}
\newcommand{\wen}[1]{{\color{black}#1}}
%
\begin{document}
\title{Rethinking Uncertainly Missing and Ambiguous Visual Modality in Multi-Modal \\ Entity Alignment}


% \author{Anonymous}
% \authorrunning{Anonymous}

\author{
Zhuo Chen\inst{1}
\and
Lingbing Guo\inst{1}
\and
 Yin Fang\inst{1}
\and
 Yichi Zhang\inst{1}
\and
Jiaoyan Chen\inst{4}
\and
Jeff Z. Pan\inst{5}
\and
Yangning Li \inst{6}
\and
Huajun Chen\inst{1,2}
\and
Wen Zhang\inst{3}\thanks{Corresponding author.}
}

\institute{College of Computer Science, Zhejiang University, Hangzhou, China
\and
Donghai laboratory, Zhoushan, China 
\and
School of Software Technology, Zhejiang University, China
\email{\{zhuo.chen,lbguo,fangyin,zhangyichi2022,zhang.wen,huajunsir\}@zju.edu.cn}
\and
The University of Manchester \& University of Oxford, UK \\
\email{jiaoyan.cheni@manchester.ac.uk}
\and
School of Informatics, The University of Edinburgh, Edinburgh, UK \\
\email{https://knowledge-representation.org/j.z.pan/}
\and
Shenzhen International Graduate School, Tsinghua University, Shenzhen, China\\
\email{liyn20@mails.tsinghua.edu.cn}
}
\authorrunning{Z. Chen et al.}
\maketitle              % typeset the header of the contribution
%
% \def\thefootnote{$\dagger$}\footnotetext{Equal Contribution.}

\begin{abstract}
As a crucial extension of entity alignment (EA), multi-modal entity alignment (MMEA) aims to identify identical entities across disparate knowledge graphs (KGs) by exploiting associated visual information. However, existing MMEA approaches primarily concentrate on the fusion paradigm of multi-modal entity features, while neglecting the challenges presented by the pervasive phenomenon of missing and intrinsic ambiguity of visual images.
In this paper, we present a further analysis of visual modality incompleteness, benchmarking latest MMEA models on our proposed dataset {MMEA-UMVM}, where
the types of alignment KGs covering bilingual and monolingual, with standard (non-iterative) and iterative training paradigms to evaluate the model performance. 
Our research indicates that, in the face of modality incompleteness, models succumb to overfitting the modality noise, and exhibit performance oscillations or declines at high rates of missing modality. This proves that the inclusion of additional multi-modal data can sometimes adversely affect EA. 
To address these challenges,  we introduce {UMAEA}, a robust multi-modal \textbf{e}ntity \textbf{a}lignment approach designed to tackle \textbf{u}ncertainly \textbf{m}issing and \textbf{a}mbiguous visual modalities.
It consistently achieves SOTA performance across all 97 benchmark splits, significantly surpassing existing baselines with limited parameters and time consumption, while effectively alleviating the identified limitations of other models. 
Our code and benchmark data are available at {\color{blue} \url{https://github.com/zjukg/UMAEA}}.

\keywords{Entity Alignment  \and Knowledge Graph \and Multi-modal Learning \and Uncertainly Missing Modality.}

\end{abstract}


The meteoric advancement of machine learning and artificial intelligence technologies has enabled the construction of neural networks that effectively emulate the complex computations of the human brain. These deep learning models have found utility in a wide range of applications, such as computer vision, natural language processing, autonomous driving, and more. With the growing complexity and sophistication of these neural network models, the computational requirements, particularly for 32-bit operations, have exponentially increased. This heightened computational demand necessitates the exploration of more efficient alternatives, such as 16-bit operations.

However, the shift to 16-bit operations is riddled with challenges. A common standpoint within the research community argues that 16-bit operations are not ideally suited for neural network computations. This belief is mainly attributable to concerns related to numerical instability during the backpropagation phase, especially when popular optimizers like Adam are employed. This instability, more pronounced during the optimizer-mediated backpropagation process rather than forward propagation, can negatively impact the performance of 16-bit operations and compromise the functioning of the neural network model. Current optimizers predominantly operate on 32-bit precision. If these are deployed in a 16-bit environment without appropriate hyperparameter fine-tuning, the neural network models encounter difficulties during learning. This issue is particularly evident in backward propagation, which heavily relies on the optimizer. Confronted with these challenges, the objective of this research is to conduct an exhaustive investigation into the feasibility and implementation of 16-bit operations for neural network training. We propose and evaluate innovative strategies aimed at reducing the numerical instability encountered during the backpropagation phase under 16-bit environments. A significant focus of this paper is also dedicated to exploring the future possibilities of developing 16-bit based optimizers. One of the fundamental aims of this research is to adapt key optimizers such as Adam to prevent numerical instability, thereby facilitating efficient 16-bit computations. These newly enhanced optimizers are designed to not only address the issue of numerical instability but also leverage the computational advantages offered by 16-bit operations, all without compromising the overall performance of the neural network models. Through this research, our intention goes beyond improving the efficiency of neural network training; we also strive to validate the use of 16-bit operations as a dependable and efficient computational methodology in the domain of deep learning. We anticipate that our research will contribute to a shift in the prevalent perceptions about 16-bit operations and will foster further innovation in the field. Ultimately, we hope our findings will pave the way for a new era in deep learning research characterized by efficient, high-performance neural network models.

The matter of numerical precision in deep learning model training has garnered substantial attention in recent years. A milestone study by Gupta et al. \cite{Gupta2015} was among the first to explore the potential of lower numerical precision in deep learning, emphasizing the critical balance between computational efficiency and precision. They suggested that with careful implementation, lower precision models can be as effective as their higher precision counterparts while requiring less computational and memory resources. Building upon these findings, significant advancements have been made in utilizing 16-bit operations for training convolutional neural networks. Courbariaux et al. \cite{Courbariaux2016} pioneered a novel technique to train neural networks using binary weights and activations, significantly reducing memory and computational demands. Similarly, Micikevicius et al. \cite{Micikevicius2017} stressed the imperative transition from 32-bit to 16-bit operations in deep learning, given the memory and computational constraints associated with training increasingly complex neural networks. Their research demonstrated that half-precision computations offer a more memory and computation-efficient alternative without compromising the model's performance. Despite these advancements, numerical instability during backpropagation remains a persistent challenge. Bengio et al. \cite{Bengio1994} illustrated the difficulties encountered in learning long-term dependencies with gradient descent due to numerical instability. Such findings underscore the need for innovative solutions to mitigate this widespread issue. One critical component to addressing this challenge lies in the development of effective optimizers. Adam, an optimizer introduced by Kingma and Ba \cite{kingma2014adam}, is known to face issues of numerical instability when employed in lower-precision environments. Another research by Yun et al. \cite{yun2023defense}, provided a comprehensive theoretical analysis, focusing on the performance of pure 16-bit floating-point neural networks. They introduced the concepts of floating-point error and tolerance to define the conditions under which 16-bit models could approximate their 32-bit counterparts. Their results indicate that pure 16-bit floating-point neural networks can achieve similar or superior performance compared to their mixed-precision and 32-bit counterparts, offering a unique perspective on the benefits of pure 16-bit networks. Lastly, in the context of efficient neural network training, Han et al. \cite{Han2015} proposed a three-stage pipeline that significantly reduces the storage requirements of the network. Their work forms an integral part of the broader discussion on efficient neural network training, further reinforcing the relevance of our research on 16-bit operations. Through our investigation, we aim to contribute to this body of work by presenting a novel approach to addressing numerical instability issues associated with 16-bit precision in the realm of deep learning model training.

\section{3D-to-2D Generative Pre-training}
\subsection{Preliminary: Generative Pre-training}

Generative pre-training is a fundamental branch of pre-training methods that aims at reconstructing integral and complete data given partial or disrupted input. Mathematically, suppose $x$ is a sample from raw data with no annotation. The pre-processing step $T(\cdot)$ either erases part of $x$ randomly or splits $x$ into pieces and intermingles them to get $\tilde{x}=T(x)$. The generative pre-training model $M$ is designed to restore from those broken input $\hat{x}=M(T(x))$ and the training loss function is designed to measure the reconstruction distance $\mathcal{L}=D(\hat{x}, x)$.
In point cloud object analysis, earlier generative pre-training methods propose various pretext tasks as $T$, including deformation~\cite{achituve2021self}, jigsaw puzzles~\cite{Jigsaw3D} and depth projection~\cite{occo} to produce disarrayed or partial point clouds. Recently, inspired by MAE~\cite{mae} in the image domain, generative pre-training in 3D domain mainly focuses on implementing random masking as $T$ and utilizing Transformers model as $M$ for reconstruction~\cite{yu2022point, pang2022masked, liu2022masked, pointm2ae}. The reconstruction distance $D$ is usually measured by the classical $l_2$ Chamfer Distance:
\begin{equation}
    D(\hat{x},x)=\frac{1}{\lvert \hat{x}\rvert}\sum_{a\in \hat{x}}\min_{b\in x}\lVert a-b \rVert_2^2 + \frac{1}{\lvert x\rvert}\sum_{b\in x}\min_{a\in \hat{x}}\lVert a-b \rVert_2^2
\label{eq:chamfer}
\end{equation}
Besides Chamfer Distance between point clouds, some methods also exploit feature distance between latents~\cite{yu2022point} or occupancy value distance~\cite{liu2022masked} as the loss function. 

The exact reason why generative pre-training would help enhance the representation ability of backbone models still remains an open question. However, abundant experimental results have conveyed that predicting missing parts according to known parts demands high reasoning ability and global comprehension capacity of the model. What's more, generative pre-training is more efficient and suitable for point cloud object analysis than contrastive pre-training, given that contrastive pre-training typically requires a large amount of training data to avoid trivial overfitting solutions but there has always been a data-starvation problem in point cloud object research field. 

\subsection{Overall Pipeline}

Different from the aforementioned generative pre-training methods that focus on uni-modal point cloud reconstruction, we propose a novel cross-modal pre-training approach of generating view images from instructed camera poses. 

The overall architecture of our proposed TAP pre-training model is depicted in Figure~\ref{fig:pipeline}. Our model takes as an input point cloud $P\in \mathbb{R}^{N\times 3}$, where $N$ is the number of points in the input point cloud. The basic building block of TAP mainly consists of: 1) a \textit{3D Backbone} that extracts 3D geometric features $F_\textrm{3d}\in \mathbb{R}^{n\times C_\textrm{3d}}$, where $n$ is the number of downsampled center points and $C_\textrm{3d}$ is the geometric feature dimension; 2) a \textit{pose-dependent Photograph Module} that takes as inputs $F_\textrm{3d}$ and pose matrix $R\in \mathbb{R}^{3\times 3}$, and predicts view image features $F_\textrm{2d}^R\in \mathbb{R}^{h\times w\times C_\textrm{2d}}$ conditioned on $R$, where $h, w$ are height and width of predicted view image feature map; 3) an \textit{2D Generator} that decodes $F_\textrm{2d}^R$ into an RGB image $I^R_\textrm{gen}\in \mathbb{R}^{H\times W\times 3}$, where $H, W$ are height and width of the output view image. 

As we place no restriction on $F_\textrm{3d}$, the \textit{3D Backbone} can be arbitrarily chosen and adopted. Therefore, our TAP is more flexible and compatible than existing generative pre-training methods that are limited to Transformer-based architecture. Experimental results in Section~\ref{sec:exp} will later verify that TAP brings consistent improvement to all kinds of point cloud models. The technical designs of the \textit{pose-dependent Photograph Module} will be thoroughly discussed in Section~\ref{sec:photo_module}. The \textit{2D Generator} consists of four Transpose Convolution layers to progressively upsample image resolution and decode RGB colors of each pixel.

\subsection{Photograph Module}
\label{sec:photo_module}

\noindent\textbf{Architectural Design.} As illustrated in Figure~\ref{fig:pipeline}, we leverage cross-attention mechanism from Transformers~\cite{vaswani2017attention} to build our \textit{pose-dependent Photograph Module}.
\begin{equation}
    \textrm{Attention}(Q,K,V) = \textrm{softmax}\left(\frac{QK^T}{\sqrt{d_k}}\right)V
\end{equation}
where $d_k$ is the scaling factor, and $Q,K,V$ are quries, keys and values matrix. More specifically, we design a Query Generator $\Phi$ to encode camera pose conditions into query tokens: $Q=\Phi(R)\in \mathbb{R}^{hw\times C_\textrm{2d}}$. We also design a Memory Builder $\Theta$ to construct $K$ and $V$ from 3D geometric features: $K=V=\Theta(F_\textrm{3d})\in \mathbb{R}^{m\times C_\textrm{2d}}$, where $m$ is the number of memory tokens. The output sequence of the cross attention layers will be rearranged from $hw \times C_\textrm{2d}$ to $h\times w \times C_\textrm{2d}$, forming the predicted view image features $F_\textrm{2d}^R$.

During the cross-attention calculation process, we do not explicitly provide any projection clues of which 3D points would project to which 2D pixel. Instead, the Photograph Module learns by itself how to arrange unordered 3D feature points to ordered 2D pixel grids, purely based on semantic similarities between 3D geometric features and our delicately-designed queries that reveal pose information. Since one sample will only have one set of memory tokens in 3D space but its view images from different poses are quite distinct from each other, learning to predict precise view images from instructed poses in a data-driven manner is not a trivial task. Therefore, during the end-to-end optimization process, the 3D backbone is trained to have a stronger perception of the object's overall geometric structure and gain a higher representative ability of the stereoscopic relations. In this way, our proposed 3D-to-2D generative pre-training would help exploit the potential and enhance the strength of 3D backbone models.

\vspace{6pt}
\noindent\textbf{Query Generator.} The query generator $\Phi$ is designed to encode pose condition $R$ into 2D grid of shape $h\times w$. In object analysis, common practice is leveraging parallel light shading to project 3D objects onto 2D grids, and pose matrix $R$ here is used to rotate objects into desired angles before projection. Therefore, each 2D grid actually represents an optical line that starts from infinity, passes through 3D objects and ends at the 2D plane. As a consequence, we choose the direction and the origin points that the optical line goes through as the delegate of the query grid. 

Before deriving formulations of optical lines for each grid, let us first revisit the parallel light shading process for better comprehension. Given 3D coordinates $\mathbf{x}=(x,y,z)$ of a point cloud $P$ and pose matrix $R$, rotation is first performed to align the object to the ideal pose position:
\begin{equation}
    \mathbf{x'}=(x',y',z')=R\mathbf{x}
\label{eq:rotate}
\end{equation}
Then we just omit the final dimension $z'$ and evenly split the first two dimensions $(x',y')$ into 2D grids $(u,v)$:
\begin{equation}
\begin{aligned}
    u = \frac{x'-x_0}{g_h} + o_h, \quad
    v = \frac{y'-y_0}{g_w} + o_w
\label{eq:proj}
\end{aligned}
\end{equation}
where $(x_0, y_0)$ is the minimum value of $(x',y')$, $(g_h, g_w)$ is the grid size, $(o_h, o_w)$ is the offset value to place the projected object at the center of the image. $0\leq u \leq h-1, 0\leq v \leq w-1$ and $(u,v)$ is a sampled pixel coordinate from the 2D grid.

Now let us begin to derive formulations of the optical line that passes through the query grid. We only know $(u,v)$ for each grid and we want to reversely trace which 3D points $(x,y,z)$ are on the same optical line during parallel light projection. According to Eq.~\ref{eq:proj}:
\begin{equation}
\begin{aligned}
    x' &= g_h u + x_0 - o_h = \Psi_h(u) \\
    y' &= g_w v + y_0 - o_w = \Psi_w(v)
\end{aligned}
\end{equation}
If we denote $A=R^{-1}$ and $A_{ij}$ as the element at $i^{th}$ row and $j^{th}$ column, then according to Eq.~\ref{eq:rotate}:
\begin{equation}
\begin{aligned}
    x &= A_{11}\Psi_h(u) + A_{12}\Psi_w(v) + A_{13}z' = \Omega_x(u,v) + A_{13}z' \\
    y &= A_{21}\Psi_h(u) + A_{22}\Psi_w(v) + A_{23}z' = \Omega_y(u,v) + A_{23}z' \\
    z &= A_{31}\Psi_h(u) + A_{32}\Psi_w(v) + A_{33}z' = \Omega_z(u,v) + A_{33}z' 
\label{eq:line}
\end{aligned}   
\end{equation}
According to the definition of line's parametric equation, Eq.~\ref{eq:line} represents a line passing through the origin point $O:(\Omega_x(u,v), \Omega_y(u,v), \Omega_z(u,v))$ with optical line direction $\mathbf{d}=(A_{13}, A_{23}, A_{33})$, where $\Omega_x, \Omega_y, \Omega_z$ are $xyz$ coordinates of $O$ and their formulations are conditioned on $u,v$. Therefore, we concatenate the coordinate of origin point $O$, normalized direction $\mathbf{d}^\dagger = \mathbf{d} / \lVert \mathbf{d} \rVert_2$ and normalized position $(u/h,v/w)$ as positional embedding together to be the initial state of our query. A multi-layer-perceptron (MLP) module is later leveraged to map the 8-dim initial query to higher dimensional space.

\vspace{6pt}
\noindent\textbf{Memory Builder.} The memory builder takes $F_\textrm{3d}$ as input to prepare for initial state of $K, V$ in cross-attention layers. We first concatenate aligned 3D coordinate $P_\textrm{3d}$ with 3D features to enhance the geometric knowledge of $F_\textrm{3d}$:
\begin{equation}
    \hat{F}_\textrm{3d} = \mathrm{MLP}(\mathrm{cat}(F_\textrm{3d}, P_\textrm{3d}))
\end{equation}
Additionally, we initialize a learnable memory token $T_\textrm{pad}$ as the pad token and concatenate it with $\hat{F}_\textrm{3d}$ to obtain the initial state of $K, V$. The reason for concatenating a learnable pad token $T_\textrm{pad}$ is that there are white background areas on the projected image (as shown in Figure~\ref{fig:pipeline}). As $F_\textrm{3d}$ only encodes foreground objects, we further need a learnable pad token to represent background regions. Otherwise, the cross-attention layers will be confused to learn how to combine foreground tokens into background features and this will inevitably diminish the pre-training effectiveness.

\subsection{Objective Function}

We perform per-pixel supervision with Mean Squared Error (MSE) loss between generated view image $I^R_\textrm{gen}$ and ground truth image $I^R_\textrm{gt}$, aligned by camera pose $R$. For simplicity, we will omit $R$ in later formulations. As the background of the rendered ground truth images is all white and reveals little information, we further design a compound loss to balance the weight between foreground regions and background regions:
\begin{equation}
    \mathcal{L}(I_\textrm{gen}, I_\textrm{gt}) = w^\textrm{fg} \mathcal{D}^\textrm{fg} + w^\textrm{bg} \mathcal{D}^\textrm{bg}
\end{equation}
\begin{equation}
    \mathcal{D}^{k}(I^{k}_\textrm{gen}, I^{k}_\textrm{gt}) = \frac{1}{HW}\sum_{h,w}(I^{k}_{\textrm{gen}}(h,w) - I^{k}_{\textrm{gt}}(h,w))^2
\end{equation}
where $k=\textrm{fg (foreground)}, \textrm{bg (background)}$ and $w^\textrm{fg}, w^\textrm{bg}$ are loss weights for foreground and background, respectively. Such per-pixel supervision is more precise than the ambiguous set-to-set Chamfer Distance introduced in Eq.~\ref{eq:chamfer}. 


\section{Experiment}
\subsection{Experiment Setup}
To guarantee a fair assessment, we use a total of seven MMEA datasets derived from three major categories ({bilingual}, {monolingual}, and {high-degree}), with two representative  pre-trained visual encoders (ResNet-152 \cite{DBLP:conf/cvpr/HeZRS16} and CLIP \cite{DBLP:conf/icml/RadfordKHRGASAM21}), and evaluated the performance of {four} models under two distinct settings ({standard} (non-iterative) and {iterative}). In this research, we intentionally set aside the surface modality (literal information) to focus on understanding the effects of absent visual modality on model performance.

\subsubsection{Datasets.}
DBP15K \cite{DBLP:conf/semweb/SunHL17} contains three  datasets ($R_{sa}=0.3$) built from the multilingual versions of DBpedia, including DBP15K$_{ZH\text{-}EN}$, DBP15K$_{JA\text{-}EN}$ and DBP15K$_{FR\text{-}EN}$. 
% Each of them contains about $400$K triples and $15$K pre-aligned entity pairs. 
We adopt their multi-model variants \cite{DBLP:conf/aaai/0001CRC21} with entity-matched images attached.
Besides, four Multi-OpenEA datasets ($R_{sa}=0.2$) \cite{DBLP:journals/corr/abs-2302-08774} are used, which are the multi-modal variants of the OpenEA benchmarks \cite{DBLP:journals/pvldb/SunZHWCAL20} with entity images achieved by searching the entity names through the Google search engine. We include two bilingual datasets \{ EN-FR-15K, EN-DE-15K \} and two monolingual datasets \{ D-W-15K-V1, D-W-15K-V2 \}, where V1 and V2 denote two versions with distinct average relation degrees.
To create our \textbf{MMEA-UMVM} (uncertainly missing visual modality) datasets, we perform random image dropping on MMEA datasets. Specifically, we randomly discard entity images to achieve varying degrees of visual modality missing, ranging from 0.05 to the maximum $R_{img}$ of the raw datasets with a step of 0.05 or 0.1. 
Finally, we get a total number of 97 data split. 
See appendix \footnote{The appendix is attached with the arXiv version of this paper.} for more details. 

%\textbf{\textit{(\rmnum{1})}} {\emph{Bilingual}}: DBP15K \cite{DBLP:conf/semweb/SunHL17} contains three  datasets built from the multilingual versions of DBpedia: DBP15K$_{ZH\text{-}EN}$, DBP15K$_{JA\text{-}EN}$ and DBP15K$_{FR\text{-}EN}$. Each of them contains about $400$K triples and $15$K pre-aligned entity pairs with 30\% of them as the seed alignments. We adopt its multi-model variant \cite{DBLP:conf/aaai/0001CRC21} with entity-matched images attached.
%% which are standard datasets in MMEA community.
%\textbf{\textit{(\rmnum{2})}} {\emph{Monolingual}}: We select FB15K-DB15K (FBDB15K) and FB15K-YAGO15K (FBYG15K) in MMKG \cite{DBLP:conf/esws/LiuLGNOR19} with three data splits which have 20\%, 50\%, and 80\% of pre-aligned EA pairs as the seed alignments, respectively.
%% \end{itemize}

\subsubsection{Iterative Training.}
Following Lin et al. \cite{DBLP:conf/coling/LinZWSW022}, we adopt a probation technique for iterative training. The probation can be viewed as a buffering mechanism, which maintains a temporary cache to store cross-graph mutual nearest entity pairs from the testing set.
Concretely, every $K_e$ (where $K_e = 5$) epochs, we propose cross-KG entity pairs that are mutual nearest neighbors in the vector space and add them to a candidate list $\mathcal{N}^{cd}$. 
Furthermore, an entity pair in $\mathcal{N}^{cd}$ will be added into the training set if it remains a mutual nearest neighbour for $K_s$ ($=$ $10$) consecutive rounds.

\subsubsection{Baselines.}
Six prominent EA algorithms proposed in recent years are selected as our baseline comparisons, excluding the surface information for a parallel evaluation.
%Besides, 
We further collect 3 latest MMEA methods as the strong baselines, including EVA \cite{DBLP:conf/aaai/0001CRC21}, MSNEA \cite{DBLP:conf/kdd/ChenL00WYC22}, and MCLEA \cite{DBLP:conf/coling/LinZWSW022}. Particularly, we reproduce them with their original pipelines unchanged in our benchmark.
\begin{table}[!htbp]
    \centering
	\tabcolsep=0.3cm
    \renewcommand\arraystretch{1.0}
    \caption{{Non-iterative} results of four models with ``w/o CMMI'' setting indicating the absence of the stage-2. 
%    with four degrees of visual modality missing: $R_{img}$ $=$ $\{0.05, 0.2, 0.4, 0.6\} \times 100\%$. 
    The best results within the baselines are marked with \underline{underline}, and we highlight our results with \textbf{bold} when we achieve SOTA.} 
    \resizebox{0.98\linewidth}{!}{
    \begin{tabular}{@{}l|l|ccc|ccc|ccc|ccc}
        \toprule
        & \multirow{2}*{\makebox[1.8cm][c]{Models}} & \multicolumn{3}{c|}{$R_{img}$ $=$ $0.05$} & \multicolumn{3}{c|}{$R_{img}$ $=$ $0.2$} & \multicolumn{3}{c|}{$R_{img}$ $=$ $0.4$} & \multicolumn{3}{c}{$R_{img}$ $=$ $0.6$} \\
        & & {\scriptsize H@1} & {\scriptsize H@10} & {\scriptsize MRR} & {\scriptsize H@1} & {\scriptsize H@10} & {\scriptsize MRR} & {\scriptsize H@1} & {\scriptsize H@10} & {\scriptsize MRR} & {\scriptsize H@1} & {\scriptsize H@10} & {\scriptsize MRR} \\
        \midrule
        % ------------------------------------ DBP ZH-EN -------------------------------------
        \parbox[t]{2mm}{\multirow{6}{*}{\rotatebox[origin=c]{90}{DBP15K$_{ZH-EN}$}}} 
        & MSNEA {\footnotesize {\cite{DBLP:conf/kdd/ChenL00WYC22}}} & .413 & .722 & .517 & .411 & .725 & .518 & .446 & .743 & .546 & .520 & .786 & .611  \\
        & EVA {\footnotesize \cite{DBLP:conf/aaai/0001CRC21}} &
        {.623} & {.878} & {.715} & \underline{.624} & \underline{.878} & \underline{.716} & \underline{.623} & \underline{.875} & \underline{.714} & .625 & .876 & .717 \\
        & MCLEA {\footnotesize {\cite{DBLP:conf/coling/LinZWSW022}}}  &
        \underline{.638} & \underline{.905} & \underline{.732} & {.588} & {.865} & {.686} & {.611} & {.874} & {.704} & \underline{.661} & \underline{.896} & \underline{.744} \\
        & \CC\textbf{w/o CMMI}  
        & \CC{.703} & \CC{.934} & \CC{.787} & \CC{.710} & \CC{.937} & \CC{.793} & \CC{.721} & \CC{.939} & \CC{.801} & \CC{.753} & \CC{.949} & \CC{.825}  \\
        & \CC\textbf{UMAEA}  &
        \CC\textbf{.720} & \CC\textbf{.938} & \CC\textbf{.800} & \CC\textbf{.727} & \CC\textbf{.941} & \CC\textbf{.806} & \CC\textbf{.727} & \CC\textbf{.941} & \CC\textbf{.806} & \CC\textbf{.758} & \CC\textbf{.951} & \CC\textbf{.829} \\
        \cmidrule(lr){2-14}
        & {Improve {$\uparrow$}}  
        & \small{ 8.2$\%$} & \small{ 3.3$\%$} & \small{ .068} & \small{ 10.3$\%$} & \small{ 6.3$\%$} & \small{ .090} & \small{ 10.4$\%$} & \small{ 6.6$\%$} & \small{ .092} & \small{ 9.7$\%$} & \small{ 5.5$\%$} & \small{ .085}  \\
        \midrule
        % ------------------------------------ DBP JA-EN -------------------------------------
        \parbox[t]{2mm}{\multirow{6}{*}{\rotatebox[origin=c]{90}{DBP15K$_{JA-EN}$}}}
        & MSNEA {\footnotesize {\cite{DBLP:conf/kdd/ChenL00WYC22}}}  & .313 & .643 & .425 & .311 & .644 & .422 & .369 & .678 & .472 & .480 & .744 & .569 \\
        & EVA {\footnotesize \cite{DBLP:conf/aaai/0001CRC21}}  & \underline{.615} & .877 & \underline{.708} & \underline{.616} & \underline{.877} & \underline{.710} & \underline{.616} & \underline{.878} & \underline{.711} & .624 & .881 & .716 \\
        & MCLEA {\footnotesize {\cite{DBLP:conf/coling/LinZWSW022}}}  &
        {.599} & \underline{.897} & {.706} & {.579} & {.846} & {.675} & {.613} & {.867} & {.703} & \underline{.686} & \underline{.898} & \underline{.761} \\
        & \CC\textbf{w/o CMMI}  
        & \CC{.708} & \CC{.943} & \CC{.794} & \CC{.712} & \CC{.947} & \CC{.798} & \CC{.730} & \CC{.950} & \CC{.810} & \CC{.772} & \CC{.962} & \CC{.843}  \\
        & \CC\textbf{UMAEA}   &
        \CC\textbf{.725} & \CC\textbf{.949} & \CC\textbf{.807} & \CC\textbf{.726} & \CC\textbf{.949} & \CC\textbf{.808} & \CC\textbf{.732} & \CC\textbf{.952} & \CC\textbf{.813} & \CC\textbf{.775} & \CC\textbf{.963} & \CC\textbf{.845} \\
        \cmidrule(lr){2-14}
        & {Improve {$\uparrow$}}  
        & \small{ 11.0$\%$} & \small{ 5.2$\%$} & \small{ .099} & \small{ 11.0$\%$} & \small{ 7.2$\%$} & \small{ .098} & \small{ 11.6$\%$} & \small{ 7.4$\%$} & \small{ .102} & \small{ 8.9$\%$} & \small{ 6.5$\%$} & \small{ .084}  \\
		\midrule
		% ------------------------------------ DBP FR-EN -------------------------------------
		\parbox[t]{2mm}{\multirow{6}{*}{\rotatebox[origin=c]{90}{DBP15K$_{FR-EN}$}}} 
		& MSNEA {\footnotesize {\cite{DBLP:conf/kdd/ChenL00WYC22}}}  & .297 & .690 & .427 & .304 & .690 & .428 & .360 & .710 & .474 & .478 & .772 & .574 \\
		& EVA {\footnotesize \cite{DBLP:conf/aaai/0001CRC21}}  & .624 & .895 & .720 & \underline{.624} & \underline{.895} & \underline{.720} & \underline{.626} & \underline{.898} & \underline{.721} & .634 & .900 & .728 \\
        & MCLEA {\footnotesize {\cite{DBLP:conf/coling/LinZWSW022}}}  &
        \underline{.634} & \underline{.930} & \underline{.741} & {.582} & {.863} & {.682} & {.601} & {.879} & {.702} & \underline{.675} & \underline{.901} & \underline{.757} \\
        & \CC\textbf{w/o CMMI}  
        & \CC{.727} & \CC{.956} & \CC{.813}& \CC{.733} & \CC{.960} & \CC{.817} & \CC{.746} & \CC{.961} & \CC{.828} & \CC{.790} & \CC{.968} & \CC{.857}\\
        & \CC\textbf{UMAEA}   &
        \CC\textbf{.752} & \CC\textbf{.970} & \CC\textbf{.830} & \CC\textbf{.755} & \CC\textbf{.960} & \CC\textbf{.832} & \CC\textbf{.763} & \CC\textbf{.962} & \CC\textbf{.838} & \CC\textbf{.792} & \CC\textbf{.970} & \CC\textbf{.859} \\
        \cmidrule(lr){2-14}
        & {Improve {$\uparrow$}}  
        & \small{ 11.8$\%$} & \small{ 4.0$\%$} & \small{ .089} & \small{ 13.1$\%$} & \small{ 6.7$\%$} & \small{ .112} & \small{ 13.7$\%$} & \small{ 6.4$\%$} & \small{ .117} & \small{ 11.7$\%$} & \small{ 6.9$\%$} & \small{ .102}  \\
		% ------------------------------------ Open EA EN-FR-------------------------------------
		\midrule
		\parbox[t]{2mm}{\multirow{6}{*}{\rotatebox[origin=c]{90}{OpenEA$_{EN-FR}$}}} 
		& MSNEA {\footnotesize {\cite{DBLP:conf/kdd/ChenL00WYC22}}}  
		& .200 & .431 & .278 & .213 & .439 & .290 & .260 & .477 & .334 & .360 & .560 & .427 \\
		& EVA {\footnotesize \cite{DBLP:conf/aaai/0001CRC21}}  
		& .528 & .833 & .634 & {.533} & {.835} & {.638} & \underline{.539} & {.835} & \underline{.642} & .547 & .830 & .647 \\
        & MCLEA {\footnotesize {\cite{DBLP:conf/coling/LinZWSW022}}}  
        & \underline{.545} & \underline{.852} & \underline{.653} & \underline{.547} & \underline{.852} & \underline{.655} & {.531} & \underline{.839} & {.637} & \underline{.597} & \underline{.852} & \underline{.688} \\
        & \CC\textbf{w/o CMMI}  
        & \CC{.587} & \CC{.893} & \CC{.695} & \CC{.590} & \CC{.893} & \CC{.697} & \CC{.614} & \CC\textbf{.900} & \CC{.715} & \CC{.664} & \CC{.912} & \CC{.753}  \\
        & \CC\textbf{UMAEA}   &
        \CC\textbf{.605} & \CC\textbf{.898} & \CC\textbf{.708} & \CC\textbf{.604} & \CC\textbf{.896} & \CC\textbf{.708} & \CC\textbf{.618} & \CC{.899} & \CC\textbf{.718} & \CC\textbf{.665} & \CC\textbf{.914} & \CC\textbf{.753} \\
        \cmidrule(lr){2-14}
        & {Improve {$\uparrow$}}  
        & \small{ 6.0$\%$} & \small{ 4.6$\%$} & \small{ .055} & \small{ 5.7$\%$} & \small{ 4.4$\%$} & \small{ .053} & \small{ 7.9$\%$} & \small{ 6.1$\%$} & \small{ .076} & \small{ 6.8$\%$} & \small{ 6.2$\%$} & \small{ .065}  \\
        % ------------------------------------ Open EA EN-DE-------------------------------------
        \midrule
		\parbox[t]{2mm}{\multirow{6}{*}{\rotatebox[origin=c]{90}{OpenEA$_{EN-DE}$}}} 
		& MSNEA {\footnotesize {\cite{DBLP:conf/kdd/ChenL00WYC22}}}  & .242 & .486 & .323 & .253 & .495 & .333 & .309 & .542 & .387 & .412 & .622 & .484 \\
		& EVA {\footnotesize \cite{DBLP:conf/aaai/0001CRC21}}  & .717 & .917 & .787 & {.718} & \underline{.918} & {.788} & \underline{.721} & \underline{.920} & \underline{.791} & .734 & \underline{.921} & .800 \\
        & MCLEA {\footnotesize {\cite{DBLP:conf/coling/LinZWSW022}}}  
        & \underline{.723} & \underline{.918} & \underline{.791} & \underline{.721} & {.915} & \underline{.789} & {.697} & {.907} & {.771} & \underline{.745} & {.906} & \underline{.803} \\
        & \CC\textbf{w/o CMMI}  
        & \CC{.752} & \CC{.938} & \CC{.818} & \CC{.757} & \CC{.941} & \CC{.822} & \CC{.771} & \CC{.946} & \CC{.833} & \CC\textbf{.804} & \CC{.954} & \CC{.858}  \\
        & \CC\textbf{UMAEA} &
        \CC\textbf{.757} & \CC\textbf{.942} & \CC\textbf{.823} & \CC\textbf{.759} & \CC\textbf{.943} & \CC\textbf{.824} & \CC\textbf{.774} & \CC\textbf{.947} & \CC\textbf{.835} & \CC\textbf{.804} & \CC\textbf{.957} & \CC\textbf{.860} \\
        \cmidrule(lr){2-14}
        & {Improve {$\uparrow$}}  
        & \small{ 3.4$\%$} & \small{ 2.4$\%$} & \small{ .032} & \small{ 3.8$\%$} & \small{ 2.5$\%$} & \small{ .035} & \small{ 5.3$\%$} & \small{ 2.7$\%$} & \small{ .044} & \small{ 5.9$\%$} & \small{ 3.6$\%$} & \small{ .057}  \\
        % ------------------------------------ Open EA D-W-V1-------------------------------------
        \midrule
		\parbox[t]{2mm}{\multirow{6}{*}{\rotatebox[origin=c]{90}{OpenEA$_{D-W-V1}$}}} 
		& MSNEA {\footnotesize {\cite{DBLP:conf/kdd/ChenL00WYC22}}}  & .238 & .452 & .31 & .254 & .465 & .326 & .318 & .514 & .385 & .432 & .601 & .490 \\
		& EVA {\footnotesize \cite{DBLP:conf/aaai/0001CRC21}}  & .570 & .801 & .653 & \underline{.575} & {.806} & {.658} & {.567} & {.797} & {.650} & .595 & .811 & .673 \\
        & MCLEA {\footnotesize {\cite{DBLP:conf/coling/LinZWSW022}}}  &
        \underline{.585} & \underline{.834} & \underline{.675} & {.574} & \underline{.824} & \underline{.663} & \underline{.581} & \underline{.813} & \underline{.665} & \underline{.655} & \underline{.848} & \underline{.726} \\
        & \CC\textbf{w/o CMMI}  
        & \CC{.640} & \CC{.879} & \CC{.727} & \CC{.644} & \CC{.882} & \CC{.730} & \CC{.667} & \CC{.891} & \CC{.749} & \CC{.722} & \CC\textbf{.908} & \CC{.790}  \\
        & \CC\textbf{UMAEA}   &
        \CC\textbf{.647} & \CC\textbf{.881} & \CC\textbf{.733} & \CC\textbf{.649} & \CC\textbf{.882} & \CC\textbf{.735} & \CC\textbf{.669} & \CC\textbf{.892} & \CC\textbf{.750} & \CC\textbf{.724} & \CC\textbf{.908} & \CC\textbf{.791} \\
        \cmidrule(lr){2-14}
        & {Improve {$\uparrow$}}  
        & \small{ 6.2$\%$} & \small{ 4.7$\%$} & \small{ .058} & \small{ 7.4$\%$} & \small{ 5.8$\%$} & \small{ .072} & \small{ 8.8$\%$} & \small{ 7.9$\%$} & \small{ .085} & \small{ 6.9$\%$} & \small{ 6.0$\%$} & \small{ .065}  \\
        % ------------------------------------ Open EA D-W-V2-------------------------------------
        \midrule
		\parbox[t]{2mm}{\multirow{6}{*}{\rotatebox[origin=c]{90}{OpenEA$_{D-W-V2}$}}} 
		& MSNEA {\footnotesize {\cite{DBLP:conf/kdd/ChenL00WYC22}}}  & .397 & .690 & .497 & .405 & .695 & .503 & .454 & .727 & .546 & .545 & .781 & .626 \\
		& EVA {\footnotesize \cite{DBLP:conf/aaai/0001CRC21}}  & \underline{.775} & .952 & .839 & \underline{.767} & {.947} & \underline{.832} & \underline{.773} & \underline{.950} & \underline{.837} & .788 & \underline{.954} & .848 \\
        & MCLEA {\footnotesize {\cite{DBLP:conf/coling/LinZWSW022}}}  &
        {.771} & \underline{.965} & \underline{.842} & {.753} & \underline{.957} & {.827} & {.757} & {.935} & {.822} & \underline{.800} & {.948} & \underline{.855} \\
        & \CC\textbf{w/o CMMI}  
        & \CC{.828} & \CC{.983} & \CC{.883} & \CC{.829} & \CC\textbf{.982} & \CC{.885} & \CC\textbf{.844} & \CC\textbf{.984} & \CC\textbf{.896} & \CC{.857} & \CC{.986} & \CC\textbf{.905}  \\
        & \CC\textbf{UMAEA}   &
        \CC\textbf{.840} & \CC\textbf{.984} & \CC\textbf{.890} & \CC\textbf{.832} & \CC\textbf{.982} & \CC\textbf{.887} & \CC\textbf{.844} & \CC\textbf{.984} & \CC\textbf{.896} & \CC\textbf{.859} & \CC\textbf{.987} & \CC\textbf{.905} \\
        \cmidrule(lr){2-14}
        & {Improve {$\uparrow$}}  
        & \small{ 6.5$\%$} & \small{ 1.9$\%$} & \small{ .048} & \small{ 6.5$\%$} & \small{ 2.5$\%$} & \small{ .055} & \small{ 7.1$\%$} & \small{ 3.4$\%$} & \small{ .059} & \small{ 5.9$\%$} & \small{ 3.3$\%$} & \small{ .050}  \\        
    \bottomrule
    \end{tabular}
    }
    \label{tab:overall}
\end{table}

\subsubsection{Implementation Details.}\label{sec:detail}
To ensure fairness, we consistently reproduce or implement all methods with the following settings:
{{(\rmnum{1})}} The hidden layer dimensions $d$ for all networks are unified into 300.  
The total epochs for baselines are set to 500 with an optional iterative training strategy applied for another 500 epochs, following \cite{DBLP:conf/coling/LinZWSW022}.
Training strategies including cosine warm-up schedule ($15\%$ steps for LR warm-up), early stopping, and gradient accumulation are adopted. The AdamW optimizer ($\beta_1=0.9$, $\beta_2=0.999$) is used, with a fixed batch size of 3500. 
{{(\rmnum{2})}} To demonstrate model stability, following \cite{DBLP:conf/ksem/ChenLWXWC20,DBLP:conf/coling/LinZWSW022}, the vision encoders $Enc_{v}$ are set to ResNet-152 \cite{DBLP:conf/cvpr/HeZRS16} on DBP15K where the vision feature dimension $d_v$ is $2048$, and set to CLIP \cite{DBLP:conf/icml/RadfordKHRGASAM21} on Multi-OpenEA with $d_v=512$.
{{(\rmnum{3})}} An alignment editing method is employed to reduce the error accumulation \cite{DBLP:conf/ijcai/SunHZQ18}.
{{(\rmnum{4})}} Following Yang et al. \cite{DBLP:conf/emnlp/YangZSLLS19}, Bag-of-Words (BoW) is selected for encoding relations ($x^r$) and attributes ($x^a$)  as fixed-length (i.e., $d_r=d_a=1000$) vectors. Specially, we firstly sort relations/attributes across KGs by frequencies in descending order. At rank $d_r$/$d_a$, we truncated or padded the list to discard the long-tail relations/attributes and obtain fixed-length all-zero vectors $x^r$ and $x^a$. For entity $e_i$: if it includes any of the top-k attributes, the corresponding position in $x_i^a$ is set to 1; if a relation of $e_i$ is among the top-k, the corresponding position in $x_i^r$ is incremented by 1.


In our UMAEA model, $\tau$ is set to 0.1 which determines how much attention the contrast loss pays to difficult negative samples. Besides, the head number $N_h$ in MHCA is set to $1$, and the training epochs are set to \{250, 50, 100\} for stage 1, 2-1, 2-2, respectively. Despite potential performance variations resulting from parameter searching, our focus remained on achieving broad applicability rather than fine-tuning for specific datasets.
During iterative training, the pipeline is repeated; but the expansion of the training set occurs exclusively in stage 1.
For MSNEA, we eliminate the attribute values for input consistency, and extend MSNEA with iterative training capability. All experiments are conducted on RTX 3090Ti GPUs.

\subsection{Overall Results} \label{sec:overall}
\subsubsection{Uncertainly Missing Modality.}
%Typical Setting and Iterative Setting.
Our primary experiment focuses on the model performances with varying missing modality proportions $R_{img}$.
In Table \ref{tab:overall}, we select four representative proportions: $R_{img} \in \{0.05, 0.2, 0.4, 0.6\} \times 100\%$ to simulate the degree of uncertainly missing modality that may exist in real-world scenarios, and evaluate the robustness of different models.
Our UMAEA demonstrates stable improvement on the DBP15K datasets across different $R_{img}$ values in comparison to the top-performing benchmark model:  $10.3\%$ ($R_{img}=0.05$), $11.6\%$ ($R_{img}=0.2$), $11.9\%$ ($R_{img}=0.4$), and $10.3\%$ ($R_{img}=0.6$). We note that it exhibits the most significant improvement when the $R_{img}$ lies between $20\%$ and $40\%$.
For the Multi-OpenEA datasets, our average improvement is: $5.5\%$ ($R_{img}=0.05$), $5.9\%$ ($R_{img}=0.2$), $7.3\%$ ($R_{img}=0.4$), and $6.4\%$ ($R_{img}=0.6$). Although the improvement is slightly lower than in DBP15K, the overall advantage range remains consistent, aligning with our motivation.
Besides, Figure \ref{fig:line} visualizes performance variation curves for three models. 
% Among them, we specify the interval range of $R_{img}$, which is from 0 to the maximum available $R_{img}$ on DBP15K data, and from 0 to 0.95 on OpenEA. 
The overall performance trend fits the conclusions drawn in Table \ref{tab:overall}, showing that our method outperforms the baseline in terms of significant performance gap, regardless of whether iterative or non-iterative learning is employed.

Additionally, we notice a phenomenon that existing models exhibit performance oscillations (EVA) or even declines (MCLEA) at higher modality missing rates. This kind of adverse effect peaks within a particular $R_{img}^1$ range and gradually recovers and gains benefits as $R_{img}$ rises to a certain level $R_{img}^2$. In other words, when $0 \leq R_{img} \leq R_{img}^2$, the additional multi-modal data negatively impacts EA. 
This observation seems counterintuitive since providing more information leads to side effects, but it is also logical. Introducing images for half of the entities means that the remaining half may become noise, which calls for a necessary trade-off.
Under the standard (non-iterative) setting, MCLEA's $R_{img}^2$ averages $63.6\%$, which is $57.14\%$ for MSNEA and $46.43\%$ for EVA across seven datasets. Our method, augmented with the CMMI module, reaches $20.71\%$ for $R_{img}^2$. Even without CMMI, the $R_{img}^2$ of UMAEA remains at $34.29\%$. 
This implies that our method can gain benefits with fewer visual modality data in entity.
Meanwhile, UMAEA exhibits less oscillation and greater robustness than other methods, as further evidenced by the entity distribution analysis in Section \ref{sec:dist}.

% Figure environment removed

We observe that our performance improvement on Multi-OpenEA is less pronounced compared to the DBP15K dataset.  This may be due to the higher image feature quality of CLIP compared to ResNet-152, which in turn diminishes the relative benefit of our model in addressing feature ambiguity. Additionally, as the appendix shows, these datasets have fewer relation and attribute types, allowing for better feature training with comparable data sizes (with a fixed 1000-word bag size, long tail effects are minimized) which partially compensates for missing image modalities.  This finding can also explain why, as seen in Figure \ref{fig:line}, our model's performance improvement decreases as $R_{img}$ increases, and our enhancement in the dense graph (D-W-V2) is slightly less pronounced than in the sparse graph (D-W-V1) which has richer graph structure information.

% 






\subsubsection{Complete Modality.}
We also evaluate our model on the standard multi-modal DBP15K \cite{DBLP:conf/aaai/0001CRC21} dataset, achieving satisfactory results with or without the visual modality (w/o IMG), as shown in Table \ref{tab:std-1}. It is noteworthy that the DBP15K dataset only has part of the entities with images attached (e.g., $78.29\%$ in DBP15K$_{ZH\text{-}EN}$, $70.32\%$ in DBP15K$_{FR\text{-}EN}$, and $67.58\%$ in DBP15K$_{JA\text{-}EN}$), which is inherent to the DBPedia database.
To further showcase our method's adaptability, in Table \ref{tab:std-2}, we evaluate it on the standard Multi-OpenEA dataset with $100\%$ image data attached, demonstrating that our method can be superior in the (MM)EA task against the potentially ambiguous modality information.
% since we achieve the best performance with full images or without images.
% where UMAEA continues to achieves SOTA performance.
\begin{table}[!htbp]
    \centering
%    \footnotesize
	\tabcolsep=0.3cm
    \renewcommand\arraystretch{1.0}
    \caption{Non-iterative (Non-iter.) and iterative (Iter.) results on three multi-modal DPB15K \cite{DBLP:conf/semweb/SunHL17} datasets, where `` * '' refers to involving the visual information for EA. 
    }
    \resizebox{0.87\linewidth}{!}{
    \begin{tabular}{@{}l|l|ccc|ccc|ccc}
        \toprule
        & \multirow{2}*{\makebox[2cm][c]{Models}} & \multicolumn{3}{c|}{DBP15K$_{ZH-EN}$} & \multicolumn{3}{c|}{DBP15K$_{JA-EN}$} & \multicolumn{3}{c}{DBP15K$_{FR-EN}$} \\
        & & {\scriptsize H@1} & {\scriptsize H@10} & {\scriptsize MRR} & {\scriptsize H@1} & {\scriptsize H@10} & {\scriptsize MRR} & {\scriptsize H@1} & {\scriptsize H@10} & {\scriptsize MRR} \\
        \midrule
        \parbox[t]{2mm}{\multirow{8}{*}{\rotatebox[origin=c]{90}{Non-iter.}}} 
        & AlignEA {\footnotesize \cite{DBLP:conf/ijcai/SunHZQ18}} & 
        .472 & .792 & .581 & .448 & .789 & .563 & .481 & .824 & .599 \\
        & KECG {\footnotesize {\cite{DBLP:conf/emnlp/LiCHSLC19}}} &
        .478 & .835 & .598 & .490 & .844 & .610 & .486 & .851 & .610 \\
        & MUGNN {\footnotesize \cite{DBLP:conf/acl/CaoLLLLC19}} &
        .494 & .844 & .611 &  .501 & .857 & .621 & .495 & .870 & .621 \\
        & AliNet {\footnotesize {\cite{DBLP:conf/aaai/SunW0CDZQ20}}} &
        .539 & .826 & .628 & .549 & .831 & .645 & .552 & .852 & .657 \\
        & MSNEA* {\footnotesize {\cite{DBLP:conf/kdd/ChenL00WYC22}}} & .609 & .831 & .685 & .541 & .776 & .620 & .557 & .820 & .643 \\
        & EVA* {\footnotesize \cite{DBLP:conf/aaai/0001CRC21}} &
        {.683} & {.906} & {.762} & {.669} & {.904} & {.752} & {.686} & \underline{.928} & {.771} \\
        & MCLEA* {\footnotesize {\cite{DBLP:conf/coling/LinZWSW022}}}  &
        \underline{.726} & \underline{.922} & \underline{.796} & \underline{.719} & \underline{.915} & \underline{.789} & \underline{.719} & {.918} & \underline{.792} \\
        & \CC\textbf{UMAEA*}  &
        \CC\textbf{.800} & \CC\textbf{.962} & \CC\textbf{.860} & \CC\textbf{.801} & \CC\textbf{.967} & \CC\textbf{.862} & \CC\textbf{.818} & \CC\textbf{.973} & \CC\textbf{.877} \\
        & \CC\textbf{~~ w/o IMG}  &
        \CC\textbf{.718} & \CC\textbf{.930} & \CC\textbf{.797} & \CC\textbf{.723} & \CC\textbf{.941} & \CC\textbf{.803} & \CC\textbf{.748} & \CC\textbf{.956} & \CC\textbf{.826} \\
        \midrule
        \parbox[t]{2mm}{\multirow{6}{*}{\rotatebox[origin=c]{90}{Iter.}}} 
        & BootEA {\footnotesize \cite{DBLP:conf/ijcai/SunHZQ18}} & 
        .629 & .847 & .703 & .622 & .854 & .701 & .653 & .874 & .731 \\
        & NAEA {\footnotesize {\cite{DBLP:conf/ijcai/ZhuZ0TG19}}} &
        .650 & .867 & .720 & .641 & .873 & .718 & .673 & .894 & .752 \\
        & MSNEA* {\footnotesize {\cite{DBLP:conf/kdd/ChenL00WYC22}}} & .648 & .881 & .728 & .557 & .804 & .643 & .583 & .848 & .672 \\
        & EVA* {\footnotesize \cite{DBLP:conf/aaai/0001CRC21}} &
        {.750} & {.912} & {.810} & {.741} & {.921} & {.807} & {.765} & {.944} & {.831} \\
        & MCLEA* {\footnotesize {\cite{DBLP:conf/coling/LinZWSW022}}}  &
        \underline{.811} & \underline{.957} & \underline{.865} & \underline{.805} & \underline{.958} & \underline{.863} & \underline{.808} & \underline{.963} & \underline{.867} \\
        & \CC\textbf{UMAEA*}  &
        \CC\textbf{.856} & \CC\textbf{.974} & \CC\textbf{.900} & \CC\textbf{.857} & \CC\textbf{.980} & \CC\textbf{.904} & \CC\textbf{.873} & \CC\textbf{.988} & \CC\textbf{.917}   \\
        & \CC\textbf{~~ w/o IMG}  &
        \CC\textbf{.793} & \CC\textbf{.952} & \CC\textbf{.852} & \CC\textbf{.794} & \CC\textbf{.960} & \CC\textbf{.857} & \CC\textbf{.820} & \CC\textbf{.976} & \CC\textbf{.880} \\
        \bottomrule
    \end{tabular}
    }
    \label{tab:std-1}
\end{table}

\begin{table}[!htbp]
    \centering
%    \footnotesize
	\tabcolsep=0.3cm
    \renewcommand\arraystretch{1.0}
    \caption{Non-iterative (Non-iter.) and iterative (Iter.) results on four standard Multi-OpenEA \cite{DBLP:journals/corr/abs-2302-08774} datasets with $R_{img}=1.0$.  
    }
    \resizebox{0.98\linewidth}{!}{
    \begin{tabular}{@{}l|l|ccc|ccc|ccc|ccc}
        \toprule
        & \multirow{2}*{\makebox[2cm][c]{Models}} & \multicolumn{3}{c|}{OpenEA$_{EN-FR}$} & \multicolumn{3}{c|}{OpenEA$_{EN-DE}$} & \multicolumn{3}{c|}{OpenEA$_{D-W-V1}$} & \multicolumn{3}{c}{OpenEA$_{D-W-V2}$} \\
        & & {\scriptsize H@1} & {\scriptsize H@10} & {\scriptsize MRR} & {\scriptsize H@1} & {\scriptsize H@10} & {\scriptsize MRR} & {\scriptsize H@1} & {\scriptsize H@10} & {\scriptsize MRR} & {\scriptsize H@1} & {\scriptsize H@10} & {\scriptsize MRR} \\
        \midrule
        \parbox[t]{2mm}{\multirow{4}{*}{\rotatebox[origin=c]{90}{Non-iter.}}} 
        & MSNEA* {\footnotesize {\cite{DBLP:conf/kdd/ChenL00WYC22}}} 
        & .692 & .813 & .734 & .753 & .895 & .804 & .800 & .874 & .826 & .838 & .940 & .873 \\
        & EVA* {\footnotesize \cite{DBLP:conf/aaai/0001CRC21}} 
        & {.785} & {.932} & {.836} & {.922} & {.983} & {.945} & {.858} & {.946} & {.891} & .890 & .981 & .922 \\
        & MCLEA* {\footnotesize {\cite{DBLP:conf/coling/LinZWSW022}}}  
        & \underline{.819} & \underline{.943} & \underline{.864} & \underline{.939} & \underline{.988} & \underline{.957} & \underline{.881} & \underline{.955} & \underline{.908} & \underline{.928} & \underline{.983} & \underline{.949} \\
        & \CC\textbf{UMAEA*}  
        & \CC\textbf{.848} & \CC\textbf{.966} & \CC\textbf{.891} & \CC\textbf{.956} & \CC\textbf{.994} & \CC\textbf{.971} & \CC\textbf{.904} & \CC\textbf{.971} & \CC\textbf{.930} & \CC\textbf{.948} & \CC\textbf{.996} & \CC\textbf{.967} \\
        \midrule
        \parbox[t]{2mm}{\multirow{4}{*}{\rotatebox[origin=c]{90}{Iter.}}}
        & MSNEA* {\footnotesize {\cite{DBLP:conf/kdd/ChenL00WYC22}}} 
        & .699 & .823 & .742 & .788 & .917 & .835 & .809 & .885 & .836 & .862 & .954 & .894 \\
        & EVA* {\footnotesize \cite{DBLP:conf/aaai/0001CRC21}} 
        & {.849} & {.974} & {.896} & {.956} & {.985} & {.968} & {.915} & {.986} & {.942} & .925 & .996 & .951 \\
        & MCLEA* {\footnotesize {\cite{DBLP:conf/coling/LinZWSW022}}}  
        & \underline{.888} & \underline{.979} & \underline{.924} & \underline{.969} & \underline{.993} & \underline{.979} & \underline{.944} & \underline{.989} & \underline{.963} & \underline{.969} & \underline{.997} & \underline{.982} \\
        & \CC\textbf{UMAEA*}  
        & \CC\textbf{.895} & \CC\textbf{.987} & \CC\textbf{.931} & \CC\textbf{.974} & \CC\textbf{998} & \CC\textbf{.984} & \CC\textbf{.945} & \CC\textbf{.994} & \CC\textbf{.965} & \CC\textbf{.973} & \CC\textbf{.999} & \CC\textbf{.984} \\
        \bottomrule
    \end{tabular}
    }
    \label{tab:std-2}
\end{table}





%\subsection{Ablation Studies} \label{sec:ablation}



% Figure environment removed
\subsection{Details Analysis} \label{sec:analysis}
\subsubsection{{Component Analysis.}}
We further analyze the impact of each training objective on our model's performance in Figure \ref{fig:ablation}, where the absence of any objective results in varying performance degradation. 
As mentioned in Section \ref{sec:iir}, IIR serves as an enhancement for ECIA, and its influence is comparatively less significant than that of $\mathcal{L}_{GMI}$ and $\mathcal{L}_{ECIA}$.
%Notably, we discuss performance changes without the CMMI module for an independent comparison. 
The CMMI module's influence is detailed in Table \ref{tab:overall}, where it becomes more significant when $R_{img}$ is low. CMMI's primary function is to mitigate noise in the missing modalities,  facilitating efficient learning at high noise levels and minimizing the noise to existing information.

\subsubsection{{Efficiency Analysis.}}
Concurrently, we briefly compare the relationship between model parameter size, training time, and performance. Our model improves the  performance with only a minor increase in parameters and time consumption. 
This indicates that in many cases, our method can directly substitute these models with minimal additional overhead.
While there is potential for enhancing UMAEA's efficiency, we view this as a direction for future research.

\begin{table}[!htbp]
    \centering
%    \footnotesize
	\tabcolsep=0.3cm
    \renewcommand\arraystretch{1.0}
    \caption{Efficiency Analysis. Non-iterative model performance on three datasets with $R_{img}=0.4$, where ``Para.'' refers to the number of learnable parameters and ``Time'' refers to the total time required for model to reach the optimal performance.
    }
    \resizebox{1.0\linewidth}{!}{
    \begin{tabular}{l|ccc|ccc|ccc}
        \toprule
         \multirow{2}*{\makebox[2cm][c]{Models}} & \multicolumn{3}{c|}{DBP15K$_{JA-EN}$} & \multicolumn{3}{c|}{DBP15K$_{FR-EN}$} & \multicolumn{3}{c}{OpenEA$_{EN-FR}$} \\
        & {\scriptsize Para. (M) } & {\scriptsize Time (Min) } & {\scriptsize MRR } & {\scriptsize Para. (M) } & {\scriptsize Time (Min) } & {\scriptsize MRR } & {\scriptsize Para. (M) } & {\scriptsize Time (Min) } & {\scriptsize MRR } \\
        \midrule
         EVA* {\footnotesize \cite{DBLP:conf/aaai/0001CRC21}} 
        & 13.27 & 30.9 & .711 & 13.29 & 30.8 & .721 & 9.81 & 17.8 & .642   \\
         MCLEA* {\footnotesize {\cite{DBLP:conf/coling/LinZWSW022}}}  
        & 13.22 & 15.3 & .703 & 13.24 & 15.7 & .702 & 9.75 & 19.5 & .637   \\
        \CC{w/o CMMI}  
        & \CC{13.82} & \CC{30.2} & \CC{.810} & \CC{13.83} & \CC{28.8} & \CC{.828} & \CC{10.35} & \CC{17.9} & \CC{.715}   \\
        \CC{UMAEA}  
        & \CC{14.72} & \CC{33.4} & \CC{.813} & \CC{14.74} & \CC{32.7} & \CC{.838} & \CC{11.26} & \CC{23.1} & \CC{.718} \\
        \bottomrule
    \end{tabular}
    }
    \label{tab:overall-iter-2}
\end{table}

% Figure environment removed
\subsubsection{Entity Distribution Analysis.} \label{sec:dist}
To further evaluate the robustness of our method,
% in various multi-modal entity pair scenarios
we analyze the model's prediction performance under different distributions of entity's visual modality.  Concretely, we compare five testing sets under $R_{img} \in  \{0.2, 0.4, 0.6\}$ with details presented in Figure \ref{fig:case},
% sorted based on the average expected number of images contained in each entity pair. 
where we exclude the CMMI module during the comparison.
We observe that EVA's performance is generally stable but underperforms when visual modality is  complete (TS 1), suggesting its overfitting to modality noise in the training stage.
In contrast, MCLEA exhibits more extreme performance fluctuations, performing worse than EVA does when there's incomplete visual information within the entity pairs (TS 2, 3, 4, 5).
Our superior performance reflects the intuition that the optimal performance occurs in TS 1, with tolerable fluctuations in other scenarios.
% More experimental results on other datasets are available in appendix.








%!TEX root = ecai-main.tex



\section{Discussion and Future Work}
\label{sec:con}


% % %   Conclusion
TKB Alignment is a new variant of the alignment problem that admits richer state and property descriptions. Our setting uses \alc-TKBs, CQs with \ltl operators, and a cost function for the edit operations. 
%
We have shown that TKB Alignment \wrt temporal CQs is solvable, by developing computation methods for both TKB and KB Alignment.

% % % %   Discussion
The TKB-alignment problem is closely related to abduction and to computing repairs of KBs, as these tasks also change a KB to either gain a desired consequence or remove an unwanted one. However, although being active research topics, neither of the two has yet been investigated for the temporalized setting and entailment of TCQs. Furthermore, TKB Alignment requires a cost-optimal solution, which is not very common in the context of abduction or repairs.

Interestingly, TKB Alignment can also be used for relaxing temporal CQ answering. Given a tuple of individuals $\bar{a}$ which is not a certain answer of a TCQ $\phi$, solve TKB Alignment for the Boolean TCQ obtained from $\phi'$ by assigning $\bar{a}$ to the answer variables of $\phi$. The costs computed during TKB Alignment for $\phi'$ then measure the \enquote{distance} to a certain answer of the query. 


% % %   Future Work
Our initial investigation on TKB Alignment uses a unitary cost measure for the edit operations mostly to ease presentation, as our approach can handle other cost measures easily. In this work, we did not regard rigid symbols, which are left for future work.
% \todo[inline]{Do we want to mention metric time  or other extensions here?} 





%
%
%


%
% ---- Bibliography ----

 \bibliographystyle{splncs04}
 \bibliography{mybibliography}
%
\appendix
\clearpage

\appendix
\section{Appendix}
\begin{table}[!htbp]
    \centering
    \vspace{-0.1cm}
    % \footnotesize
    \caption{Statistics for original datasets, where ``EA pairs'' refers to the pre-aligned entity pairs. Note that not all entities have the associated images or the equivalent counterparts in the other KG. For dataset \{ EN-FR-15K, EN-DE-15K, D-W-15K-V1, and D-W-15K-V2 \} in Multi-OpenEA, we omit the ``15K'' suffix to unify the description throughout this paper.}
    \label{tab:dataset}
    \vspace{-3pt}
    \renewcommand\arraystretch{1.0}
    \resizebox{1.\linewidth}{!}{
    \begin{tabular}{@{}l|c|cccccccc@{}}
        \toprule
        \makebox[2.5cm][c]{Dataset} & KG & \# Ent. & \# Rel. & \# Attr. & \# Rel. Triples & \# Attr. Triples & \# Image & \# EA pairs \\
        \midrule
        \multirow{2}*{DBP15K$_{ZH\text{-}EN}$} & ZH {\footnotesize (Chinese)} & 19,388 & 1,701 & 8,111 & 70,414 & 248,035 & 15,912 & \multirow{2}*{15,000} \\
        & EN {\footnotesize (English)} & 19,572 & 1,323 & 7,173 & 95,142 & 343,218 & 14,125 \\
        \midrule
        \multirow{2}*{DBP15K$_{JA\text{-}EN}$} & JA {\footnotesize (Japanese)} & 19,814 & 1,299 & 5,882 & 77,214 & 248,991 & 12,739 & \multirow{2}*{15,000} \\
        & EN {\footnotesize (English)} & 19,780 & 1,153 & 6,066 & 93,484 & 320,616 & 13,741 \\
        \midrule
        \multirow{2}*{DBP15K$_{FR\text{-}EN}$} & FR {\footnotesize (French)} & 19,661 & 903 & 4,547 & 105,998 & 273,825 & 14,174 & \multirow{2}*{15,000} \\
        & EN {\footnotesize (English)} & 19,993 & 1,208 & 6,422 & 115,722 & 351,094 & 13,858 \\
        \midrule
        \multirow{2}*{OpenEA$_{EN\text{-}FR}$} & EN {\footnotesize (English)} & 15,000 & 267 & 308 & 47,334 & 73,121 & 15,000 & \multirow{2}*{15,000} \\
        & FR {\footnotesize (French)} & 15,000 & 210 & 404 & 40,864 & 67,167 & 15,000 \\
        \midrule
        \multirow{2}*{OpenEA$_{EN\text{-}DE}$} & EN {\footnotesize (English)} & 15,000 & 215 & 286 & 47,676 & 83,755 & 15,000 & \multirow{2}*{15,000} \\
        & DE (German) & 15,000 & 131 & 194 & 50,419 & 156,150 & 15,000 \\
                \midrule
        \multirow{2}*{OpenEA$_{D\text{-}W\text{-}V1}$} & DBpedia & 15,000 & 248 & 342 & 38,265 & 68,258 & 15,000 & \multirow{2}*{15,000} \\
        & Wikidata & 15,000 & 169 & 649 & 42,746 & 138,246 & 15,000 \\
        \midrule
        \multirow{2}*{OpenEA$_{D\text{-}W\text{-}V2}$} & DBpedia & 15,000 & 167 & 175 & 73,983 & 66,813 & 15,000 & \multirow{2}*{15,000} \\
        & Wikidata & 15,000 & 121 & 457 & 83,365 & 175,686 & 15,000 \\
        \bottomrule
    \end{tabular}
    }
    \vspace{-0.6cm}
\end{table}

\begin{table}[!htbp]
  \centering
\centering
%\footnotesize
{
\vspace{-2pt}
\caption{The proportion $R_{img}$ of entities containing images for each dataset in our setting, with ``\texttt{STD}'' refers to the standard $R_{img}$ in raw datasets.
}
\label{tab:split}
}
\vspace{-2pt}
\resizebox{1.0\linewidth}{!}{
\begin{tabular}{@{}l|l}
\hline
 & \\ [-2ex]
\makebox[2.5cm][c]{Dataset} & \makebox[14cm][c]{$R_{img}$} \\
 & \\ [-2ex]
\hline
DBP15K$_{ZH\text{-}EN}$
&\texttt{\small ~0.05, 0.1, 0.15, 0.2, 0.3, 0.4, 0.45, 0.5, 0.55, 0.6, 0.7, 0.75, 0.7829~(STD)}\\
& \\ [-2ex]
\hline
 & \\ [-2ex]
DBP15K$_{JA\text{-}EN}$
&\texttt{\small ~0.05, 0.1, 0.15, 0.2, 0.3, 0.4, 0.45, 0.5, 0.55, 0.6, 0.7, 0.7032~(STD)}\\
& \\ [-2ex]
\hline
 & \\ [-2ex]
DBP15K$_{FR\text{-}EN}$
&\texttt{\small ~0.05, 0.1, 0.15, 0.2, 0.3, 0.4, 0.45, 0.5, 0.55, 0.6, 0.6758  (STD) }\\
& \\ [-2ex]
\hline
 & \\ [-2ex]
OpenEA$_{EN\text{-}FR}$
&\texttt{\small ~0.05, 0.1, 0.15, 0.2, 0.3, 0.4, 0.45, 0.5, 0.55, 0.6, 0.7, 0.8, 0.9, 0.95, 1.0~(STD)}\\ 
& \\ [-2ex]
\hline
 & \\ [-2ex]
OpenEA$_{EN\text{-}DE}$
&\texttt{\small ~0.05, 0.1, 0.15, 0.2, 0.3, 0.4, 0.45, 0.5, 0.55, 0.6, 0.7, 0.8, 0.9, 0.95, 1.0~(STD)}\\ 
& \\ [-2ex]
\hline
 & \\ [-2ex]
OpenEA$_{D\text{-}W\text{-}V1}$
&\texttt{\small ~0.05, 0.1, 0.15, 0.2, 0.3, 0.4, 0.45, 0.5, 0.55, 0.6, 0.7, 0.8, 0.9, 0.95, 1.0~(STD)}\\ 
& \\ [-2ex]
\hline
 & \\ [-2ex]
OpenEA$_{D\text{-}W\text{-}V2}$
&\texttt{\small ~0.05, 0.1, 0.15, 0.2, 0.3, 0.4, 0.45, 0.5, 0.55, 0.6, 0.7, 0.8, 0.9, 0.95, 1.0~(STD)}\\ 
[-2ex] \\
\hline
\end{tabular}}
\vspace{-0.6cm}
\end{table}

\subsection{Dataset Statistics}
Our detailed dataset statistics are presented in Table \ref{tab:dataset}.
A set of pre-aligned entity pairs is offered for guidance,  
which is proportionally split into a training set (seed alignments $\mathcal{S}$) and a testing set $\mathcal{S}_{te}$ based on the given seed alignment ratio ($R_{sa}$).
Notably,   each entity in the four Multi-OpenEA benchmark \cite{DBLP:journals/corr/abs-2302-08774} is initially associated with three images obtained from the Google search engine. In this study, we select the highest-ranked image, which is the first one, to serve as the visual information for the entity.
 The details for 97 data splits are contained in Table \ref{tab:split}, and the complete data for benchmark is accessible at {\color{blue}\url{https://github.com/zjukg/UMAEA}}.

% \vspace{-0.4cm}
\subsection{Supplementary for Experiments}
% Figure environment removed
\subsubsection{{Experiment Settings.}}
Those attribute triples $<$\textit{entity}, \textit{attribute}, \textit{value}$>$ in KGs  have been researched in many previous EA works \cite{DBLP:conf/aaai/TrisedyaQZ19,DBLP:conf/emnlp/LiuCPLC20,DBLP:conf/ijcai/Tang0C00L20,DBLP:conf/kdd/ChenL00WYC22,DBLP:conf/icde/ZhongZFD22}.
Nevertheless, in order to focus on our key subject,
we do not utilize the contents of \textit{value} parts in this work which are mainly string formats like specific date, land area or coordinate position.
Furthermore, in order to concentrate on uncertainly missing visual modality, we exclude surface-related information such as the name of entity, relation, and attribute. Our approach primarily utilizes information derived from the type of entity and relationship, as well as structure of the graph and the image data, which is inherited from previous works \cite{DBLP:conf/aaai/0001CRC21,DBLP:conf/kdd/ChenL00WYC22,DBLP:conf/coling/LinZWSW022}. Each entity is associated with multiple attributes and either $0$ or $1$ image. We achieve this association through id/index sharing, following previous works \cite{DBLP:conf/kdd/ChenL00WYC22,DBLP:journals/corr/abs-2302-08774,DBLP:conf/coling/LinZWSW022,DBLP:conf/aaai/0001CRC21}, rather than explicitly defining triples. For example, Wang et al. \cite{DBLP:journals/bdr/WangWQZ20} incorporate images as entities through the introduction of a specific \textit{Imageof} relation, allowing for a more formal structure and organization of the KG.

 Regarding the loss trade-off for multi-task learning, we attempted to use the Automatic Weighted Loss (AWL) technique \cite{DBLP:conf/cvpr/KendallGC18} to dynamically assign weights to different training objectives. However, we found that directly summing the losses after scaling resulted in similar performance ($\pm$ 0.3$\%$ in hit$@$1) compared to using AWL. Hence, we omitted this empirical study in the paper.

Regarding $R_{img}^2$ for MCLEA, as mentioned before, the adverse effect gradually recovers and gains benefits as $R_{img}$ rises to a certain level $R_{img}^2$. Here, $R_{img}^2$ represents the minimum observed $R_{img}$ at which the model's performance surpasses that without visual information ($R_{img}$=0). For MCLEA, we calculate as follows:: 
$[0.7(\text{ZH-EN})+0.7(\text{FR-EN})+0.6(\text{JA-EN})+0.55(\text{D-W-v1})+0.7(\text{D-W-v2})+0.6(\text{EN-DE})+0.6(\text{EN-FR})]/7\times100\%=63.6\%$


\vspace{-0.3cm}
\subsubsection{{Additional Experiments.}}
In this section, we provide the remaining benchmark results. 
As a supplement to Figure 5, we offer a performance comparison of models for DBP15K$_{JA-EN}$ and DBP15K$_{FR-EN}$ under different testing sets, as shown in Figure \ref{fig:appcase}, which is consistent to DBP15K$_{ZH-EN}$.

Table \ref{tab:appdbp} and Table \ref{tab:appoea} present the model performance when they are applied to typical EA tasks excluding the influence of visual modality, which obviates the need for the CMMI module during training. 
The results show that our model achieved superior performance in non-multimodal EA tasks, indicating that UMAEA can even effectively mitigate the impact of information imbalance issues arising from attribute, relation, and graph structure during model training. 
Furthermore, we provide the performance curves under the Hit$@$1 metric, as illustrated in Figure \ref{fig:appline}, where the general trend in performance change closely resembles that observed under the MRR metric (Figure \ref{fig:line}).

\begin{table}[!htbp]
    \centering
%    \footnotesize
	\tabcolsep=0.3cm
	\vspace{-0.2cm}
    \renewcommand\arraystretch{1.0}
    \caption{Non-iterative (Non-iter.) and iterative (Iter.) results on three standard DPB15K \cite{DBLP:conf/semweb/SunHL17} datasets with $R_{sa}=0.3$ without the visual modality ($R_{img}=0$). 
    }
    % \vspace{-1pt}
    \resizebox{0.84\linewidth}{!}{
    \begin{tabular}{@{}l|l|ccc|ccc|ccc}
        \toprule
        & \multirow{2}*{\makebox[2cm][c]{Models}} & \multicolumn{3}{c|}{DBP15K$_{ZH-EN}$} & \multicolumn{3}{c|}{DBP15K$_{JA-EN}$} & \multicolumn{3}{c}{DBP15K$_{FR-EN}$} \\
        & & {\scriptsize H@1} & {\scriptsize H@10} & {\scriptsize MRR} & {\scriptsize H@1} & {\scriptsize H@10} & {\scriptsize MRR} & {\scriptsize H@1} & {\scriptsize H@10} & {\scriptsize MRR} \\
        \midrule
        \parbox[t]{2mm}{\multirow{5}{*}{\rotatebox[origin=c]{90}{Non-iter.}}} 
        & MSNEA {\footnotesize {\cite{DBLP:conf/kdd/ChenL00WYC22}}} & .503 & .795 & .602 & .395 & .715 & .504 & .472 & .820 & .593 \\
        & EVA {\footnotesize \cite{DBLP:conf/aaai/0001CRC21}} &
        {.629} & {.882} & {.719} & {.627} & {.879} & {.714} & {.626} & {.896} & {.722} \\
        & MCLEA {\footnotesize {\cite{DBLP:conf/coling/LinZWSW022}}}  &
        {.672} & {.907} & {.756} & {.663} & {.904} & {.751} & {.679} & {.923} & {.769} \\
         & MEAformer {\footnotesize {\cite{chen2023meaformer}}}  &
        \underline{.708} & \underline{.925} & \underline{.787} & \underline{.699} & \underline{.934} & \underline{.785} & \underline{.722} & \underline{.947} & \underline{.805} \\
        & \CC\textbf{UMAEA}  &
        \CC\textbf{.718} & \CC\textbf{.930} & \CC\textbf{.797} & \CC\textbf{.723} & \CC\textbf{.941} & \CC\textbf{.803} & \CC\textbf{.748} & \CC\textbf{.956} & \CC\textbf{.826} \\
        \midrule
        \parbox[t]{2mm}{\multirow{5}{*}{\rotatebox[origin=c]{90}{Iter.}}} 
        & MSNEA {\footnotesize {\cite{DBLP:conf/kdd/ChenL00WYC22}}} & .545 & .850 & .648 & .451 & .788 & .567 & .531 & .872 & .648 \\
        & EVA {\footnotesize \cite{DBLP:conf/aaai/0001CRC21}} &
        {.696} & {.907} & {.774} & {.695} & {.908} & {.772} & {.708} & {.930} & {.790} \\
        & MCLEA {\footnotesize {\cite{DBLP:conf/coling/LinZWSW022}}}  &
        {.749} & {.933} & {.817} & {.752} & {.935} & {.821} & {.779} & {.955} & {.847} \\
        & MEAformer {\footnotesize {\cite{chen2023meaformer}}}  &
        \underline{.775} & \underline{.940} & \underline{.837} & \underline{.761} & \underline{.950} & \underline{.831} & \underline{.785} & \underline{.963} & \underline{.852} \\
        & \CC\textbf{UMAEA}  &
        \CC\textbf{.793} & \CC\textbf{.952} & \CC\textbf{.852} & \CC\textbf{.794} & \CC\textbf{.960} & \CC\textbf{.857} & \CC\textbf{.820} & \CC\textbf{.976} & \CC\textbf{.880} \\
        \bottomrule
    \end{tabular}
    }
    \label{tab:appdbp}
	\vspace{-0.3cm}
\end{table}

\begin{table}[!htbp]
    \centering
%    \footnotesize
	\tabcolsep=0.3cm
    \renewcommand\arraystretch{1.0}
%    \vspace{-0.2cm}
    \caption{Non-iterative (Non-iter.) and iterative (Iter.) results on four standard OpenEA \cite{DBLP:journals/pvldb/SunZHWCAL20} datasets  with $R_{sa}=0.2$ without the visual modality ($R_{img}=0$).  
    }
    \vspace{-1pt}
    \resizebox{0.98\linewidth}{!}{
    \begin{tabular}{@{}l|l|ccc|ccc|ccc|ccc}
        \toprule
        & \multirow{2}*{\makebox[2cm][c]{Models}} & \multicolumn{3}{c|}{OpenEA$_{EN-FR}$} & \multicolumn{3}{c|}{OpenEA$_{EN-DE}$} & \multicolumn{3}{c|}{OpenEA$_{D-W-V1}$} & \multicolumn{3}{c}{OpenEA$_{D-W-V2}$} \\
        & & {\scriptsize H@1} & {\scriptsize H@10} & {\scriptsize MRR} & {\scriptsize H@1} & {\scriptsize H@10} & {\scriptsize MRR} & {\scriptsize H@1} & {\scriptsize H@10} & {\scriptsize MRR} & {\scriptsize H@1} & {\scriptsize H@10} & {\scriptsize MRR} \\
        \midrule
        \parbox[t]{2mm}{\multirow{5}{*}{\rotatebox[origin=c]{90}{Non-iter.}}} 
        & MSNEA {\footnotesize {\cite{DBLP:conf/kdd/ChenL00WYC22}}} 
        & .260 & .506 & .341 & .334 & .572 & .413 & .332 & .545 & .404 & .612 & .840 & .689 \\
        & EVA {\footnotesize \cite{DBLP:conf/aaai/0001CRC21}} 
        & {.525} & {.827} & {.631} & {.721} & {.918} & {.790} & {.579} & {.809} & {.662} & .775 & .952 & .839 \\
        & MCLEA {\footnotesize {\cite{DBLP:conf/coling/LinZWSW022}}}  
        & {.571} & {.862} & {.675} & {.737} & {.921} & {.803} & {.620} & {.848} & {.704} & {.816} & {.972} & {.874} \\
        & MEAformer {\footnotesize {\cite{chen2023meaformer}}}  
        & \underline{.604} & \underline{.895} & \underline{.708} & \underline{.754} & \underline{.937} & \underline{.818} & \underline{.645} & \underline{.878} & \underline{.729} & \underline{.839} & \underline{.982} & \underline{.892} \\
        & \CC\textbf{UMAEA}  
        & \CC\textbf{.608} & \CC\textbf{.897} & \CC\textbf{.711} & \CC\textbf{.763} & \CC\textbf{.942} & \CC\textbf{.826} & \CC\textbf{.653} & \CC\textbf{.883} & \CC\textbf{.738} & \CC\textbf{.840} & \CC\textbf{.982} & \CC\textbf{.892} \\
        \midrule
        \parbox[t]{2mm}{\multirow{5}{*}{\rotatebox[origin=c]{90}{Iter.}}}
        & MSNEA {\footnotesize {\cite{DBLP:conf/kdd/ChenL00WYC22}}} 
        & .294 & .580 & .391 & .385 & .621 & .463 & .417 & .655 & .500 & .657 & .864 & .726 \\
        & EVA {\footnotesize \cite{DBLP:conf/aaai/0001CRC21}} 
        & {.602} & {.873} & {.699} & {.770} & {.936} & {.829} & {.658} & {.861} & {.734} & .848 & .980 & .899 \\
        & MCLEA {\footnotesize {\cite{DBLP:conf/coling/LinZWSW022}}}  
        & {.646} & {.899} & {.739} & {.790} & {.946} & {.846} & {.696} & {.896} & {.772} & {.881} & {.984} & {.922} \\
        & MEAformer {\footnotesize {\cite{chen2023meaformer}}}  
        & \underline{.656} & \underline{.916} & \underline{.749} & \underline{.793} & \underline{.950} & \underline{.848} & \underline{.703} & \underline{.889} & \underline{.772} & \textbf{.884} & \underline{.988} & \underline{.923} \\
        & \CC\textbf{UMAEA}  
        & \CC\textbf{.670} & \CC\textbf{.921} & \CC\textbf{.763} & \CC\textbf{.801} & \CC\textbf{958} & \CC\textbf{.857} & \CC\textbf{.715} & \CC\textbf{.910} & \CC\textbf{.789} & \CC{.882} & \CC\textbf{.993} & \CC\textbf{.925} \\
        \bottomrule
    \end{tabular}
    }
    \label{tab:appoea}
    \vspace{-0.8cm}
\end{table}

 % Figure environment removed

\vspace{-0.3cm}
\subsubsection{\textbf{Baseline Analysis.}}
We attribute the lower performance of translation based methods (e.g., MSNEA) to their reliance on semantics assumptions, which limits their ability to capture the complex structural information among entities for alignment.

Some works \cite{DBLP:conf/ijcai/WuLF0Y019,DBLP:conf/cikm/YangWZQWHH21} hold that 
the structural information plays an important role in the EA task. 
% which makes GNN-based models generally perform better. 
By performing graph convolution over an entity’s neighbors, GCNs can incorporate more structural characteristics of knowledge graphs, while the translation assumption in translation-based models focuses more on the relationship among heads, tails and relations.



\subsection{Model Details}
We reproduce EVA \cite{DBLP:conf/aaai/0001CRC21}, MSNEA \cite{DBLP:conf/kdd/ChenL00WYC22}, MCLEA \cite{DBLP:conf/coling/LinZWSW022} and MEAformer \cite{chen2023meaformer} based on their source code \footnote{\color{blue} \url{https://github.com/cambridgeltl/eva}}$^{,}$\footnote{{\color{blue} \url{https://github.com/lzxlin/MCLEA}}}$^{,}$\footnote{\color{blue} \url{https://github.com/liyichen-cly/MSNEA}}$^{,}$\footnote{\color{blue} \url{https://github.com/zjukg/MEAformer}} with their
original model pipelines unchanged but unifying hyper-parameters.
 Yuan et al. \cite{yuan2023multi} consider the inter-modal effects and mitigate the impact of weak modalities, while Hama et al. \cite{DBLP:journals/access/HamaM23} quantify the importance of modality by embedding the entities into the probability distribution.
 Guo et al. \cite{DBLP:journals/corr/abs-2305-14651} propose the GEEA framework with the mutual variational autoencoder (M-VAE) to mutually encode/decode entities between source and target KGs for both entity alignment and
entity synthesis.
Given that their methods have different goals than ours and were recently published, we did not perform direct comparisons with them in our experiments.

\subsection{Metric Details}
\subsubsection{\textbf{Hits$@$N}} describes the fraction of true aligned 
 target entities that appear in the first N entities of the sorted rank list:
\begin{equation}
  \vspace{-2pt}
    \operatorname{Hits} @ \text{N}=\frac{1}{|\mathcal{S}_{te}|} \sum_{i=1}^{|\mathcal{S}_{te}|} \mathbb{I}[{\text {rank}_i} \leqslant \text{N}]\, ,
\end{equation}
where  ${{\text{rank}}_{i}}$ refers to the rank position of the first correct mapping for the i-th query entities and $\mathbb{I}=1$ if ${\text {rank}_i} \leqslant N$ and 0 otherwise.
$\mathcal{S}_{te}$ refers to the testing alignment set.
\subsubsection{\textbf{MRR}} (Mean Reciprocal Ranking $\uparrow$) is a statistic measure for evaluating many algorithms that produces a list of possible responses to a sample of queries, ordered by probability of correctness. 
In the field of EA, the reciprocal rank of a query entity (i.e., an entity from the source KG) response is the multiplicative inverse of the rank of the first correct alignment entity in the target KG.
MRR is the average of the reciprocal ranks of results for a sample of candidate alignment entities:
\begin{equation}
    \vspace{-2pt}
    \mathbf{MRR}=\frac{1}{|\mathcal{S}_{te}|} \sum_{i=1}^{|\mathcal{S}_{te}|} \frac{1}{\text {rank}_i} \,.
\end{equation}
\subsubsection{\textbf{MR}} (Mean Rank $\downarrow$) computes the arithmetic mean over all individual ranks which is similar to MRR:
\begin{equation}
    % \vspace{-2pt}
    \mathbf{MR}=\frac{1}{|\mathcal{S}_{te}|} \sum_{i=1}^{|\mathcal{S}_{te}|} {\text {rank}_i} \,. 
\end{equation}
Note that MR is sensitive to any model performance changes, not only changes that occur below a certain cutoff and therefore reflects the average performance.

\subsection{Future Work \& Discussion}
Knowledge Graphs (KGs) have been empirically validated to provide substantial benefits in a multitude of downstream applications. They serve as significant sources of knowledge supplementation and data augmentation for diverse tasks including, but not limited to, Question Answering \cite{DBLP:conf/semweb/0007CGPYC21,DBLP:conf/jist/0007HCGFP0Z22}, Zero-shot Learning \cite{DBLP:journals/pieee/ChenGCPHZHC23,DBLP:conf/ijcai/ChenG0HPC21,chen2023duet,DBLP:conf/www/GengC0PYYJC21}, and AI4Science \cite{fang2023knowledge,DBLP:conf/aaai/FangZYZD0Q0FC22}.

Despite these advancements, the application of Multi-modal Knowledge Graphs (MMKGs) to such tasks remains relatively unexplored. One plausible reason for this gap is the inherent uncertainty, ambiguity, and occasional missing phenomena associated with various modalities in MMKGs, a challenge particularly prominent within the visual modality, as examined in this paper.

Our objective with this research is to stimulate further academic discourse and exploration in the direction of Multi-modal Entity Alignment (MMEA). We anticipate more scholarly endeavors focusing on MMKG-driven downstream tasks, and we eagerly look forward to the comprehensive understanding and exploitation of the untapped potential of multi-modal KGs within the Semantic Web community.

Moreover, there remain opportunities for future research related to this work, such as evaluating our techniques in the context of incompleteness in other modalities (e.g., attribute), and investigating effective techniques to  utilize more detailed visual contents for MMEA. There is  potential for enhancing UMAEA’s efficiency, we also view this as a direction for future research which has not been explored in depth.



\end{document}
