\section{Related Work}
\label{sec:Related Work}
\subsection{Invisible Information Hiding}
Invisible information hiding can be broadly classified into two main categories: steganography and digital watermarking. Steganography finds widespread use in the field of information security, with its primary objective being to ensure that information is accessible only to the intended recipient while remaining concealed from unauthorized individuals. Steganography can be further divided into two classes: spatial domain and transfer domain. The classical spatial domain method is Least Significant Bit (LSB), where the hidden message replaces the least significant bits of the cover image. However, this approach alters the statistical properties of the cover image, making it easily detectable by steganalysis. As a result, simple LSB steganography is ineffective in practical applications. On the other hand, steganography in the transfer domain~\cite{marvel1999spread,johnson1998exploring,wang2016rate, lu2020secure, li2019jpeg, li2019shortening, zhang2016decomposing} leverages the statistical characteristics of the image to conceal information. In recent years, numerous steganographic techniques based on CNN have been introduced. The impressive non-linear fitting capability of CNN allows for embedding and extracting information without the need for intricate manual feature extractions. Some well-established methods, such as SSGAN~\cite{shi2018ssgan} and ASDL-GAN\cite{tang2017automatic}, have proposed modifications to the redundant information of the cover image to hide the desired information. \cite{guo2023hierarchical,liu2022pscc} use CNN to realize image forgery detection and localization.

Digital watermarking is a crucial aspect of information hiding, involving the insertion of concise messages into images to protect copyright and assert authorship. Unlike steganography, digital watermarking requires a high level of robustness against transmission distortion. Traditional digital watermarking methods can be classified into two categories: spatial~\cite{karybali2006efficient, pereira2001optimal, kim2003invariant, yang2021high} and transfer~\cite{6486549,birney1995modeling, hernandez2000dct, cheng2003robust, barni2001new, zheng2003rst, xiang2008invariant, wang2023udtcwt, huang2023robust} domain approaches.

Zhu \textit{et al.}\cite{Zhu_2018_ECCV} presented an innovative approach for achieving robust image watermarking through adversarial learning. Their pioneering work demonstrated robustness against various distortions, including Gaussian blurring, pixelwise dropout, cropping, and JPEG compression. Building upon Zhu's framework, Tancik \textit{et al.}\cite{tancik2020stegastamp} introduced Stegastamp, which incorporated shooting noise to achieve robust watermarking for shooting screen and printed images. Jia \textit{et al.}~\cite{jia2020rihoop} proposed RIHOOP, utilizing differentiable 3-D rendering operations to simulate distortions resulting from camera imaging. However, unlike the aforementioned methods that embed the watermarking in RGB format, our method embeds the watermarking in RAW format and extracts it from RGB format. This distinction enables us to address specific challenges related to RAW images and attain effective copyright protection in diverse scenarios.

\subsection{Image Signal Processing Pipeline}
The Image Signal Processing (ISP) pipeline in a camera is utilized to transform RAW images captured by the camera sensor into RGB images that are perceptually optimized for the Human Visual System (HVS). To achieve exceptional visual quality, the Camera ISP pipeline consists of various modules, such as demosaicing, white balance, color correction, color mapping, gamma correction, image enhancement, noise reduction, and sharpening. However, many of these sub-modules are non-differentiable, meaning that they do not allow for the backpropagation of gradients through them. This non-differentiability poses a challenge when attempting to train the complete ISP pipeline end-to-end in a neural network, hindering the optimization process. As a result, effectively incorporating the ISP pipeline into the neural network architecture requires specialized techniques to overcome these non-differentiable components and ensure smooth training and optimization.

In contrast to the traditional ISP pipeline, where each sub-module is treated separately, the deep ISP pipeline operates on RAW images to produce RGB images using a deep neural network. Recent methodologies~\cite{C5,8259342,10.1145/2980179.2982399} based on CNN have demonstrated remarkable advancements in various ISP tasks, showcasing the superiority of CNN in this domain. Consequently, using a CNN instead of the entire ISP pipeline is feasible. Many efforts have been made in recent years to train deep networks to learn the ISP pipeline. Schwartz \textit{et al.}\cite{8478390} created a dataset containing RAW images and their corresponding RGB images and proposed the DeepISP model. This model establishes a mapping between RAW low-light images and well-lit processed RGB images. CameraNet\cite{9329084} comprises two distinct CNN modules designed to address two sets of relatively uncorrelated subtasks in an ISP pipeline: restoration and enhancement. Ignatov \textit{et al.}\cite{ignatov2020replacing} introduced an inverted pyramidal architecture named PyNET, capable of processing images at five distinct levels, to learn a diverse set of features at each level. They also collected a dataset containing paired RAW and RGB images, which was subsequently utilized in two challenges\cite{9022218,ignatov2020aim}. The top-performing methods in these challenges were MW-ISPNet~\cite{ignatov2020aim} and AWNet~\cite{dai2020awnet}, both using a Discrete Wavelet Transform (DWT)-based decomposition to replace upsampling and downsampling operations. MW-ISPNet integrates MWCNN~\cite{8575273} with RCAN~cite{zhang2018rcan} models, while AWNet employs an attention mechanism. Zhang \textit{et al.}~\cite{RAW-to-sRGB} introduced a light ISP network that builds upon the MW-ISPNet architecture and incorporates image alignment during training. This image alignment has led to the current state-of-the-art performance of the light ISP network.

The proposed RAWIW framework incorporates a deep ISP pipeline that represents the traditional ISP pipeline in a differentiable manner. This deep ISP pipeline is constructed using CNN, enabling end-to-end training of the complete framework.By utilizing a deep ISP pipeline, the proposed framework achieves a superior balance between accuracy and computational efficiency. The differentiable nature of the deep ISP pipeline enables efficient backpropagation of gradients throughout the entire framework, facilitating the optimization process and enhancing training effectiveness. In summary, the adoption of a deep ISP pipeline in the RAWIW framework not only enables end-to-end training but also improves the overall performance by efficiently managing computational resources and optimizing the training process.
