\section{Introduction}

% Figure environment removed

\IEEEPARstart{I}{n} the era of information, safeguarding intellectual property is vital, especially for multimedia resources on the Internet, such as recordings, photographs, and videos. Among these, photographs are widespread and require significant copyright protection. Invisible image watermarking plays a crucial role in the field of information security by securing the copyrights of digital images and tracing unauthorized disclosures. This technique involves creating a watermarked image that appears similar to the original image but contains unique copyright information. At the same time, the hidden watermarking must be robust enough to withstand noises and distortions that may occur during transmission, such as Gaussian noise and JPEG compression. A reliable watermarking algorithm should ensure that the hidden information remains detectable even after encountering these challenges. Traditionally, existing watermarking algorithms have focused on manipulating RGB images and hiding information in spatial or transform domains. However, professional photographers nowadays prefer working with RAW images to achieve more satisfactory results, as RAW images offer greater flexibility and ease of manipulation. Unlike RGB images, RAW images store 10-16 bits of unprocessed scene radiance, capturing a higher dynamic range and providing more room for post-processing. The popularity of smartphones like Huawei P20, iPhone 13 Pro, and Samsung Galaxy S22, which support capturing RAW images, indicates a potential surge in the number of RAW images in the near future. Moreover, RAW images are akin to works of art and should not be distributed without proper authorization. Publishers often receive RAW images and use them to generate different RGB images. As a result, applying watermarking algorithms to RAW images is critical for copyright protection but has not received sufficient attention in previous works. In conclusion, with the increasing prevalence of multimedia resources, particularly photographs, on the Internet, the importance of protecting intellectual property through invisible image watermarking cannot be overstated. Paying attention to RAW images and developing effective watermarking techniques for them is crucial in ensuring copyright protection in this digital age. 

Unlike RGB images, the copyright protection of RAW images faces significant challenges. Applying watermarking directly to RGB images poses the risk of leaking the original RAW image, and the leaked RAW image can generate unwatermarked RGB images, thereby failing to achieve the goal of copyright protection. On the other hand, attempting to decode the watermarking from RAW images presents difficulties since RAW images are not readily accessible in most cases. The existing invisible image watermarking methods typically hide and extract information within the same image format, specifically in RGB color space. If we were to apply these existing methods directly to RAW images, extracting the watermarking from RAW images would be impractical due to limited accessibility. Additionally, a RAW image can be transformed into different RGB images by various Image Signal Processing (ISP) pipelines. Hence, the process of RAW image watermarking involves encoding the watermarking into a RAW image and subsequently decoding the watermarking from the corresponding RGB images after undergoing various ISP pipelines. This multistage process aims to ensure copyright protection while accounting for the different transformations that the RAW image undergoes during image processing. In summary, protecting the copyright of RAW images requires overcoming various challenges, such as preventing leakage of the original RAW image and handling the transformation of RAW images to different RGB images by various ISP pipelines. Existing watermarking methods designed for RGB images are not directly applicable to RAW images due to the differences in accessibility and image format, emphasizing the need for specialized techniques for RAW image watermarking.

To address the aforementioned challenges, Meerwald \textit{et al.}\cite{meerwald2009watermarking} proposed frequency domain transform based RAW image watermarking. This paper introduces the pioneering deep learning watermarking-based copyright protection framework for RAW images, called \underline{RAW} \underline{I}mage \underline{W}atermarking (RAWIW). RAWIW utilizes Convolutional Neural Networks (CNN) for embedding and detecting the watermarking. The comparison between our RAWIW and the RGB watermarking method is illustrated in Fig \ref{figintro}. As discussed earlier, our proposed RAWIW encodes the watermarking information into the RAW image before applying the Image Signal Processing (ISP) pipeline. This approach enhances encoding efficiency by applying watermarking to the RAW image to undergo different retouching methods (\textit{ISP pipelines}), producing RGB images with diverse color hues and tones. The watermark decoder can then extract the watermarking information from these RGB images, irrespective of the employed retouching. In contrast, RGB watermarking necessitates encoding the watermarking information into each RGB image generated by distinct retouching methods. This process is time-consuming and computationally intensive, as each RGB image must be processed individually. By embedding the watermarking into RAW images before applying ISP pipelines, our framework streamlines the encoding process and enhances encoding efficiency. Furthermore, our method enhances decoding robustness by allowing the decoder to adapt to different retouching methods through flexible changes during the training pipeline. Consequently, our proposed method provides an effective solution for safeguarding the copyright of RAW images.

The RAWIW framework comprises five essential modules: encoder, decoder, discriminator, distortion network, and deep ISP pipeline. Within this framework, the encoder embeds watermarking information into a RAW image, while the decoder retrieves the hidden information from the corresponding RGB image. To ensure the encoded information's robustness during the transfer from RAW to RGB format, a deep differentiable ISP pipeline~\cite{ignatov2020aim,ronneberger2015u,dai2020awnet} is integrated, simulating camera image processing. Additionally, we incorporate a distortion network after the ISP module to simulate the distortions encountered during transmission, further enhancing the robustness of the watermarking against these distortions. The discriminator and encoder modules are trained adversarially to improve the concealment of the watermarking. Moreover, we employ an effective three-stage training strategy to strike a balance between the robustness and concealment of the watermarking. The framework has undergone evaluations using two datasets of RAW images, demonstrating outstanding performance in terms of decoding accuracy and visual quality. Overall, this paper presents a comprehensive approach for adding watermarking to RAW images, applicable in various scenarios, such as digital rights management and copyright protection. This robust and efficient framework can significantly contribute to safeguarding the copyright of RAW images in diverse applications.

The contributions of this paper are threefold:
\begin{itemize}

\item
{
To the best of our knowledge, this paper presents the \textbf{first deep learning-based cross-domain RAW image watermarking method named RAWIW} which can encode watermarking to RAW images while decoding watermarking from the corresponding retouched RGB images. This method can achieve the protection of copyright and ownership for RAW images.
}

\item {
We are \textbf{the first to propose a RAW image encoder that considers Bayer patterns and a distortion network that simulates the gap of different ISP pipelines and the distortion of transmission} while we use an effective three-stage training strategy for our method, which can achieve a good trade-off between robustness and concealment of watermarking.
}

\item{
Extensive experiments demonstrate that the proposed method has good concealment while being robust to different ISP pipelines.
}

\end{itemize}

The paper is structured as follows. Section \ref{sec:Related Work} provides an overview of the background and related work related to the proposed method. In Section \ref{sec:Method}, we elaborate on our proposed watermarking method in detail. Next, Section \ref{sec:Experiments} outlines the experimental setup and showcases the results obtained from our experiments. In Section \ref{sec:Limitations} and Section \ref{sec:Conclusion}, we discuss the advantages and limitations of the proposed method.