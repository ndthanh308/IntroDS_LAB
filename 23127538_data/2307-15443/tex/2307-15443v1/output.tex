\documentclass[lettersize,journal]{IEEEtran}
\usepackage{amsmath,amsfonts}
\usepackage{algorithmic}
\usepackage{algorithm}
\usepackage{array}
\usepackage[caption=false,font=normalsize,labelfont=sf,textfont=sf]{subfig}
\usepackage{textcomp}
\usepackage{stfloats}
\usepackage{url}
\usepackage{verbatim}
\usepackage{graphicx}
\usepackage{cite}

\usepackage{multirow}
\usepackage{overpic}
\usepackage{bbding}
\usepackage{booktabs}
\usepackage{xcolor}

\hyphenation{op-tical net-works semi-conduc-tor IEEE-Xplore}
% updated with editorial comments 8/9/2021

\begin{document}

\title{RAWIW: RAW Image Watermarking Robust to ISP Pipeline}

\author{Kang Fu, Xiaohong Liu, Jun Jia, Zicheng Zhang, Yicong Peng,\\ Jia Wang, and Guangtao Zhai, \emph{Senior Member, IEEE} % <-this % stops a space
\IEEEcompsocitemizethanks{\IEEEcompsocthanksitem Kang Fu, Jun Jia, Zicheng Zhang, Yicong Peng, Jia Wang, and Guangtao Zhai are with the Institute of Image Communication and Network Engineering Shanghai Jiao Tong University, 200240 Shanghai, China. E-mail:\{fuk20-20, jiajun0302, zzc1998, jack-sparrow, jiawang, zhaiguangtao\}@sjtu.edu.cn.\protect  \quad Xiaohong Liu is with John Hopcroft Center, Shanghai Jiao Tong University, Shanghai 200240, China. E-mail: xiaohongliu@sjtu.edu.cn.}
        % <-this % stops a space
}% <-this % stops a space

\maketitle

\begin{abstract}
Invisible image watermarking is essential for image copyright protection. Compared to RGB images, RAW format images use a higher dynamic range to capture the radiometric characteristics of the camera sensor, providing greater flexibility in post-processing and retouching. Similar to the master recording in the music industry, RAW images are considered the original format for distribution and image production, thus requiring copyright protection. Existing watermarking methods typically target RGB images, leaving a gap for RAW images. To address this issue, we propose the first deep learning-based \underline{RAW} \underline{I}mage \underline{W}atermarking (RAWIW) framework for copyright protection. Unlike RGB image watermarking, our method achieves cross-domain copyright protection. We directly embed copyright information into RAW images, which can be later extracted from the corresponding RGB images generated by different post-processing methods. To achieve end-to-end training of the framework, we integrate a neural network that simulates the ISP pipeline to handle the RAW-to-RGB conversion process. To further validate the generalization of our framework to traditional ISP pipelines and its robustness to transmission distortion, we adopt a distortion network. This network simulates various types of noises introduced during the traditional ISP pipeline and transmission. Furthermore, we employ a three-stage training strategy to strike a balance between robustness and concealment of watermarking. Our extensive experiments demonstrate that RAWIW successfully achieves cross-domain copyright protection for RAW images while maintaining their visual quality and robustness to ISP pipeline distortions.
\end{abstract}

\begin{IEEEkeywords}
Robust watermarking, Image signal processing, Camera pipeline, Neural networks
\end{IEEEkeywords}

% \section{Introduction}
% \IEEEPARstart{T}{his} file is intended to serve as a ``sample article file''
% for IEEE journal papers produced under \LaTeX\ using
% IEEEtran.cls version 1.8b and later. The most common elements are covered in the simplified and updated instructions in ``New\_IEEEtran\_how-to.pdf''. For less common elements you can refer back to the original ``IEEEtran\_HOWTO.pdf''. It is assumed that the reader has a basic working knowledge of \LaTeX. Those who are new to \LaTeX \ are encouraged to read Tobias Oetiker's ``The Not So Short Introduction to \LaTeX ,'' available at: \url{http://tug.ctan.org/info/lshort/english/lshort.pdf} which provides an overview of working with \LaTeX.
The problem of the presence or absence of phase transition is central in statistical mechanics. To prove the existence of phase transition, the standard idea is to define a notion of contour and use \textit{Peierls' argument} \cite{Peierls.1936}. In the usual Ising model \cite{Ising_25}, particles of the system interact only with their nearest-neighbors. On ferromagnetic long-range Ising models \cite{Anderson_Yuval_69}, there is interaction between each pair of spins in the lattice. The Hamiltonian of the model is given formally by
\begin{equation*}
    H(\sigma) = - \sum_{x,y\in \Z^d}J_{xy}\sigma_x\sigma_y,
\end{equation*}
where $J_{xy}=J|x-y|^{-\alpha}$, $J>0$, $\alpha > d$. It is well-known that the Peierls' argument in dimension 2 implies phase transition for Ising models with nearest-neighbors or long-range interactions when $d\geq 2$, using correlation inequalities. For the unidimensional lattice, it was known that short-range models do not present phase transition. In the long-range case, a different behavior was expected depending on the exponent $\alpha$ (see \cite{Kac_Thompson_69}), but the problem was challenging since contours were first created as multidimensional objects.

In dimension $d=1$, phase transition was proved first in 1969 by Dyson \cite{Dyson.69}, for $\alpha \in (1,2)$, by proving phase transition in an auxiliary model and then using correlation inequalities. In 1982, Fr{\"o}hlich and Spencer \cite{Frohlich.Spencer.82} introduced a notion of one-dimensional contours and then applied the Peierls' argument to show phase transition for the critical value $\alpha = 2$. These contours were inspired by the multiscale techniques previously introduced to study the Berezinskii-Kosterlitz-Thouless transition in two-dimensional continuous spin systems \cite{FS81}. Later, Cassandro, Ferrari, Merola and Presutti  \cite{Cassandro.05} extended the contour argument previously available for $\alpha=2$ to exponents $\alpha\in (3-\frac{\ln 3}{\ln 2}, 2)$, with the additional restriction that the nearest-neighbor interaction is strong, i.e.,  ${J(1)\gg 1}$; this restriction was removed for a subclass of interactions in \cite{Bissacot.Endo.18}. Further results were obtained using contour arguments, such as the decay of correlations, cluster expansions, phase transition with random interactions, etc; some references with these results are \cite{ Cassandro.Merola.Picco.17, Cassandro.Merola.Picco.Rozikov.14, Imbrie.82, Imbrie.Newman.88, Johansson.91}. 

In the multidimensional setting ($d\geq 2$), Ginibre, Grossmann, and Ruelle, in \cite{Ginibre.Grossmann.Ruelle.66}, proved the phase transition for $\alpha > d+1$, using an enhanced version of Peierls' argument and the usual contours. Park proposed a different notion of contour for long-range systems in \cite{Park.88.I, Park.88.II}, extending the Pirogov-Sinai theory available for short-range interactions assuming $\alpha > 3d+1$, although he can also consider Potts models with his methods. Some results in the literature suggest that truly long-range effects appear only when $d < \alpha \leq d+1$, see for instance, \cite{Biskup_Chayes_Kivelson_07}. Recently, Affonso, Bissacot, Endo and Handa \cite{Affonso.2021}, inspired by the ideas from Fr{\"o}hlich and Spencer in \cite{FS81, Frohlich.Spencer.82}, introduced a version of multiscale multidimensional contour and proved phase transition by a contour argument in the whole region $\alpha > d$. They can consider long-range Ising models with deterministic decaying fields, first introduced in the context of nearest-neighbor interactions in \cite{Bissacot_Cioletti_10}. For these models, the lack of analyticity of the free energy does not imply phase transition since these models have the same free energy as the models with zero field. It is expected that fields decaying slowly imply uniqueness. In this setting, a contour argument is useful for proofs of phase transitions as well for uniqueness, some papers with models with deterministic decaying fields are \cite{Aoun_Ott_Velenik_23, Bissacot_Cass_Cio_Pres_15, Bissacot.Endo.18, Cioletti_Vila_2016}.

The Random Field Ising model (RFIM) \cite{Imry.Ma.75} is the nearest-neighbor Ising model with an additional external field acting on each site $(h_x)_{x\in\Z^d}$ that is a family of i.i.d. Gaussian random variable with mean 0 and variance 1. Formally, the Hamiltonian of the model is given by
\begin{equation*}
    H(\sigma) = - \sum_{\substack{x,y\in \Z^d \\|x-y|=1}}J\sigma_x\sigma_y  - \varepsilon\sum_{x\in\Z^d}h_x\sigma_x,
\end{equation*}
where $J>0$, $\varepsilon>0$, $\alpha > d$ and $d \geq 1$. A detailed account of the history of the phase transition problem for this model, as well as detailed proofs, was given in \cite{Bovier.06}. Here we present a brief overview.

During the 1980s, the question of the specific dimension where phase transition for the RFIM should happen attracted much attention and was a topic of heated debate. Two convincing arguments were dividing the physics community. One of them, due to Imry and Ma \cite{Imry.Ma.75}, was a non-rigorous application of the Peierls' argument together with the use of the isoperimetric inequality. The key idea of Peierls' argument is to define a notion of contour and calculate the energy cost of "erasing" each contour, i.e., the energy cost of flipping all spins inside the contour. When there is no external field, that energy necessary to flip the spins in a region $A\subset \Z^d$ is of the order of the boundary $|\partial A|$. When we add an external field, we get an extra cost depending on this field. Imry and Ma argued that this cost should be approximately $\sqrt{|A|}$, which is smaller than $|\partial A|$ for all regions only when $d\geq 3$, so this should be the region where phase transition occurs. The other argument, due to Parisi and Sourlas \cite{Parisi.Sourlas.79}, based on dimensional reduction, predicted that the $d$-dimensional RFIM would behave like the $d-2$-dimensional nearest-neighbor Ising model, therefore presenting phase transition only when $d\geq 4$. 

The question was settled by two celebrated papers showing that Imry and Ma's prediction was correct. First, in 1988, Bricmont and Kupiainen \cite{Bricmont.Kupiainen.88} showed that there is phase transition almost surely in $d\geq3$, for low temperatures and variance $\varepsilon$ small enough. Their proof uses a rigorous renormalization group analysis for the short-range case and it is considered involved. Still, they claimed that the result works for any model with a suitable contour representation and centered sub-gaussian external field. Later on, Aizenman and Wehr \cite{Aizenman.Wehr.90} proved uniqueness for $d\leq 2$. For detailed proofs of these results, we refer the reader to \cite{Bovier.06} (see also \cite{Berretti.85, Camia.18, Frohlich.Imbre.84,  Klein.Masooman.97} for more uniqueness results). 

Recently, Ding and Zhuang, see \cite{Ding2021}, provided a simpler proof of the phase transition, not using RGM. And in  \cite{Ding.Liu.Xia.22}, Ding, Liu and Xia proved that if $\beta_c(d)$ is the critical inverse of the temperature of the Ising model with no field, for all $\beta>\beta_c(d)$ there exists a critical value $\varepsilon_0(d, \beta)$ such that the RFIM with $\varepsilon \leq \varepsilon_0$ presents phase transition. 

In the present paper, we are considering a long-range Ising model with a random field, whose Hamiltonian is given formally by
\begin{equation*}
    H(\sigma) = - \sum_{x,y\in \Z^d}J_{xy}\sigma_x\sigma_y - \varepsilon\sum_{x\in\Z^d}h_x\sigma_x,
\end{equation*}
where $J_{xy}=J|x-y|^{-\alpha}$, $J, \varepsilon>0$, $\alpha > d$ and $h_x\in\mathbb{R}$, $d\geq 3$.
Until now, the only known result in the long-range setting is for the one-dimensional long-range Ising model with a random field, by Cassandro, Orlandi, and Picco \cite{Cassandro.Picco.09}. They used the contours of \cite{Cassandro.05} to show the phase transition for the model when $\alpha\in (3-\frac{\ln 3}{\ln 2}, \frac{3}{2})$, under the assumption $J(1) \gg 1$. We stress that, as remarked by Aizenman, Greenblatt, and Lebowitz \cite{Aizenman_Greenblatt_Lebowitz_2012}, although their argument does not work for the whole region for the exponent $\alpha$, the phase transition holds for values close to the critical value $\alpha=3/2$, since by the Aizenman-Wehr theorem we know that there is uniqueness for $\alpha>3/2$.

The argument from Ding and Zhuang in \cite{Ding2021}, for $d\geq3$, involves controlling the probability of a bad event, which is closely related to controlling the quantity $$\sup_{\substack{0\in A\subset\Z^d \\ A \text{ connected }}}\frac{\sum_{x\in A}h_x}{|\partial A|},$$ known as the greedy animal lattice normalized by the boundary. The greedy animal lattice normalized by the size, instead of the boundary, was extensively studied for general distributions of $(h_x)_{x\in\Z^d}$, see \cite{Cox_Gandolfi_Griffin_Kesten_93, Gandolfi_Kesten_94, Hammond_06, Martin_02}. When we normalize by the boundary, an argument by Fisher, Fr\"{o}hlich and Spencer \cite{FFS84} shows that the expected value of the greedy animal lattice is constant. In dimension $d=2$, the expected value is not finite, see \cite{Ding.Wirth.20}. The supremum is taken over connected regions containing the origin since the interiors of the usual Peierls contours are of this form.


For the long-range model, the interior of contours is not necessarily connected. In fact, long-range contours may have considerably large diameters with respect to their size, so their interiors can be very sparse. To avoid this, we define contours, strongly inspired by the $(M,a,r)$-partition in \cite{Affonso.2021}, using a multiscaled procedure that assures that the contours have no cluster with small density.  With them, we generalize the arguments by Fisher-Fr\"{o}hlich-Spencer \cite{FFS84}, and prove that the expected value of the greedy animal lattice is constant, even considering regions not necessarily connected in the supremum. Then, we prove the phase transition for $d\geq 3$. The main result of this paper is the following.
\begin{theorem*}Given $d\geq 3$, $\alpha>d$, there exists $\beta_c\coloneqq\beta(d, \alpha)$ and $\varepsilon_c\coloneqq\varepsilon(d, \alpha)$ such that, for $\beta >\beta_c$ and $\varepsilon\leq \varepsilon_c$, the extremal Gibbs measures $\mu_{\beta, \varepsilon}^+$ and $\mu_{\beta, \varepsilon}^-$ are distinct, that is, $\mu_{\beta, \varepsilon}^+ \neq \mu_{\beta, \varepsilon}^-$ $\mathbb{P}$-almost surely. Therefore the long-range random field Ising model presents phase transition.
\end{theorem*}

This paper is divided as follows. In Section 2, we define the model and the contours, and suitable generalizations to the constructions in \cite{Affonso.2021} are introduced.  In Section 3, we define two bad events of the external field and prove that they occur with a small probability.  In Section 4, we present the proof of the phase transition.

\section{Related Work}
\label{sec:Related Work}
\subsection{Invisible Information Hiding}
Invisible information hiding can be broadly classified into two main categories: steganography and digital watermarking. Steganography finds widespread use in the field of information security, with its primary objective being to ensure that information is accessible only to the intended recipient while remaining concealed from unauthorized individuals. Steganography can be further divided into two classes: spatial domain and transfer domain. The classical spatial domain method is Least Significant Bit (LSB), where the hidden message replaces the least significant bits of the cover image. However, this approach alters the statistical properties of the cover image, making it easily detectable by steganalysis. As a result, simple LSB steganography is ineffective in practical applications. On the other hand, steganography in the transfer domain~\cite{marvel1999spread,johnson1998exploring,wang2016rate, lu2020secure, li2019jpeg, li2019shortening, zhang2016decomposing} leverages the statistical characteristics of the image to conceal information. In recent years, numerous steganographic techniques based on CNN have been introduced. The impressive non-linear fitting capability of CNN allows for embedding and extracting information without the need for intricate manual feature extractions. Some well-established methods, such as SSGAN~\cite{shi2018ssgan} and ASDL-GAN\cite{tang2017automatic}, have proposed modifications to the redundant information of the cover image to hide the desired information. \cite{guo2023hierarchical,liu2022pscc} use CNN to realize image forgery detection and localization.

Digital watermarking is a crucial aspect of information hiding, involving the insertion of concise messages into images to protect copyright and assert authorship. Unlike steganography, digital watermarking requires a high level of robustness against transmission distortion. Traditional digital watermarking methods can be classified into two categories: spatial~\cite{karybali2006efficient, pereira2001optimal, kim2003invariant, yang2021high} and transfer~\cite{6486549,birney1995modeling, hernandez2000dct, cheng2003robust, barni2001new, zheng2003rst, xiang2008invariant, wang2023udtcwt, huang2023robust} domain approaches.

Zhu \textit{et al.}\cite{Zhu_2018_ECCV} presented an innovative approach for achieving robust image watermarking through adversarial learning. Their pioneering work demonstrated robustness against various distortions, including Gaussian blurring, pixelwise dropout, cropping, and JPEG compression. Building upon Zhu's framework, Tancik \textit{et al.}\cite{tancik2020stegastamp} introduced Stegastamp, which incorporated shooting noise to achieve robust watermarking for shooting screen and printed images. Jia \textit{et al.}~\cite{jia2020rihoop} proposed RIHOOP, utilizing differentiable 3-D rendering operations to simulate distortions resulting from camera imaging. However, unlike the aforementioned methods that embed the watermarking in RGB format, our method embeds the watermarking in RAW format and extracts it from RGB format. This distinction enables us to address specific challenges related to RAW images and attain effective copyright protection in diverse scenarios.

\subsection{Image Signal Processing Pipeline}
The Image Signal Processing (ISP) pipeline in a camera is utilized to transform RAW images captured by the camera sensor into RGB images that are perceptually optimized for the Human Visual System (HVS). To achieve exceptional visual quality, the Camera ISP pipeline consists of various modules, such as demosaicing, white balance, color correction, color mapping, gamma correction, image enhancement, noise reduction, and sharpening. However, many of these sub-modules are non-differentiable, meaning that they do not allow for the backpropagation of gradients through them. This non-differentiability poses a challenge when attempting to train the complete ISP pipeline end-to-end in a neural network, hindering the optimization process. As a result, effectively incorporating the ISP pipeline into the neural network architecture requires specialized techniques to overcome these non-differentiable components and ensure smooth training and optimization.

In contrast to the traditional ISP pipeline, where each sub-module is treated separately, the deep ISP pipeline operates on RAW images to produce RGB images using a deep neural network. Recent methodologies~\cite{C5,8259342,10.1145/2980179.2982399} based on CNN have demonstrated remarkable advancements in various ISP tasks, showcasing the superiority of CNN in this domain. Consequently, using a CNN instead of the entire ISP pipeline is feasible. Many efforts have been made in recent years to train deep networks to learn the ISP pipeline. Schwartz \textit{et al.}\cite{8478390} created a dataset containing RAW images and their corresponding RGB images and proposed the DeepISP model. This model establishes a mapping between RAW low-light images and well-lit processed RGB images. CameraNet\cite{9329084} comprises two distinct CNN modules designed to address two sets of relatively uncorrelated subtasks in an ISP pipeline: restoration and enhancement. Ignatov \textit{et al.}\cite{ignatov2020replacing} introduced an inverted pyramidal architecture named PyNET, capable of processing images at five distinct levels, to learn a diverse set of features at each level. They also collected a dataset containing paired RAW and RGB images, which was subsequently utilized in two challenges\cite{9022218,ignatov2020aim}. The top-performing methods in these challenges were MW-ISPNet~\cite{ignatov2020aim} and AWNet~\cite{dai2020awnet}, both using a Discrete Wavelet Transform (DWT)-based decomposition to replace upsampling and downsampling operations. MW-ISPNet integrates MWCNN~\cite{8575273} with RCAN~cite{zhang2018rcan} models, while AWNet employs an attention mechanism. Zhang \textit{et al.}~\cite{RAW-to-sRGB} introduced a light ISP network that builds upon the MW-ISPNet architecture and incorporates image alignment during training. This image alignment has led to the current state-of-the-art performance of the light ISP network.

The proposed RAWIW framework incorporates a deep ISP pipeline that represents the traditional ISP pipeline in a differentiable manner. This deep ISP pipeline is constructed using CNN, enabling end-to-end training of the complete framework.By utilizing a deep ISP pipeline, the proposed framework achieves a superior balance between accuracy and computational efficiency. The differentiable nature of the deep ISP pipeline enables efficient backpropagation of gradients throughout the entire framework, facilitating the optimization process and enhancing training effectiveness. In summary, the adoption of a deep ISP pipeline in the RAWIW framework not only enables end-to-end training but also improves the overall performance by efficiently managing computational resources and optimizing the training process.


While it is not tractable to directly compute the integral of a function represented by a neural network, it is straightforward to take the analytical derivative. In this paper, we leverage the fundamental theorem of calculus in order to implicitly learn the integral of a function.

Suppose we wish to learn some function $f: \mathbb{R}^n \mapsto \mathbb{R}$. Instead of directly parametrising $f$, we represent it implicitly by parametrising its indefinite integral $F_\theta$ with a neural network:

\begin{equation}
    F_\theta(\vec{x}) = \int \int \cdots \int f(\vec{x}) \; dx_1 dx_2 \ldots dx_n
\end{equation}

Note that as $f$ is defined implicitly as a function of $F_\theta$, this is not an approximation---it is the \textit{exact} analytical integral. In order to solve for $f$, we must differentiate $F_\theta$:

\begin{equation}
    f(\vec{x}) = \frac{\partial}{\partial x_1} \frac{\partial}{\partial x_2} \cdots \frac{\partial}{\partial x_n} F_\theta(\vec{x})
    \label{eq:f_F}
\end{equation}

Although we parametrise its integral $F_\theta$, the function we wish to learn is $f$. Consequently, during the learning process, the loss is applied directly to $f$:

\begin{equation}
    \mathcal{L} = \mathbb{E}\left[ (y - f(\vec{x}))^2 \right]
\end{equation}

\input{schema/eps-fig}

\subsection{Integral Constraints}

Since $f$ is defined as a function of $F_\theta$, we can apply constraints directly to its integral. For example, by using an equality constraint, we can define the class of functions $f$ that integrate to a given value $\epsilon$ over a given domain $\mathcal{D}$. The same principle can be utilised to apply inequality constraints or transformations.

We start by considering rectangular domains defined by intervals $[a_i,b_i]$ for $i \in [1..n]$. The definite integral of $f$ over rectangular domain $\mathcal{D}$ is defined:

\begin{equation}
    F_\theta \Big\vert_\mathcal{D} = \sum_{p_1 \in [a_1, b_1]} \sum_{p_2 \in [a_2,b_2]} \cdots \sum_{p_n \in [a_n,b_n]} (-1)^{^{\sum \mathbbm{1}(p_i=a_i)}} \cdot F_\theta(\langle p_1, p_2, \ldots, p_n \rangle)
    \label{eq:int_eval}
\end{equation}

That is, in order to calculate the definite integral over an $n$-dimensional box, we must evaluate all of its vertices. This is because evaluating $F_\theta$ at a point will yield the ``area'' from negative infinity up to that point, so multiple points must be evaluated to determine the area of a finite region. The $(-1)$ exponent in the equation determines which regions must be subtracted and which must be added in order to determine that area (Figure \ref{fig:int_eval}).

% Figure environment removed

In order to parametrise the class of functions which integrate to $\epsilon$, we start by defining $F'_\theta$, the integral of some unconstrained function $f'$. Then, we define the integral $F_\theta$ of our constrained function $f$ by rescaling $F'_\theta$:

\begin{equation}
    F_\theta(\vec{x}) = \frac{\epsilon}{F'_\theta \big\vert_\mathcal{D}} F'_\theta(\vec{x})
\end{equation}

Since the term $\frac{\epsilon}{F'_\theta \big\vert_\mathcal{D}}$ is a scalar, we can move it inside the integral:

\begin{align}
    F_\theta(\vec{x}) &= \frac{\epsilon}{F'_\theta \big\vert_\mathcal{D}} F'_\theta(\vec{x}) \\
    F_\theta(\vec{x}) &= \frac{\epsilon}{F'_\theta \big\vert_\mathcal{D}} \int \int \cdots \int f'(\vec{x}) \; dx_1 dx_2 \ldots dx_n \\
    F_\theta(\vec{x}) &= \int \int \cdots \int \frac{\epsilon}{F'_\theta \big\vert_\mathcal{D}} f'(\vec{x}) \; dx_1 dx_2 \ldots dx_n \\
\end{align}

Therefore, we can write the constrained $f$ as a function of the unconstrained integral $F'_\theta$, which carries the learnable parameters:

\begin{align}
    f(\vec{x}) &= \frac{\epsilon}{F'_\theta \big\vert_\mathcal{D}} f'(\vec{x}) \\
    f(\vec{x}) &= \frac{\epsilon}{F'_\theta \big\vert_\mathcal{D}} \cdot \frac{\partial}{\partial x_1} \frac{\partial}{\partial x_2} \cdots \frac{\partial}{\partial x_n} F'_\theta(\vec{x})
\end{align}


% \subsection{Integration Over Arbitrary Domains}

% In the previous section we focused on the special case of rectangular domains (\textit{i.e.} where the limits of integration are constants). However, it is also possible to integrate over arbitrary domains by reparametrising with $u$-substitution.

% Consider a parametric function $\vec{x} = \vec{r}(\vec{u})$ which defines a transformation from euclidean space to some domain where the limits of integration are constant. Furthermore, as the indefinite integral depends on our choice of $\vec{r}$, we parametrise $F_\theta(\vec{u})$ instead of $F_\theta(\vec{x})$. Simplifying the notation of our iterated integral and applying this reparametrisation, our equation becomes:

% \begin{align}
%     F_\theta(\vec{u}) &= \int f(\vec{r}(\vec{u})) \; \lvert \nabla \vec{r}(\vec{u}) \rvert \, d\vec{u} \\
%     f(\vec{r}(\vec{u})) &= \frac{1}{\lvert \nabla \vec{r}(\vec{u}) \rvert} \cdot \frac{\partial}{\partial u_1} \frac{\partial}{\partial u_2} \cdots \frac{\partial}{\partial u_n} F_\theta(\vec{u}) \\
%     f(\vec{x}) &= \frac{1}{\lvert \nabla \vec{r}(\vec{u}) \rvert} \cdot \frac{\partial}{\partial u_1} \frac{\partial}{\partial u_2} \cdots \frac{\partial}{\partial u_n} F_\theta(\vec{r}^{-1}(\vec{x}))
% \end{align}

% In this formulation, we must select $\vec{r}$ according to our desired domain. For example, if we wish to integrate over the unit circle in $\mathbb{R}^2$, we can select $\langle x_1, x_2 \rangle = \vec{r}(\vec{u}) = \langle u_1 \cos(u_2), u_1 \sin(u_2) \rangle$. The differential after reparametrising with $\vec{r}$ is given by the determinant of the Jacobian $\lvert \nabla \vec{r}(\vec{u}) \rvert$. In this case, the differential is given by $|\nabla \langle u_1 \cos(u_2), u_1 \sin(u_2) \rangle| = \cos(u_2) \cdot u_1 \cos(u_2) - \sin(u_2) \cdot (-u_1 \sin(u_2)) = u_1$. 

\subsection{Positivity Constraint}

In many applications of FINN, it is necessary to constrain $f$ to be non-negative. This is useful in cases where $f$ is only defined in the positive domain (see \autoref{sec:Applications}).

Following from \autoref{eq:f_F}, in order to apply this constraint, we must ensure that the mixed partial of $F_\theta$ is non-negative. To do this, we define a new neural network layer to construct our multi-layer perceptron (MLP):

\begin{equation}
    \sigma_n \left( \lvert W \! \rvert \, \vec{x} + b \right)
\end{equation}

In this layer, we apply an absolute value to the weights (but not the bias), and we use a custom activation function $\sigma_n$, which is conditioned on the dimension of the input. Note that although we wish the mixed partial of $F_\theta$ to be non-negative, it is too constraining to restrict further derivatives to be non-negative as well. The derivative of our function $\dot{f}$ (with respect to any input dimension) should be able to represent positive \textit{or} negative values. To satisfy these criteria, we define the following activation function using the error function $\mathrm{erf}(x) = \frac{2}{\sqrt{\pi}} \int_0^x e^{-t^2} dt$:

\begin{equation}
    \sigma_n = \underbrace{\int \int \cdots \int}_{n-1} \frac{\mathrm{erf}(x)+1}{2} \; \underbrace{dx \cdots \, dx \, dx}_{n-1}
\end{equation}

For $n=1$, this simplifies to $\frac{\mathrm{erf}(x)+1}{2}$, which closely resembles sigmoid. For $n=2$, it resembles softplus, which is the integral of sigmoid. While they are similar, it is crucial that we use this custom activation instead of sigmoid, because the higher-order integrals of sigmoid evaluate to the polylogarithm, which has no closed-form solution. Conversely, all of the integrals of the error function have analytical solutions in terms of linear compositions of constants, power functions $x^k$, exponentials $e^{-x^2}$, and the error function itself $\mathrm{erf}(x)$ (all of which have efficient implementations). In practice, we use symbolic math to compute the integral once at initialisation time, and then each forward pass evaluates the resulting expression.


\section{Experimental Evaluations}\label{sec:experiment}

\textbf{Implementation.}
We implement \puma\ on top of SecretFlow~\citep{spu} in \textrm{C++} and Python. SecretFlow compiles a high-level Flax code to secure computation protocols, which are then executed by our designed cryptographic backends, and we encode the floating-ponit values as $64$-bit integers in ring $\mathbb{Z}_{2^{64}}$ with $18$-bit fractional part. 
Our experiments are run on 3 Alibaba Cloud ecs.g7.8xlarge servers with 32 vCPU and 128GB RAM each. The CPU model is Intel Xeon(Ice Lake) Platinum 8369B CPU @ 2.70GHz. We evaluate \puma\ on Ubuntu 20.04.6 LTS with Linux kernel 5.4.0-144-generic. Our bandwidth is about 5Gbps and round trip time is about 1ms. %\cheng{Describe fixed point parameters: scale, share bits.}

\textbf{Models \& Datasets.}
We evaluate \puma\ on seven NLP models: Bert-Base, Roberta-Base, and Bert-Large~\citep{bert}; GPT2-Base, GPT2-Medium, and GPT2-Large~\citep{gpt}; and LLaMA-7B~\citep{touvron2023llama}. We measure the Bert performance for three NLP tasks over the datasets of Corpus of Linguistic Acceptability (CoLA), Recognizing Textual Entailment (RTE), Stanford Question Answering Dataset (QNLI) from GLUE benchmarks~\citep{wang2018glue}, and GPT2 performance on Wikitext-103 V1~\citep{merity2016pointer}.

\textbf{Baseline.}
We compare \puma\ to the most similar prior work \mpcformer~\citep{li2023mpcformer}. But for fair comparison, we have the following considerations:
\romannumeral1) As \mpcformer\ neither supports loading pretrained transformer models nor implements LayerNorm faithfully\footnote{ As \mpcformer~does not support loading pre-trained Transformer models, we did an experiment in plaintext Bert-Base that replaced LayerNorm with BatchNorm  as \mpcformer~did. This  resulted in a significant drop in the MCC score for CoLA task from $0.616$ to $-0.020$. On the contrary, \puma~achieves an MCC score of $0.613$. }, we cannot achieve meaningful secure inference results using their framework.
Therefore, we compare our secure Transformer models inference performance to that of plaintext (floating-point) to show our precision guarantee.
\romannumeral2) \mpcformer\ with \textit{Quad} approximations (for both $\gelu$ and $\softmax$) requires retraining the  modified models. As \puma\ does not require retraining, we compare our cost to that of \mpcformer\ without \textit{Quad} approximations. Also, we re-run \mpcformer~in our environment.



\subsection{Precision}\label{sec:accuracy}

% Figure environment removed

%\begin{table}
\centering
\caption{Performance on GLUE benchmark of Bert-Base, Roberta-Base, and Bert-Large on CoLA, RTE, and QNLI, Matthews correlation is reported for CoLA. Accuracy is reported for other datasets.}\label{table:bertacc}
\begin{tabular}{c|ccc|ccc|ccc}
\hline \hline
 Model & \multicolumn{3}{c|}{Bert-Base} & \multicolumn{3}{c|}{Roberta-Base} & \multicolumn{3}{c}{Bert-Large} \\ \hline
 TASK & CoLA & RTE & QNLI & CoLA & RTE & QNLI & CoLA & RTE & QNLI \\ \hline
CPU & $0.616$     & $0.700$      & $0.916$     & $0.629$ & $0.805$ & $0.920$  & $0.686$   & $0.755$ & $0.922$ \\
\puma   & $0.613$     & $0.700$     & $0.916$     & $0.618$ & $0.805$ & $0.918$ & $0.690$ & $0.747$ & $0.918$ \\ \hline \hline
\end{tabular}
\end{table}

\begin{table}[]
    \centering
    \caption{Perplexity of GPT2-Base, GPT2-Medium, and GPT2-Large on Wikitext-103 V1.}
    \label{tab:gpot2ppl}
    \begin{tabular}{c|c|c|c}
    \hline \hline
      Model & GPT2-Base & GPT2-Medium & GPT2-Large \\ \hline
      CPU & $16.284$ & $12.536$ & $10.142$ \\
      \puma & $16.284$ & $12.540$ & $10.161$ \\
      \hline \hline
    \end{tabular}
    
\end{table}

We compare our secure model 
inference performance to that of plaintext (floating-point) in Figure~\ref{fig:performance} to show our precision guarantee.

In Figure~\ref{fig:bert-base}-\ref{fig:bert-large}, we show the Matthews correlation/accuracy of plaintext and \puma\ on the Bert-Base, Roberta-base, and Bert-Large. We observe that the accuracy achieved by \puma~ matches the accuracy of the plaintext Flax code. Specifically, the accuracy difference does
not exceed $0.011$ over all datasets. 

Moreover, in Figure~\ref{fig:gpt2}, we also compare our perplexity on dataset Wikitext-103 V1 with the plaintext baseline on models GPT2-Base, GPT2-Medium, and GPT2-Large. The results are similar and the perplexity differences do not exceed $0.02$ over all models.

The above accuracy and perplexity advantages experimentally validate that our protocols are numerically precise. 

\subsection{Inference cost}\label{sec:efficiency}
\begin{table}[h]
    \centering
    \caption{Costs of Bert-Base, Roberta-Base, and Bert-Large for one sentence of length $128$. Time is in seconds and Communication (Comm. for short) is in GB, which is the same for the following tables.}\label{tab:costbert}
    \begin{tabular}{c|cc|cc|cc}
    \hline \hline
       Model & \multicolumn{2}{c|}{Bert-Base} & \multicolumn{2}{c|}{Roberta-Base} & \multicolumn{2}{c}{Bert-Large} \\ \hline
       Costs & Time & Comm. & Time & Comm. & Time & Comm. \\ \hline
       \mpcformer & $55.320$ & $12.089$ & $57.256$ & $12.373$ & $141.222$ & $32.577$ \\
       \puma & $33.913$ & $10.773$ & $41.641$ & $11.463$ & $73.720$ & $27.246$ \\
       \cellcolor{mygray} Improv. & \cellcolor{mygray} $1.631\times$ & \cellcolor{mygray} $1.122\times$ & \cellcolor{mygray} $1.375\times$ & \cellcolor{mygray} $1.079\times$ & \cellcolor{mygray} $1.916\times$ & \cellcolor{mygray} $1.195\times$ \\
       \hline \hline
    \end{tabular}
    \vspace{-0.2cm}
\end{table}

\begin{table}[]
    \centering
    \caption{Costs of GPT2-Base, GPT2-Medium, and GPT2-Large. The input sentence is of length $32$, all of the costs are for generating $1$ token.}\label{tab:costgpt2}
    \begin{tabular}{c|cc|cc|cc}
    \hline \hline
       Model & \multicolumn{2}{c|}{GPT2-Base} & \multicolumn{2}{c|}{GPT2-Medium} & \multicolumn{2}{c}{GPT2-Large} \\ \hline
       Costs & Time & Comm. & Time & Comm. & Time & Comm. \\ \hline
       \mpcformer & $34.889$ & $4.999$ & $73.078$ & $11.766$ & $129.095$ & $22.522$  \\
       \puma & $15.506$ & $3.774$ & $30.272$ & $7.059$ & $54.154$ & $11.952$ \\
       \cellcolor{mygray} Improv. & \cellcolor{mygray} $2.250\times$ & \cellcolor{mygray} $1.325\times$ & \cellcolor{mygray} $2.414\times$ & \cellcolor{mygray} $1.667\times$ & \cellcolor{mygray} $2.383\times$ & \cellcolor{mygray} $1.884\times$ \\
       \hline \hline
    \end{tabular}
    \vspace{-0.2cm}
\end{table}

In this subsection, we compare \puma's inference cost to that of \mpcformer. 
We evaluate  three Bert models (Bert-Base, Roberta-Base, and Bert-Large) and three GPT2 models (GPT2-Base, GPT2-Medium, and GPT2-Large).
The costs are for processing one input sentence: \romannumeral1) For Bert models the input sentence is of length $128$. \romannumeral2) GPT2 models input one length-32 sentence and generate $1$ new word. 

On the 3 Bert models in Table~\ref{tab:costbert}, \puma\ is  $1.375\sim 1.916\times$ faster than  \mpcformer, and is $1.079\sim 1.195\times$ more communication-efficient. For the GPT2 models in Table~\ref{tab:costgpt2}, \puma\ is $2.250\sim 2.414\times$ faster than \mpcformer, and is $1.325\sim 1.884\times$ more communication-efficient. 
    
We observe that \puma's improvements increase as the model size grows, particularly for the GPT2 models. This trend is because our specialized optimizations are more effective when processing large-scale evaluations.



\subsection{Scalability}\label{sec:scala}

In this subsection, we measure the costs of evaluating \puma\ on Bert-Base and GPT2-Base models for varying-length inputs, and varying-length outputs (only for GPT2-Base). We also compare our costs to those of \mpcformer~to demonstrate our improvements.





\begin{table}[]
    \centering
    \caption{Costs of Bert-Base and GPT2-Base for different input length (denoted as \#Input). The input lengths for Bert-Base and GPT2-Base are respective $\{64, 128, 256, 512\}$ and $\{16, 32, 64, 128\}$. GPT2-Base generates $1$ token.}\label{tab:costbertinput}
    \begin{tabular}{cc|cc|cc|cc|cc}
    \hline \hline
       \multicolumn{2}{c|}{\#Input} & \multicolumn{2}{c|}{$64 / 16$} & \multicolumn{2}{c|}{$128 / 32$} & \multicolumn{2}{c|}{$256 / 64$} & \multicolumn{2}{c}{$512 / 128$}  \\ \hline
       \multicolumn{2}{c|}{Costs} & Time & Comm. & Time & Comm. & Time & Comm. & Time & Comm. \\ \hline
       \multirow{3}{*}{Bert}& \mpcformer & $46.428$ & $4.750$ & $85.887$ & $9.673$ & $196.372$ & $23.443$ & $582.787$ & $68.069$ \\
       & \puma & $24.345$ & $1.627$ & $42.525$ & $3.591$ & $87.561$ & $8.668$ & $212.600$ & $23.439$\\
       & \cellcolor{mygray} Improv. & \cellcolor{mygray} $1.907\times$ & \cellcolor{mygray} $2.919\times$ & \cellcolor{mygray} $2.020\times$ & \cellcolor{mygray} $2.694\times$ & \cellcolor{mygray} $2.243\times$ & \cellcolor{mygray} $2.705\times$ & \cellcolor{mygray} $2.741\times$ & \cellcolor{mygray} $2.904$ \\
       \hline
       \multirow{3}{*}{GPT2}& \mpcformer & $34.522$ & $3.767$ & $42.615$ & $4.516$ & $60.451$ & $6.281$ & $105.028$ & $11.225$  \\
       & \puma & $20.692$ & $0.625$ & $29.248$ & $1.258$ & $40.968$ & $2.607$ & $74.529$ & $5.611$\\
       &\cellcolor{mygray} Improv. & \cellcolor{mygray} $1.668\times$ & \cellcolor{mygray} $6.027\times$ & \cellcolor{mygray} $1.457\times$ & \cellcolor{mygray} $3.590\times$ & \cellcolor{mygray} $1.476\times$ & \cellcolor{mygray} $2.409\times$ & \cellcolor{mygray} $1.409\times$ & \cellcolor{mygray} $2.001\times$\\
       \hline \hline
    \end{tabular}
\end{table}
\textbf{Input Length Evaluation.}
Table~\ref{tab:costbertinput} shows our costs on varying-length inputs, we evaluate Bert-Base on the inputs of length $\{64, 128, 256, 512\}$, and GPT2-Base on the inputs of length $\{16, 32, 64, 128\}$.
For Bert-Base, \puma\ is $1.720\sim 2.282\times$ faster, and for GPT2-Base, \puma\ is $1.550\sim 2.686\times$ faster. Unlike the observations in Section~\ref{sec:efficiency}, our efficiency gains decrease with increasing input sizes in GPT2, and \puma\ requires more communication when the input length is greater than 64. This phenomenon is attributed to the interesting fact: To directly support pre-trained plaintext models, \puma\ strictly follows the plaintext model format that only accept token ids as input, so \puma\ has to compute the one-hot vectors from token ids in an MPC way. On the other hand, \mpcformer\ uses modified models that accept one-hot vectors as input, so the one-hot function could be computed at the client side in plaintext. Nevertheless, \puma\ remains faster than \mpcformer.

%\begin{table}[]
    \centering
    \caption{Costs of GPT2-small for generating different output tokens (denoted as \#Output), the input length is set as $32$.}\label{tab:costgpt2tokens}
    \begin{tabular}{c|cc|cc|cc|cc}
    \hline \hline
       \#Output & \multicolumn{2}{c|}{2} & \multicolumn{2}{c|}{4} & \multicolumn{2}{c|}{8} & \multicolumn{2}{c}{16}  \\ \hline
       Costs & Time & Comm. & Time & Comm. & Time & Comm. & Time & Comm. \\ \hline
       \mpcformer & $72.833$ & $7.676$ & $132.644$ & $13.998$ & $252.796$ & $26.648$ & $494.509$ & $51.972$ \\
       \puma & $53.191$ & $2.549$ & $111.457$ & $5.167$ & $215.352$ & $11.115$ & $457.994$ & $24.917$ \\
       Improv. & $1.369\times$ & $3.011\times$ & $1.190\times$ & $2.709\times$ & $1.174\times$ & $2.397\times$ & $1.080\times$ & $2.086\times$ \\
       \hline \hline
    \end{tabular}
\end{table}

\begin{wrapfigure}{r}{0.4\textwidth}
    % Figure removed
    \caption{Runtime of GPT2-Base for generating different number of output tokens, the input length is of length $32$.} 
    \label{fig:gptwoutcosts}
\end{wrapfigure}

\textbf{Output Length Evaluation.}
Fig~\ref{fig:gptwoutcosts} presents our costs on varying-length outputs for GPT2-Base, and compares our costs to those of \mpcformer. Our improvements in runtime range from $1.279\sim 2.700\times$ respectively.
As more output tokens are generated, both costs increase in a linear way, this is because each output token must be input back into the model to generate the next token, increasing the required one-hot embedding costs. We should emphasize
again that although the time costs might be close for long outputs, \puma\ could achieve a similar accuracy as plaintext models while \mpcformer\  could not. 


\begin{table}[]
    \centering
    \caption{Costs of the secure inference of LLaMA-7B, \#Input denotes the length of input sentence and \#Output denotes the number of generated tokens.}\label{tab:llama7b}
    \begin{tabular}{c|cc|cc|cc}
    \hline \hline
       (\#Input, \#Output) & \multicolumn{2}{c|}{$(4,1)$} & \multicolumn{2}{c|}{$(8,1)$} & \multicolumn{2}{c}{$(8,2)$} \\ \hline
       Costs & Time & Comm. & Time & Comm. & Time & Comm. \\ \hline
       \puma & $122.004$ & $0.907$ & $200.473$ & $1.794$ & $364.527$ & $3.857$ \\
       \hline \hline
    \end{tabular}
    \vspace{-0.2cm}
\end{table}

\textbf{Scale to LLaMA-7B in Five Minutes.}
We evaluated the large language model LLaMA-7B using \puma\ under 3 Alibaba Cloud
ecs.r7.32xlarge servers, each has 128 threads and 1TB RAM, with 20GB bandwidth, 0.06ms round-trip-time. 
As shown in Table~\ref{tab:llama7b}, \puma\ can support the secure inference of large language model LLaMA-7B with reasonable costs. For example, given an input sentence of 8 tokens, \puma\ can output one token in around $346.126$ seconds with communication costs of $1.865$ GB. To our knowledge, this is the first time that LLaMA-7B has been evaluated using MPC.


%Llama-7B, LAN=(20GB, 0.06ms), 128 threads, input length=8, output=1 token, costs: 346.126s, 2002213760 bytes

\section{Limitations}
\label{sec:Limitations}
Since our method uses the deep ISP pipeline, the training speed will slower than RGB watermarking methods. This is because the gradient of the entire ISP pipeline needs to be calculated. In the future, using a smaller deep ISP pipeline can improve the training speed.
At present, our method is limited to encoding and decoding RAW image pixel blocks of a fixed size. However, due to the considerable size of the RAW image in practical applications, we have opted to encode only the central area pixel block. If we choose other pixel blocks to add watermarking, we need to manually locate the watermarking pixel block from the RGB image. At the same time, since we use pixel blocks to add watermarking, the whole RGB results will have an obvious boundary around watermarked blocks.

\section{Conclusion}\label{sec:conclusion}

This paper presents our empirical domain knowledge distillation framework using ChatGPT and discusses our observations from the framework application experiments in the autonomous driving domain. The key finding is that: 1) with proper design of prompt engineering and execution flow, fully automated domain knowledge (in the ontology format) distillation is possible. However, due to the randomness in the response and the butterfly effect, the quality of fully automated distillation results is not guaranteed. To address this, we develop a web-based assistant to enable manual supervision and early intervention at runtime. We hope our findings and tools inspire future research toward revolutionizing the engineering processes of knowledge-based systems across domains.

% \section*{Acknowledgments}
% This should be a simple paragraph before the References to thank those individuals and institutions who have supported your work on this article.



% {\appendix[Proof of the Zonklar Equations]
% Use $\backslash${\tt{appendix}} if you have a single appendix:
% Do not use $\backslash${\tt{section}} anymore after $\backslash${\tt{appendix}}, only $\backslash${\tt{section*}}.
% If you have multiple appendixes use $\backslash${\tt{appendices}} then use $\backslash${\tt{section}} to start each appendix.
% You must declare a $\backslash${\tt{section}} before using any $\backslash${\tt{subsection}} or using $\backslash${\tt{label}} ($\backslash${\tt{appendices}} by itself
%  starts a section numbered zero.)}



%{\appendices
%\section*{Proof of the First Zonklar Equation}
%Appendix one text goes here.
% You can choose not to have a title for an appendix if you want by leaving the argument blank
%\section*{Proof of the Second Zonklar Equation}
%Appendix two text goes here.}



% \section{References Section}
% You can use a bibliography generated by BibTeX as a .bbl file.
%  BibTeX documentation can be easily obtained at:
%  http://mirror.ctan.org/biblio/bibtex/contrib/doc/
%  The IEEEtran BibTeX style support page is:
%  http://www.michaelshell.org/tex/ieeetran/bibtex/
 
%  % argument is your BibTeX string definitions and bibliography database(s)
% %\bibliography{IEEEabrv,../bib/paper}
% %
% \section{Simple References}
% You can manually copy in the resultant .bbl file and set second argument of $\backslash${\tt{begin}} to the number of references
%  (used to reserve space for the reference number labels box).


\bibliographystyle{IEEEtran}
\bibliography{output} 
% \bibitem{ref1}
% {\it{Mathematics Into Type}}. American Mathematical Society. [Online]. Available: https://www.ams.org/arc/styleguide/mit-2.pdf

% \bibitem{ref2}
% T. W. Chaundy, P. R. Barrett and C. Batey, {\it{The Printing of Mathematics}}. London, U.K., Oxford Univ. Press, 1954.

% \bibitem{ref3}
% F. Mittelbach and M. Goossens, {\it{The \LaTeX Companion}}, 2nd ed. Boston, MA, USA: Pearson, 2004.

% \bibitem{ref4}
% G. Gr\"atzer, {\it{More Math Into LaTeX}}, New York, NY, USA: Springer, 2007.

% \bibitem{ref5}M. Letourneau and J. W. Sharp, {\it{AMS-StyleGuide-online.pdf,}} American Mathematical Society, Providence, RI, USA, [Online]. Available: http://www.ams.org/arc/styleguide/index.html

% \bibitem{ref6}
% H. Sira-Ramirez, ``On the sliding mode control of nonlinear systems,'' \textit{Syst. Control Lett.}, vol. 19, pp. 303--312, 1992.

% \bibitem{ref7}
% A. Levant, ``Exact differentiation of signals with unbounded higher derivatives,''  in \textit{Proc. 45th IEEE Conf. Decis.
% Control}, San Diego, CA, USA, 2006, pp. 5585--5590. DOI: 10.1109/CDC.2006.377165.

% \bibitem{ref8}
% M. Fliess, C. Join, and H. Sira-Ramirez, ``Non-linear estimation is easy,'' \textit{Int. J. Model., Ident. Control}, vol. 4, no. 1, pp. 12--27, 2008.

% \bibitem{ref9}
% R. Ortega, A. Astolfi, G. Bastin, and H. Rodriguez, ``Stabilization of food-chain systems using a port-controlled Hamiltonian description,'' in \textit{Proc. Amer. Control Conf.}, Chicago, IL, USA,
% 2000, pp. 2245--2249.



\newpage

% \section{Biography Section}
% If you have an EPS/PDF photo (graphicx package needed), extra braces are
%  needed around the contents of the optional argument to biography to prevent
%  the LaTeX parser from getting confused when it sees the complicated
%  $\backslash${\tt{includegraphics}} command within an optional argument. (You can create
%  your own custom macro containing the $\backslash${\tt{includegraphics}} command to make things
%  simpler here.)
 
% \vspace{11pt}

% \bf{If you include a photo:}\vspace{-33pt}
% \begin{IEEEbiography}[{% Figure removed}]{Michael Shell}
% Use $\backslash${\tt{begin\{IEEEbiography\}}} and then for the 1st argument use $\backslash${\tt{includegraphics}} to declare and link the author photo.
% Use the author name as the 3rd argument followed by the biography text.
% \end{IEEEbiography}

% \vspace{11pt}

% \bf{If you will not include a photo:}\vspace{-33pt}
% \begin{IEEEbiographynophoto}{John Doe}
% Use $\backslash${\tt{begin\{IEEEbiographynophoto\}}} and the author name as the argument followed by the biography text.
% \end{IEEEbiographynophoto}




\vfill

\end{document}


