% \subsection{Invariant Zeros from invariant subspaces theory}
    On the other hand, \cite{basile1992controlled,morris2010invariant} studied a geometric approach to finding the invariant zeros of a linear system. Let $\mathcal{Z}$ be an $(A,\text{Range}(B))-$controlled invariant subspace. The set of invariant zeros is the set of $\lambda$ for which
    \begin{align}
        \matl
            \lambda I -A & B\\
            C & 0
        \matr
        \matl
            x\\
            u
        \matr=
        \matl
            0\\
            0
        \matr,
    \end{align}
    has a solution for some scalar $u_0$ and nonzero $x_0 \in \mathcal{F}\cap\mathcal{Z}$. Suppose that $\lambda$ is an eigenvalue of $(A,\text{Range}(B))|_{\mathcal{Z}}$, and an invariant zero of the system. Then,
    \begin{align}
        \matl
            \lambda I -A & B\\
            C & 0
        \matr
        \matl
            x_0\\
            u_0
        \matr=
        \matl
            0\\
            0
        \matr.
    \end{align}
    This shows that
    \begin{align}
        Cx_0=0,
    \end{align}
    which implies that $x_0 \in \text{Null}(C)$, and also
    \begin{align}
        \lambda x_0 + Bu_0 = Ax_0,
    \end{align}
    so $x_0$ is contained in a $(A,\text{Range}(B))-$controlled invariant subspace contained in $\text{Null}(C)$. Moreover, we want the supremum of the $(A,\text{Range}(C))$ contained in $\text{Null}(C)$. Then,
    \begin{align}
        x_0 \in \max \mathcal{V}(A,\text{Range}(B),\text{Null}(C)).
    \end{align}
    Furthermore, $\mathcal{V}^*$ has an infimum as well, which is
    \begin{align}
        R_{V^*}=V^*\cap S^*,
    \end{align}
    which represents the assignable subspace of $V^*,$ (or in other words the subspace of $V^*$ for which $\text{Range}\in \mathcal{V}^*$). From the complementability property of $\mathcal{V}^*$ we have
    \begin{align}
        R_{\mathcal{V}^*} + \mathcal{V} = \mathcal{V}^*.
    \end{align}
    where $\mathcal{V}$ is also an $(A,\text{Range}(B))-$controlled invariant subspace that represents the unassignable subspace of $\mathcal{V}$ such that,
    \begin{align}
        (A + BF)\mathcal{V}\subseteq\mathcal{V}.
    \end{align}
    Nevertheless, its implementation involves the codification of many functions, which also makes its interpretation difficult for readers who are not familiar with differential geometry concepts.\\