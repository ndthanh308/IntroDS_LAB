
\section{Known Results}

\subsection{Zero Invariance Principal}




% This approach uses the zero invariance principle to compute the zeros of the system. 
% % 
% The ZIP states that the zeros remain unchanged under output feedback and root locus technique states that the poles of the closed-loop system converge to the zeros of the system. 
% Thus, the zeros of $(A,B,C,D)$ are the poles of the system $(A+\rho BKC,BKD)$

\subsection{Generalized Eigenvalues of System Matrix}



\begin{align}
    \SZ(s) 
        =
            \matl 
                s I - A & B \\
                C & -D
            \matr
        =
            \matl 
                - A & B \\
                C & -D
            \matr
            -
            s
            \matl 
                -I & 0 \\
                0 & 0
            \matr
        =
            \SA - s \SB
\end{align}

The determinant of $\SZ(s)$ is zero for the generalized eigenvalues of $(\SA,\SB).$


\begin{definition}
    Let $A, B \in \BBR^{n \times n}.$
    Then, $\lambda \in \BBC$ is a generalized eigenvalue of $(A,B)$ 
    and $x \in \BBR^n $ is generalized eigenvector of $(A,B)$ if 
    \begin{align}
        A x = \lambda B x. 
    \end{align}
\end{definition}

\begin{proposition}
    Let $A, B \in \BBR^{n \times n}.$
    Suppose $B$
\end{proposition}


\begin{proposition}
    Let $\lambda$ be a generalized eigenvalue of $(A,B).$
    Then, $\lambda^{-1}$ is a generalized eigenvalue of $(B,A).$
\end{proposition}

Let 
\begin{align}
    G(s) = \frac{N(s)}{D(s)}.
\end{align}
The zeros of the system are the roots of $N(s).$
If $(A,B,C,D)$ is a realization of $G(s),$ then, 
\begin{align}
    N(s) = C {\rm adj \ } (\lambda I - A) B + \det (\lambda I- A) D.
\end{align}
Therefore, the zeros of the system are the roots of 
\begin{align}
    C {\rm adj \ } (\lambda I - A) B + \det (\lambda I- A) D = 0.
\end{align}





% The zeros of $G$ are contained in the generalized eigenvalues of the system matrix
% \begin{align}
%     S(\lambda)
%         =
%             \matl 
%                 \lambda I - A & B \\
%                 C & -D
%             \matr.
% \end{align}
% Note that the zeros are the roots of
% \begin{align}
%     N(\lambda) = C~{\rm adj \ } (\lambda I - A) B + \det (\lambda I- A) D
% \end{align}

% Note that 
% \begin{align}
%     \det S(\lambda)
%         =
%            C~{\rm adj \ } (\lambda I - A) B + \det (\lambda I- A) D,
% \end{align}

\begin{proposition}
    Let $\SZ(s)$ denote the system matrix. 
    Then,
    \begin{align}
        \det \SZ(s) = C {\rm adj \ } (\lambda I - A) B + \det (\lambda I- A) D.
    \end{align}
\end{proposition}
\textbf{Proof:} Note that $G(s)=C(sI-A)^{-1}B + D$, $S(\lambda)$ can be expressed as
\begin{align}
    S(\lambda) =& \matl
        \lambda I-A & B\\
        C & -D
    \matr\nonumber\\
    =& 
        \matl
            sI-A & 0\\
            C & -C(sI-A)^{-1}B - D
        \matr
        \matl
            I & (sI-A)^{-1}B\\
            0 & I
        \matr
        \nonumber\\
    =& \matl
        sI-A & 0\\
        C & -G(s)
    \matr\matl
        I & (sI-A)^{-1}B\\
        0 & I
    \matr.
\end{align}
Then,
\begin{align}
    \det(S(\lambda)) =& \det(-(\lambda I-A)G(\lambda))\nonumber\\
    =& \det(\lambda I-A)\det(-G(\lambda))\nonumber\\
    =& -\det(\lambda I-A)\det(G(\lambda))\nonumber\\
    =& -\det(\lambda I-A)\det(C(\lambda I-A)^{-1}B + D)\nonumber\\
    =& -\det(\lambda I-A)\det\left(C~\dfrac{\text{adj}(\lambda I-A)}{\det(\lambda I-A)}B + D\right)\nonumber\\
    =& \det\left(C~\text{adj}(\lambda I-A)B+\det(\lambda I-A)D\right). 
    \nn
\end{align}
Since $C~\text{adj}(\lambda I-A)B+\det(\lambda I-A)D \in \mathbb{R}$. Then,
\begin{align}
    \det S(\lambda) =C~\text{adj}(\lambda I-A)B+\det(\lambda I-A)D
\end{align}

Finally, $\det \SZ(s)=0$ implies that $\SZ(s)$ is rank deficient at $s$ and thus, there exists a vector such that
\begin{align}
    \SZ(s) x = 0
\end{align}
which implies that 
\begin{align}
    \matl
        sI-A & B\\
        C & -D
    \matr X
        =
            0.
\end{align}
or
\begin{align}
    \left(
    \matl
        -A & B\\
        C & -D
    \matr 
    -
    s 
    \matl 
        I & 0 \\
        0 & 0
    \matr
    \right)
    X
        =
            0.
\end{align}
