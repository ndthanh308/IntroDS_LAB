%%%%%%%%%%%%%%%%%%%%%%%%%%%%%%%%%%%%%%%%%%%%%%%%%%%%%%%%%%%%%%%%%%%%%%%%%%%%%%%%
%2345678901234567890123456789012345678901234567890123456789012345678901234567890
%        1         2         3         4         5         6         7         8

\documentclass[letterpaper, 10 pt, conference]{ieeeconf}  % Comment this line out
                                                          % if you need a4paper
%\documentclass[a4paper, 10pt, conference]{ieeeconf}      % Use this line for a4
                                                          % paper

\IEEEoverridecommandlockouts                              % This command is only
                                                          % needed if you want to
                                                          % use the \thanks command
% \overrideIEEEmargins
% See the \addtolength command later in the file to balance the column lengths
% on the last page of the document

%%%%%%%%%%%%%%%%%%%%%%%%%%%%%%%%%%%%%%%%%%%%%%%
%%%%%%%%%%%%%%%%% PACKAGES %%%%%%%%%%%%%%%%%%%%
%%%%%%%%%%%%%%%%%%%%%%%%%%%%%%%%%%%%%%%%%%%%%%%
\usepackage{amsmath} % assumes amsmath package installed
\usepackage{amssymb}  % assumes amsmath package installed
\usepackage{amsfonts}
\usepackage{amsthm}
\newtheoremstyle{indented}
    {3pt}
    {3pt}
    {\addtolength{\leftskip}{0em}}
    {}
    {\bfseries}
    {.}
    {1em}
    {}
\theoremstyle{indented}

% \newtheorem{property}{Property}[section]
\newtheorem{theorem}{Theorem}[section]
% \newtheorem{corollary}{Corollary}[section]
\newtheorem{proposition}{Proposition}[section]

% \newtheorem{definition}{Definition}[section]
% \newtheorem{def_chap}{Definition}[chapter]
% % \newtheorem{definitionSec}{Definition}[subsection]
\newtheorem{exmp}{Example}[section]
% \newtheorem{exmp_chap}{Example}[chapter]
% \newtheorem{exmpSec}{Example}[subsection]
% \newtheorem{fact}{Fact}[section]
% \newtheorem{question}{Question}[section]
% \newtheorem{fact_chap}{Fact}[chapter]
% \newtheorem{proposition}{Proposition}[section]
% \newtheorem{factSec}{Fact}[subsection]
% \newtheorem{theorem}{Theorem}[section]
% \newtheorem{problem}{Problem}[section]
\newtheorem{remark}{Remark}[section]
\newtheorem{lemma}{Lemma}[section]

% \newtheorem{prob}{Problem}[section]
% \newtheorem{conjecture}{Conjecture}[section]

% \theoremstyle{plain}



% \newtheoremstyle{factindented}
%     {3pt}
%     {3pt}
%     {\addtolength{\leftskip}{1.5em}}
%     {}
%     {\bfseries}
%     {.}
%     {.5em}
%     {}
% %\theoremstyle{factindented}





% \usepackage{graphicx}
% \usepackage{epsfig}
% \usepackage{epstopdf}
% \usepackage{math tools}
% \usepackage{color}
% \usepackage[utf8]{inputenc}
%\pagestyle{plain}

% \usepackage{subfigure}
% \usepackage{subfig}
% \usepackage{float} 
% \usepackage[skip=1pt,font=footnotesize]{caption}
% \usepackage{wrapfig}%   wrap figures/tables in text (i.e., Di Vinci style)
% \usepackage[export]{adjustbox}

% \usepackage{geometry}
% \usepackage{indentfirst}


% \usepackage{epsfig} % for postscript graphics files

% \usepackage{lastpage}
% \usepackage{fancyhdr}
% \pagestyle{fancy} 
% \cfoot{\thepage\ of \pageref{LastPage}}
% \fancyhead{}
% \renewcommand{\headrulewidth}{0pt}

% \usepackage{mathptmx} % assumes new font selection scheme installed
% \usepackage{times} % assumes new font selection scheme installed


% \usepackage[mathscr]{euscript}

% \usepackage{newtxtext}
% \usepackage{newtxmath}

% \usepackage{subfiles}
% \usepackage{cancel}
% \usepackage{imakeidx}

%\usepackage{citesort}
%\usepackage[fleqn]{amsmath}
%\usepackage{fullpage}
%\usepackage[top=0.75in,bottom= 0.75in,left = 0.75in, right = 0.75in]{geometry}
% \usepackage[margin = 1in]{geometry}

%\parskip = -0.5pt
%\parindent = 2pt %0in


%\usepackage{multicol}
%\usepackage{caption}
%\usepackage[skip=2pt]{caption}
%\usepackage[skip=1pt,font=footnotesize]{caption}
%\usepackage[skip=1pt,font=small]{caption}
% \usepackage{subcaption}


% \usepackage{etaremune}



% \usepackage{tikz}
% \usetikzlibrary{shapes,arrows,calc,positioning}
% \tikzstyle{bigblock} = [draw, fill=blue!20, rectangle, 
%     minimum height=6em, minimum width=8em]
% \tikzstyle{medblock} = [draw, fill=blue!20, rectangle, 
%     minimum height=4em, minimum width=4em]    
% \tikzstyle{mux} = [draw, fill=black!20, rectangle, 
%     minimum height=5em, minimum width=0.1em]    
% \tikzstyle{smallblock} = [draw, fill=blue!20, rectangle, 
%     minimum height=2em, minimum width=3em]
    
% \tikzstyle{data_block} = [draw, fill=green!20, rectangle, 
%     minimum height=2em, minimum width=3em]
% \tikzstyle{ops_block} = [draw, fill=blue!20, rectangle, 
%     minimum height=2em, minimum width=3em]    
% \tikzstyle{est_block} = [draw, fill=red!20, rectangle, 
%     minimum height=2em, minimum width=3em]    
    
% \tikzstyle{sum} = [draw, fill=blue!20, circle, node distance=1cm,minimum height=0.5cm]
% \tikzstyle{signal} = [coordinate]
% \tikzstyle{pinstyle} = [pin edge={to-,thin,black}]
% \tikzstyle{block} = [draw, fill=blue!20, rectangle, 
%     minimum height=3em, minimum width=9em]
% \tikzstyle{blockS} = [draw, fill=blue!20, rectangle, 
%     minimum height=3em, minimum width=4em]    
% \tikzstyle{input} = [coordinate]
% \tikzstyle{output} = [coordinate]
% \usetikzlibrary{matrix}

% \usetikzlibrary{positioning}
% \usetikzlibrary{math} 



\usepackage{hyperref}
\usepackage{xcolor}
\hypersetup{
    colorlinks,
    linkcolor={blue!100!black},
    citecolor={blue!50!black},
    urlcolor={blue!80!black}
}

% \usepackage{diagbox}


%\newtheorem{definition}{Definition}[section]
%\usepackage{biblatex}
% \addbibresource{mendeley_v2.bib}

% \usepackage[square, numbers, sort&compress]{natbib} 
% \setcitestyle{square braces, numbers}
%\usepackage{cite}



%\usepackage[nomarkers,figuresonly]{endfloat}

% \renewcommand{\familydefault}{\rmdefault}




%%%%%%%%%%%%%%%%%%%%%%%%%%%%%%%%%%%%%%%%%%%%%%%
%%%%%%%%%%%%%%%%% COMMANDS %%%%%%%%%%%%%%%%%%%%
%%%%%%%%%%%%%%%%%%%%%%%%%%%%%%%%%%%%%%%%%%%%%%%

\newcommand{\bc}{\begin{center}}
\newcommand{\ec}{\end{center}}
\newcommand{\benum}{\begin{enumerate}}
\newcommand{\eenum}{\end{enumerate}}
\newcommand{\nn}{\nonumber}
\newcommand{\matl}{\left[ \begin{array}}
\newcommand{\matr}{\end{array} \right]}
\newcommand{\matls}{\left[ \begin{smallmatrix}}
\newcommand{\matrs}{\end{smallmatrix} \right]}
\newcommand{\isdef}{\stackrel{\triangle}{=}}
\newcommand{\pr}[1]{{#1}^{\prime}}
\newcommand{\inv}{^{-1}}
\newcommand{\row}[1]{{\rm row}_{#1}}
\newcommand{\col}[1]{{\rm col}_{#1}}
\newcommand{\arrowup}[1]{\stackrel{#1}{\SLeftarrow}}
\newcommand{\smin}{\stackrel{\min}{\sim}}
\newcommand{\jw}{\jmath \omega}
\newcommand{\omegan}{\omega_{\rm n}}
\newcommand{\omegad}{\omega_{\rm d}}
\newcommand{\omegar}{\omega_{\rm r}}
\newcommand{\omegani}{\omega_{{\rm n}i}}
\newcommand{\omegans}[1]{\omega_{{\rm n}_{#1}}}
\newcommand{\MSig}{{\mit\Sigma}}
\newcommand{\half}{\tfrac{1}{2}}
\newcommand{\vect}[1]{\overset{\rightharpoonup}{#1}}
\newcommand{\HS}{\hat S}
\newcommand{\HI}{\hat I}
\newcommand{\rmThat}{{\hat{\rm T}}}
\newcommand{\asthat}{{\hat*}}
\newcommand{\dpder}[2]{\displaystyle\frac{\partial {#1}}{\partial {#2}}}
\newcommand{\der}[2]{\mbox{$\frac{d{#1}}{d{#2}}$}}
\newcommand{\dder}[2]{\displaystyle{\frac{d{#1}}{d{#2}}}}
\newcommand{\pder}[2]{\mbox{$\frac{\partial {#1}}{\partial {#2}}$}}
\newcommand{\dt}{{\rm d}t}
\newcommand{\dx}{{\rm d}x}
\newcommand{\p}{\partial}
\newcommand{\pt}{\partial_t}
\newcommand{\ptt}{\partial^2_t}
\newcommand{\px}{\partial_x}
\newcommand{\pxx}{\partial^2_x}
\newcommand{\pp}{\partial{^2}}
\newcommand{\tr}{{\rm tr}\,}
%%%%%%%%%%%%%%%%%%%%%%%%%%%%%%%%%%%%%%%%%%%%%%%%%%%%%%%%%%%%%%%%%%%%%%%%%%%%%%%%%%%
%%%\DeclareMathOperator{\tr}{tr}
\newcommand{\rmA}{{\rm A}}
\newcommand{\rmB}{{\rm B}}
\newcommand{\rmC}{{\rm C}}
\newcommand{\rmD}{{\rm D}}
\newcommand{\rmE}{{\rm E}}
\newcommand{\rmF}{{\rm F}}
\newcommand{\rmG}{{\rm G}}
\newcommand{\rmH}{{\rm H}}
\newcommand{\rmI}{{\rm I}}
\newcommand{\rmJ}{{\rm J}}
\newcommand{\rmK}{{\rm K}}
\newcommand{\rmL}{{\rm L}}
\newcommand{\rmM}{{\rm M}}
\newcommand{\rmN}{{\rm N}}
\newcommand{\rmO}{{\rm O}}
\newcommand{\rmP}{{\rm P}}
\newcommand{\rmQ}{{\rm Q}}
\newcommand{\rmR}{{\rm R}}
\newcommand{\rmS}{{\rm S}}
\newcommand{\rmT}{{\rm T}}
\newcommand{\rmU}{{\rm U}}
\newcommand{\rmV}{{\rm V}}
\newcommand{\rmW}{{\rm W}}
\newcommand{\rmX}{{\rm X}}
\newcommand{\rmY}{{\rm Y}}
\newcommand{\rmZ}{{\rm Z}}
\newcommand{\rma}{{\rm a}}
\newcommand{\rmb}{{\rm b}}
\newcommand{\rmc}{{\rm c}}
\newcommand{\rmd}{{\rm d}}
\newcommand{\rme}{{\rm e}}
\newcommand{\rmf}{{\rm f}}
\newcommand{\rmg}{{\rm g}}
\newcommand{\rmh}{{\rm h}}
\newcommand{\rmi}{{\rm i}}
\newcommand{\rmj}{{\rm j}}
\newcommand{\rmk}{{\rm k}}
\newcommand{\rml}{{\rm l}}
\newcommand{\rmm}{{\rm m}}
\newcommand{\rmn}{{\rm n}}
\newcommand{\rmo}{{\rm o}}
\newcommand{\rmp}{{\rm p}}
\newcommand{\rmq}{{\rm q}}
\newcommand{\rmr}{{\rm r}}
\newcommand{\rms}{{\rm s}}
\newcommand{\rmt}{{\rm t}}
\newcommand{\rmu}{{\rm u}}
\newcommand{\rmv}{{\rm v}}
\newcommand{\rmw}{{\rm w}}
\newcommand{\rmx}{{\rm x}}
\newcommand{\rmy}{{\rm y}}
\newcommand{\rmz}{{\rm z}}
\newcommand{\BBA}{{\mathbb A}}
\newcommand{\BBB}{{\mathbb B}}
\newcommand{\BBC}{{\mathbb C}}
\newcommand{\BBD}{{\mathbb D}}
\newcommand{\BBE}{{\mathbb E}}
\newcommand{\BBF}{{\mathbb F}}
\newcommand{\BBG}{{\mathbb G}}
\newcommand{\BBH}{{\mathbb H}}
\newcommand{\BBI}{{\mathbb I}}
\newcommand{\BBJ}{{\mathbb J}}
\newcommand{\BBK}{{\mathbb K}}
\newcommand{\BBL}{{\mathbb L}}
\newcommand{\BBM}{{\mathbb M}}
\newcommand{\BBN}{{\mathbb N}}
\newcommand{\BBO}{{\mathbb O}}
\newcommand{\BBP}{{\mathbb P}}
\newcommand{\BBQ}{{\mathbb Q}}
\newcommand{\BBR}{{\mathbb R}}
\newcommand{\BBS}{{\mathbb S}}
\newcommand{\BBT}{{\mathbb T}}
\newcommand{\BBU}{{\mathbb U}}
\newcommand{\BBV}{{\mathbb V}}
\newcommand{\BBW}{{\mathbb W}}
\newcommand{\BBX}{{\mathbb X}}
\newcommand{\BBY}{{\mathbb Y}}
\newcommand{\BBZ}{{\mathbb Z}}

\newcommand{\real}{{\mathbb{R}}}
\newcommand{\comp}{{\mathbb{C}}}
\newcommand{\realR}{\mathbb{R}}

\newcommand{\SA}{{\mathcal A}}
\newcommand{\SB}{{\mathcal B}}
\newcommand{\SC}{{\mathcal C}}
\newcommand{\SD}{{\mathcal D}}
\newcommand{\SE}{{\mathcal E}}
\newcommand{\SF}{{\mathcal F}}
\newcommand{\SG}{{\mathcal G}}
\newcommand{\SH}{{\mathcal H}}
\newcommand{\SI}{{\mathcal I}}
\newcommand{\SJ}{{\mathcal J}}
\newcommand{\SK}{{\mathcal K}}
\newcommand{\SL}{{\mathcal L}}
\newcommand{\SM}{{\mathcal M}}
\newcommand{\SN}{{\mathcal N}}
\newcommand{\SO}{{\mathcal O}}
\newcommand{\SP}{{\mathcal P}}
\newcommand{\SQ}{{\mathcal Q}}
\newcommand{\SR}{{\mathcal R}}
\newcommand{\SSS}{{\mathcal S}}
\newcommand{\ST}{{\mathcal T}}
\newcommand{\SU}{{\mathcal U}}
\newcommand{\SV}{{\mathcal V}}
\newcommand{\SW}{{\mathcal W}}
\newcommand{\SX}{{\mathcal X}}
\newcommand{\SY}{{\mathcal Y}}
\newcommand{\SZ}{{\mathcal Z}}
\newcommand{\BFA}{{\bf A}}
\newcommand{\BFB}{{\bf B}}
\newcommand{\BFC}{{\bf C}}
\newcommand{\BFD}{{\bf D}}
\newcommand{\BFE}{{\bf E}}
\newcommand{\BFF}{{\bf F}}
\newcommand{\BFG}{{\bf G}}
\newcommand{\BFH}{{\bf H}}
\newcommand{\BFI}{{\bf I}}
\newcommand{\BFJ}{{\bf J}}
\newcommand{\BFK}{{\bf K}}
\newcommand{\BFL}{{\bf L}}
\newcommand{\BFM}{{\bf M}}
\newcommand{\BFN}{{\bf N}}
\newcommand{\BFO}{{\bf O}}
\newcommand{\BFP}{{\bf P}}
\newcommand{\BFQ}{{\bf Q}}
\newcommand{\BFR}{{\bf R}}
\newcommand{\BFS}{{\bf S}}
\newcommand{\BFT}{{\bf T}}
\newcommand{\BFU}{{\bf U}}
\newcommand{\BFV}{{\bf V}}
\newcommand{\BFW}{{\bf W}}
\newcommand{\BFX}{{\bf X}}
\newcommand{\BFY}{{\bf Y}}
\newcommand{\BFZ}{{\bf Z}}
\def\bbR{\mbox{\bbfnt R}}
\newcommand{\vspaceAnkit}{\vspace{-3pt}}
\newcommand{\shiftq}{{\textbf{\textrm{q}}}}

\newcommand{\ddt}{\dfrac{\rmd}{\rmd t}}
\newcommand{\ddx}{\dfrac{\rmd}{\rmd x}}











% \newcommand{\SCA}{{\mathscr{A}}}
% \newcommand{\SCB}{{\mathscr{B}}}
% \newcommand{\SCU}{{\mathscr{U}}}

\newcommand{\Star}{^{*}}

%\newcommand{\scM}{{\mathscr M}}
\newcommand{\scM}{{\mathcal M}}

\newcommand{\neweqline}{\ensuremath{\nn \\ &\quad }}


\newcommand{\framedot}[2]{\stackrel{{\rm #1}\bullet}{#2}}
\newcommand{\frameddot}[2]{\stackrel{{\rm #1}\bullet \bullet}{#2}}

\newcommand{\polar}[1]{\ensuremath{r_{#1}e^{\jmath \theta_{#1}}}}

\renewcommand{\matl}{\begin{bmatrix}}
\renewcommand{\matr}{\end{bmatrix} }


\newcommand{\rotation}[2]{ {\substack{#1 \\ \boldsymbol{\longrightarrow} \\ #2}} }

\newcommand{\adj}{{\rm adj  } }
\newcommand{\rank}{{\rm rank  } }
\newcommand{\zeros}{{\rm zeros} }
\newcommand{\poles}{{\rm poles} }

\newcommand{\realization}[4]{
\left[\begin{array}{c|c}
    #1 & #2\\
    \hline
    #3 & #4 \end{array}\right]
}


\newcommand{\EndProofInEq}{\tag*{\mbox{\qed}}}


% \usepackage{multicol,lipsum}
% The following packages can be found on http:\\www.ctan.org
%\usepackage{graphics} % for pdf, bitmapped graphics files
%\usepackage{epsfig} % for postscript graphics files
%\usepackage{mathptmx} % assumes new font selection scheme installed
%\usepackage{times} % assumes new font selection scheme installed
%\usepackage{amsmath} % assumes amsmath package installed
%\usepackage{amssymb}  % assumes amsmath package installed

% \usepackage[toc,page]{appendix}

\usepackage[style=ieee ,sorting=none,maxbibnames=99,giveninits, doi=false]{biblatex}
\addbibresource{b.bib}


\title{\LARGE \bf
Zeros in the State-Space Realization
}

\title{\LARGE \bf
Zeros are the Poles of the Zero Dynamics in Linear Systems
}


\title{\LARGE \bf
Computing Invariant Zeros of a Linear System
\\
Using State-Space Realization
}



%\author{ \parbox{3 in}{\centering Huibert Kwakernaak*
%         \thanks{*Use the $\backslash$thanks command to put information here}\\
%         Faculty of Electrical Engineering, Mathematics and Computer Science\\
%         University of Twente\\
%         7500 AE Enschede, The Netherlands\\
%         {\tt\small h.kwakernaak@autsubmit.com}}
%         \hspace*{ 0.5 in}
%         \parbox{3 in}{ \centering Pradeep Misra**
%         \thanks{**The footnote marks may be inserted manually}\\
%        Department of Electrical Engineering \\
%         Wright State University\\
%         Dayton, OH 45435, USA\\
%         {\tt\small pmisra@cs.wright.edu}}
%}

% \author{Jhon Manuel Portella Delgado and Ankit Goel% <-this % stops a space
% % \thanks{The University of Maryland, Baltimore County}% <-this % stops a space
% \thanks{Jhon Manuel Portella Delgado is a Ph.D. student in the Department of Mechanical Engineering, University of Maryland, Baltimore County, 1000 Hilltop Circle, Baltimore, MD 21250. {\tt\small jportel1@umbc.edu}}%
% \thanks{Ankit Goel is an Assistant Professor in the Department of Mechanical Engineering, University of Maryland, Baltimore County,1000 Hilltop Circle, Baltimore, MD 21250. {\tt\small ankgoel@umbc.edu }}%
% }

\author{Jhon Manuel Portella Delgado and Ankit Goel% <-this % stops a space
% \thanks{The University of Maryland, Baltimore County}% <-this % stops a space
\thanks{Jhon Manuel Portella Delgado is a Ph.D. student in the Department of Mechanical Engineering, University of Maryland, Baltimore County, 1000 Hilltop Circle, Baltimore, MD 21250. 
{\href{mailto:jportel1@umbc.edu}{\tt\small jportel1@umbc.edu}}}%
\thanks{Ankit Goel is an Assistant Professor in the Department of Mechanical Engineering, University of Maryland, Baltimore County,1000 Hilltop Circle, Baltimore, MD 21250. 
{\href{mailto:ankgoel@umbc.edu}{\tt\small ankgoel@umbc.edu} }}%
}
% \usepackage{graphicx}      % include this line if your document contains figures
% % 
% \usepackage{textcomp}
% \usepackage{calc}
% \usepackage{mathtools}
% \usepackage{xparse}
% \usepackage{amssymb}
% \usepackage{amsmath}
% \usepackage{cases}
% \usepackage{mathtools}
% \usepackage{cuted}

\begin{document}



\maketitle
% \thispagestyle{empty}
% \pagestyle{empty}

%%%%%%%%%%%%%%%%%%%%%%%%%%%%%%%%%%%%%%%%%%%%%%%%%%%%%%%%%%%%%%%%%%%%%%%%%%%%%%%%
\begin{abstract}
%
It is well known that zeros and poles of a single-input, single-output system in the transfer function form are the roots of the transfer function's numerator and the denominator polynomial, respectively.  
% 
However, in the state-space form, where the poles are a subset of the eigenvalue of the dynamics matrix and thus can be computed by solving an eigenvalue problem, the computation of zeros is a non-trivial problem. 
% 
% However, the computation of zeros in the state-space form is non-trivial. 
% In fact, the zeros are computed by solving a generalized eigenvalue problem. 
% Zeros of a single-input, single-output system, given by the roots of the transfer function numerator, can be computed by solving a generalized eigenvalue problem based on its state-space representation. 
% 
% This paper shows that the zeros can be computed by solving a more straightforward eigenvalue problem. 
% 
This paper presents a realization of a linear system that allows the computation of invariant zeros by solving a simple eigenvalue problem. 
% 
% 
The result is valid for square multi-input, multi-output (MIMO) systems, is unaffected by lack of observability or controllability, and is easily extended to wide MIMO systems. 
% where the computed zeros are the invariant zeros. 
% 
% The main result is validated through several numerical examples. 
Finally, the paper illuminates the connection between the zero-subspace form and the normal form to conclude that \textit{zeros are the poles of the system's zero dynamics}.



% 
% This paper roper, we extend and validate the technique to strictly proper as well as MIMO systems through several numerical examples. 
\end{abstract}

% The main result of the paper is validated with several numerical examples. 
% Finally, we validate the main result with several numerical SISO as well as MIMO examples.
% The MIMO results show that the main result of this paper is extendible to the MIMO case. 
% We  show that the zero-subspace form is equivalent to the normal form. 
% We consider the problem of computing zeros of a system using its state-space realization. 
% % 
% It is well-known that the poles of a system are given by the eigenvalues of the dynamics matrix of its state-space realization, however, the computation of the zeros requires the solution of a generalized eigenvalue problem. 
% % 
% This paper shows that the zeros, like poles, can also be computed by solving an eigenvalue problem. 
% % 
% In particular, we present a new realization, called the \textit{zero-subspace form}, in which the zeros of the system are the eigenvalues of a special partition of the transformed dynamics matrix.
% % 
% Furthermore, we show the connection between the zero-subspace form and the well-known normal  to conclude that \textit{zeros of a system are in fact poles of its zero dynamics.}
% 
% Although we consider a SISO linear system to motivate the zero-subspace form, numerical examples confirm that the 


% It is well-known that the poles of a transfer function are given by the eigenvalues of the dynamics matrix of its state-space realization, however, the computation of the zeros requires the solution of a generalized eigenvalue problem. 
% 
% In this paper, we present a new realization, called the \textit{zero-subspace form} in which the zeros of the system are the eigenvalues of a special partition of the transformed dynamics matrix.
% 
% Specifically, we show that  a detailed proof
% We show the connection between the zero-subspace form and the well-known normal form. 
% 
% Specifically, it is shown that the zeros of the system are the eigenvalues of the zero dynamics of the system. 
% 
% Specifically, it is shown that the \textit{zeros of a system are the poles of its zero dynamics.}
% 

% Furthermore, we also show the connections between the zero dynamics, zeros, and zeroing inputs.


%%%%%%%%%%%%%%%%%%%%%%%%%%%%%%%%%%%%%%%%%%%%%%%%%%%%%%%%%%%%%%%%%%%%%%%%%%%%%%%%

% \maketitle
% \thispagestyle{empty}
% \pagestyle{empty}

% \clearpage
\section{INTRODUCTION}

Zeros are fundamental in the study of systems and control theory. 
While the poles affect the system stability, transients, and convergence rate, zeros affect the undershoot, overshoot, and zero crossings
\cite{macfarlane1976poles,desoer1974zeros,tokarzewski2006finite}.
Furthermore, nonminimum phase zeros, which are zeros in the open-right-half-plane, limit performance and bandwidth due to limited gain margin, exacerbate the tradeoff between the robustness and achievable performance of a feedback control system, and prevent input-output decoupling \cite{hoagg2007nonminimum,havre2001achievable,wonham1970decoupling}. 
Precise knowledge of zeros is thus crucial in the design of reliable control and estimation systems. 

In the transfer function representation of single-input, single-output (SISO) linear systems, 
zeros and poles are the roots of the numerator and the denominator polynomials, respectively.
% 
In the state-space form of a SISO as well as a MIMO linear system, the system's poles are a subset of the eigenvalues of the dynamics matrix.
% pole computation is thus an eigenvalue problem.  
% 
However, the zeros are not readily apparent in the state-space form. 
% 
Invariant zeros of a MIMO system with a state-space realization $(A,B,C,D)$, which are the complex numbers for which the rank of the Rosenbrock system matrix  
\begin{align}
    \SZ (\lambda) 
        =
            \matl 
                \lambda I - A & B \\C & -D
            \matr,
\end{align}
% where $(A,B,C,D)$ is a state-space realization of the system,
drops, are a subset of the generalized eigenvalues of the Rosenbrock system matrix \cite{laub1978calculation,tokarzewski2011invariant,tokarzewski2009zeros}. 
% 
% Since \eqref{eq:state_x}, \eqref{eq:output_y} is square, the zeros are given by the generalized eigenvalues of the systems's Rosenbrock matrix 
% 
% Since $\SZ(\lambda)$ is a regular pencil, its generalized eigenvalues are the zeros of the system \cite{laub1978calculation}.
% 
% The zeros of \eqref{eq:state_x}, \eqref{eq:output_y} thus can be alternatively obtained by computing the generalized eigenvalues of $\SZ(\lambda)$.


The generalized eigenvalues of the Rosenbrock system matrix can be computed by decomposing it into the generalized Schur form as shown in \cite{emami1982computation, misra1994computation,misra1989computation,misra1989minimal}. 
% 
However, this approaches yields extraneous zeros, which are removed heuristically. 
% The generalized Schur decomposition 
% Several computational techniques, all of which involve the decomposition of the Rosenbrock system matrix into the generalized Schur form, to solve this problem are described in \cite{emami1982computation, misra1994computation,misra1989computation,misra1989minimal}. 
% 
% In this paper, we consider the problem of computing the zeros of the system
% \begin{align}
%     \lambda x &= A x + Bu, \label{eq:state_x} \\
%     y &= Cx + Du, \label{eq:output_y}
% \end{align}
% where $x\in \BBR^{l_x}$ is the state, 
% $u \in \BBR$ is the input,  
% $y \in \BBR$ is the output, 
% $A,B,C,D$ are real matrices of appropriate dimensions, 
% and 
% the operator $\lambda$ is the time-derivative operator or the forward-shift operator. 
% 
Alternatively, since zeros are invariant under output feedback and the closed-loop poles approach the zeros of the system under high-gain output feedback, 
the zeros of $(A,B,C,D)$ are a subset of the eigenvalues of $\lim_{K \to \infty } A+ BK(I- D K)^{-1} C$ \cite{davison1978algorithm,garbow1977matrix}.
% 
% this approach can be used to approximate the zeros of the system \cite{davison1978algorithm}.  
% The zeros of $(A,B,C,D)$ are thus a subset of the eigenvalues of $\lim_{\rho \to \infty } A+\rho BK(I- \rho D K)^{-1} C.$
% The 
% 
Similar to the approach based on the Schur decomposition, this approach also yields extraneous zeros. 
Furthermore, both approaches are computationally expensive.
% 




% % \subsection{Invariant Zeros from invariant subspaces theory}
    On the other hand, \cite{basile1992controlled,morris2010invariant} studied a geometric approach to finding the invariant zeros of a linear system. Let $\mathcal{Z}$ be an $(A,\text{Range}(B))-$controlled invariant subspace. The set of invariant zeros is the set of $\lambda$ for which
    \begin{align}
        \matl
            \lambda I -A & B\\
            C & 0
        \matr
        \matl
            x\\
            u
        \matr=
        \matl
            0\\
            0
        \matr,
    \end{align}
    has a solution for some scalar $u_0$ and nonzero $x_0 \in \mathcal{F}\cap\mathcal{Z}$. Suppose that $\lambda$ is an eigenvalue of $(A,\text{Range}(B))|_{\mathcal{Z}}$, and an invariant zero of the system. Then,
    \begin{align}
        \matl
            \lambda I -A & B\\
            C & 0
        \matr
        \matl
            x_0\\
            u_0
        \matr=
        \matl
            0\\
            0
        \matr.
    \end{align}
    This shows that
    \begin{align}
        Cx_0=0,
    \end{align}
    which implies that $x_0 \in \text{Null}(C)$, and also
    \begin{align}
        \lambda x_0 + Bu_0 = Ax_0,
    \end{align}
    so $x_0$ is contained in a $(A,\text{Range}(B))-$controlled invariant subspace contained in $\text{Null}(C)$. Moreover, we want the supremum of the $(A,\text{Range}(C))$ contained in $\text{Null}(C)$. Then,
    \begin{align}
        x_0 \in \max \mathcal{V}(A,\text{Range}(B),\text{Null}(C)).
    \end{align}
    Furthermore, $\mathcal{V}^*$ has an infimum as well, which is
    \begin{align}
        R_{V^*}=V^*\cap S^*,
    \end{align}
    which represents the assignable subspace of $V^*,$ (or in other words the subspace of $V^*$ for which $\text{Range}\in \mathcal{V}^*$). From the complementability property of $\mathcal{V}^*$ we have
    \begin{align}
        R_{\mathcal{V}^*} + \mathcal{V} = \mathcal{V}^*.
    \end{align}
    where $\mathcal{V}$ is also an $(A,\text{Range}(B))-$controlled invariant subspace that represents the unassignable subspace of $\mathcal{V}$ such that,
    \begin{align}
        (A + BF)\mathcal{V}\subseteq\mathcal{V}.
    \end{align}
    Nevertheless, its implementation involves the codification of many functions, which also makes its interpretation difficult for readers who are not familiar with differential geometry concepts.\\

% \textcolor{red}{
% \begin{enumerate}
%     \item \cite{sanjeevini2022zero} explores the properties of the zero dynamics and output zeroing in input-output models, which have the form of transfer functions but like state space models operate in time. 
%     \item \cite{sanjeevini2020counting} studies a method to count transmission and infinite zeros in terms of the defect of a block Toeplitz matrix
% \end{enumerate}
% }

% \textbf{In fact, the computation of zeros requires the solution of a generalized eigenvalue problem, which is a considerably more difficult problem \cite{emami1982computation, misra1994computation, berg1997unfolding}.}


% 
% In particular, the invariant zeros of 
% However, all of these techniques require  



% One way is to do high gain feedback and use the zero invariancee principal. however, we get a larger set and the algorithm has to be run multiple times to find the smallest subset which contains the zeros of the system. Moreoever, the zeros are not exact. 

% shows that the eigenvalues of the systems's Rosenbrock matrix are the zeros of the system, however, the eigenvalues are the superset of the zeros. 



% \textbf{Van dooren} shows that there exists unitary transformations $P$ and $Q$ that convert the system matrix into the form 
% \begin{align}
%     PS(s)Q 
%         = 
%             \matl
%                 A_\rmr-sB_\rmr & * & * & *\\
%                 0 & A_\rmf-sB_\rmf & * & *\\
%                 0 & 0 & A_\rmi-sB_\rmi & *\\
%                 0 & 0 & 0 & A_\rml-sB_\rml
%             \matr.
% \end{align}
% The invariant zeros are then the generalized eigenvalues of $(A_\rmf, B_\rmf)$.
% However, we still need to solve an eigenvalue problem.
 


% finding the generlizd $\det \SZ(s)=0$ is eqio

% written in the transfer function form, zeros are simply the roots of the numerator polynomial, while the poles are the roots of the denominator polynomial.
% 
% In a state-space realization of the transfer function, the poles are the eigenvalues of the dynamics matrix, therefore computation of poles is an eigenvalue problem.
% On the other hand, the computation of zeros is a generalized eigenvalue problem, which is motivated by the invariant zeros of the multi-input, multi-output (MIMO) systems and is a considerably more difficult problem \cite{emami1982computation, misra1994computation, berg1997unfolding}.
% 
% The generalized eigenvalue problem-based approach to computing zeros 

% zeros are important for bla bla bla

% Zerros can be computed using these techniques

% This paper shows a new method to compute the zeros

% In SISO systems, the zeros of the system are simply the roots of the numerator polynomial. 

% In this paper, we present a new approach to computing zeros using the state-space realization of a linear system.
% % 
% In particular, we propose a change of basis that collects all the zeros of the system 

In this paper, we present a technique to compute the invariant zeros of a MIMO system that reduces the problem to an eigenvalue problem instead of a generalized eigenvalue problem. 
% 
Since the paper's main result does not depend on the minimality of the realization, the computed zeros are indeed the invariant zeros of the state-space realization. 
% 
This technique is motivated by and is closely related to the normal form of a dynamic system \cite{isidori1985nonlinear,Khalil:1173048}. 
% 
Specifically, we construct a diffeomorphism that isolates the zeros in a partition of the transformed dynamics matrix. 
Next, we show that the zeros of the system are precisely the eigenvalues of this partition.
Finally, we show that the partitioned system is in fact the zero dynamics of the system. 
% 
This observation is stated in \cite[p.~514]{Khalil:1173048}, however, the numerator polynomial of the transfer function is used to construct the state-space realization of the zero dynamics. 
% 
In contrast, in this paper, we construct the zero dynamics as well as compute the zeros using the system's state-space realization, without requiring its transfer function. 
% 
% 
 

A similar geometric approach, based on differential geometry, is described in \cite{basile1992controlled,morris2010invariant} to compute the invariant zeros of the system by solving an eigenvalue problem.  
In contrast, in this paper, we construct the eigenvalue problem, whose solution provides the invariant zeros, by constructing a simple state transformation matrix.
Furthermore, we present a simplified proof that does not require differential geometry concepts and is instead based on simple algebra.

% The contributions of this paper are 
% i) the construction of the zero-subspace form that converts the problem of computing zeros into an eigenvalue problem, 
% ii) a simplified proof that does not require differential geometry concepts and is instead based on simple algebra.


% a change of basis, where the zeros are the eigenvalues of a special partition of the dynamics matrix in the new basis. 

% Specifically, we use the fact that the zeros of the system are the eigenvalues of the zero dynamics of the system.
% % which is obtained by transforming the system into its normal form. 
% % 
% This fact is stated on \cite[p.~514]{Khalil:1173048}, however, no proof is provided. 
% % and a heuristic proof is provided. 
% In this paper, we prove this fact and provide an algorithm to compute the zeros without requiring the solution of a generalized eigenvalue problem. 
% % In this paper, we provide a linear algebra based proof without invoking the system theory.  
% Specifically, we present an algorithm to compute the transformation matrix that transforms the system into its normal form.
% Furthermore, we show that the transformation matrix is invertible, that is, the transformation matrix is a diffeomorphism.





% In this paper, we show that, like pole computation, zero computation is also an eigenvalue problem.
% 

% % 
% % 
% The contribution of this paper is thus the development of the \textit{zero-subspace form} (ZSF) for a SISO linear system. 
% % and 
% % 
% We show that the zeros of the system are the eigenvalues of a partition of the dynamics matrix of the ZSF.
% % 
% Furthermore, we show the connection between the ZSF and the normal form of a linear system \cite{isidori1985nonlinear}. 
% % 
% In particular, we show the connection of the zeros with the zero dynamics of the system; that is,  the zeros of the system are precisely the eigenvalues of the zero dynamics.
% 
% can also be obtained by transforming the system to its normal form, which expresses the system's zero dynamics explicitly.
% 
% Moreover, we show that, if the basis of the zero subspace and the basis of the zero dynamics is chosen to be the same, then the ZSF and normal form are exactly equal. 


% \begin{enumerate}
%     \item eig are Invariants zeros, not just transmission zeros. No minimal realization is required.
%     \item motiviating zeros from linear algebra concepts
%     \item zero computation from pencils
% \end{enumerate}

% the zeros of the systems are the poles of the zero dynamics. 



% \textbf{the zeros are the poles of the zero dynamics}

% \textcolor{red}{\cite{emami1982computation,misra1994computation} developed an algorithm for the computation of zeros based on the transformation of a MIMO system to the upper Schur form using unitary matrices $Q$ and $P$. This is performed to reduce the analysis only to the subsystem $A_l-sB_l$ that contains the finite zeros of the former system.
% \begin{align}
%     PS(s)Q = \matl
%         A_r-sB_r & * & * & *\\
%         0 & A_f-sB_f & * & *\\
%         0 & 0 & A_i-sB_i & *\\
%         0 & 0 & 0 & A_l-sB_l
%     \matr
% \end{align}
% Then, the QZ algorithm of \cite{laub1978calculation} is performed in the reduced system $A_l-sB_l$\\
% \hfill \\
% \cite[p.~512-515]{Khalil:1173048} presents an interpretation of the zeros of a linear system, where transmission zeros are characterized to be the eigenvalues of the matrix $A_c$ and the unobservable zeros of the system are the eigenvalues of $A_0$.
% }


Although all results presented in the paper are applicable to square MIMO systems as well, due to page limits, we restrict the proofs to the case of SISO systems. 
% 
The technique to compute the zeros can be easily extended to wide MIMO systems as shown in an example in Section \ref{sec:examples}.
However, the technique does not work in the case of tall MIMO systems since the construction of the zero-subspace form yields a singular transformation matrix. 
Ad-hoc techniques such as modifying the rows of the transformation matrix to make it nonsingular yield mixed and inconclusive results. 
The paper thus does not consider the case of tall MIMO systems.

The paper is organized as follows. 
Section \ref{sec:notation} presents the notation used in this paper, 
Section \ref{sec:ZSF} introduces the zero-subspace form of a strictly proper linear system, 
Section \ref{sec:main_result} presents and proves the main result of the paper, 
% Section \ref{sec:ZerosTF} presents a connection with the classical notion of a transfer function zero, 
Section \ref{sec:normal_form} shows the connection of the zero-subspace form with the normal form, 
% Section \ref{sec:MIMO_Zeros} extends the main result of the paper to MIMO systems, 
and 
Section \ref{sec:examples} presents numerical examples to confirm the main result of this paper.
% 
Finally, the paper concludes with a discussion in Section \ref{sec:conclusion}.





\section{Notation}
\label{sec:notation}
Let $A \in \BBR^{n\times m}.$
Then, $A_{[i,j]}$ is the matrix obtained by removing the $i$th row and $j$th column of $A.$
Note that, if $i=0,$ then only the $j$th column is removed.
Similarly, if $j=0,$ then only the $i$th row is removed.
That is, $A_{[0,0]} = A.$
% 
The set of integers between $n$ and $m$, where $n\leq m,$ that is, $\{n, n+1, \ldots, m\},$ is denoted by $\{ \iota \}_{n}^{m} .$ 
% 
$0_{n \times m}$ denotes the $n \times m$ zero matrix and 
$I_n$ denotes the $n\times n$ identity matrix. 


% \section{Zero}
% Consider the linear system
% \begin{align}
%     \dot x &= A x + Bu, \label{eq:state_x} \\
%     y &= Cx. \label{eq:output_y}
% \end{align}
% The zero-subspace form of the system \eqref{eq:state_x}-\eqref{eq:output_y} is
% \begin{align}
%     \dot \chi &= \SA \chi + \SB u, \label{eq:state_chi} \\
%     y &= \SC \chi , \label{eq:output_y_chi}
% \end{align}
% where $\chi = T x$ and $T$ is given by xxxx.
% Furthermore, 
% \begin{align}
%     \dot \eta &= \SA_\eta \eta + \SA_{\eta \xi} \xi, \\
%     \dot \xi &= \SA_{\xi \eta} \eta + \SA_{\xi} \xi + \SB_\xi u, \\
%     y &= \SC_\xi \xi , \label{eq:output_y_chi}
% \end{align}



% \begin{remark}
%     There are several mistaken assumptions in the previous discussion. 
%     The arguments are hand wavy. 
%     A signal going to zero doesnt imply that its derivative goes to zero. 
% \end{remark}

% The paper's main result shows that the eigenvalues of $A_\eta$ are exactly the zeros of the system. 



% % \clearpage
\section{zero-subspace form}
\label{sec:ZSF}
This section introduces the \textit{zero-subspace form} of a linear system.
% of a SISO linear system.
% 
The zero-subspace form is a realization in which the zeros of the system are the eigenvalues of a partition of the dynamics matrix.
% of the system represented in the zero-subspace form. 
% 
% Specifically, zeros are the eigenvalues of the upper
% 

% 



% In the following, we construct a state-transformation matrix to transform a given state-space realization of a linear system to its zero-subspace form.
% 
% and show that the zeros of the system, which are invariant under state-transformation, are the eigenvalues of a partition of the transformed dynamics matrix.


Consider a linear system
\begin{align}
    \lambda x &= A x + Bu, \label{eq:state_x} \\
    y &= Cx, \label{eq:output_y}
\end{align}
where $x\in \BBR^{l_x}$ is the state, 
$u \in \BBR^{l_u}$ is the input, 
$y \in \BBR^{l_y}$ is the output, and 
the operator $\lambda$ is the time-derivative operator or the forward-shift operator. 
The relative degree of $y_i$ is denoted by $\rho_i,$ and the relative degree of $y,$ defined as $\sum_i^{l_y} \rho_i,$ is denoted by $\rho.$
% 
% Consider a linear system
% \begin{align}
%     \lambda x &= A x + Bu, \label{eq:state_x} \\
%     y &= Cx, \label{eq:output_y}
% \end{align}
% where $x\in \BBR^{l_x}$ is the state, 
% $u \in \BBR$ is the input,  
% $y \in \BBR$ is the output, and 
% the operator $\lambda$ is the time-derivative operator or the forward-shift operator. 
% Let the relative degree of $y$ be denoted by $\rho.$
% 
% 

Note that \eqref{eq:state_x}-\eqref{eq:output_y} is a strictly proper system since $D = 0.$ 
Although the main result presented in this paper requires $\rho>0,$ that is, direct feedthrough term $D=0,$ it is not a restrictive condition since, without affecting the zeros of the system, the dynamic extension of the system with an additional pole renders the direct feedthrough term $D$ of the augmented system zero. 
Example \ref{exmp:D_nonzero} considers the case of an exactly proper system. 


Due to substantially lengthier and more complex proofs in the case of  multi-input, multi-output (MIMO) linear systems, we restrict our attention to single-input, single-output (SISO) linear systems in the following. 
% 
% In the following, we transform a realization of a single-input, single-output (SISO) linear system into its zero-subspace form. 
% 
However, the main result of this paper and the procedure to transform a realization of a MIMO system into its zero-subspace form is the same as that of a SISO system.
% 
% However, 



First, we construct a diffeomorphism to transform a realization into the zero-subspace form, as shown below.  
Let $\overline B \in \BBR^{l_x -1 \times l_x}$ be a full rank matrix such that $\overline B B = 0.$
% 
Note that $B \in \SN(\overline{B}),$ and thus, in MATLAB, $\overline B$ can be computed using \texttt{null(B')}.
Define 
\begin{align}
    \overline{C}
        \isdef 
            \matl
                C\\
                CA\\
                \vdots\\
                CA^{\rho-1}
            \matr \in \BBR^{\rho \times l_x}, 
    \label{eq:Cbar_def}
\end{align}
% 
and 
\begin{align}
    \overline T 
        \isdef
            \matl
                \overline{B} \\
                \overline{C}
            \matr
            \in \BBR^{(l_x-1+\rho) \times l_x}.
    \label{eq:Tbar_def}
\end{align}
\begin{proposition}
    \label{prop:Tbar_full_rank}
    Let $\overline T$ be given by \eqref{eq:Tbar_def}.
    Then, $\rank \ \overline T = l_x.$
\end{proposition}
% \begin{proof}
\textbf{Proof.}
    Note that $\overline B$ has $l_x-1$ linearly independent rows. 
    % 
    Since the relative degree of $y$ is $\rho,$ it follows that, for 
    % $$i \in \{ 1, \ldots, \rho-1 \}, $$
    $ i \in \{ \iota \}_{\iota =1}^{\rho-1}, $
    $C A^{i-1} B = 0,$ which implies that each element of $\{ C, CA, \ldots, CA^{\rho-2} \}$ is in the row range space of $\overline B.$
    % 
    Furthermore, since $ C A^{\rho-1} B \neq 0,$ it follows that $C A^{\rho-1}$ is not linearly dependent on the rows of $\overline B,$ thus implying that $\rank \ \overline T = l_x.$
    \qed
% \end{proof}


% \subsection{Transformation Matrix}
Next, define $l_z \isdef l_x - \rho$ and  
\begin{align}
    T 
        \isdef 
            \matl 
                B_\rmz \\
                \overline C
            \matr 
            \in \BBR^{l_x \times l_x},
    \label{eq:T_def}
\end{align}
where 
$B_\rmz \in \BBR^{l_z \times l_x}$ is chosen such that rows of $B_\rmz$ and rows of $\overline C$ are linearly independent. 
% 
Consequently, $ T$ is full-rank and thus invertible. 
% 
Note that $\SR(B_\rmz^\rmT) \subseteq \SR(\overline B^\rmT).$ 
% 
% 
% \subsection{Transformed State Space}
    % where $T$ is given by \eqref{eq:T_def}.
    % state-transformation matrix
    Using the diffeomorphism $T,$ transform the realization $(A,B,C)$ into the zero-subspace form $(\SA,\SB,\SC),$ which implies that  
    \begin{align}
        \SA \isdef T A T^{-1}, \quad 
        \SB 
            \isdef
                TB, \quad 
        \SC 
            \isdef 
                C T^{-1}.
        \label{eq:ZSF_SS}
    \end{align}

    Next, define $\eta \in \BBR^{l_z}$ as the first $l_z$ components of $Tx$ and 
    $\xi \in \BBR^{\rho}$ as the rest of $Tx, $ that is, 
    \begin{align}
        \matl 
            \eta \\
            \xi 
        \matr 
            =
                T x.
        \label{eq:state_transformation}
    \end{align}
    Note that since $\eta \in \SR(B_\rmz),$ we call $\SR(B_\rmz)$ the \textit{zero subspace} of the system. 

%     Substituting \eqref{eq:state_transformation} in \eqref{eq:state_x}-\eqref{eq:output_y} yields
%     \begin{align}
%         \matl 
%             \dot \eta \\
%             \dot \xi 
%         \matr 
%             &= 
%                 \SA 
%                 \matl 
%                     \eta \\
%                     \xi 
%                 \matr 
%                 + \SB u, \label{eq:state_chi} \\
%         y 
%             &=
%                 \SC \matl 
%                         \eta \\
%                         \xi 
%                     \matr  , \label{eq:output_y_chi}
%     \end{align}
%     where
%     \begin{align}
%     \SA \isdef T A T^{-1}, \quad 
%     \SB 
%         \isdef
%             TB, \quad 
%     \SC 
%         \isdef 
%             C T^{-1}.
%     \label{eq:ZSF_SS}
% \end{align}



    % With T defined as a section of Tbar, 
    % The new state space is this where A, B, C have special structures

Next, partition $\SA$ as
\begin{align}
    \SA 
        =
            \matl 
                \SA_{\eta} & \SA_{\eta \xi} \\
                \SA_{\xi \eta} & \SA_{\xi}
            \matr,
\end{align}
where $\SA_\eta$ is the $l_z \times l_z$ upper-left block,
$\SA_{\eta \xi}$ is the $l_z \times \rho$ upper-right block, 
$\SA_{\xi \eta}$ is the $\rho \times l_z$ lower-left block, and 
$\SA_\xi$ is the $\rho \times \rho$ lower-right block of $\SA.$
% 
% 

Finally, define $S \isdef T^{-1}$ and partition $S$ as $S = \matl S_\eta & S_\xi \matr,$ where $S_\eta \in \BBR^{l_x \times l_z}$ contains the first $l_z$ columns of $S$ and $S_\xi \in \BBR^{l_x \times \rho}$ contains the last $\rho$ columns of $S.$
% 
Substituting $T$ and $S$ in \eqref{eq:ZSF_SS} yields
\begin{align}
    \SA_\eta  &=  B_\rmz A S_\eta,
    \label{eq:SA_eta_def}
    \\ 
    \SA_{\eta \xi}  &= B_\rmz A S_\xi, \\ 
    \SA_{\xi \eta}  &= \overline C A S_\eta, \label{eq:SA_xi_eta_def} \\ 
    \SA_\xi  &= \overline C A S_\xi. \label{eq:SA_xi_def} 
\end{align}

As will be shown in the next section, the zeros of \eqref{eq:state_x}-\eqref{eq:output_y} are the 
% eigenvalues of the partition of $\SA$ corresponding to $\eta$ dynamics, that is, the zeros of the system are the
eigenvalues of $\SA_\eta$ given by \eqref{eq:SA_eta_def}.
% 


\subsection{Sparse Structure of the zero-subspace form}
The following two facts about the zero-subspace form show that the matrices $\SA_{\xi\eta},$ $\SA_\xi,$ $\SB,$ and $\SC$ have \textit{sparse} structure. 

\begin{proposition}
    \label{prop:SA_xi_eta_form}
    Let $\SA_{\xi \eta}$ and $\SA_\xi$ be defined by \eqref{eq:SA_xi_eta_def} and \eqref{eq:SA_xi_def}.
    Then, 
    \begin{align}        
        A_{\xi \eta}
            &=
                \matl
                    0_{(\rho-1) \times l_z}\\
                    CA^{\rho} S_\eta
                \matr,
        \label{eq:SA_xi_eta_def_2}
    \\ 
        A_{\xi }
            &=
                \matl
                    \matl 
                        0_{(\rho-1) \times 1} & I_{\rho-1}
                    \matr \\
                    CA^{\rho} S_\xi
                \matr. 
    % \in \BBR^{\rho \times l_z},
    \label{eq:SA_xi_def_2}
    \end{align}
\end{proposition}
\textbf{Proof.}
Since $\matl B_\rmz \\ \overline C \matr \matl S_\eta & S_\xi \matr = I_{l_x},$ it follows that 
% $\overline C S_\eta = 0_{\rho \times l_z}$ and 
% $\overline C S_\xi = I_{\rho }$
% and thus
% $\overline C S_\eta = 0_{\rho \times \rho},$
% 
\begin{align}
    \overline C S_\eta         
        &=
            \matl 
                C \\
                \vdots\\
                C A^{\rho-1} 
            \matr 
            S_\eta  
        =
            0_{\rho \times l_z}, \nn
    \\
    \overline C S_\xi         
        &=
            \matl 
                C \\
                \vdots\\
                C A^{\rho-1} 
            \matr 
            S_\xi  
        =
            I_{\rho}, \nn
\end{align}
which implies that, for each 
% $j \in \{0, \ldots, \rho-1\},$ 
$ j \in \{ \iota \}_{\iota =0}^{\rho-1}, $ $C  A^{j} S_\eta = 0$
and 
$CA^j S_\xi = e_{j+1},$ where $e_j$ is the $j$th row of $I_\rho.$ 
% 
% 
Therefore, 
\begin{align*}
    \SA_{\xi \eta}  
        &=
            \overline C A S_\eta
        = 
            \matl 
                CA S_\eta\\
                \vdots\\
                C A^{\rho-1} S_\eta \\
                C A^{\rho} S_\eta
            \matr             
        =
            \matl
                0_{(\rho-1) \times l_z}\\
                C A^{\rho} S_\eta
            \matr,
        \\
    \SA_{\xi}  
        &=
            \overline C A S_\xi
        = 
            \matl 
                CA S_\xi\\
                \vdots\\
                C A^{\rho-1} S_\xi \\
                C A^{\rho} S_\xi
            \matr             
        =
            \matl
                \matl 
                    0_{(\rho-1) \times 1} & I_{\rho-1}
                \matr \\
                CA^{\rho} S_\xi
            \matr.
    \EndProofInEq
\end{align*}
    



\begin{proposition}
    Let $\SB$ and $\SC$ be defined by \eqref{eq:ZSF_SS}.
    Then, 
    \begin{align}
    \SB 
        =
            \matl 
                0_{(l_x-1)\times 1} \\
                C A^{\rho-1}B 
            \matr, \quad 
    \SC 
        =
            \matl 
                0_{1\times l_z} &
                1
                &
                0_{1\times (\rho-1)}
            \matr.
\end{align}
\end{proposition}
\textbf{Proof.}
Note that, for each 
% $j \in \{ 0, \ldots, \rho-1\},$
$ j \in \{ \iota \}_{\iota =0}^{\rho-1}, $
$C A^j B = 0,$ which implies 
\begin{align}
    \SB
        =
            TB
        =
            \matl
                B_\rmz B\\
                \overline C B
            \matr 
        =
            \matl
                0_{l_z \times l_u}\\
                \matl
                    C\\
                    CA\\
                    \vdots\\
                    CA^{\rho-1}
                \matr
                B
            \matr 
        =
            \matl
                0_{(l_x-1) \times 1}\\
                CA^{\rho-1} B
            \matr.
            \nn
\end{align}
Next, since $C$ is the $(l_z+1)$th row of $T,$ and
\begin{align*}
    C 
        =
            \SC T 
        =
            \SC  
            \matl
                B_\rmz \\
                \overline C
            \matr 
        =
            \SC
            \matl
                B_\rmz \\
                C\\
                CA\\
                \vdots\\
                CA^{\rho-1}
            \matr, 
            % \EndProofInEq
\end{align*}
it follows that $\SC = \matl
                0_{1 \times l_z}
                &
                1
                & 
                \cdots 
                & 
                0_{1 \times \rho -1}
            \matr .$ \qed
% \clearpage

In summary, the zero-subspace form of \eqref{eq:state_x}-\eqref{eq:output_y} is
\begin{align}
    \dot \eta 
        &=
            A_\eta \eta + A_{\eta \xi} \xi,
    \label{eq:ZSF_eta_dot}
    \\
    \dot \xi 
        &=
            A_{\xi \eta} \eta + A_\xi \xi + B_\xi u, 
    \label{eq:ZSF_xi_dot}
    \\
    y 
        &=
            C_\xi \xi = \xi_1,
    \label{eq:ZSF_y}
\end{align}
where 
\begin{align}
    B_\xi 
        \isdef 
            \matl 
                0_{(\rho-1)\times 1} \\
                C A^{\rho-1}B 
            \matr
            \in \BBR^{\rho}
            ,
    \quad 
    C_\xi 
        \isdef 
            \matl 
                1
                &
                0_{1\times (\rho-1)}
            \matr
            \in \BBR^{1\times \rho}.
\end{align}



\section{Main Result}
\label{sec:main_result}
This section presents the paper's main result, that is, the zeros of a SISO linear system are the eigenvalues of a partition of the dynamics matrix represented in its zero-subspace form.
% 
Specifically, Theorem \ref{theorem:zeros} shows that the zeros of the system \eqref{eq:state_x}-\eqref{eq:output_y} are exactly the eigenvalues of $A_\eta.$ 


The following lemma appears in \cite{powell2011calculating} and is used in the proof of Theorem \ref{theorem:zeros}.
\begin{lemma}
    Let $A = \matl A_{11} & A_{12} \\ A_{21} & A_{22} \matr,$ where $A_{11}$ is square and $A_{22}$ is nonsingular. 
    Then, 
    \begin{align}
        \det A
            =
                \det 
                \left( 
                    A_{11}  - A_{12} A_{22}^{-1} A_{21}
                \right)
                \det A_{22}.
    \end{align}
\end{lemma}


% \subsection{Useful Properties of the zero-subspace form}
The following two facts about the zero-subspace form are 
% used in the proof of the main result of the paper. 
 used in the proof of Theorem \ref{theorem:zeros}.

\begin{proposition}
    \label{prop:SA_xi_eta_rho0}
    Let $\SA_{\xi \eta}$ be defined by \eqref{eq:SA_xi_eta_def}.
    Then, 
    \begin{align}
        {\SA}_{\xi \eta_{[\rho,0]}} = 0_{(\rho-1) \times l_z}.
        \label{eq:SA_xi_eta_rho0}
    \end{align}
\end{proposition}
% \begin{proof}
\textbf{Proof.}
    Note that $\SA_{\xi \eta}$ is given by \eqref{eq:SA_xi_eta_def_2}.
    % ${\SA}_{\xi \eta_{[\rho,0]}}$ is obtained by 
    Removing the $\rho$th row of ${\SA}_{\xi \eta}$ yields \eqref{eq:SA_xi_eta_rho0}.
    {\hfill {\qed}}
% \end{proof}

\begin{proposition}
    \label{prop:sI-Axi_rho1}
    Let $\SA_\xi$ be defined by \eqref{eq:SA_xi_def}.
    Then, for all $s\in \BBC,$ 
    \begin{align}
        \det \left( (sI_\rho - \SA_\xi)_{[\rho,1]} \right)= \pm 1.
        \label{eq:sI-Axi_rho1}
    \end{align}
\end{proposition}
% \begin{proof}    
\textbf{Proof.}
Note that $\SA_{\xi}$ is given by \eqref{eq:SA_xi_def_2}.
Removing the $\rho$th row and the first column of $\SA_\xi$ yields
\begin{align}
    (sI_\rho - \SA_\xi)_{[\rho,1]}
        &=
            \matl
                -1 & 0 & \cdots & 0 & 0\\
                s & -1 & \cdots & 0 & 0\\
                \vdots & \vdots & \ddots & \vdots & \vdots\\
                0 & 0 & \cdots & -1 & 0\\
                0 & 0 & \cdots & s & -1
            \matr, \nn
\end{align}
which implies \eqref{eq:sI-Axi_rho1}.
{\hfill {\qed}}
% \end{proof}

\begin{theorem}
\label{theorem:zeros}
Consider the system \eqref{eq:state_x}-\eqref{eq:output_y}
and its zero-subspace form \eqref{eq:ZSF_eta_dot}-\eqref{eq:ZSF_y}.
Then, the zeros of the system \eqref{eq:state_x}-\eqref{eq:output_y} are the eigenvalues of $\SA_\eta$. 
\end{theorem}
\textbf{Proof.}
Note that the zeros of the system \eqref{eq:state_x}-\eqref{eq:output_y} satisfy
\begin{align}
   C \adj (sI-A) B=0. \nn 
\end{align}
Since the zeros are invariant under state transformation, it follows that the zeros also satisfies
\begin{align}
    \SC \adj (sI- \SA ) \SB=0. 
    \label{eq:zeros_adj_equation}
\end{align}
Next, note that 
\begin{align}
    \SC {\rm adj  }&(sI- \SA ) \SB\nn\\
        % \nn \\
        =&
            \matl 
                0_{1\times l_z} &
                1
                &
                0_{1\times (\rho-1)}
            \matr.\nn\\
            &\adj
            \left(
            \matl 
                sI_{l_z}-\SA_{\eta} & -\SA_{\eta \xi} \\
                -\SA_{\xi \eta} & s I_{\rho} -\SA_{\xi}
            \matr 
            \right) 
            % \nn \\
            % &\quad \quad             
            % \cdot  
            \matl 
                0_{(l_x-1)\times 1} \\ CA^{\rho-1}B
            \matr
            \nn
            \\
        =&
            (-1)^n.\nn\\
            &\det \left(
            \matl
                sI_{l_z}-\SA_{\eta} & 
                -
                % \overline{\SA}_{\eta \xi} 
                {\SA}_{{\eta \xi}_{[0,1]}}
                \\
                -
                % \overline{\SA}_{\xi \eta}
                {\SA}_{\xi \eta_{[\rho,0]}}
                &
                % \overline \SA_{\xi}
                (sI_\rho - \SA_\xi)_{[\rho,1]}
            \matr
            \right)
            CA^{\rho-1}B
        \nn \\
        =& 
            (-1)^n
        \det
            \left(
                sI_{l_z}-\SA_{\eta} -
                {\SA}_{{\eta \xi}_{[0,1]}}
                (sI_\rho - \SA_\xi)_{[\rho,1]}^{-1}
                {\SA}_{\xi \eta_{[\rho,0]}}
            \right)
            \cdot 
        \nn \\ &\quad \quad 
        \det
            (
                (sI_\rho - \SA_\xi)_{[\rho,1]}
            )
            CA^{\rho-1}B, \nn 
        \nn \\
        =& 
            (-1)^n
        \det
            \left(
                sI_{l_z}-\SA_{\eta}
            \right)
        \det
            (
                (sI_\rho - \SA_\xi)_{[\rho,1]}
            )
            CA^{\rho-1}B, \nn 
\end{align}
where 
$n \isdef {2l_x-\rho+1}.$
% 
Note that the last equality uses the fact that ${\SA}_{\xi \eta_{[\rho,0]}} = 0,$ which follows from Proposition \ref{prop:SA_xi_eta_rho0}.
% 
% Where $\overline{\SA}_{\eta \xi}$ is the $\SA_{\eta \xi}$ matrix without the $l_x-\rho-1$ column, $\overline{\SA}_{\xi \eta}$ is the $\SA_{\xi \eta}$ matrix without the last row, and $\overline{s I_{(\rho-1)} -\SA_{\xi}}$ is the $s I_{\rho} -\SA_{\xi}$ matrix without the last row and without the $l_x-\rho-1$ column.
% From \cite{powell2011calculating}, it can be seen that the equation can be computed as.
% 
Since the relative degree of the system is $\rho,$ $CA^{\rho-1}B \neq 0.$
Next, it follows from Proposition \ref{prop:sI-Axi_rho1} that, for all $s\in \BBC,$ $\det
            (
                (sI_\rho - \SA_\xi)_{[\rho,1]}
            ) 
            =
                \pm 1 
            \neq
                0.$
            % which follows 
% it follows that 
Therefore, \eqref{eq:zeros_adj_equation} is satisfied if and only if $\det (sI_{l_z}-\SA_{\eta} ) = 0,$ thus implying that the zeros of the system \eqref{eq:state_x}-\eqref{eq:output_y} are the eigenvalues of $\SA_{\eta}.$
\qed


\begin{remark}
    Note that $\SA_\eta$ has $l_z$ eigenvalues and thus \eqref{eq:state_x}-\eqref{eq:output_y} has $l_z=l_x-\rho$ zeros, which is a well-known fact.

\end{remark}

\begin{remark}
%     \section{Zeros of Transfer Function}
% \label{sec:ZerosTF}
% In this section, we use a well-known interpretation of the zeros of a transfer function to provide an alternate, \textit{albeit} simplistic motivation for the zeros being eigenvalues of the dynamics matrix partition.  
A simplistic proof of the zeros being the eigenvalues of a partition of the zero-subspace form can be motivated by the blocking property of zeros. 
% 
% The zeros of a SISO transfer function are the roots of the numerator of the transfer function, that is, ${\rm zeros}(G) = {\rm roots}(N),$ where 
% \begin{align}
%     G(s) 
%         =
%             \frac{N(s)}{D(s)}.
% \end{align}
% 
% The complete response of the system $G$ is given by
% \begin{align}
%     Y(s) 
%         =
%             G(s) U(s) 
%             +
%             \sum
% \end{align}
% 
% Consider the case where the system is asymptotically stable.
Let $z$ denote a zero of an asymptotically stable system. 
% 
% It is a well-known fact that 
% For each $z \in \zeros(G)$, 
It follows that the input $u(t) = e^{zt} u_0$ yields zero asymptotic output, that is, 
% Let $z \in \BBC$ be a zero of the system \eqref{eq:state_x}-\eqref{eq:output_y}.
% It thus follows that for the input $u(t) = e^{zt} u_0,$ 
$
    \lim_{t \to \infty} y(t)= 0,
$
% Note that this is the \textit{blocking property} of the zero.
which implies that  
$
    \lim_{t \to \infty} \xi(t)= 0.
$
% since $\xi$ is a chain of integrators and the system is assumed to be asymptotically stable.
It follows from \eqref{eq:ZSF_xi_dot} that
\begin{align}
    \lim_{t \to \infty} \SA_{\xi \eta} \eta (t) + \SB_\xi u(t) = 0.
    \label{eq:eta_u_diverging}
\end{align}
Now, consider the case where $\real(z)>0.$
% Assume that $\real(z) > 0.$
Since $u(t) = e^{zt} u_0$ diverges, it follows that $\eta(t)$ must diverge to satisfy \eqref{eq:eta_u_diverging}.
In fact, $\eta(t)$ must grow exponentially as $e^{zt}$ to satisfy \eqref{eq:eta_u_diverging}.
Since $\xi(t) \to 0,$ it follows from \eqref{eq:ZSF_eta_dot} that $\eta(t)$ grows exponentially if and only if $z \in \text{spec}(A_\eta).$

% \clearpage



\end{remark}


\begin{theorem}
\label{theorem:zeros_MIMO}
Consider the square MIMO system \eqref{eq:state_x}, \eqref{eq:output_y}
and its zero-subspace form, given by \eqref{eq:ZSF_eta_dot}-\eqref{eq:ZSF_y}.
% , and consider the equivalent realization $\realization{\SA}{\SB}{\SC}{0}$.
Then, the invariant zeros of the system \eqref{eq:state_x}, \eqref{eq:output_y} are the eigenvalues of $\SA_\eta$, that is, 
\begin{align}
        {\rm izeros } \left( \realization{A}{B}{C}{0} \right) = {\rm mspec } (\SA_\eta)\nn, 
\end{align}
where ${\rm mspec}(\SA_\eta)$ denotes the multispectrum of $\SA_\eta,$ that is, the multiset consisting of the eigenvalues of $\SA_\eta$ including their algebraic multiplicity \cite[p.~506]{bernstein2018scalar}. 

\end{theorem}

\textbf{Proof.}
The proof is omitted due to restrictions on allowable number of pages. 
% \clearpage

\section{Normal Form of a Linear System}
\label{sec:normal_form}
This section shows the relation between the zero-subspace form \eqref{eq:ZSF_eta_dot}-\eqref{eq:ZSF_y} of a system and its normal form.
% 
In the nonlinear systems theory, the normal form is used to decompose a nonlinear system into its zero dynamics and the input-output feedback linearizable dynamics \cite{isidori1985nonlinear,Khalil:1173048}. 
% 
% We show that both the zero-subspace form and the normal form decompose a linear system into the same 
This section shows that a system's normal form can also be used to deduce its zeros.
% of the system by solving an eigenvalue problem. 
% 
% In fact, the normal form is the zero-subspace form. 

% and illuminate a connection between the zero dynamics of the system and the partition of the dynamics matrix of the zero-subspace form. 
% 

% In general, the normal form of a nonlinear affine system is used to construct input-output linearizing controllers \cite{isidori1985nonlinear}.
% % 
% In the normal form, the system is decomposed into its zero dynamics and a chain of integrators. 
% % 
% Arbitrary closed-loop input-output dynamics can then be imposed by cancellation using a full-state feedback control law.

% 
% In this paper, we use the normal form to expose the zero dynamics of the system and present its connection with the zero-subspace form. 
% 
To maintain consistent notation, the procedure to transform a multi-input, multi-output (MIMO), nonlinear affine system to its normal form is summarized in Appendix \hyperref[appndx:normalForm]{A}.
% 
In the case of a SISO linear system,
$
    f(x) = Ax, \
    g(x) = B, $ and $
    h(x) = Cx,
$
and thus it follows from \eqref{eq:alpha_def}, \eqref{eq:beta_def} that
\begin{align}
    \alpha(x) 
        &=
            L_f^\rho h(x)
        =
            CA^{\rho} x,
        \\
    \beta(x) 
        &=
            L_g L_f^{\rho-1} h(x)
        =
            CA^{\rho-1} B.
\end{align}
Furthermore, 
\begin{align}
    \psi(x)
        =
            \matl 
                C \\
                CA \\
                \vdots \\
                CA^{\rho-1}
            \matr x
        . 
\end{align}
Note that $\psi(x)  = \overline C x,$ where $\overline C$ is defined by \eqref{eq:Cbar_def}.
Finally, let $\phi(x) = B_\rmn x,$ where $B_\rmn$ is chosen to satisfy $L_g \phi(x) = B_\rmn B = 0.$ 
% 
Note that 
\begin{align}
        \matl 
            \eta \\
            \xi 
        \matr 
            =
                \matl 
                    \phi(x) \\
                    \psi(x) 
                \matr
            =
                \matl 
                    B_\rmn \\
                    \overline C
                \matr x,
        \label{eq:state_transformation_NF}
    \end{align}
and thus $x = R_\eta \eta + R_\xi \xi, $ where 
\begin{align}
    \matl R_\eta & R_\xi \matr
        =
            \matl 
                B_\rmn \\
                \overline C
            \matr^{-1}.
\end{align}
% 
Note that the state-transformation matrix in \eqref{eq:state_transformation_NF} is invertible.
The proof is similar to the proof of Proposition \ref{prop:Tbar_full_rank}.
% 
The normal form of \eqref{eq:state_x}-\eqref{eq:output_y} is then
\begin{align}
    \dot \eta 
        &=
            B_\rmn A x
        =
            B_\rmn A R_\eta \eta + B_\rmn A R_\xi \xi
        ,
    \label{eq:NF_eta_dot}
        \\
    \dot \xi
        &=
            B_\rmc
            CA^{\rho} 
            A R_\eta \eta 
            + 
            (A_\rmc 
            +
            B_\rmc
            CA^{\rho} 
            A R_\xi) \xi
            +
            B_\rmc
            CA^{\rho-1} B u,
    \label{eq:NF_xi_dot}
    \\
    y
        &=
            \xi_1,
\end{align}
where 
\begin{align}
    A_{\rmc}
        &\isdef 
            \matl 
                0 & 1 & 0 & \cdots & 0  \\
                0 & 0 & 1 & \cdots & 0  \\
                \vdots & \vdots & \ddots & \ddots & \vdots \\
                0 & \vdots & \ldots & 0 & 1  \\
                0 & \vdots & \ldots & 0 & 0  \\
            \matr
        \in \BBR^{\rho \times \rho},
    % \\
    \quad 
    B_{\rmc}
        \isdef 
            \matl 
                0\\
                \vdots \\
                1
            \matr
        \in \BBR^{\rho}.
\end{align}

Note that since $B_\rmz$ and $B_\rmn$ satisfy exactly the same conditions, it follows that  $\SR(B_\rmz) = \SR(B_\rmn)$ and therefore \eqref{eq:ZSF_eta_dot}, \eqref{eq:ZSF_xi_dot} are same as \eqref{eq:NF_eta_dot}, \eqref{eq:NF_xi_dot}.
In fact, if $B_\rmn$ is chosen to be equal to $B_\rmz,$ then the zero-subspace form \eqref{eq:ZSF_eta_dot}, \eqref{eq:ZSF_xi_dot} is exactly the normal form of \eqref{eq:state_x}-\eqref{eq:output_y}.
% 
% If $B_\rmn = B_\rmz,$ then \eqref{eq:NF_xi_dot} is exactly the same as \eqref{ }
% \begin{align}
%     \SA_\eta  &= B_\rmz A S_\eta = B_\rmn A R_\eta , \\ 
%     \SA_{\eta \xi}  &= B_\rmn A S_\xi, \\ 
%     \SA_{\xi \eta}  &= \overline C A S_\eta,  \\ 
%     \SA_\xi  &= \overline C A S_\xi. 
% \end{align}

In the normal form, \eqref{eq:NF_eta_dot} is called the \textit{zero dynamics} of the system.
Theorem \ref{theorem:zeros} thus shows that \textit{zeros are the eigenvalues of the dynamics matrix of the zero dynamics.}
% ; that is, \textit{the zeros are the poles of the zero dynamics}.  

% \clearpage
% \section{MIMO Zeros}
% \label{sec:MIMO_Zeros}
% In this section, we extend the main result of this paper to the case of multi-input, multi-output systems. 
% % 
% In MIMO system, the set of eigenvalues of $\SA_\eta$ contains the invariant zeros of the MIMO system. 

% % In this section, we present several numerical examples to confirm Theorem \ref{theorem:zeros} for SISO systems and Conjecture \ref{conj:MIMO_zeros} for MIMO systems.
% % 



% % works only if D = 0.

% \begin{definition}
%     Let $G \in \BBR(s)^{l_y \times l_u}$, where $G \sim 
%         \left[
%         \begin{array}{c|c}
%             A & B \\
%             \hline
%             C & D
%         \end{array} 
%         \right]
%         % \matl
%         %     A &  B\\
%         %     C & D
%         % \matr
%         .$
%     $z \in \mathbb{C}$ is an \textbf{invariant zero} of  $G$ if 
%     \begin{align}
%         \rank \  \SZ(z) < \rank \  \SZ,
%     \end{align}
%     where $\SZ(s)$ is the Rosenbrock system matrix 
%     \begin{align}
%         \SZ(s)
%             \isdef 
%                 \matl
%                     sI-A & B\\
%                     C & -D
%                 \matr,
%     \end{align}
%     and $\rank \ \SZ \isdef  \underset{s\in \BBC} \rank \ \SZ(s).$
%     The set of invariant zeros of $G$ is denoted by ${\rm izeros} 
%                     \left[
%                     \begin{array}{c|c}
%                         A & B \\
%                         \hline
%                         C & D
%                     \end{array} 
%                     \right].$

% \end{definition}

% % \begin{definition}
% %     Let $G \in \BBR(s)^{l_y \times l_u}$, assume that $G \neq 0$, let $z \in \mathbb{C}$, and assume that $z$ is not a pole of $G$. 
% %     Then, $z$ is a \textit{transmission zero} of $G$ if 
% %     \begin{align}
% %         \rank \ G(z) < \rank \ G,
% %     \end{align}
% %     where $\rank \ G \isdef  \underset{s\in \BBC \backslash {\rm poles}(G) } \rank \ G(s).$
% %     The set of transmission zeros of $G$ is denoted by ${\rm tzeros} (G).$
% % \end{definition}


% % \subsection{\textbf{Transmission Zeros}}
% % \vspace{3mm}




% % \begin{fact}
% %     Let $G \in \BBR(s)^{l_y \times l_u}$, where $G \sim 
% %         \left[
% %         \begin{array}{c|c}
% %             A & B \\
% %             \hline
% %             C & D
% %         \end{array} 
% %         \right].$ 
% %         Then, 
% %         \begin{align}
% %             {\rm tzeros}(G)
% %                 \subseteq
% %                     {\rm izeros} 
% %                     \left[
% %                     \begin{array}{c|c}
% %                         A & B \\
% %                         \hline
% %                         C & D
% %                     \end{array} 
% %                     \right].
% %         \end{align}
% %     % $IZ \supseteq TZ$
% % \end{fact}


% \begin{conjecture}
%     \label{conj:MIMO_zeros}
%     Let $G \in \BBR(s)^{l_y \times l_u}$, where $G \sim 
%         \left[
%         \begin{array}{c|c}
%             A & B \\
%             \hline
%             C & 0
%         \end{array} 
%         \right]
%         .$
%     Let $G \sim 
%         \left[
%         \begin{array}{c|c}
%             \SA & \SB \\
%             \hline
%             \SC & 0
%         \end{array} 
%         \right]
%         ,$
%     where $\SA,$ $\SB,$ and $\SC$ are the dynamics, input, and output matrices of the zero-subspace form of $G.$
%     If $l_u \leq l_y,$ then
%     \begin{align}
%         {\rm izeros} 
%         \left[
%         \begin{array}{c|c}
%             A & B \\
%             \hline
%             C & 0
%         \end{array} 
%         \right]
%             =
%                 {\rm mspec} (\SA_\eta).
%     \end{align}
%     If $l_y < l_u,$ then
%     \begin{align}
%         {\rm izeros} 
%         \left[
%         \begin{array}{c|c}
%             A & B \\
%             \hline
%             C & 0
%         \end{array} 
%         \right]
%             \subseteq
%                 {\rm mspec} (\SA_\eta).
%     \end{align}
% \end{conjecture}

% \begin{conjecture} \label{conjecture_superset}
%     ZSF zeros are the superset of IZ for the nonsquare MIMO case
% \end{conjecture}

% \clearpage
\section{Numerical Examples}
\label{sec:examples}

This section presents several examples verifying the main result of this paper. 
These examples, listed in Table \ref{tab:examples}, show that the technique presented in the paper to compute the zeros can be extended to exactly proper systems and nonsquare MIMO systems and is not affected by pole-zero cancellations. 

\begin{table}[ht]
    \centering
    \renewcommand{\arraystretch}{1.5}
    \begin{tabular}{|c|l|}
        \hline
        Example & \hspace{6em} Remark  
        \\ \hline
            Example \ref{exmp:SISOsystem} &
            Strictly proper SISO system
        \\ \hline
            Example \ref{exmp:SISOsystem_w_zero_rd} & 
            Exactly proper SISO System
        \\ \hline
            Example \ref{exmp:SISOsystem_w_pzcancellation} &
            SISO system with pole-zero cancellation 
        \\ \hline
            Example \ref{exmp:square_MIMO_2_zeros} &
            Square MIMO system
        % \\ \hline
        %     Example \ref{exmp:tall_MIMO} &
        %     Tall MIMO system
        \\ \hline
            Example \ref{exmp:wide_MIMO} &
            Wide MIMO system
        \\ \hline
        
            
        
    \end{tabular}
    \caption{List of Examples in this paper.}
    \label{tab:examples}
\end{table}

\begin{exmp}
    \label{exmp:SISOsystem}
    \textbf{[Strictly proper SISO system.]}
    Consider the system
    \begin{align}
        G(s)=\dfrac{s^2-9s+8}{s^3+11s^2+36s+36}. \nn
    \end{align}
    Note that   $\rho=1$ and $\zeros(G) = \{ 1,8\}$.
    A controllable canonical realization of $G$ is
    \begin{align*}
        A
            &=
                \matl
                    0 & 1 & 0\\
                    0 & 0 & 1\\
                    -36 & -36 & -11\\
                \matr, \quad
        B
            =
                \matl
                    0\\
                    0\\
                    1
                \matr, 
        \\
        C
            &=
                \matl
                    8 & -9 & 1
                \matr.
    \end{align*}
    % the zero-subspace form for the transformation matrix

    Letting $B_\rmz\footnote{Computed using \texttt{null(B')'} in MATLAB.} = \matl 0 & 1 & 0\\
    1 & 0 & 0\matr,$ it follows from \eqref{eq:T_def} that 
    \begin{align}
        T
            =
                \matl
                    B_\rmz\\
                    C
                \matr
            =
                \matl
                    0 & 1 & 0\\
                    1 & 0 & 0\\
                    8 & -9 & 1
                \matr,\nn
    \end{align}
    and thus
    \begin{align*}
        \SA
            &=
                \left[\begin{array}{cc|c}
                    9 & -8 & 1\\
                    1 & 0 & 0\\
                    \hline
                    -208 & 124 & -20
                \end{array}\right]
        \SB=\left[\begin{array}{c}
            0\\
            0\\
            \hline
            1
        \end{array}\right], 
        \\
        \SC
            &=
                \left[\begin{array}{cc|c}
                    0 & 0 & 1
                \end{array}\right].
    \end{align*}
    Note that $\SA_\eta = \matl9 & -8\\
    1 & 0\matr$ and thus ${\rm mspec}(\SA_\eta) = \{1,8\},$ which confirms Theorem \ref{theorem:zeros}.
    \hfill{\huge$\diamond$}
\end{exmp}

\begin{exmp}
\label{exmp:SISOsystem_w_zero_rd}
{\textbf{[SISO System with zero relative degree.]}}
\label{exmp:D_nonzero}
Consider the system
    \begin{align}
        G(s)=\dfrac{s^3+21s^2+116s+96}{s^3+11s^2+38s+40}.\nn
    \end{align}
    Note that   $\rho=0$ and $\zeros (G) = \{ -1,-8,-12 \}$.
    % 
    Furthermore, $l_z = 3$ and thus $\SA_\eta$ is $3 \times 3,$ which implies that $\SA_\eta$ and the dynamics matrix of $G$ are similar, that is, they have same eigenvalues.
    Therefore, in the case where $\rho=0,$ eigenvalues of $\SA_\eta$ are the poles of $G.$
    % and $\poles (G) = \{ -2,-4,-5 \}$
    Note that Theorem \ref{theorem:zeros} is not applicable in this case since $D \neq 0.$

    However, consider the system 
    \begin{align}
        \overline G(s)
            \isdef 
                \dfrac{G(s)}{s}
            =
                \dfrac{s^3+21s^2+116s+96}{s^4+11s^3+38s^2+40s}.\nn
    \end{align}
    Note that $\overline G(s)$ has the same set of zeros as $G(s),$ but its relative degree $\rho=1$.
    Therefore, Theorem \ref{theorem:zeros} can be used to compute the zeros of $\overline G,$ and thus the zeros of $G.$
    % 
    % Finding the zeros from the former problem using the ZSF will be impossible since $l_x=\rho$, and thus, there is no $\SA_{\eta}$. To use ZSF, we can analyse another system with the original poles and zeros and including an extra pole to make the SISO transfer function strictly proper. From the point of view of zeros, there is no restriction on the extra pole included in the new system. Nevertheless, since the analysis of most dynamical systems does not end in the identification of zeros, it is recommended to include a much more faster stable pole. In this specific example we chose to include $-100$ as the extra pole. Then, the auxiliar systems is
    % 
    % \begin{align}
    %     G_{\text{aux}}(s)=\dfrac{s^3+21s^2+116s+96}{s^4+111s^3+1138s^2+3840s+4000}.\nn
    % \end{align}
    %     
    A controllable canonical realization of $\overline G$ is
    \begin{align*}
        A
            &=
                \matl
                    0 & 1 & 0 & 0\\
                    0 & 0 & 1 & 0\\
                    0 & 0 & 0 & 1\\
                    0 & -40 & -38 & -11\\
                \matr, \quad
        B
            =
                \matl
                    0\\
                    0\\
                    0\\
                    1
                \matr, 
        \\
        C
            &=
                \matl
                    96 & 116 & 21 & 1
                \matr.
    \end{align*}
    
    Letting $B_\rmz = \matl 0 & 1 & 0 & 0\\
    0 & 0 & 1 & 0\\
    -1 & 0 & 0 & 0\matr,$ it follows from \eqref{eq:T_def} that 
    \begin{align}
        T
            =
                \matl
                    B_\rmz\\
                    C\\
                \matr
            =
                \matl
                    0 & 1 & 0 & 0\\
                    0 & 0 & 1 & 0\\
                    -1 & 0 & 0 & 0\\
                    96 & 116 & 21 & 1
                \matr, \nn
    \end{align}
    and thus
    \begin{align}
        \SA
            &=
                \left[\begin{array}{ccc|c}
                    0 & 1 & 0 & 0\\
                    -116 & -21 & 96 & 1\\
                    -1 & 0 & 0 & 0\\
                    \hline
                    -1104 & -132 & 960 & 10
                \end{array}\right]
        \SB=\left[\begin{array}{c}
            0\\
            0\\
            0\\
            \hline
            1
        \end{array}\right], \nn
        \\
        \SC
            &=
                \left[\begin{array}{ccc|c}
                    0 & 0 & 0 & 1
                \end{array}\right].\nn
    \end{align}
    Note that $\SA_\eta = \matl
        0 & 1 & 0 \\
        -116 & -21 & 96\\
        -1 & 0 & 0
    \matr$ and thus ${\rm mspec}(\SA_\eta) = \{-12,-8,-1\},$ which confirms Theorem \ref{theorem:zeros}.
    \hfill{\huge$\diamond$}
    
\end{exmp}


\begin{exmp}
\label{exmp:SISOsystem_w_pzcancellation}
{\textbf{[SISO System with pole-zero cancellation.]}}
Consider the system
    \begin{align}
        G(s)=\dfrac{s+5}{s^3+10s^2+31s+30}.\nn
    \end{align}
    Note that   $\rho=2,$  $\zeros (G) = \{ -5 \},$ and $\poles (G) = \{-2, -3, -5\}.$
    A controllable canonical realization of $G$ is
    \begin{align*}
        A
            &=
                \matl
                    0 & 1 & 0\\
                    0 & 0 & 1\\
                    -30 & -31 & -10\\
                \matr, \quad
        B
            =
                \matl
                    0\\
                    0\\
                    1
                \matr, 
        \\
        C
            &=
                \matl
                    5 & 1 & 0
                \matr.
    \end{align*}
    % the zero-subspace form for the transformation matrix

    Letting $B_\rmz = \matl 0 & 1 & 0 \matr,$ it follows from \eqref{eq:T_def} that 
    \begin{align}
        T
            =
                \matl
                    B_\rmz\\
                    C\\
                    CA
                \matr
            =
                \matl
                    0 & 1 & 0\\
                    5 & 1 & 0\\
                    0 & 5 & 1
                \matr, \nn
    \end{align}
    and thus
    \begin{align}
        \SA
            &= \left[ \begin{array}{c|cc}
                    -5 & 0 & 1\\
                    \hline
                    0 & 0 & 1\\
                    0 & -6 & -5
            \end{array}\right]
        \SB=\left[
        \begin{array}{c}
            0\\
            \hline
            0\\
            1
        \end{array}
        \right], \nn
        \\
        \SC
            &=\left[    \begin{array}{c|cc}
                    0 & 1 & 0
                \end{array}
                \right].\nn
    \end{align}
    Note that $\SA_\eta = -5$ and thus ${\rm mspec}(\SA_\eta) = \{-5\},$ which confirms Theorem \ref{theorem:zeros}.
    This example shows that Theorem \ref{theorem:zeros} is unaffected by the state-space realization's lack of observability or controllability. 
    \hfill{\huge$\diamond$}
\end{exmp}

% \begin{exmp}{\textbf{[SISO System with zero relative degree]}.}
% Consider the system
%     \begin{align}
%         G(s)=\dfrac{s^3+21s^2+116s+96}{s^3+11s^2+38s+40}.
%     \end{align}
%     Note that   $\rho=0$, and $s=-5$ is a zero and a pole of the system $\{-2, -3, -5\}$ at the same time.
%     The controllable canonical form of $G(s)$ is
%     \begin{align}
%         A
%             &=
%                 \matl
%                     0 & 1 & 0\\
%                     0 & 0 & 1\\
%                     -30 & -31 & -10\\
%                 \matr, \quad
%         B
%             =
%                 \matl
%                     0\\
%                     0\\
%                     1
%                 \matr, 
%         \\
%         C
%             &=
%                 \matl
%                     5 & 1 & 0
%                 \matr.
%     \end{align}
%     % the zero-subspace form for the transformation matrix

%     Letting $B_\rmz = \matl 0 & 1 & 0 \matr,$ it follows from \eqref{eq:T_def} that 
%     \begin{align}
%         T
%             =
%                 \matl
%                     B_\rmz\\
%                     C\\
%                     CA
%                 \matr
%             =
%                 \matl
%                     0 & 1 & 0\\
%                     5 & 1 & 0\\
%                     0 & 5 & 1
%                 \matr,
%     \end{align}
%     and thus
%     \begin{align}
%         \SA
%             &=
%                 \matl
%                     -5 & 0 & 1\\
%                     0 & 0 & 1\\
%                     0 & -6 & -5
%                 \matr
%         \SB=\matl
%             0\\
%             0\\
%             1
%         \matr, 
%         \\
%         \SC
%             &=
%                 \matl
%                     0 & 1 & 0
%                 \matr.
%     \end{align}
%     Note that $\SA_\eta = -5$ and thus ${\rm mspec}(\SA_\eta) = \{-5\},$ which confirms Theorem \ref{theorem:zeros}.
%     \hfill{\huge$\diamond$
% \end{exmp}

% \begin{exmp}
%     \label{exmp:square_MIMO}
    
%     \textbf{[Square MIMO System.]}
%     Consider the square MIMO system
%     \begin{align}
%         G(s)=\dfrac{1}{s^2+3s+2}\matl
%             1 & 3\\
%             s-8 & s-7
%         \matr,\nn
%     \end{align}
%     with two inputs and two outputs.
%     Note that $\rho_1 =2$ and $\rho_2 =1,$ and thus $\rho = 3.$
%     % Furthermore, ${\rm izeros}={8.5}$.
%     A realization of $G$, computed with MATLAB's \texttt{ss} routine, is 
%     \begin{align*}
%         A
%             &=
%                 \matl
%                     -3 & -2 & 0 & 0\\
%                     1 & 0 & 0 & 0\\
%                     0 & 0 & -3 & -2\\
%                     0 & 0 & 1 & 0
%                 \matr, \quad
%         B
%             =
%                 \matl
%                     4 & 0\\
%                     0 & 0\\
%                     0 & 4\\
%                     0 & 0
%                 \matr, 
%         \\
%         C
%             &=
%                 \matl
%                     0 & 0.25 & 0 & 0.75\\
%                     0.25 & -2 & 0.25 & -1.75
%                 \matr.
%     \end{align*}
%     % the zero-subspace form for the transformation matrix
%     Furthermore, ${\rm izeros} 
%         \left[
%         \begin{array}{c|c}
%             A & B \\
%             \hline
%             C & D
%         \end{array} 
%         \right] = \{ 8.5\},$ which is computed using the MATLAB's \texttt{tzero} routine. 

    
%     Letting $B_\rmz = \matl 0 & -1 & 0 & 0 \matr,$ it follows from \eqref{eq:T_def} that 
%     \begin{align}
%         T
%             =
%                 \matl
%                     B_\rmz\\
%                     C_1\\
%                     C_1A\\
%                     C_2
%                 \matr
%             =
%                 \matl
%                     0 & -1 & 0 & 0\\
%                     0 & 0.25 & 0 & 0.75\\
%                     0.25 & 0 & 0.75 & 0\\
%                     0.25 & -2 & 0.25 & -1.75
%                 \matr,\nn
%     \end{align}
%     and thus
%     \begin{align*}
%         \SA
%             &=\left[ \begin{array}{c|ccc}
%                     8.5 & -14 & 2 & -6\\
%                     \hline
%                     0 & 0 & 1 & 0\\
%                     0 & -2 & -3 & 0\\
%                     16.625 & -27.5 & 0.5 & -11.5
%             \end{array}
%             \right], \quad
%         \SB=\left[
%         \begin{array}{cc}
%             0 & 0\\
%             \hline
%             0 & 0\\
%             1 & 3\\
%             1 & 1
%         \end{array}
%         \right],\nn 
%         \\
%         \SC
%             &=\left[    \begin{array}{c|ccc}
%                     0 & 1 & 0 & 0\\
%                     0 & 0 & 0 & 1
%                 \end{array}
%                 \right].\nn
%     \end{align*}
%     Note that $\SA_\eta = 8.5$ and thus ${\rm mspec}(\SA_\eta) = \{8.5\},$ which confirms
%     that the invariant zeros of the system are the eigenvalues of the upper-left partition of the dynamics matrix represented in the zero-subspace form. 
%     % conjecture \ref{conj:MIMO_zeros}.
%     \hfill{\huge$\diamond$}
% \end{exmp}

\begin{exmp}
    \label{exmp:square_MIMO_2_zeros}    
    \textbf{[Square MIMO System.]}
    Consider the square MIMO system
    \begin{align}
        G(s)=\dfrac{64}{s^3+24s^2+176s + 384}\matl
            1 & s + 4\\
            s - 2 & s-8
        \matr,\nn
    \end{align}
    with two inputs and two outputs.
    Note that $\rho_1 =2$ and $\rho_2 =2,$ and thus $\rho = 4.$
    % Furthermore, ${\rm izeros}={8.5}$.
    A realization of $G$, computed with MATLAB's \texttt{ss} routine, is 
    \begin{align*}
        A
            &=
                \matl
                    -24 & -11 & -6 & 0 & 0 & 0\\
                    16 & 0 & 0 & 0 & 0 & 0\\
                    0 & 4 & 0 & 0 & 0 & 0\\
                    0 & 0 & 0 & -24 & -11 & -6\\
                    0 & 0 & 0 & 16 & 0 & 0\\
                    0 & 0 & 0 & 0 & 4 & 0
                \matr, 
        B
            =
                \matl
                    2 & 0\\
                    0 & 0\\
                    0 & 0\\
                    0 & 4\\
                    0 & 0\\
                    0 & 0
                \matr, 
        \\
        C
            &=
                \matl
                    0 & 0 & 0.5 & 0 & 1 & 1\\
                    0 & 2 & -1 & 0 & 1 & -2
                \matr.
    \end{align*}
    % the zero-subspace form for the transformation matrix
    Furthermore, ${\rm izeros} 
        \left[
        \begin{array}{c|c}
            A & B \\
            \hline
            C & D
        \end{array} 
        \right] = \{ -1,0\},$ which is computed using the MATLAB's \texttt{tzero} routine. 

    
    Letting $B_\rmz = \matl 0 & 0 & 1 & 0 & 0 & 0\\
    0 & -1 & 0 & 0 & 0 & 0\matr,$ it follows from \eqref{eq:T_def} that 
    \begin{align}
        T
            =
                \matl
                    B_\rmz\\
                    C_1\\
                    C_1A\\
                    C_2\\
                    C_2A
                \matr
            =
                \matl
                                0&   0&   1&   0&   0&   0\\
                                0&  -1&  0&   0&   0&   0\\
                                0&   0&   0.5& 0&   1&   1\\
                                0&   2&   0&   16&  4&   0\\
                                0&   2& 
                                -1&  0&   1&   -2\\
                                32&  -4&   0&  16&  -8&   0
                \matr,\nn
    \end{align}
    and thus
    \begin{align*}
        \SA
            &=\left[ \begin{array}{cc|cccc}
                     0&   -4&  0&   0&   0&   0\\
                     0&   -1&  -4&  0.5& -2&  -0.5\\
                     \hline
                     0&   0&   0&   1&   0&   0\\
                     48& -38& -88& -21& 4&   1\\
                     0&   0&   0&   0&   0&  1\\
                    -144&  268& -272&   -6 &  -88&   -26
            \end{array}
            \right],\nn\\
        \SB 
            &=\left[
        \begin{array}{cc}
            0 & 0\\
            0 & 0\\
            \hline
            0 & 0\\
            0 & 64\\
            0 & 0\\
            64 & 64
        \end{array}
        \right],\quad
        \SC
            =\left[    \begin{array}{cc|cccc}
                                        0&     0&     1&     0&     0&     0\\
                                        0&     0&     0&     0&     1&     0
                \end{array}
                \right].\nn
    \end{align*}
    Note that $\SA_\eta = \matl 0 & -4\\
    0 & -1
    \matr$ and thus ${\rm mspec}(\SA_\eta) = \{-1,0\},$ which confirms
    that the invariant zeros of the system are the eigenvalues of $\SA_\eta.$
    % the upper-left partition of the dynamics matrix represented in the zero-subspace form. 
    % conjecture \ref{conj:MIMO_zeros}.
    \hfill{\huge$\diamond$}
\end{exmp}


% \begin{exmp}
% \label{exmp:tall_MIMO}
% \textbf{[Tall MIMO System.]}
%     Consider the  system
%     \begin{align}
%         G(s)=\dfrac{1}{s^2+3s+2}\matl
%             1 & s+1\\
%             s-3 & s-5\\
%             s-4 & s-8
%         \matr,\nn
%     \end{align}
%     with two inputs and three outputs.
%     Note that $\rho_1 =1$, $\rho_2 =1,$ and $\rho_3 =1,$ and thus $\rho = 3.$
%     % Furthermore, ${\rm izeros}={8.5}$.
%     A realization of $G$ is 
%     % computed with MATLAB's \texttt{ss} routine, 
%     \begin{align*}
%         A
%             &=
%                 \matl
%                     -3 & -2 & 0 & 0\\
%                     1 & 0 & 0 & 0\\
%                     0 & 0 & -3 & -2\\
%                     0 & 0 & 1 & 0
%                 \matr, \quad
%         B
%             =
%                 \matl
%                     4 & 0\\
%                     0 & 0\\
%                     0 & 4\\
%                     0 & 0
%                 \matr, 
%         \\
%         C
%             &=
%                 \matl
%                     0 & 0.25 & 0.25 & 0.25\\
%                     0.25 & -0.75 & 0.25 & -1.25\\
%                     0.25 & -1 & 0.25 & -2
%                 \matr.
%     \end{align*}
%     % the zero-subspace form for the transformation matrix
%     Furthermore, ${\rm izeros} 
%         \left[
%         \begin{array}{c|c}
%             A & B \\
%             \hline
%             C & D
%         \end{array} 
%         \right] = \{ 2\}.$
%         % which is computed using the MATLAB's \texttt{tzero} routine. 

    
%      Letting $B_\rmz = \matl 0 & -1 & 0 & 0 \matr,$ it follows from \eqref{eq:T_def} that 
%     \begin{align}
%         T
%             =
%                 \matl
%                     B_\rmz\\
%                     C_1\\
%                     C_2\\
%                     C_3
%                 \matr
%             =
%                 \matl
%                     0 & -1 & 0 & 0\\
%                     0 & 0.25 & 0.25 & 0.25\\
%                     0.25 & -0.75 & 0.25 & -1.25\\
%                     0.25 & -1 & 0.25 & -2
%                 \matr,\nn
%     \end{align}
%     and thus
%     \begin{align*}
%         \SA
%             &=
%                 \matl
%                     2 & 4 & -12 & 8\\
%                     -1 & -3 & 3 & -2\\
%                     2 & -2 & -16 & 10\\
%                     2 & -4 & -18 & -11
%                 \matr, \quad
%         \SB=\matl
%             0 & 0\\
%             0 & 1\\
%             1 & 1\\
%             1 & 1
%         \matr,\nn 
%         \\
%         \SC
%             &=
%                 \matl
%                     0 & 1 & 0 & 0\\
%                     0 & 0 & 1 & 0\\
%                     0 & 0 & 0 & 1
%                 \matr.\nn
%     \end{align*}
%     Note that $\SA_\eta = 2$ and thus ${\rm mspec}(\SA_\eta) = \{2\},$ which confirms that the invariant zeros of the system are the eigenvalues of the upper-left partition of the dynamics matrix represented in the zero-subspace form. 
%     \hfill{\huge$\diamond$}
% \end{exmp}

% \clearpage
\begin{exmp}
\label{exmp:wide_MIMO}
\textbf{[Wide MIMO System.]}
    Consider the wide MIMO system
    \begin{align}
        G(s)
            =
                \dfrac{1}{d(s)} 
                % G_0(s)
                \matl
                    s^2+s-2 & 0 & s^2-2s+1\\
                    s^2-3s+2 & s^2-1 & s^2-1\\
                \matr
                ,\nn
    \end{align}
    where $d(s) = s^3+2s^2-s-2,$
    % where
    % \begin{align}
    %     G_0(s)=\matl
    %         s^2+s-2 & 0 & s^2-2s+1\\
    %         s^2-3s+2 & s^2-5s-6 & s^2-5s-6\\
    %     \matr\nn
    % \end{align}
    with three inputs and two outputs.
    Note that $\rho_1 =1$, and $\rho_2 =1,$ and thus $\rho = 2.$
    % Furthermore, ${\rm izeros}={8.5}$.
    A realization of $G$, computed with MATLAB's \texttt{ss} routine, is 
    \begin{align*}
        A
            &=
                \matl
                    0 & 0 & 2 & 0 & 0 & 0\\
                    1 & 0 & 1 & 0 & 0 & 0\\
                    0 & 1 & -2 & 0 & 0 & 0\\
                    0 & 0 & 0 & 0 & 0 & 2\\
                    0 & 0 & 0 & 1 & 0 & 1\\
                    0 & 0 & 0 & 0 & 1 & -2
                \matr, 
        \\
        B
            &=
                \matl
                    -2 & 0 & 1\\
                    1 & 0 & -2\\
                    1 & 0 & 1\\
                    2 & -1 & -1\\
                    -3 & 0 & 0\\
                    1 & 1 & 1
                \matr, 
        \\
        C
            &=
                \matl
                    0 & 0 & 1 & 0 & 0 & 0\\
                    0 & 0 & 0 & 0 & 0 & 1
                \matr.
    \end{align*}
    % the zero-subspace form for the transformation matrix
    Note that ${\rm izeros} 
        \left[
        \begin{array}{c|c}
            A & B \\
            \hline
            C & D
        \end{array} 
        \right] = \{ 1,1\}.$     
    %     dont worry about B1. i am going to remove it. watch!
    % \begin{align}
    %     % &B_\rmz
    %     %     =\nn\\
    %     %     &
    %     %         \matl
    %     %             0.6427 &0.4239 & 0.2051 & 0.4063 & 0.1875 & 0.4063\\
    %     %              -0.5321 & -0.0333 & 0.4654 & 0.1671 & 0.6658 & 0.1671\\
    %     %              -0.1106 & -0.3906 & -0.6706 & 0.4267 & 0.1467 & 0.4267\\
    %     %              -0.5 & 0.5 & 0 & 0.5 & -0.5 & 0
    %     %         \matr
    %     %     \nn,
    %     &B_\rmz
    %         =\nn\\
    %         &
    %             \matl
    %                 0.642 &0.423 & 0.205 & 0.406 & 0.187 & 0.406\\
    %                  -0.532 & -0.033 & 0.465 & 0.167 & 0.665 & 0.167\\
    %                  -0.110 & -0.390 & -0.670 & 0.426 & 0.146 & 0.426\\
    %                  -0.5 & 0.5 & 0 & 0.5 & -0.5 & 0
    %             \matr
    %         \nn,
    % \end{align}   
    It follows from \eqref{eq:T_def} that 
    \begin{align}
        &T
            =
                \matl
                    B_\rmz\\
                    C_1\\
                    C_2
                \matr
            =\nn\\
                &\matl
                    % 0.6427 &0.4239 & 0.2051 & 0.4063 & 0.1875 & 0.4063\\
                    %  -0.5321 & -0.0333 & 0.4654 & 0.1671 & 0.6658 & 0.1671\\
                    %  -0.1106 & -0.3906 & -0.6706 & 0.4267 & 0.1467 & 0.4267\\
                    %  -0.5 & 0.5 & 0 & 0.5 & -0.5 & 0\\
                    %  0 & 0 & 1 & 0 & 0 & 0\\
                    % 0 & 0 & 0 & 0 & 0 & 1
                    0.642 &0.423 & 0.205 & 0.406 & 0.187 & 0.406\\
                     -0.532 & -0.033 & 0.465 & 0.167 & 0.665 & 0.167\\
                     -0.110 & -0.390 & -0.670 & 0.426 & 0.146 & 0.426\\
                     -0.5 & 0.5 & 0 & 0.5 & -0.5 & 0 \\
                     0 & 0 & 1 & 0 & 0 & 0\\
                    0 & 0 & 0 & 0 & 0 & 1
                \matr,\nn
    \end{align}
    where $B_\rmz$ is computed with the MATLAB routine \texttt{null(B')}.
    Finally, the zero-subspace form of $G$ is
    \begin{align*}
        &\SA
            =\\
                % &\left[\begin{array}{cccc|cc}
                %     0.6968  &  0.2654  &  -0.0716 &  -0.2188  &  0.9844  & -0.1094\\
                %     0.6912  &  0.3949 & 0.1633 &   0.4987  & -2.2444  &  0.2494\\
                %     -0.3880  &  0.3397  & 0.9083  & -0.2800  & 1.2599  & -0.1400\\
                %     0.0679 &  -0.2750  &  -0.5429 &  -0.5  & -0.75  & 0.75\\
                %     \hline
                %     0.5787  & 0.5067  & -1.3354  &  0.5  & -3.25   & 0.25\\
                %     0.4429  &  1.0567  & -0.2496  & -0.5  & -0.75  &  -2.25
                % \end{array}\right],
                &\left[\begin{array}{cccc|cc}
                    0.696  &  0.265  &  -0.07 &  -0.218  &  0.984  & -0.109\\
                    0.691  &  0.394 & 0.163 &   0.498  & -2.244  &  0.249\\
                    -0.388  &  0.339  & 0.908  & -0.280 & 1.259  & -0.140\\
                    0.067 &  -0.275  &  -0.542 &  -0.5  & -0.75  & 0.75\\
                    \hline
                    0.578  & 0.506  & -1.335  &  0.5  & -3.25   & 0.25\\
                    0.442  &  1.056  & -0.249  & -0.5  & -0.75  &  -2.25
                \end{array}\right],
        \nn\\
        \SB&=\left[\begin{array}{ccc}
            0 & 0 & 0\\
            0 & 0 & 0\\
            0 & 0 & 0\\
            4 &  -0.5 &  -2\\
            \hline
            1 & 0 & 1\\
            1 & 1 & 1
        \end{array}\right],\nn 
        \\
        \SC
            &=
                \left[\begin{array}{cccc|cc}
                    0   &  0 &     0 &    0   &  1 &    0\\
                    0   &  0 &     0 &    0   &  0 &    1
                \end{array}\right].\nn
    \end{align*}
    % Note that
    % \begin{align}
    %     \SA_\eta = \matl
    %     0.5175  &  0.7776 &    -0.1323 &   0.4120\\
    %     0.5889 &   0.3550 &  0.9066 &    -0.4905\\
    %     0.2953  &  0.9889 &  4.6704 &   -0.1926\\
    %     -0.3251 &  0.1747 &   -1.7755&   -0.1429
    % \matr   \nn
    % \end{align}
    % and thus 
    Note that ${\rm mspec}(\SA_\eta) = \{0,-0.5,1,1\},$ which suggests
    that the invariant zeros of the system are contained in the ${\rm mspec}(\SA_\eta).$
    
    % some of the eigenvalues of the upper-left partition of the dynamics matrix represented in the zero-subspace form are the invariant zeros of the system.

    To identify the invariant zero in ${\rm mspec}(\SA_\eta),$ 
    we use the fact that the rank of the associated Rosenbrok system matrix $\SZ(s)$ drops at the invariant zero.     
    Since
    % Next, note that rank of $\SZ(z)$ for each $z \in {\rm mspec}(\SA_\eta)$ is . 
    $
        \rank \SZ(0) = \rank \SZ(-0.5) = 8, 
        % \rank \SZ(0) &= 8. 
    $
    but 
    $\rank \SZ(1) = 7,$ 
    % where $\SZ(s)$ is the Rosenbrok system matrix.
    it follows that the invariant zeros of the system are at $z=\{1,1\}$ as the rank of the Rosenbrok system matrix drops only at $z = 1.$ 
    % 
    \hfill{\huge$\diamond$}
\end{exmp}


% \clearpage
\section{Conclusion}
\label{sec:conclusion}

This paper showed that similar to the poles computation, zeros of a state-space realization of a linear system can be computed by solving an eigenvalue problem instead of a generalized eigenvalue problem.
% presented a technique to compute the zeros of a system by solving a 
% 
% This paper presented the zero-subspace form and its connection with the normal form. 
% 
% The zero-subspace form allows the computation of zeros by solving a simple eigenvalue problem. 
% 
By transforming the system into its zero-subspace form, it is shown that the zeros are the eigenvalues of a partition of the dynamics matrix represented in the zero-subspace form, which is also the dynamics matrix of the system's zero dynamics.  
% 
% 
Numerical examples validate the main result of the paper.

% Future research is focused on extending the main result of this paper to multi-input, multi-output systems. 
Future research is focused on extending the main result of this paper to nonsquare MIMO systems. 
% Numerical examples in this paper suggest that the main result is valid for square MIMO systems.
Although the technique is easily extended to the case of wide systems, where it provides spurious invariant zeros, which can be removed heuristically, it provides inconclusive results in tall systems. 
% 
% In the case of tall systems, the construction yields a singular 
% 
An alternative approach may be to consider random bordering to square the nonsquare system and thus compute invariant zeros of the MIMO system. 

% However, the nonsquare systems can be squared by random bordering.  




% We extended the zero-subspace form to MIMO systems and presented numerical examples supporting the conjecture that the set of eigenvalues of the zero dynamics contains all invariant zeros. 



% \bibliographystyle{ieeetr}
% \bibliographystyle{plain}
% \bibliography{b}
\printbibliography

% \clearpage
% \appendix
% \begin{appendices}
% \label{app:guid_law}
\section*{Appendix A: Normal Form}
\label{appndx:normalForm}
This section reviews the construction of the normal form of a nonlinear affine system \cite{kolavennu2001nonlinear,isidori1985nonlinear}.

Consider an affine system
\begin{align}
    \dot x 
        &= 
            f(x) + g(x) u,
    \label{eq:xdot_gen}
    \\
    y 
        &=
            h(x),
    \label{eq:y_gen}
\end{align}
where $x(t)\in \BBR^{l_x}$ is the state, 
$u(t)\in \BBR^{l_u}$ is the input, 
$y(t)\in \BBR^{l_y}$ is the output,
and $f, g, h$ are smooth functions of appropriate dimensions. 
% \textbf{Add citation to Khalil's book here}
For each $i\in \{ 1, \ldots, l_y \},$ let $\rho_i$ denote that relative degree of the $i$th output $y_i.$
Furthermore, let $\rho \isdef \sum_i^{l_y} \rho_i $ denote the relative degree of the system.

% The following definitions appear in \cite{isidori1985nonlinear} and are repeated here for further use in the paper. 

% \begin{definition}
%     \label{def:rel_degree}
%     In the system \eqref{eq:xdot_gen}, \eqref{eq:y_gen}.
%     the relative degree of the $i$th output $y_i$ is the smallest integer $\rho_i \ge 0$ such that $\rho_i$-th derivative of $y_i,$ that is
%     $y_i^{(\rho_i)},$ is an explicit function of input $u$. 
% \end{definition}





% \begin{definition}
%     The relative degree of the system \eqref{eq:xdot_gen}, \eqref{eq:y_gen} is the sum of the relative degree of each of its outputs, that is, $\rho \isdef \sum_i^{l_y} \rho_i.$
% \end{definition}





Consider the transformation
\begin{align}\label{eq_T}
    T : \BBR^{l_x} &\to \BBR^{l_x}
    \nn \\
        T(x)
            &=
                \matl
                    \phi(x) \\
                    \psi(x)
                \matr,
\end{align}
where $\phi(x) $ is chosen to satisfy
\begin{align}\label{eq_phi}
    L_g \phi(x) = 0,
\end{align}
and
% the $i$th component of $\psi(x)$ is given by
\begin{align}
    % \label{eq_psi}
    \psi(x) 
        =
            \matl
                \psi_1(x) \\
                \vdots \\
                \psi_{l_y}(x)
            \matr,
    \label{eq:psi_def}
\end{align}
where
\begin{align}
    % \label{eq_psi_i}
    \psi_i(x)
        \isdef
            \matl
                h_i(x) \\
                L_f h_i(x) \\
                \vdots \\
                L_f^{\rho_i-1} h_i(x)
            \matr
        \in 
            \BBR^{\rho_i}.
    \label{eq:psi_i_def}
\end{align}
Note that $\phi : \BBR^{l_x} \to \BBR^{{l_x} - \rho}$ and, for $i=1, \ldots, l_y,$
$\psi_i : \BBR^{l_x} \to \BBR^ {\rho_i},$ and thus
$\psi : \BBR^{l_x} \to \BBR^ \rho.$
Furthermore, the functions $\psi_i$ are well-defined since the functions $f,g,h$ are assumed to be smooth. 
However, $\phi$ satisfying \eqref{eq_phi} may or may not exist. 
% Note that $L_g \phi(x) $ is the Lie derivative of the function $\phi(x)$ with respect to the function $g(x).$


Assuming that $\phi$ satisfying \eqref{eq_phi} exists and defining $\eta \isdef \phi(x), $ it follows 
% from Fact \ref{fact:zeta_dot} 
that 
\begin{align}
    \dot \eta 
        &=
        %     \ddt{\phi(x)}
        % \nn \\
        % &=
        %     \dpder{\phi}{x} \dot x
        % \nn \\
        % &=
        %     \dpder{\phi}{x} [f(x) + g(x) u]
        % \nn \\
        % &=
            L_f \phi(x) + L_g \phi(x) u
        % \nn \\
        =
            L_f \phi(x),
    % \label{eq:eta_dynamics}
    \label{eq_eta_dot}
\end{align}
where $L_g \phi(x)=0$ by construction. 
% Note that such a $\phi$ may or may not exist. 
Note that \eqref{eq_eta_dot} is the \textit{zero dynamics} \cite{Khalil:1173048}.



Next, defining $\xi \isdef \psi(x),$ it follows
% from Fact \ref{fact:zeta_dot} 
that 
\begin{align}
    \dot \xi
        % &=
        %     \ddt \psi(x)
        % \nn \\
        % &=
        %     \dpder{\psi}{x} \dot x
        % \nn \\
        % &=
        %     \dpder{\psi}{x} [f(x) + g(x) u]
        % \nn \\
        &=
            L_f \psi(x) + L_g \psi(x) u.
    \label{eq:xi_dot}
\end{align}
Next, note that 
\begin{align}
    L_f \psi(x)
        % &=
        %     L_f 
        %     \matl
        %         \psi_1 \\
        %         \vdots \\
        %         \psi_{l_y}
        %     \matr
        % \nn \\
        % &=
        %     \matl
        %         L_f \psi_1 \\
        %         \vdots \\
        %         L_f \psi_{l_y}
        %     \matr
        % \nn \\
        =
            A_\rmc \xi 
            +
            B_\rmc 
            \matl 
                L_f^{\rho_1} h_1(x) \\
                \vdots \\
                L_f^{\rho_{l_y}} h_{l_y}(x)
            \matr,
    \label{eq:Lf_psi}
\end{align}
where $A_\rmc = {\rm diag  } (A_{\rmc,1}, \ldots, A_{\rmc, l_y}) \in \BBR^{\rho \times \rho }$ and 
$B_\rmc = {\rm diag} (b_{\rmc,1}, \ldots, b_{\rmc,l_y}) \in \BBR^{\rho \times l_y} $ and, for $i=1,\ldots, l_y,$ 
\begin{align}
    A_{\rmc,i}
        &\isdef 
            \matl 
                0 & 1 & 0 & \cdots & 0  \\
                0 & 0 & 1 & \cdots & 0  \\
                \vdots & \vdots & \ddots & \ddots & \vdots \\
                0 & \vdots & \ldots & 0 & 1  \\
                0 & \vdots & \ldots & 0 & 0  \\
            \matr
        \in \BBR^{\rho_i \times \rho_i},
    \\
    b_{\rmc,i}
        &\isdef 
            \matl 
                0\\
                \vdots \\
                1
            \matr
        \in \BBR^{\rho_i}.
\end{align}
Finally, 
% Next, note that
\begin{align}
    L_g \psi(x)
        % &=
        %     L_g 
        %     \matl
        %         \psi_1 \\
        %         \vdots \\
        %         \psi_{l_y}
        %     \matr
        % \nn \\
        % &=
        %     \matl
        %         L_g \psi_1 \\
        %         \vdots \\
        %         L_g \psi_{l_y}
        %     \matr
        % \nn \\
        &=
            B_\rmc  %L_g L_f^{\rho-1} h
            \matl 
                L_g L_f^{\rho_1-1} h_1(x) \\
                \vdots \\
                L_g L_f^{\rho_{l_y}-1} h_{l_y} (x)
            \matr.
    \label{eq:Lg_psi}
\end{align}

Substituting \eqref{eq:Lf_psi} and \eqref{eq:Lg_psi} in \eqref{eq:xi_dot} yields
\begin{align}
    \dot \xi
        % &=
        %     A_\rmc \xi 
        %     +
        %     B_\rmc 
        %     \matl 
        %         L_f^{\rho_1} h_1(x) \\
        %         \vdots \\
        %         L_f^{\rho_{l_y}} h_{l_y}(x)
        %     \matr
        %     +
        %     B_\rmc  %L_g L_f^{\rho-1} h
        %     \matl 
        %         L_g L_f^{\rho_i-1} h_1(x) \\
        %         \vdots \\
        %         L_g L_f^{\rho_{l_y}-1} h_{l_y} (x)
        %     \matr
        %     u
        % \\
        % &=
        %     A_\rmc \xi 
        %     +
        %     B_\rmc 
        %     \gamma(x)
        %     [
        %         u - \alpha(x)
        %     ],
        &=
            A_\rmc \xi 
            +
            B_\rmc
            \left(
                \alpha(x)
                +
                \beta(x) u
            \right),
\end{align}
where 
\begin{align}
    \alpha(x)
        &\isdef
            \matl 
                L_f^{\rho_1} h_1(x) \\
                \vdots \\
                L_f^{\rho_{l_y}} h_{l_y}(x)
            \matr
            \in \BBR^{l_y}, 
    \label{eq:alpha_def}
    \\
    \beta(x)
        &\isdef 
            \matl 
                L_g L_f^{\rho_1-1} h_1(x) \\
                \vdots \\
                L_g L_f^{\rho_{l_y}-1} h_{l_y} (x)
            \matr
            \in \BBR^{l_y \times l_u}.
    \label{eq:beta_def}
\end{align}
% 

The normal form of the nonlinear system \eqref{eq:xdot_gen}, \eqref{eq:y_gen} is then 
\begin{align}
    \dot \eta 
        &=
            L_f \phi(x),
    % \label{eq_eta_dot}
        \\
    \dot \xi
        &=
            A_\rmc \xi 
            +
            B_\rmc
            \left(
                \alpha(x)
                +
                \beta(x) u
            \right).
\end{align}


% \end{appendices}





\end{document}
