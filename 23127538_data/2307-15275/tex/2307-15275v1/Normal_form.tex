
Consider an affine system
\begin{align}
    \dot x 
        &= 
            f(x) + g(x) u,
    \label{eq:xdot_gen}
    \\
    y 
        &=
            h(x),
    \label{eq:y_gen}
\end{align}
where $x(t)\in \BBR^{l_x}$ is the state, 
$u(t)\in \BBR^{l_u}$ is the input, 
$y(t)\in \BBR^{l_y}$ is the output,
and $f, g, h$ are smooth functions of appropriate dimensions. 
% \textbf{Add citation to Khalil's book here}
For each $i\in \{ 1, \ldots, l_y \},$ let $\rho_i$ denote that relative degree of the $i$th output $y_i.$
Furthermore, let $\rho \isdef \sum_i^{l_y} \rho_i $ denote the relative degree of the system.

% The following definitions appear in \cite{isidori1985nonlinear} and are repeated here for further use in the paper. 

% \begin{definition}
%     \label{def:rel_degree}
%     In the system \eqref{eq:xdot_gen}, \eqref{eq:y_gen}.
%     the relative degree of the $i$th output $y_i$ is the smallest integer $\rho_i \ge 0$ such that $\rho_i$-th derivative of $y_i,$ that is
%     $y_i^{(\rho_i)},$ is an explicit function of input $u$. 
% \end{definition}





% \begin{definition}
%     The relative degree of the system \eqref{eq:xdot_gen}, \eqref{eq:y_gen} is the sum of the relative degree of each of its outputs, that is, $\rho \isdef \sum_i^{l_y} \rho_i.$
% \end{definition}





Consider the transformation
\begin{align}\label{eq_T}
    T : \BBR^{l_x} &\to \BBR^{l_x}
    \nn \\
        T(x)
            &=
                \matl
                    \phi(x) \\
                    \psi(x)
                \matr,
\end{align}
where $\phi(x) $ is chosen to satisfy
\begin{align}\label{eq_phi}
    L_g \phi(x) = 0,
\end{align}
and
% the $i$th component of $\psi(x)$ is given by
\begin{align}
    % \label{eq_psi}
    \psi(x) 
        =
            \matl
                \psi_1(x) \\
                \vdots \\
                \psi_{l_y}(x)
            \matr,
    \label{eq:psi_def}
\end{align}
where
\begin{align}
    % \label{eq_psi_i}
    \psi_i(x)
        \isdef
            \matl
                h_i(x) \\
                L_f h_i(x) \\
                \vdots \\
                L_f^{\rho_i-1} h_i(x)
            \matr
        \in 
            \BBR^{\rho_i}.
    \label{eq:psi_i_def}
\end{align}
Note that $\phi : \BBR^{l_x} \to \BBR^{{l_x} - \rho}$ and, for $i=1, \ldots, l_y,$
$\psi_i : \BBR^{l_x} \to \BBR^ {\rho_i},$ and thus
$\psi : \BBR^{l_x} \to \BBR^ \rho.$
Furthermore, the functions $\psi_i$ are well-defined since the functions $f,g,h$ are assumed to be smooth. 
However, $\phi$ satisfying \eqref{eq_phi} may or may not exist. 
% Note that $L_g \phi(x) $ is the Lie derivative of the function $\phi(x)$ with respect to the function $g(x).$


Assuming that $\phi$ satisfying \eqref{eq_phi} exists and defining $\eta \isdef \phi(x), $ it follows 
% from Fact \ref{fact:zeta_dot} 
that 
\begin{align}
    \dot \eta 
        &=
        %     \ddt{\phi(x)}
        % \nn \\
        % &=
        %     \dpder{\phi}{x} \dot x
        % \nn \\
        % &=
        %     \dpder{\phi}{x} [f(x) + g(x) u]
        % \nn \\
        % &=
            L_f \phi(x) + L_g \phi(x) u
        % \nn \\
        =
            L_f \phi(x),
    % \label{eq:eta_dynamics}
    \label{eq_eta_dot}
\end{align}
where $L_g \phi(x)=0$ by construction. 
% Note that such a $\phi$ may or may not exist. 
Note that \eqref{eq_eta_dot} is the \textit{zero dynamics} \cite{Khalil:1173048}.



Next, defining $\xi \isdef \psi(x),$ it follows
% from Fact \ref{fact:zeta_dot} 
that 
\begin{align}
    \dot \xi
        % &=
        %     \ddt \psi(x)
        % \nn \\
        % &=
        %     \dpder{\psi}{x} \dot x
        % \nn \\
        % &=
        %     \dpder{\psi}{x} [f(x) + g(x) u]
        % \nn \\
        &=
            L_f \psi(x) + L_g \psi(x) u.
    \label{eq:xi_dot}
\end{align}
Next, note that 
\begin{align}
    L_f \psi(x)
        % &=
        %     L_f 
        %     \matl
        %         \psi_1 \\
        %         \vdots \\
        %         \psi_{l_y}
        %     \matr
        % \nn \\
        % &=
        %     \matl
        %         L_f \psi_1 \\
        %         \vdots \\
        %         L_f \psi_{l_y}
        %     \matr
        % \nn \\
        =
            A_\rmc \xi 
            +
            B_\rmc 
            \matl 
                L_f^{\rho_1} h_1(x) \\
                \vdots \\
                L_f^{\rho_{l_y}} h_{l_y}(x)
            \matr,
    \label{eq:Lf_psi}
\end{align}
where $A_\rmc = {\rm diag  } (A_{\rmc,1}, \ldots, A_{\rmc, l_y}) \in \BBR^{\rho \times \rho }$ and 
$B_\rmc = {\rm diag} (b_{\rmc,1}, \ldots, b_{\rmc,l_y}) \in \BBR^{\rho \times l_y} $ and, for $i=1,\ldots, l_y,$ 
\begin{align}
    A_{\rmc,i}
        &\isdef 
            \matl 
                0 & 1 & 0 & \cdots & 0  \\
                0 & 0 & 1 & \cdots & 0  \\
                \vdots & \vdots & \ddots & \ddots & \vdots \\
                0 & \vdots & \ldots & 0 & 1  \\
                0 & \vdots & \ldots & 0 & 0  \\
            \matr
        \in \BBR^{\rho_i \times \rho_i},
    \\
    b_{\rmc,i}
        &\isdef 
            \matl 
                0\\
                \vdots \\
                1
            \matr
        \in \BBR^{\rho_i}.
\end{align}
Finally, 
% Next, note that
\begin{align}
    L_g \psi(x)
        % &=
        %     L_g 
        %     \matl
        %         \psi_1 \\
        %         \vdots \\
        %         \psi_{l_y}
        %     \matr
        % \nn \\
        % &=
        %     \matl
        %         L_g \psi_1 \\
        %         \vdots \\
        %         L_g \psi_{l_y}
        %     \matr
        % \nn \\
        &=
            B_\rmc  %L_g L_f^{\rho-1} h
            \matl 
                L_g L_f^{\rho_1-1} h_1(x) \\
                \vdots \\
                L_g L_f^{\rho_{l_y}-1} h_{l_y} (x)
            \matr.
    \label{eq:Lg_psi}
\end{align}

Substituting \eqref{eq:Lf_psi} and \eqref{eq:Lg_psi} in \eqref{eq:xi_dot} yields
\begin{align}
    \dot \xi
        % &=
        %     A_\rmc \xi 
        %     +
        %     B_\rmc 
        %     \matl 
        %         L_f^{\rho_1} h_1(x) \\
        %         \vdots \\
        %         L_f^{\rho_{l_y}} h_{l_y}(x)
        %     \matr
        %     +
        %     B_\rmc  %L_g L_f^{\rho-1} h
        %     \matl 
        %         L_g L_f^{\rho_i-1} h_1(x) \\
        %         \vdots \\
        %         L_g L_f^{\rho_{l_y}-1} h_{l_y} (x)
        %     \matr
        %     u
        % \\
        % &=
        %     A_\rmc \xi 
        %     +
        %     B_\rmc 
        %     \gamma(x)
        %     [
        %         u - \alpha(x)
        %     ],
        &=
            A_\rmc \xi 
            +
            B_\rmc
            \left(
                \alpha(x)
                +
                \beta(x) u
            \right),
\end{align}
where 
\begin{align}
    \alpha(x)
        &\isdef
            \matl 
                L_f^{\rho_1} h_1(x) \\
                \vdots \\
                L_f^{\rho_{l_y}} h_{l_y}(x)
            \matr
            \in \BBR^{l_y}, 
    \label{eq:alpha_def}
    \\
    \beta(x)
        &\isdef 
            \matl 
                L_g L_f^{\rho_1-1} h_1(x) \\
                \vdots \\
                L_g L_f^{\rho_{l_y}-1} h_{l_y} (x)
            \matr
            \in \BBR^{l_y \times l_u}.
    \label{eq:beta_def}
\end{align}
% 

The normal form of the nonlinear system \eqref{eq:xdot_gen}, \eqref{eq:y_gen} is then 
\begin{align}
    \dot \eta 
        &=
            L_f \phi(x),
    % \label{eq_eta_dot}
        \\
    \dot \xi
        &=
            A_\rmc \xi 
            +
            B_\rmc
            \left(
                \alpha(x)
                +
                \beta(x) u
            \right).
\end{align}
