\documentclass[aps,nofootinbib,superscriptaddress, showpacs,preprintnumbers,  nofootinbibt,twocolumn]{revtex4-2}
%%%%%%%%%%%%%%%%%%%%%%%%%%%%%%%%%%%%%%%%%%%%%%%%%%%%%%%%%%%%%%%%%%%%%%%%%%%%%%%%%%%%%%%%%%%%%%%%%%%%%%%%%%%%%%%%%%%%%%%%%%%%%%%%%%%%%%%%%%%%%%%%%%%%%%%%%%%%%%%%%%%%%%%%%%%%%%%%%%%%%%%%%%%%%%%%%%%%%%%%%%%%%%%%%%%%%%%%%%%%%%%%%%%%%%%%%%%%%%%%%%%%%%%%%%%%
\usepackage{epsfig}
\usepackage{multirow}
\usepackage{eurosym}
\usepackage{dcolumn}
\usepackage{bm}
\usepackage{enumerate}
\usepackage{float}
\usepackage{epstopdf}
\usepackage{amsmath}
\usepackage{bm}
\usepackage{amsfonts}
\usepackage{amssymb}
\usepackage{graphicx}
\usepackage{alphalph,mathtools}
\usepackage{etoolbox}
\usepackage{color}
\usepackage{booktabs}
\usepackage{footnote}
\usepackage{makecell,tabularx}
\usepackage{hyperref}
\hypersetup{colorlinks,citecolor=blue}
\hypersetup{colorlinks=true,linkcolor=red,filecolor=magenta,
    urlcolor=blue}

\setcounter{MaxMatrixCols}{10}
%TCIDATA{OutputFilter=Latex.dll}
%TCIDATA{Version=5.50.0.2953}
%TCIDATA{<META NAME="SaveForMode" CONTENT="1">}
%TCIDATA{BibliographyScheme=BibTeX}
%TCIDATA{LastRevised=Tuesday, January 17, 2023 10:00:45}
%TCIDATA{<META NAME="GraphicsSave" CONTENT="32">}

\renewcommand{\thefootnote}{\arabic{footnote}}
\def\pp{{\, \mid \hskip -1.5mm =}}
\def\cL{\mathcal{L}}
\def\be{\begin{equation}}
    \def\ee{\end{equation}}
\def\bea{\begin{eqnarray}}
    \def\eea{\end{eqnarray}}
\def\tr{\mathrm{tr}\, }
\def\nn{\nonumber \\}
\def\e{\mathrm{e}}
\def\overom{\overline{\omega}}
\def\l{\left}
\def\r{\right}
\newcommand{\R}{\ensuremath{\mathbb{R}}}
\newcommand{\de}{\mathrm{d}}
\newcommand{\cH}{\ensuremath{\mathcal{H}}}

\begin{document}

\title{\bf Observational Signatures of Modified Bardeen Black Hole: Shadow and Strong Gravitational Lensing }
\author{Niyaz Uddin Molla}
\email{niyazuddin182@gmail.com}\affiliation{Department of Mathematics, Indian Institute of Engineering Science and
Technology, Shibpur, Howrah-711 103, India,}
\author{Amna Ali}
\email{amnaalig@gmail.com} \affiliation{Department of Mathematics, Jadavpur University, Kolkata-700032, India,}
\author{Ujjal Debnath}
\email{ujjaldebnath@gmail.com}\affiliation{Department of Mathematics, Indian Institute of Engineering Science and
Technology, Shibpur, Howrah-711 103, India.}
%%%%%%%%%%%%%%%%%%%%%%%%%%%%%%%%%%%%%%%%%%%%%%%5
%%%%%%%%%%%%%%%%%
\begin{abstract}
This paper is devoted to studying the observational signatures
modified by Bardeen black hole via shadow and strong lensing
observations. Influence of the modified Bardeen black hole
parameters q, g, and the parameter $\mu$ on the shadow radius of
the black hole have been investigated numerically and graphically.
Recently, the Event Horizon Telescope (EHT) collaboration observed
the image and shadow of supermassive black holes $M87^*$ and
$SgrA^*$ where the shadow angular diameter $\theta_d=42\pm3$ for
$M87^*$ and $\theta_d=51.8\pm2.3$ for $SgrA^*$. The modified black hole
parameters q and $\mu$ for the fixed value of g  have been
constrained by the EHT collaboration data for the angular shadow
diameter of $M87^*$ and $SgrA^*$. It has been observed that the
constrain ranges of the parameters $\mu$ and $q$  of modified
Bardeen black hole as $-0.89\leq \mu/8M^2 \leq 0.4$ and $0\leq
|q|\leq 0.185$ for $M87^*$; and $-1.38\leq \mu/8M^2 \leq 0.1$ and
$0\leq |q|\leq 0.058$  for $SgrA^*$, keeping the fixed value
$g/2M=0.2$. Modified Bardeen black holes with the additional
parameters $\mu$,$g$ and $q$ besides the mass M  of the black hole
as the supermassive black holes $M87^*$ and $SgrA^*$; and it is
observed that to be a viable astrophysical black hole candidate,
the EHT result constrains the ($\mu$, $q$) parameter space.
Furthermore, Gravitational lensing in the strong field limit for
modified Bardeen black hole has been investigated numerically as
well as graphically and compared to the other ordinary
astrophysical black hole such as Schwarzschild ($\mu=\&q=0$) and
regular Bardeen ($\mu=0$) black hole. We prove how the modified
Bardeen black hole parameters would affect the various strong
lensing observables. The astrophysical consequences via strong
gravitational lensing have been explored by considering the
example of supermassive black holes in various galaxies, and the
findings show that the modified Bardeen black hole can be
quantitatively distinguished from the other astrophysical black
hole such as Schwarzschild and regular Bardeen black holes. The
findings via astrophysical consequences provide a potential way to
distinguish the modified black hole from its counterpart in the
general theory of relativity.

\textbf{ Keywords:} Shadow, Gravitational lensing, Null geodesics, Bardeen black hole.\\\\
\end{abstract}

%%%%%%%%%%%%%%%%%%%%%%%%%%%%%%%%%%%%%%%%%%%%%%%%%%

%%%%%%%%%%%%%%%%% BODY OF PAPER %%%%%%%%%%%%%%%%%%

\maketitle
\section{Introduction}

Black holes, which were postulated several decades ago, represent
a crucial prediction in the general theory of relativity and
remain enigmatic, compact entities within our universe. These
objects have currently garnered significant attention in the
fields of astronomy, astrophysics, and high-energy physics.
Remarkably significant discoveries about thermodynamics, quantum
effects, and gravitational interactions within curved spacetime
have emerged through the study of black
holes\cite{zhang1991physics,de2012black}. In recent years,
researchers have primarily been driven by various theories
regarding black holes. However, significant advancements in
observational and experimental investigations of black holes have
been witnessed over the past decade. Several observational
characteristics have provided strong evidence for the existence of
black holes, including the measurement of black hole spin in X-ray
binaries, the detection of gravitational wave signals emitted
during binary black hole mergers by
LIGO\cite{LIGOScientific:2016aoc,LIGOScientific:2020iuh,LIGOScientific:2020stg}'
the first-ever image of a black hole at the center of the galaxy
M87 captured by the Event Horizon Telescope (EHT)
collaboration\cite{EventHorizonTelescope:2019dse}, and the
discovery of a wide star-black hole binary system through radial
velocity measurements\cite{Liu:2019lfc}. These achievements
have collectively reinforced the understanding of black holes.



In addition to its significance in astronomy and astrophysics,
black holes have also been the subject of investigation in various
other branches of physics. One noteworthy endeavor in this regard
is the study of the Bardeen black hole. The modified Bardeen black
hole represents a modified version of the original Bardeen black
hole solution, which was initially proposed by John Bardeen in the
publication year \cite{bardeen1968proceedings}. This
particular line of research explores alternative aspects and
characteristics of black holes beyond traditional understanding.


The original Bardeen black hole solution was proposed as a regular
alternative to the classical Schwarzschild black hole, addressing
the singularity problem. Subsequently, variations such as the
Bardeen-anti-de Sitter and Bardeen-de Sitter black holes have been
examined \cite{tzikas2019bardeen, fernando2017bardeen}. In
recent years, researchers have introduced modifications to the
Bardeen black hole solution to incorporate various physical
effects and tackle theoretical challenges
\cite{Pourhassan:2015lfa, nag2023thermodynamics}. These
modifications involve introducing additional fields, such as
scalar or electromagnetic fields, or considering alterations to
the theory of gravity itself, including higher-order curvature
terms.


A notable example is the modified Bardeen black hole in 4D
Einstein-Gauss-Bonnet (EGB) gravity (\cite{islam2022strong}).
The EGB theory extends general relativity to higher dimensions and
incorporates quadratic curvature terms. The modified Bardeen black
hole in EGB gravity offers fresh insights into regular black hole
solutions and their astrophysical implications. Additionally,
rotating versions of the modified Bardeen black hole have been
proposed by Pourhassan and Debnath
\cite{Pourhassan:2015lfa}.



Various astronomical and astrophysical aspects, such as black hole
parameter estimation, shadows, gravitational lensing, quasinormal
modes, time delay, and particles' motion around black holes have
been investigated for different types of black holes. These
include Schwarzschild black holes \cite{Virbhadra:1999nm},
modified regular black holes, regular Bardeen black holes
\cite{Rayimbaev:2022znx, Stuchlik:2019uvf}, Bardeen black
holes in cloud string \cite{Atamurotov:2023tff,
Vishvakarma:2023csw, Liu:2023kxd}, Bardeen-Kiselev black holes
\cite{Rayimbaev:2022mrk}, modified Ads Bardeen black holes
\cite{nag2023thermodynamics}, asymptotic magnetically-charged
non-singular black holes \cite{Kumaran:2023brp}, regular
Bardeen black holes in 4D Einstein Gauss-Bonnet gravity
\cite{rayimbaev2022shadow, islam2022strong}, and others
\cite{Yan:2023pxj, Hu:2023bzy, Jana:2023sil, Molla:2023hou,
Jha:2023qqz, Qiao:2021trw, Chagoya:2020bqz, Bozza:2010xqn}
studied over the past few decades.



In the current paper, we aim to expand on previous analyses
conducted on Schwarzschild, modified regular, and regular Bardeen
black holes (explored in the literature by \cite{Bozza:2002zj,
Virbhadra:1999nm, Bozza:2003cp, Schee:2015nua, Eiroa:2010wm,
Stuchlik:2019uvf, islam2022strong} and apply them to the case of
the modified Bardeen black hole through observations of shadows
and strong gravitational lensing. Specifically, we discuss the
astrophysical consequences of the modified Bardeen black hole
using examples of supermassive black holes located near the center
of galaxies, comparing them to ordinary astrophysical black holes
like Schwarzschild and regular Bardeen black holes.


The black hole shadow is one of the most fascinating and important
astrophysical features observed through strong gravitational
lensing. The ultra-high-resolution images of $M87^*$ and $SgrA^*$
released by the Event Horizon Telescope (EHT) collaboration
provide crucial evidence for the existence of black holes
\cite{EventHorizonTelescope:2019dse,
EventHorizonTelescope:2019uob, EventHorizonTelescope:2019jan,
EventHorizonTelescope:2019ths, EventHorizonTelescope:2019pgp,
EventHorizonTelescope:2019ggy, EventHorizonTelescope:2022wkp,
EventHorizonTelescope:2022apq, EventHorizonTelescope:2022wok,
EventHorizonTelescope:2022exc, EventHorizonTelescope:2022urf,
EventHorizonTelescope:2022xqj}. Within these images, a faint
region at the center of the black hole, known as the black hole
shadow can be observed. It is generally understood that light rays
passing in the vicinity of a black hole are deflected due to the
gravitational lensing effect \cite{Einstein:1936llh,
Virbhadra:1999nm, Perlick:2004tq, Cunha:2018acu}, resulting in
the observation of a sharp-edge boundary region of brightness on
the distant image plane. The black hole shadow is deeply connected
to the spacetime geometry and serves as a robust tool for
estimating black hole parameters \cite{Kumar:2018ple,
ghosh2021parameters, Afrin:2021imp, ghosh2022constraining},
investigating general relativity and its alternatives
\cite{Mizuno:2018lxz, Psaltis:2018xkc, Stepanian:2021vvk,
younsi2023black, Perlick:2021aok, walia2022testing,
vagnozzi2022horizon}.



Gravitational lensing, an important aspect of general relativity,
is a robust astrophysical tool that involves the deflection of
light rays (photons) due to gravity. The object causing this
deflection is known as a gravitational lens. This phenomenon was
predicted in 1919 by Eddington et al. as a successful
demonstration of Einstein's general theory of relativity
\cite{dyson1920ix}. Gravitational lensing is widely used in
physics, astronomy, and cosmology to understand various properties
of spacetime, such as the distribution of matter in the universe
on both small and large scales, including galaxy clusters and
haloes \cite{Hoekstra:2013via, Brouwer:2018xnj,
Bellagamba:2018gec}.


Furthermore, gravitational lensing serves as a valuable tool for
detecting and studying dark matter and dark energy
\cite{Vanderveld:2012ec, he2017direct, Cao:2012ja,
Huterer:2017buf, Jung:2017flg, Andrade:2019wzn}, estimating the
Hubble parameter \cite{Refsdal:1964yk, Refsdal:1964nw}),
observing gravitational waves (\cite{wang1996gravitational,
Bisnovatyi-Kogan:2008yob}, and testing general relativity and
alternative theories of gravity \cite{Dyson:1920cwa,
Will:2014zpa}. Overall, gravitational lensing plays a significant
role in our understanding of the universe, allowing us to probe
the nature of gravity, explore the distribution of matter, and
investigate phenomena related to dark matter, dark energy, and
cosmological parameters.



Gravitational lensing, as an important application of general
relativity, can be categorized into two regimes: weak lensing and
strong lensing. In weak lensing, the gravitational lens is not
strong enough to produce multiple or highly magnified images. This
regime is particularly useful for studying galaxy clusters. On the
other hand, strong lensing occurs when a compact object, such as a
black hole, with a strong gravitational field or when the source
is very close to the black hole, leading to the appearance of
multiple images, arcs, and rings of the source. In this work, we
focus specifically on the phenomena of strong gravitational
lensing.


The study of strong gravitational lensing has garnered significant
interest among modern researchers due to its ability to provide
valuable information about the properties of black hole spacetime.
While relativistic images cannot be easily separated due to their
small separation and low magnification, advancements in
technology, such as the new generation Event Horizon Telescope
(EHT), offer the potential to distinguish between relativistic
images and different types of black holes. Therefore,
gravitational lensing in the strong field limit provides a useful
tool for testing general relativity and alternative theories of
gravity.


In \cite{Bozza:2001xd}, Bozza et al. introduced a useful
method for obtaining the deflection angle in the strong
gravitational field and found that the deflection angle diverges
logarithmically for the Schwarzschild black hole spacetime. They
also proposed that this method could be applied to any general
spherically symmetric black hole
\cite{Bozza:2001xd,Bozza:2002zj}. Since then, gravitational
lensing in the strong field limit has been studied for various
types of black holes
\cite{gao2021investigating,whisker2005strong,
ghosh2021parameters,zhang2017strong,chen2009strong,
eiroa2002reissner} and naked singularities
\cite{petters2012singularity,virbhadra2002gravitational,
sahu2012can, gyulchev2008gravitational,
tsukamoto2021gravitational,paul2020strong}, as well as wormholes
\cite{shaikh2019strong,tsukamoto2016strong,
sharif2015strong}. Virbhadra and Ellis
\cite{Virbhadra:1999nm} and Frittelli et al.
\cite{Frittelli:1999yf} proposed the definition of the exact
lens equation in the context of spacetime geometry, providing an
exact lens equation for the Schwarzschild black hole spacetime.
Virbhadra and Ellis \cite{Virbhadra:2002ju} also numerically
studied gravitational lensing by naked singularities.

Several studies have explicitly discussed the astrophysical consequences of different black hole spacetimes, such as the angular position, angular separation, relative magnification, Einstein ring, and time delays of relativistic images. These investigations have quantitatively examined gravitational lensing by rotating black holes as well as non-rotating black holes, focusing on the observable signatures \cite{kumar2022investigating, kumar2022testing,Kumar:2019pjp,kumar2020gravitational, afrin2023tests,Zhao:2017cwk,Ghosh:2023usx,Jha:2023qqz, Kuang:2022xjp,Chakraborty:2016lxo,Cavalcanti:2016mbe}.

Observationally, gravitational lensing phenomena by black holes have gained significant attention from researchers in recent years \cite{Millon:2023kgb, Meena2023qdq,Ng:2023jvx,Lin:2023ccz,vonBraun-Bates:2022uvy, Mahler:2022ypx}. Our work aims to study the observational signatures of the modified Bardeen black hole through shadow and strong lensing observations. We investigate various astrophysical consequences, such as the black hole shadow, angular position, separation, Einstein ring, and time delays of relativistic images, within the context of the modified Bardeen black hole. We compare these results with other astrophysical black holes, including the Schwarzschild black hole and the ordinary regular Bardeen black hole.

The structure of this paper is organized as follows: In Section \textbf{II}, we provide a brief review of the modified Bardeen black hole and analyze its null geodesics along the equatorial plane. Section \textbf{III} is devoted to studying the shadow of the modified Bardeen black hole and constraining its observables using observational data from $M87^*$ and $SgrA^*$ \cite{EventHorizonTelescope:2019pgp, EventHorizonTelescope:2019ggy, EventHorizonTelescope:2022wkp, EventHorizonTelescope:2022apq}. In Section \textbf{IV}, we investigate strong gravitational lensing by the modified Bardeen black hole and explore its observables, such as the angular image position, angular separation, relative magnification, Einstein ring, and time delays of the relativistic images. We also provide a brief overview of the astrophysical consequences of the modified Bardeen black hole. Section \textbf{V} focuses on estimating the strong lensing observables for the supermassive black hole BH $NGC 4649$, with a mass of $M=4.3 \times 10^6$ and a distance of $D_{OL}=0.0083$ Mpc \cite{Kormendy:2013dxa}, and comparing the results with those obtained for the Schwarzschild black hole and the ordinary regular Bardeen black hole. Finally, in Section \textbf{VI}, we discuss and summarize our findings.



\section{ Modified Bardeen Black holes and Null geodesics}
The modified Bardeen black hole  discussed in this paper was proposed in \cite{Pourhassan:2015lfa}
where the modified rotating version of the Bardeen black hole was discussed as particle accelerators. Here, We have discussed the static version of the modified Bardeen black hole from ref. [29] by considering $a=0$.
The  static, spherically symmetric  spacetime of a modified Bardeen black hole  is described by the following form:\cite{nag2023thermodynamics}

\begin{equation}\label{1}
ds^2=-f(r)dt^2+ \frac{1}{h(r)} dr^2 +r^2 (d\theta^2 + \sin^2\theta d\phi^2)
\end{equation}
 where
\begin{equation}\label{1a}
  f(r) =\left(1-\frac{2Mr^2}{\left(q^2+r^2\right)^{3/2}}\right) \left(1-\frac{\mu M }{\left(g^2+r^2\right)^{3/2}}\right)
  \end{equation}

\begin{equation}\label{1b}
h(r)=  \left(1-\frac{2Mr^2}{\left(q^2+r^2\right)^{3/2}}\right)
\end{equation}

 This metric is parametrized by the magnetic charge $q$ , mass parameter $M$ whereas the parameters $\mu$ and $ g$ are used to modify the modified Bardeen black hole spacetimes
The above metric maintains the following conditions :
i) The metric preserves Schwarzschild-like behaviour at large r;
ii) It incorporates the 1-loop quantum correction;
iii) It allows for a finite time dilation between the center and infinity.
In the absence of parameter $\mu$, the metric (\ref{1}) reduces to ordinary regular Bardeen black hole. Further  $\mu=0$ and $ q=0$ yield Schwarzschild black hole.

The motion of photon around the modified Bardeen black hole is described by the Lagrangian formalism $\mathcal{L}=-\frac{1}{2}g_{\mu \nu}\dot{x}^{\mu}\dot{x}^{\nu}$.
Without loss of generality, we confined the photon trajectory around the modified Bardeen black hole on the equatorial plane $\theta=\frac{\pi}{2}$.For the modified Bardeen  spacetime metric (\ref{1})
  The Lagrangian equation  for the motion of photons around the  black hole is given by
  \begin{equation}\label{2}
  \begin{split}
 & \mathcal{L}=-\frac{1}{2}g_{\mu \nu}\dot{x}^{\mu}\dot{x}^{\nu}\\
& =f(r)dt^2-\frac{1}{h(r)} dr^2 -r^2 (d\theta^2 + \sin^2\theta d\phi^2)=\delta\\
 \end{split}
  \end{equation}
  where $\dot{x}^{\mu}$  denotes the four-velocity of photon, dot represents the differentiation w.r.t the affine paramer $\tau$.
  where $\delta=-1,0,1$ indicates the spacelike, null and timelike geodesics respectively.
  The photon travels around the modified Bardeen black hole along the null geodesic, which means $\delta=0$ .
The null geodesics  obtained from the equation (\ref{2}) are as follows:
   \begin{equation}\label{3}
   \dot{t}=\frac{dt}{d\tau}=\frac{E}{\left(1-\frac{2Mr^2}{\left(q^2+r^2\right)^{3/2}}\right) \left(1-\frac{\mu M }{\left(g^2+r^2\right)^{3/2}}\right)}
       \end{equation}

   \begin{equation}\label{4}
   \dot{\phi}= \frac{d\phi}{d\tau}=\frac{L}{r^2}
   \end{equation}

   \begin{equation}\label{5}
   \dot{r}=\frac{dr}{d\tau} =\pm\sqrt{h(r)\biggr(\frac{E^2 }{f(r)}-\frac{L^2}{r^2}\biggr)}
   \end{equation}
Where '+' corresponds to clockwise and '-' corresponds to counter clockwise of photon motion.
Here $E$ and $L$ respectively  are the energy and angular momentum of the particle where as the function $f(r)$ and $h(r)$ are taken from the equations (\ref{1a}) and (\ref{1b}) .

Equation (\ref{5}) can be expressed as
\begin{equation}\label{6}
    \frac{dr}{d\tau}+V_{eff}=0
\end{equation}
where the effective potential function $V_{eff}$ is described by
\begin{equation}\label{7}
 V_{eff}=   h(r)\biggr(\frac{L^2}{r^2}-\frac{E^2 }{f(r)}\biggr)
\end{equation}
For the critical photon ring orbit, the effective potential function $V_{eff}$  satisfies the critical conditions
$V_{eff}(r)=\frac{dV_{eff}(r)}{dr}=0$ ,$\frac{d^2V_{eff}(r)}{dr^2}>0$ (for stable)and
      $\frac{d^2V_{eff}}{dr^2}<0$ (for unstable )circular orbit.
It is observed that for the modified Bardeen,or ordinary regular Bardeen black hole, $\frac{d^2V_{eff}}{dr^2}|_{r_{ph}}<0$,which corresponds to the case of unstable circular orbit of photon (See Fig.\ref{fig:1}). Therefore, the photon rays , coming from infinity to the vicinity of modified Bardeen  black hole with minimum impact parameter at the closest distance $r_0$ revolved in unstable circular orbits around the black hole and  generate a photon sphere of radius $r_{ph}$.
\section{Shadows of modified Bardeen black hole}
Black hole Shadow is one of the most important fingerprints of the space-time geometry around the horizon of the black hole. It describes the properties of the black hole which optically depend on the gravitational lensing of the near by radiation. There has been nice reviews of shadow by the black hole with observables. Reader can see in more detail \cite{Cunha:2018acu,Perlick:2021aok}. Moreover, the EHT collaboration detects the image of the black hole by using the shadow properties of black hole \cite{EventHorizonTelescope:2019dse,EventHorizonTelescope:2019pgp,EventHorizonTelescope:2019ggy},which are attracted plenty of attention. In this section, we discuss the shadow of a modified Bardeen black hole and its observable by taking the example of supermassive black holes $M87*$ and $SgrA*$ in the center of the galaxy.
The black hole shadow is directly related to the critical impact parameter of the photon orbit.
Using the above condition one can define the critical impact parameter
\begin{equation}\label{8}
    u_{cr}=\frac{L}{E}=\frac{r_{ph}}{\sqrt{f(r_{ph})}}
    \end{equation}

where the  photon sphere radius  $r_{ph}$ is the largest real  root of the equation
     \begin{equation}\label{9}
    2f(r_{ph})-r_{ph} f^{\prime}(r_{ph})=0
     \end{equation}
The black hole shadow radius $r_{sh}$, in which the observer is located  far away from the black hole, can be expressed by the celestial coordinates (X, Y) as
\begin{equation}\label{10}
    r_{sh}=\sqrt{X^2+Y^2}=\frac{r_{ph}}{\sqrt{f(r_{ph})}}
\end{equation}
where the  the celestial
co-ordinate ($X$ , $Y$ ) at the boundary curve of black hole shadow define as
\begin{equation}\label{11}
X=\lim_{r_0\rightarrow \infty}(r_0^2  \sin\theta_0)\frac{d\phi}{dr}
\end{equation}

\begin{equation}\label{12}
Y=\lim_{r_0\rightarrow \infty}(r_0^2  \frac{d\theta}{dr})
\end{equation}
Here ,$r_0$ is the radial distance between black hole and observer whereas $\theta_0$ is the inclination angle between observer and black hole.


The radius of shadow can be expressed in terms of dimensional quantity by the transformations as $t\rightarrow\frac{t}{2M}$,$r\rightarrow
\frac{r}{2M}$, $q\rightarrow\frac{q}{2M}$ ,$g\rightarrow\frac{g}{2M}$ and $\mu\rightarrow\frac{\mu}{8M^2}$ in the function f(r) and it  defines as
\begin{equation}\label{13}
    r_{sh}=r_{ph}\left[\left(1-\frac{r^2}{\left(q^2+r^2\right)^{3/2}}\right) \left(1-\frac{\mu }{\left(g^2+r^2\right)^{3/2}}\right)\right]^{-1/2}
\end{equation}


\begin{table*}
 \caption{Estimation of the photon sphere radius  and the shadows radius  for the different values of the black hole parameters $\mu=0,1,3,5,7$ ;$g=0.2,1.2$; and $|q|=0,0.1,0.2,0.4$ .\label{table:1}}.\\
\begin{tabular}{|p{1.5cm}|p{1.5cm}|p{6.7cm}| p{7cm}|}
\hline
    $\mu$ & $g$
 & $r_{ph}$\vfill {\begin{tabular}{cccc}
    \hline
     \multicolumn{4}{c}{}\\
     $|q|=0.0$&$|q|=0.1$&$|q|=0.2$&$|q|=0.4$\\
\end{tabular}} & $r_{sh}$ \vfill {\begin{tabular}{cccc}
    \hline
     \multicolumn{4}{c}{}\\
     $|q|=0.0$&$|q|=0.1$&$|q|=0.2$&$|q|=0.4$\\
\end{tabular}}\\
\hline
$0$ & $0$
 & {\begin{tabular}{cccc}
    \hline
     \multicolumn{4}{c}{}\\
     $1.50000$&$1.48309$&$1.42899$&$1.09915$\\
\end{tabular}} &  {\begin{tabular}{cccc}
    \hline
     \multicolumn{4}{c}{}\\
     $2.59808$&$2.58054$&$2.52504$&$2.22043$\\
\end{tabular}}\\
\hline
{$1$} & $0.2$
 & {\begin{tabular}{cccc}
    \hline
     \multicolumn{4}{c}{}\\
     $1.74892$&$1.73991$&$1.71281$&$1.60617$\\
\end{tabular}} &  {\begin{tabular}{cccc}
    \hline
     \multicolumn{4}{c}{}\\
     $2.95742$&$2.94772$&$2.91837$&$2.79827$\\
\end{tabular}}\\

 & $1.2$
 & {\begin{tabular}{cccc}
    \hline
     \multicolumn{4}{c}{}\\
     $1.58109$&$1.56574$&$1.51702$&$1.24412$\\
\end{tabular}} &  {\begin{tabular}{cccc}
    \hline
     \multicolumn{4}{c}{}\\
     $2.79269$&$2.77813$&$2.73253$&$2.50259$\\
\end{tabular}}\\
\hline
$3$ & $0.2$
 & {\begin{tabular}{cccc}
    \hline
     \multicolumn{4}{c}{}\\
     $2.19323$&$2.18977$&$2.17956$&$2.14160$\\
\end{tabular}} &  {\begin{tabular}{cccc}
    \hline
     \multicolumn{4}{c}{}\\
     $3.50634$&$3.50177$&$3.48813$&$3.43500$\\
\end{tabular}}\\

 & $1.2$
 & {\begin{tabular}{cccc}
    \hline
     \multicolumn{4}{c}{}\\
     $1.84867$&$1.83975$&$1.81263$&$1.69950$\\
\end{tabular}} &  {\begin{tabular}{cccc}
    \hline
     \multicolumn{4}{c}{}\\
     $3.21602$&$3.20768$&$3.18241$&$3.07755$\\
\end{tabular}}\\
\hline
$5$ & $0.2$
 & {\begin{tabular}{cccc}
    \hline
     \multicolumn{4}{c}{}\\
     $2.51788$&$2.5158$&$2.50965$&$2.48650$\\
\end{tabular}} &  {\begin{tabular}{cccc}
    \hline
     \multicolumn{4}{c}{}\\
     $3.90482$&$3.90178$&$3.89272$&$3.85741$\\
\end{tabular}}\\
 & $1.2$
 & {\begin{tabular}{cccc}
    \hline
     \multicolumn{4}{c}{}\\
     $2.15933$&$2.15468$&$2.14087$&$2.08791$\\
\end{tabular}} &  {\begin{tabular}{cccc}
    \hline
     \multicolumn{4}{c}{}\\
     $3.60475$&$3.59975$&$3.58479$&$3.52573$\\
\end{tabular}}\\
\hline
$7$ & $0.2$
 & {\begin{tabular}{cccc}
    \hline
     \multicolumn{4}{c}{}\\
     $2.77523$&$2.77374$&$2.76930$&$2.75243$\\
\end{tabular}}& {\begin{tabular}{cccc}
    \hline
     \multicolumn{4}{c}{}\\
     $4.22334$&$4.22102$&$4.21412$&$4.18715$\\
\end{tabular}}\\
 & $1.2$
 & {\begin{tabular}{cccc}
    \hline
     \multicolumn{4}{c}{}\\
     $2.2.43328$&$2.43041$&$2.42189$&$2.38949$\\
\end{tabular}} &  {\begin{tabular}{cccc}
    \hline
     \multicolumn{4}{c}{}\\
     $3.93403$&$3.93056$&$3.92019$&$3.3.87958$\\
\end{tabular}}\\

\hline

\end{tabular}

\end{table*}


% Figure environment removed
By using the equations (\ref{9} )\&(\ref{13}),
estimation of the photon sphere radius and the radius of the shadow for the different values of the black hole parameters $\mu=0,1,3,5,7$ ;$g=0.2,1.2$; and $|q|=0,0.1,0.2,0.4$ has been shown in Table.\ref{table:1}. It is observed that for the fixed value of the parameters, $\mu$ and $g$, the photon sphere radius and the shadow radius of the black hole are decreased with the parameter $q$.


\subsection{Observational Constraints using $ M87^{*}$ and $Sgr A^{*}$ observations data}

 We have already calculated,how the photon sphere radius and the shadow radius are affected by the parameters $\mu$,$ g$, and $q$ in the previous section. Here, we intend to determine the value of the parameters $\mu$,$g$, and $q$ based on the observed angular diameter of the shadow. To avoid the difficulties in our investigation of parameters estimation, we consider the parameter  $g=0.2$ as fixed and then try to find out the range of the parameters  $\mu$ and $q$.
 For a distant observer, the  shadow image of a black hole is always measured by angular diameter $\theta_d$( \cite{Perlick:2021aok})as
\begin{equation}\label{14}
  \theta_d=  \frac{2u_{ph}}{D_{ol}}
\end{equation}
where $D_{ol}$ is the distance of the black hole to the observer.
The above equation can be expressed as
\begin{equation}\label{15}
   \theta_d  (\mu as)=  \biggr(\frac{6.191165 \times 10^6}{\pi}\biggr)(\frac{\gamma}{D_{ol}/Mps} )\biggr(\frac{2u_{ph}}{M}\biggr)
\end{equation}
where $\gamma$ represents the mass ratio of a black hole to the sun and $u_{ph}=u_{cr}$ is given from the Eq.(\ref{8}).



% Figure environment removed





By using the equations (\ref{14} )\&(\ref{15}), we study the angular diameter of black hole shadow as function parameters ($\mu/8M^2$ and $q/2M$), as displayed in Fig.\ref{fig:2}. To investigate the angular diameter of black hole shadow, we consider the supermassive black holes  $ M87^*$ having mass and distance from the earth
\cite{EventHorizonTelescope:2019pgp,EventHorizonTelescope:2019ggy} are  $M\approx 6.5 \times 10^9 \dot{O}
$,$D_{ol}\approx 16.8Mpc$ respectively and  $ Sgr A^*$ having mass and distance from the earth   $M\approx
4.28\times 10^6\dot{O} $,$D_{ol}\approx 8.32kpc$ respectively; where the  shadow angular diameter $\theta_d=42\pm3$ for $M87^*$ and $\theta_d=51.8\pm2.3$
\cite{EventHorizonTelescope:2022wkp,EventHorizonTelescope:2022apq}.The modified black hole parameters $q$ and $\mu$ for the fixed value of $g$  have been constrained by the EHT collaboration data for the angular shadow diameter of $M87^*$ and $SgrA^*$.It has been observed that the constrain ranges of the parameters $\mu$ and $q$  of modified Bardeen black hole as $-0.89\leq \mu/8M^2 \leq 0.4$ and $0\leq |q|\leq 0.185$ for $M87^*$; and $-1.38\leq \mu/8M^2 \leq 0.1$ and $0\leq |q|\leq 0.058$  for $SgrA^*$, keeping the fixed value of $g/2M=0.2$. Modified Bardeen black holes with the additional parameters $\mu$,$g$ and $q$ besides the mass M  of the
 black hole as the supermassive black holes $M87^*$ and $SgrA^*$; and it is observed that to be a viable astrophysical black hole candidate, the EHT result constrains the ($\mu$, $q$) parameter space. These results suggest that the modified Bardeen black hole satisfies the EHT constraint and it is possible to detect and distinguish the modified Bardeen black hole from the other astrophysical black hole in the future.

%%%%%%%%%%%%%%%%%%%% Strong gravitational lensing and its observable%%%%%%%%%%%%%%%%%%%%%%%%%%%%%%%%%%%%%%%%%%%%%%%%%%%%%

\section{Strong gravitational lensing  and it's observable}
Here, we would like to consider the strong gravitational lensing by the modified Bardeen black hole and its observables. We intend to investigate how the modified black hole parameters $\mu$,$q$, and $g$ affect the various astrophysical consequences such as angular position, separation, magnification, Einstein's ring, and time delays for the relativistic images and compared to the correspondence case of ordinary regular Bardeen($\mu=0$). as well as standard Schwarzschild ($\mu=0,q=0$) black holes.


Here, we investigate the strong deflection angle of photon rays due to a modified Bardeen black hole for the case that both the source and observer lie in the equatorial plane ($\theta=\frac{\pi}{2}$).
 To calculate the strong deflection angle of photon rays in the
equatorial plane ($\theta=\frac{\pi}{2}$), we rewrite the
metric(\ref{1}) by the dimensionless  operation,$t\rightarrow\frac{t}{2M}$,$r\rightarrow
\frac{r}{2M}$, $q\rightarrow\frac{q}{2M}$ ,$g\rightarrow\frac{g}{2M}$ and $\mu\rightarrow\frac{\mu}{8M^2}$ as
 \begin{equation}\label{16}
d\bar{s}^2=-A(x)dt^2+ B(x) dr^2 +C(x) d\phi^2
\end{equation}
where
$$
  A(r) =\left(1-\frac{r^2}{\left(q^2+r^2\right)^{3/2}}\right) \left(1-\frac{\mu }{\left(g^2+r^2\right)^{3/2}}\right)$$

$$B(r)=  \frac{1}{\left(1-\frac{r^2}{\left(q^2+r^2\right)^{3/2}}\right)}$$

  and $$  C(r)=r^2 $$


   When the particle are coming at the  closest distance $r=r_0$ to the central black hole, where $\frac{dr}{d\tau}=0$, one can  define the minimum impact parameter $u_0$ in terms of  closest distance $r_0$ \cite{Bozza:2002zj} as
  \begin{equation}\label{17}
   u_0=\frac{r_0}{\sqrt{A(r_0)}}
   \end{equation}



% Figure environment removed





% Figure environment removed
      

Behaviour of the photon sphere radius  $r_{ph}$ has been shown as a function of the parameter $q$ in Fig.\ref{fig:3}(a);and as a function of the parameters $\mu$ and $q$ in Fig.\ref{fig:3}(b). It is found that the photon radius $r_{ph}$ is slightly decreased with the parameter $q$ for the fixed  parameters $\mu$ and $q$; and while it is increased with the parameter $\mu$ for the fixed parameters $g$ and $q$. In Fig.\ref{fig:3}(a), it is also observed that the photon radius $r_{ph}$ for the case of a modified Bardeen black hole is more than the case of an ordinary regular Bardeen black hole (red solid line) and it is more than the value of $r_{ph}=1.5$, corresponds to the case of Schwarzschild(yellow horizontal line)  black hole \cite{Bozza:2002zj}.


When $r_0 \rightarrow r_{ph}$,  deflection angle becomes divergent  and for
 $r_0 >r_{ph}$, it becomes finite only. The photon having impact parameter $u<u_{ph}$ falls into the black hole and for the case when $u>u_{ph}$, it reaches the closest distance $r_0$ near the black hole; while for the impact parameter $u=u_{cr}=u_{ph}$, photon revolved in unstable orbit around the black hole.
  The  critical impact parameter for the unstable photon orbit
 $u_{ph}$ is given by
 \begin{equation}\label{18}
u_{ph} =\frac{r_{ph}}{\sqrt{A(r_{ph})}}
 \end{equation}
and its behaviours are displayed as a function of the parameter $q$ in Fig.\ref{fig:4}(a); and as a function of the parameters $\mu$ and $q$ in Fig.\ref{fig:4}(b). It is found that the critical impact parameter $u_{ph}$ is slightly decreased with the parameter $q$ for the fixed parameters $\mu$ and $q$; and while it is increased with the parameter $\mu$ for the fixed parameters $g$ and $q$ (see Table.3 also). In Fig.\ref{fig:4}(a), it is also observed that the critical impact parameter $u_{ph}/R_s$ for the case of modified Bardeen black hole is more than the case of ordinary regular Bardeen black hole (red solid line) and it is more than the value of $u_{ph}/R_{sh}=2.59808$, corresponds to the case of Schwarzschild(yellow horizontal line)  black hole\cite{Bozza:2002zj}.


The  strong deflection angle
for the modified Bardeen black hole spacetime, as a function of  the closest approach distance  $r_0$, can be read as \cite{Virbhadra:2002ju,Claudel:2000yi}

\begin{equation}\label{19}
\alpha_D(r_0)=I(r_0)-\pi=2\int_{r_0}^\infty \frac{\sqrt{B(r)}dr}{\sqrt{C(r) \sqrt{ \frac{A(r_0)C(r)}{A(r)C(r_0)}-1} }} dr-\pi
\end{equation}

 The strong deflection angle $\alpha_D(r_0)$  depends upon the relation between $r_0$ and $r_{ph}$ and while $r_0\approx r_{ph}$, it is increased.
 So, we define a new variable z as \cite{Bozza:2002zj}
 \begin{equation}\label{20}
 z=1-\frac{r_0}{r}
 \end{equation}


For,  $r_0\approx r_{ph}$, the strong  deflection angle becomes \cite{Chagoya:2020bqz}
  \begin{equation}\label{21}
     \alpha_D(u)= -\bar{a}~ log\left(\frac{u}{u_{ph}}-1\right) +\bar{b} +\mathcal{O}(u -u_{ph})
     \end{equation}
    where

   \begin{equation}\label{22}
       \bar{a}
       =\sqrt{\frac{2A(r_{ph}) B(r_{ph})}{A(r_{ph})C^{\prime\prime}(r_{ph})-A^{\prime\prime}(r_{ph}) C(r_{ph})}}
   \end{equation}
   and
   \begin{equation}\label{23}
      \bar{b}=-\pi + I_R(r_{ph}) + \bar{a}~ log \biggr[r_{ph}^2\biggr(\frac{C^{\prime\prime}_{ph}}{c_{ph}}-\frac{A^{\prime\prime}_{ph}}{A_{ph}}\biggr)\biggr],
  \end{equation}

Here,
  $ I_R(r_{ph})= 2 \int_{0}^{1} \Bigg(r_{ph}\biggr[\sqrt{\frac{B(z)}{C(Z)}}\biggr(\frac{A(r_{ph})}{C(r_{ph})}\frac{C(z)}{A(z)}-1\biggr)\frac{1}{(1-z)^2} \biggr]-\frac{\bar{a}}{z ~r_{ph}}\Bigg)dz  $ which
  is obtained numerically.



% Figure environment removed




% Figure environment removed



 \begin{table*}
 \caption{Estimation  of strong lensing coefficients with the different value of black hole parameters  $\mu=0,1,3$ ;$g=0.2,1.2$; and $|q|=0,0.05,0.1,0.2,0.4$ .\label{table:2}}
\begin{tabular}{p{2.5cm} p{2.5cm} p{3cm} p{2.5cm} p{4cm} p{2cm} }
\hline
\hline
\multicolumn{5}{c}{Strong Lensing Coefficients }\\
$\mu$ &$g $& $|q|$& $\bar{a}$  & $\bar{b} $ &$ u_{ph}/R_{sh}$\\
\hline
$0$
& &0 & 1.00&-0.40023&2.59808\\
\hline
%\multirow{0}
%&\multirow{0}
&& 0.05 & 1.0028 &-0.401611&2.59373\\
0&{}& 0.1 & 1.01151 &-0.406055&2.58054\\
&{}&0.2 & 1.05179 &-0.429858&2.58054\\
&{}&0.4 & 1.55181 &-1.18175&2.22043\\

\hline
%\multirow{1}
%&\multirow{0.2}
&& 0.05 & 0.921272 &-0.477287&2.955\\
&{0.2}& 0.1 & 0.923386 &-.476401&2.94772\\
&{}&0.2 & 0.931409 &-0.470916&2.91837\\
1&{}&0.4 & 0.949693 &-0.399493&2.79827\\
%%%%%%%%%%%
\\
%\multirow{}
%&\multirow{1.2}
&& 0.05 & 1.03705 &-0.516087&2.78908\\
&{1.2}& 0.1 & 1.04578 &-0.52368&2.77813\\
&{}&0.2 & 1.08555 &-0.561458&2.73253\\
&{}&0.4 & 1.46926 &-1.19512&2.50259\\

\hline
%\multirow{3}
%&\multirow{}{}{0.2}
&& 0.05 & 0.761981 &-0.384548&3.5052\\
&{0.2}& 0.1 & 0.761421 &-0.381893&3.50177\\
&{}&0.2 & 0.75901 &-0.37101&3.48813\\
3&{}&0.4 & 0.746808 &-0.324242&3.435\\
%%%%%%%%%%%%%%
\\
%\multirow{}
%&\multirow{1.2}
&& 0.05 & 0.986331 &-0.641022&3.21394\\
&1.2& 0.1 & 0.989931 &-0.644732&23.20768\\
&{}&0.2 & 1.00474 &-0.660005&3.18241\\
&{}&0.4 & 1.06929 &-0.7222588&3.07755\\


\hline
\hline
\end{tabular}
\end{table*}






% Figure environment removed





We numerically obtain the
lensing coefficients $\bar{a}$ and $\bar{b}$ and $u_{ph}/R_{sh}$  with the modified Bardeen black hole  parameters $\mu=0,1 ,3$,$g=0.2,1.2 $  and $q=0,0.05,0.1,0.2,0.4$ (see Table.\ref{table:2}). From this Table, it is seen that for the fixed value of  parameters $g=0.2,1.2$
$\mu(=0,1,3)$ lensing coefficients $\bar{a}$ increases with
increasing magnitude of the parameter $q$ while  lensing
coefficients $\bar{b}$ decreases  ;except for the case when $g=0.2$ and
$\mu(=3)$ . When $\mu=0$ and $q=0$, the value of
lensing coefficients $\bar{a}=1$ and $\bar{b}=-0.40023$, which correspond to the case of the Schwarzschild black hole \cite{Bozza:2002zj}. Behaviour of the lensing coefficients $\bar{a}$ and $\bar{b}$ are displayed in Figs.(\ref{fig:5}\&\ref{fig:6}).
The behaviour of the deflection angle $\alpha_D$  of photon around the modified Bardeen black hole is displayed in Fig.\ref{fig:7}. In Figs.\ref{fig:7}(a) and \ref{fig:7}(b), it is observed that the deflection angle $\alpha_D$ is increased with the increasing magnitude of charge parameter $q$ and decreases with the increasing value of $\mu $, keeping other parameters fixed. Furthermore, it is found that
the deflection angle $\alpha_D$  for the case of a modified Bardeen black hole larger than the case of Schwarzschild ($\mu=0, q=0$)and smaller than the case of an ordinary regular Bardeen ($\mu=0$) black hole.
The deflection angle $\alpha_D$ decreases with the critical impact parameter $u$ with the different value of the parameter $\mu$ for the fixed value of $q$ and $g$ (see Fig.\ref{fig:7}(c) ); and with different magnitude values of parameter  $q$ for the fixed value of $\mu$ and $g$ (see Fig.\ref{fig:7}(d) ).

\subsection{Lensing observables}

Next, we study the strong lensing observables by modified Bardeen black hole. Here,  we assume the case where the observer and source are very far from the black hole(lens)  and they are almost aligned. Further, we assume the source is behind the black hole(lens). Therefore,
the lens equation can be defined as
\cite{Bozza:2001xd}
\begin{equation}\label{24}
 \beta=\theta-\frac{D_{ls}}{D_{os}}\Delta \alpha_{n}
\end{equation}
where $\Delta\alpha_{n}=\alpha_D(\theta) - 2n\pi $ is the offset deflection angle and $n$ indicates the number of loops of photon ray around the black hole. Here, the
angles $\beta$ and $\theta$  respectively are the angular separations between  the  black hole(lens) and source;and between the observer and source, whereas $D_{ls}$,
$D_{ol}$,$D_{os}$ are the lens-source, observer-lens,
observer-source distance respectively such that
$D{os}=D_{ol}+D_{ls}$.


Using the Eqs. (\ref{19}) and (\ref{24}), the angular separation between the  black hole(lens) to the  $n^{th}$ relativistic  image can be
expressed as
\begin{equation}\label{25}
     \theta _n =  \theta^0 _n - \frac{u_{ph}e_n(\theta_n^0-\beta)D_{os}}{\bar{a}D_{ol}D_{ls}}
 \end{equation}
where $$ e_n=e^{\frac{\bar{b}-2n\pi}{\bar{a}}},$$
$$\theta^0_n=\frac{u_{ph}(1+e_n)}{D_{ol}}$$.
Here,$\theta^0_n$ is  the angular
image position for the case when photon winds complete $2n\pi$ around the black hole(lens).

As strong gravitational lensing preserves the surface brightness, the magnification of the relativistic is the ratio of the solid angle subtended by the $n$-th  image and the source \cite{Virbhadra:1999nm}.
For the $n$-th relativistic image, the magnification is then obtained  as (\cite{Bozza:2002zj})
\begin{equation}\label{26}
\mu_n=\biggr(\frac{\beta}{\alpha}\frac{d\beta}{d\alpha}\biggr)^{-1}\biggr|_{\theta_0}=\frac{ u^2_{ph}(1+e_n)e_n D_{os}}{\beta \bar{a}D_{ls}D^2_{ol}}
\end{equation}

The above equation suggests that the first relativistic image is the brightest image and
the magnification decreases exponentially with $n$ i.e. bright of this image dominates over the other relativistic images. It is clear that the equation (\ref{26}) becomes divergent when $\beta \rightarrow 0$, suggesting that perfect alignment maximizes the possibility of detection of relativistic images.

Here, We consider the case when the brightest image, i.e. the outermost image $\theta_1$  is resolved
as a single image and the remaining inner images are packed together at $\theta_{\infty}$  ($\theta_n|_{n\rightarrow \infty}=: \theta_{\infty}$).
With the help of the deflection angle in the equation (\ref{21}), one can obtain strong lensing observables such as the angular position of the set of images $\theta_{\infty}$, angular separation between the outermost and innermost images  $S$  and  relative magnification $r_{mag}$ between the outermost relativistic image and other
pact of inner relativistic images can be defined as

\cite{Kumar:2022fqo,Bozza:2002zj}.
\begin{equation}\label{27}
\theta_{\infty}=\frac{u_{ph}}{d_{ol}}
\end{equation}
 \begin{equation}\label{28}
S= \theta_1-\theta_{\infty}\approx\theta_{\infty}e^\frac{(\bar{b} -2\pi)}{\bar{a}}
  \end{equation}

  \begin{equation}\label{29}
  r_{mag}=\frac{\mu_1}{\Sigma^\infty_{n=2}\mu_{n}}\approx \frac{5\pi}{\bar{a}log(10)}
  \end{equation}

If the strong lensing observables   $\theta_{\infty}$,$S$, and $r_{mag}$ are measured from the observation, the lensing coefficients $\bar{a}$,$\bar{b}$ and the minimum impact parameter $u_{ph}$  can be obtained easily by inverting the equations (\ref{27}),(\ref{28}) and (\ref{29}), and Further, compared to the theoretically obtained values. Using these findings, one can identify the nature of the modified Bardeen, ordinary regular Bardeen, and Schwarzschild black hole; and distinguish among them.\\

Considering the supermassive black holes $ M87^*$, $ Sgr A^*$ and  $NGC 7457$
in the nearby galaxies, we estimate the observable quantities
$\theta_{\infty}$,$S$, and $r_{mag}$ in the context of a modified Bardeen black hole (See Table.4). The mass and distance from the earth for $ M87^*$
\cite{EventHorizonTelescope:2019pgp,EventHorizonTelescope:2019ggy} are  $M\approx 6.5 \times 10^9 \dot{O}
$,$D_{ol}\approx 16.8Mpc$, for $ Sgr A^*$ are  $M\approx
4.28\times 10^6\dot{O} $,$D_{ol}\approx 8.32kpc$
\cite{EventHorizonTelescope:2022wkp,EventHorizonTelescope:2022apq},  and for $NGC 7457$ are  $M\approx 8.95
\times 10^6 \dot{O}$, $D_{ol}\approx 12.53~Mpc$ \cite{Kormendy:2013dxa} .



% Figure environment removed



% Figure environment removed










% Figure environment removed


      


 \begin{table*}
\begin{center}
 \caption{Estimation  of strong lensing observables for supermassive BHs $ M87^*$,$ Sgr A^*$, $NGC 7457$ with the different value of black hole parameters  $\mu=0,1,3$ ;$g=0.2,1.2$; and $|q|=0,0.05,0.1,0.2,0.4$.The observable quantity $r_{mag}$ does not depend on the mass  or distance of the black hole from the observer.\label{table:3}}
 \begin{tabular}{ccc|cc|cc|cc|c}
 \hline
 \hline
&{parameters} && {$ M87^*$}& &{$ Sgr A^*$ }&&{ $NGC 7457$}&&$ M87^*$,$ Sgr A^*$ , $NGC 7457$
\\
$\mu$& $g$ & $|q|$ & $ \theta_{\infty} (\mu as)$&$S(\mu as)$&
$\theta_{\infty} (\mu as)$ &  $S (\mu as)$ & $\theta_{\infty} (\mu as)$ &$S(\mu as)$&$r_{mag}$ \\

\hline
\hline
{0} & 0 & 0&19.9633&0.024984&26.3315&0.0329538&0.0365211&$4.57\times 10^{-5}$&6.82188\\
\hline
%\multirow{0}
&  & 0.05 &19.9298&0.025377&26.2874&0.0334722&0.03646&$4.63\times 10^{-5}$&6.80283\\
0&  & 0.1 &19.8285&0.0266224&26.1537&0.0351149&0.0362745&$4.87\times 10^{-5}$&6.74426\\
& & 0.2 &19.402&0.0328069&025.5912&0.0432723&0.0354944&$6.0\times 10^{-5}$&6.48597\\
&  & 0.4 &17.0615&0.138949&22.504&0.183274&0.0312125&$2.54\times 10^{-4}$&4.39608\\
\hline
%\multirow{1}&
%\multirow{0.2}
&& 0.05 &22.7058&0.0147639&29.9489&0.0194735&0.0415383&$2.70\times 10^{-5}$&7.40485\\
&0.2 &0.1 &22.6499&0.0149914&29.8751&0.0197736&0.041436&$2.74\times 10^{-5}$&7.3879\\
& &0.2 &22.4243&0.0159015&29.5776&0.020974&0.0410234&$2.91\times 10^{-5}$&7.32426\\
1& &0.4&21.5015&0.0189008&28.3604&0.0249302&0.0393352&$3.46\times 10^{-5}$&7.18325\\
\\
%\multirow{}{}{}&
%\multirow{}{}{1.2}
&& 0.05 &21.4309&0.0304544&28.2673&0.0401692&0.039206&$5.57\times 10^{-5}$&6.57816\\
&1.2 &0.1 &21.3467&0.0318095&28.1563&0.0419567&0.0390521&$5.82\times 10^{-5}$&6.52325\\
& &0.2 &20.9964&0.0383546&27.6942&0.0505896&0.038411&$7.02\times 10^{-5}$&6.28426\\
& &0.4&19.2295&0.118434&25.3637&0.156215&0.0351788&$2.17\times 10^{-4}$&4.64307\\
\hline
%\multirow{}{}{3}&
%\multirow{}{}{0.2}
&& 0.05 &26.9335&0.00426567&35.5252&0.00562641&0.0492725&$7.8\times 10^{-6}$&8.95282\\
&0.2 &0.1 &26.9071&0.00424895&35.4904&0.00560436&0.0492242&$7.77\times 10^{-6}$&8.95941\\
& &0.2 &26.8023&0.00417578&35.3522&0.00550785&0.0490325&$7.64\times 10^{-6}$&8.98787\\
3& &0.4&26.394&0.00379368&34.8137&0.00500385&0.0482857&$6.94\times 10^{-6}$&9.13472\\
\\
%\multirow{}{}{}&
%\multirow{}{}{1.2}
&& 0.05 &24.6955&0.0220698&32.5732&0.02911&0.0451782&$4.03\times 10^{-5}$&6.91642\\
&1.2 &0.1 &24.6473&0.0225118&32.5098&0.029693&0.0450902&$4.11\times 10^{-5}$&6.89127\\
& &0.2 &24.4532&0.0243877&32.2537&10.0321674&0.044735&$4.46\times 10^{-5}$&6.7897\\
& &0.4&23.6474&0.0337682&31.1909&0.0445402&0.043261&$6.17\times 10^{-5}$&6.37982\\
\hline
\hline
\end{tabular}
\end{center}
\end{table*}



The behaviour of the strong lensing observables angular image position $\mathit{\theta_{\infty}}$, angular image separation $S$, and relative magnification $\mathit{r_{mag}}$   as a function of the parameter $q$  and as the function of the parameters $q$ and $\mu$ for the fixed value of $g=0.2$ for $M87^{*}$  and for $SgrA^{*}$  has been shown in Figs.\ref{fig:8},\ref{fig:9}\& \ref{fig:10} (see Table.3 also).  It is observed that the angular image position $\mathit{\theta_{\infty}}$ and relative magnification $\mathit{r_{mag}}$    decrease with the increasing magnitude of charge parameter $q$ while angular image separation $S$ increases with the increasing magnitude of both the parameters $\mu $ and $q$, keeping other parameters fixed. But the relative magnification $\mathit{r_{mag}}$ increases with increasing value of the parameter $\mu $  for the fixed value of $g $ and $q$. Furthermore, it is found that
the relative magnification $\mathit{r_{mag}}$  for the case of modified Bardeen black hole larger than the case of Schwarzschild ($\mu=0, q=0$)as well as ordinary regular Bardeen ($\mu=0$) black hole.



\subsection{Einestein Ring}

When the source, black hole (lens), and observer are perfectly aligned
i.e., when $\beta=0$, a black hole (lens) deflects the light rays  in all
direction  such that a ring-shaped image is produced, which is called an Einstein ring
\cite{Einstein:1936llh,Liebes:1964zz,Mellier:1998pk,Bartelmann:1999yn,Schmidt:2008hc,Guzik:2009cm}.

By simplifying the equation (\ref{25}) for $\beta=0$,we obtain the angular radius of $n^{th}$ relativistic images as follows
\begin{equation}\label{30}
     \theta _n =  \theta^0 _n \biggr(1 - \frac{u_{ph} e_n  D_{os}}{\bar{a}D_{ls}D_{ol}}\biggr)
 \end{equation}

Considering  the case where the black hole (lens) is  located at a half distance between the source
and receiver i.e., $D_{os}=2D_{ol}$ and taking $D_{ol}>>u_{ph}$, thus the
angular radius of the $n^{th}$ relativistic Einstein ring in the context of a modified Bardeen black hole is given by

\begin{equation}\label{31}
\theta^E_n=\frac{ u_{ph}(1+e_n)}{D_{ol}}
\end{equation}



% Figure environment removed


      
The angular radius $\theta^E_1$ denotes the outermost Einstein ring and which is shown in Fig.\ref{fig:11} for the supermassive black holes  $ M87^*$ (Fig.\ref{fig:11}(a)\&(b)) and $ Sgr A^*$(Fig.\ref{fig:11}(c)\&(d)).
It is observed that for  the fixed parameter $\mu$ and $g$, the outermost Einstein rings decrease with the increasing magnitude of the parameter $q$ in the context of both the supermassive black holes  $
Sgr A^*$ and  $ M87^*$.Further, it is found that the outermost Einstein rings for the modified Bardeen black hole are more than the ordinary regular Bardeen black hole.

\subsection{Time delay in strong field Limit}

Time delay is one of the most important observable by the strong gravitational lensing phenomenon, which is obtained by the time difference between the formation of two relativistic images. The time difference is caused when the photon travels in a different path around the black hole. The time travel by the different photon paths for the different relativistic images is different and hence, there is a time difference between the different relativistic images. If the time signals of two relativistic images are distinguished from the observation, one can calculate the time delay between two signals\cite{Bozza:2003cp}.The time taken by a photon to revolve  around the black hole \cite{Bozza:2003cp} is read as
 \begin{equation}\label{32}
\tilde{T}=\tilde{a}log\biggr(\frac{u}{u_{ph}}-1\biggr)+\tilde{b}+\mathcal{O}(u-u_{ph})
\end{equation}

With the help of the above  Eq.(\ref{32}), one can compute the time difference between two relativistic images.


For spherically static symmetric black hole spacetime, the time delay between  two relativistic
images, when the relativistic images are on the same side of the black hole, are obtained as
\begin{equation}\label{33}
\Delta T_{2,1}=2\pi u_c=2\pi D_{ol} \theta_{\infty}
\end{equation}
If  the time delay
$\Delta T_{2,1}$ between two relativistic images with an accuracy $5\%$ and critical impact parameter $u_{ph}$ with a negligible error are obtained, then it can be possible to measure the black hole distance with an accuracy of $5 \%$.

The time delay $\Delta T_{2,1}$ for various supermassive black holes in the context of the standard Schwarzschild  ($\mu=0$,$q=0 $), ordinary regular Bardeen ($\mu =0$,$q=0.3 $) and modified  Bardeen ($\mu =3$,$q=0.3 $)  black holes
have been estimated numerically (see Table.5).
It is found that the time delay
$\Delta T_{2,1}$ between two relativistic images in the context of modified Bardeen black hole ($\mu =3$,$g=0.2$,$q=0.3 $)  much more than cases of standard Schwarzschild  ($\mu=0$,$q=0 $) as well as ordinary regular Bardeen ($\mu=0$,$q=0.3 $).


 \begin{table*}
 \caption{Estimation of time delay for some supermassive BHs in the context of Schwarzschild  ($\mu=0$,$q=0 $), ordinary regular Bardeen ($q=0.3,\mu=0$) and modified Bardeen ($\mu=3$,$q=0.3 $) black hole spacetimes. Mass(M) and distance $D_{ol}$
respectively are taken in solar mass and Mpc units \cite{Kormendy:2013dxa}.  Time delays $\Delta T_{2,1}$ are estimated in minutes.
 \label{table:4}}
\begin{tabular}{ p{2.5cm}p{2.cm}p{2.cm}p{2.5cm}p{2.5cm}p{2.5cm} p{3.5cm} p{2.5cm}}
\hline
\hline
 Galaxy  &$M(M_{\odot})$& $D_{ol}(Mpc)$&
$\Delta T_{2,1}$\vfill ($\mu=0,q=0$) &  $\Delta T_{2,1}$\vfill($\mu=0,q=0.3$)   &$\Delta T_{2,1}$ \vfill ($\mu=3,g=0.2,q=0.3$) \\

\hline
\hline
%\multirow
M87 & $6.5\times 10^9 $& $16.68$ & $17378.9$ & $16186.6$ & $21000.2$\\
NGC 4472 & $2.54\times 10^9 $& 16.72&6791.12&6325.3&8206.24\\
NGC 4395 & $3.6\times 10^5 $& 4.3&0.962522&0.896499&1.16309\\
NGC1332 & $1.47\times 10^9 $& 22.66&3930.3&3660.71&4749.28\\
NGC 7457 & $8.95\times 10^6 $& 12.53&23.9294&28.2880&28.9157\\
NGC 1399 & $8.81\times 10^8 $& 20.85&2355.5&2193.93&2846.34\\
NGC 1374 & $5.90\times 10^8 $& 19.57&1577.47&1469.26&1906.17\\
NGC 4649 & $4.72\times 10^9 $& 16.46&12619.7&11754.1&15249.4\\
NGC 3607 & $1.37\times 10^8 $& 22.65&366.293&341.168&442.62\\
NGC 4459 & $6.96\times 10^7 $& 16.01&186.088&173.323&224.864\\
NGC 4486A & $1.44\times 10^7 $& 18.36&38.5009&35.86&46.5236\\
NGC 1316 & $1.69\times 10^8 $& 20.95&451.85&420.857&546.006\\
NGC 4382  & $1.30\times 10^7 $& 17.88&34.7577&32.3736&42.0004\\
NGC 5077 & $8.55\times 10^8 $& 38.7&2285.9&2129.9&2762.34\\
NGC 7768  & $1.34\times 10^9 $& 116.0&3582.72&3336.97&4329.28\\
NGC 4697  & $2.02\times 10^8 $& 12.54&546&503.036&652.622\\
NGC 5128  & $5.69\times 10^7 $& 3.62&152.132&141.697&183.833\\
NGC 5576  & $2.73\times 10^8 $& 25.68&729.912&679.845&882.009\\
NGC 3608  & $4.65\times 10^8 $& 22.75&1243.26&1157.98&1502.32\\
M32  & $2.45\times 10^6 $& 0.806&6.5505&6.10118&7.91547\\
Cygnus A  & $2.66\times 10^9 $& 242.7&7111.97&6624.13&8593.93\\




\hline
\end{tabular}
\end{table*}

\section{Comparison with observation:}
It is easily known that the standard astrophysical black holes like ordinary regular Bardeen, and Schwarzschild black holes are a special form modified Bardeen black holes. A few works have been investigated on the shadow cast and gravitational lensing in the context of ordinary regular Bardeen as well as Schwarzschild black holes, which are already discussed previously. In the present work, we extend the work done by some literature, Bozza, Virbhadra Ellis for Schwarzschild black hole \cite{Bozza:2002zj,
Virbhadra:1999nm, Bozza:2003cp}, He et al. \cite{He:2021htq} and also Stuchik and Schee  \cite{Schee:2015nua,Stuchlik:2019uvf} work's for regular Bardeen black holes; and Islam et al.'s \cite{islam2022strong}  works for Bardeen black hole in 4D Einstein's gravity.
Islam's et al. investigated the strong gravitational lensing by  Bardeen black hole in 4 dimensional  Einstein's Gauss-Bonnet gravity and constrained the black hole parameters using some supper massive black hole data. In their works, it is also mentioned that the regular Bardeen black hole solution is a special solution of Bardeen black hole in 4-dimensional Einstein's Gauss-Bonnet gravity. Recently,  He et al. studied the shadow and observed properties of Bardeen black holes surrounded by different accretion models.
In this paper, we discuss the Shadow and strong gravitational lensing effects of the modified Bardeen black hole and compared it to the Schwarzschild ($\mu=0$ and $q=0$) as well as ordinary regular Bardeen($\mu=0$) black hole.

In the Shadow discussion, the radius of the black hole shadow has been obtained numerically  (See. Table.\ref{table:1}). It is found that the radius of the black hole corresponds to the modified Bardeen black hole is larger than the regular Bardeen as well as Schwarzschild black holes. The angular diameter of shadow for modified Bardeen black hole as a function of parameters ($\mu/8M^2$ and $q/2M$) has been displayed in Fig.\ref{fig:2} . It is observed that the angular diameter of the modified Bardeen black hole is larger than the regular Bardeen black hole in the context of supermassive black holes $M87^{*}$ and $Sgr A^{*}$ (see Fig.\ref{fig:2}). Furthermore, it is observed that the red solid curve corresponds to $\theta_d=39.4615 \mu as$ for modified Bardeen black hole, blue solid curve corresponds to $\theta_d=38.80 \mu as$ for  Bardeen black hole and the Green solid curve correspond to $\theta_d=39.9265 \mu as$ for Schwarzschild black hole within $1\sigma$ region of the measured angular diameter $\theta_d=42\pm3 \mu as$ for $M87^{*}$(Fig.(Fig.\ref{fig:2}(a)); and for $SgrA^{*}$(\ref{fig:2}),the red solid curve corresponds to $\theta_d=52.15 \mu as$ for modified Bardeen black hole, blue solid curve correspond to $\theta_d=49.15 \mu as$ for  Bardeen black hole and the Green solid curve correspond to $\theta_d=52.77 \mu as$ for Schwarzschild black hole within $1\sigma$ region of the measured angular diameter $\theta_d=51.8\pm 2.3 \mu as$ .

In the strong gravitational lensing investigation, we apply the method proposed by \cite{Bozza:2002zj} which can be used to distinguish the various types of static spherically symmetric black holes and also investigate the various astrophysical consequences by considering the supermassive black holes(see Table.\ref{table:3} \& Table.\ref{table:5} ) in the context of modified Bardeen, ordinary regular Bardeen, and Schwarzschild black holes. Furthermore, we estimate the strong observable quantities such as angular position $\theta_{\infty}$, $S$ and $r_{mag}$ in the context of modified Bardeen, ordinary regular Bardeen, and Schwarzschild black holes by considering the supermassive black hole $NGC 4649$  having mass $M=4.72\times 10^9$ and distance $D_{OL=0.008}Mpc$ and numerically compared among them.

Considering the supermassive black hole having the same mass and distance, it is found from our estimations that the angular position of innermost image $\theta_{\infty}$  and angular separation $S$ is always greater than the case ordinary regular Bardeen as well as Schwarzschild black holes. Also, the modified Bardeen black hole has larger relative magnifications.
the numerical differences of $\theta_{\infty}$, $S$ and $r_{mag}$ in between modified Bardeen ($\mu=3$,$g=0.2$, $q=0.3$ and ordinary regular Bardeen($\mu=0$, $q=0.3$  are respectively as $\sim 5.9 \mu as$, $\sim 0.3 \mu as$ and $\sim 3.1$ magnitude while the differences in between modified Bardeen ($\mu=3$,$g=0.2$, $q=0.3$ and standard Schwarzschild
 black hole ($\mu=0$, $q=0$ ) are respectively as $\sim 4.9 \mu as$, $\sim 0.18 \mu as$ and $\sim 2.2$ magnitude .It is also observed that the angular position $\theta_{\infty}\in (19.59,19.82)$ , separation $S\in (0.00297,0.003138) $ and relative magnification $r_{mag}\in (3.47,3.5)$; when $\mu=03$,$g=0.2$ and $0\leq |q|\leq 3$.These findings suggest that the outermost image for the modified Bardeen black hole is very nearer to the innermost images and it may be possible to separate the images from the other black hole images.
In other words, If the outermost relativistic image can be detected, one can distinguish the modified Bardeen black hole from the other standard astrophysical black holes like Schwarzschild, ordinary regular Bardeen black holes, etc, using the technology method. However, it is so difficult in observations as angular separation of the relativistic images is not more than $\sim 0.3 \mu as$. It also has been observed that Einstein's ring $\theta^{E}_{\infty}$for the modified Bardeen black hole is larger than the other standard astrophysical black holes such as Schwarzschild and ordinary regular Bardeen black holes  (See in Fig.\ref{fig:11}).
Further, it is observed (See Table.\ref{table:4}) that time delays between two relativistic images for the case of a modified Bardeen black hole ($\sim 15249.4 minutes$) are significantly much more than the other astrophysical black holes such as  Schwarzschild black hole ($\sim 12619.7 minutes$ ) as well as ordinary regular Bardeen black holes ($\sim 11754.1 minutes$ ) in the context of supermassive black hole  $NGC 4649$. These results suggest that if one can distinguish the first and second relativistic images from the observation, the time delay between them may provide a better chance to distinguish the modified Bardeen black hole from the other astrophysical black holes such as a Schwarzschild or an ordinary regular Bardeen black hole. Thus, a modified Bardeen black hole
could be distinguished quantitatively from a Schwarzschild or an ordinary regular Bardeen black hole.


If a modified Bardeen black hole is identified and confirmed to exist, it would have several significant implications and consequences for our understanding of black holes and general relativity. Here are some potential implications:
\begin{itemize}
    \item
 The existence of a modified Bardeen black hole would provide empirical evidence for alternative theories of gravity that deviate from the predictions of general relativity. It would indicate that the standard black hole solution of general relativity is not the only viable description of black holes and that modifications are necessary to explain the observed astrophysical phenomena.
\end{itemize}
\begin{itemize}
    \item  The modified Bardeen black hole would exhibit distinct astrophysical signatures and observational features compared to standard black holes. These could include variations in the gravitational lensing effects, the structure of the event horizon, the formation of accretion disks, and the emission of gravitational waves. Identifying and characterizing these unique signatures would deepen our understanding of the underlying physics and properties of black holes.
\end{itemize}
\begin{itemize}
    \item  The modified Bardeen black hole could provide insights into the nature of dark matter. Some modified gravity theories propose that the effects attributed to dark matter can be explained by modifications to the gravitational laws at large scales. Observations of the modified Bardeen black hole could offer constraints on such theories and provide clues about the nature of dark matter.
\end{itemize}
\begin{itemize}
    \item  The no-hair theorem in general relativity states that a black hole is characterized solely by its mass, electric charge, and angular momentum. If the modified Bardeen black hole possesses additional parameters or properties beyond these three, it would challenge the no-hair theorem. Confirming the existence of a modified Bardeen black hole would require revisiting our understanding of black hole uniqueness and the fundamental properties of black holes.
\end{itemize}
\begin{itemize}
    \item  The discovery of a modified Bardeen black hole would push the boundaries of our current understanding of fundamental physics. It would inspire further theoretical investigations, stimulate new research directions, and potentially lead to the development of more comprehensive theories that can explain the behavior of black holes in modified gravity scenarios.
\end{itemize}
In summary, identifying a modified Bardeen black hole would have profound implications for our understanding of gravity, astrophysics, and fundamental physics. It would open up new avenues of exploration and deepen our knowledge of the nature and properties of black holes.

 %\renewcommand{\arraystretch}{1.3}


\begin{table*}
 \caption{Estimation of observables by taking the supermassive black hole $NGC 4649$  having mass $M=4.72\times 10^9 M_{\odot}$ and distance $D_{OL}=16.46 Mpc$ in the context of  Schwarzschild, ordinary regular Bardeen,   and modified Bardeen black hole spacetimes.
 \label{table:5}}
\begin{tabular}{ |p{1.6cm}|p{1.5cm}| p{6.5cm}|p{6.4cm}|  }
\hline
%\multicolumn{7}{|c|}{}
& {Schwarz\vfill -schild black hole}&  { regular Bardeen black hole($\mu=0$ ) }\vfill {\begin{tabular}{cccc}
    \hline
    \multicolumn{1}{c}{}\\
    $|q|=0.05$&$|q|=0.1$&$|q|=0.2$&$|q|=0.3$\\
\end{tabular}} & {modified Bardeen black hole($\mu=3$ ,$g=0.2$)}\vfill{\begin{tabular}{cccc}
\hline
    \multicolumn{1}{c}{}\\
     $|q|=0.05$& $|q|=0.1$&$|q| =0.2$&$|q|=0.3$\\
\end{tabular}}\\
\hline
\textbf {$\theta_{\infty}$\vfill($\mu$arcsecs)} & {14.6901}  &{\begin{tabular}{cccc}
    \hline
    \multicolumn{1}{c}{}\\
    {14.6656}& {14.591}& {14.27} & {13.6825}\\
\end{tabular}} &{\begin{tabular}{cccc}
    \hline
    \multicolumn{2}{c}{}\\
    { 19.8192}& {19.7998}&{19.7227}&{19.5959}\\
\end{tabular}}\\
\hline
\textbf {$S$ \vfill($\mu$arcsecs) }&{0.0184}  &{\begin{tabular}{cccc}
    \hline
    \multicolumn{1}{c}{}\\
   { 0.018}&{0.0196}&{0.024}&{0.037}\\
\end{tabular}} &{\begin{tabular}{cccc}
    \hline
    \multicolumn{1}{c}{}\\
    {0.0.00313}&{0.0.00312}& {0.00307}&{0.00297}\\
\end{tabular}}\\
\hline
\textbf {$r_{mag}$} &{6.82188 } &{\begin{tabular}{cccc}
    \hline
    \multicolumn{1}{c}{}\\
   { 6.80282}&{6.74425}& {6.48595}& {5.92957}\\
\end{tabular}} &{\begin{tabular}{cccc}
    \hline
    \multicolumn{1}{c}{}\\
     {8.95283}&{8.95941}& {8.98787}& {9.04274}\\
\end{tabular}}\\
\hline
{$u_c/R_{sh}$} & {2.59808}  &{\begin{tabular}{cccc}
    \hline
    \multicolumn{1}{c}{}\\
   {2.59373}&{2.58054}& {2.52504}& {2.41987}\\
\end{tabular}}&{\begin{tabular}{cccc}
    \hline
    \multicolumn{1}{c}{}\\
   { 3.5052}& {3.50177}& {3.48813}& {3.4657}\\
\end{tabular}} \\
\hline
 {$\bar{a}$ }& {1 } &{\begin{tabular}{cccc}
    \hline
    \multicolumn{1}{c}{}\\
    {1.0028}& {1.01151}&{1.05179}& {1.15041}\\
\end{tabular}}&{\begin{tabular}{cccc}
    \hline
    \multicolumn{1}{c}{}\\
     {0.761981}& {10.761421}&{0.75901}& {0.754404}\\
\end{tabular}}\\
\hline
{$\bar{b}$ }& {-0.40023} &{\begin{tabular}{cccc}
    \hline
    \multicolumn{1}{c}{}\\
    { -0.401608}& {-0.406055}& {-0.429858}& {-0.511002}\\
\end{tabular}} &{\begin{tabular}{cccc}
    \hline
    \multicolumn{1}{c}{}\\
    {-0.384548}&{-0.381893}& {-0.37101}& {-0.35205}\\
\end{tabular}}\\
\hline
\end{tabular}
\end{table*}





\section{Results and Conclusions}


In this paper, we have discussed the observational signatures of
the modified Bardeen black hole through shadow and strong
gravitational lensing observations. We compared these signatures
with those of other astrophysical black holes, such as the
Schwarzschild black hole and the ordinary regular Bardeen black
hole. We examined how the parameters of the modified Bardeen black
hole affect the shadow and strong lensing observables.

First, we derived the null geodesics for the modified Bardeen
black hole using the Hamiltonian-Jacobi action and reviewed the
photon orbit around this black hole. Numerically estimating the
photon sphere radius and shadow radius, we found that for fixed
values of the parameters $\mu$ and $g$, these radii decrease with
increasing magnitude of the charge parameter $q$, while they
increase with increasing magnitude of $\mu$ for fixed values of
$q$ and $g$. Furthermore, we observed that the shadow radius for
the modified Bardeen black hole is larger than that of the
Schwarzschild black hole and the ordinary regular Bardeen black
hole. We also obtained the angular diameter of the black hole
shadow with the parameters $\mu$ and $q$ for a fixed parameter $g
(=0.2)$, considering the supermassive black holes $M87^*$ and
$SgrA^*$. It can be seen that the angular diameter of the black
hole shadow for the modified Bardeen is larger than the
Schwarzschild, as well as regular Bardeen black holes. The
modified black hole parameters q and $\mu$ for the fixed value of
g  have been constrained by the EHT collaboration data for the
angular shadow diameter of $M87^*$ and $SgrA^*$. It has been
observed that the constrain ranges of the parameters $\mu$ and $q$
of modified Bardeen black hole as $-0.89\leq \mu/8M^2 \leq 0.4$
and $0\leq |q|\leq 0.185$ for $M87^*$; and $-1.38\leq \mu/8M^2
\leq 0.1$ and $0\leq |q|\leq 0.058$  for $SgrA^*$, keeping the
fixed value $g/2M=0.2$. Modified Bardeen  black holes with the
additional parameters $\mu$ ,$g$ and $q$ besides the mass M  of
the black hole as the supermassive black holes $M87^*$ and
$SgrA^*$; and it is observed to be a viable astrophysical black
hole candidate, the EHT result constrains the ($\mu$, $q$)
parameter space.



Next, we investigated strong gravitational lensing by the modified
Bardeen black hole and examined its astrophysical consequences. We
studied the effects of the modified Bardeen black hole parameters
$\mu$, $g$, and $q$ on the strong deflection angle and strong
lensing observables. By revisiting the null geodesic equations and
numerically estimating the photon radius, we obtained the lensing
coefficients $\bar{a}$, $\bar{b}$, and $u_{ph}/R_{sh}$. Our
results showed that for fixed values of $\mu$ and $g$, $\bar{a}$
increases while $\bar{b}$ decreases with increasing magnitude of
the charge parameter $q$. Conversely, for fixed values of $q$ and
$g$, $\bar{a}$ decreases while $\bar{b}$ increases with increasing
value of the parameter $\mu$. We observed that the deflection
angle $\alpha_D$ slightly increases initially, reaches a maximum
value, and then decreases with increasing magnitude of the charge
parameter $q$ for fixed values of $\mu$ and $g$. However, it
always decreases with increasing value of $\mu$ for fixed values
of $q$ and $g$. Comparatively, the deflection angle for the
modified Bardeen black hole is smaller than that for other
astrophysical black holes, such as the Schwarzschild black hole
and the ordinary regular Bardeen black hole.



We also numerically estimated the strong lensing observables for
the relativistic images in the context of the modified Bardeen
black hole, considering the supermassive black holes $M87^*$,
$SgrA^*$, and $NGC7457$. Our results showed that the angular
position $\theta_\infty$ and magnification $r_{\text{mag}}$ of the
relativistic images in the context of the modified Bardeen black
hole are larger than those for the Schwarzschild black hole and
the ordinary regular Bardeen black hole. However, the angular
separation $S$ of the relativistic images in the context of the
modified Bardeen black hole is smaller than that for the
Schwarzschild black hole and the ordinary regular Bardeen black
hole. We provided specific numerical ranges for the observables
$\theta_\infty$ and $S$ for different values of the parameters
$\mu$, $g$, and $q$, considering the supermassive black holes
$M87^*$, $SgrA^*$, and $NGC7457$. In the cases, where $\mu=0$ and
$0\leq |q|\leq 0.4$,$\theta_{\infty}\in (17.06,19.97)\mu as$ for
$M87^{*}$,$\theta_{\infty}\in (22.5,26.3)\mu as$ for $SgrA^{*}$
and $\theta_{\infty}\in (0.031,0.037)\mu as$ for $NGC 7457$; when
$\mu=1$ ,$g=0.2$ and $0< |q|\leq 0.4$,$\theta_{\infty}\in
(21.5,22.8)\mu as$ for $M87^{*}$,$\theta_{\infty}\in (28.3,30)\mu
as$ for $SgrA^{*}$ and $\theta_{\infty}\in (0.03,0.042)\mu as$ for
$NGC 7457$;  when $\mu=1$ ,$g=1.2$ and $0< |q|\leq
0.4$,$\theta_{\infty}\in (19.2,21.44)\mu as$ for
$M87^{*}$,$\theta_{\infty}\in (25.3,28.3)\mu as$ for $SgrA^{*}$
and $\theta_{\infty}\in (0.35,0.4)\mu as$ for $NGC 7457$; when
$\mu=3$ ,$g=0.2$ and $0< |q|\leq 0.4$,$\theta_{\infty}\in
(26.3,26.94)\mu as$ for $M87^{*}$,$\theta_{\infty}\in
(34.8,35.53)mu as$ for $SgrA^{*}$ and $\theta_{\infty}\in
(0.048,0.05)\mu as$ for $NGC 7457$; and  when $\mu=1$ ,$g=0.2$ and
$0< |q|\leq 0.4$,$\theta_{\infty}\in (23.6,24.7)\mu as$ for
$M87^{*}$,$\theta_{\infty}\in (31.19,32.58)\mu as$ for $SgrA^{*}$
and $\theta_{\infty}\in (0.048,0.05)\mu as$ for $NGC 7457$.
Moreover, the angular separation $S\in (0.024,0.14)\mu as$ for
$M87^{*}$, $S\in (0.032,0.184)\mu as$ for $SgrA^{*}$ ,$S\in (4.5
\times 10^{-5},2.54 \times 10^{-4})\mu as$ for  NGC $7457 $ for
the case when $\mu=0$ and $0\leq|q|\leq 0.4$; $S\in
(0.014,0.019)\mu as$ for $M87^{*}$, $S\in (0.019,0.025)\mu as$ for
$SgrA^{*}$ ,$S\in (2.6 \times 10^{-5},3.5 \times 10^{-5})\mu as$
for  NGC $7457 $ for the case when $\mu=1$ ,$g=0.2$ and $0<|q|\leq
0.4$;
 $S\in (0.03,0.19)\mu
as$ for $M87^{*}$, $S\in (0.04,0.16)\mu as$ for $SgrA^{*}$
,$S\in (5.5 \times 10^{-5},2.18 \times 10^{-4})\mu as$ for  NGC $7457 $ for the case when
$\mu=1$,$g=1.2$ and $0<|q|\leq 0.4$;
$S\in (0.003,0.0043)\mu
as$ for $M87^{*}$, $S\in (0.005,0.0057)\mu as$ for $SgrA^{*}$
,$S\in (6.9 \times 10^{-6},7.8 \times 10^{-6})\mu as$ for  NGC $7457 $ for the case when
$\mu=3$,$g=0.2$ and $0<|q|\leq 0.4$;
$S\in (0.022,0.025)\mu
as$ for $M87^{*}$, $S\in (0.02,0.033)\mu as$ for $SgrA^{*}$
,$S\in (4 \times 10^{-5},6.02 \times 10^{-5})\mu as$ for  NGC $7457 $ for the case when
$\mu=3$,$g=1.2$ and $0<|q|\leq 0.4$. Considering the supermassive black holes $M87^{*}$ and $SgrA^{*}$, the outermost Einstein's ring  $\theta^E_n$ has been displayed in Fig.\ref{fig:11} for the cases of modified Bardeen ($\mu=3,g=0.2$)  as well as ordinary regular Bardeen ($\mu=0$)  black holes.



Our analysis revealed that the outermost Einstein rings
$\theta^E_n$ for the modified Bardeen black hole are larger
compared to those of the ordinary regular Bardeen black hole. This
implies that the modified Bardeen black hole exhibits a larger
angular separation between multiple relativistic images, enhancing
the detectability and distinguishing it from the ordinary regular
Bardeen black hole.


Moreover, when considering various supermassive black holes, we
examined the time delay between the first and second-order
relativistic images for the modified Bardeen, ordinary Bardeen,
and Schwarzschild black holes. Notably, we found that the time
delay for the modified Bardeen black hole ($\sim 15249.4 minutes$)
is significantly greater than that for the Schwarzschild black
hole ($\sim 12619.7 minutes$ ) and the ordinary regular Bardeen
black holes ($\sim 11754.1 minutes$ ) in the context of the
supermassive black hole $NGC 4649$. This suggests that the
modified Bardeen black hole exhibits distinct temporal signatures,
providing an avenue for its detection and differentiation from
other astrophysical black holes.\\


\section*{Acknowledgements}

N.U.M would like to thank  CSIR, Govt. of
India for providing Senior Research Fellowship (No. 08/003(0141))/2020-EMR-I).

%%%%%%%%%%%%%%%%%%%%%%%%%%%%%%%%%%%%%%%%%%%%%%%%%%
%\section*{Data Availability}


%%%%%%%%%%%%%%%%%%%% REFERENCES %%%%%%%%%%%%%%%%%%

% The best way to enter references is to use BibTeX:
\bibliographystyle{apsrev4-2}%
\bibliography{Niyaz2.bib} 

%%%%%%%%%%%%%%%%%%%%%%%%%%%%%%%%%%%%%%%%%%%%%%%%%%

\end{document}


