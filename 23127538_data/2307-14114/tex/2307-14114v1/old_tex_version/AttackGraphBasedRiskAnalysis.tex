\documentclass[table]{scrartcl}

\usepackage{amssymb}
\usepackage{amsthm}
\usepackage{authblk}
\usepackage{booktabs}
\usepackage{eurosym}
\usepackage{float}
\restylefloat{table}
\usepackage{listliketab}
\usepackage{listings}
\usepackage{multirow}
\usepackage{tabularx}

\newtheorem{definition}{Definition}
\usepackage{amsmath}
\usepackage{slashbox}
\usepackage{graphicx}
\usepackage[hyphens]{url}
\usepackage[hidelinks]{hyperref}
\usepackage{hyphenat}
\usepackage[backend=biber,sorting=none]{biblatex}
\addbibresource{AttackGraphBasedRiskAnalysis.bib}
\usepackage{pifont}
\newcommand{\cmark}{\ding{51}}
\newcommand{\xmark}{\ding{55}}
\usepackage{csquotes}
\usepackage[capitalise]{cleveref}

\lstdefinelanguage{JavaScript}{
  keywords={typeof, new, true, false, catch, function, return, null, catch, switch, var, if, in, while, do, else, case, break},
  keywordstyle=\color{blue}\bfseries,
  ndkeywords={class, export, boolean, throw, implements, import, this},
  ndkeywordstyle=\color{darkgray}\bfseries,
  identifierstyle=\color{black},
  sensitive=false,
  comment=[l]{//},
  morecomment=[s]{/*}{*/},
  commentstyle=\color{purple}\ttfamily,
  stringstyle=\color{red}\ttfamily,
  morestring=[b]',
  morestring=[b]"
}

\title{New Title}
\author{}

% \author[1]{Simon Unger}
% \author[1]{Ektor Arzoglou}
% \author[1]{Stefan Katzenbeisser}

% \affil[1]{Chair of Computer Engineering, University of Passau, \authorcr
% \{Simon.Unger, Ektor.Arzoglou, Stefan.Katzenbeisser\}@uni-passau.de}

\begin{document}

\maketitle

\section{Introduction}\label{sec: intro}

Graphical security modelling is a widely-used and well-established approach for representing and analysing security scenarios that examine vulnerabilities of systems and organisations. One of the primary strengths of graphical security models is that they allow for the inclusion of user-friendly visual elements with formal semantics and algorithms and therefore enable both qualitative and quantitative analyses. Over the last couple of decades, security researchers have been progressively focusing on graphical security modelling, which has gradually evolved into a valuable tool for the assessment of risks in real-life systems, such as automotive and railway environments.

Security scenarios include (1) malicious actions of an attacker, whose goal is to harm or damage one or more assets of a system or organisation, and (2) countermeasures for either preventing or mitigating such malicious actions. The first \emph{tree-based approach} for graphical security modelling was the \emph{threat logic trees}, which was introduced by Weiss in 1991~\cite{weiss1991}, thereby motivating the development of several subsequent frameworks, such as attack trees, which are still considered one of the most important and favoured tools for the assessment of risks to date.

In all tree-based approaches, the modelling process begins with the identification of a feared event, which is shown as a root node, and it continues with the deduction of the event's consequences, which are shown as refined nodes, hence resulting in a tree model. However, tree structures are limited to only one path between a pair of nodes. In other words, with tree structures, each refined node can only have one parent node. This limitation is addressed by the \emph{directed acyclic graph (DAG) structure}, which enables refined nodes to have multiple parent nodes. As a result, DAG structures can provide a higher level of detail, but they can also come with a higher level of complexity, which can nevertheless be dealt with modularisation, thereby allowing the model to be subdivided into loosely-coupled, independent, and interchangeable parts that can be studied individually and in parallel. Finally, while the one-to-one relationship between nodes in tree structures results in a linear analysis of security scenarios, the one-to-many relationship between nodes in DAG structures can theoretically result in an exponential analysis. However, in practice, exponents are kept small due to the acyclic structure, and the analysis of security scenarios is eventually possible.

\section{Related Work}\label{sec: related work}

Kordy et al. categorise thirty-three frameworks for graphical analysis of attack and defense scenarios into (1) \emph{attack and/or defense modelling}, which focus on the formal aspects of attacks or defenses, and (2) \emph{static or sequential modelling}, which focus on the temporal aspects or dependencies between actions~\cite{DAGpaper}. Using the same categorisation, this section provides an overview of all the frameworks in chronological ascending order, and it describes these frameworks that fulfill the majority of properties incorporated in the framework of this article.

By reviewing frameworks from current literature, this article identifies six properties for graphical analysis of attack and defense scenarios. The first property is \emph{attack vectors}, which enable the relations (shown as edges) between attack steps (shown as nodes) and therefore the formation of attack paths (i.e. attack vectors). The second property is the \emph{directed acyclic graph (DAG) structure}, thereby enabling linear (i.e. directed) and finite (i.e. acyclic) series of attack steps towards multiple potential attack goals (i.e. graph). The third property is \emph{node attributes}, which enable the quantification and therefore the evaluation of attack steps. The fourth property is \emph{dynamic connectors}, thereby enabling extensive attack refinements (besides the basic AND-OR refinements). The fifth property is \emph{edge attributes}, which enable the quantification and therefore the evaluation of relations between attack steps. The final property is \emph{defense nodes}, thereby enabling actions to reduce the negative consequences of attacks.

Each one of the thirty-three frameworks presented in this section considers only subsets of the six identified properties, hence ... To overcome this limitation, this article incorporates all six identified properties into a framework for graphical analysis of attack and defense scenarios and examines ...

\subsection{Static Attack Modelling}\label{sec: static attack modelling}

\begin{table}[h]
\rowcolors{2}{gray!10}{gray!40}
\renewcommand{\arraystretch}{1.2}
\caption{Static attack modelling frameworks compared to the six defined properties.}
\label{tab: static attack modelling}
\noindent\makebox[\linewidth]{%
\begin{tabular}[t]{>{\raggedright}p{0.22\linewidth}>{\raggedright}p{0.13\linewidth}>{\raggedright}p{0.13\linewidth}>{\raggedright}p{0.13\linewidth}>{\raggedright}p{0.13\linewidth}>{\raggedright\arraybackslash}p{0.13\linewidth}>{\raggedright\arraybackslash}p{0.13\linewidth}}
\toprule
 & Attack Vectors & DAG Structure & Node Attributes & Dynamic Connectors & Edge Attributes & Defense Nodes
\tabularnewline
\midrule
% \textbf{Static Attack Modelling} & & & & &
% \tabularnewline
Attack Trees & \checkmark & - & (\checkmark) & - & - & -
\tabularnewline
Augmented Vulnerability Trees & \checkmark & - & (\checkmark) & - & - & -
\tabularnewline
Augmented Attack Trees & \checkmark & - & (\checkmark) & - & - & -
\tabularnewline
OWA Trees & \checkmark & - & - & (\checkmark)  & (\checkmark) & -
\tabularnewline
Parallel Model for Multi-Parameter Attack Trees & \checkmark & - & (\checkmark) & - & - & -
\tabularnewline
Extended Fault Trees & \checkmark & - & (\checkmark) & - & - & -
\tabularnewline
\bottomrule
\end{tabular}}
\end{table}

Six frameworks for \emph{static attack modelling}, namely \emph{Attack Trees}~\cite{weiss1991}, \emph{Augmented Vulnerability Trees}~\cite{AugmentedVulnerabilityTrees}, \emph{Augmented Attack Trees}~\cite{AugmentedAttackTrees}, \emph{OWA Trees}~\cite{Yager2006OWATA}, \emph{Parallel Model for Multi-Parameter Attack Trees}~\cite{ParallelModelForMultiParameterAttackTrees}, and \emph{Extended Fault Trees}~\cite{ExtendedFaultTrees}, are summarised in Table~\ref{tab: static attack modelling}. All frameworks fulfill the attack vectors property, but none of them fulfills the DAG structure and defense nodes properties. Of the six frameworks, OWA trees stand out as they at least partially fulfill the dynamic connectors and edge attributes properties, despite being the only framework that does not fulfill the node attributes property. This section describes attack trees, which was the first graphical security modelling framework, and OWA trees, which is the framework that at least partially fulfills most of the six identified properties.

\subsubsection{Attack Trees}\label{sec: attack trees}

The first \emph{tree-based approach}, shown as an AND-OR tree structure, for graphical security modelling was the \emph{threat logic trees}, which was introduced by Weiss in 1991~\cite{weiss1991}. Today, all AND-OR tree structures are referred to as \emph{attack trees}, which is a term that was first introduced by Salter et al. in 1998~\cite{Salter1998}.

In attack trees, the root node (i.e., the root of the tree) indicates the main goal of the attack. The main goal is then conjunctively (AND) or disjunctively (OR) refined into sub-goals until they represent basic actions, which correspond to atomic components that can be easily understood and quantified. Conjunctive refinements indicate that \emph{all} sub-goals need to be fulfilled in order to achieve the main goal, whereas disjunctive refinements indicate that \emph{at least one} sub-goal needs to be fulfilled for achieving the main goal~\cite{weiss1991}.

\subsubsection{OWA Trees}\label{sec: owa trees}

\emph{Ordered weighted averaging (OWA) trees} were proposed by Yager in 2005 to include the concept of \emph{uncertainty} into attack trees~\cite{Yager2006OWATA}. This was made possible by replacing the AND-OR nodes with OWA nodes (i.e., quantifiers, such as \emph{most}, \emph{some}, \emph{half of}, etc.) and therefore taking into consideration situations where the number of sub-goals that need to be fulfilled in order to achieve the main goal remains unknown. Finally, OWA trees allow for the evaluation of success probability and cost attributes, which can be jointly used to calculate the cheapest and most probable attack.

\subsection{Sequential Attack Modelling}\label{sec: sequential attack modelling}

\begin{table}[h]
\rowcolors{2}{gray!10}{gray!40}
\renewcommand{\arraystretch}{1.2}
\caption{Sequential attack modelling frameworks compared to the six defined properties.}
\label{tab: sequential attack modelling}
\noindent\makebox[\textwidth]{%
\begin{tabular}[t]{>{\raggedright}p{0.22\linewidth}>{\raggedright}p{0.13\linewidth}>{\raggedright}p{0.13\linewidth}>{\raggedright}p{0.13\linewidth}>{\raggedright}p{0.13\linewidth}>{\raggedright\arraybackslash}p{0.13\linewidth}>{\raggedright\arraybackslash}p{0.13\linewidth}}
\toprule
 & Attack Vectors & DAG Structure & Node Attributes & Dynamic Connectors & Edge Attributes & Defense Nodes
\tabularnewline
\midrule
% \textbf{Sequential Attack Modelling} & & & & &
% \tabularnewline
Cryptographic DAGs & \checkmark & \checkmark & - & - & - & -
\tabularnewline
Fault Trees for Security & \checkmark & - & \checkmark & (\checkmark) & - & -
\tabularnewline
Bayesian Networks for Security & \checkmark & \checkmark & \checkmark & - & \checkmark & -
\tabularnewline
Bayesian Attack Graphs & \checkmark & \checkmark & \checkmark & - & \checkmark & -
\tabularnewline
Compromise Graphs & \checkmark & \checkmark & - & - & (\checkmark) & -
\tabularnewline
Enhanced Attack Trees & \checkmark & - & \checkmark & - & (\checkmark) & -
\tabularnewline
Vulnerability Cause Graphs & (\checkmark) & \checkmark & - & - & - & -
\tabularnewline
Dynamic Fault Trees for Security & \checkmark & - & (\checkmark) & - & - & -
\tabularnewline
Serial Model for Multi-Parameter Attack Trees & \checkmark & - & (\checkmark) & - & - & -
\tabularnewline
Improved Attack Trees & \checkmark & - & (\checkmark) & - & - & -
\tabularnewline
Time-dependent Attack Trees & \checkmark & \checkmark & (\checkmark) & - & - & -
\tabularnewline
\bottomrule
\end{tabular}}
\end{table}

Eleven frameworks for \emph{sequential attack modelling}, namely \emph{Cryptographic DAGs}~\cite{Meadows1996ARO}, \emph{Fault Trees for Security}~\cite{FaultTreesForSecurity}, \emph{Bayesian Networks for Security}~\cite{BayesianNetworksForSecurity}, \emph{Bayesian Attack Graphs}~\cite{BayesianAttackGraphs}, \emph{Compromise Graphs}~\cite{CompromiseGraphs}, \emph{Enhanced Attack Trees}~\cite{EnhancedAttackTrees}, \emph{Vulnerability Cause Graphs}~\cite{VulnerabilityCauseGraphs}, \emph{Dynamic Fault Trees for Security}~\cite{DynamicFaultTreesForSecurity}, \emph{Serial Model for Multi-Parameter Attack Trees}~\cite{SerilModelForMultiParameterAttackTrees}, \emph{Improved Attack Trees}~\cite{ImprovedAttackTrees}, and \emph{Time-dependent Attack Trees}~\cite{TimeDependentAttackTrees}, are summarised in Table~\ref{tab: sequential attack modelling}. Again, none of the frameworks fulfills the defense nodes property. In addition, only Fault Trees for Security offer a wide range of dynamic connectors, and only Bayesian-based models fulfill the edge attributes property. Finally, Compromise Trees and Enhanced Attack Trees are two frameworks that at least partially fulfill the edge attributes property, and Vulnerability Cause Graphs are the only framework that does not fulfill the attack vectors property. This section describes Cryptographic DAGs, which was the first graph-based approach for security modelling, and Bayesian Attack Graphs, which combine attack trees and Bayesian networks and also fulfill four of the six identified properties.

\subsubsection{Cryptographic DAGs}\label{sec: cryptographic dags}

\emph{Cryptographic directed acyclic graphs (DAGs)} were proposed by Meadows in 1996 to provide a \emph{novel} simple representation of sequences and dependencies of attack steps towards the main goal of the attack. Instead of a tree-based approach, Cryptographic DAGs introduced a \emph{graph-based approach} for security modelling, however they eventually do not offer the possibility to perform risk assessment.

\subsubsection{Bayesian Networks and Bayesian Attack Graphs}\label{sec: bayesian Networks and Attack Graphs}

For the last couple of decades, researchers have been focusing on \emph{Bayesian networks} for the purposes of security modelling. The origin of Bayesian networks, which are also known as \emph{belief} or \emph{causal networks}, lies in artificial intelligence. In Bayesian networks, nodes represent events or objects and are associated with probabilistic variables. Hence, analysing the uncertainty of events is also possible. Bayesian networks follow a DAG structure, where the directed edges represent the causal dependencies between the nodes~\cite{BayesianAttackGraphs}.

\emph{Bayesian attack graphs} are a fusion of (general) attack trees and (computational procedures) of Bayesian networks, and they were first introduced by Liu and Man in 2005 to analyse network vulnerability scenarios~\cite{BayesianAttackGraphs}. Subsequently, calculating general security metrics regarding information system networks~\cite{Frigault2008, Noel2010} and capturing dynamic behavior~\cite{Frigault2008Dyn} were also made possible.

Finally, although Bayesian attack graphs allow for assigning values to nodes and for performing computations using the graphs, they do not allow for assigning values to connectors and for including countermeasures. As a result, Bayesian attack graphs cannot be used to perform risk assessment.

\subsection{Static Attack and Defense Modelling}\label{sec: static attack and defense modelling}

\begin{table}[h]
\rowcolors{2}{gray!10}{gray!40}
\renewcommand{\arraystretch}{1.2}
\caption{Static attack and defense modelling frameworks compared to the six defined properties.}
\label{tab: static attack and defense modelling}
\noindent\makebox[\linewidth]{%
\begin{tabular}[t]{>{\raggedright}p{0.22\linewidth}>{\raggedright}p{0.13\linewidth}>{\raggedright}p{0.13\linewidth}>{\raggedright}p{0.13\linewidth}>{\raggedright}p{0.13\linewidth}>{\raggedright\arraybackslash}p{0.13\linewidth}>{\raggedright\arraybackslash}p{0.13\linewidth}}
\toprule
 & Attack Vectors & DAG Structure & Node Attributes & Dynamic Connectors & Edge Attributes & Defense Nodes
\tabularnewline
\midrule
% \textbf{Static Attack and Defense Modelling}
% \tabularnewline
Anti-Models & \checkmark & - & - & - & - & \checkmark
\tabularnewline
Defense Trees & \checkmark & - & (\checkmark) & - & - & \checkmark
\tabularnewline
Protection Trees & - & - & \checkmark & - & - & \checkmark
\tabularnewline
Security Activity Graphs & \checkmark & \checkmark & (\checkmark) & - & - & \checkmark
\tabularnewline
Attack Countermeasure Trees & \checkmark & - & \checkmark & - & - & \checkmark
\tabularnewline
Attack-Defense Trees & \checkmark & - & \checkmark & - & - & \checkmark
\tabularnewline
Countermeasure Graphs & \checkmark & \checkmark & \checkmark & - & - & \checkmark
\tabularnewline
\bottomrule
\end{tabular}}
\end{table}

Seven frameworks for \emph{static attack and defense modelling}, namely \emph{Anti-Models}~\cite{AntiModels}, \emph{Defense Trees}~\cite{DefenseTrees}, \emph{Protection Trees}~\cite{ProtectionTrees}, \emph{Security Activity Graphs}~\cite{SecurityActivityGraphs}, \emph{Attack Countermeasure Trees}~\cite{AttackCountermeasureTrees}, \emph{Attack-Defense Trees}~\cite{AttackDefenseTrees}, and \emph{Countermeasure Graphs}~\cite{CountermeasureGraphs}, are summarised in Table~\ref{tab: static attack and defense modelling}. All frameworks fulfill the defense nodes property, and only Protection Trees do not fulfill the attack vectors property. In addition, Anti-Models is the only framework that does not at least partially fulfill the node attributes property. This section describes Security Activity Graphs and Countermeasure Graphs, which are the two frameworks that fulfill four of the six identified properties.

\subsubsection{Security Activity Graphs}\label{sec: security activity graphs}

\emph{Security activity graphs (SAGs)} were developed by Ardi et al. in 2006 to improve security throughout the software development process~\cite{SecurityActivityGraphs}. SAGs are loosely based on fault trees, and the root of a SAG is associated with a vulnerability. Vulnerability mitigations are modelled using activities (i.e., leaf nodes), which are assigned boolean variables to indicate whether an activity \enquote{is implemented perfectly during software development} (true) or not (false). Finally, besides AND-OR gates, which follow a strictly Boolean logic, SAGs also include \emph{split gates}, which allow one activity to be used in several parent activities, thus creating a DAG structure.

However, the lack of edge attributes together with the limited options of connectors and node attributes render SAGs impractical for risk assessment.

\subsubsection{Countermeasure Graphs}\label{sec: countermeasure graphs}

\emph{Countermeasure graphs} were introduced by Baca and Petersen in 2010 to simplify countermeasure selection through cumulative voting~\cite{CountermeasureGraphs}. Countermeasure graphs are created by identifying actors, goals, attacks, and countermeasures. Related events are connected with edges. That is, actors are connected to pursued goals and likely executable attacks, and countermeasures are connected to preventable attacks. Finally, actors, goals, attacks, and countermeasures are assigned priorities according to the rules of hierarchical cumulative voting. Higher assigned priorities imply higher threat levels of the corresponding events, and vice versa. Using hierarchical cumulative voting, the most effective countermeasures can be identified.

Countermeasure Graphs provide a useful system overview, but the computational rules focus on finding the most effective countermeasure instead of the most likely and severe attack. This limitation could at least partially be addressed with the threat level, however the threat level value is determined by the subjective assessment of the graph creator rather than by calculations over meaningful attributes, thereby raising issues of validity.

\subsection{Sequential Attack and Defense Modelling}\label{sec: sequential attack and defense modelling}

\begin{table}[h]
\rowcolors{2}{gray!10}{gray!40}
\renewcommand{\arraystretch}{1.2}
\caption{Sequential attack and defense modelling frameworks compared to the six defined properties.}
\label{tab: sequential attack and defense modelling}
\noindent\makebox[\linewidth]{%
\begin{tabular}[t]{>{\raggedright}p{0.22\linewidth}>{\raggedright}p{0.13\linewidth}>{\raggedright}p{0.13\linewidth}>{\raggedright}p{0.13\linewidth}>{\raggedright}p{0.13\linewidth}>{\raggedright\arraybackslash}p{0.13\linewidth}>{\raggedright\arraybackslash}p{0.13\linewidth}}
\toprule
 & Attack Vectors & DAG Structure & Node Attributes & Dynamic Connectors & Edge Attributes & Defense Nodes
\tabularnewline
\midrule
% \textbf{Sequential Attack and Defense Modelling} & & & & &
% \tabularnewline
Insecurity Flows & \checkmark & \checkmark & \checkmark & - & - & \checkmark
\tabularnewline
Intrusion DAGs & \checkmark & \checkmark & - & - & - & \checkmark
\tabularnewline
Bayesian Defense Graphs & \checkmark & \checkmark & (\checkmark) & - & - & \checkmark
\tabularnewline
Security Goal Indicator Trees & - & - & - & - & - & \checkmark
\tabularnewline
Attack Response Trees & \checkmark & -  & \checkmark & - & - & \checkmark
\tabularnewline
Boolean Logic Driven Markov Processes & \checkmark & \checkmark & (\checkmark) & (\checkmark) & - & \checkmark
\tabularnewline
Cyber Security Modelling Language & \checkmark & \checkmark & (\checkmark) & - & (\checkmark) & \checkmark
\tabularnewline
Security Goal Models & \checkmark & \checkmark & - & - & - & \checkmark
\tabularnewline
Unified Parameterizable Attack Trees & \checkmark & - & \checkmark & - & (\checkmark) & \checkmark
\tabularnewline
\bottomrule
\end{tabular}}
\end{table}

Finally, nine frameworks for \emph{sequential attack and defense modelling}, namely \emph{Insecurity Flows}~\cite{InsecurityFlows}, \emph{Intrusion DAGs}~\cite{IntrusionDAGs}, \emph{Bayesian Defense Graphs}~\cite{BayesianDefenseGraphs}, \emph{Security Goal Indicator Trees}~\cite{SecurityGoalIndicatorTrees}, \emph{Attack Response Trees}~\cite{AttackResponseTrees}, \emph{Boolean Logic Driven Markov Process}~\cite{BooleanLogicDrivenMarkovProcess}, \emph{Cyber Security Modelling Language}~\cite{CyberSecurityModelingLanguage2010}, \emph{Security Goal Models}~\cite{SecurityGoalModels}, and \emph{Unified Parameterizable Attack Trees}~\cite{UnifiedParameterizableAttackTrees}, are summarised in Table~\ref{tab: sequential attack and defense modelling}. All frameworks fulfill the defense nodes property, and only Security Goal Indicator Trees do not fulfill the attack vectors property. In addition, Boolean Logic Driven Markov Processes (BDMPs) is the only framework that offers a a wide range of connectors and therefore at least partially fulfills the dynamic connectors property. This section describes BDMPs and Cyber Security Modelling Language (CySeMoL), which are the two frameworks that at least partially fulfill five of the six identified properties.

\subsubsection{Boolean Logic Driven Markov Processes}\label{sec: boolean logic driven markov processes}

\emph{Boolean logic driven Markov processes (BDMPs)} are a security modelling framework, which can also be used to used to perform risk assessment~\cite{BooleanLogicDrivenMarkovProcess}. It was invented by Bouissou and Bon in 2003 for the safety and reliability domains, and it was later adapted to security modelling by Piètre-Cambacédès and Bouissou in 2010. BDMPs combine the readability of attack trees with the modelling power of Markov chains. The root (top event) of a BDMP represents the main goal of the attack, and the leaves represent the attack steps or security events. BDMPs offer a wide range of node attributes, including time-domain metrics, such as mean-time to success, attack tree related metrics, such as costs of attacks, Boolean indicators, such as specific requirements, and risk assessment tools, such as sensibility graphs.

However, the lack of edge attributes in addition to issues of usability with respect to leaf nodes and connectors~\cite{BDMPCritic} render BDMPs impractical for risk assessment.

\subsubsection{Cyber Security Modelling Language}\label{sec: cyber security modelling language}

\emph{Cyber security modelling language (CySeMoL)} was developed by Sommestad et al. in 2010 to assess the cyber security of \emph{supervisory control and data acquisition (SCADA)} system architectures~\cite{CyberSecurityModelingLanguage2010, CyberSecurityModelingLanguage2013}. Simply modelling the system architecture and the characteristics of the involved assets is sufficient, as CySeMoL already includes information about how attacks and defenses are quantitatively related. The attacker is assumed to be a professional penetration tester with a fixed time of one week to perform an attack. CySeMoL was extended by Holm in 2014 and renamed to \emph{predictive, probabilistic cyber security modelling language ((P$^2$)CySeMoL)}, introducing more flexible and useful computations, the possibility to model assets, attacks, and defenses that are not necessarily SCADA-related, and the option to specify the time needed to perform an attack~\cite{PredictiveProbabilisticCyberSecurityModelingLanguage}. Computations can be conducted automatically (i.e., without personalised inputs) as (P$^2$)CySeMoL already includes qualitative information gathered from literature reviews, empirical studies, as well as surveys involving domain experts~\cite{CyberSecurityModelingLanguage2010, CyberSecurityModelingLanguage2013, PredictiveProbabilisticCyberSecurityModelingLanguage}.

The results of the computations show the likelihood of an attack. However, the severity of an attack is not considered, and therefore the risk of an attack cannot be properly assessed. Furthermore, (P$^2$)CySeMoL does not include connectors, and therefore it seems an inconvenient tool for graphical risk assessment.

\subsection{Summary of Remarks}\label{sec2: summary of remarks}

This section provides an overview of thirty-three frameworks for analysis of attack and defense scenarios, and it describes eight of these frameworks in more detail. Thirty frameworks fulfill the attack vectors property, sixteen frameworks fulfill the defense nodes property, and only thirteen frameworks fulfill the DAG structure property. In addition, node/edge attributes and connectors are in most cases fixed and limited, thereby reducing the usability and usefulness of the frameworks with respect to the purposes of risk assessment. The complex nature and rapid development of (information) systems, attacks, and defenses motivates ...

\section{Attack Graphs}\label{sec: attack graphs}

This article follows a \emph{DAG structure approach} for the development of the security modelling framework. DAGs are graphs of nodes, which are connected with directed edges that do not form any loops. The main DAG components are illustrated in Weiss's attack tree shown in Figure~\ref{fig: attack tree weiss}.

% Figure environment removed

\subsection{Attack Graph Components}\label{sec: attack graph components}

This section summarises the main Attack Graph components.\\

\noindent
\textbf{Nodes}\\
\emph{Nodes} indicate the attacker's goals. There are three different types of nodes: (1) Root Nodes, a set of which is defined as $\mathcal{R}$, (2) Leaf Nodes, a set of which is defined as $\mathcal{L}$, and (3) Inner Nodes, a set of which is defined as $\mathcal{I}$. Finally, all nodes $\emph{n} \in \mathcal{N}$ are formally defined as:

\begin{center}
$\mathcal{N} = \mathcal{R} \cup \mathcal{L} \cup \mathcal{I}$
\end{center}

\noindent
\textbf{Edges}\\
\emph{Edges} indicate the relation between nodes. Attack Graphs use directed edges, thereby enabling nodes to have \emph{predecessors} (also called \emph{parents} or \emph{ancestors}) and/or \emph{successors} (also called \emph{children}). A set of predecessors of a node $n \in \mathcal{N}$ is defined as $Pred_n$, and a set of successors of a node $n \in \mathcal{N}$ is defined as $Succ_n$. For example, for node $n$ := \enquote{Access System Console} in Figure~\ref{fig: attack tree weiss}, $Pred_n$ = \{\enquote{Obtain Admin. Privileges}\} and $Succ_n$ = \{\enquote{Enter Computer Center}, \enquote{Corrupt Operator}\}.\\

\noindent
\textbf{Root Nodes}\\
A node that has no predecessor is a Root Node $\emph{r} \in \mathcal{R}$. \emph{Root Nodes} indicate the main goal of the attack. For example, the Root Node in Figure~\ref{fig: attack tree weiss} is \enquote{Obtain Admin. Privileges}.\\

\noindent
\textbf{Leaf Nodes}\\
A node that has no successor is a Leaf Node $\emph{l} \in \mathcal{L}$. \emph{Leaf Nodes} indicate basic actions that can easily be understood and quantified. For example, three Leaf Nodes in Figure~\ref{fig: attack tree weiss} are \enquote{Break In to Comp. Center}, \enquote{Obtain Password File}, and \enquote{Corrupt Sys. Admin}.\\

\noindent
\textbf{Inner Nodes}\\
A node that has at least one predecessor and at least one successor is an Inner Node $\emph{i} \in \mathcal{I}$. On one hand, \emph{Inner Nodes} are specialised actions, goals, or sub-goals of their predecessor nodes. On the other hand, they are generalised actions or goals of their successor nodes. For example, two Inner Nodes in Figure~\ref{fig: attack tree weiss} are \enquote{Access System Console} and \enquote{Enter Computer Center}.\\

\noindent
\textbf{Node Attributes}\\
Nodes are refined into sub-goals until they represent basic actions. These basic actions can then be quantified by assigning \emph{attributes}, which are are quantifiable properties, to nodes. A set of node attributes is defined as $\mathcal{A}$. For every $\emph{a}_k \in \mathcal{A}$, there exists a value $\emph{v}_m \in \mathcal{V}_k $, where $\mathcal{V}_k$ is the finite set of acceptable values of the attribute $\emph{a}_k \in \mathcal{A}$. The process of assigning values to an attribute is defined by function $g$:

\begin{center}
$g(\emph{a}_k) = \emph{v}_m$
\end{center}

where $\emph{a}_k \in \mathcal{A}$ is an attribute of a node $\emph{n} \in \mathcal{N} $, and $\emph{v}_m \in \mathcal{V}_k$ is a value of this attribute. The set $\mathcal{F}_g$ contains all functions $g$ that assign values to attributes.\\

% A set of attributes of a specific node $\emph{n}_i \in \mathcal{N}$ is defined as $\mathcal{A}_i \subseteq \mathcal{A}$. Hence, every node can be expressed as a tuple of itself, and tuples of its attributes and their values $\emph{n}_i := (\emph{n}_i, (\emph{a}_1, \emph{v}_1), (\emph{a}_2, \emph{v}_2), ..., (\emph{a}_j, \emph{v}_j))$ with $\emph{n}_i \in \mathcal{N}$, $\emph{a}_1, ..., \emph{a}_j \in \mathcal{A}_i$ and $\emph{v}_1 \in \mathcal{V}_1, \emph{v}_2 \in \mathcal{V}_2, ..., \emph{v}_j \in \mathcal{V}_j$.\\ % (\emph{n}_i, (a_i, v_i)\in\mathcal{A}\times\mathcal{V})=

\noindent
\textbf{Edge Attributes}\\
The relation between nodes can be quantified by assigning attributes to edges. A set of edge attributes is defined as $\mathcal{A}_\mathcal{E}$. Edge attributes cannot be assigned to nodes, and similarly node attributes cannot be assigned to edges. Therefore, $\mathcal{A}_\mathcal{E} \neq \mathcal{A}$.\\

\noindent
\textbf{Connectors}\\
\emph{Connectors} $\emph{c} \in \mathcal{C}$ indicate the refinements of nodes. Conjunctive refinements (AND) indicate that \emph{all} successor nodes need to be fulfilled in order to achieve the goal of the refined node, whereas disjunctive refinements (OR) indicate that \emph{at least one} successor node needs to be fulfilled for achieving the goal of the refined node.\\

% Other refinements can be e.g., XOR ($\oplus$) or k-out-of-n ($\binom{n}{k}$).

\noindent
\textbf{Aggregated Attributes}\\
A node $n \in \mathcal{N}$ and its successor nodes $Succ_n$ = \{$n_i, n_{i+1}, ... \in \mathcal{N}$\} have the same set of attributes. Hence, the attribute values of node $n \in \mathcal{N}$ are determined by aggregating the attribute values of its successor nodes $Succ_n$ = \{$n_i, n_{i+1}, ... \in \mathcal{N}$\}. For example, for an attribute $a_k \in \mathcal{A}$, there exists a set of values $\mathcal{V}_k := \{v_m, v_{m+1}, ..., v_{m+n}\}$. Then, the attributes of the successor nodes $Succ_n$ = \{$n_i, n_{i+1}, ... \in \mathcal{N}$\} are assigned values from the set $\mathcal{V}_k$. Finally, these values are aggregated to determine value $v \in \mathcal{V}_k$ of attribute $a_k$ of node $n$. To compute value $v \in \mathcal{V}_k$, a function that takes values from the set $\mathcal{V}_k$ as input and returns value $v \in \mathcal{V}_k$ as output needs to be defined.

An attribute value can be typically computed using a function that returns (1) the \emph{maximum}, (2) the \emph{minimum}, (3) the \emph{sum}, or (4) the \emph{product} of a set of given attribute values. In this case, the set of attribute values $v_m \in \mathcal{V}_k$ is arranged in a natural sort order, where $x_{min}$ is the smallest and $x_{max}$ is the largest value of the set. The set of functions $\mathcal{F}_f$ for aggregating attribute values is defined as follows.\\

\noindent
\textbf{Maximum.} The maximum function takes only the highest value.
\begin{center}
$f_{max}(g(a_1), g(a_{2}), ..., g(a_{k})) = \max\{ g(a_1), g(a_{2}), ..., g(a_{k})\}$
\end{center}

\noindent
\textbf{Minimum.} The minimum function takes only the lowest value.
\begin{center}
$f_{min}(g(a_1), g(a_{2}), ..., g(a_{k})) = \min\{ g(a_1), g(a_{2}), ..., g(a_{k})\}$
\end{center}

\noindent
\textbf{Sum.} The sum function adds the values.
\begin{center}
$f_{sum}(g(a_1), g(a_{2}), ..., g(a_{k})) = \min\{\sum\limits_{j = 1}^{k} g(a_j), x_{max}\}$
\end{center}

\noindent
\textbf{Product.} The product function multiplies the values.
\begin{center}
$f_{prod}(g(a_1), g(a_{2}), ..., g(a_{k})) = \min\{\prod\limits_{j = 1}^{k} g(a_j), x_{max}\}$
\end{center}

\subsection{Attack Graph Definition}\label{sec: attack graph definition}

This article defines an Attack Graph as follows.\\

\noindent
\emph{An Attack Graph $\mathcal{G}$ is a directed acyclic graph (DAG), containing logical connectors and depicting an attack scenario. The edges are weighted and the nodes include attributes that represent the difficulty to perform this attack step. Attributes have a predefined set of values. Finally, functions $g$ assign the values to the attributes, and functions $f$ aggregate values of successor nodes.}\\

\begin{center}
$\mathcal{G} = \{\mathcal{N},\mathcal{E}, \mathcal{C},\mathcal{A},\mathcal{V},\mathcal{F}_f,\mathcal{F}_g\}$
\end{center}

\section{Attack Graph Risk Assessment}\label{sec: attack graph risk assessment}

The Attack Graph adaptation of Weiss's attack tree is shown in Figure~\ref{fig: attack tree weiss modified}. Here, the assessment of risks is made possible by integrating the main Attack Graph components summarised in Section~\ref{sec: attack graphs} with (1) \emph{Consequence Nodes}, which indicate the negative consequences of the main goal of the attack, and (2) \emph{Attack Feasibility Attributes}, which indicate the ease of launching an attack.

% Figure environment removed

% The purpose of Attack Graphs is ... \\

\noindent
\textbf{Consequence Nodes}\\
In Attack Graphs, \emph{Consequence Nodes} (i.e., \enquote{Data Leakage} and \enquote{Denial of Rightful Access to the System}), shown as rounded corner rectangles, become the highest nodes (i.e. the Root Nodes) in the tree structure, and they indicate the negative consequences of the main goal of the attack, which is now represented by the topmost Inner Node (i.e., \enquote{Obtain Admin. Privileges}). For the rest of the article, Root Nodes are referred to as Consequence Nodes.\\

\noindent
\textbf{Attack Feasibility Attributes}\\
\emph{Attack Feasibility Attributes}, shown as coloured squares on top right of Leaf and Inner Nodes, indicate the ease of launching an attack, and they are computed using either (1) a \emph{function} of a set of given attribute values or (2) a \emph{matrix} of a set of given attribute values.

In the first case, a value $\emph{v}_m \in \mathcal{V}_k$ of an Attack Feasibility Attribute $\emph{a}_k \in \mathcal{A}$ of a node $n$ can be computed using a function that aggregates the attributes $\emph{a}_{1}, \emph{a}_{2}, ..., a_{k-1} \in \mathcal{A} \cup \mathcal{A}_\mathcal{E}$ of node $n$, with $a_k \neq a_{k+1}, a_{k+2}, ..., a_{k-1}$:

\begin{center}
$\emph{v}_m = f(\emph{a}_{1}, \emph{a}_{2}, ..., a_{k-1})$
\end{center}

Similarly, in the second case, a set of attributes $d := |\{a_1, a_{2}, ..., a_{k-1}\}|$ are aggregated over a $d$-dimensional matrix. An example of a 2-dimensional matrix is shown in Table~\ref{tab: 2d matrix}. Here, for a node $n$ with attributes $a_1$ and $a_2$, there exist a set of values $\mathcal{V}_1 := \{v_0, v_1, ..., v_{m_1}\}$ for $a_1$ and a set of values $\mathcal{V}_2 := \{u_0, u_1, ..., u_{m_2}\}$ for $a_2$. As such, the aggregation of $\mathcal{V}_1$ and $\mathcal{V}_2$ returns a set of values $\mathcal{V}_3 := \{v_{00}, v_{01}, ..., v_{{m_1}{m_2}}\}$ for an Attack Feasibility Attribute $a_3$.

\begin{table}[h]
\renewcommand{\arraystretch}{1.2}
\caption{2D matrix for an Attack Feasibility Attribute $a_3$, with $a_1, a_2, a_3$ being attributes of a node $n$.}
\label{tab: 2d matrix}
\noindent\makebox[\linewidth]{%
\begin{tabular}[t]{>{\raggedright}p{0.05\linewidth}>{\raggedright}p{0.05\linewidth}>{\raggedright}p{0.05\linewidth}>{\raggedright}p{0.05\linewidth}>{\raggedright\arraybackslash}p{0.05\linewidth}}
\toprule
 & \multicolumn{4}{c}{$a_2$}
\tabularnewline
\cmidrule{2-5}
$a_1$ & $u_0$ & $u_1$ & ... & $u_{m_2}$
\tabularnewline
\midrule
$v_0$ & $v_{00}$ & $v_{01}$ & ... & $v_{0{m_2}}$
\tabularnewline
$v_1$ & $v_{10}$ & $v_{11}$  & ... & $v_{1{m_2}}$
\tabularnewline
... & ... & ... & ... & ...
\tabularnewline
$v_{m_1}$ & $v_{{m_1}0}$ & $v_{{m_1}1}$ & ... & $v_{{m_1}{m_2}}$
\tabularnewline
\bottomrule
\end{tabular}}
\end{table}

% The values need to be in the acceptable range of values for attribute $a_3$, hence $v_{00}, v_{01}, v_{10}, ..., v_{{m_1}{m_2}} \in \mathcal{V}_3$. Acceptable values need to be chosen for every attribute and do not have to be numerical values. So, if $g(a_1) = v_{m_1}$ and $g(a_2) = u_{m_2}$, then $g(a_3) = v_{{m_1}{m_2}}$.

\noindent
\textbf{Countermeasure Nodes} \\
To perform continuous risk assessment it is necessary to include planned countermeasures against attacks as well as already applied ones. 
There are two ways how those countermeasures can influence the risk. 
First, there are countermeasures that help against attack steps. These are attached as new leaf nodes, as shown in~\cref{fig: attack tree weiss modified} with the (blue) nodes \enquote{Physical Access Restriction}, \enquote{Firewall} and \enquote{Vulnerability / Malware Scans}. 
Using the aggregation functions the node attributes of the parent nodes are influenced. 
In the example of~\cref{fig: attack tree weiss modified} the values of the countermeasure nodes are added to their parent nodes which complicates the attack and therefore lowers the attack feasibility. 
Second, the countermeasure can influence the impact an attack has on a specific consequence. 
This case is a bit more complicated to illustrate as it is influencing an edge instead of a node. 
However, the principle is the same as for nodes.

\subsection{Attack Graph Risk Estimation}\label{sec: attack graph risk estimation}

% $Risk = Likelihood \times Impact$

According to DIN VDE V 0831-104, the \emph{Attack Feasibility (AF)} of the Leaf and Inner Nodes shown in Figure~\ref{fig: attack tree weiss modified} is computed based on the \emph{attacker's capabilities} and \emph{mitigation factors}. First, the attacker's capabilities are described by two attributes: (1) \emph{Resources (R)}, reflecting the financial and workforce capacity of the attacker to prepare and launch an attack, and (2) \emph{Knowledge (K)}, reflecting the information that the attacker holds about the system they intend to attack. The Resources and Knowledge of the attacker are each rated as \emph{low} ($R,K = 2$), \emph{medium} ($R,K = 3$), or \emph{high} ($R,K = 4$). As such, attackers with \emph{basic} capabilities ($R,K = 1$) are not considered. Second, the mitigation factors relate to the risk of the attacker being discovered, and they are described by the \emph{Location (L)} attribute, which reflects whether an attack can be launched remotely ($L = 0$) or locally ($L = 1$). The values of the three attributes are summarised in Table~\ref{tab: resources, knowledge, and location matrix}.

\begin{table}[h]
\renewcommand{\arraystretch}{1.2}
\caption{Resources, Knowledge, and Location matrix.}
\label{tab: resources, knowledge, and location matrix}
\noindent\makebox[\linewidth]{%
\begin{tabular}[t]{>{\raggedright}p{0.15\linewidth}>{\raggedright}p{0.1\linewidth}>{\raggedright}p{0.1\linewidth}>{\raggedright}p{0.1\linewidth}>{\raggedright}p{0.1\linewidth}>{\raggedright\arraybackslash}p{0.05\linewidth}}
\toprule
 & 0 & 1 & 2 & 3 & 4
\tabularnewline
\midrule
Resources & - & - & Low & Medium & High
\tabularnewline
Knowledge & - & - & Low & Medium & High
\tabularnewline
Location & Remote & Local & - & - & -
\tabularnewline
\bottomrule
\end{tabular}}
\end{table}

In this regard, every Leaf Node (green node) is assigned with Resources (hammer) and Knowledge (light bulb) attributes, whose value range from 2 to 4, and a Location (pin) attribute, whose value range from 0 to 1. First, the Resources and Knowledge of the attacker are related in a matrix, such as the one shown in Table~\ref{tab: preliminary attack feasibility matrix}, to initially determine a \emph{Preliminary Attack Feasibility (PAF)}, which indicates the ease of launching an attack without taking into consideration the risk of being discovered. For example, Figure~\ref{fig: attack tree weiss modified} shows that the \enquote{Break In to Comp. Center} attack (i.e., bottom left Leaf Node) requires \emph{Low Resources} ($R = 2$) and \emph{Low Knowledge} ($K = 2$) to be launched. In this case, Table~\ref{tab: preliminary attack feasibility matrix} shows that for $R = 3$ and $K = 3$, $PAF = 4$. Second, the Location of the attacker is subtracted from the Preliminary Attack Feasibility, to eventually determine the Attack Feasibility. Here, Figure~\ref{fig: attack tree weiss modified} shows that the \enquote{Break In to Comp. Center} attack requires \emph{Local Access} ($L = 1$) to be launched. In this case, the Attack Feasibility is equal to 3 ($AF = PAF - L = 4 - 1 = 3$).

\begin{table}[h]
\renewcommand{\arraystretch}{1.2}
\caption{Preliminary Attack Feasibility matrix.}
\label{tab: preliminary attack feasibility matrix}
\noindent\makebox[\linewidth]{%
\begin{tabular}[t]{>{\raggedright}p{0.05\linewidth}>{\raggedright}p{0.05\linewidth}>{\raggedright}p{0.05\linewidth}>{\raggedright}p{0.05\linewidth}>{\raggedright}p{0.05\linewidth}>{\raggedright\arraybackslash}p{0.025\linewidth}}
\toprule
 & \multicolumn{5}{c}{K}
\tabularnewline
\cmidrule{2-6}
R & 1 & 2 & 3 & 4 & 5
\tabularnewline
\midrule
1 & 5 & 5 & 4 & 3 & 2
\tabularnewline
2 & 5 & 4 & 4 & 3 & 2
\tabularnewline
3 & 4 & 4 & 4 & 3 & 2
\tabularnewline
4 & 3 & 3 & 3 & 3 & 2
\tabularnewline
5 & 2 & 2 & 2 & 2 & 2
\tabularnewline
\bottomrule
\end{tabular}}
\end{table}

When the Attack Feasibility of all Leaf Nodes is determined, the Resources, Knowledge, and Location attributes of Inner Nodes are obtained from their successor nodes either conjunctively (AND) or disjunctively (OR). Similar to Weiss's attack tree~\cite{weiss1991}, only conjunctive refinements are shown explicitly. That is, all refinements shown in the Attack Graph are disjunctive unless a predecessor node is connected to its successor nodes through an AND gate, in which case the refinement is conjunctive. Regarding conjunctive refinements, an Inner Node obtains the sum of values for each Resources, Knowledge, and Location attribute exhibited among its successor nodes, and the Attack Feasibility is determined based on the resulting values. For example, ... Regarding disjunctive refinements, an Inner Node obtains the Resources, Knowledge, and Location attributes of its successor node with the highest Attack Feasibility. If the highest Attack Feasibility is exhibited by multiple successor nodes, ... For example... The AND-OR functions are listed in the~\hyperref[appendix]{Appendix}.

In addition, the edges relating Consequence Nodes to the topmost Inner Node are assigned with an \emph{Impact} attribute, which indicates the magnitude of damage or physical harm caused by negative consequences on the system. The impact of negative consequences is typically rated using numerical ranges and qualitative scales, as shown in Table~\ref{tab: impact matrix}, however the impact rating terminology remains rather inconsistent across risk assessment standards. Here, if the \enquote{Obtain Admin. Privileges} attack is eventually achieved, the impact of \enquote{Data Leakage} is 3 (\emph{moderate)}, and the the impact of \enquote{Denial of Rightful Access to the System} is 2 (\emph{minor)}.

% The Attack Graph shows that the highest risk results from \enquote{Trojan Horse SA Account}.

\begin{table}[h]
\renewcommand{\arraystretch}{1.2}
\caption{Impact matrix.}
\label{tab: impact matrix}
\noindent\makebox[\linewidth]{%
\begin{tabular}[t]{>{\raggedright}p{0.05\linewidth}>{\raggedright}p{0.1\linewidth}>{\raggedright\arraybackslash}p{0.7\linewidth}}
\toprule
Value & Impact & Description
\tabularnewline
\midrule
1 & Negligible & Impact can be readily absorbed, without requiring management effort.
\tabularnewline
2 & Minor & Impact can be readily absorbed, requiring some management effort.
\tabularnewline
3 & Moderate & Impact cannot be readily absorbed, requiring modest level of resources and management effort.
\tabularnewline
4 & Major & Impact requires high level of resources and management effort to rectify.
\tabularnewline
5 & Severe & Disaster with potential to lead to business collapse, requiring total management effort to rectify.
\tabularnewline
\bottomrule
\end{tabular}}
\end{table}

Finally, the Impact and Attack Feasibility attributes are related in a matrix, such as the one shown in Table~\ref{tab: risk matrix}, to determine a Risk attribute, shown as coloured squares on top right of Consequence Nodes, indicating ... Here, the Attack Graph shows a \emph{significant (S)} risk of \enquote{Data Leakage} and a \emph{moderate (M)} risk of \enquote{Denial of Rightful Access to the System}.

\begin{table}[h]
\renewcommand{\arraystretch}{1.2}
\caption{Risk matrix.}
\label{tab: risk matrix}
\noindent\makebox[\linewidth]{%
\begin{tabular}[t]{>{\raggedright}p{0.1\linewidth}>{\raggedright}p{0.15\linewidth}>{\raggedright}p{0.15\linewidth}>{\raggedright}p{0.15\linewidth}>{\raggedright}p{0.15\linewidth}>{\raggedright\arraybackslash}p{0.15\linewidth}}
\toprule
\multirow{2}[3]{*}{Impact} & \multicolumn{5}{c}{Attack Feasibility}
\tabularnewline
\cmidrule{2-6}
 & 1 & 2 & 3 & 4 & 5
\tabularnewline
\midrule
1 & Low & Low & Low & Low & Low
\tabularnewline
2 & Low & Low & Moderate & Moderate & Moderate
\tabularnewline
3 & Low & Moderate & Moderate & Significant & Significant
\tabularnewline
4 & Low & Moderate & Significant & Very High & Very High
\tabularnewline
5 & Low & Moderate & Significant & Very High & Very High
\tabularnewline
\bottomrule
\end{tabular}}
\end{table}

\subsection{Summary of Remarks}\label{sec4: summary of remarks}



\section{Application of Attack Graphs to Risk Management Standards}\label{sec: application of attack graphs to risk management standards}

This sections discusses the application of Attack Graphs to ISO/SAE 21434~\cite{21434} and CLC/TS 50701~\cite{50701}, which specify engineering requirements for cybersecurity risk management in the automotive and railway environments, respectively.

\subsection{Risk Assessment According to ISO/SAE 21434}\label{sec: 21434 risk assessment}

The ISO/SAE 21434~\cite{21434} defines (1) an \emph{item} as a component or set of components that implements a function at the vehicle level and (2) an \emph{asset} as an object that has value or contributes to value. Assets have properties, whose compromise may realise \emph{damage scenarios}, which refer to the negative consequences imposed on items. In this regard, the first step in the risk assessment process according to ISO/SAE 21434 is the \emph{asset identification}, where the assets of an item are specified, and the possible damage scenarios are evaluated. The second step is the \emph{threat scenario identification}, where the potential causes (i.e. threats) of compromise of the assets' properties are analysed. Here, a threat scenario can lead to multiple damage scenarios, and a damage scenario can correspond to multiple threat scenarios. The third step is the \emph{impact rating}, where the magnitude of damage or physical harm (i.e impact) from a damage scenario is estimated. The damage scenarios are assessed against potential negative consequences, and the impact rating of the damage scenarios is determined to be (1) \emph{negligible}, (2) \emph{moderate}, (3) \emph{major}, or (4) \emph{severe}. The fourth step is the \emph{attack path analysis}, where threat scenarios are analysed for identifying attack paths. Here, attack paths are linked to the threat scenarios that can be realised by these attack paths. The fifth step is the \emph{attack feasibility rating}, where the ease of attack path exploitation is rated as (1) \emph{very low}, (2) \emph{low}, (3) \emph{medium}, or (4) \emph{high}. The sixth step is the \emph{risk determination}, where the risk of threat scenarios is determined from the impact of the associated damage scenarios and the attack feasibility of the associated attack paths. The risk value ranges from 1 (lowest risk) to 5 (highest risk). Here, risk matrices, such as the one shown in Table~\ref{tab: risk 21434}, can also be used for risk determination. In addition, if a threat scenario is linked to multiple attack paths, the attack feasibility of these attack paths is aggregated (i.e., the threat scenario is assigned the maximum attack feasibility level of the attack paths). The final step in the risk assessment process according to ISO/SAE 21434 is the \emph{risk treatment decision}, where treatment decisions for the identified risks are taken mainly based on the impact and attack feasibility ratings. Here, the risk treatment options are (1) \emph{risk avoidance}, by removing risk sources, (2) \emph{risk reduction}, by e.g., inserting countermeasures, (3) \emph{risk sharing or transference}, through e.g., contracts or insurances, or (4) \emph{risk acceptance}, in case of low impact and attack feasibility~\cite{21434}.

\begin{table}[h]
\renewcommand{\arraystretch}{1.2}
\caption{Risk matrix from ISO/SAE 21434~\cite{21434}.}
\label{tab: risk 21434}
\noindent\makebox[\linewidth]{%
\begin{tabular}[t]{>{\raggedright}p{0.12\linewidth}>{\centering}p{0.12\linewidth}>{\centering}p{0.12\linewidth}>{\centering}p{0.12\linewidth}>{\centering\arraybackslash}p{0.06\linewidth}}
\toprule
\multirow{2}[3]{*}{Impact} & \multicolumn{4}{c}{Attack Feasibility}
\tabularnewline
\cmidrule(lr){2-5}
 & Very Low & Low & Medium & High
\tabularnewline
\midrule
Negligible & 1 & 1 & 1 & 1
\tabularnewline
Moderate & 1 & 2 & 2 & 3
\tabularnewline
Major & 1 & 2 & 3 & 4
\tabularnewline
Severe & 1 & 3 & 4 & 5
\tabularnewline
\bottomrule
\end{tabular}}
\end{table}

\subsection{Application of Attack Graphs to ISO/SAE 21434}\label{sec: 21434 attack graphs}

The application of Attack Graphs to ISO/SAE 21434 is shown in Figure~\ref{fig: risk graph 21434}. The risk assessment process is carried out as follows. First (\emph{asset identification}), in this case, the \emph{item} is a UNIX system, and the \emph{asset} is the administrator privileges. Second (\emph{threat scenario identification}), the threat that may compromise the administrator privileges is represented by the topmost Inner Node (i.e., \enquote{Obtain Admin. Privileges}), which leads to two \emph{damage scenarios} (i.e., negative consequences) represented by Consequence Nodes (i.e., \enquote{Data Leakage} and \enquote{Denial of Rightful Access to the System}). Third (\emph{impact rating}), the edges relating Consequence Nodes to the topmost Inner Node are assigned with an Impact attribute. Here, if the \enquote{Obtain Admin. Privileges} threat is eventually realised, the impact of \enquote{Data Leakage} is \emph{moderate}, and the impact of \enquote{Denial of Rightful Access to the System} is \emph{major}. Fourth (\emph{attack path analysis}), attack paths are identified through the refinements of Inner Nodes. That is, the topmost Inner Node (i.e., \enquote{Obtain Admin. Privileges}) represents the high-level threat (i.e., the main goal of the attack), which is refined into low-level threats (i.e., sub-goals of the attack), represented by successor Inner Nodes, until the Leaf Nodes (green nodes) ultimately represent the least significant threats (i.e., elementary attacks). Hence, all Leaf and Inner Nodes of the same attack path need to be fulfilled for the \enquote{Obtain Admin. Privileges} threat to be realised (i.e., the main goal of the attack to be achieved). Fifth (\emph{attack feasibility rating}), every Leaf Node is assigned with Resources (hammer) and Knowledge (light bulb) attributes, whose value range from ... to ..., and a Location (pin) attribute, whose value range from ... to ... Then, the Attack Feasibility (red bubble) of every Leaf Node threat is computed, and Inner Nodes obtain the Resources, Knowledge, and Location attributes from their successor nodes either conjunctively or disjunctively, as described in Section~\ref{sec: attack graph risk estimation}. Sixth (\emph{risk determination}), the Impact attribute and the Attack Feasibility of the topmost Inner Node are used to determine the risk (red bubble) of Consequence Nodes. Here, using the risk matrix shown in Table~\ref{tab: risk 21434}, the Attack Graph shows a risk of 3 for \enquote{Data Leakage} and a risk of 4 for \enquote{Denial of Rightful Access to the System}. Finally (\emph{risk treatment decision}), the options to \emph{avoid}, \emph{reduce}, or \emph{share/transfer} the identified risks are represented by Defense Nodes. Conversely, if the Impact attribute and the Attack Feasibility of the topmost Inner Node are low, and therefore the identified risks are \emph{accepted}, Defense Nodes do not need to be added to the Attack Graph.

% The Attack Graph shows that the highest risk results from both \enquote{Corrupt Operator} and \enquote{Corrupt Sys. Admin}.

% Figure environment removed

\subsection{Risk Assessment According to CLC/TS 50701}\label{sec: 50701 risk assessment}

Similar to ISO/SAE 21434, CLC/TS 50701 is also describing an \emph{asset-based} risk assessment. In this regard, the first step in the risk assessment process according to CLC/TS 50701 is the \emph{system definition}, where the purpose, scope, operational environment, and applicable security standards of the \emph{system under consideration (SUC)} are specified. In addition, the valuable objects (i.e., assets) supporting the essential functions of the SuC are identified. Assets are classified into \emph{information technology (IT) assets}, whose compromise may lead to business consequences (e.g., loss of revenue), and \emph{operational technology (OT) assets}, whose compromise may lead to physical consequences (e.g., sustained service outages). The second step is the \emph{threat landscape}, where a list of potential negative actions or events (i.e., threats) capable of jeopardising the assets of the SUC is established and maintained. Here, threats are classified into internal (i.e. arising from within the system) and external (i.e. arising from without the system), and the skills and motivations driving the threats to exploit the vulnerabilities of the SUC and affect the assets of the SUC are documented. The third step is the \emph{impact assessment}, where the negative consequences, in terms of \emph{confidentiality, integrity, and availability (CIA)}, imposed on assets are evaluated. The CIA impact is assessed qualitatively and ranges from A (highest impact) to D (lowest impact). The fourth step is the \emph{likelihood assessment}, where the attack surfaces (i.e., exposure) and the level of expertise and/or resources required to exploit the flaws (i.e., vulnerabilities) of the SUC are evaluated. The value of both exposure and vulnerability ranges from 1 (highly restricted logical or physical access for attacker, vulnerability can only be exploited with high effort) to 3 (easy logical or physical access for attacker, vulnerability can be exploited with low effort). In addition, the likelihood function is $L = EXP + VUL - 1$, and therefore the likelihood of attacks range from 1 (highly unlikely) to 5 (highly likely). The final step in the risk assessment process according to CLC/TS 50701 is the \emph{risk evaluation}, where the risk of the documented threats being realised is determined usually by translating the threat landscape into a risk matrix, in which the assessed impact and likelihood of documented threats are related. An example of such a risk matrix is shown in Table~\ref{tab: risk 50701}.

\begin{table}[h]
\renewcommand{\arraystretch}{1.2}
\caption{Risk matrix from CLC/TS 50701~\cite{50701}.}
\label{tab: risk 50701}
\noindent\makebox[\linewidth]{%
\begin{tabular}[t]{>{\raggedright}p{0.10\linewidth}>{\centering}p{0.12\linewidth}>{\centering}p{0.12\linewidth}>{\centering}p{0.12\linewidth}>{\centering}p{0.12\linewidth}>{\centering\arraybackslash}p{0.12\linewidth}}
\toprule
\multirow{2}[3]{*}{Impact} & \multicolumn{5}{c}{Likelihood}
\tabularnewline
\cmidrule(lr){2-6}
 & 1 & 2 & 3 & 4 & 5
\tabularnewline
\midrule
D & Low & Low & Low & Medium & Significant
\tabularnewline
C & Low & Low & Medium & Significant & High
\tabularnewline
B & Low & Medium & Significant & High & High
\tabularnewline
A & Medium & Significant & High & High & Very High
\tabularnewline
\bottomrule
\end{tabular}}
\end{table}

\subsection{Application of Attack Graphs to CLC/TS 50701}\label{sec: 50701 attack graphs}

The application of Attack Graphs to CLC/TS 50701 is shown in Figure~\ref{fig: risk graph 50701}. The risk assessment process is carried out as follows. First (\emph{system definition}), in this case, the \emph{SUC} is a Linux system. Second (\emph{threat landscape}), the threats (that may jeopardise the assets of the Linux system) are represented by Leaf and Inner Nodes. Similar to ISO/SAE 21434, the topmost Inner Node represents the most significant threat, successor Inner Nodes represent less significant threats, and the Leaf Nodes (green nodes) ultimately represent the least significant threats. Hence, the topmost Inner Node is fulfilled when all Leaf and Inner Nodes of the same attack path are fulfilled. Third (\emph{impact assessment}), the negative consequences are represented by Consequence Nodes (i.e., \enquote{Data Leakage} and \enquote{Denial of Rightful Access to the System}). In addition, the edges relating Consequence Nodes to the topmost Inner Node are assigned with an Impact attribute, whose value ranges from A to D. Here, if the \enquote{Obtain Admin. Privileges} threat is eventually realised, the impact of \enquote{Data Leakage} is C, and the the impact of \enquote{Denial of Rightful Access to the System} is A. Fourth (\emph{likelihood assessment}), every Leaf Node is assigned with Exposure (lock) and Vulnerability (shield) attributes, whose value range from 1 to 3, and the likelihood (red bubble) of every Leaf Node threat is computed using the function $L = EXP + VUL - 1$. Then, Inner Nodes obtain the Exposure and Vulnerability attributes from their successor nodes either conjunctively or disjunctively, as described in Section~\ref{sec: attack graph risk estimation}. Finally (\emph{risk evaluation}), the Impact attribute and the likelihood of the topmost Inner Node are used to determine the risk (red bubble) of Consequence Nodes. Here, using the risk matrix shown in Table~\ref{tab: risk 50701}, the Attack Graph shows a significant (S) risk of \enquote{Data Leakage} and a high (H) risk of \enquote{Denial of Rightful Access to the System}.

% The Attack Graph shows that the highest risk results from \enquote{Corrupt Operator}.

% Figure environment removed

\subsection{Summary of Remarks}\label{sec5: summary of remarks}

This sections discusses the application of Attack Graphs to ISO/SAE 21434~\cite{21434} and CLC/TS 50701~\cite{50701}, which specify engineering requirements for cybersecurity risk management in the automotive and railway environments, respectively. The Attack Graphs were created using diagrams.net and the open source Attack Graphs plugin\footnote{\label{foot:Plugin}https://incyde-gmbh.github.io/drawio-plugin-attackgraphs/}.

\section{Conclusion}\label{sec: conclusion}

% The contributions of Attack Graphs are ...

Section~\ref{sec: related work} discusses that none of the graphical security modelling frameworks from current literature fulfills all six properties for graphical analysis of attack and defense scenarios. The eight in more detail discussed representatives are again shown in Table~\ref{tab: summary of frameworks}. The most relevant framework is the Boolean Logic Driven Markov Processes. However, even this one falls short when it gets to attributes, especially edge attributes.

\begin{table}[h]
\rowcolors{2}{gray!10}{gray!40}
\renewcommand{\arraystretch}{1.2}
\caption{Attack Graphs fulfill all six identified properties, whereas graphical security modelling frameworks from current literature fulfill at most five properties.}
\label{tab: summary of frameworks}
\noindent\makebox[\linewidth]{%
\begin{tabular}[t]{>{\raggedright}p{0.22\linewidth}>{\raggedright}p{0.13\linewidth}>{\raggedright}p{0.13\linewidth}>{\raggedright}p{0.13\linewidth}>{\raggedright}p{0.13\linewidth}>{\raggedright\arraybackslash}p{0.13\linewidth}>{\raggedright\arraybackslash}p{0.13\linewidth}}
\toprule
 & Attack Vectors & DAG Structure & Node Attributes & Dynamic Connectors & Edge Attributes & Defense Nodes
\tabularnewline
\midrule
Attack Trees & \checkmark & - & (\checkmark) & - & - & -
\tabularnewline
OWA Trees & \checkmark & - & - & (\checkmark)  & (\checkmark) & -
\tabularnewline
Cryptographic DAGs & \checkmark & \checkmark & - & - & - & -
\tabularnewline
Bayesian Networks for Security & \checkmark & \checkmark & \checkmark & - & \checkmark & -
\tabularnewline
Bayesian Attack Graphs & \checkmark & \checkmark & \checkmark & - & \checkmark & -
\tabularnewline
Security Activity Graphs & \checkmark & \checkmark & \checkmark & - & - & \checkmark
\tabularnewline
Countermeasure Graphs & \checkmark & \checkmark & \checkmark & - & - & \checkmark
\tabularnewline
Boolean Logic Driven Markov Processes & \checkmark & \checkmark & - & (\checkmark) & \checkmark & \checkmark
\tabularnewline
Cyber Security Modelling Language & \checkmark & \checkmark & (\checkmark) & - & (\checkmark) & \checkmark
\tabularnewline
Attack Graphs & \checkmark & \checkmark & \checkmark & \checkmark & \checkmark & \checkmark
\tabularnewline
\bottomrule
\end{tabular}}
\end{table}

Attack Graphs fulfill all six properties. As shown in this work, the framework is able to represent attack vectors in a DAG structure using any connectors necessary. To perform the necessary calculations user-defined attributes can be added to nodes or edges. Enabling the user to choose attributes that are relevant for the use-case at hand. Hence, the risk can be calculated according to the user defined functions using the connectors and directed edges. It is also possible to include defense nodes that represent any countermeasures, be it functional, operational, or any other kind of countermeasure. 

Attack Graphs were used in a research project supervised and financed by the German Centre of Rail Traffic Research\footnote{https://www.dzsf.bund.de}. In this project, 21 Attack Graphs were designed to analyse attack vectors of future rail systems and risk assessment was performed to evaluate what technologies pose the highest risk. The Attack Graphs\footnote{https://github.com/INCYDE-GmbH/attackgraphs} are publicly available as well as the developed open-source plugin\textsuperscript{\ref{foot:Plugin}} for diagrams.net\footnote{https://app.diagrams.net/}.

% But trees are not sufficient enough. Those Attack Trees for complex systems can get pretty big. And with the increasing connectivity of systems they are getting more complex, in terms of multiple technologies are in use, different systems are connected with each other to one big system, which in turn requires more interfaces. The more interfaces, systems and parties operate on one big system the bigger the attack surface.

\section*{Appendix: }\label{appendix}

\begin{lstlisting}[caption = AND function, numbers = left, breaklines = true, frame = single, firstnumber = 1, language = JavaScript]
function (collection) { 
    var result = {}; 
    collection.childAttributes.forEach(function(child) { 
    for (var attribute in child.attributes) { 
            if (attribute in result) { 
                result[attribute] + = parseInt(child.attributes[attribute]); 
            } else { 
                result[attribute] = parseInt(child.attributes[attribute]); 
            } 
        } 
    }); 
    for (var attribute in result) { 
        if (attribute in collection.globalAttributes) { 
            result[attribute] = Math.min(collection.globalAttributes[attribute].max, result[attribute]); 
        } 
    } 
    return result; 
}
\end{lstlisting}

\begin{lstlisting}[caption = OR function, numbers = left, breaklines = true, frame = single, language = JavaScript]
function (collection) { 
    var result = null; 
    if (collection.childAttributes.length = = 1) { 
        result = collection.childAttributes[0].attributes; 
    } else { 
        var candidates = []; 
        var worstValue = 0; 
        collection.childAttributes.forEach(function(child) { 
            var value = parseInt(child.computedAttribute); 
            if (value > worstValue) { 
                worstValue = value; 
                candidates = []; 
                candidates.push(child); 
            } else if (value = = worstValue) { 
                candidates.push(child); 
            } 
        }); 
        var tiebreaker = function(candidates, name, max) { 
            var min_value = max; 
            candidates.forEach(function(node) { 
                min_value = Math.min(min_value, node.attributes[name]); 
            }); 
            result = []; 
            candidates.forEach(function(node) { 
                if (node.attributes[name] = = min_value) { 
                    result.push(node); 
                } 
            }); 
            return result; 
        }; 
        if (candidates.length > 1) { 
            candidates = tiebreaker(candidates, "Resources", collection.globalAttributes["Resources"].max); 
            if (candidates.length > 1) { 
                candidates = tiebreaker(candidates, "Knowledge", collection.globalAttributes["Knowledge"].max); 
            } 
        } 
        result = candidates[0].attributes; 
    } 
    return result; 
}
\end{lstlisting}

% \bibliography{references}
% \bibliographystyle{ieeetr}

\printbibliography

\end{document}
