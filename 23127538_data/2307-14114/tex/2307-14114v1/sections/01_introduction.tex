% !TEX root = ../AttackGraphBasedRiskAnalysis.tex
% !TEX spellcheck = en_US
% !TEX encoding = UTF-8 Unicode

\section{Introduction}\label{sec: intro}

Traditional cities are becoming smarter. 
One of the core smart city concepts is smart mobility, which has attracted considerable attention from security researchers due to the emergence of smart vehicles and V2X communication that have given rise to novel cybersecurity threats.

Over the last decade, several trends have contributed to the automotive and railway threat landscape. 
First, sophisticated features in smart vehicles come with a higher volume of lines of code, aggravating testability and auditing and increasing the likelihood and severity of vulnerabilities. 
Second, (wireless) communication interfaces in smart vehicles come with a higher volume of external peripheral devices that can connect to smart vehicles, hence increasing the attackers' access point options, and also with a higher volume of connections, hence increasing the risk of malicious interactions. 
Finally, a higher volume of connections between smart vehicles comes with a higher volume of exchanged data, which in most cases is personal and, therefore, immensely valuable. In other words, more data is generated and needs to be considered and protected.

Graphical security modeling is a widely-used and well-established approach for representing and analyzing threat landscapes that examine vulnerabilities of systems and organizations. 
One of the primary strengths of graphical security models is that they allow for the inclusion of user-friendly visual elements with formal semantics and algorithms, enabling both qualitative and quantitative analyses. 
Over the last couple of decades, security researchers have been progressively focusing on graphical security modeling, which has gradually evolved into a valuable tool for the assessment of risks in real-life systems, such as automotive and railway environments.

Threat landscapes include (1) malicious actions of an attacker, whose goal is to harm or damage one or more assets of a system or organization, and (2) countermeasures for either preventing or mitigating such malicious actions. 
The first \emph{tree-based approach} for graphical security modeling was the \emph{threat logic trees}, which was introduced by Weiss in 1991~\cite{weiss1991}, thereby motivating the development of several subsequent frameworks, such as attack trees, which are still considered one of the most important and favored tools for the assessment of risks to date.

In all tree-based approaches, the modeling process begins with identifying a feared event, which is shown as a root node, and continues with the refinement of the attack steps, resulting in a tree model.
However, tree structures are limited to only one path between a pair of nodes. 
In other words, with tree structures, each refined node can only have one parent node. 
This limitation is addressed by the \emph{directed acyclic graph (DAG) structure}, which enables refined nodes to have multiple parent nodes. 
As a result, DAG structures can provide a higher level of detail, but they can also come with a higher level of complexity, which can nevertheless be dealt with modularization, thereby allowing the model to be subdivided into loosely-coupled, independent, and interchangeable parts that can be studied individually and in parallel. 
Finally, while the one-to-many relationship between nodes in tree structures results in a linear analysis of the threat landscape, the many-to-many relationship between nodes in DAG structures can theoretically result in an exponential analysis.
However, the complexity is kept small in practice due to the acyclic structure, and the threat landscape analysis is eventually possible.

Ensuring the security of systems is not a static process that is over after going through once.
The conditions are constantly changing, on the one hand attackers and their capabilities are evolving, and on the other hand, systems themselves are being extended and evolving.
To effectively perform the necessary continuous security management, it is necessary to know not just the threat landscape but to be able to understand the consequences and impacts if attacks are performed successfully.
Hence, it is necessary to continuously perform a risk analysis to identify the potential exposure.
Nowadays, risk management is primarily done using large tables filled with a lot of information and use cases.
Large tables only offer limited visibility, as it is challenging to maintain a comprehensive overview of risks.
With numerous rows and columns, it becomes difficult to identify trends and patterns or prioritize risks effectively.
Furthermore, managing risk can be a tedious and time-consuming process.
Updating and maintaining tables with evolving risks and mitigation measures can require significant effort, especially when dealing with a complex system or multiple risk factors.
This gets even harder when dealing with large tables that often fail to provide the necessary context and connections between different risks.
Additionally, analyzing and interpreting data from large tables can be daunting. 
It may require specialized tools or skills to extract meaningful insights from the extensive amount of information presented in the table format.
Large tables may further lack the flexibility to accommodate changing risk scenarios or evolving requirements. 
Modifying or updating the table structure to incorporate new risks or factors can be cumbersome and may hinder agility in risk management.
With numerous cells and data entries, there is also an increased risk of errors, inaccuracies, or inconsistencies in the large table. 
These issues can undermine the reliability and integrity of the risk management process.

We propose a graphical solution for the risk management process to mitigate these disadvantages of tables.
A visual representation can enhance the understanding and communication of complex risk information and make it easier to identify patterns, trends, and relationships among risks, facilitating effective decision-making.
Complex risk data is further simplified by presenting it in a clear and concise manner.
Understanding  the relationships, dependencies, and interactions between various risk elements is necessary to understand the overall risk landscape.
Visual representations of the entire risk landscape provide this overview, allowing for the identification of interdependencies, hotspots, or areas of high vulnerability.
Graphical solutions can also aid in developing and evaluating risk mitigation strategies. 
By visually representing the potential consequences and effectiveness of different mitigation measures, decision-makers can make more informed choices and allocate resources more efficiently.
Furthermore, it allows for the exploration of different risk scenarios. 
By manipulating variables or parameters within the visual representation, it becomes possible to assess the potential impact of various risk factors and evaluate the effectiveness of different response strategies.
Additionally, as graphical solutions can be more adaptable to changing requirements and evolving risks, they allow for easier updates and modifications, enabling risk management processes to be more responsive and agile.

Consequently, we believe a graphical solution for the risk assessment process improves the maintenance of risk scenarios and facilitates accessibility to different stakeholders, including non-technical audiences.
However, the existing graphical solutions are momentarily used to describe the threat landscape.
Which, of course, is helpful for the risk management process but not sufficient to represent the entire risk management process.
Therefore, motivating us to define a new graphical method for risk assessment by extending existing graphical methods for depicting the threat landscape.
Besides ways to depict attack vectors, their probability, and countermeasures, our method includes a way to depict the consequences of attack vectors and the impact level, enabling us to calculate a risk value.

The remainder of the paper is structured as follows:
After the introduction,~\cref{sec: related work} discusses the related work.
Our definition of attack graphs is given in~\cref{sec: attack graphs}.
The necessary adjustments to use these attack graphs are presented in~\cref{sec: attack graph risk assessment}, including an example of how the risk assessment is performed in our project.
\cref{sec: applicability of attack graphs to risk management standards} validates our defined method by combing it with the risk assessment processes of ISO/SAE 21434~\cite{21434} and CLC/TS 50701~\cite{50701} respectively.
The scalability and practicality are evaluated in~\cref{sec: evaluation}.
Finally,~\cref{sec: conclusion} concludes this paper.