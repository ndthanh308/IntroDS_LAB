% !TEX root = ../AttackGraphBasedRiskAnalysis.tex
% !TEX spellcheck = en_US
% !TEX encoding = UTF-8 Unicode

\section{Attack Graph Risk Assessment}\label{sec: attack graph risk assessment}

We enhance the Attack Graph components summarized in Section~\ref{sec: attack graphs} with (1) \emph{Consequence Nodes}, which indicate the negative consequences of the main goal of the attack and create a relation to constitute the impact of an attack, (2) \emph{Attack Feasibility Attributes}, which indicate the ease of launching an attack, and (3) \emph{Countermeasure Nodes}, which indicate security measures for either preventing or mitigating attack steps or the impact of those.
The Attack Graph adaptation of Weiss's attack tree is shown in Figure~\ref{fig: attack tree weiss modified}.

% Figure environment removed

% The purpose of Attack Graphs is ... \\

\noindent
\textbf{Consequence Nodes}\\
In Attack Graphs, \emph{Consequence Nodes} (i.e., \enquote{Data Leakage} and \enquote{Denial of Rightful Access to the System}), shown as rounded corner rectangles, become the highest nodes (i.e. the Root Nodes) in the tree structure.
They indicate the negative consequences of the main goal of the attack, which is now represented by the topmost Inner Node (i.e., \enquote{Obtain Admin. Privileges}). 
For the rest of the article, Root Nodes are referred to as Consequence Nodes.

Consequence nodes are not part of the set $\mathcal{N}$ and build their own set $\mathcal{N}_C$, with $\mathcal{N} \cap \mathcal{N}_C = \emptyset$, and with a different set of attributes $\mathcal{A}_R$, with $\mathcal{A}_R \cap \mathcal{A} = \emptyset$.
This is necessary as nodes of the set $\mathcal{N}$ represent actions with their attributes, consequently describing properties necessary to perform the action or how easy it is to perform this action.
Whereas nodes of the set $\mathcal{N}_C$ describe the consequences of actions, and thus their attributes  typically describe the risk of that consequence to arise. \\

\noindent
\textbf{Attack Feasibility Attributes}\\ 
\emph{Attack Feasibility Attributes} are shown as colored squares on the top right of Inner Nodes, indicating the ease of launching an attack.
They are computed using either (1) a \emph{function} of a set of given attribute values, (2) a \emph{matrix} of a set of given attribute values or (3) a combination of these methods.

In the first case, a value $\emph{v}_m \in \mathcal{V}_k$ of an Attack Feasibility Attribute $\emph{a}_k \in \mathcal{A}$ of a node $n$ can be computed using a function that aggregates the attributes $\emph{a}_{1}, \emph{a}_{2}, ..., a_{k-1} \in \mathcal{A}$ of node $n$, with $a_k \neq a_{1}, a_{2}, ..., a_{k-1}$:

\begin{center}
$\emph{v}_m = f(\emph{a}_{1}, \emph{a}_{2}, ..., a_{k-1})$.
\end{center}

\noindent
Similarly, in the second case, a set of attributes $d := |\{a_1, a_{2}, ..., a_{k-1}\}|$ are aggregated over a $d$-dimensional matrix. 
An example of a 2-dimensional matrix is shown in Table~\ref{tab: 2d matrix}. 
Here, for a node $n$ with attributes $a_1$ and $a_2$, there exist a set of values $\mathcal{V}_1 := \{v_0, v_1, ..., v_{m_1}\}$ for $a_1$ and a set of values $\mathcal{V}_2 := \{u_0, u_1, ..., u_{m_2}\}$ for $a_2$. 
As such, the aggregation of $\mathcal{V}_1$ and $\mathcal{V}_2$ returns a set of values $\mathcal{V}_3 := \{v_{00}, v_{01}, ..., v_{{m_1}{m_2}}\}$ for an Attack Feasibility Attribute $a_3$.

\begin{table}[h]
\renewcommand{\arraystretch}{1.2}
\caption{2D matrix for an Attack Feasibility Attribute $a_3$, with $a_1, a_2, a_3$ being attributes of a node $n$.}
\label{tab: 2d matrix}
\noindent\makebox[\linewidth]{%
\begin{tabular}[t]{>{\raggedright}p{0.05\textwidth}>{\raggedright}p{0.05\textwidth}>{\raggedright}p{0.05\textwidth}>{\raggedright}p{0.05\textwidth}>{\raggedright\arraybackslash}p{0.05\textwidth}}
\toprule
 & \multicolumn{4}{c}{$a_2$}
\tabularnewline
\cmidrule{2-5}
$a_1$ & $u_0$ & $u_1$ & ... & $u_{m_2}$
\tabularnewline
\midrule
$v_0$ & $v_{00}$ & $v_{01}$ & ... & $v_{0{m_2}}$
\tabularnewline
$v_1$ & $v_{10}$ & $v_{11}$  & ... & $v_{1{m_2}}$
\tabularnewline
... & ... & ... & ... & ...
\tabularnewline
$v_{m_1}$ & $v_{{m_1}0}$ & $v_{{m_1}1}$ & ... & $v_{{m_1}{m_2}}$
\tabularnewline
\bottomrule
\end{tabular}}
\end{table}

In the third case, the two previously described methods are combined in an arbitrary way.
An example that is also used later, is that the (genuine) subset $s \subset \{a_1, a_{2}, ..., a_{k-1}\}$ is aggregated using an $|s|$-dimensional matrix, which generates a preliminary result $v_{pre}$.
The remaining attributes $r := \{a_1, a_{2}, ..., a_{k-1}\} \setminus s$ are then aggregated together with $v_{pre}$ in a function $f$, generating the final result.
E.g., let s be the set of values $\{a_1, a_{2}, ..., a_{k-2}\}$ which are aggregated to form the preliminary result $v_{pre}$.
Then $r = \{a_{k-1}\}$ and the result $v$ is calculated by using the function $f$ over the preliminary result $v_{pre}$ and the value of $a_{k-1}$: $v = f(v_{pre}, g(a_{k-1}))$.

We want to acknowledge the fact that the \emph{Attack Feasibility} attribute can have different names, like attack or success probability or likelihood, especially in some standards.
However, \emph{probability} is not a useful name and can lead to some irritation, as it is not mathematically computable probability in terms of statistics or percentages.
Hence, we chose a different wording that is also present in some standards.\\

% The values need to be in the acceptable range of values for attribute $a_3$, hence $v_{00}, v_{01}, v_{10}, ..., v_{{m_1}{m_2}} \in \mathcal{V}_3$. Acceptable values need to be chosen for every attribute and do not have to be numerical values. So, if $g(a_1) = v_{m_1}$ and $g(a_2) = u_{m_2}$, then $g(a_3) = v_{{m_1}{m_2}}$.

\noindent
\textbf{Countermeasure Nodes}\\
Finally, \emph{Countermeasure Nodes} (i.e., \enquote{Physical Access Restriction}, \enquote{Firewall} and \enquote{Vulnerability / Malware Scans}), shown as light blue nodes in~\cref{fig: attack tree weiss modified}, are included to indicate the security measures taken to mitigate the risk.
%become the lowest nodes (i.e. the Leaf Nodes) in the tree structure, and they indicate the countermeasures for either preventing or mitigating the least significant threats (i.e., elementary attacks), which are now represented by the lowermost Inner Nodes, shown as green nodes.
%To perform continuous risk assessment it is necessary to include planned countermeasures against attacks as well as already applied ones. 
There are two ways how those countermeasures can influence the risk. 
First, there are countermeasures that mitigate or prevent specific attack steps. 
These are attached as new leaf nodes, as shown in~\cref{fig: attack tree weiss modified}. %with the (light blue) nodes \enquote{Physical Access Restriction}, \enquote{Firewall} and \enquote{Vulnerability / Malware Scans}. 
Using the aggregation functions, the node attributes of the parent nodes are influenced. 
In the example of~\cref{fig: attack tree weiss modified} the values of the countermeasure nodes are added to their parent nodes which complicate the attack and therefore lower the attack feasibility of the predecessor nodes. 
Taking the countermeasure node \enquote{Physical Access Restriction} and the lowermost inner node \enquote{Break in to Comp. Center} on the lower left of~\cref{fig: attack tree weiss modified}.
The node \enquote{Physical Access Restriction} is rated to increase the value of \emph{Resources} and \emph{Knowledge} by one as the aggregation function chosen here is the sum function.
Hence, the values for \emph{Resources} and \emph{Knowledge} of \enquote{Break in to Comp. Center} are both increased by one to their new value of two.
The originally rated value for \enquote{Break in to Comp. Center} is still visible in blue but crossed out in~\cref{fig: attack tree weiss modified} to illustrate the effect of countermeasure nodes.

Second, the countermeasure can influence an attack's impact on a specific consequence. 
This case is a bit more complicated to illustrate as it is influencing an edge (between a \emph{Consequence Node} and a topmost Inner Node) instead of a node. 
However, the principle is the same as for nodes.

\subsection{Application of Risk Assessment Graphs}\label{sec: attack graph risk estimation}

We now show an example of how to adapt the method of Attack Graphs to facilitate risk management.
When the described definitions are used to create a graphical representation or risk landscape, we refer to these graphs as \emph{Risk Assessment Graphs}.

\begin{table*}[ht]
\renewcommand{\arraystretch}{1.2}
\caption{Resources, Knowledge, and Location matrix.}
\label{tab: resources, knowledge, and location matrix}
\noindent\makebox[\textwidth]{%
\begin{tabular}[t]{>{\raggedright}p{0.15\textwidth}>{\centering}p{0.1\textwidth}>{\centering}p{0.15\textwidth}>{\centering}p{0.1\textwidth}>{\centering}p{0.1\textwidth}>{\centering}p{0.1\textwidth}>{\centering\arraybackslash}p{0.15\textwidth}}
\toprule
 & 0 & 1 & 2 & 3 & 4 & 5
\tabularnewline
\midrule
Resources & - & Basic & Low & Medium & High & Extraordinary
\tabularnewline
Knowledge & - & Basic & Low & Medium & High & Extraordinary
\tabularnewline
Location & Remote & Local & - & - & - & -
\tabularnewline
\bottomrule
\end{tabular}}
\end{table*}

According to DIN VDE V 0831-104, the \emph{Attack Feasibility (AF)} of the Inner Nodes shown in Figure~\ref{fig: attack tree weiss modified} is computed based on the \emph{attacker's capabilities} and \emph{mitigation factors}. 
First, the attacker's capabilities are described by two attributes: (1) \emph{Resources (R)}, reflecting the financial and workforce capacity of the attacker to prepare and launch an attack, and (2) \emph{Knowledge (K)}, reflecting the information that the attacker holds about the system they intend to attack. 
The Resources and Knowledge of the attacker are each rated as \emph{low} ($R,K = 2$), \emph{medium} ($R,K = 3$), or \emph{high} ($R,K = 4$) in the example given in DIN VDE 0831-104. As such, attackers with \emph{basic} and \emph{extraordinary }capabilities ($R,K = 1$ and $R,K = 5$) are not considered.
The values for Resources and Knowledge were extended to distinguish better between different attacks.
Second, the mitigation factors relate to the risk of the attacker being discovered, and they are described by the \emph{Location (L)} attribute, which reflects whether an attack can be launched remotely ($L = 0$) or locally ($L = 1$). 
The values of the three attributes are summarised in Table~\ref{tab: resources, knowledge, and location matrix}.



In this regard, every lowermost Inner Node (green node) is assigned the Resources (hammer) and Knowledge (light bulb) attributes, whose value range from 2 to 4, and a Location (pin) attribute, whose value is either 0 or 1.
First, the Resources and Knowledge of the attacker are related in a matrix, such as the one shown in Table~\ref{tab: preliminary attack feasibility matrix}, to initially determine a \emph{Preliminary Attack Feasibility (PAF)}, which indicates the ease of launching an attack without taking into consideration the risk of being discovered. 
For example, Figure~\ref{fig: attack tree weiss modified} shows that the \enquote{Break in to Comp. Center} attack (i.e., bottom left Leaf Node) requires \emph{Low Resources} ($R = 2$) and \emph{Low Knowledge} ($K = 2$) to be launched. 
In this case, Table~\ref{tab: preliminary attack feasibility matrix} shows that for $R = 3$ and $K = 3$, $PAF = 4$. 
Second, the Location of the attacker is subtracted from the Preliminary Attack Feasibility, to eventually determine the Attack Feasibility. 
Here, Figure~\ref{fig: attack tree weiss modified} shows that the \enquote{Break in to Comp. Center} attack requires \emph{Local Access} ($L = 1$) to be launched.
In this case, the Attack Feasibility is equal to 3 ($AF = PAF - L = 4 - 1 = 3$). 

\begin{table}[h]
\renewcommand{\arraystretch}{1.2}
\caption{Preliminary Attack Feasibility matrix.}
\label{tab: preliminary attack feasibility matrix}
\noindent\makebox[\linewidth]{%
\begin{tabular}[t]{>{\raggedright}p{0.05\textwidth}>{\raggedright}p{0.05\textwidth}>{\raggedright}p{0.05\textwidth}>{\raggedright}p{0.05\textwidth}>{\raggedright}p{0.05\textwidth}>{\raggedright\arraybackslash}p{0.025\textwidth}}
\toprule
 & \multicolumn{5}{c}{K}
\tabularnewline
\cmidrule{2-6}
R & 1 & 2 & 3 & 4 & 5
\tabularnewline
\midrule
1 & 5 & 5 & 4 & 3 & 2
\tabularnewline
2 & 5 & 4 & 4 & 3 & 2
\tabularnewline
3 & 5 & 4 & 4 & 3 & 2
\tabularnewline
4 & 4 & 3 & 3 & 2 & 1
\tabularnewline
5 & 3 & 2 & 2 & 1 & 1
\tabularnewline
\bottomrule
\end{tabular}}
\end{table}


When the Attack Feasibility of all lowermost Inner Nodes is determined, the Resources, Knowledge, and Location attributes of all predecessor Inner Nodes are obtained from their successor Inner Nodes, in this case, either conjunctively (AND) or disjunctively (OR).
Other graphs can include different connectors as well.
Similar to Weiss's attack tree~\cite{weiss1991}, only conjunctive refinements are shown explicitly. 
That is, all refinements shown in the Risk Assessment Graph are disjunctive unless a predecessor Inner Node is connected to its successor Inner Node through an AND gate, in which case the refinement is conjunctive. 
Regarding conjunctive refinements, an Inner Node obtains, in this case, the sum of values for each Resource, Knowledge, and Location attribute exhibited among its successor nodes, and the Attack Feasibility is determined based on the resulting values. 
For example, the attack \emph{Guess Password} can only succeed if an attacker obtains the password file and encounters a guessable password.
Hence, the nodes \emph{Obtain Password File} ($R=1, K=3, L=0$) and \emph{Encounter Guessable Password} ($R=1, K=2, L=0$) are conjunctively joined and their predecessor node \emph{Guess Password} obtains the sum of their values ($R=2, K=5, L=0$).
This particular sum function has an upper limit, meaning that if the addition of the successor values would result in a higher value than defined in the range of this attribute, then the maximum value for this attribute will be chosen for the predecessor node.
For example, if the sum of the values for Knowledge would exceed the value 5, the predecessor node will obtain 5 as the value for Knowledge.
Regarding disjunctive refinements, an Inner Node obtains its successor node's Resources, Knowledge, and Location attributes with the highest Attack Feasibility. 
If the highest Attack Feasibility is exhibited by multiple successor nodes, we defined an order on the attributes, deeming Resources more critical than knowledge.
If the lowest Resource value of the successor nodes exhibiting the highest Attack Feasibility is exhibited by multiple successor nodes, the values for Knowledge are compared.
If multiple of these nodes exhibit the same lowest value for Knowledge, any node can be chosen as they must have the same rating.
Because, if the values for Attack Feasibility, Resources, and Knowledge are identical, then the values of Location have to be identical as well or it would not result in the same Attack Feasibility.
For example, in the case of \emph{Look over Sys. Admin. Shoulder} and \emph{Corrupt Sys. Admin}, both have an Attack Feasibility of 4, so to decide which one is more critical, the algorithm checks the Resource Attribute.
The Resource Attribute for those nodes is different and as \emph{Look Over Sys. Admin. Shoulder} requires only \emph{Basic Resources} the predecessor node \emph{Obtain Admin Password} obtains its values. 
%The implemented AND-OR functions are listed in the~\hyperref[appendix]{Appendix}.


\begin{table*}[h]
\renewcommand{\arraystretch}{1.2}
\caption{Impact matrix.}
\label{tab: impact matrix}
\noindent\makebox[\textwidth]{%
\begin{tabular}[t]{>{\raggedright}p{0.05\textwidth}>{\raggedright}p{0.1\textwidth}>{\raggedright\arraybackslash}p{0.7\textwidth}}
\toprule
Value & Impact & Description
\tabularnewline
\midrule
1 & Negligible & Impact can be readily absorbed, without requiring management effort.
\tabularnewline
2 & Minor & Impact can be readily absorbed, requiring some management effort.
\tabularnewline
3 & Moderate & Impact cannot be readily absorbed, requiring a modest level of resources and management effort.
\tabularnewline
4 & Major & Impact requires a high level of resources and management effort to rectify.
\tabularnewline
5 & Severe & Disaster with the potential to lead to business collapse, requiring total management effort to rectify.
\tabularnewline
\bottomrule
\end{tabular}}
\end{table*}

\begin{table*}[h]
\renewcommand{\arraystretch}{1.2}
\caption{Risk matrix.}
\label{tab: risk matrix}
\noindent\makebox[\textwidth]{%
\begin{tabular}[t]{>{\raggedright}p{0.1\textwidth}>{\raggedright}p{0.15\textwidth}>{\raggedright}p{0.15\textwidth}>{\raggedright}p{0.15\textwidth}>{\raggedright}p{0.15\textwidth}>{\raggedright\arraybackslash}p{0.15\textwidth}}
\toprule
\multirow{2}[3]{*}{Impact} & \multicolumn{5}{c}{Attack Feasibility}
\tabularnewline
\cmidrule{2-6}
 & 1 & 2 & 3 & 4 & 5
\tabularnewline
\midrule
1 & Low & Low & Low & Low & Low
\tabularnewline
2 & Low & Low & Moderate & Moderate & Moderate
\tabularnewline
3 & Low & Moderate & Moderate & Significant & Significant
\tabularnewline
4 & Low & Moderate & Significant & Very High & Very High
\tabularnewline
5 & Low & Moderate & Significant & Very High & Very High
\tabularnewline
\bottomrule
\end{tabular}}
\end{table*}

In addition, the edges relating Consequence Nodes to the topmost Inner Node are assigned with an \emph{Impact} attribute, which indicates the magnitude of damage or physical harm caused by negative consequences on the system. 
The impact of negative consequences is typically rated using numerical ranges and qualitative scales, as shown in Table~\ref{tab: impact matrix}, however, the impact rating terminology remains rather inconsistent across risk assessment standards. 
Here, if the \enquote{Obtain Admin. Privileges} attack is eventually achieved, the impact of \enquote{Data Leakage} is 3 (\emph{moderate)}, and the impact of \enquote{Denial of Rightful Access to the System} is 2 (\emph{minor)}.


% The Attack Graph shows that the highest risk results from \enquote{Trojan Horse SA Account}.




Finally, the Impact and Attack Feasibility attributes are related in a matrix, such as the one shown in Table~\ref{tab: risk matrix}, to determine a Risk attribute, shown as colored squares on the top right of Consequence Nodes, indicating whether the current risk level is acceptable or not.
Here, the Risk Assessment Graph shows a  risk of \emph{significant (S)} \enquote{Data Leakage} and a \emph{moderate (M)} risk of \enquote{Denial of Rightful Access to the System}.


\subsection{Summary of Remarks}\label{sec4: summary of remarks}

In this section, we defined the components required to perform risk assessment using attack graphs:
(1) \emph{consequence nodes},
(2) \emph{attack feasibility attributes}, and
(3) \emph{countermeasure nodes}.
These attributes contribute to extending the threat landscape to a risk landscape.
Apart from the formal definitions, we further showed a practical example of how to utilize this method.
The example is grounded on the risk assessment process described in the DIN VDE V 0831-104.
The described risk assessment process was also used in the project \enquote{Forecast of security requirements and evaluation of possible security concepts for the railway system} provided by the German Center for Rail Traffic Research (DZSF) at the Federal Railway Authority (EBA).