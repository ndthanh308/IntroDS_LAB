% !TEX root = ../AttackGraphBasedRiskAnalysis.tex
% !TEX spellcheck = en_US
% !TEX encoding = UTF-8 Unicode

\section{Applicability of Risk Assessment Graphs to Risk Management Standards}\label{sec: applicability of attack graphs to risk management standards}

This section discusses the applicability of Risk Assessment Graphs to ISO/SAE 21434~\cite{21434} and CLC/TS 50701~\cite{50701}, which specify engineering requirements for cybersecurity risk management in the automotive and railway environments, respectively.
We show that our methodology is able to assist different risk analysis methodologies.

\subsection{Risk Assessment According to ISO/SAE 21434}\label{sec: 21434 risk assessment}

\begin{table*}[h]
\renewcommand{\arraystretch}{1.2}
\caption{Example aggregation of attack potential.}
\label{tab: attack potential}
\noindent\makebox[\textwidth]{%
\begin{tabular}{>{\raggedright}p{0.1\textwidth}>{\raggedright}p{0.05\textwidth}|>{\raggedright}p{0.07\textwidth}>{\raggedright}p{0.05\textwidth}|>{\raggedright}p{0.1\textwidth}>{\raggedright}p{0.05\textwidth}|>{\raggedright}p{0.1\textwidth}>{\raggedright}p{0.05\textwidth}|>{\raggedright}p{0.08\textwidth}>{\raggedright}p{0.05\textwidth}}
\toprule
\multicolumn{2}{c}{Elapsed Time} & \multicolumn{2}{c}{\parbox{0.1\textwidth}{Specialist Expertise}}  & \multicolumn{2}{c}{\parbox{0.15\textwidth}{Knowledge of the item (or Component)}}  & \multicolumn{2}{c}{\parbox{0.1\textwidth}{Window of Opportunity}}  & \multicolumn{2}{c}{Equipment}
\tabularnewline
\cmidrule{1-10}
 Enumerate & Value &  Enumerate & Value & Enumerate & Value & Enumerate & Value & Enumerate & Value 
\tabularnewline
\midrule
$<1$ day & 0 & Layman & 0 & Public & 0 & Unlimited & 0 & Standard & 0
\tabularnewline
$<1$ week & 1 & Proficient & 3 & Restricted & 3 & Easy & 1 & Specialized & 4
\tabularnewline
$<1$ month & 4 & Expert & 6 & Confidential & 7 & Moderate & 4 & Bespoke & 7
\tabularnewline
$<6$ months & 17 & Multiple Experts & 8 & Strictly Confidential & 11 & Difficult/None & 10 & Multiple Bespoke & 9
\tabularnewline
$>6$ months & 19 & & & & & & & &
\tabularnewline
\bottomrule
\end{tabular}}
\end{table*}


The ISO/SAE 21434~\cite{21434}, a widely used standard in the automotive domain, defines (1) an \emph{item} as a component or set of components that implement a function at the vehicle level and (2) an \emph{asset} as an object that has value or contributes to value. 
Assets have properties whose compromise may realize \emph{damage scenarios}, which refer to the negative consequences imposed on items.
In this regard, the first step in the risk assessment process, according to ISO/SAE 21434 is the \emph{asset identification}, where the assets of an item are specified, and the possible damage scenarios are evaluated. 
The second step is the \emph{threat scenario identification}, where the potential causes (i.e., threats) of compromise of the assets' properties are analyzed. 
Here, a threat scenario can lead to multiple damage scenarios, and a damage scenario can correspond to multiple threat scenarios. 
The third step is the \emph{impact rating}, where the magnitude of damage or physical harm (i.e impact) from a damage scenario is estimated. 
The damage scenarios are assessed against potential negative consequences, and the impact rating of the damage scenarios is determined to be (1) \emph{negligible}, (2) \emph{moderate}, (3) \emph{major}, or (4) \emph{severe}.
The fourth step is the \emph{attack path analysis}, where threat scenarios are analyzed for identifying attack paths. Here, attack paths are linked to the threat scenarios that can be realized by these attack paths.
The fifth step is the \emph{attack feasibility rating}, where the ease of attack path exploitation is rated as (1) \emph{very low}, (2) \emph{low}, (3) \emph{medium}, or (4) \emph{high}. 
According to the standard, the attack feasibility rating should be based on either (a) an \emph{attack potential}-based approach, (b) a $CVSS^2$-based approach, or (c) an \emph{attack vector}-based approach.
If the rating is based on the attack potential, the attack feasibility should be determined based on core factors including \emph{elapsed time}, \emph{specialist expertise}, \emph{knowledge of the item or component}, \emph{window of opportunity}, and \emph{equipment}.
For each attribute, numerical values can be defined.
The informative parts of the standard propose an example based on the ISO/IEC 18045~\cite{18045}, see~\cref{tab: attack potential}.
The attack potential is then calculated by adding all parameters, the result is then mapped according to~\cref{tab: attack feasibility matrix 21434}. 



\begin{table}[h]
\renewcommand{\arraystretch}{1.2}
\caption{Attack Feasibility mapping according to ISO/SAE 21434.}
\label{tab: attack feasibility matrix 21434}
\noindent\makebox[\linewidth]{%
\begin{tabular}[t]{>{\raggedright}p{0.1\textwidth}>{\raggedright}p{0.15\textwidth}}
\toprule
Values & Attack Feasibility
\tabularnewline
\midrule
0-13 & High
\tabularnewline
%\midrule
14-19 & Medium
\tabularnewline
%\midrule
20-24 & Low
\tabularnewline
%\midrule
$>=$25 & Very Low 
\tabularnewline
\bottomrule
\end{tabular}}
\end{table}

The sixth step is the \emph{risk determination}, where the risk of threat scenarios is determined from the impact of the associated damage scenarios and the attack feasibility of the associated attack paths. 
The risk value ranges from 1 (lowest risk) to 5 (highest risk). Here, risk matrices, such as the one shown in Table~\ref{tab: risk 21434}, also proposed by the informative parts of the standard, can also be used for risk determination. 
In addition, if a threat scenario is linked to multiple attack paths, the attack feasibility of these attack paths is aggregated (i.e., the threat scenario is assigned the maximum attack feasibility level of the attack paths). 
The final step in the risk assessment process according to ISO/SAE 21434 is the \emph{risk treatment decision}, where treatment decisions for the identified risks are taken mainly based on the impact and attack feasibility ratings. 
Here, the risk treatment options are (1) \emph{risk avoidance}, by removing risk sources, (2) \emph{risk reduction}, by e.g., inserting countermeasures, (3) \emph{risk sharing or transference}, through e.g., contracts or insurances, or (4) \emph{risk acceptance}, in case of low impact and attack feasibility~\cite{21434}.

\begin{table}[h]
\renewcommand{\arraystretch}{1.2}
\caption{Risk matrix from ISO/SAE 21434~\cite{21434}.}
\label{tab: risk 21434}
\noindent\makebox[\linewidth]{%
\begin{tabular}[t]{>{\raggedright}p{0.12\linewidth}>{\centering}p{0.12\linewidth}>{\centering}p{0.12\linewidth}>{\centering}p{0.12\linewidth}>{\centering\arraybackslash}p{0.06\linewidth}}
\toprule
\multirow{2}[3]{*}{Impact} & \multicolumn{4}{c}{Attack Feasibility}
\tabularnewline
\cmidrule(lr){2-5}
 & Very Low & Low & Medium & High
\tabularnewline
\midrule
Negligible & 1 & 1 & 1 & 1
\tabularnewline
Moderate & 1 & 2 & 2 & 3
\tabularnewline
Major & 1 & 2 & 3 & 4
\tabularnewline
Severe & 2 & 3 & 4 & 5
\tabularnewline
\bottomrule
\end{tabular}}
\end{table}

\subsection{Applicability of Risk Assessment Graphs to ISO/SAE 21434}\label{sec: 21434 attack graphs}

% Figure environment removed

\cref{fig: risk graph 21434} shows an example of a Risk Assessment Graph using ISO/SAE 21434.
The risk assessment process is carried out as follows. First (\emph{asset identification}), in this case, the \emph{item} is a UNIX system, and the \emph{asset} is the administrator privileges. 
Second (\emph{threat scenario identification}), the threat that may compromise the administrator privileges is represented by the topmost Inner Node (i.e., \enquote{Obtain Admin. Privileges}), which leads to two \emph{damage scenarios} (i.e., negative consequences) represented by Consequence Nodes (i.e., \enquote{Data Leakage} and \enquote{Denial of Rightful Access to the System}). 
Third (\emph{impact rating}), the edges relating Consequence Nodes to the topmost Inner Node are assigned with an Impact attribute. 
Here, if the \enquote{Obtain Admin. Privileges} threat is eventually realized, the impact of \enquote{Data Leakage} is \emph{major}, and the impact of \enquote{Denial of Rightful Access to the System} is \emph{moderate}.
Fourth (\emph{attack path analysis}), attack paths are identified through the refinements of Inner Nodes. 
That is, the topmost Inner Node (i.e., \enquote{Obtain Admin. Privileges}) represents the high-level threat (i.e., the main goal of the attack), which is refined into low-level threats (i.e., sub-goals of the attack), represented by successor Inner Nodes until the lowermost Inner Nodes (green nodes) ultimately represent the least significant threats (i.e., elementary attacks).
Hence, all Inner Nodes of the same attack path need to be fulfilled for the \enquote{Obtain Admin. Privileges} threat to be realized (i.e., the main goal of the attack to be achieved). 
Fifth (\emph{attack feasibility rating}), every lowermost Inner Node (green nodes) is assigned with \emph{Elapsed Time} (sandglass), \emph{Specialist Expertise} (star), \emph{Knowledge of the Item (or Component)} (light bulb), \emph{Window of Opportunity} (calendar), and \emph{Equipment} (hammer) attributes, whose value ranges are described in~\cref{tab: attack potential}. 
Then, the Attack Feasibility of every lowermost Inner Node threat is assigned using Table~\ref{tab: attack feasibility matrix 21434}, and predecessor Inner Nodes obtain the attributes from their successor Inner Nodes either conjunctively or disjunctively, as described in Section~\ref{sec: attack graph risk estimation}.


For inner nodes, no specific values are shown as for the risk calculation only the attack feasibility is necessary and if multiple successor nodes have the same attack feasibility, the most critical path is the one which has the lowest sum of attribute values.
Furthermore, the conjunctive calculation is an addition of the values of the successor nodes, resulting in a value of $22$ which according to~\cref{tab: attack feasibility matrix 21434} results in \emph{Low} for the attack feasibility. 
Sixth (\emph{risk determination}), the Impact attribute and the Attack Feasibility of the topmost Inner Node are used to determine the risk of Consequence Nodes. 
Here, using the risk matrix shown in Table~\ref{tab: risk 21434}, the Risk Assessment Graph shows a risk of 4 for \enquote{Data Leakage} and a risk of 3 for \enquote{Denial of Rightful Access to the System}. 
Finally (\emph{risk treatment decision}), the options to \emph{avoid}, \emph{reduce}, or \emph{share/transfer} the identified risks are represented by Countermeasure Nodes. 
Conversely, if the Impact attribute and the Attack Feasibility of the topmost Inner Node are low, and therefore the identified risks are \emph{accepted}, Countermeasure Nodes do not need to be added to the Risk Assessment Graph.


% The Attack Graph shows that the highest risk results from both \enquote{Corrupt Operator} and \enquote{Corrupt Sys. Admin}.


\subsection{Risk Assessment According to CLC/TS 50701}\label{sec: 50701 risk assessment}



Similar to ISO/SAE 21434, CLC/TS 50701 is also describing an \emph{asset-based} risk assessment. 
In this regard, the first step in the risk assessment process according to CLC/TS 50701 is the \emph{system definition}, where the purpose, scope, operational environment, and applicable security standards of the \emph{system under consideration (SUC)} are specified. 
In addition, the valuable objects (i.e., assets) supporting the essential functions of the SuC are identified. 
Assets are classified into \emph{information technology (IT) assets}, whose compromise may lead to business consequences (e.g., loss of revenue), and \emph{operational technology (OT) assets}, whose compromise may lead to physical consequences (e.g., sustained service outages). 
The second step is the \emph{threat landscape}, where a list of potential negative actions or events (i.e., threats) capable of jeopardizing the assets of the SUC is established and maintained.
% Figure environment removed
Here, threats are classified into internal (i.e., arising from within the system) and external (i.e., arising from without the system), and the skills and motivations driving the threats to exploit the vulnerabilities of the SUC and affect the assets of the SUC are documented.
The third step is the \emph{impact assessment}, where the negative consequences, in terms of \emph{confidentiality, integrity, and availability (CIA)}, imposed on assets are evaluated. 
The CIA impact is assessed qualitatively and ranges from A (highest impact) to D (lowest impact). 
The fourth step is the \emph{likelihood assessment}, where the attack surfaces (i.e., exposure) and the level of expertise and/or resources required to exploit the flaws (i.e., vulnerabilities) of the SUC are evaluated. 
The value of both exposure and vulnerability ranges from 1 (highly restricted logical or physical access for the attacker, vulnerability can only be exploited with high effort) to 3 (easy logical or physical access for the attacker, vulnerability can be exploited with low effort). 
In addition, the likelihood function is $L = EXP + VUL - 1$, and therefore the likelihood of attacks ranges from 1 (highly unlikely) to 5 (highly likely). 
The final step in the risk assessment process, according to CLC/TS 50701, is the \emph{risk evaluation}, where the risk of the documented threats being realized is determined usually by translating the threat landscape into a risk matrix, in which the assessed impact and likelihood of documented threats are related. 
Table~\ref{tab: risk 50701} shows an example of such a risk matrix.


\begin{table*}[h]
\renewcommand{\arraystretch}{1.2}
\caption{Risk matrix from CLC/TS 50701~\cite{50701}.}
\label{tab: risk 50701}
\noindent\makebox[\textwidth]{%
\begin{tabular}[t]{>{\raggedright}p{0.10\linewidth}>{\centering}p{0.12\linewidth}>{\centering}p{0.12\linewidth}>{\centering}p{0.12\linewidth}>{\centering}p{0.12\linewidth}>{\centering\arraybackslash}p{0.12\linewidth}}
\toprule
\multirow{2}[3]{*}{Impact} & \multicolumn{5}{c}{Likelihood}
\tabularnewline
\cmidrule(lr){2-6}
 & 1 & 2 & 3 & 4 & 5
\tabularnewline
\midrule
D & Low & Low & Low & Medium & Significant
\tabularnewline
C & Low & Low & Medium & Significant & High
\tabularnewline
B & Low & Medium & Significant & High & High
\tabularnewline
A & Medium & Significant & High & High & Very High
\tabularnewline
\bottomrule
\end{tabular}}
\end{table*}

\subsection{Applicability of Risk Assessment Graphs to CLC/TS 50701}\label{sec: 50701 attack graphs}



The applicability is demonstrated through the example of a Risk Assessment Graph using CLC/TS 50701, shown in~\cref{fig: risk graph 50701}.
The risk assessment process is carried out as follows. 
First (\emph{system definition}), in this case, the \emph{SUC} is a Linux system. Second (\emph{threat landscape}), the threats (that may jeopardize the assets of the Linux system) are represented by Inner Nodes. 
Similar to ISO/SAE 21434, the topmost Inner Node represents the most significant threat, successor Inner Nodes represent less significant threats, and the lowermost Inner Nodes (green nodes) ultimately represent the least significant threats. 
Hence, the topmost Inner Node is fulfilled when all Inner Nodes of the same attack path are fulfilled. Third (\emph{impact assessment}), the negative consequences are represented by Consequence Nodes (i.e., \enquote{Data Leakage} and \enquote{Denial of Rightful Access to the System}). 
In addition, the edges relating Consequence Nodes to the topmost Inner Node are assigned with an Impact attribute, whose value ranges from A to D. Here, if the \enquote{Obtain Admin. Privileges} threat is eventually realized, the impact of \enquote{Data Leakage} is A, and the impact of \enquote{Denial of Rightful Access to the System} is C. 
Fourth (\emph{likelihood assessment}), every lowermost Inner Node is assigned the Exposure (lock) and Vulnerability (shield) attributes, whose value range from 1 to 3, and the likelihood (red bubble) of every lowermost Inner Node threat is computed using the function $L = EXP + VUL - 1$. 
Then, predecessor Inner Nodes obtain the Exposure and Vulnerability attributes from their successor Inner Nodes either conjunctively or disjunctively, as described in Section~\ref{sec: attack graph risk estimation}.
Finally (\emph{risk evaluation}), the Impact attribute and the likelihood of the topmost Inner Node are used to determine the risk of Consequence Nodes. 
Here, using the risk matrix shown in Table~\ref{tab: risk 50701}, the Risk Assessment Graph shows a high (H) risk of \enquote{Data Leakage} and a significant (S) risk of \enquote{Denial of Rightful Access to the System}.

% The Attack Graph shows that the highest risk results from \enquote{Corrupt Operator}.



%\subsection{Summary of Remarks}\label{sec5: summary of remarks}

%This sections discusses the applicability of Risk Assessment Graphs to ISO/SAE 21434~\cite{21434} and CLC/TS 50701~\cite{50701}, which specify engineering requirements for cybersecurity risk management in the automotive and railway environments, respectively. 
%The Risk Assessment Graphs were created using diagrams.net and the open source Attack Graphs plugin\footnote{\label{foot:Plugin}https://incyde-gmbh.github.io/drawio-plugin-attackgraphs/}.