% !TEX root = ../AttackGraphBasedRiskAnalysis.tex
% !TEX spellcheck = en_US
% !TEX encoding = UTF-8 Unicode

\section{Evaluation}\label{sec: evaluation}



We identified seven properties that are necessary for a graphical risk assessment process: 
(1) \emph{Attack vectors} to represent the threat landscape.
(2) \emph{Directed acyclic graph} structure due to interdependencies.
(3) \emph{Node attributes} to evaluate the danger that originates from threats.
(4) \emph{Dynamic connectors} due to the complexity and dependencies of attacks/attack steps.
(5) \emph{Edge attributes} for attributes dependent on node connections, like the impact.
(6) \emph{Countermeasure nodes} to represent the existing system and to plan against attacks.
(7) \emph{Consequence nodes} represent what a successful attack impacts.
Section~\ref{sec: related work} discusses that none of the graphical security modeling frameworks from the current literature fulfills all of these seven properties for graphical risk assessment. 
The eight in more detail discussed representatives are again shown in Table~\ref{tab: summary of frameworks}. 
The most relevant framework is the Boolean Logic Driven Markov Processes. 
However, even this one falls short when it gets to attributes, especially edge attributes, and it is not possible to represent consequences either.
Hence, we defined \emph{Risk Assessment Graphs} to fulfill all seven properties. 

\begin{table*}[h]
\rowcolors{2}{gray!10}{gray!40}
\renewcommand{\arraystretch}{1.2}
\caption{Risk Assessment Graphs fulfill all seven identified properties, whereas graphical security modeling frameworks from current literature fulfill at most five properties.}
\label{tab: summary of frameworks}
\noindent\makebox[\textwidth]{%
\begin{tabular}[t]{>{\raggedright}p{0.15\textwidth}>{\raggedright}p{0.06\textwidth}>{\raggedright}p{0.08\textwidth}>{\raggedright}p{0.1\textwidth}>{\raggedright}p{0.09\textwidth}>{\raggedright\arraybackslash}p{0.08\textwidth}>{\raggedright\arraybackslash}p{0.15\textwidth}>{\raggedright\arraybackslash}p{0.1\textwidth}}
\toprule
 & Attack Vectors & DAG Structure & Node Attributes & Dynamic Connectors & Edge Attributes & Countermeasure Nodes & Consequence Nodes
\tabularnewline
\midrule
Attack Trees & \checkmark & - & (\checkmark) & - & - & - & -
\tabularnewline
OWA Trees & \checkmark & - & - & (\checkmark)  & (\checkmark) & - & -
\tabularnewline
Cryptographic DAGs & \checkmark & \checkmark & - & - & - & - & -
\tabularnewline
Bayesian Networks for Security & \checkmark & \checkmark & \checkmark & - & \checkmark & - & -
\tabularnewline
Bayesian Attack Graphs & \checkmark & \checkmark & \checkmark & - & \checkmark & - & -
\tabularnewline
Security Activity Graphs & \checkmark & \checkmark & \checkmark & - & - & \checkmark & -
\tabularnewline
Countermeasure Graphs & \checkmark & \checkmark & \checkmark & - & - & \checkmark & -
\tabularnewline
Boolean Logic Driven Markov Processes & \checkmark & \checkmark & - & (\checkmark) & \checkmark & \checkmark & -
\tabularnewline
Cyber Security Modeling Language & \checkmark & \checkmark & (\checkmark) & - & (\checkmark) & \checkmark & -
\tabularnewline
Risk Assessment Graphs & \checkmark & \checkmark & \checkmark & \checkmark & \checkmark & \checkmark & \checkmark
\tabularnewline
\bottomrule
\end{tabular}}
\end{table*}

Furthermore, we evaluated the \emph{Risk Assessment Graph} method in the project \enquote{Forecast of security requirements and evaluation of possible security concepts for the railway system} provided by the German Center for Rail Traffic Research (DZSF) at the Federal Railway Authority (EBA). 
The project aimed to identify research and standardization demands for security measures of future railway systems. 
In order to achieve this goal, 21 use cases were created, describing how future technologies can be utilized as well as their connections, data flows, and benefits. 
These use cases built the basis for a further security analysis, which was done using Attack Graphs.
First, an Attack Graph for each use case was created to identify the threats and vulnerabilities.
Then, a risk analysis was performed using the method described in~\cref{sec: attack graph risk estimation}.
Next, security measures to mitigate the threats and vulnerabilities were researched and attached to the Risk Assessment Graphs until the risk reached the lowest level.
The security measures were then further investigated to determine the technological level and applicability to the specific railway system as well as their coverage by standards.
So we also wanted to know if the standards that would most likely be applicable for these use cases already require the necessary countermeasures.

During this process, 21 Risk Assessment Graphs with more than 900 Nodes in total (including the Countermeasure Nodes) were created, with the biggest graph containing more than 70 nodes.
It is worth mentioning that if paths were the exact same for different use cases, the attack path was at least once completely evaluated and depicted, whereas it was in other use cases shortened and refers to the other graph.
Furthermore, countermeasures could be used for different attack paths, which resulted in two options for depicting them.
The countermeasure node could have multiple relations to attack paths or the countermeasure node appears multiple times in the same graph.
This is dependent on the specific graph and was decided regarding readability issues.
Additionally, some countermeasure nodes do not contain a single countermeasure, as only the combination of countermeasures would be sufficient to ensure security and can not be evaluated as a stand-alone measure.

Therefore, Risk Assessment Graphs also scale for bigger graphs and for multiple use cases. 
The Risk Assessment Graphs were created using a plugin tool for diagrams.net, which is publicly available on GitHub~\footnote{\label{foot:Plugin}https://incyde-gmbh.github.io/drawio-plugin-attackgraphs/}. 
The Attack Graphs are also publicly available~\footnote{\label{foot: attackgraphs}https://github.com/INCYDE-GmbH/attackgraphs}, and the final versions that include the risk assessment and the countermeasures will also be uploaded publicly in the near future.

Regarding usability, we found the Attack Graphs to be much clearer than alternatives like tables.
Especially using the tool helps as it is also implemented to highlight parent nodes and the attack path with the highest risk.
This makes it easy to follow which path needs to be mitigated to reduce the risk level.
However, countermeasures can defend against multiple attacks, so for bigger graphs duplicating the node is sometimes better than having multiple outgoing edges that cross over the complete graph for better visibility.

