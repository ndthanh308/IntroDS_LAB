% !TEX root = ../AttackGraphBasedRiskAnalysis.tex
% !TEX spellcheck = en_US
% !TEX encoding = UTF-8 Unicode

\section{Conclusion}\label{sec: conclusion}

% The contributions of Attack Graphs are ...
This paper presented that the entire process of risk assessment can be performed using a graphical solution, enhancing clarity for all stakeholders and adaptability regarding changed conditions.
We showed that our method enables the replication of the risk assessment process visually and is also compatible with existing risk assessment standards, like the ISO/SAE 21434 or the CLC/TS 50701.
We further verified this method in our project \enquote{Forecast of security requirements and evaluation of possible security concepts for the railway system}, financed by the German Centre of Rail Traffic Research\footnote{https://www.dzsf.bund.de}, by creating 21 graphs of varying sizes in the domain of railway systems.
The Attack Graphs\textsuperscript{\ref{foot: attackgraphs}} are publicly available, as well as the developed open-source plugin\textsuperscript{\ref{foot:Plugin}} for diagrams.net\footnote{https://app.diagrams.net/}, which was used to implement the described method.
Using our method enables a continuous evaluation and reevaluation of the risk landscape, including changing systems, countermeasures, and attacker capabilities.

We identified seven properties necessary to utilize attack graphs for risk assessment: (1) \emph{attack vectors}, (2) \emph{directed acyclic graph structure}, (3) \emph{node attributes}, (4) \emph{dynamic connectors}, (5) \emph{edge attributes}, (6) \emph{countermeasure nodes}, and(7) \emph{consequence nodes}.
These properties were combined to develop the methodology for \emph{Risk Assessment Graphs}, as no existing graphical method combined them.
Risk assessment graphs were further evaluated in the project \enquote{Forecast of security requirements and evaluation of possible security concepts for the railway system} provided by the German Center for Rail Traffic Research (DZSF) at the Federal Railway Authority (EBA).
There, we created 21 graphs of various sizes for future railway systems, using the risk assessment process to identify the need for further research and standardization of countermeasures.

A limitation of a graphical solution is that there is no direct way to include detailed explanations.
Furthermore, especially for bigger graphs, the structure is not as organized as there are a lot of connections.
Resulting in at least some work effort to rearrange the nodes and edges or to double some nodes.
Table-based solutions are usually able to cross-reference data, which should also be able for graph-based solutions, but is not that intuitive and at least in our developed tool not been implemented yet, neither within one graph nor between multiple graphs.

As we move forward, we recognize the potential for further advancements in the field of risk assessment using attack graphs. 
Future research may focus on incorporating real-time threat intelligence feeds and exploring the integration of automation for continuous risk monitoring.

In conclusion, our graphical solution using Risk Assessment Graphs represents a significant step toward a robust and efficient risk assessment.
By harnessing the power of visual representations, we aim to strengthen cybersecurity defenses, ultimately ensuring the resilience and trustworthiness of modern computer systems in the face of evolving threats.