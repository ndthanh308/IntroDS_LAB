% !TEX root = ../AttackGraphBasedRiskAnalysis.tex
% !TEX spellcheck = en_US
% !TEX encoding = UTF-8 Unicode

\section{Related Work}\label{sec: related work}
%\todo{Discuss all frameworks regarding consequences.}

Kordy et al.~\cite{DAGpaper} categorize thirty-three frameworks for graphical analysis of attack and defense scenarios into (1) \emph{attack and/or defense modeling}, which focus on the formal aspects of attacks or defenses, and (2) \emph{static or sequential modeling}, which focus on the temporal aspects or dependencies between actions. 
Using the same categorization, this section provides an overview of all the frameworks, and it describes these frameworks that fulfill the majority of properties incorporated in the framework of this article.

By reviewing frameworks from current literature, we identify seven properties for graphically modeling and managing an entire risk landscape.
The first property is \emph{attack vectors}, which enables the relations (shown as edges) between attack steps (shown as nodes) and, therefore, the formation of attack paths (i.e., attack vectors). 
The second property is the \emph{directed acyclic graph (DAG) structure}, thereby enabling linear (i.e., directed) and finite (i.e., acyclic) series of attack steps towards multiple potential attack goals (i.e., graph). 
The third property is \emph{node attributes}, which enables the quantification and, therefore, the evaluation of attack steps. 
The fourth property is \emph{dynamic connectors}, thereby enabling extensive attack refinements (besides the basic AND-OR refinements). 
The fifth property is \emph{edge attributes}, which enables the quantification and, therefore, the evaluation of relations between attack steps. 
The sixth property is \emph{countermeasure nodes}, thereby enabling actions to reduce the negative consequences of attacks.
The final property is \emph{consequence nodes}, enabling the presentation of consequences of successful attacks, which is also necessary to constitute the impact.

\begin{table*}[h]
\rowcolors{2}{gray!10}{gray!40}
\renewcommand{\arraystretch}{1.2}
\caption{Static attack modeling frameworks compared to the seven defined properties.}
\label{tab: static attack modeling}
\noindent\makebox[\textwidth]{%
\begin{tabular}[t]{>{\raggedright}p{0.15\textwidth}>{\raggedright}p{0.06\textwidth}>{\raggedright}p{0.08\textwidth}>{\raggedright}p{0.1\textwidth}>{\raggedright}p{0.09\textwidth}>{\raggedright\arraybackslash}p{0.08\textwidth}>{\raggedright\arraybackslash}p{0.15\textwidth}>{\raggedright\arraybackslash}p{0.1\textwidth}}
\toprule
 & Attack Vectors & DAG Structure & Node Attributes & Dynamic Connectors & Edge Attributes & Countermeasure Nodes & Consequence Nodes
\tabularnewline
\midrule
% \textbf{Static Attack Modeling} & & & & &
% \tabularnewline
Attack Trees & \checkmark & - & (\checkmark) & - & - & - & -
\tabularnewline
Augmented Vulnerability Trees & \checkmark & - & (\checkmark) & - & - & -  & -
\tabularnewline
Augmented Attack Trees & \checkmark & - & (\checkmark) & - & - & -  & -
\tabularnewline
OWA Trees & \checkmark & - & - & (\checkmark)  & (\checkmark) & -  & -
\tabularnewline
Parallel Model for Multi-Parameter Attack Trees & \checkmark & - & (\checkmark) & - & - & -  & -
\tabularnewline
Extended Fault Trees & \checkmark & - & (\checkmark) & - & - & -  & -
\tabularnewline
\bottomrule
\end{tabular}}
\end{table*}

Each one of the thirty-three frameworks presented in this section considers only subsets of the seven identified properties. 
None of these frameworks are suitable, as all seven properties are necessary to perform a full risk assessment.
To overcome this limitation, this article incorporates all seven identified properties into a framework for a graphical solution for performing risk analysis and examines its applicability to different risk analysis standards.

\subsection{Static Attack Modeling}\label{sec: static attack modeling}




Six frameworks for \emph{static attack modeling}, namely \emph{Attack Trees}~\cite{weiss1991}, \emph{Augmented Vulnerability Trees}~\cite{AugmentedVulnerabilityTrees}, \emph{Augmented Attack Trees}~\cite{AugmentedAttackTrees}, \emph{OWA Trees}~\cite{Yager2006OWATA}, \emph{Parallel Model for Multi-Parameter Attack Trees}~\cite{ParallelModelForMultiParameterAttackTrees}, and \emph{Extended Fault Trees}~\cite{ExtendedFaultTrees}, are summarised in Table~\ref{tab: static attack modeling}.
All frameworks fulfill the attack vectors property, but none of them supports the DAG structure, countermeasure nodes, and consequence nodes properties. 
Of the six frameworks, OWA trees stand out as they at least partially fulfill the dynamic connectors and edge attributes properties, despite being the only framework that does not fulfill the node attributes property. 
This section describes attack trees, which was the first graphical security modeling framework, and OWA trees, which is the framework that at least partially fulfills most of the seven identified properties.

\subsubsection{Attack Trees}\label{sec: attack trees}

The first \emph{tree-based approach}, shown as an AND-OR tree structure for graphical security modeling, was the \emph{threat logic trees}, which was introduced by Weiss in 1991~\cite{weiss1991}.
Today, all AND-OR tree structures are referred to as \emph{attack trees}, a term first introduced by Salter et al. in 1998~\cite{Salter1998}.

In attack trees, the root node (i.e., the tree's root) indicates the attack's main goal. 
The main goal is then conjunctively (AND) or disjunctively (OR) refined into sub-goals until they represent basic actions corresponding to atomic components that can be easily understood and quantified. 
Conjunctive refinements indicate that \emph{all} sub-goals need to be fulfilled in order to achieve the main goal, whereas disjunctive refinements indicate that \emph{at least one} sub-goal needs to be fulfilled for achieving the main goal~\cite{weiss1991}.

\subsubsection{OWA Trees}\label{sec: owa trees}

\emph{Ordered weighted averaging (OWA) trees} were proposed by Yager in 2005 to include the concept of \emph{uncertainty} into attack trees~\cite{Yager2006OWATA}. 
This was made possible by replacing the AND-OR nodes with OWA nodes (i.e., quantifiers, such as \emph{most}, \emph{some}, \emph{half of}, etc.) and therefore taking into consideration situations where the number of sub-goals that need to be fulfilled in order to achieve the main goal remains unknown. 
Finally, OWA trees allow for the evaluation of success probability and cost attributes, which can be jointly used to calculate the cheapest and most probable attack.

\subsection{Sequential Attack Modeling}\label{sec: sequential attack modeling}

\begin{table*}[h]
\rowcolors{2}{gray!10}{gray!40}
\renewcommand{\arraystretch}{1.2}
\caption{Sequential attack modeling frameworks compared to the seven defined properties.}
\label{tab: sequential attack modeling}
\noindent\makebox[\textwidth]{%
\begin{tabular}[t]{>{\raggedright}p{0.15\textwidth}>{\raggedright}p{0.06\textwidth}>{\raggedright}p{0.08\textwidth}>{\raggedright}p{0.1\textwidth}>{\raggedright}p{0.09\textwidth}>{\raggedright\arraybackslash}p{0.08\textwidth}>{\raggedright\arraybackslash}p{0.15\textwidth}>{\raggedright\arraybackslash}p{0.1\textwidth}}
\toprule
 & Attack Vectors & DAG Structure & Node Attributes & Dynamic Connectors & Edge Attributes & Countermeasure Nodes & Consequence Nodes
\tabularnewline
\midrule
% \textbf{Sequential Attack Modeling} & & & & &
% \tabularnewline
Cryptographic DAGs & \checkmark & \checkmark & - & - & - & - & - 
\tabularnewline
Fault Trees for Security & \checkmark & - & \checkmark & (\checkmark) & - & -  & -
\tabularnewline
Bayesian Networks for Security & \checkmark & \checkmark & \checkmark & - & \checkmark & - & -
\tabularnewline
Bayesian Attack Graphs & \checkmark & \checkmark & \checkmark & - & \checkmark & - & -
\tabularnewline
Compromise Graphs & \checkmark & \checkmark & - & - & (\checkmark) & - & -
\tabularnewline
Enhanced Attack Trees & \checkmark & - & \checkmark & - & (\checkmark) & - & -
\tabularnewline
Vulnerability Cause Graphs & (\checkmark) & \checkmark & - & - & - & - & -
\tabularnewline
Dynamic Fault Trees for Security & \checkmark & - & (\checkmark) & - & - & - & -
\tabularnewline
Serial Model for Multi-Parameter Attack Trees & \checkmark & - & (\checkmark) & - & - & - & -
\tabularnewline
Improved Attack Trees & \checkmark & - & (\checkmark) & - & - & - & -
\tabularnewline
Time-dependent Attack Trees & \checkmark & \checkmark & (\checkmark) & - & - & - & -
\tabularnewline
\bottomrule
\end{tabular}}
\end{table*}

Eleven frameworks for \emph{sequential attack modeling}, namely \emph{Cryptographic DAGs}~\cite{Meadows1996ARO}, \emph{Fault Trees for Security}~\cite{FaultTreesForSecurity}, \emph{Bayesian Networks for Security}~\cite{BayesianNetworksForSecurity}, \emph{Bayesian Attack Graphs}~\cite{BayesianAttackGraphs}, \emph{Compromise Graphs}~\cite{CompromiseGraphs}, \emph{Enhanced Attack Trees}~\cite{EnhancedAttackTrees}, \emph{Vulnerability Cause Graphs}~\cite{VulnerabilityCauseGraphs}, \emph{Dynamic Fault Trees for Security}~\cite{DynamicFaultTreesForSecurity}, \emph{Serial Model for Multi-Parameter Attack Trees}~\cite{SerilModelForMultiParameterAttackTrees}, \emph{Improved Attack Trees}~\cite{ImprovedAttackTrees}, and \emph{Time-dependent Attack Trees}~\cite{TimeDependentAttackTrees}, are summarised in Table~\ref{tab: sequential attack modeling}. 
Again, none of the frameworks fulfills the countermeasure and consequence nodes property. In addition, only Fault Trees for Security offer a wide range of dynamic connectors, and only Bayesian-based models fulfill the edge attributes property. 
Finally, Compromise Graphs and Enhanced Attack Trees are two frameworks that at least partially fulfill the edge attributes property, and Vulnerability Cause Graphs are the only framework that only partially fulfills the attack vectors property. 
This section describes Cryptographic DAGs, which was the first graph-based approach for security modeling, and Bayesian Attack Graphs, which combine attack trees and Bayesian networks and also fulfill four of the seven identified properties.

\subsubsection{Cryptographic DAGs}\label{sec: cryptographic dags}

\emph{Cryptographic directed acyclic graphs} were proposed by Meadows~\cite{Meadows1996ARO} in 1996 to provide a \emph{novel} simple representation of sequences and dependencies of attack steps towards the main goal of the attack. 
Instead of a tree-based approach, Cryptographic DAGs introduced a \emph{graph-based approach} for security modeling. 
However, they eventually do not offer the possibility to perform risk assessment as other properties are still not fulfilled.

\subsubsection{Bayesian Networks and Bayesian Attack Graphs}\label{sec: bayesian Networks and Attack Graphs}

For the last couple of decades, researchers have been focusing on \emph{Bayesian networks} for the purposes of security modeling.
The origin of Bayesian networks, which are also known as \emph{belief} or \emph{causal networks}, lies in artificial intelligence.
In Bayesian networks, nodes represent events or objects and are associated with probabilistic variables. 
Hence, analyzing the uncertainty of events is also possible. 
Bayesian networks follow a DAG structure, where the directed edges represent the causal dependencies between the nodes~\cite{BayesianAttackGraphs}.

\emph{Bayesian attack graphs} are a fusion of (general) attack trees and (computational procedures) of Bayesian networks, and they were first introduced by Liu and Man in 2005 to analyze network vulnerability scenarios~\cite{BayesianAttackGraphs}. 
Subsequently, calculating general security metrics regarding information system networks~\cite{Frigault2008, Noel2010} and capturing dynamic behavior~\cite{Frigault2008Dyn} was also made possible.

Finally, although Bayesian attack graphs allow for assigning values to nodes and for performing computations using the graphs, they do not allow for a dynamic selection of connectors and for including countermeasures. 
As a result, Bayesian attack graphs cannot be used to perform risk assessment.

\subsection{Static Attack and Defense Modeling}\label{sec: static attack and defense modeling}

\begin{table*}[h]
\rowcolors{2}{gray!10}{gray!40}
\renewcommand{\arraystretch}{1.2}
\caption{Static attack and defense modeling frameworks compared to the seven defined properties.}
\label{tab: static attack and defense modeling}
\noindent\makebox[\textwidth]{%
\begin{tabular}[t]{>{\raggedright}p{0.15\textwidth}>{\raggedright}p{0.06\textwidth}>{\raggedright}p{0.08\textwidth}>{\raggedright}p{0.1\textwidth}>{\raggedright}p{0.09\textwidth}>{\raggedright\arraybackslash}p{0.08\textwidth}>{\raggedright\arraybackslash}p{0.15\textwidth}>{\raggedright\arraybackslash}p{0.1\textwidth}}
\toprule
 & Attack Vectors & DAG Structure & Node Attributes & Dynamic Connectors & Edge Attributes & Countermeasure Nodes & Consequence Nodes
\tabularnewline
\midrule
% \textbf{Static Attack and Defense Modeling}
% \tabularnewline
Anti-Models & \checkmark & - & - & - & - & \checkmark & -
\tabularnewline
Defense Trees & \checkmark & - & (\checkmark) & - & - & \checkmark & -
\tabularnewline
Protection Trees & - & - & \checkmark & - & - & \checkmark & -
\tabularnewline
Security Activity Graphs & \checkmark & \checkmark & (\checkmark) & - & - & \checkmark & -
\tabularnewline
Attack Countermeasure Trees & \checkmark & - & \checkmark & - & - & \checkmark & -
\tabularnewline
Attack-Defense Trees & \checkmark & - & \checkmark & - & - & \checkmark & -
\tabularnewline
Countermeasure Graphs & \checkmark & \checkmark & \checkmark & - & - & \checkmark & -
\tabularnewline
\bottomrule
\end{tabular}}
\end{table*}

Seven frameworks for \emph{static attack and defense modeling}, namely \emph{Anti-Models}~\cite{AntiModels}, \emph{Defense Trees}~\cite{DefenseTrees}, \emph{Protection Trees}~\cite{ProtectionTrees}, \emph{Security Activity Graphs}~\cite{SecurityActivityGraphs}, \emph{Attack Countermeasure Trees}~\cite{AttackCountermeasureTrees}, \emph{Attack-Defense Trees}~\cite{AttackDefenseTrees}, and \emph{Countermeasure Graphs}~\cite{CountermeasureGraphs}, are summarised in Table~\ref{tab: static attack and defense modeling}. 
All frameworks fulfill the countermeasure nodes property, and only Protection Trees do not fulfill the attack vectors property. 
In addition, Anti-Models is the only framework that does not at least partially fulfill the node attributes property.
However, none of these frameworks considers consequence nodes in their design.
This section describes Security Activity Graphs and Countermeasure Graphs, which are the two frameworks that fulfill four of the seven identified properties.

\subsubsection{Security Activity Graphs}\label{sec: security activity graphs}

\emph{Security activity graphs (SAGs)} were developed by Ardi et al.~\cite{SecurityActivityGraphs} in 2006 to improve security throughout the software development process. 
SAGs are loosely based on fault trees, and the root of a SAG is associated with a vulnerability. 
Vulnerability mitigations are modeled using activities (i.e., leaf nodes), which are assigned boolean variables to indicate whether an activity \enquote{is implemented perfectly during software development} (true) or not (false). 
Finally, besides AND-OR gates, which follow a strictly Boolean logic, SAGs also include \emph{split gates}, which allow one activity to be used in several parent activities, thus creating a DAG structure.

However, SAGs lack the ability to represent the consequences of threats and edge attributes, and both are necessary to calculate a risk value.
Furthermore, there is only a limited option for connectors and node attributes.
Therefore, rendering SAGs impractical for risk assessment.

\subsubsection{Countermeasure Graphs}\label{sec: countermeasure graphs}

\emph{Countermeasure graphs} were introduced by Baca and Petersen~\cite{CountermeasureGraphs} in 2010 to simplify countermeasure selection through cumulative voting. 
Countermeasure graphs are created by identifying actors, goals, attacks, and countermeasures. Related events are connected with edges. 
That is, actors are connected to pursued goals and likely executable attacks, and countermeasures are connected to preventable attacks. 
Finally, actors, goals, attacks, and countermeasures are assigned priorities according to the rules of hierarchical cumulative voting. 
Higher assigned priorities imply higher threat levels of the corresponding events and vice versa. 
Using hierarchical cumulative voting, the most effective countermeasures can be identified.

Countermeasure Graphs provide a useful system overview, but the computational rules focus on finding the most effective countermeasure instead of the most likely and severe attack. 
This limitation could at least partially be addressed with the threat level. 
However, the threat level value is determined by the subjective assessment of the graph creator rather than by calculations over meaningful attributes, thereby raising issues of validity.

\subsection{Sequential Attack and Defense Modeling}\label{sec: sequential attack and defense modeling}

\begin{table*}[h]
\rowcolors{2}{gray!10}{gray!40}
\renewcommand{\arraystretch}{1.2}
\caption{Sequential attack and defense modeling frameworks compared to the seven defined properties.}
\label{tab: sequential attack and defense modeling}
\noindent\makebox[\textwidth]{%
\begin{tabular}[t]{>{\raggedright}p{0.15\textwidth}>{\raggedright}p{0.06\textwidth}>{\raggedright}p{0.08\textwidth}>{\raggedright}p{0.1\textwidth}>{\raggedright}p{0.09\textwidth}>{\raggedright\arraybackslash}p{0.08\textwidth}>{\raggedright\arraybackslash}p{0.15\textwidth}>{\raggedright\arraybackslash}p{0.1\textwidth}}
\toprule
 & Attack Vectors & DAG Structure & Node Attributes & Dynamic Connectors & Edge Attributes & Countermeasure Nodes & Consequence Nodes
\tabularnewline
\midrule
% \textbf{Sequential Attack and Defense Modeling} & & & & &
% \tabularnewline
Insecurity Flows & \checkmark & \checkmark & \checkmark & - & - & \checkmark & -
\tabularnewline
Intrusion DAGs & \checkmark & \checkmark & - & - & - & \checkmark & -
\tabularnewline
Bayesian Defense Graphs & \checkmark & \checkmark & (\checkmark) & - & - & \checkmark & -
\tabularnewline
Security Goal Indicator Trees & - & - & - & - & - & \checkmark & -
\tabularnewline
Attack Response Trees & \checkmark & -  & \checkmark & - & - & \checkmark & -
\tabularnewline
Boolean Logic Driven Markov Processes & \checkmark & \checkmark & (\checkmark) & (\checkmark) & - & \checkmark & -
\tabularnewline
Cyber Security Modeling Language & \checkmark & \checkmark & (\checkmark) & - & (\checkmark) & \checkmark & -
\tabularnewline
Security Goal Models & \checkmark & \checkmark & - & - & - & \checkmark & -
\tabularnewline
Unified Parameterizable Attack Trees & \checkmark & - & \checkmark & - & (\checkmark) & \checkmark & -
\tabularnewline
\bottomrule
\end{tabular}}
\end{table*}

Finally, nine frameworks for \emph{sequential attack and defense modeling}, namely \emph{Insecurity Flows}~\cite{InsecurityFlows}, \emph{Intrusion DAGs}~\cite{IntrusionDAGs}, \emph{Bayesian Defense Graphs}~\cite{BayesianDefenseGraphs}, \emph{Security Goal Indicator Trees}~\cite{SecurityGoalIndicatorTrees}, \emph{Attack Response Trees}~\cite{AttackResponseTrees}, \emph{Boolean Logic Driven Markov Process}~\cite{BooleanLogicDrivenMarkovProcess}, \emph{Cyber Security Modeling Language}~\cite{CyberSecurityModelingLanguage2010}, \emph{Security Goal Models}~\cite{SecurityGoalModels}, and \emph{Unified Parameterizable Attack Trees}~\cite{UnifiedParameterizableAttackTrees}, are summarized in Table~\ref{tab: sequential attack and defense modeling}. 
All frameworks fulfill the countermeasure nodes property, and only Security Goal Indicator Trees do not fulfill the attack vectors property. 
In addition, Boolean Logic Driven Markov Processes (BDMPs) is the only framework that offers a wide range of connectors and, therefore, at least partially fulfills the dynamic connectors property. 
This section describes BDMPs and Cyber Security Modeling Language (CySeMoL), which are the two frameworks that at least partially fulfill five of the seven identified properties.

\subsubsection{Boolean Logic Driven Markov Processes}\label{sec: boolean logic driven markov processes}

\emph{Boolean logic driven Markov processes (BDMPs)} are a security modeling framework, which can also be used to perform risk assessment~\cite{BooleanLogicDrivenMarkovProcess}. 
It was invented by Bouissou and Bon~\cite{BooleanLogicDrivenMarkovProcess} in 2003 for the safety and reliability domains, and it was later adapted to security modeling by Piètre-Cambacédès and Bouissou in 2010. 
BDMPs combine the readability of attack trees with the modeling power of Markov chains. 
The root (top event) of a BDMP represents the main goal of the attack, and the leaves represent the attack steps or security events.
BDMPs offer a wide range of node attributes, including time-domain metrics, such as mean-time to success, attack tree-related metrics, such as costs of attacks, boolean indicators, such as specific requirements, and risk assessment tools, such as sensibility graphs.

However, the lack of edge attributes, in addition to issues of usability with respect to leaf nodes and connectors~\cite{BDMPCritic}, render BDMPs impractical for risk assessment.

\subsubsection{Cyber Security Modeling Language}\label{sec: cyber security modeling language}

\emph{Cyber security modeling language (CySeMoL)} was developed by Sommestad et al. in 2010 to assess the cyber security of \emph{supervisory control and data acquisition (SCADA)} system architectures~\cite{CyberSecurityModelingLanguage2010, CyberSecurityModelingLanguage2013}.
Simply modeling the system architecture and the characteristics of the involved assets is sufficient, as CySeMoL already includes information about how attacks and defenses are quantitatively related. 
The attacker is assumed to be a professional penetration tester with a fixed time of one week to perform an attack.
CySeMoL was extended by Holm in 2014 and renamed to \emph{predictive, probabilistic cyber security modeling language ((P$^2$)CySeMoL)}, introducing more flexible and useful computations, the possibility to model assets, attacks, and defenses that are not necessarily SCADA-related, and the option to specify the time needed to perform an attack~\cite{PredictiveProbabilisticCyberSecurityModelingLanguage}.
Computations can be conducted automatically (i.e., without personalized inputs) as (P$^2$)CySeMoL already includes qualitative information gathered from literature reviews, empirical studies, as well as surveys involving domain experts~\cite{CyberSecurityModelingLanguage2010, CyberSecurityModelingLanguage2013, PredictiveProbabilisticCyberSecurityModelingLanguage}.

The results of the computations show the likelihood of an attack. 
However, the severity of an attack is not considered, and therefore the risk of an attack cannot be properly assessed. 
Furthermore, (P$^2$)CySeMoL does not include connectors, and therefore it seems an inconvenient tool for graphical risk assessment.

\subsection{Summary of Remarks}\label{sec2: summary of remarks}

This section provides an overview of thirty-three frameworks for analysis of attack and defense scenarios, and it describes eight of these frameworks in more detail. 
Thirty frameworks fulfill the attack vectors property, sixteen frameworks fulfill the countermeasure nodes property, and only thirteen frameworks fulfill the DAG structure property.
In addition, node/edge attributes and connectors are, in most cases, fixed and limited, thereby reducing the usability and usefulness of the frameworks with respect to the purposes of risk assessment. 
The complex nature and rapid development of (information) systems, attacks, and defenses motivate the need for proper risk management.
Existing methods are mainly consisting of tables with graphical solutions mostly utilized for support, if at all.
As shown in this section, current graphical solutions support threat or vulnerability management and sometimes even calculations to determine which attack vector might be the easiest to execute or, in other terms, which is most probable to occur.
The risk value cannot be equated with probability, though, and is usually determined using the probability of an event and its impact.
However, none of the methods described in this section can represent an event's consequences and impact, rendering them incapable of performing risk assessment.

