

%!TEX root = ../paper_main.tex

\section{Conclusions}


In this work, we have proposed Information Gained Subgroup Discovery (IGSD), a new SD algorithm for pattern discovery that combines Information Gain and Odds Ratio as a multi-criteria for pattern selection. Additionally, two versions of IGSD are proposed to evaluate the dynamic adjustment of the search optimization thresholds during subgroup space exploration. Also, main and general limitations of state-of-the-art SD algorithms are discussed, identifying the following ones: need for fine-tuning of key parameters for each dataset, usage of a single pattern search criteria set by hand, usage of non-overlapping data structures for subgroup space exploration, and impossibility to search for patterns by fixing some relevant dataset variables. The proposed IGSD algorithm tries to tackle all these limitations and thus is evaluated using up to eleven datasets with different characteristics to uncover patterns. For comparison purposes, the same datasets are also used with two state-of-the-art SD algorithms: FSSD and SSD++.

Results obtained showed that FSSD provides more complex patterns and SSD++ provides less complex patterns than IGSD. In turn, IGSD usually finds more larger patterns sets than FSSD and SSD++. Thus, it can be concluded that IGSD produces larger sets of patterns with less amount of information or variables for each returned pattern. On the other hand, FSSD and SSD++ confidence average values are 83\% and 63\%, respectively, significantly lower than IGSD confidence average values of around 90\%. This lower reliability of FSSD and SSD++ is also reflected in ORR average values providing 3.58 and 3, respectively, stating a medium-high dependence between patterns and targets. In turn, IGSD provided an ORR average value of around 4, stating a high dependence between patterns and targets. The fact that IGSD obtained better results than FSSD and SSD++ without manual setting of any search parameter also validates the proposed method.

In the performance evaluation of patterns obtained by compared algorithms for all datasets, we propose to complement standard SD measures and include some metrics: Information Gain, ORR and p\-value, not considered typically in SD literature. Also, results obtained for P4Lucat dataset have been validated by a group of experts. Thus, patterns acceptance rates show that results provided by IGSD, are more in agreement with the experts than results obtained using FSSD and SSD++ algorithms. For the P4Lucat dataset, better-accepted patterns also have higher ORR and confidence values while being statistically significant with a p\-value below 0.05. Hence, we consider that the inclusion of the proposed non-standard SD metrics allows to better evaluate discovered patterns. 

Finally, as mentioned above, the proposed IGSD algorithm uses sets of subgroups and follows a non-greedy pattern search strategy. This makes IGSD perform a wider exploration of the search space, allowing to obtain potentially more relevant patterns but at the cost of significantly longer computational times. As a future work, we plan to explore a similar strategy we adopted in a previous work \cite{CBMS23} for selecting statistically significant variables. Then, the set of variables in datasets with a large number of columns can be reduced, and search patterns based on this reduced set of significant variables.









