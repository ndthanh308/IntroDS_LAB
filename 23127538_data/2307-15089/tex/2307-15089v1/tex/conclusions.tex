

%!TEX root = ../paper_main.tex

\section{Conclusions}

% \todo[inline]{Discutir cruce de métricas de algoritmos y validación para este dataset. Para este dataset, estrategia de IG y OR parece buena. Hemos obtenido patrones más largos y con más información. La mayoría de patrones siguen las guías clínicas (por eso mostramos los que no la siguen). Hay patrones que no las siguen o no están aceptados --> Esto lo tienen que comentar los médicos.}
In this work, we have compared two algorithms presented in the state of the art related to SD, the FSSD and SSD++ algorithms. After understanding how these algorithms work and realizing some limitations and issues that these algorithms might have when used in our data, we propose a new SD algorithm, IGSD, based on the IG and OR measures, which tries to solve the limitations presented before. \\
The results provided by FSSD and SSD++ algorithms contain patterns with less information, hence, less complexity, than IGSD algorithm. It is remarkable that patterns from SSD++ algorithm present a very low amount of information. Thus, they have more representation in the dataset, as they have higher coverage. However, these patterns are less reliable due to the fewer confidence values reported. FSSD and SSD++ confidence values are 49\% and 31\%, respectively, against the patterns produced by the IGSD algorithm with a confidence value above 80\%. This low reliability is also reflected in ORR values, with FSSD having 2.84, which represents low dependence between the patterns and the targets, and SSD++, 1.69, which represents a very low dependence between the patterns and the targets, respectively, instead of the ORR reported by IGSD, which represents a medium-high dependence between patterns and targets. \\
In relation to the quality functions, the acceptance rate shows that the results provided by IGSD, are more in agreement with the experts than the results obtained using FSSD and SSD++ algorithms. Therefore, it validates that confidence and ORR values seem to indicate a higher acceptance rate in this dataset. This is also validated through a p-value below 0.05. Thus, IGSD is able to discover patterns with acceptable values in terms of quality functions, and in accordance with the experts preference, in this field of study. Moreover, this result is achieved because cancer stage and first treatment were fixed as clinicians stated these were key items in the patterns.

This work had several limitations, such as the use of only one dataset, which does not completely validate the algorithm performance in a general context; and the use of a small cohort of patients, which could not represent fairly the global population. Thus, in further studies, general validation is a goal to achieve, having ideally, larger datasets. 

% In this work we have used IGSD for finding treatment patterns of clinical interest relying on treatment outcomes in a dataset concerning lung cancer patients. Additionally, discovered patterns have been compared with clinical guidelines as a reference for
% evaluating results and for identifying new potential treatments. SD has proven to be useful for identifying, within our cohort of patients, treatment patterns of high clinical interest, taking treatment results as a reference. In addition, it allows evaluating the degree of adherence of said patterns with clinical guidelines and identifying new potential treatments.

% The comparison reveals that for stages I and II, surgery is the first prescribed treatment in agreement with clinical recommendations, since neoadjuvant CT before surgery is optional depending on tumor characteristics. These groups did not developed progression-relapses on the cancer status nor toxicities as a side effect. Several studies have demonstrated that curative-intent surgery, when coupled with regional lymph node examination, is generally associated with the best long-term overall survival in patients with early-stage NSCLC \cite{chi_comparison_2019}.

% Stage III NSLC is a heterogeneous and complex disease that could be classified into subgroups: resectable, potentially resectable and unresectable locally advanced NSCLC \cite{majem_seom_2019}. In patients with potentially resectable disease, the optimal treatment strategy remains unclear. Several phase III trials and a meta-analysis showed that induction therapy followed by surgery might be better than surgery alone \cite{burdett_chemotherapy_2007}.
% Apart from the controversy regarding the sequence of treatments, the results of our study provide a description of treatment patterns in an academic center and reveal that the treatment of stage III was in line with guidelines recommendations, where chemotherapy and surgery are prescribed as first treatments. Different progression-relapse and toxicity targets were found in these subgroups.

% Regarding stage IV, agreement with clinical guidelines is observed in most cases since chemotherapy, targeted oral therapy, and immunotherapy, are prescribed as initial treatments. A discrepancy, with high statistical relevance, is found in the dataset since curative surgery is found as a first treatment, while it is not present in clinical guidelines for stage IV. However, this discrepancy is probably related to oligometastatic disease (eg, single brain metastasis) or metastatic disease limited to the chest. Several studies have demonstrated that this subgroup may benefit from aggressive local therapy to both primary chest and metastasic sites \cite{ashworth_individual_2014}. Also, different progression-relapse and toxicity targets were found in these subgroups.

% Our study had several limitations. As a retrospective cohort, not all patients’ information had been provided. Probably, our model may benefit from being developed and validated with a larger cohort of patients. However, the results presented in this paper are a preliminary study of treatment patterns found in lung cancer patients based on the outcomes of the first prescribed treatments. In future work, we plan to extend patient and disease related variables to characterize and find treatment prescription patterns based on patients’ profiles.