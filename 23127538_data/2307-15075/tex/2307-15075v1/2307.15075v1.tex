\documentclass[12pt]{amsart}
\usepackage{}

\usepackage{amsmath}
\usepackage{amsfonts}
\usepackage{amssymb}
\usepackage[all]{xy}           %xypic macro for latex2.09
\usepackage{bbm}
\usepackage{bbding}
\usepackage{txfonts}
\usepackage{amscd}
\usepackage{tikz}
\usetikzlibrary{matrix}
\usepackage[shortlabels]{enumitem}
%\usepackage[xetex]{graphicx}

%\usepackage[xetex,colorlinks,final,backref=page,hyperindex]{hyperref}


\usepackage{ifpdf}
\ifpdf
\usepackage[colorlinks,final,backref=page,hyperindex]{hyperref}
\else
\usepackage[colorlinks,final,backref=page,hyperindex,hypertex]{hyperref}
\fi
\usepackage{tikz}
\usepackage[active]{srcltx}

%======================================================================
\renewcommand\baselinestretch{1}    %was    1, 1.5 for double sp
%======================================================================
%%standard setting
%\topmargin -0.3truein \textheight 8.4truein
%\oddsidemargin 0.2truein
%\evensidemargin 0.2truein \textwidth 440pt
%======================================================================
%%little larger standard setting: good setting
\topmargin 0cm \textheight 22cm \oddsidemargin 0cm \evensidemargin -0cm \textwidth 16.3cm
%========================================================================================%%wide
%%lower setting for 1920x1080
%%\topmargin -.9cm \textheight 21cm \oddsidemargin 0cm \evensidemargin -0cm \textwidth 16.3cm
%%%%%%%%%%%%%%

\makeatletter


\newtheorem{thm}{Theorem}[section]
\newtheorem{lem}[thm]{Lemma}
\newtheorem{cor}[thm]{Corollary}
\newtheorem{pro}[thm]{Proposition}
\newtheorem{ex}[thm]{Example}
\newtheorem{rmk}[thm]{Remark}
\newtheorem{defi}[thm]{Definition}


\setlength{\baselineskip}{1.8\baselineskip}
\newcommand {\comment}[1]{{\marginpar{*}\scriptsize\textbf{Comments:} #1}}
\newcommand {\emptycomment}[1]{}
\newcommand {\yh}[1]{{\marginpar{*}\scriptsize\textcolor{purple}{yh: #1}}}
\newcommand {\tr}[1]{{\marginpar{*}\scriptsize\textcolor{read}{tr: #1}}}
\newcommand {\nrn}[1]{   [    #1   ]^{\rm{3Lie}}}
\newcommand{\li}[1]{\textcolor{blue}{#1}}
\newcommand{\lir}[1]{\textcolor{blue}{\underline{Li:}#1 }}
\newcommand{\liu}[1]{\textcolor{blue}{ #1}}


\newcommand{\dr}{\delta^{\rm reg}}
\newcommand{\pom}{$\mathsf{PoiMod}$~}
\newcommand{\asm}{$\mathsf{ComMod}$~}
\newcommand{\liem}{$\mathsf{LieMod}$~}
\newcommand{\asbm}{$\mathsf{AssMod}$~}

\newcommand{\lon }{\,\rightarrow\,}
\newcommand{\be }{\begin{equation}}
\newcommand{\ee }{\end{equation}}

\newcommand{\defbe}{\triangleq}
%\newcommand{\pf}{\noindent{\bf Proof.}\ }

\newcommand{\g}{\mathfrak g}
\newcommand{\h}{\mathfrak h}

\newcommand{\HH}{\mathbb H}
\newcommand{\Real}{\mathbb R}
\newcommand{\Comp}{\mathbb C}
\newcommand{\Numb}{\mathbb N}
\newcommand{\huaB}{\mathcal{B}}%{{\mathcal{E}}}%{\mathcal{B}}
\newcommand{\huaS}{\mathcal{S}}
%\newcommand{\A}{\mathcal{A}}
\newcommand{\huaA}{\mathcal{A}}%{{\mathcal{F}}}%{\mathcal{A}}
\newcommand{\huaL}{\mathcal{L}}
\newcommand{\huaR}{\mathcal{R}}
\newcommand{\huaE}{\mathcal{E}}
\newcommand{\huaF}{\mathcal{F}}
\newcommand{\huaG}{\mathcal{G}}


\newcommand{\huaU}{\mathcal{U}}
\newcommand{\huaV}{\mathcal{V}}
\newcommand{\huaW}{\mathcal{W}}
\newcommand{\huaX}{\mathcal{X}}
\newcommand{\huaY}{\mathcal{Y}}
\newcommand{\huaQ}{\mathcal{Q}}
\newcommand{\huaP}{\mathcal{P}}
\newcommand{\huaC}{{\mathfrak{C}}}%{\mathcal{C}}
\newcommand{\huaD}{\mathcal{D}}
\newcommand{\huaI}{\mathcal{I}}
\newcommand{\huaJ}{\mathcal{J}}
\newcommand{\huaH}{\mathcal{H}}
\newcommand{\huaK}{\mathcal{K}}
\newcommand{\huaO}{{\mathcal{O}}}
\newcommand{\huaT}{\mathcal{T}}
\newcommand{\huaZ}{\mathcal{Z}}
\newcommand{\huaN}{\mathcal{N}}
\newcommand{\CWM}{C^{\infty}(M)}
\newcommand{\CWN}{C^{\infty}(N)}
\newcommand{\adjoint}{{ad}}
\newcommand{\CWGamma}{C^{\infty}(\Gamma)}
\newcommand{\XGamma}{\huaX(\Gamma)}
\newcommand{\XM}{\mathcal{X}(M)}
\newcommand{\XN}{\mathcal{X}(N)}
\newcommand{\XalphaGamma}{\chi^\phi(\Gamma)}
\newcommand{\TalphaGamma}{T^\phi\Gamma}
\newcommand{\set}[1]{\left\{#1\right\}}
\newcommand{\norm}[1]{\left\Vert#1\right\Vert}
\newcommand{\abs}[1]{\left\vert#1\right\vert}
\newcommand{\huaMC}{\mathcal{MC}}

\newcommand{\preLie}{{preLie}}
\newcommand{\Perm}{{Perm}}
\newcommand{\LeftMove}[1]{\overleftarrow{#1}}
\newcommand{\RightMove}[1]{\overrightarrow{#1}}

\newcommand{\frka}{\mathfrak a}
\newcommand{\frkb}{\mathfrak b}
\newcommand{\frkc}{\mathfrak c}
\newcommand{\frkd}{\mathfrak d}
\newcommand{\frke}{\mathfrak e}
\newcommand{\frkf}{\mathfrak f}
\newcommand{\frkg}{\mathfrak g}
\newcommand{\frkh}{\mathfrak h}
\newcommand{\frki}{\mathfrak i}
\newcommand{\frkj}{\mathfrak j}
\newcommand{\frkk}{\mathfrak k}
\newcommand{\frkl}{\mathfrak l}
\newcommand{\frkm}{\mathfrak m}
\newcommand{\frkn}{\mathfrak n}
\newcommand{\frko}{\mathfrak o}
\newcommand{\frkp}{\mathfrak p}
\newcommand{\frkq}{\mathfrak q}
\newcommand{\frkr}{\mathfrak r}
\newcommand{\frks}{\mathfrak s}
\newcommand{\frkt}{\mathfrak t}
\newcommand{\frku}{\mathfrak u}
\newcommand{\frkv}{\mathfrak v}
\newcommand{\frkw}{\mathfrak w}
\newcommand{\frkx}{\mathfrak x}
\newcommand{\frky}{\mathfrak y}
\newcommand{\frkz}{\mathfrak z}

\newcommand{\frkA}{\mathfrak A}
\newcommand{\frkB}{\mathfrak B}
\newcommand{\frkC}{\mathfrak C}
\newcommand{\frkD}{\mathfrak D}
\newcommand{\frkE}{\mathfrak E}
\newcommand{\frkF}{\mathfrak F}
\newcommand{\frkG}{\mathfrak G}
\newcommand{\frkH}{\mathfrak H}
\newcommand{\frkI}{\mathfrak I}
\newcommand{\frkJ}{\mathfrak J}
\newcommand{\frkK}{\mathfrak K}
\newcommand{\frkL}{\mathfrak L}
\newcommand{\frkM}{\mathfrak M}
\newcommand{\frkN}{\mathfrak N}
\newcommand{\frkO}{\mathfrak O}
\newcommand{\frkP}{\mathfrak P}
\newcommand{\frkQ}{\mathfrak Q}
\newcommand{\frkR}{\mathfrak R}
\newcommand{\frkS}{\mathfrak S}
\newcommand{\frkT}{\mathfrak T}
\newcommand{\frkU}{\mathfrak U}
\newcommand{\frkV}{\mathfrak V}
\newcommand{\frkW}{\mathfrak W}
\newcommand{\frkX}{\mathfrak X}
\newcommand{\frkY}{\mathfrak Y}
\newcommand{\frkZ}{\mathfrak Z}
%\def\qed{\hfill ~\vrule height6pt width6pt depth0pt}
\newcommand{\rhowx}{\rho^{\star}}
\newcommand{\Cinf}{C^\infty}%{\mathbf{Diff}}
\newcommand{\half}{\frac{1}{2}}
\newcommand{\Deg}{\mathrm{deg}}
\newcommand{\re}{\mathrm{Re}}
\newcommand{\aff}{\mathrm{aff}}
\newcommand{\pomnib}{\mathbf{b}}
\newcommand{\pair}[1]{\left\langle #1\right\rangle}
\newcommand{\ppairingE}[1]{\left ( #1\right )_E}
\newcommand{\starpair}[1]{\left ( #1\right )_{*}}
\newcommand{\ppairing}[1]{\left ( #1\right )}
\newcommand{\pibracket}[1]{\left [ #1\right ]_{\pi}}
\newcommand{\conpairing}[1]{\left\langle  #1\right\rangle }
\newcommand{\Courant}[1]{\left\llbracket  #1\right\rrbracket }
\newcommand{\Dorfman}[1]{\{ #1\}}
\newcommand{\CDorfman}[1]{   [    #1   ]_{\huaK}   }
\newcommand{\jetc}[1]{   [    #1   ]_{\jet{C}}   }
\newcommand{\Lied}{\frkL}
\newcommand{\jet}{\mathfrak{J}}
\newcommand{\WJ}{\mathbf{wJ}}
\newcommand{\jetd}{\mathbbm{d}}
\newcommand{\dev}{\mathfrak{D}}
\newcommand{\spray}[2]{\frkX(#1)}%{\frac{#1}{#2}}
\newcommand{\prolong}[3]{{#1}\diamond[{{#2}},{{#3}}]}
\newcommand{\pie}{^\prime}
\newcommand{\Id}{\rm{Id}}
\newcommand{\huaAStar}{\mathcal{A}^*}
\newcommand{\e}{\mathbbm{e}}
\newcommand{\p}{\mathbbm{p}}
\newcommand{\br}[1]{   [ \cdot,    \cdot  ]   }
\newcommand{\id}{\mathbbm{i}}
\newcommand{\jd}{\mathbbm{j}}
\newcommand{\jE}{\mathscr{J}{E}}
\newcommand{\dE}{\mathscr{D}{E}}
\newcommand{\wedgeE}{\wedge_{E}}
\newcommand{\dM}{\mathrm{d}}
\newcommand{\dt}{\delta^{T}}
\newcommand{\dA}{\mathrm{d}^A}
\newcommand{\dB}{\mathrm{d}^B}
\newcommand{\djet}{\mathrm{d}_{\jet}}
\newcommand{\dimension}{\mathrm{dim}}
\newcommand{\omni}{\mathcal{E}}
\newcommand{\omnirho}{\rho_{\varepsilon}}
\newcommand{\huai}{\mathrm{I}}
\newcommand{\Hom}{\mathrm{Hom}}
\newcommand{\CHom}{\mathrm{CHom}}
\newcommand{\Sym}{\mathrm{Sym}}
\newcommand{\Nat}{\mathbb N}
\newcommand{\Der}{\mathrm{Der}}
\newcommand{\ODer}{\mathrm{OutDer}}
\newcommand{\IDer}{\mathrm{InnDer}}
\newcommand{\Lie}{\mathrm{Lie}}
\newcommand{\Rep}{\mathrm{Rep}}
\newcommand{\ab}{\mathrm{ab}}



\newcommand{\Set}{\mathrm{Set}}
\newcommand{\Nij}{\mathrm{Nij}}
\newcommand{\Ob}{\mathsf{Ob^2_{T}}}
\newcommand{\Inn}{\mathrm{Inn}}
\newcommand{\Out}{\mathrm{Out}}
\newcommand{\Ad}{\mathrm{Ad}}
\newcommand{\Aut}{\mathrm{Aut}}
\newcommand{\diver}{\mathrm{div}}
\newcommand{\gl}{\mathfrak {gl}}
\newcommand{\sln}{\mathfrak {sl}}
\newcommand{\so}{\mathfrak {so}}
\newcommand{\Symm}{\mathrm {Symm}}
\newcommand{\Hgroup}{\mathrm{H}}
\newcommand{\splinear}{\mathrm{sl}}
\newcommand{\jetwedge}{\curlywedge}
\newcommand{\Ker}{\mathrm{Ker}}
\newcommand{\kup}{$\huaO$-operator }
\newcommand{\kups}{$\huaO$-operators }
\newcommand{\AD}{\mathfrak{ad}}


\newcommand{\MC}{\mathrm{MC}}
\newcommand{\SMC}{\mathrm{SMC}}
\newcommand{\Mod}{\mathrm{Mod}}
\newcommand{\End}{\mathrm{End}}
\newcommand{\ad}{\mathrm{ad}}
\newcommand{\pr}{\mathrm{pr}}
\newcommand{\inv}{\mathrm{inv}}
\newcommand{\Img}{\mathrm{Im}}
\newcommand{\ve}{\mathrm{v}}
\newcommand{\Vect}{\mathrm{Vect}}
\newcommand{\sgn}{\mathrm{sgn}}
\newcommand{\Ksgn}{\mathrm{Ksgn}}
\newcommand{\diag}{\mathrm{diag}}
\newcommand{\pat}{\partial_t}
\newcommand{\pau}{\partial_u}
\newcommand{\pae}{\partial_s}
\newcommand{\V}{\mathbb{V}}
\newcommand{\W}{\mathbb{W}}
\newcommand{\K}{\mathbb{K}}
\newcommand{\LL}{\mathbb{L}}
\newcommand{\D}{\mathbb{D}}
\newcommand{\T}{\mathbb{T}}
\newcommand{\Poisson}[1]{\{ #1\}}
%\nc{\oop}{$\mathcal{O}$-operator\xspace}
%\nc{\oops}{$\mathcal{O}$-operators\xspace}
\newcommand{\perm}{\mathbb S}
\newcommand{\nat}{\mathbb Z}

\newcommand{\GRB}{\huaO}
\newcommand{\GRBN}{\mathrm{\huaO N}}

\newcommand{\KVN}{\mathrm{KVN}}
\newcommand{\FGV}{\mathrm{FGV}}
\newcommand{\KVB}{\mathrm{KV\Omega}}

\newcommand{\HN}{\mathrm{H N}}

\newcommand{\OmN}{\mathrm{\Omega N}}
\newcommand{\PN}{\mathrm{PN}}
\newcommand{\POm}{\mathrm{P\Omega}}
\newcommand{\NR}{\mathrm{NR}}
\newcommand{\CE}{\mathrm{CE}}
\newcommand{\jetrho}{\rho_{\pi}}
\newcommand{\eomnib}{\mathbf{a}}

\newcommand{\wrelated}[1]{ ~\stackrel{#1}{\sim}~ }
\newcommand{\connection}{\gamma}
\newcommand{\rhoAWuXing}{\rho_A^{\star}}
\newcommand{\rhoBWuXing}{\rho_B^{\star}}
\newcommand{\PYBE}{\mathrm{PYBE}}
\newcommand{\Li}{\mathsf{3Lie}}
\newcommand{\cC}{{\rm c}\mathsf{3Lie}}
\begin{document}


\setlength{\baselineskip}{1.2\baselineskip}
%%%%%%%%%%%%%%%%%%%%%%%%%%%%%%%%%%%%%%%%%%%%%%%%%%%%%%%%%%%%%%%%%%%%%%%%%%%
%%%%%%%%%%%%%%%%%%%%%%    Title    %%%%%%%%%%%%%%%%%%%%%%%%%%%%%%%%%%%%%%%%
\title[Manin triples associated to $n$-Lie bialgebras]
{Manin triples associated to $n$-Lie bialgebras}
\author{Ying Chen}
\address{
School of Mathematical Sciences  \\
Zhejiang Normal University\\
Jinhua 321004\\
China}
\email{yingchen@zjnu.edu.cn}

\author{Chuangchuang Kang}
\address{
School of Mathematical Sciences  \\
Zhejiang Normal University\\
Jinhua 321004 \\
China}
\email{kangcc@zjnu.edu.cn}

\author{Jiafeng L\"u}
\address{
School of Mathematical Sciences    \\
Zhejiang Normal University\\
Jinhua 321004              \\
China}
\email{jiafenglv@zjnu.edu.cn}


\author{Shizhuo Yu}
\address{
School of Mathematical Sciences and LPMC    \\
Nankai University \\
Tianjin 300317              \\
China}
\email{yusz@nankai.edu.cn}




\dedicatory{\it{Dedicated to the Memory of Professor Yuri I. Manin (1937-2023)}}
\begin{abstract}
In this paper, we study the Manin triples associated to $n$-Lie bialgebras. We  develop the method of  double constructions as well as operad matrices to make $n$-Lie bialgebras into Manin triples. Then,
the related Manin triples lead to a natural construction of metric $n$-Lie algebras. Moreover, a one-to-one correspondence between the double of $n$-Lie bialgebras and Manin triples
of $n$-Lie algebras be established.

\end{abstract}


\subjclass[2010]{17B62,17A42,17B37,17B60}

\keywords{Manin triples, $n$-Lie bialgebras, double of $n$-Lie bialgebras, $n$-Lie algebras}

%\maketitle

%\tableofcontents




\maketitle
%\tableofcontents
%%%%%%%%%%%%%%%%%%%%   Introduction   %%%%%%%%%%%%%%%%%%%%%%%%%%%%%%%%%%%%%%%%
\allowdisplaybreaks

















\section{Introduction}\label{sec:intr}

The aim of this paper is to extend  the  Manin triples structure from Lie algebras to  $n$-Lie algebras and get some applications,
with the goal of establishing a one-to-one correspondence between the double of $n$-Lie bialgebras and Manin triples of $n$-Lie algebras.

Manin triples structures naturally induce a class of quasi-triangular $r$-matrix, which provide a class of
examples of Poisson manifolds and Poisson homogeneous spaces in Lie theory \cite{Semenov-Tian-Shansky}. In 1983, Drinfeld \cite{Drinfeld} introduced the notion of Lie bialgebras, which is well established as the infinitesimalisation of quantum groups \cite{Manin1}.
A Lie bialgebra consists of a Lie algebra $\g$ and a compatible Lie cobracket $\delta_\g$, such that the cobracket induces a Lie bracket on the dual space and satisfies the 1-cocycle condition \cite{Drinfeld,Quantum2}. Moreover, Lie bialgebras exponentiate to Poisson-Lie groups, which has attracted considerable interest from Poisson and symplectic geometers \cite{Quantum}.
%Quasi-triangular Lie bialgebras are probably the most important of all Lie bialgebras, as they give rise to the classical Yang-Baxter equation.
In fact, let $(\g,\delta_\g)$ be a Lie bialgebra, then there exists a canonical Lie bialgebra structure on $\g\oplus \g^*$ induced by Manin triples of Lie algebras. The Lie bialgebra on $\g\oplus \g^*$ is called the double Lie bialgebra of $\g$, and it can be used to construct examples of Poisson manifolds \cite{Lu-1}.
The usual interpretation of the 1-cocycle condition for Lie bialgebras is that $\delta_\g$ is a 1-cocycle of $\g$ associated to the representation $\ad\otimes1+1\otimes\ad$ on the tensor space $\g\otimes\g$. Another way to interpret the 1-cocycle condition is to decompose $\delta_\g$ into $\delta_\g^1$ and $\delta_\g^2$. Here,
$\delta_\g^1$ and $\delta_\g^2$ are 1-cocycles of $\g$ associated to $\ad\otimes1$ and $1\otimes\ad$, respectively, satisfying a compatibility condition \cite{Bai}. This equivalent interpretation leads to the local cocycle condition and can be understood from an operadic point of view \cite{Loday}. It is natural to extend such structures to the $n$-Lie bialgebras, that is, consider the operad matrices of $n$-Lie bialgebras.

In 1985, Filippov \cite{Filippov} introduced the definition of $n$-Lie algebras (also known as Filippov algebras). His paper considered $n$-ary multi-linear and skew-symmetric operation that satisfy the generalized Jacobi identity, which appear in many fields in mathematics and mathematical physics\cite{$n$-ary}. Particularly, $3$-Lie algebras play an important role in string theory \cite{Bagger,Drinfeld2,Gustavsson,Matsuo,Matsuo2}.
%It is natural to construct a suitable bialgebra theory for $3$-Lie algebras.
In 2016, \cite{Bai} introduced two types of $3$-Lie bialgebras, whose compatibility conditions are given by local cocycles and double constructions respectively. The notion of 3-Lie classical Yang-Baxter equation (3-Lie CYBE) is derived from local cocycle 3-Lie bialgebras, and the solutions to this equation give rise to coboundary local cocycle 3-Lie bialgebras. Meanwhile, $3$-pre-Lie algebras give rise to solutions of the $3$-Lie CYBE. In 2017, \cite{Bai2} classified the double construction 3-Lie bialgebras for complex 3-Lie algebras in dimensions 3 and 4 and provided the corresponding pseudo-metric 3-Lie algebras of dimension 8. In \cite{Bai-R}, $n$-Lie coalgebras with rank $r$ are defined, their structures are discussed, and $n$-Lie bialgebras are introduced and their structures investigated.

However,  there is currently no known coboundary theory or structure for the double space  $\g\oplus \g^*$ of $n$-Lie bialgebras.
Inspired by the notations of local cocycle 3-Lie bialgebras and double
construction 3-Lie bialgebra, it is natural to consider the analogue
of Manin triples associated to $n$-Lie bialgebras.


%
%In 2017, \cite{Bai2} classified the double construction $3$-Lie bialgebras for complex $3$-Lie algebras in dimensions $3$ and $4$ and gave the corresponding $8$-dimensional pseudo-metric $3$-Lie algebras. Based on the previous work, another purpose of our article is to develop a Manin triple on a $n$-Lie algebra to define a bialgebra structure on a $n$-Lie algebra by following the approach of these scholars.
%
%
%The
%
%
%to a connection with quantum groups. Since then, there have been many articles on Lie bialgebras. The bialgebra theory is an essential aspect of Lie algebra, because it has important applications. For example, the Lie bialgebra is the algebraic structure corresponding to a Poisson-Lie group and the classical structure of a quantized universal enveloping algebra \cite{Quantum,Drinfeld2}. In fact, Lie bialgebras were suggested by a study of Hamiltonian mechanics and Poisson Lie group, which are a vector space equipped with the structure of Lie algebra and Lie coalgebra, and satisfying some compatibility conditions \cite{Drinfeld,Quantum2}.
%
%
%In 1985, V. T. Filippov \cite{Filippov} introduced the definition of $n$-Lie algebras (also known as Filippov algebras). In his paper, $n$-ary multi-linear and skew-symmetric operation $[\cdot,\cdots,\cdot]$ was considered, satisfying the generalized Jacobi identity:
%\begin{equation*}
%  [x_1,\cdots,x_{n-1},[y_1,\cdots,y_n]]=\sum\limits_{i=1}^n[y_1,\cdots,y_{i-1},[x_1,\cdots,x_{n-1},y_i],y_{i+1},\cdots,y_n].
%\end{equation*}
% $n$-Lie structures appear in many fields in mathematics and mathematical physics. Motivated by problems of quark dynamics, in 1973, Nambu proposed a generalization of classical Hamiltonian mechanics by the $3$-ary Poisson bracket\cite{Nambu}. Based on Nambu's work, Takhtajan \cite{Takhtajan} studied the general theory of $n$-Poisson or Nambu-Poisson manifolds.
%
%
%
%
% geometries and string theory \cite{symmetry,gauge}, which have applied to several fields of mathematics and physics. For example $3$-Lie algebras have appeared in multiple $M2$-Branes theory in the context of the Bagger-Lambert-Gustavsson model \cite{$n$-ary}.  In the infinite-dimensional case, Nambu algebras are a particular $3$-Lie algebra, and Nambu-Poisson structures have been generally studied for arbitrary $n$ by Takhtajan \cite{Takhtajan} in $1994$. Actually, a Nambu-Poisson structure is defined by a $n$-linear mapping, satisfying skew-symmetric, Leibniz rule and Jacobi identity \cite{Poisson}. The Nambu-Poisson tensors provide a generalization of the Poisson structure on a manifold \cite{Lichnerowicz}.
%
%
%
%In $1994$, W. Michaelis \cite{Michaelis} studied a class of Witt type Lie bialgebras, and introduced how to construct coboundary or triangular Lie bialgebras from a special kind of Lie algebras. In $2000$, S. H. Ng and E. J. Taft \cite{Taft,Taft2} proved that all structures of Lie bialgebras on the one sided Witt algebra, the Witt algebra and the Virasoro algebra are coboundary triangular, and they completed the classification of all structures of Lie bialgebras on the one sided Witt algebra. In $2006$, in the context of deformation of the Witt algebra and the Virasoro algebra, the notion of Hom-Lie algebra was introduced by J. T. Hartwig, D. Larsson and S. D. Silvestrov \cite{Hartwig}. Thus, how to extend the theory of Lie bialgebras to the Hom structures was considered naturally. In $2015$, D.Yau \cite{Hom-Liebialg} introduced a notion of Hom-Lie bialgebras, and studied coboundary and quasi-triangular these two cases in detail.
%
%Yang-Baxter equations(YBE) were first introduced by R. J. Baxter, C. N. Yang and McGuire in statistical mechanics \cite{Baxter,Baxter2,Yang}. In $1979-1980$, E. K. Sklyanin \cite{Sklyanin,Sklyanin2} introduced the classical Yang-Baxter equation(CYBE). In $2009$, as a Hom type generalization of the YBE, D. Yau \cite{Yau} introduced the Hom-Yang-Baxter equation(HYBE). Based on previous research on Hom-Lie algebras and the HYBE, D. Yau studied a twisted generalization of the CYBE and the related structure of Hom-Lie bialgebras in $2015$ \cite{Hom-Liebialg}. Therefore, bialgebraic theory and some characteristic solutions of YBE bridge mathematics and physics.
%
%
%
%Metric $3$-Lie algebras are interesting in physics. For example, $3$-Lie algebras are required to admit an invariant, in order to obtain the correct equations of motion for the Begger-Lambert theory from a Lagrangian. The relative signs of the kinetic terms for scalar and fermion fields in the Bagger-Lambert Lagrangian are determined by the signature of this metric \cite{Bagger,Bagger2,Gustavsson}. In fact, Lie bialgebras can be described in terms of Manin triples equivalently. It is natural to consider an analogue of Manin triples for $3$-Lie algebras, because the double structure of Manin triples can characterize Lie bialgebras. In $2016$, Chengming Bai, Li Guo and YunHe Sheng \cite{Bai} utilized a Manin triple on a $3$-Lie algebra to define a bialgebra structure on a $3$-Lie algebra by following the theory of Lie bialgebras. They called it a double construction $3$-Lie bialgebra, and also showed that it can be regarded as a special case of the local cocycle $3$-Lie bialgebra. Their approach provides a natural construction of pseudo-metric $3$-Lie algebras with signature $(n,n)$ for the aforementioned study of Bagger-Lambert Lagrangian. In $2017$, Chengming Bai, Li Guo and Chengdu Du \cite{Bai2} classified the double construction $3$-Lie bialgebras for complex $3$-Lie algebras in dimensions $3$ and $4$ and gave the corresponding $8$-dimensional pseudo-metric $3$-Lie algebras. Based on the previous work, another purpose of our article is to develop a Manin triple on a $n$-Lie algebra to define a bialgebra structure on a $n$-Lie algebra by following the approach of these scholars.

The paper is organized as follows. Theorem \ref{main} in Section \ref{sec:double} is the main result and Section \ref{sec:preliminary} as well as Section \ref{sec:n-Lie-bi} are the preparation for it. Concretely, in Section \ref{sec:preliminary}, we introduce some concepts and known results about $n$-Lie algebras that will be used later. In Section \ref{sec:n-Lie-bi}, we summarize the coboundary theory of $n$-Lie algebras. We also define $n$-Lie bialgebras and show that each $n$-Lie bialgebra must have a dual $n$-Lie bialgebra whose dual is the $n$-Lie bialgebra itself. In Section \ref{sec:double}, we define an operad matrix of $n$-Lie bialgebras and a local cocycle $n$-Lie bialgebra. We also establish a one-to-one correspondence between the double of $n$-Lie bialgebras and  Manin triples of $n$-Lie algebras.

Throughout this paper, all algebras are finite-dimensional and over a field $F$ of
characteristic zero.
\section{Preliminary results on $n$-Lie algebras}\label{sec:preliminary}
In this section, we will give some preliminaries and basic results on $n$-Lie algebras from \cite{Filippov}.
\begin{defi}\label{defi:n-Lie-alg}
An {\bf $n$-Lie algebra} is a vector space $\g$ with a skew-symmetric n-linear map $[\cdot,\cdots,\cdot]:\otimes^n\g\rightarrow \g$ such that the following Filippov-Jacobi identity holds, for all $x_i,y_i \in \g, 1\leq i\leq n$,
\begin{equation}\label{eq:FJi}
  [x_1,\cdots,x_{n-1},[y_1,\cdots,y_n]]=\sum\limits_{i=1}^n[y_1,\cdots,y_{i-1},[x_1,\cdots,x_{n-1},y_i],y_{i+1},\cdots,y_n].
\end{equation}

\end{defi}

The Filippov-Jacobi identity can be described in another way. For $X=(x_1,\cdots,x_{n-1}) \in \wedge^{n-1}\g$, the operator
\begin{equation*}
  \ad(X):\g\rightarrow\g,\quad\ad(X)(y):=[x_1,\cdots,x_{n-1},y],\quad\forall~ y \in \g,
\end{equation*}
is a derivation in the sense that
\begin{equation}\label{eq:FJi-deri}
  \ad(X)([y_1,\cdots,y_n])=\sum\limits_{i=1}^n[y_1,\cdots,y_{i-1},\ad(X)(y_i),y_{i+1},\cdots,y_n].
\end{equation}

\begin{defi}\label{defi:$n$-Lie-alg-rep}
A {\bf representation} of an $n$-Lie algebra $(\g,[\cdot,\cdots,\cdot])$ on a vector space M is a skew-symmetric linear map $\rho:\wedge^{n-1}\g\rightarrow \gl(M)$ satisfying
\begin{align}
\rho([x_1,\cdots,x_n],y_1,\cdots,y_{n-2}) &= \sum\limits_{i=1}^n(-1)^{n-i}\rho(x_1,\cdots,\hat{x_i},\cdots,x_n)\rho(x_i,y_1,\cdots,y_{n-2}), \label{eq:rep-1}\\
[ \rho(x_1,\cdots,x_{n-1}),\rho(y_1,\cdots,y_{n-1})] &= \sum\limits_{i=1}^{n-1}\rho(y_1,\cdots,y_{i-1},[x_1,\cdots,x_{n-1},y_i],y_{i+1},\cdots,y_{n-1}),
\label{eq:rep-2}
\end{align}
for all $x_i,y_i \in \g,~1\leq i\leq n$, where $\hat{x_i}$ means that the element $x_i$ is omitted.

\end{defi}
We denote the representation by the pair $(M,\rho)$, and say that $M$ is a $\g$-module as well. When $\rho=\ad:\wedge^{n-1}\g\rightarrow \gl(\g)$ given by
$$
\ad(x_1,\cdots,x_{n-1})(x_n)=[x_1,\cdots,x_{n-1},x_n],\quad\forall~x_1,\cdots,x_n\in \g,
$$
then the pair $(\g,\ad)$ is a $\g$-module and is called the adjoint module of $\g$.

\begin{defi}\label{defi:$n$-Lie-subalg}
Denote by $[A_1, A_2,\cdots,A_n]$ the subspace of $A$ generated by all vectors $[x_1, x_2, \cdots , x_n]$, where $x_i\in A_i$, for $i = 1, 2, \cdots, n$. Let $\g$ be an $n$-Lie algebra over field $F$ and let $\h$ be a subspace of $\g$. If $[\h,\h,\cdots,\h]_\g\subset \h$, then $\h$ is called an {\bf $n$-Lie subalgebra} of $\g$.
\end{defi}

\begin{pro}
Let $(\g,[\cdot,\cdots,\cdot])$ be an $n$-Lie algebra, then we have
\begin{eqnarray}\label{eq:FJi-biyao}
% \nonumber to remove numbering (before each equation)
  && \sum\limits_{i=1}^{n-1}[y_1,\cdots,y_{i-1},[x_1,\cdots,x_{n-1},y_i],y_{i+1},\cdots,y_{n-1},y_n] \\
\nonumber  &+&\sum\limits_{j=1}^{n-1}[x_1,\cdots,x_{j-1},[y_1,\cdots,y_{n-1},x_j],x_{j+1},\cdots,x_{n-1},y_n]=0.
\end{eqnarray}
\end{pro}

\begin{proof}
By \eqref{eq:FJi}, we have
\begin{eqnarray*}
% \nonumber to remove numbering (before each equation)
  [x_1,\cdots,x_{n-1},[y_1,\cdots,y_n]] &=& \sum\limits_{i=1}^{n-1}[y_1,\cdots,y_{i-1},[x_1,\cdots,x_{n-1},y_i],y_{i+1},\cdots,y_n]\\
  &+&[y_1,\cdots,y_{n-1},[x_1,\cdots,x_{n-1},y_n]] \\
  &=& \sum\limits_{i=1}^{n-1}[y_1,\cdots,y_{i-1},[x_1,\cdots,x_{n-1},y_i],y_{i+1},\cdots,y_{n-1},y_n]\\
  &+&\sum\limits_{j=1}^{n-1}[x_1,\cdots,x_{j-1},[y_1,\cdots,y_{n-1},x_j],x_{j+1},\cdots,x_{n-1},y_n]\\
  &+&[x_1,\cdots,x_{n-1},[y_1,\cdots,y_n]],
\end{eqnarray*}
implying \eqref{eq:FJi-biyao}.
\end{proof}
Let $V$ be a vector space, $V^*$ be the dual space of $V$. For each positive integer $k$, we identify the tensor product $\otimes^k V$ with the space of multi-linear
maps from $\underbrace{V^*\times \cdots \times V^*}_{k-times}\rightarrow \Comp$, such that:
$$
\langle\xi_1\otimes \cdots\otimes \xi_n,v_1\otimes\cdots\otimes v_n\rangle=\langle\xi_1,v_1\rangle\cdots\langle\xi_n,v_n\rangle,\quad \forall ~\xi_1,\cdots,\xi_n\in V^*,v_1,\cdots,v_n\in V,
$$
where $\langle \xi_i,v_i\rangle=\xi_i(v_i)$, $1\leq i\leq n$. For $v_1,\cdots,v_n\in V$, define that
$$
v_1\wedge v_2 \wedge \cdots\wedge v_k=\sum_{\sigma\in S_k}sgn(\sigma)v_{\sigma(1)}\otimes v_{\sigma(2)}\cdots v_{\sigma(k)}\in \wedge^kV\subset \otimes^kV.
$$


\section{$n$-Lie bialgebras}\label{sec:n-Lie-bi}

In this section, we first introduce the coboundary theory of $n$-Lie algebras. We then give the notions of $n$-Lie bialgebras and the coadjoint representation of $n$-Lie algebras.
Finally we  show that each $n$-Lie bialgebra must have a dual $n$-Lie bialgebra whose dual is the $n$-Lie bialgebra itself.



\subsection{$n$-Lie algebra cohomology}

%In order to formulate the definition and properties of $n$-Lie bialgebras in general, we need a few definitions from the theory of $n$-Lie-algebra cohomology.
Let $\g$ be an $n$-Lie algebra over the field $F$ %$F=\Real$ or $\Comp$
and let $(M,\rho)$ be a representation of $\g$ on $M$. For all $X^1,\cdots,X^{n-1}\in \g$, denote the elements $X=X^1\wedge X^2\wedge\cdots\wedge X^{n-1}$ in $\wedge^{n-1}\g$ by $X=(X^1,\cdots,X^{n-1})$. An  action of $\g$ on $M$ is a $\g$-module  $\rho(X)$ and for $a \in M$ we denote $\g$ acts on $M$ by $X(X^1,\cdots,X^{n-1})_\cdot a$. For example, $\g$ acts on itself is the adjoint representation $(\g,\ad)$.

More generally, $\g$ acts on any tensor product of $\g$ in the following way. For decomposable elements, $y_1\otimes\cdots\otimes y_p$ in $\otimes^p\g=\g\otimes\cdots\otimes\g$ ($p$ times),
\begin{eqnarray*}
% \nonumber to remove numbering (before each equation)
  X_\cdot (y_1\otimes\cdots\otimes y_p) &:=& \ad_{x_1,\cdots,x_{n-1}}^{(p)}(y_1\otimes \cdots\otimes y_p), \\
   &=& \ad_{x_1,\cdots,x_{n-1}}y_1\otimes y_2\otimes\cdots\otimes y_p+y_1\otimes \ad_{x_1,\cdots,x_{n-1}}y_2\otimes y_3\otimes\cdots\otimes y_p \\
   &+& \cdots +y_1\otimes y_2\otimes\cdots\otimes y_{p-1}\otimes \ad_{x_1,\cdots,x_{n-1}}y_p.
\end{eqnarray*}

\begin{defi}
Let $\g$ be an $n$-Lie algebra and let $(M,\rho)$ be a representation of $\g$. {\bf $k$-cochains} on $\g$ with values in $M$ is the set
\begin{equation*}\label{defi:k-cochains}
  C^k(\g;M):=~\{~linear~maps~u:\underbrace{\wedge^{n-1}\g\otimes\cdots \otimes\wedge^{n-1}\g}_{k-1 -times}\wedge\g\rightarrow M~\}.
\end{equation*}
\end{defi}

A $1$-cochain on $\g$ with values in $M$ is just a linear map $u$  from $\g$ to $M$, i.e.
\begin{equation*}
  k=1,~u:\g\rightarrow M.
\end{equation*}
For all $X=X^1\wedge\cdots\wedge X^{n-1} \in \wedge^{n-1}\g,~z\in\g$, the coboundary operator  $\delta:C^1(\g;M)\rightarrow C^{2}(\g;M)$ of a $1$-cochain $u$  is given by
\begin{eqnarray*}
% \nonumber to remove numbering (before each equation)
  \delta u(X,z) &=& X(X^1,\cdots,X^{n-1})_\cdot u(z) \\
   & +&\sum\limits_{i=1}^{n-1}(-1)^{i+1}X(X^1,\cdots,\hat{X^i},\cdots,X^{n-1},z)_\cdot u(X^i)-u([X^1,\cdots,X^{n-1},z]).
\end{eqnarray*}

We can deduce that for any $1$-cochain $u$ on $\g$ with values in $M$,
\begin{equation*}
  \delta(\delta u)=0.
\end{equation*}
In fact, for any $X_1, X_2 \in \wedge^{n-1}\g$, $z \in \g$,
\begin{eqnarray*}
% \nonumber to remove numbering (before each equation)
  (\delta(\delta u))(X_1, X_2, z) &=& \rho(X_1^1,\cdots,X_1^{n-1})\delta u(X_2,z)-\rho(X_2^1,\cdots,X_2^{n-1})\delta u(X_1,z) \\
   &+&\sum\limits_{i=1}^{n-1}(-1)^{n-i+1}\rho(X_2^1,\cdots,\hat{X_2^i},\cdots,X_2^{n-1},z)\delta u(X_1,X_2^i) \\
   &-&\delta u(X_2,[X_1^1,\cdots,X_1^{n-1},z])+\delta u(X_1,[X_2^1,\cdots,X_2^{n-1},z]) \\
   &-&\sum\limits_{m=1}^{n-1}(-1)^{m+1}\delta u([X_1^1,\cdots,X_1^{n-1},X_2^m]\wedge X_2^1\wedge\cdots\wedge\hat{X_2^m}\wedge\cdots\wedge X_2^{n-1},z).
\end{eqnarray*}
Since $X\mapsto \rho(X)$ is a representation of $\g$ in $M$, then $\delta(\delta u)=0$.
\begin{defi}
The coboundary of a $k$-cochain $u$ on $\g$ with values in $M$ is the $(k+1)$-cochain $\delta u:C^k(\g;M)\rightarrow C^{k+1}(\g;M)$,  such that for all $X_1,X_2,\cdots,X_k \in \wedge^{n-1}\g$, $z \in \g$,
\begin{eqnarray*}\label{defi:k-coboundary}
% \nonumber to remove numbering (before each equation)
  &&\delta u(X_1,X_2,\cdots,X_k,z) \\
  =&&\sum\limits_{i=1}^{k}(-1)^{i+1}\rho(X_i^1,\cdots,X_i^{n-1})u(X_1,\cdots,\hat{X_i},\cdots,X_k,z) \\
  +&&\sum\limits_{i=1}^{n-1}(-1)^{n+k-i+1}\rho(X_k^1,\cdots,\hat{X_k^i},\cdots,X_k^{n-1},z)u(X_1,\cdots,X_{k-1},\hat{X_k},X_k^i) \\
  +&&\sum\limits_{i=1}^{k}(-1)^{i}u(X_1,\cdots,\hat{X_i},\cdots,X_k,[X_i^1,\cdots,X_i^{n-1},z])  \\
  +&&\sum\limits_{1\leq i\leq j}^{k}(-1)^{i}u(X_1,\cdots,\hat{X_i},\cdots,X_{j-1},\sum\limits_{m=1}^{n-1}[X_i^1,\cdots,X_i^{n-1},X_j^m]\wedge X_j^1\wedge\cdots\wedge\hat{X_j^m}\wedge\cdots\wedge X_j^{n-1},\cdots,X_k,z),
\end{eqnarray*}
where $\hat{X_i}$ indicates that the element $X_i$ is omitted.

\end{defi}

\begin{pro}\label{pro:complex-$n$-Lie}\cite{Casas}
For any $k$-cochain $u$, $k\geq1$, $\delta(\delta u)=0$.
\end{pro}
This is a standard result, which generalizes the property proved above for $k=1$.

\begin{defi}\label{defi:k-cocycle}
A $k$-cochain $u$ is called a {\bf $k$-cocycle} if it satisfies
\begin{equation*}
  \delta u=0.
\end{equation*}
A $k$-cochain $u$ is called a {\bf $k$-coboundary} $(k\geq2)$ if there exists a $(k-1)$-cochain $v$, such that
\begin{equation*}
  u=\delta v.
\end{equation*}
\end{defi}
By Proposition \ref{pro:complex-$n$-Lie}, any $k$-coboundary is a $k$-cocycle. By Definition \ref{defi:k-cocycle}, the quotient of the vector space of $k$-cocycles by the vector space of $k$-coboundaries is called the $k$-th cohomology vector space of $\g$ with values in $M$.
See \cite{Takhtajan-2} for more details.
\subsection{$n$-Lie bialgebras}
Now assume that $\g$ is an $n$-Lie algebra and that $\gamma$ is a linear map from $\g$ to $\otimes^n\g$ whose transpose is denoted by $^t\gamma:\otimes^n\g^\ast\rightarrow\g^\ast$. (If $\g$ is infinite-dimensional, $\otimes^n\g^\ast$ is a subspace of $(\otimes^n\g)^\ast$, and what we are considering is in fact the restriction of the transpose of $\gamma$ .) Recall that a linear map on $\otimes^n\g^\ast$ can be identified with a $n$-linear map on $\g^\ast$.

\begin{defi}\label{defi:$n$-Lie-bialg}
An {\bf $n$-Lie bialgebra} is an $n$-Lie algebra $\g$ with a linear map $\gamma:\g\rightarrow\otimes^n\g$ such that
\begin{enumerate}[(i)]
  \item $^t\gamma:\otimes^n\g^\ast\rightarrow\g^\ast$ defines an $n$-Lie bracket on $\g^\ast$, i.e., $^t\gamma$ is a skew-symmetric $n$-linear map on $\g^\ast$ satisfying the Filippov-Jacobi identity;
  \item $\gamma$ is a $1$-cocycle on $\g$ with values in $\otimes^n\g$, where $\g$ acts on $\otimes^n\g$ by the adjoint representation $\ad^{(n)}$.
\end{enumerate}
\end{defi}
Condition (ii) means that the 2-cochain $\delta\gamma$ vanishes, i.e., for all $x_1,\cdots,x_n \in \g$,
\begin{equation}\label{eq:ii'}
 \gamma([x_1,\cdots,x_n])=\sum\limits_{i=1}^{n}(-1)^{n-i}\ad^{(n)}_{x_1,\cdots,\hat{x_i},\cdots,x_n}(\gamma(x_i)).
\end{equation}
Let $[\cdot,\cdots,\cdot]_{\g^*}:\otimes^n \g^*\rightarrow \g^*$ be the $n$-Lie bracket defined by $\gamma$.
Denote $^t\gamma:\otimes^n \g^*\rightarrow \g^*$ by
\begin{equation}
  [\xi_1,\cdots,\xi_n]_{\g^\ast}=^t\gamma(\xi_1\otimes\cdots\otimes\xi_n),\quad\forall~\xi_1,\cdots,\xi_n\in\g^\ast.
\end{equation}
Then by Definition \ref{defi:$n$-Lie-bialg}, for all $x \in \g$,
\begin{equation}\label{eq:gamma-tran}
  \langle~[\xi_1,\cdots,\xi_n]_{\g^\ast},x~\rangle = \langle~\gamma(x),\xi_1\otimes\cdots\otimes\xi_n~\rangle.
\end{equation}

Condition (i) is equivalent to the following two identities:
$$
[\xi_{\sigma(1)},\cdots,\xi_{\sigma(n)}]_{\g^\ast} =sgn(\sigma)[\xi_1,\cdots,\xi_n]_{\g^\ast},
$$
$$
[\xi_1,\cdots,\xi_{n-1},[\eta_1,\cdots,\eta_n]_{\g^\ast}]_{\g^\ast}=\sum\limits_{i=1}^{n}[\eta_1,\cdots,\eta_{i-1},[\xi_1,\cdots,\xi_{n-1},\eta_i]_{\g^\ast},\eta_{i+1},\cdots,\eta_n]_{\g^\ast}.
$$

An alternate way of writing \eqref{eq:ii'} is
\begin{eqnarray*}\label{eq:expandii'}
% \nonumber to remove numbering (before each equation)
  &&\langle~[\xi_1,\cdots,\xi_n]_{\g^\ast},[x_1,\cdots,x_n]~\rangle\\
  &=&\sum\limits_{i=1}^{n}\langle~\xi_1\otimes\cdots\otimes\xi_n,(\ad_{x_1,\cdots,\hat{x_i},\cdots,x_n}\otimes1\otimes\cdots\otimes1\\
  &+&1\otimes\ad_{x_1,\cdots,\hat{x_i},\cdots,x_n}\otimes1\otimes\cdots\otimes1
  +\cdots+1\otimes\cdots\otimes1\otimes\ad_{x_1,\cdots,\hat{x_i},\cdots,x_n})(\gamma(x_i))~\rangle.
\end{eqnarray*}
Using the Sweedler’s notation, write $\gamma(x_i)=z_1\otimes z_2\cdots\otimes z_n$, we have
\begin{eqnarray*}
% \nonumber to remove numbering (before each equation)
  &&(\ad_{x_1,\cdots,x_{n-1}}\otimes1\otimes\cdots\otimes1
  +\cdots+1\otimes\cdots\otimes1\otimes\ad_{x_1,\cdots,x_{n-1}})
  (z_1\otimes z_2\cdots\otimes z_n)  \\
  &=& [x_1,\cdots,x_{n-1},z_1]\otimes z_2\otimes\cdots\otimes z_n
  +z_1\otimes[x_1,\cdots,x_{n-1},z_2]\otimes z_3\otimes\cdots\otimes z_n\\
  &+&\cdots+z_1\otimes z_2\otimes\cdots\otimes[x_1,\cdots,x_{n-1},z_n].
\end{eqnarray*}

\subsection{The coadjoint representation of $n$-Lie algebras}
Now we introduce the definition of the coadjoint representation of $n$-Lie algebras on the dual vector space.

\begin{pro}\label{pro:co-ad}
Let $\g$ be a finite-dimensional $n$-Lie algebra and $\g^\ast$ be its dual vector space. Set
\begin{equation}\label{eq:$n$-Lie-coadjoint-rep}
  \ad^\ast_{x_1,\cdots,x_{n-1}}=-^t(\ad_{x_1,\cdots,x_{n-1}}):\wedge^{n-1}\g\rightarrow End(\g^*),\quad\forall~x_1, \cdots,x_{n-1} \in \g,
\end{equation}
 i.e., $\ad^\ast_{x_1,\cdots,x_{n-1}}$ is the endomorphism of $\g^\ast$ satisfying
\begin{equation}\label{eq:dual-rep}
  \langle~\xi, \ad_{x_1,\cdots,x_{n-1}}x~\rangle=-\langle~\ad^\ast_{x_1,\cdots,x_{n-1}}\xi, x~\rangle,\quad\forall~x \in \g, \xi \in \g^\ast.
\end{equation}
Then $(\g^\ast,\ad^\ast_{x_1,\cdots,x_{n-1}})$ is a representation of $\g$ on $\g^\ast$.
\end{pro}

\begin{proof}
By \eqref{eq:FJi-biyao} and \eqref{eq:dual-rep}, for all $x_1,\cdots,x_n,~y_1,\cdots,y_{n-1},~z \in \g, ~\xi \in \g^\ast$, we have
\begin{eqnarray*}
  &&\langle~\ad_{[x_1,\cdots,x_n],y_1,\cdots,y_{n-2}}^\ast\xi,z~\rangle \\
  =&& -\langle~\xi,[[x_1,\cdots,x_n],y_1,\cdots,y_{n-2},z]~\rangle \\
  =&&(-1)^n\langle~\xi,[y_1,\cdots,y_{n-2},z,[x_1,\cdots,x_n]]~\rangle,\\
  =&&(-1)^n\sum\limits_{i=1}^{n}\langle~\xi,[x_1,\cdots,x_{i-1},[y_1,\cdots,y_{n-2},z,x_i],x_{i+1},\cdots,x_n]~\rangle\\
=&&(-1)^{n-i}\sum\limits_{i=1}^{n}\langle~\xi,[x_i,y_1,\cdots,y_{n-2},[x_1,\cdots,\hat{x_i},\cdots,x_n,z]]~\rangle\\
=&&\langle~\sum\limits_{i=1}^{n}(-1)^{n-i}\ad_{x_1,\cdots,\hat{x_i},\cdots,x_n}^\ast\ad_{x_i,y_1,\cdots,y_{n-2}}^\ast\xi,z~\rangle.
\end{eqnarray*}
And
\begin{eqnarray*}
% \nonumber to remove numbering (before each equation)
  &&\langle~\ad_{x_1,\cdots,x_{n-1}}^\ast\ad_{y_1,\cdots,y_{n-1}}^\ast\xi,z~\rangle-\langle~\ad_{y_1,\cdots,y_{n-1}}^\ast\ad_{x_1,\cdots,x_{n-1}}^\ast\xi,z~\rangle \\
  =&&\langle~\xi,[y_1,\cdots,y_{n-1},[x_1,\cdots,x_{n-1},z]]~\rangle-\langle~\xi,[x_1,\cdots,x_{n-1},[y_1,\cdots,y_{n-1},z]]~\rangle \\
  =&& \langle \xi,\sum\limits_{i=1}^{n-1}[x_1,\cdots,x_{i-1},[y_1,\cdots,y_{n-1},x_i],x_{i+1},\cdots,x_{n-1},z]\rangle\\
=&& \langle~\xi,-\sum\limits_{i=1}^{n-1}[y_1,\cdots,y_{i-1},[x_1,\cdots,x_{n-1},y_i],y_{i+1},\cdots,y_{n-1},z]~\rangle\\
=&&\langle~\sum\limits_{i=1}^{n-1}\ad_{y_1,\cdots,y_{i-1},[x_1,\cdots,x_{n-1},y_i],y_{i+1},\cdots,y_{n-1}}^\ast\xi,z~\rangle.
\end{eqnarray*}

Thus, the following two equalities hold:
\begin{eqnarray*}
% \nonumber to remove numbering (before each equation)
  \ad_{[x_1,\cdots,x_n],y_1,\cdots,y_{n-2}}^\ast &=& \sum\limits_{i=1}^{n}(-1)^{n-i}\ad_{x_1,\cdots,\hat{x_i},\cdots,x_n}^\ast\ad_{x_i,y_1,\cdots,y_{n-2}}^\ast,\\
\ad_{x_1,\cdots,x_{n-1}}^\ast\ad_{y_1,\cdots,y_{n-1}}^\ast-\ad_{y_1,\cdots,y_{n-1}}^\ast\ad_{x_1,\cdots,x_{n-1}}^\ast &=& \sum\limits_{i=1}^{n-1}\ad_{y_1,\cdots,y_{i-1},[x_1,\cdots,x_{n-1},y_i],y_{i+1},\cdots,y_{n-1}}^\ast.
\end{eqnarray*}
Therefore,$(\g^\ast,\ad^\ast_{x_1,\cdots,x_{n-1}})$ is a representation of $\g$ on $\g^\ast$.
\end{proof}

\begin{defi}\label{defi:$n$-Lie-alg-coadjoint-rep}
The representation $(\g^\ast, \ad^\ast_{x_1,\cdots,x_{n-1}})$  is called the {\bf coadjoint representation} of $\g$.
\end{defi}

\subsection{The dual of $n$-Lie bialgebras}

Let  $(\g^\ast, \ad^\ast_{x_1,\cdots,x_{n-1}})$  be the  coadjoint representation of $\g$.
By \eqref{eq:dual-rep}, for all $\xi_1,\cdots,\xi_n\in \g^*$ and $y_1\otimes\cdots\otimes y_p$ in $\otimes^p\g$, we have
\begin{equation}\label{eq:dual-ten}
  \langle~\xi_1\otimes\cdots\otimes\xi_n,\ad_{x_1,\cdots,x_{n-1}}^{(p)}(y_1\otimes\cdots\otimes y_n)~\rangle
  =-\langle~\ad_{x_1,\cdots,x_{n-1}}^{*(p)}(\xi_1\otimes\cdots\otimes\xi_n),y_1\otimes\cdots\otimes y_n~\rangle.
\end{equation}
Let $(\g,\gamma)$ be an $n$-Lie bialgebra and let $[\cdot,\cdots,\cdot]_{\g^*}:\otimes^n \g^*\rightarrow \g^*$ be the $n$-Lie bracket defined by $\gamma$.
 By \eqref{eq:ii'} and \eqref{eq:dual-ten}, for all $\xi_1,\cdots,\xi_n\in \g^*$, $x_1,\cdots,x_n\in \g$, we have
\begin{eqnarray*}
  &&\langle~[\xi_1,\cdots,\xi_n]_{\g^\ast}, [x_1,\cdots,x_n]~\rangle\\
  =&&\langle~\xi_1\otimes\cdots \otimes\xi_n, \gamma([x_1,\cdots,x_n])~\rangle\\
  =&&\langle~\xi_1\otimes\cdots \otimes\xi_n, \sum\limits_{i=1}^{n}(-1)^{n-i}\ad^{(n)}_{x_1,\cdots,\hat{x_i},\cdots,x_n}(\gamma(x_i))~\rangle\\
  =&&\sum\limits_{i=1}^{n}(-1)^{n-i}\langle~\xi_1\otimes\cdots\otimes\xi_n,(\ad_{x_1,\cdots,\hat{x_i},\cdots,x_n}\otimes1\otimes\cdots\otimes1
  +\cdots+1\otimes\cdots\otimes1\otimes\ad_{x_1,\cdots,\hat{x_i},\cdots,x_n})(\gamma(x_i))~\rangle\\
  =&&\sum\limits_{i=1}^{n}(-1)^{n-i+1}\langle~(\ad_{x_1,\cdots,\hat{x_i},\cdots,x_n}^*\otimes1\otimes\cdots\otimes1
  +\cdots
  +1\otimes\cdots\otimes1\otimes\ad_{x_1,\cdots,\hat{x_i},\cdots,x_n}^*)(\xi_1\otimes\cdots\otimes\xi_n),\gamma(x_i)~\rangle\\
  =&&\sum\limits_{i=1}^{n}(-1)^{n-i+1}\langle~[\ad_{x_1,\cdots,\hat{x_i},\cdots,x_n}^*(\xi_1),\cdots,\xi_n]_{\g^\ast}
  +\cdots+[\xi_1,\cdots,\ad_{x_1,\cdots,\hat{x_i},\cdots,x_n}^*(\xi_n)]_{\g^\ast},x_i~\rangle\\
  =&&\sum\limits_{i,j=1}^{n}(-1)^{n-i+1}\langle~[\xi_1,\cdots,\ad_{x_1,\cdots,\hat{x_i},\cdots,x_n}^*(\xi_j),\cdots,\xi_n]_{\g^\ast},x_i~\rangle\\
  =&&\sum\limits_{i,j=1}^{n}(-1)^{n-i+1}(-1)^{n-j} \langle~[\xi_1,\cdots,\xi_{j-1},\xi_{j+1},\cdots,\xi_n,\ad^\ast_{x_1,\cdots,\hat{x_i},\cdots,x_n}(\xi_j)]_{\g^\ast},x_i~\rangle\\
  =&&\sum\limits_{i,j=1}^{n}(-1)^{i+j-1}
  \langle~[\xi_1,\cdots,\xi_{i-1},\xi_{i+1},\cdots,\xi_n,\ad^\ast_{x_1,\cdots,\hat{x_j},\cdots,x_n}(\xi_i)]_{\g^\ast},x_j~\rangle.
\end{eqnarray*}
Then \eqref{eq:ii'} can be written by
\begin{equation}\label{eq:1-cocycle-equ}
  \langle~[\xi_1,\cdots,\xi_n]_{\g^\ast}, [x_1,\cdots,x_n]~\rangle
  =\sum\limits_{i,j=1}^{n}(-1)^{i+j-1}
  \langle~[\xi_1,\cdots,\xi_{i-1},\xi_{i+1},\cdots,\xi_n,\ad^\ast_{x_1,\cdots,\hat{x_j},\cdots,x_n}(\xi_i)]_{\g^\ast},x_j~\rangle.
\end{equation}
For example, in \cite{Bai}, the 1-cocycle condition of Lie bialgebras can be written by
\begin{eqnarray*}
  \langle~ [\xi_1,\xi_2]_{\g^*},[x_1,x_2]\rangle
  &=&-\langle~[\xi_2,\ad_{x_2}^*(\xi_1)],x_1~\rangle
  +\langle~[\xi_1,\ad_{x_2}^*(\xi_2)],x_1~\rangle\\
  &+&\langle~[\xi_2,\ad_{x_1}^*(\xi_1)],x_2~\rangle
  -\langle~[\xi_1,\ad_{x_1}^*(\xi_2)],x_2~\rangle.
\end{eqnarray*}

Set the adjoint representation $\ad:\wedge^{n-1}\g^*\rightarrow End(\g^*)$ of $\g^*$ by
$$
  \ad_{\xi_1,\cdots,\xi_{n-1}}(\xi_n)=[\xi_1,\cdots,\xi_n]_{\g^\ast}.
$$
 Define a skew-symmetric linear map $\ad_{\xi_1,\cdots,\xi_{n-1}}^*:\wedge^{n-1} \g^*\rightarrow End (\g)$ satisfying
\begin{equation}\label{eq:dual-rep-2}
  \langle~\ad_{\xi_1,\cdots,\xi_{n-1}}(\xi_n),x~\rangle = -\langle~\xi_n,\ad^\ast_{\xi_1,\cdots,\xi_{n-1}}(x)~\rangle,\quad\forall~\xi_1, \cdots, \xi_n \in \g^\ast, x \in \g.
\end{equation}
Then, by Proposition \ref{pro:co-ad}, $(\g,\ad_{\xi_1,\cdots,\xi_{n-1}}^*)$ is the coadjoint representation of $\g^\ast$ on $\g$.

By \eqref{eq:1-cocycle-equ} and \eqref{eq:dual-rep-2}, we have
\begin{equation}\label{eq:1-cocycle-equ2}
  \langle~[\xi_1,\cdots,\xi_n]_{\g^\ast}, [x_1,\cdots,x_n]~\rangle =\sum\limits_{i,j=1}^{n}(-1)^{i+j}\langle~\ad^\ast_{x_1,\cdots,\hat{x_j},\cdots,x_n}(\xi_i)
  ,\ad^\ast_{\xi_1,\cdots,\hat{\xi_i},\cdots,\xi_n}(x_j)~\rangle.
\end{equation}
\begin{rmk}
There is a symmetry between $\g$ with its $n$-Lie bracket $[\cdot,\cdots,\cdot]$ and $\g^\ast$ with its $n$-Lie bracket $[\cdot,\cdots,\cdot]_{\g^\ast}$ defined by $\gamma$. Therefore, $\g$ and $\g^\ast$ play symmetric roles.
\end{rmk}
\begin{pro}\label{pro:$n$-Lie-bialg-dual}
Let $(\g,\gamma)$ be an $n$-Lie bialgebra, $^t\gamma$ be the $n$-Lie bracket of $\g^\ast$, and let $\mu$ be the $n$-Lie bracket of $\g$. Then $(\g^\ast, ^t\mu)$ is an $n$-Lie bialgebra.

\end{pro}
\begin{proof}
Let $\mu:\otimes^n\g\rightarrow\g$ be the skew-symmetric $n$-Lie bracket of $\g$.
For all $\xi_1,\cdots,\xi_n\in \g^\ast$, $x_1,\cdots,x_n\in \g$ we have
\begin{equation}\label{eq:mu-tran}
  \langle~[\xi_1,\cdots,\xi_n]_{\g^\ast}, [x_1,\cdots,x_n]~\rangle=\langle~^t\mu([\xi_1,\cdots,\xi_n]_{\g^\ast}), x_1\otimes \cdots\otimes x_n~\rangle.
\end{equation}
By \eqref{eq:1-cocycle-equ2} and \eqref{eq:mu-tran}, we have
\begin{eqnarray*}
% \nonumber to remove numbering (before each equation)
  && \langle~^t\mu([\xi_1,\cdots,\xi_n]_{\g^\ast}), x_1\otimes \cdots\otimes x_n~\rangle \\
  &=& \sum\limits_{i,j=1}^{n}(-1)^{i+j+1}\langle~\xi_i,[x_1,\cdots,x_{j-1},\hat{x_j},x_{j+1},\cdots,x_n,\ad^\ast_{\xi_1,\cdots,\hat{\xi_i},\cdots,\xi_n}x_j]~\rangle\\
  &=& \sum\limits_{i,j=1}^{n} (-1)^{i+j+1}\langle~^t\mu(\xi_i),x_1\otimes\cdots\otimes x_{j-1}\otimes\hat{x_j}\otimes x_{j+1}\otimes\cdots\otimes x_n\otimes\ad^\ast_{\xi_1,\cdots,\hat{\xi_i},\cdots,\xi_n}x_j~\rangle\\
  &=&\sum\limits_{i,j=1}^{n} (-1)^{i+j+1}(-1)^{n-j}\langle~^t\mu(\xi_i),x_1\otimes\cdots\otimes x_{j-1}\otimes\ad^\ast_{\xi_1,\cdots,\hat{\xi_i},\cdots,\xi_n}x_j\otimes x_{j+1}\otimes\cdots\otimes x_n~\rangle\\
  &=&\sum\limits_{i=1}^{n}(-1)^{n+i-2i}\langle~(\ad_{\xi_1,\cdots,\hat{\xi_i},\cdots,\xi_n}\otimes1\otimes\cdots\otimes1
  +1\otimes\ad_{\xi_1,\cdots,\hat{\xi_i},\cdots,\xi_n}\otimes1\otimes\cdots\otimes1 \\
  &+&\cdots+1\otimes\cdots\otimes1\otimes\ad_{\xi_1,\cdots,\hat{\xi_i},\cdots,\xi_n})(^t\mu(\xi_i)),x_1\otimes\cdots\otimes x_{j-1}\otimes{x_j}\otimes x_{j+1}\otimes\cdots\otimes x_n~\rangle.
\end{eqnarray*}
Therefore,
\begin{equation*}
  (^t\mu)([\xi_1,\cdots,\xi_n]_{\g^\ast})=\sum\limits_{i=1}^{n}(-1)^{n-i}\ad^{(n)}_{\xi_1,\cdots,\hat{\xi_i},\cdots,\xi_n}(^t\mu(\xi_i)),
\end{equation*}
which implies that $^t\mu$ is a $1$-cocycle. Then $(\g^\ast, ^t\mu)$ is an $n$-Lie bialgebra.
\end{proof}

\begin{rmk}
By Proposition \ref{pro:$n$-Lie-bialg-dual}, each $n$-Lie
bialgebra has a dual $n$-Lie bialgebra whose dual is the $n$-Lie bialgebra itself. The $n$-Lie bialgebra $(\g^\ast, ^t\mu)$ is called the {\bf dual of $n$-Lie bialgebra} $(\g,\gamma)$.
\end{rmk}
\section{The double of $n$-Lie bialgebras.}\label{sec:double}

In this section, we define an operad matrix of $n$-Lie bialgebras and a local cocycle $n$-Lie bialgebra. It is a generalization of local cocycle $3$-Lie bialgebras introduced by \cite{Bai}. We also establish a one-to-one correspondence between the double of $n$-Lie bialgebras and a Manin triple of $n$-Lie algebras.





\subsection{Local cocycle $n$-Lie bialgebras}
In this subsection, we first introduce the notion of operad matrix, which can be utilized to represent the 1-cocycle condition of $n$-Lie bialgebras. Using the operad matrix, we define $R_i$-operad $n$-Lie bialgebras and $C_j$-operad $n$-Lie bialgebras based on rows and columns respectively. We also demonstrate that $R_i$-operad matrices generalize the local cocycle 3-Lie bialgebras to local cocycle $n$-Lie bialgebras. Finally, we give the relationship between $n$-Lie bialgebras and local cocycle $n$-Lie bialgebras in Proposition \ref{lem:lo-co-$n$-Lie-bialg}.





\begin{pro}\label{pro:$n$-Lie-alg-ope}
Let $(\g,\gamma)$ be an $n$-Lie bialgebra. For all $x_1,\cdots,x_n\in \g$, define an {\bf operad matrix} of $(\g,\gamma)$ by
\begin{eqnarray*}
A=\left(
    \begin{array}{cccc}
       (-1)^{n-1}\ad_{x_2,x_3,\cdots,x_n}\otimes^{n-1}1& (-1)^{n-2}\ad_{x_1,x_3,\cdots,x_n}\otimes^{n-1}1 & \cdots & \ad_{x_1,\cdots,x_{n-2},x_{n-1}}\otimes^{n-1}1 \\
      (-1)^{n-1}1\otimes \ad_{x_2,x_3,\cdots,x_n}\otimes^{n-2}1 & (-1)^{n-2}1\otimes \ad_{x_1,x_3,\cdots,x_n}\otimes^{n-2}1 & \cdots & 1\otimes \ad_{x_1,\cdots,x_{n-2},x_{n-1}}\otimes^{n-2}1 \\
      \vdots & \vdots & \ddots & \vdots \\
      (-1)^{n-1}1\otimes^{n-2}1\otimes \ad_{x_2,x_3,\cdots,x_n} & (-1)^{n-2}1\otimes^{n-2}1\otimes \ad_{x_1,x_3,\cdots,x_n} & \cdots & 1\otimes^{n-2}1\otimes \ad_{x_1,\cdots,x_{n-2},x_{n-1}} \\
    \end{array}
  \right),
\end{eqnarray*}
where $ \ad:\wedge^{n-1}\g \rightarrow End(\g)$ is the  adjoint representation of $\g$, $1$ is a symbol playing a similar role of unit. Then $\gamma$ is a 1-cocycle if and only if
\begin{equation}\label{eq:1-cocy}
  \gamma([x_1,\cdots,x_n])=(1,1,\cdots,1)_{1\times n}A(\gamma(x_1), \gamma(x_2), \cdots ,\gamma(x_n) )^T.
\end{equation}
\end{pro}
\begin{proof}
 By \eqref{eq:ii'}, we have
 \begin{eqnarray*}
 % \nonumber to remove numbering (before each equation)
   && \gamma([x_1,\cdots,x_n]) \\
   =&& [\gamma(x_1),x_2,\cdots,x_n]+[x_1,\gamma(x_2),\cdots,x_n]+\cdots+[x_1,\cdots,x_{n-1},\gamma(x_n)], \\
   =&& (-1)^{n-1}[x_2,x_3,\cdots,x_n,\gamma(x_1)]+(-1)^{n-2}[x_1,x_3,\cdots,x_n,\gamma(x_2)]+\cdots+[x_1,\cdots,x_{n-2},x_{n-1},\gamma(x_n)], \\
   =&& (-1)^{n-1}(\ad_{x_2,x_3,\cdots,x_n}\otimes^{n-1}1+1\otimes\ad_{x_2,x_3,\cdots,x_n}\otimes^{n-2}1+\cdots+1\otimes^{n-2}1\otimes\ad_{x_2,x_3,\cdots,x_n})\gamma(x_1) \\
   +&& (-1)^{n-2}(\ad_{x_1,x_3,\cdots,x_n}\otimes^{n-1}1+1\otimes\ad_{x_1,x_3,\cdots,x_n}\otimes^{n-2}1+\cdots+1\otimes^{n-2}1\otimes\ad_{x_1,x_3,\cdots,x_n})\gamma(x_2) \\
   +&& \cdots \\
   +&& (\ad_{x_1,\cdots,x_{n-2},x_{n-1}}\otimes^{n-1}1+1\otimes\ad_{x_1,\cdots,x_{n-2},x_{n-1}}\otimes^{n-2}1+\cdots+1\otimes^{n-2}1\otimes\ad_{x_1,\cdots,x_{n-2},x_{n-1}})\gamma(x_n),
   \\
   =&& (1,1,\cdots,1)_{1\times n}A(\gamma(x_1), \gamma(x_2), \cdots ,\gamma(x_n) )^T.
 \end{eqnarray*}
 This completes the proof.
\end{proof}

\begin{ex}
Let $(\g,\gamma)$ be a Lie bialgebra, then $\gamma$ is a 1-cocycle if and only if
$$
\gamma([x_1,x_2])=(1,1)A(\gamma(x_1), \gamma(x_2))^T,
$$
where
$$
A=\left(
    \begin{array}{cc}
      -\ad_{x_2}\otimes1 & \ad_{x_1}\otimes1 \\
      -1\otimes\ad_{x_2} & 1\otimes\ad_{x_1} \\
    \end{array}
  \right).
$$
\end{ex}

\begin{ex}
Let $(\g,\gamma)$ be a 3-Lie bialgebra, then $\gamma$ is a 1-cocycle if and only if
$$
\gamma([x_1,x_2,x_3])=(1,1,1)A(\gamma(x_1), \gamma(x_2), \gamma(x_3))^T,
$$
where
$$
A=\left(
    \begin{array}{ccc}
      \ad_{x_2,x_3}\otimes1\otimes1 & -\ad_{x_1,x_3}\otimes1\otimes1 & \ad_{x_1,x_2}\otimes1\otimes1 \\
      1\otimes\ad_{x_2,x_3}\otimes1 & -1\otimes\ad_{x_1,x_3}\otimes1 & 1\otimes\ad_{x_1,x_2}\otimes1 \\
      1\otimes1\otimes\ad_{x_2,x_3} & -1\otimes1\otimes\ad_{x_1,x_3} & 1\otimes1\otimes\ad_{x_1,x_2} \\
    \end{array}
  \right).
$$
\end{ex}

\begin{defi}
Let the operad matrix $A$ keep the $i$-th row, all other positions are zero, i,e., the operad matrix is given by
 $$
A_{R_i}=\left(
          \begin{array}{cccc}
            0 & 0 & 0 & 0 \\
            \cdots & \cdots & \cdots & \cdots \\
(-1)^{n-1}\otimes^{i-1}1\otimes \ad_{\hat{x}_1}\otimes^{n-i}1 & (-1)^{n-2}\otimes^{i-1}1\otimes \ad_{\hat{x}_2}\otimes^{n-i}1 & \cdots & \otimes^{i-1}1\otimes \ad_{\hat{x}_n}\otimes^{n-i}1 \\
            \cdots & \cdots & \cdots & \cdots \\
            0 & 0 & 0 & 0 \\
          \end{array}
        \right),
 $$
 where $\ad_{\hat{x}_i}=\ad_{x_1,\cdots,\hat{x}_i,\cdots,x_n}$, $i=1,~2,\cdots,~n$. If the linear map $\gamma_R^i:\g\rightarrow \otimes^n \g$ satisfying
 \begin{equation}\label{$R$}
   \gamma_R^i([x_1,\cdots,x_n])=(1,1,\cdots,1)_{1\times n}A_{R_i}(\gamma_R^i(x_1), \gamma_R^i(x_2), \cdots ,\gamma_R^i(x_n) )^T,
 \end{equation}
 such that $^t\gamma_R^i:\otimes^n\g^\ast\rightarrow\g^\ast$ defines an $n$-Lie algebra structure on $\g^\ast$, then we call $(\g,\gamma_R^i)$ the {\bf $R_i$-operad} $n$-Lie bialgebra.
\end{defi}

\begin{ex}\cite{Bai}
Let $\g$ be a Lie algebra, and let $\gamma_R^1, \gamma_R^2: \g\rightarrow\g\otimes\g$ be two linear maps such that
\begin{eqnarray*}
% \nonumber to remove numbering (before each equation)
  \gamma_R^1([x_1,x_2]) &=& (1,1)A_{R_1}(\gamma_R^1(x_1), \gamma_R^1(x_2))^T, \\
  \gamma_R^2([x_1,x_2]) &=& (1,1)A_{R_2}(\gamma_R^2(x_1), \gamma_R^2(x_2))^T,
\end{eqnarray*}
where
$$
A_{R_1}=\left(
          \begin{array}{cc}
            -\ad_{x_2}\otimes1 & \ad_{x_1}\otimes1 \\
            0 & 0 \\
          \end{array}
        \right),
$$
and
$$
A_{R_2}=\left(
          \begin{array}{cc}
            0 & 0 \\
            -1\otimes\ad_{x_2} & 1\otimes\ad_{x_1} \\
          \end{array}
        \right).
$$
If $\gamma=\gamma_R^1+\gamma_R^2$ satisfy $^t\gamma:\g^\ast\otimes\g^\ast\rightarrow\g^\ast$ defines a Lie algebra structure on $\g^\ast$, then $(\g,\gamma)$ is called a local cocycle Lie bialgebra.
\end{ex}

\begin{rmk}\cite{Bai}
Let $(\g,\gamma)$ be a local cocycle Lie bialgebra. If the following compatibility condition holds:
\begin{equation}\label{eq:compatibility}
  (1\otimes\ad_{x_1})\gamma_1(x_2)+(\ad_{x_1}\otimes1)\gamma_2(x_2)-(1\otimes\ad_{x_2})\gamma_1(x_1)-(\ad_{x_2}\otimes1)\gamma_2(x_1)=0,
\end{equation}
then $(\g,\gamma)$ is a Lie bialgebra. Conversely, let $(\g,\gamma)$ be a Lie bialgebra. If $\gamma=\gamma_1+\gamma_2$ such that for any $x_1,x_2 \in \g$, \eqref{eq:compatibility} holds, then $(\g,\gamma)$ is a local cocycle Lie bialgebra.
\end{rmk}

\begin{defi}
Let the operad matrix $A$ keep the $j$-th column, all other position are zero,  i,e., the operad matrix is given by
$$
A_{C_j}=\left(
          \begin{array}{ccccc}
            0 & \cdots & (-1)^{n-j} \ad_{\hat{x}_j}\otimes^{n-1}1
             & \cdots & 0 \\
            0 & \cdots & (-1)^{n-j}1\otimes \ad_{\hat{x}_j}\otimes^{n-2}1
            & \cdots & 0 \\
            \vdots & \ddots & \vdots
            & \ddots & \vdots \\
            0 & \cdots & (-1)^{n-j}1\otimes^{n-2}1\otimes \ad_{\hat{x}_j}
            & \cdots & 0 \\
          \end{array}
        \right).
$$
If the linear map $\gamma_C^j:\g\rightarrow \otimes^n \g$ satisfying
\begin{equation}\label{$C$}
  \gamma_C^j([x_1,\cdots,x_n])=(1,1,\cdots,1)_{1\times n}A_{C_j}(\gamma_C^j(x_1), \gamma_C^j(x_2), \cdots ,\gamma_C^j(x_n) )^T,
\end{equation}
such that $^t\gamma_C^j:\otimes^n\g^\ast\rightarrow\g^\ast$ defines an $n$-Lie algebra structure on $\g^\ast$, then we call $(\g,\gamma_C^j)$ the {\bf $C_j$-operad} $n$-Lie bialgebra.
\end{defi}

\begin{ex}
Let $\g$ be a Lie algebra, and let $\gamma_C^1, \gamma_C^2: \g\rightarrow\g\otimes\g$ be two linear maps such that
\begin{eqnarray*}
% \nonumber to remove numbering (before each equation)
  \gamma_C^1([x_1,x_2]) &=& (1,1)A_{C_1}(\gamma_C^1(x_1), \gamma_C^1(x_2))^T, \\
  \gamma_C^2([x_1,x_2]) &=& (1,1)A_{C_2}(\gamma_C^2(x_1), \gamma_C^2(x_2))^T,
\end{eqnarray*}
where
$$
A_{C_1}=\left(
          \begin{array}{cc}
            -\ad_{x_2}\otimes1 & 0 \\
            -1\otimes\ad_{x_2} & 0 \\
          \end{array}
        \right),
$$
and
$$
A_{C_2}=\left(
          \begin{array}{cc}
            0 & \ad_{x_1}\otimes1 \\
            0 & 1\otimes\ad_{x_1} \\
          \end{array}
        \right).
$$
If $\gamma_C^1$ (or $\gamma_C^2$) satisfy $^t\gamma_C^1$ (or $^t\gamma_C^2$) $:\g^\ast\otimes\g^\ast\rightarrow\g^\ast$ defines a Lie algebra structure on $\g^\ast$, then $(\g,\gamma_C^1)$ or $((\g,\gamma_C^2))$ is called a $C_1$ ( or $C_2$)-operad Lie bialgebra.
\end{ex}

\begin{rmk}
Let $(\g,\gamma_C^1)$ be a $C_1$-operad Lie bialgebra, and let $(\g,\gamma_C^2)$ be a $C_2$-operad Lie bialgebra. If the following compatibility condition holds:
\begin{equation*}
  (\ad_{x_1}\otimes1)\gamma_1(x_2)+(1\otimes\ad_{x_1})\gamma_1(x_2)-(\ad_{x_2}\otimes1)\gamma_2(x_1)-(1\otimes\ad_{x_2})\gamma_2(x_1)=0,
\end{equation*}
then $(\g,\gamma=\gamma_C^1+\gamma_C^2)$ is a Lie bialgebra.
\end{rmk}

%For example, if $(A,\gamma)$ is a $R_1$-Livernet operad, then we have
%\begin{equation*}
%  \gamma([x_1,\cdots,x_n])=\sum\limits_{i=1}^{n}(-1)^{i+1}(\ad_{x_1,\cdots,x_{i-1},\hat{x_i},x_{i+1},\cdots,x_n}\otimes1\otimes\cdots\otimes1)\gamma(x_i),
%\end{equation*}
%$\gamma([x_1,\cdots,x_n])$ simply denoted by $\gamma_R^1$.

%If $(A,\gamma)$ is a $C_1$-Livernet operad, then we have
%\begin{eqnarray*}
%% \nonumber to remove numbering (before each equation)
%  \gamma([x_1,\cdots,x_n]) &=& (\ad_{x_1,\cdots,x_{i-1},\hat{x_i},x_{i+1},\cdots,x_n}\otimes1\otimes\cdots\otimes1+1\otimes\ad_{x_1,\cdots,x_{i-1},\hat{x_i},x_{i+1},\cdots,x_n}\otimes1\otimes\cdots\otimes1 \\
%   &+& \cdots+1\otimes\cdots\otimes1\otimes\ad_{x_1,\cdots,x_{i-1},\hat{x_i},x_{i+1},\cdots,x_n})\gamma(x_1),
%\end{eqnarray*}
%$\gamma([x_1,\cdots,x_n])$ simply denoted by $\gamma_C^1$.
%Let $(\g,\gamma)$ be an $n$-Lie bialgebra, by the above notation and \eqref{eq:ii'}, we have
%\begin{equation}\label{eq:$n$-Lie-gamma}
%  \gamma=\sum\limits_{i=1}^{n}\gamma_R^i=\sum\limits_{j=1}^{n}\gamma_C^j.
%\end{equation}

\begin{defi}\label{defi:local-coc}
Let $\g$ be an $n$-Lie algebra and let $\gamma_1,\gamma_2,\cdots,\gamma_n:\g\rightarrow\otimes^n\g$ be linear maps such that
\begin{equation}\label{eq:gamma-12n-cc}
  \gamma_i = \gamma_{R}^i,\quad i=1,2,\cdots,n.
\end{equation}

If $\gamma=\gamma_1+\gamma_2+\cdots+\gamma_n$ be a linear map, such that $^t\gamma:\otimes^n\g^\ast\rightarrow\g^\ast$ defines an $n$-Lie algebra structure on $\g^\ast$,
then the pair $(\g,\gamma)$ is called a {\bf local cocycle} $n$-Lie bialgebra.
\end{defi}
For any $x_1,\cdots,x_n \in \g$, we can rewrite \eqref{eq:gamma-12n-cc} as,
\begin{eqnarray*}
% \nonumber to remove numbering (before each equation)
  \gamma_1([x_1,\cdots,x_n]) &=& \sum\limits_{i=1}^{n}(-1)^{n-i}(\ad_{x_1,\cdots,x_{i-1},\hat{x_i},x_{i+1},\cdots,x_n}\otimes1\otimes\cdots\otimes1)\gamma_1(x_i), \\
  \nonumber  \gamma_2([x_1,\cdots,x_n]) &=& \sum\limits_{i=1}^{n}(-1)^{n-i}(1\otimes\ad_{x_1,\cdots,x_{i-1},\hat{x_i},x_{i+1},\cdots,x_n}\otimes\cdots\otimes1)\gamma_2(x_i), \\
  \nonumber & \cdots & \\
  \nonumber  \gamma_n([x_1,\cdots,x_n]) &=& \sum\limits_{i=1}^{n}(-1)^{n-i}(1\otimes\cdots\otimes1\otimes\ad_{x_1,\cdots,x_{i-1},\hat{x_i},x_{i+1},\cdots,x_n})\gamma_n(x_i),
\end{eqnarray*}

\begin{ex}
Let $\g$ be a $3$-Lie algebra, and let $\gamma=\gamma_1+\gamma_2+\gamma_3:\g\rightarrow \g\otimes\g\otimes\g$ be a linear map, such that $^t\gamma:\g^\ast\otimes\g^\ast\otimes\g^\ast\rightarrow\g^\ast$ defines a $3$-Lie algebra structure on $\g^\ast$, and for any $x_1,x_2,x_3 \in \g$, the following conditions are satisfied:
\begin{eqnarray*}
% \nonumber to remove numbering (before each equation)
  \gamma_1([x_1,x_2,x_3]) &=& (\ad_{x_2,x_3}\otimes1\otimes1)\gamma_1(x_1)-(\ad_{x_1,x_3}\otimes1\otimes1)\gamma_1(x_2)+(\ad_{x_1,x_2}\otimes1\otimes1)\gamma_1(x_3); \\
  \gamma_2([x_1,x_2,x_3]) &=& (1\otimes\ad_{x_2,x_3}\otimes1)\gamma_2(x_1)-(1\otimes\ad_{x_1,x_3}\otimes1)\gamma_2(x_2)+(1\otimes\ad_{x_1,x_2}\otimes1)\gamma_2(x_3); \\
  \gamma_3([x_1,x_2,x_3]) &=& (1\otimes1\otimes\ad_{x_2,x_3})\gamma_3(x_1)-(1\otimes1\otimes\ad_{x_1,x_3})\gamma_3(x_2)+(1\otimes1\otimes\ad_{x_1,x_2})\gamma_3(x_3).
\end{eqnarray*}
Then the pair $(\g,\gamma)$ is a local cocycle $3$-Lie bialgebra.
\end{ex}

\begin{pro}\label{lem:lo-co-$n$-Lie-bialg}
Let $(\g,\gamma)$ be a local cocycle $n$-Lie bialgebra. If the following compatibility condition holds:
\begin{equation}
  \sum\limits_{k=1}^n\sum\limits_{i=1,i\neq k}^n\sum\limits_{j=1}^n(-1)^{n-j}(\otimes^{k-1}1\otimes\ad_{x_1,\cdots,x_{j-1},\hat{x_j},x_{j+1},\cdots,x_n}\otimes^{n-k}1)\gamma_{i}(x_j)=0,
\end{equation}
then the pair $(\g,\gamma)$ is a $n$-Lie bialgebra.
\end{pro}
\begin{proof}
By \eqref{eq:ii'}, \eqref{eq:gamma-12n-cc} and the Definition \ref{defi:local-coc}, we can complete the proof directly.
\end{proof}

\begin{ex}
Let $(\g,\gamma)$ be a local cocycle Lie bialgebra, and the compatibility condition holds: for any $x_1,x_2 \in \g$
\begin{eqnarray*}
% \nonumber to remove numbering (before each equation)
  && \sum\limits_{j=1}^2(-1)^{2-j}(\ad_{\hat{x_j}}\otimes1)\gamma_{2}(x_j)
  +  \sum\limits_{j=1}^2(-1)^{2-j}(1\otimes\ad_{\hat{x_j}})\gamma_{1}(x_j)\\
  =&&-(\ad_{x_2}\otimes1)\gamma_{2}(x_1)+(\ad_{x_1}\otimes1)\gamma_{2}(x_2)
  -(1\otimes\ad_{x_2})\gamma_{1}(x_1)+(1\otimes\ad_{x_1})\gamma_{1}(x_2) \\
  =&&0,
\end{eqnarray*}
then the pair $(\g,\gamma)$ is a Lie bialgebra.
\end{ex}

\begin{ex}
Let $(\g,\gamma)$ be a local cocycle $3$-Lie bialgebra, and the compatibility condition holds: for any $x_1,x_2,x_3 \in \g$
\begin{eqnarray*}
% \nonumber to remove numbering (before each equation)
  && \sum\limits_{j=1}^3(-1)^{3-j}(\ad_{\hat{x_j}}\otimes1\otimes1)\gamma_{2}(x_j)
  +  \sum\limits_{j=1}^3(-1)^{3-j}(\ad_{\hat{x_j}}\otimes1\otimes1)\gamma_{3}(x_j)\\
  +&&\sum\limits_{j=1}^3(-1)^{3-j}(1\otimes\ad_{\hat{x_j}}\otimes1)\gamma_{1}(x_j)
  +  \sum\limits_{j=1}^3(-1)^{3-j}(1\otimes\ad_{\hat{x_j}}\otimes1)\gamma_{3}(x_j)\\
  +&&\sum\limits_{j=1}^3(-1)^{3-j}(1\otimes1\otimes\ad_{\hat{x_j}})\gamma_{1}(x_j)
  +  \sum\limits_{j=1}^3(-1)^{3-j}(1\otimes1\otimes\ad_{\hat{x_j}})\gamma_{2}(x_j)\\
  =&&(\ad_{x_2,x_3}\otimes1\otimes1)\gamma_{2}(x_1)-(\ad_{x_1,x_3}\otimes1\otimes1)\gamma_{2}(x_2)+(\ad_{x_1,x_2}\otimes1\otimes1)\gamma_{2}(x_3)\\
  +&&(\ad_{x_2,x_3}\otimes1\otimes1)\gamma_{3}(x_1)-(\ad_{x_1,x_3}\otimes1\otimes1)\gamma_{3}(x_2)+(\ad_{x_1,x_2}\otimes1\otimes1)\gamma_{3}(x_3)\\
  +&&(1\otimes\ad_{x_2,x_3}\otimes1)\gamma_{1}(x_1)-(1\otimes\ad_{x_1,x_3}\otimes1)\gamma_{1}(x_2)+(1\otimes\ad_{x_1,x_2}\otimes1)\gamma_{1}(x_3)\\
  +&&(1\otimes\ad_{x_2,x_3}\otimes1)\gamma_{3}(x_1)-(1\otimes\ad_{x_1,x_3}\otimes1)\gamma_{3}(x_2)+(1\otimes\ad_{x_1,x_2}\otimes1)\gamma_{3}(x_3)\\
  +&&(1\otimes1\otimes\ad_{x_2,x_3})\gamma_{1}(x_1)-(1\otimes1\otimes\ad_{x_1,x_3})\gamma_{1}(x_2)+(1\otimes1\otimes\ad_{x_1,x_2})\gamma_{1}(x_3)\\
  +&&(1\otimes1\otimes\ad_{x_2,x_3})\gamma_{2}(x_1)-(1\otimes1\otimes\ad_{x_1,x_3})\gamma_{2}(x_2)+(1\otimes1\otimes\ad_{x_1,x_2})\gamma_{2}(x_3)\\
  =&&0,
\end{eqnarray*}
then the pair $(\g,\gamma)$ is a $3$-Lie bialgebra.
\end{ex}

\begin{rmk}
Let $(\g,\gamma)$ be a local cocycle $n$-Lie bialgebra, and $\mu$ is the $n$-Lie bracket of $\g$, but we cannot obtain that $(\g^\ast, ^t\mu)$ is also a local cocycle $n$-Lie bialgebra, where $^t\gamma$ is the $n$-Lie bracket of $\g^\ast$. For example, the dual of a local cocycle Lie bialgebra is not a local cocycle Lie bialgebra.
\end{rmk}

\subsection{Manin triples of $n$-Lie algebras and the double of $n$-Lie bialgebras.}

We now show that there is a one-to-one correspondence between the double of $n$-Lie bialgebras and a Manin triple of $n$-Lie algebras.

\begin{defi}
A {\bf metric $n$-Lie algebra} is a triple $\big(\g, [\cdot,\cdots,\cdot], (\cdot,\cdot)\big)$, where $\g$ is an $n$-Lie algebra with the $n$-Lie bracket $[\cdot,\cdots,\cdot]$,  and $(\cdot,\cdot):\g\times\g\rightarrow \mathbf{k}$ is a non-degenerate symmetric bilinear form  satisfies the invariant condition:
\begin{equation}\label{eq:$n$-Lie-inva}
  ([x_1,\cdots,x_{n-1},x_n],t)+(x_n,[x_1,\cdots,x_{n-1},t])=0,\quad\forall x_1,\cdots,x_n,t \in \g.
\end{equation}
\end{defi}

\begin{defi}\label{defi:Manin-tri}
Let $\big(\g, [\cdot,\cdots,\cdot], (\cdot,\cdot)\big)$ be a metric $n$-Lie algebra.
A {\bf Manin triple of $n$-Lie algebras} is a triple
$((\g, [\cdot,\cdots,\cdot], (\cdot,\cdot)), \g_1, \g_2)$ such that
 \begin{enumerate}[(i)]
   \item $\g_1$, $\g_2$ are isotropic $n$-Lie subalgebras of $\g$, such that $\g=\g_1\oplus\g_2$ as vector spaces.
   \item For all $x_1,\cdots,x_{n-1}\in \g_1$, $\xi_1,\cdots,\xi_{n-1}\in \g_2$ and $x\in \g$, the following conditions hold:
\begin{eqnarray}
% \nonumber to remove numbering (before each equation)
   &&(\xi_2,[x_1,\cdots,x_{n-1},\xi_1])=0,\\
   &&(x_2,[\xi_1,\cdots,\xi_{n-1},x_1])=0,
\end{eqnarray}
and
\begin{equation}
\left\{
\begin{split}
&(x,[x_1,\cdots,x_{n-2},\xi_1,\xi_2])=0,\\
   &\quad\quad\quad\quad\quad\vdots \\
   &(x,[x_1,x_2,\xi_1,\cdots,\xi_{n-2}])=0.
\end{split}
\right.
\end{equation}
\end{enumerate}
\end{defi}



Let  $(\g,[\cdot,\cdots,\cdot],\gamma_{\g})$ be a $R_1$-operad $n$-Lie bialgebra, where $\gamma_{\g}:\g\rightarrow\otimes^n\g$ is a skew-symmetric linear map  such that $[\cdot,\cdots,\cdot]_{\g^\ast}:\otimes^n\g^\ast\rightarrow\g^\ast$ defines an $n$-Lie bracket on $\g^\ast$.
%i.e. for all $x_1,\cdots,x_n \in \g,\xi_1,\cdots,\xi_n \in \g^\ast$,
%\begin{equation*}
%  \langle~\gamma_{\g}[x_1,\cdots,x_n],\xi\otimes\cdots\otimes\xi_n~\rangle=\langle~[x_1,\cdots,x_n],[\xi\otimes\cdots\otimes\xi_n]_{\g^\ast}~\rangle,
%\end{equation*}
%Let $(\g,[\cdot,\cdots,\cdot],\gamma_{\g})$ be an $n$-Lie bialgebra and let $(\g^*,[\cdot,\cdots,\cdot]_{\g^*},\gamma_{\g^*})$  be its dual $n$-Lie bialgebra.
Set $\frkd=\g\oplus\g^\ast$ and denote its elements by $x+\xi$, where $x\in \g$, $\xi \in \g^*$.
Define a linear map $[\cdot,\cdots,\cdot]_\frkd:\wedge^n \frkd\rightarrow \frkd$ by
\begin{eqnarray}\label{eq:$n$-Lie-on-d}
% \nonumber to remove numbering (before each equation)
  &&[x_1+\xi_1,x_2+\xi_2,\cdots,x_n+\xi_n]_\frkd =
  [x_1,x_2,\cdots,x_n]+\sum\limits_{i=1}^{n}(-1)^{n-i}\ad^\ast_{x_1,\cdots,x_{i-1},\hat{x_i},x_{i+1},\cdots,x_n}\xi_i \nonumber \\ &&+[\xi_1,\xi_2,\cdots,\xi_n]_{\g^\ast}+\sum\limits_{i=1}^{n}(-1)^{n-i}\ad^\ast_{\xi_1,\cdots,\xi_{i-1},\hat{\xi_i},\xi_{i+1},\cdots,\xi_n}x_i,
\quad \forall ~x_1,\cdots,x_n\in \g, \xi_1,\cdots,\xi_n\in\g^*,
\end{eqnarray}
where $\ad^*$ denotes both the coadjoint of $(\g,[\cdot,\cdots,\cdot])$ on $(\g^*,[\cdot,\cdots,\cdot]_{\g^*})$ and of $(\g^*,[\cdot,\cdots,\cdot]_{\g^*})$ on $(\g,[\cdot,\cdots,\cdot])$, i.e., for all
$x,x_1,\cdots,x_n\in \g, \xi,\xi_1,\cdots,\xi_n\in\g^*$,
\begin{eqnarray}
\langle~\ad^\ast_{x_1,\cdots,x_{n-1}}(\xi),x_n~\rangle&=&-\langle~\xi,[x_1,\cdots,x_{n-1},x_n]~\rangle,
\label{eq:ad*g}\\
\langle~\ad^\ast_{\xi_1,\cdots,\xi_{n-1}}(x),\xi_n~\rangle&=&-\langle~x,[\xi_1,\cdots,\xi_{n-1},\xi_n]_\g^*~\rangle.
\label{eq:ad*g*}
\end{eqnarray}

%\begin{rmk}
%Let $(\g,[\cdot,\cdot,\cdot],\gamma_{\g})$ be a 3-Lie bialgebra, and let $(\g^*,[\cdot,\cdot,\cdot]_{\g^*},\gamma_{\g^*})$  be its dual 3-Lie bialgebra. For all $~x_1,\cdots,x_3\in \g, \xi_1,\cdots,\xi_3\in\g^*$, the linear map $[\cdot,\cdot,\cdot]_\frkd:\wedge^3 \frkd\rightarrow \frkd$ is given by
%\begin{eqnarray*}
%  [x_1+\xi_1,x_2+\xi_2,x_3+\xi_3]_\frkd &=&
%  [x_1,x_2,x_3]+\ad^\ast_{x_2,x_{3}}\xi_1 -\ad^\ast_{x_1,x_{3}}\xi_2+\ad^\ast_{x_1,x_{2}}\xi_3\\ &+&[\xi_1,\xi_2,\xi_3]_{\g^\ast}+\ad^\ast_{\xi_2,\xi_{3}}x_1
%-\ad^\ast_{\xi_1,\xi_{3}}x_2+\ad^\ast_{\xi_1,\xi_{2}}x_3.
%\end{eqnarray*}
%\end{rmk}

\begin{defi}
Let $(\g,[\cdot,\cdots,\cdot])$ be an $n$-Lie algebra and $\gamma:\g\rightarrow\otimes^n\g$ be a linear map.  If $\gamma$ satisfies
\begin{equation}\label{eq:n-centroid}
  \gamma([x_1,x_2,\cdots,x_n]) = [\gamma(x_1),x_2,\cdots,x_n],\quad \forall~ x_1,\cdots,x_n\in\g,
\end{equation}
then we call $\gamma$ a {\bf centroid} map.
\end{defi}

\begin{defi}
Let $(\g,[\cdot,\cdots,\cdot])$ be an $n$-Lie algebra and $\gamma:\g\rightarrow\otimes^n\g$ be a linear map. For all $x_1,\cdots,x_n\in\g$, if $\gamma$ satisfies
\begin{eqnarray}
%\gamma_{\g} &=& \gamma_{R}^1, \label{eq:n-local-co}\\
(\otimes^{j-1}1\otimes\ad_{x_2,x_3,\cdots,x_n}\otimes^{n-j}1&+&
\otimes^{n-1}1\otimes\ad_{x_2,x_3,\cdots,x_n})\gamma(x_1)=0,\label{eq:1}\\
(\otimes^{i-1}1\otimes\ad_{x_2,x_3,\cdots,x_{n-1},x_n}\otimes^{n-i}1)\gamma(x_1)
&+&(\otimes^{k-1}1\otimes\ad_{x_2,x_3,\cdots,x_{n-1},x_1}\otimes^{n-k}1)\gamma(x_n)=0,\label{eq:2}
\end{eqnarray}
where $x_i\in \g$, $1\leq j \leq n-1$, $1\leq i,k\leq n, ~i\neq k$,
then we call $\gamma$ a {\bf local operad map}.
\end{defi}

\begin{pro}\label{pro:double-Lie}
Let  $(\g,[\cdot,\cdots,\cdot],\gamma_{\g})$ be a local centroid $R_1$-operad $n$-Lie bialgebra and define a linear map $[\cdot,\cdots,\cdot]_\frkd:\wedge^n \frkd\rightarrow \frkd$ by \eqref{eq:$n$-Lie-on-d}.
Then  $(\frkd,[\cdot,\cdots,\cdot]_\frkd)$ is an $n$-Lie algebra.
\end{pro}
\begin{proof}
By  the definition of $R_1$-operad $n$-Lie bialgebra, we have
\begin{equation}
 \gamma_\g([x_1,\cdots,x_n]) = \sum\limits_{i=1}^{n}(-1)^{n-i}(\ad_{x_1,\cdots,x_{i-1},
\hat{x_i},x_{i+1},\cdots,x_n}\otimes1\otimes\cdots\otimes1)\gamma_\g(x_i).
\end{equation}
On the one hand, for the tensor $\xi_1\otimes\cdots\otimes\xi_n\in \otimes^n\g^*$, we have
\begin{eqnarray*}
 \langle \gamma_\g([x_1,\cdots,x_n]), \xi_1\otimes\cdots\otimes\xi_n\rangle
%&=& \langle [x_1,\cdots,x_n], [\xi_1,\cdots,\xi_n]_{\g^*}\rangle\\
&=&-\langle  \ad_{\xi_1,\cdots,\xi_{n-1}}^\ast([x_1,\cdots,x_n]), \xi_n\rangle.
\end{eqnarray*}
On the other hand, by \eqref{eq:gamma-tran} and \eqref{eq:ad*g} we have
\begin{eqnarray*}
&&\sum\limits_{i=1}^{n}(-1)^{n-i}\langle(\ad_{x_1,\cdots,x_{i-1},
\hat{x_i},x_{i+1},\cdots,x_n}\otimes1\otimes\cdots\otimes1)\gamma_\g(x_i),
\xi_1\otimes\cdots\otimes\xi_n\rangle\\
%&=&\sum\limits_{i=1}^{n}(-1)^{n-i}(-1)^{n-1}\langle(\ad_{x_1,\cdots,x_{i-1},
%\hat{x_i},x_{i+1},\cdots,x_n}\otimes1\otimes\cdots\otimes1)\gamma_\g(x_i),
%\xi_n\otimes\cdots\otimes\xi_1\rangle\\
%&=&-\sum\limits_{i=1}^{n}(-1)^{i+1}\langle\gamma_\g(x_i),
%\ad_{x_1,\cdots,x_{i-1},\hat{x_i},x_{i+1},\cdots,x_n}^*(\xi_n)\otimes\cdots\otimes\xi_1\rangle\\
%&=&-\sum\limits_{i=1}^{n}(-1)^{i+1}\langle x_i,
%[\ad_{x_1,\cdots,x_{i-1},\hat{x_i},x_{i+1},\cdots,x_n}^*\xi_n,\cdots,\xi_1]_{\g^*}\rangle\\
%&=&-\sum\limits_{i=1}^{n}(-1)^{i+1}(-1)^{n-1}\langle x_i,
%[\xi_1,\cdots,\ad_{x_1,\cdots,x_{i-1},\hat{x_i},x_{i+1},\cdots,x_n}^*\xi_n]_{\g^*}\rangle\\
%&=&\sum\limits_{i=1}^{n}(-1)^{n-i}\langle \ad_{\xi_1,\cdots,\xi_{n-1}}^\ast(x_i),
%\ad_{x_1,\cdots,x_{i-1},\hat{x_i},x_{i+1},\cdots,x_n}^*\xi_n\rangle\\
%&=&-\sum\limits_{i=1}^{n}(-1)^{n-i}\langle [x_1,\cdots,x_{i-1},\hat{x_i},x_{i+1},\cdots,x_n,\ad_{\xi_1,\cdots,\xi_{n-1}}^\ast(x_i)],
%\xi_n\rangle\\
&=&-\sum\limits_{i=1}^{n}\langle[x_1,\cdots,x_{i-1},\ad_{\xi_1,\cdots,\xi_{n-1}}^\ast x_i,x_{i+1},\cdots,x_n], \xi_n\rangle.
\end{eqnarray*}
Then
\begin{eqnarray}
 \ad_{\xi_1,\cdots,\xi_{n-1}}^\ast([y_1,\cdots,y_n])
  = \sum\limits_{i=1}^{n}[y_1,\cdots,y_{i-1},\ad_{\xi_1,\cdots,\xi_{n-1}}^\ast y_i,y_{i+1},\cdots,y_n].\label{eqs:61}
\end{eqnarray}

By \eqref{eq:n-centroid}, for all $y_1,\cdots,y_n\in\g$ we have
\begin{equation}\label{eq:n-centroid-2}
 \gamma_\g([x_1,\cdots,x_{n-1},\sum_{i=1}^ny_i]) = \sum_{i=1}^n\sum\limits_{k=1}^{n}(\otimes^{k-1}1\otimes\ad_{x_1,
\cdots,x_{n-1}}\otimes^{n-k}1)\gamma_\g(y_i).
\end{equation}
For the tensor $\eta_1\otimes\cdots\otimes\eta_n\in \otimes^n\g^*$, the left-hand side of \eqref{eq:n-centroid-2} is equal to
\begin{eqnarray*}
&&\langle \gamma_\g([x_1,\cdots,x_{n-1},\sum_{i=1}^ny_i]), \eta_1\otimes\cdots\otimes\eta_n\rangle\\
%&=& \sum_{i=1}^n\langle [x_1,\cdots,x_{n-1},y_i], [\eta_1,\cdots,\eta_n]_{\g^*}\rangle\\
%&=& \sum_{i=1}^n\langle [x_1,\cdots,x_{n-1},y_i], (-1)^{n-i}[\eta_1,\cdots,\eta_{i-1},\hat{\eta}_i,\eta_{i+1},\cdots\eta_n,\eta_i]_{\g^*}\rangle\\
&=&-\sum_{i=1}^n(-1)^{n-i}\langle  \ad_{\eta_1,\cdots,\eta_{i-1},\hat{\eta}_i,\eta_{i+1},\cdots\eta_n}^\ast([x_1,\cdots,x_{n-1},y_i]), \eta_i\rangle.
\end{eqnarray*}
For the tensor $\eta_1\otimes\cdots\otimes\eta_n\in \otimes^n\g^*$,  the right-hand side of \eqref{eq:n-centroid-2} is equal to
\begin{eqnarray*}
&&\sum\limits_{i=1}^{n}\sum_{k=1}^{n}\langle (\otimes^{k-1}1\otimes\ad_{x_1,
\cdots,x_{n-1}}\otimes^{n-k}1)\gamma_\g(y_i),\eta_1\otimes\cdots\otimes\eta_n\rangle \\
%&=&-\sum\limits_{i=1}^{n}\sum_{k=1}^{n}\langle \gamma_\g(y_i),\eta_1\otimes\cdots\otimes\eta_{k-1}
% \otimes \ad_{x_1,\cdots,x_{n-1}}^*(\eta_k)\otimes \eta_{k+1}\otimes\cdots\otimes\eta_n\rangle \\
%&=&-\sum\limits_{i=1}^{n}\sum_{k=1}^{n}\langle y_i,[\eta_1,\cdots,\eta_{k-1},
% \ad_{x_1,\cdots,x_{n-1}}^*(\eta_k), \eta_{k+1},\cdots,\eta_n]_{\g^*}\rangle \\
%&=&-\sum\limits_{i=1}^{n}\sum_{k=1}^{n}\langle y_k,[\eta_1,\cdots,\eta_{i-1},
% \ad_{x_1,\cdots,x_{n-1}}^*(\eta_i), \eta_{i+1},\cdots,\eta_n]_{\g^*}\rangle \\
%&=&-\sum_{i=1}^{n}\sum_{k=1}^{i-1}(-1)^{(n-k)}\langle y_k,
%[\eta_1,\cdots,\eta_{k-1},\hat{\eta}_k,\eta_{k+1},\cdots,
% \ad_{x_1,\cdots,x_{n-1}}^*(\eta_i), \cdots,\eta_n,\eta_k]_{\g^*}\rangle \\
%&&-\sum_{i=1}^{n}\sum_{k=i+1}^{n}(-1)^{(n-k)}\langle y_k,
%[\eta_1,\cdots,\ad_{x_1,\cdots,x_{n-1}}^*(\eta_i), \cdots,\eta_{k-1},\hat{\eta}_k,\eta_{k+1},\cdots\eta_n,\eta_k]_{\g^*}\rangle\\
%&&-\sum_{i=1}^{n}(-1)^{(n-i)}\langle y_i,[\eta_1,\cdots,\eta_{i-1},\hat{\eta}_{i},
%  \eta_{i+1},\cdots,\eta_n,\ad_{x_1,\cdots,x_{n-1}}^*(\eta_i)]_{\g^*}\rangle \\
%&=&\sum\limits_{i=1}^{n}\sum_{k=1}^{i-1}(-1)^{(n-k)}\langle
%\ad^*_{\eta_1,\cdots,\eta_{k-1},\hat{\eta}_k,\eta_{k+1},\cdots,
% \ad_{x_1,\cdots,x_{n-1}}^*(\eta_i),\cdots,\eta_n}(y_k),
%\eta_k\rangle \\
%&&+\sum_{i=1}^{n}\sum_{k=i+1}^{n}(-1)^{(n-k)}\langle
%\ad^*_{\eta_1,\cdots,\ad_{x_1,\cdots,x_{n-1}}^*(\eta_i), \cdots,\eta_{k-1},\hat{\eta}_k,\eta_{k+1},\cdots\eta_n}
% (y_k),
%\eta_k\rangle\\
%&&+\sum_{i=1}^{n}(-1)^{(n-i)}\langle \ad_{\eta_1,\cdots,\eta_{i-1},\hat{\eta}_{i},
%  \eta_{i+1},\cdots,\eta_n}^* (y_i),\ad_{x_1,\cdots,x_{n-1}}^*(\eta_i)\rangle \\
&=&\sum\limits_{i=1}^{n}\sum_{k=1}^{i-1}(-1)^{(n-k)}\langle
\ad^*_{\eta_1,\cdots,\eta_{k-1},\hat{\eta}_k,\eta_{k+1},\cdots,
 \ad_{x_1,\cdots,x_{n-1}}^*(\eta_i),\cdots,\eta_n}(y_k),
\eta_k\rangle \\
&&+\sum_{i=1}^{n}\sum_{k=i+1}^{n}(-1)^{(n-k)}\langle
\ad^*_{\eta_1,\cdots,\ad_{x_1,\cdots,x_{n-1}}^*(\eta_i), \cdots,\eta_{k-1},\hat{\eta}_k,\eta_{k+1},\cdots\eta_n}
 (y_k),
\eta_k\rangle\\
&&-\sum_{i=1}^{n}(-1)^{(n-i)}\langle [x_1,\cdots,x_{n-1},\ad_{\eta_1,\cdots,\eta_{i-1},\hat{\eta}_{i},
  \eta_{i+1},\cdots,\eta_n}^* (y_i)],\eta_i\rangle. \\
\end{eqnarray*}
Thus, we can obtain
\begin{eqnarray}
% \nonumber to remove numbering (before each equation)
  \label{eqs:62}&&{[x_1,\cdots,x_{n-1},\sum\limits_{i=1}^{n}(-1)^{n-i}\ad_{\eta_1,\cdots,\eta_{i-1},\hat{\eta_i},\eta_{i+1},\cdots,\eta_n}^\ast y_i]}
\end{eqnarray}
\begin{eqnarray*}
% \nonumber to remove numbering (before each equation)
  \nonumber=&& \sum\limits_{i=1}^{n}(-1)^{n-i}\ad_{\eta_1,\cdots,\eta_{i-1},\hat{\eta_i},\eta_{i+1},\cdots,\eta_n}^\ast([x_1,\cdots,x_{n-1},y_i]) \\
  \nonumber+&&\sum\limits_{i=1}^{n}\sum\limits_{k=1}^{i-1}(-1)^{n-k}\ad_{\eta_1,\cdots,\eta_{k-1},\hat{\eta_k},\eta_{k+1},\cdots,\eta_{i-1},\ad_{x_1,\cdots,x_{n-1}}^\ast\eta_i,\eta_{i+1},\cdots,\eta_n}^\ast y_k \\
  \nonumber+&&\sum\limits_{i=1}^{n}\sum\limits_{k=i+1}^{n}(-1)^{n-k}\ad_{\eta_1,\cdots,\eta_{i-1},\ad_{x_1,\cdots,x_{n-1}}^\ast\eta_i,\eta_{i+1},\cdots,\eta_{k-1},\hat{\eta_k},\eta_{k+1},\cdots,\eta_n}^\ast y_k .
\end{eqnarray*}

By \eqref{eq:1}, for the integer $1\leq i \leq n$, $y_i\in \g$, we can get the following $n(n-1)$ equations
\begin{eqnarray*}
% \nonumber to remove numbering (before each equation)
  (\ad_{y_1,\cdots,y_{i-1},\hat{y_i},y_{i+1},\cdots,y_n}\otimes^{n-1}1+\otimes^{n-1}1\otimes\ad_{y_1,\cdots,y_{i-1},\hat{y_i},y_{i+1},\cdots,y_n})\gamma(x_1) &=& 0, \\
  (1\otimes\ad_{y_1,\cdots,y_{i-1},\hat{y_i},y_{i+1},\cdots,y_n}\otimes^{n-2}1+\otimes^{n-1}1\otimes\ad_{y_1,\cdots,y_{i-1},\hat{y_i},y_{i+1},\cdots,y_n})\gamma(x_2) &=& 0, \\
  &\cdots&  \\
  (\otimes^{n-2}1\otimes\ad_{y_1,\cdots,y_{i-1},\hat{y_i},y_{i+1},\cdots,y_n}\otimes1+\otimes^{n-1}1\otimes\ad_{y_1,\cdots,y_{i-1},\hat{y_i},y_{i+1},\cdots,y_n})\gamma(x_{n-1}) &=& 0.
\end{eqnarray*}
Sum over the above equations, we can obtain
\begin{equation}\label{eq:3}
  \sum\limits_{i=1}^{n}\sum\limits_{j=1}^{n-1}(-1)^{n-i}\big((\otimes^{j-1}1\otimes\ad_{y_1,\cdots,\hat{y_i},\cdots,y_n}\otimes^{n-j}1) +(\otimes^{n-1}1\otimes\ad_{y_1,\cdots,\hat{y_i},\cdots,y_n})\big)\gamma(x_j)
  =0.
\end{equation}
For the tensor $\xi_1\otimes\cdots\otimes\xi_{n-1}\otimes\eta_i \in \otimes^n\g^\ast$, the left-hand side of \eqref{eq:3} is equal to
\begin{eqnarray*}
% \nonumber to remove numbering (before each equation)
&&\langle~\sum\limits_{i=1}^{n}\sum\limits_{j=1}^{n-1}(-1)^{n-i}\big((\otimes^{j-1}1\otimes\ad_{y_1,\cdots,\hat{y_i},\cdots,y_n}\otimes^{n-j}1)\\
&&+ (\otimes^{n-1}1\otimes\ad_{y_1,\cdots,\hat{y_i},\cdots,y_n})\big)\gamma(x_j), \xi_1\otimes\cdots\otimes\xi_{n-1}\otimes\eta_i~\rangle\\
%&=&-\sum\limits_{i=1}^{n}\sum\limits_{j=1}^{n-1}(-1)^{n-i}\langle~x_j,[\xi,\cdots,\xi_{j-1},\ad_{y_1,\cdots,\hat{y_i},\cdots,y_n}^\ast\xi_j,\xi_{j+1},\cdots,\xi_{n-1},\eta_i]_{\g^\ast}\\
%&&+[\xi,\cdots,\xi_{n-1},\ad_{y_1,\cdots,\hat{y_i},\cdots,y_n}^\ast\eta_i]_{\g^\ast}~\rangle \\
%&=&-\sum\limits_{i=1}^{n}\sum\limits_{j=1}^{n-1}(-1)^{n-i}(-1)^{n-j}\langle~x_j,[\xi,\cdots,\xi_{j-1},\hat{\xi_j},\xi_{j+1},\cdots,\xi_{n-1},\eta_i,\ad_{y_1,\cdots,\hat{y_i},\cdots,y_n}^\ast\xi_j]_{\g^\ast}\\
%&&+[\xi,\cdots,\xi_{j-1},\hat{\xi_j},\xi_{j+1},\cdots,\xi_{n-1},\ad_{y_1,\cdots,\hat{y_i},\cdots,y_n}^\ast\eta_i,\xi_j]_{\g^\ast}~\rangle\\
%&=&\sum\limits_{i=1}^{n}\sum\limits_{j=1}^{n-1}(-1)^{n-i}(-1)^{n-j}(\langle~\ad^\ast_{\xi,\cdots,\xi_{j-1},\hat{\xi_j},\xi_{j+1},\cdots,\xi_{n-1},\eta_i}x_j,\ad_{y_1,\cdots,\hat{y_i},\cdots,y_n}^\ast\xi_j~\rangle\\
%&&+\langle~\ad^\ast_{\xi,\cdots,\xi_{j-1},\hat{\xi_j},\xi_{j+1},\cdots,\xi_{n-1},\ad_{y_1,\cdots,\hat{y_i},\cdots,y_n}^\ast\eta_i}x_j,\xi_j~\rangle)\\
%&=&-\sum\limits_{i=1}^{n}\sum\limits_{j=1}^{n-1}(-1)^{n-i}(-1)^{n-j}\langle~[y_1,\cdots,\hat{y_i},\cdots,y_n,\ad^\ast_{\xi,\cdots,\xi_{j-1},\hat{\xi_j},\xi_{j+1},\cdots,\xi_{n-1},\eta_i}x_j],\xi_j~\rangle\\
%\end{eqnarray*}
%\begin{eqnarray*}
%&&+\sum\limits_{i=1}^{n}\sum\limits_{j=1}^{n-1}(-1)^{n-i}(-1)^{n-j}\langle~\ad^\ast_{\xi,\cdots,\xi_{j-1},\hat{\xi_j},\xi_{j+1},\cdots,\xi_{n-1},\ad_{y_1,\cdots,\hat{y_i},\cdots,y_n}^\ast\eta_i}x_j,\xi_j~\rangle)\\
&=&-\sum\limits_{i=1}^{n}\sum\limits_{j=1}^{n-1}(-1)^{n-j}\langle~[y_1,\cdots,y_{i-1},\ad^\ast_{\xi,\cdots,\xi_{j-1},\hat{\xi_j},\xi_{j+1},\cdots,\xi_{n-1},\eta_i}x_j,y_{i+1},\cdots,y_n],\xi_j~\rangle\\
&&+\sum\limits_{i=1}^{n}\sum\limits_{j=1}^{n-1}(-1)^{n-i}(-1)^{n-j}\langle~\ad^\ast_{\xi,\cdots,\xi_{j-1},\hat{\xi_j},\xi_{j+1},\cdots,\xi_{n-1},\ad_{y_1,\cdots,y_{i-1},\hat{y_i},y_{i+1},\cdots,y_n}^\ast\eta_i}x_j,\xi_j~\rangle).
\end{eqnarray*}

The right-hand side of \eqref{eq:3} is equal to $0$, then we have
\begin{eqnarray}\label{eqs:1}
  &&\sum\limits_{j=1}^{n-1}(-1)^{n-j}\ad_{\xi_1,\cdots,\xi_{j-1},\hat{\xi_j},\xi_{j+1},\cdots,\xi_{n-1},\sum\limits_{i=1}^{n}(-1)^{n-i}\ad_{y_1,\cdots,y_{i-1},\hat{y_i},y_{i+1},\cdots,y_n}^\ast \eta_i}^\ast x_j \\
  \nonumber=&& \sum\limits_{i=1}^{n}[y_1,\cdots,y_{i-1},\sum\limits_{j=1}^{n-1}(-1)^{n-j}\ad_{\xi_1,\cdots,\xi_{j-1},\hat{\xi_j},\xi_{j+1},\cdots,\xi_{n-1},\eta_i}^\ast x_j,y_{i+1},\cdots,y_n].
\end{eqnarray}
By \eqref{eq:2}, for the integer $1\leq j \leq n$, $y_i\in \g$, $i\neq k$ we can get the following $n(n-1)$ equations:
\begin{eqnarray*}
% \nonumber to remove numbering (before each equation)
  (\otimes^{i-1}1\otimes\ad_{x_1,\cdots,\hat{x_j},\cdots,x_{n-1},y_i}\otimes^{n-i}1)\gamma(y_1)
  +(\otimes^{k-1}1\otimes\ad_{x_1,\cdots,\hat{x_j},\cdots,x_{n-1},y_1}\otimes^{n-k}1)\gamma(y_i) &=& 0, \\
  (\otimes^{i-1}1\otimes\ad_{x_1,\cdots,\hat{x_j},\cdots,x_{n-1},y_i}\otimes^{n-i}1)\gamma(y_2)
  +(\otimes^{k-1}1\otimes\ad_{x_1,\cdots,\hat{x_j},\cdots,x_{n-1},y_2}\otimes^{n-k}1)\gamma(y_i) &=& 0, \\
   &\cdots&  \\
  (\otimes^{i-1}1\otimes\ad_{x_1,\cdots,\hat{x_j},\cdots,x_{n-1},y_i}\otimes^{n-i}1)\gamma(y_n)
  +(\otimes^{k-1}1\otimes\ad_{x_1,\cdots,\hat{x_j},\cdots,x_{n-1},y_n}\otimes^{n-k}1)\gamma(y_i) &=& 0.
\end{eqnarray*}
sum over the above equations, for all $i=1,2,\cdots,n$, we can obtain
\begin{equation*}
  \sum\limits_{j=1}^{n-1}\sum\limits_{k=1,k\neq i}^{n}\big((\otimes^{i-1}1\otimes\ad_{x_1,\cdots,\hat{x_j},\cdots,x_{n-1},y_i}\otimes^{n-i}1)\gamma(y_1)
  +(\otimes^{k-1}1\otimes\ad_{x_1,\cdots,\hat{x_j},\cdots,x_{n-1},y_1}\otimes^{n-k}1)\gamma(y_i)\big)=0,
\end{equation*}
thus, we have
\begin{eqnarray}\label{eq:4}
% \nonumber to remove numbering (before each equation)
  &&\sum\limits_{i=1}^{n}\sum\limits_{j=1}^{n-1}\sum\limits_{k=1}^{n}(-1)^{n-j}(\otimes^{i-1}1\otimes\ad_{x_1,\cdots,x_{j-1},\hat{x_j},x_{j+1},\cdots,x_{n-1},y_i}\otimes^{n-i}1)\gamma(y_k)  \\
  \nonumber&=&\sum\limits_{i=1}^{n}\sum\limits_{j=1}^{n-1}(-1)^{n-j}(\otimes^{i-1}1\otimes\ad_{x_1,\cdots,x_{j-1},\hat{x_j},x_{j+1},\cdots,x_{n-1},y_i}\otimes^{n-i}1)\gamma(y_i).
\end{eqnarray}
For the tensor $\eta_1\otimes\cdots\otimes\eta_{i-1}\otimes\xi_j\otimes\eta_{i+1}\otimes\cdots\otimes\eta_n \in \otimes^n\g^\ast$, the left-hand side of \eqref{eq:4} is equal to
\begin{eqnarray*}
% \nonumber to remove numbering (before each equation)
  &&\sum\limits_{i=1}^{n}\sum\limits_{j=1}^{n-1}\sum\limits_{k=1}^{n}(-1)^{n-j}\langle~(\otimes^{i-1}1\otimes\ad_{x_1,\cdots,x_{j-1},\hat{x_j},x_{j+1},\cdots,x_{n-1},y_i}\otimes^{n-i}1)\gamma(y_k),\\
  &&\eta_1\otimes\cdots\otimes\eta_{i-1}\otimes\xi_j\otimes\eta_{i+1}\otimes\cdots\otimes\eta_n~\rangle\\
%\end{eqnarray*}
%\begin{eqnarray*}
 % &=&-\sum\limits_{i=1}^{n}\sum\limits_{j=1}^{n-1}\sum\limits_{k=1}^{i-1}(-1)^{n-j}\langle~y_k,[\eta_1,\cdots,\eta_k,\cdots,\eta_{i-1},\ad_{x_1,\cdots,x_{j-1},\hat{x_j},x_{j+1},\cdots,x_{n-1},y_i}\xi_j,\eta_{i+1},\cdots,\eta_n]_{\g^\ast}~\rangle  \\
%  &&-\sum\limits_{i=1}^{n}\sum\limits_{j=1}^{n-1}\sum\limits_{k=i+1}^{n}(-1)^{n-j}\langle~y_k,[\eta_1,\cdots,\eta_{i-1},\ad_{x_1,\cdots,x_{j-1},\hat{x_j},x_{j+1},\cdots,x_{n-1},y_i}\xi_j,\eta_{i+1},\cdots,\eta_k,\cdots,\eta_n]_{\g^\ast}~\rangle  \\
%  &&-\sum\limits_{i=1}^{n}\sum\limits_{j=1}^{n-1}(-1)^{n-j}\langle~y_i,[\eta_1,\cdots,\eta_{i-1},\ad_{x_1,\cdots,x_{j-1},\hat{x_j},x_{j+1},\cdots,x_{n-1},y_i}\xi_j,\eta_{i+1},\cdots,\eta_n]_{\g^\ast}~\rangle \\
%  &=&-\sum\limits_{i=1}^{n}\sum\limits_{j=1}^{n-1}\sum\limits_{k=1}^{i-1}(-1)^{n-j}(-1)^{n-k}\langle~y_k,[\eta_1,\cdots,\hat{\eta_k},\cdots,\eta_{i-1},\ad_{x_1,\cdots,\hat{x_j},\cdots,x_{n-1},y_i}\xi_j,\eta_{i+1},\cdots,\eta_n,\eta_k]_{\g^\ast}~\rangle \\
%  &&-\sum\limits_{i=1}^{n}\sum\limits_{j=1}^{n-1}\sum\limits_{k=i+1}^{n}(-1)^{n-j}(-1)^{n-k}\langle~y_k,[\eta_1,\cdots,\eta_{i-1},\ad_{x_1,\cdots,\hat{x_j},\cdots,x_{n-1},y_i}\xi_j,\eta_{i+1},\cdots,\hat{\eta_k},\cdots,\eta_n,\eta_k]_{\g^\ast}~\rangle
%  \\
%  &&-\sum\limits_{i=1}^{n}\sum\limits_{j=1}^{n-1}(-1)^{n-j}\langle~y_i,[\eta_1,\cdots,\eta_{i-1},\ad_{x_1,\cdots,x_{j-1},\hat{x_j},x_{j+1},\cdots,x_{n-1},y_i}\xi_j,\eta_{i+1},\cdots,\eta_n]_{\g^\ast}~\rangle \\
  &=&\sum\limits_{i=1}^{n}\sum\limits_{j=1}^{n-1}\sum\limits_{k=1}^{i-1}(-1)^{n-j}(-1)^{n-k}\langle~\ad^\ast_{\eta_1,\cdots,\hat{\eta_k},\cdots,\eta_{i-1},\ad_{x_1,\cdots,x_{j-1},\hat{x_j},x_{j+1},\cdots,x_{n-1},y_i}\xi_j,\eta_{i+1},\cdots,\eta_n}y_k,\eta_k~\rangle \\
  &&+\sum\limits_{i=1}^{n}\sum\limits_{j=1}^{n-1}\sum\limits_{k=i+1}^{n}(-1)^{n-j}(-1)^{n-k}\langle~\ad^\ast_{\eta_1,\cdots,\eta_{i-1},\ad_{x_1,\cdots,x_{j-1},\hat{x_j},x_{j+1},\cdots,x_{n-1},y_i}\xi_j,\eta_{i+1},\cdots,\hat{\eta_k},\cdots,\eta_n}y_k,\eta_k~\rangle
  \\
  &&-\sum\limits_{i=1}^{n}\sum\limits_{j=1}^{n-1}(-1)^{n-j}\langle~y_i,[\eta_1,\cdots,\eta_{i-1},\ad_{x_1,\cdots,x_{j-1},\hat{x_j},x_{j+1},\cdots,x_{n-1},y_i}\xi_j,\eta_{i+1},\cdots,\eta_n]_{\g^\ast}~\rangle,
\end{eqnarray*}
the right-hand side of \eqref{eq:4} is equal to
\begin{eqnarray*}
% \nonumber to remove numbering (before each equation)
\sum\limits_{i=1}^{n}\sum\limits_{j=1}^{n-1}(-1)^{n-j}&&\langle~(\otimes^{i-1}1\otimes\ad_{x_1,\cdots,x_{j-1},\hat{x_j},x_{j+1},\cdots,x_{n-1},y_i}\otimes^{n-i}1)\gamma(y_i),\\
&&\eta_1\otimes\cdots\otimes\eta_{i-1}\otimes\xi_j\otimes\eta_{i+1}\otimes\cdots\otimes\eta_n~\rangle \\
=-\sum\limits_{i=1}^{n}\sum\limits_{j=1}^{n-1}(-1)^{n-j}&&\langle~y_i,[\eta_1,\cdots,\eta_{i-1},\ad_{x_1,\cdots,x_{j-1},\hat{x_j},x_{j+1},\cdots,x_{n-1},y_i}\xi_j,\eta_{i+1},\cdots,\eta_n]_{\g^\ast}~\rangle.
\end{eqnarray*}
Then we have
\begin{eqnarray}\label{eqs:2}
% \nonumber to remove numbering (before each equation)
  &&\sum\limits_{i=1}^{n}\sum\limits_{k=1}^{i-1}(-1)^{n-k}\ad_{\eta_1,\cdots,\eta_{k-1},\hat{\eta_k},\eta_{k+1},\cdots,\eta_{i-1},\sum\limits_{j=1}^{n-1}(-1)^{n-j}\ad_{x_1,\cdots,x_{j-1},\hat{x_j},x_{j+1},\cdots,x_{n-1},y_i}^\ast\xi_j,\eta_{i+1},\cdots,\eta_n}^\ast y_k \\
  \nonumber+&& \sum\limits_{i=1}^{n}\sum\limits_{k=i+1}^{n}(-1)^{n-k}\ad_{\eta_1,\cdots,\eta_{i-1},\sum\limits_{j=1}^{n-1}(-1)^{n-j}\ad_{x_1,\cdots,x_{j-1},\hat{x_j},x_{j+1},\cdots,x_{n-1},y_i}^\ast\xi_j,\eta_{i+1},\cdots,\eta_{k-1},\hat{\eta_k},\eta_{k+1},\cdots,\eta_n}^\ast y_k \\
  \nonumber=&&0.
\end{eqnarray}
Similarly, we can obtain
\begin{eqnarray}
% \nonumber to remove numbering (before each equation)
  &&\ad_{x_1,\cdots,x_{n-1}}^\ast([\eta_1,\cdots,\eta_n]_{\g^\ast}) \\
  \nonumber=&& \sum\limits_{i=1}^{n}[\eta_1,\cdots,\eta_{i-1},\ad_{x_1,\cdots,x_{n-1}}^\ast\eta_i,\eta_{i+1},\cdots,\eta_n]_{\g^\ast};\label{eqs:64} \\
  &&{[\xi_1,\cdots,\xi_{n-1},\sum\limits_{i=1}^{n}(-1)^{n-i}\ad_{y_1,\cdots,y_{i-1},\hat{y_i},y_{i+1},\cdots,y_n}^\ast \eta_i]_{\g^\ast}} \\
  \nonumber=&& \sum\limits_{i=1}^{n}(-1)^{n-i}\ad_{y_1,\cdots,y_{i-1},\hat{y_i},y_{i+1},\cdots,y_n}^\ast([\xi_1,\cdots,\xi_{n-1},\eta_i]_{\g^\ast}) \\
  \nonumber+&&\sum\limits_{i=1}^{n}\sum\limits_{k=1}^{i-1}(-1)^{n-k}\ad_{y_1,\cdots,y_{k-1},\hat{y_k},y_{k+1},\cdots,y_{i-1},\ad_{\xi_1,\cdots,\xi_{n-1}}^\ast y_i,y_{i+1},\cdots,y_n}^\ast \eta_k \\
  \nonumber+&&\sum\limits_{i=1}^{n}\sum\limits_{k=i+1}^{n}(-1)^{n-k}\ad_{y_1,\cdots,y_{i-1},\ad_{\xi_1,\cdots,\xi_{n-1}}^\ast y_i,y_{i+1},\cdots,y_{k-1},\hat{y_k},y_{k+1},\cdots,y_n}^\ast \eta_k;\label{eqs:65} \\
  &&\sum\limits_{j=1}^{n-1}(-1)^{n-j}\ad_{x_1,\cdots,x_{j-1},\hat{x_j},x_{j+1},\cdots,x_{n-1},\sum\limits_{i=1}^{n}(-1)^{n-i}\ad_{\eta_1,\cdots,\eta_{i-1},\hat{\eta_i},\eta_{i+1},\cdots,\eta_n}^\ast y_i}^\ast \xi_j \\
  \nonumber=&& \sum\limits_{i=1}^{n}[\eta_1,\cdots,\eta_{i-1},\sum\limits_{j=1}^{n-1}(-1)^{n-j}\ad_{x_1,\cdots,x_{j-1},\hat{x_j},x_{j+1},\cdots,x_{n-1},y_i}^\ast \xi_j,\eta_{i+1},\cdots,\eta_n]_{\g^\ast};\label{eqs:66} \\
  \label{eqs:67}&&\sum\limits_{i=1}^{n}\sum\limits_{k=1}^{i-1}(-1)^{n-k}\ad_{y_1,\cdots,y_{k-1},\hat{y_k},y_{k+1},\cdots,y_{i-1},\sum\limits_{j=1}^{n-1}(-1)^{n-j}\ad_{\xi_1,\cdots,\xi_{j-1},\hat{\xi_j},\xi_{j+1},\cdots,\xi_{n-1},\eta_i}^\ast x_j,y_{i+1},\cdots,y_n}^\ast \eta_k \\
  \nonumber+&&\sum\limits_{i=1}^{n}\sum\limits_{k=i+1}^{n}(-1)^{n-k}\ad_{y_1,\cdots,y_{i-1},\sum\limits_{j=1}^{n-1}(-1)^{n-j}\ad_{\xi_1,\cdots,\xi_{j-1},\hat{\xi_j},\xi_{j+1},\cdots,\xi_{n-1},\eta_i}^\ast x_j,y_{i+1},\cdots,y_{k-1},\hat{y_k},y_{k+1},\cdots,y_n}^\ast \eta_k \\
  \nonumber=&&
  0.
\end{eqnarray}

By \eqref{eq:FJi}, for all $y_1,\cdots,y_n \in \g,\xi_1,\cdots,\xi_{n-1},\eta_1,\cdots,\eta_n \in \g^\ast$ we have
\begin{eqnarray}\label{9}
% \nonumber to remove numbering (before each equation)
  && \langle~\sum\limits_{i=1}^{n}y_i,[\xi_1,\cdots,\xi_{n-1},[\eta_1,\cdots,\eta_n]_{\g^\ast}]_{\g^\ast}~\rangle \\
  \nonumber=&&\langle~\sum\limits_{i=1}^{n}y_i,\sum\limits_{k=1}^{n}[\eta_1,\cdots,\eta_{k-1},[\xi_1,\cdots,\xi_{n-1},\eta_k]_{\g^\ast},\eta_{k+1},\cdots,\eta_n]_{\g^\ast}~\rangle.
\end{eqnarray}
The left-hand side of \eqref{9} is equal to
\begin{eqnarray*}
  &&\langle~\sum\limits_{i=1}^{n}y_i,[\xi_1,\cdots,\xi_{n-1},[\eta_1,\cdots,\eta_n]_{\g^\ast}]_{\g^\ast}~\rangle \\
%\end{equation*}
%\begin{eqnarray*}
%% \nonumber to remove numbering (before each equation)
%  =&& \langle~-\sum\limits_{i=1}^{n}\ad^\ast_{\xi_1,\cdots,\xi_{n-1}}y_i,[\eta_1,\cdots,\eta_n]_{\g^\ast}~\rangle \\
%  =&& \langle~-\sum\limits_{i=1}^{n}\ad^\ast_{\xi_1,\cdots,\xi_{n-1}}y_i,(-1)^{n-i}[\eta_1,\cdots,\eta_{i-1},\hat\eta_i,\eta_{i+1},\cdots,\eta_n,\eta_i]_{\g^\ast}~\rangle \\
  =&& \langle~\sum\limits_{i=1}^{n}(-1)^{n-i}\ad^\ast_{\eta_1,\cdots,\eta_{i-1},\hat\eta_i,\eta_{i+1},\cdots,\eta_n}(\ad^\ast_{\xi_1,\cdots,\xi_{n-1}}y_i),\eta_i~\rangle,
\end{eqnarray*}
the right-hand side of \eqref{9} is equal to
\begin{eqnarray*}
% \nonumber to remove numbering (before each equation)
  && \langle~\sum\limits_{i=1}^{n}y_i,\sum\limits_{k=1}^{n}[\eta_1,\cdots,\eta_{k-1},[\xi_1,\cdots,\xi_{n-1},\eta_k]_{\g^\ast},\eta_{k+1},\cdots,\eta_n]_{\g^\ast}~\rangle \\
  %=&& \langle~\sum\limits_{k=1}^{n}y_k,\sum\limits_{i=1}^{n}[\eta_1,\cdots,\eta_{i-1},[\xi_1,\cdots,\xi_{n-1},\eta_i]_{\g^\ast},\eta_{i+1},\cdots,\eta_n]_{\g^\ast}~\rangle \\
%  =&& \langle~y_i,\sum\limits_{i=1}^{n}[\eta_1,\cdots,\eta_{i-1},[\xi_1,\cdots,\xi_{n-1},\eta_i]_{\g^\ast},\eta_{i+1},\cdots,\eta_n]_{\g^\ast}~\rangle \\
%  +&& \langle~\sum\limits_{k=1}^{i-1}y_k,\sum\limits_{i=1}^{n}[\eta_1,\cdots,\eta_{i-1},[\xi_1,\cdots,\xi_{n-1},\eta_i]_{\g^\ast},\eta_{i+1},\cdots,\eta_n]_{\g^\ast}~\rangle
%  \\
%  +&& \langle~\sum\limits_{k=i+1}^{n}y_k,\sum\limits_{i=1}^{n}[\eta_1,\cdots,\eta_{i-1},[\xi_1,\cdots,\xi_{n-1},\eta_i]_{\g^\ast},\eta_{i+1},\cdots,\eta_n]_{\g^\ast}~\rangle
%  \\
%  =&& \langle~y_i,\sum\limits_{i=1}^{n}[\eta_1,\cdots,\eta_{i-1},[\xi_1,\cdots,\xi_{n-1},\eta_i]_{\g^\ast},\eta_{i+1},\cdots,\eta_n]_{\g^\ast}~\rangle \\
%  +&& \langle~\sum\limits_{i=1}^{n}\sum\limits_{k=1}^{i-1}y_k,(-1)^{n-k}[\eta_1,\cdots,\hat\eta_k,\cdots,\eta_{i-1},[\xi_1,\cdots,\xi_{n-1},\eta_i]_{\g^\ast},\eta_{i+1},\cdots,\eta_n,\eta_k]_{\g^\ast}~\rangle
%  \\
%  +&& \langle~\sum\limits_{i=1}^{n}\sum\limits_{k=i+1}^{n}y_k,(-1)^{n-k}[\eta_1,\cdots,\eta_{i-1},[\xi_1,\cdots,\xi_{n-1},\eta_i]_{\g^\ast},\eta_{i+1},\cdots,\hat\eta_k,\cdots,\eta_n,\eta_k]_{\g^\ast}~\rangle
%  \\
%  =&& \langle~\sum\limits_{i=1}^{n}y_i,[\eta_1,\cdots,\eta_{i-1},[\xi_1,\cdots,\xi_{n-1},\eta_i]_{\g^\ast},\eta_{i+1},\cdots,\eta_n]_{\g^\ast}~\rangle \\
%  -&&\langle~\sum\limits_{i=1}^{n}\sum\limits_{k=1}^{i-1}(-1)^{n-k}\ad^\ast_{\eta_1,\cdots,\hat\eta_k,\cdots,\eta_{i-1},[\xi_1,\cdots,\xi_{n-1},\eta_i]_{\g^\ast},\eta_{i+1},\cdots,\eta_n}y_k,\eta_k~\rangle
%  \\
%  -&&\langle~\sum\limits_{i=1}^{n}\sum\limits_{k=i+1}^{n}(-1)^{n-k}\ad^\ast_{\eta_1,\cdots,\eta_{i-1},[\xi_1,\cdots,\xi_{n-1},\eta_i]_{\g^\ast},\eta_{i+1},\cdots,\hat\eta_k,\cdots,\eta_n}y_k,\eta_k~\rangle
%  \\
%=&& \langle~\sum\limits_{i=1}^{n}(-1)^{n-i}y_i,[\eta_1,\cdots,\hat\eta_i,\cdots,\eta_n,[\xi_1,\cdots,\xi_{n-1},\eta_i]_{\g^\ast}]_{\g^\ast}~\rangle \\
%  -&&\langle~\sum\limits_{i=1}^{n}\sum\limits_{k=1}^{i-1}(-1)^{n-k}\ad^\ast_{\eta_1,\cdots,\hat\eta_k,\cdots,\eta_{i-1},[\xi_1,\cdots,\xi_{n-1},\eta_i]_{\g^\ast},\eta_{i+1},\cdots,\eta_n}y_k,\eta_k~\rangle
%  \\
%  -&&\langle~\sum\limits_{i=1}^{n}\sum\limits_{k=i+1}^{n}(-1)^{n-k}\ad^\ast_{\eta_1,\cdots,\eta_{i-1},[\xi_1,\cdots,\xi_{n-1},\eta_i]_{\g^\ast},\eta_{i+1},\cdots,\hat\eta_k,\cdots,\eta_n}y_k,\eta_k~\rangle\\
%% \nonumber to remove numbering (before each equation)
  =&& \langle~\ad^\ast_{\xi_1,\cdots,\xi_{n-1}}(\sum\limits_{i=1}^{n}(-1)^{n-i}\ad^\ast_{\eta_1,\cdots,\hat\eta_i,\cdots,\eta_n}y_i),\eta_i~\rangle \\
  -&&\langle~\sum\limits_{i=1}^{n}\sum\limits_{k=1}^{i-1}(-1)^{n-k}\ad^\ast_{\eta_1,\cdots,\hat\eta_k,\cdots,\eta_{i-1},[\xi_1,\cdots,\xi_{n-1},\eta_i]_{\g^\ast},\eta_{i+1},\cdots,\eta_n}y_k,\eta_k~\rangle
  \\
  -&&\langle~\sum\limits_{i=1}^{n}\sum\limits_{k=i+1}^{n}(-1)^{n-k}\ad^\ast_{\eta_1,\cdots,\eta_{i-1},[\xi_1,\cdots,\xi_{n-1},\eta_i]_{\g^\ast},\eta_{i+1},\cdots,\hat\eta_k,\cdots,\eta_n}y_k,\eta_k~\rangle.
\end{eqnarray*}
Then we have
\begin{eqnarray}\label{eq:9}
% \nonumber to remove numbering (before each equation)
  &&\ad^\ast_{\xi_1,\cdots,\xi_{n-1}}(\sum\limits_{i=1}^{n}(-1)^{n-i}\ad^\ast_{\eta_1,\cdots,\hat\eta_i,\cdots,\eta_n}y_i)
  = \sum\limits_{i=1}^{n}(-1)^{n-i}\ad^\ast_{\eta_1,\cdots,\hat\eta_i,\cdots,\eta_n}(\ad^\ast_{\xi_1,\cdots,\xi_{n-1}}y_i) \\
  \nonumber&+& \sum\limits_{i=1}^{n}\sum\limits_{k=1}^{i-1}(-1)^{n-k}\ad^\ast_{\eta_1,\cdots,\hat\eta_k,\cdots,\eta_{i-1},[\xi_1,\cdots,\xi_{n-1},\eta_i]_{\g^\ast},\eta_{i+1},\cdots,\eta_n}y_k \\
  \nonumber&+& \sum\limits_{i=1}^{n}\sum\limits_{k=i+1}^{n}(-1)^{n-k}\ad^\ast_{\eta_1,\cdots,\eta_{i-1},[\xi_1,\cdots,\xi_{n-1},\eta_i]_{\g^\ast},\eta_{i+1},\cdots,\hat\eta_k,\cdots,\eta_n}y_k.
\end{eqnarray}
Similarly, we can obtain
\begin{eqnarray}\label{eq:10}
% \nonumber to remove numbering (before each equation)
  &&\ad^\ast_{x_1,\cdots,x_{n-1}}(\sum\limits_{i=1}^{n}(-1)^{n-i}\ad^\ast_{y_1,\cdots,\hat y_i,\cdots,y_n}\eta_i)
  \nonumber= \sum\limits_{i=1}^{n}(-1)^{n-i}\ad^\ast_{y_1,\cdots,y_{i-1},\hat y_i,y_{i+1},\cdots,y_n}(\ad^\ast_{x_1,\cdots,x_{n-1}}\eta_i) \\
  \nonumber&+& \sum\limits_{i=1}^{n}\sum\limits_{k=1}^{i-1}(-1)^{n-k}\ad^\ast_{y_1,\cdots,\hat y_k,\cdots,y_{i-1},[x_1,\cdots,x_{n-1},y_i],y_{i+1},\cdots,y_n}\eta_k \\
  \nonumber&+& \sum\limits_{i=1}^{n}\sum\limits_{k=i+1}^{n}(-1)^{n-k}\ad^\ast_{y_1,\cdots,y_{i-1},[x_1,\cdots,x_{n-1},y_i],y_{i+1},\cdots,\hat y_k,\cdots,y_n}\eta_k.
\end{eqnarray}

By \eqref{eq:FJi}, we have
\begin{eqnarray*}
% \nonumber to remove numbering (before each equation)
  &&[\xi_1,\cdots,\xi_{n-1},[\eta_1,\cdots,\eta_n]_{\g^\ast}]_{\g^\ast} \\
  %\nonumber&&= \sum\limits_{i=1}^{n}[\eta_1,\cdots,\eta_{i-1},[\xi_1,\cdots,\xi_{n-1},\eta_i]_{\g^\ast},\eta_{i+1},\cdots,\eta_n]_{\g^\ast} \\
%  \nonumber&&= \sum\limits_{i=1}^{n}(-1)^{n-i}[\eta_1,\cdots,\eta_{i-1},\hat{\eta_i},\eta_{i+1},\cdots,\eta_n,[\xi_1,\cdots,\xi_{n-1},\eta_i]_{\g^\ast}]_{\g^\ast} \\
%  \nonumber&&= \sum\limits_{i=1}^{n}(-1)^{n-i}\sum\limits_{j=1}^{n-1}[\xi_1,\cdots,\xi_{j-1},[\eta_1,\cdots,\eta_{i-1},\hat{\eta_i},\eta_{i+1},\cdots,\eta_n,\xi_j]_{\g^\ast},\xi_{j+1},\cdots,\xi_{n-1},\eta_i]_{\g^\ast} \\
%  \nonumber&&+ \sum\limits_{i=1}^{n}(-1)^{n-i}[\xi_1,\cdots,\xi_{n-1},[\eta_1,\cdots,\eta_{i-1},\hat{\eta_i},\eta_{i+1},\cdots,\eta_n,\eta_i]_{\g^\ast}]_{\g^\ast}\\
  \nonumber=&&\sum\limits_{i=1}^{n}\sum\limits_{j=1}^{n-1}(-1)^{n-i}(-1)^{n-j}[\xi_1,\cdots,\hat{\xi_j},\cdots,\xi_{n-1},\eta_i,[\eta_1,\cdots,\eta_{i-1},\hat{\eta_i},\eta_{i+1},\cdots,\eta_n,\xi_j]_{\g^\ast}]_{\g^\ast} \\
  \nonumber&&+ \sum\limits_{i=1}^{n}[\xi_1,\cdots,\xi_{n-1},[\eta_1,\cdots,\eta_n]_{\g^\ast}]_{\g^\ast},
\end{eqnarray*}
thus, we obtain
\begin{eqnarray}\label{eq:FJi-biyao2}
&&-(n-1)[\xi_1,\cdots,\xi_{n-1},[\eta_1,\cdots,\eta_n]_{\g^\ast}]_{\g^\ast}\\
\nonumber&&=\sum\limits_{i=1}^{n}\sum\limits_{j=1}^{n-1}(-1)^{n-i}(-1)^{n-j}[\xi_1,\cdots,\hat{\xi_j},\cdots,\xi_{n-1},\eta_i,[\eta_1,\cdots,\eta_{i-1},\hat{\eta_i},\eta_{i+1},\cdots,\eta_n,\xi_j]_{\g^\ast}]_{\g^\ast}.
\end{eqnarray}
By \eqref{eq:FJi-biyao2}, we have
\begin{eqnarray}\label{eq:FJi-biyao22}
% \nonumber to remove numbering (before each equation)
&&-\sum\limits_{j=1}^{n-1}\langle~[\xi_1,\cdots,\xi_{n-1},[\eta_1,\cdots,\eta_n]_{\g^\ast}]_{\g^\ast},x_j~\rangle \\
\nonumber&&=\sum\limits_{i=1}^{n}\sum\limits_{j=1}^{n-1}(-1)^{n-i}(-1)^{n-j}\langle~[\xi_1,\cdots,\hat{\xi_j},\cdots,\xi_{n-1},\eta_i,[\eta_1,\cdots,\eta_{i-1},\hat{\eta_i},\eta_{i+1},\cdots,\eta_n,\xi_j]_{\g^\ast}]_{\g^\ast},x_j~\rangle,
\end{eqnarray}
the left-hand side of \eqref{eq:FJi-biyao22} is equal to
\begin{eqnarray*}
% \nonumber to remove numbering (before each equation)
  &&-\sum\limits_{j=1}^{n-1}\langle~[\xi_1,\cdots,\xi_{n-1},[\eta_1,\cdots,\eta_n]_{\g^\ast}]_{\g^\ast},x_j~\rangle
 % &=& -\sum\limits_{j=1}^{n-1}(-1)^{n-j}\langle~[\xi_1,\cdots,\xi_{j-1},\hat{\xi_j},\xi_{j+1},\cdots,\xi_{n-1},[\eta_1,\cdots,\eta_n]_{\g^\ast},\xi_j]_{\g^\ast},x_j~\rangle \\
  = \sum\limits_{j=1}^{n-1}(-1)^{n-j}\langle~\xi_j,\ad^\ast_{\xi_1,\cdots,\xi_{j-1},\hat{\xi_j},\xi_{j+1},\cdots,\xi_{n-1},[\eta_1,\cdots,\eta_n]_{\g^\ast}}x_j~\rangle,
\end{eqnarray*}
the right-hand side of \eqref{eq:FJi-biyao22} is equal to
\begin{eqnarray*}
% \nonumber to remove numbering (before each equation)
  && \sum\limits_{i=1}^{n}\sum\limits_{j=1}^{n-1}(-1)^{n-i}(-1)^{n-j}\langle~[\xi_1,\cdots,\hat{\xi_j},\cdots,\xi_{n-1},\eta_i,[\eta_1,\cdots,\eta_{i-1},\hat{\eta_i},\eta_{i+1},\cdots,\eta_n,\xi_j]_{\g^\ast}]_{\g^\ast},x_j~\rangle\\
  %&=& -\sum\limits_{i=1}^{n}\sum\limits_{j=1}^{n-1}(-1)^{n-i}(-1)^{n-j}\langle~[\eta_1,\cdots,\eta_{i-1},\hat{\eta_i},\eta_{i+1},\cdots,\eta_n,\xi_j]_{\g^\ast},\ad^\ast_{\xi_1,\cdots,\xi_{j-1},\hat{\xi_j},\xi_{j+1},\cdots,\xi_{n-1},\eta_i}x_j~\rangle \\
  &=& \sum\limits_{i=1}^{n}\sum\limits_{j=1}^{n-1}(-1)^{n-i}(-1)^{n-j}\langle~\xi_j,\ad^\ast_{\eta_1,\cdots,\eta_{i-1},\hat{\eta_i},\eta_{i+1},\cdots,\eta_n}(\ad^\ast_{\xi_1,\cdots,\xi_{j-1},\hat{\xi_j},\xi_{j+1},\cdots,\xi_{n-1},\eta_i}x_j)~\rangle.
\end{eqnarray*}
Then we have
\begin{equation}\label{eq:11}
  \sum\limits_{j=1}^{n-1}(-1)^{n-j}\ad^\ast_{\xi_1,\cdots,\hat\xi_j,\cdots,\xi_{n-1},[\eta_1,\cdots,\eta_n]_{\g^\ast}}x_j
  =\sum\limits_{i=1}^{n}(-1)^{n-i}\ad^\ast_{\eta_1,\cdots,\hat\eta_i,\cdots,\eta_n}(\sum\limits_{j=1}^{n-1}(-1)^{n-j}\ad^\ast_{\xi_1,\cdots,\hat\xi_j,\cdots,\xi_{n-1},\eta_i}x_j).
\end{equation}
Similarly, we can obtain
\begin{equation}\label{eq:12}
  \sum\limits_{j=1}^{n-1}(-1)^{n-j}\ad^\ast_{x_1,\cdots,\hat x_j,\cdots,x_{n-1},[y_1,\cdots,y_n]}\xi_j
  =\sum\limits_{i=1}^{n}(-1)^{n-i}\ad^\ast_{y_1,\cdots,\hat y_i,\cdots,y_n}(\sum\limits_{j=1}^{n-1}(-1)^{n-j}\ad^\ast_{x_1,\cdots,\hat x_j,\cdots,x_{n-1},y_i}\xi_j).
\end{equation}

Then by \eqref{eqs:61},\eqref{eqs:62},\eqref{eqs:1},\eqref{eqs:2}-\eqref{eqs:67},\eqref{eq:9},\eqref{eq:10},\eqref{eq:11},\eqref{eq:12} and the fact that $\g,~\g^\ast$ are $n$-Lie algebras, next we can proof the Filippov-Jacobi Identity on $\g\oplus\g^\ast$.
\begin{eqnarray*}
% \nonumber to remove numbering (before each equation)
&&\big[x_1+\xi_1,\cdots,x_{n-1}+\xi_{n-1},[y_1+\eta_1,\cdots,y_n+\eta_n]_{\frkd}\big]_{\frkd} \\
=&&\big[x_1,\cdots,x_{n-1},[y_1,\cdots,y_n]\big]
+[x_1,\cdots,x_{n-1},\sum\limits_{i=1}^{n}(-1)^{n-i}\ad_{\eta_1,\cdots,\hat{\eta_i},\cdots,\eta_n}^\ast y_i]\\
+&&\ad_{\xi_1,\cdots,\xi_{n-1}}\big([y_1,\cdots,y_n]
+\sum\limits_{i=1}^{n}(-1)^{n-i}\ad_{\eta_1,\cdots,\hat{\eta_i},\cdots,\eta_n}^\ast y_i\big)\\
+&&\sum\limits_{j=1}^{n-1}(-1)^{n-j}\ad_{\xi_1,\cdots,\hat{\xi_j},\cdots,\xi_{n-1},[\eta_1,\cdots,\eta_n]_{\g^\ast}
+\sum\limits_{i=1}^{n}(-1)^{n-i}\ad_{y_1,\cdots,\hat{y_i},\cdots,y_n}^\ast \eta_i}^\ast x_j \\
+&&\sum\limits_{j=1}^{n-1}(-1)^{n-j}\ad_{x_1,\cdots,\hat{x_j},\cdots,x_{n-1},[y_1,\cdots,y_n]
+\sum\limits_{i=1}^{n}(-1)^{n-i}\ad_{\eta_1,\cdots,\hat{\eta_i},\cdots,\eta_n}^\ast y_i}^\ast \xi_j \\
+&&\ad_{x_1,\cdots,x_{n-1}}\big([\eta_1,\cdots,\eta_n]_{\g^\ast}
+\sum\limits_{i=1}^{n}(-1)^{n-i}\ad_{y_1,\cdots,\hat{y_i},\cdots,y_n}^\ast \eta_i\big)\\
+&&\big[\xi_1,\cdots,\xi_{n-1},[\eta_1,\cdots,\eta_n]_{\g^\ast}\big]_{\g^\ast}
+[\xi_1,\cdots,\xi_{n-1},\sum\limits_{i=1}^{n}(-1)^{n-i}\ad_{y_1,\cdots,\hat{y_i},\cdots,y_n}^\ast \eta_i]_{\g^\ast}.
\end{eqnarray*}

\begin{eqnarray*}
% \nonumber to remove numbering (before each equation)
&&\sum\limits_{i=1}^{n}\big[y_1+\eta_1,\cdots,y_{i-1}+\eta_{i-1},[x_1+\xi_1,\cdots,x_{n-1}+\xi_{n-1},y_i+\eta_i]_{\frkd},y_{i+1}+\eta_{i+1},\cdots,y_n+\eta_n\big]_{\frkd}\\
=&&\sum\limits_{i=1}^{n}\big[y_1,\cdots,y_{i-1},[x_1,\cdots,x_{n-1},y_i],y_{i+1},\cdots,y_n\big]
+\sum\limits_{i=1}^{n}[y_1,\cdots,y_{i-1},\ad_{\xi_1,\cdots,\xi_{n-1}}^\ast y_i,y_{i+1},\cdots,y_n]\\
+&&\sum\limits_{i=1}^{n}[y_1,\cdots,y_{i-1},\sum\limits_{j=1}^{n-1}(-1)^{n-j}\ad_{\xi_1,\cdots,\hat{\xi_j},\cdots,\xi_{n-1},\eta_i}^\ast x_j,y_{i+1},\cdots,y_n]\\
+&&\sum\limits_{i=1}^{n}(-1)^{n-i}\ad_{\eta_1,\cdots,\hat{\eta_i},\cdots,\eta_n}^\ast\big([x_1,\cdots,x_{n-1},y_i]
+\sum\limits_{j=1}^{n-1}(-1)^{n-j}\ad_{\xi_1,\cdots,\hat{\xi_j},\cdots,\xi_{n-1},\eta_i}^\ast x_j
+\ad_{\xi_1,\cdots,\xi_{n-1}}^\ast y_i\big)\\
+&&\sum\limits_{i=1}^{n}\sum\limits_{k=1}^{i-1}(-1)^{n-k}\ad_{\eta_1,\cdots,\hat{\eta_k},\cdots,\eta_{i-1},[\xi_1,\cdots,\xi_{n-1},\eta_i]_{\g^\ast}
+\sum\limits_{j=1}^{n-1}(-1)^{n-j}\ad_{x_1,\cdots,\hat{x_j},\cdots,x_{n-1},y_i}^\ast\xi_j
+\ad_{x_1,\cdots,x_{n-1}}^\ast\eta_i,\eta_{i+1},\cdots,\eta_n}^\ast y_k\\
+&&\sum\limits_{i=1}^{n}\sum\limits_{k=i+1}^{n}(-1)^{n-k}\ad_{\eta_1,\cdots,\eta_{i-1},[\xi_1,\cdots,\xi_{n-1},\eta_i]_{\g^\ast}
+\sum\limits_{j=1}^{n-1}(-1)^{n-j}\ad_{x_1,\cdots,\hat{x_j},\cdots,x_{n-1},y_i}^\ast\xi_j
+\ad_{x_1,\cdots,x_{n-1}}^\ast\eta_i,\eta_{i+1},\cdots,\hat{\eta_k},\cdots,\eta_n}^\ast y_k\\
+&&\sum\limits_{i=1}^{n}\big[\eta_1,\cdots,\eta_{i-1},[\xi_1,\cdots,\xi_{n-1},\eta_i]_{\g^\ast},\eta_{i+1},\cdots,\eta_n\big]_{\g^\ast}\\
+&&\sum\limits_{i=1}^{n}[\eta_1,\cdots,\eta_{i-1},\ad_{x_1,\cdots,x_{n-1}}^\ast\eta_i,\eta_{i+1},\cdots,\eta_n]_{\g^\ast}\\
+&&\sum\limits_{i=1}^{n}[\eta_1,\cdots,\eta_{i-1},\sum\limits_{j=1}^{n-1}(-1)^{n-j}\ad_{x_1,\cdots,\hat{x_j},\cdots,x_{n-1},y_i}^\ast \xi_j,\eta_{i+1},\cdots,\eta_n]_{\g^\ast}\\
+&&\sum\limits_{i=1}^{n}(-1)^{n-i}\ad_{y_1,\cdots,\hat{y_i},\cdots,y_n}^\ast\big([\xi_1,\cdots,\xi_{n-1},\eta_i]_{\g^\ast}
+\sum\limits_{j=1}^{n-1}(-1)^{n-j}\ad_{x_1,\cdots,\hat{x_j},\cdots,x_{n-1},y_i}^\ast \xi_j
+\ad_{x_1,\cdots,x_{n-1}}^\ast \eta_i\big)\\
+&&\sum\limits_{i=1}^{n}\sum\limits_{k=1}^{i-1}(-1)^{n-k}\ad_{y_1,\cdots,\hat{y_k},\cdots,y_{i-1},[x_1,\cdots,x_{n-1},y_i]
+\sum\limits_{j=1}^{n-1}(-1)^{n-j}\ad_{\xi_1,\cdots,\hat{\xi_j},\cdots,\xi_{n-1},\eta_i}^\ast x_j
+\ad_{\xi_1,\cdots,\xi_{n-1}}^\ast y_i,y_{i+1},\cdots,y_n}^\ast \eta_k\\
+&&\sum\limits_{i=1}^{n}\sum\limits_{k=i+1}^{n}(-1)^{n-k}\ad_{y_1,\cdots,y_{i-1},[x_1,\cdots,x_{n-1},y_i]
+\sum\limits_{j=1}^{n-1}(-1)^{n-j}\ad_{\xi_1,\cdots,\hat{\xi_j},\cdots,\xi_{n-1},\eta_i}^\ast x_j
+\ad_{\xi_1,\cdots,\xi_{n-1}}^\ast y_i,y_{i+1},\cdots,\hat{y_k},\cdots,y_n}^\ast\eta_k.
\end{eqnarray*}
Therefore, $(\frkd,[\cdot,\cdots,\cdot]_\frkd)$ is an $n$-Lie algebra.
\end{proof}

\begin{lem}\label{lem:$n$-Lie-alg-d}
Let  $(\g,[\cdot,\cdots,\cdot],\gamma_{\g})$ be a local centroid $R_1$-operad $n$-Lie bialgebra
and
let $\langle\cdot,\cdot\rangle_\frkd:\frkd\times \frkd\rightarrow \mathbf{k}$ be a symmetric bilinear form given by
\begin{equation}\label{eq:d-bili}
  \langle~x+\xi,y+\eta~\rangle_\frkd =\langle~x,\eta~\rangle+\langle~\xi,y~\rangle, \quad\forall~ x,y \in \g, \xi,\eta \in \g^\ast.
\end{equation}
For all $x_1,\cdots,x_{n-1},x\in \g$, $\xi_1,\cdots,\xi_{n-1},\xi\in \g^*$, if $\langle\cdot,\cdot\rangle_\frkd$ satisfying
\begin{eqnarray}
% \nonumber to remove numbering (before each equation)
   &&\langle\xi_2,[x_1,\cdots,x_{n-1},\xi_1]_\frkd\rangle_\frkd=0,
\label{eq:one-g*}\\
   &&\langle x_2,[\xi_1,\cdots,\xi_{n-1},x_1]_\frkd\rangle_\frkd=0,
\label{eq:one-g}
\end{eqnarray}
and
\begin{equation}\label{eq:other-g}
\left\{
\begin{split}
&\langle x+\xi,[x_1,\cdots,x_{n-2},\xi_1,\xi_2]_\frkd\rangle_\frkd=0,\\
   &\quad\quad\quad\quad\quad\vdots \\
   &\langle x+\xi,[x_1,x_2,\xi_1,\cdots,\xi_{n-2}]_\frkd\rangle_\frkd=0.
\end{split}
\right.
\end{equation}
where $[\cdot,\cdots,\cdot]_\frkd$ is given by \eqref{eq:$n$-Lie-on-d},
then $[\cdot,\cdots,\cdot]_\frkd$ is a unique $n$-Lie bracket
such that $(\g,[\cdot,\cdots,\cdot])$ and $(\g^*,[\cdot,\cdots,\cdot]_{\g^*})$ are $n$-Lie subalgebras of $(\frkd,[\cdot,\cdots,\cdot]_\frkd)$ and that the symmetric bilinear form $\langle\cdot,\cdot\rangle_\frkd$ is invariant.
\end{lem}

\begin{proof}
It is straightforward to deduce that $\langle\cdot,\cdot\rangle_\frkd$  is a non-degenerate symmetric bilinear form.
By the invariant condition \eqref{eq:$n$-Lie-inva} and $(\g,[\cdot,\cdots,\cdot])$ is an $n$-Lie subalgebra of $(\frkd,[\cdot,\cdots,\cdot]_\frkd)$, for all  $x_1,\cdots,x_n\in \g, \xi\in\g^*$, we have
\begin{align*}
  &\langle x_n,[x_1,\cdots,x_{n-1},\xi]_\frkd\rangle_{\frkd}
  = -\langle [x_1,\cdots,x_{n-1},x_n]_\frkd,\xi\rangle_{\frkd}
  = -\langle[x_1,\cdots,x_{n-1},x_n],\xi\rangle_{\frkd}.
\end{align*}
By \eqref{eq:d-bili} and \eqref{eq:ad*g}, the right-hand side of the above equation is  equal to
\begin{align*}
  -\langle~\xi,[x_1,\cdots,x_{n-1},x_n]~\rangle
  = \langle~\ad^\ast_{x_1,\cdots,x_{n-1}}\xi,x_n~\rangle
  = \langle x_n,\ad^\ast_{x_1,\cdots,x_{n-1}}\xi\rangle_{\frkd}.
\end{align*}
By \eqref{eq:one-g*} and  $\langle\cdot,\cdot\rangle_\frkd$  is a non-degenerate bilinear form, we have
\begin{equation}\label{eq:double-one-g*}
[x_1,\cdots,x_{n-1},x+\xi]_\frkd=\ad^\ast_{x_1,\cdots,x_{n-1}}\xi.
\end{equation}

Similarly, for all $\xi_1,\cdots,\xi_n\in \g^*,x\in\g$, by \eqref{eq:d-bili} and \eqref{eq:ad*g*} we have
$$\langle\xi_n,[\xi_1,\cdots,\xi_{n-1},x]_\frkd\rangle_{\frkd}
=\langle\xi_n,\ad^\ast_{\xi_1,\cdots,\xi_{n-1}}x\rangle_{\frkd}.$$
Thus, by \eqref{eq:one-g}, we have
\begin{equation}\label{eq:double-g}
[\xi_1,\cdots,\xi_{n-1},x+\xi]_\frkd=\ad^\ast_{\xi_1,\cdots,\xi_{n-1}}x.
\end{equation}

By \eqref{eq:other-g}, we have
\begin{equation}\label{eq:other-brackt}
  [x_1,\cdots,x_{n-2},\xi_1,\xi_2]_\frkd=0,\cdots,[x_1,x_2,\xi_1,\cdots,\xi_{n-2}]_\frkd=0.
\end{equation}
Therefore, for all $x_1,\cdots,x_n \in \g,~\xi_1,\cdots,\xi_n \in \g^\ast$, we get the unique linear map $[\cdot,\cdots,\cdot]_\frkd:\wedge^n \frkd\rightarrow \frkd$ satisfying  \eqref{eq:$n$-Lie-on-d}. Then, the conclusion holds.
\end{proof}


\begin{defi}\label{defi:doub-Li-bialg}
Let  $(\g,[\cdot,\cdots,\cdot],\gamma_{\g})$ be a local centroid $R_1$-operad $n$-Lie bialgebra, where $\gamma_{\g}:\g\rightarrow\otimes^n\g$ is a linear map that defines an $n$-Lie bracket on $\g^\ast$ through the dual map $\gamma_{\g}^\ast:\otimes^n\g^\ast\rightarrow\g^\ast$. Then we call $(\g,\gamma_{\g})$ a \bf{double construction $n$-Lie bialgebra}.
\end{defi}
%Note that $d=\g\bowtie\g^\ast$ is also the double of $g^\ast$. In the $n$-Lie algebra $d$, the subspaces $\g$ and $\g^\ast$ are complementary $n$-Lie subalgebras, and both are isotropic, which means that the scalar product vanishes on $\g$ and on $\g^\ast$.

%\begin{pro}\label{pro:d-loco-nLB}
%If $(\g,\gamma)$ is an $n$-Lie bialgebra and $d$ is the double of $\g$, then we can get a local cocycle $n$-Lie bialgebra.
%\end{pro}
%
%\begin{proof}
%Let $k_1,k_2,\cdots,k_n$ be complex numbers such that $k_1+k_2+\cdots+k_n=1$. Denote $\gamma_i=k_i\gamma,~i=1,2,\cdots,n$, then $\gamma=\gamma_1+\gamma_2+\cdots+\gamma_n$ defines a local cocycle $n$-Lie bialgebra.
%Since $d$ is the double of $\g$, we can obtain Eq?, then we get
%\begin{equation*}
%  (k_1+k_2+\cdots+k_n)\gamma=k_1\gamma_R^1+k_2\gamma_R^2+\cdots+k_n\gamma_R^n.
%\end{equation*}
%Thus, $\gamma=\gamma_{1R}^1+\gamma_{2R}^2+\cdots+\gamma_{nR}^n?$,~$(\g,\gamma)$ is a local cocycle $n$-Lie bialgebra.
%\end{proof}
%
%\begin{defi}\label{defi:quasi-loco-nLB}
%Let $\g$ be a $n$-Lie bialgebra and $\gamma$ satisfies Eqs\eqref{eq:n-local-co}-\eqref{eq:2}, then we call $(\g,\gamma)$ is a quasi-$n$-Lie bialgebra.
%\end{defi}
%
%Therefore, for any quasi-$n$-Lie bialgebra $\g$, $(d, \g, \g^\ast)$ is an example of a Manin triple.

\begin{thm}\label{main}
%Let  $(\g,[\cdot,\cdots,\cdot],\gamma_{\g})$ be a local centroid $R_1$-operad $n$-Lie bialgebra and $\gamma_{\g}:\g\rightarrow\otimes^n\g$ a linear map satisfies $\gamma_{\g}^\ast:\otimes^n\g^\ast\rightarrow\g^\ast$ defines a $n$-Lie bracket on $\g^\ast$. Then
Let $(\cdot,\cdot)$ be a bilinear form  defined by \eqref{eq:d-bili} and let $[\cdot,\cdots,\cdot]_\frkd$ be a linear map  defined by \eqref{eq:$n$-Lie-on-d}. Then $((\g\oplus\g^\ast,[\cdot,\cdots,\cdot]_\frkd,(\cdot,\cdot)),\g,\g^\ast)$ is a Manin triple if and only if $(\g,\gamma_{\g})$ is a double construction $n$-Lie bialgebra.
\end{thm}

\begin{proof}
It can be directly obtained from Proposition \ref{pro:double-Lie} and  Lemma \ref{lem:$n$-Lie-alg-d}.
\end{proof}

\begin{pro}
Let  $(\g,\gamma_{\g})$ be a  double construction $n$-Lie bialgebra with a basis $\{e_1,\cdots,e_n\}$. For the positive integers $1 \leq a_1,\cdots,a_n, s_1,\cdots,s_n, i,k\leq n$ and structural constants $T_{a_1,\cdots,a_n}^k $, $C_{i}^{s_1,\cdots,s_n}e_{s_1}\in F$,
set
\begin{equation*}
  [e_{a_1},\cdots,e_{a_n}]=\sum\limits_{k=1}^nT_{a_1,\cdots,a_n}^k e_k,
  \quad \gamma(e_{i})=\sum\limits_{s_1,\cdots,s_n=1}^nC_{i}^{s_1,\cdots,s_n}e_{s_1}\otimes\cdots\otimes e_{s_n},
\end{equation*}
then we have
%$(1)\gamma_{\g}$ satisfies Eq\eqref{eq:n-local-co} if and only if the following equation holds:
%\begin{equation}\label{eq:iff29}
%  \sum\limits_{s_1,\cdots,s_n,k=1}^n\big(T_{a_1,\cdots,a_n}^k C_k^{s_1,\cdots,s_n}-\sum\limits_{i=1}^n(-1)^{n-i}T_{a_1,\cdots,a_{i-1},\hat{a_i},a_{i+1},\cdots,a_n,k}^{s_1} C_{a_i}^{k,s_2,\cdots,s_n}\big)=0.
%\end{equation}

(1) $\gamma_{\g}$ satisfies \eqref{eq:n-centroid} if and only if the following equation holds:
\begin{equation}\label{eq:iff30}
  \sum\limits_{s_1,\cdots,s_n,k=1}^n\big(T_{a_1,\cdots,a_n}^k C_k^{s_1,\cdots,s_n}-\sum\limits_{i=1}^n(-1)^{n-1}T_{a_2,\cdots,a_n,k}^{s_i} C_{a_1}^{s_1,\cdots,s_{i-1},\hat{s_i},s_{i+1},\cdots,s_n,k}\big)=0.
\end{equation}

(2) $\gamma_{\g}$ satisfies \eqref{eq:1} if and only if the following equations hold:
\begin{equation}\label{eq:iff31}
  \sum\limits_{s_1,\cdots,s_n,k=1}^n T_{a_2,\cdots,a_n,s_j}^k C_{a_1}^{s_1,\cdots,s_n}=\sum\limits_{s_1,\cdots,s_n,k=1}^n T_{a_2,\cdots,a_n,s_n}^{k} C_{a_1}^{s_1,\cdots,s_n}=0,\quad \forall j=1,2,\cdots,n-1.
\end{equation}

(3) $\gamma_{\g}$ satisfies \eqref{eq:2} if and only if the following equations hold:
\begin{equation}\label{eq:iff32}
  \sum\limits_{s_1,\cdots,s_n,j=1}^n T_{a_2,\cdots,a_n,s_i}^j C_{a_1}^{s_1,\cdots,s_n}=\sum\limits_{s_1,\cdots,s_n,j=1}^n T_{a_1,\cdots,a_{n-1},s_k}^j C_{a_n}^{s_1,\cdots,s_n}=0,\quad \forall i,k=1,2,\cdots,n;~i\neq k.
\end{equation}
\end{pro}

\begin{proof}
It is obtained by a straightforward computation of \eqref{eq:n-centroid}-\eqref{eq:2} followed by comparing the coefficients.
\end{proof}


{\bf Acknowledgements:}
The authors would like to thank the referees for helpful comments. The fourth  author acknowledges support from the NSF China (12101328).

\begin{thebibliography}{a}
%\bibitem{Nambu}Y. Nambu, Generalized Hamiltonian dynamics. \emph{Phys. Rev. D} {\bf 7} (1973), 2405-2412.

%\bibitem{Takhtajan} L. Takhtajan, On foundation of the generalized Nambu mechanics. \emph{Comm. Math. Phys.} {\bf 160} (1994), 295-315.



\bibitem{$n$-ary} J. A. de Azcarraga and J. M. Izquierdo, $n$-ary algebras: a review with applications, \emph{J. Phys. A: Math. Theor.} {\bf 43} (2010), 293001.

\bibitem{Bagger}
J. Bagger and N. Lambert, Gauge symmetry and supersymmerey of multiple $M2$-branes, \emph{Phys. Rev. D} {\bf 77} (2008), 065008.

\bibitem{Bagger2}
J. Bagger and N. Lambert, Comments on multiple $M2$-branes, \emph{J. High Energy Phys.} (2008), no. 2, 105, 15 pp.

\bibitem{Bai}
C. Bai, G. Li and Y. Sheng, Bialgebras, the classical Yang-Baxter equation and Manin triples for $3$-Lie algebras, \emph{Adv. Theor. Math. Phys.} {\bf 23} (2019), no. 1, 27-74.


\bibitem{Bai-R} R. Bai, W. Guo, L. Lin and Y. Zhang, $n$-Lie bialgebras, \emph{Linear Multilinear Algebra} {\bf 66} (2018), no. 2, 382-397.

 \bibitem{Casas} J. M. Casas, J. L. Loday and T. Pirashvili, Leibniz $n$-algebras, \emph{Forum Math.} {\bf 14} (2002), no. 2, 189-207.

\bibitem{Quantum}
V. Chari and A. Pressley, A Guide to Quantum Groups, Cambridge University Press, Cambridge(1994).

\bibitem{Drinfeld2}
V. G. Drinfeld, Hamiltonian structure on the Lie groups, Lie bialgebras and the geometric sense of the classical Yang-Baxter equations. \emph{Soviet Math. Dokl.} {\bf 27} (1983), 68-71.

\bibitem{Drinfeld}
V. G. Drinfeld, Quantum groups, In: Proceedings of ICM (Berkeley, 1986), \emph{Amer. Math. Soc., Providence, RI,} (1987), 798-820.

\bibitem{Bai2}
C. Du, C. Bai and G. Li, $3$-Lie bialgerbas and $3$-Lie classical Yang-Baxter equations in low dimensions, \emph{Linear Multilinear Algebra} {\bf 66} (2018), no. 8, 1633-1658.

\bibitem{Filippov}
V. T. Filippov, $n$-Lie algebras, \emph{Sib. Mat. Zh.} {\bf 26} (1985), 126-140.

\bibitem{Gustavsson}
A. Gustavsson, Algebraic structures on parallel $M2$-branes, \emph{Nuclear Phys. B} {\bf 811} (2009), no. 1-2, 66-76.




\bibitem{Matsuo}
P. Ho, R. Hou and Y. Matsuo, Lie $3$-algebra and multiple $M2$-branes, \emph{J. High Energy Phys.} (2008), no. 6, 020, 30 pp.

\bibitem{Matsuo2}
P. Ho, Y. Matsuo and S. Shiba, Lorentzian Lie ($3$-)algebra and toroidal compactification of M/string theory, \emph{J. High Energy Phys.} (2009), no. 3, 045, 34 pp.


\bibitem{Loday}J. L. Loday, Generalized bialgebras and triples of operads. \emph{Asterisque} {\bf 320} (2008), x+116 pp.

\bibitem{Lu-1} J.-H. Lu, and M. Yakimov,     Group orbits and regular partitions of Poisson manifolds, \emph{Comm. Math. Phys.} {\bf 283} (2008), no. 3, 729-748.

\bibitem{Quantum2}
S. Majid, Foundations of quantum group theory, Cambridge University Press, Cambridge(1995).

\bibitem{Manin1}
Y. I. Manin,  Quantum Groups and Noncommutative Geometry, Universit${\rm \acute{e}}$ de Montreal, Centre de Recherches Math${\rm \acute{e}}$matiques,  Springer, Montreal, 1988.

\bibitem{Semenov-Tian-Shansky} M. A. Semenov-Tian-Shansky, Dressing transformations and Poisson group actions, \emph{Publ. Res. Inst. Math. Sci.} {\bf 21} (1985), 1237-1260.

\bibitem{Takhtajan-2} L. A. Takhtajan, Higher order analog of Chevalley-Eilenberg complex and deformation theory of $n$-gebras, \emph{St. Petersburg Math. J.} {\bf 6} (1995), no. 2, 429-438.














%\bibitem{Poisson}J. A. de Azcarraga, J. M. Izquierdo, and J. C. Perez Bueno, On the generalizations of Poisson structures. \emph{J. Phys.} {\bf A30} (1997), L607-L616.

%\bibitem{Lichnerowicz}
%A. Lichnerowicz, Les varietes de Poisson et leurs algebras de Lie associees. \emph{J. Diff. Geom.} {\bf 12} (1977), 253-300.
%
%\bibitem{symmetry}
%J. Bagger and N. Lambert, Gauge symmetry and supersymmetry of multiple M2-branes gauge theories. \emph{Phys. Rev. D} {\bf 77} (2008), 065008.
%
%\bibitem{gauge}
%J. Bagger and N. Lambert, Three-algebras and N=6 Chern-Simons gauge theories. \emph{Phys. Rev. D} {\bf 79} (2009), 025002.
%
%\bibitem{Yau}
%D. Yau, Hom-algebras and homology. \emph{J. Lie Theory} {\bf 19} (2009), 409.
%
%\bibitem{Hom-Liebialg}
%D. Yau, The classical Hom-Yang-Baxter equation and Hom-Lie bialgebras. \emph{Int. Electron. J. Algebra} {\bf 17} (2015), 11-45.

%\bibitem{Baxter}
%R. J. Baxter, Partition function for the eight-vertex lattice model. \emph{Ann. Physics} {\bf 70} (1972), 193-228.
%
%\bibitem{Baxter2}
%R. J. Baxter, Exactly solved models in statistical mechanics. \emph{Academic Press}, London (1982).
%
%\bibitem{Yang}
%C. N. Yang, Some exact results for the many-body problem in one dimension with replusive delta-function interaction. \emph{Phys. Rev. Lett.} {\bf 19} (1967),1312-1315.
%
%\bibitem{Sklyanin}
%E. K. Sklyanin, On complete integrability of the Landau-Lifshitz equation. \emph{LOMI preprint E-3-1979, Leningrad} (1979).
%
%\bibitem{Sklyanin2}
%E. K. Sklyanin, The quantum version of the inverse scattering method. \emph{Zap. Nauchn. Sem. LOMI 95} (1980), 55-128.








%\bibitem{Michaelis}
%W. Michaelis, A class of infinite dimensional Lie bialgebras containing the Virasoro algebras. \emph{Adv. Math. } {\bf 107} (1994), 365-392.
%
%\bibitem{Taft}
%E. J. Taft, Witt and Virasoro algebras as Lie bialgebras. \emph{J. Pure Appl. Algebra} {\bf 87} (1993), 301-312.
%
%\bibitem{Taft2}
%S. H. Ng, E. J. Taft, Classification of the Lie bialgebra structures on the Witt and Virasoro algebras. \emph{J. Pure Appl. Algebra} {\bf 151} (2000), 67-88.
%
%\bibitem{Hartwig}
%J. T. Hartwig, D. Larsson, and S. D. Silvestrov, Deformations of Lie algebras using $\sigma$-derivations. \emph{J. Algebra} {\bf 295} (2006), 314.











%\bibitem{Kasymov}
%S. M. Kasymov, Theory of $n$-Lie algebras. \emph{Algebra i Logika} {\bf 26} (1987), no. 3, 277-297.
%
%\bibitem{Ling}
%W. Ling, On the structure of $n$-Lie algebras, Dissertation. \emph{University-GHS-Siegen, Siegn}(1993).














\end{thebibliography}

\end{document}
