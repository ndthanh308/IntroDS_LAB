\documentclass[journal]{IEEEtran}
\usepackage[edges]{forest}
\usetikzlibrary{arrows.meta}
\usepackage{graphicx,times,amsmath,cite,enumerate}
\usepackage{kantlipsum}
\usepackage{threeparttable}
\usepackage{epsfig}
\usepackage{amssymb}
\usepackage{latexsym}
\usepackage{psfrag}
\usepackage{setspace}
\usepackage{color}
\usepackage{verbatim}
\usepackage{amsthm}
\usepackage{footnote}
\usepackage{textcase}
\usepackage{array}
\newcolumntype{P}[1]{>{\centering\arraybackslash}p{#1}}
\usepackage{float} 
\usepackage{multirow}
\usepackage{textcomp}
\usepackage{forest}
\PassOptionsToPackage{hyphens}{url}
\usepackage{url}
\usepackage{physics}
\usepackage{hyperref}
\usepackage{etoolbox}
\usepackage[edges]{forest}
\usepackage[ruled,vlined]{algorithm2e}
\usepackage{siunitx}
\let\oldemptyset\emptyset
\let\emptyset\varnothing
\usepackage{subfigure}
\usepackage{physics}
\usepackage{tikz}
\usetikzlibrary{automata, positioning}
\usepackage{nomencl}
\Urlmuskip=0mu plus 1mu
\usepackage{float}
\usepackage{tabularx,booktabs}
\setlength{\subfigtopskip}{0pt}

\renewcommand{\baselinestretch}{0.975}

% \usepackage{titlesec}
% \titlespacing{\subsubsection}
%   {0pt}
%   {0ex plus 0ex minus 0ex}
%   {0ex plus 0ex minus 0ex}
% \renewcommand{\thesubsubsection}{\thesubsection.\arabic{subsubsection}}


\usepackage{lipsum}

\newcommand\blfootnote[1]{%
  \begingroup
  \renewcommand\thefootnote{}\footnote{#1}%
  \addtocounter{footnote}{-1}%
  \endgroup
}



\title{\vspace{-10pt}Weather Sensitive High Spatio-Temporal Resolution Transportation Electric Load Profiles For Multiple Decarbonization Pathways \vspace{-0.15em}}

% \title{Climate Sensitive Spatio-Temporal Transportation Electric Loads For Multiple Decarbonization Pathways}

%\title{Developing Spatio_Temporal Transportation Electric Load from a multi-sector Dynamics Model}
\IEEEoverridecommandlockouts

\begin{document}
\bstctlcite{IEEEexample:BSTcontrol} % for mutiple papers by same authors

\author{
\IEEEauthorblockN{Samrat Acharya\IEEEauthorrefmark{1}, Malini Ghosal\IEEEauthorrefmark{1}, Travis Thurber\IEEEauthorrefmark{1}, Casey D. Burleyson\IEEEauthorrefmark{1}, Yang Ou\IEEEauthorrefmark{2}, Allison Campbell\IEEEauthorrefmark{1}, Gokul Iyer\IEEEauthorrefmark{2}, Nathalie Voisin\IEEEauthorrefmark{1}\IEEEauthorrefmark{3}, and Jason Fuller\IEEEauthorrefmark{1} }\\
\IEEEauthorblockA{\IEEEauthorrefmark{1} \textit{Pacific Northwest National Laboratory, Richland, WA 99354, USA}} \\
\IEEEauthorblockA{\IEEEauthorrefmark{2} \textit{Joint Global Change Research Institute, Pacific Northwest National Laboratory, College Park, MD 20740, USA}} \\
\IEEEauthorblockA{\IEEEauthorrefmark{2} \textit{University of Washington, Seattle, WA 98195, USA}} \\
\{samrat.acharya, malini.ghosal\}@pnnl.gov \vspace{-10pt}
}

\maketitle
\begin{abstract}
% Transportation electrification is a major pathway towards reducing greenhouse gas emissions. Although the transportation electrification sector has focused on light-duty vehicles in the past, electrification of medium-and heavy-duty vehicles has started as well with recent decarbonization policies. Thus, the electrification of all-duty vehicles is expected to grow aggressively in the foreseeable future. Power grid and electric vehicle charging station operators require accurate spatio-temporal charging load forecasts of the electrified transportation sector to maintain normal techno-economic operation. However, such charging load forecasts are subject to multi-sectoral changes. In this paper, we use Global Change Analysis Model (GCAM) that internalizes such changes and report global yearly energy usage of transportation sector. Then, we downscale the yearly energy use to forecasts the data-driven time-series electric vehicle charging load for the years 2025 to 2050 across balancing authorities in Western Electricity Coordinating Council (WECC). Furthermore, we show the effect of temperature, charging strategies, and electric vehicle charging station capacities on the time-series electric vehicle charging load. This work assists power utilities and charging station operators to plan their operation and helps researchers to advance their study via public data and code. 

%This paper aims to provide accurate spatio-temporal charging load forecasts for the transportation electrification sector, focusing on electric vehicles. We use the Global Change Analysis Model (GCAM) to report global yearly energy usage of the transportation sector and downscale the data to forecast the time-series electric vehicle charging load for the years 2025 to 2050 across balancing authorities in the Western Electricity Coordinating Council (WECC). The study also examines the impact of temperature, charging strategies, and electric vehicle charging station capacities on the time-series electric vehicle charging load. The goal is to assist power utilities and charging station operators in their operations and to aid researchers in their studies by providing public data and code. 

%This paper outlines the methodology to generate aggregate spatiotemporal electric load demand from the transportation sector at the balancing authority levels for future years based on various decarbonization and climate model scenarios. The methodology downscales the state-wise, yearly-energy use by transportation modes and classes from the multi-sector dynamics model Global Change Analysis Model (GCAM) to hourly load shapes across balancing authorities in the Western Electricity Coordinating Council (WECC). The aggregate load shapes are created using the charging behavior of each vehicle weight class. The study also examines the impact of temperature, charging strategies, and charging rates on transportation load shapes. The generated load shape will next inform production cost modeling studies of the western grid impacted by various decarbonization pathways.
%We downscale the state-wise, yearly-energy use of the transportation sector obtained from a multi-sector dynamics model- Global Change Analysis Model (GCAM)- to hourly electric load profiles across balancing authorities in the Western Electricity Coordinating Council (WECC). 

%We present an approach to generating hourly electric load profiles for the transportation sector that we demonstrate over the western U.S. interconnection for a range of decarbonization pathways, climate scenarios, and future years. The approach consists of downscaling annual state-scale sectoral load projections from the multi-sectoral Global Change Analysis Model (GCAM) into hourly electric load profiles for various decarbonization pathways. The approach is unique in that the load profiles represent an evolving sensitivity to temperature and charging strategies associated with transportation electrification. The load profiles are generated at Balancing Authority level in the western U.S. interconnection to represent spatial diversity in the projections and are compatible with production cost model analysis. Our open source approach can be adapted for other grid regions.
Electrification of transport compounded with climate change will transform hourly load profiles and their response to weather. Power system operators and EV charging stakeholders require such high-resolution load profiles for their planning studies. However, such profiles accounting whole transportation sector is lacking. Thus, we present a novel approach to generating hourly electric load profiles that considers charging strategies and evolving sensitivity to temperature. The approach consists of downscaling annual state-scale sectoral load projections from the multi-sectoral Global Change Analysis Model (GCAM) into hourly electric load profiles leveraging high resolution climate and population datasets. Profiles are developed and evaluated at the Balancing Authority scale, with a 5-year increment until 2050 over the Western U.S. Interconnect for multiple decarbonization pathways and climate scenarios. The datasets are readily available for production cost model analysis. Our open source approach is transferable to other regions.

% We utilize a multi-sector dynamics model to estimate the yearly energy use of the transportation sector and downscales it to hourly electric load profiles for various decarbonization pathways and climate change scenarios. We also examine the impact of temperature, charging strategies, and charging rates on the transportation load profiles. The load profiles are generated at a Balancing Authority level in the western U.S. interconnection so that they can later inform production cost modeling studies in the western U.S. interconnection. This study can be adapted for other grid regions using our open-source code.

\end{abstract}

% \vspace{-15pt}
\section{Introduction}
\label{sec:intro}

% Electrification of the transportation sector is crucial for decarbonization because the transportation sector is one of the biggest producers of greenhouse gas (GHG) emissions. For example, the transportation sector constituted 29\% \cite{us_epa} of the total GHG emissions in 2019 in the United States (U.S.) and 27\% \cite{iea_ghg_sector} globally. This makes the transportation sector the first and second biggest GHG emitter in the U.S. and globally, respectively. 
% To date transportation electrification has been focused on light-duty vehicles (LDVs). For example, global electric car sales have tripled from 2018-2021, with 16.5 millions in 2021 -- 9\% of the global car market \cite{IEA}, while global penetration of electric buses and trucks are 4\% and 0.1\% in 2021, respectively \cite{IEA}. Although LDVs continue to have largest portion in transportation electrification, electrification of medium- and heavy-duty vehicles (MHDVs) (e.g., commercial and freight trucks, buses) and non-road vehicles (e.g., trains, aviation, and ships) is projected to increase in the future \cite{mai2018electrification}. Thus, it is important to consider the electrification of the entire transportation sector. 
\blfootnote{This paper is accepted for publication in IEEE ISGT NA 2024. The complete copyright version will be available on IEEE Xplore when the conference proceedings are published.}
The transportation sector is a major greenhouse gas (GHG) emitter, accounting for 29\% of United States (U.S.) \cite{us_epa} and 27\% of global emissions in 2019 \cite{iea_ghg_sector}. It is the largest GHG emitter in the U.S. and second largest globally. While the focus has primarily been on electrifying light-duty vehicles (LDVs), the electrification of medium- and heavy-duty vehicles (MHDVs) and non-road vehicles (such as trains, aviation, and ships) is gaining momentum \cite{mai2018electrification}. Currently, LDVs have seen significant electrification, with global electric car sales reaching 16.5 million in 2021, representing 9\% of the global car market \cite{IEA}. Electric bus and truck penetration is lower at 4\% and 0.1\% respectively \cite{IEA}. However, achieving comprehensive decarbonization requires electrifying the entire transportation sector, including MHDVs and non-road vehicles. 

% Rapid large-scale transportation electrification imposes economic \cite{kintner2020electric}, operational \cite{lopes2010integration}, and cybersecurity \cite{acharya2020public} challenges to power grids and Electric Vehicle (EV) chargers. Accurate projections of spatio-temporal transportation electric loads are required to mitigate the aforementioned challenges. Furthermore, granularity in the projections allows the power grid and charging station operators to enhance the efficiency of their planning and operational decisions. However, highly granular projections (e.g., 1 second) over longer time horizons can have low accuracy \cite{jain2014forecasting}. Thus, we project future transportation loads at an hourly resolution. Recent studies \cite{gaete2021open, borlaug2021heavy, osti_1836645} use data-driven approaches to project EV charging load. Morales et. al \cite{gaete2021open} develop time-series charging load profiles for LDVs using data on mobility, battery characteristics, and charging strategies in Germany. Borlaug et al. \cite{borlaug2021heavy} develop average daily charging load profiles for depot-based heavy-duty trucks using data on truck mobility and charging strategies. Wang et al. \cite{osti_1836645} develop daily charging load profiles for MHDVs in 2030 using data on mobility and future charger capacities in California. In contrast to the studies \cite{gaete2021open, borlaug2021heavy, osti_1836645}, we develop charging load time-series for the whole transportation sector.

Rapid large-scale transportation electrification poses challenges to economics \cite{kintner2020electric}, operations \cite{lopes2010integration}, and cybersecurity \cite{acharya2020public} of power grids and EV chargers. To address these challenges, accurate and highly granular projections of spatio-temporal transportation charging load profiles are essential for effective planning and decision-making.
% However, highly granular projections over long time horizons may suffer from reduced accuracy \cite{jain2014forecasting}. In our study, we focus on hourly resolution projections of future transportation loads. 
Recent studies \cite{gaete2021open, borlaug2021heavy, osti_1836645} have employed data-driven approaches to project EV charging load profiles. For example, Gaete et al. \cite{gaete2021open} developed LDV charging profiles in Germany using data on mobility, battery characteristics, and charging strategies. Borlaug et al. \cite{borlaug2021heavy} created average daily charging profiles for depot-based heavy-duty trucks based on truck mobility and charging data. Wang et al. \cite{osti_1836645} generated daily charging profiles for MHDVs in California by considering mobility patterns and future charger capacities. In contrast to these studies \cite{gaete2021open, borlaug2021heavy, osti_1836645}, our approach encompasses the entire transportation sector, providing comprehensive time-series data for charging load profiles.

% Figure environment removed

% In this paper, we study transportation electrification in the western U.S. interconnection region. Annual projections are shown in Fig.~\ref{fig:penetration} under two decarbonization pathways: i) Business As Usual (BAU) and ii) Net-Zero (NZ) as modeled by the U.S. version of the Global Change Analysis Model (GCAM-USA). GCAM is a global, economically-driven, multi-sector dynamics model to simulate the interactions between human and natural systems and assess global changes and their impacts. GCAM has been widely used by organizations working on climate change including Intergovernmental Panel on Climate Change (IPCC) \cite{ipcc}. The BAU scenario in GCAM is based on existing GHG emission reduction policies. The NZ scenario is based on ambitious plans to achieve a clean U.S. grid by 2035 and net-zero GHG emission by 2050. Using the two GCAM decarbonization pathways, we generate hourly electric load profiles for the transportation sector across Balancing Authorities (BAs) in the western U.S. interconnection. BAs are entities that plan resources ahead-of-time, maintain the balance of electricity supply and demand within a specific geographic area, and support frequency regulation in real-time. Please see \cite{ba_map} for the spatial locations of the BAs in the western U.S. interconnection.
% Our unique contributions are summarized below:
% \begin{enumerate}

%     \item We develop a spatially distributed statistical downscaling approach for projecting annual transportation energy onto hourly time series loads at balancing authority level in the western U.S. interconnection using GCAM- a multi-sector dynamic model that incorporates global changes.

%     \item We conduct sensitivity analysis of transportation electric loads with temperature representing climate change scenarios, decarbonization pathways and charging strategies.

%     \item We provide open source code to accelerate the dissemination of the approach and support the community of practice, available at \url{https://doi.org/10.5281/zenodo.7888569}\cite{acharya_samrat_2023_7888569}.

% \end{enumerate}
In this paper we analyze transportation electrification in the western U.S. interconnection region. Our analysis encompasses two decarbonization pathways: i) Business As Usual (BAU) and ii) Net-Zero (NZ), modeled using the U.S. version of the Global Change Analysis Model (GCAM-USA) as depicted in Fig.~\ref{fig:penetration}. GCAM is a widely utilized, economically-driven, multi-sector dynamics model that simulates the interactions between human and natural systems, playing a crucial role in assessing global changes and their impacts. Notably, GCAM has been employed by organizations like the Intergovernmental Panel on Climate Change (IPCC) \cite{ipcc} and the U.S. Department of Energy. The BAU scenario in GCAM reflects existing GHG emission reduction policies. The NZ scenario embodies ambitious plans to achieve a clean U.S. grid by 2035 and net-zero GHG emissions by 2050. Leveraging these decarbonization pathways, we generate hourly transportation electric load profiles across Balancing Authorities (BAs) in the western U.S. interconnection. BAs oversee resource planning, electricity supply-demand balance, and real-time frequency regulation in specific geographic areas. For spatial details on BAs in the western U.S. interconnection, we refer to \cite{ba_map}. To this end, we make the following unique contributions:
% \vspace{-3pt}
\begin{enumerate}
\item We develop a spatially-distributed statistical downscaling approach, leveraging multi-sector dynamics in GCAM, to project annual transportation energy onto hourly BA-level time-series profiles in the western U.S. interconnection. 

%time series load profiles at the level of BAs within the western U.S. interconnection.
\item We conduct a detailed sensitivity analysis of transportation charging profiles considering climate change scenarios, decarbonization pathways, and charging strategies.
\item We provide open source code at \url{https://doi.org/10.5281/zenodo.7888569}\cite{acharya_samrat_2023_7888569} to accelerate the dissemination of the approach and support the community of practice.
\end{enumerate}


\section{Data and Tools}
\label{sec:methodology}
This section outlines the key data and tools employed to project spatio-temporal transportation charging load profiles.

% \vspace{-11pt}
\subsection{Global Change Analysis Model (GCAM)}
\label{sec:gcam}

% GCAM is an open-source, integrated assessment model developed and maintained at PNNL\cite{GCAMv6documentation}. GCAM incorporates energy, economy, agriculture and land use, water, and climate systems at different spatial scales. The modeling period in GCAM spans from 2015 to 2100, with 5-year time-steps, and the final calibration year is 2015. This analysis utilizes GCAM-USA v6, which is built within the global GCAM but contains 50 states plus the District of Columbia, representing the sub-national economy and energy systems. These state-level regions explicitly consider various socio-economic drivers, energy transformation sectors, and final energy services. The population and economic growth assumptions are broadly consistent with the Shared Socioeconomic Pathway-2 (SSP2) \cite{o2017roads}. In this analysis, GCAM-USA’s electricity technology cost assumptions are updated to National Renewable Energy Laboratory’s (NREL) 2022 Annual Technology Baseline (ATB) \cite{EIA}. Transportation cost and energy intensity assumptions for various technologies, including EVs, are primarily based on NREL’s Electrification Futures Study \cite{jadun2017electrification}. 

The Global Change Analysis Model (GCAM) \cite{GCAMv6documentation} is an open-source integrated assessment model that encompasses energy, economy, agriculture and land use, water, and climate systems at global spatial scales. For this study GCAM spans from 2015 to 2100, with 5-year time-steps, and the final calibration year is 2015. We utilize GCAM-USA v6 which focuses on 50 states plus the District of Columbia and simulates sub-national economy and energy systems. GCAM-USA v6 incorporates socioeconomic drivers, energy transformation trends, and final energy services at the state level. It utilizes the Shared Socioeconomic Pathway-2 (SSP2) growth assumptions \cite{o2017roads}. GCAM-USA v6 incorporates updated electricity technology cost assumptions from the National Renewable Energy Laboratory's (NREL) 2022 Annual Technology Baseline (ATB) \cite{EIA}. Transportation cost and energy intensity assumptions, including those for EVs, are primarily based on NREL's Electrification Futures Study \cite{jadun2017electrification}.

% \vspace{-11pt}
\subsection{Thermodynamic Global Warming Simulations (TGW)}
\label{sec:wrf}
% The hourly temperatures for historic and future years are based on the Thermodynamic Global Warming (TGW) simulations using the Weather Research and Forecasting (WRF) model \cite{tgw-wrf, tgw-wrf1}. WRF is a widely adopted numerical weather prediction model designed to calculate atmospheric weather variables at horizontal and vertical grid cells above the Earth's surface \cite{wrf, Skamarock2019}. The TGW simulations provide meteorological variables at hourly and three-hourly temporal and 12 km$^2$ spatial resolution. The TGW simulations cover the conterminous U.S. plus portions of Canada and Mexico. The future weather scenarios (2020-2099) are created by replaying historical weather events under different levels of global warming based on multiple Representative Concentration Pathways (RCP) and Global Climate Models (GCMs). Warming levels are derived from average temperature and humidity changes from models that are ``cooler'' and ``hotter'' than the multi-model mean. Both the cooler and hotter ensemble scenarios combined with RCP 4.5 and RCP 8.5 have been used in this work, resulting in four future climate scenarios.

The Thermodynamic Global Warming (TGW) simulations \cite{tgw-wrf, tgw-wrf1} are based on the Weather Research and Forecasting (WRF) model \cite{wrf, Skamarock2019}. WRF is a widely used numerical weather prediction model that calculates atmospheric variables at horizontal and vertical grid cells above the Earth's surface \cite{wrf, Skamarock2019}. The TGW simulations provide hourly and three-hourly meteorological data at a 12 km$^2$ resolution, covering the conterminous U.S. and parts of Canada and Mexico. For 2020-2099, the TGW simulations replay historical weather events under different levels of global warming based on multiple Representative Concentration Pathways (RCPs) and Global Climate Models. Warming levels are derived from average temperature and humidity changes from models that are ``cooler'' and ``hotter''  compared to the multi-model mean. We use four future climate scenarios obtained by combining the cooler and hotter scenarios with RCP 4.5 and RCP 8.5. 

% \vspace{-11pt}
\subsection{Total ELectricity Loads Model (TELL)}
\label{sec:tell}

% The Total ELectricity Loads (TELL) model is used to downscale the annual state-level electricity demand projections from GCAM-USA to an hourly resolution \cite{McGrath2022}. TELL is based on a series of multilayer perceptron (MLP) models for each BA. The MLP models are trained to predict historical hourly electricity demands based on variations in meteorology. The hourly BA-level loads from TELL are then scaled so that they quantitatively match the annual state-level loads from GCAM-USA. For this experiment, TELL was used to generate hourly non-transportation electricity demands for each BA that can be combined with the transportation electricity demands to provide a complete picture of how projected total loads change over time and across decarbonization scenarios.

The Total Electricity Loads (TELL) model \cite{McGrath2022} downscales annual state-level electricity demand projections from GCAM-USA to a hourly resolution. TELL utilizes multilayer perceptron models trained on historical hourly electricity demands and meteorological variations for each BA. The hourly BA-level loads from TELL are scaled to match the annual state-level loads from GCAM-USA, providing hourly non-transportation electricity demands for each BA. Combining these with transportation charging demands allows for a comprehensive analysis of change in total loads over time and across decarbonization scenarios.

% \vspace{-11pt}
\subsection{EVI-Pro Lite}
\label{sec:evi_pro}
% EVI-Pro Lite \cite{center2020electric} is a tool that helps project the aggregated charging demand for EVs. It uses various inputs such as detailed data on personal vehicle travel patterns, EV attributes, and charging station characteristics to generate charging electric load profiles. The tool is based on training models using advanced PEV simulations based on millions of miles of real-world daily driving schedules sourced from large public and commercial travel data sets in the U.S. The model takes input such as fleet size, daily mean temperature (only a few discrete values), the distribution of battery electric vehicles and plug-in hybrid vehicles, battery size of the EVs, percentage of vehicles having access to the charging station at home, preference of charging at home, access to charging at work, and levels of charging.
% EVI-Pro Lite is based on historical data and certain assumptions are made that might not hold for future years due to technological advancements (e.g., the prevalence of wireless charging), changes in charging behavior, and adoption of EVs in larger demographics. However, we chose our parameters and assumptions carefully to best represent the future scenarios. The subsection \ref{sec:ldv_profile_method} outlines the input assumptions and the procedures to harmonize EVI-Pro inputs with the GCAM-USA, TGW, and TELL models.

EVI-Pro Lite \cite{center2020electric} is a data-driven tool that projects the aggregated charging demand of EVs. It uses detailed data on travel patterns, EV attributes, and charging infrastructure characteristics to generate charging electric load profiles. The tool is based on advanced PEV simulations trained on real-world driving data from large U.S. travel databases. Inputs to the tool include fleet size, discrete daily mean temperature, EV distribution, battery size, charging access, charging preferences (home vs work), and charging levels. We acknowledge that future technological advancements (e.g., wireless charging), changes in charging behavior, and broader EV adoption may challenge EVI-Pro assumptions based on historical data. Thus, we select parameters to represent realistic future charging scenarios. We detail our input assumptions and their relation with GCAM-USA, TGW, and TELL in Section \ref{sec:ldv_profile_method}.

% \vspace{-12pt}
\subsection{Fleet DNA Data}
\label{sec:fleetdna}
% Fleet DNA is an anonymous dataset on mobility of commercial MHDVs, including delivery vans, transit buses, and refuse trucks in the U.S. \cite{fleetDNA}. NREL collects, processes, and publish the data. The data headings we use are “Vid", “start\textunderscore{ts}", “end\textunderscore{ts}", and “distance\textunderscore{total}". “Vid" is an unique identifier for a vehicle, “start\textunderscore{ts}" and “end\textunderscore{ts}" are the start and end times of the vehicle recordings, and “distance\textunderscore{total}" is the distance travelled by a vehicle (in miles) during a trip. Since all vehicles in Fleet DNA data do not operate with return-to-base schedules, we omit vehicle mobility instances that are less likely to be return-to-base. Furthermore, the data is filtered such that vehicles return to the depot by midnight and charge only during their dwelling time in depots.

The Fleet DNA dataset \cite{fleetDNA} contains anonymous data on the mobility of commercial MHDVs in the U.S., including delivery vans, transit buses, and refuse trucks. We utilize specific data headings, such as "Vid" (vehicle identifier), "start\textunderscore{ts}" (vehicle record start time), "end\textunderscore{ts}" (vehicle record end time), and "distance\textunderscore{total}" (distance traveled by vehicle). To focus on return-to-base schedules, we exclude mobility instances that are less likely to be return-to-base. Also, we filter the data to include only vehicles that return to the depot by midnight and charge during their dwelling time in depots.

% \vspace{-11pt}
\subsection{Non-Road Vehicles Data}
Non-road vehicles data includes enplanements at U.S. commercial airports published by the U.S. Federal Aviation Administration \cite{enplanemnet}, route miles of rails, and shipping docks in U.S. published by the U.S. Department of Transportation \cite{rails_county_data}.

% \begin{itemize}
%     \item Global Change Analysis Model (GCAM): 
%     % \cite{calvin2019gcam}--- Yang
%     \item EVI-Pro: \cite{osti_1764904}--- Malini
%     \item The Weather Research and Forecasting Model (WRF): \cite{WRF}---Kendall/Allison
%     \item Total Electricity Loads Model (TELL): \cite{TELL}---Casey
%     \item Fleet DNA Data : 
% \end{itemize}

% \vspace{-5pt}
\section{Methodology}

Fig.~\ref{fig:workflow} illustrates our methodology for developing spatio-temporal charging load profiles across BAs in the western U.S. interconnection. Below, we detail our approach.

% Figure environment removed

% \vspace{-12pt}
\subsection{Developing LDV Charging Load Profiles}
\label{sec:ldv_profile_method}

Fig.~\ref{fig:workflow} illustrates the generation of LDV charging load profiles. This section outlines the methodology for translating the state-level annual energy consumption of electrified LDVs in GCAM-USA into hourly load shapes at the BA level.
\subsubsection*{Spatial Downscaling}
We use county-level resolution data to spatially downscale LDV charging load profiles. This allows us to incorporate temperature data from the TGW simulations (Section~\ref{sec:wrf}) and consider the diverse climatic regions within the BAs. The state-to-county distribution of energy use is determined by analyzing the distribution of registered EVs across counties using vehicle distribution data from 2018-2021. Each county's proportion of the state's total EVs is calculated, following the methodology outlined in \cite{kintner2020electric}.
\subsubsection*{Temporal Downscaling}
The county-level aggregate charging load profiles require various inputs from different modules. One crucial input for the EVI-Pro module is the fleet size. While direct fleet size data is not available from GCAM-USA, the annual energy consumption (in PJ/yr), energy usage per vehicle travel distance (in MJ/kvm), and average travel distance are utilized to estimate the fleet size. Daily mean temperature from TGW simulations (Section \ref{sec:wrf}) is used in EVI-Pro, with mapping to the nearest permissible discrete values. Additional inputs for EVI-Pro are outlined in Table \ref{T:evi_input}, independent of other modules in this study.

While EVI-Pro \cite{center2020electric} does not cover all future charging scenarios, we outline our key parameter choices in the model.
\begin{itemize}
    \item Battery EVs have a 250-mile range- the max. battery size in EVI-Pro. Plug-in hybrid EVs are excluded in this study.

    \item Predominant use of level-2 chargers (208-240 V) over level-1 chargers (120 V AC).
    
    \item Anticipated decrease in home charging preference from 80\% to 60\%, influenced by increasing chargers in public and work. Also, as EV ownership reaches 75-90\%, chargers in multi-family residential dwellers increases, reducing home charging preferences. In 2022, $\approx 63\%$ of houses are single-family units \cite{homeowner}, which also bounds the preference for home charging.
    \item Study of two charging strategies: \emph{min\_delay} for immediate maximum-speed charging after arrival and \emph{load\_level} for slow charging during the dwelling time. By 2035, 30\% of LDVs are expected to use load leveling, which is projected to increase to 70\% by 2050. Managed charging with price incentives and larger battery capacities reduce range anxiety, increasing load-leveling adoption.
\end{itemize}





% Fig.~\ref{fig:workflow} depicts the process of generating BA-level LDV charging load profiles. This section elucidates the methodology for translating the annual energy consumption of electrified LDVs at the state level, derived from the GCAM-USA, into hourly load shapes at the BA level. The translation process entails both spatial and temporal downscaling of energy consumption.

% \subsubsection*{Spatial downscaling}
% We utilize a county-level spatial resolution to generate the aggregate downscaled load profile for LDVs. This resolution is adopted to ensure that temperature data from the TGW simulations as described in Section \ref{sec:wrf} is effectively utilized as BAs might encompass diverse climatic regions. In order to obtain the state- to county-level distribution of energy use, the vehicle distribution data from 2018-2021 is used. Specifically, we analyze the distribution of registered EVs across counties and calculate each county's proportion of the state's total EVs. Our approach aligns with the methodology outlined in \cite{kintner2020electric} and incorporates the most current data available.


% \subsubsection*{Temporal downscaling}
% Several inputs are required from different modules to derive the county-level aggregate charging load profiles. One critical input for the EVI-Pro module is the fleet size. Despite the absence of direct fleet size data from the GCAM-USA, the annual energy consumption \unit{(PJ/yr)}, energy usage per vehicle travel distance \unit{(MJ/kvm)}, and average travel distance have been utilized. Likewise, in EVI-Pro, discrete values of daily mean temperature are utilized. TGW is the source of county-level daily mean temperature, which is subsequently mapped to the nearest permissible discrete values.
% Other inputs which do not interact with any other modules of the framework and their assumed values for given years and can be observed in Table \ref{T:evi_input}.


% Input labels and parameters can be understood through our EVI-Pro assumptions \cite{center2020electric}. While the model does not encompass all future technology and behavior parameters, we have outlined some key choices with closest match.


% \begin{itemize}
%     \item We assume all BEVs have a 250-mile range, which is the maximum battery size considered in EVI-Pro as this would be typical for future years. Plug-in hybrid EVs are excluded from this distribution.
%     \item We assume chargers are predominantly level-2 (L2- operates at 208-240 V) rather than level-1 (L1 - common residential 120V AC outlet).
%     \item We expect the preference for home charging among LDV owners to decrease from 80\% to 60\% in the future. Limited public and work chargers, along with the growing adoption of EVs among multi-family residential dwellers, will contribute to this shift. Currently, EV ownership is primarily associated with homeowners who benefit from dedicated overnight charging. However, as EV ownership reaches 75-90\% according to our study, the availability and preference for home charging will decrease. Approximately 63\% of houses are single-family units \cite{homeowner}, suggesting that the number of such units may bind the preference for home charging.

%     \item We study two charging strategies: \emph{min\_delay} which charges immediately after arrival at maximum speed and \emph{load\_level} which charges slowly during the dwelling time. By 2035, only 30\% of LDVs are expected to use load leveling, but this is expected to increase to 70\% by 2050. We anticipate that managed charging with price incentives and larger battery capacities will reduce range anxiety, leading to greater adoption of the load-leveling strategy.
% \end{itemize}



% \vspace{-15pt}
\subsection{Developing MHDV Charging Load Profiles} 
\label{sec:mhdv_profile_method}

% First, we generate normalized MHDV charging load profiles (separate for MDVs and HDVs) using the MHDV load shape generator in Fig.~\ref{fig:workflow}. 
% The inputs to the MHDV load shape generator include mobility data for MHDVs (delivery vans, delivery trucks, school buses, transit buses, bucket trucks, tractors, and refuse trucks) obtained from Fleet DNA in Section~\ref{sec:fleetdna}, MHDV charging strategies, EVCS capacities (\unit[]{kW}), unit energy use \unit[]{kWh/miles}, fleet size, and number of sample fleets. We use three charging strategies -- immediate, delay, and constant minimum power as developed in \cite{borlaug2021heavy}. In immediate charging, vehicles start to charge as soon as they arrive in the depot and charge till the battery is fully recharged or a subsequent trip begins. In delay charging, the charging is delayed to an extent such that the vehicles are fully recharged just before the subsequent trip. In minimum power charging, the vehicles are charged at a constant minimum power throughout the dwelling period in depot such that the vehicle is fully recharged for a subsequent trip. 
% We envisage that future EV charging is a mix of the charging strategies. Thus, we allow weights to the charging strategies to incorporate the uncertainty in charging strategies. Similarly, mulitple charging capacities will be available in the future. Thus, we allow weights to the charger capacities as well. The unit energy use of vehicle (\unit[]{kWh/miles}) depends on its weight, which we assign using the survey in \cite{smith2020medium}. To decrease the risk of bias in MHDV profiles, we simulate the MHDV load shape generator in Fig.~\ref{fig:workflow} for a user-defined fleet size and number of fleet samples and the profiles as in \cite{borlaug2021heavy}. After receiving the inputs, the MHDV load shape generator aggregates Fleet DNA mobility data by month of the year and generates normalized average daily MHDV charging profiles with a 1-hour resolution for each month. Such daily profiles are extended to create yearly time series of MHDV charging load. Due to insufficient mobility data we assume that the daily MHDV charging profile is the same across a month of the year.  

% Second, as Fig.~\ref{fig:workflow} shows, we scale the normalized MHDV charging load shapes by the annual MHDV energy use (in \unit[]{EJ}) projected by GCAM-USA in the western U.S. interconnection for a given year and decarbonization pathway. Scaled MHDV charging profiles in the western U.S. interconnection are further downscaled to BAs using the relative penetration of MHDVs across BAs. Penetration of MDVs in some BAs are reported in \cite{kintner2020electric}. We extrapolate the penetration of MDVs in the remaining BAs using the penetration of LDVs. We also assume that the relative penetration of MDVs and HDVs are the same in BAs. 
Fig.~\ref{fig:workflow} illustrates the two-step process for generating MHDV charging load profiles. 

In Step 1, using the MHDV load shape generator in Fig.\ref{fig:workflow}, we generate separate normalized MHDV charging load profiles for MDVs and HDVs. Inputs to the generator include mobility data for MHDVs (delivery vans, delivery trucks, school buses, transit buses, bucket trucks, tractors, and refuse trucks) obtained from Fleet DNA (Section\ref{sec:fleetdna}), MHDV charging strategies, EVCS capacities (\unit[]{kW}), unit energy use (\unit[]{kWh/miles}), fleet size, and number of sample fleets \cite{borlaug2021heavy}.
Three charging strategies—immediate, delay, and constant minimum power—are utilized. Immediate charging begins upon depot arrival and continues until the battery is fully recharged or the next trip commences. Delayed charging delays charging to ensure full recharge just before the subsequent trip. Minimum power charging involves charging at a constant minimum power level throughout the depot dwelling period to guarantee a full recharge for the next trip. We incorporate the uncertainty of future EV charging, considering a mix of charging strategies and multiple charging capacities, by assigning weights to both the charging strategies and charger capacities. The unit energy use (\unit[]{kWh/mi}) of vehicles depends on their weight, which is determined using a survey \cite{smith2020medium}. To minimize bias in MHDV profiles, we simulate the MHDV load shape generator (Fig.~\ref{fig:workflow}) for a user-defined fleet size and number of fleet samples, following the approach in \cite{borlaug2021heavy}. The load shape generator in Fig.\ref{fig:workflow} aggregates Fleet DNA mobility data by month and generates normalized average daily MHDV charging shapes with a 1-hour resolution for each month. These daily shapes are extended to create normalized yearly MHDV charging load profiles. Due to insufficient mobility data, we assume that the daily MHDV charging shape remains consistent throughout a month.

In Step 2, we scale the normalized MHDV charging load shapes by the annual MHDV energy use (in \unit[]{EJ}) projected by GCAM-USA in the western U.S. interconnection for a given year and decarbonization pathway. The scaled MHDV charging profiles in the western U.S. interconnection are further downscaled to BAs by considering the relative penetration of MHDVs across BAs. The penetration of MDVs in some BAs is reported in \cite{kintner2020electric}. For the remaining BAs we extrapolate the penetration of MDVs using the penetration of LDVs as a reference. Additionally, we assume that the relative penetration of MDVs and HDVs is the same across BAs.


\begin{table}[!t]
\centering
% \vspace{-30pt}
\caption{Choice of input parameter for EVI-Pro}
\vspace{-2pt}
\label{T:evi_input}
\vspace{-1mm}
\resizebox{1\columnwidth}{!}{
\begin{threeparttable}
\begin{tabular}{|c|c|c|c|}
\hline
%PARAM.& VALUE & PARAM. & VALUE\\ \hline 
 \rule{0pt}{2ex}\textbf{PEV}& BEV250 & \textbf{Class} & Equal\\ \hline 
 \textbf{Preference} & Home60 &
\textbf{Home access} & HA75, HA100 \\ \hline

 \rule{0pt}{2ex}\textbf{Home power} & MostL2 & \textbf{Home ch. strategy} & min\_delay, load\_leveling \\ \hline
 
 \rule{0pt}{2ex}\textbf{Work power} &MostL2  &\textbf{Work ch. strategy} & min\_delay \\ \hline
\end{tabular}
\end{threeparttable}
}
% \vspace{-15pt}
\end{table}

\vspace{-11pt}
\subsection{Developing Non-Road Vehicles Charging Load 
Profiles}
\label{sec:non-road}

As Fig.\ref{fig:workflow} shows, we generate constant charging profiles for non-road vehicles (aviation, rails, and ships). To downscale the state-wise yearly energy use by non-road vehicles in GCAM-USA to hourly charging power across BAs in the western U.S. interconnection, we use the ratio of i) airport enplanements for aviation, ii) route miles travelled in railroads and transits for rails and trains, and iii) shipping docks for ships, in a given BA to that of the entire western U.S. interconnection. The energy at the western U.S. interconnection level is determined by aggregating the state-level energies in GCAM-USA for the 11 states within the region. In this study, we assume that electrified rails, ship, and aviation consume constant power over each hour of the year. We make this assumption due to the lack of data on their charging behavior as they are not yet widely deployed. Therefore, this study may not reflect accurate temporal downscaling of non-road vehicles. However, their spatial downscaling is based on county-level data.



% \vspace{-5pt}
\section{Case Study and Results}
This section examines transportation load profiles in the western U.S. interconnection for 2035 and 2050, considering different decarbonization pathways and climate scenarios. It compares the sensitivity of these profiles and projected loads to the total system load. We present profiles for NZ decarbonization, highlighting its aggressiveness compared to BAU pathway. Our open-source code allows for profile generation in 5-year increments until 2050 for both BAU and NZ pathways.


% \vspace{-12pt}
\subsection{Sensitivity of LDV Load Profiles}
\label{sec:ldv_sensitivity}
Fig. \ref{fig:sensitivity:ldv} demonstrates the impact of LDV charging strategies, temperature, and weekdays vs. weekends on the LDV charging profiles. In Fig. \ref{fig:sensitivity:ldv}(a), load-leveling charging  (C2) shows a flat, grid-friendly charging profile compared to immediate charging (C1). With a 70\% adoption rate of C2 by 2050, the peak load can be reduced by 40\% compared to the C1-only charging. Fig. \ref{fig:sensitivity:ldv}(b) shows the impact of temperature on LDV charging. A temperature of $20 ^\circ$C requires the least charging, while $-10$, $0$, and $40^\circ$C  days consume  $28\%$, $21\%$, and $36\%$ more power, respectively. Temperature affects charging efficiency and energy consumption for the same travel distance. Fig. \ref{fig:sensitivity:ldv}(c) shows LDV charging peak load during weekends is approximately 10\% lower than on weekdays.

% Figure environment removed

% \subsection{Sensitivity of LDV Load Shapes}
% \label{sec:ldv_sensitivity}

% Fig. \ref{fig:sensitivity:ldv} demonstrates the impact of LDV charging strategies, temperature, and weekdays vs. weekends on the LDV charging profiles generted using EVI-Pro Lite. 
% In Fig. \ref{fig:sensitivity:ldv} (a), we compare immediate and load-leveling charging strategies (C1 and C2). The graph shows that load-leveling charging results in a flat, grid-friendly profile. By assuming a 70\% adoption rate of C2 by 2050, we can reduce the peak by $40\%$ from the C1-only charging strategy as seen from the purple line in \ref{fig:sensitivity:ldv} (a). Fig. \ref{fig:sensitivity:ldv}(b) reveals that a temperature of 20 degrees Celsius requires the least charging, while $-10$, $0$, and $40$ degrees Celsius days demand $28\%$, $21\%$, and $36\%$ more power. Temperature impacts the charging and discharging efficiency of the model, thereby impacting the energy drawn to travel the same distance. Fig. \ref{fig:sensitivity:ldv}(c) shows that weekends have a shifted peak, 10\% lower than weekdays. These insights enhance our understanding of LDV charging profiles. 

% \vspace{-9pt}
\subsection{Sensitivity of MHDV Load Profiles}
\label{sec:mhdv_sensitivity}
Fig.~\ref{fig:sensitivity:mhdv} shows the sensitivity of average daily MHDV charging load profiles to charging strategies, charger capacities, and number of sample fleets for stochastic simulation. As Figs.~\ref{fig:sensitivity_charging_strategy_mdv} and \ref{fig:sensitivity_charging_strategy_hdv} show, the choice of charging strategy significantly impacts charging power's magnitude and timing, resulting in varying charging peaks. For instance, as Fig.~\ref{fig:sensitivity_charging_strategy_hdv} shows, the HDV peak is 0.67 GW for minimum power charging at 00:00 while it is 2.84 GW for delay charging at 9:00; representing a 323\% increase. To achieve a more balanced and grid-friendly charging profile, we mix 40\% immediate, 10\% delay, and 50\% constant minimum power charging strategies as an example. This mix means majority of MHDVs charge with constant power throughout their dwelling time or start charging upon arrival at depots and a small portion wait for a suitable time to charge, such as during periods of cheaper electricity prices, before their departure. Charger capacities also influence the charging profiles, with slower charging leading to smaller peaks. For example, as Fig.\ref{fig:sensitivity_charger_hdv} shows, the 50 kW charger exhibits a peak of 0.76 GW at 20:00, while fast chargers have a peak of 1.08 GW at 19:00, representing a 42\% increase. We anticipate that larger battery capacities will drive the adoption of high power chargers. Thus, we consider a mix of chargers: 5\% 50 kW, 5\% 125 kW, 10\% 250 kW, 40\% 350 kW, and 40\% 500 kW. Furthermore, as Figs.~\ref{fig:sensitivity_n_samples_mdv} and \ref{fig:sensitivity_n_samples_hdv} show, the charging profiles remain stable across different fleet samples, despite variations in the number of sample fleets.

% Figure environment removed



% \vspace{-9pt}
\subsection{Transportation Load Profiles}
\label{sec:load_profiles}
% Figs.~\ref{fig:profiles_NZ}(a) and (b) show the  charging load profiles (00:00 to 23:00 UTC hours, one for each month) of LDVs, MHDVS, rails, aviation, ship, and total transportation in the western U.S. interconnection in 2035 and 2050 for the NZ decarbonization pathway. We observe five key dynamics in Figs.~\ref{fig:profiles_NZ}(a) and (b). First, charging load profiles vary with the type of vehicles.   
% Second, not only does the transportation peak load increase in 2050 as compared to 2035 (peak is 27.77 GW in 2035, while it is 59.85 GW in 2050 - a 115\% increase), but also the variation in the charging load increases (19.19 GW in 2035 and 20.35 GW in 2050 - a 6\% increase). Third, the total transportation charging load profile is dominated by LDVs. Fourth, charging load for non-road transportation (aviation, rail, and ship) increases from 2035 to 2050. Fifth, MHDV charging load slightly decreases in 2050 as compared to 2035 in NZ pathway (e.g., peak for HDVs in 2035 is 4.9 GW but only 4.1 GW in 2050) because the GCAM-USA decarbonization pathways envisage
% two features: i) replacement of MHDVs, mostly MDVs, by mass transportation (e.g., rails and ships) increases over years, and ii) MHDVs will adopt other clean fuel technologies (e.g. hydrogen) as we go further.

Figs.~\ref{fig:profiles_NZ}(a) and (b) depict the  charging load profiles (00:00 to 23:00 UTC hours discontinuous for each month) for LDVs, MHDVs, rails, aviation, ship, and total transportation in the western U.S. interconnection in 2035 and 2050 under the NZ decarbonization pathway. Several key dynamics are observed. Firstly, the charging load profiles differ by vehicle type. Secondly, the transportation peak load significantly increases from 27.77 GW in 2035 to 59.85 GW in 2050, marking a 115\% surge. Additionally, the variation in charging load experiences a 6\% increase, with values of 19.19 GW in 2035 and 20.35 GW in 2050. Thirdly, LDVs contribute prominently to the total transportation charging load profile. Fourthly, charging load for non-road transportation modes (aviation, rail, and ship) exhibits an upward trend from 2035 to 2050. Finally, the MHDV charging load slightly decreases in 2050 compared to 2035 (e.g., HDV peak in 2035 is 4.9 GW but 4.1 GW in 2050) due to the NZ decarbonization pathway's emphasis on transitioning from MHDVs, particularly MDVs, to mass transportation options such as rail and ships. This decline is further driven by the adoption of alternative clean fuel technologies, such as hydrogen, in later years.

% \vspace{-10pt}
\subsection{Transportation Electric Load Relative to System Load}
\label{sec:load_ratio}
% We use three metrics (M1, M2, and M3) for assessing the spatial and temporal aspects of the impact of transportation load on the electric power system. M1 measures the ratio of yearly total electric load to transportation electric load, M2 measures the ratio of transportation load to system load at the system peak, and M3 measures the ratio of transportation load to system load at the transportation peak. The study shows that M1 rapidly increases in 2050 relative to 2035, indicating steep increase in electrification across all BAs. The results also reveal that all metrics vary significantly across BAs. For instance, southern California BAs such as LDWPD and IID have high M1 (e.g., in IID the M1 is up to 0.5 in 2035 and 0.677  in 2050). This is due to two reasons, 1) the existing aggressive EV adoption trend in the region and b) the system load in the region is predominantly residential in nature with a moderate climate. Similarly, CHPD in Chelan County, Washington has a light non-transportation electric load but an aggressive EV policy. Finally, the transportation peak and system peak do not coincide, reducing system stress due to transportation charging loads.

Three metrics (M1, M2, and M3) in Fig.~\ref{fig:load_ratio} assess the spatial and temporal impact of transportation load on the electric power system. M1 represents the yearly ratio of total electric energy to transportation electric energy, M2 measures the ratio of transportation load to system load at the system peak, and M3 measures the ratio of transportation load to system load at the transportation peak. M1 rapidly increases in 2050 relative to 2035, indicating steep increase in electrification across all BAs. The results also reveal that all metrics vary significantly across BAs. For instance, southern California BAs such as LDWPD and IID have high M1 (e.g., in IID the M1 is 0.50 in 2035 and 0.68 in 2050). This is attributed to the aggressive EV adoption trend in the region and the predominantly residential system load with a moderate climate. Similarly, CHPD in Chelan County, Washington, has high M1 due to its low non-transportation electric load and ambitious EV policy. Notably, the transportation peak and system peak do not coincide, reducing system stress from transportation charging.

% Figure environment removed

% \vspace{-10pt}
\subsection{Load Profiles with BAU Decarbonization}
% The comparison between the NZ pathway and the BAU pathway reveals significant differences in the charging peak load and its variation. In 2035, the charging peak load for NZ is about 4.7\% higher than that for BAU (27.77 GW  vs. 26.51 GW), while in 2050, it is approximately 65.4\% higher (59.85 GW  vs. 36.17 GW). The charging load variations also show a significant difference, with NZ experiencing about 21.5\% higher variation in 2035 (19.19 GW vs. 15.79 GW) and 10.5\% higher variation in 2050 (20.35 GW vs. 18.41 GW) compared to BAU. Additionally, the ratio of transportation to system energy increases in NZ than BAU especially in 2050. For example, it is around 9.8\% higher for NZ than for BAU in IID in 2050 (0.667 in Fig.~\ref{fig:load_ratio}(b) vs. 0.61).
In comparison to the BAU pathway (not shown in this paper), the NZ pathway (Figs.~\ref{fig:profiles_NZ} and \ref{fig:load_ratio}) demonstrates significant differences in charging peak load and variation. In 2035, NZ has a 4.7\% higher charging peak load than BAU (27.77 GW vs. 26.51 GW), increasing to approximately 65.4\% in 2050 (59.85 GW vs. 36.17 GW). NZ also exhibits higher load variation, with about 21.5\% more variation in 2035 (19.19 GW vs. 15.79 GW) and 10.5\% more variation in 2050 (20.35 GW vs. 18.41 GW) compared to BAU. Additionally, the transportation-to-system energy ratio is notably higher in the NZ pathway, particularly in 2050. For example, in IID, NZ shows a 9.8\% higher ratio than BAU in 2050 (0.67 in Fig.~\ref{fig:load_ratio}(b) vs. 0.61).



% \vspace{-8pt}
\section{Conclusion}
% We present a novel spatially distributed statistical downscaling approach for creating time-series of transportation charging load profiles using a  multi-sector global change analysis model, demonstrated over the western interconnect for 2035 and 2050. We also conduct a sensitivity analysis of the profiles. The study quantifies factors that affect charging load profiles, such as weather conditions, charging behaviors, socioeconomic and decarbonization policies. The study's key finding is that transportation charging loads can comprise up to 2.4-56.6\% of system electric peak in BAs, despite accounting for less than 20\% and 17\% of total electric load for the 2050 Net-Zero case and the 2050 BAU case, respectively, in the western U.S. interconnection. The large variation in the ratio of the transportation to system load is due to nature of load (commercial, residential, and industrial), climate zones, and EV adoption trends in the BAs.
 
% The non-uniform spatial and temporal impact of transportation loads on different BAs is crucial for understanding the effects of decarbonization and transportation electrification policies on the grid. The dataset and code are publicly available at \url{https://doi.org/10.5281/zenodo.7888569}and can be used to develop charging load profiles for other regions.

This paper presents a novel approach for generating spatially-distributed hourly time-series of transportation charging load profiles. Applied to the western U.S. interconnection in 2035 and 2050, our analysis reveals that transportation charging loads can contribute significantly to the system electric peak, ranging from 2.4\% to 56.6\% in different BAs, despite accounting for less than 20\% and 17\% of total electric load for the 2050 NZ case and the 2050 BAU case, respectively. This variation is influenced by factors such as load nature, climate zones, and EV adoption trends. Understanding this non-uniform spatial and temporal impact is crucial for effective decarbonization and transportation electrification policies. To facilitate further research and policy analysis, we provide a publicly available dataset and code at \url{https://doi.org/10.5281/zenodo.7888569}. Researchers and decision-makers can use this resource to develop charging load profiles for other regions, enabling informed decision-making in the pursuit of sustainable transportation electrification.

% \vspace{-11pt}
\section*{Acknowledgment}
This research was supported by the GODEEEP Investment at Pacific Northwest National Laboratory (PNNL). PNNL is a multi-program national laboratory operated for the U.S. Department of Energy (DOE) by Battelle Memorial Institute under Contract No. DE-AC05-76RL01830. We thank Kate Forrest and Brian Tarroja at University of California, Irvine, USA for the discussions on the transportation profiles.

% Figure environment removed

\vspace{-11pt}
\bibliographystyle{IEEEtran}
\bibliography{ref}
\end{document}
