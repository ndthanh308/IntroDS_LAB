\section{Conclusion and Discussion}
\label{sec:conclusion}

We introduce a physics-informed TC tracking framework, {\tool}, that utilizes tools from vector field topology. Based only on a 2D wind vector field, {\tool} is able to produce results comparable to (and sometimes even better than) those obtained with a widely used TC tracking algorithm---{\TE}---while requiring far less input data.
Although {\tool} does not consider the air temperature field (e.g., the warm cores) of TCs, the symmetric eye structures of TCs allow {\tool} to detect and track them. \myedit{Furthermore, our framework may be used in uncertainty visualization to understand  uncertainty due to different model physics or setup, or feature comparison of geoscientific data with space and time dimensions.}

\myedit{{\tool} has a number of limitations. First, our robustness-based framework is very useful for hurricane tracking, since hurricanes have symmetric structures and hurricane tracks are one of the most important factors in assessing their risks. However, once the symmetric structures (i.e., the eyes) are weakened or disappear with the cyclones moving to higher latitudes, {\tool} does not consider them as TCs anymore. This is because the cyclone at higher latitudes may get their energy from one or more front systems dividing warm air from the south and cold air front the north, see~\cref{fig:MissedTraj}. Such frontal systems lead to asymmetric structures, which are not detected by {\tool}. 
In general, our technique may not be suitable for asymmetric feature tracking such as extratropical cyclones.  
Second, to obtain optimal tracking results, we may need to fine-tune the parameters for each single event. 
%This is not always required if we study the climate change impacts on TC features by tracking a current against a future time period, where same parameters would ensure a comparison between the two time periods. 
Third, the current framework only considers the near surface (10 meters) winds. However, higher altitude winds (dozens to hundreds of meters) are also big concerns when it comes to real-world applications such as wind energy. This is left for future work.}

\myedit{From an application perspective, to the best of our knowledge, it is an open challenge to incorporate 3D data in the study of TCs, based on domain scientist feedback. 
Because adding more variables may reduce the overall efficiencies of TROPHY yet may not guarantee a better performance.
From an algorithmic perspective, it may be possible to extend {\tool} to utilize 3D data for TC tracking. First, we may use horizontal layers of a 3D vector field along the vertical direction. Second, we may utilize a third variable called the vertical motion (i.e., upward and downward), in addition to zonal wind (U) and meridional (V) wind. Expanding {\tool} to either of these directions could be useful in better detecting, tracking and understanding hurricanes such as their genesis, intensification, and landfall (important factors to be considered for risk assessment). The robustness framework has been extended previously to study 2D symmetric tensor fields~\cite{WangHotz2017,JankowaiWangHotz2019} and critical points in 3D vector fields~\cite{SkrabaRosenWang2016}. It may be feasible to integrate the robustness framework for 3D critical points in {\tool}. However, there are more complex features that need to be considered in 3D, such as vortex regions or vortex core lines. 
We would need to develop theoretical foundations to quantify their robustness first before establishing their physical interpretability in studying TCs.}
%i moved this to the first paragraph when talking about strength. \myedit{Furthermore, our framework may be used in uncertainty visualization to understand  uncertainty due to different model physics or setup, or feature comparison of geoscientific data with space and time dimensions.}



%Moreover, although this study tracks TC eyes with symmetric features, the framework can take into account more dynamic and thermodynamic variables to enable more functionalities in studying asymmetric features, such as extratropical cyclones, which can bring storm surges, high winds, and heavy precipitation.
%\myedit{For example, \tool can detect extratropical cyclones when their winds do some rotations, \eg, during the stage of ''occluded cyclone''. Adding the gradient of temperature can differentiate such detected extratropical cyclones with TCs during TCs analysis and visualization. For the stages that cannot be detected by \tool,  \eg, during the stage of ''open wave'', identifying stationary fronts created by cold air and warm air may help to study asymmetric features.}
