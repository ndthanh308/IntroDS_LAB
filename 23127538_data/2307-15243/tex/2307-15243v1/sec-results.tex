\section{Cyclone Tracking Results With {\tool}} 
\label{sec:results}
We demonstrate cyclone-tracking results using {\tool} and their comparisons with a state-of-the-art TC tracking algorithm. Observations from official forecast centers are provided for reference.

% Figure environment removed

% Figure environment removed


\subsection{Overview of Results}
We apply {\tool} to the $\EW$ dataset on a cluster with 664 nodes (128GB DDR4 and 36 cores per node). 
We utilize the method of Tricoche~\etal~\cite{TricocheScheuermannHagen2001} to calculate degrees of critical points for the vector field in each time step and parallelize the computation of multilevel robustness with Eden~\cite{SimmermanOsborneHuang2012}, which can schedule and manage a number of tasks on a high-performance computing cluster.

Even though we have obtained the tracking results using {\tool} for all 30 years of data, we demonstrate TC tracking results for 1981, 1990, 2004, and 2009 in~\cref{fig:teaser} to highlight the tracking differences between various datasets and algorithms. 
%The third column shows the results with default thresholds for filters and the fourth column shows the results whose thresholds of filters for each year are tuned by user considering trajectories from IBTrACS as guideline. The TC tracking results using {\TE} are shown on the second column of~\cref{fig:teaser}. 
{\TE} tracks TC eyes as positions with the local minimum SLP, whereas the TC eyes from {\tool} are located where wind speeds are zeros. Therefore, TC tracks from {\TE} and {\tool} do not overlap exactly. 
Both methods consider the area within a two-degree great circle of a detected TC eye to find a local maximum and use it to measure storm intensity~\cite{ZarzyckiUllrich2017,UllrichZarzycki2017}. Tracks in~\cref{fig:teaser,fig:Tails,fig:MissedTraj} are colored by storm intensity~\wrt~this \emph{two-degree maximum wind speed}, referred to as \emph{wind speed} for simplicity. We discuss some preliminary findings based on~\cref{fig:teaser} in the following paragraphs


First, since {\TE} requires both dynamic and thermodynamic variables to meet its specified criteria, it usually cannot detect the beginnings and endings of TC where wind speeds are low. {\tool} can detect such tails as long as the centers of flows can be identified as critical points; see tracks $a$, $b$, and $c$ in~\cref{fig:teaser}. 

Second, {\tool} can detect some discrete storm events that are not captured by \TE, such as tracks $d$ and $e$ in~\cref{fig:teaser}. Track $d$ can be found in IBTrACS, whereas track $e$ cannot. To further investigate these discrepancies, including tracks that cannot be captured by \TE/IBTrACS but are found by {\tool}, and tracks that are in IBTrACS but are undetected by {\tool}, we conducted case studies (reported in \cref{sec:case-study,sec:terminate}). 
%in~\cref{sec:case-study}. Actually, {\tool} may also fail to capture some trajectories that are detected by {\TE} and recorded in IBTrACS. We investigate one missed trajectory with {\tool} in~\cref{sec:terminate}.
 
Third, using fine-tuned filter thresholds for {\tool}, we can get TC tracking results that are more similar to IBTrACS. 
For example, track $f$ is missed by {\tool} with a default threshold for smoothness, since $f$ is severely bent and has a relatively low smoothness value. When, however, we decrease the threshold for $s(\gamma)$ from 0.967 to 0.95, track $f$ is shown in the TC tracking result as indicated by the red arrow in~\cref{fig:teaser}. Similarly, track $e$, which is not recorded in IBTrACS, can be filtered out if we increase the threshold of stability $b(\gamma)$ from 0.029 to 0.03. 
These sensitivities to parameters in the tracking algorithm are, in general, expected \cite{enz2022parallel}. 
This human-in-the-loop option provides users the opportunity of fine-tuning, especially for short-term forecasting or weather-scale studies, as accuracy is more important at this scale. 
At the climate scale and for climate change impacts, users  may use the same thresholds for all TC cases in both historic and future periods, to avoid the impacts of parameter uncertainties. 
 
\subsection{Case Study: Tracking TC during Dissipation}
\label{sec:case-study}

We now investigate a Category 2 hurricane track named Hurricane Bonnie. 
As illustrated in \cref{fig:Tails}, (A) shows the TC tracks from {\TE} (colored by wind speed) and IBTrACS (in black), whereas (B) shows the TC tracks detected by {\tool}.
We observe a much longer tail in (B) with low wind speed compared with (A).
(C) gives the zoomed-in views of tails of TC tracks detected by {\TE} (1st row) and {\tool} (2nd row). Vector fields from different time steps are shown as background. We also highlight the TC eyes detected by each method in corresponding time steps by arrows on (C).
We see TC eyes detected by {\tool} are exactly located where the wind vanishes, whereas the TC eyes from {\TE} are slightly shifted to the outside of the eyes. 
The last two columns of (C) show that {\tool} can track the TC eyes during dissipation because the wind flow is still spinning even if the wind speed is low. 
{\tool} tracks such flow behaviors as sources until the sources disappear.
%\jiali{okay this might be why it's not in real data or TempestExtreem, because in reality, hurricane flow can not be a source, which is a divergence, corresponding good weather in meteorology. if you are comfortable let's remove this part of discussion?} 

%% Figure environment removed

%\subsection{Case Study: Terminating Tracking When a TC Loses Its Distinct Eye}
\subsection{Case Study: Terminating Tracking} 
%  When a TC Loses Its Distinct Eye
\label{sec:terminate} 

We next investigate a Category 1 hurricane named Hurricane Lisa, which is captured by {\TE} but is partially missed by {\tool}.
\cref{fig:MissedTraj} (A--C) show TCs of 1998, which are recorded in IBTrACS and detected by {\TE} and {\tool} with the $\EEight$ dataset, respectively. 
We focus on the analysis of Hurricane Lisa, marked as $\gamma_1$, $\gamma_2$, and $\gamma_3$ in different methods. 
We note that $\gamma_3$, detected by {\tool}, has a clearly shorter track compared with the other two datasets because the eyes of Hurricane Lisa were detected as centers only up to 10/08/1998 00:00 AM, after which Hurricane Lisa lost its distinct eye and could not be detected by {\tool}; see~\cref{fig:MissedTraj} (M), as well as a zoomed-in view in~\cref{fig:MissedTraj} (d). 

Such a process could be related to TC's extratropical transitions \cite{KnaffLongmoreMolenar2014}, which may occur as Hurricane Lisa moves over cooler water and into areas of stronger wind shear at higher latitudes. During this transition, the cyclone loses its symmetric and distinct eye and, thus, {\tool} terminates its tracking.

In fact, Hurricane Lisa produced distinct eyes again at time step 10/09/1998 at 12:00 PM and 10/10/1998 at 00:00 AM, so {\tool} was able to start the track again. However, because these tracks were short-lived, they were filtered out during the initial feature selection. 
%another eyewall \jiali{you meant eye not eyewall, right? i changed 'eyewall' to 'eye' in above paragraph.} at time step 10/09/1998 12:00 PM, as shown in~\cref{fig:MissedTraj} (H) and (a). But it loses its eyewall \jiali{you meant eye not eyewall, right?} again at 10/10/1998 00:00 AM, as shown in~\cref{fig:MissedTraj} (I) and (b).
%The loss of eyewall \jiali{you meant eye structure not eyewall, right?} makes {\tool} terminate its TC track. (Lin: Can we use it to evaluate the storm simulation model?) \jiali{if Tempest captured that and we don't, it could be the tracking algorithm, if none of us captured it, it could be the data. but i think we've explained it well.}

\subsection{Evaluations: Annual Domain-Averaged Climatology}
\label{sec:AnnualClimatology}
This section and the following present the evaluations of {\tool} using the metrics described in \cref{sec:metrics}. For annual climatology, cumulative statistics are calculated over the entire data period (30 years), which are then normalized to a per-year basis. 
\cref{tab:annual} shows annually averaged statistics for TCs detected by {\tool} and \TE. 

\begin{table}[t]
 \caption{Annually averaged metrics (frequency, TC days,  and latitude of lifetime-maximum intensity) for TCs detected by {\TE} and {\tool} relative to the reference (IBTrACS). }
 \vspace{-2mm}
 \label{tab:annual}
\scriptsize%
  \centering%
  \begin{tabu}{%
  *{4}{c}%
  }
\toprule
&$\overline{count} (\#)$&$\overline{tcd} (days)$&$\overline{lmi}$ ($^{\circ}lat.$)\\
	\midrule   
  IBTrACS &10.74 &79.34&25.65\\
  {\TE} & 8.17&50.01&38.30\\
{\tool} (default) &5.3&52.62&31.55\\
{\tool} (human-in-loop) &7.77&66.43&29.98\\
\bottomrule
  \end{tabu}%
  \vspace{-4mm}
\end{table}
%The top line represents average statistics from IBTrACS for reference.  
%The use of IBTrACS observations as the reference allows us to use these metrics to evaluate TC tracking algorithms. 

According to annual TC count ($\overline{count}$), IBTrACS contains approximately 11 TCs per year within the studied region, whereas {\tool} with default setting produces the least number of TCs. However, when considering annual storm lifetime (\ie, $\overline{tcd}$), {\tool} detects longer storms than {\TE} and is closer to IBTrACS on average. 
This result is consistent with what we observed in our experiments. 
{\tool} can detect TCs not only during their movements but also, at least partially, during their formation and dissipation periods, even if their wind speed is low; see~\cref{fig:Tails} for an example.
Also, since {\TE} requires dynamic and thermodynamic variables to meet specified criteria, TC tracks may break into pieces when some parts of the track do not meet the requirement; see the track indicated by a purple circle from~\cref{fig:teaser} for an example.
Overall, {\tool} produces TC tracks for a longer time than does \TE, whereas {\TE} detects more TC tracks than does {\tool}.

In addition, although {\tool}'s TC tracks have closer hurricane genesis $\overline{lmi}$ \wrt~the observation, both {\tool} and {\TE} show a poleward bias compared with IBTrACS's genesis. %in term of the hurricane genesis $\overline{lmi}$. 
One of the reasons for this bias could be that the reanalysis data including ERA5 are not able to  simulate storm structures near the equator well, according to Knaff~\etal~\cite{KnaffLongmoreMolenar2014}, indicating that the input data to any tracking algorithms is the most important factor when studying TCs.
%As pointed out by Knaff~\etal~\cite{KnaffLongmoreMolenar2014}, the manifestation of reanalysis data (including ERA5) is likely more poorly simulating storm structure closer to the equator. The basis of $\overline{lmi}$ from~\cref{tab:annual} is in agreement with this statement, since TCs detected from the $\EW$ dataset tend to increase in size at higher latitudes. But {\tool} produces TC tracks has closer $\overline{lmi}$ \wrt~the observation compared with \TE. 



%%jiali here
\subsection{Evaluations: Spatial Climatology}
\label{sec:spatialClimatology}
For spatial pattern climatology, density maps are generated by aggregating occurrences into $4^\circ$ by $4^\circ$ bins.
%, showing how frequent each dataset can produce TCs. 
An example of the spatial track density patterns is shown in~\cref{fig:rxyTrack}. The spatical correlation between IBTrACS and {\tool} (default) is caculated using patterns from~\cref{fig:rxyTrack} (A) and (C). To quantify the similarity between IBTrACS and {\tool} (or TempeExtremes), we calculate Pearson correlation coefficient $r_{xy}$ for total occurrence ($r_{xy,track}$), genesis occurrence ($r_{xy,gen}$), maximum wind speed, and minimum SLP. 
%For instance, to calculate the total occurrence frequency ($r_{xy,track}$) and genesis ($r_{xy,gen}$) frequency correlations, the raw number of hits (per year) are summed in each bin. 
%An example of the spatial track density patterns used to derive the correlations is shown in~\cref{fig:rxyTrack}. 
%After getting the density maps, we can evaluate the spatial patterns between {\tool}/{\TE} and observation using Pearson correlation coefficient described in~\cref{eq:PearsonCor}. 
\cref{tab:spatical} shows the pattern correlation of {\tool}/{\TE} with observations. 
%The first metric is cumulative track density ($r_{xy,track}$). 

Both {\tool} and {\TE} can produce reasonable distributions of storm occurrence when compared with observations ($>0.89$). 
The spatial patterns of genesis ($r_{xy,gen}$) from {\tool} and {\TE} show lower correlations ($<0.64$) with observations. This is not surprising because the initialization and development of a clear eye depend on many other factors such as their 3D evolutions, which are beyond the capability of what our 2D vector fields can represent. In fact, predicting the genesis of TC is one of the scientific challenges in TC research fields, indicating that the currently available data cannot  capture TC genesis and that new frameworks of such calculations are needed \cite{yang2021hurricane}. 
%The storm first appears when the data assimilation system is employed to generate new initial conditions for the forecast, suggesting that its development was not easy to predict from a forecast run initialized 12 hours earlier. Thus data preparation is likely one factor in the inability to extract a clear center when the storm is a tropical depression. Tropical depressions are also not well-organized systems, the development of a clear eye is dependent on their 3D evolution beyond our study of 2D vector fields, which make {\tool} also has low consistency with observation in detecting the genesis of TCs.
%The last two correlations query spatial patterns of different intensity measures, $r_{xy,u10}$ and $r_{xy,slp}$ measure the spatial correlation as measured by maximum u10 and minimum sea level pressure, respectively, over each grid box over the duration of the data period.  
Both {\tool} and {\TE} are able to produce  storm strength similar to that of observations ($\ge 0.89$) in terms of maximum $u_{10}$ and minimum sea level pressure. 

% Figure environment removed

\begin{table}[t]
\caption{Spatial correlation for TCs detected by {\TE} and {\tool} with the reference (IBTrACS). 
%The TC activity is binned into $4^{\circ}\times 4^{\circ}$ boxes prior to correlation calculation. Spatial climatology includes spatial pattern of cumulative track density ($r_{xy,track}$), genesis ($r_{xy,gen}$), maximum $u10$ ($r_{xy,u10}$), and minimum SLP ($r_{xy,spl}$). 
}
\vspace{-2mm}
 \label{tab:spatical}
\scriptsize%
  \centering%
  \begin{tabu}{%
  *{5}{c}%
  }
\toprule
$4^{\circ}\times 4^{\circ}$ & $r_{xy,track}$ & $r_{xy,gen}$ & $r_{xy,u10}$ & $r_{xy,slp}$\\ [0.5ex] 
\midrule   
 IBTrACS & 1.00 & 1.00 & 1.00 & 1.00  \\ 
 {\TE} & 0.892 & 0.638 & 0.920 &0.889 \\
{\tool} (default) & 0.897 & 0.569 & 0.925 & 0.901 \\
{\tool} (human-in-loop) & 0.914 & 0.634 & 0.933 & 0.920 \\
\bottomrule
  \end{tabu}%
  \vspace{-4mm}
\end{table}



