\section{Technical Background}
\label{sec:background}

We first review the classic notion of robustness and multilevel robustness, which are customized to build {\tool} (see \cref{sec:TROPHY}).

\subsection{Robustness}
\label{sec:classicRobustness}

\para{Critical points of a 2D vector field.} 
Unless otherwise specified, we work with a 2D \emph{vector field} $f: \Xspace \subseteq \Rspace^2 \to \Rspace^2$, which assigns a 2D vector to each point in $\Xspace$. 
We use $u_{10}$ and $v_{10}$ to represent the 10-meter zonal (west-east) and meridional (south-north) wind vector components, respectively.  
Then, $f$ is expressed as $f(x) = (u_{10}(x), v_{10}(x))^T$. 

A \emph{critical point} $x \in \Xspace$ in $f$ is where the vector vanishes, that is, $|f(x)| = 0$. 
A critical point $x$ can be classified {\wrt} its \emph{degree} $\mydeg(x)$, defined as the number of field rotations while traveling along a closed curve counterclockwise surrounding $x$ (enclosing no other critical point). 
A source/sink/center has degree $+1$, whereas a saddle point has degree $-1$.
Critical points are important features in studying flow behavior in many applications; see \cref{fig:CriticalPoints} as an example.
%\cref{fig:CriticalPoints} shows four types of critical points from the $\EW$ dataset (see~\cref{sec:data} for details). 
% Figure environment removed


% Figure environment removed

In most cases, the center of a TC can be detected as a center in a vector field when it is intensified into a strong hurricane, with very low wind speed in the eye and extremely high wind speed along the eyewall. 
During the dissipating phase of a TC, such as at its landfall, the center of the TC can be detected as either a source or a sink. 
If a center transforms to a source, then it indicates a divergence in meteorology, which means the weather can be clear and calm. 
If a center transforms to a sink, then it indicates a convergence, which is associated with clouds and precipitation. 
An example of this phenomenon is Hurricane Florence in 2018: a clear sink forms in the 2D wind field  during its landfall, bringing 1-in-500-year expected flooding due to heavy precipitation.

%and can be detected as source points when it is weaken on its dissipating phase, such as at its landfall, \jiali{sentences after this need to be changed if there is no sink. what i wrote there is not true for source, it's true for sink..} which can still bring severe weather such as heavy precipitation, as the source is still a low pressure system, favorable for air rising, cooling and condensation.
 
% Arrow glyphs are placed on sampled points in the domain to indicate the directions of vectors. 






\para{Merge trees.} 
The computation of robustness relies on an \emph{augmented merge tree} modified from the classic merge tree. 
Given a scalar function $f_0$ defined in a 2D domain $\Xspace$, $f_0:\Xspace \to \Rspace$, let $\Xspace_r=f_0^{-1} (-\infty, r]$ denote the \emph{sublevel set} of $f_0$ for some $r \geq 0$. 
A classic \emph{merge tree} is constructed by tracking the evolution of (connected) components in $\Xspace_r$ as $r$ increases. Leaves in a merge tree represent the creation of a component at a local minimum of $f_0$, internal nodes represent the merging of components, and the root represents the entire space as a single component.

% Figure environment removed

% Figure environment removed

\myedit{To construct an augmented merge tree from a 2D vector field $f$, first we define a scalar field $f_0:\Xspace \to \Rspace$ by assigning the vector magnitude to each point $x \in \Xspace$, that is, $f_0(x) = ||f(x)||_2$. In this paper, $f_0$ can be expressed as wind speed.
Second, instead of using local minima of $f_0$ as leaves of the merge tree, the leaves of our augmented merge tree consist of $\Xspace_0$, which is precisely the set of critical points of $f$. 
The tracking of the merging behavior of components is the same as classic merge tree construction. 
Third, once the merge tree is constructed, it can be further augmented with the degrees of critical points (on leaves) and the degrees of components (on internal nodes). The degree of a component is defined as the sum of degrees of critical points the component contains. See~\cref{sec:exampleMT} for an example.}

\para{Robustness calculation.}
The topological notion of \emph{robustness} quantifies the stability of a critical point \wrt~perturbations of the vector field. Let us define the concept of vector field \emph{perturbation} first. A continuous mapping $h: \Xspace \to \Rspace^2$ is an \emph{$r$-perturbation} of $f$, if $d(f, h) \leq r$, where $d(f, h)=\sup_{x\in \Xspace}||f(x)-h(x)||_2$, and $\sup$ means supremum. 
See~\cite{WangRosenSkraba2013} for some mathematical properties of robustness and lemmas to support critical points cancellation under vector field perturbation. 

The robustness of a critical point can be calculated as the function value of its lowest zero-degree ancestor in the augmented merge tree~\cite{WangRosenSkraba2013}. \myedit{See~\cref{sec:exampleRob} for an example.}



%which ignores the possibility of the occurrences of perturbation within a local neighborhood.
%; see~\cite[Sect. 4]{YanUllrichVan-Roekel2022} for an example. 



\subsection{Multilevel Robustness}
\label{sec:ml-Robustness}
In practice, vector fields generated from large-scale ocean and atmospheric datasets contain features at different scales. The drawback of classic robustness comes from building a single merge tree with critical points in the entire domain, which suffer from undesirable boundary effects~\cite{YanUllrichVan-Roekel2022}. 
To mitigate such drawbacks, Yan~\etal\cite{YanUllrichVan-Roekel2022} introduced a notion of multilevel robustness (reviewed in~\cref{sec:MRdef}).
This notion captures the multiscale nature of the data and mitigates the boundary effects suffered by classic robustness computation. 
It also shows initial promise in critical point tracking in practice. 
%They also proposed a multilevel robustness framework to realize the robustness-based critical point tracking in practice. 
We review how the notion of multilevel robustness can improve the feature-tracking results in~\cref{sec:intrgrateWithFTK}, and we give the pipeline to implement the multilevel robustness framework in~\cref{sec:pipelineMR}. For simplicity and comparative purposes, we refer to this original multilevel robustness-based tracking~\cite{YanUllrichVan-Roekel2022} as the {\MR} framework in the remainder of this paper.

\subsubsection{The Multilevel Robustness}
\label{sec:MRdef}
Roughly speaking, the multilevel robustness of a critical point $x \in \Xspace$ can be defined as a sequence of robustness values computed from its neighborhoods of increasing radii. 
Formally, let $B_x(a)$ denote a ball of radius $a$ with a critical point $x \in \Xspace$ as its center.
The multilevel robustness of $x$ can be expressed as 
$R_x: [0,\infty) \to \Rspace$, where $R_x(a)$ is the (classic) robustness of $x$ computed {\wrt} the domain $B_x(a)$ for $a \in [0,\infty)$.
%see~\cref{fig:AdaptiveRegions} for multiple neighborhoods of $x_1$ and $x_3$ with different radii.
Assuming the domain $\Xspace$ contains $n$ critical points, then for a fixed critical point $x \in \Xspace$, its multilevel robustness will change at most $n-1$ times as $a$ increases, since $x$ gets one more candidate as its the cancellation partner as $B_x(a)$ passes through each critical point. 

In~\cref{fig:Partners} (D-F), we give the exact multilevel robustness of $x_1$, $x_2$, and $x_3$, respectively, where the $x$-axis corresponds to the increasing radii and the $y$-axis represents their classic robustness values. We highlight the radii when the neighborhood includes new critical points with blue points in~\cref{fig:Partners} (D-F).  
In~\cref{fig:Partners} (A-C), we visualize all cancellation partners for selected critical points when we use different sizes of neighborhoods in classic robustness computation. 
The cancellation partners are wrapped in bubbles and colored by the number of times that are referred to as cancellation partners of selected critical points. 
For example, $x_1$ and $y_1$ are paired as cancellation partners 12 times, whereas $x_2$ and $y_2$ are paired 122 times.
The classic robustness of $x_1$ calculated with the entire  domain is infinity, even if it can be canceled with $y_1$ within a $7.85$-degree region under a $17.6$-perturbation. 
This phenomenon happens because $x_1$ represents the center of a large-scale cyclone and is surrounded by flows of a large magnitude. If we build an augmented merge tree from the entire input domain during classic robustness calculation, the lowest ancestor of $x_1$ will be the ancestor of the most critical points in the domain. 
This limitation explains why $x_1$ has potential cancellation partners across the entire domain and may not be able to find its cancellation partner if the degree of the entire domain is not equal to zero. See~\cite[Fig. 2]{YanUllrichVan-Roekel2022} for another example. 
 
Therefore, the drawback of classic robustness comes from building a single merge tree with critical points in the entire domain, which ignores the possibility of the occurrences of cancellation within a local neighborhood. The definition of multilevel robustness successfully captures the multiscale nature of the data and mitigates the drawbacks of the classic robustness computation. However, computing the multilevel robustness exactly is time-consuming. For the vector field containing $n$ critical points, we need to conduct $n \times (n-1)$ classic robustness computations. 
In~\cite{YanUllrichVan-Roekel2022}, the {\MR} framework approximates the exact multilevel robustness by using $N$-level robustness. That is, for a critical point $x \in \Xspace$, the authors considered $N$ number of its neighborhoods at radius $\{a_0, \dots, a_{N-1}\}$, where each $a_i := L \times (i+1)/N$ and $L$ is the diameter of the domain $\Xspace$. 
In this case, the approximations of multilevel robustness for all critical points require $n \times N$ classic robustness computations and work well in their applications.




\subsubsection{Enhancing Feature Tracking with Multilevel Robustness}
\label{sec:intrgrateWithFTK} 
The multilevel robustness can be integrated with any existing feature-tracking algorithms to improve the understanding of vector field dynamics.
Yan et al.~\cite{YanUllrichVan-Roekel2022} utilized the minimum multilevel robustness $\minR_{x} := \min_{a \in [0,~L)} R_x(a) $ for their visualization tasks, since $\minR_{x}$ approximates the smallest possible amount of perturbation to the vector field necessary to cancel each critical point. 
The authors 
%\cite{YanUllrichVan-Roekel2022} 
integrated the $\minR_{x}$ with FTK~\cite{GuoLenzXu2021}, a state-of-the-art feature-tracking technique. 
We also utilize FTK in {\tool}.

The initial critical point tracks from FTK suffer from visual clutter when we deal with large-scale datasets. 
\cref{fig:filterBeforeMR} (A) shows the FTK tracking result for the $\EFour$ dataset whose time steps range from 06/01/2004 to 10/31/2004 with a six-hour time gap (see~\cref{sec:evaluation,sec:data} for details). 
Because of visual clutter among thousands of tracks, it is hard for us to identify the dominant features. 
Since the FTK algorithm considers only the correspondences of critical points based on 0-levelset extraction, some important features (\eg,~centers of cyclones) will be included in the same track with other noisy features. 
\cref{fig:filterBeforeMR} (C) shows one of the FTK tracks from \cref{fig:filterBeforeMR} (A). 
This long track contains a Category 3 hurricane, named Jeanne, as highlighted with the blue curve in~\cref{fig:filterBeforeMR} (C). However, it also contains unstable features on the Gulf of Mexico; indicated within the orange box of ~\cref{fig:filterBeforeMR} (C). 

The main idea of enhancing feature tracking with multilevel robustness is to segment and reconnect the initial tracks obtained by FTK considering the minimum multilevel robustness. 
The {\MR} framework can break initial FTK tracks into more meaningful segments with similar robustness values.  
In the example of~\cref{fig:filterBeforeMR} (C), the {\MR} framework can extract the part highlighted with the blue curve from the other part of the track.
This framework can also remove unstable features in the middle of a meaningful track and reconnect remaining parts as a new track after examining spatial faces and spacetime edges~\cite{GuoLenzXu2021} of breakpoints; see~\cite[Sect. 5.1]{YanUllrichVan-Roekel2022} for a concrete example.  


\subsubsection{Pipeline of the Multilevel Robustness Framework}
\label{sec:pipelineMR}
As shown in~\cref{fig:pipeline} (orange arrows and indices), the implementation of {\MR} framework involves the following three steps: 

\para{Step 1:~multilevel robustness calculation.} 
%Multilevel robustness is defined as a sequence of robustness values computed from its neighborhoods with increasing radii. 
The {\MR} framework calculates the multilevel robustness for all detected critical points with evenly increased radii until the neighborhood includes the entire input domain. 
Then, the minimum multilevel robustness is calculation for postprocessing.

\para{Step 2:~integration with feature tracking.}
The {\MR} framework integrates the minimum multilevel robustness with FTK~\cite{GuoLenzXu2021} to enhance the original FTK tracking results.
%The main idea is to add a step to segment/reconnect the original FTK tracks into more meaningful segments, based on the multilevel robustness.

\para{Step 3:~feature selection.}
The {\MR} framework utilizes two filters based on multilevel robustness and degree information of tracks for feature selection. 
These feature selection strategies can help users reduce visual clutter and highlight dominant features in the domain.

{\tool} 
%inherits two key capabilities of the multilevel robustness framework: capture the multiscale nature of the data and enhance critical points tracking results to understand the vector field dynamics. It 
reuses the notion of multilevel robustness, described in~\cref{sec:MRdef}, and the method to integrate the minimum multilevel robustness with FTK; see~\cref{sec:intrgrateWithFTK}. 
In the following section, we customize the {\MR} framework to {\tool} by integrating the physical knowledge of TCs in feature extraction and tracking.



