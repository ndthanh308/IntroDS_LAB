\section{Related Work}
\label{sec:related}

We review related work on traditional TC tracking algorithms and critical point tracking from vector field topology.

\subsection{Tropical Cyclone Tracking Algorithms}
TC tracking has been studied over the past decades for weather forecasting and climate analysis to issue early warnings, assess impacted areas, and provide risk assessments for public and critical infrastructures.
Most of the \myedit{traditional} TC tracking algorithms require multiple dynamic and thermodynamic variables at different altitudes to detect and track a TC. These algorithms determine a detected feature as a TC candidate by tuning parameter thresholds; however, the choices of thresholds are generally subjective~\cite{EnzEngelmannLohmann2022}.
An example of a traditional TC tracking algorithms is TRACK~\cite{HodgesCobbVidale2017}, which uses relative vorticity at 850, 700, 600, 500, and 250 hPa as the key variable. 
TRACK uses certain criteria (still based on vorticity) during post-tracking to isolate the warm core of TCs. 
The criteria must be jointly attained for at least one day. 
Another example of \myedit{a traditional TC tracker }is {\TE}~\cite{ZarzyckiUllrich2017,UllrichZarzycki2017}, which is used in this study for comparison with {\tool} in~\cref{sec:results}. 
{\TE} uses sea level pressure (SLP) as its key feature-tracking variable. 
TC candidates are initially identified by minima and a closed contour of the SLP field. Next, a geopotential height difference between 250 to 500 hPa is used to refine the candidate definitions. 
Instead of the living time, {\TE}~requires that a tracked storm travels at least $8^\circ$ between 10 N to 40 N latitudes. 
Both TRACK and {\TE} algorithms and many others (\eg,~\cite{biswas2018hurricane}) require many input variables that are often not readily available from raw model output and need additional calculations, which can involve handling a big amount of data. 
Also, these algorithms are not exactly tracking the eyes of TCs. They typically identify minimum SLP and maximum vorticity, where the tracked path of a TC is not exactly along the eye of the TC, and sometimes the path can even be biased toward the eyewall (a ring of tall thunderstorms that produce heavy rains and usually the strongest winds).
%, which has significantly different wind speeds. 

\myedit{Recently, several contour-based and topology-based feature tracking approaches have been proposed~\cite{WischgollScheuermann2001, EdelsbrunnerHarerMascarenhas2004, SohnBajaj2005, ReininghausKastenWeinkauf2012, DoraiswamyNatarajanNanjundiah2013, SkrabaWang2014, BremerWeberTierny2010} that have potential use in TC tracking. 
Most approaches identify features of interest with regions enclosed by streamlines or level sets (\ie, isosurfaces or contours).
Wischgol~\etal~\cite{WischgollScheuermann2001} discussed closed streamlines in 2D fields, which can represent the eyes of the TCs and indicate locations and sizes of the eyes in TC tracking. 
Correspondences between features in consecutive time steps can be identified via spatial overlap~\cite{SohnBajaj2005}, critical point tracking~\cite{SkrabaWang2014}, tree structure~\cite{BremerWeberTierny2010}, or combinatorial feature flow fields~\cite{ReininghausKastenWeinkauf2012}. Critical point tracking is most relevant to our framework, which is reviewed next. 
}

%These bias will further influence downstream applications as well as risk assessment when considering the potential impact of identified TCs.   

%\lin{Can shorten the following paragraph if exceed the page limit.}
%Tracking the TC eyes traditionally was done manually by tracing the movement of spiral rainbands or by overlaying spiral templates on remote sensing images for the best match~\cite{SivaramakrishnanSelvam1966}. However, the process of eye tracing is  difficult in practice because not every TC has the easily recognizable feature of an eye for forecasters to pinpoint. 

%Several methods also were developed to identify typhoon centers by using mass observations such as global positioning system dropsonde and aircraft data~\cite{Kepert2005} and Doppler radar velocity and reflectivity fields (\eg,~\cite{LeeMarks2000}). These methods characterized the eye region with weak or no echoes and surrounded by a more or less complete wall of deep convection %where the extreme tangential and ascending wind velocities are located~
%\cite{Willoughby1998,Blackwell2000}. While these technologies are vital for forecasting the TC eye and impactful areas, they are subjective and take time to develop. %which may not meet the requirement of timely forecasting to the general public. 
%In addition, observations like drop sonde and aircraft are  available only for a relatively short time period and when the TCs are not too strong. Such data limitations prevent us from studying the long-term trend of TCs and features of the very strong TCs, which is a bigger concern than weaker TCs when assessing the impacts and risks at both weather and climate scales. %such as category 3-5 hurricanes.  

\subsection{Vector Field Topology: Critical Point Tracking}
Critical point tracking establishes the correspondences between critical points in successive time steps. 
It is a key tool from vector field topology and plays an important role in understanding the behavior of time-varying vector fields.
Algorithms for critical point tracking may be classified as proximity-, integral-, and interpolation-based methods. 
Proximity-based methods (e.g.,~\cite{HelmanHesselink1989, HelmanHesselink1990}) find correspondences of critical points based on distance proximity in the domain. 
Integral-based approaches represent the tracking of critical points as streamlines of a higher dimensional field, called the feature flow field (FFF)~\cite{TheiselSeidel2003,WeinkaufTheiselVan-Gelder2010, ReininghausKastenWeinkauf2012}, and compute feature tracks based on tangent curves in FFF. 
Interpolation-based methods take into account the time as an additional dimension in addition to the space domain~\cite{TricocheScheuermannHagen2001a,TricocheWischgollScheuermann2002, GarthTricocheScheuermann2004,GuoLenzXu2021}.

The robustness of critical points has been introduced to quantify the structural stability of critical points~\cite{SkrabaRosenWang2016} and has been used in vector field simplification~\cite{SkrabaWangChen2014, SkrabaWangChen2015}, feature extraction~\cite{WangBujackRosen2017}, and visualization~\cite{WangRosenSkraba2013}. 
Skraba and Wang~\cite{SkrabaWang2014} showed the potential use of robustness in feature tracking, that is, finding correspondences between critical points based on their closeness in stability, measured by robustness, instead of just distance proximity within the domain.
Building on the theoretical basis established by~\cite{SkrabaWang2014}, Yan~\etal~\cite{YanUllrichVan-Roekel2022} proposed a multilevel robustness ({\MR}) framework to realize critical point tracking in practice for large-scale scientific simulations, see~\cref{sec:ml-Robustness} for details.


%\subsection{{\tool} vs. the Multilevel Robustness Framework}
%\label{sec:diffsWithMLRob}

%While {\tool} is customized from the multilevel robustness framework~\cite{YanUllrichVan-Roekel2022}, it improves upon the multilevel robustness framework in the following aspects: %terms of efficiency, accuracy, and filtering strategies for specific applications.

%First, as a physics-informed tracking algorithm, {\tool} focuses only on tracking cyclone-like features and can conduct multilevel robustness calculations on these candidates. Such a strategy helps {\tool} filter out about $90\%$ critical points and makes {\tool} much more efficient than the multilevel robustness framework.

%Second, %the multilevel robustness framework approximates the ground truth of multilevel robustness with a discrete set of evenly spaced radii. 
%{\tool} considers the all possible cancellation partners for cyclone-like critical points within a physics-informed size of neighborhood using adaptive levels during multilevel robustness computation. Such a strategy is more appropriate for TC tracking compared with the multilevel robustness framework. The multilevel robustness framework approximates the ground truth of multilevel robustness with a discrete set of evenly spaced radii, which involves a much larger space than a real TC event usually covers. This can cause cancellation of a pair of partners that are far away from each other, which is not necessarily in one TC event. {\tool} however, considers the true minimum amount of perturbation that is necessary for critical point cancellation within the selected neighborhood. %It can also avoid getting a pair of cancellation partners that are far away from each other, which is counter-intuitive in studying real-world cyclones.

%Third, since the centers of cyclones can be usually modeled as sources or centers, trajectories consists saddles would be removed when we postprocess FTK trajectories. We also do not need the degree-based filter function in feature selection, since critical points whose degrees are $-1$ are silently excluded before feature selection. On the other hand, we include two more additional filters: the maximum 5-degree-maximum wind speed along trajectory and smoothness of trajectory, considering the physical properties of real-world cyclones.

% by Jiali: i am trying to rewrite a bit of the third one, but need to understand a few terms before making it better.
%Third, {\tool} is designed to track TCs which has very low or zero wind speed at the center, and has anti-clock wise circulations around the wall of TC structure. Therefore, trajectories consists saddles are removed for further tracking or analysis or calculations. We also do not need the degree-based filter function in feature selection, since critical points whose degrees are $-1$ are silently excluded before feature selection. Importantly, we include two additional filters: the maximum wind speed along the trajectory and smoothness of trajectory, considering the physical properties of real-world cyclones  
