\section{\myedit{An Example of Merge Tree}}
\label{sec:exampleMT}

We now give an example of an augmented merge tree that is constructed from the $\EW$ dataset.
First, we define a scalar field $f_0:\Xspace \to \Rspace$ by assigning the vector magnitude to each point $x \in \Xspace$, that is, $f_0(x) = ||f(x)||_2$; see~\cref{fig:CalRob}(A) and (B), which visualize the vector field $f$ and scalar field $f_0$.
Second, we track the merging behavior of components that contain critical points of $f$. For example, the component containing $x_1$ and $x_2$ merges with the component containing $x_3$ at $r=0.84$ and forms $C_3$ in $\Xspace_{0.84}$, which is represented by the purple and green region in~\cref{fig:CalRob} (D).
Third, we augment the merge tree with the degrees of critical points (on leaves) and the degrees of components (on internal nodes). For example, component $C_1$ in~\cref{fig:CalRob} (D) contains critical points $x_1$ and $x_2$, whose degrees are $+1$ and $-1$, respectively. The degree of $C_1$ is $\mydeg(x_1)+ \mydeg(x_2)=0$, i.e., $\mydeg(C_1)=0$. The augmented merge tree of~\cref{fig:CalRob} (A) is shown in~\cref{fig:CalRob} (E).



\section{\myedit{An Example of Robustness Calculation}}
\label{sec:exampleRob}

Using the example in~\cref{fig:CalRob} (A)-(E), we now show how to calculate robustness with an augmented merge tree.
As pointed out in~\cref{sec:classicRobustness}, the robustness of a critical point can be calculated as the function value of its lowest zero-degree ancestor in the augmented merge tree.
The robustness of $x_1$ and $x_2$ is 0.65, whereas the robustness of $x_3$ and $x_4$ is 14.7.
Intuitively, for the example in~\cref{fig:CalRob} (A)-(E), it is easier for $x_1$ and $x_2$ to be canceled with each other than $x_3$ and $x_4$, since they have much lower robustness values. In~\cref{fig:CalRob} (F), we give the vector field from the same dataset but one time step (6 hours) behind the vector field of ~\cref{fig:CalRob} (A). We see $x_1$ and $x_2$ disappear, whereas $x_3$ and $x_4$ remain.

% Figure environment removed


\section{Details on Dataset and Methods}
\label{sec:data}

We demonstrate the performance of {\tool} using 30-year (1981--2010) near-surface wind vector field from the ECMWF Reanalysis v5 (ERA5). It is produced by the Copernicus Climate Change Service (C3S)~\cite{C3S}. ERA5 provides hourly estimates of the global climate information with a spatial grid resolution of 30 km. 
%In this paper, we focuses on the analysis of tropical cyclones/storms with the time period 1981-2010. 
Since tropical cyclones/storms usually occur during June and October, we limit our dataset with a time window from June $1$ to October $31$ every year at standard synoptic reporting times (0000, 0600, 1200, and 1800 UTC).
A rectangle region on the Atlantic Ocean ($5$ N$^\circ$ to $49.5$ N$^\circ$ and $98$ W$^\circ$ to $18$ W$^\circ$) is selected.
We utilize 10-meter zonal and meridional wind speed as the 2D vector field, since in the near-surface the hurricane core represents a region of strong convergence and associated vertical motion.
We annotate this 30-year dataset as the $\EW$ dataset. We also mark the one-year subset data from the $\EW$ dataset as $\EY$; for example, \cref{fig:filterBeforeMR} uses the $\EFour$ dataset. 
%% Figure environment removed

%\para{Method}.

We use the International Best Track Archive for Climate Stewardship (IBTrACS~\cite{KnappKrukLevinson2010} version 4) observations as the reference.
IBTrACS is compiled from quality-controlled records from various forecasting centers. 
In this paper, we select TCs reported by World Meteorological Organization (WMO) official forecast centers. Again, only tropical cyclones/storms within the region of the $\EW$ dataset are visualized.
For comparison purposes, we include the TC tracking results of the \TE~software package~\cite{UllrichZarzycki2017} applied to the $\EW$ dataset with the parameter setting following~\cite{ZarzyckiUllrich2017}. 
%We use these tracking results to make a comparison between {\tool} and the state-of-the-art method using the same dataset.