Tropical cyclones (TCs) are the largest drivers of losses among natural hazards, bringing wind gusts, high waves, storm surges, and heavy rainfall. In order to achieve improved forecasts, robust risk assessment, and confident future projections of TCs, realistically detecting and tracking TCs are critical \cite{walsh2016tropical, marchok2021important, bourdin2022intercomparison}. In particular, the eye (a region of mostly calm weather at the center of TCs) is a signature feature of a mature TC. 
Therefore, knowing the location and the movement of the eye precisely and in a timely manner is crucial for weather centers issuing warnings to the general public \cite{wong2008automatic,chang2009algorithm}. Potential forecast track errors, due to forecasting models or tracker issues, could  have negative impacts on downstream applications, such as the selection of areas under watch and warnings, and false input for storm surge models over certain coastal regions~\cite{TaylorGlahn2008}. 
In addition, since the 1980s, there have been  increasing trends in TC intensities (particularly in strong categories, such as Categories 4 and 5 hurricanes) based on the long-term observation data in the Atlantic basin~\cite{HollandBruyere2014,KossinOlanderKnapp2013}.

In the past decades, many TC trackers were developed by research institutes and weather forecast centers~\cite{CamargoZebiak2002,KleppekMuccioneRaible2008, UllrichZarzycki2017, HodgesCobbVidale2017, BellChandTory2018}. 
Most of them focused on tracking the TC regions instead of TC eyes, and they used different weather variables (e.g., minimum sea level pressure or maximum vorticity) as the basis to identify a TC candidate before applying various thresholds~\cite{HodgesCobbVidale2017, ZarzyckiUllrich2017,UllrichZarzycki2017}. 
Some efforts focused on fixing the TC eyes during tracking (e.g.,~\cite{SivaramakrishnanSelvam1966, Kepert2005, LeeMarks2000, Willoughby1998,Blackwell2000}). 

Vector field topology has seen widespread applications in science and engineering since its introduction to visualization more than 30 years ago~\cite{HelmanHesselink1989}, including climate study and ocean modeling~\cite{EngelkeMasoodBeren2020,NilssonEngelkeFriederici2020,DoraiswamyNatarajanNanjundiah2013}. 
It has been one of the most promising tools to describe and interpret vector field behaviors by providing meaningful abstraction and summarization, especially for large-scale scientific data~\cite{BujackYanHotz2020}. 
Critical points (\ie, where a vector field vanishes) are core features of vector field topology, and they can be used for studying TCs, since TC eyes can naturally be identified as critical points of the wind vector fields. 
Thus, the tracking of TCs can be converted to critical point tracking in vector field topology. 

Many algorithms have been developed to find the correspondences between critical points in successive time steps in the form of  tracks (trajectories). 
Most critical point tracking algorithms infer correspondences between critical points based on distance proximity~\cite{HelmanHesselink1990, ReininghausKastenWeinkauf2012, GuoLenzXu2021}, which may produce artifacts in TC tracking.
Wang~\etal~\cite{WangRosenSkraba2013} introduced a topological notion of robustness  to quantify the structural stability of critical points. The robustness of a critical point is the minimum amount of perturbation to the vector field necessary to cancel it. 
Skraba and Wang~\cite{SkrabaWang2014} established the theoretical foundation to relate critical point tracking with robustness: critical points with high robustness values could be tracked more easily and more accurately. 

Recently, Yan~\etal~\cite{YanUllrichVan-Roekel2022} brought this theory to practice by introducing a multilevel robustness framework for the study of 2D time-varying vector fields, which has demonstrated its potential in cyclone tracking (this is referred to as the {\MR} framework for comparison purpose).  
The multilevel robustness can be integrated with state-of-the-art feature-tracking algorithms, such as the Feature Tracking Kit (FTK)~\cite{GuoLenzXu2021}, to improve tracking results. 
An advantage is that it identifies cyclonic features using only 2D wind vector fields, which is encouraging as most TC tracking algorithms require multiple dynamic and thermodynamic variables at different altitudes. A disadvantage is that the framework does not scale well for datasets containing a large number of cyclones. 

\para{Contributions.} 
We introduce a topologically robust physics-informed tracking framework ({\tool}) for TC tracking. 
The main idea is to integrate physical knowledge of TC to drastically improve the computational efficiency of the multilevel robustness framework for large-scale climate datasets. 
Our newly designed framework, {\tool}, inherits the capability of critical point tracking based on multilevel robustness~\cite{YanUllrichVan-Roekel2022}. 
First, {\tool} tracks the TC eyes instead of the TC impact areas. 
Second, it is super lightweight, requiring only near-surface wind speeds and directions. 
Third, it is able to work with any TC/critical point tracking algorithms.  
In particular, {\tool} is customized by adding a number of physics-informed strategies on top of the {\MR} framework, making TC tracking more efficient and accurate for large-scale climate datasets. 
Our contributions are three-fold. 

First, we introduce a physics-informed feature selection strategy to filter short-lived and unstable features. Such a strategy removes $90\%$ of critical points in the multilevel robustness computation and makes {\tool} much more efficient than the previous approach~\cite{YanUllrichVan-Roekel2022}.

Second, we propose an adaptive strategy to use physics-informed local neighborhoods for the multilevel robustness computation, making it more efficient and physically meaningful under the real-world scenario. 

Third, we apply {\tool} to 30 years of reanalysis data from ERA5. 
We demonstrate that {\tool} can achieve TC tracking results comparable to and sometimes even better than those of the traditionally well-validated TC tracking algorithm,  TempestExtemes~\cite{ZarzyckiUllrich2017,UllrichZarzycki2017}.
These experimental results are encouraging since {\tool} only requires 2D wind vector field data at the near-surface, whereas the traditional TC tracking algorithms need far more variables at various altitudes.
As pointed out by Bujack~\etal~\cite{BujackYanHotz2020}, it is difficult to interpret flow topology~\wrt~physical meaning in the time-varying setting.
Our comparison between {\tool} and TempestExtemes builds a bridge between tracking methods based on vector field topology and those based on multivariate scalar fields, which helps increase the physical interpretability of vector field topology.

%\jiali{Lin, if you have time, can you read this Bourdin paper and see whether they mention any other ERA5 problems regarding TCs~\cite{walsh2016tropical, marchok2021important, bourdin2022intercomparison}}
	

