\section{Method: {\tool} for Cyclone Tracking}
\label{sec:TROPHY}

Our physics-informed tracking framework, {\tool}, encodes several criteria considering the physical properties of cyclones.  
These criteria make {\tool} more efficient and appropriate for cyclone tracking than the {\MR} framework. 

\para{Overview of {\tool}.}
An overview of our pipeline is shown in~\cref{fig:pipeline}. 
First, we compute the FTK tracking result for the whole dataset and the classic robustness of the critical points using the entire domain for each time step. 
Second, the FTK tracks and the robustness of critical points are used in the physics-informed feature selection (\cref{sec:FeatureSeclectionBeforeMR}) to filter out noise-liked tracks. 
Third, an adaptive-level strategy (\cref{sec:adaptiveLevel}) is applied in the multilevel robustness calculation for selected critical points.
Fourth, {\tool} integrates the minimum multilevel robustness with FTK to enhance the original FTK tracking results. 
This step uses the same strategy with the {\MR} framework; see~\cref{sec:intrgrateWithFTK} or~\cite[Sect. 5.1]{YanUllrichVan-Roekel2022} for a detailed discussion.
Finally, we utilize one \emph{stability filter} function from~\cite{YanUllrichVan-Roekel2022} and propose two additional physics-informed filter functions to highlight cyclones; see~\cref{sec:FeatureSelection}.


\subsection{Physics-Informed Feature Selection}
\label{sec:FeatureSeclectionBeforeMR}

We now introduce a physics-informed feature selection strategy during preprocessing, making {\tool} much more efficient than the {\MR} framework in studying large-scale datasets.  
The {\MR} framework computes multilevel robustness for all detected critical points. 
For the example in~\cref{fig:filterBeforeMR} (A), 102,720 critical points and 22,743 tracks are detected with FTK. 
Suppose we set the number of levels $N= 50$, the {\MR} framework needs to conduct $102720 \times 50$ times classic robustness computation. 
{\tool} instead focuses on tracking cyclone-liked features, which selects a subset of critical points/tracks for in-processing. 
Considering the physical properties of real-world cyclones, we incorporate two criteria in physics-informed feature selection before multilevel robustness calculations. 

First, since the duration of a tropical storm/cyclone is usually larger than 1 day, we require the selected tracks to contain at least one 1-day segment that consists of $+1$ degree critical points only. 

Second, since a critical point representing the center of a cyclone usually has a high stability measure across time before it hits the land and dissipates, we require the segment detected from the previous step to have a high average robustness value.
Based on domain knowledge, a tropical storm must have maximum sustained winds of at least 17.5 m/s. 
We set the threshold to 1.75 for track filtering. 

With these two requirements, the numbers of critical points and tracks decrease to 9,784 and 86, respectively; compare~\cref{fig:filterBeforeMR} (A) and (B). 
Thus, {\tool} needs to compute multilevel robustness for only $9.5\%$ of critical points compared with the original {\MR} framework.

% Figure environment removed





\subsection{Adaptive Levels for Multilevel Robustness}
\label{sec:adaptiveLevel}

We next propose an adaptive level strategy for the multilevel robustness calculation. 
Such a strategy also considers physical properties of real-world cyclones and leads to more reasonable minimum multilevel robustness values for {\tool} than does the {\MR} framework. 
As discussed in~\cref{sec:MRdef}, the {\MR} framework approximates the exact multilevel robustness with a set of evenly spaced radii. 
This strategy may lead to two consequences that are counterintuitive in real-world cyclone analysis. 
First, a critical point can be canceled with other critical points that are far away in the known data domain due to boundary effects; see partners for $x_1$ and $x_2$ in~\cref{fig:Partners} (A) and (B). 
Although such cancellations are mathematically justifiable, it is almost impossible to happen in real-world scenarios   since no perturbation could happen across the whole Atlantic Ocean. 
Second, the {\MR} framework may not find the true minimum multilevel robustness value due to sampling. 
It may also waste computational resources when an enlarged neighborhood does not include potential cancellation partners. For example, increasing the radius of the neighborhood from 60 to 80 for $x_3$ in~\cref{fig:Partners} (C) does not lead to new cancellation candidates.  

To mitigate the drawbacks of the {\MR} framework, we introduce an adaptive-level strategy. 
First, we set a physics-informed neighborhood size, where real-world perturbation could happen. 
We set this radius to be 10 degrees since hurricanes are among the most destructive real-world perturbations to wind field and are typically about 4.7 degrees wide. 
The {\tool} considers all possible cancellation partners within this neighborhood using varying radii.  
\cref{fig:AdaptiveRegions} (A) illustrates our adaptive level strategy in calculating the multilevel robustness. 
For a critical point $x$, we consider all critical points within its neighborhood at radius 10. 
Suppose there are $N$ ($N \ge 10$) critical points in this neighborhood and their Euclidean distances to $x$ are $\{a_0, \dots, a_{N-1}\}$. We compute the classic robustness of $x$ within neighborhood defined by these radii, giving rise to its multilevel robustness. 
If $N < 10$, we select an additional $10-N$ closest critical points outside of the selected neighborhood for more candidates of cancellation partners; see~\cref{fig:AdaptiveRegions} (B). 
However, in most cases, the cancellation partners for the true minimum multilevel robustness are located in our physics-informed neighborhood, for example, $7.85$ degree for $x_1$, $4.15$ degree for $x_2$, and $6.05$ degree for $x_3$ in~\cref{fig:Partners}.

% Figure environment removed

\subsection{TC Feature Selection for Visualization} 
\label{sec:FeatureSelection}

We now present feature selection aided by the minimum multilevel robustness and physical properties of TCs.
We inherit one \emph{stability filter} from the {\MR} framework.  
This filter considers the minimum multilevel robustness $\minR_x$ and its temporal stability in terms of lifespan. 
Let $l$ denote a logistic transformation of $\minR_x$, which maps the $\minR_x \in [0, \infty]$ to $l(\minR_x) \in [0,1]$. This normalization is defined as 
 \[
 l(\minR_x)=\frac{2}{1+e^{-k\cdot \minR_x}}-1,
 \]
where $k$ is the logistic growth rate; see~\cite[Sect. 5.1]{YanUllrichVan-Roekel2022} for a detailed discussion on the benefit of this normalization and parameter selection for $k$.
Now let $\gamma$ denote a track,  $|\gamma|$ is its total length. 
The \emph{stability} of a track $\gamma$ is defined as
\begin{equation}
\label{eqn:stability}
  b(\gamma) := \frac{\sum_{x \in \gamma}l(\minR_x)}{|\gamma|} \cdot \frac{t_\gamma}{T},
\end{equation}
where $T$ is the temporal span of the input dataset and $t_\gamma$ is the temporal span of $\gamma$.
The first term in Eq.~\eqref{eqn:stability} captures the average pointwise stability (in a logistic scale), whereas the second term encodes the lifespan of the track. 
By definition, $b(\gamma) \in [0, 1]$. 
 
Our second feature selection strategy is referred to as  \emph{maximum wind speed} (MWS) \emph{filter}.  
Since the storm systems are usually categorized by the Saffir--Simpson Hurricane Wind Scale, which considers only a hurricane's maximum sustained wind speed, we use this filter to control the categories of storm that users want to visualize.
Formally, for a track $\gamma$, its \emph{maximum wind speed} is
\begin{equation}
\label{eqn:windSpeed}
  w(\gamma) := \max_{x \in \gamma} \omega(x),
\end{equation}
where $\omega(x)$ is the MWS within the two-degree region of the center $x$. 

Our third feature selection strategy is referred to as  \emph{smoothness filter}. 
It is used to filter out tracks that are too tortuous to be considered as TC tracks. 
Roughly speaking, we define the smoothness of a track by the average distance between the normalized track and its smooth univariate spline. 
Formally, for a track $\gamma$, its \emph{smoothness} is
\begin{equation}
\label{eqn:smooth}
  s(\gamma) := 1-\frac{\sum_{x \in \gamma}||J(x)-U(J(x))||}{|\gamma|} ,
\end{equation}
where $J$ is a normalization term mapping $x$ to $[0,1]\times [0,1]$ and $U$ maps $J(x)$ to the point on the smooth univariate spline of the normalized $\gamma$. 
{\tool} uses the SciPy Python library to calculate the (degree 3) smooth univariate splines for detected tracks.
\cref{fig:FittingCurve} shows three curves from our climate dataset and their smooth univariate splines. The left track has the lowest average distance and hence the highest smoothness value, whereas the right track has the lowest smoothness value and usually cannot be considered as a TC track.

% Figure environment removed 

In~\cref{fig:filters}, we apply the above three filters to the $\ETen$ dataset (see~\cref{sec:evaluation,sec:data} for details). After integrating multilevel robustness with FTK tracking results, we obtain tracks in~\cref{fig:filters} (A), which still suffer from visual clutter.
By applying the stability filter with a threshold of $0.029$ for $b(\gamma)$, we obtain tracks in~\cref{fig:filters} (B). 
If we keep enlarging the stability threshold, track $a$ will be filtered out before track $b$, even if track $a$ represents the Category 1 hurricane Otto, and track $c$ cannot be found in the National Hurricane Center’s Tropical Cyclone Reports. 
Therefore, we use our MSF filter $w(\gamma)$ based on the MSF along the track and set the threshold to be $13.5$. 
The remaining tracks are shown in~\cref{fig:filters} (D). 
Again, if we keep increasing the threshold for $w(\gamma)$, track $c$ will be filtered out before track $d$, whereas track $c$ represents the Category 1 hurricane Lisa. 
Therefore, we apply our third smoothness filter $s(\gamma)$ in~\cref{fig:filters} (E) and (F). 
Now, {\tool} highlights 8 tracks representing hurricanes/tropical storms that can all be found in the National Hurricane Center’s Tropical Cyclone Reports.
 



