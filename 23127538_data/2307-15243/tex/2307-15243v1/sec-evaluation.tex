\section{Metrics, Data, and Methods for Evaluation}
\label{sec:evaluation}

We present experimental results using using 30-year (1981--2010) near-surface wind vector field from the ECMWF Reanalysis v5 (ERA5). We annotate this 30-year dataset as the $\EW$ dataset. We also mark the one-year subset data from the $\EW$ dataset as $\EY$.
We use the International Best Track Archive for Climate Stewardship (IBTrACS~\cite{KnappKrukLevinson2010} version 4) observations as the reference and the TC tracking results of the \TE~software package~\cite{UllrichZarzycki2017} for comparision. 
The details of datasets, IBTrACS, and \TE~are described in~\cref{sec:data}.
We now describe the metrics used for evaluating the performance of {\tool} in~\cref{sec:metrics}. 
We also discuss the parameter tuning for {\tool} in~\cref{sec:parameter}.

\subsection{Metrics for Evaluating Tropical Cyclones}
\label{sec:metrics}

We review several metrics used for evaluating TCs in climate data. See~\cite{ZarzyckiUllrichReed2021} for detailed descriptions. 

% Figure environment removed

\subsubsection{Storm Climatology and Characteristics}
\para{Annual frequency}, marked as count or $m$ (\#), is measured by the number of discrete storm events.

\para{Annual duration}, marked as $\TCD$ (days), can be defined as 
$ \TCD_m = \frac{1}{4}\sum_{i\in [1, m]} \ocr_{6h, i}$,
where $\ocr_{6h, i}$ is occurrence of 6 hourly-tracked points during the lifetime of storm $i$. 

\para{Storm genesis}, marked as gen, is defined as the first entry for each individual storm’s lifetime.

\para{Storm intensity} can be measured by the minimum sea level pressure at the cyclone center, marked as SLP (hPa), and two-degree maximum 10-meter wind speed, marked as $u_{10}$ ($m/s$).

\para{Latitude of lifetime-maximum intensity}, marked as LMI, is defined as the absolute value of the latitude where a TC reaches its maximum intensity (as defined by maximum $u_{10}$). 

\subsubsection{Statistics}
We employ two statistical techniques to evaluate the above metrics on a 30-year (1981--2010) dataset.

One is the \emph{arithmetic mean}, $\bar{x} = \frac{1}{n}\sum_{i=1}^n x_i$,
which is used in~\cref{sec:AnnualClimatology} to study annual domain-averaged climatology. 
%For example, $\overline{\TCD}$ represents the average annual duration of storms.

The second is the \emph{Pearson correlation} coefficient $r_{xy}$,
 defined as
%\begin{equation}
%\label{eq:PearsonCor}
\[
 r_{xy} = \frac{\sum_{i=1}^n (x_i-\bar{x})(y_i-\bar{y})}{\sqrt{\sum_{i=1}^n (x_i-\bar{x})^2}\sqrt{\sum_{i=1}^n (y_i-\bar{y})^2}},
 \]
%\end{equation}
%where $x$ can be measured items from {\tool} and $y$ can be items from reference. 
which is used in~\cref{sec:spatialClimatology} to evaluate the similarity of metrics {\wrt} temporal (\eg, storm frequency) or spatial (\eg, genesis density) patterns, which are generated by TC tracking results from the tested tracking algorithm and reference observations.

\subsection{Configuration for Feature Selection}
\label{sec:parameter}
For filters introduced in~\cref{sec:FeatureSelection}, we recommend a default value as a threshold for each filter function based on the TC tracks provided by IBTrACS.
First, we calculate the spatial pattern of cumulative track density using TC tracks detected by {\tool} with thresholds of stability $b(\gamma)$ varying from 0.02 to 0.035, MWS $w(\gamma)$ varying from 10 to 14, and smoothness $s(\gamma)$ varying from 0.95 to 0.97.
We also calculate the spatial pattern of cumulative track density from IBTrACS as a baseline. 
Then, we evaluate the similarity between the density patterns from {\tool} and IBTrACS using Pearson correlation, marked as $r_{xy,track}$, which is the most exhaustive measure of TC activity.
We provide a detailed example for $r_{xy,track}$ computation in~\cref{sec:spatialClimatology}.
We suggest the thresholds for $b(\gamma)$, $w(\gamma)$ , and $s(\gamma)$ with the values when $r_{xy,track}$ reaches its maximum, that is, 0.029 for stability $b(\gamma)$, 13.5 for MWS $w(\gamma)$, and 0.967 for smoothness $s(\gamma)$. 
{\tool} also allows users to fine-tune these thresholds independently for individual TC tracks of interest. 
We report performances of {\tool} with both default thresholds and human-in-the-loop thresholds in~\cref{sec:results}. 
We demonstrate that, although the tracking results using default threshold are encouraging, results with the human-in-the-loop option can further improve the performance of {\tool}. 