\documentclass{INTERSPEECH2023}
%\documentclass[a4paper,12pt,]{article}

% 2023-01-06 modified by Simon King (Simon.King@ed.ac.uk)  

% **************************************
% *    DOUBLE-BLIND REVIEW SETTINGS    *
% **************************************
% Comment out \interspeechcameraready when submitting the 
% paper for review.
% If your paper is accepted, uncomment this to produce the
%  'camera ready' version to submit for publication.

\interspeechcameraready 


% **************************************
% *                                    *
% *      STOP !   DO NOT DELETE !      *
% *          READ THIS FIRST           *
% *                                    *
% * This template also includes        *
% * important INSTRUCTIONS that you    *
% * must follow when preparing your    *
% * paper. Read it BEFORE replacing    *
% * the content with your own work.    *
% **************************************

\usepackage{amsmath,graphicx}

% Example definitions.
% --------------------
\def\x{{\mathbf x}}
\def\L{{\cal L}}

% Title.
% ------
\usepackage{booktabs}
\usepackage{multirow}
\usepackage{subcaption}
\usepackage[belowskip=0pt,aboveskip=0pt]{caption}
%\title{Diarisation subtasks are still tasks}
%\title{Joint speech and overlap detection: a comparative study on multiple datasets across speech domains}
\title{Joint speech and overlap detection: a benchmark over multiple audio setup and speech domains}
\name{Martin Lebourdais$^{1\star}$, Théo Mariotte$^{1,2\star}$,  Marie Tahon$^{1}$, Anthony Larcher$^{1}$, \\
\textit{Antoine Laurent$^{1}$, Silvio Montrésor$^{2}$,  Sylvain Meignier$^{1}$, Jean-Hugh Thomas$^{2}$}
\thanks{$^{\star}$ Both authors contributed equally.}
}
%The maximum number of authors in the author list is 20. If the number of contributing authors is more than this, they should be listed in a footnote or the acknowledgement section.
\address{$^{1}$LIUM, $^{2}$LAUM UMR 6613 IA-GS, Le Mans Université 
}
\email{\{first name\}.\{last name\}@univ-lemans.fr}


\begin{document}


\maketitle
 
\begin{abstract}
% 1000 characters. ASCII characters only. No citations.
Voice activity and overlapped speech detection (respectively VAD and OSD) are key pre-processing tasks for speaker diarization.
The final segmentation performance highly relies on the robustness of these sub-tasks.
Recent studies have shown VAD and OSD can be trained jointly using a multi-class classification model.
However, these works are often restricted to a specific speech domain, lacking information about the generalization capacities of the systems.
This paper proposes a complete and new benchmark of different VAD and OSD models, on multiple audio setups (single/multi-channel) and speech domains (e.g. media, meeting...). 
Our 2/3-class systems, which combine a Temporal Convolutional Network with speech representations adapted to the setup, outperform state-of-the-art results.
%(66.8\% F&-score on DiHard)
We show that the joint training of these two tasks offers similar performances in terms of F1-score to two dedicated VAD and OSD systems while reducing the training cost. This unique architecture can also be used for single and multi-channel speech processing.
\end{abstract}

%!TEX root = ../main.tex

\section{Introduction}
\label{sec:intro}

Climate change and the decline of species richness are severe challenges that influence the living conditions of humans around the world.
Especially the dramatic loss of insects~\cite{hallmann2017more,wagner2021insect} plays a crucial role in many ecological processes that affect agriculture and others.
Hence, monitoring insect species populations becomes more important nowadays to better understand insect decline and long-term trends in species distributions.
Furthermore, there are about one million named species on our planet~\cite{stork2018many}, making manual counting of individuals unrealistic.
Consequently, automated monitoring of insects is inevitably required to infer abundance estimations across larger regions.
One possible way is to use camera traps to collect images of insects that computer vision algorithms can then process to recognize the depicted species automatically.

In this paper, we focus on nocturnal insects, mainly nocturnal moths (Lepidoptera).
Even for this subset, there exist hundred thousands of different species worldwide and depending on the habitat, species lists can be narrowed down based on the study region.
For example, image datasets containing hundreds of moth species from Ecuador and Costa Rica are publicly available and can directly be used for evaluating fine-grained recognition algorithms~\cite{Rodner15:FRD}.
Here, we are interested in monitoring moth species in Central Europe.
We present datasets of moth images we have collected so far and our analysis of algorithms for insect localization and species classification.

% Figure environment removed

Our work is part of a larger project called AMMOD\footnote{\scriptsize{AMMOD = \textbf{A}utomated \textbf{M}ultisensor Station for \textbf{M}onitoring \textbf{o}f Bio\textbf{d}iversity (\url{https://ammod.de/})}}, which aims at developing self-sustaining multi-sensor stations for monitoring species diversity~\cite{Waegele22:TAM}.
One component of these stations is a light-based camera trap for nocturnal insects, called the \emph{moth~scanner}~\cite{Radig2021:AVL,Korsch21_DLP}.
It is a non-invasive monitoring system for automatically gathering images at nighttime.
A UV-LED lamp illuminates a white planar surface to attract the insects that land on this surface.
A high-resolution camera takes an image of the whole surface every two minutes.
Our prototype is shown in Figure~\ref{fig:prototype}.

With this setup, we can collect large-scale datasets of nocturnal insects over a long period that can then be used to develop and evaluate appropriate fine-grained species recognition algorithms.
The moth scanner takes several hundred images during one night, and within five months, we collected more than \num{27000} images with our prototype.
In this paper, we refer to the resulting dataset as the \emph{nocturnal insects dataset~(NID)}, and more details are given in Section~\ref{sec:dataset}.
Note that this dataset is supposed to be extended over time as our system will be in operation within the following years.
We plan to maintain multiple sensor stations in parallel at different locations.
Hence, it has the potential to become a valuable source for large-scale learning and continuous learning within a fine-grained domain.

Besides its impact on research in fine-grained recognition, our developments for automated visual monitoring of nocturnal insects are beneficial for ecologists.
Until now, insect monitoring is mainly done by hand and supported by citizen scientists who manually take images of individual insects in their gardens. 
Previously, we published an image dataset of nocturnal moths captured manually by citizen scientists, called \emph{\mbox{EU-Moths}} dataset at a local workshop~\cite{Korsch21_DLP}.
This paper also includes a dataset description and our baseline results for insect localization and species classification.
There are two reasons for this.
First, we want to announce this dataset to a broader audience interested in fine-grained recognition because it can directly be used for algorithm development and evaluation.
Second, we want to highlight the challenges for recognition algorithms that arise when processing automatically captured camera trap images compared to manually taken images with hand-held cameras.

In general, our paper aims to promote the application of moth species identification as a fine-grained visual recognition problem.
We underpin this with existing datasets, results of baseline algorithms, and a light-based camera trap setup that will be used during the following years to automatically collect further large-scale image data.
We believe that research on automated visual identification of hundreds to thousands of different nocturnal moth species can have a major impact on developing fine-grained recognition algorithms in general, and we, therefore, want to share our insights and datasets with the community.

%The rest of the paper is structured as follows. 
%After a short review of related work in Section~\ref{sec:related_work}, we describe the two abovementioned datasets containing images of nocturnal moths in Section~\ref{sec:dataset}.
%The algorithms we applied to both datasets are described in Section~\ref{sec:methods} and we present the achieved results in Section~\ref{sec:results}. 
%We discuss challenges of processing automatically captured images with light-based camera traps in Section~\ref{sec:challenges} that are also important to consider for similar projects, followed by conclusions in Section~\ref{sec:conclusions}.

%\todo{REWRITE from here}

%Before the classification can be performed, we need to perform a detection of the insects.
%At this stage, the application of the state-of-the-art detection models like SSD~\cite{liu2016ssd} or YOLO~\cite{redmon2016you} is an obvious step.
%On the other hand, these models are computationally expensive and other light-weight methods like the MCC blob detector~\cite{bjerge2021automated} are more suitable for the application in the field.
%Unfortunately, to evaluate and compare different detection methods suitable benchmark datasets are missing.

%In this paper, we present a new dataset collected with the help of our prototype.
%In the period of five months, we captured over \num{27000} images in suburban area in Middle Germany.
%For bootstrapping and first evaluations we annotated a subset of these images with bounding boxes for the captured insects.
%The image data and the annotations will be soon publicly available.

%As a first baseline for insect detection task, we evaluated two different methods on the data and present these results further in our paper.
%First, we used a well-established Deep Learning detection model capable of identifying multiple objects in an image, namely the single-shot MultiBox detector (SSD)~\cite{liu2016ssd}.
%As a light-weight alternative that can be easily deployed directly at the computationally limited hardware of the moth scanner, we developed and evaluated a multi-step blob detection algorithm.
% \todo{Edge Computing as buzzword? EdgeAI may be wrong here?}
%First, it reduces the power consumption due to the reduction of computations.
%Further, applying the detection directly at the moth scanner, we can drastically reduce the amount of data that needs to be transmitted  when the system will gather data autonomously in the field.
%The algorithm is closely related to the blob detection method proposed by Bjerge~\etal\cite{bjerge2021automated} but mitigates some of the method's limits.
%We present the idea and the improvement in more detail in Sect.~\ref{sec:methods}.

\section{Datasets}
\label{sect:2_datasets}
Our benchmark datasets combine multiple speech domains including far-field audio recordings.
For each dataset, VAD and OSD labels are derived from the provided ground-truth segmentation. 
%The comparative study is conducted over several datasets covering multiple speech domains
%Experiments are conducted on several corpora covering multiple speech domains.
Table~\ref{tab:corpus_desc} summarizes corpus characteristics.
%Datasets are detailed in the following sections.

%Our system is evaluated on several datasets covering different domains.
\begin{table}[ht!]
\centering
\caption{Corpus characteristics. $^\star$multi-microphone data.}
\begin{tabular}{@{}lccc@{}}
\toprule
Corpus   & Domain & Duration & Overlap prop. \\ \midrule
DIHARD   & Multiple    & 34~h      & 11.6\%             \\
ALLIES   & Media    & 328~h     & 3.2\%              \\
ALLIES-clean & Media & 6~h      & 13.9\%             \\
AMI$^\star$      & Meeting   & 100~h     & 24.7\%                   \\
CHiME-5$^\star$  & Dinner party   & 60~h      & 22.9\%            \\ \bottomrule
\end{tabular}

\label{tab:corpus_desc}
\end{table}

%Each corpus is presented in the following sections.
\vspace{-15pt}

\subsection{Single Channel}
%
Single channel experiments are conducted on 3 datasets: ALLIES~\cite{larcher:hal-03262914}, DIHARD~\cite{ryant19_interspeech} and AMI~\cite{Mccowan05theami_short}.
The ALLIES corpus is a soon-to-be-available French meta-corpus designed to gather and extend previous French data collected for diarization and transcription evaluation campaigns.
It consists of 328~h of audio extracted from 1998 to 2014 in 1008 shows with 5901 different speakers. 
The overlap proportion (in duration) fluctuates widely between broadcast news with little to no interaction and debates (around 10\% of overlaps). 
Despite a harmonization effort, the data collected and annotated under different protocols introduces some homogeneity problems~\cite{lebourdais:hal-03660323}. 
15 debate shows, referred to as ALLIES-clean, 
%for which a manual segmentation of speaker activity including overlapping speech is provided, 
were selected in order to get a high overlap proportion, a manual and homogeneous speech segmentation, and diversity in the shows represented. 

%The DIHARD corpus was provided for the eponymous challenge in 2020. 
The DIHARD corpus contains data from 7 domains with various recording qualities, situations, and degrees of spontaneity from read speech to phone conversations.
Since spontaneous speech naturally contains a high proportion of overlapped speech, this corpus is well-suited for OSD. 
This corpus is partitioned as intended for the challenge and evaluated on the official evaluation partition.

The AMI meeting corpus contains recordings of realistic meetings involving up to 5 participants in various environments.
The \textit{headset-mix} is used for single-channel experiments on this dataset. 
The data partition follows the protocol proposed in \cite{landini_but_2020}.

\subsection{Multiple Channels}
%
Multiple-channel experiments are conducted on 2 corpora: AMI~\cite{Mccowan05theami_short} and CHiME-5~\cite{barker2018fifth}.
We select AMI audio data captured by the \textit{Array 1} as a distant multi-microphone signal.
It consists of a uniform circular array (UCA) composed of 8 omnidirectional microphones placed in the center of the table during meetings.

The CHiME-5 dataset contains 20 dinner-party sessions involving 4 participants in a real-home environment. 
Speakers were asked to move between 3 rooms during the party.
Audio signals thus feature a strong background noise diversity with varying acoustic conditions.
Audio signals are captured with 6 linear arrays composed of 4 microphones.
For our experiments, only the first microphone of each array is selected.
Finally, the resulting signal contains 6~channels.

% \subsection{ALLIES}
% The ALLIES corpus~\cite{larcher:hal-03262914_short} is a French metacorpus designed to gather and extend previous French data collected for diarization and transcription evaluation campaigns.
% %
% %from television and radio shows. 
% %It gathers data from diarization evaluation campaigns (Ester1\&2~\cite{galliano-etal-2006-corpus} and Repere\cite{giraudel_repere_2012_short}) and new data gathered for this corpus. 
% It consists of 307h of audio extracted from 1998 to 2014 in more than 1000 TV and radio shows with 5711 different speakers. 
% The overlap proportion (in duration) fluctuates widely between broadcast news in which there is mainly single speakers, and debates which contain around 10\% of overlaps in duration. Despite a harmonization effort, data collected and annotated under different protocols introduces homogeneity problems~\cite{lebourdais:hal-03660323}. 
% 15 debate shows, referred as ALLIES-clean, for which a manual segmentation of speaker activity including overlapping speech is provided, were selected in order to get a high overlap proportion,an homogeneous speech segmentation and the diversity in the shows represented.
% %We manually extract a sub-corpus designated as ALLIES-clean, composed of 15 shows selected to contain overlapping speech and to be representative of the corpus content.

% \subsection{AMI}

% The AMI meeting corpus \cite{Mccowan05theami_short} consists of 100~h of realistic meeting recordings involving up to 5 participants.
% This dataset features close-talk speech signals recorded using individual microphones (e.g. headset, lapel) and distant speech signals collected using microphone arrays.
% In this study, we will only consider the \textit{headset-mix} as close-talk signals and the \textit{array 1} recordings as multi-channel data (referred to as AMI$^\star$).
% The microphone array consists of a uniform circular array (UCA) composed of eight omnidirectional microphones placed in the center of the table during meetings.
% A manual annotation of the speaker activity is provided. VAD and OSD labels are then extracted from this ground truth segmentation.

% \subsection{CHiME-5}
% The CHiME-5 dataset \cite{watanabe2020chime_short} is over 60~h of dinner party recordings divided into 20 sessions.
% Each session involves 4 participants in a real-home environment. 
% Speakers were asked to move between 3 rooms during the party.
% Audio signals are captured with 6 linear arrays composed of 4 microphones.
% For our experiments, we only select the first channel of each microphone array.
% The resulting signal is thus composed of 6 channels.
% This corpus then features realistic recordings with a strong background noise diversity.
% Ground-truth segmentation obtained via manual annotation and forced-alignment is provided.
% VAD and OSD labels are extracted from those.


% \subsection{DIHARD}
% The DIHARD corpus~\cite{ryant19_interspeech} was provided for the eponymous challenge in 2020. 
% It contains data from 7 domains with various recording qualities, situations and degrees of spontaneity from read speech to phone conversations.
% Since spontaneous speech naturally contains a high proportion of overlapped speech, this corpus is well-suited for OSD. 
% This corpus is partitioned as intended for the challenge and evaluated on the official evaluation partition.
% The labels for OSD and VAD are extracted from the ground-truth speaker segmentation provided in the evaluation campaign.

\vspace{-5pt}
\section{System overview}
\label{sect:3_system}
%The proposed segmentation is composed of a feature extraction layer preceding a sequence modeling network.
%The feature extraction is adapted accordingly with the number of channels in the input signal.
Figure~\ref{fig:archi} depicts an overview of VAD, OSD, and VAD+OSD systems.
While the feature extractor (in blue) is adapted with respect to the number of input channels, the sequence modeling network (in purple) processes the sequence of features before the frame classification.
The frame classification is done at a rate of 100~Hz, while the raw waveform is sampled at 16~kHz.
%In the target sequences each class is given with a frequency of 100~Hz, while the raw waveform is sampled at 16~kHz.
\vspace{-10pt}

% Figure environment removed
\vspace{-10pt}

\subsection{Single channel features ($M=1$)}
The single channel feature extractor is based on the WavLM pre-trained model~\cite{chen_wavlm_2022_short}.
This choice is motivated by the performance obtained on the diarization task according to the SUPERB benchmark~\cite{yang_superb_2021_short}.
Furthermore, WavLM has been trained using simulated overlapped speech and is then more robust to this type of data. 
WavLM outputs speech representations every 20~ms. 
In order to align this representation with the target sequence, we decide to add a linear layer on top of the frozen WavLM.
%Our feature extractor is composed of a freezed WavLM model followed by a linear layer. Our target is sampled with a frequency of 100~Hz, WavLM only output features every 20~ms, a transformation is then needed. 
The linear layer aims to transform a segment of 99 features extracted with WavLM over a 2~s window, into a 200-frame vector, supposedly aligned with our target.


\subsection{Multiple channel features ($M>1$)}

When multiple channels are available, feature extraction is performed using the Self-Attention Channel Combinator (SACC) \cite{gong_self-attention_2021_short}.
This architecture has previously shown its efficiency for OSD under distant speech conditions \cite{mariotte22_interspeech}.
The algorithm consists of a self-attention module \cite{vaswani_attention_2017_short} which computes per-channel weights from the multi-channel Short-Time Fourier Transform (STFT) of the input signal.
The channels are then weighted and combined in order to get a single-channel representation.
Combination weights are computed from the multichannel STFT calculated on 25~ms segments with 10~ms shift.
The attention module is composed of a single attention head of size $d=256$.
The combined representation is converted to the log-mel scale using $N_f=64$ filters.
Global Mean and Variance Normalization (MVN) is also applied before feeding the sequence modeling network.

\vspace{-5pt}

\begin{table*}[ht]
\centering
\caption{Overview of the F1-score (\%) for each system on the evaluation set of several corpora covering various domains, $^\star$ indicates multi-microphone data, $\dagger$ indicates that the results are taken from the original article.}
\setlength\tabcolsep{4.5pt}
\begin{tabular}{@{}rlcccccccccc@{}}

\toprule
            && \multicolumn{5}{c}{VAD} & \multicolumn{5}{c}{OSD} \\ 
            \cmidrule(l){3-7} \cmidrule(l){8-12}\\
            && DIHARD    & ALLIES    & AMI   & AMI$^{\star}$   & CHiME$^{\star}$    & DIHARD    & ALLIES   & AMI   & AMI$^{\star}$ & CHiME$^{\star}$\\ \midrule
%            \cmidrule(l){2-6} \cmidrule(l){7-11}\\
%Baselines   & 63.4   & /      & /      & /           &  \\
\multirow{4}{*}{\rotatebox[origin=c]{90}{2-class}}&
VAD (ours)    & 97.0      &  99.8     & 97.4     & 96.4      & 99.8          & -         & -         & -         & -         & -\\
&OSD (ours)    & -         & -         & -        & -         & -          & \textbf{66.2}      & \textbf{71.6}  & \textbf{79.6}   & \textbf{72.2}      & \textbf{75.9}   \\
& Mel+CRNN~\cite{Cornell2022} & - & - & - & - & - & 51.3 & - & 66.0 & 57.2 & -\\ 
& Mel+TCN~\cite{cornell_detecting_2020} & - & - & - & - & - & 54.7 & - & 73.4 & 65.8 & -\\ \midrule
\multirow{4}{*}{\rotatebox[origin=c]{90}{3-class}}&
VAD+OSD (ours) & 97.0      & 89.2     &  97.2     & 96.6     &   99.3      & \textbf{66.8} & \textbf{75.4}      & \textbf{80.4}     & \textbf{71.8}      & \textbf{75.5} \\ 
& Mel+CRNN~\cite{Cornell2022} & - & - & - & - & - & 50.8 & - & 69.6 & 61.2 & -\\ 
& Mel+TCN~\cite{cornell_detecting_2020} & - & - & - & - & - & 54.5 & - & 73.8 & 67.9 & -\\ 
& SincNet+BLSTM~\cite{bredin:hal-03257524}$\dagger$ & - & - & - & - & - & 59.9 & - & 75.3 & - & -\\ \bottomrule
\end{tabular}
\label{tab:overview}
\end{table*}

\subsection{Sequence modeling and classification}

%VAD and OSD tasks rely on the classification of frames.
The sequence modeling network (in purple in Fig.~\ref{fig:archi}) takes as input a sequence $\boldsymbol{x}$ of single or multi-channel features and assigns a class to each frame of this sequence.
%The temporal representations extracted either by single- or multichannel feature extractors are then processed by a sequence modeling network.
%This network learns a representation of the sequence of features and classifies each frame.
This task is performed using a TCN \cite{bai_empirical_2018} since this architecture has shown noticeable results on both VAD and OSD tasks \cite{lebourdais22_interspeech,mariotte22_interspeech,Cornell2022,cornell_detecting_2020}.
It is composed of 5 residual convolutional blocks repeated 3 times.
Classification is performed by a 1-d convolutional layer followed by a $\mathrm{softmax}$ activation function. %to compute classification pseudo-probabilities.
%The global architecture with each configuration is presented in Figure \ref{fig:archi}.

For each frame in the output sequence, the VAD outputs the pseudo-probability to contain at least one speaker $p(N_{spk} \geq 1|\boldsymbol{x})$. 
The OSD outputs the pseudo-probability to contain speech from more than one speaker $p(N_{spk} \geq 2|\boldsymbol{x})$. 
Both VAD and OSD are then binary classifiers.
The joint VAD+OSD system outputs the pseudo-probability of either containing any speech $p(N_{spk}=0|\boldsymbol{x})$, speech from a single speaker $p(N_{spk}=1|\boldsymbol{x})$, or speech from more than one speaker $p(N_{spk}\geq2|\boldsymbol{x})$. 
The 3-class approach is then converted to 2-class VAD and OSD by merging the relevant classes.
%One output is then dedicated to VAD while another one is dedicated to OSD.

\vspace{-5pt}
\subsection{Training and Evaluation}
In order to estimate the robustness over different speech domains, the three systems are trained and evaluated independently on the 5 datasets aforementioned. 
%to evaluate their performance on different speech domains.
%The evaluation in the cross-dataset scenario is not considered and will take part of a future study.
%Segmentation systems are trained on 2 seconds audio segments with associated frame-level labels.
To counteract the small number of overlap segments, 50\% of the training segments are augmented on-the-fly by summing them to another randomly sampled training segment. Associated labels of each segment are also combined \cite{bredin_pyannoteaudio_2020_short}.
The loss function is a cross-entropy, and we used the ADAM optimizer with a learning rate of $lr=10^{-3}$.
Single-channel audio data is augmented with noise extracted from MUSAN~\cite{snyder2015musan} and additional reverberation using simulated room impulse responses.
Preliminary experiments have shown that data augmentation did not bring significant improvement in the far-field scenario. 

%To ensure an adequate proportion of overlap in the training data, artificial overlap is added as a weighted sum of two speech segments.
% the input of each of the channel have also been augmented with noise extracted from MUSAN~\cite{Snyder2015MUSANAM} and reverb by simulated room impulse responses. 

%Each system is evaluated on the test subset of the considered corpus. 
Following the DIHARD evaluation plan, we use the F1-score obtained on the evaluation set as a performance metric.
%The evaluation procedure slightly differs between the 2-class systems -- VAD and OSD -- and joint VAD+OSD.
In the 2-class approach, only the positive class output ($N_{spk} \geq 1$ for VAD, and $N_{spk} \geq 2$ for OSD) is used for prediction and two detection thresholds are applied to predict binary labels \cite{bredin_pyannoteaudio_2020_short}.
In the 3-class approach, the class associated with the maximum $\mathrm{softmax}$ output is selected at the frame level. A working version of the code will soon be released\footnote{Hidden link for anonymous submission}. %\url{https://git-lium.univ-lemans.fr/speaker}

\section{Results}
\label{sec:results}

\subsection{Pareto-optimal YOLO models}

By computing the proxy metric for model accuracy (mAP$_{50-95}$ in VOC training from scratch) and latency values for the whole \textit{YOLOBench} architecture space on several hardware platforms, we have determined the Pareto sets containing the most promising models (in terms of latency-accuracy trade-off) for each HW platform. Merging the first and second Pareto sets for each device into a single list of best architectures, we arrived at a set of 52 backbone/neck combinations for COCO pre-training (same architectures with different input resolutions are considered as the same data points here, since COCO training is regardless done on a fixed resolution of 640x640). After performing the COCO training for these selected models, we fine-tune these models at 11 different resolutions (from 160x160 to 480x480 with a step of 32) on all downstream datasets (except for COCO), resulting in 572 models total for each dataset.

% Figure environment removed


Finally, given fine-tuned model accuracy on several datasets and latency measurements on several devices, we compute the actual Pareto sets for each particular dataset/HW platform combination. Figure \ref{fig:voc_pareto} shows the Pareto frontiers of \textit{YOLOBench} models fine-tuned on the VOC dataset on 4 different devices. One could observe significant differences in the Pareto frontiers between devices. In particular, the Pareto-optimal set for VIM3 NPU is mostly comprised of YOLOv6 models, with some YOLOv5, YOLOv7, and YOLOv8 models present in the higher accuracy region. This is not the case for Pareto sets of Intel and ARM CPUs that, despite containing a few YOLOv6 models in the lower latency region, also contain many YOLOv5 and YOLOv7 variations in the higher accuracy region (with some other models families like YOLOv3 and YOLOv4 also present in a limited capacity). While the Pareto sets for Intel and ARM CPUs are found to be similar to each other, the Pareto set on Jetson Nano GPU is different from the rest of the devices and is very non-uniform in terms of model family distribution, with YOLOv5, YOLOv6, YOLOv7 and YOLOv8 models all represented across the whole accuracy/latency range. Table \ref{tab:pareto_table} shows representative Pareto-optimal models for 3 different datasets (VOC, SKU-110k, WIDERFACE) and 3 hardware platforms under certain latency thresholds. Note that although there are similarities of model family distributions in Pareto sets computed for different datasets (see Appendix B), the exact optimal model for a given latency threshold depends on the specific dataset of interest.

Next, we looked at statistics of Pareto-optimal models depending on the dataset and hardware platform. Figure \ref{fig:pareto_scaling} shows the distribution of depth factor, width factor, and input resolution values in Pareto frontier models for VOC and SKU-110k datasets on Jetson Nano GPU (data for other datasets and devices available in Appendix B). The general trend observed is that models at lower input resolutions mostly have lower depth and width factors. This means that to achieve optimal results in terms of latency-accuracy trade-off, one has to scale down the width and depth of the architecture before lowering the input resolution. This effect is more pronounced in some datasets (SKU-110k and WIDERFACE), where almost all optimal models are either at the maximal resolution we considered (480x480) with variation in width and depth, or at lower resolutions with minimal width and depth factors. This effect is dataset-dependent, as it is observed to be more relaxed for VOC and COCO datasets, where many optimal models with a variation in width and depth factor are found at resolutions lower than 480x480.

To summarize, we have demonstrated that with a state-of-the-art training pipeline and detection head structure, YOLO-based models with various backbone/neck combinations could achieve good latency-accuracy trade-off in various deployment scenarios, including older backbone/neck structures from YOLOv4 and YOLOv3 models. Furthermore, we have shown that depth/width reduction precedes input resolution down-scaling in optimal YOLO-based detectors.

\subsection{Ranking training-free accuracy predictors}
\label{sec:ranking_zc}


With an increasing number of architecture blocks and hyperparameter combinations, the size of the candidate model space in \textit{YOLOBench} can further grow exponentially. Hence, it is important to develop efficient methods of filtering out bad architecture proposals before running them through the full training pipeline, including pre-training on the COCO dataset. In the field of neural architecture search, recent works have proposed a handful of training-free, {\em zero-cost} (ZC) estimators that have been shown to perform well on various (relatively simple) benchmarks \cite{mellor2021neural, abdelfattah2021zero, li2023zico}.

Zero-cost estimators were originally proposed by Mellor et al. \cite{mellor2021neural}, and later expanded by Abdelfattah et al. \cite{abdelfattah2021zero} as a means to quickly evaluate the performance of an architecture using only a mini-batch of data. These estimators work by extracting statistics obtained from a forward (and/or backward) pass of a few mini-batches of data through the network, hence eliminating the need for full training of the model. Despite the fact that over 20 different zero-cost accuracy estimators have been introduced in recent years, simple baselines like the number of parameters and MAC count are still found to be hard to outperform \cite{li2023zico}.

The vast search space of YOLO-like architectures necessitates the development of effective training-free estimators to filter out bad candidates and reduce the search space. We have examined the performance of a representative subset of zero-cost estimators on \textit{YOLOBench}, namely: 
Fisher~\cite{turner2019blockswap}, GradNorm~\cite{abdelfattah2021zero}, GraSP~\cite{wang2020picking}, JacobCov~\cite{abdelfattah2021zero}, Plain~\cite{abdelfattah2021zero}, SNIP~\cite{lee2018snip}, SynFlow~\cite{tanaka2020pruning}, ZiCo~\cite{li2023zico}, Zen-score~\cite{lin2021zen} and NWOT~\cite{mellor2021neural}. The NWOT metric is computed by measuring the Hamming distance between binary codes produced by each layer's activations \cite{mellor2021neural}. Although originally proposed for ReLU-based networks, we found that it works well in practice for YOLO variations, most of which contain SiLU activations. The NWOT metric can be also computed by looking at signs of each layer's output features before the activation layer to form the binary code. We refer to that version of the NWOT metric as  \textit{NWOT (pre-act)} ("pre-activation"), and find that its performance might differ significantly from the original NWOT metric, primarily because the binary codes are computed before the normalization layers followed by the activations. We also compare the performance of the above metrics with simple baselines such as the number of trainable parameters and MAC count, as well as with a training-based proxy that we have used for pre-select models for \textit{YOLOBench} (mAP$_{50-95}$ of training from scratch on the VOC dataset). 

All zero-cost metrics are computed on randomly initialized models using the same loss function as used for training of all \textit{YOLOBench} models, using a single mini-batch of data with a corresponding image resolution (except for ZiCo, which requires two different mini-batches of data). We empirically evaluate the considered set of zero-cost proxies on \textit{YOLOBench} using the following metrics:

\begin{itemize}
    \item Kendall $\tau$ (global): Kendall rank correlation coefficient evaluated on all \textit{YOLOBench} models
    \item Kendall $\tau$ (top-15\%): Kendall rank correlation coefficient evaluated on the top-15\% performing \textit{YOLOBench} models (in terms of mAP$_{50-95}$ value)
    \item Percentage of all actual Pareto-optimal models in the Pareto set determined with the zero-cost estimator in the zero-cost proxy-latency space (recall for Pareto-optimal model prediction using the ZC-based Pareto set)
\end{itemize}

The last metric effectively measures how accurate the computed Pareto set would be if the proxy values are used instead of actual mAP to rank models. It is calculated by determining Pareto fronts for model rankings based on zero-cost proxies (and real latency measurements) and looking at how many models present in the actual Pareto set are also present in the ZC-based Pareto set. In other words, a recall value of 0.7 would mean that by taking the models from the ZC-based Pareto set as candidates, we find 70\% of all actual Pareto-optimal models in that candidate set. We report values for Pareto fronts computed with latency measurements on the Jetson Nano GPU in Table \ref{tab:zc_table}.

We generally found that all of the zero-cost predictors we considered (except for NWOT) were outperformed by the simple baseline of MAC count both in terms of Kendall-Tau scores as well as in the percentage of predicted Pareto-optimal models (see Table \ref{tab:zc_table}). Furthermore, when compared with using mAP$_{50-95}$ on VOC training from scratch as a predictor, we observed that only NWOT came close to it in terms of ranking scores. We have also found that a pre-activation version of NWOT tends to work better than standard NWOT on \textit{YOLOBench}. For the task of predicting mAP$_{50-95}$ of models fine-tuned on SKU-110k, we notably found that pre-activation NWOT outperforms VOC training from scratch metric in terms of Kendall-Tau scores (possibly due to domain difference between VOC and SKU-110k datasets), but the VOC-based proxy metric still performs better for Pareto-optimal model prediction on SKU-110k.

\begin{table}
  \small
    \caption{COCO mAP and inference latency on Raspberry Pi 4 CPU (TFLite, FP32) for YOLOv8s vs. a model identified from the NWOT-latency Pareto frontier. Mean and standard deviation of inference time over 5 runs (each run done for 100 iterations) shown, with 640x640 input resolution.} %
    \label{tab:timm_coco}
    \vspace*{2mm}
    \begin{tabularx}{\columnwidth}{X|X|X} %
      \hline
      {Model} & {COCO mAP$^{val}_{50-95}$} & {Latency, ms}\\
      \hline
      {YOLOv8s} & {0.4364} & {1476.09 $\pm$ 1.49} \\
      \hline
      {YOLOv8s (HSwish)} & {0.4355} & {1381.62 $\pm$ 7.34} \\
      \hline
      {YOLO-FBNetV3-D-PAN-C3} & {0.4463} & {1355.21 $\pm$ 9.93} \\
      \hline
    \end{tabularx}
\end{table}


In trying to capture all the real Pareto-optimal models using ZC scores, one could increase the size of the ZC-based candidate pool by computing the second (third, fourth) ZC-based Pareto set and adding it to the pool of ZC-based candidates. Such a procedure naturally increases the percentage of actual Pareto-optimal models in that pool, and the full set of actual Pareto-optimal models could be found this way by looking only at a portion of the full dataset (e.g. at first $N$ ZC-based Pareto fronts). In this context, we compute candidate pools consisting of $N$ ZC Pareto fronts for each ZC metric and look at the percentage of actual Pareto-optimal models found in the pool versus the pool size (as \% of the full dataset size). Looking at the pool size is motivated by the observation that the number of models in ZC-based Pareto fronts can significantly vary depending on the ZC metric.

Figure \ref{fig:zc_pareto_voc} shows the percentage of predicted real Pareto-optimal models on the VOC dataset contained in pools of $N$ first Pareto fronts for 4 different predictors (VOC training from scratch, NWOT, pre-activation NWOT and MAC count). For ARM and Intel CPUs, we observe a general trend of VOC training from scratch being the best predictor and MAC count being the worst at all points. Interestingly, for Jetson Nano GPU NWOT performs close to VOC training from scratch for $N = 1,2$ but starts to perform worse with more models in the pool. Surprisingly, the training-free predictors MAC count and pre-activation NWOT outperform VOC training from scratch in predicting Pareto-optimal models on VIM3 NPU.

\subsection{Pareto-optimal detector identification using NWOT score}

To demonstrate the potential of using ZC-based Pareto sets in identifying promising detector architectures with good accuracy-latency trade-off, we have additionally generated multiple candidate architectures based on CNN backbones provided by the \texttt{timm} library \cite{rw2019timm}. The architectures were generated by using one of the 347 CNN-based backbones available in \texttt{timm} as a feature extractor followed by a modified Path Aggregation Network (PAN) (same structure with C3 blocks as in YOLOv5 was used, with the number of channels corresponding to YOLOv5s, without the SPPF block) and a YOLOv8 detection head, as in all other \textit{YOLOBench} models. 

We computed the pre-activation NWOT scores as well as measured inference latency on Raspberry Pi 4 ARM CPU with TFLite for all candidate models. We then used the NWOT score and latency values for each model to compute the Pareto frontier in the NWOT-latency space (see Appendix D). We then trained one of the models identified to belong to the NWOT-based Pareto frontier (YOLO with FBNetV3-D backbone) on the COCO dataset with the same setup used to pre-train \textit{YOLOBench} models (640x640 input resolution, 300 epochs, batch size 64, other hyperparameters set to default of Ultralytics YOLOv8 \cite{ultralytics})\footnote{Note that YOLOv8s results provided by Ultralytics \cite{ultralytics} are higher than the ones we have reported, as the models were trained for 500+ epochs on COCO. However, no script to reproduce those results has been released to date.}. The resulting model was found to be more accurate and faster than YOLOv8s (a model in a similar latency range) when tested on Raspberry Pi 4 CPU with TFLite (FP32, XNNPACK backend) (see Table \ref{tab:timm_coco}). Furthermore, we have looked at the accuracy and latency of a YOLOv8s modification with SiLU activations replaced with Hardswish activations (Table \ref{tab:timm_coco}), as we have observed the choice of activation function to be a significant factor affecting TFLite inference latency. We found that the identified NWOT-Pareto model (also containing Hardswish activations in the backbone, neck and head) still outperformed YOLOv8s-HSwish in terms of latency and accuracy.

\vspace{-5pt}

\section{Analysis}
\label{sect:5_analysis}

%The previous section has shown that a joint VAD+OSD system can reach similar or better performance than two dedicated systems on several speech domains.
This section evaluates the benefits of such an approach in terms of training time, speech domains, and spatial information in the multi-channel scenario.
%The error distribution according to speech domain and insights about the spatial information used in the multiple channel scenario are also presented.
\vspace{-5pt}
\subsection{Training time}

In order to assess the value of training a joint VAD+OSD system against two dedicated models, we compare the training time required for each approach.
Each system is trained on an RTX6000 GPU card until it reaches its best F1-score on the validation set.
Figure \ref{fig:train_time_chart} presents the elapsed time to obtain the best-performing model.

% Figure environment removed

2-class OSD task clearly requires more resources than VAD.
%, probably because of the difficulty of the OSD task. 
Indeed the discrimination of the spectral information between the presence of one speaker or several speakers is more difficult than between speech and non-speech signals.
Multi-channel VAD+OSD system converges as fast as the 2-class VAD system, as observed on the AMI$^\star$ and CHiME-5$^\star$ datasets.
In the single channel scenario, the gain is less significant (and no gain at all with DIHARD), probably because the spatial information helps to detect multiple speakers.

\subsection{Influence of the speech domain on performance}

In order to study the influence of the speech domain on OSD, we analyze the OSD F1-score distributions for each of the DIHARD evaluation files, manually separated into 7 domains (see Fig.~\ref{fig:domains}). \textit{Clinical} contains conversations between a clinician and a child, \textit{facetoface} contain interviews, \textit{phone} contains phone conversations, \textit{map task} contains a game in which someone guides a person remotely on a map, \textit{group chat} contains spontaneous conversations, \textit{court} contains court recordings and \textit{audiobook} contains read speech.

% Figure environment removed

Fig.~\ref{fig:domains} shows that the F1-score is globally better for \textit{phone} conversations than for \textit{clinical} and \textit{face-to-face} conversations, despite the fact that the three domains are dyadic interactions. 
We can then hypothesize that the absence of visual cues in phone conversations limits the diversity of overlaps contained in the audio files. 
%The \textit{map task} has a similar distribution as \textit{phone} conversations while showing slightly better results. 
%This shows that the acquisition modality influences the type of overlaps distribution. %created and recognized. 
Another difference between domains is the quality of the recordings. 
For example, \textit{group chat} and \textit{face-to-face} files 
%all have a lot of background noises and poorly recorded segments.
feature strong background noise and low-quality recordings, which could explain the low performance obtained in these domains.
This analysis concludes that the speech domain is of major importance for OSD.
The presence of noise, the diversity of overlaps, and the differences in turn-taking driven by the speech domain is clearly a major issue for OSD.\\ 

\vspace{-10pt}
\subsection{Spatialisation}

In the CHiME-5 dataset, the rooms where participants are located are annotated for each utterance in the evaluation set.
We can thus study which microphone the SACC feature extractor activates as a function of the speakers' positions.
Since the VAD+OSD system is trained using one microphone per array in the CHiME data, we can visualize the combination weights for each array in each room.
Two arrays are located in the kitchen, two are located in the dining area and two are in the living room.
Figure~\ref{fig:spat} shows the SACC combination weights of each channel depending on the location of the speakers.
On these utterances, the SACC system mostly activates the channels placed in the areas where speakers are located.
The system seems able to select microphones with the most information for the VAD+OSD task.
An in-depth study should however be conducted to better assess the information used by the system in the multiple-channel scenario.

% Figure environment removed


\vspace{-0.5cm}
\section{Conclusion}
\label{sect:6_ccl}
This article presents a benchmark on two speech segmentation tasks -- Voice Activity Detection and Overlapped Speech Detection -- over multi/mono channel and various domains in 5 datasets.
Two approaches are compared by solving jointly or independently VAD and OSD.
%Experiments have shown that the joint training of VAD+OSD offers similar performance as two dedicated systems.
%This behavior has been observed on both single-channel audio data and distant multi-microphone signals.
%proposition:
The VAD+OSD joint training offers similar performance as the traditional 2-class OSD or VAD approaches on both single-channel audio data and distant multi-microphone signals.
The proposed system reaches a new state-of-the-art for OSD on DIHARD (66.8\%) and AMI (80.4\%) data.
Particularly in the case of ALLIES data with domain adaptation, joint training brings an improvement of +11.8\% for VAD and +5.3\% for OSD.
Furthermore, joint training requires fewer resources as it reduces the training time on most of the datasets, especially in the case of multi-channel data.

%autre proposition (mais ça ne rentre plus :( )
A deeper analysis demonstrates that background noise and face-to-face conversations are clearly hard to segment. We also visualize how the combination weights obtained with the SACC multi-channel feature extractor are prone to locate active speakers within a session.

%proposition by MT
Since VAD and OSD performances on multi-microphone data highly depend on the number of microphones during training, we intend to evaluate our system in a cross-domain scenario with different types of adaptation to go towards a robust multi-corpus segmentation model.
The impact of the proposed VAD+OSD system on diarization will also be evaluated.












%In further experiments, we intend to evaluate our system in a cross domain scenario with different types of adaptation to go towards a robust multi-corpus segmentation model.
%Furthermore, VAD and OSD performances on multi-microphone data highly depends on the number of microphones during training.
%Future work will focus on generalizing training to various microphone array set-up.


% This article presents a comparative study between two approaches for Voice Activity Detection (VAD) and Overlapped Speech Detection (OSD).
% The results shows that jointly training VAD and OSD offers similar performances as two dedicated systems, whatever the speech domain.
% This trend has been observed on both single channel and multiple channels audio data, with different types of feature extractors.
% Per-domain performance analysis has also shown that a joint system may perform better on some domains (e.g. phone calls) compared to others (e.g. face-to-face).
% Moreover, the use of multiple-class training requires less resources since the training time is reduced.

%In this article, we presented a benchmark on two segmentation tasks, for 5 corpora and 2 models per task. The results obtained overtake the current state of the art for both of the method in mono channel and achieve promising results in multi channel.
%We have also shown the interest of adaptation of an existing model to obtain results faster and with less data on new corpora. This conclusion has been made possible by a study of speech domain impact on overlapped speech detection.


\section{Acknowledgments}

This work was performed using HPC resources from GENCI–IDRIS (Grant 2022-AD011012565), the French ANR GEM (ANR-19-CE38-0012), and LMAC grant from Région Pays de la Loire.


\bibliographystyle{IEEEtran}
\bibliography{bib}

\end{document}
