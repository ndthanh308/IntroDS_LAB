\documentclass[10pt,superscriptaddress,twocolumn,amsmath,amssymb,aps,prl,reprint]{revtex4-1}

\usepackage{mathrsfs}
\usepackage{graphicx}
\usepackage{dcolumn}
\usepackage{bm}
\usepackage{color}
\newcommand{\lr}[1]{\langle#1\rangle}
\usepackage{multirow}
\usepackage[colorlinks,bookmarks=false,citecolor=blue,linkcolor=red,urlcolor=blue]{hyperref}
\usepackage{ulem}


\renewcommand{\Re}{\operatorname{Re}}
\renewcommand{\Im}{\operatorname{Im}}

\newcommand{\be}{\begin{equation}}
\newcommand{\ee}{\end{equation}}
\newcommand{\bea}{\begin{eqnarray}}
\newcommand{\eea}{\end{eqnarray}}

\definecolor{purple}{RGB}{128,0,128}
\newcommand{\zd}{\textcolor{red}}

\begin{document}
\title{Crossover from Non-Fermi-Liquid to Pseudogap Behavior in the Spectral of Local Impurity in Power-Law Diverging Multichannel Kondo Model}
%Crossover from Non-Fermi-Liquid to Pseudogap Phase  in a Diverging Power-Law Multichannel Kondo Model
\author{Zuodong Yu}
\affiliation{School of Information and Electronic Engineering, Zhejiang Gongshang University, Hangzhou 310018, China}
\affiliation{National Laboratory of Solid State Microstructure, Department of Physics, Nanjing University, Nanjing 210093, China}
\author{Danqing Hu}
\affiliation{Department of Physics and Chongqing Key Laboratory for Strongly Coupled Physics, Chongqing University, Chongqing 401331, China}
\author{Jiangfan Wang}
\affiliation{School of Physics, Hangzhou Normal University, Hangzhou, Zhejiang 311121, China}
\author{Xinloong Han}
\email{hanxinloong@gmail.com}
\affiliation{Kavli Institute for Theoretical Sciences, University of Chinese Academy of Sciences, Beijing 100190, China}

\date{\today}
\begin{abstract}
%Non-Fermi liquid (NFL) behaviour is found to exist in various strongly correlated materials, from high $T_c$ cuprates and iron-based superconductors to heavy fermion systems. However it remains elusive to understand its origin due to lack of controllable parameters. 

Motivated by the emergence of higher-order van Hove singularities (VHS) with power-law divergent density of states (DOS) ($\rho_c(\omega)=\rho_0/|\omega|^{r}$, $0<r<1$) in materials, we investigate a multichannel Kondo model involving conduction electrons near the higher-order van Hove filling. This model considers $M$ channel and $N$ spin degrees of freedom. Employing a renormalization group analysis and dynamical large-$N$ approach, our results reveal a crossover from a non-Fermi liquid to pseudogap behavior in the spectral properties of the local impurity at the overscreened fixed point. At this critical fixed point, we precisely determine the conditions under which the crossover occurs, either by tuning the exponent $r$ or the ratio $\kappa=M/N$ to a critical value. The results of this study provide novel insights into the non-Fermi liquid and pseudogap behaviors observed in strongly correlated systems, shedding light on the intriguing interplay between higher-order van Hove singularities and multichannel Kondo physics.

%auxiliary fermionic field behaves a non-Fermi liquid (NFL) both at the ferromagnetic coupling and overscreened fixed points. 

%Remarkably, the self-energy of spinons shows a power-law diverging behaviour when the ratio $\kappa$ ($M/N$) reaches a critical value $\kappa_c=-4 \Gamma(1-r)\Gamma(r-1)\sin^2(r\pi/2)$ at the overscreened fixed point. 




\end{abstract}
 \maketitle



\textit{Introduction.}$-$ 
Understanding exotic phenomena in strongly correlated electronic systems is a central problem in condensed matter physics. Among these phenomena, the Kondo effect is of particular interest, describing the screening of a single magnetic impurity by itinerant electrons, resulting in the formation of a many-body singlet state below a characteristic Kondo temperature $T_{K}$ in metals \cite{hewson1997kondo}.
When the magnetic impurity is screened by multiple conduction channels symmetrically, intriguing physics emerges, including non-Fermi liquid (NFL) behaviors \cite{nozieres1980kondo,ludwig1991exact} and fractionalized quasiparticles \cite{emery1992mapping,lopes2020anyons}. These exotic effects may have relevance for real heavy fermion materials \cite{cox1987quadrupolar,onimaru2016exotic}.
In certain limits, such as the large-$N$ limit, the multichannel Kondo model becomes exactly solvable, making it an ideal playground for studying strong electron correlation effects \cite{cox1993spin,parcollet1997transition,parcollet1998overscreened}. Unlike its single-channel counterpart, whose low-temperature properties can be described using Fermi liquid theory around a strong-coupling fixed point \cite{nozieres1974fermi-liquid}, the multichannel Kondo model exhibits a stable intermediate overscreening fixed point and a non-Fermi liquid ground state. Analytical studies based on conformal field theory have shed light on this intriguing behavior\cite{parcollet1998overscreened}.

% Figure environment removed

In strongly correlated systems, two intriguing and extensively studied novel states are the non-Fermi liquid (NFL) metallic state and the pseudogap phase. The former is characterized by anomalous electrical transport and thermodynamic properties, exemplified by the $T$-linear resistivity observed in cuprates \cite{hill2001breakdown,cooper2009anomalous,yuan2022scaling}, iron-based superconductors \cite{doiron2009correlation,Dai2013Hidden}, and heavy-fermion systems \cite{custers2003break,gegenwart2008quantum,si2010heavy,badoux2016change,shen2020strange}. Despite several decades of theoretical studies \cite{sachdev2010strange,varma2020colloquium,phillips2022stranger}, its microscopic origin remains elusive.
In contrast, the pseudogap phase, associated with superconducting pairing, has been extensively investigated in cuprate superconductors \cite{timusk1999pseudogap,lawler2010intra,badoux2016change,zhao2017global,varma1989phenomenology,varma1999pseudogap,kyang2006_YRZ,Rice_2012}. A notable distinction between these two states lies in the low-energy behavior of their self-energies. The NFL state typically exhibits a sublinear power-law vanishing behavior, characterized by $\Sigma_{NFL}(\omega)\propto \omega^{\alpha_1}$, with a parameter $\alpha_1\in (0,1)$. On the other hand, the pseudogap phase features a diverging self-energy, generally following $\Sigma_{PG}(\omega)\propto \omega^{\alpha_2}$, where $\alpha_2\in\left[-1,0\right)$.
For instance, the self-energy of the pseudogap phase takes the form $\Sigma(\omega,{\bf k}) \propto [\omega+\varepsilon^{\prime}_{\bf k}]^{-1}$ in the doped RVB spin liquid with a kinetic energy $\varepsilon_{\bf k}=-2t (\cos(k_x)+\cos(k_y))$ involving nearest-neighbor hopping $t$ \cite{kyang2006_YRZ,Rice_2012}, as well as in the Hatsugai–Kohmoto model with $\varepsilon^{\prime}_{\bf k}=\varepsilon_{\bf k}+U/2$, where $\varepsilon_{\bf k}$ represents the bare kinetic energy, and $U$ denotes the long-range interaction strength \cite{phillips2020exact}.
Despite the intensive research on both the NFL and pseudogap phases, a simple model capable of accommodating both phenomena and capturing their transitions is still lacking.

The electron density of states (DOS) $\rho(\omega)=\rho_0|\omega|^{-r}$ (where $|\omega|\leq D$, and $D$ is the bandwidth) of the bath plays a crucial role in determining the low-temperature behaviors of the Kondo model. In the framework of the renormalization group (RG), different forms of DOS lead to distinct fixed points along the RG trajectories. For the case of a flat DOS ($r=0$), the RG analysis reveals an unstable local moment (LM) fixed point and a stable overscreened (OS) fixed point when the channel number exceeds $2S$ ($S$ is the impurity spin size) \cite{parcollet1998overscreened}. On the other hand, for the pseudogap DOS ($-1<r<0$), the RG flow diagrams exhibit more intricate structures: a LM fixed point at weak coupling and an OS phase at intermediate coupling \cite{vojta2001multichannel}.
However, investigations of the single-channel Kondo model with $S=1/2$ and a diverging DOS indicate the presence of a stable ferromagnetic coupling fixed point and an antiferromagnetic strong-coupling fixed point with a power-law dependence of the coupling strength on the Kondo temperature $T_K$ \cite{mitchell2013quantum}.

In this study, we conduct a comprehensive investigation of the multichannel Kondo model with SU($N$) spin and SU($M$) channel symmetry, considering power-law diverging density of states of conduction electrons in the large-$N$ limit. The impurity is represented by pseudofermions, capturing its spin degrees of freedom. Notably, our calculations reveal a remarkable crossover from a non-Fermi liquid (NFL) to pseudogap behavior in the spectral properties of the local impurity, achieved by tuning the exponent $r$ or the ratio $\kappa=M/N$ at the antiferromagnetic coupling side.
At the critical exponent $r_c$ with a fixed $\kappa$, both the resistivity and dynamical local spin susceptibility exhibit power-law behaviors, with $\rho(T)\propto T^{2r_c}$ and $\chi(\omega)\propto \omega$. These vanishing power-law behaviors of resistivity and local susceptibilities deviate from the usual NFL, marginal Fermi liquid, and Landau-Fermi liquid behaviors. Our findings provide new insights into the NFL-pseudogap crossover and offer a simple model to study correlated electrons beyond the conventional Landau-Fermi liquid paradigm.

\textit{Model}.$-$\label{Model}
We start from the following large-$N$ Hamiltonian for the multichannel Kondo model 
\bea
\mathcal{H}_{MK}=\sum_{\bf{k}}c^{\dagger}_{\alpha\sigma,{\bf k}}(\epsilon_{{\bf k}}-\mu) c_{\alpha\sigma,{\bf k}}+\frac{J_K}{N}\sum_{\alpha}\mathbf{S}\cdot\mathbf{s}_{\alpha},
\eea
% Figure environment removed
with a power-law divergent DOS, $\rho_{c}(\omega)=\rho_0/|\omega|^{r}$, where $0<r<1$ is a positive number. The exponent $r=1/3$ describes the biased bilayer graphenene \cite{PhysRevB.95.035137}, while $r=1/4$ describes the kagome metals AV$_3$Sb$_5$ (A=K, Rb, Cs) due to the higher-order van Hove singularities (VHS)\cite{kang2022twofold,hu2022rich,PhysRevB.107.184504}. $c_{\alpha\sigma,{\bf k}}$ is the electron annihilation operator with the SU($N$) spin flavor $\sigma=1,\dots,N$ and the channel index $\alpha=1,\dots,M$. 
We consider the higher-order VHS filling such that $\mu=0$.
Here the local moment $\mathbf{S}$ couples to the electron bath with a coupling constant $J_K$, which can be either AFM or FM. We consider the conduction electrons as the fundamental representation of symmetry SU($N$)$\times$SU($M$), and the local moment is expressed by $S_{\sigma\sigma^{\prime}}=f^{\dagger}_{\sigma}f_{\sigma^{\prime}}-\delta_{\sigma\sigma^{\prime}}/2$ after introducing the pseudo-fermion (spinon) field $f_{\sigma}$, with the constraint $\sum_{\sigma}f^{\dagger}_{\sigma}f_{\sigma}=N/2$ which can be enforced by introducing a Lagrange multiplier $\lambda$. Then the Kondo term can be expressed as
\bea
\frac{J_K}{N}\sum_{\alpha}\mathbf{S}\cdot\mathbf{s}_{\alpha}=\frac{J_K}{N}\sum_{\sigma\sigma^{\prime}} (f^\dagger_\sigma f_{\sigma^{\prime}}-\delta_{\sigma\sigma^{\prime}}/2) c^{\dagger}_{\alpha\sigma^{\prime},{\bf k}} c_{\alpha\sigma,{\bf k}^{\prime}}.
\eea
Denoting $G_{c}(\tau,{\bf k})=-\langle \mathcal{T}_{\tau} c_{\alpha\sigma}(\tau)c^{\dagger}_{\alpha\sigma}(0)\rangle$ and $G_f(\tau)=-\langle \mathcal{T}_{\tau} f_{\alpha}(\tau)f^{\dagger}_{\alpha}(0)\rangle$ as the Greens' functions of conduction electrons and spinons, respectively. We define the ratio $\kappa=M/N<1$ between the number of channel and spin degrees of freedom. In the following, we perform RG analysis for the AFM Kondo coupling at large-N limit with fixing $\kappa$, and then present the dynamical and transport properties near the crossover region based on the dynamical large-$N$ approach.

\textit{RG analysis by small $r$ expansion with AFM coupling.}$-$\label{RG} We perform a renormalization group analysis based on the dimensional regularization with a minimal subtraction of poles. The renormalization field $f_{\sigma,\Lambda}$ and dimensionless coupling constant $J_{K,\Lambda}$ at the energy cutoff $\Lambda$ running from the bandwidth $D$ to 0, can be defined as $f_{\sigma}=Z_f^{1/2} f_{\sigma,\Lambda}$ and $J_K=Z_J Z_f^{-1} J_{K,\Lambda} \Lambda^{r}$ where $Z_f$ and $Z_J$ is the renormalization factor for $f_{\sigma,\Lambda}$ and $J_{K,\Lambda}$.
 In a large-$N$ limit, only the two diagrams shown in Fig. \ref{fig:Fig2}
 contribute to $Z_f$ and $Z_J$, 
 % Figure environment removed
\begin{equation}
\begin{aligned}
Z_f=1-\frac{\kappa\rho_0^2 J_{K,\Lambda}^2}{2 r}, \quad Z_J=1-\frac{\rho_0 J_{K,\Lambda}}{r}.
\end{aligned}
\end{equation}
Therefore, the RG equation for the coupling constant $J_K$ can be derived as
\begin{equation}
\begin{aligned}
\frac{d \bar{J}_K}{d \ln(D/\Lambda)}=r \bar{J}_K+\bar{J}^2_K-\kappa\bar{J}^3_K,
\end{aligned}
\end{equation} 
where the dimensionless coupling constant $\bar{J}_K=J_K\rho_c(\Lambda)$. As usual, there contains a stable fixed point at $\bar{J}_{OS}=(1+\sqrt{1+4\kappa r})/2\kappa$, which marks the overscreened Kondo phase, and an unstable trivial fixed point $\bar{J}_{LM}=0$ which denotes a local moment phase. Despite this two fixed points, the above RG equation also demonstrates a stable ferromagnetic coupling fixed point with $\bar{J}_{FM}=(1-\sqrt{1+4\kappa r})/2\kappa$. In this work, we focus the overscreened fixed point $J_{OS}$ which controls the low-energy physics in the antiferromagnetic coupling side.


\textit{Large-$N$ limit with AFM coupling.}$-$To analyze the Kondo overscreened phase with AFM coupling analytically and numerically in the large-$N$ limit, we derive saddle point equations by introducing charged bosonic fields in each channel $B_{\alpha}(\tau)$ which conjugates to $\sum_{\sigma,{\bf k}} f^{\dagger}_{\sigma}(\tau) c_{\alpha \sigma,{\bf k}}(\tau)/\sqrt{N}$ to decouple the Kondo interaction. The Greens' function $G_B(\tau)$ for the bosonic fields $B_{\alpha}$ are defined as $G_B(\tau)=\sum_{\alpha}\langle \mathcal{T}_{\tau} B_{\alpha}(0) B^{\dagger}_{\alpha}(\tau)\rangle/M$.  In the large-$N$ limit, the saddle point equations for the self-energy of $f_{\sigma}$ and $B_{\alpha}$ fields at the zero temperature limit are
\bea\label{Eq:Selfenergies}
&&\Sigma_f(\omega)=\kappa\int \frac{d\omega^{\prime}}{2\pi}G_0(\omega^{\prime}) G_B(\omega-\omega^{\prime}), \\
&&\Pi(\Omega)=-\int\frac{d\omega^{\prime}}{2\pi}G_0(\omega^{\prime}) G_f(\omega^{\prime}-\Omega),
\eea
and they satisfy the Schwinger-Dyson equations $G_f^{-1}(\omega)=\omega-\lambda-\Sigma_f(\omega)$ and $G^{-1}_B(\Omega)=1/J_K-\Pi(\Omega)$. The local Green's function for conduction electrons is $G_0(\omega)=C\rho_0|\omega|^{-r}sgn(\omega)$ with constant $C$. At the low energy and zero temperature limit, we assume the following scaling forms,
 % Figure environment removed
\bea
G_f(\omega)=iA_1 |\omega|^{\Delta_1-1} sgn(\omega),G_B(\omega)=A_2|\omega|^{\Delta_2-1},
\eea
where $A_{1/2}>0$ is some constant. The exponent $\Delta_{1/2}>0$ can be obtained once we put the scaling ansatz back into Eq. (\ref{Eq:Selfenergies}). After some algebra, one can get
\bea\label{Eq:ScalingEq}
&&\Delta_1+\Delta_2=1+r, \label{eq:exponents1}\\
&& \kappa \Gamma(\Delta_1-1) \Gamma(1-\Delta_1)\cos^2(\frac{\Delta_1\pi}{2})= \nonumber \\
&&\Gamma(r-\Delta_1)\Gamma(\Delta_1-r)\cos^2(\frac{r-\Delta_1}{2}\pi),\label{eq:exponents2}
\eea
which the scaling exponents $\Delta_{1/2}$ must satisfy. In the $r=0$ limit, there exists only one solution $\Delta_1=1/(1+\kappa)$ and $\Delta_2=\kappa/(1+\kappa)$
to guarantee the causality which requires $A_{1,2}>0$, coinciding with the fact that the oversreened phase is the only fixed point for the flat DOS conduction bath. For fixed $\kappa=1/2$, let us now tune $r$ slightly away from exact zero, say $r=0.1$, the exponent $\Delta_1$ extracted from Eqs (\ref{eq:exponents1}) and (\ref{eq:exponents2}) will approximately be $0.767>2/3$. At a critical diverging DOS with $r_c\approx 0.314$, the crossover from NFL to pseudogap phase emerges. Our analytic results reveal the strongly divergent DOS of conduction electrons promote a much stronger correction to the self-energy $\Sigma_f(\omega)$ of spinons, eventually leading $\Sigma_f(\omega)$ to be singular in the pseudogap region.

Now we turn to discuss the observable in the multichannel Kondo model with power-law diverging DOS. As an important observable, the resistivity $\rho(T)$ shows a major distinction with the flat and pseudogap multichannel Kondo model. In the flat DOS case ($r=0$), $\rho(T)$ approaches to a constant at the low temperature limit as $\rho(T)/\rho(0)=1-\gamma T^{\kappa/(\kappa+1)}$ for $\kappa<1$ where $\gamma$ is a constant. In the pseudogap case ($r<0$), at the scaling invariant oversreened fixed point, $\rho(T)$ diverges as a power-law behavior $\rho(T)\propto 1/T^{-2r}$ where $-2r$ is contributed from the pseudogap DOS and the scattering matrix $\mathcal{T}(\omega,T)$ which is a convolution of $G_f$ and $G_B$.% Figure environment removedIn the diverging DOS case ($r>0$), the resistivity vanishes at the zero temperature with the form $\rho(T)\propto T^{2r}$, which strongly deviates from the Fermi liquid for $r\ll 1$. This nontrivial power-law behavior of $\rho(T)$ in the DPLMCK has two-fold implications. First, it implies the Kondo screening effect because the resistivity is strongly modified compared to $\rho(T)\propto T^{1-r}$ in the absence of local impurity\cite{isobe2019supermetal}. Second, the vanishing behavior at $T=0$ of $\rho(T)$ is universal, it only depends on the exponent $r$.

The dynamical susceptibility of the local impurity $\chi(\tau)=-G_f(\tau) G_f(-\tau)$ shows a anomalous scaling behavior. In the low-energy and zero temperature limit, $\chi(\omega)\propto \omega^{1+\eta}$, where $\eta$ is a positive anomalous scaling exponent and can be given by $\eta=2\Delta_1-2$, in the pseudogap phase when $r>r_c$. Exactly at $r_c$, $\eta$ vanishes and the dynamic susceptibility is linear-in-$\omega$ at low energy limit as $\chi(\omega)\propto \omega$. This kind of anomalous scaling behavior for dynamic or static susceptibility can be a hallmark to diagnose the occurrence of the crossover from NFL to the pseudogap phase in experiments. This is one of the key points of this work.

% % Figure environment removed


\textit{Numerical results.}$-$ To confirm above analysis and to elucidate the crossover beyond the low energy and temperature limit, we solve the dynamical large-$N$ equations numerically. In Fig. \ref{fig:Fig3}(a), we illustrate the spectral function $A_f(\omega,T)$ of spinons at different $r$ at relative low temperature. As we can see, the scaling behavior is developed for all values of $r$ considered here,  consistent with our scaling analysis. Moreover, the pseudogap phase with positive scaling exponent begins to emerge at a critical $r_{c,N}$, whose value  agrees well with the analytical $r_c$. This gives rise to a peak at finite energy $\omega_0$ as marked in Fig. \ref{fig:Fig3}(a). Of interestingly, we find that $\omega_0$ follows a power-law relation $\omega_0\propto (\rho_0J_K/r)^{1/r}$ for small value of $\rho_0J_K/r$, which is similar with the Kondo temperature $T_K$ for diverging power-law DOS case\cite{mitchell2013quantum}. The astonishing result implies  $T_K$ is the only relevant energy scale in our model. The exponent in the scaling region of $\Im G_{f,B}(\omega)$ in Fig. \ref{fig:Fig3}(a,b) by numerical fitting as illustrated also agree well with Eqs. \ref{eq:exponents1} and \ref{eq:exponents2}.

In Fig. \ref{fig:Fig4}, we move to the dynamical physical quantities T-matrix and local spin susceptibility. These quantities can be calculated by the convolution of the single particle Green's functions. The scaling form of $G_f$ and $G_B$ leads to the power-law behavior in T-matrix and susceptibility with the exponent controlled by $r$. The results confirm the statement from analytical analysis that the local spin susceptibility at the critical value of $r$ has vanishing anomalous scaling exponent. With T-matrix, one can calculate the resistivity from $\rho^{-1}\sim -\int d\omega \frac{\partial f(\omega)}{\partial \omega}\frac{ \rho_{c}(\omega)}{\mathrm{Im}[\mathcal{T}(\omega)]}$. For $r\leq 0$ case, the T-matrix is either constant or singular for low frequency at zero temperature, which lead to NFL behavior in resistivity as show in Fig.~\ref{fig:Fig5}(a). While the resistivity behavior differently when $r>0$. The exponent equals to $2r$ as expected with vanishing residue resistivity at zero temperature. % which implies an continues evolution from NFL behavior $T^{\ll 2}$ to the Fermi liquid behavior $T^{\sim 2}$.
In Fig.~\ref{fig:Fig5}(b) we show temperature dependency of local static spin response. At high temperature, as the magnetic impurity is weakly coupled to the conducting electron, the spin response follows the well known Curie's law $\sim T^{-1}$. When the temperature decreases, the susceptibility starts to saturate, give rise to $T\chi_{loc}=0$ at $T=0$. This indicate the local spin is completely quenched due to strongly coupled to electron. As $\Delta_1=1>0.5$ at critical $r$, we do not expect powerlaw behavior at low temperature\cite{vojta2001multichannel}.  % This Pauli paramagnetism in low-temperature together with the resistivity result shows a crossover from NFL to nearly FL by controling the power-law exponent of the conduction electron DOS.
 
 
\textit{Stability of pseudogap phase at a finite $N$.}$-$ A natural question arises in the context of the large-$N$ limit is whether the system remains stable after considering the fluctuations in the finite $N$ case. Following Ref. \cite{cox1993spin}, we show that the saddle-point exponents stay the same in a $1/N$ level in our DPLMCK model. It is seen from the structure of functional integral formulation that a generic diagram with $L$ loops to $\Im G_f^{-1}$ must contain $L$ propagators of $B$-field, $L$ propagators of conduction electrons, and $L-1$ propagators of spinons. The most singular part thus behaves as $\delta\Im G_f^{-1}(\omega)\propto |\omega|^{\zeta_f(L)}/N$ where $\zeta_f(L)=L-r L+(\Delta_2-1)L+(L-1)(\Delta_1-1)$. At the saddle point, $\zeta_f(L)$ fulfills $\zeta_f(L)=1-\Delta_1$. We also have $\delta \Im G_B^{-1}\propto |\omega|^{\zeta_B(L)}/N$ where $\zeta_B=1-\Delta_2$. This argument is valid for both NFL and pseudogap phase. Hence the fluctuations are irrelevant to alter behaviors governed by saddle points, thus the pseudogap phase is robust against $1/N$ fluctuations. This is also the reason why our NRG results are consistent with dynamical large-$N$ calculations in the AFM coupling side.

% % Figure environment removed

\textit{Conclusions and remarks.}$-$\label{conclusion}
In conclusion, we have investigated the power-law multichannel Kondo model both analytically and numerically in the large-$N$ limit. At the overscreened fixed point, there emerges scaling behaviors for the spinons and $B$ fields, and their scaling exponents satisfy scaling ansatz. Our analysis shows there is a crossover from the non-Fermi liquid to pseudogap behavior in the spectral of spinons. We uncover that the dynamic spin susceptibility and resistivity have diverse behaviour compared to the case in multichannel Kondo model with flat or pseudogap DOS.
 % The supermetal phase can appear in certain parameter regime and an $s$-wave Pomeranchuk order in degenerate with $p\pm ip$ pairing states can be stabilized for a strong initial attractive $\gamma_2$ for the case of $\kappa=\frac{1}{3}$. 



\textit{Acknowledgments.}$-$
X.L.H acknowledges the supports from China Postdoctoral Science Foundation Fellowship (No. 2022M723112). Z.D.Y acknowledges the supports from National Natural Science Foundation of China Grants No. 12204411 and No.12075205


\bibliographystyle{apsrev4-1}
%\bibliography{reference}

\begin{thebibliography}{39}%
\makeatletter
\providecommand \@ifxundefined [1]{%
 \@ifx{#1\undefined}
}%
\providecommand \@ifnum [1]{%
 \ifnum #1\expandafter \@firstoftwo
 \else \expandafter \@secondoftwo
 \fi
}%
\providecommand \@ifx [1]{%
 \ifx #1\expandafter \@firstoftwo
 \else \expandafter \@secondoftwo
 \fi
}%
\providecommand \natexlab [1]{#1}%
\providecommand \enquote  [1]{``#1''}%
\providecommand \bibnamefont  [1]{#1}%
\providecommand \bibfnamefont [1]{#1}%
\providecommand \citenamefont [1]{#1}%
\providecommand \href@noop [0]{\@secondoftwo}%
\providecommand \href [0]{\begingroup \@sanitize@url \@href}%
\providecommand \@href[1]{\@@startlink{#1}\@@href}%
\providecommand \@@href[1]{\endgroup#1\@@endlink}%
\providecommand \@sanitize@url [0]{\catcode `\\12\catcode `\$12\catcode
  `\&12\catcode `\#12\catcode `\^12\catcode `\_12\catcode `\%12\relax}%
\providecommand \@@startlink[1]{}%
\providecommand \@@endlink[0]{}%
\providecommand \url  [0]{\begingroup\@sanitize@url \@url }%
\providecommand \@url [1]{\endgroup\@href {#1}{\urlprefix }}%
\providecommand \urlprefix  [0]{URL }%
\providecommand \Eprint [0]{\href }%
\providecommand \doibase [0]{http://dx.doi.org/}%
\providecommand \selectlanguage [0]{\@gobble}%
\providecommand \bibinfo  [0]{\@secondoftwo}%
\providecommand \bibfield  [0]{\@secondoftwo}%
\providecommand \translation [1]{[#1]}%
\providecommand \BibitemOpen [0]{}%
\providecommand \bibitemStop [0]{}%
\providecommand \bibitemNoStop [0]{.\EOS\space}%
\providecommand \EOS [0]{\spacefactor3000\relax}%
\providecommand \BibitemShut  [1]{\csname bibitem#1\endcsname}%
\let\auto@bib@innerbib\@empty
%</preamble>
\bibitem [{\citenamefont {Hewson}(1997)}]{hewson1997kondo}%
  \BibitemOpen
  \bibfield  {author} {\bibinfo {author} {\bibfnamefont {A.~C.}\ \bibnamefont
  {Hewson}},\ }\href@noop {} {\emph {\bibinfo {title} {The Kondo problem to
  heavy fermions}}},\ \bibinfo {number} {2}\ (\bibinfo  {publisher} {Cambridge
  university press},\ \bibinfo {year} {1997})\BibitemShut {NoStop}%
\bibitem [{\citenamefont {Nozieres}\ and\ \citenamefont
  {Blandin}(1980)}]{nozieres1980kondo}%
  \BibitemOpen
  \bibfield  {author} {\bibinfo {author} {\bibfnamefont {P.}~\bibnamefont
  {Nozieres}}\ and\ \bibinfo {author} {\bibfnamefont {A.}~\bibnamefont
  {Blandin}},\ }\href {https://doi.org/10.1051/jphys:01980004103019300}
  {\bibfield  {journal} {\bibinfo  {journal} {Journal de Physique}\ }\textbf
  {\bibinfo {volume} {41}},\ \bibinfo {pages} {193} (\bibinfo {year}
  {1980})}\BibitemShut {NoStop}%
\bibitem [{\citenamefont {Ludwig}\ and\ \citenamefont
  {Affleck}(1991)}]{ludwig1991exact}%
  \BibitemOpen
  \bibfield  {author} {\bibinfo {author} {\bibfnamefont {A.~W.}\ \bibnamefont
  {Ludwig}}\ and\ \bibinfo {author} {\bibfnamefont {I.}~\bibnamefont
  {Affleck}},\ }\href {https://link.aps.org/doi/10.1103/PhysRevLett.67.3160}
  {\bibfield  {journal} {\bibinfo  {journal} {Physical review letters}\
  }\textbf {\bibinfo {volume} {67}},\ \bibinfo {pages} {3160} (\bibinfo {year}
  {1991})}\BibitemShut {NoStop}%
\bibitem [{\citenamefont {Emery}\ and\ \citenamefont
  {Kivelson}(1992)}]{emery1992mapping}%
  \BibitemOpen
  \bibfield  {author} {\bibinfo {author} {\bibfnamefont {V.}~\bibnamefont
  {Emery}}\ and\ \bibinfo {author} {\bibfnamefont {S.}~\bibnamefont
  {Kivelson}},\ }\href@noop {} {\bibfield  {journal} {\bibinfo  {journal}
  {Physical Review B}\ }\textbf {\bibinfo {volume} {46}},\ \bibinfo {pages}
  {10812} (\bibinfo {year} {1992})}\BibitemShut {NoStop}%
\bibitem [{\citenamefont {Lopes}\ \emph {et~al.}(2020)\citenamefont {Lopes},
  \citenamefont {Affleck},\ and\ \citenamefont {Sela}}]{lopes2020anyons}%
  \BibitemOpen
  \bibfield  {author} {\bibinfo {author} {\bibfnamefont {P.~L.}\ \bibnamefont
  {Lopes}}, \bibinfo {author} {\bibfnamefont {I.}~\bibnamefont {Affleck}}, \
  and\ \bibinfo {author} {\bibfnamefont {E.}~\bibnamefont {Sela}},\ }\href@noop
  {} {\bibfield  {journal} {\bibinfo  {journal} {Physical Review B}\ }\textbf
  {\bibinfo {volume} {101}},\ \bibinfo {pages} {085141} (\bibinfo {year}
  {2020})}\BibitemShut {NoStop}%
\bibitem [{\citenamefont {Cox}(1987)}]{cox1987quadrupolar}%
  \BibitemOpen
  \bibfield  {author} {\bibinfo {author} {\bibfnamefont {D.}~\bibnamefont
  {Cox}},\ }\href {https://link.aps.org/doi/10.1103/PhysRevLett.59.1240}
  {\bibfield  {journal} {\bibinfo  {journal} {Physical review letters}\
  }\textbf {\bibinfo {volume} {59}},\ \bibinfo {pages} {1240} (\bibinfo {year}
  {1987})}\BibitemShut {NoStop}%
\bibitem [{\citenamefont {Onimaru}\ and\ \citenamefont
  {Kusunose}(2016)}]{onimaru2016exotic}%
  \BibitemOpen
  \bibfield  {author} {\bibinfo {author} {\bibfnamefont {T.}~\bibnamefont
  {Onimaru}}\ and\ \bibinfo {author} {\bibfnamefont {H.}~\bibnamefont
  {Kusunose}},\ }\href@noop {} {\bibfield  {journal} {\bibinfo  {journal}
  {Journal of the Physical Society of Japan}\ }\textbf {\bibinfo {volume}
  {85}},\ \bibinfo {pages} {082002} (\bibinfo {year} {2016})}\BibitemShut
  {NoStop}%
\bibitem [{\citenamefont {Cox}\ and\ \citenamefont
  {Ruckenstein}(1993)}]{cox1993spin}%
  \BibitemOpen
  \bibfield  {author} {\bibinfo {author} {\bibfnamefont {D.~L.}\ \bibnamefont
  {Cox}}\ and\ \bibinfo {author} {\bibfnamefont {A.~E.}\ \bibnamefont
  {Ruckenstein}},\ }\href
  {https://link.aps.org/doi/10.1103/PhysRevLett.71.1613} {\bibfield  {journal}
  {\bibinfo  {journal} {Physical review letters}\ }\textbf {\bibinfo {volume}
  {71}},\ \bibinfo {pages} {1613} (\bibinfo {year} {1993})}\BibitemShut
  {NoStop}%
\bibitem [{\citenamefont {Parcollet}\ and\ \citenamefont
  {Georges}(1997)}]{parcollet1997transition}%
  \BibitemOpen
  \bibfield  {author} {\bibinfo {author} {\bibfnamefont {O.}~\bibnamefont
  {Parcollet}}\ and\ \bibinfo {author} {\bibfnamefont {A.}~\bibnamefont
  {Georges}},\ }\href {https://link.aps.org/doi/10.1103/PhysRevLett.79.4665}
  {\bibfield  {journal} {\bibinfo  {journal} {Physical review letters}\
  }\textbf {\bibinfo {volume} {79}},\ \bibinfo {pages} {4665} (\bibinfo {year}
  {1997})}\BibitemShut {NoStop}%
\bibitem [{\citenamefont {Parcollet}\ \emph {et~al.}(1998)\citenamefont
  {Parcollet}, \citenamefont {Georges}, \citenamefont {Kotliar},\ and\
  \citenamefont {Sengupta}}]{parcollet1998overscreened}%
  \BibitemOpen
  \bibfield  {author} {\bibinfo {author} {\bibfnamefont {O.}~\bibnamefont
  {Parcollet}}, \bibinfo {author} {\bibfnamefont {A.}~\bibnamefont {Georges}},
  \bibinfo {author} {\bibfnamefont {G.}~\bibnamefont {Kotliar}}, \ and\
  \bibinfo {author} {\bibfnamefont {A.}~\bibnamefont {Sengupta}},\ }\href
  {https://link.aps.org/doi/10.1103/PhysRevB.58.3794} {\bibfield  {journal}
  {\bibinfo  {journal} {Physical Review B}\ }\textbf {\bibinfo {volume} {58}},\
  \bibinfo {pages} {3794} (\bibinfo {year} {1998})}\BibitemShut {NoStop}%
\bibitem [{\citenamefont {Nozières}(1974)}]{nozieres1974fermi-liquid}%
  \BibitemOpen
  \bibfield  {author} {\bibinfo {author} {\bibfnamefont {P.}~\bibnamefont
  {Nozières}},\ }\href {\doibase 10.1007/BF00654541} {\bibfield  {journal}
  {\bibinfo  {journal} {Journal of Low Temperature Physics}\ }\textbf {\bibinfo
  {volume} {17}},\ \bibinfo {pages} {31} (\bibinfo {year} {1974})}\BibitemShut
  {NoStop}%
\bibitem [{\citenamefont {Hill}\ \emph {et~al.}(2001)\citenamefont {Hill},
  \citenamefont {Proust}, \citenamefont {Taillefer}, \citenamefont {Fournier},\
  and\ \citenamefont {Greene}}]{hill2001breakdown}%
  \BibitemOpen
  \bibfield  {author} {\bibinfo {author} {\bibfnamefont {R.}~\bibnamefont
  {Hill}}, \bibinfo {author} {\bibfnamefont {C.}~\bibnamefont {Proust}},
  \bibinfo {author} {\bibfnamefont {L.}~\bibnamefont {Taillefer}}, \bibinfo
  {author} {\bibfnamefont {P.}~\bibnamefont {Fournier}}, \ and\ \bibinfo
  {author} {\bibfnamefont {R.}~\bibnamefont {Greene}},\ }\href
  {https://doi.org/10.1038/414711a} {\bibfield  {journal} {\bibinfo  {journal}
  {Nature}\ }\textbf {\bibinfo {volume} {414}},\ \bibinfo {pages} {711}
  (\bibinfo {year} {2001})}\BibitemShut {NoStop}%
\bibitem [{\citenamefont {Cooper}\ \emph {et~al.}(2009)\citenamefont {Cooper},
  \citenamefont {Wang}, \citenamefont {Vignolle}, \citenamefont {Lipscombe},
  \citenamefont {Hayden}, \citenamefont {Tanabe}, \citenamefont {Adachi},
  \citenamefont {Koike}, \citenamefont {Nohara}, \citenamefont {Takagi} \emph
  {et~al.}}]{cooper2009anomalous}%
  \BibitemOpen
  \bibfield  {author} {\bibinfo {author} {\bibfnamefont {R.}~\bibnamefont
  {Cooper}}, \bibinfo {author} {\bibfnamefont {Y.}~\bibnamefont {Wang}},
  \bibinfo {author} {\bibfnamefont {B.}~\bibnamefont {Vignolle}}, \bibinfo
  {author} {\bibfnamefont {O.}~\bibnamefont {Lipscombe}}, \bibinfo {author}
  {\bibfnamefont {S.}~\bibnamefont {Hayden}}, \bibinfo {author} {\bibfnamefont
  {Y.}~\bibnamefont {Tanabe}}, \bibinfo {author} {\bibfnamefont
  {T.}~\bibnamefont {Adachi}}, \bibinfo {author} {\bibfnamefont
  {Y.}~\bibnamefont {Koike}}, \bibinfo {author} {\bibfnamefont
  {M.}~\bibnamefont {Nohara}}, \bibinfo {author} {\bibfnamefont
  {H.}~\bibnamefont {Takagi}},  \emph {et~al.},\ }\href
  {https://www.science.org/doi/10.1126/science.1165015} {\bibfield  {journal}
  {\bibinfo  {journal} {Science}\ }\textbf {\bibinfo {volume} {323}},\ \bibinfo
  {pages} {603} (\bibinfo {year} {2009})}\BibitemShut {NoStop}%
\bibitem [{\citenamefont {Yuan}\ \emph {et~al.}(2022)\citenamefont {Yuan},
  \citenamefont {Chen}, \citenamefont {Jiang}, \citenamefont {Feng},
  \citenamefont {Lin}, \citenamefont {Yu}, \citenamefont {He}, \citenamefont
  {Zhang}, \citenamefont {Jiang}, \citenamefont {Zhang} \emph
  {et~al.}}]{yuan2022scaling}%
  \BibitemOpen
  \bibfield  {author} {\bibinfo {author} {\bibfnamefont {J.}~\bibnamefont
  {Yuan}}, \bibinfo {author} {\bibfnamefont {Q.}~\bibnamefont {Chen}}, \bibinfo
  {author} {\bibfnamefont {K.}~\bibnamefont {Jiang}}, \bibinfo {author}
  {\bibfnamefont {Z.}~\bibnamefont {Feng}}, \bibinfo {author} {\bibfnamefont
  {Z.}~\bibnamefont {Lin}}, \bibinfo {author} {\bibfnamefont {H.}~\bibnamefont
  {Yu}}, \bibinfo {author} {\bibfnamefont {G.}~\bibnamefont {He}}, \bibinfo
  {author} {\bibfnamefont {J.}~\bibnamefont {Zhang}}, \bibinfo {author}
  {\bibfnamefont {X.}~\bibnamefont {Jiang}}, \bibinfo {author} {\bibfnamefont
  {X.}~\bibnamefont {Zhang}},  \emph {et~al.},\ }\href
  {https://doi.org/10.1038/s41586-021-04305-5} {\bibfield  {journal} {\bibinfo
  {journal} {Nature}\ }\textbf {\bibinfo {volume} {602}},\ \bibinfo {pages}
  {431} (\bibinfo {year} {2022})}\BibitemShut {NoStop}%
\bibitem [{\citenamefont {Doiron-Leyraud}\ \emph {et~al.}(2009)\citenamefont
  {Doiron-Leyraud}, \citenamefont {Auban-Senzier}, \citenamefont {de~Cotret},
  \citenamefont {Bourbonnais}, \citenamefont {J{\'e}rome}, \citenamefont
  {Bechgaard},\ and\ \citenamefont {Taillefer}}]{doiron2009correlation}%
  \BibitemOpen
  \bibfield  {author} {\bibinfo {author} {\bibfnamefont {N.}~\bibnamefont
  {Doiron-Leyraud}}, \bibinfo {author} {\bibfnamefont {P.}~\bibnamefont
  {Auban-Senzier}}, \bibinfo {author} {\bibfnamefont {S.~R.}\ \bibnamefont
  {de~Cotret}}, \bibinfo {author} {\bibfnamefont {C.}~\bibnamefont
  {Bourbonnais}}, \bibinfo {author} {\bibfnamefont {D.}~\bibnamefont
  {J{\'e}rome}}, \bibinfo {author} {\bibfnamefont {K.}~\bibnamefont
  {Bechgaard}}, \ and\ \bibinfo {author} {\bibfnamefont {L.}~\bibnamefont
  {Taillefer}},\ }\href@noop {} {\bibfield  {journal} {\bibinfo  {journal}
  {Physical Review B}\ }\textbf {\bibinfo {volume} {80}},\ \bibinfo {pages}
  {214531} (\bibinfo {year} {2009})}\BibitemShut {NoStop}%
\bibitem [{\citenamefont {Dai}\ \emph {et~al.}(2013)\citenamefont {Dai},
  \citenamefont {Xu}, \citenamefont {Shen}, \citenamefont {Xiao}, \citenamefont
  {Wen}, \citenamefont {Qiu}, \citenamefont {Homes},\ and\ \citenamefont
  {Lobo}}]{Dai2013Hidden}%
  \BibitemOpen
  \bibfield  {author} {\bibinfo {author} {\bibfnamefont {Y.~M.}\ \bibnamefont
  {Dai}}, \bibinfo {author} {\bibfnamefont {B.}~\bibnamefont {Xu}}, \bibinfo
  {author} {\bibfnamefont {B.}~\bibnamefont {Shen}}, \bibinfo {author}
  {\bibfnamefont {H.}~\bibnamefont {Xiao}}, \bibinfo {author} {\bibfnamefont
  {H.~H.}\ \bibnamefont {Wen}}, \bibinfo {author} {\bibfnamefont {X.~G.}\
  \bibnamefont {Qiu}}, \bibinfo {author} {\bibfnamefont {C.~C.}\ \bibnamefont
  {Homes}}, \ and\ \bibinfo {author} {\bibfnamefont {R.~P. S.~M.}\ \bibnamefont
  {Lobo}},\ }\href {\doibase 10.1103/PhysRevLett.111.117001} {\bibfield
  {journal} {\bibinfo  {journal} {Phys. Rev. Lett.}\ }\textbf {\bibinfo
  {volume} {111}},\ \bibinfo {pages} {117001} (\bibinfo {year}
  {2013})}\BibitemShut {NoStop}%
\bibitem [{\citenamefont {Custers}\ \emph {et~al.}(2003)\citenamefont
  {Custers}, \citenamefont {Gegenwart}, \citenamefont {Wilhelm}, \citenamefont
  {Neumaier}, \citenamefont {Tokiwa}, \citenamefont {Trovarelli}, \citenamefont
  {Geibel}, \citenamefont {Steglich}, \citenamefont {P{\'e}pin},\ and\
  \citenamefont {Coleman}}]{custers2003break}%
  \BibitemOpen
  \bibfield  {author} {\bibinfo {author} {\bibfnamefont {J.}~\bibnamefont
  {Custers}}, \bibinfo {author} {\bibfnamefont {P.}~\bibnamefont {Gegenwart}},
  \bibinfo {author} {\bibfnamefont {H.}~\bibnamefont {Wilhelm}}, \bibinfo
  {author} {\bibfnamefont {K.}~\bibnamefont {Neumaier}}, \bibinfo {author}
  {\bibfnamefont {Y.}~\bibnamefont {Tokiwa}}, \bibinfo {author} {\bibfnamefont
  {O.}~\bibnamefont {Trovarelli}}, \bibinfo {author} {\bibfnamefont
  {C.}~\bibnamefont {Geibel}}, \bibinfo {author} {\bibfnamefont
  {F.}~\bibnamefont {Steglich}}, \bibinfo {author} {\bibfnamefont
  {C.}~\bibnamefont {P{\'e}pin}}, \ and\ \bibinfo {author} {\bibfnamefont
  {P.}~\bibnamefont {Coleman}},\ }\href {https://doi.org/10.1038/nature01774}
  {\bibfield  {journal} {\bibinfo  {journal} {Nature}\ }\textbf {\bibinfo
  {volume} {424}},\ \bibinfo {pages} {524} (\bibinfo {year}
  {2003})}\BibitemShut {NoStop}%
\bibitem [{\citenamefont {Gegenwart}\ \emph {et~al.}(2008)\citenamefont
  {Gegenwart}, \citenamefont {Si},\ and\ \citenamefont
  {Steglich}}]{gegenwart2008quantum}%
  \BibitemOpen
  \bibfield  {author} {\bibinfo {author} {\bibfnamefont {P.}~\bibnamefont
  {Gegenwart}}, \bibinfo {author} {\bibfnamefont {Q.}~\bibnamefont {Si}}, \
  and\ \bibinfo {author} {\bibfnamefont {F.}~\bibnamefont {Steglich}},\ }\href
  {https://doi.org/10.1038/nphys892} {\bibfield  {journal} {\bibinfo  {journal}
  {nature physics}\ }\textbf {\bibinfo {volume} {4}},\ \bibinfo {pages} {186}
  (\bibinfo {year} {2008})}\BibitemShut {NoStop}%
\bibitem [{\citenamefont {Si}\ and\ \citenamefont
  {Steglich}(2010)}]{si2010heavy}%
  \BibitemOpen
  \bibfield  {author} {\bibinfo {author} {\bibfnamefont {Q.}~\bibnamefont
  {Si}}\ and\ \bibinfo {author} {\bibfnamefont {F.}~\bibnamefont {Steglich}},\
  }\href {https://www.science.org/doi/10.1126/science.1191195} {\bibfield
  {journal} {\bibinfo  {journal} {Science}\ }\textbf {\bibinfo {volume}
  {329}},\ \bibinfo {pages} {1161} (\bibinfo {year} {2010})}\BibitemShut
  {NoStop}%
\bibitem [{\citenamefont {Badoux}\ \emph {et~al.}(2016)\citenamefont {Badoux},
  \citenamefont {Tabis}, \citenamefont {Lalibert{\'e}}, \citenamefont
  {Grissonnanche}, \citenamefont {Vignolle}, \citenamefont {Vignolles},
  \citenamefont {B{\'e}ard}, \citenamefont {Bonn}, \citenamefont {Hardy},
  \citenamefont {Liang} \emph {et~al.}}]{badoux2016change}%
  \BibitemOpen
  \bibfield  {author} {\bibinfo {author} {\bibfnamefont {S.}~\bibnamefont
  {Badoux}}, \bibinfo {author} {\bibfnamefont {W.}~\bibnamefont {Tabis}},
  \bibinfo {author} {\bibfnamefont {F.}~\bibnamefont {Lalibert{\'e}}}, \bibinfo
  {author} {\bibfnamefont {G.}~\bibnamefont {Grissonnanche}}, \bibinfo {author}
  {\bibfnamefont {B.}~\bibnamefont {Vignolle}}, \bibinfo {author}
  {\bibfnamefont {D.}~\bibnamefont {Vignolles}}, \bibinfo {author}
  {\bibfnamefont {J.}~\bibnamefont {B{\'e}ard}}, \bibinfo {author}
  {\bibfnamefont {D.}~\bibnamefont {Bonn}}, \bibinfo {author} {\bibfnamefont
  {W.}~\bibnamefont {Hardy}}, \bibinfo {author} {\bibfnamefont
  {R.}~\bibnamefont {Liang}},  \emph {et~al.},\ }\href
  {https://doi.org/10.1038/nature16983} {\bibfield  {journal} {\bibinfo
  {journal} {Nature}\ }\textbf {\bibinfo {volume} {531}},\ \bibinfo {pages}
  {210} (\bibinfo {year} {2016})}\BibitemShut {NoStop}%
\bibitem [{\citenamefont {Shen}\ \emph {et~al.}(2020)\citenamefont {Shen},
  \citenamefont {Zhang}, \citenamefont {Komijani}, \citenamefont {Nicklas},
  \citenamefont {Borth}, \citenamefont {Wang}, \citenamefont {Chen},
  \citenamefont {Nie}, \citenamefont {Li}, \citenamefont {Lu} \emph
  {et~al.}}]{shen2020strange}%
  \BibitemOpen
  \bibfield  {author} {\bibinfo {author} {\bibfnamefont {B.}~\bibnamefont
  {Shen}}, \bibinfo {author} {\bibfnamefont {Y.}~\bibnamefont {Zhang}},
  \bibinfo {author} {\bibfnamefont {Y.}~\bibnamefont {Komijani}}, \bibinfo
  {author} {\bibfnamefont {M.}~\bibnamefont {Nicklas}}, \bibinfo {author}
  {\bibfnamefont {R.}~\bibnamefont {Borth}}, \bibinfo {author} {\bibfnamefont
  {A.}~\bibnamefont {Wang}}, \bibinfo {author} {\bibfnamefont {Y.}~\bibnamefont
  {Chen}}, \bibinfo {author} {\bibfnamefont {Z.}~\bibnamefont {Nie}}, \bibinfo
  {author} {\bibfnamefont {R.}~\bibnamefont {Li}}, \bibinfo {author}
  {\bibfnamefont {X.}~\bibnamefont {Lu}},  \emph {et~al.},\ }\href
  {https://doi.org/10.1038/s41586-020-2052-z} {\bibfield  {journal} {\bibinfo
  {journal} {Nature}\ }\textbf {\bibinfo {volume} {579}},\ \bibinfo {pages}
  {51} (\bibinfo {year} {2020})}\BibitemShut {NoStop}%
\bibitem [{\citenamefont {Sachdev}(2010)}]{sachdev2010strange}%
  \BibitemOpen
  \bibfield  {author} {\bibinfo {author} {\bibfnamefont {S.}~\bibnamefont
  {Sachdev}},\ }\href {https://dx.doi.org/10.1088/1742-5468/2010/11/P11022}
  {\bibfield  {journal} {\bibinfo  {journal} {Journal of Statistical Mechanics:
  Theory and Experiment}\ }\textbf {\bibinfo {volume} {2010}},\ \bibinfo
  {pages} {P11022} (\bibinfo {year} {2010})}\BibitemShut {NoStop}%
\bibitem [{\citenamefont {Varma}(2020)}]{varma2020colloquium}%
  \BibitemOpen
  \bibfield  {author} {\bibinfo {author} {\bibfnamefont {C.~M.}\ \bibnamefont
  {Varma}},\ }\href {https://link.aps.org/doi/10.1103/RevModPhys.92.031001}
  {\bibfield  {journal} {\bibinfo  {journal} {Reviews of Modern Physics}\
  }\textbf {\bibinfo {volume} {92}},\ \bibinfo {pages} {031001} (\bibinfo
  {year} {2020})}\BibitemShut {NoStop}%
\bibitem [{\citenamefont {Phillips}\ \emph {et~al.}(2022)\citenamefont
  {Phillips}, \citenamefont {Hussey},\ and\ \citenamefont
  {Abbamonte}}]{phillips2022stranger}%
  \BibitemOpen
  \bibfield  {author} {\bibinfo {author} {\bibfnamefont {P.~W.}\ \bibnamefont
  {Phillips}}, \bibinfo {author} {\bibfnamefont {N.~E.}\ \bibnamefont
  {Hussey}}, \ and\ \bibinfo {author} {\bibfnamefont {P.}~\bibnamefont
  {Abbamonte}},\ }\href {https://doi.org/10.1126/science.abh4273} {\bibfield
  {journal} {\bibinfo  {journal} {Science}\ }\textbf {\bibinfo {volume}
  {377}},\ \bibinfo {pages} {eabh4273} (\bibinfo {year} {2022})}\BibitemShut
  {NoStop}%
\bibitem [{\citenamefont {Timusk}\ and\ \citenamefont
  {Statt}(1999)}]{timusk1999pseudogap}%
  \BibitemOpen
  \bibfield  {author} {\bibinfo {author} {\bibfnamefont {T.}~\bibnamefont
  {Timusk}}\ and\ \bibinfo {author} {\bibfnamefont {B.}~\bibnamefont {Statt}},\
  }\href {https://doi.org/10.1088/0034-4885/62/1/002} {\bibfield  {journal}
  {\bibinfo  {journal} {Reports on Progress in Physics}\ }\textbf {\bibinfo
  {volume} {62}},\ \bibinfo {pages} {61} (\bibinfo {year} {1999})}\BibitemShut
  {NoStop}%
\bibitem [{\citenamefont {Lawler}\ \emph {et~al.}(2010)\citenamefont {Lawler},
  \citenamefont {Fujita}, \citenamefont {Lee}, \citenamefont {Schmidt},
  \citenamefont {Kohsaka}, \citenamefont {Kim}, \citenamefont {Eisaki},
  \citenamefont {Uchida}, \citenamefont {Davis}, \citenamefont {Sethna} \emph
  {et~al.}}]{lawler2010intra}%
  \BibitemOpen
  \bibfield  {author} {\bibinfo {author} {\bibfnamefont {M.}~\bibnamefont
  {Lawler}}, \bibinfo {author} {\bibfnamefont {K.}~\bibnamefont {Fujita}},
  \bibinfo {author} {\bibfnamefont {J.}~\bibnamefont {Lee}}, \bibinfo {author}
  {\bibfnamefont {A.}~\bibnamefont {Schmidt}}, \bibinfo {author} {\bibfnamefont
  {Y.}~\bibnamefont {Kohsaka}}, \bibinfo {author} {\bibfnamefont {C.~K.}\
  \bibnamefont {Kim}}, \bibinfo {author} {\bibfnamefont {H.}~\bibnamefont
  {Eisaki}}, \bibinfo {author} {\bibfnamefont {S.}~\bibnamefont {Uchida}},
  \bibinfo {author} {\bibfnamefont {J.}~\bibnamefont {Davis}}, \bibinfo
  {author} {\bibfnamefont {J.}~\bibnamefont {Sethna}},  \emph {et~al.},\ }\href
  {https://doi.org/10.1038/nature09169} {\bibfield  {journal} {\bibinfo
  {journal} {Nature}\ }\textbf {\bibinfo {volume} {466}},\ \bibinfo {pages}
  {347} (\bibinfo {year} {2010})}\BibitemShut {NoStop}%
\bibitem [{\citenamefont {Zhao}\ \emph {et~al.}(2017)\citenamefont {Zhao},
  \citenamefont {Belvin}, \citenamefont {Liang}, \citenamefont {Bonn},
  \citenamefont {Hardy}, \citenamefont {Armitage},\ and\ \citenamefont
  {Hsieh}}]{zhao2017global}%
  \BibitemOpen
  \bibfield  {author} {\bibinfo {author} {\bibfnamefont {L.}~\bibnamefont
  {Zhao}}, \bibinfo {author} {\bibfnamefont {C.}~\bibnamefont {Belvin}},
  \bibinfo {author} {\bibfnamefont {R.}~\bibnamefont {Liang}}, \bibinfo
  {author} {\bibfnamefont {D.}~\bibnamefont {Bonn}}, \bibinfo {author}
  {\bibfnamefont {W.}~\bibnamefont {Hardy}}, \bibinfo {author} {\bibfnamefont
  {N.}~\bibnamefont {Armitage}}, \ and\ \bibinfo {author} {\bibfnamefont
  {D.}~\bibnamefont {Hsieh}},\ }\href {https://doi.org/10.1038/nphys3962}
  {\bibfield  {journal} {\bibinfo  {journal} {Nature Physics}\ }\textbf
  {\bibinfo {volume} {13}},\ \bibinfo {pages} {250} (\bibinfo {year}
  {2017})}\BibitemShut {NoStop}%
\bibitem [{\citenamefont {Varma}\ \emph {et~al.}(1989)\citenamefont {Varma},
  \citenamefont {Littlewood}, \citenamefont {Schmitt-Rink}, \citenamefont
  {Abrahams},\ and\ \citenamefont {Ruckenstein}}]{varma1989phenomenology}%
  \BibitemOpen
  \bibfield  {author} {\bibinfo {author} {\bibfnamefont {C.}~\bibnamefont
  {Varma}}, \bibinfo {author} {\bibfnamefont {P.~B.}\ \bibnamefont
  {Littlewood}}, \bibinfo {author} {\bibfnamefont {S.}~\bibnamefont
  {Schmitt-Rink}}, \bibinfo {author} {\bibfnamefont {E.}~\bibnamefont
  {Abrahams}}, \ and\ \bibinfo {author} {\bibfnamefont {A.}~\bibnamefont
  {Ruckenstein}},\ }\href@noop {} {\bibfield  {journal} {\bibinfo  {journal}
  {Physical Review Letters}\ }\textbf {\bibinfo {volume} {63}},\ \bibinfo
  {pages} {1996} (\bibinfo {year} {1989})}\BibitemShut {NoStop}%
\bibitem [{\citenamefont {Varma}(1999)}]{varma1999pseudogap}%
  \BibitemOpen
  \bibfield  {author} {\bibinfo {author} {\bibfnamefont {C.~M.}\ \bibnamefont
  {Varma}},\ }\href {\doibase 10.1103/PhysRevLett.83.3538} {\bibfield
  {journal} {\bibinfo  {journal} {Phys. Rev. Lett.}\ }\textbf {\bibinfo
  {volume} {83}},\ \bibinfo {pages} {3538} (\bibinfo {year}
  {1999})}\BibitemShut {NoStop}%
\bibitem [{\citenamefont {Yang}\ \emph {et~al.}(2006)\citenamefont {Yang},
  \citenamefont {Rice},\ and\ \citenamefont {Zhang}}]{kyang2006_YRZ}%
  \BibitemOpen
  \bibfield  {author} {\bibinfo {author} {\bibfnamefont {K.-Y.}\ \bibnamefont
  {Yang}}, \bibinfo {author} {\bibfnamefont {T.~M.}\ \bibnamefont {Rice}}, \
  and\ \bibinfo {author} {\bibfnamefont {F.-C.}\ \bibnamefont {Zhang}},\ }\href
  {\doibase 10.1103/PhysRevB.73.174501} {\bibfield  {journal} {\bibinfo
  {journal} {Phys. Rev. B}\ }\textbf {\bibinfo {volume} {73}},\ \bibinfo
  {pages} {174501} (\bibinfo {year} {2006})}\BibitemShut {NoStop}%
\bibitem [{\citenamefont {Rice}\ \emph {et~al.}(2011)\citenamefont {Rice},
  \citenamefont {Yang},\ and\ \citenamefont {Zhang}}]{Rice_2012}%
  \BibitemOpen
  \bibfield  {author} {\bibinfo {author} {\bibfnamefont {T.~M.}\ \bibnamefont
  {Rice}}, \bibinfo {author} {\bibfnamefont {K.-Y.}\ \bibnamefont {Yang}}, \
  and\ \bibinfo {author} {\bibfnamefont {F.~C.}\ \bibnamefont {Zhang}},\ }\href
  {\doibase 10.1088/0034-4885/75/1/016502} {\bibfield  {journal} {\bibinfo
  {journal} {Reports on Progress in Physics}\ }\textbf {\bibinfo {volume}
  {75}},\ \bibinfo {pages} {016502} (\bibinfo {year} {2011})}\BibitemShut
  {NoStop}%
\bibitem [{\citenamefont {Phillips}\ \emph {et~al.}(2020)\citenamefont
  {Phillips}, \citenamefont {Yeo},\ and\ \citenamefont
  {Huang}}]{phillips2020exact}%
  \BibitemOpen
  \bibfield  {author} {\bibinfo {author} {\bibfnamefont {P.~W.}\ \bibnamefont
  {Phillips}}, \bibinfo {author} {\bibfnamefont {L.}~\bibnamefont {Yeo}}, \
  and\ \bibinfo {author} {\bibfnamefont {E.~W.}\ \bibnamefont {Huang}},\ }\href
  {https://doi.org/10.1038/s41567-020-0988-4} {\bibfield  {journal} {\bibinfo
  {journal} {Nature Physics}\ }\textbf {\bibinfo {volume} {16}},\ \bibinfo
  {pages} {1175} (\bibinfo {year} {2020})}\BibitemShut {NoStop}%
\bibitem [{\citenamefont {Vojta}(2001)}]{vojta2001multichannel}%
  \BibitemOpen
  \bibfield  {author} {\bibinfo {author} {\bibfnamefont {M.}~\bibnamefont
  {Vojta}},\ }\href {https://link.aps.org/doi/10.1103/PhysRevLett.87.097202}
  {\bibfield  {journal} {\bibinfo  {journal} {Physical Review Letters}\
  }\textbf {\bibinfo {volume} {87}},\ \bibinfo {pages} {097202} (\bibinfo
  {year} {2001})}\BibitemShut {NoStop}%
\bibitem [{\citenamefont {Mitchell}\ \emph {et~al.}(2013)\citenamefont
  {Mitchell}, \citenamefont {Vojta}, \citenamefont {Bulla},\ and\ \citenamefont
  {Fritz}}]{mitchell2013quantum}%
  \BibitemOpen
  \bibfield  {author} {\bibinfo {author} {\bibfnamefont {A.~K.}\ \bibnamefont
  {Mitchell}}, \bibinfo {author} {\bibfnamefont {M.}~\bibnamefont {Vojta}},
  \bibinfo {author} {\bibfnamefont {R.}~\bibnamefont {Bulla}}, \ and\ \bibinfo
  {author} {\bibfnamefont {L.}~\bibnamefont {Fritz}},\ }\href {\doibase
  10.1103/PhysRevB.88.195119} {\bibfield  {journal} {\bibinfo  {journal} {Phys.
  Rev. B}\ }\textbf {\bibinfo {volume} {88}},\ \bibinfo {pages} {195119}
  (\bibinfo {year} {2013})}\BibitemShut {NoStop}%
\bibitem [{\citenamefont {Shtyk}\ \emph {et~al.}(2017)\citenamefont {Shtyk},
  \citenamefont {Goldstein},\ and\ \citenamefont
  {Chamon}}]{PhysRevB.95.035137}%
  \BibitemOpen
  \bibfield  {author} {\bibinfo {author} {\bibfnamefont {A.}~\bibnamefont
  {Shtyk}}, \bibinfo {author} {\bibfnamefont {G.}~\bibnamefont {Goldstein}}, \
  and\ \bibinfo {author} {\bibfnamefont {C.}~\bibnamefont {Chamon}},\ }\href
  {\doibase 10.1103/PhysRevB.95.035137} {\bibfield  {journal} {\bibinfo
  {journal} {Phys. Rev. B}\ }\textbf {\bibinfo {volume} {95}},\ \bibinfo
  {pages} {035137} (\bibinfo {year} {2017})}\BibitemShut {NoStop}%
\bibitem [{\citenamefont {Kang}\ \emph {et~al.}(2022)\citenamefont {Kang},
  \citenamefont {Fang}, \citenamefont {Kim}, \citenamefont {Ortiz},
  \citenamefont {Ryu}, \citenamefont {Kim}, \citenamefont {Yoo}, \citenamefont
  {Sangiovanni}, \citenamefont {Di~Sante}, \citenamefont {Park} \emph
  {et~al.}}]{kang2022twofold}%
  \BibitemOpen
  \bibfield  {author} {\bibinfo {author} {\bibfnamefont {M.}~\bibnamefont
  {Kang}}, \bibinfo {author} {\bibfnamefont {S.}~\bibnamefont {Fang}}, \bibinfo
  {author} {\bibfnamefont {J.-K.}\ \bibnamefont {Kim}}, \bibinfo {author}
  {\bibfnamefont {B.~R.}\ \bibnamefont {Ortiz}}, \bibinfo {author}
  {\bibfnamefont {S.~H.}\ \bibnamefont {Ryu}}, \bibinfo {author} {\bibfnamefont
  {J.}~\bibnamefont {Kim}}, \bibinfo {author} {\bibfnamefont {J.}~\bibnamefont
  {Yoo}}, \bibinfo {author} {\bibfnamefont {G.}~\bibnamefont {Sangiovanni}},
  \bibinfo {author} {\bibfnamefont {D.}~\bibnamefont {Di~Sante}}, \bibinfo
  {author} {\bibfnamefont {B.-G.}\ \bibnamefont {Park}},  \emph {et~al.},\
  }\href {https://doi.org/10.1038/s41567-021-01451-5} {\bibfield  {journal}
  {\bibinfo  {journal} {Nature Physics}\ }\textbf {\bibinfo {volume} {18}},\
  \bibinfo {pages} {301} (\bibinfo {year} {2022})}\BibitemShut {NoStop}%
\bibitem [{\citenamefont {Hu}\ \emph {et~al.}(2022)\citenamefont {Hu},
  \citenamefont {Wu}, \citenamefont {Ortiz}, \citenamefont {Ju}, \citenamefont
  {Han}, \citenamefont {Ma}, \citenamefont {Plumb}, \citenamefont {Radovic},
  \citenamefont {Thomale}, \citenamefont {Wilson} \emph {et~al.}}]{hu2022rich}%
  \BibitemOpen
  \bibfield  {author} {\bibinfo {author} {\bibfnamefont {Y.}~\bibnamefont
  {Hu}}, \bibinfo {author} {\bibfnamefont {X.}~\bibnamefont {Wu}}, \bibinfo
  {author} {\bibfnamefont {B.~R.}\ \bibnamefont {Ortiz}}, \bibinfo {author}
  {\bibfnamefont {S.}~\bibnamefont {Ju}}, \bibinfo {author} {\bibfnamefont
  {X.}~\bibnamefont {Han}}, \bibinfo {author} {\bibfnamefont {J.}~\bibnamefont
  {Ma}}, \bibinfo {author} {\bibfnamefont {N.~C.}\ \bibnamefont {Plumb}},
  \bibinfo {author} {\bibfnamefont {M.}~\bibnamefont {Radovic}}, \bibinfo
  {author} {\bibfnamefont {R.}~\bibnamefont {Thomale}}, \bibinfo {author}
  {\bibfnamefont {S.~D.}\ \bibnamefont {Wilson}},  \emph {et~al.},\ }\href
  {https://doi.org/10.1038/s41467-022-29828-x} {\bibfield  {journal} {\bibinfo
  {journal} {Nature Communications}\ }\textbf {\bibinfo {volume} {13}},\
  \bibinfo {pages} {2220} (\bibinfo {year} {2022})}\BibitemShut {NoStop}%
\bibitem [{\citenamefont {Han}\ \emph {et~al.}(2023)\citenamefont {Han},
  \citenamefont {Schnyder},\ and\ \citenamefont {Wu}}]{PhysRevB.107.184504}%
  \BibitemOpen
  \bibfield  {author} {\bibinfo {author} {\bibfnamefont {X.}~\bibnamefont
  {Han}}, \bibinfo {author} {\bibfnamefont {A.~P.}\ \bibnamefont {Schnyder}}, \
  and\ \bibinfo {author} {\bibfnamefont {X.}~\bibnamefont {Wu}},\ }\href
  {\doibase 10.1103/PhysRevB.107.184504} {\bibfield  {journal} {\bibinfo
  {journal} {Phys. Rev. B}\ }\textbf {\bibinfo {volume} {107}},\ \bibinfo
  {pages} {184504} (\bibinfo {year} {2023})}\BibitemShut {NoStop}%
\bibitem [{\citenamefont {Isobe}\ and\ \citenamefont
  {Fu}(2019)}]{isobe2019supermetal}%
  \BibitemOpen
  \bibfield  {author} {\bibinfo {author} {\bibfnamefont {H.}~\bibnamefont
  {Isobe}}\ and\ \bibinfo {author} {\bibfnamefont {L.}~\bibnamefont {Fu}},\
  }\href {\doibase 10.1103/PhysRevResearch.1.033206} {\bibfield  {journal}
  {\bibinfo  {journal} {Physical Review Research}\ }\textbf {\bibinfo {volume}
  {1}},\ \bibinfo {pages} {033206} (\bibinfo {year} {2019})}\BibitemShut
  {NoStop}%
\end{thebibliography}%



\end{document}
