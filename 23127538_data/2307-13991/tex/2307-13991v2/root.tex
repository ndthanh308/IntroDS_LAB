%%%%%%%%%%%%%%%%%%%%%%%%%%%%%%%%%%%%%%%%%%%%%%%%%%%%%%%%%%%%%%%%%%%%%%%%%%%%%%%%

\documentclass[letterpaper, 10 pt, conference]{ieeeconf}


\IEEEoverridecommandlockouts                           
\overrideIEEEmargins    

\usepackage{times}
% numbers option provides compact numerical references in the text. 
\usepackage{graphics} % for pdf, bitmapped graphics files
\usepackage{mathtools}
\usepackage{graphicx}
\usepackage{tabularx}
\usepackage{caption}
\usepackage{subcaption}
\usepackage{wrapfig}
\usepackage{romannum}
\usepackage{amsmath,amssymb}
\usepackage{url}
\usepackage{bm}
\usepackage[table,xcdraw]{xcolor}
% \usepackage{hyperref}
\usepackage[colorlinks=true]{hyperref}

\usepackage{cite}

\newcommand{\Tref}[1]{Table~\ref{#1}}
\newcommand{\Eref}[1]{Eq.~(\ref{#1})}
% \newcommand{\Eref}[1]{(\ref{#1})}
\newcommand{\Fref}[1]{Fig.~\ref{#1}}
\newcommand{\Sref}[1]{Section~\ref{#1}}

\usepackage{multirow}
\usepackage{booktabs}
\usepackage{pifont}
\usepackage{arydshln}

\usepackage[ruled,vlined]{algorithm2e}
\usepackage[noend]{algpseudocode}
\makeatletter
\let\OldStatex\Statex
\renewcommand{\Statex}[1][3]{%
  \setlength\@tempdima{\algorithmicindent}%
  \OldStatex\hskip\dimexpr#1\@tempdima\relax}
\renewcommand{\ALG@beginalgorithmic}{\normalsize}

\renewcommand{\labelenumi}{\roman{enumi})}
\newcommand{\rpartial}{{\mathrm{\partial}}}
\newcommand{\rinf}{{\mathrm \inf}}
\newcommand{\rd}{{\mathrm d}}
\newcommand{\rdelta}{{\mathrm \delta}}
\newcommand{\rtr}{{\mathrm{tr}}}
\newcommand{\rP}{{\mathrm{P}}}
\newcommand{\rT}{{\mathrm{T}}}
\newcommand{\rvec}{{\mathrm{vec}}}
\newcommand{\rtau}{{\mathrm{\tau}}}
\newcommand{\va}{{\bf a}}
\newcommand{\vb}{{\bf b}}
\newcommand{\vc}{{\bf c}}
\newcommand{\vx}{{\bf x}}
\newcommand{\vy}{{\bf y}}
\newcommand{\vz}{{\bf z}}
\newcommand{\vp}{{\bf p}}
\newcommand{\dvx}{{\bf dx}}
\newcommand{\vdu}{{\bf du}}
\newcommand{\vdy}{{\bf dy}}
\newcommand{\vq}{{\bf q}}
\newcommand{\vo}{{\bf o}}
\newcommand{\vt}{{\bf t}}
\newcommand{\vr}{{\bf r}}
\newcommand{\vs}{{\bf  s}}
\newcommand{\vf}{{\bf f}}
\newcommand{\vg}{{\bf g}}
\newcommand{\vG}{{\bf G}}
\newcommand{\vu}{{\bf u}}
\newcommand{\vU}{{\bf U}}
\newcommand{\vv}{{\bf  v}}
\newcommand{\vI}{{\bf I}}
\newcommand{\vh}{{\bf h}}
\newcommand{\vk}{{\bf k}}
\newcommand{\vl}{{\bf l}}
\newcommand{\vdx}{{\bf dx}}
\newcommand{\vdw}{{\bf dw}}
\newcommand{\vw}{{\bf w}}
\newcommand{\vW}{{\bf W}}
\newcommand{\vA}{{\bf A}}
\newcommand{\vR}{{\bf R}}
\newcommand{\vE}{{\bf E}}
\newcommand{\vL}{{\bf L}}
\newcommand{\vK}{{\bf K}}
\newcommand{\vJ}{{\bf J}}
\newcommand{\vF}{{\bf F}}
\newcommand{\vM}{{\bf M}}
\newcommand{\vD}{{\bf D}}
\newcommand{\vN}{{\bf N}}
\newcommand{\vV}{{\bf V}}
\newcommand{\vH}{{\bf H}}
\newcommand{\vT}{{\bf T}}
\newcommand{\vC}{{\bf C}}
\newcommand{\vO}{{\bf O}}
\newcommand{\vQ}{{\bf Q}}
\newcommand{\vB}{{\bf B}}
\newcommand{\vS}{{\bf S}}
\newcommand{\vX}{{\bf X}}
\newcommand{\vY}{{\bf Y}}
\newcommand{\vP}{{\bf P}}
\newcommand{\rank}{{\text{rank}}}
\newcommand{\vect}{{\text{vec}}}
\newcommand{\vxi}{{\mbox{\boldmath$\xi$}}}
\newcommand{\vpi}{{\mbox{\boldmath$\pi$}}}
\newcommand{\vdomega}{{\bf d\omega}}
\newcommand{\vlambda}{{\mbox{\boldmath$\lambda$}}}
\newcommand{\vBamma}{{\mbox{\boldmath$\Gamma$}}}
\newcommand{\VTheta}{{\mbox{\boldmath$\Theta$}}}
\newcommand{\VPhi}{{\mbox{\boldmath$\Phi$}}}
\newcommand{\vphi}{{\mbox{\boldmath$\phi$}}}
\newcommand{\VPsi}{{\mbox{\boldmath$\Psi$}}}
\newcommand{\ve}{{\mbox{\boldmath$\epsilon$}}}
\newcommand{\VSigma}{{\mbox{\boldmath$\Sigma$}}}
\newcommand{\valpha}{{\mbox{\boldmath$\alpha$}}}
\newcommand{\vmu}{{\mbox{\boldmath$\mu$}}}
\newcommand{\vbeta}{{\mbox{\boldmath$\beta$}}}
\newcommand{\vomega}{{\mbox{\boldmath$\omega$}}}
\newcommand{\vtau}{{\mbox{\boldmath$\tau$}}}
\newcommand{\vdtau}{{\mbox{\boldmath$d\tau$}}}
\newcommand{\vtheta}{{\mbox{\boldmath$\theta$}}}\newcommand{\dataset}{{\cal D}}
\newcommand{\fracpartial}[2]{\frac{\partial #1}{\partial  #2}}
\newcommand{\vcalS}{{\mbox{\boldmath$\cal{S}$}}}
\newcommand{\vcalU}{{\mbox{\boldmath$\cal{U}$}}}
\newcommand{\vcalD}{{\mbox{\boldmath$\cal{D}$}}}
\newcommand{\vcalJ}{{\mbox{\boldmath$\cal{J}$}}}
\newcommand{\vcalE}{{\mbox{\boldmath$\cal{E}$}}}
\newcommand{\vcalF}{{\mbox{\boldmath$\cal{F}$}}}
\newcommand{\vcalL}{{\mbox{\boldmath$\cal{L}$}}}
\newcommand{\vcalZ}{{\mbox{\boldmath$\cal{Z}$}}}
\newcommand{\vcalG}{{\mbox{\boldmath$\cal{G}$}}}
\newcommand{\vcalN}{{\mbox{\boldmath$\cal{N}$}}}
\newcommand{\vcalM}{{\mbox{\boldmath$\cal{M}$}}}
\newcommand{\vcalH}{{\mbox{\boldmath$\cal{H}$}}}
\newcommand{\vcalC}{{\mbox{\boldmath$\cal{C}$}}}
\newcommand{\vcalO}{{\mbox{\boldmath$\cal{O}$}}}
\newcommand{\vcalP}{{\mbox{\boldmath$\cal{P}$}}}
\newcommand{\vcalB}{{\mbox{\boldmath$\cal{B}$}}}
\newcommand{\vcalA}{{\mbox{\boldmath$\cal{A}$}}}
\newcommand{\vcalg}{{\mbox{\boldmath$\cal{g}$}}}
\newcommand{\argmax}{\operatornamewithlimits{argmax}}
\newcommand{\argmin}{\operatornamewithlimits{argmin}}
\newcommand{\T}{^\mathsf{T}}
\newcommand{\E}{\mathbb{E}}
\newcommand{\calJ}{\mathcal{J}}
\newcommand{\calF}{\mathcal{F}}
\newcommand{\calL}{\mathcal{L}}
\newcommand{\calD}{\mathcal{D}}
\newcommand{\calG}{\mathcal{G}}
\newcommand{\calE}{\mathcal{E}}
\newcommand{\calN}{\mathcal{N}}

\newcommand{\fu}{\frac{\partial \vf}{\partial \vu}}

\DeclareMathOperator{\Tr}{Tr}

\newcommand{\Qb}{\mathbb{Q}}
\newcommand{\Db}{\mathbb{D}}
\newcommand{\Pb}{\mathbb{P}}
\newcommand{\Fb}{\mathbb{F}}
\newcommand{\Hb}{\mathbb{H}}
\newcommand{\Rb}{\mathbb{R}}
\newcommand{\Xspace}{\mathcal{X}}
\newcommand{\snew}{\mathbf{x}^\prime}
\newcommand{\s}{\mathbf{x}}
\newcommand{\Pas}[2]{\mathcal{P}\left({#1}|{#2}\right)}
\newcommand{\Pol}[2]{\mathcal{U}\left({#1}|{#2}\right)}
\newcommand{\PolOpt}[2]{\mathcal{U}^{*}\left({#1}|{#2}\right)}
\newcommand{\PolPasT}[1]{\mathcal{P}\left({#1}\right)}
\newcommand{\PolT}[1]{\mathcal{U}\left({#1}\right)}
\newcommand{\traj}{\mathbf{X}}
\newcommand{\KL}[2]{\mathbb{D}_{\mathrm{KL}}\left({#1}\parallel {#2}\right)}
\newcommand{\costt}[1]{\mathcal{J}\left({#1}\right)}
\newcommand{\logb}[1]{\log\left({#1}\right)}
\newcommand{\expb}[1]{\exp\left({#1}\right)}
\newcommand{\val}[2]{\mathit{V}_{#2}\left({#1}\right)}
\newcommand{\zt}[2]{\Phi_{#2}\left({#1}\right)}
\newcommand{\nZ}[2]{\mathcal{G}_t[\Phi]\left({#1}\right)}
\newcommand{\ExP}[2]{\E_{{#1}}{\left[#2\right]}}
\newcommand{\vnu}{{\bf \nu}}
\newcommand{\pluseq}{\mathrel{+}=}


\DeclareCaptionFont{mysize}{\fontsize{8}{9.6}\selectfont}
\captionsetup{font=mysize}

\newcommand{\red}[1]{\textcolor{red}{#1}}
% \newcommand{\red}[1]{\textcolor{black}{#1}}

%%%%%%%%%%%%%%%%%%%%%%%%%%%%%%%%%%%%%%%%%%%%%%%%%%%%%%%%%%%%%%%%%%%%%%%%%%%%%%%%
\title{\LARGE \bf
METAVerse: Meta-Learning Traversability Cost Map for \\ Off-Road Navigation
}

\author{Junwon Seo$^{1}$, Taekyung Kim$^{2}$, Seongyong Ahn$^{1}$, and Kiho Kwak$^{1}$%
\thanks{This work was supported by the Agency For Defense Development Grant funded by the Korean Government in 2024.}% <-this % stops a space
\thanks{$^{1}$Junwon Seo, Seongyong Ahn, and Kiho Kwak are with the Agency for Defense Development, Daejeon 34186, Republic of Korea
        {\tt\footnotesize \{junwon.vision, 	seongyong.ahn, kkwak.add\}@gmail.com}}%
% \thanks{$^{2}$Taekyung Kim was with the Agency for Defense Development. He is now with the Department of Robotics, University of Michigan, Ann Arbor, MI, 48109 USA
        % {\tt\footnotesize taekyung@umich.edu}}%
\thanks{$^{2}$Taekyung Kim is with the Department of Robotics, University of Michigan, Ann Arbor, MI, 48109, USA {\tt\footnotesize taekyung@umich.edu}}%
\thanks{Our video can be found at \href{https://youtu.be/4rIAMM1ZKMo}{\tt\footnotesize https://youtu.be/4rIAMM1ZKMo}}%
}
\begin{document}

% https://youtu.be/i55Zk_76xcs
\maketitle


%%%%%%%%%%%%%%%%%%%%%%%%%%%%%%%%%%%%%%%%%%%%%%%%%%%%%%%%%%%%%%%%%%%%%%%%%%%%%%%%
\begin{abstract}
Autonomous navigation in off-road conditions requires an accurate estimation of terrain traversability. However, traversability estimation in unstructured environments is subject to high uncertainty due to the variability of numerous factors that influence vehicle-terrain interaction. Consequently, it is challenging to obtain a generalizable model that can accurately predict traversability in a variety of environments. This paper presents \textit{METAVerse}, a meta-learning framework for learning a global model that accurately and reliably predicts terrain traversability across diverse environments. We train the traversability prediction network to generate a dense and continuous-valued cost map from a sparse LiDAR point cloud, leveraging vehicle-terrain interaction feedback in a self-supervised manner. Meta-learning is utilized to train a global model with driving data collected from multiple environments, effectively minimizing estimation uncertainty. During deployment, online adaptation is performed to rapidly adapt the network to the local environment by exploiting recent interaction experiences. To conduct a comprehensive evaluation, we collect driving data from various terrains and demonstrate that our method can obtain a global model that minimizes uncertainty. Moreover, by integrating our model with a model predictive controller, we demonstrate that the reduced uncertainty results in safe and stable navigation in unstructured and unknown terrains.
\end{abstract}





%%%%%%%%%%%%%%%%%%%%%%%%%%%%%%%%%%%%%%%%%%%%%%%%%%%%%%%%%%%%%%%%%%%%%%%%%%%%%%%%
\section{INTRODUCTION}
\input{_I.Introduction/1.1.introduction}

%%%%%%%%%%%%%%%%%%%%%%%%%%%%%%%%%%%%%%%%%%%%%%%%%%%%%%%%%%%%%%%%%%%%%%%%%%%%%%%%
\section{RELATED WORKS}
\subsection{Traversability Estimation in Off-Road}
Earlier works estimate traversability based on rule-based features derived from geometric and visual appearances, such as the terrain's roughness and slope~\cite{sock2016probabilistic, ahtiainen2017normal, kim2017traversable}. With the advent of deep neural networks, semantic segmentation has been widely utilized to classify off-road terrains according to their navigability levels~\cite{contrastive_offroad, guan2022ga}. Recent work~\cite{shaban2022semantic} has utilized semantic scene completion to generate a dense terrain classification map in bird’s eye view~(BEV) from a sparse point cloud~\cite{fei2021pillarsegnet}.
%The predictions are converted into a dense map in bird’s eye view (BEV) for path planning. 

Unfortunately, these human-supervised methods cannot provide adequate information for effective navigation in complex and unstructured off-road environments. The cost assigned to a predetermined terrain class would be irrelevant to a given environment or inaccurately depict navigability~\cite{frey2023fast}. Self-supervised approaches use terrain interaction feedback to circumvent such limitations~\cite{acoustic, wellhausen_2019should, gasparino2022wayfast}. Using information about the terrain traversed by a vehicle, they identify traversable regions or classify terrains into multiple classes to designate differential costs~\cite{kim2006traversability, seo2023learning, guan2023vinet}. Nevertheless, these approaches abstract away the subtle variations in traversability within the same class~\cite{hdif2023}.

%Interactions with terrains measured by proprioceptive sensors are used to label exteroceptive observations.

Recent research has shifted toward predicting a continuous-valued cost for more effective navigation in unstructured environments~\cite{yao2022rca, sathyamoorthy2022terrapn, karnan2023self, chen2023learning}. To define the continuous costs, inertial information pertinent to navigational stability is processed~\cite{hdif2023, yao2022rca, karnan2023self}, or a reinforcement learning framework is leveraged~\cite{zhu2020off, weerakoon2022terp, frey2022locomotion}. By incorporating the local cost map into path planning and control, these methods improve navigational performance in terms of stability and safety. Nevertheless, these methods prioritize distinguishing between terrain types over evaluating subtle differences in traversability among terrains of the same class. Also, these approaches cannot account for terrains with unknown properties~\cite{seo2023learning}. While recent works detect and avoid unobserved terrains~\cite{cai2022probabilistic, seo2023scate}, they cannot obtain a global model that adapts or generalizes to numerous environments.

\subsection{Meta-Learning}
The goal of meta-learning is to train a model that can rapidly adapt to new tasks~\cite{hospedales2021meta}. Model-agnostic meta-learning (MAML)~\cite{finn2017model} obtains the initial parameters of a neural network such that taking a few gradient descent steps from this initialization results in effective generalization. By considering environments or situations with distinct characteristics as different tasks, meta-learning can learn a global initial parameter from heterogeneous datasets without confusion. During inference, the global model is generalizable to numerous tasks, including a novel one, through a few gradient updates~\cite{li2018learning}.

Various works in robotics literature have adopted meta-learning to learn a global model that can reduce uncertainty through rapid adaptation~\cite{Wortsman_2019_CVPR}. Nagabandi et al.~\cite{nagabandi2018learning} trained the dynamics model of a legged robot, which rapidly adapts to its local environment. Recently, Visca et al.~\cite{visca2022deep} proposed a meta-adaptive energy predictor for path planning in unknown terrains. 

% Figure environment removed

%%%%%%%%%%%%%%%%%%%%%%%%%%%%%%%%%%%%%%%%%%%%%%%%%%%%%%%%%%%%%%%%%%%%%%%%%%%%%%%%
\section{METHODS}
This section details our proposed framework for learning traversability. First, we present our traversability cost prediction network, which predicts the continuous traversability cost derived from vehicle-terrain interaction and generates a dense cost map in BEV using a single sweep LiDAR point cloud. Then, we describe our meta-learning method for training the network~(\textit{METAVerse}) to acquire a global model that is generalizable in various environments. 

%Using online adaptation, it can accurately and reliably predict terrain traversability with unknown characteristics.



\subsection{Dense Traversability Cost Map \label{sec:baseline}}
% Motivation
Off-road environments are fraught with bumps and obstacles of varying shapes, despite being in the same terrain class. For safe and effective navigation in off-road terrain, estimating the nuanced traverse cost of the terrain is necessary. A path planner can optimize a trajectory that minimizes disturbances during navigation with the predicted cost. Therefore, we generate dense and continuous traverse cost maps in BEV from a single LiDAR point cloud.

% data generation
The z-axis linear acceleration measured by an IMU is utilized to define traversability cost derived from vehicle-terrain interaction. This component effectively captures the terrain's bumpiness related to the stability of the vehicle in off-road navigation~\cite{Bekhti_verticala}. In addition, the definition of traversability based on vertical acceleration can be advantageous for control performance because model-based controllers frequently employ a vehicle dynamics model ignorant of vertical motions for computational simplicity~\cite{kim_smooth_2022, kim2023bridging}.

Motivated by recent work~\cite{hdif2023}, we define traversability cost using the spectral analysis of z-axis linear acceleration. The wavelet power spectrum is used to precisely characterize the costs of a time series signal, as it eliminates the need to segment signals and apply Fourier transform to each segment. A continuous wavelet transformation with the Morlet wavelet is performed on the z-acceleration, $a_z(t)$, to generate the wavelet coefficient $w_z(f_n, t)$ for each frequency scale $f_n = 2^{n} \cdot f_0$ and time step $t$. Then, the traversability cost $c_t$ is defined by the wavelet power spectrum as follows:
\begin{equation}
    {c}_t = \sum_{n=0}^{j} \frac{\|w_z(f_n, t)\|^2}{f_n},  
\end{equation} where squares of the coefficient are divided by frequency scale to rectify the power spectrum, as suggested by Liu et al.~\cite{liu2007rectification}, and summed over a certain frequency scale range of $f_0=0.16$ to $f_j=5.12$. The calculated ground-truth costs are then assigned to BEV grids along the positions of the trajectory and used for training the traversability cost prediction network.

% base network structure
The traversability cost prediction network is trained to produce a dense top-view cost map using a sparse single-sweep LiDAR point cloud, as shown in Fig.~\ref{fig:pipeline}. Following PointPillars~\cite{lang2019pointpillars}, each point is discretized into sparse pillars. The point in each pillar is encoded as a $4$-dimensional feature consisting of offset from the pillar center and distance from origin $(\Delta x, \Delta y, \Delta z, d)$. Using a simplified PointNet~\cite{qi2017pointnet} that consists of a linear layer, BatchNorm, and ReLU, each pillar of size $(N, 4)$ is converted into sparse pillar features of size $C$, where $N$ is the maximum number of points per pillar. Each pillar feature is scattered back to the pillar locations to create a BEV sparse feature representation of size $(H, W, C)$, where $H$ and $W$ denote the width and height of the grid, respectively. The empty pillars are zero-initialized. A U-Net~\cite{unet} structured network, which has an encoder-decoder architecture with skip connections, is employed to generate a dense pillar feature map of size $D$. It progressively reduces the spatial size of features and captures higher-level semantic information while the decoder upsamples feature maps to recover spatial information.

The dense pillar features are concatenated with parameterized velocity to produce velocity-conditioned cost maps. Fourier feature mapping~\cite{tancik2020fourier} is used to incorporate the vehicle's velocity into the cost prediction~\cite{hdif2023}. The velocity vector is mapped into a higher dimensional representation:
\begin{equation} \label{eq:ffm}
    \begin{split}
    \gamma(v) = [\cos(2\pi b_1 v), &\sin(2\pi b_1 v), \dots, \\
    &\cos(2\pi b_P v), \sin(2\pi b_P v)],
    \end{split}
\end{equation} where $v$ is the norm of the velocity vector, $b_i \sim \mathcal{N}(0, 5^2)$ are sampled from a Gaussian distribution, and $P=10$ is the number of samples. Finally, the MLP head predicts the mean ${\mu}_i$ and standard deviation ${\sigma}_i$ of the traversability for each pillar $i$. The network is trained to minimize the Gaussian log-likelihood:
\begin{equation}\label{gaussiannll}
    \mathcal{L}^{\text{traverse}}\left(\tau, \bm{\theta}\right) = \frac{1}{2}\sum_i \left(\log({\sigma}_i) +  \frac{({\mu}_i - c_i)^2}{{\sigma}_i} \right),
\end{equation} where $\bm{\theta}$, $\tau$, and $c_i$ represent the model parameter, driving data along a trajectory segment, and the ground truth cost associated with the pillar $i$, respectively. The loss calculation is restricted to pillars assigned with ground truth, that is, the vehicle has traversed. Multiple data augmentations, such as random flip, rotation, and translation, are implemented during training to prevent overfitting and produce a dense cost map.

\subsection{METAVerse: Meta-Learning Traversability Cost Map}
% Motivation
Learning a global traversability model using a large dataset~$\mathcal{D}$ of multiple environments leads to high aleatoric uncertainty. To address this problem, we propose a method that can effectively learn a global traversability model capable of rapidly adapting to a new environment based on its recent experiences. Meta-learning can be used to learn a global model of predicting self-supervised traversability cost because it can handle the variability of interaction data obtained from distinct environments. In addition, by performing online adaptation with self-labeled data, it would be able to accurately predict traversability costs during deployment in a variety of environments, including unknown ones.


% MAML(Meta Learning) Concept
MAML~\cite{finn2017model} is used to learn the global traversability model. MAML aims to find the initial parameters of the network so that adaptation with a few gradient descent steps from this initialization leads to effective generalization to the current circumstances. This meta-objective enables the model for predicting traversability cost to incorporate driving data collected in various environments without confusion caused by aleatoric uncertainty in the training phase. During the deployment phase, the network is updated based on recent vehicle-terrain interaction experiences to adapt to dynamic environments and generate an accurate cost map.

% MAML Explanation
While terrain properties vary significantly in different environments, we assume the environment is locally consistent. Consequently, vehicle-terrain interaction of each local trajectory segment, denoted as $\tau$, is regarded as a separate \textit{task}. Instead of considering the entire dataset with distinct properties as a single task, the network is trained with the meta-objective that the recent experiences can provide information about the current task. The past $M$ timesteps provide insight into how to adapt the model to predict the traversability costs of future trajectories precisely. The network is trained to adapt using the \textit{meta-train} data of the past $M$ timesteps, $\tau\left(t-M,t\right)$, to predict the traversability cost of \textit{meta-eval} data from the next $K$ timesteps, $\tau\left(t,t+K\right)$, as follows: 
\begin{equation}\label{eq:meta}
    \begin{aligned}
    \argmin_{\bm{\theta}} \hspace{5pt} & \mathbb{E}_{\tau\left(t-M, t+K\right) \sim \mathcal{D}}  \big[ \mathcal{L}^{\text{traverse}}\left(\tau(t, t+K), \bm{\theta}'\right)\big] \\   
    & \text{s.t.:} \hspace{5pt} \bm{\theta}' = \bm{\theta} - \alpha \nabla_{\bm{\theta}} \mathcal{L}^{\text{traverse}}\left(\tau(t-M, t), \bm{\theta}\right).
    \end{aligned}
\end{equation} 

Algorithm~\ref{Algorithm:meta} outlines the meta-learning-based training procedure of \textit{METAVerse} for obtaining the global traversability model. The inner loops adapt the model with meta-train data by taking $N_A$ adaptation steps via gradient descent. The outer loop updates the initial parameters with losses calculated with $N$ trajectories within a minibatch. 

During deployment, the traversability prediction network is online adapted utilizing meta-train data, as illustrated in Fig.~\ref{fig:concept}. Meta-train data is automatically generated by self-labeling sensor data from LiDAR and IMU that are stored in queues. The online adaptation of the network is performed asynchronously, and only inner loops are executed to accomplish rapid adaptation. The updated network generates an accurate representation of environments, which is then employed for navigation by a model predictive controller. The online adaptation enables the trained global model to be applicable in various environments and even adapt to terrains with unknown properties.


\begin{algorithm}[h]
\small
\SetKwInOut{Input}{Given}
\Input{
%$\bm{\theta}$: Traversability cost prediction model parameters\;
$\mathcal{D}$: Traversability data from various environments\;
$M, K$: Number of past and future timesteps\;
$N$: Number of sampled trajectories within a batch\;
$N_A$: Number of the inner loops\;
$\alpha, \beta$: Learning rates for the inner and outer loops\;
Randomly initialize $\bm{\theta}$\;
\For{$i \leftarrow 0$ \KwTo \text{maximum iterations}}{
    \For{$j \leftarrow 0$ \KwTo $N-1$}{
        Sample $\tau(t-M, t)$, $\tau(t, t+K)$ $\sim \mathcal{D}$\;
        Self-Label $\tau(t-M, t)$ and $\tau(t, t+K)$\;
        $\bm{\theta}' \leftarrow \bm{\theta}$\;
        % start adaptation
        \For{$k \leftarrow 0$ \KwTo $N_A-1$}{
            $\bm{\theta}' \leftarrow \bm{\theta}' - \alpha \nabla_{\bm{\theta'}} \mathcal{L}^{\text{traverse}}\left(\tau(t-M, t), \bm{\theta'}\right)$\;
        }
        $\mathcal{L}_j \leftarrow \mathcal{L}^{\text{traverse}}\left(\tau(t, t+K), \bm{\theta'}\right)$
    }
    $\bm{\theta} \leftarrow  \bm{\theta} - \beta \nabla_{\bm{\theta}} \frac{1}{N} \sum\limits_{j=1}^N \mathcal{L}_j$
}
}
\caption{Meta Learning of Traversability Cost}\label{Algorithm:meta}
\end{algorithm}


%%%%%%%%%%%%%%%%%%%%%%%%%%%%%%%%%%%%%%%%%%%%%%%%%%%%%%%%%%%%%%%%%%%%%%%%%%%%%%%%
\section{EXPERIMENTS}
In this section, we validate the efficacy of our proposed framework. We first demonstrate that our method is capable of learning a global model that minimizes uncertainty in prediction, resulting in a more accurate prediction of traversability costs in diverse environments. We then verify that this accurate and dependable traversability cost prediction leads to stable and effective off-road navigation. 

Our experiments address the following key questions: \textbf{(Q1)}~Can our method learn a global model that minimizes uncertainty for learning traversability in various environments? \textbf{(Q2)}~Does our method facilitate safe and effective navigation in unstructured environments? \textbf{(Q3)}~Can our traversability prediction network adapt effectively to unknown terrains during navigation?

\subsection{Implementation Details}
For all experiments, the input point cloud is cropped at [($0, 51.2$), ($-25.6, 25.6$), ($-5, 10$)] meters along the $x$, $y$, $z$ axes, and a pillar grid size of $0.2m\times0.2m$ is used. For the traversability prediction network, the maximum number of points per pillar is set to ${N=20}$, and the channels of sparse and dense pillar features are set to ${C=128}$ and ${D=64}$, respectively. Each encoder and decoder has five layers, each consisting of a max pooling or transposed convolution layer and two $3 \times 3$ convolution layers, with a ReLU and a BatchNorm layer in the middle.

Our model is trained for $60$ epochs using the Adam optimizer with the outer loop learning rate of $\beta=3e^{-4}$ and a batch size of $16$. Each trajectory data within a batch comprises \textit{meta-train} data for the inner loop and \textit{meta-eval} data for the outer loop. Each meta-train data is composed of eight LiDAR point clouds and the ground-truth traversability costs, which are generated from the trajectory of the previous $M=8$ seconds from a reference time. The meta-eval data also consists of eight point clouds and the ground truth, but they are generated based on the trajectory of the upcoming $K=8$ seconds.  All network parameters are subject to adaptation and are updated $N_A=3$ times at a learning rate for inner loops of $\alpha=1e^{-4}$. During training, random horizontal flipping is applied with a probability of $50\%$, random rotation along the z-axis is applied between ($-\frac{\pi}{4},\frac{\pi}{4}$) radians, and random translation is applied $(-5, 5)$ meters in the $x$ and $y$ axes.

\subsection{Learning Global Traversability Model}\label{sec:global}
\newcommand{\cmark}{\ding{51}}%
\newcommand{\xmark}{\ding{55}}%

\begin{table}[t]
\centering
\renewcommand {\arraystretch}{1.1}
\caption{The sequence length in time~(\textit{Total Len}) and the standard deviations of the ground-truth traversability costs~(\textit{STDEV of Cost}) are displayed to illustrate the details of each evaluation dataset category.}
\tiny
\label{tab:eval_dataset_sum}
\resizebox{0.9\linewidth}{!}{%
    \begin{tabular}{ccccc}
        \toprule
        & \bf{Unpaved} & \bf{Grassland} & \bf{Profiled Road} & \bf{Simulation} \\
        \midrule
        \multicolumn{1}{c|}{\textit{Total Len} (s)} &  531.5 & 390.2 & 1056.9 & 545.9 \\
        \multicolumn{1}{c|}{\textit{STDEV of Cost}}&  0.4751 & 0.5166 & 0.4638 & 5.5651 \\
        \bottomrule
\end{tabular}%(w.o. Adaptation)
}
\vspace*{-0.2in}
\end{table}


We validate the efficacy of our meta-learning method for traversability cost prediction~(\textbf{Q1}) with real-world off-road driving data. Specifically, the validity of the global traversability model trained with meta-objective~(\ref{eq:meta}) is evaluated in terms of prediction accuracy. 

\subsubsection{Experimental Setup}
Driving data is collected in off-road environments with various types of terrain using our platform equipped with OS1-128 LiDAR and IMU~(See Fig.~\ref{fig:concept}). Also, the driving data is obtained in a simulation environment consisting of randomly patterned rough terrain and bumps, which is hazardous to interact with in real-world environments~\cite{seo2023scate}. Approximately three hours of driving data are utilized to train the network. Then, the trained traversability prediction network is evaluated using an evaluation dataset consisting of separate trajectory sequences not included in the training dataset. 

According to the terrain characteristics, we divide our real-world evaluation dataset into $3$ categories, namely, \textit{unpaved}, \textit{grassland}, and \textit{profiled road}. Fig.~\ref{fig:real_data} shows example images of these terrains along with their corresponding visualizations of traversability cost maps. The \textit{unpaved} consists of rough and unpaved dirt tracks with numerous bumps and puddles. The \textit{grassland} represents grassy roads with vegetation, bushes, and cobblestones. The \textit{profiled road} is composed of roads with five different types of artificial road profiles of varying sizes that are used for assessing driving stability. Lastly, \textit{simulation} category is added for the evaluation, which refers to data collected while navigating the off-road evaluation track in the simulation setup~(See Fig.~\ref{fig:driving}). The details of our evaluation dataset are presented in Table.~\ref{tab:eval_dataset_sum}.

For comparison, a network is trained using all train data without meta-objective and adaptation~(\textit{Baseline}). Also, the network trained with meta-objective is evaluated with zero adaptation steps~($N_A$=0) during inference as well as with varying numbers of adaptation steps. Mean Square Error~(MSE) between the ground-truth traversability and the predicted traversability cost of corresponding grids is calculated for the evaluation.

% Figure environment removed



\begin{table}[b]
% \vspace*{-0.1in}
\centering
\renewcommand {\arraystretch}{1.3}
\caption{Validation error of the experiment with real-world driving data. Our method shows a significant margin compared to the baseline, and it can predict traversability more accurately by conducting adaptations during inference.}
\scriptsize
\label{tab:quantitative}
\resizebox{1.0\linewidth}{!}{%
    \begin{tabular}{cccccc}
        \toprule
        \multirow{2}{*}{\textbf{Method}} & \multirow{2}{*}{\textbf{Adaptation}}& \multicolumn{4}{c}{\bf{Evaluation Dataset Category}}\\ 
        \cmidrule(l{4pt}r{4pt}){3-6} 
        & & \bf{Unpaved} & \bf{Grassland} & \bf{Profiled Road} & \bf{Simulation} \\
        \midrule
        \multicolumn{1}{c}{\textit{Baseline}} & \xmark & 0.1222 & 0.2460 & 0.2039 & 0.7263 \\
        \multicolumn{1}{c}{\textit{METAVerse}}& \xmark & 0.0713 & 0.1961 & 0.1907  & 0.6668 \\
        \multicolumn{1}{c}{\textit{METAVerse}}& \checkmark & \textbf{0.0114} & \textbf{0.1767} & \textbf{0.1523} & \textbf{0.5725} \\
        \bottomrule
\end{tabular}%(w.o. Adaptation)
}
% \vspace*{-0.1in}
\end{table}

\subsubsection{Experimental Result}
The quantitative results are presented in Table~\ref{tab:quantitative}. In every category, the model trained with the meta-objective outperforms the baseline model trained without the meta-objective. It indicates that the non-meta-learned baseline failed to converge well due to high aleatoric uncertainty in ground-truth traversability collected across various terrains. In contrast, our meta-learned model can converge well by incorporating such uncertainty during training and adapting the model during inference. Even without adaptation in the evaluation phase, our method outperforms the baseline, implying that utilizing the meta-objective leads to a finding of a better initial parameter~\cite{finn2017model}. In addition, the model's performance improves as it adapts during inference using recent interaction experiences. 



Fig.~\ref{fig:real_adapt_graph} shows the experimental results on the whole evaluation data with varying numbers of adaptation steps during inference. The accuracy improves as the number of adaptation steps increases. Also, our method with meta-objective shows a significant margin over the baseline that does not conduct adaptation during both training and inference. After adaptation, the initially erroneous cost map is adjusted to more accurately represent the cost of terrains by incorporating recent experiences, as illustrated in Fig.~\ref{fig:real_adapt_vis}.




% Figure environment removed

\subsection{Autonomous Navigation in Unstructured Environments}\label{sec:autonomous}
We demonstrate that our method can result in stable navigation (\textbf{Q2}-\textbf{Q3}) by integrating our traversability cost map with a sampling-based model-predictive controller. A high-fidelity vehicle dynamics simulator~-~IPG CarMaker is utilized for the experiments, which enables navigation in a controlled setup and, thus, enables more comprehensive comparisons and analyses of the results. In all experiments, navigation is performed only using local traversability maps generated from LiDAR point clouds with no prior knowledge of the environments. Note that our network runs in real time along with the LiDAR update rate of $10$ hz.
%with SEMIL-1748GC.

\subsubsection{Experimental Setup}

For navigation, we employ the Smooth Model Predictive Path Integral~(SMPPI)~\cite{kim_smooth_2022} controller, a sampling-based model predictive controller that can generate smooth actions during deployment. Table ~\ref{table:mppi} lists the controller's parameters. Based on our previous work \cite{kim_smooth_2022}, we formulate a simple state-dependent running cost function $q(\vx_t)$ of the controller:
\begin{equation}
    q(\vx_t) = \alpha_1{\text{Track}(\vx_t)} + \alpha_2{\text{Stable}(\vx_t)} + \alpha_3{\text{Speed}(\vx_t)},
\end{equation} where $\vx_t$ represents the vehicle state and $\text{Stable}(\vx_t)$ is based on the traversability cost map generated by our method. Based on our previous work~\cite{seo2023scate}, $\text{Track}(\vx_t)$ imposes a significant penalty on regions with high uncertainty, or namely non-traversable regions. $\text{Speed}(\vx_t)$ is a simple quadratic cost that penalizes the difference between the target speed and the vehicle's current speed. Each cost is normalized into $[0, 1]$, and the weight coefficients are set as  $\alpha_1 > \alpha_2 > \alpha_3$. This setting ensures that the vehicle navigates only through traversable regions, and within those regions, it optimizes the trajectory to minimize traversability costs while maintaining the target speed as much as possible.


\begin{table}[hbt]
\caption{Control parameters of SMPPI. Note that the definitions of the variables and the remaining parameters not specified in this paper are provided in the original paper~\cite{kim_smooth_2022}.}
\vspace*{-0.1in}
\renewcommand{\arraystretch}{1.5}
\begin{center}
\resizebox{0.99\linewidth}{!}{
    \begin{tabular}{cccccc}
    \toprule
    \bf{Control Frequency} & \bf{Target Speed} & \bf{Trajectories} & \bf{Horizon} & \bf{Sampling Variance} \\
    \midrule
    $10$ Hz & $30$ km/h & 5,000 & 4 s & ${\bf{\text{Diag}}}(1.6, 0.4)$\\
    \bottomrule
    \end{tabular}
}
\end{center}
\label{table:mppi}
\vspace*{-0.1in}
\end{table}


Based on our previous work~\cite{kim2023bridging}, we use a probabilistic ensemble neural network as the vehicle dynamics. The state and action inputs follow the simplified bicycle model, and it also leverages the history of state action pairs to extract contextual information. For real-time implementation, we use a four-layer MLP and five ensembles.

\subsubsection{Safe Navigation \label{sec:navigation}}
% Figure environment removed


An off-road environment (see Fig.~\ref{fig:driving}) is designed to conduct navigation~(\textbf{Q2}) where challenging unstructured terrains with large and irregularly patterned bumps make navigation solely with classification-based traversability maps inadequate. The control vehicle is a Volvo XC90, which is distinct from the vehicle used for training data collection. To evaluate the efficacy of our global model in enhancing navigation performance, navigation is performed with our method~(\textit{METAVerse}) with and without online adaptation. Experiments are also conducted with a self-supervised method that predicts point-wise traversability~(\textit{Point-wise})~\cite{seo2023scate} instead of generating a dense map. In addition, rule-based methods are compared, including an elevation-map based method~(\textit{Elevation Based})~\cite{fankhauser2014robot,miki2022elevation} and a slope-based method~(\textit{Slope Based})~\cite{sock2016probabilistic,kim2017traversable}. We conduct navigation $15$~times for each method. 

The trajectories taken during navigation are shown in Fig.~\ref{fig:driving}, and the statistics about angular and vertical motions of the vehicle and success rates of navigations are shown in Table~\ref{tab:results}. The navigational performance of \textit{METAVerse} employing online adaptation is superior to that of our model without the adaptation. The traversability network is adjusted online to the novel environment and vehicle of deployment, resulting in improved navigational performance.

In addition, our method with voxelization to efficiently construct a dense cost map improves navigation performance compared to the point-wise self-supervised method, which predicts traversability cost using a point-wise network and generates local traversability maps via interpolation~\cite{seo2023scate}. Due to real-time constraints, the network for point-wise prediction becomes shallower than the voxel-based network, which is sufficiently deep to generate a continuous cost map in BEV. The efficient network structure that directly generates maps in BEV enables our model to embed richer information to predict traversability more precisely in a limited amount of time, thereby enhancing real-time navigational performance. Lastly, the inability of the rule-based methods to reason about nuanced interactions with unstructured terrain leads to instability and an increase in navigational failure rates. On the other hand, the vehicle utilizing a self-supervised traversability map effectively navigates along paths that minimize disturbances. Our method can effectively identify risky regions where terrain adversely impacts vehicle stability.


\subsubsection{Effect of Online Adaptation}


% % Figure environment removed

% Figure environment removed

To validate the efficacy of online adaptation to unknown terrains~(\textbf{Q3}), which ultimately leads to safe navigation, an additional off-road environment is designed with several unknown types of bumps (see Fig.~\ref{fig:adapt_graph}). The vehicle begins adaptation after experiencing three types of unknown bumps. This allows the vehicle to generate trajectories that minimize disturbance when re-encountering the bumps. The navigation is conducted for five trials using our trained traversability prediction networks, with and without online adaptation during inference.

%The vehicle begins adaptation after experiencing three unseen types of bumps. This allows the vehicle to optimize trajectories that minimize disturbance when re-encountering the bumps. T

Fig.~\ref{fig:adapt_graph}a illustrates the navigation results. The vehicle experiences unseen bumps and effectively adapts the model to accommodate the experience. Initially, the predicted traversability of all unobserved bumps is not relatively differentiated because the model has no information about the bumps. As the network begins to be online-adapted using the experience, the predicted costs for bump $\#2$ become lower than the costs for bump $\#1$ and $\#3$. Therefore, the controller chooses to circumvent the challenging bumps, resulting in successful navigation. In contrast, the vehicle without adaptation fails to discern the difficulty and avoid them, eventually resulting in a rollover.

\begin{table}[b]
\vspace*{-0.2in}
\centering
\renewcommand {\arraystretch}{1.2}
\scriptsize
\caption{Navigation results in the scene for evaluating online adaptation (\textbf{Q3}). The average vehicle motions across $5$~trials are shown.
}
\resizebox{1.0\linewidth}{!}{
    \begin{tabular}{cccccccccccccc}
        \toprule
        \multirow{3}{*}{\textbf{Method}} & 
        \multirow{3}{*}{\textbf{\shortstack{Online\\Adaptation}}} & 
        \multirow{2}{*}{\textbf{\shortstack{Vertical\\Vel.}}} &
        \multirow{2}{*}{\textbf{\shortstack{Vertical\\Acc.}}} & 
        \multirow{2}{*}{\textbf{\shortstack{Roll\\Rate}}}&
        \multirow{2}{*}{\textbf{\shortstack{Pitch\\Rate}}}&
        \multirow{2}{*}{\textbf{\shortstack{Roll\\Acc.}}}&
        \multirow{2}{*}{\textbf{\shortstack{Pitch\\Acc.}}}\\
        &&&&&&& \\
        &&\bf{[m/s]}&
        \bf{[m/s$^\text{2}$]}&
        \bf{[rad/s]}&
        \bf{[rad/s]}&
        \bf{[rad/s$^\text{2}$]}&
        \bf{[rad/s$^\text{2}$]} \\
        \midrule
        \textit{METAVerse} & \xmark & 0.3319 & 2.7385 & 0.2563 & 2.1254 & 2.9504 & 2.7333 \\
        \textit{METAVerse} & \checkmark & \textbf{0.1910} & \textbf{1.4347} & \textbf{0.1409} & \textbf{1.9994} & \textbf{1.4919} & \textbf{1.6075}\\
        \bottomrule
    \end{tabular}
    }
\label{table:adapt}
\end{table}

Fig.~\ref{fig:adapt_graph}b shows the vertical acceleration experienced by the vehicle during navigation. By beginning to online-adapt the network, the vehicle can reduce the impact exerted on it by adjusting its trajectory based on recent experiences, whereas the vehicle that does not conduct adaptation continues to experience enormous impacts due to the inability to overcome uncertainty. The navigational stability measured by averaging the vertical and angular motions of the vehicle is presented in Table.~\ref{table:adapt}. It verifies that adaptation can induce stable vehicle motions by adjusting the traversability prediction model to incorporate experiences with unknown terrains.

% \begin{wrapfigure}{R}{0.5\linewidth}
% \vspace*{-0.1in}
% \centering
% % Figure removed
% \caption{Vertical acceleration of the vehicle during navigation.}
% \label{fig:adapt_graph}
% \vspace*{-0.2in}
% \end{wrapfigure}







%%%%%%%%%%%%%%%%%%%%%%%%%%%%%%%%%%%%%%%%%%%%%%%%%%%%%%%%%%%%%%%%%%%%%%%%%%%%%%%%
\section{CONCLUSION}
This paper proposes a meta-learning framework for off-road traversability estimation. Our traversability prediction network predicts terrain traversability derived from vehicle-terrain interactions and generates a dense and continuous-valued cost map from a single-sweep LiDAR point cloud. Meta-learning is used to train a global model that can accurately predict terrain traversability in a variety of environments by minimizing uncertainty. During deployment, the network performs online adaptation utilizing recent interaction experiences to improve the accuracy of predictions. Extensive experiments demonstrate that the proposed method can reduce the uncertainty of the global model, resulting in stable off-road navigation in unstructured and unknown terrains. We believe this concept can be used for the broader deployment of autonomous robots in unstructured environments and improve the reliability of off-road navigation.

We are currently evaluating our navigation method in real-world off-road environments. In future work, we intend to extend this framework using multiple sensor fusions to reduce uncertainty, such as LiDAR-camera fusion. 


\addtolength{\textheight}{0cm}   % This command serves to balance the column lengths
                                  % on the last page of the document manually. It shortens
                                  % the textheight of the last page by a suitable amount.
                                  % This command does not take effect until the next page
                                  % so it should come on the page before the last. Make
                                  % sure that you do not shorten the textheight too much.

\bibliographystyle{IEEEtran}
\bibliography{mybib.bib}

\end{document}
