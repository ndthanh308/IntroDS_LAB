% Motivation
Learning a global traversability model using a large dataset~$\mathcal{D}$ of multiple environments leads to high aleatoric uncertainty. To address this problem, we propose a method that can effectively learn a global traversability model capable of rapidly adapting to a new environment based on its recent experiences. Meta-learning can be used to learn a global model of predicting self-supervised traversability cost because it can handle the variability of interaction data obtained from distinct environments. In addition, by performing online adaptation with self-labeled data, it would be able to accurately predict traversability costs during deployment in a variety of environments, including unknown ones.


% MAML(Meta Learning) Concept
MAML~\cite{finn2017model} is used to learn the global traversability model. MAML aims to find the initial parameters of the network so that adaptation with a few gradient descent steps from this initialization leads to effective generalization to the current circumstances. This meta-objective enables the model for predicting traversability cost to incorporate driving data collected in various environments without confusion caused by aleatoric uncertainty in the training phase. During the deployment phase, the network is updated based on recent vehicle-terrain interaction experiences to adapt to dynamic environments and generate an accurate cost map.

% MAML Explanation
While terrain properties vary significantly in different environments, we assume the environment is locally consistent. Consequently, vehicle-terrain interaction of each local trajectory segment, denoted as $\tau$, is regarded as a separate \textit{task}. Instead of considering the entire dataset with distinct properties as a single task, the network is trained with the meta-objective that the recent experiences can provide information about the current task. The past $M$ timesteps provide insight into how to adapt the model to predict the traversability costs of future trajectories precisely. The network is trained to adapt using the \textit{meta-train} data of the past $M$ timesteps, $\tau\left(t-M,t\right)$, to predict the traversability cost of \textit{meta-eval} data from the next $K$ timesteps, $\tau\left(t,t+K\right)$, as follows: 
\begin{equation}\label{eq:meta}
    \begin{aligned}
    \argmin_{\bm{\theta}} \hspace{5pt} & \mathbb{E}_{\tau\left(t-M, t+K\right) \sim \mathcal{D}}  \big[ \mathcal{L}^{\text{traverse}}\left(\tau(t, t+K), \bm{\theta}'\right)\big] \\   
    & \text{s.t.:} \hspace{5pt} \bm{\theta}' = \bm{\theta} - \alpha \nabla_{\bm{\theta}} \mathcal{L}^{\text{traverse}}\left(\tau(t-M, t), \bm{\theta}\right).
    \end{aligned}
\end{equation} 

Algorithm~\ref{Algorithm:meta} outlines the meta-learning-based training procedure of \textit{METAVerse} for obtaining the global traversability model. The inner loops adapt the model with meta-train data by taking $N_A$ adaptation steps via gradient descent. The outer loop updates the initial parameters with losses calculated with $N$ trajectories within a minibatch. 

During deployment, the traversability prediction network is online adapted utilizing meta-train data, as illustrated in Fig.~\ref{fig:concept}. Meta-train data is automatically generated by self-labeling sensor data from LiDAR and IMU that are stored in queues. The online adaptation of the network is performed asynchronously, and only inner loops are executed to accomplish rapid adaptation. The updated network generates an accurate representation of environments, which is then employed for navigation by a model predictive controller. The online adaptation enables the trained global model to be applicable in various environments and even adapt to terrains with unknown properties.


\begin{algorithm}[h]
\small
\SetKwInOut{Input}{Given}
\Input{
%$\bm{\theta}$: Traversability cost prediction model parameters\;
$\mathcal{D}$: Traversability data from various environments\;
$M, K$: Number of past and future timesteps\;
$N$: Number of sampled trajectories within a batch\;
$N_A$: Number of the inner loops\;
$\alpha, \beta$: Learning rates for the inner and outer loops\;
Randomly initialize $\bm{\theta}$\;
\For{$i \leftarrow 0$ \KwTo \text{maximum iterations}}{
    \For{$j \leftarrow 0$ \KwTo $N-1$}{
        Sample $\tau(t-M, t)$, $\tau(t, t+K)$ $\sim \mathcal{D}$\;
        Self-Label $\tau(t-M, t)$ and $\tau(t, t+K)$\;
        $\bm{\theta}' \leftarrow \bm{\theta}$\;
        % start adaptation
        \For{$k \leftarrow 0$ \KwTo $N_A-1$}{
            $\bm{\theta}' \leftarrow \bm{\theta}' - \alpha \nabla_{\bm{\theta'}} \mathcal{L}^{\text{traverse}}\left(\tau(t-M, t), \bm{\theta'}\right)$\;
        }
        $\mathcal{L}_j \leftarrow \mathcal{L}^{\text{traverse}}\left(\tau(t, t+K), \bm{\theta'}\right)$
    }
    $\bm{\theta} \leftarrow  \bm{\theta} - \beta \nabla_{\bm{\theta}} \frac{1}{N} \sum\limits_{j=1}^N \mathcal{L}_j$
}
}
\caption{Meta Learning of Traversability Cost}\label{Algorithm:meta}
\end{algorithm}
