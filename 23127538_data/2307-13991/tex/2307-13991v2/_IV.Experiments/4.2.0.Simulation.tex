We demonstrate that our method can lead to stable navigation (\textbf{Q2}-\textbf{Q3}) by integrating our traversability cost map with a sampling-based model-predictive controller. A high-fidelity vehicle dynamics simulator~-~IPG CarMaker is utilized for the experiments, which enables navigation in a controlled setup and, thus, enables more comprehensive comparisons and analyses of the results. In all experiments, navigation is performed only using local traversability maps generated in real-time from LiDAR point clouds, with no prior knowledge of the environments or global path planning.

\vspace*{-0.05in}
\begin{table}[hbt]
\caption{Control parameters of SMPPI. Note that the definitions of the variables and the remaining parameters not specified in this paper are provided in the original paper~\cite{kim_smooth_2022}.}
\vspace*{-0.05in}
\renewcommand{\arraystretch}{1.4}
\begin{center}
\resizebox{0.9\linewidth}{!}{
    \begin{tabular}{cccccc}
    \toprule
    \bf{Control Frequency} & \bf{Target Speed} & \bf{Trajectories} & \bf{Horizon} & \bf{Sampling Variance} \\
    \midrule
    $10$ Hz & $30$ km/h & 5,000 & 4 s & ${\bf{\text{Diag}}}(1.6, 0.4)$\\
    \bottomrule
    \end{tabular}
}
\end{center}
\label{table:mppi}
\vspace*{-0.15in}
\end{table}

\subsubsection{Experimental Setup}

For navigation, we employ the Smooth Model Predictive Path Integral (SMPPI)~\cite{kim_smooth_2022} controller, a sampling-based model predictive controller that can generate smooth actions during deployment. Table~\ref{table:mppi} lists the controller's parameters. Based on our previous work \cite{kim_smooth_2022}, we formulate a simple state-dependent running cost function $q(\vx_t)$ of the controller:
\begin{equation}
    q(\vx_t) = \alpha_1{\text{Track}(\vx_t)} + \alpha_2{\text{Stable}(\vx_t)} + \alpha_3{\text{Speed}(\vx_t)},
\end{equation} where $\vx_t$ represents the vehicle state and $\text{Stable}(\vx_t)$ is the predicted traversability cost. Based on our previous work for identifying non-traversable regions~\cite{seo2023scate}, $\text{Track}(\vx_t)$ imposes a significant penalty on regions with high uncertainty to prevent collisions. $\text{Speed}(\vx_t)$ is a simple quadratic cost that penalizes the difference between the target speed and the vehicle's current speed. Each cost is normalized into $[0, 1]$, and the weight coefficients are set as $\alpha_1=10000$, $\alpha_2=10$, and $\alpha_3=1$, namely $\alpha_1 > \alpha_2 > \alpha_3$. This setting ensures that the vehicle navigates only through traversable regions, and within those regions, it optimizes the trajectory to minimize traversability costs while maintaining the target speed as much as possible.

% Figure environment removed



Based on our previous work~\cite{kim2023bridging}, we use a probabilistic ensemble neural network as the vehicle dynamics. The state and action inputs follow the simplified bicycle model, and it also leverages the history of state action pairs to extract contextual information. For real-time implementation, we use a four-layer MLP and five ensembles.

