To validate the efficacy of online adaptation to unknown terrains~(\textbf{Q3}), which ultimately leads to safe navigation, an additional off-road environment is designed with several unknown types of bumps (see Fig.~\ref{fig:adapt_graph}). The vehicle begins adaptation after experiencing three types of unknown bumps. This allows the vehicle to generate trajectories that minimize disturbance when re-encountering these bumps. The navigation is conducted for five trials using our trained traversability prediction networks, with and without online adaptation during inference.

%The vehicle begins adaptation after experiencing three unseen types of bumps. This allows the vehicle to optimize trajectories that minimize disturbance when re-encountering the bumps. T

Fig.~\ref{fig:adapt_graph}\red{a} illustrates the navigation results. The vehicle experiences unseen bumps and effectively adapts the model to accommodate the experience. Initially, the predicted traversability of all unobserved bumps is not relatively differentiated because the model has no information about the bumps. As the network begins to be online-adapted using the experience, the predicted costs for bump $\#2$ become lower than the costs for bump $\#1$ and $\#3$. Therefore, the controller chooses to circumvent the challenging bumps, resulting in successful navigation. In contrast, the vehicle without adaptation fails to discern the difficulties and avoid them, eventually resulting in a rollover.


\begin{table}[t]
% \vspace*{-0.2in}
\centering
\renewcommand {\arraystretch}{1.2}
% \scriptsize
\caption{Navigation results in the scene for evaluating online adaptation (\textbf{Q3}). The average vehicle motions across $5$~trials are shown.
}
\small{
\resizebox{0.9\linewidth}{!}{
    \begin{tabular}{ccccccccccccc}
        \toprule 
        \multirow{3}{*}{\textbf{\shortstack{Online\\Adaptation}}} & 
        \multirow{2}{*}{\textbf{\shortstack{Vertical\\Vel.}}} &
        \multirow{2}{*}{\textbf{\shortstack{Vertical\\Acc.}}} & 
        \multirow{2}{*}{\textbf{\shortstack{Roll\\Rate}}}&
        \multirow{2}{*}{\textbf{\shortstack{Pitch\\Rate}}}&
        \multirow{2}{*}{\textbf{\shortstack{Roll\\Acc.}}}&
        \multirow{2}{*}{\textbf{\shortstack{Pitch\\Acc.}}}\\
        &&&&&&& \\
        &{[m/s]}&
        {[m/s$^\text{2}$]}&
        {[rad/s]}&
        {[rad/s]}&
        {[rad/s$^\text{2}$]}&
        {[rad/s$^\text{2}$]} \\
        \midrule
        \xmark & 0.3319 & 2.7385 & 0.2563 & 2.1254 & 2.9504 & 2.7333 \\
        \checkmark & \textbf{0.1910} & \textbf{1.4347} & \textbf{0.1409} & \textbf{1.9994} & \textbf{1.4919} & \textbf{1.6075}\\
        \bottomrule
    \end{tabular}
    }
}
\label{table:adapt}
\vspace*{-0.2in}
\end{table}

Fig.~\ref{fig:adapt_graph}\red{b} shows the vertical acceleration experienced by the vehicle during navigation. By beginning to online-adapt the network, the vehicle can reduce the impact exerted on it by adjusting its trajectory based on recent experiences, whereas the vehicle that does not conduct adaptation continues to experience enormous impacts due to the inability to overcome uncertainty. The navigational stability measured by averaging the vertical and angular motions of the vehicle is presented in Table~\ref{table:adapt}. It verifies that adaptation can induce stable vehicle motions by adjusting the traversability prediction model to incorporate experiences with unknown terrains.

% \begin{wrapfigure}{R}{0.5\linewidth}
% \vspace*{-0.1in}
% \centering
% % Figure removed
% \caption{Vertical acceleration of the vehicle during navigation.}
% \label{fig:adapt_graph}
% \vspace*{-0.2in}
% \end{wrapfigure}





