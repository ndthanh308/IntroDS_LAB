% Figure environment removed


An off-road environment (see Fig.~\ref{fig:driving}) is designed to conduct navigation~(\textbf{Q2}) where challenging unstructured terrains with large and irregularly patterned bumps make navigation solely with classification-based traversability maps inadequate. The control vehicle is a Volvo XC90, which is distinct from the vehicle used for training data collection. To evaluate the efficacy of our global model in enhancing navigation performance, navigation is performed with our method~(\textit{METAVerse}) with and without online adaptation. Experiments are also conducted with a self-supervised method that predicts point-wise traversability~(\textit{Point-wise})~\cite{seo2023scate} instead of generating a dense map. In addition, rule-based methods are compared, including an elevation-map based method~(\textit{Elevation Based})~\cite{fankhauser2014robot,miki2022elevation} and a slope-based method~(\textit{Slope Based})~\cite{sock2016probabilistic,kim2017traversable}. We conduct navigation $15$~times for each method. 

The trajectories taken during navigation are shown in Fig.~\ref{fig:driving}, and the statistics about angular and vertical motions of the vehicle and success rates of navigations are shown in Table~\ref{tab:results}. The navigational performance of \textit{METAVerse} employing online adaptation is superior to that of our model without the adaptation. The traversability network is adjusted online to the novel environment and vehicle of deployment, resulting in improved navigational performance.

In addition, our method with voxelization to efficiently construct a dense cost map improves navigation performance compared to the point-wise self-supervised method, which predicts traversability cost using a point-wise network and generates local traversability maps via interpolation~\cite{seo2023scate}. Due to real-time constraints, the network for point-wise prediction becomes shallower than the voxel-based network, which is sufficiently deep to generate a continuous cost map in BEV. The efficient network structure that directly generates maps in BEV enables our model to embed richer information to predict traversability more precisely in a limited amount of time, thereby enhancing real-time navigational performance. Lastly, the inability of the rule-based methods to reason about nuanced interactions with unstructured terrain leads to instability and an increase in navigational failure rates. On the other hand, the vehicle utilizing a self-supervised traversability map effectively navigates along paths that minimize disturbances. Our method can effectively identify risky regions where terrain adversely impacts vehicle stability.

