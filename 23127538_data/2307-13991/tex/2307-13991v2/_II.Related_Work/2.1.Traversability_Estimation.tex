Earlier works estimate traversability based on rule-based features derived from geometric and visual appearances, such as the terrain's roughness and slope~\cite{sock2016probabilistic, ahtiainen2017normal, kim2017traversable}. With the advent of deep neural networks, semantic segmentation has been widely utilized to classify off-road terrains according to their navigability levels~\cite{contrastive_offroad, guan2022ga}. Recent work~\cite{shaban2022semantic} has utilized semantic scene completion to generate a dense terrain classification map in bird’s eye view~(BEV) from a sparse point cloud~\cite{fei2021pillarsegnet}.
%The predictions are converted into a dense map in bird’s eye view (BEV) for path planning. 

Unfortunately, these human-supervised methods cannot provide adequate information for effective navigation in complex and unstructured off-road environments. The cost assigned to a predetermined terrain class would be irrelevant to a given environment or inaccurately depict navigability~\cite{frey2023fast}. Self-supervised approaches use terrain interaction feedback to circumvent such limitations~\cite{acoustic, wellhausen_2019should, gasparino2022wayfast}. Using information about the terrain traversed by a vehicle, they identify traversable regions or classify terrains into multiple classes to designate differential costs~\cite{kim2006traversability, seo2023learning, guan2023vinet}. Nevertheless, these approaches abstract away the subtle variations in traversability within the same class~\cite{hdif2023}.

%Interactions with terrains measured by proprioceptive sensors are used to label exteroceptive observations.

Recent research has shifted toward predicting a continuous-valued cost for more effective navigation in unstructured environments~\cite{yao2022rca, sathyamoorthy2022terrapn, karnan2023self, chen2023learning}. To define the continuous costs, inertial information pertinent to navigational stability is processed~\cite{hdif2023, yao2022rca, karnan2023self}, or a reinforcement learning framework is leveraged~\cite{zhu2020off, weerakoon2022terp, frey2022locomotion}. By incorporating the local cost map into path planning and control, these methods improve navigational performance in terms of stability and safety. Nevertheless, these methods prioritize distinguishing between terrain types over evaluating subtle differences in traversability among terrains of the same class. Also, these approaches cannot account for terrains with unknown properties~\cite{seo2023learning}. While recent works detect and avoid unobserved terrains~\cite{cai2022probabilistic, seo2023scate}, they cannot obtain a global model that adapts or generalizes to numerous environments.