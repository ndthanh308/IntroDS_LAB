The goal of meta-learning is to train a model that can rapidly adapt to new tasks~\cite{hospedales2021meta}. Model-agnostic meta-learning (MAML)~\cite{finn2017model} obtains the initial parameters of a neural network such that taking a few gradient descent steps from this initialization results in effective generalization. By considering environments or situations with distinct characteristics as different tasks, meta-learning can learn a global initial parameter from heterogeneous datasets without confusion. During inference, the global model is generalizable to numerous tasks, including a novel one, through a few gradient updates~\cite{li2018learning}.

Various works in robotics literature have adopted meta-learning to learn a global model that can reduce uncertainty through rapid adaptation~\cite{Wortsman_2019_CVPR}. Nagabandi et al.~\cite{nagabandi2018learning} trained the dynamics model of a legged robot, which rapidly adapts to its local environment. Recently, Visca et al.~\cite{visca2022deep} proposed a meta-adaptive energy predictor for path planning in unknown terrains. 

% Figure environment removed