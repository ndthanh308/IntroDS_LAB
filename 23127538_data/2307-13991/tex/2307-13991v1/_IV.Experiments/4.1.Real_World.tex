\newcommand{\cmark}{\ding{51}}%
\newcommand{\xmark}{\ding{55}}%

\begin{table}[t]
\centering
\renewcommand {\arraystretch}{1.1}
\caption{The sequence length in time~(\textit{Total Len}) and the standard deviations of the ground-truth traversability costs~(\textit{STDEV of Cost}) are displayed to illustrate the details of each evaluation dataset category.}
\tiny
\label{tab:eval_dataset_sum}
\resizebox{0.9\linewidth}{!}{%
    \begin{tabular}{ccccc}
        \toprule
        & \bf{Unpaved} & \bf{Grassland} & \bf{Profiled Road} & \bf{Simulation} \\
        \midrule
        \multicolumn{1}{c|}{\textit{Total Len} (s)} &  531.5 & 390.2 & 1056.9 & 545.9 \\
        \multicolumn{1}{c|}{\textit{STDEV of Cost}}&  0.4751 & 0.5166 & 0.4638 & 5.5651 \\
        \bottomrule
\end{tabular}%(w.o. Adaptation)
}
\vspace*{-0.2in}
\end{table}


We validate the efficacy of our meta-learning method for traversability cost prediction~(\textbf{Q1}) with real-world off-road driving data. Specifically, the validity of the global traversability model trained with meta-objective~(\ref{eq:meta}) is evaluated in terms of prediction accuracy. 

\subsubsection{Experimental Setup}
Driving data is collected in off-road environments with various types of terrain using our platform equipped with OS1-128 LiDAR and IMU~(See Fig.~\ref{fig:concept}). Also, the driving data is obtained in a simulation environment consisting of randomly patterned rough terrain and bumps, which is hazardous to interact with in real-world environments~\cite{seo2023scate}. Approximately three hours of driving data are utilized to train the network. Then, the trained traversability prediction network is evaluated using an evaluation dataset consisting of separate trajectory sequences not included in the training dataset. 

According to the terrain characteristics, we divide our real-world evaluation dataset into $3$ categories, namely, \textit{unpaved}, \textit{grassland}, and \textit{profiled road}. Fig.~\ref{fig:real_data} shows example images of these terrains along with their corresponding visualizations of traversability cost maps. The \textit{unpaved} consists of rough and unpaved dirt tracks with numerous bumps and puddles. The \textit{grassland} represents grassy roads with vegetation, bushes, and cobblestones. The \textit{profiled road} is composed of roads with five different types of artificial road profiles of varying sizes that are used for assessing driving stability. Lastly, \textit{simulation} category is added for the evaluation, which refers to data collected while navigating the off-road evaluation track in the simulation setup~(See Fig.~\ref{fig:driving}). The details of our evaluation dataset are presented in Table.~\ref{tab:eval_dataset_sum}.

For comparison, a network is trained using all train data without meta-objective and adaptation~(\textit{Baseline}). Also, the network trained with meta-objective is evaluated with zero adaptation steps~($N_A$=0) during inference as well as with varying numbers of adaptation steps. Mean Square Error~(MSE) between the ground-truth traversability and the predicted traversability cost of corresponding grids is calculated for the evaluation.

% Figure environment removed



\begin{table}[b]
% \vspace*{-0.1in}
\centering
\renewcommand {\arraystretch}{1.3}
\caption{Validation error of the experiment with real-world driving data. Our method shows a significant margin compared to the baseline, and it can predict traversability more accurately by conducting adaptations during inference.}
\scriptsize
\label{tab:quantitative}
\resizebox{1.0\linewidth}{!}{%
    \begin{tabular}{cccccc}
        \toprule
        \multirow{2}{*}{\textbf{Method}} & \multirow{2}{*}{\textbf{Adaptation}}& \multicolumn{4}{c}{\bf{Evaluation Dataset Category}}\\ 
        \cmidrule(l{4pt}r{4pt}){3-6} 
        & & \bf{Unpaved} & \bf{Grassland} & \bf{Profiled Road} & \bf{Simulation} \\
        \midrule
        \multicolumn{1}{c}{\textit{Baseline}} & \xmark & 0.1222 & 0.2460 & 0.2039 & 0.7263 \\
        \multicolumn{1}{c}{\textit{METAVerse}}& \xmark & 0.0713 & 0.1961 & 0.1907  & 0.6668 \\
        \multicolumn{1}{c}{\textit{METAVerse}}& \checkmark & \textbf{0.0114} & \textbf{0.1767} & \textbf{0.1523} & \textbf{0.5725} \\
        \bottomrule
\end{tabular}%(w.o. Adaptation)
}
% \vspace*{-0.1in}
\end{table}

\subsubsection{Experimental Result}
The quantitative results are presented in Table~\ref{tab:quantitative}. In every category, the model trained with the meta-objective outperforms the baseline model trained without the meta-objective. It indicates that the non-meta-learned baseline failed to converge well due to high aleatoric uncertainty in ground-truth traversability collected across various terrains. In contrast, our meta-learned model can converge well by incorporating such uncertainty during training and adapting the model during inference. Even without adaptation in the evaluation phase, our method outperforms the baseline, implying that utilizing the meta-objective leads to a finding of a better initial parameter~\cite{finn2017model}. In addition, the model's performance improves as it adapts during inference using recent interaction experiences. 



Fig.~\ref{fig:real_adapt_graph} shows the experimental results on the whole evaluation data with varying numbers of adaptation steps during inference. The accuracy improves as the number of adaptation steps increases. Also, our method with meta-objective shows a significant margin over the baseline that does not conduct adaptation during both training and inference. After adaptation, the initially erroneous cost map is adjusted to more accurately represent the cost of terrains by incorporating recent experiences, as illustrated in Fig.~\ref{fig:real_adapt_vis}.




% Figure environment removed