\begin{table*}[t!]
\centering
\scriptsize
\renewcommand {\arraystretch}{1.2}
\caption{Quantitative results for the navigation experiments (\textbf{Q2}). The average and maximum motions of the vehicle across all trials are shown. %The results indicate that our traversability map can facilitate stable off-road navigation.
}
\resizebox{1.0\textwidth}{!}{
\label{tab:results}
    \begin{tabular}{cccccccccccccccc}
        \toprule
        \multirow{2}{*}{\textbf{Method}} & \multirow{2}{*}{\textbf{\shortstack{Self\\Supervised}}} & \multirow{2}{*}{\textbf{\shortstack{Online\\Adaptation}}} & \multirow{2}{*}{\textbf{Success Rate}} & \multicolumn{2}{c}{\bf{Vertical Vel. [m/s]}} & \multicolumn{2}{c}{\bf{Vertical Acc. [m/s$^\text{2}$]}} & \multicolumn{2}{c}{\bf{Roll Rate [rad/s]}} &
        \multicolumn{2}{c}{\bf{Pitch Rate [rad/s]}} &
        \multicolumn{2}{c}{\bf{Roll Acc. [rad/s$^\text{2}$]}} &
        \multicolumn{2}{c}{\bf{Pitch Acc. [rad/s$^\text{2}$]}} \\
        \cmidrule(l{4pt}r{4pt}){5-6} \cmidrule(l{4pt}r{4pt}){7-8} \cmidrule(l{4pt}r{4pt}){9-10} \cmidrule(l{4pt}r{4pt}){11-12} \cmidrule(l{4pt}r{4pt}){13-14} \cmidrule(l{4pt}r{4pt}){15-16} & & & &\textbf{Mean} & \textbf{Max} & \textbf{Mean} & \textbf{Max} & \textbf{Mean} & \textbf{Max} & \textbf{Mean} & \textbf{Max} & \textbf{Mean} & \textbf{Max} & \textbf{Mean} & \textbf{Max}\\
        \midrule
        \textit{Elevation Based} & \xmark & \xmark & 0.46 %7 / 15 
        & 0.1522 & 2.7664 & 1.2543 & 40.1081 & 0.1433 & 0.1516 & 1.4675 & 2.3831 & 1.2877 & 54.7836 & 1.1770 & 34.0290\\
        \textit{Slope Based} & \xmark & \xmark & 0.53 %8 / 15 
        & 0.1286 & 1.8467 & 0.9994 & 30.4818 & 0.1288 & 1.6243 & 0.1273 & 1.3686 & 1.0184 & 38.9109 & 0.9531 & 26.8223\\
        \textit{Point-wise} & \checkmark &\xmark & 0.80 %12 / 15 
        & 0.1091 & 1.7433 & 0.8862 & 26.3765 & \textbf{0.1007} & 1.4877 & 0.1180 & 1.4961 & \textbf{0.8528} & 36.2358 & 0.8516 & 23.7459 \\
        \textit{METAVerse} & \checkmark & \xmark  & 0.93 %14 / 15 
        & 0.1094 & 1.6901 &  0.8684 & 27.9480 & 0.1071 & 1.6461 & 0.1180 & 1.5383 & 0.8990 & 36.0222 & 0.8780 & 27.0468\\
        \textit{METAVerse} & \checkmark  & \checkmark & \textbf{1.00}%\textbf{15 / 15} 
        & \bf{0.1064} & \textbf{1.5913} & \textbf{0.8542} & \textbf{25.6867} & 0.1038 & \textbf{1.4582} & \textbf{0.1134} & \textbf{1.4349} & 0.8926 & \textbf{32.2179} & \textbf{0.8495} & \textbf{22.0796}\\
        \bottomrule
    \end{tabular}
    }
\vspace*{-0.2in}
\end{table*}




% Figure environment removed

An off-road environment (see Fig.~\ref{fig:driving}) is designed to conduct navigation~(\textbf{Q2}) where challenging unstructured terrains with large and irregularly patterned bumps make navigation solely with classification-based traversability maps inadequate. The control vehicle is a Volvo XC90, which is distinct from the vehicle used for training data collection. Experiments are conducted using traversability maps that generate continuous cost maps, including an elevation-map based method~(\textit{Elevation Based})~\cite{fankhauser2014robot,miki2022elevation}, a slope-based method~(\textit{Slope Based})~\cite{sock2016probabilistic,kim2017traversable}, and a self-supervised method that predicts point-wise traversability~(\textit{Point-wise})~\cite{seo2023scate}. Also, navigation is performed with our method~(\textit{METAVerse}) with and without online adaptation. We conduct navigation $15$~times for each method.



% 실제 경험을 토대로 최적화된 반응과 위험(불안정)을 예측하여 이를 피해간다
% 1. 반면, rule 기반은 복잡한 interaction이 고려되지 않아서 unstructured terrain과의 반응을 제대로 이해할 수 없다. 따라서 failure rate가 높고, impact가 높아진다(수치상), 불안정한 navigation이다.
% 2. 또한 online-adaptation을 켜면 global model에서 simulation 환경 및 차량에 더 최적화된 costmap을 생성할 수 있고, 이는 더 안정적인 주행으로 이루어진다.
% 3. 마지막으로 point-wise로 예측하는 것보다도 성능이 좋았다. point단위로 예측해서 interpolation및 accumulation하는 것은 real-time issue 때문에 network가 얕아지는 반면, 우리의 bev map 기반은 네트워크를 더 깊게 구성할 수 있어서 성능이 더 좋다.

The trajectories taken during navigation are shown in Fig.~\ref{fig:driving}. Also, the statistics about angular and vertical motions of the vehicle and success rates of navigations are shown in Table~\ref{tab:results} for quantitative evaluation. Compared to other rule-based methods, the vehicle using a self-supervised traversability map navigates along paths that minimize disturbances. Our method can identify risky regions where terrain can adversely impact vehicle stability. The inability of rule-based methods to reason about nuanced interactions with unstructured terrain, on the other hand, leads to instability and an increase in navigational failure rates.

In addition, the navigational performance of \textit{METAVerse} employing online adaptation is superior to that of our model without the adaptation. The traversability network is adjusted online to the environment and vehicle of deployment, resulting in improved navigational performance. Lastly, our method, which employs voxelization to construct a dense cost map efficiently, improves navigation performance by providing richer information than the self-supervised method, which predicts traversability cost using a point-wise network and generates local traversability maps through interpolation~\cite{seo2023scate}. Due to real-time constraints, the network for point-wise prediction becomes shallower than the voxel-based network, which is sufficiently deep to generate a continuous cost map in BEV. The efficient network structure that directly generates maps in BEV enables our model to predict traversability more precisely in a limited amount of time, thereby enhancing real-time navigational performance.
 % and measure the navigation failure rate