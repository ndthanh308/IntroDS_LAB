% Motivation
Off-road environments are fraught with bumps and obstacles of varying shapes, despite being in the same terrain class. For safe and effective navigation in off-road terrain, estimating the nuanced traverse cost of the terrain is necessary. A path planner can optimize a trajectory that minimizes disturbances during navigation with the predicted cost. Therefore, we generate dense and continuous traverse cost maps in BEV from a single LiDAR point cloud.

% data generation
The z-axis linear acceleration measured by an IMU is utilized to define traversability cost derived from vehicle-terrain interaction. This component effectively captures the terrain's bumpiness related to the stability of the vehicle in off-road navigation~\cite{Bekhti_verticala}. In addition, the definition of traversability based on vertical acceleration can be advantageous for control performance because model-based controllers frequently employ a vehicle dynamics model ignorant of vertical motions for computational simplicity~\cite{kim_smooth_2022, kim2023bridging}.

Motivated by recent work~\cite{hdif2023}, we define traversability cost using the spectral analysis of z-axis linear acceleration. The wavelet power spectrum is used to precisely characterize the costs of a time series signal, as it eliminates the need to segment signals and apply Fourier transform to each segment. A continuous wavelet transformation with the Morlet wavelet is performed on the z-acceleration, $a_z(t)$, to generate the wavelet coefficient $w_z(f_n, t)$ for each frequency scale $f_n = 2^{n} \cdot f_0$ and time step $t$. Then, the traversability cost $c_t$ is defined by the wavelet power spectrum as follows:
\begin{equation}
    {c}_t = \sum_{n=0}^{j} \frac{\|w_z(f_n, t)\|^2}{f_n},  
\end{equation} where squares of the coefficient are divided by frequency scale to rectify the power spectrum, as suggested by Liu et al.~\cite{liu2007rectification}, and summed over a certain frequency scale range of $f_0=0.16$ to $f_j=5.12$. The calculated ground-truth costs are then assigned to BEV grids along the positions of the trajectory and used for training the traversability cost prediction network.

% base network structure
The traversability cost prediction network is trained to produce a dense top-view cost map using a sparse single-sweep LiDAR point cloud, as shown in Fig.~\ref{fig:pipeline}. Following PointPillars~\cite{lang2019pointpillars}, each point is discretized into sparse pillars. The point in each pillar is encoded as a $4$-dimensional feature consisting of offset from the pillar center and distance from origin $(\Delta x, \Delta y, \Delta z, d)$. Using a simplified PointNet~\cite{qi2017pointnet} that consists of a linear layer, BatchNorm, and ReLU, each pillar of size $(N, 4)$ is converted into sparse pillar features of size $C$, where $N$ is the maximum number of points per pillar. Each pillar feature is scattered back to the pillar locations to create a BEV sparse feature representation of size $(H, W, C)$, where $H$ and $W$ denote the width and height of the grid, respectively. The empty pillars are zero-initialized. A U-Net~\cite{unet} structured network, which has an encoder-decoder architecture with skip connections, is employed to generate a dense pillar feature map of size $D$. It progressively reduces the spatial size of features and captures higher-level semantic information while the decoder upsamples feature maps to recover spatial information.

The dense pillar features are concatenated with parameterized velocity to produce velocity-conditioned cost maps. Fourier feature mapping~\cite{tancik2020fourier} is used to incorporate the vehicle's velocity into the cost prediction~\cite{hdif2023}. The velocity vector is mapped into a higher dimensional representation:
\begin{equation} \label{eq:ffm}
    \begin{split}
    \gamma(v) = [\cos(2\pi b_1 v), &\sin(2\pi b_1 v), \dots, \\
    &\cos(2\pi b_P v), \sin(2\pi b_P v)],
    \end{split}
\end{equation} where $v$ is the norm of the velocity vector, $b_i \sim \mathcal{N}(0, 5^2)$ are sampled from a Gaussian distribution, and $P=10$ is the number of samples. Finally, the MLP head predicts the mean ${\mu}_i$ and standard deviation ${\sigma}_i$ of the traversability for each pillar $i$. The network is trained to minimize the Gaussian log-likelihood:
\begin{equation}\label{gaussiannll}
    \mathcal{L}^{\text{traverse}}\left(\tau, \bm{\theta}\right) = \frac{1}{2}\sum_i \left(\log({\sigma}_i) +  \frac{({\mu}_i - c_i)^2}{{\sigma}_i} \right),
\end{equation} where $\bm{\theta}$, $\tau$, and $c_i$ represent the model parameter, driving data along a trajectory segment, and the ground truth cost associated with the pillar $i$, respectively. The loss calculation is restricted to pillars assigned with ground truth, that is, the vehicle has traversed. Multiple data augmentations, such as random flip, rotation, and translation, are implemented during training to prevent overfitting and produce a dense cost map.