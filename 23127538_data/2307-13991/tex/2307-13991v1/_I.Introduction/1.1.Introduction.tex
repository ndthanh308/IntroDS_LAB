% 1. Traversability Estimation in Off-Road and its difficulty
During off-road navigation, autonomous vehicles encounter diverse unstructured terrains with distinct characteristics. To ensure safe and stable navigation in off-road environments, it is necessary to predict the interaction of a vehicle with terrains in an upcoming trajectory~\cite{borges2022survey}. The predicted difficulty of interaction, which represents the traversability of the terrain, plays critical roles in various navigational strategies such as local path planning and control~\cite{fan2021learning, frey2022locomotion}. However, estimating terrain traversability accurately in off-road environments with limited sensors is challenging. Even though human-annotated datasets can be utilized to train a semantic classifier, off-road environments are fraught with unseen and ambiguous terrains, resulting in inaccurate predictions of traversability. Furthermore, terrains of the same terrain class may have varying degrees of traversability due to their complex and variable traversability-related properties, which cannot be captured through manually labeled data.

% 2. Emergence of Self-Supervised Traversability Estimation
Recently, off-road navigation has benefited significantly from self-supervised approaches that utilize vehicle-terrain interaction of actual navigation experiences to learn terrain traversability~\cite{kim2006traversability, acoustic, wellhausen_2019should}. These methods learn a mapping from exteroceptive data (e.g., RGB and LiDAR point clouds) to the traversability cost defined by the vehicle-terrain interaction measured with proprioceptive sensors (e.g., Inertial Measurement Unit~(IMU)). The resultant traversability cost maps with continuous-valued scores can precisely represent the difficulty of navigation. Moreover, these cost maps reflect the navigation capabilities of a vehicle and the terrain characteristics based on actual experiences, resulting in enhanced navigational performance.

% 3. Shortcomings of Self-supervised methods: generalizability.
However, these methods cannot be generalized to various environments since it is challenging to acquire a global model that operates reliably in numerous environments. As the interaction data only provides supervision for regions the vehicle has interacted with, the model's predictions for unexplored terrains are subject to high epistemic uncertainty~\cite{seo2023scate}. Even if interaction data are obtained on a variety of terrains to reduce epistemic uncertainty, the estimation of the model is still subject to a substantial amount of aleatoric uncertainty. In real-world off-road environments, the traversability of the terrain is intricately influenced by a multitude of interconnected and complex factors (e.g., platform, geometry, deformability, bumpiness, friction, and roughness)~\cite{hdif2023}. Nonetheless, such subtle variations cannot be captured precisely with a limited sensor configuration. While the geometric characteristics of the terrain captured by a sensor would be comparable, its ground-truth traversability would vary considerably, resulting in a high level of aleatoric uncertainty. These uncertainties result in less precise predictions and make it impossible to estimate traversability costs that are more nuanced than simple terrain classification, resulting in suboptimal off-road navigational performances.

% Figure environment removed

% Contribution.
This paper presents a meta-learning-based framework for learning terrain traversability~(\textit{METAVerse}), which is capable of learning a global model that accurately predicts terrain traversability in various off-road environments. During training, the traversability cost prediction network is learned to generate a dense, continuous-valued cost map in real time from a single sweep LiDAR point clouds. To minimize the uncertainty of the network trained with data collected from various terrains, meta-learning is employed to discover initial network parameters that enable effective generalization with a few gradient descent steps. During deployment, the model quickly adapts to the local context with gradient descents based on the recent history of vehicle-terrain interactions, allowing for effective off-road navigation. We demonstrate that our method can learn a global model that reduces prediction uncertainty using real-world driving data collected on unstructured terrains with varying properties. In addition, by integrating our framework with a sampling-based model predictive controller, we demonstrate that our method facilitates stable and safe navigation in unstructured environments. An overview of our framework is presented in Fig.~\ref{fig:concept}.

In summary, the main contributions of our work are:
\begin{itemize}
    \item We introduce a deep meta-learning framework for learning a global terrain traversability prediction network that can reliably generate a dense cost map from a sparse LiDAR point cloud in various environments.
    \item We propose a method to minimize uncertainty in estimation by online adapting the network based on the vehicle's navigation experience.
    \item We demonstrate that our method can reduce uncertainty in prediction using data collected in various terrains.
    \item We validate that our method leads to stable navigation by integrating our traversability cost map with a sampling-based model predictive controller.
\end{itemize}


