\documentclass[prl,showpacs,twocolumn,amsmath,amssymb,floatfix,superscriptaddress]{revtex4-2}
\usepackage{amsmath}
\usepackage{amssymb}
\usepackage{graphicx}
\usepackage{bm}
\usepackage{color}
\usepackage{hyperref}
\usepackage{graphicx}
\usepackage{upgreek}
\usepackage{scalerel}
\usepackage{multirow}
 
\usepackage{pgffor}
\usepackage{pdfpages}
\usepackage{mathdots}
\usepackage{float}
\usepackage{silence}
\WarningFilter{revtex4-1}{Repair the float}
\usepackage{physics} 
\usepackage{lscape}   
\usepackage{color, colortbl}
\usepackage[table]{xcolor}
\usepackage{comment}  
\makeatletter
\AtBeginDocument{\let\LS@rot\@undefined}
\makeatother
\newcommand{\edit}[1] {\textcolor{black}{#1}}
\newcommand{\editJ}[1] {\textcolor{red}{#1}}



\begin{document}
\title{Non-Hermitian phase-biased Josephson junctions}
\author{Jorge Cayao}
\affiliation{Department of Physics and Astronomy, Uppsala University, Box 516, S-751 20 Uppsala, Sweden}
\author{Masatoshi Sato}
\affiliation{Yukawa Institute for Theoretical Physics, Kyoto University, Kyoto 606-8502, Japan}

 
\date{\today} 
\begin{abstract}
We study non-Hermitian Josephson junctions formed by  conventional superconductors with a finite phase difference under non-Hermiticity naturally appearing due  to coupling  to normal reservoirs. Depending on the structure of non-Hermiticity, captured here in terms of retarded self-energies, the low-energy spectrum hosts topologically stable exceptional points either at zero or finite real energies  as a function of the superconducting phase difference. Interestingly, we find that the corresponding phase-biased supercurrents acquire divergent profiles at such exceptional points, an instance that turns out to be a natural and unique non-Hermitian effect  that signals a possible way to enhance the sensitivity of Josephson junctions. Our work thus opens the way for realizing unique non-Hermitian phenomena due to the interplay between non-Hermitian topology and the Josephson effect.
\end{abstract}
\maketitle
%%%%%%%%%%%%%%%%%%%%%%%%%%%%%%%
% SECTION 1:                 INTRODUCTION                                %
%%%%%%%%%%%%%%%%%%%%%%%%%%%%%%%
Non-Hermitian (NH) systems have recently attracted an enormous interest due to their potential for realizing unexpected states of matter with applications in several areas of physics that are impossible in the Hermitian realm \cite{el2018non,ozdemir2019parity,RevModPhys.93.015005,doi:10.1080/00018732.2021.1876991,OS23,wiersig2020review, parto2020non}. The majority of the intriguing NH properties stems from the fact that  NH systems host a complex spectrum with spectral degeneracies known as exceptional points (EPs) where eigenvalues and eigenvectors coalesce \cite{TKato, heiss2004exceptional, berry2004physics, Heiss_2012, PhysRevLett.86.787, PhysRevLett.103.134101, PhysRevLett.104.153601, gao2015observation, doppler2016dynamically,PhysRevB.99.121101}.  EPs have shown to be topological objects with topology defined in the complex spectrum that introduces the concept of point gaps unlike what occurs in Hermitian  systems \cite{SZF18, PhysRevX.9.041015,KBS19,ZSHC21}.   Another interesting property of EPs is that  the NH spectrum is very sensitive to perturbations, producing strong responses at EPs and offering a way for realizing non-Hermitian sensing devices with no Hermitian analog \cite{wiersig2020review,PhysRevLett.125.180403,arouca2022exceptionally,parto2023enhanced}.

NH physics has been shown to emerge in  open systems \cite{RevModPhys.93.015005,doi:10.1080/00018732.2021.1876991}, with already well-stablished concepts in photonics and optics where non-Hermiticity  has been experimentally shown to be controlled almost at will \cite{PhysRevLett.86.787,feng2014single,gao2015observation,doppler2016dynamically,peng2016chiral,hodaei2017enhanced,chen2017exceptional,Longhi:17}. Another less explored  but promising route is offered by material junctions where non-Hermiticity naturally arises when coupling the system to normal reservoirs or leads \cite{datta1997electronic}. These ideas have recently been explored using normal state systems \cite{PhysRevB.98.155430, PhysRevB.98.245130, PhysRevResearch.1.012003,cayao2023exceptional,NIOS23} but also limited examples exits in setups involving superconductors \cite{PhysRevB.87.235421,pikulin2013two,JorgeEPs,avila2019non,PhysRevB.105.094502,PhysRevB.107.104515}. While these initial studies have shed some light on the interplay  between non-Hermiticity and superconductivity, they have overlooked the impact of non-Hermiticity on the  phase of the superconducting order parameter. 

The simplest systems where the superconducting phase is crucial are Josephson junctions (JJs), commonly formed by two superconductors where their superconducting order  parameters exhibit a finite superconducting phase difference \cite{Tinkham}. This enables  the formation of Andreev bound states (ABS) and the transfer of Cooper pairs that gives rise to a dissipationaless supercurrent across the junction \cite{furusaki1991dc,PhysRevB.45.10563,Beenakker:92,Furusaki_1999,RevModPhys.76.411,sauls2018andreev}. Due to these unique superconducting properties, JJs are not only excellent devices for revealing the origin of the underlying superconducting state \cite{tiira17,PhysRevLett.121.047001,PhysRevX.9.011010,ren2019topological,Fornieri_2019,PhysRevLett.124.226801,PhysRevLett.125.116803,lutchyn2018majorana,prada2019andreev,frolov2019quest,flensberg2021engineered} but they are also  the  most relevant superconducting building blocks with great promise for future quantum  applications \cite{PhysRevLett.89.117901,devoret2005implementing,PhysRevLett.90.226806,PhysRevB.81.144519,sarma2015majorana,acin2018quantum,krantz2019quantum,aguado2020perspective,benito2020hybrid,aguado2020majorana,beenakker2019search,PRXQuantum.2.040204,siddiqi2021engineering,bargerbos2022singlet,doi:10.1146/annurev-conmatphys-031119-050605,pita2023direct}. Thus, even though JJs offer a natural ground for NH physics, it is still unknown the response of ABSs and supercurrents to non-Hermiticity, specially, to the presence of EPs.  

In this work we consider NH phase-biased JJs formed by conventional superconductors coupled to normal reservoirs are depicted in Fig.\,\ref{Fig1} and demonstrate the formation and control of EPs  by means of the superconducting phase difference. In particular, we find that ABSs are strongly sensitive to the amount of non-Hermiticity and enable the emergence of EPs at  finite or zero-real energies  in a well-controlled manner.   Interestingly, we find that the phase-dependent supercurrents exhibit sharp enhanced profiles at  EPs but only the EPs at zero-real energy produce sizeable signals in the total supercurrents. Finally, we provide symmetry arguments showing the stability of the obtained EPs    in phase-biased JJs.
 

    % Figure environment removed

  
%%%%%%%%%%%%%%%%%%%%%%%%%%%%%%%
% SECTION 2:                            MODEL:                              %
%%%%%%%%%%%%%%%%%%%%%%%%%%%%%%%
\textit{NH Josephson junctions}.---We consider JJs under non-Hermiticity induced by normal leads, as schematically shown in Fig.\,\ref{Fig1}. We model these junctions by an effective Hamiltonian given by
\begin{equation}
\label{Heff}
H_{\rm eff}=H_{\rm JJ}+\Sigma^{r}(\omega=0)\,,
\end{equation}
where $H_{\rm JJ}$ describes the Hermitian JJ with superconductors (Ss)  in orange having a finite phase difference ($\phi$) between their order parameters, while $\Sigma^{r}(\omega=0)$ represents the retarded zero-frequency self-energy which captures the effect of the normal (ferromagnet) leads and renders the JJ non-Hermitian. In the Hermitian regime,  the spectrum   of the JJs described by  $H_{\rm JJ}$ become strongly dependent on  $\phi$ and reveals the formation of ABSs within the induced superconducting gap, thus producing a dissipationless supercurrent $I(\phi)$ \footnote{ Hermitian JJs formed by conventional superconductors have supercurrents behaving as $I(\phi)\sim{\rm sin(\phi)}$ but more exotic ABSs  have the potential to induce anomalous supercurrent profiles beyond sine-like curves \cite{kashiwaya2000tunnelling}.}.    As mentioned, non-Hermiticity arises from $\Sigma^{r}(\omega=0)$:  under general circumstances, the self-energy can exhibit both real (Re) and imaginary (Im) parts $\Sigma^{r}(\omega=0)={\rm Re}\Sigma^{r}-i{\rm Im}\Sigma^{r}$.  While the Re part only causes a shift in the energies of the closed system $H_{\rm JJ}$, the Im part makes the effective Hamiltonian $H_{\rm eff}$ non-Hermitian producing
  unexpected changes in the spectrum that are uniquely associated to the non-Hermitian  mechanism.  Here we are interested on investigating the impact of non-Hermiticity originating from $\Sigma^{r}(\omega=0)$ on both the ABSs and phase-dependent supercurrents of NH phase-biased JJs.   

 
Before going further, we point out the relation between ABSs and supercurrents. The spectrum, and hence ABSs, can be obtained from ${\rm det}(\omega-H_{\rm eff})=0$, which corresponds to the poles of the retarded Green's function \cite{datta1997electronic,mahan2013many,zagoskin},  $G^{r}(\omega)=(\omega-H_{\rm eff})^{-1}$, where $H_{\rm eff}$  is given by Eq.\,(\ref{Heff}). Due to the non-Hermiticity induced by the self-energy, the spectrum is expected to become complex with eigenvalues coming in pairs due to particle-hole symmetry intrinsically present in superconducting systems \cite{PhysRevB.87.235421,pikulin2013two,JorgeEPs,avila2019non,PhysRevB.105.094502,PhysRevB.107.104515}, namely, ($E_{n},-E_{n}^{*}$), $E_{n} (\phi)=  {\rm Re}E_{n}(\phi)-i{\rm Im}E_{n}(\phi)$. The Re part represents the quasiparticle energy  while the Im part characterizes its lifetime in the superconductor \cite{datta1997electronic}. Thus, in the same spirit as in the Hermitian regime where the phase dependent quasiparticle energies determine the supercurrent \cite{furusaki1991dc,PhysRevB.45.10563,Beenakker:92},  it is natural here to define a supercurrent associated to the complex phase-dependent spectrum of NH JJs as \cite{PhysRevX.8.031079} $I=-(e/h)\sum_{n>0}dE_{n}(\phi)/d\phi$, which leads to
\begin{equation}
\label{INH}
I(\phi)=-\frac{e}{h}\sum_{n>0}\left[\frac{d {\rm Re}E_{n}(\phi)}{d \phi}-i  \frac{d [{\rm Im}E_{n}(\phi)]}{d \phi}\right]\,,
\end{equation}
where the first (second) term in square brackets correspond to the contribution from the Re (Im) part of the Andreev spectrum to the supercurrent. Since the Re part corresponds to the physical quasiparticle energy, the Re part of  $I(\phi)$ can be understood as the quasiparticle supercurrent. For interpretation purposes, however, the Im part of $I(\phi)$ is important as the inverse of ${\rm Im} E_{n}$ represents the quasiparticle lifetime in the superconductor \cite{datta1997electronic}.  We also point out that superconductors forming JJs can be coupled directly or there can be a mediating (normal, insulating, or superconducting) region, as shown in Fig.\,\ref{Fig1}(a,b): when the length of this mediating region is shorter than the superconducting coherence length, only a pair of ABSs appears within the gap  which then fully determine $I(\phi)$ \cite{Beenakker:92}. Below, we consider short JJs with non-Hermiticity arising from coupling to normal/ferromagnet leads as in Fig.\,\ref{Fig1}(a,b) and then investigate  how EPs emerge on the complex Andreev spectrum and how they impact the supercurrents.

\textit{NH JJs with superconductors coupled to different normal leads}.---To begin, we consider     NH JJs with single site superconductors (Ss)   coupled to distinct normal (N) leads [Fig.\,\ref{Fig1}]. This   NH JJs are modelled by Eq.\,(\ref{Heff}), with
 \begin{equation}
 \label{NHJJtoy1}
\begin{split}
H_{\rm JJ}^{(1)}&=
\begin{pmatrix}
H_{L}&V\\
V^{\dagger}&H_{R}
\end{pmatrix}\,,\\
\Sigma^{(1)}&={\rm diag}(\Sigma_{1},\Sigma_{2})\\
 \end{split}
\end{equation} 
 where $H_{\alpha}=\varepsilon_{\alpha}\tau_{z}+{\rm Re}(\Delta_{\alpha})\tau_{x}-{\rm Im}(\Delta_{\alpha})\tau_{x}$ the Hamiltonian of left (right) S $\alpha=L/R$ with   $s$-wave pair potential $\Delta_{\alpha}=\Delta{\rm e}^{i\phi_{\alpha}}$  and  superconducting phase $\phi_{\alpha}$, and $\varepsilon_{\alpha}$ is the onsite energy in $\alpha=L/R$, while $V=t\tau_{z}$ is the coupling between Ss. Moreover,  $\Sigma_{1(2)}=-i\Gamma_{1(2)}\tau_{0}$, with $\Gamma_{1(2)}$ denoting the couplings of left (right) S to the left (right) N lead and $\tau_{i}$   is the $i$-th Pauli matrix in Nambu  space.  Despite the simplicity, $H_{\rm JJ}^{(1)}$ captures key properties of recent experiments trying to engineer Kitaev chains with quantum dots, see e.g. Ref.\,\cite{dvir2023realization,bordin2023crossed}. Since we are interested in   the formation of EPs, next we obtain the eigenvalues which, at $\varepsilon_{\alpha}={0}$, $\phi_{L}=0$, $\phi_{R}=\phi$,   read   
\begin{equation}
\label{Eval1}
E_{j}=-i\Gamma\pm\sqrt{\delta^{2}-\gamma^{2}\pm2\Delta\sqrt{t^{2}{\rm sin}^{2}(\phi/2)-\gamma^{2}}}
\end{equation}
where $\delta^{2}=t^{2}+\Delta^{2}$ and $j=1,2,3,4$ label the four eigenvalues, $\gamma=(\Gamma_{1}-\Gamma_{2})/2$, and $\Gamma=(\Gamma_{1}+\Gamma_{2})/2$.  To visualize $E_{j}$ in Fig.\,\ref{Fig2}(a) we plot the Re and Im eigenvalues as a function of $\phi$ with (blue and red curves) and without  (gray curves) non-Hermiticity. At vanishing non-Hermiticity, the spectrum of ABSs is gapless for the chosen parameters. Interestingly,  for a finite amount of non-Hermiticity,  with $\gamma\neq0$, the two positive and negative eigenvalues coalesce, with their real parts merging at finite Re energies. We have verified that the eigenfunctions also  coalesce at $\phi^{(1)}_{\rm EP}$ where eigenvalues merge, thus signalling the formation of EPs, whose positions correspond to the ends of the shaded yellow region in  Fig.\,\ref{Fig2}(a). By inspecting Eq.\,(\ref{Eval1}), we realize that such EPs form when $t^{2}{\rm sin}^{2}(\phi/2)=\gamma^{2}$, which allows us to find the values of $\phi$ at which   EPs form, namely, at $\phi^{(1)}_{\rm EP}=\pm2{\rm arcsin}(|\gamma|/|t|)+2\pi n$, where $n\in\mathbb{Z}$. We note that $\phi_{\rm EP}^{(1)}$ appear around $2n\pi$, $n=0,1,\cdots$\, for small $\gamma$.
   The tunability of these EPs and be further seen in Fig.\,\ref{Fig2}(b) where we plot the  the Re part of the difference between the two positive eigenvalues (${\rm Re}E_{\rm ee}$)  as a function of $\Gamma_{1}$ and $\phi$. In this case, the blue region indicates ${\rm Re}E_{\rm ee}=0$ with its border marking the position of EPs, which are purely emergent and controlled by virtue of $\phi$. We also point out  that the EPs found here are protected  by a NH topological number, as we demonstrate in the  Supplementary Material (SM) \cite{SM}.


    % Figure environment removed

It is now natural to ask about the current-phase curves associated to phase-dependent Andreev spectrum obtained in Eq.\,(\ref{Eval1}).  To answer this question, in Fig.\,\ref{Fig2}(c,d) we present the supercurrents without  and with non-Hermiticity as a function of $\phi$ obtained by using Eq.\,(\ref{INH}). In the Hermitian regime, the supercurrent  exhibits a sawtooth profile at $\phi=\pi$ due to the   zero-energy crossing of ABSs at $\phi=\pi$, see gray curves in Fig.\,\ref{Fig2}(c,d). In the non-Hermitian regime, with $\gamma\neq0$, the Re and Im supercurrent of each individual ABS exhibits sharp divergent profiles at the EPs $\phi_{\rm EP}$, see red and blue curves in (a,b) and  cyan arrows. However, the divergent contributions of  the two positive (negative) ABSs cancels out when obtaining the total Re supercurrent, giving rise to a smooth curve across EPs with an overall sine-like profile, see green curve in Fig.\,\ref{Fig2}(c); note that for the same reason, the total Im supercurrent acquires vanishing values. Nevertheless, it is worth noting that, even though no EP signatures are seen in the total $I(\phi)$, its Hermitian sawtooth profile at $\phi=\pi$ gets smooth out  by non-Hermiticity.  In sum,  NH JJs can host highly tunable EPs at finite Re energies as a pure NH topological effect but the total supercurrent is largely insensitive to such EPs.

 
% Figure environment removed

 \textit{NH JJs with a middle N region coupled to a ferromagnet lead}.---Here we consider   spinful NH JJs with a  short mediating N region  coupled to a ferromagnet lead, as in Fig.\,\ref{Fig1}(b). For simplicity we restrict our analysis to  single site S and N regions. The NH  JJ   is then modelled by
\begin{equation}
\label{NH3}
\begin{split}
H_{\rm JJ}^{(2)}&=
\begin{pmatrix}
H_{L}&V&0\\
V&H_{N}&V\\
0&V&H_{R}
\end{pmatrix}\,,\\
\Sigma^{(2)}&={\rm diag}(0,\Sigma(\omega=0),0)
\end{split}
\end{equation}
where $H_{\alpha}=\epsilon_{\alpha}\tau_{z}+{\rm Re}(\Delta_{\alpha})\sigma_{y}\tau_{y}-{\rm Im}(\Delta_{\alpha})\sigma_{y}\tau_{x}$ the Hamiltonian of left (right) S $\alpha=L/R$ with   $s$-wave pair potential $\Delta_{\alpha}=\Delta{\rm e}^{i\phi_{\alpha}}$  and  superconducting phase $\phi_{\alpha}$. Moreover, $H_{\rm N}=\varepsilon_{\rm N}\tau_{z}$ describes the N region, $V=t\sigma_{0}\tau_{z}$ the hopping matrix between N and S. Here,   N   is coupled to the ferromagnet lead, with $\Sigma={\rm diag}(\Sigma^{r}_{\rm e},\Sigma^{r}_{\rm h})$ and $\Sigma^{r}_{\rm e,h}(\omega=0)=-i \Gamma \sigma_{0}-i\gamma \sigma_{z}$: here $\Gamma=(\Gamma_{\uparrow}+\Gamma_{\downarrow})/2$ and $\gamma=(\Gamma_{\uparrow}-\Gamma_{\downarrow})/2$, where $\Gamma_{\sigma}$ is the coupling of spin $\sigma$  to the ferromagnet lead. Note that the choice of non-Hermiticity here is different in comparison to the previous section, where the asymmetry originated from coupling distinct Ss to different N leads.  The eigenvalues  are not simple to obtain but it is still possible to explore the fact that they solutions of ${\rm det}(\omega-H_{\rm JJ}^{(2)}-\Sigma^{(2)})=0$. Thus, taking up to second order in $\omega$ to capture only the lowest two eigenvalues, we obtain for $\varepsilon_{\alpha}=0$, $\phi_{L}=0$, $\phi_{R}=\phi$,  
\begin{equation}
\label{Eval2}
E_{\pm}=-i\Gamma \frac{\Delta^{2}(2t^{2}+\Delta^{2})}{A}\pm\frac{\Delta}{A}\sqrt{C+2t^{4} A\,{\rm cos}\phi}\,,
\end{equation}
 where $A=4 t^4+\Delta ^4+2 \Delta ^2 \left(-\gamma ^2+\Gamma ^2+2 t^2\right)$, $C=4t^{4}[2t^{4}+(2t^{2}-2\gamma^{2}+\Gamma^{2})\Delta^{2}]+2[(t^{2}-\gamma^{2})^{2}+\Gamma^{2}(\Gamma^{2}-2\gamma^{2})]\Delta^{4}-\gamma^{2}\Delta^{6}$.  From Eq.\,(\ref{Eval2}) we notice that EPs here appear  with the Re parts merging at zero-energy    when the square root vanishes $C+2t^{4} A\,{\rm cos}\phi=0$, with   $|C|\leq2|t^{4} A|$ ensuring that the Re parts sticks to zero.  Thus, EPs appear at $\phi^{(2)}_{\rm EP}=\pm{\rm arcos}(-C/(2t^{4}A))+2\pi n$, where $n\in\mathbb{Z}$. For the visualization of  EPs, in  Fig.\,\ref{Fig3}(a) we show the Re and Im parts as a function $\phi$. While in the Hermitian regime (gray curves) the ABSs develop a zero-energy crossing for the chosen parameters, in the NH case they reveal the formation of EPs at zero-real energy and  with phases   $\phi^{(2)}_{\rm EP}$ around $(2n+1)\pi$, $n=0,1,\cdots$  marked by the ends of the shaded yellow region. These EPs are thus different to the seen in previous section which  have Re parts merging at finite values. In contrast to the previous section,  here the Re part sticks to zero energy between EPs, giving rise to zero-energy lines  as a unique NH effect that resembles the NH Bogoliubov Fermi arcs that here are fully controlled by $\phi$. The tunability of the EPs and zero-energy lines is further seen in Fig.\,\ref{Fig3}(b), which shows the Re part of the difference between the lowest positive and lowest negative eigenvalues as a function of $\Gamma_{\uparrow}$ and $\phi$. We also note that the  EPs obtained here are protected by a NH topological number, see SM \cite{SM}.
 
When it comes to the supercurrent   $I(\phi)$  due to ABSs in Eq.\,(\ref{Eval2}), Fig.\,\ref{Fig3}(c,d)   shows its Re and Im parts  as a function of $\phi$ for different $\gamma$  obtained by using Eq.\,(\ref{INH}). In the Hermitian case, $\gamma=0$, $I(\phi)$ exhibits a sawtooth profile at $\phi=\pi$ due to the zero-energy crossing in the ABSs, see gray curves. At  $\gamma\neq0$, the Re and Im parts of $I(\phi)$ exhibit divergences at the values of $\phi$ where the ABSs host EPs, see cyan arrows. At first sight, this   seems similar to $I(\phi)$ from  EPs at finite Re energy in Fig.\,\ref{Fig2}(c,d) but there is a key difference.  $I(\phi)$ here is entirely determined by one ABS which then suggests that the strong divergence at the EPs remains robust and cannot be wash out by the contribution of other energy levels, in contrast to the EPs at finite Re energy.  Moreover, the Re $I(\phi)$  vanishes between EPs as a result of the zero-energy lines in the Re ABSs, in contrast to what occurs for EPs at finite real energy  in previous section. The fact that $I(\phi)$  exhibits robust divergences at EPs fully controlled by $\phi$ demonstrates that this effect is purely of NH origin.

 
     % Figure environment removed

%%%%%%%%%%%%%%%%%%%%%%%%%%%%%%%
% SECTION 4:      Physical realization in Rashba nanowires      %
%%%%%%%%%%%%%%%%%%%%%%%%%%%%%%%
\textit{Physical realization in  superconductor-semiconductor hybrids}.---Having established  EPs and EP-enhanced supercurrents in simple NH phase-biased  JJs, here we demonstrate their realization in  NH Rashba JJs. The Hermitian system is modelled by $H_{\rm R}=\xi_{k}\tau_{z}+i\alpha k \sigma_{y}\tau_{z}+B\sigma_{x}\tau_{z}+\Delta\sigma_{y}\tau_{y}$, where $\xi_{k}=(\hbar^{2}k^{2}/2m-\mu)$ is the kinetic energy with $k$ along $x$, $\mu$ is the chemical potential, $\alpha$ is the Rashba spin-orbit coupling strength, and $\Delta$ characterizes the  spin-singlet $s$-wave pair potential. Also,  $B$ is the Zeeman field  from an applied magnetic field along the $x$-axis, inducing a (Hermitian) topological phase transition at $B_c=\pm \sqrt{\Delta^{2}+\mu^{2}}$ that predicts Majorana bound states (MBSs) at the ends of the system \cite{PhysRevLett.105.077001,PhysRevLett.105.177002}, see also  \cite{sato2017topological,Aguadoreview17,prada2019andreev,lutchyn2018majorana,frolov2019quest,beenakker2019search,flensberg2021engineered}. For computational purposes, $H_{\rm R}$ is discretized into  a tight-binding lattice with spacing $a=10$\,nm  \cite{cayao2018andreev} and then divided into three regions of finite length forming a JJ: The left and right S regions host a  finite pair potential $\Delta$ with a finite phase difference $\phi$, while the central N region has $\Delta=0$. Moreover, each region has length $L_{\rm N, S}$ and finite chemical potential $\mu_{\rm N, S}$. We further consider realistic system parameters: $\alpha_{\rm R}=40$\,meVnm and $\Delta=0.5$\,meV, which are in the range of experimental values reported for InSb and InAs nanowires, and Nb and Al Ss \cite{lutchyn2018majorana}. We finally model a short NH JJ by coupling the N region  to a ferromagnet lead via a  self-energy as in Eq.\,(\ref{NH3}) [Fig.\,\ref{Fig1}(b)], and obtain the Andreev spectrum and supercurrents.
  
In Fig.\,\ref{Fig4} we present the Re Andreev spectrum and Re supercurrents as a function of $\phi$ demonstrating the formation of EPs. For $B<B_{\rm c}$ in the Hermitian regime the JJ develops zero-energy crossings   which evolve into stable zero-energy lines at finite non-Hermiticity $\gamma\neq0$ whose ends mark the presence of EPs, see gray and blue curves and also the shaded yellow region in Fig.\,\ref{Fig4}(a). These EPs do not depend on the length of the S regions, which occurs because there are not MBSs in this regime.  Above $B_{\rm c}$, the Hermitian JJ hosts four MBSs at $\phi=\pi$ (two at inner side of the junction and two at the outer ends) but only two at $\phi=0$ (two outer), revealed in the gray curves in (b) or (c) \cite{PhysRevLett.108.257001,PhysRevB.91.024514,cayao2018andreev,PhysRevB.96.205425}. Interestingly, for short S regions $L_{\rm S}<2\xi_{\rm M}$, with $\xi_{\rm M}$ the Majorana localization length, a finite amount of non-Hermiticity induces EPs between lowest positive and lowest negative ABSs (outer MBS) and forms a zero-energy line between them [Fig.\,\ref{Fig4}(b)]. For $L_{\rm S}>2\xi_{\rm M}$, the outer MBSs acquire zero-energy for all $\phi$ and   the inner MBSs develop a zero-energy crossing at $\phi=\pi$ in the Hermitian regime (gray curves), which, at $\gamma\neq0$, transforms into a pair of EPs between inner MBSs connected by a stable zero-energy    [Fig.\,\ref{Fig4}(c)]. This EP regime is analogous to the case found in   Fig.\,\ref{Fig3}(a).  

With respect to the supercurrents $I(\phi)$, in Fig.\,\ref{Fig4}(e-f) we obtain them  using Eq.\,(\ref{INH}) for the Andreev spectrum in Fig.\,\ref{Fig4}(a-c), which includes the contribution of  ABSs and quasicontinuum above the induced gap. The first observation is that in all cases we find divergent profiles at EPs, which remains robust even under the presence of a phase-dependent quasicontinuum, see ends of yellow regions in Fig.\,\ref{Fig4}(d-f). Note that EPs between inner MBSs produce larger supercurrents for $L_{\rm S}>2\xi_{\rm M}$ as the quasicontinuum is weakly dependent on $\phi$ and the MBSs carry almost all the contribution. As argued in previous sections, this surprising behaviour,  the enhancement of Re $I(\phi)$ at EPs,  originates from the anomalous contributions of ABSs at EPs occurring at zero-real energy and represents  a unique   and robust NH effect.    We have also verified  that EPs at finite energy are possible to find in these realistic NH Rashba JJs but their respective supercurrents  do not sense the presence of EPs, in agreement with the simple models discussed in previous sections.





 
%%%%%%%%%%%%%%%%%%%%%%%%%%%%%%%
% SECTION 5:                       CONCLUSIONS                       %
%%%%%%%%%%%%%%%%%%%%%%%%%%%%%%%
In conclusion, we  studied non-Hermitian phase-biased Josephson junctions and demonstrated the formation of exceptional points at finite and zero real energies in the phase-dependent Andreev spectrum. We  discovered that the Andreev exceptional points at zero real energy give rise to a drastically enhanced   phase-dependent supercurrents, thus boosting the sensing   capability of the Josephson junction. We stress that our findings hold experimental relevance  because similar non-Hermitian   junctions have already been fabricated. This is the case, for instance, in recent  studies on few-site Kitaev chains based on quantum dots,  where attaching normal leads for transport measurements has also been achieved \cite{dvir2023realization,bordin2023crossed}. Moreover, Josephson junctions based on superconductor-semiconductor hybrids have already been fabricated, and normal leads seem to be relatively simple to attach \cite{Doh:S05,nphys1811,Nilsson:NL12,PhysRevLett.110.217005,tiira17,Goffman17,PhysRevLett.121.047001,PhysRevX.9.011010,PhysRevLett.124.226801,PhysRevLett.125.116803,razmadze2022supercurrent}. Taken together, these recent works clearly place our findings   within experimental reach.  Our work thus opens a route for the realization of exceptional points and enhanced supercurrents entirely by the interplay of non-Hermitian topology and  the Josephson effect. 

 \emph{Note added:} In the final stages of preparing this manuscript, a preprint was posted online Ref.\,\cite{li2023anomalous} which partially overlaps with some of our results related to EPs at zero real energy but using  a different methodology and setup. 


%%%%%%%%%%%%%%%%%%%%%%%%%%%%%%%
%                        ACKNOWLEDGMENTS                               %
%%%%%%%%%%%%%%%%%%%%%%%%%%%%%%%
We thank  R. Aguado and Y. Tanaka  for insightful discussions.   
J. C. acknowledges financial support from the Swedish Research Council (Vetenskapsr{\aa}det Grant No. 2021-04121), the Royal Swedish Academy of Sciences (Grant No. PH2022-0003), and the Carl Trygger's Foundation (Grant No. 22: 2093), and the Japan Society for the Promotion of Science via the International Research Fellow Program. M. S. was supported by JST CREST Grant No.JPMJCR19T2. 
 
 
\bibliography{biblio}
\onecolumngrid
\foreach \x in {1,...,2}
{%
\clearpage
	\includepdf[pages={\x}]{NHJJ_SM.pdf} 
}
\end{document}
