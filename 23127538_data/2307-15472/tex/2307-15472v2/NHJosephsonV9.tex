\documentclass[prl,showpacs,twocolumn,amsmath,amssymb,floatfix,superscriptaddress]{revtex4-2}
\usepackage{amsmath}
\usepackage{amssymb}
\usepackage{graphicx}
\usepackage{bm}
\usepackage{color}
\usepackage{hyperref}
\usepackage{graphicx}
\usepackage{upgreek}
\usepackage{scalerel}
\usepackage{multirow}
 
\usepackage{pgffor}
\usepackage{pdfpages}
\usepackage{mathdots}
\usepackage{float}
\usepackage{silence}
\WarningFilter{revtex4-1}{Repair the float}
\usepackage{physics} 
\usepackage{lscape}   
\usepackage{color, colortbl}
\usepackage[table]{xcolor}
\usepackage{comment}  
\makeatletter
\AtBeginDocument{\let\LS@rot\@undefined}
\makeatother
\newcommand{\edit}[1] {\textcolor{black}{#1}}
\newcommand{\editJ}[1] {\textcolor{red}{#1}}



\begin{document}
\title{Non-Hermitian phase-biased Josephson junctions}
\author{Jorge Cayao}
\affiliation{Department of Physics and Astronomy, Uppsala University, Box 516, S-751 20 Uppsala, Sweden}
\author{Masatoshi Sato}
\affiliation{Yukawa Institute for Theoretical Physics, Kyoto University, Kyoto 606-8502, Japan}

 
\date{\today} 
\begin{abstract}
We study non-Hermitian Josephson junctions formed by superconductors with a finite phase difference under non-Hermiticity, which naturally appears due to coupling to normal reservoirs. Depending on the structure of non-Hermiticity, captured in terms of retarded self-energies, the low-energy spectrum hosts topologically stable exceptional points either at zero or finite real energies as a function of the superconducting phase difference. Interestingly, the corresponding phase-biased supercurrents may acquire divergent profiles at such exceptional points. This instance is a natural and unique non-Hermitian effect that signals a possible way to enhance the sensitivity of Josephson junctions. Our work opens a way for realizing unique non-Hermitian phenomena due to the interplay between non-Hermitian topology and the Josephson effect.
\end{abstract}

\maketitle

%%%%%%%%%%%%%%%%%%%%%%%%%%%%%%%
% SECTION 1:                 INTRODUCTION                                %
%%%%%%%%%%%%%%%%%%%%%%%%%%%%%%%
Non-Hermitian (NH) systems have recently attracted enormous interest due to their potential for realizing unexpected states of matter that are impossible in the Hermitian realm \cite{el2018non,ozdemir2019parity,RevModPhys.93.015005,doi:10.1080/00018732.2021.1876991,OS23,wiersig2020review, parto2020non}. NH systems exhibit intriguing properties, being the most relevant to their complex spectrum with degeneracies known as exceptional points (EPs), where eigenvalues and eigenvectors coalesce
\cite{TKato, heiss2004exceptional, berry2004physics, Heiss_2012, PhysRevLett.86.787, PhysRevLett.103.134101, PhysRevLett.104.153601, gao2015observation, doppler2016dynamically,PhysRevB.99.121101}.  EPs have shown to be topological objects with topology defined in the complex spectrum, thus introducing the concept of point gaps unlike Hermitian topology \cite{SZF18, PhysRevX.9.041015,KBS19,ZSHC21}.  Also, the NH spectrum is very sensitive to perturbations, causing strong responses at EPs and offering a way for sensing devices with no Hermitian analog \cite{wiersig2020review,PhysRevLett.125.180403,arouca2022exceptionally,parto2023enhanced}.

NH physics has been shown to emerge in open systems \cite{RevModPhys.93.015005,doi:10.1080/00018732.2021.1876991}, with well-established mechanisms in photonics and optics where non-Hermiticity can be controlled almost at will \cite{PhysRevLett.86.787,feng2014single,gao2015observation,doppler2016dynamically,peng2016chiral,hodaei2017enhanced,chen2017exceptional,Longhi:17}. Another less explored but promising route is offered by material junctions where non-Hermiticity naturally arises when coupling a closed system to normal reservoirs \cite{datta1997electronic}. These ideas have recently been explored in normal systems \cite{PhysRevB.98.155430, PhysRevB.98.245130, PhysRevResearch.1.012003,cayao2023exceptional,NIOS23} and few examples also exist in setups with superconductors \cite{PhysRevB.87.235421,pikulin2013two,JorgeEPs,avila2019non,PhysRevB.105.094502,PhysRevB.107.104515,PhysRevB.105.155418,cayao2024nonhermitian}. While these initial studies have shed some light on the interplay between non-Hermiticity and superconductivity, they have overlooked the impact of non-Hermiticity on the phase of the superconducting order parameter. 

The simplest systems where the superconducting phase is crucial are Josephson junctions (JJs), formed by two superconductors with their order parameters exhibiting a finite phase difference \cite{RevModPhys.51.101,Tinkham}. This enables  the formation of Andreev bound states (ABSs) and the transfer of Cooper pairs, giving rise to a dissipationaless supercurrent across the JJ \cite{kulik1975,furusaki1991dc,PhysRevB.45.10563,Beenakker:92,Furusaki_1999,RevModPhys.76.411,sauls2018andreev}. These properties make JJs excellent devices for revealing the type of  superconductivity \cite{tiira17,PhysRevLett.121.047001,PhysRevX.9.011010,ren2019topological,Fornieri_2019,PhysRevLett.124.226801,PhysRevLett.125.116803,lutchyn2018majorana,prada2019andreev,frolov2019quest,flensberg2021engineered}  and also key superconducting building blocks for future quantum  applications 
\cite{PhysRevLett.89.117901,devoret2005implementing,PhysRevLett.90.226806,PhysRevB.81.144519,sarma2015majorana,acin2018quantum,krantz2019quantum,aguado2020perspective,benito2020hybrid,aguado2020majorana,beenakker2019search,PRXQuantum.2.040204,siddiqi2021engineering,bargerbos2022singlet,doi:10.1146/annurev-conmatphys-031119-050605,pita2023direct}. Thus, even though JJs offer a natural ground for NH physics, it is still unknown the response of ABSs and supercurrents to non-Hermiticity, specially, to the presence of EPs.  

In this work we consider NH phase-biased JJs formed by conventional superconductors coupled to normal leads [Fig.\,\ref{Fig1}] and discover topologically robust EPs and EP-enhanced supercurrents fully controlled by the phase difference. We find that the ABSs   reveal the formation of EPs at  finite and zero real energies. Interestingly, their  phase-dependent supercurrents become sharply enhanced   at  EPs, but only the EPs at zero real energy produce sizeable signals in the total supercurrents. Our work thus demonstrates the potential of NH topology for realizing   EPs and enhanced supercurrents  in phase-biased JJs.


 

    % Figure environment removed

  
%%%%%%%%%%%%%%%%%%%%%%%%%%%%%%%
% SECTION 2:                            MODEL:                              %
%%%%%%%%%%%%%%%%%%%%%%%%%%%%%%%
\textit{NH Josephson junctions}.---We consider JJs under non-Hermiticity induced by normal leads, as shown in Fig.\,\ref{Fig1}. We model these JJs by an effective Hamiltonian given by
\begin{equation}
\label{Heff}
H_{\rm eff}=H_{\rm JJ}+\Sigma^{r}(\omega=0)\,,
\end{equation}
where $H_{\rm JJ}$ describes the Hermitian JJ with superconductors (Ss) having a finite phase difference ($\phi$) between their order parameters \cite{PhysRevB.54.7366,zagoskin}. $\Sigma^{r}(\omega=0)$ is the retarded zero-frequency self-energy originating from the normal (ferromagnet) leads, which renders the JJ NH, as derived in Supplementary Material (SM) \cite{SM}. In the Hermitian regime,  the spectrum of $H_{\rm JJ}$ acquires a strong dependence on  $\phi$ and reveals the ABSs within the induced superconducting gap, producing a dissipationless supercurrent $I(\phi)$ \footnote{Hermitian JJs formed by conventional superconductors exhibit $I(\phi)\sim{\rm sin(\phi)}$ \cite{RevModPhys.51.101,RevModPhys.76.411} but more exotic ABSs can induce anomalous $I(\phi)$\cite{PhysRevB.53.R11957,PhysRevB.56.892,kashiwaya2000tunnelling,PhysRevLett.96.097007,PhysRevLett.103.107002,PhysRevB.92.155437,mizushima2018multifaceted,PhysRevB.101.184501,PhysRevB.104.L020501,PhysRevB.105.054504,baldo2023zero}. We also note that Ss forming JJs can be coupled directly or by a mediating region [Fig.\,\ref{Fig1}(a,b)]: when the length of the mediating region is shorter than the superconducting coherence length, only a pair of ABSs appears within the gap which then fully determines $I(\phi)$ \cite{Beenakker:92}.}.    
NH effects arise from  $\Sigma^{r}(\omega=0)$ \cite{datta1997electronic}, which  
in general has real (Re) and imaginary (Im) parts, $\Sigma^{r}(\omega=0)={\rm Re}\Sigma^{r}-i{\rm Im}\Sigma^{r}$ \cite{SM}.  While ${\rm Re}\Sigma^{r}$ only causes a shift in the energies of  $H_{\rm JJ}$,  ${\rm Im}\Sigma^{r}$ makes $H_{\rm eff}$  NH, inducing changes in the spectrum that are entirely due to the NH mechanism.   

 
Before going further, we stress the relation between ABSs and $I(\phi)$. The spectrum, and hence ABSs, can be obtained from ${\rm det}(\omega-H_{\rm eff})=0$, which corresponds to the poles of the retarded Green's function \cite{datta1997electronic,mahan2013many,zagoskin} $G^{r}(\omega)=(\omega-H_{\rm eff})^{-1}$, where $H_{\rm eff}$  is given by Eq.\,(\ref{Heff}). Due to non-Hermiticity, the spectrum becomes complex with eigenvalues coming in pairs due to particle-hole symmetry  \cite{PhysRevB.87.235421,pikulin2013two,JorgeEPs,avila2019non,PhysRevB.105.094502,PhysRevB.107.104515}, namely, ($E_{n},-E_{n}^{*}$), $E_{n} (\phi)=  {\rm Re}E_{n}(\phi)-i{\rm Im}E_{n}(\phi)$. The Re part represents the quasiparticle energy while the Im part characterizes its lifetime  \cite{datta1997electronic}. Thus, in the same spirit as in  Hermitian JJs, where the  Andreev spectrum determines the supercurrent \cite{furusaki1991dc,PhysRevB.45.10563,Beenakker:92},  it is natural here to define a supercurrent associated to the complex spectrum of NH JJs as \cite{PhysRevX.8.031079} $I=-(e/\hbar)\sum_{n\ge0}dE_{n}(\phi)/d\phi$, which leads to
\begin{equation}
\label{INH}
I(\phi)=-\frac{e}{\hbar}\sum_{n\ge0}\left[\frac{d {\rm Re}E_{n}(\phi)}{d \phi}-i  \frac{d [{\rm Im}E_{n}(\phi)]}{d \phi}\right]\,,
\end{equation}
where the first (second) term in square brackets is the contribution from the Re (Im) part of the Andreev spectrum with ${\rm Re}E_{n}\ge0$ \footnote{It is common to find versions of Eq.\,(\ref{INH}) with a factor of $2e/\hbar$ instead of $\frac{e}{\hbar}$, which occurs because spin degeneracy is usually assumed, see e. g.\,Refs.\,\cite{Beenakker:92,PhysRevB.96.205425}. Here, however, we consider models which are effectively spinless or with broken spin-degeneracy.}. Since the Re part stems from the physical quasiparticle energy, ${\rm Re}I(\phi)$ can be understood as the physical supercurrent while ${\rm Im}I(\phi)$ are associated with its lifetime \footnote{${\rm Im}I(\phi)$ reveals a supercurrent due to ${\rm Im} E_{n}$, whose inverse gives the quasiparticle lifetime \cite{datta1997electronic}.}.  Below, we consider two types of NH JJs with non-Hermiticity due to coupling to normal and ferromagnet leads and study EPs in the complex Andreev spectrum and their impact on $I(\phi)$.
 
\textit{NH JJ with superconductors coupled to different normal leads}.---To begin, we consider the NH JJ with single site superconductors (Ss) coupled to distinct normal (N) leads, as shown in Fig.\,\ref{Fig1}.
Equation (\ref{Heff}) 
with
 \begin{equation}
 \label{NHJJtoy1}
%\begin{split}
H_{\rm JJ}^{(1)}=
\begin{pmatrix}
H_{L}&V\\
V^{\dagger}&H_{R}
\end{pmatrix}\,,\quad
\Sigma^{(1)}={\rm diag}(\Sigma_{L},\Sigma_{R})\,,%\\
% \end{split}
\end{equation} 
describes ABSs in this NH JJ,  where $H_{\alpha}=\varepsilon_{\alpha}\tau_{z}+{\rm Re}(\Delta_{\alpha})\tau_{x}-{\rm Im}(\Delta_{\alpha})\tau_{y}$ for $\alpha=L$ $(R)$ is the Bogoliubov de-Gennes Hamiltonian of the left (right) S with an $s$-wave pair potential $\Delta_{\alpha}=\Delta{\rm e}^{i\phi_{\alpha}}$ and the onsite normal energy $\varepsilon_\alpha$, while $V=t\tau_{z}$ is the coupling between Ss  \footnote{The parameter $t$ controls the transmission across the junction, enabling to reach vanishing transmission and transmission equal to 1 which, respectively, characterize the tunnel and full transparent regimes \cite{Beenakker:92,PhysRevB.54.7366,datta1997electronic}.}. The self-energy term gives non-Hermiticity $\Sigma_{L(R)}=-i\Gamma_{L(R)}\tau_{0}$  \footnote{Different couplings $\Gamma_{L(R)}$ give a finite asymmetry in the amount of non-Hermiticity, shown before to be the key for inducing EPs \cite{PhysRevB.105.094502,PhysRevB.107.104515}}, where 
the coupling of the left (right) S to the left (right) N lead determines the decay rate $\Gamma_{L(R)}$  \cite{PhysRevB.105.094502,PhysRevB.107.104515,PhysRevB.105.155418}. ($\tau_{i}$  is the $i$-th Pauli matrix in the Nambu space.) For $\varepsilon_{\alpha}={0}$, $\phi_{L}=0$, and $\phi_{R}=\phi$, the eigenvalues of $H_{\rm eff}$ read   
\begin{equation}
\label{Eval1}
E_{j}=-i\Gamma\pm\sqrt{\delta^{2}-\gamma^{2}\pm2\Delta\sqrt{\smash[b]{t^{2}{\rm sin}^{2}(\phi/2)-\gamma^{2}}}}
\end{equation}
where $\delta^{2}=t^{2}+\Delta^{2}$, 
$\gamma=(\Gamma_{L}-\Gamma_{R})/2$, $\Gamma=(\Gamma_{L}+\Gamma_{R})/2$, and $j=1,\cdots,4$ labels the eigenvalues.  Fig.\,\ref{Fig2}(a) shows the Re and Im parts of $E_j$ as a function of $\phi$ without and with non-Hermiticity. 
For $\Delta=t$, $E_j$ without non-Hermiticity shows level crossings at $\phi=0, \pi$, which qualitatively reproduces the standard ABSs in transparent JJs  \cite{SM}. 
Notably, once turning on $\gamma\neq0$, the modes crossing at $\phi=0$ coalesce, with their Re parts merging at finite energy, signaling the formation of EPs \cite{RevModPhys.93.015005}. See yellow region ends in  Fig.\,\ref{Fig2}(a). From  Eq.\,(\ref{Eval1}), EPs appear when $t^{2}{\rm sin}^{2}(\phi/2)=\gamma^{2}$, and thus, they exist even for small but nonzero $\gamma$.
One can also confirm this feature in Fig.\,\ref{Fig2}(b), which shows the Re part of the difference between the two positive eigenvalues (${\rm Re}E_{\rm ee}$)  as a function of $\gamma$ and $\phi$. 
The border of the blue region (${\rm Re}E_{\rm ee}=0$) marks the EPs, 
which persist for $|\gamma|<|t|$. In contrast, the modes crossing at $\phi=\pi$ do not evolve into EPs.  The system has parity-time, spin-rotation, and particle-hole symmetries. %fully controlled by   
An NH topology protected by these symmetries ensures the presence and absence of these EPs \cite{SM}; thus, the discussed EP feature is universal.


    % Figure environment removed

 We next analyze the current-phase curve $I(\phi)$ for the ABSs in Eq.\,(\ref{Eval1}), obtained from Eq.\,(\ref{INH}). See Fig.\,\ref{Fig2}(c,d).   In the Hermitian regime, $I(\phi)$  exhibits a sawtooth profile at $\phi=\pi$ due to the zero-energy crossing in the ABSs, as the gray curve indicates. In the NH case, with $\gamma\neq0$, the ${\rm Re}I(\phi)$  and ${\rm Im}I(\phi)$ of each ABS develop sharp divergences at the EP since each ABS carries $\smash[b]{I_{1(2)}(\phi)}\sim \pm {\rm sin}(\phi)/\sqrt{\smash[b]{t^{2}{\rm sin}^{2}(\phi/2)-\gamma^{2}}}$, whose denominator vanishes at the EPs \cite{SM}. See red and blue curves in (a,b) and cyan arrows. %This can be understood by noting that   
 However, the divergences of $I_1(\phi)$ and $I_2(\phi)$ cancel out in the total  ${\rm Re}I(\phi)$, leading to a smooth curve across EPs with an overall sine-like profile, as shown by the green dotted curve  \cite{SM}; similarly, the total ${\rm Im}I(\phi)$  acquires vanishing values [Fig.\,\ref{Fig2}(d)]. Nevertheless, it is worth noting that, even though no EP effect is evident in the total $I(\phi)$, the Hermitian sawtooth profile at $\phi=\pi$  smooths out by $\gamma\neq0$.  In sum,  NH JJs can host tunable EPs at finite Re energies as a pure NH topological effect, but the total supercurrent is insensitive to such EPs.

 


 \textit{NH JJ with a middle N region coupled to a ferromagnet lead}.---Here we consider spinful NH JJs with a short mediating N region coupled to a ferromagnet lead, as shown in Fig.\,\ref{Fig1}(b). For simplicity, we focus on single site S and N regions  \footnote{JJs with a very small N region belongs to the short junction regime where the length of the N region is smaller than the superconducting coherence length \cite{Beenakker:92}. The same applies to the JJ studied in Eq.\,(\ref{NHJJtoy1})}. The NH  JJ   is then modeled by
\begin{equation}
\label{NH3}
%\begin{split}
H_{\rm JJ}^{(2)}=
\begin{pmatrix}
H_{L}&V&0\\
V&H_{N}&V\\
0&V&H_{R}
\end{pmatrix}\,,\quad
\Sigma^{(2)}={\rm diag}(0,\Sigma,0)\,,
%\end{split}
\end{equation}
where $H_{\alpha}=\varepsilon_{\alpha}\tau_{z}+{\rm Re}(\Delta_{\alpha})\sigma_{y}\tau_{y}-{\rm Im}(\Delta_{\alpha})\sigma_{y}\tau_{x}$ for $\alpha=L$ $(R)$ describes the left (right) S with an $s$-wave pair potential $\Delta_{\alpha}=\Delta{\rm e}^{i\phi_{\alpha}}$, $H_{\rm N}=\varepsilon_{\rm N}\tau_{z}$ presents the N region, and $V=t\sigma_{0}\tau_{z}$ is the hopping between N and S  \footnote{Here, $t$ controls the transmission across the junction \cite{Beenakker:92,PhysRevB.54.7366,datta1997electronic}.}. 
($\sigma_i$ and $\tau_i$ are the Pauli matrices in the spin and Nambu spaces.)
The coupling between N and a ferromagnet lead induces a spin-dependent self-energy $\Sigma={\rm diag}(\Sigma^{r}_{\rm e},\Sigma^{r}_{\rm h})$ where $\Sigma^{r}_{\rm e,h}(\omega=0)=-i \Gamma \sigma_{0}-i\gamma \sigma_{z}$, $\Gamma=(\Gamma_{\uparrow}+\Gamma_{\downarrow})/2$, $\gamma=(\Gamma_{\uparrow}-\Gamma_{\downarrow})/2$;   $\Gamma_{\sigma}$ is the coupling of spin $\sigma$   to the lead \footnote{The choice of non-Hermiticity here is different in comparison to the previous section, where the asymmetry therein originated from coupling  Ss to different N leads.}.  The eigenvalues of $H_{\rm eff}$ are not simple, but it is still possible to explore them as solutions of ${\rm det}(\omega-H_{\rm JJ}^{(2)}-\Sigma^{(2)})=0$. Taking up to second order in $\omega$, the solutions for $\varepsilon_{\alpha}=0$, $\phi_{L}=0$, $\phi_{R}=\phi$ read
\begin{equation}
\label{Eval2}
E_{\pm}=-i\Gamma \frac{\Delta^{2}(2t^{2}+\Delta^{2})}{A}\pm\frac{\Delta}{A}\sqrt{\smash[b]{C+2t^{4} A\,{\rm cos}\phi}}\,,
\end{equation}
where $A=4 t^4+\Delta ^4+2 \Delta ^2 \left(-\gamma ^2+\Gamma ^2+2 t^2\right)$, and 
$C=2t^4 A-4t^2\gamma^2\Delta^4+2(\gamma^2+\Gamma^2)^2\Delta^4-\gamma^2\Delta^6$.
From Eq.\,(\ref{Eval2}), we find that the ABSs are degenerate when $C+2t^{4} A\,{\rm cos}\phi=0$.
In the absence of non-Hermiticity ($\Gamma=\gamma=0$), it holds that $C=2t^4A>0$; thus, the degeneracy appears at $\phi=\pi$ as a level crossing at zero energy, as illustrated by gray curves in Fi.\ref{Fig3} (a). 
Then, in the presence of non-Hermiticity, the degeneracy evolves into a pair of EPs at $\phi_{\rm EP}=\pm {\rm arccos}(-C/2t^4 A)$ once we have $|C|<2t^4|A|$ by non-Hermiticity \footnote{We have verified that the corresponding wavefunctions also coalesce at these EPs.}.  
Intriguingly, the Re part of ABSs sticks to zero energy between the EPs, in contrast to the EPs in the previous model.
We note that a weak but nonzero spin-dependent non-Hermiticity $\gamma$ suffices to realize the EPs because $C$ decreases as $C=2t^2A-4t^2\gamma^2\Delta^4$ for $0< |\gamma|, |\Gamma|\ll |t|, |\Delta|$ so to meet the condition $|C|<2t^4|A|$.
The tunability of the EPs is also evident in Fig.\,\ref{Fig3}(b), which shows the Re part of the difference between the eigenvalues $E_{\pm}$ as a function of $\Gamma_{\uparrow}$ and $\phi$; EPs appear at  weak $\Gamma_{\rm \uparrow}$. The border of the blue region marks the EPs. 
Contrary to the previous system, this JJ breaks parity-time and spin-rotation symmetries and only keeps particle-hole symmetry intrinsic to superconductors. 
This difference enables the topologically stable EPs at zero energy,
as argued in the SM \cite{SM}.

 
 % Figure environment removed


When it comes to the supercurrent   $I(\phi)$  due to ABSs in Eq.\,(\ref{Eval2}), Fig.\,\ref{Fig3}(c,d)   shows its Re and Im parts as a function of $\phi$. In the Hermitian case,  $I(\phi)$ develops a sawtooth profile at $\phi=\pi$ due to the zero-energy crossing in the ABSs, see the gray curve. In the NH regime, the Re and Im parts of $I(\phi)$ interestingly develop divergences at the EPs of the ABSs, see the cyan arrows.  Here, $I(\phi)\sim {\rm sin}(\phi)/\sqrt{\smash[b]{C+2t^{4} A\,{\rm cos}\phi}}$, whose denominator thus vanishes at the EPs \cite{SM}. At first sight, this seems similar to $I(\phi)$ from  EPs at finite Re energy in Fig.\,\ref{Fig2}(c,d), but there is a key difference. In the present case, ${\rm Re}I(\phi)$ is entirely determined by only the non-negative  ABS, which then implies that the divergence at the EPs remains robust and cannot be washed out by the contribution of another ABS, in contrast to $I(\phi)$ from EPs at finite Re energy  \footnote{While reaching the EPs in real systems might be challenging,  but being close to them
already induces a drastic enhanced $I(\phi)$.}. Moreover,  while the total ${\rm Im}I(\phi)$ vanishes because the two ABSs merging into a single level at zero Re energy (yellow region) necessarily involves taking both Im ABSs, the ${\rm Re}I(\phi)$ intriguingly vanishes between EPs as a result of the zero-energy lines in the Re ABSs. The fact that $I(\phi)$  has robust divergences at EPs fully controlled by $\phi$ reveals that this effect is purely of NH origin.

 


%%%%%%%%%%%%%%%%%%%%%%%%%%%%%%%
% SECTION 4:      Physical realization in Rashba nanowires      %
%%%%%%%%%%%%%%%%%%%%%%%%%%%%%%%
\textit{NH JJs in superconductor-semiconductor hybrids}.---Having found  EPs and EP-enhanced $I(\phi)$ in simple NH JJs, here we show their realization in experimentally feasible NH Rashba JJs. The Hermitian part is modelled by $H_{\rm R}=\xi_{k}\tau_{z}+i\alpha k \sigma_{y}\tau_{z}+B\sigma_{x}\tau_{z}+\Delta\sigma_{y}\tau_{y}$, where $\xi_{k}=(\hbar^{2}k^{2}/2m-\mu)$ is the kinetic term with $k$ along $x$, $\mu$ is the chemical potential, $\alpha$ is the Rashba spin-orbit coupling (SOC), and $\Delta$  is the  spin-singlet $s$-wave pair potential.   Since SOC is intrinsic  in  semiconductors, we consider it to be finite \footnote{Similar results can be obtained without SOC but its intrinsic nature in semiconductors makes it    necessary in realistic systems \cite{sato2017topological,Aguadoreview17,prada2019andreev,lutchyn2018majorana,frolov2019quest,flensberg2021engineered,tanaka2024theory}.}. $B$ is a Zeeman field   along   $x$  inducing a Hermitian topological phase transition at $|B_c|=\sqrt{\smash[b]{\Delta^{2}+\mu^{2}}}$ that predicts Majorana bound states (MBSs) \cite{PhysRevLett.105.077001,PhysRevLett.105.177002,sato2017topological,Aguadoreview17,prada2019andreev,lutchyn2018majorana,frolov2019quest,flensberg2021engineered,tanaka2024theory}. Then, $H_{\rm R}$ is discretized into  a tight-binding lattice with spacing $a=10$\,nm  \cite{cayao2018andreev} and   divided into three finite regions: The left and right S regions host a  finite pair potential $\Delta$ with a finite phase difference $\phi$, while the central N region has $\Delta=0$; each region has length $L_{\rm N, S}$ and  the chemical potential $\mu_{\rm N, S}$. We take realistic parameters: $\alpha_{\rm R}=40$\,meVnm and $\Delta=0.5$\,meV, in the range of values for InSb and InAs nanowires, and Nb and Al Ss \cite{lutchyn2018majorana}. We finally model a short NH JJ by coupling the middle N region to a ferromagnet lead as in Eq.\,(\ref{NH3}) [Fig.\,\ref{Fig1}(b)], and obtain the Andreev spectrum and $I(\phi)$.
  
In Fig.\,\ref{Fig4}, we present the Re Andreev spectrum and ${\rm Re}I(\phi)$ as a function of $\phi$. For $B<B_{\rm c}$ in the Hermitian regime, the JJ develops zero-energy crossings, which evolve into stable zero-energy lines at finite spin-dependent non-Hermiticity $\gamma\neq0$ whose ends mark the presence of EPs. See the gray and blue curves and also the yellow region in Fig.\,\ref{Fig4}(a). These EPs do not depend on the length of the S regions
\footnote{MBSs appear as end states, with their wavefunctions decaying towards the bulk and overlapping in short systems \cite{PhysRevLett.105.077001,PhysRevLett.105.177002,sato2017topological,Aguadoreview17,prada2019andreev,lutchyn2018majorana,frolov2019quest,flensberg2021engineered}. This finite overlap is also accompanied with a zero-energy splitting, which becomes vanishing small for long systems and is thus seen as a signature of Majorana nonlocality \cite{PhysRevB.96.205425,PhysRevB.104.L020501}.}.  Above $B_{\rm c}$, the Hermitian JJ hosts four (nearly zero energy) MBSs at $\phi=\pi$ (two at the inner side of the junction and two at the outer ends) but only two at $\phi=0$ (two at the outer), as revealed in the gray curves in (b) or (c) \cite{PhysRevLett.108.257001,PhysRevB.91.024514,cayao2018andreev,cayao2018finite,PhysRevB.96.205425}. Interestingly, for the short S case $L_{\rm S}<2\xi_{\rm M}$, with $\xi_{\rm M}$ the Majorana localization length, a finite amount of non-Hermiticity induces EPs between lowest positive and highest negative ABSs (outer MBSs) and form a zero-energy line between them [Fig.\,\ref{Fig4}(b)]. For $L_{\rm S}>2\xi_{\rm M}$, the outer MBSs acquire zero-energy for all $\phi$ and the inner MBSs develop a zero-energy crossing at $\phi=\pi$ in the Hermitian regime (gray curves), which, at $\gamma\neq0$, transforms into a pair of EPs between inner MBSs connected by a stable zero-energy line  [Fig.\,\ref{Fig4}(c)]. This EP regime is analogous to the case found in  Fig.\,\ref{Fig3}(a).   

     % Figure environment removed


In Fig.\,\ref{Fig4}(e-f), we show $I(\phi)$ of Eq.\,(\ref{INH}) for the Andreev spectrum in Fig.\,\ref{Fig4}(a-c), which includes the contribution from ABSs and the quasicontinuum. In  all the cases, we find divergent profiles at EPs, which remain robust even under the presence of a phase-dependent quasicontinuum, as indicated by cyan arrows in Fig.\,\ref{Fig4}(d-f). Note that EPs between inner MBSs produce larger supercurrents for $L_{\rm S}>2\xi_{\rm M}$ since the quasicontinuum is weakly dependent on $\phi$ so the MBSs carry almost all the contribution. As argued above, this surprising behavior,  the enhancement of Re $I(\phi)$ at EPs,  originates from the anomalous contributions of ABSs at EPs occurring at zero-real energy and represents a unique and robust NH effect \footnote{Here EPs at finite energy might emerge accidentally, but their respective supercurrents do not sense the EPs, in line with the first simple model.}. 




%%%%%%%%%%%%%%%%%%%%%%%%%%%%%%%
% SECTION 5:                       CONCLUSIONS                       %
%%%%%%%%%%%%%%%%%%%%%%%%%%%%%%%
In conclusion, we studied non-Hermitian phase-biased Josephson junctions and demonstrated the formation of exceptional points at finite and zero real energies in the phase-dependent Andreev spectrum. We discovered that the Andreev exceptional points at zero real energy give rise to drastically enhanced phase-dependent supercurrents, thus boosting the sensing capability of Josephson junctions.  Our findings hold experimental relevance because similar non-Hermitian junctions have already been fabricated, including  Josephson junctions based on superconductor-semiconductor hybrids \cite{Doh:S05,nphys1811,Nilsson:NL12,PhysRevLett.110.217005,tiira17,Goffman17,PhysRevLett.121.047001,PhysRevX.9.011010,PhysRevLett.124.226801,PhysRevLett.125.116803,razmadze2022supercurrent} and few  Kitaev chains \cite{dvir2023realization,bordin2023crossed},  where coupling to normal leads seems to be a feasible task. Taken together, these recent works clearly place our findings within experimental reach.  Our work thus opens a route for realizing exceptional points and enhanced supercurrents entirely by the interplay of non-Hermitian topology and the Josephson effect. 

\emph{Note added:} In the final stages of preparing this work, the preprint Ref.\,\cite{li2023anomalous}  was posted online, which partially overlaps with our results but uses a different method and distinct setups.   We also note that a recent calculation by Shen et al. shows a similar enhancement of the Josephson currents by EPs \cite{shen2024nonhermitian}, consistent with our results  as we demonstrate in Section 5 of Ref.\,\cite{SM}.


%%%%%%%%%%%%%%%%%%%%%%%%%%%%%%%
%                        ACKNOWLEDGMENTS                               %
%%%%%%%%%%%%%%%%%%%%%%%%%%%%%%%
We thank  R. Aguado and Y. Tanaka for their insightful discussions.   
J. C. acknowledges financial support from the Swedish Research Council (Vetenskapsr{\aa}det Grant No. 2021-04121), the Royal Swedish Academy of Sciences (Grant No. PH2022-0003),  the Carl Trygger's Foundation (Grant No. 22: 2093), and the Japan Society for the Promotion of Science via the International Research Fellow Program. M. S. was supported by JST CREST Grant No. JPMJCR19T2. 
 
 
\bibliography{biblio}
\onecolumngrid
\foreach \x in {1,...,10}
{%
\clearpage
	\includepdf[pages={\x}]{NHJJ_SM_V3.pdf} 
}
\end{document}
