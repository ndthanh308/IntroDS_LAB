\documentclass[11pt,a4paper]{article}
\pdfoutput=1 

% ************** Used Packages **************

\usepackage[table]{xcolor}
\usepackage{colortbl}
\RequirePackage{ifpdf} 
\usepackage{amsmath} 
\usepackage{mathtools}
\usepackage{jheppub}
\usepackage{pstricks}
\usepackage[final]{pdfpages} 
\usepackage{ifpdf} 
\usepackage{slashed}
\usepackage{float}
\usepackage{comment}
\usepackage[caption = false]{subfig}
\usepackage[normalem]{ulem}
\usepackage{color}
\usepackage{xcolor}
\definecolor{urlblue}{rgb}{0.2,0.4,0.7}
\definecolor{citegreen}{rgb}{0,0.6,0.2}
\definecolor{linkred}{rgb}{0.9,0.2,0.1}
\usepackage{hyperref}
\hypersetup{
colorlinks=true, citecolor=citegreen, linkcolor=blue, urlcolor=urlblue}

\usepackage{graphics}
\usepackage{etoolbox} % Load before 'breqn' package to make 'lineno' work
\usepackage{fixmath}
\usepackage{psfrag}
\usepackage{textgreek}

\usepackage{notoccite} % If you have \cite commands in \section-like commands, or in \caption, the
% citation will also appear in the table of contents, or list of
% whatever.  If you are also using an unsrt-like bibliography style,
% these citations will come at the very start of the bibliography,
% which is confusing. This package suppresses the effect.

\usepackage{amsfonts}
\usepackage{autobreak}
\usepackage{marginnote}
\usepackage{enumitem}
\usepackage{appendix}
\usepackage{caption} 
\captionsetup[table]{skip=10pt}

\newcommand{\NOdisplay}[1]{ }

% ******************************* 
% Title & details of the authors
% *******************************

\title{QCD corrections to the Golden decay channel of Higgs boson}

\author{Mandeep Kaur$^{a}$, Maguni Mahakhud$^{a,b}$, Ambresh Shivaji$^{a}$, Xiaoran Zhao$^{c}$}
\emailAdd{mandeepkaur.iiser@gmail.com, maguni.mahakhud@kolhanuniversity.ac.in, ashivaji@iisermohali.ac.in, 
xiaoran.zhao@uniroma3.it}
\affiliation{
$^a$Indian Institute of Science Education and Research Mohali,\\
Knowledge City, Sector 81, SAS Nagar, Manauli, Punjab 140306, India
\\$^b$University Department of Physics, Kolhan University,\\
Chaibasa, West Singhbhum, Jharkhand- 833201
\\$^c$Dipartimento di Matematica e Fisica, Universit‘a di Roma Tre and INFN,\\sezione di Roma
Tre, Italy}


\preprint{}


% ********** Abstract **********

\abstract{{Future colliders aim to provide highly precise experimental measurements of the properties of the Higgs boson. In order to benefit from these precision machines, theoretical errors in the Higgs sector observables have to match at least with the experimental uncertainties. The theoretical uncertainties in the Higgs sector observables can be reduced by including missing higher-order terms in perturbative calculations.} In this direction, we compute mixed QCD-electroweak corrections at ${\mathcal O}(\alpha \alpha_s)$ to the Higgs decay into four charged leptons by considering the golden decay channel, $ H \to e^+e^-\mu^+\mu^-$. 
 Due to color conservation, these corrections receive contribution only from two-loop virtual diagrams. 
 In the complex mass scheme, 
we find that the mixed QCD-electroweak corrections to the partial decay width, relative to leading order predictions, are positive and are about $0.27\%$ ($0.30\%$) for fixed (running) QCD coupling. Relative to next-to-leading order electroweak corrections, the mixed QCD-electroweak corrections are found to be approximately 
 $18\%$ ($21\%$) for fixed (running) strong coupling. {With respect to the leading order, we observe a flat effect of mixed QCD-electroweak corrections on the invariant mass distribution of lepton pairs. The $\phi$ distribution, due to mixed QCD-electroweak corrections, follows $(1-\cos \phi)$ dependence.}
}



% ********** Beginning **********

\begin{document}
\allowdisplaybreaks[4]
\unitlength1cm
\keywords{}
\maketitle
\flushbottom
%%%%%%%%%%%%%%%%%%%%%%%%%%%%%%%%%%%%%%

\section{Introduction}
\label{sec:intro}
With the groundbreaking discovery of the Higgs boson by the CMS and ATLAS~\citep{CMS:2012qbp,ATLAS:2012yve} collaborations at the LHC in 2012, particle physics has entered into a realm of precision studies. The Higgs boson is one of its kind in the Standard Model (SM), therefore, precision measurements in the Higgs sector provide an opportunity to look for the physics beyond the Standard Model (BSM). {All the present and future colliders, such as the Large Hadron Collider (LHC)~\citep{Peskin:2013xra}, High-Luminosity LHC (HL-LHC)~\citep{Apollinari:2015wtw}, Future Circular Collider (FCC-ee)~\citep{TLEPDesignStudyWorkingGroup:2013myl}, Circular Electron Positron Collider (CEPC)~\citep{CEPC-SPPCStudyGroup:2015csa,CEPC-SPPCStudyGroup:2015esa}, and the International Linear Collider (ILC)~\citep{Baer:2013cma} aim to explore the uncharted territory of fundamental interactions by measuring various Higgs properties with higher statistics. On the theory side, highly precise predictions for Higgs production and decay channels are needed for a fairer comparison with future experimental data.}

Among the five prominent decay modes of Higgs, the rare but most important one is the decay of Higgs boson into four charged leptons, also known as the ``Golden decay channel". This decay channel played a significant role during the Higgs discovery in 2012 as it provides a particularly clean signature around the Higgs mass ($\sim 125 $ GeV) in the invariant mass spectrum of final state leptons. Furthermore, kinematic distributions of final state leptons for this decay mode not only allow for precision mass measurements of the Higgs boson but also serve as a powerful tool to study its spin and $CP$ properties~\citep{ATLAS:2014euz,CMS:2013fjq,Bolognesi:2012mm,CMS:2014nkk,ATLAS:2013xga}. The data collected in the off-shell production and decay of Higgs to four leptons via $Z$-boson pairs can constrain its total decay width~\citep{ATLAS:2014lla,CMS:2014quz}. 
Thus, improved predictions for the golden decay mode are paramount to have a better understanding of Higgs properties. In this direction, several works have been reported in the literature.

The exact one-loop QED corrections of $\mathcal{O}(\alpha )$ to the Higgs decay into four leptons with off-shell $Z$-bosons have been evaluated in~\citep{Piccinini:2005iu,CarloniCalame:2006vr}. The complete one-loop electroweak corrections for leptonic, semi-leptonic, and hadronic final states, and one-loop QCD corrections for semi-leptonic and hadronic final states to the Higgs decay $H \to WW/ZZ \to 4f$ have already been evaluated in~\citep{Bredenstein:2006rh,Bredenstein:2006ha} and are encoded in a Monte Carlo (MC) code {\tt Prophecy4f}~\citep{Bredenstein:2006nk,Bredenstein:2007ec}. The $\mathcal{O}(\alpha)$ corrections, reported for the case of four charged leptons in the final state, are of the order of 2-4\% for moderate Higgs masses ($M_H \leq 200$ GeV) and increase with growing Higgs mass, reaching upto 13\%. In addition to that, one-loop electroweak and QCD corrections to the Higgs decay into four fermions in the context of a simple extension of SM have also been studied and are implemented in the code {\tt Prophecy4f}~\citep{Altenkamp:2017ldc,Altenkamp:2017kxk,Altenkamp:2018bcs}. The Next-to-leading order electroweak corrections to the Higgs decay into the charged leptonic final state $H \to Z^{(*)} Z^{(*)} \to 4\ell$ with $4\ell = 4e, \;4\mu,\; 2e2\mu$ matched with QED Parton Shower (PS) have also been calculated, for which the results are available in a public event generator, {\tt Hto4l}~\citep{Boselli:2015aha}. 

In the present work, we compute QCD corrections to the decay $H \to e^+e^-\mu^+\mu^-$ on top of electroweak corrections that mainly receive contributions from two-loop diagrams, appearing at order $\alpha \alpha_s$. These mixed QCD-electroweak corrections are expected to be small because of the two-loop effect compared to the NLO electroweak corrections. But they are essential to provide precise predictions for Higgs sector observables at the LHC and future colliders and also to test the validity of perturbative QFT calculations. Our motive is to quantify these corrections, simulate decay events and provide improved numerical predictions for partial decay width of $H\to e^+e^-\mu^+\mu^-$ with the accuracy $\mathcal{O}(\alpha \alpha_s)$. {The two-loop diagrams contributing to the amplitude at $\mathcal{O}(\alpha \alpha_s)$ are very similar to those appearing in the processes $e^+ e^- \to Z H$ and $e^+ e^- \to \mu^+ \mu^- H$, for which mixed QCD-electroweak corrections of $\mathcal{O}(\alpha \alpha_s)$ have been evaluated in Refs.~\citep{Sun:2016bel,Gong:2016jys,Chen:2018xau}}. In this work, numerical calculation of two-loop amplitude is performed systematically using our in-house codes, and finally, to provide improved predictions for partial decay width, phase space integration over final-state leptons is performed by interfacing our codes with the publicly available code {\tt{Hto4l}}~\citep{Boselli:2015aha}.

The rest of the paper is organized as follows. In section~\ref{sec:QCD-cor}, we classify the Feynman diagrams which contribute at $\mathcal{O}(\alpha \alpha_s)$.
The organization of matrix elements in terms of form factors and their divergence structure is discussed in section~\ref{sec:FF}. In section~\ref{sec:renorm}, the UV renormalization of two-loop matrix elements, along with the opted renormalization scheme, is described. The numerical implementation of two-loop matrix elements for event generation and the checks performed on them are given in section~\ref{sec:imple} followed by our numerical results in section~\ref{sec:results}. Finally, we draw conclusions of our work in section~\ref{sec:conclusions}. 

\section{QCD correction to $H \to e^+e^- \mu^+ \mu^-$}
\label{sec:QCD-cor}
{In the SM, the leading order (LO) contribution to the on-shell decay of Higgs boson into four charged leptons ($ H \to e^+ e^- \mu^+ \mu^-$) mediated by $Z$-bosons comes from a tree-level diagram, shown in Fig.~\ref{fig:LO}. Due to energy conservation, at least one of the two $Z$-bosons has to be off-shell, depicted by $Z^*$. In this work, our calculation considers a more general case by treating both the $Z$-bosons off-shell.} We choose the following momenta assignments for the particles in the decay.  
\begin{equation}
H(q) \rightarrow Z^{(*)}(p_1)Z^{(*)} (p_2) \rightarrow e^+(p_3) e^-(p_4) \mu^+(p_5) \mu^-(p_6),
\end{equation}
% Figure environment removed
where momentum conservation requires $q =( p_1 +p_2)$, $p_1 = (p_3 + p_4)$ and $p_2 = (p_5 + p_6)$. In our calculations, we are neglecting the masses of final-state leptons. Therefore, the scalar products of momenta lead to
\begin{equation}
q^2 = (p_1+p_2)^2 = m_H^2,\; \;p_1^2 \neq 0,\; \;p_2^2 \neq 0,\; \;p_3^2 = p_4^2=p_5^2=p_6^2=0,
\end{equation}
where $m_H$ is the Higgs mass. As the decay width is proportional to squared amplitude, the amplitude for $H \to Z^{(*)} Z^{(*)} \to e^+ e^- \mu^+ \mu^-$ in the perturbative expansion upto two-loop order can be written as
\begin{equation}
M_{\text{Total}} = M_0 + M_1^{\alpha} + M_2^{\alpha \alpha_s}+ \dots
\end{equation}
Here $M_0$, $M_1$, and $M_2$ are the LO, one-loop, and two-loop amplitudes. {In the SM, $M_1$ receives contribution only 
from the electroweak (EW) sector, as the particles involved at the LO are color neutral. However, at two-loop, both the EW and QCD sectors can contribute. Therefore, there can be contributions of $\mathcal{O}(\alpha^2)$ and $\mathcal{O}(\alpha \alpha_s)$ at the two-loop, but one can neglect the contributions of $\mathcal{O}(\alpha^2)$ due to the smallness of EW coupling $\alpha$ in comparison to the strong coupling $\alpha_s$. Thus, we only consider mixed QCD-electroweak corrections of $\mathcal{O}(\alpha \alpha_s)$  at the two-loop level and focus on the evaluation of 
$M_2^{\alpha \alpha_s}$.

{In going beyond the LO, $\mathcal{O}(\alpha)$ amplitude receives contributions from several one-loop diagrams mediated by weak bosons and quarks. However, only quarks are susceptible to couple with gluons and take part in QCD corrections, therefore at $\mathcal{O}(\alpha \alpha_s)$, only the diagrams with quark loop along with the gluon dressing will contribute. The correction to the amplitude at $\mathcal{O}(\alpha \alpha_s)$ can be divided into three categories as follows:
\begin{equation}
\label{eqn:Total_Amp}
 M_2^{\alpha \alpha_s} = \delta M_{HV_1V_2}^{\alpha \alpha_s} + \delta M_{S.E.}^{\alpha \alpha_s} + \delta M_{Z\ell \bar{ \ell}}^{\alpha \alpha_s},
\end{equation}
where, $\delta M_{HV_1V_2}^{\alpha \alpha_s}$ consists of corrections coming from $HV_1V_2$ vertex, $\delta M_{S.E.}^{\alpha \alpha_s}$ contains corrections due to self-energy insertions on the vector-boson legs and $\delta M_{Z\ell \bar{ \ell}}^{\alpha \alpha_s}$ appears due to $\mathcal{O}(\alpha \alpha_s)$ counter-term for $Z\ell \bar{ \ell}$ vertex respectively. These contributions are described below.}

\subsection{$HV_1V_2$ vertex corrections}
{At two-loop, in addition to the decay of Higgs into $e^+ e^- \mu^+ \mu^-$ through $Z^{(*)}Z^{(*)}$ channel, we also have to consider the contributions coming from $Z^{(*)}\gamma^{(*)}$, and $\gamma^{(*)} \gamma^{(*)}$ channels. Thus, we consider the most general vertex correction denoted by $\delta M_{HV_1V_2}^{\alpha \alpha_s} (V_1,V_2 = Z,\gamma)$, where 
both $V_1$ and $V_2$ are taken off-shell, and write,}
\begin{equation}
\delta M_{HV_1V_2}^{\alpha \alpha_s} = \delta M_{HZZ}^{\alpha \alpha_s} +\delta M_{HZ\gamma}^{\alpha \alpha_s} +\delta M_{H\gamma Z}^{\alpha \alpha_s} + \delta M_{H\gamma \gamma}^{\alpha \alpha_s} .
\end{equation}
In the above, the $HZ\gamma$ and $H\gamma Z$ contributions are written explicitly to take care of the fact that $e^+e^-$ can come from either $Z$ or $\gamma$.
The leptonic decay of vector bosons is not affected by the QCD corrections to $HV_1V_2$ vertex. Therefore, we can decompose the amplitude for $HV_1V_2$ vertex corrections as
\begin{equation}
\delta M_{HV_1V_2} = M^{\mu \nu} J_{\mu}(p_1)J_{\nu}(p_2),
\end{equation}
where, $M^{\mu\nu}$ is the two-loop amplitude for $H (q) \rightarrow V_1(p_1)V_2(p_2)$ decay, $J_{\mu}(p_1)$ and $J_{\nu}(p_2)$ are the fermionic currents corresponding to $V_1 \rightarrow e^+ e^-$ and $V_2 \rightarrow \mu^+ \mu^-$. {Since the coupling of the top quark with Higgs is the largest among all quark flavours, we neglect the contributions from diagrams with any particle other than the top quark in the loop.} There are in total 48 two-loop triangle diagrams contributing to the $H (q) \rightarrow V_1(p_1)V_2(p_2)$ decay at $\mathcal{O}(\alpha \alpha_s)$ out of which some representative diagrams are shown in Fig. \ref{fig:triangle}, 
% Figure environment removed
which are generated with the help of the \texttt{QGRAF}~\citep{NOGUEIRA1993279} package. The amplitude for each of these diagrams is organized in \texttt{FORM} \citep{Vermaseren:2000nd,Vermaseren:2008kw} and is manipulated using \texttt{Mathematica}.

Instead of using conventional methods, the projector technique has been opted to perform the amplitude evaluation more systematically. In this technique, the amplitude for $H \to V_1 V_2$ using Lorentz covariance can be expressed as
\begin{equation}
\label{eqn:general amplitude}
 M^{\mu\nu} = (A\; g^{\mu\nu} + B\; p_1^{\nu} p_2^{\mu} + C\; \epsilon^{\mu\nu p_1p_2} + D\; p_1^{\nu} p_1^{\mu} + E\; p_2^{\nu} p_2^{\mu} + F\; p_1^{\mu} p_2^{\nu}),
\end{equation}
where, $A$, $B$, $C$, $D$, $E$ and $F$ are scalar functions called form factors and $\epsilon^{\mu\nu p_1p_2} = \epsilon^{\mu\nu \rho\sigma}p_{1\rho}p_{2\sigma}$. The Form factors can be obtained by applying suitable projectors $P^i_{\mu\nu}$ ($i=A,B,C,D,E,F$) on the amplitude $M^{\mu\nu}$. These form factors are, in general, functions of Mandelstam variables present in the problem under consideration. The two-loop form factors contributing to the amplitude at $\mathcal{O}(\alpha \alpha_s)$ are discussed in section~\ref{sec:FF}.

\subsection{Self-energy corrections}
The self-energy part of the two-loop amplitude denoted by $\delta M_{S.E.}^{\alpha \alpha_s}$ receives contribution from $\mathcal{O}(\alpha\alpha_s)$ corrections to $ZZ$ and mixed $Z\gamma$ self-energies. For the self-energy part, in addition to the top quark, two-loop amplitude also receives contributions from light quarks, which are taken to be massless. In total, 72 self-energy diagrams contribute to $\delta M_{S.E.}^{\alpha \alpha_s}$ as shown in Fig.~\ref{fig:SE_bare}. In order to calculate $\delta M_{S.E.}^{\alpha \alpha_s}$, we need $\mathcal{O}{(\alpha\alpha_s)}$ expressions of gauge boson self-energies. Their analytical expressions are available in \citep{Djouadi:1993ss,Dittmaier:2015rxo}.
% Figure environment removed

\subsection{$Z\ell \bar{ \ell}$  vertex correction} 
The amplitude at $\mathcal{O}(\alpha \alpha_s)$ also receives contribution from two diagrams shown in Fig.~\ref{fig:Zll_vertex} involving $Z\ell \bar{ \ell}$ vertex counterterm. 
The contribution to the vertex counterterm at $\mathcal{O}(\alpha \alpha_s)$ comes from the self-energies of vector bosons. It depends on the 
wave function renormalization constants of vector bosons, the charge renormalization
constant, and the renormalization constant for weak mixing angle  \citep{Dittmaier:2015rxo}. The pure EW nature of $Z\ell \bar{ \ell}$ vertex does not allow any kind of QCD corrections of $\mathcal{O}(\alpha \alpha_s)$, therefore the counterterm for $Z\ell \bar{ \ell}$ vertex is non-divergent in nature. One needs to be careful about the renormalization scheme in which the vertex correction is computed. 

% Figure environment removed

\section{Two-loop form factors and divergences}
\label{sec:FF}
The form factors for $HV_1V_2$ vertex corrections can be obtained by applying the projectors given in Appendix~\ref{app:A} on the two-loop bare amplitude of all the contributing diagrams shown in Fig.~\ref{fig:triangle}. 
In order to calculate the trace involving $\gamma^5$ matrix, the prescription given in~\cite{Larin:1993tq} is used.
{Furry's theorem forbids the appearance of single $\gamma^5$ in the trace over a closed fermionic loop due to charge invariance, which gives $C=0$ on adding the contributions from all the triangle diagrams together.} Furthermore, as the current conservation is associated with massless leptons in the final state, only the form factors $A$ and $B$ contribute at the amplitude squared level. These form factors can be written as linear combinations of scalar two-loop integrals of the type
\begin{align}
\label{eqn:integral-family}
 I_{\{\nu _i\}}\left( D,p_1^2, p_2^2, m_t^2, \mu^2 \right)
 & = 
 e^{2 \gamma_E \epsilon}
 \left(\mu^2\right)^{\nu-D}
 \int \frac{d^Dk_1}{i \pi^{\frac{D}{2}}} \frac{d^Dk_2}{i \pi^{\frac{D}{2}}}
 \prod\limits_{i=1}^7 \frac{1}{ P_i^{\nu_i} }.
\end{align}
Here, $k_1$ and $k_2$ are loop momenta, $D$ denotes the space-time dimension, $\gamma_E$ is the Euler-Mascheroni constant, $\mu$ represents an arbitrary scale introduced to maintain the dimensionlessness of the integral, $\{\nu _i\} $ = $\{\nu_1 \nu_2 \nu_3 \nu_4 \nu_5 \nu_6 \nu_7\}$ are powers of inverse propagators $P_i$, and $\nu = \sum_{i=1}^7 \nu_i$.
The inverse propagators $P_i$ are given by
\begin{align}
\label{eqn:propagators}
 P_1 & = k_1^2 - m_t^2,
 &
 P_2 & = k_2^2 - m_t^2,
 &
 P_3 & = \left(k_1-k_2\right)^2,
 \nonumber \\
 P_4 & = \left(k_1-p_1\right)^2 - m_t^2,
 &
 P_5 & = \left(k_2-p_1\right)^2 - m_t^2,
 &
 P_6 & = \left(k_1-p_1-p_2\right)^2 -m_t^2,
 \nonumber \\
 P_7 & = \left(k_2-p_1-p_2\right)^2 - m_t^2.
\end{align} 


The set of two-loop integrals in these form factors is reduced to a minimal set of integrals called master integrals (MIs) using integration-by-parts (IBP)~\citep{Tkachov:1981wb,Chetyrkin:1981qh} and  Lorentz-invariance (LI) \citep{Gehrmann:1999as} identities with the programs \texttt{LiteRed}~\citep{Lee:2012cn,Lee_2014} combined with \texttt{Mint}~\citep{Lee:2013hzt} and \texttt{FIRE}~\citep{A.V.Smirnov_2008,SMIRNOV20132820,SMIRNOV2015182}. We find in total 41 master integrals after IBP reduction. The choice of master integrals is not unique. The basis set $\vec{I}$ of 41 master integrals we get is
\begin{equation*}
\begin{aligned}
& \{I_{0000011},\; I_{0000111},\; I_{0001111},\; I_{0010110},\; I_{0020110},\; I_{0100011},\; I_{0100110},\; I_{0100111},\; I_{0101011},\; I_{0101110},\\
& I_{0101111},\; I_{0110010},\; I_{0110110}, \;I_{0111000},\; I_{0111001},\; I_{0111010},\; I_{0111011}, \;I_{0111110},\; I_{0112011},\; I_{0112110},\\
& I_{0120110},\; I_{0121001},\; I_{0121010},\; I_{0121011},\; I_{0121110},\; I_{0210010}, \;I_{0210110},\; I_{0211000},\; I_{0211001},\; I_{0211010},  \\
& I_{0211011},\; I_{0211110},\; I_{1100011},\; I_{1100110},\; I_{1100111},\;
I_{1101100},\; I_{1101101},\; I_{1110110},\; I_{1120110},\; I_{1210110}, \\
& I_{2110110}\}.
\end{aligned}
\end{equation*}

The involvement of two-loop integrals makes the evaluation of these form factors highly non-trivial. The master integrals can be evaluated numerically with the help of publicly available codes such as pySecDec~\citep{BOROWKA2018313}, AMFlow~\citep{Liu:2022chg} etc. The analytical results for these 41 master integrals in the canonical basis are now available in terms of iterated integrals~\citep{Chaubey:2022hlr}. {In our work, for an efficient evaluation, keeping the required accuracy of the MIs in mind, we use an in-house code for the numerical evaluation of all the two-loop master integrals involved using the sector-decomposition method given in~\citep{Binoth:2000ps,Binoth:2003ak,Li:2015foa}. The results obtained for MIs using the numerical integration are in good agreement with the analytical results.}

Due to the presence of loop integrals and massless particles in the loop, the two-loop form factors develop both Ultraviolet (UV) and Infrared (IR) divergences. 
We regularize both the divergences in dimensional regularization by taking $D = 4-2\epsilon$. After regularization, the divergences are encoded in the two-loop master integrals as poles in $\epsilon$, $\frac{1}{\epsilon^4}$ being the highest order of pole that can 
appear. The $\frac{1}{\epsilon^3}$ and $\frac{1}{\epsilon^4}$ poles are exclusively 
due to IR singularities, while $\frac{1}{\epsilon}$ and $\frac{1}{\epsilon^2}$ poles
can be due to both IR and UV singularities. According to the KLN (Kinoshita-Lee-Nauenberg) theorem~\citep{Kinoshita:1962ur,PhysRev.133.B1549,PhysRevD.17.2773,PhysRevD.17.2789}, the IR singularities eventually get cancelled against real emission Feynman diagrams to give IR safe observables. 

{In our case, the possible real emission Feynman diagrams involve the emission of a gluon from the closed quark loop, as shown in Fig.~\ref{fig:IR}.}
% Figure environment removed
{Due to the presence of a closed fermionic loop, the amplitudes of these diagrams are proportional to trace over $T^a$, the generator of $SU(3)$ gauge group. Since ${\rm tr}(T^a)=0$, diagrams for real corrections give zero. Moreover, these real emission diagrams can contribute only at the amplitude-squared level, which is an $\mathcal{O}(\alpha^2 \alpha_s)$ effect with respect to the LO. 
Therefore, real emission diagrams do not contribute at $\mathcal{O}(\alpha \alpha_s)$. The absence of real corrections thus demands for cancellation of virtual IR divergences among contributing diagrams, and the two-loop amplitude is expected to be free from any IR divergences in our case. Since the form factors $A$ and $B$ 
for $HV_1V_2$ vertex are independent, at two-loop, they should be separately free from $\frac{1}{\epsilon^3}$ and $\frac{1}{\epsilon^4}$ poles. This fact will provide one of the important checks on our calculation.} Furthermore, since the form factor $B$ is zero at the tree-level, and the first non-zero contribution to it arises at one-loop, we expect that the two-loop form factor $B$ does not have $\frac{1}{\epsilon^2}$ UV pole dependence. This can serve as another consistency check on our calculation. 

\section{UV renormalization and complex mass scheme}
\label{sec:renorm}
{In order to remove UV divergences from the matrix elements contributing at $\mathcal{O}(\alpha \alpha_s)$, standard on-shell renormalization scheme is used to renormalize all the fields and masses involved. The renormalization process involves evaluating required counter term (CT) diagrams and then adding CT amplitude back to the UV divergent amplitude to get finite results after the cancellation of all the divergences}. There are in total 48 one-loop triangle, 96 one-loop self-energy and 5 tree-level CT diagrams contributing to the amplitude at $\mathcal{O}(\alpha \alpha_s)$. The representative CT diagrams are shown in Fig.~\ref{fig:triangleCT}, Fig.~\ref{fig:SE_CT} and Fig.~\ref{fig:Tree_CT}, respectively. 

 
 As shown in Fig.~\ref{fig:triangleCT}, the triangle counterterm diagrams mainly involve $Vt\bar{t}$, $Ht\bar{t}$ vertex counterterms and counterterms for top-quark mass and wave function. However, the renormalization of mass and wave function of the top-quark is coupled with the renormalization of the $Ht\bar{t}$ and $Vt\bar{t}$ vertex. Thus on adding these contributions together, the $Ht\bar{t}$ and $Vt\bar{t}$ vertex counterterms are cancelled with the quark wave function counterterms. Therefore, at the one-loop level, we only need to evaluate diagrams with top-quark mass counterterm insertions.
On the other hand, for the evaluation of self-energy and tree-level counterterm diagrams, one needs $\mathcal{O}(\alpha \alpha_s)$ expressions for renormalization constants $\delta{Z_e}$, $\delta{Z_{ZZ}}$, $\delta{Z_H}$, $\delta{M_Z^2}$, $\delta{M_W^2}$, $\delta{Z_{\gamma Z}}$ and $\delta{Z_{Z\gamma}}$, which we have deduced from the self-energies of Higgs and gauge bosons given in~\citep{Djouadi:1993ss,PhysRevD.51.218}. 
 
 
% Figure environment removed
% Figure environment removed
% Figure environment removed
 

{The diagrams for the process under consideration involve unstable $Z$-bosons in the propagators. Therefore, one needs to introduce the finite $Z$-width in the propagators to ensure the stability of perturbative calculation at the $Z$-pole.} However, this incorporation of the finite $Z$-width can lead to several problems, such as the violation of gauge invariance due to the mixing of different perturbative orders~\citep{Grunewald:2000ju}. These problems are tackled via the adoption of the ``Complex Mass Scheme (CMS)"~\citep{Denner:1999gp,Denner:2005es,Denner:2005fg,DENNER200622}. In this scheme, we analytically continue the masses of weak gauge bosons to the complex plane; therefore, it is just a generalization of the on-shell scheme. Hence, we declare the $Z$-mass as a complex quantity and change the other parameters and counterterms accordingly to perform the whole calculation consistently. There is no need to introduce the finite width for the top-quark running in the loop as Higgs mass is taken to be 125 GeV which is below the $t\bar{t}$ threshold. 

After combining the CT amplitude with the UV divergent amplitude in the complex mass scheme, we get the finite results for amplitudes of $HV_1 V_2$ vertex and self-energy corrections.

\section{Numerical implementation and checks}
\label{sec:imple}
We combine the UV finite amplitudes for $HV_1 V_2$ vertex and self-energy corrections with the fermionic currents to get the two-loop contribution to $H \to e^+ e^-\mu^+\mu^-$. Adding the finite $Z\ell \bar \ell$ contribution to it, 
we get the matrix element $M_2^{\alpha \alpha_s}$ given in Eq.~\ref{eqn:Total_Amp}.
In the perturbative expansion of amplitude-squared upto two-loop level, this matrix element contribute via the interference with LO amplitude as
\begin{equation}
|M|^2 = |M_0|^2 + 2\;Re(M_0^*\; M_1^{\alpha}) + 2\;Re(M_0^*\; M_2^{\alpha \alpha_s}).
\end{equation} 
The interference term, in terms of two-loop form factors, is calculated using \texttt{FORM} and a {\tt FORTRAN} output 
is obtained. To obtain the partial decay width, we need to perform the phase-space integration over the final state leptons. For which we have used a publicly available {\tt{Hto4l}}~\citep{Boselli:2015aha} code. It is a Monte Carlo code which generates events for the process $ H \to 4\ell\;(\ell=e,\mu)$. A huge number of phase points needs to be generated to achieve good accuracy on observables. Calculating the form factors for every phase point is very time-consuming. To manage this issue, {we have prepared a two-dimensional grid with an accuracy of $\mathcal{O}(10^{-3})$ for form factors $A$ and $B$ using pre-specified phase-space points ($p_1^2$ and $p_2^2$) by numerically evaluating 41 MIs for input parameters given in Sec.~\ref{sec:results}. 
The grid is then used to estimate form factors at random phase space points with the 
help of a linear interpolation code developed in-house. Note that a new grid would be needed for any change in the input parameter set.}
After that, we interface the squared matrix elements along with the grid of form factors and interpolation code with the {\tt{Hto4l}} code to perform the phase-space integration and to obtain the corrected partial decay width and kinematical distributions for final state leptons. In order to prove the reliability of our implementation, we have performed the following checks:
\begin{enumerate}
    \item To a good numerical accuracy, we find that the $1/\epsilon^4$ and $1/  \epsilon^3$ poles cancel in both form factors $A$ and $B$. In form factor $B$, the 
         $1/\epsilon^2$ pole also vanishes. The UV poles in form factors $A$ 
         and $B$ cancel after adding the CTs, and the result does not depend on the choice of dimensional regularization scale $\mu$. These checks have been performed for several phase space points. 
    \item {As mentioned earlier the two-loop diagrams for $H \to e^+e^-\mu^+\mu^-$ are closely related to the ones appearing in production process $e^+ e^- \to ZH$. In Ref.~\citep{Gong:2016jys} analytical expressions for contributing form factors are given upto order $m_t^{0}$ after series expanding them in $\frac{1}{m_t}$. In order to check the accuracy of the grid prepared for form factors using an in-house code, we produced the grid for $e^+ e^- \to Z H $ for a large value of top-quark mass ($m_t$). Further, we match the numerical values of form factors from the grid with the one given in~\citep{Gong:2016jys}, and find an excellent agreement between the two for different values of center-of-mass energies.}
     \item The correctness of our numerical implementation is checked via reproducing the results for mixed QCD-electroweak corrections for $e^+ e^- \to ZH$ process given in Ref.~\citep{Gong:2016jys} in $G_\mu$ and $\alpha(0)$ scheme. We performed this check by implementing our calculation in \texttt{MadGraph} \citep{Alwall:2014hca}, and we found that the calculated corrections matched the available results in both schemes with relative error less than $1\%$.
\end{enumerate}

\section{Numerical results}
\label{sec:results}

In this section, we will present the numerical results for our calculation obtained by its implementation in {\tt{Hto4l}} code. We work in the $G_{\mu}$ scheme and use the following set of input parameters,
\begin{equation} 
\begin{aligned}
G_\mu  &= 1.16637\times 10^{-5}\; {\rm GeV}^{-2}, & M_Z  &= 91.1876 \;{\rm GeV},& M_W  &= 80.379 \;{\rm GeV}, \\
\Gamma_Z&= 2.4952 \;{\rm GeV},& M_H  &= 125 \;{\rm GeV},& m_t &= 173\;{\rm GeV}. 
\end{aligned}
\end{equation}
To analyze the mixed QCD-electroweak corrections, we take fixed $\alpha_s(M_Z)$ = 0.1185. For running of QCD coupling, we use the one-loop result and choose the invariant mass of $\ell^+\ell^-$ pair as the scale. 

With $\alpha_s(M_Z)=0.1185$, we find that the mixed QCD-electroweak correction
to partial decay width, with respect to the LO, is around 0.27$\%$. With running coupling, the correction becomes 0.30\%, shown by dotted lines in Fig.~\ref{fig:dist1} (a). 
In order to draw a comparison, we note that the two-loop QCD corrections in $H \to Z\gamma$ decay have been found around 0.22\% of the LO~\cite{SPIRA1992350,Bonciani:2015eua}. Similar corrections in the $H \to \gamma \gamma$ decay lie in the range of 1-2\% for the intermediate Higgs mass below $t\bar t$ threshold~\cite{DJOUADI1991187}.    
With respect to NLO EW corrections, mixed QCD-electroweak corrections amount to 18\% for fixed and 21\% for running QCD coupling. 
It is well known that higher-order corrections are sensitive to the kinematics of the events. In Fig.~\ref{fig:dist1}, we investigate the impact of two-loop corrections with respect to LO and NLO EW corrections on the invariant mass distribution of the final state lepton pair, where $\delta_i = \Gamma_{\textit{two-loop}} / \Gamma_{i}$ ($i=$ LO, NLO) indicates the relative correction.
% Figure environment removed 
For fixed scale, the mixed QCD-EW corrections relative to LO are roughly the same in all the bins and are of the order of 0.27\%, as seen in Fig.~\ref{fig:dist1} (a). This suggests that the nature of events for LO and two-loop corrections is kinematically similar. However, the two-loop corrections for running $\alpha_s$ differ in each bin and are higher with respect to fixed scale corrections in lower mass bins reaching up to 0.36\%. Beyond the $Z$ pole, the corrections become slightly smaller than the prediction at the inclusive level. This behaviour is dictated by the one-loop running of $\alpha_s$.

In Fig.~\ref{fig:dist1} (b), we plot the two-loop corrections with respect to the NLO EW corrections. Since the electroweak corrections are also sensitive to the kinematics of the events, we note that in certain bins, between 20 GeV and 40 GeV, the 
mixed EW-QCD corrections reach 40\% of the electroweak corrections. 
 The ratio of NLO and LO contributions in each bin, as shown in the lower panel of Fig.~\ref{fig:dist1} (b), can be used to understand the features in the upper panel. We get a larger contribution from two-loop corrections than NLO EW corrections in the region 20-40 GeV when the NLO EW corrections are negligible. The 2-loop corrections appear flat in bins between 40 GeV and 80 GeV. However, it is not the case, as shown 
in Fig.~\ref{fig:dist1} (c). Moreover, since the nature of LO and two-loop events are similar, 
$\delta_{\rm NLO}$ follows a pattern inverse to the pattern in the lower panel.

Apart from invariant mass distribution, angular distributions are also helpful for studying the Higgs properties. Therefore, we estimate the effect of mixed QCD-EW corrections in $\phi$ distribution, which is one of the most sensitive observables for   
BSM studies. It is defined as the angle between the decay planes of the intermediate $Z$-bosons in the rest frame of the Higgs boson. This angle $\phi$ is the main observable for spin-parity assignment of Higgs boson \citep{Soni:1993jc,Chang:1993jy,Skjold:1993jd,Barger:1993wt,Arens:1994wd,Buszello:2002uu,Choi:2002jk}.
% Figure environment removed
In contrast to the invariant mass distribution, mixed QCD-electroweak corrections relative to LO do not exhibit a flat behaviour for angular distribution, as shown in Fig.~\ref{fig:dist3} (a). We observe a $(1-\cos \phi)$ dependence in the shape of $\phi$-distribution due to mixed QCD-electroweak corrections. This is very different from the LO behaviour, which follows a {$\cos^2 \phi$} dependence~\cite{PhysRevD.37.1220,Choi:2002jk}. 
This difference can be attributed to the change in effective $HZZ$ coupling due to two-loop corrections.

The two-loop corrections with respect to LO are insignificant at the edges before rising and peaking at $\phi = 180^\circ$ with 0.49\% for fixed and 0.56\% for running QCD coupling. Compared to fixed $\alpha_s$, relative corrections are higher for running $\alpha_s$ with respect to LO across all the bins. 
In Fig.~\ref{fig:dist3} (b), we have shown the relative effect of two-loop corrections with respect to NLO electroweak corrections. The contribution of two-loop corrections is small with respect to NLO EW corrections at the edges where the NLO EW corrections are more significant, as shown in the lower panel of Fig.~\ref{fig:dist3} (b), whereas in the mid-range of angle $\phi$ where the NLO EW corrections are negligible with respect to LO, more prominent peaks for two-loop corrections are observed in the angular distribution. Once again, $\delta_{\rm NLO}$ for $\phi$ looks flat near the edges. However, it is indeed not the case as shown in Fig.~\ref{fig:dist3} (c), and it is consistent with the pattern of $d\Gamma_{\rm NLO}/d\Gamma_{\rm LO}$.

\section{Conclusions and outlook} 
\label{sec:conclusions}
In this paper, we have computed the mixed QCD-electroweak corrections to the partial decay width of $H \to e^+ e^- \mu^+ \mu^-$ channel. It is one of the most important decay channels to study Higgs properties at LHC and for new physics searches in the Higgs sector. In-house codes are developed to systematically compute contributing two-loop matrix elements at $\mathcal{O}(\alpha \alpha_s)$ using the projector technique. The bare two-loop matrix elements are found to be free from IR divergences but contain UV divergences, which are regularized using dimensional regularization and are eliminated by on-shell counter-term renormalization procedure. The whole calculation is implemented in publicly available event generator {\tt{Hto4l}} code to perform the phase space integration over final state leptons and to obtain the improved predictions for partial decay width. In $G_\mu$ scheme, the mixed QCD-electroweak correction to the partial decay width for Higgs mass 125 GeV is found to be 0.27$\%$
of the LO prediction for fixed QCD coupling. With respect to the NLO electroweak correction, the mixed QCD-electroweak correction is about 18\%.
The corrections can be significantly larger for Higgs mass above $t \bar t$ threshold, which is relevant to various BSM scenarios. 

For invariant mass distribution of final state lepton pairs, the two-loop corrections with respect to LO are flat, while with respect to NLO electroweak corrections, they can reach up to 40\% in certain bins. For angular distributions, which are crucial in measurements of spin-$CP$ properties of the Higgs boson, the corrections with respect to the LO are of the order 0.56\% in bins around $\phi =180^\circ$ when using the running $\alpha_s$. We also note that there are kinematic and angular bins in which the mixed QCD-EW corrections dominate the NLO EW corrections.
These results may be helpful in probing new physics in the Higgs sector. Needless to say that our computational framework also allows predictions for $H \to \gamma \gamma, \; \gamma Z, \;\gamma \ell^+ \ell^-, Z\ell^+ \ell^-$ decays. In addition to that, we can also use the ingredients calculated in this paper to predict $\mathcal{O}(\alpha \alpha_s)$ corrections for $H \to \ell^+ \ell^- \ell^+ \ell^- \; (\ell = e,\mu)$.

\section*{Acknowledgement}
MK would like to acknowledge financial support from IISER Mohali for this work. MK would like to thank Biswajit Das for the useful discussions. XZ is supported by the Italian Ministry of Research (MUR) under grant PRIN 20172LNEEZ.
\appendix
\section{Projection operators}
\label{app:A}
The projectors required to obtain form factors given in Eq. \ref{eqn:general amplitude} are
\begin{align}
P^A_{\mu\nu} &= \frac{1}{d-2}\left(g_{\mu\nu} + \frac{p_2^2 p_{1\mu}p_{1\nu}+ p_1^2 p_{2\mu}p_{2\nu} - (p_1.p_2) (p_{1\nu}p_{2\mu} + p_{1\mu}p_{2\nu})}{(p_1.p_2)^2 - p_1^2p_2^2}\right)\nonumber\\
P^B_{\mu\nu} &= \frac{1}{(d-2)((p_1.p_2)^2 - p_1^2p_2^2)^2}\bigg((p_1.p_2)^3\left(\frac{p_1^2p_2^2}{(p_1.p_2)^2}-1\right)g_{\mu\nu} +(d-1)(p_1.p_2)^2p_{1\mu}p_{2\nu}\nonumber\\
&-(d-1)(p_1.p_2)(p_2^2 p_{1\mu}p_{1\nu}+ p_1^2 p_{2\mu}p_{2\nu})+\left((p_1.p_2)^2+(d-2)p_1^2p_2^2\right)p_{1\nu}p_{2\mu}\bigg)\nonumber\\
P^C_{\mu\nu} &= \frac{\epsilon_{\mu\nu p_1p_2}}{(d-2)(d-3)((p_1.p_2)^2 - p_1^2p_2^2)}\nonumber\\
	P^D_{\mu\nu} &= \frac{1}{(d-2)((p_1.p_2)^2 - p_1^2p_2^2)^2}\bigg((p_1.p_2)^2 p_2^2\left(1-\frac{p_1^2p_2^2}{(p_1.p_2)^2}\right)g_{\mu\nu} + (p_2.p_2)^2(d-1)p_{1\mu}p_{1\nu}\nonumber\\
&-(d-1)p_2^2(p_1.p_2)(p_{1\mu}p_{2\nu} + p_{1\nu}p_{2\mu})+ (p_1^2p_2^2+(d-2)(p_1.p_2)^2)p_{2\mu}p_{2\nu})\bigg)\nonumber\\
	P^E_{\mu\nu} &= \frac{1}{(d-2)((p_1.p_2)^2 - p_1^2p_2^2)^2}\bigg((p_1.p_2)^2 p_1^2\left(1-\frac{p_1^2p_2^2}{(p_1.p_2)^2}\right)g_{\mu\nu} + (p_1.p_1)^2(d-1)p_{2\mu}p_{2\nu}\nonumber\\
&-(d-1)p_1^2(p_1.p_2)(p_{1\mu}p_{2\nu} + p_{1\nu}p_{2\mu})+ (p_1^2p_2^2+(d-2)(p_1.p_2)^2)p_{1\mu}p_{1\nu})\bigg)\nonumber\\
P^F_{\mu\nu} &= \frac{1}{(d-2)((p_1.p_2)^2 - p_1^2p_2^2)^2}\bigg((p_1.p_2)^3\left(\frac{p_1^2p_2^2}{(p_1.p_2)^2}-1\right)g_{\mu\nu} +(d-1)(p_1.p_2)^2p_{2\mu}p_{1\nu}\nonumber\\
&-(d-1)(p_1.p_2)(p_1^2 p_{2\mu}p_{2\nu}+ p_2^2 p_{1\mu}p_{1\nu})+\left((p_1.p_2)^2+(d-2)p_1^2p_2^2\right)p_{2\nu}p_{1\mu}\bigg)\nonumber\\
\end{align}
Where, $d=4-2\epsilon$ is space-time dimension with dimensional regulator $\epsilon$. 
\newpage
\bibliography{literature} 
\bibliographystyle{JHEP}
% ********** Ending **********
\end{document}
