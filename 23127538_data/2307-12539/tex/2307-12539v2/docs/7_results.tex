% !TEX root = ../main.tex

\section{Discussion}
\subsection{Top-down Analysis Approach for Sports Videos}
% \subsection{Data-Driven Video Analysis without Losing Contexts}
\label{sec:top-down-analysis}

% 1. top down approach is preferred by experts as it shows summary and provides data to support detailed analysis
% 2. however, it is not prevalence for two reasons. (1) to collect high-quality data, abundance resources is required. Only a fragment of athletes have access to it, like NBA league, national teams of popular sports. (2) with fewer resources, no data experts can provide good analysis and good data. Therefore, sports experts do not trust the data; or have to manually collect the data, like coaches in our study.
%  3. Our design has shown that, by maintaining the contextual understanding of the data, sports experts can use top-down approach to analyze game, without sacraficing their domain knowledge nor compromising their coaching time.
% therefore, it is important to provide context during the analysis.

% \jui{This section is not very focused. It seems at first about the top-down approach we designed and how it is preferred, but the last paragraph is for combining CV and human-in-the-loop, and the mix use of video and reconstruction. To make it worse, the section title is ambiguous. Make up your mind on what you want to say here and just say it, with the proper title.}

We proposed a top-down data analysis approach in our study to enhance badminton coaches' video analysis workflow. 
The approach involved providing an overview of the match and then using filters and visualizations to narrow down the area of interest and identify specific game moments.
% The approach involved coaches first getting an overview of the match and then using filters and visualizations to narrow down the area of interest and drill down to specific game moments. 
 All experts agreed that this approach was effective in supporting coaches to generate and verify insights in the case study.
% All experts in our study favored this approach as it supported coaches in generating and verifying insights effectively in the case study. 


While the top-down analysis workflow is a well-established visual analytics approach~\cite{shneiderman2003eyes}, 
% described in \textit{``Overview first, zooming and filtering, details on demand''}~\cite{shneiderman2003eyes}, 
it is rarely used by sports domain experts for analyzing sports videos. 
Instead, sports professionals, such as scouts and coaches, typically watch individual game videos to evaluate a player's performance 
due to the inaccessibility of high-quality data from videos, and limited resources that lead to experts relying on their own interpretation and annotation of videos.
% There are two main reasons for this. First, collecting high-quality data from videos requires advanced sensing technologies and input from data experts, making it inaccessible to most sports, e.g., no player tracking data is available for professional badminton matches. 
% Second, sports experts often rely on their own annotation and interpretation of videos due to limited resources.  This can lead to a gap between data and context during the analysis, resulting in less comprehensive and communicable insights, as found in Sec.~\ref{sec:gaps}.  
However, this approach can result in a gap between data and context, leading to less comprehensive and communicable insights, as found in Sec.~\ref{sec:gap3}. 
% 

To tackle this problem, we combined two promising avenues for sports analytics: CV-based data collection and human-in-the-loop analysis.
% with a direct contextual understanding of the data. 
Our study found that experts leveraged immediate access to both summary data and video to perform top-down analysis,
and integrate multiple data sources to develop and communicate their insights, such as domain knowledge, static and dynamic data. 
% to develop coaching advice and communicate with the players. 
Even when automation fails, experts can use the actual video to verify the data. As computer vision techniques continue to improve, we envision automatic data collection to benefit more sports domains,
% that more sports can benefit from data automation, 
while human-centered design empowers experts with an effective top-down analysis approach without losing contextual understanding of the data.
% empowering experts to perform top-down analysis for sports videos with a direct contextual understanding of the data.


% While computer vision can automatically detect sports events and statistics at an increasingly high accuracy~\cite{}, 
% in a highly personalized and collaborative task like sports coaching, expert knowledge will still be the dominating factor in the analysis process.
% Therefore, it is important to maintain the underlying context to allow experts performing analysis in depth while saving time on the low-level tasks. 
% In our study, we found that experts leveraged the immediate access to both static data and video to perform top-down analysis, integrating multiple data sources in their insights (expert knowledge, static data, dynamic data) and developing a coaching plan to communicate with player. 
% Even when the automation fails (e.g., inaccurate player pose), experts were able to use the actual video to verify. As CV and ML techniques continue to improve, we envision more sports can benefit from the data automation and empower experts to perform data-driven video analysis without losing contexts. 


% \textbf{Human-in-the-loop analysis vs. Data-in-the-hand analysis}

\subsection{Immersive Video Analysis for Sports Coaching}
% benefits of VR for sports analytics

Based on our case studies, we found that an end-to-end immersive analytic pipeline like VIRD can be suitable for sports coaching in badminton.
% \jui{personally, I am not too fond of excessive "bolding" everywhere. The section title is bold; the paragraph start is bold, and the bullet points are bold. This context switching makes it harder to get the structure of the article from a glance.}
Both coaches and the player in the studies were able to achieve their match analysis goals throughout the entire analysis pipeline in VR, from data exploration, insight generation to communication (using VIRD to showcase their insights to viewers on a TV screen in our study). 
% \jui{exactly how does communication happen in VR? Maybe I missed it...}. 
Unlike immersive analytics for scientific data analysis, where several analytic steps such as analyzing abstract data are better conducted in traditional desktop environments~\cite{bach2019immersive, hubenschmid2022relive}, sports data are intrinsically spatial and dynamic, making analyzing and presenting insights using videos and visualizations in 3D desirable. Furthermore, analyzing sports videos for coaching relies heavily on domain knowledge without the need for complex data manipulation. Thus, we found immersive analytics provide several advantages for sports coaching.

\para{1. Situated visualization reduces context-switching costs and shortens the path from hypotheses to insights.}
As both coaches commented, the most beneficial feature of VIRD was the immediate access to all the required data, \textit{``when I put the headset on I already have information that I may need without even having to watch the video first''} (C1).
% 
In the standard workflow on the desktop, coaches spend much time navigating and finding the critical game moments (e.g., when an error happens) while tallying rallies of interest on a separate note. This process induces high context-switching costs~\cite{wang2000guidelines} as the coaches need to constantly re-interpret the changed views between data in the notes and the game moments in the video, leading to a longer analysis cycle.  
% \jui{this seems to be a hypothesis. how did we confirm?}.
With large screen space and situated visualization placed in context (e.g., shots on the badminton court), coaches are presented with all required data in multiple views with spatial-continuous movement, which was found to reduce context-switching costs~\cite{yang_embodied_2021, plumlee_zooming_2002}. 
Therefore, experts can leverage their visual working memory~\cite{plumlee_zooming_2002} and focus exclusively on the match analysis.
% which Plumlee et al. described in their predictive model that multiple views are more efficient for tasks requiring high visual memory compared to zooming interface. 
% \jui{what does "visual working memory" mean? Clarify, cite, explain when appropriate.}.
This was shown in the hypothesis-driven workflow with VIRD, where experts can plan their analysis and immediately verify their insights with data, \textit{``I'd like to see where I made a mistake on the court. Almost all of it was on the forehand''} (P1).

% \textbf{2. Embodied interaction improves visibility of critical game moments.}
\para{2. Multi-modal data analysis improves visibility of critical game moments.}
An essential task in video coaching is identifying critical game moments to reveal root causes, \textit{``when we coach a player, we have to pinpoint the exact cause and outcome''} (C2).  
% These critical moments were identified by experts based on scores (e.g, last few rallies in a tied game), duration (e.g., short rally), and when winners and errors happened.
The ability to breakdown a rally stroke by stroke and instantly preview each shot in the video with embodied interaction in VR allows experts to directly access and focus on these critical moments.
One coach even suggested \textit{``it would be good if there's a way to play a loop of all the winning shots. Just because I'll spend a lot of time between going into different strokes.''} 
% 
Beyond individual shots, experts also dive into selected rallies in further detail. By comparing static data and 3D visualization from the game, they can develop a more comprehensive analysis from data summary to key moments. For instance, the coach (C1) went through all error rallies and pointed out the player's weakness on defensive shots as he pinpointed the location of error shots on the court with the VR pointer. 



% Experts examined player movement, reactions, shot sequence in the rally to determine the cause and develop strategy to improve the performance. 
\para{3. Immersive 3D visualizations deepen game understanding and engagement.} 
Experts expressed excitement that they could move freely in the virtual court and watch the game from different angles. Further, all of them were excited seeing the moving bird and the 3D reconstructed game.
% commenting with surprise like \textit{``That’s pretty cool!''} 
Throughout the analysis, experts felt engaged and interactive, \textit{``I like how I can move my body around and face the shot. I find that more beneficial than just looking at a TV screen or computer''} (C1). 
Experts also expressed their tendency to view a match in 3D and refer to the video only when the 3D view did not make sense. A coach also suggested adding rackets to improve the 3D view.
% The player felt it was fun to just rewatching himself in 3D and compare it with the video.
Using VIRD, experts obtained deeper insights
% \jui{can you provide an example?} 
on the spatial aspects in their analysis. P1 observed most of his winners were hit downwards from his backhand side from analyzing the shot locations and trajectories. 
  % as 3D visualizations provides shot breakdown and objective viewpoints
%However, with the errors in shot and pose detection, experts expressed they had to refer to the video to compare actions in the actual game. A coach suggested fixing the postures and adding rackets to improve the 3D view. 
% 
One interesting finding was the sense of presence in VIRD.
A coach used the first-person view to describe his analysis for a player in the match, saying \textit{``these are my errors''}, while the other coach moved to the coaching position by the court and the player moved to his side of the court to view the game from a first-person viewpoint.

% - Allow users to engage more deeply with the data through embodied interaction (with haptic feedback) 
% - Enables users to test hypothesis and adjust their analysis in real-time, facilitating a more iterative and data-driven approach to analysis.

% - Allow users to better visualize the match multiple angles and identify key moments
% - Facilitate better understanding and communication of game insights with 3D spatial visualization
% - Provides a more immersive and engaging experience, allowing users to feel like they are in the game

% -visceral experiences
% 1. promote deeper analysis on spatial aspects - shot location, distribution, shot trajectory
% 2. immersive and make user feel like being in the game - first person view point
% 


% \subsection{Progressive User-Centered Design with Pro Athletes}
% \label{sec:proathletes}
% % what works and what doesn't?
% Ideally, we would benefit from collaborating with a dedicated team of experts throughout the design process. 
% However, with limited access to high-performance coaches and players, we adjusted our human-centered design process based on experts' availability. In the formative study, we interviewed five experts online. This allowed us to understand the current practice and pain points from their diverse experiences. In the design iteration, we conducted in-person user testing in different cities with three coaches focusing on different design aspects. This approach enabled us to obtain necessary design feedback from multiple experts while minimizing their time commitment (1 hour per expert). We took a similar approach to evaluate VIRD in case studies with three experts, each with a different focus, such as developing strategy vs. coaching, and player vs. coach perspective.

% % 
% However, we also faced challenges in obtaining consistent feedback from coaches due to differences in their familiarity with new technology and analytical approaches. Coaches who were more technologically adept provided more constructive feedback on specific design aspects, while others were hindered by incomplete data or technical limitations and struggled to visualize the user flow. For instance, one coach suggested including shot types (e.g., drop) in the design to facilitate detailed shot analysis, 
% which was not prioritized in our design requirements to avoid overfitting individual coaching styles.
% Nonetheless, getting user feedback on distinct areas helped us progressively refine design based on consistent goals and tasks, as described in Sec.~\ref{sec:design_iteration}  

% Through this process, we found gaining continuous access to sports experts challenging due to the intense competition cycles in the sports domain.
% Our progressive approach offers a pragmatic solution to the challenges of a user-centered design process with professional athletes to advance visualization research in sports.




\subsection{Limitations \& Generalizability}
\textbf{Limitations.} 
% shorten: We discuss the limitations of automatic data collection accuracy and the small number of domain experts in our study. 
% Further, we envision extending our work in the direction of remote collaboration, natural language-based interaction, and leisure game viewing.
% \jui{I will start the section by a paragraph summarizing/enumerating the limitations, before jumping into each of them in order of importance.}
% 
Our computer vision models are around 90\% and 96\% accurate in detecting shots and player poses, which causes confusion when inaccuracy occurs. While experts can still perform match analysis by accessing the video view in VIRD, this was mentioned as an area for improvement by all experts. As CV techniques advance, we envision the limitation on automatic data collection can be largely improved with better-trained models, making our approach more reliable. 

\re{Due to the limited access to high-performance badminton experts, such as Olympian coaches and players,}
\re{our study reports the feedback from a few domain experts. 
% we were fortunate to get connected with. 
We believe the identified problem is significant and common among badminton athletes, but our solution might not generalize to all experts due to varying analysis approaches and resources among coaches and players.} Instead, we consider our main contribution to be a design study exploring the use of immersive analytics in real-world sports coaching. 


% generalizability
\noindent
\textbf{Generalizability.} We believe our established data preprocessing pipeline (based on MonoTrack~\cite{liu-2022} and CLIFF~\cite{li-2022}) and the immersive and interactive way to analyze multi-modal game data in badminton can be applied to general match videos and
benefit the broader community beyond professional coaches, such as players at all levels, and other racket sports.
With the VR benefits in visualizing spatial data and revisiting critical game moments, we also envision expanding VIRD beyond match analysis for leisure game viewing or broadcasting, as noted by the player that watching the game in 3D was fun.

% \noindent
% \textbf{Future Work.} Some exciting future work includes supporting remote collaboration between the coach and player in a shared immersive space, which can address the gap of limited coaching time and lack of support for video discussions found in the formative study. 
% We also foresee integrating natural language as an input method to further lower the context-switching costs throughout match analysis, as using VR controllers to select and filter data was found clunky for first-time users. 

% shorten: For example, coaches wanted to filter all winners shot won by Marin in game 1; Instead of clicking three buttons on the UI, this instruction can be parsed automatically by a large language model such as GPT~\cite{brown2020language}.
% 


% \jui{I would stress that many parts of offline preprocessing and VIRD can be repurpose for exactly this.}.




% \subsection{Implications for Sports Analytics in XR}
% % technology & research
% - computer vision for game reconstruction and analysis
% >> it is getting much better with video training data

% - XR technologies 
% >> technology is getting better and more available

% - human-AI interaction, personalization
% >> expert in the loop

% - future work: remote collaboration, game viewing for athletes and fans, simulated training (AR)
