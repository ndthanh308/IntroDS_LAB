% !TEX root = ../main.tex

\section{Formative Study with Olympian Coaches \& Players}
\label{sec:formative_study}
We applied a user-centered design process to develop VIRD and involved target users at every design stage. All experts involved in our study are Olympic or national team coaches and players.
% 
Section \ref{sec:formative_study} presents the gaps in match analysis based on expert interviews with coaches and players.
Section \ref{sec:goal_task_analysis} presents our goal and task analysis, which informed the design of VIRD.
Section \ref{sec:vird} presents VIRD's design based on three rounds of user testing with coaches.
Section \ref{sec:user-study} presents the evaluation of VIRD with both coaches and players on match analysis for developing game strategy and communicating insights. 
% Note that due to the specific domain we target, i.e., high-performance badminton coaching, our design is guided by a small number of domain experts involved in the study. While this is the nature of professional sports, we discuss the implications of conducting research with professional athletes in Sec.~\ref{sec:proathletes}.

% \jui{unprofessional question here: should this section really be called "Design Requirement"? Doesn't seem to match what comes next... But maybe it is a field thing..}

% We interviewed 5 badminton professionals on their coaching experiences to obtain insights into gaps in their current video analysis. 
% We iterated with 3 coaches to propose an ideal match analysis workflow and conducted goal and task analysis to form design requirements. 

% \jui{One thing I think will be good to clarify here or in intro is why we have so few interviewees -- due to our focus on high level players.}

\subsection{Procedures}
\label{sec:formativestudy}
To understand current practice and identify gaps in badminton coaching, we interviewed five professional badminton players and coaches (I1-I5; M = 2, F = 3; Age:
30-45). All of them are former Olympic players representing Canada, Taiwan, and the US. All of them have at least 10 years of player experience and four became professional coaches after their playing careers with 2 to 15 years of coaching experience.

We conducted 1-hour semi-structured interviews online to elicit the interviewee's background, overall coaching workflow, video and data usage in their coaching, and how they evaluate the player's performance in the video analysis. Finally, we asked interviewees to analyze a short match video to demonstrate their typical analysis workflow. 

All interviews were transcribed and analyzed using affinity mapping. Our analysis focused on understanding the current badminton coaching practice and identifying gaps in their match analysis workflow. 



%%%
\subsection{Findings and Gaps}
\label{sec:findings}
Overall, we observed that coaching practices varied widely among the interviewees due to varying resource levels, player skill levels, and coaching styles. 
%Despite differences in coaching style, 
The coaching process typically involves a significant amount of video analysis for both coaches and players. 
% why videos
For players, 
videos are crucial as they help players become aware of their playing technique and style and allow them to create a mental model of other players.
Players are often told to record their own match and watch the videos multiple times, 
e.g., \textit{``Most coaches recommend we watch it several times and break it down to focus on one thing at a time''} (I3). 
Coaches also rely on videos to direct coaching by analyzing the root cause of player performance and
communicating insights to players,
e.g., \textit{``They won’t understand what I am saying unless they see it physically''} (I2).

% This practice is useful for developing strategic insights into the game. 

% Therefore, analyzing matches for coaching is crucial for both coaches and players.
All interviewees agreed that video analysis is time-consuming. 
I4 noted that
\textit{``if the match is 30-40 minutes, it took 3 to 5 hours to rewatch and discuss with your coach''}. 
When analyzing the videos, 
coaches and players watch matches, take notes, analyze them for insights, and discuss their findings. 
With more resources, coaches can proactively share their insights with players, but this requires a significant time investment (I2). 
Alternatively, in cases where access to a coach is limited, 
players may seek coaching by requesting feedback on areas to improve (I3, I5)
or by coaching each other (I1, I3, I4). 
Therefore, analyzing matches for coaching is crucial for both coaches and players.

For clarity on badminton terminology, note that a match comprises the best-of-three games. Each game has multiple rallies, with each rally awarding a point. Within every rally, players execute a series of shots.
Below, we summarize four gaps in the current match video analysis.




% Coaches found it beneficial to analyze multiple matches of their players over a period of time (I1, I2, I4).
% \textit{``If you had like 20 matches that you played for an entire year, then you have a better idea where you can dictate training''} (I1). 

% This variation can be explained by a few variables. First, the level of resources available to a player can vary significantly depending on their level of competition, with high performers or older players typically receiving more attention. 
% Second, the coach-player relationship 
% can impact the dynamics and ultimately the outcome of the coaching, as emphasized by a coach that \textit{``you really have to understand the individual and know how they respond to certain things''} (I2). 
% Third, coaches' own preference and familiarity with the available technology stack shape their coaching style. 
% Some coaches use data analysis (e.g., notational analysis) to reveal errors and shot patterns (I1, I2), while others gain insights on playing styles and shot quality from match videos (I3, I4).
% Finally, the player's skill level impacts the instruction types. 
% Novices receive guidance on basic tactics and pose correctness, while elite athletes focus on tactical-level instruction.
% In this work, we summarized the current practice and pain points in video analysis.

\subsubsection{The Current Bottom-Up Workflow is Inefficient}
\label{sec:gap1}

We observed that the experts we interviewed followed a \emph{bottom-up} approach to generate their insights. 
They began by scanning videos to detect insights into a player's playing style and weaknesses.
For example, during video scanning, I3 promptly observed, \emph{``She's a lefty ... [so] she tends to lean more to her left side for a big forehand.''} 
Upon forming an insight hypothesis, they scrutinized additional videos to identify similar patterns and validate their observations, e.g., \textit{``Is this just an outlier or some of these random matches where we didn’t do well, or concrete things that we need to work on?''} (I2). 
Coaches repeatedly watched games until insights emerged and revisited the games to gather further evidence.
Because of the limited time and resources to review footage, coaches and players might choose to only look at the most important parts, which leads to incomplete analysis
and potentially less effective communication. 
In summary, the current bottom-up analysis workflow lacks support for efficient iteration and analytic reasoning.
% \zt{I feel we need a sentence to transit to the top-down workflow. maybe quote ``Overview first, details on demand''}
% leading us to the first design goal:


\subsubsection{Manual Data Collection from Videos is Time-Consuming}
\label{sec:gap2}

During the current workflow,
experts manually collect summary statistics from watching the videos to reveal patterns quantitatively. This allows them to compare the player performance (I1, I2, I4) and
communicate better with the players (I1, I2).
For example, 
having these data benefit their coaching greatly, e.g., \emph{``If you had like 20 matches that you played for an entire year,
then you have a better idea where you can dictate training''} (I1).
Further, concrete evidence like \textit{``70\% of time when you do this, you win the point''} (I2) allows players to immediately grasp the concept.
% 
% Players also compare their performance in tournaments and training by counting patterns from selected rallies, such as shot types, locations, sequence, and body movement.
However, 
when watching a video,
paying attention to multiple metrics and patterns simultaneously is difficult. 
As a result, coaches and players have to watch the videos multiple times and
focus on different aspects one at a time, 
such as opponent versus their player,
and winner shots versus unforced errors. 
Such a manual process \emph{``takes up a lot of time''} (I2). In summary, manually collecting data from videos hinders experts to perform match analysis efficiently.
% This manual process is time-consuming and might result in incomplete analysis and less effective communication. Therefore, it is crucial to incorporate a design goal that addresses this issue.


\subsubsection{Data Insights and Contexts are Presented Separately}
% \para{Presenting Data Insights with Videos is Necessary}
\label{sec:gap3}
% \zt{need presenting insights with videos}

Currently, data insights and videos are often presented separately. Coaches typically provide players with a summary of key insights, such as 
\textit{``you are pushing the tempo and make too many mistakes (6 errors in 11 points you lost)''},
without directly connecting to specific moments in the video. 
This disjointed presentation hinders a comprehensive understanding of the game, as players may struggle to visualize the context behind the numbers. 
To bridge this gap, coaches might manually note timestamps of critical game moments with the help of some tools (e.g., YouTube, Hudl~\cite{hudl}, Clutch~\cite{clutch}). However, collecting the video moments is still very tedious, as noted by I2 that \textit{``I have to spend 30 minutes per player writing things down, and another 1 hour to review notes with them to show them here’s what happened''}.
Moreover, without ample video evidence, sometimes it can be hard to convince the players of a particular finding. For example, I1 mentioned that 
\textit{``The kids don't realize that they make a lot of unforced errors. If they watched the video clips, they'd understand it better''}.
Therefore, the current way of presenting data and video separately may impede a holistic understanding of the game due to an inefficient workflow.



\subsubsection{2D Game Representation is Insufficient}
\label{sec:gap4}

When asked about the limitation of analyzing matches with videos, 
multiple coaches expressed that single-camera recordings might fail to capture essential game aspects, 
such as environmental conditions (e.g., wind), 
shot timing (\textit{``Speed seems slower in the video''}), 
and player reactions.
Although official badminton match are limited by monocular videos,
some coaches use 360-degree videos in training for comparing players and their opposition, as 
\textit{``you can see both sides and like 
how the player reacts to the opponent in real-time''} (I1). 
Slow-motion or zoomed views also assist in breaking down techniques and offering objective perspectives (I1, I3).
I3 noted \textit{``when you're hitting the shot, you only know how it feels, but you can't see how it looks''}. 
These remarks indicate that traditional 2D videos fall short in providing spatial comprehension and flexible viewing angles necessary for analyzing specific shot attributes. %monocular
This finding aligns with previous work~\cite{ye2020shuttlespace, chu2021tivee}.





\subsection{Summary}

Based on the formative study, we identified that high-performance badminton
coaches and players perform match analysis to reveal the
strengths and weaknesses of players and to develop playing
or training strategies. However, data collection (i.e., bottom-up workflow and manual note-taking) and presentation (i.e., separation of data and videos and 2D game representations) gaps exist in their current analysis workflow, leading to inefficient match analysis and communication for coaching.   

