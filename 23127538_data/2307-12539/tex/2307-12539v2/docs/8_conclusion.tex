% !TEX root = ../main.tex

\section{Conclusions and Future Work}
In this study, we introduced VIRD, an immersive badminton match video analysis tool for high-performance coaching based on a formative study with Olympic coaches and players. 
VIRD employs a top-down analytic approach in VR with 3D reconstructed game views and multi-modal data analysis.
Experts successfully developed game strategies and effectively communicated insights using VIRD in case studies, showcasing the advantages of immersive analytics in badminton coaching. 
% 
These benefits include reduced context-switching costs, enhanced visibility of critical game moments, and a deeper understanding of and engagement with the game through situated 3D visualizations.

% 
Promising future work includes enabling remote collaboration between coaches and players in a shared immersive VR space, addressing the limited coaching time and insufficient support for video discussions. Additionally, VR could be employed to simulate game scenarios for enhanced athlete training.  
Incorporating natural language input methods, such as GPT, may also help minimize context-switching costs during match analysis, enabling more efficient analytical iteration.

% Results suggest that when presented with video context, CV-based data collection can support
% effective top-down analysis for sports coaching.
% % human-in-the-loop analysis for sports game analysis. 
% Experts expressed high satisfaction 
% analyzing match videos with VIRD. 
% % compared to current methods. 
% Further, immersive analytics show benefits for sports coaching in presenting data in-situ to reduce context-switching costs, improving the visibility of critical game moments with embodied interaction, and deepening game understanding and engagement with immersive 3D visualizations.
