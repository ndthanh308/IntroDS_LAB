% !TEX root = ../main.tex
\section{Goal \& Task Analysis} 
\label{sec:goal_task_analysis}


\subsection{Design Goals}
To support coaches and players in analyzing matches and communicating
insights effectively, we characterized four design goals with respect to the four identified gaps in  Sec.~\ref{sec:findings}. 


\para{G1. Providing a top-down analysis workflow.} 
Our tool needs to present summary data to enable an immediate overview of the match, and support effective data exploration to discover regions of interest for detailed analysis. 
Unlike the traditional bottom-up approach that requires users to watch the videos sequentially to observe insights, the top-down approach supports users to analyze game details on demand, driven by observed patterns from summary data,
\re{as captured in ``Overview first, zoom and filter,
details on demand''~\cite{shneiderman2003eyes}}.
% Per coaches' feedback, essential metadata include scores, rally count, match duration, errors and winners count, and length of each rally. 



\para{G2. Collecting data from videos automatically.}
To avoid tedious manual data collection from users, our tool must provide the necessary data, including summary statistics and annotation of critical shots (i.e., winners and errors). This data should be automatically collected without user input during the match analysis.
% \jui{The goal says it needs to be "automatic", but our preprocessing is not automatic.}
   
\para{G3. Integrating abstract data with game contexts.} 
% \zt{merge abstract data with the physical context} 
% to support comprehensive analysis and communication.} 
Even though summary data can help coaches identify patterns in the game, 
it is important to investigate the actual game moments in the video to verify observations and analyze root causes, as well as to present the insights to players.
\re{Similar to the concept of "Search, show context, expand on demand” on large graphs~\cite{van2009search}, the user goal is to search for a meaningful context.}
Our tool should provide an easy transition between statistics and videos to support iterative analytical reasoning and communication.

\para{G4. Visualizing spatial data in 3D space.} 
3D data should be 
visualized within a 3D space to support an accurate interpretation of 
their spatial attributes, e.g., shot speed and trajectory. 
In addition, our visualizations should support an analysis from flexible viewpoints to enhance users' spatial perception and support objective perspectives. 


\subsection{Task Abstraction} 
\label{sec:task_abstraction}
We abstract six analytic tasks users perform when analyzing matches. Currently, users have to manually gather summary data to identify rallies of interest and manually navigate to each rally to extract insights.


\para{T1. Identify the rallies of interest based on game summary data.}
 % \zt{a}
 % A user first needs a game breakdown to obtain an overview of the match (e.g., game length, winning rallies). Then, they often want to filter a set of interesting rallies based on the metadata of the games.
Users first obtain an overview of the match performance at the game level (e.g., game length, scores).
Based on the game metadata, they can focus on a subset of interesting rallies, such as the winning rallies by their player in the first game.
% \textit{Gap: There is currently no direct way to filter and navigate to rallies based on game information.}

\para{T2. Identify the rallies of interest based on rally summary data.}
 % \zt{b}
%  In addition to the metadata of the game,
% the user often wants to obtain an overall impression of the rallies (e.g., playing style, rhythm) from a shot level,
% such as the ratio of winner shots and distribution of the shot locations.
% Such statistics allow them to further filter rallies with specific patterns for the next step of analysis.
After filtering rallies based on the game summary, users observe patterns of the game (e.g., playing style, competition) at the rally level,
such as the ratio of winner versus error rallies.
Such statistics allow them to identify specific patterns and further filter rallies for deeper analysis.
% \textit{Gap: Currently, users have to watch the entire video and manually collect this data.}
% - Shots and rallies are not categorized and have to be found manually.  

% Identify patterns of winner and error shots.
\para{T3. Gain a statistical overview of rallies of interest.}
 % \zt{c}
Focusing on a set of filtered rallies, users compare the statistics of individual rallies to obtain high-level insights, such as assessing the pace of a rally from shot counts and stress levels from score differences. 
% the user then focuses on analyzing the winner and error shots of each rally to derive insights. 
% This requires a dashboard to show the summary statistics for each rally.
% \textit{Gap: Users have to manually count data for each rally with the current tools.}

\para{T4. Gain a spatial overview of rallies of interest.}
 % \zt{d}
 Additionally, users examine spatial and temporal aspects of individual shots among these rallies (e.g., shot trajectory and speed) and compare them against rally summary data (e.g., shot location distributions).
The spatial information can help users gain an in-depth understanding of the shot-level data and form insights on specific rallies.
% Additionally, 
% showing the spatial and temporal aspects of the shots (e.g., trajectories and shuttle speed) and their comparison with the summarized attributes (e.g., shot types and placement distributions)
% in space is essential
% for badminton analysis.
% The spatial information can help the users can gain an in-depth understanding of the data and derive more insights.
% \textit{Gap: Current methods do not present spatial and dynamic data in 3D space. }

\para{T5.  Investigate game details of specific shots.}
 % \zt{e}
 \re{
 After gaining an overview, users can dive into rallies to examine game details, such as player movement and shot sequence, to get deeper insights into a player's performance.
 This often requires users to watch the game moment multiple times from different angles (e.g., player vs. opponent).}
% Once some insights are observed, 
%the user dives into the respective game moments of the rally to examine game details missing from the summary statistics, such as player movement and shot sequence, to verify the observations and dive deeper into the cause of the performance. 
%This often requires the user to watch the game moment multiple times from different angles (e.g., player vs. opponent).
% Once some patterns are observed, 
% the user dives into the respective game moments of the rally to examine game details missing from the summary statistics, such as player movement and strategy, to contemplate deeper into the cause of the performance. 
% This often requires the users to watch the game moment from different angles.
% \textit{Gap: When watching a video, the users cannot interpret dynamic shots and player movement from different angles.}

%%%% shot level
\para{T6.  Verify insights across rallies.} 
 % \zt{c}
Before concluding their analysis, users need to cross-validate other rallies with similar or contrasting patterns to update and verify their insights.
Thus, users need to efficiently navigate to other game moments based on observed patterns.
% \textit{Gap: With a mainstream video player, it is tedious to navigate and replay specific game moments based on their data patterns.}
% Navigating and replaying specific game moments in the video is tedious, which causes high mental load to compare summary statistics with game details. }
% - It is time-consuming to compare and organize video clips to support insights.
