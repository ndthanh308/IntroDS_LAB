% !TEX root = ../main.tex

\section{Formative Study with Olympian Coaches & Players}
\tl{WIP}
% We applied a user-centered design process to develop VIRD and involved target users (i.e., professional coaches and players) at every design stage. 
% Section \ref{sec:goal_task_analysis} presents our process of obtaining design requirements based on expert interviews. 
% Section \ref{sec:vird} presents VIRD's design based on three rounds of user testing.
% Finally, section \ref{sec:user-study} presents the evaluation of VIRD in analyzing actual matches in case studies.

% Note that due to the specific domain we target, i.e., high-performance badminton coaching, our design is guided by a small number of domain experts involved in the study. While this is the nature of professional sports, we discuss the implications of conducting research with professional athletes in Sec.~\ref{sec:proathletes}.

% \jui{unprofessional question here: should this section really be called "Design Requirement"? Doesn't seem to match what comes next... But maybe it is a field thing..}

% We interviewed 5 badminton professionals on their coaching experiences to obtain insights into gaps in their current video analysis. 
% We iterated with 3 coaches to propose an ideal match analysis workflow and conducted goal and task analysis to form design requirements. 

% \jui{One thing I think will be good to clarify here or in intro is why we have so few interviewees -- due to our focus on high level players.}

\subsection{Procedures}
\label{sec:formativestudy}
To understand current practice and identify gaps in badminton coaching, we interviewed five professional badminton players and coaches (I1-I5; M = 2, F = 3; Age:
30-40). All of them are former Olympic-level players representing Canada, Taiwan, and the US. All of them have at least 10 years of player experience and four became official coaches after their playing careers with 2 to 15 years of coaching experience.

We conducted 1-hour semi-structured interviews online to elicit the interviewee's background, overall coaching workflow, video and data usage in their coaching, and how they evaluate the player's performance in the video analysis. Finally, we asked interviewees to analyze a short match video to demonstrate their typical analysis workflow. 

All interviews were transcribed and analyzed using affinity mapping. Our analysis focused on understanding the current badminton coaching practice and identifying gaps in their match analysis workflow. 


\subsection{Findings and Discussions}
We present findings on coaching workflow, the current practice of match analysis, and the usage of videos. Finally, we highlight the pain points of performing match analysis on videos.

\subsubsection{Coaching Workflow}
% \para{Effective analysis and communication are essential for coaching, regardless of the method.}

\para{Coaching resources and styles vary widely.}
We observed coaching practices vary widely among the interviewees. 
This variation can be explained by a few variables. First, the level of resources available to a player can vary significantly depending on their level of competition, with high performers or older players typically receiving more attention. 
% 
Second, the coach-player relationship 
% (e.g., familiarity, personalities)
can impact the dynamics and ultimately the outcome of the coaching, as emphasized by a coach that \textit{``you really have to understand the individual and know how they respond to certain things''} (I2). 
% 
Third, coaches' own preference and familiarity with the available technology stack shape their coaching style. 
Some coaches use data analysis (e.g., notational analysis) to reveal errors and shot patterns (I1, I2), while others gain insights on playing styles and shot quality from match videos (I3, I4).
% Some coaches perform data analysis (i.e., notational analysis) to reveal errors and shot patterns (I1, I2), while others obtain general insights on playing styles and shot quality from the match videos (I3, I4). 
% 
Finally, the player's skill level impacts the instruction types. 
Novices receive guidance on basic tactics and pose correctness, while elite athletes focus on tactical-level instruction.
% Less experienced players will get instructions on basic tactics and pose correctness, whereas elite athletes get instructions at a tactical level.
%At lower level more instructions will be given on basic tactics and pose correctness, whereas at higher level, the instructions are given at a tactic level.

%Through coaching, players receive guidance on the areas they need to focus on to improve their game throughout their careers. However, the level of resources available to players can vary significantly depending on the level of competition.
%Coaches often have limited time to spend with lower-ranked or younger players compared to the high-performance groups.
%Particularly, the individual coach-player relationship has a huge impact on how a player can benefit from coaching. Each coach has a different style and 
%players' needs for coaching evolve as they grow and develop (I1-I5). 
%From a coach's perspective,
%\textit{``you really have to understand the individual and know how they respond to certain things''} (I2). Some coaches perform data analysis (i.e., notational analysis) to reveal errors and shot patterns (I1, I2), while others obtain general insights on playing styles and shot quality from the match videos (I4).
%Players' needs for coaching also change from basic tactics at a lower level to studying the game at a higher level. 

\para{Coaching involves a common two-step process, requiring effective analysis and communication.}
Despite the difference in coaching style, the overall coaching process includes an analysis and a communication step.
Coaches or players take notes from watching the match, analyze the notes to obtain insights, and communicate and discuss the findings together. 
% For example, the coach watches the game to synthesize insights, and discusses them with the players to develop their future game strategy, e.g., 
With more resources, coaches can share their insights with the player,
\textit{``I have to spend 30 minutes per player writing things down, and another 1 hour to review notes with them to show them here’s what happened''} (I2). 
In other cases when access to the coach is limited, 
% when less resources are available to individual players,
the players will initiate the contact and request coaching on areas to improve (I3, I5) and 
% as the access to one's coach is sometimes limited, 
may also have to help coach each other (I1, I3, I4). This practice is helpful for players to develop strategic insights towards the game, \textit{``having that coaching lens allows you to see what patterns they scope out, or how they diagnose a problem''} (I3). 
Therefore, analyzing matches for coaching is crucial for both coaches and players.
% \jui{should we point out that our users, therefore, CAN be coach and players?}

% \para{The key to effective coaching is two-way communication and collaboration.}
% Nonetheless, 
% the key to effective coaching is communication and collaboration, enabling coaches to develop personalized strategies for players. 
% % the most effective coaching builds upon communication and collaboration to develop a personalized strategy for players. 
% As one player put it, \textit{``most of the best coaches watch [the videos] with you, and point out the pattern again and again''} (I3). 




\subsubsection{Current Workflow for Match Analysis}
\label{sec:gaps}
% Based on the interview results, we identified gaps in the current coaching workflow and propose a top-down analysis approach.
To sample their current video analysis process, we asked coaches to analyze a badminton game for 10 minutes and think-aloud their observations and coaching advice. 

% \para{Current workflow.} 
\para{Coaches obtain instant insights from reviewing matches but require cross-referencing to verify the insights.}
We observed that these high-performance coaches can identify immediate insights on the opponent's playing style and weakness the first time watching the game. For example, \textit{``She's a lefty, and her stance is a little bit untraditional. She tends to lean more to her left side so she can have a big forehand''} (I3), or in another game, \textit{``The blue guy always returns with a backhand technique. I can tell that Blue guy is very confident and comfortable in his backhand''} (I2).
However, to give their player coaching advice, coaches all requested to watch more films in order to verify their observations with more complete data,
\textit{``Right there we saw all of her forehand shots were straight mostly. But don't know until we watch more''} (I3). Further analysis is required, \textit{``I’ll switch and change more into his forehand and see if he’s more aggressive or less''} (I2). 

As we found in the interviews, the current workflow is primarily a \emph{bottom-up approach}, which involves sequentially watching the game and taking notes, until some insights emerge and are then verified.


\noindent
\textbf{Gaps.} We found gaps in data collection and presentation in the current analysis workflow.
%to support match analysis for coaches. 
On the one hand, it is difficult to  observe, take notes, extract statistics, and develop strategies at the same time. Therefore, coaches have to watch films multiple times. For each watch, they focus on different aspects, one at a time, such as opponent versus their player,
% number of shots in a rally,
and winner shots versus unforced errors. They manually note down any of these findings,
\textit{``As you can tell that process is very manual and takes up a lot of time''} (I2).
On the other hand, with limited time and resources to review footage, coaches and players might choose to only look at the most important parts, which leads to incomplete analysis and potentially less effective communication. 
Coaches have to cross-reference multiple videos to make confident conclusions, 
\textit{``Is this just an outlier or some of these random matches where we didn't do well, or concrete things that we need to work on?''} (I2).
More experienced coaches might do better at this sifting process, but they might also miss important insights. 
In addition, without ample video evidence, sometimes it can be hard to convince the players of a particular finding,
% \jui{can we quote one here?}.
\textit{``The kids don’t realize that they make a lot of unforced errors. If they knew what the rate was, they'll understand it better''} (I1).
%Coaches will leverage their experiences and observations to point out the most useful and practical part to the players, but it remains challenging for them to communicate insights effectively without concrete evidence such as video.


%Because of limited resources in badminton, even at the high-performance level, coaches we interviewed have to manually annotate everything, such as the number of shots in a rally, winners and unforced errors. Besides, they focus on different aspects one at a time from watching the raw footage (e.g., opponent vs. player), leading to much time on navigating and comparing the videos with collected data. 
%On the other hand, with limited time and resources to review films, coaches and players choose to only look at the most important parts, which leads to less comprehensive analysis and less effective communication. Coaches will leverage their experiences and observations to point out the most useful and practical part to the players, but it remains challenging for them to communicate insights effectively without concrete evidence such as video.


\subsubsection{Usage of Videos in Match Analysis}
\para{Video analysis is core to coaching despite its limitations.} 
Both players and coaches analyze videos to evaluate performance and derive insights. Videos are crucial as they help players become aware of their playing technique and style, and allow players to create a mental model of other players.
% why videos
Players are often told to record their own match and watch the videos multiple times, \textit{``Most coaches recommend we watch it several times and break it down to focus on one thing at a time''} (I3). Coaches also rely on videos to communicate insights to players. \textit{``They won’t understand what I am saying unless they see it physically''} (I2).

% video analysis focus
The focus of video analysis depends on opponents and individual matches.  During the competition, they analyze 1 to 3 games about their opponent in depth to scout the opponent's overall style and occasions when errors happen (I2, I3, I4). 
Players also compare their performance in tournaments and training to track improvement. 
Specifically, they identify patterns
% extract useful information 
from selected rallies, 
such as shot types,
% (e.g., offensive or defensive)
locations, sequence, and body movement,  where
\textit{``Some rallies are more informative and impactful than others, like longer rallies or the ones that repeat every time. I would review those more in depth''} (I3). 
% Important factors in a rally include the used shot types (e.g., offensive or defensive), shot locations, sequence of shots, and change in body movement. 
Some coaches manually extract summary statistics (and spend even more time) to reveal patterns quantitatively to better communicate with the players (I1, I2), e.g., \textit{``70\% of time when you do this, you win the point.''}   

% video analysis resource
All interviewees agreed that video analysis is critical but time-consuming. \textit{``If the match is 30-40 minutes, it took 3 to 5 hours to rewatch and discuss with your coach''} (I4). Coaches found it beneficial to analyze multiple matches of their players over a period of time (I1, I2, I4).
% \textit{``Right now I have annotated one or two matches because I don’t have time to do everything. 
\textit{``If you had like 20 matches that you played for an entire year, then you have a better idea where you can dictate training''} (I1). 


When asked about the limitation of analyzing matches with videos, 
some coaches expressed that videos 
recorded by a single camera may lose fidelity on real game aspects, 
 % especially those recorded by a single camera in badminton tournaments, can lose fidelity on real game aspects, 
 such as environmental conditions (e.g., wind), shot timing (\textit{``Speed seems slower in the video''}), and player reactions (e.g., emotional cues).
% 
% Furthermore, some coaches express that videos from a single camera itself are limiting on how accurate information can be obtained from the analysis, such as court condition, player movement, shot timing and speed \jui{I don't understand this statement. Is it better to say single camera videos might alter perceptions of the game?}, \textit{``Speed seems slower in the video''}(I2). 
Some coaches found 360-degree videos useful for comparing players and their opposition, as 
\textit{``you can 
see both sides and 
% of things happen and their interaction,
% , and you can see the interaction 
like 
how the player reacts to the opponent in real-time''} (I1). Slow-motion or zoomed views also help to break down the technique and provide objective viewpoints (I1, I3), \textit{``when you're hitting the shot, you only know how it feels, but you can't see how it looks''} (I3). 


\subsection{Summary}
Even though videos are core to coaching, currently the value obtained from these videos is sub-optimal due to the limitation on coaching resources. 
Single-viewpoint videos also have limitations on the spatial perception of the real game.
Experts have developed ways to enhance the coaching workflow, such as using notational analysis to collect stats or capturing 360 videos to allow multi-angle analysis; they also leverage tools to manage videos and communicate with the players, such as YouTube and Clutch~\cite{clutch}. 
% \jui{the 360 thing seems out of place here}. 
However, there is currently no ideal solution for efficiently conducting comprehensive video analysis.