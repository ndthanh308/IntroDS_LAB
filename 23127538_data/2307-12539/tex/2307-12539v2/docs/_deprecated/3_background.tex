% !TEX root = ../main.tex

% \section{Background}

% \subsection{Badminton Match Analysis and Coaching}
% match analysis
% Professional badminton players compete at the highest level, and the outcomes of their matches are often decided by the narrowest of margins. 
% As a result, players and coaches have come to rely heavily on match analysis as a means of identifying key patterns and strategies that can be used to gain a competitive advantage against their opponents. 
% % coaching
% Different from training, coaching provides crucial guidance for players to develop training and playing strategies by extracting and prioritizing valuable insights from match analyses. 
% The success of badminton coaching relies on several factors, including the coach's expertise, resources available to collect and analyze data, and the amount and quality of the coaching time with the player.
% \jui{This paragraph seems largely overlapping with intro. Consider removing or at least shortening}

% rule: https://olympics.com/en/news/badminton-guide-how-to-play-rules-olympic-history
% A badminton match consists of the best-of-three games. 
% A game is won by the first side to win 21 points or two clear points if the score is 20-20. 
% Each point is earned by winning a rally, which involves multiple shots (strokes). 
% % The side that won the previous rally serves first in the next rally. 
% % A point is scored when the bird (shuttlecock) hits the ground in the opponent's court.
% Players change sides after each game and at the midway point of the third game, when one side reaches 11 points. 
% A break is allowed at the midpoint of each game. A typical match takes 40-50 minutes, but the duration can vary depending on the players and circumstances. \jui{the only useful thing in this paragraph seems to be the duration of a typical game. Can we just add it to the intro and remove this paragraph?}

% \subsection{CV Techniques}

% \subsubsection{Pose and shape estimation}
% Using statistical models and a top-down approach, CLIFF (Carrying Location Information in Full Frames)\cite{li-2022} performs a 3D human pose and shape estimation from a single RGB image. The first step is the detection of players. This is done using YoloV3 \cite{yolov3}, a fast and accurate model that predicts object class and bounding boxes by dividing the image into a grid of cells. Then, the prediction of the SMPL parameters from image features is done using regression-based methods. This estimation is used to obtain the human body meshes through a linear function, resulting in a complete 3D representation of the players. \jui{Why do we need these details? Isn't it just applying off-the-shelf technique? If so, we should focus on WHY we need the pose and for the implementation details, just cite the paper. Same for the trajectory reconstruction; my view is we need only a very short description of what we did there, too.}

% \section{Methodologies}
% We applied a user-centered design process to develop VIRD and involved target users (i.e., professional coaches and players) at every design stage. 
% We interviewed experts to identify the design requirements (Sec.\ref{sec:goal_task_analysis}}), 
% and iterate our design decision based on three rounds of  user testing (Sec.\ref{sec:vird}), 
% and finally, we evaluated the use of our tool in analyzing actual matches in a user study (Sec.\ref{sec:user-study}).
% Note that due to the specific domain we target, i.e., high-performance badminton coaching, our design is guided by a small number of domain experts involved in the study. While this is the nature of professional sports, we discuss the implications of conducting research with professional athletes in Sec.~\ref{sec:proathletes}.
% \jui{This is important. However, if we remove 3.1-3.2 (or combine them with intro, then we can put this at the end of the intro as well?}