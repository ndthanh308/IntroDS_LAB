% !TEX root = ../main.tex

\firstsection{Introduction}
\maketitle

% importance of video analysis in badminton coaching and training

Professional badminton players compete at the highest level, and the outcomes of their matches are often decided by the narrowest of margins. 
Players and coaches have come to rely heavily on match analysis to identify key patterns and strategies that can be used to gain a competitive advantage against their opponents. 
As a result, match analysis is a crucial part of badminton coaching, which requires a combination of data analytics, domain expertise, 
and coaching experiences to develop useful winning strategies and training plans for the player.

% coaching
Different from training, coaching provides crucial guidance for players to develop training and playing strategies by extracting and prioritizing valuable insights from match analyses. 
The success of badminton coaching relies on several factors, including the coach's expertise, resources available to collect and analyze data, and the amount and quality of the coaching time with the player.
% 
%% challenges
% At the competitive level, coaches and players have to study the game and develop strategies against different opponents through video analysis. 
To analyze one game, a coach has to watch hundreds of rallies, each of which contains tens of shots, while jotting down notes on a piece of paper. To make things worse, to develop and confirm insights, they typically have to do this for each video from start to finish a dozen of times. This painful process requires a significant amount of time (3 to 5 times the video length) and a lot of mental effort. This in turn compresses the time coaches can spend with their players. Combining it with a lack of effective means to communicate these tactical insights to each player, we found that there is a huge demand for more efficient video analysis and coaching tool.


%However, current badminton coaching is still very manual and lacks support for iteration. 
%The current video analysis process requires repetitive video watching and manual note-taking. 
%For individual sports like badminton, coaches often have limited access to data, means of expression, and coaching time with individual players. 
%This causes a gap in video analysis and communication between coaches and players due to a lack of supporting coaching materials.

% Overview of current methods and tools for video analysis

\jui{It might make sense to have a paragraph here to contrast what existing work is doing in more qualitative ways, and then specify the gaps in these work in addressing our problems}

% Statement of the problem and research questions

In this work, 
we propose an integrated immersive analysis pipeline for badminton matches based on close collaboration with top Olympic-level badminton professionals. 
We aim to answer the following three questions from our study:
\begin{itemize}
\item What data are required for analyzing matches and developing coaching insights? 
\item What is the ideal coaching workflow for badminton video analysis and communication?
\item How to design an integrated video analysis tool to support badminton coaching for professional coaches and players?
\end{itemize}

Using computer vision (CV) techniques, 
we extract rally and shot-level data from a match video and reconstruct dynamic trajectories in 3D. 
We design a top-down approach where coaches can overview all shots in a summary view, and examine individual rally videos in a detailed view. 
Through 3D immersive visualization, coaches can interactively analyze the spatial data with visceral experiences of dynamic shots from the player's perspectives and perform video analysis with efficient playback and searching.

% WHY VR? <--- better to say it upfront than having reviewers complain why we didn't compare with 2D.
To propel the immersive
analytics in sports, we experiment the use of immersive 3D visualizations and embodied interaction in VR. 
Our case studies suggest (VR benefits for sports analytics)

% Objectives and contributions of the proposed VR tool

\jui{Another question I have is whether we can elevate the framing of the paper to include other competitive individual sports like Tennis, and then specify why and how we focus on badminton.}


Our contribution is three folds, 
(1) a formative study with Olympic-level badminton athletes to characterize current workflow and gaps in match analysis for coaching, 
(2) an end-to-end immersive video analytic tool for badminton coaching, VIRD, and (3) case studies of VIRD with high-performance badminton experts on analyzing real matches.
We concluded with research implications
for applying immersive analytics to sports coaching with automatic data collection and a human-in-the-loop analysis approach.