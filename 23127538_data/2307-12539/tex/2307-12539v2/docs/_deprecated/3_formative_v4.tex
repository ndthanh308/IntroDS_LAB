% !TEX root = ../main.tex

\section{Formative Study with Olympian Coaches & Players \zt{WIP}}
% We applied a user-centered design process to develop VIRD and involved target users (i.e., professional coaches and players) at every design stage. 
% Section \ref{sec:goal_task_analysis} presents our process of obtaining design requirements based on expert interviews. 
% Section \ref{sec:vird} presents VIRD's design based on three rounds of user testing.
% Finally, section \ref{sec:user-study} presents the evaluation of VIRD in analyzing actual matches in case studies.

% Note that due to the specific domain we target, i.e., high-performance badminton coaching, our design is guided by a small number of domain experts involved in the study. While this is the nature of professional sports, we discuss the implications of conducting research with professional athletes in Sec.~\ref{sec:proathletes}.

% \jui{unprofessional question here: should this section really be called "Design Requirement"? Doesn't seem to match what comes next... But maybe it is a field thing..}

% We interviewed 5 badminton professionals on their coaching experiences to obtain insights into gaps in their current video analysis. 
% We iterated with 3 coaches to propose an ideal match analysis workflow and conducted goal and task analysis to form design requirements. 

% \jui{One thing I think will be good to clarify here or in intro is why we have so few interviewees -- due to our focus on high level players.}

\subsection{Procedures}
\label{sec:formativestudy}
To understand current practice and identify gaps in badminton coaching, we interviewed five professional badminton players and coaches (I1-I5; M = 2, F = 3; Age:
30-40). All of them are former Olympic-level players representing Canada, Taiwan, and the US. All of them have at least 10 years of player experience and four became official coaches after their playing careers with 2 to 15 years of coaching experience.

We conducted 1-hour semi-structured interviews online to elicit the interviewee's background, overall coaching workflow, video and data usage in their coaching, and how they evaluate the player's performance in the video analysis. Finally, we asked interviewees to analyze a short match video to demonstrate their typical analysis workflow. 

All interviews were transcribed and analyzed using affinity mapping. Our analysis focused on understanding the current badminton coaching practice and identifying gaps in their match analysis workflow. 


\subsection{Findings and Discussions}
Overall, we observed coaching practices vary widely among the interviewees. 
\zt{common points: 1) two steps, 2) videos}
% This variation can be explained by a few variables. First, the level of resources available to a player can vary significantly depending on their level of competition, with high performers or older players typically receiving more attention. 
% Second, the coach-player relationship 
% can impact the dynamics and ultimately the outcome of the coaching, as emphasized by a coach that \textit{``you really have to understand the individual and know how they respond to certain things''} (I2). 
% Third, coaches' own preference and familiarity with the available technology stack shape their coaching style. 
% Some coaches use data analysis (e.g., notational analysis) to reveal errors and shot patterns (I1, I2), while others gain insights on playing styles and shot quality from match videos (I3, I4).
% Finally, the player's skill level impacts the instruction types. 
% Novices receive guidance on basic tactics and pose correctness, while elite athletes focus on tactical-level instruction.
Despite the difference in coaching style, 
the coaching process typically involves two steps: analysis and communication.
Coaches and players watch matches, take notes, analyze them for insights, and discuss their findings. 
With more resources, coaches can share their insights with players, although this requires a significant time investment (I2). 
Alternatively, in cases where access to a coach is limited, 
players may initiate contact and request coaching on areas to improve (I3, I5)
or help coach each other (I1, I3, I4). 
This practice is useful for developing strategic insights towards the game. 
Therefore, analyzing matches for coaching is crucial for both coaches and players.

Both players and coaches analyze videos to evaluate performance and derive insights. 
Videos are crucial as they help players become aware of their playing technique and style, 
and allow players to create a mental model of other players.
% why videos
Players are often told to record their own match and watch the videos multiple times, \textit{``Most coaches recommend we watch it several times and break it down to focus on one thing at a time''} (I3). 
Coaches also rely on videos to communicate insights to players. 
\textit{``They won’t understand what I am saying unless they see it physically''} (I2).

% video analysis is tedious
All interviewees agreed that video analysis is critical but time-consuming. 
\textit{``If the match is 30-40 minutes, it took 3 to 5 hours to rewatch and discuss with your coach''} (I4). 
Coaches found it beneficial to analyze multiple matches of their players over a period of time (I1, I2, I4).
\textit{``If you had like 20 matches that you played for an entire year, then you have a better idea where you can dictate training''} (I1). 


In this work, we focus on the analysis step of badminton coaching
and summarized the current practice and pain points in match analysis.

\subsubsection{Current Bottom-Up Workflow is Time-Consuming}
\label{sec:gaps}
We observed that the coaches we interviewed followed a \emph{bottom-up} approach to generate their insights. 
They began by scanning videos to catch insights into the opponent's playing style and weaknesses. 
For example, during video scanning, I3 immediately noticed that \emph{``She's a lefty ... [so] She tends to lean more to her left side for a big forehand.''} 
Once they formed an insight hypothesis, they watched more videos to search for similar patterns and verify their observations. 
I3 explained, \emph{`` ... we saw all of her forehand shots were straight mostly, but don't know until we watch more.''} 
In summary, coaches repeatedly watched games until insights emerged, then re-watched the games to collect more evidence. 
However, this workflow is significantly time-consuming,
\zt{I feel we need a sentence to transit to the top-down workflow. maybe quote ``Overview first, details on demand''}
leading us to the first design goal:

\begin{quote}
\textbf{G1. Providing a top-down analysis workflow.} 
\zt{@Tica, plz refine and flesh out this a bit}
Summary data enables an immediate overview of the match and support effective exploration to discover region-of-interest for analysis. 
Per coaches' feedback, essential metadata include scores, rally count, match duration, errors and winners count, and length of each rally. 
\end{quote}

\subsubsection{Collecting Data from Videos is Tedious}
During the current workflow,
coaches manually collect summary statistics (and spend even more time) to reveal patterns quantitatively to better communicate with the players (I1, I2), e.g., \textit{``70\% of time when you do this, you win the point.''}
Players also compare their performance in tournaments and training 
by counting patterns from selected rallies, such as shot types, locations, sequence, and body movement.
However, it is difficult to pay attention to multiple metrics and patterns simultaneously.
Therefore, coaches and players have to watch films multiple times and
focus on different aspects, one at a time, such as opponent versus their player,
and winner shots versus unforced errors. 
Such a manual process \emph{``takes up a lot of time''} (I2).
Consequently,
coaches and players might choose to only look at the most important parts, 
which leads to incomplete analysis and potentially less effective communication. 
Therefore, it is essential to incorporate the design goal that

\begin{quote}
    \textbf{G2. Collecting data from videos automatically.}
    \zt{@Tica, add one more goal here to talk about auto data collection.}
\end{quote}

% with limited time and resources to review footage, 
% They manually note down any of these findings, 
% \textit{``As you can tell that process is very manual and 
% The focus of video analysis depends on opponents and individual matches.  
% During the competition, they analyze 1 to 3 games about their opponent in depth to scout the opponent's overall style and occasions when errors happen (I2, I3, I4). 

% \textit{``Some rallies are more informative and impactful than others, like longer rallies or the ones that repeat every time. I would review those more in depth''} (I3). 

%% collecting data from video
% On the one hand, it is difficult to observe, take notes, extract statistics, and develop strategies simultaneously. 
% For each watch, they 

% video analysis focus



\subsection{Presenting Insights with Videos is Necessary}
%% presenting data with video
\zt{need presenting insights with videos}
In addition, without ample video evidence, sometimes it can be hard to convince the players of a particular finding,
\textit{``The kids don’t realize that they make a lot of unforced errors. If they knew what the rate was, they'll understand it better''} (I1).

Coaches have to cross-reference multiple videos to make confident conclusions, 
\textit{``Is this just an outlier or some of these random matches where we didn't do well, or concrete things that we need to work on?''} (I2).
More experienced coaches might do better at this sifting process, but they might also miss important insights. 


\textbf{G2. Integrate static data and video to support comprehensive analysis.} 
\zt{merge abstract data with the physical context} 
Even though summary data can help coaches identify patterns in the game, it is still important to investigate the actual game moments in the video to verify observations and analyze the root cause, as well as present the insights to players. 
Providing an easy transition between statistics and videos is thus crucial to support iterative analytical reasoning and communication.


\subsubsection{2D displays are insufficient}
% \para{Video analysis is core to coaching despite its limitations.} 

\zt{3D reconstruction}
When asked about the limitation of analyzing matches with videos, 
some coaches expressed that videos 
recorded by a single camera may lose fidelity on real game aspects, 
such as environmental conditions (e.g., wind), shot timing (\textit{``Speed seems slower in the video''}), and player reactions (e.g., emotional cues).
Some coaches found 360-degree videos useful for comparing players and their opposition, as 
\textit{``you can see both sides and like 
how the player reacts to the opponent in real-time''} (I1). Slow-motion or zoomed views also help to break down the technique and provide objective viewpoints (I1, I3), \textit{``when you're hitting the shot, you only know how it feels, but you can't see how it looks''} (I3). 


\textbf{G3. Visualize spatial and temporal data within context.} 3D data (e.g., shot trajectory or player position) should be represented within the appropriate context to support an accurate interpretation of their spatial and temporal attributes, such as speed and timing. In addition, our visualizations should support experts analyzing from flexible viewpoints to allow accurate spatial perception and objective perspectives. 

%%%%%%%%%%%%%%%%%%%%%%%%%%G1 - G3
% Based on user interviews, we identified that high-performance badminton coaches and players perform match analysis to reveal the strengths and weaknesses of the opponent/player and develop playing or training strategies. However, data collection and presentation gaps exist in their current analysis workflow.

% To support these coaches and players in analyzing matches and communicating insights effectively, we characterized three design goals to support an ideal analytic workflow.


% % 
% % Since match analysis requires investigation into action details like dynamic movement and player pose, it is important to allow easy reference to the actual video when providing statistics. \jui{the conclusion is ok, but the reasoning seems weak -- what part of the analysis requires action details? Did we support this statement in the formative studies? We need to explain things better here and potentially re-word G2.}

% % \jui{same here. I feel like this is the right goal, but insufficiently supported in the studies we presented so far. Need to adjust the formative studies part to accommodate this statement.}

% From the user's perspective, we abstract five visualization tasks that are performed throughout the match analysis. 


\subsection{Summary}
Even though videos are core to coaching, currently the value obtained from these videos is sub-optimal due to the limitation on coaching resources. 
Single-viewpoint videos also have limitations on the spatial perception of the real game.
Experts have developed ways to enhance the coaching workflow, such as using notational analysis to collect stats or capturing 360 videos to allow multi-angle analysis; they also leverage tools to manage videos and communicate with the players, such as YouTube and Clutch~\cite{clutch}. 
% \jui{the 360 thing seems out of place here}. 
However, there is currently no ideal solution for efficiently conducting comprehensive video analysis.


\noindent
\textbf{T1.  Break down a match into rallies.}
A user first needs a breakdown of the match to obtain necessary metadata (e.g., winning rallies, game length) and easy access to the rallies of interest. \textit{Gap: There is currently no direct way to navigate to or filter rallies based on game information.}

\noindent
\textbf{T2.  Extract statistics from videos at rally and shot levels.}
The user explores summary data, such as rally lengths and shot locations, to obtain an overall impression of the game (e.g., playing style, rhythm). The statistics allow them to obtain an overview, which leads to identifying specific patterns in the next steps. \textit{Gap: Currently, users have to watch the entire video to annotate and collect this data.}

\noindent
\textbf{T3.  Identify patterns of winner \& error shots.}
% The user then focuses on rallies that led to winner and error shots to derive insights based on summarizing and comparing the identified patterns. 
The user then focuses on analyzing winner and error shots to derive insights. 
Necessary aspects to consider include spatial and temporal analysis of shot trajectories and player poses, and their comparison with the summarized attributes such as shot types and distributions. \textit{Gap: Shots and rallies are not categorized and have to be found manually.  
Spatial and dynamic data are not presented in-situ and in 3-dimensional perspectives. }

\noindent
\textbf{T4.  Investigate game details of a rally.}
% \jui{I think they look at more than just the pose etc in the video. Should we broaden this task to obtaining more detailed information in the video but missing from our higher-level summary and statistics.}
Once some patterns are observed, the user dives into the respective game moments of the rally to examine game details that are missing from the summary statistics, such as player movement and strategy, to contemplate deeper into the cause of the performance. 
% \jui{In general, we should proabably lay the foundation better in earlier sections on answering: why is video important and what is missing from our reconstruction}
\textit{Gap: Navigating and replaying specific game moments in the video is tedious, which causes high mental load to compare summary statistics with game details. 
In addition, interpreting dynamic shots and player movement is constrained by the video angle.}
% \textit{Gap: Interpretation of the spatial and temporal aspects is constrained by the video angle.}
% \jui{not sure if this is the right gap}

\noindent
\textbf{T5.  Generate and verify insights across rallies.} Before concluding  coaching advice, the user cross-references other rallies with similar or contrasting patterns to update and verify insights. \textit{Gap: It is time-consuming to compare and organize video clips to support insights.}

