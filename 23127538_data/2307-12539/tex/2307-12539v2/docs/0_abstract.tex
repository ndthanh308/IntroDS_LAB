% !TEX root = ../main.tex

%% Abstract section.
\abstract{
% 
Badminton is a fast-paced sport that requires a strategic combination of spatial, temporal, and technical tactics. 
To gain a competitive edge at high-level competitions, badminton professionals frequently analyze match videos to gain insights and develop game strategies. 
However, the current process for analyzing matches is time-consuming and relies heavily on manual note-taking, due to the lack of automatic data collection and appropriate visualization tools. 
 As a result, there is a gap in effectively analyzing matches and communicating insights among badminton coaches and players. 
% 
This work proposes an end-to-end immersive match analysis pipeline designed in close collaboration with badminton professionals, including Olympic and national coaches and players. 
%We present VIRD, VR Bird immersive analysis tool, that features 3D reconstructed game views of the match video and interactive analysis workflow in an immersive environment.
We present \emph{VIRD}, a VR Bird (i.e., shuttle) immersive analysis tool, that supports interactive badminton game analysis in an immersive environment based on 3D reconstructed game views of the match video.
We propose a top-down analytic workflow that allows users to seamlessly move from a high-level match overview to a detailed game view of individual rallies and shots, using situated 3D visualizations and video. 
We collect 3D spatial and dynamic shot data and player poses with computer vision models and visualize them in VR. Through immersive visualizations, coaches can interactively analyze situated spatial data (player positions, poses, and shot trajectories) with flexible viewpoints while navigating between shots and rallies effectively with embodied interaction. 
We evaluated the usefulness of VIRD with Olympic and national-level coaches and players in real matches. Results show that immersive analytics supports effective badminton match analysis with reduced context-switching costs and enhances spatial understanding with a high sense of presence. 


% High-performance badminton experts, such as Olympic-level coaches and players, often analyze match videos to acquire insights and develop game strategies. 
% However, the current video analysis process is time-consuming and relies on manual note-taking, leading to data collection and presentation gaps that hinder comprehensive match analysis and communication between coach and player.
% % However, the current video analysis process requires repetitive video watching and manual note-taking. For individual sports like badminton, coaches often have limited access to data, means of expression, and coaching time with individual players. This leads to data collection and presentation gaps to support comprehensive match analysis from insight generation to communication. 
% % 
% In this work, we propose an end-to-end immersive match video analysis pipeline based on close collaboration with badminton professionals.  
% We present VIRD, a VR immersive analysis tool, featuring a top-down analytic workflow with smooth interfaces between summary data and detailed game view with 3D visualizations and video.
% Spatial and dynamic shot data and player poses were collected with CV-based models and reconstructed in 3D. 
% Through immersive visualizations, coaches can interactively analyze spatial data with flexible viewpoints while navigating across shots and rallies effectively with embodied interaction. We evaluated the usefulness of VIRD with experts on actual matches. Results show that immersive analytics support effective video analysis with reduced context-switching costs and enhance spatial understanding with a high sense of presence. We discuss the implications of immersive video analysis for sports coaching.

} % end of abstract