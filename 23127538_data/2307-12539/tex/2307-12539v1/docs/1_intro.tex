% !TEX root = ../main.tex

\firstsection{Introduction}
\maketitle

% Professional badminton players compete at the highest level, 
% and the outcomes of their matches are often decided by the narrowest of margins. 
% Players and coaches have come to rely heavily on match analysis to identify key patterns and strategies that can be used to gain a competitive advantage against their opponents. 
In the highly competitive world of professional badminton, 
coaches and players constantly seek ways to gain an edge over their opponents. 
Detailed match analysis has become indispensable for identifying key patterns, 
developing winning strategies, 
and creating tailored training plans. 
Traditional methods of match analysis require coaches 
to review video footage of matches,
take notes, and identify trends and potential areas for improvement.

%%%% talk about why the traditional methods are bad
%%% "the coach needs to context switch a lot between video and spatial data" + "ineffective communication using the current medium".
However, these traditional methods have significant drawbacks.
% First, traditional methods require coaches to manually annotate videos to collect in-game data about the players,
% such as their statistical summaries, 
% distributions of the shuttle placements,
% and trajectory types.
% Coaches often have to watch countless rallies multiple times, limiting the time they can spend with their players. 
% 
First, traditional methods require coaches to manually annotate videos while analyzing the match. In order to gather insights from match footage, coaches have to watch videos several times to collect data, such as players' statistical summaries, shuttle locations and types, and playing styles. After that, coaches have to compare the summary data with their observations to iterate on their hypotheses, verify insights, and draw conclusions.
This data collection and analysis process can be highly time-consuming and mentally demanding, as coaches have to watch countless rallies multiple times and maintain a high level of attention to detail. While the analysis efforts vary widely, a badminton match can contain 50 to over 100 rallies and the coaches typically spend 3 to 5 hours analyzing a full match video according to our formative study.
% \jui{might be good to cite some real numbers here. How many rallies, how much time they have to spend wrt to video length} 
As a result, this can increase their cognitive load and limit the time coaches spend working with their players.
% 
% Coaches may spend three to five times the duration of a match collecting statistics from crucial game moments, 
% cross-referencing data and video to verify insights, 
% and communicating their advice to the players.
Furthermore, traditional methods struggle to capture and communicate badminton's inherently spatial nature effectively. 
Badminton players use height- and speed-varying tactics to control the pace and aggression of their games, which is harder to perceive in projected 2D videos.
In turn, coaches rely on physical demonstrations to help players conceptualize the insights. The limited coaching time available for high-performance athletes can restrict the amount of information conveyed during coaching sessions.
% 
% this can lead to missing insights or communicating difficulty \jui{can we support this claim? it looks weak and hand wavy}.
% crucial to effective match analysis and coaching.
% Thus, players often require seeing actual movement to understand the concepts. 
% Traditional medium also limits the coach's ability to fully convey these spatial aspects to players.
Finally, traditional methods separate the in-game data from its physical context.
Consequently, coaches often need to switch back and forth between the videos and the collected data,
inevitably increasing the mental effort of the analysis process.
As shown in our formative studies (Sec.~\ref{sec:formative_study}), there is a demand for more efficient match analysis and coaching tools.



In this work, 
we closely collaborated with Olympic badminton professionals
to design VR Bird (VIRD)\footnote{A badminton shuttlecock is also informally called a \emph{bird}.}, a novel interactive coaching tool for badminton match analysis in VR.
To address the limitations of traditional methods, VIRD adopts a top-down analysis approach to support an efficient iterative analysis process. It leverages computer vision (CV) techniques to extract game statistics and 3D shot and player data, and enables situated analysis of spatial and dynamic game data in immersive 3D space with an interactive approach. 
% 
\re{Immersive analytics were found beneficial in supporting 3D data analysis with spatial understanding and immersion~\cite{t_chandler_immersive_2015}.
With its high portability and affordability, 
we chose VR HMD as our targeted platform to design immersive analysis solutions for badminton video match analysis.}


% Taking only a match video as input, 
% VIRD's preprocessing pipeline extracts useful game statistics and 3D shot and player data. 
% Through immersive visualizations and 3D reconstructed game views on top of the original match video,
% VIRD features a top-down analysis approach where coaches can overview all shots in a summary view, 
% and examine individual rally videos in a detailed view. The visual components are outlined in Sec.~\ref{sec:visual_components}. 
% VIRD outperforms traditional analysis methods and tackles the aforementioned drawbacks.
% % 
% First, to eliminate the tedious manual data collection process,
% VIRD leverages computer vision (CV) techniques to 
% extract rally and shot-level data from match videos and reconstruct dynamic 3D trajectories. 
% Second, to reveal the spatial nature of the data,
% VIRD takes advantage of virtual reality (VR) to visualize the data in an immersive 3D space,
% providing a more accurate representation of spatial information and height-related tactics. 
% Third, to support in-situ analysis, VIRD reconstructs the 3D shot and player movements from the video and simultaneously displays the 3D match and 2D video in the immersive space,
% allowing the coach to interactively inspect the data within its physical context.
% These features lead to more efficient match analysis and improved communication of tactical insights. \jui{I will consider shortening this paragraph. VIRD's design is detailed in the later section already. Here in intro we can focus on the high-level message. If it were me, I will just briefly say to address these issues, we do X, Y, Z, and refer readers to the VIRD section for more details. I will also avoid hastily come to a conclusion "VIRD outperforms..." before presenting results.}

To design VIRD, we performed a design study to answer the following three research questions.
First, we conduct a formative study with badminton professionals to understand 
\textit{``What data are required for analyzing matches and developing coaching insights?''} 
Second, we identified gaps and iterated solutions with coaches to answer
\textit{``What is the ideal coaching workflow for badminton video analysis and communication?''}. 
Finally, we conducted a multi-staged user-centered design to address \textit{``How to design an integrated video analysis tool to support badminton coaching for professional coaches and players?''}
We conducted case studies with high-performance badminton experts, including Olympic and national team coaches and players, on match analysis using VIRD. Both coaches and the player were able to perform effective match analyses, and verify and present insights using VIRD with high satisfaction. They leveraged immersive 3D visualizations to generate new insights with concrete evidence, such as observing shot distributions or pinpointing specific game moments, and used VR interaction to accelerate their iteration from hypotheses to insights.  
% leveraging CV-based data collection and immersive analysis approach to accelerate their analysis process. 
% \begin{itemize}
% \item What data are required for analyzing matches and developing coaching insights? 
% \item What is the ideal coaching workflow for badminton video analysis and communication?
% \item How to design an integrated video analysis tool to support badminton coaching for professional coaches and players?
% \end{itemize}

% \jui{Another question I have is whether we can elevate the framing of the paper to include other competitive individual sports like Tennis, and then specify why and how we focus on badminton.}

Our research has four main contributions:
1) a formative study with Olympic badminton professionals to identify gaps in current match analysis workflow for coaching, 
2) a characterization of goals and tasks for badminton match analysis in coaching, 
3) an end-to-end immersive video analysis tool, VIRD, with state-of-the-art CV-based data collection and a VR analytic interface for badminton coaching,
and 4) case studies with high-performance badminton experts to evaluate the usefulness of VIRD.
Our results suggest that applying immersive analytics to sports videos, with CV-based data collection and human-in-the-loop analysis, can be highly effective for sports professionals.


% Through 3D immersive visualization, 
% coaches can interactively analyze the spatial data with visceral experiences of dynamic shots from the player's perspectives and perform video analysis with efficient playback and searching.
% count statistical summaries of the players by annotating rallies and shots.
% Furthermore,
% make extensive notes on the spatial data such as shuttle placements and trajectories,
% and switch back and forth between these data and the videos.
%%%%%% - are we talking about the drawback of manual note-taking
% Additionally,inevitably adding extra mental effort to the 
% the mental effort required to 
% take notes,
% analyze complex patterns,
% and extract meaningful insights can be exhausting. 

% It provides an immersive top-down approach for coaches to overview all shots, 
% examine individual rallies, 
% search and playback videos efficiently,
% and interact with spatial data with visceral experiences from the player's perspective, 
% significantly enhancing the efficiency of match analysis and communication of tactical insights.

% VR can be seen as a superset of traditional desktop environments, 
% meaning that it can encompass and extend the capabilities of traditional methods while also providing additional benefits.
% First, VR enhances spatial understanding, as it allows for the presentation of 3D trajectory data in a ``real 3D'' form, 
% %%% this is not an issue of the existing methods
% Second, it enables coaches to 
% allowing them to analyze trajectories and spatial data from the player's perspective, 
% thereby enhancing their understanding of the game's dynamics. 




% WHY VR? <--- better to say it upfront than having reviewers complain why we didn't compare with 2D.
% To propel the immersive
% analytics in sports, we experiment the use of immersive 3D visualizations and embodied interaction in VR. 
% Our case studies suggest (VR benefits for sports analytics)

% Objectives and contributions of the proposed VR tool
