% !TEX root = ../main.tex

\section{Evaluation}
\label{sec:user-study}
% We designed three case studies to evaluate how well VIRD supports badminton experts analyze match videos. 
% We described the case studies and the computational performance of selected matches.


\subsection{Case Study Design}
\label{sec:case-study-design}
%Due to 
% the scarcity of the target users and 
%the complexity of analysis tasks, 
We conducted in-person case studies~\cite{lam_empirical_2012} with domain experts to evaluate VIRD on match analysis in four aspects: 1) data analysis method, 2) derived insights, 3) useful components, and 4) overall user experience.

\noindent
\textbf{Participants \& Data.} 
% Our study goal is to evaluate VIRD on match analysis in four aspects: 1) data analysis method, 2) derived insights, 3) useful components, and 4) overall user experiences. 
% 
We invited two high-performance coaches from the user testing phase (C1 \& C2; M=2; Age: 40-60) along with a US national team player (P1; M; Age: 20-25) who had been mentored by C1 for a decade. 
None of them had prior experience using VR outside of our study.
% 
We selected three professional matches, including two public matches (M1, M2) and one personal match provided by P1 (M3).

\begin{itemize}
    \item M1: 2021 Denmark Open Final, MS, Momota vs. Axelson~\cite{match_axelson_momota}  
    \item M2: 2022 BWF World Champ. QF, WS, Yamaguchi vs. Marin~\cite{match_marin_yamaguchi} 
    \item M3: 2022 Mexican International, R32, MS, Ma (P1) vs. Castillo
\end{itemize}

\re{
We processed the match data as described in Sec.~\ref{sec:data_preprocessing}.
%We performed data preprocessing on these three matches using techniques described in Sec.~\ref{sec:data_preprocessing}.
}
% M1 was a quarter-final game between Marin and Yamaguchi in 2022 BWF Women's Single, lasting for 47 minutes. M2 was a final match between Axelson and Momota in 2021 Denmark Open Men's Single, lasting for 1 hour 33 minutes. M3 was a round 32 match between Ma and Castillo in 2022 XIII Mexican International Men's Single, lasting for 1 hour.

% [M1] 2021 Denmark Open - Final MS, Axelson vs. Momota
% [M2]  2022 BWF World Championship - QF WS, Marin vs. Yamaguchi
% [M3] 2022 Mexican International - R32 MS, Ma vs. Castillo

\para{Experiment Set-up.}
% The user study was conducted in a 300 sq ft meeting room. The participant wore Meta Quest 2 to use VIRD.
% We ran VIRD in Unity3D~\cite{unity} on a PC with a i7-11800H
% 2.30GHz processer and an NVIDIA GeForce RTX 3060 graphics card.
% We displayed VIRD on a Meta Quest 2 virtual reality headset with $1,920 \times
% 1,832$ resolution per eye and a 90 Hz refresh rate, connected to the PC through a 5m USB3 Type-C cable. 
% The VIRD view was also projected onto a 65" 4k TV screen connected to PC so the instructor can see the VR view. 
% 
The user study took place in a 300 sq ft meeting room, where participants used VIRD with a Meta Quest 2 headset. VIRD was run 
% from Unity3D~\cite{unity} 
on a PC equipped with an i7-11800H 2.30GHz processor and an NVIDIA GeForce RTX 3060 graphics card. The Meta Quest 2 VR headset has a resolution of 1,920 x 1,832 per eye and a 90 Hz refresh rate, connected to the PC via a 5m USB3 Type-C cable. The VIRD view was also projected onto a 65" 4K TV screen connected to the PC, allowing the instructor to observe the VR view.




\noindent
\textbf{Study Design.}
To evaluate how VIRD supports match analysis, 
each coach analyzed one public match for developing game strategy, where C1 and C2 analyzed M1 and M2, respectively.
% 
In addition, to evaluate how VIRD helps derive and communicate insights for coaching, 
both C1 and P1 analyze M3 in the same session.
% , we scheduled C1 and P1 together.   
During the study, C1 analyzed M3 using VIRD and provided coaching advice directly to P1, who watched C1's interaction on a TV screen.




\noindent
\textbf{Procedures.} 
We first introduced the study to the expert and obtained their consent to participate and be recorded. They agreed to disclose their identity in the paper. 
\re{
The experts first watched the match video on the desktop for 10 minutes to familiarize themselves with the players in the match, as they had not coached them before.
%The experts first warmed up by watching the match video on the desktop for 10 minutes. This step allowed coaches to familiarize themselves with the players in the match as they had not coached either player in the two chosen public matches before. % and did not apply to P1.
}
Next, we introduced key features of VIRD with a list of example tasks, such as \textit{``select all winners by Momota in G1''}, and asked the expert to explore the features freely. This training step took around 10 minutes. The expert then analyzed the assigned match for 10 minutes in think-aloud fashion. After match analysis with VIRD, they were asked to conclude their coaching advice. In addition, C1 performed another match analysis of M3 and shared his advice with P1 in the study for 10 minutes. Finally, we gathered feedback from the expert about their experience with VIRD in a post-study survey and a follow-up interview.  
Each study took 60 to 75 minutes and we compensated each participant with a \$50 gift card.

\noindent
\textbf{Measure \& Data Analysis.}
We recorded the user interaction, voices, and VR screen records for analysis. In the post-study survey, we collected subjective ratings on a five-point Likert Scale, including learnability, usability, usefulness of each feature, and overall satisfaction of VIRD. In the follow-up interview, experts commented on the most useful features, pros and cons, and suggestions for using VIRD in actual coaching.
% 
To evaluate the experts' analysis and the insights they obtained, we performed text analysis on audio transcripts. We labeled user comments based on knowledge type, including prior knowledge, analysis, or insight. We also mapped user comments to VR screen records to extract the visualizations used in the analysis.


% \subsection{Computational Performance}
% We evaluated the performance of the shot and player pose detection of the three selected matches in the case study.  \jui{Do we need these? They just show that CLIFF performs well in our videos. Both MonoTrack and CLIFF are published work; I suggest we simply cite the papers and maybe provide a few numbers in 6.1 in 1-2 sentences to show how well they do on our games, and remove this section.}
% \begin{table}[h!]
% \small
% \begin{tabular}{|c|c|c|c|c|c|}
% \hline
% \textbf{Match} & \textbf{Length} & \textbf{\# Rally} & \textbf{Missing Frame} & \textbf{Total Frame} & \textbf{Accuracy} \\ \hline
% M1        &  93 mins & 109                        & 800                    & 56' 570              & \textbf{98,59\%}  \\ \hline
% M2        & 47 mins  & 73                         & 505                    & 24' 517              & \textbf{97,94\%}  \\ \hline
% M3         & 60 mins & 106                        & 2 976                  & 29 '086              & \textbf{89,77\%}  \\ \hline
% \textbf{Total} & 200 mins & \textbf{288}               & \textbf{4 281}         & \textbf{110 '173}    & \textbf{96,11\%}  \\ \hline
% \end{tabular}%
% \end{table}

% The missing frame column in the table corresponds to frames where at least one player was not detected by the algorithm. M1 and M2 have high accuracy ($>$97.9\%) with broadcast-quality video while M3,
% filmed with a phone camera, had approximately 90\% accuracy.
% The overall accuracy rate of over 96\% demonstrates the strong robustness and effectiveness of Cliff [] in detecting players. 


\subsection{Case Study Results}
We present the results of two case studies.
%, each with a different focus on the use of VIRD for match analysis. 
Case 1 examines a coach developing game strategies using VIRD. Case 2 explores a pair of coach and player
communicating insights for coaching. For both cases, we describe coaches' findings using the match player's last name. 


\subsubsection{Case 1: Developing Game Strategy in a Match}

We demonstrate C2's analysis workflow on M2 with both desktop and VIRD, highlighting his analysis approach, insights, and interactions.

 % 2022 BWF QF Yamaguchi vs. Marin
 \noindent
 \textbf{Desktop.} 
 During the 10-minute warm-up phase, the coach went through M2's first half (11 points) of game 1.
 \re{
  Using his usual video analysis approach, the coach went through the YouTube video sequentially, pausing or fast-forwarding to the rally, and manually recorded statistics (the number of winners, errors, and short rallies) on a spreadsheet after each rally (\autoref{fig:case}a).}
% Using his usual video analysis approach, the coach interacted with the video on YouTube by going through the video sequentially, pausing or fast-forwarding to the rally, and manually recording statistics 
 % \jui{can we use statistics throughout the paper? stats is not formal use}
% (the number of winners, errors, and short rallies) on a spreadsheet after each rally, as shown in \autoref{fig:case}a. 
 % analsyis
% Upon analysis, he focused on 
Using the collected stats and observations in the video, he discovered that Marin had won 8 out of 11 points very quickly, with 4 winners versus 5 unforced errors.
 Further, he observed that Marin was playing very flat and trying to push the tempo, leading to Yamaguchi only playing from a small area on the court. 
 
 % insights
 These observations led to two coaching insights.  The coach stressed that these were initial observations that he would usually first validate in more detail. 
 % With the limited time, the coach stressed that he points out the trend but would cross-reference for validations.\jui{this sentence seems a bit out of place. logically it will connect better if you say right away what are the two insights.}
 First, Marin is playing very fast and not moving the opponent. 
 %With Yamaguchi barely moving out of the box, 
 \textit{``[Yamaguchi]'s getting more comfortable with what Marin is doing.''} The advice is that Marin needs to utilize the backcourt and open the court more. To explain this insight, the coach pointed to the mid-court areas on a court diagram.
 Second, Marin is pressing the match and only playing flat shots. \textit{``These are world-class players. You can't just do the same thing the whole time.''} The coach thinks she needs to change techniques, such as varying the speed and angle of the shot. % shorten, or use more fake shots.
Lastly, the coach commented that 5 unforced errors were a little too high for the first half of the game. Yamaguchi is not moving much, leading Marin to waste her energy, \textit{``Marin's going to make more mistakes in the long-term if she doesn't change the strategy.''}

\noindent
\textbf{VIRD.} 
During the 10-minute match analysis with VIRD, the coach used the filters to focus on each player's winners and errors separately. He also drilled down to specific rallies or the shot video to verify observations, and focused on spatial aspects such as shot location, distribution, and shot trajectory in the analysis.
% analysis
To continue his analysis of M2 from the warm-up, he selected the first half of game 1 and explored the shot distribution by players and shot outcomes.
% \jui{he actually studied the same portion of the game as in desktop?}. 
He found that Marin's winners came mostly from the back and she attacked the bottom right corner (\autoref{fig:case}b left), while most of Yamaguchi's winning shots were from the front (\autoref{fig:case}b right). 
%Looking at errors, 
He also found that Marin's errors were pretty evenly distributed while Yamaguchi had more errors in the front. 
After an overview, the coach continued his analysis based on different hypotheses, such as wanting to see how Marin did on her winners because that's an important part of her game. He went through each of the winner rallies in detail and examined the shot locations on the court. He commented \textit{``Look at all the dots on the Yamaguchi's court, none of them pass this white line back here''}, pointing at the court with the VR controller (\autoref{fig:case}c).
He further focused on short rallies (less than 10 shots) using the Rally Menu, and observed that Marin's backhand serve was really flat, giving Yamaguchi scoring opportunities.



% Figure environment removed

% insights
The coach was able to verify his previous insights obtained on the desktop with concrete evidence while pointing out additional details.
% 1st insight: not moving the opponenet
First, he referred to the shot locations on the court (\autoref{fig:case}c) to demonstrate that Marin is barely moving the opponent. Further, based on the heatmap showing Yamaguchi has most of her winning shots from the front (\autoref{fig:case}b right), he suggested
\textit{``Marin should try to avoid this corner because Yamaguchi is creating scoring opportunities from this corner.''} 
% 2nd insight: flat shot
Second, he replayed a short rally and pointed out the flat shot, \textit{``Marin's backhand serve was really flat. [Yamaguchi] didn't have to move at all. She scored right away''}. Based on examining the short rallies, the coach suggested that Marin should
\textit{``either make the serve higher ... or don't use that type of serve because it's not working.''}
 % 
 \re{
 To explain these findings, the coach used actual video clips and spatial data and compared patterns between players and rallies. 
 %To explain these findings, the coach used actual video clips and spatial data to showcase the results, and compared patterns between players and rallies to verify observations. 
 }
 Throughout the analysis, the coach also used the first-person view to describe findings, such as \textit{``I want to avoid this corner and play the other ones.''}

% \jui{Should we add a short summary paragraph here to strengthen the argument of using VIRD? It seems that the message is "what C2 can do on desktop, he can do in VIRD. In addition, using VIRD he can do X, Y, Z better/more efficiently.} 

\noindent
\textbf{Case 1 Summary.} 
\re{
C2 analyzed the first half of game 1 in match M2 with Desktop and VIRD. The coach effectively verified two initial observations from Desktop using  VIRD and explained his insights with spatial data visualizations and specific rallies. He also seamlessly iterated between summary data and detailed game views to support his analytic reasoning.
}
%In Case 1, C2 analyzed the same match duration of M2 (first half of game 1) with Desktop and VIRD. The coach effectively verified two initial observations from Desktop using  VIRD, including that Marin was not moving the opponent and her shots were too flat, and explained his insights with spatial data visualizations and specific rallies. He also seamlessly navigated between summary data and detailed game views to support his analytic reasoning and iteration.

\subsubsection{Case 2: Verifying and Communicating Coaching Insights}
We describe how C1 verified and shared coaching advice with P1 using VIRD and how P1, as a player, obtained analysis insights. % from the player's perspective. %from the coach's perspective

% \jui{This section has two main paragraphs: Coach and Player. Before going into these paragraphs, you should explain what they are and why they are structured like these}

\noindent
\textbf{Coach.} Prior to our study, C1 had already spent 2 hours analyzing M3 and shared his insights with P1 virtually. 
\re{He had asked P1 questions about his strategy and his opinion about the match.}
%He coached P1 by asking questions about his strategy and his own understanding and opinions about the match. 
Based on the discussion, the coach followed up with match insights and statistical trends to explain his advice. % shorten They did not get into specific rallies in the actual video due to limited time.
  % 
  In this case study, the coach used VIRD to explore the match data in more depth, to communicate directly with P1, and to verify previous insights and update some original hypotheses.
  
The match between Castillo and Ma was won by Ma  (P1) with 21-11 (G1), 19-21 (G2), and 13-21 (G3). 
% \jui{repeated and colliding use of G1, G2, G3} 
The coach first read the score of the match from the Match Summary, and selected G2 with a score of 19-21 won by Ma because it was a close game. He focused on 18 errors made by Ma and examined each rally. Utilizing the virtual red shot arc to immediately pinpoint where the errors were coming from in each rally, he quickly browsed three rallies of errors and identified that Ma made all mistakes on defensive shots. 
This confirmed one of his previous analysis insights, where he pointed out P1's recovery shots gave the opponent too many opportunities, and suggested P1 working on his defense, 
\textit{``this shows we're working on the right thing because he's making mistakes here''}.
To demonstrate this insight, he continued to select another rally and used the VR pointer to point at the red shot arc, \textit{``you can see the mistake from this mid court when his opponent attacks.''} Based on this analysis, he asked follow-up questions to drill down to the root cause with the player, such as footwork issues. %or physical fitness issues.  

% 
The coach analyzed the 14 winners by Ma and found that 43\% of the winners were hit to the front left area on the heatmap. He filtered the shot location to focus on the 6 winners hit to the front left. He was surprised, since P1 is not confident in their net play. He pointed to the heatmap to show P1 that \textit{``your front is better than what you expected. You scored from the front the most''}.  He ended the analysis with a comment that he would go through each rally with the player in more depth, e.g., \textit{``Did you score because you’re always in the front, or was it your skill?''}

\para{Player.} 
The case study with P1 focused on how VIRD helps players analyze their own match in an immersive environment.
%shorten and improves their understanding on top of their existing knowledge.

After being introduced to the features of VIRD, the player analyzed M3 for 10 minutes. He first looked at winners and focused on comparing his shot locations and trajectories across each game. He found that his winners came majority from the back left, \textit{``the first game was 60\% from the back left''}. Further, he found almost all of his winners were going down or flat across three games by examining the shot arcs on the virtual court. 
He then confirmed his observations with the rally video.
%drilled down to the rally video to confirm the observations. 
By comparing data across games, he found that he had more winners in the front but not in the back in G2, but upon further investigation of a rally in G2, \textit{``for that rally it seemed like I did win it in the front but a lot of it was set up in the back actually''}.
\re{
Looking at the winner shot analysis, he was surprised that the majority of winners came from his backhand side, \textit{``I thought for me it was a lot easier in general to attack from the forehand side''}.
%he found that the majority of winners coming from his backhand side was surprising to him as \textit{``I thought for me it was a lot easier in general to attack from the forehand side''}. 
He also contemplated that he might need to find more ways to keep the shot down as it seemed beneficial if he did not hit it up as often.}

The player continued to analyze his errors in each game and compared the shot heat map. Seeing too many errors shown in a game, he chose to split each game into half. 
He found that most of his errors came from the back across all games. He moved in the virtual court to match his position on the court from the first-person view and pointed at the shot arcs and heat map on the virtual court to communicate his observations. He found his errors were mostly on the backhand side in the second half of G2
% \jui{undefined terminology?}
, while the majority were on the forehand side. %shorten (five out of six half game sets). 
He contemplated that \textit{``maybe he started changing strategy...''} as G2 had a tight score (19-21) where the opponent was close to winning the match.
Upon analyzing the shots on the virtual court and previewing them in the video view, he also found that all of the errors were very high arcing, \textit{``the arc tells me I'm losing because of defensive shots. It's either because I'm hitting into the back or my shots in the front are too high''}.
%, reflecting the coach's advice.
%To conclude, 
The player commented he would use this tool to go through every rally, and considered VIRD very helpful to get a sense of his overall performance and identify areas to discuss with the coach. 

\noindent
\textbf{Case 2 Summary.} In Case 2, C1 analyzed M3 and shared coaching insights with P1, while P1 also analyzed his own performance. The coach used VIRD to verify one previous insight (P1 needs to improve defensive shots) and discovered a new insight (P1 has better net shots than he thought), and used a combination of spatial visualizations and rally videos to explain his insights to the player. The player focused on finding patterns in his performance with spatial visualizations and drew conclusions with two new findings about his shots, including most winners coming from his backhand side and downwards, and most errors coming from the back across all games.

% Figure environment removed


\subsubsection{Post-study Survey \& Interview}
\autoref{fig:rating} shows the average subjective ratings collected in the post-study survey, ranging from 1 (strongly disagree) to 5 (strongly agree).
%The average subjective ratings of the post-study survey are shown in \autoref{fig:rating}, ranging from 1 (strongly disagree) to 5 (strongly agree).
Overall, experts rated VIRD with high learnability, usability and usefulness ($\mu \geq$ 4.0 except 3D player posture). % shorten We discussed each item with qualitative feedback from the follow-up interview.


\para{Learnability.} Experts found it easy to learn the features in VIRD. Particularly, the data provided in each visual element (Match Summary, Shot Filter, Rally Menu) were clearly understandable. The overall top-down analysis method, and spatial data visualizations in shot trajectories and heat maps were also properly learned during training. 
%shorten: Given the short study, coaches also expressed they will be able to use the tool to more extent with practices.

\para{Usability.} Experts rated the ease of use of VIRD for analyzing matches highly in each of the analysis tasks, including getting overviews, filtering shots and rallies, and navigating across rallies.
% 
The main issues arose in operating the VR controllers, such as clicking on the trigger button. Some suggestions included adding an onboarding tutorial to train users on accurately interacting with each component.   

\para{Usefulness.} 
Experts found VIRD helpful for match analysis from finding shot patterns, verifying insights, and explaining coaching advice. The coaches found the static data panels (Match Summary \& Shot Filter) most helpful as they provide the foundation of analysis to link to videos and spatial data visualizations, and visually showcase data and videos to the player. 
% 
Further, experts found it helpful to have a 3D virtual court with flexible view points, use a VR controller to select and navigate, and view dynamic 3D shot trajectories. 
The interactive approach was highlighted by both the coach and player as a benefit of VIRD, which supports an iterative analysis loop as well as linking static data to dynamic video and shots. 
The player found 3D visualizations (heatmap and trajectory) very useful in finding shot patterns and generating insights on his performance, especially being able to compare the video with 3D game from different angles.
% 
However, 3D player posture was considered less helpful ($\mu$=2.7) as 
coaches found the players off balance occasionally due to technical limitations. Experts mentioned they found the player positions in 3D game views helpful, and did not pay much attention to the actual posture. We observed that coaches were mostly interested in player movement and whether they were in good positions when hitting the shot. According to experts, observing slightly off-balanced player postures did not significantly impede their ability to comprehend the 3D gameplay. In cases where it was necessary, they would refer to video views for comparison.
% \jui{Should we expand on this a bit more given it is the lowest score by far? Why is player's pose off-balance a bad thing? What about their positioning? One motivation for having pose is their on-court position. Did that help? The other is for the pose when players hit the shots (e.g., in backhand backcourt). Did it help for those cases?} 
% In some occasions, coaches found the player poses off balance and had to look at the video to confirm. 
% Although pose and shot detection models are not the main contribution of our research, the accuracy is crucial to support a complete match analysis workflow.
% We discuss the limitations and opportunities of computer vision for sports further in Sec.~\ref{sec:top-down-analysis}.

\para{User Experience.} Experts felt satisfied, engaged, and prefer to use VIRD for analyzing match videos (all $\mu$=5.0). 
% pros
The major advantages for coaches were getting instant access to data, using an interactive approach, and access to 3D visualizations, which could lead to a huge reduction on the time to perform match analysis (less than 30 minutes vs. 3-5 hours). For the player, the pros are having spatial data to help dive into one's own strengths and weaknesses.
% cons
On the downside, coaches found the shot and pose detection occasionally inaccurate. Although they could refer to the actual video to verify, it hindered the experience of viewing the entire game in 3D. The player felt that the VR environment made him feel like he was in a game and he would get distracted. %shorten,  \textit{``there's so many things I want to do or play stuff''}.
As a player, he considered using this system for game watching instead of analyzing, since \textit{``If I'm training a lot and really tired, then going into this [VR], of course I want to have fun''}.
% If I'm training a lot and also competing, I'm really tired. Then going into this [VR], of course I want to have fun