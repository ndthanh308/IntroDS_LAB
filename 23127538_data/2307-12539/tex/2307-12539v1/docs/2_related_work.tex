% !TEX root = ../main.tex
\section{Related Work}

%\subsection{Visual Analytics of Sports Games}
\noindent
\textbf{Visual Analytics of Sports Games.}
\re{
Sports visualization research often aims to visualize spatial and dynamic data in context, e.g., on a court diagram (basketball\cite{pingali2001visualization}, tennis\cite{wu2018forvizor}, baseball\cite{dietrich2014baseball4d}), embedded in the game videos (soccer~\cite{stein2017bring}, basketball~\cite{lin2022quest, chen2023iball}) or projected on a virtual court in immersive environments (baseball~\cite{zou_evaluation_2019}, badminton\cite{chu2021tivee}).}
This is largely because contextual understanding is crucial for deriving meaningful and actionable insights from sports games~\cite{tuyls2021game,patton2021predicting,heaton2023perform}.
%There has been a long interest in sports visualization research to visualize spatial and dynamic data in context, e.g., on a court diagram (basketball\cite{pingali2001visualization}, tennis\cite{wu2018forvizor}, baseball\cite{dietrich2014baseball4d}), embedded in the game videos (soccer~\cite{stein2017bring}, basketball~\cite{lin2022quest, chen2023iball}) or projected on a virtual court in immersive environments (baseball~\cite{zou_evaluation_2019}, badminton\cite{chu2021tivee}).
%This is largely because contextual understanding is crucial for understanding sports games and for deriving meaningful and actionable insights~\cite{shea_2014,tuyls2021game,patton2021predicting,heaton2023perform}.

For racket sports in particular, spatial and temporal data, such as shot locations and trajectories, are crucial for game analysis and regulations.
LucentVision~\cite{pingali2000lucentvision} is a commercial tennis visualization system based on real-time ball/player tracking. It offers virtual replays of ball trajectories and presents a color heatmap to show the coverage of player movements.
%They designed several novel visualizations such as a color heatmap on a court diagram showing the coverage of player movements, and a virtual replay of a detected ball trajectory from flexible angles.
Similarly, Hawk-eye~\cite{owens2003hawk} system 
provides 3D views of tracked tennis balls using multiple cameras, and extends to other sports like baseball and soccer for enhanced game viewing and officiating.
% provides 3D views of the tracked tennis ball based on multiple tracking cameras and has been expanded to other sports, such as baseball pitches and soccer goals, where spatial visualizations are shown in virtual game views to enhance game viewing and officiating.

Two prior studies used immersive analytics to  analyze spatial badminton stroke data. 
ShuttleSpace~\cite{ye2020shuttlespace} visualizes badminton shot trajectories in a VR environment to support coaches in analyzing shot data from a first-person perspective.
 They provide an integrated visual design that augments 3D trajectory data with 2D statistical information using a first-person perspective visualization and peripheral vision. ShuttleSpace also enables natural and efficient trajectory selection in VR with a stroke metaphor, allowing analysts to select trajectories by imitating badminton strokes. The system has been evaluated through case studies conducted by domain experts, demonstrating its potential for facilitating badminton data analysis.
% 
\re{TIVEE~\cite{chu2021tivee} designed an immersive VR system for experts to analyze sequential stroke trajectories in badminton.  It allows experts to explore different tactics from a third-person perspective, and provides a detailed court view for inspecting and explaining tactics that lead to wins and losses. Case studies with professional badminton experts demonstrate the system's effectiveness in identifying patterns of commonly used tactics.}

\re{
In contrast to tennis, official badminton games do not use tracking systems. Thus, 
most badminton professionals lack access to manually collected shot datasets~\cite{ye2020shuttlespace, chu2021tivee} and rely on
video-based match analysis (Sec.~\ref{sec:formativestudy}).
Our study fills this gap by providing an end-to-end immersive video analytic tool, VIRD, that integrates computer vision-based data collection directly from match videos. Compared to prior work~\cite{chu2021tivee,ye2020shuttlespace}, our study provides comprehensive match analysis for coaching, emphasizing both analysis and communication of insights.
}
% While prior work~\cite{chu2021tivee,ye2020shuttlespace} concentrate on immersive visualization of static stroke trajectories, VIRD provides a dynamic and immersive 3D representation of complete badminton match data, including situated visualizations and 3D shot and player models. 
% Our study emphasizes  a comprehensive approach to support match analysis for coaching, both in analyzing and communicating insights.

% Our study focuses on match video analysis for badminton coaches and visualizes dynamic shot and player data based on automatic data collection generalizable to all match videos (Sec.~\ref{sec:data_preprocessing}). 

%\subsection{3D Game Reconstruction of Sports Videos}
\para{Game Reconstruction of Sports Videos.}
Reconstructing 3D sports games from videos offers opportunities to improve game understanding and analytics. Computer vision research has 
focused on reconstructing game scenes and player or ball movement for various sports, such as basketball~\cite{chen2009physics, zhu2020reconstructing}, tennis~\cite{pingali2000lucentvision,owens2003hawk}, soccer~\cite{rematas2018soccer}, volley ball~\cite{chen20113d}, or human motion in sports~\cite{rematas2018soccer}.
% 
Badminton games present unique challenges,
such as single-camera recordings, fast shuttle speed (the fastest shuttle speed can be over 250 mph), and complex player movement.
% Among all sports, badminton games present several unique challenges. First, badminton
% games are captured by single-camera videos, as usually no multi-angled or tracking cameras are available in tournaments. Second, tracking the shuttlecock and reconstructing its trajectory can be difficult due to its fast speed and unique physics. The fastest shuttle speed can be over 250 mph. Third, reconstructing 3D player movement on the court is not trivial due to the required accuracy on player shapes and poses.
% Third, reconstructing 3D player movement on the court is not trivial due to the required accuracy on player shapes and poses.
% Two main challenges exist. First, tracking the small shuttlecock and reconstructing its trajectory can be difficult due to its fast speed and unique physics. The second challenge is to accurately detect and reconstruct the player's poses and shape on the court from the single-angled video footage.

\re{Prior studies have addressed shuttlecock detection and tracking, including an instant review system to determine whether a shot was in or out~\cite{kopania-2022}.
%Kopania et al.\cite{kopania-2022} developed an instant review system that determines whether a shot was in or out.
} 
Other studies focused on tracking the speed, rotation angle, and athlete's body transformation using sensors and path-tracking algorithms\cite{lyu-2021}. Recently, MonoTrack~\cite{liu-2022} improved state-of-the-art models~\cite{huang2019tracknet, farin2003robust,wang2020deep} on court recognition and 2D trajectory estimation based on badminton domain knowledge. \re{
We use MonoTrack~\cite{liu-2022} to accurately extract and segment 3D shuttle trajectories in match videos.
%This method accurately extracts and segments 3D shuttle trajectories, providing a more precise analysis of badminton games. We use MonoTrack~\cite{liu-2022} to reconstruct dynamic shots in the match videos.
}

While there has been some research on the detection and tracking of badminton players \cite{haq-2022,rahmad-2019}, there has been relatively little focus on estimating their poses and shapes. Fortunately, SMPL~\cite{SMPL:2015},
a learned model of human body shape, allows accurate human shape representation from basic human 3D models. 
In addition, recent state-of-the-art algorithms are capable of performing 3D human pose and shape estimation \cite{zou-2021, monet-2022}. Among them, CLIFF~\cite{li-2022} can estimate SMPL parameters from a 2D video, which we use to reconstruct player poses.

% Our study used MonoTrack~\cite{liu-2022} to reconstruct dynamic shots and track player positions, which were then combined with pose estimation by CLIFF~\cite{li-2022} to reconstruct player poses in the badminton game.



%\subsection{Immersive Analytics for Spatial and Dynamic Data}
\para{Immersive Analytics for Spatial and Dynamic Data.}
Immersive analytics (IA) has gained significant attention among visualization researchers due to its ability to facilitate analytical reasoning and collaboration for analyzing high-dimensional and multivariate data~\cite{t_chandler_immersive_2015, 
% cordeil_imaxes:_2017,
% marriott_immersive_2018, 
ens2021grand}. 
With its large screen spaces and embodied interaction offered by immersive technologies, such as AR and VR displays, IA offers several advantages over traditional desktop visualizations~\cite{kraus2022immersive}, such as immersion~\cite{millais2018exploring, helbig2014concept}, multi-modal interaction~\cite{butscher2018clusters, lopez2015towards, ready2018immersive}, and presenting data in-situ~\cite{benko2004collaborative, hachet2011toucheo, moran2015improving}, which improves spatial understanding and user experience throughout the visual analysis process.
% 
% However, shortcomings of IA exist in certain steps in the analytic workflow, such as annotation tasks that require precise manipulation~\cite{dube2019text}, or displaying statistical and abstract information~\cite{munzner2014visualization}.
% 
Specifically, prior studies found several benefits of analyzing spatial and dynamic data in immersive environments, i.e., identifying spatial attributes such as distance~\cite{yang2018maps, kraus2019impact}, transitioning between 2D and 3D views~\cite{yang2020tilt}, and the visceral experience of viewing dynamic, physical data in VR~\cite{lee2020data}.

With these identified benefits of IA, recent work investigate applying immersive analytics to sports~\cite{lin2020sportsxr}. Besides ShuttleSpace~\cite{ye2020shuttlespace} and TIVEE~\cite{chu2021tivee} for badminton shuttle trajectory analysis in VR, 
Lin et al.~\cite{lin2021} present real-time basketball shot arcs for situated analysis in AR during free-throw training. Sumiya et al.~\cite{sumiya2022anywhere} applied a similar approach for basketball shooting training in VR. 
Rezzil~\cite{rezzil} provides a commercial soccer training system in VR that allows players to evaluate their performance with instant visual feedback. Zou et al.~\cite{zou_evaluation_2019} presents real-time bat swing spatial data for baseball batting training in VR.

% With more spatial and dynamic data involved in data analysis, 
% recent research proposed immersive analytics tool kits to allow authoring and analyzing spatial data
% \cite{buschel2021miria, sicat2018dxr, lee2023deimos, cordeil2019iatk}, or integrated analysis methods of 3D and 2D data with mixed reality ~\cite{hubenschmid2022relive, cavallo2019immersive}.
% These works establish the groundwork for domain-agnostic data exploration and analysis, such as observing user interaction in a lab study.
%While many previous work applied 
% IA has been applied to scientific visualization fields, such as archaeology~\cite{benko2004collaborative, smith2013artifactvis2} and biomedicine~\cite{nowke2013visnest, maes2018minomics}. 
% Few IA works address domain-specific analysis tasks for non-scientific users, such as sports~\cite{lin2020sportsxr, chu2021tivee}, IoT~\cite{ens2017ivy}, and facility management~\cite{prouzeau2020corsican, coupry2021bim}. 

While most IA work in sports focuses on providing real-time feedback for training, our study focuses on immersive video analytics for sports coaching with spatial and dynamic data extracted from sports videos. To the best of our knowledge, we are the first study to propose an immersive analysis system for sports video analysis.

% We aim to empirically evaluate the benefits of applying IA in sports and tackle some of the challenges outlined by Ens et al.~\cite{ens2021grand}, such as how situated contexts benefit users' cognitive load.
