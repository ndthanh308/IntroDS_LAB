%%%%%%%%%%%%%%%%%%%%%%%%%
% ellmultinomialtex; LaTeX file,
% Michael J. Schlosser
% ``An elliptic extension of the multinomial theorem''
% First version: July 24, 2023
% 1st revision: ***
% Author's email: michael.schlosser@univie.ac.at 
%%%%%%%%%%%%%%%%%%%%%%%%%
\documentclass[reqno,12pt]{amsart} 

\usepackage{hyperref}

\newtheorem{theorem}{Theorem}
\newtheorem{proposition}{Proposition}
\newtheorem{corollary}{Corollary}
\newtheorem{conjecture}{Conjecture}
\newtheorem{lemma}{Lemma}

\theoremstyle{remark}
\newtheorem{remark}{Remark}
\newtheorem*{fremark}{Final Remark}
\newtheorem{definition}{Definition}

%\setlength{\textwidth}{160.0mm}
%\setlength{\textheight}{240.0mm}
%\setlength{\oddsidemargin}{0mm}
%\setlength{\evensidemargin}{0mm}
%\setlength{\topmargin}{-8mm}
%\setlength{\parindent}{5.0mm}

\setlength{\textwidth}{160.0mm}
\setlength{\oddsidemargin}{0mm}
\setlength{\evensidemargin}{0mm}
\addtolength{\topmargin}{-1.3cm}
%\addtolength{\textwidth}{2.6cm}
\addtolength{\textheight}{2.6cm}

\numberwithin{equation}{section}

\allowdisplaybreaks

\newcommand{\ta}{\theta}
\newcommand{\C}{\mathbb C}
\newcommand{\E}{\mathbb E}
\newcommand{\N}{\mathbb N}
\newcommand{\Z}{\mathbb Z}
\newcommand{\F}{\mathbb F}

\author{Michael J.\ Schlosser}
\address{Fakult\"at f\"ur Mathematik, Universit\"at Wien,
Oskar-Morgenstern-Platz 1, A-1090 Vienna, Austria}
\email{michael.schlosser@univie.ac.at}
%\urladdr{http://www.mat.univie.ac.at/{\textasciitilde}schlosse}
\thanks{The author's research was partly supported by
FWF Austrian Science Fund grant P32305.}

\title[An elliptic multinomial theorem]{An elliptic extension
  of the multinomial theorem}

\subjclass[2010]{Primary 05A10; Secondary 11B65, 33D67, 33D80, 33E90}

\keywords{multinomial theorem, commutation relations,
elliptic weights, elliptic hypergeometric series}

\begin{document}

\begin{abstract}
We present a multinomial theorem for elliptic commuting
variables. This result extends the author's previously obtained
elliptic binomial theorem to higher rank. Two essential
ingredients are a simple elliptic star-triangle relation,
ensuring the uniqueness of the normal form coefficients,
and, for the recursion of the closed form elliptic multinomial
coefficients, the Weierstra{\ss} type $\mathsf A$ elliptic
partial fraction decomposition. From our elliptic multinomial
theorem we obtain, by convolution, an identity that is equivalent
to Rosengren's type $\mathsf A$ extension of the
Frenkel--Turaev ${}_{10}V_9$ summation, which in the
trigonometric or basic limiting case reduces to Milne's type
$\mathsf A$ extension of the Jackson ${}_8\phi_7$ summation.
Interpreted in terms of a weighted counting of lattice paths in
the integer lattice $\mathbb Z^r$, our derivation of the
$\mathsf A_r$ Frenkel--Turaev summation constitutes the first
combinatorial proof of that fundamental identity, and,
at the same time, of important special cases including
the $\mathsf A_r$ Jackson summation.
\end{abstract}

\maketitle

\section{Introduction}\label{sec:intro}
The area of {\em elliptic combinatorics} (or maybe, to be more precise,
 {\em elliptic hypergeometric combinatorics},
for a better distinction with other areas of a more geometric flavor
that might bear the same name)
is relatively young and intimately tied to elliptic lattice models.
In the concluding remarks of the paper \cite{S} published in 2007, which dealt
with the enumeration of lattice paths using elliptic weight functions
and which exhibited the first combinatorial proof of the fundamental
Frenkel--Turaev summation~\cite{FT} (see Equation~\eqref{propfteq}
below), the author proposed the development of a theory of elliptic
combinatorics. A lot has happened since then. 
For a sample of further papers on this subject, see
\cite{BK18,BCK,Be,BGR,HKKS,Ka15,Ka17,Ka19a,Ka19b,Ka23,S1,SY1,SY2}.

\textit{Elliptic hypergeometric series} (which, in a nutshell, form
a natural extension of ordinary, or ``rational'', hypergeometric series
and of basic, or ``trigonometric'', hypergeometric series to a more
general type of series in which the respective terms contain elliptic,
i.e., doubly-periodic and meromorphic, functions) made their
first implicit appearance in 1987 in the work of the mathematical
physicists Date, Jimbo, Kuniba, Miwa and Okado~\cite{DJKMO}
as elliptic $6$-$j$ symbols, representing elliptic solutions of the
star-triangle relation (a.k.a.\ Yang--Baxter equation).
Ten years later, Frenkel and Turaev~\cite{FT}, by exploiting the
tetrahedral symmetries of those  $6$-$j$ symbols and making the
expressions explicit, wrote out the first identities for (what they called)
``modular hypergeometric series'' (now commonly called
\textit{elliptic hypergeometric series}). In particular, they discovered
what is now called the ${}_{12}V_{11}$
transformation (an elliptic extension of Bailey's very-well-poised
${}_{12}\phi_{11}$ transformation) and, by applying specialization,
the ${}_{10}V_9$ summation (which is an elliptic extension of
Jackson's very-well-poised ${}_8\phi_7$ summation).

We start with explaining some important notions from the theory of elliptic
hypergeometric series (cf.\ \cite[Ch.~11]{GR} and \cite{R21})
which we shall need.
Let $\C^\times:=\C\setminus\{0\}$. Let the \textit{modified
Jacobi theta function} (in short: \textit{theta function}) with
argument $x$ and fixed nome $p$ be defined by
\begin{equation*}
\ta(x)=\ta (x; p):= (x; p)_\infty (p/x; p)_\infty\,,\quad\quad
\ta (x_1, \ldots, x_m): = \prod^m_{i=1} \ta (x_i),
\end{equation*}
where $ x, x_1, \ldots, x_m, p\in \C^\times,\ |p| < 1,$ and $(x; p)_\infty=
\prod^\infty_{k=0}(1-x p^k)$ is an infinite $p$-shifted factorial.

The theta function satisfies the following simple properties, namely
the \textit{inversion}
\begin{subequations}
\begin{equation}
\ta(x)=-x\,\ta(1/x),
\end{equation}
the \textit{quasi-periodicity}
\begin{equation}\label{p1id}
\ta(px)=-\frac 1x\,\ta(x),
\end{equation}
\end{subequations}
and the \textit{three-term addition formula}
(cf.\ \cite[p.~451, Example 5]{WW})
\begin{equation}\label{addf}
\ta(xy,x/y,uv,u/v)-\ta(xv,x/v,uy,u/y)
=\frac uy\,\ta(yv,y/v,xu,x/u).
\end{equation}
The addition formula in \eqref{addf} is a special case of the
following more general identity due to Weierstra{\ss}
(cf.\ \cite[p.~451, Example 3]{WW}),
which we refer to as \textit{elliptic partial fraction identity}
of type $\mathsf A$:
let $a_1,\ldots,a_r,b_1,\ldots,b_r\in\C^\times$, then
\begin{equation}\label{pfdA}
\sum_{i=1}^r\frac{\prod_{j=1}^r\ta(a_i/b_j)}{\prod_{j\neq i}\ta(a_i/a_j)}=0,
\end{equation}
under the assumption that the \textit{elliptic balancing condition}
$a_1\cdots a_r=b_1\cdots b_r$ holds\footnote{The notion
\textit{elliptic balancing condition} was introduced by
Spiridonov~\cite{Sp02} who made significant contributions in the
development of the theory of elliptic hypergeometric functions.}.
The addition formula in \eqref{addf} is a rewriting of the $r=3$
special case of \eqref{pfdA}. While the relation in \eqref{addf}
serves as  key ingredient in the theory of elliptic hypergeometric
series, the partial fraction decomposition in \eqref{pfdA}
is underlying the theory of multivariate elliptic hypergeometric
series associated to the root system $\mathsf A_r$
(cf.\ \cite{R04,RW}). Indeed, in the theory of (multivariate)
elliptic hypergeometric series inductive proofs and functional
equations typically make use of the identities in \eqref{addf}
and \eqref{pfdA}
(or, in the setting of root systems other than $\mathsf A_r$,
of other suitable elliptic partial fraction identities which exist).

Now define the {\em theta shifted factorial} (or
{\em $q,p$-shifted factorial}) by
\begin{equation*}
(a;q,p)_n := \begin{cases}
\prod^{n-1}_{k=0} \ta (aq^k),& n = 1, 2, \ldots\,,\cr
1,& n = 0,\cr
1/\prod^{-n-1}_{k=0} \ta (aq^{n+k}),& n = -1, -2, \ldots.
\end{cases}
\end{equation*}
For compact notation, we write
\begin{equation*}
(a_1, a_2, \ldots, a_m;q, p)_n := \prod^m_{k=1} (a_k;q,p)_n.
\end{equation*}
Notice that for $p=0$ one has $\ta (x;0) = 1-x$,
in which case $(a;q, 0)_n = (a;q)_n$
is a {\em $q$-shifted factorial} in base $q$ (cf.\ \cite{GR}).

Notice that
\begin{equation*}%\label{pid}
(pa;q,p)_n=(-1)^na^{-n}q^{-\binom n2}\,(a;q,p)_n,
\end{equation*}
which follows from repeated use of \eqref{p1id}. 
A list of other useful identities for manipulating the
$q,p$-shifted factorials is given in \cite[Sec.~11.2]{GR}.

By definition, a function $g(u)$ is {\em elliptic}, if it is
a doubly-periodic meromorphic function of the complex variable $u$.

Without loss of generality, one may assume
(see \cite[Theorem~1.3.3]{R21}) that
\begin{equation*}
g(u)=\frac{\ta(a_1q^u,a_2q^u,\dots,a_sq^u)}
{\ta(b_1q^u,b_2q^u,\dots,b_sq^u)}\,z
\end{equation*}
(i.e., $g$ is an abelian function of some degree $s$), 
for a constant $z$ and some
$a_1,a_2,\dots,a_s$, $b_1,b_2,\dots,b_s$, $q,p\in\C^\times$ with $|p|<1$,
where the elliptic balancing condition, namely
\begin{equation*}
a_1a_2\cdots a_s=b_1b_2\cdots b_s,
\end{equation*}
holds. If one writes $q=e^{2\pi\sqrt{-1}\sigma}$, $p=e^{2\pi\sqrt{-1}\tau}$,
with complex $\sigma$, $\tau$, then $g(u)$ is indeed periodic in $u$
with periods $\sigma^{-1}$ and $\tau\sigma^{-1}$
(which can be verified by applying \eqref{p1id} to each of the $2s$
theta functions appearing in $g(u)$).
Keeping this notation for $p$ and $q$, we denote the {\em field
of elliptic functions} over $\C$ of the complex variable $u$,
meromorphic in $u$ with the two periods $\sigma^{-1}$ and
$\tau\sigma^{-1}$, by $\E_{q^u;q,p}$.

More generally, we denote the {\em field of totally elliptic multivariate
functions} over $\C$ of the complex variables $u_1,\dots,u_n$,
meromorphic in each variable with equal periods,
$\sigma^{-1}$ and $\tau\sigma^{-1}$, of double periodicity, by
$\E_{q^{u_1},\dots,q^{u_n};q,p}$.
The notion of totally elliptic multivariate functions
was first introduced by Spiridonov \cite{Sp02,Sp11}.

We are ready to recall the definition of an \textit{elliptic
hypergeometric series}. This is defined to be a series $\sum_{k\ge 0}c_k$
with $c_0=1$ such that $g(k):=c_{k+1}/c_k$ is an elliptic function
in $k$ (viewed as a complex variable).

We already conclude our brief introduction by explicitly reproducing
Frenkel and Turaev's ${}_{10}V_9$ summation \cite{FT}
(see also \cite[Eq.~(11.4.1)]{GR}),
an identity which is fundamental to the
theory of elliptic hypergeometric series:
Let $m\in\N_0$ and $a,b,c,d,e,q,p\in\C$ with $|p|<1$.
Then there holds the following identity:
\begin{align}\label{propfteq}
\sum_{k=0}^m\frac{\ta(aq^{2k})}{\ta(a)}
\frac{(a,b,c,d,e,q^{-m};q,p)_k}
{(q,aq/b,aq/c,aq/d,aq/e,aq^{m+1};q,p)_k}q^k
&\notag\\=
\frac{(aq,aq/bc,aq/bd,aq/cd;q,p)_m}
{(aq/b,aq/c,aq/d,aq/bcd;q,p)_m}&,
\end{align}
where $a^2q^{m+1}=bcde$.
It is easy to see that the series in \eqref{propfteq}
is indeed an elliptic hypergeometric series.
The convention of referring to the above series as a
${}_{10}V_9$ series follows Spiridonov's arguments
in \cite{Sp02b} and has become standard
(see also \cite[Ch.~11]{GR}).

The rest of of paper is organized as follows:
In Section~\ref{sec:bmcoeffs} we introduce the specific
elliptic weights which we use and define corresponding
elliptic binomial and multinomial coefficients.
The elliptic binomial coefficients are those which we
introduced in \cite{S} in the context of lattice path
enumeration (different elliptic binomial coefficients
were considered by Rains~\cite[Definition~11]{R06}
and, at least implicitly, also by Coskun and Gustafson~\cite{CG},
both in the context of convolutions for families of multivariate
special functions that are recursively defined by vanishing
properties and a branching rule)
and also appeared as the normal form coefficients in an
elliptic extension of the binomial theorem, featured in
\cite[Sec.~4]{S1}. The elliptic multinomial coefficients,
defined in the same section,
extend our elliptic binomial coefficients and are new.
What is interesting about our specific elliptic weights in \eqref{wdef}
is that they satisfy a simple elliptic star-triangle relation,
see \eqref{eq:estr}. Our analysis in Section~\ref{sec:ellmthm}
crucially depends on this star-triangle relation.
In Section~\ref{sec:ellmthm} we first introduce an
algebra of elliptic commuting variables and then
turn to the main result of the paper, an elliptic extension
of the multinomial theorem, as an identity in that algebra,
a result which extends the elliptic binomial
theorem from \cite[Sec.~4]{S1}. We show how our elliptic
multinomial theorem can be used to rederive
Rosengren's~\cite[Theorem~5.1]{R04} $\mathsf A_r$
extension of the Frenkel--Turaev summation,
which in the basic case was first obtained by
Milne~\cite[Theorem~6.17]{M88}.
In a concluding remark we explain how our our algebraic derivation
of the $\mathsf A_r$ Frenkel--Turaev summation admits
a direct combinatorial interpretation in terms of elliptic
weighted lattice paths in the integer lattice $\mathbb Z^r$.

\section{Elliptic weights, elliptic binomial and
  multinomial coefficients}\label{sec:bmcoeffs}

For indeterminants $a$, $b$, complex numbers $q$, $p$ (with $|p|<1$),
and integers $s$, $t$, we define the \textit{small elliptic weights} by
\begin{equation}\label{wdef}
w_{a,b;q,p}(s,t):=
\frac{\ta(aq^{s+2t},bq^{2s+t-2},aq^{t-s-1}/b)}
{\ta(aq^{s+2t-2},bq^{2s+t},aq^{t-s+1}/b)}q.
\end{equation}
The corresponding \textit{big elliptic weights} are defined by
\begin{align}\label{Wdef}
W_{a,b;q,p}(s,t):={}&\prod_{j=1}^tw_{a,b;q,p}(s,j)\notag\\
={}&\frac{\ta(aq^{s+2t},bq^{2s},bq^{2s-1},aq^{1-s}/b,aq^{-s}/b)}
{\ta(aq^s,bq^{2s+t},bq^{2s+t-1},aq^{1+t-s}/b,aq^{t-s}/b)}q^t.
\end{align}
Clearly, $W_{a,b;q,p}(s,0)=1$, for all $s$.
For $p\to 0$ followed by $a\to 0$ and $b\to 0$, in this order
(or $p\to 0$ followed by $b\to \infty$ and $a\to \infty$, in this order),
the small elliptic weights $w_{a,b;q,p}(s,t)$ all reduce to $q$
and the big elliptic weights $W_{a,b;q,p}(s,t)$ reduce to $q^t$.
For convenience, we also define the following shifted variant of
a big elliptic weight,
\begin{equation}\label{sWdef}
W_{a,b;q,p}^{(\rho)}(s,t):=W_{aq^{2\rho},bq^{2\rho};q,p}(s,t),
\end{equation}
and further the \textit{big $Q$-weights} by the product
\begin{align}\label{Qweights}
&Q_{a,b;q,p}(\ell,\rho,s,t):=\prod_{i=1}^\ell
W_{a,b;q,p}^{(\rho)}(i+s,t)\notag\\
&=\frac{(aq^{1+2\rho+s+t};q,p)_\ell\,(bq^{1+2\rho+2s};q,p)_{2\ell}\,
(aq^{1-\ell-s}/b,aq^{-\ell-s}/b;q,p)_\ell}
{(aq^{1+2\rho+s};q,p)_\ell\,(bq^{1+2\rho+2s+t};q,p)_{2\ell}\,
(aq^{1+t-\ell-s}/b,aq^{t-\ell-s}/b;q,p)_\ell}q^{\ell t}.
\end{align}
(We decided to use the capital letter $Q$ to denote
the expression on the right-hand side of \eqref{Qweights},
since we view that expression as an extension of $q^{\ell t}$,
a ``big product'' of $q$'s.)

Assuming $c$ to be an additional indeterminant,
we would like to highlight the following relation satisfied by the
small elliptic weights \eqref{wdef}, for all $s$ and $t$,
which we refer to as the (simple) \textit{elliptic star-triangle
  relation}\footnote{Other, more complicated,
  elliptic star-triangle relations have appeared in the literature.
  They usually involve a sum, an integral, or matrices.
See in particular \cite{Ke15} and the references therein,
and \cite{MS18}, a selection which is not meant to be exhaustive.
Star-triangle relations are also known as Yang--Baxter equations
and are studied as the master equation in  integrable models
in statistical mechanics and quantum field theory, see \cite{J89}.
The first explicit mentioning of a star triangle relation was made
by Onsager~\cite{O44} in 1944 in connection with his solution
of the Ising model.}.
\begin{align}\label{eq:estr}
&w_{aq^2,bq^2;q,p}(s,t)\,w_{a,c;q,p}(s,t)\,w_{bq^2,cq^2;q,p}(s,t)\notag\\*
&=w_{b,c;q,p}(s,t)\,w_{aq^2,cq^2;q,p}(s,t)\,w_{a,b;q,p}(s,t).
\end{align}
This specific equality of simple products
(which, somewhat surprisingly, we were unable to find
in the existing literature)
is readily verified using the explicit expression
for the small elliptic weights in \eqref{wdef}.
Marking and connecting the positions of the three weights
in \eqref{eq:estr} where the respective parameters
(appearing in the subindices) are both shifted by $q^2$,
and (a level below) the same for the other three weights
where the respective parameters are not shifted, both
from left-to right, an overlapping of the two chains
becomes visible, see Figure~\ref{fig1}.
% Figure environment removed

For indeterminants $a$, $b$, complex numbers $q$, $p$
(with $|p|<1$), and integers $n$, $k$, we
define the {\em elliptic binomial coefficient} as follows
(which is exactly the expression for $w(\mathcal P((0,0)\to(k,n-k)))$
in \cite[Th.~2.1]{S}, and in \cite{S1} it was shown that these
elliptic binomial coefficients indeed appear as the coefficients
in a noncommutative elliptic binomial theorem):
\begin{equation}\label{ellbc}
\begin{bmatrix}n\\k\end{bmatrix}_{a,b;q,p}:=
\frac{(q^{1+k},aq^{1+k},bq^{1+k},aq^{1-k}/b;q,p)_{n-k}}
{(q,aq,bq^{1+2k},aq/b;q,p)_{n-k}}.
\end{equation}
Note that this definition of the elliptic binomial coefficient
reduces to the usual $q$-binomial coefficient
after taking the limits $p\to 0$, $a\to 0$, and $b\to 0$,
in this order (or after taking the limits in the order $p\to 0$,
$b\to\infty$, and $a\to\infty$).
As pointed out in \cite{S}, the expression in
\eqref{ellbc} is {\em totally elliptic}, i.e.\
elliptic in each of $\log_qa$, $\log_qb$, $k$, and $n$
(viewed as complex parameters), with equal periods of double periodicity, 
which fully justifies the notion ``elliptic''.
In particular,
$\left[\begin{smallmatrix}n\\k\end{smallmatrix}\right]_{a,b;q,p}
\in\E_{a,b,q^n,q^k;q,p}$.

It is immediate from the definition of \eqref{ellbc} that, for
integers $n,k$, there holds
\begin{subequations}\label{qbinrel}
\begin{equation}
\begin{bmatrix}n\\0\end{bmatrix}_{a,b;q,p}=
\begin{bmatrix}n\\n\end{bmatrix}_{a,b;q,p}=1,
\end{equation}
and
\begin{equation}
\begin{bmatrix}n\\k\end{bmatrix}_{a,b;q,p}=0,\qquad\text{whenever}\quad
k<0,\quad\text{or}\quad k> n.
\end{equation}
Furthermore, using the theta additional formula in \eqref{addf}
one can verify the following recursion formula for the
elliptic binomial coefficients:
\begin{equation}\label{rec}
\begin{bmatrix}n+1\\k\end{bmatrix}_{a,b;q,p}=
\begin{bmatrix}n\\k\end{bmatrix}_{a,b;q,p}+
\begin{bmatrix}n\\k-1\end{bmatrix}_{a,b;q,p}\,W_{a,b;q,p}(k,n+1-k),
\end{equation}
\end{subequations}
for non-negative integers $n$ and $k$.

If one lets $p\to 0$, $a\to 0$, then $b\to 0$, in this order
(or after taking the limits in the order $p\to 0$,
$b\to\infty$, and $a\to\infty$), the
relations in \eqref{qbinrel} reduce to
\begin{equation*}
\begin{bmatrix}n\\0\end{bmatrix}_{q}=
\begin{bmatrix}n\\n\end{bmatrix}_{q}=1,
\end{equation*}
\begin{equation*}
\begin{bmatrix}n+1\\k\end{bmatrix}_{q}=
\begin{bmatrix}n\\k\end{bmatrix}_{q}+
\begin{bmatrix}n\\k-1\end{bmatrix}_{q}\,q^{n+1-k},
\end{equation*}
for positive integers $n$ and $k$ with $n\ge k$, which is
a well-known recursion for the $q$-binomial coefficients.

As was shown in \cite{S1}, the elliptic binomial coefficients
in \eqref{ellbc} can be interpreted as the (area) generating function
for all lattice paths in the integer lattice $\Z^2$ from $(0,0)$ to
$(k,n-k)$ consisting of East and North steps of unit length
where each path is weighted with respect to the product of the
weights of the respective squares covered by the path (by which
one means that they are in the region between the $x$-axis and below the
path). In this interpretation, the weight of the single square with
north-east corner $(s,t)$ is given by $w_{a,b;q,p}(s,t)$,
whereas $W_{a,b;q,p}(s,t)$ can be regarded as
the weight of of the $s$-th column having height $t$.

To prepare the reader for a better understanding of our main result
of this paper, namely the elliptic multinomial theorem in
Section~\ref{sec:ellmthm}, it will be convenient to recall the
author's elliptic binomial theorem from \cite[Theorem~2]{S1}.
We start with the definition of
the algebra of elliptic-commuting variables in which the
elliptic binomial coefficients manifestly appear as the
coefficients in a binomial expansion after normal ordering of the
respective variables.

\begin{definition}\label{defea}
For two complex numbers $q$ and $p$ with $|p|<1$,
let $\C_{q,p}[X,Y,\E_{a,b;q,p}]$ denote
the associative unital algebra over $\C$,
generated by $X$, $Y$, and the commutative
subalgebra $\E_{a,b;q,p}$,
satisfying the following three relations:
%\begin{subequations}\label{defeaeq}
\begin{align*}
YX&=W_{a,b;q,p}(1,1)\,XY,%\label{elleq}
\\
Xf(a,b)&=f(aq,bq^2)\,X,\label{xf}\\%\label{yf}
Yf(a,b)&=f(aq^2,bq)\,Y,
\end{align*}
%\end{subequations}
for all $f\in\E_{a,b;q,p}$.
\end{definition} 
We refer to the variables $X,Y,a,b$
forming $\C_{q,p}[X,Y,\E_{a,b;q,p}]$
as {\em elliptic commuting} variables.
The algebra $\C_{q,p}[X,Y,\E_{a,b;q,p}]$ reduces to
$\C_{q}[X,Y]$ if one formally lets $p\to 0$, $a\to 0$,
then $b\to 0$, in this order, or lets $p\to 0$, $b\to\infty$,
then $a\to\infty$, in this order, while (having eliminated the nome $p$) 
relaxing the condition of ellipticity.
It should be noted that the monomials $X^kY^l$ form a basis for the
algebra $\C_{q,p}[X,Y,\E_{a,b;q,p}]$ as a left module over $\E_{a,b;q,p}$,
i.e., any element can be written uniquely as a finite sum
$\sum_{k,l\ge 0} f_{kl}X^kY^l$ with $f_{kl}\in \E_{a,b;q,p}$ which
we call the \textit{normal form} of the element.

The following result from \cite[Theorem~2]{S1}
shows that the normal form of the binomial
$(X+Y)^n$ is ``nice''; each (left) coefficient of $X^kY^{n-k}$ completely
factorizes as an expression in $\E_{a,b;q,p}$.

\begin{theorem}[Binomial theorem for %elliptic commuting
  variables in \mbox{$\C_{q,p}[X,Y,\E_{a,b;q,p}]$}]\label{ebthm}
  Let $n\in\N_0$.
  % Then the following identity is valid in $\C_{q,p}[X,Y,\E_{a,b;q,p}]$:
  Then, as an identity in  $\C_{q,p}[X,Y,\E_{a,b;q,p}]$, we have
\begin{equation*}%\label{eqbinth}
(X+Y)^n=\sum_{k=0}^n\begin{bmatrix}n\\k\end{bmatrix}_{a,b;q,p}X^kY^{n-k}.
\end{equation*}
\end{theorem}

In \cite[Corollary~4]{S1}, convolution was applied to this result
(together with comparison of coefficients)
yielding the Frenkel and Turaev ${}_{10}V_9$ summation \cite{FT}
in a form equivalent to \eqref{propfteq} by analytic continuation.

\begin{remark}\label{remHKKS}
In the recent work \cite[Definition~5.6 and Theorem~5.7]{HKKS}, the
author, in collaboration with Hoshi, Katori, and Koornwinder, defined
a similar but different elliptic commuting algebra (denoted there by
$\C_{q,p}[X,Y,\E_{x,a,b,c;q,p}]$) with a corresponding binomial theorem.  
\end{remark}

Before we extend the elliptic binomial coefficients in \eqref{ellbc}
to elliptic multinomial coefficients, we rewrite the elliptic
partial fraction decomposition \eqref{pfdA} in a form that
is suitable for our purpose.
Replacing $r$ by $r+1$ in \eqref{pfdA},
isolating the $r+1$-th term of the sum, putting the first $r$
terms to the other side and dividing both sides of the
equation by the $r+1$-th term, we obtain the following
form of the type $\mathsf A$ elliptic partial fraction identity:
\begin{equation}\label{pfd}
  1=\frac{\prod_{j=1}^r\ta(a_{r+1}/a_j)}
  {\prod_{j=1}^{r+1}\ta(a_{r+1}/b_j)}
  \sum_{i=1}^r\frac{\prod_{j=1}^{r+1}\ta(b_j/a_i)}
  {\prod_{\substack{1\le j\le r+1\\j\neq i}}\ta(a_j/a_i)},
\end{equation}
now subject to the elliptic balancing condition
$a_1\cdots a_{r+1}=b_1\cdots b_{r+1}$.

We are ready to define (for the first time)
elliptic multinomial coefficients.
Let $r>1$ be an integer and $a_1,\ldots,a_r\in\C^\times$ be variables
(in addition to the base $q$ and nome $p$; $p,q\in\C^\times$, $|p|<1$).
Further, let $k_1\ldots,k_r$ be integers satisfying $k_1+\cdots+k_r\ge 0$.
Here and throughout, we write $K_i:=\sum_{\nu=1}^i k_\nu$,
for $i=0,\ldots,r$, and we will later similarly use the notations
$N_i:=\sum_{\nu=1}^i n_\nu$ and $L_i:=\sum_{\nu=1}^i l_\nu$.
We define the elliptic multinomial coefficients explicitly as
\begin{align}\label{ellmc}
&\begin{bmatrix}k_1+\cdots+k_r\\k_1,\ldots,k_r\end{bmatrix}
_{a_1,\ldots,a_r;q,p}\notag\\
&:=
\frac{(q;q,p)_{k_1+\cdots+k_r}}{\prod_{i=1}^r(q;q,p)_{k_i}}
\prod_{i=1}^r\frac{(a_iq^{1+K_r-k_i};q,p)_{k_i}}
{(a_iq^{1+2K_{i-1}};q,p)_{k_i}}
\prod_{1\le i<j\le r}\frac{(a_iq^{1-k_i}/a_j;q,p)_{k_j}}
{(a_iq/a_j;q,p)_{k_j}}.
\end{align}
For $r=2$, the elliptic multinomial coefficients
$\left[\begin{smallmatrix}k_1+k_2\\k_1,k_2\end{smallmatrix}\right]_{a_1,a_2;q,p}$
reduce to the elliptic binomial coefficients
$\left[\begin{smallmatrix}k_1+k_2\\k_1\end{smallmatrix}\right]_{a_1,a_2;q,p}$
(which in general is different from
$\left[\begin{smallmatrix}k_1+k_2\\k_2\end{smallmatrix}\right]_{a_1,a_2;q,p}$)
given in \eqref{ellbc}. That is, for $r=2$ we have two short notations for the
elliptic multinomial coefficients in \eqref{ellmc},
just as in the familiar ordinary case.

The elliptic multinomial coefficients in \eqref{ellmc} satisfy
\begin{subequations}
\begin{equation}
  \begin{bmatrix}0\\0,\ldots,0\end{bmatrix}_{a_1,\ldots,a_r;q,p}
=1,
\end{equation}
and (remember that we are assuming $k_1+\cdots+k_r\ge 0$)
\begin{equation}
  \begin{bmatrix}k_1+\cdots+k_r\\k_1,\ldots,k_r\end{bmatrix}_{a_1,\ldots,a_r;q,p}
=0,\quad\text{whenever}\quad
k_j<0\quad\text{for some}\quad j=1,\ldots,r,
\end{equation}
and for $k_1+\cdots+k_r>0$
the recurrence relation
\begin{align}\label{rec-ellmc}
  \notag
  &\begin{bmatrix}k_1+\cdots+k_r\\
    k_1,\ldots,k_r\end{bmatrix}_{a_1,\ldots,a_r;q,p}\\*
  &=\sum_{i=1}^r
  \begin{bmatrix}k_1+\cdots+k_r-1\\k_1,\ldots,k_{i-1},k_i-1,
    k_{i+1},\ldots,k_r\end{bmatrix}_{a_1,\ldots,a_r;q,p}
 \prod_{j>i}W_{a_i,a_j;q,p}^{(K_{j-1}-k_i)}(k_i,k_j).
\end{align}
\end{subequations}
The latter is readily established by using the elliptic
partial fraction decomposition \eqref{pfd}.
Indeed, dividing both sides of \eqref{rec-ellmc} by
the elliptic multinomial coefficient on the left-hand side and
replacing the elliptic multinomial coefficients and the shifted big
elliptic weights and by their explicit expressions in \eqref{ellmc},
\eqref{sWdef} and \eqref{Wdef}, we obtain, after cancellation
of common factors, \eqref{pfd} with respect to the following simultaneous
substitutions:
\begin{alignat*}{3}
  &a_i\mapsto q^{k_i}/a_i,&\quad&\text{for $1\le i\le r$},&\qquad\;
  &a_{r+1}\mapsto q^{k_1+\cdots+k_r},\\
 &b_i\mapsto 1/a_i,&&\text{for $1\le i\le r$},&
  &b_{r+1}\mapsto q^{2(k_1+\cdots+k_r)}.
\end{alignat*}
This confirms \eqref{rec-ellmc}.

\section{An elliptic multinomial theorem}\label{sec:ellmthm}

Recall (from Section~\ref{sec:intro})
that $\E_{a_1,\ldots,a_r;q,p}$ denotes the field of totally
elliptic functions over $\C$, in the complex variables 
$\log_qa_i$, $1\le i\le r$, with equal periods $\sigma^{-1}$,
$\tau\sigma^{-1}$ (where $q=e^{2\pi\sqrt{-1}\sigma}$, $p=e^{2\pi\sqrt{-1}\tau}$,
$\sigma,\tau\in\C$), of double periodicity.

We present an elliptic extension of the multinomial theorem
that extends our elliptic binomial theorem from \cite[Theorem~2]{S1}
to several (elliptic-commuting) variables.

We shall work in the following algebra.
\begin{definition}\label{defear}
For $2r$ noncommuting variables $X_1,\ldots,X_r$, and $a_1,\ldots,a_r$,
where the variables $a_1,\ldots,a_r$ commute with each other,
and two complex numbers $q$, $p$ with $|p|<1$, let
$\C_{q,p}[X_1,\ldots,X_r,\E_{a_1,\ldots,a_r;q,p}]$ denote the associative
unital algebra over $\C$, generated by $X_1,\ldots,X_r$,
and the commutative subalgebra $\E_{a_1,\ldots,a_r;q,p}$,
satisfying the following relations:
\begin{subequations}\label{defeaeqr}
\begin{align}
X_jX_i&=w_{a_i,a_j;q,p}(1,1)\,X_iX_j,\label{elleqr}
\quad\text{for}\quad 1\le i<j\le r,\\
X_i\,f(a_1,\ldots,a_r)&=f(a_1q^2,\ldots,a_{i-1}q^2,a_iq,
a_{i+1}q^2,\ldots,a_r q^2)\,X_i,\quad\text{for}\quad 1\le i\le r,\label{xfr}
\end{align}
\end{subequations}
for all $f\in\E_{a_1,\ldots,a_r;q,p}$, and where the elliptic weights
$w_{a_i,a_j;q,p}$ are defined in \eqref{wdef}.
\end{definition}
We refer to the $2r$ variables $X_1,\ldots,X_r$, $a_1,\ldots,a_r$
forming $\C_{q,p}[X_1,\ldots,X_r,\E_{a_1,\ldots,a_r;q,p}]$
as {\em elliptic-commuting} variables\footnote{The algebra
  $\C_{q,p}[X_1,\ldots,X_r,\E_{a_1,\ldots,a_r;q,p}]$
reduces to the well-known algebra of $q$-commuting
variables, that we may denote by $\C_q[X_1,\ldots,X_r]$,
%that is
defined by $X_jX_i=qX_iX_j$ for $1\le i<j\le r$,
if one formally lets $p\to 0$ and $a_1\to 0,\ldots,a_r\to 0$,
in this order, or lets $p\to 0$ and $a_r\to\infty,\ldots,a_1\to\infty$,
in this order, while (having eliminated the nome $p$) relaxing the condition
of ellipticity.}.

The following commutation relations, for $1\le i<j\le r$
and $1\le k\le r$, arise as a consequence of
\eqref{xfr} combined with \eqref{wdef}:
\begin{align*}
  X_i\,w_{a_i,a_j;q,p}(s,t)&=w_{a_i,a_j;q,p}(s+1,t)\,X_i,\\
  X_j\,w_{a_i,a_j;q,p}(s,t)&=w_{a_i,a_j;q,p}(s,t+1)\,X_j,\\
  X_k\,w_{a_i,a_j;q,p}(s,t)&=w_{a_iq^2,a_jq^2;q,p}(s,t)\,X_k,
 \quad\text{for}\quad k\neq i\quad\text{and}\quad k\neq j.
\end{align*}

The relations in \eqref{defeaeqr} are
well-defined as any expression in
$\C_{q,p}[X_1,\ldots,X_r,\E_{a_1,\ldots,a_r;q,p}]$
can be put into a unique canonical form regardless in which order the
commutation relations are applied for this purpose.
(With other words, in $\C_{q,p}[X_1,\ldots,X_r,\E_{a_1,\ldots,a_r;q,p}]$)
Bergman's diamond lemma~\cite{B} applies.)
This can be shown by induction on the number of steps
required to bring any monomial into canonical form.
By locally applying commutation relations of the form \eqref{defeaeqr}
the number of steps needed to bring a monomial into canonical form
is reduced. The order of the steps to achieve this is not unique,
but (thanks to the simple elliptic star-triangle relation
in \eqref{eq:estr}) the final result is unique.

We demonstrate this claim by considering the normalization of
the monomial $X_cX_bX_a$, for $1\le a<b<c\le r$, which, using
\eqref{defeaeqr}, can be brought in two
distinct ways into canonical form.

On one hand, we have
\begin{subequations}
\begin{align}\label{eq:diamond1}
X_cX_bX_a&=X_c(X_bX_a)=X_c\, w_{a,b;q,p}(1,1)\,X_aX_b\notag\\
         &=w_{aq^2,bq^2;q,p}(1,1)\,X_cX_aX_b=w_{aq^2,bq^2;q,p}(1,1)\,
           w_{a,c;q,p}(1,1)\,X_aX_bX_c\notag\\
&=w_{aq^2,bq^2;q,p}(1,1)\,w_{a,c;q,p}(1,1)\,X_a\,w_{b,c;q,p}(1,1)\,X_bX_c\notag\\
&=w_{aq^2,bq^2;q,p}(1,1)\,w_{a,c;q,p}(1,1)\,w_{bq^2,cq^2;q,p}(1,1)\,X_aX_bX_c.
\end{align}
On the other hand, we have
\begin{align}\label{eq:diamond2}
X_cX_bX_a&=(X_cX_b)X_a=w_{b,c;q,p}(1,1)\,X_bX_cX_a\notag\\
         &=w_{b,c;q,p}(1,1)\,X_b\,w_{a,c;q,p}(1,1)\, X_aX_c
           =w_{b,c;q,p}(1,1)\,w_{aq^2,cq^2;q,p}(1,1)\,X_bX_aX_c\notag\\
&=w_{b,c;q,p}(1,1)\,w_{aq^2,cq^2;q,p}(1,1)\,w_{a,b;q,p}(1,1)\,X_aX_bX_c.
\end{align}
\end{subequations}
Comparison of the left coefficients of $X_aX_bX_c$ in \eqref{eq:diamond1}
and \eqref{eq:diamond2} gives
\begin{align*}
&w_{aq^2,bq^2;q,p}(1,1)\,w_{a,c;q,p}(1,1)\,w_{bq^2,cq^2;q,p}(1,1)\notag\\
&=w_{b,c;q,p}(1,1)\,w_{aq^2,cq^2;q,p}(1,1)\,w_{a,b;q,p}(1,1),
\end{align*}
which is an instance of the elliptic star-triangle relation
\eqref{eq:estr}.

For bringing expressions in $\C_{q,p}[X_1,\ldots,X_r,\E_{a_1,\ldots,a_r;q,p}]$
into normal form the following lemma is useful.
\begin{lemma}\label{lem:com}
Let $k_1,\ldots,k_r$ and $l_1,\ldots,l_r$ be non-negative integers
The following commutation relation holds as an identity in
$\C_{q,p}[X_1,\ldots,X_r,\E_{a_1,\ldots,a_r;q,p}]$:
\begin{align*}
&X_1^{k_1}\cdots X_r^{k_r}X_1^{l_1}\cdots X_r^{l_r}\notag\\*
&=\bigg(\prod_{1\le i<j\le r}Q_{a_i,a_j;q,p}(l_i,K_{j-1}-k_i+L_{i-1},k_i,k_j)\bigg)
X_1^{k_1+l_1}\cdots X_r^{k_r+l_r}.
\end{align*}
\end{lemma}
\begin{proof}
The identity is readily proved by multiple induction using
\begin{equation*}%\label{lemeq}
X_j^{k}X_i^{l}=Q_{a_i,a_j;q,p}(l,0,0,k)\,X_i^{l}X_j^k,
\end{equation*}
where $1\le i<j\le r$, for any pair of non-negative integers $k$ and $l$
(which is equivalent to the elliptic specialization of \cite[Lemma~1]{S1}),
combined with repeated application of the commutation rule \eqref{xfr}.
\end{proof}

The following is our main result.
\begin{theorem}[Elliptic multinomial theorem]\label{ellmthm}
  Let $n\in\N_0$.
  Then the following identity is valid in
$\C_{q,p}[X_1,\ldots,X_r,\E_{a_1,\ldots,a_r;q,p}]$:
\begin{equation*}%\label{eqminth}
(X_1+\cdots+X_r)^n=\sum_{k_1+\cdots+k_r=n}
\begin{bmatrix}n\\k_1,\ldots,k_r\end{bmatrix}_{a_1,\ldots,a_r;q,p}
X_1^{k_1}\cdots X_r^{k_r}.
\end{equation*}
\end{theorem}
\begin{proof}
We proceed by induction on $n$. For $n=0$ the formula is trivial.
Now let $n>0$ ($n$ being fixed) and assume that we have already
shown the formula for all non-negative integers less than $n$.
We have (by separating the last factor, applying induction,
applying a special case of Lemma~\ref{lem:com}, shifting the summation,
and finally combining terms using the recurrence relation
\eqref{rec-ellmc})
\begin{align*}
(X_1+\cdots+X_r)^n&=(X_1+\cdots+X_r)^{n-1}(X_1+\cdots+X_r)\\
&=\sum_{k_1+\cdots+k_r=n-1}
\begin{bmatrix}n-1\\k_1,\ldots,k_r\end{bmatrix}_{a_1,\ldots,a_r;q,p}
X_1^{k_1}\cdots X_r^{k_r}(X_1+\cdots+X_r)\\
&=\sum_{k_1+\cdots+k_r=n-1}
\sum_{i=1}^r\Bigg(
\begin{bmatrix}n-1\\k_1,\ldots,k_r\end{bmatrix}_{a_1,\ldots,a_r;q,p}
\bigg(\prod_{j>i}W_{a_i,a_j;q,p}^{(K_{j-1}-k_i)}(1+k_i,k_j)\bigg)\\*
&\qquad\qquad\qquad\qquad\quad\times
X_1^{k_1}\cdots X_{i-1}^{k_{i-1}}X_i^{k_i+1}
X_{i+1}^{k_{i+1}}\cdots X_r^{k_r}\Bigg)\\
&=\sum_{k_1+\cdots+k_r=n}
\sum_{i=1}^r\Bigg(
\begin{bmatrix}n-1\\k_1,\ldots,k_{i-1},k_i-1,k_{i+1},\ldots,k_r\end{bmatrix}_{a_1,\ldots,a_r;q,p}\\*
&\qquad\qquad\qquad\qquad\times
\bigg(\prod_{j>i}W_{a_i,a_j;q,p}^{(K_{j-1}-k_i)}(k_i,k_j)\bigg)
X_1^{k_1}\cdots  X_r^{k_r}\Bigg)\\
&=\sum_{k_1+\cdots+k_r=n}
\begin{bmatrix}n\\k_1,\ldots,k_r\end{bmatrix}_{a_1,\ldots,a_r;q,p}
X_1^{k_1}\cdots X_r^{k_r},
\end{align*}
which is what was to be shown.
\end{proof}

\section{The $\mathsf A_r$ Frenkel--Turaev summation by convolution}

By convolution, applied to the elliptic multinomial theorem
in Theorem~\ref{ellmthm}, we obtain the following result
which turns out to be equivalent to Rosengren's $A_r$ extension of
the Frenkel--Turaev $_{10}V_9$ summation.
\begin{theorem}\label{thm:cf-Ar-FT}
Let $0\le M\le N$ be two integers, an let $n_1,\ldots,n_r\in\N_0$
satisfying $N_r=n_1+\cdots+n_r=N$. Then we have
\begin{align}\label{eq:cf-Ar-FT}
&\begin{bmatrix}N\\n_1,\ldots,n_r\end{bmatrix}_{a_1,\ldots,a_r;q,p}\notag\\
  &=\sum_{k_1+\cdots+k_r=M}\Bigg(
    \begin{bmatrix}M\\k_1,\ldots,k_r\end{bmatrix}_{a_1,\ldots,a_r;q,p}
\begin{bmatrix}N-M\\n_1-k_1,\ldots,n_r-k_r\end{bmatrix}_{a_1q^{2M-k_1},\ldots,a_rq^{2M-k_r};q,p}\notag\\*
&\qquad\qquad\qquad\quad\times\prod_{1\le i<j\le r}Q_{a_i,a_j;q,p}
(n_i-k_i,N_{i-1}+K_{j-1}-K_i,k_i,k_j)\Bigg).
\end{align}
\end{theorem}
\begin{proof}
Working in $\C_{q,p}[X_1,\ldots,X_r,\E_{a_1,\ldots,a_r;q,p}]$, we expand
$(X_1+\cdots+X_r)^N=(X_1+\cdots+X_r)^M(X_1+\cdots+X_r)^{N-M}$
in two different ways and extract left coefficients of the monomial
$X_1^{n_1}\cdots X_r^{n_r}$ where $n_1+\cdots+n_r=N$.
On the left-hand side, the expansion is achieved by a single application
of Theorem~\ref{ellmthm}, which is simply
$$
(X_1+\cdots+X_r)^N=
\sum_{k_1+\cdots+k_r=N}
\begin{bmatrix}N\\k_1,\ldots,k_r\end{bmatrix}_{a_1,\ldots,a_r;q,p}
X_1^{k_1}\cdots X_r^{k_r},
$$
whose coefficient of $X_1^{n_1}\cdots X_r^{n_r}$ is clearly
$\left[\begin{smallmatrix}N\\
    n_1,\ldots,n_r\end{smallmatrix}\right]_{a_1,\ldots,a_r;q,p}$.
On the right-hand side we apply Theorem~\ref{ellmthm} twice
and bring the expression into normal form by multiple applications
of \eqref{xfr} (to the second elliptic multinomial coefficient)
and final apply Lemma~\ref{lem:com} (to bring the product of two monomials
into normal form). The details are as follows:
\begin{align*}
&(X_1+\cdots+X_r)^M(X_1+\cdots+X_r)^{N-M}\\
&=\sum_{k_1+\cdots+k_r=M}
\begin{bmatrix}M\\k_1,\ldots,k_r\end{bmatrix}_{a_1,\ldots,a_r;q,p}
X_1^{k_1}\cdots X_r^{k_r}\sum_{l_1+\cdots+l_r=N-M}
\begin{bmatrix}N-M\\l_1,\ldots,l_r\end{bmatrix}_{a_1,\ldots,a_r;q,p}
X_1^{l_1}\cdots X_r^{l_r}\\
&=\sum_{\substack{k_1+\cdots+k_r=M\\l_1+\cdots+l_r=N-M}}
\begin{bmatrix}M\\k_1,\ldots,k_r\end{bmatrix}_{a_1,\ldots,a_r;q,p}
\begin{bmatrix}N-M\\l_1,\ldots,l_r\end{bmatrix}_{a_1q^{2M-k_1},\ldots,a_rq^{2M-k_r};q,p}
X_1^{k_1}\cdots  X_r^{k_r}X_1^{l_1}\cdots X_r^{l_r}\\
&=\sum_{\substack{k_1+\cdots+k_r=M\\l_1+\cdots+l_r=N-M}}\Bigg(
\begin{bmatrix}M\\k_1,\ldots,k_r\end{bmatrix}_{a_1,\ldots,a_r;q,p}
\begin{bmatrix}N-M\\l_1,\ldots,l_r\end{bmatrix}_{a_1q^{2M-k_1},\ldots,a_rq^{2M-k_r};q,p}\\*
  &\qquad\qquad\qquad\qquad\times
    \bigg(\prod_{1\le i<j\le r}Q_{a_i,a_j;q,p}(l_i,K_{j-1}-k_i+L_{i-1},k_i,k_j)\bigg)
X_1^{k_1+l_1}\cdots X_r^{k_r+l_r}\Bigg).
\end{align*}
Taking left coefficients of  $x_1^{n_1}\cdots x_r^{n_r}$ evidently
gives the right-hand side of \eqref{eq:cf-Ar-FT}.
\end{proof}

The convolution identity in Theorem~\ref{thm:cf-Ar-FT} can be regarded
as the combinatorial form of the $\mathsf A_r$ Frenkel--Turaev summation:
\begin{align}\label{eq:Ar-FT}
&\frac{(b/a_1,\ldots,b/a_{r+1};q,p)_M}{(q,bz_1,\ldots,bz_r;q,p)_M}\notag\\
&=\sum_{k_1+\cdots+k_r=M}\prod_{1\le i<j\le r}
\frac{q^{k_i}\,\ta(z_jq^{k_j-k_i}/z_i)}{\ta(z_j/z_i)}
\prod_{i=1}^r\frac{\prod_{j=1}^{r+1}(a_jz_i;q,p)_{k_i}}
{(bz_i;q,p)_{k_i}\prod_{j=1}^r(z_iq/z_j;q,p)_{k_i}}.
\end{align}
This identity (which in essentially the same form is also stated
by Gasper and Rahman in their textbook \cite[Equation~(11.7.8)]{GR})
was first obtained by Rosengren in \cite[Theorem~5.1]{R04}.
The $r=2$ case of the identity in \eqref{eq:Ar-FT} is the single-sum
Frenkel--Turaev summation in \eqref{propfteq}.
The $p\to 0$ case of  the summation in \eqref{eq:Ar-FT}
was derived earlier by Milne~\cite[Theorem~6.17]{M88}.
Now, \eqref{eq:Ar-FT} contains \eqref{eq:cf-Ar-FT} as a special case:
In \eqref{eq:Ar-FT}, perform the following simultaneous substitutions: 
\begin{alignat*}{3}
  &a_i\mapsto q^{-n_i},&\quad&\text{for $1\le i\le r$},&\qquad\;
  &a_{r+1}\mapsto q^{-M},\\
 &z_i\mapsto 1/a_i,&&\text{for $1\le i\le r$},&
  &b\mapsto q^{-M-N}.
\end{alignat*}
These substitutions yield \eqref{eq:cf-Ar-FT} (after some rewriting).
On the contrary, after rewriting the elliptic multinomial coefficients
and weights in \eqref{eq:cf-Ar-FT} explicitly in terms of products
of theta-shifted factorials,
the restriction that  $n_1,\ldots,n_r$ are non-negative integer parameters
can be removed  by repeated analytic continuation.
This means that \eqref{eq:cf-Ar-FT} is actually equivalent to
\eqref{eq:Ar-FT}.

\begin{remark}
While the above derivation of \eqref{eq:cf-Ar-FT}
involved elliptic commuting variables and algebraic manipulations,
it is not difficult to give combinatorial interpretations of the
respective algebraic expressions
in terms of weighted lattice paths in the $r$-dimensional integer lattice
$\mathbb Z^r$. The multinomial $(X_1+\cdots+X_r)^N$
can be interpreted as the generating function for lattice paths
starting in the origin and consisting of $N$ unit steps
where the $i$th of the $r$ different unit steps increase
the $i$th coordinate in $\mathbb Z^r$ by one
while not changing the other coordinates.
With other words, starting in the origin,
after $N$ steps, the path reaches a point in the intersection of
$\mathbb Z^r$ with the hyperplane $z_1+\dots+z_r=N$.
In this interpretation, for any $r$-tuple of non-negative integers
$(k_1,\ldots,k_r)$ whose $i$th component is positive,
the weight of the unit step
$$
(k_1,\ldots,k_{i-1},k_i-1,k_{i+1},\ldots,k_r)\to
(k_1,\ldots,k_r)
$$
is
\begin{equation}\label{eq:weightsr}
 \prod_{i<j\le r}W_{a_i,a_j;q,p}^{(K_{j-1}-k_i)}(k_i,k_j),
\end{equation}
for any $i=1,\ldots,r$, in accordance with the recurrence relation of the
elliptic multinomial coefficients in \eqref{rec-ellmc}.
Assuming the weight of a lattice path in $\mathbb Z^r$
to be the product of the weights (which all are of the form
\eqref{eq:weightsr}) of the unit steps it is composed of,
the weighted generating function of the family of all lattice paths
that start in the origin $(0,\ldots,0)$ and, after
$N=n_1+\cdots+n_r$ unit steps, end in $(n_1\ldots,n_r)$,
is the elliptic multinonomial coefficient
$$
\begin{bmatrix}N\\n_1,\ldots,n_r\end{bmatrix}_{a_1,\ldots,a_r;q,p}.
$$
In this lattice path interpretation the convolution in
Theorem~\ref{thm:cf-Ar-FT} then concerns the generating function
of paths that start in the origin $(0,\ldots,0)$ and, after
$N=n_1+\cdots+n_r$ unit steps, end exactly in $(n_1\ldots,n_r)$
but is refined according to where, after $M$ steps (for fixed $M$
satisfying $0\le M\le N$),  the path crosses the hyperplane
$z_1+\dots+z_r=M$.

We believe that our derivation of \eqref{eq:cf-Ar-FT}
by convolution (which as we just explained, can be interpreted
in terms of a weighted counting of lattice paths)
constitutes the first combinatorial proof of
the $\mathsf A_r$ Frenkel--Turaev summation,
an identity that is of fundamental importance in the theory of
elliptic hypergeometric series associated with root
systems (cf.\ \cite{RW}).
\end{remark}

\begin{remark}
  It would be interesting to find a higher rank extension of
  the elliptic binomial theorem from
  \cite[Definition~5.6 and Theorem~5.7]{HKKS} that was mentioned
  in Remark~\ref{remHKKS} and to derive a corresponding
  multivariate Frenkel--Turaev summation by convolution in the same
  way as Theorem~\ref{thm:cf-Ar-FT} was derived in this section.
  We find this an interesting open problem worthwhile to pursuit.
  It is not clear whether such the obtained identity would be equivalent
  to Rosengren's $\mathsf A_r$ Frenkel--Turaev summation or whether
  it would be of a different type such as one of the multivariate
  Frenkel--Turaev summations listed in \cite{RW}.
  On the contrary, one can simply ask whether any of those other
  multivariate Frenkel--Turaev summations in \cite{RW} admit similar
  algebraic or combinatorial interpretations as \eqref{eq:Ar-FT} does.
\end{remark}

\begin{fremark} We hope this paper will inspire more research on
elliptic commuting variables and on
weighted lattice path enumeration in $\mathbb Z^r$. 
\end{fremark}

\section*{Statements and Declarations}
Data sharing is not applicable to this article as no datasets were
generated or analysed during the current study.
The Author further declares to have no relevant financial
or non-financial interests to disclose.

\begin{thebibliography}{99}

\bibitem{BK18} H.~Baba and M.~Katori,
``Excursion processes associated with elliptic combinatorics'',
{\em J.\ Stat.\ Phys.\ }\textbf{171} (2018), no.\ 6, 1035--1066.

\bibitem{BCK} N.~Bergeron, C.~Ceballos, and J.~K\"ustner.
``Elliptic and $q$-analogs of the Fibonomial numbers'',
{\em SIGMA Symmetry Integrability Geom.\ Methods Appl.\ }
\textbf{16} (2020), \#076.

\bibitem{B} G.~M.~Bergman,
``The diamond lemma for ring theory'',
{\em Adv.\ Math. }\textbf{29} (1978), 178--218.

\bibitem{Be} D.~Betea,
  ``Elliptically distributed lozenge tilings of a hexagon'',
{\em SIGMA} \textbf{14} (2018), 032, 39 pp.

\bibitem{BGR} A.~Borodin, V.~Gorin, and E.~M.~Rains,
``$q$-Distributions on boxed plane partitions'',
{\em Sel.\ Math.\ (N.S.)} \textbf{16} (2010), no.~4, 731--789.

\bibitem{CG} H.~Coskun and R.~A.~Gustafson,
``Well-poised Macdonald functions $W_\lambda$ and Jackson
coefficients $\omega_\lambda$ on $BC_n$'',
Proceedings of the workshop on Jack, Hall--Littlewood and
Macdonald polynomials,
{\em Contemp. Math. AMS} \textbf{417} (2006), 127--155.

\bibitem{DJKMO} E.~Date, M.~Jimbo, A.~Kuniba, T.~Miwa, and M.~Okado,
``Exactly solvable SOS models: local height probabilities and
theta function identities'', {\em Nuclear Phys.\ B}
\textbf{290} (1987), 231--273.

\bibitem{FT} I.~B.~Frenkel and V.~G.~Turaev,
``Elliptic solutions of the Yang--Baxter equation and modular
hypergeometric functions'', in V.I.~Arnold et al.\ (eds.),
{\em The Arnold--Gelfand Mathematical Seminars}, pp.~171--204,
Birkh\"auser, Boston, 1997.

\bibitem{GR} G.~Gasper and M.~Rahman,
{\em Basic hypergeometric series}, second edition,
Encyclopedia of Mathematics and Its Applications~\textbf{96},
Cambridge University Press, Cambridge, 2004.

%\bibitem{GK} I.~Gahramanov and A.~P.~Kels,
%``The star-triangle relation, lens partition function, and
%hypergeometric sum/integrals'',
%{\em J.\ High Energy Phys.\ }\textbf{2017}(2), 40, (2017).

\bibitem{HKKS} N.~Hoshi, M.~Katori, T.~H.~Koornwinder, and M.~J.~Schlosser,
``On an identity of Chaundy and Bullard. III. Basic and elliptic extensions'',
{\em Contemp.\ Math.\ AMS}; to appear; 
\href{https://arxiv.org/abs/2304.10003/}{arXiv:2304.10003}
  
\bibitem{J89} M.~Jimbo,
``Introduction to the Yang--Baxter equation'',
{\em  Internat.\ J.\ Modern Phys.\ A} \textbf{4}
(1989), no.~15, 3759--3777.
  
\bibitem{Ka15} M.~Katori,
``Elliptic determinantal processes of type $A$'',
{\em Probab.\ Theory Related Fields} \textbf{162} (2015),
no.\ 3--4, 637--677.

%\bibitem{Ka16} M.~Katori,
%``Elliptic Bessel processes and elliptic Dyson models realized as
%inhomogeneous processes'',
%{\em J.\ Math.\ Phys.\ }\textbf{57} (2016), no.\ 10, 103302, 32 pp.

\bibitem{Ka17} M.~Katori,
  ``Elliptic determinantal processes and elliptic Dyson models'',
{\em SIGMA} \textbf{13} (2017), \#079, 36 pp.

\bibitem{Ka19a} M.~Katori,
  ``Macdonald denominators for affine root systems, orthogonal theta
  functions, and elliptic determinantal point processes'',
{\em J.\ Math.\ Phys.\ }\textbf{60} (2019), no.\ 1, 013301, 27 pp.

\bibitem{Ka19b} M.~Katori,
``Two-dimensional elliptic determinantal point processes and
related systems'',
{\em Comm.\ Math.\ Phys.\ }\textbf{371} (2019), no.\ 3, 1283--1321.

\bibitem{Ka23} M.~Katori,
``Elliptic extensions in statistical and stochastic systems'',
{\em SpringerBriefs in Math.\ Phys.\ }\textbf{47}, xiv+125 pp.,
Springer Nature Singapore, 2023;
\url{https://doi.org//10.1007/978-981-19-9527-9}

\bibitem{Ke15} A.~P.~Kels,
  ``New solutions of the star-triangle relation with discrete
  and continuous spin variables'',
{\em J.\ Phys.\ A: Math.\ Theor.\ }\textbf{48} (2015), 435201.

\bibitem{MS18} K.~Y.~Magadov and V.~P.~Spiridonov,
``Matrix Bailey lemma and the star-triangle relation'',
{\em SIGMA} \textbf{14} (2018), 121, 13 pp.

%Theorem 6.17 is the $A_r$ Jackson summation
\bibitem{M88} S.\ C.\ Milne,
``Multiple $q$-series and $\mathrm U(n)$ generalizations of
Ramanujan’s ${}_1\Psi_1$ sum'', in: {\em Ramanujan revisited},
(Urbana--Champaign, 1987), 473--524, Academic Press, Boston, MA, 1988.

\bibitem{O44} L.~Onsager,
``Crystal statistics. I. A two-dimensional model with an
order-disorder transition'',
{\em Phys.\ Rev.\ (2)} \textbf{65} (1944), no.~3--4, 117--149.

\bibitem{R06} E.~M.~Rains,
``$BC_n$ symmetric abelian functions'',
{\em Duke Math.\ J.\ }\textbf{135} (2006),  no.~1, 99--180. 

%Theorem 5.1 is the $A_r$ Frenkel--Turaev summation
\bibitem{R04} H.~Rosengren,
``Elliptic hypergeometric series on root systems'',
{\em Adv.\ Math.\ }\textbf{181} (2004), 417--447.

\bibitem{R21} H.~Rosengren,
``Elliptic hypergeometric functions'',
in H.~S.~Cohl and M.~E.~H.~Ismail (eds.),
\emph{Lectures on Orthogonal Polynomials and Special Functions},
London Math. Soc. Lecture Note Ser.\ \text{464},
Cambridge University Press, 2021, pp. 213--279.
\href{https://arXiv.org/abs/1608.06161}{arXiv:math/1608.06161v3}.

%\bibitem{RS} H.~Rosengren and M.~J.~Schlosser,
%``Elliptic determinant evaluations and the Macdonald identities
%for affine root systems'',
%{\em Compos.\ Math.\ }\text{142} (2006), no.~4, 937--961.

\bibitem{RW} H.~Rosengren and S.~O.~Warnaar,
``Elliptic hypergeometric functions associated with root systems'',
in: {\em Encyclopedia of Special Functions: The Askey-Bateman Project}
(ser.\ eds. M.~E.~H.~Ismail and W.~van~Assche),
Volume 2. Multivariable Special Functions
(vol.~eds. T.~H.~Koornwinder and J.~Stokman), Cambridge Univ. Press,
Cambridge, 2020, pp. 159--186.

\bibitem{S} M.~J.~Schlosser,
``Elliptic enumeration of nonintersecting lattice paths'',
{\em J.\ Combin.\ Theory Ser.\ A} \textbf{114} (2007), no.\ 3, 505--521.

\bibitem{S1} M.~J.~Schlosser,
``A noncommutative weight-dependent generalization of the binomial theorem'',
{\em S\'em.\ Lothar.\ Combin.\ }\textbf{B81j} (2020), 24 pp.

\bibitem{SY1} M.~J.~Schlosser and M.~Yoo,
``Elliptic rook and file numbers'',
{\em Electron.\ J.\ Combin.\ }\textbf{24}(1) (2017), \#P1.31, 47 pp.

\bibitem{SY2} M.~J.~Schlosser and M.~Yoo,
``Elliptic extensions of the alpha-parameter model and
the rook model for matchings''
{\em Adv.\ Appl.\ Math.\ }\textbf{84} (2017), 8--33.

\bibitem{Sp02} V.~P.~Spiridonov, 
``Theta hypergeometric series'',
in V.A.~Malyshev and A.M.~Vershik (eds.),
{\em Asymptotic Combinatorics with Applications to Mathematical Physics},
pp.~307--327, Kluwer Acad.\ Publ., Dordrecht, 2002.

\bibitem{Sp02b} V.~P.~Spiridonov, 
``An elliptic incarnation of the Bailey chain'',
{\em Int.\ Math.\ Res.\ Not.\ }\textbf{37} (2002), 1945--1977.

\bibitem{Sp11} V.~P.~Spiridonov, 
``Theta hypergeometric terms'',
in {\em Arithmetic and Galois theories of differential equations},
pp.~325--345,
S\'emin.\ Congr.\ \textbf{23}, Soc.\ Math.\ France, Paris, 2011;
\href{https://arXiv.org/abs/1003.4491}{arXiv:math/1003.4491v3}.

%\bibitem{Sp2} V.P.~Spiridonov,
%``Essays on the theory of elliptic hypergeometric functions'',
%Russian Math.\ Surveys \textbf{63} (2008), 405--472.
 
%\bibitem{W} S.O.~Warnaar,
%``Summation and transformation formulas
%for elliptic hypergeometric series'',
%{\em Constr.\ Approx.\ }\textbf{18} (2002), 479--502.

\bibitem{WW} E.~T.~Whittaker and G.~N.~Watson,
{\em A Course of Modern Analysis}, 4th ed.,
Cambridge University Press, Cambridge, 1962.

\end{thebibliography}

\end{document}

