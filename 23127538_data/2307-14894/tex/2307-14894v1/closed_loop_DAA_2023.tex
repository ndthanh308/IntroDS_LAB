%%%%%%%%%%%%%%%%%%%%%%%%%%%%%%%%%%%%%%%%%%%%%%%%%%%%%%%%%%%%%%%%%%%%%%%%%%%%%%%%
%2345678901234567890123456789012345678901234567890123456789012345678901234567890
%        1         2         3         4         5         6         7         8

\documentclass[letterpaper, 10 pt, conference]{ieeeconf}  % Comment this line out if you need a4paper

%\documentclass[a4paper, 10pt, conference]{ieeeconf}      % Use this line for a4 paper

\IEEEoverridecommandlockouts                              % This command is only needed if 
                                                          % you want to use the \thanks command

\overrideIEEEmargins                                      % Needed to meet printer requirements.

%In case you encounter the following error:
%Error 1010 The PDF file may be corrupt (unable to open PDF file) OR
%Error 1000 An error occurred while parsing a contents stream. Unable to analyze the PDF file.
%This is a known problem with pdfLaTeX conversion filter. The file cannot be opened with acrobat reader
%Please use one of the alternatives below to circumvent this error by uncommenting one or the other
%\pdfobjcompresslevel=0
%\pdfminorversion=4

% See the \addtolength command later in the file to balance the column lengths
% on the last page of the document

% The following packages can be found on http:\\www.ctan.org
%\usepackage{graphics} % for pdf, bitmapped graphics files
%\usepackage{epsfig} % for postscript graphics files
\usepackage{graphicx}
%\usepackage{mathptmx} % assumes new font selection scheme installed
%\usepackage{times} % assumes new font selection scheme installed
%\usepackage{amsmath} % assumes amsmath package installed
%\usepackage{amssymb}  % assumes amsmath package installed
\usepackage{courier}
\usepackage{tabularx}

\title{\LARGE \bf
Analyzing the Closed-Loop Performance of Detect-And-Avoid Systems
}


\author{\'Italo Romani de Oliveira, Thiago Matsumoto, Aaron Mayne, Antonio Gracia Berna
\thanks{*The authors are with Boeing Research \& Technology, which supported this work. The corresponding author is 
        {\tt\small italo.romanideoliveira@boeing.com}
}
}

\begin{document}



\maketitle
\thispagestyle{empty}
\pagestyle{empty}


%%%%%%%%%%%%%%%%%%%%%%%%%%%%%%%%%%%%%%%%%%%%%%%%%%%%%%%%%%%%%%%%%%%%%%%%%%%%%%%%
\begin{abstract}
        The Detect-And-Avoid (DAA) algorithms for unmanned air vehicles in civil airspace have industry standards called Minimum Operational Performance Standards (MOPS), which establish clear criteria to check whether they can ensure safe separation for all plausible operational conditions. However, these MOPS ensure performance for the avoidance maneuvers, which are open-loop, but not for the maneuvers that bring the air vehicles back to their intended courses after they are deemed clear of conflict, which close the control loop of the missions. In this paper, we analyze the closed-loop performance of existing DAA algorithms that fulfill certain MOPS', by experimenting large numbers of traffic configurations with 4 aircraft in a delimited airspace. 

        We perform this analysis by obtaining their rates of loss of separation and timeout events, the latter happening when a chain of maneuvers exceeds the maximum supply of energy in a vehicle. In pair with these indicators, we measure the efficiency of the closed-loop logic, expressed as the excess fuel rate, and study the relationship between safety and efficiency in these scenarios. Our results show that the inefficiency caused by DAA algorithms is very significant in dense airspaces and that it is necessary to do considerable research in devising closed-loop mission management that takes into account the necessity of acting on DAA advisories and then having a safe and efficient logic to resume to mission.
        
        Performing the simulations of the closed-loop mission management logic can be highly time consuming, depending on the implementation of the DAA algorithm. Despite there being Neural Network approximations of DAA logic that alleviate the computational load, none of those that we found available can properly handle cases with more than one intruder in the ownship surveillance range. Therefore, in an attempt to overcome the high computational cost of analyzing the closed-loop performance of DAA algorithms, we study the correlation between inefficiency and safety of closed-loop scenarios, and some indicators from the corresponding open-loop scenarios, such as angle of deviation and number of deviations per flight time.  
\end{abstract}


%%%%%%%%%%%%%%%%%%%%%%%%%%%%%%%%%%%%%%%%%%%%%%%%%%%%%%%%%%%%%%%%%%%%%%%%%%%%%%%%
\section{INTRODUCTION}
\section{Introduction}
Current quantum hardware is unable to carry out universal quantum computations due to the buildup of errors that occur during the computation. 
The magnitude of the individual error is currently above the value that the Threshold Theorem requires in order to kick-start quantum error correction and fault-tolerant quantum computation~\cite[Section 10.6]{nielsen_chuang_2010}. 
Although the experimentally achieved fidelity rates are promising and the error bounds are inching closer to the required threshold, we will have to work for the foreseeable future with quantum hardware with errors that build-up during the computation.  This implies that we can only do a limited number of steps before the output of the computation has become completely uncorrelated with the intended one.

For fault-tolerant quantum computing, we repeat four steps: 
1) We apply a number of single and two-qubit quantum gates, in parallel whenever possible; 
2) We perform a syndrome measurement on a subset of the qubits; 
3) We perform fast classical computations to determine which errors have occurred and how to correct them; 
and, 4) We apply correction terms based on the classical computations.
We then repeat these four steps with a next sequence of gates. 
These four steps are essential to fault-tolerant quantum computing. 


The starting point of this work is to use the four steps outlined above, not to carry out error correction and fault-tolerant computation, but to enhance short, constant-depth, {\em uncorrected} quantum circuits that perform single qubit gates and {\em nearest-neighbor} two qubit gates. 
Since in the long run we will have to implement error-correction and fault-tolerant computation anyhow, and this is done by such a four-step process, why not make other use of this architecture? Moreover, on some of the quantum hardware platforms, these operations are already in place.
Embracing this idea we naturally arrive at the question: what is the computational power of \textit{low-depth} quantum-classical circuits organized as in the four steps outlined above? 
We thus investigate circuits that execute a small, ideally constant, number of stages, where at each stage we may apply, in parallel, single qubit gates and {\em nearest-neighbor} two qubit gates, followed by measurements, followed by low-depth classical computations of which the outcome can control quantum gates in later stages. 
It is not clear, at first, whether such circuits, especially with constant depth, can do anything remotely useful. 
But we will see that this is indeed the case: many quantum computations can be done by such circuits in constant depth. 
By parallelizing quantum computations in this way, we improve the overall computational capabilities of these circuits, as we do not incur errors on qubits that are idle, simply because qubits are not idle for a very long time. 
Furthermore, reducing the depth of quantum circuits, at the cost of increasing width, allows the circuit to be run faster even if errors occur.

The first usage of such a four-step layout, not to do error correction, but to perform computations, can be found in the paradigm of measurement-based quantum computing~\cite{gottesman1999demonstrating,raussendorf2001one,jozsa2006introduction,clark2007generalised}: 
A universal form of quantum computing where a quantum state is prepared and operations are performed by measuring qubits in different bases, depending on previous measurements and intermediate measurements.

\citeauthor{PhamSvore2013} were the first to formalize the four-step protocol for performing computations~\cite{PhamSvore2013}. They included specific hardware topologies by considering two-dimensional graphs for imposing constraints on qubit interactions. In their model, they develop circuits for particularly useful multi-qubit gates, including specifying costs in the width, number of qubits, depth, number of concurrent time steps, size, and total number of non-Identity operations.
As a result, they find an algorithm that factors integers in polylogarithmic depth.
\citeauthor{Browne:2011} showed that the main tool in the work by \citeauthor{PhamSvore2013}, the fan-out gate, can also be replaced by additional log-depth classical computations in the measurement-based quantum computing setting~\cite{Browne:2011}.

More recently, \citeauthor{Cirac:2021} introduced a scheme to implement unitary operations involving quantum circuits combined with Local Operations and Classical Communication ($\mathsf{LOCC}$) channels: $\mathsf{LOCC}$-assisted quantum circuits~\cite{Cirac:2021}. Similarly to the four-step scheme we just described, they allow for a short depth circuit to be run on the qubits, followed by one round of $\mathsf{LOCC}$, in which ancilla qubits are measured and local unitaries are applied based on the measurement outcomes. They show that in this model any 1D transitionally invariant matrix-product state (MPS) with fixed bond dimension is in the same phase of matter as the trivial state. Similar ideas can be found in~\cite{TVV_NonAbelianTopologicalOrder_2022, tantivasadakarn2021long}.

In this work, we introduce a new model, called \textit{Local Alternating Quantum-Classical Computations} ($\LAQCC$). In this model we alternate between running quantum circuits (constrained by locality), ending in the measurement of a subset of qubits, and fast classical computations based on the measurement results. The outcome of the classical computations are then used to control future quantum circuits. We allow for flexibility in this model, by giving different constraints to the power of both the quantum circuits and the classical circuits as well as the number of alternations between them. 
Most attention will be given to $\LAQCC$ containing quantum circuits of constant depth, classical circuits of logarithmic depth and at most a constant number of alternations between them. 
Any circuit constructed in this model is considered to be of constant depth. 
We restrict ourselves to logarithmic depth classical computations, as this is the first natural and non-trivial extension beyond constant-depth classical computations. 
Constant-depth classical computations do however also have an equivalent constant-depth quantum implementation.

The definition of $\LAQCC$ sharpens the original definition of \citeauthor{PhamSvore2013} by adding constraints to the intermediate classical computations. This allows us to bound the power of $\LAQCC$ from above. 

The main result of \citeauthor{Cirac:2021}, that 1D translational invariant MPS with fixed bond dimension can be prepared by $\mathsf{LOCC}$-assisted circuits, relies on local symmetries of the MPS. These symmetries allow them to prepare local states (on a constant number of qubits) and glue them together by doing one round of the appropriate entangling measurement and corrections, after which they run a round of local unitaries to get the desired result. This general scheme for preparing states that exhibit an MPS description with the appropriate local symmetries requires only geometrically local unitaries and one round of measurement and corrections an therefore is accessible in $\LAQCC$. Studying different local symmetries, known as Symmetry Protected Topological (SPT) phases of matter, to find measurement-based constant depth circuits for states is a broad ongoing field of research~\cite{TVV_NonAbelianTopologicalOrder_2022, tantivasadakarn2021long, smith2023deterministic}. 
All these schemes have a $\LAQCC$ implementation.

%$\LAQCC$-circuits also exist for general schemes of preparing local states, based on the local tensors, and gluing them together using one round of entangled measurement and corrections, based on the local symmetry. 
%The main result of \citeauthor{Cirac:2021}, that 1D translational invariant MPS with fixed bond dimension can be prepared by $\mathsf{LOCC}$-assisted circuits, relies heavily on local symmetries of the MPS and as a result also has an equivalent $\LAQCC$ implementation. 
%The corrections applied after the measurement round are local unitaries depending on the local symmetries of the MPS. 

 

%This general scheme of preparing local states, based on the local tensors, and gluing it together by doing one round of entangled measurement and corrections, based on the local symmetry, is accessible in $\LAQCC$.
Note however that \citeauthor{Cirac:2021} also suggest a circuit for the $W$-state.
This circuit uses sequentially and dependent measurement-based corrections of the ancilla qubits. 
These dependent measurements translate to sequential alternations between the quantum and classical circuits and therefore increase the total depth to linear depth, exceeding the constant-depth constraints imposed by $\LAQCC$-circuits. 

We study the power of the $\LAQCC$ model with respect to state preparation, showing that even with only constant quantum-depth and logarithmic classical depth it remains possible to prepare states with long-range entanglement.
Another surprising result is that it is unlikely that $\LAQCC$ circuits are classically simulatable. We show that any instantaneous quantum polynomial-time (IQP) circuit~\cite{Bremner2010,Shepherd2009} has an $\LAQCC$ implementation.
Classical simulation of IQP circuits implies the collapse of the polynomial hierarchy to the third level, which is not believed to be true~\cite{Bremner2017}. Therefore, we expect that $\LAQCC$ circuits are unlikely to be classically simulatable. We bound the power of $\LAQCC$ by showing that it is contained in $\QNC^1$, the class of polynomial-size, log-depth circuits.

Next, we also study the power that intermediate classical calculations can add to quantum computations, by considering a new model that alternates between polynomially many polynomial-depth quantum circuits and unbounded classical computations
We study this model by doing a complexity theoretical analysis, where we draw inspiration from the notions of complexity given by \citeauthor{RosenthalYuen:2022}, \citeauthor{MetgerYuen:2023}, and \citeauthor{Aaronson:2004}.
All three complexity notions are based on the notion of state preparation, instead of more traditional definition of complexity such as the decidability of a computational problem. 
The first two consider classes based on sequences of quantum states preparable by a polynomial-sized quantum circuit, where the circuits are uniformly generated by a computational class, for instance, the class $\mathsf{PSPACE}$, which results in the complexity class $\mathsf{StatePSPACE}$~\cite{RosenthalYuen:2022,MetgerYuen:2023}.
The third notion considers a relative complexity, where the complexity is measured between two given states, and is measured by the number of gates, from a given gate-set, required to transform one state in another state~\cite{Aaronson:2004}. 
For our definition of state preparation complexity, we drop the uniformity constraint from~\cite{RosenthalYuen:2022,MetgerYuen:2023} and define a class as $\mathsf{StateX}$, which refers to states preparable by circuits of type $\mathsf{X}$. 
As an example, if $\mathsf{X} = \QNC^0$, this results in the class $\mathsf{StateQNC^0}$, which is the set of states preparable from the $\ket{0}^n$ state by poly-size constant-depth circuits. 
This notion is similar to the relative complexity from~\cite{Aaronson:2004}, where one state is the  $\ket{0}^n$ state and instead of counting the number of gates we consider the set of states preparable by a fixed number of gates. Using this notion of complexity we show that any state preparable by an $\LAQCC^*$ circuit is also preparable by a $\mathsf{PostQPoly}$ circuit, the class of circuits of polynomial depth with an additional post-selection gate. 

All Clifford circuits have a constant-depth $\LAQCC$ implementation, implying that any stabilizer state can be implemented by a constant-depth $\LAQCC$ circuit, see Section~\ref{sec:clifford_circuits} for a proof of this statement. 
Efficient circuits for stabilizer states have been known already through measurement-based quantum computing. Therefore this paper focuses on the preparation of non-stabilizer states, and as a surprising result we find novel constant-depth protocols for four very natural classes of non-stabilizer states.
Despite the extensive research into these four classes of non-stabilizer states and the many applications of them, no efficient constant- or low-depth state preparation protocols are known yet. We specifically consider these four classes as they are all often used as initial states in other algorithms.

The first state is a uniform superposition over an arbitrary number of states. 
This state finds applications in many quantum algorithms, as they often start with a uniform superposition over multiple states. 
This superposition is often achieved by applying Hadamard gates to every qubit due to its simplicity to prepare. 
Yet, the analysis of many algorithms, such as Shor's algorithm~\cite{Shor:1997}, would benefit from a different initial superposition. 
The circuit to prepare the uniform superposition over an arbitrary number of states uses an exact version of Grover search as a subroutine, that turns a probabilistic circuit, with a known constant probability of success, into a deterministic circuit. 
We use the circuit for preparing a uniform superposition over an arbitrary number of states as a subroutine in the next two quantum state preparation protocols. 

The second state is the $W$-state, the uniform superposition over all computational basis states of Hamming-weight~$1$, a natural long-ranged entangled state that displays a fundamentally nonequivalent type of entanglement from the Greenberger–Horne–Zeilinger state~\cite{WState:2000}, for which $\LAQCC$-type constant-depth circuits were previously known~\cite{PhamSvore2013, Cirac:2021}. 
The $W$-state is often used as benchmark for new quantum hardware~\cite{Haffner2005,Neeley2010,GarciaPerez:2021}. 
A novel way to prepare the $W$-state therefore gives a new way to benchmark different quantum devices with each other. 
A circuit for preparing the $W$-state was given in~\cite{Cirac:2021}, but this implementation requires sequentially alternating measurements followed by local unitaries, which in the $\LAQCC$ model is not considered to be of constant depth. 
We improve this protocol by giving an $\LAQCC$ implementation of the $W$-state, based on a compress-uncompress method that links the one-hot and binary encoding of integers.

The third state considered is the Dicke state, a generalization of the $W$-state, a superposition over all computational basis states with Hamming-weight $k$~\cite{Dicke:1954}. 
Dicke states have relevance in various practical settings.
For instance, for quantum game theory~\cite{zdemir2007}, quantum storage~\cite{Bacon_Compress:2006,Plesch:2010}, quantum error correction~\cite{ouyang2014permutation}, quantum metrology~\cite{toth2012multipartite}, and quantum networking~\cite{prevedel2009experimental}. 
Dicke states have been used as a starting state for variational optimization algorithms, most notably Quantum Alternating Operator Ansatz (QAOA)~\cite{Hadfield2019}, to find solutions to problems such as Maximum k-vertex Cover~\cite{Brandhofer2022,cook2020quantum}.
The ground states of physical Hamiltonians describing one-dimensional chains tend to show a resemblance to Dicke states such as states resulting from the Bethe ansatz, making them an ideal starting state when investigating the ground state behavior of these Hamiltonians~\cite{TDL_BetheAnsatzDerivation:2010,B_ExcitedStateQuantumPhaseTransitions:2013,DickeTransitions:2021}. 
For instance, the algorithm by \citeauthor{van2021preparing}, who give an algorithm to prepare the Bethe ansatz eigenstates of the spin-1/2 XXZ spin chain, starts by first preparing a Dicke state~\cite{van2021preparing}. 
A Dicke-state preparation protocol based on the compress-uncompress methodology used in the $W$-state furthermore finds applications in entanglement distillation, where the entanglement of a large state is concentrated on only a few qubits. 
Efficient deterministic circuits for preparing Dicke states have been proposed by \citeauthor{bartschi2019deterministic}~\cite{bartschi2019deterministic, bartschi2022deterministic_short_depth}. 
They provide a quantum circuit of depth $\mathO(k \log(\frac{n}{k}))$, allowing arbitrary connectivity, to prepare a Dicke state, which they conjecture to be optimal when $k$ is constant. 
In this work, we provide a constant-depth $\LAQCC$ circuit below their conjectured bound already for constant $k$. 
However, this does not directly disprove their conjecture, as we allow for intermediate measurements and classical computations. 
More significantly, we even construct constant-depth $\LAQCC$ circuits for $k = \mathO(\sqrt{n})$ greatly improving their bound.
This construction extends the compress-uncompress method for the $W$-state combined with additional subroutines. 

We continue with a log-depth state preparation protocol for the Dicke-state for arbitrary $k$. 
This protocol implements an efficient transformation between the factoradic number representation and the combinatorial number representation of a positive integer. 
The combinatorial number representation relates directly to the Dicke state. 
The provided efficient transformation between number representation systems might be of independent interest. 

We conclude by modifying our protocol for preparing a Dicke-state to a protocol that prepares quantum many-body scar states in constant-depth. 
These states have low entanglement and longer coherence times than states with similar energy density.
These characteristics make many-body scar states interesting to analyze and relevant within physics.
Many-body scar states appear for instance in the AKLT model~\cite{AKLT:1987,MRBAR:2018,MRB:2018} and different spin models~\cite{SI:2019,MOBFR:2020}.
Known methods for preparing these states have polynomial-depth~\cite{Gustafson:2023}, whereas our circuit has constant depth. 

% We conclude by studying the power that intermediate classical calculations can add to quantum computations. 
% In this study, we define a new model that relaxes constant-depth quantum circuits to polynomial depth quantum circuits, log-depth classical calculations to unbounded classical computations and a constant number of alternations to a polynomial number of alternations. 
% We call this model $\LAQCC^*$. 
% We study this model by doing a complexity theoretical analysis, where we draw inspiration from the notions of complexity given by \citeauthor{RosenthalYuen:2022}, \citeauthor{MetgerYuen:2023}, and \citeauthor{Aaronson:2004}.
% All three complexity notions are based on the notion of state preparation, instead of more traditional definition of complexity such as the decidability of a computational problem. 
% The first two consider classes based on sequences of quantum states preparable by a polynomial-sized quantum circuit, where the circuits are uniformly generated by a computational class, for instance, the class $\mathsf{PSPACE}$, which results in the complexity class $\mathsf{StatePSPACE}$~\cite{RosenthalYuen:2022,MetgerYuen:2023}.
% The third notion considers a relative complexity, where the complexity is measured between two given states, and is measured by the number of gates, from a given gate-set, required to transform one state in another state~\cite{Aaronson:2004}. 
% For our definition of state preparation complexity, we drop the uniformity constraint from~\cite{RosenthalYuen:2022,MetgerYuen:2023} and define a class as $\mathsf{StateX}$, which refers to states preparable by circuits of type $\mathsf{X}$. 
% As an example, if $\mathsf{X} = \QNC^0$, this results in the class $\mathsf{StateQNC^0}$, which is the set of states preparable from the $\ket{0}^n$ state by poly-size constant-depth circuits. 
% This notion is similar to the relative complexity from~\cite{Aaronson:2004}, where one state is the  $\ket{0}^n$ state and instead of counting the number of gates we consider the set of states preparable by a fixed number of gates. Using this notion of complexity we show that any state preparable by an $\LAQCC^*$ circuit is also preparable by a $\mathsf{PostQPoly}$ circuit, the class of circuits of polynomial depth with an additional post-selection gate. 

\paragraph{Summary of results}
\begin{itemize}
    \item We give a new definition of a computational model that captures the power of the four step process: applying a constant number of layers of one- and two-qubit gates; performing a syndrome measurement; perform a fast classical computation determining corrections; apply corrections. We call this model \emph{Local Alternating Quantum Classical Computations}, or $\LAQCC$ for short. In this model we bound the allowed quantum operations, intermediate classical calculations, and number of rounds separately. In Section~\ref{sec:LAQCC_model} we define this model and give a list of operations based on results from literature contained in this computational model. In some of these operations we explicitly use that we allow for multiple, but at most constant, rounds  of corrections.
    \item  We show show that there exist $\LAQCC$ circuits that can not be weakly simulated in Section~\ref{sec:IQP_in_LAQCC}. We further show that for every $\LAQCC$ circuit there exists a $\QNC^1$ circuit simulating it perfectly, in Section~\ref{sec:LAQCC_in_QNC1}.
    \item We introduce a new type computational complexity for preparing states and show that the extension of $\LAQCC$ where we allow a polynomial number of rounds and unbounded classical computation, is contained in $\mathsf{PostQPoly}$, the class of polynomial circuits with post-selection, in Section~\ref{sec:Complexity results}.
    \item We show a protocol to prepare the uniform superposition state of size $q$ in $\LAQCC$ using $\mathO(\ceil{\log_2(q)}^2)$ qubits in Section~\ref{sec:superposition_modulo_q}. 
    \item We show a protocol to prepare the $W_n$ state in $\LAQCC$ using $\mathO(n\log(n))$ qubits in Section~\ref{sec:W_state_in_LAQCC}.
    \item We show two ways of preparing the Dicke-$(n,k)$ state. The first method is in $\LAQCC$, works up to $k = \mathO(\sqrt{n})$, uses $\mathO(n^2\log(n))$ qubits, and is found in Section~\ref{sec:dicke:small_k}. The second method is in $\LAQCC\text{-}\mathsf{LOG}$ (an extension of $\LAQCC$ allowing for logarithmic number of alterations instead of constant), works for any $k$, uses $\mathO(\text{poly}(n))$ qubits, and is found in Section~\ref{sec:Dicke_in_LAQCC_LOG}. 
    \item We extend on our $\LAQCC$ method of generating Dicke-$(n,k)$ states for $k = \mathO(\sqrt{n})$ and show a protocol to generate many-body scar states for a particular Hamiltonian in $\LAQCC$ (Section~\ref{sec:many_body_scar}). 
\end{itemize}
Summarized in a table, we provide the following state generation protocols:
\begin{table}[htb]
\centering
\begin{tabular}{l|l|l|l}
\textbf{State description} & \textbf{Width} & \textbf{Depth} & \textbf{Implementation}\\
\hline 
Uniform superposition mod $q$: $\frac{1}{\sqrt{q}} \sum_{i = 0}^{q-1}\ket{i}$ & $\mathO(\ceil{\log^2 q})$ & $\mathO(1)$ & Section~\ref{sec:superposition_modulo_q}\\

$W$-state: $\frac{1}{\sqrt{n}}\sum_{i = 0}^{n-1}\ket{e_i}$ & $\mathO(n \log n)$ & $\mathO(1)$ & Section~\ref{sec:W_state_in_LAQCC}\\

Dicke-$(n,k)$, $k = \mathO(\sqrt{n})$: $\binom{n}{k}^{-1/2}\sum_{x \in \{0,1\}^n: |x| = k} \ket{x}$ &  $\mathO(n^2\log n)$ & $\mathO(1)$ 
&Section~\ref{sec:dicke:small_k}\\

Dicke-$(n,k)$: $\binom{n}{k}^{-1/2}\sum_{x \in \{0,1\}^n: |x| = k} \ket{x}$ & $\mathO(\text{poly}(n))$ & $\mathO(\log n)$ &Section~\ref{sec:Dicke_in_LAQCC_LOG}\\

QMBS: $\ket{S_k} = \frac{1}{k! \sqrt{\mathcal N(n,k)}}(Q^\dagger)^k \ket{\Omega}$ &  $\mathO(n^2\log n)$ & $\mathO(1)$  &  Section~\ref{sec:many_body_scar}
\end{tabular}
\caption{Summary of state preparation protocols given in this paper.}
\label{tab:sate_prep}
\end{table}
In the entry for the quantum many-body scar state $Q$ denotes the raising operator and $\mathcal N(n,k)=\binom{n-k-1}{k}$. 
Section~\ref{sec:many_body_scar} will provide more details on the variables and the implementation. 

\paragraph{Organization of the paper}
\noindent We first introduce relevant preliminaries in Section~\ref{sec:preliminaries}. 
In Section~\ref{sec:LAQCC_model} we formally define the class of Local Alternating Quantum-Classical Computations ($\LAQCC$). We also show that any Clifford circuit can be implemented in constant depth $\LAQCC$ (a result based on a result from measurement-based quantum computing~\cite{jozsa2006introduction}). 
This result allows us to give many useful multi-qubit gates and routines in Section~\ref{sec:gates_created_in_LAQCC}. 
Beyond that we show that constant depth $\LAQCC$ circuits are contained in $\QNC^1$ and that any $\mathsf{IQP}$ circuit has an $\LAQCC$ implementation.
We conclude this section with an analysis of a more powerful instantiation of $\LAQCC$ and show an inclusion with respect to the class $\mathsf{PostQPoly}$, which is the class of circuits of polynomial depth with one additional post-selection gate. 
In Section~\ref{sec:state_prep_in_LAQCC} we give $\LAQCC$ circuit implementations for preparing the uniform superposition over an arbitrary number of states, the $W$-state and the Dicke state up to $k = \mathO(\sqrt{n})$. We furthermore give a log-depth circuit implementation for preparing the Dicke state for any $k$. We conclude by showing a $\LAQCC$ circuit for generating many body scar states of a particular type of Hamiltonian.



\section{MULTIVEHICLE SCENARIO DEFINITIONS}\label{section:scenario_definitions}
The multi-vehicle scenarios consist of an airspace of approximately 20 km of diameter, with 4 aircraft and definitions similar to \cite{IROliveira2021}, 
but here, one of the cases allows 3-D maneuvering. In this section, we present a summary of these common scenario definitions. The airspace was initially designed
with a cellular structure, in order to facilitate coordination among the agents, as shown in  Fig.~\ref{fig:cellular_airspace}.

% Figure environment removed

In this paper's analyses, the cells are not needed for coordination, however that structure is still needed to generate the large number of different traffic configurations, because the origin and destination points of an air vehicle's mission are placed in the outer cells (7-18). The traffic configurations are generated by a factorial algorithm with the following constraints:
\begin{enumerate}
	\item \label{itm:aircraft_id} An aircraft is uniquely identified by a numerical id between 0 and 3;
	\item An aircraft's mission is defined by an Origin and a Destination point, both being cell centroids;
	\item Origin points are all distinct among the aircraft, so as that they begin the scenario properly separated;
	\item \label{itm:o-d_distance} The Destination point must be distant from the Origin point by at least three cells in-between them;
\end{enumerate}
Constraint \#\ref{itm:aircraft_id} allows to define priority rules among the aircraft, and contributes to generate a higher number of distinct traffic configurations. Constraint \#\ref{itm:o-d_distance} is thought to exclude trajectories that are too short and, in a certain way, to promote convergence and crossing of the trajectories and obtain more conflicts. Implicitly, it forces that the origin and destination points be at the outer ring of the airspace. Then, by generating all possible combinations allowed by these constraints, one obtains 122,416 distinct traffic configurations. Still, it can be observed that some of these configurations are rotated versions of others, so, if the cell numbers did not matter, some of these configurations could be eliminated and we would remain with 46,660 traffic configurations (the full set does not contain 6 rotated replica subsets because constraint \#\ref{itm:o-d_distance} creates asymmetries). Anyway, we still chose to keep the full set because some conflict resolution algorithms use the reference to the north (and consequently, to the cell numbers) to decide tie-break situations, and even because some of them use look up tables or decision rules in which the absolute heading of the aircraft is an input. In some of our experiments, it was observed that ACAS sXu has slightly different resolution to traffic configurations that are rotated versions, besides occurrence of special events with DAIDALUS happening for just one scenario among the total 122,416, confirming that there is no strong rotational symmetry in the DAA implementations. 

An aircraft can go out of the cell area to execute a deviation maneuver, and we observed such occurrences with a small frequency, depending on the scenario specification. In reality, there might be borders with forbidden airspaces or obstacles, and those would be more significant safety threats. In any case, the aircraft tend to remain inside the airspace because they have to reach their destination through an efficient path. For the scenario specifications with only vertical maneuvers, the aircraft remain symbolically at 500 ft of altitude, while in the spec with vertical maneuvers, the aircraft have their origin and destination points at this altitude, and there will remain if no vertical maneuver happen. Vertical maneuvers are allowed only above this altitude, without any bound up, other than the requirement to return to the base altitude in the end of the flight mission. 

\subsection{Use of priorities}

Priorities can be used to determine a collective decision on how each aircraft will alter its trajectory in case of a traffic encounter of two or more aircraft \cite{ROMANIDEOLIVEIRA201792}. DAA attributes priorities to each aircraft in an encounter according to built-in, safety-driven rules. Although DO-365B \cite{DO_365} does not establish any decision rule based on priority, it requires a high priority flag indication for the remote pilot in case of an intruder is in emergency state. On the other hand, both DAIDALUS \cite{DaidalusPage} and DO-386 \cite{DO_386} define, respectively, priority rules for coordinated encounters, where priorities are defined according to observed aircraft equipage and the  aircraft identification code (the so-called ICAO address). The intention is to use objective rules to decrease the chances of aircraft performing maneuvers that worsen the conflict, as for example, turning onto each other. We distinguish these built-in priorities from the priorities explored in \cite{ROMANIDEOLIVEIRA201792}, by calling the latter as \emph{extrinsic} priorities, because they can be arbitrarily chosen among equal vehicles. And it is necessary bearing in mind that the DAA built-in priorities are used only for the divergent maneuvers, not for the return-to-mission phase. 

Considering the case where aircraft are equally equipped, DAA maneuver tiebreaking is performed by aircraft id, and is fair only if the probability distribution of aircraft ids per encounter follow a symmetric curve, which would hardly be the case in practice. Apart from tiebreaking, there is a general understanding in the DAA regulations that priorities must be observed when encounters occur with public service and medevac missions, vehicles in various sorts of emergency, etc. However, the afore mentioned standards do not need to define specific rules for encounters with such intruders, because they assume that such intruders will be treated as non-cooperative targets, either by proactive identification by the remote pilot, or by interaction with Air Traffic Control, or, as the last means, by the repeated refusal of cooperation by the intruder. 

It was observed in our previous works \cite{ROMANIDEOLIVEIRA201792,IROliveira2021} that, for encounters involving multiple (more than two) equally-righted air vehicles in dense airspaces, the use of extrinsic priorities improves the efficiency of the collective traffic system, because it decreases the total number of deviations and gives more predictability to the scenarios. In the case of DAA, they can be used to make DAA to suppress any alert associated to a cooperative intruder with lower priority than the ownship's. While this greatly improves efficiency of the scenarios, it also may increase the occurrence of losses of separation, and this trade-off has to be taken into account. Such trade-off also includes the occurrence of timeout events, already explored in our previous works. 

A timeout is the forced termination of a long chain of maneuvers without all the aircraft having accomplished their missions. In practice, this would be most likely associated with a fuel or energy emergency, when the aircraft's energy supply becomes critically low. Thus, at least one of the aircraft has to start a procedure of emergency landing, which is an undesirable event, both from the efficiency and the safety perspectives. The long chain of maneuvers is caused by the inability of the DAA logic to plan a coordinated return to mission for all the aircraft involved in an encounter, thus each aircraft decides independently on how to resume its intended path, and this lack of coordination causes maneuver repetitions, extensions, alternating cycles, and other bizarre sequences of maneuvers, that could also be called \emph{livelocks}. 

In some of our scenario specifications, we used the aircraft id as the sole criterion for extrinsic priorities. This choice is technically correct, however, in practice, it is not  an equitable policy, because the id remains unchanged for the lifetime of the vehicle, and that either blesses or condemns the vehicle with a persistent tendency to being promoted or being penalized. Thus, a randomized priority assignment would be needed, but that would have to be cheat-proof and fault-proof \cite{IROliveira2022}, which would bring some extra cost to the system.  

\subsection{Vertical maneuvering}

Most of the scenarios that we studied have only horizontal maneuvers, with all aircraft flying at the same altitude of 500 ft. This might happen in practice near airports, where the higher altitudes are reserved for manned aviation, and lower altitudes are not allowed due to ground obstacles or noise regulations. In any case, 2-D aircraft dynamics and mission control is much simpler and greatly helps in grasping the critical aspects of distributed Conflict Detection and Resolution (CD\&R). Later in this research, we developed an initial 3-D simulation model in which, if DAA presents avoidance bands in the vertical dimension, only the up sense is taken, a decision depending on a priority order for choosing which of the aircraft will maneuver or not. This was a simplification to treat the indeterminacy left by DO-365B and DAIDALUS. DO-365B associates coordinated encounters to certain classes of equipage, among which the Class 1 does not allow such coordination. 

In DAIDALUS, there is no explicit coordination among the aircraft and, because of that, the ownship ``pilot" can choose either direction and end up choosing the same direction of the intruder ``pilot". Thus, its resolution logic will be complete only with additional assumptions. It is easy to see that, in the vertical plane, if both aircraft maneuver up for conflict avoidance, there will be no avoidance. In the horizontal plane, where the aircraft dynamics has rotational symmetry, it is desirable that aircraft turn to the same relative direction, however sensor uncertainty and multi-aircraft encounters remain problematic without coordination. In our implementation of 3-D maneuvering, if DAIDALUS is prescribing an upwards vertical speed, that becomes a control input simultaneously to any horizontal deviation input, thus the resulting maneuver can happen in both planes at the same time. Once DAIDALUS signals that the aircraft is clear of conflict, the mission control logic will try to return to the destination altitude, provided that it does not create a new conflict immediately. With regards to selection of upward or downward resolution maneuver, the use of id-based priorities allowed our implementation to disregard downward advisories with acceptable results and solve the vertical indeterminacy in a simplified way.  For an in-depth study on the interoperability between horizontal and vertical advisories, one can refer to \cite{Londner2015}. 


\section{RESULTS AND ANALYSIS}
\subsection{Performance Indicators}\label{subsection:performance_indicators_definition}
We use three categories of indicators to analyze the performance of a scenario definition:

\textbf{Inefficiency Rate}: it is the difference in fuel spent between a traffic scenario in which the aircraft comply to DAA resolution advisories (RA), and the analogous scenario in which each vehicle follows its optimal path, without observation of traffic separation rules. In case of vertical deviations, a higher fuel rate is required for climb maneuvers, a lower fuel rate in descent maneuver, which increase the net total for the mission. The resulting value of this indicator is the average value over all traffic configurations in a scenario set. 

\textbf{Loss of Separation (LoS) Rate}: despite there being several ways of defining traffic separation, we examine just the simplest one, which is checking whether or not the vehicles are separated by at least a fixed \emph{minimum distance}. In order to include both DAIDALUS and ACAS sXu in the same tables, we use one separation distance from each one, respectively: 4,000 ft, which is the Horizontal Miss Distance, or HMD, used in DO-365B \cite{DO_365} to define the so-called \emph{Hazard Alert Zone} (HAZ), associated to the DAA Well-Clear (DWC) concept of separation; and 2,000 ft, used in DO-396 \cite{DO_396}, that defines the Loss of Well-Clear (LoWC) event in relation to large UAVs or manned aircraft. These indicators will denote the rate of scenarios, in a scenario set, where the distance between any aircraft pair fell below the afore mentioned threshold values. 

\textbf{Timeout Rate}: this indicates, in a scenario set, the rate of scenarios where any aircraft exceeded a maximum time without reaching its destination point. As pointed out in section \ref{section:scenario_definitions}, this phenomenon occurs because DAA Resolution Advisories (RAs) cause long chains of maneuvers that extend beyond the energy/fuel allowance of the vehicle, due to shortcomings in coordination. We use a time threshold of 1,000 seconds but, in practice, the timeout would be determined by the energy/fuel capacity of the vehicle.

\textbf{Scenario Computing Time}: the time needed to simulate a single scenario instance, in a single core of an Intel Xeon CPU, discounted the fact that multiple scenario instances can be run in parallel in a multi-core CPU. In our simulated scenarios, the DAA algorithm is called at least each 2 seconds, for each aircraft, but when the aircraft is in avoidance mode, that can happen more often. In the case of ACAS sXu, the requirement of receiving various messages to update a single track contribute to result in multiple calls per simulated second.


\subsection{Scenario Specifications and Labels}
In this study, a scenario specification is defined by features such as: the DAA algorithm used, the dimensionality (2-D or 3-D), if it uses extrinsic priorities or not, the target separation parameter, and possibly other features. The scenario labels used in this section encode these attributes:
\begin{itemize}
\item \texttt{dai\_ip\_2d\_4k}: DAIDALUS without extrinsic priorities, 2-D maneuvering and regular 4 kft Horizontal Miss Distance (HMD);
\item \texttt{dai\_ep\_2d\_4k}: similar to the above, with extrinsic priorities;
\item \texttt{sxu\_ip\_2d\_2k}: ACAS sXu with intrinsic priorities only, 2-D maneuvering and regular 2 kft LoWC threshold;
\item \texttt{sxu\_ep\_2d\_2k}: similar to the above, with extrinsic priorities;
\item \texttt{dai\_ep\_3d\_4k}: similar to \texttt{dai\_ep\_2d\_4k}, with 3-D maneuvering;
\item \texttt{dai\_ep\_2d\_2k}: similar to \texttt{dai\_ep\_2d\_4k}, with HMD reduced to 2 kft.
\end{itemize}

And there are other features and labels that will be mentioned below as needed.

\subsection{Performance Analysis}
The analysis of selected scenario specifications is summarized in table~\ref{table:performance_analysis}. The first notorious observation in this table is the effect of extrinsic priorities to decrease inefficiency. With regards to safety indicators, their effect is mixed, and we have to observe each case separately. In the case of DAIDALUS, priorities decreased the 2 kft LoS rate and, most drastically, the timeout rate, while increased the 4 kft LoS rate. In the case of ACAS sXu, priorities increased both LoS rate indicators, but decreased the timeout rate drastically. Based on our rule of thumb assessment, it can be said that DAIDALUS works better with extrinsic priorities, while ACAS sXu works better without them. We conjecture that the following reasons explain this fact: i) that DAIDALUS has more built-in symmetries than ACAS sXu; ii) that ACAS sXu has already built-in priority rules for multi-aircraft encounters, and extrinsic priorities may contradict with them.

\begin{table}[h]
    \caption{Summary of closed-loop performance indicators per scenario.}
    \label{table:performance_analysis}
    \begin{center}
    \begin{tabularx}{\columnwidth}{|p{0.20\columnwidth}|p{0.11\columnwidth}|p{0.09\columnwidth}|p{0.09\columnwidth}|p{0.1\columnwidth}|p{0.105\columnwidth}|}
        \hline
    Scenario spec. & Ineffici-ency rate & LoS rate 4~kft & LoS rate 2~kft & Timeout rate & Scenario comp. time (s)\\
    \hline
    \texttt{dai\_ip\_2d\_4k} & 9.71\% & 1.4E-2 & 6.5E-5 & 1.6E-2 & 6.5E-2\\
    \texttt{dai\_ep\_2d\_4k} & 4.83\% & 2.4E-2 & 4.1E-5 & 0 & 5.1E-2\\
    \texttt{sxu\_ip\_2d\_2k} & 20.7\% & 8.7E-1 & 4.1E-2 & 1.6E-3 & 7.8E+1\\
    \texttt{sxu\_ep\_2d\_2k} & 10.9\% & 9.0E-1 & 2.0E-1 & 1.8E-5 & 7.8E+1\\
    \texttt{dai\_ep\_3d\_4k} & 4.38\% & 8.9E-6 & 0 & 0 & 7.4E-2\\
    \texttt{dai\_ep\_2d\_2k} & 1.3\% & 9.0E-1 & 1.1E-1 & 0 & 3.9E-2\\
    \hline
    \end{tabularx}
    \end{center}
\end{table}

According to a line of reasoning, it would be expected, that, in the more efficient scenarios, the aircraft fly closer to each other and, therefore, there should be a higher probability of losing separation. But this is not the only principle at play, because, if the aircraft perform deviations with the least extra distance, while keeping separation, they stay less in the air and decrease the total number of conflicts. This becomes more understandable when we compare \texttt{dai\_ep\_2d\_4k} with \texttt{dai\_ep\_3d\_4k}, where the latter achieved a small advantage in efficiency, but a huge one in safety. \texttt{dai\_ep\_3d\_4k} is capable of shortening the total distances, but has a residual cost associated to vertical maneuvers, where the climb maneuvers spend fuel at higher rates. 

It cannot escape from observation that DAIDALUS performed much better than ACAS sXu in almost all indicators. In our opinion, it would be reasonable to expect that ACAS sXu would not excel in the LoS rates, especially that of 4 kft, because its first protection criterion is 2 kft, as a built-in feature. However, with smaller protection volumes, the deviations should be smaller and, by this reasoning, its expected inefficiency would be lower than that of DAIDALUS. But our results show otherwise when we compare the cases of DAIDALUS with those of ACAS sXu. The only case in which ACAS sXu obtained an advantage was for the LoS rates comparison between \texttt{sxu\_ip\_2d\_2k} and \texttt{dai\_ep\_2d\_2k}, which have the same separation target. In any case, the Los rate obtained for ACAS sXu meets the performance requirement of ASTM F3442 \cite{ASTM}, which uses the definition of LoWC Ratio (LR), which is the ratio between the LoS rate of 2 kft shown in table~\ref{table:performance_analysis}, with DAA active, and corresponding LoS rate with DAA inactive, the latter being 0.785 according  to our simulations. Thus, the resulting LR scores for ACAS sXu here are 0.052 and 0.253, respectively for the two ACAS sXu specs, which are well below the value of 0.4 from \cite{ASTM}, and consistent with the performance analysis of \cite{DO_396}. We conjecture that these scores would be lower in a future 3-D scenario spec of a ACAS sXu, by following the same improvement obtained with DAIDALUS. 

\subsection{Possible approximations to the closed-loop behavior}
Trying to alleviate the heavy computational load to simulate large numbers of different traffic configurations, in this multi-aircraft, closed-loop setup, we considered some approximated solutions, such as the use of Deep Neural Networks to emulate the ACAS Xu/sXu behavior, in the lines followed by \cite{Julian2018,Bak2022}. However, the existing solutions that we found available were developed for just one intruder aircraft, so they were not suitable for our study. Another possibility would be the exploitation of symmetry transformations \cite{Sibai2020}, however the history-dependent nature of the ACAS Xu/sXu algorithms, associated to the present closed-loop setup, make this possibility unpractical. Thus, we started exploring simpler ways of deducing closed-loop behavior without having to perform the full simulation of a scenario. So far, we tried to analyze correlations between measures of open-loop maneuvers and the closed loop performance. The features that we explored are:
\begin{itemize}
    \item \textbf{Distance flown until the end of the first deviation maneuver} ($\overline{M/D}$): we consider the total distance flown until a ``Clear-of-Conflict'' (CoC) event happens, that is, after one or more divergent maneuvers start in a scenario instance, we stop the scenario when the first divergent maneuver of any aircraft finishes and that individual aircraft is clear of conflict, the moment from which some decision must be made on how to continue the mission. We count the total number of maneuvers started, and divide it by the sum of the flown distances, across all scenario instances in an execution set associated to a scenario spec.
    \item \textbf{Average angle deviation maneuver} ($\overline{\alpha}$): using the same stopping rule of above, we account the angle difference between the heading angles of the aircraft at the beginning of the divergent maneuver and at the stopping moment. 
\end{itemize}
For each of these measures, we ran the 122,416 traffic configuration instances with the open-loop stopping rule. Here, all the scenario specifications are 2-dimensional, and we use abbreviated lables to achieve a better display in the graph legends. Namely, the scenario specifications in this subsection are defined as:
\begin{itemize}
    \item \texttt{D1}: DAIDALUS without extrinsic priorities and with deterministic sensor data;
    \item \texttt{D2}: DAIDALUS with extrinsic priorities and deterministic sensor data;
    \item \texttt{D3}: DAIDALUS with extrinsic priorities and Sensor Uncertainty Mitigation (SUM). This is a design feature \cite{Narkawicz2018} to mitigate uncertainty in sensor data, as for example, to determine the position of an intruder aircraft;
    \item \texttt{D4}: DAIDALUS with its Horizontal Miss Distance (HMD) set to 200 ft (the standard is 4,000 ft) and uncertain sensor data;
    \item \texttt{X1}: ACAS sXu without extrinsic priorities;
    \item \texttt{X2}: ACAS sXu with extrinsic priorities;
    \item \texttt{X3}: ACAS sXu with scenario downscaled to speed of 43 knots and cell radius of 1 km.
\end{itemize}
We used the values of $\overline{M/D}$ and $\overline{\alpha}$ obtained for each of the specs above as inputs to a linear regressor of inefficiency, as defined in section~\ref{subsection:performance_indicators_definition}, and generated a plot with the pairs of (true, predicted) values from this regression, as shown in fig.~\ref{fig:inefficiency_regression}. The trend line in the figure, which depicts the regressor, seems to represent a strong correlation, which is confirmed by the value of $R^2$. When considering each of the regression inputs separately, we obtain $R^2=0.73$ for $\overline{M/D}$ and $R^2=0.77$ for $\overline{\alpha}$, which show that they contribute with approximately equal predicting power. 
% Figure environment removed

We performed a similar analysis for the LoS indicators, but we found very little correlation, as $R^2=0.11$ for the 2~kft LoS indicator. Nevertheless, it can be concluded that these open-loop measurements are a good proxy for the closed-loop inefficiency, with the advantage that the predictor discounts the bias that may have been introduced by the closed-loop mission management system, which is not part of the DAA specification. A rough estimate for the computing time saved is of 68\% in the inefficiency case.      



\section{CONCLUSIONS}
\section{Conclusion and Future Work}
In this work, I design corruption-robust algorithms for the Lipschitz contextual search problem. I present the \emph{agnostic checking} technique and demonstrate its effectiveness in designing corruption-robust algorithms. There are several open problems for future research. First, in the algorithm I propose for pricing loss, the schedule for agnostic checks is fixed upfront. Can the learner design an adaptive checking schedule for the pricing loss? Second, this work assumes the learner has knowledge of the Lipschitz constant $L$. Can the learner design efficient no-regret algorithms without knowledge of $L$? 

\addtolength{\textheight}{-12cm}   % This command serves to balance the column lengths
                                  % on the last page of the document manually. It shortens
                                  % the textheight of the last page by a suitable amount.
                                  % This command does not take effect until the next page
                                  % so it should come on the page before the last. Make
                                  % sure that you do not shorten the textheight too much.

%%%%%%%%%%%%%%%%%%%%%%%%%%%%%%%%%%%%%%%%%%%%%%%%%%%%%%%%%%%%%%%%%%%%%%%%%%%%%%%%



%%%%%%%%%%%%%%%%%%%%%%%%%%%%%%%%%%%%%%%%%%%%%%%%%%%%%%%%%%%%%%%%%%%%%%%%%%%%%%%%



%%%%%%%%%%%%%%%%%%%%%%%%%%%%%%%%%%%%%%%%%%%%%%%%%%%%%%%%%%%%%%%%%%%%%%%%%%%%%%%%

\section*{ACKNOWLEDGMENT}
We highly appreciate the support of the FAA TCAS Program Office, for authorizing and helping us to obtain the latest version of the ACAS sXu prototype package. Other invaluable contributions were provided by the members of the ACAS sXu development team, including Joshua Silbermann and Tyler Young, who guided and troubleshot the software installation, and Charles Leeper, who taught us basic and advanced features of the software package.

%%%%%%%%%%%%%%%%%%%%%%%%%%%%%%%%%%%%%%%%%%%%%%%%%%%%%%%%%%%%%%%%%%%%%%%%%%%%%%%%

\bibliographystyle{IEEEtran}
\bibliography{./closed_loop_DAA_2023.bib}




\end{document}
