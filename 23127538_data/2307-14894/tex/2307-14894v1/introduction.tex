The social and economic drives for Advanced Air Mobility (AAM) will soon lead to congested skies overhead cities. Thus, higher automation levels will become a requirement for coordinating this traffic, because of its higher complexity, which will not be effectively managed by humans. The establishment of fixed air routes can reduce complexity, however at the cost of limiting traffic capacity and decreasing flight efficiency. Another alternative is the use of a powerful central system equipped with Artificial Intelligence (AI), which would allow flexible trajectories and higher efficiency. However, such system could contain Single-Points-of-Failure (SPoFs), would be a highly sought target of malicious attacks, and would be subject to periods of unavailability. Furthermore, such system would have to impose certain constraints to the trajectories, in order to be computationally feasible, which occasionally may decrease flight efficiency. 

A simpler solution already exists to keep unmanned aircraft safely separated in the sky, called Detect-And-Avoid (or simpy DAA), which addresses the need of avoiding collisions with other aircraft and, in principle, to any detectable flying vehicle. This capability is so essential for future aviation that is already specified in several industry standards, such as DO-365B \cite{DO_365}, DO-386 \cite{DO_386}, DO-396 \cite{DO_396} and others. However, these standards focus on specifying the avoidance maneuver, which maintains separation, and are not concerned with what happens next, when the aircraft are deemed clear of conflict by the DAA algorithm. In other words, they ensure performance of divergent maneuvers, but do not cover the maneuvers needed to converge to a waypoint or terminal point.   

In a previous paper \cite{IROliveira2021}, we analyzed the performance of the DAIDALUS \cite{DaidalusPage} algorithm, which meets the standard DO-365B \cite{DO_365}, as the representative of a fully distributed traffic management strategy, comparing it to two other strategies, one of them fully centralized, and the other one being a distributed, but coordinated strategy. In the follow-on work of that paper, we wanted to explore other versions of DAA and, with that, confirm or disprove our findings, so this paper reports what has been accomplished towards this goal. The closest alternative to DAIDALUS is ACAS-Xu \cite{Manfredi2016,Owen2019}, which not only meets DO-365B \cite{DO_365}, but has a standard of its own, DO-386 \cite{DO_386}, because it includes the capability of processing raw track data from various sensors and doing its own probabilistic data fusion for surveillance, as opposed to \cite{DO_365}, which assumes a unified track/vector information. A systematic comparison between DAIDALUS and ACAS-Xu was already performed in \cite{Davies18}, which pointed to an overall agreement between both, with an expected superiority of ACAS Xu in dealing with noise in the input sensors. However, all the analyses in that reference are concerned only with the open-loop maneuvers, with just one intruder aircraft, and do not consider the resume of the flights towards a destination.

In the present paper, we advance some steps towards generalization of the performance analysis of DAA algorithms in a closed-loop context. We did not have an authorized ACAS Xu software package immediately available to use, but we did obtain authorization from the FAA TCAS Program Office to use the ACAS sXu API (herein version 4.1), which is a variant of ACAS Xu for small Unmanned Air Vehicles (UAVs), limited to 55 pounds of take-off weight, as specified in the standard DO-396 \cite{DO_396}. ACAS sXu inherits the same core algorithm of ACAS Xu, albeit with different look up tables, thus any comparison with DAIDALUS has to take into account their different scopes. Among other differences, the distance-based separation requirement that the aircraft controlled by ACAS sXu has to comply towards manned aircraft or large UAVs is, horizontally, 2,000 feet (in \cite{DO_396}, this distance is the threshold of a \emph{Loss of Well-Clear} event, or LoWC; and this distance is also prescribed in the standard ASTM F3442/F3442M-20 \cite{ASTM}). The analogous parameter in DO-365B has the value of 4,000 feet or 0.66 nmi (according to Table 2-24 of \cite{DO_365}, section ``HAZ'', row ``HMD''), and is part of the definition of \emph{DAA Well-Clear} or DWC. It may be possible to tweak their parameters to make them more comparable, but they have dozens of other parameters that are not directly translatable and, furthermore, ACAS sXu has large look-up tables. Thus, we kept most of their original parameters values intact and consider any comparison between them as just notional.   