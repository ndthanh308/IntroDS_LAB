\subsection{Performance Indicators}\label{subsection:performance_indicators_definition}
We use three categories of indicators to analyze the performance of a scenario definition:

\textbf{Inefficiency Rate}: it is the difference in fuel spent between a traffic scenario in which the aircraft comply to DAA resolution advisories (RA), and the analogous scenario in which each vehicle follows its optimal path, without observation of traffic separation rules. In case of vertical deviations, a higher fuel rate is required for climb maneuvers, a lower fuel rate in descent maneuver, which increase the net total for the mission. The resulting value of this indicator is the average value over all traffic configurations in a scenario set. 

\textbf{Loss of Separation (LoS) Rate}: despite there being several ways of defining traffic separation, we examine just the simplest one, which is checking whether or not the vehicles are separated by at least a fixed \emph{minimum distance}. In order to include both DAIDALUS and ACAS sXu in the same tables, we use one separation distance from each one, respectively: 4,000 ft, which is the Horizontal Miss Distance, or HMD, used in DO-365B \cite{DO_365} to define the so-called \emph{Hazard Alert Zone} (HAZ), associated to the DAA Well-Clear (DWC) concept of separation; and 2,000 ft, used in DO-396 \cite{DO_396}, that defines the Loss of Well-Clear (LoWC) event in relation to large UAVs or manned aircraft. These indicators will denote the rate of scenarios, in a scenario set, where the distance between any aircraft pair fell below the afore mentioned threshold values. 

\textbf{Timeout Rate}: this indicates, in a scenario set, the rate of scenarios where any aircraft exceeded a maximum time without reaching its destination point. As pointed out in section \ref{section:scenario_definitions}, this phenomenon occurs because DAA Resolution Advisories (RAs) cause long chains of maneuvers that extend beyond the energy/fuel allowance of the vehicle, due to shortcomings in coordination. We use a time threshold of 1,000 seconds but, in practice, the timeout would be determined by the energy/fuel capacity of the vehicle.

\textbf{Scenario Computing Time}: the time needed to simulate a single scenario instance, in a single core of an Intel Xeon CPU, discounted the fact that multiple scenario instances can be run in parallel in a multi-core CPU. In our simulated scenarios, the DAA algorithm is called at least each 2 seconds, for each aircraft, but when the aircraft is in avoidance mode, that can happen more often. In the case of ACAS sXu, the requirement of receiving various messages to update a single track contribute to result in multiple calls per simulated second.


\subsection{Scenario Specifications and Labels}
In this study, a scenario specification is defined by features such as: the DAA algorithm used, the dimensionality (2-D or 3-D), if it uses extrinsic priorities or not, the target separation parameter, and possibly other features. The scenario labels used in this section encode these attributes:
\begin{itemize}
\item \texttt{dai\_ip\_2d\_4k}: DAIDALUS without extrinsic priorities, 2-D maneuvering and regular 4 kft Horizontal Miss Distance (HMD);
\item \texttt{dai\_ep\_2d\_4k}: similar to the above, with extrinsic priorities;
\item \texttt{sxu\_ip\_2d\_2k}: ACAS sXu with intrinsic priorities only, 2-D maneuvering and regular 2 kft LoWC threshold;
\item \texttt{sxu\_ep\_2d\_2k}: similar to the above, with extrinsic priorities;
\item \texttt{dai\_ep\_3d\_4k}: similar to \texttt{dai\_ep\_2d\_4k}, with 3-D maneuvering;
\item \texttt{dai\_ep\_2d\_2k}: similar to \texttt{dai\_ep\_2d\_4k}, with HMD reduced to 2 kft.
\end{itemize}

And there are other features and labels that will be mentioned below as needed.

\subsection{Performance Analysis}
The analysis of selected scenario specifications is summarized in table~\ref{table:performance_analysis}. The first notorious observation in this table is the effect of extrinsic priorities to decrease inefficiency. With regards to safety indicators, their effect is mixed, and we have to observe each case separately. In the case of DAIDALUS, priorities decreased the 2 kft LoS rate and, most drastically, the timeout rate, while increased the 4 kft LoS rate. In the case of ACAS sXu, priorities increased both LoS rate indicators, but decreased the timeout rate drastically. Based on our rule of thumb assessment, it can be said that DAIDALUS works better with extrinsic priorities, while ACAS sXu works better without them. We conjecture that the following reasons explain this fact: i) that DAIDALUS has more built-in symmetries than ACAS sXu; ii) that ACAS sXu has already built-in priority rules for multi-aircraft encounters, and extrinsic priorities may contradict with them.

\begin{table}[h]
    \caption{Summary of closed-loop performance indicators per scenario.}
    \label{table:performance_analysis}
    \begin{center}
    \begin{tabularx}{\columnwidth}{|p{0.20\columnwidth}|p{0.11\columnwidth}|p{0.09\columnwidth}|p{0.09\columnwidth}|p{0.1\columnwidth}|p{0.105\columnwidth}|}
        \hline
    Scenario spec. & Ineffici-ency rate & LoS rate 4~kft & LoS rate 2~kft & Timeout rate & Scenario comp. time (s)\\
    \hline
    \texttt{dai\_ip\_2d\_4k} & 9.71\% & 1.4E-2 & 6.5E-5 & 1.6E-2 & 6.5E-2\\
    \texttt{dai\_ep\_2d\_4k} & 4.83\% & 2.4E-2 & 4.1E-5 & 0 & 5.1E-2\\
    \texttt{sxu\_ip\_2d\_2k} & 20.7\% & 8.7E-1 & 4.1E-2 & 1.6E-3 & 7.8E+1\\
    \texttt{sxu\_ep\_2d\_2k} & 10.9\% & 9.0E-1 & 2.0E-1 & 1.8E-5 & 7.8E+1\\
    \texttt{dai\_ep\_3d\_4k} & 4.38\% & 8.9E-6 & 0 & 0 & 7.4E-2\\
    \texttt{dai\_ep\_2d\_2k} & 1.3\% & 9.0E-1 & 1.1E-1 & 0 & 3.9E-2\\
    \hline
    \end{tabularx}
    \end{center}
\end{table}

According to a line of reasoning, it would be expected, that, in the more efficient scenarios, the aircraft fly closer to each other and, therefore, there should be a higher probability of losing separation. But this is not the only principle at play, because, if the aircraft perform deviations with the least extra distance, while keeping separation, they stay less in the air and decrease the total number of conflicts. This becomes more understandable when we compare \texttt{dai\_ep\_2d\_4k} with \texttt{dai\_ep\_3d\_4k}, where the latter achieved a small advantage in efficiency, but a huge one in safety. \texttt{dai\_ep\_3d\_4k} is capable of shortening the total distances, but has a residual cost associated to vertical maneuvers, where the climb maneuvers spend fuel at higher rates. 

It cannot escape from observation that DAIDALUS performed much better than ACAS sXu in almost all indicators. In our opinion, it would be reasonable to expect that ACAS sXu would not excel in the LoS rates, especially that of 4 kft, because its first protection criterion is 2 kft, as a built-in feature. However, with smaller protection volumes, the deviations should be smaller and, by this reasoning, its expected inefficiency would be lower than that of DAIDALUS. But our results show otherwise when we compare the cases of DAIDALUS with those of ACAS sXu. The only case in which ACAS sXu obtained an advantage was for the LoS rates comparison between \texttt{sxu\_ip\_2d\_2k} and \texttt{dai\_ep\_2d\_2k}, which have the same separation target. In any case, the Los rate obtained for ACAS sXu meets the performance requirement of ASTM F3442 \cite{ASTM}, which uses the definition of LoWC Ratio (LR), which is the ratio between the LoS rate of 2 kft shown in table~\ref{table:performance_analysis}, with DAA active, and corresponding LoS rate with DAA inactive, the latter being 0.785 according  to our simulations. Thus, the resulting LR scores for ACAS sXu here are 0.052 and 0.253, respectively for the two ACAS sXu specs, which are well below the value of 0.4 from \cite{ASTM}, and consistent with the performance analysis of \cite{DO_396}. We conjecture that these scores would be lower in a future 3-D scenario spec of a ACAS sXu, by following the same improvement obtained with DAIDALUS. 

\subsection{Possible approximations to the closed-loop behavior}
Trying to alleviate the heavy computational load to simulate large numbers of different traffic configurations, in this multi-aircraft, closed-loop setup, we considered some approximated solutions, such as the use of Deep Neural Networks to emulate the ACAS Xu/sXu behavior, in the lines followed by \cite{Julian2018,Bak2022}. However, the existing solutions that we found available were developed for just one intruder aircraft, so they were not suitable for our study. Another possibility would be the exploitation of symmetry transformations \cite{Sibai2020}, however the history-dependent nature of the ACAS Xu/sXu algorithms, associated to the present closed-loop setup, make this possibility unpractical. Thus, we started exploring simpler ways of deducing closed-loop behavior without having to perform the full simulation of a scenario. So far, we tried to analyze correlations between measures of open-loop maneuvers and the closed loop performance. The features that we explored are:
\begin{itemize}
    \item \textbf{Distance flown until the end of the first deviation maneuver} ($\overline{M/D}$): we consider the total distance flown until a ``Clear-of-Conflict'' (CoC) event happens, that is, after one or more divergent maneuvers start in a scenario instance, we stop the scenario when the first divergent maneuver of any aircraft finishes and that individual aircraft is clear of conflict, the moment from which some decision must be made on how to continue the mission. We count the total number of maneuvers started, and divide it by the sum of the flown distances, across all scenario instances in an execution set associated to a scenario spec.
    \item \textbf{Average angle deviation maneuver} ($\overline{\alpha}$): using the same stopping rule of above, we account the angle difference between the heading angles of the aircraft at the beginning of the divergent maneuver and at the stopping moment. 
\end{itemize}
For each of these measures, we ran the 122,416 traffic configuration instances with the open-loop stopping rule. Here, all the scenario specifications are 2-dimensional, and we use abbreviated lables to achieve a better display in the graph legends. Namely, the scenario specifications in this subsection are defined as:
\begin{itemize}
    \item \texttt{D1}: DAIDALUS without extrinsic priorities and with deterministic sensor data;
    \item \texttt{D2}: DAIDALUS with extrinsic priorities and deterministic sensor data;
    \item \texttt{D3}: DAIDALUS with extrinsic priorities and Sensor Uncertainty Mitigation (SUM). This is a design feature \cite{Narkawicz2018} to mitigate uncertainty in sensor data, as for example, to determine the position of an intruder aircraft;
    \item \texttt{D4}: DAIDALUS with its Horizontal Miss Distance (HMD) set to 200 ft (the standard is 4,000 ft) and uncertain sensor data;
    \item \texttt{X1}: ACAS sXu without extrinsic priorities;
    \item \texttt{X2}: ACAS sXu with extrinsic priorities;
    \item \texttt{X3}: ACAS sXu with scenario downscaled to speed of 43 knots and cell radius of 1 km.
\end{itemize}
We used the values of $\overline{M/D}$ and $\overline{\alpha}$ obtained for each of the specs above as inputs to a linear regressor of inefficiency, as defined in section~\ref{subsection:performance_indicators_definition}, and generated a plot with the pairs of (true, predicted) values from this regression, as shown in fig.~\ref{fig:inefficiency_regression}. The trend line in the figure, which depicts the regressor, seems to represent a strong correlation, which is confirmed by the value of $R^2$. When considering each of the regression inputs separately, we obtain $R^2=0.73$ for $\overline{M/D}$ and $R^2=0.77$ for $\overline{\alpha}$, which show that they contribute with approximately equal predicting power. 
% Figure environment removed

We performed a similar analysis for the LoS indicators, but we found very little correlation, as $R^2=0.11$ for the 2~kft LoS indicator. Nevertheless, it can be concluded that these open-loop measurements are a good proxy for the closed-loop inefficiency, with the advantage that the predictor discounts the bias that may have been introduced by the closed-loop mission management system, which is not part of the DAA specification. A rough estimate for the computing time saved is of 68\% in the inefficiency case.      
