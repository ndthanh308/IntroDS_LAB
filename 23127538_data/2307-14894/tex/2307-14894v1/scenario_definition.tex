The multi-vehicle scenarios consist of an airspace of approximately 20 km of diameter, with 4 aircraft and definitions similar to \cite{IROliveira2021}, 
but here, one of the cases allows 3-D maneuvering. In this section, we present a summary of these common scenario definitions. The airspace was initially designed
with a cellular structure, in order to facilitate coordination among the agents, as shown in  Fig.~\ref{fig:cellular_airspace}.

% Figure environment removed

In this paper's analyses, the cells are not needed for coordination, however that structure is still needed to generate the large number of different traffic configurations, because the origin and destination points of an air vehicle's mission are placed in the outer cells (7-18). The traffic configurations are generated by a factorial algorithm with the following constraints:
\begin{enumerate}
	\item \label{itm:aircraft_id} An aircraft is uniquely identified by a numerical id between 0 and 3;
	\item An aircraft's mission is defined by an Origin and a Destination point, both being cell centroids;
	\item Origin points are all distinct among the aircraft, so as that they begin the scenario properly separated;
	\item \label{itm:o-d_distance} The Destination point must be distant from the Origin point by at least three cells in-between them;
\end{enumerate}
Constraint \#\ref{itm:aircraft_id} allows to define priority rules among the aircraft, and contributes to generate a higher number of distinct traffic configurations. Constraint \#\ref{itm:o-d_distance} is thought to exclude trajectories that are too short and, in a certain way, to promote convergence and crossing of the trajectories and obtain more conflicts. Implicitly, it forces that the origin and destination points be at the outer ring of the airspace. Then, by generating all possible combinations allowed by these constraints, one obtains 122,416 distinct traffic configurations. Still, it can be observed that some of these configurations are rotated versions of others, so, if the cell numbers did not matter, some of these configurations could be eliminated and we would remain with 46,660 traffic configurations (the full set does not contain 6 rotated replica subsets because constraint \#\ref{itm:o-d_distance} creates asymmetries). Anyway, we still chose to keep the full set because some conflict resolution algorithms use the reference to the north (and consequently, to the cell numbers) to decide tie-break situations, and even because some of them use look up tables or decision rules in which the absolute heading of the aircraft is an input. In some of our experiments, it was observed that ACAS sXu has slightly different resolution to traffic configurations that are rotated versions, besides occurrence of special events with DAIDALUS happening for just one scenario among the total 122,416, confirming that there is no strong rotational symmetry in the DAA implementations. 

An aircraft can go out of the cell area to execute a deviation maneuver, and we observed such occurrences with a small frequency, depending on the scenario specification. In reality, there might be borders with forbidden airspaces or obstacles, and those would be more significant safety threats. In any case, the aircraft tend to remain inside the airspace because they have to reach their destination through an efficient path. For the scenario specifications with only vertical maneuvers, the aircraft remain symbolically at 500 ft of altitude, while in the spec with vertical maneuvers, the aircraft have their origin and destination points at this altitude, and there will remain if no vertical maneuver happen. Vertical maneuvers are allowed only above this altitude, without any bound up, other than the requirement to return to the base altitude in the end of the flight mission. 

\subsection{Use of priorities}

Priorities can be used to determine a collective decision on how each aircraft will alter its trajectory in case of a traffic encounter of two or more aircraft \cite{ROMANIDEOLIVEIRA201792}. DAA attributes priorities to each aircraft in an encounter according to built-in, safety-driven rules. Although DO-365B \cite{DO_365} does not establish any decision rule based on priority, it requires a high priority flag indication for the remote pilot in case of an intruder is in emergency state. On the other hand, both DAIDALUS \cite{DaidalusPage} and DO-386 \cite{DO_386} define, respectively, priority rules for coordinated encounters, where priorities are defined according to observed aircraft equipage and the  aircraft identification code (the so-called ICAO address). The intention is to use objective rules to decrease the chances of aircraft performing maneuvers that worsen the conflict, as for example, turning onto each other. We distinguish these built-in priorities from the priorities explored in \cite{ROMANIDEOLIVEIRA201792}, by calling the latter as \emph{extrinsic} priorities, because they can be arbitrarily chosen among equal vehicles. And it is necessary bearing in mind that the DAA built-in priorities are used only for the divergent maneuvers, not for the return-to-mission phase. 

Considering the case where aircraft are equally equipped, DAA maneuver tiebreaking is performed by aircraft id, and is fair only if the probability distribution of aircraft ids per encounter follow a symmetric curve, which would hardly be the case in practice. Apart from tiebreaking, there is a general understanding in the DAA regulations that priorities must be observed when encounters occur with public service and medevac missions, vehicles in various sorts of emergency, etc. However, the afore mentioned standards do not need to define specific rules for encounters with such intruders, because they assume that such intruders will be treated as non-cooperative targets, either by proactive identification by the remote pilot, or by interaction with Air Traffic Control, or, as the last means, by the repeated refusal of cooperation by the intruder. 

It was observed in our previous works \cite{ROMANIDEOLIVEIRA201792,IROliveira2021} that, for encounters involving multiple (more than two) equally-righted air vehicles in dense airspaces, the use of extrinsic priorities improves the efficiency of the collective traffic system, because it decreases the total number of deviations and gives more predictability to the scenarios. In the case of DAA, they can be used to make DAA to suppress any alert associated to a cooperative intruder with lower priority than the ownship's. While this greatly improves efficiency of the scenarios, it also may increase the occurrence of losses of separation, and this trade-off has to be taken into account. Such trade-off also includes the occurrence of timeout events, already explored in our previous works. 

A timeout is the forced termination of a long chain of maneuvers without all the aircraft having accomplished their missions. In practice, this would be most likely associated with a fuel or energy emergency, when the aircraft's energy supply becomes critically low. Thus, at least one of the aircraft has to start a procedure of emergency landing, which is an undesirable event, both from the efficiency and the safety perspectives. The long chain of maneuvers is caused by the inability of the DAA logic to plan a coordinated return to mission for all the aircraft involved in an encounter, thus each aircraft decides independently on how to resume its intended path, and this lack of coordination causes maneuver repetitions, extensions, alternating cycles, and other bizarre sequences of maneuvers, that could also be called \emph{livelocks}. 

In some of our scenario specifications, we used the aircraft id as the sole criterion for extrinsic priorities. This choice is technically correct, however, in practice, it is not  an equitable policy, because the id remains unchanged for the lifetime of the vehicle, and that either blesses or condemns the vehicle with a persistent tendency to being promoted or being penalized. Thus, a randomized priority assignment would be needed, but that would have to be cheat-proof and fault-proof \cite{IROliveira2022}, which would bring some extra cost to the system.  

\subsection{Vertical maneuvering}

Most of the scenarios that we studied have only horizontal maneuvers, with all aircraft flying at the same altitude of 500 ft. This might happen in practice near airports, where the higher altitudes are reserved for manned aviation, and lower altitudes are not allowed due to ground obstacles or noise regulations. In any case, 2-D aircraft dynamics and mission control is much simpler and greatly helps in grasping the critical aspects of distributed Conflict Detection and Resolution (CD\&R). Later in this research, we developed an initial 3-D simulation model in which, if DAA presents avoidance bands in the vertical dimension, only the up sense is taken, a decision depending on a priority order for choosing which of the aircraft will maneuver or not. This was a simplification to treat the indeterminacy left by DO-365B and DAIDALUS. DO-365B associates coordinated encounters to certain classes of equipage, among which the Class 1 does not allow such coordination. 

In DAIDALUS, there is no explicit coordination among the aircraft and, because of that, the ownship ``pilot" can choose either direction and end up choosing the same direction of the intruder ``pilot". Thus, its resolution logic will be complete only with additional assumptions. It is easy to see that, in the vertical plane, if both aircraft maneuver up for conflict avoidance, there will be no avoidance. In the horizontal plane, where the aircraft dynamics has rotational symmetry, it is desirable that aircraft turn to the same relative direction, however sensor uncertainty and multi-aircraft encounters remain problematic without coordination. In our implementation of 3-D maneuvering, if DAIDALUS is prescribing an upwards vertical speed, that becomes a control input simultaneously to any horizontal deviation input, thus the resulting maneuver can happen in both planes at the same time. Once DAIDALUS signals that the aircraft is clear of conflict, the mission control logic will try to return to the destination altitude, provided that it does not create a new conflict immediately. With regards to selection of upward or downward resolution maneuver, the use of id-based priorities allowed our implementation to disregard downward advisories with acceptable results and solve the vertical indeterminacy in a simplified way.  For an in-depth study on the interoperability between horizontal and vertical advisories, one can refer to \cite{Londner2015}. 
