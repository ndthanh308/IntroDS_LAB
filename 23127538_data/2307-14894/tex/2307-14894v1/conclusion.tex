In this paper, we presented an extension of our work on the analysis of closed-loop air traffic management algorithms, aimed at fully autonomous applications. This extension focused solely on DAA algorithms, using two implementations fulfilling industry standards, namely DAIDALUS and ACAS sXu. We analyzed the effect of using extrinsic (or arbitrarily decided) priorities as a means of coordination, and found that they cause a drastic decrease in inefficiency and timeout rates, for all algorithms, with a positive effect also for the rate of Loss of Separation (LoS) in the case of DAIDALUS with the threshold of 2,000 ft prescribed by the standard ASTM F3442 \cite{ASTM}, while in the case of ACAS sXu it had a negative effect on LoS rate. Our results show that ACAS sXu has higher inefficiency and LoS rates than DAIDALUS in physically similar scenarios, apart from considerations on noise and errors in sensor data. This higher inefficiency of ACAS sXu is confirmed by prediction based on open-loop scenario data, in which the aircraft perform maneuvers based solely on DAA advisories. The use of open-loop indicators for predicting inefficiency showed to be promising, with a high degree of correlation registered for a small number of scenario specifications, while the prediction of LoS rates based on open-loop simulation turned out to have little correlation. The use of only open-loop data allows to save 68\% of simulation time. 

A final confirmation remains to be done in this phase of the research, which is to integrate the vertical maneuvers of ACAS sXu with our closed-loop mission control logic, in order to obtain the performance of the complete ACAS sXu logic in 3-D. Because this will certainly increase the already massive computational effort for the many traffic configurations, we also need to find an unbiased sampling rule to decrease the number of scenario instances, or perhaps develop a Deep Neural Network model of ACAS Xu/sXu for 3 or more intruders. Besides that, the difficulties that we found in achieving an efficient decentralized strategy for traffic management subject to DAA advisories raised the need of finding more powerful methods to solve this problem, and the method that seems more promising, based on a preliminary literature search, is Multi-Agent Reinforcement Learning (MARL), which we intend to explore in the next steps. 