\documentclass[10pt,conference]{IEEEtran}
\IEEEoverridecommandlockouts
% The preceding line is only needed to identify funding in the first footnote. If that is unneeded, please comment it out.
%\documentclass[sigconf,review,anonymous]{acmart}
%\acmConference[ASE 2022]{The37th IEEE/ACM International Conference on Automated Software Engineering}{Sept. 26 -- Oct. 1, 2022}{Ann Arbor, MI, USA}
\usepackage{cite}
\usepackage{amsmath,amssymb,amsfonts}
\usepackage{amsmath,amsfonts}
\usepackage{algorithmic}
\usepackage{graphicx}
\usepackage{textcomp}
\usepackage{rotating}
\usepackage{array}
\usepackage{xcolor}
\usepackage{tikz}
\usepackage{makecell}
\usepackage{multirow}
\usepackage[T1]{fontenc}
\usepackage{comment}
%\usepackage{floatrow}
\usepackage{etoolbox}
\usepackage{enumitem}
\def\BibTeX{{\rm B\kern-.05em{\sc i\kern-.025em b}\kern-.08em
    T\kern-.1667em\lower.7ex\hbox{E}\kern-.125emX}}

\begin{document}

\newcommand*\circled[1]{\tikz[baseline=(char.base)]{
		\node[shape=circle,draw,inner sep=0.8pt] (char) {#1};}}
	
%\setcopyright{acmcopyright}

\title{Statement-based Memory for Neural Source Code Summarization}


\author{\IEEEauthorblockN{Aakash Bansal, Siyuan Jiang, Sakib Haque, and Collin McMillan}
\IEEEauthorblockA{\textit{Dept. of Computer Science and Engineering} \\
	\textit{University of Notre Dame}\\
	Notre Dame, IN, USA \\
	\{abansal1, sjiang1, shaque, cmc\}@nd.edu
}}

%\author{\IEEEauthorblockN{\emph{Author list redacted for blind review.}}
%	\IEEEauthorblockA{\textit{~} \\
%		\textit{~}\\
%		~ \\
%		~
%}}

%\author{\emph{Author list redacted for blind review.}
%\author{Author One, Author Two, Author Three, and Author Four}
%\email{{one, two, three, four}@anonymous}
%\affiliation{%
%	\institution{Anonymous\\Anonymous}
%	\city{Anonymous}
%	\state{Anon}
%	\country{Anon}
%	\postcode{00000}
%}


\maketitle

\begin{abstract}
Source code summarization is the task of writing natural language descriptions of source code behavior.  Code summarization underpins software documentation for programmers.  Short descriptions of code help programmers understand the program quickly without having to read the code itself.  Lately, neural source code summarization has emerged as the frontier of research into automated code summarization techniques.  By far the most popular targets for summarization are program subroutines.  The idea, in a nutshell, is to train an encoder-decoder neural architecture using large sets of examples of subroutines extracted from code repositories.  The encoder represents the code and the decoder represents the summary.  However, most current approaches attempt to treat the subroutine as a single unit.  For example, by taking the entire subroutine as input to a Transformer or RNN-based encoder.  But code behavior tends to depend on the flow from statement to statement.  Normally dynamic analysis may shed light on this flow, but dynamic analysis on hundreds of thousands of examples in large datasets is not practical.  In this paper, we present a statement-based memory encoder that learns the important elements of flow during training, leading to a statement-based subroutine representation without the need for dynamic analysis.  We implement our encoder for code summarization and demonstrate a significant improvement over the state-of-the-art.
\end{abstract}

\begin{IEEEkeywords}
neural networks, neural models of source code, dynamic memory networks
\end{IEEEkeywords}

\input intro

\input background

\input approach

\input dataset

\input experiment

\input results

\input discussion

\input conclusion
\vspace{-0.25cm}
\section{Reproducibility}
\label{sec:repo}
To encourage reproducibility and future work by other researchers, we provide all our code, datasets, predictions, and instructions for replication using an online repository:

\emph{github.com/aakashba/smncode2022}

\section*{Acknowledgment}
This work is supported in part by NSF CCF-2100035 and CCF-2211428. Any opinions, findings, and conclusions expressed herein are the authors and do not necessarily reflect those of the sponsors.

\bibliographystyle{IEEEtran}
\bibliography{main}

\end{document}
