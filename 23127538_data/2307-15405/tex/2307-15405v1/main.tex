\documentclass[aps,prl,twocolumn,superscriptaddress,amsfont,graphicx,nofootinbib,preprintnumbers]{revtex4}
\bibliographystyle{apsrev}

\usepackage{amssymb}
\usepackage{amsmath}
\usepackage{graphicx}
\usepackage{subcaption}
\usepackage{float}
\captionsetup{subrefformat=parens}
\usepackage[colorlinks=true]{hyperref}
\usepackage{appendix}
\usepackage{amsbsy}
\usepackage{slashed}
\usepackage{tikz}
\usepackage{tikz-feynman}
\DeclareMathAlphabet\mathbfcal{OMS}{cmsy}{b}{n}
\def\e{\epsilon}
\newcommand{\code}[1]{\textsc{#1}}

\DeclareMathOperator{\Tr}{Tr}


\begin{document}
\def\UZH{Physik-Institut, Universit\"{a}t Zurich, 
            Winterthurerstrasse 190,
           	CH-8057 Z\"{u}rich,
            Switzerland}
\def\TUM{Technical University of Munich, TUM School of Natural Sciences, Physics Department, James-Franck-Straße 1, 85748 Garching, Germany}
\preprint{ZU-TH 38/23, TUM-HEP 1467/23}

\title{\boldmath Planar three-loop QCD helicity amplitudes for $V$+jet production at hadron colliders}

\author{Thomas Gehrmann}            
\email[Electronic address: ]{thomas.gehrmann@uzh.ch}
\affiliation{\UZH} 

\author{Petr Jakub\v{c}\'{i}k}            
\email[Electronic address: ] {petr.jakubcik@physik.uzh.ch}
\affiliation{\UZH}

\author{Cesare Carlo Mella}            
\email[Electronic address: ]{cesarecarlo.mella@tum.de}
\affiliation{\TUM}

\author{Nikolaos Syrrakos}            
\email[Electronic address: ]{nikolaos.syrrakos@tum.de}
\affiliation{\TUM}

\author{Lorenzo Tancredi}            
\email[Electronic address: ]{lorenzo.tancredi@tum.de}
\affiliation{\TUM}


\begin{abstract}
We compute the planar three-loop Quantum Chromodynamics (QCD) corrections to the helicity amplitudes involving a 
vector boson $V=Z,W^\pm,\gamma^*$, two quarks and a gluon. These amplitudes are 
relevant to vector-boson-plus-jet production at hadron 
colliders and other precision QCD observables. The 
planar corrections encompass the leading colour factors $N^3$, $N^2 N_f$, $N N_f^2$ and $N_f^3$.
 We provide the finite remainders of the independent helicity amplitudes in terms of multiple polylogrithms, continued to all kinematic regions and in a form which is compact and lends itself to efficient numerical evaluation.
\end{abstract}

\maketitle

The problem of the presence or absence of phase transition is central in statistical mechanics. To prove the existence of phase transition, the standard idea is to define a notion of contour and use \textit{Peierls' argument} \cite{Peierls.1936}. In the usual Ising model \cite{Ising_25}, particles of the system interact only with their nearest-neighbors. On ferromagnetic long-range Ising models \cite{Anderson_Yuval_69}, there is interaction between each pair of spins in the lattice. The Hamiltonian of the model is given formally by
\begin{equation*}
    H(\sigma) = - \sum_{x,y\in \Z^d}J_{xy}\sigma_x\sigma_y,
\end{equation*}
where $J_{xy}=J|x-y|^{-\alpha}$, $J>0$, $\alpha > d$. It is well-known that the Peierls' argument in dimension 2 implies phase transition for Ising models with nearest-neighbors or long-range interactions when $d\geq 2$, using correlation inequalities. For the unidimensional lattice, it was known that short-range models do not present phase transition. In the long-range case, a different behavior was expected depending on the exponent $\alpha$ (see \cite{Kac_Thompson_69}), but the problem was challenging since contours were first created as multidimensional objects.

In dimension $d=1$, phase transition was proved first in 1969 by Dyson \cite{Dyson.69}, for $\alpha \in (1,2)$, by proving phase transition in an auxiliary model and then using correlation inequalities. In 1982, Fr{\"o}hlich and Spencer \cite{Frohlich.Spencer.82} introduced a notion of one-dimensional contours and then applied the Peierls' argument to show phase transition for the critical value $\alpha = 2$. These contours were inspired by the multiscale techniques previously introduced to study the Berezinskii-Kosterlitz-Thouless transition in two-dimensional continuous spin systems \cite{FS81}. Later, Cassandro, Ferrari, Merola and Presutti  \cite{Cassandro.05} extended the contour argument previously available for $\alpha=2$ to exponents $\alpha\in (3-\frac{\ln 3}{\ln 2}, 2)$, with the additional restriction that the nearest-neighbor interaction is strong, i.e.,  ${J(1)\gg 1}$; this restriction was removed for a subclass of interactions in \cite{Bissacot.Endo.18}. Further results were obtained using contour arguments, such as the decay of correlations, cluster expansions, phase transition with random interactions, etc; some references with these results are \cite{ Cassandro.Merola.Picco.17, Cassandro.Merola.Picco.Rozikov.14, Imbrie.82, Imbrie.Newman.88, Johansson.91}. 

In the multidimensional setting ($d\geq 2$), Ginibre, Grossmann, and Ruelle, in \cite{Ginibre.Grossmann.Ruelle.66}, proved the phase transition for $\alpha > d+1$, using an enhanced version of Peierls' argument and the usual contours. Park proposed a different notion of contour for long-range systems in \cite{Park.88.I, Park.88.II}, extending the Pirogov-Sinai theory available for short-range interactions assuming $\alpha > 3d+1$, although he can also consider Potts models with his methods. Some results in the literature suggest that truly long-range effects appear only when $d < \alpha \leq d+1$, see for instance, \cite{Biskup_Chayes_Kivelson_07}. Recently, Affonso, Bissacot, Endo and Handa \cite{Affonso.2021}, inspired by the ideas from Fr{\"o}hlich and Spencer in \cite{FS81, Frohlich.Spencer.82}, introduced a version of multiscale multidimensional contour and proved phase transition by a contour argument in the whole region $\alpha > d$. They can consider long-range Ising models with deterministic decaying fields, first introduced in the context of nearest-neighbor interactions in \cite{Bissacot_Cioletti_10}. For these models, the lack of analyticity of the free energy does not imply phase transition since these models have the same free energy as the models with zero field. It is expected that fields decaying slowly imply uniqueness. In this setting, a contour argument is useful for proofs of phase transitions as well for uniqueness, some papers with models with deterministic decaying fields are \cite{Aoun_Ott_Velenik_23, Bissacot_Cass_Cio_Pres_15, Bissacot.Endo.18, Cioletti_Vila_2016}.

The Random Field Ising model (RFIM) \cite{Imry.Ma.75} is the nearest-neighbor Ising model with an additional external field acting on each site $(h_x)_{x\in\Z^d}$ that is a family of i.i.d. Gaussian random variable with mean 0 and variance 1. Formally, the Hamiltonian of the model is given by
\begin{equation*}
    H(\sigma) = - \sum_{\substack{x,y\in \Z^d \\|x-y|=1}}J\sigma_x\sigma_y  - \varepsilon\sum_{x\in\Z^d}h_x\sigma_x,
\end{equation*}
where $J>0$, $\varepsilon>0$, $\alpha > d$ and $d \geq 1$. A detailed account of the history of the phase transition problem for this model, as well as detailed proofs, was given in \cite{Bovier.06}. Here we present a brief overview.

During the 1980s, the question of the specific dimension where phase transition for the RFIM should happen attracted much attention and was a topic of heated debate. Two convincing arguments were dividing the physics community. One of them, due to Imry and Ma \cite{Imry.Ma.75}, was a non-rigorous application of the Peierls' argument together with the use of the isoperimetric inequality. The key idea of Peierls' argument is to define a notion of contour and calculate the energy cost of "erasing" each contour, i.e., the energy cost of flipping all spins inside the contour. When there is no external field, that energy necessary to flip the spins in a region $A\subset \Z^d$ is of the order of the boundary $|\partial A|$. When we add an external field, we get an extra cost depending on this field. Imry and Ma argued that this cost should be approximately $\sqrt{|A|}$, which is smaller than $|\partial A|$ for all regions only when $d\geq 3$, so this should be the region where phase transition occurs. The other argument, due to Parisi and Sourlas \cite{Parisi.Sourlas.79}, based on dimensional reduction, predicted that the $d$-dimensional RFIM would behave like the $d-2$-dimensional nearest-neighbor Ising model, therefore presenting phase transition only when $d\geq 4$. 

The question was settled by two celebrated papers showing that Imry and Ma's prediction was correct. First, in 1988, Bricmont and Kupiainen \cite{Bricmont.Kupiainen.88} showed that there is phase transition almost surely in $d\geq3$, for low temperatures and variance $\varepsilon$ small enough. Their proof uses a rigorous renormalization group analysis for the short-range case and it is considered involved. Still, they claimed that the result works for any model with a suitable contour representation and centered sub-gaussian external field. Later on, Aizenman and Wehr \cite{Aizenman.Wehr.90} proved uniqueness for $d\leq 2$. For detailed proofs of these results, we refer the reader to \cite{Bovier.06} (see also \cite{Berretti.85, Camia.18, Frohlich.Imbre.84,  Klein.Masooman.97} for more uniqueness results). 

Recently, Ding and Zhuang, see \cite{Ding2021}, provided a simpler proof of the phase transition, not using RGM. And in  \cite{Ding.Liu.Xia.22}, Ding, Liu and Xia proved that if $\beta_c(d)$ is the critical inverse of the temperature of the Ising model with no field, for all $\beta>\beta_c(d)$ there exists a critical value $\varepsilon_0(d, \beta)$ such that the RFIM with $\varepsilon \leq \varepsilon_0$ presents phase transition. 

In the present paper, we are considering a long-range Ising model with a random field, whose Hamiltonian is given formally by
\begin{equation*}
    H(\sigma) = - \sum_{x,y\in \Z^d}J_{xy}\sigma_x\sigma_y - \varepsilon\sum_{x\in\Z^d}h_x\sigma_x,
\end{equation*}
where $J_{xy}=J|x-y|^{-\alpha}$, $J, \varepsilon>0$, $\alpha > d$ and $h_x\in\mathbb{R}$, $d\geq 3$.
Until now, the only known result in the long-range setting is for the one-dimensional long-range Ising model with a random field, by Cassandro, Orlandi, and Picco \cite{Cassandro.Picco.09}. They used the contours of \cite{Cassandro.05} to show the phase transition for the model when $\alpha\in (3-\frac{\ln 3}{\ln 2}, \frac{3}{2})$, under the assumption $J(1) \gg 1$. We stress that, as remarked by Aizenman, Greenblatt, and Lebowitz \cite{Aizenman_Greenblatt_Lebowitz_2012}, although their argument does not work for the whole region for the exponent $\alpha$, the phase transition holds for values close to the critical value $\alpha=3/2$, since by the Aizenman-Wehr theorem we know that there is uniqueness for $\alpha>3/2$.

The argument from Ding and Zhuang in \cite{Ding2021}, for $d\geq3$, involves controlling the probability of a bad event, which is closely related to controlling the quantity $$\sup_{\substack{0\in A\subset\Z^d \\ A \text{ connected }}}\frac{\sum_{x\in A}h_x}{|\partial A|},$$ known as the greedy animal lattice normalized by the boundary. The greedy animal lattice normalized by the size, instead of the boundary, was extensively studied for general distributions of $(h_x)_{x\in\Z^d}$, see \cite{Cox_Gandolfi_Griffin_Kesten_93, Gandolfi_Kesten_94, Hammond_06, Martin_02}. When we normalize by the boundary, an argument by Fisher, Fr\"{o}hlich and Spencer \cite{FFS84} shows that the expected value of the greedy animal lattice is constant. In dimension $d=2$, the expected value is not finite, see \cite{Ding.Wirth.20}. The supremum is taken over connected regions containing the origin since the interiors of the usual Peierls contours are of this form.


For the long-range model, the interior of contours is not necessarily connected. In fact, long-range contours may have considerably large diameters with respect to their size, so their interiors can be very sparse. To avoid this, we define contours, strongly inspired by the $(M,a,r)$-partition in \cite{Affonso.2021}, using a multiscaled procedure that assures that the contours have no cluster with small density.  With them, we generalize the arguments by Fisher-Fr\"{o}hlich-Spencer \cite{FFS84}, and prove that the expected value of the greedy animal lattice is constant, even considering regions not necessarily connected in the supremum. Then, we prove the phase transition for $d\geq 3$. The main result of this paper is the following.
\begin{theorem*}Given $d\geq 3$, $\alpha>d$, there exists $\beta_c\coloneqq\beta(d, \alpha)$ and $\varepsilon_c\coloneqq\varepsilon(d, \alpha)$ such that, for $\beta >\beta_c$ and $\varepsilon\leq \varepsilon_c$, the extremal Gibbs measures $\mu_{\beta, \varepsilon}^+$ and $\mu_{\beta, \varepsilon}^-$ are distinct, that is, $\mu_{\beta, \varepsilon}^+ \neq \mu_{\beta, \varepsilon}^-$ $\mathbb{P}$-almost surely. Therefore the long-range random field Ising model presents phase transition.
\end{theorem*}

This paper is divided as follows. In Section 2, we define the model and the contours, and suitable generalizations to the constructions in \cite{Affonso.2021} are introduced.  In Section 3, we define two bad events of the external field and prove that they occur with a small probability.  In Section 4, we present the proof of the phase transition.
\section{Setup and Calculation}\label{sec:setup}
We consider the production of a vector boson $V$ through lepton-antilepton annihilation and its subsequent decay into a quark, anti-quark and a gluon
\begin{equation}
l^{+}(p_5)+l^{-}(p_6) \to V(p_4) \rightarrow q(p_1) + \bar{q}(p_2) + g(p_3)\, .
\end{equation}
The calculation is performed assuming a purely vector coupling for $V$ and the effect of non-singlet axial couplings is reconstructed later by reweighting the
helicity amplitudes by the appropriate couplings, see eq.~\eqref{eq:HALpL}.
We introduce the usual Mandelstam invariants
\begin{align}\nonumber
s_{12} = (p_1+p_2)^2\,, \quad   s_{13} = (p_1+p_3)^2\,, \quad  s_{23} = (p_2+p_3)^2
\end{align}
which satisfy the conservation equation
\begin{equation}
s_{12} + s_{13} + s_{23} = q^2\,,
\label{eqn:mandelstamsum}
\end{equation}
where $q^2=p_4^2$ is the momentum-squared of the vector boson. It is convenient to work with the following dimensionless ratios
\begin{align}
    x = {s_{12}}/{q^2}\,, \quad \quad  y = {s_{13}}/{q^2}\,, \quad \quad z = {s_{23}}/{q^2}\,, 
    \label{eqn:xyz}
\end{align}
such that \eqref{eqn:mandelstamsum} implies $x+y+z=1$. It is easy to see that in the decay kinematic region all these invariants are non-negative and
\begin{equation}
 z \geq 0\,, \quad \quad  0 \leq  y  \leq 1 - z\,, \quad  \quad x = 1 - y -z\,. 
\end{equation}


We work in dimensional regularization, taking ${d = 4 -2 \epsilon}$. In particular, we adopt the 't Hooft-Veltman dimensional regularization scheme~\cite{tHooft:1972tcz}, in which loop momenta are $d-$dimensional but external states are kept in four dimensions. The amplitude for the production of a vector boson can then be written as 
\begin{align}
\mathcal{M}_{} &= -i\sqrt{4\pi\bar{\alpha}_s}\, {\mathbb T_{ij}^{a}} A_{ \mu \nu} \epsilon_3^{\mu} \epsilon_4^{\nu}\,,
\label{eq:Ampl}
\end{align}
where $\bar{\alpha}_s$ is the bare strong coupling and ${\mathbb T_{ij}^a}$ is the generator of $SU(N)$ in the fundamental representation.
We can then proceed by decomposing the amplitude $A^{\mu\nu}$ in eq.~\eqref{eq:Ampl} into a basis of tensor structures in $d=4$ spacetime dimensions following a method introduced in~\cite{Peraro:2019cjj,Peraro:2020sfm}, see ref.~\cite{Gehrmann:2023zpz} for details.

Instead of calculating the corresponding form factors, we prefer to start from this decomposition 
and introduce projector operators that directly extract the amplitude for fixed helicities of the external states and contracted with a leptonic current.
To this end, we define
\begin{equation}
    {\rm M}_{\lambda_{q_2}\lambda_3 \lambda_{l_5}} = \epsilon_{3,\mu}^{\lambda_3}   \, 
	A^{\mu \nu}_{\lambda_{\bar q_1}\lambda_{q_2} } C_{\lambda_{l_5}}^\nu(p_5,p_6)\,.
	 \label{eq:HAnocouplings}
\end{equation}
Working in spinor-helicity formalism, we write for the left- and right-handed fermionic currents 
\begin{align}\label{eq:leptonspinhel1}
    C_L^\mu(p,q) = [q \gamma^\mu p\rangle\,, \qquad C_R^\mu(p,q) = [p \gamma^\mu q\rangle\,, 
\end{align}
and for the polarization vector of the outgoing external gluon 
\begin{equation}\label{eq:gluonpol}
\epsilon_{3,-}^\mu = \frac{\langle 3 \gamma^\mu 2 ]}{\sqrt{2} [3 2]}\,, \quad 
\epsilon_{3,+}^\mu = \frac{\langle 2 \gamma^\mu 3 ]}{\sqrt{2} \langle 2 3 \rangle}\,,
\end{equation}
where we adopted an axial gauge with $p_2$ as the reference vector.
In this setup, the two independent helicity amplitudes read
\begin{alignat}{3}
    {\rm M}_{L+L} &= \frac{1}{\sqrt{2}}  
	 \Big[ \langle 1 2 \rangle [1 3]^2 \Big( \alpha_1 \langle 536] 
	&&+ \alpha_2 \langle 526] \Big)\nonumber\\
        &{}&&+ \alpha_3 \langle 25 \rangle [13] [36] \Big]\,,
\label{eq:mLLv}\\
	 {\rm M}_{L-L} &=  \frac{1}{\sqrt{2}}  
	 \Big[ \langle 23 \rangle^2 [1 2] \Big( \gamma_1 \langle 536] 
	&&+ \gamma_2 \langle 516] \Big)\nonumber\\        
        &{}&&+ \gamma_3 \langle 23 \rangle \langle 35 \rangle [16] \Big], \label{eq:LpLv}
\end{alignat}
and the remaining two helicity amplitudes, ${\rm M}_{R+L}$ and ${\rm M}_{R-L}$, can be obtained by means of parity and charge conjugation~\cite{Gehrmann:2022vuk,Gehrmann:2023zpz}.
Additionally, for each amplitude, the lepton current handedness can be reversed by a simple swap $p_5 \leftrightarrow p_6$.
The contribution of a Feynman diagram to the independent helicity coefficients $\alpha_i$ and $\gamma_i$ with $i=1,...,3$ can then be obtained by applying suitable projector operators. Their explicit form (immaterial to the present discussion) is given in the supplemental material.


The helicity amplitude coefficients 
$\Omega = (\alpha_i,\gamma_i)$ can be expanded in powers of the  strong coupling constant
\begin{align}
\Omega &= \Omega^{(0)} + \left( \frac{\alpha_s}{2 \pi}\right) \Omega^{(1)}
     + \left( \frac{{\alpha}_s}{2 \pi}\right)^2 \Omega^{(2)} \nonumber \\
    &+ \left( \frac{{\alpha}_s}{2 \pi}\right)^3 \Omega^{(3)} +\mathcal{O}({\alpha}_s^4) \,.   \label{eq:omegaexp}
\end{align}
This expansion holds for the bare coefficients (with $\alpha_s$ replaced by $\bar{\alpha}_s$), the renormalized coefficients, as well as for the IR-subtracted coefficients defined below. 
The three-loop coefficients 
are in turn decomposed into color factors as
\begin{align}\label{eq:colordecomp}
\Omega^{(3)}&=   
N^3\:\Omega^{(3)}_1
+ N \Omega^{(3)}_2
+ \frac{1}{N} \Omega^{(3)}_3
+ \frac{1}{N^{3}} \Omega^{(3)}_4\nonumber\\
&+ N_{f}  \:N^2 \Omega^{(3)}_5
+ N_{f} \Omega^{(3)}_6
+ \frac{N_{f}}{N^{2}} \Omega^{(3)}_7\nonumber\\
&+ N_{f}^2  N \:\Omega^{(3)}_8
+ \frac{N_{f}^2}{N} \Omega^{(3)}_9 
+ N_{f}^3 \Omega_{10}^{(3)}
\nonumber\\
&+ N_{f,V}  N^2 \Omega^{(3)}_{11}
+ N_{f,V} \Omega^{(3)}_{12}\nonumber\\
&+ \frac{N_{f,V}}{ N^{2}} \Omega^{(3)}_{13}
+ N_{f} N_{f,V}  N \Omega^{(3)}_{14}
+ \frac{N_{f} N_{f,V}}{ N} \Omega^{(3)}_{15}, %%  \nonumber\\& + N_{f}^2 N_{f,V} \Omega^{(3)}_{16}
\end{align}
where $N_f$ is the number of active quark flavors in the loops, and $N_{f, V}$ denotes contributions where the vector boson couples to an internal fermion loop. In this letter we focus on 
the dominant terms in this expansion. 
Using $N_f \sim N$, these are all the terms with total power of three in $N$ and $N_f$
and they correspond to $\Omega_j$ with $j=1,5,8,10$.
Importantly, the leading color factors only receive contribution from planar diagrams, see Figure~\ref{fig:representative_diags} for example diagrams. 
The contributions proportional to $N_f^2 N_{f,V}$ vanish for a vector-like coupling by Furry's theorem. 
Eq.~\eqref{eq:colordecomp} is also valid for any helicity amplitude coefficient -- bare, renormalized or IR subtracted. 

% Figure environment removed

In the following, we fix the renormalization scale in 
$\Omega$ as $\mu^2=q^2$.
The full scale dependence can then be recovered through
\begin{align}
     \Omega^{(3)}(\mu) & = 
     \left(\frac{5}{16} \beta _0^3 L(\mu)^3+\beta _0 \beta _1 L(\mu)^2+\frac{1}{2} \beta _2 L(\mu)\right)\Omega^{(0)}\nonumber\\
    &+  \left(\frac{15}{8} \beta _0^2 L(\mu)^2+\frac{3}{2} \beta _1 L(\mu)\right) \Omega^{(1)}\nonumber\\
   &     +\frac{5}{2} \beta _0 L(\mu) \Omega ^{(2)} + \Omega ^{(3)} 
\end{align}
with $L(\mu) = \log \left(\mu ^2/q^2\right)$. 


The helicity amplitudes for the decay of a Standard Model vector boson $V$ can finally be related to the helicity amplitudes obtained above by dressing with the appropriate electroweak couplings
\begin{align}
\mathcal{M}_{\lambda_{q_2}\lambda_3 \lambda_{l_5}}^V =& - 
 \frac{i \sqrt{4 \pi \alpha_s} (4 \pi \alpha)\, L_{l_5 l_6}^{V} L_{q_1 q_2}^V }{D(p_{56}^2,m_V^2)} \nonumber \\ 
 &\times
 {\mathbb T^{a}_{ij}} \,{\rm M}_{\lambda_{q_2}\lambda_3 \lambda_{l_5}} \,, \label{eq:HALpL}
 \end{align}
where $p_{56} = p_5 + p_6$, the vector boson propagator reads
\begin{align}
D\left(q^2,m_V^2\right) &= q^2 - m_V^2 + i \Gamma_V m_V
\end{align}
and the couplings for the bosons ${V=Z,W^\pm,\gamma^*}$ are
\begin{align}
    R_{f_1 f_2}^\gamma = L_{f_1 f_2}^\gamma &= - e_{f_1} \delta_{f_1 f_2}\,,\\
    L_{f_1 f_2}^Z &= \frac{I_3^{f_1} - \sin^2{\theta_w} e_{f_1}}{\sin{\theta_w} \cos{\theta_w}} \delta_{f_1 f_2} \,, \\
    L_{f_1 f_2}^W &= \frac{\epsilon_{f_1,f_2}}{\sqrt{2}\sin{\theta_w}}  \,,\\
    R_{f_1 f_2}^Z &=-\frac{\sin{\theta_w} e_{f_1}}{\cos{\theta_w}} \delta_{f_1 f_2}\,,\\
    R_{f_1 f_2}^W &=0  \,.
\end{align}  
In the formulas above, $\alpha$ is the electroweak coupling constant,
$\theta_w$ is the Weinberg angle, $I_3 = \pm 1/2$ is the third component
of the weak isospin and the charges $e_i$
are measured in terms of the fundamental electric charge $e>0$.
Moreover, $\epsilon_{f_1,f_2}=1$ if $f_1 \neq f_2$ but belonging to the same
isospin doublet, and zero otherwise. 

In order to compute the (unrenormalized) corrections to the helicity amplitude coefficient, we use the same unified workflow as for the tree-level, one- and two-loop amplitudes for $Vq\bar{q}g$~\cite{Gehrmann:2023zpz}, whose agreement with older results in the literature up to the finite part in $\epsilon$ provides an additional check on our method. In summary, the relevant three-loop diagrams are  generated using \code{QGRAF}~\cite{Nogueira:1991ex} and every manipulation including insertion of Feynman rules, evaluation of  Dirac and Lorentz algebra and application of the projectors are performed in \code{FORM}~\cite{Vermaseren:2000nd}. 
Once the helicity projectors have been applied, all Feynman diagrams are expressed in terms of scalar integrals, which can be written in terms of a single planar auxiliary topology of the form
\begin{equation}
\mathcal{I}_{n_1,...,n_{15}} = e^{3 \gamma_E \epsilon} \int \prod_{i=1}^3 \frac{d^d k_i}{i \pi^{d/2}} \frac{1}{D_1^{n_1}... D_{15}^{n_{15}}}
\end{equation}
with $\gamma_E=0.5772\ldots$ the Euler constant and propagators
\begin{align*}
\begin{tabular}{l@{\hskip 0.2in}l@{\hskip 0.2in}l}
$D_1 = k_1 $    & $D_6 = k_3 - p_1$               & $D_{11} = k_2 - p_{123}$ \\
$D_2 =  k_2$    & $D_7 = k_1 - p_{12}$          & $D_{12 }= k_3 - p_{123}$  \\
$D_3 = k_3 $    & $D_8 = k_2 - p_{12} $         & $D_{13} = k_1 - k_2 $           \\
$D_4 = k_1-p_1$ & $D_9 = k_3 - p_{12} $         & $D_{14} = k_1 - k_3$            \\
$D_5 = k_2-p_1$ & $D_{10} = k_1 - p_{123}$ & $D_{15} = k_2 - k_3$          
\end{tabular}
\end{align*}
with $p_{ij(k)} = p_i + p_j (+p_k)$.
The integrals can be reduced to a set of master integrals using integration-by-parts (IBP)
identities~\cite{Tkachov:1981wb,Chetyrkin:1981qh}. For the actual reduction, 
we use the implementation of the Laporta
algorithm~\cite{Laporta:2001dd} in the automated code \code{Kira2}~\cite{Maierhoefer:2017hyi,Klappert:2020nbg} and express all integrals
directly in terms of the canonical basis for the three-loop planar family defined in~\cite{Canko:2023yoe}.
Here it was shown that, in line with the one- and two-loop results, the three-loop planar integrals can be evaluated to arbitrary orders in the dimensional regularization
parameter $\epsilon$ in terms of multiple polylogarithms (MPLs)~\cite{Goncharov:1998kja,Remiddi:1999ew,Gehrmann:2000zt,Vollinga:2004sn}
with alphabet
$\{y,z,1-y,1-z,y+z,1-y-z\}$.

The amplitude before reduction can be expressed in terms of 95625 scalar integrals, which in turn are reduced to 291 canonical basis elements and their crossings. 
The size and complexity of intermediate expressions makes the use of traditional methods for symbolic insertion of the IBP reduction into the unreduced amplitude highly non-trivial. 
Therefore, 
 in view of the expected increase in complexity of the subleading layers in the color expansion~\eqref{eq:colordecomp}, 
 we also devised a hybrid method involving finite field reconstruction, in parallel to a standard fully analytic approach. 
 
 In particular, in the standard approach, we produced the IBP identities with~\code{Kira2} and used~\code{Mathematica} 
 and~\code{Fermat} to include them into the unreduced amplitude and simplify the resulting coefficients. We then expanded the master integrals in $\epsilon$ and obtained the final expression for the amplitude in terms of MPLs.
 In the hybrid approach, we used the same IBP reduction obtained with \code{Kira2}. However, we then performed an insertion of the reduction and MI solutions in the amplitude for a handful of rational numerical values of the variables $y$ and $z$ in the coefficients of the MPLs. This step allowed us to determine which polylogarithms appear in the final expression for a given helicity coefficient. With this information, we proceeded to reconstruct the rational prefactors in $y$ and $z$ only for the relevant polylogarithms, leveraging the implementation provided by \code{FiniteFlow}~\cite{Peraro:2019svx}. This approach lends itself to parallelization and gives us the possibility to recycle the symbolic reductions in future calculations. Compared to the traditional approach of reconstructing only the finite part, it allows us to present also the renormalized amplitudes with explicit IR poles for greater versatility in phenomenological applications with different pole schemes. We verified that the reconstructed results exactly matched those obtained through the fully analytic approach.
\section{UV renormalization and IR subtraction}\label{sec:UVandIR}
All UV divergences are removed by expressing the helicity coefficients in terms of the
$\overline{\rm MS}$-renormalized strong coupling $\alpha_s=\alpha_s(\mu)$.
Since this is a standard procedure, we only summarize the necessary $\beta$-function coefficients in the supplemental material.
The remaining IR divergences are universal and can be extracted multiplicatively 
in the formalism of Soft-Collinear Effective Theory (SCET)~\cite{Becher:2009qa, Catani:1998bh, Dixon:2009ur,Gardi:2009qi} as
\begin{equation}
    \mathbf{\mathbf{\Phi}}_{\text{finite}}(\{p\}) = \lim_{\epsilon \rightarrow 0} \mathbfcal{Z}^{-1}(\{p\}; \epsilon) \,\, \mathbf{\mathbf{\Phi}}({p}, \epsilon)\,,
    \label{eqn: IR-Renormalization}
\end{equation}
where we indicate with $\mathbf{\mathbf{\Phi}}_{\text{finite}}(\{p\})$ the finite remainder of the amplitude $\mathbf{\mathbf{\Phi}}({p}, \epsilon)$. As indicated by
the bold font, 
$\mathbfcal Z$ is generally an operator in color space, but it diagonalizes when acting on the $Vq\bar q g$ amplitude coefficients, which factorize onto 
a single color structure,
\begin{equation}
    \mathbf{\Phi}={\mathbb T^a_{ij}}\,\Omega\,.
    \label{eq:colvec}
\end{equation} 
$\mathbfcal Z$ can be formally expressed in terms of the anomalous dimension matrix $\mathbf\Gamma$ as
\begin{align}\nonumber
    \mathbfcal{Z}(\epsilon, \{p\}, \mu) &= \mathbb{P} \,  \text{exp} \Big[  \int_{\mu}^{\infty} \frac{d\mu'}{\mu'} \mathbf{\Gamma}(\{p\}, \mu') \Big] \\
    &= \sum_{l = 0}^{\infty} \Big( \frac{\alpha_s}{2 \pi} \Big)^l \mathbfcal{Z}^{(l)} \,.
    \label{eqn: Z_operator}
\end{align}

Importantly, for the first time at three loops, the anomalous dimension operator is not entirely captured by a dipole term but receives contributions also from quadrupole color correlation operators,
\begin{equation}\label{eq:dipole_+_quadrupole}
\mathbf{\Gamma}(\{p\}, \mu)=  \mathbf{\Gamma}_\text{dip}(\{p\}, \mu)  + \mathbf{\Delta}_3(\{p\}, \mu)  \; .
\end{equation}
The former is due to the pairwise exchange of color charge between 
external legs and reads
\begin{align}
 \mathbf{\Gamma}_{\text{dip}}(\{p\},\mu) &= \sum_{1 \leq i < j \leq n} \mathbf{T}_i^a\mathbf{T}_j^a \, \gamma^{\text{K}}(\alpha_s) \log\Big(-\frac{\mu^2}{s_{ij} + i \eta }\Big)\nonumber\\
 &\phantom{\quad}+  \, \sum_{i=1}^n \gamma^i(\alpha_s)\,,
 \label{DipoleExpression}
\end{align}
where the $\gamma^{\text{K}}$ is the cusp anomalous dimension and $\gamma^i$ is the  anomalous dimension of parton $i$. 

The quadrupole contribution $\mathbf{\Delta}_3$ in
eq.~\eqref{eq:dipole_+_quadrupole} accounts for the exchange
of color charge among three or four soft gluons emitted from (a subset of) the partonic legs. Unlike in three-loop four-parton scattering~\cite{Henn:2016jdu,Caola:2021rqz,Caola:2021izf,Caola:2022dfa} where this term was observed in N=4 SYM and QCD amplitudes for the first time, soft gluon emissions from four external partons are impossible here and thus the quadrupole term consists only of the entirely kinematics-independent term~\cite{Almelid:2015jia}
\begin{multline} \label{eq:quadrupole}
\mathbf{\Delta}^{(3), ijk}_3 = -f_{abe} f_{cde}\,
\bigg[
16 \,C   \, \sum_{i=1}^3 \sum_{\substack{1\leq j < k \leq3 \\ j,k\neq i}}  \left\{ \mathbf{T}^a_i,\mathbf{T}^d_i \right\} \mathbf{T}^b_j \mathbf{T}^c_k 
\bigg]\,,
\end{multline}
with  $C = \zeta_5 + 2 \zeta_2 \zeta_3$ a constant. However, this term only enters the color layers that are suppressed by at least $N^{-2}$ relative to the 
leading color contributions considered here.


\begin{table*}[t]
\begin{tabular}{|c||c|c|c|c|}
\hline
{} &  $V\to gq\bar{q}$ & $q\bar{q}\to Vg$ & $qg\to Vq$ & $\bar{q}g\to V\bar{q}$  \\ \hline
 {} & $y=z=1/3$ & \multicolumn{3}{c|}{$u=1/7, v=3/5$}  \\ \hline\hline
$\alpha_1$ & $0.59426734 + 0.03099861 i$ & $ -0.49538117 +    0.65353908 i$ & $ -0.22197101 - 0.03494388 i$ & $ -0.22197101 -    0.03494388 i$ \\ \hline
$\alpha_2$ & $ 1.7192645 - 8.7893957 i$ & $   2.0901940 + 8.1000376 i$ & $ -1.6429881 + 0.5830600 i$ & $ -1.6429881 +    0.5830600 i$ \\ \hline
$\alpha_3$ & $ 1.2128210 + 5.9479256 i$ & $ -6.0948209 - 9.3383011 i$ & $   0.94229209 - 0.78134227 i$ & $   0.94229209 - 0.78134227 i$ \\ \hline
$\gamma_1$ & $ -0.59426734 - 0.03099861 i$ & $   0.46544666 - 1.05443048 i$ & $ -0.05750758 +    0.15110700 i$ & $ -0.05750758 + 0.15110700 i$ \\ \hline
$\gamma_2$ & $ -1.7192645 +    8.7893957 i$ & $ -1.0680269 - 9.5317293 i$ & $ 0.24066096 - 0.32042746 i$ & $   0.24066096 - 0.32042746 i$ \\ \hline
$\gamma_3$ & $ 1.2128210 + 5.9479256 i$ & $ -5.2865869 -    7.5133951 i$ & $ 0.18231060 + 0.07136413 i$ & $ 0.18231060 + 0.07136413 i$ \\ \hline
\end{tabular}
\caption{Numerical values for the finite remainders of the helicity coefficients at a phase-space point in each channel to 8 significant figures, setting $q^2=1$, $N=3$ and $N_f=5$. All values are to be multiplied by $10^4$.}
\label{tab: numerics}
\end{table*}


In summary, given the $l$-loop renormalized helicity amplitudes $\mathbf{\Phi}^{(l)}$, one can remove the IR singularities with
\begin{align}
\mathbf{\Phi}^{(0)}_{\text{finite}} & = \mathbf{\Phi}^{(0)}, \nonumber \\
\mathbf{\Phi}^{(1)}_{\text{finite}} & = \mathbf{\Phi}^{(1)} - \mathbf{I}^{(1)} \, \mathbf{\Phi}^{(0)} ,
%\label{eqn: perturbative IR subtraction one loop}\\
\nonumber \\
\mathbf{\Phi}^{(2)}_{\text{finite}} & = \mathbf{\Phi}^{(2)} - \mathbf{I}^{(2)} \, \mathbf{\Phi}^{(0)}  - \mathbf{I}^{(1)} \, \mathbf{\Phi}^{(1)}\,,
\nonumber \\
%\label{eqn: perturbative IR subtraction two loops}\\
\mathbf{\Phi}^{(3)}_{\text{finite}} & = \mathbf{\Phi}^{(3)} - \mathbf{I}^{(3)} \, \mathbf{\Phi}^{(0)}  - \mathbf{I}^{(2)} \, \mathbf{\Phi}^{(1)} - \mathbf{I}^{(1)} \, \mathbf{\Phi}^{(2)}\,.
\label{eqn: perturbative IR subtraction three loops}
\end{align}
The subtraction operators in \eqref{eqn: perturbative IR subtraction three loops} can be expressed as
\begin{align}
 \mathbf{I}^{(1)} & = \mathbfcal{Z}^{(1)}\,,\nonumber \\
 \mathbf{I}^{(2)} & = \mathbfcal{Z}^{(2)} - \big(\mathbfcal{Z}^{(1)}\big)^2\,,\nonumber \\
 \mathbf{I}^{(3)} & = \mathbfcal{Z}^{(3)} - 2\mathbfcal{Z}^{(2)}\mathbfcal{Z}^{(1)} + \big(\mathbfcal{Z}^{(1)}\big)^3\,.
 \label{eqn: IR scet}
\end{align}
After performing the color algebra in (\ref{eqn: perturbative IR subtraction three loops}), we recover the simple factorizing structure (\ref{eq:colvec}) for 
the finite parts of the helicity amplitude. Consequently, 
(\ref{eqn: perturbative IR subtraction three loops})
also holds for the amplitude coefficients $\Omega$. 

As explained in detail in~\cite{Gehrmann:2023etk}, the explicit form of the operators $\mathbfcal{Z}_l$ in terms of the universal constants in the perturbative expansion for $\gamma^K$, $\gamma^q$ and $\gamma^g$ is implied by the dipole formula \eqref{DipoleExpression} and the new term $\mathbf{\Delta}^{(3)}_3$. A summary of the relevant expressions for UV renormalization and IR subtraction is given in the supplemental material.



\section{Results}\label{sec:helamps}
In view of future phenomenological applications 
of the three-loop $Vq\bar qg$ amplitudes to LHC physics, we perform the analytic continuation of the $V$ decay amplitudes to the kinematic regions where the electroweak boson is produced in addition to a jet from the scattering of two partons: $ q g \rightarrow V q$, $\bar{q} g \rightarrow V \bar{q}$ and $q \bar{q} \rightarrow V g $.

The general strategy for performing the analytic continuation of the MPLs appearing in $2\to 2$ scattering involving four-point functions with one off-shell leg and massless propagators was outlined in  detail in~\cite{Gehrmann:2002zr} and adapted for the case of $Vq\bar{q}g$ in~\cite{Gehrmann:2023zpz}. 
In summary, we can access all the production helicity amplitudes by combinations of kinematic crossings of the external momenta, parity flips and the analytic continuation of MPLs to the region denoted as (3a) in~\cite{Gehrmann:2002zr} via the substitution
\begin{align}
y = {1}/{v}\,, \quad \quad z = -{u}/{v}\,.
\end{align}

The finite remainders of the independent helicity amplitudes in the decay and production regions constitute the main result of this work and are linked in electronic format to the \code{arXiv} submission of this letter.

%\subsection{Numerical results}\label{subsec:numerics}
The provided results are compact and can easily be evaluated numerically for phenomenological studies. To demonstrate this, in table \ref{tab: numerics} we present values for the helicity coefficients at the symmetric phase space point $y=z=1-y-z=1/3$ in decay kinematics (where the coefficients $\alpha$ and $\gamma$ take the same absolute value and where all imaginary parts are generated from the expansion of the overall factors $(-q^2)^{-\epsilon}$) and at a selected point in each of the production regimes.
The values were obtained using \code{GiNaC}~\cite{Vollinga:2004sn} as implemented in \code{PolyLogTools}~\cite{Duhr:2019tlz}.

We verified that in all helicity configurations, the symbol of the leading colour $Vgq\bar{q}$ amplitude satisfies the conjectured non-adjacency conditions~\cite{Dixon:2020bbt}. It was found in~\cite{Henn:2023vbd} that in the tennis-court topology, a single master integral with 8 propagators and its crossing have a symbol with $1-z$ and $1-x$ appearing next to each other. This violation propagates to several higher topologies during integration and the cancellation of the violating symbol in the amplitude is therefore non-trivial.
\section{Summary and Conclusions}
\label{sec:conclusions}
We have outlined the design of a small electromagnetic calorimeter, the Few-Degree Calorimeter (FDC), which is designed to cover the range of $-4.6 < \eta < -3.6$. The primary objective of this detector is to tag electrons within the $Q^2$ range of 0.1 to 1.0 GeV$^2$, thus enabling future research on the transition to perturbative QCD and the gluon-saturation regime.

The FDC design we present here incorporates the latest advancements in SiPM-on-tile calorimetry to create a modern and improved version of the ZEUS Beam Pipe Calorimeter and H1 Very Low $Q^{2}$ calorimeter. The incorporation of high-granularity 5D shower measurements (position, time, and energy) offered by this technology holds great potential for background tagging.

In conclusion, this document presents the first design that has the potential to close the EIC $Q^2$ gap while maintaining a compact and cost-effective solution. Considering the larger crossing-angle envisioned for the second-interaction region at the EIC, which results in a larger hole in the crystal ECAL acceptance, this design may offer further opportunities for optimization for the EIC Detector 2.




\section*{ACKNOWLEDGEMENTS}
This work was supported in part by the Excellence Cluster ORIGINS funded by the Deutsche Forschungsgemeinschaft (DFG, German Research Foundation) under Germany’s Excellence Strategy – EXC-2094-390783311, by the Swiss National Science Foundation (SNF) under contract 200020-204200, and by the European Research Council (ERC) under the European Union’s research and innovation programme grant agreements 949279 (ERC Starting Grant HighPHun) and 101019620 (ERC Advanced Grant TOPUP).

\bibliography{main}
\newpage
\onecolumngrid
\appendix

\section*{Supplemental material}
\makeatletter
\makeatother
In this supplemental material, we collect key formulas relevant to the calculation of the three-loop helicity amplitude coeffients and to their renormalization and infrared subtraction. 
\subsection{UV renormalization}
The first three $\beta$-function coefficients are
\begin{align}
    \beta_0 &= \frac{11 C_A}{6} - \frac{2 T_R N_f}{3}\,, \\
    \beta_1 &= \frac{17 C_A^2}{6} - \frac{5 C_A T_R N_f}{3} -  C_F T_R N_f\,,\\
    \beta_2 &= -\frac{205}{72} C_A C_F N_f T_R-\frac{1415}{216} C_A^2 N_f T_R+\frac{79}{108} C_A N_f^2
   T_R^2+\frac{2857 C_A^3}{432}+\frac{11}{18} C_F N_f^2 T_R^2+\frac{1}{4} C_F^2 N_f T_R\,,
\end{align}
with the QCD colour factors
\begin{equation}
    C_A = N,\quad C_F = \frac{N^2-1}{2N},\quad T_R = \frac{1}{2}\,.
\end{equation}
\subsection{IR subtraction}
We define the expansions
\begin{equation}
\mathbf{\Gamma}_{\text{dip}} = \sum_{l=0}^{\infty} \mathbf{\Gamma}_{l} \, \,  \Big(\frac{\alpha_s}{2 \pi} \Big)^{l+1}, \quad \quad \, \mathbf{\Gamma}' = \frac{\partial \mathbf{\Gamma}_{\text{dip}}}{\partial \log(\mu)} = \sum_{l=0}^{\infty} \mathbf{\Gamma}'_{l} \, \,  \Big(\frac{\alpha_s}{2 \pi} \Big)^{l+1}\,.
\label{DipoleExpansion}
\end{equation}
Both of the operators diagonalize in colour space for a process with three partons so we can drop the boldface notation. Substituting into \eqref{eqn: Z_operator}, we obtain
\begin{align}
    \mathcal{Z}^{(1)} &= \frac{\Gamma_{0}'}{4 \epsilon ^2} + \frac{\Gamma _0}{2 \epsilon }\\
    \mathcal{Z}^{(2)} &= \frac{\Gamma_{0}^{\prime 2}}{32 \epsilon ^4}+\frac{\Gamma'_0}{8 \epsilon
   ^3}\left(\Gamma _0-\frac{3 \beta _0}{2}\right)+\frac{1}{4 \epsilon ^2}\left(-\beta _0 \Gamma _0+\frac{\Gamma _0^2}{2}+\frac{\Gamma'
   _1}{4}\right)+\frac{\Gamma _1}{4 \epsilon }\\
    \mathcal{Z}^{(3)} &= +\frac{\Gamma_0^{\prime 3}}{384 \epsilon ^6} +\frac{\Gamma_0^{\prime 2}}{64 \epsilon ^5}\left(\Gamma _0-3\beta _0\right) +\frac{\Gamma_{0}'}{ 9\epsilon ^4}\left(-\frac{5}{4} \beta _0 \Gamma _0 +\frac{11}{9}
   \beta _0^2 +\frac{1}{4} \Gamma _0^2 +\frac{
   \Gamma_{1}'}{8}\right)\nonumber\\
   &+ \frac{1}{\epsilon ^3}\left(\frac{1}{54} \Gamma_{0}' \left(10 C_A N_f T_R-17 C_A^2+6 C_F N_f T_R\right)+\Gamma _0
   \left(\frac{\beta _0^2}{6}+\frac{\Gamma_{1}'}{32}\right)-\frac{1}{8} \beta _0 \Gamma
   _0^2-\frac{5 \beta _0 \Gamma_{1}'}{72}+\frac{\Gamma _1 \Gamma_{0}'}{16}+\frac{\Gamma
   _0^3}{48}\right)\nonumber\\
   &+\frac{1}{\epsilon ^2}\left(\Gamma _0 \left(\frac{1}{36} \left(10 C_A N_f T_R-17 C_A^2+6 C_F N_f
   T_R\right)+\frac{\Gamma _1}{8}\right)-\frac{\beta _0 \Gamma _1}{6}+\frac{\Gamma'
   _2}{36}\right)+\frac{\Gamma _2 + \Delta_4^{(3)}}{6 \epsilon }\,.
\end{align}
Defining a perturbative expansion for the anomalous dimensions with $i=K,q,\overline{q},g$\,,
\begin{align}
\gamma^{i} = \sum_{l = 0}^{\infty} \gamma_l^{i} \Big( \frac{\alpha_s}{2 \pi} \Big)^{l+1}\,,
\end{align}
we can use the dipole formula to write for the $gq\bar{q}$ system
\begin{align}
\Gamma_{l} &=  -C_F \, L_{12} \,  \gamma^{K}_l - \frac{C_A}{2} \Big(-L_{12} + L_{23} + L_{13} \Big) \gamma^{K}_l  + 2 \gamma_l^{q} + \gamma_l^{g},
\label{eqn: GammaExpansion}\\
\Gamma_{l}' &= - \gamma^{K}(\alpha_s) (2C_F+C_A)\,,
\label{eqn: gamma prime}
\end{align}
with
\begin{equation}
    L_{ij} = \log{\Big(-\frac{\mu^2}{s_{ij}+ i \eta} \Big)}\,.
\end{equation}
For convenience, we also reproduce the first three coefficients for the cusp anomalous dimension
\begin{flalign}
    \gamma^{K}_0 &=2&&\\
    \gamma^{K}_1 &=\left(\frac{67}{9}-\frac{\pi ^2}{3}\right) C_A-\frac{10 N_f}{9}&&\\
    \gamma^{K}_2 &=\left(-\frac{14 \zeta _3}{3}+\frac{10 \pi ^2}{27}-\frac{209}{54}\right) C_A N_f+\left(\frac{11 \zeta _3}{3}+\frac{11 \pi ^4}{90}-\frac{67 \pi ^2}{27}+\frac{245}{12}\right) C_A^2+\left(4 \zeta _3-\frac{55}{12}\right) C_F
   N_f-\frac{2 N_f^2}{27}\,,&&
\end{flalign}
the quark anomalous dimension
\begin{flalign}
    \gamma^{q}_0 &=-\frac{3 C_F}{2}&&\\
    \gamma^{q}_1 &=\left(\frac{13 \zeta _3}{2}-\frac{11 \pi ^2}{24}-\frac{961}{216}\right) C_A C_F+\left(\frac{65}{108}+\frac{\pi ^2}{12}\right) C_F N_f+\left(-6 \zeta _3+\frac{\pi ^2}{2}-\frac{3}{8}\right) C_F^2&&\\
    %%
    \gamma^{q}_2 &=\left(-\frac{241 \zeta _3}{54}+\frac{11 \pi ^4}{360}+\frac{1297 \pi ^2}{1944}-\frac{8659}{5832}\right) C_A C_F N_f+\left(-\frac{1}{3} \pi ^2 \zeta _3-\frac{211 \zeta _3}{6}-15 \zeta _5+\frac{247 \pi
   ^4}{1080}+\frac{205 \pi ^2}{72}-\frac{151}{32}\right) C_A C_F^2\nonumber&&\\
   %
   &+\left(-\frac{11}{18} \pi ^2 \zeta _3+\frac{1763 \zeta _3}{36}-17 \zeta _5-\frac{83 \pi ^4}{720}-\frac{7163 \pi ^2}{3888}-\frac{139345}{23328}\right)
   C_A^2 C_F+\left(\frac{32 \zeta _3}{9}-\frac{7 \pi ^4}{108}-\frac{13 \pi ^2}{72}+\frac{2953}{432}\right) C_F^2 N_f\nonumber&&\\
   %
   &+\left(-\frac{\zeta _3}{27}-\frac{5 \pi ^2}{108}+\frac{2417}{5832}\right) C_F N_f^2+\left(\frac{2 \pi
   ^2 \zeta _3}{3}-\frac{17 \zeta _3}{2}+30 \zeta _5-\frac{\pi ^4}{5}-\frac{3 \pi ^2}{8}-\frac{29}{16}\right) C_F^3\,,&&
\end{flalign}
and the gluon anomalous dimension
\begin{flalign}
    \gamma^{g}_0 &=-\beta _0&&\\
    \gamma^{g}_1 &=\left(\frac{32}{27}-\frac{\pi ^2}{36}\right) C_A N_f+\left(\frac{\zeta _3}{2}+\frac{11 \pi ^2}{72}-\frac{173}{27}\right) C_A^2+\frac{C_F N_f}{2}&&\\
    \gamma^{g}_2 &=\left(-\frac{19 \zeta _3}{9}-\frac{\pi ^4}{90}-\frac{\pi ^2}{24}+\frac{1217}{216}\right) C_A C_F N_f+\left(-\frac{7 \zeta _3}{27}+\frac{5 \pi ^2}{324}-\frac{269}{11664}\right) C_A N_f^2-\frac{11}{72} C_F N_f^2-\frac{1}{8} C_F^2 N_f\nonumber&&\\
    &+\left(\frac{89 \zeta
   _3}{54}+\frac{41 \pi ^4}{1080}-\frac{599 \pi ^2}{1944}+\frac{30715}{11664}\right) C_A^2 N_f+\left(-\frac{5}{18} \pi ^2 \zeta _3+\frac{61 \zeta _3}{12}-2 \zeta _5-\frac{319 \pi ^4}{2160}+\frac{6109 \pi
   ^2}{3888}-\frac{48593}{2916}\right) C_A^3\,.&&
\end{flalign}


\subsection{Projectors onto helicity amplitude coefficients}
With the common denominator $\mathcal{D} = 2(d-3)y^3z^2(1-y-z)^2$ and the tensor basis defined as in ref.~\cite{Gehrmann:2023zpz},
\begin{flalign}
\mathcal{P}_{\alpha_1} &=\frac{1}{\mathcal{D}}\Big[-d (1-y)^2 z \:T^{\dagger }_1+z (-2 y (-1+y+2 z)+d (-1+y+z+y z)) \:T^{\dagger }_2-(1-y) y z (1-y-z) \:T^{\dagger }_3\nonumber &&\\
&\phantom{=\frac{1}{\mathcal{D}}\Big[\:}+(1-y)^2 y z \:T^{\dagger }_4-y z (-1+y+z+y z)
   \:T^{\dagger }_5+(1-y) y z^2 \:T^{\dagger }_6\Big]\,,&&\\
   %
\mathcal{P}_{\alpha_2} &=\frac{1}{\mathcal{D}}\Big[-z \left(-(d-4) y^2+(d-4) y (1-z)+d z\right) \:T^{\dagger }_1+z \left(2 y^2-d (1-z) z+y (-4+d+(2+d) z)\right) \:T^{\dagger
   }_2\nonumber &&\\
&\phantom{=\frac{1}{\mathcal{D}}\Big[\:}-y z (1-y-z) (y+z) \:T^{\dagger }_3-y z (z+y (1-y-z)) \:T^{\dagger }_4-y z \left(y-(1-y) z+z^2\right) \:T^{\dagger }_5+y z^2 (y+z) \:T^{\dagger }_6\Big]\,,&&\\
%
\mathcal{P}_{\alpha_3} &=\frac{1}{\mathcal{D}}\Big[+4 (1-y) y z
   (1-y-z) \:T^{\dagger }_1+4 y z (1-y-z)^2 \:T^{\dagger }_2+2 y^2 z (1-y-z)^2 \:T^{\dagger }_3-2 y^2 z^2 (1-y-z) \:T^{\dagger }_6\Big]\,, &&\\
   %
\mathcal{P}_{\gamma_1} &=\frac{1}{\mathcal{D}}\Big[+(d-4) y (-1+y+z+y z)
   \:T^{\dagger }_1+y (1-z) (-4+d+2 y+4 z-d z) \:T^{\dagger }_2-y^2 (1-z) (1-y-z) \:T^{\dagger }_3\nonumber &&\\
&\phantom{=\frac{1}{\mathcal{D}}\Big[\:}-y^2 (-1+y+z+y z) \:T^{\dagger }_4+y^2 (1-z)^2 \:T^{\dagger
   }_5-y^2 (1-z) z \:T^{\dagger }_6\Big]\,, &&\\
   %
\mathcal{P}_{\gamma_2} &=\frac{1}{\mathcal{D}}\Big[+(d-4) y (z-y (1-y-z)) \:T^{\dagger }_1+y \left(2 y^2+(d-4) (1-z) z+y (-4+d-(d-2) z)\right) \:T^{\dagger }_2\nonumber &&\\
&\phantom{=\frac{1}{\mathcal{D}}\Big[\:}-y^2
   (1-y-z) (y+z) \:T^{\dagger }_3-y^2 (z-y (1-y-z)) \:T^{\dagger }_4+y^2 \left(y-(1-y) z+z^2\right) \:T^{\dagger }_5-y^2 z (y+z) \:T^{\dagger }_6\Big]\,, &&\\
   %
\mathcal{P}_{\gamma_3} &=\frac{1}{\mathcal{D}}\Big[+4 y^2 z (1-y-z)
   \:T^{\dagger }_2-2 y^2 z (1-y-z)^2 \:T^{\dagger }_3-2 y^2 z^2 (1-y-z) \:T^{\dagger }_6\Big]\,.
\end{flalign}
\end{document}