\section{Introduction}\label{sec:intro}
Scattering amplitudes provide the connection between the Lagrangian formulation of the Standard Model of particle physics and observable quantities at particle collider experiments. 
Their perturbation theory expansion amounts to the computation of loop corrections, enabling increasingly accurate predictions.  
Multi-loop amplitudes in QCD provide moreover an 
important gateway for new mathematical ideas to enter particle physics. 

The helicity amplitudes for processes involving a vector boson and three partons have been of pivotal
importance to QCD precision physics. In their different 
kinematical crossings, these amplitudes describe 
three-jet production in $e^+e^-$ 
annihilation~\cite{Ellis:1976uc}, (2+1)-jet production in deep inelastic $ep$ scattering, and $V$+jet production at hadron colliders. Measurements of all 
three types of processes allowed establishing
QCD as the correct theory of the strong interactions. Precision data on these benchmark reactions continue to play a key role in enabling accurate determinations of the QCD coupling constant and of the partonic content of the proton. 

On the theoretical side, these processes are also 
a testing ground for the development of new 
calculational methods in perturbative QCD. $e^+e^- \to 3$~jets was the first jet production 
process to be computed to 
next-to-leading order (NLO,~\cite{Ellis:1980wv}) and next-to-next-to-leading order (NNLO,~\cite{Gehrmann-DeRidder:2007vsv,Weinzierl:2008iv}), 
thereby driving the development of 
 systematic formulations of both subtraction and phase-space slicing methods for handling infrared singularities and 
 their implementation in 
 flexible parton-level event generation programs~\cite{Kunszt:1989km,Giele:1991vf,Catani:1996jh,Gehrmann-DeRidder:2014hxk}.  
It is likely that $e^+e^- \to$ 3 jets will be among the first processes with jets to be tackled at
third order (N$^3$LO) in perturbative QCD.

At the LHC, leptonic decays of produced vector bosons leave clear signatures in the detectors, allowing observables such as the 
transverse momentum distribution of the lepton pair to be measured to a precision well below the percent level. 
To match this level of accuracy requires going beyond
the currently available NNLO predictions~\cite{Boughezal:2015dva,Gehrmann-DeRidder:2015wbt,Neumann:2022lft} for 
vector-boson-plus-jet production, which
relates directly 
to the vector boson transverse momentum spectrum. 

The Born-level helicity amplitudes for these processes 
couple the vector boson to a quark-antiquark pair and a gluon: $Vq\bar qg$. QCD corrections to these amplitudes were computed previously up to one loop~\cite{Giele:1991vf} and two loops~\cite{Garland:2002ak}, recently extended to
include singlet axial vector coupling 
terms~\cite{Gehrmann:2022vuk} and higher orders 
in the dimensional regulator~\cite{Gehrmann:2023zpz}. 

The extension of scattering amplitudes
to the next loop order is always associated with conceptual advances. More than twenty years ago, the calculation of the two-loop master integrals for four-point functions with one off-shell leg~\cite{Gehrmann:2000zt, Gehrmann:2001ck} served as the first advanced example of the use of the method of differential equations~\cite{Kotikov:1990kg, Gehrmann:1999as}, now ubiquitous in multi-loop calculations. It also necessitated the study of a new class of functions, generalized harmonic polylogarithms (multiple polylogarithms, MPLs)~\cite{Gehrmann:2000zt,Goncharov:1998kja}. 

The concept of a canonical basis of master integrals~\cite{ArkaniHamed:2010gh,Henn:2013nsa} has facilitated the extension of the two-loop $Vq\bar qg$
master integrals to transcendental weight six~\cite{Gehrmann:2023etk}. At three loops, the ladder-box topology was considered some time ago~\cite{DiVita:2014pza}, while the more challenging tennis court topologies have only been addressed more recently~\cite{Canko:2021xmn, Canko:2023yoe}. Interestingly, all planar integrals can be expressed in terms of MPLs with the same alphabet as at two loops. On the other hand, the first results for non-planar topologies~\cite{Henn:2023vbd, Canko:2023yoe} indicate an extended alphabet 
with the appearance of square roots. 
The increasing complexity at intermediate stages of amplitude calculations has also led to the adoption of methods based on finite-field arithmetic in particle physics~\cite{vonManteuffel:2014ixa,Peraro:2019svx}, allowing the reconstruction of analytic results for the amplitudes from 
multiple numerical evaluations. 

This combination of conceptual and technical developments enables us to 
derive the planar three-loop QCD $Vq\bar qg$ helicity amplitudes,  which we present in this letter. These amplitudes  constitute the  virtual N$^3$LO QCD corrections to $e^+e^- \to$ 3 jets and processes related to it by kinematical crossing, and are the stepping stone to reach a new level of precision in QCD studies at colliders. 



