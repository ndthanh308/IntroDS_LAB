\section{UV renormalization and IR subtraction}\label{sec:UVandIR}
All UV divergences are removed by expressing the helicity coefficients in terms of the
$\overline{\rm MS}$-renormalized strong coupling $\alpha_s=\alpha_s(\mu)$.
Since this is a standard procedure, we only summarize the necessary $\beta$-function coefficients in the supplemental material.
The remaining IR divergences are universal and can be extracted multiplicatively 
in the formalism of Soft-Collinear Effective Theory (SCET)~\cite{Becher:2009qa, Catani:1998bh, Dixon:2009ur,Gardi:2009qi} as
\begin{equation}
    \mathbf{\mathbf{\Phi}}_{\text{finite}}(\{p\}) = \lim_{\epsilon \rightarrow 0} \mathbfcal{Z}^{-1}(\{p\}; \epsilon) \,\, \mathbf{\mathbf{\Phi}}({p}, \epsilon)\,,
    \label{eqn: IR-Renormalization}
\end{equation}
where we indicate with $\mathbf{\mathbf{\Phi}}_{\text{finite}}(\{p\})$ the finite remainder of the amplitude $\mathbf{\mathbf{\Phi}}({p}, \epsilon)$. As indicated by
the bold font, 
$\mathbfcal Z$ is generally an operator in color space, but it diagonalizes when acting on the $Vq\bar q g$ amplitude coefficients, which factorize onto 
a single color structure,
\begin{equation}
    \mathbf{\Phi}={\mathbb T^a_{ij}}\,\Omega\,.
    \label{eq:colvec}
\end{equation} 
$\mathbfcal Z$ can be formally expressed in terms of the anomalous dimension matrix $\mathbf\Gamma$ as
\begin{align}\nonumber
    \mathbfcal{Z}(\epsilon, \{p\}, \mu) &= \mathbb{P} \,  \text{exp} \Big[  \int_{\mu}^{\infty} \frac{d\mu'}{\mu'} \mathbf{\Gamma}(\{p\}, \mu') \Big] \\
    &= \sum_{l = 0}^{\infty} \Big( \frac{\alpha_s}{2 \pi} \Big)^l \mathbfcal{Z}^{(l)} \,.
    \label{eqn: Z_operator}
\end{align}

Importantly, for the first time at three loops, the anomalous dimension operator is not entirely captured by a dipole term but receives contributions also from quadrupole color correlation operators,
\begin{equation}\label{eq:dipole_+_quadrupole}
\mathbf{\Gamma}(\{p\}, \mu)=  \mathbf{\Gamma}_\text{dip}(\{p\}, \mu)  + \mathbf{\Delta}_3(\{p\}, \mu)  \; .
\end{equation}
The former is due to the pairwise exchange of color charge between 
external legs and reads
\begin{align}
 \mathbf{\Gamma}_{\text{dip}}(\{p\},\mu) &= \sum_{1 \leq i < j \leq n} \mathbf{T}_i^a\mathbf{T}_j^a \, \gamma^{\text{K}}(\alpha_s) \log\Big(-\frac{\mu^2}{s_{ij} + i \eta }\Big)\nonumber\\
 &\phantom{\quad}+  \, \sum_{i=1}^n \gamma^i(\alpha_s)\,,
 \label{DipoleExpression}
\end{align}
where the $\gamma^{\text{K}}$ is the cusp anomalous dimension and $\gamma^i$ is the  anomalous dimension of parton $i$. 

The quadrupole contribution $\mathbf{\Delta}_3$ in
eq.~\eqref{eq:dipole_+_quadrupole} accounts for the exchange
of color charge among three or four soft gluons emitted from (a subset of) the partonic legs. Unlike in three-loop four-parton scattering~\cite{Henn:2016jdu,Caola:2021rqz,Caola:2021izf,Caola:2022dfa} where this term was observed in N=4 SYM and QCD amplitudes for the first time, soft gluon emissions from four external partons are impossible here and thus the quadrupole term consists only of the entirely kinematics-independent term~\cite{Almelid:2015jia}
\begin{multline} \label{eq:quadrupole}
\mathbf{\Delta}^{(3), ijk}_3 = -f_{abe} f_{cde}\,
\bigg[
16 \,C   \, \sum_{i=1}^3 \sum_{\substack{1\leq j < k \leq3 \\ j,k\neq i}}  \left\{ \mathbf{T}^a_i,\mathbf{T}^d_i \right\} \mathbf{T}^b_j \mathbf{T}^c_k 
\bigg]\,,
\end{multline}
with  $C = \zeta_5 + 2 \zeta_2 \zeta_3$ a constant. However, this term only enters the color layers that are suppressed by at least $N^{-2}$ relative to the 
leading color contributions considered here.


\begin{table*}[t]
\begin{tabular}{|c||c|c|c|c|}
\hline
{} &  $V\to gq\bar{q}$ & $q\bar{q}\to Vg$ & $qg\to Vq$ & $\bar{q}g\to V\bar{q}$  \\ \hline
 {} & $y=z=1/3$ & \multicolumn{3}{c|}{$u=1/7, v=3/5$}  \\ \hline\hline
$\alpha_1$ & $0.59426734 + 0.03099861 i$ & $ -0.49538117 +    0.65353908 i$ & $ -0.22197101 - 0.03494388 i$ & $ -0.22197101 -    0.03494388 i$ \\ \hline
$\alpha_2$ & $ 1.7192645 - 8.7893957 i$ & $   2.0901940 + 8.1000376 i$ & $ -1.6429881 + 0.5830600 i$ & $ -1.6429881 +    0.5830600 i$ \\ \hline
$\alpha_3$ & $ 1.2128210 + 5.9479256 i$ & $ -6.0948209 - 9.3383011 i$ & $   0.94229209 - 0.78134227 i$ & $   0.94229209 - 0.78134227 i$ \\ \hline
$\gamma_1$ & $ -0.59426734 - 0.03099861 i$ & $   0.46544666 - 1.05443048 i$ & $ -0.05750758 +    0.15110700 i$ & $ -0.05750758 + 0.15110700 i$ \\ \hline
$\gamma_2$ & $ -1.7192645 +    8.7893957 i$ & $ -1.0680269 - 9.5317293 i$ & $ 0.24066096 - 0.32042746 i$ & $   0.24066096 - 0.32042746 i$ \\ \hline
$\gamma_3$ & $ 1.2128210 + 5.9479256 i$ & $ -5.2865869 -    7.5133951 i$ & $ 0.18231060 + 0.07136413 i$ & $ 0.18231060 + 0.07136413 i$ \\ \hline
\end{tabular}
\caption{Numerical values for the finite remainders of the helicity coefficients at a phase-space point in each channel to 8 significant figures, setting $q^2=1$, $N=3$ and $N_f=5$. All values are to be multiplied by $10^4$.}
\label{tab: numerics}
\end{table*}


In summary, given the $l$-loop renormalized helicity amplitudes $\mathbf{\Phi}^{(l)}$, one can remove the IR singularities with
\begin{align}
\mathbf{\Phi}^{(0)}_{\text{finite}} & = \mathbf{\Phi}^{(0)}, \nonumber \\
\mathbf{\Phi}^{(1)}_{\text{finite}} & = \mathbf{\Phi}^{(1)} - \mathbf{I}^{(1)} \, \mathbf{\Phi}^{(0)} ,
%\label{eqn: perturbative IR subtraction one loop}\\
\nonumber \\
\mathbf{\Phi}^{(2)}_{\text{finite}} & = \mathbf{\Phi}^{(2)} - \mathbf{I}^{(2)} \, \mathbf{\Phi}^{(0)}  - \mathbf{I}^{(1)} \, \mathbf{\Phi}^{(1)}\,,
\nonumber \\
%\label{eqn: perturbative IR subtraction two loops}\\
\mathbf{\Phi}^{(3)}_{\text{finite}} & = \mathbf{\Phi}^{(3)} - \mathbf{I}^{(3)} \, \mathbf{\Phi}^{(0)}  - \mathbf{I}^{(2)} \, \mathbf{\Phi}^{(1)} - \mathbf{I}^{(1)} \, \mathbf{\Phi}^{(2)}\,.
\label{eqn: perturbative IR subtraction three loops}
\end{align}
The subtraction operators in \eqref{eqn: perturbative IR subtraction three loops} can be expressed as
\begin{align}
 \mathbf{I}^{(1)} & = \mathbfcal{Z}^{(1)}\,,\nonumber \\
 \mathbf{I}^{(2)} & = \mathbfcal{Z}^{(2)} - \big(\mathbfcal{Z}^{(1)}\big)^2\,,\nonumber \\
 \mathbf{I}^{(3)} & = \mathbfcal{Z}^{(3)} - 2\mathbfcal{Z}^{(2)}\mathbfcal{Z}^{(1)} + \big(\mathbfcal{Z}^{(1)}\big)^3\,.
 \label{eqn: IR scet}
\end{align}
After performing the color algebra in (\ref{eqn: perturbative IR subtraction three loops}), we recover the simple factorizing structure (\ref{eq:colvec}) for 
the finite parts of the helicity amplitude. Consequently, 
(\ref{eqn: perturbative IR subtraction three loops})
also holds for the amplitude coefficients $\Omega$. 

As explained in detail in~\cite{Gehrmann:2023etk}, the explicit form of the operators $\mathbfcal{Z}_l$ in terms of the universal constants in the perturbative expansion for $\gamma^K$, $\gamma^q$ and $\gamma^g$ is implied by the dipole formula \eqref{DipoleExpression} and the new term $\mathbf{\Delta}^{(3)}_3$. A summary of the relevant expressions for UV renormalization and IR subtraction is given in the supplemental material.


