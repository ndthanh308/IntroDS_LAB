\section{Setup and Calculation}\label{sec:setup}
We consider the production of a vector boson $V$ through lepton-antilepton annihilation and its subsequent decay into a quark, anti-quark and a gluon
\begin{equation}
l^{+}(p_5)+l^{-}(p_6) \to V(p_4) \rightarrow q(p_1) + \bar{q}(p_2) + g(p_3)\, .
\end{equation}
The calculation is performed assuming a purely vector coupling for $V$ and the effect of non-singlet axial couplings is reconstructed later by reweighting the
helicity amplitudes by the appropriate couplings, see eq.~\eqref{eq:HALpL}.
We introduce the usual Mandelstam invariants
\begin{align}\nonumber
s_{12} = (p_1+p_2)^2\,, \quad   s_{13} = (p_1+p_3)^2\,, \quad  s_{23} = (p_2+p_3)^2
\end{align}
which satisfy the conservation equation
\begin{equation}
s_{12} + s_{13} + s_{23} = q^2\,,
\label{eqn:mandelstamsum}
\end{equation}
where $q^2=p_4^2$ is the momentum-squared of the vector boson. It is convenient to work with the following dimensionless ratios
\begin{align}
    x = {s_{12}}/{q^2}\,, \quad \quad  y = {s_{13}}/{q^2}\,, \quad \quad z = {s_{23}}/{q^2}\,, 
    \label{eqn:xyz}
\end{align}
such that \eqref{eqn:mandelstamsum} implies $x+y+z=1$. It is easy to see that in the decay kinematic region all these invariants are non-negative and
\begin{equation}
 z \geq 0\,, \quad \quad  0 \leq  y  \leq 1 - z\,, \quad  \quad x = 1 - y -z\,. 
\end{equation}


We work in dimensional regularization, taking ${d = 4 -2 \epsilon}$. In particular, we adopt the 't Hooft-Veltman dimensional regularization scheme~\cite{tHooft:1972tcz}, in which loop momenta are $d-$dimensional but external states are kept in four dimensions. The amplitude for the production of a vector boson can then be written as 
\begin{align}
\mathcal{M}_{} &= -i\sqrt{4\pi\bar{\alpha}_s}\, {\mathbb T_{ij}^{a}} A_{ \mu \nu} \epsilon_3^{\mu} \epsilon_4^{\nu}\,,
\label{eq:Ampl}
\end{align}
where $\bar{\alpha}_s$ is the bare strong coupling and ${\mathbb T_{ij}^a}$ is the generator of $SU(N)$ in the fundamental representation.
We can then proceed by decomposing the amplitude $A^{\mu\nu}$ in eq.~\eqref{eq:Ampl} into a basis of tensor structures in $d=4$ spacetime dimensions following a method introduced in~\cite{Peraro:2019cjj,Peraro:2020sfm}, see ref.~\cite{Gehrmann:2023zpz} for details.

Instead of calculating the corresponding form factors, we prefer to start from this decomposition 
and introduce projector operators that directly extract the amplitude for fixed helicities of the external states and contracted with a leptonic current.
To this end, we define
\begin{equation}
    {\rm M}_{\lambda_{q_2}\lambda_3 \lambda_{l_5}} = \epsilon_{3,\mu}^{\lambda_3}   \, 
	A^{\mu \nu}_{\lambda_{\bar q_1}\lambda_{q_2} } C_{\lambda_{l_5}}^\nu(p_5,p_6)\,.
	 \label{eq:HAnocouplings}
\end{equation}
Working in spinor-helicity formalism, we write for the left- and right-handed fermionic currents 
\begin{align}\label{eq:leptonspinhel1}
    C_L^\mu(p,q) = [q \gamma^\mu p\rangle\,, \qquad C_R^\mu(p,q) = [p \gamma^\mu q\rangle\,, 
\end{align}
and for the polarization vector of the outgoing external gluon 
\begin{equation}\label{eq:gluonpol}
\epsilon_{3,-}^\mu = \frac{\langle 3 \gamma^\mu 2 ]}{\sqrt{2} [3 2]}\,, \quad 
\epsilon_{3,+}^\mu = \frac{\langle 2 \gamma^\mu 3 ]}{\sqrt{2} \langle 2 3 \rangle}\,,
\end{equation}
where we adopted an axial gauge with $p_2$ as the reference vector.
In this setup, the two independent helicity amplitudes read
\begin{alignat}{3}
    {\rm M}_{L+L} &= \frac{1}{\sqrt{2}}  
	 \Big[ \langle 1 2 \rangle [1 3]^2 \Big( \alpha_1 \langle 536] 
	&&+ \alpha_2 \langle 526] \Big)\nonumber\\
        &{}&&+ \alpha_3 \langle 25 \rangle [13] [36] \Big]\,,
\label{eq:mLLv}\\
	 {\rm M}_{L-L} &=  \frac{1}{\sqrt{2}}  
	 \Big[ \langle 23 \rangle^2 [1 2] \Big( \gamma_1 \langle 536] 
	&&+ \gamma_2 \langle 516] \Big)\nonumber\\        
        &{}&&+ \gamma_3 \langle 23 \rangle \langle 35 \rangle [16] \Big], \label{eq:LpLv}
\end{alignat}
and the remaining two helicity amplitudes, ${\rm M}_{R+L}$ and ${\rm M}_{R-L}$, can be obtained by means of parity and charge conjugation~\cite{Gehrmann:2022vuk,Gehrmann:2023zpz}.
Additionally, for each amplitude, the lepton current handedness can be reversed by a simple swap $p_5 \leftrightarrow p_6$.
The contribution of a Feynman diagram to the independent helicity coefficients $\alpha_i$ and $\gamma_i$ with $i=1,...,3$ can then be obtained by applying suitable projector operators. Their explicit form (immaterial to the present discussion) is given in the supplemental material.


The helicity amplitude coefficients 
$\Omega = (\alpha_i,\gamma_i)$ can be expanded in powers of the  strong coupling constant
\begin{align}
\Omega &= \Omega^{(0)} + \left( \frac{\alpha_s}{2 \pi}\right) \Omega^{(1)}
     + \left( \frac{{\alpha}_s}{2 \pi}\right)^2 \Omega^{(2)} \nonumber \\
    &+ \left( \frac{{\alpha}_s}{2 \pi}\right)^3 \Omega^{(3)} +\mathcal{O}({\alpha}_s^4) \,.   \label{eq:omegaexp}
\end{align}
This expansion holds for the bare coefficients (with $\alpha_s$ replaced by $\bar{\alpha}_s$), the renormalized coefficients, as well as for the IR-subtracted coefficients defined below. 
The three-loop coefficients 
are in turn decomposed into color factors as
\begin{align}\label{eq:colordecomp}
\Omega^{(3)}&=   
N^3\:\Omega^{(3)}_1
+ N \Omega^{(3)}_2
+ \frac{1}{N} \Omega^{(3)}_3
+ \frac{1}{N^{3}} \Omega^{(3)}_4\nonumber\\
&+ N_{f}  \:N^2 \Omega^{(3)}_5
+ N_{f} \Omega^{(3)}_6
+ \frac{N_{f}}{N^{2}} \Omega^{(3)}_7\nonumber\\
&+ N_{f}^2  N \:\Omega^{(3)}_8
+ \frac{N_{f}^2}{N} \Omega^{(3)}_9 
+ N_{f}^3 \Omega_{10}^{(3)}
\nonumber\\
&+ N_{f,V}  N^2 \Omega^{(3)}_{11}
+ N_{f,V} \Omega^{(3)}_{12}\nonumber\\
&+ \frac{N_{f,V}}{ N^{2}} \Omega^{(3)}_{13}
+ N_{f} N_{f,V}  N \Omega^{(3)}_{14}
+ \frac{N_{f} N_{f,V}}{ N} \Omega^{(3)}_{15}, %%  \nonumber\\& + N_{f}^2 N_{f,V} \Omega^{(3)}_{16}
\end{align}
where $N_f$ is the number of active quark flavors in the loops, and $N_{f, V}$ denotes contributions where the vector boson couples to an internal fermion loop. In this letter we focus on 
the dominant terms in this expansion. 
Using $N_f \sim N$, these are all the terms with total power of three in $N$ and $N_f$
and they correspond to $\Omega_j$ with $j=1,5,8,10$.
Importantly, the leading color factors only receive contribution from planar diagrams, see Figure~\ref{fig:representative_diags} for example diagrams. 
The contributions proportional to $N_f^2 N_{f,V}$ vanish for a vector-like coupling by Furry's theorem. 
Eq.~\eqref{eq:colordecomp} is also valid for any helicity amplitude coefficient -- bare, renormalized or IR subtracted. 

% Figure environment removed

In the following, we fix the renormalization scale in 
$\Omega$ as $\mu^2=q^2$.
The full scale dependence can then be recovered through
\begin{align}
     \Omega^{(3)}(\mu) & = 
     \left(\frac{5}{16} \beta _0^3 L(\mu)^3+\beta _0 \beta _1 L(\mu)^2+\frac{1}{2} \beta _2 L(\mu)\right)\Omega^{(0)}\nonumber\\
    &+  \left(\frac{15}{8} \beta _0^2 L(\mu)^2+\frac{3}{2} \beta _1 L(\mu)\right) \Omega^{(1)}\nonumber\\
   &     +\frac{5}{2} \beta _0 L(\mu) \Omega ^{(2)} + \Omega ^{(3)} 
\end{align}
with $L(\mu) = \log \left(\mu ^2/q^2\right)$. 


The helicity amplitudes for the decay of a Standard Model vector boson $V$ can finally be related to the helicity amplitudes obtained above by dressing with the appropriate electroweak couplings
\begin{align}
\mathcal{M}_{\lambda_{q_2}\lambda_3 \lambda_{l_5}}^V =& - 
 \frac{i \sqrt{4 \pi \alpha_s} (4 \pi \alpha)\, L_{l_5 l_6}^{V} L_{q_1 q_2}^V }{D(p_{56}^2,m_V^2)} \nonumber \\ 
 &\times
 {\mathbb T^{a}_{ij}} \,{\rm M}_{\lambda_{q_2}\lambda_3 \lambda_{l_5}} \,, \label{eq:HALpL}
 \end{align}
where $p_{56} = p_5 + p_6$, the vector boson propagator reads
\begin{align}
D\left(q^2,m_V^2\right) &= q^2 - m_V^2 + i \Gamma_V m_V
\end{align}
and the couplings for the bosons ${V=Z,W^\pm,\gamma^*}$ are
\begin{align}
    R_{f_1 f_2}^\gamma = L_{f_1 f_2}^\gamma &= - e_{f_1} \delta_{f_1 f_2}\,,\\
    L_{f_1 f_2}^Z &= \frac{I_3^{f_1} - \sin^2{\theta_w} e_{f_1}}{\sin{\theta_w} \cos{\theta_w}} \delta_{f_1 f_2} \,, \\
    L_{f_1 f_2}^W &= \frac{\epsilon_{f_1,f_2}}{\sqrt{2}\sin{\theta_w}}  \,,\\
    R_{f_1 f_2}^Z &=-\frac{\sin{\theta_w} e_{f_1}}{\cos{\theta_w}} \delta_{f_1 f_2}\,,\\
    R_{f_1 f_2}^W &=0  \,.
\end{align}  
In the formulas above, $\alpha$ is the electroweak coupling constant,
$\theta_w$ is the Weinberg angle, $I_3 = \pm 1/2$ is the third component
of the weak isospin and the charges $e_i$
are measured in terms of the fundamental electric charge $e>0$.
Moreover, $\epsilon_{f_1,f_2}=1$ if $f_1 \neq f_2$ but belonging to the same
isospin doublet, and zero otherwise. 

In order to compute the (unrenormalized) corrections to the helicity amplitude coefficient, we use the same unified workflow as for the tree-level, one- and two-loop amplitudes for $Vq\bar{q}g$~\cite{Gehrmann:2023zpz}, whose agreement with older results in the literature up to the finite part in $\epsilon$ provides an additional check on our method. In summary, the relevant three-loop diagrams are  generated using \code{QGRAF}~\cite{Nogueira:1991ex} and every manipulation including insertion of Feynman rules, evaluation of  Dirac and Lorentz algebra and application of the projectors are performed in \code{FORM}~\cite{Vermaseren:2000nd}. 
Once the helicity projectors have been applied, all Feynman diagrams are expressed in terms of scalar integrals, which can be written in terms of a single planar auxiliary topology of the form
\begin{equation}
\mathcal{I}_{n_1,...,n_{15}} = e^{3 \gamma_E \epsilon} \int \prod_{i=1}^3 \frac{d^d k_i}{i \pi^{d/2}} \frac{1}{D_1^{n_1}... D_{15}^{n_{15}}}
\end{equation}
with $\gamma_E=0.5772\ldots$ the Euler constant and propagators
\begin{align*}
\begin{tabular}{l@{\hskip 0.2in}l@{\hskip 0.2in}l}
$D_1 = k_1 $    & $D_6 = k_3 - p_1$               & $D_{11} = k_2 - p_{123}$ \\
$D_2 =  k_2$    & $D_7 = k_1 - p_{12}$          & $D_{12 }= k_3 - p_{123}$  \\
$D_3 = k_3 $    & $D_8 = k_2 - p_{12} $         & $D_{13} = k_1 - k_2 $           \\
$D_4 = k_1-p_1$ & $D_9 = k_3 - p_{12} $         & $D_{14} = k_1 - k_3$            \\
$D_5 = k_2-p_1$ & $D_{10} = k_1 - p_{123}$ & $D_{15} = k_2 - k_3$          
\end{tabular}
\end{align*}
with $p_{ij(k)} = p_i + p_j (+p_k)$.
The integrals can be reduced to a set of master integrals using integration-by-parts (IBP)
identities~\cite{Tkachov:1981wb,Chetyrkin:1981qh}. For the actual reduction, 
we use the implementation of the Laporta
algorithm~\cite{Laporta:2001dd} in the automated code \code{Kira2}~\cite{Maierhoefer:2017hyi,Klappert:2020nbg} and express all integrals
directly in terms of the canonical basis for the three-loop planar family defined in~\cite{Canko:2023yoe}.
Here it was shown that, in line with the one- and two-loop results, the three-loop planar integrals can be evaluated to arbitrary orders in the dimensional regularization
parameter $\epsilon$ in terms of multiple polylogarithms (MPLs)~\cite{Goncharov:1998kja,Remiddi:1999ew,Gehrmann:2000zt,Vollinga:2004sn}
with alphabet
$\{y,z,1-y,1-z,y+z,1-y-z\}$.

The amplitude before reduction can be expressed in terms of 95625 scalar integrals, which in turn are reduced to 291 canonical basis elements and their crossings. 
The size and complexity of intermediate expressions makes the use of traditional methods for symbolic insertion of the IBP reduction into the unreduced amplitude highly non-trivial. 
Therefore, 
 in view of the expected increase in complexity of the subleading layers in the color expansion~\eqref{eq:colordecomp}, 
 we also devised a hybrid method involving finite field reconstruction, in parallel to a standard fully analytic approach. 
 
 In particular, in the standard approach, we produced the IBP identities with~\code{Kira2} and used~\code{Mathematica} 
 and~\code{Fermat} to include them into the unreduced amplitude and simplify the resulting coefficients. We then expanded the master integrals in $\epsilon$ and obtained the final expression for the amplitude in terms of MPLs.
 In the hybrid approach, we used the same IBP reduction obtained with \code{Kira2}. However, we then performed an insertion of the reduction and MI solutions in the amplitude for a handful of rational numerical values of the variables $y$ and $z$ in the coefficients of the MPLs. This step allowed us to determine which polylogarithms appear in the final expression for a given helicity coefficient. With this information, we proceeded to reconstruct the rational prefactors in $y$ and $z$ only for the relevant polylogarithms, leveraging the implementation provided by \code{FiniteFlow}~\cite{Peraro:2019svx}. This approach lends itself to parallelization and gives us the possibility to recycle the symbolic reductions in future calculations. Compared to the traditional approach of reconstructing only the finite part, it allows us to present also the renormalized amplitudes with explicit IR poles for greater versatility in phenomenological applications with different pole schemes. We verified that the reconstructed results exactly matched those obtained through the fully analytic approach.