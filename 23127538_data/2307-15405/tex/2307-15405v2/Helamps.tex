\section{Results}\label{sec:helamps}
In view of future phenomenological applications 
of the three-loop $Vq\bar qg$ amplitudes to LHC physics, we perform the analytic continuation of the $V$ decay amplitudes to the kinematic regions where the electroweak boson is produced in addition to a jet from the scattering of two partons: $ q g \rightarrow V q$, $\bar{q} g \rightarrow V \bar{q}$ and $q \bar{q} \rightarrow V g $.

The general strategy for performing the analytic continuation of the MPLs appearing in $2\to 2$ scattering involving four-point functions with one off-shell leg and massless propagators was outlined in  detail in~\cite{Gehrmann:2002zr} and adapted for the case of $Vq\bar{q}g$ in~\cite{Gehrmann:2023zpz}. 
In summary, we can access all the production helicity amplitudes by combinations of kinematic crossings of the external momenta, parity flips and the analytic continuation of MPLs to the region denoted as (3a) in~\cite{Gehrmann:2002zr} via the substitution
\begin{align}
y = {1}/{v}\,, \quad \quad z = -{u}/{v}\,.
\end{align}

The finite remainders of the independent helicity amplitudes in the decay and production regions constitute the main result of this work and are linked in electronic format to the \code{arXiv} submission of this letter.

%\subsection{Numerical results}\label{subsec:numerics}
The provided results are compact and can easily be evaluated numerically for phenomenological studies. To demonstrate this, in table \ref{tab: numerics} we present values for the helicity coefficients at the symmetric phase space point $y=z=1-y-z=1/3$ in decay kinematics (where the coefficients $\alpha$ and $\gamma$ take the same absolute value and where all imaginary parts are generated from the expansion of the overall factors $(-q^2)^{-\epsilon}$) and at a selected point in each of the production regimes.
The values were obtained using \code{GiNaC}~\cite{Vollinga:2004sn} as implemented in \code{PolyLogTools}~\cite{Duhr:2019tlz}.

We verified that in all helicity configurations, the symbol of the leading colour $Vgq\bar{q}$ amplitude satisfies the conjectured non-adjacency conditions~\cite{Dixon:2020bbt}. It was found in~\cite{Henn:2023vbd} that in the tennis-court topology, a single master integral with 8 propagators and its crossing have a symbol with $1-z$ and $1-x$ appearing next to each other. This violation propagates to several higher topologies during integration and the cancellation of the violating symbol in the amplitude is therefore non-trivial.