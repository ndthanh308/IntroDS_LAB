\section{Semi-structured Interview with Stakeholders}
\label{sec:als-interview}

To gain a better understanding of the potential usefulness of our autocalibration method for people with disabilities who use gaze typing, we conducted a semi-structured interview with stakeholders from the ALS community.

\subsection{Procedure}
We ran a single virtual semi-structured interview session with a group of ALS stakeholders, with IRB approval. The session consisted of (1) a demonstration of \systemName{} and (2) a guided group discussion. The group discussion centered around participants' relationship with the ALS community and gaze typing, barriers to gaze typing technologies for ALS users, resources to assist ALS users during gaze typing, and feedback on the demoed \systemName{} prototype. The discussion was transcribed for further analysis.

Two authors analyzed the transcript via deductive thematic coding \citep{braun2006using}. 
Three primary themes were derived from the semi-structured interview questions and any additional content covered during the interview. Each primary theme was subdivided into multiple secondary themes.
The transcript was categorized sentence-by-sentence into one or more secondary codes, if applicable. During the coding process, three authors met regularly to discuss the coding assignments. All authors discussed the findings of the thematic coding, which we summarize  below.

\subsection{Participants}

We recruited 7 participants from support organizations for ALS. 
Two of the participants were direct caregivers for family members who have/had ALS, three were members of organizations which support people with ALS, 
and one was a speech pathologist who treats ALS clients in the states of Washington, New Jersey, and Maryland in the United States. One of the participants has 20 years of experience in software and hardware development, with a focus on accessibility programming and building eye trackers. Another has 40 years of software development experience and got involved in the ALS association 18 years ago, building software for missing pieces of eye tracking to make them do things they weren't programmed or capable of doing at the time. In addition, we were joined by a participant with cerebral palsy who used eye trackers for gaze typing in her daily life. 

In the ALS community, caregivers are heavily relied on for setting up and using gaze tracking or other assistive devices. In particular, support staff who specialize in assistive devices and care for people with ALS have a wealth of experience from working with many individuals, and we aimed to learn from their broad experience with many people. 


\subsection{Resulting Themes}
We discuss our findings from the semi-structured interview categorized by the three identified primary themes and their subsequent secondary themes.

\subsubsection{Obstacles to Eye-Gaze Typing} 
Participants discussed various obstacles ALS users face in using eye tracker for gaze typing, which is an essential form of communication.

\begin{itemize}
    \item \textbf{Medications:} Participants noted that during the progression of ALS, people ingest stronger doses of medications such as muscle relaxants, anti-allergens, and opioids to manage symptoms and discomfort. Each of these three categories of medications creates problems for gaze-based interaction. Muscle relaxants make it harder to focus on a precise location on the screen; anti-allergens cause eye dryness which induces blinking, thereby diminishing eye tracking quality; and opioids cause pupil constrictions and dilations, making it harder to track the eyes. These issues 
    could also create autocalibration problems for users with progressed disease.
    
    
    \item \textbf{Eye Function:} Participants noted that glasses often diminish eye tracking quality as they reflect both external light sources and the screen itself. 
    Thus, many people with ALS do not wear glasses during gaze typing. Additionally, towards later stages of the disease, one side of the body often tightens, causing the head to drop and rotate to one side and making eye tracking more difficult. As a result, in later phases of the disease, only one eye will be used for tracking. Most modern eye trackers (including Tobii PCEye, used in our work) can reliably detect gaze by tracking only one eye.

    \item \textbf{Changing Positions:} Participants shared that people with ALS often move out of their chair or have their screens removed for medical care. Every time they return, they must recalibrate their eye tracker.  
    They suggested that autocalibration could benefit such users returning to their device after intermissions without having to recalibrate each time. One participant explained, \emph{``When tracking errors are position specific... an autocalibration algorithm that was the right one, that would be very useful and would operationalize eye trackers properly.''}
    
    \item \textbf{Calibration Process:} 
    Participants shared that ALS users often adapt and find workarounds for calibration drift, excelling at compensating for miscalibrations. One care worker explained, \emph{``Some of their brain muscle memory has already developed like they know how they can be so off and all of a sudden they shoot to the right character and get it. And it's because they've learned how to adapt to that.''}

    \item \textbf{Recalibration Frequency:} Participants explained that ALS users often recalibrate as much as 50 times a day, in an attempt to fix poor tracking. This causes frustration and can discourage use of the device at all. \emph{`` It's annoying to them and they need to calibrate again and again'; `..they'll need to have 20 calibrations in a 2 hour period. Well, there's no solution other than recalibrating. So like this would be amazing, them having the ability to have that autocalibration.''} 

    
\end{itemize}

\subsubsection{Resources for Eye-Gaze Typing}
Participants discussed resources that help alleviate obstacles to gaze typing for people with ALS. 
\begin{itemize}
    \item \textbf{Human debugging:}
    Participants noted that gaze typing can cause fatigue and overwhelm ALS users new to the system. Keyboard layouts are modified over time with additional features or buttons to help them use the device for longer and make the learning curve easier.
    In debugging, feedback from calibration (e.g. patterns in mistakes across the calibration targets) is a tool that care givers use. 
    For example, such feedback can indicate that the device positioning is off, or that lighting from a window is interfering with tracking 
    \emph{``There are so many situations where you need someone to figure out what you see from the calibration''}. 
    \item \textbf{Education and training:} Participants noted that there is a lack of awareness and education 
    about using eye tracking technology effectively with ALS. For example, they noted that education about the impacts of medications could significantly advance eye tracking efficacy.
\end{itemize}

\subsubsection{Prototype Feedback}
Finally, participants shared their perspectives on the demoed prototype's potential, pitfalls and suggested future improvements for ALS users.
\begin{itemize}
    \item \textbf{Use Cases:} Participants found \systemName{} to be promising for ALS users, and were interested in ALS users having the opportunity to use the system. \emph{``We would definitely try it with our clients 100\%.''} In particular, they thought that \systemName's autocalibration could help alleviate the frustration around repeated calibration. \emph{``It would be amazing for them to have the ability for autocalibration.''} 
    They also recognized the potential for \systemName{} to reduce the learning curve for new users and improve communication during the progression of ALS. \emph{``The trackers work really well on some [people] and then they get towards the more progressed stage of the disease, towards the end of life, and then that's when a lot of the times things get complex with medications. So I'm just really happy to see anything that could help get somebody started on eye gaze, because  a lot of people start too late.''}; \emph{`'For someone who's starting off and they have a significant offset, this would be very, very significant.''}
    \item \textbf{Desired Improvements:} Participants noted that for people in later stages of ALS or people who are otherwise unable to dwell on a precise location, having larger keys areas and fonts could help increase the error tolerance of the algorithm and still prove helpful. 
    They suggested that specific keys could further enhance autocalibration -- for example, the spacebar could serve as an indicator that the previous word was typed correctly, or typing the next likely character successfully could be used as a point of calibration. They also suggested that alternate sensory feedback during calibration such as audio could increase transparency and enhance the user experience by letting the user know that the calibration is changing. \emph{``I think even being able to alert the user to what has changed specifically before and after autocalibration would be really important.''} 
    They also noted that logs or reports generated during autocalibration could be useful for caregivers debugging the system and for physicians diagnosing other medical conditions like cataracts or medication effects.  
    
    \item \textbf{Pitfalls:} Participants reiterated that people using medications which cause dry eyes or pupil dilations might not benefit as much from autocalibration because their main challenge with gaze typing is the inability to focus or the eyes not being detected by the eye trackers. They also mentioned that miscalibrations can differ across the screen, but our work only calibrated to a fixation point on the top of the screen.
    \item \textbf{Control to turn autocalibration on/off:} Participants recommended providing the option to turn autocalibration on or off, and suggested that  
    AI could recommend to the user when to turn it on. \emph{``The one thing that would concern me would be breaking things for people that aren't having a problem. You know, someone whose brain muscle memory has already developed [for compensating to miscalibration]. One suggestion would be the system detects things are not working great for them and an AI [tells] them that they're a little bit off. The system could suggest OK, you're not doing quite as well as you were before. Let's do a little bit of autocalibration to see if we can get you back on track?''}
\end{itemize}