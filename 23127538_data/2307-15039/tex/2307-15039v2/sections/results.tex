\subsection{Results}
\label{sec:results}


\subsubsection{Typing Efficiency}

We found that \systemName{} exhibited faster typing speeds (characters/minute) (Fig. \ref{fig:quant}(a); $p<0.05$), lower abort frequency (Fig. \ref{fig:quant}(b); $p<0.05$), and required fewer backspaces (Fig. \ref{fig:quant}(c); $p=0.53$) in comparison to the static control system. Differences between typing speed (\systemName{}: M=$22.73$, SE=$0.78$; Control: M=$20.31$, SE=$0.65$) and abort frequency (\systemName{}: M=$0.05$, SE=$0.05$; Control: M=$0.6$, SE=$0.19$) were statistically significant (Fig.~\ref{fig:quant}(a),(c)). 
When we consider the gaze data for \systemName{} after the first reading attempt (when the system gets the first opportunity to autocalibrate), differences for typing speed %(Fig. \ref{fig:quant}(d); 
(\systemName{}: M=$25.91$, SE=$0.98$; Control: M=$20.31$, SE=$0.65$; $p<0.001$) and number of backspaces used 
(\systemName{}: M=$9.13$, SE=$2.0$; Control: M=$13.19$, SE=$2.12$; $p=0.18$) were even larger. With this consideration, typing speed is  strikingly different between the two systems ($p<0.001$), hinting towards the effectiveness of \systemName{}'s autocalibration technique. 



% Figure environment removed

\subsubsection{Mental Workload}

Participants consistently rated \systemName{} more favorably than the control manual calibration in the NASA TLX survey, which assesses a system's mental workload in terms of mental demand, physical demand, temporal demand, performance, effort, and frustration. In Fig.~\ref{fig:nasa-survey}, we observe that along four of the six dimensions, differences between \systemName{} and the static control are statistically significant (p$<0.05$). Participants perceive reduced mental demand (\systemName{}: M=$3.68$, SE=$0.36$; Control: M=$4.95$, SE=$0.35$), improved performance (\systemName{}: M=$5.84$, SE=$0.23$; Control: M=$4.89$, SE=$0.36$), reduced effort (\systemName{}: M=$3.84$, SE=$0.39$; Control: M=$4.95$, SE=$0.32$), and reduced frustration (\systemName{}: M=$3.10$, SE=$0.35$; Control: M=$4.37$, SE=$0.36$).
These results highlight an improved and more seamless user experience with autocalibration via \systemName{} compared to a static calibration approach. Qualitative feedback (Sec.~\ref{sec:pref}) provides further support for these findings along the dimensions of mental comfort, performance, effort, and frustration.

We note that several participants asked us clarifying questions about the questions for physical demand and temporal demand. It is possible that the results for these two dimensions may reflect discrepancies in their interpretations.


% Figure environment removed


\subsubsection{Overall Preferences}
\label{sec:pref}
We asked participants ``Which system do you prefer to use as an end-user of this device?''. Their preferences between the systems were: \systemName{} 14 (73.68\%), control 3 (15.79\%), no preference 2 (10.53\%). Open feedback from participants, summarized below, sheds light on these preferences.


Participants who preferred \systemName{} cited increased comfort and ease. They explained, \emph{``I think [\systemName{}] is more comfortable. With [the control], I had to compensate for each word and found it harder to select the letter''}; \emph{``I prefer [\systemName{}], because it was on average easier and required less mental and physical load.''} Another noted the accuracy of the system, \emph{``As an end-user I prefer [\systemName{}] as it was more accurate for most of the sentences I typed. It was only frustrating for 2/5 sentences, as opposed to [the control] which was frustrating for most of the sentences.''}. Participants also appreciated the real-time updates, one user noting \emph{``[\systemName{}] adapted quickly to where I was looking''}. Others said, \emph{``[\systemName{}] seemed to adjust to my typing and improve as I performed the task whereas  [the control] was constantly bad (and while the consistency helped and I learned to adjust where the system faltered, I was frustrated I needed to apply additional effort to complete the task)''}; \emph{``In [\systemName{}], I would start off having to recognize the offset and type accordingly, but after a few characters it would adapt and then I could actually look at the intended character, so it got progressively easier. In [the control] interface, when it worked (1 case) it was very smooth, but for the other 4 cases I had to identify the offset and consistently use it to type the whole phrase.''} Some users also perceived that their own ability to type improved with \systemName{}, \emph{``When starting with [\systemName{}], I did notice I was better at using the input method (typing with eyes) as I used more sentences, irrespective of how difficult the given sentence was to type as I was more comfortable overall.''}. 


Since we purposefully introduced miscalibration to evaluate \systemName{}'s corrections, some users also noted the struggle with \systemName{} before the autocalibration started with the first reading attempt, even though overall they preferred it over the static control. \emph{``I found [the control] to be substantially more difficult to use. I felt significant strain while using it and felt noticeably tired after completing the five [control] tasks.
[\systemName{}] was not easy to use, and required significant focus to complete, but I did not feel as strained while using it.
Overall, I would describe [the control] as difficult to use and very uncomfortable, and [\systemName{}] as difficult to use but only somewhat uncomfortable''}. When asked further to explain their reasoning for preferring [\systemName{}], they noted, \emph{``I much preferred [\systemName{}]. I don't think I could have completed more than five sentences using [the control] without having to take a break. With [\systemName{}] I believe I could have done eight to ten sentences before needing a break. Even with a break, however, I do not think I could complete more than two five-sentence sessions using [the control] without needing to walk away from the computer for an extended (10+ minute) period of time. While answering the survey for [the control], I felt like I had a headache. I was worried that the headache would get worse during the next session of typing, but it did not. [\systemName{}] was not comfortable to use, but it did not cause me the sort of discomfort and mental strain I felt during and immediately after using [the control].''}.


Participants who did not prefer \systemName{} cited some pitfalls of our approach. While most users read the typed text, we noticed one user instead relied on activated keys turning red. If a user does not look at the text box to read the typed text, \systemName{} does not update the calibration. A few users also preferred the reliability of errors in the static calibration system, despite additional cognitive load.
As one participant explained, \emph{``The error or offset between where I gaze and the detected cursor seems constant in [the control]. In [\systemName{}], the error is more random.''} We note that since users were not aware of the autocalibration feature relying on their reading, this contributed to unpredictability for a minority of the users. Though the purpose of the study was to evaluate `implicit' autocalibration, we believe if users were made aware of the underlying autocalibration technique, that could further enhance their experience.



One participant summarized the tradeoff between systems: \emph{``It felt as if [\systemName{}] was adapting to the miscalibration in the eye tracker, whereas [the control] was not. The adaptability of [\systemName{}] has benefits and tradeoffs. It meant that [the control] was more predictable, whereas [\systemName{}] was less predictable. That said, there was perhaps less overall motor coordination effort involved in using [\systemName{}]''}. The same user noted, \emph{``I think that overall I would prefer [\systemName{}] in the long-term.''} 