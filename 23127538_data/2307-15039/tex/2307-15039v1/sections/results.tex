\section{Results}
\label{sec:results}

We compare performance of the autocalibration system (A: Learning Interface) and static calibration system (B: No Learning) through the metrics introduced in Sec.~\ref{sec:metrics}. We also report qualitative findings observed through survey (Sec.~\ref{sec:static} and Sec.~\ref{sec:learning} and interview responses (Sec.~\ref{sec:interview}).

\subsection{Typing Efficiency}

We found that the autocalibrated system exhibits faster typing speeds (characters/minute), lower abort frequency, and requires lower number of backspaces to be used in comparison to the static system (Fig. \ref{fig:quant}).
The difference in average typing speed (Autocalibration: M$=24.09$, SE$=0.84$; Static: M$=19.91$, SE$=0.75$; p$<0.001$) and average abort frequency (Autocalibration: M$=0.07$, SE$=0.06$, Static: M$=0.8$, SE$=0.23$, p$<0.001$) is strongly statistically significant, but the difference between the number of backspaces used is not (Autocalibration: M$=9.76$, SE$=2.33$, Static: M$=13.47$, SE$=2.63$, p$=0.33$). The latter effect may be explained by the fact that we did not instruct users to correct mistakes if they activated unintended keys, instead allowing them to complete the phrase by spelling it incorrectly if that was more comfortable. Several users noted that they found it difficult to use the backspace key when they typed characters incorrectly under miscalibration and preferred to continue typing incorrectly to complete the phrase typing session. 

% Figure environment removed

\subsection{Mental Workload}

In our NASA-TLX surveys, participants consistently rated our autocalibration system more favorably than the manual callibration: mental demand (Autocalibration: M$=3.93$, SE$=0.38$; Static: M$=5.33$, SE$=0.37$; p$<0.05$), physical demand (Autocalibration: M$=2.93$, SE$=0.39$; Static: M$=4.07$, SE$=0.43$; p$<0.05$), temporal demand (Autocalibration: M$=3.33$, SE$=0.29$; Static: M$=3.93$, SE$=0.42$; p$=0.16$), performance (Autocalibration: M$=5.87$, SE$=0.26$; Static: M$=4.8$, SE$=0.45$; p$=0.06$), effort (Autocalibration: M$=3.87$, SE$=0.43$; Static: M$=5.0$, SE$=0.34$; p$<0.05$), and frustration (Autocalibration: M$=3.00$, SE$=0.38$; Static: M$=4.53$, SE$=0.38$; p$<0.01$). These findings are an encouraging sign of the potential to enhance the user experience of gaze typing technologies via autocalibration. 


% Figure environment removed


\subsection{Overall Preferences}

We observed that 12 out of the 15 participants preferred the autocalibrated system, 1 participant preferred the static calibration system, and 2 participants had no preference (Fig.~\ref{fig:pref}). The latter two participant groups did not prefer the autocalibration system due to some pitfalls of our approach. While most users read the typed text, some users did not read at all and instead relied on the visual feedback from the interface (the key turns red when activated). Without looking at the text box to read the typed text, the autocalibrated system does not have an opportunity to update the calibration. Moreover, if the autocalibrated system is inaccurate in updating the gaze coordinates but corrects itself with more reading related signals, a user might prefer the reliability of errors in the static calibration system. Assuming other environmental factors to be constant, the miscalibration amount for a static system does not change in a session and thus users can learn to consistently compensate for it, albeit under cognitive strain.

% Figure environment removed


Specifically, one user reports that for the autocalibrated system, ``I would start off having to recognize the offset and type accordingly, but after a few characters it would adapt and then I could actually look at the intended character, so it got progressively easier''. Another user states ``I think [the autocalibrated system] is more comfortable. With [the manually calibrated system], I had to compensate for each word and found it harder to select the letter''. Some users also noted challenges of using [the autocalibrated system] which only autocalibrates when a user reads a typed text (particpants were unaware of this functinoality): ``Interface A was not easy to use, and required significant focus to complete, but I did not feel as strained while using it''; ``It felt as if [the autocalibrated system] was adapting to the miscalibration in the eye tracker, whereas [the manual calibration] was not. The adaptability of [the autocalibrated system] has benefits and tradeoffs. It meant that [the manual calibration] was more predictable, whereas [the autocalibrated system] was less predictable. That said, there was perhaps less overall motor coordination effort involved in using [the autocalibrated system]''.

We also asked participants why they preferred one system over the other, if at all. While one participant preferred the consistent miscalibration of the static system, several of them express the adaptability of the autocalibration system leads to a positive user experience: ``I would prefer [the manual calibration]. The error or offset between where I gaze and the detected cursor seems constant in [the manual calibration]. In [the autocalibrated system], the error is more random.''; ``[the autocalibrated system] was on average easier and required less mental and physical load''; ``[the autocalibrated system] adapted quickly to where I was looking''. Additionally, one participant shared their personal thoughts about the potential of gaze typing technology and the need to improve the accuracy of calibration for deployment in real-world applications: ``As someone who is open to alternatives means of typing due to physical restrictions, I like the potential of the technology but also acknowledge the potential strain for persons mentally and physically. It would be cool for persons typing on TVs though''. 

\begin{comment}
    
Q1. How would you compare the experience of gaze typing between the different interfaces in terms of ease and comfort? 

Users respond in the following ways:

``System B is harder for me, when I wanted to type some letter, I have to look at the border of that letter or even some other letter instead of the center of the target letter.''\\

``I think system A is more comfortable. With system B, I had to compensate for each word and found it harder to select the letter.''

``I found interface B to be substantially more difficult to use. I felt significant strain while using interface B and felt noticeably tired after completing the five interface B tasks. 

Interface A was not easy to use, and required significant focus to complete, but I did not feel as strained while using it. 

Overall I would describe interface B as difficult to use and very uncomfortable, and interface A as difficult to use but only somewhat uncomfortable.''

``It felt as if system a was adapting to the miscalibration in the eye tracker, whereas system b was not. The adaptability of system a has benefits and tradeoffs. It meant that system b was more predictable, whereas system a was less predictable. That said, there was perhaps less overall motor coordination effort involved in using system a.''

``A was much easier to use, it corrected itself quite quickly compared to B which never corrected itself and it was me who had to adjust in order to use B. If the sentence was long I felt that A was starting to loose its focus again but it didn't miss type I just had to adjust very slight to where I was looking.''

``In the first interface, I would start off having to recognize the offset and type accordingly, but after a few characters it would adapt and then I could actually look at the intended character, so it got progressively easier. In the second interface, when it worked (1 case) it was very smooth, but for the other 4 cases I had to identify the offset and consistently use it to type the whole phrase.''

``I didn't notice a difference. Maybe I felt better in the first system but that may be because I was more tired during the second one.''



Q. As an end-user, which interface do you prefer? Why?

``System A. Since the gaze tracker follows the letter I wanna type more easily.''

``System A. In comparison, system A feels easier to type and even if I started with compensation on the focused position, it can help me select on the letters along the way.''

``I much preferred interface A. 

I don't think I could have completed more than five sentences using interface B without having to take a break. With interface A I believe I could have done eight to ten sentences before needing a break. 

Even with a break, however, I do not think I complete more than two five sentence sessions using interface B without needing to walk away from the computer for an extended (10+ minute) period of time. While answering the post-interface B surveys, I felt like I had a headache. I was worried that the headache would get worse during the next session of typing, but it did not. Interface A was not comfortable to use, but it did not cause me the sort of discomfort and mental strain I felt during and immediately after using interface B.''

``I think that overall I would prefer system A in the long-term. There is a difficulty however with continuously adjusting to the system's own adaptations which requires somewhat more mental effort.''

``A, it adapted quickly to where I was looking''

``First interface, because it was on average easier and required less mental and physical load.''

``I felt equally with both systems. I don't have a preference.''

Q. Any other feedback?
``As someone who is open to alternatives means of typing due to physical restrictions, I like the potential of the technology but also acknowledge the potential strain for persons mentally and physically. It would be cool for persons typing on TVs though.''

``It seems if my gesture changes much during the test, we need to redo the calibration before we can continue typing. In reality, I don't think people will like doing calibration every time they sit in front of a computer. ''
\end{comment}

