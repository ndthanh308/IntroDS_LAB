
\section{Conclusion}
\label{sec:conclusion}

Eye tracking technology has enabled computer interactions for many users, in particular those with physical disabilities who cannot use traditional input devices. Gaze typing is a common application of eye tracking technology. However, the accuracy and efficiency of gaze typing can be significantly impacted by the calibration of the eye tracking system. Conventional calibration methods are time-consuming and require users to periodically perform manual calibrations, which can be inconvenient and disruptive to the user experience. We present a novel autocalibration technique for improving the user experience of gaze typing systems. Our approach dynamically compensates for miscalibration using the difference in users' gaze during typing (when the user may compensate for miscallibration) versus reading the text they have typed (when the user does not). Results from our user study suggest that by leveraging the natural gaze behavior of users during reading, our autocalibration system provides a more efficient, less mentally demanding, and overall preferable experience compared to traditional manual calibration.