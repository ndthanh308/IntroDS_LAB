\section{Conclusions}\label{sec:conclusions}
% added another layer of complexity to the scaling methodologies
% differences not large
% can advise EM design experts with promising points
In this paper, we instantiated a framework to optimize the size of electric motors in powertrains, by scaling already existing electric motor designs using high-fidelity tools.
We scale the motors in axial and radial direction, and add complexity by rescaling the sizes of the magnets and slots independently, giving the user more design freedom while maintaining a manageable number of design variables and tractability on system-level powertrain design optimization.
Our results show that, while the energy consumption reduction is fairly small, the framework does converge to a different optimum design w.r.t. the proportional scaling method.
This provides electric motor design engineers with multiple promising starting points for further design optimization.

In the full version of this paper, we aim to extend our experiments to optimize the motor design of a heavier cross-over type car, and to run on different drive cycles.
We also plan to focus on using our framework to validate the linearly stretching method described in Section~\ref{sec:introduction}, and the Willans line models~\cite{GuzzellaSciarretta2007}.
For future work, we would be interested in including integer design variables, such as the rewinding factors and the number of poles and slots.

