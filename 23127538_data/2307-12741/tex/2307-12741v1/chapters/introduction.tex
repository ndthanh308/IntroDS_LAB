\section{Introduction}\label{sec:introduction}
% - Large, multidisciplinary design space of electric vehicle powertrains: cheaper, longer range
% - Especially in electric motors
% - Scaling in optimization procedures
% - How to do it accurately but maintain an overview?
In order to accelerate the adoption of electric vehicles (EVs), their purchasing price and driving range should be improved.
One strategy to achieve this, is to advance the design of the powertrain holistically.
However, the powertrain system is highly multidisciplinary and interconnected, and contains many design variables.
Focusing specifically on electric motors (EMs), which are complex machines, renders it difficult to optimize the component-level design with a system-level perspective.

One widely used method to optimize the size of EMs in powertrain design, is by scaling the EM, which can be achieved in multiple ways.
Two options are named here: The measurement data can be scaled along the torque axis, or the geometrical dimensions of an already existing design are scaled.
Both of these methods are built on very strong assumptions.
This calls for methods that EMs more accurately, but retain simplicity and tractability on the system level.

\subsubsection*{Related literature}
% - Linear scaling methods
% - High level scaling
% - Detailed EM design
This paper addresses three closely connected research streams.
The first stream treats the optimization of (hybrid) EV powertrain sizing, such as in~\cite{BorsboomFahdzyanaEtAl2021,MurgovskiJohannessonEtAl2012,SilvasHofmanEtAl2016}.
In these cases, the sizing of the EM is carried out by linearly stretching the efficiency map along the torque axis.
However, this linear stretching method is considered to be valid only in small scalar ranges.

The second stream relates to the component-level design optimization of EMs, in order to, for instance, optimize total harmonic distortion or torque ripple \cite{LeiZhuEtAl2017,BramerdorferTapiaEtAl2018}.
The models that are used for this process are numerous in their design variables and finite-element methods, and thus viewed as fairly accurate, but this would be intractable for powertrain design purposes.

The final stream attempts to connect the first two streams: devising accurate EM models for powertrain design~\cite{RamakrishnanStipeticEtAl2018,StipeticGossEtAl2018,ClementeBorsboomEtAl2023}.
This is achieved by (re)scaling an already existing EM design in its geometrical dimensions, and then obtaining its performance by either simulating it with a higher fidelity tool, devising analytical scaling laws, or constructing surrogate models.
While all of these scaling methods are useful, they all consider proportional scaling of all design variables in the same direction with \textit{one} scaling factor.
This includes the internal design of the EM, which can be of great influence on the performance if considered separately.

Summarizing, to the best of the authors' knowledge, there are no methods for EM design optimization that consider internal design variables, while still retaining computational tractability on the powertrain system level.

% Figure environment removed%

\subsubsection*{Statement of Contributions}
% - New EM scaling method: combined scaling -> internal and external?
% - Bayesian optimization
% - Vehicle framework with results on different approaches
In order to address this challenge, this paper contributes to the field by presenting a framework that optimizes the gear ratio value and the design of the EM in an electric powertrain directly, using high-fidelity simulations and derivative-free solvers. 
To this end, we scale the EM in its main geometric dimensions---axially and radially---and include the rescaling of the internal design variables related to the magnets and slots.
The scaling procedure is encapsulated in a powertrain and vehicle model, and showcased by optimizing the EM design for a small city car on a drive cycle.

\subsubsection*{Organization}
This paper is organized as follows:
Section~\ref{sec:methodology} presents the optimization framework, with a strong focus on the EM scaling models, and poses the optimization problem.
Section~\ref{sec:results} displays the results of showcasing the presented framework on a drive cycle, and the conclusions are drawn in Section~\ref{sec:conclusions}.



