\section{Methodology}\label{sec:methodology}
We explain the optimization framework by introducing the vehicle and powertrain models, with emphasis on the scaling procedure of the EM (Section~\ref{subsec:longvehdyn} to \ref{subsec:em}).
Afterwards, we pose the optimization problem and solution approach in Section~\ref{subsec:optprob} and \ref{subsec:sol}, followed by a discussion in Section~\ref{subsec:discussion}.
For reasons of brevity, we will drop the notation of time dependency whenever it is clear from the context.

\subsection{Longitudinal Vehicle Dynamics}\label{subsec:longvehdyn}
We adopt a quasi-static modeling approach in this paper, as is common in powertrain design studies~\cite{GuzzellaSciarretta2007}.
Considering a flat road, we calculate the torque required at the wheels to complete the drive cycle $T_\mathrm{req}$ by
\begin{equation}
	T_\mathrm{req} = r_\mathrm{w} \cdot \left( m_\mathrm{v} \cdot a + \frac{1}{2}\cdot \rho_\mathrm{a} \cdot c_\mathrm{d} \cdot A_\mathrm{f} \cdot v^2 + c_\mathrm{r} \cdot m_\mathrm{v} \cdot g \right),
\end{equation}
where $v$ and $a$ are the drive cycle's velocity and acceleration, respectively, $r_\mathrm{w}$ is the radius of the wheels, $m_\mathrm{v}$ is the mass of the vehicle, $\rho_\mathrm{a}$ is the density of air, $c_\mathrm{d}$ is the aerodynamic drag coefficient, $A_\mathrm{f}$ is the frontal area of the vehicle, $c_\mathrm{r}$ is the rolling resistance coefficient, and $g$ is the gravitational constant.

\subsection{Transmission}\label{subsec:gb}
The one-speed transmission and final drive are modeled with a fixed efficiency $\eta_\mathrm{g}$ and gear ratio $\gamma$, which is subject to optimization within bounds $\underline{\gamma}$ and $\overline{\gamma}$.
Then, the torque at the EM shaft $T_\mathrm{m}$ can be calculated by
\begin{align}
	T_\mathrm{m} =
	\begin{cases}
		\frac{T_\mathrm{req}}{\gamma \cdot \eta_\mathrm{g}} \quad &\text{ if } T_\mathrm{req} \geq 0 \\
		\frac{T_\mathrm{req}\cdot \eta_\mathrm{g}}{\gamma} \quad &\text{ if } T_\mathrm{req} < 0,
	\end{cases} 
\end{align}
\obmargin{whereas}{regen braking fraction?} the EM speed $\omega_\mathrm{m}$ is given by
\begin{equation}
	\omega_\mathrm{m} = \gamma \cdot \frac{v}{r_\mathrm{w}}. 
\end{equation}
%In this paper, we consider both one- and two-speed transmissions.

\subsection{Electric Motor}\label{subsec:em}
In this paper, we take inspiration from the \obmargin{proportional}{uniform? external?} scaling procedure for recalculation in~\cite{StipeticGossEtAl2018} of brushless permanent-magnet machines, without taking rewinding into consideration.
The proportional scaling procedure is shown in Fig.~\ref{fig:scaling} and described by two scaling factors: $k_\mathrm{ax} \in [\underline{k}_\mathrm{ax}, \overline{k}_\mathrm{ax}]$ and $k_\mathrm{rad}\in [\underline{k}_\mathrm{rad}, \overline{k}_\mathrm{rad}]$, which scale all design variables uniformly in the axial and radial direction, respectively, within the boundaries $\underline{k}_{\{\cdot\}}$ and $\overline{k}_{\{\cdot\}}$.
We adopt the same assumptions from~\cite{StipeticGossEtAl2018}, meaning that the cross-section of a \obmargin{proportionally}{} scaled EM is identical with the unscaled version, and the magnetic and electric loading remain unchanged~\cite{HendershotMiller2010}.
%the current is adjusted with $k_\mathrm{rad}$ to match the magnetic fields of both versions, and the .

While all internal design parameters are altered with the proportional scaling factors, we add another layer of design freedom in our framework, where we allow the magnets and the slots to be rescaled separately in two dimensions each, on top of the proportional scaling factor.
In this way, we can change the magnetic and electric loading of the motor, which are the two primary elements that contribute to the torque per unit volume.
This can be viewed in Fig.~\ref{fig:scaling}, and the main equations are 
\begin{align}
	d_\mathrm{mw} &= d_\mathrm{mw,0} \cdot k_\mathrm{rad}\cdot k_\mathrm{mw}, \\
	d_\mathrm{ml} &= d_\mathrm{ml,0} \cdot k_\mathrm{rad}\cdot k_\mathrm{ml}, \\
	d_\mathrm{sd} &= d_\mathrm{sd,0} \cdot k_\mathrm{rad}\cdot k_\mathrm{sd}, \\
	d_\mathrm{tw} &= d_\mathrm{tw,0} \cdot k_\mathrm{rad}\cdot k_\mathrm{tw},
\end{align}
where $d_\mathrm{mw}$ and $d_\mathrm{ml}$ are the width and length of the magnets, respectively, $d_\mathrm{sd}$ is the depth of the slots, $d_\mathrm{tw}$ are the width of the teeth, $d_{\{\cdot\},0}$ are the dimensions of the unscaled motor, and all $k_{\{\cdot\}} \neq k_\mathrm{rad}$ are the individual rescaling factors of the internal design parameters with boundaries $\underline{k}_{\{\cdot\}}$ and $\overline{k}_{\{\cdot\}}$.

Each EM that is scaled proportionally and rescaled internally, a procedure that we will call \obmargin{\textit{combined scaling}}{??}, is simulated in Motor-CAD~\cite{MotorCAD}.
After the data on the operational limits and the losses $P_\mathrm{loss}$ are obtained, we calculate the power supplied to the motor $P_\mathrm{ac}$ as
\begin{equation}
	P_\mathrm{ac} =	T_\mathrm{m}\cdot \omega_\mathrm{m} + P_\mathrm{loss}\left(T_\mathrm{m}, \omega_\mathrm{m}\right).
\end{equation}
The energy used at the input of the EM $E_\mathrm{ac}$ is then computed by
\begin{equation}
	\frac{\mathrm{d}}{\mathrm{d}t} E_\mathrm{ac} = P_\mathrm{ac}.
\end{equation}

\subsection{Optimization Problem}\label{subsec:optprob}
The objective of the optimization problem in our framework is to minimize the used energy over the drive cycle $E_\mathrm{ac}(T)$, where $T$ is the duration of the cycle.
This is subject to the constraints on the design and input variables, and performance constraints.
The design variables $p$ are the scaling factors and the transmission ratio. 
The possible input variable $u$ is the selected gear.
The constraints on the performance comprise a desired maximum speed $v_\mathrm{max}$, acceleration time $t_\mathrm{acc}$ to reach velocity $v_\mathrm{acc}$, and gradeability on a slope $\alpha_{\mathrm{max}}$.

\begin{prob}[EM Design Problem]\label{prob:main}
	The optimal EM design is the solution of
	\begin{equation*}
		\begin{aligned}
			&\!\min & &E_\mathrm{ac}(T) \\
			& \textnormal{s.t. } & &\textnormal{Design Variable Constraints}\\
			& & &\textnormal{Input Variable Constraints}\\
			& & &\textnormal{Performance Constraints}.
		\end{aligned}
	\end{equation*}
\end{prob}

\subsection{Solution Approach}\label{subsec:sol}
The approach to obtain a design solution is displayed in Fig.~\ref{fig:solstrategy}.
Because each candidate EM design is simulated in Motor-CAD, which is black-box and time-consuming, Bayesian optimization is adopted as the optimizer in Matlab.
The interface between Matlab and Motor-CAD is facilitated by ActiveX scripting.
After the data is obtained of each candidate design, the vehicle and its powertrain are simulated over the drive cycle.
When the optimizer has found a solution, the procedure is terminated.

% Figure environment removed%

\subsection{Discussion}\label{subsec:discussion}
A few comments are in order.
First, we choose to only consider continuous optimization variables in this paper.
Our framework could be extended by including integer design variables, such as the rewinding factor~\cite{StipeticGossEtAl2018}, and the number of poles pairs and slots, at the price of a longer computational time.
Second, since we select Bayesian optimization as the solver in our framework, we do not have any global optimality guarantees and our framework is not deterministic.
However, if we select sufficient optimization iterations, we are confident that the solver will have converged and we are provided with a promising design, from which EM experts can move forward to tune the design further, to yield higher efficiency, torque density, etc.
\obmargin{}{Willans line}

% - no integer steps (rewinding, number of poles)
% - no validation
% - no comparison with linear stretching
% - no global optimality guarantees
% - scaling limits



