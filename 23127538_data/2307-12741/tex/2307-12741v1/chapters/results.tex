\section{Results}\label{sec:results}
% - ?
In this section, we present \obmargin{the preliminary results}{bit insecure} of applying our framework to find an optimal EM design.
We use a template interior permanent magnet motor design in Motor-CAD as our reference.
This template is based on the motor designed for the 2012 Nissan Leaf, which is considered a compact hatchback type car, and is rated at \unit[120]{kW} of maximum power.
We employ our framework to design a new EM for a small city car, which is lighter than the Nissan Leaf reference vehicle, with both the proportional scaling and the combined scaling methods, on the Worldwide harmonized Light-duty vehicle Test Cycle (WLTC).
The vehicle parameters and constraint values are given in Table~\ref{tab:params}.
We allow the Bayesian optimization solver to run 50 iterations, where a Motor-CAD calculation of a scaled design and a drive cycle simulation take about \unit[220]{s} to evaluate on an ASUS VivoBook 15 notebook with an Intel i7-10750H 2.6 GHz CPU and 16 GB of RAM.
The total computation time amounts to an average of \unit[2]{h} and \unit[45]{min} for the proportional case (3 optimization variables) and \unit[3]{h} and \unit[45]{min} for the combined case (7 optimization variables).
%\obmargin{Settings, computation times, platforms}{}

\begin{small}
	\begin{table}
		\centering 
		\caption{Parameters} 
		\label{tab:params} 
		\begin{tabular}{c c c | c c c } 
			\toprule
			\bfseries Parameter & \bfseries Value & \bfseries Unit & \bfseries Parameter & \bfseries Value & \bfseries Unit\\ 
			\midrule 
			$m_\mathrm{v}$ & 1085 & \unit{kg}&					$r_\mathrm{w}$ & 0.295 & \unit{m} \\
			$A_\mathrm{f}$ & 0.72 & \unit{m$^2$} &				$c_\mathrm{d}$ & 0.35 & \unit{1} \\
			$g$ & 9.81 & \unit{m/s$^2$} &						$\rho$ & 1.2 & \unit{kg/m$^3$} \\		
			$\underline{k}_{\{\mathrm{ax,rad}\}}$ & 0.8 & \unit{-} & 		$\overline{k}_{\{\mathrm{ax,rad}\}}$ & 1.2 & \unit{-} \\
			$\underline{k}_{\{\mathrm{mw,ml,sd,tw}\}}$ & 0.9 & \unit{-} & 	$\overline{k}_{\{\mathrm{mw,ml,sd,tw}\}}$ & 1.1 & \unit{-} \\
			$\underline{\gamma}$ & 1 & \unit{-}& 				$\overline{\gamma}$ & 10 & \unit{-} \\
			$\eta_\mathrm{g}$ & 95 & \unit{\%}&					$r_\mathrm{w}$ & 0.295 & \unit{m} \\
			$v_\mathrm{max}$ & 180 & \unit{km/h} &				$v_{\mathrm{acc}}$ & 100 & \unit{km/h} \\
			$t_\mathrm{acc}$ & 9.6 & \unit{s} &					$\alpha_{\mathrm{max}}$ & 20 & \unit{\%} \\
			\bottomrule 
		\end{tabular} 
	\end{table} 
\end{small}

\subsection{Numerical Results}
The optimal designs that are obtained by our framework are summarized in Table~\ref{tab:solutions}.
For clarification, in the proportional case, the scaling factors related to the magnets and slots are not free variables in the optimization and are all fixed to have a value of 1.
Therefore they are excluded from this column in the table. 
As can be observed, the energy consumption reduction is fairly small: a difference of \unit[-0.13]{\%}.
However, the design solutions are significantly different, especially in the radial direction.
Moreover, in the combined scaling case, the slot and magnet sizes are wider but shorter than in the proportional one (note that a smaller $k_\mathrm{tw}$ refers to narrower teeth, and thus a wider slot).

\begin{small}
	\begin{table}
		\centering 
		\caption{Design Solutions} 
		\label{tab:solutions} 
		\begin{tabular}{c | c c } 
			\toprule
			\bfseries Solution & \bfseries Combined & \bfseries Proportional \\
			\midrule
			\bfseries $k_\mathrm{ax}$ & 0.81 & 0.80 \\
			\bfseries $k_\mathrm{rad}$ & 1.10 & 1.03 \\
			\bfseries $\gamma$ & 5.44 & 5.55 \\
			\bfseries $k_\mathrm{mw}$ & 1.05 & - \\
			\bfseries $k_\mathrm{ml}$ & 0.92 & - \\
			\bfseries $k_\mathrm{sd}$ & 0.95 & - \\
			\bfseries $k_\mathrm{tw}$ & 0.90 & - \\
			\bfseries $E_\mathrm{ac}(T)$ & \unit[7.81]{MJ} & \unit[7.82]{MJ} \\ 
			\bottomrule 
		\end{tabular} 
	\end{table} 
\end{small}


