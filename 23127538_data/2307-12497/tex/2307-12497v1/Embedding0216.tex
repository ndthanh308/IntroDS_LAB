% This is samplepaper.tex, a sample chapter demonstrating the
% LLNCS macro package for Springer Computer Science proceedings;
% Version 2.21 of 2022/01/12
%
\documentclass[runningheads]{llncs}
%
\usepackage[T1]{fontenc}
% T1 fonts will be used to generate the final print and online PDFs,
% so please use T1 fonts in your manuscript whenever possible.
% Other font encondings may result in incorrect characters.
%
\usepackage{graphicx}
% Used for displaying a sample figure. If possible, figure files should
% be included in EPS format.
%
% If you use the hyperref package, please uncomment the following two lines
% to display URLs in blue roman font according to Springer's eBook style:
%\usepackage{color}
%\renewcommand\UrlFont{\color{blue}\rmfamily}
\usepackage{amsmath}
\usepackage{amsfonts,amssymb}
\newtheorem{algorithm}[theorem]{Algorithm}


\usepackage{algorithm}
\usepackage{algorithmicx}
\usepackage{algpseudocode}
\renewcommand{\algorithmicrequire}{\textbf{Input:}}
\renewcommand{\algorithmicensure}{\textbf{Output:}}



\begin{document}
%
\title{A Coefficient-Embedding Ideal Lattice can be Embedded into Infinitely Many Polynomial Rings}
%
%\titlerunning{Abbreviated paper title}
% If the paper title is too long for the running head, you can set
% an abbreviated paper title here
%

\author{Yihang Cheng\inst{1,2} \and
Yanbin Pan\inst{1,2}\orcidID{0000-0002-5591-0234}}
%%
\authorrunning{Yihang Cheng \textit{et al.}}
%% First names are abbreviated in the running head.
%% If there are more than two authors, '\textit{et al.}' is used.
%%
\institute{Key Laboratory of Mathematics Mechanization,   Academy of Mathematics and Systems Science, Chinese Academy of Sciences \and School of Mathematical Sciences, University of Chinese Academy of Sciences, Beijing, China.  \\
\email{chengyihang15@mails.ucas.ac.cn} \\{panyanbin@amss.ac.cn}}
%
\maketitle              % typeset the header of the contribution
%
\begin{abstract}
Many lattice-based crypstosystems employ ideal lattices for high efficiency. However, the additional algebraic structure of ideal lattices usually makes us worry about the security, and it is widely believed that the algebraic structure will help us solve the hard problems in ideal lattices more efficiently. In this paper, we study the additional algebraic structure of ideal lattices further and find that a given ideal lattice in some fixed polynomial ring can be embedded  as  an ideal in infinitely many different polynomial rings. We explicitly present all these polynomial rings for any given ideal lattice. The interesting phenomenon tells us that a single ideal lattice may have  more abundant algebraic structures than we imagine, which will
impact the security of  corresponding crypstosystems. For example, it increases the difficulties to evaluate the security of crypstosystems based on ideal lattices, since it seems that we need consider all the polynomial rings that the given ideal lattices can be embedded into if we believe that the algebraic structure will contribute to solve the corresponding hard problem. It also inspires us a new method to solve the ideal lattice problems by embedding the given ideal lattice into another well-studied polynomial ring. As a by-product, we also introduce an efficient algorithm to identify if a given lattice is an ideal lattice or not.
\keywords{ideal lattice  \and coefficient embedding \and complexity.}
\end{abstract}
%
%
%
\section{Introduction}
The research on lattice-based cryptography was pioneered by Ajtai \cite{ref1} in 1996. He presented a family of one-way function  based on the Short Integer Solution (SIS) problem, which has the average-case hardness under the worst-case assumptions for some lattice problems. In 1997,  Ajtai and Dwork \cite{ref2} introduced a public-key cryptosystem, whose average-case security can be based on the worst-case hardness of the unique-Shortest Vector Problem.  In 2005,  Regev \cite{ref4} proposed another problem with average-case hardness,  the Learning with Errors problem (LWE), and also a  public-key encryption scheme based on LWE. Because of the average-case security, lattice-based cryptography has drawn considerable attentions from then on. 

Although there have been many cryptographic schemes based on LWE and SIS,   the main drawback of such schemes is their limited efficiency, due to its large key size and slow computations. Especially, as  the development of quantum computers, it becomes more urgent to design more practical lattice-based cryptosystems, since lattice-based cryptosystems are widely believed to be quantum-resistant.  
To improve the efficiency, additional algebraic structure is involved in the  lattice to construct more practical schemes. Among them,  ideal lattice plays an important role. 
     
In fact, as early as in 1998, Hoffstein, Pipher, and Silverman \cite{ref3} introduced a lattice-based public-key encryption scheme known as NTRU, whose security is related to the ideal in the ring $\mathbb{Z}[x]/(x^n-1)$. Due to the cyclic structure of the ideal lattice, the efficiency of NTRU is very high. Later, in 2010, Lyubashevsky, Peikert and Regev \cite{ref5} presented a ring-based variant of LWE, called Ring-LWE, whose average-case  hardness is based on worst-case assumptions on ideal lattices. In 2017, Peikert, Regev and Stephens-Davidowitz \cite{ref6} refined the proof of the security of Ring-LWE for more algebraic number field. After the introduction of Ring-LWE, more and more practical cryptosystems based on ideal lattices have be constructed.

  
     There are two different ways to define ideal lattices. 
     
     One is induced by the coefficient embedding from ring $\mathbb{Z}[x]/f(x)$ into $\mathbb{Z}^n$. NTRU uses coefficient embedding to define its lattice. It is very convenient to implement cryptosystems based on  Ring-LWE with the coefficient embedding. In fact, almost all the ideal lattice-based cryptosystems are implemented via the coefficient embedding. However, it seems not easy to clarify the hardness of  problems for the coefficient-embedding ideal lattice in general. 
      
      
     The other one is defined by the canonical embedding from the algebraic integer ring of some number field $K$ into $\mathbb{C}^n$. This type of ideal lattice is usually employed in the security proof or hardness reduction in Ring-LWE based cryptography.
     
     It is widely believed that the additional algebraic structure of ideal lattice will help us solve its hard problems  more efficiently. 
     
     In 2016, Cramer, Ducas, Peikert and Regev \cite{ref7} introduced a polynomial-time quantum algorithm to solve $2^{\sqrt{n\text{log}n}}$-SVP in principal ideal lattices in the algebraic integer ring of $\mathbb{Q}(\zeta_m)$, where $m$ is a power of some prime. In 2017, Cramer, Ducas and Wesolowski \cite{ref8} extended the result to general ideals. In the same year,  Holzer, Wunderer and Buchmann \cite{ref9} extended the field to be $\mathbb{Q}(\zeta_m)$, where $m=p^aq^b$ and $p$, $q$ are different primes. 
     
     In 2019, Pellet-Mary, Hanrot and Stehl\'{e} \cite{ref10} introduced a pre-processing method (PHS algorithm) to solve $\gamma$-SVP for ideal lattices in any number field. The pre-processing phasing takes exponential time. Let $n$ be the dimension of the number field $K$ viewed as a $\mathbb{Q}$-vector space. Pellet-Mary \textit{et al.} showed that by performing pre-processing on $K$ in exponential time, their algorithm can, given any ideal lattice $I$ of $O_K$, for any $\alpha \in [0,1/2]$ output a $\exp(\widetilde{O}((n\log n)^{\alpha+1}/n))$ approximation of a shortest none-zero vector of $I$ in time $\exp(\widetilde{O}((n\log n)^{1-2\alpha}/n))+T$.  For the classical method, $T=\exp(\widetilde{O}((n\log n)^{1/2})$ if $K$ is  a cyclotomic field or $T=\exp(\widetilde{O}((n\log n)^{2/3})$ for an arbitrary number field $K$.
     
     In 2020, Bernard and Roux-Langlois \cite{ref19} proposed a new “twisted” version of the PHS  algorithm. They proved that Twisted-PHS algorithm performs at least as well as the original PHS algorithm and their algorithm suggested that much better approximation factors were achieved. In 2022, Bernard,  Lesavourey,  Nguyen and  Roux-Langlois \cite{ref20} extended the experiments of \cite{ref19} to cyclotomic fields of degree up to 210 for most conductors $m$. 
     
 In 2021, Pan, Xu, Wadleigh and Cheng \cite{ref21} found the connection between the complexity of the shortest vector problem (SVP) of prime ideals in number fields and their decomposition groups, and revealed lots of weak instances of  ideal lattices in which SVP can be solved efficiently. In 2022, Boudgoust, Gachon and Pellet-Mary \cite{ref22} generalized the work of Pan \textit{et al.} \cite{ref21} and provided a simple condition under which an ideal lattice defines an easy instance of the shortest vector problem. Namely, they showed that the more automorphisms stabilize the ideal, the easier it was to find a short vector in it.
 
 As mentioned above, almost all the research on SVP is in the canonical-embedding ideal lattices and the research on SVP in the coefficient-embedding ideal lattices is few.  In some rings, such as the algebraic integer rings of cyclotomic fields, the SVPs induced by the two different embeddings are connected with each other.  

In 2017, Baston \cite{ref11} discussed the norm connection between the coefficient embedding and the canonical embedding in the cyclotomic fields. Let $K=Q(\zeta_m)$ and for any ideal $I\in O_K$, let $T$ be the transformation matrix from the coefficient-embedding lattice $\mathcal{L}(B)$ to the canonical-embedding lattice $\mathcal{L}(B')$, which means $TB=B'$. Consider the singular value decomposition (SVD) of $T$ and \[T=U\begin{pmatrix}s_1&0&\cdots&0\\0&s_2&\cdots&0\\0&\vdots&\ddots&0\\0&0&\cdots&s_n\end{pmatrix}V\]$s_1\geq s_2\geq \cdots \geq s_n>0$, $V$ and $M$ unitary matrices. Define $k_2=\frac{s_1}{s_n}$ and he showed that the less $k_2$ is the more relevant of the SVPs are in the ideal lattices induced by two different embeddings of a fixed ideal. 	 

More specifically, according to lemma 3.5 of \cite{ref11}, given $T\in \mathbb{C}^{n\times n}$ and $x\in \mathbb{C}^n$, $s_1,s_2,\cdots,s_n$ is the singular value of $T$, then  $s_n(T)\cdot\|x\|_2\leq\|Tx\|_2\leq s_1(T)\cdot\|x\|_2$. This conclusion shows a direct reduction between the SVPs in two ideal lattices induced by different embeddings of a fixed ideal. Using the notion above, if there is an oracle to solve $\gamma$-SVP in $\mathcal{L}(B)$, then using lemma 3.5 of \cite{ref11} and the relation $TB=B'$ we can solve $k_2 \gamma$-SVP in $\mathcal{L}(B')$ in polynomial time. 

We recall that a number field $K$ is called monogenic if $O_K=\mathbb{Z}[\alpha]$ for some $\alpha \in K$. All the cyclotomic and quadratic fields are monogenic. Only in monogenic fields, $O_K$ is isomorphic to a polynomial ring $\mathbb{Z}[x]/f(x)$ for some monic irreducible integer polynomial $f(x)$ and only in this case the ideal lattices induced by the coefficient embedding have the same algebraic structure with the one induced by canonical embedding of the same ideal in the sense of ring isomorphism. 

In theorem 3.1 of \cite{ref11}, Baston showed that $k_2$ is only related to the monogenic number field $K$ and has nothing to do with the chosen ideal or fractional ideal. Though the research on $k_2$ in general monogenic number field is little, when $K=Q(\zeta_m)$ is a cyclotomic field, $k_2=(\text{rad}(m))^{1/2}$ for $m$ is odd and $k_2=(\text{rad}(m)/2)^{1/2}$ for $m$ is even, where rad(m) means the different prime multiple of m (see lemma 3.4 of \cite{ref11}).  Therefore when $m=a^l$ with $l$ large enough, the SVPs in ideal lattices induced by two embeddings of the same ideal are connected closely. When $m=2^l$ for any $l\geq2$, $k_2=1$ and the SVPs in two embedding ideal lattices are essentially the same.

With the discussion of the Baston's results above, in some monogenic number fields especially some cyclotomic fields, we can use the result of the canonical-embedding ideal lattices to handle the coefficient-embedding ideal lattices.


\subsubsection{Our contribution}
In this paper, we focus on the coefficient embedding. Our main contribution is to show that an ideal lattice in the ring $\mathbb{Z}[x]/f(x)$, where $f(x)$ is monic and $f(x)\in\mathbb{Z}[x]$, can be embedded into infinitely many rings $\mathbb{Z}[x]/g(x)$, where $g(x)$ is monic and $g(x)\in\mathbb{Z}[x]$ (\textbf{Theorem 1}). Besides, we show an efficient algorithm for computing all the rings that an ideal lattice can be embedded into and also judging whether a given integer lattice can be embedded into a polynomial ring (\textbf{Algorithm 1}).

As we all know, a lattice is actually a discrete additive subgroup of $\mathbb{R}^{m}$. The only difference between the general integer lattice and the ideal lattice is the multiplication structure of the ideal lattice. In fact, an integer lattice may be embedded into a polynomial ring $\mathbb{Z}[x]/f(x)$, and it can be viewed as an ideal of $\mathbb{Z}[x]/f(x)$.  Hence, with this embedding, the integer lattice as an ideal of  $\mathbb{Z}[x]/f(x)$ is equipped with the multiplication of the ring $\mathbb{Z}[x]/f(x)$. A natural question is that what will happen if we equip the same lattice with different "multiplication" or is the ''multiplication'' unique ? Obviously if this can be done, the lattice will not change, but the ring changes, which means that a fixed integer lattice may be viewed as different ideals in different rings.


We show that it is possible to embed a given ideal lattice  as another ideal into infinitely many different polynomial rings by the coefficient embedding. We explicitly present all the polynomial rings for any given ideal lattice.

It is widely believed that additional algebraic structure may lead a more efficient algorithm to solve the hard problems in ideal lattice than  general lattices, such as the method of recovering a short generator proposed by Cramer  \textit{et al.} \cite{ref7} and the method of pre-processing any number field $K$ proposed by Pellet-Mary \textit{et al.} \cite{ref10}. The researches above are all in the canonical-embedding ideal lattices.

The results of the SVP in canonical-embeddinng ideal lattices can also fit the case in coefficient-embedding ideal lattices in lots of rings. Though in a general monogenic number field the relation betweem SVPs in ideal lattices induced by two different embeddings of a fixed ideal is unclear, by the discussion of Baston's \cite{ref11} above in lots of special cyclotomic fields the connection between the SVPs in ideal lattices induced by two different embeddings of the same ideal is very close and solving the $\gamma$-SVP in one embedding means solving the $\beta$-SVP in another embedding, where $\gamma$ and $\beta$ are close. 

If we happen to find that the algebraic structure for  polynomial ring  $R$  that can help us  solve the ideal lattice problems in $R$ more efficiently, then our results shows that it's possible to solve the problem for other ideal lattices not in $R$ as long as the ideal lattices can be embedded as ideals into $R$. Similarly, when using the method in  \cite{ref10}, the pre-processing of   $R$ can also be used to solve the problems for some  ideal lattices not in $R$, which implies that  we may not need pre-process the new polynomial ring for every new ideal lattices.


Moreover, once we find a weak ideal lattice in which the lattice problem can be solved more efficiently,  we can solve the problems for infinite ideal lattices in different polynomial rings. Though the integer lattice is fixed, they are different as ideals in different polynomial rings. It seems that a weak ideal will spread as infinite weak ideals.


On the other hand, the abundant embedding relations will impact the security of the crypstosystems based on ideal lattices. When considering the security, it's necessary to evaluate all the corresponding ideals in the polynomial rings that the given ideal lattices can be embedded into instead of just one single ideal lattice.

We have to point out that all of the observations above shed a shadow on the security of ideal lattice-based cryptosystems. 

As a by-product, an efficient algorithm to identify an ideal lattice is introduced. We first show an equivalent condition between the integer lattice and the coefficient-embedding ideal lattice. According to this condition, we introduce a  polynomial-time algorithm that is more efficient than the algorithm proposed by Ding and Lindner \cite{ref13}. Moreover, we present an explicit form for all possible polynomial rings that the ideal lattice can be embedded into in theory instead of the implicit form obtained in the experiment as in \cite{ref13}. The explicit form can help us theoretically analyze the algebraic properties of there ideals directly.


%Hence, it's not proper anymore to judge the security of a crypstosystem based on ideal lattice by a single ring. And the embedding relation no doubts increases the difficulties of the security proof for any crypstosystem based on ideal lattices. Instead of relying on the worst case of the given ring, we must evaluate the whole rings class that the used ideal lattice can be embedded into.
%
%However, avoiding the use of certain well-studied rings doesn't means it's more secure, since we find that 

\subsubsection{Roadmap}The paper is organized as follows. In Section 2, some preliminaries are presented. In Section 3, we reveal the embedding relation in details and give some application scenes. In Section 4, an algorithm for identifying a coefficient-embedding ideal lattice is introduced together with the complexity analysis. In the final section, we give a brief conclusion.

\section{Preliminaries}

In this paper we denote by $\mathbb{C}$, $\mathbb{R}$, $\mathbb{Q}$ and $\mathbb{Z}$ the complex number field, the real number field, the rational number field and the integer ring respectively.

We denote a matrix by a  capital letter in bold and denote a vector by a lower-case letter in bold. To  represent the element of a matrix, we use the lower-case letter. For example,  the element of matrix $\mathbf{A}$ at the $i$-th row and $j$-th column is denoted by $a_{ij}$, while its $i$-th row is denoted by $\mathbf{a}_i$. Since we have the inner products in $\mathbb{R}^n$ and $\mathbb{C}^n$ respectively, we can define the norm of vectors,  that is, $ \Vert \mathbf{v} \Vert :=<\mathbf{v},\mathbf{v}>$ in $\mathbb{R}^n$ and  $ \Vert \mathbf{v} \Vert :=<\mathbf{v},\overline{\mathbf{v}}>$ in $\mathbb{C}^n$.

For two integers $a$ and $b$, $a| b$ means that $b$ is divisible by $a$. Otherwise, we write $a\not| \ b$. For integer $a$ and a matrix $\mathbf{A}$, $a| \mathbf{A}$ means that every entry of $\mathbf{A}$ can be divisible by $a$.
%Given a set $H$ of vectors, span($H$) means the space generated by the elements of $H$. For example, $\text{span}(\mathbf{A})=\{\sum_{i=1}^n x_i\mathbf{a}_i : x_i\in\mathbb{R}\}$. Denote by dim(span($H$)) the dimension of  span($H$).


%Denote by $\mathcal{B}(x,r)$ the ball centered at $x$ with radius $r$. 

For a polynomial $f(x)\in\mathbb{Z}[x]$, denote by $\mathbb{Z}[x]/f(x)$ for simplicity the quotient ring $\mathbb{Z}[x]/(f(x)\mathbb{Z}[x])$.

For a map $\sigma$, and a set $S$, denote by $\sigma(S)$ the set $\{\sigma(x):x\in S\}$.


 
 
 
\subsection{Lattice}
Lattices are  discrete subgroups of $ \mathbb{R}^m$, or equivalently,
\begin{definition}{(Lattice)}
Given n linearly independent vectors $\mathbf{B}=\begin{pmatrix}\mathbf{b}_1 \\ \mathbf{b}_2 \\\vdots \\ \mathbf{b}_n\end{pmatrix}$, where $\mathbf{b}_i\in\mathbb{R}^m$, the lattice $\mathcal{L}(\mathbf{B})$ generated by $\mathbf{B}$ is defined as  follows: \[\mathcal{L}(\mathbf{B})=\{\sum_{i=1}^n x_i\mathbf{b}_i : x_i\in\mathbb{Z}\}=\{\mathbf{xB} : \mathbf{x}\in\mathbb{Z}^n\}.\] 	
\end{definition}
We call $\mathbf{B}$ a basis of $\mathcal{L}(\mathbf{B})$, m and n the dimension and  rank of $\mathcal{L}(\mathbf{B})$ respectively. When $m=n$, we say $\mathcal{L}(\mathbf{B})$ is full-rank.

When $n\textgreater1$, there are infinitely many bases for a lattice $\mathbf{\mathcal{L}}$, and any two bases are related to each other by a unimodular matrix, which is an invertible integer matrix. More precisely, given a lattice $\mathcal{L}(\mathbf{B}_1)$, $\mathbf{B}_2$ is also a basis of the lattice if and only if there exists a unimodular matrix $\mathbf{U}$ s.t. $\mathbf{B}_1=\mathbf{U} \mathbf{B}_2$.

\paragraph{Hard problems in lattices}


The shortest vector problem (SVP)  is one  of the most famous hard problems in lattices.

SVP is the  question of finding a nonzero shortest  vector in a given lattice $\mathcal{L}$, whose length is denoted by $\lambda_1(\mathcal{L})$. 
The approximating-SVP with factor $\gamma$, denoted by $\gamma$-SVP, asks to find a short nonzero lattice vector $\mathbf{v}$ such that $$\|\mathbf{v}\|\le\gamma\cdot\lambda_1(\mathcal{L}).$$

In fact, The hardness of $\gamma$-SVP depends on $\gamma$. When $\gamma=1$, $\gamma$-SVP is exactly the original SVP, and for constant $\gamma$, this problem is known to be NP-hard under randomized reduction \cite{ajtai98}. 
Many cryptosystems are based on the hardness of (decision) $\gamma$-SVP when $\gamma$ is in polynomial size. By now we have not found any  polynomial-time classical algorithm to deal with such cases. The existing polynomial algorithms such as LLL \cite{ref14}, BKZ \cite{ref15} can find the situation when $\gamma =\text{exp}(n)$. 




 \subsection{Hermite Normal Form}
For the integer matrix, there is a very important standard form known as the Hermite Normal Form (HNF). For simplicity, we just present the definition of HNF for the non-singular integer matrix.
\begin{definition}{(Hermite Normal Form)} A non-singular matrix $\mathbf{H}\in\mathbb{Z}^{n \times n}$ is said to be in HNF, if
	\begin{itemize}
		\item $h_{i,i} \textgreater 0$ for $1\leq i\leq n.$
		\item $h_{j,i} =0 $ for $1\leq j \textless i \leq n.$
		\item $0 \leq h_{j,i}\textless h_{i,i}$ for $1\leq i\textless j\leq n.$
	\end{itemize}
\end{definition}


The Hermite Normal Form has some important properties. See \cite{DKT87,MW01,LP19} for more details. 
\begin{lemma}
For any  integer matrix $\mathbf{A}$, there exists a unimodular matrix $\mathbf{U}$ such that $\mathbf{H}$=$\mathbf{U} \mathbf{A}$ is in HNF. Moreover, HNF can be computed in polynomial time.
\end{lemma}

For integer lattices, we have 
\begin{lemma}\label{hnfbasis}
For any lattice $\mathcal{L}\subset \mathbb{Z}^n$, there exists a unique basis $\mathbf{H}$ in HNF. We call $\mathbf{H}$ the HNF basis of $\mathcal{L}$.
\end{lemma}


  Sometimes we do not need the whole HNF of an integer matrix. So we  introduce the Incomplete Hermite Normal Form of an integer matrix, which is also  a special basis of the integer lattice.
  
  \begin{definition}{(Incomplete Hermite Normal Form)} A non-singular matrix $\mathbf{B}\in\mathbb{Z}^{n \times n}$ is said to be in Incomplete Hermite Normal Form, if
  \begin{itemize}
  	\item $b_{n,n}>0$;
  	\item $b_{i,n} = 0 \mbox{ for } 1\leq i\leq n-1.$
  \end{itemize}
  \end{definition}

Given a full-rank integer matrix  $\mathbf{B}$,
\[ \mathbf{B}= \begin{pmatrix} b_{1,1}&b_{1,2}&\cdots&b_{1,n} \\ b_{2,1}&b_{2,2}&\cdots&b_{2,n} \\ \vdots&\vdots&\ddots&\vdots \\b_{n,1}&b_{n,2}&\cdots&b_{n,n} \end{pmatrix},\]
it is well known that by the Extended  Euclidean Algorithm we can find a unimodular matrix $\mathbf{U}$, such that 
\[ U \begin{pmatrix} b_{1,n} \\ b_{2,n} \\ \vdots \\b_{n,n} \end{pmatrix} = \begin{pmatrix}0 \\ 0 \\ \vdots \\d \end{pmatrix},\]
where $d = \gcd( b_{1,n}, b_{2,n},...,b_{n,n})$. Then we have 
$$ \mathbf{B}'= U \mathbf{B} =  \begin{pmatrix} \mathbf{D}&\mathbf{0} \\ \mathbf{b'}& d \end{pmatrix} $$
is in Incomplete Hermite Normal Form, where  $\mathbf{D} \in \mathbb{Z}^{(n-1) \times (n-1)}$, $\mathbf{b}' \in \mathbb{Z}^{n-1}$. 

About the Incomplete Hermite Normal Form, it is easy to conclude the following lemma. So we omit the proof.

\begin{lemma} \label{ihnflemma}
	For any non-singular matrix $\mathbf{B}\in\mathbb{Z}^{n \times n}$, $\mathbf{B}$  is said to be in Incomplete Hermite Normal Form, if
	\begin{itemize}
		\item we can find a unimodular matrix $\mathbf{U}$ in polynomial time, such that $ \mathbf{B}'= \mathbf{U} \mathbf{B}$ is in Incomplete Hermite Normal Form.
		\item For any unimodular matrix $\mathbf{U}$ and $\mathbf{V}$ such that $ \mathbf{B}'= \mathbf{U} \mathbf{B}$ and $ \mathbf{B}''= \mathbf{V} \mathbf{B}$ are in Incomplete Hermite Normal Form,  $ \mathbf{B}'$ and $ \mathbf{B}''$ are not necessarily equal, but $$b'_{n,n} = b''_{n,n} = \gcd( b_{1,n}, b_{2,n},...,b_{n,n}).$$  Specially, notice that the HNF  $\mathbf{H}$ of $\mathbf{B}$ is also in  Incomplete Hermite Normal Form. We immediately have
		$$ h_{n,n} = \gcd( b_{1,n}, b_{2,n},...,b_{n,n}).$$
	\end{itemize}
\end{lemma}

\subsection{Ideal lattices}

 An algebraic number field $K$ is an extension field of the rationals $\mathbb{Q}$ such that its dimension $ [K : \mathbb{Q}]$ as a $\mathbb{Q}$-vector space (i.e., its degree) is finite. 
 
 
 
 
 An element $x$ in the algebraic number field $K$ is said to be integral over $\mathbb{Z}$ if the coefficients of the minimal polynomial of $x$ over $\mathbb{Q}$ are all integers. All the elements which are integral over $\mathbb{Z}$ in $K$ make up a set  $O_K$. $O_K$ is actually a ring called the algebraic integer ring of $K$ over $\mathbb{Z}$. 
 
 
  $O_K$ is a finitely generated free $\mathbb{Z}$-module of dimension $[K :\mathbb{Q}]$. The basis of $O_K$ as a free $\mathbb{Z}$-module is called the integer basis, which is also a basis of $K$ as a $\mathbb{Q}$-vector space.
 
\paragraph{Canonical-embedding ideal lattice} 


 If $\Omega\supset K$ is an extension field such that $\Omega$ is algebraically closed over $\mathbb{Q}$, then there are exactly $[K :\mathbb{Q}]$ field embeddings of $K$ into $\Omega$. For convenience, we regard $\Omega$ as the complex field $\mathbb{C}$.


Any ideal of $O_K$ is a full-rank submodule of $O_K$. Let $[K :\mathbb{Q}]=n$.  This structure induces a canonical embedding:
\begin{align*}
\Sigma: O_K&\rightarrow\mathbb{C}^n \\ a&\mapsto (\Sigma_i(a))_{i=1,...,n},
\end{align*}
where $\Sigma_i$'s are the $n$ different embeddings from $K$ into $\mathbb{C}$.

\begin{definition}{(Canonical-embedding Ideal Lattice)}
Given a number field $K$ and  any ideal I of $O_K$, $\Sigma$(I) is called the canonical-embedding ideal lattice.
\end{definition}

\paragraph{Coefficient-embedding ideal lattice} Denote by $\mathbb{Z}^{(n)}[x]$  the set of all the  polynomials in $\mathbb{Z}[x]$ with degree $\leq$$n-1$. We use the symbol $\sigma$ to represent the following linear map: 
\begin{align*} \sigma : \mathbb{Z}^{(n)}[x]&\rightarrow\mathbb{Z}^n \\ \sum_{i=1}^{n} a_ix^{i-1} &\mapsto (a_1,a_1,...,a_n), \end{align*}
where linear map means that 
\begin{itemize}
	\item For any $f(x)$, $g(x)\in \mathbb{Z}^{(n)}[x]$, $\sigma(f(x)+g(x)) = \sigma(f(x))+\sigma(g(x));$
	\item For any $f(x)\in \mathbb{Z}^{(n)}[x]$ and $z\in\mathbb{Z}$, $\sigma(zf(x)) = z\sigma(f(x)).$
\end{itemize}

We can also define its inverse, which is linear too:
\begin{align*} \sigma^{-1} : \mathbb{Z}^n &\rightarrow\mathbb{Z}^{(n)}[x]\\ (a_1,a_1,\cdots,a_n)&\mapsto \sum_{i=1}^{n} a_ix^{i-1}. \end{align*} 




In what follows, we focus on ideal lattices induced by ideals of the ring $\mathbb{Z}[x]/f(x)$, where $f(x)$ is a monic polynomial of degree $n$.  Obviously, any element in $\mathbb{Z}^{(n)}[x]$ can be viewed as a  representative in the ring $\mathbb{Z}[x]/f(x)$.
So we abuse the symbol $\sigma$ to represent the  the following coefficient embedding. 
\begin{align*} \sigma : \mathbb{Z}[x]/f(x)&\rightarrow\mathbb{Z}^n \\ \sum_{i=1}^{n} a_ix^{i-1} &\mapsto (a_1,a_1,...,a_n). \end{align*} 

Therefore, under the coefficient embedding, any ideal of $\mathbb{Z}[x]/f(x)$ can be viewed as an integer lattice.

\begin{definition}{(Coefficient-embedding Ideal Lattice)}
Given $\mathbb{Z}[x]/f(x)$, where  $f(x)$ is a monic polynomial of degree n, and any ideal $I$ of $\mathbb{Z}[x]/f(x)$, $\sigma$(I) is called the coefficient-embedding ideal lattice, which is of course an integer lattice.
\end{definition}






Roughly speaking,  the canonical-embedding ideal lattice are usually used in the theoretical analysis of lattice-related hard problems and lattice-based cryptosystems whereas the coefficient-embedding ideal lattices are usually used in the implementation of lattice-based cryptosystems. Most of the practical lattice-based cryptosystems are employing the  coefficient-embedding ideal lattices.     


In some cases, the SVP problem in canonical-embedding ideal lattices  and  coefficient-embedding ideal lattices  are equivalent or very closely connected with each other as we've mentioned in the latter of the introduction.

In the third section, we first show and prove a naturally equivalent condition (\textbf{Lemma 4}) that whether an integer lattice can be embedded into a given polynomial ring. It's not complicated and it's a direct application of the \textbf{Definition 5}. Though the result of \textbf{Lemma 4} may have been used in the early research, we haven't found a detailed description. Hence, we rewrite and prove the \textbf{Lemma 4} formally.

	Our main theorem (\textbf{Theorem 1}) is motivated by \textbf{Lemma 5} proposed by Zhang, Liu and Lin \cite{ref16}. They show that the coefficients of HNF of a coefficient-embedding ideal lattice satisfy some special condition. Using the result of  \textbf{Lemma 5}
	($h_{n,n}|h_{i,j}$) together with the equivalent condition of \textbf{Lemma 4}, we prove the \textbf{Theorem 1}.

Next, we show some potential application of our main theorem, and give some examples. Though the examples may not be practical in the reality for now, it do supply us a new angle to deal with the SVP in ideal lattices or integer lattices.

In the fourth section, we propose \textbf{Algorithm 1} to judge whether an integer lattice can be embedded into a polynomial ring as an ideal and compute all the rings that the lattice can be embedded into as an ideal if it's an ideal lattice. We originally want to make use of the property of \textbf{Lemma 5}, but we find the property of \textbf{Lemma 5} is too weak to judge whether an integer lattice is an ideal lattice. Hence, based on this ideal we introduce \textbf{Definition 3} the Incomplete Hermite Normal Form and propose another equivalent condition (\textbf{Theorem 2 }) to judge an ideal lattice. Based on \textbf{Theorem 2}, we propose \textbf{Algorithm 1}, and give a simple analysis of the complexity.








\section{An ideal lattice can be embedded into different rings}
We stress that in the following, we focus on the coefficient-embedding ideal lattice, and in this section, we'll show how an coefficient-embedding ideal lattice can be embedded into different rings. 

The idea behind is much simple. It is well known that a lattice is just an additive group. However, when it is also equipped  with some "multiplication", then it becomes an ideal lattice. A natural question is that what will happen if we equip the same lattice with different "multiplication"? Obviously if this can be done, the lattice will not change, but the ideal changes, which means that an ideal lattice can be viewed as different ideals in different rings.



\subsection{Two Properties of ideal lattices}

Before stating our main theorem, we present two crucial properties 
of coefficient-embedding ideal lattices.

\subsubsection{Deciding an ideal lattice} We next present an easy way to tell if a given lattice is a  coefficient-embedding ideal lattice in $\mathbb{Z}[x]/f(x)$ or not.

\begin{lemma} \label{klemma4ideal}
	For any monic polynomial $f(x)\in\mathbb{Z}[x]$ with degree $n$, a lattice $\mathcal{L}(\mathbf{B})$ with any basis $\mathbf{B}$ is a  coefficient-embedding ideal lattice in $\mathbb{Z}[x]/f(x)$  if and only if $\sigma(x\sigma^{-1}(\mathbf{b}_i) \mod f(x))\in \mathcal{L}(\mathbf{B})$ for $i=1,\cdots,n$, where $\mathbf{b}_i$ is the $i$-th row vector of $\mathbf{B}$, and $\sigma$ is the map defined in Section 2.3.
\end{lemma}
\begin{proof} 
	If  $\mathcal{L}(\mathbf{B})$ is a  coefficient-embedding ideal lattice in $\mathbb{Z}[x]/f(x)$, then $\sigma^{-1}(\mathbf{b}_i)$'s are in the corresponding ideal. It is obvious that  $x\sigma^{-1}(\mathbf{b}_i) \mod f(x)$ must be in the ideal too, which means that $\sigma(x\sigma^{-1}(\mathbf{b}_i) \mod f(x))\in \mathcal{L}(\mathbf{B})$.
	
	
	If there exists a  monic polynomial $f(x)\in\mathbb{Z}[x]$ with degree $n$, such that $\sigma(x\sigma^{-1}(\mathbf{b}_i) \mod f(x))\in \mathcal{L}(\mathbf{B})$ for $i=1,\cdots,n$, we show that $\sigma^{-1}(\mathcal{L}(\mathbf{B}))$ must be an ideal in $\mathbb{Z}[x]/f(x)$. It is easy to check that   $\sigma^{-1}(\mathcal{L}(\mathbf{B}))$  is an additive group, due to the fact that $\sigma$ is an additive homomorphism. Since  $\sigma(x\sigma^{-1}(\mathbf{b}_i) \mod f(x))\in \mathcal{L}(\mathbf{B})$, then for any lattice vector $\mathbf{v} = \sum_{i=1}^n z_i\mathbf{b}_i$, $z_i\in\mathbb{Z}$, we have   $$\sigma(x\sigma^{-1}(\mathbf{v}) \mod f(x))=\sum_{i=1}^n z_i \sigma( x \sigma^{-1}(\mathbf{b}_i) \mod f(x)) \in \mathcal{L}(\mathbf{B}). $$
	Applying the result on the lattice vector $\sigma(x\sigma^{-1}(\mathbf{v}) \mod f(x))$, we will have
	$$\sigma(x^2\sigma^{-1}(\mathbf{v}))= \sigma(x \sigma^{-1}(\sigma(x\mathbf{v}\mod f(x)))) \in \mathcal{L}(\mathbf{B}). $$
	Hence, for any positive integer $k$, we know that 
		$$\sigma(x^k\sigma^{-1}(\mathbf{v})) \in \mathcal{L}(\mathbf{B}). $$
	Then for any $g(x)=\sum_{i=1}^n g_ix^{i-1}\in\mathbb{Z}[x]/f(x)$ and any lattice vector $\mathbf{v}$, 
		$$\sigma(g(x)\sigma^{-1}(\mathbf{v}) \mod f(x))=\sum_{i=1}^n g_i \sigma( x^{i-1} \sigma^{-1}(\mathbf{v}) \mod f(x)) \in \mathcal{L}(\mathbf{B}). $$
	The lemma follows.
	

\end{proof}



\subsubsection{HNF of ideal lattices}

 The following lemma tells us some  divisibility relation among the elements in the HNF basis of a coefficient-embedding ideal lattice, which has been proved in \cite{ref16}. For completeness, we present the whole proof simply.

%The proof is exactly the same with the lemma 2 of Liu \textit{et al.}'s paper \cite{ref16}, though our condition is slightly different. For convenience, we rewrite their proof.


%Let $\mathcal{L}(\mathbf{B})$ be a full-rank coefficient-embedding ideal lattice in the ring $\mathbb{Z}[x]/f(x)$, and $\mathbf{H}$ is the HNF of $\mathcal{L}(\mathbf{B})$.


\begin{lemma}[\cite{ref16}] \label{klemma}
Let $\mathbf{H}$ be the HNF basis of the full-rank coefficient-embedding ideal lattice $\mathcal{L}(\mathbf{B})$ in the ring $\mathbb{Z}[x]/f(x)$.
 \[ \mathbf{H}= \begin{pmatrix} h_{1,1}&0&\cdots&0 \\ h_{2,1}&h_{2,2}&\cdots&0 \\ \vdots&\vdots&\ddots&\vdots \\h_{n,1}&\cdots&\cdots&h_{n,n} \end{pmatrix}.\] Then $h_{i,i}|h_{j,l}$, for ${1\leq l \leq j \leq i \leq n}$. Specially,
 $h_{n,n} \vert h_{i,j}$, ${i,j\leq n}$.
\end{lemma}
\begin{proof} 
By induction on $i$,  it's trivial for $i=1$.
         
         Assume the result holds for $i \leq k \leq n-1$. It remains to show that for $i=k+1$, $h_{k+1,k+1}|h_{j,l}$ where ${1\leq l \leq j \leq k+1 \leq n}$.
          
          Let $\mathbf{h}_i$ be the $i$-th row of $\mathbf{H}$. Note that for any ideal $I$ of $\mathbb{Z}[x]/f(x)$ and for all $g(x)\in I $, $xg(x) \in I$. Specially $x\sigma^{-1}(\mathbf{h}_k) \in I$, where $\sigma$ is the coefficient-embedding. Since  $\mathbf{H}$ is a basis of the ideal lattice, it is very simple to imply that there must exist $y_i \in \mathbb{Z}$, for $i=1,2,\cdots,k+1$ such that:\[ \begin{pmatrix} 0&h_{k,1}&\cdots&h_{k,k}&0&\cdots&0 \end{pmatrix}=\sum_{i=1}^{k+1}y_i\mathbf{h}_i.\]
          
         Hence, \begin{align*}h_{k,k}&=y_{k+1}h_{k+1,k+1} \\ h_{k,k-1}&=y_kh_{k,k}+y_{k+1}h_{k+1,k}\\\vdots\\h_{k,1}&=\sum_{i=2}^{k+1}y_ih_{i,2}\\0&=\sum_{i=1}^{k+1}y_ih_{i,1} \end{align*}
         
         
         From the first equation, we get $y_{k+1}=\frac{h_{k,k}}{h_{k+1,k+1}}\in \mathbb{Z}$, and \begin{align*} h_{k+1,k}&=\frac{h_{k,k-1}-y_kh_{k,k}}{h_{k,k}}h_{k+1,k+1}\\h_{k+1,k-1}&=\frac{h_{k,k-2}-y_{k-1}h_{k-1,k-1}-y_kh_{k,k-1}}{h_{k,k}}h_{k+1,k+1}\\\vdots\\h_{k+1,2}&=\frac{h_{k,1}-\sum_{i=2}^ky_ih_{i,2}}{h_{k,k}}h_{k+1,k+1}\\h_{k+1,1}&=\frac{-\sum_{i=1}^ky_ih_{i,1}}{h_{k,k}}h_{k+1,k+1} \end{align*}
          
          From the induction hypothesis, we have $h_{k,k}|h_{j,l}$ for $1\leq l \leq j \leq k \leq n$. So the coefficient of $h_{k+1,k+1}$ in each equation is in fact an integer. Therefore, $h_{k+1,k+1}|h_{k+1,l},1 \leq l \leq k+1$. Since $h_{k+1,k+1}|h_{k,k}$, we know $h_{k+1,k+1}|h_{j,l}$, where $1 \leq l \leq j \leq k+1 \leq n$. Thus, the result holds for $i=k+1$. 
          
          By induction,  $h_{i,i}|h_{j,l}$, ${1\leq l \leq j \leq i \leq n}$. So $h_{n,n}|h_{i,j}$, ${1\leq i \leq j \leq n}$.  Lemma \ref{klemma} follows.
\end{proof}

\begin{remark}\label{remarkhnf}
	Note that in the proof of Lemma \ref{klemma}, to conclude that $h_{n,n} \vert h_{i,j}$, ${i,j\leq n}$, what we need is $\sigma(x\sigma^{-1}(\mathbf{h}_k)) \in \mathcal{L}(\mathbf{B})$ for $k =1,\cdots, n-1$. We do not care about  if  $\sigma(x\sigma^{-1}(\mathbf{h}_n))$ is in $\mathcal{L}(\mathbf{B})$ or not.		
\end{remark}


Lemma \ref{klemma} presents us the divisibility relation among the elements in the HNF basis of a coefficient-embedding ideal lattice. Actually, not only it's crucial to our main theorem, but also we can regard it as a  tool to quickly rule out some integer lattices not being an ideal lattice in any polynomial ring. See Section 4 for more details.


\subsection{Main theorem}
%The main difference between the lattice and the ideal is the multiplication structure.  To simulate the ideal multiplication structure in an integer lattice, we need to define a column transformation  on a row vector.


We next present our main theorem.
\begin{theorem}\label{kthm}
For any  full-rank coefficient-embedding ideal lattice $\mathcal{L}(\mathbf{B})$ in the ring $\mathbb{Z}[x]/f(x)$, where $f(x)$ is monic and $\text{deg}(f(x))=n$, there exists infinitely many monic $g(x)\in\mathbb{Z}[x]$ with degree $n$, s.t.  $\mathcal{L}(\mathbf{B})$ is also a coefficient-embedding  ideal lattice in $\mathbb{Z}[x]/g(x)$.

More precisely, 
let $d = \gcd( b_{1,n}, b_{2,n},...,b_{n,n})$. Then $\mathcal{L}(\mathbf{B})$ is also a coefficient-embedding  ideal lattice in $\mathbb{Z}[x]/g(x)$, where $g(x)\in\mathbb{Z}[x]$ is a monic polynomial with degree $n$, if and only if 
$$\sigma(f(x)-g(x)) \in  \mathcal{L}(\frac{\mathbf{B}}{d}),$$
or equivalently,
$$g(x) \in f(x) + \sigma^{-1}(\mathcal{L}(\frac{\mathbf{B}}{d})).$$

\end{theorem}
\begin{proof}
%It's sufficient to find an injection from $\mathcal{L}(\mathbf{B})$ to some polynomial ring $\mathbb{Z}[x]/g(x)$. 

Consider the HNF basis of $\mathcal{L}(\mathbf{B})$, 
 \[ \mathbf{H}= \begin{pmatrix} h_{1,1}&0&\cdots&0 \\ h_{2,1}&h_{2,2}&\cdots&0 \\ \vdots&\vdots&\ddots&\vdots \\h_{n,1}&\cdots&\cdots&h_{n,n} \end{pmatrix}.\]
For convenience, we denote  by $\mathbf{h}_i$ the $i$-th row of $\mathbf{H}$, and then $\mathbf{h}_i$ is a vector in $\mathbb{Z}^n$. 

(i) If there is a monic $g(x)\in\mathbb{Z}[x]$ with degree $n$, s.t.  $\mathcal{L}(\mathbf{B})$ is also a coefficient-embedding  ideal lattice in $\mathbb{Z}[x]/g(x)$, we next prove that $\sigma(f(x)-g(x)) \in  \mathcal{L}(\frac{\mathbf{B}}{d})$.  

By Lemma \ref{klemma4ideal}, 
 $\mathcal{L}(\mathbf{H}) = \mathcal{L}(\mathbf{B})$ is  a coefficient-embedding  ideal lattice in $\mathbb{Z}[x]/f(x)$, then we  have
$$\sigma(x\sigma^{-1}(\mathbf{h}_n) \mod f(x))\in \mathcal{L}(\mathbf{B}).$$
Note that 
$$x\sigma^{-1}(\mathbf{h}_n) \mod f(x) = \sum_{i=1}^{n-1} h_{n,i}x^i - h_{n,n} (f(x)-x^n).$$
We have
 \begin{equation} \label{equ1}
\begin{pmatrix} 0&h_{n,1}&...&h_{n,n-1} \end{pmatrix}-h_{n,n}\sigma(f(x)-x^n) \in  \mathcal{L}(\mathbf{B}).
 \end{equation}
 
 
 Similarly, since $\mathcal{L}(\mathbf{B})$ is also a coefficient-embedding  ideal lattice in $\mathbb{Z}[x]/g(x)$, we have 
  \begin{equation} \label{equ2}
 \begin{pmatrix} 0&h_{n,1}&...&h_{n,n-1} \end{pmatrix}-h_{n,n}\sigma(g(x)-x^n) \in  \mathcal{L}(\mathbf{B}).
  \end{equation}
 Subtracting the left side of (\ref{equ1}) from the left side of (\ref{equ2}), we immediately have
 $$h_{n,n}\sigma(f(x)-g(x)) \in \mathcal{L}(\mathbf{B}).$$
 By Lemma \ref{ihnflemma}, $h_{n,n} = d$, we have
 $$\sigma(f(x)-g(x)) \in \mathcal{L}(\frac{\mathbf{B}}{d}).$$
 
 
 
 (ii) We next prove that for any  polynomial $g(x)$, such that $\sigma(f(x)-g(x)) \in  \mathcal{L}(\frac{\mathbf{B}}{d})$, any  full-rank coefficient-embedding ideal lattice $\mathcal{L}(\mathbf{B})$ in the ring $\mathbb{Z}[x]/f(x)$ can also be viewed as a coefficient-embedding  ideal lattice in $\mathbb{Z}[x]/g(x)$.
 
 
 First, $g(x)$ is obviously a monic polynomial with degree $n$. Note that by Lemma \ref{klemma}, $h_{n,n}|h_{i,j}$, then $d = h_{n,n}$ divide all the components of every lattice vector in $\mathcal{L}(\mathbf{B})$, which means that $ \mathcal{L}(\frac{\mathbf{B}}{d})$ is an integer lattice and once $\sigma(f(x)-g(x)) \in  \mathcal{L}(\frac{\mathbf{B}}{d})$, $g(x)\in\mathbb{Z}[x]$.
 
 
 By Lemma \ref{klemma4ideal} again, 
 $\mathcal{L}(\mathbf{H}) = \mathcal{L}(\mathbf{B})$ is  a coefficient-embedding  ideal lattice in $\mathbb{Z}[x]/f(x)$, then we  have
 $$\sigma(x\sigma^{-1}(\mathbf{h}_i) \mod f(x))\in \mathcal{L}(\mathbf{B}),$$
 for $i = 1,\cdots, n$.
 
 To prove that  $\mathcal{L}(\mathbf{B})$ is also a coefficient-embedding  ideal lattice in $\mathbb{Z}[x]/g(x)$, by Lemma \ref{klemma4ideal}  it is enough to show that 
  $\sigma(x\sigma^{-1}(\mathbf{h}_i) \mod g(x))\in \mathcal{L}(\mathbf{B}),$
 for $i = 1,\cdots, n$.
 
 
 Note that for $i = 1,\cdots, n-1$, 
 $$\sigma(x\sigma^{-1}(\mathbf{h}_i) \mod g(x)) = \sigma(x\sigma^{-1}(\mathbf{h}_i) \mod f(x))\in \mathcal{L}(\mathbf{B}).$$
 
 Since$\sigma(f(x)-g(x)) \in  \mathcal{L}(\frac{\mathbf{B}}{d})$, there exists a lattice vector $\mathbf{v} \in \mathcal{L}(\mathbf{B})$ such that $d(f(x)-g(x)) = h_{n,n}(f(x)-g(x))= \sigma^{-1}(\mathbf{v})$. Then for $i = n$, 
 \begin{align*}
 \sigma(x\sigma^{-1}(\mathbf{h}_n) \mod g(x)) &=\sigma (\sum_{i=1}^{n-1} h_{n,i}x^i - h_{n,n} (g(x)-x^n))\\& = \sigma (\sum_{i=1}^{n-1} h_{n,i}x^i- h_{n,n} (f(x)-x^n) + \sigma^{-1}(\mathbf{v}))\\
 &= \sigma(x\sigma^{-1}(\mathbf{h}_n) \mod f(x)) + \mathbf{v} \in \mathcal{L}(\mathbf{B}).
 \end{align*}
 The  theorem  follows.
\end{proof}

\begin{remark}
	The HNF $\mathbf{H}$ in the proof can be replaced by any Incomplete Hermite Normal Form.
	
\end{remark}
\subsection{Applications}

For most lattice-based cryptosystems, their security is guaranteed by the hardness of lattice problems such as $\gamma$-SVP.  Hence, the hardness of lattice problem in ideal lattice is widely considered as the security foundation of Ring-LWE based cryptosystems. 


 
 Due to the additional algebraic structure, the problem for the ideal lattice is usually conjectured to be easier than that for the general integer lattice. Some recent progress supports the argument well. Obviously, the algebraic structure depends on the polynomial ring that the ideal belongs to. 
 




 However, Theorem \ref{kthm} shows us that an ideal lattice can be embedded as some ideals into different polynomial rings, which means that an ideal lattice may have different "algebraic structure" in different rings although the lattice stays the same.

This phenomenon inspires us to consider the following method to solve the hard problems for a given ideal lattice. By changing the polynomial ring, is it possible to transform the given ideal lattice as another ideal in which the lattice problems can be solved more efficiently by using the new algebraic structure? It seems hard to present a negative answer if  the algebraic structure can indeed help solve the hard problems,  since we have to consider infinite ideals and hence infinite algebraic structures. This no doubt increases the difficulty to show that the lattice problem for some fixed ideal lattice is hard.


On the other hand, if we can utilize the algebraic structure to solve the  lattice problems in some ideal lattice, then we can solve the problems for infinite ideal lattice in different rings.  We would like to stress that as a lattice, these ideal lattices are same. However, as ideals, they are different. It seems that a weak ideal will spread as infinite weak ideals.
  
Next we present some concrete examples to show the potential risk inspired by Theorem \ref{thmidentify}.

\subsubsection{Pre-processing  a fixed ring brings more.}


 In \cite{ref10},   Pellet-Mary \textit{et al.} showed  pre-processing the number field can help solve $\gamma$-SVP in canonical-embedding ideal lattices more efficiently. However, pre-processing usually costs too much time. One may think for different number fields, we have to do different pre-processing. By Theorem \ref{thmidentify}, we know that  pre-processing  a fixed number field will also  help us solve  $\gamma$-SVP  more efficiently in ideals that is not in the algebraic integer ring of the fixed number field. 
 
 
 Consider the ring $\mathbb{Z}[x]/(x^n+1)$, where $n=2^k$, which is one of the most used rings in cryptosystems. It is well known that the lengths of  vectors induced by the same element  under  the coefficient embedding and the canonical embedding are  the same up to a fixed factor, which means the hardness of SVP in such two embedding ideal lattices  are equivalent. Hence, by the method in \cite{ref10}, we can pre-process the  ring $\mathbb{Z}[x]/(x^n+1)$, and then can solve $\gamma$-SVP in its any coefficient-embedding ideal lattice more efficiently.  
 
 Below we give a simple example to show how to apply the pre-processing on the ideal in other polynomial ring.
\begin{example}
Given a coefficient-embedding ideal lattice in the ring $\mathbb{Z}[x]/(x^n+x^{n-1}+2x^{n-2}+1)$ induced by the ideal $<x+2>$, where $n=2^k$,  the basis has the form \[\mathbf{B}=\begin{pmatrix}2&1&0&\cdots&\cdots&0\\0&2&1&\cdots&\cdots&0\\\vdots&\vdots&\vdots&\vdots&\vdots&\vdots\\0&0&\cdots&0&2&1\\-1&0&\cdots&0&-2&1\end{pmatrix}.\]
Note that the greatest common divisor $d$ of the entries in the last column is $1$.
To verify that $ \mathcal{L}(\mathbf{B})$ can be embedded as an ideal into the ring $\mathbb{Z}[x]/(x^n+1)$, according to Theorem \ref{kthm}, it's sufficient to verify the following relation:
\[\sigma((x^n+x^{n-1}+2x^{n-2}+1)-(x^n+1))=\begin{pmatrix}0&0&\cdots&0&2&1\end{pmatrix}\in\mathcal{L}(\mathbf{B}),\]
which is obvious.

Therefore, by pre-processing the field $\mathbb{Q}[x]/(x^n+1)$, we can also handle the hard problems on the ideal lattice  $ \mathcal{L}(\mathbf{B})$ in $\mathbb{Z}[x]/(x^n+x^{n-1}+2x^{n-2}+1)$.
\end{example}


By the discussion above, our theorem has a good chance to amplify the results of the research on the ideal lattice of certain rings.
 
%\begin{remark}
%In fact, this argument can be extended to all  cyclotomic fields.
%As we all know, there exists irreducible polynomial $f(x)$ over $\mathbb{Z}[x]$, s.t. $O(Q(\zeta))\cong \mathbb{Z}[x]/f(x)$, where $\zeta$ is a primitive unit root. Similarly to the ring $\mathbb{Z}[x]/(x^n+1)$, it's possible to handle the ideal lattices in the algebraic integer rings of any cyclotomic fields.  But the norm relation between the two embeddings in any cyclotomic field is slightly complex. In fact, there is a linear transformation between the two embeddings in cyclotomic field. We refer to \cite{ref11} for more details. Then our theory can extend Pellet-Marry's \cite{ref10} method to any integer lattice that's in the same class of the algebraic integer ring of some cyclotomic field.
%\end{remark}


\subsubsection{Changing the ring  may be not enough for the security.}
Sometimes, we want to choose a special ring for the cryptosystems to resist some potential attacks. This may work in general. However, for some fixed ideal lattices, this may be not enough to obtain the desired security.


For example, NTRUPrime \cite{ref29} uses the ring $\mathbb{Z}[x]/(x^p-x-1)$ to resist the potential subfield attacks against NTRU, where $p$ is an odd prime. We next present a simple example to show that some ideals generated by polynomials with small coefficients in the ring $\mathbb{Z}[x]/(x^p-x-1)$   can also be embedded as an ideal into some $\mathbb{Z}[x]/f(x)$, where $f(x)$ is reducible. However, a  reducible $f(x)$ may cause some potential security risk.
\begin{example}
	For convenience, we assume that $p$ is large enough. Consider the coefficient-embedding ideal lattice induced by the principal ideal $<x^{p-1}-x^2-x>$    in the ring $\mathbb{Z}[x]/(x^p-x-1)$.  We show that such ideal lattice can be embedded as an ideal into the ring $\mathbb{Z}[x]/f(x)$, where $f(x) = (x+1)(x^{p-1}-x-1)$ is reducible.
	
	The lattice basis of is 
	\[\mathbf{B}=\begin{pmatrix} 0&-1&-1&0&0&\cdots&1\\1&1&-1&-1&0&\cdots&0\\\vdots&\vdots&\vdots&\vdots&\vdots&\vdots&\vdots\\\cdots&\cdots&\cdots&\cdots&\cdots&\cdots&\cdots\end{pmatrix}.\]
	
	Note that the greatest common divisor $d$ of the entries in the last column is $1$.  According to Theorem \ref{kthm}, it's sufficient to verify the following relation:
	$$\sigma((x+1)(x^{p-1}-x-1)-(x^p-x-1))=(0,-1,-1,0,\cdots,1) \in \mathcal{L}(\mathbf{B}),$$
	and $ (0,-1,-1,0,\cdots,1)$ is exactly the first row of $\mathbf{B}$.
\end{example}




\section{Identifying an Ideal Lattice}

We also present an algorithm to identify the ideal lattice, which is faster than that in \cite{ref13}.

\subsection{Main theorem}
Inspired by Theorem \ref{kthm}, we find a new equivalent condition between integer lattices and coefficient-embedding ideal lattices, which is described as below.
    
  
\begin{theorem}\label{thmidentify}
	Given a full-rank integer lattice $\mathcal{L}(\mathbf{B})$, let $\mathbf{B}'= \begin{pmatrix} \mathbf{D}&\mathbf{0} \\ \mathbf{b'}&b_{n,n}' \end{pmatrix}$ be any  Incomplete Hermit Normal Form of $\mathbf{B}$. Then $\mathcal{L}(\mathbf{B})$ is an ideal lattice if and only if there exists a $\mathbf{T} \in \mathbb{Z}^{(n-1) \times n}$, s.t.$\begin{pmatrix} \mathbf{0}&\mathbf{D}\end{pmatrix}=\mathbf{T}\mathbf{B}$.
\end{theorem}
\begin{proof}
It can be easily check the ``only if'' part by Lemma \ref{klemma4ideal}, since for an ideal lattice $\mathcal{L}(\mathbf{B})$ in $\mathbb{Z}[x]/g(x)$, there exists a $\mathbf{T} \in \mathbb{Z}^{(n-1) \times n}$, s.t. $\begin{pmatrix} \mathbf{0}&\mathbf{D}\end{pmatrix}=\mathbf{T}\mathbf{B}$ if and only if  $\sigma(x\sigma^{-1}(\mathbf{b}'_i) \mod g(x))\in \mathcal{L}(\mathbf{B})$ for $i=1,\cdots,n-1$.

For ``if'' part,
to indicate that $\mathcal{L}(\mathbf{B})$ is an ideal lattice, we need to find a monic polynomial $g(x)$ of degree $n$, s.t. $\mathcal{L}(\mathbf{B})$ can be embedded as an ideal into $\mathbb{Z}[x]/g(x)$, or $\sigma(x\sigma^{-1}(\mathbf{b}'_i) \mod g(x))\in \mathcal{L}(\mathbf{B})$ for $i=1,\cdots,n$  by Lemma \ref{klemma4ideal}.

Note that for any polynomial $g(x)$ with degree $n$, $\sigma(x\sigma^{-1}(\mathbf{b}'_i) \mod g(x))\in \mathcal{L}(\mathbf{B})$ for $i=1,\cdots,n-1$ since there exists a $\mathbf{T} \in \mathbb{Z}^{(n-1) \times n}$, s.t. $\begin{pmatrix} \mathbf{0}&\mathbf{D}\end{pmatrix}=\mathbf{T}\mathbf{B}$. 

It remains to show that there exists a monic polynomial $g(x)$ of degree $n$, such that $\sigma(x\sigma^{-1}(\mathbf{b}'_n) \mod g(x))\in \mathcal{L}(\mathbf{B})$.

We first present a lemma, which will be proven later.
\begin{lemma} \label{identify}	
	If $\begin{pmatrix} \mathbf{0}&\mathbf{D}\end{pmatrix}=\mathbf{T}\mathbf{B}$, then $ \mathbf{B}'/b_{n,n}' \in \mathbb{Z}^{(n-1) \times (n-1)}$
\end{lemma}

 By Lemma \ref{identify},  $\frac{1}{b_{n,n}'}(\begin{pmatrix}0&\mathbf{b}'\end{pmatrix} +\mathcal{L}(\mathbf{B})) \subset \mathbb{Z}^n$. Taking any
 \begin{equation} \label{equg}
  \mathbf{g}=\begin{pmatrix}g_1&g_2&\cdots&g_n\end{pmatrix} \in\frac{1}{b_{n,n}'}(\begin{pmatrix}0&\mathbf{b}'\end{pmatrix} +\mathcal{L}(\mathbf{B})),
 \end{equation}
   the integer polynomial $g(x)=x^n+g_nx^{n-1}+\cdots+g_1$ is what we want, since
$$
 \sigma(x\sigma^{-1}(\mathbf{b}'_n) \mod g(x)) = \begin{pmatrix}0&\mathbf{b}'\end{pmatrix} - {b_{n,n}'}\begin{pmatrix}g_1&g_2&\cdots&g_n\end{pmatrix}\in \mathcal{L}(\mathbf{B}). $$
 
 

%Hence, all the polynomials corresponding to the vectors in the coset $\frac{1}{b_{n,n}'}(\begin{pmatrix}0&\mathbf{b}'\end{pmatrix} +\mathcal{L}(\mathbf{B}))$ can induce a coefficient embedding from $\mathcal{L}(\mathbf{B})$ into polynomial ring.  $\mathcal{L}(\mathbf{B})$ is a coefficient-embedding ideal lattice.
\end{proof}

It remains to prove Lemma \ref{identify}.
\begin{proof}{(Lemma  \ref{identify})} According to Lemma \ref{hnfbasis}, 	
 $\mathcal{L}(\mathbf{B}')$ has a unique HNF basis, denoted by $\mathbf{H}=(h_{i,j})_{1\leq i \leq n,1\leq j \leq n}$. By Lemma \ref{ihnflemma}, we know that $b_{n,n}'=h_{n,n}$.
 
 It can be easily concluded that the lattice $\mathcal{L}(\mathbf{D})$ has a unique HNF basis,  $\mathbf{H}'=(h_{i,j})_{1\leq i \leq n-1,1\leq j \leq n-1}$, which implies that there exists a unimodular matrix $\mathbf{U}\in \mathbb{Z}^{(n-1)\times (n-1)}$ such that $\mathbf{H}' = \mathbf{U} \mathbf{D}$.
	
Since $\begin{pmatrix} \mathbf{0}&\mathbf{D}\end{pmatrix}=\mathbf{T}\mathbf{B}$, we have 
$\mathbf{U}\begin{pmatrix} \mathbf{0}&\mathbf{D}\end{pmatrix}=\mathbf{U}\mathbf{T}\mathbf{B}$, which is exactly
\begin{equation}\label{equ3}
\begin{pmatrix}0&h_{1,1}&0&...&0\\0&h_{2,1}&h_{2,2}&...&0\\.&.&.&.&.\\0&h_{n-1,1}&h_{n-1,2}&...&h_{n-1,n-1}\end{pmatrix}=\mathbf{U}\mathbf{T}\mathbf{B}.
\end{equation}


Note that $\mathcal{L}(\mathbf{U}\mathbf{T}\mathbf{B})\subset \mathcal{L}(\mathbf{B})$. What Equation (\ref{equ3}) tells us is 
$$\sigma(x\sigma^{-1}(\mathbf{h}_k)) \in \mathcal{L}(\mathbf{B}),\mbox{ for } k =1,\cdots, n-1.$$

By the discussion in Remark \ref{remarkhnf}, we have  $b_{n,n}'=h_{n,n}\vert\mathbf{B}$. 
\end{proof}

\subsection{Algorithm and analysis}

According to Theorem \ref{thmidentify},  there is a very simple algorithm to identify if a given integer lattice is an ideal lattice or not.

\begin{algorithm}[htb]
	\caption{Identifying an ideal lattice}
	\label{alg:identify}
	\begin{algorithmic}[1]
		\Require $\mathbf{B} \in \mathbb{Z}^{n \times n}$, $\text{rank}(\mathbf{B})=n$.
		\Ensure False if $\mathcal{L}(\mathbf{B})$ is not a coefficient-embedding ideal lattice; Otherwise output a set $S\subset \mathbb{Z}^n$  s.t. for any $(g_1,g_2,...,g_n)\in S$, $\mathcal{L}(\mathbf{B})$ can be embedded as an ideal  into $\mathbb{Z}[x]/(g_1x+g_2x^2+...+g_nx^{n-1}+x^n)$.
		\State  Compute  any  Incomplete Hermit Normal Form $\mathbf{B}'= \begin{pmatrix} \mathbf{D}&\mathbf{0} \\ \mathbf{b'}&b_{n,n}' \end{pmatrix}$ of $\mathbf{B}$ by unimodular transformation;
		\If{$b_{n,n}' \not\vert\  \mathbf{B}$} return False;
		\EndIf
	    \If {$\begin{pmatrix} \mathbf{0}&\mathbf{D}\end{pmatrix}\mathbf{B}^{-1}\notin\mathbb{Z}^{(n-1)\times n} $} return False;
		\EndIf
		\State  Output $S = \frac{1}{b'_{n,n}}(\begin{pmatrix}0&\mathbf{b}'\end{pmatrix}+\mathcal{L}(\mathbf{B}))$.
	\end{algorithmic}
\end{algorithm}

\begin{remark}\label{HNFremark}
	In Step 1, we can also compute the HNF of $\mathcal{L}(\mathbf{B})$, and then use the divisibility relation described in Lemma \ref{klemma} to rule out some integer lattices that can't be embedded as an ideal into any polynomial ring. This may speedup the algorithm in practice, since many "random" integer lattices can not pass such check.
\end{remark}

\paragraph{Correctness} The correctness of our algorithm is guaranteed by Theorem \ref{thmidentify}. 

\paragraph{Complexity} We next analyze the time complexity.

For Step 1, we can use the following algorithm to compute an Incomplete Hermite Normal Form for   $\mathbf{B} \in \mathbb{Z}^{n \times n}$ with a unimodular transformation, whose idea has already been described in Section 2.2.


\begin{algorithm}[htb]
	\caption{Computing an Incomplete Hermite Normal Form}
	\label{alg:ihnf}
	\begin{algorithmic}[1]
		\Require $\mathbf{B} \in \mathbb{Z}^{n \times n}$, $\text{rank}(\mathbf{B})=n$.
		\Ensure An Incomplete Hermit Normal Form of $\mathbf{B}$ by unimodular transformation.
		\For{$i$ from 1 to $n-1$}
		\State Use Extended Euclidean Algorithm with input $(b_{i,n},b_{i+1,n})$ to find $x$, $y$, $d$ s.t. $xb_{i,n}+yb_{i+1,n}=\gcd(b_{i,n},b_{i+1,n})=d$;
		\State  Update	$\begin{pmatrix} \mathbf{b_i} \\ \mathbf{b_{i+1}}\end{pmatrix}$:= $\begin{pmatrix}-b_{i+1,n}/d&b_{i,n}/d\\x&y\end{pmatrix}\begin{pmatrix} \mathbf{b_i} \\ \mathbf{b_{i+1}}\end{pmatrix}$;
		\EndFor
		\State  Output $\mathbf{B}$.
	\end{algorithmic}
\end{algorithm}

It is easy to check that the integer matrix $\begin{pmatrix}-b_{i+1,n}/d&b_{i,n}/d\\x&y\end{pmatrix}$ is unimodular since its determinant is $-1$. Hence, the transformation in Step 3 will not change the lattice $\mathcal{L}(\mathbf{B})$. After Step 3 for each $i$, we have $b_{i,n} = 0$ and $b_{i+1,n}= d$ computed by Step 2, which means that the output is in Incomplete Hermite Normal Form.

For the time complexity, we assume that for the input  $\mathbf{B}$, the absolute value of its every entry is bounded by $2^B$. 

It is easy to conclude that for the $i$-th loop, at the beginning, we have 
\begin{itemize}
	\item $|b_{i,j}|< 2^{i*B+1}$, $|b_{i+1,j}|< 2^{B}$ for $j =1,\cdots, n$, especially we have $|b_{i,n}|< 2^{B}$;
	\item $|x|<  2^{B}$, $|y|< 2^{B}$, $d<  2^{B}$.
\end{itemize}
Note that the Extended Euclidean Algorithm takes $O(\text{log}\vert a \vert \text{log}\vert b\vert)$ bit operations on input $(a, b)$. Then for the $i$-th loop, with the plain integer multiplication we have:
\begin{itemize}
	\item Step 2 costs $O(B^2)$ bit operations;
	\item Step 3 costs $O(i*nB^2)$ bit operations;
\end{itemize} 
Hence, for the total $n$ loops, Algorithm \ref{alg:ihnf} needs $O(n^3B^2)$ bit operations, and we have the following result.

\begin{lemma}\label{clemma}
	For a non-singular matrix $\mathbf{B}\in\mathbb{Z}^{n \times n}$, the absolute value of whose entries is bounded by $2^B$, Algorithm \ref{alg:ihnf} takes $O(n^3B^2)$ bit operations to compute an Incomplete Hermite Normal Form of $\mathbf{B}$ by a unimodular transformation.
\end{lemma}




















For Step 4, we refer to Theorem 37 of \cite{ref12} for more details.

\begin{theorem}[Theorem 37 in \cite{ref12}] \label{thmsto}
	There exists a Las Vegas algorithm that takes as input a non-singular $\mathbf{A} \in \mathbb{Z}^{n \times n}$ 
    and $\mathbf{b} \in \mathbb{Z}^{n}$, and returns as output the vector $\mathbf{b}\mathbf{A}^{-1}\in\mathbb{Q}^n$. If the absolute value of the entries of $\mathbf{A}$ is bounded by $2^B$, and  the absolute value of the entries of $\mathbf{b}$ is bounded by $2^{nB}$, then
	the expected cost of the algorithm is $O((\log n)\mathbf{MM}(n)\mathbf{MZ}(B + \log n))$ bit operations, where 
	$\mathbf{MM}(n$) means two $n\times n$ matrices can be multiplied using at most $\mathbf{MM}(n)$ integer multiplications and $\mathbf{MZ}(B$) means two $B$-bits integer  can be multiplied using at most $\mathbf{MZ}(B)$ bit operations.	
	This result assumes
	that $\mathbf{MZ}(t) = O(\mathbf{MM}(t)/t)$.

\end{theorem}


It is It is well known that the classical plain multiplication method allows $\mathbf{MZ}(B)= O(B^2)$, and 
the Sch\"{o}nhage-Strassen algorithm \cite{ref24} allows $\mathbf{MZ}(B)=O(B(\text{log}B$)((\text{log} $\text{log}B$).


For $\mathbf{MM}(n)$, the classical plain multiplication method allows $\mathbf{MM}(n)=2n^3-n^2$,  and the asymptotically faster method allows $\mathbf{MM}(n)=O(n^{2.376})$. We refer to \cite{ref17} and \cite{ref18} for more details and further discussion. 

For simplicity, we adopt the classical plain method for both integer multiplication and matrix multiplication, that is  $\mathbf{MZ}(B)= O(B^2)$ and $\mathbf{MM}(n)=O(n^3)$. Then by Theorem \ref{thmsto} a simple analysis shows that Step 4 in Algorithm \ref{alg:identify} costs  $O(n^4\log n(B + \log n)^2)$ bit operations



Together with Lemma \ref{clemma}, we have
\begin{theorem}\label{complexity}
Given $\mathbf{B} \in \mathbb{Z}^{n \times n}$, $\text{rank}(\mathbf{B})=n$, and the absolute value of the entries of $\mathbf{B}$ is bounded by $2^B$, then there is a Las Vegas algorithm with expected complexity  $O(n^4\log n(B + \log n)^2)$ to identify whether $\mathcal{L}(\mathbf{B})$ is an ideal lattice or not.
\end{theorem}


\begin{remark}\label{compare}
It is claimed  in \cite{ref13}  that the algorithm presented by Ding and Lindner to identify an ideal lattice  costs $O(n^4B^2)$ bit operations. However, we have to point out that there is some flaw in the complexity analysis in  $O(n^4B^2)$.  The algorithm in \cite{ref13}  needs to compute $n-2$ powers of $\mathbf{B}$, that is, $\mathbf{B}^k$ for $k = 2, \cdots, n-1$. It is claimed this can be done within $O(n^4B^2)$ bit operations. However, when $k$ grows bigger, the bit size of the entries in $\mathbf{B}^k$ will be $O(kB)$ instead of $B$. Hence the correct time complexity  should be 
$$\sum_{k=2}^{n-1}O(n^3*k*B^2) =  O(n^5B^2).$$
So our algorithm is faster than the algorithm in \cite{ref13}, due to the fact that our algorithm just checks if the systems of equations have integer solutions or not.

%Besides, the algorithm in \cite{ref13}  outputs a single polynomial ring of the ring class if the input lattice is an ideal lattice but we compute the entire class.
\end{remark}

\section{Conclusion}
In this paper, we reveal the embedding relation between the coefficient-embedding ideal lattice and the integer lattice, which gives us a new method to solve the ideal lattice problems by embedding the given ideal lattice into the well-studied polynomial ring. 
 
 Hence, it's not proper anymore to judge the security of a crypstosystem based on ideal lattice by just considering a single ring. The embedding relation no doubt increases the difficulties of evaluating the security for any crypstosystem based on ideal lattices. 

Since the ideal lattice is a special case of the module lattice, it's possible that there is a similar embedding relation between integer lattices and module lattices. Therefore, it's worth researching that how to generalize our theory to module lattice.

%As a direct production of our theorem, we introduce an efficient method to identifying an ideal lattice. Since our algorithm only computes the gcd of the last column of the given lattice, it avoids the explosion of the matrices coefficients. Besides,our algorithm not only identify an ideal lattice, but also computes the entire class of the rings which the given integer lattice can be embedded into. 


\begin{thebibliography}{99}  

\bibitem{ref1}Ajtai, M. Generating Hard Instances of Lattice Problems. Proceedings of the Twenty-Eighth Annual ACM Symposium on Theory of Computing. pp. 99–108. CiteSeerX 10.1.1.40.2489. doi:10.1145/237814.237838. ISBN 978-0-89791-785-8. S2CID 6864824.(1996)

\bibitem{ajtai98}
Ajtai, M. The shortest vector problem in L2 is NP-hard for randomized reductions
(extended abstract). Proceedings of the Thirtieth Annual ACM Symposium on Theory of Computing. pp. 10–19. STOC ’98, ACM, New York, NY, USA (1998)


\bibitem{ref2}Ajtai, M.; Dwork, C. A Public-Key Crypstosystem with Worst-Case/ Average-Case Equivalence. Electronic Colloquium on Computational Complexity-Report Series 1996.-available via: http://www.eccc.uni-trier.de/eccc/(1997)





\bibitem{ref11}Batson, S.C. The linear transformation that relates the canonical and coefficient embeddings of ideal in cyclotomic integer rings.International Journal of Number Theory. Vol.13, No.09, pp. 2277-2297.https://www.worldscientific.com/doi/abs/10.1142/S1793042117501251(2017)

\bibitem{ref19}Bernard, O.; Roux-Langlois, A. Twisted-PHS: Using the Product Formula
to Solve Approx-SVP in Ideal Lattices. International Conference on the Theory and Application of Cryptology and Information Security. ASIACRYPT 2020: Advances in Cryptology – ASIACRYPT 2020 pp 349–380. https://doi.org/10.1007/978-3-030-64834-3\_12. (2020)

\bibitem{ref20}Bernard, O.; Lesavourey,  A.; Nguyen T.; Roux-Langlois, A. Log-S-unit Lattices Using Explicit Stickelberger Generators to Solve Approx Ideal-SVP. International Conference on the Theory and Application of Cryptology and Information Security. ASIACRYPT 2022: Advances in Cryptology – ASIACRYPT 2022 pp 677–708.https://doi.org/10.1007/978-3-031-22969-5\_23 (2022)

\bibitem{ref29} Bernstein, D.; Brumley, B.; Chen, M.; Chuengsatiansup, C.; Lange, T.; Marotzke, A.; Peng, B.; Tuveri, N.; Vredendaal, C.; Yang, B. NTRU Prime: round 3(20201007). https://ntruprime.cr.yp.to/nist/ntruprime-20201007.pdf (2020)

\bibitem{ref22}Boudgoust, K.; Gachon, E.; Pellet-Mary, A. Some Easy Instances of Ideal-SVP and Implications on the Partial Vandermonde Knapsack Problem. Annual International Cryptology Conference. CRYPTO 2022: Advances in Cryptology – CRYPTO 2022 pp 480–509.https://doi.org/10.1007/978-3-031-15979-4\_17 (2022)



\bibitem{ref23}Buchmann, J.; Lindner, R. Density of Ideal Lattices. Algorithms and Number Theory 2009 Mathematics, Computer Science.(2009)

\bibitem{ref18}Buergisser, P.; Clausen, M.; Shokrollahi, M. A.  Algebraic Complexity Theory, volume 315 of Grundlehren der mathematischen Wissenschaften. Springer-Verlag.(1996)


\bibitem{ref7}Cramer, R.; Ducas, L.; Peikert, C.; Regev, O. Recovering short generators of principal ideals in cyclotomic rings. In: Fischlin, M., Coron, J.-S. (eds.)EUROCRYPT 2016. LNCS, vol. 9666, pp. 559–585. Springer, Heidelberg. doi:10.1007/978-3-662-49896-520(2016)

\bibitem{ref8}Cramer, R.; Ducas, L.; Wesolowski, B. Short Stickelberger class relations and application to ideal-SVP. In: Coron, J.-S., Nielsen, J.B. (eds.) EUROCRYPT 2017. LNCS, vol. 10210, pp. 324–348. Springer, Cham. https://doi.org/10.1007/978-3-319-56620-7 12(2017)


\bibitem{ref13}Ding, J.; Lindner, R. Identifying Ideal Lattices. https://eprint.iacr.org/2007/322.pdf(2007)

\bibitem{DKT87}Domich, P.D., Kannan, R., Trotter, L.E.: Hermite Normal Form
Computation Using Modulo Determinant Arithmetic. Mathematics of Operations Research 12(1), 50-59 (1987).



\bibitem{ref3}Hoffstein, J.; Pipher, J.; Silverman, J. H. NTRU: A ring-based public key cryptosystem. Algorithmic Number Theory. Lecture Notes in Computer Science. Vol. 1423. pp. 267–288. CiteSeerX 10.1.1.25.8422. doi:10.1007/bfb0054868. ISBN 978-3-540-64657-0.(1998)

\bibitem{ref9}Holzer, P.; Wunderer, T.; Buchmann, J.A.: Recovering short generators of principal fractional ideals in cyclotomic fields of conductor $p^{\alpha}q^{\beta}$. In: Patra, A., Smart, N.P. (eds.) INDOCRYPT 2017. LNCS, vol. 10698, pp. 346–368. Springer, Cham. https://doi.org/10.1007/978-3-319-71667-1 18(2017)



\bibitem{ref14}Lenstra, A. K.; Lenstra, H. W., Jr.; Lovász, L. "Factoring polynomials with rational coefficients". Mathematische Annalen. 261 (4): 515–534. CiteSeerX 10.1.1.310.318. doi:10.1007/BF01457454. hdl:1887/3810. MR 0682664. S2CID 5701340.(1982)

\bibitem{LP19}  Liu, R., Pan, Y.: Computing Hermite Normal Form Faster via Solving System of Linear Equations. In: Proceedings of the
2019 International Symposium on Symbolic and Algebraic Computation. pp. 283-290. ACM (2019).

\bibitem{ref5}Lyubashevsky, V.; Peikert, C.; Regev, O. On Ideal Lattices and Learning with Errors over Rings. Advances in Cryptology – Eurocrypt 2010. Lecture Notes in Computer Science. Vol. 6110. pp. 1–23. CiteSeerX 10.1.1.352.8218. doi:10.1007/978-3-642-13190-5\_1. ISBN 978-3-642-13189-9.(2010-05-30)




\bibitem{MW01} Micciancio, D., Warinschi, B.: A Linear Space Algorithm for
Computing the Hermite Normal Form. In: Proceedings of the
2001 International Symposium on Symbolic and Algebraic Com-
putation. pp. 231-236. ACM (2001).

\bibitem{ref21}Pan, Y.; Xu, J.; Wadleigh, N.; Cheng, Q. On the Ideal Shortest Vector Problem over Random Rational Primes. Annual International Conference on the Theory and Applications of Cryptographic Techniques. EUROCRYPT 2021: Advances in Cryptology – EUROCRYPT 2021 pp 559–583. https://doi.org/10.1007/978-3-030-77870-5\_20 (2021)

\bibitem{ref6}Peikert, C., Regev, O., Stephens-Davidowitz, N. Pseudorandomness of ring-LWE for any ring and modulus. In: Proceedings of the 49th Annual ACM SIGACT Symposium on Theory of Computing, STOC 2017, pp. 461–473. https://doi.org/10.1145/3055399.3055489(2017)

\bibitem{ref10}Pellet-Mary, A.; Hanrot, G.; Stehlé, D. Approx-svp in ideal lattices with preprocessing. In: Annual International Conference on the Theory and Applications of Cryptographic Techniques. pp. 685–716. Springer(2019)

\bibitem{ref4}Regev, O. On lattices, learning with errors, random linear codes, and cryptography. Proceedings of the thirty-seventh annual ACM symposium on Theory of computing - STOC '05. ACM. pp. 84–93. CiteSeerX 10.1.1.110.4776. doi:10.1145/1060590.1060603. ISBN 978-1581139600. S2CID 53223958.(2005-01-01)

\bibitem{ref24}Sch\"{o}nhage, A.; Strassen, V. Schnelle Multiplikation grosser.  Zahlen Computing, 7:281-292 (1971)

\bibitem{ref15}Schnorr. A hierarchy of polynomial time lattice basis reduction algorithms. Theoretical Computer Science (1987)

\bibitem{ref12}Storjohann, A. The shifted number system for fast linear algebra on integer matrices. In Journal of Complexity, Pages 609-650.https://www.sciencedirect.com/science/article/pii/S0885064X05000312(2005)

\bibitem{ref17} von zur Gathen, J.;  Gerhard, J.  Mordern Computer Algebra. Cambridge University Press, 2 edition. (2003)


\bibitem{ref16} Zhang, Y.; Liu, R.; Lin, D. Improved Key Generation Algorithm for Gentry's Fully Homomorphic Ecryption Scheme. In Computer Science , Cryptography and Security.arXiv:1709.06724.(2017)


\end{thebibliography}
%\appendix
%\section{Computing an Incomplete Hermite Normal Form} \label{app:ihnf}




%\begin{lemma}\label{clemma}
%	For a non-singular matrix $\mathbf{B}\in\mathbb{Z}^{n \times n}$, the absolute value of whose entries is bounded by $2^B$, Algorithm \ref{alg:ihnf} takes $O(n^3B(\text{log}nB)(\text{loglog}nB))$ bits operation to get an Incomplete Hermite Normal Form of $\mathbf{B}$ by a unimodular transformation.
%\end{lemma}

%\begin{proof}		
%
%	
%	Then let's consider the multiplication operations in the algorithm. For convenience, we define $\mathbf{M}(B$) to be the complexity of the multiplication of two integers bounded by $2^B$ and $\mathbf{M}'(B$) to be  the complexity of the multiplication between an integer bounded by $2^B$ and  an integer vector of dimension n whose coefficients are bounded by $2^B$, too. 
%	
%	The Sch\"{o}nhage-Strassen algorithm \cite{ref24} allows $\mathbf{M}(B)=O(B(\text{log}B$)((\text{log} $\text{log}B$). Hence,$\mathbf{M}'(B$)=$n$$\mathbf{M}(B$)=$O(nB(\text{log}B$)(log $\text{log}B$)). There are (n-1)-times  iterations in step 1 and the complexity of the multiplication in the i-th iteration is $4\mathbf{M}'(iB$). Therefore, the complexity of multiplication in step 1 is 
%	\[\Sigma_{i=1}^{n-1}4\mathbf{M}'(iB)=O(n^3B(\text{log}nB)(loglognB))\]
%	
%	And the coefficients of $\mathbf{B}$ after step 1 can be bounded by $B^n$. Hence, step 1 takes
%	
%	$\text{max}\{(n-1)O((\text{log}B), n\text{log}B, O(n^3B((\text{log}nB)((\text{loglog}nB))\}$
%	
%	$=O(n^3B((\text{log}nB)((\text{loglog}nB)$
%\end{proof}

\end{document}
