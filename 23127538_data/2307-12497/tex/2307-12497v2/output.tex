%%
%% This is file `sample-lualatex.tex',
%% generated with the docstrip utility.
%%
%% The original source files were:
%%
%% samples.dtx  (with options: `sigconf')
%% 
%% IMPORTANT NOTICE:
%% 
%% For the copyright see the source file.
%% 
%% Any modified versions of this file must be renamed
%% with new filenames distinct from sample-lualatex.tex.
%% 
%% For distribution of the original source see the terms
%% for copying and modification in the file samples.dtx.
%% 
%% This generated file may be distributed as long as the
%% original source files, as listed above, are part of the
%% same distribution. (The sources need not necessarily be
%% in the same archive or directory.)
%%
%%
%% Commands for TeXCount
%TC:macro \cite [option:text,text]
%TC:macro \citep [option:text,text]
%TC:macro \citet [option:text,text]
%TC:envir table 0 1
%TC:envir table* 0 1
%TC:envir tabular [ignore] word
%TC:envir displaymath 0 word
%TC:envir math 0 word
%TC:envir comment 0 0
%%
%%
%% The first command in your LaTeX source must be the \documentclass
%% command.
%%
%% For submission and review of your manuscript please change the
%% command to \documentclass[manuscript, screen, review]{acmart}.
%%
%% When submitting camera ready or to TAPS, please change the command
%% to \documentclass[sigconf]{acmart} or whichever template is required
%% for your publication.
%%
%%
\documentclass[sigconf]{acmart}
\newtheorem{algorithm}{Algorithm}
\newtheorem{remark}{Remark}

\usepackage{hyperref}
\usepackage{xcolor}
\usepackage{tabularx}
\usepackage{booktabs}
\usepackage{subcaption}% for two table
\usepackage{algorithm}
\usepackage{algorithmicx}
\usepackage{algpseudocode}
\renewcommand{\algorithmicrequire}{\textbf{Input:}}
\renewcommand{\algorithmicensure}{\textbf{Output:}}
%%
%% \BibTeX command to typeset BibTeX logo in the docs
\AtBeginDocument{%
  \providecommand\BibTeX{{%
    Bib\TeX}}}

%% Rights management information.  This information is sent to you
%% when you complete the rights form.  These commands have SAMPLE
%% values in them; it is your responsibility as an author to replace
%% the commands and values with those provided to you when you
%% complete the rights form.
\setcopyright{acmlicensed}
\copyrightyear{2024}
\acmYear{2024}
\acmDOI{}

%% These commands are for a PROCEEDINGS abstract or paper.
\acmConference[]{}{}{ }
%%
%%  Uncomment \acmBooktitle if the title of the proceedings is different
%%  from ``Proceedings of ...''!
%%
%%\acmBooktitle{Woodstock '18: ACM Symposium on Neural Gaze Detection,
%%  June 03--05, 2018, Woodstock, NY}
\acmISBN{}


%%
%% Submission ID.
%% Use this when submitting an article to a sponsored event. You'll
%% receive a unique submission ID from the organizers
%% of the event, and this ID should be used as the parameter to this command.
%%\acmSubmissionID{123-A56-BU3}

%%
%% For managing citations, it is recommended to use bibliography
%% files in BibTeX format.
%%
%% You can then either use BibTeX with the ACM-Reference-Format style,
%% or BibLaTeX with the acmnumeric or acmauthoryear sytles, that include
%% support for advanced citation of software artefact from the
%% biblatex-software package, also separately available on CTAN.
%%
%% Look at the sample-*-biblatex.tex files for templates showcasing
%% the biblatex styles.
%%

%%
%% The majority of ACM publications use numbered citations and
%% references.  The command \citestyle{authoryear} switches to the
%% "author year" style.
%%
%% If you are preparing content for an event
%% sponsored by ACM SIGGRAPH, you must use the "author year" style of
%% citations and references.
%% Uncommenting
%% the next command will enable that style.
%%\citestyle{acmauthoryear}


%%
%% end of the preamble, start of the body of the document source.
\begin{document}

%%
%% The "title" command has an optional parameter,
%% allowing the author to define a "short title" to be used in page headers.
\title{Embedding Integer Lattices as Ideals into  Polynomial Rings}

%%
%% The "author" command and its associated commands are used to define
%% the authors and their affiliations.
%% Of note is the shared affiliation of the first two authors, and the
%% "authornote" and "authornotemark" commands
%% used to denote shared contribution to the research.
\author{Yihang Cheng, Yansong Feng and Yanbin Pan}
\affiliation{%
  \institution{Key Laboratory of Mathematics Mechanization, Academy of Mathematics and Systems Science, Chinese Academy of Sciences\\
  School of Mathematical Sciences, University of Chinese Academy of Sciences}
  \streetaddress{Zhongguancun East Road 55}
  \city{Beijing 100190}
  \country{China}
}
\email{chengyihang15@mails.ucas.ac.cn, {fengyansong, panyanbin}@amss.ac.cn}
%\author{Yansong Feng}
%\email{fengyansong@amss.ac.cn}
%\affiliation{%
  %\institution{Key Laboratory of Mathematics Mechanization, Academy of Mathematics and Systems Science, Chinese Academy of Sciences, Beijing, China \\
%School of Mathematical Sciences, University of Chinese Academy of Sciences, Beijing, China}
  %\streetaddress{Zhongguancun East Road 55}
  %\city{Beijing}
  %\country{China}
  %\postcode{100190}
%}

%\author{Yanbin Pan}
%\email{ panyanbin@amss.ac.cn}

%\affiliation{%
  %\institution{Key Laboratory of Mathematics Mechanization, Academy of Mathematics and Systems Science, Chinese Academy of Sciences, Beijing, China \\
%School of Mathematical Sciences, University of Chinese Academy of Sciences, Beijing, China}
  %\streetaddress{Zhongguancun East Road 55}
  %\city{Beijing}
  %\country{China}
  %\postcode{100190}
%}

%%
%% By default, the full list of authors will be used in the page
%% headers. Often, this list is too long, and will overlap
%% other information printed in the page headers. This command allows
%% the author to define a more concise list
%% of authors' names for this purpose.
\renewcommand{\shortauthors}{Cheng et al.}

%%
%% The abstract is a short summary of the work to be presented in the
%% article.
\begin{abstract}
  Many lattice-based crypstosystems employ ideal lattices for high efficiency. However, the additional algebraic structure of ideal lattices usually makes us worry about the security, and it is widely believed that the algebraic structure will help us solve the hard problems in ideal lattices more efficiently. In this paper, we study the additional algebraic structure of ideal lattices further and find that a given ideal lattice in a polynomial ring can be embedded as  an ideal into infinitely many different polynomial rings by the coefficient embedding. We design an algorithm to verify whether a given full-rank lattice in $\mathbb{Z}^n$ is an ideal lattice and output all the polynomial rings that the given lattice can be embedded into as an ideal with time complexity $\mathcal{O}(n^3B(B+\log n)$, where $n$ is the dimension of the lattice and $B$ is the upper bound of the bit length of the entries of the input lattice basis. We would like to point out that Ding and Lindner proposed an algorithm for identifying ideal lattices and outputting a single polynomial ring that the input lattice can be embedded into with time complexity $\mathcal{O}(n^5B^2)$ in 2007. However, we find a flaw in Ding and Lindner's algorithm that causes some ideal lattices can't be identified by their algorithm.
  
  %In 2007, Ding and Linder gave an algorithm for identifying ideal lattices with the time complexity $\mathcal{O}(n^5B^2)$. However, we find there is a flaw in Ding and Linder's algorithm. Some ideal lattices can't be identified as ideal lattices in Ding and Linder's algorithm and we give a non-trivial example. Ignoring the flaw, our algorithm also has two advantages over Ding and Linder's algorithm: 1. Our algorithm is more efficient with the time complexity $\mathcal{O}(n^3B(B+\log n)$; 2. Ding and Linder's algorithm only presented one polynomial ring while ours reveals all the polynomial rings that the given lattice can be embedded into as an ideal.
\end{abstract}

%%
%% The code below is generated by the tool at http://dl.acm.org/ccs.cfm.
%% Please copy and paste the code instead of the example below.
%%
\begin{CCSXML}
<ccs2012>
   <concept>
       <concept_id>10002950.10003624</concept_id>
       <concept_desc>Mathematics of computing~Discrete mathematics</concept_desc>
       <concept_significance>500</concept_significance>
       </concept>
   <concept>
       <concept_id>10002978.10002979.10002985</concept_id>
       <concept_desc>Security and privacy~Mathematical foundations of cryptography</concept_desc>
       <concept_significance>500</concept_significance>
       </concept>
   <concept>
       <concept_id>10003752.10003809</concept_id>
       <concept_desc>Theory of computation~Design and analysis of algorithms</concept_desc>
       <concept_significance>500</concept_significance>
       </concept>
 </ccs2012>
\end{CCSXML}

\ccsdesc[500]{Mathematics of computing~Discrete mathematics}
\ccsdesc[500]{Security and privacy~Mathematical foundations of cryptography}
\ccsdesc[500]{Theory of computation~Design and analysis of algorithms}


%%
%% Keywords. The author(s) should pick words that accurately describe
%% the work being presented. Separate the keywords with commas.
\keywords{Ideal lattice, Coefficient embedding, Complexity}


%%

%%
%% This command processes the author and affiliation and title
%% information and builds the first part of the formatted document.
\maketitle

\section{Introduction}
\subsection{The Development of Ideal Lattices}
The research on lattice-based cryptography was pioneered by Ajtai \cite{ref1} in 1996. He presented a family of one-way function  based on the Short Integer Solution (SIS) problem, which has the average-case hardness under the worst-case assumptions for some lattice problems. In 1997,  Ajtai and Dwork \cite{ref2} introduced a public-key cryptosystem, whose average-case security can be based on the worst-case hardness of the unique-Shortest Vector Problem.  In 2005,  Regev \cite{ref4} proposed another problem with average-case hardness,  the Learning with Errors problem (LWE), and also a  public-key encryption scheme based on LWE. Because of the average-case security, lattice-based cryptography has drawn considerable attentions from then on. 

Although there have been many cryptographic schemes based on LWE and SIS,   the main drawback of such schemes is their limited efficiency, due to its large key size and slow computations. Especially, as  the development of quantum computers, it becomes more urgent to design more practical lattice-based cryptosystems, since lattice-based cryptosystems are widely believed to be quantum-resistant.  
To improve the efficiency, additional algebraic structure is involved in the  lattice to construct more practical schemes. Among them,  ideal lattice plays an important role. 
     
In fact, as early as in 1998, Hoffstein, Pipher, and Silverman \cite{ref3} introduced a lattice-based public-key encryption scheme known as NTRU, whose security is related to the ideal in the ring $\mathbb{Z}[x]/(x^n-1)$. Due to the cyclic structure of the ideal lattice, the efficiency of NTRU is very high. Later, in 2010, Lyubashevsky, Peikert and Regev \cite{ref5} presented a ring-based variant of LWE, called Ring-LWE, whose average-case  hardness is based on worst-case assumptions on ideal lattices. In 2017, Peikert, Regev and Stephens-Davidowitz \cite{ref6} refined the proof of the security of Ring-LWE for more algebraic number field. After the introduction of Ring-LWE, more and more practical cryptosystems based on ideal lattices have be constructed.

  
     There are two different ways to define ideal lattices. 
     
     One is induced by the coefficient embedding from ring $\mathbb{Z}[x]/f(x)$ into $\mathbb{Z}^n$. NTRU uses coefficient embedding to define its lattice. It is very convenient to implement cryptosystems based on  Ring-LWE with the coefficient embedding. In fact, almost all the ideal lattice-based cryptosystems are implemented via the coefficient embedding. However, it seems not easy to clarify the hardness of  problems for the coefficient-embedding ideal lattice in general. 
      
      
     The other one is defined by the canonical embedding from the algebraic integer ring of some number field $K$ into $\mathbb{C}^n$. This type of ideal lattice is usually employed in the security proof or hardness reduction in Ring-LWE based cryptography.
     
     It is widely believed that the additional algebraic structure of ideal lattice will help us solve its hard problems  more efficiently. 
     
     In 2016, Cramer, Ducas, Peikert and Regev \cite{ref7} introduced a polynomial-time quantum algorithm to solve $2^{\sqrt{n\text{log}n}}$-SVP in principal ideal lattices in the algebraic integer ring of $\mathbb{Q}(\zeta_m)$, where $m$ is a power of some prime. In 2017, Cramer, Ducas and Wesolowski \cite{ref8} extended the result to general ideals. In the same year,  Holzer, Wunderer and Buchmann \cite{ref9} extended the field to be $\mathbb{Q}(\zeta_m)$, where $m=p^aq^b$ and $p$, $q$ are different primes. 
     
     In 2019, Pellet-Mary, Hanrot and Stehl\'{e} \cite{ref10} introduced a pre-processing method (PHS algorithm) to solve $\gamma$-SVP for ideal lattices in any number field. The pre-processing phasing takes exponential time. Let $n$ be the dimension of the number field $K$ viewed as a $\mathbb{Q}$-vector space. Pellet-Mary \textit{et al.} showed that by performing pre-processing on $K$ in exponential time, their algorithm can, given any ideal lattice $I$ of $O_K$, for any $\alpha \in [0,1/2]$ output a $\exp(\widetilde{O}((n\log n)^{\alpha+1}/n))$ approximation of a shortest none-zero vector of $I$ in time $\exp(\widetilde{O}((n\log n)^{1-2\alpha}/n))+T$.  For the classical method, $T=\exp(\widetilde{O}((n\log n)^{1/2})$ if $K$ is  a cyclotomic field or $T=\exp(\widetilde{O}((n\log n)^{2/3})$ for an arbitrary number field $K$.
     
     In 2020, Bernard and Roux-Langlois \cite{ref19} proposed a new “twisted” version of the PHS  algorithm. They proved that Twisted-PHS algorithm performs at least as well as the original PHS algorithm and their algorithm suggested that much better approximation factors were achieved. In 2022, Bernard,  Lesavourey,  Nguyen and  Roux-Langlois \cite{ref20} extended the experiments of \cite{ref19} to cyclotomic fields of degree up to 210 for most conductors $m$. 
     
 In 2021, Pan, Xu, Wadleigh and Cheng \cite{ref21} found the connection between the complexity of the shortest vector problem (SVP) of prime ideals in number fields and their decomposition groups, and revealed lots of weak instances of  ideal lattices in which SVP can be solved efficiently. In 2022, Boudgoust, Gachon and Pellet-Mary \cite{ref22} generalized the work of Pan \textit{et al.} \cite{ref21} and provided a simple condition under which an ideal lattice defines an easy instance of the shortest vector problem. Namely, they showed that the more automorphisms stabilize the ideal, the easier it was to find a short vector in it.
 
 As mentioned above, almost all the research on SVP is in the canonical-embedding ideal lattices and the research on SVP in the coefficient-embedding ideal lattices is few.  However, in some rings, such as $\mathbb{Z}[X]/(x^n+1)$ where $n=2^k$ for $k\geq 1$, the SVPs induced by the two different embeddings are almost equal. We refer to  \cite{ref11} for more details.
 
 %We recall that a number field $K$ is called monogenic if $O_K=\mathbb{Z}[\alpha]$ for some $\alpha \in K$. All the cyclotomic and quadratic fields are monogenic. Only in monogenic fields, $O_K$ is isomorphic to a polynomial ring $\mathbb{Z}[x]/f(x)$ for some monic irreducible integer polynomial $f(x)$ and only in this case the ideal lattices induced by the coefficient embedding have the same algebraic structure with the one induced by canonical embedding of the same ideal in the sense of ring isomorphism. 

%In 2017, Baston \cite{ref11} discussed the norm connection between the coefficient embedding and the canonical embedding in the cyclotomic fields. Let $K=Q(\zeta_m)$ and for any ideal $I\in O_K$, let $T$ be the transformation matrix from the coefficient-embedding lattice $\mathcal{L}(B)$ to the canonical-embedding lattice $\mathcal{L}(B')$, which means $TB=B'$. Consider the singular value decomposition (SVD) of $T$ and \[T=U\begin{pmatrix}s_1&0&\cdots&0\\0&s_2&\cdots&0\\0&\vdots&\ddots&0\\0&0&\cdots&s_n\end{pmatrix}V\]$s_1\geq s_2\geq \cdots \geq s_n>0$, $V$ and $U$ unitary matrices. Define $k_2=\frac{s_1}{s_n}$ and Baston showed that the less $k_2$ is the more relevant of the SVPs are in the ideal lattices induced by two different embeddings of the same ideal. 	 

%More specifically, according to Lemma 3.5 of \cite{ref11}, given $T\in \mathbb{C}^{n\times n}$ and $x\in \mathbb{C}^n$, $s_1,s_2,\cdots,s_n$ is the singular value of $T$, then  $s_n(T)\cdot\|x\|_2\leq\|Tx\|_2\leq s_1(T)\cdot\|x\|_2$. This conclusion shows a direct reduction between the SVPs in two ideal lattices induced by different embeddings of a fixed ideal. Using the notion above, if there is an oracle to solve $\gamma$-SVP in $\mathcal{L}(B)$, then combining Lemma 3.5 of \cite{ref11} and the relation $TB=B'$ we can solve $k_2 \gamma$-SVP in $\mathcal{L}(B')$ in polynomial time.  The similar result still holds if we interchange $\mathbf{B}$ and $\mathbf{B'}$.

%In Theorem 3.1 of \cite{ref11}, Baston showed that $k_2$ is only related to the monogenic number field $K$ and has nothing to do with the chosen ideal or fractional ideal. Though the research on $k_2$ in general monogenic number field is little, when $K=Q(\zeta_m)$ is a cyclotomic field, $k_2=(\text{rad}(m))^{1/2}$ for $m$ is odd and $k_2=(\text{rad}(m)/2)^{1/2}$ for $m$ is even, where rad(m) means the different prime multiple of m (see Lemma 3.4 of \cite{ref11}).  Therefore when $m=a^l$ with $l$ large enough, the SVPs in ideal lattices induced by two embeddings of the same ideal are connected closely. When $m=2^l$ for any $l\geq2$, $k_2=1$ and the SVPs in two embedding ideal lattices are essentially the same.

%With the discussion of the Baston's results above, in some monogenic number fields especially some cyclotomic fields, we can use the results of the research on $\gamma$-SVP in canonical-embedding ideal lattices to handle the $\gamma$-SVP in the coefficient-embedding ideal lattices.


\subsection{Our contribution}
In this paper, our main contribution is to find that an ideal lattice in the ring $\mathbb{Z}[x]/f(x)$ can be embedded into infinitely many rings $\mathbb{Z}[x]/g(x)$ as ideals, where $f(x)$ and $g(x)$ are monic and $f(x)$, $g(x)\in\mathbb{Z}[x]$ (Theorem~\ref{kthm}). Besides, corresponding to our finding, we show an efficient algorithm for computing all the rings that an ideal lattice can be embedded into as ideals and also judging whether a given integer lattice can be embedded as an ideal into a polynomial ring like $\mathbb{Z}[x]/f(x)$ with time complexity $\mathcal{O}(n^3B(B+\log n)$, where $n$ is the dimension of the lattice and $B$ is the upper bound of the bit length of the entries of the input lattice basis (Algorithm~\ref{alg:identify}).
%In this paper, we find an interesting property of the coefficient-embedding ideal lattice. Our main contribution is to show that an ideal lattice in the ring $\mathbb{Z}[x]/f(x)$, where $f(x)$ is monic and $f(x)\in\mathbb{Z}[x]$, can be embedded into infinitely many rings $\mathbb{Z}[x]/g(x)$ as ideals of the rings $\mathbb{Z}[x]/g(x)$, where $g(x)$ is monic and $g(x)\in\mathbb{Z}[x]$ (Theorem 1). And then we give some promising applications of this property. At last, we show an efficient algorithm for computing all the rings that an ideal lattice can be embedded into as ideals and also judging whether a given integer lattice can be embedded as an ideal into a polynomial ring like $\mathbb{Z}[x]/f(x)$, where $f(x)$ is monic and $f(x)\in\mathbb{Z}[x]$.(Algorithm 1).

Although, in 2007, Ding and Lindner \cite{ref13} proposed an algorithm for identifying ideal lattice that output a single polynomial ring which the input lattice can be embedded into as an ideal with time complexity $\mathcal{O}(n^5B^2)$, we find that there is a flaw in Ding and Lindner's algorithm. More exactly, some ideal lattices can't be identified by their algorithm and we give a non-trivial example in Section 4. Besides, ignoring the flaw, our algorithm is more efficient and output more polynomial rings than Ding and Lindner's algorithm.

On one hand, our finding reveals that an ideal lattice in $\mathbb{Z}[x]/f(x)$ can be viewed as an ideal lattice in $\mathbb{Z}[x]/g(x)$ for infinitely many different $g(x)$ and it is widely believed that some additional algebraic structures may lead a more efficient algorithm to solve the hard problems in ideal lattice than  general lattices, such as \cite{ref7}, \cite{ref10}.  Hence, we may embed the given ideal lattice into a well-studied ring as an ideal lattice and use the algebraic structure of the well-studied ring to solve the hard lattice problems more efficiently.

On the other hand, we test the proportion of ideal lattices in plain integer lattices by our algorithm and find that the proportion decreases very fast with the increase of the lattice dimension and upper bound of the bit length of the entries of the input lattice basis. Our test data indicates that the ideal lattice is actually very rare.

Finally, we provide an efficient open source implementation of our algorithm for identifying ideal lattices in SageMath. The source code is available at:\begin{center}
  \textcolor{blue}{\url{https://github.com/fffmath/Identifying-Ideal-Lattice}}.
\end{center}
With this implementation, we conducted several experiments, and the experimental results are presented in Appendix~\ref{AppendixA}.
%As we all know, a lattice is actually a discrete additive subgroup of $\mathbb{R}^{m}$. The only difference between the general integer lattice and the ideal lattice is the multiplication structure of the ideal lattice. In fact, an integer lattice may be embedded into a polynomial ring $\mathbb{Z}[x]/f(x)$, and it can be viewed as an ideal of $\mathbb{Z}[x]/f(x)$.  Hence, with this embedding, the integer lattice as an ideal of  $\mathbb{Z}[x]/f(x)$ is equipped with the multiplication of the ring $\mathbb{Z}[x]/f(x)$. A natural question is that what will happen if we equip the same lattice with  multiplication of different polynomial rings or is such polynomial ring unique?

%We show that it is possible to embed a given ideal lattice  as another ideal into infinitely many different polynomial rings by the coefficient embedding. For any ideal lattice, we explicitly present all the polynomial rings that the given ideal lattice can be embedded into as ideals. 

%We stress that these embeddings won't change the original lattice at all, which means all the hard lattice problems remain the same after these embeddings. And what do change after these embeddings is the multiplication structure of the original ideal lattice. More specifically, if the original ideal lattice is $\mathcal{L}(\mathbf{B})$ in the polynomial ring  $\mathbb{Z}[x]/f(x)$, $f(x)$ monic and $f(x)\in\mathbb{Z}[x]$, then our discussion shows that $\mathcal{L}(\mathbf{B})$ can be viewed as an ideal in another polynomial ring $\mathbb{Z}[x]/g(x)$, $g(x)$ monic and $g(x)\in\mathbb{Z}[x]$. Hence, these embeddings change the multiplication of the given ideal lattice $\mathcal{L}(\mathbf{B})$ from the multiplication of the ring $\mathbb{Z}[x]/f(x)$ to the multiplication of the ring $\mathbb{Z}[x]/g(x)$ and won't change the additive structure of the lattice. 


%It is widely believed that additional algebraic structure may lead a more efficient algorithm to solve the hard problems in ideal lattice than  general lattices, such as the method of recovering a short generator proposed by Cramer  \textit{et al.} \cite{ref7} and the method of pre-processing any number field $K$ proposed by Pellet-Mary \textit{et al.} \cite{ref10}. 

%However, unfortunately for us, for now the application of our discovery to handle the hard lattice problems is very limited. Firstly, the researches mentioned above are all in the canonical-embedding ideal lattices, so our discovery in the coefficient-embedding ideal lattices can not make use of the results in canonical-embedding ideal lattices directly. Secondly, because the researches on the hard lattice problems in coefficient-embedding ideal lattices are few, compared to canonical-embedding ideal lattices, the current tools are very limited, and we don't know exactly which kinds of polynomial rings like $\mathbb{Z}[x]/f(x)$ with $f(x)$ monic have a good structure that help to deal with the hard lattice problems  more easily in the coefficient-embedding ideal lattices than those in plain lattices.


%But our discovery is also full of development potential. In some rings, the results of the $\gamma$-SVP in canonical-embeddinng ideal lattices can fit the cases in coefficient-embedding ideal lattices.  By the discussion of Baston's \cite{ref11} above in some special cyclotomic fields the connection between the $\gamma$-SVPs in ideal lattices induced by two different embeddings of the same ideal is very close, which means solving the $\gamma$-SVP in one embedding is equal to solving the $\beta$-SVP in the other embedding, where $\gamma$ and $\beta$ are close. 

%Besides, our discussion makes the research on the $\gamma$-SVP in the coefficient-embedding ideal lattice more valuable. If we happen to find that the algebraic structure for  polynomial ring  $R$  that can help us  solve the ideal lattice problems in $R$ more efficiently, then our results shows that it's possible to solve the problem for other ideal lattices not in $R$ as long as the ideal lattices can be embedded as ideals into $R$. 
 
 %In some cases, our property can expand the research on $\gamma$-SVP in the canonical-embedding ideal lattice. As discussed above, in some monogenic fields, especially some cyclotomic fields, the $\gamma$-SVPs in the lattices induced by the two different embeddings of the same ideal are connected closely.  Hence, we assume $R$ is such a  good algebraic integer ring mentioned above, and when using the method in  \cite{ref10}, our property shows that the pre-processing of   $R$ can also be used to solve the problems for some coefficient-embedding ideal lattices not in $R$.


 %Once we find a weak ideal lattice in which the lattice problem can be solved more efficiently,  we can solve the problems for infinite ideal lattices in different polynomial rings. Though the integer lattice is fixed, they are different as ideals in different polynomial rings. It seems that a weak ideal will spread as infinite weak ideals. Hence, our discovery points out when considering the security, it's necessary to evaluate all the corresponding ideals in the polynomial rings that the given ideal lattices can be embedded into instead of just one single ideal lattice.


%On the other hand, the abundant embedding relations will impact the security of the crypstosystems based on ideal lattices. When considering the security, it's necessary to evaluate all the corresponding ideals in the polynomial rings that the given ideal lattices can be embedded into instead of just one single ideal lattice.

%We have to point out that all of the observations above shed a shadow on the security of ideal lattice-based cryptosystems. 

%As a by-product, an efficient algorithm is introduced to identify an ideal lattice and give the class of all the rings that the original lattice can be embedded into as different ideals. We first show an equivalent condition between the integer lattice and the coefficient-embedding ideal lattice. According to this condition, we introduce a  polynomial-time algorithm that is more efficient than the algorithm proposed by Ding and Lindner \cite{ref13}. Moreover, we present an explicit form for all possible polynomial rings that the ideal lattice can be embedded into in theory instead of the implicit form obtained in the experiment as in \cite{ref13}. The explicit form can help us theoretically analyze the algebraic properties of there ideals directly.


%Hence, it's not proper anymore to judge the security of a crypstosystem based on ideal lattice by a single ring. And the embedding relation no doubts increases the difficulties of the security proof for any crypstosystem based on ideal lattices. Instead of relying on the worst case of the given ring, we must evaluate the whole rings class that the used ideal lattice can be embedded into.
%
%However, avoiding the use of certain well-studied rings doesn't means it's more secure, since we find that 

\subsection{Roadmap}The paper is organized as follows. In Section 2, some preliminaries are presented. In Section 3, we show embedding relation between integer lattices and polynomial rings, and the theoretic basis of Algorithm \ref{alg:identify} is also presented. In Section 4, we propose the algorithm for identifying a coefficient-embedding ideal lattice together with the complexity analysis and the comparison to Ding and Lindner's algorithm. The appendix contains our experimental results and reference.

\section{Preliminaries}
\subsection{Notation}

In this paper we denote by $\mathbb{C}$, $\mathbb{R}$, $\mathbb{Q}$ and $\mathbb{Z}$ the complex number field, the real number field, the rational number field and the integer ring respectively.

We denote a matrix by a  capital letter in bold and denote a vector by a lower-case letter in bold. To  represent the element of a matrix, we use the lower-case letter. For example,  the element of matrix $\mathbf{A}$ at the $i$-th row and $j$-th column is denoted by $a_{ij}$, while its $i$-th row is denoted by $\mathbf{a}_i$. Since we have the inner products in $\mathbb{R}^n$ and $\mathbb{C}^n$ respectively, we can define the norm of vectors,  that is, $ \Vert \mathbf{v} \Vert :=<\mathbf{v},\mathbf{v}>$ in $\mathbb{R}^n$ and  $ \Vert \mathbf{v} \Vert :=<\mathbf{v},\overline{\mathbf{v}}>$ in $\mathbb{C}^n$.

For two integers $a$ and $b$, $a| b$ means that $b$ is divisible by $a$. Otherwise, we write $a\not| \ b$. For integer $a$ and a matrix $\mathbf{A}$, $a| \mathbf{A}$ means that every entry of $\mathbf{A}$ can be divisible by $a$.
%Given a set $H$ of vectors, span($H$) means the space generated by the elements of $H$. For example, $\text{span}(\mathbf{A})=\{\sum_{i=1}^n x_i\mathbf{a}_i : x_i\in\mathbb{R}\}$. Denote by dim(span($H$)) the dimension of  span($H$).


%Denote by $\mathcal{B}(x,r)$ the ball centered at $x$ with radius $r$. 

For a polynomial $f(x)\in\mathbb{Z}[x]$, denote by $\mathbb{Z}[x]/f(x)$ for simplicity the quotient ring $\mathbb{Z}[x]/(f(x)\mathbb{Z}[x])$.

For a map $\sigma$, and a set $S$, denote by $\sigma(S)$ the set $\{\sigma(x):x\in S\}$.


 
 
 
\subsection{Lattice}
Lattices are  discrete subgroups of $ \mathbb{R}^m$, or equivalently,
\begin{definition}{(Lattice)}
Given n linearly independent vectors $\mathbf{B}=\begin{pmatrix}\mathbf{b}_1 \\ \mathbf{b}_2 \\\vdots \\ \mathbf{b}_n\end{pmatrix}$, where $\mathbf{b}_i\in\mathbb{R}^m$, the lattice $\mathcal{L}(\mathbf{B})$ generated by $\mathbf{B}$ is defined as  follows: \[\mathcal{L}(\mathbf{B})=\{\sum_{i=1}^n x_i\mathbf{b}_i : x_i\in\mathbb{Z}\}=\{\mathbf{xB} : \mathbf{x}\in\mathbb{Z}^n\}.\] 	
\end{definition}
We call $\mathbf{B}$ a basis of $\mathcal{L}(\mathbf{B})$, $m$ and $n$ the dimension and the rank of $\mathcal{L}(\mathbf{B})$ respectively. When $m=n$, we say $\mathcal{L}(\mathbf{B})$ is full-rank.

When $n>1$, there are infinitely many bases for a lattice $\mathbf{\mathcal{L}}$, and any two bases are related to each other by a unimodular matrix, which is an invertible integer matrix. More precisely, given a lattice $\mathcal{L}(\mathbf{B}_1)$, $\mathbf{B}_2$ is also a basis of the lattice if and only if there exists a unimodular matrix $\mathbf{U}$ s.t. $\mathbf{B}_1=\mathbf{U} \mathbf{B}_2$.

\paragraph{Hard problems in lattices}


The shortest vector problem (SVP)  is one  of the most famous hard problems in lattices.

SVP is the  question of finding a nonzero shortest  vector in a given lattice $\mathcal{L}$, whose length is denoted by $\lambda_1(\mathcal{L})$. 
The approximating-SVP with factor $\gamma$, denoted by $\gamma$-SVP, asks to find a short nonzero lattice vector $\mathbf{v}$ such that $$\|\mathbf{v}\|\le\gamma\cdot\lambda_1(\mathcal{L}).$$

In fact, The hardness of $\gamma$-SVP depends on $\gamma$. When $\gamma=1$, $\gamma$-SVP is exactly the original SVP, and for constant $\gamma$, this problem is known to be NP-hard under randomized reduction \cite{ajtai98}. 
Many cryptosystems are based on the hardness of (decision) $\gamma$-SVP when $\gamma$ is in polynomial size. By now we have not found any  polynomial-time classical algorithm to deal with such cases. The existing polynomial algorithms such as LLL \cite{ref14}, BKZ \cite{ref15} can find the situation when $\gamma =\text{exp}(n)$. 




 \subsection{Hermite Normal Form And Smith Normal Form}
For the integer matrix, there is a very important standard form known as the Hermite Normal Form (HNF). For simplicity, we just present the definition of HNF for the non-singular integer matrix.
\begin{definition}{(Hermite Normal Form)} A non-singular matrix $\mathbf{H}\in\mathbb{Z}^{n \times n}$ is said to be in HNF, if
	\begin{itemize}
		\item $h_{i,i} > 0$ for $1\leq i\leq n.$
		\item $h_{j,i} =0 $ for $1\leq j < i \leq n.$
		\item $0 \leq h_{j,i}< h_{i,i}$ for $1\leq i< j\leq n.$
	\end{itemize}
\end{definition}


The Hermite Normal Form has some important properties. See \cite{DKT87,MW01,LP19} for more details. 
\begin{lemma}
For any  integer matrix $\mathbf{A}$, there exists a unimodular matrix $\mathbf{U}$ such that $\mathbf{H}$=$\mathbf{U} \mathbf{A}$ is in HNF. Moreover, HNF can be computed in polynomial time.
\end{lemma}

For integer lattices, we have 
\begin{lemma}\label{hnfbasis}
For any lattice $\mathcal{L}\subset \mathbb{Z}^n$, there exists a unique basis $\mathbf{H}$ in HNF. We call $\mathbf{H}$ the HNF basis of $\mathcal{L}$.
\end{lemma}


  Sometimes we do not need the whole HNF of an integer matrix. So we  introduce the Incomplete Hermite Normal Form of an integer matrix, which is also  a special basis of the integer lattice.
  
  \begin{definition}{(Incomplete Hermite Normal Form)} A non-singular matrix $\mathbf{B}\in\mathbb{Z}^{n \times n}$ is said to be in Incomplete Hermite Normal Form, if
  \begin{itemize}
  	\item $b_{n,n}>0$;
  	\item $b_{i,n} = 0 \mbox{ for } 1\leq i\leq n-1.$
  \end{itemize}
  \end{definition}

Given a full-rank integer matrix  $\mathbf{B}$,
\[ \mathbf{B}= \begin{pmatrix} b_{1,1}&b_{1,2}&\cdots&b_{1,n} \\ b_{2,1}&b_{2,2}&\cdots&b_{2,n} \\ \vdots&\vdots&\ddots&\vdots \\b_{n,1}&b_{n,2}&\cdots&b_{n,n} \end{pmatrix},\]
it is well known that by the Extended  Euclidean Algorithm we can find a unimodular matrix $\mathbf{U}$, such that 
\[ \mathbf{U} \begin{pmatrix} b_{1,n} \\ b_{2,n} \\ \vdots \\b_{n,n} \end{pmatrix} = \begin{pmatrix}0 \\ 0 \\ \vdots \\d \end{pmatrix},\]
where $d = \gcd( b_{1,n}, b_{2,n},...,b_{n,n})$. Then we have 
$$ \mathbf{B}'= \mathbf{U} \mathbf{B} =  \begin{pmatrix} \mathbf{D}&\mathbf{0} \\ \mathbf{b'}& d \end{pmatrix} $$
is in Incomplete Hermite Normal Form, where  $\mathbf{D} \in \mathbb{Z}^{(n-1) \times (n-1)}$, $\mathbf{b}' \in \mathbb{Z}^{n-1}$. 

About the Incomplete Hermite Normal Form, it is easy to conclude the following lemma. So we omit the proof.

\begin{lemma} \label{ihnflemma}
	For any non-singular matrix $\mathbf{B}\in\mathbb{Z}^{n \times n}$, the following properties are satisfied:
	\begin{itemize}
		\item we can find a unimodular matrix $\mathbf{U}$ in polynomial time, such that $ \mathbf{B}'= \mathbf{U} \mathbf{B}$ is in Incomplete Hermite Normal Form.
		\item For any unimodular matrix $\mathbf{U}$ and $\mathbf{V}$ such that $ \mathbf{B}'= \mathbf{U} \mathbf{B}$ and $ \mathbf{B}''= \mathbf{V} \mathbf{B}$ both in Incomplete Hermite Normal Form,  $ \mathbf{B}'$ and $ \mathbf{B}''$ are not necessarily equal, but $$b'_{n,n} = b''_{n,n} = \gcd( b_{1,n}, b_{2,n},...,b_{n,n}).$$  Specially, notice that the HNF  $\mathbf{H}$ of $\mathbf{B}$ is also in  Incomplete Hermite Normal Form. We immediately have
		$$ h_{n,n} = \gcd( b_{1,n}, b_{2,n},...,b_{n,n}).$$
	\end{itemize}
\end{lemma}

\begin{definition}{(Smith Normal form)}Let $\mathbf{A}$ be nonzero $m\times n$ matrix over a principal ideal domain $R$, there exist invertible $m\times m$ and $n\times n$-matrices $\mathbf{P},\mathbf{T}$ (with coefficients in $R$) such that the product \[\mathbf{S}=\mathbf{PAT}=\begin{pmatrix}\alpha_1&0&0&\cdots&0\\0&\alpha_2&0&\cdots&0\\0&0&\ddots&\ddots&0\\\vdots&\vdots&\ddots&\alpha_r&\vdots\\\vdots&\vdots&0&0&\vdots\\0&\cdots&\cdots&\cdots& 0\end{pmatrix}\] And the diagonal elements satisfy $\alpha_i|\alpha_{i+1}$ for all $1\leq i < r$. $\mathbf{S}$ is the Smith Normal Form of $\mathbf{A}$, and the elements $\alpha _{i}$ are unique up to multiplication by a unit in $R$ and are called the elementary divisors, invariants, or invariant factors.
\end{definition}

\begin{definition}{(Smith Massager)}Let $\mathbf{A}\in \mathbb{Z}^{n\times n}$ be a non-singular integer matrix with Smith Normal Form $\mathbf{S}$. A matrix $\mathbf{M}\in\mathbb{Z}^{n\times n}$ is a Smith Massager for $\mathbf{A}$ if 

(i) it satisfies that $\mathbf{AM}\equiv 0$ cmod $\mathbf{S}$, and

(ii) there exists a matrix $\mathbf{W}\in \mathbb{Z}^{n\times n} $ such that $\mathbf{WM}\equiv \mathbf{I_n}$ \text{cmod} $\mathbf{S}$.
\end{definition}
\begin{definition}{(cmod)}
Given $\mathbf{B}\in\mathbb{Z}^{m\times n}$ and $\mathbf{S}\in\mathbb{Z}^{n\times n}$, where \[\mathbf{B}=\begin{pmatrix}
    \mathbf{b_1}&\mathbf{b_2}&\cdots&\mathbf{b_n}
\end{pmatrix}\]
\[\mathbf{S}=\begin{pmatrix}
    s_1&0&\cdots&0\\0&s_2&\cdots&0\\\vdots&\vdots&\ddots&\vdots\\0&0&\cdots&s_n
\end{pmatrix}\]
$\mathbf{b_i}$ is the $i$-th column of $\mathbf{B}$ and $\mathbf{S}$ is a diagonal matrix. 

$\mathbf{B}$ cmod $\mathbf{S}$ :=  $\begin{pmatrix}
    \mathbf{b_1}\mod{s_1}&\mathbf{b_2}\mod{s_2}&\cdots&\mathbf{b_n}\mod{s_n}
\end{pmatrix}$
\end{definition}

The definitions of Smith Normal form and Smith Massager will only be used in Theorem~\ref{LatMemthm}, Section~\ref{section:Identifying an Ideal Lattice}.
\subsection{Ideal lattices}

 An algebraic number field $K$ is an extension field of the rationals $\mathbb{Q}$ such that its dimension $ [K : \mathbb{Q}]$ as a $\mathbb{Q}$-vector space (i.e., its degree) is finite. 
 
 
 
 
 An element $x$ in the algebraic number field $K$ is said to be integral over $\mathbb{Z}$ if the coefficients of the minimal polynomial of $x$ over $\mathbb{Q}$ are all integers. All the elements which are integral over $\mathbb{Z}$ in $K$ make up a set  $O_K$. $O_K$ is actually a ring called the algebraic integer ring of $K$ over $\mathbb{Z}$. 
 
 
  $O_K$ is a finitely generated $\mathbb{Z}$-module of dimension $[K :\mathbb{Q}]$. The basis of $O_K$ as a $\mathbb{Z}$-module is called the integer basis, which is also a basis of $K$ as a $\mathbb{Q}$-vector space.
 
\paragraph{Canonical-embedding ideal lattice} 


 If $\Omega\supset K$ is an extension field such that $\Omega$ is algebraically closed over $\mathbb{Q}$, then there are exactly $[K :\mathbb{Q}]$ field embeddings of $K$ into $\Omega$. For convenience, we regard $\Omega$ as the complex field $\mathbb{C}$.


Any ideal of $O_K$ is a full-rank submodule of $O_K$. Let $[K :\mathbb{Q}]=n$.  This structure induces a canonical embedding:
\begin{align*}
\Sigma: O_K&\rightarrow\mathbb{C}^n \\ a&\mapsto (\Sigma_i(a))_{i=1,...,n},
\end{align*}
where $\Sigma_i$'s are the $n$ different embeddings from $K$ into $\mathbb{C}$.

\begin{definition}{(Canonical-embedding Ideal Lattice)}
Given a number field $K$ and  any ideal I of $O_K$, $\Sigma$(I) is called the canonical-embedding ideal lattice.
\end{definition}

\paragraph{Coefficient-embedding ideal lattice} Denote by $\mathbb{Z}^{(n)}[x]$  the set of all the  polynomials in $\mathbb{Z}[x]$ with degree $\leq$$n-1$. We use the symbol $\sigma$ to represent the following linear map: 
\begin{align*} \sigma : \mathbb{Z}^{(n)}[x]&\rightarrow\mathbb{Z}^n \\ \sum_{i=1}^{n} a_ix^{i-1} &\mapsto (a_1,a_1,...,a_n), \end{align*}
where linear map means that 
\begin{itemize}
	\item For any $f(x)$, $g(x)\in \mathbb{Z}^{(n)}[x]$, $\sigma(f(x)+g(x)) = \sigma(f(x))+\sigma(g(x));$
	\item For any $f(x)\in \mathbb{Z}^{(n)}[x]$ and $z\in\mathbb{Z}$, $\sigma(zf(x)) = z\sigma(f(x)).$
\end{itemize}

We can also define its inverse, which is linear too:
\begin{align*} \sigma^{-1} : \mathbb{Z}^n &\rightarrow\mathbb{Z}^{(n)}[x]\\ (a_1,a_1,\cdots,a_n)&\mapsto \sum_{i=1}^{n} a_ix^{i-1}. \end{align*} 




In what follows, we focus on ideal lattices induced by ideals of the ring $\mathbb{Z}[x]/f(x)$, where $f(x)$ is a monic polynomial of degree $n$.  Any element in $\mathbb{Z}^{(n)}[x]$ can be viewed as a  representative in the ring $\mathbb{Z}[x]/f(x)$ with $\text{degree}(f(x))\geq n$ \cite{ref13}.
So we abuse the symbol $\sigma$ to represent the  the following coefficient embedding. 
\begin{align*} \sigma : \mathbb{Z}[x]/f(x)&\rightarrow\mathbb{Z}^n \\ \sum_{i=1}^{n} a_ix^{i-1} &\mapsto (a_1,a_2,...,a_n). \end{align*} 

Therefore, under the coefficient embedding, any ideal of $\mathbb{Z}[x]/f(x)$ can be viewed as an integer lattice.

\begin{definition}{(Coefficient-embedding Ideal Lattice)}
Given $\mathbb{Z}[x]/f(x)$, where  $f(x)$ is a monic polynomial of degree n, and any ideal $I$ of $\mathbb{Z}[x]/f(x)$, $\sigma$(I) is called the coefficient-embedding ideal lattice, which is of course an integer lattice.
\end{definition}






%Roughly speaking,  the canonical-embedding ideal lattice are usually used in the theoretical analysis of lattice-related hard problems and lattice-based cryptosystems whereas the coefficient-embedding ideal lattices are usually used in the implementation of lattice-based cryptosystems. Most of the practical lattice-based cryptosystems are employing the  coefficient-embedding ideal lattices.

Roughly speaking, due to the abundant algebraic structures of the corresponding algebraic integer domains, the hard lattice problems in canonical-embedding ideal lattices are easier to analyse than that in coefficient-embedding ideal lattices. However, as we've introduced in the introduction, in some cases, the results in canonical-embedding ideal lattices can be converted to the results in the coefficient-embedding ideal lattices with small loss.

The following is an important property of ideal lattices, it was proposed by  Zhang, Liu and Lin \cite{ref16}. We present their proof detail for readers to check conveniently.

\begin{lemma}[\cite{ref16}] \label{klemma}
Let $\mathbf{H}$ be the HNF basis of the full-rank coefficient-embedding ideal lattice $\mathcal{L}(\mathbf{B})$ in the ring $\mathbb{Z}[x]/f(x)$.
 \[ \mathbf{H}= \begin{pmatrix} h_{1,1}&0&\cdots&0 \\ h_{2,1}&h_{2,2}&\cdots&0 \\ \vdots&\vdots&\ddots&\vdots \\h_{n,1}&\cdots&\cdots&h_{n,n} \end{pmatrix}.\] Then $h_{i,i}|h_{j,l}$, for ${1\leq l \leq j \leq i \leq n}$. Specially,
 $h_{n,n} \vert h_{i,j}$, ${i,j\leq n}$.
\end{lemma}
\begin{proof} 
By induction on $i$,  it's trivial for $i=1$.
         
         Assume the result holds for $i \leq k \leq n-1$. It remains to show that for $i=k+1$, $h_{k+1,k+1}|h_{j,l}$ where ${1\leq l \leq j \leq k+1 \leq n}$.
          
          Let $\mathbf{h}_i$ be the $i$-th row of $\mathbf{H}$. Note that for any ideal $I$ of $\mathbb{Z}[x]/f(x)$ and for all $g(x)\in I $, $xg(x) \in I$. Specially $x\sigma^{-1}(\mathbf{h}_k) \in I$, where $\sigma$ is the coefficient-embedding. Since  $\mathbf{H}$ is a basis of the ideal lattice, it is very simple to imply that there must exist $y_i \in \mathbb{Z}$, for $i=1,2,\cdots,k+1$ such that:\[ \begin{pmatrix} 0&h_{k,1}&\cdots&h_{k,k}&0&\cdots&0 \end{pmatrix}=\sum_{i=1}^{k+1}y_i\mathbf{h}_i.\]
          
         Hence, \begin{align*}h_{k,k}&=y_{k+1}h_{k+1,k+1} \\ h_{k,k-1}&=y_kh_{k,k}+y_{k+1}h_{k+1,k}\\\vdots\\h_{k,1}&=\sum_{i=2}^{k+1}y_ih_{i,2}\\0&=\sum_{i=1}^{k+1}y_ih_{i,1} \end{align*}
         
         
         From the first equation, we get $y_{k+1}=\frac{h_{k,k}}{h_{k+1,k+1}}\in \mathbb{Z}$, and \begin{align*} h_{k+1,k}&=\frac{h_{k,k-1}-y_kh_{k,k}}{h_{k,k}}h_{k+1,k+1}\\h_{k+1,k-1}&=\frac{h_{k,k-2}-y_{k-1}h_{k-1,k-1}-y_kh_{k,k-1}}{h_{k,k}}h_{k+1,k+1}\\\vdots\\h_{k+1,2}&=\frac{h_{k,1}-\sum_{i=2}^ky_ih_{i,2}}{h_{k,k}}h_{k+1,k+1}\\h_{k+1,1}&=\frac{-\sum_{i=1}^ky_ih_{i,1}}{h_{k,k}}h_{k+1,k+1} \end{align*}
          
          From the induction hypothesis, we have $h_{k,k}|h_{j,l}$ for $1\leq l \leq j \leq k \leq n$. So the coefficient of $h_{k+1,k+1}$ in each equation is in fact an integer. Therefore, $h_{k+1,k+1}|h_{k+1,l},1 \leq l \leq k+1$. Since $h_{k+1,k+1}|h_{k,k}$, we know $h_{k+1,k+1}|h_{j,l}$, where $1 \leq l \leq j \leq k+1 \leq n$. Thus, the result holds for $i=k+1$. 
          
          By induction,  $h_{i,i}|h_{j,l}$, ${1\leq l \leq j \leq i \leq n}$. So $h_{n,n}|h_{i,j}$, ${1\leq i \leq j \leq n}$.  Lemma \ref{klemma} follows.
\end{proof}
 


%In some cases, the $\gamma$-SVP problem in canonical-embedding ideal lattices  and  coefficient-embedding ideal lattices  are equivalent or very closely connected with each other as we've mentioned in the introduction.

\subsection{Overview}In the third section, we first show and prove a naturally equivalent definition (Lemma \ref{klemma4ideal}) of integer lattices. It's a direct application of the definition of the coefficient-embedding ideal lattice. Though the result of Lemma \ref{klemma4ideal} may have been used in some earlier research, we haven't found a detailed description. Hence, we rewrite and prove Lemma \ref{klemma4ideal} formally.

Inspired by Lemma \ref{klemma} proposed by Zhang, Liu and Lin \cite{ref16}, we propose Theorem \ref{thmidentify}, another equivalent definition of ideal lattices in Section 3.2. Using this equivalent definition, we design Algorithm \ref{alg:identify} to verify whether an integer lattice is an ideal lattice.

In Section 3.3, Theorem \ref{kthm} shows that a coefficient-embedding ideal lattice can be embedded into another polynomial ring denoted by $R$ as an ideal of $R$, and for a fixed coefficient-embedding ideal lattice the number of such $R$ is infinite. The proof is also motivated by Lemma \ref{klemma}. Theorem \ref{kthm} guarantees that Algorithm \ref{alg:identify} can output all the polynomial rings which the input integer lattice can be embedded into as ideals.

%Based on theorem \ref{kthm}, we present the theorem \ref{thmidentify}, an equivalent condition for identifying an ideal lattice. The key step in the proof of theorem \ref{kthm} relies on the lemma \ref{klemma}, and by observing the proof detail of lemma \ref{klemma} we give the lemma \ref{identify}, a weaker version of lemma \ref{klemma}. It's easy to proof theorem \ref{thmidentify} by using the lemma \ref{identify}.
 

%Next, we introduce some applications of Theorem 1, and give some examples. Though the examples may not be practical in the reality for now, it do supply us a new method to deal with the $\gamma$-SVP in ideal lattices and integer lattices.

In the fourth section, we propose Algorithm \ref{alg:identify} to judge whether an integer lattice can be embedded into a polynomial ring as ideals and compute all the rings that the lattice can be embedded into as an ideal if the given lattice is a coefficient-embedding ideal lattice. We analysis the time complexity of Algorithm \ref{alg:identify} and also compare our algorithm to related work.

Finally, we give a brief conclusion. Out experimental data is presented in the Appendix~\ref{AppendixA}.

%We originally want to make use of special coefficients relation of the HNF of the coefficient-embedding ideal lattice given by Lemma 5, but we find the result of Lemma 5 is too weak to judge whether an integer lattice is an ideal lattice. Hence, we introduce the Incomplete Hermite Normal Form (Definition 3) and propose a stronger condition (Theorem 2 ) than the result of Lemma 5 to judge an ideal lattice. Based on Theorem 2, we propose Algorithm 1,which can judge whether an integer lattice is an ideal lattice and compute all the rings the input lattice can be embedded into as ideals if the input lattice is an ideal lattice, and give an analysis of the complexity of Algorithm 1.








\section{An ideal lattice can be embedded into different rings}
We stress that in the following, we focus on the coefficient-embedding ideal lattice, and in this section, we'll show how an coefficient-embedding ideal lattice can be embedded into different rings. 

%It is well known that a lattice is just an additive group. However, when it is also equipped  with some "multiplication", then it becomes an ideal lattice. A natural question is that what will happen if we equip the same lattice with different "multiplication"? Obviously if this can be done, the lattice will not change, but the ideal changes, which means that an ideal lattice can be viewed as different ideals in different rings.





\subsection{Deciding an ideal lattice} We next present an easy way to tell if a given lattice is a  coefficient-embedding ideal lattice in $\mathbb{Z}[x]/f(x)$ or not.

\begin{lemma} \label{klemma4ideal}
	For any monic polynomial $f(x)\in\mathbb{Z}[x]$ with degree $n$, a lattice $\mathcal{L}(\mathbf{B})$ with any basis $\mathbf{B}$ is a  coefficient-embedding ideal lattice in $\mathbb{Z}[x]/f(x)$  if and only if $\sigma(x\sigma^{-1}(\mathbf{b}_i)\mod f(x))\in \mathcal{L}(\mathbf{B})$ for $i=1,\cdots,n$, where $\mathbf{b}_i$ is the $i$-th row vector of $\mathbf{B}$, and $\sigma$ is the map defined in Section 2.3.
\end{lemma}
\begin{proof} 
	If  $\mathcal{L}(\mathbf{B})$ is a  coefficient-embedding ideal lattice in $\mathbb{Z}[x]/f(x)$, then $\sigma^{-1}(\mathbf{b}_i)$'s are in the corresponding ideal. It is obvious that  $x\sigma^{-1}(\mathbf{b}_i)\mod f(x)$ must be in the ideal too, which means that $\sigma(x\sigma^{-1}(\mathbf{b}_i)\mod f(x))\in \mathcal{L}(\mathbf{B})$.
	
	
	If there exists a  monic polynomial $f(x)\in\mathbb{Z}[x]$ with degree $n$, such that $\sigma(x\sigma^{-1}(\mathbf{b}_i)\mod f(x))\in \mathcal{L}(\mathbf{B})$ for $i=1,\cdots,n$, we show that $\sigma^{-1}(\mathcal{L}(\mathbf{B}))$ must be an ideal in $\mathbb{Z}[x]/f(x)$. It is easy to check that   $\sigma^{-1}(\mathcal{L}(\mathbf{B}))$  is an additive group, due to the fact that $\sigma$ is an additive homomorphism. Since  $\sigma(x\sigma^{-1}(\mathbf{b}_i)\mod f(x))\in \mathcal{L}(\mathbf{B})$, then for any lattice vector $\mathbf{v} = \sum_{i=1}^n z_i\mathbf{b}_i$, $z_i\in\mathbb{Z}$, we have   $$\sigma(x\sigma^{-1}(\mathbf{v})\mod f(x))=\sum_{i=1}^n z_i \sigma( x \sigma^{-1}(\mathbf{b}_i)\mod f(x)) \in \mathcal{L}(\mathbf{B}). $$
	Applying the result on the lattice vector $\sigma(x\sigma^{-1}(\mathbf{v})\mod f(x))$, we will have
	$$\sigma(x^2\sigma^{-1}(\mathbf{v}))= \sigma(x \sigma^{-1}(\sigma(x\mathbf{v}\mod f(x)))) \in \mathcal{L}(\mathbf{B}). $$
	Hence, for any positive integer $k$, we know that 
		$$\sigma(x^k\sigma^{-1}(\mathbf{v})) \in \mathcal{L}(\mathbf{B}). $$
	Then for any $g(x)=\sum_{i=1}^n g_ix^{i-1}\in\mathbb{Z}[x]/f(x)$ and any lattice vector $\mathbf{v}$, 
		$$\sigma(g(x)\sigma^{-1}(\mathbf{v})\mod f(x))=\sum_{i=1}^n g_i \sigma( x^{i-1} \sigma^{-1}(\mathbf{v})\mod f(x)) \in \mathcal{L}(\mathbf{B}). $$
	The lemma follows.
	

\end{proof}



%\subsubsection{HNF of ideal lattices}

 %The following lemma tells us some  divisibility relation among the elements in the HNF basis of a coefficient-embedding ideal lattice, which has been proved in \cite{ref16}. For completeness, we present the whole proof simply.

%The proof is exactly the same with the lemma 2 of Liu \textit{et al.}'s paper \cite{ref16}, though our condition is slightly different. For convenience, we rewrite their proof.


%Let $\mathcal{L}(\mathbf{B})$ be a full-rank coefficient-embedding ideal lattice in the ring $\mathbb{Z}[x]/f(x)$, and $\mathbf{H}$ is the HNF of $\mathcal{L}(\mathbf{B})$.


%\begin{lemma}[\cite{ref16}] \label{klemma}
%Let $\mathbf{H}$ be the HNF basis of the full-rank coefficient-embedding ideal lattice $\mathcal{L}(\mathbf{B})$ in the ring $\mathbb{Z}[x]/f(x)$.
 %\[ \mathbf{H}= \begin{pmatrix} h_{1,1}&0&\cdots&0 \\ h_{2,1}&h_{2,2}&\cdots&0 \\ \vdots&\vdots&\ddots&\vdots \\h_{n,1}&\cdots&\cdots&h_{n,n} \end{pmatrix}.\] Then $h_{i,i}|h_{j,l}$, for ${1\leq l \leq j \leq i \leq n}$. Specially,
 %$h_{n,n} \vert h_{i,j}$, ${i,j\leq n}$.
%\end{lemma}
%\begin{proof} 
%By induction on $i$,  it's trivial for $i=1$.
         
         %Assume the result holds for $i \leq k \leq n-1$. It remains to show that for $i=k+1$, $h_{k+1,k+1}|h_{j,l}$ where ${1\leq l \leq j \leq k+1 \leq n}$.
          
          %Let $\mathbf{h}_i$ be the $i$-th row of $\mathbf{H}$. Note that for any ideal $I$ of $\mathbb{Z}[x]/f(x)$ and for all $g(x)\in I $, $xg(x) \in I$. Specially $x\sigma^{-1}(\mathbf{h}_k) \in I$, where $\sigma$ is the coefficient-embedding. Since  $\mathbf{H}$ is a basis of the ideal lattice, it is very simple to imply that there must exist $y_i \in \mathbb{Z}$, for $i=1,2,\cdots,k+1$ such that:\[ \begin{pmatrix} 0&h_{k,1}&\cdots&h_{k,k}&0&\cdots&0 \end{pmatrix}=\sum_{i=1}^{k+1}y_i\mathbf{h}_i.\]
          
         %Hence, \begin{align*}h_{k,k}&=y_{k+1}h_{k+1,k+1} \\ h_{k,k-1}&=y_kh_{k,k}+y_{k+1}h_{k+1,k}\\\vdots\\h_{k,1}&=\sum_{i=2}^{k+1}y_ih_{i,2}\\0&=\sum_{i=1}^{k+1}y_ih_{i,1} \end{align*}
         
         
         %From the first equation, we get $y_{k+1}=\frac{h_{k,k}}{h_{k+1,k+1}}\in \mathbb{Z}$, and \begin{align*} h_{k+1,k}&=\frac{h_{k,k-1}-y_kh_{k,k}}{h_{k,k}}h_{k+1,k+1}\\h_{k+1,k-1}&=\frac{h_{k,k-2}-y_{k-1}h_{k-1,k-1}-y_kh_{k,k-1}}{h_{k,k}}h_{k+1,k+1}\\\vdots\\h_{k+1,2}&=\frac{h_{k,1}-\sum_{i=2}^ky_ih_{i,2}}{h_{k,k}}h_{k+1,k+1}\\h_{k+1,1}&=\frac{-\sum_{i=1}^ky_ih_{i,1}}{h_{k,k}}h_{k+1,k+1} \end{align*}
          
          %From the induction hypothesis, we have $h_{k,k}|h_{j,l}$ for $1\leq l \leq j \leq k \leq n$. So the coefficient of $h_{k+1,k+1}$ in each equation is in fact an integer. Therefore, $h_{k+1,k+1}|h_{k+1,l},1 \leq l \leq k+1$. Since $h_{k+1,k+1}|h_{k,k}$, we know $h_{k+1,k+1}|h_{j,l}$, where $1 \leq l \leq j \leq k+1 \leq n$. Thus, the result holds for $i=k+1$. 
          
          %By induction,  $h_{i,i}|h_{j,l}$, ${1\leq l \leq j \leq i \leq n}$. So $h_{n,n}|h_{i,j}$, ${1\leq i \leq j \leq n}$.  Lemma \ref{klemma} follows.
%\end{proof}

%\begin{remark}\label{remarkhnf}
	%Note that in the proof of Lemma \ref{klemma}, to conclude that $h_{n,n} \vert h_{i,j}$, ${i,j\leq n}$, what we need is $\sigma(x\sigma^{-1}(\mathbf{h}_k)) \in \mathcal{L}(\mathbf{B})$ for $k =1,\cdots, n-1$. We do not care about  if  $\sigma(x\sigma^{-1}(\mathbf{h}_n))$ is in $\mathcal{L}(\mathbf{B})$ or not.		
%\end{remark}


%Lemma \ref{klemma} presents us the divisibility relation among the elements in the HNF basis of a coefficient-embedding ideal lattice. Actually, not only it's crucial to our main theorem, but also we can regard it as a  tool to quickly rule out some integer lattices not being an ideal lattice in any polynomial ring. See Section 4 for more details.
\subsection{Equivalent condition}



%\section{Identifying an Ideal Lattice}

%We also present an algorithm to identify the ideal lattice, which is faster than that in \cite{ref13}.

%\subsection{Main theorem}
%Inspired by Theorem \ref{kthm}, we find a new equivalent condition between integer lattices and coefficient-embedding ideal lattices, which is described as below.
Inspired by Lemma \ref{klemma}, we find a new equivalent definition of coefficient-embedding ideal lattices.    
  
\begin{theorem}\label{thmidentify}
	Given a full-rank integer lattice $\mathcal{L}(\mathbf{B})$, let $\mathbf{B}'= \begin{pmatrix} \mathbf{D}&\mathbf{0} \\ \mathbf{b'}&b_{n,n}' \end{pmatrix}$ be any  Incomplete Hermit Normal Form of \ $\mathbf{B}$. Then $\mathcal{L}(\mathbf{B})$ is an ideal lattice if and only if there exists a $\mathbf{T} \in \mathbb{Z}^{(n-1) \times n}$, s.t.$\begin{pmatrix} \mathbf{0}&\mathbf{D}\end{pmatrix}=\mathbf{T}\mathbf{B}$. Specially, if $\mathcal{L}(\mathbf{B})$ is an ideal lattice, then taking any $g(x)=x^n+g_nx^{n-1}+\cdots+g_1$ with $\begin{pmatrix}g_1&g_2&\cdots&g_n\end{pmatrix} \in\frac{1}{b_{n,n}'}(\begin{pmatrix}0&\mathbf{b}'\end{pmatrix} +\mathcal{L}(\mathbf{B}))$, $\mathcal{L}(\mathbf{B})$ is also an ideal lattice in the ring $\mathbb{Z}[X]/g(x)$.
\end{theorem}
\begin{proof}
It can be easily check the ``only if'' part by Lemma \ref{klemma4ideal}, since for an ideal lattice $\mathcal{L}(\mathbf{B})$ in $\mathbb{Z}[x]/g(x)$, there exists a $\mathbf{T} \in \mathbb{Z}^{(n-1) \times n}$, s.t. $\begin{pmatrix} \mathbf{0}&\mathbf{D}\end{pmatrix}=\mathbf{T}\mathbf{B}$ if and only if  $\sigma(x\sigma^{-1}(\mathbf{b}'_i)\mod g(x))\in \mathcal{L}(\mathbf{B})$ for $i=1,\cdots,n-1$.

For ``if'' part,
to indicate that $\mathcal{L}(\mathbf{B})$ is an ideal lattice, we need to find a monic polynomial $g(x)$ of degree $n$, s.t. $\mathcal{L}(\mathbf{B})$ can be embedded as an ideal into $\mathbb{Z}[x]/g(x)$, or $\sigma(x\sigma^{-1}(\mathbf{b}'_i)\mod g(x))\in \mathcal{L}(\mathbf{B})$ for $i=1,\cdots,n$  by Lemma \ref{klemma4ideal}.

Note that for any polynomial $g(x)$ with degree $n$, $\sigma(x\sigma^{-1}(\mathbf{b}'_i)\mod g(x))\in \mathcal{L}(\mathbf{B})$ for $i=1,\cdots,n-1$ since there exists a $\mathbf{T} \in \mathbb{Z}^{(n-1) \times n}$, s.t. $\begin{pmatrix} \mathbf{0}&\mathbf{D}\end{pmatrix}=\mathbf{T}\mathbf{B}$. 

It remains to show that there exists a monic polynomial $g(x)$ of degree $n$, such that $\sigma(x\sigma^{-1}(\mathbf{b}'_n)\mod g(x))\in \mathcal{L}(\mathbf{B})$.

We first present a lemma, which will be proven later.
\begin{lemma} \label{identify}	
	If $\begin{pmatrix} \mathbf{0}&\mathbf{D}\end{pmatrix}=\mathbf{T}\mathbf{B}$, then $ \mathbf{B}'/b_{n,n}' \in \mathbb{Z}^{n \times n}$
\end{lemma}

 By Lemma \ref{identify},  $\frac{1}{b_{n,n}'}(\begin{pmatrix}0&\mathbf{b}'\end{pmatrix} +\mathcal{L}(\mathbf{B})) \subset \mathbb{Z}^n$. Taking any
 \begin{equation} \label{equg}
  \mathbf{g}=\begin{pmatrix}g_1&g_2&\cdots&g_n\end{pmatrix} \in\frac{1}{b_{n,n}'}(\begin{pmatrix}0&\mathbf{b}'\end{pmatrix} +\mathcal{L}(\mathbf{B})),
 \end{equation}
   the integer polynomial $g(x)=x^n+g_nx^{n-1}+\cdots+g_1$ is what we want, since
$$
 \sigma(x\sigma^{-1}(\mathbf{b}'_n)\mod g(x)) = \begin{pmatrix}0&\mathbf{b}'\end{pmatrix} - {b_{n,n}'}\begin{pmatrix}g_1&g_2&\cdots&g_n\end{pmatrix}\in \mathcal{L}(\mathbf{B}). $$
 
 

%Hence, all the polynomials corresponding to the vectors in the coset $\frac{1}{b_{n,n}'}(\begin{pmatrix}0&\mathbf{b}'\end{pmatrix} +\mathcal{L}(\mathbf{B}))$ can induce a coefficient embedding from $\mathcal{L}(\mathbf{B})$ into polynomial ring.  $\mathcal{L}(\mathbf{B})$ is a coefficient-embedding ideal lattice.
%\end{proof}

It remains to prove Lemma \ref{identify}. Actually, the proof is exactly the same with Lemma \ref{klemma}
%\begin{proof}{(Lemma  \ref{identify})} According to Lemma \ref{hnfbasis}, 	
 %$\mathcal{L}(\mathbf{B}')$ has a unique HNF basis, denoted by $\mathbf{H}=(h_{i,j})_{1\leq i \leq %n,1\leq j \leq n}$. By Lemma \ref{ihnflemma}, we know that $b_{n,n}'=h_{n,n}$.
 
 %It can be easily concluded that the lattice $\mathcal{L}(\mathbf{D})$ has a unique HNF basis,  $\mathbf{H}'=(h_{i,j})_{1\leq i \leq n-1,1\leq j \leq n-1}$, which implies that there exists a unimodular matrix $\mathbf{U}\in \mathbb{Z}^{(n-1)\times (n-1)}$ such that $\mathbf{H}' = \mathbf{U} \mathbf{D}$.
	
%Since $\begin{pmatrix} \mathbf{0}&\mathbf{D}\end{pmatrix}=\mathbf{T}\mathbf{B}$, we have 
%$\mathbf{U}\begin{pmatrix} \mathbf{0}&\mathbf{D}\end{pmatrix}=\mathbf{U}\mathbf{T}\mathbf{B}$, which is exactly
%\begin{equation}\label{equ3}
%\begin{pmatrix}0&h_{1,1}&0&...&0\\0&h_{2,1}&h_{2,2}&...&0\\.&.&.&.&.\\0&h_{n-1,1}&h_{n-1,2}&...&h_{n-1,n-1}\end{pmatrix}=\mathbf{U}\mathbf{T}\mathbf{B}.
%\end{equation}


%Note that $\mathcal{L}(\mathbf{U}\mathbf{T}\mathbf{B})\subset \mathcal{L}(\mathbf{B})$. What Equation (\ref{equ3}) tells us is 
%$$\sigma(x\sigma^{-1}(\mathbf{h}_k)) \in \mathcal{L}(\mathbf{B}),\mbox{ for } k =1,\cdots, n-1.$$

%By the discussion in Remark \ref{remarkhnf}, we have  $b_{n,n}'=h_{n,n}\vert\mathbf{B}$. 
\end{proof}

\subsection{An ideal lattice can be embedded into infinitely many different polynomial rings as ideals}
%The main difference between the lattice and the ideal is the multiplication structure.  To simulate the ideal multiplication structure in an integer lattice, we need to define a column transformation  on a row vector.
Given a full-rank ideal lattice $\mathcal{L}(\mathbf{B})$ together with the Incomplete Hermit Normal Form $\mathbf{B}'= \begin{pmatrix} \mathbf{D}&\mathbf{0} \\ \mathbf{b'}&b_{n,n}' \end{pmatrix}$, Theorem \ref{thmidentify} shows that for any $g(x)=x^n+g_nx^{n-1}+\cdots+g_1$ with $\begin{pmatrix}g_1&g_2&\cdots&g_n\end{pmatrix} \in\frac{1}{b_{n,n}'}(\begin{pmatrix}0&\mathbf{b}'\end{pmatrix} +\mathcal{L}(\mathbf{B}))$,  $\mathcal{L}(\mathbf{B})$ is also an ideal lattice in the ring $\mathbb{Z}[X]/g(x)$. The following theorem proves that only if we take $g(x)$ in this way, $\mathcal{L}(\mathbf{B})$ can be viewed as an ideal lattice in the ring $\mathbb{Z}[X]/g(x)$. In other words, the coset $\frac{1}{b_{n,n}'}(\begin{pmatrix}0&\mathbf{b}'\end{pmatrix} +\mathcal{L}(\mathbf{B}))$ can represent the class of all the polynomial rings which the given ideal lattice $\mathcal{L}(\mathbf{B})$ can be embedded into as ideals.


\begin{theorem}\label{kthm}
For any  full-rank coefficient-embedding ideal lattice $\mathcal{L}(\mathbf{B})$ in the ring $\mathbb{Z}[x]/f(x)$, where $f(x)$ is monic and $\text{deg}(f(x))=n$, there exists infinitely many monic $g(x)\in\mathbb{Z}[x]$ with degree $n$, s.t.  $\mathcal{L}(\mathbf{B})$ is also a coefficient-embedding  ideal lattice in $\mathbb{Z}[x]/g(x)$.

More precisely, 
let $d = \gcd( b_{1,n}, b_{2,n},...,b_{n,n})$. Then $\mathcal{L}(\mathbf{B})$ is also a coefficient-embedding  ideal lattice in $\mathbb{Z}[x]/g(x)$, where $g(x)\in\mathbb{Z}[x]$ is a monic polynomial with degree $n$, if and only if 
$$\sigma(f(x)-g(x)) \in  \mathcal{L}(\frac{\mathbf{B}}{d}),$$
or equivalently,
$$g(x) \in f(x) + \sigma^{-1}(\mathcal{L}(\frac{\mathbf{B}}{d})).$$

\end{theorem}
\begin{proof}
%It's sufficient to find an injection from $\mathcal{L}(\mathbf{B})$ to some polynomial ring $\mathbb{Z}[x]/g(x)$. 

Consider the HNF basis of $\mathcal{L}(\mathbf{B})$, 
 \[ \mathbf{H}= \begin{pmatrix} h_{1,1}&0&\cdots&0 \\ h_{2,1}&h_{2,2}&\cdots&0 \\ \vdots&\vdots&\ddots&\vdots \\h_{n,1}&\cdots&\cdots&h_{n,n} \end{pmatrix}.\]
For convenience, we denote  by $\mathbf{h}_i$ the $i$-th row of \ $\mathbf{H}$, and then $\mathbf{h}_i$ is a vector in $\mathbb{Z}^n$. 

(i) If there is a monic $g(x)\in\mathbb{Z}[x]$ with degree $n$, s.t.  $\mathcal{L}(\mathbf{B})$ is also a coefficient-embedding  ideal lattice in $\mathbb{Z}[x]/g(x)$, we next prove that $\sigma(f(x)-g(x)) \in  \mathcal{L}(\frac{\mathbf{B}}{d})$.  

By Lemma \ref{klemma4ideal}, 
 $\mathcal{L}(\mathbf{H}) = \mathcal{L}(\mathbf{B})$ is  a coefficient-embedding  ideal lattice in $\mathbb{Z}[x]/f(x)$, then we  have
$$\sigma(x\sigma^{-1}(\mathbf{h}_n)\mod f(x))\in \mathcal{L}(\mathbf{B}).$$
Note that 
$$x\sigma^{-1}(\mathbf{h}_n)\mod f(x) = \sum_{i=1}^{n-1} h_{n,i}x^i - h_{n,n} (f(x)-x^n).$$
We have
 \begin{equation} \label{equ1}
\begin{pmatrix} 0&h_{n,1}&...&h_{n,n-1} \end{pmatrix}-h_{n,n}\sigma(f(x)-x^n) \in  \mathcal{L}(\mathbf{B}).
 \end{equation}
 
 
 Similarly, since $\mathcal{L}(\mathbf{B})$ is also a coefficient-embedding  ideal lattice in $\mathbb{Z}[x]/g(x)$, we have 
  \begin{equation} \label{equ2}
 \begin{pmatrix} 0&h_{n,1}&...&h_{n,n-1} \end{pmatrix}-h_{n,n}\sigma(g(x)-x^n) \in  \mathcal{L}(\mathbf{B}).
  \end{equation}
 Subtracting the left side of (\ref{equ1}) from the left side of (\ref{equ2}), we immediately have
 $$h_{n,n}\sigma(f(x)-g(x)) \in \mathcal{L}(\mathbf{B}).$$
 By Lemma \ref{ihnflemma}, $h_{n,n} = d$, we have
 $$\sigma(f(x)-g(x)) \in \mathcal{L}(\frac{\mathbf{B}}{d}).$$
 
 
 
 (ii) We next prove that for any  polynomial $g(x)$, such that $\sigma(f(x)-g(x)) \in  \mathcal{L}(\frac{\mathbf{B}}{d})$, any  full-rank coefficient-embedding ideal lattice $\mathcal{L}(\mathbf{B})$ in the ring $\mathbb{Z}[x]/f(x)$ can also be viewed as a coefficient-embedding  ideal lattice in $\mathbb{Z}[x]/g(x)$.
 
 
 First, $g(x)$ is obviously a monic polynomial with degree $n$. Note that by Lemma \ref{klemma}, $h_{n,n}|h_{i,j}$, then $d = h_{n,n}$ divide all the components of every lattice vector in $\mathcal{L}(\mathbf{B})$, which means that $ \mathcal{L}(\frac{\mathbf{B}}{d})$ is an integer lattice and once $\sigma(f(x)-g(x)) \in  \mathcal{L}(\frac{\mathbf{B}}{d})$, $g(x)\in\mathbb{Z}[x]$.
 
 
 By Lemma \ref{klemma4ideal} again, 
 $\mathcal{L}(\mathbf{H}) = \mathcal{L}(\mathbf{B})$ is  a coefficient-embedding  ideal lattice in $\mathbb{Z}[x]/f(x)$, then we  have
 $$\sigma(x\sigma^{-1}(\mathbf{h}_i)\mod f(x))\in \mathcal{L}(\mathbf{B}),$$
 for $i = 1,\cdots, n$.
 
 To prove that  $\mathcal{L}(\mathbf{B})$ is also a coefficient-embedding  ideal lattice in $\mathbb{Z}[x]/g(x)$, by Lemma \ref{klemma4ideal}  it is enough to show that 
  $\sigma(x\sigma^{-1}(\mathbf{h}_i)\mod g(x))\in \mathcal{L}(\mathbf{B}),$
 for $i = 1,\cdots, n$.
 
 
 Note that for $i = 1,\cdots, n-1$, 
 $$\sigma(x\sigma^{-1}(\mathbf{h}_i)\mod g(x)) = \sigma(x\sigma^{-1}(\mathbf{h}_i)\mod f(x))\in \mathcal{L}(\mathbf{B}).$$
 
 Since$\sigma(f(x)-g(x)) \in  \mathcal{L}(\frac{\mathbf{B}}{d})$, there exists a lattice vector $\mathbf{v} \in \mathcal{L}(\mathbf{B})$ such that $d(f(x)-g(x)) = h_{n,n}(f(x)-g(x))= \sigma^{-1}(\mathbf{v})$. Then for $i = n$, 
 \begin{align*}
 \sigma(x\sigma^{-1}(\mathbf{h}_n)\mod g(x)) &=\sigma (\sum_{i=1}^{n-1} h_{n,i}x^i - h_{n,n} (g(x)-x^n))\\& = \sigma (\sum_{i=1}^{n-1} h_{n,i}x^i- h_{n,n} (f(x)-x^n) + \sigma^{-1}(\mathbf{v}))\\
 &= \sigma(x\sigma^{-1}(\mathbf{h}_n)\mod f(x)) + \mathbf{v} \in \mathcal{L}(\mathbf{B}).
 \end{align*}
 The  theorem  follows.
\end{proof}

\begin{remark}
	The HNF \ $\mathbf{H}$ in the proof can be replaced by any Incomplete Hermite Normal Form.
	
\end{remark}
\begin{remark}\label{rmk:app}
  For most lattice-based cryptosystems, their security is guaranteed by the hardness of lattice problems such as $\gamma$-SVP.  Hence, the hardness of lattice problem in ideal lattice is widely considered as the security foundation of Ring-LWE based cryptosystems. 

  However, the worst-case hardness of ideal lattice $\gamma$-SVP in different polynomial rings are not the same exactly. For example, in the ring $\mathbb{Z}[x]/(x^n+1)$ $n=2^k$ $k\geq 1$, there is a quantum polynomial time algorithm for ideal lattice $\text{exp}(n^{1/2})$-SVP \cite{ref7} \cite{ref8}, but the coefficient is no less than $\text{exp}(n)$ in the majority of polynomial rings.

  Theorem \ref{kthm} indicates that an ideal lattice can be viewed as an ideal lattice in infinitely different polynomial rings. Hence, it's possible to embed the given ideal lattice into a special ring such as $\mathbb{Z}[x]/(x^n+1)$ $n=2^k$ $k\geq 1$ which can help to the solve the hard lattice problems.
	
\end{remark}
%\subsection{Applications}

%For most lattice-based cryptosystems, their security is guaranteed by the hardness of lattice problems such as $\gamma$-SVP.  Hence, the hardness of lattice problem in ideal lattice is widely considered as the security foundation of Ring-LWE based cryptosystems. 


 
% Due to the additional algebraic structure, the problem for the ideal lattice is usually conjectured to be easier than that for the general integer lattice. Some recent progress supports the argument well. Obviously, the algebraic structure depends on the polynomial ring that the ideal belongs to. 
 




% However, Theorem \ref{kthm} shows us that an ideal lattice can be embedded as ideals into different polynomial rings, which means that an ideal lattice may have different "algebraic structure" in different rings.
 
 %And we notice that these embeddings won't change the original lattice at all, and hence the hard lattice problems remain the same.

%This phenomenon inspires us to consider the following method to solve the hard problems for a given ideal lattice. By changing the polynomial ring, is it possible to transform the given ideal lattice as another ideal in which the lattice problems can be solved more efficiently by using the new algebraic structure? It seems hard to present a negative answer if  the algebraic structure can indeed help solve the hard problems,  since we have to consider infinite ideals and hence infinite algebraic structures. This no doubt increases the difficulty to show that the lattice problem for some fixed ideal lattice is hard.


%On the other hand, if we can utilize the algebraic structure to solve the  lattice problems in some ideal lattice, then we can solve the problems for infinite ideal lattice in different rings.  We would like to stress that as a lattice, these ideal lattices are same. However, as ideals, they are different. It seems that a weak ideal will spread as infinite weak ideals.
  
%Next we present some concrete examples to show the potential risk inspired by Theorem \ref{thmidentify}.

%\subsubsection{Pre-processing  a fixed ring brings more.}

%As we've mentioned in the introduction that in some monogenic number fields, the $\gamma$-SVPs in ideal lattices induced by two different embeddings of the same ideal are connected closely. Hence, in such number fields we can make use of the research on $\gamma$-SVP in the canonical embedding ideal lattices to deal with the $\gamma$-SVP in coefficient-embedding ideal lattices.

 %In \cite{ref10},   Pellet-Mary \textit{et al.} showed  pre-processing the number field can help solve $\gamma$-SVP in canonical-embedding ideal lattices more efficiently. However, pre-processing usually costs too much time. One may think for different number fields, we have to do different pre-processing. By Theorem \ref{thmidentify}, we know that  pre-processing  a fixed number field will also  help us solve  $\gamma$-SVP  more efficiently in ideals that is not in the algebraic integer ring of the fixed number field. 
 
 
 %Consider the ring $\mathbb{Z}[x]/(x^n+1)$, where $n=2^k$, which is one of the most used rings in cryptosystems. It is well known that the lengths of  vectors induced by the same element  under  the coefficient embedding and the canonical embedding are  the same up to a fixed factor, which means the hardness of SVP in such two embedding ideal lattices  are equivalent. Hence, by the method in \cite{ref10}, we can pre-process the  ring $\mathbb{Z}[x]/(x^n+1)$, and then can solve $\gamma$-SVP in its any coefficient-embedding ideal lattice more efficiently.  
 
 %Below we give a simple example to show how to apply the pre-processing on the ideal in other polynomial ring.
%\begin{example}
%Given a coefficient-embedding ideal lattice in the ring $\mathbb{Z}[x]/(x^n+x^{n-1}+2x^{n-2}+1)$ induced by the ideal $<x+2>$, where $n=2^k$,  the basis has the form \[\mathbf{B}=\begin{pmatrix}2&1&0&\cdots&\cdots&0\\0&2&1&\cdots&\cdots&0\\\vdots&\vdots&\vdots&\vdots&\vdots&\vdots\\0&0&\cdots&0&2&1\\-1&0&\cdots&0&-2&1\end{pmatrix}.\]
%Note that the greatest common divisor $d$ of the entries in the last column is $1$.
%To verify that $ \mathcal{L}(\mathbf{B})$ can be embedded as an ideal into the ring $\mathbb{Z}[x]/(x^n+1)$, according to Theorem \ref{kthm}, it's sufficient to verify the following relation:
%\[\sigma((x^n+x^{n-1}+2x^{n-2}+1)-(x^n+1))=\begin{pmatrix}0&0&\cdots&0&2&1\end{pmatrix}\in\mathcal{L}(\mathbf{B}),\]
%which is obvious.

%Therefore, by pre-processing the field $\mathbb{Q}[x]/(x^n+1)$, we can also handle the hard problems on the ideal lattice  $ \mathcal{L}(\mathbf{B})$ in $\mathbb{Z}[x]/(x^n+x^{n-1}+2x^{n-2}+1)$.
%\end{example}


%By the discussion above, our theorem has a good chance to amplify the results of the research on the ideal lattice of certain rings.
 
%\begin{remark}
%In fact, this argument can be extended to all  cyclotomic fields.
%As we all know, there exists irreducible polynomial $f(x)$ over $\mathbb{Z}[x]$, s.t. $\mathcal{O}(Q(\zeta))\cong \mathbb{Z}[x]/f(x)$, where $\zeta$ is a primitive unit root. Similarly to the ring $\mathbb{Z}[x]/(x^n+1)$, it's possible to handle the ideal lattices in the algebraic integer rings of any cyclotomic fields.  But the norm relation between the two embeddings in any cyclotomic field is slightly complex. In fact, there is a linear transformation between the two embeddings in cyclotomic field. We refer to \cite{ref11} for more details. Then our theory can extend Pellet-Marry's \cite{ref10} method to any integer lattice that's in the same class of the algebraic integer ring of some cyclotomic field.
%\end{remark}


%\subsubsection{Changing the ring  may be not enough for the security.}
%Sometimes, we want to choose a special ring for the cryptosystems to resist some potential attacks. This may work in general. However, for some fixed ideal lattices, this may be not enough to obtain the desired security.


%For example, NTRUPrime \cite{ref29} uses the ring $\mathbb{Z}[x]/(x^p-x-1)$ to resist the potential subfield attacks against NTRU, where $p$ is an odd prime. We next present a simple example to show that some ideals generated by polynomials with small coefficients in the ring $\mathbb{Z}[x]/(x^p-x-1)$   can also be embedded as an ideal into some $\mathbb{Z}[x]/f(x)$, where $f(x)$ is reducible. However, a  reducible $f(x)$ may cause some potential security risk.
%\begin{example}
	%For convenience, we assume that $p$ is large enough. Consider the coefficient-embedding ideal lattice induced by the principal ideal $<x^{p-1}-x^2-x>$    in the ring $\mathbb{Z}[x]/(x^p-x-1)$.  We show that such ideal lattice can be embedded as an ideal into the ring $\mathbb{Z}[x]/f(x)$, where $f(x) = (x+1)(x^{p-1}-x-1)$ is reducible.
	
	%The lattice basis of is 
	%\[\mathbf{B}=\begin{pmatrix} 0&-1&-1&0&0&\cdots&1\\1&1&-1&-1&0&\cdots&0\\\vdots&\vdots&\vdots&\vdots&\vdots&\vdots&\vdots\\\cdots&\cdots&\cdots&\cdots&\cdots&\cdots&\cdots\end{pmatrix}.\]
	
	%Note that the greatest common divisor $d$ of the entries in the last column is $1$.  According to Theorem \ref{kthm}, it's sufficient to verify the following relation:
	%$$\sigma((x+1)(x^{p-1}-x-1)-(x^p-x-1))=(0,-1,-1,0,\cdots,1) \in \mathcal{L}(\mathbf{B}),$$
	%and $ (0,-1,-1,0,\cdots,1)$ is exactly the first row of $\mathbf{B}$.
%\end{example}

\section{Identifying an Ideal Lattice}
\label{section:Identifying an Ideal Lattice}
\subsection{Algorithm}

According to Theorem \ref{thmidentify} and Theorem \ref{kthm},  we propose an algorithm to identify whether a given integer lattice is an ideal lattice or not (Algorithm 1).

\begin{algorithm}[htb]

	\caption{Identifying an ideal lattice}
	\label{alg:identify}
	\begin{algorithmic}[1]
		\Require $\mathbf{B} \in \mathbb{Z}^{n \times n}$, $\text{rank}(\mathbf{B})=n$.
		\Ensure False if $\mathcal{L}(\mathbf{B})$ is not a coefficient-embedding ideal lattice; Otherwise output a set $S\subset \mathbb{Z}^n$  s.t. for any $(g_1,g_2,...,g_n)\in S$, $\mathcal{L}(\mathbf{B})$ can be embedded as an ideal  into $\mathbb{Z}[x]/(g_1+g_2x^1+...+g_nx^{n-1}+x^n)$.
		\State  Compute  any  Incomplete Hermit Normal Form $\mathbf{B}'= \begin{pmatrix} \mathbf{D}&\mathbf{0} \\ \mathbf{b'}&b_{n,n}' \end{pmatrix}$ of $\mathbf{B}$ by unimodular transformation;
		\If{$b_{n,n}' \not\vert\  \mathbf{B}$} return False;
		\EndIf
	    \If {$\begin{pmatrix} \mathbf{0}&\mathbf{D}\end{pmatrix}\mathbf{B}^{-1}\notin\mathbb{Z}^{(n-1)\times n} $} return False;
		\EndIf
		\State  Output $S = \frac{1}{b'_{n,n}}(\begin{pmatrix}0&\mathbf{b}'\end{pmatrix}+\mathcal{L}(\mathbf{B}))$.
	\end{algorithmic}
\end{algorithm}

\begin{remark}\label{HNFremark}
	In Step 1, we can also compute the HNF of $\mathcal{L}(\mathbf{B})$, and then use the divisibility relation described in Lemma \ref{klemma} to rule out some integer lattices that can't be embedded as an ideal into any polynomial ring. This may speedup the algorithm in practice, since many "random" integer lattices can not pass such check.
\end{remark}

The correctness of Algorithm 1 is guaranteed by Theorem \ref{thmidentify} and Theorem \ref{kthm}
%\paragraph{Correctness} The correctness of Algorithm 1 is guaranteed by Theorem \ref{thmidentify}. 

%\paragraph{Complexity} We next analyze the time complexity.
\subsection{Complexity}
For Step 1, we can use Algorithm 2 to compute an Incomplete Hermite Normal Form for   $\mathbf{B} \in \mathbb{Z}^{n \times n}$ with a unimodular transformation, whose idea has already been described in Section 2.3.


\begin{algorithm}[htb]
	\caption{Computing an Incomplete Hermite Normal Form}
	\label{alg:ihnf}
	\begin{algorithmic}[1]
		\Require $\mathbf{B} \in \mathbb{Z}^{n \times n}$, $\text{rank}(\mathbf{B})=n$.
		\Ensure An Incomplete Hermit Normal Form of $\mathbf{B}$ by unimodular transformation.
		\For{$i$ from 1 to $n-1$}
		\State Use Extended Euclidean Algorithm with input $(b_{i,n},b_{i+1,n})$ to find $x$, $y$, $d$ s.t. $xb_{i,n}+yb_{i+1,n}=\gcd(b_{i,n},b_{i+1,n})=d$;
		\State  Update	$\begin{pmatrix} \mathbf{b_i} \\ \mathbf{b_{i+1}}\end{pmatrix}$:= $\begin{pmatrix}-b_{i+1,n}/d&b_{i,n}/d\\x&y\end{pmatrix}\begin{pmatrix} \mathbf{b_i} \\ \mathbf{b_{i+1}}\end{pmatrix}$;
		\EndFor
		\State  Output $\mathbf{B}$.
	\end{algorithmic}
\end{algorithm}

It is easy to check that the integer matrix $\begin{pmatrix}-b_{i+1,n}/d&b_{i,n}/d\\x&y\end{pmatrix}$ is unimodular since its determinant is $-1$. Hence, the transformation in Step 3 will not change the lattice $\mathcal{L}(\mathbf{B})$. After Step 3 for each $i$, we have $b_{i,n} = 0$ and $b_{i+1,n}= d$ computed by Step 2, which means that the output is in Incomplete Hermite Normal Form.

For the time complexity, we assume that for the input  $\mathbf{B}$, the absolute value of its every entry is bounded by $2^B$. 

It is easy to conclude that for the $i$-th loop, at the beginning, we have 
\begin{itemize}
	\item $|b_{i,j}|< 2^{i*B+1}$, $|b_{i+1,j}|< 2^{B}$ for $j =1,\cdots, n$, especially we have $|b_{i,n}|< 2^{B}$;
	\item $|x|<  2^{B}$, $|y|< 2^{B}$, $d<  2^{B}$.
\end{itemize}
Note that the Extended Euclidean Algorithm takes $\mathcal{O}(\text{log}\vert a \vert \text{log}\vert b\vert)$ bit operations on input $(a, b)$. Then for the $i$-th loop, with the plain integer multiplication we have:
\begin{itemize}
	\item Step 2 costs $\mathcal{O}(B^2)$ bit operations;
	\item Step 3 costs $\mathcal{O}(i*nB^2)$ bit operations;
\end{itemize} 
Hence, for the total $n$ loops, Algorithm \ref{alg:ihnf} needs $\mathcal{O}(n^3B^2)$ bit operations, and we have the following result.

\begin{lemma}\label{clemma}
	For a non-singular matrix $\mathbf{B}\in\mathbb{Z}^{n \times n}$, the absolute value of whose entries is bounded by $2^B$, Algorithm \ref{alg:ihnf} takes $\mathcal{O}(n^3B^2)$ bit operations to compute an Incomplete Hermite Normal Form of $\mathbf{B}$ by a unimodular transformation.
\end{lemma}

The most time-consuming part of Algorithm 1 is to judge whether $\begin{pmatrix}0&\mathbf{D}\end{pmatrix}\mathbf{B}^{-1}\in\mathbb{Z}^{(n-1)\times n}$ or not. In fact, there is an equivalent description for this and we refer to the results of Birmpilis et al \cite{ref30}.

\begin{theorem}[See Theorem 4 of \cite{ref30} ]\label{LatMemthm}
Let $\mathbf{B}\in\mathbb{Z}^{n\times n}$ be nonsingular with Smith form $\mathbf{S}$ and Smith massager $\mathbf{M}$. Let $s$ be the largest invariant factor of $\mathbf{S}$. The following lattices are identical:

$L_1=\{v|v\mathbf{B}^{-1}\in\mathbb{Z}^{1\times n}\}$

$L_2=\{v|v\mathbf{M}\equiv0_{1\times n}$ cmod $\mathbf{S}\}$
\end{theorem}

 By Theorem \ref{LatMemthm}, $L_1=L_2$, which means to judge whether $\begin{pmatrix}0&\mathbf{D}\end{pmatrix}\mathbf{B}^{-1}$ $\in\mathbb{Z}^{(n-1)\times n}$ or not , it's sufficient to verify $\begin{pmatrix}0&\mathbf{D}\end{pmatrix}\mathbf{M}\equiv 0_{(n-1)\times n}$ cmod $\mathbf{S}$. $\mathbf{S}$ is the Smith Norm Form of $\mathbf{B}$, and it's diagonal.  %Hence, $\begin{pmatrix}0&\mathbf{D}\end{pmatrix}\mathbf{M}\equiv 0_{(n-1)\times n}\mathbf{c}\mod$$\mathbf{S}$ means for any $i=1,2,\cdots,n$ the $i$-th column of  $\begin{pmatrix}0&\mathbf{D}\end{pmatrix}\mathbf{M}$ is a $\mathbf{0}$ vector mod $s_i$, where $s_i$ is the $i$-th element of the diagonal of $\mathbf{S}$. 
 The following theorem is also proposed by Birmpilis et al
\cite{ref30} to compute the Smith Normal Form $\mathbf{S}$ and a reduced Smith Massager $\mathbf{M}$ of the input matrix ($\mathbf{M}$ is reduced $\mathbf{c}$mod $\mathbf{S}$)

\begin{theorem}[See Theorem 19 of \cite{ref30}]\label{Sthm}
There exists a Las Vegas algorithm that takes as input a nonsingular $\mathbf{A}\in\mathbb{Z}^{n\times n}$, and returns as output the Smith Normal Form $\mathbf{S}\in\mathbb{Z}^{n\times n}$ and a reduced Smith Massager $\mathbf{M}\in\mathbb{Z}^{n\times n}$ of the input matrix. The cost of the algorithm is $\mathcal{O}(n^{\omega}\text{B}(\log n+\log \|A\|)(\log n)^2)$ bit operations. The algorithm returns Fail with probability at most 1/2.
\end{theorem}

$\text{B}(d)=\mathcal{O}(M(d)\log d)$ and $M(d)$ bounds the number of bit operations required to multiply two integers bounded in magnitude by $2^d$.  We take $M(d)=\mathcal{O}(d^2)$. $\omega$ is a valid exponent of matrix multiplication: two $n\times n$ matrices can be multiplied in $\mathcal{O}(n^{\omega})$ operations from the domain of the entries, and the best known upper bound is $\omega<2.37286$ by Alman and Williams \cite{ref31}.

After computing $\mathbf{M}$ and $\mathbf{S}$, it remains to compute $\begin{pmatrix}0&\mathbf{D}\end{pmatrix}\mathbf{M}$ $\mathbf{c}$mod $\mathbf{S}$. It's easy to check that the entries of the output of Algorithm 2 $\begin{pmatrix}0&\mathbf{D}\end{pmatrix}$ are bounded by $2^{3B}$.  We first consider the computation of the remainder modulo $Y$ of the product of two integers. Recall that $\text{Rem}(ab,Y)$ has cost bounded by $\mathcal{O}((\log ab/Y)(\log Y))$, so the cost of computing the $i$-th element of  $\begin{pmatrix}0&\mathbf{D}\end{pmatrix}\mathbf{M}$ $\mathbf{c}$mod $\mathbf{S}$ is bounded by $\mathcal{O}(n(3B)(\log s_i))$. Hence, the cost of computing  $\begin{pmatrix}0&\mathbf{D}\end{pmatrix}\mathbf{M}$ $\mathbf{c}$mod $\mathbf{S}$ is bounded by $\mathcal{O}(3nB\log |\text{det}(\mathbf{B})|)=\mathcal{O}(3n^2B(B+\log n))$

Combining Lemma 7, Theorem 4 and the discussion above, we have

\begin{theorem}\label{complexity}
Given $\mathbf{B} \in \mathbb{Z}^{n \times n}$, $\text{rank}(\mathbf{B})=n$, and the absolute value of the entries of $\mathbf{B}$ is bounded by $2^B$, then there is a Las Vegas algorithm with expected complexity  $\mathcal{O}(n^3B(B+\log n))$ to identify whether $\mathcal{L}(\mathbf{B})$ is an ideal lattice or not.
\end{theorem}














%For Step 4, we refer to Theorem 37 of \cite{ref12} for more details.

%\begin{theorem}[Theorem 37 in \cite{ref12}] \label{thmsto}
	%There exists a Las Vegas algorithm that takes as input a non-singular $\mathbf{A} \in \mathbb{Z}^{n \times n}$ 
    %and $\mathbf{b} \in \mathbb{Z}^{n}$, and returns as output the vector $\mathbf{b}\mathbf{A}^{-1}\in\mathbb{Q}^n$. If the absolute value of the entries of $\mathbf{A}$ is bounded by $2^B$, and  the absolute value of the entries of $\mathbf{b}$ is bounded by $2^{nB}$, then
	%the expected cost of the algorithm is $\mathcal{O}((\log n)\mathbf{MM}(n)\mathbf{MZ}(B + \log n))$ bit operations, where 
	%$\mathbf{MM}(n$) means two $n\times n$ matrices can be multiplied using at most $\mathbf{MM}(n)$ integer multiplications and $\mathbf{MZ}(B$) means two $B$-bits integer  can be multiplied using at most $\mathbf{MZ}(B)$ bit operations.	
	%This result assumes
	%that $\mathbf{MZ}(t) = O(\mathbf{MM}(t)/t)$.

%\end{theorem}


%It is It is well known that the classical plain multiplication method allows $\mathbf{MZ}(B)= O(B^2)$, and 
%the Sch\"{o}nhage-Strassen algorithm \cite{ref24} allows $\mathbf{MZ}(B)=O(B(\text{log}B$)((\text{log} $\text{log}B$).


%For $\mathbf{MM}(n)$, the classical plain multiplication method allows $\mathbf{MM}(n)=2n^3-n^2$,  and the asymptotically faster method allows $\mathbf{MM}(n)=O(n^{2.376})$. We refer to \cite{ref17} and \cite{ref18} for more details and further discussion. 

%For simplicity, we adopt the classical plain method for both integer multiplication and matrix multiplication, that is  $\mathbf{MZ}(B)= O(B^2)$ and $\mathbf{MM}(n)=O(n^3)$. Then by Theorem \ref{thmsto} a simple analysis shows that Step 4 in Algorithm \ref{alg:identify} costs  $\mathcal{O}(n^4\log n(B + \log n)^2)$ bit operations



%Together with Lemma \ref{clemma}, we have
%\begin{theorem}\label{complexity}
%Given $\mathbf{B} \in \mathbb{Z}^{n \times n}$, $\text{rank}(\mathbf{B})=n$, and the absolute value of the entries of $\mathbf{B}$ is bounded by $2^B$, then there is a Las Vegas algorithm with expected complexity  $\mathcal{O}(n^4\log n(B + \log n)^2)$ to identify whether $\mathcal{L}(\mathbf{B})$ is an ideal lattice or not.
%\end{theorem}
\subsection{Related research}

In 2007, Ding and Lindner \cite{ref13} already proposed an algorithm for identifying ideal lattices, but we find that there is a flaw in their algorithm. More exactly, some ideal lattices can't be identified by their algorithm.

We find some non-trivial ideal lattices which can't be identified by Ding and Lindner's algorithm. The following is an example:
\[\mathbf{B}=\begin{pmatrix}6&-8&-5\\3&-7&-4\\6&1&-1\end{pmatrix}\]

The row vectors of $\mathbf{B}$ span a full-rank ideal lattice in the ring $\mathbb{Z}[x]/x^3+3x^2+x^1-3$. However, with the input $\mathbf{B}$, Ding and Lindner's algorithm return false. 

More exactly, in their algorithm, the lattice is spanned by column vectors, so the input matrix should be $\mathbf{B}^T$. They first transform $\mathbf{B}^T$ into an upper-triangular Hermite Normal Form $\mathbf{H}$.
\[\mathbf{H}=\begin{pmatrix}9&6&0\\0&1&0\\0&0&1\end{pmatrix}\]
Then they compute the adjugate matrix $\mathbf{A}$ of $\mathbf{H}$.
\[\mathbf{A}=\begin{pmatrix}1&-6&0\\0&9&0\\0&0&9\end{pmatrix}\]
Let $\mathbf{I_n}$ be the unit matrix of dimension $n$, and $\mathbf{M}$ be a matrix only related to the dimension $n$ (For this example, the dimension is 3).
\[\mathbf{M}=\begin{pmatrix}\mathbf{0}&0\\\mathbf{I_{n-1}}&\mathbf{0}\end{pmatrix}\]
In step 4 of their algorithm, they need to verify whether only the last column $\mathbf{AMH}$ mod $\det(\mathbf{B})$ is equal to $\mathbf{0}$ or not. If the input lattice basis $\mathbf{B}$ spans an ideal lattice, they believe by default only the last column $\mathbf{AMH}$ mod $\det(\mathbf{B})$ is not equal to $\mathbf{0}$. However, $\mathbf{AMH} \equiv \mathbf{0}$ mod $\det(\mathbf{B})$, which causes their algorithm to return "false". Apparently, they ignore the situation that all the column of $\mathbf{AMH}$ mod $\det(\mathbf{B})$ is equal to $\mathbf{0}$.

Ignoring the flaw above, our algorithm still performs better than theirs in two aspects:
\begin{itemize}
    \item Our algorithm outputs more. Ding and Lindner's algorithm outputs a single polynomial ring of the ring class if the input lattice is an ideal lattice but ours outputs the entire ring class.
    \item The time complexity of our algorithm is lower. It is claimed  in \cite{ref13}  that the algorithm presented by Ding and Lindner to identify an ideal lattice  costs $\mathcal{O}(n^4B^2)$ bit operations. However, we have to point out that there is also a flaw in the complexity analysis in  $\mathcal{O}(n^4B^2)$.  The algorithm in \cite{ref13}  needs to compute $n-2$ powers of $\mathbf{B}$, that is, $\mathbf{B}^k$ for $k = 2, \cdots, n-1$. It is claimed this can be done within $\mathcal{O}(n^4B^2)$ bit operations. However, when $k$ grows bigger, the bit size of the entries in $\mathbf{B}^k$ will be $\mathcal{O}(kB)$ instead of $B$. Hence the correct time complexity  should be 
$$\sum_{k=2}^{n-1}\mathcal{O}(n^3*k*B^2) =  \mathcal{O}(n^5B^2).$$
\end{itemize}

%\begin{remark}\label{compare}
%It is claimed  in \cite{ref13}  that the algorithm presented by Ding and Lindner to identify an ideal lattice  costs $\mathcal{O}(n^4B^2)$ bit operations. However, we have to point out that there is some flaw in the complexity analysis in  $\mathcal{O}(n^4B^2)$.  The algorithm in \cite{ref13}  needs to compute $n-2$ powers of $\mathbf{B}$, that is, $\mathbf{B}^k$ for $k = 2, \cdots, n-1$. It is claimed this can be done within $\mathcal{O}(n^4B^2)$ bit operations. However, when $k$ grows bigger, the bit size of the entries in $\mathbf{B}^k$ will be $\mathcal{O}(kB)$ instead of $B$. Hence the correct time complexity  should be 
%$$\sum_{k=2}^{n-1}O(n^3*k*B^2) =  O(n^5B^2).$$

%So our algorithm is faster than the algorithm in \cite{ref13}, due to the fact that our algorithm just checks if the systems of equations have integer solutions or not.

%Besides, the algorithm in \cite{ref13}  outputs a single polynomial ring of the ring class if the input lattice is an ideal lattice but we compute the entire class.
%\end{remark}
\subsection{Experiment}
Using our algorithm, we conducted several experiments, and the experimental results are presented in Appendix~\ref{AppendixA}.
\section{Conclusion}
In this paper, we explore the connection between integer lattices and coefficient-embedding ideal lattices. We have three main contributions:

Firstly, we find and proof an ideal lattice can be viewed as an ideal lattice in infinitely many different polynomial rings. This interesting phenomenon may contribute to the solution to hard ideal lattice problems as mentioned in Remark \ref{rmk:app}.

Secondly, we propose an efficient algorithm for identifying ideal lattices, and compared to related work, our algorithm has more advantages.

Finally, we provide an efficient open source implementation of our algorithm for identifying ideal lattices in SageMath. Our experimental results are presented in Appendix~\ref{AppendixA}.

%In this paper, we reveal the embedding relation between the coefficient-embedding ideal lattice and the integer lattice, which gives us a new method to solve the ideal lattice problems by embedding a given ideal lattice into the well-studied polynomial ring. 
 
 %Hence, it's not proper anymore to judge the security of a crypstosystem based on ideal lattice by just considering a single ring. The embedding relation introduced in this article no doubt increases the difficulties of evaluating the security for any crypstosystem based on ideal lattices. 

%Since the ideal lattice is a special case of the module lattice, it's possible that there is a similar embedding relation between integer lattices and module lattices. Therefore, it's worth researching that how to extend our theory to module lattice.

%As a direct production of our theorem, we introduce an efficient method to identifying an ideal lattice. Since our algorithm only computes the gcd of the last column of the given lattice, it avoids the explosion of the matrices coefficients. Besides,our algorithm not only identify an ideal lattice, but also computes the entire class of the rings which the given integer lattice can be embedded into
\newpage
\bibliographystyle{unsrt} 
\bibliography{sample-base}

\appendix
\section{Experiments}
\label{AppendixA}
In this section, we present our experimental results and some intersting findings about density of ideal lattice. The experiments were conducted in the SageMath 9 environment on a personal computer equipped with an Intel Core i7-13700KF 3.40 GHz processor. The source code for the experiments is open-sourced and available at \begin{center}
  \textcolor{blue}{\url{https://github.com/fffmath/Identifying-Ideal-Lattice}}.
\end{center} It allows simulations of experiments with input dimensions, bounds, and the number of experiments.

We compared our algorithm with the one proposed by Ding and Lindner~\cite{ref13}. Under the same parameters, our algorithm demonstrated a significant advantage in terms of runtime.

%During the experiments, we found a flaw in their algorithm. It misclassified some examples that should have been ideal lattices. Not only did it fail to correctly identify unit lattices, but it also struggled with non-trivial lattice bases. The algorithm ignored the case of $zF(Aq) \equiv 0 \pmod{\det(B)}$ and assumed $zF(Aq) \not\equiv 0 \pmod{\det(B)}$ in the proof of their Theorem 1. However, the error rate decreased with increasing matrix dimensions in the simulation. This could be attributed to the decreasing proportion of ideal lattices as dimensions increased, reducing the likelihood of misclassifying them. Additionally, we guess the probability of $zF(Aq) \equiv 0 \pmod{\det(B)}$ decreased with increasing dimensions, further reducing the error rate.

Regarding algorithm runtime, we conducted multiple experiments with different variables. For input parameters \textit{dim} and \textit{bound}, we randomly generated a \textit{dim}-dimensional matrix within the specified \textit{bound} as the lattice basis. In other words, this results in the generation of a dim $\times$ dim matrix, where each element of the matrix falls within the range of $-2^{\textit{bound}}$ to $2^{\textit{bound}}$.


Two scenarios were considered:

\begin{itemize}
    \item Fixing the dimension (\textit{dim}): We kept \textit{dim} constant and recorded the runtime as \textit{bound} increased gradually.
    \item Fixing the bound (\textit{bound}): We kept \textit{bound} constant and recorded the runtime as \textit{dim} increased.
\end{itemize}

The relevant experimental results can be found in Figure~\ref{fig:lattice}.

For parameters with \textit{dim} less than 300, we conducted 100 experiments for each parameter and recorded the average time consumption as the time record. We observed that these data have very low variance, with each data point closely approaching the mean.

For parameters with large \textit{dim}, due to the longer individual runtime, we performed five experiments for each group and used the average of these five values as the time consumption.


% Figure environment removed

Note that as dimensions or bounds increased, the proportion of ideal lattices became very small. Therefore, most of the generated lattices in the former experiments weren't ideal lattice, resulting in runtime data just be not suitable for ideal lattice input.

To further explore ideal lattices, we conducted additional experiments using ideal lattice as input. We randomly selected polynomials $f$ with coefficients in \{-1,0,1\} and $g$ with coefficients in (${-2^{\textit{bound}}, 2^{\textit{bound}}}$) and computed the lattice basis of the principal ideal generated by $g$ in $\mathbb{Z}[x]/f(x)$, ensuring it is an ideal lattice. In such case, we take the coefficient vectors of $x^ig(x)\text{mod}f(x)$ as the lattice basis, and the reason why we limit the coefficients of $f(x)$ in \{-1,0,1\} is to decrease the exploration of the coefficients of ideal lattice basis generated by $g$. Similarly as former experiments, we also performed experiments with fixed dimensions, recording the runtime as \textit{bound} varied, and fixed bounds, recording the runtime as \textit{dim} varied. The relevant experimental results can be found in Figure~\ref{fig:ideal-lattice}. 

% Figure environment removed

To facilitate the comparison of different parameters and the runtime under various inputs, you can refer to the data table in Table~\ref{table:experiments}.
\begin{table}[htbp]
    \centering
\begin{tabular}{ccc}
   \toprule
   (dim, bound) & lattice (s)& ideal lattice (s)\\
   \midrule
    (100, 5) & 0.406 & 0.467 \\
    (100, 10) & 0.555 & 0.598 \\
    (100, 15) & 0.713 & 0.759 \\
    (100, 20) & 0.894 & 0.934 \\
    (200, 5) & 3.999 & 5.538 \\
    (200, 10) & 5.607 & 7.503 \\
    (200, 15) & 7.494 & 8.203 \\
    (200, 20) & 9.365 & 11.140 \\
    (300, 5) & 16.426 & 30.870 \\
    (300, 10) & 23.916 & 37.507 \\
    (300, 15) & 30.485 & 44.475 \\
    (300, 20) & 39.398 & 57.703 \\
    (400, 5) & 46.075 & 93.985 \\
    (400, 10) & 61.436 & 103.909 \\
    (400, 15) & 87.487 & 136.954 \\
    (400, 20) & 115.221 & 153.318 \\
    (500, 5) & 110.583 & 192.532 \\
    (500, 10) & 144.965 & 297.249 \\
    (500, 15) & 204.832 & 313.888 \\
    (500, 20) & 270.002 & 393.900 \\
   \bottomrule
\end{tabular}
\caption{Experimental results for cost time when using random lattice/ideal lattice as input.}
\label{table:experiments}
\end{table}

Finally, although finding an ideal lattice in high dimensions is challenging, we conducted experiments in lower dimensions to estimate the reduction factor. We investigated the density of ideal lattices in low dimensions and small bounds. We performed 100,000 experiments for each parameter $\textit{dim}=3$, $\textit{bound}=3,4,5,6,7$ and $\textit{bound}=3$, $\textit{dim}=2,3,4,5,6$, recording the quantity of ideal lattices under different parameters. We observed a rapid decrease in the proportion of ideal lattices in Figure~\ref{fig:density}.
% Figure environment removed

\end{document}
\endinput
%%
%% End of file `sample-lualatex.tex'.
