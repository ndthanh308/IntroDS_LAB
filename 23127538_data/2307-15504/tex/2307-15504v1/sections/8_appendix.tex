\appendix

\section*{Appendices}
\label{sec:appendix}

\section{Case Study}
We list some examples of our format transfer process in this section. 

\clearpage
\centering
% Figure removed

\clearpage

\centering
% Figure removed

% \begin{tcolorbox}[width=\textwidth,
%                   %%enhanced,
%                   %%frame hidden,
%                   interior hidden,
%                   boxsep=0pt,
%                   left=6pt,
%                   right=6pt,
%                   ]%%
% {\color{blue} \textbf{\texttt{Transferring from P3 to Ni-v2 (Task: trec):}}}

% {\color{red} \textbf{\texttt{Instruction:}}}
% \begin{markdown}
% You are given a sentence which contains a question and a possible answer type. Your task is to identify the correct answer type from the suggested options. You may need to read the sentence and its context carefully in order to determine the correct answer type.
% \end{markdown}

% {\color{orange} \textbf{\texttt{Positive Examples: }}}

% \begin{markdown}
% **Input:** What do bee hives do in cranberry bogs ?\nIs this asking about Description, Entity, Abbreviation, Person, Quantity, Location?

% **Output:** Description

% **Explanation:** The question is asking for a description of what bee hives do in cranberry bogs. So the correct answer type is Description.

% \end{markdown}

% {\color{orange} \textbf{\texttt{Negative Examples: }}}
% \\
% \begin{markdown}
% **Input:** What golfing accessory was patented by George Grant on December 12\nIs this asking about Description, Entity, Abbreviation, Person, Quantity, Location?

% **Output:** Quantity

% **Explanation:** The sentence is asking about a golfing accessory. These types of questions typically require an answer about an entity (i.e. a specific object or thing), so the correct answer type is 'Entity' rather than 'Quantity'.
% \end{markdown}


% {\color{black} \rule{\linewidth}{0.2mm}}

% {\color{blue} \textbf{\texttt{Original P3(trec):}}}

% {\color{red} \textbf{\texttt{Instruction: }}}
% \begin{markdown}
% {Input}\n\nDescriptors: Description, Entity, Abbreviation, Person, Quantity, Location\n\nBest Descriptor?\n
% \end{markdown}

% {\color{orange} \textbf{\texttt{Positive Examples: }}}

% \begin{markdown}

% **Input:** What do bee hives do in cranberry bogs ?\n

% **Output:** \nDescription

% \end{markdown}

% \end{tcolorbox}

% % examples for flan to ni-v2
% \begin{tcolorbox}[width=\textwidth,
%                   %%enhanced,
%                   %%frame hidden,
%                   interior hidden,
%                   boxsep=0pt,
%                   left=6pt,
%                   right=6pt,
%                   ]%%
% {\color{blue} \textbf{\texttt{Transferring from Flan to Ni-v2 (Task: wsc):}}}

% {\color{red} \textbf{\texttt{Instruction:}}}
% \begin{markdown}
% In this task, you are given two sentences (sentence 1 and sentence 2). If sentence 1 implies that sentence 2 is true, answer "Yes", otherwise "No".
% \end{markdown}

% {\color{orange} \textbf{\texttt{Positive Examples: }}}

% \begin{markdown}
% **Input:** sentence 1: As Ollie carried Tommy up the long winding steps, his legs ached. sentence 2: Tommy's legs ached. options: - no - yes.

% **Output:** no

% **Explanation:** The sentence 1 does not imply that Tommy's legs ached. So, the output should be "No".

% \end{markdown}

% {\color{orange} \textbf{\texttt{Negative Examples: }}}
% \\
% \begin{markdown}
% **Input:** Sentence 1: Paul couldn't find his car keys, so he had to walk. Sentence 2: Paul had to walk.

% **Output:** Yes

% **Explanation:** Sentence 1 implies that sentence 2 is true, so the correct output should be "Yes".
% \end{markdown}


% {\color{black} \rule{\linewidth}{0.2mm}}

% {\color{blue} \textbf{\texttt{Original Flan(wsc):}}}

% {\color{red} \textbf{\texttt{Instruction: }}}
% \begin{markdown}
% If "{sentence1}", can we conclude that "{sentence2}"\n{options_}
% \end{markdown}

% {\color{orange} \textbf{\texttt{Positive Examples: }}}

% \begin{markdown}

% **Input:** If "As Ollie carried Tommy up the long winding steps, his legs ached.", can we conclude that "Tommy's legs ached."\n\nOPTIONS:\n- no\n- yes.

% **Output:** no

% \end{markdown}

% \end{tcolorbox}


% % examples for crossfit to ni-v2
% \clearpage
% \begin{tcolorbox}[width=\textwidth,
%                   %%enhanced,
%                   %%frame hidden,
%                   interior hidden,
%                   boxsep=0pt,
%                   left=6pt,
%                   right=6pt,
%                   ]%%
% {\color{blue} \textbf{\texttt{Transferring from CrossFit to Ni-v2 (Task: imdb):}}}

% {\color{red} \textbf{\texttt{Instruction:}}}
% \begin{markdown}
% You are given a sentence or a paragraph describing a particular topic. Your task is to classify the sentiment of the sentence/paragraph as either negative or positive. Label the sentiment in the output as per the given parameters without incorporating any additional information into your answer.
% \end{markdown}

% {\color{orange} \textbf{\texttt{Positive Examples: }}}

% \begin{markdown}
% **Input:** I am a big fan a Faerie Tale Theatre and I've seen them all and this is one of the best! It's funny, romantic, and a classic. I recommend this for all ages. It's great for little kids because it's well, Cinderella and great for adults and teen because it's funny and not over the top. I watched it when I was little and I still watch it now. It has great lines that my family and I quote all the time. The acting is great and it never gets old. If you like fairy tales and romances you will love this. I've watched many a Cinderella movie in my time and this is the best of them all. (Sorry Disney) I highly recommend this movie and all the Faerie Tale Theatre shows. They all appeal to all ages and are all unique and very entertaining.

% **Output:** positive

% **Explanation:** The sentence describes a very positive opinion on the chosen topic. The opinion is supported by facts, like the uniqueness of the show, its lasting values, great acting, and so on. Hence, the sentiment of the sentence is classified as positive.

% \end{markdown}

% {\color{orange} \textbf{\texttt{Negative Examples: }}}
% \\
% \begin{markdown}
% **Input:** I know a few things that are worst. A few. It had a couple of funny scenes. It is a movie not appropriate for kids but, only a child would find this movie hilarious. This is definetly a movie that you would like to use a free rental coupon for. Don't waste your money just to laugh a couple of times.

% **Output:** Positive

% **Explanation:** The given sentence is mainly negative in nature as it suggests not to waste money on the movie. The word \"definetly\" and \"a couple of funny scenes\" are used in the sentence to provide a bit of contrast, yet it does not make the overall sentiment of the sentence positive. Therefore, the correct answer should be \"negative\" instead of \"positive\".
% \end{markdown}


% {\color{black} \rule{\linewidth}{0.2mm}}

% {\color{blue} \textbf{\texttt{Original CrossFit(imdb):}}}

% {\color{red} \textbf{\texttt{Instruction: }}}
% \begin{markdown}

% \end{markdown}

% {\color{orange} \textbf{\texttt{Positive Examples: }}}

% \begin{markdown}

% **Input:** I am a big fan a Faerie Tale Theatre and I've seen them all and this is one of the best! It's funny, romantic, and a classic. I recommend this for all ages. It's great for little kids because it's well, Cinderella and great for adults and teen because it's funny and not over the top. I watched it when I was little and I still watch it now. It has great lines that my family and I quote all the time. The acting is great and it never gets old. If you like fairy tales and romances you will love this. I've watched many a Cinderella movie in my time and this is the best of them all. (Sorry Disney) I highly recommend this movie and all the Faerie Tale Theatre shows. They all appeal to all ages and are all unique and very entertaining.

% **Output:** positive

% \end{markdown}

% \end{tcolorbox}


% % examples for ni-v2 to flan
% \begin{tcolorbox}[width=\textwidth,
%                   %%enhanced,
%                   %%frame hidden,
%                   interior hidden,
%                   boxsep=0pt,
%                   left=6pt,
%                   right=6pt,
%                   ]%%
% {\color{blue} \textbf{\texttt{Transferring from Ni-v2 to Flan (Task: dialogre):}}}

% {\color{red} \textbf{\texttt{Instruction:}}}
% \begin{markdown}
% {input} Identify the name of one of the speakers in the given dialog.
% \end{markdown}

% {\color{black} \rule{\linewidth}{0.2mm}}

% {\color{blue} \textbf{\texttt{Original Ni-v2(dialogre):}}}

% {\color{red} \textbf{\texttt{Instruction: }}}
% \begin{markdown}
% You are given a dialog between 2 or more individuals. Within the dialog, there will be clues as to the names of the speakers. You will be asked at the end of the dialog to identify the name of one of the speakers.
% \end{markdown}
% \end{tcolorbox}


% examples for adding cot in ni-v2
% \begin{tcolorbox}[width=\textwidth,
%                   %%enhanced,
%                   %%frame hidden,
%                   interior hidden,
%                   boxsep=0pt,
%                   left=6pt,
%                   right=6pt,
%                   ]%%
% {\color{blue} \textbf{\texttt{Ni-v2 with COT(doqa):}}}

% {\color{red} \textbf{\texttt{Instruction:}}}
% \begin{markdown}
% Given a paragraph about cooking, and a set of conversational question answers about the paragraph, generate a relevant question to the topic of the paragraph. The paragraph has the prefix 'CONTEXT:'. Each conversation question has a prefix `Q:` followed by the answer prefix `A:`.
% \end{markdown}

% {\color{orange} \textbf{\texttt{Positive Examples: }}}

% \begin{markdown}
% **Input:** CONTEXT: It not huge, it's just the difference from freezer to room temperature you are worrying aboutE.g. -20°C to 20°C, is A 40°C shift. The shift was going to be 20°C to 100°C anyway. There is no physical reasons why this would be anymore stressfulFrom room temperature you are raising it 80°C, from frozen you are raising it 120°C. Not a problem in the normal temperature range for glassThe freezing temperature of water (most common item in food) has no relation to the freezing temperature of glass etc Pyrex and other glasses can be damaged if one part of them is instantly heated or cooled by a 100°C or so. <sep> Q: Is it safe to microwave Pyrex containers immediately after removing them from the freezer and removing the plastic lid? A: Pyrex and other glasses can be damaged if one part of them is instantly heated or cooled by a 100°C or so.

% **Output:** Do you think they will they shatter?

% **Explanation:** This is a valid question as it is a follow-up question of the conversation regaring whether pyrex and other glasses (refered to as they) will shatter or not on microwaving immediately after removing them from the freezer.

% **Chain-of-Thoughts:** The conversation questions are about microwaving Pyrex containers after removing them from the freezer. The answer to the question states that Pyrex and other glasses can be damaged if one part of them is instantly heated or cooled by a 100°C or so. <<Can be damaged -> Shatter>> Therefore, the relevant question to the topic of the paragraph is \"Do you think they will they shatter?\".

% \end{markdown}


% \end{tcolorbox}



% \clearpage
% % examples for denoising with ppl scoring
% \begin{tcolorbox}[width=\textwidth,
%                   %%enhanced,
%                   %%frame hidden,
%                   interior hidden,
%                   boxsep=0pt,
%                   left=6pt,
%                   right=6pt,
%                   ]%%
% {\color{blue} \textbf{\texttt{Denoising with Perplexity Scoring: }}}

% {\color{red} \textbf{\texttt{Sample 1:}}}
% \begin{markdown}
% You are given questions about various topics and asked to provide the correct answers. The questions could relate to geographical facts, biographical information, cultural data, dates, and more. The answers should be accurate and detailed enough to provide a complete understanding of the question.
% \end{markdown}

% {\color{orange} \textbf{\texttt{Perplexity Score:}}}
% \begin{markdown}
% 3.97
% \end{markdown}
% \\

% {\color{red} \textbf{\texttt{Sample 2:}}}
% \begin{markdown}
% You are given a question in the input. Your task is to find the corresponding answer to the question. Depending on the complexity of the answer, the output can be in the form of a text string or a URL link to a web page with more detailed information.
% \end{markdown}

% {\color{orange} \textbf{\texttt{Perplexity Score:}}}
% \begin{markdown}
% 7.28
% \end{markdown}
% \\

% {\color{red} \textbf{\texttt{Sample 3:}}}
% \begin{markdown}
% You are provided with a question in the input. Your task is to provide the correct answer to the given question. Make sure that the answer is accurate, relevant and specific.
% \end{markdown}

% {\color{orange} \textbf{\texttt{Perplexity Score:}}}
% \begin{markdown}
% 3.69
% \end{markdown}
% \\

% {\color{red} \textbf{\texttt{Sample 4:}}}
% \begin{markdown}
% You are given a question in the input. Your task is to provide a correct answer to that question based on the knowledge you have. The output should be a short, direct and accurate response to the given question.
% \end{markdown}

% {\color{orange} \textbf{\texttt{Perplexity Score:}}}
% \begin{markdown}
% 4.15
% \end{markdown}

% \end{tcolorbox}


% \clearpage


% Denoising testing time experiment
% \begin{table*}[!htb]
%   \centering
%     \begin{tabular}{l l cc}
%     \toprule
%     \multicolumn{4}{c}{\textbf{Train on Ni-v2, Denoising at Testing Time}} \\
%     \midrule
%     \textbf{Benchmark} & \textbf{Method} & \multicolumn{2}{c}{\textbf{\underline{DPN}}} \\
%     & & EM & Rouge-L\\ 
  
%     \midrule
%     PromptSource & conventional & 30.8 & 36.1\\
%     PromptSource & denoising, sample scale = 2 & 33.0 & 38.9\\
%     PromptSource & denoising, sample scale = 4 & 33.5 & 39.4\\
%     PromptSource & denoising, sample scale = 8 & 33.7 & 39.7\\
%     PromptSource & denoising, sample scale = 16 & 33.8 & 39.8\\
%     PromptSource & denoising, sample scale = 32 & \textbf{34.0} & \textbf{40.0}\\
  
%     \bottomrule
%     \end{tabular}%
%     \caption{\label{testing-time-denoise}
% Results during testing time with denoising strategy of different sampling scales on PromptSource. Other experiment setting is consistent with testing time in experiemnt1.
% }
% \end{table*}


% % testing time scaling exp em.
% % Figure environment removed

% % testing time scaling exp rougel.
% % Figure environment removed

% % Ni-v2 and Crossfit task numbers exp em.
% % Figure environment removed

% % Ni-v2 and Crossfit task numbers exp rougel.
% % Figure environment removed

% Human evaluation.
% \begin{table}[!htb]
%   \centering
%     \begin{tabular}{l c c c c}
%     \toprule
%     \textbf{Item} & \textbf{crossfit} & \textbf{p3} & \textbf{flan} & \textbf{avg}\\
%     \midrule
%      def. correctness & $88$ & $89$ & $100$ & $92$\\
%      def. alignment & $75$ & $73$ & $88$ & $79$ \\
%      pos. explanation & $83$ & $87$ & $91$ & $87$ \\
%      neg. example & $54$ & $51$ & $21$ & $42$ \\
%      neg. explanation & $47$ & $53$ & $24$ & $41$\\
%     \bottomrule
%     \end{tabular}%
%     \caption{\label{human-evaluation}
% Human evaluation results before denoising.
% }
% \end{table}


% % Figure environment removed
