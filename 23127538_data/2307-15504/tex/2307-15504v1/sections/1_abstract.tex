% 
\begin{abstract}
Instruction tuning has emerged as a promising approach to enhancing large language models in following human instructions. It is shown that increasing the diversity and number of instructions in the training data can consistently enhance generalization performance, which facilitates a recent endeavor to collect various instructions and integrate existing instruction tuning datasets into larger collections. However, different users have their unique ways of expressing instructions, and there often exist variations across different datasets in the instruction styles and formats, i.e., format inconsistency. In this work, we study how format inconsistency may impact the performance of instruction tuning. We propose a framework called ``\underline{U}nified \underline{I}nstruction \underline{T}uning'' (UIT), which calls OpenAI APIs for automatic format transfer among different instruction tuning datasets. We show that UIT successfully improves the generalization performance on unseen instructions, which highlights the importance of format consistency for instruction tuning. 
To make the UIT framework more practical, we further propose a novel perplexity-based denoising method to reduce the noise of automatic format transfer. We also train a smaller offline model that achieves comparable format transfer capability than OpenAI APIs to reduce costs in practice. The codes and trained models are publicly at \url{https://github.com/thunlp/UnifiedInstructionTuning}.
\end{abstract}