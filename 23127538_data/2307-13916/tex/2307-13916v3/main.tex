\documentclass[11pt, letterpaper]{article}
\pdfoutput=1
\usepackage[authoryear]{natbib}
\usepackage{amsmath,amssymb,amsfonts,amsthm,bbm}
\usepackage{epic,eepic,epsfig,longtable}
\usepackage{multirow,verbatim}
\usepackage{array}

\usepackage{epsfig}
% \usepackage{euler}
\usepackage{setspace}
%\usepackage{CJK}
%\usepackage{color}
% \usepackage{tikz}
% \usetikzlibrary{arrows,positioning}
\usepackage{pb-diagram}
\usepackage{fancyhdr}
\usepackage{graphicx}
\usepackage[title]{appendix}
%\usepackage{appendix}
\usepackage{listings}
\usepackage{longtable}
\usepackage{url}
\usepackage{lineno}
%\usepackage{hyperref}       % hyperlinks
%\hypersetup{
%  colorlinks = true,
%  urlcolor = blue,
%  linkcolor = blue,
%  citecolor = blue,
%}
%\usepackage{subfigure}
\usepackage{mathrsfs}
\usepackage{stmaryrd}
\usepackage{appendix}
\usepackage{algorithmic}
\usepackage{algorithm}


%\renewcommand{\algorithmicrequire}{\textbf{Input:}}
%\renewcommand{\algorithmicensure}{\textbf{Output:}}

%\providecommand{\abs}[1]{\left\lvert#1\right\rvert}
%\providecommand{\norm}[1]{\left\lVert#1\right\rVert}
\newcommand\blfootnote[1]{%
  \begingroup
  \renewcommand\thefootnote{}\footnote{#1}%
  \addtocounter{footnote}{-1}%
  \endgroup
}

%%%%%%%%%% page setup %%%%%%%%%%
\textheight 8.5 in
\textwidth 6.5 in
\topmargin -0.5 in
\oddsidemargin -0.1 in
\renewcommand{\topfraction}{1}
\renewcommand{\bottomfraction}{1}
\renewcommand{\textfraction}{0}
\renewcommand{\floatpagefraction}{0.90}
\makeatletter
\def\singlespace{\def\baselinestretch{1}\@normalsize}
\def\endsinglespace{}
\renewcommand{\topfraction}{1}
\renewcommand{\bottomfraction}{1}
\renewcommand{\textfraction}{0}
\renewcommand{\floatpagefraction}{0.90}
\makeatletter
\def\singlespace{\def\baselinestretch{1}\@normalsize}
\def\endsinglespace{}

%%%%%%%%%% numbering %%%%%%%%%%
%\renewcommand{\theequation}{\thesection.\arabic{equation}}
%\numberwithin{equation}{section}
%\renewcommand{\thefootnote}{\fnsymbol{footnote}}
%\renewcommand{\hat}{\widehat}
\def\c{\centerline}

%\renewcommand{\hat}{\widehat}

%\newcommand{\bfm}[1]{\ensuremath{\mathbf{#1}}}

\allowdisplaybreaks
%%%%%%%%%%%%%%%%%%%%%%%%%%%%%%%%%%%%%%%%%%%%%%%%%%%%%%
\setcounter{section}{0}
\def\thesection{\arabic{section}}
\setcounter{page}{1}
% \pagestyle{myheadings}
\usepackage{verbatim}
\pagestyle{plain}

%%%%%%%%%%%%%%%%%%%%%%%%%%%%%%%%%%%%%%%%%%%%%%%%%%%%%%%%%%%%%%%%%%%%%%
%%%%%%%%%%%%%%%%%%%%%%%%%%%%%%%%%%%%%%%%%%%%%%%%%%%%%%
%%%%%%%%%%%%%%%%%%%%%%%%%%%%%%%%%%%%%%%%%%%%%%%%%%%%%%%%%%%%%%%%%%%%%%
%%%%%%%%%%%%%%%%%%%%%%%%%%%%%%%%%%%%%%%%%%%%%%%%%%%%%%

\renewcommand{\baselinestretch}{1.66}
\baselineskip=22pt


%%%%%%%%%%%%%%%%%%%% more definitions %%%%%%%%%%%%%%%%%%%%

\numberwithin{equation}{section}
\theoremstyle{plain}
\newtheorem{thm}{Theorem}[section]
\newtheorem{claim}{Claim}[section]
\newtheorem{defn}{Definition}[section]
\newtheorem{lem}{Lemma}[section]
\newtheorem{cor}{Corollary}[section]
\newtheorem{prop}{Proposition}[section]
\newtheorem{ass}{Assumption}[section]
\newtheorem{fact}{Fact}[section]
\theoremstyle{definition}
\newtheorem{exm}{Example}[section]
\newtheorem{rem}{Remark}[section]


\newcounter{CondCounter}
\newenvironment{con}{
	\refstepcounter{CondCounter} % increment the environment's counter
	(C\theCondCounter)}{\par} % move to the next line


\usepackage{mystyle}
\date{\vspace{-5ex}}


%%%%%%%%%%%%%%%%%%%%%%%%% document starts here %%%%%%%%%%%%%%%%%%%%%%%%%

\makeatother

\begin{document}
% \renewcommand{\baselinestretch}{1.2}

\title{Online learning in bandits with predicted context}

\author{Yongyi Guo\thanks{Department of Statistics, University of Wisconsin-Madison, Madison, WI, 53706. Email: \texttt{guo98@wisc.edu}.}
\and Ziping Xu\thanks{Department of Statistics, Harvard University, Cambridge, MA, 02138. Email: \texttt{zipingxu@fas.harvard.edu}.}
\and Susan Murphy\thanks{Department of Statistics, Harvard University, Cambridge, MA, 02138. Email: \texttt{samurphy@g.harvard.edu}.}
}

%\date{\today}

\maketitle
\onehalfspacing

\begin{abstract}
We consider the contextual bandit problem where at each time, the agent only has access to a noisy version of the context and the error variance (or an estimator of this variance). This setting is motivated by a wide range of applications where the true context for decision-making is unobserved, and only a prediction of the context by a potentially complex machine learning algorithm is available. When the context error is non-vanishing, classical bandit algorithms fail to achieve sublinear regret. We propose the first online algorithm in this setting with sublinear regret guarantees under mild conditions. The key idea is to extend the measurement error model in classical statistics to the online decision-making setting, which is nontrivial due to the policy being dependent on the noisy context observations. We further demonstrate the benefits of the proposed approach in simulation environments based on synthetic and real digital intervention datasets.
\end{abstract}

\section{Introduction}\label{section_intro}

The rapid growth of renewable energy sources as a sustainable alternative to traditional power generation requires the development of effective energy storage solutions capable of mitigating the power grid fluctuations inherent to clean energy technologies \cite{zhao2020renewable}. In this context, vanadium redox flow batteries (VRFBs) offer several advantages that make them a promising large-scale stationary energy storage solution: high round-trip energy efficiency, excellent scalability, long cycle life, decoupled power and energy capacity, deep discharge capability, and high safety compared to other types of batteries. These advantages posit VRFBs as a promising energy storage technology for various applications, including grid-level energy storage, renewable energy integration, load balancing, and backup power systems. However, VRFBs also face challenges such as lower energy and power density, high cost, complex system design, electrolyte degradation, and environmental concerns \cite{cunha2015vanadium, yuan2019review, lourenssen2019vanadium}. 

In particular, one of the main drawbacks of VRFBs is their capacity decay during cycling, mainly caused by ion crossover \cite{yang2015effects}, hydrogen evolution \cite{wei2017insitu}, or imperfect electrolyte mixing in the tanks \cite{prieto2023fluid}. This makes the total vanadium concentration and the state of charge (SOC) to drift away from symmetry between the negative and positive sides, creating a charge imbalance. To address and correct these issues effectively, real-time monitoring of the total vanadium concentration and SOC of the positive and negative half-cells would be highly desirable. Upon charge and discharge, the absorbance spectra of vanadium compounds vary significantly. Thus, a direct, non-invasive method to measure the SOC is through Ultraviolet-Visible (UV-Vis) absorbance spectroscopy. 

The method comes as a valuable alternative to the classical approach of open circuit voltage (OCV) measurements, typically conducted offline in a smaller electrochemical cell separate from the main redox reactor. These OCV sensors are considered invasive due to the potential for crossover and self-discharge through the membrane of the smaller cell. Moreover, they are susceptible to temperature variations, and their measurement accuracy heavily depends on the precise design and operation of the cell to mitigate issues such as mass transfer limitations, preferential paths, or shunt currents, among others. Moreover, OCV measurements solely provide information about the overall SOC of the VRFB system, lacking the ability to determine the individual SOC of each electrolyte (positive and negative). Additionally, they do not provide insights into the total Vanadium concentration in each stream~\cite{knehr2011open,munoz2023exploring}.

More recent approaches, such as the online monitoring of density or viscosity to establish their correlation with the state of charge (SOC), still have limitations, including temperature dependency and the introduction of an additional pressure drop. These factors contribute to a decrease in the overall round-trip efficiency of VRFBs~\cite{li2018investigation, ressel2018state}. In the case of membraneless redox flow batteries, which have gained significant research attention, an optical non-invasive measurement becomes crucial due to the limitations associated with the aforementioned methods. Furthermore, in membraneless systems, the simultaneous measurement of SOC and the total vanadium concentration in both electrolytes is required due to the absence of a membrane.
Specifically, for micro membraneless VRFBs, it is imperative to minimize the dead volume in the microsensor, a challenge effectively addressed through non-invasive optical methods~\cite{navalpotro2023neutral,lee2013microfluidic}. Overall, optical sensors enable monitoring VRFB performance and provide more profound insights into the specific processes that control efficiency and capacity decay during operation, offering optimal real-time readings and enabling closed-loop control strategies~\cite{di2022general,monbaliu2023will}.

In the literature, the UV-Vis spectroscopy method has been extensively studied~\cite{geiser_photometrical_2019-1, roznyatovskaya_detection_2016, brooker_determining_2015, choi_analysis_2013}, uncovering early the complex chemistry of the positive electrolyte with Blanc et al.~\cite{blanc_spectrophotometric_1982} and more recently with Buckley et al.~\cite{buckley_towards_2014, quill_factors_2015, gao_spectroscopic_2013, petchsingh_spectroscopic_2016}. Experimental tests have also been conducted using the UV-Vis method to determine the state of charge (SOC) both offline~\cite{liu_state_2012, petchsingh_spectroscopic_2016} and online/in-operando~\cite{tang_monitoring_2012, zhang_-line_2015, shin_real-time_2020}. However, upon browsing the extensive literature on the subject, a critical question still emerges: How can the total vanadium concentration and the SOC of both electrolytes in a VRFB be independently measured using UV-Vis spectroscopy, and what level of accuracy can be achieved with this method? While partial answers can be found in scattered publications, the available information often focuses solely on one specific electrolyte and lacks detailed analysis. In addition, certain reported calibrations only provide the SOC as an output, disregarding the total vanadium concentration. Moreover, limited information is available regarding measurement errors, making it difficult to evaluate the robustness of the reported calibration methods. Furthermore, existing studies often present calibrations for a narrow range of concentrations.

To effectively address these knowledge gaps, the primary objective of this article is to present a comprehensive document that outlines calibration methods capable of accurately quantifying both the total vanadium concentration and the SOC across all possible vanadium electrolyte mixtures, encompassing a wide concentration range of 0.91 to 1.83~M. Moreover, we provide significant calibration improvements compared with the existing literature, resulting in better accessibility for the reader and improved measurement accuracy.

This study deals with the three types of vanadium electrolyte mixtures employed in VRFBs: \textit{i)} the anolyte $\rm V^{2+}/V^{3+}$ ($\VII/\VIII$), \textit{ii)} the $\rm V^{3+}/VO_2^{+}$ ($\VIII/\VIV$) mixture and \textit{iii)} the catholyte $\rm VO^{2+}/VO_2^{+}$ ($\VIV/\VV$). Following standard practice, the four oxidation states of vanadium are denoted here $\VII$, $\VIII$, $\VIV$, and $\VV$. During normal VRFB operation, the $\VIII/\VIV$ electrolyte should not be present and is typically less relevant. Nonetheless, a failing VRFB with high electrolyte imbalance could overdischarge and result in $\VIII/\VIV$ appearing in one of the tanks. Moreover, commercial vanadium electrolyte is usually distributed as an equimolar $\VIII/\VIV$ mixture, often referred to as $\rm V{3.5+}$, meant to undergo a preconditioning process to achieve zero SOC either as anolyte or catholyte. Other vanadium electrolyte mixtures cannot exist since self-discharge reactions would occur yielding one of the three composition pairs indicated above~\cite{Tang_thermal_2012}.

For each electrolyte we provide two calibration methods: \textit{i)} a fast empirical method based on linear regression, which requires only one or two hand-picked absorbances as input, and \textit{ii)} a more computationally intensive method based on spectral deconvolution~\cite{loktionov_operando_2022}, which uses the whole spectrum as input. The methods to be outlined below are based on the generalized Beer-Lambert law. This law states that the total absorbance of a mixture containing multiple absorbing compounds is equal to the sum of the individual absorbances contributed by each compound. This principle assumes that the absorbing compounds in the mixture do not interact with each other and that their absorbance contributions are independent~\cite{atkins2014physical, harris2015quantitative, skoog2013fundamentals}. 

The calibration methods yield the total vanadium concentration $C$ and the mole fractions $X_i$ for the three electrolyte mixtures under study. In the case of the $\VII/\VIII$ and $\VIV/\VV$ mixtures, anolyte and catholyte, the calibration also provides the SOC, represented by the mole fractions $\XII$ and $\XV$, which is the desired piece of information for most readers. Table \ref{tab:notation} defines the notation used in this paper, relating the total vanadium concentration $C$, the mole fractions $X_i$, and the SOC to the molar concentrations $C_i$ of the four vanadium ions, $i = \{2, 3, 4, 5\}$. Note that in the $\VIV/\VV$ mixture we include the presence of the 1:1 stoichiometric mixed-valence complex $\rm V_2O_3^{3+}$ in equilibrium with $\VIV$ and $\VV$ according to reaction \eqref{eq:V2O33p_formation}, whose concentration is denoted by subscript $i = 45$ \cite{blanc_spectrophotometric_1982}. Hereafter, the SOC and mole fractions are expressed as percentages unless otherwise stated. 

The data set and the Python calibration algorithms generated in this work are publicly shared in an online repository (\href{https://github.com/AngeAM/SOC_Vanadium_Spectra_2023.git}{GitHub}). We believe this decision will be helpful for both the scientific community and the industry. In general, adopting an open data approach proves most valuable in advancing the field, as it allows anyone to enhance the existing calibration procedures by employing more advanced chemometric techniques.

\begin{table}[b!]
    \centering
    \begin{tabular}{cccc}
        & \multicolumn{3}{c}{Electrolyte mixture} \\[1mm] \cline{2-4} \\[-3.5mm]
        Variable & $\VII/\VIII$ & $\VIII/\VIV$ & $\VIV/\VV$\\[1mm] 
        \hline \\[-3.5mm]
        $C$     & $\CII+\CIII$ & $\CIII+\CIV$ & $\CIV+\CV+2C_{45}$ \\[1mm] 
        \hline \\[-3.5mm]
        $\XII$  & $\dfrac{\CII}{\CII+\CIII}$ & -- & -- \\[4mm]
        $\XIII$ & $\dfrac{\CIII}{\CII+\CIII}$ & $\dfrac{\CIII}{\CIII+\CIV}$ & -- \\[4mm]
        $\XIV$  & -- & $\dfrac{\CIV}{\CIII+\CIV}$ & $\dfrac{\CIV+C_{45}}{\CIV+\CV+2C_{45}}$ \\[4mm]
        $\XV$   & -- & -- & $\dfrac{\CV+C_{45}}{\CIV+\CV+2C_{45}}$ \\[4mm] 
        \hline \\[-3.5mm]
        SOC     & $X_2$ & -- & $X_5 = 1-X_4$
    \end{tabular}
    \caption{Definition of the total vanadium concentration $C$, molar fractions $X_i$, and SOC for the different electrolyte mixtures under study as a function of the molar concentration of the four vanadium ions $C_i$, $i = \{2, 3, 4, 5\}$. In the $\VIV/\VV$ mixture $C_{45}$ denotes the concentration of the 1:1 stoichiometric mixed-valence complex $\rm V_2O_3^{3+}$ at equilibrium with $\VIV$ and $\VV$ according to reaction \eqref{eq:V2O33p_formation}.}
    \label{tab:notation}
\end{table}

The paper is organized as follows. Section \ref{sec:experimental} describes the experimental procedures employed to prepare the calibration samples with the best accuracy. Section \ref{sec:results_and_calibration} presents the calibration methods for the different mixtures, starting with the anolyte $\VII/\VIII$ and the $\VIII/\VIV$ mixture, which exhibit a common linear response. Subsequently, the calibration of the catholyte $\VIV/\VV$ is discussed, showcasing its non-linear spectral response. In Section \ref{sec:discussion}, we discuss the accuracy, advantages and disadvantages of the different methods. Finally, in Section \ref{sec:conclusions} we summarize the key findings and draws the conclusions.

\section{Experimental}\label{sec:experimental}

\subsection{Electrochemical preparation of the reference electrolytes} \label{sec:prep_chem}

The calibration of UV-Vis spectroscopic methods requires the preparation of single-species reference electrolytes, and of mixtures thereof, with a high degree of compositional precision. In this section we describe the experimental procedures employed to generate these samples. We prepared the reference solutions of vanadium electrolytes via electrochemical reduction and oxidation in an in-house 3 $\rm cm^2$ electrochemical cell. To avoid hydrogen evolution reactions and significant crossover effects, the applied cell voltage was kept at 1.6~V. Throughout the charge process, we acquired UV-Vis absorbance spectra periodically using flow cuvettes. As the starting electrolyte, we used an equimolar $\VIII/\VIV$ commercial solution (Oxkem Limited, UK) containing an average oxidation number (AOS) of $+3.5$. Its exact composition is listed in Table \ref{tab:fresh_composition}. Since all samples originated from this batch solution, we performed induction coupled plasma-optical emission spectroscopy (ICP-OES) to measure the total vanadium concentration with high accuracy, with the results given in the table. It is interesting to note that the solution contains 0.5~M of phosphoric acid, which is commonly used in commercial vanadium electrolytes to improve their solubility properties~\cite{oldenburg_revealing_2018}.
\begin{table}[ht!]
    \centering
    \begin{tabular}{cc}
         Species & Concentration \\[1mm]
         \hline \\[-3.5mm]
        Vanadium & $1.83 \pm 0.092$~M \\
        Sulfuric Acid & 4.6~M \\
        Phosphoric Acid & 0.5~M \\[1mm]
         \hline \\[-3.5mm]
        Oxidation number & $\approx 3.5$\\
    \end{tabular}
    \caption{Composition of the commercial vanadium electrolyte (Oxkem Limited, UK) used in the study, with an average oxidation number (AOS) of $+3.5$.}
    \label{tab:fresh_composition}
\end{table}

Slight variations of the total vanadium concentration can significantly affect the absorbance spectra measurements. This can occur, for instance, due to ion crossover across the membrane. To minimize such phenomenon, we stacked 3 layers of Nafion\textsuperscript{\tiny\textregistered}~N112 membrane following the results published by Ashraf et al.~\cite{ashraf_gandomi_direct_2020}. With this membrane stacking and taking into account our charge duration of 5-15 hours, the total vanadium concentration should not vary more than 5 mM, which represents a variation of about 0.27\%. Note that the calibrations and data shown in this paper are provided for a given acidity level. Minimal variations of spectrum shape could appear when varying the acidity~\cite{loktionov_operando_2022}. However, given the small magnitude of these variations, we expect negligible impacts on the accuracy of the calibration. Additionally, the methodology presented in this paper remains valid regardless of the acidity and could be used to update the calibration for various acidic conditions, a potential research work that is outside the scope of this paper.

\subsection{UV-Vis spectroscopy}\label{sec:UV_vis}

We acquired the spectra using 0.1~mm and 1~mm path length commercial flow cuvettes (Starna Scientific Ltd., UK). Because of its lower absorption, we measured the $\VII$/$\VIII$ electrolyte using the 1~mm path length cuvette whilst for the other mixtures we used the 0.1~mm path length cuvette. The spectrometer (Flame T-VIS-NIR-ES, Ocean Insight, USA), the lamp (HL-2000-FHSA-LL, Ocean Insight, USA), and the cuvette holder (SQUARE ONE, Ocean Insight, USA) were connected via optical fibers. The Flame T-VIS-NIR-ES spectrometer has a wavelength detection range of 345-1033 nm and an optical resolution of 1.33 nm FWHM. However, considering the high absorbance of the electrolyte and the lamp emission range, the actual measurable range is about 420-1000nm depending on the concentration and the species. Upon measurement, we injected at least 1 mL of sample with a syringe in the flow cuvette in order to ensure ample clearing of the cuvette optical window. In this work, all absorbances are normalized with the optical path length, and thus their units are in $\rm cm^{-1}$.

\subsubsection{Preparation of $\VII$ and $\VV$}

The preparation of the charged $\VII$ and $\VV$ electrolyte solutions was achieved by starting from the equimolar 
$\rm V{3.5+}$ solution and bringing the electrochemical oxidation and reduction of the electrolyte to completion. In other words, fully oxidizing or reducing the vanadium applying a fixed 1.6~V potential to the cell. The completion point was deemed to be reached when three independent criteria were met: \textit{i)} the current density was lower than $2~\rm mA/cm^2$, \textit{ii)} the UV-Vis spectrum remained stable for more than 10 min, and \textit{iii)} the shape of the spectrum was coherent with previous literature results. It is worth noting that we had to increase the applied potential up to 1.7-1.8~V for a short period of time (10-15~min) to achieve a stable UV-Vis spectrum.

\subsubsection{Preparation of $\VIII$ and $\VIV$}

Similarly, we prepared the discharged $\VIII$ and $\VIV$ electrolytes by electrochemical charging the equimolar $\rm V{3.5+}$ solution. However, reaching the exact transition state, or endpoint, where only pure $\VIII$ or $\VIV$ exist is more complex than for the fully charged electrolytes $\VII$ and $\VV$. In this case, the use of online spectrometry is critical. The procedure, called spectrophotometric titration \cite{skoog2013fundamentals}, consists in lowering the applied voltage or current when approaching the purity point and monitoring the spectra at specific wavelengths looking for an inflection point. Further elaboration on the procedure and specific criteria for its application to vanadium electrolytes can be found in the Supplementary Information (Figure~S1). Obtaining $\VIII$ is relatively easier through this method, as an overcharge leads to the formation of $\VII$, which subsequently undergoes air oxidation. Indeed, when the negative electrolyte is exposed to air, it naturally tends to become pure $\VIII$. Nevertheless, in order to prevent any potential impact on the total vanadium concentration due to water evaporation, efforts were made to minimize exposure to air.

\subsubsection{Calibration samples}

We prepared the calibration samples via pipetting and mixing/diluting of the reference solutions and a 4.6M sulfuric acid solution. The total sample volumes were 1-2~mL, depending on the desired vanadium concentration. Using deionized water and a 5-digit analytical weighing scale, we estimated the pipetting accuracy to be 0.4\%. For each mixture, we prepared 11 samples at SOCs ranging from 0 to 100\% and concentrations of 1.83, 1.52, 1.22, and 0.91~M (44 samples per mixture, 132 samples in total). We disregarded concentrations lower than 0.9~M for the calibration step since most industrial applications use electrolytes with concentrations higher than 1~M to maintain high capacity/volume ratios. Nonetheless, to verify the Beer-Lambert law, we additionally prepared samples with 0.6 and 0.3~M containing only pure compounds (see Section~\ref{sec:ref_spectra}).

\section{Results and calibration}
\label{sec:results_and_calibration}

\subsection{Reference spectra}
\label{sec:ref_spectra}
Figure \ref{fig:ref_spectra}a shows the reference absorbance spectra $A_i$ of the single-species electrolyte solutions, obtained following the procedures outlined in section~\ref{sec:experimental}. The spectra $A_2$, $A_3$, $A_4$, and $A_5$ correspond, respectively, to the oxidation states $\VII$, $\VIII$, $\VIV$, and $\VV$. The shapes of the spectra are consistent with the visual appearance of the electrolytes depicted in Figure \ref{fig:ref_spectra}b, as well as the color band shown in Figure \ref{fig:ref_spectra}c. For instance, $A_5$ acts as a high-pass filter, suppressing the blue part of the visible spectrum and having negligible absorbance above 650 nm, which yields an orange/yellow color. By contrast, $A_4$ has very low absorption between 420-550 nm, resulting in a deep blue appearance. Figures \ref{fig:ref_spectra}d, \ref{fig:ref_spectra}e, \ref{fig:ref_spectra}f, and \ref{fig:ref_spectra}g show the measured absorbance as a function of the total vanadium concentration for the four vanadium oxidation states. The plots correspond to selected wavelengths, approximately aligned with the absorbance peaks depicted in Figure \ref{fig:ref_spectra}b for $\VII$, $\VIII$, and $\VIV$, and at 440 nm for $\VV$, which lacks a well defined peak in the observed range. The linear fits (red dotted lines) obtained for $\VII$, $\VIII$, and $\VIV$ confirm the validity of the Beer-Lambert law 
\begin{equation}
\label{eq:beer_lambert}
    A_i = \epsilon_i C_i \quad \text{for} \quad i=\{2, 3, 4\}
\end{equation}
at least up to total vanadium concentrations of 1.83~M. In this expression $A_i$ is the path-length normalized absorbance spectrum in $\rm cm^{-1}$, $C_i$ is the molar concentration in M, and $\epsilon_i$ represents the molar absorptivity (or molar absorption coefficient) of the $i$-th oxidation state, in $\rm cm^{-1}~M^{-1}$.

By contrast, the relation between absorbance and concentration for $\VV$ is non-linear, deviating from the classical Beer-Lambert law. This non-linear behavior can be effectively described by a power law equation of the form
\begin{equation}
\label{eq:beer_lambert_power}
A_5 = \epsilon_5 C_5^k
\end{equation}
where the fitting exponent remains approximately constant for different wavelengths with a value of roughly $k \approx 2$, and $\epsilon_5$ represents the molar absorptivity of $\VV$. 
The non-linear absorbance of $\VV$ had been only briefly reported for $\VV$ at ultra-low concentrations ($<0.03$~M) by Bannard et al.~\cite{bannard_spectrophotometric_1968}. Consequently, the deviation observed here at high concentrations is a novel result that, to the best of our knowledge, had not been previously reported in the literature. Non-linear deviations from the Beer-Lambert law at high concentrations may be due to different factors, such as intermolecular interactions or changes in the refractive index of the analyte. These deviations may result in complex spectral behaviors that cannot be accurately described by a simple linear relationship between absorbance and concentration \cite{mayerhofer2020bouguer, nolte2021IR, shin_real-time_2020}.

The values of $\epsilon_i$ and $k$ for the chosen wavelengths are annotated in the graphs of Figure \ref{fig:ref_spectra} along with their corresponding statistical uncertainties given by the root mean square error (RMSE, or $E$ in shorthand notation) between the predicted and the measured values. 
Table \ref{tab:val_ref_epsilon} lists values of the molar absorptivities $\epsilon_i$ for $i=\{2,3,4\}$ at selected wavelengths. As previously discussed, these wavelengths correspond to the local absorbance peaks of $\VII$, $\VIII$, and $\VIV$, with the corresponding maxima of $\epsilon_i$ being highlighted in boldface in each case.

% Figure environment removed

\begin{table}[ht!]
    \centering
    \begin{tabular}{cccc}
        & \multicolumn{3}{c}{Electrolyte} \\[3mm]
        & $\VII$ & $\VIII$ & $\VIV$ \\[1mm] 
        \cline{2-4} \\[-3.5mm]
        Wavelength&$\epsilon_2$ & $\epsilon_3$ & $\epsilon_4$ \\[1mm]
        \hline \\[-3.5mm]
        564 nm & $\mathbf{4.36} \pm 0.026$ & $5.46 \pm 0.079$          & $2.91 \pm 0.043$ \\
        605 nm & $3.21 \pm 0.020$          & $\textbf{7.40}\pm 0.053$ & $7.18 \pm 0.061$ \\
        766 nm & $2.11 \pm 0.014$          & negligible          & $\textbf{19.72} \pm 0.095$ \\
        850 nm & $\textbf{3.18} \pm  0.032$ & negligible          & $10.82 \pm 0.19$
    \end{tabular}
    \caption{Values of the 1.83~M molar absorptivities $\epsilon_i$ for $i=\{2,3,4\}$ at selected wavelengths determined by the local maxima of the absorbance curves of $\VII$, $\VIII$, and $\VIV$. All units are in $\rm cm^{-1} M^{-1}$.}
    \label{tab:val_ref_epsilon}
\end{table}

The following sections provide detailed information about the spectral responses of the mixtures $\VII/\VIII$, $\VIII/\VIV$, and $\VIV/\VV$. While the $\VII/\VIII$ and $\VIII/\VIV$ mixtures show a linear response, the catholyte $\VIV/\VV$ exhibits a non-linear response. This non-linear behavior is attributed both to the non-linear absorbance of $\VV$ and to the formation of a high absorbance 1:1 stoichiometric complex between $\VIV$ and $\VV$~\cite{blanc_spectrophotometric_1982}. Consequently, the calibration methods for the three electrolyte mixtures are presented in two distinct sections addressing the linear and non-linear cases separately.

\subsection{Linear case: $\VII/\VIII$ and $\VIII/\VIV$ }

% Figure environment removed

Figures \ref{fig:spectra_soc}a and \ref{fig:spectra_soc}b show, respectively, the molar absorptivity spectra of the $\VII/\VIII$ and $\VIII/\VIV$ mixtures at varying SOCs, or mole fractions, for a total vanadium concentration $C = 1.83$~M. The former electrolyte is characterized by the mole fraction of $\VII$, or $X_2$, which coincides with the SOC, while the later is characterized by the mole fraction of $\VIV$, or $X_4$. The vertical arrows indicate the location of the isosbestic points in which the absorbance remains constant regardless of the mole fraction. In UV-Vis spectroscopy, an isosbestic point refers to a specific wavelength at which the absorbance of the sample remains constant during a chemical reaction or a change in the sample's composition. 
The presence of such points suggests that the reaction occurring in the sample proceeds via a direct conversion between species, without any intermediates, e.g., reaction $\VII \rightleftharpoons \VIII$. 

When dealing with mixtures, the Beer-Lambert law \eqref{eq:beer_lambert} can be extended to account for multiple absorbing species. Therefore, for a binary mixture of two pure compounds A and B, with mole fractions $X_{\rm A}$ and $X_{\rm B} = 1- X_{\rm A}$, the absorption spectrum is the sum of the individual absorbances of the two components
\begin{equation}\label{eq:lin_comb}
    A_{\rm M} = [\epsilon_{\rm A}X_{\rm A} + \epsilon_{\rm B}(1-X_{\rm A})]C
\end{equation}
where $A_{\rm M}$ is the absorbance of the mixture, $\epsilon_{\rm A}$ and $\epsilon_{\rm B}$ are the molar absorptivities of compounds A and B, and $C$ is the total (vanadium) concentration.
At the isosbestic point $\epsilon_{\rm A} = \epsilon_{\rm B} \equiv \epsilon^{\rm iso}$ and Equation~\eqref{eq:lin_comb} reduces to
\begin{equation}\label{eq:iso}
    A_{\rm M}^{\rm iso} = \epsilon^{\rm iso} C
\end{equation}
In other words, at the isosbestic point the absorbance is independent of the mole fraction $X_{\rm A}$ and grows linearly with the total concentration $C$.
Table \ref{tab:val_isosbestic} lists the isosbestic wavelengths, $\lambda^{\rm iso}$, and the corresponding absorbance values, $\epsilon^{\rm iso}$, for the two optically linear vanadium mixtures. As illustrated below, the presence of isosbestic points enables the estimation of the total concentration $C$ independently of the mole fraction $X_{\rm A}$ by directly using the values of $A_{\rm M}^{\rm iso}$ and $\epsilon^{\rm iso}$ in Equation~\eqref{eq:iso}.

The negative electrolyte ($\VII$/$\VIII$) appears to have three isosbestic points. However, only two of them are listed in Table \ref{tab:val_isosbestic} because the third one, located around 500~nm, is too out of focus to have practical application. Equation~\eqref{eq:lin_comb} may lose validity at lower wavelengths ($\lambda<550$) due to uncharted interactions among the vanadium compounds. This hypothesis is consistent with the findings of Loktionov et al.~\cite{loktionov_operando_2022}, whose results suggest the presence of additional interactions at lower wavelengths within the negative electrolyte.

\begin{table}
    \centering
    \begin{tabular}{ccc}
        Electrolyte mixture & $\lambda^{\rm iso}~(\rm nm)$ & $\epsilon^{\rm iso}~(\rm cm^{-1} M^{-1})$ \\[1mm]
      \hline \\[-3.5mm]
      $\VII$/$\VIII$ & $722.11 \pm 0.69 $ & $1.24 \pm 0.001$ \\[1mm]
       & $542.73 \pm 7.03 $ & $4.06 \pm 0.023$ \\[1mm]
      \hline \\[-3.5mm]
      $\VIII$/$\VIV$ & $608.22 \pm 2.33 $ & $7.41 \pm 0.05$ \\ 
    \end{tabular}
    \caption{Wavelengths of the isosbestic points and their corresponding absorbances for the the two optically linear vanadium mixtures.}
    \label{tab:val_isosbestic}
\end{table}

%It is important to note that the presence of sulfuric acid and phosphoric acid in the solutions considered here does not invalidate the assumption of a binary sample underlying Equation \eqref{eq:lin_comb}. Just like water, the main component of the vanadium electrolyte solutions, the absorbances of these acids are several orders of magnitude less than those of the vanadium compounds under study. As a result, their impact on the spectra can be safely disregarded, and the solutions can be effectively treated as binary mixtures for all optical purposes.

\subsubsection{Fast empirical method via linear regression}
\label{sec:empirical234}

According to Equation~\eqref{eq:lin_comb}, by plotting the mole fraction $X_{\rm A}$ versus the absorbance $A_{\rm M}$ at a specific wavelength and performing a linear regression analysis we could obtain a linear relationship that provided $X_{\rm A}$ in terms of $A_{\rm M}$. However, it is important to note that despite the simplicity of this method, it may produce inaccurate results. This is primarily because the absorbance varies with the concentration $C$ and the path length, which may be unknown or altered due to experimental conditions. For example, in the case of VRFBs, species crossover may have a significant impact on the overall concentration of vanadium ions in both the positive and negative electrolytes, thereby affecting the accuracy of the results.

We can address this problem by dividing the total mixture absorbance $A_{\rm M}$ by the absorbance at the isosbestic point $A_{\rm M}^{\rm iso}$, effectively merging Equations~\eqref{eq:lin_comb} and~\eqref{eq:iso} into a single equation
\begin{equation}
\frac{A_{\rm M}}{A_{\rm M}^{\rm iso}} = \frac{(\epsilon_{\rm A} - \epsilon_{\rm B})X_{\rm A} + \epsilon_{\rm B}} {\epsilon^{\rm iso}}
\end{equation}
This equation can be solved explicitly for $X_{\rm A}$ to yield a linear relation of the form
\begin{equation}
X_{\rm A} = a\frac{A_{\rm M}}{A_{\rm M}^{\rm iso}} + b 
\label{eq:SOClin_a_b}
\end{equation}
with $a = \epsilon^{\rm iso}/(\epsilon_{\rm A} - \epsilon_{\rm B})$ and $b = - \epsilon_{\rm B}/(\epsilon_{\rm A}-\epsilon_{\rm B})$. According to this expression, $X_{\rm A}$ should vary linearly with $A_{\rm M}/A_{\rm M}^{\rm iso}$ and be independent of the total concentration $C$. Moreover, the total concentration $C$ could be estimated from Equation~\eqref{eq:iso} as follows
\begin{equation}
\label{eq:totalC_lin}
    C = \frac{A_{\rm M}^{\rm iso}}{\epsilon^{\rm iso}}
\end{equation}
Hence $C$ should grow linearly with the absorbance regardless of the mole fraction $X_{\rm A}$, provided that the absorbance is measured at the isosbestic point.

% Figure environment removed

Figures \ref{fig:lin_reg}a and \ref{fig:lin_reg}b show respectively the mole fractions $X_2$ and $X_4$ of the 44 calibration samples of the $\VII/\VIII$ and $\VIII/\VIV$ mixtures plotted as a function of $A_{\rm M}^\lambda/A_{\rm M}^{\rm iso}$ at a particular wavelength for each mixture, along with the corresponding linear fits \eqref{eq:SOClin_a_b} as detailed in Table~\ref{tab:val_lin_reg}, using the first isosbestic wavelength reported in Table~\ref{tab:val_isosbestic} for each mixture. Similarly, Figures \ref{fig:lin_reg}c and \ref{fig:lin_reg}d show the total concentration of vanadium ions as a function of the isosbestic absorbances $A_{\rm M}^{\rm iso}$ along with the corresponding linear fits \eqref{eq:totalC_lin} also detailed in Table~\ref{tab:val_lin_reg}. The experimental points are shown as black circles, while the linear fits are plotted as red dotted lines. In these plots, all 44 calibration samples are included, meaning that the fits are performed within different vanadium concentrations. The excellent linearity observed in all cases provides strong evidence for the validity and consistency of Equations~\eqref{eq:SOClin_a_b} and \eqref{eq:totalC_lin} across a wide range of total vanadium concentrations and mole fractions. Table~\ref{tab:val_lin_reg} summarizes the correlations thus obtained with their corresponding coefficients calculated from the linear fits, as well as the estimated RMSE, commonly used for reporting the error of calibration methods. 

Thus, when using the linear fits from the current calibration reported in Table~\ref{tab:val_lin_reg} to determine the mole fractions, the SOC, or the total vanadium concentration of a new set of experimental data, the reported values should include, as a minimum, the uncertainty in the measurement by employing the usual notation, e.g., $X = X_{\rm exp} \pm \RMSE_X$.

\begin{table}[ht]
    \centering
\begin{tabular}{ccccc}
    & \multicolumn{4}{c}{Electrolyte mixture} \\[1mm] \cline{2-5} \\[-3.5mm]
    & \multicolumn{2}{c}{$\VII$/$\VIII$} & \multicolumn{2}{c}{$\VIII$/$\VIV$} \\[1mm] \hline \\[-3.5mm]
    Variable & Linear fit & $\RMSE_X$ & Linear fit & $\RMSE_C$ \\[1mm] \hline \\[-3.5mm]
X (\%) & $X_2 = 40.51 \dfrac{A^{850}}{A^{723}}$ & $\pm 1.46$ & $ X_4 = 38.26 \dfrac{A^{760}}{A^{608}} - 1.91$ & $\pm 0.52$  \\
  &  &  &  &  \\
C (M) & $C = \dfrac{1}{1.34} A^{723}$ & $\pm 0.03$ & $C = \dfrac{1}{7.71} A^{608}$  & $\pm 0.015$  \\
 \end{tabular}
\caption{Linear fits for the mole fractions of $\VII$ and $\VIV$ versus $A_{\rm M}/A_{\rm M}^{\rm iso}$ and total vanadium concentration versus $A_{\rm M}^{\rm iso}$ in the $\VII/\VIII$ and $\VIII/\VIV$ mixtures with indication of the corresponding RMSE errors.}
    \label{tab:val_lin_reg}
\end{table}

\subsubsection{Spectral deconvolution}\label{sec:deconv_lin}

The spectral deconvolution method in UV-Visible spectroscopy is a technique used to separate overlapping absorption spectra of multiple components in a mixture. It involves mathematically extracting the individual contributions of each component from the observed composite spectrum, enabling the quantification of their respective concentrations or absorbances. Its application to a mixture of two pure compounds, A and B, starts by expressing the individual absorbances in terms of the known molar absorptivities, $\epsilon_{\rm A}$ and $\epsilon_{\rm B}$, and the unknown molar fractions, $X_{\rm A}$ and $X_{\rm B} = 1 - X_{\rm A}$, and total (vanadium) concentration, $C$. The calibration then proceeds by fitting the best pair ($X_{\rm A}$, $C$) that minimizes the sum of the error squares between the experimentally measured mixture absorbance $A_{\rm exp}$ and the mixture absorbance predicted by the Beer-Lambert law~\eqref{eq:lin_comb}
\begin{equation}
\label{eq:deconv_linear}
\sum_{\lambda} \left\{A_{\rm exp} - \left[\epsilon_{\rm A}X_{\rm A} + \epsilon_{\rm B}(1-X_{\rm A})\right]C\right\}^2
\end{equation}
As the problem involves two unknowns, a minimum of two wavelengths is required for minimization. However, in order to obtain the highest accuracy, the fit was performed over a wide spectral bandwidth (420-1000~nm) using $\epsilon_{\rm A}$ and $\epsilon_{\rm B}$ as reference spectra. As discussed above, the calculation range was slightly narrower than the detection range of the spectrometer (345-1033 nm). The extreme wavelengths were excluded because the signal-to-noise ratio was below the detection limit of the instrument.

Loktionov et al.~\cite{loktionov_operando_2022} recently implemented this method for determining the SOC and the total concentration of vanadium ions, yielding solid results for vanadium concentrations of 0.5~M and 1~M. In this study, we extend the investigation to a broader concentration range and also incorporate the characterization of the $\VIII$/$\VIV$ mixture. Additionally, we report the measurement error when applying the method to calibrated samples. Figures~\ref{fig:lin_deconv}a and \ref{fig:lin_deconv}b illustrate the application of spectral deconvolution to the $\VII/\VIII$ and $\VIII/\VIV$ mixtures with two representative examples. The experimental absorbance spectrum for $C=1.83$~M and $X_{\rm A}=50$\%, with A denoting $\VII$ and $\VIV$, respectively, is shown in black. The dashed lines represent the spectral contributions of each pure compound: $\epsilon_{\rm A} X_{\rm A} C$ and $\epsilon_{\rm B} (1 - X_{\rm A})C$. According to Equation~\eqref{eq:deconv_linear}, the sum of the two dashed curves yields the red curve, which is the fit. Overall, the fit is accurate for all samples although a slight mismatch can sometimes be found at the lower wavelengths for the negative electrolyte, $\lambda < 600$~nm in Figure~\ref{fig:lin_deconv}a. This is consistent with the suggestion stated above that the spectral response of the mixture could lose linearity at lower wavelengths~\cite{loktionov_operando_2022}. 

For the established molar fraction and total concentration of each of the calibration samples, referred to as true values ($X_{\rm A,T}$, $C_{\rm T}$), spectral deconvolution was applied to obtain the so-called predicted values ($X_{\rm A,P}$, $C_{\rm P}$). Figures~\ref{fig:lin_deconv}c, \ref{fig:lin_deconv}d, \ref{fig:lin_deconv}e, and \ref{fig:lin_deconv}f show the calibration plots for $\VII$/$\VIII$ (c, d) and $\VIII$/$\VIV$ (e, f) for the total vanadium concentration $C$ and the mole fraction $X_A$. Essentially, these are scatter plots showing the predicted values versus the true values. Black dots represent the obtained results for all concentrations. In an ideal situation with perfect calibration and measurements with 100\% accuracy, the predicted values would align with the true values along the bisector of the first quadrant, represented by the blue line. In practical scenarios, deviations from the blue line indicate reduced calibration accuracy and increased measurement errors. To quantify accuracy, the RMSE was computed for all concentrations, as shown in Table \ref{tab:RMSE_deconv_lin}. The RMSE values were consistent over the entire concentration range, indicating uniform accuracy. Specifically, the average measurement errors for the mole fraction $X_{\rm A}$ and the concentration $C$ were found to be 0.79\% and 17 mM, respectively.

% Figure environment removed

\begin{table}[h!]
    \centering
\begin{tabular}{ccccc}
  & \multicolumn{4}{c}{Electrolyte mixture} \\[1mm] \cline{2-5} \\[-3.5mm]
  & \multicolumn{2}{c}{$\VII$/$\VIII$} & \multicolumn{2}{c}{$\VIII$/$\VIV$} \\[1mm] \hline \\[-3.5mm]
$C$ (M) & $\RMSE_X$ (\%) & $\RMSE_C$ (M) & $\RMSE_X$ (\%) & $\RMSE_C$ (M) \\[1mm] \hline \\[-3.5mm]
1.83 & 1.11  & 0.038 & 0.65 &  0.009 \\
1.52 & 0.66  & 0.020 & 1.09 & 0.012 \\
1.22 & 0.75  & 0.019 & 0.52 &  0.015 \\
0.91 & 0.86  & 0.012 & 0.74 &  0.007 \\[1mm] \hline \\[-3.5mm]
 Average &\textbf{0.85}  & \textbf{0.022} & \textbf{0.74} & \textbf{0.012}
 \end{tabular}
\caption{Calibration results for the spectral deconvolution method showing the RMSE for the mole fraction in \% and the total concentration in~M for total vanadium concentrations between 0.91~M and 1.83~M.}
    \label{tab:RMSE_deconv_lin}
\end{table}

\subsection{Non-linear case: $\VIV/\VV$}

% Figure environment removed

Figure \ref{fig:spectra_pos} shows the absorbance spectra of the positive electrolyte $\VIV/\VV$ as the SOC increases from 0\% ($\VIV$) to 100\% ($\VV$) for a total concentration $C = 1.83$~M. As indicated by the red arrow, the overall absorbance (550-1000~nm) first increases up to ${\rm SOC} \approx 40\%$ (Figure \ref{fig:spectra_pos}a). Subsequently, the absorbance declines, first gradually within the 50-70\% SOC range and then sharply as the SOC approaches 80-100\% (Figure \ref{fig:spectra_pos}b). This observation reveals a radically different spectral response compared to that of $\VII/\VIII$ and $\VIII/\VIV$.
Another difference is that the catholyte $\VIV/\VV$ exhibits significantly higher absorbances, reaching values as high as 65~$\rm cm^{-1}$ over a wide range of wavelengths (560-850 nm) and SOCs (25-75\%).

According to Blanc et al.~\cite{blanc_spectrophotometric_1982}, the observed high-absorbance non-linear spectral behavior can be attributed to the formation of a 1:1 stoichiometric mixed-valence complex, $\rm V_2O_3^{3+}$. This complex is believed to form according to the equilibrium reaction
\begin{equation}
\label{eq:V2O33p_formation}
{\underbrace{\rm VO^{2+}\vphantom{_2}}_{\VIV} + \underbrace{\rm VO_2^{+}}_{\VV}} \; {\stackrel{K_c}{\rightleftharpoons}} {\rm \; V_2O_3^{3+}}    
\end{equation}
characterized by the equilibrium constant
\begin{equation} \label{eq:eqcondC4C4C45}
K_c = \frac{C_{45,\rm  }}{C_{4}C_{5}}    
\end{equation}
written here in terms of the equilibrium concentrations of $\VIV$, $\VV$, and that of the mixed-valence complex, denoted here as $C_{45}$. The overall stoichiometry of reaction \eqref{eq:V2O33p_formation} provides the following relations between the equilibrium concentrations
\begin{align}
C_4 & = C_4^*-C_{45} = X_4 C - C_{45} \label{eq:C4eq}\\
C_5 & = C_5^*-C_{45} = (1-X_4)C - C_{45} \label{eq:C5eq}
\end{align}
where $C_4^* = C_4 + C_{45}$ and $C_5^* = C_5 + C_{45}$ denote the nominal concentrations of $\VIV$ and $\VV$ in the absence of the mixed-valence complex, i.e., in the hypothetical case $K_c = 0$. In the above equations, $C_4^*$ and $C_5^*$ are conveniently rewritten as the product of the total vanadium concentration $C = C_4^*+C_5^* = C_4 + C_5 + 2C_{45}$ and the nominal mole fractions $X_4 = C_4^*/C$ and $X_5 = C_5^*/C$, which are in turn related to the SOC, given by $X_5 = 1 - X_4$. For further reference, the relations between $C_4$, $C_5$, $C_{45}$, $C$, $X_4$, $X_5$, and SOC are gathered in the last column of Table~\ref{tab:notation}.

Summarizing, the positive electrolyte contains 3 different compounds accounting for the total theoretical absorbance
\begin{equation}
    A_{\rm M} = A_{4} + A_{5} + A_{45}
\end{equation}
where $A_4$, $A_{5}$, and $A_{45}$ are the absorbances of $\VIV$, $\VV$, and $\rm V_2O_3^{3+}$, respectively. With use made of Equations~\eqref{eq:beer_lambert} and \eqref{eq:beer_lambert_power}, the absorbance of the mixture can be written as
\begin{equation}\label{eq:abs_pos}
        A_{\rm M} = \epsilon_4 C_{4} + \epsilon_5 C_{5}^k + \epsilon_{45} C_{45}
\end{equation}
Here we assume that the mixed valence complex $\rm V_2O_3^{3+}$ still follows the Beer-Lambert law~\eqref{eq:beer_lambert}, but generalize the analysis of Blanc et al.~\cite{blanc_spectrophotometric_1982} by considering 
the non-linear dependence between $A_5$ and the concentration of $\VV$ described by Equation\eqref{eq:beer_lambert_power}.

Combining Equations~\eqref{eq:C4eq}, \eqref{eq:C5eq}, and \eqref{eq:abs_pos}, we can write
\begin{equation}
\label{eq:abs_pos_2}
A_{\rm M} = \epsilon_4(X_4C-C_{45}) + \epsilon_5[(1-X_4)C-{C_{45}}]^k + \epsilon_{45}C_{45}
\end{equation}
Moreover, substituting Equations~\eqref{eq:C4eq} and \eqref{eq:C5eq} into the equilibrium condition~\eqref{eq:eqcondC4C4C45} yields a quadratic equation for $C_{45}$ that can be solved analytically to give~\cite{blanc_spectrophotometric_1982, quill_factors_2015}
\begin{equation}\label{eq:c45}
    C_{45} = \frac{1-\sqrt{1-4\chi^2X_4(1-X_4)C^2}}{2\chi}
\end{equation}
where
\begin{equation}\label{eq:chi}
   \chi = \frac{K_c}{K_cC + 1}
\end{equation}
According to the above analysis, during the charge and discharge of the $\VIV/\VV$ electrolyte the molar concentrations of the three compounds are expected to vary non-linearly with the state of charge, ${\rm SOC} = 1 - X_4$. This non-linear behavior poses a greater challenge for interpreting the spectra compared to the $\VII/\VIII$ and $\VIII/\VIV$ electrolytes.

It is useful to note that for wavelengths above 600~nm the molar absorptivity of $\VV$ is virtually zero, $\epsilon_5 \approx 0$, so that  Equation~\eqref{eq:abs_pos_2} reduces to
\begin{equation}
A_{\rm M}^{\lambda > 600{\rm nm}} =\epsilon_4(X_4 C-C_{45}) + \epsilon_{45}C_{45}
\end{equation}
which, upon substitution of $C_{45}$ from Equations~\eqref{eq:c45} and \eqref{eq:chi}, yields
\begin{equation}\label{eq:abs_pos_3}
A_{\rm M}^{\lambda > 600{\rm nm}} =\epsilon_4X_4C + (\epsilon_{45}-\epsilon_4)\left[\frac{1-\sqrt{1-4X_4(1-X_4)(K_cC)^2/(K_cC + 1)^2}}{2K_c/(K_cC + 1)}\right]
\end{equation}
In this equation, $K_c$ and $\epsilon_{45}$ are unknown constants, while $C$ and $X_4$ are known parameters for each  $\VIV/\VV$ calibration sample. Thus, since we have an experimental database of 44 reference spectra $A_{\rm exp}$ for different values of $C$ and $X_4$, we can perform a multivariate function fit to find the best pair ($\epsilon_{45}$, $K_c$) that, for each wavelength $\lambda > 600$~nm, minimizes the sum of the error squares
\begin{equation}
    \sum_{\underset{\lambda > 600{\rm nm}}{C,\hspace{1pt} X_4}} (A_{\rm exp} - A_{\rm M}^{\lambda > 600{\rm nm}})^2
\end{equation}
with the summation extending over all the 44 pairs ($C$, $X_4$) available in the $\VIV/\VV$ database.

% Figure environment removed

The black dots plotted in Figure~\ref{fig:pos_fit}a represent the experimentally measured absorbance at 660~nm, $A_{\rm exp}^{\rm 660nm}$, as a function of $X_4$. The red lines are the predicted absorbances resulting from the fitted values of $\epsilon_{45}$ and $K_c$ for that particular wavelength. Figure \ref{fig:pos_fit}b plots the coefficient of determination $R^2$ of the fit versus the wavelength. Details on the calculation of $R^2$ are given in the Supplementary Information (Section~S2). Overall, the model demonstrates excellent agreement with the experimental data, with a robust and stable value of $R^2$ averaging $\overline{R^2} \approx 0.99$ for $\lambda>600$~nm. The fitted values of $\epsilon_{45}$ and $K_c$ are represented in Figures~\ref{fig:pos_fit}c and \ref{fig:pos_fit}d as a function of the wavelength. It is important to clarify that the observed variation of approximately $\pm 10\%$ in $K_c$ is either a numerical artifact that results from the multivariate regression process or could be attributed to model limitations and experimental errors. The reaction constant $K_c$ is a chemical property of the electrolyte, and therefore it should remain constant regardless of the wavelength sampled.

The average value of $K_c$ indicated in the plot ($\overline{K}_c = 0.87 \pm 0.088~\rm M^{-1}$) falls within the higher end of the range 0.2-0.8 $\rm M^{-1}$ reported in the literature~\cite{blanc_spectrophotometric_1982, buckley_towards_2014, quill_factors_2015, loktionov_operando_2022}. Although the discrepancy is small, it could be traced back to the lower total vanadium concentrations used in the cited experiments. Additionally, by expressing the absorbance of the isolated complex in the form
\begin{equation}
A_{45} = \epsilon_{45}C_{45} = \epsilon_{45} K_c C_4C_5
\end{equation}
it becomes clear that there is an inverse relationship between $K_c$ and the molar absorptivity $\epsilon_{45}$, so that the fitting could perform equally well for different pairs ($\epsilon_{45}$, $K_c$) just by increasing $K_c$ and reducing $\epsilon_{45}$. Please note that the values of $K_c$ reported in the literature have always been obtained indirectly, as we do in this study, without direct measurements of the complex concentration.

The last subplot, Figure~\ref{fig:pos_fit}e, shows $X_4(A_{\rm max})$ as a function of the wavelength. This is essentially the value of $X_4$ that gives maximum absorbance for a given wavelength, e.g., $X_4 = 55.1\%$ for $\lambda = 660$~nm, as shown in Figure~\ref{fig:pos_fit}a. This turning point is of practical importance as it serves as a reference marker for estimating the total concentration between charging cycles. 

The results presented in this section are used below to define two calibration methods allowing to estimate the ${\rm SOC} = 1 - X_4$ and the total vanadium concentration $C$ of the catholyte $\VIV/\VV$.

\subsubsection{Fast empirical method}

The nonlinear character of Equation~\eqref{eq:abs_pos_3} makes it impossible to solve it analytically for either $X$ or $C$. An alternative approach involves fitting the absorbance of the mixture to a quadratic function of $X_4$ and $C$ of the form
\begin{equation}\label{eq:abs_pos_empir}
    A_{\rm M} = a_0X_4C + a_1X_4C^2 + a_2X_4^2C + a_3X_4^2C^2
\end{equation}
Table~\ref{tab:param_pos_direct} lists the values of the fitting parameters $a_i$ for two selected wavelengths in the range $\lambda > 600$~nm. 
The reported values were obtained through multivariate polynomial regression of the experimental absorbance data from the 44 calibration samples of the $\VIV/\VV$ mixture at $660$~nm and $760$~nm. 
As demonstrated in the Supplementary Information (Figure~S2), the quadratic fit \eqref{eq:abs_pos_empir} exhibits comparable performance to the theoretical equation \eqref{eq:abs_pos_3} with an average $R^2$ 
of 0.99.

\begin{table}[hb]
    \centering
\begin{tabular}{ccc}
& \multicolumn{2}{c}{Wavelength $\lambda$} \\[1mm]
\cline{2-3} \\[-3.5mm]
  Fitting parameter  & 660 nm & 760 nm \\[1mm]
\hline \\[-3.5mm]
$a_0$ ($\rm M^{-1}$) & $\phantom{-}62.12$ & $\phantom{-}71.29$ \\
$a_1$ ($\rm M^{-2}$)& $\phantom{-}41.83$ & $\phantom{-}33.62$ \\
$a_2$ ($\rm M^{-1}$)& $-50.65$ & $-51.71$ \\
$a_3$ ($\rm M^{-2}$)& $-42.63$ & $-34.56$ \\
\end{tabular}
\caption{Values of the fitting parameters $a_i$ appearing in Equation~\eqref{eq:abs_pos_empir} obtained from the absorbances of the 44 calibration samples of the $\VIV/\VV$ mixture at two selected wavelengths in the range $\lambda > 600$~nm.}
    \label{tab:param_pos_direct}
\end{table}

If the total vanadium concentration $C$ is known, then $X_4$ can be obtained  analytically by solving the quadratic Equation~\eqref{eq:abs_pos_empir} to give
\begin{equation}\label{eq:root_X}
    X_4 =  \frac{-(a_0C + a_1C^2) \pm \sqrt{(a_0C + a_1C^2)^2+4A_{\rm M}(a_2C+a_3C^2)}}{2(a_2C+a_3C^2)}
\end{equation}
which yields two possible roots ($X_{4-}$, $X_{4+}$). However, one of these roots is spurious. To elucidate which is the correct value of $X_4$ it is necessary to use Equation~\eqref{eq:root_X} with two different wavelengths ($\lambda_1$, $\lambda_2$). By comparing the solutions obtained with these two wavelengths, the false value of $X_4$ can be disregarded~\cite{petchsingh_spectroscopic_2016}. Figure \ref{fig:pos_direct} illustrates the process, based on evaluating the squared differences between the roots $X_{4-}$ and $X_{4+}$ provided by the two wavelengths
\begin{align}
    D_- & = \left(X^{\lambda_1}_{4-} - X^{\lambda_2}_{4-}\right)^2 \\
    D_+ & = \left(X^{\lambda_1}_{4+} - X^{\lambda_2}_{4+}\right)^2
\end{align}
and selecting the root that yields the smallest difference. The value of $X_4$ is then computed as the arithmetic mean of the selected root obtained from both wavelengths.

% Figure environment removed

As an illustrative example, Figure \ref{fig:pos_direct}a represents the absorbance versus $X_4$ curves computed using Equation~\eqref{eq:abs_pos_empir} with $C~=~1.22$~M and the fitting parameters reported in Table~\ref{tab:param_pos_direct} for 660~nm and 760~nm. An illustrative example demonstrating the application of the process for determining the composition $X_{4,P}$ of a particular mixture is also provided. In this case, we used the calibration sample corresponding to $C = 1.21$~M and $X_{4,\rm T} = 80\%$, plotting the experimentally measured absorbances $A_{\rm M}^{\rm 660nm}$ and $A_{\rm M}^{\rm 760nm}$ as horizontal lines. The intersection points of these lines with the parabolic curves represent the potential values of $X_{4,\rm P}$. Upon analysis, it is observed that $D_-<D_+$, indicating that the correct root corresponds to the one associated with the $-$ sign. Thus, the value of $X_{4,\rm P}$ can be obtained by averaging, in this case $X_{4,\rm P} = (X_{4-}^{\rm 660nm} + X_{4-}^{\rm 760nm})/2 = {\rm 80.9\%}$.

Figure \ref{fig:pos_direct}b shows the calibration curves representing the predicted ($X_{\rm 4,P}$) versus the true ($X_{\rm 4,T}$) values of $X_4$ for each of the four concentrations within the 44 calibration samples of the $\VIV/\VV$ mixture. 
The error $\RMSE_X$ is also indicated in the plots. The red circle located in the 1.22~M curve corresponds to the calibration data used as the example in Figure~\ref{fig:pos_direct}a.
It is worth noting that the method presented here is weaker than the one proposed in Section~\ref{sec:empirical234} for the $\VII/\VIII$ and $\VIII/\VIV$ mixtures, since in this case $X_4$ and $C$ cannot be obtained at once using only two absorbance values. Indeed, the total concentration must be either known \textit{a-priori} or assumed, which may give rise to systematic errors over the long-term cycling of the battery due to the accumulated effect of ion crossover.

One possible way to determine the total vanadium concentration is to monitor the absorbance of the $\VIV/\VV$ mixture during cycling in real time. This enables the estimation of $C$ from the peak absorbance, $A_{\rm max}$, observed during the charge or discharge process. As previously discussed, the mole fraction at the turning point, $X_4(A_{\rm max})$, is well known for all wavelengths (see Figure \ref{fig:pos_fit}e). Thus, by using the pair $(A_{\rm M}, X_4) = (A_{\rm max}, X_4(A_{\rm max}))$ one can estimate the total concentration by solving the quadratic Equation~\eqref{eq:abs_pos_empir} to give
\begin{equation}\label{eq:root_C}
    C =  \frac{-(a_0X_4 + a_2X_4^2) + \sqrt{(a_0X_4 + a_2X_4^2)^2+4A_{\rm M}(a_1X_4+a_3X_4^2))}}{2(a_1X_4+a_3X_4^2)}
\end{equation}

\subsubsection{Spectral deconvolution}

% Figure environment removed

This method is similar to the one presented in Section~\ref{sec:deconv_lin}, but the equations here are slightly different due to the presence of the complex $\rm V_2O_3^{3+}$. The first requirement is to determine the molar absorptivity spectrum of the complex $\epsilon_{45}$. Since this species does not appear in isolation from $\VIV$ and $\VV$, the experimental determination of $\epsilon_{45}$ must be done indirectly. To achieve this, we can rewrite Equation~\eqref{eq:abs_pos_2} as
\begin{equation}\label{eq:e45}
    \epsilon_{45} = \frac{A_{\rm exp} - \epsilon_4(X_4C-{C_{45}}) - \epsilon_5\left[(1-X_4) C-C_{45}\right]^k} { C_{45}}
\end{equation}
where $C_{45}$ is evaluated from Equation~\eqref{eq:c45} using the approximate value of $\chi \approx \overline{K}_c/(\overline{K}_c + 1)$ obtained from Equation~\eqref{eq:chi} with the average value of the equilibrium constant reported above, $\overline{K}_c=0.87 \rm~M^{-1}$. The equation is applied for all the calibration samples (36 in total, disregarding the 8 samples with either $X_4$ or $X_5 = 0$) using $A_{\rm exp}$ as the measured absorbance. The reference $\epsilon_{45}$ curve, plotted in Figure \ref{fig:pos_deconv}a, is computed by averaging over all the calibration samples (see Figure~S3). Further details regarding the averaging of $\epsilon_{45}$ and the limits of the model are given in the Supplementary Information (Section~S3). The high molar absorptivity of $\rm V_2O_3^{3+}$ is noteworthy, particularly when compared to those of $\VIV$ and $\VV$; e.g., while $\epsilon_{45}$ reaches values as high as $200~\rm cm^{-1}M^{-1}$, $\epsilon_4$ peaks at about $20~\rm cm^{-1} M^{-1}$ (see Figure~\ref{fig:spectra_soc})

Once the absorptivity spectrum $\epsilon_{45}$ is known, we can apply the spectral deconvolution method to any arbitrary mixture of positive electrolyte $\VIV/\VV$. The procedure involves determining the pair ($X_4$, $C$) that minimizes the sum of the error squares between the measured mixture absorbance $A_{\rm exp}$ and the absorbance predicted by Equation~\eqref{eq:abs_pos_2}, that is,
\begin{equation}
\label{eq:deconv_pos}
    \sum_{\lambda} \left\{A_{\rm exp} - \epsilon_4(X_4C - C_{45}) - \epsilon_5\left[(1-X_4)C- C_{45}\right]^k - \epsilon_{45} C_{45}\right\}^2
\end{equation}
where the summation ranges from 440~nm to 1000~nm, the widest wavelength range that the spectrometry system offers with a sufficient signal-to-noise ratio for the positive electrolyte.

The method proposed here includes several enhancements with respect to that proposed by Loktionov et al.~\cite{loktionov_operando_2022}. A key innovation in our case is the incorporation of the power-law absorbance versus concentration response for $\VV$ stated in Equation~\eqref{eq:beer_lambert_power}, which translates into the nonlinear term on the right hand side of Equation~\eqref{eq:abs_pos_2}. This generalization significantly improves the calibration results. Indeed, the use of $k=1$ in Equation~\eqref{eq:beer_lambert_power} leads to poor fitting results for the $\VV$ spectrum at wavelengths lower than 600~nm.
A parametric study exploring various values of $k$ is provided in the Supplementary Information (Section~S4), demonstrating that the optimal power-law exponent is $k = 2.09$. This is close to the value $1.88$ reported in Figure \ref{fig:ref_spectra}, fitted using only one wavelength.
Additionally, our method introduces other calibration improvements. By considering the entire range of wavelengths instead of dividing it into two ranges, we eliminate one fitting step. Moreover, the use of the analytical expressions \eqref{eq:abs_pos_2} and \eqref{eq:c45} directly provides the desired values of $C$ and $X_4$ for a broader range (0.91-1.83~M) of total vanadium concentrations.

Figure \ref{fig:pos_deconv}b provides an illustrative example of the spectral deconvolution process showing the individual contribution of all compounds. The agreement is excellent, with the red curve (fitted) accurately overlapping the black curve (measured). Figures \ref{fig:pos_deconv}c and \ref{fig:pos_deconv}d show the calibration curves for $X_4$ and $C$. Finally, Table \ref{tab:RMSE_deconv_pos} indicates the RMSE errors resulting from the application of the spectral deconvolution method to both variables. 

\begin{table}[htbp]
    \centering
\begin{tabular}{ccc}
$C$ (M) & $\RMSE_X$ (\%) & $\RMSE_C$ (M) \\[1mm]
\hline \\[-3.5mm]
1.83 & 2.26  & 0.078  \\
1.52 & 1.70  & 0.028  \\
1.22 & 1.07  & 0.029 \\
0.91 & 1.34  & 0.013 \\[1mm]
\hline \\[-3.5mm]
 Average &\textbf{1.59}  & \textbf{0.037}
 \end{tabular}
\caption{RMSE error of the spectral deconvolution method of the positive electrolyte $\VIV/\VV$ corresponding to the $\VIV$ mole fraction, $X_4$, and the total vanadium concentration, $C$.}
    \label{tab:RMSE_deconv_pos}
\end{table}

\section{Discussion}
\label{sec:discussion}

Figure \ref{fig:summary_sigma} shows histogram plots of the average RMS errors $\RMSE_X$ and $\RMSE_C$ for the three vanadium mixtures under study, comparing the empirical and spectral deconvolution methods.
Among the various vanadium mixtures, the calibrations perform slightly worse for the positive electrolyte $\VIV$/$\VV$. This is expected because the spectral response is significantly altered by the presence of the high absorbance complex $\rm V_2O_3^{3+}$, as discussed in the previous section. On average, the spectral deconvolution method performs better in both measuring the total concentration $C$ and the representative vanadium mole fractions $X_i$, $i = \{2, 4\}$. Additionally, no red bar is plotted for the $\RMSE_C$ error of the the positive electrolyte since the empirical calibration measures only $X_4$. 
The two calibration approaches differ radically in the quantity of information given to the algorithm. One the one hand, the empirical method is designed to use only a few hand-picked values while the spectral deconvolution uses the entirety of the spectrum. The deconvolution method essentially takes advantage of the whole spectrum to extract the maximum amount of information contained in it. We also expect the deconvolution approach to be more resilient to spectral distortions due to experimental errors. Such distortions can occur due to poor referencing, contaminated flow cuvettes (e.g., particles, bubbles), light scattering, change of light intensity over time, etc. These induces the spectra to be distorted or offsetted. We observed experimentally that, in such poor conditions the deconvolution was somehow able to absorb the distortions and delivered very acceptable values while the empirical methods yielded incorrect values. To delve further and assess this resilience, it would be useful to purposefully induce these distortions in a controlled manner to test the robustness of the calibration methods, but this is outside the scope of this paper. Moreover, other calibration methods could be envisioned such as correlation analysis, principal component regression, partial least square as well as advanced machine learning, such methods being already applied in other chemometrics studies~\cite{liu_state_2012, guo_chemometric_2021, meza_ramirez_applications_2021}.

% Figure environment removed

\section{Conclusions}
\label{sec:conclusions}

This study presents an exhaustive overview of UV-Vis absorbance spectroscopy calibration applied to vanadium electrolytes to estimate the total vanadium concentration $C$ and the nominal mole fraction of the mixture compounds $X_i$. We considered the three known mixtures of vanadium electrolytes: $\rm V^{2+}/V^{3+}$ ($\VII/\VIII$), $\rm V^{3+}/VO_2^{+}$ ($\VIII/\VIV$), and $\rm VO_2^{+}/VO^{2+}$ ($\VIV/\VV$).
In total, we provide 6 calibration methods, 3 empirical while the other 3 require computer-based fits based on spectral deconvolution. On average, we obtained errors for the mole fraction $\RMSE_X \approx 1.0$-$1.5 \%$ and for the total concentration $\RMSE_C \approx 25$-$35$~mM.
Using a spectrum as the input, almost all calibrations provide both $C$ and $X_i$ as an output for a wide range of concentration (0.91-1.83M).

The empirical calibrations are aimed to be used in industrial applications, where a low-cost optical sensor picking up only two absorbance values could measure the SOC of the battery in real-time. This would supposedly come with a slightly worse accuracy but at a lower cost. The spectral deconvolution method is considered more powerful but computationally intensive, so it is better suited for laboratory applications where spectrometers and computers are common, but comes at a higher cost.

The data provided with this paper is publicly shared in open access repositories (\href{https://github.com/AngeAM/SOC_Vanadium_Spectra_2023.git}{GitHub}) along with the Python code used to perform the calibrations. As such, anyone collecting spectral data of their vanadium electrolyte could measure the state of charge and the concentration of their electrolytes using our code and data. The main idea of this study is to provide an exhaustive, clear and useful guide to both industry and academia practitioners on how to accurately measure the total and relative concentrations of the different species present in vanadium electrolytes using UV-Vis absorbance spectroscopy.
This study brings valuable insights into the constantly evolving body of knowledge on vanadium electrolytes, which will hopefully lead to the design and development of more efficient and competitive vanadium redox flow batteries in the near future.

\section*{Data availability}
The calibration data and software tools developed in this work are publicly shared in the open repository \href{https://github.com/AngeAM/SOC_Vanadium_Spectra_2023.git}{https://github.com/AngeAM/SOC\_Vanadium\_Spectra\_2023.git} under the open access MIT license system. Detailed information on the metadata and the calibration scripts are given in the readme file located at the root of the repository. 

\section*{Acknowledgments}

We would like to acknowledge the critical hardware support provided by Micro Electrochemical Technologies S.L. (B5tec). We would also like to thank Alberto Bernaldo for his support in setting up the experimental platform, Sonia Sevilla for the help in preparing the samples, and Dr.~Vanesa Muñoz-Perales for the design of the electrochemical cell. Finally, we acknowledge the support of Jesús Palma from IMDEA energía for the ICP analysis and the electrolyte supply.

\section*{Author contributions}
Ange A. Maurice: Conceptualization, Methodology, Software, Validation, Formal analysis, Investigation, Resources, Data Curation, Writing - Original Draft, Visualization, Project administration, Funding acquisition. Alberto E. Quintero: Conceptualization, Methodology, Investigation, Resources, Writing - Original Draft, Supervision. Marcos Vera: Conceptualization, Formal analysis, Investigation, Resources, Writing - Review \& Editing, Supervision, Project administration, Funding acquisition.

\section*{Funding}
This work has been partially funded by FEDER/MICINN-
Agencia Estatal de Investigación under projects TED2021-129378B-C21 and PID2019-106740RB-I00/AEI/10.13039/501100011033. A.A.~Maurice acknowledges the support of an MCSF-Cofund “Energy for Future” (E4F) postdoctoral research fellowship by the Spanish Iberdrola Foundation (GA-101034297).

\section{Simulation results}\label{section:simulations}

In this section, we complement our theoretical analyses with simulation results on a synthetic environment with artificial noise and reward models as well as a simulation environment based on the real dataset, HeartStep V1 \citep{10.1093/abm/kay067}.

\paragraph{Compared algorithms.} In both simulation environments, we compare the following algorithms: Thompson sampling (\texttt{TS}) with normal priors \citep{russo2018tutorial}, Linear Upper Confidence Bound (\texttt{UCB}) approach \citep{chu2011contextual}, \texttt{MEB} (Algorithm \ref{alg1}), and \texttt{MEB-naive} (\texttt{MEB} plugged in with the naive measurement error estimator (\ref{eq:naive-estimator}) instead of (\ref{eq:proposed-estimator})). See Appendix \ref{appendix-simulation+} for a detailed description of the algorithms.

\subsection{Synthetic environment}

We first test our algorithms on a synthetic environment. We consider a contextual bandit environment with $d = 5$, $T = 50000$. In the reward model, we set $\btheta_0^* = (5, 6, 4, 6, 4)$, $\btheta_1^* = (6, 5, 5, 5, 5)$, and $\eta_t$ drawn i.i.d. from $\mathcal{N}(0, \sigma_{\eta}^2)$. Let $(\xb_t)_{t\in[T]}$ be independently sampled from $\mathcal{N}(\bmu_{x}, \bI_d)$, where $\bmu_x = \one_d$. We further set $\bSigma_{e, t} \equiv \bSigma_e \coloneqq \bI_d / 4$ and consider independent $(\bepsilon_{t})_{t\in[T]}$ with Normal distribution%; (ii) Multivariate $t(3)$ distribution 
with covariance $\bSigma_{e}$. We independently generate bandit data for $n_{exp} = 100$ times, and compare among the candidate algorithms in terms of estimation quality and cumulative regret with a moderate exploration probability $p_0 = 0.2$. 

% Table \ref{tab:synthetic} (a) shows the average regret (cumulative regret divided by $T$) over 50000 steps under under $\sigma^2_{\eta} \in \{0.01, 0.1, 1.0\}$ and $\sigma_{\epsilon}^2 \in \{0.1, 1.0, 2.0\}$. \texttt{MEB} showed significantly smaller average regret compared to other baseline methods under all choices of $\sigma^2_{\eta}$ and $\sigma_{\epsilon}^2$. An L2 estimation error plot can be found in Appendix Figure \ref{fig:synthetic_error}, which also demonstrates that \texttt{MEB} has lower estimation error. In general, a higher $\sigma_{\epsilon}^2$ leads to larger context noise leading to a larger benefit of using \texttt{MEB}.

% % Figure environment removed

% \begin{table}[h]
%     \centering
%     \begin{tabular}{c|cccc}
%        Algorithm & \texttt{TS} & \texttt{UCB} & MEB & MEB naive \\
%        \hline
%        Normal    & 2.38   & 2.33    &  0.123   & 1.50 \\
%        $t(3)$    & 1.81   & 1.63    &  0.617   & 1.66
%     \end{tabular}
%     \caption{Simple regret at $T = 10000$ with $p = 0.2$. For each independent run, we calculate the simple regret with 10000 samples. All the numbers are reported with a $\times 10^{-3}$ omitted.}
%     \label{tab:simple_regret}
% \end{table}

\subsection{HeartStep V1 simulation environment}

We also construct a simulation environment with HeartSteps dataset. HeartSteps is a physical activity mobile health application, whose primary goal is to help the user prevent negative health outcomes and adopt and maintain healthy behaviors, for example, higher physical activity level. HeartSteps V1 is a 42-day mobile health trial \citep{dempsey2015randomised,klasnja2015microrandomized,liao2016sample}, where participants are provided a Fitbit tracker and a mobile phone application. One of the intervention components is a contextually-tailored physical activity suggestion that may be delivered at any of the five user-specified times during each day. The delivery times are roughly separated by 2.5 hours.

% [TODO: mention data sensible consent in the appendix]

\paragraph{Construction of the simulated environment.} We follow the simulation setups in \cite{liao2020personalized}.  The true context at the time $t$ is denoted by $\xb_t$ with three main components $\xb_t = (I_t, Z_t, B_t)$. Here, $I_t$ is an indicator variable of whether an intervention ($A_t = 1$) is feasible (e.g. $I_t$ is 0 when the participant is driving a car, a situation where the suggestion should not be sent). $Z_t$ contains some features at time $t$. $B_t$ is the true treatment burden, which is a function of the participant's treatment history\footnote{Note this violates contextual bandits assumption and leads to an MDP. We believe this is a good setup to test the robustness of our proposed approach.}. Specifically, $B_{t+1} = \lambda B_t + 1_{\{A_{t} = 1\}}$. We assume that $(I_t)_{t\in[T]}$ and $(Z_t)_{t\in[T]}$ are sampled i.i.d with the empirical distribution from the Heartstep V1 dataset, and $(B_t)_{t\in[T]}$ is given by the aforementioned transition model.

The reward model is 
$
    r_t(\xb, a; \btheta) = \xb^{\top} \balpha + a f(\xb)^{\top} \bbeta + \eta_t,
$
where $\xb$ is the full context, $f(\xb)$ is a subset of $\xb$ that is considered to have an impact on the treatment effects, and $\btheta = (\balpha^\top, \bbeta^\top)^\top \in \mathbb{R}^9$. Here $\eta_t$ is the Gaussian noise on the reward observation, whose variance $\sigma_{\eta}^2$ is chosen to be $0.1, 1.0$, and $5.0$ respectively \citep{liao2016sample}. For a detailed list of variables in the context, see Table \ref{tab:variables} in Appendix \ref{appendix-simulation+}.

The true parameters $(\btheta_a^*)_{a\in\{0,1 \}}$ is estimated from GEE (Generalized Estimating Equations) with rewards being the log-transformed step count collected 30 minutes after the decision time.

In light of the measurement error setting in this paper, we consider an observation noise on $B_t$ for the following reasons: 1) The burden $B_t$ can be understood as a prediction of the burden level of the participant, which is particularly crucial in mobile health studies; 2) Other variables are normally believed to have low or no observation noise. Thus, we assume that the agent only observes
$
\tilde{\xb}_t = (I_t, Z_t, \tilde{B}_t),
$
where $\tilde{B}_t = B_t + \epsilon_t$ and $\epsilon_t$ is drawn i.i.d. from normal distribution with mean zero and variance $\sigma_{\epsilon}^2$.

\subsection{Results} Table \ref{tab:synthetic} (a) and (b) shows the average regret (cumulative regret divided by $T$) in both the synthetic environment and the real-data environment based on HeartStep V1. We use the same set of $\sigma_{\epsilon}^2 \in \{0.1, 1.0, 2.0\}$, while different $\sigma^2_{\eta}$ reflect the change of absolute values in coefficients in two different environments ($\sigma_{\eta}^2 = 5.0$ is the level of reward noise in HeartStep V1). \texttt{MEB} shows significantly smaller average regret compared to other baseline methods under most combinations of $\sigma^2_{\eta}$ and $\sigma_{\epsilon}^2$. An estimation error plot can be found in Appendix \ref{appendix-simulation+},
%Figure \ref{fig:synthetic_error} and \ref{fig:real-data_error}
which also demonstrates that \texttt{MEB} has a lower estimation error.

\begin{table}[ht]
\centering
\caption{Average regret for both synthetic environment and real-data environment under different combinations of $\sigma_{\eta}^2$ and $\sigma_{\epsilon}^2$. The results are averages over 100 independent runs and the standard deviations are reported in the full table in Appendix \ref{appendix-simulation+}.}
\begin{subtable}[h]{0.5\textwidth}
\caption{Average regret in the synthetic environment over $50000$ steps with clipping probability $p = 0.2$. }
\begin{tabular}{ll|llll}
$\sigma_{\eta}^2$ & $\sigma_{\epsilon}^2$ & \texttt{TS} & \texttt{UCB} & \texttt{MEB} & \texttt{MEB-naive} \\
\hline
\hline
0.01 & 0.1 & 0.047 & 0.046 & \textbf{0.027} & 0.038 \\
0.1  & 0.1 & 0.047 & 0.047 & \textbf{0.026} & 0.039 \\
1.0  & 0.1 & 0.048 & 0.048 & \textbf{0.027} & 0.038 \\
% 0.01 & 0.5 & 0.301 & 0.310 & \textbf{0.099} & 0.221 \\
% 0.1  & 0.5 & 0.328 & 0.320 & \textbf{0.110} & 0.221 \\
% 1.0  & 0.5 & 0.299 & 0.301 & \textbf{0.119} & 0.224 \\
0.01 & 1.0 & 0.757 & 0.647 & \textbf{0.198} & 0.371 \\
0.1  & 1.0 & 0.769 & 0.721 & \textbf{0.205} & 0.392 \\
1.0  & 1.0 & 0.714 & 0.697 & \textbf{0.218} & 0.404 \\
0.01 & 2.0 & 1.492 & 1.504 & \textbf{0.358} & 0.616 \\
0.1  & 2.0 & 1.195 & 1.333 & \textbf{0.368} & 0.584 \\
1.0  & 2.0 & 1.299 & 1.476 & \textbf{0.416} & 0.625\\
\hline
\hline
\end{tabular}
\end{subtable}

\vspace{3mm}

\begin{subtable}[h]{0.5\textwidth}
\caption{Average regret in the real-data environment over $2500$ steps with clipping probability $p = 0.2$. }
\begin{tabular}{ll|llll}
$\sigma_{\eta}^2$ & $\sigma_{\epsilon}^2$ & \texttt{TS} & \texttt{UCB} & \texttt{MEB} & \texttt{MEB-naive} \\
\hline
\hline
0.05 & 0.1 & 0.027 & 0.027 & \textbf{0.022} & 0.024 \\
0.1  & 0.1 & 0.026 & 0.024 & \textbf{0.020} & \textbf{0.020} \\
5.0  & 0.1 & 1.030 & \textbf{0.743} & 0.831 & 1.173 \\
0.05 & 1.0 & 0.412 & 0.408 & 0.117 & \textbf{0.112} \\
0.1  & 1.0 & 0.309 & 0.316 & \textbf{0.085} & 0.087 \\
5.0  & 1.0 & 1.321 & \textbf{0.918} & 1.458 & 1.322 \\
0.05 & 2.0 & 0.660 & 0.634 & \textbf{0.144} & 0.148 \\
0.1  & 2.0 & 0.740 & 0.704 & \textbf{0.151} & 0.155 \\
5.0  & 2.0 & 1.585 & 2.415 & 1.577 & \textbf{1.436}
\end{tabular}
\end{subtable}
\label{tab:synthetic}
\end{table}

\section{Discussion and conclusions}

We propose a new algorithm, \texttt{MEB}, which is the first algorithm with sublinear regret guarantees in contextual bandits with noisy context, where we have limited knowledge of the noise distribution. This setting is common in practice, especially where only predictions for unobserved context are available.
%the true context for decision-making can only be detected or learned approximately from observable auxiliary data. 
\texttt{MEB} leverages the novel estimator (\ref{eq:proposed-estimator}), which extends the conventional measurement error adjustment techniques by considering the interplay between the policy and the measurement error. 

\noindent\textbf{Limitations and future directions.} Several questions remain for future investigation. First, is $\tilde\cO(T^{2/3})$ the optimal rate of regret compared to the standard benchmark policy (\ref{eq:oracle-policy}), as in some other bandits with semi-parametric reward model (e.g. \cite{xu2022towards})? Providing lower bounds on the regret helps us understand the limit of improvement in the online algorithm. Second, we assume that the agent has an unbiased prediction of the true context. It is important to understand how biased predictions affect the results. %, as in practice, machine learning algorithms may generate biased predictions. 
%Last but not least, all analyses in this paper are restricted to bandits, where future contexts are not affected by previous actions.
Last but not least, it's interesting to see our method can be extended to more complicated decision-making settings (e.g. Markov decision processes).


\newpage 
\bibliographystyle{ims}
\bibliography{rl_aistats}
\newpage

\newpage
\appendix

\section{Proof of Lemma \ref{lemma, equivalence of two def of MDDO}}
\begin{proof}
For any ${\bs{\beta}}\in\mc H$, according to the definition of $G_{\bs s}$ (see Definition $\ref{def: MDDO}$), one has
\begin{align*}
\langle G_{\bs s},{\bs{\beta}}\rangle&=\int_{[0,1]} G_{\bs s}(t){\bs{\beta}}(t)~\mathrm{d}t=\int_{[0,1]}\mathrm{cov}\hspace{-0.9mm}\left(\bs{X}(t),\mathrm{e}^{\mi\langle \bs s,\Y\rangle}\right){\bs{\beta}}(t)~\mathrm{d}t\\
&=\int_{[0,1]}\mathrm{cov}\hspace{-0.9mm}\left(\bs{X}(t){\bs{\beta}}(t),\mathrm{e}^{\mi \langle \bs s,\Y\rangle}\right)~\mathrm{d}t.
\end{align*}
By Fubini theorem, under Assumption $\ref{as:joint distribution assumption}$, one can exchange the order of integration and covariance above and get that
\begin{align*}
 \langle G_{\bs s},{\bs{\beta}}\rangle&=\int_{[0,1]}\mathrm{cov}\hspace{-0.9mm}\left(\bs{X}(t){\bs{\beta}}(t),\mathrm{e}^{\mi \langle \bs s,\Y\rangle}\right)~\mathrm{d}t\\ &=\mathrm{cov}\hspace{-0.9mm}\left(\int_{[0,1]}\bs{X}(t){\bs{\beta}}(t)~\mathrm{d}t,\mathrm{e}^{\mi \langle \bs s,\Y\rangle}\right)=\mathrm{cov}\hspace{-0.9mm}\left(\langle \bs{X},{\bs{\beta}}\rangle,\mathrm{e}^{\mi \langle \bs s ,\Y\rangle}\right).
\end{align*}
Thus for any $\bs\alpha(t),{\bs{\beta}}(t)\in\mc H$, one can get
\begin{align*}
\big\langle \big(G_{\bs s}\otimes \overline{G}_{\bs s}\big)\bs\alpha,{\bs{\beta}}\big\rangle=\langle G_{\bs s},\bs\alpha\rangle\langle \overline{G}_{\bs s},{\bs{\beta}}\rangle=\mathrm{cov}\hspace{-0.9mm}\left(\langle \bs{X},\bs\alpha\rangle,\mathrm{e}^{\mi \langle \bs s,\Y\rangle}\right)\hspace{-0.9mm}\mathrm{cov}\hspace{-0.9mm}\left(\langle \bs{X},{\bs{\beta}}\rangle,\mathrm{e}^{-\mi\langle \bs s,\Y\rangle}\right)\\
=\mb{E}\hspace{-0.9mm}\left(\langle \bs{X},\bs\alpha\rangle\mathrm{e}^{\mi \langle \bs s,\Y\rangle}\right)\mb{E}\hspace{-0.8mm}\left(\langle \bs{X},{\bs{\beta}}\rangle\mathrm{e}^{-\mi \langle \bs s,\Y\rangle}\right)=\mb{E}\Big(\langle \bs{X},\bs\alpha\rangle\langle \bs{X}',{\bs{\beta}}\rangle\mathrm{e}^{\mi \langle \bs s,\Y-\Y'\rangle}\Big).
\end{align*}
Considering that $\mb{E}\big(\langle \bs{X},\alpha\rangle\langle \bs{X}',{\bs{\beta}}\rangle\big)=0$, one has
\begin{align*}
\big\langle \big(G_{\bs s}\otimes \overline{G}_{\bs s}\big)\bs\alpha,{\bs{\beta}}\big\rangle
=- \mb{E}\Big(\langle \bs{X},\bs\alpha\rangle\langle \bs{X}',{\bs{\beta}}\rangle\big(1-\mr{e}^{\mi \langle \bs s,\Y-\Y'\rangle}\big)\Big)&\\
=- \mb{E}\Big(\langle \bs{X},\bs\alpha\rangle\langle \bs{X}',{\bs{\beta}}\rangle\big[1-\cos\big(\langle \bs s,\Y-\Y'\rangle\big)\big]\Big)&\\
+\mi\mb{E}\Big(\langle \bs{X},\bs\alpha\rangle\langle \bs{X}',{\bs{\beta}}\rangle\big[\sin\big(\langle\bs s,\Y-\Y'\rangle\big)\big]\Big)&.
\end{align*}
It is easy to check that
\[\int_{\mb R^q}\frac{\sin \big(\langle\bs s,\Y-\Y'\rangle)\big)}{\|\bs s\|^{1+q}}~\mr{d}\bs s=\lim_{\varepsilon\to0^+}\int_{\bs s\in\mb{R}^q:\varepsilon\leqslant\|\bs s\|\leqslant \varepsilon^{-1}}\frac{\sin \big(\langle \bs s,\Y-\Y'\rangle\big)}{\|\bs s\|^{1+q}}~\mr{d}\bs s=0,\]
because the integrand is an odd function. By Lemma 1 in \cite{szekely2007measuring},  one can also get
\[\int_{\R^q}\frac{1-\cos\big(\langle \bs s,\Y-\Y'\rangle\big)}{\|\bs s\|^{1+q}}~\mr{d}\bs s=c_q\|\Y-\Y'\|.
\]
Combining above results with Definition $\ref{def: MDDO}$, one can obtain that 
\begin{align}\label{proof: lemma MDDO}
\langle\mathrm{MDDO}(\bs{X}|Y)\bs\alpha,{\bs{\beta}}\rangle=- \mb{E}\Big(\langle \bs{X},\bs\alpha\rangle\langle \bs{X}',{\bs{\beta}}\rangle\|\Y-\Y'\|\Big) .
\end{align}
Then by the arbitrariness of $\bs\alpha,{\bs{\beta}}\in\mc H$, the proof is completed. 
\end{proof}

\section{Proof of Theorem \ref{theorem, MDDO and conditional mean independence}}



According to \eqref{proof: lemma MDDO}, one can get the following useful lemma.
\begin{lemma}\label{lemma, MDDO and FMDD}
Under Assumption $\ref{as:joint distribution assumption}$, for all ${\bs{\beta}}\in\mathcal H$, $\|{\bs{\beta}}\|=1$, we have
\begin{align*}
\langle \mathrm{MDDO}(\boldsymbol{X}|\Y)({\bs{\beta}}),{\bs{\beta}}\rangle &=- \mathbb E\Big[ \langle \boldsymbol{X},{\bs{\beta}}\rangle \langle \boldsymbol{X}',{\bs{\beta}}\rangle \|\Y-\Y'\|\Big]\\
&=- \mathbb E\Big[\big\langle\langle \boldsymbol{X},{\bs{\beta}}\rangle{\bs{\beta}},\langle \boldsymbol{X}',{\bs{\beta}}\rangle{\bs{\beta}}\big\rangle\|\Y-\Y'\|\Big].
\end{align*}
\end{lemma}
This conclusion links MDDO with functional martingale
difference divergence  (FMDD, \citealt{lee2020testing}). 
Next we give the following two lemmas to finish the proof of Theorem $\ref{theorem, MDDO and conditional mean independence}$.
\begin{lemma}\label{lem: Txx=0tuiTx=0}If $T$ is a positive semi-definite operator on a Hilbert space $\wt{\mathcal{H}}$, then for all $x\in\wt{\mathcal{H}}$, one has $\langle Tx,x\rangle=0\Longleftrightarrow Tx=0$.
\end{lemma}
\begin{proof}
`$\Longleftarrow$': It is obvious.

`$\Longrightarrow$': It is easy to check that $f(a,b)=\langle Ta,b\rangle$ $(a,b\in\wt{\mc H})$ is a 
positive semi-definite Hermitian form. Thus, for any $y\in\wt{\mathcal{H}}$, one can use Cauchy inequality to get
\[|\langle Tx,y\rangle|^2\leqslant\langle Tx,x\rangle\langle Ty,y\rangle=0\Longrightarrow \langle Tx,y\rangle=0.\]
By the arbitrariness of $y\in\wt{\mc H}$, one has $Tx=0$.
\end{proof}

Our proof of Theorem $\ref{theorem, MDDO and conditional mean independence}$ is mainly inspired by the following property of
FMDD in \cite{lee2020testing}.
\begin{lemma}[Proposition 1 of \cite{lee2020testing}]\label{lem:prop1inlee}
If $\E[\|\X\|+\|\Y\|]<\infty$ and $\E[\|\bs X\|\|\Y\|]<\infty$, then we have
\[\E[\langle \X,\X'\rangle\|\Y-\Y'\|]=0\Longleftrightarrow \E[\X|\Y]=0\quad\text{almost surely},\]
where $(\X',\Y')$ is an i.i.d. copy of $(\X,\Y)$.
\end{lemma}
\paragraph{Proof of Theorem $\ref{theorem, MDDO and conditional mean independence}$}
\begin{proof}
Clearly, (ii) is a direct consequence of Lemma $\ref{lemma, equivalence of two def of MDDO}$ and the following lemma.

\begin{lemma}[Lemma 15 in \citealt{chen2023optimality}]\label{lem:cov TX}
If $T$ is an operator defined on $\mc H_1\to\mc H_2$ where $\mc H_i,i=1,2$ is a Hilbert space. $\bs X\in\mc H_1$ is a random element satisfying $\mb E[\bs X]=0$ . Then we have $\mr{var}(T\bs X)=T\mr{var}(\bs X)T^*$.
\end{lemma}

Now we start  to prove (i).
 First, one has
\begin{align*}\mathrm{MDDO}(\boldsymbol{X}|\Y)=0 &\Longleftrightarrow \mathrm{MDDO}(\boldsymbol{X}|\Y)({\bs{\beta}})=0,\quad\forall{\bs{\beta}}\in\mb{S}_{\mathcal H};\\
\mathbb E[\boldsymbol{X}|\Y]=0~~\text{a.s.}&\Longleftrightarrow\langle\mb E[\boldsymbol{X}|\Y],{\bs{\beta}}\rangle{\bs{\beta}}=0~~\text{a.s.} \quad\forall{\bs{\beta}}\in\mb{S}_{\mathcal H},
\end{align*}
where $\mb{S}_\mc{H}=\{{\bs{\beta}}\in\mc H:\|{\bs{\beta}}\|=1\}$. Second, from Lemma $\ref{lem: Txx=0tuiTx=0}$, one knows that
\begin{align*}
\mathrm{MDDO}(\boldsymbol{X}|\Y)({\bs{\beta}})=0&\Longleftrightarrow\langle\mathrm{MDDO}(\boldsymbol{X}|\Y)({\bs{\beta}}),{\bs{\beta}}\rangle=0.
\end{align*}
 Then under Assumption $\ref{as:joint distribution assumption}$, by Lemma $\ref{lemma, MDDO and FMDD}$ and $\ref{lem:prop1inlee}$, one has
\begin{align*}
&\langle\mathrm{MDDO}(\boldsymbol{X}|\Y)({\bs{\beta}}),{\bs{\beta}}\rangle=0\Longleftrightarrow\mathbb E[\big\langle\langle \boldsymbol{X},{\bs{\beta}}\rangle{\bs{\beta}},\langle \boldsymbol{X}',{\bs{\beta}}\rangle{\bs{\beta}}\rangle\|\Y-\Y'\|]=0\\
&\qquad\qquad\qquad\qquad\qquad\Longleftrightarrow\mathbb E[\langle \bs X,{\bs{\beta}}\rangle{\bs{\beta}}|\Y]=\langle \mb E[\boldsymbol{X}|\Y],{\bs{\beta}}\rangle{\bs{\beta}}=0~~\text{a.s.}
\end{align*}
This finishes the proof of Theorem $\ref{theorem, MDDO and conditional mean independence}$.
\end{proof}

% {\color{blue}\paragraph{Proof of Lemma \ref{lem:cov TX} (Repeated)}
% \begin{proof}
% For any $\u_1,\u_2\in\mc H_2$, we have
% \begin{align*}
% &\left\langle  T\mr{var}(\vX)T^*\u_1,\u_2  \right\rangle=\left\langle  T\mb E[\vX\otimes\vX]T^*\u_1,\u_2  \right\rangle
% =\left\langle  \mb E[\vX\otimes\vX]T^*\u_1,T^*\u_2  \right\rangle    
% \end{align*}
% since $\mb E[\vX]=0$. By the definition of convariance operator and expectation, we have 
% \begin{align*}
% \left\langle  \mb E[\vX\otimes\vX]T^*\u_1,T^*\u_2  \right\rangle=&\left\langle  \mb E[\left\langle\vX,  T^*\u_1 \right\rangle       \vX            ],T^*\u_2  \right\rangle
% =\mb E[  \left\langle\vX,  T^*\u_1 \right\rangle      \left\langle \vX            ,T^*\u_2  \right\rangle].
% \end{align*}
% Similarly, we have
% \begin{align*}
%  \left\langle  \mr{var}(T\vX)\u_1,\u_2  \right\rangle=\left\langle  \mb E[T\vX\otimes T\vX]\u_1,\u_2  \right\rangle=\mb E[  \left\langle T\vX,  \u_1 \right\rangle      \left\langle T\vX            ,\u_2  \right\rangle].\\    
% \end{align*}
% Then the proof is completed by noticing the following
% \begin{align*}
% \mb E[  \left\langle T\vX,  \u_1 \right\rangle      \left\langle T\vX            ,\u_2  \right\rangle]=\mb E[  \left\langle\vX,  T^*\u_1 \right\rangle      \left\langle \vX            ,T^*\u_2  \right\rangle].
% \end{align*}
% \end{proof}}



\section{Proof of Lemma \ref{lemma: SE=GammaS}}
Recall the following fact in FSIR.
\begin{lemma}\label{lemma, direct result of linearity condition}~\\
Under Assumption $\ref{as:Linearity condition and Coverage condition}~ \boldsymbol{\mathrm{{i)}}}$, we have $\mathcal S_{\mathbb E(\boldsymbol{X}|\Y)}\subseteq \Gamma \mc S_{\Y|\bs X}\subseteq \mc H$.
\end{lemma}
It is a trivial generalization of    \cite[Theorem 2.1]{ferre2003functional} from univariate response to multivariate response.
\paragraph{Proof of Lemma $\ref{lemma: SE=GammaS}$}
\begin{proof}
First, we prove that $\mathcal{S}_{\mathbb{E}(\bs X|\Y)}^\perp\subseteq \mathrm{Im}\{\mathrm{var(\mb{E}(\bs X|\Y))}\}^\perp$. For any ${\bs{\beta}}\in\mathcal{S}_{\mathbb{E}(\bs X|\Y)}^\perp$, one has $\langle{\bs{\beta}},\mb{E}(\bs X|\Y)\rangle=0$ a.s. Then for any $\bs\alpha\in\mathcal{H}$, one can get
\begin{align*}
\langle{\bs{\beta}},\mathrm{var}(\mb{E}(\bs X|\Y))\bs\alpha\rangle&=\langle{\bs{\beta}},\mb E\lmi\mb{E}(\bs X|\Y)\otimes \mb{E}(\bs X|\Y)\rmi\bs\alpha\rangle\\
&=\mb E\big(\langle\mb{E}(\bs X|\Y),\bs\alpha\rangle\langle{\bs{\beta}},\mb{E}(\bs X|\Y)\rangle\big)=0,
\end{align*}
which means that ${\bs{\beta}}\in\mathrm{Im}\{\mathrm{var}(\mb{E}(\bs X|\Y))\}^\perp$. Moreover, one has
\begin{align*}\mathcal{S}_{\mathbb{E}(\bs X|\Y)}^\perp\subseteq \mathrm{Im}\{\mathrm{var}(\mb{E}(\bs X|\Y))\}^\perp
%&\Rightarrow\left(\mathcal{S}_{\mathbb{E}(\bs X|Y)}^\perp\right)^\perp\supseteq \left(\mathrm{Im}\{\mathrm{var(\mb{E}(\bs X|Y))}\}^\perp\right)^\perp\\&
\Longrightarrow\overline{\mathcal{S}_{\mathbb{E}(\bs X|\Y)}}\supseteq\overline{\mathrm{Im}}\{\mathrm{var}(\mb{E}(\bs X|\Y))\}.
\end{align*}
Thus, $\overline{\mathrm{Im}}\{\mathrm{var}(\mb{E}(\bs X|\Y))\}\subseteq\overline{\mathcal{S}_{\mathbb{E}(\bs X|\Y)}}\subseteq\overline{\Gamma\mathcal{S}_{\Y|\bs X}}$ by Lemma $\ref{lemma, direct result of linearity condition}$. According to Assumption $\ref{as:Linearity condition and Coverage condition}$ \textbf{ii)}, one can get
\[\mathrm{dim}\left(\overline{\mathrm{Im}}\{\mathrm{var}(\mb{E}(\bs X|\Y))\}\right)=\mathrm{dim}\left(\overline{\mathcal{S}_{\mathbb{E}(\bs X|\Y)}}\right)=\mathrm{dim}(\overline{\Gamma\mathcal{S}_{\Y|\bs X}})=d.\]
One can complete the proof since finite dimension subspaces are closed.
\end{proof}

\section{Proof of Theorem \ref{theorem, MDDO and IRS}}
\begin{proof}
For convenience, we abbreviate $\mathrm{MDDO}(\boldsymbol{X}|\Y)$ to ${M}$. According to Theorem $\ref{theorem, MDDO and conditional mean independence}$ and Lemma $\ref{lem: Txx=0tuiTx=0}$, one can get
\begin{align*}{\bs{\beta}}\in\mathcal S_{\mb E(\boldsymbol{X}|\Y)}^\perp&\Longleftrightarrow\langle {\bs{\beta}},\mathbb E(\boldsymbol{X}|\Y)\rangle=0~~\text{a.s.}\Longleftrightarrow\mathbb E(\langle {\bs{\beta}},\boldsymbol{X}\rangle|\Y)=0~~\text{a.s.}\\
&\Longleftrightarrow\mathrm{MDDO}(\langle {\bs{\beta}},\boldsymbol{X}\rangle|\Y)=0\Longleftrightarrow\langle {M}{\bs{\beta}},{\bs{\beta}}\rangle=0\\
&\Longleftrightarrow{M}{\bs{\beta}}=0\Longleftrightarrow {\bs{\beta}}\in\mathrm{null}(M)=\overline{\mathrm{Im}}(M)^\perp,
\end{align*}
which means that $\mathcal S_{\mb E(\boldsymbol{X}|\Y)}^\perp=\overline{\mathrm{Im}}(M)^\perp$ and $\overline{\mathcal S_{\mb E(\boldsymbol{X}|\Y)}}=\overline{\mathrm{Im}}(M)$.
One can complete the proof since finite dimension subspaces are closed.
\end{proof}
\section{Proof of Lemma \ref{lemma, way of estimate truncate central subspace}}
Before proving Lemma $\ref{lemma, way of estimate truncate central subspace}$, we give the following lemma.
\begin{lemma}\label{lem: colPBP equal colPB operator}
Assume that $P$ is a bounded linear operator from a Hilbert space $\wt{\mc H}$ to itself and $B$ is a positive semi-definite operator from $\wt{\mc H}$ to itself. 
Then we have $\overline{\mathrm{Im}}(PBP^*)=\overline{\mathrm{Im}}(PB)$.
\end{lemma}
\begin{proof}
It suffices to show that $\mnull(BP^*)=\mnull(PBP^*)$. First, since $B$ is positive semi-definite, one has $\langle x,PBP^*x\rangle = \langle P^*x,BP^*x \rangle\geqslant 0~(\forall x\in\wt{\mc H})$. Thus $PBP^*$ is a positive semi-definite operator on $\wt{\H}$.
For any $y\in\wt{\H}$, we have 
\begin{align*}
PBP^*y=0\overset{(a)}{\Longleftrightarrow}\langle y,PBP^*y\rangle = \langle P^*y,BP^*y \rangle=0\overset{(b)}{\Longleftrightarrow} BP^*y=0. 
\end{align*}
where $(a)$ and $(b)$ come from Lemma $\ref{lem: Txx=0tuiTx=0}$.
Thus $\mnull(PBP^*)=\mnull(BP^*)$.
\end{proof}

\paragraph{Proof of Lemma $\ref{lemma, way of estimate truncate central subspace}$}
\begin{proof}
For convenience, we abbreviate $\mathrm{MDDO}(\boldsymbol{X}|\Y)$ and $\mathrm{MDDO}(\boldsymbol{X}^{(m)}|\Y)$ to ${M}$ and $M_m$ respectively. 

By Corollary $\ref{corollary, MDDO and central subspace}$, one can get $\Gamma\mathcal{S}_{\Y|\boldsymbol{X}}=\mathrm{Im}(M)$. Thus,
\begin{align}\label{eq: corollary, MDDO and central subspace}
\Pi_m\Gamma\mathcal{S}_{\Y|\boldsymbol{X}}=\Pi_m\mathrm{Im}(M)=\mathrm{Im}(\Pi_m M).
\end{align}
It is easy to check that
\begin{align}
\Gamma_m&:=\mathrm{var}(\bs X^{(m)})=\Pi_m\Gamma\Pi_m=\Pi_m\Gamma=\Gamma\Pi_m=\sum\limits_{i=1}^m\lambda_i\phi_i\otimes\phi_i.\label{eq: Gamma m def}
\end{align}
On the one hand, by the definition of $\mathcal{S}^{(m)}_{{\Y|\boldsymbol{X}}}$ and $\Gamma_m$ (see \eqref{def: truncated central subspace} and \eqref{eq: Gamma m def}), one can get
\begin{align}\label{eq:Pim Gamma S}
\Pi_m\Gamma\mathcal{S}_{\Y|\boldsymbol{X}}&=\Pi_m\Gamma\Pi_m\mathcal{S}_{\Y|\boldsymbol{X}}=(\Pi_m\Gamma)(\Pi_m\mathcal{S}_{\Y|\boldsymbol{X}})=\Gamma_m\mathcal{S}^{(m)}_{{\Y|\boldsymbol{X}}}.
\end{align}
On the other hand, one has $\overline{\mathrm{Im}}(\Pi_m M)=\overline{\mathrm{Im}}(\Pi_m M\Pi_m)$ by Lemma $\ref{lem: colPBP equal colPB operator}$. Since $\Pi_m M$ and $\Pi_m M\Pi_m$ are both of finite rank, one can further get
\begin{align*}
\mathrm{Im}(\Pi_m M)&=\overline{\mathrm{Im}}(\Pi_m M)=\overline{\mathrm{Im}}(\Pi_m M\Pi_m)=\mathrm{Im}(\Pi_mM\Pi_m).
\end{align*}
Then according to Theorem $\ref{theorem, MDDO and conditional mean independence}$(ii), one has
\begin{align}
\mathrm{Im}(\Pi_m M)=\mathrm{Im}(\Pi_mM\Pi_m)=\mathrm{Im}(M_m).\label{eq:Pim span M}
\end{align}
Combining \eqref{eq:Pim Gamma S}, \eqref{eq:Pim span M} with \eqref{eq: corollary, MDDO and central subspace}, one has $\Gamma_m\mathcal{S}^{(m)}_{{\Y|\boldsymbol{X}}}=\mathrm{Im}\{M_m\}$.
Finally, one can get $ \Gamma_m^\dagger\mathrm{Im}\{M_m\}=\Gamma_m^\dagger\Gamma_m\mathcal{S}^{(m)}_{{\Y|\boldsymbol{X}}}=\Pi_m\mathcal{S}^{(m)}_{{\Y|\boldsymbol{X}}}=\mathcal{S}^{(m)}_{{\Y|\boldsymbol{X}}}$.
\end{proof}

\section{Wely Inequality for a Self-adjoint and Compact Operator}\label{ap:Wely inequality for self-adjoint and compact operators}
First, we show the following three results in standard functional analysis textbook.
\begin{lemma}[Spectral theorem]\label{thm: Spectral theorem}Let $\wt{\mathcal{H}}$ be a Hilbert space and $A:\wt{\mc{H}}\to\wt{\mc{H}}$ be a compact, self-adjoint operator. There is an at most countable orthonormal basis $\{\wt e_j\}_{j\in J}$ ($J=\{1,\cdots,n\}$ or $\mathbb{Z}_{\geqslant1}$) of $\wt{\mathcal{H}}$ and eigenvalues $\{\wt\lambda_j\}_{j\in J}$ with $|\wt\lambda_1|\geqslant|\wt\lambda_2|\geqslant\cdots\geqslant0$ converging to zero, such that
\begin{align*}
x=\sum_{j\in J}\langle x,\wt e_j\rangle \wt e_j;\qquad Ax=\sum_{j\in J}\wt\lambda_j\langle x,\wt e_j\rangle \wt e_j,\qquad x \in\wt{\mathcal{H}}.
\end{align*}
\end{lemma}

\begin{lemma}[Rayleigh's principle]\label{lem:Rayleigh operator}Let $A$ be a compact, self-adjoint operator. If $\{\wt e_j\}_{j\in J}$ and $\{\wt\lambda_j\}_{j\in J}$ are eigenvectors and eigenvalues define in Lemma $\ref{thm: Spectral theorem}$ respectively. Then
\[|\wt\lambda_1|=\mathop{\sup\limits_{\|u\|=1}}|\langle Au,u\rangle|;\qquad|\wt\lambda_n|=\mathop{\sup\limits_{\|u\|=1}}_{u\in\{\wt e_1,\cdots,\wt e_{n-1}\}^\perp}|\langle Au,u\rangle|~(n\geqslant 2).\]
\end{lemma}
\begin{lemma}[Minimax theorem]\label{lem:minimax operator}
Assume that $A$ is a positive semi-definite and compact operator with its eigenvalues $\{\wt\lambda_i\}$ ordered as $\wt\lambda_1\geqslant\dots\geqslant \wt\lambda_n\geqslant\dots\geqslant 0$, then
$$
\wt\lambda_n=\inf_{E_{n-1}}\sup_{x\in E_{n-1}^\perp,\|x\|=1}\langle Ax,x\rangle
$$
where $E_{n-1}$ with dimension $n-1$ is a closed linear subspace of $\wt{\mc H}$.
\end{lemma}
Then we give the Wely inequality for a self-adjoint and compact operator.
\begin{proposition}\label{prop: wely operator}
Let $M=N+R$ where $M$, $N$ and $R$ are three self-adjoint and compact operators defined on a Hilbert space $\wt{\mc H}$. Also, $M$ and $N$ are positive semi-definite with their respective eigenvalues $\{\mu_i\},\{\nu_i\}$ ordered as follows
\begin{align*}
M:\mu_1\geqslant\dots\geqslant \mu_n\geqslant\dots\geqslant 0;\qquad
N:\nu_1\geqslant\dots\geqslant \nu_n\geqslant\dots\geqslant 0,
\end{align*}
while $R$'s eigenvalues are $\{\rho_i\}$ ordered as follows:
\[R:|\rho_1|\geqslant\dots\geqslant |\rho_n|\geqslant\dots\geqslant 0.\]
Then the following inequalities hold: $|\mu_k-\nu_k|\leqslant|\rho_1|=\|R\| $, $k\geqslant1$.
\end{proposition}
\begin{proof}
From Lemma $\ref{lem:minimax operator}$, we have:
\[\mu_n=\inf_{E_{n-1}}\sup_{x\in E_{n-1}^\perp,\|x\|=1}\langle Mx,x\rangle;\qquad\nu_n=\inf_{E_{n-1}}\sup_{x\in E_{n-1}^\perp,\|x\|=1}\langle Nx,x\rangle,\]
where $E_{n-1}$ with dimension $n-1$ is a closed linear subspace of $\wt{\mc H}$.
By Lemma $\ref{lem:Rayleigh operator}$, we have:
$$
\sup_{\|u\|=1}|\langle Ru,u\rangle|=|\rho_1|.
$$
Since $\langle Mu,u\rangle=\langle Nu,u\rangle+\langle Ru,u\rangle$, for any $\|u\|=1$, we have:
$$
\langle Nu,u\rangle-|\rho_1|\leqslant\langle Mu,u\rangle \leqslant \langle Nu,u\rangle+|\rho_1|.
$$
Then for any given $n-1$ dimensional closed linear subspace of $\wt{\mc H}$, we conclude
\begin{equation}\label{eq:max ineq}
\sup_{u\in E_{n-1}^\perp,\|u\|=1}\langle Nu,u\rangle-|\rho_1|\leqslant\sup_{u\in E_{n-1}^\perp,\|u\|=1}\langle Mu,u\rangle\leqslant \sup_{u\in E_{n-1}^\perp,\|u\|=1}\langle Nu,u\rangle+|\rho_1|.
\end{equation}
Take the infimum with respective to $E_{n-1}$ in \eqref{eq:max ineq}, we have
\[\nu_n-|\rho_1|\leqslant\mu_n\leqslant \nu_n+|\rho_1|\]
by Lemma $\ref{lem:minimax operator}$.
\end{proof}
The next result is a direct corollary of Proposition $\ref{prop: wely operator}$.
\begin{corollary}\label{coro:wely ineq operator}
Let $M$ and $N$ be two self-adjoint, positive semi-definite and compact operators defined on a Hilbert space $\wt{\mc H}$ with their respective eigenvalues $\{\mu_i\},\{\nu_i\}$ ordered as follows
\begin{align*}
M:\mu_1\geqslant\dots\geqslant \mu_n\geqslant\dots\geqslant 0\quad\text{and}\quad
N:\nu_1\geqslant\dots\geqslant \nu_n\geqslant\dots\geqslant 0.
\end{align*}
Then the following inequalities hold: $|\mu_k-\nu_k|\leqslant\|M-N\| $, $ k\geqslant1$.
\end{corollary}




\section{Proof of Proposition \ref{prop:bound hatMmd Mm}}
Before proving Proposition $\ref{prop:bound hatMmd Mm}$, we give the following conclusion, whose proof is deferred to the end of this section.
\begin{proposition}\label{proposition, concentration of MDDO}
Under Assumptions $\ref{as:joint distribution assumption}$ and $\ref{assumption: sub-Gaussian}$, for all $\gamma\in(0,1/2)$, there exist positive constants $D_0=D_0(\gamma,\sigma_0,\sigma_1)$, $D_1=D_1(\sigma_1)$, $D_2=D_2(\sigma_0,\sigma_1)$ and $n_0=n_0(\gamma,\sigma_0,\sigma_1)$ such that for all $n\geqslant n_0$ and
\[C\in \l D_0n^{\frac{2\gamma}{5}}-\ln\l D_1m^2n \r,D_2 n^{\frac{1}{5}}-\ln\l D_1m^2n \r \rmi,\]
we have
\begin{equation*}
\mathbb{P}\l\left\|\wh M_m- M_m\right\| <\l \frac{C+\ln( D_1m^2n)}{D_2}\r^{\frac52}\frac{12m}{\sqrt n}\r\geqslant 1-\exp(- C).
\end{equation*}
\end{proposition}
\paragraph{Proof of Proposition $\ref{prop:bound hatMmd Mm}$}
\begin{proof}




Using Corollary $\ref{coro:wely ineq operator}$, one can get
$
\lambda_i\l\wh M_m\r\leqslant \lno\wh M_m-M_m\rno +\lambda_i\l M_m\r
$. 
Since $\rank(M_m)=d$, one can get $\lambda_i(M_m)=0,~i\geqslant d+1$. Thus by Proposition $\ref{proposition, concentration of MDDO}$, one has
\begin{align}\label{eq:lambdai hat Mm upper bound}
\mathbb{P}\l\lambda_{d+1}(\wh M_m)<\l \frac{C+\ln\l D_1m^2n\r}{D_2}\r^{\frac52}\frac{12m}{\sqrt n}\r\geqslant 1-\exp(- C)\qquad(i\geq d+1). 
\end{align}
Notice that 
\begin{align*}\lno\wh M_m^d- M_m\rno &\leqslant\lno M_m-\wh M_m\rno +\lno\wh M_m-\wh M_m^d\rno ;\\
\lno\wh M_m-\wh M_m^d\rno &=\left\|\sum_{i=d+1}^\infty\wh\mu_i\wh\gamma_i\otimes \wh\gamma_i\right\| =\widehat{\lambda}_{d+1}=\lambda_{d+1}(\widehat{M}_m)
\end{align*}
by \eqref{wh M_m spectral decomposition}.
Then combing Proposition $\ref{proposition, concentration of MDDO}$ with \eqref{eq:lambdai hat Mm upper bound} can complete the proof.
\end{proof}


\paragraph{Proof of Proposition \ref{proposition, concentration of MDDO}}
\begin{proof}
Note that $\boldsymbol{X}^{(m)}=\sum\limits_{j=1}^m\langle \boldsymbol{X},\phi_j\rangle\phi_j$, then a simple calculation leads to
\begin{align*}
M_m&=-\sum_{i,j=1}^m\mathbb E\big[\langle \boldsymbol{X},\phi_i\rangle\langle \boldsymbol{X}',\phi_{j}\rangle\|\Y-\Y'\|\big]\phi_i\otimes\phi_j;\\
\widehat{M}_m&=-\sum_{i,j=1}^m\frac1{n^2}\sum_{k,\ell=1}^n\langle \boldsymbol{X}_k,\phi_i\rangle\langle \boldsymbol{X}_\ell,\phi_j\rangle\|\Y_k-\Y_\ell\|\phi_i\otimes\phi_j.
\end{align*}

For a operator $\Gamma'$ that can be expanded as $\Gamma':=\sum\limits_{i,j=1}^\infty a_{ij}\phi_i\otimes\phi_{j}$, let us define its maximal norm as $\|\Gamma'\|_{\mathrm{max}}=\sup\limits_{i,j}|a_{ij}|$.



\begin{lemma}\cite[Theorem 1]{mai2021slicing}\label{lemma, concentration of MDDOnm}
Under Assumptions $\ref{as:joint distribution assumption}$ and $\ref{assumption: sub-Gaussian}$, for all
$\gamma\in(0,1/2)$, there exist positive
constants $C_0=C_0(\gamma,\sigma_0,\sigma_1)$, $C_1=C_1(\sigma_1)$, $C_2 = C_2(\sigma_0;\sigma_1)$ and $n_0 = n_0(\gamma,\sigma_0,\sigma_1)$
such that for all $n\geqslant n_0$ and $\varepsilon\in(C_0 n^{-(1/2-\gamma)},1]$, we have
\begin{equation*}
\mathbb{P}\l\lno \widehat{M}_m-M_m\rno_{\max}>12\varepsilon\r\leqslant C_1 m^2n\exp\l- C_2\l\varepsilon^2 n\r^{1/5}\r.
\end{equation*}
\end{lemma}
\noindent Since $\lno\widehat{M}_m-M_m\rno \leqslant m\lno\widehat{M}_m-M_m\rno_{\mathrm{max}}$, one has
\begin{equation*}
\mathbb{P}\l\lno\widehat{M}_m-M_m\rno >12m\varepsilon\r\leqslant C_1 m^2n\exp\l-C_2\l\varepsilon^2 n\r^{1/5}\r.
\end{equation*}
Let $C=C_2\l\ve^2n\r^{1/5}-\ln\l C_1m^2n\r$ satisfying 
\begin{align*}
C\in\l C_2C_0^{2/5}n^{2\gamma/5}-\ln\l C_1m^2n\r,C_2n^{1/5}-\ln\l C_1m^2n\r\rmi,
\end{align*}
then one has
\begin{equation*}
\mathbb{P}\l\lno\widehat{M}_m-M_m\rno \leqslant\l \frac{C+\ln\l C_1m^2n\r}{C_2}\r^{\frac52}\frac{12m}{\sqrt{n}}\r>1- \exp(- C).
\end{equation*}
Then in order to complete the proof, one only need to choose $D_0$, $D_1$ and $D_2$ to be $C_2C_0^{2/5}$, $C_1$ and $C_2$ respectively. 
\end{proof}





\section{Properties of Sub-Gaussian Random Vectors}
We first review the definition of sub-Gaussian random variables.
\begin{definition}[Sub-Gaussian random variable and its upper-exponentially bounded constant]\label{def:sub gaussian variable}
A random variable $X$ is called a sub-Gaussian random variable if $X$ satisfies one of the following equivalent properties:
\begin{itemize}
 \item[1).] Tails. $\P(|X|>t)\leqslant \exp(1-t^{2}/K^{2}_{1})$ for any $t>0$;
 \item[2).] Moments. $\E[|X|^{p}]^{1/p}\leqslant K_{2}\sqrt{p}$ for any $p\geqslant 1$;
 \item[3).]Super-exponential moment: $\E[\exp(X^{2}/K^{2}_{3})]\leqslant \mr{e}$.

\noindent Moreover, if $\E[X]=0$, then the properties $1)-3)$ are also equivalent to the following one:
\item[4).] Moment generating function: $\E[\exp(tX)]\leqslant \exp(t^{2}K^{2}_{4})$ for all $t\in\R$.
\end{itemize}
Here $K_1$, $K_2$, $K_3$ and $K_4$ are four constants.
$K$ is called an upper-exponentially bounded constant of $X$ if 
$K\geqslant \max\{K_{1},K_{2},K_{3},K_{4}\}$.
\end{definition}
\begin{definition}[Sub-Gaussian random vector and its upper-exponentially bounded constant]\label{def,sub-Gaussian random vector,upper-exponentially bounded constant}
 ${X}\in\R^m$ is called a sub-Gaussian random vector if for all $x\in\R^m$, one-dimensional marginal $\langle{X},x\rangle$ is sub-Gaussian random variable. $K$ is called an upper-exponentially bounded constant of $X$ if $K$ satisfies:
 \begin{align*}
K\geqslant \sup_{x\in\mb{S}^{m-1}}K(\langle X,x\rangle) 
 \end{align*}
 where $K(\langle X,x\rangle)$ denotes an upper-exponentially bounded constant of $\langle X,x\rangle$.
Moreover, $K$ is called a uniform (about $m$) upper-exponentially bounded constant of $X$ if $K$ satisfies:
 \begin{align*}
K\geqslant \sup_m\sup_{x\in\mb{S}^{m-1}}K\l \langle X,x\rangle\r.
 \end{align*}
Furthermore, $X$ is called a uniform (about $m$) sun-Gaussian random vector.
 \end{definition}
The following is an application of sub-Gaussian random vectors.
\begin{lemma}[\citealt{vershynin2010introduction}]\label{lem:esgrm}
 Let $\M=[\bs m_1~\cdots~\bs m_n]$ be an $m\times n$ matrix ($n>m$) whose columns $\m_{i}$ are 
 independent centered sub-Gaussian random vectors with 
 covariance matrix $\mathbf{I}_{m}$. Let $\sigma^{+}_{\min}(\M)$ and $\sigma_{\max}(\M)$ be the infimum and supremum of positive singular values of $\M$ respectively. Then, for any $t>0$, with probability at least $1-2\exp(- C^{\prime}t^{2})$, we have
 \begin{equation*}
 \sqrt{n}-C_0\sqrt{m}-t\leqslant \sigma^{+}_{\min}(\M)\leqslant \sigma_{\max}(\M)\leqslant \sqrt{n}+C_0\sqrt{m}+t
 \end{equation*}
 where $C'$ and $C_0$ are two positive constants depending only on $K(\bs m_1)$:
 the upper-exponentially bounded constant of $\bs m_1$.
\end{lemma}
\noindent Let $t=\sqrt m$, then one can easily get
\begin{align}\label{equation, min max eval}
\begin{split}
\lambda_{\max}\left(\frac1n \M\M^\top\right)\leqslant \left(1+\frac{(C_0+1)\sqrt m}{\sqrt n}\right)^2;\\
\lambda_{\min}^+\left(\frac1n \M\M^\top\right)\geqslant \left(1-\frac{(C_0+1)\sqrt m}{\sqrt n}\right)^2, 
\end{split}
\end{align}
with probability at least $1-2\exp(- C'm)$ where $\lambda^{+}_{\min}(\cdot)$ and $\lambda_{\max}(\cdot)$ stands for the infimum and supremum of the positive spectrum respectively.



\begin{lemma}\label{lemma, estiamtion error of inverse sample cov}
Assume that $\x_1,\x_2,...,\x_n$ are $n$ i.i.d. samples from an $m$-dimensional centered sub-Gaussian vector with an invertible covariance matrix $\Sigma$. Let $\wh\Sigma:=\frac1n\sum_i \x_i\x_i^\top$.
Then there exists a positive constant $n_1'=n_1'(K(\bs m_1),c_1)$ ($c_1$ is defined in \eqref{eq: m n relationship}), such that when $n\geqslant n_1'$, we have
\begin{align*}
\lno\wh{\Sigma}-\Sigma\rno\hspace{-1.5mm}&\leqslant (C_0+2)^2\lambda_{\max}(\Sigma)\sqrt{\frac mn}~~\text{and}~~ \lno\wh{\Sigma}^{-1}-\Sigma^{-1}\rno\hspace{-1.5mm}\leqslant \frac{4(C_0+2)^2}{\lambda_{\min}(\Sigma)}\sqrt{\frac mn},
 \end{align*}
 with probability at least $1-2\exp(- C'm)$, where $C_0$ is defined in Lemma $\ref{lem:esgrm}$.
\end{lemma}
\begin{proof}
Let $\x_i=\Sigma^{\frac12}\m_i$ and $\bs{M}=[\bs m_1~\cdots~\bs m_n]$ where $\m_i$ is a centered sub-Gaussian random vector with covariance $\mathbf I_m$. Then one has 
\begin{align*}
\lno\wh\Sigma-\Sigma\rno&\leqslant\lno\Sigma^{\frac12}\rno\cdot\left\|\frac1n \M\M^\top-\mathbf I\right\|\cdot\lno\Sigma^{\frac12}\rno\\
&= \lambda_{\max}(\Sigma)\cdot\left[\lambda_{\max}\left(\frac1n \M\M^\top\right)-1\right]
\end{align*}
and 
\begin{align*}
\lno\wh{\Sigma}^{- 1}-\Sigma^{- 1}\rno
&\leqslant \lno\Sigma^{-\frac12}\rno\cdot\left\|\frac1n \M\M^\top-\mathbf I\right\|\cdot\lno\l\frac1n \M\M^\top\r^{-1}\rno\cdot\lno\Sigma^{-\frac12}\rno\\
&=\frac{1}{\lambda_{\min}(\Sigma)}\left[\lambda_{\max}\left(\frac1n \M\M^\top\right)-1\right]\cdot\lambda_{\min}\left(\frac1n \M\M^\top\right)^{-1}.
\end{align*}
By \eqref{equation, min max eval}, it is easy to check that
\begin{align*}&\lambda_{\max}\left(\frac1n \M\M^\top\right)-1\leqslant\left(1+\frac{(C_0+1)\sqrt m}{\sqrt n}\right)^2-1\leqslant\frac{(C_0+2)^2\sqrt m}{\sqrt n};\\
&\lambda_{\min}\left(\frac1n \M\M^\top\right)\geqslant \left(1-\frac{(C_0+1)\sqrt m}{\sqrt n}\right)^2\geqslant \frac14~\text{for}~n\geqslant [2(C_0+1)]^{\frac2{1-c_1}},
\end{align*}
with probability at least $1-2\exp(- C'm)$. Thus the proof is completed by choosing $n_1'(C_0,c_1):=[2(C_0+1)]^{\frac{2}{1-c_1}}$. 
\end{proof}

\section{Proof of Proposition \ref{prop:concentration Gammam dag Mmd}}\label{ap:concentration inequality}
We first give the following lemma whose proof is deferred to the end of this section.
\begin{lemma}\label{lem:PimTPimtoT}If $T$ is of finite rank, then we have $\lim\limits_{m\to \infty}\|\Pi_m T\Pi_m-T\| =0$.
\end{lemma}
A direct corollary of this lemma is as follows.
\begin{corollary}\label{lemma, M go to Mm}
%For any $\varepsilon>0$, one has $\|M-M_m\| <\varepsilon$ when $m$ is sufficiently large.
Under Assumptions $\ref{as:joint distribution assumption}$ and $\ref{as:Linearity condition and Coverage condition}$, we have $\lim\limits_{m\to\infty}\|M-M_m\| =0$.
\end{corollary}
\noindent We denote by $m_M(\varepsilon)$ the minimal integer $m_M$ satisfying $\|M-M_m\| \leqslant \varepsilon$ for all $m\geqslant m_M$.

Proposition $\ref{prop:concentration Gammam dag Mmd}$ is a direct corollary of the following Proposition.
\begin{proposition}
\label{prop:bound of finite estimate}
 Suppose that Assumptions $\ref{as:joint distribution assumption}$ to $\ref{assumption: rate-type condition}$ hold, then $\forall \gamma\in(0,1/2)$, there exist positive constants
 \begin{align*}
 n_1=n_1(\gamma,\sigma_0,\sigma_1,\bs K,m_M(1),c_1),\quad D_3=D_3(\|M\| ,\wt C,\bs K) 
 \end{align*}
and $C'=C'(\bs K)$
, such that when $n\geqslant n_1$, we have
\begin{equation*}
\begin{aligned}
\mb P\l \lno\widehat\Gamma_m^\dagger \widehat M_m^d-\Gamma_m^\dagger M_m\rno  \leqslant \left[\frac{C+\ln(D_1m^2n)}{D_2}\right]^{\frac52}\frac{24m^{\alpha_1+1}}{\wt C\sqrt n}+D_3\frac{m^{(2\alpha_1+1)/2}}{n^{1/2}} \r&\\
\geqslant 1-\exp(- C)-2\exp(- C'm).&
\end{aligned}
\end{equation*}
Here $D_1,D_2$ and $C$ are defined in Proposition $\ref{prop:bound hatMmd Mm}$ and $\bs K$ is the uniform upper-exponentially bounded constant of $(\sqrt{\lambda_1}w_1,\dots,\sqrt{\lambda_m}w_m)$. 
\end{proposition}
\begin{proof}
By triangle inequality, one has
\begin{align*}
&\lno\widehat{\Gamma}_m^\dagger \widehat M_m^d-\Gamma_m^\dagger M_m\rno 
=\lno\widehat\Gamma_m^\dagger \widehat M_m^d-\wh\Gamma_m^\dagger M_m+\wh\Gamma_m^\dagger M_m-\Gamma_m^\dagger M_m\rno 
\\&\qquad\leqslant\lno\Gamma_m^\dagger\rno \cdot \lno\widehat M_m^d-M_m\rno +\lno\widehat\Gamma_m^\dagger-\Gamma_m^\dagger\rno \cdot \lno M_m\rno .
\end{align*}
Thus one can bound $\lno\Gamma_m^{\dag}M_m-\widehat\Gamma_m^{\dag}\widehat M_m^d\rno $ by bound $\lno\Gamma_m^\dagger\rno $, $\lno\widehat\Gamma_m^\dagger-\Gamma_m^\dagger\rno $, $\lno\widehat M_m^d-M_m\rno $ and $\lno M_m\rno $ respectively.
\begin{itemize}
 \item\textbf{Bound of $\lno\Gamma_m^\dagger\rno $}: By Assumption $\ref{assumption: rate-type condition}$, one has 
\begin{align}\label{eq:bound Gammam dagger}
\lambda_j\geqslant \wt C j^{-\alpha_1}\Rightarrow\lno\Gamma_m^\dagger\rno =\lambda_m^{-1}\leqslant \wt{C}^{-1} m^{\alpha_1}. 
\end{align} 
 \item\textbf{Bound of $\lno\widehat\Gamma_m^\dagger-\Gamma_m^\dagger\rno $}:
 Let us define $\mc H_m:=\mathrm{span}\{\phi_1,\dots,\phi_m\}$ where $\{\phi_i\}$ is introduced in Equation $\eqref{eq:X expansion}$. It is easy to check that
$\lno\widehat\Gamma_m^\dagger-\Gamma_m^\dagger\rno =\lno(\widehat\Gamma_m^\dagger-\Gamma_m^\dagger)|_{\mc H_m}\rno $ since $\l\widehat\Gamma_m^\dagger-\Gamma_m^\dagger\r{\bs{\beta}}=0$ for any ${\bs{\beta}}\in\mc{H}_m^\perp$. 
Because $\l\widehat\Gamma_m^\dagger-\Gamma_m^\dagger\r|_{\mc H_m}$ can be represented by matrix $\widehat{\Sigma}^{-1}-\Sigma^{-1}$ defined in Lemma $\ref{lemma, estiamtion error of inverse sample cov}$ under orthonormal basis $\{\phi_i\}_{i=1}^m$, one can get $\lno\widehat\Gamma_m^\dagger-\Gamma_m^\dagger\rno =\|\widehat{\Sigma}^{-1}-\Sigma^{-1}\|$.
Similarly, one can also get $\lno\Gamma_m^\dagger\rno =\lno\Sigma^{-1}\rno=\lambda_{\min}^{-1}(\Sigma)$. Thus, by Lemma $\ref{lemma, estiamtion error of inverse sample cov}$ one has
\[\mb P\l\lno\widehat\Gamma_m^\dagger-\Gamma_m^\dagger\rno \leqslant {4(C_0+2)^2}\lno\Gamma^{\dag}_m\rno \sqrt{\frac mn}\r\geqslant 1-2\exp(- C'm)\]
for sufficiently large $n\geqslant n_1'(\bs K,c_1)$
. Combing with $\lno\Gamma_m^\dagger\rno \hspace{-1mm}\leqslant \wt{C}^{-1} m^{\alpha_1}$, one can get
\begin{equation}\label{eq: distance hat gamma m dagger hat gamma m dagger}
\mb P\l\lno\widehat\Gamma_m^\dagger-\Gamma_m^\dagger\rno \leqslant \frac{4(C_0+2)^2m^{(2\alpha_1+1)/2}}{\wt Cn^{1/2}}\r\geqslant 1-2\exp(- C'm)
\end{equation}
for sufficiently large $n\geqslant n_1'(\bs K,c_1)$.
 \item\textbf{Bound of $\lno\widehat M_m^d-M_m\rno $}:
 See Proposition $\ref{prop:bound hatMmd Mm}$.
 \item \textbf{Bound of $\lno M_m\rno $}: By Corollary $\ref{lemma, M go to Mm}$, $\|M-M_m\| \leqslant 1$ for sufficiently large $m\geqslant m_M(1)$. Then by triangle inequality, one can get
\[\|M_m\| -\|M\| \leqslant \|M-M_m\| \leqslant 1.\]
Hence,
\begin{align}\label{eq:Mm leq M C}
\|M_m\| \leqslant \|M\| +1.
\end{align}
\end{itemize}
Combing \eqref{eq:bound Gammam dagger}, \eqref{eq: distance hat gamma m dagger hat gamma m dagger}, Proposition $\ref{prop:bound hatMmd Mm}$ with \eqref{eq:Mm leq M C}, one can choose $D_3$ and $n_1$ to be $\frac{4(C_0+2)^2(\|M\| +1)}{\wt C}$ and $\max\{n_0,n_1'(\bs K,c_1),m_M(1)^{1/c_1}\}$ respectively to
complete the proof where $n_0$ is defined in Proposition $\ref{prop:bound hatMmd Mm}$.
\end{proof}

\paragraph{Proof of Lemma \ref{lem:PimTPimtoT}}
\begin{proof}By the triangle inequality and compatibility of operator norm, one has
\begin{align*}
\|\Pi_m T\Pi_m-T\| &\leqslant\|\Pi_mT\Pi_m-\Pi_mT\| +\|\Pi_mT-T\| \\
&\leqslant\|(\Pi_m-I)T^*\| +\|(\Pi_m-I)T\| 
\end{align*}
where $I=\sum\limits_{i=1}^\infty\phi_i\otimes\phi_i$ for $\{\phi_i\}_{i\in\mb{Z}_{\geqslant 1}}$ defined in \eqref{eq:X expansion} being an orthonormal basis of $\mc H$. 
% Since the adjoint of $M(\Pi_m-I)$ is $(\Pi_m-I)M$, we have
% \begin{align*}&\|M(\Pi_m-I)\| +\|(\Pi_m-I)M\| \\
% =&
% \end{align*}

Since $T$ is of finite rank, let us assume that $\{e_i\}_{i=1}^k$ is an orthonormal basis of $\mathrm{Im}(T)$ where $k=\mr{rank}(T)$. For any ${\bs{\beta}}\in\mathcal{H}$ such that $\|{\bs{\beta}}\|=1$, one has $\|T{\bs{\beta}}\|\leqslant\|T\| \|{\bs{\beta}}\|=\|T\| $, so one can assume that $T{\bs{\beta}}\in\mathrm{Im}(T)$ admits the following expansion under basis $\{e_i\}_{i=1}^k$:
\[T{\bs{\beta}}=\sum_{i=1}^k b_ie_i,\quad \sum_{i=1}^k b^2_i\leqslant\|T\| ^2<\infty.\]
Thus
\[\|(I-\Pi_m)T{\bs{\beta}}\|=\left\|\sum_{i=1}^k(I-\Pi_m) b_ie_i\right\|\leqslant\sum_{i=1}^k |b_i|\cdot\|(I-\Pi_m) e_i\|.\]
Clearly, $\|(\Pi_m-I)\alpha\|~(\forall\alpha\in\H)$ tends to $0$ as $m\to\infty$ since 
\[(I-\Pi_m)\alpha=\left(\sum_{i={m+1}}^\infty\phi_i\otimes\phi_i\right)\left(\sum\limits_{i=1}^\infty c_i\phi_i\right)=\sum_{i=m+1}^\infty c_i\phi_i\xrightarrow{m\to\infty} 0\]
where we have assumed that $\alpha=\sum\limits_{i=1}^\infty c_i\phi_i$ .

Thus $\forall\varepsilon>0$, there exists some $N_i>0$ such that $\forall m> N_i$ one has $\|(\Pi_m-I)e_i\|<\varepsilon$, $(\forall i=1,...,k)$. Let $N=\max\{N_1,\cdots,N_k\}$, then $\forall m>N$ one has
\[\|(I-\Pi_m)T{\bs{\beta}}\|\leqslant\sum_{i=1}^k |b_i|\cdot\|(I-\Pi_m) e_i\|\leqslant\sum_{i=1}^k |b_i|\varepsilon\leqslant k\varepsilon\|T\| ,\]
which means that $\forall m>N$, one has
\begin{align*}
\|(\Pi_m-I)T\| &=\sup_{\|{\bs{\beta}}\|=1}\|(\Pi_m-I)T{\bs{\beta}}\|\leqslant k\varepsilon\|T\| . 
\end{align*}
Thus $\lim\limits_{m\to\infty}\|(\Pi_m-I)T\| =0$. 

Similarly, one can also get $\lim\limits_{m\to\infty}\|(\Pi_m-I)T^*\| =0$. Then the proof of Lemma $\ref{lem:PimTPimtoT}$ is completed.
\end{proof}
% \section{Sin Theta Theorem}\label{ap:Sin Theta theorem}
% \subsection{Sin Theta Theorem for Self-adjoint Operators}
% \begin{lemma}[Proposition 2.3 in \cite{seelmann2014notes}]\label{lemma, sin theta of infinite dimension operator}
% Let $B$ be a self-adjoint operator on a separable Hilbert space $\widetilde{\mathcal{H}}$, and let ${V}\in\mathcal{L}(\widetilde{\mathcal{H}})$ be another self-adjoint operator where $\mathcal{L}\l\widetilde{\mc H}\r$ stands for the space of bounded linear operators from a Hilbert space $\widetilde{\mc H}$ to $\widetilde{\mc H}$.
% Write \[\mathrm{spec}( B)=\sigma\cup\Sigma\quad\text{and}\quad \mathrm{spec}( B+ V)=\omega\cup\Omega
% \]
% with $\sigma\cap\Sigma=\varnothing=\omega\cap\Omega$, and suppose that there is $\widehat d>0$ such that
% \[\mathrm{dist}(\sigma,\Omega)\geqslant \widehat d\quad\text{and}\quad\mathrm{dist}(\Sigma,\omega)\geqslant \wh d\]
% where $\mathrm dist(\sigma,\Sigma):=\min\{|a-b|:a\in\sigma,b\in\Omega\}$.
% Then, the operator angle $\Theta=\Theta(P_{ B}(\sigma),P_{ B+ V}(\omega))$ satisfies the bound
% \[\|\sin\Theta\|:=\|P_{{B}}(\sigma)-P_{{B}+{V}}(\omega)\| \leqslant\frac\pi2\frac{\| V\| }{\wh d}\]
% where $P_{ B}(\sigma)$ denotes the spectral projection for $ B$ associated with $\sigma$, i.e., 
% \[P_{B}(\sigma):=\frac{1}{2\pi\mathrm{i}}\oint_{\gamma}\frac{\mathrm{d}z}{z-B},\]
% where $\gamma$ is a contour on $\mathbb{C}$ that encloses $\sigma$ but no other elements of $\mathrm{spec}( B)$.
% \end{lemma}
% \begin{remark}
% We note that, 
% if further $ B$ is compact, 
% the spectral projection coincide with projection operator onto the closure of the space spanned by the eigenfunctions associated with the eigenvalues in $\sigma$.

% If $B$ is compact, by the spectral decomposition theorem one has
% \[B=\sum_{i=1}^\infty\mu_ie_i\otimes e_i\quad\text{and}\quad(z- B)^{-1}=\sum_{i=1}^\infty(z-\mu_i)^{-1}e_i\otimes e_i,\]
% where $\mr{spec}(B):=\{\mu_i\}_{i=1}^\infty$ satisfies $|\mu_i|\xrightarrow{i\to\infty} 0$.
% Then $\forall v\in \mathcal{H}$,
% \begin{align*}P_{B}(\sigma)v&=\frac{1}{2\pi\mathrm{i}}\oint_{\gamma}({z-B})^{-1}v~{\mathrm{d}z}=\frac{1}{2\pi\mathrm{i}}\oint_{\gamma}\sum_{i=1}^\infty(z-\mu_i)^{- 1}\langle e_i,v\rangle e_i~{\mathrm{d}z}\\
% &=\sum_{i=1}^\infty\left[\left(\frac{1}{2\pi\mathrm{i}}\oint_{\gamma}(z-\mu_i)^{-1}~{\mathrm{d}z}\right)\langle e_i,v\rangle e_i\right]=\sum_{i\in\{i:\mu_i\in\sigma\}}\langle e_i,v\rangle e_i.
% \end{align*}
% Especially, if $\sigma=\mr{spec}(B)\backslash\{0\}$, then $P_{B}(\sigma)$ is the projection operator onto the $\overline{\mathrm{Im}}(B)$.
% \end{remark}
% Splitting eigenvalues into nonzero part and zero part yields the following useful corollary.
% \begin{corollary}\label{cor: sin theta self adjoint}
% Let $B$ and $B'$ be two positive semi-definite {and compact} operators with finite rank on a separable Hilbert space $\widetilde{\mathcal{H}}$. Let $\lambda_{\min}^+( B)$ and $\lambda_{\min}^+(B')$ be the infimum of the positive eigenvalues of ${B}$ and ${B}'$ respectively. Then we have
% \[\left\|P_{ B}-P_{ B'}\right\| \leqslant\frac\pi2\frac{\| B- B'\| }{\min\{\lambda_{\min}^+( B),\lambda_{\min}^+( B')\}}.\]
% \end{corollary}
% \subsection{Sin Theta Theorem for General Operators}
% When ${B}$ and ${V}$ in Lemma $\ref{lemma, sin theta of infinite dimension operator}$ are not self-adjoint, we use the symmetrization trick, which mainly depends on the following Lemma.
% \begin{lemma}\label{lem:projection equality}
% $P_A=P_{AA^*}$ for any bounded linear operator $A$ from a Hilbert space $\wt\H$ to $\wt\H$. Especially, $P_A=P_{AA^{\top}}$ for any matrix $A$.
% \end{lemma}
% \begin{proof}This lemma is a direct corollary of Lemma $\ref{lem: colPBP equal colPB operator}$.
% \end{proof}

% Then we have the following Sin Theta theorem for general operator.
% \begin{lemma}\label{lemma, sin theta of nonadjoint operator}
% Let $ B,B'\in\mathcal{L}(\widetilde{\mathcal{H}})$ be two compact operators (not necessarily self-adjoint) with finite rank.
% Then we have
% \begin{align*}
% \left\|P_{ B}-P_{ B'}\right\| &\leqslant\frac\pi2\frac{\| B B^*- B'B'^*\| }{\min\lb\sigma_{\min}^+( B)^2,\sigma_{\min}^+(B')^2\rb}\\
% &\leqslant \frac\pi2\frac{\| B- B'\| ^2+2\| B- B'\| \| B'\| }{\min\lb\sigma_{\min}^+( B)^2,\sigma_{\min}^+( B')^2\rb}.
% \end{align*}
% \end{lemma}
% \begin{proof}By Lemma $\ref{lem:projection equality}$, one can get $\left\|P_{ B}-P_{ B'}\right\| =\left\|P_{ B B^*}-P_{ B' B'^*}\right\| $.
% Since $ BB^*, B'B'^*$ are both self-adjoint and compact, by Lemma $\ref{cor: sin theta self adjoint}$, one has
% \begin{align*}
% \left\|P_{ B B^*}-P_{ B' B'^*}\right\| \leqslant \frac{\pi}{2}\frac{\| B B^*- B' B'^*\| }{\min\lb\lambda_{\min}^+\l B B^*\r,\lambda_{\min}^+\l B' B'^*\r\rb}.
% \end{align*}
% Then the proof is completed in view of the following inequality:
% % of $\| B B^*- B' B'^*\| $:
% \begin{align}
% \lno B B^*- B' B'^*\rno &= \|( B- B')( B- B')^*\hspace{-0.5mm}+\hspace{-0.5mm}( B-B')(B')^*\hspace{-0.5mm}+\hspace{-0.5mm} B'( B- B')^*\| \nonumber\\
% &\leqslant \| B- B'\| ^2+2\| B- B'\| \| B'\| . \label{eq:sy ineq}
% \end{align}
% \end{proof}


\section{Sin Theta Theorem}\label{ap:Sin Theta theorem}
\subsection{Sin Theta Theorem for Self-adjoint Operators}
\begin{lemma}[Proposition 2.3 in \cite{seelmann2014notes}]\label{lemma, sin theta of infinite dimension operator}
Let $B$ be a self-adjoint operator on a separable Hilbert space $\widetilde{\mathcal{H}}$, and let ${V}\in\mathcal{L}(\widetilde{\mathcal{H}})$ be another self-adjoint operator where $\mathcal{L}\left(\widetilde{\mc H}\right)$ stands for the space of bounded linear operators from a Hilbert space $\widetilde{\mc H}$ to $\widetilde{\mc H}$.
Write the spectra of $B$ and $B+V$ as \[\mathrm{spec}( B)=\sigma\cup\Sigma\quad\text{and}\quad \mathrm{spec}( B+ V)=\omega\cup\Omega
\]
with $\sigma\cap\Sigma=\varnothing=\omega\cap\Omega$, and suppose that there is $\widehat d>0$ such that
\[\mathrm{dist}(\sigma,\Omega)\geqslant \widehat d\quad\text{and}\quad\mathrm{dist}(\Sigma,\omega)\geqslant \wh d\]
where $\mathrm dist(\sigma,\Sigma):=\min\{|a-b|:a\in\sigma,b\in\Omega\}$.
Then it holds that
\[\|P_{{B}}(\sigma)-P_{{B}+{V}}(\omega)\| \leqslant\frac\pi2\frac{\| V\| }{\wh d}\]
where $P_{ B}(\sigma)$ denotes the spectral projection for $ B$ associated with $\sigma$, i.e., 
\[P_{B}(\sigma):=\frac{1}{2\pi\mathrm{i}}\oint_{\gamma}\frac{\mathrm{d}z}{z-B},\]
where $\gamma$ is a contour on $\mathbb{C}$ that encloses $\sigma$ but no other elements of $\mathrm{spec}( B)$.
\end{lemma}
\begin{remark}
We note that, 
if further $ B$ is compact, 
the spectral projection coincide with projection operator onto the closure of the space spanned by the eigenfunctions associated with the eigenvalues in $\sigma$. 
% For more details, see, e.g., Remark 1 in \cite{chen2023optimality}.

Specifically, if $B$ is compact, by the spectral decomposition theorem one has
\[B=\sum_{i=1}^\infty\mu_ie_i\otimes e_i\quad\text{and}\quad(z- B)^{-1}=\sum_{i=1}^\infty(z-\mu_i)^{-1}e_i\otimes e_i,\]
where $\mr{spec}(B):=\{\mu_i\}_{i=1}^\infty$ satisfies $|\mu_i|\xrightarrow{i\to\infty} 0$.
Then $\forall v\in \mathcal{H}$, it holds that
\begin{align*}P_{B}(\sigma)v&=\frac{1}{2\pi\mathrm{i}}\oint_{\gamma}({z-B})^{-1}v~{\mathrm{d}z}=\frac{1}{2\pi\mathrm{i}}\oint_{\gamma}\sum_{i=1}^\infty(z-\mu_i)^{- 1}\langle e_i,v\rangle e_i~{\mathrm{d}z}\\
&=\sum_{i=1}^\infty\left[\left(\frac{1}{2\pi\mathrm{i}}\oint_{\gamma}(z-\mu_i)^{-1}~{\mathrm{d}z}\right)\langle e_i,v\rangle e_i\right]=\sum_{i\in\{i:\mu_i\in\sigma\}}\langle e_i,v\rangle e_i.
\end{align*}
In particular, if $\sigma=\mr{spec}(B)\backslash\{0\}$, then $P_{B}(\sigma)$ is the projection operator onto the $\overline{\mathrm{Im}}(B)$.
\end{remark}

Splitting eigenvalues into nonzero part and zero part yields the following useful corollary.
\begin{corollary}\label{cor: sin theta self adjoint}
Let $B$ and $B'$ be two positive semi-definite {and compact} operators with finite rank on a separable Hilbert space $\widetilde{\mathcal{H}}$. Let $\lambda_{\min}^+( B)$ and $\lambda_{\min}^+(B')$ be the infimum of the positive eigenvalues of ${B}$ and ${B}'$ respectively. Then we have
\[\left\|P_{ B}-P_{ B'}\right\| \leqslant\frac\pi2\frac{\| B- B'\| }{\min\{\lambda_{\min}^+( B),\lambda_{\min}^+( B')\}}.\]
\end{corollary}
\subsection{Sin Theta Theorem for General Operators}
When ${B}$ and ${V}$ in Lemma $\ref{lemma, sin theta of infinite dimension operator}$ are not self-adjoint, we use the symmetrization trick, which mainly depends on the following Lemma.
\begin{lemma}\label{lem:projection equality}
$P_A=P_{AA^*}$ for any bounded linear operator $A$ from a Hilbert space $\wt\H$ to $\wt\H$. Especially, $P_A=P_{AA^{\top}}$ for any matrix $A$.
\end{lemma}
\begin{proof}First we show that the null space of  $A^*$ is the same as the null space of $AA^*$.
On the one hand, 
\[x\in\mathrm{null}(A^*)\Longrightarrow
A^*x=0\Longrightarrow AA^*x=0\Longrightarrow x\in\mathrm{null}(AA^*); 
\]
One the other hand,
\begin{align*}x\in\mathrm{null}(AA^*)&\Longrightarrow
AA^*x=0\Longrightarrow \langle x,AA^*x\rangle=\langle A^*x,A^*x\rangle=\|A^*x\|^2=0\\
&\Longrightarrow A^*x=0\Longrightarrow x\in\mathrm{null}(A^*).
\end{align*}
Hence, we have $\mathrm{null}(A^*)=\mathrm{null}(AA^*)$. Take the orthogonal complement of the both sides of this equality, we can get
\[\mathrm{null}(A^*)^{\perp}=\mathrm{null}(AA^*)^{\perp}\Longrightarrow {\mathrm{Im}(A)}={\mathrm{Im}(AA^*)}.\]
\end{proof}
Then we have the following Sin Theta theorem for general operator.
\begin{lemma}\label{lemma, sin theta of nonadjoint operator}
Let $ B,B'\in\mathcal{L}(\widetilde{\mathcal{H}})$ be two compact operators (not necessarily self-adjoint) with finite rank.
Then we have
\begin{align*}
\left\|P_{ B}-P_{ B'}\right\| &\leqslant\frac\pi2\frac{\| B B^*- B'B'^*\| }{\min\left\{\sigma_{\min}^+( B)^2,\sigma_{\min}^+(B')^2\right\}}\\
&\leqslant \frac\pi2\frac{\| B- B'\| ^2+2\| B- B'\| \| B'\| }{\min\left\{\sigma_{\min}^+( B)^2,\sigma_{\min}^+( B')^2\right\}}.
\end{align*}
\end{lemma}
\begin{proof}By Lemma $\ref{lem:projection equality}$, one can get $\left\|P_{ B}-P_{ B'}\right\| =\left\|P_{ B B^*}-P_{ B' B'^*}\right\| $.
Since $ BB^*, B'B'^*$ are both self-adjoint and compact, by Lemma $\ref{cor: sin theta self adjoint}$, one has
\begin{align*}
\left\|P_{ B B^*}-P_{ B' B'^*}\right\| \leqslant \frac{\pi}{2}\frac{\| B B^*- B' B'^*\| }{\min\left\{\lambda_{\min}^+\left( B B^*\right),\lambda_{\min}^+\left( B' B'^*\right)\right\}}.
\end{align*}
Then the proof is completed in view of the following inequality:
% of $\| B B^*- B' B'^*\| $:
\begin{align}
\left\| B B^*- B' B'^*\right\| &= \|( B- B')( B- B')^*\hspace{-0.5mm}+\hspace{-0.5mm}( B-B')(B')^*\hspace{-0.5mm}+\hspace{-0.5mm} B'( B- B')^*\| \nonumber\\
&\leqslant \| B- B'\| ^2+2\| B- B'\| \| B'\| . \label{eq:sy ineq}
\end{align}
\end{proof}


\section{Proof of Theorem \ref{theorem, total convergence rate}}
Thanks to the triangle inequality, one can bound the subspace estimation error by bounding the error term (i): $\mathbf{ Loss}_1:=\left\|P_{\mc S_{\Y|\X}^{(m)}}-P_{ \widehat {\mc S}_{\Y|\X}^{(m)}}\right\| $ and error term (ii): $\mathbf{ Loss}_2:= \left\|P_{\mathcal S_{\Y|\boldsymbol{X}}}-P_{\mathcal S_{\Y|\boldsymbol{X}}^{(m)}}\right\| $ respectively.
\subsection{Upper bound of error term (i)}
We first give the following lemmas, whose proofs are all deferred to the end of this section.
\begin{lemma}\label{lem:Gammam dagger Mm uniformly bounded}
% Under Assumptions $\ref{as:joint distribution assumption}$ and $\ref{as:Linearity condition and Coverage condition}$,
% $\{\|\Gamma_m^\dagger M_m\| \}_{m=1}^\infty$ is uniformly (about $m$) bounded by $\|\Gamma^{-1}M\| $.
Under Assumptions $\ref{as:joint distribution assumption}$ and $\ref{as:Linearity condition and Coverage condition}$, it holds that $\|\Gamma_m^\dagger M_m\| \leq \|\Gamma^{-1}M\| (\forall m).$
% \begin{align*}
% \|\Gamma_m^\dagger M_m\| \leq \|\Gamma^{-1}M\| \quad\forall m.
% \end{align*}
% $\{\|\Gamma_m^\dagger M_m\| \}_{m=1}^\infty$ is uniformly (about $m$) bounded by $\|\Gamma^{-1}M\| $.
\end{lemma}
\begin{lemma}\label{lem: Gamma inverse M to Gammam dagger Mm}Under Assumptions $\ref{as:joint distribution assumption}$ and $\ref{as:Linearity condition and Coverage condition}$, we have \[\lim\limits_{m\to\infty}\lno\Gamma^{-1}M-\Gamma_m^\dagger M_m\rno =0.\]
\end{lemma}
\noindent We denote by $m_T(\varepsilon)$ the minimal integer $m_T$ satisfying $\lno\Gamma^{- 1}M-\Gamma_m^\dagger M_m\rno \hspace{-1mm}\leqslant \varepsilon$ for all $m\geqslant m_T$ and define an event 
$$\ttE:=\lb \left\|\widehat\Gamma_m^\dagger \widehat M_m^d-\Gamma_m^\dagger M_m\right\|  \leqslant\hspace{-0.5mm}\left(\tfrac{D_0+1}{D_2}\right)^{\frac52}\tfrac{24}{\wt C}n^{c_1(\alpha_1+1)+\gamma-\frac{1}{2}}+D_3n^{\frac{c_1(2\alpha_1+1)-1}{2}}\rb.$$
Then by taking $C$ to be $(D_0+1)n^{\frac{2\gamma}{5}}-\ln\l D_1m^2n \r$ in  Proposition \ref{prop:bound of finite estimate}, one has: for $n\geqslant \l\frac{D_0+1}{D_2}\r^{\frac{5}{1-2\gamma}}$,
$$\P(\ttE)\geq 1-D_1m^2n\exp\left[-(D_0+1)n^{\frac{2\gamma}{5}}\right] -2\exp(- C'm).$$
\begin{lemma}\label{lem:lower bound sigma min total}
Introducing $
\bigtriangleup :=\max\lb \frac{\sigma_d(\Gamma^{-1} M)}{2},\frac{\sigma_d(\Gamma^{-1} M)^2}{4\|\Gamma^{-1}M\| } \rb$.
Suppose that Assumptions $\ref{as:joint distribution assumption}$ to $\ref{assumption: rate-type condition}$ hold, $c_1(2\alpha_1+1)-1<0$ and $2(c_1(\alpha_1+1)+\gamma)-1<0$. Then there exists a positive constant
\begin{align*}
n_2'=n_2'\l\sigma_d(\Gamma^{-1}M),\|\Gamma^{-1}M\| , \gamma,\sigma_0,\sigma_1,\bs K,m_M(1),c_1,m_T\l \tfrac{\bigtriangleup}{2}\r,\wt C,\alpha_1\r
\end{align*}
such that when $n\geqslant n_2'$, we have
\begin{align}
\sigma_{\min}^+(\Gamma_m^{\dagger} M_m)^2\geqslant \tfrac{\sigma_d(\Gamma^{-1}M)^2}{2} \label{eq: lower bound of sigma min}. 
\end{align}
Furthermore, Conditioning on $\ttE$, we have
\begin{align}\label{eq: lower bound of sigma min hat}
&\sigma_{\min}^+(\wh\Gamma_m^{\dagger}\wh M_m^d)^2\geqslant \tfrac{\sigma_d(\Gamma^{-1}M)^2}{2}.
\end{align}
\end{lemma}
The following proposition is an upper bound of error term (i):
\begin{proposition}\label{proposition, estimation error}
Positive constants $D_1$, $D_2$ and $C'$  as in Proposition $\ref{prop:bound of finite estimate}$,
suppose that Assumptions $\ref{as:joint distribution assumption}$ to $\ref{assumption: rate-type condition}$ hold, then $\forall \gamma\in(0,1/2)$, if $c_1$ satisfies $2c_1(\alpha_1+1)+2\gamma-1<0$ and $c_1(2\alpha_1+1)-1<0$, there exists a positive constant $C_1:=C_1\l \|\Gamma^{-1}M\| ,\sigma_d(\Gamma^{-1}M) ,\wt C,\gamma,\sigma_0,\sigma_1\r$ such that
\begin{align*}
\P\l
\lno P_{\mc{S}_{\Y|\X}^{(m)}}-P_{ \widehat{\mc{S}}_{\Y|\X}^{(m)}}\rno \leqslant C_1\frac{m^{\alpha_1+1}}{n^{1/2-\gamma}}\r\geqslant1-2\exp(- C'm)&\\
- D_1m^2n\exp\l -(D_0+1)n^{\frac{2\gamma}{5}} \r&,
\end{align*}
when 
\begin{align*}
n\geqslant\max\Bigg\{ n_1,\l\tfrac{D_0+1}{D_2}\r^{\frac{5}{1-2\gamma}},\left[\tfrac{\|\Gamma^{-1}M\|  \wt C}{48}\l\tfrac{D_2}{D_0+1}\r^{\frac52}\right]^{\frac{2}{2(c_1(\alpha_1+1)+\gamma)-1}}&,\\
\l \tfrac{\|\Gamma^{-1}M\| }{2D_3}\r^{\frac{2}{c_1(2\alpha_1+1)-1}},n_2',\left[ \tfrac{D_3\wt C}{24}\l \tfrac{D_2}{D_0+1} \r^{\frac52} \right]^{\frac2{2\gamma+c_1}}&\Bigg\}
\end{align*}
where $n_2'$ is defined in Lemma $\ref{lem:lower bound sigma min total}$.
\end{proposition}
\begin{proof}
By Lemma $\ref{lemma, way of estimate truncate central subspace}$, $\eqref{def: estimator central subspace}$ and Lemma $\ref{lemma, sin theta of nonadjoint operator}$, one has
\begin{align}
&\left\|P_{\mc S_{\Y|\vX}^{(m)}}-P_{\wh{\mc{S}}_{\Y|\vX}^{(m)}}\right\| =\left\|P_{\Gamma_m^{\dagger}M_m}-P_{\wh\Gamma_m^{\dagger}\wh M_m^d}\right\| \nonumber\\
&\qquad\leqslant\frac{\pi}{2}\frac{\lno\widehat\Gamma_m^\dagger \widehat M_m^d-\Gamma_m^\dagger M_m\rno ^2+\lno\widehat\Gamma_m^\dagger \widehat M_m^d-\Gamma_m^\dagger M_m\rno \lno\Gamma_m^\dagger M_m\rno }{\min\lb\sigma_{\min}^+\l\wh\Gamma_m^\dagger \wh M_m^d\r^2,\sigma_{\min}^+\l\Gamma_m^\dagger M_m\r^2\rb}\label{eq: PS minus P hat S norm}.
% &\leqslant C_5\|\widehat\Gamma_m^\dagger \widehat M_m^d-\Gamma_m^\dagger M_m\|\\
% &=\widetilde O_{\mathbb{P}}\l\frac{m^{\alpha_1+1}}{n^{1/2}}\r,
\end{align}
% with probability at least $1-\exp(- C)-2\exp(- C'm)$.
Because of $c_1(2\alpha_1+1)-1<0$ and $2(c_1(\alpha_1+1)+\gamma)-1<0$, it is easy to check that when
\[n\geqslant\max\lb\left[\tfrac{\|\Gamma^{-1}M\|  \wt C}{48}\l\tfrac{D_2}{D_0+1}\r^{\frac52}\right]^{\frac{2}{2(c_1(\alpha_1+1)+\gamma)-1}},\l \tfrac{\|\Gamma^{-1}M\| }{2D_3}\r^{\frac{2}{c_1(2\alpha_1+1)-1}}\rb,\]
both $\l\tfrac{D_0+1}{D_2}\r^{\frac52}\tfrac{24}{\wt C}n^{c_1(\alpha_1+1)+\gamma-\frac{1}{2}}$ and $D_3n^{\frac{c_1(2\alpha_1+1)-1}{2}}$ are less than or equal to $\frac{\|\Gamma^{-1}M\| }{2}$. Thus, on the event $\ttE$,
\begin{align}\label{eq: high prob upper bound is Gamma minus 1 M}
\lno\widehat\Gamma_m^\dagger \widehat M_m^d-\Gamma_m^\dagger M_m\rno \leqslant \lno\Gamma^{-1}M\rno .
\end{align}
By Lemma $\ref{lem:Gammam dagger Mm uniformly bounded}$, inserting \eqref{eq: high prob upper bound is Gamma minus 1 M} into \eqref{eq: PS minus P hat S norm} leads to
$$
\lno P_{\mc{S}_{\Y|\X}^{(m)}}-P_{ \widehat{\mc{S}}_{\Y|\X}^{(m)}}\rno
\leqslant \frac{\pi\lno\widehat\Gamma_m^\dagger \widehat M_m^d-\Gamma_m^\dagger M_m\rno \lno\Gamma^{-1}M\rno }{\min\lb\sigma_{\min}^+\l\wh\Gamma_m^\dagger \wh M_m^d\r^2,\sigma_{\min}^+\l\Gamma_m^\dagger M_m\r^2\rb},
$$
on the event $\ttE$.
Furthermore, when $n\geqslant \left[ \frac{D_3\wt C}{24}\l \frac{D_2}{D_0+1} \r^{\frac52} \right]^{\frac2{2\gamma+c_1}}$ and $n\geq n_2'$, one can get
$\l \tfrac{D_0+1}{D_2}\r^{\frac52}\tfrac{24m^{\alpha_1+1}}{\wt C n^{1/2-\gamma}}$ is greater than or equal to $D_3\tfrac{m^{(2\alpha_1+1)/2}}{n^{1/2}}$
and then on the event $\ttE$,
\begin{align*}
\lno P_{\mc{S}_{\Y|\X}^{(m)}}-P_{ \widehat{\mc{S}}_{\Y|\X}^{(m)}}\rno \leqslant \tfrac{96\pi\|\Gamma^{-1}M\| }{\sigma_d(\Gamma^{-1}M)^2}\l \tfrac{D_0+1}{D_2}\r^{\frac52}\tfrac{m^{\alpha_1+1}}{\wt C n^{1/2-\gamma}}.
\end{align*}
 by
Lemma $\ref{lem:lower bound sigma min total}$.
Then choosing $C_1=\tfrac{96\pi\|\Gamma^{-1}M\| }{\wt C\sigma_d(\Gamma^{-1}M)^2}\l \tfrac{D_0+1}{D_2}\r^{\frac52}$ can complete the proof.
\end{proof}



\paragraph{Proof of Lemma \ref{lem:Gammam dagger Mm uniformly bounded}}
\begin{proof}
First, it is easy to check that:
\begin{align}
\Gamma^\dag_m=\Pi_m\Gamma^{-1}\Pi_m=\Pi_m\Gamma^{-1}=\Gamma^{-1}\Pi_m=\sum\limits_{i=1}^m\lambda_i^{-1}\phi_i\otimes\phi_i.\label{eq: Gamma m dag def}
\end{align}
According to \eqref{eq: Gamma m dag def} and $M_m=\Pi_mM\Pi_m$, it is easy to check that $\Gamma_m^\dagger M_m=\Pi_m \Gamma^{- 1}M\Pi_m$. Then by the compatibility of operator norm, one can get
\begin{align*}
\lno\Gamma_m^\dagger M_m\rno =\lno\Pi_m \Gamma^{-1}M\Pi_m\rno \leqslant \lno\Pi_m\rno  \lno\Gamma^{-1}M\rno \lno\Pi_m\rno =\lno\Gamma^{-1}M\rno .
\end{align*}
Note that $\Gamma^{-1}M$ is bounded since $\Gamma^{-1}M$ is of finite rank by Corollary $\ref{corollary, MDDO and central subspace}$. Thus the proof is completed. 
\end{proof}


\paragraph{Proof of Lemma \ref{lem: Gamma inverse M to Gammam dagger Mm}}
\begin{proof}
It is easy to check that
$\Gamma_m^\dagger M_m=\Pi_m\Gamma^{-1}M\Pi_m$ and $\Gamma^{-1}M$ is of finite rank by Corollary $\ref{corollary, MDDO and central subspace}$.
Thus the proof is completed by Lemma $\ref{lem:PimTPimtoT}$.
\end{proof}
\paragraph{Proof of Lemma \ref{lem:lower bound sigma min total}}
\begin{proof}
We first prove \eqref{eq: lower bound of sigma min}.
By Corollary $\ref{corollary, MDDO and central subspace}$ and Lemma $\ref{lem:projection equality}$, one has $\rank(\Gamma^{- 1}M)=\rank\l\Gamma^{- 1}M(\Gamma^{- 1}M)^*\r=d$. Thus
\begin{align*}
\sigma_{\min}^+(\Gamma^{-1}M)^2=\lambda_{\min}^+\l\Gamma^{-1}M(\Gamma^{-1}M)^*\r=\lambda_d\l \Gamma^{-1}M(\Gamma^{-1}M)^*\r. 
\end{align*}
 It is easy to see $\rank(\Gamma_m^\dagger M_m)=\rank\l \Gamma_m^\dagger M_m(\Gamma_m^\dagger M_m)^*\r\leqslant d$ by $\Gamma_m^\dagger M_m=\Pi_m \Gamma^{-1} M \Pi_m$ and Lemma $\ref{lem:projection equality}$, thus one can assume that 
 \begin{align*}
\sigma_{\min}^+(\Gamma^\dagger_m M_m)^2=\lambda_{\min}^+\l\Gamma_m^\dagger M_m(\Gamma_m^\dagger M_m)^*\r=\lambda_j\l \Gamma_m^\dagger M_m(\Gamma_m^\dagger M_m)^*\r
\end{align*}
for some $j\leqslant d$.
By Corollary $\ref{coro:wely ineq operator}$, $\eqref{eq:sy ineq}$ and
% (Notice that $M_m$ and $M$ are both compact and self-adjoint)
Lemma $\ref{lem: Gamma inverse M to Gammam dagger Mm}$
%and Lemma \ref{lem:Gammam dagger Mm uniformly bounded}
, one has
\begin{align*}
&\left|\sigma_{\min}^+(\Gamma^\dagger_m M_m)^2\hspace{-0.5mm}-\hspace{-0.5mm}\sigma_j(\Gamma^{-1} M)^2\right|\hspace{-0.5mm}=\hspace{-0.5mm}\left|\lambda_{j}\hspace{-1mm}\l\Gamma^\dagger_m M_m(\Gamma^\dagger_m M_m)^{*}\hspace{-0.5mm}\r\hspace{-0.5mm}-\hspace{-0.5mm}\lambda_j\hspace{-1mm}\l \Gamma^{-1} M(\Gamma^{-1} M)^*\hspace{-0.5mm}\r\right|\\
&\qquad\leqslant
\|\Gamma^{-1} M(\Gamma^{-1} M)^*- \Gamma_m^\dagger M_m(\Gamma_m^\dagger M_m)^*\| \\
&\qquad\leqslant \|\Gamma^{-1} M- \Gamma_m^\dagger M_m\| ^2+
\|\Gamma^{-1} M- \Gamma_m^\dagger M_m\| \cdot\|\Gamma^{-1} M\| \xrightarrow{m\to\infty} 0. 
% &\leqslant\|\Gamma^{-1} M- \Gamma_m^\dagger M_m\|\cdot3\|\Gamma^{-1} M\|
\end{align*}
Thus for 
$
n\geqslant m_T(\bigtriangleup)^{\frac1{c_1}}=m_T\l\max\lb\frac{\sigma_d(\Gamma^{-1} M)}{2},\frac{\sigma_d(\Gamma^{-1} M)^2}{4\|\Gamma^{-1}M\| }\rb\r^{\frac1{c_1}}, 
$
one has $\|\Gamma^{-1} M- \Gamma_m^\dagger M_m\| ^2$ and $\|\Gamma^{-1} M- \Gamma_m^\dagger M_m\| \cdot\|\Gamma^{-1} M\| $ are both less than or equal to $\frac{1}{4}\sigma_d(\Gamma^{-1} M)^2$. Hence one can get
$\left|\sigma_{\min}^+(\Gamma^\dagger_m M_m)^2-\sigma_j(\Gamma^{-1} M)^2\right|\leqslant\frac{1}{2}\sigma_d(\Gamma^{-1} M)^2$
% \begin{align*}\label{eq:sigma min Mm}
% \left|\sigma_{\min}^+(\Gamma^\dagger_m M_m)^2-\sigma_j(\Gamma^{-1} M)^2\right|\leqslant\frac{1}{2}\sigma_d(\Gamma^{-1} M)^2
% \|
% \lambda_j\l \Gamma_m^\dagger M_m\l\Gamma_m^\dagger M_m\r^*\r\geqslant \lambda_j\l \Gamma^{-1} M\l\Gamma^{-1} M\r^*\r-\frac{\lambda_d\l \Gamma^{-1} M\l\Gamma^{-1} M\r^*\r}{2}
% \geqslant\frac{\lambda_d\l \Gamma^{-1} M\l\Gamma^{-1} M\r^*\r}{2}. 
% \end{align*}
and
\begin{equation}
\sigma_{\min}^+(\Gamma^\dagger M_m)^2\geqslant \sigma_j(\Gamma^{-1} M)^2-\frac{1}{2}\sigma_d(\Gamma^{-1} M)^2\geqslant\frac{1}{2}\sigma_d(\Gamma^{-1} M)^2
\end{equation}
for sufficiently large $n$. This completes the proof of \eqref{eq: lower bound of sigma min}.





Next we prove $\eqref{eq: lower bound of sigma min hat}$. Combining Proposition $\ref{prop:bound of finite estimate}$ with Lemma $\ref{lem: Gamma inverse M to Gammam dagger Mm}$ leads to that on the event $\ttE$, 
$$
\lno\wh \Gamma_m^\dag\wh M^d_m- \Gamma^{-1}M\rno \leqslant\ve+\l\tfrac{D_0+1}{D_2}\r^{\frac52}\tfrac{24}{\wt C}n^{c_1(\alpha_1+1)+\gamma-\frac{1}{2}}+D_3n^{\frac{c_1(2\alpha_1+1)-1}{2}}
$$
for  $n\geqslant \max\{n_1,m_T( \ve)^{1/c_1}\}$.
Assuming that $c_1(2\alpha_1+1)-1<0$ and $2(c_1(\alpha_1+1)+\gamma)-1<0$, it is easy to check that when $$n\geqslant\max\lb\left[\frac{\bigtriangleup \wt C}{96}\l\frac{D_2}{D_0+1}\r^{\frac52}\right]^{\frac{2}{2(c_1(\alpha_1+1)+\gamma)-1}},\l \frac{\bigtriangleup}{4D_3}\r^{\frac{2}{c_1(2\alpha_1+1)-1}}\rb$$, both $\l\tfrac{D_0+1}{D_2}\r^{\frac52}\tfrac{24}{\wt C}n^{c_1(\alpha_1+1)+\gamma-\frac{1}{2}}$ and $D_3n^{\frac{c_1(2\alpha_1+1)-1}{2}}$ are less than or equal to $\frac{\bigtriangleup}{4}$. Letting $\varepsilon=\frac12\bigtriangleup$, one can get on the event $\ttE$,
when
\begin{align*}
n&\geqslant n_2'=n_2'\hspace{-0.5mm}\l\hspace{-0.5mm}\sigma_d(\Gamma^{-1}M),\|\Gamma^{-1}M\| , \gamma,\sigma_0,\sigma_1,\bs K,m_M(1),c_1,m_T\l \tfrac{\bigtriangleup}{2}\r,\wt C,\alpha_1\hspace{-0.5mm}\r\\
&:=\max\bigg\{ n_1,m_T\l \tfrac{\bigtriangleup}{2}\r^{1/c_1}, \left[\tfrac{\bigtriangleup \wt C}{96}\l\tfrac{D_2}{D_0+1}\r^{\frac52}\right]^{\frac{2}{2(c_1(\alpha_1+1)+\gamma)-1}},\l \tfrac{\bigtriangleup}{4D_3}\r^{\frac{2}{c_1(2\alpha_1+1)-1}}\bigg\},
\end{align*}
one has $\lno\wh \Gamma_m^\dag\wh M^d_m- \Gamma^{-1}M\rno \leqslant\bigtriangleup$ and further
$\sigma_{\min}^+(\wh\Gamma^\dagger \wh M^d_m)^2\hspace{-1mm}\geqslant\hspace{-1mm} \tfrac{\sigma_d(\Gamma^{-1} M)^2}{2}$ by the same argument as the proof of \eqref{eq: lower bound of sigma min}.
 This completes the proof of \eqref{eq: lower bound of sigma min hat}.
Considering that $m_T(\bigtriangleup)\leqslant m_{T}\l\frac\bigtriangleup2\r$, one can also get $\eqref{eq: lower bound of sigma min}$ when $n\geqslant n_2'$. Thus the proof is completed.
\end{proof}
\subsection{Upper bound of error term (ii)}\label{ap, subs, truncation error}
\begin{proposition}\label{proposition, truncation error}
Under Assumption $\ref{assumption: rate-type condition}$, there exists a positive constant $C_2:=C_2\l d,\wt C,\lambda_d(\mc{B}),\alpha_2\r$ where $\mc{B}:=\sum\limits_{i=1}^d {\bs{\beta}}_i\otimes{\bs{\beta}}_i$ for ${\bs{\beta}}_i$ defined in \eqref{def: central subspace}, such that when $n\geqslant \l \frac{\lambda_d({\mc{B}})}{4d\wt C^2}\sqrt{\frac{2\alpha_2-1}{\zeta(2\alpha_2)}}\r^{\frac{2}{c_1(1-2\alpha_2)}}$, we have
\begin{equation}\label{equation, truncation error}
 \left\|P_{\mathcal S_{\Y|\boldsymbol{X}}}-P_{\mathcal S_{\Y|\boldsymbol{X}}^{(m)}}\right\| \leqslant C_2m^{-\frac{2\alpha_2-1}{2}},
\end{equation}
where $\zeta(\cdot)$ is Riemann $\zeta$ function.
% \begin{equation}
% \|P_{\mathcal S_{Y|\boldsymbol{X}}}-P_{\mathcal S_{Y|\boldsymbol{X}}^{(m)}}\|\leqslant O_{\mathbb{P}}(dn^{-(\alpha_2-1)/(2\alpha_1+\alpha_2)}) 
% \end{equation}
\end{proposition}
\begin{proof}
Let ${\mc{B}^{(m)}}:=\sum\limits_{i=1}^d {\bs{\beta}}_i^{(m)}\otimes{\bs{\beta}}_i^{(m)}$ for ${\bs{\beta}}_i^{(m)}$ defined in \eqref{def: truncated central subspace}.
Combing with Equation $\eqref{def: central subspace}$, it is easy to check that $\left\|P_{\mathcal S_{\Y|\boldsymbol{X}}}-P_{\mathcal S_{\Y|\boldsymbol{X}}^{(m)}}\right\| =\|P_{\mc{B}}-P_{\mc{B}^{(m)}}\| $. By Corollary $\ref{cor: sin theta self adjoint}$, we have
\begin{align}\label{eq:sin theta for B Bm}
\|P_{\mc{B}}-P_{\mc{B}^{(m)}}\| \leqslant \frac{\pi}{2}\frac{\|{\mc{B}}-{\mc{B}^{(m)}}\| }{\min\{\lambda_{\min}^+({\mc{B}}),\lambda_{\min}^+({\mc{B}^{(m)}})\}}.
\end{align}

Note that ${\mc{B}}-{\mc{B}^{(m)}}$ is self-adjoint, then
\begin{align*}
&\lno{\mc{B}}-{\mc{B}^{(m)}}\rno =\sup_{{\bs{\beta}}\in\mathbb{S}_{ \mathcal H}}|\langle ({\mc{B}}-{\mc{B}^{(m)}})({\bs{\beta}}),{\bs{\beta}}\rangle|=\sup_{{\bs{\beta}}\in\mathbb{S}_{\mathcal H}}|\langle {\mc{B}}{\bs{\beta}},{\bs{\beta}}\rangle-\langle {\mc{B}^{(m)}}{\bs{\beta}},{\bs{\beta}}\rangle|\\
&~~=\sup_{{\bs{\beta}}\in\mathbb{S}_{\mathcal H}}\hspace{-0.9mm}\left|\sum_{i=1}^d\hspace{-0.9mm}\left[\langle{\bs{\beta}}_i,{\bs{\beta}}\rangle^2-\langle{\bs{\beta}}_i^{(m)},{\bs{\beta}}\rangle^2\right]\right|=\sup_{{\bs{\beta}}\in\mathbb{S}_{\mathcal H}}\hspace{-0.9mm}\left| \sum_{i=1}^d\langle{\bs{\beta}}_i-{\bs{\beta}}_i^{(m)},{\bs{\beta}}\rangle\langle{\bs{\beta}}_i+{\bs{\beta}}_i^{(m)},{\bs{\beta}}\rangle\right|\\
&~~\leqslant\sup_{{\bs{\beta}}\in\mathbb{S}_{\mathcal H}}\sum_{i=1}^d\left| \langle{\bs{\beta}}_i-{\bs{\beta}}_i^{(m)},{\bs{\beta}}\rangle\langle{\bs{\beta}}_i+{\bs{\beta}}_i^{(m)},{\bs{\beta}}\rangle\right|
\leqslant\sum_{i=1}^d\left\|{\bs{\beta}}_i-{\bs{\beta}}_i^{(m)}\right\|\left\|{\bs{\beta}}_i+{\bs{\beta}}_i^{(m)}\right\|,
\end{align*}
where the first inequality comes from the triangle inequality, and the 
second inequality comes from the Cauchy-Schwarz inequality and $\|{\bs{\beta}}\|=1$. 
 Then one has ${\bs{\beta}}_i=\sum\limits_{j=1}^\infty b_{ij}\phi_j$ and 
\[{\bs{\beta}}^{(m)}_i=\Pi_m{\bs{\beta}}_i=\sum_{j'=1}^m\phi_{j'}\otimes\phi_{j'}\sum_{j=1}^\infty b_{ij}\phi_j=\sum_{j'=1}^m\sum_{j=1}^\infty\langle\phi_{j'},\phi_j\rangle b_{ij}\phi_{j'}=\sum_{j=1}^mb_{ij}\phi_j.\]
According to Assumption $\ref{assumption: rate-type condition}$, one can get
\begin{align*}
\left\|{\bs{\beta}}_i-{\bs{\beta}}_i^{(m)}\right\|&=\left\|\sum_{j=m+1}^\infty b_{ij}\phi_j\right\|=\sqrt{\sum_{j=m+1}^\infty b_{ij}^2}\leqslant \wt C\sqrt{\sum_{j=m+1}^\infty j^{-2\alpha_2}};\\
\left\|{\bs{\beta}}_i+{\bs{\beta}}_i^{(m)}\right\|&\leqslant\|{\bs{\beta}}_i\|+\lno{\bs{\beta}}_i^{(m)}\rno\leqslant2\|{\bs{\beta}}_i\|=2\sqrt{\sum_{j=1}^\infty b_{ij}^2}\leqslant 2\wt C\sqrt{\sum_{j=1}^\infty j^{- 2\alpha_2}}.
\end{align*}
Because $\alpha_2>1/2$, one has
\[\sum\limits_{j=m+1}^\infty \frac{1}{j^{2\alpha_2}}\leqslant \frac{1}{2\alpha_2-1}\frac{1}{m^{2\alpha_2-1}};\qquad \sum_{j=1}^\infty \frac 1{j^{2\alpha_2}}=\zeta(2\alpha_2)\text{ is convergent},\]
where $\zeta(\cdot)$ is Riemann $\zeta$ function. Thus, one can get
\begin{equation}\label{eq: upper bound of operator norm of A minus B}
\lno{\mc{B}}-{\mc{B}^{(m)}}\rno \leqslant 2d\wt C^2\sqrt{\frac{\zeta(2\alpha_2)}{2\alpha_2-1}}m^{-\frac{2\alpha_2-1}{2}}.
\end{equation}

Furthermore, 
{since $\mr{rank}(\mc{B})=d$, one can get that $\lambda_{\min}^+(\mc{B})=\lambda_{d}(\mc{B})$. It is easy to see $\rank(\mc{B}^{(m)})\leqslant d$ by $\mc{B}^{(m)}=\Pi_m \mc{B} \Pi_m$, thus one can assume that $\lambda_{\min}^+(\mc{B}^{(m)})=\lambda_j( \mc{B}^{(m)})$ for some $j\leqslant d$.
By Corollary $\ref{coro:wely ineq operator}$
% (Notice that $M_m$ and $M$ are both compact and self-adjoint)
and \eqref{eq: upper bound of operator norm of A minus B}, one has:
$$
|\lambda_j( \mc{B}^{(m)})-\lambda_j\l \mc{B}\r|\leqslant\lno \mc{B}-\mc{B}^{(m)}\rno \leqslant 2d\wt C^2\sqrt{\frac{\zeta(2\alpha_2)}{2\alpha_2-1}}m^{-\frac{2\alpha_2-1}{2}}.
$$
Thus for sufficiently large {$n\geqslant \l \frac{\lambda_d({\mc{B}})}{4d\wt C^2}\sqrt{\frac{2\alpha_2-1}{\zeta(2\alpha_2)}} \r^{\frac{2}{c_1(1-2\alpha_2)}}$}, one has
\begin{align}
&\lambda_j\l \mc{B}^{(m)}\r\geqslant \lambda_j\l \mc{B}\r-\frac{\lambda_d\l \mc{B}\r}{2}
\geqslant\frac{\lambda_d\l \mc{B}\r}{2}\nonumber\\
&\qquad\Longrightarrow \min\{\lambda_{\min}^+({\mc{B}}),\lambda_{\min}^+({\mc{B}^{(m)}})\}\geqslant \frac{\lambda_d({\mc{B}})}{2}. \label{eq:lower bound lambda min plus B Bm}
\end{align}}
Inserting \eqref{eq: upper bound of operator norm of A minus B} and \eqref{eq:lower bound lambda min plus B Bm} into \eqref{eq:sin theta for B Bm} leads to
\begin{align*}
\left\|P_{\mathcal S_{\Y|\boldsymbol{X}}}-P_{\mathcal S_{\Y|\boldsymbol{X}}^{(m)}}\right\| \leqslant \frac{2\pi d\wt C^2}{\lambda_{d}(\mc{B})}\sqrt{\frac{\zeta(2\alpha_2)}{2\alpha_2-1}}m^{-\frac{2\alpha_2-1}{2}}.
\end{align*}
Then choosing $C_2:=\frac{2\pi d\wt C^2}{\lambda_d({\mc{B}})}\sqrt{\frac{\zeta(2\alpha_2)}{2\alpha_2-1}}$ can complete the proof.
\end{proof}



\subsection{Proof of Theorem \ref{theorem, total convergence rate}}
\begin{proof}
Note that
\begin{equation}
\begin{aligned}
\left\|P_{\mc{S}_{\Y|\X}}-P_{\widehat{\mc{S}}_{\Y|\X}^{(m)}}\right\| 
&\leqslant \left\|P_{\mc{S}_{\Y|\X}}-P_{\mc{S}_{\Y|\X}^{(m)}}\right\| +\left\|P_{\mc{S}_{\Y|\X}^{(m)}}-P_{ \widehat{\mc{S}}_{\Y|\X}^{(m)}}\right\| .\\
\end{aligned}
\end{equation}
Next we select $m$ to be $n^{\frac{1-2\gamma}{2\alpha_1+2\alpha_2+1}}$, i.e.,  $c_1:=\frac{1-2\gamma}{2\alpha_1+2\alpha_2+1}$. And it is easy to check that $c_1$ satisfies $2c_1(\alpha_1+1)+2\gamma-1=-\frac{(1-2\gamma)(2\alpha_2-1)}{2\alpha_1+2\alpha_2+1}<0$ and $c_1(2\alpha_1+1)-1=-\frac{2[\gamma(2\alpha_1+1)+\alpha_2]}{2\alpha_1+2\alpha_2+1}<0$.
Then combining Proposition $\ref{proposition, estimation error}$ with Proposition $\ref{proposition, truncation error}$ leads to
\begin{align*}
\P\left[\left\|P_{\mc S_{\Y|\X}}-P_{\widehat{\mc{S}}_{\Y|\X}^{(m)}}\right\| \leqslant\hspace{-0.5mm} (C_1+C_2)n^{-\frac{(2\alpha_2-1)(1-2\gamma)}{2(2\alpha_1+2\alpha_2+1)}}\right]\hspace{-1mm}\geqslant\hspace{-1mm} 1-2\exp\hspace{-0.5mm}\l\hspace{-1mm}- C'n^{\frac{1-2\gamma}{2\alpha_1+2\alpha_2+1}}\r&\\
-\exp\left[\ln\l D_1n^{\frac{2\alpha_1+2\alpha_2+3-4\gamma}{2\alpha_1+2\alpha_2+1}} \r-(D_0+1)n^{\frac{2\gamma}{5}}\right]&
\end{align*}
when $n\geqslant n_3'$, where
\begin{align*}
n_3'=\max\Bigg\{n_1,n_2',\left[\tfrac{\|\Gamma^{-1}M\|  \wt C}{48}\l\tfrac{D_2}{D_0+1}\r^{\frac52}\right]^{\frac{2}{2(c_1(\alpha_1+1)+\gamma)-1}}\hspace{-0.9mm},\l \tfrac{\|\Gamma^{-1}M\| }{2D_3}\r^{\frac{2}{c_1(2\alpha_1+1)-1}}\hspace{-0.9mm},\\
\l\tfrac{D_0+1}{D_2}\r^{\frac{5}{1-2\gamma}},\left[ \tfrac{D_3\wt C}{24}\l \tfrac{D_2}{D_0+1} \r^{\frac52} \right]^{\frac2{2\gamma+c_1}},\l\tfrac{\lambda_d(\mc{B})}{4d\wt C^2}\sqrt{\tfrac{{2\alpha_2-1}}{\zeta(2\alpha_2)}} \r^{\frac{2}{c_1(1-2\alpha_2)}}\Bigg\}
\end{align*}

It is easy to check that as long as $\frac{2\gamma}{5}<\frac{1-2\gamma}{2\alpha_1+2\alpha_2+1}\Longrightarrow\gamma<\frac{5}{4(\alpha_1+\alpha_2+3)}$, 
there exists a constant $n_3''=n_3''\l \gamma,\alpha_1,\alpha_2,D_0,D_1,C'\r$ such that when $n\geqslant n_3'$ further, we have 
\begin{align*}
\P\l\left\|P_{\mc S_{\Y|\X}}-P_{\widehat{\mc{S}}_{\Y|\X}^{(m)}}\right\| \leqslant (C_1+C_2)n^{-\frac{(2\alpha_2-1)(1-2\gamma)}{2(2\alpha_1+2\alpha_2+1)}} \r
\geqslant1-2\exp\l-\tfrac{D_0+1}{2}n^{\frac{2\gamma}{5}} \r.
\end{align*}
Thus one can choose $n_3=\max\{n_3',n_3''\}$ to get the following conclusion.
\begin{proposition}
Under Assumptions $\ref{as:joint distribution assumption}$ to $\ref{assumption: rate-type condition}$, for any $\gamma\in\l0,\tfrac{5}{4(\alpha_1+\alpha_2+3)}\r$, choosing 
$m=n^{\frac{1-2\gamma}{2\alpha_1+2\alpha_2+1}}$ (i.e.,  $c_1=\frac{1-2\gamma}{2\alpha_1+2\alpha_2+1}$) yields a positive constant
\begin{align*}
D_4:=D_4\l \|\Gamma^{-1}M\| ,\sigma_d(\Gamma^{-1}M) ,\gamma,\sigma_0,\sigma_1,d,\wt C,\lambda_d\l\sum\limits_{i=1}^d {\bs{\beta}}_i\otimes{\bs{\beta}}_i\r,\alpha_2\r 
\end{align*}
such that when $n$ is sufficiently large, we have:
\begin{align*}
\P\l\left\|P_{\mc{S}_{\Y|\X}}-P_{\widehat{\mc{S}}_{\Y|\X}^{(m)}}\right\| \leqslant D_4n^{-\frac{(2\alpha_2-1)(1-2\gamma)}{2(2\alpha_1+2\alpha_2+1)}} \r
\geqslant1-2\exp\l -\tfrac{D_0+1}{2}n^{\frac{2\gamma}{5}} \r,
\end{align*}
where $D_0$ and $D_1$ are defined in Proposition $\ref{prop:bound hatMmd Mm}$.
\end{proposition}
\noindent
% Theorem $\ref{theorem, total convergence rate}$ is a direct corollary of above proposition.
Define 
$$\mathtt F:=\left\{\left\|P_{\mc{S}_{\Y|\X}}-P_{\widehat{\mc{S}}_{\Y|\X}^{(m)}}\right\| \leqslant D_4n^{-\frac{(2\alpha_2-1)(1-2\gamma)}{2(2\alpha_1+2\alpha_2+1)}}\right\}.$$
Then 
\begin{align*}
 \mb E\left[\left\|P_{\mc{S}_{\Y|\X}}-P_{\widehat{ \mc{S}}_{\Y|\X}^{(m)}}\right\|^2\right] =&
  \mb E\left[\left\|P_{\mc{S}_{\Y|\X}}-P_{\widehat{ \mc{S}}_{\Y|\X}^{(m)}}\right\|^21_{\mathtt{F}}\right] +
   \mb E\left[\left\|P_{\mc{S}_{\Y|\X}}-P_{\widehat{ \mc{S}}_{\Y|\X}^{(m)}}\right\|^21_{\mathtt{F}^c}\right]\\ 
 \leqslant &
 D_4^2n^{-\frac{(2\alpha_2-1)(1-2\gamma)}{2\alpha_1+2\alpha_2+1}}+4\mb P\left( \mathtt F^c\right)\\
 \lesssim&n^{-\frac{(2\alpha_2-1)(1-2\gamma)}{2\alpha_1+2\alpha_2+1}}+\exp\l -\tfrac{D_0+1}{2}n^{\frac{2\gamma}{5}} \r\\
\lesssim&n^{-\frac{(2\alpha_2-1)(1-2\gamma)}{2\alpha_1+2\alpha_2+1}}.
\end{align*}
This completes the proof of  Theorem \ref{theorem, total convergence rate}.
\end{proof}







\section{Additional Simulation Results of Section \ref{sec:Synthetic}}
This section contains the additional  simulation results  of Sections \ref{sec:Synthetic}  when $\varepsilon\sim N(0,1)$.



We show the average $\mc D(\bs B;\bs{\wh B})$ with different $m$ or $\rho$ for three methods under $\mc M_1$ to $\mc M_3$ in Figure \ref{fig:error 3models,noise1},
where we mark minimal error in each model with red `$\times$'. The shaded areas represent the standard error associated with these estimates and all of them are less than  $0.01$. For FSFIR, the  minimal errors for $\mc M_1-\mc M_3$ are  $0.08,0.02,0.01$ respectively.
For TFSIR, the  minimal errors are  $0.08,0.02,0.01$ and for regularized FSIR,  the  minimal errors are $0.13,0.06,0.01$.  

% Figure environment removed


Figure \ref{fig:error 3models,noise1} shows that FSFIR attains the best performance among  all models. 
Moreover, FSFIR is easier to practice as it does not need a slice number $H$ in advance. 






%\appendix
%\begin{appendices}
%\begin{comment}
\section{System Architecture}
\label{appendix:architecture}
\system has a novel modularized system architecture with three key components: 
\emph{StreamManager}, 
\emph{TxnManager} and \emph{TxnScheduler}. 
These components are instantiated in each thread locally.
The execution outline of \system is presented in Algorithm~\ref{alg:algo}.
Transactional stream processing is continuous and potentially never ends (Line 1$\sim$8).
The dependency resolution and execution of state transactions are separated into two non-overlapping phases by punctuations~\cite{Tucker:2003:EPS:776752.776780} (Line 2 and 5), which guarantees that no subsequent input event will have a smaller timestamp. 
Effectively, a batch of state transactions is collected during the first phase, and processed during the second phase.

In the first phase (i.e., stream processing phase), 
the \emph{StreamManager} conducts preprocessing for every input event ($e$). Similar to some prior works~\cite{tstream}, state transactions may be issued but not immediately processed during preprocessing (Line 3).
The \emph{pre\_processing} and \emph{post\_processing} functions are exposed as APIs to users.
The \emph{TxnManager} handles dependency resolution (Line 4) among state transactions and insert decomposed operations to construct a \tpg. We discuss the detailed two-phase \tpg construction process in Section~\ref{subsec:construction}.

In the second phase  (i.e., transaction processing phase), 
the \emph{TxnManager} is first involved again to refine (Line 6) the constructed \tpg with further dependency resolution.
The \emph{TxnScheduler} 
schedules operations for concurrent execution based on the constructed \tpg according to the three dimensions of scheduling decisions (Line 7). 
In particular, a scheduling decision model $M$ is instantiated based on the constructed \tpg (Line 14).
\textbf{\circled{1}} Guided by $M$, execution threads adopt an exploration strategy (Section~\ref{subsec:explore}) to explore the constructed \tpg for operations available to be scheduled constrained by dependencies. 
\textbf{\circled{2}} 
During exploration, one or multiple operations may be treated as the 
% basic 
unit of scheduling (Section~\ref{subsec:granularity}). 
Subsequently, \textbf{\circled{3}} every thread executes operation(s) in the unit of scheduling with various abort handling mechanisms (Section~\ref{subsec:abort_handling}).
Only when state transactions are processed (i.e., committed or aborted) can the associated input events be postprocessed (Line 8) by the \emph{StreamManager} based on transaction processing results.
\end{comment}

\begin{comment}
\begin{algorithm}
\footnotesize
    \KwData{$e$ \tcp{Input event}}
    \KwData{$txn_{ts}$ \tcp{State transaction}}
    \KwData{$G$ \tcp{The currently constructed TPG}}
    \While{!finish processing of input streams}{
        \eIf(\tcp*[h]{Phase 1}){\text{$e$ is not a $punctuation$}}{
                $txn_{ts}$ $\gets$ PRE\_Processing($e$)\;
                \textbf{TPG\_Construction}($G$, $txn_{ts}$)\; 
          }(\tcp*[h]{Phase 2}){
                \textbf{TPG\_Refinement}($G$)\; 
                \textbf{TXN\_Scheduling}($G$)\; 
                POST\_Processing()\;
          }
    }
    
    \SetKwFunction{FMain}{TPG\_Construction}
    \SetKwProg{Fn}{Function}{:}{}
    \Fn{\FMain{$G$, $txn_{ts}$}}{
        $O_{1..k}$ $\gets$ \textbf{Partition} $txn_{ts}$\;
        \ForEach{\text{operation $O_{i}$ $\in$ $O_{1..k}$}}{
            \textbf{Identify} its \ld\;
            $G$ $\gets$ $G$ + $O_{i}$ \;
        }
    }
    \SetKwFunction{FMain}{TPG\_Refinement}
    \SetKwProg{Fn}{Function}{:}{}
    \Fn{\FMain{$G$}}{
        \ForEach{\text{vertex $e_{i}$ $\in$ $G$}}{
            \textbf{Identify} its \td, \pd\;
        }
    }
    
    \SetKwFunction{FMain}{TXN\_Scheduling}
    \SetKwProg{Fn}{Function}{:}{}
    \Fn{\FMain{$G$}}{
        $M$ $\gets$ Instantiated with $G$;\tcp{A decision model}
        \While{!finish scheduling of $G$
        }{
          \textbf{\circled{2}} $Scheduling Unit$ $\gets$ \textbf{\circled{1}} \emph{Explore}($G$, $M$)\; 
            \textbf{\circled{3}} \emph{Execute with Abort Handling} ($Scheduling Unit$)\; 
        }
    }
  \caption{Execution Outline of \system}
  \label{alg:algo}
\end{algorithm}
\end{comment}

%\end{appendices}


\end{document}
