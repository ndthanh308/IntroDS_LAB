\setlength{\tabcolsep}{2pt}
\begin{table}[h]
\centering
\footnotesize
\begin{tabular}{lccccccccccccccc}
\toprule

& Spleen & RKidney & LKidney & Gallbladder & Esophagus & Liver & Stomach & Aorta & IVC & PV\&SV & Pancreas & RAdrenal & LAdrenal & Total\\
\midrule
\multicolumn{15}{c}{Dice Scores}\\
\midrule
% TotalSegmentor & 0.96 & 0.91 & 0.95 & 0.73 & 0.79 & 0. & 0. & 0. & 0. & 0. & 0. & 0. & 0. & 0.33 \\
Atlas (V1) & 0.96 & 0.91 & 0.94 & 0.62 & 0.72 & 0.97 & 0.84 & 0.91 & 0.81 & 0.76 & 0.75 & 0.64 & 0.59 &  
0.81\\
Atlas (V2) & 0.96 & 0.86 & 0.92 & 0.72 & 0.80 & 0.96 & 0.86 & 0.92 & 0.85 & 0.77  & 0.83 & 0.75 & 0.74 & 
0.85\\
\midrule
\multicolumn{15}{c}{Mean Surface Distance}\\
\midrule
Atlas (V1) & 1.14 & 2.21 & 0.83 & - & 1.68 & 0.79 &  3.26  & 1.51 & 2.03 & 1.48 & 2.53 & 1.30 & 1.67 & 2.03\\
Atlas (V2) & 0.65 & 4.28 & 1.70 & - & 1.38 & 1.21 &  2.50 & 1.29 & 1.87 & 1.77 & 1.51 & 0.80 & 0.90 & 1.85\\
\bottomrule
\hline
\end{tabular}
\caption{Class-wise Dice and Mean Surface Distance performance on the BTCV dataset for the standard Atlas model and the robust Atlas model. Both predictions have not been post-processed. 
%We also include the predictions of the TotalSegmentor model. 
It can clearly be seen that the robust Atlas V2 model benefited from the adapted training procedure.}
\label{tab:btcv}

\end{table}