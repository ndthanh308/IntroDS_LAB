\documentclass[fleqn,10pt]{wlscirep}
\usepackage[utf8]{inputenc}
\usepackage[T1]{fontenc}
\usepackage{lineno}

\usepackage{algorithm}
\usepackage{algorithmic}
\usepackage{adjustbox}

\linenumbers

\newcommand{\todo}[1]{\textcolor{red}{TODO: #1}}

%Added by Alex
\usepackage{subfig}

\usepackage{dirtree}

\nolinenumbers

%\title{Dense Predictions on Whole-Body CT Images for 144 Anatomical Structures by Leveraging Anatomical Knowledge}

%\title{Automated Generation of an Anatomical full Body CT Dataset via Knowledge Aggregation and Anatomical Rules}

\title{Towards Unifying Anatomy Segmentation: Automated Generation of a Full-body CT Dataset via Knowledge Aggregation and Anatomical Guidelines}

\author[1,7$\dag$,*]{Alexander Jaus}
\author[1,3,$\dag$]{Constantin Seibold}
\author[3]{Kelsey Hermann}
\author[2,4,5,7]{Alexandra Walter}
\author[4,5]{Kristina Giske}
\author[6]{Johannes Haubold}
\author[3,7,**]{Jens Kleesiek}
\author[1,7,**]{Rainer Stiefelhagen}
\affil[1]{Institute for Anthropomatics and Robotics, Karlsruhe Institute of
Technology, Karlsruhe, Germany}
\affil[2]{Steinbuch Center for Computing, Karlsruhe Institute of Technology (KIT), Karlsruhe,
%Hermann-von-Helmholtz-Platz 1, Eggenstein-Leopoldshafen, 
Germany}
\affil[3]{Institute for AI in Medicine, University Hospital Essen, Essen, Germany}
\affil[4]{Department of Medical Physics in Radiation Oncology, German Cancer Research Center (DKFZ), 
%Im Neuenheimer Feld 280, 
Heidelberg, Germany}
\affil[5]{Heidelberg Institute of Radiation Oncology (HIRO) \& National Center for Radiation Research in Oncology (NCRO), Heidelberg/Dresden, Germany}
\affil[6]{Department of Diagnostic and Interventional Radiology and Neuroradiology, University Hospital Essen, Essen, Germany}
\affil[7]{Helmholtz Information and Data Science School for Health (HIDSS4Health), Karlsruhe/Heidelberg, Germany}

\affil[*]{corresponding author(s): Alexander Jaus (alexander.jaus@kit.edu)}

\affil[$\dag$]{these authors contributed equally to this work}

\affil[**]{Shared last author}



\begin{abstract}
\label{sec:abstract}
With the advent of decentralised digital currencies powered by blockchain technology, a new era of peer-to-peer transactions has commenced. The rapid growth of the cryptocurrency economy has led to increased use of transaction-enabling wallets, making them a focal point for security risks. As the frequency of wallet-related incidents rises, there is a critical need for a systematic approach to measure and evaluate these attacks, drawing lessons from past incidents to enhance wallet security.

In response, we introduce a multi-dimensional design taxonomy for existing and novel wallets with various design decisions. We classify existing industry wallets based on this taxonomy, identify previously occurring vulnerabilities and discuss the security implications of design decisions. We also systematise threats to the wallet mechanism and analyse the adversary's goals, capabilities and required knowledge. We present a multi-layered attack framework and investigate 84 incidents between 2012 and 2024, accounting for \$5.4B. Following this, we classify defence implementations for these attacks on the precautionary and remedial axes. We map the mechanism and design decisions to vulnerabilities, attacks, and possible defence methods to discuss various insights. 

\end{abstract}
\begin{document}

\flushbottom
\maketitle
%  Click the title above to edit the author information and abstract

\thispagestyle{empty}

\section*{Background \& Summary}
\label{Sec: Background}
Medical image segmentation has shown tremendous success ever since the advent of Deep Learning within the medical field which can be dated back to the introduction of the U-Net~\cite{ronneberger2015u}. While there has been a lot of architectural advancement of models to generalize U-Net architectures to the prevalent 3D medical imaging modalities~\cite{cciccek20163d,milletari2016v} such as Computed Tomography (CT), Magnetic Resonance Imaging (MRI) or  Positron Emission Tomography (PET), there has been less effort in providing a holistic view on the entire human body and its anatomy. 
Current models for medical data predominantly specialize in partially annotated datasets on sub-areas of the human body. 
These datasets range from single organ annotations such as the segmentation of spleen or pancreas~\cite{simpson2019large} to multi-organ datasets such as the BTCV~\cite{landman2015miccai} containing annotations for 13 abdominal organs. The recently introduced Medical Segmentation Decathlon~\cite{antonelli2022medical} aims at generalizing models to multiple tasks. Each of the individual tasks is however still limited to a specific body region and a certain organ of interest which does not address a complete anatomical view.  % and even proceeding towards a comprehensive view of the human body~\cite{} providing 50 organs-at-risk-annotations. 
%Why is it bad to treat the human body in a partially annotated and subpar fashion? Provide an intutition
Focusing on only a subsection of the entire anatomy has multiple downsides. From a technical point of view, it limits potential downstream applications which could be used within clinical settings. Additionally, it is a well-known fact that the human body consists of multiple systems and structures which are in constant interaction with each other to maintain a homeostatic state. Thus also from a medical point of view, it appears beneficial to look at interactions between structures instead of treating them in an isolated fashion. This requires overcoming the current constrained setting and striving towards a holistic view of anatomical structures which is a limitation of the current literature.
%Homeostasis. Wechselwirkung zwischen Organen kann besser erkannt werden. 


%Describe the problem of more and more labels
A common problem for Deep Learning applications is the need for large annotated datasets. The medical domain is no exception and poses additional difficulties due to the required expertise to annotate medical images combined with the labor-intensive annotation process for common 3D imaging modalities. While this is already cumbersome for a small number of labels, it is effectively impossible to find a team of radiologists willing to annotate the combination of hundreds of categories for hundreds of CTs. For example, the recently proposed ATM-dataset~\cite{zhang2023multi}, comparable in size with the proposed Dense Anatomical Prediction (DAP) Atlas dataset, utilizes three expert radiologists working on a single label with each CT volume requiring about 60-90 minutes~\cite{zhang2023multi}. Assuming an average time of $75$ minutes per CT volume, this single-label dataset requires almost $80$ days of work, assuming $8$h of uninterrupted work per day. Scaling to more labels is infeasible as it will not only increase the workload but also might cause the radiologist to lose track due to the higher complexity of the task. 

%MOOSE
The creators of MOOSE~\cite{sundar2022fully}, a multi-organ segmentation model use a hybrid approach consisting of expert annotation and automated annotations. The experts annotate on a small dataset of $50$ CT images $13$ organ structures, $20$ bone segments, and $4$ tissue structures which were semi-automatically extracted. For the majority of their labels which are $83$ cerebral structures, the authors use an automatic segmentation approach by leveraging the Hammersmith atlas~\cite{hammers2003three}.


%Describe the approach of chen2021 deep and total segmentation and how we are different from them
TotalSegmentator~\cite{wasserthal2022totalsegmentator}  approaches this problem via active learning in which an expert improves model predictions which are again fed to the model to improve the predictions. While this procedure noticeably reduces an expert's time spent on annotation, it still relies on direct interaction with experts for multiple weeks to generate a dataset of 104 anatomical structures. It also has to be acknowledged that large datasets with this many labels face a concrete challenge when evaluating the label quality since it becomes increasingly infeasible to have pixel-wise alignments checked for all anatomical structures. As such, TotalSegmentor~\cite{wasserthal2022totalsegmentator} relied upon quality insurance via 3D renderings instead of manual voxel-wise alignment checks. 

% pseudo-label filtering
When we compare this to the natural image domain, several works utilize entirely automated annotation for classification and segmentation purposes\cite{xie2020self,zhai2022scaling,kirillov2023segment}. One example is the recent \textit{Segment Anything}-dataset~\cite{kirillov2023segment}. Here, the authors train a base model on manually annotated data and make predictions on unlabeled images through multi-scale inference and filter predictions via non-maximum suppression leading to 11 million annotated images. The general concept of pseudo-label filtering from weak annotations like image-level class labels improves the unlabeled training data and leads to more stable models\cite{reiss2023decoupled,wang2022omni,bae2022one,mlynarski2019deep,qu2023annotating}.  
Other works~\cite{hu2023label} have shown the successful application of pseudo-generated tumor labels on CT data of the liver leading to accurate segmentation results on real liver tumors.

%Describe your approach
Motivated by these recent advancements and the difficulty to scale expert annotations to multi-label large-scale medical datasets, we want to develop a dataset that enables the training of models suitable to serve a variety of clinically relevant downstream tasks which profit from extensive anatomical knowledge like body composition analysis, surgery planning or cancer treatment monitoring. 

We build upon this core idea of pseudo-label filtering and refinement to employ an expert-free dataset generation approach that aggregates the scattered anatomical knowledge of multiple source datasets and aggregates them into a single dataset. 
We combine anatomical information from various sources through pseudo-label-based label aggregation and pseudo-label-refinement in post-processing strategies that leverage anatomical textbook knowledge to assert the anatomical plausibility of the labels. 
%By integrating these steps into self-training concepts, we generate our Dense Anatomical Prediction (DAP) Atlas dataset with anatomical labels. 
Our dataset has been approved by experts, despite not having contributed to its creation.  
The dataset consists of $533$ whole-body CT images with labels for $142$ anatomical structures containing information ranging from body composition, organs to various vessels. We are the first to release a dataset containing dense annotations for every voxel in a full-body human CT. We show an example of the dataset in Fig.~\ref{fig:intro_fig}. 
%The dataset enables the training of models which can serve a variety of clinically relevant downstream tasks which profit from extensive anatomical knowledge like a body composition analysis, surgery planning or cancer treatment monitoring. 

% Figure environment removed

%Summary of the dataset and the contributions

The remainder of this paper is structured as follows: Section \textit{Methods} discusses how the full-body CT data and the labels are acquired and how they are processed to generate our Atlas dataset. In Section \textit{Data Records} we briefly discuss the structure and availability of the dataset to ease the usage for the community. This is followed by Section \textit{Technical Validation} in which the quality of the dataset is examined. We conclude the paper with the final chapter \textit{Discussion and Conclusion}. 
\section*{Methods}
%Why AutoPet dataset
\label{S: Methods}
\subsection*{Data Acquisition}
%Why CT

Two of the most common volumetric modalities are CT and MRI. 
While MRI often focuses on soft tissue analysis and brain imaging, CT is a common choice in the clinical routine due to its acquisition time and broad field of use. As we aim to generate models to segment any anatomy utilizing various sources, we start by selecting a dataset that acts as a solid basis for full-body label aggregation.
% Why AuotPET
The recently published AutoPET dataset~\cite{gatidis2022whole} is a PET-CT dataset that perfectly fits our requirements since nuclear medicine often requires full-body CT scans to track therapy. In addition to the full-body CTs, this dataset might enable future multi-modal segmentation tasks~\cite{xue2021multi, marinov2023mirror} due to the separate PET domain and lesion annotations. Future multimodal tasks could make use of the provided anatomical structures, which, however, is not the focus of this work.

%Which volumes have been selected? 
%Why did we choose a subset?
We select a subset of $566$ CTs of the AutoPET dataset. The selection criterion is based on similar slice thickness in the axial dimension leading to a homogeneous dataset. Furthermore, we make sure that the images show important regions of interest. Our region of interest starts from the head and ends slightly below the hip which includes all thoracic and abdominal organs. The chosen subset consists of the CTs which contain between $336$ and $400$ slices in the original dicom files. We exclude CTs with fewer slices as these tend to show an insufficient subpart of the body contradicting the desired full-body dataset. CT images containing more than 400 slices tend to include more irrelevant content. %background and anatomical structures outside of our region of interest such as the legs. 
This leaves us with a homogeneous dataset of size $566$. Our final DAP Atlas dataset does deviate from this selection, as we filter out implausible predictions, in a final post-processing step leaving us with $533$ CT images. The filter criteria will be described in detail later. In the following, we will refer to the DAP Atlas dataset as the dataset containing $533$ images. 

%Description of the subset
Our DAP Atlas is similar to AutoPET regarding the age and gender distributions as well as pathological findings. We show a descriptive analysis of the dataset regarding the aforementioned dimensions in Fig.~\ref{fig:descriptive_statistics}. 

% Figure environment removed

%Maybe we need a table showing average 

\subsection*{Knowledge Acquisition}


 % Figure environment removed
% % Figure environment removed


%Where do the individual labels come from
The dataset aggregates multiple sources of anatomical segmentation knowledge which we differentiate into public knowledge which is present in the form of publicly available datasets and private knowledge which are private datasets available to us. Besides the segmentation knowledge, we leverage rule-based knowledge which is derived from anatomical textbook knowledge and represents what could be described as the common sense of a radiologist. These rules contain for instance, which anatomical structures are possible in which part of the human body. We display the DAP Atlas construction workflow in Fig.~\ref{fig:merging} and discuss the details in the following.

A large amount of labels of the dataset is derived from public segmentation knowledge which is present in fragmented form through publicly available datasets. These contain annotations of organs of interest on CT images showing parts of the human body. We extract this knowledge by training neural networks on these public datasets, which learn to predict the labels of the respective datasets. Typically, training a neural network is a data and task-specific problem and requires finetuning a large set of hyperparameters which is impracticable for our desired applications as we intend to train a vast number of models on various heterogeneous datasets. To overcome this problem, we use the nnU-Net~\cite{isensee2021nnu}, a framework that automatically configures a U-Net and adapts the training procedure to the data at hand. This Auto-ML framework provides segmentation results surpassing several more complex works without any of their additional engineering overhead. We employ standard nnU-Nets and train them on publicly available datasets. The used publicly available datasets are shown along with their obtained label category in Fig.~\ref{Tab:Label_Usage}. After training, these networks are used to carry over the extracted knowledge by predicting the learned labels into our full-body selected DAP Atlas CT images. We describe the used datasets and the merging procedure in the following. %and show more detailed information about the used datasets and the source of each label in the supplementary information. 

%TODO: Include pathological labels?

% \begin{table}[]
% \centering
% \begin{tabular}{|l||r|r|r|r|r|r|r|r|r|r|r|r|r|r|}
% \hline
%                & \multicolumn{1}{l|}{P} & \multicolumn{1}{l|}{TS} & \multicolumn{1}{l|}{V} & \multicolumn{1}{l|}{RS} & \multicolumn{1}{l|}{Hip} & \multicolumn{1}{l|}{A} & \multicolumn{1}{l|}{C} & \multicolumn{1}{l|}{ATM} & \multicolumn{1}{l|}{P} & \multicolumn{1}{l|}{ST} & \multicolumn{1}{l|}{Abd} & \multicolumn{1}{l|}{Rule} & \multicolumn{1}{l|}{BCA} & \multicolumn{1}{l|}{HN} \\ \hline \hline
% Musculature    & 0                      & 10                      & 0                      & 0                       & 0                        & 0                      & 0                      & 0                        & 0                      & 0                       & 0                        & 2                         & 1                        & 0                       \\ \hline
% Tissues        & 2                      & 0                       & 0                      & 0                       & 0                        & 0                      & 0                      & 0                        & 0                      & 0                       & 0                        & 0                         & 4                        & 0                       \\ \hline
% Digestive      & 9                      & 9                       & 0                      & 0                       & 0                        & 0                      & 0                      & 0                        & 0                      & 1                       & 1                        & 0                         & 0                        & 0                       \\ \hline
% Urinary        & 3                      & 3                       & 0                      & 0                       & 0                        & 0                      & 0                      & 0                        & 0                      & 0                       & 0                        & 0                         & 0                        & 0                       \\ \hline
% Endocrine      & 4                      & 2                       & 0                      & 0                       & 0                        & 2                      & 0                      & 0                        & 0                      & 0                       & 0                        & 0                         & 2                        & 0                       \\ \hline
% Reproductive   & 2                      & 0                       & 0                      & 0                       & 0                        & 0                      & 1                      & 0                        & 0                      & 0                       & 0                        & 0                         & 0                        & 0                       \\ \hline
% Nervous        & 1                      & 1                       & 0                      & 0                       & 0                        & 0                      & 0                      & 0                        & 0                      & 0                       & 0                        & 0                         & 0                        & 0                       \\ \hline
% Immune         & 2                      & 1                       & 0                      & 0                       & 0                        & 1                      & 0                      & 0                        & 0                      & 0                       & 0                        & 0                         & 0                        & 0                       \\ \hline
% Vertebras      & 0                      & 24                      & 24                     & 0                       & 0                        & 0                      & 0                      & 0                        & 0                      & 0                       & 0                        & 0                         & 0                        & 0                       \\ \hline
% Ribs           & 0                      & 24                      & 0                      & 24                      & 0                        & 0                      & 0                      & 0                        & 0                      & 0                       & 0                        & 1                         & 1                        & 0                       \\ \hline
% Bones          & 2                      & 10                      & 0                      & 0                       & 1                        & 0                      & 0                      & 0                        & 0                      & 0                       & 0                        & 1                         & 0                        & 1                       \\ \hline
% Cardiovascular & 1                      & 11                      & 0                      & 0                       & 0                        & 2                      & 0                      & 0                        & 0                      & 2                       & 1                        & 0                         & 1                        & 11                      \\ \hline
% Vessels        & 0                      & 2                       & 0                      & 0                       & 0                        & 0                      & 0                      & 0                        & 0                      & 0                       & 0                        & 0                         & 0                        & 0                       \\ \hline
% Respiratory    & 0                      & 6                       & 0                      & 0                       & 0                        & 0                      & 0                      & 1                        & 1                      & 1                       & 0                        & 1                         & 0                        & 0                       \\ \hline
% Visual         & 0                      & 0                       & 0                      & 0                       & 0                        & 0                      & 0                      & 0                        & 0                      & 0                       & 0                        & 2                         & 0                        & 0                       \\ \hline 
%                & \multicolumn{1}{l|}{}  & \multicolumn{1}{l|}{}   & \multicolumn{1}{l|}{}  & \multicolumn{1}{l|}{}   & \multicolumn{1}{l|}{}    & \multicolumn{1}{l|}{}  & \multicolumn{1}{l|}{}  & \multicolumn{1}{l|}{}    & \multicolumn{1}{l|}{}  & \multicolumn{1}{l|}{}   & \multicolumn{1}{l|}{}    & \multicolumn{1}{l|}{}     & \multicolumn{1}{l|}{}    & \multicolumn{1}{l|}{}   \\ \hline
% Used Labels    & 26                     & 103                     & 24                     & 24                      & 1                        & 5                      & 1                      & 1                        & 1                      & 4                       & 2                        & 7                         & 9                        & 12                      \\ \hline
% \end{tabular}
% \caption{Overview of the different source datasets from which the DAP Atlas dataset is derived. Within this table we cluster the individual labels according to their anatomical system. We show a full table with individual labels in the supplementary material. On the bottom row we show the number of labels in the DAP Atlas dataset which are influenced by the respective source dataset. The sum of columns cannot be calculated, because predictions of the same label are fused. The abbreviations for the datasets are as follows: \textbf{P}:Pediatric~\cite{jordan2022pediatric}, \textbf{TS}: Total Segmentator~\cite{wasserthal2022totalsegmentator}, \textbf{V}: Verse~\cite{sekuboyina2021verse}, \textbf{RS}:RibSeg~\cite{yang2021ribseg}, \textbf{Hip}: Pelvis CT~\cite{liu2021deep}, \textbf{A}:Amos~\cite{ji2022amos}, \textbf{C}:MAL Cervix~\cite{landman2015miccai}, \textbf{ATM}: ATM Airway Tree Modelling~\cite{zhang2023multi}, \textbf{PARSE}: Pulmonary Artery Segmentation Challenge~\cite{kuanquan_wang_2022_6361906}, \textbf{ST}:SegThor~\cite{lambert2020segthor}, \textbf{Abd}: CT50 Abdomen~\cite{ma2021abdomenct}, \textbf{Rule}: Rule based derived label, \textbf{BCA}: Body composition analysis model~\cite{koitka2021fully}, \textbf{HN}: Private Head-neck dataset. \textbf{Self} refers to the self-annotated labels.}
% \label{Tab:Label_Usage}
% \end{table}

% Figure environment removed

%Brief description of the datasets
\begin{itemize}
    \item \textbf{Pediatric~\cite{jordan2022pediatric}:} This dataset consists of $359$ chest-abdomen-pelvis and abdomen-pelvis CT images of patients between the age of $5$ and $16$ years. It provides $29$ anatomical structures annotated by experts. Patients were selected based on random clinical indications from the university clinic of Children's Wisconsin.
    \item \textbf{Total Segmentator~\cite{wasserthal2022totalsegmentator}:} The TotalSegmentator dataset is large and diverse with $1024$ CT images of different body parts with labels for $104$ anatomical structures. The dataset was collected by randomly sampling from the PACs systems of multiple sites. Its annotation is based on an interactive semi-automatic approach. Here, models are first trained on a few manual annotations. These models infer predictions on unlabeled scans which are lastly refined by an expert. This cycle repeats with an ever-increasing number of training images.
    \item \textbf{SegThor~\cite{lambert2020segthor}:} A dataset consisting of $60$ thoracic CTs collected at the Henri Becquerel Center. The patients were selected based on lung cancer or Hodkin's lymphoma diagnosis. The CTs contain annotations for four organs at risk whose tissues must remain intact during radiation therapy. The annotations of the dataset are provided by an experienced radiotherapist. 
    \item \textbf{CT50Abdomen~\cite{ma2021abdomenct}:} The dataset is part of the CT1k Abdomen datasets extension in which the authors provide 50 abdominal CT images with previously less annotated structures such as the adrenal glands. Annotations are provided by multiple junior annotators and checked by senior radiologists.
    \item \textbf{MAL Cervix~\cite{landman2015miccai}:} This dataset is part of the Beyond the Cranial Vault challenge. It consists of $30$ training and $20$ testing abdominal CT images acquired via a full bladder drinking protocol and annotated by a trained radiation oncologist. It focuses on the digestive and reproductive systems of female cervical cancer patients.
    \item \textbf{Amos~\cite{ji2022amos}:} A diverse dataset with $500$ CT images collected from different scanners and sites covering 15 abdominal organ categories. The selection of patients relates to abdominal tumors or abnormalities examinations. Annotations rely on a combination of junior and senior radiologist labor.
    \item \textbf{RibSeg~\cite{yang2021ribseg}:} The RibSeg dataset consists of $490$ CT Scans taken from publicly available RibFrac~\cite{jin2020deep} dataset. The authors use a semi-automatic morphology-based segmentation approach based on thresholding, point cloud segmentation, and morphological operations. They check the proposed segmentations by hand and refine them if necessary. 
    \item \textbf{Verse ~\cite{sekuboyina2021verse}:} A large dataset for vertebra segmentation. It consists of two subsets and has a total of $374$ CT scans of $355$ patients from multiple detectors and sites with voxel-wise annotations for individual vertebras. Segmentations have been performed semi-automatically with initial proposals being generated by an in-house pipeline. The proposals are refined by a team of trained medical students and experts and finally approved by a radiologist with more than $30$ years of experience. 
    \item \textbf{ATM~\cite{zhang2023multi}:} This dataset establishes a benchmark for Airway Tree Modelling by providing $500$ chest CT scans from different sites and includes scans of healthy patients, patients with pulmonary diseases, and even noisy COVID-19 CTs. Annotations of the pulmonary airways were performed by a team of three experts with each radiologist having more than five years of experience.
    \item \textbf{PARSE~\cite{kuanquan_wang_2022_6361906}:} The PARSE dataset is part of the Pulmonary Artery Segmentation Challenge and contains a total of $203$ CT images from $203$ patients which have been diagnosed with pulmonary nodular diseases. The CTs were generated using devices from two different manufacturers, with data collected from four distinct sites. Each of the images has been annotated by five experts with each expert having at least five years of experience in the field. 
    \item \textbf{Pelvic CT~\cite{liu2021deep}:} A large-scale dataset that focuses on the segmentation of pelvic bone structures such as hip bones or sacrum. It consists of $1184$ CT images collected from different source datasets combining images from multiple sites, scanners, and even metal artifacts. The labeling was conducted by a team of junior and senior radiologists.
    
\end{itemize}

%Include Labels
% \begin{table*}[t]
    \centering
    \small
    \begin{tabular}{lccccccc}
    \toprule
    Source & Series & Labels & Label Domain & Spacing & Slices & IoU \\
    \midrule
         Pediatric~\cite{jordan2022pediatric} & $ 359 $ & 29 & Full Body & [$  0.52 \pm 0.11 , 0.52 \pm 0.11 , 1.76 \pm 0.52  $]  & $ 306.6 \pm 245.4 $ & \todo{-}\\
         
         Total Segmentator~\cite{wasserthal2022totalsegmentator} & $ 1024 $ & 104 & Full Body & [$  1.5 \pm 0.0 , 1.5 \pm 0.0 , 1.5 \pm 0.0  $]  & $ 259.0 \pm 130.3 $ & \todo{-} \\

         SegTHOR~\cite{lambert2020segthor} &  $ 60 $ & 4 & Thoracic & [$  1.00 \pm 0.09 , 1.00 \pm 0.09 , 2.39 \pm 0.23  $]  & $ 184.7 \pm 30.35 $ & \todo{-} \\
         
         CT50Abdomen~\cite{ma2021abdomenct} & $ 50 $ & 13 & Abdominal & [$  0.81 \pm 0.07 , 0.81 \pm 0.07 , 2.63 \pm 0.52  $]  & $ 95.88 \pm 9.000 $& \todo{-}\\

         MAL Cervix~\cite{landman2015miccai} & $ 50 $ & 4 & Reproductive & \todo{-}  &  \todo{-} & \todo{-}\\

         Amos~\cite{ji2022amos} & $ 500 $ & 5 & Abdominal & \todo{-}  & \todo{-} & \todo{-}\\
         
         RibSeg~\cite{yang2021ribseg} & $ 490 $ & 1 & Ribs & [$  0.74 \pm 0.07 , 0.74 \pm 0.07 , 1.13 \pm 0.14  $]  & $ 359.5 \pm 59.06 $ & \todo{-}\\

         Verse~\cite{sekuboyina2021verse} & $374$ & 28 & Spine & [$  0.79 \pm 0.23 , 0.79 \pm 0.23 , 1.29 \pm 0.65  $]  & $ 444.8 \pm 348.6 $ & \todo{-}\\
         
         ATM~\cite{zhang2023multi} & $ 500 $ & 1 & Respiratory & \todo{-}  & \todo{-} & \todo{-} \\
         
         PARSE~\cite{kuanquan_wang_2022_6361906} & $ 203 $ & 1 & Vessels & [$  0.67 \pm 0.07 , 0.67 \pm 0.07 , 0.99 \pm 0.01  $]  & $ 301.4 \pm 31.22 $ & \todo{-} \\
         
         Pelvic CT~\cite{liu2021deep} & $ 1184 $ & \todo{-} & Bones & \todo{-}  & \todo{-} & \todo{-} \\
         
        
        
        
         \bottomrule
    \end{tabular}
    \caption{Comparison of considered CT datasets for datasets used for label aggregation regarding size, number of labels, label domain, volume spacing, number of slices and segmentation performance of a nnUNet~\cite{isensee2021nnu} in 5-fold cross validation.}
    \label{tab:ch5_datasets_overview}
\end{table*}


%Third source of knowledge: Own annotations, BCA and Alex's Labels
Besides the previously described dataset, we leverage non-publicly available datasets and models. One of the models is the body composition analysis model~\cite{koitka2021fully} which differentiates between different types of tissues. From this model, we obtain labels such as \textit{fat} or the general class \textit{muscles}. In total, we extract $9$ labels from the body composition model source. 
A second private source dataset consisting of $104$ diverse head and neck contrast CT images from four different source cohorts~\cite{giske2011local, stoiber2017analyzing, bejarano2019longitudinal, bejarano2018head, clark2013cancer}.
%The total of $104$ images have been annotated by medical students with a focus on anatomical structures required for the. 
This dataset focuses on the diagnosis and treatment of oropharyngeal or hypopharyngeal head and neck cancer and has been annotated by medical students. It provides fine-grained classes for head neck vessels and bone structures. We train a standard nnU-Net on $86$ images to extract the dataset knowledge and add $12$ unique, previously unavailable labels from this dataset, mostly vessels in the head-neck region. 

%Second source of truth: Rules and Remapping
After obtaining the labels from the different nnU-Net predictions, we use anatomically derived rules to refine the current predictions and generate $7$ additional labels. An intuitive example for a new label that can be derived from the combination of obtained labels and medical common sense is the skull. It can be derived from a thresholding procedure obtained by the bone window present in CT images. Bones typically lead to CT values between 350 and 3000 Hounsfield Units (HU) which serve as the described thresholds. The obtained set of voxels can be restricted to the area above the C5 vertebra which previously was obtained. Finally, we remove already predicted vertebras from the thresholded voxels which leaves us with an accurate mask for the skull. We furthermore exploit the behavior of the neural network predictions which have only been trained on parts of the anatomy and typically confuse structures that look similar in the CT images. Common systematic errors are to predict gonads as the eyeballs or colon as the nasal cavity. We exploit these systematic mistakes and remap the produced labels according to the location within the human body. By employing these simple rules we add $7$ additional labels. 


% Figure environment removed


\subsection*{Knowledge Aggregation}
In order to aggregate the predictions of the individual models, we define a common labeling scheme %which is derived from established medical nomenclature standards.  
to which we map the obtained masks. Since some of these labels present multiple versions of the same anatomical structure, such as the class \textit{aorta} which is present in Total Segmentator, Amos, and SegThor it is necessary to combine these predictions. 
%Might need to delete this sentence
%We check the label quality of the models for the individual predictions on randomly sampled volumes and exclude predictions of suboptimal quality. 
Unless stated otherwise, we merge the predictions of models trained on the different source datasets into a single mask which is the union of all individual masks. 
This procedure is simple and stable. It also helps in aggregating masks of the same anatomical structures which are only predicted within certain regions of the human body on which the respective models have been trained. An example of this behavior can be found in the mask for the class \textit{aorta} predicted by the SegThor~\cite{lambert2020segthor} model. While the aorta spans outside the thorax, this model only predicts it within the thorax region. Only by merging the mask for \textit{aorta} of this model with additional masks from other models leaves us with a full mask for the aorta spanning over the entire anatomy. Thus merging these labels combines the knowledge present in different parts of the human body into a single, unified anatomy which is the goal of this work. 

When integrating the different anatomical structures into the Atlas labeling scheme, we aggregate them according to their anatomical hierarchical level from course to fine starting from general tissues such as \textit{muscles} or \textit{fat}. On top, we gradually add the different organs and finally fine-grained vessel structures such as \textit{Pulmonary Arteries}. During the aggregation process, we employ basic anatomical knowledge to improve individual predictions on the fly. One of these operations is that we split the predicted voxels into left and right clusters for paired organs such as the hip bones, kidneys or adrenal glands and resolve conflicts. Furthermore, we restrict predictions based on previous labels or eliminate non-largest connected components if it is deemed appropriate. %Furthermore, we employ basic morphological operations such morphological growing exclude predictions of structures which are likely to be confused such are adjacent bone structures. Examples for this are hip bones and the two femurs

%Volle List der Labens angeben

\subsubsection*{From Aggregated Predictions to a Unified Dataset}
\label{SS: Label Aggregation}
After integrating the labels into the common DAP Atlas CT volumes, it is an integrated dataset, but the different masks are still predictions of models which were trained on heterogeneous source datasets and thus generate heterogeneous masks. To unite these different, integrated masks into a single seamless dataset, we perform one iteration of self-training. The benefits of this procedure are four-fold: As previously mentioned, we bring the labels which originate from datasets of different resolutions into the common Atlas resolution leading to a truly seamless integration. A second reason to perform self-training is to eliminate non-systematic random noise. The network receives consistent feedback from consistent predictions, while noisy predictions are non-systematic. This observation is a well-known fact in image classification~\cite{liu2020early} which states that before the memorization of training data, networks tend to ignore noisy predictions and focus on consistent feedback. We make sure to not overfit the network on the dataset by closely monitoring training and validation losses. A third reason, to perform self-training is to distill the fragmented knowledge into a single model capable to predict the entire anatomy, this massively decreases the necessary time to predict the anatomy, since it reduces the inference time from $n$ expert models to a single model. Finally, self-training hampers the exact reconstruction of private data from expert models which were directly trained on private source datasets. 

%Beschreibung von V1
We generate the first version of the DAP Atlas dataset by applying the obtained unified anatomical model on the selected Atlas target volumes. While the overall label quality is good, we notice certain patterns which were repeatedly done wrong and with which the networks seemed to struggle. These systematic exceptions are the confusion of voxels that belong to paired structures such as the left and right kidney or adjacent vertebrae. Further, we observe implausible predictions of structures within body regions that are not possible, e.g. colon being predicted outside the abdomen. Finally, we observe structures belonging to the reproductive system to be predicted for the wrong sex. 
These errors are relatively easy to correct by once again applying anatomical rules. To address these, we propose Algorithm~\ref{Alg: Post-Processing}. 
We furthermore use two sets of rules to filter out implausible prediction: During Algorithm ~\ref{Alg: Rib counting}: We exclude predictions leading to different orderings induced by median points and minimum points of the ribs. Additionally, we examine the normal vector of the hyperplane during Algorithm~\ref{Alg: Left-Right-Splitting} and exclude predictions leading to hyperplanes that deviate too much from the axial directions. This reduces the number of images in the Atlas dataset from $566$ to $533$ CTs.

After applying Algorithm~\ref{Alg: Post-Processing} to the raw labels, we receive the final version of the dataset, which is rated as very impressive by a consulted radiologist. We describe the extensive validation procedure of the dataset in Section \textit{Technical Validation}. 

While the dataset is convincing, we acknowledge that the performance of the developed anatomical model is dependent on Algorithm~\ref{Alg: Post-Processing} which is undesirable as it requires the availability of anchor predictions which may not always be available for arbitrary CTs. As an additional contribution besides the dataset, we develop a more robust, model based on the available Atlas Knowledge which is more suitable for a clinical environment in which the model has to process arbitrary CT Volumes. We will refer to the previous model used to generate the dataset as the Atlas dataset model (V1) and the novel model as the Atlas prediction model (V2).

\subsubsection*{Developing a Prediction Model from the Atlas Dataset}
The goal of the Atlas prediction model is to eliminate the need for post-processing which is impractical within a clinical setting in which the model should be able to deliver convincing results on arbitrary CT volumes. When examining the different steps of Algorithm~\ref{Alg: Post-Processing}, we notice two steps that are easy to address algorithmically: sex-based consistency and non-largest connected component suppression as defined in Algorithms~\ref{Alg: Sex-based consistency} and~\ref{Alg: Non largest CC supression} respectively, as these methods simply suppress predictions and do not rely on other anchor predictions such as Algorithm~\ref{Alg: Left-Right-Splitting}. We thus aim to develop a training procedure that eliminates the need for left-right splitting (Algorithm~\ref{Alg: Left-Right-Splitting}), area-restrictions (Algorithm~\ref{Alg: Area Restriction}), and rib counting (Algorithm~\ref{Alg: Rib counting}).

To tackle these challenges we develop a custom training strategy for the Atlas prediction model. First, we apply Algorithm ~\ref{Alg: Post-Processing} during the aggregation phase of the individual expert models to maximize the agreement with the desired output which has been approved by experts. Next, we observe that due to the large number of classes, the standard nnU-Net~\cite{isensee2021nnu} learning rate schedule is suboptimal as it closely follows a linear learning rate schedule allocating approximately the same number of epochs for small and large learning rates. We find that the proposed task is more difficult than most standard segmentation tasks and thus increase the number of training epochs from $1000$ to $5000$. Finally, we fine-tune the network for another $1000$ epochs with a fixed learning rate of $0.001$ and without the standard mirror augmentation. This allows the network to focus on the improvements on smaller structures and helps to mitigate the right-left and rib confusion. We show a comparison of the raw output of the Atlas dataset model, the post-processed volume, and the raw output of the Atlas prediction model in Fig.~\ref{fig:V1_V2_comparison}. As it can be seen, the output of the robust model has a large agreement with the post-processed predictions of the first model without relying on Algorithm~\ref{Alg: Post-Processing}. We analyze this behavior and find that the vast majority of predicted structures have an agreement of more than $90\%$ IoU between the post-processed V1 Model and the raw V2 predictions.

Besides the DAP Atlas dataset, we also release the robust segmentation model which can be used to perform inference without post-processing. It furthermore tends to perform better for out-of-distribution tasks which are common within a clinical setting. We examine this behavior in Section \textit{Technical Validation}.

% Figure environment removed

%We examine the same 25 volumes and compare these to previously discussed limitations. 

%TODO Do the comparison The discussion and the figure

%Besides the proposed dataset generated by the previously described procedure and approved by experts, we also release a second version of the dataset which has been generated by the robust model, while there is no substantial difference among the two datasets as they have a mean IOU of almost $90\%$ we do not perform the same costly evaluation procedure twice. We generally recommend the usage of the approved standard V1 dataset and the robust models for custom inference without post-processing, as it tends to perform better for out-of-distribution tasks. We examine this behaviour in Section~\ref{S: Technical Validation}.



\section*{Data Records}
\label{S: Data Record}
%The Data Records section should be used to explain each data record associated with this work, including the repository where this information is stored, and to provide an overview of the data files and their formats. Each external data record should be cited numerically in the text of this section, for example \cite{Hao:gidmaps:2014}, and included in the main reference list as described below. A data citation should also be placed in the subsection of the Methods containing the data-collection or analytical procedure(s) used to derive the corresponding record. Providing a direct link to the dataset may also be helpful to readers (\hyperlink{https://doi.org/10.6084/m9.figshare.853801}{https://doi.org/10.6084/m9.figshare.853801}).

%Tables should be used to support the data records, and should clearly indicate the samples and subjects (study inputs), their provenance, and the experimental manipulations performed on each (please see 'Tables' below). They should also specify the data output resulting from each data-collection or analytical step, should these form part of the archived record.
We make use of the recently published AutoPET dataset~\cite{gatidis2022whole} which can be accessed on The Cancer Imaging Archive (TCIA) under its collection name “FDG-PET-CT-Lesions” to download the raw PET/CT data. We publish the segmentation masks representing the DAP Atlas dataset at \href{https://github.com/alexanderjaus/AtlasDataset}{github}. In our DAP Atlas dataset, we have retained the AutoPET naming convention to ensure that the masks can be easily matched with their corresponding original CT volume.
\vspace{5mm}

\dirtree{%
.1 DAP Atlas Anatomical Labels.
.2 AutoPET\_0011f3deaf\_10445.nii.gz.
.2 AutoPET\_01140d52d8\_56839.nii.gz.
.2 AutoPET\_0143bab87a\_33529.nii.gz.
.2 \dots .
}
\vspace{5mm}
 The given name consists of the subject ID followed by the last $5$ digits of the Study UID which allows a unique matching of the segmentation masks to the AutoPET CTs. 
\section*{Technical Validation}
\label{S: Technical Validation}

As previously discussed, we propose the DAP Atlas dataset as a knowledge aggregation dataset from multiple fragmented source datasets, which are impractical to train neural networks on, as they only offer partial supervision for the presented anatomical structure and label everything else as background. As the DAP Atlas dataset consists of many volumes and is rich in labels, it is nearly impossible to have experts check every voxel for correctness. As previously mentioned, other datasets containing few annotations can still use manual label checking and correction. The Airway Tree Modeling dataset~\cite{zhang2023multi}, which is comparable in size, provides annotations for a single label. With the previously discussed arguments in Section~\textit{Background and Summary}, the creation time was $80$ radiologist days. This example demonstrates that even checking and correcting for the DAP atlas dataset on a voxel level by humans is nearly impossible.
The TotalSegmentator dataset~\cite{wasserthal2022totalsegmentator} proposes to use 2D renderings of 3D organs to improve the required time to check for correctness. This type of shape check gains in speed, but it also does not guarantee the correctness of each voxel. 

\subsection*{Evaluation Setup:}
To tackle the aforementioned problem of evaluation, we propose a hybrid approach combining human experts, anatomical plausibility, and usefulness for the Deep Learning community. 
\begin{itemize}
    \item \textbf{Deep Learning Applicability:} We verify the usefulness of our dataset for the development of Deep Learning algorithms by taking our anatomical segmentation models which were trained on DAP Atlas and perform inference on the BTCV~\cite{landman2015miccai} abdomen dataset. This dataset has not been used for the dataset construction and provides an unbiased performance check. We compare the performance of the Atlas dataset model and the Atlas prediction model. 
    \item \textbf{Expert Checks:} We sample $25$ volumes from the DAP Atlas dataset and let an expert radiologist evaluate them. We ask the radiologist to report on overall label quality and provide insights on potential use.
    \item \textbf{Anatomical Insights of DAP Atlas:} To verify the general, global anatomical plausibility of the dataset, we use the labels of the DAP Atlas dataset to calculate the volumes and mean intensities as characteristic descriptors of the different anatomical structures. We plot these descriptors against the age and gender of the patients and identify if they follow characteristic medical curves. 
    Further, we compare the volume distributions of Atlas organs with several source datasets to investigate the deviation of the different volume distributions. Finally, we investigate which anatomical structures in the Atlas dataset are most affected by which type of cancer.
    
\end{itemize}

By combining these three approaches we combine the thoroughness of local, voxel-wise checks with the scalability of global overall checks and make sure that the dataset introduces merit to the Deep Learning community. 

\subsection*{Results:}
\subsubsection*{Deep Learning Applicability}

% Figure environment removed

Regarding the usefulness of the provided dataset for Deep Learning models, we use our unified anatomical models to predict the Atlas anatomy onto the BTCV~\cite{landman2015miccai} abdomen dataset. This dataset has not been used to construct the DAP Atlas dataset and provides an unbiased performance check. We emphasize, that we do not fine-tune the models using the BTCV training data, but perform inference on the BTCV testset without post-processing. 
We report the performance measures of the Atlas dataset model and the Atlas prediction model in Table~\ref{tab:btcv}. 

Our Atlas dataset model achieves an average dice score of about $81\%$. 
The Atlas prediction model (V2), developed through iterative training and post-processing cycles, achieves an $85\%$ dice score, an improvement of $\sim 4\%$, demonstrating its increased robustness due to the adapted training schedule. $85\%$ dice is on par with state-of-the-art medical segmentation models such as UNETR~\cite{hatamizadeh2022unetr} which are trained in a standard supervised fashion on the training dataset. 
We see that the V2 model shows significant improvements in abdominal structures, i.e. $81\%$ vs $85\%$ for the \textit{vena cava inferior (IVC)} or $75\%$ vs $83\%$ for the pancreas and in particular smaller structures such as \textit{Adrenal Glands}. 
But we also notice small decreases in performance for \textit{Left and Right Kidneys}. 
Regarding the Mean Surface Distance performance, we see an overall improvement of the Atlas V2 model compared to the Atlas dataset model (V1), however, the individual organs improvements do not follow a clear pattern. 
In summary, we find that the DAP Atlas dataset allows for the creation of high-quality anatomical models and our proposed adapted training strategy for the development of a more robust Atlas V2 model as described previously seems to deliver the expected results.

\setlength{\tabcolsep}{2pt}
\begin{table}[h]
\centering
\footnotesize
\begin{tabular}{lccccccccccccccc}
\toprule

& Spleen & RKidney & LKidney & Gallbladder & Esophagus & Liver & Stomach & Aorta & IVC & PV\&SV & Pancreas & RAdrenal & LAdrenal & Total\\
\midrule
\multicolumn{15}{c}{Dice Scores}\\
\midrule
% TotalSegmentor & 0.96 & 0.91 & 0.95 & 0.73 & 0.79 & 0. & 0. & 0. & 0. & 0. & 0. & 0. & 0. & 0.33 \\
Atlas (V1) & 0.96 & 0.91 & 0.94 & 0.62 & 0.72 & 0.97 & 0.84 & 0.91 & 0.81 & 0.76 & 0.75 & 0.64 & 0.59 &  
0.81\\
Atlas (V2) & 0.96 & 0.86 & 0.92 & 0.72 & 0.80 & 0.96 & 0.86 & 0.92 & 0.85 & 0.77  & 0.83 & 0.75 & 0.74 & 
0.85\\
\midrule
\multicolumn{15}{c}{Mean Surface Distance}\\
\midrule
Atlas (V1) & 1.14 & 2.21 & 0.83 & - & 1.68 & 0.79 &  3.26  & 1.51 & 2.03 & 1.48 & 2.53 & 1.30 & 1.67 & 2.03\\
Atlas (V2) & 0.65 & 4.28 & 1.70 & - & 1.38 & 1.21 &  2.50 & 1.29 & 1.87 & 1.77 & 1.51 & 0.80 & 0.90 & 1.85\\
\bottomrule
\hline
\end{tabular}
\caption{Class-wise Dice and Mean Surface Distance performance on the BTCV dataset for the standard Atlas model and the robust Atlas model. Both predictions have not been post-processed. 
%We also include the predictions of the TotalSegmentor model. 
It can clearly be seen that the robust Atlas V2 model benefited from the adapted training procedure.}
\label{tab:btcv}

\end{table}

A qualitative assessment of the label quality and the provided labels beyond the required labels is shown in Fig.~\ref{fig:BTCV Inference}. Our model predictions exhibit a remarkable level of alignment when compared to the ground truth. The only noticeable differences are the slightly smoother surface compared to the ground truth and minor shape differences in the liver.
Beyond that, we also show how the BTCV organs are a well-integrated fraction of the anatomical structures of the human body which our model is able to segment.
\subsubsection*{Expert Checks:}
We include a human expert in the quality check pipeline. 
To gather human feedback on the DAP Atlas dataset, we randomly sampled $25$ volumes and let an expert radiologist evaluate the quality and discuss shortcomings and applications. The following section discusses the feedback we received and displays the strengths and weaknesses of the DAP Atlas dataset.
\\The feedback that we received was mostly positive:

\begin{quote}
\textit{
    Overall, for whole-body segmentation of a normal patient, it's very impressive. [...] It was also good on some patients pointing out a small hiatal hernia. Otherwise, I think it is more useful for medical students and internal medicine doctors who may not be as familiar with anatomy on CT.}
\end{quote}

Besides this general feedback, we gained some insights into the structural mechanics of the dataset, which we summarize in the following. The expert noted that some structures seem to not always be homogeneously segmented, naming predominantly the spinal canal. Further, for tree-like structures such as the pulmonary artery, the fine-grained branch endings lose detail and become under-segmented. Lastly, it was noted, that the borders of abutting abdominal organs are at times offset and differ from the expert's estimation.

% in the limitations in the form of minor systematic errors which we carefully collected from the feedback of the specialist who investigated the label quality on the described sample. 
% %All discovered systematic errors are described in the following and can be taken into consideration when using the Atlas Dataset: 
% We group the feedback according to the risk in poses for potential downstream application and the granularity of the feedback. This 

% \begin{itemize}
%     \item structures not always homogeneously segmented
%     \item finer tree-like structures are sometimes undersegmented
%     \item borders of abutting abdominal structures are sometimes offset
% \end{itemize}

\subsubsection*{Anatomical Insights of DAP Atlas:}
As a first global check, we compare the volume distributions of our proposed DAP Atlas dataset against other datasets which were annotated by experts. The different volume distributions are shown in Fig.~\ref{fig:All_distplots}. We find that our proposed dataset is placed well within the volume distributions of other expert datasets in both distribution shape and distribution support. 

The selection of anatomical structures for which we show the distribution plots is based on the criteria that at least two additional datasets next to the proposed DAP Atlas dataset contain the structure. By observing the distributions we find, that the volume distributions for the same organs in different datasets do vary by small amounts, but the general shape of the distributions are very similar among the datasets. Small differences in the distributions of organ volumes across the datasets are quite plausible and may stem from limited samples, different annotation schemes, or the selection criteria of the patients. For instance, the distributions of organs in the Pediatric dataset tend to be shifted to the left, which can be easily explained since the dataset focuses on patients below the age of $18$ years. Larger variations and in particular distributions deviating towards the origin can stem from CT images only covering parts of organs which is common in the Total Segmentator dataset. When comparing the DAP Atlas dataset to the family of organ distributions, we find it to be well-integrated regarding its distribution support and shape.  
%Include the description of the boxplot

To analytically confirm this, we calculate the Jensen–Shannon Divergence (JSD) as a symmetric, finite measure to calculate the deviations of distributions. For each of the analyzed anatomical structures, we calculate the JSD of a dataset's volume distribution to all other volume distributions within the same structure. We average these values to receive the distribution's average distance to all other distributions. The greater the average value, the more distant the volume distribution is from its peers. Finally, we draw a box-plot to compare the distribution of average JSD distances per dataset. We find that the DAP Atlas dataset is well placed among the other datasets with distributional agreements very similar to those of voxel-wise expert annotated datasets.

As a second global check, we calculate the volume and the mean intensity of each of the anatomical structures in the DAP Atlas dataset and plot them against the age of the patients in Fig.~\ref{fig:Age_distplots}. 

On the left, we show the volume in milliliters of characteristic organs of the DAP Atlas dataset plotted against the age of the patients. We observe that the volumes of organs for female patients tend to be smaller compared to the organs of men. Further, we fit a quadratic model to explain organ volumes as a function of age and find plausible medical relationships. Whereas the volume of the liver follows a downward-facing parabola with increasing and decreasing characteristics, the hearth atrium tends to only increase with progressing age. Both of these behaviors are well-known medical facts~\cite{keller2021right}, confirming the anatomical plausibility of the dataset.

In the middle column of Fig.~\ref{fig:Age_distplots}, we show the calculated mean intensities of the respective structure in the CT volume indicated by our
atlas labels. We choose three exciting examples and observe very plausible curves when examining the relationship between the mean-intensities of the atlas labels and the age of the patients. For instance, the relationship between hip bone density and age almost linearly decreases with increasing age due to osteoporosis. Organ tissues also tend to become less dense with increasing age. Finally, an interesting observation can be derived from the last plot of the middle column in which we examine the reported outlier. The reason for this extreme behavior is dental implants pushing up the mean intensity of the skull.

%In the right column of Fig.~\ref{fig:Age_distplots}, we examine the effect of pathologies on the Atlas labels. We obtain the pathology label of the respective patient from the metadata of the AutoPET dataset and investigate the distribution of the volume of the lung for healthy patients and patients suffering from lung cancer. It can be seen that the distribution is different for the two groups, with the pathological distribution shifted towards the right, indicating larger lung volumes under the presence of cancer.

In the right column, we show an example of a potential future use case in which the Atlas dataset may serve as a cornerstone in the joint investigation of the entire anatomy and pathologies. We calculate which of the known structures in the Atlas dataset are most affected by which type of cancer. For each patient in the Atlas dataset, we examine which anatomical structures are affected by cancer. When determining if an anatomical structure has been affected by cancer, we consider it to be cancerous if there is at least one voxel labeled as cancerous tissue. During this analysis, we do not distinguish between metastasis and primary cancer cells. Finally, we normalize by the total number of patients with the respective disease to obtain how likely it is that an anatomical structure is affected given the respective diagnosis.

% \subsection*{Limitations}
% \label{SS: Limitations}
% As previously elaborated, we proved the overall anatomical plausibility of the dataset as well as its usefullness for deep-learning tasks. Furthermore, we believe in the future of only limited expert supervision as it won't be possible to create large-scale datasets matching those of natural images in the medical field due to the required expertise. Thus any semi-automatic validation approach comes with the downside that the correctness of each individual voxel cannot be guaranteed. 
% %We share this limitation with the Total Segmentator~\cite{wasserthal2022totalsegmentator} dataset, cannot guarantee voxel-wise accuracy. 
% We however want to emphasize, that non-systematic noise behaviour can be handled for instance via the regularization approach described in Section~\ref{SS: Label Aggregation}.




% Figure environment removed

%% Figure environment removed

% Figure environment removed




% \begin{itemize}[noitemsep]
%     \item The spinal canal is sometimes not homogeneously segmented
%     \item Parts of the posterior globes are segmented, but not uniformly among patients
%     \item Suboptimal distinction between breast tissue and body wall fat
%     \item Mistakes where the duodenum and pancreas are abutting each other
%     \item The uterus is sometimes only heterogeneously segmented
%     \item Different fat planes and fat/vessel borders are messy in the abdomen
%     \item Under estimation of the mastoid air cells 
%     \item Border between sinuses and nasal concha/turbinates 
%     \item Only partial segmentation of the vessels in the arms and legs
%     \item The smaller branches of the pulmonary arteries, vein and hepatic vessels are not segmented
%     \item The ileal cecal junction is usually wrong
%    \item The anterior border or posterior border of the stomach are are sometimes confused with parts of the small bowel
% \end{itemize}

%\todo{Die Limitations an beispielen Zeigen und darstellen, dass sie im Vergleich zum Overall positive Feedback minor sind}

\begin{table}[htbp]
  \centering
  \begin{tabular}{|c|c|c|c|c|c|c|c|}
    %\hline
    %\multicolumn{2}{|c|}{Section 1} & \multicolumn{2}{c|}{Section 2} & \multicolumn{2}{c|}{Section 3} & \multicolumn{2}{c|}{Section 4} \\
    \hline
    \textbf{ID} & \textbf{Label} & \textbf{ID} & \textbf{Label} & \textbf{ID} & \textbf{Label} & \textbf{ID} & \textbf{Label} \\
    \hline
    \hline
    0                            & Background                      & 38                           & Iliopsoas Left                  & 75                           & Costa 5 Left                    & 112                          & Iliac Artery Left               \\
1                            & Unknown Tissue                & 39                           & Iliopsoas Right                 & 76                           & Costa 5 Right                   & 113                          & Iliac Artery Right              \\
2                            & Muscles                         & 40                           & Autochthon Left                 & 77                           & Costa 6 Left                    & 114                          & Aorta                           \\
3                            & Fat                             & 41                           & Autochthon Right                & 78                           & Costa 6 Right                   & 115                          & Iliac Vena Left                 \\
4                            & Abdominal Tissue                & 42                           & Skin                            & 79                           & Costa 7 Left                    & 116                          & Iliac Vena Right                \\
5                            & Mediastinal Tissue              & 43                           & Vertebrae C1                    & 80                           & Costa 7 Right                   & 117                          & Inferior Vena Cava              \\
6                            & Esophagus                       & 44                           & Vertebrae C2                    & 81                           & Costa 8 Left                    & 118                          & Portal Vein and Splenic Vein    \\
7                            & Stomach                         & 45                           & Vertebrae C3                    & 82                           & Costa 8 Right                   & 119                          & Celiac Trunk                    \\
8                            & Small Bowel                     & 46                           & Vertebrae C4                    & 83                           & Costa 9 Left                    & 120                          & Lung Lower Lobe Left            \\
9                            & Duodenum                        & 47                           & Vertebrae C5                    & 84                           & Costa 9 Right                   & 121                          & Lung Upper Lobe Left            \\
10                           & Colon                           & 48                           & Vertebrae C6                    & 85                           & Costa 10 Left                   & 122                          & Lung Lower Lobe Right           \\
12                           & Gallbladder                     & 49                           & Vertebrae C7                    & 86                           & Costa 10 Right                  & 123                          & Lung Middle Lobe Right          \\
13                           & Liver                           & 50                           & Vertebrae T1                    & 87                           & Costa 11 Left                   & 124                          & Lung Upper Lobe Right           \\
14                           & Pancreas                        & 51                           & Vertebrae T2                    & 88                           & Costa 11 Right                  & 125                          & Bronchus                        \\
15                           & Kidney Left                     & 52                           & Vertebrae T3                    & 89                           & Costa 12 Left                   & 126                          & Trachea                         \\
16                           & Kidney Right                    & 53                           & Vertebrae T4                    & 90                           & Costa 12 Right                  & 127                          & Pulmonary Artery                \\
17                           & Bladder                         & 54                           & Vertebrae T5                    & 91                           & Rib Cartilage                   & 128                          & Cheek Left                      \\
18                           & Gonads                          & 55                           & Vertebrae T6                    & 92                           & Sternum Corpus                  & 129                          & Cheek Right                     \\
19                           & Prostate                        & 56                           & Vertebrae T7                    & 93                           & Clavicula Left                  & 130                          & Eyeball Left                    \\
20                           & Uterocervix                     & 57                           & Vertebrae T8                    & 94                           & Clavicula Right                 & 131                          & Eyeball Right                   \\
21                           & Uterus                          & 58                           & Vertebrae T9                    & 95                           & Scapula Left                    & 132                          & Nasal Cavity                    \\
22                           & Breast Left                     & 59                           & Vertebrae T10                   & 96                           & Scapula Right                   & 133                          & Artery Common Carotid Right     \\
23                           & Breast Right                    & 60                           & Vertebrae T11                   & 97                           & Humerus Left                    & 134                          & Artery Common Carotid Left      \\
24                           & Spinal Canal                    & 61                           & Vertebrae T12                   & 98                           & Humerus Right                   & 135                          & Sternum Manubrium               \\
25                           & Brain                           & 62                           & Vertebrae L1                    & 99                           & Skull                           & 136                          & Artery Internal Carotid Right   \\
26                           & Spleen                          & 63                           & Vertebrae L2                    & 100                          & Hip Left                        & 137                          & Artery Internal Carotid Left    \\
27                           & Adrenal Gland Left              & 64                           & Vertebrae L3                    & 101                          & Hip Right                       & 138                          & Internal Jugular Vein Right                      \\
28                           & Adrenal Gland Right             & 65                           & Vertebrae L4                    & 102                          & Sacrum                          & 139                          & Internal Jugular Vein Left                       \\
29                           & Thyroid Left                    & 66                           & Vertebrae L5                    & 103                          & Femur Left                      & 140                          & Artery Brachiocephalic          \\
30                           & Thyroid Right                   & 67                           & Costa 1 Left                    & 104                          & Femur Right                     & 141                          & Vein Brachiocephalic Right     \\
31                           & Thymus                          & 68                           & Costa 1 Right                   & 105                          & Heart                           & 142                          & Vein Brachiocephalic Left      \\
32                           & Gluteus Maximus Left            & 69                           & Costa 2 Left                    & 106                          & Heart Atrium Left               & 143                          & Artery Subclavian Right        \\
33                           & Gluteus Maximus Right           & 70                           & Costa 2 Right                   & 107                          & Heart Tissue                    & 144                          & Artery Subclavian Left         \\
34                           & Gluteus Medius Left             & 71                           & Costa 3 Left                    & 108                          & Heart Atrium Right              &                              &                                 \\
35                           & Gluteus Medius Right            & 72                           & Costa 3 Right                   & 109                          & Heart Myocardium                &                              &                                 \\
36                           & Gluteus Minimus Left            & 73                           & Costa 4 Left                    & 110                          & Heart Ventricle Left            &                              &                                 \\
37                           & Gluteus Minimus Right           & 74                           & Costa 4 Right                   & 111                          & Heart Ventricle Right           &                              &                                 \\ \hline
\end{tabular}
\label{Tab: All Labels}
\caption{Full list of available labels in the DAP Atlas dataset. Besides the self-explanatory tissues, we include a class \textit{Unknown Tissue} indicating tissue that likely still needs to be annotated. It typically contains tissue structures that have not been annotated explicitly but were obtained by morphological operations. We still include this class as it has the potential to be useful for future work.}
\end{table}

%Include the following into the text
%This procedure however comes with the downside that due to potential individual dataset specific labeling policies are not preserved in the DAP Atlas dataset.



%TODO: Look in the dataset for examples. Formulate a bit differently

%\begin{itemize}
%    \item The spinal canal is sometimes not homogeneously segmented
%    \item Parts of the posterior globes are segmented, but not uniformly among patients
%    \item Suboptimal distinction between breast tissue and body wall fat
%    \item Mistakes where the duodenum and pancreas are abutting each other
%    \item The uterus is sometimes only heterogeneously segmented
%    \item Different fat planes and fat/vessel borders are messy in the abdomen
%    \item Under estimation of the mastoid air cells 
%    \item Border between sinuses and nasal concha/turbinates 
%    \item Only partial segmentation of the vessels in the arms and legs
%    \item The smaller branches of the pulmonary arteries, vein and hepatic vessels are not segmented
%    \item The ileal cecal junction is usually wrong
 %   \item The anterior border or posterior border of the stomach are are sometimes confused with parts of the small bowel
%\end{itemize}



%-parts of the posterior globes are segmented, but not uniformly among patients, not sure if it is supposed to be the whole globe 
%-On a few, normal lung parenchyma is missed, but often a lung lesion is misclassified as mediastinum in the upper lobes 
%-poor distinction between breast tissue and body wall fat (heterogeneously labeled in the women) 
%-mistakes where the duodenum and pancreas are abutting each other 
%-the uterus is only heterogeneously segmented 
%-The different fat planes and fat/vessel borders are messy in the abdomen 
%-under estimation of the mastoid air cells 
%-border between sinuses and nasal concha/turbinates 
%-only partial segmentation of the vessels in the arms and legs
%-the smaller branches of the pulmonary arteries/vein and hepatic vessels are not segmented
%-ileal cecal junction is usually wrong 
%-often the anterior border or posterior border of the stomach are given a different color, or parts of the small bowel are missed 
%-sometimes the spinal canal is not homogeneously segmented 


%Include Feedback from Calsey
%Point out that our new labels cannot be checked and copared by the previously mentioned radiological feature comparions and volume comparsisons. 
%We cannot guarantee that voxel-wise labels are correct for all the 

%\section*{Usage Notes}

%The Usage Notes should contain brief instructions to assist other researchers with reuse of the data. This may include discussion of software packages that are suitable for analysing the assay data files, suggested downstream processing steps (e.g. normalization, etc.), or tips for integrating or comparing the data records with other datasets. Authors are encouraged to provide code, programs or data-processing workflows if they may help others understand or use the data. Please see our code availability policy for advice on supplying custom code alongside Data Descriptor manuscripts.

%For studies involving privacy or safety controls on public access to the data, this section should describe in detail these controls, including how authors can apply to access the data, what criteria will be used to determine who may access the data, and any limitations on data use. 

\section{Conclusion}
\label{sec:conclusion}


We presented DCA, an algorithm to address disparity in outcomes of ranking processes using compensatory bonus points. We showed that DCA, by relying on a sampling-based approach, successfully reduces disparity in a wide range of settings, while being significantly more efficient than state-of-the-art approaches, running in sub-linear time. This makes DCA a good candidate for iterative processes that would allow users to identify the ranking function that best fits their needs while checking for its fairness impacts and the required compensatory bonus points.  


Our approach relies on the use of compensatory bonus points, a departure from previous work, which has mostly focused on modifying the ranking function directly, or on the use of quotas. A significant advantage of compensatory bonus points is that they are transparent, interpretable, and easily explainable to all stakeholders.



%\section{Conclusion}\label{sec:conclusion}

This paper presents our empirical domain knowledge distillation framework using ChatGPT and discusses our observations from the framework application experiments in the autonomous driving domain. The key finding is that: 1) with proper design of prompt engineering and execution flow, fully automated domain knowledge (in the ontology format) distillation is possible. However, due to the randomness in the response and the butterfly effect, the quality of fully automated distillation results is not guaranteed. To address this, we develop a web-based assistant to enable manual supervision and early intervention at runtime. We hope our findings and tools inspire future research toward revolutionizing the engineering processes of knowledge-based systems across domains.



\section*{Usage Notes}
The AutoPET dataset~\cite{gatidis2022whole} contains three different image types for each examination: The PET and the corresponding SUV, the tumor segmentation mask, and two different CT images. One of them is in the original resolution whereas the second one is resampled to match the PET/SUV resolution. DAP Atlas is based on the original CT image resolution, however, resampling of the masks can be performed to match the PET resolution for future multimodel PET/CT work. 

\section*{Code availability}

The code for the dataset aggregation and for the post-processing will be made publicly available under \href{https://github.com/alexanderjaus/AtlasDataset}{github}. 
Besides the dataset, we publish the trained models V1 as well as the robust model V2 at \href{https://github.com/alexanderjaus/AtlasDataset}{github}. 

\section*{Conflicts of Interest}
The authors disclose no conflicts of interest.

\section*{Informed Consent Statement:}
The DAP Atlas uses a subset of the AutoPET dataset. As the AutoPET dataset was part of a public competition, ethical approval from our side was not required, as confirmed by the license attached with the open-access data. 
\bibliography{main}

%\noindent LaTeX formats citations and references automatically using the bibliography records in your .bib file, which you can edit via the project menu. Use the cite command for an inline citation, e.g. \cite{Kaufman2020, Figueredo:2009dg, Babichev2002, behringer2014manipulating}. For data citations of datasets uploaded to e.g. \emph{figshare}, please use the \verb|howpublished| option in the bib entry to specify the platform and the link, as in the \verb|Hao:gidmaps:2014| example in the sample bibliography file. For journal articles, DOIs should be included for works in press that do not yet have volume or page numbers. For other journal articles, DOIs should be included uniformly for all articles or not at all. We recommend that you encode all DOIs in your bibtex database as full URLs, e.g. https://doi.org/10.1007/s12110-009-9068-2.

\section*{Acknowledgements} %(not compulsory)
%Check if HIDS4HEALTH needs to be acknowledged
The present contribution is supported by the Helmholtz Association under the joint research school “HIDSS4Health – Helmholtz Information and Data Science School for Health”.
It was performed on the HoreKa supercomputer funded by the Ministry of Science, Research and the Arts Baden-Württemberg and by the Federal Ministry of Education and Research and is supported by the Helmholtz Association Initiative and Networking Fund on the HAICORE@KIT partition. 

\section*{Author contributions statement}
A.J., C.S.: conception, design and creation of the Atlas dataset and evaluation procedure. 
K.H., J.H.: Human Feedback of the dataset.
A.W., K.G.: data collection of private Head-Neck dataset and training of the expert model.  
J.K., R.S.: discussion and supervision of the project.

%Must include all authors, identified by initials, for example:
%A.A. conceived the experiment(s), A.A. and B.A. conducted the experiment(s), C.A. and D.A. analysed the results. All authors reviewed the manuscript. 

%\section*{Competing interests} (mandatory statement)

%The corresponding author is responsible for providing a \href{https://www.nature.com/sdata/policies/editorial-and-publishing-policies#competing}{competing interests statement} on behalf of all authors of the paper. This statement must be included in the submitted article file.

%\section*{Figures \& Tables}

%Figures, tables, and their legends, should be included at the end of the document. Figures and tables can be referenced in \LaTeX{} using the ref command, e.g. Figure \ref{fig:stream} and Table \ref{tab:example}. 

%Authors are encouraged to provide one or more tables that provide basic information on the main ‘inputs’ to the study (e.g. samples, participants, or information sources) and the main data outputs of the study. Tables in the manuscript should generally not be used to present primary data (i.e. measurements). Tables containing primary data should be submitted to an appropriate data repository.

%Tables may be provided within the \LaTeX{} document or as separate files (tab-delimited text or Excel files). Legends, where needed, should be included here. Generally, a Data Descriptor should have fewer than ten Tables, but more may be allowed when needed. Tables may be of any size, but only Tables which fit onto a single printed page will be included in the PDF version of the article (up to a maximum of three). 

%Due to typesetting constraints, tables that do not fit onto a single A4 page cannot be included in the PDF version of the article and will be made available in the online version only. Any such tables must be labelled in the text as ‘Online-only’ tables and numbered separately from the main table list e.g. ‘Table 1, Table 2, Online-only Table 1’ etc.

\end{document}