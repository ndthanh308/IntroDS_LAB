\section*{Discussion and Conclusion}
\label{S: Discussion}
% - Retraining besser als normal
% - Spezifische -> Allgemein
% - Menschliches Feedback
% - Expert-free annotation mechanism 
% - 
% \todo{Brauchen wir wirklich nochmal eine Discussion}
In this work, we generated the full-body DAP Atlas dataset through the aggregation of fragmented knowledge in combination with self-training, guided by medicine-derived rules.  
The dataset enables the training of holistic anatomy segmentation models in CT, which we evaluated through performances in transfer learning, expert inspection, and comparability to established knowledge about the human body.

We have shown that the proposed DAP Atlas dataset carries valuable, accurate anatomical knowledge which can be used to train neural networks capable to perform high-quality segmentation on previously unseen data. The proposed robust training procedure of the Atlas prediction model outperforms the standard nnU-Net training in out-of-distribution tasks confirming its usefulness for general prediction tasks. 

We believe in the future of only limited expert supervision as it will no longer be possible to create large-scale datasets matching those of natural images in the medical field due to the required expertise. Thus, any semi-automatic validation approach comes with the downside that the correctness of each individual voxel cannot be guaranteed. 
However, we want to emphasize, that non-systematic noise behavior can be handled for instance via the regularization approach described in Section \textit{Methods}.

Based on the global checks and the Atlas Model's strong performance on BTCV, we can confidently say that the expert's favorable opinion likely applies to the other CT volumes that have not undergone voxel-wise analysis. 

% Part1 :
% Diskussion aus den Ergebnissen in die Discussion 
% Part2:
% Bragging 

% the performance of a retraining method to the standard training approach for a natural language processing task. Our results showed that retraining the model with additional data was significantly better than the traditional approach.

% We also explored the impact of using specific domain data versus more general data for training the model. Interestingly, we found that training with specific data first and then generalizing to more general data yielded better results than training solely on general data. This suggests that a specific-to-general approach may be beneficial in certain contexts.

% In addition to training approaches, we investigated the use of human feedback to improve the model's performance. We found that incorporating feedback from human annotators led to a significant improvement in the model's accuracy. This underscores the importance of incorporating human expertise and judgement in machine learning processes.

% Finally, we evaluated the effectiveness of an expert-free annotation mechanism for data labeling. Our results indicated that this approach was effective and yielded results comparable to expert-labeled data. This suggests that expert-free annotation mechanisms could be a promising avenue for improving the efficiency and scalability of data labeling processes in natural language processing tasks.

Our proposed expert-free dataset generation approach was used to construct our full-body DAP Atlas dataset which is a large, full-body CT dataset with dense annotations for most of the important anatomical structures. Our methodology involves consolidating existing but scattered information found in partially annotated datasets, using self-training methods guided by medical principles. 

We point out, that the future of large-scale medical datasets will most likely be expert free at least in the sense of voxel-wise alignment checks due to the inconvenient combination of limited experts in the medical field and required expertise to guarantee voxel-wise correctness. Thus, we propose a hybrid approach to evaluate the proposed DAP Atlas dataset, partially combining voxel-wise alignment checks on randomly sampled subsets and automated checks. We argue that due to the global overall anatomical consistency and the positive feedback on a random sample, the overall quality of the dataset is convincing which is further supported by the already mentioned impressive performance on the BTCV benchmark without leveraging its training dataset. 

%We furthermore demonstrate the usefullness of the DAP Atlas dataset by taking the dataset generating models to the test on a public benchmark dataset which has not been used to generate the dataset. We receive impressive performances on the benchmark without leveraging the training data of the benchmark dataset.

To foster progress on dense anatomical predictions in the medical community, both, the dataset and the models will be made publicly available.

Overall, we believe that this work is a first step towards automated large-scale dataset generation and validation with the usage of anatomical knowledge to overcome the limited expert problem within the medical field.
