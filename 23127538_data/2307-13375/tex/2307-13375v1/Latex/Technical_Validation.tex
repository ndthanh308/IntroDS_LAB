\section*{Technical Validation}
\label{S: Technical Validation}

As previously discussed, we propose the DAP Atlas dataset as a knowledge aggregation dataset from multiple fragmented source datasets, which are impractical to train neural networks on, as they only offer partial supervision for the presented anatomical structure and label everything else as background. As the DAP Atlas dataset consists of many volumes and is rich in labels, it is nearly impossible to have experts check every voxel for correctness. As previously mentioned, other datasets containing few annotations can still use manual label checking and correction. The Airway Tree Modeling dataset~\cite{zhang2023multi}, which is comparable in size, provides annotations for a single label. With the previously discussed arguments in Section~\textit{Background and Summary}, the creation time was $80$ radiologist days. This example demonstrates that even checking and correcting for the DAP atlas dataset on a voxel level by humans is nearly impossible.
The TotalSegmentator dataset~\cite{wasserthal2022totalsegmentator} proposes to use 2D renderings of 3D organs to improve the required time to check for correctness. This type of shape check gains in speed, but it also does not guarantee the correctness of each voxel. 

\subsection*{Evaluation Setup:}
To tackle the aforementioned problem of evaluation, we propose a hybrid approach combining human experts, anatomical plausibility, and usefulness for the Deep Learning community. 
\begin{itemize}
    \item \textbf{Deep Learning Applicability:} We verify the usefulness of our dataset for the development of Deep Learning algorithms by taking our anatomical segmentation models which were trained on DAP Atlas and perform inference on the BTCV~\cite{landman2015miccai} abdomen dataset. This dataset has not been used for the dataset construction and provides an unbiased performance check. We compare the performance of the Atlas dataset model and the Atlas prediction model. 
    \item \textbf{Expert Checks:} We sample $25$ volumes from the DAP Atlas dataset and let an expert radiologist evaluate them. We ask the radiologist to report on overall label quality and provide insights on potential use.
    \item \textbf{Anatomical Insights of DAP Atlas:} To verify the general, global anatomical plausibility of the dataset, we use the labels of the DAP Atlas dataset to calculate the volumes and mean intensities as characteristic descriptors of the different anatomical structures. We plot these descriptors against the age and gender of the patients and identify if they follow characteristic medical curves. 
    Further, we compare the volume distributions of Atlas organs with several source datasets to investigate the deviation of the different volume distributions. Finally, we investigate which anatomical structures in the Atlas dataset are most affected by which type of cancer.
    
\end{itemize}

By combining these three approaches we combine the thoroughness of local, voxel-wise checks with the scalability of global overall checks and make sure that the dataset introduces merit to the Deep Learning community. 

\subsection*{Results:}
\subsubsection*{Deep Learning Applicability}

% Figure environment removed

Regarding the usefulness of the provided dataset for Deep Learning models, we use our unified anatomical models to predict the Atlas anatomy onto the BTCV~\cite{landman2015miccai} abdomen dataset. This dataset has not been used to construct the DAP Atlas dataset and provides an unbiased performance check. We emphasize, that we do not fine-tune the models using the BTCV training data, but perform inference on the BTCV testset without post-processing. 
We report the performance measures of the Atlas dataset model and the Atlas prediction model in Table~\ref{tab:btcv}. 

Our Atlas dataset model achieves an average dice score of about $81\%$. 
The Atlas prediction model (V2), developed through iterative training and post-processing cycles, achieves an $85\%$ dice score, an improvement of $\sim 4\%$, demonstrating its increased robustness due to the adapted training schedule. $85\%$ dice is on par with state-of-the-art medical segmentation models such as UNETR~\cite{hatamizadeh2022unetr} which are trained in a standard supervised fashion on the training dataset. 
We see that the V2 model shows significant improvements in abdominal structures, i.e. $81\%$ vs $85\%$ for the \textit{vena cava inferior (IVC)} or $75\%$ vs $83\%$ for the pancreas and in particular smaller structures such as \textit{Adrenal Glands}. 
But we also notice small decreases in performance for \textit{Left and Right Kidneys}. 
Regarding the Mean Surface Distance performance, we see an overall improvement of the Atlas V2 model compared to the Atlas dataset model (V1), however, the individual organs improvements do not follow a clear pattern. 
In summary, we find that the DAP Atlas dataset allows for the creation of high-quality anatomical models and our proposed adapted training strategy for the development of a more robust Atlas V2 model as described previously seems to deliver the expected results.

\setlength{\tabcolsep}{2pt}
\begin{table}[h]
\centering
\footnotesize
\begin{tabular}{lccccccccccccccc}
\toprule

& Spleen & RKidney & LKidney & Gallbladder & Esophagus & Liver & Stomach & Aorta & IVC & PV\&SV & Pancreas & RAdrenal & LAdrenal & Total\\
\midrule
\multicolumn{15}{c}{Dice Scores}\\
\midrule
% TotalSegmentor & 0.96 & 0.91 & 0.95 & 0.73 & 0.79 & 0. & 0. & 0. & 0. & 0. & 0. & 0. & 0. & 0.33 \\
Atlas (V1) & 0.96 & 0.91 & 0.94 & 0.62 & 0.72 & 0.97 & 0.84 & 0.91 & 0.81 & 0.76 & 0.75 & 0.64 & 0.59 &  
0.81\\
Atlas (V2) & 0.96 & 0.86 & 0.92 & 0.72 & 0.80 & 0.96 & 0.86 & 0.92 & 0.85 & 0.77  & 0.83 & 0.75 & 0.74 & 
0.85\\
\midrule
\multicolumn{15}{c}{Mean Surface Distance}\\
\midrule
Atlas (V1) & 1.14 & 2.21 & 0.83 & - & 1.68 & 0.79 &  3.26  & 1.51 & 2.03 & 1.48 & 2.53 & 1.30 & 1.67 & 2.03\\
Atlas (V2) & 0.65 & 4.28 & 1.70 & - & 1.38 & 1.21 &  2.50 & 1.29 & 1.87 & 1.77 & 1.51 & 0.80 & 0.90 & 1.85\\
\bottomrule
\hline
\end{tabular}
\caption{Class-wise Dice and Mean Surface Distance performance on the BTCV dataset for the standard Atlas model and the robust Atlas model. Both predictions have not been post-processed. 
%We also include the predictions of the TotalSegmentor model. 
It can clearly be seen that the robust Atlas V2 model benefited from the adapted training procedure.}
\label{tab:btcv}

\end{table}

A qualitative assessment of the label quality and the provided labels beyond the required labels is shown in Fig.~\ref{fig:BTCV Inference}. Our model predictions exhibit a remarkable level of alignment when compared to the ground truth. The only noticeable differences are the slightly smoother surface compared to the ground truth and minor shape differences in the liver.
Beyond that, we also show how the BTCV organs are a well-integrated fraction of the anatomical structures of the human body which our model is able to segment.
\subsubsection*{Expert Checks:}
We include a human expert in the quality check pipeline. 
To gather human feedback on the DAP Atlas dataset, we randomly sampled $25$ volumes and let an expert radiologist evaluate the quality and discuss shortcomings and applications. The following section discusses the feedback we received and displays the strengths and weaknesses of the DAP Atlas dataset.
\\The feedback that we received was mostly positive:

\begin{quote}
\textit{
    Overall, for whole-body segmentation of a normal patient, it's very impressive. [...] It was also good on some patients pointing out a small hiatal hernia. Otherwise, I think it is more useful for medical students and internal medicine doctors who may not be as familiar with anatomy on CT.}
\end{quote}

Besides this general feedback, we gained some insights into the structural mechanics of the dataset, which we summarize in the following. The expert noted that some structures seem to not always be homogeneously segmented, naming predominantly the spinal canal. Further, for tree-like structures such as the pulmonary artery, the fine-grained branch endings lose detail and become under-segmented. Lastly, it was noted, that the borders of abutting abdominal organs are at times offset and differ from the expert's estimation.

% in the limitations in the form of minor systematic errors which we carefully collected from the feedback of the specialist who investigated the label quality on the described sample. 
% %All discovered systematic errors are described in the following and can be taken into consideration when using the Atlas Dataset: 
% We group the feedback according to the risk in poses for potential downstream application and the granularity of the feedback. This 

% \begin{itemize}
%     \item structures not always homogeneously segmented
%     \item finer tree-like structures are sometimes undersegmented
%     \item borders of abutting abdominal structures are sometimes offset
% \end{itemize}

\subsubsection*{Anatomical Insights of DAP Atlas:}
As a first global check, we compare the volume distributions of our proposed DAP Atlas dataset against other datasets which were annotated by experts. The different volume distributions are shown in Fig.~\ref{fig:All_distplots}. We find that our proposed dataset is placed well within the volume distributions of other expert datasets in both distribution shape and distribution support. 

The selection of anatomical structures for which we show the distribution plots is based on the criteria that at least two additional datasets next to the proposed DAP Atlas dataset contain the structure. By observing the distributions we find, that the volume distributions for the same organs in different datasets do vary by small amounts, but the general shape of the distributions are very similar among the datasets. Small differences in the distributions of organ volumes across the datasets are quite plausible and may stem from limited samples, different annotation schemes, or the selection criteria of the patients. For instance, the distributions of organs in the Pediatric dataset tend to be shifted to the left, which can be easily explained since the dataset focuses on patients below the age of $18$ years. Larger variations and in particular distributions deviating towards the origin can stem from CT images only covering parts of organs which is common in the Total Segmentator dataset. When comparing the DAP Atlas dataset to the family of organ distributions, we find it to be well-integrated regarding its distribution support and shape.  
%Include the description of the boxplot

To analytically confirm this, we calculate the Jensen–Shannon Divergence (JSD) as a symmetric, finite measure to calculate the deviations of distributions. For each of the analyzed anatomical structures, we calculate the JSD of a dataset's volume distribution to all other volume distributions within the same structure. We average these values to receive the distribution's average distance to all other distributions. The greater the average value, the more distant the volume distribution is from its peers. Finally, we draw a box-plot to compare the distribution of average JSD distances per dataset. We find that the DAP Atlas dataset is well placed among the other datasets with distributional agreements very similar to those of voxel-wise expert annotated datasets.

As a second global check, we calculate the volume and the mean intensity of each of the anatomical structures in the DAP Atlas dataset and plot them against the age of the patients in Fig.~\ref{fig:Age_distplots}. 

On the left, we show the volume in milliliters of characteristic organs of the DAP Atlas dataset plotted against the age of the patients. We observe that the volumes of organs for female patients tend to be smaller compared to the organs of men. Further, we fit a quadratic model to explain organ volumes as a function of age and find plausible medical relationships. Whereas the volume of the liver follows a downward-facing parabola with increasing and decreasing characteristics, the hearth atrium tends to only increase with progressing age. Both of these behaviors are well-known medical facts~\cite{keller2021right}, confirming the anatomical plausibility of the dataset.

In the middle column of Fig.~\ref{fig:Age_distplots}, we show the calculated mean intensities of the respective structure in the CT volume indicated by our
atlas labels. We choose three exciting examples and observe very plausible curves when examining the relationship between the mean-intensities of the atlas labels and the age of the patients. For instance, the relationship between hip bone density and age almost linearly decreases with increasing age due to osteoporosis. Organ tissues also tend to become less dense with increasing age. Finally, an interesting observation can be derived from the last plot of the middle column in which we examine the reported outlier. The reason for this extreme behavior is dental implants pushing up the mean intensity of the skull.

%In the right column of Fig.~\ref{fig:Age_distplots}, we examine the effect of pathologies on the Atlas labels. We obtain the pathology label of the respective patient from the metadata of the AutoPET dataset and investigate the distribution of the volume of the lung for healthy patients and patients suffering from lung cancer. It can be seen that the distribution is different for the two groups, with the pathological distribution shifted towards the right, indicating larger lung volumes under the presence of cancer.

In the right column, we show an example of a potential future use case in which the Atlas dataset may serve as a cornerstone in the joint investigation of the entire anatomy and pathologies. We calculate which of the known structures in the Atlas dataset are most affected by which type of cancer. For each patient in the Atlas dataset, we examine which anatomical structures are affected by cancer. When determining if an anatomical structure has been affected by cancer, we consider it to be cancerous if there is at least one voxel labeled as cancerous tissue. During this analysis, we do not distinguish between metastasis and primary cancer cells. Finally, we normalize by the total number of patients with the respective disease to obtain how likely it is that an anatomical structure is affected given the respective diagnosis.

% \subsection*{Limitations}
% \label{SS: Limitations}
% As previously elaborated, we proved the overall anatomical plausibility of the dataset as well as its usefullness for deep-learning tasks. Furthermore, we believe in the future of only limited expert supervision as it won't be possible to create large-scale datasets matching those of natural images in the medical field due to the required expertise. Thus any semi-automatic validation approach comes with the downside that the correctness of each individual voxel cannot be guaranteed. 
% %We share this limitation with the Total Segmentator~\cite{wasserthal2022totalsegmentator} dataset, cannot guarantee voxel-wise accuracy. 
% We however want to emphasize, that non-systematic noise behaviour can be handled for instance via the regularization approach described in Section~\ref{SS: Label Aggregation}.




% Figure environment removed

%% Figure environment removed

% Figure environment removed




% \begin{itemize}[noitemsep]
%     \item The spinal canal is sometimes not homogeneously segmented
%     \item Parts of the posterior globes are segmented, but not uniformly among patients
%     \item Suboptimal distinction between breast tissue and body wall fat
%     \item Mistakes where the duodenum and pancreas are abutting each other
%     \item The uterus is sometimes only heterogeneously segmented
%     \item Different fat planes and fat/vessel borders are messy in the abdomen
%     \item Under estimation of the mastoid air cells 
%     \item Border between sinuses and nasal concha/turbinates 
%     \item Only partial segmentation of the vessels in the arms and legs
%     \item The smaller branches of the pulmonary arteries, vein and hepatic vessels are not segmented
%     \item The ileal cecal junction is usually wrong
%    \item The anterior border or posterior border of the stomach are are sometimes confused with parts of the small bowel
% \end{itemize}

%\todo{Die Limitations an beispielen Zeigen und darstellen, dass sie im Vergleich zum Overall positive Feedback minor sind}

\begin{table}[htbp]
  \centering
  \begin{tabular}{|c|c|c|c|c|c|c|c|}
    %\hline
    %\multicolumn{2}{|c|}{Section 1} & \multicolumn{2}{c|}{Section 2} & \multicolumn{2}{c|}{Section 3} & \multicolumn{2}{c|}{Section 4} \\
    \hline
    \textbf{ID} & \textbf{Label} & \textbf{ID} & \textbf{Label} & \textbf{ID} & \textbf{Label} & \textbf{ID} & \textbf{Label} \\
    \hline
    \hline
    0                            & Background                      & 38                           & Iliopsoas Left                  & 75                           & Costa 5 Left                    & 112                          & Iliac Artery Left               \\
1                            & Unknown Tissue                & 39                           & Iliopsoas Right                 & 76                           & Costa 5 Right                   & 113                          & Iliac Artery Right              \\
2                            & Muscles                         & 40                           & Autochthon Left                 & 77                           & Costa 6 Left                    & 114                          & Aorta                           \\
3                            & Fat                             & 41                           & Autochthon Right                & 78                           & Costa 6 Right                   & 115                          & Iliac Vena Left                 \\
4                            & Abdominal Tissue                & 42                           & Skin                            & 79                           & Costa 7 Left                    & 116                          & Iliac Vena Right                \\
5                            & Mediastinal Tissue              & 43                           & Vertebrae C1                    & 80                           & Costa 7 Right                   & 117                          & Inferior Vena Cava              \\
6                            & Esophagus                       & 44                           & Vertebrae C2                    & 81                           & Costa 8 Left                    & 118                          & Portal Vein and Splenic Vein    \\
7                            & Stomach                         & 45                           & Vertebrae C3                    & 82                           & Costa 8 Right                   & 119                          & Celiac Trunk                    \\
8                            & Small Bowel                     & 46                           & Vertebrae C4                    & 83                           & Costa 9 Left                    & 120                          & Lung Lower Lobe Left            \\
9                            & Duodenum                        & 47                           & Vertebrae C5                    & 84                           & Costa 9 Right                   & 121                          & Lung Upper Lobe Left            \\
10                           & Colon                           & 48                           & Vertebrae C6                    & 85                           & Costa 10 Left                   & 122                          & Lung Lower Lobe Right           \\
12                           & Gallbladder                     & 49                           & Vertebrae C7                    & 86                           & Costa 10 Right                  & 123                          & Lung Middle Lobe Right          \\
13                           & Liver                           & 50                           & Vertebrae T1                    & 87                           & Costa 11 Left                   & 124                          & Lung Upper Lobe Right           \\
14                           & Pancreas                        & 51                           & Vertebrae T2                    & 88                           & Costa 11 Right                  & 125                          & Bronchus                        \\
15                           & Kidney Left                     & 52                           & Vertebrae T3                    & 89                           & Costa 12 Left                   & 126                          & Trachea                         \\
16                           & Kidney Right                    & 53                           & Vertebrae T4                    & 90                           & Costa 12 Right                  & 127                          & Pulmonary Artery                \\
17                           & Bladder                         & 54                           & Vertebrae T5                    & 91                           & Rib Cartilage                   & 128                          & Cheek Left                      \\
18                           & Gonads                          & 55                           & Vertebrae T6                    & 92                           & Sternum Corpus                  & 129                          & Cheek Right                     \\
19                           & Prostate                        & 56                           & Vertebrae T7                    & 93                           & Clavicula Left                  & 130                          & Eyeball Left                    \\
20                           & Uterocervix                     & 57                           & Vertebrae T8                    & 94                           & Clavicula Right                 & 131                          & Eyeball Right                   \\
21                           & Uterus                          & 58                           & Vertebrae T9                    & 95                           & Scapula Left                    & 132                          & Nasal Cavity                    \\
22                           & Breast Left                     & 59                           & Vertebrae T10                   & 96                           & Scapula Right                   & 133                          & Artery Common Carotid Right     \\
23                           & Breast Right                    & 60                           & Vertebrae T11                   & 97                           & Humerus Left                    & 134                          & Artery Common Carotid Left      \\
24                           & Spinal Canal                    & 61                           & Vertebrae T12                   & 98                           & Humerus Right                   & 135                          & Sternum Manubrium               \\
25                           & Brain                           & 62                           & Vertebrae L1                    & 99                           & Skull                           & 136                          & Artery Internal Carotid Right   \\
26                           & Spleen                          & 63                           & Vertebrae L2                    & 100                          & Hip Left                        & 137                          & Artery Internal Carotid Left    \\
27                           & Adrenal Gland Left              & 64                           & Vertebrae L3                    & 101                          & Hip Right                       & 138                          & Internal Jugular Vein Right                      \\
28                           & Adrenal Gland Right             & 65                           & Vertebrae L4                    & 102                          & Sacrum                          & 139                          & Internal Jugular Vein Left                       \\
29                           & Thyroid Left                    & 66                           & Vertebrae L5                    & 103                          & Femur Left                      & 140                          & Artery Brachiocephalic          \\
30                           & Thyroid Right                   & 67                           & Costa 1 Left                    & 104                          & Femur Right                     & 141                          & Vein Brachiocephalic Right     \\
31                           & Thymus                          & 68                           & Costa 1 Right                   & 105                          & Heart                           & 142                          & Vein Brachiocephalic Left      \\
32                           & Gluteus Maximus Left            & 69                           & Costa 2 Left                    & 106                          & Heart Atrium Left               & 143                          & Artery Subclavian Right        \\
33                           & Gluteus Maximus Right           & 70                           & Costa 2 Right                   & 107                          & Heart Tissue                    & 144                          & Artery Subclavian Left         \\
34                           & Gluteus Medius Left             & 71                           & Costa 3 Left                    & 108                          & Heart Atrium Right              &                              &                                 \\
35                           & Gluteus Medius Right            & 72                           & Costa 3 Right                   & 109                          & Heart Myocardium                &                              &                                 \\
36                           & Gluteus Minimus Left            & 73                           & Costa 4 Left                    & 110                          & Heart Ventricle Left            &                              &                                 \\
37                           & Gluteus Minimus Right           & 74                           & Costa 4 Right                   & 111                          & Heart Ventricle Right           &                              &                                 \\ \hline
\end{tabular}
\label{Tab: All Labels}
\caption{Full list of available labels in the DAP Atlas dataset. Besides the self-explanatory tissues, we include a class \textit{Unknown Tissue} indicating tissue that likely still needs to be annotated. It typically contains tissue structures that have not been annotated explicitly but were obtained by morphological operations. We still include this class as it has the potential to be useful for future work.}
\end{table}

%Include the following into the text
%This procedure however comes with the downside that due to potential individual dataset specific labeling policies are not preserved in the DAP Atlas dataset.



%TODO: Look in the dataset for examples. Formulate a bit differently

%\begin{itemize}
%    \item The spinal canal is sometimes not homogeneously segmented
%    \item Parts of the posterior globes are segmented, but not uniformly among patients
%    \item Suboptimal distinction between breast tissue and body wall fat
%    \item Mistakes where the duodenum and pancreas are abutting each other
%    \item The uterus is sometimes only heterogeneously segmented
%    \item Different fat planes and fat/vessel borders are messy in the abdomen
%    \item Under estimation of the mastoid air cells 
%    \item Border between sinuses and nasal concha/turbinates 
%    \item Only partial segmentation of the vessels in the arms and legs
%    \item The smaller branches of the pulmonary arteries, vein and hepatic vessels are not segmented
%    \item The ileal cecal junction is usually wrong
 %   \item The anterior border or posterior border of the stomach are are sometimes confused with parts of the small bowel
%\end{itemize}



%-parts of the posterior globes are segmented, but not uniformly among patients, not sure if it is supposed to be the whole globe 
%-On a few, normal lung parenchyma is missed, but often a lung lesion is misclassified as mediastinum in the upper lobes 
%-poor distinction between breast tissue and body wall fat (heterogeneously labeled in the women) 
%-mistakes where the duodenum and pancreas are abutting each other 
%-the uterus is only heterogeneously segmented 
%-The different fat planes and fat/vessel borders are messy in the abdomen 
%-under estimation of the mastoid air cells 
%-border between sinuses and nasal concha/turbinates 
%-only partial segmentation of the vessels in the arms and legs
%-the smaller branches of the pulmonary arteries/vein and hepatic vessels are not segmented
%-ileal cecal junction is usually wrong 
%-often the anterior border or posterior border of the stomach are given a different color, or parts of the small bowel are missed 
%-sometimes the spinal canal is not homogeneously segmented 


%Include Feedback from Calsey
%Point out that our new labels cannot be checked and copared by the previously mentioned radiological feature comparions and volume comparsisons. 
%We cannot guarantee that voxel-wise labels are correct for all the 

%\section*{Usage Notes}

%The Usage Notes should contain brief instructions to assist other researchers with reuse of the data. This may include discussion of software packages that are suitable for analysing the assay data files, suggested downstream processing steps (e.g. normalization, etc.), or tips for integrating or comparing the data records with other datasets. Authors are encouraged to provide code, programs or data-processing workflows if they may help others understand or use the data. Please see our code availability policy for advice on supplying custom code alongside Data Descriptor manuscripts.

%For studies involving privacy or safety controls on public access to the data, this section should describe in detail these controls, including how authors can apply to access the data, what criteria will be used to determine who may access the data, and any limitations on data use. 