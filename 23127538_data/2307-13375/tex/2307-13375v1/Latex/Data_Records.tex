\section*{Data Records}
\label{S: Data Record}
%The Data Records section should be used to explain each data record associated with this work, including the repository where this information is stored, and to provide an overview of the data files and their formats. Each external data record should be cited numerically in the text of this section, for example \cite{Hao:gidmaps:2014}, and included in the main reference list as described below. A data citation should also be placed in the subsection of the Methods containing the data-collection or analytical procedure(s) used to derive the corresponding record. Providing a direct link to the dataset may also be helpful to readers (\hyperlink{https://doi.org/10.6084/m9.figshare.853801}{https://doi.org/10.6084/m9.figshare.853801}).

%Tables should be used to support the data records, and should clearly indicate the samples and subjects (study inputs), their provenance, and the experimental manipulations performed on each (please see 'Tables' below). They should also specify the data output resulting from each data-collection or analytical step, should these form part of the archived record.
We make use of the recently published AutoPET dataset~\cite{gatidis2022whole} which can be accessed on The Cancer Imaging Archive (TCIA) under its collection name “FDG-PET-CT-Lesions” to download the raw PET/CT data. We publish the segmentation masks representing the DAP Atlas dataset at \href{https://github.com/alexanderjaus/AtlasDataset}{github}. In our DAP Atlas dataset, we have retained the AutoPET naming convention to ensure that the masks can be easily matched with their corresponding original CT volume.
\vspace{5mm}

\dirtree{%
.1 DAP Atlas Anatomical Labels.
.2 AutoPET\_0011f3deaf\_10445.nii.gz.
.2 AutoPET\_01140d52d8\_56839.nii.gz.
.2 AutoPET\_0143bab87a\_33529.nii.gz.
.2 \dots .
}
\vspace{5mm}
 The given name consists of the subject ID followed by the last $5$ digits of the Study UID which allows a unique matching of the segmentation masks to the AutoPET CTs. 