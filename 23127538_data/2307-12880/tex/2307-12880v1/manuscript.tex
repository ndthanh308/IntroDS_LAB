\documentclass[11pt]{article} %
\usepackage{graphicx}
%\usepackage{subfigure}
%\usepackage[english]{babel}
\usepackage{fancyhdr}
\usepackage{makeidx}
\usepackage{amssymb,amsmath,pdfpages}
\usepackage{physics}
\usepackage{gensymb}
\usepackage{upquote}
\usepackage{ulem}
\usepackage{xcolor}
\usepackage{cite}
%\usepackage[version=4]{mhchem}
%\usepackage{chemgreek}
\usepackage[utf8]{inputenc}
\usepackage[nottoc]{tocbibind}
\usepackage{setspace}
\usepackage{enumerate}
\DeclareUnicodeCharacter{0308}{HERE!HERE!}
\DeclareUnicodeCharacter{030C}{HERE!HERE!}
\DeclareUnicodeCharacter{0301}{HERE!HERE!}
%\usepackage[round, sort & compress]{natbib}
%\usepackage{algorithm}
%\usepackage{algorithmic}
%\usepackage{IEEEtran}
%\usepackage{aip}
%\usepackage{psfig}
%\usepackage{epsfig}
%\usepackage{natbib}
\newcommand{\bd}{\begin{document}}
	\newcommand{\ed}{\end{document}}
\newcommand{\bc}{\begin{center}}
	\newcommand{\ec}{\end{center}}
\newcommand{\vs}{\vspace}
\newcommand{\hs}{\hspace}
\newcommand{\beq}{\begin{equation}}
\newcommand{\eeq}{\end{equation}}
\newcommand{\beqs}{\begin{eqn*}}
	\newcommand{\eeqs}{\end{eqn*}}
\newcommand{\bq}{\begin{quote}}
	\newcommand{\eq}{\end{quote}}
\newcommand{\lb}{\linebreak}
%\renewcommand{\pb}{\pagebreak}
\newcommand{\mb}{\makebox}
\newcommand{\fb}{\framebox}
\newcommand{\mc}{\multicolumn}
\newcommand{\ben}{\begin{enumerate}}
	\newcommand{\een}{\end{enumerate}}
\newcommand{\bit}{\begin{itemize}}
	\newcommand{\eit}{\end{itemize}}
\newcommand{\ov}{\overline}
\newcommand{\un}{\underline}
\newcommand{\lt}{\left}
\newcommand{\rt}{\right}
\newcommand{\ba}{\begin{array}}
	\newcommand{\ea}{\end{array}}
\newcommand{\beqa}{\begin{eqnarray}}
\newcommand{\eeqa}{\end{eqnarray}}
\newcommand{\beqas}{\begin{eqnarray*}}
	\newcommand{\eeqas}{\end{eqnarray*}}
\newcommand{\bfg}{% Figure environment removed}
\newcommand{\pad}{\partial}
\newcommand{\nn}{\nonumber}
\newcommand{\la}{\leftarrow}
\newcommand{\ra}{\rightarrow}
\newcommand{\lgla}{\longleftarrow}
\newcommand{\lgra}{\longrightarrow}
\newcommand{\La}{\Leftarrow}
\newcommand{\Ra}{\Rightarrow}
\newcommand{\Lra}{\Leftrightarrow}
\newcommand{\Lgla}{\Longleftarrow}
\newcommand{\Lgra}{\Longrightarrow}
\renewcommand{\a}{\alpha}
\renewcommand{\b}{\beta}
\newcommand{\g}{\gamma}
\newcommand{\G}{\Gamma}
\renewcommand{\d}{\delta}
\newcommand{\D}{\Delta}
\newcommand{\e}{\epsilon}
\newcommand{\eps}{\epsilon}
%\newcommand{\th}{\theta}
\newcommand{\s}{\sigma}
\renewcommand{\l}{\lamda}
\newcommand{\m}{\mu}
\newcommand{\n}{\nu}
\renewcommand{\S}{\Sigma}
\newcommand{\p}{\pi}
\newcommand{\om}{\omega}
\newcommand{\Om}{\Omega}
\newcommand{\tri}{\triangle}
\newcommand{\ti}{\times}
\newcommand{\f}{\frac}
\newcommand{\ds}{\displaystyle}
\newcommand{\bm}[1]{\mb{{\boldmath $#1$}}}
\newcommand{\alter}[2]{\lt\{ \ba{ll}#1 \\ #2 \ea \rt.}
\newcommand{\alt}[4]{\lt\{ \ba{ll}#1 & \mb{if \, \,}#2 \\ #3 & \mb{}#4 \ea
	\rt.}
\newcommand{\altn}[4]{\lt\{ \ba{rl}#1 & \mb{if \, \,}#2 \\ #3 & \mb{}#4 \ea
	\rt.}
\newcommand{\altif}[4]{\lt\{ \ba{ll}#1 & \mb{if \, \,}#2 \\ #3 &
	\mb{if \, \,}#4 \ea \rt.}
\newcommand{\altnif}[4]{\lt\{ \ba{rl}#1 & \mb{if \, \,}#2 \\ #3 &
	\mb{if \, \,}#4 \ea \rt.}
\newcounter{algc}
\newcounter{romc}
\newcounter{Alphc}
\newcommand{\bl}{\begin{list}{{\it Step} ~\arabic{algc}~:} {\usecounter{algc}
			\setlength{\topsep}{0pt} \setlength{\itemsep}{0pt}}}
	\newcommand{\el}{\end{list}}
\newcommand{\blr}{\begin{list}{~\roman{romc}~:} {\usecounter{romc}
			\setlength{\topsep}{0pt} \setlength{\itemsep}{0pt}}}
	\newcommand{\elr}{\end{list}}
\newcommand{\bla}{\begin{list}{~\Alph{Alphc}~:} {\usecounter{Alphc}
			\setlength{\topsep}{0pt} \setlength{\itemsep}{0pt}}}
	\newcommand{\ela}{\end{list}}
\newcommand{\tsup}{\textsuperscript}
\newcommand{\tsub}{\textsubscript}


\newtheorem{theorem}{Theorem}
\setlength{\topmargin}{-0.5in} \setlength{\textwidth}{6.25in}
\setlength{\textheight}{8.5in} \setlength{\oddsidemargin}{0.2in}
\setlength{\evensidemargin}{0.2in}
\linespread{1.5}

\begin{document}
\title{Harmonic to anharmonic tuning of moir\'e potential leading to unconventional Stark effect and giant dipolar repulsion in WS$_2$/WSe$_2$ heterobilayer}
\author{Suman Chatterjee$^{1,||}$, Medha Dandu$^{1,2,||}$, Pushkar Dasika$^{1,||}$, Rabindra Biswas$^{1}$, \\Sarthak Das$^{1,3}$, Kenji Watanabe$^4$, Takashi Taniguchi$^5$,\\ Varun Raghunathan$^1$, and Kausik Majumdar$^{1*}$\\
	$^1$Department of Electrical Communication Engineering, \\Indian Institute of Science, Bangalore 560012, India\\	
	$^2$Currently with Molecular Foundry, Lawrence Berkeley National Laboratory,\\  Berkeley, CA 94720, United States\\
	$^3$Currently with Institute of Materials Research and Engineering (IMRE),\\ Agency for Science, Technology and Research (A*STAR),\\ Singapore 138634, Republic of Singapore\\
    $^4$Research Center for Functional Materials,\\ National Institute for Materials Science, 1-1 Namiki, Tsukuba 305-044, Japan\\
	$^5$International Center for Materials Nanoarchitectonics,\\ National Institute for Materials Science, 1-1 Namiki, Tsukuba 305-044, Japan\\
	\\$^{||}$These authors contributed equally\\
	$^*$Corresponding author, email: kausikm@iisc.ac.in}
\date{}
\maketitle
\newpage
\begin{abstract}
Excitonic states trapped in harmonic moir\'e wells of twisted heterobilayers is an intriguing testbed. However, the moir\'e potential is primarily governed by the twist angle, and its dynamic tuning remains a challenge. Here we demonstrate anharmonic tuning of moir\'e potential in a WS$_2$/WSe$_2$ heterobilayer through gate voltage and optical power. A gate voltage can result in a local in-plane perturbing field with odd parity around the high-symmetry points. This allows us to simultaneously observe the first (linear) and second (parabolic) order Stark shift for the ground state and first excited state, respectively, of the moir\'e trapped exciton - an effect opposite to conventional quantum-confined Stark shift. Depending on the degree of confinement, these excitons exhibit up to twenty-fold gate-tunability in the lifetime ($100$ to $5$ ns). Also, exciton localization dependent dipolar repulsion leads to an optical power-induced blueshift of $\sim$1 meV/$\mu$W - a five-fold enhancement over previous reports.
  \end{abstract}

\section*{Introduction}
Interlayer van der Waals interaction allows us to stack layers of transition metal dichalcogenides (TMDCs) onto each other with an arbitrary lattice mismatch \cite{chiu2015determination,cheng2014electroluminescence,dandu2022electrically}. This leads to an additional degree of freedom, the twist angle ($\theta$) between two successive layers, that governs the moir\'e pattern arising in the corresponding superlattice \cite{tran2019evidence,mak2022semiconductor,wu2018theory,lau2022reproducibility}. The lattice constant of the moir\'e is given by $a_M \approx \frac{a}{\sqrt{\theta^2+\delta^2}}$ where $\delta$ is the lattice constant difference and $a$ being the average lattice constant  \cite{yuan2020twist,jin2019identification,wu2018theory}. Different atomic registries present in this moir\'e superlattice (Figure \ref{fig:introduction to three moire ILEs}a) form a periodic potential fluctuation [$V_M(\mathbf{r})$] resulting from local strain and interlayer coupling \cite{naik2020origin,naik2018ultraflatbands}. Varying twist angle can dramatically change the material properties, drawing attention from the researchers in the recent past \cite{lin2023room,chuang2022emergent,shi2019twisted,yuan2020twist}. Moir\'e superlattice in TMDC heterobilayer has been widely explored including observation of neutral moir\'e exciton \cite{tran2019evidence,alexeev2019resonantly,seyler2019signatures}, moir\'e trion \cite{liu2021signatures,wang2021moire,marcellina2021evidence}, single photon emission \cite{mukherjee2020observation,kremser2020discrete}, and correlated states \cite{xu2020correlated,mak2022semiconductor,liu2021excitonic}.

Due to type-II band alignment, WS\tsub2/WSe\tsub2 heterobilayer supports an ultrafast charge transfer \cite{jin2018ultrafast,hong2014ultrafast} with electrons staying in the WS\tsub{2} conduction band, and holes in the WSe\tsub2 valance band, forming interlayer exciton (ILE) \cite{yuan2020twist,jin2019identification} under optical excitation (Figure \ref{fig:introduction to three moire ILEs}b). The moir\'e wells behave as two-dimensional harmonic traps for the ILEs \cite{tran2019evidence,tan2022signature,lohof2023confined}.

The depth of the exciton moir\'e potential is determined by the twist angle and the degree of lattice mismatch between the two heterobilayers. Hence, dynamic tuning of moir\'e potential remains a challenge, which, if realised, will be of great importance for both scientific exploration and applications. One could perturb the moir\'e potential by external stimulus, however, the perturbing potential may not necessarily be harmonic, breaking down the usual harmonic potential approximation for moir\'e well. In this work, we explore two such anharmonic perturbations to the WS$_2$/WSe$_2$ moir\'e potential well: the first one through a gate voltage which introduces anharmonic perturbation through screening at high doping regime; and the second one is through optical excitation which introduces the perturbing potential through ILE dipolar repulsion. In both cases, the harmonic to anharmonic switching of the moir\'e potential manifests through a corresponding change from an equal to unequal inter-excitonic spectral separation. In such a scenario, we explore several intriguing features of the moir\'e excitons, including giant lifetime tunability, anomalous Stark shift, and dipolar repulsion induced large spectral blueshift.

\section*{Results and Discussion}
We prepare hBN-capped $\sim 59^\circ$ twisted (confirmed by second harmonic generation (SHG) spectroscopy in \textbf{Supplementary Note 1 and Figure 1}) WS\tsub2/WSe\tsub2 heterobilayer (sample D1) with a back gate (see \textbf{Methods} for sample preparation). The schematic and the optical image of sample D1 are illustrated in Figure \ref{fig:introduction to three moire ILEs}c and d. This twist angle creates a moir\'e superlattice with a lattice constant $\sim 7.3$ nm. Figure \ref{fig:introduction to three moire ILEs}e shows a representative photoluminescence (PL) spectrum from the sample with 532 nm excitation at 4 K. The emission spectrum exhibits three separate, strong interlayer moir\'e excitonic resonances \cite{sun2022enhanced} $X_0$, $X_1$, and $X_2$ at $\approx$ 1.392, 1.418 and 1.442 eV, respectively (marked by black dashed line). The peaks exhibit alternating sign of the degree of circular polarization (DOCP) (\textbf{Supplementary Figure 2}), indicating the existence of moir\'e superlattice \cite{tran2019evidence,yu2018brightened,wu2018theory}.

%introduction to three ILE states
The near-equal inter-excitonic separation suggests that the three exciton resonances appear from excitonic states in the harmonic moir\'e potential well (Figure \ref{fig:introduction to three moire ILEs}f) \cite{tran2019evidence,tan2022signature,wu2018theory,lohof2023confined}. This inter-excitonic separation can be tuned by varying the twist angle, which regulates the depth of the moir\'e potential well \cite{tran2019evidence, choi2021twist}. We verified this by measuring twist angle dependent PL spectra from three samples [D1 ($\sim 59^\circ$), D2 ($\sim 54^\circ$) and D3 (large angle misalignment)] in \textbf{Supplementary Figure 3}. The time-resolved PL (TRPL) measurement (see \textbf{Methods}) from sample D1 in Figure \ref{fig:introduction to three moire ILEs}g shows that the lifetime of the three species ($\tau_{X_0}=100$ ns, $\tau_{X_1}=15.3$ ns, and $\tau_{X_2}=9$ ns) increases significantly with stronger confinement. Accordingly, their PL intensity also exhibits significantly different power law with varying optical power ($P$): $I \propto P^{\alpha_i}$ with $\alpha_0=0.34\pm0.02$, $\alpha_1=0.59\pm0.03$, and $\alpha_2=1.1\pm0.11$ (Figure \ref{fig:introduction to three moire ILEs}h). The corresponding spectra at three different $P$ values are shown in Figure \ref{fig:introduction to three moire ILEs}i. At low power ($30$ nW), $X_0$ emission is the dominant one, with negligible emission from $X_2$. However, at higher power ($5.95$ $\mu$W), three peaks are clearly discernable, and the fractional contribution of $X_0$ reduces, while $X_2$ emission becomes appreciable. All these observations indicate that the three different excitonic species correspond to moir\'e trapped excitonic states with varying degrees of localization (Figure \ref{fig:introduction to three moire ILEs}f). From the spectral separation between the quantized states, we calculate peak-to-peak moir\'e potential fluctuation of $\approx 150$ meV (see \textbf{Supplementary Note 2}), as shown in Figure \ref{fig:gate dependent ILE}c. Possible alternative explanations, such as phonon-sidebands and defect-bound excitons, are unlikely in our samples based on the observations including alternating signs of the DOCP and systematic tuning of the ILE peak separation with twist angle, doping, and optical power (discussed later).

%gate dependent manipulation of ILE
\textbf{Gate tunability:} Figure \ref{fig:gate dependent ILE}a shows a color plot of the interlayer exciton emission spectra as a function of gate voltage ($V_g$). The estimated n-doping density at the highest applied $V_g$ ($=5$ V) is $<1.5 \times10^{12}$ cm$^{-2}$ (see \textbf{Supplementary Figure 4}). This is well below the moir\'e trap density ($n_0) \approx 2\times10^{12}$ cm$^{-2}$ for $a_M \sim 7.3$ nm. The fitted peak positions are shown in the left panel of Figure \ref{fig:gate dependent ILE}b (see individual spectra in \textbf{Supplementary Figure 5}). While the $V_g<$ 0 V region is nearly featureless, $V_g>$ 0 V (n-doping) region has three conspicuous features: (a) there is a reduction in emission intensity for all the three ILE peaks, with $X_0$ disappearing at high $V_g$; (b) there is a large and unequal redshift for the peaks for $V_g>0$; and (c) the inter-excitonic separation changes at higher $V_g$, indicating induced anharmonicity. The reduction in emission intensity with an increase in $V_g$ rules out the charged excitonic (trion) nature of any of the three peaks. Figure \ref{fig:gate dependent ILE}b (right panel) schematically explains the origin of the strong redshift with $V_g$. At positive $V_g$, the WS$_2$ layer becomes n-doped. Due to small thermal energy at 4 K, the wave function of the induced electrons remains primarily in the WS$_2$ layer, with a fraction of it extends into the WSe$_2$ bandgap as an evanescent state with imaginary wave vector. Such a wave function distribution creates a screening of the gate field, and in turn a relative potential difference between WS$_2$ and WSe$_2$ layers, reducing the interlayer bandgap. Note that, the presence of the charge density from the evanescent state in WSe$_2$ is essential to create such relative potential difference between the two layers, else dictated by the self-consistent electrostatics, a zero induced charge density in WSe$_2$ layer would result in pinning of the WSe$_2$ potential with that of WS$_2$, and no relative interlayer bandgap change would be allowed.

\textbf{Unconventional Stark effect:} Interestingly, the average slope (indicated by black dashed line in Figure \ref{fig:gate dependent ILE}b) of the redshift of $X_2$ is almost similar (about 5 meV/V) to that of the intra-layer WS$_2$ trion (X$^-$) or charged (XX$^-$) biexciton \cite{chatterjee2022trion} (See \textbf{Supplementary Figure 6}), but the average slope is higher for $X_1$ ($\sim$ 7 meV/V) and $X_0$ ($\sim$ 15 meV/V). The redshift of the intra-layer WS$_2$ trion emission peak with $V_g$ is directly related to the enhanced trion dissociation energy due to the extra energy required to place the remaining electron into the increasingly filled conduction band. Hence it can be correlated with the change in the Fermi energy due to doping \cite{chatterjee2022trion,mak2013tightly,kallatt2019interlayer}. This change is nearly equal to the shift in the WS$_2$ conduction band with respect to the WSe$_2$ valence band, making the average slopes of $X_2$ and WS$_2$ trion shift similar. This also is in agreement with the weak confinement of $X_2$.

However, the enhancement in the slope of the redshift for $X_1$ and $X_0$ cannot be explained from doping dependent interlayer bandgap reduction and suggests a strong additional effect of localization. To understand this further, we solve the 1D Poisson equation to obtain the movement of bands with $V_g$ (see \textbf{Supplementary Note 3} for the details of the calculation). The results are summarized in Figure \ref{fig:gate dependent ILE}d. At small positive $V_g$, the bands shift downward in energy (middle panel, $V_g=$ 0.5 V). However, at larger positive $V_g$, the central part of region I (right panel, $V_g>$ 0.5 V) of the conduction band moir\'e well being energetically closer to the Fermi energy supports more electron density than region II. Accordingly, due to the screening by the induced carrier density, region I starts moving down slower than region II. The net effect is a suppression in the local moir\'e fluctuation of the conduction band. Interestingly, the self-consistent electrostatics forces an amplification in the moir\'e potential fluctuation in the valence band of WSe$_2$: The suppressed movement of WS$_2$ bands in region I also reduces the movement of bands in WSe$_2$, while the stronger movement of WS$_2$ bands in region II (with relatively less carrier density) also pushes the WSe$_2$ bands more downward. The net result is a flattening of the electron moir\'e well in the WS\tsub2 conduction band, causing a delocalization of the electron state, coupled with a deeper hole moir\'e well in the WSe$_2$ valence band, resulting in an enhanced localization of the hole state (zoomed in Figure \ref{fig:gate dependent ILE}d, bottom panel). This modification of the moir\'e trapping potential, in turn, causes a reduction in the energy of the trapped electron state and an enhancement in the energy of the trapped hole state. The negative net change gives rise to an additional redshift in the localized exciton resonance ($X_0$ and $X_1$).

This results in an in-plane perturbation potential ($\Delta V$) with even parity about the high-symmetry points (Figure \ref{fig:gate dependent ILE}e). $\Delta V$ is maximum at the center of the moir\'e well and reduces symmetrically away from the center. On the other hand, the wave function ($\psi$) has an even and odd parity for the ground ($X_0$) and first excited ($X_1$) states, respectively. This, in turn, results in a large (small) value of $|\psi_0|^2$ ($|\psi_1|^2$) around the center of the trap for $X_0$ ($X_1$), as shown in Figure \ref{fig:gate dependent ILE}e. Due to such a strong overlap (non-overlap) of $\Delta V$ and $|\psi_0|^2$ ($|\psi_1|^2$), the first order Stark effect ($\bra{\psi}\Delta V\ket{\psi}$) is nonzero (negligible) for $X_0$ ($X_1$). Accordingly, we expect $X_0$ and $X_1$ to exhibit linear and parabolic Stark shift, respectively, with the in-plane local electric field ($\xi$), and hence with $V_g$, since our simulation suggests that $\xi$ is approximately linearly dependent on $V_g$ (see \textbf{Supplementary Figure 7}). Such local field effect will cancel out for the less-localized $X_2$ state. In Figure \ref{fig:gate dependent ILE}f, the respective Stark shifts [$\delta_{X_{0,1}}(V_g)-\delta_{X_{0,1}}(V_g=0)$ where $\delta_{X_0}=E_{X_2}-E_{X_0}$ and $\delta_{X_1}=E_{X_2}-E_{X_1}$] exhibit linear and parabolic variation with $V_g$ (reproduced in sample D4 as well, see \textbf{Supplementary Figure 8}), in excellent agreement with the above analysis. We note that such Stark effect is unconventional since the usual quantum confined Stark  effect (QCSE) in quantum wells, where the applied vertical electric field is uniform, results in a perturbing potential having odd parity. Thus the first-order QCSE (linear) is usually negligible, and we only observe a parabolic shift in the emission energy due to the second-order correction \cite{singh2007electronic, abraham2021anomalous,das2020highly,verzhbitskiy2019suppressed,klein2016stark}.

\textbf{Gate tunable exciton lifetime:} Figure \ref{fig:Gate dependent trpl}a shows the peak-resolved (spectral resolution of 0.8 meV) TRPL spectra (see \textbf{Methods}) for $X_0$, $X_1$, and $X_2$, at $V_g=0$ and $3$ V, suggesting a faster decay at higher $V_g$ for all the ILE peaks. The transient response is captured well (solid black lines in Figure \ref{fig:Gate dependent trpl}a) by a set of rate equations and Gaussian formation model (see \textbf{Methods}, equations \ref{diff eqn_0}-\ref{diff eqn_2}). The extracted decay ($\tau_i$) and formation time ($\tau_{fi}$) are plotted for the exciton $X_i$, $i=0,1,2$ in Figure \ref{fig:Gate dependent trpl}b-c. Around $V_g=0$ V, the decay time varies over 10-fold from $X_0$ ($\sim 100$ ns) to $X_2$ ($\sim 9$ ns). However, at large $V_g$, all the three ILEs show similar decay time (4-6 ns). On the other hand, the formation times are relatively weaker function of $V_g$ and reduces slightly with increasing $V_g$.

The kinetics can be understood by the cascaded processes schematically depicted in Figure \ref{fig:Gate dependent trpl}d. At small $V_g$, the respective net lifetimes follow the trend $\tau_0 \gg \tau_1 > \tau_2$ (Figure \ref{fig:Gate dependent trpl}b), which is readily understood due to the additional non-radiative decay paths $\gamma_{20}$ and $\gamma_{21}$ for $X_2$, and $\gamma_{10}$ for $X_1$. The order of the respective formation times ($\tau_{f0}= 5.6$ ns, $\tau_{f1}= 3.6$ ns, and $\tau_{f2}= 1.1$ ns) in Figure \ref{fig:Gate dependent trpl}c, also supports the model of cascaded formation. In addition, a longer lifetime would mean the state is blocked for a longer duration, increasing the formation time.

The strong gate dependence of the ILE lifetime is captured through a simple model where the gate dependent non-radiative process is considered as proportional to induced carrier density (see equations \ref{decay time scales}-\ref{eq:decay time scales2} in \textbf{Methods}):
\begin{equation}\label{eq:lifetime_model}
  \tau_i(V_g)= \Bigg{[}\frac{1}{\tau_i(V_g=0)} + C_i(e^{\alpha V_g} - 1)\Bigg{]}^{-1}
\end{equation}
The model (solid traces in Figure \ref{fig:Gate dependent trpl}b) accurately reproduces the $V_g$ dependent lifetime values (symbols) by using $\alpha$ and $C_i$ as fitting parameters. We observe a $V_g$-modulation of $\tau_0$ by more than 20-fold from $100$ to $5$ ns (Figure \ref{fig:Gate dependent trpl}b), which correlates well with the PL intensity reduction of $X_0$ with $V_g$, in Figure \ref{fig:gate dependent ILE}a. This is a direct evidence of the gate-induced non-radiative process due to the delocalization of the electron in the flattened conduction band (Figure \ref{fig:gate dependent ILE}d). $X_0$ being the ground state of the well, the inter-excitonic transfer related non-radiative decay channels (Figure \ref{fig:Gate dependent trpl}d) are suppressed. On the other hand, At low $V_g$, $\tau_1$ and $\tau_2$ are dominated by the (gate independent) non-radiative decay channels to other lower energy states (that is, $\gamma_{10}$, $\gamma_{20}$, and $\gamma_{21}$), hence remain nearly unchanged up to $V_g=2$ V (Figure \ref{fig:Gate dependent trpl}b). The $V_g$-dependent non-radiative decay rate starts dominating only at large $V_g$ for $X_1$ and $X_2$, resulting in a reduction of $\tau_{1}$ and $\tau_{2}$.

 %power dependent manipulation of ILE
\textbf{Optical power induced anharmonicity:} We now vary $P$ over nearly two decades using a pulsed laser (531 nm) at $V_g=0$ V and plot the ILE peak positions in Figure \ref{fig:power dependent trpl}a. While $X_0$ exhibits a strong blueshift ($\approx 1$ meV$/\mu$W), the shift for $X_1$ and $X_2$ is negligible. Hence, the inter-excitonic separations ($\delta E_{21}$ and $\delta E_{10}$) do not remain equal at higher $P$, suggesting departure from harmonic behaviour. Such anharmonicity and power-dependent blueshift can be understood by the perturbing potential ($U_{dd}$) arising from ILE dipolar repulsion \cite{sun2022excitonic,laikhtman2009exciton}:
\begin{equation} \label{eq:blueshift}
     U_{dd} = \int nU(r)d^2r = n\frac{q^2d}{\epsilon_0\epsilon_r}
 \end{equation}
where $U(r) = \frac{q^2}{2\pi\epsilon_0\epsilon_r}(\frac{1}{r} - \frac{1}{\sqrt{r^2 + d^2}})$ is the repulsion between two ILE dipoles placed at a distance r (schematically shown in Figure \ref{fig:power dependent trpl}b, left panel). $\epsilon_0$ is the vacuum permittivity, $\epsilon_r$ is the effective relative permittivity of the heterojunction, $n$ is the effective concentration of exciton dipoles, and $d$ is the interlayer separation. Due to this induced anharmonicity, it is expected to observe a lifting of degeneracy for $X_1$ and $X_2$, as shown schematically in Figure \ref{fig:power dependent trpl}b (right panel). Since the lifetime of $X_0$ is significantly larger than that of $X_1$ and $X_2$, the steady-state density (generation rate $\times$ lifetime) of ILE dipoles is dominated by the population of $X_0$ ($n_{X0}$). Since $I_{X0} (\propto n_{X_0})  \propto P^{0.34}$ (see Figure \ref{fig:introduction to three moire ILEs}h), equation \ref{eq:blueshift} indicates that the blueshift ($E_{dd}$) of $X_0$ should follow $E_{dd} \propto P^{0.34}$, in good agreement with the linear fit in Figure \ref{fig:power dependent trpl}c. From equation \ref{eq:blueshift}, $n_{X0}$ is estimated to be $\approx 9.5\times 10^{11}$ cm$^{-2}$ (which is less than $n_0/2$) at the highest optical power used ($17.7$ $\mu$W).

To the best of our knowledge, the observed average rate of the blueshift with power for $X_0$ ($\approx$ 1 meV/$\mu$W) is the highest reported value for ILE to date \cite{sun2022excitonic,nagler2017interlayer,rivera2015observation,unuchek2019valley}, indicating a strong inter-excitonic interaction. The strong confinement of $X_0$ does not allow it to drift out of the moir\'e trap in the presence of such dipole-dipole repulsion, resulting in a large blueshift. On the other hand, weaker confinement of ${X_1}$ and ${X_2}$ allows them to drift away under such dipolar repulsion, resulting in a suppressed blueshift in this small power regime.

Figure \ref{fig:power dependent trpl}d, top panel (open symbols) shows the optical power dependent lifetime of $X_0$, $X_1$, and $X_2$. We notice that the lifetime for all the three species is a weak function of $P$. This is in stark contrast with intra-layer free exciton where Auger effect drastically reduces the lifetime at higher $P$ \cite{kumar2014exciton,kuroda2020dark}. Such a weak dependence of lifetime on $P$ is a result of protection from Auger-induced exciton-exciton annihilation due to a combined effect of moir\'e trapping and strong dipolar repulsion.

For a perfect two-dimensional harmonic well, $X_0$, $X_1$, and $X_2$ are expected to exhibit a degeneracy of 1, 2, and 3, respectively. Through the optically induced anharmonicity, we expect the degeneracy of $X_1$ and $X_2$ to be lifted (Figure \ref{fig:power dependent trpl}b, right panel). However, our simulation suggests only $< 2$ meV fine-splitting, and the inhomogeneous broadening of the peaks does not allow us to observe such small splitting in the emission spectra.

Interestingly, while $X_2$ exhibits a mono-exponential decay at low power, its dynamics becomes bi-exponential at higher power ($P > 3.9$ $\mu$W) with an additional lifetime of $\tau_a \sim 1$ ns, as indicated by the blue solid symbols in Figure \ref{fig:power dependent trpl}d (top panel), and the TRPL spectra in the top panels of Figure \ref{fig:power dependent trpl}e-f. In the bottom panel of Figure \ref{fig:power dependent trpl}d, we quantify the degree of anharmonic perturbation by plotting, from Figure \ref{fig:power dependent trpl}a, the relative magnitude of the peak separation ($\delta E = \frac{\delta E_{21} - \delta E_{10}}{\delta E_{21}}\times100\%$) with incident power (0\% corresponding to the harmonic case). The strong correlation between the appearance of the faster additional decay (in region 2) and the strength of the anharmonic perturbation is evident. The faster additional decay likely arises from the fine-split higher energy state of $X_2$, which has reduced confinement into the moir\'e trap, thus having enhanced decay rate (schematically shown in Figure \ref{fig:power dependent trpl}b, right panel). Note that the decay of $X_0$ remains mono-exponential even at higher power since the ground state is non-degenerate (bottom panels of Figure \ref{fig:power dependent trpl}e-f).

In summary, we have shown that the exciton moir\'e potential in heterobilayer can be dynamically tuned through external stimuli, such as gate voltage and optical power. The usual harmonic approximation of moir\'e potential breaks down under such perturbation. The strength of such tunability is evidenced through moir\'e excitons exhibiting (a) confinement dependent tuning of features, (b) anomalous Stark shift where parity is reversed with respect to conventional quantum-confined Stark effect, (c) strong modulation of the lifetime and the inter-excitonic separation, and (d) a giant spectral blueshift through dipolar repulsion. The results will lead to intriguing experiments and applications exploiting dynamic tuning of moir\'e potential.

\section*{Methods}
\textbf{Device fabrication:} We prepared the hBN capped WS\tsub2/WSe\tsub2 heterojunctions using a sequential dry-transfer method (with micromanipulators) where the individual layers were exfoliated from flux grown crystals (HQ-Graphene) on polydimethylsiloxane (PDMS) using Scotch tape. For back gated samples, the pre-patterned metal electrodes are prepared using photolithography followed by sputtering of Ni/Au (10/50 nm) and lift-off. The entire stack (for D1 and D4) is gated from the backside (from the WS$_2$ side) through hBN layer (dielectric) and the pre-patterned metal line. The WS\tsub2 layer is contacted to a different electrode (Gr) for carrier injection. After completion of the transfer process, the devices are annealed inside a vacuum chamber ($10^{-6}$ mbar) at 250$^\circ$C for 5 hours for better adhesion of the layers and removal of air bubbles. The angle and stacking between WS\tsub2/WSe\tsub2 layers are confirmed using SHG (see \textbf{Supplementary Figure 1}).

\textbf{PL measurement:} All the PL measurements on the samples are carried out in a closed-cycle cryostat at 4.5 K using a $\times$50 objective (0.5 numerical aperture) lense. The bottom gate voltages are applied using a Keithley 2636B source meter (for both PL and TRPL), and then the PL spectra are collected using a spectrometer with 1800 lines per mm grating and CCD (Renishaw spectrometer). We use the 532 nm CW and 531 nm pulsed lasers to excite the sample. The spot size for both pulsed and CW laser is $\sim$1.5 $\mu$m. All the power values are measured using a silicon photodetector from Edmund Optics. All the error bars in different plots in the manuscript indicate mean $\pm$ standard deviation.

\textbf{TRPL measurement:} Our custom-built TRPL setup comprises of a 531 nm pulsed laser head (LDH-D-TA-530B from PicoQuant) controlled by the PDL-800 D driver, a photon-counting detector (SPD-050-CTC from Micro Photon Devices), and a time-correlated single photon counting (TCSPC) system (PicoHarp 300 from PicoQuant). The pulse width of the laser is 40 ps. For the spectrally resolved TRPL from moir\'e ILEs, a combination of a long pass filter (cut in wavelength of 650 nm) and a wavelength-tunable monochromator (Edmund optics, 2 cm$^2$ Square holographic gratings) with 0.5 nm resolution (corresponding to about 0.8 meV resolution in the ILE spectral regime) are placed in front of the SPD. The peak position of the emission from ILEs are simultaneously measured along with TRPL measurement by performing in-situ PL (see Supplemental Material in ref. \cite{chatterjee2022trion} for setup schematic). The instrument response function (IRF) has a full-width-at-half-maximum (fwhm) of $52$ ps.

\textbf{Exciton formation and decay model:} To fit the experimentally obtained TRPL data, we use three differential equations:
\begin{equation}\label{diff eqn_0}
    \frac{dn_0(t)}{dt} = f_0(t) - \frac{n_0(t)}{\tau_0}
    \end{equation}
    \begin{equation}\label{diff eqn_1}
    \frac{dn_1(t)}{dt} = f_1(t) - \frac{n_1(t)}{\tau_1}
    \end{equation}
      \begin{equation}\label{diff eqn_2}
    \frac{dn_2(t)}{dt} = f_2(t) - \frac{n_2(t)}{\tau_2}
\end{equation}
Here $n_i(t)$ is the time dependent population density, $\tau_i$ is the net decay time, and $f_i(t) = \frac{1}{\sigma_i \sqrt{2\pi}} e^{\frac{-(t-\tau_{fi})^2}{2\sigma_i^2}}$ is the Gaussian formation function, and $\tau_{fi}$ is the formation time measured from the laser excitation time for exciton $X_i$, $i=0,1,2$. After solving these equations numerically, we fit the measured TRPL data from the three moir\'e exciton emissions using $\tau_{fi}$, $\sigma_i$, and $\tau_i$ as fitting parameter.

\textbf{Model for gate-voltage dependent lifetime:} The net decay time ($\tau_i$) measured in TRPL (Figure \ref{fig:Gate dependent trpl}b), for exciton $X_i$ ($i=0, 1, 2$) is given by:
\begin{equation}\label{decay time scales}
  \frac{1}{\tau_i(V_g)}= \frac{1}{\tau_{r,i}} + \frac{1}{\tau_{nr0,i}} + \frac{1}{\tau_{nrg,i}(V_g)}
\end{equation}
where $\tau_{r,i}$, $\tau_{nr0,i}$, and $\tau_{nrg,i}(V_g)$ represent the radiative lifetime, gate voltage independent non-radiative lifetime, and the gate voltage dependent non-radiative lifetime, respectively. From Figure \ref{fig:Gate dependent trpl}d, $\frac{1}{\tau_{nr0,2}}= \gamma_{20} + \gamma_{21} + \gamma^\prime_{2}$ for $X_2$, and $\frac{1}{\tau_{nr0,1}}= \gamma_{10} + \gamma^\prime_{1}$ for $X_1$, and $\frac{1}{\tau_{nr0,0}}= \gamma^\prime_{0}$, where $\gamma^\prime_{i}$ is the rate of any other unaccounted non-radiative process for exciton $X_i$. Considering that the rate of the gate dependent non-radiative process is proportional to induced carrier density, which in turn is an exponential function of $V_g$, we write $\frac{1}{\tau_{nrg,i}}= C_ie^{\alpha V_g}$, where $C_i$ and $\alpha$ are fitting parameters. By noting that $\frac{1}{\tau_{r,i}}$ is relative small (in equation \ref{decay time scales}) and becomes smaller with an increase in $V_g$, we write
\begin{equation}\label{eq:decay time scales2}
  \frac{1}{\tau_i(V_g)} \approx \frac{1}{\tau_i(V_g=0)} + C_i(e^{\alpha V_g} - 1)
\end{equation}
\section*{Data Availability}
The data that support the findings of this study are available within the main text and Supplementary Information. Any other relevant data are available from the corresponding authors upon request.
%\bibliographystyle{unsrt}
%\bibliography{references}
\begin{thebibliography}{10}

\bibitem{chiu2015determination}
Ming-Hui Chiu, Chendong Zhang, Hung-Wei Shiu, Chih-Piao Chuu, Chang-Hsiao Chen,
  Chih-Yuan~S Chang, Chia-Hao Chen, Mei-Yin Chou, Chih-Kang Shih, and Lain-Jong
  Li.
\newblock Determination of band alignment in the single-layer
  {MoS$_2$}/{WSe$_2$} heterojunction.
\newblock {\em Nature Communications}, 6(1):7666, 2015.

\bibitem{cheng2014electroluminescence}
Rui Cheng, Dehui Li, Hailong Zhou, Chen Wang, Anxiang Yin, Shan Jiang, Yuan
  Liu, Yu~Chen, Yu~Huang, and Xiangfeng Duan.
\newblock Electroluminescence and photocurrent generation from atomically sharp
  {WSe$_2$/MoS$_2$} heterojunction p--n diodes.
\newblock {\em Nano Letters}, 14(10):5590--5597, 2014.

\bibitem{dandu2022electrically}
Medha Dandu, Garima Gupta, Pushkar Dasika, Kenji Watanabe, Takashi Taniguchi,
  and Kausik Majumdar.
\newblock Electrically tunable localized versus delocalized intralayer moir\'e
  excitons and trions in a twisted {MoS$_2$} bilayer.
\newblock {\em ACS Nano}, 16(6):8983--8992, 2022.

\bibitem{tran2019evidence}
Kha Tran, Galan Moody, Fengcheng Wu, Xiaobo Lu, Junho Choi, Kyounghwan Kim,
  Amritesh Rai, Daniel~A Sanchez, Jiamin Quan, Akshay Singh, et~al.
\newblock Evidence for {moir{\'e}} excitons in van der waals heterostructures.
\newblock {\em Nature}, 567(7746):71--75, 2019.

\bibitem{mak2022semiconductor}
Kin~Fai Mak and Jie Shan.
\newblock Semiconductor{ moir\'e} materials.
\newblock {\em Nature Nanotechnology}, 17(7):686--695, 2022.

\bibitem{wu2018theory}
Fengcheng Wu, Timothy Lovorn, and AH~MacDonald.
\newblock Theory of optical absorption by interlayer excitons in transition
  metal dichalcogenide heterobilayers.
\newblock {\em Physical Review B}, 97(3):035306, 2018.

\bibitem{lau2022reproducibility}
Chun~Ning Lau, Marc~W Bockrath, Kin~Fai Mak, and Fan Zhang.
\newblock Reproducibility in the fabrication and physics of moir{\'e}
  materials.
\newblock {\em Nature}, 602(7895):41--50, 2022.

\bibitem{yuan2020twist}
Long Yuan, Biyuan Zheng, Jens Kunstmann, Thomas Brumme, Agnieszka~Beata Kuc,
  Chao Ma, Shibin Deng, Daria Blach, Anlian Pan, and Libai Huang.
\newblock Twist-angle-dependent interlayer exciton diffusion in
  {WS$_2$}-{WSe$_2$} heterobilayers.
\newblock {\em Nature Materials}, 19(6):617--623, 2020.

\bibitem{jin2019identification}
Chenhao Jin, Emma~C Regan, Danqing Wang, M~Iqbal Bakti~Utama, Chan-Shan Yang,
  Jeffrey Cain, Ying Qin, Yuxia Shen, Zhiren Zheng, Kenji Watanabe, et~al.
\newblock Identification of spin, valley and {moir\'e} quasi-angular momentum
  of interlayer excitons.
\newblock {\em Nature Physics}, 15(11):1140--1144, 2019.

\bibitem{naik2020origin}
Mit~H Naik, Sudipta Kundu, Indrajit Maity, and Manish Jain.
\newblock Origin and evolution of ultraflat bands in twisted bilayer transition
  metal dichalcogenides: Realization of triangular quantum dots.
\newblock {\em Physical Review B}, 102(7):075413, 2020.

\bibitem{naik2018ultraflatbands}
Mit~H Naik and Manish Jain.
\newblock Ultraflatbands and shear solitons in {moir\'e} patterns of twisted
  bilayer transition metal dichalcogenides.
\newblock {\em Physical Review Letters}, 121(26):266401, 2018.

\bibitem{lin2023room}
Qiaoling Lin, Hanlin Fang, Yuanda Liu, Yi~Zhang, Moritz Fischer, Juntao Li,
  Joakim Hagel, Samuel Brem, Ermin Malic, Nicolas Stenger, et~al.
\newblock A room-temperature {moir\'e} interlayer exciton laser.
\newblock {\em arXiv preprint arXiv:2302.01266}, 2023.

\bibitem{chuang2022emergent}
Hsun-Jen Chuang, Madeleine Phillips, Kathleen~M McCreary, Darshana
  Wickramaratne, Matthew~R Rosenberger, Vladimir~P Oleshko, Nicholas~V Proscia,
  Mark Lohmann, Dante~J O’Hara, Paul~D Cunningham, et~al.
\newblock Emergent {moir\'e} phonons due to zone folding in {WSe$_2$}-{WS$_2$}
  van der waals heterostructures.
\newblock {\em ACS Nano}, 16(10):16260--16270, 2022.

\bibitem{shi2019twisted}
Jia Shi, Yuanzheng Li, Zhepeng Zhang, Weiqiang Feng, Qi~Wang, Shuliang Ren, Jun
  Zhang, Wenna Du, Xianxin Wu, Xinyu Sui, et~al.
\newblock Twisted-angle-dependent optical behaviors of intralayer excitons and
  trions in {WS$_2$}/{WSe$_2$} heterostructure.
\newblock {\em ACS Photonics}, 6(12):3082--3091, 2019.

\bibitem{alexeev2019resonantly}
Evgeny~M Alexeev, David~A Ruiz-Tijerina, Mark Danovich, Matthew~J Hamer,
  Daniel~J Terry, Pramoda~K Nayak, Seongjoon Ahn, Sangyeon Pak, Juwon Lee,
  Jung~Inn Sohn, et~al.
\newblock Resonantly hybridized excitons in moir{\'e} superlattices in van der
  waals heterostructures.
\newblock {\em Nature}, 567(7746):81--86, 2019.

\bibitem{seyler2019signatures}
Kyle~L Seyler, Pasqual Rivera, Hongyi Yu, Nathan~P Wilson, Essance~L Ray,
  David~G Mandrus, Jiaqiang Yan, Wang Yao, and Xiaodong Xu.
\newblock Signatures of moir{\'e}-trapped valley excitons in
  {MoSe$_2$}/{WSe$_2$} heterobilayers.
\newblock {\em Nature}, 567(7746):66--70, 2019.

\bibitem{liu2021signatures}
Erfu Liu, Elyse Barr{\'e}, Jeremiah van Baren, Matthew Wilson, Takashi
  Taniguchi, Kenji Watanabe, Yong-Tao Cui, Nathaniel~M Gabor, Tony~F Heinz,
  Yia-Chung Chang, et~al.
\newblock Signatures of moir{\'e} trions in {WSe$_2$}/{MoSe$_2$}
  heterobilayers.
\newblock {\em Nature}, 594(7861):46--50, 2021.

\bibitem{wang2021moire}
Xi~Wang, Jiayi Zhu, Kyle~L Seyler, Pasqual Rivera, Huiyuan Zheng, Yingqi Wang,
  Minhao He, Takashi Taniguchi, Kenji Watanabe, Jiaqiang Yan, et~al.
\newblock Moir{\'e} trions in {MoSe$_2$}/{WSe$_2$} heterobilayers.
\newblock {\em Nature Nanotechnology}, 16(11):1208--1213, 2021.

\bibitem{marcellina2021evidence}
Elizabeth Marcellina, Xue Liu, Zehua Hu, Antonio Fieramosca, Yuqing Huang, Wei
  Du, Sheng Liu, Jiaxin Zhao, Kenji Watanabe, Takashi Taniguchi, et~al.
\newblock Evidence for moir{\'e} trions in twisted {MoSe$_2$} homobilayers.
\newblock {\em Nano Letters}, 21(10):4461--4468, 2021.

\bibitem{mukherjee2020observation}
Arunabh Mukherjee, Kamran Shayan, Lizhong Li, Jie Shan, Kin~Fai Mak, and A~Nick
  Vamivakas.
\newblock Observation of site-controlled localized charged excitons in
  {CrI$_3$}/{WSe$_2$} heterostructures.
\newblock {\em Nature Communications}, 11(1):5502, 2020.

\bibitem{kremser2020discrete}
Malte Kremser, Mauro Brotons-Gisbert, Johannes Kn{\"o}rzer, Janine
  G{\"u}ckelhorn, Moritz Meyer, Matteo Barbone, Andreas~V Stier, Brian~D
  Gerardot, Kai M{\"u}ller, and Jonathan~J Finley.
\newblock Discrete interactions between a few interlayer excitons trapped at a
  {MoSe$_2$}/{WSe$_2$} heterointerface.
\newblock {\em npj 2D Materials and Applications}, 4(1):8, 2020.

\bibitem{xu2020correlated}
Yang Xu, Song Liu, Daniel~A Rhodes, Kenji Watanabe, Takashi Taniguchi, James
  Hone, Veit Elser, Kin~Fai Mak, and Jie Shan.
\newblock Correlated insulating states at fractional fillings of {moir\'e}
  superlattices.
\newblock {\em Nature}, 587(7833):214--218, 2020.

\bibitem{liu2021excitonic}
Erfu Liu, Takashi Taniguchi, Kenji Watanabe, Nathaniel~M Gabor, Yong-Tao Cui,
  and Chun~Hung Lui.
\newblock Excitonic and valley-polarization signatures of fractional correlated
  electronic phases in a {WSe$_2$}/{WS$_2$} {moir{\'e}} superlattice.
\newblock {\em Physical Review Letters}, 127(3):037402, 2021.

\bibitem{jin2018ultrafast}
Chenhao Jin, Eric~Yue Ma, Ouri Karni, Emma~C Regan, Feng Wang, and Tony~F
  Heinz.
\newblock Ultrafast dynamics in van der waals heterostructures.
\newblock {\em Nature Nanotechnology}, 13(11):994--1003, 2018.

\bibitem{hong2014ultrafast}
Xiaoping Hong, Jonghwan Kim, Su-Fei Shi, Yu~Zhang, Chenhao Jin, Yinghui Sun,
  Sefaattin Tongay, Junqiao Wu, Yanfeng Zhang, and Feng Wang.
\newblock Ultrafast charge transfer in atomically thin {MoS$_2$}/{WS$_2$}
  heterostructures.
\newblock {\em Nature Nanotechnology}, 9(9):682--686, 2014.

\bibitem{tan2022signature}
Qinghai Tan, Abdullah Rasmita, Zhaowei Zhang, KS~Novoselov, and Wei-bo Gao.
\newblock Signature of cascade transitions between interlayer excitons in a
  moir{\'e} superlattice.
\newblock {\em Physical Review Letters}, 129(24):247401, 2022.

\bibitem{lohof2023confined}
Frederik Lohof, Johannes Michl, Alexander Steinhoff, Bo~Han, Martin von
  Helversen, Sefaattin Tongay, Kenji Watanabe, Takashi Taniguchi, Sven
  H{\"o}fling, Stephan Reitzenstein, et~al.
\newblock Confined-state physics and signs of fermionization of {moir\'e}
  excitons in {WSe$_2$}/{MoSe$_2$} heterobilayers.
\newblock {\em arXiv preprint arXiv:2302.14489}, 2023.

\bibitem{sun2022enhanced}
Xueqian Sun, Yi~Zhu, Hao Qin, Boqing Liu, Yilin Tang, Tieyu L{\"u}, Sharidya
  Rahman, Tanju Yildirim, and Yuerui Lu.
\newblock Enhanced interactions of interlayer excitons in free-standing
  heterobilayers.
\newblock {\em Nature}, 610(7932):478--484, 2022.

\bibitem{yu2018brightened}
Hongyi Yu, Gui-Bin Liu, and Wang Yao.
\newblock Brightened spin-triplet interlayer excitons and optical selection
  rules in van der waals heterobilayers.
\newblock {\em 2D Materials}, 5(3):035021, 2018.

\bibitem{choi2021twist}
Junho Choi, Matthias Florian, Alexander Steinhoff, Daniel Erben, Kha Tran,
  Dong~Seob Kim, Liuyang Sun, Jiamin Quan, Robert Claassen, Somak Majumder,
  et~al.
\newblock Twist angle-dependent interlayer exciton lifetimes in van der waals
  heterostructures.
\newblock {\em Physical Review Letters}, 126(4):047401, 2021.

\bibitem{chatterjee2022trion}
Suman Chatterjee, Garima Gupta, Sarthak Das, Kenji Watanabe, Takashi Taniguchi,
  and Kausik Majumdar.
\newblock Trion-trion annihilation in monolayer {WS$_2$}.
\newblock {\em Physical Review B}, 105(12):L121409, 2022.

\bibitem{mak2013tightly}
Kin~Fai Mak, Keliang He, Changgu Lee, Gwan~Hyoung Lee, James Hone, Tony~F
  Heinz, and Jie Shan.
\newblock Tightly bound trions in monolayer {MoS$_2$}.
\newblock {\em Nature Materials}, 12(3):207--211, 2013.

\bibitem{kallatt2019interlayer}
Sangeeth Kallatt, Sarthak Das, Suman Chatterjee, and Kausik Majumdar.
\newblock Interlayer charge transport controlled by exciton--trion coherent
  coupling.
\newblock {\em npj 2D Materials and Applications}, 3(1):15, 2019.

\bibitem{singh2007electronic}
Jasprit Singh.
\newblock {\em Electronic and optoelectronic properties of semiconductor
  structures}.
\newblock Cambridge University Press, 2007.

\bibitem{abraham2021anomalous}
Nithin Abraham, Kenji Watanabe, Takashi Taniguchi, and Kausik Majumdar.
\newblock Anomalous stark shift of excitonic complexes in monolayer {WS$_2$}.
\newblock {\em Physical Review B}, 103(7):075430, 2021.

\bibitem{das2020highly}
Sarthak Das, Medha Dandu, Garima Gupta, Krishna Murali, Nithin Abraham,
  Sangeeth Kallatt, Kenji Watanabe, Takashi Taniguchi, and Kausik Majumdar.
\newblock Highly tunable layered exciton in bilayer {WS$_2$}: linear quantum
  confined stark effect versus electrostatic doping.
\newblock {\em ACS Photonics}, 7(12):3386--3393, 2020.

\bibitem{verzhbitskiy2019suppressed}
Ivan Verzhbitskiy, Daniele Vella, Kenji Watanabe, Takashi Taniguchi, and Goki
  Eda.
\newblock Suppressed out-of-plane polarizability of free excitons in monolayer
  {WSe$_2$}.
\newblock {\em ACS Nano}, 13(3):3218--3224, 2019.

\bibitem{klein2016stark}
Julian Klein, Jakob Wierzbowski, Armin Regler, Jonathan Becker, Florian
  Heimbach, K~Muller, Michael Kaniber, and Jonathan~J Finley.
\newblock Stark effect spectroscopy of mono-and few-layer {MoS$_2$}.
\newblock {\em Nano Letters}, 16(3):1554--1559, 2016.

\bibitem{sun2022excitonic}
Zhe Sun, Alberto Ciarrocchi, Fedele Tagarelli, Juan~Francisco Gonzalez~Marin,
  Kenji Watanabe, Takashi Taniguchi, and Andras Kis.
\newblock Excitonic transport driven by repulsive dipolar interaction in a van
  der waals heterostructure.
\newblock {\em Nature Photonics}, 16(1):79--85, 2022.

\bibitem{laikhtman2009exciton}
B~Laikhtman and Ronen Rapaport.
\newblock Exciton correlations in coupled quantum wells and their luminescence
  blue shift.
\newblock {\em Physical Review B}, 80(19):195313, 2009.

\bibitem{nagler2017interlayer}
Philipp Nagler, Gerd Plechinger, Mariana~V Ballottin, Anatolie Mitioglu,
  Sebastian Meier, Nicola Paradiso, Christoph Strunk, Alexey Chernikov,
  Peter~CM Christianen, Christian Sch{\"u}ller, et~al.
\newblock Interlayer exciton dynamics in a dichalcogenide monolayer
  heterostructure.
\newblock {\em 2D Materials}, 4(2):025112, 2017.

\bibitem{rivera2015observation}
Pasqual Rivera, John~R Schaibley, Aaron~M Jones, Jason~S Ross, Sanfeng Wu,
  Grant Aivazian, Philip Klement, Kyle Seyler, Genevieve Clark, Nirmal~J
  Ghimire, et~al.
\newblock Observation of long-lived interlayer excitons in monolayer
  {MoSe$_2$}-{WSe$_2$} heterostructures.
\newblock {\em Nature Communications}, 6(1):6242, 2015.

\bibitem{unuchek2019valley}
Dmitrii Unuchek, Alberto Ciarrocchi, Ahmet Avsar, Zhe Sun, Kenji Watanabe,
  Takashi Taniguchi, and Andras Kis.
\newblock Valley-polarized exciton currents in a van der waals heterostructure.
\newblock {\em Nature Nanotechnology}, 14(12):1104--1109, 2019.

\bibitem{kumar2014exciton}
Nardeep Kumar, Qiannan Cui, Frank Ceballos, Dawei He, Yongsheng Wang, and Hui
  Zhao.
\newblock Exciton-exciton annihilation in {MoSe$_2$} monolayers.
\newblock {\em Physical Review B}, 89(12):125427, 2014.

\bibitem{kuroda2020dark}
Takashi Kuroda, Yusuke Hoshi, Satoru Masubuchi, Mitsuhiro Okada, Ryo Kitaura,
  Kenji Watanabe, Takashi Taniguchi, and Tomoki Machida.
\newblock Dark-state impact on the exciton recombination of {WS$_2$} monolayers
  as revealed by multi-timescale pump-probe spectroscopy.
\newblock {\em Physical Review B}, 102(19):195407, 2020.

\end{thebibliography}
\section*{Acknowledgements}
S.C. and K.M. acknowledge useful discussions with Garima Gupta, Nithin Abraham, Mayank Chhaperwal, and Manish Jain. K.W. and T.T. acknowledge support from the JSPS KAKENHI (Grant Numbers 19H05790 and 20H00354). K.M. acknowledges the support from a grant from Science and Engineering Research Board (SERB) under Core Research Grant, a grant from the Indian Space Research Organization (ISRO), a grant from MHRD under STARS, and support from MHRD, MeitY, and DST Nano Mission through NNetRA.
\section*{Author contribution}
K.M. designed the experiment. M.D., S.C., and S.D. fabricated the devices and conducted the measurements. P.D. conducted the electrostatic simulation. R.B. and V.R. performed the SHG measurements for all samples. K.W. and T.T. grew the hBN crystals. S.C., M.D., and K.M. conducted the data analysis and wrote the manuscript with inputs from others.
\section*{Competing Interests}
The authors declare no competing interests.
\newpage
% Figure environment removed
\newpage
% Figure environment removed
\newpage
% Figure environment removed
\newpage
% Figure environment removed
\AtEndDocument{\includepdf[pages={2-13}]{Supplementary_information}}
\end{document} 