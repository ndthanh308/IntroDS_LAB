


\subsection{General (full) identifiability $\implies$ (local) $\mathcal{F}$-identifiability }

In the following, we demonstrate that classical identifiability results from the literature can be used to assess the $\mathcal{F}$-plausibility of a set $S$. Informally, we show that if all variables in the SCM follow an identifiable $\mathcal{F}$-model, then any set $S$ containing a child of $Y$ cannot be $\mathcal{F}$-plausible. To state the result, we require the following definition.

We define a \textbf{projection} of a graph $\mathcal{G} = (V, E)$ onto $S \subseteq V$, denoted as $\mathcal{G}[S]$, as a graph with vertices $S$ and the following edges: for distinct $i, j \in S$, there is an edge $i \to j$ in $\mathcal{G}[S]$ if there is a directed path from $i$ to $j$ in $\mathcal{G}$ such that all vertices on this path except $i, j$ do not belong to $S$ ($i \to \dots \to j$). Moreover, there is a bidirected edge $i - j$ if a path exists between $i, j$ in $\mathcal{G}$ that does not contain a collider and the first edge points towards $i$ and the last edge points towards $j$ ($i \leftarrow \dots \to j$).

For simplicity, we focus on the case when $\textbf{X} = (X_1, \dots, X_p)$ are neighbors (either direct causes or direct effects) of $Y$ in the corresponding SCM. Using the classical conditional independence approach and d-separation \citep{Pearl_causal_diagrams_biometrika}, we can eliminate other variables from being potential parents of $X_0 = Y$. Nevertheless, the theory can be extended to non-neighbors as well.

\begin{proposition}
\label{TheoremFidentifiabilityWithChild}
Let $(X_0, \textbf{X})$ follow an SCM with DAG $\mathcal{G}_0$ such that (\ref{SCM_for_Y}) holds. Assume that all $\textbf{X}$ are neighbors of $X_0$ in $\mathcal{G}_0$. Let $S \subseteq \{1, \dots, p\}$ and denote $S_0 = S \cup \{0\}$. Suppose $S$ contains a child of $X_0$ in $\mathcal{G} := \mathcal{G}_0[S_0]$; that is, $\exists j \in ch_0(\mathcal{G})$. Assume that $\mathcal{G}$ is a DAG.

Let $(X_0, \textbf{X}_S)$ follow an identifiable $\mathcal{F}_A$-model with DAG $\mathcal{G}$. Then, $S$ is not $\mathcal{F}_A$-plausible.
\end{proposition}

\begin{proof}
Without loss of generality, let $X_1$ be a childless child of $X_0$ in $S$ (the set of children of $X_0$ is nonempty by assumption, and one of them must be childless to avoid cycles). Denote $k = |S| \in \mathbb{N}$. Before we start with the proof, note the following trivial observation:

\textbf{Observation:} In every DAG $\mathcal{G}$ where all vertices are neighbors of $X_0$ and $X_1$ is the childless child of $X_0$, there exists a vertex $i$ such that either 1) $pa_i = \emptyset$, or 2) $pa_0 = \emptyset$ and $pa_i \subseteq \{0\}$.

We start with the case when $k = 1$ and $X_0 \to X_1$. Then, we use the principle of induction and Lemma~\ref{lemma_global_local_identifiability} to prove that any $S$ containing the vertex $X_1$ is not $\mathcal{F}_A$-plausible in $\mathcal{G}$.

\textbf{Step 1 ($k=1$)}: Let $\mathcal{G}_0 := \{0 \to 1\}$. Assume that $(X_0, X_1)$ follows an identifiable $\mathcal{F}_A$-model with DAG $\mathcal{G} = \mathcal{G}_0$. Then, trivially, $S = \{1\}$ is not $\mathcal{F}_A$-plausible. By definition, there does not exist $f \in \mathcal{F}_A$ such that $X_0 = f(X_1, \varepsilon_0)$. This is precisely the definition of $\mathcal{F}_A$-plausibility. Hence, $S = \{1\}$ is not $\mathcal{F}_A$-plausible.

\textbf{Step 2 ($k = 2$)}: Recall that all variables are neighbors of $X_0$ and $1 \in ch_0(\mathcal{G})$. There are four graphs satisfying these conditions: $\{2 \to 0 \to 1\}$, $\{2 \to 0 \to 1 \leftarrow 2\}$, $\{0 \to 1 \leftarrow 2 \leftarrow 0\}$, $\{2 \leftarrow 0 \to 1\}$. Notice that in all of these cases, $pa_2 \subseteq \{0\}$. Using Lemma~\ref{lemma_global_local_identifiability}, we condition on $X_2 = x$ and obtain a bivariate identifiable $\mathcal{F}_A$-model with $1 \in ch_0$. Using Step 1, we showed that $f \in \mathcal{F}_A$ cannot exist such that $Y = f(X_1, x, \varepsilon_Y)$, and therefore (\ref{SCM_for_Y}) cannot be satisfied. Hence, $S = \{1, 2\}$ is not $\mathcal{F}_A$-plausible.

\textbf{Step 3 ($k \to k-1$)}: Using the Observation, find a vertex $i$ such that either $pa_i = \emptyset$, or $pa_0 = \emptyset$ and $pa_i \subseteq \{0\}$. Using Lemma~\ref{lemma_global_local_identifiability}, we condition on $X_i = x$ and obtain an identifiable $\mathcal{F}_A$-model with $1 \in ch_0$. Using the previous step, $f \in \mathcal{F}_A$ cannot exist such that $Y = f(X_1, \dots, X_{i-1}, x, X_{i+1}, \dots, X_k, \varepsilon_Y)$, and therefore (\ref{SCM_for_Y}) cannot be satisfied. Hence, $S$ is not $\mathcal{F}_A$-plausible.
\end{proof}

\begin{lemma}
\label{lemma_global_local_identifiability}
Let $(X_0, \textbf{X})$ follow an identifiable $\mathcal{F}_A$-model with DAG $\mathcal{G}$, where all $\textbf{X}$ are neighbours of $X_0$.  Let $S=\{2, \dots, p\}$.  Let one of the following conditions hold: 

\begin{enumerate}
    \item  Let $X_1$ be the source variable ($pa_1=\emptyset)$. 
    \item Let  $X_0$ be the source variable ($pa_0=\emptyset)$ and $pa_1 \subseteq \{0\}$ (that is, all arrows from $X_0, X_1$ are only outgoing except with a possible $X_0\to X_1$).  
\end{enumerate}

Then, there exist $x\in supp(X_1)\subseteq \mathbb{R}$ such that  $(X_0, \textbf{X}_{S})\mid X_1=x$ follow an identifiable $\mathcal{F}_A$-model; that is, there exist exactly one DAG $\mathcal{G}_0$ such that the joint distribution $P_{(X_0, \textbf{X}_{S})\mid X_1=x}$ can be generated via an $\mathcal{F}_A$-model with graph $\mathcal{G}_0$. In particular, $\mathcal{G}_0$ is equal to $\mathcal{G}$ where we erase vertex $\{1\}$ and all edges connected with this vertex. 
\end{lemma}

\begin{proof}
\textit{Example and discussion about the main idea:} Is it possible to construct a counterexample where the full  $\mathcal{F}_A$-model is identifiable but if we condition on $X_1=x$ then multiple $\mathcal{F}_A$-models can generate the same conditional distribution? Related question was considered in the Example 26 from \cite{Peters2014}. There, the SCM has the form $X_1\to X_2\to X_3\leftarrow X_1$; notice that $X_1$ is the source, and $$X_1 = \eta_1, X_2 = f_2(X_2)+\eta_2, X_3 = f_3(X_1) + aX_2+\eta_3,$$ where $\eta_1\sim t_{v=3}, \eta_2\sim N(0, \sigma_2^2),\eta_3\sim N(0, \sigma_3^2) $ are independent, $a\neq 0$. Here, $X_2, X_3$ are non-Gaussian but $(X_2, X_3\mid X_1=x)$ is linear Gaussian. Hence,  $\mathcal{F}_A$-model  is not identifiable from $P_{X_2, X_3\mid X_1=x}$ . Therefore, we can revert this equation and obtain the same joint distribution generated by an  $\mathcal{F}_A$-model of the form $$X_1 = \eta_1, X_2 = g_2(X_1)+bX_3 + \tilde{\eta}_2, X_3 = g_3(X_1) +\tilde{\eta}_3,$$ where $\eta_1\sim t_{v=3}, \tilde{\eta}_2\sim N(0, \tilde{\sigma}_2^2),\tilde{\eta}_3\sim N(0, \tilde{\sigma}_3^2) $, for some $g_1, g_2$. Hence, the original   $\mathcal{F}_A$-model is not identifiable. 

\textbf{Proof. }

\textbf{Part 1:} Consider the following notation of the SCM over  $(X_0, \textbf{X})$: 
$$
X_1 = \eta_1,\,\,\,\,\,\,\,\,\, X_i = f_i(\textbf{X}_{pa_i})+\eta_i, \,\,\,i=0,2,3,\dots, p, \,\,\,\,\,{\eta}_i \text{ independent}. 
$$
Let $x$ be in the support of $X_1$. Then, consider a SCM with $\mathcal{G}_0$ defined in the statement and link functions
$$
X_i=\begin{cases} 
 & f_i^x(\textbf{X}_{pa_i})+\eta_i =f_i(\textbf{X}_{pa_i})+\eta_i, \,\,\,\,\,\text{if} \,i\not\in ch_1, \\
 & f_i^x(\textbf{X}_{pa_i})+\eta_i = f_i(x, \textbf{X}_{pa_i\setminus \{1\}})+\eta_i, \,\,\,\,\,\text{if} \,i\in ch_1. 
\end{cases}
$$
where the link functions have the form
$$
f_i^x(\textbf{x}_{pa_i})=\begin{cases} 
 & f_i(\textbf{x}_{pa_i}) \,\,\,\,\,\text{if} \,i\not\in ch_1, \\
 & f_i(x, \textbf{x}_{pa_i\setminus \{1\}}), \,\,\,\,\,\text{if} \,i\in ch_1. 
\end{cases}
$$
This is obviously an $\mathcal{F}_A$-model with DAG  $\mathcal{G}_0$ and the joint distribution is equal to  $P_{(X_0, \textbf{X}_{S})\mid X_1=x}$ . We will show that this model is identifiable for at least one choice of $x$. 

For a contradiction, suppose the that model is not identifiable for any $x$. We will find an $\mathcal{F}_A$-model with DAG  $\tilde{\mathcal{G}}\neq \mathcal{G}$ that generates the joint distribution $P_{(X_0, \textbf{X}_{S}) }$. If we manage to prove that, it gives us the desired contradiction. 

The non-identifiability for any $x$  gives us an existence of an $\mathcal{F}_A$-model defined by  $(\tilde{\mathcal{G}}_0,{\tilde{f}}^x_i, \tilde{\eta}^x_i)$ generating the same joint distribution  $P_{(X_0, \textbf{X}_{S})\mid X_1=x}$ (here, trivially $\tilde{\mathcal{G}}_0$ does not depend on $x$ from continuity of variables). 
Define $\tilde{\mathcal{G}}$ as $\tilde{\mathcal{G}}_0$ with additional one vertex $\{1\}$ and edges from $\{1\}$ to $ch_1(\mathcal{G})$. Obviously  $\tilde{\mathcal{G}}\neq \mathcal{G}$.  Consider a SCM over  $(X_0, \textbf{X})$ with DAG $\tilde{\mathcal{G}}$ and 
\begin{equation*} \begin{split}
&X_1 = \eta_1,\\&    
X_i=\tilde{f}_i^{X_1}(\textbf{X}_{pa_i(\tilde{\mathcal{G}})})+\tilde{\eta}^{X_1}_i,
    \end{split}
\end{equation*}
where $\tilde{\eta}^{X_1}_i$ are all jointly independent given $X_1$. However, for all $i\in ch_1(\mathcal{G})$, we know that $X_1$ must have an additive effect of $X_i$ (affects only mean, and not variance for example) and therefore $\tilde{\eta}^{X_1}_i = g_i(X_1) + \tilde{\eta}_i$ for some functions $g_i$ noise variables $\tilde{\eta}_i$. Therefore, we found an  $\mathcal{F}_A$-model with DAG $\tilde{\mathcal{G}}$, which is the desired contradiction. 

\textbf{Part 2: }
 Original SCM over  $(X_0, \textbf{X})$ can be written as 
$$
X_0 = \eta_0,\,\,\,X_1 = f_1(X_0)+\eta_1, \,\,\,\,\,\,\,\,\, X_i = f_i(\textbf{X}_{pa_i})+\eta_i, \,\,\,i=0,2,3,\dots, p, \,\,\,\,\,{\eta}_i \text{ independent}. 
$$
Let $x$ be in the support of $X_1$. Then, consider a SCM with $\mathcal{G}_0$ defined in the statement and link functions
$$
X_0 = \tilde{\eta}_0, \,\,\,\,\text{ where } \tilde{\eta}_0\sim X_0\mid X_1,
$$
$$
X_i=\begin{cases} 
 & f_i^x(\textbf{X}_{pa_i})+\eta_i =f_i(\textbf{X}_{pa_i})+\eta_i, \,\,\,\,\,\text{if} \,i\not\in ch_1, \\
 & f_i^x(\textbf{X}_{pa_i})+\eta_i = f_i(x, \textbf{X}_{pa_i\setminus \{1\}})+\eta_i, \,\,\,\,\,\text{if} \,i\in ch_1. 
\end{cases}
$$
where the link functions have the form
$$
f_i^x(\textbf{x}_{pa_i})=\begin{cases} 
 & f_i(\textbf{x}_{pa_i}) \,\,\,\,\,\text{if} \,i\not\in ch_1, \\
 & f_i(x, \textbf{x}_{pa_i\setminus \{1\}}), \,\,\,\,\,\text{if} \,i\in ch_1. 
\end{cases}
$$
This is obviously an $\mathcal{F}_A$-model and the joint distribution is equal to  $P_{(X_0, \textbf{X}_{S})\mid X_1=x}$ . We will show that this model is identifiable for at least one choice of $x$. 

For a contradiction, suppose the that model is not identifiable for any $x$. We will find an $\mathcal{F}_A$-model with DAG  $\tilde{\mathcal{G}}\neq \mathcal{G}$ that generates the joint distribution $P_{(X_0, \textbf{X}_{S}) }$. If we manage to prove that, it gives us the desired contradiction. 

The non-identifiability for any $x$  gives us an existence of an $\mathcal{F}_A$-model defined by  $(\tilde{\mathcal{G}}_0,{\tilde{f}}^x_i, \tilde{\eta}^x_i)$ generating the same joint distribution  $P_{(X_0, \textbf{X}_{S})\mid X_1=x}$ (here, trivially $\tilde{\mathcal{G}}_0$ does not depend on $x$ from continuity of variables). 
Define $\tilde{\mathcal{G}}$ as $\tilde{\mathcal{G}}_0$ with additional one vertex $\{1\}$ and edges from $\{1\}$ to $ch_1(\mathcal{G})$. Obviously  $\tilde{\mathcal{G}}\neq \mathcal{G}$.  Consider a SCM over  $(X_0, \textbf{X})$ with DAG $\tilde{\mathcal{G}}$ and  
\begin{equation*} \begin{split}
&X_0 = \eta_0,\\&    
X_i=\tilde{f}_i^{X_1}(\textbf{X}_{pa_i(\tilde{\mathcal{G}})})+\tilde{\eta}^{X_i}_i.
    \end{split}
\end{equation*}
Using the same argument as in the part 1, this is an $\mathcal{F}_A$-model with DAG  $\tilde{\mathcal{G}}$ that generates the joint distribution $P_{(X_0, \textbf{X}_{S}) }$. This is a contradiction with identifiability of  $\mathcal{G}$ in the original SCM.   

\end{proof}

We showed that if all variables in the SCM follow an identifiable $\mathcal{F}_A$-model and $\mathcal{G}$ is a DAG, then any set $S$ containing a child of $Y$ cannot be $\mathcal{F}_A$-plausible. In practice, this implies that we can use classical identifiability results to assess the $\mathcal{F}$-plausibility of a given set $S$.





