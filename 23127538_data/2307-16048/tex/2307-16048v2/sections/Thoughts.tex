





\begin{lemma}\label{CoolLemma}
Let $\textbf{X}=(X_1, \dots, X_k)$ be a continuous random vector with independent components and $s<k$. 
\begin{enumerate}
\item Let $f:\mathbb{R}^{k}\to\mathbb{R}$ be continuous injective real function such that there does not exist a decomposition  $ f(\textbf{x}) = f_1(\textbf{x}_S) + f_2(\textbf{x}_{\{1, \dots, k\}\setminus S}), \textbf{x}\in\mathbb{R}^{k}$ for any non-empty $S\subset \{1, \dots, k\}$, where $f_1, f_2$ are some measurable functions. 

 Then, a non-zero function $h$ does not exist such that for $s<k$ holds
\begin{equation}\label{yuiMAIN}
f(X_1, \dots, X_k)+  h(X_1, \dots, X_s) \indep (X_1, \dots, X_s).
\end{equation}

\item Let $\textbf{X}=(X_1, \dots, X_k)$ be non-degenerate Gaussian random vector (possibly with dependent components). Let $f:\mathbb{R}^{k}\to\mathbb{R}$ be continuous injective additively-transitive \footnote{Function is said to be ``additively transitive'', if $f(x, y+z) = f(x, y) + f(x, z)$. For example, $f(x, y) = xy$ is additively transitive. 
} real function such that there does not exist a decomposition  $ f(\textbf{x}) = f_1(\textbf{x}_S) + f_2(\textbf{x}_{\{1, \dots, k\}\setminus S}), \textbf{x}\in\mathbb{R}^{k}$ for any non-empty $S\subset \{1, \dots, k\}$, where $f_1, f_2$ are some measurable functions. 

 Then, a non-zero function $h$ does not exist such that for $s<k$ holds (\ref{yuiMAIN}). 



\end{enumerate}
\end{lemma}
\begin{proof}
We use notation $\textbf{X}_S=(X_1, \dots, X_s)^\top$, $\textbf{X}_{\setminus S}=(X_{s+1},\dots X_k)^\top$. 
Let us introduce functionals (not norms, we only use them to simplify notation) $||\cdot||_{plus}$ and  $||\cdot||_{times}$ , defined by $|| \textbf{a} ||_{plus} = a_1 +\dots + a_d$ , $|| \textbf{a} ||_{times} = a_1 a_2\dots a_d$ , for $\textbf{a}=(a_1, \dots, a_d)^\top\in\mathbb{R}^d$. 

\textbf{Part 5: } We use the following well-known result. For a multivariate normal vector  \begin{align*}
    \textbf{Z}=(\textbf{Z}_1, \textbf{Z}_2)^\top &\sim N\bigg(\begin{pmatrix}
           \mu_1 \\
            \mu_2 
         \end{pmatrix}, 
         \begin{pmatrix}
           \Sigma_{1,1} ,   \Sigma_{1,2} \\
             \Sigma_{2,1},   \Sigma_{2,2}
         \end{pmatrix}\bigg),
 \end{align*}
where $\textbf{Z}_1, \textbf{Z}_2$ is a partition of $\textbf{Z}$ into smaller sub-vectors, it holds that $(\textbf{Z}_1|\textbf{Z}_2=a)$, the conditional distribution of the first partition given the second, has distribution equal to $N(\mu_a, \tilde{\Sigma})$, where
$\mu_a = \mu_1 + \Sigma_{1,2}\Sigma_{2,2}^{-1}(a-\mu_2)$, $\tilde{\Sigma} = \Sigma_{1,1} - \Sigma_{1,2}\Sigma^{-1}_{2,2}\Sigma_{2,1}$. Specifically,  $(\textbf{Z}_1|\textbf{Z}_2=a)-\mu_a\overset{D}{=}(\textbf{Z}_1|\textbf{Z}_2=b) - \mu_b$ for any $a, b$.

\textit{Proof of Part 5: }
For a contradiction, let such $h$ exist. Define $\xi:=h(\textbf{X}_S) + f(\textbf{X}_S, \textbf{X}_{\setminus S}) $, which is the left hand side of (\ref{yuiMAIN}). 

Fix $\textbf{a}_0\in\mathbb{R}^s$ in the support of $\textbf{X}_S$ . Since $\xi\indep \textbf{X}_S$, for all $\textbf{x}\in\mathbb{R}^{s}$ holds   $\xi\mid [\textbf{X}_S=\textbf{a}_0] \overset{D}{=}\xi\mid [\textbf{X}_S=\textbf{x}]$. Using the ``well-known result'', we know that  $\textbf{X}_{\setminus S} - \mu_a\mid \textbf{X}_S =a$ has the same distribution as   $\textbf{X}_{\setminus S} - \mu_b\mid \textbf{X}_S =\textbf{x}$.  

Hence, 
\begin{align*}
&h(\textbf{x}) + f(\textbf{x}, \textbf{X}_{\setminus S} - \mu_{\textbf{x}})\overset{D}{=}h(\textbf{a}_0) + f(\textbf{a}_0, \textbf{X}_{\setminus S} -\mu_{\textbf{a}_0} )
\end{align*}
Using the property  $f(\textbf{x}, y+z) = f(\textbf{x}, y) + f(\textbf{x}, z)$, we obtain  
$$
h(\textbf{x}) + f(\textbf{x}, \textbf{X}_{\setminus S}) +  f(\textbf{x}, -\mu_\textbf{x}) \overset{D}{=}h(\textbf{a}_0) + f(\textbf{a}_0, \textbf{X}_{\setminus S} ) + f(\textbf{a}_0,  -\mu_{\textbf{a}_0})
$$

Define

$$f_1(\textbf{x}):=h(\textbf{a}_0) - h(\textbf{x}) -f(\textbf{x}, -\mu_\textbf{x}) + f(\textbf{a}_0, -\mu_{\textbf{a}_0}), \,\,\text{for}\,\,\textbf{x}\in\mathbb{R}^{s}, \,\,\,\,\,and\,\,\,\,\,\, f_2(\textbf{x}) := f(\textbf{a}_0, \textbf{x})\,\,\text{for}\,\,  \textbf{x}\in\mathbb{R}^{k-s}.   $$
Then we have $$
f(\textbf{x}, \textbf{X}_{\setminus S})\overset{D}{=} f_1(\textbf{x}) +f_2(\textbf{X}_{\setminus S}). $$
This is analogous to Equation~\ref{awrefarefs} and we follow the same steps: we extend the equality from equality in distribution to equality everywhere using Lemma~\ref{lemma_additivity}. We found an additive decomposition of $f$, which is the desired contradiction. 
 
 
\end{proof}
