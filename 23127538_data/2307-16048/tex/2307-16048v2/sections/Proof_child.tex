\begin{customprop}{
\ref{TheoremFidentifiabilityWithChild}}
Let $(X_0, \textbf{X})$ follow an (identifiable) restricted $\mathcal{F}_A$-model with DAG $\mathcal{G}$, such that all $\textbf{X}$ are neighbors of $X_0$ in $\mathcal{G}$.   Let $S \subseteq \{1, \dots, p\}$ contain a child of $X_0$ in $\mathcal{G}$.  Then, $S$ is not $\mathcal{F}_A$-plausible. 
\end{customprop}

\begin{proof}
 \label{proof of TheoremFidentifiabilityWithChild} 
    For a contradiction, let $S$ be $\mathcal{F}_A$-plausible. Without loss of generality, let $X_1$ satisfy 
    
    \begin{enumerate}
        \item $1\in S$,
        \item $X_1$ is a child of $X_0$,
        \item $X_1$ is childless.
    \end{enumerate}
    
We explain why this is without loss of generality.  By assumptions, the set $\tilde{S}:=\{i\in S : \,X_i \text{ is a child of }X_0\}$ is non-empty. Moreover, there must exist $j\in \{1, \dots, p\}$ that is a childless child of $X_0$ in $\mathcal{G}$ (otherwise there would be a cycle). If $j\in\tilde{S}$ then $X_j$ satisfies all conditions 1, 2 and 3. Without loss of generality, we rename $j=1$. 
If  $j\not\in\tilde{S}$, then $(X_0, \textbf{X}_{\{1, \dots, p\}\setminus j})$ also follows a restricted $\mathcal{F}_A$-model and since $j\not\in S$, we can focus only on proving Proposition~\ref{TheoremFidentifiabilityWithChild} restricted to variables  $(X_0,  \textbf{X}_{\{1, \dots, p\}\setminus j})$. Iteratively, after a finite number of steps, we find that restricted $\mathcal{F}_A$-model  $(X_0, \textbf{X}_{\{1, \dots, p\} \setminus \{j_1, \dots, j_k\}})$ satisfies $j\in\tilde{S}$. 


The idea of the proof is that we define two bivariate $\mathcal{F}_A$-models, one with $X_0\to X_1$ and one with $X_1\to X_0$, which will lead to a contradiction with the identifiability of the original restricted $\mathcal{F}_A$-model. 

Since $(X_0, \textbf{X})$ follow an $\mathcal{F}_A$-model, we can write  $X_i = f_i(\textbf{X}_{pa_i}) + \eta_i$, where $f_i$ are some measurable functions and $\eta_i$ are jointly independent, $i\in \{0, \dots, p\}$. Specifically, we have 
\begin{equation*}
    \begin{split}
        X_0 = f_0(\textbf{X}_{pa_0}) + \eta_0, \,\,\,\,\,\,\,X_1 = f_1(X_0, \textbf{X}_{pa_1\setminus\{1\}}) + \eta_1, 
    \end{split}
\end{equation*}
where $\eta_1 \indep \textbf{X}_{\{0, 2, 3, \dots, p\}}$ .Conditioning on $\textbf{X}_{\{ 2, 3, \dots, p\}} = \textbf{x}$, we obtain 
\begin{equation*}
    \begin{split}
\textbf{SCM\,1:} \,\,\,\,\,\,\,\,\,\,\,\,\,       X_0 = \tilde{\eta}_0, \,\,\,\,\,\,\,X_1 = f_1(X_0, \textbf{x}_{pa_1\setminus\{1\}}) + \eta_1, 
    \end{split}
\end{equation*}
where $\tilde{\eta}_0\sim X_0\mid\textbf{X}_{\{2, 3, \dots, p\}} = \textbf{x}$ and $\eta_1\indep X_0$. 


From the fact that $S$ is $\mathcal{F}_A$-plausible, we can find a function $f$ such that $\eta_S:=X_0 - f(\textbf{X}_S)$ satisfies $\eta_S\indep \textbf{X}_S$. Hence, we can write $$
X_0 =f(X_1, \textbf{X}_{S\setminus\{1\}}) + \eta_S,$$ where $\eta_S\indep \textbf{X}_S$.  Conditioning on  $\textbf{X}_{\{2, 3, \dots, p\}} = \textbf{x}$, we obtain
$$
\textbf{SCM\,2:} \,\,\,\,\,\,\,\,\,\,\,\,\,      X_1 = \tilde{\eta}_1, \,\,\,\,\,X_0 =f(X_1, \textbf{x}_{S\setminus\{1\}}) + \eta_S,$$ where $\tilde{\eta}_1\sim X_1\mid X_{\{2, 3, \dots, p\}} = \textbf{x}$  and $\eta_S\indep X_1$. 

Notice that in both Models 1 and 2, the joint distribution of $(X_0, X_1)$ is equal to $P_{X_0, X_1\mid (X_2, \dots, X_p)=\textbf{x}}$ and hence, we were able to find two additive noise models generating the same joint distribution, where the first model follows restricted additive noise model. This is a direct contradiction with the identifiability of restricted additive noise model (Theorem~\ref{theorem20}). Therefore, $S$ is not $\mathcal{F}_F$-plausible.
 
\end{proof}










