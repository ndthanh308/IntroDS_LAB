
To illustrate our methodology using a real-world example, we consider data on the fertility rate. This example was also used by \cite{Christina}, who employed the ICP methodology. We show that our results are in line with the findings of  \cite{Christina}, while we drop the assumption of different environments and consider only one environment.

The target variable of interest is $Y = \text{`Fertility rate'},$ measured yearly in more than 200 countries.  Developing countries exhibit a significantly higher fertility rate than Western countries \citep{hirschman1994}. The fertility rate can be predicted by considering covariates such as the `infant mortality rate' or `GDP.' However, if one wants to explore the potential effect of a particular law or a policy change, it becomes necessary to leverage the causal knowledge of the underlying system.

Randomized studies are not possible to design in this context since factors like `infant mortality rate' cannot be isolated for manipulation. Even so, understanding the impact of policies to reduce infant mortality rates within a country remains an important question, even if randomized studies are unfeasible.

Here, we consider covariates $\textbf{X} = (X_1, X_2, X_3, X_4)^\top$, where $X_1=$`GDP (in US dollars)', $X_2 = $`Education expenditure (\% of GDP)', $X_3 = $`Infant mortality rate (infant deaths per 1,000 live births)', $X_4 = $`Continent'. The data are taken from \cite{worldbank_data, unitednations_data_about_fertility}. 

We apply the methodology developed in this paper to estimate the causes of $Y$. Since our variables are continuous and regular, it seems natural and justifiable to use the following choices for $\mathcal{F}$:  $\mathcal{F}_A, \mathcal{F}_{F}$ or $ \mathcal{F}_{LS}$, where $F = Gaussian$. Note that $\mathcal{F}_A, \mathcal{F}_F\subset\mathcal{F}_{LS}$.

For the choice $\mathcal{F} = \mathcal{F}_A$, we observe that all sets $S\subseteq \{1, \dots, 4\}$ are strongly rejected as $\mathcal{F}$-plausible and our estimate is an empty set. Our data show heteroschedasticity and much more complex relations than those that can be described by just one parameter (the mean).  

Applying our methodology with the choices $\mathcal{F}_{LS}$ and $ \mathcal{F}_{F}$, we obtain the results described in Table~\ref{Table_application}. The results suggest that $X_3$ is the identifiable cause of $Y$. This is in line with findings from \cite{Christina} (backed up by research from sociology in \cite{hirschman1994}), who also discovered the variable $X_3$ to be causal. Furthermore, the score-based estimate  indicates that $X_2$ is a member of $\widehat{pa}_Y$ across both selections of $\mathcal{F}$. This suggests that $X_2$ is a cause of $Y$ as well, even though the score-based estimate does not have the same guarantees as the set $\hat{S}_{\mathcal{F}}(Y)$. Note that sets $\{2,3\}, \{1,2,3\}$ are $\mathcal{F}$-plausible for both choices of $\mathcal{F}$. 

Explaining changes in fertility rate is still a topical issue. In our study, we focus on using our developed framework to provide data-driven answers about the potential causes of changes in fertility rates. While models for the fertility rate often have a DAG structure when dynamics are measured, marginalizing to a cross-section may produce relationships which violate the acyclicity constraint \citep{koyama2022how}. This application serves as an illustration of our methodology, while we do not discuss validity of a DAG structure of the variables measured. Moreover the findings rely on the local causal sufficiency of $Y$, an assumption that can surely be questioned. For instance, other variables such as `religious beliefs' or a `political situation' may explain the fertility rate, but are hard to measure. 


% Please add the following required packages to your document preamble:
% \usepackage{multirow}
\begin{table}[h]
\centering
\begin{tabular}{|c|c|c|c|}
\hline
\multirow{2}{*}{$\mathcal{F}$}       & \multirow{2}{*}{$\mathcal{F}$-plausible sets} & ISD estimate of the                                       & Score-based\\
                                     &                                               & $\mathcal{F}$-identifiable set $\hat{S}_{\mathcal{F}}(Y)$ &                                               estimate of $\widehat{pa}_Y$\\ \hline
$\mathcal{F}_{LS}$                   & \{2,3\}, \{3,4\}, \{1,2,3\}, \{1,3,4\}        & \{3\}                                                     & \{2,3\}                                       \\ \hline
$\mathcal{F}_{F}$& \{2,3\}, \{2,3,4\}, \{1,2,3\}, \{1,3,4\}      & \{3\}                                                     & \{1,2,3\}                                     \\ \hline
\end{tabular}
\caption{Estimated causal predictors of fertility rate under different function classes $\mathcal{F}$. Here, $F$ denotes the Gaussian distribution.}
\label{Table_application}
\end{table}