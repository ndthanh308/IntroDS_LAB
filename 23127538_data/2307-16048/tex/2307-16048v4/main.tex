
%--------------- Personalize your document here ---------------

\author{} % Enter your name
\newcommand{\studentID}{} % enter your student ID
\newcommand{\supervisorone}{} % Enter your supervisor's name
\newcommand{\supervisortwo}{}% Leave it empty or enter your second supervisor's name 
\newcommand{\department}{}
\newcommand{\exam}{}

\title{} %Enter the title of your report 
\date{\today} % insert a specific date	

%--------------------------------------------------------------

% This document was adapted from the
% TEMPLATE FOR PHYS250 WORKSHEET created by Alastair McLean
% URL: https://www.overleaf.com/latex/templates/phys250-worksheet-template/xxftvfhmwqdt

% Jefferson Silveira
% Email: 19jdls1@queensu.ca
% Last update: 09-Jun-2021
% If you have any questions or concerns, do not hesitate to contact me.
%--------------------------------------------------------------

\documentclass[a4paper,11pt]{article}
\usepackage[left=30mm,top=30mm,right=30mm,bottom=30mm]{geometry}
\usepackage{etoolbox} %required for cover page
\usepackage{booktabs}
\usepackage[table,xcdraw]{xcolor}
\usepackage[usestackEOL]{stackengine}
\usepackage[round]{natbib}
\usepackage[T1]{fontenc}
\usepackage[utf8]{inputenc}
\usepackage{bm}
\usepackage{graphicx}
\usepackage{subcaption}
\usepackage{amsmath}
\usepackage{amsfonts}
\usepackage{mathtools}
\usepackage{xcolor}
\usepackage{float}
\usepackage{hyperref}
\usepackage[capitalise]{cleveref}
\usepackage{enumitem,kantlipsum}
\usepackage{amssymb}
\usepackage{amsbsy}
\usepackage{amsthm}
\usepackage{bbm}% theorems, definitions, etc.
\usepackage{pifont}
 \usepackage{multirow}
 \usepackage{footnote}
 \makesavenoteenv{tabular}
 \usepackage{url}
 \usepackage{eucal}
 \usepackage{wrapfig}
 
\usepackage[ruled,vlined]{algorithm2e}
\usepackage{listings}
\usepackage{dirtytalk}
\usepackage{graphicx}
\usepackage{multirow}

\usepackage{chngcntr}
\usepackage{apptools}
\AtAppendix{\counterwithin{lemma}{section}}
\AtAppendix{\counterwithin{theorem}{section}}
\AtAppendix{\counterwithin{proposition}{section}}
\AtAppendix{\counterwithin{consequence}{section}}

%\renewcommand{\listingscaption}{Algorithm}
%\renewcommand{\listoflistingscaption}{List of Algorithms}
\newcommand{\E}{\mathbb{E}}
\newcommand{\R}{\mathbb{R}}
\DeclareMathOperator*{\argmin}{arg\,min}
\DeclareMathOperator*{\argmax}{arg\,max}
\newcommand{\F}{\mathcal{F}}
\newcommand{\M}{ (\mathcal{M}_1, \mathcal{M}_2)}


\newtheorem{proposition}{Proposition}
\newtheorem{definition}{Definition}

\newtheorem{example}{Example}
\newtheorem{remark}{Remark}
\newtheorem{assumption}{Assumption}
\newtheorem*{assumption*}{Assumption}
\newtheorem{observation}{Observation}
\newtheorem*{terminology}{Terminology}
\newtheorem{consequence}{Consequence}
\newtheorem{identity}{Identity}


\newtheorem{innercustomthm}{Theorem}
\newenvironment{customthm}[1]
  {\renewcommand\theinnercustomthm{#1}\innercustomthm}
  {\endinnercustomthm}
  

\newtheorem{innercustomlem}{Lemma}
\newenvironment{customlem}[1]
  {\renewcommand\theinnercustomlem{#1}\innercustomlem}
  {\endinnercustomlem}


\newtheorem{innercustomprop}{Proposition}
\newenvironment{customprop}[1]
  {\renewcommand\theinnercustomprop{#1}\innercustomprop}
  {\endinnercustomprop}


  
\newtheorem{innercustomremark}{Remark}
\newenvironment{customremark}[1]
  {\renewcommand\theinnercustomremark{#1}\innercustomremark}
  {\endinnercustomremark}


  
\newtheorem{innercustomconsequence}{Consequence}
\newenvironment{customconsequence}[1]
  {\renewcommand\theinnercustomconsequence{#1}\innercustomconsequence}
  {\endinnercustomconsequence}
  
  \newtheorem{theorem}{Theorem}
\newtheorem{lemma}{Lemma}
\newtheorem{notation}{Notation}

\def\b#1{{\color{red}\bf #1}}%

\def\lm#1{{\textcolor{purple}{LM: \bf #1}}}

\newenvironment{myproof}{
  \par\medskip\noindent
  \textit{Proof}.
}{
\newline
\rightline{$\qedsymbol$}
}

\newenvironment{hproof}{%
  \renewcommand{\proofname}{Idea of the proof}\proof}{\endproof}

\newcommand\independent{\protect\mathpalette{\protect\independenT}{\perp}}
\newcommand{\indep}{\perp \!\!\! \perp}
\def\independenT#1#2{\mathrel{\rlap{$#1#2$}\mkern2mu{#1#2}}}

\setlength{\bibsep}{1pt plus 0.1ex}

\bibliographystyle{plainnat}

\hypersetup{
    colorlinks,
    linkcolor={black},
    citecolor={blue!50!black},
    urlcolor={blue!80!black}
}
\newtheorem{condition}{Condition}
\linespread{1}

\graphicspath{{figures/}}	

% Keywords command
\providecommand{\keywords}[1]
{
  \small	
  \textbf{\textit{Keywords: }} #1
}
%----------------------------------TITLE PAGE -----------------------------------
\title{Structural restrictions in local causal discovery: identifying direct causes of a target variable} 
\author{Juraj Bodik$^{1 \footnote{ Email of the corresponding author: Juraj.Bodik@unil.ch. \\Published in \textit{Biometrika}, 2025, \href{https://doi.org/10.1093/biomet/asaf042}{10.1093/biomet/asaf042}}}$, Valérie Chavez-Demoulin$^1$}
\date{%
    $^1$ {\small HEC, Université de Lausanne, Switzerland} \\%
    }

%-------------------------------- END TITLE PAGE ----------------------------------

\begin{document}

\pagenumbering{gobble}% Remove page numbers (and reset to 1)

\maketitle
\begin{abstract}
We consider the problem of learning a set of direct causes of a target variable from an observational joint distribution. Learning directed acyclic graphs (DAGs) that represent the causal structure is a fundamental problem in science. Several results are known when the full DAG is identifiable from the distribution, such as assuming a nonlinear Gaussian data-generating process. Here, we are only interested in identifying the direct causes of one target variable (local causal structure), not the full DAG. This allows us to relax the identifiability assumptions and develop possibly faster and more robust algorithms. In contrast to the Invariance Causal Prediction framework, we only assume that we observe one environment without any interventions. We discuss different assumptions for the data-generating process of the target variable under which the set of direct causes is identifiable from the distribution. While doing so, we put essentially no assumptions on the variables other than the target variable. In addition to the novel identifiability results, we provide two practical algorithms for estimating the direct causes from a finite random sample and demonstrate their effectiveness on several benchmark and real datasets. 
\end{abstract}
%TC:ignore
\keywords{Causal discovery, Identifiability, Target variable, Local causal discovery, Structural causal models, Causal inference}
%TC:endignore
  
%\newpage
%\listoffigures
%\newpage
%\listoftables
%\newpage
%\listofalgorithms % List of algorithms in pseudocode format
%\newpage
%\listoflistings % List of algorithms in code format
%\newpage


\pagenumbering{arabic}% Arabic page numbers (and reset to 1)

% This is how you can organize your document
\section{Introduction}
%TODO
%Note: eta and epsilon incoherence
%Change the plot with c=0.3, 0.6, 0.9

Causal reasoning holds great significance in numerous fields, including public policy, decision-making and medicine \citep{Holland, TheBookOfWhy}. Randomized control experiments are widely accepted as the gold standard method for determining causal relationships \citep{Rubin}. However, the feasibility of such experiments is often hindered by their high costs and ethical concerns. Therefore, it is important to estimate causal relations from observational data, which are obtained by observing a system without any interventions \citep{Elements_of_Causal_Inference}. 

In this paper, we deal with the problem of estimating a set of direct causes of a target variable from a random sample. Typically, research focus lies in estimating the full causal structure, while we are interested only in a local causal structure around one variable of interest. The main issue arises when several causal structures produce the same observed distribution; the set of direct causes can be unidentifiable. We can generally estimate only the Markov equivalence class (MEC). 

A variety of research papers have proposed a methodology to deal with unidentifiable structures. These methods are either structural-restriction-based, meaning we add some additional assumptions about the functional relations between the variables, such as assuming nonlinear Gaussian data-generation process \citep{hoyer2008, Peters2014, reviewANMMooij, Elements_of_Causal_Inference,immer2022identifiability, Nonlinear_Causal_Discovery_with_Confounders, bodik2023identifiability}; score-based, meaning we pick a causal structure with the best fit on the data according to some score function \citep{Greedy_search, Score-based_causal_learning}; or information-theory-based, using mutual information and approximations of Kolmogorov complexity \citep{IGCI, Slope, Natasa_Tagasovska}.  However, these methods were designed either for a bivariate case or to infer the entire causal structure of the system. 

Several methodologies have been proposed to infer a local structure around a target variable. These methods are typically divided into three categories: learning a local skeleton (unoriented graph), learning a minimal Markov blanket (a sufficient set), or learning a set of direct causes of the target variable (goal of this paper). Under causal sufficiency and faithfulness \citep{Pearl_book}, the PC algorithm \citep{PCalgorithm} can identify the MEC and consistently learn the skeleton of the full structure. \cite{Local_causal_discovery_2008}, \cite{Local_Causal_discovery2010} and \cite{Wang2014_local_DAG_learning} discuss modifications of the PC algorithm focusing only on the local structure. 
\cite{Local_Causal_discovery_Gao_2021} suggest a methodology for estimating the minimal Markov blanket based on comparing entropies of the variables.  
\cite{Local_Causal_discovery_Mona_Azadkia} propose a method to learn the direct causes of the target variable under the assumption that the causes are identifiable (specifically, assuming that the underlying structure is a polytree). In contrast, our work aims to distinguish between different local causal structures (local MEC) using an structural-restrictions-based approach, where we take the ideas from classical approaches and use them locally. 

In the following, we present the main ideas of the paper. The theory is based on a structural causal model (SCM; \citealp{Pearl}) where a target variable $Y$ is structurally generated as $Y = f_Y(\textbf{X}_{pa_Y}, \varepsilon_Y)$, where $\textbf{X} = (X_1, \dots, X_p)^\top$ are other variables in the system and $pa_Y\subseteq\{1, \dots, p\}$ are called parents (or direct causes) of $Y$, and $\varepsilon_Y\indep \textbf{X}_{pa_Y}$ is a noise variable. Our goal is to estimate the set $pa_Y$ from a random sample from $(Y, \textbf{X})$. Following the structural-restrictions-based ideology, we assume $f_Y\in\mathcal{F}$, where $\mathcal{F}$ is a subset of all measurable functions (for example, all linear functions). 

Throughout the paper, we restrict the class of functions $\mathcal{F}$ in the following way. We assume that $f_Y$ is invertible (notation $f_Y\in\mathcal{I}$) in the sense that $$\text{a function } f_Y^\leftarrow \text{ exists such that }\varepsilon_Y = f_Y^\leftarrow(\textbf{X}_{pa_Y},Y).$$ In other words, the noise variables can be recovered from the observed variables.  Moreover, we assume that $f_Y\in\mathcal{I}_m$, where 
$$
\mathcal{I}_{m} = \{f\in\mathcal{I} : \,\,f\text{ is not constant in any of its arguments} \}.
$$
This assumption is closely related to causal minimality (the subscript $m$ in $\mathcal{I}_{m}$ represents the word ``minimality''). For more details and a rigorous definition of the class  $\mathcal{I}_m$, see Appendix \ref{Appendix_A.1.}. Overall, we assume that $\mathcal{F}\subseteq\mathcal{I}_m$. 

Our framework is built on a notion of $\mathcal{F}$-identifiability.  Without loss of generality, we assume $\varepsilon_Y\sim U(0,1)$\footnote{To understand why this assumption is without loss of generality, consider the equality $f_Y(\textbf{X}_{pa_Y}, \varepsilon) =f_Y(\textbf{X}_{pa_Y}, q^{-1}(\varepsilon_Y))$, where $\varepsilon_Y\sim U(0,1)$ and $q$ is a distribution function of $\varepsilon$. We define $ \tilde{f}_Y(\textbf{X}_{pa_Y}, \varepsilon_Y)=f_Y(\textbf{X}_{pa_Y}, q^{-1}(\varepsilon_Y))$ and only work with $\tilde{f}_Y$. }. 
\begin{definition}\label{Definition1}
A non-empty set $S\subseteq\{1, \dots, p\}$ is called an $\mathcal{F}$-\textbf{plausible} set of parents of $Y$, if 
\begin{equation}\label{Definition_F_plausible}
f\in\mathcal{F} \text{ exists such that for } \varepsilon_S := f^{\leftarrow}(\textbf{X}_{S}, Y)\text{, it holds that }\varepsilon_S\indep \textbf{X}_S, \,\,\,  \varepsilon_S\sim U(0,1).
\end{equation}
We define a set of $\mathcal{F}$-\textbf{identifiable} parents of $Y$ as follows:
\begin{equation*}\label{definition_F_identifiable}
S_\mathcal{F}(Y):= \bigcap_{S\subseteq \{1, \dots, p\}}\{S:   \text{ S is } \mathcal{F} \text{-plausible set of parents of Y}\}.
\end{equation*}
    
\end{definition}

The functional space $\mathcal{F}$ corresponds to the assumptions that we are willing to make about the data-generation process of $Y$. 
If we assume linearity of the covariates, this represents the assumption $f_Y\in\mathcal{F}_L$, where (recall that without loss of generality, we assume $\varepsilon_Y\sim U(0,1)$)
\begin{equation}\label{mathcalF_L}
\mathcal{F}_L = \{f\in\mathcal{I}_m: f(\textbf{x}, \varepsilon) = \beta^T\textbf{x} + q^{-1}(\varepsilon) \text{ for some quantile function }q^{-1} \text{ and } \beta\in\mathbb{R}^{|\textbf{x}|}\}.
\end{equation}
Note that the restriction  $f\in\mathcal{I}_m$ in (\ref{mathcalF_L}) implies that the arguments $\beta_i\neq 0$. On the other hand, if we assume a Conditionally Parametric Causal Model  ($CPCM(F)$, see (\ref{CPCM_def}) in Section~\ref{section_notation}), this corresponds to assuming $f_Y\in\mathcal{F}_F$ where 
\begin{equation}\label{mathcalF_F}
\mathcal{F}_F := \{f\in\mathcal{I}_m: f(\textbf{x},\varepsilon) =  F^{-1}(\varepsilon;\theta(\textbf{x})) \text{ for some function } \theta\}.
\end{equation}
Table~\ref{tableDefinitions} describes all functional spaces considered in this paper.  

The concept of  $\mathcal{F}$-identifiability provides theoretical limitations for the causal estimates under assumption $f_Y\in\mathcal{F}$. If $S_\mathcal{F}(Y) $ contains one element, we can only identify one cause of $Y$, even with infinite number of observations. The main part of the paper consists of inferring which elements belong to $S_\mathcal{F}(Y) $: When does it hold that  $S_\mathcal{F}(Y) = pa_Y$? 

From a practical point of view, we propose two algorithms for estimating the direct causes of a target variable from a random sample. One provides an estimate of $S_\mathcal{F}(Y)$, that is, it tests the $\mathcal{F}$-plausibility of several sets and outputs their intersection. This shields us against the mistake of including a non-parent in the output. However, the output does not have to contain all direct causes. The second is a score-based algorithm estimating $pa_Y$ based on a goodness-of-fit; even if several sets are $\mathcal{F}$-plausible, the output is the set with the best score. The first algorithm has strong theoretical guarantees for containing only the direct causes of $Y$. In contrast, the score-based algorithm outputs the ``best looking'' set of direct causes without theoretical guarantees for the output. 


We primarily focus on the case when $\textbf{X}=(X_1, \dots, X_p)$ are neighbors (either direct causes or direct effects) of $Y$ in the corresponding SCM. Using the classical conditional independence approach and d-separation (see Section~\ref{section_notation}), we can eliminate other variables from being potential parents of $Y$. Nevertheless, the theory can be extended to non-neighbors as well. 


In the following example, we provide an initial assessment of the findings presented in Section~\ref{Section_3}.
\begin{example}[Teaser example with Gaussian assumptions]\label{Teaser example with Gaussian assumptions}
Consider an SCM defined as follows (see Figure \ref{Figure_DAG_for_teaser}): let $(X_1, X_2, X_3)$ be normally distributed, and $X_5$ be non-degenerate. Let
$$Y = \mu_Y(X_1, X_2, X_3) + \sigma_Y(X_1, X_2, X_3)\varepsilon_Y,\,\,\,  \varepsilon_Y\text{ is Gaussian,}\,\,\,\,\,\,\,\,\varepsilon_Y\indep (X_1, X_2, X_3),$$
$$X_4 = \mu_4(X_2, Y, X_5) + \sigma_4(X_2, Y, X_5)\varepsilon_4,\,\,\,  \varepsilon_4\text{ is Gaussian,}\,\,\,\,\,\,\,\,\,\,\,\,\varepsilon_4\indep (X_1, X_2, X_3, Y),$$
where $(\mu_4, \sigma_4)^\top$ and  $(\mu_Y, \sigma_Y)^\top$ are real functions that satisfy some (weak) assumptions presented in Section~\ref{Section_3}. In particular,  $(\mu_4, \sigma_4)^\top$ are not in the form (\ref{norm}) and $\mu_Y$ is not additive (or $\sigma_Y$ is not multiplicative) in its arguments. 

Then, $S_{\mathcal{F}_F}(Y) = S_{\mathcal{F}_{LS}}(Y) = pa_Y = \{1,2,3\}$, where $F$ is a Gaussian distribution function. This result follows from the theory presented in  Section~\ref{Section_3}; in particular, it is a consequence of Lemma \ref{GaussianSeparabilita} in combination with Lemma \ref{lemma o inseparability=unplausibility}, Consequence~\ref{PropositionOAdditiveParents} and results in Example \ref{Gaussian case} combined with Proposition~\ref{TheoremFidentifiabilityWithChild}.
\end{example}


This paper is structured as follows. Section~\ref{section_notation} provides a standard notation and classical preliminary results from a theory of graphical models.  We also briefly summarize causal discovery methods from the literature that are based on structural restrictions. In Section~\ref{Section_3}, we dive deeper into mathematical properties of $S_\mathcal{F}(Y)$, where the aim is to find conditions under which $S_\mathcal{F}(Y) = pa_Y$. In Section~\ref{section_algorithm}, we describe our proposed algorithms for estimating $S_\mathcal{F}(Y)$ and $pa_Y$ from a random sample. Section~\ref{section_simulations} contains a short simulation study followed by an application on a real dataset. The paper has four appendices: Appendix~\ref{Speci_appendix} contains some detailed notions omitted from the main text for clarity,  Appendix~\ref{Appendix_Auxiliary} provides some auxiliary results needed for the proofs (in particular,  Lemma~\ref{CoolLemma} is the core mathematical result of the paper), the proofs can be found in Appendix~\ref{Section_proofs}, and Appendix~\ref{Appendix_Simulations}  contains some details about the simulations and the application. 


% Figure environment removed


















\begin{table}[]
\begin{tabular}{|l|}
\hline
\multicolumn{1}{|c|}{Summary of different $\mathcal{F}\subset \mathcal{I}_m$ used in the paper}                                                                                                                    \\ \hline
$\mathcal{F}_L = \{f\in\mathcal{I}_m: f(\textbf{x}, \varepsilon) = \beta^T\textbf{x} + q^{-1}(\varepsilon) \text{ for some quantile function }q^{-1} \text{ and }\beta\neq 0\}$                                \\ \hline
$\mathcal{F}_A = \{f\in\mathcal{I}_m: f(\textbf{x}, \varepsilon) = \mu(\textbf{x}) + g^{-1}(\varepsilon) \text{ for some }\mu(\cdot)\text{ and quantile function }q^{-1}     \}$        \\ \hline
$\mathcal{F}_{LS} = \{f\in\mathcal{I}_m: f(\textbf{x}, \varepsilon) = \mu(\textbf{x}) + \sigma(\textbf{x}) q^{-1}(\varepsilon) $                                                     \\
\,\,\,\,\,\,\,\,\,\,\,\,\,\,\,\,\,\,\,\,\,$\text{ for some functions  }\mu, \sigma>0\,\,\,\,\,\text{ and for some quantile function }q^{-1}\}$ \\ \hline
$\mathcal{F}_F := \{f\in\mathcal{I}_m: f(\textbf{x},\varepsilon) =  F^{-1}(\varepsilon;\theta(\textbf{x})) \text{ for some function } \theta:\mathbb{R}^{|\textbf{x}|}\to\mathbb{R}^q\}$         \\ \hline
\end{tabular}
\caption{The table summarizes different functional spaces $\mathcal{F}$ used in the paper. $\mathcal{F}_L$, $\mathcal{F}_A$, $\mathcal{F}_{LS}$, and $\mathcal{F}_F$ correspond to the linearity assumption, additivity assumption, location-scale assumption, and $CPCM(F)$ assumption, respectively.   }
\label{tableDefinitions}
\end{table}











\section{Properties of $\mathcal{F}$-identifiable parents}
\label{Section_3}

Recall that we assume the data-generation process of $Y$ in the form 
\begin{equation}\label{SCM_for_Y}
Y=f_Y(\textbf{X}_{pa_Y}, \varepsilon_Y),\,\,\,\,\,\,\, f_Y\in\mathcal{F}, \,\,\,\,\,\,\,\varepsilon_Y\indep \textbf{X}_{pa_Y}, \,\,\,\,\,\,\,\varepsilon_Y\sim U(0,1). \tag{\ding{170}} 
\end{equation}
The principle of independence of the cause and the mechanism directly implies that the set $S=pa_Y$ is always  $\mathcal{F}$-plausible; under (\ref{SCM_for_Y}), it always holds that
\begin{equation}\label{subseteq_parents}
S_\mathcal{F}(Y) \subseteq pa_Y.
\end{equation}
However, the equality $S_\mathcal{F}(Y)= pa_Y$ does not need to hold. Observe that 
$$\text{if }\mathcal{F}_1\subseteq \mathcal{F}_2, \text{ then }S_{\mathcal{F}_1}(Y)\supseteq S_{\mathcal{F}_2}(Y).$$
This is not surprising, as the more restrictions we put on the data-generation process, the larger the set of identifiable parents. 

In this section, we discuss which elements belong to $S_{\mathcal{F}}(Y)$. We find out that under linearity, we typically get $S_{\mathcal{F}_L}(Y)=\emptyset$, that is, if the link function $f_Y$ is linear, we cannot identify any parents of $Y$. However, if the link function $f_Y$ is ``ugly,'' it typically holds that $S_{\mathcal{F}}(Y) = pa_Y$ unless $\mathcal{F}$ contains too many functions.

In Section \ref{Section3.1}, our focus is on the case $\mathcal{F} = \mathcal{F}_L$. In Section \ref{Subsection3.2}, we focus mostly on the case where $\mathcal{F} = \mathcal{F}_F$ for different distribution functions $F$. Consequence~\ref{PropositionOAdditiveParents}  and Proposition \ref{Support_proposition} in Section \ref{Section3_children_case} discuss the modifications of these results for  $\mathcal{F}=\mathcal{F}_{A}$ and $\mathcal{F}=\mathcal{F}_{LS}$. 

Section~\ref{subsection3.3} includes several examples illustrating the results presented in Sections \ref{Section3.1} and \ref{Subsection3.2}.


\subsection{Case \texorpdfstring{$S_{\mathcal{F}}(Y) = \emptyset$}{Empty set of identifiable parents} and linear models }
\label{Section3.1}
The following remark shows why the concept of  $\mathcal{F}$-identifiable parents is usually not very interesting for linear SCM.   Recall that we interchangeably use the notation $X_0=Y$ for the target variable, and that an SCM follows an $\mathcal{F}$-model if each structural equation in the SCM satisfies $f_i\in\mathcal{F}, i=0, \dots, p$.  Following this definition, note that a statement ``$(Y, \textbf{X})$ follow a linear SCM'' is equivalent to a statement ``$(Y, \textbf{X})$ follow an $\mathcal{F}_L$-model.'' 

\begin{remark}\label{ExampleEasyDAG}
Consider $\mathcal{F}=\mathcal{F}_L$, where $(Y, \textbf{X})$ follow an $\mathcal{F}_L$-model with DAG drawn in Figure \ref{Mixed_Graph}A, and $$Y = \beta_1X_1+\beta_2X_2 + q^{-1}(\varepsilon_Y), \,\,\,\,\,\,\,\,\,\,\,\,\,\,\,\,\,\varepsilon_Y\indep (X_1, X_2), \,\,\varepsilon_Y\sim U(0,1),$$ for some $\beta_1\neq 0, \beta_2\neq 0$ and a quantile function $q^{-1}$. Then $S_{\mathcal{F}_L}(Y) = \emptyset$. 
\end{remark}
\begin{proof}
We show that the set $S=\{1\}$  is $\mathcal{F}_L$-plausible. Intuitively, this holds since $Y - \beta_1 X_1 \indep X_1$, and we do not put any restrictions on the noise variable. More precisely, we find $f\in\mathcal{F}_L$ such that (\ref{Definition_F_plausible}) holds. Such $f$ can be defined as $f(x,\varepsilon) = \beta_1x + \tilde{q}^{-1}(\varepsilon)$ for $x\in\mathbb{R}, \varepsilon\in (0,1)$, where $\tilde{q}$ is the distribution function of $[\beta_2X_2 + q^{-1}(\varepsilon_Y)]$. Then trivially, $f\in\mathcal{F}_L$ and its inverse satisfies $f^\leftarrow(x,y) = \tilde{q}(y-\beta_1x)$ for $x,y\in\mathbb{R}$. Hence, $\varepsilon_{S} = f^{\leftarrow}(X_1, Y) = \tilde{q}(Y - \beta_1X_1) = \tilde{q}[\beta_2X_2 + q^{-1}(\varepsilon_Y)]\indep X_1$  and $\varepsilon_{S}\sim U(0,1)$. We have shown that $f$ satisfies (\ref{Definition_F_plausible}). 

With a similar reasoning, we get that  $S=\{2\}$ is also $\mathcal{F}_L$-plausible. Therefore, $S_{\mathcal{F}_L}(Y)\subseteq \{1\}\cap \{2\}=\emptyset$. 
\end{proof}
A similar argument can be used in a more general case. The following lemma demonstrates that in the linear structural causal models, $S_{\mathcal{F}_L}(Y)$ is usually empty, even if $pa_Y(\mathcal{G}_0)\neq \emptyset$. 

\begin{lemma}\label{LemmaAboutUnidentifiabilityFL}
Let $(Y, \textbf{X})\in\mathbb{R}\times \mathbb{R}^p$ follow an $\mathcal{F}_L$-model with DAG $\mathcal{G}_0$ and $pa_Y(\mathcal{G}_0)\neq\emptyset$. Then, $|S_{\mathcal{F}_L}(Y)| \leq 1$. Moreover, if there are any $a,b\in an_Y(\mathcal{G}_0)$ that are d-separated in $\mathcal{G}_0$, then $S_{\mathcal{F}_L}(Y) = \emptyset$. 
\end{lemma}

The proof is in \hyperref[Proof of LemmaAboutUnidentifiabilityFL]{Appendix} \ref{Proof of LemmaAboutUnidentifiabilityFL}. Lemma~\ref{LemmaAboutUnidentifiabilityFL} shows a more general principle that goes beyond the linear models. If we can \textit{marginalize} a causal model to a smaller submodel without breaking $f_Y\in\mathcal{F}$, then only the submodel is relevant for inference about  $S_{\mathcal{F}}(Y)$. We provide a more rigorous explanation of this.

\begin{definition}
Let $\mathcal{F}\subseteq\mathcal{I}_m$ and let $(X_0, \textbf{X})\in\mathbb{R}\times \mathbb{R}^p$ follow an $\mathcal{F}$-model with DAG $\mathcal{G}_0$. For a non-empty set $S\subseteq \{0, \dots, p\}$, we say that the $\mathcal{F}$-model is \textbf{marginalizable} to $S$ if  $\textbf{X}_{S}$ can also be written as an $\mathcal{F}$-model with some underlying DAG $\mathcal{G}_S$. 
%We say that an $\mathcal{F}$-model is \textbf{unmarginalizable} wrt $X_0$, if for all non-empty sets $ S\subseteq\{1, \dots, p\}$ holds that $\textbf{X}_{(0,S)}$ do not follow an $\mathcal{F}$-model. 
\end{definition}

We provide an illustration of the marginalizability on a specific example. 

\begin{remark}\label{example158}
Consider the following $\mathcal{F}_L$-model with DAG drawn in Figure \ref{Mixed_Graph}B: $X_1=\eta_1, X_2 = X_1 + \eta_2,  X_0 = X_1+X_2+\eta_0$, where the noise variables ($\eta_1, \eta_2, \eta_0$ are not necessary uniformly distributed) are jointly independent. 
\begin{itemize}
\item If $\eta_1, \eta_2, \eta_0\overset{iid}{\sim} N(0,\sigma^2)$, then the $\mathcal{F}_L$-model is marginalizable to $S = \{0,1\}$ and to $S=\{0,2\}$.
\item If $\eta_1, \eta_2, \eta_0$ are \textit{not} Gaussian, then the $\mathcal{F}_L$-model is marginalizable to $S = \{0,1\}$, but not to $S=\{0,2\}$.
\end{itemize}
\end{remark}
\begin{proof}
The marginalizability with respect to $S = \{0,1\}$ follows from the fact that we can rewrite $X_1=\eta_1, X_0=2X_1 + \tilde{\eta}_0$, where $\tilde{\eta}_Y = \eta_2 + \eta_0\indep \eta_1$. For  $S=\{0,2\}$ and Gaussian noise, we can rewrite $X_2=\tilde{\eta}_2, X_0= \frac{3}{2}X_2 +\tilde{\eta}_0$, where $\tilde{\eta}_2 = \eta_1 + \eta_2$ and $ \tilde{\eta}_0 =\frac{1}{2}\eta_1 -\frac{1}{2}\eta_2 + \eta_0\indep \tilde{\eta}_2$. 

For $S=\{0,2\}$ and non-Gaussian noise, we can never find $X_2=\tilde{\eta}_2, X_0= \beta X_2 +\tilde{\eta}_0$, where $ \tilde{\eta}_2 \indep \tilde{\eta}_0$ and $\beta\neq 0$, since for non-Gaussian variables, $a\eta_1 + b\eta_2 \not\indep c\eta_1 + d\eta_2$ always holds, for $a,b,c,d\in\mathbb{R}\setminus\{0\}$ (Darmois-Skitovič theorem \citep{Skitovic}). 
\end{proof}


The following lemma states that marginalizable models typically have a small number of identifiable parents. 

\begin{lemma}\label{lemma158}
 Let $\mathcal{F}\subseteq\mathcal{I}_m$. Let $(X_0, \textbf{X})\in\mathbb{R}\times \mathbb{R}^p$ follow an $\mathcal{F}$-model with DAG $\mathcal{G}_0$ and $pa_{X_0}(\mathcal{G}_0)\neq \emptyset$. Let  $S\subseteq \{1, \dots, p\}$ be a non-empty set. If $(X_0, \textbf{X})$ is marginalizable to $S\cup\{0\}$, then $S_{\mathcal{F}}(X_0)\subseteq S$. 
\end{lemma}
\begin{proof}
 \label{Proof of lemma158}
 Since  $(X_0, \textbf{X})$ is marginalizable to $S\cup\{0\}$,  $(X_0, \textbf{X}_S)$ follows an $\mathcal{F}$-model. Therefore, $f_0\in\mathcal{F}$ exists, such that $X_0 = f_0(X_{\tilde{S}}, \varepsilon_0)$ for some $\tilde{S}\subseteq S$, $\varepsilon_0\indep X_{\tilde{S}}$, $\varepsilon_0\sim U(0,1)$. 
In other words, $ f_0^{\leftarrow}(X_{\tilde{S}}, X_0)\indep X_{\tilde{S}}$,  $f_0^{\leftarrow}(X_{\tilde{S}}, X_0)\sim U(0,1)$, which is exactly the definition of $\mathcal{F}$-plausibility. Hence, $\tilde{S}$ is $\mathcal{F}$-plausible and consequently,  $S_{\mathcal{F}}(X_0)\subseteq \tilde{S}\subseteq S$. 
 \end{proof}
Continuing with Example \ref{example158}, Lemma \ref{lemma158} gives us $S_{\mathcal{F}_L}(X_0)=\emptyset$ in the Gaussian case, and  $S_{\mathcal{F}}(X_0)=\{1\}$ in the non-Gaussian case. Merely assuming linearity is insufficient to determine whether $X_2$ is a parent of $X_0$.


% Figure environment removed








\subsection{Deriving assumptions under which \texorpdfstring{$S_{\mathcal{F}}(Y) = pa_Y$}{Complete set of identifiable parents}}\label{Subsection3.2}

We provide conditions under which all sets $S\subseteq \{1, \dots, p\}, S\neq pa_Y$ are not $\mathcal{F}$-plausible.
We focus on two main cases: when $S\cap ch_Y\neq\emptyset$ and $S\subset pa_Y$. We start with the case when  $S\cap ch_Y\neq\emptyset$.

\subsubsection{ \texorpdfstring{Case $S\cap ch_Y\neq\emptyset$}{Focusing on descendants}}
\label{Section3_children_case}
In the following, we show that we can use classical identifiability results from the literature for assessing $\mathcal{F}$-plausibility of a set $S$. Informally, if all variables in the SCM follow an identifiable $\mathcal{F}$-model, then any $S$ containing a child of $Y$ cannot be $\mathcal{F}$-plausible. We use a notion of pairwise identifiability of an $\mathcal{F}$-model, defined in Appendix \ref{Appendix_pairwise_identifiability}. Pairwise identifiability describes the identifiability of the causal relation between each pair of random variables, conditioned on any other variables. 

\begin{proposition}
\label{TheoremFidentifiabilityWithChild}
Let $(X_0, \textbf{X})\in\mathbb{R}\times \mathbb{R}^p$ follow an SCM with DAG $\mathcal{G}_0$. Let $S\subseteq\{1, \dots, p\}$ and denote $S_0 = S\cup \{0\}$. Assume that  $\mathcal{G}=\mathcal{G}_0[S_0]$, the projection of $\mathcal{G}_0$ on $S$ defined in Section \ref{section_notation}, is a DAG. Let $S$ contain a childless child of $X_0$, that is, $\exists j\in  ch_0(\mathcal{G})$ such that $ch_{j}(\mathcal{G})=\emptyset$. 

Let $\mathcal{F}\subseteq \mathcal{I}_{m}$ and let $(X_0, \textbf{X}_S)$ follow an  $\mathcal{F}$-model with graph $\mathcal{G}$, that is pairwise identifiable. Then, $S$ is not $\mathcal{F}$-plausible. 
\end{proposition}
The proof is in \hyperref[Proof of TheoremFidentifiabilityWithChild]{Appendix} \ref{Proof of TheoremFidentifiabilityWithChild}. To provide an example of the usage of Proposition~\ref{TheoremFidentifiabilityWithChild}, consider $\mathcal{F} = \mathcal{F}_A$ and $X_1\to Y\to X_2$. Let $X_2 = f_2(Y) + \eta_2$, where $Y\indep\eta_2$ and $\eta_2$ has the Gaussian distribution and $f_2$ is non-linear. Combining Proposition~\ref{TheoremFidentifiabilityWithChild} with Lemma 6 in \cite{Zhang2009}, we find that $S=\{2\}$ is not $\mathcal{F}_A$-plausible. Since $S=\{1\} = pa_Y$ is $\mathcal{F}_A$-plausible as long as $f_Y\in\mathcal{F}_A$, we get $S_{\mathcal{F}_A}(Y)=pa_Y=\{1\}$.

The following proposition discusses a different case when $\mathcal{F}$-implausibility results from restricting support of $Y$ by conditioning on the child of $Y$. This result is specific for a location-scale space of functions $\mathcal{F}_{LS}$, but can be modified for other types of $\mathcal{F}$. 

\begin{proposition}[Assuming bounded support]
\label{Support_proposition}
Let $(Y, \textbf{X})\in\mathbb{R}\times \mathbb{R}^p$ follow an SCM with DAG $\mathcal{G}_0$. Let $S\subseteq\{1, \dots, p\}$ be a non-empty set. Let  $\underline{\Psi},\overline{\Psi}: \mathbb{R}^{\mid S\mid}\to \mathbb{R}$ be real functions such that
\begin{equation*} 
supp(Y\mid \textbf{X}_S=\textbf{x}) = \big(\underline{\Psi} (\textbf{x}),\overline{\Psi}(\textbf{x})\big), \,\,\,\,\,\,\, \forall \textbf{x}\in supp(\textbf{X}_S).
\end{equation*}
Moreover, let
\begin{equation} \label{eq9987}
\frac{Y - \underline{\Psi}(\textbf{X}_S)}{\overline{\Psi}(\textbf{X}_S) - \underline{\Psi}(\textbf{X}_S)}\not\indep \textbf{X}_S.
\end{equation}
Then, $S$ is not $\mathcal{F}_{LS}$-plausible. 
\end{proposition}
The proof can be found in \hyperref[Proof of Support_proposition]{Appendix} \ref{Proof of Support_proposition}. Proposition \ref{Support_proposition} can be expressed as follows. If the support of $Y$ given $\textbf{X}_S = \textbf{x}_S$ is bounded, then $S$ can be  $\mathcal{F}_{LS}$-plausible only in a very specific case when (\ref{eq9987}) does not hold. 
Typically, (\ref{eq9987}) holds if $S$ contains a child of $Y$. 
\begin{example}\label{Example_o_Supporte}
Consider SCM where $Y$ is a parent of $X_1$ and $X_1 = Y + \eta$, where $Y\indep \eta$. Assume that  $Y, \eta$ are non-negative ($supp(Y) = supp(\eta) = (0, \infty)$). 
Then, $\underline{\Psi}(x)=0$ and  $\overline{\Psi}({x})=x$, since the support of $[Y\mid Y+\eta=x]$ is $(0,x)$. Hence, (\ref{eq9987}) reduces to $\frac{Y}{X_1} \not\indep X_1$. If $\frac{Y}{X_1} \not\indep X_1$, then $S=\{1\}$ is not $\mathcal{F}_{LS}$-plausible. 

How strong is the assumption $\frac{Y}{X_1} \not\indep X_1$? We claim that it holds in typical situations. A notable exception when  $\frac{Y}{X_1} \indep X_1$ holds is when $Y, \eta$ have Gamma distributions with equal scales. 
\end{example}

Proposition \ref{Support_proposition} is applicable only when $S$ contains a child of $Y$. If $S\subseteq pa_Y$, then (\ref{eq9987}) typically does not hold, as the following example illustrates.  
\begin{example}
Consider a bivariate SCM with $X_1\to Y$. Let $Y = X_1 + \eta$, where $X_1\indep \eta$. Assume that $supp(X) = supp(\eta) = (0, 1)$. Then, $\underline{\Psi}(x)=x$ and  $\overline{\Psi}({x})=1+x$. Hence, (\ref{eq9987}) reduces to $Y-X_1 \not\indep X_1$, which is not satisfied, so Proposition \ref{Support_proposition} is not applicable. 
\end{example}
Proposition \ref{Support_proposition} can be also stated for a case when  $\overline{\Psi}({x})=\infty$. In that case, we require stronger assumptions; we replace assumption (\ref{eq9987}) with  $Y - \underline{\Psi}(\textbf{X}_S)\not\indep \textbf{X}_S$ and replace $\mathcal{F}_{LS}$ with $\mathcal{F}_A$ (more restricted set where the scale is fixed). 

\subsubsection{Case \texorpdfstring{$S\subset pa_Y$}{Focusing on parents}}
In the following, we discuss the case when $S\subset pa_Y$. First, we introduce the notion of the inseparability of a real function. This is a fundamental notion since we will show that if $f_Y$ is inseparable, then every $S\subset pa_Y$ is not $\mathcal{F}$-plausible. Later, we provide a characterization of inseparable functions, which leaves us with a powerful tool for inferring  $\mathcal{F}$-plausible sets. For the characterization of inseparable functions, we mainly restrict our attention to sets $\mathcal{F} = \mathcal{F}_F$ for some distribution function $F$. 

\begin{definition}
Let $\textbf{X}=(X_1, \dots, X_k)$ be a random vector and $\mathcal{F}\subseteq\mathcal{I}_m$. A function $f\in\mathcal{I}_m {:}\,\, \mathbb{R}^{k+1}\to\mathbb{R}$ is called $\mathcal{F}-$\textbf{inseparable} \textbf{wrt} $\textbf{X}$, if for all $S\subset\{1, \dots, k\}$, $z\in \mathbb{R}$ exists such that 
\begin{equation}\label{Definitioninseparability}
\text{for all } g\in\mathcal{F}\,\, \text{ holds } \,\,g^{\leftarrow}\big(\textbf{X}_S, f(\textbf{X}, z)\big)\not\indep \textbf{X}_S.
\end{equation}
 \end{definition}
The notion of inseparability of a function means that we are not able to ``erase'' the effect of $\textbf{X}_S$ on $Y=f(\textbf{X}, z)$ without considering other variables. Note that the notion of $\mathcal{F}$-inseparability is a property of a real function; it does not depend on causal relations (only on the distribution of $\textbf{X}$). We provide an illustration of the notion of inseparability on the following example. 

\begin{remark}
Let $k=2$ and $\textbf{X}= (X_1, X_2)$ be continuous with independent components. Consider the function $f(x_1, x_2, z) = x_1x_2z$ and $\mathcal{F}_1, \mathcal{F}_2\subseteq\mathcal{I}_m$, such that $\mathcal{F}_1$ contains only linear functions and $\mathcal{F}_2$ contains only multiplicative functions, that is, $\mathcal{F}_1 = \mathcal{F}_L$ and 
\begin{align*}
    \mathcal{F}_2 = \{f\in\mathcal{I}_m&: f(\textbf{x}, \varepsilon) = g_1(x_1)\ldots g_k(x_k)q^{-1}(\varepsilon) \\&\text{ for some measurable functions }g_1, \dots, g_k \text{ and a quantile function 
 } q^{-1}\}.
\end{align*}
Then, $f$ is $\mathcal{F}_1-$inseparable wrt $\textbf{X}$ but is not $\mathcal{F}_2-$inseparable wrt $\textbf{X}$.
\end{remark}
\begin{proof}
 Intuitively, $f$ is $\mathcal{F}_1-$inseparable wrt $\textbf{X}$ because we cannot find $\beta\in\mathbb{R}$ such that $X_1X_2\varepsilon -\beta X_1 \indep X_1$. However, $f$ is not  $\mathcal{F}_2-$inseparable wrt $\textbf{X}$, since we can find $g$ such that $X_1X_2\varepsilon  g(X_1)\indep X_1$. We provide a more rigorous explanation of this. 
 
 First, we show that $f$ is not $\mathcal{F}_2-$inseparable wrt $\textbf{X}$. Take $S=\{1\}$. Choose a function $g\in\mathcal{F}_2$ such that  $g(x, z) = xz$. Its inverse has a form $g^{\leftarrow}(x, xz) = z$. Hence, $g^{\leftarrow}(X_S, f(\textbf{X}, z))=g^{\leftarrow}(X_1, X_1X_2z) = X_2z\indep X_1$. Hence, $f$ is not $\mathcal{F}_2-$inseparable wrt $\textbf{X}$, since we found a $g\in\mathcal{F}_2$ such that (\ref{Definitioninseparability}) is violated. 

Second, we  show that $f$ is $\mathcal{F}_1-$inseparable wrt $\textbf{X}$. Consider $S=\{1\}$ and consider any function $g\in\mathcal{F}_1$. Let us write $g$ in a form $g(x,z) = \beta x + {q}^{-1}(z)$ for some quantile function ${q}^{-1}$ and $\beta\neq 0$. The inverse of $g$ satisfies $g^{\leftarrow}(x, xz) ={q}(xz - \beta x)$. In order to show (\ref{Definitioninseparability}), we need to show that $q(X_1X_2z - \beta X_1)\not\indep X_1$. In the remainder of the section (and Appendix~\ref{Appendix_Auxiliary}), we develop a framework for proving statements such as this one. In particular, $X_1X_2z - \beta X_1\not\indep X_1$ holds due to Lemma \ref{CoolLemma} part 2. Hence, for $S=\{1\}$, (\ref{Definitioninseparability}) is satisfied. For $S=\{2\}$, the proof follows analogously. 
\end{proof}


The following lemma shows the connection between the  $\mathcal{F}$-inseparability and  $\mathcal{F}$-plausibility. It shows that the $\mathcal{F}$-inseparability implies the $\mathcal{F}$-inplausibility of all subsets of the parents. 


\begin{lemma}\label{lemma o inseparability=unplausibility}
Let $(Y, \textbf{X})\in\mathbb{R}\times \mathbb{R}^p$ follow an SCM with DAG $\mathcal{G}_0$ and $pa_Y(\mathcal{G}_0)\neq\emptyset$. Let $f_Y\in\mathcal{F}\subseteq\mathcal{I}_m$. If $f_Y$ is  $\mathcal{F}-$inseparable wrt $\textbf{X}_{pa_Y}$, then every $S\subset pa_Y(\mathcal{G}_0)$ is \textit{not} an $\mathcal{F}$-plausible set of parents of $Y$. 
\end{lemma}
\begin{proof}\label{proof of lemma o inseparability=unplausibility}
%"$\implies$": 
For a contradiction, let $S\subset pa_Y(\mathcal{G}_0)$ be an $\mathcal{F}$-plausible set of parents of $Y$. Then, we can find $f\in\mathcal{F} \text{ such that } f^{\leftarrow}(\textbf{X}_{S}, Y)\indep \textbf{X}_S.$ Since we have $Y = f_Y(\textbf{X}_{pa_Y}, \varepsilon_Y)$, we can rewrite $f^{\leftarrow}\big(\textbf{X}_{S}, f_Y(\textbf{X}_{pa_Y}, \varepsilon_Y)\big)\indep \textbf{X}_S.$ Since $\varepsilon_Y\indep \textbf{X}_{pa_Y}$, conditioning on $[\varepsilon_Y=z]$ for arbitrary $z\in(0,1)$ will give us $f^{\leftarrow}\big(\textbf{X}_S, f(\textbf{X}_{pa_Y}, z)\big)\indep \textbf{X}_S$, which is a contradiction with inseparability.
%"$\impliedby$" For a contradiction, let $S\subsetneq\{1, \dots, p\}$ and let there exist $g\in \mathcal{F}$ such that $g^{-1}(X_S, f_Y(\textbf{X}_{pa_Y}, z))\indep X_S$ for any $z\in (0,1)$. This implies (law of total probability) that for $\varepsilon_Y\indep X_{pa_Y}$ holds $g^{-1}(X_S, f_Y(\textbf{X}_{pa_Y}, \varepsilon_Y))\indep X_S$. Using $Y= f_Y(\textbf{X}_{pa_Y}, \varepsilon_Y)$ rewrite $g^{-1}(X_S, Y)\indep X_S$. Since $g\in \mathcal{F}\subset\mathcal{I}$, this is a contradiction with $\mathcal{F}$-plausibility. 
\end{proof}
In the following, we focus on characterizing the $\mathcal{F}_F$-inseparability for different distributions $F$. We show that some large classes of $f\in\mathcal{F}_F$ are indeed $\mathcal{F}_F$-inseparable. We will restrict our attention to specific types of $F$, where the parameters act \textbf{additively}/\textbf{multiplicatively}/\textbf{location-scale}. Rigorous definitions of these types can be found in Appendix \ref{Appendix_location_scale_definition}. The Gaussian distribution with fixed variance is an example of $F$ whose parameter acts additively. The Gaussian distribution with the fixed expectation or the Pareto distribution are examples of $F$ whose parameter acts multiplicatively. Examples of Location-Scale types of distributions include Gaussian distribution, logistic distribution, or Cauchy distribution, among many others.

First, we consider one parameter case ($q=1$ in (\ref{CPCM_def})). The following two propositions are the main results of this subsection. They characterize $\mathcal{F}_F-$inseparability (under some weak assumptions on the distribution function $F$). Combining these results with Lemma \ref{lemma o inseparability=unplausibility} gives us a powerful tool for inferring $\mathcal{F}_F-$plausible sets.  



\begin{proposition}\label{LemmaOParetoinseparabilite}
Let $F$ be a distribution function whose parameter acts post-multiplicatively. Let  $\textbf{X}=(X_1, \dots, X_k)$ be a continuous random vector with independent components.  
\begin{itemize}
\item Consider $f\in\mathcal{F}_F$ in the form $f(\textbf{x}, \varepsilon)=F^{-1}\big(\varepsilon, \theta(\textbf{x})\big)$ with additive function $\theta(x_1, \dots, x_k) = h_1(x_1)+\dots + h_k(x_k)$, where $h_i$ are continuous non-constant real functions. Then, $f$ is $\mathcal{F}_F-$inseparable wrt $\textbf{X}$. 
\item Consider $f\in\mathcal{F}_F$ in the form $f(\textbf{x}, \varepsilon)=F^{-1}\big(\varepsilon, \theta(\textbf{x})\big)$ with multiplicative function   $\theta(x_1, \dots, x_k) = h_1(\textbf{x}_S)\cdot h_2(\textbf{x}_{\{1, \dots, k\}\setminus S})$ for some $S\subsetneq \{1, \dots, k\}$, where $h_1, h_2$ are continuous non-constant non-zero real functions. Then, $f$ is not $\mathcal{F}_F-$inseparable wrt $\textbf{X}$. 
\end{itemize}
\end{proposition}
\begin{hproof}
Full proof is in \hyperref[Proof of LemmaOParetoinseparabilite]{Appendix} \ref{Proof of LemmaOParetoinseparabilite}. Here, we show the main steps of the first bullet-point in the case when $k=2$, $h_1(x)=h_2(x)=x$ and $F$ is the Pareto distribution function. Consider   $S=\{1\}$ (the case $S=\{2\}$ follows similarly). We show that for $S=\{1\}$ and any $z\in(0,1)$, there does not exist $g\in\mathcal{F}_F$ such that $g^{\leftarrow}\big(X_1, f(\textbf{X}, z)\big)\indep X_1.$ 

For a contradiction, assume that such $g$ exists and write $g^{\leftarrow}\big(x, \cdot) = F(\cdot, \theta_g(x)\big)$ for some non-constant positive function $\theta_g$. Using the form of the Pareto distribution, rewrite $X_1\indep g^{\leftarrow}\big(X_1, f(\textbf{X}, z)\big) = F\bigg(f(\textbf{X}, z),\theta_g(X_1)\bigg) =  F\bigg(F^{-1}[z, \theta(\textbf{X})],\theta_g(X_1)\bigg) = z^{-\frac{\theta(\textbf{X})}{\theta_g(X_1)}} = z^{-\frac{(X_1+X_2)}{\theta_g(X_1)}}$. Using the identity $Z_1\indep Z_2 \implies f(Z_1)\indep Z_2$ for any measurable function $f$ and random variables $Z_1, Z_2$, we get $$X_1\indep \theta_g(X_1)(X_1+X_2).$$ 
We show that this is not possible. Denote $\xi=\theta_g(X_1)(X_1+X_2)$ and choose \textit{distinct} $a,b,c$ in the support of $X_1$ such that $\theta_g(b)\neq 0$ (since $\theta_g$ is non-constant and $X_1$ is non-binary, this is possible). 

Since $\xi\indep X_1$, then $\xi\mid [X_1=a] \overset{D}{=}\xi\mid [X_1=b] \overset{D}{=}\xi\mid [X_1=c]$. Hence, 
\begin{equation}\label{asdf}
\theta_g(a)(a+X_2)\overset{D}{=}\theta_g(b)(b+X_2)\overset{D}{=}\theta_g(c)(c+X_2).
\end{equation}
By dividing by a non-zero constant $\theta_g(b)$ and subtracting $b$, we get
$$
\{\frac{\theta_g(a)}{\theta_g(b)}a-b\}+\frac{\theta_g(a)}{\theta_g(b)}X_2\overset{D}{=}X_2\overset{D}{=}\{\frac{\theta_g(c)}{\theta_g(b)}c-b\}+\frac{\theta_g(c)}{\theta_g(b)}X_2.
$$
It
holds that (see lemma \ref{distributionalequalitylemma}) if $z_1+z_2X_2\overset{D}{=}X_2$ for some constants $z_1, z_2$, then $z_2=\pm 1$. In our case, it leads to $\frac{\theta_g(a)}{\theta_g(b)}=\pm 1$ and  $\frac{\theta_g(c)}{\theta_g(b)}=\pm 1$. Therefore, at least two values of $\theta_g(a),\theta_g(b), \theta_g(c)$ have to be equal, and neither of them is zero. Without loss of generality, $\theta_g(a)= \theta_g(b)$. Plugging this into equation (\ref{asdf}), we get $a=b$, which is a contradiction since we chose them to be distinct. Therefore, the non-constant function $\theta_g$ does not exist, and $g$ also does not exist. 
\end{hproof}


\begin{proposition}\label{LemmaOAdditiveinseparabilite}
Let $F$ be a distribution function whose parameter acts post-additively. Let  $\textbf{X}=(X_1, \dots, X_k)$ be a continuous random vector with independent components.  
\begin{itemize}
\item Consider $f\in\mathcal{F}_F$ in the  $f(\textbf{x}, \varepsilon)=F^{-1}\big(\varepsilon, \theta(\textbf{x})\big)$ with an additive function $\theta(x_1, \dots, x_k) = h_1(\textbf{x}_S) + h_2(\textbf{x}_{\{1, \dots, k\}\setminus S})$ for some non-empty $S\subset \{1, \dots, k\}$, where $h_1, h_2$ are continuous non-constant non-zero real functions. Then, $f$ is not $\mathcal{F}_F-$inseparable wrt $\textbf{X}$. 
\item Consider $f\in\mathcal{F}_F$ in the form $f(\textbf{x}, \varepsilon)=F^{-1}\big(\varepsilon, \theta(\textbf{x})\big)$ with multiplicative function   $\theta(x_1, \dots, x_k) = h_1(x_1). h_2(x_2)\dots h_k(x_k)$, where $h_i$ are continuous non-constant non-zero real functions. Then, $f$ is $\mathcal{F}_F-$inseparable wrt $\textbf{X}$. 
\end{itemize}
\end{proposition}

 The proof follows similar steps as the proof of Proposition \ref{LemmaOParetoinseparabilite} and can be found in \hyperref[Proof of LemmaOAdditiveinseparabilite]{Appendix} \ref{Proof of LemmaOAdditiveinseparabilite}.  Proposition \ref{LemmaOParetoinseparabilite} and Proposition \ref{LemmaOAdditiveinseparabilite} reveal that to obtain  $\mathcal{F}_F$-inseparability, the form of the parameter cannot match how the parameter affects the distribution. If the parameter acts additively resp. multiplicatively, the parameter cannot have an additive resp. multiplicative form.  However,  $\mathcal{F}_F$-inseparability typically holds if the forms do not match. Note that the multiplicative assumption of $f_Y$ in the second bullet-point of Proposition~\ref{LemmaOAdditiveinseparabilite} is not necessary and is only technical for proving Lemma~\ref{CoolLemma}.  As long as $f_Y$ is not additive in any of its arguments, Lemma~\ref{CoolLemma} can be modified for much larger space of functions. 
 
 
Proposition \ref{LemmaOParetoinseparabilite} and Proposition \ref{LemmaOAdditiveinseparabilite} are formulated only for one-parameter cases (when $q=1$). The following proposition discusses the case when $q=2$, where we restrict ourselves to a location-scale family of distributions.


\begin{proposition}\label{LemmaOLocationScaleinseparabilite}
Let $F$ have a location-scale type with $q=2$ parameters. Let $\textbf{X}=(X_1, \dots, X_k)$ be a continuous random vector with independent components. Consider $f\in\mathcal{F}_F$ in the form $f(\textbf{x}, \varepsilon)=F^{-1}\big(\varepsilon, \theta(\textbf{x})\big)$, where $\theta(\textbf{x}) = \big(\mu(\textbf{x}), \sigma(\textbf{x})\big)^\top$ is additive in both components, that is,  
$\mu(\textbf{x}) = h_{1, \mu}(x_1)+\dots + h_{k, \mu}(x_k)$ and 
$\sigma(\textbf{x}) =h_{1, \sigma}(x_1)+\dots + h_{k, \sigma}(x_k)$ for some continuous non-constant non-zero functions $h_{i,\cdot}$, where we also assume $h_{i,\sigma}>0$, $i=1, \dots, k$.  Then, $f$ is $\mathcal{F}_F-$inseparable wrt $\textbf{X}$. 
\end{proposition}

The proof is in \hyperref[Proof of LemmaOLocationScaleinseparabilite]{Appendix}~\ref{Proof of LemmaOLocationScaleinseparabilite}. The crucial assumption lies in the additivity of $\sigma(\textbf{x})$; since this parameter acts multiplicatively in $F$ and has an additive form, a similar reasoning as in Proposition  \ref{LemmaOParetoinseparabilite} can be used (Proposition \ref{LemmaOLocationScaleinseparabilite} shows that even if another parameter $\mu$ depends on $\textbf{X}$, it does not ruin the results). Proposition \ref{LemmaOLocationScaleinseparabilite} can also be reformulated such that both parameters have a multiplicative form. 


The biggest drawback of Propositions \ref{LemmaOParetoinseparabilite}, \ref{LemmaOAdditiveinseparabilite}, and  \ref{LemmaOLocationScaleinseparabilite} lies in the fact that we assume independent components of $\textbf{X}$, which is rarely the case. This assumption is only a technical assumption that simplifies the proof and is not necessary. It allows us to explicitly express the conditional distributions $h(X_{i})\mid \textbf{X}_S$, that we used in the proof. However, an explicit form of a conditional distribution can also be found in other cases, such as when $\textbf{X}$ is Gaussian. 

\begin{lemma}\label{GaussianSeparabilita}
Let the assumptions from Proposition \ref{LemmaOLocationScaleinseparabilite} hold, and let $\textbf{X}=(X_1, \dots, X_k)$ be a non-degenerate Gaussian random vector (possibly with dependent components) and $h_{i,\sigma}$ be linear functions. Then, $f$ is $\mathcal{F}_F-$inseparable wrt $\textbf{X}$. 
\end{lemma}
\begin{proof}
The proof follows the same steps as the proof of Proposition \ref{LemmaOLocationScaleinseparabilite}, with the only difference being that we use  Lemma \ref{CoolLemma} part 4 instead of Lemma \ref{CoolLemma} part 3 in the last step. 
\end{proof}

The results of Propositions \ref{LemmaOParetoinseparabilite}, \ref{LemmaOAdditiveinseparabilite}, and  \ref{LemmaOLocationScaleinseparabilite} are not restricted for $\mathcal{F}_F$, but can be easily modified for the case $\mathcal{F}=\mathcal{F}_A$.

\begin{consequence}\label{PropositionOAdditiveParents}
 Consider $f_Y\in\mathcal{F}_A$ and let $(Y, \textbf{X})\in\mathbb{R}\times \mathbb{R}^p$ follow an SCM with DAG $\mathcal{G}_0$ where $pa_Y$ are $d$-separated.  
 \begin{itemize}
     \item If $f_Y$ has a form $f_Y(\textbf{x}, e) = h_1(\textbf{x}_S) + h_2(\textbf{x}_{pa_Y\setminus S}) + q^{-1}(e), \,\,\,\,\,\textbf{x}\in\mathbb{R}^{|pa_Y|}, e\in(0,1),$ for some non-empty $S\subset pa_Y$, where $h_1, h_2$ are continuous non-constant real functions and $q^{-1}$ is a quantile function. Then, $S_{\mathcal{F}_A}(Y) = \emptyset$.
     \item If $f_Y$ has a form $f_Y(\textbf{x}, e) = h_1(x_1) \dots h_{|pa_Y|}(x_{|pa_Y|}) + q^{-1}(e),\,\,\,\,\,\textbf{x}\in\mathbb{R}^{|pa_Y|}, e\in(0,1),$ where $h_i$ are continuous non-constant non-zero real functions and  $q^{-1}$ is a quantile function. Then every $S\subset pa_Y$ is not $\mathcal{F}_A$-plausible.
 \end{itemize}
\end{consequence}

The proof is in \hyperref[Proof of PropositionOAdditiveParents]{Appendix} \ref{Proof of PropositionOAdditiveParents}. As before, the assumption of $d$-separability can be replaced by assuming the normality of $\textbf{X}_{pa_Y}$, the assumption of multiplicative form of $f_Y$ in the second bullet-point can be weakened, and the statement can be modified for $\mathcal{F}=\mathcal{F}_{LS}$. 

\subsection{Some examples}\label{subsection3.3}

\begin{example}[Extending Example \ref{Pareto case}]\label{ExampleParetoFinal}
Consider an SCM with DAG drawn in Figure \ref{Mixed_Graph}A, where the structural equation corresponding to $Y$ is in the form (\ref{CPCM_def}) with $F$ being the Pareto distribution with $\theta(X_1, X_2) = h_1(X_1)+h_2(X_2)$ for some continuous non-constant functions  $h_1, h_2$. Moreover, let $X_3 = F^{-1}\big(\varepsilon_3, \theta_3(Y)\big)$, where $\theta_3(x)\neq a\log(x) + b$ for any $a,b\in\mathbb{R}$, $\varepsilon_3\indep Y$. Then $S_{\mathcal{F}_F}(Y) = \{1,2\}=pa_Y$. 

The reason is as follows: if $3\in S$, then $S$ is not $\mathcal{F}_F$-plausible (Proposition~\ref{TheoremFidentifiabilityWithChild} combined with results presented in Example \ref{Pareto case}). Cases $S=\{1\}$ or $S=\{2\}$ are also  not $\mathcal{F}_F$-plausible, since the function $\theta(x_1, x_2) = h_1(x_1)+h_2(x_2)$ is $\mathcal{F}_F$-inseparable wrt $(X_1, X_2)$ from Proposition \ref{LemmaOParetoinseparabilite}. Since $S=\{1,2\}$ is trivially $\mathcal{F}_F$-plausible from (\ref{subseteq_parents}), we get    $S_{\mathcal{F}_F}(Y) = \{1,2\}=pa_Y$.
\end{example}

\begin{example}[Additive noise models]\label{ANMexample}
Consider an SCM with DAG drawn in Figure \ref{Figure_DAG_for_teaser} and $\mathcal{F}=\mathcal{F}_A$. Let $(X_1, X_2, X_3)$ be normally distributed, 
 $$Y = \mu_Y(X_1, X_2, X_3) + \eta_Y,\,\,\, X_4 = \mu_1(Y)+ \mu_2(X_5, \eta_4),\,\,\,\,X_5 = \eta_5$$for some independent (not necessarily uniformly distributed) noise variables  $\eta_Y, \eta_4, \eta_5$, where $\mu_1, \mu_2$ are continuous non-zero functions. Let $(Y, X_4)$ follow an identifiable bivariate additive noise model (that is, $\mu_1$ does not satisfy a differential equation, described in Section 3.1 in \cite{Peters2014}). 

$S = \{5\}$ is not  $\mathcal{F}_A$-plausible, since $Y - f(X_3)\indep X_3$ can happen only if $f$ is a constant function (and constant functions do not belong to  $\mathcal{I}_m$). The same argument can be used whenever $5\in S\not\ni 4$. 

$S=\{4\}$ is not   $\mathcal{F}_A$-plausible since $(Y, {X}_4)$ follow an identifiable (additive) model, and Proposition~\ref{TheoremFidentifiabilityWithChild} can be used.
Sets $S \supseteq \{4, 5\}$ can be dealt with analogously.


If $\mu_Y$ is additive in the first component, that is, $\mu_Y(X_1, X_2, X_3) = h_1(X_1) + h(X_2, X_3)$ for some continuous non-constant functions $h_1, h$, then $S=\{1\}$ is  $\mathcal{F}_A$-plausible and hence $S_{\mathcal{F}_A}(Y)\subseteq \{1\}$. However, if  $\mu_Y$ is not additive in any of its components, then any $S\subset \{1,2,3\}$ is not  $\mathcal{F}_A$-plausible. 

If $\mu_Y$ does not satisfy the differential equation described in Section 3.1 in \cite{Peters2014}, we also get that every $S$ with $4\in S$ is not   $\mathcal{F}_A$-plausible since $(Y, {X}_4)$ follow an identifiable (additive) model and Proposition~\ref{TheoremFidentifiabilityWithChild} can be applied.

We do not have to pay attention to the sets $S = \{1,2,3,4\}, \{1,2,3,5\}, \{1,2,3,4,5\}$ since $pa_Y\subset S$ and the set of $\mathcal{F}_A$-identifiable parents remain the same, regardless of the plausibility of these sets.   
\end{example}


In Example \ref{ANMexample}, we considered a variable $X_5$, which is not a neighbor of $Y$. In general, we can directly discard this variable as a potential parent using the classical graphical argument of d-separation. However, if $X_i\indep Y$ for some $i\in\{1, \dots, p\}$, typically $i\not\in S_{\mathcal{F}}(Y)$. In particular, it is easy to see that if $X_i\indep Y$, then a set $S=\{i\}$ is not $\mathcal{F}$-plausible. 





\section{Estimation}
\label{section_algorithm}

We introduce two algorithms for estimating the set of direct causes of a target variable. The first algorithm is based on estimating $S_{\mathcal{F}}(Y)$ via statistical testing, and the second is a score-based algorithm that finds the best-fitting set for $pa_Y$. 

We consider a random sample of size $n\in\mathbb{N}$ from $(Y, \textbf{X})\in\mathbb{R}\times\mathbb{R}^p$, where $Y$ is the target variable, $\textbf{X}$ are the covariates, and $\mathcal{F}\subseteq\mathcal{I}_{m}$. 

\subsection{ISD algorithm for estimating $S_{\mathcal{F}}(Y)$ }
\label{ISD}

For a non-empty set $S\subseteq\{1, \dots, p\}$, define the hypothesis $$H_{0, S} (\mathcal{F}): S \text{   is an }\mathcal{F} \text{-plausible set of parents of Y}. $$ Suppose for the moment that a statistical test for $H_{0, S} (\mathcal{F})$ with size smaller than a significance level $\alpha$ is available. Then, we define $$\hat{S}_{\mathcal{F}}(Y):=\bigcap_{ \emptyset\neq S\subseteq\{1, \dots, p\}:   H_{0, S} (\mathcal{F}) \text{ is not rejected}}S$$ as an intersection of all sets for which  $H_{0, S} (\mathcal{F})$ was not rejected. 

In order to test  $H_{0, S} (\mathcal{F})$,  we propose a procedure called ISD (Independence $+$ Significance $ +$ Distribution). The idea is to decompose  $H_{0, S} (\mathcal{F})$ into three sub-hypothesis. In particular, $H_{0, S} (\mathcal{F})$ is true if and only if there exist a function $\hat{f}_S$ such that $\hat{\varepsilon}_S:=\hat{f}_S^{\leftarrow}(\textbf{X}_S, Y)$  satisfies: 
\begin{enumerate}
    \item   $H_{0, S}^I: \hat{\varepsilon}_S \indep \textbf{X}_S$, \noindent\hfill (\textbf{I}ndependence)
    \item   $H_{0, S}^S$: $\hat{f}_S\in\mathcal{F}$,  \noindent\hfill (\textbf{S}ignificance) \newline (recall that $\hat{f}_S$ must be irreducible almost surely or in other words, all inputs are significant)
    \item    $H_{0, S}^D:$ $\hat{\varepsilon}_S \sim U(0,1)$. \noindent\hfill (\textbf{D}istribution)
\end{enumerate}

We reject  $H_{0, S} (\mathcal{F})$ if and only if one of  $H_{0, S}^I, H_{0, S}^S, H_{0, S}^D$  is rejected. 

\begin{theorem}
\label{theorem_consistency_ISD}
Assume that $(Y, \textbf{X})$ satisfies ( \ref{SCM_for_Y}) with $pa_Y\neq \emptyset$.  Assume that the estimator  $\hat{S}_{\mathcal{F}}(Y)$  is constructed as described above with $\hat{f}_{pa_Y} = f_Y$ and with valid tests $H_{0, S}^I, H_{0, S}^S, H_{0, S}^D$  for all non-empty set $S\subseteq \{1, \dots, p\}$ at level $\alpha$ in a sense that for all $S$, $\sup_{P: H_{0, S}^\cdot \text{is true}}P(H_{0, S}^\cdot \text{ is rejected})\leq \alpha$ for all $\cdot \in \{S, I, D\}$. Then
$$
P(\hat{S}_{\mathcal{F}}(Y) \subseteq pa_Y) \geq 1-3\alpha. 
$$
\end{theorem}
\textit{Proof.} \, The proof follows directly from 
\begin{align*}
    & P(\hat{S}_{\mathcal{F}}(Y) \subseteq pa_Y) \geq  P(H_{0, pa_Y}(\mathcal{F}) \text{ is not rejected}) =  P(H_{0, pa_Y}^I, H_{0, pa_Y}^S, H_{0, pa_Y}^D \text{ are not rejected})\\& \geq \prod_{\cdot \in \{I, S, D\} } P(H_{0, pa_Y}^\cdot\text{ is not rejected}) \geq (1-\alpha)^3 > 1-3\alpha \,\,\,\,\,\,\,\,for\,\,\alpha\in(0,1). \,\,\,\,\,\,\,\,\,\,\,\,\,\,\,\, \qedsymbol
\end{align*} 
In order to find a suitable candidate for the function $\hat{f}_S$, we use classical methods from machine learning. 
If $\mathcal{F} = \mathcal{F}_A$ or $\mathcal{F} = \mathcal{F}_{LS}$, we can apply random forest, neural networks, GAM, or other classical methods \citep{GAM,GAMLSS, Paul2014StatisticallyII}. Using one of these methods, we estimate the conditional mean ${\mu}$ (and variance ${\sigma}$ in $\mathcal{F}_{LS}$ case) and output the residuals $\hat{\eta}_S:=Y -\hat{\mu}(\textbf{X}_S)$ (or $ \hat{\eta}_S:=\frac{Y -\hat{\mu}(\textbf{X}_S)}{\hat{\sigma}(\textbf{X}_S)}$ in $\mathcal{F}_{LS}$ case). Possibly, re-scale the residuals $\hat{\varepsilon}_S := \hat{q}(\hat{\eta}_S)$, where $\hat{q}$ is the empirical distribution function of $\hat{\eta}$ (see the discussion about $H_{0, S}^D$ below). If  $\mathcal{F} = \mathcal{F}_F$ for some distribution function $F$, we can use GAMLSS \citep{GAMLSS} or GAM algorithms for estimating $\theta$. Then, we define $\hat{\varepsilon}_S:=F\big(Y, \hat{\theta}(\textbf{X}_S)\big)$.

Notice that if the chosen method is consistent and ( \ref{SCM_for_Y}) holds, $\hat{f}_{pa_Y}$ converges to $f_{Y}$. Therefore, the choice $\hat{f}_{pa_Y} = f_Y$ in Theorem~\ref{theorem_consistency_ISD} is justified in large sample sizes. The following tests can be used for practical testing of  $H_{0, S}^I, H_{0, S}^S$ and $H_{0, S}^D$:

\begin{enumerate}
    \item $H_{0, S}^I$: We can use a kernel-based HSIC test \citep{Kernel_based_tests} or a copula-based test \citep{copula_based_independence_test}.
    \item $H_{0, S}^S$:  This test ensures that $\hat{f}_S$ is irreducible, meaning that we do not include non-significant (and hence non-causal) covariates into an $\mathcal{F}$-plausible set. In practice, we test the alternative hypothesis \({H}_{0, S}^{S, \text{alt}}: \hat{f}_S \not\in \mathcal{F}\), and we reject \(H_{0, S}^S\) if and only if we do not reject \({H}_{0, S}^{S, \text{alt}}\). The reason is that many methods have been developed for testing \({H}_{0, S}^{S, \text{alt}}\). For example, in the case of linear regression \(Y = \beta \textbf{X}_S + \eta_S\), we test if \(\beta_i \neq 0\) for all \(i \in S\) via classical significance testing. Analogously for GAM or GAMLSS. Alternatively, we can use a permutation test to assess the significance of the covariates \citep{Paul2014StatisticallyII}.
    \item $H_{0, S}^D$: This step is only relevant when a specific noise distribution is assumed. However, the hypothesis \( H_{0, S}^D \) is automatically true in cases such as \(\mathcal{F} = \mathcal{F}_L\), \(\mathcal{F}_A\), or \(\mathcal{F}_{LS}\). In these instances, we omit this test. The reason is that we can use a probability integral transform of the estimated noise to obtain $\hat{\varepsilon}_S\sim U(0,1)$. However in cases such as  $\mathcal{F}=\mathcal{F}_F$, the integral transform breaks the condition $\hat{f}_S\in\mathcal{F}_F$ and testing for $\hat{\varepsilon}_S\sim U(0,1)$ is necessary.  \newline If we opt for testing $H_{0, S}^D:$ $\hat{\varepsilon}_S \sim U(0,1)$, we can use a Kolmogorov-Smirnov or Anderson-Darling test \citep{AD-KStest}.
\end{enumerate}

In our implementation, we opt for HSIC test, GAM estimation, and the Anderson-Darling test. We summarize the algorithm in case of $\mathcal{F} = \mathcal{F}_A$ as follows:

\begin{algorithm}[H]
  \SetAlgoLined
  \KwData{Random sample $(y_1, x_1^1, \dots, x_1^p), \dots, (y_n, x_n^1, \dots, x_n^p)$}
  \KwResult{ REJECT or NOT REJECT  }

1) Estimate \(\hat{f}_S\) in the model \(Y = f_S(\textbf{X}_S) + \eta_S\) (using GAM estimation, for example). Define \(\hat{\eta}_S := Y - \hat{f}_S(\textbf{X}_S)\).

2) Test \(\hat{\eta}_S \indep \textbf{X}_S\) at level \(\alpha\) (using the HSIC test, for example). Set \(H_{0, S}^I = \text{REJECT}\) if this test was rejected, otherwise set \(H_{0, S}^I = \text{NOT REJECT}\).

3) Set \(H_{0, S}^S = \text{NOT REJECT}\) if all covariates are significant at level \(\alpha\) in the model from step 1 (using the permutation test for covariate significance, for example). Otherwise, set \(H_{0, S}^S = \text{REJECT}\).

4) Automatically define \(H_{0, S}^D = \text{NOT REJECT}\) (this step is not relevant in the case \(\mathcal{F} = \mathcal{F}_A\)).

5) Return \(\text{NOT REJECT}\) if all \(H_{0, S}^I\), \(H_{0, S}^S\), and \(H_{0, S}^D\) were not rejected. Otherwise, return \(\text{REJECT}\).


\caption{Testing $H_{0, S} (\mathcal{F})$ in case of $\mathcal{F} = \mathcal{F}_A$ }
  \label{Algorithm}
\end{algorithm}




\subsection{Score-based estimation of $pa_Y$}\label{Section_algorithm2}
We propose a score-based algorithm for estimating the set of direct causes of $Y$. It is a local counterpart of score-based algorithms for estimating the full DAG $\mathcal{G}_0$, following the ideas from \cite{Score-based_causal_learning}, \cite{Peters2014}, and \cite{bodik2023identifiability}. Recall that 
under ( \ref{SCM_for_Y}), the set $S=pa_Y$ should satisfy that $\varepsilon_S\indep \textbf{X}_S$, every ${X}_i, i\in S$ is significant and $\varepsilon_S\sim U(0,1)$. 
Therefore, we use the following score function: 
\begin{equation*}
\begin{split}
\widehat{pa}_Y =  \argmax_{\substack{S\subseteq\{1, \dots, p\}\\
                  S\neq\emptyset}}score(S) &= \argmax_{\substack{S\subseteq\{1, \dots, p\}\\
                  S\neq\emptyset}}\lambda_1 (Independence) + \lambda_2(Significance)+ \lambda_3 (Distribution),
\end{split}
\end{equation*}
where $\lambda_1, \lambda_2, \lambda_3\in [0, \infty)$,  ``\textit{Independence}'' is a measure of independence between $ (\hat{\varepsilon}_S, \textbf{X}_S)$, ``\textit{Significance}'' is a measure of significance of covariates $\textbf{X}_S$, and ``\textit{Distribution}'' is a distance between the distribution of $\hat{\varepsilon}_S$ and $U(0,1)$, where $\hat{\varepsilon}_S$ is the noise estimate defined in Section~\ref{ISD}. 

\textit{The measure of independence} can be chosen as the p-value of the independence test (such as the kernel-based HSIC test or the copula-based test). 
\textit{The measure of significance} corresponds to the estimation method analogously to the ISD case. For linear regression (similarly for GAM or GAMLSS), we compute the corresponding p-values for the hypotheses $\beta_i=0, i\in S$.  Then, \textit{Significance} is the minus of the maximum of the corresponding p-values (worst case option). We can also use a permutation test to assess the covariate's significance in terms of the predictability power and choose the largest p-value. 
\textit{The distance between the distribution of} $\hat{\varepsilon}_S$ \textit{and} $U(0,1)$ can be chosen as the p-value of the Anderson-Darling test. 

The choice of $\lambda_1, \lambda_2, \lambda_3$ describes weights we put on each of the three scores: if $\lambda_1>\lambda_2, \lambda_3$, then our estimate will be very sensitive against the violation of the independence $\varepsilon_S\indep \textbf{X}_S$, but not as sensitive against the violation of the other two properties. 

In our implementation, we opt for the following choices. The \textit{Independence} term is the logarithm of the p-value of the Kernel-based HSIC test, and the \textit{Distribution} term is the logarithm of the p-value of the Anderson-Darling test. We use GAM for the estimation of $\hat{f}_S$ and minus the logarithm of the maximum of the corresponding p-values for the \textit{Significance} term. The logarithmic transformation of the three p-values is used to re-scale the values from $[0,1]$ to $(-\infty, 0]$. The practical choice for the weights is   $\lambda_1 = \lambda_2 = \lambda_3 =1 $ (unless $\mathcal{F}=\mathcal{F}_L,\mathcal{F}_A$, or $\mathcal{F}_{LS}$ when we put $\lambda_3 = 0$). 

\subsubsection{Consistency}

Consistency of the proposed algorithm follows from the results presented in \cite{reviewANMMooij}, who showed consistency of the score-based DAG estimation for additive noise models. In the following, we consider $\mathcal{F} = \mathcal{F}_A$, although it is straightforward to generalize these results for other types of $\mathcal{F}$ (for a discussion about $\mathcal{F} = \mathcal{F}_{LS}$, see \cite{sun2023causeeffect}, and for $\mathcal{F} = \mathcal{F}_F$, see \cite{bodik2023identifiability}). For simplicity, we assume that the measure of independence is the negative value of HSIC test itself (not its p-value as we use in our implementation), and the estimate $\hat{f}_S$ is \textit{suitable} in the sense that noise estimate $\hat{\varepsilon} = \hat{f}_S^\leftarrow(\textbf{X}_S, Y)$ satisfies 
$$
\lim_{n\to\infty}\mathbb{E}_{}\bigg( \frac{1}{n}\sum_{i=1}^n(\varepsilon_i-\hat{\varepsilon}_i)^2   \bigg) = 0,
$$
where the expectation is taken with respect to the distribution of the random sample  \cite[Appendix A.2]{reviewANMMooij}. 

\begin{proposition}\label{Proposition_consistency}
Consider $\mathcal{F} = \mathcal{F}_A$ and 
let $(Y, \textbf{X})\in\mathbb{R}\times \mathbb{R}^p$ follow an SCM with DAG $\mathcal{G}_0$ satisfying ( \ref{SCM_for_Y}). Assume that every $S \neq pa_Y\neq \emptyset$ is not $\mathcal{F}$-plausible. 
Then,   

\begin{equation}
  \lim_{n\to\infty} \mathbb{P}(\widehat{pa}_Y  \neq pa_Y) = 0,
\end{equation}   
where $n$ is the size of the random sample and $\widehat{pa}_Y$ is our score-based estimate from Section~\ref{Section_algorithm2} with $\lambda_1, \lambda_2>0, \lambda_3 = 0$, suitable estimation procedure, and HSIC independence measure. 
\end{proposition}

The proof is in \hyperref[Proof of Proposition_consistency]{Appendix} \ref{Proof of Proposition_consistency}.  If several sets are $\mathcal{F}$-plausible, the score-based algorithm provides no guarantees that the set $S=pa_Y$ will have the best score among them (as opposed to the ISD algorithm that outputs their intersection). 

\subsection{What is a suitable $\mathcal{F}$ in practice?}

The choice of the class $\mathcal{F}$ is a crucial step in the algorithm. The choice of an appropriate model is a common problem in classical statistics; however, it is more subtle in causal discovery. It has been shown \citep{Peters2014} that methods based on restricted structural equation models can outperform traditional methods (these results were shown when estimating the entire DAG). Even if assumptions such as additive noise or Gaussian distribution of the effect given the causes can appear to be strong, such methods have turned out to be rather useful, and small violations of the model still lead to a good estimation procedure. 

The size of  $\mathcal{F}$ is the most important part. If $\mathcal{F}$ contains too many functions, we find that most sets are  $\mathcal{F}$-plausible. On the other hand, overly restrictive  $\mathcal{F}$ can lead to rejecting all sets as potential causes.  If we have knowledge about the data-generation process (such as when $Y$ is a sum of many small events), choosing $\mathcal{F}_F$ for a distribution function $F$ (such as Gaussian) is reasonable. For the choices $\mathcal{F} = \mathcal{F}_A$ or $\mathcal{F}_{LS}$,  there are numerous papers justifying such assumptions in several settings (when estimating the full DAG, \cite{reviewANMMooij}, \cite{Elements_of_Causal_Inference},  \cite{immer2022identifiability}). 

  







\section{Simulations and application}
\label{section_simulations}

We conducted a simulation study to evaluate the performance of the algorithms. These simulations can be found in \hyperref[section_appendix_simulations]{Appendix} \ref{section_appendix_simulations}. The first simulation study (in \hyperref[section_appendix_simulations]{Appendix} \ref{Section_Additive})  demonstrates a congruence between theoretical insights and the outcomes obtained from simulations.  The second simulation study (in \hyperref[section_appendix_simulations]{Appendix} \ref{Section_benchmarks}) involves a comparison between our algorithms and classical approaches using three benchmark datasets.  Table~\ref{Table_results} provides a concise overview of the results (see \hyperref[section_appendix_simulations]{Appendix} \ref{section_appendix_simulations} for more details).

% Please add the following required packages to your document preamble:
% \usepackage{multirow}
\begin{table}[h]
\centering
\begin{tabular}{|l|c|c|c|c}
\cline{1-4}
\multirow{2}{*}{}        & \textbf{First}     & \textbf{Second}                         & \textbf{Third}                          & \textbf{Total}  \\
                         & \textbf{Benchmark} & \multicolumn{1}{l|}{\textbf{Benchmark}} & \multicolumn{1}{l|}{\textbf{Benchmark}} & \textbf{(Mean)} \\ \hline
\textbf{IDE algorithm}   & 98\%/ 100\%        & 82\%/ 100\%                             & 72\%/ 100\%                             & 84\%/ 100\%     \\ \cline{1-4}
\textbf{Score algorithm} & 100\%/ 100\%       & 98\%/ 100\%                             & 92\%/ 88\%                              & 93\%/ 96\%      \\ \cline{1-4}
RESIT                    & 52\%/ 18\%         & 36\%/ 100\%                             & 2\%/ 94\%                               & 30\%/ 71\%      \\ \cline{1-4}
CAM-UV                      & 96\%/ 40\%         &  2\%/ 100\%                             & 0\%/ 100\%                              & 32\%/ 80\%      \\ \cline{1-4}
Pairwise bQCD            & 100\%/ 0\%         & 56\%/ 100\%                             & 80\%/ 26\%                              & 78\%/ 42\%      \\ \cline{1-4}
Pairwise IGCI            & 0\%/ 48\%          & 0\%/ 100\%                              & 70\%/ 34\%                              & 17\%/ 70\%      \\ \cline{1-4}
Pairwise Slope           & 100\%/ 2\%         & 100\%/ 100\%                            & 100\%/ 22\%                             & 100\%/ 41\%     \\ \cline{1-4}
\end{tabular}
\caption{Performance of different algorithms on the three benchmark datasets. The first number represents the ``percentage of discovered correct causes'', and the second is the ``percentage of no false positives''. All information can be found in \hyperref[section_appendix_simulations]{Appendix} \ref{section_appendix_simulations}.   }
\label{Table_results}
\end{table}









\subsection{Real-world example}
\label{section_application}
%Change 2 to X_2
%Erase Appendix A.2.
To illustrate our methodology on a real-world example, we consider data on the fertility rate like in \cite{Christina}. The target variable of interest is $Y = \text{`Fertility rate'},$ measured yearly in more than 200 countries.  Developing countries exhibit a significantly higher fertility rate than Western countries \citep{Cheng2022}. The fertility rate can be predicted by considering covariates such as the `infant mortality rate' or `GDP.' However, if one wants to explore the potential effect of a particular law or a policy change, it becomes necessary to leverage the causal knowledge of the underlying system.

Randomized studies are not possible to design in this context since factors like `infant mortality rate' cannot be isolated for manipulation. Even so, understanding the impact of policies to reduce infant mortality rates within a country remains an important question, even if randomized studies are unfeasible.

Here, we consider covariates $\textbf{X} = (X_1, X_2, X_3, X_4)^\top$, where $X_1=$`GDP (in US dollars)', $X_2 = $`Education expenditure (\% of GDP)', $X_3 = $`Infant mortality rate (infant deaths per 1,000 live births)', $X_4 = $`Continent'. The data come from \cite{worldbank_data_about_Education, worldbank_data_about_GDP, unitednations_data_about_fertility}. Visualizations of the data can be found in Appendix~\ref{Appendix_Application}. 

Explaining changes in fertility rate is still a topical issue for which no apparent rational explanation exists. In our study, we focus on using our developed framework to provide data-driven answers about the potential causes of changes in fertility rates. 

In a study by \cite{Christina}, the authors employed a method called invariant causal prediction (ICP; \cite{Peters_invariance}, \cite{PfisterTimeSeries}) to determine the possible causes of $Y$. The ICP approach relies on observing an environmental variable (closely related to a context variable or an instrumental variable; \cite{Mooij2020}). However, selecting a suitable environmental variable can be a subject of debate, and in many cases, it may be challenging to identify one. Nevertheless, we acknowledge the potential benefits of utilizing such a variable, as it can enhance the reliability of the results. 

We apply the methodology developed in this paper to estimate the causes of $Y$. We tried several choices of $\mathcal{F}$; in particular, $\mathcal{F}_A, \mathcal{F}_{LS}, \mathcal{F}_{F_1},\mathcal{F}_{F_2},\mathcal{F}_{F_3} $ for the choices $F_1, F_2, F_3 = Gaussian$, $Gamma$, and $Pareto$, respectively. These candidate choices come from a preliminary inference on the marginal distributions of the data. For the choice $\mathcal{F} = \mathcal{F}_A, \text{ or }\mathcal{F}_{F_3}$, we observe that all sets $S\subseteq \{1, \dots, 4\}$ are strongly rejected as $\mathcal{F}$-plausible and our estimate is an empty set. One reason for this is the restricted aspects provided by these specific choices of $\mathcal{F}$; our data show much more complex relations than those that can be described by just one parameter (the mean in the case $\mathcal{F}_A$ and the tail index in the case $\mathcal{F}_{F_3}$).  

Applying our methodology with the choices $\mathcal{F}_{LS}, \mathcal{F}_{F_1},\mathcal{F}_{F_2}$, we obtain the results described in Table~\ref{Table_application}. The results suggest that $X_3$ is the identifiable cause of $Y$. This is in line with findings from \cite{Christina} (backed up by research from sociology in \cite{hirschman1994}), who also discovered the variable $X_3$ to be causal. Furthermore, the score-based estimate  indicates that $X_2$ is a member of $\widehat{pa}_Y$ across all three selections of $\mathcal{F}$. This suggests that $X_2$ is a cause of $Y$ as well, even though the score-based estimate does not have the same guarantees as the set $\hat{S}_{\mathcal{F}}(Y)$. Note that sets $\{2,3\}, \{1,2,3\}$ are $\mathcal{F}$-plausible for all three choices of $\mathcal{F}$. 

We emphasize that the findings rely on the causal sufficiency of the variables employed, an assumption that can surely be questioned. For instance, other variables such as `religious beliefs' or a `political situation' may explain the fertility rate, but are hard to measure. 


% Please add the following required packages to your document preamble:
% \usepackage{multirow}
\begin{table}[]
\begin{tabular}{|c|c|c|c|}
\hline
\multirow{2}{*}{$\mathcal{F}$}       & \multirow{2}{*}{$\mathcal{F}$-plausible sets} & ISD estimate of the                                       & \multirow{2}{*}{Score-based estimate of $\widehat{pa}_Y$} \\
                                     &                                               & $\mathcal{F}$-identifiable set $\hat{S}_{\mathcal{F}}(Y)$ &                                               \\ \hline
$\mathcal{F}_{LS}$                   & \{2,3\}, \{3,4\}, \{1,2,3\}, \{1,3,4\}        & \{3\}                                                     & \{2,3\}                                       \\ \hline
$\mathcal{F}_{F_1}$                  & \{2,3\}, \{2,3,4\}, \{1,2,3\}, \{1,3,4\}      & \{3\}                                                     & \{1,2,3\}                                     \\ \hline
\multirow{2}{*}{$\mathcal{F}_{F_2}$} & \{3\}, \{2,3\}, \{3,4\}, \{1,2,3\},           & \multirow{2}{*}{\{3\}}                                    & \{1, 2,3,4\} (with almost equal               \\
                                     & \{1,3,4\}, \{2,3,4\},\{1,2,3,4\}              &                                                           & scores for \{1,2,3\}, \{2.3.4\})              \\ \hline
\end{tabular}
\caption{Resulting estimates of causes of changes in fertility rates from real data of Section \ref{section_application}. Here, $F_1$ is a Gaussian distribution, and $F_2$ is a Gamma distribution. }
\label{Table_application}
\end{table}


\section{Discussion and future work}

In this work, we studied the problem of estimating the direct causes of a target variable \( Y \). We introduced a general framework that leverages identifiability theory for full causal graphs \( \mathcal{G} \) in a localized setting. This allows for more flexible and scalable applications of causal discovery. 

Several avenues for future work remain. It would be valuable to adapt other causal discovery methods, such as IGCI or those based on Kolmogorov complexity \citep{IGCI, Natasa_Tagasovska}, to the local setting. Similarly, extensions of our framework to time series data \citep{bodik, bodik2024grangercausalityextremes} are a natural next step. Ideas from recent advances in the invariance framework, such as defining sets and simultaneous false discovery bounds \citep{Christina, li2024simultaneousICP}, could also be incorporated into our ISD algorithm to improve its power and computational efficiency. Our local framework is also well-suited for prediction under distribution shift, such as covariate shift or extrapolation \citep{jin2024reweightingpredictiverolecovariate, bodik_extrapolation}, where identifying direct causes enables more robust predictions. Furthermore, leveraging tools such as instrumental variables \citep{Imbens2014} or anchor regression \citep{10.1111/rssb.12398} may offer a principled way to address unobserved confounding and improve robustness in complex settings. 

A central limitation of our approach lies in the choice of the functional class $\mathcal{F}$. As in most causal discovery methods, such as those based on LiNGAM, post-nonlinear models, or location-scale assumptions, the validity and interpretability of the results depend critically on how well the chosen model class matches the underlying data-generating process. While the computational complexity may increase with the dimensionality of $\textbf{X}$, especially when testing many subsets, this also opens opportunities for improvement through scalable algorithms, dimension reduction techniques, or regularization strategies tailored to the local setting.

Overall, the theory developed in this work contributes to a deeper understanding of causal structure and the fundamental limitations of purely data-driven approaches to causal inference. We believe that causal discovery on a local scale provides a promising path toward practical applications, particularly in high-dimensional settings, and this work takes an important step toward making such methods more accessible, interpretable, and robust.


\section*{Acknowledgment}
This study was supported by the Swiss National Science Foundation, grant number 201126. 


\section*{Supplementary material}
\label{SM}
The supplementary material contains some detailed definitions and more technical explanations of the theory omitted from the main text for clarity, simulation study, some auxiliary lemmas, and all the technical details and proofs.

The code and data are available in the online repository \url{https://github.com/jurobodik/Structural-restrictions-in-local-causal-discovery.git} or on request from the author.   


\bibliography{bibliography}
\newpage
\appendix
\section{Appendix} \label{Speci_appendix}

\subsection{Class of invertible and minimal functions, invertible causal model, and causal minimality}
\label{Appendix_A.1.}

First, let us formally introduce the notions of invertibility and minimality of a real function. Next, we define a class of measurable functions $\mathcal{I}_m$ and a subclass of SCM called invertible causal models. We show that minimality of a link function is equivalent with causal minimality of the causal model. 

\begin{definition}[Invertibility]\label{I}
Let $\mathcal{X}_x\subseteq\mathbb{R}^{p},\mathcal{X}_y\subseteq\mathbb{R}, \mathcal{X}_z\subseteq\mathbb{R}$ be measurable sets. A measurable function $f:\mathcal{X}_x\times\mathcal{X}_y\to \mathcal{X}_z$  is called \textbf{invertible for the last element}, notation $f\in \mathcal{I}$, if there exists a function $f^{\leftarrow}:\mathcal{X}_x\times\mathcal{X}_z\to \mathcal{X}_y$ that fulfills the following: $\forall \textbf{x}\in\mathcal{X}_x, \forall y\in\mathcal{X}_y,z\in\mathcal{X}_z$ such that $y=f(\textbf{x},z)$, then $z=f^{\leftarrow}(\textbf{x},y)$. 
\end{definition}
The previous definition indicates that the element $z$ in a relationship $y=f(\textbf{x},z)$ can be uniquely recovered from $(\textbf{x},y)$. To provide an example, for the function $f(x,z) = x+z$, it holds that $f^{\leftarrow}(x, y) = y-x$, since $f^{\leftarrow}(x, f(x,z)) = f(x,z) - x = z$. More generally, for the additive function defined as $f(\textbf{x},z) = g_1(\textbf{x})+ g_2(z)$, where $g_2$ is invertible, it holds that $f^{\leftarrow}(\textbf{x}, y) = g_2^{-1}(y - g_1(\textbf{x}))$, $\textbf{x}\in\mathbb{R}^d, y, z\in \mathbb{R}$. Overall, if $f$ is differentiable and the partial derivative of $f(\textbf{x},z)$ with respect to $z$ is monotonic, then $f\in\mathcal{I}$ (follows from inverse function theorem). 

\begin{definition}[Minimality]
    We say that a function $f:\mathbb{R}^n\to\mathbb{R}$ \textit{is minimal almost surely}, notation $f\in\mathcal{M}$, if there does not exist a function $g:\mathbb{R}^{n-1}\to \mathbb{R}$ and $k\leq n$, such that $f(x_1, \dots, x_n) = g(x_1, \dots, x_{k-1}, x_{k+1}, \dots, x_n)$ for almost all $ \textbf{x}\in\mathbb{R}^n$ in the support of $f$. Recall that the notion 'almost all' represents the fact that the measure of a set $\{\textbf{x}\in\mathbb{R}^n: f(x_1, \dots, x_n) \neq g(x_1, \dots, x_{k-1}, x_{k+1}, \dots, x_n)\}$ has a Lebesgue measure zero. 
\end{definition}

\begin{definition}[Invertible causal model]
 We denote the set of invertible and almost surely minimal functions by
$$
\mathcal{I}_{m} = \{f\in\mathcal{I}\cap \mathcal{M} \}.
$$
We define the \textbf{ICM} (invertible causal model) as a SCM (\ref{definition_general_SCM}) with structural equations satisfying $f_i\in\mathcal{I}_{m}$ for all $i=0, \dots, p$.  
\end{definition}


Note that $f_i\in\mathcal{I}_{m}$ implies causal minimality of the ICM model, as the following lemma suggests. Recall that a distribution $P_\mathbf{X}$ over $\textbf{X}$ satisfies \textbf{causal minimality} with respect to $\mathcal{G}$ if it is Markov with respect to $\mathcal{G}$, but not to any proper subgraph of $\mathcal{G}$. 

\begin{lemma}\label{Lemma_about_ICM_minimality}
Consider a distribution generated by \hyperref[I]{ICM} with graph $\mathcal{G}_0$ (see Definition  \ref{I}). Let all structural equations $f_j\in\mathcal{I}_{m}$, $\forall j=0, \dots, p$.  Then, the distribution is causally minimal with respect to  $\mathcal{G}_0$. Conversely, if $f_j\not \in\mathcal{M}$ for some $j\in\{0, \dots, p\}$, then the causal minimality is violated. 
\end{lemma}
\begin{proof}
The second claim follows directly from Proposition 4 in \cite{Peters2014}. For the first claim, we use a similar approach as in Proposition 17 from \cite{Peters2014}. 

Let $f_j\in\mathcal{I}_{m}$ for all $j=0, \dots, p$ and let the causal minimality be violated, i.e., let $\tilde{\mathcal{G}}$ be a subgraph of $\mathcal{G}_0$ such that the distribution is Markov wrt $\tilde{\mathcal{G}}$. Find $i,j\in\mathcal{G}_0$ such that $i\to j$ in $\mathcal{G}_0$ but $i\not\to j$ in $\tilde{\mathcal{G}}$. 

In graph $\mathcal{G}_0$ we have a structural equation  $X_j = f_j(\textbf{X}_{pa_j(\mathcal{G}_0)}, \varepsilon_j) = f_j(X_i, \textbf{X}_{pa_j(\mathcal{G}_0)\setminus\{i\}}, \varepsilon_j)$  but in $\tilde{\mathcal{G}}$ we have $X_j = \tilde{f}_j(\textbf{X}_{pa_j(\mathcal{G}_0)\setminus\{i\}}, \varepsilon_j)$. Hence, functions $f_j(X_i, \textbf{X}_{pa_j(\mathcal{G}_0)\setminus\{i\}}, \varepsilon_j)$ and $\tilde{f}_j( \textbf{X}_{pa_j(\mathcal{G}_0)\setminus\{i\}}, \varepsilon_j)$ have to be equal almost surely, which contradicts $f_j\in\mathcal{I}_{m}$. 
\end{proof}




\subsection{ Lemma~\ref{LemmaAboutUnidentifiabilityFL} modified for hidden confounders}
\label{Appendix_A_Lemma}

In the following, we use the notion of m-separability, a generalization of d-separability for mixed-type graphs. For details see \cite{Richardson}. Moreover, we say that a node in a graph is a source node, if all edges associated to the node are directed out-going edges (i.e. only $v\to \cdot$ are allowed). 

\begin{lemma}\label{LemmaAboutUnidentifiabilityFL2}
Let $(Y, \textbf{X})\in\mathbb{R}\times \mathbb{R}^p$ follow an $\mathcal{F}_L$-model with DAG ${\mathcal{G}}_0$. Let $\tilde{\textbf{X}}\subseteq \textbf{X}$ be observed variables (and $\textbf{X}\setminus \tilde{\textbf{X}}$ are unobserved hidden confounders). Let $\tilde{\mathcal{G}}_0$ be a projection of ${\mathcal{G}}_0$ on the observed variables. If there exist a source variable $a\in pa_Y(\tilde{\mathcal{G}}_0)$ then $|S_{\mathcal{F}_L}(Y)| \leq 1$. Moreover, if there exist a pair of source variables  $a,b\in an_Y(\tilde{\mathcal{G}}_0)$ that are m-separated in $\tilde{\mathcal{G}}_0$, then $S_{\mathcal{F}_L}(Y) = \emptyset$. 
\end{lemma}

\begin{proof}\label{Proof of LemmaAboutUnidentifiabilityFL 2}
The proof is fully analogous to the proof of Lemma~\ref{Proof of LemmaAboutUnidentifiabilityFL}. Since we added the assumption that $a,b$ are source variables, the fact that some variables are unobserved does not change any step in the proof. 
\end{proof}



\subsection{Definition of restricted additive noise model from \cite{Peters2014}}
\label{Appendix_restricted_additive_noise_model}

We restate the definition of restricted additive noise model presented in Section 3 in \cite{Peters2014}. 

\begin{definition}
An  $\mathcal{F}_A$-model is called a restricted additive noise model if for all \(j \in V\), \(i \in \text{PA}_j\) and all sets \(S \subseteq V\) with \(\text{PA}_j \setminus \{i\} \subseteq S \subseteq \text{ND}_j \setminus \{i, j\}\), there is an \(x_S\) with \(p_S(x_S) > 0\), such that

\[
\left( f_j ( x_{\text{PA}_j \setminus \{i\}}, \underbrace{\cdot}_{X_i} ), P_{(X_i \mid \textbf{X}_S = \textbf{x}_S)}, P_{\eta_j} \right)
\]
satisfies Condition~\ref{Condition19}. Here, the underbrace indicates the input component of \(f_j\) for variable \(X_i\). In particular, we require the noise variables to have non-vanishing densities and the functions \(f_j\) to be continuous and three times continuously differentiable.
\end{definition}


\begin{condition}\label{Condition19}
    The triple \((f_j, P_{X_i}, P_{\eta_j})\) does not solve the following differential equation for all \(x_i, x_j\) with \(\nu''(x_j - f(x_i)) f'(x_i) \neq 0\):

\[
\xi''' = \xi'' \left( -\frac{\nu''' f'}{\nu''} + \frac{f''}{f'} \right) - 2 \nu'' f'' f' + \nu' f''' + \frac{\nu' \nu''' f'' f'}{\nu''} - \nu' \frac{(f'')^2}{f'},
\]
where \(f := f_j\), \(\xi := \log p_{X_i}\), and \(\nu := \log p_{\eta_j}\) are the logarithms of the strictly positive densities. To improve readability, we have skipped the arguments \(x_j - f(x_i)\), \(x_i\), and \(x_i\) for \(\nu\), \(\xi\), and \(f\) and their derivatives, respectively.
\end{condition}

\cite[Theorem 28]{Peters2014} showed that $\mathcal{G}$ is  identifiable from the joint distribution under a causally minimal restricted additive noise model.

\begin{theorem}[Theorem 20 in \cite{Peters2014}]
\label{theorem20}
     Let \(P_{(X_0, X_1)} \) be generated by a bivariate additive noise model with graph \( \mathcal{G}_0 \) satisfying Condition~\ref{Condition19} and assume causal minimality, i.e., a non-constant function \( f_j \). Then, \( \mathcal{G}_0 \) is identifiable from the joint distribution.
\end{theorem}


\begin{theorem}[Theorem 28 in \cite{Peters2014}]
\label{theorem28}
 Let \( P_{(X_1, \ldots, X_p)} \) be generated by a restricted additive noise model with graph \( \mathcal{G}_0 \) and let \( P_{(X_1, \ldots, X_p) }\) satisfy causal minimality with respect to \( \mathcal{G}_0 \) (which holds for example if the functions \( f_j \) are minimal). Then, \( \mathcal{G}_0 \) is identifiable from the joint distribution.

\end{theorem}




\subsection{General identifiability $\implies$ $\mathcal{F}$-identifiability for non-additive $\mathcal{F}$}
\label{Appendix_pairwise_identifiability}

In the following, we restate the result from Section~\ref{section_general_implies_local_identifiability} for general $\mathcal{F}$. One may expect that $\mathcal{F}$-identifiability of $Y$ follows automatically if we assume an identifiable $\mathcal{F}$-model for all variables in the SCM. This is not the case in general.  
\begin{example}
Consider the following SCM: 
\begin{equation*}
X_1 = \eta_1; \,\,\,\,\,\,\,\,Y = X_1^2 + \eta_Y; \,\,\,\,\,\,\,\,X_2 = \beta_2 Y + \eta_2, 
\end{equation*}
where $\eta_1, \eta_2, \eta_Y$ are independent, $\eta_1\sim U(0,1), \eta_Y, \eta_2\sim N(0, 1)$. This SCM is identifiable (for example, SCM with graph $X_2\to Y\to X_1$ does not allow writing $X_1 = f_1(Y) + \tilde{\eta}_1$ for any $f_1, \tilde{\eta}_1$). However, notice that under conditioning on $X_1 = x\in (0,1)$ we obtain a linear Gaussian case and we can revert the equation between $(Y, X_2)$ and obtain $Y = X_1^2 + \tilde{\beta}_2X_2 + \tilde{\eta}_Y$ for some $\tilde{\beta}_2\in\mathbb{R}, \tilde{\eta}_Y\sim N(0, \sigma^2)$. 
\end{example}

We require a slightly stronger notion of identifiability of $\mathcal{G}$ that we call ``pairwise identifiability''. 

\begin{definition}
Let $(X_0, \textbf{X})\in\mathbb{R}\times \mathbb{R}^p$ follow a SCM (\ref{definition_general_SCM}) with DAG $\mathcal{G}_0$. Let $\mathcal{F}$ be a subset of all measurable functions. We say that the  $\mathcal{F}$-model is \textbf{identifiable}, if there does not exist a graph $\mathcal{G}'\neq \mathcal{G}_0$ and functions $f_i'\in\mathcal{F}, i=0, \dots, p$ generating the same joint distribution. 

We say that the $\mathcal{F}$-model is \textbf{pairwise identifiable}, if for all $i,j\in\mathcal{G}_0, i\in pa_j$ hold the following: $\forall S\subseteq V$ such that  $pa_j\setminus \{i\}\subseteq S \subseteq nd_j\setminus\{i,j\}$ there exist $\textbf{x}_{S}: p_S(\textbf{x}_S)>0$ satisfying that a bivariate model defined as $Z_1=\tilde{\varepsilon}_1, Z_2 = \tilde{f}(Z_1, {\varepsilon}_j)$ is identifiable, where $P_{\tilde{\varepsilon}_1} = P_{X_i\mid \textbf{X}_{S} =\textbf{ x}_S}$, $\tilde{f}(x, \varepsilon) = f(\textbf{x}_{pa_j\setminus\{i\}}, x, \varepsilon)$, $\tilde{\varepsilon}_1\indep {\varepsilon_j}$.  
\end{definition}

In the bivariate case, the notion of identifiability and pairwise identifiability trivially coincides. Note the following observation. 

\begin{lemma}\label{pairwise_implies_global}
Pairwise identifiable $\mathcal{F}$-model is identifiable.  
\end{lemma}
 
The proof can be found in  \hyperref[Proof of pairwise_implies_global]{Appendix} \ref{Proof of pairwise_implies_global}. The following proposition is a counterpart of Proposition~\ref{TheoremFidentifiabilityWithChild}
 from Section~\ref{section_general_implies_local_identifiability} with general $\mathcal{F}$. 


\begin{proposition}
\label{proposition_for_pairwise_F_model_identifiability}
Let $(X_0, \textbf{X})$ follow a pairwise identifiable $\mathcal{F}$-model with DAG $\mathcal{G}$, such that all $\textbf{X}$ are neighbors of $X_0$ in $\mathcal{G}$.   Let $S \subseteq \{1, \dots, p\}$ contain a child of $X_0$ in $\mathcal{G}$.  Then, $S$ is not $\mathcal{F}$-plausible. 
\end{proposition}
 The proof can be found in  \hyperref[Proof of proposition_for_pairwise_F_model_identifiability]{Appendix} \ref{Proof of proposition_for_pairwise_F_model_identifiability}. 

\subsection{ $\mathcal{F}$-plausibility under restricted support        }
\label{Appendix_support}
 
The following proposition discusses a case when $\mathcal{F}$-implausibility results from restricting the support of $Y$ by conditioning on the child of $Y$. This result is specific for a location-scale space of functions $\mathcal{F}_{LS}$, but can be easily modified for other types of $\mathcal{F}$. 

\begin{proposition}[Assuming bounded support]
\label{Support_proposition}
Let $(Y, \textbf{X})\in\mathbb{R}\times \mathbb{R}^p$ follow an SCM with DAG $\mathcal{G}_0$. Let $S\subseteq\{1, \dots, p\}$ be a non-empty set. Let  $\underline{\Psi},\overline{\Psi}: \mathbb{R}^{\mid S\mid}\to \mathbb{R}$ be real functions such that
\begin{equation*} 
supp(Y\mid \textbf{X}_S=\textbf{x}) = \big(\underline{\Psi} (\textbf{x}),\overline{\Psi}(\textbf{x})\big), \,\,\,\,\,\,\, \forall \textbf{x}\in supp(\textbf{X}_S).
\end{equation*}
If
\begin{equation} \label{eq9987}
\frac{Y - \underline{\Psi}(\textbf{X}_S)}{\overline{\Psi}(\textbf{X}_S) - \underline{\Psi}(\textbf{X}_S)}\not\indep \textbf{X}_S,
\end{equation}
then $S$ is not $\mathcal{F}_{LS}$-plausible. 
\end{proposition}
The proof can be found in \hyperref[Proof of Support_proposition]{Appendix} \ref{Proof of Support_proposition}. Proposition \ref{Support_proposition} can be expressed as follows. If the support of $Y$ given $\textbf{X}_S = \textbf{x}_S$ is bounded, then $S$ can be  $\mathcal{F}_{LS}$-plausible only in a very specific case when (\ref{eq9987}) does not hold. 
Typically, (\ref{eq9987}) holds if $S$ contains a child of $Y$. 
\begin{example}\label{Example_o_Supporte}
Consider SCM where $Y$ is a parent of $X_1$ and $X_1 = Y + \eta$, where $Y\indep \eta$. Assume that  $Y, \eta$ are non-negative ($supp(Y) = supp(\eta) = (0, \infty)$). 
Then, $\underline{\Psi}(x)=0$ and  $\overline{\Psi}({x})=x$, since the support of $[Y\mid Y+\eta=x]$ is $(0,x)$. Hence, (\ref{eq9987}) reduces to $\frac{Y}{X_1} \not\indep X_1$. If $\frac{Y}{X_1} \not\indep X_1$, then $S=\{1\}$ is not $\mathcal{F}_{LS}$-plausible. 

How strong is the assumption $\frac{Y}{X_1} \not\indep X_1$? We claim that it holds in typical situations. A notable exception when  $\frac{Y}{X_1} \indep X_1$ holds is when $Y, \eta$ have Gamma distributions with equal scales. 
\end{example}

Proposition \ref{Support_proposition} is applicable only when $S$ contains a child of $Y$. If $S\subseteq pa_Y$, then (\ref{eq9987}) typically does not hold, as the following example illustrates.  
\begin{example}
Consider a bivariate SCM with $X_1\to Y$. Let $Y = X_1 + \eta$, where $X_1\indep \eta$. Assume that $supp(X) = supp(\eta) = (0, 1)$. Then, $\underline{\Psi}(x)=x$ and  $\overline{\Psi}({x})=1+x$. Hence, (\ref{eq9987}) reduces to $Y-X_1 \not\indep X_1$, which is not satisfied, so Proposition \ref{Support_proposition} is not applicable. 
\end{example}
Proposition \ref{Support_proposition} can be also stated for a case when  $\overline{\Psi}({x})=\infty$. In that case, we require stronger assumptions; we replace assumption (\ref{eq9987}) with  $Y - \underline{\Psi}(\textbf{X}_S)\not\indep \textbf{X}_S$ and replace $\mathcal{F}_{LS}$ with $\mathcal{F}_A$ (more restricted set where the scale is fixed). 




\subsection{Theorem~\ref{Theorem_in_section2} restated for location-scale models $\mathcal{F}_{LS}$ and CPCM($F$) models $\mathcal{F}_F$}
\label{Appendix_location_scale_definition}

We restate similar results to Theorem~\ref{Theorem_in_section2} for $\mathcal{F} = \mathcal{F}_{LS}$ and $\mathcal{F}_F$ functional classes. We start with the location-scale result. 

\begin{proposition}[Location-scale]\label{LemmaOLocationScaleinseparabilite}
 Consider $(Y, \textbf{X})\in\mathbb{R}\times \mathbb{R}^p$ satisfying ( \ref{SCM_for_Y}) with  $pa_Y\neq \emptyset$ and $\mathcal{F} = \mathcal{F}_{LS}$. 
 Let $\textbf{X}_{pa_Y}$ have full support and independent components. 

 Let $f_Y\in\mathcal{F}_{LS}$  have the form $f_Y(\textbf{x}, \varepsilon)=\mu(\textbf{x}) + \sigma(\textbf{x})\varepsilon$, where $\theta(\textbf{x}) = \big(\mu(\textbf{x}), \sigma(\textbf{x})\big)^\top$ is additive in both components, that is,  
$\mu(\textbf{x}) = h_{1, \mu}(x_1)+\dots + h_{k, \mu}(x_k)$ and 
$\sigma(\textbf{x}) =h_{1, \sigma}(x_1)+\dots + h_{k, \sigma}(x_k)$ for some continuous non-constant non-zero functions $h_{i,\cdot}$, where we also assume $h_{i,\sigma}>0$, $i=1, \dots, k$.  Then, then every $S\subsetneq pa_Y$ is not $\mathcal{F}_{LS}$-plausible. 
\end{proposition}
Proof can be found in \hyperref[Proof of LemmaOLocationScaleinseparabilite]{Appendix} \ref{Proof of LemmaOLocationScaleinseparabilite}. 

Now we focus on the case $\mathcal{F} = \mathcal{F}_{F}$, where a distribution function $F$ has $q\in\mathbb{N}$ parameters $\theta = (\theta_1, \dots, \theta_q)^\top$. We restrict to such $F$ satisfying the following definition. 


\begin{definition}\label{DefLS}
Let $F$ be a distribution function with one ($q=1$) parameter $\theta$. We say that the \textbf{parameter acts additively} in $F$, if an invertible real function $f_2$ and a function $f_1\in\mathcal{I}_m$ exist such that for all $\theta_1, \theta_2$ holds \footnote{Notation $F_{\theta_1}\big(F^{-1}_{\theta_2}(z)\big)$ is equivalent to $F(F^{-1}(z, \theta_2), \theta_1)$. We believe that this improves the readability.  } 
\begin{equation}\label{postAdditiveDefinition}
F_{\theta_1}\big(F^{-1}_{\theta_2}(z)\big) = f_1\big(z, f_2(\theta_1) + \theta_2\big), \,\,\,\forall z\in(0,1). 
\end{equation}

We say that the \textbf{parameter acts  multiplicatively} in $F$ if an invertible real function $f_2$ and a function $f_1\in\mathcal{I}_m$ exist such that for all $\theta_1, \theta_2$ holds  \begin{equation}\label{postMultiplDefinition}
F_{\theta_1}\big(F^{-1}_{\theta_2}(z)\big) = f_1\big(z, f_2(\theta_1) \cdot\theta_2\big), \,\,\,\forall z\in(0,1).
\end{equation}

Let $F$ be a distribution function with two ($q=2$) parameters $\theta = (\mu, \sigma)^\top\in\mathbb{R}\times \mathbb{R}_+$. We say that $F$ is a \textbf{Location-Scale} distribution, if for all $\theta$ holds  
\begin{equation*}
F_{\theta}\left( \frac{x-\mu}{\sigma}\right) = F_{\theta_0}(x),\,\,\,\,\forall x\in\mathbb{R},
\end{equation*} 
where $F_{\theta_0}$ is called standard distribution and corresponds to a parameter $\theta_0 = (0,1)^\top$.  
\end{definition}
Examples of $F$ whose parameter acts  additively include a Gaussian distribution with fixed variance or a Logistic distribution/Gumbel distribution with fixed scales.  Note that typically, $f_2(x) = -x$,  since $F_{\theta_1}\big(F^{-1}_{\theta_1}(z)\big)=z$ needs to hold.

Examples of $F$ whose parameter acts  multiplicatively include a Gaussian distribution with the fixed expectation or a Pareto distribution (where $F_{\theta_1}\big(F^{-1}_{\theta_2}(z)\big) = z^{\frac{\theta_1}{\theta_2}}= f_1\big(z, f_2(\theta_1) \cdot\theta_2\big)$ for $f_1(z,x) = z^{-1/x}$ and $f_2(x)=-1/x$). Functions $ f_1, f_2$ are not necessarily uniquely defined. 

Examples of Location-Scale types of distributions include Gaussian distribution, logistic distribution, or Cauchy distribution, among many others.







\begin{proposition}\label{LemmaOParetoinseparabilite}
 Consider $(Y, \textbf{X})\in\mathbb{R}\times \mathbb{R}^p$ satisfying ( \ref{SCM_for_Y}) with  $pa_Y\neq \emptyset$ and $\mathcal{F} = \mathcal{F}_{F}$. where $F$ be a distribution function whose parameter acts multiplicatively. 
 Let $\textbf{X}_{pa_Y}$ be a continuous random vector with full support and independent components. 
\begin{itemize}
\item Consider $f_Y\in\mathcal{F}_F$ in the form $f_Y(\textbf{x}, \varepsilon)=F^{-1}\big(\varepsilon, \theta(\textbf{x})\big)$ with additive function $\theta(x_1, \dots, x_k) = h_1(x_1)+\dots + h_k(x_k)$, where $h_i$ are continuous non-constant real functions. Then, then every $S\subsetneq pa_Y$ is not $\mathcal{F}_{F}$-plausible. 
\item Consider $f_Y\in\mathcal{F}_F$ in the form $f_Y(\textbf{x}, \varepsilon)=F^{-1}\big(\varepsilon, \theta(\textbf{x})\big)$ with multiplicative function   $\theta(x_1, \dots, x_k) = h_1(\textbf{x}_S)\cdot h_2(\textbf{x}_{\{1, \dots, k\}\setminus S})$ for some $S\subsetneq \{1, \dots, k\}$, where $h_1, h_2$ are continuous non-constant non-zero real functions. Then, $S_{\mathcal{F}_F}(Y)=\emptyset$.
\end{itemize}
\end{proposition}
Proof can be found in \hyperref[Proof of LemmaOParetoinseparabilite]{Appendix} \ref{Proof of LemmaOParetoinseparabilite}. 
Analogous Proposition~\ref{LemmaOParetoinseparabilite} can be stated for $F$ being additive or location-scale type, where Lemma \ref{CoolLemma} part 3 and 4 would be used instead of part 2. 


Finally, we present the modification of Theorem~\ref{Theorem_in_section2} for (possibly dependent) Gaussian parents. 
\begin{lemma}[Gaussian case]\label{lemma_Gaussian_parents}
 Consider $(Y, \textbf{X})\in\mathbb{R}\times \mathbb{R}^p$ satisfying ( \ref{SCM_for_Y}) with  $pa_Y\neq \emptyset$ and $\mathcal{F} = \mathcal{F}_A$. 
 Let $\textbf{X}_{pa_Y}$ have Gaussian distribution. If $f_Y\in\mathcal{F}_A$ is continuous injective additively transitive \footnote{A function is said to be \textbf{additively transitive}, if $f(x, y+z) = f(x, y) + f(x, z)$. For example, $f(x, y) = xy$ is additively transitive.} function that cannot be written as (\ref{efaetgfas}), then every $S\subsetneq pa_Y$ is not $\mathcal{F}_A$-plausible. In particular, if $pa_Y = \{1, \dots, p\}$ then $S_{\mathcal{F}_A}(Y) = pa_Y$. 
\end{lemma}

\begin{proof}
    The proof is analogous to the proof of Theorem~\ref{Theorem_in_section2}, where we use Lemma~\ref{CoolLemma} part~5 instead of part~1 in the last step. 
\end{proof}



\section{Simulations and application}
\label{section_appendix_simulations}

Appendix~\ref{Section_Additive} offers an illustrative  simulations to demonstrate the theoretical findings discussed in Section \ref{Section_3}. Appendix~\ref{Section_benchmarks} pertains to the evaluation of the algorithm's performance on three benchmark datasets.

To randomly generate a $d$-dimensional function, we use the concept of the Perlin noise generator \citep{PerlinNoise}. Examples of such generated functions can be found in Appendix~\ref{Appendix_D1}. The two algorithms presented in Section~\ref{section_algorithm}, all simulations, and the Perlin noise generator are coded in the programming language \texttt{R} \citep{Rstudio} and can be found in the supplementary package or at \url{https://github.com/jurobodik/Structural-restrictions-in-local-causal-discovery.git}.


\subsection{Highlighting the results from Section \ref{Section_3}}
\label{Section_Additive}

Consider a target variable $Y$ with two parents $X_1, X_2$, where $\textbf{X} = (X_1, X_2)$ is a centered normal random vector with correlation $c\in\mathbb{R}$. The generation process of $Y$ is as follows: \begin{equation}\label{eq576}
    Y = g_1(X_1)+g_2(X_2) + \gamma\cdot g_{1,2}(X_1, X_2)+\eta,\,\,\,\,\,\,\text{ with }\eta\sim N(0,1)\,\text{ and  }\gamma\in\mathbb{R},
\end{equation}
 where $g_1, g_2, g_{1,2}$ are fixed functions generated using the Perlin noise approach.

Theorem~\ref{Theorem_in_section2} suggests that if $c= \gamma= 0 $ then we should find that  $S_{\mathcal{F}}(Y) = \emptyset$. Moreover, if  $c\in\mathbb{R},$ and $\gamma\neq 0 $, then  $S_{\mathcal{F}}(Y) = pa_Y = \{1,2\}$. Moreover, the choice of $c$ can affect the finite sample properties. 

Figure~\ref{Plot_sim_additive} confirms these results. For a range of parameters $c\in[0,0.9], \gamma\in[0,1]$, we generate 50 times such a random dataset of size $n=500$ 
and estimate $S_{\mathcal{F}}(Y)$ using the ISD algorithm. If $\gamma$ is small, we discover direct causes of $Y$ only in a small number of cases. However, the larger the $\gamma$, the larger the number of discovered parents. Figure~\ref{Plot_sim_additive} also suggests that the correlation between the parents can actually be beneficial. The reason is that even if (\ref{eq576}) is additive in each component, the correlation between the components can create a bias in estimating $g_1$ (resp. $g_2$). This results in a dependency between the residuals and $X_1$ (resp. $X_2$) in the model where we regress $Y$ on $X_1$ (or on $X_2$), and we are more likely to reject the plausibility of $S = \{1\}$ (or $S = \{2\}$). 

% Figure environment removed



\subsection{Benchmarks}\label{Section_benchmarks}
We created three benchmark datasets to assess the performance of our methodology. Two of them correspond to additive noise models ($\mathcal{F}=\mathcal{F}_A$), and the third to $\mathcal{F}=\mathcal{F}_F$ with the Pareto distribution $F$.

The first benchmark dataset consists of $\textbf{X} = (X_1, X_2, X_3, X_4)^\top$ and the response variable $Y$, where $pa_Y = \{1\}$ with the corresponding graph drawn in Figure~\ref{Figure_DAG_for_sim1}A. The data-generation process is as follows: 
\begin{equation*}
 X_1 =\eta_1, \,\,\,Y = g_Y(X_1)+\eta_Y, \,\,\,X_i = g_i(Y,\eta_i),\,\,\, i=2,3,4,     
\end{equation*}
where $g_Y, g_2, g_3, g_4$ are fixed functions generated using the Perlin noise approach, $\eta_1, \eta_2, \eta_3, \eta_4$ are correlated uniformly distributed noise variables, and $\eta_Y\sim N(0,1)$ is independent of $\eta_1, \dots, \eta_4$. The challenge is to find the (one) direct cause among all variables. 
 
The second benchmark dataset consists of $\textbf{X} = (X_1, X_2, X_3)^\top$ and the response variable $Y$, where $pa_Y = \{1,2,3\}$ with the corresponding graph drawn in Figure~\ref{Figure_DAG_for_sim1}B. The data-generation process is as follows:
\begin{align*}
   \begin{pmatrix}
          X_1 \\
            X_2 \\
            X_3
         \end{pmatrix} &\sim N\bigg(\begin{pmatrix}
           0 \\
            0 \\
            0
         \end{pmatrix}, 
         \begin{pmatrix}
          1 ,   c , c \\
            c,   1, c\\
            c,c,1
         \end{pmatrix}\bigg),\,\,\,\, Y = g_Y(X_1, X_2, X_3) + \eta_Y, \,\,\,\text{ where }\eta_Y\overset{}{\sim} N(0,1),
 \end{align*}
for $c=0.5$ and a fixed function $g_Y$ generated using the Perlin noise approach. The challenge is to estimate as many direct causes of $Y$ as possible. 

The third benchmark dataset consists of $\textbf{X} = (X_1, X_2, X_3)^\top$ and the response variable $Y$ corresponding to the DAG C of Figure~\ref{Figure_DAG_for_sim1}. Here, every edge is randomly oriented; either  $\to$ or $\leftarrow$ with probability $\frac{1}{2}$. The source variables (variables without parents) are generated following the standard Gaussian distribution. $Y$ is generated as (\ref{CPCM_def}) with the Pareto distribution $F$ with a fixed function $\theta(\textbf{X}_{pa_Y})$ generated using the Perlin noise approach. Finally, if $X_i$ is the effect of $Y$, it is generated as $X_i = f_i(Y, \eta_i)$, where $\eta_i\sim U(0,1)$,  $ \eta_i\indep Y$ and $f_i$ is a fixed function generated as a combination of functions generated using the Perlin noise approach. 


In all datasets, we consider a sample size of $n=500$. 
% Figure environment removed




We compare our proposed algorithms with specific methods for causal discovery, which are: RESIT \citep{Peters2014}, CAM-UV \citep{maeda2021causal}, pairwise bQCD \citep{Natasa_Tagasovska}, pairwise IGCI with the Gaussian reference measure \citep{IGCI}, and pairwise slope \citep{Slope}. When we use the term ``pairwise,'' we are referring to orienting each edge between $(X_i, Y)$ separately, $i=1, \dots, p$. 

For evaluating the performance, we simulate 100 repetitions of each of the three benchmark datasets and use two metrics: ``percentage of discovered correct causes'' and ``percentage of no false positives'' which measures the percentage of cases with no incorrect variable in the set of estimated causes. As an example, consider $pa_Y = \{1,2,3\}$. If we estimate $\widehat{pa}_Y = \{1,2\}$ in $80\%$ of cases and  $\widehat{pa}_Y = \{1,4,5\}$ in $20\%$ of cases, the percentage of discovered correct causes is $\frac{2}{3}\frac{8}{10} + \frac{1}{3}\frac{2}{10} \approx 60\%$ and the percentage of no false positives is $80\%$. 

Table \ref{Table_results} shows the performance of all methodologies. As shown, our two algorithms outperform the other approaches by a significant margin. The IDE algorithm never includes a wrong covariate in the set of causes. On the other hand, although the scoring algorithm demonstrates better overall performance and power, it tends to include non-causal variables more frequently.




\subsection{Visualization of benchmark datasets}
\label{Appendix_Simulations}

\subsubsection{Functions generated using the Perlin noise approach}\label{Appendix_D1}
In the following, we provide examples of functions generated using the Perlin noise approach. For a one-dimensional case, let $X_1, \eta_Y\overset{iid}{\sim} N(0,1)$ and $Y = g(X_1)+\eta_Y$, where $g$ is generated using the Perlin noise approach. Such (typical) datasets are plotted in Figure~\ref{Figure_perlin_1}. 

For a two-dimensional case, let $X_1,X_2, \eta_Y\overset{iid}{\sim} N(0,1)$ and $Y = g(X_1, X_2)+\eta_Y$, where $g$ is generated using the Perlin noise approach. Such (typical) datasets are plotted in Figure~\ref{Figure_perlin_2}. 

For a three-dimensional case, let $X_1,X_2,X_3, \eta_Y\overset{iid}{\sim} N(0,1)$ and $Y = g(X_1, X_2,X_3)+\eta_Y$, where $g$ is generated using the Perlin noise approach. The visualisation of a four-dimensional dataset is a bit tricky; Figure~\ref{Figure_perlin_3} represents the three-dimensional slices of the function. 
% Figure environment removed






% Figure environment removed











% Figure environment removed









%\subsubsection{Application}\label{Appendix_Application}
%Figure~\ref{Figure_application_data_visualization} represents a visualization of the covariates from Section~\ref{section_simulations}. 

%% Figure environment removed

















\section{Auxiliary results}\label{Appendix_Auxiliary}
The median $med(X)$ is defined as $m\in\mathbb{R}$ such that   $P(X\leq m )\geq \frac{1}{2}\leq P(X\geq m )$. It always exists, and is unique for continuous random variables.
\begin{lemma}\label{distributionalequalitylemma}
Let $X$ be a non-degenerate continuous real random variable. Let $a,b\in\mathbb{R}$ such that 
\begin{equation}\label{qwerty}
a+bX\overset{D}{=}X.
\end{equation}
Then, either $(a,b) = (0,1)$ or $(a,b) = (2med(X), -1)$. 
\end{lemma}
\begin{proof}
\textit{Idea of the proof assuming a finite variance of $X$: } If $X$ has finite variance, then (\ref{qwerty}) implies $var(a+bX) = var(X)$, rewriting gives us  $b^2 var(X)=var(X)$, and hence, $b=\pm 1$. Now, (\ref{qwerty}) also implies $\mathbb{E}(a+bX) = \mathbb{E}(X)$, hence $a=(1-b)\mathbb{E}(X)$. Therefore, if $b=1$, then $a= 0$, and if $b=-1$, then $a=2\mathbb{E}(X)$. 

\textit{Proof without the moment assumption:}  (\ref{qwerty}) implies that for any $q\in (0.5,1)$, the difference between the $q$ quantile and $(1-q)$ quantile should be the same on both sides of (\ref{qwerty}). Denote $F^{-1}_X(q)$ a $q-$quantile of $X$ and assume that $F^{-1}_X(q)\neq F^{-1}_X(1-q)$ (since $X$ is non-degenerate, such $q$ exist). We get 
$$
F^{-1}_{a+bX}(q) - F^{-1}_{a+bX}(1-q) = F^{-1}_X(q)- F^{-1}_X(1-q)=:D.
$$
Consider $b\geq 0$. Using linearity of the quantile function, we obtain $a+bF^{-1}_X(q) - \big(a+bF^{-1}_X(1-q)\big) = D$ and hence, $bD=D$, which gives us $b=1$. If $b<0$, then an identity $F^{-1}_{a+bX}(q) = a+\big(1-F^{-1}_{-bX}(1-q)\big) = a+\big(1+bF^{-1}_{X}(1-q)\big)$ hold. Hence, we get  $a+[1+bF^{-1}_X(1-q)] - [a+\big(1+bF^{-1}_X(1-q)\big)] = D$. Rewriting the left side, we get $-bD=D$,  which gives us $b=-1$. 

In the case when $b=1$, trivially $a=0$, since otherwise, $med(a+X)\neq med(X)$. If $b=-1$, then applying median on both sides of (\ref{qwerty}) gives us $med(a-X)= med(X)$ and hence, $a=2med(X)$, as we wanted to show. 
\end{proof}


\begin{lemma}\label{CoolLemma}
Let $\textbf{X}=(X_1, \dots, X_k)$ be a continuous random vector with independent components and $s<k$.  Let $h_1, \dots, h_k$ be continuous non-constant real functions. 
\begin{enumerate}
\item A non-zero function $f$ does not exist such that 
\begin{equation}\label{yui}
  f(X_1, \dots, X_s)\big(h_1(X_1)+\dots+h_k(X_k)\big)\indep (X_1, \dots, X_s).
\end{equation}
\item Moreover, let $h_1, \dots, h_k$ be non-zero. Then, a non-zero function $f$ does not exist such that 
\begin{equation}\label{yuiDVA}
  f(X_1, \dots, X_s) + h_1(X_1)h_2(X_2)\dots h_k(X_k)\indep (X_1, \dots, X_s).
\end{equation}
\item Let $h:\mathbb{R}^{k-s}\to\mathbb{R}$ be measurable function such that $h(X_{s+1}, \dots, X_k)$ is non-degenerate continuous random variable. Functions $f_1, f_2$ does not exist, such that $f_2$ is positive non-constant and
\begin{equation}\label{yuiTRI}
f_1(X_1, \dots, X_s) + f_2(X_1, \dots, X_s)h(X_{s+1}, \dots, X_k)\indep (X_1, \dots, X_s).
\end{equation}
\item Let $\textbf{X}=(X_1, \dots, X_k)$ be non-degenerate Gaussian random vector (possibly with dependent components) and $\beta = (\beta_{s+1}, \dots, \beta_{k})^\top\in \mathbb{R}^{k-s}$ . Functions $f_1, f_2$ does not exist, such that $f_2$ is positive non-constant and
\begin{equation}\label{yuiSTYRI}
f_1(X_1, \dots, X_s) + f_2(X_1, \dots, X_s)(\beta_{s+1}X_{s+1}+ \dots+ \beta_{k}X_k)\indep (X_1, \dots, X_s).
\end{equation}

\end{enumerate}
\end{lemma}
\begin{proof}
Let us introduce functionals (not norms, we only use them to simplify notation) $||\cdot||_{plus}$ and  $||\cdot||_{times}$ , defined by $|| \textbf{a} ||_{plus} = a_1 +\dots + a_d$ , $|| \textbf{a} ||_{times} = a_1 a_2\dots a_d$ , for $\textbf{a}=(a_1, \dots, a_d)^\top\in\mathbb{R}^d$. 
We use notation $\textbf{X}_S=(X_1, \dots, X_s)^\top$ and denote a function $h_S: \mathbb{R}^s\to\mathbb{R}^s: h_S(\textbf{x})= (h_1(x_1), \dots, h_s(x_s))^\top$. 

\textbf{Part 1: }For a contradiction, let such $f$ exist. First, some notation: Let $Y=h_{s+1}(X_{s+1})+\dots+h_{k}(X_k)$ and define $\xi:=f(\textbf{X}_S)(||h_S(\textbf{X}_S)||_{plus}+Y)$, which is the left hand side of (\ref{yui}). 

Choose $\textbf{a}, \textbf{b}, \textbf{c}\in\mathbb{R}^s$ in the support of $\textbf{X}_S$ such that $||h_S(\textbf{a})||_{plus}, ||h_S(\textbf{b})||_{plus}, ||h_S(\textbf{c})||_{plus}$ are distinct and $f(\textbf{b})\neq 0$ (it is possible since $h_i$ are non-constant). 

Since $\xi\indep \textbf{X}_S$, then  $\xi\mid [\textbf{X}_S=\textbf{a}] \overset{D}{=}\xi\mid [\textbf{X}_S=\textbf{b}] \overset{D}{=}\xi\mid [\textbf{X}_S=\textbf{c}]$. Hence, 
\begin{equation}\label{asdfgh}
f(\textbf{a})(||h_S(\textbf{a})||_{plus}+Y)\overset{D}{=}f(\textbf{b})(||h_S(\textbf{b})||_{plus}+Y)\overset{D}{=}f(\textbf{c})(||h_S(\textbf{c})||_{plus}+Y).
\end{equation}
By dividing by a non-zero constant $f(\textbf{b})$ and subtracting a constant $||h_S(\textbf{b})||_{plus}$, we get
$$
\frac{f(\textbf{a})}{f(\textbf{b})}||h_S(\textbf{a})||_{plus}-||h_S(\textbf{b})||_{plus}+\frac{f(\textbf{a})}{f(\textbf{b})}Y\overset{D}{=}Y\overset{D}{=}\frac{f(\textbf{c})}{f(\textbf{b})}||h_S(\textbf{c})||_{plus}-||h_S(\textbf{b})||_{plus}+\frac{f(\textbf{c})}{f(\textbf{b})}Y.
$$
Now we use lemma \ref{distributionalequalitylemma}. It gives us that $\frac{f(\textbf{a})}{f(\textbf{b})}=\pm 1$ and also  $\frac{f(\textbf{c})}{f(\textbf{b})}=\pm 1$. Therefore, at least two values of $f(\textbf{a}), f(\textbf{b}), f(\textbf{c})$ must be equal (and neither of them are zero). WLOG $f(\textbf{a})= f(\textbf{c})$. Plugging this into equation (\ref{asdfgh}), we get $||h_S(\textbf{a})||_{plus}=||h_S(\textbf{c})||_{plus}$, which is a contradiction since we chose them to be distinct. 

\textbf{Part 2: } We proceed in a similar way to the previous part. For a contradiction, let such $f$ exist. First, some notation: let $Y=h_{s+1}(X_{s+1})\dots h_{k}(X_k)$ and define $\xi:=f(\textbf{X}_S) + (||h_S(\textbf{X}_S)||_{times} \cdot Y)$, which is the left hand side of (\ref{yuiDVA}). 

Choose $\textbf{a}, \textbf{b}, \textbf{c}\in\mathbb{R}^s$ in the support of $\textbf{X}_S$ such that $||h_S(\textbf{a})||_{times}, ||h_S(\textbf{b})||_{times}, ||h_S(\textbf{c})||_{times}$ are distinct and $f(\textbf{b})\neq 0$. 

Since $\xi\indep \textbf{X}_S$, then  $\xi\mid [\textbf{X}_S=\textbf{a}] \overset{D}{=}\xi\mid [\textbf{X}_S=\textbf{b}] \overset{D}{=}\xi\mid [\textbf{X}_S=\textbf{c}]$. Hence, 


\begin{equation}\label{asdfghDVA}
f(\textbf{a}) + ||h_S(\textbf{a})||_{times}\cdot Y\overset{D}{=}f(\textbf{b})+||h_S(\textbf{b})||_{times}\cdot Y\overset{D}{=}f(\textbf{c}) + ||h_S(\textbf{c})||_{times}\cdot Y.
\end{equation}
By dividing by a non-zero constant $f(\textbf{b})$ and subtracting a constant $||h_S(\textbf{b})||_{times}$, we get
$$
\frac{f(\textbf{a})}{f(\textbf{b})}||h_S(\textbf{a})||_{times}-||h_S(\textbf{b})||_{times}+\frac{f(\textbf{a})}{f(\textbf{b})}Y\overset{D}{=}Y\overset{D}{=}\frac{f(\textbf{c})}{f(\textbf{b})}||h_S(\textbf{c})||_{times}-||h_S(\textbf{b})||_{times}+\frac{f(\textbf{c})}{f(\textbf{b})}Y.
$$
Now we use lemma \ref{distributionalequalitylemma}. It gives us that $\frac{f(\textbf{a})}{f(\textbf{b})}=\pm 1$ and also  $\frac{f(\textbf{c})}{f(\textbf{b})}=\pm 1$. Therefore, at least two values of $f(\textbf{a}), f(\textbf{b}), f(\textbf{c})$ must be equal (and neither of them are zero). WLOG $f(\textbf{a})= f(\textbf{b})$. Plugging this into equation (\ref{asdfghDVA}), we get $||h_S(\textbf{a})||_{times}=||h_S(\textbf{b})||_{times}$, which is a contradiction since we chose them to be  distinct. 

\textbf{Part 3: } For a contradiction, let $f_1, f_2$ exist. Denote $Y = h(X_{s+1}, \dots, X_k)$. Choose $\textbf{a}, \textbf{b}\in\mathbb{R}^s$ in the support of $\textbf{X}_S$ such that $f_2(\textbf{a})\neq f_2(\textbf{b})\neq 0$. From (\ref{yuiTRI}), we get $f_1(\textbf{a}) + f_2(\textbf{a})Y \overset{D}{=}f_1(\textbf{b}) + f_2(\textbf{b})Y$. By rewriting, we get $\frac{f_1(\textbf{a})-f_1(\textbf{b})}{f_2(\textbf{b})} + \frac{f_2(\textbf{a})}{f_2(\textbf{b})}Y \overset{D}{=}Y$. Applying Lemma  \ref{distributionalequalitylemma}, we obtain $\frac{f_2(\textbf{a})}{f_2(\textbf{b})}=\pm 1$. Since $f_2$ is positive, we get $f_2(\textbf{a}) = f_2(\textbf{b})$. This is a contradiction. 


\textbf{Part 4: } We use the following well-known result. For a multivariate normal vector  \begin{align*}
    \textbf{Z}=(\textbf{Z}_1, \textbf{Z}_2)^\top &\sim N\bigg(\begin{pmatrix}
           \mu_1 \\
            \mu_2 
         \end{pmatrix}, 
         \begin{pmatrix}
           \Sigma_{1,1} ,   \Sigma_{1,2} \\
             \Sigma_{2,1},   \Sigma_{2,2}
         \end{pmatrix}\bigg),
 \end{align*}
where $\textbf{Z}_1, \textbf{Z}_2$ is a partition of $\textbf{Z}$ into smaller sub-vectors, it holds that $(\textbf{Z}_1|\textbf{Z}_2=a)$, the conditional distribution of the first partition given the second, has distribution equal to $N(\mu_a, \tilde{\Sigma})$, where
$\mu_a = \mu_1 + \Sigma_{1,2}\Sigma_{2,2}^{-1}(a-\mu_2)$, $\tilde{\Sigma} = \Sigma_{1,1} - \Sigma_{1,2}\Sigma^{-1}_{2,2}\Sigma_{2,1}$. Specifically, the covariance structure does not depend on $a$. 

\textit{Proof of Part 4: }
 For a contradiction, let $f_1, f_2$ exist. Choose $\textbf{a}, \textbf{b}\in\mathbb{R}^s$ in the support of $\textbf{X}_S$ such that $f_2(\textbf{a})\neq f_2(\textbf{b})\neq 0$. Denote $Y = \beta_{s+1}X_{s+1}+ \dots+ \beta_{k}X_k$ and $Y_a := (Y\mid \textbf{X}_S=a)$ and $Y_b := (Y\mid \textbf{X}_S=b)$ (do not mistake the notation with a do-intervention; $Y_a$ is simply a conditional distribution of $Y$ given $\textbf{X}_S=a$). Notice that $Y$ is also Gaussian. 
 
 The well-known result gives us that $\mu_a + Y_a \overset{D}{=}\mu_b + Y_b$, where $\mu_a, \mu_b$ are constants depending on the mean and covariance structure of $\textbf{X}$, and on $a,b$. 
 
 From (\ref{yuiSTYRI}), we get $$f_1(\textbf{a}) + f_2(\textbf{a})Y_a \overset{D}{=}f_1(\textbf{b}) + f_2(\textbf{b})Y_b.$$
 
 
 By rewriting, we get $\frac{f_1(\textbf{a})-f_1(\textbf{b}) + f_2(a)(\mu_a - \mu_b)}{f_2(\textbf{b})} + \frac{f_2(\textbf{a})}{f_2(\textbf{b})}Y_b \overset{D}{=}Y_b$. Applying Lemma  \ref{distributionalequalitylemma}, we obtain $\frac{f_2(\textbf{a})}{f_2(\textbf{b})}=\pm 1$. Since $f_2$ is positive, we get $f_2(\textbf{a}) = f_2(\textbf{b})$. This is a contradiction. 
 
 
\end{proof}


\begin{lemma}
Let $X,Y$ be continuous random variables and $f$ is a (non-random) strictly increasing function. Then, 
\begin{equation}\label{trivial_identity}
X\indep Y \iff f(X)\indep Y. \tag{\ding{95}} 
\end{equation}
\end{lemma}
\begin{proof}
This statement is trivial. 
\end{proof}





\section{Proofs}
\label{Section_proofs}

\begin{customprop}{
\ref{TheoremFidentifiabilityWithChild}}
Let $(X_0, \textbf{X})$ follow an (identifiable) restricted $\mathcal{F}_A$-model with DAG $\mathcal{G}$, such that all $\textbf{X}$ are neighbors of $X_0$ in $\mathcal{G}$.   Let $S \subseteq \{1, \dots, p\}$ contain a child of $X_0$ in $\mathcal{G}$.  Then, $S$ is not $\mathcal{F}_A$-plausible. 
\end{customprop}

\begin{proof}
 \label{proof of TheoremFidentifiabilityWithChild} 
    For a contradiction, let $S$ be $\mathcal{F}_A$-plausible. Without loss of generality, let $X_1$ satisfy 
    
    \begin{enumerate}
        \item $1\in S$,
        \item $X_1$ is a child of $X_0$,
        \item $X_1$ is childless.
    \end{enumerate}
    
We explain why this is without loss of generality.  By assumptions, the set $\tilde{S}:=\{i\in S : \,X_i \text{ is a child of }X_0\}$ is non-empty. Moreover, there must exist $j\in \{1, \dots, p\}$ that is a childless child of $X_0$ in $\mathcal{G}$ (otherwise there would be a cycle). If $j\in\tilde{S}$ then $X_j$ satisfies all conditions 1, 2 and 3. Without loss of generality, we rename $j=1$. 
If  $j\not\in\tilde{S}$, then $(X_0, \textbf{X}_{\{1, \dots, p\}\setminus j})$ also follows a restricted $\mathcal{F}_A$-model and since $j\not\in S$, we can focus only on proving Proposition~\ref{TheoremFidentifiabilityWithChild} restricted to variables  $(X_0,  \textbf{X}_{\{1, \dots, p\}\setminus j})$. Iteratively, after a finite number of steps, we find that restricted $\mathcal{F}_A$-model  $(X_0, \textbf{X}_{\{1, \dots, p\} \setminus \{j_1, \dots, j_k\}})$ satisfies $j\in\tilde{S}$. 


The idea of the proof is that we define two bivariate $\mathcal{F}_A$-models, one with $X_0\to X_1$ and one with $X_1\to X_0$, which will lead to a contradiction with the identifiability of the original restricted $\mathcal{F}_A$-model. 

Since $(X_0, \textbf{X})$ follow an $\mathcal{F}_A$-model, we can write  $X_i = f_i(\textbf{X}_{pa_i}) + \eta_i$, where $f_i$ are some measurable functions and $\eta_i$ are jointly independent, $i\in \{0, \dots, p\}$. Specifically, we have 
\begin{equation*}
    \begin{split}
        X_0 = f_0(\textbf{X}_{pa_0}) + \eta_0, \,\,\,\,\,\,\,X_1 = f_1(X_0, \textbf{X}_{pa_1\setminus\{1\}}) + \eta_1, 
    \end{split}
\end{equation*}
where $\eta_1 \indep \textbf{X}_{\{0, 2, 3, \dots, p\}}$ .Conditioning on $\textbf{X}_{\{ 2, 3, \dots, p\}} = \textbf{x}$, we obtain 
\begin{equation*}
    \begin{split}
\textbf{SCM\,1:} \,\,\,\,\,\,\,\,\,\,\,\,\,       X_0 = \tilde{\eta}_0, \,\,\,\,\,\,\,X_1 = f_1(X_0, \textbf{x}_{pa_1\setminus\{1\}}) + \eta_1, 
    \end{split}
\end{equation*}
where $\tilde{\eta}_0\sim X_0\mid\textbf{X}_{\{2, 3, \dots, p\}} = \textbf{x}$ and $\eta_1\indep X_0$. 


From the fact that $S$ is $\mathcal{F}_A$-plausible, we can find a function $f$ such that $\eta_S:=X_0 - f(\textbf{X}_S)$ satisfies $\eta_S\indep \textbf{X}_S$. Hence, we can write $$
X_0 =f(X_1, \textbf{X}_{S\setminus\{1\}}) + \eta_S,$$ where $\eta_S\indep \textbf{X}_S$.  Conditioning on  $\textbf{X}_{\{2, 3, \dots, p\}} = \textbf{x}$, we obtain
$$
\textbf{SCM\,2:} \,\,\,\,\,\,\,\,\,\,\,\,\,      X_1 = \tilde{\eta}_1, \,\,\,\,\,X_0 =f(X_1, \textbf{x}_{S\setminus\{1\}}) + \eta_S,$$ where $\tilde{\eta}_1\sim X_1\mid X_{\{2, 3, \dots, p\}} = \textbf{x}$  and $\eta_S\indep X_1$. 

Notice that in both Models 1 and 2, the joint distribution of $(X_0, X_1)$ is equal to $P_{X_0, X_1\mid (X_2, \dots, X_p)=\textbf{x}}$ and hence, we were able to find two additive noise models generating the same joint distribution, where the first model follows restricted additive noise model. This is a direct contradiction with the identifiability of restricted additive noise model (Theorem~\ref{theorem20}). Therefore, $S$ is not $\mathcal{F}_F$-plausible.
 
\end{proof}












 
 
\begin{customlem}{\ref{pairwise_implies_global}}
Pairwise identifiable $\mathcal{F}$-model defined in Appendix~\ref{Appendix_pairwise_identifiability} is identifiable.  
\end{customlem}

 
\begin{proof}\label{Proof of pairwise_implies_global}
For a contradiction, let there be two $\mathcal{F}$-models with causal graphs $\mathcal{G}\neq \mathcal{G}'$ that both generate the same joint distribution $P_{(X_0, \textbf{X})}$.   Using Proposition 29 in \cite{Peters2014}, variables $L,K\in \{X_0, \dots, X_p\}$ exist, such that 
\begin{itemize}
\item $K\to L$ in $\mathcal{G}$ and $L\to K$ in $\mathcal{G}'$,
\item $S:=\underbrace{\big\{pa_L(\mathcal{G})\setminus\{K\}\big\}}_\text{\textbf{Q}}\cup\underbrace{\big\{pa_K(\mathcal{G}')\setminus\{L\}\big\}}_\text{\textbf{R}}\subseteq \big\{nd_L(\mathcal{G}) \cap nd_K(\mathcal{G}')\setminus\{K,L\}\big\} $. 
\end{itemize}
For this $S$, choose $x_S$ according to the condition in the definition of pairwise identifiability. We use the notation $x_S=(x_q, x_r)$, where $q\in \textbf{Q}, r\in \textbf{R}$, and we define $K^\star := K\mid \{X_S=x_S\}$ and $L^\star := L\mid \{X_S=x_S\}$.  Now we use Lemma 36 and Lemma 37 from \cite{Peters2014}.  Since $K\to L$ in $\mathcal{G}$, we get $$K^\star=\tilde{\varepsilon}_{K^\star},\,\,\,\,\,\,\,\,\,\, L^\star = f_{L^\star}(K^\star, \varepsilon_L),$$ 
where $\tilde{\varepsilon}_{K^\star} = K\mid \{X_S=x_S\}$ and $\varepsilon_L\indep K^\star$. We obtained a bivariate $\mathcal{F}$-model with $K^\star\to L^\star$. However, the same holds for the other direction; from $L\to K$ in $\mathcal{G}'$, we get $$L^\star=\tilde{\varepsilon}_{L^\star},\,\,\,\,\,\,\,\,\,\, K^\star = f_{K^\star}(L^\star, \varepsilon_K),$$ 
 where $\tilde{\varepsilon}_{L^\star} =L\mid \{X_S=x_S\}$ and $\varepsilon_K\indep L^\star$. We obtained a bivariate $\mathcal{F}$-model with $L^\star\to K^\star$, which is a contradiction. 
\end{proof}



 
\begin{customprop}{\ref{proposition_for_pairwise_F_model_identifiability}}
Let $(X_0, \textbf{X})$ follow a pairwise identifiable $\mathcal{F}$-model with DAG $\mathcal{G}$, such that all $\textbf{X}$ are neighbors of $X_0$ in $\mathcal{G}$.   Let $S \subseteq \{1, \dots, p\}$ contain a child of $X_0$ in $\mathcal{G}$.  Then, $S$ is not $\mathcal{F}$-plausible. 
\end{customprop}

 
\begin{proof}\label{Proof of proposition_for_pairwise_F_model_identifiability}
    For a contradiction, let $S$ be $\mathcal{F}_A$-plausible. Without loss of generality, let $X_1$ be a childless child of $X_0$ such that $1\in S$ (the set of children of $X_0$ is nonempty by assumption, and one of them must be childless to avoid cycles). The idea of the proof is that we define two bivariate $\mathcal{F}$-models, one with $X_0\to X_1$ and one with $X_1\to X_0$, which will lead to a contradiction with the pairwise identifiability. 

  Since $(X_0, \textbf{X})$ follow an $\mathcal{F}$-model, we can write  $X_i = f_i(\textbf{X}_{pa_i(\mathcal{G})}, \varepsilon_i)$, where $f_i\in\mathcal{F}$ and $\varepsilon_i$ are jointly independent, $i\in \{0, 1, \dots, p\}$. We use the  pairwise identifiability condition. For a specific choice $(X_0, X_1)$ and $\tilde{S} = nd_1(\mathcal{G})\setminus\{0,1\} = \{2, \dots, p\}$ (the second equality holds since $X_1$ is a childless child),  $\textbf{x}_{\tilde{S}}: p_{\tilde{S}}(\textbf{x}_{\tilde{S}})>0$ exists, satisfying the condition that a bivariate $\mathcal{F}$-model defined as
\begin{equation}\label{tyuiop}
\tilde{X}_0=\tilde{\varepsilon}_0, \tilde{X}_1 = \tilde{f}_1(\tilde{X}_0, \tilde{\varepsilon}_1)
\end{equation}
is identifiable, where  $P_{\tilde{\varepsilon}_0} = P_{X_0\mid \textbf{X}_{\tilde{S}} = x_{\tilde{S}}}    $ and $\tilde{f}_1(x, \varepsilon) =f(\textbf{x}_{pa_1\setminus\{0\}}, x, \varepsilon)$, $\tilde{\varepsilon}_1\indep \tilde{\varepsilon}_0$ . 

From the fact that $S$ is $\mathcal{F}$-plausible, $f\in\mathcal{F}$ exists, such that $\varepsilon_S:=f^\leftarrow(\textbf{X}_S, X_0)$ satisfies $\varepsilon_S\indep \textbf{X}_S, \varepsilon_S\sim U(0,1)$. Hence, we can define a model  $$\tilde{\tilde{X}}_1 = \tilde{\tilde{\varepsilon}}_1 , \tilde{\tilde{X}}_0 = \tilde{\tilde{f}}(\tilde{\tilde{X}}_1, \varepsilon_S),$$ where $P_{\tilde{\tilde{\varepsilon}}_1} = P_{X_1\mid \textbf{X}_{\tilde{S}} = x_{\tilde{S}}}    $ and $\tilde{\tilde{f}}(\textbf{x}, \varepsilon) =f(\textbf{x}_{\tilde{S}}, x, \varepsilon)$. In this model, $\varepsilon_S\indep \tilde{\tilde{\varepsilon}}_1$. 

Now, note that $(\tilde{X}_0,\tilde{X}_1)\overset{D}{=}(\tilde{\tilde{X}}_0, \tilde{\tilde{X}}_1)$, since both sides are distributed as $\big[(X_0, X_1)\mid X_{\tilde{S}}\big]$.  This is a contradiction with the identifiability of (\ref{tyuiop}). Therefore,  $S$ is not $\mathcal{F}_F$-plausible.
 

\end{proof}




\begin{customlem}{\ref{LemmaAboutUnidentifiabilityFL}}
Let $(Y, \textbf{X})\in\mathbb{R}\times \mathbb{R}^p$ follow an $\mathcal{F}_L$-model with DAG $\mathcal{G}_0$ and $pa_Y(\mathcal{G}_0)\neq\emptyset$. Then, $|S_{\mathcal{F}_L}(Y)| \leq 1$ ($|S|$ represents the number of elements of the set $S$). Moreover, if  $a,b\in an_Y(\mathcal{G}_0)$ that are d-separated in $\mathcal{G}_0$ exist, then $S_{\mathcal{F}_L}(Y) = \emptyset$. 
\end{customlem}

\begin{proof}\label{Proof of LemmaAboutUnidentifiabilityFL}
First, we show $|S_{\mathcal{F}_L}(Y)| \leq 1$. Let $a\in an_Y(\mathcal{G}_0)\cap Source(\mathcal{G}_0)$. Such $a$ exists since $pa_Y(\mathcal{G}_0)\neq\emptyset$. We show that $S = \{a\}$ is an $\mathcal{F}_L-$plausible set. 

Denote $X_0:= Y$.  Since $\mathcal{G}_0$ is acyclic, it is possible to express recursively each variable $X_j , j= 0, \dots, p, $ as a weighted sum of the noise terms $\varepsilon_0, \dots, \varepsilon_p$ that belong to the ancestors of $X_j$. Let us write the Linear SCM with notation 
\begin{equation*}
X_i = \sum_{j\in pa_i}\beta_{j,i}X_j + \varepsilon_i  = \sum_{j\in an_i}\beta_{j\to i}\varepsilon_j,
\end{equation*}
where $\beta_{j,i}$ are non-zero constants and $\beta_{j\to i}$ is the sum of distinct weighted directed paths from node $j$ to node $i$, with a convention $\beta_{j\to j} := 1$.\footnote{To provide an example of the notation, if $X_1 = \varepsilon_1, X_2 = 2X_1 + \varepsilon_2, X_3 = 3X_1 + 4X_2+\varepsilon_3$, then $X_3 = 11\varepsilon_1 + 4\varepsilon_2 + 1\varepsilon_3 = \beta_{1\to 3}\varepsilon_1 +\beta_{2\to 3}\varepsilon_2 + \beta_{3\to 3}\varepsilon_1 $.}

Using this notation, note that 
$$X_0 = \sum_{j\in an_0}\beta_{j\to i}\varepsilon_j = \beta_{a\to 0}\varepsilon_a + \sum_{j\in an_0\setminus \{a\}}\beta_{j\to i}\varepsilon_j =\beta_{a\to 0} X_a + \sum_{j\in an_0\setminus \{a\}}\beta_{j\to i}\varepsilon_j ,$$ where $X_a\indep \sum_{j\in an_i\setminus \{a\}}\beta_{j\to i}\varepsilon_j$ since $a\in Source(\mathcal{G}_0)$. Hence, $Y - \beta_{a\to 0} X_a\indep X_a$, which is almost the definition of $\mathcal{F}_L$-plausibility of set $S = \{a\}$. More rigorously, for  $S = \{a\}$, we can find $f\in\mathcal{F}_L$ such that  $f_Y^{\leftarrow}({X}_{S}, Y)\indep X_S$ and $f_Y^{\leftarrow}({X}_{S}, Y)\sim U(0,1)$. This function can be defined as 
$$
f(x, \varepsilon) = \beta_{a\to 0}x + g^{-1}(\varepsilon), \,\,\,x\in\mathbb{R}, \varepsilon\in (0,1),
$$
where $g$ is the distribution function of $(Y - \beta_{a\to 0}X_S)$. This function obviously satisfies $f\in\mathcal{F}_L$. Moreover, since $f_Y^{\leftarrow}(X_{S}, Y) = g(Y - \beta_{a\to 0}X_S)$, it holds that $f_Y^{\leftarrow}({X}_{S}, Y)\indep X_S$ and $f_Y^{\leftarrow}({X}_{S}, Y)\sim U(0,1)$, which is what we wanted to show. Hence,  $|S_{\mathcal{F}_L}(Y)|\leq 1$, since $S_{\mathcal{F}_L}(Y)\subseteq S = \{a\}$. 

Now, let $a,b\in an_Y(\mathcal{G}_0)$ that are d-separated in $\mathcal{G}_0$. Let $a', b'\in \mathcal{G}_0$ such that $a'\in \big\{an_a(\mathcal{G}_0)\cup \{a\}\big \}\cap Source(\mathcal{G}_0)$,   $b'\in \big\{an_b(\mathcal{G}_0)\cup \{b\}\big \}\cap Source(\mathcal{G}_0)$. They are well defined since the sets $an_a(\mathcal{G}_0)\cup \{a\}$,  $an_b(\mathcal{G}_0)\cup \{b\}$  must contain some source node. Since $a,b$ are d-separated,   $ \big\{an_a(\mathcal{G}_0)\cup \{a\}\big \}$ and $\big\{an_b(\mathcal{G}_0)\cup \{b\}\big \}$ are disjoint sets, $a'\neq b'$ (they are even d-separated in $\mathcal{G}_0$ \citep{Pearl}). 

Using the same argument as in the first part of the proof, since $a'\in an_Y(\mathcal{G}_0)\cap Source(\mathcal{G}_0)$, it holds that  $S = \{a\}$ is an $\mathcal{F}_L-$plausible set.  $S = \{b\}$ is also an $\mathcal{F}_L-$plausible set since $b'\in an_Y(\mathcal{G}_0)\cap Source(\mathcal{G}_0)$. Together, $S_{\mathcal{F}_L}(Y)\subseteq \{a\}$ and $S_{\mathcal{F}_L}(Y)\subseteq \{b\}$. We showed that  $S_{\mathcal{F}_L}(Y) = \emptyset$. 
\end{proof}


\begin{customlem}{\ref{lemma158}}
 Let $\mathcal{F}\subseteq\mathcal{I}_m$. Let $(X_0, \textbf{X})\in\mathbb{R}\times \mathbb{R}^p$ follow an $\mathcal{F}$-model with DAG $\mathcal{G}_0$ and $pa_{X_0}(\mathcal{G}_0)\neq \emptyset$. Let  $S\subseteq \{1, \dots, p\}$ be a non-empty set. If $(X_0, \textbf{X})$ is marginalizable to $S\cup\{0\}$, then $S_{\mathcal{F}}(X_0)\subseteq S$. 
\end{customlem}
\begin{proof}
 \label{Proof of lemma158}
 Since  $(X_0, \textbf{X})$ is marginalizable to $S\cup\{0\}$,  $(X_0, \textbf{X}_S)$ follows an $\mathcal{F}$-model. Therefore, $f_0\in\mathcal{F}$ exists, such that $X_0 = f_0(X_{\tilde{S}}, \varepsilon_0)$ for some $\tilde{S}\subseteq S$, $\varepsilon_0\indep X_{\tilde{S}}$, $\varepsilon_0\sim U(0,1)$. 
In other words, $ f_0^{\leftarrow}(X_{\tilde{S}}, X_0)\indep X_{\tilde{S}}$,  $f_0^{\leftarrow}(X_{\tilde{S}}, X_0)\sim U(0,1)$, which is exactly the definition of $\mathcal{F}$-plausibility. Hence, $\tilde{S}$ is $\mathcal{F}$-plausible and consequently,  $S_{\mathcal{F}}(X_0)\subseteq \tilde{S}\subseteq S$. 
 \end{proof}






























 
 
 
\begin{customprop}{\ref{Support_proposition}}
Let $(Y, \textbf{X})\in\mathbb{R}\times \mathbb{R}^p$ follow an SCM with DAG $\mathcal{G}_0$. Let $S\subseteq\{1, \dots, p\}$ be a non-empty set. Let  $\underline{\Psi},\overline{\Psi}: \mathbb{R}^{\mid S\mid}\to \mathbb{R}$ be real functions such that
\begin{equation*}
supp(Y\mid \textbf{X}_S=\textbf{x}) = (\underline{\Psi} (\textbf{x}),\overline{\Psi}(\textbf{x})), \,\,\,\,\,\,\, \forall \textbf{x}\in supp(\textbf{X}_S).
\end{equation*}
Moreover, let
\begin{equation} \tag{\ref{eq9987}}
\frac{Y - \underline{\Psi}(\textbf{X}_S)}{\overline{\Psi}(\textbf{X}_S) - \underline{\Psi}(\textbf{X}_S)}\not\indep \textbf{X}_S.
\end{equation}
Then, $S$ is not $\mathcal{F}_{LS}$-plausible. 
\end{customprop}

\begin{proof}\label{Proof of Support_proposition}
For a contradiction, let $S$ be  $\mathcal{F}_{LS}$-plausible. Hence, $f\in\mathcal{F}_{LS}$ exists such that 
\begin{equation}\label{eq74}
f^{\leftarrow}(\textbf{X}_S, Y)\indep\textbf{ X}_S.
\end{equation}
Since $f\in\mathcal{F}_{LS}$, we can write $f^{\leftarrow}(\textbf{x},y) = q\big(\frac{y - \mu(\textbf{x})}{\sigma(\textbf{x})}\big)$  for some functions $\mu(\cdot), \sigma(\cdot)>0$ and for some (continuous) distribution function $q(\cdot)$. Using this notation, (\ref{eq74}) is equivalent to
\begin{equation} \label{eq989}
\frac{Y - {\mu}(\textbf{X}_S)}{{\sigma}(\textbf{X}_S)}\indep \textbf{X}_S.
\end{equation}
Denote $W_{\textbf{x}}:= (Y\mid \textbf{X}_S=\textbf{x})$. From (\ref{eq989}), we get that for all $\textbf{x},\textbf{y}$ in the support of $\textbf{X}_S$, it must hold that
\begin{equation}\label{eq7285}
\frac{W_\textbf{x} - {\mu}(\textbf{x})}{{\sigma}(\textbf{x})}\overset{D}{=}\frac{W_\textbf{y} - {\mu}(\textbf{y})}{{\sigma}(\textbf{y})}.
\end{equation}
Hence, supports must also match, i.e., (\ref{eq7285}) implies
\begin{align*}
\frac{ \underline{\Psi} (\textbf{x}) - {\mu}(\textbf{x})}{{\sigma}(\textbf{x})}&\overset{}{=}\frac{ \underline{\Psi} (\textbf{y}) - {\mu}(\textbf{y})}{{\sigma}(\textbf{y})}, \,\,\,\,\,\,\,\,\,\,\,\,\,\,\,\,\,\,\,\,\,\,\,\,\,\,
\frac{ \overline{\Psi}(\textbf{x}) - {\mu}(\textbf{x})}{{\sigma}(\textbf{x})}\overset{}{=}\frac{ \overline{\Psi}(\textbf{y}) - {\mu}(\textbf{y})}{{\sigma}(\textbf{y})},
\end{align*}
for all $\textbf{x},\textbf{y}$ in the support of $\textbf{X}_S$. Solving for $\mu, \sigma$  gives us 
\begin{align*}
{\mu}(\textbf{x})&\overset{}{=}c_1+ \underline{\Psi} (\textbf{x}) , \,\,\,\,\,\,\,\,\,\,\,
\sigma(\textbf{x})\overset{}{=}c_2\cdot [\overline{\Psi} (\textbf{x})-\underline{\Psi}(\textbf{x})],
\end{align*}
where $c_1 \in  \mathbb{R}, c_2 \in \mathbb{R}_{+}$ are some constants. Plugging this into (\ref{eq989}) gives us a contradiction with (\ref{eq9987}). 
\end{proof}


\begin{customprop}{\ref{LemmaOParetoinseparabilite}}
 Consider $(Y, \textbf{X})\in\mathbb{R}\times \mathbb{R}^p$ satisfying ( \ref{SCM_for_Y}) with  $pa_Y\neq \emptyset$ and $\mathcal{F} = \mathcal{F}_{F}$. where $F$ be a distribution function whose parameter acts multiplicatively. 
 Let $\textbf{X}_{pa_Y}$ be a continuous random vector with full support and independent components. 
\begin{itemize}
\item Consider $f_Y\in\mathcal{F}_F$ in the form $f_Y(\textbf{x}, \varepsilon)=F^{-1}\big(\varepsilon, \theta(\textbf{x})\big)$ with additive function $\theta(x_1, \dots, x_k) = h_1(x_1)+\dots + h_k(x_k)$, where $h_i$ are continuous non-constant real functions. Then, then every $S\subsetneq pa_Y$ is not $\mathcal{F}_{F}$-plausible. 
\item Consider $f_Y\in\mathcal{F}_F$ in the form $f_Y(\textbf{x}, \varepsilon)=F^{-1}\big(\varepsilon, \theta(\textbf{x})\big)$ with multiplicative function   $\theta(x_1, \dots, x_k) = h_1(\textbf{x}_S)\cdot h_2(\textbf{x}_{\{1, \dots, k\}\setminus S})$ for some $S\subsetneq \{1, \dots, k\}$, where $h_1, h_2$ are continuous non-constant non-zero real functions. Then, $S_{\mathcal{F}_F}(Y)=\emptyset$.
\end{itemize}
\end{customprop}
\begin{proof}\label{Proof of LemmaOParetoinseparabilite}
\textbf{The first bullet-point}: For a contradiction, consider that $S=\{1, \dots, s\}\subset \{1, \dots, k\}$ is $\mathcal{F}_{F}$-plausible. Then for almost all $z\in(0,1)$, there exist $g\in\mathcal{F}_F$ such that  $g^{\leftarrow}\big(\textbf{X}_S, f(\textbf{X}, z)\big)\indep \textbf{X}_S$. Since $g\in\mathcal{F}_F$, we can write $g^{\leftarrow}(\textbf{x}_S, \cdot) = F\big(\cdot, \theta_g(\textbf{x}_S)\big)$ for some non-constant function $\theta_g$. Hence, 
$$\textbf{X}_S\indep g^{\leftarrow}\big(\textbf{X}_S, f(\textbf{X}, z)\big)
= F[F^{-1}\big(z, \theta(\textbf{X})\big), \theta_g(\textbf{X}_S)]
= f_1[z, f_2\big(\theta_g(\textbf{X}_S)\big) \cdot\theta(\textbf{X})].$$ 
We use identity (\ref{trivial_identity}). Since $f_1$ is invertible, we obtain 
\begin{equation}\label{dfrge}
\textbf{X}_S\indep f_2\big(\theta_g(\textbf{X}_S)\big) \cdot\theta(\textbf{X}).
\end{equation}
Define $\tilde{\theta}_g(\textbf{X}_S):=f_2\big(\theta_g(\textbf{X}_S)\big)$. Finally, since $\theta(\textbf{X})$ is an additive function from the assumptions, (\ref{dfrge}) is equivalent to
$$
\tilde{\theta}_g(\textbf{X}_S)[h_1(X_1) + \dots + h_k(X_k)]\indep \textbf{X}_S. 
$$
However, that is a contradiction with Lemma \ref{CoolLemma} part 2. 

\textbf{The second bullet-point}:  
We show that $S$ is $\mathcal{F}_F$-plausible set by finding an appropriate function $g\in\mathcal{F}_F$ such that $g^{\leftarrow}\big(\textbf{X}_S, f_Y(\textbf{X}, \varepsilon_Y)\big)\indep \textbf{X}_S$. Since it must hold that  $g\in\mathcal{F}_F$, we write  $g^{\leftarrow}(\textbf{x}_S, \cdot) = F\big(\cdot, \theta_g(\textbf{x}_S)\big)$ for some $\theta_g$. 

Rewrite 
 $$g^{\leftarrow}\big(\textbf{X}_S, f(\textbf{X}, \varepsilon_Y)\big) = F[F^{-1}\big(\varepsilon_Y, \theta(\textbf{X})\big)     ,\theta_g(\textbf{X}_S)]
 = f_1[\varepsilon_Y, f_2\big(\theta_g(\textbf{X}_S)\big) \cdot\theta(\textbf{X})],$$
where $f_1, f_2$ are from (\ref{postMultiplDefinition}). 
We choose $\theta_g$ such that $f_2\big(\theta_g(\textbf{x}_S)\big) = \frac{1}{h_1(\textbf{x}_S)}$. Obviously, $g\in\mathcal{F}_F$. Then, by extending $\theta$ to its multiplicative form, we get 
$$ f_1[\varepsilon_Y, f_2\big(\theta_g(\textbf{X}_S)\big) \cdot\theta(\textbf{X})] =  f_1[\varepsilon_Y,  h_2(\textbf{X}_{\{1, \dots, k\}\setminus S})]\indep \textbf{X}_S.$$
Together, we found $g\in\mathcal{F}_F$ defined by $g^{\leftarrow}(\textbf{x}_S, \cdot) = F\big(\cdot, f_2^{-1}(\frac{1}{h_1(\textbf{x}_S)})\big)$ that satisfy   $g^{\leftarrow}\big(\textbf{X}_S, f(\textbf{X}, \varepsilon_Y)\big)\indep \textbf{X}_S$. Hence, $S$ is $\mathcal{F}_F$-plausible. The analogous argument can be given for the set $\{1, \dots, k\}\setminus S$. Hence,  $S_{\mathcal{F}_F}(Y)\subseteq S\cap( \{1, \dots, k\}\setminus S) = \emptyset$.
\end{proof}


\begin{customprop}{\ref{LemmaOLocationScaleinseparabilite}}
 Consider $(Y, \textbf{X})\in\mathbb{R}\times \mathbb{R}^p$ satisfying ( \ref{SCM_for_Y}) with  $pa_Y\neq \emptyset$ and $\mathcal{F} = \mathcal{F}_{LS}$. 
 Let $\textbf{X}_{pa_Y}$ have full support and independent components. 

 Let $f_Y\in\mathcal{F}_{LS}$  have the form $f_Y(\textbf{x}, \varepsilon)=\mu(\textbf{x}) + \sigma(\textbf{x})\varepsilon$, where $\theta(\textbf{x}) = \big(\mu(\textbf{x}), \sigma(\textbf{x})\big)^\top$ is additive in both components, that is,  
$\mu(\textbf{x}) = h_{1, \mu}(x_1)+\dots + h_{k, \mu}(x_k)$ and 
$\sigma(\textbf{x}) =h_{1, \sigma}(x_1)+\dots + h_{k, \sigma}(x_k)$ for some continuous non-constant non-zero functions $h_{i,\cdot}$, where we also assume $h_{i,\sigma}>0$, $i=1, \dots, k$.  Then, then every $S\subsetneq pa_Y$ is not $\mathcal{F}_{LS}$-plausible. 
\end{customprop}
\begin{proof}
\label{Proof of LemmaOLocationScaleinseparabilite}
For a contradiction, consider that $S\subsetneq \{1, \dots, pa_Y\}$, is $\mathcal{F}_{LS}$-plausible. Without loss of generality, let $S=\{1, \dots, s\}$ for $s<k=|pa_Y|$. Then, $g\in\mathcal{F}_{LS}$ exist such that  $g^{\leftarrow}\big(\textbf{X}_S, Y\big)\indep \textbf{X}_S$. Since $g\in\mathcal{F}_{LS}$, we can write $g(\textbf{x}_S, e) = \mu_g(\textbf{x}_S) + \sigma_g(\textbf{x}_S)q^{-1}(e)$ for some function $\theta_g=(\mu_g, \sigma_g)$ that is irreducible and hence non-constant in neither of the arguments. Inverse of such a function is in the form $g^{\leftarrow}(\textbf{x}_S, e) = q(\frac{e - \mu_g(\textbf{x}_S)}{\sigma_g(\textbf{x}_S)})$. 

Hence, simply rewriting   
$$\textbf{X}_S\indep g^{\leftarrow}\big(\textbf{X}_S, Y\big)
= q\bigg(\frac{Y - \mu_g(\textbf{X}_S)}{\sigma_g(\textbf{X}_S)}\bigg),$$
and using \ref{trivial_identity} and $Y = \mu(\textbf{X}_{pa_Y}) + \sigma(\textbf{X}_{pa_Y})\varepsilon_Y$ we get 
\begin{equation}\label{dfrgeTRI}
\textbf{X}_S\indep \frac{\mu(\textbf{X}_{pa_Y}) + \sigma(\textbf{X}_{pa_Y})\varepsilon_Y - \mu_g(\textbf{X}_S)}{\sigma_g(\textbf{X}_S)}. 
\end{equation}
Equation (\ref{dfrgeTRI}) can be equivalently rewritten into
\begin{equation}\label{erty}
\textbf{X}_S\indep f_1(\textbf{X}_S)  + f_2(\textbf{X}_S)h(\textbf{X}_{S^c}, \varepsilon_Y),
\end{equation}
where $ S^c=\{1, \dots, k\}\setminus S$, and 
%Mozno tam je clen navyse, ale na tom nezalezi
$$f_1(\textbf{x}) =\frac{ h_{1, \mu}(x_1)+\dots + h_{s, \mu}(x_s)- \mu_g(\textbf{x})}{\sigma_g(\textbf{x})},$$ $$f_2(\textbf{x}) =  \frac{ h_{1, \mu}(x_1)+\dots + h_{s, \mu}(x_s)}{\sigma_g(\textbf{x})},$$
$$
h(\textbf{x}, \varepsilon) = h_{s+1, \mu}(x_{s+1})+\dots + h_{k, \mu}(x_k) + [h_{s+1, \sigma}(x_{s+1})+\dots + h_{k, \sigma}(x_k)]\varepsilon.
$$
However, independence (\ref{erty}) is a contradiction with Lemma \ref{CoolLemma} part 4. 
    
\end{proof}


\begin{customthm}{\ref{Theorem_in_section2}}

 Consider $(Y, \textbf{X})\in\mathbb{R}\times \mathbb{R}^p$ satisfying (\ref{SCM_for_Y}) with  $pa_Y\neq \emptyset$ and $\mathcal{F} = \mathcal{F}_A$. 
 Let $\textbf{X}_{pa_Y}$ have independent components. 
 
  \begin{itemize}
     \item If $f_Y$ has a form \begin{equation}
        f_Y(\textbf{x}, e) = h_1(\textbf{x}_S) + h_2(\textbf{x}_{pa_Y\setminus S}) + q^{-1}(e), \,\,\,\,\,\textbf{x}\in\mathbb{R}^{|pa_Y|}, e\in(0,1),\tag{\ref{efaetgfas}}
     \end{equation}
 for some non-empty $S\subset pa_Y$, where $h_1, h_2$ are measurable functions and $q^{-1}$ is a quantile function. Then, $S_{\mathcal{F}_A}(Y) = \emptyset$.
     \item If $f_Y\in\mathcal{F}_A$ is continuous injective\footnote{Remark \ref{remark_lemma_additivity} in Appendix~\ref{Appendix_Auxiliary} shows an example where a periodic function $f$ contradicts this statement. We conjecture that the assumption about injectivity of $f$ can be relaxed to non-periodicity of $f$.    } function that cannot be written as (\ref{efaetgfas}), then every $S\subsetneq pa_Y$ is not $\mathcal{F}_A$-plausible. 
 \end{itemize}
\end{customthm}
 

\begin{proof}
\label{Proof of Theorem_in_section2}

    \textbf{The first bullet-point}:  We show that sets $S$ and $S^c:=pa_Y\setminus S$ are both $\mathcal{F}_A$-plausible. Consider a function $f\in\mathcal{F}_A$ satisfying $f^{\leftarrow}(\textbf{X}_S, Y):= \tilde{q}\big(Y - h_1(\textbf{X}_S)\big)$, where $\tilde{q}$ is a distribution function of $[h_2(\textbf{X}_{pa_Y\setminus S}) + q^{-1}(\varepsilon_Y)]$. Then, $f^{\leftarrow}(\textbf{X}_S, Y)\indep \textbf{X}_S$ since $Y - h_1(\textbf{X}_S) = h_2(\textbf{X}_{pa_Y\setminus S}) + q^{-1}(\varepsilon_Y)\indep \textbf{X}_S$ and we apply identity (\ref{trivial_identity}). Moreover, $f^{\leftarrow}(\textbf{X}_S, Y)\sim U(0,1)$ trivially. We showed that set $S$ satisfies every property for being  $\mathcal{F}_A$-plausible. Set $S = pa_Y\setminus S$ can be analogously shown to be  $\mathcal{F}_A$-plausible as well. Therefore,  $S_{\mathcal{F}_A}(Y)\subseteq S\cap (pa_Y\setminus S)) = \emptyset$.

\textbf{The second bullet-point}: Let us write $Y = f_0(X_1, \dots, X_{|pa_Y|}) + q^{-1}(\varepsilon_Y)$, where $\varepsilon_Y\indep \textbf{X}_{pa_Y}$ for some continuous injective function $f_0$ and continuous quantile function $q^{-1}$.  For a contradiction, consider that an  $\mathcal{F}_A$-plausible non-empty set $S\subset pa_Y$ exists. That means, $f\in\mathcal{F}_A: \mathbb{R}^{|S|+1}\to\mathbb{R}$ exists such that (\ref{Definition_F_plausible}) holds. Since $f\in\mathcal{F}_A$, we can write $f(\textbf{x}, e) = \mu(\textbf{x}) + \tilde{q}^{-1}(e), \,\,\textbf{x}\in\mathbb{R}^{|S|}, e\in (0,1)$ for some measurable function $\mu$ and quantile function $\tilde{q}^{-1}$. Additive functions have an inverse in a form $f^{\leftarrow}(\textbf{x}, y) = \tilde{q}\big(y-\mu(\textbf{x})\big),\,\,\textbf{x}\in\mathbb{R}^{|S|}, y\in\mathbb{R}$ (see discussion in Appendix~\ref{Appendix_A.1.}).   Using (\ref{Definition_F_plausible}) and identity \ref{trivial_identity}, we have $Y- \mu(\textbf{X}_S)\indep \textbf{X}_S$. 

Hence, we have  
\begin{align*}
    Y- \mu(\textbf{X}_S)&\indep \textbf{X}_S 
    \\ f_0(X_1, \dots, X_{|pa_Y|}) + q^{-1}(\varepsilon_Y) - \mu(\textbf{X}_S) &\indep \textbf{X}_S 
    \\f_0(X_1, \dots, X_{|pa_Y|}) - \mu(\textbf{X}_S) &\indep \textbf{X}_S. 
\end{align*}
This is a contradiction with Lemma \ref{CoolLemma} part 1.    
\end{proof}

\begin{customlem}{\ref{lemma_hidden_confounder}}

 Consider $(Y, \textbf{X})\in\mathbb{R}\times \mathbb{R}^p$  satisfy ( \ref{SCM_for_Y}) with  $\mathcal{F} = \mathcal{F}_A$. Consider $\emptyset\neq hid\subset pa_Y$. Let $S\subseteq pa_Y \cap obs$ and $\tilde{S}:=(pa_Y\cap obs)\setminus S$ such that
 $(\textbf{X}^{hid}, \textbf{X}_S)\indep \textbf{X}_{\tilde{S}}$ (one can consider that $\textbf{X}^{hid}$ cause $\textbf{X}_S$ and $Y$, but not $\textbf{X}_{\tilde{S}}$). 
 
 If $f_Y$ has a form \begin{equation*}
        f_Y(\textbf{x}, e) = h_1(\textbf{X}^{hid}, \textbf{X}_S) + h_2(\textbf{X}_{\tilde{S}}) + q^{-1}(e), \,\,\,\,\,\textbf{x}\in\mathbb{R}^{|pa_Y|}, e\in(0,1),
     \end{equation*}
for some continuous non-constant real functions  $h_1, h_2$ and a quantile function $q^{-1}$. Then, $S_{\mathcal{F}_A}(Y) \subseteq \tilde{S}\subset pa_Y$.
\end{customlem}
\begin{proof}
\label{Proof of lemma_hidden_confounder}
The set $\tilde{S}$ is $\mathcal{F}_A$-plausible since  $Y - h_2(\textbf{X}_{\tilde{S}}) =h_1(\textbf{X}^{hid}, \textbf{X}_S) + q^{-1}(\varepsilon_Y)  \indep \textbf{X}_{\tilde{S}}$. Therefore $S_{\mathcal{F}_A}(Y)\subseteq\tilde{S}$. 
\end{proof}


\begin{customprop}{\ref{Proposition_consistency}}
Consider $\mathcal{F} = \mathcal{F}_A$ and 
let $(Y, \textbf{X})\in\mathbb{R}\times \mathbb{R}^p$ follow an SCM with DAG $\mathcal{G}_0$ satisfying ( \ref{SCM_for_Y}). Assume that every $S \neq pa_Y$ is not $\mathcal{F}$-plausible. 
Then,   

\begin{equation}
  \lim_{n\to\infty} \mathbb{P}(\widehat{pa}_Y  \neq pa_Y) = 0,
\end{equation}   
where $n$ is the size of the random sample and $\widehat{pa}_Y$ is our score-based estimate from Section~\ref{Section_algorithm2} with $\lambda_1, \lambda_2>0, \lambda_3 = 0$, suitable estimation procedure, and HSIC independence measure. 
\end{customprop}
\begin{proof}
\label{Proof of Proposition_consistency}
This result is a simple consequence of Theorem 20 in \cite{reviewANMMooij}. We use the same notation. For a rigorous definition of $HSIC$ and $\widehat{HSIC}$, see Appendix A.1 in \cite{reviewANMMooij}. 



We show that $score(S)> score(pa_Y)$ as $n\to\infty$ for any $S\neq pa_Y$. 
The $score(S)$ is defined as the weighted sum of  \textit{Independence} and \textit{Significance} terms. Let us first concentrate on the former. By definition, we write $\textit{Independence} = -\widehat{HSIC}(\textbf{X}_S, \hat{\varepsilon}_S)$. On a population level, it holds (Lemma~12 in \cite{reviewANMMooij}) that  ${HSIC}(\textbf{X}_S, {\varepsilon}_S) > 0 $ and ${HSIC}(\textbf{X}_{pa_Y}, {\varepsilon}_{pa_Y}) = 0$, since $\textbf{X}_S$ and ${\varepsilon}_S$ are not independent (because $S$ is not $\mathcal{F}$-plausible) and  $\textbf{X}_{pa_Y}$ and ${\varepsilon}_{pa_Y}$ are independent (by definition of the SCM). 
By Theorem~20 in \cite{reviewANMMooij}, we obtain $\widehat{HSIC}(\textbf{X}_{pa_Y}, \hat{\varepsilon}_{pa_Y})\to {HSIC}(\textbf{X}_{pa_Y}, {\varepsilon}_{pa_Y})=0$ and $\widehat{HSIC}(\textbf{X}_{S}, \hat{\varepsilon}_{S})\to {HSIC}(\textbf{X}_{S}, {\varepsilon}_{S})>0$, as $n\to\infty$. Therefore, the independence term is strictly smaller (for some large $n$) for $S$ than for $pa_Y$. 

Let us focus on \textit{Significance} term\footnote{We work with the \textit{Significance} term somewhat vaguely in this proof. However, we only need $Significance\to 0$ as $n\to\infty$ for $pa_Y$, which is satisfied for any reasonable method of assessing significance of covariates.}. Since all $\textbf{X}_{pa_Y}$ are significant (otherwise $f_Y\notin\mathcal{I}_m$) we get that $Significance\to 0$ as $n\to\infty$ for $pa_Y$. Moreover, by definition, always $Significance\geq 0$.

Together, we find that $score(pa_Y)>score(S)$ for large $n$, since 
the \textit{Independence} term is strictly smaller (for large $n$) for $S$ than for $pa_Y$ and \textit{Significance} term converges to $0$ for $pa_Y$ and is non-negative.  
We showed that $pa_Y$ has the largest score among all $S\subseteq \{1, \dots, p\}$ (for $n$ large enough). 
\end{proof}



%


\subsection{General (full) identifiability $\implies$ (local) $\mathcal{F}$-identifiability }

In the following, we demonstrate that classical identifiability results from the literature can be used to assess the $\mathcal{F}$-plausibility of a set $S$. Informally, we show that if all variables in the SCM follow an identifiable $\mathcal{F}$-model, then any set $S$ containing a child of $Y$ cannot be $\mathcal{F}$-plausible. To state the result, we require the following definition.

We define a \textbf{projection} of a graph $\mathcal{G} = (V, E)$ onto $S \subseteq V$, denoted as $\mathcal{G}[S]$, as a graph with vertices $S$ and the following edges: for distinct $i, j \in S$, there is an edge $i \to j$ in $\mathcal{G}[S]$ if there is a directed path from $i$ to $j$ in $\mathcal{G}$ such that all vertices on this path except $i, j$ do not belong to $S$ ($i \to \dots \to j$). Moreover, there is a bidirected edge $i - j$ if a path exists between $i, j$ in $\mathcal{G}$ that does not contain a collider and the first edge points towards $i$ and the last edge points towards $j$ ($i \leftarrow \dots \to j$).

For simplicity, we focus on the case when $\textbf{X} = (X_1, \dots, X_p)$ are neighbors (either direct causes or direct effects) of $Y$ in the corresponding SCM. Using the classical conditional independence approach and d-separation \citep{Pearl_causal_diagrams_biometrika}, we can eliminate other variables from being potential parents of $X_0 = Y$. Nevertheless, the theory can be extended to non-neighbors as well.

\begin{proposition}
\label{TheoremFidentifiabilityWithChild}
Let $(X_0, \textbf{X})$ follow an SCM with DAG $\mathcal{G}_0$ such that (\ref{SCM_for_Y}) holds. Assume that all $\textbf{X}$ are neighbors of $X_0$ in $\mathcal{G}_0$. Let $S \subseteq \{1, \dots, p\}$ and denote $S_0 = S \cup \{0\}$. Suppose $S$ contains a child of $X_0$ in $\mathcal{G} := \mathcal{G}_0[S_0]$; that is, $\exists j \in ch_0(\mathcal{G})$. Assume that $\mathcal{G}$ is a DAG.

Let $(X_0, \textbf{X}_S)$ follow an identifiable $\mathcal{F}_A$-model with DAG $\mathcal{G}$. Then, $S$ is not $\mathcal{F}_A$-plausible.
\end{proposition}

\begin{proof}
Without loss of generality, let $X_1$ be a childless child of $X_0$ in $S$ (the set of children of $X_0$ is nonempty by assumption, and one of them must be childless to avoid cycles). Denote $k = |S| \in \mathbb{N}$. Before we start with the proof, note the following trivial observation:

\textbf{Observation:} In every DAG $\mathcal{G}$ where all vertices are neighbors of $X_0$ and $X_1$ is the childless child of $X_0$, there exists a vertex $i$ such that either 1) $pa_i = \emptyset$, or 2) $pa_0 = \emptyset$ and $pa_i \subseteq \{0\}$.

We start with the case when $k = 1$ and $X_0 \to X_1$. Then, we use the principle of induction and Lemma~\ref{lemma_global_local_identifiability} to prove that any $S$ containing the vertex $X_1$ is not $\mathcal{F}_A$-plausible in $\mathcal{G}$.

\textbf{Step 1 ($k=1$)}: Let $\mathcal{G}_0 := \{0 \to 1\}$. Assume that $(X_0, X_1)$ follows an identifiable $\mathcal{F}_A$-model with DAG $\mathcal{G} = \mathcal{G}_0$. Then, trivially, $S = \{1\}$ is not $\mathcal{F}_A$-plausible. By definition, there does not exist $f \in \mathcal{F}_A$ such that $X_0 = f(X_1, \varepsilon_0)$. This is precisely the definition of $\mathcal{F}_A$-plausibility. Hence, $S = \{1\}$ is not $\mathcal{F}_A$-plausible.

\textbf{Step 2 ($k = 2$)}: Recall that all variables are neighbors of $X_0$ and $1 \in ch_0(\mathcal{G})$. There are four graphs satisfying these conditions: $\{2 \to 0 \to 1\}$, $\{2 \to 0 \to 1 \leftarrow 2\}$, $\{0 \to 1 \leftarrow 2 \leftarrow 0\}$, $\{2 \leftarrow 0 \to 1\}$. Notice that in all of these cases, $pa_2 \subseteq \{0\}$. Using Lemma~\ref{lemma_global_local_identifiability}, we condition on $X_2 = x$ and obtain a bivariate identifiable $\mathcal{F}_A$-model with $1 \in ch_0$. Using Step 1, we showed that $f \in \mathcal{F}_A$ cannot exist such that $Y = f(X_1, x, \varepsilon_Y)$, and therefore (\ref{SCM_for_Y}) cannot be satisfied. Hence, $S = \{1, 2\}$ is not $\mathcal{F}_A$-plausible.

\textbf{Step 3 ($k \to k-1$)}: Using the Observation, find a vertex $i$ such that either $pa_i = \emptyset$, or $pa_0 = \emptyset$ and $pa_i \subseteq \{0\}$. Using Lemma~\ref{lemma_global_local_identifiability}, we condition on $X_i = x$ and obtain an identifiable $\mathcal{F}_A$-model with $1 \in ch_0$. Using the previous step, $f \in \mathcal{F}_A$ cannot exist such that $Y = f(X_1, \dots, X_{i-1}, x, X_{i+1}, \dots, X_k, \varepsilon_Y)$, and therefore (\ref{SCM_for_Y}) cannot be satisfied. Hence, $S$ is not $\mathcal{F}_A$-plausible.
\end{proof}

\begin{lemma}
\label{lemma_global_local_identifiability}
Let $(X_0, \textbf{X})$ follow an identifiable $\mathcal{F}_A$-model with DAG $\mathcal{G}$, where all $\textbf{X}$ are neighbours of $X_0$.  Let $S=\{2, \dots, p\}$.  Let one of the following conditions hold: 

\begin{enumerate}
    \item  Let $X_1$ be the source variable ($pa_1=\emptyset)$. 
    \item Let  $X_0$ be the source variable ($pa_0=\emptyset)$ and $pa_1 \subseteq \{0\}$ (that is, all arrows from $X_0, X_1$ are only outgoing except with a possible $X_0\to X_1$).  
\end{enumerate}

Then, there exist $x\in supp(X_1)\subseteq \mathbb{R}$ such that  $(X_0, \textbf{X}_{S})\mid X_1=x$ follow an identifiable $\mathcal{F}_A$-model; that is, there exist exactly one DAG $\mathcal{G}_0$ such that the joint distribution $P_{(X_0, \textbf{X}_{S})\mid X_1=x}$ can be generated via an $\mathcal{F}_A$-model with graph $\mathcal{G}_0$. In particular, $\mathcal{G}_0$ is equal to $\mathcal{G}$ where we erase vertex $\{1\}$ and all edges connected with this vertex. 
\end{lemma}

\begin{proof}
\textit{Example and discussion about the main idea:} Is it possible to construct a counterexample where the full  $\mathcal{F}_A$-model is identifiable but if we condition on $X_1=x$ then multiple $\mathcal{F}_A$-models can generate the same conditional distribution? Related question was considered in the Example 26 from \cite{Peters2014}. There, the SCM has the form $X_1\to X_2\to X_3\leftarrow X_1$; notice that $X_1$ is the source, and $$X_1 = \eta_1, X_2 = f_2(X_2)+\eta_2, X_3 = f_3(X_1) + aX_2+\eta_3,$$ where $\eta_1\sim t_{v=3}, \eta_2\sim N(0, \sigma_2^2),\eta_3\sim N(0, \sigma_3^2) $ are independent, $a\neq 0$. Here, $X_2, X_3$ are non-Gaussian but $(X_2, X_3\mid X_1=x)$ is linear Gaussian. Hence,  $\mathcal{F}_A$-model  is not identifiable from $P_{X_2, X_3\mid X_1=x}$ . Therefore, we can revert this equation and obtain the same joint distribution generated by an  $\mathcal{F}_A$-model of the form $$X_1 = \eta_1, X_2 = g_2(X_1)+bX_3 + \tilde{\eta}_2, X_3 = g_3(X_1) +\tilde{\eta}_3,$$ where $\eta_1\sim t_{v=3}, \tilde{\eta}_2\sim N(0, \tilde{\sigma}_2^2),\tilde{\eta}_3\sim N(0, \tilde{\sigma}_3^2) $, for some $g_1, g_2$. Hence, the original   $\mathcal{F}_A$-model is not identifiable. 

\textbf{Proof. }

\textbf{Part 1:} Consider the following notation of the SCM over  $(X_0, \textbf{X})$: 
$$
X_1 = \eta_1,\,\,\,\,\,\,\,\,\, X_i = f_i(\textbf{X}_{pa_i})+\eta_i, \,\,\,i=0,2,3,\dots, p, \,\,\,\,\,{\eta}_i \text{ independent}. 
$$
Let $x$ be in the support of $X_1$. Then, consider a SCM with $\mathcal{G}_0$ defined in the statement and link functions
$$
X_i=\begin{cases} 
 & f_i^x(\textbf{X}_{pa_i})+\eta_i =f_i(\textbf{X}_{pa_i})+\eta_i, \,\,\,\,\,\text{if} \,i\not\in ch_1, \\
 & f_i^x(\textbf{X}_{pa_i})+\eta_i = f_i(x, \textbf{X}_{pa_i\setminus \{1\}})+\eta_i, \,\,\,\,\,\text{if} \,i\in ch_1. 
\end{cases}
$$
where the link functions have the form
$$
f_i^x(\textbf{x}_{pa_i})=\begin{cases} 
 & f_i(\textbf{x}_{pa_i}) \,\,\,\,\,\text{if} \,i\not\in ch_1, \\
 & f_i(x, \textbf{x}_{pa_i\setminus \{1\}}), \,\,\,\,\,\text{if} \,i\in ch_1. 
\end{cases}
$$
This is obviously an $\mathcal{F}_A$-model with DAG  $\mathcal{G}_0$ and the joint distribution is equal to  $P_{(X_0, \textbf{X}_{S})\mid X_1=x}$ . We will show that this model is identifiable for at least one choice of $x$. 

For a contradiction, suppose the that model is not identifiable for any $x$. We will find an $\mathcal{F}_A$-model with DAG  $\tilde{\mathcal{G}}\neq \mathcal{G}$ that generates the joint distribution $P_{(X_0, \textbf{X}_{S}) }$. If we manage to prove that, it gives us the desired contradiction. 

The non-identifiability for any $x$  gives us an existence of an $\mathcal{F}_A$-model defined by  $(\tilde{\mathcal{G}}_0,{\tilde{f}}^x_i, \tilde{\eta}^x_i)$ generating the same joint distribution  $P_{(X_0, \textbf{X}_{S})\mid X_1=x}$ (here, trivially $\tilde{\mathcal{G}}_0$ does not depend on $x$ from continuity of variables). 
Define $\tilde{\mathcal{G}}$ as $\tilde{\mathcal{G}}_0$ with additional one vertex $\{1\}$ and edges from $\{1\}$ to $ch_1(\mathcal{G})$. Obviously  $\tilde{\mathcal{G}}\neq \mathcal{G}$.  Consider a SCM over  $(X_0, \textbf{X})$ with DAG $\tilde{\mathcal{G}}$ and 
\begin{equation*} \begin{split}
&X_1 = \eta_1,\\&    
X_i=\tilde{f}_i^{X_1}(\textbf{X}_{pa_i(\tilde{\mathcal{G}})})+\tilde{\eta}^{X_1}_i,
    \end{split}
\end{equation*}
where $\tilde{\eta}^{X_1}_i$ are all jointly independent given $X_1$. However, for all $i\in ch_1(\mathcal{G})$, we know that $X_1$ must have an additive effect of $X_i$ (affects only mean, and not variance for example) and therefore $\tilde{\eta}^{X_1}_i = g_i(X_1) + \tilde{\eta}_i$ for some functions $g_i$ noise variables $\tilde{\eta}_i$. Therefore, we found an  $\mathcal{F}_A$-model with DAG $\tilde{\mathcal{G}}$, which is the desired contradiction. 

\textbf{Part 2: }
 Original SCM over  $(X_0, \textbf{X})$ can be written as 
$$
X_0 = \eta_0,\,\,\,X_1 = f_1(X_0)+\eta_1, \,\,\,\,\,\,\,\,\, X_i = f_i(\textbf{X}_{pa_i})+\eta_i, \,\,\,i=0,2,3,\dots, p, \,\,\,\,\,{\eta}_i \text{ independent}. 
$$
Let $x$ be in the support of $X_1$. Then, consider a SCM with $\mathcal{G}_0$ defined in the statement and link functions
$$
X_0 = \tilde{\eta}_0, \,\,\,\,\text{ where } \tilde{\eta}_0\sim X_0\mid X_1,
$$
$$
X_i=\begin{cases} 
 & f_i^x(\textbf{X}_{pa_i})+\eta_i =f_i(\textbf{X}_{pa_i})+\eta_i, \,\,\,\,\,\text{if} \,i\not\in ch_1, \\
 & f_i^x(\textbf{X}_{pa_i})+\eta_i = f_i(x, \textbf{X}_{pa_i\setminus \{1\}})+\eta_i, \,\,\,\,\,\text{if} \,i\in ch_1. 
\end{cases}
$$
where the link functions have the form
$$
f_i^x(\textbf{x}_{pa_i})=\begin{cases} 
 & f_i(\textbf{x}_{pa_i}) \,\,\,\,\,\text{if} \,i\not\in ch_1, \\
 & f_i(x, \textbf{x}_{pa_i\setminus \{1\}}), \,\,\,\,\,\text{if} \,i\in ch_1. 
\end{cases}
$$
This is obviously an $\mathcal{F}_A$-model and the joint distribution is equal to  $P_{(X_0, \textbf{X}_{S})\mid X_1=x}$ . We will show that this model is identifiable for at least one choice of $x$. 

For a contradiction, suppose the that model is not identifiable for any $x$. We will find an $\mathcal{F}_A$-model with DAG  $\tilde{\mathcal{G}}\neq \mathcal{G}$ that generates the joint distribution $P_{(X_0, \textbf{X}_{S}) }$. If we manage to prove that, it gives us the desired contradiction. 

The non-identifiability for any $x$  gives us an existence of an $\mathcal{F}_A$-model defined by  $(\tilde{\mathcal{G}}_0,{\tilde{f}}^x_i, \tilde{\eta}^x_i)$ generating the same joint distribution  $P_{(X_0, \textbf{X}_{S})\mid X_1=x}$ (here, trivially $\tilde{\mathcal{G}}_0$ does not depend on $x$ from continuity of variables). 
Define $\tilde{\mathcal{G}}$ as $\tilde{\mathcal{G}}_0$ with additional one vertex $\{1\}$ and edges from $\{1\}$ to $ch_1(\mathcal{G})$. Obviously  $\tilde{\mathcal{G}}\neq \mathcal{G}$.  Consider a SCM over  $(X_0, \textbf{X})$ with DAG $\tilde{\mathcal{G}}$ and  
\begin{equation*} \begin{split}
&X_0 = \eta_0,\\&    
X_i=\tilde{f}_i^{X_1}(\textbf{X}_{pa_i(\tilde{\mathcal{G}})})+\tilde{\eta}^{X_i}_i.
    \end{split}
\end{equation*}
Using the same argument as in the part 1, this is an $\mathcal{F}_A$-model with DAG  $\tilde{\mathcal{G}}$ that generates the joint distribution $P_{(X_0, \textbf{X}_{S}) }$. This is a contradiction with identifiability of  $\mathcal{G}$ in the original SCM.   

\end{proof}

We showed that if all variables in the SCM follow an identifiable $\mathcal{F}_A$-model and $\mathcal{G}$ is a DAG, then any set $S$ containing a child of $Y$ cannot be $\mathcal{F}_A$-plausible. In practice, this implies that we can use classical identifiability results to assess the $\mathcal{F}$-plausibility of a given set $S$.







\end{document}


