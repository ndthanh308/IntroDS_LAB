\section{Appendix:  Auxiliary results}
\label{Appendix_Auxiliary}

\begin{lemma}\label{distributionalequalitylemma}
Let $X$ be a non-degenerate continuous real random variable. Let $a,b\in\mathbb{R}$ such that 
\begin{equation}\label{qwerty}
a+bX\overset{D}{=}X.
\end{equation}
Then, either $(a,b) = (0,1)$ or $(a,b) = (2med(X), -1)$. Here, $med(X)$ is the median of $X$.
\end{lemma}
\begin{proof}
\textit{Idea of the proof assuming a finite variance of $X$: } If $X$ has finite variance, then (\ref{qwerty}) implies $var(a+bX) = var(X)$, rewriting gives us  $b^2 var(X)=var(X)$, and hence, $b=\pm 1$. Now, (\ref{qwerty}) also implies $\mathbb{E}(a+bX) = \mathbb{E}(X)$, hence $a=(1-b)\mathbb{E}(X)$. Therefore, if $b=1$, then $a= 0$, and if $b=-1$, then $a=2\mathbb{E}(X)$. 

\textit{Proof without the moment assumption:}  (\ref{qwerty}) implies that for any $q\in (0.5,1)$, the difference between the $q$ quantile and $(1-q)$ quantile should be the same on both sides of (\ref{qwerty}). Denote $F^{-1}_X(q)$ a $q-$quantile of $X$ and assume that $F^{-1}_X(q)\neq F^{-1}_X(1-q)$ (since $X$ is non-degenerate, such $q$ exist). We get 
$$
F^{-1}_{a+bX}(q) - F^{-1}_{a+bX}(1-q) = F^{-1}_X(q)- F^{-1}_X(1-q)=:D.
$$
Consider $b\geq 0$. Using linearity of the quantile function, we obtain $a+bF^{-1}_X(q) - \big(a+bF^{-1}_X(1-q)\big) = D$ and hence, $bD=D$, which gives us $b=1$. If $b<0$, then an identity $F^{-1}_{a+bX}(q) = a+\big(1-F^{-1}_{-bX}(1-q)\big) = a+\big(1+bF^{-1}_{X}(1-q)\big)$ hold. Hence, we get  $a+[1+bF^{-1}_X(1-q)] - [a+\big(1+bF^{-1}_X(1-q)\big)] = D$. Rewriting the left side, we get $-bD=D$,  which gives us $b=-1$. 

In the case when $b=1$, trivially $a=0$, since otherwise, $med(a+X)\neq med(X)$. If $b=-1$, then applying median on both sides of (\ref{qwerty}) gives us $med(a-X)= med(X)$ and hence, $a=2med(X)$, as we wanted to show. 
\end{proof}



\begin{lemma}\label{CoolLemma}
Let $\textbf{X}=(X_1, \dots, X_k)$ be a continuous random vector with independent components and $s<k$. 
\begin{enumerate}
\item Let $f:\mathbb{R}^{k}\to\mathbb{R}$ be an injective  function such that there does not exist a decomposition  $ f(\textbf{x}) = f_1(\textbf{x}_S) + f_2(\textbf{x}_{\{1, \dots, k\}\setminus S}), \textbf{x}\in\mathbb{R}^{k}$ for any non-empty $S\subset \{1, \dots, k\}$, where $f_1, f_2$ are some measurable functions. 

 Then, a measurable function $h$ does not exist such that for $s<k$ holds
\begin{equation}\label{yuiMAIN}
f(X_1, \dots, X_k)+  h(X_1, \dots, X_s) \indep (X_1, \dots, X_s).
\end{equation}

\item  Let $f_1, \dots, f_k$ be continuous non-constant real functions. A non-zero function $h$ does not exist such that 
\begin{equation}\label{yui}
  h(X_1, \dots, X_s)\big(f_1(X_1)+\dots+f_k(X_k)\big)\indep (X_1, \dots, X_s).
\end{equation}
\item  Let $f_1, \dots, f_k$ be continuous non-constant non-zero real functions. Then, a non-zero function $h$ does not exist such that 
\begin{equation}\label{yuiDVA}
 h(X_1, \dots, X_s) + f_1(X_1)f_2(X_2)\dots f_k(X_k)\indep (X_1, \dots, X_s).
\end{equation}
\item Let $f:\mathbb{R}^{k-s}\to\mathbb{R}$ be measurable function such that $f(X_{s+1}, \dots, X_k)$ is non-degenerate continuous random variable. Functions $h_1, h_2$ does not exist, such that $h_2$ is positive non-constant and
\begin{equation}\label{yuiTRI}
h_1(X_1, \dots, X_s) + h_2(X_1, \dots, X_s)f(X_{s+1}, \dots, X_k)\indep (X_1, \dots, X_s).
\end{equation}

\end{enumerate}
\end{lemma}
\begin{proof}
We use notation $\textbf{X}_S=(X_1, \dots, X_s)^\top$, $\textbf{X}_{\setminus S}=(X_{s+1},\dots X_k)^\top$. 
Let us introduce functionals (not norms, we only use them to simplify notation) $||\cdot||_{plus}$ and  $||\cdot||_{times}$, defined by $|| \textbf{a} ||_{plus} = a_1 +\dots + a_d$, $|| \textbf{a} ||_{times} = a_1 a_2\dots a_d$, for $\textbf{a}=(a_1, \dots, a_d)^\top\in\mathbb{R}^d$. 

\textbf{Part 1:} For a contradiction, let such $h$ exist. Define $\xi:=h(\textbf{X}_S) + f(\textbf{X}_S, \textbf{X}_{\setminus S}) $, which is the left hand side of (\ref{yuiMAIN}). Fix $\textbf{a}_0\in\mathbb{R}^s$ in the support of $\textbf{X}_S$ and define
$$f_1(\textbf{x}):=h(\textbf{a}_0) - h(\textbf{x}), \,\,\text{for}\,\,\textbf{x}\in\mathbb{R}^{s}, \,\,\,\,\,and\,\,\,\,\,\, f_2(\textbf{x}) := f(\textbf{a}_0, \textbf{x})\,\,\text{for}\,\,  \textbf{x}\in\mathbb{R}^{k-s}.   $$

Since $\xi\indep \textbf{X}_S$, for all $\textbf{x}\in\mathbb{R}^{s}$ holds   $\xi\mid [\textbf{X}_S=\textbf{a}_0] \overset{D}{=}\xi\mid [\textbf{X}_S=\textbf{x}]$. Hence, 
\begin{align*}
h(\textbf{x}) + f(\textbf{x}, \textbf{X}_{\setminus S})\overset{D}{=}h(\textbf{a}_0) + f(\textbf{a}_0, \textbf{X}_{\setminus S})\end{align*}
\begin{equation}
    \label{awrefarefs}
    f(\textbf{x}, \textbf{X}_{\setminus S})\overset{D}{=} f_1(\textbf{x}) +f_2(\textbf{X}_{\setminus S}). 
\end{equation}
To extend the equality from equality in distribution to equality everywhere, we use Lemma~\ref{lemma_additivity}. We found an additive decomposition of $f$, which is the desired contradiction. 

\textbf{Part 2: }For a contradiction, let such $h$ exist. First, some notation: Let $Y=f_{s+1}(X_{s+1})+\dots+f_{k}(X_k)$ and define $\xi:=h(\textbf{X}_S)(||f_S(\textbf{X}_S)||_{plus}+Y)$, where  $f_S: \mathbb{R}^s\to\mathbb{R}^s:f_S(\textbf{x})= (f_1(x_1), \dots, f_s(x_s))^\top$, which is the left hand side of (\ref{yui}). 

Choose $\textbf{a}, \textbf{b}, \textbf{c}\in\mathbb{R}^s$ in the support of $\textbf{X}_S$ such that $||f_S(\textbf{a})||_{plus}, ||f_S(\textbf{b})||_{plus}, ||f_S(\textbf{c})||_{plus}$ are distinct and $h(\textbf{b})\neq 0$ (it is possible since $h_i$ are non-constant). 

Since $\xi\indep \textbf{X}_S$, then  $\xi\mid [\textbf{X}_S=\textbf{a}] \overset{D}{=}\xi\mid [\textbf{X}_S=\textbf{b}] \overset{D}{=}\xi\mid [\textbf{X}_S=\textbf{c}]$. Hence, 
\begin{equation}\label{asdfgh}
h(\textbf{a})(||f_S(\textbf{a})||_{plus}+Y)\overset{D}{=}h(\textbf{b})(||f_S(\textbf{b})||_{plus}+Y)\overset{D}{=}h(\textbf{c})(||f_S(\textbf{c})||_{plus}+Y).
\end{equation}
By dividing by a non-zero constant $h(\textbf{b})$ and subtracting a constant $||f_S(\textbf{b})||_{plus}$, we get
$$
\frac{h(\textbf{a})}{h(\textbf{b})}||f_S(\textbf{a})||_{plus}-||f_S(\textbf{b})||_{plus}+\frac{h(\textbf{a})}{h(\textbf{b})}Y\overset{D}{=}Y\overset{D}{=}\frac{h(\textbf{c})}{h(\textbf{b})}||f_S(\textbf{c})||_{plus}-||f_S(\textbf{b})||_{plus}+\frac{h(\textbf{c})}{h(\textbf{b})}Y.
$$
Now we use Lemma \ref{distributionalequalitylemma}. It gives us that $\frac{f(\textbf{a})}{f(\textbf{b})}=\pm 1$ and also  $\frac{f(\textbf{c})}{f(\textbf{b})}=\pm 1$. Therefore, at least two values of $f(\textbf{a}), f(\textbf{b}), f(\textbf{c})$ must be equal (and neither of them are zero). WLOG $f(\textbf{a})= f(\textbf{c})$. Plugging this into equation (\ref{asdfgh}), we get $||h_S(\textbf{a})||_{plus}=||h_S(\textbf{c})||_{plus}$, which is a contradiction since we chose them to be distinct. 

\textbf{Part 3: } We proceed in a similar way to the previous part. For a contradiction, let such $h$ exist. First, some notation: let $Y=f_{s+1}(X_{s+1})\dots f_{k}(X_k)$ and define $\xi:=h(\textbf{X}_S) + (||f_S(\textbf{X}_S)||_{times} \cdot Y)$, where  $f_S: \mathbb{R}^s\to\mathbb{R}^s: f_S(\textbf{x})= (f_1(x_1), \dots, f_s(x_s))^\top$, which is the left hand side of (\ref{yuiDVA}). 

Choose $\textbf{a}, \textbf{b}, \textbf{c}\in\mathbb{R}^s$ in the support of $\textbf{X}_S$ such that $||f_S(\textbf{a})||_{times}, ||f_S(\textbf{b})||_{times}, ||f_S(\textbf{c})||_{times}$ are distinct and $||f_S(\textbf{b})||_{times}\neq 0$. 

Since $\xi\indep \textbf{X}_S$, then  $\xi\mid [\textbf{X}_S=\textbf{a}] \overset{D}{=}\xi\mid [\textbf{X}_S=\textbf{b}] \overset{D}{=}\xi\mid [\textbf{X}_S=\textbf{c}]$. Hence, 


\begin{equation}\label{asdfghDVA}
h(\textbf{a}) + ||f_S(\textbf{a})||_{times}\cdot Y\overset{D}{=}h(\textbf{b})+||f_S(\textbf{b})||_{times}\cdot Y\overset{D}{=}h(\textbf{c}) + ||f_S(\textbf{c})||_{times}\cdot Y.
\end{equation}
By dividing by a non-zero constant $||f_S(\textbf{b})||_{times}$ and subtracting constant $h(\textbf{b})$, we get
$$
h(\textbf{a}) - h(\textbf{b}) + \frac{||f_S(\textbf{a})||_{times}}{||f_S(\textbf{b})||_{times}}Y \overset{D}{=}Y\overset{D}{=} h(\textbf{c}) - h(\textbf{b}) + \frac{||f_S(\textbf{c})||_{times}}{||f_S(\textbf{b})||_{times}}Y
$$


Now we use lemma \ref{distributionalequalitylemma}. It gives us that $\frac{||f_S(\textbf{a})||_{times}}{||f_S(\textbf{b})||_{times}}=\pm 1$ and also  $\frac{||f_S(\textbf{c})||_{times}}{||f_S(\textbf{b})||_{times}}=\pm 1$. Therefore, at least two values of $||f_S(\textbf{a})||_{times}, ||f_S(\textbf{b})||_{times}, ||f_S(\textbf{c})||_{times}$ must be equal, which is a contradiction since we chose them to be  distinct. 

\textbf{Part 4: } For a contradiction, let $h_1, h_2$ exist. Denote $Y = f(X_{s+1}, \dots, X_k)$. Choose $\textbf{a}, \textbf{b}\in\mathbb{R}^s$ in the support of $\textbf{X}_S$ such that $h_2(\textbf{a})\neq h_2(\textbf{b})\neq 0$. From (\ref{yuiTRI}), we get $h_1(\textbf{a}) + h_2(\textbf{a})Y \overset{D}{=}h_1(\textbf{b}) + h_2(\textbf{b})Y$. By rewriting, we get $\frac{h_1(\textbf{a})-h_1(\textbf{b})}{h_2(\textbf{b})} + \frac{h_2(\textbf{a})}{h_2(\textbf{b})}Y \overset{D}{=}Y$. Applying Lemma  \ref{distributionalequalitylemma}, we obtain $\frac{h_2(\textbf{a})}{h_2(\textbf{b})}=\pm 1$. Since $h_2$ is positive, we get $h_2(\textbf{a}) = h_2(\textbf{b})$. This is a contradiction. 
\end{proof}

\begin{lemma}
\label{lemma_additivity}
Let $X$ be a random variable with strictly increasing distribution function $F_X$. Let $f(x,y)$ be a function for which an inverse with respect to $y$ exists (e.g. if $f$ is injective or strictly increasing and continuous in y, as stated by the Inverse Function Theorem \citep{Inverse_function_theorem}). Let
\begin{equation}
\label{lemma_additivity_equation}
    f(x, X) \overset{D}{=}h_1(x) + h_2(X), \,\,\,\,\,\,\,\,for\,\,all\,\,\,x\in\mathbb{R}, 
\end{equation}
for some functions $h_1, h_2$, where $h_2$ is measurable.

Then, there exist functions $\tilde{h}_1, \tilde{h}_2$, where $\tilde{h}_2$ is measurable, such that 
$$
f(x, y) \overset{}{=}\tilde{h}_1(x) + \tilde{h}_2(y), \,\,\,\,\,\,\,\,for\,\,all\,\,\,x,y\in\mathbb{R}.
$$
\end{lemma}

\begin{proof}
\label{proof of lemma_additivity}Let $g(x, y):=f(x, y) - h_1(x)$. We have that $g(x, X) \overset{D}{=}g(\tilde{x}, X)$ for all $x, \tilde{x}$. 
Let $g^{-1}(x, y)$ denote the inverse of $g(x, y)$ with respect to $y$ for a given $x$. The existence of this inverse follows directly from the existence of an inverse of $f$.  

Let $a\in\mathbb{R}$. Then 
$$ \mathbb P \left (g(x,X) \le a \right)=\mathbb P \left (X \le g^{-1}(x, a) \right)=F_X(g^{-1}(x, a)), \, \forall x\in\mathbb{R},$$where $F_X$ is the distribution function of $X$. 

Therefore
$F_X(g^{-1}(x, a)) = F_X(g^{-1}(\tilde{x}, a))$ for all $x, \tilde{x}\in\mathbb{R}$. Since $F_X$ is strictly increasing function, we obtain $g^{-1}(x, a) = g^{-1}(\tilde{x}, a)$. We showed that $g^{-1}$ does not depend on $x$. Since
$$y=g^{-1}(x, a) \Leftrightarrow g(x,y)=a,$$ we also obtain that $g(x, a) = g(\tilde{x}, a)$. Therefore, $g(x, a)$ does not depend on $x$ and we can write $g(x,y) = \tilde{h}_2(y)$ for some function $\tilde{h}_2$. We showed that 
$f(x, y) = h_1(x) + \tilde{h}_2(y)$ for all $x, y$.     
\end{proof}
\textit{Remark:} We show a periodic function $f$ such that the statement of Lemma~\ref{lemma_additivity} is not valid.

Let $Y \sim N(0,1)$. Define the continuous function $F_Y(y) = P[Y\leq y]$ for $y \in \mathbb{R}$. Note that $F_Y(Y) \sim \mbox{U}(0,1)$.
Define the continuous functions $f:\mathbb{R}^2\rightarrow\mathbb{R}$, $h_1:\mathbb{R}\rightarrow\mathbb{R}$, $h_2:\mathbb{R}\rightarrow\mathbb{R}$ by
\begin{align*}
f(x,y) &= \cos(2\pi F_Y(y) + x) \quad \forall (x,y)\in\mathbb{R}^2\\
h_1(x) &= 0,  \,\,\,\,\,\,\,\,\,
h_2(y)= \cos(2\pi F_Y(y)) \quad \forall x,y \in \mathbb{R}.
\end{align*} 
Then 
$$ f(x,Y) = \cos(2\pi F_Y(Y) + x)\overset{D}{=} \cos(2\pi F_Y(Y)) \quad \forall x \in \mathbb{R}.$$
In particular, 
$$f(x,Y) \overset{D}{=} h_1(x) + h_2(Y) \quad \forall x \in \mathbb{R}.$$
However, $f(x,y)$ does not have the form $\tilde{h}_1(x)+\tilde{h}_2(y)$. 

 


\begin{lemma}
Let $X,Y$ be continuous random variables and $f$ is a (non-random) injective function on the support of $X$. Then, 
\begin{equation}\label{trivial_identity}
X\indep Y \iff f(X)\indep Y.
\end{equation}
\end{lemma}
\begin{proof}
This statement is trivial. 
\end{proof}




