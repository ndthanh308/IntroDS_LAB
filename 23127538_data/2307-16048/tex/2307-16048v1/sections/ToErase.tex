
Given that we have infinite data, a consistent estimation method for (\ref{equation_SID})  and a perfect independence test (“independence oracle”), ISD is correct in a sense that  $\hat{S}_{\mathcal{F}_A}(Y) = S_{\mathcal{F}_A}(Y)$. This statement is made more precise in Lemma~\ref{lemmaISDconsistency}. 


\begin{lemma}\label{lemmaISDconsistency}
Consider $\mathcal{F} = \mathcal{F}_A$ and let $(Y, \textbf{X})\in\mathbb{R}\times \mathbb{R}^p$ follow an SCM with DAG $\mathcal{G}_0$ satisfying (\ref{SCM_for_Y}), with $S_{\mathcal{F}_A}(Y)\neq\emptyset$. 
Then, ISD algorithm used with a consistent estimation (\ref{equation_SID}) and an independence-oracle is guaranteed to estimate $\hat{S}_{\mathcal{F}_A}(Y) = S_{\mathcal{F}_A}(Y)$. 
\end{lemma}
\begin{proof}
   Similar statement can be found about RESIT algorithm \cite{Peters2014}. However here, the proof is trivial.  For every $\mathcal{F}_A$-plausible set $S$, we have that  ${\varepsilon}_S \indep \textbf{X}_S$. Since our estimation is consistent, we also have (in the limit) that
 $\hat{\varepsilon}_S \indep \textbf{X}_S$. Therefore, our independence oracle will output YES on question 1. Trivially, all $\textbf{X}_S$ are significant since  $S$ is $\mathcal{F}_A$-plausible set, and $\hat{\varepsilon}_S \sim U(0,1)$ is trivially satisfied since for $\mathcal{F} = \mathcal{F}_A$ is the third question redundant. Therefore, ISD algorithm will mark the set $S$ as  $\mathcal{F}_A$-plausible and the ISD estimation is correct.  
 \end{proof}




















