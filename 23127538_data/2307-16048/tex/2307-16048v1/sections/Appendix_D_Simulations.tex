\section{Appendix: Simulations and application}
\label{Appendix_Simulations}

\subsection{Functions generated using the Perlin noise approach}\label{Appendix_D1}
In the following, we provide examples of functions generated using the Perlin noise approach. For a one-dimensional case, let $X_1, \eta_Y\overset{iid}{\sim} N(0,1)$ and $Y = g(X_1)+\eta_Y$, where $g$ is generated using the Perlin noise approach. Such (typical) datasets are plotted in Figure~\ref{Figure_perlin_1}. 

For a two-dimensional case, let $X_1,X_2, \eta_Y\overset{iid}{\sim} N(0,1)$ and $Y = g(X_1, X_2)+\eta_Y$, where $g$ is generated using the Perlin noise approach. Such (typical) datasets are plotted in Figure~\ref{Figure_perlin_2}. 

For a three-dimensional case, let $X_1,X_2,X_3, \eta_Y\overset{iid}{\sim} N(0,1)$ and $Y = g(X_1, X_2,X_3)+\eta_Y$, where $g$ is generated using the Perlin noise approach. The visualisation of a four-dimensional dataset is a bit tricky; Figure~\ref{Figure_perlin_3} represents the three-dimensional slices of the function. 
% Figure environment removed






% Figure environment removed











% Figure environment removed









\subsection{Application}\label{Appendix_Application}


% Figure environment removed











