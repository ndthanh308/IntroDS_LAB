
The following lemma shows that the set $\mathcal{F} = \mathcal{I}_m$ is too large to identify any parents of a target variable. 

\begin{lemma}
Let $(X_0,\textbf{X})$ follow ICM model with DAG $\mathcal{G}$, such that its distribution $P_{(X_0,\textbf{X})}$  has a density with respect to Lebesgue measure. Consider a DAG $\tilde{\mathcal{G}}\neq \mathcal{G}$ such that $P_{(X_0,\textbf{X})}$ is Markov with respect to $\tilde{\mathcal{G}}$. Then, there exist ICM model with DAG $\tilde{\mathcal{G}}$ that entails the distribution $P_{(X_0,\textbf{X})}$.    
\end{lemma}
\begin{proof}
In the proof of Proposition 7.1 in \cite{Elements_of_Causal_Inference}
 is described a method how to define a SCM with DAG $\tilde{\mathcal{G}}$  that entails the distribution $P_{(X_0,\textbf{X})}$. However, such defined SCM also satisfies $f_i\in\mathcal{I}_m$ for all $i=0, \dots, p$. 
\end{proof}
