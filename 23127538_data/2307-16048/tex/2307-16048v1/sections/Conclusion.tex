\section{Conclusion and future work}

In this work, we have discussed the problem of estimating direct causes of a target variable $Y$ from an observational joint distribution. We introduced the concept of $\mathcal{F}$-identifiability, a crucial concept that describes the variables that can be inferred as causal under the assumption $f_Y\in\mathcal{F}$. Our theory mainly focuses on describing the set of the $\mathcal{F}$-identifiable parents of $Y$. We explored various choices of $\mathcal{F}$ and developed a formal theory to determine when it is possible to identify all causes of $Y$. Additionally, we presented two algorithms for estimating the set of parents of $Y$ from a random sample. To evaluate the performance of our algorithms, we created several benchmark datasets and demonstrated their effectiveness. We then applied our methodology to a real-world problem related to the causes of the fertility rate.

In this study, we attempted to leverage established results from the theory of identifiability of the entire graph $\mathcal{G}$, but numerous research directions still need to be explored. Our theoretical results apply only to covariates that are neighbours of the target variable. Do our findings encompass non-adjacent variables as well? One limitation of our framework is the necessity of a choice of $\mathcal{F}$. Employing an automatic model selection approach could enhance performance and expand the applicability of this approach. Modifying other causal discovery methodologies on a local scale could also be an interesting future research direction. 

Ultimately, although our findings can provide valuable insights into the causal structure, further research is necessary to thoroughly assess the practical applicability of our approach in real-world settings. We believe that the theory developed in this work can be beneficial for a deeper understanding of causal structure and the theoretical limitations of purely data-driven methodologies for causal inference. 


\section*{Conflict of interest and data availability}
The code and data are available in the \href{https://github.com/jurobodik/Structural-restrictions-in-local-causal-discovery.git}{repository} or on request from the author.   

The authors declare that they have no known competing financial interests or personal relationships that could have appeared to influence the work reported in this paper.


\section*{Acknowledgements}
This study was supported by the Swiss National Science Foundation. 



