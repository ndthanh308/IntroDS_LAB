\subsection{Real-world example}
\label{section_application}
%Change 2 to X_2
%Erase Appendix A.2.
To illustrate our methodology on a real-world example, we consider data on the fertility rate like in \cite{Christina}. The target variable of interest is $Y = \text{`Fertility rate'},$ measured yearly in more than 200 countries.  Developing countries exhibit a significantly higher fertility rate than Western countries \citep{Cheng2022}. The fertility rate can be predicted by considering covariates such as the `infant mortality rate' or `GDP.' However, if one wants to explore the potential effect of a particular law or a policy change, it becomes necessary to leverage the causal knowledge of the underlying system.

Randomized studies are not possible to design in this context since factors like `infant mortality rate' cannot be isolated for manipulation. Even so, understanding the impact of policies to reduce infant mortality rates within a country remains an important question, even if randomized studies are unfeasible.

Here, we consider covariates $\textbf{X} = (X_1, X_2, X_3, X_4)^\top$, where $X_1=$`GDP (in US dollars)', $X_2 = $`Education expenditure (\% of GDP)', $X_3 = $`Infant mortality rate (infant deaths per 1,000 live births)', $X_4 = $`Continent'. The data come from \cite{worldbank_data_about_Education, worldbank_data_about_GDP, unitednations_data_about_fertility}. Visualizations of the data can be found in Appendix~\ref{Appendix_Application}. 

Explaining changes in fertility rate is still a topical issue for which no apparent rational explanation exists. In our study, we focus on using our developed framework to provide data-driven answers about the potential causes of changes in fertility rates. 

In a study by \cite{Christina}, the authors employed a method called invariant causal prediction (ICP; \cite{Peters_invariance}, \cite{PfisterTimeSeries}) to determine the possible causes of $Y$. The ICP approach relies on observing an environmental variable (closely related to a context variable or an instrumental variable; \cite{Mooij2020}). However, selecting a suitable environmental variable can be a subject of debate, and in many cases, it may be challenging to identify one. Nevertheless, we acknowledge the potential benefits of utilizing such a variable, as it can enhance the reliability of the results. 

We apply the methodology developed in this paper to estimate the causes of $Y$. We tried several choices of $\mathcal{F}$; in particular, $\mathcal{F}_A, \mathcal{F}_{LS}, \mathcal{F}_{F_1},\mathcal{F}_{F_2},\mathcal{F}_{F_3} $ for the choices $F_1, F_2, F_3 = Gaussian$, $Gamma$, and $Pareto$, respectively. These candidate choices come from a preliminary inference on the marginal distributions of the data. For the choice $\mathcal{F} = \mathcal{F}_A, \text{ or }\mathcal{F}_{F_3}$, we observe that all sets $S\subseteq \{1, \dots, 4\}$ are strongly rejected as $\mathcal{F}$-plausible and our estimate is an empty set. One reason for this is the restricted aspects provided by these specific choices of $\mathcal{F}$; our data show much more complex relations than those that can be described by just one parameter (the mean in the case $\mathcal{F}_A$ and the tail index in the case $\mathcal{F}_{F_3}$).  

Applying our methodology with the choices $\mathcal{F}_{LS}, \mathcal{F}_{F_1},\mathcal{F}_{F_2}$, we obtain the results described in Table~\ref{Table_application}. The results suggest that $X_3$ is the identifiable cause of $Y$. This is in line with findings from \cite{Christina} (backed up by research from sociology in \cite{hirschman1994}), who also discovered the variable $X_3$ to be causal. Furthermore, the score-based estimate  indicates that $X_2$ is a member of $\widehat{pa}_Y$ across all three selections of $\mathcal{F}$. This suggests that $X_2$ is a cause of $Y$ as well, even though the score-based estimate does not have the same guarantees as the set $\hat{S}_{\mathcal{F}}(Y)$. Note that sets $\{2,3\}, \{1,2,3\}$ are $\mathcal{F}$-plausible for all three choices of $\mathcal{F}$. 

We emphasize that the findings rely on the causal sufficiency of the variables employed, an assumption that can surely be questioned. For instance, other variables such as `religious beliefs' or a `political situation' may explain the fertility rate, but are hard to measure. 


% Please add the following required packages to your document preamble:
% \usepackage{multirow}
\begin{table}[]
\begin{tabular}{|c|c|c|c|}
\hline
\multirow{2}{*}{$\mathcal{F}$}       & \multirow{2}{*}{$\mathcal{F}$-plausible sets} & ISD estimate of the                                       & \multirow{2}{*}{Score-based estimate of $\widehat{pa}_Y$} \\
                                     &                                               & $\mathcal{F}$-identifiable set $\hat{S}_{\mathcal{F}}(Y)$ &                                               \\ \hline
$\mathcal{F}_{LS}$                   & \{2,3\}, \{3,4\}, \{1,2,3\}, \{1,3,4\}        & \{3\}                                                     & \{2,3\}                                       \\ \hline
$\mathcal{F}_{F_1}$                  & \{2,3\}, \{2,3,4\}, \{1,2,3\}, \{1,3,4\}      & \{3\}                                                     & \{1,2,3\}                                     \\ \hline
\multirow{2}{*}{$\mathcal{F}_{F_2}$} & \{3\}, \{2,3\}, \{3,4\}, \{1,2,3\},           & \multirow{2}{*}{\{3\}}                                    & \{1, 2,3,4\} (with almost equal               \\
                                     & \{1,3,4\}, \{2,3,4\},\{1,2,3,4\}              &                                                           & scores for \{1,2,3\}, \{2.3.4\})              \\ \hline
\end{tabular}
\caption{Resulting estimates of causes of changes in fertility rates from real data of Section \ref{section_application}. Here, $F_1$ is a Gaussian distribution, and $F_2$ is a Gamma distribution. }
\label{Table_application}
\end{table}