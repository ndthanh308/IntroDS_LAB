\section*{Appendix C: Log-concave Noises}

In this appendix, we extend the main model in Section 3 by considering noises that have log-concave densities. Formally, $(\varepsilon_i)_{i \in N}$ are independently drawn from distribution $F$, which has a positive continuous density $f$ on the entire real line. We assume, in addition, that $f$ is strictly log-concave and symmetric about zero. Common distributions, such as Gaussian, Laplace, and logistic distributions with mean zero, satisfy these assumptions.

Suppose that the leader uses a monotone strategy with threshold $z \in \RR$. Then a follower with type $x$ has the following posterior density about $\theta$:
\begin{equation*}
    \psi^h(\theta; x, z) = \begin{cases}
    \frac{\frac{1}{\sigma_F}f\left(\frac{x - \theta}{\sigma_F}\right)}{F\left(\frac{x - z}{\sigma_F}\right)}\one(\theta > z) & \text{if $h = \invest$} \\
    ~ & ~ \\
    \frac{\frac{1}{\sigma_F}f\left(\frac{x - \theta}{\sigma_F}\right)}{1 - F\left(\frac{x - z}{\sigma_F}\right)}\one(\theta \leq z) & \text{if $h = \notinvest$}
    \end{cases}.
\end{equation*}
Let $\Psi^h(\cdot; \,x, z)$ be the corresponding CDF.

\begin{lemma} \label{app_c_fosd}
$\Psi^h(\cdot; \,x, z)$ is strictly increasing in $x$ and $z$ in the sense of strict first-order stochastic dominance.
\end{lemma}
\begin{proof}
We only prove the case $h = \invest$, for the proof of the other case, is similar.
Fix $z$ and $\theta > z$, and let $x < x'$. Note that
\begin{equation*}
    \frac{ \psi^\invest (\theta; \,x', z) }{ \psi^\invest (\theta; \,x', z) } = \frac{ F\left(\frac{x - z}{\sigma_F}\right) }{ F\left(\frac{x' - z}{\sigma_F}\right) } \cdot \frac{ f\left(\frac{x' - \theta}{\sigma_F}\right) }{ f\left(\frac{x - \theta}{\sigma_F}\right) }.
\end{equation*}
Since $f$ is logconcave, $f((x'-\theta)/\sigma_F)/f((x-\theta)/\sigma_F)$ is weakly increasing in $\theta$; that is, $\Psi^\invest(\cdot; \,x', z)$ dominates $\Psi^\invest(\cdot; \,x, z)$ in the monotone likelihood ratio order. This follows from the equivalence between log concavity and P\'{o}lya Frequency of order 2 (See, for example, Proposition 1 in \cite{an_1998} or Proposition 2.3 in \cite{saumard_wellner_2014}). 
Thus, $\Psi^\invest(\cdot; \,x', z)$ first-order stochastically dominates $\Psi^\invest(\cdot; \,x, z)$.

The claim that $\Psi^\invest(\cdot; \,x, z)$ is strictly increasing in the first-order stochastic dominance sense is because
\begin{equation*}
    \Psi^\invest(\theta; \,x, z) = \frac{1}{ F\left(\frac{x-z}{\sigma_F}\right)  }\int_z^\theta \frac{1}{\sigma_F} f\left(\frac{x-t}{\sigma_F}\right)  \dd t = 1 - \frac{ F\left(\frac{x-\theta}{\sigma_F}\right) }{ F\left(\frac{x-z}{\sigma_F}\right) }
\end{equation*}
is strictly increasing in $z$.
\end{proof}



Let $\eta = \lim_{x \to -\infty} F(x)/f(x)$. It is worth noting that $\eta$ is the scale parameter if the noises follow a Laplace or logistic distribution.
\begin{lemma} \label{app_c_exp}
The posterior expectations $\EE_{\theta \sim \Psi^h(\cdot; \,x, z)}[\theta]$ are strictly increasing in $x$ and $z$. Moreover,
\begin{equation*}
    \lim_{x \to -\infty} \EE_{\theta \sim \Psi^h(\cdot; \,x, z)}[\theta] = \begin{cases}
    \sigma_F \eta + z & \text{if $h = \invest$} \\
    - \infty & \text{if $h = \notinvest$}
    \end{cases},
\end{equation*}
and 
\begin{equation*}
    \lim_{x \to \infty} \EE_{\theta \sim \Psi^h(\cdot ; \,x, z)}[\theta] = \begin{cases}
    \infty & \text{if $h = \invest$} \\
    -\sigma_F \eta + z & \text{if $h = \notinvest$}
    \end{cases}.
\end{equation*}
\end{lemma}
\begin{proof}
(Part 1) The first part of this lemma is given by Lemma \ref{app_c_fosd}.

\noindent (Part 2) Consider first history $h = \invest$.
By a change of variable, we have
\begin{equation*}
    \EE_{\theta \sim \Psi^\invest(\cdot; \,x, z)}[\theta]  = \frac{1 }{F\left(\frac{x - z}{\sigma_F}\right)} \int_z^\infty  \frac{t}{\sigma_F} f\left(\frac{x - t}{\sigma_F}\right) \dd t = \sigma_F \delta\left( \frac{x - z}{\sigma_F} \right) + z.
\end{equation*}
where
\begin{equation*}
    \delta(u) = u - \frac{ \int_{-\infty}^u t f(t) \dd t }{ F(u) }.
\end{equation*}
It is clear that $\lim_{u \to \infty} \delta(u) = \infty$, and hence
\begin{equation*}
    \lim_{x \to \infty} \EE_{\theta \sim \Psi^h(\cdot ; \,x, z)}[\theta] = \infty.
\end{equation*}
Since $f$ is log-concave, it has a light right tail; that is,
\begin{equation*}
    \lim_{x \to \infty} \frac{ f(x) }{ \ee^{-cx} } = 0
\end{equation*}
for some $c > 0$.\footnote{~See, for example, Corallary 1 in \cite{an_1998}.} Thus, by symmetry (about zero),
\begin{equation*}
    \lim_{x \to -\infty} x F(x) = \lim_{x \to -\infty} \frac{ F(x) }{ x^{-1} } = \lim_{x \to -\infty} \frac{ f(x) }{ -x^{-2} } = \lim_{x \to -\infty} \frac{ f(x) }{ \ee^{cx} } \cdot \frac{ \ee^{cx} }{ -x^{-2} } = 0.
\end{equation*} 
Integrating by parts now gives 
\begin{equation*}
    \delta(u) = \frac{ \int_{-\infty}^u F(t) \dd t }{ F(u) }.
\end{equation*}
It follows that $\lim_{u \to -\infty} \delta(u) = \eta$. Thus,
\begin{equation*}
    \lim_{x \to -\infty} \EE_{\theta \sim \Psi^h(\cdot ; \,x, z)}[\theta] = \sigma_F \eta + z.
\end{equation*}

The proof for history $h = \notinvest$ is similar.
We have
\begin{equation*}
    \EE_{\theta \sim \Psi^\notinvest(\cdot; \,x, z)}[\theta] = \frac{1}{ 1 - F\left(\frac{x - z}{\sigma_F}\right) } \int_{-\infty}^z \frac{t}{\sigma_F} f\left( \frac{x - t}{\sigma_F} \right) \dd t = -\sigma_F \varsigma \left(\frac{x - z}{\sigma_F} \right) + z,
\end{equation*}
where
\begin{equation*}
    \varsigma(u) = \frac{ \int_u^\infty t f(t) \dd t }{ 1 - F(u) } - u.
\end{equation*}
Thus, $\lim_{u \to -\infty} \varsigma(u) = \infty$, and
\begin{equation*}
    \lim_{x \to -\infty} \EE_{\theta \sim \Psi^\notinvest(\cdot; \,x, z)}[\theta] = -\infty.
\end{equation*}
Since $\lim_{u \to \infty} u \left( 1 - F(u) \right) = 0$ (because $f$ is light-tailed),
\begin{equation*}
    \varsigma(u) = \frac{\int_u^\infty [1 - F(t)] \dd t}{1 - F(u)}
\end{equation*}
by integration by parts. Thus,
\begin{equation*}
    \lim_{u \to \infty} \varsigma(u) = \lim_{u \to \infty} \frac{1 - F(u)}{f(u)} = \lim_{u \to \infty} \frac{ F(-u) }{ f(-u) } = \eta,
\end{equation*}
which implies that
\begin{equation*}
    \lim_{x \to \infty} \EE_{\theta \sim \Psi^\notinvest(\cdot; \,x, z)}[\theta] = -\sigma_F \eta + z.
\end{equation*}
The proof is complete.
\end{proof}

Suppose, in addition, that a follower with type $x$ believes that other followers use monotone strategies with threshold $x_h$ under history $h$, then his payoff under history $h$
\begin{equation*}
    \pi_F^h(x; z, x_h) = \EE_{\theta \sim \Psi^h(\cdot; \,x, z)} \left[ \theta - \frac{n-1}{n}F\left( \frac{x_h - \theta}{\sigma_F} \right) \right] - \frac{\chi_\notinvest}{n}
\end{equation*}
has the following properties:


\begin{lemma} \label{app_c_fol_payoff}
Type $x$'s payoffs $\pi_F^h(x; z, x_h)$ are strictly increasing in $x$ and $z$ but is strictly decreasing in $x_h$. Moreover,
\begin{equation*}
   \lim_{x \to -\infty} \pi_F^h(x; z, x_h) = \begin{cases}
        \sigma_F \eta + z - \frac{n-1}{n}F\left( \frac{x_\invest - z}{\sigma_F} \right) & \text{if $h = \invest$} \\
        -\infty & \text{if $h = \notinvest$}
    \end{cases}
\end{equation*}
and
\begin{equation*}
    \lim_{x \to \infty} \pi_F^h(x; z, x_h) = \begin{cases}
        \infty & \text{if $h = \invest$} \\
        -\sigma_F \eta + z - \frac{1}{n} - \frac{n-1}{n}F\left( \frac{x_\notinvest - z}{\sigma_F} \right) & \text{if $h = \notinvest$}
    \end{cases}
\end{equation*}
\end{lemma}
\begin{proof}
(Part 1) Note that $\theta - ((n-1)/n)F((x_h - \theta)/\sigma_F)$ is strictly increasing in $\theta$; therefore Lemma \ref{app_c_fosd} implies that $\pi_F^h(x; z, x_h)$ is strictly increasing in $x$ and $z$. But since $\theta - ((n-1)/n)F((x_h - \theta)/\sigma_F)$ is strictly decreasing in $x_h$, so is $\pi_F^h(x; z, x_h)$.

(Part 2)
It follows immediately from Lemma \ref{app_c_exp} that $\pi_F^\invest(x; z, x_\invest) \to \infty$ as $x \to \infty$
and $\pi_F^\notinvest(x; z, x_\notinvest) \to -\infty$ as $x \to -\infty$
because $F((x_h - \theta)/\sigma_F)$ is bounded.



Now under $h = \invest$, a change of variable $u = (x - \theta)/\sigma_F$ gives that
\begin{align*}
    \EE_{\theta \sim \Psi^\invest(\cdot;\,x, z)}\left[ F\left(\frac{x_\invest - \theta}{\sigma_F}\right)\right] & = \frac{1}{F\left(\frac{x - z}{\sigma_F}\right)} \int_z^\infty F\left(\frac{x_\invest - \theta}{\sigma_F}\right) \frac{1}{\sigma_F}f\left(\frac{x - \theta}{\sigma_F}\right) \dd \theta \\
    & = \frac{1}{F\left(\frac{x - z}{\sigma_F}\right)} \int_{-\infty}^\frac{x-z}{\sigma_F} f(u) F\left(\frac{x_\invest - x}{\sigma_F} + u \right) \dd u.
\end{align*}
By L'H\^{o}pital's rule, the above expectation has the following limit:
\begin{equation*}
    \lim_{x \to -\infty} \EE_{\theta \sim \Psi^\invest(\cdot;\,x, z)}\left[ F\left(\frac{x_\invest - \theta}{\sigma_F}\right)\right] = F \left( \frac{x_\invest - z}{\sigma_F} \right).
\end{equation*}
Thus, by Lemma \ref{app_c_exp}, we can conclude that
\begin{equation*}
    \lim_{x \to -\infty} \pi_F^\invest(x; z, x_\invest) = \sigma_F \eta + z - \frac{n-1}{n} F \left( \frac{x_\invest - z}{\sigma_F} \right).
\end{equation*}
Similarly, under $h = \notinvest$,
\begin{align*}
    \lim_{x \to \infty} \EE_{\theta \sim \Psi^\notinvest(\cdot;\,x, z)}\left[ F\left(\frac{x_\notinvest - \theta}{\sigma_F}\right)\right] & = \lim_{x \to \infty} \frac{1}{F\left( \frac{z-x}{\sigma_F} \right)} \int_{-\infty}^z F \left( \frac{x_\notinvest - \theta}{\sigma_F} \right) \frac{1}{\sigma_F} f\left( \frac{x-\theta}{\sigma_F} \right) \dd \theta \\
    & = \lim_{x \to \infty} \frac{1}{F\left( \frac{z-x}{\sigma_F} \right)} \int_{\frac{x-z}{\sigma_F}}^\infty f(u) F \left( \frac{x_\notinvest - x}{\sigma_F} + u \right) \dd u \\
    & = F \left( \frac{x_\notinvest - z}{\sigma_F} \right).
\end{align*}
Thus, 
\begin{equation*}
    \lim_{x \to \infty} \pi_F^\notinvest (x; z, x_\notinvest) = -\sigma_F \eta + z - \frac{1}{n} - \frac{n-1}{n} F \left( \frac{x_\notinvest - z}{\sigma_F} \right),
\end{equation*}
as desired.
\end{proof}

Let $(\thetalow^k, \thetaup^k, \xiup^k, \xilow^k, \xnup^k, \xnlow^k)_{k=0}^\infty$ be the six $\Delta$-rationalizable sequences, where $\thetalow^0 = \xilow^0 = \xnlow^0 = -\infty$ and $\thetaup^0 = \xiup^0 = \xnup^0 = \infty$.
Let $\thetalow$, $\thetaup$, $\xilow$, $\xiup$, $\xnlow$, $\xnup$ be their limits, respectively, as $k \to \infty$. If one sequence diverges, then its limit is either $-\infty$ or $\infty$. 

\begin{proposition} \label{app_c_rat_seq}
    If $\sigma_F \geq (n\eta)^{-1}(n-1)$, then the above sequences have the following properties: \\
    (a) $(\thetalow^k)_{k=0}^\infty$ is such that $\thetalow^k = \thetalow = 0$ for all $k \geq 1$; \\
    (b) $(\thetaup^k)_{k=0}^\infty$ is decreasing and  such that $\thetaup^k = \thetaup = 0$ for all $k \geq 2$; \\
    (c) $(\xilow^k)_{k=0}^\infty$ is such that $\xilow^k = \xilow = -\infty$ for all $k \geq 0$; \\
    (d) $(\xiup^k)_{k=0}^\infty$ is decreasing and such that $\xiup^k = \xiup -\infty$ for all $k \geq 1$; \\
    (e) $(\xnlow^k)_{k=0}^\infty$ is increasing and such that $\xnlow^k = \xnlow = \infty$ for all $k \geq 1$; \\
    (f) $(\xnup^k)_{k=0}^\infty$ is such that $\xnup^k = \xnup = \infty$ for all $k \geq 0$.
\end{proposition}
\begin{proof}
\textit{Round $k = 1$}: First note that the best-case and worst-case payoffs to leader $\theta$ are given by
\[
\pi_L(\theta; \xilow^0) = \theta - F \left( \frac{\xilow^0 - \theta}{\sigma_F} \right) = \theta
\]
and 
\[
\pi_L(\theta; \xiup^0) = \theta - F \left( \frac{\xiup^0 - \theta}{\sigma_F} \right) = \theta -1,
\]
respectively.
This implies that $\thetalow^1 = 0$ and $\thetaup^1 = 1$. That is, it is dominant for all leader types below $\thetalow^1$ to take $a_L = \notinvest$ and dominant for all leader types above $\thetaup^1$ to choose $a_L = \invest$.

Given $\thetalow^1$ and $\thetaup^1$. By Lemma \ref{app_c_fol_payoff}, follower $x$'s best-case payoff under $h = \invest$ is
\[
\pi_F^\invest(x; \thetaup^1, \xilow^0) = \EE_{\theta \sim \Psi^\invest(\cdot; x, \thetaup^1)}\left[\theta - \frac{n-1}{n} F\left( \frac{\xilow^0 - \theta}{\sigma_F} \right) \right] = \EE_{\theta \sim \Psi^\invest(\cdot; x, \thetaup^1)} [\theta].
\]
Since $\lim_{x \to -\infty} \EE_{\theta \sim \Psi^\invest(\cdot; x, \thetaup^1)} [\theta] = \sigma_F \eta + 1 > 0$ by Lemma \ref{app_c_exp}, we have $\xilow^1 = -\infty$, meaning that there is no follower type for whom action $\invest$ is strictly dominated. Analogously, the worst-case payoff to follower $x$ yields
\begin{align*}
  \pi_F^\invest(x; \thetalow^1, \xiup^0) & = \EE_{\theta \sim \Psi^\invest(\cdot; x, \thetalow^1)}\left[\theta - \frac{n-1}{n} F\left( \frac{\xiup^0 - \theta}{\sigma_F} \right) \right] \\
  & = \EE_{\theta \sim \Psi^\invest(\cdot; x, \thetalow^1)} [\theta] - \frac{n-1}{n},
\end{align*}
which has the limit of $\sigma_F \eta - (n-1)/n$ as $x \to -\infty$. But since $\sigma_F \geq (n-1)/(n\eta)$, it follows that
\[
\lim_{x \to -\infty} \pi_F^\invest(x; \thetalow^1, \xiup^0) = \sigma_F \eta - \frac{n-1}{n} \geq 0.
\]
Thus, $\xiup^1 = -\infty$ and it is dominant for all follower types to take action $\invest$.

Under $h = \notinvest$, the worst-case payoff to follower $x$ is
\begin{align*}
    \pi_F^\notinvest(x; \thetalow^1, \xnup^0) & = \EE_{\theta \sim \Psi^\notinvest(\cdot; x, \thetalow^1)} \left[\theta - \frac{n-1}{n} F\left( \frac{\xnup^0 - \theta}{\sigma_F} \right) \right] - \frac{1}{n} \\ 
    & = \EE_{\theta \sim \Psi^\notinvest(\cdot; x, \thetalow^1)} [\theta] - 1.
\end{align*}
By Lemma \ref{app_c_exp}, we have
\[
\lim_{x \to \infty} \pi_F^\notinvest(x; \thetalow^1, \xnup^0) = - \sigma_F \eta - 1 < 0.
\]
Therefore $\xnup^1 = \infty$; i.e., there exists no follower type for whom $\invest$ is a dominant action. Follower $x$'s best-case payoff exhibits 
\begin{align*}
    \pi_F^\notinvest(x; \thetaup^1, \xnlow^0) & = \EE_{\theta \sim \Psi^\notinvest(\cdot; x, \thetaup^1)} \left[\theta - \frac{n-1}{n} F\left( \frac{\xnlow^0 - \theta}{\sigma_F} \right) \right] - \frac{1}{n} \\
    & = \EE_{\theta \sim \Psi^\notinvest(\cdot; x, \thetaup^1)} [\theta] - \frac{1}{n}
\end{align*}
and
\begin{align*}
    \lim_{x \to \infty} \pi_F^\notinvest(x; \thetaup^1, \xnlow^0) = -\sigma_F \eta + 1 - \frac{1}{n} \leq -\left(\frac{n-1}{n \eta}\right) \eta + \frac{n-1}{n} = 0. 
\end{align*}
Thus we have $\xnlow^1 = \infty$; i.e., $\notinvest$ is a dominant action for all follower types.

\textit{Round $k = 2$}: Given $\xilow^1 = \xiup^1 = -\infty$, leader $\theta$'s worst-case and best-case payoffs coincide:
\begin{equation*}
    \pi_L(\theta; \xiup^1) = \theta - F \left( \frac{\xiup^1 - \theta}{\sigma_F} \right) = \theta.
\end{equation*}
This immediately implies that $\thetalow^2 = \thetaup^2 = 0$.

Given $\thetalow^2$ and $\thetaup^2$, follower $x$'s worst-case and best-case payoffs, under history $h = \invest$, also coincide because $\thetalow^2 = \thetaup^2 = 0$ and $\xilow^1 = \xiup^1 = -\infty$. Moreover, by Lemma \ref{app_c_exp},
\begin{equation*}
    \lim_{x \to -\infty} \pi_F^\invest(x; \thetaup^2, \xilow^2) = \sigma_F \eta > 0.
\end{equation*}
Thus, $\xilow^2 = \xiup^2 = -\infty$. One can show analogously that $\xnlow^2 = \xnup^2 = \infty$ because $\thetalow^2 = \thetaup^2 = 0$ and $\xnlow^1 = \xnup^1 = \infty$.

Now we can conclude that (a)-(f) hold true via a simple induction argument.
\end{proof}

\begin{corollary}
If $\sigma_F \geq (n\eta)^{-1}(n-1)$, then the game has a unique $\Delta$-rationalizable strategy profile in which all followers imitate the leader's choice of action.
\end{corollary}

Proposition \ref{app_c_rat_seq} says that sequences $(\thetalow^k)_{k=0}^\infty$, $(\xilow^k)_{k=0}^\infty$, and $(\xnup^k)_{k=0}^\infty$ always converge. However, it is 
worth noting that we have provided a strong sufficient condition so that a unique $\Delta$-rationalizable behavior is achieved in Round 2 (i.e., 
the other three sequences to converge in Round 2). In fact, 
if the following condition holds
\begin{equation*}
    \max\left\{ \iota_\invest^{k}(\sigma_F), \iota_\notinvest^{k}(\sigma_F) \right\} \leq \sigma_F < \min \left\{ \iota_\invest^{k-1}(\sigma_F), \iota_\notinvest^{k-1}(\sigma_F)  \right\},
\end{equation*}
where
\begin{equation*}
\iota_\invest^k(\sigma_F) = \frac{n-1}{n\eta} F\left( \frac{\xiup^{k-1}}{\sigma_F} \right)
\end{equation*}
and
\begin{equation*}
\iota_\notinvest^k(\sigma_F) = \frac{1}{\eta}\left[ \thetaup^k - \frac{1}{n} - \frac{n-1}{n} F \left( \frac{\xnlow^{k-1} - \thetaup^k}{\sigma_F} \right) \right],
\end{equation*}
then the six sequences will converge to the unique $\Delta$-rationalizable profile in Round $k \geq 2$.
The next proposition shows that there exist multiple $\Delta$-rationalizable profiles when $\sigma_F \to 0$.

\begin{proposition}
    In the limit as $\sigma_F \to 0$, we have $\thetaup \to (n-1)/(2n)$, $\xiup \to (n-1)/(2n)$, and $\xnlow \to \infty$.
\end{proposition}
\begin{proof}
    The proof is identical to that of Proposition \ref{prop_limit}.
\end{proof}
