\section*{Appendix A: Proofs }

\subsection*{Proof of Lemma \ref{lemma_truncated_expectation}} First note that  type $x$'s belief about $\theta$ has a Gaussian distribution with mean $x$ and variance $\sigma_F^2$. 
Recall that $\lambda(x) = \phi(x)/\Phi(x)$ is the reversed hazard rate of a standard Gaussian random variable. 
Rewrite Equation (\ref{eq_expectation_of_theta}) as follows:
\begin{equation*} 
    \EE_{\theta \sim \Psi^h(\cdot; \,x, z)}[\theta] = \begin{cases}
   \sigma_F \left[y + \lambda(y) \right] + z  & \text{if $h = \invest$} \\
   ~ & ~ \\
   -\sigma_F \left[-y + \lambda(-y) \right] + z & \text{if $h = \notinvest$}
   \end{cases},
\end{equation*}
where $y = (x - z)/\sigma_F$. 
Since $-1 < \lambda'(x) < 0$ \citep{sampford_1953},
$x + \lambda(x)$ is strictly increasing in $x$. It is now straightforward to see that $\EE_{\theta \sim \Psi^h(\cdot; \,x, z)}[\theta]$ is strictly increasing in $x$ and $z$ because $\lambda(\cdot)$ is a  strictly decreasing function and $\sigma_F > 0$.

Using the derivative formula $\phi'(x) = -x \phi(x)$ and L'H\^{o}pital's rule, we have $\lim_{x \to -\infty} x + \lambda(x) = \lim_{x \to -\infty} \phi(x)/\phi'(x) = 0$ and hence $x + \lambda(x) > 0$ for all $x \in \RR$. We also have $\lim_{x \to \infty} x + \lambda(x) = \infty$ because $\lim_{x \to \infty} \lambda(x) = 0$. Thus,
\begin{equation*}
    \EE_{\theta \sim \Psi^\invest(\cdot; \,x, z)}[\theta] \to \begin{cases}
   z & \text{as $x \to -\infty$} \\
   \infty & \text{as $x \to \infty$}
   \end{cases},
\end{equation*}
and
\begin{equation*}
    \EE_{\theta \sim \Psi^\notinvest(\cdot; \,x, z)}[\theta] \to \begin{cases}
   -\infty & \text{as $x \to -\infty$} \\
   z & \text{as $x \to \infty$}
   \end{cases}.
\end{equation*}
This completes the proof. \qed


\medskip
\subsection*{Proof of Lemma \ref{lemma_x_payoff}}
(Part 1)
By Lemma \ref{lemma_truncated_expectation}, it suffices to show that $\Psi^h(\theta; x, z)$ is increasing in $x$ and $z$ in the sense of first-order stochastic dominance
because $\Phi\left((x_h - \theta)/\sigma_F\right)$ is strictly increasing in $\theta$ and strictly decreasing in $x_h$.
 
Given $h = \invest$. For $x' > x$, we have
\begin{equation*}
    \frac{\psi^\invest(\theta; x', z)}{\psi^\invest(\theta; x
    , z)} = \frac{\phi\left(\frac{\theta - x'}{\sigma_F}\right)}{\phi\left(\frac{\theta - x}{\sigma_F}\right)} \cdot \frac{\Phi\left(\frac{x - z}{\sigma_F}\right)}{\Phi\left(\frac{x' - z}{\sigma_F}\right)}.
\end{equation*}
Since $\phi(\cdot)$ is   log-concave, $\phi\left((\theta - x')/\sigma_F\right)/\phi\left((\theta - x)/\sigma_F\right)$ is increasing in $\theta$; i.e., $\phi(\cdot)$ is a P\'{o}lya frequency function of order 2 \citep[Proposition 2.3]{saumard_wellner_2014}, and so is $\psi^\invest(\theta; x', z)/\psi^\invest(\theta; x, z)$. This implies that $\psi^\invest(\theta; x, z)$ is log-supermodular in $(\theta, x)$, or, equivalently, $\Psi^\invest(\theta; x',z)$   dominates $\Psi^\invest(\theta; x,z)$ in the monotone likelihood ratio order. Thus, $\Psi^\invest(\theta; x',z)$   first-order stochastically dominates $\Psi^\invest(\theta; x,z)$. 

By definition, we have
\begin{equation*}
    \Psi^\invest(\theta; x, z) = 1 - \frac{\Phi\left(\frac{x - \theta}{\sigma_F}\right)}{\Phi\left(\frac{x - z}{\sigma_F}\right)}.
\end{equation*}
It is decreasing in $z$; that is, $\Psi^\invest(\theta; x, z') < \Psi^\invest(\theta; x, z)$ for $z' > z$ and for all $\theta \in (z, \infty)$, where $\Psi^\invest(\theta; x, z') = 0$ when $\theta \in (z, z']$. This proves that $\Psi(\theta; x, z')$ dominates $\Psi(\theta; x, z)$ in the first-order stochastic dominance sense. The proof for $\Psi^\notinvest(\theta; x, z)$ is analogous.

\noindent (Part 2)  Since $\Phi\left( (x_h - \theta)/\sigma_F \right)$ is bounded, it follows from Lemma \ref{lemma_truncated_expectation} that $\pi_F^\notinvest(x; z, x_\notinvest)$ $\to -\infty$ as $x \to -\infty$ and $\pi_F^\invest(x; z, x_\invest)$ $\to \infty$ as $x \to \infty$. Now under $h = \invest$,
\begin{align*}
    \EE_{\theta \sim \Psi^\invest(\cdot; \,x, z)} \left[ \Phi\left(\frac{x_\invest - \theta}{\sigma_F}\right) \right] & = \frac{1}{\Phi\left( \frac{x - z}{\sigma_F} \right)} \int_z^\infty \Phi\left( \frac{x_\invest - t}{\sigma_F} \right) \frac{1}{\sigma_F} \phi\left( \frac{x - t}{\sigma_F} \right) \dd t \\
    & = \frac{1}{\Phi\left( \frac{x - z}{\sigma_F} \right)} \int_{-\infty}^{\frac{x - z}{\sigma_F}} \phi(\eta) \Phi\left( \frac{x_\invest - x}{\sigma_F} + \eta \right) \dd \eta \\
    & \to \Phi\left( \frac{x_\invest - z}{\sigma_F} \right) \quad \text{as $x \to -\infty$}.
\end{align*}
The second equality is given by a change of variable $\eta = (x - t)/\sigma_F$, and the limiting result is a consequence of applying L'H\^{o}pital's rule. Thus, Lemma \ref{lemma_truncated_expectation} implies that $\lim_{x \to -\infty} \pi_F^\invest(x; z, x_\invest)$ $= z - ((n-1)/n)\Phi(( x_\invest - z )/\sigma_F)$.

Similarly, under $h = \notinvest$, we have 
\begin{align*}
    \EE_{\theta \sim \Psi^\notinvest(\cdot; \,x, z)} \left[ \Phi\left(\frac{x_\notinvest - \theta}{\sigma_F}\right) \right] & = \frac{1}{\Phi\left( \frac{z - x}{\sigma_F} \right)} \int_{-\infty}^z \Phi\left( \frac{x_\notinvest - t}{\sigma_F} \right) \frac{1}{\sigma_F} \phi\left( \frac{x - t}{\sigma_F} \right) \dd t \\
    & = \frac{1}{\Phi\left( \frac{z - x}{\sigma_F} \right)} \int_{\frac{x - z}{\sigma_F}}^\infty \phi(\eta) \Phi\left( \frac{x_\notinvest - x}{\sigma_F} + \eta \right) \dd \eta \\
    & \to \Phi\left( \frac{x_\notinvest - z}{\sigma_F} \right) \quad \text{as $x \to \infty$}.
\end{align*}
Thus, by Lemma \ref{lemma_truncated_expectation}, $\lim_{x \to \infty} \pi_F^\notinvest(x; z, x_\notinvest) = z - 1/n - ((n-1)/n) \Phi((x_\notinvest - z)/\sigma_F)$.
\qed

\medskip
\subsection*{Proof of Lemma \ref{lemma_rat_seq} }
Define iteratively six sequences as follows. Let $\thetalow^0 = \xilow^0 = \xnlow^0 = -\infty$ and $\thetaup^0 = \xiup^0 = \xnup^0 = \infty$, and for $k \geq 1$, 
\begin{equation*}
    \begin{cases}
        \thetalow^k = \br_L(\xilow^{k-1}) & \\
        \thetaup^k = \br_L(\xiup^{k-1}) & \\
        \xilow^k = \br_F^\invest(\thetaup^k, \xilow^{k-1}) & \\
        \xiup^k = \br_F^\invest(\thetalow^k, \xiup^{k-1}) & \\
        \xnlow^k = \br_F^\notinvest(\thetaup^k, \xnlow^{k-1}) & \\
        \xnup^k = \br_F^\notinvest(\thetalow^k, \xnup^{k-1})
    \end{cases},
\end{equation*}
where $\theta = \br_L(x)$, $x \in \RR \cup \{-\infty, \infty\}$, is 
the unique solution to
\begin{equation*}
   \pi_L(\theta; x) = \theta - \Phi\left(\frac{x - \theta}{\sigma_F}\right) = 0,
\end{equation*}
and, $\br_F^h(\theta', x')$, $(\theta', x') \in \RR \times \RR \cup \{-\infty, \infty\}$, is the unique value of $x$, if exists, that solves
\begin{equation*} 
     \pi_F^h(x; \theta', x') = \EE_{\theta \sim \Psi^h(\cdot; \,x, \theta')}\left[ \theta - \frac{n-1}{n} \Phi\left( \frac{x' - \theta}{\sigma_F} \right) \right] - \frac{\chi_\notinvest}{n} = 0; 
\end{equation*}
otherwise
\begin{equation*}
    \br_F^h(\theta', x') = \begin{cases}
        -\infty & \text{if $h = \invest$} \\
        \infty & \text{if $h = \notinvest$}
    \end{cases}.
\end{equation*}
Now we prove statements (a)-(f) as follows by induction.

\vskip 0.5 \baselineskip
\noindent Parts (a), (b), (c) \& (d): For $k = 1$, 
we have $\thetalow^1 = 0$ and $1 = \thetaup^1 < \thetaup^0$
because $\xilow^0 = -\infty$ and $\xiup^0 = \infty$. It follows that $\xilow^1 = \br_F^\invest(1, -\infty) = - \infty$ because $\EE_{\theta \sim \Psi^\invest(\cdot; \,x, 1)}[\theta] > 0$ for all $x$ by Lemma \ref{lemma_truncated_expectation}. Also, $\xiup^1 = \br_F^\invest(0, \infty)$ is the unique solution to $\EE_{\theta \sim \Psi^\invest(\cdot; \,\xiup^1, 0)}[\theta] = (n-1)/n$; therefore $\xiup^1 < \xiup^0$.

Suppose now that $\thetalow^k = 0$, $\thetaup^k < \thetaup^{k-1}$, $\xilow^k = -\infty$, and $\xiup^k < \xiup^{k -1}$ for any given $k \geq 2$. Then $\thetalow^{k+1} = \br_L(\xilow^k) = \br_L(-\infty) =  0$, and $0 < \thetaup^{k+1} = \br_L(\xiup^k) < \br_L(\xiup^{k-1}) = \thetaup^k$ because the leader's payoff is strictly decreasing in $x_\invest$. Furthermore, 
$\xilow^{k+1} = \br_F^\invest(\thetaup^{k+1}, \xilow^k) = \br_F^\invest(\thetaup^{k+1}, -\infty) = -\infty$
because $\EE_{\theta \sim \Psi^\invest(\cdot; \,x, \thetaup^{k+1})}[\theta] > \EE_{\theta \sim \Psi^\invest(\cdot; \,x, 0}[\theta] > 0$ for all $x$ by Lemma \ref{lemma_truncated_expectation}. Since Lemma \ref{lemma_x_payoff} implies that $\br_F^\invest(0, x)$ is strictly increasing in $x$, it is now clear that
\begin{equation*}
    \xiup^{k+1} = \br_F^\invest(\thetalow^{k+1}, \xiup^k) = \br_F^\invest(0, \xiup^k) < \br_F^\invest(0, \xiup^{k-1}) = \br_F^\invest(\thetalow^k, \xiup^{k-1}) = \xiup^k.
\end{equation*}
This completes the proof of parts (a), (b), (c), and (d).

\vskip 0.5 \baselineskip
\noindent Part (e): For $k = 1$, $\xnlow^1 = \br_F^\notinvest(\thetaup^1, \xnlow^0) = \br_F^\notinvest(\thetaup^1, -\infty) > \xnlow^0$ because, by Lemma \ref{lemma_truncated_expectation}, $\EE_{\theta \sim \Psi^\notinvest(\cdot; \,x, \thetaup^1)}[\theta] = 1/n$ has a unique solution.
Suppose now that $\xnlow^k > \xnlow^{k-1}$ for any given $k \geq 2$.
We know from Lemma \ref{lemma_x_payoff} that $\br_F^\notinvest(\theta, x)$ is strictly decreasing in $\theta$ but strictly increasing in $x$. Thus, $\xnlow^{k+1} = \br_F^\notinvest(\thetaup^{k+1}, \xnlow^k) > \br_F^\notinvest(\thetaup^k, \xnlow^k) > \br_F^\notinvest(\thetaup^k, \xnlow^{k-1}) = \xnlow^k$.

\vskip 0.5 \baselineskip
\noindent Part (f): For $k=1$, we know by Lemma \ref{lemma_truncated_expectation} that $\pi_F^\notinvest(x; \thetalow^1, \xnup^0) = \EE_{\theta \sim \Psi^\notinvest(\cdot; \,x, 0)}[\theta] - 1 < -1$ for all $x$. This implies that $\xnup^1 = \infty$.
Suppose that $\xnup^k = \infty$ for any given $k \geq 2$. We again have $\pi_F^\notinvest(x; \thetalow^k, \xnup^0) = \EE_{\theta \sim \Psi^\notinvest(\cdot; \,x, 0)}[\theta] - 1 < -1
$ for all $x$. Thus, $\xnup^{k+1} = \infty$. \qed




\medskip
\subsection*{Proof of Proposition \ref{prop_unique_rat} }
The model has a unique $\Delta$-rationalizable behavior if and only if $\thetaup = 0$, $\xiup = -\infty$, and $\xnlow = \infty$.
If $\xiup = -\infty$, Equation (\ref{leader_upper}) implies that
$\thetaup = 0$. By Equation (\ref{follower_notinvest_lower}),
it follows that $\xnup = \infty$ because we know from Equations (\ref{eq_expectation_of_theta}) and (\ref{follower_payoff_rank_belief}) that
\begin{equation*}
    \pi_F^\notinvest(x; 0, x) = x - \sigma_F \lambda\left( -\frac{x}{\sigma_F} \right) - \frac{n-1}{2n} \Phi \left(\frac{x}{\sigma_F}\right) - \frac{n+1}{2n} < 0
\end{equation*}
for all $x$. However, if there exists a value of $x$ such that $\pi_F^\invest(x; 0, x) = 0$, then 
\begin{equation*}
    \xiup = \max \left\{x \vl \pi_F^\invest(x; 0, x) =0 \right\}
\end{equation*}
and hence $\thetaup > \thetalow$. Thus, a unique $\Delta$-rationalizable behavior obtains if and only if $\xiup = -\infty$. It is worth noting that 
$\xiup > -\infty$ does not necessarily imply
that $\xnlow < \infty$.

Now we show that there exists a unique $\widehat{\sigma}_F$ such that $\xiup = -\infty$ if and only if $\sigma_F > \widehat{\sigma}_F$.
Again by Equations (\ref{eq_expectation_of_theta}) and (\ref{follower_payoff_rank_belief}) we can write $\pi_F^\invest(x; 0, x) = 0$ as
\begin{equation} \label{app_eq_fb}
    x = \frac{n-1}{2n}\Phi\left(\frac{x}{\sigma_F}\right) - \sigma_F \lambda\left(\frac{x}{\sigma_F}\right). \tag{A.1}
\end{equation}
Define $\rho(x, \sigma_F)$ to be the right-hand side of Equation (\ref{app_eq_fb}).
Observe that $\partial \rho(x, \sigma_F)/\partial x > 0$, $\lim_{x \to \infty}\rho(x, \sigma_F) = (n-1)/(2n)$, $\lim_{x \to -\infty} \rho(x, \sigma_F) = 1$, and $x > \rho(x, \sigma_F)$ for $x < 0$ but $|x|$ sufficiently large. 
Let $\bar{\sigma}_F = (n-1)/(8n\phi(0))$. It suffices to consider the following two cases.

\textit{Case 1}: Suppose that $\sigma_F \leq \bar{\sigma}_F$. It is equivalent to $\rho(0, \sigma_F) \geq 0$. Since
\begin{equation*}
    \frac{\partial \rho(0, \sigma_F)}{\partial x} = \frac{n-1}{2n\sigma_F}\phi(0) - \lambda'(0) \geq 8\phi(0)^2 > 1
\end{equation*}
by the derivative formula $\lambda'(x) = -\lambda(x)\left[ x + \lambda(x) \right]$,
Equation (\ref{app_eq_fb}) must have at least two solutions.

\textit{Case 2}: Suppose that $\sigma_F > \bar{\sigma}_F$.
For $x > 0$, 
\begin{equation} \label{app_rho_sig}
   \frac{\partial \rho(x, \sigma_F)}{\partial \sigma_F} = - \frac{x}{\sigma_F} \left(\frac{\partial \rho(x, \sigma_F)}{\partial x}\right) < 0, \tag{A.2}
\end{equation}
and
\begin{equation} \label{app_rho_sd}
    \frac{\partial^2 \rho(x, \sigma_F)}{\partial x^2} = \frac{n-1}{2n\sigma_F^2}\phi'\left(\frac{x}{\sigma_F}\right) - \frac{1}{\sigma_F}\lambda''\left(\frac{x}{\sigma_F}\right) < 0 \tag{A.3}
\end{equation}
because $\lambda''(x) > 0$. By (\ref{app_rho_sig}) and (\ref{app_rho_sd}), there exists a unique $\widehat{\sigma}_F$ such that $x$ and $\rho(x, \widehat{\sigma}_F)$ are tangent to each other at some $x > 0$ because $\rho(x, \sigma_F)$ is strictly concave. Moreover, for $x > 0$, $x > \rho(x, \sigma_F)$ if and only if $\sigma_F > \widehat{\sigma}_F$. If we can prove that $x > \rho(x, \sigma_F)$ for $x \leq 0$ whenever $\sigma_F > \widehat{\sigma}_F$, then we are done. Note that, at $\sigma_F = \bar{\sigma}_F$, 
\begin{equation*}
    \rho(x, \bar{\sigma}_F) - x = \bar{\sigma}_F \bigg[2\lambda(0)\Phi(z) - \lambda(z) - z \bigg] < 0,
\end{equation*}
where $z = x/\bar{\sigma}_F$ because $\lambda(0) = 2\phi(0)$ and $2\lambda(0)\Phi(x) - \lambda(x) - x < 0$ for $x < 0$. It follows that $x > \rho(x, \sigma_F)$ for all $x \leq 0$. Thus, (\ref{app_eq_fb}) can 
only have solutions if $\sigma_F \leq \widehat{\sigma}_F$.

Taking Case 1 and Case 2 together, we can conclude that (\ref{app_eq_fb}) has no solution (i.e., $\xiup = -\infty$) if and only if $\sigma > \widehat{\sigma}_F$. \qed


\medskip
\subsection*{Proof of Proposition \ref{prop_limit}}

\underline{Step 1}. We first show that $\xiup \to (n-1)/2n$ as $\sigma_F \to 0$. 
For sufficiently small $\sigma_F$,
note that $\xiup > -\infty$ is determined by Equation (\ref{follower_invest_upper}); that is,
\begin{equation*}
    \xiup + \sigma_F \lambda \left( \frac{\xiup - \thetalow}{\sigma_F}  \right) - \frac{n-1}{2n} \Phi\left( \frac{\xiup - \thetalow}{\sigma_F}  \right) = 0,
\end{equation*}
where $\thetalow = 0$.
We know from the proof of Proposition \ref{prop_unique_rat} that $\xiup > 0$. So, as $\sigma_F \to 0$, there are three cases to consider: (i) $\xiup \to 0$ and $\xiup/\sigma_F \to k \geq 0$, (ii) $\xiup \to 0$ and $\xiup/\sigma_F \to \infty$, and  (iii) $\xiup \to \aleph > 0$ and $\xiup/\sigma_F \to \infty$, 

\emph{Case (i)}: Suppose that $\xiup \to 0$ and $\xiup/\sigma_F \to k$, where $k \geq 0$ is a constant. It follows that
\begin{equation*}
    \xiup + \sigma_F \lambda \left( \frac{\xiup}{\sigma_F}  \right) - \frac{n-1}{2n} \Phi\left( \frac{\xiup}{\sigma_F}  \right) \to 0 + 0 \cdot \lambda(k) - \frac{n-1}{2n}\Phi(k) \neq 0,
\end{equation*}
which leads to a contradiction.

\emph{Case (ii)}: Suppose that $\xiup \to 0$ and $\xiup/\sigma_F \to \infty$. Then we have a contradiction because
\begin{equation*}
    \xiup + \sigma_F \lambda \left( \frac{\xiup}{\sigma_F}  \right) - \frac{n-1}{2n} \Phi\left( \frac{\xiup}{\sigma_F}  \right) \to - \frac{n-1}{2n} < 0.
\end{equation*}

\emph{Case (iii)}: Suppose that $\xiup \to \aleph > 0$. Then it must be the case that
\begin{equation*}
    \xiup + \sigma_F \lambda \left( \frac{\xiup}{\sigma_F}  \right) - \frac{n-1}{2n} \Phi\left( \frac{\xiup}{\sigma_F}  \right) \to \aleph - \frac{n-1}{2n} = 0,
\end{equation*}
which results in a contradiction except for the case $\aleph = (n-1)/2n$.

Combining all three cases above, we can conclude that $\xiup \to (n-1)/2n$ as $\sigma_F \to 0$.

\vskip 0.5 \baselineskip
\noindent \underline{Step 2}. We show next that $\thetaup \to (n-1)/2n$ as $\sigma_F \to 0$. We consider, for the sake of contradiction, the following two cases: (i) $\thetaup \to  \tau < (n-1)/2n$, and (iii) $\thetaup \to  \tau > (n-1)/2n$.

\emph{Case (i)}: Suppose that $\thetaup \to  \tau$, where $\tau \in [0, (n-1)/2n)$ is a constant. Note that $\thetaup$ is given by Equation (\ref{leader_upper}):
\begin{equation*}
    \thetaup - \Phi \left( \frac{\xiup - \thetaup}{\sigma_F} \right) = 0.
\end{equation*}
Since $(\xiup - \thetaup)/\sigma_F \to \infty$, we have
\begin{equation*}
\thetaup - \Phi \left( \frac{\xiup - \thetaup}{\sigma_F} \right) \to \tau - 1 < - \frac{n+1}{2n} < 0,     
\end{equation*}
which leads to a contradiction.

\emph{Case (ii)}: Suppose that $\thetaup \to  \tau \in ((n-1)/2n, 1]$. Then it must be the case that $(\xiup - \thetaup)/\sigma_F \to -\infty$. But we have a contradiction because 
\begin{equation*}
\thetaup - \Phi \left( \frac{\xiup - \thetaup}{\sigma_F} \right) \to \tau > \frac{n-1}{2n} > 0.     
\end{equation*}

Thus, we must have $\thetaup \to (n-1)/2n$ as $\sigma_F \to 0$.

\medskip
\noindent \underline{Step 3}. Lastly, we show that $\xnlow \to \infty$ as $\sigma_F \to 0$. 
By way of contradiction, suppose that $\xnlow \to \aleph < \infty$. This means that $\aleph$ solves  
Equation (\ref{follower_notinvest_lower}):
\begin{equation*}
    \aleph - \sigma_F \lambda \left( \frac{\thetaup - \aleph}{\sigma_F} \right) + \frac{n-1}{2n} \Phi\left( \frac{\thetaup - \aleph}{\sigma_F}\right) = 1.
\end{equation*}
There are three possible cases for the limit of $(\thetaup - \xnlow)/\sigma_F$ as $\sigma_F \to 0$: (i) $-\infty$, (ii) $\infty$, or (iii) a constant $k \in \RR$.

\emph{Case (i)}: Suppose that $(\thetaup - \xnlow)/\sigma_F \to -\infty$. Since $\lim_{x \to -\infty} x + \lambda(x) = 0$, we have
\begin{align*}
    \xnlow - & \sigma_F \lambda \left( \frac{\thetaup - \xnlow}{\sigma_F} \right) + \frac{n-1}{2n} \Phi\left( \frac{\thetaup - \xnlow}{\sigma_F}\right) \\
    & = -\sigma_F \left( \frac{\thetaup - \xnlow}{\sigma_F} + \lambda \left( \frac{\thetaup - \xnlow}{\sigma_F} \right) \right) + \thetaup + \frac{n-1}{2n} \Phi\left( \frac{\thetaup - \xnlow}{\sigma_F}\right) \\
    & \to \frac{n-1}{2n} < 1.
\end{align*}
Thus, we have a contradiction.

\emph{Case (ii)}: Suppose that $(\thetaup - \xnlow)/\sigma_F \to \infty$. In this case, we must have $\aleph \leq (n-1)/2n$. It follows that
\begin{equation*}
    \xnlow - \sigma_F \lambda \left( \frac{\thetaup - \xnlow}{\sigma_F} \right) + \frac{n-1}{2n} \Phi\left( \frac{\thetaup - \xnlow}{\sigma_F}\right) \to \aleph + \frac{n-1}{2n} \leq \frac{n-1}{n} < 1,
\end{equation*}
which leads to a contradiction.

\emph{Case (iii)}: Suppose that $(\thetaup - \xnlow)/\sigma_F \to k$, where $k$ is a constant. This implies that $\aleph = (n-1)/2n$. But since
\begin{equation*}
    \xnlow - \sigma_F \lambda \left( \frac{\thetaup - \xnlow}{\sigma_F} \right) + \frac{n-1}{2n} \Phi\left( \frac{\thetaup - \xnlow}{\sigma_F}\right) \to \aleph + \frac{n-1}{2n}\Phi(k) < \frac{n-1}{n} < 1,
\end{equation*}
we have a contradiction.

Thus, we must have $\xnlow \to \infty$ as $\sigma_F \to 0$. This completes the proof. \qed


\medskip
\subsection*{Derivation of the conditional rank beliefs in (\ref{rank_belief}) } 
By definition, we have
\begin{align*}
    R^h(x; z) & = \prob(x_k \leq x_j \vl x_j = x, z) \\
    & = \prob(\theta + \sigma_F \varepsilon_k \leq x_j \vl x_j = x, z) \\
    & = \int_{-\infty}^\infty \left( \int_{-\infty}^\frac{x - \theta}{\sigma_F} \phi(\varepsilon) \dd \varepsilon \right) \dd \Psi^h(\theta; x, z) \\
    & = \int_{-\infty}^\infty \Phi\left( \frac{x - \theta}{\sigma_F} \right) \dd \Psi^h(\theta; x, z).
\end{align*}
Now under $h = \invest$, by (\ref{interim_belief})
\begin{align*}
    R^\invest(x; z) & = \frac{1}{\Phi\left( \frac{x - z}{\sigma_F}\right) } \int_z^\infty \Phi\left( \frac{x - \theta}{\sigma_F} \right) \frac{1}{\sigma_F} \phi\left( \frac{\theta - x}{\sigma_F}  \right) \dd \theta \\
    & = \frac{1}{\Phi\left( \frac{x - z}{\sigma_F}\right) } \int_{-\infty}^{\Phi\left(\frac{x-z}{\sigma_F}\right)} \eta \dd \eta \\ 
    & = \frac{1}{2}\Phi\left( \frac{x-z}{\sigma_F} \right).
\end{align*}
The second equality is given by a change of variable $\eta = \Phi((x - \theta)/\sigma_F)$. Similarly,
\begin{align*}
    R^\notinvest(x; z) & = \frac{1}{\Phi\left( \frac{z - x}{\sigma_F}\right) } \int_{-\infty}^z \Phi\left( \frac{x - \theta}{\sigma_F} \right) \frac{1}{\sigma_F} \phi\left( \frac{\theta - x}{\sigma_F} \right) \dd \theta \\
    & = \frac{1}{\Phi\left( \frac{z - x}{\sigma_F}\right) } \int_{\Phi\left(\frac{x-z}{\sigma_F}\right)}^1 \eta \dd \eta \\
    & = \frac{1}{2} \Phi\left( \frac{x-z}{\sigma_F} \right) + \frac{1}{2}.
\end{align*}


\medskip
\subsection*{Derivation of the conditional rank beliefs in (\ref{ext_rank_belief})}

Under history $h = \invest$,
\begin{align*}
    R^\invest(x; z) & = \prob(x_k \leq x_j \vl x_j = x, x_L > z) \\
    & = \int_{-\infty}^\infty \left( \int_{-\infty}^\frac{x - \theta}{\sigma_F} \phi(\varepsilon) \dd \varepsilon \right) g^\invest (\theta; x , z) \dd \theta \\ 
    & = \frac{1}{\Phi \left( \frac{x - z }{\sigma} \right) }\int_{-\infty}^\infty \Phi(-y) \phi(y) \Phi \left( \frac{x - z + \sigma_F y }{\sigma_L} \right) \dd y \\
    & = \frac{1}{2} - \frac{1}{\Phi \left( \frac{x - z }{\sigma} \right)} T \left( \frac{x - z}{\sigma},  \frac{\sigma_F}{(2\sigma_L^2 + \sigma_F^2)^{\frac{1}{2}}} \right).
\end{align*}
The second equality is given by the independence between 
$\theta$ and $\varepsilon_k$,
the third equality is due to a change of variable $y = (\theta - x_\invest)/\sigma_F$, and the last equality is derived from applying the following integral identity
\begin{equation*}
    \int_{-\infty}^\infty \Phi(a + bx) \Phi(cx) \phi(x) \dd x = \frac{1}{2} \Phi \left( \frac{a}{\sqrt{1 + b^2}} \right) + T \bigg( \frac{a}{\sqrt{1 + b^2}}, \frac{bc}{\sqrt{1 + b^2 + c^2}}\bigg)
\end{equation*}
and $T(y, -a) = -T(y, a)$, where $T(y, a)$ is Owen's T-function (see footnote \ref{owen_t}).

Under history $h = \notinvest$, a similar argument yields
\begin{align*}
    R^\notinvest(x; z) & = \prob (x_k \leq x_j \vl x_j = x, x_L \leq z) \\
    & = \int_{-\infty}^\infty \Phi \left( \frac{x - \theta }{\sigma_F} \right)  g^\notinvest(\theta; x, z) \dd \theta \\
    & = \frac{1}{2} + \frac{1}{\Phi \left( \frac{z - x}{\sigma} \right)} T \left( \frac{z - x}{\sigma},  \frac{\sigma_F}{(2\sigma_L^2 + \sigma_F^2)^{\frac{1}{2}}} \right).
\end{align*}



\medskip




\begin{lemma} \label{lemma_exp_ext}
Under history $h$, $\EE_{\theta \sim G^h(\cdot; \,x, z)}[\theta]$ is increasing in $x$ and $z$. Moreover,
\[
\EE_{\theta \sim G^h(\cdot; \,x, z)}[\theta] \to \begin{cases}
     \infty & \text{as $x \to \infty$} \\
     -\infty & \text{as $x \to -\infty$}
\end{cases}.
\]
\end{lemma}
\begin{proof}[Proof of Lemma \ref{lemma_exp_ext}]
Note that under history $h = \invest$,
\begin{align} \label{app_fol_exp}
    \EE_{\theta \sim G^\invest(\cdot; \,x, z)} [\theta] & = \frac{1}{\Phi\left(\frac{x-z}{\sigma}\right)}\int_{-\infty}^\infty \frac{t}{\sigma_F} \phi\left(\frac{t-x}{\sigma_F}\right) \Phi\left(\frac{t-z}{\sigma_L}\right) \dd t  \nonumber \\
    & = \frac{1}{\Phi\left(\frac{x-z}{\sigma}\right)}\int_{-\infty}^\infty (x + \sigma_F \eta) \phi(\eta) \Phi\left(\frac{x + \sigma_F \eta - z}{\sigma_L}\right) \dd \eta \nonumber \\
    & = x + \frac{\sigma_F^2}{\sigma} \lambda \left(\frac{x-z}{\sigma}\right). \tag{A.4}
\end{align}
The second inequality is due to a change of variable $\eta = (t-x)/\sigma_F$, and the third equality is derived by applying integral identities $\int_{-\infty}^\infty \phi(\eta)\Phi(a + b\eta) \dd \eta = \Phi(a/\sqrt{1 + b^2})$ and $\int_{-\infty}^\infty \eta \phi(\eta) \Phi(a+b\eta) \dd \eta = (b/\sqrt{1+b^2})\Phi(a/\sqrt{1+b^2})$ (see, for example, \cite{owen_1980}). Therefore, (\ref{app_fol_exp}) is increasing in $x$ and $z$ because
\begin{align*}
    \EE_{\theta \sim G^\invest(\cdot; \,x, z)} [\theta] & = \frac{\sigma_L^2}{\sigma}\left(\frac{x-z}{\sigma}\right) + \frac{\sigma_F^2}{\sigma}\left(\frac{x-z}{\sigma} + \lambda\left(\frac{x-z}{\sigma}\right)\right) + z,
\end{align*}
$\eta + \lambda(\eta)$ is increasing in $\eta$, and $-1 < \lambda'(\eta) < 0$. The last inequality is due to \cite{sampford_1953}.
Now given the fact that $\lambda(\eta)/\eta \to -1$ as $\eta \to -\infty$ and $\lambda(\eta)/\eta \to 0$ as $\eta \to \infty$, we can conclude that
\[
\EE_{\theta \sim G^\invest(\cdot; \,x, z)} [\theta] \to \begin{cases}
    \infty & \text{as $x \to \infty$} \\
    -\infty & \text{as $x \to -\infty$}
\end{cases}.
\]


Similarly, under history $h = \notinvest$, one can show that
\begin{align*}
    \EE_{\theta \sim G^\notinvest(\cdot; \,x, z)}[\theta] & = x - \frac{\sigma_F^2}{\sigma} \lambda\left(\frac{z-x}{\sigma}\right) \\
    & = -\frac{\sigma_L^2}{\sigma}\left(\frac{z-x}{\sigma}\right) - \frac{\sigma_F^2}{\sigma}\left(\frac{z-x}{\sigma} + \lambda\left(\frac{z-x}{\sigma}\right)\right) + z,
\end{align*}
which is increasing in $x$ and $z$, approaches to $-\infty$ as $x \to -\infty$, and approaches to $\infty$ as $x \to \infty$. The proof is complete.
\end{proof}

\begin{lemma} \label{lemma_payoff_ext}
Under history $h$, $\pi_F^h(x; z, x_h)$ is increasing in $x$ and $z$, and is decreasing in $x_h$. Moreover, $\lim_{x \to -\infty}\pi_F^h(x; z, x_h) = -\infty$ and $\lim_{x \to \infty} \pi_F^h(x; z, x_h) = \infty$.
\end{lemma}
\begin{proof}[Proof of Lemma \ref{lemma_payoff_ext}]
Recall that
\[
\pi_F^h(x; z, x_h) = \EE_{\theta \sim G^h(\cdot; \,x, z)} \left[\theta - \frac{n-1}{n}\Phi\left(\frac{x_h - \theta}{\sigma_F}\right)\right] - \frac{\chi_\notinvest}{n}.
\]
It is immediate to see that $\pi_F^h(x; z, x_h)$ is decreasing in $x_h$ since $\Phi((x_h - \theta)/\sigma_F)$ is increasing in $x_h$.
To show $\pi_F^h(x; z, x_h)$ is increasing in both $x$ and $z$, it suffices to prove that $G^h(\cdot; \,x, z)$ is increasing in $x$ and $z$ with respect to the first-order stochastic dominance order. We ignore the proof as it is similar to that of Lemma \ref{lemma_x_payoff}.
The limits at infinity are given directly by Lemma \ref{lemma_exp_ext} and the boundedness of $\Phi((x_h - \theta)/n)$.
\end{proof}

\begin{lemma} \label{lemma_seq_ext} 
Let $\xlow^0 = \xhlow^0 = -\infty$ and $\xup^0 = \xhup^0 = \infty$ for each history $h$. Then
the iterative procedure of $\Delta$-rationalizability yields six sequences: \\
(i) $(\xlow^k)_{k = 0}^\infty, (\xilow^k)_{k = 0}^\infty$, and $(\xnlow^k)_{k = 0}^\infty$ are strictly increasing and bounded above;\\
(ii) $(\xup^k)_{k = 0}^\infty, (\xiup^k)_{k = 0}^\infty$, and $(\xnup^k)_{k = 0}^\infty$ are strictly decreasing and bounded below.
\end{lemma}
\begin{proof}[Proof of Lemma \ref{lemma_seq_ext}]
Let $x_L = \br_L(x_\invest)$ denote the unique solution to
\[
\pi_L(x_L; x_\invest) = x_L - \Phi\left(\frac{x_\invest-x_L}{\sigma}\right),
\]
and $x = \br_F^h(z, x_h)$ the unique solution to $\pi_F^h(x; z, x_h) = 0$ for each history $h$.
The latter is guaranteed by both Lemma \ref{lemma_exp_ext} and Lemma \ref{lemma_payoff_ext}.

Let $\xlow^0 = \xhlow^0 = -\infty$ and $\xup^0 = \xhup^0 = \infty$. Define for $k \in \NN$,
\begin{equation} \label{ext_seq_br}
\begin{cases}
\xlow^k = \br_L(\xilow^{k-1}) \\
\xup^k = \br_L(\xiup^{k-1}) \\
\xilow^k = \br_F^\invest(\xup^k, \xilow^{k-1}) \\
\xiup^k  = \br_F^\invest(\xlow^k, \xiup^{k-1}) \\
\xnlow^k = \br_F^\notinvest(\xup^k, \xnlow^{k-1}) \\
\xnup^k = \br_F^\notinvest(\xlow^k, \xnup^{k-1}) \\
\end{cases}. \tag{A.5}
\end{equation}
Note that, by Lemma \ref{lemma_payoff_ext}, $\br_L(x_\invest)$ is strictly increasing in $x_\invest$ and $\br_F^h(z, x_h)$ is strictly decreasing in $z$ and strictly increasing in $x_h$.  We prove the lemma by induction. 

\noindent For $k = 1$, consider the leader first. Since $\xlow^1 = \br_L(\xilow^0) = 0$ and $\xup^0 = \br_L(\xiup^0) = 1$, it is immediate that $\xlow^0 < \xlow^1 < \xup^1 < \xup^0$. Now under history $h$, Lemma \ref{lemma_exp_ext} implies that $\xhlow^1 > \xhlow^0$ and $\xhup^1 < \xhup^0$. But since
\[
\xhlow^1 = \br_F^h(\xup^1, \xhlow^0) < \br_F^h(\xlow^1, \xhlow^0) < \br_F^h(\xlow^1, \xhup^0) = \xhup^1,
\]
it follows that $\xhlow^0 < \xhlow^1 < \xhup^1 < \xhup^0$.


\noindent Assume that, for $k \geq 1$, $\xlow^{k-1} < \xlow^k < \xup^k < \xup^{k-1}$ for the leader, and $\xhlow^{k-1} < \xhlow^k < \xhup^k < \xhup^{k+1}$ for followers under history $h$. The induction hypothesis implies
\begin{align*}
    \br_L(\xilow^{k-1}) < \br_L(\xilow^k) < \br_L(\xiup^k) < \br_L(\xiup^{k-1});
\end{align*}
therefore we have $\xlow^k < \xlow^{k+1} < \xup^{k+1} < \xup^k$.
The proof for $\xhlow^k < \xhlow^{k+1} < \xhup^{k+1} < \xhup^k$ is straightforward.
\end{proof}

\medskip
\noindent Define $\tilde{S}(y, \nu) = 1/2 - T(y, \nu)/\Phi(y)$, where $T(y, \nu)$ is Owen's T-function with slope parameter $\nu > 0$ and $y \in \RR$ (see footnote \ref{owen_t}).
\begin{lemma} \label{lemma_s_func}
    The function $\tilde{S}(y, \nu)$ is strictly increasing and differentiable everywhere in $y$. Moreover, $\lim_{y \to -\infty} \tilde{S}(y, \nu) = 0$, $\lim_{y \to \infty} \tilde{S}(y, \nu) = 1/2$, and if $\nu \in (0, 1)$, then $\partial \tilde{S}(y, \nu)/\partial y$ vanishes at infinity ; i.e.,
    \begin{equation*}
         \lim_{y \to -\infty} \frac{\partial  \tilde{S}(y, \nu)}{\partial y} =          \lim_{y \to \infty} \frac{\partial  \tilde{S}(y, \nu)}{\partial y} = 0.
    \end{equation*}
\end{lemma}
\begin{proof}[Proof of Lemma \ref{lemma_s_func}]
Since $T(y, \nu)$ is differentiable everywhere in $y$, so is $\tilde{S}(y, \nu)$. Note that
\begin{align*}
    \frac{\dd }{\dd y}\left[ \frac{T(y, \nu)}{\Phi(y)} \right] & = \frac{1}{2\Phi(y)^2} \left[-\phi(y) \erf\left( \frac{\nu y}{\sqrt{2}} \right) \Phi(y) - 2\phi(y) T(y, \nu)\right]  \\
 & \propto - \phi(y) \erf\left( \frac{\nu y}{\sqrt{2}} \right) \Phi(y) - \phi(y) \int_{-y}^\infty \phi(t) \erf\left(
 \frac{\nu t}{\sqrt{2}} \right) \dd t \\
 & = - \phi(y) \int_{-y}^\infty \phi(t) \left[ \erf\left( 
 \frac{\nu y}{\sqrt{2}} \right) + \erf\left( 
 \frac{\nu t}{\sqrt{2}} \right) \right] \dd t \\
 & < 0,
\end{align*}
where $\erf(\cdot)$ is the error function.
The second equality is derived using the integral representation of Owen's T-function (see Equation (3.1) in \cite{brychkov_savischenko_2016}). The inequality holds because the strict monotonicity of $\erf(\cdot)$ implies that
\begin{equation*}
    \erf\left( \frac{\nu y}{\sqrt{2}} \right) + \erf\left( 
 \frac{\nu t}{\sqrt{2}} \right) >  \erf\left( 
 \frac{\nu y}{\sqrt{2}} \right) + \erf\left( 
 -\frac{\nu y}{\sqrt{2}} \right) = 0
\end{equation*}
for all $t > -y$. Thus, $\tilde{S}(y, \nu)$ is strictly increasing. Moreover, since $T(y, \nu) \to 0$ as $y \to \infty$ or $y \to -\infty$, we have $\lim_{y \to \infty} \tilde{S}(y, \nu) = 1/2$ and
\begin{equation*}
    \lim_{y \to -\infty} \tilde{S}(y, \nu) = \frac{1}{2} -\lim_{y \to -\infty}\frac{T(y, \nu)}{\Phi(y)} = \frac{1}{2} + \frac{1}{2}\lim_{y \to -\infty} \erf\left( \frac{\nu y}{\sqrt{2}} \right)  = 0.
\end{equation*}
The last equality is due to $\erf(\nu y/\sqrt{2}) \to -1$ as $y \to -\infty$.

Now we define
\begin{equation*}
    M(y, \nu) = \int_{-y}^\infty \phi(t) \erf\left( 
 \frac{\nu t}{\sqrt{2}} \right)  \dd t + \erf\left( \frac{\nu y}{\sqrt{2}}\right)\Phi(y),
\end{equation*}
and write $\tilde{S}_y(y, \nu)$ for the partial derivative $\partial \tilde{S}(y, \nu)/\partial y$. Then we know from above that
\begin{equation*}
    \tilde{S}_y(y, \nu) = \frac{\phi(y) M(y, \nu)}{2 \Phi(y)^2}.
\end{equation*}
To show that $\tilde{S}_y(y, \nu)$ vanishes as $y \to \infty$, note that
\begin{equation*}
    \lim_{y \to \infty} M(y, \nu) = \int_{-\infty}^\infty \phi(t) \erf\left( \frac{\nu t}{\sqrt{2}}\right) \dd t + 1
    = 2 \int_{-\infty}^\infty \phi(t) \Phi(\nu t) \dd t
    = 2 \Phi(0) = 1.
\end{equation*}
The first equality holds because $\erf(\nu y/\sqrt{2}) \to 1$ as $y \to \infty$, the second equality is due to $\erf(\nu t/\sqrt{2}) = 2 \Phi(\nu t) - 1$, and the last equality is given by applying the integral identity $\int_{-\infty}^\infty \phi(t) \Phi(a + bt) \dd t = \Phi(a/\sqrt{1+b^2})$. Thus, $\lim_{y \to \infty} \tilde{S}_y(y, \nu) = 0$. 

Suppose that $\nu \in (0, 1)$.
Since
\begin{equation*}
    M_y(y,\nu) = \sqrt{\frac{2\nu^2}{\pi}} \Phi(y) \exp{\left(-\frac{\nu^2 y^2}{2}\right)},
\end{equation*}
by L'H\^{o}pital's rule and $\lim_{y \to -\infty} \phi(y)/(y \Phi(y)) = -1$, we have
\begin{equation*}
    \lim_{y \to -\infty} \frac{y M(y, \nu)}{\Phi(y)} = \lim_{y \to -\infty} \frac{M(y, \nu)}{\Phi(y) y^{-1}} = \sqrt{\frac{2\nu^2}{\pi}} \lim_{y \to -\infty}  \frac{ \exp{\left(-\frac{\nu^2 y^2}{2}\right)}}{\frac{\phi(y)}{y \Phi(y)} - \frac{1}{y^2}} = 0.
\end{equation*}
This implies that
\begin{align*}
    \lim_{y \to -\infty} \frac{\phi(y) M(y, \nu)}{\Phi(y)^2} & = \lim_{y \to -\infty} \frac{-y M(y, \nu) + M_y(y, \nu)}{2 \Phi(y)} \\
    & = \sqrt{\frac{2\nu^2}{\pi}} \lim_{y \to -\infty} \exp{\left(-\frac{\nu^2 y^2}{2}\right)} \\
    & = 0.
\end{align*}
Thus, $\lim_{y \to -\infty} \tilde{S}_y(y, \nu) = 0$.
The proof is complete.
\end{proof}



\medskip
\subsection*{Proof of Proposition \ref{prop_unique_rat_ext}}
\emph{Proof of Part (i)}:
Fix $\sigma_F = \gamma \sigma_L$, $\gamma > 0$. Let $\xlow, \xup, \xilow, \xiup, \xnlow$, and $\xnup$ be the limits of the six sequences described in (\ref{ext_seq_br}), respectively. By Lemma \ref{lemma_seq_ext}, we know that they must solve (\ref{ext_sys_eqs}). Moreover $0 = \xlow^1 < \xlow \leq \xup < \xup^1 = 1$ and $\xhlow^1 < \xhlow \leq \xhup < \xhup^1$, where $\xhlow^1$ and $\xhup^1$ are the lower and upper dominance bounds in Round 1 for the followers under history $h$. Let $\Xi_L = [0, 1]$ and $\Xi_F^h = [\xhlow^1, \xhup^1]$.

Let $S(y) = \tilde{S}(y, \alpha)$ with slope parameter $\alpha = \gamma/\sqrt{2 + \gamma^2}$.
Since $1 + \gamma^2 (1 + \lambda'(y))$ is positive and bounded for all $y \in \RR$,
the following function is well-defined by Lemma \ref{lemma_s_func}:
\begin{equation*}
    \Lambda(\gamma) = \max_{y \in \RR} \frac{S'(y)}{1 + \gamma^2(1 + \lambda'(y))}.
\end{equation*}
Moreover, $\Lambda(\gamma) > 0$.  Now let
\begin{equation} \label{ext_sufficiency_1} 
     \widehat{\sigma}_L^1(\gamma) = \left(\frac{n-1}{n}\right)  \sqrt{(1+\gamma^2) \Lambda(\gamma)^2}. \tag{A.6}
\end{equation}
We prove this part in two steps. First, we show that if $\sigma_L > \widehat{\sigma}_L^1(\gamma)$
then the following system of equations with $x_L \in \Xi_L$ and $x_h \in \Xi_F^h$ 
\begin{equation} \label{ext_sys_small}
    \begin{cases}
        \pi_L(x_L;x_{\invest}) = 0 \\
        \pi_F^\invest(x_{\invest}; x_L, x_{\invest}) = 0 \\ 
        \pi_F^\notinvest(x_{\notinvest}; x_L,x_{\notinvest}) = 0 \\
    \end{cases} \tag{A.7}
\end{equation}
has a unique solution $(x_L^*, x_\invest^*, x_\notinvest^*)$. Second, we show that there exists $\widehat{\sigma}_L^2(\gamma)$ such that $(x_L^*, x_\invest^*, x_\notinvest^*)$ is also the unique solution to (\ref{ext_sys_eqs}) whenever
\begin{equation} \label{ext_suff_cond}
    \sigma_L > \widehat{\sigma}_L(\gamma) = \max \left\{\widehat{\sigma}_L^1(\gamma), \widehat{\sigma}_L^2(\gamma) \right\}. \tag{A.8}
\end{equation}


\noindent \underline{Step 1}: Assume from now on that $\sigma_L > \widehat{\sigma}_L^1(\gamma)$.
By Equations (\ref{ext_rank_belief}), (\ref{ext_follower_fp_eqn}), and (\ref{app_fol_exp}), we have
\begin{equation} \label{app_fol_fp_eqn}
    \pi_F^\invest(x_\invest; x_L, x_\invest) = x_\invest + \frac{\sigma_F^2}{\sigma} \lambda\left( \frac{x_\invest - x_L}{\sigma}\right) - \frac{n-1}{n}S\left( \frac{x_\invest - x_L}{\sigma} \right). \tag{A.9}
\end{equation}
If $\sigma_L > \widehat{\sigma}_L^1(\gamma)$, then $\sigma_F^2/\sigma^2 = \gamma^2/(1 + \gamma^2)$ gives that
\begin{align*}
    \frac{\partial \pi_F^\invest}{\partial x_\invest} & = 1 + \frac{\gamma^2}{1 + \gamma^2} \lambda'\left( \frac{x_\invest - x_L}{\sigma} \right) - \frac{n-1}{n\sigma} S'\left(\frac{x_\invest - x_L}{\sigma} \right)\\
    & > 1 + \frac{\gamma^2}{1 + \gamma^2} \lambda'\left( \frac{x_\invest - x_L}{\sigma} \right) - \frac{1}{(1 + \gamma^2) \Lambda(\gamma)} S'\left(\frac{x_\invest - x_L}{\sigma} \right) \geq 0.
\end{align*}
This implies that, for any given $x_L$, $\pi_F^\invest(x_\invest; x_L, x_\invest) = 0$ admits a unique solution. 
Similarly, since
\begin{equation*}
    \pi_F^\notinvest (x_\notinvest; x_L, x_\notinvest) = x_\notinvest - \frac{\sigma_F^2}{\sigma} \lambda\left( \frac{ x_L - x_\notinvest }{\sigma}\right) + \frac{n-1}{n}S\left( \frac{x_L - x_\notinvest}{\sigma} \right) - 1,
\end{equation*}
the condition $\sigma_L > \widehat{\sigma}_L^1(\gamma)$ implies that
\begin{align*}
    \frac{\partial \pi_F^\notinvest}{\partial x_\notinvest} & = 1 + \frac{\gamma^2}{1 + \gamma^2} \lambda'\left( \frac{ x_L - x_\notinvest }{\sigma}\right) - \frac{n-1}{n\sigma} S'\left(\frac{x_L - x_\notinvest}{\sigma} \right) \\
    & > 1 + \frac{\gamma^2}{1 + \gamma^2} \lambda'\left( \frac{x_L - x_\notinvest}{\sigma} \right) - \frac{1}{(1 + \gamma^2) \Lambda(\gamma)} S'\left(\frac{x_L - x_\notinvest}{\sigma} \right) \geq 0
\end{align*}
and hence, given $x_L$, $\pi_F^\notinvest (x_\notinvest; x_L, x_\notinvest) = 0$ has a unique solution.
Thus, (\ref{ext_sys_small}) has a unique solution if and only if there is a unique solution to its first two equations.

Since $\partial \pi_F^\invest/\partial x_\invest$ is continuous on $\Xi_F^\invest \times \Xi_L$, the extreme value theorem ensures that there exists $d_\invest > 0$ such that $\partial \pi_F^\invest/\partial x_\invest \geq d_\invest$. Thus, a global implicit function theorem (see, e.g., Lemma 2 of \cite{zhang_ge_2006}) implies that there is a unique function $f: \Xi_L \to \Xi_F^\invest$ such that $\pi_F^\invest(f(x_L); x_L, f(x_L)) = 0$. Moreover, $f \in C^1$ and $f' < 0$. It follows that
\begin{equation*}
    \frac{\dd \pi_L}{\dd x_L} = \frac{\partial \pi_L}{\partial x_L} + \frac{\partial \pi_L}{\partial x_\invest}f'(x_L) > 0.
\end{equation*}
This says that $\pi_L(x_L; f(x_L)) = 0$ has a unique solution. Thus, (\ref{ext_sys_small}) has a unique solution $(x_L^*, x_\invest^*, x_\notinvest^*)$.

\vskip 0.5 \baselineskip
\noindent \underline{Step 2}: We next show that there exists $\widehat{\sigma}_L^2(\gamma)$ such that
$(x_L^*, x_\invest^*, x_\notinvest^*)$ is the unique solution to (\ref{ext_sys_eqs}) if (\ref{ext_suff_cond}) holds.
Note that
\begin{equation*}
    \frac{\partial \pi_L}{\partial x_L} = 1 + \frac{1}{\sigma}\phi\left( \frac{x_\invest - x_L}{\sigma} \right) > 0
\end{equation*}
is continuous on $\Xi_F^\invest \times \Xi_L$, so there exists $d_L > 0$ such that $\partial \pi_L/\partial x_L \geq d_L$. Thus, there exists a global implicit function $g: \Xi_F^\invest \to \Xi_L$ such that $\pi_L(g(x_\invest); x_\invest) = 0$. In addition, we have $g \in C^1$ and $g' > 0$. 
Then the first four equations of (\ref{ext_sys_eqs}) implies that
\begin{equation*}
    \xlow = g(\xilow) = (g \circ f)(\xup) = (g\circ f \circ g) (\xiup) =  (g\circ f \circ g \circ f)(\xlow).
\end{equation*}
Consider $h: \Xi_L \to \Xi_L$ such that $h = g \circ f \circ g \circ f$. Clearly, $x_L^*$ is a fixed point of $h$
(shown in Step 1). We also note that $h(0) > 0$ and $h' > 0$ because $f' < 0$ and $g' > 0$.
Define
\begin{equation*}
    M(\sigma_L, \gamma) = \max_{y \in \RR} - \frac{\gamma^2}{1+\gamma^2} \lambda'(y) + \frac{n-1}{n\sigma_L \sqrt{1 + \gamma^2}} S'(y).
\end{equation*}
We have $M(\sigma_L, \gamma) < 1$ because $\sigma_L > \widehat{\sigma}_L^1(\gamma)$. It follows that
\begin{equation*}
    f' \geq \frac{M(\sigma_L, \gamma)}{M(\sigma_L, \gamma) - 1} = M_f(\sigma_L, \gamma).
\end{equation*}
By the envelope theorem, $\partial M(\sigma_L, \gamma)/\partial \sigma_L < 0$. Thus, $\partial M_f(\sigma_L, \gamma)/\partial \sigma_L > 0$.
By the definition of $g$, we have
\begin{equation*}
    g' \leq \frac{\phi(0)}{\sigma_L \sqrt{1 + \gamma^2} + \phi(0)} = M_g(\sigma_L, \gamma).
\end{equation*}
Moreover, $\partial M_g(\sigma_L, \gamma)/\partial \sigma_L < 0$.
It follows that
\begin{equation} \label{h_prime}
    h' \leq \left[M_g(\sigma_L, \gamma) \cdot M_f(\sigma_L,\gamma)\right]^2. \tag{A.10}
\end{equation}
The right-hand side of (\ref{h_prime}) is strictly decreasing in $\sigma_L$ and approaches zero as $\sigma_L \to \infty$. Thus, there exists $\widehat{\sigma}_L^2(\gamma)$ such that $h' < 1$ if $\sigma_L > \max\{\widehat{\sigma}_L^1(\gamma),  \widehat{\sigma}_L^2(\gamma)\}$ (i.e., (\ref{ext_suff_cond}) holds). It is immediate that $h$ has a unique fixed point when $h' < 1$. The proof of part (i) is complete.   

\medskip
\noindent \emph{Proof of Part (ii)}:
Fix $\sigma_F = \gamma \sigma_L$, $\gamma > 0$. We prove this part in three steps. We first show the existence of a monotone equilibrium characterized by a tuple of thresholds $(x_L^*(\sigma_L, \gamma)$, $x_\invest^*(\sigma_L, \gamma)$, $x_\notinvest^*(\sigma_L, \gamma))$ such that
$a_L = \invest$ if and only if $x_L > x_L^*(\sigma_L, \gamma)$, and $s_j(\invest) = \invest$ if and only if $x_j > x_\invest^*(\sigma_L, \gamma)$ and $s_j(\notinvest) = \invest$ if and only if $x_j > x_\notinvest^*(\sigma_L, \gamma)$ for all followers $j \in F$. We will simply write the thresholds as $(x_L^*, x_\invest^*, x_\notinvest^*)$ if no confusion will arise.
In Step 2, we show that $x_\invest^* < x_L^* < x_\notinvest^*$. We finally show, in Step 3, that...

\vskip 0.5 \baselineskip
\noindent \underline{Step 1}.
Recall that $x_L = g(x_\invest)$ is the solution to $\pi_L(x_L; x_\invest) = 0$. Substituting $g(x_\invest)$ into Equation (\ref{app_fol_fp_eqn}) yields
\begin{equation*}
    \pi_F^\invest(x_\invest; g(x_\invest), x_\invest) = x_\invest + \frac{\sigma_F^2}{\sigma} \lambda\left( \frac{x_\invest - g(x_\invest)}{\sigma}\right) - \frac{n-1}{n}S\left( \frac{x_\invest - g(x_\invest)}{\sigma} \right).
\end{equation*}
Since $g(x_\invest) \in (0, 1)$, $\pi_F^\invest(x_\invest; g(x_\invest), x_\invest)\to -\infty$ as $x_\invest \to -\infty$, and $\pi_F^\invest(x_\invest; g(x_\invest), x_\invest)$ $\to \infty$ as $x \to \infty$ by Lemma \ref{lemma_payoff_ext}. Thus, by continuity, there must exists $x_\invest^*$ such that $\pi_F^\invest(x_\invest^*;$ $ g(x_\invest^*), x_\invest^*) = 0$. Let $x_L^* = g(x_\invest^*)$. It follows, by Lemma \ref{lemma_payoff_ext}, that there exists $x_\notinvest^*$ such that $\pi_F^\notinvest(x_\notinvest^*; g(x_\invest^*), x_\notinvest^*) = 0$.

\vskip 0.5 \baselineskip
\noindent \underline{Step 2}.
We now prove that $x_\invest^* < x_L^*$. By way of contradiction, assume $x_\invest^* \geq x_L^*$. Then we have $(x_\invest^* - x_L^*)/\sigma \geq 0$ and hence $x_L^* = \Phi((x_\invest^* - x_L^*)/\sigma) \geq 1/2$. It follows that
\begin{align*}
    x_\invest^* + \frac{\sigma_F^2}{\sigma} \lambda \left( \frac{x_\invest^* - x_L^*}{\sigma}  \right) - \frac{n-1}{n} S \left( \frac{x_\invest^* - x_L^*}{\sigma} \right) 
    \geq x_L^* - \frac{n-1}{2n} \geq \frac{1}{2n} > 0,
\end{align*}
which leads to a contradiction. The first inequality is due to $\lambda > 0$, $S < 1/2$, and the assumption that $x_\invest^* \geq x_L^*$. The second inequality is given by $x_L^* \geq 1/2$. Thus, it must be that $x_\invest^* < x_L^*$.

% In a similar vein, one can get another contradiction by assuming $x_L^* \geq x_\notinvest^*$. That is,
% \begin{align*}
%     x_\notinvest^* - \frac{\sigma_F^2}{\sigma} \left( \frac{x_L^* - x_\notinvest^*}{\sigma} \right) + \frac{n-1}{n} S \left( \frac{x_L^* - x_\notinvest^*}{\sigma} \right) - 1 \leq x_L^* + \frac{n-1}{2n} - 1 < - \frac{1}{2n} < 0. 
% \end{align*}
% The second inequality is because $x_\invest^* < x_L^*$ implies that $x_L^* < 1/2$. Thus, $x_L^* < x_\notinvest^*$.

\vskip 0.5 \baselineskip
\noindent \underline{Step 3}.
Let $\widecheck{x}_L^*$ and $\widecheck{x}_h^*$ be the limits as $\sigma_L \to 0$ while keeping the ratio $\sigma_F/\sigma_L = \gamma$ fixed.
Step 2 implies that $(x_\invest^* - x_L^*)/\sigma$ can approach either a constant $k \leq 0$ or $-\infty$ as $\sigma_L \to 0$ . We show in both cases, $\widecheck{x}_L^*$ is strictly less than $(n-1)/2n$ and so is $\widecheck{x}_\invest^*$.
In the former case, $x_L^*$ and $x_\invest^*$ must have the same limit; otherwise $(x_\invest^* - x_L^*)/\sigma \to -\infty$ leading to a contradiction. But since
\begin{equation*}
    \frac{\sigma_F^2}{\sigma} \lambda \left( \frac{x_\invest^* - x_L^*}{\sigma} \right) - \frac{n-1}{n} S \left( \frac{x_\invest^* - x_L^*}{\sigma} \right) \to 0 \cdot \lambda(k) - \frac{n-1}{n} S(k) = - \frac{n-1}{n}S(k),
\end{equation*}
we have $\widecheck{x}_\invest^* = ((n-1)/n)S(k) < (n-1)/2n$ by Lemma \ref{lemma_s_func}, and so does $\widecheck{x}_L^*$. In the latter case, $\widecheck{x}_L^* = 0 < (n-1)/2n$ because $\Phi((x_\invest^* - x_L^*)/\sigma) \to 0$.



Now consider a function $\widehat{x}_\invest(\sigma_L)$ that approaches $(n-1)/2n$ as $\sigma_L \to 0$. Observe that
\begin{equation*}
    \widehat{x}_\invest(\sigma_L) + \frac{\sigma_F^2}{\sigma} \lambda \left( \frac{\widehat{x}_\invest(\sigma_L) - x_L^*}{\sigma} \right) - \frac{n-1}{n} S \left( \frac{\widehat{x}_\invest(\sigma_L) - x_L^*}{\sigma} \right) \to 0
\end{equation*}
because $(\widehat{x}_\invest(\sigma_L) - x_L^*)/\sigma \to \infty$. This means that $(n-1)/2n$ is a solution to $\pi_F^\invest(x_\invest; $ $\widecheck{x}_L^*, x_\invest) = 0$. It implies further that $g(\widehat{x}_\invest(\sigma_L)) \to 1$ as $\sigma_L \to 0$.
Therefore, both actions are rationalizable for leader types $x_L \in (\widecheck{x}_L^*,1)$ and for follower types $x \in (\widecheck{x}_\invest^*, (n-1)/2n)$ under $h = \invest$. The proof is complete. \qed









