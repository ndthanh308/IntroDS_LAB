
\section{So, Does One Soros Make a Difference?}

Although our model is stylized, the forces we uncover are present in a range of economic settings, in particular, financial stability and crises. In financial economics, a substantial body of literature highlights coordination failures as a key contributor to fragility. These failures occur when economic agents undertake ``destabilizing'' actions based on the anticipation that others will do the same. The outcome is a self-fulfilling crisis, commonly referred to as panic. On the other hand, it is often a reason for discussion about whether a crisis happens due to flawed fundamentals or due to panic. The global games approach has become very popular as it provides a bridge that connects these two perspectives. Using this approach, those papers (for example, \cite{corsetti_et_al_2004}) predict that the presence of a large player (our leader) may lead to a higher degree of (de)stabilization if that is the large player's preferred outcome. Our results show that this might not be necessarily the case.

% In standard models of currency attacks (\cite{morris_shin_1998}) or bank runs (\cite{goldstein_pauzner_2005}), crises happen as a combination of panic and fundamentals: specifically, the possibility of bad fundamentals triggers panic, that is, a possible coordination failure on a destabilizing action. More formally, in these static games, the space $\Theta$ in which the fundamentals belong (usually the entire real line), can be partitioned into three regions: a region where the destabilizing action will be taken for sure solely due to fundamentals being flawed (specifically, lower than a threshold $\underline{\theta}$), a region where the destabilizing action will not be taken due to very strong fundamentals (higher than a threshold $\bar{\theta}$) and an intermediate region in which there are multiple equilibria and panic comes into play. In particular, the global games approach delivers the prediction that as noise tends to zero, there will be a threshold $\theta^*$ such that the destabilizing action will be taken for fundamentals worse than $\theta^*$ and not taken for fundamentals above $\theta^*$. In this case, one can call the destabilization that occurs for fundamentals between $\underline{\theta}$ and $\theta^*$ a panic-driven crisis.
% The main implication of our results is that when a large player (our leader) is present, nonequilibrium beliefs will also matter in these settings, and such predictions are fragile when one uses rationalizability as a solution concept. In particular, destabilization may not happen if the fundamentals are between $\underline{\theta}$ and $\theta^*$, but on the other hand, it may happen for fundamentals between $\theta^*$ and $\overline{\theta}$. 



%It is interesting to discuss the implications of our results for the study in \cite{corsetti_et_al_2004}. 
\cite{corsetti_et_al_2004} analyze the question of whether large visible traders increase the vulnerability of a currency to speculative attacks. Their model features a single large investor and a continuum of small investors who decide whether to attack a currency based on their private information about the fundamentals. One of their results is that the large player's action makes small traders more aggressive compared to the case where the large player is absent.

Although our framework differs from that of \cite{corsetti_et_al_2004}, our results provide valuable insights into the question at hand. In their study, the visible trader serves as our equivalent of a leader. However, in our model, the leader is not characterized by size or power, as each follower possesses equal influence. Nonetheless, the leader's presence induces more aggressive behavior among the followers. This effect arises because our model involves a finite number of players, resulting in each individual's actions affecting the payoffs of others. Additionally, the leader's actions serve as signals to the followers, conveying a portion of her information. Even as we approach the assumption of a continuum by increasing the number of followers infinitely, this overall impact remains, driven solely by informational factors.

Our findings demonstrate that a powerful player is not necessary to elicit this effect; a visible agent is sufficient. However, this holds true only if the condition that yields unique rationalizable behavior is satisfied. In \cite{corsetti_et_al_2004}, the focus is on monotone strategies to derive their results, while our approach complements theirs by examining the uniqueness or multiplicity of rationalizable behavior rather than equilibrium behavior. In cases where our sufficient condition is met, our model aligns with the predictions of \cite{corsetti_et_al_2004} (see Proposition 7 of that paper).

If that condition is not satisfied, however, then for specific pairs of ($\sigma_L,\sigma_F)$  the game features multiple rationalizable strategy profiles. This, in the language of \cite{corsetti_et_al_2004}, implies that even if the large trader is visible, the fear of follower miscoordination may deter certain types of the leader from attacking the currency. Thus, it may be the case that the attack never happens even if it were going to be successful, as long as the large trader cannot take down the currency on her own. Moreover, even if the large trader does attack the currency, there is still a possibility that it will survive, resulting in a negative payoff for her. This is due to the lack of guarantee that the small traders will coordinate their actions with the leader. Furthermore, if the visible trader does not deem the attack worthwhile, the followers may still opt to do so, given their uncertainty regarding the leader's motives. Was it because the fundamentals were too strong or due to her fear of followers miscoordinating? Thus, they might still coordinate on attacking, as long as it is possible that they, on their own, can take down the currency.

A similar argument applies to models of bank runs, such as \cite{goldstein_pauzner_2005}. The role of the leader is now played by a large and visible creditor. While equilibrium beliefs would make the smaller creditors imitate the action of the large player, when non-equilibrium beliefs are taken into account, this no longer necessarily holds. Thus, even if the large creditor would ideally not run on the bank for some intermediate values of fundamentals, she may be forced to do so out of fear that smaller creditors will, even after observing her action.


In a broader sense, these models predict that as noise diminishes, the large player can attain the most favorable outcome from her perspective. This implies that a significant investor contributes to increased fragility in the context of a currency attack, whereas in a bank-run model, she fosters greater stability. However, our paper emphasizes that this prediction becomes fragile when considering rationalizable behavior. Specifically, the leader can only achieve the optimal outcome from her viewpoint when a unique rationalizable outcome obtains. Consequently, our model highlights the significance of precise information on the part of the followers, rendering the presence of the large player less influential. In fact, smaller players may coordinate their actions against her, leading to a scenario where the large player refrains from taking actions that would (de)stabilize the economy, even if the underlying fundamentals justify such actions perfectly. Of course, it should be noted that what we have just discussed is a novel effect that will be present in regime change games. Whether this effect will play a significant role, like in our setting, or not, may depend on the intricacies of the specific model considered. 


% \section{Discussion of Solution Concept}
\section{Why $\Delta$-rationalizability?}
Before we conclude, we discuss our choice of $\Delta$-rationalizability as our solution concept. % \subsection{Why Rationalizability} 
The global games approach has gained significant popularity due to its ability to provide unique predictions about equilibrium behavior. However, when we deviate from the static benchmark or introduce signaling, achieving uniqueness becomes more challenging. Although one can usually establish uniqueness within a specific class of strategies (i.e., monotone strategies), this may exclude other equilibria involving more complex strategies which are difficult to characterize. We, therefore, choose rationalizability as our solution concept since it provides the best possible bound on equilibrium outcomes.




% \subsection{Why $\Delta$-rationalizability?}
While the canonical rationalizability concept for static games of incomplete information, interim correlated rationalizability (ICR) (see \cite{dekel_et_al_2007}), is well established, this is not the case for multistage games. In this paper, we choose $\Delta$-rationalizability as our preferred concept because it is fully consistent with  Harsanyi’s approach and is particularly suitable for addressing the questions we investigate. Additionally, it simplifies notation and analysis by reducing the need for technical details.\footnote{~Specifically, there is no need to define a Harsanyi type space.} Another possible candidate is interim sequential rationalizability (ISR), a generalization of ICR to multistage games developped in \cite{penta_2012}. Another approach would involve applying ICR to the normal form of the game as described by \cite{chen_2012}. As it turns out, for our model, these three approaches yield equivalent results. This equivalence stems from the following observation: if we were to specify a type space as mandated by ICR and ISR, our game would become a game with no information in the sense of \cite{penta_2012}. Consequently, the predictions generated by ISR would be identical to those obtained by applying ICR to the normal form of the game. Given that our restrictions on first-order beliefs, $\Delta$, can be derived from the type space, Proposition 3 of \cite{battigalli_et_al_2011}  implies that these predictions coincide with the ones derived from ICR.


% \subsection{Discussion of Assumptions}
% Our model is built upon a set of explicit and implicit assumptions. For our main result to remain valid, it is crucial that the miscoordination effect outweighs the signaling effect, as we have described in detail. In this section, we will discuss the potential impact on the results if we were to relax these assumptions. First, we assumed an improper prior over the real line. Of course, one should not interpret this as an objective distribution, but rather as a convenient way to parametrize the type space. Had we assumed a non-flat prior distribution, multiplicity of equilibria and thus, rationalizable behavior, could then come due to the interplay between public and private information as in \cite{hellwig_2002}. Therefore, by assuming an improper prior, we can arrive at the multiplicity result and the conditions under which this will be obtained more cleanly, isolating and emphasizing the new forces we uncover. 

% Moreover, the fact that noise comes from a Gaussian distribution is also innocuous, while making the analysis analytically tractable. The main result of the paper goes through for general log-concave and symmetric distributions in the limit as the noise on followers' information becomes arbitrarily small. To see this, consider the setting of the main model, and observe that irrespective of the distribution of shocks, the leader choosing action $\invest$ whenever $\theta\geq 0$ and followers imitating the leader is always rationalizable since it is always an equilibrium behavior. Now, consider the marginal type of a follower, that is, the follower whose signal is exactly equal to the conjectured threshold that other follower use when playing a monotone strategy. In the subgame that follows action $\invest$, if this type's signal is negative (which means the threshold used by other followers is a negative number), then the probability he attaches to other followers' signals being higher than his own converges to 0 as noise vanishes, irrespective of the distribution that noise comes from. But when this signal is positive, \footnote{~This means that in this subgame, our follower believes that other followers use a positive threshold. These types of beliefs cannot be part of an equilibrium but as our discussion shows, they can be part of rationalizable behavior.} this probability converges to a number which is at least 1/2.\footnote{~In the case of normality, this number is exactly 1/2 which is why $\xiup$ corresponds to the risk dominant threshold of the subgame that follows action $\invest$ of the leader.} This happens because, in this limit, the conditional rank belief function $R^h (x_{\invest},z)$ is always strictly smaller than 1/2 in the subgame where the leader chooses action $\invest$. Thus, the best response of a follower to the leader's strategy of choosing action $\invest$ if $\theta\geq 0$ is a set with at least two values. This immediately implies that the game features multiplicity of rationalizable behavior. What the normality assumption allows us to do, is to derive a necessary and sufficient condition for uniqueness, something that may be intractable for other distributions (especially the necessity part). 











\section{Conclusion}
In this paper, we adopt an informational approach to the study of leadership and analyze whether efficient leadership can be obtained in a framework that features strategic complementarities. Our main result shows that it is indeed possible for highly informed followers to undermine the leadership's role in facilitating coordination and even cause the leader herself to choose inefficient actions. We have identified the tension between two main forces that drive our results: the signaling effect of the leader's actions and the miscoordination effect stemming from the followers' dispersed information. We have established conditions regarding the precision of information of the leader and the followers under which either effect can dominate, leading to a unique or multiple rationalizable play and efficient or inefficient outcomes. 


It is important to note that our model is highly stylized, and several questions remain open for further exploration. The most promising and significant one, we believe, is the endogenous determination of the leader, particularly in relation to the costs associated with acquiring information. In our model, the leader is exogenously chosen, and access to information sources is assumed to be cost-free. However, if a team of individuals can acquire costly information about the state, would a leader endogenously emerge? Would players understand the negative effects of everyone acquiring information and opt to have only one player, the leader, do so? We leave these and other questions for future research.

Finally, it should be noted that while our results may support the centralization of information, this conclusion only applies within the examined framework. In different scenarios, although we anticipate that precise information on the followers' part would still undermine coordination and efficient leadership as we have described, there may be other reasons why information dissemination is desired. Therefore, a broader avenue for future research is to examine under what circumstances dispersed information is preferred, despite its negative impact on facilitating coordination.