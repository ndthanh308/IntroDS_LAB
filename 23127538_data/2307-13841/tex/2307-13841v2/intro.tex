\section{Introduction}

Coordination problems are pervasive in a wide range of
socio-economic phenomena, including but not limited to corporate culture, bank runs, currency attacks and political change. 
% Commonly, in these settings, it is the case that one of the actors, a leader, is endowed with a special role: determining the correct course of action and aligning followers' incentives in order to facilitate coordination.
In many of these scenarios, a leader is responsible for determining the correct course of action and aligning the incentives of followers to facilitate coordination.
This task is often challenging due to the inherent uncertainty of certain economic environments and the positive spillover effect of coordination that typically benefits everyone involved more than each actor separately. 
%Such a task is challenging both due to the uncertainty inherent in many economic situations and due to the fact that coordination usually has positive spill-overs that benefit everyone more than each follower or the leader personally. %In such scenarios the role of the leader is at least two-fold. First, the leader should act as a motivator for her team of followers. 
Leaders can potentially succeed in their role by \textit{leading by example}. Exerting costly effort or taking a risky action can signal to the followers the high return potential of this course of action. 
%Moreover, by doing so, the leader can also act as a communicator of information. 
However, this may not be enough to discipline the followers on coordinating on the correct action due to the dispersed information that creates strategic uncertainty about what each follower believes. This strategic uncertainty that agents face concerning the actions and beliefs of others may lead to undesirable outcomes. It is, therefore, natural to ask whether such coordination can be achieved and what are the specific characteristics of the leader and the followers that facilitate or undermine it. This paper explores this question from an informational perspective and argues that precise information held by followers can potentially undermine efficient leadership.



Our framework focuses on a leader who is perfectly informed about the state of nature, and a team of followers who have access to private information. The leader leads by example---she makes the first move, which the followers observe.
Then, they simultaneously make their choices, incorporating the information that becomes available due to the leader's action. Our model, thus, features signaling on the part of the leader. It is important to note that the leader is not special in any other way besides acting first and having an informational advantage. We do not consider issues of leader's credibility or special characteristics that might make her a better or worse leader. We take this route in order to isolate the effect that is solely due to information. Our main finding is that if the followers' information is precise enough, efficient leadership can be undermined. More importantly, the leader might act inefficiently due to fear of miscoordination by the followers. Conversely, if the followers have sufficiently imprecise information, the fully efficient outcome is achieved.

Formally, our game is a two-stage game with binary, irreversible actions that features strategic complementarities both within and across stages. We use $\Delta$-rationalizability as our solution concept, which extends extensive-form rationalizability \`{a} la \cite{pearce_1984} to games with incomplete information \citep{battigalli_siniscalchi_2003}.  We fully characterize the rationalizable set for the leader and followers, that is, the set of strategies that are rationalizable for certain types of each player. Our main results translate to finding necessary and sufficient conditions about the magnitude of the noise of followers' information that induce uniqueness or multiplicity of rationalizable play. When rationalizable play is unique, the leader disciplines the followers on the correct course of action. When this is not the case, followers might play against the leader which in turn may make the leader choose not to take the desired action despite being perfectly informed about the state. It is important to note that our results obtain under alternative solution concepts such as interim sequential rationalizability \citep{penta_2012} or interim correlated rationalizability applied to the normal form of the game \citep{dekel_et_al_2007, chen_2012}. 

Our main result is driven by the tension between two opposing effects: the signaling effect of the leader's choice on the followers and the miscoordination effect that arises from the followers' dispersed information.
Simply put, the leader's action reveals some information to the followers and can influence their decisions.
% Heuristically, these forces act as follows: The leader's choice reveals to the followers part of the information that the leader has. Moreover, her action has a spillover effect on the followers.
However, each follower can be certain neither about which leader's type chose the observed action nor about the types and strategies of other followers, which leads to the miscoordination effect.
% However, despite these two facts, a follower can never be sure neither about the particular type or strategy the leader followed nor the type or strategy of other followers. Thus, the miscoordination effect arises. 
Knowing this, the leader may be forced to choose the undesired action in the first place. The signaling effect is captured by the truncation of the conditional distribution of the state that followers deem possible after observing the leader's choice. The miscoordination effect is captured by the conditional higher-order beliefs a follower holds about the state and the beliefs of other followers. In the limit as followers' information becomes arbitrarily precise, this effect is captured exactly by the \textit{conditional rank belief function}, a generalization of the rank belief function introduced in \cite{morris_et_al_2016} and \cite{morris_yildiz_2019}. This function yields the probability that a follower assigns to the event that another follower has received a signal about the state less than his own, which now depends on the behavior of the leader. Specifically, we show that our conditions that ensure unique rationalizable play, essentially bound the derivative of the conditional rank belief function so that the signaling effect dominates the miscoordination effect.

Dominance of the signaling effect over the miscoordination effect means that each follower wants to imitate the leader irrespective of his signal and without worrying about the actions of other followers. For this to be the case, a necessary condition is that the leader knows the true state or he is arbitrarily well informed. If this is not true, then a follower will always worry about potential miscoordination. The tension comes because both these forces depend on the leader's behavior but in opposite ways: more ``aggressive'' (and, thus, more efficient in the case where the state is positive) behavior by the leader weakens the signaling effect and strengthens the miscoordination effect by increasing the strategic uncertainty that followers face. This undermines coordination on the correct course of action on the part of followers. The only way this can be overcome, so that the efficient behavior of the leader and followers is supported as the unique rationalizable behavior, is for the noise in the followers' information to be sufficiently high. If this is not the case, the miscoordination effect dominates: even if it is common knowledge that the state is positive, followers may still choose not to take the ``correct" action. The fact that this behavior on the followers' part is rationalizable, makes the choice of the leader not to take this action in the first place rationalizable as well. In the language of financial economics, this inefficient outcome may prevail solely due to panic, that is, solely due to the fear of miscoordination and not due to fundamentals being weak. 

We extend our results to a setting where the leader is not perfectly informed about the state but has access to private information like the followers. The novel effect is that this uncertainty “noises up” the learning induced by the action of the leader: whereas in the benchmark model this knowledge leads to a truncation in the support of posterior beliefs about the state, now posterior beliefs retain full support over the entire real line. This extension, which from an applied point of view is clearly more natural, establishes the generality of our result from a methodological perspective because it restores two-sided dominance, a property that is absent in our main model but is key to the global games literature, to which this paper broadly belongs. 

While our model is very stylized, it plays the role of a metaphor that can be interpreted in a way that fits a variety of situations. As we stated, it features strategic complementarities and does not allow for free riding, an issue that is usually central in organizational settings or public good provision games. However, by doing this, we can analyze more diverse phenomena, in which there might not be a leader, in the standard use of the term, but a player whose action is visible to smaller players. What is important, is the visibility of this player that grants them the leadership role. The leader, thus, can be a nation that initiates an environmentally friendly policy in the hope that more countries will follow or the manager of a firm that wishes to induce her employees to exert costly effort when her only instrument is her own choice to work or shirk. It can also be a prominent investor contemplating whether to attack a currency or not or whether to roll over debt or not. Or, the leader can be a vanguard in revolution. By attacking the regime first, she can inspire other citizens to do so. As we point out, the (application-specific) ideal outcome from the point of view of the leader is supported as the unique rationalizable outcome only when the conditions we identify are satisfied. As an example, we discuss the implications of the forces involved in our model for financial stability and panic-driven crises. Specifically, we compare the predictions of our model to results in the finance literature according to which the leader can achieve her desired outcome. This need not be the case in our framework. In particular, when followers' information is accurate, they may coordinate against the leader, resulting in a negative payoff for her. This makes the presence of the leader less influential, in the sense that she may not take the action that (de)stabilizes the economy even if the fundamentals perfectly justify doing so.

Of course, our results are not to be interpreted as suggesting that followers should always be ``kept in the dark.'' While this does facilitate coordination and efficient leadership in our setting, there are many scenarios in which one would not only want the followers to hold precise information, but also the leader might wish to learn from them. What our results point out is the negative effect of precise information on the followers' part. However, this effect could be dominated by other positive ones that are application specific.








\subsection{Related Literature}

Our paper contributes to two strands of literature. Primarily, it adds to the literature on global games \citep{carlsson_van_damme_1993, morris_shin_2003} by characterizing rationalizable outcomes when one of the players moves first and her action is observable by the others. \cite{corsetti_et_al_2004} is the most closely related paper as they investigate a large player's role in currency attacks. Although the settings are different, our results complement theirs in showing that predictions such as the ones they derive can also be made in our framework only if followers hold imprecise information. If this is not the case, the large player may be less influential on the outcome of the game and even receive a negative payoff if followers coordinate agains her. We discuss in detail how our paper relates to \cite{corsetti_et_al_2004} in Section 6. \cite{loeper_et_al_2014} study the role of experts whose actions can influence agents' behavior and show that the outcome is biased towards experts' interests. This feature is also present in our paper, again, under the qualification that there is unique rationalizable play. The difference between the two models is that in \cite{loeper_et_al_2014}, the experts' preferences differ from those of the agents, whereas in our model, the leader and followers share the same objective. Moreover, experts' actions in \cite{loeper_et_al_2014} have no spillover effect and do not affect directly the outcome of the game. Finally, while the main question of interest in \cite{loeper_et_al_2014} is how experts influence agents' actions, we ask how information on the part of followers affects coordination and the choice of the leader.
In \cite{angeletos_et_al_2006} there is a perfectly informed policymaker whose action is observable by the players before the coordination stage so the game also features signaling. In their framework, this leads to equilibrium multiplicity. The difference between the two papers is that the leader and the followers have perfectly aligned incentives in our model, while in \cite{angeletos_et_al_2006} there is a conflict of interest.  \cite{basak_zhou_2020} study a regime change game where agents choose sequentially and there is a principal who can dynamically disclose information to dissuade them from attacking. The disclosure policies studied also remove the two-sided dominance property. However,  in our model, the leader and followers share the exact same goal and, moreover, we do not have disclosure of information but signaling by the leader. \cite{angeletos_et_al_2007} extend the standard static benchmark of a regime change game,  to a dynamic setting in which players may attack a regime multiple times and where learning features a prominent role. They derive a multiplicity result similar to ours, due to the role of the truncation of conditional beliefs that is central to both papers. However, in their work, agents' behavior in the continuation does not affect how they act when the game begins, while, in our setting, followers' behavior in the sub-game obviously affects the decision of the leader at the beginning of the game.


Moreover, our paper is related to the economics of leadership literature pioneered by \cite{hermalin_1998}.
Our work is most closely related to  \cite{komai_et_al_2007} and \cite{komai_stegeman_2010}.
The former examines the informational role of the leader in a moral hazard in teams model, while the latter extends this setting by exploring leader selection and investment in information.
% The second paper generalizes this setting substantially and derives a similar result. Moreover, it studies questions such as leader selection and investment in information.  
 Our results validate those in \cite{komai_et_al_2007} and \cite{komai_stegeman_2010} in that we also derive efficient outcomes when the leader has a sufficient informational advantage and reveals, through signaling, part of her information to followers. 
However, our contribution lies in the converse direction: whenever the leader does not have a sufficient informational advantage, then it might be the case that the leader herself acts inefficiently. In addition, as our extension shows, the result does not necessarily rely on who is better informed but rather on whether the signaling effect of the leader's action dominates the miscoordination effect inherent to the followers' problem due to dispersed information. Other papers in the economics of leadership literature that relate to our work include \cite{bolton_et_al_2013} and \cite{dewan_myatt_2008}. 
These papers explore in a beauty contest framework, the qualities a leader must possess to successfully determine an organization's mission in changing environments while ensuring coordination among followers. Our paper differs with respect to \cite{bolton_et_al_2013} and \cite{dewan_myatt_2008} in that, in our model, the leader neither chooses the signals that followers observe as in \cite{dewan_myatt_2008} nor has the final say about which action will be taken as in \cite{bolton_et_al_2013}. Her role is simpler: she just chooses an action that the followers observe. By simplifying the environment in this way, we are able to isolate and analyze the effect of the information that followers possess on leadership in a clearer manner: the leader may choose incorrectly in the first place and potentially trap herself and the followers in the wrong course of action, a feature absent in \cite{bolton_et_al_2013} and \cite{dewan_myatt_2008}.
% Both papers study from different perspectives the qualities a leader must have in order to successfully determine the mission of an organization in a changing environment while ensuring coordination of followers on this mission. 

 %Our paper also adds to the literature on global games \citep{carlsson_van_damme_1993, morris_shin_2003}. 
% Ever since, this literature has vastly expanded with the papers most closely related to ours being  \cite{corsetti_et_al_2004}, \cite{angeletos_et_al_2006}, \cite{angeletos_et_al_2007}, \cite{dasgupta_2007} and \cite{loeper_et_al_2014}. 
%The paper by \cite{corsetti_et_al_2004}  and \cite{dasgupta_2007} feature two-stage games where a large player in the former or a continuum of players in the latter have the option to move first or delay their choice until the second period. Both these papers feature signaling. In \cite{corsetti_et_al_2004} the action of a large player is observable by the followers while in \cite{dasgupta_2007}, followers observe a noisy signal of the aggregate first stage action.\footnote{~Noisy signaling is necessary, since with a continuum of first-stage players, perfect observation of the aggregate action makes the followers able to correctly infer the true state which directly leads to equilibrium multiplicity.} Our generalized model of Section 4 provides some new insights regarding the predictions of \cite{corsetti_et_al_2004} and identifies conditions under which these predictions remain valid or not, when one considers rationalizable behavior. The paper by \cite{loeper_et_al_2014} studies the role of opinion leaders in coordination. The informational role of an opinion leader is similar to the one of our leader. However, an opinion leader does not affect the outcome of the game through his action directly, so there is no spillover effect. Their main result corresponds to the outcome of our game when there is unique equilibrium behavior in a similar way in which this happens with the \cite{corsetti_et_al_2004} paper. The main difference is that we show that this is the case only if our condition for unique rationalizable behavior is satisfied. If that is not the case, our results imply that our leader and the large investor in \cite{corsetti_et_al_2004} may choose the inefficient action, while the expert in \cite{loeper_et_al_2014} may choose not to disclose information at all. In \cite{angeletos_et_al_2006} there is a perfectly informed policymaker whose action is observable by the players before the coordination stage so the game also features signaling. In their framework, this leads to equilibrium multiplicity. There is an important difference between the two papers. The leader and the followers have perfectly aligned incentives in our model, while in \cite{angeletos_et_al_2006} there is a conflict of interest. This leads to different effects that are important for the questions studied in each paper. \cite{angeletos_et_al_2007} extend the standard static benchmark of a regime change game,  to a dynamic setting in which players may attack a regime multiple times and where learning features a prominent role. They restrict attention to monotone equilibria and they show that multiplicity might occur in the limit as noise vanishes.  This, of course, implies that had they used an appropriate version of rationalizability as their solution concept, they would obtain multiple rationalizable profiles. 



%One key similarity between our model and the existing literature that involves multi-stage games is the important role of conditional beliefs. While in our model as in \cite{corsetti_et_al_2004} the conditional belief about the fundamental and the conditional higher order beliefs about other followers' information is the result of the action of one agent, the leader, in \cite{angeletos_et_al_2007} the conditional belief is the result of observing that the regime survived the first period attack. This shows that, from a theoretical point of view, whether this information is revealed by the actions of a single entity (our leader), or by the aggregate action of a group of agents is irrelevant. 


%The limit multiplicity result in \cite{angeletos_et_al_2007}, is similar to ours, albeit for different reasons. Specifically, in their model, an equilibrium when an attack happens in period 1 only and never thereafter always exists. This corresponds to the monotone equilibrium behavior our model features, which, moreover, can be obtained as the unique rationalizable play under the conditions we specify. Their multiplicity result is then driven mainly by the fact that they assume a prior mean and the fact that learning occurs, which results in enabling more (less) aggressive behavior when the prior mean is high (low) in the sequel. In our model, even if the subgame following the choice of the leader appears to be a standard global game, it is not. When our necessary and sufficient condition is not satisfied, the subgame features at least two Bayes-Nash equilibria (both in monotone strategies) which, in turn, make the leader have a continuum of rationalizable profiles. That is, our continuation game features multiple equilibria even though our assumptions are chosen to be the more conducive to uniqueness. To put it differently, our result hinges on the fact that it can be potentially the case in the \cite{angeletos_et_al_2007} paper, that even if the initial attack is unsuccessful and the prior mean low, it may still be rationalizable for agents to attack even if it is not an equilibrium behavior. 

%The main difference between our model and existing literature, however, is the opposite direction.  As we argued, we view our main contribution to be the fact that the leader herself might choose the inefficient action simply because she is worried that followers might miscoordinate. This logic is, of course, absent in \cite{angeletos_et_al_2007} since the first-period choice remains unaffected by the fear of miscoordination of agents in the following periods and has not been analyzed in the rest of the papers we discussed.



On the methodological side, our model is an extensive-form game with incomplete information that features strategic complementarities within and across stages. It, therefore, shares elements with \cite{echenique_2004} and \cite{van_zandt_vives_2007}. Moreover, we interpret our results in terms of the \textit{rank belief function}, introduced by \cite{morris_et_al_2016} and \cite{morris_yildiz_2019}. Even though we ask different questions, our theoretical framework is similar to that of \cite{morris_yildiz_2019}, since we also use properties of rank belief functions to analyze the behavior of followers after observing the leader's action. Essentially, as we mentioned, our conditions for unique rationalizable play are derived by bounding the derivative of the conditional rank belief function at its symmetry point.  



\subsection{Organization of the Paper}


The remainder of the paper is organized as follows: Section 2 introduces the main model. In Section 3, we present and discuss the main results. Section 4 extends the results of Section 3 to an alternative information structure for the leader. In Section 5, we explore the implications of the results for the leader's role in financial stability and crises. Section 6 discusses the choice of the solution concept. Finally, Section 7 concludes. All proofs are in Appendix A.

