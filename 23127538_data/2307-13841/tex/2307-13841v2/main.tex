\documentclass[12pt]{article}
\usepackage
[
    a4paper,
    % left=3cm,
    % right=2cm,
    % top=3cm,
    % bottom=3cm,
    margin = 3cm,
]
{geometry}
\usepackage{amsfonts}
\usepackage{amsmath}
\usepackage{amssymb}
\usepackage{amsthm}
\usepackage{graphicx}
\usepackage{tikz}
\usepackage{sgame}
\usetikzlibrary{decorations.pathreplacing}
\usepackage{color}
\usepackage{mathpazo}
\usepackage{times}
\usepackage{caption}
\usepackage{subcaption}
\usepackage{natbib}
\usepackage[labelsep=space]{caption}
\captionsetup{justification=centering, labelfont=bf}
\usepackage{soul}
\usepackage{url}
\usepackage[multiple]{footmisc}
\usepackage{comment}


\newtheorem{theorem}{Theorem}
\theoremstyle{definition}
\newtheorem{assumption}{Assumption}
\newtheorem{claim}{Claim}
\newtheorem{corollary}{Corollary}
\newtheorem{definition}{Definition}
\newtheorem{example}{Example}
\newtheorem{lemma}{Lemma}
\newtheorem{proposition}{Proposition}
\newtheorem{remark}{Remark}

%%%%%%%%%%%%%%%%%%%%%%%%%%%%%%%%%%%

\newcommand\citeapos[1]{\citeauthor{#1}'s (\citeyear{#1})}

\DeclareMathOperator\erf{erf}
\DeclareMathOperator*{\argmax}{arg\,max}\DeclareMathOperator*{\argmin}{arg\,min}

\DeclareFontFamily{U}{mathx}{\hyphenchar\font45}
\DeclareFontShape{U}{mathx}{m}{n}{
      <5> <6> <7> <8> <9> <10>
      <10.95> <12> <14.4> <17.28> <20.74> <24.88>
      mathx10
      }{}
\DeclareSymbolFont{mathx}{U}{mathx}{m}{n}
\DeclareFontSubstitution{U}{mathx}{m}{n}
\DeclareMathAccent{\widecheck}{0}{mathx}{"71}


\newcommand{\RR}{\mathbb{R}}
\newcommand{\NN}{\mathbb{N}}
\newcommand{\EE}{\mathbb{E}}
\newcommand{\dd}{\mathrm{d}}
\newcommand{\ee}{\mathrm{e}}
\newcommand{\Action}{\mathcal{A}}
\newcommand{\vl}{\,\vert\,}
\newcommand{\lcb}{\big\{}
\newcommand{\rcb}{\big\}}
\newcommand{\llcb}{\bigg\{}
\newcommand{\rrcb}{\bigg\}}
\newcommand{\lsb}{\big[}
\newcommand{\rsb}{\big]}
\newcommand{\llsb}{\bigg[}
\newcommand{\rrsb}{\bigg]}
\newcommand{\lb}{\big(}
\newcommand{\rb}{\big)}

\newcommand{\cmmt}[1]{{\color{red} #1}}
\newcommand{\note}[1]{{\color{blue} #1}}

\newcommand{\prob}{\mathrm{Pr}}
\newcommand{\invest}{\mathcal{E}}
\newcommand{\notinvest}{\mathcal{N}}

\newcommand{\nbar}{\overline{N}}
\newcommand{\one}{\mathbb{1}}
\newcommand{\thetat}{\tilde{\theta}}
\newcommand{\thetazerostar}{\theta_L^*}
\newcommand{\noise}{\tilde{\varepsilon}}
\newcommand{\xzerohat}{\widehat{x}_L}
\newcommand{\thetazerohat}{\widehat{\theta}_L}
\newcommand{\xhhat}{\widehat{x}_h}
\newcommand{\xihat}{\widehat{x}_{\invest}}
\newcommand{\xnhat}{\widehat{x}_{\notinvest}}
\newcommand{\xistar}{x_{\invest}^*}
\newcommand{\xnstar}{x_{\notinvest}^*}
\newcommand{\xzerostar}{x_L^*}
\newcommand{\cdfinvest}{G^\invest(\theta; x, z)}
\newcommand{\scdfinvest}{\Psi^\invest(\theta; x_j, \widehat{\theta}_L)}
\newcommand{\scdfinvesteq}{\Psi^\invest(\theta; \xistar, \thetazerostar)}
\newcommand{\cdfnotinvest}{G^\notinvest(\theta; x, z)}
\newcommand{\scdfnotinvest}{\Psi^\notinvest(\theta; x_j, \widehat{\theta}_L)}
\newcommand{\scdfnotinvesteq}{\Psi^\notinvest(\theta; \xnstar, \thetazerostar)}
\newcommand{\cdfhist}{G^h(\theta; x, z)}
\newcommand{\scdfhist}{\Psi^h(\theta; x_j, \widehat{\theta}_L)}
\newcommand{\scdfhistnobar}{\Psi^h(\theta; x_j, \widehat{\theta}_L)}
\newcommand{\pdfinvest}{g^\invest(\theta; x_j, \widehat{x}_L)}
\newcommand{\spdfinvest}{\psi^\invest(\theta; x_j, \widehat{\theta}_L)}
\newcommand{\pdfnotinvest}{g^\notinvest(\theta; x_j, \widehat{x}_L)}
\newcommand{\spdfnotinvest}{\psi^\notinvest(\theta; x_j, \widehat{\theta}_L)}
\newcommand{\pdfhist}{g^h(\theta; x_j, z)}
\newcommand{\spdfhist}{\psi^h(\theta; x_j, z)}
\newcommand{\spdfhistnobar}{\psi^h(\theta; x_j, \widehat{\theta}_L)}



\newcommand{\thetalow}{\underline{\theta}_L}
\newcommand{\thetaup}{\overline{\theta}_L}
\newcommand{\thetastar}{\theta_L^{*}}
\newcommand{\thetadoublestar}{\theta_L^{**}}
\newcommand{\xilow}{\underline{x}_\invest}
\newcommand{\xiup}{\overline{x}_\invest}

\newcommand{\xnlow}{\underline{x}_\notinvest}
\newcommand{\xnup}{\overline{x}_\notinvest}

\newcommand{\xhlow}{\underline{x}_h}
\newcommand{\xhup}{\overline{x}_h}
\newcommand{\xhstar}{x_h^*}
\newcommand{\xhdoublestar}{x_h^{**}}
\newcommand{\xidoublestar}{x_\invest^{**}}
\newcommand{\xndoublestar}{x_\notinvest^{**}}

\newcommand{\br}{\mathrm{BR}}
\newcommand{\xlow}{\underline{x}_L}
\newcommand{\xup}{\overline{x}_L}
\newcommand{\marg}{\mathrm{marg}}

\linespread{1.286}

\begin{document}

\title{It's Not Always the Leader's Fault: How Informed Followers Can Undermine Efficient Leadership\thanks{~An earlier draft of this paper was titled ``The Signaling Role of Leaders in Global Games". We owe special thanks to Alessandro Pavan for his support and guidance throughout the writing process. We also wish to thank George-Marios Angeletos, Sandeep Baliga, Georgy Egorov, Stephen Morris, Wojciech Olszewski, Harry Pei, Alvaro Sandroni, Marciano Siniscalchi, Zhen Zhou as well as conference participants at the 34th Stony Brook International Conference on Game Theory, the C.R.E.T.E. 2022 conference and various seminar audiences at Northwestern University for valuable comments and suggestions.}} 

\author{Panagiotis Kyriazis\thanks{~Corresponding author. Department of Economics, Northwestern University. Email: pkyriazis@u.northwestern.edu.}
\and Edmund Y. Lou\thanks{~Department of Economics, Northwestern University. Email: edmund.lou@u.northwestern.edu.}}
\date{\today}

\maketitle
\thispagestyle{empty}

\begin{abstract}
Coordination facilitation and 
efficient decision-making are two essential components of successful leadership. In this paper, we take an informational approach and investigate how followers' information impacts coordination and efficient leadership in a model featuring a leader and a team of followers. We show that efficiency is achieved as the unique rationalizable outcome of the game when followers possess sufficiently imprecise information. In contrast, if followers have accurate information, the leader may fail to coordinate them toward the desired outcome or even take an inefficient action herself. We discuss the implications of the results for the role of leaders in the context of financial fragility and crises. \\

\noindent Keywords: Global games, leadership, coordination.

\noindent JEL Classification: C72, C73, D83.
\end{abstract}



\newpage
\pagenumbering{arabic}

% Figure environment removed

\section{Introduction}
Automatic 3D reconstruction of clothed humans using image inputs has gained increasing significance due to its potential applications in a wide array of AR/VR scenarios. High-fidelity reconstructions typically depend on sophisticated capture systems, which are developed with dense camera arrays~\cite{collet2015high,joo2015panoptic,joo2018total}, programmable light-stages~\cite{Vlasic2009, guo2019relightables}, and depth sensors~\cite{newcombe2011kinectfusion,DoubleFusion,BodyFusion,dou2016fusion4d,newcombe2015dynamicfusion}. However, stringent capture environments equipped with complex hardware pose significant challenges for consumer-level applications.


In this context, considerable research effort has been dedicated to developing methods that allow for more flexible capture configurations, such as utilizing a few RGB inputs. Among these works, learning implicit functions \cite{iccv2020PIFu, saito2020pifuhd, hong2021stereopifu} has proven effective in achieving highly detailed reconstructions by integrating the advancements of deep neural networks. These methods employ large multi-layer perceptrons (MLPs) to predict the occupancy probability or truncated signed distance function (TSDF) value of every queried 3D point based on its associated local feature, which is extracted from images. They can recover a continuous surface at arbitrary resolutions without topology restrictions.


However, in typical MLP-based implicit networks, the occupancy or TSDF value at each location is solved independently with planar image features, rendering them less capable of addressing challenging cases such as occlusions. Consequently, these methods suffer from generalization and robustness issues, particularly when tackling strong occlusions caused by large motion or multiple interacting humans. 
Some follow-up studies  \cite{zheng2021deepmulticap,zheng2021pamir,huang2020arch} utilize an extra geometric model, SMPL~\cite{Loper2015}, to improve robustness by introducing strong shape priors. 
Their success typically relies on the assumption of geometrical similarity \cite{huang2020arch} between the shape prior and target reconstruction, making them intractable for handling complex cases with loose clothes and sensitive to errors in SMPL model fitting.



%\ping{this paragraph sounds like `TSDF is better than MLP/SMPL, and we use TSDF to solve the problem'. But in Sec 3, we are telling a different story, saying `MLP needs a 3D convolutional encoder'. We need to make these two sections consistent.}\sicong{I think in this paragraph we claim that the TSDF}


%We opt for Trucated Signed Distance Funtion (TSDF) volumetric representations as they are naturally suitable for convolution operations, which have shown remarkable performance for learning hierarchical features on 2D visual perception tasks \cite{SunXLW19}. 
%Meanwhile, TSDF also describes the gradual geometry change around shape surface, which is not reflected by occupancy volume. 

We instead revisit the 3D volumetric representation and resort to 3D convolutional neural networks (CNNs) for feature learning, due to their impressive performance in feature learning and the ability to incorporate spatial context. However, volumetric methods and 3D convolution involve discretization, which might raise concerns regarding whether a discretized volume can preserve subtle geometric details as continuous representations learned in implicit functions. We investigate the relationship between volume resolution and quantization error on synthetic data by converting target mesh objects to TSDF volumes, as shown in Figure~\ref{fig:quantization_error}. We observe that the quantization errors are significantly reduced by increasing volume resolution and become nearly negligible when reaching a relatively high resolution (e.g., 512 or higher). In other words, achieving fine-detailed reconstruction is not supposed to be restricted by the use of volume representations as long as a proper volume resolution is utilized. Therefore, we present a method with high-resolution feature volumes, e.g., 256 and 512, while traditional volumetric methods \cite{varol18_bodynet,gilbert2018volumetric} are often limited to much lower resolutions, such as 32 or 128.



On the other hand, an increase in volume resolution may lead to a cubic growth of memory overhead \cite{8100085}. Reducing memory costs while guaranteeing the granularity of volumetric representations is necessary for pursuing high-quality reconstruction. Thus, we adopt a coarse-to-fine approach and cull away irrelevant voxels to build a sparse high-resolution feature volume. At the coarse level, the network computes an initial TSDF by applying a U-Net with sparse 3D CNN \cite{3DSemanticSegmentationWithSubmanifoldSparseConvNet} on the sparse feature volume, which is carved by a visual hull. Through our experiments, it turns out that more than 95\% of the volume grids are discarded by the visual hull culling, making the sparse 3D CNN efficient. At the fine level, the network focuses on a narrow band near the zero-level set of the initial TSDF and discretizes the narrow band with smaller voxels. By employing this narrow-band culling, we further shrink the sampling space, resulting in a relatively small range of grid numbers (usually 300K--500K in our experiments) even with a high volume resolution of 512. The remaining voxels in the narrow band are associated with features that fuse high-frequency information from the computed normal maps upon the low-frequency shape from the coarse level to compute the TSDF at high resolution. The final mesh is then extracted from the TSDF using the Marching-Cube algorithm ~\cite{Lorensen87marchingcubes}.
% Different from the u-net sturcture to preserve global topology context, we then apply a shallow 3dcnn to compute the final TSDF $D_{final}$ which contain more local geometry detail.




% \ping{this paragraph can be expanded. It is an important contribution and often ignored by other works. stress on the novel idea of regressing blending weights instead of colors}

In addition to geometry, high-quality mesh texture is also a crucial factor contributing to visual appearance. Directly computing a color field in 3D space, as in \cite{iccv2020PIFu}, struggles to capture high-frequency texture details, while the neural radiance field (NeRF) \cite{yu2020pixelnerf} or the DoubleField~\cite{shao2022doublefield} require expensive per-instance optimization and are often unstable for sparse input images. In contrast, we adopt an image-based rendering approach to compute a texture atlas map, which is efficient and widely supported in existing computer graphics tools. 
Specifically, we compute a blending weight at each 3D point on the mesh surface to determine its color as a weighted average of the colors at its image projections. The blending weights can be computed at a relatively coarse resolution, e.g., 512 volume resolution in our case, and leave texture details to the high-resolution images, such as 1K or 2K. Unlike previous methods that generate blurry texturing results under sparse input, our method generalizes well on both synthetic and real data with just a few input views. 
Figure~\ref{fig:teaser} shows two examples reconstructed by our method. Despite the challenging garment, pose, and occlusion, our method recovers faithful shape, normal, and texture on the right.

%with a wide variety of poses and clothing styles, and it is also adaptive to handle input image with arbitrary resolutions.
%\sicong{For this concern we claim that when the resolution of dicretized volume meets certain threshold (which is 256 in our experiment), the quantization error can be neglected.} 



In summary, the main contributions of this paper are as follows:
\begin{itemize}
\vspace{-0.1in}
  \item 
  We revisit the 3D volumetric representation and demonstrate that it can support clothed human reconstruction with equal or even better performance compared to implicit representation. 
  \item 
  We develop a memory and computation-efficient method for high-resolution volumetric reconstruction using sophisticated sparse 3D CNN, coarse-to-fine estimation, and voxel culling by visual hull and narrow bands. 
  \item 
  We introduce a novel method to compute a texture atlas map, which captures rich appearance details from high-resolution input images.
  \item 
  We achieve impressive results on standard benchmark datasets Twindom and MultiHuman, significantly reducing the point-2-surface (P2S) precision to approximately 0.2cm from just six input views, with more than $50\%$ error reduction compared to the state-of-the-art methods, including DoubleField~\cite{shao2022doublefield} and PIFuHD~\cite{saito2020pifuhd}.
\end{itemize}
\medskip

% !TEX program = pdflatex
% !TEX root = main.tex


\section{The Model}

We represent a series of interactions between $N$ individuals as a sequence of weighted directed networks with adjacency matrix $A^t$ for $t=0,1,2,\ldots,T$. For each $t$, its entry $A_{ij}^t$ is the outcome of interactions $i \rightarrow j$ suggesting that $i$ is ranked above $j$. This allows both cardinal and ordinal inputs. For instance, in team sports, $A_{ij}^t$ could be the number of points by which team $i$ beat team $j$, or we could simply set $A_{ij}^t=1$ to indicate that $i$ won and $j$ lost. We can include the case where individuals interact multiple times at time $t$ by summing the corresponding entries.

We assume that the values of $A_{ij}^t$ are influenced by a vector of real-valued ranks $\v{s}^t=(s_{1}^t,\dots, s_{N}^t)$, where $s_i^t$ is $i$'s skill, strength or prestige at time $t$.
To model these interactions, we follow SpringRank's approach of imagining the network as a physical system~\cite{de2018physical}. Specifically, each node $i$ is embedded in $\mathbb{R}$ at position $s_i^t$, and each directed edge $i \rightarrow j$ becomes an oriented spring with a non-zero resting length and displacement $s_i^t-s_j^t$. Since we are free to rescale latent space and the energy scale, we set the spring constant and resting length to $1$. The spring corresponding to an edge $i \rightarrow j$ at time $t$ then has energy
\be\label{eqn:staticH}
H_{ij}(s_i^t,s_j^t)=\f{1}{2} \bup{s_i^t-s_j^t-1}^{2} \, .
\ee
If there were no other effects, the total energy of the system at time $t$ would then be 
\be\label{eqn:totalstaticH}
H^t(\v{s}^t) = \sum_{i,j=1}^{N} A_{ij}^t \,H_{ij}(s_i^t,s_j^t) \, .
\ee
If we determined $\v{s}^t$ by minimizing $H^t$ for each $t$ separately, we would simply be applying the static SpringRank model separately to each ``snapshot'' of the network. This would ignore all previous (and future) interactions, and ignore the hypothesis that ranks change smoothly from one time-step to the next.

% Figure environment removed

To model this smoothness, we also assume a dependence between ranks at successive time-steps. Specifically, we extend the Hamiltonian~\eqref{eqn:totalstaticH} with an extra term that models the \emph{self-interaction} between past and current ranks,
\begin{equation}\label{eqn:selfH}
\Hself^t(\v{s}^t,\v{s}^{t-1}) 
= \frac{\kself}{2} \sum_{i=1}^N (s_i^t-s_i^{t-1})^2 \, .
\end{equation}
This can be seen as a set of additional ``self-springs'' that connect the rank of each individual with its own previous rank. The spring constant $\kself$ parametrizes how smoothly we want the ranks to change from one step to the next. In inference terms, $\kself$ is a hyperparameter which we tune using cross-validation.

Summing over all time-steps $0 < t \le T$ and adding this to the pairwise interactions at each time-step then gives a total energy

\begin{align}\label{eqn:fullH}
\Htotal(\{\v{s}^t\}) = \sum_{t=0}^T H^t(\v{s}^t) + \sum_{t=1}^T \Hself^t(\v{s}^t,\v{s}^{t-1}) \, .
\end{align}
We call this the dynamical SpringRank Hamiltonian. The optimal ranks $\v{s}^0,\v{s}^1,\ldots,\v{s}^T$ are those that minimize it.


There are two ways to minimize $\Htotal$. One is to proceed in an online way, moving forward in time. In this approach, we use the static SpringRank model Eq.~\eqref{eqn:totalstaticH} to find the initial ranks $\v{s}^0$ by minimizing $H^0(\v{s}^0)$. As in Ref.~\cite{de2018physical}, the energy is unchanged if we add a constant to all the ranks; we can break this translational symmetry by setting the mean initial rank $(1/N) \sum_{i=1}^N v_i^0$ to zero.
Then, at each subsequent time-step $t \ge 1$, we update the ranks by taking into account both the new pairwise interactions and the self-springs connecting the ranks with their previous values. Namely, given $\v{s}^{t-1}$ and $A^t$, we find the ranks $\v{s}^t$ that minimize $H^t(\v{s}^t) + \Hself^t(\v{s}^t,\v{s}^{t-1})$.

Since this is a convex function of $\v{s}^t$, we can find its minimum by setting its gradient to zero, or equivalently by balancing all the forces $v_i^t$. This yields a system of linear equations:
\begin{align}\label{eqn:fullsolution}
\rup{ D^{out,t}+D^{in,t}- \bup{A^t + (A^t)^\dagger}+\kself\id} \,\v{s}^t
&=\rup{D^{out,t}-D^{in,t}}\v{1} \nonumber \\& +\kself\, \v{s}^{t-1} \, . 
\end{align}

Here 
$D^{out,t}$ and $D^{in,t}$ are diagonal matrices whose entries are the weighted out- and in-degrees $D^{out,t}_{ii}=\sum_{j}A^t_{ij}$ and $D^{in,t}_{ii}=\sum_{j}A^t_{ji}$; 
$\dagger$ denotes the transpose; 
$\id$ is the identity matrix; 
and $\v{1}$ is the all-ones vector.

The matrix on the left side of~\Cref{eqn:fullsolution} is invertible if $\kself > 0$. In particular, its eigenvector $\v{1}$ has eigenvalue $N \kself$. Thus for each $A^t$ and each $\v{s}^{t-1}$, Eq.~\eqref{eqn:fullsolution} has a unique solution $\v{s}^t$. Overall, Eq.~\eqref{eqn:fullsolution} is similar to the regularized version of SpringRank~\cite{de2018physical} with regularization parameter $\alpha= \kself$. However, unlike the static model, there is a term on the right-hand side containing the previous ranks $\v{s}^{t-1}$, creating a Markovian dependence between successive time-steps. We refer to this model as \dsrfull\ (\dsr).

Importantly the online DSR approach does not actually minimize $\Htotal$, instead solving a sequence of minimization problems, one for each time step. To minimize $\Htotal$ instead, we set $\nabla \Htotal(\v{s}^t) = 0$, solving for the minimizers $\v{s}^t$ over all $N(T+1)$ ranks simultaneously, yielding the following system of equations (SI \Cref{sec:h_total_derive}):

\begin{align}\label{eqn:h_total}
\rup{ D^{out,t}+D^{in,t} - \bup{A^t+(A^t)^\dagger} + 2\kself\id}\,\v{s}^t 
&=\rup{D^{out,t}-D^{in,t}}\v{1} \nonumber\\ 
& +\kself \,\bup{\v{s}^{t-1} + \v{s}^{t+1}} \, . 
\end{align}
This differs from \Cref{eqn:fullH} in that the right-hand side now includes both past and future ranks (which doubles the contribution of $\kself$ on the left). We remove the terms $\v{s}^{t-1}$ and $\v{s}^{t+1}$ for $t=0$ and $t=T$ respectively. This entire system has translational symmetry, since the energy Eq.~\eqref{eqn:fullH} remains the same if we add the same constant to all ranks at all times, but we can again break this symmetry by setting the mean rank to zero.

Additionally, in contrast to \Cref{eqn:fullsolution}, the ranks at $t$ now depend on both $t-1$ and $t+1$, which themselves depend on ranks at adjacent time-steps, so that ranks are affected by interactions in both the past and the future. In computer science, methods like this where the entire history is provided to the algorithm are called \emph{offline}, to distinguish them from \emph{online} approaches that update their results in real time as data becomes available. Thus we refer to this model as \nmdsrfull\ (\nmdsr).  

The cost of solving \Cref{eqn:fullsolution} for a single time-step is the same as static SpringRank with only one additional parameter to be tuned using cross-validation, and there are $T$ such $N$-dimensional equations to be solved successively. On the other hand, \Cref{eqn:h_total} requires solving a single  system of dimension $NT$, whose operator consists of $T$ blocks, each of dimension $N\times N$. While these two approaches feature numbers of non-zero entries that are fundamentally determined by the number of total edges across all time steps, the cost of solving \dsr vs \nmdsr will depend on the particular choice of linear solver~\cite{peng2021solving}.

Philosophically, Eqns.~\eqref{eqn:fullsolution} and~\eqref{eqn:h_total} are trying to do two different things. If we are given all the data $A^0,A^1,\ldots,A^T$ and we want to infer retrospectively how each individual's rank changed over time, it makes sense to include both past and future interactions as in~\eqref{eqn:h_total} so that $s_i^t$ is affected by $i$'s entire history. 

In contrast, \eqref{eqn:fullsolution} can be viewed as modeling each individual's perceived rank at the time, based only on the interactions that have occurred so far.

In principle, one could envisage other ways to formally incorporate an explicit dependence on  $\v{s}^{t-1}$ into the model, and we provide one example in SI \Cref{sec:sidynl}. However, we found that the approaches presented in this Section provide a natural interpretation, result in good prediction performance on both real and synthetic datasets (see \Cref{sec:results}) and are computationally scalable. 

We close this section with two possible extensions to these models. First, in some settings we might have timestamps $t$ that are not successive integers $0,1,\ldots,T$. In this case, if the time interval between two successive times is $\Delta t$, one could scale the spring constant of the self-springs between time-steps as $\kself/\Delta t$. This corresponds to the fact that if we have $\Delta$ identical springs in series, each of which is stretched by $(s^t-s^{t-1})/\Delta$, their total energy is $(1/2)(\kself/\Delta)(s^t-s^{t-1})^2$. The same expression applies if the timestamps are real-valued so that $\Delta$ is not an integer.

Second, if we believe that not just the ranks themselves but their rates of change behave smoothly over time, one could add a momentum term to the Hamiltonian which is quadratic in the discrete second derivative of the ranks. Since
\begin{gather*}
\left( (s^{t+1}-s^t) - (s^t-s^{t-1}) \right)^2
= \left( s^{t+1} - 2 s^t + s^{t-1} \right)^2 \\
= 2 (s^t-s^{t-1})^2 + 2 (s^{t+1}-s^t)^2 - (s^{t+1} - s^{t-1})^2 \, ,
\end{gather*}
this is equivalent to adding a repulsive force, i.e., a spring with negative spring constant, between ranks two time-steps apart. Note that the system nevertheless remains convex: this momentum term is positive semidefinite, so adding it to~\eqref{eqn:fullH} keeps the coupling matrix positive definite except for translational symmetry. Of course, these terms are second-order in time. In the online approach, one would have to determine $\v{s}^0$ from the static model, $\v{s}^1$ from the first-order model~\eqref{eqn:fullsolution}, and then use the model including this momentum term for $\v{s}^t$ for $t \ge 2$. We have not pursued this here, but it may make sense for certain datasets.


\subsection{Moving-window SpringRank}\label{subsec:mwsr}

Before we test the various versions of \dsrfull\ defined above, we consider a simpler model as a baseline. 
The simplest way to extend SpringRank to a dynamical context is to apply the static model to the interactions in a series of ``windows,'' where in each window we sum the interactions over a series of consecutive time-steps. For instance, we can compute $\v{s}^t$ for each $t$ by applying the static model to a window of width $\tau$, i.e., replacing $A^t$ with $\sum_{t'=t}^{t+\tau-1} A^{t'}$. Since these windows overlap, the resulting estimates $\v{s}^t$ will be smooth to some extent, even without imposing an explicit dependence between $\v{s}^t$ and $\v{s}^{t-1}$. We use this method, which we call \mwsrfull\ (\mwsr), as a baseline to compare with the dynamical models presented above.

Roughly speaking, a larger $\tau$ is like a larger self-spring constant $\kself$, since it induces more overlap between windows and thus a stronger correlation between the inferred ranks. However, like a decaying-history approach, \mwsr\ assumes a particular kernel for the importance of past time-steps: namely, that all $t'$ in the window are equally important. In contrast, \dsrfull\ infers the importance of past time-steps by coupling $\v{s}^t$ with $\v{s}^{t-1}$.

However, both models have a free parameter that needs to be tuned, i.e., $\kself$ and $\tau$. A shorter window $\tau$ or smaller spring constant $\kself$ allows the ranks to respond quickly to new interactions, while a longer window or larger spring constant more tightly couples nearby estimates. This trade-off suggests the existence of an optimal window length $\tau_{\opt}$. We tune $\tau$ using a cross-validation procedure as explained in SI \Cref{sisec:tuning}.


\subsection{Generative Model and Synthetic Data}
\label{sec:genmod}

Analogous to a model presented in~\cite{de2018physical}, we propose a probabilistic generative model for dynamic data. It takes as input the ranks $\v{s}^t$ and generates a sequence of weighted directed networks with adjacency matrix $A^t$ at time $t$. One can also imagine models that generate the ranks, for instance with a random walk with Gaussian steps whose log-probability is the self-spring Hamiltonian~\eqref{eqn:selfH}, but we treat $\v{s}^t$ as an input since we want the user of this model to have control over how the ground-truth ranks vary with time.  For instance, in our experiments below we generate synthetic data where the ranks vary sinusoidally.

The generative model has two real-valued parameters: a signal-to-noise ratio or inverse temperature $\beta$, and an overall density of edges $c$. Given the ranks $\v{s}^t$, it generates weighted, directed edges between each pair of nodes $i,j$ independently, as follows. The probability $P_{ij}^t(\beta)$ of $i$ ``beating'' $j$ at time $t$, giving a directed edge $i \to j$, is a logistic function as in~\cite{de2018physical} or the Bradley-Terry-Luce model~\cite{bradley1952,luce1959}:
\bea
\nonumber P_{ij}^t(\beta)=\frac{1}{1+\e^{-2\beta(s_i^t-s_j^t)}} \, .
\eea
The number of such edges, which gives the integer weight $A_{ij}^t$, is then drawn from a Poisson distribution whose mean $\lambda_{ij}^t$ is $cP^t_{ij}\,(\beta)$: 
\be
\label{generative_poiss}
A^t_{ij} \sim \Poi\left(\lambda_{ij}^t=\frac{c}{1+\e^{-2\beta(s_i^t-s_j^t)}}\right).
\ee
Since $P_{ij}^t(\beta) + P_{ji}^t(\beta)=1$, for any pair $i,j$ the total number of interactions $A_{ij}^t + A_{ji}^t$ is Poisson-distributed with mean $c$. The rank differences $s_i^t-s_j^t$ are used only to choose the directions of these edges. This  is equivalent to a model where we define a random multigraph where the number of edges between $i$ and $j$ is $\Poi(c)$, and then we choose the direction of each edge independently according to $P_{ij}^t$.

This is different from the generative model proposed in the static case in~\cite{de2018physical}. In that model the probability that $i$ and $j$ interact depends on $s_i-s_j$ so that nodes are more likely to interact if their ranks are fairly close. This is consistent with SpringRank's assumption that if $i$ beats $j$ then $j$ is below $i$, but not too far below it (since the springs have resting length $1$). This assumption makes sense for some datasets but not for others. By generating synthetic data without this dependence, our intent is to pose a greater challenge to SpringRank by modeling (for example) round-robin tournaments where every team plays each other.

\subsection{Model Evaluation}
\label{sec:testing}

Assessing a ranking model on real datasets is not straightforward since we do not know the true values of the underlying ranks. Nevertheless, we may measure the extent to which inferred ranks are accurate in the sense that they can predict the outcome of new observations. 

There are several performance metrics that can be used for prediction evaluation. From coarse-grained measures capable of predicting the likely winner to more fine-grained measures that also estimate odds, we consider four main metrics in our experiments, detailed in \Cref{sisec:evaluation}. We measure prediction performance using a cross-validation protocol where datasets are divided into training and test sets. The training set is used for hyperparameter tuning and parameter estimation while performance is evaluated on the test set. In order to preserve the chronological ordering of the data, the test set contains future observations, i.e., observations that chronologically follow those used in training. Hyperparameters for each method are tuned using grid-search in order to maximize the performance metrics as described in SI \Cref{sisec:tuning}.





%%% Local Variables:
%%% mode: latex
%%% TeX-master: "main"
%%% End:

\medskip

\section{Analysis and Main Results} \label{sect_signaling_game}



In this section, we present the main results of our analysis. First, we define our solution concept, $\Delta$-\textit{rationalizability}, and proceed to derive the sets of rationalizable type-strategy profiles for the leader and the followers. Then, we identify a necessary 
and sufficient condition under which the model exhibits unique rationalizable behavior. Finally, we discuss our results and the analysis's implications for efficiency.

\medskip


\subsection{Rationalizable Behavior}
Our solution concept is $\Delta$-rationalizability of \cite{battigalli_siniscalchi_2003}, which extends \citeapos{pearce_1984} notion of extensive-form rationalizability to games with incomplete information.
The ``$\Delta$'' in $\Delta$-rationalizability indicates a specific set of restrictions on beliefs that are required to be satisfied at each round of the iterative procedure. In our case, it is the signal structure commonly known to all players. We will show that in general, the set of action-type pairs that are $\Delta$-rationalizable constitute an interval both for the leader and the followers unless the followers' information is sufficiently noisy. This result hinges on the possibility of ``knowledge traps'': more precise information on the followers' part induces multiplicity of  $\Delta$-rationalizable type-strategy profiles which can lead to serious inefficiencies.


% \subsubsection*{Actions, Strategies and Beliefs}

Before providing a formal definition of the procedure, we introduce the following notation. Recall that $\Action_L=\{\invest,\notinvest\}$ is the action set of the leader. To simplify notation, we also consider it to be the set of possible (non-terminal) histories of the game. We, therefore, let $a_L \in \Action_L$ denote the action chosen by the leader and $h \in \Action_L$ the corresponding history.
A strategy for follower $j \in F$ is a mapping, $s_j: \Action_L \to \Action_j$, that maps history $h$ into action $s_j(h)$.
Let $S_j$ be the set of strategies for follower $j$. The sets of all possible types of the leader and follower $j$ are $\Theta$ and $X_j$, respectively. We call $(\theta_L , a_L )\in \Theta \times \Action_L$ a type-strategy pair for the leader. Likewise, $(x_j , s_j)\in X_j \times S_j$ is a type-strategy pair for follower $j$.
Players' interim beliefs are conditional probabilities, derived from the Bayes' rule, about the type-strategy pairs of their opponents. Specifically, an interim belief of leader $\theta$ is $\mu_L (\cdot \vl \theta) \in \Delta \left(X \times S \right)$, 
where $X \times S = \prod_{j\in F} X_j \times S_j$
with a generic element $(x, s) = (x_j, s_j)_{j \in F}$, 
and the interim belief for follower $j$ given type $x_j$ and history $h$ is $\mu_j (\cdot \vl x_j,h) \in \Delta \left(\Theta \times \prod_{k \neq j} (X_{k}\times S_{k})\right)$.

 
For leader $\theta$, exerting effort, $a_L = \invest$, is the best response with respect to a belief $\mu_L(x, s \vl \theta)$
if
\[
\int_{(x, s)} u(\theta, A_{-L}(s))\dd \mu_L(x, s \vl \theta) > 0,
\]
where $A_{-L}(s(\invest)) = \sum_{j \in F} \one\left(s_j(\invest) = \invest \right)$.\footnote{~We assume, without loss of generality, that players break the tie by choosing not to exert effort.}
Similarly, for type $x_j$ of follower $j$ under history $h$, action $s_j(h) = \invest$ is the best response to a belief $\mu_j(\theta, x_{-j}, s_{-j} \vl x_j, h)$ if
\[
\int_{(\theta, x_{-j}, s_{-j})} u(\theta, A_{-j}(h, s_{-j}))\dd \mu_j(\theta, x_{-j}, s_{-j} \vl x_j, h) > 0,
\]
where $A_{-j}(h, s_{-j}(h)) = \chi_\invest  + \sum_{k \neq j,~k \in F} \one\left(s_k(h) = \invest \right)$ and $\chi_\invest = \one\left(h = \invest \right)$.\footnote{~Likewise, $\chi_\notinvest = \one(h = \notinvest)$.} The notion of $\Delta$-rationalizability is defined as follows.


\begin{definition}[$\Delta$-rationalizability]
Consider the following procedure.\\
(Round 0) Let $R_L^0 = \Theta \times \Action_L$ and $R_{F, \,j}^0 = X_j \times S_j$ for each $j \in F$. \\
(Round $k \geq 1$) Let $R_F^m = \prod_{j \in F} R_{F, \,j}^m$ and $R_{F, -j}^m = \prod_{\ell \neq j} R_{F, \ell}^m, m \in \{0\} \cup \NN$. Then
\begin{enumerate}
    \item[(i)] $(\theta, a_L) \in R_L^k$ if and only if $(\theta, a_L) \in R_L^{k-1}$ and there exists a belief $\mu_L(\cdot \vl \theta) \in \Delta(R_F^0)$ such that $\mu_L(R_F^{k-1} \vl \theta) = 1$ and $a_L$ is a best response with respect to $\mu_L(\cdot \vl \theta)$.
    
    \item[(ii)] For every follower $j \in F$, $(x_j, s_j) \in R_j^{k-1}$ if and only if $(x_j, s_j) \in R_j^{k-1}$ and for each history $h$ there exists a belief $\mu_j(\cdot \vl x_j, h) \in \Delta(R_L^0 \times R_{F, \,-j}^0)$ such that $\mu_j(R_L^k \times R_{F, \,-j}^{k-1} \vl x_j, h) = 1$ and $s_j(h)$ is a best response with respect to $\mu_j(\cdot \vl x_j, h)$.
\end{enumerate}
Finally, let $R_L^\infty = \bigcap_{k=0}^\infty R_L^k$ and $R_{F, \,j}^\infty = \bigcap_{k=0}^\infty R_{F, \,j}^k$. Then an action $a_L$ is $\Delta$-rationalizable for type $\theta$ of the leader if $(\theta, a_L) \in R_L^\infty$. Analogously, a strategy $s_j$ is $\Delta$-rationalizable for type $x_j$ of follower $j$ if $(x_j, s_j) \in R_{F, \,j}^\infty$.
\end{definition}


\subsubsection*{Follower Problem}
Consider type $x$ of follower $j \in F$. 
Suppose that he believes that the leader uses a monotone strategy with threshold $z \in \RR$; that is, $a_L = \invest$ for all $\theta > z$.
Therefore, type $x$'s interim belief about $\theta$ has a truncated Gaussian distribution with density
\begin{equation} \label{interim_belief}
    \psi^h(\theta; x, z) = \begin{cases}
    \frac{\frac{1}{\sigma_F}\phi\left(\frac{\theta - x}{\sigma_F}\right)}{1 - \Phi\left(\frac{z - x}{\sigma_F}\right)}\one(\theta > z) & \text{if $h = \invest$} \\
    ~ & ~ \\
    \frac{\frac{1}{\sigma_F}\phi\left(\frac{\theta - x}{\sigma_F}\right)}{\Phi\left(\frac{z - x}{\sigma_F}\right)}\one(\theta \leq z) & \text{if $h = \notinvest$}
    \end{cases},
\end{equation}
where $\phi(\cdot)$ and $\Phi(\cdot)$ denote the standard Gaussian density function (PDF) and cumulative distribution function (CDF), respectively.
Let $\Psi^h(\cdot; \,x, z)$ be the corresponding CDF under history $h$. 
We denote by $\lambda(x) = \phi(x)/\Phi(x)$ the \textit{reversed hazard rate}. Then 
type $x$'s expectation of $\theta$ can be written as
\begin{equation} \label{eq_expectation_of_theta}
    \EE_{\theta \sim \Psi^h(\cdot; \,x, z)}[\theta] = \begin{cases}
   x + \sigma_F \lambda \left(\frac{x - z}{\sigma_F}\right)  & \text{if $h = \invest$} \\
   ~ & ~ \\
   x - \sigma_F \lambda \left( \frac{z - x}{\sigma_F} \right) & \text{if $h = \notinvest$}
   \end{cases},
\end{equation}
which has the following properties.

\begin{lemma} \label{lemma_truncated_expectation}
The interim expectations $\EE_{\theta \sim \Psi^h(\cdot; \,x, z)}[\theta]$ are strictly increasing in $x$ and $z$. Moreover,
\begin{equation*}
    \lim_{x \to -\infty} \EE_{\theta \sim \Psi^h(\cdot; \,x, z)}[\theta] = \begin{cases}
    z & \text{if $h = \invest$} \\
    - \infty & \text{if $h = \notinvest$}
    \end{cases},
\end{equation*}
and
\begin{equation*}
    \lim_{x \to \infty} \EE_{\theta \sim \Psi^h(\cdot ; \,x, z)}[\theta] = \begin{cases}
    \infty & \text{if $h = \invest$} \\
    z & \text{if $h = \notinvest$}
    \end{cases}.
\end{equation*}
\end{lemma}

Now suppose further that follower $j$ believes that other followers are using monotone strategies with threshold $x_h$ under history $h$; that is, for any 
follower $\ell \neq j$ with type $x$, 
$s_\ell(h) = \invest$ if and only if $x > x_h$. This implies that, at a given state $\theta$, the probability that follower $j$ assigns to
$k$ other followers investing equals
$\left[1 - \Phi\left((x_h - \theta)/\sigma_F\right)\right]^k$, $k \in \{0, 1, \dots, n-1\}$. Therefore
follower $j$'s expected proportion of other players investing at state $\theta$ is
\begin{align*}
    A_{-j}(\theta) & = \sum_{k=0}^{n-1}  \binom{n-1}{k} \frac{k }{n} \left[1 - \Phi\left(\frac{x_h - \theta}{\sigma_F}\right)\right]^k \Phi\left(\frac{x_h - \theta}{\sigma_F}\right)^{n-1-k} +\frac{\chi_{\invest}}{n}\\
    & = \frac{n-1}{n}\left[\left(1 - \Phi\left(\frac{x_h - \theta}{\sigma_F}\right)\right) \right] + \frac{\chi_\invest}{n}.
\end{align*}
The second equality follows from
the binomial identity $\sum_{k=0}^{n-1} \binom{n-1}{k} k (1-q)^k q^{n-1-k} = (n-1)(1-q)$. Since the leader's action is observable, follower $j$ has certainty about receiving the network benefit $\chi_\invest/n$.
Thus, we may write the payoff to choosing $a_j =\invest$ for type $x$, under history $h$, as
\begin{equation} \label{follower_payoff}
   \pi_F^h(x; z, x_h) = \EE_{\theta \sim \Psi^h(\cdot; \,x, z)} \left[ \theta - \frac{n-1}{n} \Phi\left(\frac{x_h -\theta}{\sigma_F}\right)   \right] - \frac{\chi_\notinvest}{n}.
\end{equation}
We then have the next lemma.

\begin{lemma} \label{lemma_x_payoff}
The follower payoffs $\pi_F^h(x; z, x_h)$ are strictly increasing in $x$ and $z$, and are strictly decreasing in $x_h$. Moreover,
\begin{equation*}
    \lim_{x \to -\infty} \pi_F^h(x; z, x_h) = \begin{cases}
z - \frac{n-1}{n}\Phi\left( \frac{x_\invest - z}{\sigma_F} \right) & \text{if $h = \invest$} \\
-\infty & \text{if $h = \notinvest$}
\end{cases}
\end{equation*}
and
\begin{equation*}
    \lim_{x \to \infty} \pi_F^h(x; z, x_h) = \begin{cases}
\infty & \text{if $h = \invest$} \\
z - \frac{1}{n} + \frac{n-1}{n}\Phi\left( \frac{x_\notinvest - z}{\sigma_F} \right) & \text{if $h = \notinvest$}
\end{cases}.
\end{equation*}
\end{lemma}

\medskip
\subsubsection*{Leader Problem}
Suppose that, under history $h = \invest$, followers use monotone strategies with threshold $x_\invest \in \RR$; that is, $s_j(\invest) = \invest$ for $x_j > x_\invest$.\footnote{~Since followers are \emph{ex ante} identical, assuming a common threshold is without loss.} If the leader chooses $a_L=\invest$, then the expected aggregate action is given by
\begin{equation*}
    A_{-L}(\theta) = 1 - \Phi\left(\frac{x_\invest - \theta}{\sigma_F}\right).
\end{equation*}
Note that the behavior of followers matters to the leader only when $a_L = \invest$; otherwise, she obtains a payoff of zero by taking the safe action $a_L = \notinvest$. The payoff to choosing $a_L=\invest$ for type $\theta$ is therefore
\begin{equation} \label{leader_payoff}
    \pi_L (\theta; x_\invest)= \theta - \Phi\left(\frac{x_\invest - \theta}{\sigma_F}\right).
\end{equation}
It is immediate to see that $\pi_L(\theta; x_\invest)$ is strictly increasing in $\theta$ and crosses zero only once from below. Thus, the leader's best response to $x_\invest$ is the unique solution to $\pi_L(\theta; x_\invest) = 0$.



\subsection{Main Results} \label{main_results}


We first provide an intuitive explanation of how $\Delta$-rationalizability proceeds. Before the procedure starts, all players deem all type-strategy pairs possible. Let $\thetalow^0 = \xhlow^ 0 = -\infty$ and $\thetaup^0 = \xhup^0 = \infty$ for each history $h$. We call the former \textit{lower dominance bounds} and the latter \textit{upper dominance bounds}.
Note that the payoff to the leader, given $\xilow^0$ and $\xiup^0$, satisfies the standard two-sided ``limit dominance'' property of global games \citep{morris_shin_2003}, with the \textit{dominance regions} being $(-\infty, 0)$ and $(1, \infty)$. That is, exerting no effort ($a_L = \notinvest$) is dominant for all types $\theta < 0$, and exerting effort ($a_L = \invest$) is dominant for all types $\theta > 1$.
This implies that
the leader will eliminate, in Round 1, all type-action pairs $(\theta, \invest)$ with $\theta < \thetalow^1 = 0$ and $(\theta, \notinvest)$ with $\theta > \thetaup^1 = 1$. 

By knowing the leader's dominance bounds $\thetalow^1$ and $\thetaup^1$, each follower can infer from $h = \invest$ that this decision cannot be made by a type $\theta < \thetalow^1$. 
This, in turn, 
determines each follower's dominance regions. For type $x$ of a follower, Lemma \ref{lemma_x_payoff} implies that
the worst-case payoff equals 
\begin{equation*}
   \pi_F^\invest(x; \thetalow^1, \xiup^0) =   \EE_{\theta \sim \Psi^\invest(\cdot; \,x, \thetalow^1)} [\theta] - \frac{n-1}{n}.
\end{equation*}
Let $\xiup^1$ be the unique solution to $\EE_{\theta \sim \Psi^\invest(\cdot; \,\xiup^1, 0)} [\theta] = \frac{n-1}{n}$. Exerting no effort is never a best response for $x > \xiup^1$ because the worst-case payoff is strictly increasing in $x$. But since the best-case payoff is positive for all $x$:
\begin{equation*}
   \pi_F^\invest(x; \thetaup^1, \xilow ^0) = \EE_{\theta \sim \Psi^\invest(\cdot; \,x, \thetaup^1)}[\theta] > 0,
\end{equation*}
the subgame under $h = \invest$ violates 
the two-sided limit dominance property because
exerting effort is not strictly dominated for any type $x$.\footnote{~See \cite{baliga_sjostrom_2004} and \cite{bueno_de_mesquita_2010} for applications with one-sided limit dominance but different signal structures.} 
This implies that the lower dominance bound yields $\xilow^1 = -\infty$. 

Now, under history $h = \notinvest$, followers know that it must be leader $\theta \leq \thetaup^1$ that has chosen not to exert effort. The subgame exhibits no upper dominance region because the worst-case payoff to any follower type $x$
\begin{equation*}
    \pi_F^\notinvest(x; \thetalow^1, \xiup ^0) = \EE_{\theta \sim \Psi^\notinvest(\cdot; \,x, \thetalow^1)}[\theta] - 1 < 0
\end{equation*}
is negative. Thus, $\xnup^1 = \infty$. The lower dominance bound is given by the unique solution $\xnlow^1$ to 
\begin{equation*}
    \pi_F^\notinvest(\xnlow^1; \thetaup^1, \xnlow^0) = \EE_{\theta \sim \Psi^\notinvest(\cdot; \,\xnlow^1, \thetaup^1)}[\theta] - \frac{1}{n} = 0. 
\end{equation*}
In sum, each follower $j$ will delete type-strategy pairs $(x, s_j)$ such that (i) $x > \xiup^1$ and $s_j(\invest) = \notinvest$, and (ii) $x < \xnlow^1$ and $s_j(\notinvest) = \invest$.

In Round 2, $\thetalow^2$, $\xilow^2$, and $\xnup^2$ are given analogously.  The leader's upper dominance bound, $\thetaup^2$, is the unique solution to 
\begin{equation*}
    \pi_L(\thetaup^2; \xiup^1) = \thetaup^2 - \Phi\bigg( \frac{ \xiup^1 - \thetaup^2 }{\sigma_F} \bigg) = 0.
\end{equation*}
Moreover, Lemma \ref{lemma_x_payoff} implies that followers' upper dominance bound under history $h = \invest$ is the unique value of $\xiup^2$ that solves 
\begin{equation*}
    \pi_F^\invest(\xiup^2; \thetalow^2, \xiup^1) = \EE_{\theta \sim \Psi^\invest(\cdot; \, \xiup^2, \thetalow^2)} \left[\theta - \frac{n-1}{n}\Phi\left(\frac{\xiup^1 - \xiup^2}{\sigma_F}\right)\right ] = 0,
\end{equation*}
and the lower dominance bound under history $h = \notinvest$ is given by the unique solution $\xnlow^2$ to
\begin{equation*}
        \pi_F^\notinvest(\xnlow^2; \thetaup^2, \xnlow^1) = \EE_{\theta \sim \Psi^\invest(\cdot; \, \xnlow^2, \thetaup^2)} \left[\theta - \frac{n-1}{n}\Phi\left(\frac{\xnlow^1 - \xnlow^2}{\sigma_F}\right)\right ] - \frac{1}{n} = 0.
\end{equation*}

A similar argument goes for all Rounds $k > 2$.
The iteration procedure ultimately yields six sequences. We summarize their properties in the following lemma.


\begin{lemma} \label{lemma_rat_seq}
The sequences are such that:\\
(a) $(\thetalow^k)_{k=0}^\infty$ is such that $\thetalow^k = \thetalow = 0$ for all $k \geq 1$; \\
(b) $(\thetaup^k)_{k=0}^\infty$ is strictly decreasing and bounded below; \\
(c) $(\xilow^k)_{k=0}^\infty$ is such that $\xilow^k = \xilow = -\infty$ for all $k \geq 0$; \\
(d) $(\xiup^k)_{k=0}^\infty$ is strictly decreasing; \\
(e) $(\xnlow^k)_{k=0}^\infty$ is strictly increasing; \\
(f) $(\xnup^k)_{k=0}^\infty$ is such that $\xnup^k = \xnup = \infty$ for all $k \geq 0$.
\end{lemma}

By the monotone convergence theorem, $\thetaup^k$ converges to $\thetaup$ as $k \to \infty$. Moreover, $\thetaup$ is the unique solution to
\begin{equation} \label{leader_upper}
    \pi_L(\thetaup; \xiup) = 0,
\end{equation}
where $\xiup = \lim_{k \to \infty} \xiup^k$, and hence $\thetaup < 1$.
If $\xiup > - \infty$, it solves 
\begin{equation} \label{follower_invest_upper}
    \pi_F^\invest(\xiup; \thetalow, \xiup) = 0;
\end{equation}
otherwise $\xiup = -\infty$. Similarly, let $\xnlow = \lim_{k \to \infty}\xnlow^k$, and $\xnlow$ solves
\begin{equation} \label{follower_notinvest_lower}
        \pi_F^\notinvest(\xnlow; \thetaup, \xnlow) = 0
\end{equation}
if a solution exists. Otherwise $\xnlow^k$ diverges to $\xnlow = \infty$.
We now state the main result of the paper.

\begin{proposition} \label{prop_unique_rat}
The $\Delta$-rationalizable sets are $R_L^\infty = R_L^0 \setminus \overline{R}_L^\infty$ and $R_{F, \,j}^\infty = R_{F, \,j}^0 \setminus \overline{R}_{F, \,j}^\infty$, where
\begin{equation*}
     \overline{R}_L^{\infty} = \left\{(\theta, a_L) \vl~ a_L = \invest ~\text{if}~ \theta \leq 0 ~\text{and}~ a_L = \notinvest ~\text{if}~ \theta > \thetaup \right\}
\end{equation*}
and
\begin{equation*}
    \overline{R}_{F, \,j}^\infty = \left\{(x_j, s_j) \vl ~ s_j(\invest) = \notinvest ~\text{if}~x_j > \xiup ~\text{and}~s_j(\notinvest) = \invest ~\text{if}~ x < \xnlow \right\}.
\end{equation*}
Moreover, there exists a unique $\widehat{\sigma}_F$ such that there is a unique $\Delta$-rationalizable strategy profile with 
$(\thetaup, \xiup, \xnlow) = (0, -\infty, \infty)$ if and only if $\sigma_F > \widehat{\sigma}_F$.

\end{proposition}







In words, Proposition \ref{prop_unique_rat} conveys the message that if the leader has a sufficient informational advantage (i.e.,
$\sigma_F > \widehat{\sigma}_F$), then  
the unique $\Delta$-rationalizable strategy profile features leader type $\theta$ choosing $a_L = \invest$ when $\theta > 0$ and $a_L = \notinvest$ otherwise, and all follower types imitating the leader's action. This leads to a fully efficient outcome. However, 
when followers have relatively precise information (i.e.,
$\sigma_F \leq \widehat{\sigma}_F$), both actions become rationalizable for leader types $\theta \in (0,\thetaup]$ and for follower types in $(-\infty,\xiup]$ if the leader chooses to exert effort and in $(\xnlow, \infty)$ if the leader chooses to exert no effort. Thus, the leader does not necessarily choose the efficient action ($a_L = \invest$) when $\theta \in (0, \thetaup]$ for fear that followers might coordinate against her and choose the inefficient action.




Moreover, whenever we obtain unique rationalizable behavior under history $h=\invest$, we also do so under history $h=\notinvest$. This is because followers understand that the state is negative which implies that $\pi_F^{\notinvest} (\xnlow;\thetaup ,\xnlow)=0$ has no solution. If, however, we get multiplicity of rationalizable profiles under history $h=\invest$, we may or may not get multiplicity under history $h=\notinvest$. This will depend on whether $\pi_F^{\notinvest} (\xnlow;\thetaup ,\xnlow)=0$ has solutions or not for the particular value of $\sigma_F < \widehat{\sigma}_F$ considered. It should also be noted that when the necessary and sufficient condition is not satisfied, then the values of $(\thetaup, \xiup, \xnlow)$ depend on the value of $\sigma_F$ and, thus, the game features noise-dependent selection. We plot the values of $\thetaup$ in Figure 2. Figure \ref{fig_n_vs_sigma_f_hat} illustrates how the value of $\widehat{\sigma}_F$ changes with the number of followers. 



% Figure environment removed


The next result shows what happens in the limit as $\sigma_F$ tends to zero.  
\begin{proposition} \label{prop_limit}
    In the limit as $\sigma_F \to 0$, the dominance bounds $\thetaup \to (n-1)/(2n)$, $\xiup \to (n-1)/(2n)$, and $\xnlow \to \infty$.
\end{proposition}

It is worth noting that the leader's and followers' upper threshold is given by $(n-1)/2n$ which corresponds to the ``risk dominant'' strategy profile of the subgame given the spillover benefit of the leader's action. This is because, in this extreme, the followers are allowed to have beliefs that completely shut down the informational role of the leader. In particular, this would be the unique rationalizable behavior of an alternative game, with the same payoffs as in this subgame, where there is no leader and the beliefs about the state are given by the Bayesian updating of the prior after followers receive their signals. This will be better illustrated in the next section where we interpret our results in terms of "conditional rank beliefs". 


% Figure environment removed

\subsection{The Signaling Role of Leader and Information Traps}

It is now evident that when the condition $\sigma_F > \widehat{\sigma}_F$ is met, we obtain \emph{efficient leadership}: the leader chooses action $a_L = \invest$ whenever it socially desirable to do so (i.e., $\theta > 0$) and chooses $a_L = \notinvest$ otherwise. In this case, the outcome of the game corresponds to the fully efficient subgame perfect equilibrium of the complete information game, which may not come as a surprise.
Indeed, \cite{komai_et_al_2007} reach a similar conclusion in a different framework, which focuses on the signaling role of leaders primarily in organizational economics settings. However, our global games perspective speaks to leadership in a variety of scenarios, such as regime change, bank runs, and currency attacks. Additionally, our result reinforces theirs by deriving it using a weaker solution concept.
% the same conclusion, albeit in a different framework. While their framework captures the signaling role of leaders mostly in the organizational economics settings, our global games perspective speaks about leadership in other types of scenarios as well, such as regime change, bank runs, or currency attacks. Moreover, our result strengthens theirs in that we derive it using a weaker solution concept.

However, the significance of our contribution lies in the converse direction, demonstrating that if followers possess sufficiently precise private information, the leader may choose a socially undesirable action. The leader may fear that her well-informed followers might make the ``wrong'' decision, which could, in turn, compel her to act inefficiently by selecting $a_L = \notinvest$ even when $\theta > 0$.
% However, we believe that the importance of our contribution comes from the converse direction: if the followers have sufficiently precise private information, then the leader herself might choose the socially undesirable action. This is because the leader might be afraid that her followers will choose the ``wrong'' action when they are well informed. 
% This, in turn, might lead the leader herself to act inefficiently; that is, she might choose $a_L=\notinvest$ even if she knows that $\theta > 0$. 
Furthermore, even if $\theta < 0 $ and the leader chooses $\notinvest$, well-informed followers might be tempted to choose $\invest$, which is socially undesirable in this scenario because they do not know what type of leader chose action $\notinvest$. Therefore, we may encounter an \textit{information trap}: better-informed followers might make the leader choose incorrectly, or even if the leader does choose the efficient action, they, themselves, might not do so. On the other hand, if these followers were ``kept in the dark,'' efficient coordination would be achieved as the unique rationalizable outcome of the game. This is surprising in the sense that, we do not obtain ``limit uniqueness'' but rather ``limit multiplicity'' of rationalizable profiles, contrary to the standard results in the global games literature (see, for example, \cite{frankel_et_al_2003}). 


Therefore, it is clear that the role of the leader is undermined by the more precise information the followers hold. Our model uncovers two opposite forces that compete with each other: we term them the signaling effect and the miscoordination effect. The necessary and sufficient condition derived in Proposition \ref{prop_unique_rat} makes certain that the signaling effect dominates the miscoordination effect and, as a result, ensures that the efficient outcome is realized. We now proceed to shed more light on these two forces and make the tension uncovered clearer. 


\subsubsection{Signaling Effect versus Miscoordination Effect: Why Multiplicity Happens?}

To see why multiplicity presents itself, we will analyze the subgame after history $h$ and consider the rationalizable profiles of followers' type-strategy pairs. Notice that given the action of the leader, the subgame appears to be a global game except we have one-sided dominance regions. To make the intuition clearer it is useful to consider \textit{monotone strategies}. In that regard, consider type $x$ of follower $j$.  This type of follower $j$ does not know which threshold the leader used to make the choice that led to history $h$ being realized, so let $z$ denote this threshold.  Now, define follower $j$'s \textit{conditional rank belief} as the probability he assigns to the event that the other follower's type $x_k$ is at most his own ($x_j = x$) conditional on history $h$. We have that:

\begin{equation} \label{rank_belief}
    R^h(x; z)= \prob(x_k \leq x_j \vl x_j = x, h) = \frac{1}{2}\left[\Phi\left(\frac{x - z}{\sigma_F}\right) + \chi_\notinvest \right] 
\end{equation}
This definition is a direct extension of the rank belief function introduced in \cite{morris_et_al_2016} and \cite{morris_yildiz_2019}. 

Assume that follower $j$ conjectures that his opponents in the subgame will use a threshold $x_h$. Then, the expected payoff to choosing $\invest$ under history $h$ is given by:
\begin{equation} \label{follower_payoff_rank_belief}
    \pi_F^h(x; z, x_h) = \EE_{\theta \sim \Psi^h(\cdot; \,x, z)} \left[ \theta - \frac{n-1}{n} \Phi\left(\frac{x_h-\theta}{\sigma_F}\right)   \right] - \frac{\chi_\notinvest}{n}.
\end{equation}
Now, consider the type of follower $j$ whose signal is equal to the conjectured threshold of followers $-j$. Then, we can write this type's expected payoff to choosing $\invest$ as 
\begin{equation*}
    \pi_F^h(x_h; z, x_h) = \EE_{\theta \sim \Psi^h(x_h; \,z, x_h)} \left[ \theta \right] + \frac{n-1}{n}\left[1 - R^h(x_h; z)\right] + \frac{\chi_\invest}{n} - 1.
\end{equation*}

In order for type $x_h$ of follower $j$ to be indifferent between choosing $\invest$ and $\notinvest$ it must be the case that $x_h$ must solve for all $h$
\begin{equation} \label{eq_eqm_follower}
   \EE_{\underbrace{\theta \sim \Psi^h(\cdot; \,z, x_h)}_{\text{signaling effect}}} \left[ \theta \right] -  \underbrace{\frac{\chi_\notinvest}{n}}_{\text{spillover from leader's action}}= \underbrace{\frac{n-1}{n}R^h(x_h; z)}_{\text{miscoordination effect}} 
\end{equation}
The left-hand side of this equation captures the twofold effect of the leader's choice. First, the signaling effect simply says that by observing history $\invest$, it must be the case that $\theta>z$. The rationality of the leader implies that $z\geq 0$ and this is common knowledge among followers. Second, if the leader chose $\notinvest$ there is a negative spillover benefit to followers. The right-hand side captures the miscoordination effect: given the leader's threshold, if a follower $-j$ invests only if his type is greater than $x_h$, then type $x_h$ of follower $j$ faces an expected loss equal to (1/n) times the probability of this event, which is given exactly by the conditional rank belief function. 
Notice that if type $x_h$ of follower $j$ is indifferent between $\invest$ and $\notinvest$, then any type lower than $x_h$ will choose $\notinvest$ given the conjectured strategies. 



Focus on history $h=\invest$. An analogous argument holds for history $h=\notinvest$. Notice that we can write Equation (\ref{eq_eqm_follower}) under $h = \invest$ as 
\begin{equation} \label{eq_eqm_follower_rewrite}
    x_\invest + \sigma_F \lambda\left(\frac{x_\invest - z}{\sigma_F}\right) = \frac{n-1}{n}\left[\frac{1}{2}\Phi\left(\frac{x_\invest - z}{\sigma_F}\right)\right]
\end{equation}
 Since the leader choosing action $\invest$ whenever $\theta>0$ and followers imitating the leader's action is always rationalizable behavior, to get uniqueness,
Equation (\ref{eq_eqm_follower_rewrite}) must either have no solution or have exactly one solution for $z=\thetalow=0$. In the former case we obtain the efficient outcome, while in the latter, we obtain uniqueness but full efficiency is not achieved.
Observe that
the noise, $\sigma_F$, affects both the signaling and the miscoordination effect. Moreover, the former is increasing in $\sigma_F$. In contrast, the latter does not change monotonically with $\sigma_F$ but features rapid slope change around $x_h=z$.\footnote{~It changes monotonically for $x_h<z$ and $x_h>z$ but not for all $x$.} For large values of $\sigma_F$  (i.e., $\sigma_F \geq \widehat{\sigma}_F$), the signaling effect dominates the miss-coordination effect, implying that Equation (\ref{eq_eqm_follower_rewrite}) has no solution, that is, the expected loss is always smaller than the expected return to choosing $\invest$. This means that $\xiup^k \to \xilow$ as $k \to \infty$ and hence $\invest$ is the unique rationalizable action in the subgame following $h = \invest$. See Figure \ref{fig_invest_unique} for an illustration.
As a consequence, $\thetalow = \thetaup$, $\xnlow = \xnup$ and we obtain a unique $\Delta$-rationalizable strategy profile.

% Figure environment removed

On the contrary, for small values of $\sigma_F$ (i.e., $\sigma_F \leq \widehat{\sigma}_F$), the signaling effect stops playing a dominant role, meaning that the expected loss may be higher than the expected return. Hence Equation (\ref{eq_eqm_follower_rewrite}) may have multiple solutions (see Figure \ref{fig_invest_multiple}). Indeed, this is the case. In particular, the largest solution corresponds to the limit $\xiup$ to which the sequence $(\xiup^k)$ will converge. This yields the fact that both actions are rationalizable for follower types $(\xilow,\xiup]$ with $\xilow=-\infty$.


At this point, the fact that $\theta>0$ has become common knowledge plays a minimal role. In fact, as $\sigma_F$ approaches zero, the signaling effect is completely obliterated and followers behave as if $\theta$ can take any value on the real line\footnote{~In fact, the monotone strategy profiles with thresholds $\xilow$ and $\xiup$ are the least and greatest Bayesian Nash equilibria that bound all rationalizable strategies in the sub-game that follows action $\invest$ of the leader \citep{van_zandt_vives_2007}.}. An interesting aspect of the model is that 
$R^\invest(\xiup, \thetalow) \to 1/2$ as $\sigma_F \to 0^+$. This guarantees that the monotone strategy profile with threshold $\xiup$ which bounds the set of rationalizable profiles in the subgame, corresponds to the ``risk dominant'' equilibrium \citep{harsanyi_selten_1988} of the subgame game given history $\invest$. That is, a follower will only choose $\invest$ if $\invest$ is a best response to a uniform belief over other followers choosing each action. However, this point cannot be supported as an equilibrium in monotone strategies of the whole game. When followers play according to the threshold $\xiup$, the leader will best respond by using a monotone strategy with threshold $\thetaup > \thetalow$, which is the limit of $\thetaup^k$ as $k \to \infty$. Thus, any leader type $\theta \in (\thetalow, \thetaup)$ will find both actions rationalizable. 

% Figure environment removed





\subsection{Discussion}

\subsubsection{Equilibrium Behavior}

If the necessary and sufficient condition derived in Proposition \ref{prop_unique_rat} is satisfied, the game features unique rationalizable behavior. This immediately implies that the game features unique equilibrium behavior. In particular, the unique rationalizable strategy profile corresponds to the unique Perfect Bayesian Equilibrium of the game, which is in monotone strategies, with thresholds for the leader and followers given by $\theta_L^*=\thetaup=0 $, $\xilow=\xiup=-\infty$ and $\xnlow=\xnup=\infty$. 

As we prove in Appendix B, it is the case that when one considers monotone strategies only, the game always has a unique monotone equilibrium. However, restricting attention to these types of strategies only may be with loss, since one cannot rule out the existence of other equilibria in more complicated, non-monotone strategies. This is one of the reasons that we chose rationalizability as our solution concept.










\subsubsection{Peer-confirming Equilibrium}

Although our model features incomplete information about a fundamental state, the result is consistent with the prediction delivered by the peer-confirming equilibrium of \cite{lipnowski_sadler_2019}. In a leader-centered star network where all followers observe the leader's strategy, \cite{lipnowski_sadler_2019} argue that ``imitation'' may arise as the unique extensive-form peer-confirming equilibrium since followers do not observe any information that could contradict the leader's rationality. 
The leader, thus, has the advantage of inducing others to imitate her behavior.
In this regard, Proposition 1 deals with the interplay between strategic uncertainty and fundamental uncertainty by
identifying a necessary and sufficient condition under which the unique $\Delta$-rationalizable strategy profile that arises also corresponds to the unique peer-confirming equilibrium that highlights imitation.









\medskip








\medskip

\section{Extension: Leader is not Perfectly Informed}
\subsection{Alternative Information Structure} \label{sect_noisy_signaling}


Suppose, now, that instead of perfectly learning the state $\theta$, the leader observes a noisy signal $x_L = \theta + \sigma_L \varepsilon_L$ with $\varepsilon_L$ being a standard Gaussian noise independent of $\theta$ and $\varepsilon_j$ for any $j \in F$. We deem this extension important for three reasons. First, from an applied perspective, it is more natural to assume that the leader observes a noisy signal about the state rather than its true value. Second, we document that multiplicity of rationalizable behavior is not a consequence of the one-sided dominance in the subgames. That is, the reason for multiplicity is not that the action of the leader when she is perfectly informed introduces ``too much'' common knowledge. Rather, the main result of the paper continues to hold with the (noisy) information structure that brings back the standard two-sided dominance. Finally, this extension will show that while it must be the case that the leader is fully informed to obtain efficiency, it is not enough for the leader to be better informed than the followers to obtain unique rationalizable behavior. The requirement that the noise in leader's information is smaller than the one in the followers' (i.e., $\sigma_L < \sigma_F$), is neither necessary nor sufficient for uniqueness.  

\subsection{Analysis and Rationalizable Behavior}


Call $x_L \in X_L = \RR$ the leader's type. A strategy for the leader is now a mapping $s_L: X_L \to \Action_L$. Let $S_L$ denote the strategy space for the leader. The actions and strategies of the followers remain the same as defined in Section \ref{sect_signaling_game}.
Let $\mu_L(\cdot\vl x_L) \in \Delta \left(\Theta \times X \times S \right)$ be leader $x_L$'s belief about the state and the type-strategy pairs of the followers, where $X \times S = \prod_{j\in F} (X_j \times S_j)$. Since there is no learning for the leader, the marginal of $\mu_L(\cdot \vl x_L)$ about $\theta$ has a Gaussian distribution with mean $x_L$ and variance $\sigma_L^2$.
Let $\mu_j(\cdot \vl x_j, h) \in \Delta \left(\Theta \times X_L \times X_{-j} \times S_{-j} \right)$ be follower $j$'s belief about 
the state and the type-strategy pairs of his opponents given type $x_j$ and history $h$, where $X_{-j} \times S_{-j} = \prod_{k \neq j \in F} (X_k \times S_k)$.


For type $x_L$ of the leader, exerting effort is the best response to a belief $\mu_L(\cdot \vl x_L)$ if
\begin{equation*}
    \int_{(\theta, x, s)} u(\theta, A_{-L}(s)) \dd \mu_L(\theta, x, s \vl x_L) > 0,
\end{equation*}
where $A_{-L}(s(\invest)) = \sum_{j \in F} \one(s_j(\invest) = \invest)$.
For a follower with type $x_j$, $s_j(h) = \invest$ is the best response to $\mu_j(\cdot \vl x_j, h)$ when
\begin{equation*}
    \int_{(\theta, x_L, x_{-j}, s_{-j})} u\left(\theta, A_{-j}(h, s_{-j}) \right) \dd \mu(\theta, x_L, x_{-j}, s_{-j} \vl x_j, h) > 0
\end{equation*}
where $A_{-j}(h, s_{-j}(h)) = \chi_\invest + \sum_{k \in F, \,k \neq j} \one(s_k(h) = \invest)$. The initial set of type-strategy pairs for the leader in the definition of $\Delta$-rationalizability is now given by $R_L^0 =X_L \times S_L$.


Suppose that a follower type $x$ believes that the leader uses a monotone strategy with threshold $z$; i.e., $a_L = \invest$ if and only if $x_L > z$. Then, 
upon observing $h = \invest$, type $x$'s interim belief has CDF
\begin{equation} \label{posterior_invest}
    G^\invest(\theta; \,x, z) = \frac{1}{ \Phi\left( \frac{x - z}{\sigma}\right) } \int_{-\infty}^{\theta} \frac{1}{\sigma_F} \phi\left(\frac{t- x}{\sigma_F}\right) \Phi\left(\frac{t - z}{\sigma_L}\right)\dd t,
\end{equation}
where $\sigma^2 = \sigma_F^2 + \sigma_L^2$.
Similarly, type $x$'s interim CDF under history $h = \notinvest$ is
\begin{equation} \label{posterior_not_invest}
    G^\notinvest(\theta; \,x, z) = \frac{1}{\Phi \left(\frac{z - x}{\sigma}\right) } \int_{-\infty}^{\theta}
    \frac{1}{\sigma_F} \phi\left(\frac{t-x}{\sigma_F}\right) \Phi\left(\frac{z - t}{\sigma_L}\right)\dd t.
\end{equation}
It is worth noting that $G^h(\cdot; \,x, z)$, unlike the interim beliefs $\Psi^h(\cdot; \,x, z)$ in the main model, has support over the entire real line. Thus, the subgames no longer feature one-sided dominance in the $\Delta$-rationalizability procedure.

If type $x$ believes further that other followers use monotone strategies with threshold $x_h$ under history $h$, then his payoff to choosing $\invest$ yields
\begin{equation*}
    \pi_F^h(x_j; z, x_h) = \EE_{\theta \sim G^h(\cdot; x, z) } \left[\theta - \frac{n-1}{n} \Phi\left(\frac{x_h - \theta}{\sigma_F}\right) \right]  -\frac{\chi_\notinvest}{n}.
\end{equation*}
We show in Appendix A that $\pi_F^h(x; z, x_h)$ is strictly increasing in $x$ and crosses zero once from below. Furthermore, it is strictly increasing in $z$ and strictly decreasing in $x_h$.
Type $x$'s 
conditional rank belief is given by
\begin{equation} \label{ext_rank_belief}
\begin{cases}
    R^\invest(x; z) = \prob(x_k \leq x_j \vl x_j = x, x_L > z) = \frac{1}{2} - \frac{T\left(\frac{x-z}{\sigma},  
 ~\alpha \right)}{\Phi\left(\frac{x - z}{\sigma}\right)} & \\
    ~ & \\
    R^\notinvest(x; z)  = \prob(x_k \leq x_j \vl x_j = x, x_L \leq z) = \frac{1}{2} + \frac{T\left(\frac{z - x}{\sigma}, ~\alpha \right)}{\Phi\left(\frac{ z - x}{\sigma}\right)} 
\end{cases},
\end{equation}
where $\alpha = \sigma_F/(2\sigma_L^2 + \sigma_F^2)^{1/2}$ and $T(y, a)$ is \textit{Owen's T-function}.\footnote{\label{owen_t}~Owen's T-function, first introduced by \cite{owen_1956}, is defined by 
\[
T(y,a) = \frac{1}{2\pi}\int_0^a \frac{\ee^{-(1+t^2)y^2/2}}{1+t^2} \dd t.
\]
It gives the probability of the event $\{X > y, ~0 < Y < a X \}$ when $X$ and $Y$ are independent standard Gaussian random variables. 
See \cite{savischenko_2014} and \cite{brychkov_savischenko_2016} for an overview of the function.} The derivation of (\ref{ext_rank_belief}) is given in Appendix A. When $x = x_h$, it can be shown that
\begin{equation} \label{ext_follower_fp_eqn}
    \pi_F^h(x_h; z, x_h) = \EE_{\theta \sim G^h(\cdot; \,x_h, z)}[\theta] - \frac{n-1}{n}R^h(x_h; z) - \frac{\chi_\notinvest}{n}.
\end{equation}


Now consider type $x_L$ of the leader. Suppose that leader $x_L$ believes that
followers use monotone strategies with threshold $x_\invest$ under history $h = \invest$. Then her payoff to choosing $\invest$ is
\begin{equation*}
    \pi_L(x_L; y_\invest) = x_L - \Phi\left(\frac{x_\invest -x_L}{\sigma}\right).
\end{equation*}
which is strictly increasing in $x_L$ and strictly decreasing in $x_\invest$.


The $\Delta$-rationalizability procedure again yields six sequences. We prove in Appendix A that $(\xlow^k)_{k=0}^\infty$, $(\xilow^k)_{k=0}^\infty$, and $(\xnlow^k)_{k=0}^\infty$ are increasing and bounded above, and $(\xup^k)_{k=0}^\infty$, $(\xiup^k)_{k=0}^\infty$, and $(\xnup^k)_{k=0}^\infty$ are decreasing and bounded below.
The monotone convergence theorem therefore guarantees that they converge to $\xlow$, $\xilow$, $\xnlow$, $\xup$, $\xiup$, and $\xnup$, respectively. In addition, the limits together solve the following system of equations:
\begin{equation} \label{ext_sys_eqs}
    \begin{cases}
        \pi_L(\xlow; \xilow) = 0 \\
        \pi_L(\xup; \xiup) = 0 \\
        \pi_F^\invest(\xilow; \xup, \xilow) = 0 \\ \pi_F^\invest(\xiup; \xlow, \xiup) = 0 \\
        \pi_F^\notinvest(\xnlow; \xup, \xnlow) = 0 \\
        \pi_F^\notinvest(\xnup; \xlow, \xnup) = 0
    \end{cases}.
\end{equation}

For a given pair of $(\sigma_L, \sigma_F)$, note that we can view it as a point on the ray from the origin with slope $\sigma_F/\sigma_L$. To understand which pair of $(\sigma_L, \sigma_F)$ induces unique $\Delta$-rationalizable behavior, we establish a sufficient condition on each fixed ray $\sigma_F = \gamma \sigma_L$, $\gamma \geq 0$, along which the slope parameter of Owen's T-function is given by $\alpha = \gamma/\sqrt{2 + \gamma^2}$.



\begin{proposition} \label{prop_unique_rat_ext}

The $\Delta$-rationalizable sets are $R_L^\infty = R_L^0 \setminus \overline{R}_L^\infty$ and $R_{F, \,j}^\infty = R_{F, \,j}^0 \setminus \overline{R}_{F, \,j}^\infty$, where
\begin{equation*}
     \overline{R}_L^{\infty} = \left\{(x_L, a_L) \vl~ a_L = \invest ~\text{if}~ x_L \leq \xlow ~\text{and}~ a_L = \notinvest ~\text{if}~ x_L > \xup \right\}
\end{equation*}
and
\begin{equation*}
\overline{R}_{F, \,j}^\infty =  \left\{(x_j, s_j) \vl
\text{$s_j(h) = \invest$ if $x_j \leq \underline{x}_h$ and  $s_j(h) = \notinvest$ if $x_j > \overline{x}_h$, \text{for all}~$h \in \Action_L$} \right\}
\end{equation*} 
Moreover, \\
(i) there exists $\widehat{\sigma}_L(\gamma)$ such that the game has unique $\Delta$-rationalizable behavior if $\sigma_L > \widehat{\sigma}_L(\gamma)$; \\
(ii) in the limit as $\sigma_L \to 0$ (while keeping $\sigma_F/\sigma_L = \gamma$ fixed), the game features multiplicity of $\Delta$-rationalizable behavior.
\end{proposition}


Proposition \ref{prop_unique_rat_ext} is an analog of Proposition \ref{prop_unique_rat}. It establishes that the game generally features multiplicity of rationalizable behavior. In particular, this is necessarily the case, given the ray $\gamma$, in the limit where both noises approach zero in a way that their ratio is always given by $\gamma$. Moreover, for each $\gamma$, one can increase the noises $(\sigma_F,\sigma_L)$ in a way that their ratio is given by $\gamma$ and the game features unique rationalizable behavior. Note, however, that now, even when unique rationalizable play is obtained, the thresholds the agents use depend on the noises $(\sigma_F,\sigma_L)$. This was not the case in our main model. There, as long as $\sigma_F >\widehat{\sigma}_F$, the unique rationalizable play was always the fully efficient one. Finally, the leader having more accurate information than the followers is neither necessary nor sufficient to obtain unique rationalizable behavior, since this can happen irrespective of whether $\gamma$ is greater, equal, or less than one. On the other hand, the leader being arbitrarily better informed than followers is necessary to obtain the efficient outcome.


\subsection{Discussion}

\subsubsection{Signaling Effect, Miscoordination Effect, and Multiplicity}

In a similar spirit to the analysis of the main model, one can analyze the subgame after history $h$ and consider the rationalizable profiles of followers' type-strategy pairs.  Let $z=x_L^* $, that is, $z$ be equal to the leader's threshold in the case where unique rationalizable behavior obtains. Assume that $\widehat{x}$ is the type of follower who is indifferent between choosing $\invest$ and $\notinvest$. One can rewrite Equation \ref{eq_eqm_follower} as
\begin{equation} \label{eq_f_ext}
    \EE_{\underbrace{{\theta \sim G^h(\widehat{x}; z, \widehat{x})}}_\text{signaling effect}} \left[ \theta \right] -  \underbrace{\frac{\chi_\notinvest}{n}}_{\text{externality from leader's action}}= \underbrace{\frac{n-1}{n}R^h(\widehat{x}; z)}_{\text{miss-coordination effect}} 
\end{equation} 

As we stated, in this case, Equation (\ref{eq_f_ext}) has at least one solution. To get the uniqueness of rationalizable play, it must be the case that the derivative of the conditional rank belief function is sufficiently bounded. This is not generally the case, since around $z$, for certain values of $\sigma_L $ and $\sigma_F$\footnote{~In particular, this is necessarily the case in the limit as $\sigma_L\rightarrow 0$ and $\sigma_F \rightarrow 0$ with $\sigma_F/\sigma_L= \gamma$.}, the rank belief function abruptly changes, which means that the expected payoff of the indifferent type of follower $j$ changes sign more than once. The condition of Proposition \ref{prop_unique_rat_ext} ensures that this rapid change is not enough to make the expected payoff of follower $j$ cross the $x$-axis multiple times. Similarly to the main model, if the sign change occurred more than once, the subgame would feature at least two Bayesian Nash equilibria that would correspond to the solutions of Equation \ref{eq_f_ext}. In this case, multiplicity of rationalizable behavior immediately obtains.  Such a case is given in Figure \ref{fig_invest_noisy_multiple}.

% Figure environment removed


\begin{comment}
Notice that the subgame after the leader chooses action $\invest$ has three Bayesian Nash equilibria, in which the followers use thresholds $\xilow<x_{\invest}^* < \xiup$. The leader's best responses are given by $\xlow<x_L^* <\xup$ respectively. Now, every type of leader in $[\xlow,\xup]$ finds both actions rationalizable and the same holds for every type of a follower in $[\xilow,\xiup]$\footnote{The values $(x_L^*,x_{\invest}^*)$ together with the value $x_{\notinvest}^*$ (which is not shown in the graph) correspond to the PBE of the game.}.
\end{comment}



\subsubsection{Inefficiency of the Unique Outcome}
Contrary to the main model, the extension features an inefficient outcome irrespective of the uniqueness of rationalizable play. This result obtains whenever $\sigma_L$ is bounded away from zero. In the limiting case where $\sigma_L \to 0^+$ and for $\sigma_F$ sufficiently large, we recover the unique efficient $\Delta$-rationalizable profile of the main model. It is worth noting, though, that in all cases, the extensive form game leads to outcomes at least as efficient as the ones that would obtain if the game was a simultaneous move game, a prediction consistent with existing literature. This means that the presence of the leader is always helpful, even if her information is very imprecise. This is not surprising, since the leader's action apart from information carries a benefit that spills over to followers.

% \subsubsection{Equilibrium Behavior}

% If the sufficient condition derived in Proposition \ref{prop_unique_rat_ext} is satisfied, the game features unique rationalizable behavior. This immediately implies that the game features unique equilibrium behavior. In particular, the unique rationalizable strategy profile corresponds to the unique Perfect Bayesian Equilibrium of the game, which is in monotone strategies, with thresholds for the leader and followers given by $x_L^*$, $\xistar$ and $\xnstar$. 



\subsection{A Synthesis of the Results}
One can think of the results established in Propositions 1 and 3 in the following way: In the $(\sigma_L , \sigma_F)$ space, when $\sigma_L=0$, Proposition 1 derives a necessary and sufficient condition under which the game features unique rationalizable behavior which delivers the fully efficient outcome. When $\sigma_F=0$, multiplicity immediately obtains since the followers are perfectly informed about the state. In the limit where both noises vanish and $\sigma_F/\sigma_L\to 0$,\footnote{~ This means that followers are arbitrarily better informed than the leader.} the subgame becomes a standard global game, where the leader's action carries only the positive spillover and no information. In this case, the unique rationalizable behavior features a monotone strategy for the leader and the followers with threshold $(n-1)/2n$.  When both $\sigma_L$ and $\sigma_F$ are nonzero and vanish at a rate such that their ratio is given by $\gamma$ for any $\gamma>0$, Proposition 3 establishes the multiplicity of rationalizable behavior in the limit when we move towards the origin along the fixed ray $\gamma$ and the existence of a value $\widehat{\sigma}_L(\gamma)$ such that when one is sufficiently away from the origin the game features unique rationalizable behavior. Thus, in general, the leader cannot ``discipline'' the followers on imitating her behavior, and thus, she may choose the inefficient action in the first place. 













\medskip


\section{Conclusion and Future Work}
In this work, I design corruption-robust algorithms for the Lipschitz contextual search problem. I present the \emph{agnostic checking} technique and demonstrate its effectiveness in designing corruption-robust algorithms. There are several open problems for future research. First, in the algorithm I propose for pricing loss, the schedule for agnostic checks is fixed upfront. Can the learner design an adaptive checking schedule for the pricing loss? Second, this work assumes the learner has knowledge of the Lipschitz constant $L$. Can the learner design efficient no-regret algorithms without knowledge of $L$? 
\newpage


\bibliographystyle{te}
\bibliography{references}

\newpage

\section*{Appendix A: Proofs }

\subsection*{Proof of Lemma \ref{lemma_truncated_expectation}} First note that  type $x$'s belief about $\theta$ has a Gaussian distribution with mean $x$ and variance $\sigma_F^2$. 
Recall that $\lambda(x) = \phi(x)/\Phi(x)$ is the reversed hazard rate of a standard Gaussian random variable. 
Rewrite Equation (\ref{eq_expectation_of_theta}) as follows:
\begin{equation*} 
    \EE_{\theta \sim \Psi^h(\cdot; \,x, z)}[\theta] = \begin{cases}
   \sigma_F \left[y + \lambda(y) \right] + z  & \text{if $h = \invest$} \\
   ~ & ~ \\
   -\sigma_F \left[-y + \lambda(-y) \right] + z & \text{if $h = \notinvest$}
   \end{cases},
\end{equation*}
where $y = (x - z)/\sigma_F$. 
Since $-1 < \lambda'(x) < 0$ \citep{sampford_1953},
$x + \lambda(x)$ is strictly increasing in $x$. It is now straightforward to see that $\EE_{\theta \sim \Psi^h(\cdot; \,x, z)}[\theta]$ is strictly increasing in $x$ and $z$ because $\lambda(\cdot)$ is a  strictly decreasing function and $\sigma_F > 0$.

Using the derivative formula $\phi'(x) = -x \phi(x)$ and L'H\^{o}pital's rule, we have $\lim_{x \to -\infty} x + \lambda(x) = \lim_{x \to -\infty} \phi(x)/\phi'(x) = 0$ and hence $x + \lambda(x) > 0$ for all $x \in \RR$. We also have $\lim_{x \to \infty} x + \lambda(x) = \infty$ because $\lim_{x \to \infty} \lambda(x) = 0$. Thus,
\begin{equation*}
    \EE_{\theta \sim \Psi^\invest(\cdot; \,x, z)}[\theta] \to \begin{cases}
   z & \text{as $x \to -\infty$} \\
   \infty & \text{as $x \to \infty$}
   \end{cases},
\end{equation*}
and
\begin{equation*}
    \EE_{\theta \sim \Psi^\notinvest(\cdot; \,x, z)}[\theta] \to \begin{cases}
   -\infty & \text{as $x \to -\infty$} \\
   z & \text{as $x \to \infty$}
   \end{cases}.
\end{equation*}
This completes the proof. \qed


\medskip
\subsection*{Proof of Lemma \ref{lemma_x_payoff}}
(Part 1)
By Lemma \ref{lemma_truncated_expectation}, it suffices to show that $\Psi^h(\theta; x, z)$ is increasing in $x$ and $z$ in the sense of first-order stochastic dominance
because $\Phi\left((x_h - \theta)/\sigma_F\right)$ is strictly increasing in $\theta$ and strictly decreasing in $x_h$.
 
Given $h = \invest$. For $x' > x$, we have
\begin{equation*}
    \frac{\psi^\invest(\theta; x', z)}{\psi^\invest(\theta; x
    , z)} = \frac{\phi\left(\frac{\theta - x'}{\sigma_F}\right)}{\phi\left(\frac{\theta - x}{\sigma_F}\right)} \cdot \frac{\Phi\left(\frac{x - z}{\sigma_F}\right)}{\Phi\left(\frac{x' - z}{\sigma_F}\right)}.
\end{equation*}
Since $\phi(\cdot)$ is   log-concave, $\phi\left((\theta - x')/\sigma_F\right)/\phi\left((\theta - x)/\sigma_F\right)$ is increasing in $\theta$; i.e., $\phi(\cdot)$ is a P\'{o}lya frequency function of order 2 \citep[Proposition 2.3]{saumard_wellner_2014}, and so is $\psi^\invest(\theta; x', z)/\psi^\invest(\theta; x, z)$. This implies that $\psi^\invest(\theta; x, z)$ is log-supermodular in $(\theta, x)$, or, equivalently, $\Psi^\invest(\theta; x',z)$   dominates $\Psi^\invest(\theta; x,z)$ in the monotone likelihood ratio order. Thus, $\Psi^\invest(\theta; x',z)$   first-order stochastically dominates $\Psi^\invest(\theta; x,z)$. 

By definition, we have
\begin{equation*}
    \Psi^\invest(\theta; x, z) = 1 - \frac{\Phi\left(\frac{x - \theta}{\sigma_F}\right)}{\Phi\left(\frac{x - z}{\sigma_F}\right)}.
\end{equation*}
It is decreasing in $z$; that is, $\Psi^\invest(\theta; x, z') < \Psi^\invest(\theta; x, z)$ for $z' > z$ and for all $\theta \in (z, \infty)$, where $\Psi^\invest(\theta; x, z') = 0$ when $\theta \in (z, z']$. This proves that $\Psi(\theta; x, z')$ dominates $\Psi(\theta; x, z)$ in the first-order stochastic dominance sense. The proof for $\Psi^\notinvest(\theta; x, z)$ is analogous.

\noindent (Part 2)  Since $\Phi\left( (x_h - \theta)/\sigma_F \right)$ is bounded, it follows from Lemma \ref{lemma_truncated_expectation} that $\pi_F^\notinvest(x; z, x_\notinvest)$ $\to -\infty$ as $x \to -\infty$ and $\pi_F^\invest(x; z, x_\invest)$ $\to \infty$ as $x \to \infty$. Now under $h = \invest$,
\begin{align*}
    \EE_{\theta \sim \Psi^\invest(\cdot; \,x, z)} \left[ \Phi\left(\frac{x_\invest - \theta}{\sigma_F}\right) \right] & = \frac{1}{\Phi\left( \frac{x - z}{\sigma_F} \right)} \int_z^\infty \Phi\left( \frac{x_\invest - t}{\sigma_F} \right) \frac{1}{\sigma_F} \phi\left( \frac{x - t}{\sigma_F} \right) \dd t \\
    & = \frac{1}{\Phi\left( \frac{x - z}{\sigma_F} \right)} \int_{-\infty}^{\frac{x - z}{\sigma_F}} \phi(\eta) \Phi\left( \frac{x_\invest - x}{\sigma_F} + \eta \right) \dd \eta \\
    & \to \Phi\left( \frac{x_\invest - z}{\sigma_F} \right) \quad \text{as $x \to -\infty$}.
\end{align*}
The second equality is given by a change of variable $\eta = (x - t)/\sigma_F$, and the limiting result is a consequence of applying L'H\^{o}pital's rule. Thus, Lemma \ref{lemma_truncated_expectation} implies that $\lim_{x \to -\infty} \pi_F^\invest(x; z, x_\invest)$ $= z - ((n-1)/n)\Phi(( x_\invest - z )/\sigma_F)$.

Similarly, under $h = \notinvest$, we have 
\begin{align*}
    \EE_{\theta \sim \Psi^\notinvest(\cdot; \,x, z)} \left[ \Phi\left(\frac{x_\notinvest - \theta}{\sigma_F}\right) \right] & = \frac{1}{\Phi\left( \frac{z - x}{\sigma_F} \right)} \int_{-\infty}^z \Phi\left( \frac{x_\notinvest - t}{\sigma_F} \right) \frac{1}{\sigma_F} \phi\left( \frac{x - t}{\sigma_F} \right) \dd t \\
    & = \frac{1}{\Phi\left( \frac{z - x}{\sigma_F} \right)} \int_{\frac{x - z}{\sigma_F}}^\infty \phi(\eta) \Phi\left( \frac{x_\notinvest - x}{\sigma_F} + \eta \right) \dd \eta \\
    & \to \Phi\left( \frac{x_\notinvest - z}{\sigma_F} \right) \quad \text{as $x \to \infty$}.
\end{align*}
Thus, by Lemma \ref{lemma_truncated_expectation}, $\lim_{x \to \infty} \pi_F^\notinvest(x; z, x_\notinvest) = z - 1/n - ((n-1)/n) \Phi((x_\notinvest - z)/\sigma_F)$.
\qed

\medskip
\subsection*{Proof of Lemma \ref{lemma_rat_seq} }
Define iteratively six sequences as follows. Let $\thetalow^0 = \xilow^0 = \xnlow^0 = -\infty$ and $\thetaup^0 = \xiup^0 = \xnup^0 = \infty$, and for $k \geq 1$, 
\begin{equation*}
    \begin{cases}
        \thetalow^k = \br_L(\xilow^{k-1}) & \\
        \thetaup^k = \br_L(\xiup^{k-1}) & \\
        \xilow^k = \br_F^\invest(\thetaup^k, \xilow^{k-1}) & \\
        \xiup^k = \br_F^\invest(\thetalow^k, \xiup^{k-1}) & \\
        \xnlow^k = \br_F^\notinvest(\thetaup^k, \xnlow^{k-1}) & \\
        \xnup^k = \br_F^\notinvest(\thetalow^k, \xnup^{k-1})
    \end{cases},
\end{equation*}
where $\theta = \br_L(x)$, $x \in \RR \cup \{-\infty, \infty\}$, is 
the unique solution to
\begin{equation*}
   \pi_L(\theta; x) = \theta - \Phi\left(\frac{x - \theta}{\sigma_F}\right) = 0,
\end{equation*}
and, $\br_F^h(\theta', x')$, $(\theta', x') \in \RR \times \RR \cup \{-\infty, \infty\}$, is the unique value of $x$, if exists, that solves
\begin{equation*} 
     \pi_F^h(x; \theta', x') = \EE_{\theta \sim \Psi^h(\cdot; \,x, \theta')}\left[ \theta - \frac{n-1}{n} \Phi\left( \frac{x' - \theta}{\sigma_F} \right) \right] - \frac{\chi_\notinvest}{n} = 0; 
\end{equation*}
otherwise
\begin{equation*}
    \br_F^h(\theta', x') = \begin{cases}
        -\infty & \text{if $h = \invest$} \\
        \infty & \text{if $h = \notinvest$}
    \end{cases}.
\end{equation*}
Now we prove statements (a)-(f) as follows by induction.

\vskip 0.5 \baselineskip
\noindent Parts (a), (b), (c) \& (d): For $k = 1$, 
we have $\thetalow^1 = 0$ and $1 = \thetaup^1 < \thetaup^0$
because $\xilow^0 = -\infty$ and $\xiup^0 = \infty$. It follows that $\xilow^1 = \br_F^\invest(1, -\infty) = - \infty$ because $\EE_{\theta \sim \Psi^\invest(\cdot; \,x, 1)}[\theta] > 0$ for all $x$ by Lemma \ref{lemma_truncated_expectation}. Also, $\xiup^1 = \br_F^\invest(0, \infty)$ is the unique solution to $\EE_{\theta \sim \Psi^\invest(\cdot; \,\xiup^1, 0)}[\theta] = (n-1)/n$; therefore $\xiup^1 < \xiup^0$.

Suppose now that $\thetalow^k = 0$, $\thetaup^k < \thetaup^{k-1}$, $\xilow^k = -\infty$, and $\xiup^k < \xiup^{k -1}$ for any given $k \geq 2$. Then $\thetalow^{k+1} = \br_L(\xilow^k) = \br_L(-\infty) =  0$, and $0 < \thetaup^{k+1} = \br_L(\xiup^k) < \br_L(\xiup^{k-1}) = \thetaup^k$ because the leader's payoff is strictly decreasing in $x_\invest$. Furthermore, 
$\xilow^{k+1} = \br_F^\invest(\thetaup^{k+1}, \xilow^k) = \br_F^\invest(\thetaup^{k+1}, -\infty) = -\infty$
because $\EE_{\theta \sim \Psi^\invest(\cdot; \,x, \thetaup^{k+1})}[\theta] > \EE_{\theta \sim \Psi^\invest(\cdot; \,x, 0}[\theta] > 0$ for all $x$ by Lemma \ref{lemma_truncated_expectation}. Since Lemma \ref{lemma_x_payoff} implies that $\br_F^\invest(0, x)$ is strictly increasing in $x$, it is now clear that
\begin{equation*}
    \xiup^{k+1} = \br_F^\invest(\thetalow^{k+1}, \xiup^k) = \br_F^\invest(0, \xiup^k) < \br_F^\invest(0, \xiup^{k-1}) = \br_F^\invest(\thetalow^k, \xiup^{k-1}) = \xiup^k.
\end{equation*}
This completes the proof of parts (a), (b), (c), and (d).

\vskip 0.5 \baselineskip
\noindent Part (e): For $k = 1$, $\xnlow^1 = \br_F^\notinvest(\thetaup^1, \xnlow^0) = \br_F^\notinvest(\thetaup^1, -\infty) > \xnlow^0$ because, by Lemma \ref{lemma_truncated_expectation}, $\EE_{\theta \sim \Psi^\notinvest(\cdot; \,x, \thetaup^1)}[\theta] = 1/n$ has a unique solution.
Suppose now that $\xnlow^k > \xnlow^{k-1}$ for any given $k \geq 2$.
We know from Lemma \ref{lemma_x_payoff} that $\br_F^\notinvest(\theta, x)$ is strictly decreasing in $\theta$ but strictly increasing in $x$. Thus, $\xnlow^{k+1} = \br_F^\notinvest(\thetaup^{k+1}, \xnlow^k) > \br_F^\notinvest(\thetaup^k, \xnlow^k) > \br_F^\notinvest(\thetaup^k, \xnlow^{k-1}) = \xnlow^k$.

\vskip 0.5 \baselineskip
\noindent Part (f): For $k=1$, we know by Lemma \ref{lemma_truncated_expectation} that $\pi_F^\notinvest(x; \thetalow^1, \xnup^0) = \EE_{\theta \sim \Psi^\notinvest(\cdot; \,x, 0)}[\theta] - 1 < -1$ for all $x$. This implies that $\xnup^1 = \infty$.
Suppose that $\xnup^k = \infty$ for any given $k \geq 2$. We again have $\pi_F^\notinvest(x; \thetalow^k, \xnup^0) = \EE_{\theta \sim \Psi^\notinvest(\cdot; \,x, 0)}[\theta] - 1 < -1
$ for all $x$. Thus, $\xnup^{k+1} = \infty$. \qed




\medskip
\subsection*{Proof of Proposition \ref{prop_unique_rat} }
The model has a unique $\Delta$-rationalizable behavior if and only if $\thetaup = 0$, $\xiup = -\infty$, and $\xnlow = \infty$.
If $\xiup = -\infty$, Equation (\ref{leader_upper}) implies that
$\thetaup = 0$. By Equation (\ref{follower_notinvest_lower}),
it follows that $\xnup = \infty$ because we know from Equations (\ref{eq_expectation_of_theta}) and (\ref{follower_payoff_rank_belief}) that
\begin{equation*}
    \pi_F^\notinvest(x; 0, x) = x - \sigma_F \lambda\left( -\frac{x}{\sigma_F} \right) - \frac{n-1}{2n} \Phi \left(\frac{x}{\sigma_F}\right) - \frac{n+1}{2n} < 0
\end{equation*}
for all $x$. However, if there exists a value of $x$ such that $\pi_F^\invest(x; 0, x) = 0$, then 
\begin{equation*}
    \xiup = \max \left\{x \vl \pi_F^\invest(x; 0, x) =0 \right\}
\end{equation*}
and hence $\thetaup > \thetalow$. Thus, a unique $\Delta$-rationalizable behavior obtains if and only if $\xiup = -\infty$. It is worth noting that 
$\xiup > -\infty$ does not necessarily imply
that $\xnlow < \infty$.

Now we show that there exists a unique $\widehat{\sigma}_F$ such that $\xiup = -\infty$ if and only if $\sigma_F > \widehat{\sigma}_F$.
Again by Equations (\ref{eq_expectation_of_theta}) and (\ref{follower_payoff_rank_belief}) we can write $\pi_F^\invest(x; 0, x) = 0$ as
\begin{equation} \label{app_eq_fb}
    x = \frac{n-1}{2n}\Phi\left(\frac{x}{\sigma_F}\right) - \sigma_F \lambda\left(\frac{x}{\sigma_F}\right). \tag{A.1}
\end{equation}
Define $\rho(x, \sigma_F)$ to be the right-hand side of Equation (\ref{app_eq_fb}).
Observe that $\partial \rho(x, \sigma_F)/\partial x > 0$, $\lim_{x \to \infty}\rho(x, \sigma_F) = (n-1)/(2n)$, $\lim_{x \to -\infty} \rho(x, \sigma_F) = 1$, and $x > \rho(x, \sigma_F)$ for $x < 0$ but $|x|$ sufficiently large. 
Let $\bar{\sigma}_F = (n-1)/(8n\phi(0))$. It suffices to consider the following two cases.

\textit{Case 1}: Suppose that $\sigma_F \leq \bar{\sigma}_F$. It is equivalent to $\rho(0, \sigma_F) \geq 0$. Since
\begin{equation*}
    \frac{\partial \rho(0, \sigma_F)}{\partial x} = \frac{n-1}{2n\sigma_F}\phi(0) - \lambda'(0) \geq 8\phi(0)^2 > 1
\end{equation*}
by the derivative formula $\lambda'(x) = -\lambda(x)\left[ x + \lambda(x) \right]$,
Equation (\ref{app_eq_fb}) must have at least two solutions.

\textit{Case 2}: Suppose that $\sigma_F > \bar{\sigma}_F$.
For $x > 0$, 
\begin{equation} \label{app_rho_sig}
   \frac{\partial \rho(x, \sigma_F)}{\partial \sigma_F} = - \frac{x}{\sigma_F} \left(\frac{\partial \rho(x, \sigma_F)}{\partial x}\right) < 0, \tag{A.2}
\end{equation}
and
\begin{equation} \label{app_rho_sd}
    \frac{\partial^2 \rho(x, \sigma_F)}{\partial x^2} = \frac{n-1}{2n\sigma_F^2}\phi'\left(\frac{x}{\sigma_F}\right) - \frac{1}{\sigma_F}\lambda''\left(\frac{x}{\sigma_F}\right) < 0 \tag{A.3}
\end{equation}
because $\lambda''(x) > 0$. By (\ref{app_rho_sig}) and (\ref{app_rho_sd}), there exists a unique $\widehat{\sigma}_F$ such that $x$ and $\rho(x, \widehat{\sigma}_F)$ are tangent to each other at some $x > 0$ because $\rho(x, \sigma_F)$ is strictly concave. Moreover, for $x > 0$, $x > \rho(x, \sigma_F)$ if and only if $\sigma_F > \widehat{\sigma}_F$. If we can prove that $x > \rho(x, \sigma_F)$ for $x \leq 0$ whenever $\sigma_F > \widehat{\sigma}_F$, then we are done. Note that, at $\sigma_F = \bar{\sigma}_F$, 
\begin{equation*}
    \rho(x, \bar{\sigma}_F) - x = \bar{\sigma}_F \bigg[2\lambda(0)\Phi(z) - \lambda(z) - z \bigg] < 0,
\end{equation*}
where $z = x/\bar{\sigma}_F$ because $\lambda(0) = 2\phi(0)$ and $2\lambda(0)\Phi(x) - \lambda(x) - x < 0$ for $x < 0$. It follows that $x > \rho(x, \sigma_F)$ for all $x \leq 0$. Thus, (\ref{app_eq_fb}) can 
only have solutions if $\sigma_F \leq \widehat{\sigma}_F$.

Taking Case 1 and Case 2 together, we can conclude that (\ref{app_eq_fb}) has no solution (i.e., $\xiup = -\infty$) if and only if $\sigma > \widehat{\sigma}_F$. \qed


\medskip
\subsection*{Proof of Proposition \ref{prop_limit}}

\underline{Step 1}. We first show that $\xiup \to (n-1)/2n$ as $\sigma_F \to 0$. 
For sufficiently small $\sigma_F$,
note that $\xiup > -\infty$ is determined by Equation (\ref{follower_invest_upper}); that is,
\begin{equation*}
    \xiup + \sigma_F \lambda \left( \frac{\xiup - \thetalow}{\sigma_F}  \right) - \frac{n-1}{2n} \Phi\left( \frac{\xiup - \thetalow}{\sigma_F}  \right) = 0,
\end{equation*}
where $\thetalow = 0$.
We know from the proof of Proposition \ref{prop_unique_rat} that $\xiup > 0$. So, as $\sigma_F \to 0$, there are three cases to consider: (i) $\xiup \to 0$ and $\xiup/\sigma_F \to k \geq 0$, (ii) $\xiup \to 0$ and $\xiup/\sigma_F \to \infty$, and  (iii) $\xiup \to \aleph > 0$ and $\xiup/\sigma_F \to \infty$, 

\emph{Case (i)}: Suppose that $\xiup \to 0$ and $\xiup/\sigma_F \to k$, where $k \geq 0$ is a constant. It follows that
\begin{equation*}
    \xiup + \sigma_F \lambda \left( \frac{\xiup}{\sigma_F}  \right) - \frac{n-1}{2n} \Phi\left( \frac{\xiup}{\sigma_F}  \right) \to 0 + 0 \cdot \lambda(k) - \frac{n-1}{2n}\Phi(k) \neq 0,
\end{equation*}
which leads to a contradiction.

\emph{Case (ii)}: Suppose that $\xiup \to 0$ and $\xiup/\sigma_F \to \infty$. Then we have a contradiction because
\begin{equation*}
    \xiup + \sigma_F \lambda \left( \frac{\xiup}{\sigma_F}  \right) - \frac{n-1}{2n} \Phi\left( \frac{\xiup}{\sigma_F}  \right) \to - \frac{n-1}{2n} < 0.
\end{equation*}

\emph{Case (iii)}: Suppose that $\xiup \to \aleph > 0$. Then it must be the case that
\begin{equation*}
    \xiup + \sigma_F \lambda \left( \frac{\xiup}{\sigma_F}  \right) - \frac{n-1}{2n} \Phi\left( \frac{\xiup}{\sigma_F}  \right) \to \aleph - \frac{n-1}{2n} = 0,
\end{equation*}
which results in a contradiction except for the case $\aleph = (n-1)/2n$.

Combining all three cases above, we can conclude that $\xiup \to (n-1)/2n$ as $\sigma_F \to 0$.

\vskip 0.5 \baselineskip
\noindent \underline{Step 2}. We show next that $\thetaup \to (n-1)/2n$ as $\sigma_F \to 0$. We consider, for the sake of contradiction, the following two cases: (i) $\thetaup \to  \tau < (n-1)/2n$, and (iii) $\thetaup \to  \tau > (n-1)/2n$.

\emph{Case (i)}: Suppose that $\thetaup \to  \tau$, where $\tau \in [0, (n-1)/2n)$ is a constant. Note that $\thetaup$ is given by Equation (\ref{leader_upper}):
\begin{equation*}
    \thetaup - \Phi \left( \frac{\xiup - \thetaup}{\sigma_F} \right) = 0.
\end{equation*}
Since $(\xiup - \thetaup)/\sigma_F \to \infty$, we have
\begin{equation*}
\thetaup - \Phi \left( \frac{\xiup - \thetaup}{\sigma_F} \right) \to \tau - 1 < - \frac{n+1}{2n} < 0,     
\end{equation*}
which leads to a contradiction.

\emph{Case (ii)}: Suppose that $\thetaup \to  \tau \in ((n-1)/2n, 1]$. Then it must be the case that $(\xiup - \thetaup)/\sigma_F \to -\infty$. But we have a contradiction because 
\begin{equation*}
\thetaup - \Phi \left( \frac{\xiup - \thetaup}{\sigma_F} \right) \to \tau > \frac{n-1}{2n} > 0.     
\end{equation*}

Thus, we must have $\thetaup \to (n-1)/2n$ as $\sigma_F \to 0$.

\medskip
\noindent \underline{Step 3}. Lastly, we show that $\xnlow \to \infty$ as $\sigma_F \to 0$. 
By way of contradiction, suppose that $\xnlow \to \aleph < \infty$. This means that $\aleph$ solves  
Equation (\ref{follower_notinvest_lower}):
\begin{equation*}
    \aleph - \sigma_F \lambda \left( \frac{\thetaup - \aleph}{\sigma_F} \right) + \frac{n-1}{2n} \Phi\left( \frac{\thetaup - \aleph}{\sigma_F}\right) = 1.
\end{equation*}
There are three possible cases for the limit of $(\thetaup - \xnlow)/\sigma_F$ as $\sigma_F \to 0$: (i) $-\infty$, (ii) $\infty$, or (iii) a constant $k \in \RR$.

\emph{Case (i)}: Suppose that $(\thetaup - \xnlow)/\sigma_F \to -\infty$. Since $\lim_{x \to -\infty} x + \lambda(x) = 0$, we have
\begin{align*}
    \xnlow - & \sigma_F \lambda \left( \frac{\thetaup - \xnlow}{\sigma_F} \right) + \frac{n-1}{2n} \Phi\left( \frac{\thetaup - \xnlow}{\sigma_F}\right) \\
    & = -\sigma_F \left( \frac{\thetaup - \xnlow}{\sigma_F} + \lambda \left( \frac{\thetaup - \xnlow}{\sigma_F} \right) \right) + \thetaup + \frac{n-1}{2n} \Phi\left( \frac{\thetaup - \xnlow}{\sigma_F}\right) \\
    & \to \frac{n-1}{2n} < 1.
\end{align*}
Thus, we have a contradiction.

\emph{Case (ii)}: Suppose that $(\thetaup - \xnlow)/\sigma_F \to \infty$. In this case, we must have $\aleph \leq (n-1)/2n$. It follows that
\begin{equation*}
    \xnlow - \sigma_F \lambda \left( \frac{\thetaup - \xnlow}{\sigma_F} \right) + \frac{n-1}{2n} \Phi\left( \frac{\thetaup - \xnlow}{\sigma_F}\right) \to \aleph + \frac{n-1}{2n} \leq \frac{n-1}{n} < 1,
\end{equation*}
which leads to a contradiction.

\emph{Case (iii)}: Suppose that $(\thetaup - \xnlow)/\sigma_F \to k$, where $k$ is a constant. This implies that $\aleph = (n-1)/2n$. But since
\begin{equation*}
    \xnlow - \sigma_F \lambda \left( \frac{\thetaup - \xnlow}{\sigma_F} \right) + \frac{n-1}{2n} \Phi\left( \frac{\thetaup - \xnlow}{\sigma_F}\right) \to \aleph + \frac{n-1}{2n}\Phi(k) < \frac{n-1}{n} < 1,
\end{equation*}
we have a contradiction.

Thus, we must have $\xnlow \to \infty$ as $\sigma_F \to 0$. This completes the proof. \qed


\medskip
\subsection*{Derivation of the conditional rank beliefs in (\ref{rank_belief}) } 
By definition, we have
\begin{align*}
    R^h(x; z) & = \prob(x_k \leq x_j \vl x_j = x, z) \\
    & = \prob(\theta + \sigma_F \varepsilon_k \leq x_j \vl x_j = x, z) \\
    & = \int_{-\infty}^\infty \left( \int_{-\infty}^\frac{x - \theta}{\sigma_F} \phi(\varepsilon) \dd \varepsilon \right) \dd \Psi^h(\theta; x, z) \\
    & = \int_{-\infty}^\infty \Phi\left( \frac{x - \theta}{\sigma_F} \right) \dd \Psi^h(\theta; x, z).
\end{align*}
Now under $h = \invest$, by (\ref{interim_belief})
\begin{align*}
    R^\invest(x; z) & = \frac{1}{\Phi\left( \frac{x - z}{\sigma_F}\right) } \int_z^\infty \Phi\left( \frac{x - \theta}{\sigma_F} \right) \frac{1}{\sigma_F} \phi\left( \frac{\theta - x}{\sigma_F}  \right) \dd \theta \\
    & = \frac{1}{\Phi\left( \frac{x - z}{\sigma_F}\right) } \int_{-\infty}^{\Phi\left(\frac{x-z}{\sigma_F}\right)} \eta \dd \eta \\ 
    & = \frac{1}{2}\Phi\left( \frac{x-z}{\sigma_F} \right).
\end{align*}
The second equality is given by a change of variable $\eta = \Phi((x - \theta)/\sigma_F)$. Similarly,
\begin{align*}
    R^\notinvest(x; z) & = \frac{1}{\Phi\left( \frac{z - x}{\sigma_F}\right) } \int_{-\infty}^z \Phi\left( \frac{x - \theta}{\sigma_F} \right) \frac{1}{\sigma_F} \phi\left( \frac{\theta - x}{\sigma_F} \right) \dd \theta \\
    & = \frac{1}{\Phi\left( \frac{z - x}{\sigma_F}\right) } \int_{\Phi\left(\frac{x-z}{\sigma_F}\right)}^1 \eta \dd \eta \\
    & = \frac{1}{2} \Phi\left( \frac{x-z}{\sigma_F} \right) + \frac{1}{2}.
\end{align*}


\medskip
\subsection*{Derivation of the conditional rank beliefs in (\ref{ext_rank_belief})}

Under history $h = \invest$,
\begin{align*}
    R^\invest(x; z) & = \prob(x_k \leq x_j \vl x_j = x, x_L > z) \\
    & = \int_{-\infty}^\infty \left( \int_{-\infty}^\frac{x - \theta}{\sigma_F} \phi(\varepsilon) \dd \varepsilon \right) g^\invest (\theta; x , z) \dd \theta \\ 
    & = \frac{1}{\Phi \left( \frac{x - z }{\sigma} \right) }\int_{-\infty}^\infty \Phi(-y) \phi(y) \Phi \left( \frac{x - z + \sigma_F y }{\sigma_L} \right) \dd y \\
    & = \frac{1}{2} - \frac{1}{\Phi \left( \frac{x - z }{\sigma} \right)} T \left( \frac{x - z}{\sigma},  \frac{\sigma_F}{(2\sigma_L^2 + \sigma_F^2)^{\frac{1}{2}}} \right).
\end{align*}
The second equality is given by the independence between 
$\theta$ and $\varepsilon_k$,
the third equality is due to a change of variable $y = (\theta - x_\invest)/\sigma_F$, and the last equality is derived from applying the following integral identity
\begin{equation*}
    \int_{-\infty}^\infty \Phi(a + bx) \Phi(cx) \phi(x) \dd x = \frac{1}{2} \Phi \left( \frac{a}{\sqrt{1 + b^2}} \right) + T \bigg( \frac{a}{\sqrt{1 + b^2}}, \frac{bc}{\sqrt{1 + b^2 + c^2}}\bigg)
\end{equation*}
and $T(y, -a) = -T(y, a)$, where $T(y, a)$ is Owen's T-function (see footnote \ref{owen_t}).

Under history $h = \notinvest$, a similar argument yields
\begin{align*}
    R^\notinvest(x; z) & = \prob (x_k \leq x_j \vl x_j = x, x_L \leq z) \\
    & = \int_{-\infty}^\infty \Phi \left( \frac{x - \theta }{\sigma_F} \right)  g^\notinvest(\theta; x, z) \dd \theta \\
    & = \frac{1}{2} + \frac{1}{\Phi \left( \frac{z - x}{\sigma} \right)} T \left( \frac{z - x}{\sigma},  \frac{\sigma_F}{(2\sigma_L^2 + \sigma_F^2)^{\frac{1}{2}}} \right).
\end{align*}



\medskip




\begin{lemma} \label{lemma_exp_ext}
Under history $h$, $\EE_{\theta \sim G^h(\cdot; \,x, z)}[\theta]$ is increasing in $x$ and $z$. Moreover,
\[
\EE_{\theta \sim G^h(\cdot; \,x, z)}[\theta] \to \begin{cases}
     \infty & \text{as $x \to \infty$} \\
     -\infty & \text{as $x \to -\infty$}
\end{cases}.
\]
\end{lemma}
\begin{proof}[Proof of Lemma \ref{lemma_exp_ext}]
Note that under history $h = \invest$,
\begin{align} \label{app_fol_exp}
    \EE_{\theta \sim G^\invest(\cdot; \,x, z)} [\theta] & = \frac{1}{\Phi\left(\frac{x-z}{\sigma}\right)}\int_{-\infty}^\infty \frac{t}{\sigma_F} \phi\left(\frac{t-x}{\sigma_F}\right) \Phi\left(\frac{t-z}{\sigma_L}\right) \dd t  \nonumber \\
    & = \frac{1}{\Phi\left(\frac{x-z}{\sigma}\right)}\int_{-\infty}^\infty (x + \sigma_F \eta) \phi(\eta) \Phi\left(\frac{x + \sigma_F \eta - z}{\sigma_L}\right) \dd \eta \nonumber \\
    & = x + \frac{\sigma_F^2}{\sigma} \lambda \left(\frac{x-z}{\sigma}\right). \tag{A.4}
\end{align}
The second inequality is due to a change of variable $\eta = (t-x)/\sigma_F$, and the third equality is derived by applying integral identities $\int_{-\infty}^\infty \phi(\eta)\Phi(a + b\eta) \dd \eta = \Phi(a/\sqrt{1 + b^2})$ and $\int_{-\infty}^\infty \eta \phi(\eta) \Phi(a+b\eta) \dd \eta = (b/\sqrt{1+b^2})\Phi(a/\sqrt{1+b^2})$ (see, for example, \cite{owen_1980}). Therefore, (\ref{app_fol_exp}) is increasing in $x$ and $z$ because
\begin{align*}
    \EE_{\theta \sim G^\invest(\cdot; \,x, z)} [\theta] & = \frac{\sigma_L^2}{\sigma}\left(\frac{x-z}{\sigma}\right) + \frac{\sigma_F^2}{\sigma}\left(\frac{x-z}{\sigma} + \lambda\left(\frac{x-z}{\sigma}\right)\right) + z,
\end{align*}
$\eta + \lambda(\eta)$ is increasing in $\eta$, and $-1 < \lambda'(\eta) < 0$. The last inequality is due to \cite{sampford_1953}.
Now given the fact that $\lambda(\eta)/\eta \to -1$ as $\eta \to -\infty$ and $\lambda(\eta)/\eta \to 0$ as $\eta \to \infty$, we can conclude that
\[
\EE_{\theta \sim G^\invest(\cdot; \,x, z)} [\theta] \to \begin{cases}
    \infty & \text{as $x \to \infty$} \\
    -\infty & \text{as $x \to -\infty$}
\end{cases}.
\]


Similarly, under history $h = \notinvest$, one can show that
\begin{align*}
    \EE_{\theta \sim G^\notinvest(\cdot; \,x, z)}[\theta] & = x - \frac{\sigma_F^2}{\sigma} \lambda\left(\frac{z-x}{\sigma}\right) \\
    & = -\frac{\sigma_L^2}{\sigma}\left(\frac{z-x}{\sigma}\right) - \frac{\sigma_F^2}{\sigma}\left(\frac{z-x}{\sigma} + \lambda\left(\frac{z-x}{\sigma}\right)\right) + z,
\end{align*}
which is increasing in $x$ and $z$, approaches to $-\infty$ as $x \to -\infty$, and approaches to $\infty$ as $x \to \infty$. The proof is complete.
\end{proof}

\begin{lemma} \label{lemma_payoff_ext}
Under history $h$, $\pi_F^h(x; z, x_h)$ is increasing in $x$ and $z$, and is decreasing in $x_h$. Moreover, $\lim_{x \to -\infty}\pi_F^h(x; z, x_h) = -\infty$ and $\lim_{x \to \infty} \pi_F^h(x; z, x_h) = \infty$.
\end{lemma}
\begin{proof}[Proof of Lemma \ref{lemma_payoff_ext}]
Recall that
\[
\pi_F^h(x; z, x_h) = \EE_{\theta \sim G^h(\cdot; \,x, z)} \left[\theta - \frac{n-1}{n}\Phi\left(\frac{x_h - \theta}{\sigma_F}\right)\right] - \frac{\chi_\notinvest}{n}.
\]
It is immediate to see that $\pi_F^h(x; z, x_h)$ is decreasing in $x_h$ since $\Phi((x_h - \theta)/\sigma_F)$ is increasing in $x_h$.
To show $\pi_F^h(x; z, x_h)$ is increasing in both $x$ and $z$, it suffices to prove that $G^h(\cdot; \,x, z)$ is increasing in $x$ and $z$ with respect to the first-order stochastic dominance order. We ignore the proof as it is similar to that of Lemma \ref{lemma_x_payoff}.
The limits at infinity are given directly by Lemma \ref{lemma_exp_ext} and the boundedness of $\Phi((x_h - \theta)/n)$.
\end{proof}

\begin{lemma} \label{lemma_seq_ext} 
Let $\xlow^0 = \xhlow^0 = -\infty$ and $\xup^0 = \xhup^0 = \infty$ for each history $h$. Then
the iterative procedure of $\Delta$-rationalizability yields six sequences: \\
(i) $(\xlow^k)_{k = 0}^\infty, (\xilow^k)_{k = 0}^\infty$, and $(\xnlow^k)_{k = 0}^\infty$ are strictly increasing and bounded above;\\
(ii) $(\xup^k)_{k = 0}^\infty, (\xiup^k)_{k = 0}^\infty$, and $(\xnup^k)_{k = 0}^\infty$ are strictly decreasing and bounded below.
\end{lemma}
\begin{proof}[Proof of Lemma \ref{lemma_seq_ext}]
Let $x_L = \br_L(x_\invest)$ denote the unique solution to
\[
\pi_L(x_L; x_\invest) = x_L - \Phi\left(\frac{x_\invest-x_L}{\sigma}\right),
\]
and $x = \br_F^h(z, x_h)$ the unique solution to $\pi_F^h(x; z, x_h) = 0$ for each history $h$.
The latter is guaranteed by both Lemma \ref{lemma_exp_ext} and Lemma \ref{lemma_payoff_ext}.

Let $\xlow^0 = \xhlow^0 = -\infty$ and $\xup^0 = \xhup^0 = \infty$. Define for $k \in \NN$,
\begin{equation} \label{ext_seq_br}
\begin{cases}
\xlow^k = \br_L(\xilow^{k-1}) \\
\xup^k = \br_L(\xiup^{k-1}) \\
\xilow^k = \br_F^\invest(\xup^k, \xilow^{k-1}) \\
\xiup^k  = \br_F^\invest(\xlow^k, \xiup^{k-1}) \\
\xnlow^k = \br_F^\notinvest(\xup^k, \xnlow^{k-1}) \\
\xnup^k = \br_F^\notinvest(\xlow^k, \xnup^{k-1}) \\
\end{cases}. \tag{A.5}
\end{equation}
Note that, by Lemma \ref{lemma_payoff_ext}, $\br_L(x_\invest)$ is strictly increasing in $x_\invest$ and $\br_F^h(z, x_h)$ is strictly decreasing in $z$ and strictly increasing in $x_h$.  We prove the lemma by induction. 

\noindent For $k = 1$, consider the leader first. Since $\xlow^1 = \br_L(\xilow^0) = 0$ and $\xup^0 = \br_L(\xiup^0) = 1$, it is immediate that $\xlow^0 < \xlow^1 < \xup^1 < \xup^0$. Now under history $h$, Lemma \ref{lemma_exp_ext} implies that $\xhlow^1 > \xhlow^0$ and $\xhup^1 < \xhup^0$. But since
\[
\xhlow^1 = \br_F^h(\xup^1, \xhlow^0) < \br_F^h(\xlow^1, \xhlow^0) < \br_F^h(\xlow^1, \xhup^0) = \xhup^1,
\]
it follows that $\xhlow^0 < \xhlow^1 < \xhup^1 < \xhup^0$.


\noindent Assume that, for $k \geq 1$, $\xlow^{k-1} < \xlow^k < \xup^k < \xup^{k-1}$ for the leader, and $\xhlow^{k-1} < \xhlow^k < \xhup^k < \xhup^{k+1}$ for followers under history $h$. The induction hypothesis implies
\begin{align*}
    \br_L(\xilow^{k-1}) < \br_L(\xilow^k) < \br_L(\xiup^k) < \br_L(\xiup^{k-1});
\end{align*}
therefore we have $\xlow^k < \xlow^{k+1} < \xup^{k+1} < \xup^k$.
The proof for $\xhlow^k < \xhlow^{k+1} < \xhup^{k+1} < \xhup^k$ is straightforward.
\end{proof}

\medskip
\noindent Define $\tilde{S}(y, \nu) = 1/2 - T(y, \nu)/\Phi(y)$, where $T(y, \nu)$ is Owen's T-function with slope parameter $\nu > 0$ and $y \in \RR$ (see footnote \ref{owen_t}).
\begin{lemma} \label{lemma_s_func}
    The function $\tilde{S}(y, \nu)$ is strictly increasing and differentiable everywhere in $y$. Moreover, $\lim_{y \to -\infty} \tilde{S}(y, \nu) = 0$, $\lim_{y \to \infty} \tilde{S}(y, \nu) = 1/2$, and if $\nu \in (0, 1)$, then $\partial \tilde{S}(y, \nu)/\partial y$ vanishes at infinity ; i.e.,
    \begin{equation*}
         \lim_{y \to -\infty} \frac{\partial  \tilde{S}(y, \nu)}{\partial y} =          \lim_{y \to \infty} \frac{\partial  \tilde{S}(y, \nu)}{\partial y} = 0.
    \end{equation*}
\end{lemma}
\begin{proof}[Proof of Lemma \ref{lemma_s_func}]
Since $T(y, \nu)$ is differentiable everywhere in $y$, so is $\tilde{S}(y, \nu)$. Note that
\begin{align*}
    \frac{\dd }{\dd y}\left[ \frac{T(y, \nu)}{\Phi(y)} \right] & = \frac{1}{2\Phi(y)^2} \left[-\phi(y) \erf\left( \frac{\nu y}{\sqrt{2}} \right) \Phi(y) - 2\phi(y) T(y, \nu)\right]  \\
 & \propto - \phi(y) \erf\left( \frac{\nu y}{\sqrt{2}} \right) \Phi(y) - \phi(y) \int_{-y}^\infty \phi(t) \erf\left(
 \frac{\nu t}{\sqrt{2}} \right) \dd t \\
 & = - \phi(y) \int_{-y}^\infty \phi(t) \left[ \erf\left( 
 \frac{\nu y}{\sqrt{2}} \right) + \erf\left( 
 \frac{\nu t}{\sqrt{2}} \right) \right] \dd t \\
 & < 0,
\end{align*}
where $\erf(\cdot)$ is the error function.
The second equality is derived using the integral representation of Owen's T-function (see Equation (3.1) in \cite{brychkov_savischenko_2016}). The inequality holds because the strict monotonicity of $\erf(\cdot)$ implies that
\begin{equation*}
    \erf\left( \frac{\nu y}{\sqrt{2}} \right) + \erf\left( 
 \frac{\nu t}{\sqrt{2}} \right) >  \erf\left( 
 \frac{\nu y}{\sqrt{2}} \right) + \erf\left( 
 -\frac{\nu y}{\sqrt{2}} \right) = 0
\end{equation*}
for all $t > -y$. Thus, $\tilde{S}(y, \nu)$ is strictly increasing. Moreover, since $T(y, \nu) \to 0$ as $y \to \infty$ or $y \to -\infty$, we have $\lim_{y \to \infty} \tilde{S}(y, \nu) = 1/2$ and
\begin{equation*}
    \lim_{y \to -\infty} \tilde{S}(y, \nu) = \frac{1}{2} -\lim_{y \to -\infty}\frac{T(y, \nu)}{\Phi(y)} = \frac{1}{2} + \frac{1}{2}\lim_{y \to -\infty} \erf\left( \frac{\nu y}{\sqrt{2}} \right)  = 0.
\end{equation*}
The last equality is due to $\erf(\nu y/\sqrt{2}) \to -1$ as $y \to -\infty$.

Now we define
\begin{equation*}
    M(y, \nu) = \int_{-y}^\infty \phi(t) \erf\left( 
 \frac{\nu t}{\sqrt{2}} \right)  \dd t + \erf\left( \frac{\nu y}{\sqrt{2}}\right)\Phi(y),
\end{equation*}
and write $\tilde{S}_y(y, \nu)$ for the partial derivative $\partial \tilde{S}(y, \nu)/\partial y$. Then we know from above that
\begin{equation*}
    \tilde{S}_y(y, \nu) = \frac{\phi(y) M(y, \nu)}{2 \Phi(y)^2}.
\end{equation*}
To show that $\tilde{S}_y(y, \nu)$ vanishes as $y \to \infty$, note that
\begin{equation*}
    \lim_{y \to \infty} M(y, \nu) = \int_{-\infty}^\infty \phi(t) \erf\left( \frac{\nu t}{\sqrt{2}}\right) \dd t + 1
    = 2 \int_{-\infty}^\infty \phi(t) \Phi(\nu t) \dd t
    = 2 \Phi(0) = 1.
\end{equation*}
The first equality holds because $\erf(\nu y/\sqrt{2}) \to 1$ as $y \to \infty$, the second equality is due to $\erf(\nu t/\sqrt{2}) = 2 \Phi(\nu t) - 1$, and the last equality is given by applying the integral identity $\int_{-\infty}^\infty \phi(t) \Phi(a + bt) \dd t = \Phi(a/\sqrt{1+b^2})$. Thus, $\lim_{y \to \infty} \tilde{S}_y(y, \nu) = 0$. 

Suppose that $\nu \in (0, 1)$.
Since
\begin{equation*}
    M_y(y,\nu) = \sqrt{\frac{2\nu^2}{\pi}} \Phi(y) \exp{\left(-\frac{\nu^2 y^2}{2}\right)},
\end{equation*}
by L'H\^{o}pital's rule and $\lim_{y \to -\infty} \phi(y)/(y \Phi(y)) = -1$, we have
\begin{equation*}
    \lim_{y \to -\infty} \frac{y M(y, \nu)}{\Phi(y)} = \lim_{y \to -\infty} \frac{M(y, \nu)}{\Phi(y) y^{-1}} = \sqrt{\frac{2\nu^2}{\pi}} \lim_{y \to -\infty}  \frac{ \exp{\left(-\frac{\nu^2 y^2}{2}\right)}}{\frac{\phi(y)}{y \Phi(y)} - \frac{1}{y^2}} = 0.
\end{equation*}
This implies that
\begin{align*}
    \lim_{y \to -\infty} \frac{\phi(y) M(y, \nu)}{\Phi(y)^2} & = \lim_{y \to -\infty} \frac{-y M(y, \nu) + M_y(y, \nu)}{2 \Phi(y)} \\
    & = \sqrt{\frac{2\nu^2}{\pi}} \lim_{y \to -\infty} \exp{\left(-\frac{\nu^2 y^2}{2}\right)} \\
    & = 0.
\end{align*}
Thus, $\lim_{y \to -\infty} \tilde{S}_y(y, \nu) = 0$.
The proof is complete.
\end{proof}



\medskip
\subsection*{Proof of Proposition \ref{prop_unique_rat_ext}}
\emph{Proof of Part (i)}:
Fix $\sigma_F = \gamma \sigma_L$, $\gamma > 0$. Let $\xlow, \xup, \xilow, \xiup, \xnlow$, and $\xnup$ be the limits of the six sequences described in (\ref{ext_seq_br}), respectively. By Lemma \ref{lemma_seq_ext}, we know that they must solve (\ref{ext_sys_eqs}). Moreover $0 = \xlow^1 < \xlow \leq \xup < \xup^1 = 1$ and $\xhlow^1 < \xhlow \leq \xhup < \xhup^1$, where $\xhlow^1$ and $\xhup^1$ are the lower and upper dominance bounds in Round 1 for the followers under history $h$. Let $\Xi_L = [0, 1]$ and $\Xi_F^h = [\xhlow^1, \xhup^1]$.

Let $S(y) = \tilde{S}(y, \alpha)$ with slope parameter $\alpha = \gamma/\sqrt{2 + \gamma^2}$.
Since $1 + \gamma^2 (1 + \lambda'(y))$ is positive and bounded for all $y \in \RR$,
the following function is well-defined by Lemma \ref{lemma_s_func}:
\begin{equation*}
    \Lambda(\gamma) = \max_{y \in \RR} \frac{S'(y)}{1 + \gamma^2(1 + \lambda'(y))}.
\end{equation*}
Moreover, $\Lambda(\gamma) > 0$.  Now let
\begin{equation} \label{ext_sufficiency_1} 
     \widehat{\sigma}_L^1(\gamma) = \left(\frac{n-1}{n}\right)  \sqrt{(1+\gamma^2) \Lambda(\gamma)^2}. \tag{A.6}
\end{equation}
We prove this part in two steps. First, we show that if $\sigma_L > \widehat{\sigma}_L^1(\gamma)$
then the following system of equations with $x_L \in \Xi_L$ and $x_h \in \Xi_F^h$ 
\begin{equation} \label{ext_sys_small}
    \begin{cases}
        \pi_L(x_L;x_{\invest}) = 0 \\
        \pi_F^\invest(x_{\invest}; x_L, x_{\invest}) = 0 \\ 
        \pi_F^\notinvest(x_{\notinvest}; x_L,x_{\notinvest}) = 0 \\
    \end{cases} \tag{A.7}
\end{equation}
has a unique solution $(x_L^*, x_\invest^*, x_\notinvest^*)$. Second, we show that there exists $\widehat{\sigma}_L^2(\gamma)$ such that $(x_L^*, x_\invest^*, x_\notinvest^*)$ is also the unique solution to (\ref{ext_sys_eqs}) whenever
\begin{equation} \label{ext_suff_cond}
    \sigma_L > \widehat{\sigma}_L(\gamma) = \max \left\{\widehat{\sigma}_L^1(\gamma), \widehat{\sigma}_L^2(\gamma) \right\}. \tag{A.8}
\end{equation}


\noindent \underline{Step 1}: Assume from now on that $\sigma_L > \widehat{\sigma}_L^1(\gamma)$.
By Equations (\ref{ext_rank_belief}), (\ref{ext_follower_fp_eqn}), and (\ref{app_fol_exp}), we have
\begin{equation} \label{app_fol_fp_eqn}
    \pi_F^\invest(x_\invest; x_L, x_\invest) = x_\invest + \frac{\sigma_F^2}{\sigma} \lambda\left( \frac{x_\invest - x_L}{\sigma}\right) - \frac{n-1}{n}S\left( \frac{x_\invest - x_L}{\sigma} \right). \tag{A.9}
\end{equation}
If $\sigma_L > \widehat{\sigma}_L^1(\gamma)$, then $\sigma_F^2/\sigma^2 = \gamma^2/(1 + \gamma^2)$ gives that
\begin{align*}
    \frac{\partial \pi_F^\invest}{\partial x_\invest} & = 1 + \frac{\gamma^2}{1 + \gamma^2} \lambda'\left( \frac{x_\invest - x_L}{\sigma} \right) - \frac{n-1}{n\sigma} S'\left(\frac{x_\invest - x_L}{\sigma} \right)\\
    & > 1 + \frac{\gamma^2}{1 + \gamma^2} \lambda'\left( \frac{x_\invest - x_L}{\sigma} \right) - \frac{1}{(1 + \gamma^2) \Lambda(\gamma)} S'\left(\frac{x_\invest - x_L}{\sigma} \right) \geq 0.
\end{align*}
This implies that, for any given $x_L$, $\pi_F^\invest(x_\invest; x_L, x_\invest) = 0$ admits a unique solution. 
Similarly, since
\begin{equation*}
    \pi_F^\notinvest (x_\notinvest; x_L, x_\notinvest) = x_\notinvest - \frac{\sigma_F^2}{\sigma} \lambda\left( \frac{ x_L - x_\notinvest }{\sigma}\right) + \frac{n-1}{n}S\left( \frac{x_L - x_\notinvest}{\sigma} \right) - 1,
\end{equation*}
the condition $\sigma_L > \widehat{\sigma}_L^1(\gamma)$ implies that
\begin{align*}
    \frac{\partial \pi_F^\notinvest}{\partial x_\notinvest} & = 1 + \frac{\gamma^2}{1 + \gamma^2} \lambda'\left( \frac{ x_L - x_\notinvest }{\sigma}\right) - \frac{n-1}{n\sigma} S'\left(\frac{x_L - x_\notinvest}{\sigma} \right) \\
    & > 1 + \frac{\gamma^2}{1 + \gamma^2} \lambda'\left( \frac{x_L - x_\notinvest}{\sigma} \right) - \frac{1}{(1 + \gamma^2) \Lambda(\gamma)} S'\left(\frac{x_L - x_\notinvest}{\sigma} \right) \geq 0
\end{align*}
and hence, given $x_L$, $\pi_F^\notinvest (x_\notinvest; x_L, x_\notinvest) = 0$ has a unique solution.
Thus, (\ref{ext_sys_small}) has a unique solution if and only if there is a unique solution to its first two equations.

Since $\partial \pi_F^\invest/\partial x_\invest$ is continuous on $\Xi_F^\invest \times \Xi_L$, the extreme value theorem ensures that there exists $d_\invest > 0$ such that $\partial \pi_F^\invest/\partial x_\invest \geq d_\invest$. Thus, a global implicit function theorem (see, e.g., Lemma 2 of \cite{zhang_ge_2006}) implies that there is a unique function $f: \Xi_L \to \Xi_F^\invest$ such that $\pi_F^\invest(f(x_L); x_L, f(x_L)) = 0$. Moreover, $f \in C^1$ and $f' < 0$. It follows that
\begin{equation*}
    \frac{\dd \pi_L}{\dd x_L} = \frac{\partial \pi_L}{\partial x_L} + \frac{\partial \pi_L}{\partial x_\invest}f'(x_L) > 0.
\end{equation*}
This says that $\pi_L(x_L; f(x_L)) = 0$ has a unique solution. Thus, (\ref{ext_sys_small}) has a unique solution $(x_L^*, x_\invest^*, x_\notinvest^*)$.

\vskip 0.5 \baselineskip
\noindent \underline{Step 2}: We next show that there exists $\widehat{\sigma}_L^2(\gamma)$ such that
$(x_L^*, x_\invest^*, x_\notinvest^*)$ is the unique solution to (\ref{ext_sys_eqs}) if (\ref{ext_suff_cond}) holds.
Note that
\begin{equation*}
    \frac{\partial \pi_L}{\partial x_L} = 1 + \frac{1}{\sigma}\phi\left( \frac{x_\invest - x_L}{\sigma} \right) > 0
\end{equation*}
is continuous on $\Xi_F^\invest \times \Xi_L$, so there exists $d_L > 0$ such that $\partial \pi_L/\partial x_L \geq d_L$. Thus, there exists a global implicit function $g: \Xi_F^\invest \to \Xi_L$ such that $\pi_L(g(x_\invest); x_\invest) = 0$. In addition, we have $g \in C^1$ and $g' > 0$. 
Then the first four equations of (\ref{ext_sys_eqs}) implies that
\begin{equation*}
    \xlow = g(\xilow) = (g \circ f)(\xup) = (g\circ f \circ g) (\xiup) =  (g\circ f \circ g \circ f)(\xlow).
\end{equation*}
Consider $h: \Xi_L \to \Xi_L$ such that $h = g \circ f \circ g \circ f$. Clearly, $x_L^*$ is a fixed point of $h$
(shown in Step 1). We also note that $h(0) > 0$ and $h' > 0$ because $f' < 0$ and $g' > 0$.
Define
\begin{equation*}
    M(\sigma_L, \gamma) = \max_{y \in \RR} - \frac{\gamma^2}{1+\gamma^2} \lambda'(y) + \frac{n-1}{n\sigma_L \sqrt{1 + \gamma^2}} S'(y).
\end{equation*}
We have $M(\sigma_L, \gamma) < 1$ because $\sigma_L > \widehat{\sigma}_L^1(\gamma)$. It follows that
\begin{equation*}
    f' \geq \frac{M(\sigma_L, \gamma)}{M(\sigma_L, \gamma) - 1} = M_f(\sigma_L, \gamma).
\end{equation*}
By the envelope theorem, $\partial M(\sigma_L, \gamma)/\partial \sigma_L < 0$. Thus, $\partial M_f(\sigma_L, \gamma)/\partial \sigma_L > 0$.
By the definition of $g$, we have
\begin{equation*}
    g' \leq \frac{\phi(0)}{\sigma_L \sqrt{1 + \gamma^2} + \phi(0)} = M_g(\sigma_L, \gamma).
\end{equation*}
Moreover, $\partial M_g(\sigma_L, \gamma)/\partial \sigma_L < 0$.
It follows that
\begin{equation} \label{h_prime}
    h' \leq \left[M_g(\sigma_L, \gamma) \cdot M_f(\sigma_L,\gamma)\right]^2. \tag{A.10}
\end{equation}
The right-hand side of (\ref{h_prime}) is strictly decreasing in $\sigma_L$ and approaches zero as $\sigma_L \to \infty$. Thus, there exists $\widehat{\sigma}_L^2(\gamma)$ such that $h' < 1$ if $\sigma_L > \max\{\widehat{\sigma}_L^1(\gamma),  \widehat{\sigma}_L^2(\gamma)\}$ (i.e., (\ref{ext_suff_cond}) holds). It is immediate that $h$ has a unique fixed point when $h' < 1$. The proof of part (i) is complete.   

\medskip
\noindent \emph{Proof of Part (ii)}:
Fix $\sigma_F = \gamma \sigma_L$, $\gamma > 0$. We prove this part in three steps. We first show the existence of a monotone equilibrium characterized by a tuple of thresholds $(x_L^*(\sigma_L, \gamma)$, $x_\invest^*(\sigma_L, \gamma)$, $x_\notinvest^*(\sigma_L, \gamma))$ such that
$a_L = \invest$ if and only if $x_L > x_L^*(\sigma_L, \gamma)$, and $s_j(\invest) = \invest$ if and only if $x_j > x_\invest^*(\sigma_L, \gamma)$ and $s_j(\notinvest) = \invest$ if and only if $x_j > x_\notinvest^*(\sigma_L, \gamma)$ for all followers $j \in F$. We will simply write the thresholds as $(x_L^*, x_\invest^*, x_\notinvest^*)$ if no confusion will arise.
In Step 2, we show that $x_\invest^* < x_L^* < x_\notinvest^*$. We finally show, in Step 3, that...

\vskip 0.5 \baselineskip
\noindent \underline{Step 1}.
Recall that $x_L = g(x_\invest)$ is the solution to $\pi_L(x_L; x_\invest) = 0$. Substituting $g(x_\invest)$ into Equation (\ref{app_fol_fp_eqn}) yields
\begin{equation*}
    \pi_F^\invest(x_\invest; g(x_\invest), x_\invest) = x_\invest + \frac{\sigma_F^2}{\sigma} \lambda\left( \frac{x_\invest - g(x_\invest)}{\sigma}\right) - \frac{n-1}{n}S\left( \frac{x_\invest - g(x_\invest)}{\sigma} \right).
\end{equation*}
Since $g(x_\invest) \in (0, 1)$, $\pi_F^\invest(x_\invest; g(x_\invest), x_\invest)\to -\infty$ as $x_\invest \to -\infty$, and $\pi_F^\invest(x_\invest; g(x_\invest), x_\invest)$ $\to \infty$ as $x \to \infty$ by Lemma \ref{lemma_payoff_ext}. Thus, by continuity, there must exists $x_\invest^*$ such that $\pi_F^\invest(x_\invest^*;$ $ g(x_\invest^*), x_\invest^*) = 0$. Let $x_L^* = g(x_\invest^*)$. It follows, by Lemma \ref{lemma_payoff_ext}, that there exists $x_\notinvest^*$ such that $\pi_F^\notinvest(x_\notinvest^*; g(x_\invest^*), x_\notinvest^*) = 0$.

\vskip 0.5 \baselineskip
\noindent \underline{Step 2}.
We now prove that $x_\invest^* < x_L^*$. By way of contradiction, assume $x_\invest^* \geq x_L^*$. Then we have $(x_\invest^* - x_L^*)/\sigma \geq 0$ and hence $x_L^* = \Phi((x_\invest^* - x_L^*)/\sigma) \geq 1/2$. It follows that
\begin{align*}
    x_\invest^* + \frac{\sigma_F^2}{\sigma} \lambda \left( \frac{x_\invest^* - x_L^*}{\sigma}  \right) - \frac{n-1}{n} S \left( \frac{x_\invest^* - x_L^*}{\sigma} \right) 
    \geq x_L^* - \frac{n-1}{2n} \geq \frac{1}{2n} > 0,
\end{align*}
which leads to a contradiction. The first inequality is due to $\lambda > 0$, $S < 1/2$, and the assumption that $x_\invest^* \geq x_L^*$. The second inequality is given by $x_L^* \geq 1/2$. Thus, it must be that $x_\invest^* < x_L^*$.

% In a similar vein, one can get another contradiction by assuming $x_L^* \geq x_\notinvest^*$. That is,
% \begin{align*}
%     x_\notinvest^* - \frac{\sigma_F^2}{\sigma} \left( \frac{x_L^* - x_\notinvest^*}{\sigma} \right) + \frac{n-1}{n} S \left( \frac{x_L^* - x_\notinvest^*}{\sigma} \right) - 1 \leq x_L^* + \frac{n-1}{2n} - 1 < - \frac{1}{2n} < 0. 
% \end{align*}
% The second inequality is because $x_\invest^* < x_L^*$ implies that $x_L^* < 1/2$. Thus, $x_L^* < x_\notinvest^*$.

\vskip 0.5 \baselineskip
\noindent \underline{Step 3}.
Let $\widecheck{x}_L^*$ and $\widecheck{x}_h^*$ be the limits as $\sigma_L \to 0$ while keeping the ratio $\sigma_F/\sigma_L = \gamma$ fixed.
Step 2 implies that $(x_\invest^* - x_L^*)/\sigma$ can approach either a constant $k \leq 0$ or $-\infty$ as $\sigma_L \to 0$ . We show in both cases, $\widecheck{x}_L^*$ is strictly less than $(n-1)/2n$ and so is $\widecheck{x}_\invest^*$.
In the former case, $x_L^*$ and $x_\invest^*$ must have the same limit; otherwise $(x_\invest^* - x_L^*)/\sigma \to -\infty$ leading to a contradiction. But since
\begin{equation*}
    \frac{\sigma_F^2}{\sigma} \lambda \left( \frac{x_\invest^* - x_L^*}{\sigma} \right) - \frac{n-1}{n} S \left( \frac{x_\invest^* - x_L^*}{\sigma} \right) \to 0 \cdot \lambda(k) - \frac{n-1}{n} S(k) = - \frac{n-1}{n}S(k),
\end{equation*}
we have $\widecheck{x}_\invest^* = ((n-1)/n)S(k) < (n-1)/2n$ by Lemma \ref{lemma_s_func}, and so does $\widecheck{x}_L^*$. In the latter case, $\widecheck{x}_L^* = 0 < (n-1)/2n$ because $\Phi((x_\invest^* - x_L^*)/\sigma) \to 0$.



Now consider a function $\widehat{x}_\invest(\sigma_L)$ that approaches $(n-1)/2n$ as $\sigma_L \to 0$. Observe that
\begin{equation*}
    \widehat{x}_\invest(\sigma_L) + \frac{\sigma_F^2}{\sigma} \lambda \left( \frac{\widehat{x}_\invest(\sigma_L) - x_L^*}{\sigma} \right) - \frac{n-1}{n} S \left( \frac{\widehat{x}_\invest(\sigma_L) - x_L^*}{\sigma} \right) \to 0
\end{equation*}
because $(\widehat{x}_\invest(\sigma_L) - x_L^*)/\sigma \to \infty$. This means that $(n-1)/2n$ is a solution to $\pi_F^\invest(x_\invest; $ $\widecheck{x}_L^*, x_\invest) = 0$. It implies further that $g(\widehat{x}_\invest(\sigma_L)) \to 1$ as $\sigma_L \to 0$.
Therefore, both actions are rationalizable for leader types $x_L \in (\widecheck{x}_L^*,1)$ and for follower types $x \in (\widecheck{x}_\invest^*, (n-1)/2n)$ under $h = \invest$. The proof is complete. \qed











\newpage

\section*{Appendix B: Unique Monotone Equilibrium}
In this appendix, we show that the game has a unique equilibrium when we restrict attention to monotone strategies. Although the result is standard in the literature, it is included for completeness.

Let $s_L: \Theta \to \Action_L$ be the leader's strategy. 
The leader is said to follow a \emph{monotone strategy} if her strategy takes the form:
\begin{equation*}
    s_L(\theta) = \begin{cases}
        \invest & \text{if $\theta > \thetazerohat$} \\
        \notinvest & \text{if $\theta \leq \thetazerohat$}
    \end{cases}.
\end{equation*}
A strategy for any follower $j$ is a mapping $s_j: X_j \times \Action_L \to \Action_j$. Follower $j$'s strategy is monotone if
\begin{equation*}
    s_j(x_j, h) = \begin{cases}
        \invest & \text{if $x_j > \xhhat$} \\
        \notinvest & \text{if $x_j \leq \xhhat$}
    \end{cases}.
\end{equation*}
A \emph{monotone equilibrium} is a symmetric perfect Bayesian equilibrium in monotone strategies with thresholds $(\theta_L^*, x_\invest^*, x_\notinvest^*)$.




\begin{lemma} \label{lemma_monotone_b}
There exists a monotone equilibrium with thresholds $\thetazerostar = 0$, $\xistar = -\infty$, and $\xnstar = \infty$.
\end{lemma}

\begin{proof}
    Fix a follower type $x$. Suppose that the leader uses threshold $\thetazerostar = 0$ and other followers use thresholds $\xistar = -\infty$ and $\xnstar = \infty$. If the leader exerts effort, then type $x$'s payoff yields
\begin{equation*}
    \pi_F^\invest(x; \thetazerostar, \xistar) = \EE_{\theta \sim \Psi^\invest(\cdot; \, x, 0)} \left[\theta\right] > 0.
\end{equation*}
This means that all types $x$ will exert effort under history $h = \invest$. Thus, follower $j$'s best response is a monotone strategy with threshold $\xistar = -\infty$. In contrast, if the leader does not exert effort, then the payoff for type $x$ is
\begin{equation*}
    \pi_F^\notinvest(x; \thetazerostar, \xnstar) = \EE_{\theta \sim \Psi^\notinvest(\cdot; \,x, 0)} [\theta] - 1< 0.
\end{equation*}
Thus, under history, $h = \notinvest$, follower $j$ will best respond by using a monotone strategy with threshold $\xnstar = \infty$.

Consider now type $\theta$ of the leader. Since all followers will invest if they see the leader invests, investing generates a payoff of $\theta$ for type $\theta$. Therefore, type $\theta$ invests if and only if $\theta > 0$. In other words, the leader will best respond by choosing threshold $\thetazerostar = 0$. The proof is complete.
\end{proof}


\begin{proposition}
There is no monotone equilibrium other than the one given in Lemma \ref{lemma_monotone_b}.

\end{proposition}
\begin{proof}
By way of contradiction, suppose that $\thetazerostar$ and $\xistar$ are the equilibrium thresholds. Then they must solve the indifference conditions
\begin{equation} \label{app_b_leader_eq}
    \pi_L(\thetazerostar; \xistar) = \thetazerostar - \Phi\left(\frac{\xistar - \thetazerostar}{\sigma_F}\right) = 0 \tag{B.1}
\end{equation}
and
\begin{equation} \label{app_b_follower_eq}
    \pi_F^\invest(x_\invest^*; \theta_L^*, x_\invest^*) =  \EE_{\theta \sim \Psi^\invest(\cdot; \,\xistar, \thetazerostar)} \left[ \theta - \frac{n-1}{n}\Phi\left(\frac{\xistar - \thetazerostar}{\sigma_F}\right) \right] = 0. \tag{B.2}
\end{equation}
By Equations (\ref{rank_belief}) and (\ref{eq_eqm_follower}) we can write (\ref{app_b_follower_eq}) as
\begin{equation} \label{app_b_follower_eq_2}
    \xistar + \sigma_F \lambda\left(\frac{\xistar - \thetazerostar}{\sigma_F}\right) = \frac{n-1}{2n}\Phi\left(\frac{\xistar - \thetazerostar}{\sigma_F}\right). \tag{B.3}
\end{equation}
Subtracting Equation (\ref{app_b_leader_eq}) from Equation (\ref{app_b_follower_eq_2}) yields
\begin{equation} \label{app_eq_diff}
    \xistar - \thetazerostar + \sigma_F \lambda\left( \frac{\xistar - \thetazerostar}{\sigma_F} \right) = - \frac{n+1}{2n} \Phi\left(\frac{\xistar - \thetazerostar}{\sigma_F}\right). \tag{B.4}
\end{equation}
Note that $x + \lambda(x)$ is increasing in $x$ with $\lim_{x \to -\infty} x + \lambda(x) = 0$, and hence $x + \lambda(x) > 0$ for all $x$. This implies that the left-hand side of (\ref{app_eq_diff}) is positive. But since the right-hand side of (\ref{app_eq_diff}) is negative, this leads to a contradiction.
\end{proof}


\newpage
\section*{Appendix C: Log-concave Noises}

In this appendix, we extend the main model in Section 3 by considering noises that have log-concave densities. Formally, $(\varepsilon_i)_{i \in N}$ are independently drawn from distribution $F$, which has a positive continuous density $f$ on the entire real line. We assume, in addition, that $f$ is strictly log-concave and symmetric about zero. Common distributions, such as Gaussian, Laplace, and logistic distributions with mean zero, satisfy these assumptions.

Suppose that the leader uses a monotone strategy with threshold $z \in \RR$. Then a follower with type $x$ has the following posterior density about $\theta$:
\begin{equation*}
    \psi^h(\theta; x, z) = \begin{cases}
    \frac{\frac{1}{\sigma_F}f\left(\frac{x - \theta}{\sigma_F}\right)}{F\left(\frac{x - z}{\sigma_F}\right)}\one(\theta > z) & \text{if $h = \invest$} \\
    ~ & ~ \\
    \frac{\frac{1}{\sigma_F}f\left(\frac{x - \theta}{\sigma_F}\right)}{1 - F\left(\frac{x - z}{\sigma_F}\right)}\one(\theta \leq z) & \text{if $h = \notinvest$}
    \end{cases}.
\end{equation*}
Let $\Psi^h(\cdot; \,x, z)$ be the corresponding CDF.

\begin{lemma} \label{app_c_fosd}
$\Psi^h(\cdot; \,x, z)$ is strictly increasing in $x$ and $z$ in the sense of strict first-order stochastic dominance.
\end{lemma}
\begin{proof}
We only prove the case $h = \invest$, for the proof of the other case, is similar.
Fix $z$ and $\theta > z$, and let $x < x'$. Note that
\begin{equation*}
    \frac{ \psi^\invest (\theta; \,x', z) }{ \psi^\invest (\theta; \,x', z) } = \frac{ F\left(\frac{x - z}{\sigma_F}\right) }{ F\left(\frac{x' - z}{\sigma_F}\right) } \cdot \frac{ f\left(\frac{x' - \theta}{\sigma_F}\right) }{ f\left(\frac{x - \theta}{\sigma_F}\right) }.
\end{equation*}
Since $f$ is logconcave, $f((x'-\theta)/\sigma_F)/f((x-\theta)/\sigma_F)$ is weakly increasing in $\theta$; that is, $\Psi^\invest(\cdot; \,x', z)$ dominates $\Psi^\invest(\cdot; \,x, z)$ in the monotone likelihood ratio order. This follows from the equivalence between log concavity and P\'{o}lya Frequency of order 2 (See, for example, Proposition 1 in \cite{an_1998} or Proposition 2.3 in \cite{saumard_wellner_2014}). 
Thus, $\Psi^\invest(\cdot; \,x', z)$ first-order stochastically dominates $\Psi^\invest(\cdot; \,x, z)$.

The claim that $\Psi^\invest(\cdot; \,x, z)$ is strictly increasing in the first-order stochastic dominance sense is because
\begin{equation*}
    \Psi^\invest(\theta; \,x, z) = \frac{1}{ F\left(\frac{x-z}{\sigma_F}\right)  }\int_z^\theta \frac{1}{\sigma_F} f\left(\frac{x-t}{\sigma_F}\right)  \dd t = 1 - \frac{ F\left(\frac{x-\theta}{\sigma_F}\right) }{ F\left(\frac{x-z}{\sigma_F}\right) }
\end{equation*}
is strictly increasing in $z$.
\end{proof}



Let $\eta = \lim_{x \to -\infty} F(x)/f(x)$. It is worth noting that $\eta$ is the scale parameter if the noises follow a Laplace or logistic distribution.
\begin{lemma} \label{app_c_exp}
The posterior expectations $\EE_{\theta \sim \Psi^h(\cdot; \,x, z)}[\theta]$ are strictly increasing in $x$ and $z$. Moreover,
\begin{equation*}
    \lim_{x \to -\infty} \EE_{\theta \sim \Psi^h(\cdot; \,x, z)}[\theta] = \begin{cases}
    \sigma_F \eta + z & \text{if $h = \invest$} \\
    - \infty & \text{if $h = \notinvest$}
    \end{cases},
\end{equation*}
and 
\begin{equation*}
    \lim_{x \to \infty} \EE_{\theta \sim \Psi^h(\cdot ; \,x, z)}[\theta] = \begin{cases}
    \infty & \text{if $h = \invest$} \\
    -\sigma_F \eta + z & \text{if $h = \notinvest$}
    \end{cases}.
\end{equation*}
\end{lemma}
\begin{proof}
(Part 1) The first part of this lemma is given by Lemma \ref{app_c_fosd}.

\noindent (Part 2) Consider first history $h = \invest$.
By a change of variable, we have
\begin{equation*}
    \EE_{\theta \sim \Psi^\invest(\cdot; \,x, z)}[\theta]  = \frac{1 }{F\left(\frac{x - z}{\sigma_F}\right)} \int_z^\infty  \frac{t}{\sigma_F} f\left(\frac{x - t}{\sigma_F}\right) \dd t = \sigma_F \delta\left( \frac{x - z}{\sigma_F} \right) + z.
\end{equation*}
where
\begin{equation*}
    \delta(u) = u - \frac{ \int_{-\infty}^u t f(t) \dd t }{ F(u) }.
\end{equation*}
It is clear that $\lim_{u \to \infty} \delta(u) = \infty$, and hence
\begin{equation*}
    \lim_{x \to \infty} \EE_{\theta \sim \Psi^h(\cdot ; \,x, z)}[\theta] = \infty.
\end{equation*}
Since $f$ is log-concave, it has a light right tail; that is,
\begin{equation*}
    \lim_{x \to \infty} \frac{ f(x) }{ \ee^{-cx} } = 0
\end{equation*}
for some $c > 0$.\footnote{~See, for example, Corallary 1 in \cite{an_1998}.} Thus, by symmetry (about zero),
\begin{equation*}
    \lim_{x \to -\infty} x F(x) = \lim_{x \to -\infty} \frac{ F(x) }{ x^{-1} } = \lim_{x \to -\infty} \frac{ f(x) }{ -x^{-2} } = \lim_{x \to -\infty} \frac{ f(x) }{ \ee^{cx} } \cdot \frac{ \ee^{cx} }{ -x^{-2} } = 0.
\end{equation*} 
Integrating by parts now gives 
\begin{equation*}
    \delta(u) = \frac{ \int_{-\infty}^u F(t) \dd t }{ F(u) }.
\end{equation*}
It follows that $\lim_{u \to -\infty} \delta(u) = \eta$. Thus,
\begin{equation*}
    \lim_{x \to -\infty} \EE_{\theta \sim \Psi^h(\cdot ; \,x, z)}[\theta] = \sigma_F \eta + z.
\end{equation*}

The proof for history $h = \notinvest$ is similar.
We have
\begin{equation*}
    \EE_{\theta \sim \Psi^\notinvest(\cdot; \,x, z)}[\theta] = \frac{1}{ 1 - F\left(\frac{x - z}{\sigma_F}\right) } \int_{-\infty}^z \frac{t}{\sigma_F} f\left( \frac{x - t}{\sigma_F} \right) \dd t = -\sigma_F \varsigma \left(\frac{x - z}{\sigma_F} \right) + z,
\end{equation*}
where
\begin{equation*}
    \varsigma(u) = \frac{ \int_u^\infty t f(t) \dd t }{ 1 - F(u) } - u.
\end{equation*}
Thus, $\lim_{u \to -\infty} \varsigma(u) = \infty$, and
\begin{equation*}
    \lim_{x \to -\infty} \EE_{\theta \sim \Psi^\notinvest(\cdot; \,x, z)}[\theta] = -\infty.
\end{equation*}
Since $\lim_{u \to \infty} u \left( 1 - F(u) \right) = 0$ (because $f$ is light-tailed),
\begin{equation*}
    \varsigma(u) = \frac{\int_u^\infty [1 - F(t)] \dd t}{1 - F(u)}
\end{equation*}
by integration by parts. Thus,
\begin{equation*}
    \lim_{u \to \infty} \varsigma(u) = \lim_{u \to \infty} \frac{1 - F(u)}{f(u)} = \lim_{u \to \infty} \frac{ F(-u) }{ f(-u) } = \eta,
\end{equation*}
which implies that
\begin{equation*}
    \lim_{x \to \infty} \EE_{\theta \sim \Psi^\notinvest(\cdot; \,x, z)}[\theta] = -\sigma_F \eta + z.
\end{equation*}
The proof is complete.
\end{proof}

Suppose, in addition, that a follower with type $x$ believes that other followers use monotone strategies with threshold $x_h$ under history $h$, then his payoff under history $h$
\begin{equation*}
    \pi_F^h(x; z, x_h) = \EE_{\theta \sim \Psi^h(\cdot; \,x, z)} \left[ \theta - \frac{n-1}{n}F\left( \frac{x_h - \theta}{\sigma_F} \right) \right] - \frac{\chi_\notinvest}{n}
\end{equation*}
has the following properties:


\begin{lemma} \label{app_c_fol_payoff}
Type $x$'s payoffs $\pi_F^h(x; z, x_h)$ are strictly increasing in $x$ and $z$ but is strictly decreasing in $x_h$. Moreover,
\begin{equation*}
   \lim_{x \to -\infty} \pi_F^h(x; z, x_h) = \begin{cases}
        \sigma_F \eta + z - \frac{n-1}{n}F\left( \frac{x_\invest - z}{\sigma_F} \right) & \text{if $h = \invest$} \\
        -\infty & \text{if $h = \notinvest$}
    \end{cases}
\end{equation*}
and
\begin{equation*}
    \lim_{x \to \infty} \pi_F^h(x; z, x_h) = \begin{cases}
        \infty & \text{if $h = \invest$} \\
        -\sigma_F \eta + z - \frac{1}{n} - \frac{n-1}{n}F\left( \frac{x_\notinvest - z}{\sigma_F} \right) & \text{if $h = \notinvest$}
    \end{cases}
\end{equation*}
\end{lemma}
\begin{proof}
(Part 1) Note that $\theta - ((n-1)/n)F((x_h - \theta)/\sigma_F)$ is strictly increasing in $\theta$; therefore Lemma \ref{app_c_fosd} implies that $\pi_F^h(x; z, x_h)$ is strictly increasing in $x$ and $z$. But since $\theta - ((n-1)/n)F((x_h - \theta)/\sigma_F)$ is strictly decreasing in $x_h$, so is $\pi_F^h(x; z, x_h)$.

(Part 2)
It follows immediately from Lemma \ref{app_c_exp} that $\pi_F^\invest(x; z, x_\invest) \to \infty$ as $x \to \infty$
and $\pi_F^\notinvest(x; z, x_\notinvest) \to -\infty$ as $x \to -\infty$
because $F((x_h - \theta)/\sigma_F)$ is bounded.



Now under $h = \invest$, a change of variable $u = (x - \theta)/\sigma_F$ gives that
\begin{align*}
    \EE_{\theta \sim \Psi^\invest(\cdot;\,x, z)}\left[ F\left(\frac{x_\invest - \theta}{\sigma_F}\right)\right] & = \frac{1}{F\left(\frac{x - z}{\sigma_F}\right)} \int_z^\infty F\left(\frac{x_\invest - \theta}{\sigma_F}\right) \frac{1}{\sigma_F}f\left(\frac{x - \theta}{\sigma_F}\right) \dd \theta \\
    & = \frac{1}{F\left(\frac{x - z}{\sigma_F}\right)} \int_{-\infty}^\frac{x-z}{\sigma_F} f(u) F\left(\frac{x_\invest - x}{\sigma_F} + u \right) \dd u.
\end{align*}
By L'H\^{o}pital's rule, the above expectation has the following limit:
\begin{equation*}
    \lim_{x \to -\infty} \EE_{\theta \sim \Psi^\invest(\cdot;\,x, z)}\left[ F\left(\frac{x_\invest - \theta}{\sigma_F}\right)\right] = F \left( \frac{x_\invest - z}{\sigma_F} \right).
\end{equation*}
Thus, by Lemma \ref{app_c_exp}, we can conclude that
\begin{equation*}
    \lim_{x \to -\infty} \pi_F^\invest(x; z, x_\invest) = \sigma_F \eta + z - \frac{n-1}{n} F \left( \frac{x_\invest - z}{\sigma_F} \right).
\end{equation*}
Similarly, under $h = \notinvest$,
\begin{align*}
    \lim_{x \to \infty} \EE_{\theta \sim \Psi^\notinvest(\cdot;\,x, z)}\left[ F\left(\frac{x_\notinvest - \theta}{\sigma_F}\right)\right] & = \lim_{x \to \infty} \frac{1}{F\left( \frac{z-x}{\sigma_F} \right)} \int_{-\infty}^z F \left( \frac{x_\notinvest - \theta}{\sigma_F} \right) \frac{1}{\sigma_F} f\left( \frac{x-\theta}{\sigma_F} \right) \dd \theta \\
    & = \lim_{x \to \infty} \frac{1}{F\left( \frac{z-x}{\sigma_F} \right)} \int_{\frac{x-z}{\sigma_F}}^\infty f(u) F \left( \frac{x_\notinvest - x}{\sigma_F} + u \right) \dd u \\
    & = F \left( \frac{x_\notinvest - z}{\sigma_F} \right).
\end{align*}
Thus, 
\begin{equation*}
    \lim_{x \to \infty} \pi_F^\notinvest (x; z, x_\notinvest) = -\sigma_F \eta + z - \frac{1}{n} - \frac{n-1}{n} F \left( \frac{x_\notinvest - z}{\sigma_F} \right),
\end{equation*}
as desired.
\end{proof}

Let $(\thetalow^k, \thetaup^k, \xiup^k, \xilow^k, \xnup^k, \xnlow^k)_{k=0}^\infty$ be the six $\Delta$-rationalizable sequences, where $\thetalow^0 = \xilow^0 = \xnlow^0 = -\infty$ and $\thetaup^0 = \xiup^0 = \xnup^0 = \infty$.
Let $\thetalow$, $\thetaup$, $\xilow$, $\xiup$, $\xnlow$, $\xnup$ be their limits, respectively, as $k \to \infty$. If one sequence diverges, then its limit is either $-\infty$ or $\infty$. 

\begin{proposition} \label{app_c_rat_seq}
    If $\sigma_F \geq (n\eta)^{-1}(n-1)$, then the above sequences have the following properties: \\
    (a) $(\thetalow^k)_{k=0}^\infty$ is such that $\thetalow^k = \thetalow = 0$ for all $k \geq 1$; \\
    (b) $(\thetaup^k)_{k=0}^\infty$ is decreasing and  such that $\thetaup^k = \thetaup = 0$ for all $k \geq 2$; \\
    (c) $(\xilow^k)_{k=0}^\infty$ is such that $\xilow^k = \xilow = -\infty$ for all $k \geq 0$; \\
    (d) $(\xiup^k)_{k=0}^\infty$ is decreasing and such that $\xiup^k = \xiup -\infty$ for all $k \geq 1$; \\
    (e) $(\xnlow^k)_{k=0}^\infty$ is increasing and such that $\xnlow^k = \xnlow = \infty$ for all $k \geq 1$; \\
    (f) $(\xnup^k)_{k=0}^\infty$ is such that $\xnup^k = \xnup = \infty$ for all $k \geq 0$.
\end{proposition}
\begin{proof}
\textit{Round $k = 1$}: First note that the best-case and worst-case payoffs to leader $\theta$ are given by
\[
\pi_L(\theta; \xilow^0) = \theta - F \left( \frac{\xilow^0 - \theta}{\sigma_F} \right) = \theta
\]
and 
\[
\pi_L(\theta; \xiup^0) = \theta - F \left( \frac{\xiup^0 - \theta}{\sigma_F} \right) = \theta -1,
\]
respectively.
This implies that $\thetalow^1 = 0$ and $\thetaup^1 = 1$. That is, it is dominant for all leader types below $\thetalow^1$ to take $a_L = \notinvest$ and dominant for all leader types above $\thetaup^1$ to choose $a_L = \invest$.

Given $\thetalow^1$ and $\thetaup^1$. By Lemma \ref{app_c_fol_payoff}, follower $x$'s best-case payoff under $h = \invest$ is
\[
\pi_F^\invest(x; \thetaup^1, \xilow^0) = \EE_{\theta \sim \Psi^\invest(\cdot; x, \thetaup^1)}\left[\theta - \frac{n-1}{n} F\left( \frac{\xilow^0 - \theta}{\sigma_F} \right) \right] = \EE_{\theta \sim \Psi^\invest(\cdot; x, \thetaup^1)} [\theta].
\]
Since $\lim_{x \to -\infty} \EE_{\theta \sim \Psi^\invest(\cdot; x, \thetaup^1)} [\theta] = \sigma_F \eta + 1 > 0$ by Lemma \ref{app_c_exp}, we have $\xilow^1 = -\infty$, meaning that there is no follower type for whom action $\invest$ is strictly dominated. Analogously, the worst-case payoff to follower $x$ yields
\begin{align*}
  \pi_F^\invest(x; \thetalow^1, \xiup^0) & = \EE_{\theta \sim \Psi^\invest(\cdot; x, \thetalow^1)}\left[\theta - \frac{n-1}{n} F\left( \frac{\xiup^0 - \theta}{\sigma_F} \right) \right] \\
  & = \EE_{\theta \sim \Psi^\invest(\cdot; x, \thetalow^1)} [\theta] - \frac{n-1}{n},
\end{align*}
which has the limit of $\sigma_F \eta - (n-1)/n$ as $x \to -\infty$. But since $\sigma_F \geq (n-1)/(n\eta)$, it follows that
\[
\lim_{x \to -\infty} \pi_F^\invest(x; \thetalow^1, \xiup^0) = \sigma_F \eta - \frac{n-1}{n} \geq 0.
\]
Thus, $\xiup^1 = -\infty$ and it is dominant for all follower types to take action $\invest$.

Under $h = \notinvest$, the worst-case payoff to follower $x$ is
\begin{align*}
    \pi_F^\notinvest(x; \thetalow^1, \xnup^0) & = \EE_{\theta \sim \Psi^\notinvest(\cdot; x, \thetalow^1)} \left[\theta - \frac{n-1}{n} F\left( \frac{\xnup^0 - \theta}{\sigma_F} \right) \right] - \frac{1}{n} \\ 
    & = \EE_{\theta \sim \Psi^\notinvest(\cdot; x, \thetalow^1)} [\theta] - 1.
\end{align*}
By Lemma \ref{app_c_exp}, we have
\[
\lim_{x \to \infty} \pi_F^\notinvest(x; \thetalow^1, \xnup^0) = - \sigma_F \eta - 1 < 0.
\]
Therefore $\xnup^1 = \infty$; i.e., there exists no follower type for whom $\invest$ is a dominant action. Follower $x$'s best-case payoff exhibits 
\begin{align*}
    \pi_F^\notinvest(x; \thetaup^1, \xnlow^0) & = \EE_{\theta \sim \Psi^\notinvest(\cdot; x, \thetaup^1)} \left[\theta - \frac{n-1}{n} F\left( \frac{\xnlow^0 - \theta}{\sigma_F} \right) \right] - \frac{1}{n} \\
    & = \EE_{\theta \sim \Psi^\notinvest(\cdot; x, \thetaup^1)} [\theta] - \frac{1}{n}
\end{align*}
and
\begin{align*}
    \lim_{x \to \infty} \pi_F^\notinvest(x; \thetaup^1, \xnlow^0) = -\sigma_F \eta + 1 - \frac{1}{n} \leq -\left(\frac{n-1}{n \eta}\right) \eta + \frac{n-1}{n} = 0. 
\end{align*}
Thus we have $\xnlow^1 = \infty$; i.e., $\notinvest$ is a dominant action for all follower types.

\textit{Round $k = 2$}: Given $\xilow^1 = \xiup^1 = -\infty$, leader $\theta$'s worst-case and best-case payoffs coincide:
\begin{equation*}
    \pi_L(\theta; \xiup^1) = \theta - F \left( \frac{\xiup^1 - \theta}{\sigma_F} \right) = \theta.
\end{equation*}
This immediately implies that $\thetalow^2 = \thetaup^2 = 0$.

Given $\thetalow^2$ and $\thetaup^2$, follower $x$'s worst-case and best-case payoffs, under history $h = \invest$, also coincide because $\thetalow^2 = \thetaup^2 = 0$ and $\xilow^1 = \xiup^1 = -\infty$. Moreover, by Lemma \ref{app_c_exp},
\begin{equation*}
    \lim_{x \to -\infty} \pi_F^\invest(x; \thetaup^2, \xilow^2) = \sigma_F \eta > 0.
\end{equation*}
Thus, $\xilow^2 = \xiup^2 = -\infty$. One can show analogously that $\xnlow^2 = \xnup^2 = \infty$ because $\thetalow^2 = \thetaup^2 = 0$ and $\xnlow^1 = \xnup^1 = \infty$.

Now we can conclude that (a)-(f) hold true via a simple induction argument.
\end{proof}

\begin{corollary}
If $\sigma_F \geq (n\eta)^{-1}(n-1)$, then the game has a unique $\Delta$-rationalizable strategy profile in which all followers imitate the leader's choice of action.
\end{corollary}

Proposition \ref{app_c_rat_seq} says that sequences $(\thetalow^k)_{k=0}^\infty$, $(\xilow^k)_{k=0}^\infty$, and $(\xnup^k)_{k=0}^\infty$ always converge. However, it is 
worth noting that we have provided a strong sufficient condition so that a unique $\Delta$-rationalizable behavior is achieved in Round 2 (i.e., 
the other three sequences to converge in Round 2). In fact, 
if the following condition holds
\begin{equation*}
    \max\left\{ \iota_\invest^{k}(\sigma_F), \iota_\notinvest^{k}(\sigma_F) \right\} \leq \sigma_F < \min \left\{ \iota_\invest^{k-1}(\sigma_F), \iota_\notinvest^{k-1}(\sigma_F)  \right\},
\end{equation*}
where
\begin{equation*}
\iota_\invest^k(\sigma_F) = \frac{n-1}{n\eta} F\left( \frac{\xiup^{k-1}}{\sigma_F} \right)
\end{equation*}
and
\begin{equation*}
\iota_\notinvest^k(\sigma_F) = \frac{1}{\eta}\left[ \thetaup^k - \frac{1}{n} - \frac{n-1}{n} F \left( \frac{\xnlow^{k-1} - \thetaup^k}{\sigma_F} \right) \right],
\end{equation*}
then the six sequences will converge to the unique $\Delta$-rationalizable profile in Round $k \geq 2$.
The next proposition shows that there exist multiple $\Delta$-rationalizable profiles when $\sigma_F \to 0$.

\begin{proposition}
    In the limit as $\sigma_F \to 0$, we have $\thetaup \to (n-1)/(2n)$, $\xiup \to (n-1)/(2n)$, and $\xnlow \to \infty$.
\end{proposition}
\begin{proof}
    The proof is identical to that of Proposition \ref{prop_limit}.
\end{proof}


\end{document}
