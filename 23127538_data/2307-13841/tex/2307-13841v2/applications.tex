\section{Implications for Applications}
Our results have important implications for applications of the global games framework. In reality, most phenomena are dynamic in nature and for some, it can be the case that there exists a "special" player, the leader, whose actions may have payoff externalities as well as an informational role. In this case, we have identified conditions that can guarantee the uniqueness of rationalizable play, thus making easier the goal of analysts to perform comparative statics analysis. We have also illustrated the sensitivity of such predictions to the specific details of the information structures assumed. 

Consider for example a currency attack game, in which there is a large player, in the spirit of \cite{corsetti_et_al_2004}, who acts first. Assume that the large player perfectly knows the state as in our baseline model. As we have shown, uniqueness of rationalizable play is not guaranteed, unless the variance in the noise of the small players' signals is sufficiently high. If this is the case, then the qualitative predictions do not change. If, however, this is not true, then in principle, one cannot rule out non monotone equilibrium play.  \cmmt{more tuff here}

