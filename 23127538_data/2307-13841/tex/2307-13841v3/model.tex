\section{The Model} \label{sect_model}

Consider the following two-stage game. There is a \textit{leader} (she), $L$, and a team of $n \geq 2$ \textit{followers} (he). Each player $i \in N \equiv \{L, 1, \dots, n\}$ has to decide whether to take an action ($a_i = \invest$) or not ($a_i = \notinvest$). This action can be interpreted as exerting costly effort, investing into a new project, or attacking a regime or currency. The cost of taking the action is $c > 0$. 
Not taking the action is a safe option (e.g., shirking or staying with the existing technology) with benefits normalized to zero. Let $\Action_i = \{\invest, \notinvest\}$ be the set of actions for player $i \in N$. To fix ideas, we will henceforth refer to the action as the exertion of effort.

The leader moves first in stage 1. The followers, $j \in F = \{1, \dots, n\}$, make decisions in stage 2 after observing the leader's action. Our game thus constitutes a multi-stage game with observable actions.
We assume that the leader's action is irreversible; that is, she can neither ``exit'' nor ``delay'', which may be understood as the commitment made by the leader or the consequence of a high reputation cost of exit or delay.\footnote{~We borrow the terms from \cite{kovac_steiner_2013}.} Thus, our leader has maximum credibility.


We consider the situation where coordination exhibits positive externalities; that is, the more people choose to exert effort the more benefit everybody accrues. In other words, players'  decisions are strategic complements. Let $\tilde{b}(\theta, a_{-i})$ be an additively separable benefit function for player $i$ when action $a_i = \invest$ is taken (otherwise it is zero). The parameter $\theta \in \Theta = \RR$ is a payoff-relevant state (``fundamentals'') that affects the gross return. The state $\theta$ can be though of as a parametrization of the environment in which an organization operates, the strength of a regime, or the solvency of a bank.  Let $a_{-i} = (a_k)_{k \in N; \,k \neq i}$ be the vector of other players' actions that generates positive \emph{spillover benefits} when at least one other player chooses to exert effort. Thus, player $i$'s payoff is given by $u_i = \tilde{b}(\theta, a_{-i}) - c$. 
We further assume that $\tilde{b}$ is strictly increasing in both $\theta$ and $a_{-i}$ and 
is symmetric in $a_{-i}$. Let $\one(\cdot)$ be the indicator function and $A_{-i} = n^{-1}\sum_{k \neq i,~k \in N} \one(a_k = \invest)$ be the proportion of other players choosing to exert effort. Then we could write player $i$'s payoff as\footnote{~Suppose that $\tilde{b}(\theta, a_{-i}) = v(\theta) + \sum_{k \neq i}w(a_k)$, where $v$ is an increasing function and $w$ is such that $w(0) = 0$ and $w(d_k) = \omega > 0$. Then player $i$'s payoff is $\tilde{u}(\theta, a_{-i}) = \tilde{b}(\theta, a_{-i}) - c = v(\theta) + \omega \sum_{k \neq i}\one(a_k = \invest) - c$. A monotone transformation, $u = \alpha \tilde{u} + \beta$, gives that $u(\theta, A_{-i}) = \alpha v(\theta) + \beta + A_{-i} - 1$ by letting $\alpha = 1/(\omega n)$ and $\beta = 1 - \alpha c$. We may then assume, without loss of generality, that $\alpha v + \beta = \mathrm{id}$, the identity map from $\RR$ to $\RR$.}
\begin{equation*}
    u_i = u(\theta, A_{-i}) = \theta + A_{-i} - 1.
\end{equation*}
This payoff function is familiar in the global games literature, for example, see \cite{morris_shin_2003} and \cite{morris_yildiz_2019}. Figure \ref{fig:example} illustrates an example with two followers ($n = 2$). 


\input{gametree}


Assume that the leader observes the realization of $\theta$ (henceforth her type). Followers, on the other hand,  have \textit{fundamental uncertainty} about $\theta$. The initial common prior is an improper uniform distribution over the real line.\footnote{~Later we discuss why this assumption, though simplifying, actually strengthens our results.} Each follower $j \in F$ receives a signal $x_j = \theta + \sigma_F \varepsilon_j$, where $\sigma_F > 0$ measures the quality of private information and $\varepsilon_j$ is an idiosyncratic standard Gaussian noise that is independent of $\theta$ and independently and identically distributed (IID) across all followers. In Appendix C we show that our results hold for a set of noise distributions with positive densities, continuously differentiable, symmetric (around zero), and log-concave over $\RR$. We refer to $x_j$ as follower $j$'s type and let $X_j = \RR$ be the corresponding type space. 



From a global game perspective, the assumption that the leader knows $\theta$ is common knowledge is critical. In particular, under this assumption, the subgames do not have two-sided dominance regions, but rather there is only one-sided dominance. In Section 4, we investigate an alternative information structure where the leader also observes a noisy private signal about $\theta$, which brings back the standard two-sided dominance. The current model can be understood as the limiting case when the noise of the leader's information approaches zero while keeping $\sigma_F$ fixed. The spirit of the main result is similar in both cases.


Note that, under complete information, it is straightforward to see that the model admits multiple subgame perfect equilibria when $\theta \in (0, 1/n)$. In the two-follower case, for example, we have two subgame perfect equilibria---$(\invest, \invest\notinvest, \invest\notinvest)$ and$(\notinvest, \notinvest\notinvest, \notinvest\notinvest)$---with the former being fully efficient.\footnote{~By $\invest\notinvest$ we mean a follower exerts effort when $a_L = \invest$ and does not exert effort when $a_L = \notinvest$.} 












