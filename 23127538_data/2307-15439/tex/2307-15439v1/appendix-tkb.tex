%!TEX root = ecai-main.tex

\subsection{Proofs from Section~\ref{sec:tkb-align}}

\typing*


\begin{proof}	
	Consider a FO interpretation $\I$ and the type defined as $\Phi_{\I} = \set{\pquery \in \props(\ptquery) \mid \I \models \query}$.
	Clearly, we have $\I \models \chformula(\Phi_{I})$ and $\I \not\models \chformula(\Phi)$ for every other type $\Phi$.
	
	On the other hand, note that every type $\Phi \in 2^{\props(\ptquery)}$ identifies a combination of positive and negative boolean BCQs that are consistent by construction, and therefore admit a model $\I$ for all of them, which means that $\I \models \chformula(\Phi)$, and so that $\I$ is of type $\Phi$.
	
	The proof easily lifts to (infinite) sequences of FO traces and types.
	
\end{proof}

\thmbridge*
%
\begin{proof}	
	Observe that from Theorem~\ref{thm:sat-bstates}, we obtain that $\word \in \L(\PA_{\ptquery})$ iff $\word \models \ptquery$, regarded as propositional \ltl.
	Therefore, we need to prove that, for every $\seqI$, it holds that $\seqI \models \tquery$ iff $\word \models \ptquery$, with $\word$ being the trace type of $\seqI$.
	We do it by induction on the structure of $\tquery$.
	
	\begin{itemize}
		\item 
			Assume $\tquery = \psi \in \BCQ$ being a boolean conjunctive query.
			It holds that $\seqI \models \psi$ iff $\seqI_{0} \models \psi$, which, from Lemma~\ref{lem:typing}, is true iff $\hat{\psi} \in \word_0$, being $\word_0$ the type of $\seqI_{0}$.
			The latter holds iff $\word \models \hat{\psi}$ and so iff $\word \in \L(\PA_{\ptquery})$.
	\end{itemize}
	%
	All the other (boolean and temporal) cases are standard proofs that build on top of the semantics of \tcq and of propositional \ltl.
	%
\end{proof}


\reductcomplexity*
\begin{proof}
Direct consequence of Thm.~\ref{thm:sat-bstates}, which states that the state space of $\PA_\tquery$ has doubly exponential size in $\tquery$, and the EXPTIME-completeness of deciding satisfiability of $\chi(\Phi)$ wrt $\Tmc$~\cite{Lutz08}.
\end{proof}


\lemtail*
%
\begin{proof}	
	We prove the two implications separately.
	First, assume that $q_k \notin \Acc(\PA^\Tmc_{\ptquery})$ and let $\word^k = \Phi_{k} \Phi_{k + 1} \cdots$ be a type trace and $\seqI^{k} = \I_{k} \I_{k + 1}$ a corresponding FO-trace.
	If $\I_i \not\models \Tmc$ for some $i \geq k$ then the trace type $\word$ already satisfies the statement.
	Assume instead that $\I_i \models \Tmc$ fore every $i \geq k$ and  consider the run
	$$\ppath^{\Tmc} = q_k \trans{\Phi_{k}} q_{k + 1} \trans{\Phi_{k + 1}} \cdots$$
	of $\word^{k}$ in $\PA^{\Tmc}_{\ptquery}$.
	First, observe that, since $\I_i \models \Tmc$ for every $i \geq k$, we have that $\chformula(\Phi_i)$ is consistent with $\Tmc$ for every $i \geq k$.
	This means that the extra state $q^{*}$ in $\PA^{\Tmc}_{\ptquery}$ does not occur in $\ppath^{\Tmc}$, which implies that $\ppath^{\Tmc}$ is also the run of $\word^{k}$ in $\PA_{\ptquery}$.
	By concatenation, we obtain that
	$$\ppath' = q_0 \trans{\Phi_0} \cdots \trans{\Phi_{k - 1}} q_k \trans{\Phi_{k}} q_{k + 1} \trans{\Phi_{k + 1}} \cdots$$
	is the run in $\PA_{\ptquery}$ of the word $\word$.
	Now, since $q_k \notin \Acc(\PA^\Tmc_{\ptquery})$, we obtain that $\min\set{\painf(\ppath^{\Tmc}, \PA^\Tmc_{\ptquery})}$ is odd.
	%
	% \anni{Why does it follow that it is odd? I think this requires an explanation.}
	%
	Therefore, by the definition of $\col^\Tmc$, we obtain that 
	$\min\set{\painf(\ppath', \PA_{\ptquery})} = \min\set{\painf(\ppath^{\Tmc}, \PA_{\ptquery})} = \min\set{\painf(\ppath^{\Tmc}, \PA^{\Tmc}_{\ptquery})} - 1$ is even and then that $\word \in \L(\PA_{\ptquery})$.
	By Theorem~\ref{thm:bridge} we derive that the FO trace $\seqI$, being of type $\word$, is such that $\seqI \models \tquery$.
	
	We now prove the other direction by counter-nominal argument.
	Assume $q_k \in \Acc(\PA^{\Tmc}_{\ptquery})$.
	%
	This means that there exists a word $\word^k = \Phi_{k} \Phi_{k + 1} \ldots$ whose run $$\ppath^{\Tmc} = q_k \trans{\Phi_{k}} q_{k + 1} \trans{\Phi_{k + 1}} \cdots$$ of $\word^{k}$ in $\PA^{\Tmc}_{\ptquery}$ starting from $q_k$ is accepting.
	This implies that $q^{*}$ never appears in $\ppath^{\Tmc}$, which means that $\chformula(\Phi_i)$ is consistent with $\Tmc$, for every $i \geq k$, and that $$\ppath' = q_0 \trans{\Phi_0} \cdots \trans{\Phi_{k - 1}}q_k \trans{\Phi_{k}} q_{k + 1} \trans{\Phi_{k + 1}} \cdots$$ is the run in $\PA_{\ptquery}$ of the word $\word = \Phi_0 \cdots \Phi_{k - 1} \Phi_{k} \Phi_{k + 1} \cdots$.
	Being that $\ppath^{\Tmc}$ is accepting in $\PA^{\Tmc}_{\ptquery}$, we have that $\min\set{\painf(\ppath^{\Tmc}, \PA^\Tmc_{\ptquery})}$ is even, which implies that $\min\set{\painf(\ppath', \PA_{\ptquery})} = \min\set{\painf(\ppath^{\Tmc}, \PA_{\ptquery})} = \min\set{\painf(\ppath^{\Tmc}, \PA^{\Tmc}_{\ptquery})} - 1$ is odd and then that $\word \notin \L(\PA_{\ptquery})$.
	By Theorem~\ref{thm:bridge}, every FO trace $\seqI$ of type $\word$ is such that $\seqI \not\models \tquery$.
	Observe that, being $\chformula(\Phi_i)$ consistent with $\Tmc$ for every $i \geq k$, this means that $\I_i \models \Tmc$ for every $i \geq k$.
	This proves the statement.
	
\end{proof}

\rtdacomplexity*
\begin{proof}
Let $\tkb=(\Tmc,\Aseq)$ and $\PA_{\ptquery}^{\Tmc}=(2^{\props(\ptquery)},Q^\Tmc,\delta^{\Tmc},q_0,\col^{\Tmc})$ 
	be the $\Tmc$-reduct of $\PA_{\ptquery}=(2^{\props(\ptquery)},Q,\delta,q_0,\col)$. 
 Define the RT-DFA for $\tkb$ and $\tquery$ as $\DFA=(\alphabet,S,s_0,\gamma,F)$.

The result follows directly from the definition of RT-DFA (Def.~\ref{def:rtdfa})
and the following facts: the sizes of $Q$ and \alphabet are doubly exponential wrt \tquery, which implies a triply exponential size of $S$; checking whether there exists an ABox $\Amc$ s.t.~$(\Amc,\Tmc)\models \phi$ is time-exponential wrt both $\Tmc$ (this is shown in the proof of Thm~\ref{th:solvability:kb:alignment:no:nominals:CQ}) and $\phi$, with $\phi=\bigvee_{\Phi \in \Upsilon}\chformula(\Phi)\land
	                        			\bigwedge_{\Phi \not\in \Upsilon}\lnot\chformula(\Phi)$ 
doubly exponential wrt $\tquery$; and computing $\Acc(\PA_{\ptquery}^{\Tmc})$ is doubly exponential wrt \tquery.
\end{proof}


In order to prove Thm.~\ref{thm:rt-sound-compl}, the following auxiliary result is needed.
%%
\begin{lemma}\label{lem:abstraction-instantiation}
	For every word $w \in \alphabet^{*}$ there exists a TKB-modification $\tmods$ that instantiates $w$.
	Moreover, for every TKB-modification $\mods$ there exists a unique abstraction $w$ of it.
\end{lemma}

\begin{proof}
	We first prove that for every word $w$ there exists an instantiation $\tmods$ of it.
	Consider also the run $\ppath = s_0 \trans{w_0} \ldots \trans{w_{m}} s_{m + 1}$ be the corresponding run on the RT-DFA.
	Then, for each $j \leq m$, we have:
	
	\begin{itemize}
		\item 
			If $w_j = \delsym$ then define $\tmods_j = \del$;
			
		\item 
			if $w_j = (\fixsym, \Upsilon)$ and $s_j = (Z, i)$, then consider an ABox $\Amc$ consistent with $\Tmc$ such that $(\Tmc, \Amc) \models \bigvee_{\Phi \in \Upsilon} \chformula(\Phi)\land
			\bigwedge_{\Phi \not\in \Upsilon} \lnot\chformula(\Phi)$.
			Observe that such ABox always exists.
			Now, let $\mods$ be a KB-modification such that $\Amc = \apply(\mods, \Amc_i)$.
			Also such $\mods$ always exists.
			Then, define $\tmods_{j} = \fix{\mods}$;
			
		\item 
			if $w_j = (\addsym, \Upsilon)$ and $s_j = (Z, i)$, then consider an ABox $\Amc$ consistent with $\Tmc$ such that $(\Tmc, \Amc) \models \bigvee_{\Phi \in \Upsilon} \chformula(\Phi)\land
			\bigwedge_{\Phi \not\in \Upsilon} \lnot\chformula(\Phi)$.
			Observe that such ABox always exists.
			Now, let $\mods$ be a KB-modification such that $\Amc = \apply(\mods, \emptyA)$.
			Also such $\mods$ always exists.
			Then, define $\tmods_{j} = \add{\mods}$.
	\end{itemize}
	
	Clearly, the TKB-modification $\tmods$ satisfies Definition~\ref{def:abstraction} and so it is an instantiation of $w$.
	
	For the second statement, consider a TKB modification $\tmods$ of length $m$.
	We construct by induction on $m$ the unique abstraction of $\tmods$.
	
	As base case, if $m = 0$, we have three possible cases:
	
	\begin{enumerate}
		\item 
			If $\tmods_0 = \delsym$ then define $w_{0} = \delsym$;
			
		\item 
			If $\tmods_0 = \fix{\mods}$ then consider $\Amc = \apply(\mods, \Amc_0)$ and $\Upsilon$ being the \emph{unique} element in $2^{2^{\props}}$ such that $(\Tmc, \Amc) \models \bigvee_{\Phi \in \Upsilon} \chformula(\Phi)\land \bigwedge_{\Phi \not\in \Upsilon} \lnot\chformula(\Phi)$.
			Therefore, define $w_0 = (\fixsym, \Upsilon)$;
			
		\item 
			If $\tmods_0 = \add{\mods}$ then consider $\Amc = \apply(\mods, \emptyA)$ and $\Upsilon$ being the \emph{unique} element in $2^{2^{\props}}$ such that $(\Tmc, \Amc) \models \bigvee_{\Phi \in \Upsilon} \chformula(\Phi)\land \bigwedge_{\Phi \not\in \Upsilon} \lnot\chformula(\Phi)$.
			Therefore, define $w_0 = (\addsym, \Upsilon)$.
	\end{enumerate}

	Clearly, in all the cases, $w_0$ is the only abstraction of $\tmods_0$.
	
	For the induction case, assume there is a unique abstraction for every TKB modification of length $m$ and consider the TKB-modification $\tmods$ of length $m + 1$.
	First, consider $\tmods' = \tmods_0 \ldots \tmods_m$ be the prefix of $\tmods$ up to $m$.
	By induction hypothesis, let $w'$ be the only abstraction of $\tmods'$ and $\ppath = s_0 \trans{w_0} \ldots \trans{w_{m}} s_{m + 1}$ be the corresponding run on the RT-DFA, with $s_m = (Z, i)$.
	We distinguish three cases:
	
	\begin{enumerate}
		\item 
			If $\tmods_{m + 1} = \del$, then define $w_{m + 1} = \delsym$;
			
		\item 
			If $\tmods_{m + 1} = \fix{\mods}$, then consider $\Amc = \apply(\mods, \Amc_i)$ and $\Upsilon$ being the \emph{unique} element in $2^{2^{\props}}$ such that $(\Tmc, \Amc) \models \bigvee_{\Phi \in \Upsilon} \chformula(\Phi)\land \bigwedge_{\Phi \not\in \Upsilon} \lnot\chformula(\Phi)$.
			Therefore, define $w_{m + 1} = (\fixsym, \Upsilon)$;
			
		\item 
			If $\tmods_{m + 1} = \add{\mods}$, then consider $\Amc = \apply(\mods, \emptyA)$ and $\Upsilon$ being the \emph{unique} element in $2^{2^{\props}}$ such that $(\Tmc, \Amc) \models \bigvee_{\Phi \in \Upsilon} \chformula(\Phi)\land \bigwedge_{\Phi \not\in \Upsilon} \lnot\chformula(\Phi)$.
			Therefore, define $w_{m + 1} = (\addsym, \Upsilon)$;
	\end{enumerate}

	Clearly, also in this case, the abstraction $w = w' \cdot w_{m + 1}$ is the unique abstraction of $\tmods$.
	
\end{proof}



\thmsoundcompl*

\begin{proof}
	First, observe that thanks to Lemma~\ref{lem:abstraction-instantiation} we need to prove only Item~2 of the theorem.
	Indeed, Item~1 would follow from Item~2 and from the fact that every abstraction admits at least an instantiation.
	Analogously, Item~3 would follow from Item~2 and from the fact that every TKB-modification admits a unique abstraction.
	
	We then prove Item~2.
	For technical convenience, we prove a slightly more general result.
	First, for every TKB $\tkb = (\Tmc, \Aseq)$ with $\Aseq = \Amc_0 \cdots \Amc_\ell$ and an index $i$, define $\tkb_{\geq i} = (\Tmc, \Aseq_{\geq i})$ where $\Aseq_{\geq i} = \Amc_i \cdots \Amc_{\ell}$.
	We prove that, for every word $w \in \alphabet^{*}$ and state $s = (Z, i)$ of the automaton $\DFA$, it holds that the run 
	
	\[
	\ppath = s \trans{w_0} s_1 \trans{w_1} \ldots \trans{w_m} s_{m + 1}
	\]
	
	is accepting in $\DFA$ iff, for all instantiations $\tmods$ of $w$ and every trace type $\word$ such that  $\seqI \models \apply(\tmods, \tkb_{\geq i})$ for every $\seqI$ of type $\word$, it holds that the run in $\PA_{\ptquery}$ starting from some $q \in Z$ over $\word$ is accepting.
	
	Item~2 of the theorem then follows by setting $s = (\{q_0, 0\})$ the initial state of $\DFA$, which in turns applies to the TKB $\tkb$.
	
	The proof is by induction on the length $m$ of $w$ and, subsequently, $\tmods$.
	
	As base case, assume $m = 0$, therefore both $m$ and $\tmods$ are the empty sequence.
	This implies that the path $\ppath = s$ is made by just $s = (Z, i)$ and it is accepting iff $s \in F$.
	By construction of $\DFA$, first we have that $i = \ell + 1$ and so that $\tkb_{\geq i} = (\Tmc, \emptyAseq)$.
	Moreover, we have that $Z \cap \Acc(\PA^{\Tmc}_{\ptquery}) = \emptyset$ and so that every state $q$ in $Z$ does not belong to $\Acc(\PA^{\Tmc}_{\ptquery})$.
	By Lemma~\ref{lem:tail}, we have that every trace type $\word$ such that $\seqI \models (\Tmc, \emptyAseq)$ for every $\seqI$ of type $\word$, is such that the run starting from $q$ is accepting in $\PA_{\ptquery}$, which proves the statement.
	
	For the induction case, assume that the property holds for a given $m$, and that $w$ and $\tmods$ are of length $m + 1$.
	Observe that, since $\ppath$ is accepting, then also the path
	
	\[
	\ppath' = s_1 \trans{w_1} \ldots \trans{w_m} s_{m + 1}
	\]
	%
	is accepting, which means that, by induction hypothesis, for every instantiation $\tmods'$ of $w' = w_{1} \cdots w_{m}$, every trace type $\word'$ such that  $\seqI \models \apply(\tmods, \tkb_{\geq 1})$ for every $\seqI'$ of type $\word'$, it holds that the run in $\PA_{\ptquery}$ starting from some $q \in Z_1$, with $s_1 = (Z_1, \iota)$ over $\word'$ is accepting.
	We distinguish three cases:
	
	\begin{itemize}
		\item 
			$w_0 = \delsym$, then $\tmods_{0} = \del$ and every instantiation of $w$ is of the form $\tmods_{0} \cdot \tmods'$.
			We also obtain that $s_1 = \gamma((Z, 0), \delsym) = (Z, 1)$.
			In addition, observe that 
			
			\[
			\apply(\tmods_{0} \cdot \tmods', \tkb_{\geq 0}) = \apply(\del \cdot \tmods', \tkb_{\geq 0}) = \apply(\tmods', \tkb_{\geq 1})
			\]
			which, combined with the induction hypothesis, proves the statement.
			
		\item 
			$w_0 = (\fixsym, \Upsilon)$ and so $\tmods_{0} = \fix{\mods}$ for some $\mods$ such that $(\Tmc, \apply(\mods, \Amc_0)) \models \bigvee_{\Phi \in \Upsilon}\chformula(\Phi)\land
			\bigwedge_{\Phi \not\in \Upsilon}\lnot\chformula(\Phi)$.
			Also, every instantiation of $w$ is of the form $\tmods_{0} \cdot \tmods'$.
			Observe that 
			
			\[
			s_1 = \gamma((Z,0), (\fixsym, \Upsilon)) = (Z_1, 1)
			\]
			
			where $Z_1$ contains only states $q'$ such that $\delta(q,\Phi)=q'$, for some $q \in Z$.
			
			Observe that			
			
			\begin{align*}
				\apply(\tmods_{0} \cdot \tmods', \tkb_{\geq 0}) & =\apply(\fix{\mods} \cdot \tmods', \tkb_{\geq 0}) \\ 
				& =(\Tmc, \apply(\mods, \Amc_0) \cdot \apply(\tmods', \Aseq_{\geq 1}))
			\end{align*}
		
			By induction hypothesis, every trace type $\word'$ such that $\seqI' \models (\Tmc, \apply(\tmods', \Aseq_{\geq 1}))$ for every $\seqI'$ of type $\word'$ has an accepting run in $\PA_{\ptquery}$ starting from $s_1$.
			Moreover, the only types $\Phi_0$ entailed by $(\Tmc, \apply(\mods, \Amc_0))$ must belong to $\Upsilon$ by construction.
			Therefore, the word $\word = \Phi_0 \cdot \word'$ clearly is accepted by $\PA_{\ptquery}$ from some state $q \in Z$ such that $\delta(q, \Phi_0) \in Z_1$ and so the statement holds.
			
			\item 
				$w_0 = (\addsym, \Upsilon)$ and so $\tmods_{0} = \add{\mods}$ for some $\mods$ such that $(\Tmc, \apply(\mods, \emptyA)) \models \bigvee_{\Phi \in \Upsilon}\chformula(\Phi)\land
				\bigwedge_{\Phi \not\in \Upsilon}\lnot\chformula(\Phi)$.
				Also, every instantiation of $w$ is of the form $\tmods_{0} \cdot \tmods'$.
				Observe that 
				
				\[
				s_1 = \gamma((Z,0), (\fixsym, \Upsilon)) = (Z_1, 0)
				\]
%				
				where $Z_1$ contains only states $q'$ such that $\delta(q,\Phi)=q'$, for some $q \in Z$.
%				
				Observe that
				
				\begin{align*}
					\apply(\tmods_{0} \cdot \tmods', \tkb_{\geq 0}) & =\apply(\add{\mods} \cdot \tmods', \tkb_{\geq 0}) \\ 
					& =(\Tmc, \apply(\mods, \emptyA) \cdot \apply(\tmods', \Aseq_{\geq 0}))
				\end{align*}
%				
				By induction hypothesis, every trace type $\word'$ such that $\seqI' \models (\Tmc, \apply(\tmods', \Aseq_{\geq 0}))$ for every $\seqI'$ of type $\word'$ has an accepting run in $\PA_{\ptquery}$ starting from $s_1$.
				Moreover, the only types $\Phi_0$ entailed by $(\Tmc, \apply(\mods, \emptyA))$ must belong to $\Upsilon$ by construction.
				Therefore, the word $\word = \Phi_0 \cdot \word'$ clearly is accepted by $\PA_{\ptquery}$ from some state $q \in Z$ such that $\delta(q, \Phi_0) \in Z_1$ and so the statement holds.
	
	\end{itemize}

	Note that all the arguments in both base and induction cases are equivalences, which means both directions of Item~2 are proved.
\end{proof}

\migcomplexity*
\begin{proof}
Consequence of Def.~\ref{def:mmg}, Lemma~\ref{lem:rtda-complexity}, which states that RT-DFA $\DFA = (\alphabet, S, s_0, \gamma, F)$ for $\tkb$ and $\tquery$ has triply eponential size wrt $\tquery$ and the fact that KB-Alignment is solvable in doubly exponential time (see Thm.~\ref{th:solvability:kb:alignment:no:nominals:CQ}).
\end{proof}


\tkbalgsolv*
\begin{proof}
    Immediate consequence of the fact that the Minimal-instantiation Graph $G$ has triply exponential size and that Shortest Path can be solved in polynomial time on $G$.
\end{proof}



%\thmtkbalign*

%\begin{proof}
%	content
%\end{proof}