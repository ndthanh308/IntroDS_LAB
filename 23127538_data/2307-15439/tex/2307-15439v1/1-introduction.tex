%!TEX root = ecai-main.tex

\section{Introduction}


Observing complex systems over time and drawing conclusions about their behavior 
is a core task for many AI systems. In particular, adaptive systems have to recognize situations in which an adaptation is useful. A well-investigated approach to do this is ontology-based situation recognition \cite{BBKTT-KIJ-20,AKRWZ-AIJ-21,Optique-17}. This approach is usually realized by modeling the observed system by a temporal knowledge base (TKB), where the data from the observed system is collected over time and stored in a sequence of ABoxes and a TBox that models important notions from the application domain. 
%
The situation to be recognized by the system is then modeled by a temporal query that will be answered over the sequence of ABoxes and the TBox. The situation recognition is then to detect predefined situations that are formalized as temporal (conjunctive) queries over the observed and enriched ABox sequence.
%
As in classical ontology-mediated query answering \cite{BO-RW-15}, the TBox enriches the data in the ABox sequence as it restricts its interpretation and allows for more conclusions. 
The semantics of  TKBs is given by an infinite  sequence of first-order interpretations. TKBs can be queried by temporal conjunctive queries (TCQs), which combine \ltl with conjunctive queries. Methods for answering temporal queries over TKBs and testing entailment of Boolean TCQs have been intensively investigated (\cite{BaaderBL15,BoLT-JWS15,AKRWZ-AIJ-21}). 


Now, in many applications, the data is  collected from several sources and need not always be accurate. Consider the medical domain, where deviations of classical symptoms are frequent for certain patient groups or where examination methods such as blood test results can be inaccurate or discretized unsuitably. Thus the query need not return the expected answer, although the patient or, in the general case, the observed system is in a critical state that requires adaptation. The problem is to find a version of the TKB that admits to detect \enquote{near misses}. 

%
There are mainly two approaches developed to address the problem of errors or inaccuracies in DL knowledge bases. In case of inconsistent TKBs, ontology repairs restore consistent versions by deleting statements from the ABoxes \cite{BouKooTur2019,Bien-KIJ-20}. In case that information is missing in the ABoxes for the query to return answers, ABox abduction, i.e., adding new statements to the ABoxes has been investigated---mostly in the atemporal setting \cite{DR-AAAI-19,KoDeToSc-KR-20}.

In this paper, we investigate the new task of TKB Alignment, i.e., to modify the sequence of ABoxes by deletions or additions of statements so as to yield answers for the TCQ. 
%
Surprisingly, this problem has not been addressed in the literature yet. The goal of this paper is to develop an approach to solve instances of this new problem.


The well-known problem of Trace Alignment realizes a very similar task to TKB alignment: for a finite trace of observations  and a property specification expressed in Linear Temporal Logic (\ltl), a minimal modification of the 
trace is produced that satisfies the specification. This task has been extensively studied by the 
Business Process (BP) and AI communities, leading to effective solutions and implemented tools; 
see, e.g.,~\cite{DMMP17,Leoni2012,LeoniMA15}.
%%
In all these settings, the observations recorded in a trace are propositional, i.e., each time point of the trace 
represents one of finitely many possible observables, modeled as propositions.
%%



%----  old version ----


In this paper, we address the problem of 
TKB Alignment as a Trace Alignment problem in a much richer setting, where 
observables 
are described by DL concepts and roles, and properties are specified by a  temporalized query using DL atoms. Furthermore, the open world semantics of DLs is adopted, since entailment is considered instead of satisfaction as in classical propositional trace alignment.


We investigate the following setting for TKB alignment: a TCQ %temporal conjunctive query 
using (future) \ltl operators and a TKB written in the DL \alc, 
together with a  cost measure for edit operations on the ABox sequence. 
Solving TKB alignment is then to compute an ABox sequence  which, together with the TBox, entails the Boolean TCQ,
while guaranteeing cost-optimality of the modification. Intuitively, the cost-optimal version of the TKB states which minimal changes of the TKB would result in answers to the TCQ.

The technique we develop builds on an approach for deciding temporal query entailment over TKBs by \cite{BaaderBL15} 
and one for solving \ltl Trace Alignment for finite traces by \cite{DMMP17}, and extends them non-trivially.  
Our technique extends the former approach from verification to synthesis 
and the latter from the propositional to the DL setting, from propositional traces to TKBs, and from finite to infinite traces.
%
%This ultimately results in an effective solution approach which allows for automatically identifying noisy observations
%in a knowledge-based setting, and defining corrective actions to recover the original observations.
This ultimately results in an effective solution approach which can assess the deviation of irregular observations wrt standard ones
and define corrective actions to recover a standard observation.

%
%%The general approach (as in \cite{BaaderBL15}) is to decompose the Boolean TCQ into two parts. The first one is the outer abstraction, where the DL queries are replaced by propositional variables. This gives an \ltl formula over (Boolean combinations) of propositional formulas and makes, in principle, the automata-based approach for alignment of finite \ltl traces from \cite{DMMP17} applicable.
%%\todo[inline]{For Fabio and Giuseppe: short description on   automaton-based approach to TKB alignment.}
%The second and inner part is a sequence of classical Boolean conjunctive queries over DL atoms corresponding to each time point in the outer abstraction.
%
%%In order to employ the automata-based approach for TKB alignment by \cite{DMMP17}, we had to modify it in several ways.
%%%
%%(1) While alignment in case of \ltl modifies the model directly, in case of TKBs it modifies the sequence of ABoxes (and thus infinite sets of infinite models).
%%\todo[inline]{For Fabio and Giuseppe: describe the "remedy" to this in the automaton construction.}
%%
%%%
%%\todo[inline]{To add:
%%\\ (2)satisfiability vs.\ entailment: instead of simply testing unsatisfiability of negated query, we need to \dots}
%%%
%%(3) \ltl trace alignment uses simple cost computation for edit operations, but for TKBs the cost computation has to combine the costs of edit operations on the sequence of ABoxes with those on statements in an ABox. 
%%The cost computation for editing ABox statements requires to solve the (atemporal) KB alignment problem. We develop an algorithm for KB alignment.
%% 

\smallskip \noindent The detailed proofs for all results 
are supplied in the Appendix.

%%This paper is structured as follows. After the preliminaries,  Sect.~\ref{sec:tkb-align} defines the TKB alignment problem and extends the automata-based method for solving it. Sect.~\ref{sec:kbalign} describes a method for solving KB alignment. The paper ends with conclusions and future work.
%%	