\documentclass[draft]{ecai}  % use for more realistic space assessment! %\documentclass[final,doubleblind]{ecai}  % use option [doubleblind] for double blind submission and hiding the authors section

\usepackage{graphicx}
\usepackage{latexsym}
\usepackage[numbers]{natbib}

%\ecaisubmission      % inserts page numbers. Use only for submission of paper.
                      % Do NOT use for camera-ready version of paper.

\paperid{1742}        % paper id for double blind submission

\usepackage{paralist}

\usepackage{csquotes}

\usepackage{thmtools,thm-restate} %to restate results in the Appendix

%\usepackage{amsmath}
\usepackage{algorithm}
\usepackage[noend]{algorithmic}
%\usepackage[switch]{lineno}

%%****************************************************************************%%
%% Packages                                                          %%
%%****************************************************************************%%

\usepackage{xargs}

\usepackage{xspace}

\usepackage{xstring}

\usepackage{boolexpr}

\usepackage[cmex10]{mathtools}
\usepackage{amsfonts}
\usepackage{mathrsfs}
%\usepackage{latexsym}
\usepackage{textcomp}
\usepackage{pifont}
\usepackage{amssymb,bm,xargs}
\usepackage[shortlabels]{enumitem}


%\usepackage{hyperref}
%\usepackage{natbib}

%\usepackage[ruled,vlined,linesnumbered]{algorithm2e}
\usepackage{wrapfig}

%Only for drafts
\usepackage{lipsum}
\usepackage[obeyDraft]{todonotes}
\usepackage{cancel}
%\usepackage{newproof}

%%****************************************************************************%%
%% Environments                                                               %%
%%****************************************************************************%%


\newtheorem{definition}{Definition}
%\newtheorem{theorem}{Theorem}
\newtheorem{lemma}{Lemma}
\newtheorem{proposition}{Proposition}
%\newtheorem{corollary}{Corollary}
%\newtheorem{conjecture}{Conjecture}
%\newtheorem{example}{Example}
\newtheorem{remark}{Remark}

\newenvironment{proof}{\emph{Proof.} }{\hfill $\square$ \\} 


%%****************************************************************************%%
%% Misc                                                                       %%
%%****************************************************************************%%

%%****************************************************************************%%

%%****************************************************************************%%
%% Commands	                                                                  %%
%%****************************************************************************%%

\newcommand{\argemp}[2]{\if&#1&\else#2\fi}

\newcommand{\argdef}[2]{\if&#1&#2\else#1\fi}

%%****************************************************************************%%

\newcommand{\argint}[3]{\if&#2&\else#1#2#3\fi}

\newcommand{\argext}[3]{\if&#1&#3\else#1\if&#3&\else#2#3\fi\fi}

%%****************************************************************************%%


\newcommandx{\mthfnt}[3][1=, 2=0]{{
	\IfStrEqCase{#1}
	{%
		{}%
		{#3}%
		{Name}%
		{%
			\IfStrEqCase{#2}
			{%
				{0}{\mathcal{#3}}%
				{1}{\mathscr{#3}}%
				{2}{\mathfrak{#3}}%
				{3}{\mathbb{#3}}%
			}
			[\ensuremath{\clubsuit}]%
		}%
		{Set}%
		{%
			\IfStrEqCase{#2}
			{%
				{0}{\mathrm{#3}}%
				{1}{\mathsf{#3}}%
				{2}{\mathbb{#3}}%
				{3}{\mathbf{#3}}%
			}
			[\ensuremath{\clubsuit}]%
		}%
		{Fun}%
		{%
			\IfStrEqCase{#2}
			{%
				{0}{\mathsf{#3}}%
				{1}{\mathrm{#3}}%
			}
			[\ensuremath{\clubsuit}]%
		}%
		{Rel}%
		{%
			\IfStrEqCase{#2}
			{%
				{0}{\mathit{#3}}%
				{1}{\mathtt{#3}}%
			}
			[\ensuremath{\clubsuit}]%
		}%
		{Sym}%
		{%
			\IfStrEqCase{#2}
			{%
				{0}{\mathtt{#3}}%
				{1}{\mathbf{#3}}%
			}
			[\ensuremath{\clubsuit}]%
		}%
		{Elm}%
		{\mathnormal{#3}}
	}
[\ensuremath{\clubsuit}]%
}}
			
			
\newcommand{\mthsub}[1]{\argemp{#1}{\ensuremath{_{\mathnormal{#1}}}}}

\newcommand{\mthsup}[1]{\argemp{#1}{\ensuremath{^{\mathnormal{#1}}}}}

\newcommandx{\mth}[5][1=, 2=0, 4=, 5=]{{\ensuremath{\mthfnt[#1][#2]{#3}\mthsub{#4}\mthsup{#5}}}}

\newcommandx{\mtharg}[6][1=, 2=0, 4=, 5=]{{\mth[#1][#2]{#3}[#4][#5]\ensuremath{\argint{(}{#6}{)}}}}
			
			
\newcommand{\mthempty}{\mth[][]}
\newcommand{\mthargempty}{\mtharg[][]}
			
\newcommand{\mthstyname}{0}
\newcommand{\mthname}[1][]{\mth[Name][\argdef{#1}{\mthstyname}]}
\newcommand{\mthargname}[1][]{\mtharg[Name][\argdef{#1}{\mthstyname}]}

\newcommand{\mthstyset}{0}
\newcommand{\mthset}[1][]{\mth[Set][\argdef{#1}{\mthstyset}]}
\newcommand{\mthargset}[1][]{\mtharg[Set][\argdef{#1}{\mthstyset}]}

\newcommand{\mthstyfun}{0}
\newcommand{\mthfun}[1][]{\mth[Fun][\argdef{#1}{\mthstyfun}]}
\newcommand{\mthargfun}[1][]{\mtharg[Fun][\argdef{#1}{\mthstyfun}]}

\newcommand{\mthstyrel}{0}
\newcommand{\mthrel}[1][]{\mth[Rel][\argdef{#1}{\mthstyrel}]}
\newcommand{\mthargrel}[1][]{\mtharg[Rel][\argdef{#1}{\mthstyrel}]}

\newcommand{\mthstysym}{0}
\newcommand{\mthsym}[1][]{\mth[Sym][\argdef{#1}{\mthstysym}]}
\newcommand{\mthargsym}[1][]{\mtharg[Sym][\argdef{#1}{\mthstysym}]}

\newcommand{\mthstyelm}{0}
\newcommand{\mthelm}[1][]{\mth[Elm][\argdef{#1}{\mthstyelm}]}
\newcommand{\mthargelm}[1][]{\mtharg[Elm][\argdef{#1}{\mthstyelm}]}
			

\newcommand{\tuple}[1]
{\ensuremath{\!\argint{\langle}{#1}{\rangle}}}


%%% Figure packages and styles
%
%\usepackage{tikz}
%\usetikzlibrary{arrows,shapes}
%
%\usepackage{wrapfig}
%% \usepackage[caption=false,font=footnotesize]{subfig}
%
%\tikzstyle{every node} =
%[draw = none, fill = white, thin]
%\tikzstyle{every edge} +=
%[black, thick]
%
%\tikzstyle{noall} =
%[draw = none, fill = none]
%\tikzstyle{nodraw} =
%[draw = none, fill = white]
%\tikzstyle{nofill} =
%[draw = black, fill = none]
%
%\tikzstyle{cnode} =
%[circle, draw = black]
%\tikzstyle{snode} =
%[regular polygon, regular polygon sides = 4, draw = gray!50]
%\tikzstyle{lnode} =
%[diamond, draw = gray!75]
%\tikzstyle{pnode} =
%[regular polygon, regular polygon sides = 5, draw = gray]

\newcommand\calF{\mathcal{F}}
\newcommand\calG{\mathcal{G}}
\newcommand\calM{\mathcal{M}}
\newcommand\calV{\mathcal{V}}
\newcommand\calU{\mathcal{U}}
\newcommand\calW{\mathcal{W}}
\newcommand\calP{\mathcal{P}}
\newcommand\calD{\mathbb{D}}
%%%%%%%%%%%%%%%%%
%% macros introduced by Luke 
\newcommand\mydef[1]{{\bf\em #1}}
%%%%%%%%%%%%%%%%%

\newcommand{\numviparams}{{| \lambda |}}
\newcommand{\scoreaccvars}[1]{s_1^{#1}, \ldots, s_{\numviparams}^{#1}}
\newcommand{\scoreaccvar}[2]{s_{#1}^{#2}}
\newcommand{\isdeterm}[1]{\text{Deterministic}({#1})}


\newcommand{\expect}[1]{\mathbb{E}\left[{#1}\right]}
\newcommand{\var}[1]{\mathbb{V}\left[ {#1} \right]}
\newcommand{\expectdist}[2]{\mathbb{E}_{#1}\left[ {#2} \right]}
\newcommand{\vardist}[2]{\mathbb{V}_{#1}\left[ {#2} \right]}
\newcommand{\cov}[2]{\mathbb{C}\text{ov}[{#1}][{#2}]}
\newcommand{\covv}[1]{\mathbb{C}\text{ov}[{#1}]}
\newcommand{\corr}[1]{\mathbb{C}\text{orr}[{#1}]}

\newcommand{\fix}[1]{\mathit{fix}\left({#1}\right)}
\newcommand{\sbr}[1]{\left\llbracket {#1} \right\rrbracket}
\newcommand{\ctxtype}[3]{{#1} \cong_\text{ctx} {#2} : {#3}}
\newcommand{\bigstep}[3]{{#1} \Downarrow_{#2} {#3}}


% PCF types
\newcommand{\bool}{\mathit{bool}}
\newcommand{\nat}{\mathit{nat}}

\newcommand{\ctx}[1]{\mathcal{C}\left[ {#1}\right] }
\newcommand{\pcft}[1]{\text{PCF}_{#1}}

\newcommand{\nfl}{\mathbb{N}_\bot}
\newcommand{\bfl}{\mathbb{B}_\bot}

% PCF constructs
\newcommand{\succc}[1]{\mathbf{succ}({#1})}
\newcommand{\succcn}[2]{\mathbf{succ}^{#1}({#2})}
\newcommand{\zero}{\mathbf{0}}
\newcommand{\zerotest}[1]{\mathbf{zero}\left({#1}\right)}
\newcommand{\pred}[1]{\mathbf{pred}\left( {#1} \right)}
\newcommand{\predn}[2]{\mathbf{pred}^{#1}\left( {#2} \right)}
\def\solvable{\#}

\newcommand{\true}{\mathbf{true}}
\newcommand{\false}{\mathbf{false}}
\newcommand{\pcffix}[1]{\mathbf{fix}\left({#1}\right)}
\newcommand{\pcffn}[3]{\mathbf{fn}~{#1}:{#2}\mathpunct{.}{#3}}
\newcommand{\pairtype}[2]{{#1} * {#2}}
\newcommand{\pairexp}[2]{\mathbf{pair}({#1}, {#2})}
\newcommand{\leftexp}[1]{\mathbf{left}({#1})}
\newcommand{\rightexp}[1]{\mathbf{right}({#1})}

\newcommand{\RationalPos}{\mathbb{Q}^{+}}

\newcommand{\meas}[1]{\mathbb{M}\left( {#1} \right) }
\newcommand{\integ}[1]{\sbr{#1}_I}

\newcommand{\notbigstep}[2]{{#1}~\cancel{\Downarrow}_{#2}}
\newcommand{\subtrace}[3]{{#1}^{{#2} \ldots {#3}}}
\newcommand{\supp}[1]{\textsf{supp}\left({#1}\right)}
\newcommand{\dom}[1]{\textsf{Dom}\left({#1}\right)}
\newcommand{\suppk}[2]{\textsf{Supp}^{#1}\left({#2}\right)}
\newcommand{\tracespace}{\bigcup_{n \in \mathbb{N}}[0, 1]^n}
\newcommand{\generictracespace}{\mathbb{T}}
\newcommand{\nnreals}{\mathbb{R}_{\geq 0}}
\newcommand{\posreals}{\mathbb{R}_{> 0}}
\newcommand{\reals}{\mathbb{R}}

\newcommand{\unrollkM}[2]{\textsf{unroll}_{#1}\left({#2}\right)}
\newcommand{\nphmcint}[5]{\Psi_\textsf{NP}\left({#1}, {#2}, {#3}, {#4}, {#5}\right)}

%SPCF constructs
\newcommand{\spcfvalues}{\Lambda^0_v}

\newcommand{\prevalueM}[1]{\textsf{value}^{-1}_{#1}(\spcfvalues{})}
\newcommand{\num}[1]{\underline{#1}}

% \theoremstyle{definition}
% \newtheorem{thm}{Theorem}
% \newtheorem{lem}{Lemma}
% \newtheorem{defn}{Definition}
% \newtheorem{conj}{Conjecture}
% \newtheorem{prop}{Proposition}

%\theoremstyle{definition}
%\newtheorem{defn}{Definition}[section]
%\newtheorem{example}[defn]{Example}
%
%
%\theoremstyle{plain}
%\newtheorem{thm}{Theorem}[section]
%\newtheorem{lem}[thm]{Lemma}
%\newtheorem{cor}[thm]{Corollary}
%\newtheorem{conj}[thm]{Conjecture}
%\newtheorem{prop}[thm]{Proposition}
%\newtheorem{remark}[thm]{Remark}

%% Proofs
%\let\oldproof\proof
%\renewcommand{\proof}{\color{blue}\oldproof}


\definecolor{codegreen}{rgb}{0,0.6,0}
\definecolor{codegray}{rgb}{0.5,0.5,0.5}
\definecolor{codepurple}{rgb}{0.58,0,0.82}
\definecolor{backcolour}{rgb}{0.95,0.95,0.92}

\lstdefinestyle{myStyle}{
    belowcaptionskip=1\baselineskip,
    breaklines=true,
    frame=none,
    basicstyle=\footnotesize\ttfamily,
    keywordstyle=\bfseries\color{green!40!black},
    commentstyle=\itshape\color{purple!40!black},
    identifierstyle=\color{blue},
    backgroundcolor=\color{gray!10!white},
    %backgroundcolor=\color{backcolour}, 
    numberstyle=\tiny\color{codegray},
    stringstyle=\color{codepurple},
    breakatwhitespace=false,                          
    keepspaces=true,                 
    numbers=left,       
    numbersep=5pt,                  
    showspaces=false,                
    showstringspaces=false,
    showtabs=false,                  
    tabsize=2,
}

% argmin/argmax
\DeclareMathOperator*{\argmax}{arg\,max}
\DeclareMathOperator*{\argmin}{arg\,min}

% Concatenation of lists
\newcommand\doubleplus{+\kern-1.3ex+\kern0.8ex}

% Program configurations
\newcommand{\tuple}[1]{\ensuremath{\langle #1 \rangle}}
% Rule based definitions
\newcommand{\Rule}[4][]{\ensuremath{\inferrule*[lab={\hypertarget{#2}{(\TirName{#2})}},#1]{#3}{#4}}}

% Calligraphic symbols
\newcommand{\calI}{{\mathcal I}} 
\newcommand{\calT}{{\mathcal T}}

%  Macro for new Y operator.
\newcommand{\yBounded}[3]{\mu^{#1}_{#2}\rvert_{#3}}

%%%%%%%%%%%%%%%%%
 
%%%%%%%%%%%%%%%%%

\newcommand{\expv}{\mathbb{E}}

\newcommand{\combTr}[2]{\left[\begin{matrix}
		#1\\
		#2
	\end{matrix} \right]}

\newcommand{\exType}[2]{\left\{\begin{matrix}
		#1\\
		#2
	\end{matrix} \right\}}
\newcommand{\myint}[1]{ [#1]}
\newcommand{\Uniform}{\ensuremath{\mathrm{Uniform}}}
\newcommand{\Normal}{\ensuremath{\mathrm{normal}}}
\DeclareMathOperator{\abs}{abs}
\DeclareMathOperator{\pdf}{pdf}

\newcommand{\intConf}[1]{\lceil#1\rceil}
\newcommand{\tr}{\boldsymbol{t}}

\newcommand{\sample}{\tt{sample}}
%\newcommand{\fix}{\texttt{fix}}
%\newcommand{\num}[1]{\underline{#1}}
\newcommand{\myif}{\texttt{if}}
\newcommand{\mylet}{\texttt{let} \, }
\newcommand{\myin}{\, \texttt{in} \,}
\newcommand{\mythen}{\, \texttt{then} \,}
\newcommand{\myelse}{\, \texttt{else} \,}
\newcommand{\score}{\tt{score}}
\newcommand{\tick}{\tt{tick}}

\newcommand{\term}{\tt{term}}
\newcommand{\pv}{\mathbf{v}}
\newcommand{\rv}{\mathbf{r}}

\newcommand{\interval}{\mathfrak{I}}

\newcommand{\typeReal}{\textbf{\textsf{R}}}

\newcommand{\symbolInt}{\myint{\cdot}}

\newcommand{\LambdaInterval}{\Lambda_{\interval}}
\newcommand{\LambdaSymbolic}{\Lambda_{\text{sym}}}

\newcommand{\toIntervalTerm}[1]{#1^{2\interval}}

%Others
\newcommand{\Sset}{\mathbb{S}}
\newcommand{\Iset}{\mathbb{I}}
\newcommand{\Rset}{\mathbb{R}}
\newcommand{\Nset}{\mathbb{N}}
\newcommand{\Zset}{\mathbb{Z}}

\newcommand{\Term}{\mathbb{T}}
\newcommand{\prob}{\mathbb{P}}
\newcommand{\expt}{\mathbb{E}}


\newcommand{\Leb}{\tt{Leb}}
\newcommand{\Red}{\tt{Red}}
\newcommand{\cost}{\text{cost}}

%\newcommand{\intervalab}[2]{\underline{[#1,#2]}}
\newcommand{\intervalab}{\underline{[a,b]}}
\newcommand{\interI}{\mathcal{I}}
\newcommand{\trans}{\mathcal{T}}

\newcommand{\iv}{\mathbb{I}}

% Programming language constructs
\newcommand{\lit}[1]{\underline{#1}}
\newcommand{\letIn}[1]{\mathsf{let}\,{#1}\,\mathsf{in}\,}
\newcommand{\fixLam}[2]{\mu {#1} {#2}.}
\newcommand{\ifElse}[3]{\mathsf{if} (#1 \le \num{0}) \, {#2} \,\mathsf{else}\, {#3}}

%%Basic notions
\newcommand{\pspace}{(\Omega,\mathcal{F},\probm)}
\newcommand{\probm}{\mathbb{P}}
\newcommand{\condexpv}[2]{{\expt}{\left[{#1} \mid {#2}\right]}}

\newcommand{\stdConf}[1]{(#1)}
%\newcommand{\intConf}[1]{\lceil#1\rceil}
%\newcommand{\intConf}[1]{(#1)}
%\newcommand{\symConf}[1]{\langle\!\langle  #1 \rangle\!\rangle}
%\newcommand\symPath[1]{(#1)}
\newcommand{\symPath}[1]{\langle\!\langle  #1 \rangle\!\rangle}
\newcommand\symConf[1]{(#1)}

\newcommand{\ifSimple}[3]{\mathsf{if}(#1, #2, #3)}
%\newcommand{\ifElse}[3]{\mathsf{if} (#1 \le 0) \, \allowbreak {#2} \, \allowbreak \mathsf{else}\, {#3}}
%\newcommand{\ifElse}[3]{\ifSimple{#1}{#2}{#3}}

%\newcommand{\trace}{\mathsf{s}}
%
%\newcommand\defn[1]{{\bf \em #1}}
\newcommand{\traces}{\mathbb{T}}
%
%\newcommand{\stdConf}[1]{(#1)}
%%\newcommand{\intConf}[1]{\lceil#1\rceil}
%\newcommand{\intConf}[1]{(#1)}
%%\newcommand{\symConf}[1]{\langle\!\langle  #1 \rangle\!\rangle}
%%\newcommand\symPath[1]{(#1)}
%\newcommand{\symPath}[1]{\langle\!\langle  #1 \rangle\!\rangle}
%\newcommand\symConf[1]{(#1)}

\newcommand{\valueSem}[1]{\mathsf{val}_{#1}} % value (semantics)
\newcommand{\weightSem}[1]{\mathsf{wt}_{#1}} % weight (semantics)
\newcommand{\measureSem}[1]{\llbracket #1 \rrbracket}
\newcommand{\posterior}{\mathsf{posterior}}


%%%%%%%%%
% 
%%%%%%%%
\newcommand{\loc}{\ell}
\newcommand{\locs}{\mathit{L}}
\newcommand{\blocs}{\mathit{L}_{\mathrm{b}}}

\newcommand{\iflocs}{\mathit{L}_{\mathrm{if}}}
\newcommand{\looplocs}{\mathit{L}_{\mathrm{while}}}

\newcommand{\alocs}{\mathit{L}_{\mathrm{a}}}
\newcommand{\wlocs}{\mathit{L}_{\mathrm{w}}}
\newcommand{\rlocs}{\mathit{L}_{\mathrm{r}}}
\newcommand{\Alocs}[1]{\mathit{L}_{\mathrm{A}}^{\mathsf{#1}}}
\newcommand{\Dlocs}{\mathit{L}_{\mathrm{nd}}}
\newcommand{\transitions}{{\rightarrow}}

%%% 
\newcommand{\plocs}{\mathit{L}_{\mathrm{p}}}
\newcommand{\tlocs}{\mathit{L}_{\mathrm{t}}}

\newcommand{\lin}{\loc_\mathrm{init}}
\newcommand{\lout}{\loc_\mathrm{out}}
\newcommand{\val}[1]{\mbox{\sl Val}_{#1}}

\newcommand{\pvars}{V_\mathrm{p}}
\newcommand{\rvars}{V_{\mathrm{r}}}
\newcommand{\pre}{\mathrm{pre}}

\newcommand{\sle}{\sqsubseteq}
\newcommand{\sge}{\sqsupseteq}

\newcommand{\lfp}{\mathrm{lfp}}
\newcommand{\gfp}{\mathrm{gfp}}

\newcommand{\rdvarjdis}{\mathcal D}
\newcommand{\sampset}{\textit{supp}}

\newcommand{\upd}{\mbox{\sl upd}}
\newcommand{\wet}{\mbox{\sl wt}}
\newcommand{\transset}{\mathfrak T}
\newcommand{\valin}{\pv_{\mathrm{init}}}
\newcommand{\ret}{\mbox{\sl ret}}

\newcommand{\win}{w_{\mathrm{init}}}

\newcommand{\sampdpd}{\overline{\Upsilon}}

\newcommand{\outmap}{\text{O}}
\newcommand{\sat}[1]{\langle #1 \rangle}
\newcommand{\monoid}{\mbox{\sl Monoid}}
\newcommand{\handelmanformat}{(\dagger)}

\newcommand{\trunc}{\mathcal{B}}

\newcommand{\ewt}{\mbox{\sl ewt}}
\newcommand{\statemap}{\text{St}}

\newcommand{\valrd}{{\mathbf{r}}}
\newcommand{\frmloc}{\ell^{\mathrm{src}}}
\newcommand{\toloc}{\ell^{\mathrm{dst}}}

\newcommand{\monomials}{\mathbf{M}}

\begin{document}

\begin{frontmatter}

\title{Optimal Alignment of Temporal Knowledge Bases}

\author[A,C]{\fnms{Oliver}~\snm{Fern\'andez Gil}
%	\orcid{....-....-....-....}
\thanks{Corresponding Author. Email: oliver.fernandez@tu-dresden.de}
}
\author[B]{\fnms{Fabio}~\snm{Patrizi}
%	\orcid{....-....-....-....}
\thanks{Email: patrizi@diag.uniroma1.it}}
\author[B]{\fnms{Giuseppe}~\snm{Perelli}
%	\orcid{....-....-....-....}
\thanks{Email: perelli@di.uniroma1.it}} % use of \orcid{} is optional
\author[A,C]{\fnms{Anni-Yasmin}~\snm{Turhan}
%	\orcid{....-....-....-....}
\thanks{Email: anni-yasmin.turhan@tu-dresden.de}} % use of \orcid{} is optional


\address[A]{Theoretical Computer Science, TU Dresden, Germany}
\address[B]{Sapienza University of Rome}
\address[C]{Center for Scalable Data Analytics and Artificial Intelligence (ScaDS.AI) Dresden/Leipzig, Germany}

\begin{abstract}
%
%	old
%	
%Trace alignment is a well-investigated problem underlying many AI applications. Given a trace, i.e.~a sequence of observations, and a property $\varphi$ in Linear Temporal Logic, it amounts to finding a minimal-cost modification of the trace that satisfies $\varphi$.
%While observables are typically modeled as propositions, temporal knowledge bases (TKBs) 
%written in description logics 
%%(DLs) 
%can model the temporal evolution of relational data, enriched by background knowledge,  thus facilitating more expressive modeling.
%%%
%This paper introduces a new variant of the alignment problem defined over TKBs, written in the description logic
%%the DL  
%\alc, and temporal conjunctive queries for the property $\varphi$.
%For this variant, we devise a solution 
%% \\
%technique to compute (cost-optimal) alignments of TKBs.
%%%
%%  provide computation algorithms for (cost optimal) alignments of temporal KBs. 
%%
%
%   NEW
%
Answering temporal CQs over temporalized Description Logic knowledge bases (TKB) is a main technique to realize ontology-based situation recognition. In case the collected data in such a knowledge base is inaccurate, important query answers can be missed. 
In this paper we introduce the TKB Alignment problem, which computes a variant of the TKB that minimally changes the TKB, but entails the given temporal CQ and is in that sense (cost-)optimal. We investigate this problem for \alc TKBs and conjunctive queries with LTL operators and devise a solution technique to compute (cost-optimal) alignments of TKBs that extends techniques for the alignment problem for propositional LTL over finite traces.
\end{abstract}

\end{frontmatter}

\section{Introduction}
\label{sec:introduction}

The recent surge of Large Language Models (LLMs), such as GPT-3.5/4~\cite{bubeck_sparks_2023}, PaLM~\cite{chowdhery_palm_2022}, FLAN-T5~\cite{chung_scaling_2022}, and Alpaca~\cite{taori_stanford_2023}, has shown a promising trend of large pre-trained models to do a variety of tasks in a zero-shot setting (\ie without any new training data). Example tasks include question answering~\cite{omar2023chatgpt,robinson2023leveraging}, logic reasoning~\cite{wei_chain--thought_2023,zhou_least--most_2023}, machine translation~\cite{brants2007large,gulcehre2017integrating} \etc\ 
A number of experiments have revealed that, built on hundreds of billions of parameters, these LLMs have started to show the capability to understand the human common sense beneath the natural language and do proper reasoning and inference accordingly~\cite{bubeck_sparks_2023,nori_capabilities_2023}.

Among different applications, one particular question yet to be answered is how well LLMs can understand human mental health states through natural language.
Mental health problems represent a significant burden for individuals and societies worldwide.
A recent report suggested that more than 20\% of adults in the U.S. would experience at least one mental disorder in their lifetime~\cite{mental2022state} and 5.6\% of adults experienced a serious psychotic disorder that significantly impairs functioning~\cite{mental2023stats}. The global economy loses around \$1 trillion annually in productivity due to depression and anxiety alone~\cite{mentalcost2023}.

In the past decade, there has been a plethora of research in natural language processing (NLP) and computational social science on detecting mental health issues via online text data such as social media~(\eg \cite{guntuku_detecting_2017,eichstaedt2018facebook,coppersmith_clpsych_2015,de_choudhury_social_2013,de_choudhury_mental_2014}). However, most of these studies have focused on building domain-specific machine learning (ML) models (\ie one model for one particular task, such as stress detection~\cite{nijhawan2022stress,guntuku2019understanding}, depression prediction~\cite{eichstaedt2018facebook,tadesse2019detection,xu_leveraging_2019}, or suicide risk assessment~\cite{de_choudhury_discovering_2016,coppersmith2018natural}). Even for traditional pre-trained language models such as BERT, it needs to be finetuned for specific downstream tasks~\cite{devlin_bert_2019,liu_roberta_2019}.
Since natural language is a major component of mental health assessment and treatment~\cite{sharma2018mental,gkotsis2016language}, LLMs might be a potentially powerful tool to understand end-users' mental states based on the language users' wrote. These instruction-finetuned and general-purpose models can understand a variety of inputs and obviate the need to train multiple models for different tasks. Thus, we can envision using one LLM for a variety of mental-health-related tasks, such as multiple question-answering, reasoning, and inference.
Such a vision opens up a wide range of opportunities for UbiComp, Human-Computer Interaction (HCI), and mental health communities, such as online public health monitoring systems~\cite{patel2018psyheal,graham2019artificial}, intelligent assistants for mental counselors and supporters~\cite{sharma_towards_2021,sharma_humanai_2023}, mental-health-aware personal chatbots~\cite{abd2021perceptions,denecke2020mental}, to just name a few.
However, there is a lack of investigation into understanding, evaluating, and improving the capability of LLMs for mental health prediction tasks.

There are few very recent studies on the evaluation of LLMs (\eg ChatGPT) on mental-health-related tasks, most of which are in zero-shot settings with simple prompt engineering~\cite{yang_evaluations_2023,amin_will_2023,lamichhane_evaluation_2023}. Researchers have shown preliminary results that LLMs have some initial capability of predicting mental health disorders with natural language with some promising but still limited performance compared to state-of-the-art domain-specific NLP models~\cite{yang_evaluations_2023,lamichhane_evaluation_2023}.
This remaining gap is expected since existing general-purpose LLMs are not specifically trained on mental health tasks.
However, to achieve our vision of leveraging LLMs for mental health support and assistance, we need to answer the research question: \textbf{How to empower LLMs with more mental health domain knowledge and become an expert}?

We conducted a series of experiments with multiple LLMs, including Alpaca~\cite{noauthor_stanford_2023}, Alpaca-LoRA~\cite{hu_lora_2021}, and GPT-3.5~\cite{noauthor_introducing_2022}.
Considering the data availability, we focused on online social media data with high-quality human-generated mental health labels.
Our experiments contained three stages: (1) zero-shot prompting, where we experimented with various prompts related to mental health, (2) few-shot prompting, where we inserted examples into prompt inputs, and (3) instruction-finetuning, where we finetuned LLMs on multiple mental-health datasets with various tasks.

Our results indicate that zero-shot obtained promising but limited performance on multiple mental health prediction tasks across all models. GPT-3.5 had relatively better results since it has a larger scale. But their performance is still far from task-specific models. 
Meanwhile, providing a few shots in the prompt can improve the model performance to some extent ($\overline{\Delta}$ = 4.7\%), but the advantage is limited.
Finally and most importantly, we found that instruction-finetuning can significantly improve the model performance across multiple mental-health-related tasks at the same time. Our finetuned Alpaca, namely \textbf{Mental-Alpaca}, significantly outperforms the original GPT-3.5 ($\times$25 times of model size) by an average of 16.7\% on balance accuracy. 
Meanwhile, Mental-Alpaca can further perform on par with the task-specific state-of-the-art Mental-RoBERTa~\cite{ji_mentalbert_2021}. It is noteworthy that Mental-RoBERTa needs to be trained on each task individually, 
while our Mental-Alpaca can solve different tasks off the shelf. 
% We open-source our training code and model at [github link].
Our experiments present the first comprehensive evaluation of various techniques to enhance LLMs' capability in the mental health domain.

The contribution of our paper can be summarized as follows:
\begin{s_enumerate}
\item We present the first comprehensive evaluation of prompt engineering, few-shot, and finetuning techniques on multiple LLMs in the mental health domain.
\item With online social media data, our results reveal that finetuning on a variety of datasets can significantly improve LLM's capability on multiple mental-health-specific tasks simultaneously.
% We release our model \textbf{Mental-Alpaca} as the first open-source LLM targeted at mental health prediction tasks.
\item We provide a few technical guidelines for future researchers and developers on turning LLMs into experts in specific domains.
\end{s_enumerate}

%
\section{Preliminaries}
\label{sec:preliminaries}

% Figure environment removed
\noindent

\noindent
\textbf{Fuzz Driver Basics} \tab 
% key aspects of a fuzz driver
% The prerequisites can include the initialization of the target API arguments and the setup of the correct execution context.
The key components of a fuzz driver are illustrated in Figure~\ref{fig:fuzz-driver-internal}.
A typical fuzz driver includes at least three components: prerequisites initialization, execution, and post-cleaning of the target API, as mentioned in lines 3, 4, and 7.
In addition, there are three optional components listed in lines 2, 5, and 6 that can improve a driver's effectiveness.
The component mentioned in line 2 allows the driver to reject inputs with large sizes to reduce execution costs, or to split input data into several parts for testing multiple arguments of the target API.
The component mentioned in line 5 enables the driver to call additional APIs to trigger more program behaviors, which helps reveal more bugs.
Finally, the component mentioned in line 6 allows the driver to add semantic oracles to find more logical bugs.
These oracles are similar to \texttt{assert} statements used in unit tests, which abort execution when certain properties of the program are not satisfied.
Since the fuzz drivers are extensively executed with randomly mutated input data, there is a high requirement on the correctness and robustness of its API usage.
The incorrect or unrobust usage can lead to both false positives and negatives.
For instance, if a driver failed to feed the mutated data into the API, it can never find any bug inside the target.
Or if a driver passed an incorrectly initialized argument to the API, false crashes may be raised.
% In this paper, an effective fuzz driver represents the drivers which have correct API usage and produce no false positives.
% Precisely validating the effectiveness of fuzz drivers is crucial for evaluating fuzz driver generation methods.
% However, general validation techniques do not work well due to the diverse semantics on the API usages.

\noindent
\textbf{Minimum Requirements of an Effective Fuzz Driver} \tab 
% what is an effective fuzz driver
To be effective, a fuzz driver must correctly use the API and produce no false positive results.
The minimal requirements for an effective fuzz driver includes satisfying the necessary control flow dependencies and initializing the arguments correctly.
Argument initialization can be one of the following cases (in the order of simplicity):
\ding{182} \textbf{C1}: If the argument value is irrelevant or should be a naive value such as \texttt{0} or \texttt{NULL}, a variable is declared or a literal constant is used directly;
\ding{183} \textbf{C2}: If the argument is supposed to be a macro or a global variable that is already defined in common libraries or the target API's project, it is located and used;
\ding{184} \textbf{C3}: If creating the argument requires the use of common library APIs, such as creating a file and writing specific content, common practices are followed;
\ding{185} \textbf{C4}: If initializing the argument requires the output of other APIs within the project, those APIs are initialized first following the above initialization cases.

\noindent
\textbf{LLM-Based Code Generation} \tab 
LLMs provide a natural language interface that allows users to generate code through conversational queries and answers.
With this interface, code generation tasks can be completed more efficiently and with less technical expertise required.
In this study, prompt represents the content of a single query while the conversation represents one or more rounds of queries and answers sharing the same communication context.

% The type of prompt involved in the study is \textit{prefix prompt}~\cite{prompt-engineering-survey}, which expects the LLM to continue the content
% llm is query based, in a conversational style
% what is prompt, what is query, what is conversation
% we only consider the second form of prompt

%
%!TEX root = ecai-main.tex


\section{The TKB Alignment Problem}
\label{sec:tkb-align}

We generalize the verification problem of TQE
to a \emph{synthesis} version, consisting
in \emph{minimally} modifying the sequence 
$\Aseq$ of a TKB $\tkb=(\Tmc,\Aseq)$, 
to obtain a TKB $\tkb'=(\Tmc,\Aseq')$, s.t.~$\tkb'\models\tquery$. 
%% 
Observe that if $\tkb\models\varphi$, the problem amounts to checking TQE.
%%
To define the problem, we  formalize next the notions of ABox- and TKB-modification, and minimality.

To modify the ABoxes occurring in a TKB, we consider two kinds of ABox operations, namely 
\emph{insertion} and \emph{removal} of a (concept or role) assertion $\assert$, 
respectively denoted as $\ins{\alpha}$ and $\rem{\alpha}$.
%
The result of applying such operations to an ABox $\Amc$ is given by the function $\apply$:
\begin{inparaenum}[(\it i)]
    \item $\apply(\ins{\assert}, \Amc) =\Amc\cup\set{\assert}$;
    \item $\apply(\rem{\assert}, \Amc) =\Amc\setminus\set{\assert}$.
\end{inparaenum}
%
An \emph{ABox-modification} is a (possibly empty) sequence $\mods=\op_0\cdots \op_n$ of ABox operations 
$\op_i$. By $\emptymods$ we denote the empty ABox-modification.
%
The semantics of applying an ABox-modification $\mods$ to an ABox $\Amc$ is obtained by 
inductively extending $\apply$ to sequences of operations: 
%
\begin{inparaenum}[(\it i)]
	\item $\apply(\emptymods, \Amc) = \Amc;$
	\item $\apply(\op\sep\mods,\Amc) = \apply(\mods,\apply(\op,\Amc))$, where $\sep$ is the concatenation operator.
\end{inparaenum}

We assume every operation $\op$ has a strictly positive cost $\cost(\op)\in\mathbb{R^+}$.
The \emph{cost} of an ABox-modification 
$\mods=\op_1\cdots \op_n$ is defined as 
$\cost(\mods) :=\sum_{i=1}^n \cost(\op_i)$, with $\cost(\emptymods)=0$.

In addition to modifying its ABoxes, 
a TKB can be modified by adding or removing ABoxes.
Let $\atmods=\set{\fix{\mods},\add{\mods}, \del \mid \mods \text{ is an ABox-modification}}$
be the set of \emph{atomic TKB-modifications}. 
Intuitively, 
$\fix{\mods}$ stands for the modification of an ABox through the application of $\mods$,
$\add{\mods}$ for the addition of the empty ABox followed by the application of $\mods$, 
and $\del$ for ABox deletion. 

A \emph{TKB-modification} is a finite sequence $\tmods$ of atomic TKB-modifications, with $\emptymods$ denoting the empty TKB-modification.
Notice that,  by a slight abuse of notation, we use $\emptymods$ to denote both the empty ABox-modification and the empty TKB-modification; the intended 
meaning is clear from the context.
%%

The result of applying a TKB-modification 
$\tmods$ to a sequence %of ABoxes 
$\Aseq=\Amc_0\cdots\Amc_\ell$ is the sequence $\apply(\tmods,\Aseq)$, % of ABoxes, 
inductively defined as follows, where 
$\emptyAseq$ denotes the empty sequence of ABoxes,
$\emptyA$ the empty ABox, and $\Amc|\Aseq=\Amc\Amc_0\cdots\Amc_\ell$:

\begin{itemize}
    \item $\apply(\varepsilon,\Aseq) = \Aseq$;
  
    \item $\apply(\fix{\mods}, \emptyAseq) = \emptyAseq$;
  
    \item $\apply(\fix{\mods} \sep \tmods, \Amc|\Aseq) = \apply(\mods,\Amc)|\apply(\tmods,\Aseq)$;
    \item $\apply(\add{\mods}\sep\tmods, \Aseq) = \apply(\mods,\emptyA)|\apply(\tmods,\Aseq)$;
	
	\item $\apply(\del \sep \tmods, \emptyAseq)) = \apply(\tmods, \emptyAseq)$;
	
	\item $\apply(\del \sep \tmods, \Amc|\Aseq) = \apply(\tmods, \Aseq)$.
\end{itemize}
%%
For a given TKB $\tkb = (\Tmc, \Aseq)$ and a TKB-modification $\tmods$, 
we define $\apply(\tmods,\tkb) = (\Tmc, \apply(\tmods, \Aseq))$.
%%
The cost function naturally extends to TKB-modifications and ABox sequences:
%%
\begin{itemize}
	\item $\cost(\emptytmods, \Aseq) = 0$;
	\item $\cost(\fix{\mods} \sep \tmods, \emptyAseq) = \cost(\tmods,\emptyAseq)$;
	\item $\cost(\fix{\mods} \sep \tmods, \Amc|\Aseq) = \cost(\mods) + \cost(\tmods,\Aseq)$;
	\item $\cost(\add{\mods} \sep \tmods, \Aseq) = 1+ \cost(\mods) + \cost(\tmods,\Aseq)$;
	\item $\cost(\del \sep \tmods, \emptyAseq) = \cost(\tmods, \emptyAseq)$;
	\item $\cost(\del \sep \tmods, \Amc|\Aseq) = \big(\sum_{\assert \in \Amc} \cost(\rem{\assert})\big) + 1 + \cost(\tmods, \Aseq)$.
\end{itemize}
The cost of $\add{\mods}$ is that of adding the empty ABox (taken as 1) and applying $\mods$ to it; similarly, the cost of $\del$ is that of emptying 
the ABox, by removing all of its assertions, and removing the resulting empty ABox (i.e., 1).
For a TKB $\tkb=(\Tmc,\Aseq)$ and a TKB-modification $\tmods$, we let $\cost(\tmods,\tkb) =\cost(\tmods,\Aseq)$.

We can now introduce the TKB-alignment problem.
%%
\begin{definition}[TKB Alignment] \label{def:tkb-alignment}
Given a \TKB $\tkb$ and a TCQ $\varphi$, %% s.t.~$\tkb \not \models \varphi$, 
the \emph{\TKB-alignment} problem
 consists in finding a minimal-cost TKB-modification $\tmods^*$ (if any) 
s.t.~$\apply(\tmods^*,\tkb) \models \varphi$.
\end{definition}
%%

Observe that, for every TKB-modification $\tmods$ and TKB $\tkb$, 
it holds that $\apply(\tmods,\tkb) = \apply(\tmods \cdot \fix{\epsilon},\tkb)$ and
$\cost(\tmods, \tkb) = \cost(\tmods \cdot \fix{\epsilon}, \tkb)$.
That is, appending a sequence of $\fix{\epsilon}$ to $\tmods$ does not affect the result or
the cost of modifying $\tkb$.
Thus, we can always extend $\tmods$ 
to guarantee that the combined number of occurrences of 
deletions $\del$ and fixes $\fix{\mods}$ in $\tmods$ equals at least the number of ABoxes in $\Aseq$.
For technical convenience, from now on, we assume this is the case for every 
$\tmods$.

\subsection{Solving TKB Alignment}
% \anni{Wouldn't it be the better structure to first describe the \enquote{decomposition} into propositional abstraction and inner part and then get to the automata constructions?}

Our solution approach consists in reducing TKB Alignment to Shortest Path. To this end, we construct a graph, called \emph{Minimal-instantiation Graph}, with each edge labelled by an atomic TKB-modification 
and its corresponding cost, s.t.~every shortest path from a suitably defined \emph{initial} node to one node from a (suitably defined) \emph{target} set, represents an optimal solution to the original TKB Alignment instance.
The construction of such a graph is based on several intermediate structures, which we present and discuss next.

Consider a TKB $\tkb=(\Tmc,\Aseq)$ and a TCQ $\tquery$. We start with the construction of a DPA intended to accept the set of models of $\tquery$.
%%Observe that, since traces are sequences of FO interpretations, which are infinitely many (for infinite $\Delta$),
%%such a DPA cannot exist, as DPAs have finite input alphabets.
To this end, we adopt an approach similar to that of~\cite{BaaderBL15}, which uses
the  \emph{propositional abstraction} of $\tquery$.
%
%%Let $\bcqs(\tquery)$ be the (finite) set  of \bcq s occurring in $\tquery$.
If we view every \bcq~$\query\in\PBCQ$ as a proposition $\pquery$, then $\tquery$ can be viewed as an \ltl formula.
This is called the \emph{propositional abstraction} of $\tquery$, denoted $\ptquery$.
%%, and let $\pBCQ$ be the set of all propositionalized \bcq s from $\BCQ$.
Obviously, $\props(\ptquery)$ is the set of all propositions $\pquery$ occurring 
in $\ptquery$, each corresponding to exactly one BCQ $\query\in\PBCQ$.

Since $\ptquery\in\ltl$, we can now use the \buchi automaton (BA) construction 
of~\cite{Var95}, to obtain a BA that recognizes $\L(\ptquery)$ and then use the 
BA-to-DPA construction of~\cite{Pit07} to obtain the 
DPA $\PA_{\ptquery}=(2^{\props(\ptquery)},Q,q_0,\delta,\col)$ of $\ptquery$.
%%
The importance of $\PA_{\ptquery}$ lies in the fact that, 
although it reads input words $\word=\Phi_0 \Phi_1 \cdots\in {(2^{\props(\ptquery)})}^\omega$
and not FO traces $\seqI=\Imc_0\Imc_1\cdots$, it fully characterizes $\L(\varphi)$, 
as discussed below.

For $\Phi\in 2^{\props(\ptquery)}$, let 
$\chformula(\Phi)=\bigwedge_{\pquery \in \Phi} \query \wedge \bigwedge_{\pquery \in \props(\ptquery) \setminus \Phi} \neg \query$.
Since the conjuncts in $\chformula(\Phi)$ are possibly negated \bcq s from $\PBCQ$, and
not propositional abstractions,
$\chformula(\Phi)\in\bcBCQ$.
If $\chformula(\Phi)$ is consistent, i.e., admits at least one model,
$\Phi$ is called a \emph{type}.
When $\Imc\models\chformula(\Phi)$, we call $\Phi$ the \emph{type} of $\Imc$. 
This notion naturally extends to traces by
defining the \emph{trace type} of a FO trace 
$\seqI=\Imc_0\Imc_1\cdots$ 
as the word $\word=\Phi_0\Phi_1\cdots\in{(2^{\props(\ptquery)})}^\omega$ 
s.t.~$\Phi_i$ is the type of $\Imc_i$, for all $i\geq 0$.
%%
We have the following result.

\begin{restatable}{lemma}{typing}
	\label{lem:typing}
    Every FO interpretation has a unique type and every type admits an FO interpretation.
    Moreover, 
    every FO trace has a unique trace type and every trace type admits an FO trace.
\end{restatable}
The following result relates $\L(\PA_{\ptquery})$ and $\L(\tquery)$.
%
\begin{restatable}{theorem}{thmbridge}\label{thm:bridge}
Consider a TCQ $\tquery$.
For every FO trace $\seqI$ of type $\word$, 
it holds that $\seqI\in\L(\tquery)$ iff $\,\word\in\L(\PA_{\ptquery})$.
\end{restatable}
%%This result allows us to see $\PA_{\ptquery}$ as an automaton accepting $\L(\tquery)$.
%%Indeed, given $\seqI$, checking whether $\seqI\in\L(\tquery)$ is equivalent to checking
%%whether the type $\word$ of $\seqI$ is s.t.~$\word\in\L(\PA_{\ptquery})$.
%%Thus, given a trace $\seqI$, we say that $\PA_{\ptquery}$ \emph{accepts} $\seqI$ iff
%%$\PA_{\ptquery}$ accepts the type $\word$ of $\seqI$. In this sense, we can 
%%say that $\PA_{\ptquery}$ recognize $\L(tquery)$.


Since $\PA_{\ptquery}$ is independent of $\tkb$, it cannot 
be used to search for the desired minimal-cost modification.
For this, we can use a deterministic finite-state automaton (DFA) 
$\DFA$,
called the \emph{repair-template} DFA for $\tkb$ and $\tquery$.
The definition of $\DFA$ requires an auxiliary DPA, 
called the $\Tmc$-\emph{reduct} of $\PA_{\ptquery}$, 
to define the final states of $\DFA$.


\begin{definition}[$\Tmc$-reduct of $\PA_{\ptquery}$]\label{def:reduct}
Given a TBox $\Tmc$ and a TCQ $\tquery$, let 
$\PA_{\ptquery}=(2^{\props(\ptquery)},Q,\delta,q_0,\col)$. 
The $\Tmc$-reduct of $\PA_{\ptquery}$ 
is the DPA 
$\PA_{\ptquery}^{\Tmc}=(2^{\props(\ptquery)},Q^\Tmc,\delta^{\Tmc},q_0,\col^{\Tmc})$ 
s.t.:
\begin{itemize}
	\item $Q^\Tmc=Q\cup\set{q^*}$, with $q^*\notin Q$;
	\item $\delta^\Tmc(q,\Phi)=q'$, iff either:
		\begin{itemize}
			\item $\chformula(\Phi)$ is satisfiable wrt $\Tmc$ and $\delta(q,\Phi)=q'$; or
			\item $\chformula(\Phi)$ is not satisfiable wrt $\Tmc$ and $q'=q^*$; or
			\item $q=q'=q^*$;
		\end{itemize}
	\item $\col^\Tmc(q^*)=1$ and $\col^{\Tmc}(q)=\col(q)+1$, for all $q\in Q$.
\end{itemize}
\end{definition}
%%

\begin{restatable}{lemma}{reductcomplexity}\label{lem:t-reduct-complexity}
The $\Tmc$-reduct of $\PA_{\ptquery}$ can be computed in doubly exponential time and has doubly exponential size wrt \tquery.
\end{restatable}


Intuitively, the $\Tmc$-reduct of $\PA_{\ptquery}$ accepts a 
trace type $\word$ iff there exists a FO trace $\seqI$ of type $\word$ 
that does not satisfy $\tquery$ and contains only interpretations satisfying $\Tmc$.
%%
Let $\Acc(\PA^\Tmc_{\ptquery})$ be the acceptance set of $\PA^\Tmc_{\ptquery}$. We have the following.

\begin{restatable}{lemma}{lemtail}\label{lem:tail}
	
	Consider a finite sequence $\word'=\Phi_0 \cdots \Phi_{k - 1}$ of 
	types and 
	the finite run $\ppath=q_0 \trans{\Phi_0} \cdots \trans{\Phi_{k - 1}}q_k$ induced in $\PA_{\ptquery}$.
	Then,  
	$q_k \notin \Acc(\PA^\Tmc_{\ptquery})$ iff
	for every type $\word = \Phi_0 \cdots \Phi_{k - 1} \Phi_{k} 
	\Phi_{k + 1} \cdots$, having $\word'$ as a prefix, and for all 
	traces $\seqI = \Imc_0 \Imc_1 \cdots$ 
	of type $\word$, if, for all $i \geq k$, it holds that 
	$\Imc_i\models\Tmc$, then $\seqI \models \tquery$.
		
\end{restatable}
%%
Recall that we are looking for a TKB $\tkb' = (\Tmc, \Aseq' = \Amc_0\cdots\Amc_{\ell'})$, 
obtained as modification of $\tkb=(\Tmc,\Aseq)$, 
s.t.~$\tkb' \models \tquery$, i.e., all models $\seqI$ of $\tkb'$ 
(Def.~\ref{def:tkb-model}) are s.t.~$\seqI \models \tquery$ (Def.~\ref{def:tcq-entailment}).
%%
Lemma~\ref{lem:tail} implies that every model $\seqI$ of $\tkb'$ must belong to some trace type $\word_{\seqI}$ whose induced run in $\PA_{\ptquery}$ touches some $q_k \notin \Acc(\PA^\Tmc_{\ptquery})$, for $k = \ell' + 1$.
%%
Based on this, we next define the \emph{repair-template} DFA.
%%
\begin{definition}[Repair-template DFA]\label{def:rtdfa}
	Given a TKB $\tkb=(\Tmc,\Aseq)$ with $\Aseq=\Amc_0\cdots\Amc_\ell$ 
	and a TCQ $\tquery$, let $\PA_{\ptquery}^{\Tmc}=(2^{\props(\ptquery)},Q^\Tmc,\delta^{\Tmc},q_0,\col^{\Tmc})$ 
	be the $\Tmc$-reduct of $\PA_{\ptquery}=(2^{\props(\ptquery)},Q,\delta,q_0,\col)$.

	The \emph{repair-template DFA (RT-DFA) for $\tkb$ and $\tquery$} is the 
	DFA $\DFA=(\alphabet,S,s_0,\gamma,F)$ s.t.:
	\begin{itemize}
		\item $\alphabet= (\set{\fixsym,\addsym}\times2^{2^{\props(\ptquery)}})\cup\set{\delsym}$ is the alphabet;
		\item $S=2^Q\times\set{0,\ldots,\ell+1}$ is the set of states;
		\item $s_0=(\set{q_0},0)$ is the initial state;
		\item $\gamma: S \times \alphabet \ra S$ is the transition function 
		s.t.~$\gamma((Z,i),X)=(Z',i')$ iff either:
			\begin{enumerate}[leftmargin=*]
			    \item 
			        $X=\delsym$, $Z=Z'$, and $i'=\min\set{i + 1, \ell + 1}$; or
			\item all of the following hold:
				\begin{enumerate}
					\item\label{it:dfa-1} $X=(\sigma,\Upsilon)$, with $\sigma\in\set{\fixsym,\addsym}$;
					\item $q'\in Z'$ iff $\delta(q,\Phi)=q'$, for $q\in Z$ and $\Phi\in\Upsilon$;
					\item\label{it:dfa-2} there exists an ABox $\Amc$ consistent with $\Tmc$ s.t.:
					$(\Tmc,\Amc)\models 
	                        			\bigvee_{\Phi \in \Upsilon}\chformula(\Phi)\land
	                        			\bigwedge_{\Phi \not\in \Upsilon}\lnot\chformula(\Phi)$;
    			        		\item $i'=
							\begin{cases}
								\min\set{i+1,\ell+1}, \text{ if } \sigma=\fixsym\\
								i, \text{ if } \sigma=\addsym
							\end{cases}
						$
				\end{enumerate}
			\end{enumerate}
		\item $F = \set{(Z, \ell + 1) \in S \mid Z\cap\Acc(\PA_{\ptquery}^{\Tmc})=\emptyset}$ is
			the set of final states.
	\end{itemize}
\end{definition}
%%

\begin{restatable}{lemma}{rtdacomplexity}\label{lem:rtda-complexity}
The RT-DFA for a TKB  $\tkb$ and a TCQ $\tquery$ can be computed in triply exponential time and has triply exponential size wrt $\tquery$.
\end{restatable}

Observe that the right-hand side expression of the entailment ($\models$) in
Item~\ref{it:dfa-2} above is a Boolean combination of \bcq s.
The purpose of the RT-DFA is to capture the solution space of TKB Alignment for 
$(\Tmc,\Aseq)$ and $\tquery$, in the following sense:
\begin{inparaenum}[(\it i)]
    \item from every accepted word $w$, some TKB modification $\tmods$ can be derived s.t.~$\apply(\tmods, \tkb) \models \tquery$,
    and
    \item every TKB modification $\tmods$ s.t.~$\apply(\tmods, \tkb) \models \tquery$ can be derived from some accepted word $w$.
\end{inparaenum}
This is formalized next, by the notion of TKB-modification \emph{abstraction} and the subsequent result.
%%
\begin{definition}[TKB-modification abstraction]
		\label{def:abstraction}
    Consider a TKB $\tkb=(\Tmc, \Aseq)$, with $\Aseq = \Amc_0 \cdots \Amc_\ell$,
    a TCQ $\tquery$, and let $\DFA = (\alphabet, S, s_0, \gamma, F)$ 
    be the RT-DFA for $\tkb$ and $\tquery$.
    %%
    A word $w=w_0\cdots w_m\in\alphabet^*$, 
    inducing a finite run $\ppath=s_0\trans{w_0}\cdots\trans{w_{m}}s_{m+1}$
    in $\DFA$,
    is an \emph{abstraction} of 
    (or \emph{abstracts}) a TKB-modification
    $\tmods=\tmods_0\cdots\tmods_m$ iff, 
    for $j=0,\ldots, m$: 
    
        \begin{itemize}
            \item $w_j=\delsym$ and $\tmods_j = \del$; or
            
            \item
            $w_j = (\fixsym, \Upsilon)$ and, for $s_j=(Z,i)$,
            $\tmods_j=\fix{\mods}$, 
            with $(\Tmc, \apply(\mods, \Amc_i))\models \bigvee_{\Phi \in \Upsilon} \chformula(\Phi)\land
            \bigwedge_{\Phi \not\in \Upsilon} \lnot\chformula(\Phi)$; or
            
            \item 
            $w_j = (\addsym, \Upsilon)$ and
            $\tmods_j=\add{\mods}$, 
            with $(\Tmc, \apply(\mods, \emptyA))\models \bigvee_{\Phi \in \Upsilon} \chformula(\Phi)\land
            \bigwedge_{\Phi \not\in \Upsilon} \lnot\chformula(\Phi)$.
        \end{itemize}
        When this holds, $\tmods$ is an \emph{instantiation} of (or \emph{instantiates}) $w$.
\end{definition}

\begin{restatable}{theorem}{thmsoundcompl}\label{thm:rt-sound-compl}
	Consider a TKB $\tkb=(\Tmc,\Aseq)$, with $\Aseq=\Amc_0 \cdots \Amc_\ell$,
	a TCQ $\tquery$, and let $\DFA=(\alphabet,S,s_0,\gamma,F)$ 
	be the RT-DFA for $\tkb$ and $\tquery$.
	Then:
	\begin{enumerate}
		\item
			for every word $w \in \alphabet^*$, there exists an instantiation $\tmods$ s.t.~$w \in \L(\DFA)$ iff $\apply(\tmods,\tkb) \models \tquery$;
		\item%~\label{it:sound-comp-2}
			for every word $w \in \alphabet^*$ and every instantiation $\tmods$ of $w$,  it holds that $w \in \L(\DFA)$ iff $\apply(\tmods,\tkb) \models \tquery$;
		\item
			for every TKB-modification $\tmods$ there exists a unique abstraction $w_\tmods$ s.t.~$w_\tmods \in \L(\DFA)$ iff $\apply(\tmods,\tkb) \models \tquery$.
	\end{enumerate}
\end{restatable}
%%
%In other words, $\L(\DFA)$ partitions the set of solutions in such a way that a solution $\tmods$ belongs to the cell of a word $w$ iff $\tmods$ instantiates $w$.

Thm.~\ref{thm:rt-sound-compl} states that the language of $\DFA$ 
characterizes the set of solutions for the TKB-alignment of $\tkb$ against the 
specification $\tquery$; in particular, Item~2 ensures that every 
instantiation of some TKB-modification abstraction 
$w \in \L(\DFA)$ is a solution to TKB Alignment. 
Then, every optimal solution $\tmods^*$ is s.t.:

\begin{center}
	$\tmods^*=\argmin_\eta\set{\cost(\tmods,\tkb)\mid\tmods\text{ instantiates some } w\in\L(\DFA)}$.
\end{center}

Based on this, we can reduce the problem of finding $\tmods^*$ to that of finding a minimal path in a suitably weighted graph.

\begin{definition}[Minimal-instantiation Graph]\label{def:mmg}
	Consider a TKB $\tkb=(\Tmc,\Aseq)$, with $\Aseq = \Amc_0 \cdots \Amc_\ell$, a TCQ $\tquery$, and let $\DFA = (\alphabet, S, s_0, \gamma, F)$ be the RT-DFA for $\tkb$ and $\tquery$.
	The \emph{minimal-instantiation graph} for $\tkb$ and $\tquery$ is the weighted graph $\MMG = (N, E, \weight)$, where:
\begin{enumerate}
	\item
		$N = S$ is the finite set of nodes;
	\item
		$E \subseteq N \times \atmods \times N$, is the finite set of edges, labelled by atomic TKB-modifications;
	\item
		$\weight:E \ra \mathbb{R}^+$ is the edge weight function;
	\item\label{it:mmg-1}
		it holds that $e=((Z,i),\tmods,(Z', i'))\in E$ and $\weight(e)=c$ iff, for some $X$, $(Z',i')=\gamma((Z,i), X)$, and:
	\begin{itemize}[leftmargin=*, label=-]
		\item
			if $X = \delsym$ then $\tmods = \del$ and $c = \cost(\del,\Amc_i)$;
		\item
			if $X = (\fixsym, \Upsilon)$ then $\tmods = \\ \argmin_{\fix{\mods}} \bigset{\cost(\fix{\mods}, \Amc_i) \mid (\Tmc, \apply(\mods, \Amc_i)) \models \bigvee_{\Phi \in \Upsilon} \chformula(\Phi) \land \bigwedge_{\Phi \not\in \Upsilon} \lnot \chformula(\Phi)}$ and $c = \cost(\tmods, \Amc_i)$;
		\item
			if $X = (\addsym, \Upsilon)$ then $\tmods =\\ \argmin_{\add{\mods}} \bigset{\cost(\add{\mods}, \emptyAseq) \mid \\ (\Tmc, \apply(\mods, \emptyA)) \models \bigvee_{\Phi \in \Upsilon} \chformula(\Phi) \land \bigwedge_{\Phi \not\in \Upsilon} \lnot \chformula(\Phi)}$ and $c = \cost(\tmods, \emptyA)$.
	\end{itemize}
\end{enumerate}
\end{definition}
%%
%%
\begin{restatable}{lemma}{migcomplexity}\label{lem:mig-complexity}
The Minimal-instantiation Graph for a TKB  $\tkb$ and a TCQ $\tquery$ can be computed in triply exponential time and has triply exponential size wrt $\tquery$.
\end{restatable}

%%
The minimal-instantiation graph $\MMG$ is a graph whose edges are labelled with atomic TKB-modifications
and weighted with the corresponding cost. Through its labels, every finite path $s_0\trans{\tmods_0}\cdots\trans{\tmods_{m-1}}s_m$ 
of $\MMG$ defines an instantiation $\tmods=\tmods_0\cdots\tmods_{m}$ of some input word (not necessarily accepted)
$w=w_0\cdots  w_{m}$ of $D$.
Also the viceversa holds, i.e., every input word $w$ of $\DFA$ is an abstraction of the TKB-modification $\tmods$ 
defined by some path of $G$. 

Observe that, by Item~\ref{it:mmg-1} of Def.~\ref{def:mmg}, $\tmods$ includes only minimal-cost atomic TKB-modifications $\tmods_i$,
thus it is a minimal-cost TKB-modification among all those that instantiate the same $w$.
Moreover, recall that, by Theorem~\ref{thm:rt-sound-compl}, every solution to TKB Alignment is associated to an abstraction $w \in \L(\DFA)$.
Thus, since every such $w$ has a minimal-cost instantiation in some path of $\MMG$, we can search for the minimal-cost
solution by exploring the paths of $\MMG$. Indeed, it is enough to search for the minimal-cost paths of $\MMG$ which 
correspond to the words $w$ accepted by $\DFA$;  since the nodes of $\MMG$ correspond to the states of $\DFA$, this corresponds
to searching for a minimal-path of $G$ starting in $s_0$ and ending in some node that is an accepting state for $\DFA$.

%%Observe that every run in $\DFA$ is a path in $\MMG$ and vice-versa.
%%Moreover, the edges in $\MMG$ are labeled with atomic TKB-modifications and weighted with costs.
%%Consider an edge $e = ((Z, i), \tmods, (Z', i'))$ between two nodes $(Z, i)$ and $(Z', i')$.
%%Item~4 of the definition requires $\tmods$ to be a minimal-cost atomic TKB-modification such that $\gamma((Z,i), \word_\tmods) = (Z', i')$, with $\word_\tmods$ being the abstraction of $\tmods$.
%%Moreover, the weight $\weight(e)$ expresses the cost of such atomic TKB-modification. 
%%Essentially, the labels and weights capture the optimal modification $\tmods$ that is needed to trigger the transitions in $\DFA$.
%%
%%Notice that computing the correct labels and weights amounts to solve a \emph{local minimization} problem which we call KB-alignment.
%%Section~\ref{sec:kbalign} is devoted to formally define and solve such problem.
%%
%%For now, let us assume that we have such technique at hand.
%%Moreover, recall that, from Theorem~\ref{thm:rt-sound-compl} every solution to the TKB-alignment is associated to an abstraction $\word \in \L(\DFA)$.
%%From the construction of $\MMG$ we also know how to compute the cost of such solution $\word$, which amounts to the cost of the path $\pi$ in $\MMG$ that corresponds to the accepting run of $\word$ in $\DFA$.
%%Therefore, computing the optimal solution to the TKB alignment amounts to solving a minimum-cost path problem in $\MMG$, as it is stated in the next result.

\begin{restatable}{theorem}{thmtkbalign}\label{thm:tkb-alignment}
	Consider a TKB $\tkb=(\Tmc,\Aseq)$, with $\Aseq = \Amc_0 \cdots \Amc_\ell$, a TCQ $\tquery$, and let $\DFA=(\alphabet,S,s_0,\gamma,F)$ 
	be the RT-DFA for $\tkb$ and $\tquery$.
	A TKB-modification $\tmods=\tmods_0\cdots\tmods_m$ is an optimal solution of TKB Alignment for $\tkb$ and $\tquery$ iff 
	there exists a minimum-cost path $\pi=n_0\trans{\tmods_0}\cdots\trans{\tmods_{m-1}}n_m$ in $\MMG$, s.t.~$n_0=s_0$
	and $n_m\in F$.
\end{restatable}
%%
Thus, with $\MMG$ at hand, the search can easily be performed by, e.g., Dijkstra's algorithm.
However, in order to state the effective solvability of TKB Alignment, we must still guarantee that $\MMG$ is actually 
computable. In this respect, observe that we still need to explain how the labels and the weights of $\MMG$ can be obtained. 

By Def.~\ref{def:mmg} (Item~\ref{it:mmg-1}), 
computing the labels and the weights of $\MMG$ 
requires to solve, for every edge, one \emph{local minimization problem} of 
the form
$\argmin_{(\sigma,\mods)} \set{\cost((\sigma,\mods),\Aseq)}$, 
subject to a constraint of the form
$(\Tmc, \apply(\mods, \Amc)) \models \query$, with $\query\in \bcBCQ$.
%%
This is the \emph{KB-alignment} problem, 
which we formally define and 
solve in Section~\ref{sec:kbalign}.
%%
We report here that the problem can be solved in doubly exponential time.
This, together with the complexity results reported above, leads to the 
following.
%%
\begin{restatable}{theorem}{tkbalgsolv}\label{thm:tkb-alg-solv}
TKB Alignment is solvable in triply exponential time.
\end{restatable}

% \giuseppe{I added a short discussion on the comparison with Baader semantics.}

As mentioned above, one might consider the semantics on TCQs given in~\cite{BaaderBL15} and instantiate TKB alignment with it.
Such variation on the semantics has minimal impact on the solution technique, which can be adapted to that setting with minimal changes. Details are omitted, due to space constraints.

%This means that we can associate $w$ with the minimal cost of $\tmods_w$ among all possible instantiations of $w$, that is:
%
%\begin{center}
%	$\cost(w) = \min\{\cost(\tmods, \tkb) \mid \tmods \text{ is an instantiation of } w\}$.
%\end{center}
%
%\begin{itemize}
%    \item $w = w_0 \cdots w_m$
%    
%    \item $\cost(w) = \cost(w_0) + \cost(w_1 \cdots w_m)$
%    
%    \item $\cost(w_0) = $ vedi sotto
%\end{itemize}
%
%
%
%
%\begin{center}
%	$w^*=\argmin_{w\in\L(\DFA)}
%        \set{\cost(w)}$.
%\end{center}
%
%
%
%
%
%Assume $\tmods_w = \tmods_0 \cdots \tmods_m$ to be a minimum-cost instantiation of $w$, that is $\cost(w) = \cost(\tmods_w, \tkb)$.
%Observe that $\cost(\tmods_w, \tkb)$ is computed as the sum of the costs for the modifications  
%\begin{itemize}
%    \item if $w_j=\delsym$ then $\tmods_j=\del$ and $\cost(w_j) = \cost(\del,\Amc)$ 
%
%    \item if $w_j=(\fixsym,\Upsilon)$ then 
%        $\cost(w_j) = \min\set{\cost(\mods,\Amc)|(\Tmc, \apply(\mods, \Amc))\models \bigvee_{\Phi \in \Upsilon} \chformula(\Phi)\land
%            \bigwedge_{\Phi \not\in \Upsilon} \lnot\chformula(\Phi)}\}$
%     
%\end{itemize}
%
%
%
%\giuseppe{Continue ... }
%
%Notice that, at each reading step, we account for deleting, fixing, or adding the corresponding ABoxes.
%Such modifications come with a cost, which means we need to solve a local alignment problem to find the minimal cost, namely a KB-alignment.
%
%
%
%
%For now, let us assume that we know how to compute the (minimal) cost of deleting, fixing, or adding an ABox to entail a query of the form $\chformula(\Phi_{1}) \vee \ldots \vee \chformula(\Phi_{n})$, and so that we know how to associate a cost to each transition of the RT-DFA in Definition~\ref{def:rtdfa}.

% Then, regard $\DFA$ as a weighted graph where nodes are the states and weighted edges are obtained from the transition function $\gamma$, where the weights are the costs computed by solving KB-alignment.
% Observe that a given path $\pi$ in the graph corresponds to a TKB-modification $\tmods_{\pi}$ that is obtained by lining up all the minimal-cost modifications that are found by solving the KB-alignment on each edge in $\pi$.
% This also determines the TKB $\tkb_{\pi} = \tmods_{\pi}(\tkb)$ that is obtained from $\tkb$ by applying the TKB-modification $\tmods_{\pi}$.
% In addition, the cost $\cost(\tmods_{\pi})$ corresponds to the sum of weights in $\pi$.


%%\begin{lemma}\label{lem:repair-template-soundness}
%%    Consider a TKB $\tkb=(\Tmc,\Aseq)$ with $\Aseq=\Amc_0,\ldots,\Amc_\ell$ 
%%	and a TCQ $\tquery$, let
%%    $\DFA=(\alphabet,S,s_0,\gamma,F)$
%%    be the RT-DFA for $\tkb$ and $\tquery$.
%%    Then, for every word 
%%    $w=w_0\cdots w_m\in \L(\DFA)$ and 
%%    induced accepting path 
%%    $\ppath=s_0\trans{w_0}\cdots\trans{w_{m}}s_{m+1}$,
%%    there exists a TKB-modification 
%%    $\tmods_w=\tmods_0\cdots\tmods_m$
%%    s.t.~$\tmods_w(\tkb) \models \tquery$ and, for $j=0,\ldots, m$: 
%%    \begin{itemize}
%%        \item if $w_j=\delsym$ then $\tmods_j = \del$;
%%        \item if $w_j = (\fixsym, \Upsilon)$ and 
%%            $s_j=(Z,i)$ then 
%%            $\tmods_j=\fix{\mods}$ with $\mods$ s.t.~$(\Tmc, \apply(\mods, \Amc_i))\models \bigvee_{\Phi \in \Upsilon} \Phi$;
%%        \item if $w_j = (\addsym, \Upsilon)$ then 
%%            $\tmods_j=\add{\mods}$ with $\mods$ s.t.~$(\Tmc, \apply(\mods, \emptyA)) \models \bigvee_{\Phi \in \Upsilon} \chformula(\Phi)$.
%%    \end{itemize}
%%\end{lemma}
%%
%%\begin{lemma}\label{lem:repair-template-completeness}
%%    Consider a TKB $\tkb=(\Tmc,\Aseq)$ with $\Aseq=\Amc_0,\ldots,\Amc_\ell$ 
%%	and a TCQ $\tquery$, let
%%    $\DFA=(\alphabet,S,s_0,\gamma,F)$
%%    be the RT-DFA for $\tkb$ and $\tquery$.
%%    Then, for every TKB-modification $\tmods = \tmods_0 \cdots \tmods_m$ s.t.~$\apply(\tmods,\tkb) \models\tquery$ there exists a word $w = w_0 \cdots w_m \in \L(\DFA)$ 
%%    inducing the accepting path $\ppath=s_0\trans{w_0}\cdots\trans{w_{m}}s_{m+1}$ s.t., for $j = 0, \ldots, m$:    
%%        \begin{itemize}
%%            \item if $\tmods_j = \del$ then $w_j=\delsym$;
%%            \item if $\tmods_j = \fix{\mods}$ and $s_j = (Z,i)$ then $w_j = (\fixsym, \Upsilon)$ for $\Upsilon$ s.t.~$s_{j+1}=\gamma(s_j,((\fixsym, \Upsilon)))$;
%%            \item if $\tmods_j = \add{\mods}$ and $s_j = (Z,i)$ then $w_j = (\addsym, \Upsilon)$ for $\Upsilon$ s.t.~$s_{j+1}=\gamma(s_j,((\addsym, \Upsilon)))$.
%%        \end{itemize}
%%\end{lemma}

%\begin{theorem}\label{thm:solvingTKBalign}
%	
%	The solution to TKB-Alignment with input $\tkb$ and $\tquery$ correspond to the shortest path over $\DFA$.
%	
%\end{theorem}

%
%!TEX root = ecai-main.tex

\section{The KB-Alignment Problem}
\label{sec:kbalign}

In this section, we define the \emph{knowledge base alignment problem (KB Alignment)}, 
and show how to solve it for the query language $\bcBCQ$ and the DL $\alc$.
%

%\begin{definition}[KB-Alignment]
%	\label{def:kb-alignment}
%	Let $\Kmc=\tup{\Tmc, \Amc}$  be a knowledge base and  $\fquery$ a query in $\bcBCQ$. An ABox-modification $\mods$ is a \emph{KB-alignment for $\Kmc$ and $\fquery$} if:        
%	%
%	\begin{enumerate}
%		\item\label{prop:cons}
%		$\tup{\Tmc,\Amc'}$ is consistent, where $\Amc' = \apply(\mods,\Amc)$, and
%		%
%		\item\label{prop:entail}
%		$\tup{\Tmc,\Amc'}\models\fquery$.
%		
%	\end{enumerate}
%	%
%	The ABox-modification $\mods$ is an \emph{optimal KB-alignment for $\Kmc$ and $\fquery$} if it additionally satisfies:
%	%
%	\begin{enumerate}
%		\item[3.]
%		$\cost(\mods) \leq \cost(\mods')$ for all KB-alignments $\mods'$ for $\Kmc$ and $\fquery$.
%	\end{enumerate}
%	%            
%	The \emph{KB-alignment problem} consists in finding, given a KB $\Kmc$  and a query $\fquery \in \bcBCQ$, an optimal KB-alignment for $\Kmc$ and $\fquery$.
%\end{definition}


\begin{definition}[KB Alignment]
    \label{def:kb-alignment}
    Given a KB $\Kmc=\tup{\Tmc, \Amc}$ and  a query $\bquery \in \bcBCQ$, the 
    problem of KB Alignment consists in finding an ABox-modification $\mods$ 
    such that for $\Amc' = \apply(\mods,\Amc)$: 
    %
    \begin{enumerate}
    	\item\label{prop:cons}
    	$\tup{\Tmc,\Amc'}$ is consistent;
    	%
    	\item\label{prop:entail}
    	$\tup{\Tmc,\Amc'}\models\bquery$; and
    	%
    	\item\label{prop:optimal}
    	$\cost(\mods) \leq \cost(\mods')$ for all ABox-modifications $\mods'$ satisfying Conditions~\ref{prop:cons} and \ref{prop:entail}. 
    \end{enumerate}
    %
    An ABox-modification $\mods$ satisfying 
    Condition~\ref{prop:cons} and~\ref{prop:entail} is a 
    \emph{solution to KB Alignment} for $\tup{\Kmc,\bquery}$; if it also 
    satisfies~\ref{prop:optimal}, it is an \emph{optimal} solution.
    %    
\end{definition}
% \oliver{I find the use of ``solution'' to KB Alignment vs. ``optimal solution'' a bit confusing. One could understand the term "solution" as the answer to instance of KB Alignment, which is not the case.}
%
%
To solve an instance of KB Alignment, it is sufficient to consider ABox-modifications with operations defined using concept and role names from the input KB and query, i.e., using names from their signature. A \emph{signature} is a finite subset of $\cset \cup \rset$.  We use $\sig(X)$ to denote the set of concept and role names occurring in $X$, where $X$ can be a KB or a query.
%%
\begin{restatable}{lemma}{sigrestr}\label{lem:signature:restriction}	
%\begin{lemma}\label{lem:signature:restriction}
   Let $\Kmc$ be a KB and $\bquery \in \bcBCQ$. If KB Alignment has a solution for $\tup{\Kmc,\bquery}$, then it has an optimal solution $\mods$ where all ABox-operations $\ins{\assert}$ and $\rem{\assert}$ in $\mods$ are such that $\assert$ is an assertion defined over $\sig(\Kmc) \cup \sig(\bquery)$.
%\end{lemma}
\end{restatable}
%
An algorithm that solves KB Alignment for $\alc$ KBs and queries in $\bcBCQ$ is introduced next.

\subsection{Solving KB-Alignment}\label{sub:sec:kb:alignment}

Algorithm~\ref{alg:kb:alignment} describes our method to solve KB Alignment. 
The approach is to  
(1) compute an upper bound on the cost of optimal 
solutions (if any), %obtained from a trivial brute-force solution, 
(2) iterate over all ABox-modifications of decreasing 
cost starting from the upper bound, and 
finally output a modification that satisfies Condition~\ref{prop:cons} to \ref{prop:optimal} from 
Definition~\ref{def:kb-alignment}, i.e., which is optimal.
 
More precisely, for the input  TBox $\Tmc$ and query $\bquery$, the algorithm first computes an ABox $\Amc'$ 
such that $\tup{\Tmc, \Amc'}$ is consistent and $\tup{\Tmc, \Amc'} \models \bquery$. 
Then, a  modification $\mods^*$ from the input ABox $\Amc$ into $\Amc'$ is 
obtained which always exists:
given two ABoxes $\Amc$ and $\Amc'$, a 
\emph{trivial modification from $\Amc$ into $\Amc'$} is defined as a 
sequence $\mods = \op_1\cdots \op_k \cdots \op_n$, where 
$\op_1 \cdots \op_k$ consists of the removal of all assertions in 
$\Amc$ and $\op_{k+1} \cdots \op_n$ consists of the insertion of all 
assertions occurring in $\Amc'$.
 %
% %  \begin{itemize}
% %  	\item
% %  	given two ABoxes $\Amc$ and $\Amc'$, a \emph{trivial modification from $\Amc$ into $\Amc'$} is a sequence $\mods = \op_1\cdots \op_\ell \cdots \op_n$, where $\op_1 \cdots \op_\ell$ consists of the removal of all assertions in $\Amc$ and $\op_{\ell+1} \cdots \op_n$ consists of the insertion of all assertions occurring in $\Amc'$.
% %  	% 	
% %  	Clearly, $\Amc'=\apply(\mods,\Amc)$.% and the order of the operations in the two sub-sequences of $\mods$ does not affect the cost $\cost(\mods)$. 	 	
% %  \end{itemize}
  %  
  %
  
  The algorithm computes one trivial modification, which, by 
  Definition~\ref{def:kb-alignment}, is a (possibly non-optimal) solution, thus $\cost(\mods^*)$ realizes the first step of the approach as it  
  is an upper bound on the cost of optimal  
  solutions. 
  Then, the \emph{for-loop} enumerates all ABox-modifications with cost smaller 
  than $\cost(\mods^*)$, that satisfy Condition~\ref{prop:cons} 
  and~\ref{prop:entail} of Definition~\ref{def:kb-alignment}, and returns one ABox modification of 
  minimal cost.

By using Lemma~\ref{lem:signature:restriction}, it is not hard to show that the 
output of Algorithm~\ref{alg:kb:alignment} is always an optimal solution.
%
 \begin{restatable}{lemma}{algcorrect}\label{lem:alg:1:sound:complete}
 %\begin{lemma}\label{lem:alg:1:sound:complete}
    If Algorithm~\ref{alg:kb:alignment} returns $\mods^*$ on input $\Kmc$ and 
    $\bquery$, then $\mods^*$ is an optimal solution to KB Alignment for 
    $\tup{\Kmc,\bquery}$.
 	%
 	If Algorithm~\ref{alg:kb:alignment} returns ``no solution'', 
 	then KB Alignment has no solution for $\tup{\Kmc,\bquery}$.
 %\end{lemma}
\end{restatable}

\begin{algorithm}[t]
	\caption{KB Alignment.}\label{alg:kb:alignment}
	\textbf{Input}: An $\alc$ KB $\Kmc=\tup{\Tmc, \Amc}$ and a query $\bquery \in \bcBCQ$.
	
	\textbf{Output}: An optimal solution of KB Alignment for $\tup{\Kmc,\bquery}$, if a solution exists; or ``no solution'', otherwise.
	
	\begin{algorithmic}[1]	
		\STATE Compute an ABox $\Amc'$ s.t.~$\tup{\Tmc, \Amc'}$ is consistent, $\tup{\Tmc, \Amc'} \models \bquery$ and $\sig(\Amc') \subseteq \sig(\Kmc) \cup \sig(\bquery)$. If  no such ABox exists, \textbf{return} ``no solution'';
		\label{alg1:line:nosolution}
		%
		\STATE Define a trivial modification $\mods^*$ from $\Amc$ into $\Amc'$;\label{alg1:line:trivial}%
		%
		\STATE Let $M$ be the set of ABox-modifications $\mods$ defined over $\sig(\Kmc) \cup \sig(\bquery)$ s.t.~$\cost(\mods) < \cost(\mods^*)$;  		 
		%  		
		\FORALL{$\mods \in M$}
		%\STATE $\Amc' := \apply(\mods,\Amc)$;  		   
		\IF{($\mods$ satisfies conditions~\ref{prop:cons} and \ref{prop:entail} in Definition~\ref{def:kb-alignment}) \AND \\ \quad ($\cost(\mods) < \cost(\mods^*)$)  }\label{alg1:line:if}%
		\STATE \hspace*{0.5cm}
        $\mods^* := \mods$;
		\ENDIF
		\ENDFOR
		\RETURN{$\mods^*$};
	\end{algorithmic}
\end{algorithm}

Hence, to see that Algorithm~\ref{alg:kb:alignment} solves the KB-alignment 
problem, it remains to show that it terminates, i.e., all its steps can \emph{effectively} 
be computed. The following arguments 
%prove that this is the case 
show this for most of the 
steps:
\begin{itemize}[leftmargin=*] 		
	%
	\item
	Since  $\Amc$ and $\Amc'$ are  finite sets, a trivial modification  $\mods^*$ can easily be computed from the ABox $\Amc'$.
	%
	\item
	Consistency of $\alc$ KBs is a decidable problem (\cite{DL-Handbook-03}), as well as entailment of $\bcBCQ$-queries in $\alc$ (see Sec.~\ref{sub:sec:cqs}), hence the conditions at line~\ref{alg1:line:if} can be effectively verified for each $\mods \in M$.
	%
	\item
	The set $M$ contains only ABox-modifications defined over the finite 
	signature $\sig(\Kmc) \cup \sig(\bquery)$. Hence, given $n > 0$, $M$ 
	contains finitely many $\mods$ with $n$ ABox-operations, and each such 
	$\mods$ has cost of at least $c\cdot n$, where $c$ is the minimal cost of 
	an ABox operation defined over $\sig(\Kmc) \cup \sig(\bquery)$.
	%
	This implies that $M$ is a finite set and contains only modifications with 
	no more than $\cost(\mods^*) / c$ operations. Thus, $M$ can be visited in 
	finite time.
\end{itemize}
%
It remains to specify how to compute the initial ABox $\Amc'$ (or to determine that it does not exist).  This requires a more involved argument presented in the following.


\subsubsection{Computing the initial ABox $\Amc'$}

The computation of the initial ABox $\Amc'$ in our algorithm is closely related to the \emph{query emptiness problem} in ontology-mediated query answering.
%based data access. 
This problem was introduced and investigated in \cite{BaaderBL16} for various DLs (including $\alc$) and the query language \CQ. We define this problem here for the more general query language $\bcCQ$.
%
It uses the notion of $\Sigma$-ABox, which refers to ABoxes that use only names from a signature $\Sigma$.

\begin{definition}[$\bcCQ$-query Emptiness]\label{def:query:emptiness}
	Let $\Tmc$ be a TBox, $\Sigma$ a signature and $\bquery \in \bcCQ$. 
	The query $\bquery$ is called \emph{empty for $\Sigma$ given $\Tmc$} if
	for all $\Sigma$-ABoxes $\Amc$ such that $(\Tmc,\Amc)$ is consistent, we have 
	$\cans{(\Tmc,\Amc)}{\bquery} = \emptyset$.

	\emph{$\bcCQ$-query Emptiness} is the problem of deciding, given a TBox 
	$\Tmc$, a signature $\Sigma$, and $\bquery \in \bcCQ$, whether $\bquery$ is 
	empty for $\Sigma$ \wrt $\Tmc$. 	
\end{definition}
 
 In \cite{BaaderBL16}, it is shown that to decide \CQ-query Emptiness in $\alc$, 
 it suffices to consider a single $\Sigma$-ABox $\Amc_{\Tmc,\Sigma}$. 
 This ABox is of exponential size and can be computed (from a given satisfiable TBox $\Tmc$ and a signature $\Sigma$) in exponential time, in the size of $\Tmc$ and the cardinality of $\Sigma$. Moreover, it satisfies the following:
 %
 \begin{itemize}[leftmargin=*]
 	\item
 	the KB $(\Tmc,\Amc_{\Tmc,\Sigma})$ is consistent, and
 	%
 	\item
 	given a pure \cq\ $\query$, $\query$ is empty for $\Sigma$ \wrt $\Tmc$ iff \\ $\cans{(\Tmc,\Amc_{\Tmc,\Sigma})}{\query} = \emptyset$.
 \end{itemize}
 %
  The arguments used to prove the second property can be easily extended to pure 
	${\bcCQ}$-queries, as the following result shows. % ^The proof can be found in the  appendix.
  
  \begin{restatable}{lemma}{ASigmaenough}\label{lem:ASigmaT:is:enough:for:BbcCO}
  %\begin{lemma}\label{lem:ASigmaT:is:enough:for:BbcCO}
  	Let $\Tmc$ be a satisfiable $\alc$ TBox, $\Sigma$ a signature and $\bquery \in \bcCQ$ a pure query. Then, $\bquery$ is empty for $\Sigma$ wrt\ $\Tmc$ iff $\cans{(\Tmc,\Amc_{\Tmc,\Sigma})}{\query} = \emptyset$.
  %\end{lemma}
\end{restatable}

For \emph{non-pure} queries, query emptiness can be reduced to the case of pure queries. This can be done as follows. 
%
Let $\Tmc$ be an $\alc$ TBox, $\Sigma$ a signature, and $\bquery \in \bcCQ$ a \emph{non-pure} query with $\ind(\bquery)=\{a_1,\ldots,a_m\}$. We select $m$ \emph{fresh} concept names $A_1,\ldots,A_m$, i.e., concept names neither occurring in $\Tmc, \bquery$ nor $\Sigma$. 
%
Then, we  define an $\alc$ TBox  $\Tmc_p$, a signature $\Sigma_p$ and a query $\bquery_p\in\bcCQ$, as follows:
%
\begin{itemize}
	\item
	$\Tmc_p = \Tmc \cup \Tmc_{\disj}$, where $\Tmc_{\disj} =\{A_i \sqcap A_j \sqsubseteq \bot \mid 1 \leq i < j \leq m\}$,
	%
	%
	\item
	$\bquery_p = \bquery_x \land \query_{\disj}$, where $\bquery_x$ is obtained from $\bquery$ by replacing each $a \in \ind(\bquery)$ by a fresh free variable $x_a$, whereas $\query_{\disj}$ is the \cq\ $\query_{\disj} = A_1(x_{a_1}) \land \ldots \land A_m(x_{a_m})$, and
	%
	%
	\item
	$\Sigma_p = \Sigma \cup \{A_1,\ldots,A_m\}$.
\end{itemize}
%
%For \emph{non-pure} queries, query emptiness can be easily reduced to the case of pure queries. Namely, it is enough to consider the query $\bquery_x$ obtained from a \emph{non-pure} query $\bquery$ by replacing each $a \in \ind(\bquery)$ by a fresh free variable $x_a$.
%%
%This is shown in the following lemma.	
%
The following lemma shows that testing whether $\bquery$ is empty for $\Sigma$ \wrt $\Tmc$ reduces to checking emptiness of $\bquery_p$ for $\Sigma_p$ \wrt  $\Tmc_p$. Since $\bquery_p$ is a pure query, this yields a reduction from query emptiness of \emph{non-pure} queries to the case of pure queries.
%
% . Its proof (see the appendix) describes how to use a \emph{non-empty} $\bquery_x$ to compute an extension $\Amc^\bquery_{\Tmc,\Sigma}$ of $\Amc_{\Tmc,\Sigma}$ that witnesses non-emptiness of $\bquery$.
 %
 \begin{restatable}{lemma}{removeindividuals}\label{lem:remove:individuals}
%\begin{lemma}\label{lem:remove:individuals}	
	%
	Let $\Tmc$ be an $\alc$ TBox, $\Sigma$ a signature, and $\bquery$ a query in $\bcCQ$. The following holds:
	%
	\begin{enumerate}
		\item
		$\bquery$ is empty for $\Sigma$ \wrt $\Tmc$ iff $\bquery_p$ is empty for $\Sigma_p$ \wrt $\Tmc_p$.
		%$\bquery$ is empty for $\Sigma$ \wrt $\Tmc$ iff $\bquery_x$ is empty for $\Sigma$ \wrt $\Tmc$.
		%
		%
		\item\label{point:2:lemma:nonpure:to:pure}
		%
		%If there is a $\Sigma_p$-ABox $\Amc_p$ s.t.~$\tup{\Tmc_p, \Amc_p}$ is consistent and $\cans{(\Tmc_p,\Amc_p)}{\query_p} \neq \emptyset$, then $\Amc_p$ can be transformed in linear time into a $\Sigma$-ABox $\Amc$ witnessing non-emptiness of $\bquery$ for $\Sigma$ \wrt $\Tmc$. %
		If there is a $\Sigma_p$-ABox $\Amc_p$ that witnesses non-emptiness of $\bquery_p$ for $\Sigma_p$ \wrt $\Tmc_p$, then, given $t \in \cans{(\Tmc_p,\Amc_p)}{\query_p}$, $\Amc_p$ can be transformed in polynomial time (in the size of $\Amc_p$ and $t$) into a $\Sigma$-ABox $\Amc$ witnessing non-emptiness of $\bquery$ for $\Sigma$ \wrt $\Tmc$. %
		%If there is a $\Sigma$-ABox $\Amc_x$ such that $\cans{(\Tmc,\Amc_x)}{\query_x} \neq \emptyset$, then $\Amc_x$ can be transformed in polynomial time into a $\Sigma$-ABox $\Amc$ such that $\cans{(\Tmc,\Amc)}{\query} \neq \emptyset$.		
	\end{enumerate}
	%
\end{restatable}
%\end{lemma}

Hence, if the input query $\bquery$ of Algorithm~\ref{alg:kb:alignment} is pure, by Lemma~\ref{lem:ASigmaT:is:enough:for:BbcCO} the search space for the ABox $\Amc'$ can simply be restricted to $\{\Amc_{\Tmc,\Sigma}\}$ where $\Sigma=\sig(\Kmc)\cup\sig(\bquery)$.
%
%Otherwise, Lemma~\ref{lem:remove:individuals} tells us how to obtain $\Amc'$ (if it exists). Namely, the algorithm checks whether $\bquery_x$ is non-empty for $\Sigma$ \wrt $\Tmc$, again by only looking at $\Amc_{\Tmc,\Sigma}$.
%
%If the latter is true, then  $\Amc'$ is selected as the ABox obtained from applying to $\Amc_{\Tmc,\Sigma}$ the polynomial time transformation from Lemma~\ref{lem:remove:individuals} (this transformation is part of the proof of Lemma~\ref{lem:remove:individuals}, which can be found in the Appendix).
Otherwise, Lemma~\ref{lem:remove:individuals} tells us how to obtain $\Amc'$ (if it exists). Namely, the algorithm first constructs $\Tmc_p, \bquery_p$ and $\Sigma_p$ from $\Tmc, \bquery$ and $\Sigma$. It then checks whether $\bquery_p$ is non-empty for $\Sigma_p$ \wrt $\Tmc_p$, by using $\Amc_{\Tmc_p,\Sigma_p}$.
%
If the latter is true, then  $\Amc'$ is selected as the ABox obtained from applying to $\Amc_{\Tmc_p,\Sigma_p}$ the transformation from the second statement in Lemma~\ref{lem:remove:individuals}. %(this transformation is described in the proof of Lemma~\ref{lem:remove:individuals}, which can be found in the Appendix).
%
Overall, this provides a way to compute $\Amc'$ whenever it exists.

Hence, Algorithm~\ref{alg:kb:alignment} always terminates. This, together with Lemma~\ref{lem:alg:1:sound:complete}, yields solvability of KB Alignment. A closer look at Algorithm~\ref{alg:kb:alignment} reveals that it runs in \emph{double exponential time} in the size of the input KB and query. Thus, we obtain the following result.
%
%\begin{theorem}\label{th:solvability:kb:alignment:no:nominals:CQ}
\begin{restatable}{theorem}{solvabilitykbalignment}\label{th:solvability:kb:alignment:no:nominals:CQ}
	%KB Alignment is solvable for $\alc$ and $\bcBCQ$.
	KB Alignment is solvable for $\alc$ and $\bcBCQ$ in \emph{double exponential time}.
\end{restatable}
%\end{theorem}
%

%The computational problem associated to  $\bcCQ$-query \emph{non-emptiness} is the problem of computing a $\Sigma$-ABox $\Amc$ (if it exists) that witnesses non-emptiness of a query. For a Boolean query $\bquery$, this corresponds to find a $\Sigma$-ABox $\Amc$ s.t.~$(\Tmc,\Amc) \models \bquery$.

%For a Boolean query $\bquery$, we know that $\cans{(\Tmc,\Amc)}{\bquery} \neq \emptyset$ iff $(\Tmc,\Amc) \models \bquery$.
%%
%Hence, the problem of computing the ABox $\Amc'$ in Algorithm~\ref{alg:kb:alignment} is an instance of the computational problem associated to $\bcCQ$-query non-emptiness (i.e. computing a witness $\Sigma$-ABox, if it exists), where $\Sigma=\sig(\Kmc) \cup \sig(\bquery)$.


%\begin{definition}\label{def:query:emptiness:problem}
%	Let $\Tmc$ be a TBox, $\Sigma$ a signature and $\qlang$ a query language. Then, an $\qlang$-query $\query$ is called empty for $\Sigma$ and $\Tmc$ if for all $\Sigma$-ABoxes $\Amc$ s.t.~$(\Tmc,\Amc)$ is consistent, we have $cert_{\Tmc,\Amc}(\query)=\emptyset$.
%	
%	\emph{$\qlang$-query emptiness} is the problem of deciding, given $\Tmc$, $\Sigma$ and an $\qlang$-query $\query$, whether $\query$ is empty for $\Sigma$ and $\Tmc$.
%\end{definition}


%finding $\tilde{\Amc}$ above reduces to check whether the A-Box $\Amc_\psi$ is a solution, where $\Amc_\psi$ is obtained from $\psi$ as follows:
% %
%$$
% 	\Amc_\psi := \{A(a_y) \mid A(y) \in \psi\} \cup \{r(a_y,a_z) \mid r(y,z) \in \psi\}.
%$$
%%
% The following lemma shows that it is enough to look at $\Amc_\psi$.
% 
% \begin{restatable}{lemma}{lemApsiisenough}\label{lem:A:psi:is:enough}
	% 	Let $\dl$ be a DL not including nominals, $\Tmc$ a T-Box and $\psi$ a \CQ-query. Then, an A-Box $\tilde{\Amc}$ exists iff $\Amc_\psi$ is such an A-Box.
	% \end{restatable}
% %
% Testing whether $\Amc_\psi$ satisfies the conditions required for $\tilde{\Amc}$ reduces to check whether $(\Tmc,\Amc_\psi)$ is a consistent knowledge base.  Note that $\Amc_\psi$ represents the structure of $\psi$, which means that every model of $(\Tmc,\Amc_\psi)$ entails $\psi$.
% %
% Hence, we obtain the following result.
%
%   it is shown that query emptiness is decidable for all fragments of the DL $\alci$ and $\qlang \in \{\IQ,\CQ\}$. In particular, it is shown that to decide query emptiness it suffices to consider a single A-Box $\Amc_{\Sigma,\Tmc}$. More precisely, it is shown that $\psi$ is non-empty for $\Sigma$ and $\Tmc$ iff $cert_{\Tmc,\Amc_{\Sigma,\Tmc}} (\psi) \neq \emptyset$.
%%
%Hence, the first step in the algorithm above can be effectively computed by defining $\Sigma$ as the set of concept and role names occurring in $\Tmc$ and $\psi$, computing $\Amc_{\Sigma,\Tmc}$ as described in \cite{BaaderBL16} and checking whether $cert_{\Tmc,\Amc_{\Sigma,\Tmc}} (\psi) \neq \emptyset$. From this, we obtain the following result.
%%


%Open points:
%\begin{itemize}
%\item 
%%Define cost model for modifications: a basic cost model can be based on \emph{edit} operations that are applied to $\Amc$. More precisely,
% %        \begin{itemize}
	% %            \item \emph{insert} and \emph{delete} relations between individuals in $\Amc$, i.e., edges $r(a,b)$,
	% %            \item \emph{replace} an edge $r(a,b)$ in $\Amc$ for another edge $s(a,b)$,
	% %            \item define similar operations for concept assertions $A(a)$ in $\Amc$, and
	% %            \item one question is whether we want to allow the introduction of new individuals or the deletion of existing ones, which may be necessary to obtain a path to satisfy $\psi$.
	% %        \end{itemize}
% %      All these operations have a cost. A transformation of $\Amc$ into $\tilde\Amc$ consists of a sequence of basic operations. The cost of the whole transformation can be defined as the sum of the costs of the basic operations in the sequence.

%	Some observations/examples to illustrate the cost model and its interaction with the knowledge in $\Tmc$:
%	\begin{itemize}
	%		\item I have added the requirement that $\tilde\Kmc$ should be consistent, because depending on the considered DL, the resulting $\tilde{\Kmc}$ may turn out to be inconsistent.
	%		%
	%		For instance, if $\Amc$ contains $r(a,b)$ and $\Tmc$ an inconsistency statement $\exists r \sqcap \exists s \sqsubseteq \bot$, then adding $s(a,b)$ to $\Amc$ would make $\Kmc$ inconsistent.
	%		\item The knowledge in $\Tmc$ can be relevant:
	%		
	%		\begin{tikzpicture}[font=\small]
		%		\node(T) {$\Tmc=\{\exists r \sqsubseteq \exists p.\exists s.\exists r\}$};
		%		\node(L) [below of=T,yshift=0.5cm,xshift=-0.4cm] {$\psi=s.ps.(ps)^*$};
		%		\node(A) [right of=T,xshift=2cm]{$\Amc:$};
		%		\node(a1) [right of=A,xshift=-0.5cm,fill=black,circle,inner sep=0pt,minimum size=3pt] { };
		%		\node(a2) [right of=a1, fill=black,circle,inner sep=0pt,minimum size=3pt] { };
		%		\node(a3) [right of=a2, fill=black,circle,inner sep=0pt,minimum size=3pt] { };
		%		\draw[->] (a1) to node[above]{$s$} (a2);
		%		\draw[->,dotted,red] (a2) to node[above]{$r$} (a3);
		%		\end{tikzpicture}
	%		
	%		One possibility to modify $\Amc$ to satisfy $\psi$ is to explicitly add two \emph{loop} edges labeled with $p$ and $s$ to the middle element.
	%		%
	%		However, a better solution is to just add the $r$-labelled edge. The implication in $\Tmc$ forces the middle element to have a path labelled with $ps$.
	%		
	%		\item 
	%		There may be no solution:
	%		
	%		\begin{tikzpicture}[font=\small]
		%		\node(T) {$\Tmc=\{\exists r \sqcap \exists s \sqsubseteq \bot\}$};
		%		\node(L) [right of=T,xshift=1.5cm] {$\psi=\{sr\}$};
		%		\node(A) [right of=T,xshift=3cm]{$\Amc:$};
		%		\node(a1) [right of=A,xshift=-0.5cm,fill=black,circle,inner sep=0pt,minimum size=3pt] { };
		%		\end{tikzpicture}
	%		
	%		The word $sr$ can only be obtained by adding self-edges with labels $s$ and $r$ to the unique element in $\Amc$. But this would make the resulting $\tilde\Kmc$ inconsistent. This would be different if our cost model permits insertions of new individuals.
	%	\end{itemize}
%	\item Problem should be decidable for strictly positive costs: we build a path satisfying the \tcrpq and from this we
%	can easily define $\tilde\Amc$. Details to be discussed together (but we believe this can be done by building the automaton for the \tcrpq and then finding all the direct accepting paths).(or \tcrpq?) 
%\end{itemize}
%\anni{This is definition is rather KB alignment. This could also be interesting, since it has a bit of a (weighted?) abduction flavour: The query is an observation and the old KB is not \enquote{in line} with the observation. Now the abduction problem is, how to update the KB to incorporate the observation.}

%\anni{For the approximate semantics of (C2)RPQs, the $\psi$ is (minimally) altered into a $\psi'$ s.t.\  $\mathcal{K} \models \psi'$.}		

%
%Then, for every model $\I$ of $\tilde\Kmc$ there exist $d,e\in \Delta^\I$ and $w=R_1\ldots R_\ell \in \lang(\psi)$ s.t.~$d$ is connected to $e$ in $\I$ by a role path labeled with $w$. This implies that $e \in (\exists R_1.\cdots \exists R_\ell.\top)^\I$ and that $\exists R_1.\cdots \exists R_\ell.\top$ is satisfiable wrt\ $\Tmc$.

%Consider a new ABox $\tilde\Amc_w=\{R_1(a_0,a_1),\ldots,R_\ell(a_{\ell-1},a_\ell)\}$, where $R_1\ldots R_\ell$ is one of the words $w$ satisfying the previous properties. Since $\exists R_1.\cdots \exists R_\ell.\top$ is satisfiable wrt\ $\Tmc$, the KB $\tilde\Kmc_w=(\Tmc,\tilde\Amc_w)$ is consistent\footnote{Depending on the logic, this may not be true under the UNA}. A witness of this is the interpretation $\I$ s.t.~$d \in (\exists R_1.\cdots \exists R_\ell.\top)^\I$, where each $a_i$ is mapped to the corresponding element in the path from to $e$. Furthermore, $\tilde\Kmc_w \models \psi$.
%
%As a consequence we have that: if there exists an alignment $\tilde\Kmc=(\Tmc, \tilde\Amc)$, then there is always one of the form $\tilde\Kmc_w$ for some word $w \in \L(\psi)$. 
%

%	
%!TEX root = ecai-main.tex



\section{Discussion and Future Work}
\label{sec:con}


% % %   Conclusion
TKB Alignment is a new variant of the alignment problem that admits richer state and property descriptions. Our setting uses \alc-TKBs, CQs with \ltl operators, and a cost function for the edit operations. 
%
We have shown that TKB Alignment \wrt temporal CQs is solvable, by developing computation methods for both TKB and KB Alignment.

% % % %   Discussion
The TKB-alignment problem is closely related to abduction and to computing repairs of KBs, as these tasks also change a KB to either gain a desired consequence or remove an unwanted one. However, although being active research topics, neither of the two has yet been investigated for the temporalized setting and entailment of TCQs. Furthermore, TKB Alignment requires a cost-optimal solution, which is not very common in the context of abduction or repairs.

Interestingly, TKB Alignment can also be used for relaxing temporal CQ answering. Given a tuple of individuals $\bar{a}$ which is not a certain answer of a TCQ $\phi$, solve TKB Alignment for the Boolean TCQ obtained from $\phi'$ by assigning $\bar{a}$ to the answer variables of $\phi$. The costs computed during TKB Alignment for $\phi'$ then measure the \enquote{distance} to a certain answer of the query. 


% % %   Future Work
Our initial investigation on TKB Alignment uses a unitary cost measure for the edit operations mostly to ease presentation, as our approach can handle other cost measures easily. In this work, we did not regard rigid symbols, which are left for future work.
% \todo[inline]{Do we want to mention metric time  or other extensions here?} 






\clearpage

\ack
% The work of Perelli was partially funded by MUR under the PRIN programme, grant B87G22000450001 (PINPOINT).
% The work of Patrizi and Perelli has been supported by PNRR MUR project PE0000013-FAIR.
The work of Giuseppe Perelli was partially funded by MUR under the PRIN programme, grant B87G22000450001 (PINPOINT), and by the PNRR MUR project PE0000013-FAIR.
The work of Fabio Patrizi was partially funded by MUR under the PNRR MUR project PE0000013-FAIR, the ERC Advanced Grant WhiteMech
(No. 834228), and the Sapienza Project MARLeN.


\bibliography{ecai}

\iftrue % Set this to true to enable appendix, false to disable
\clearpage
\appendix
\section{Proofs}
%!TEX root = ecai-main.tex

\subsection{Proofs from Section~\ref{sec:tkb-align}}

\typing*


\begin{proof}	
	Consider a FO interpretation $\I$ and the type defined as $\Phi_{\I} = \set{\pquery \in \props(\ptquery) \mid \I \models \query}$.
	Clearly, we have $\I \models \chformula(\Phi_{I})$ and $\I \not\models \chformula(\Phi)$ for every other type $\Phi$.
	
	On the other hand, note that every type $\Phi \in 2^{\props(\ptquery)}$ identifies a combination of positive and negative boolean BCQs that are consistent by construction, and therefore admit a model $\I$ for all of them, which means that $\I \models \chformula(\Phi)$, and so that $\I$ is of type $\Phi$.
	
	The proof easily lifts to (infinite) sequences of FO traces and types.
	
\end{proof}

\thmbridge*
%
\begin{proof}	
	Observe that from Theorem~\ref{thm:sat-bstates}, we obtain that $\word \in \L(\PA_{\ptquery})$ iff $\word \models \ptquery$, regarded as propositional \ltl.
	Therefore, we need to prove that, for every $\seqI$, it holds that $\seqI \models \tquery$ iff $\word \models \ptquery$, with $\word$ being the trace type of $\seqI$.
	We do it by induction on the structure of $\tquery$.
	
	\begin{itemize}
		\item 
			Assume $\tquery = \psi \in \BCQ$ being a boolean conjunctive query.
			It holds that $\seqI \models \psi$ iff $\seqI_{0} \models \psi$, which, from Lemma~\ref{lem:typing}, is true iff $\hat{\psi} \in \word_0$, being $\word_0$ the type of $\seqI_{0}$.
			The latter holds iff $\word \models \hat{\psi}$ and so iff $\word \in \L(\PA_{\ptquery})$.
	\end{itemize}
	%
	All the other (boolean and temporal) cases are standard proofs that build on top of the semantics of \tcq and of propositional \ltl.
	%
\end{proof}


\reductcomplexity*
\begin{proof}
Direct consequence of Thm.~\ref{thm:sat-bstates}, which states that the state space of $\PA_\tquery$ has doubly exponential size in $\tquery$, and the EXPTIME-completeness of deciding satisfiability of $\chi(\Phi)$ wrt $\Tmc$~\cite{Lutz08}.
\end{proof}


\lemtail*
%
\begin{proof}	
	We prove the two implications separately.
	First, assume that $q_k \notin \Acc(\PA^\Tmc_{\ptquery})$ and let $\word^k = \Phi_{k} \Phi_{k + 1} \cdots$ be a type trace and $\seqI^{k} = \I_{k} \I_{k + 1}$ a corresponding FO-trace.
	If $\I_i \not\models \Tmc$ for some $i \geq k$ then the trace type $\word$ already satisfies the statement.
	Assume instead that $\I_i \models \Tmc$ fore every $i \geq k$ and  consider the run
	$$\ppath^{\Tmc} = q_k \trans{\Phi_{k}} q_{k + 1} \trans{\Phi_{k + 1}} \cdots$$
	of $\word^{k}$ in $\PA^{\Tmc}_{\ptquery}$.
	First, observe that, since $\I_i \models \Tmc$ for every $i \geq k$, we have that $\chformula(\Phi_i)$ is consistent with $\Tmc$ for every $i \geq k$.
	This means that the extra state $q^{*}$ in $\PA^{\Tmc}_{\ptquery}$ does not occur in $\ppath^{\Tmc}$, which implies that $\ppath^{\Tmc}$ is also the run of $\word^{k}$ in $\PA_{\ptquery}$.
	By concatenation, we obtain that
	$$\ppath' = q_0 \trans{\Phi_0} \cdots \trans{\Phi_{k - 1}} q_k \trans{\Phi_{k}} q_{k + 1} \trans{\Phi_{k + 1}} \cdots$$
	is the run in $\PA_{\ptquery}$ of the word $\word$.
	Now, since $q_k \notin \Acc(\PA^\Tmc_{\ptquery})$, we obtain that $\min\set{\painf(\ppath^{\Tmc}, \PA^\Tmc_{\ptquery})}$ is odd.
	%
	% \anni{Why does it follow that it is odd? I think this requires an explanation.}
	%
	Therefore, by the definition of $\col^\Tmc$, we obtain that 
	$\min\set{\painf(\ppath', \PA_{\ptquery})} = \min\set{\painf(\ppath^{\Tmc}, \PA_{\ptquery})} = \min\set{\painf(\ppath^{\Tmc}, \PA^{\Tmc}_{\ptquery})} - 1$ is even and then that $\word \in \L(\PA_{\ptquery})$.
	By Theorem~\ref{thm:bridge} we derive that the FO trace $\seqI$, being of type $\word$, is such that $\seqI \models \tquery$.
	
	We now prove the other direction by counter-nominal argument.
	Assume $q_k \in \Acc(\PA^{\Tmc}_{\ptquery})$.
	%
	This means that there exists a word $\word^k = \Phi_{k} \Phi_{k + 1} \ldots$ whose run $$\ppath^{\Tmc} = q_k \trans{\Phi_{k}} q_{k + 1} \trans{\Phi_{k + 1}} \cdots$$ of $\word^{k}$ in $\PA^{\Tmc}_{\ptquery}$ starting from $q_k$ is accepting.
	This implies that $q^{*}$ never appears in $\ppath^{\Tmc}$, which means that $\chformula(\Phi_i)$ is consistent with $\Tmc$, for every $i \geq k$, and that $$\ppath' = q_0 \trans{\Phi_0} \cdots \trans{\Phi_{k - 1}}q_k \trans{\Phi_{k}} q_{k + 1} \trans{\Phi_{k + 1}} \cdots$$ is the run in $\PA_{\ptquery}$ of the word $\word = \Phi_0 \cdots \Phi_{k - 1} \Phi_{k} \Phi_{k + 1} \cdots$.
	Being that $\ppath^{\Tmc}$ is accepting in $\PA^{\Tmc}_{\ptquery}$, we have that $\min\set{\painf(\ppath^{\Tmc}, \PA^\Tmc_{\ptquery})}$ is even, which implies that $\min\set{\painf(\ppath', \PA_{\ptquery})} = \min\set{\painf(\ppath^{\Tmc}, \PA_{\ptquery})} = \min\set{\painf(\ppath^{\Tmc}, \PA^{\Tmc}_{\ptquery})} - 1$ is odd and then that $\word \notin \L(\PA_{\ptquery})$.
	By Theorem~\ref{thm:bridge}, every FO trace $\seqI$ of type $\word$ is such that $\seqI \not\models \tquery$.
	Observe that, being $\chformula(\Phi_i)$ consistent with $\Tmc$ for every $i \geq k$, this means that $\I_i \models \Tmc$ for every $i \geq k$.
	This proves the statement.
	
\end{proof}

\rtdacomplexity*
\begin{proof}
Let $\tkb=(\Tmc,\Aseq)$ and $\PA_{\ptquery}^{\Tmc}=(2^{\props(\ptquery)},Q^\Tmc,\delta^{\Tmc},q_0,\col^{\Tmc})$ 
	be the $\Tmc$-reduct of $\PA_{\ptquery}=(2^{\props(\ptquery)},Q,\delta,q_0,\col)$. 
 Define the RT-DFA for $\tkb$ and $\tquery$ as $\DFA=(\alphabet,S,s_0,\gamma,F)$.

The result follows directly from the definition of RT-DFA (Def.~\ref{def:rtdfa})
and the following facts: the sizes of $Q$ and \alphabet are doubly exponential wrt \tquery, which implies a triply exponential size of $S$; checking whether there exists an ABox $\Amc$ s.t.~$(\Amc,\Tmc)\models \phi$ is time-exponential wrt both $\Tmc$ (this is shown in the proof of Thm~\ref{th:solvability:kb:alignment:no:nominals:CQ}) and $\phi$, with $\phi=\bigvee_{\Phi \in \Upsilon}\chformula(\Phi)\land
	                        			\bigwedge_{\Phi \not\in \Upsilon}\lnot\chformula(\Phi)$ 
doubly exponential wrt $\tquery$; and computing $\Acc(\PA_{\ptquery}^{\Tmc})$ is doubly exponential wrt \tquery.
\end{proof}


In order to prove Thm.~\ref{thm:rt-sound-compl}, the following auxiliary result is needed.
%%
\begin{lemma}\label{lem:abstraction-instantiation}
	For every word $w \in \alphabet^{*}$ there exists a TKB-modification $\tmods$ that instantiates $w$.
	Moreover, for every TKB-modification $\mods$ there exists a unique abstraction $w$ of it.
\end{lemma}

\begin{proof}
	We first prove that for every word $w$ there exists an instantiation $\tmods$ of it.
	Consider also the run $\ppath = s_0 \trans{w_0} \ldots \trans{w_{m}} s_{m + 1}$ be the corresponding run on the RT-DFA.
	Then, for each $j \leq m$, we have:
	
	\begin{itemize}
		\item 
			If $w_j = \delsym$ then define $\tmods_j = \del$;
			
		\item 
			if $w_j = (\fixsym, \Upsilon)$ and $s_j = (Z, i)$, then consider an ABox $\Amc$ consistent with $\Tmc$ such that $(\Tmc, \Amc) \models \bigvee_{\Phi \in \Upsilon} \chformula(\Phi)\land
			\bigwedge_{\Phi \not\in \Upsilon} \lnot\chformula(\Phi)$.
			Observe that such ABox always exists.
			Now, let $\mods$ be a KB-modification such that $\Amc = \apply(\mods, \Amc_i)$.
			Also such $\mods$ always exists.
			Then, define $\tmods_{j} = \fix{\mods}$;
			
		\item 
			if $w_j = (\addsym, \Upsilon)$ and $s_j = (Z, i)$, then consider an ABox $\Amc$ consistent with $\Tmc$ such that $(\Tmc, \Amc) \models \bigvee_{\Phi \in \Upsilon} \chformula(\Phi)\land
			\bigwedge_{\Phi \not\in \Upsilon} \lnot\chformula(\Phi)$.
			Observe that such ABox always exists.
			Now, let $\mods$ be a KB-modification such that $\Amc = \apply(\mods, \emptyA)$.
			Also such $\mods$ always exists.
			Then, define $\tmods_{j} = \add{\mods}$.
	\end{itemize}
	
	Clearly, the TKB-modification $\tmods$ satisfies Definition~\ref{def:abstraction} and so it is an instantiation of $w$.
	
	For the second statement, consider a TKB modification $\tmods$ of length $m$.
	We construct by induction on $m$ the unique abstraction of $\tmods$.
	
	As base case, if $m = 0$, we have three possible cases:
	
	\begin{enumerate}
		\item 
			If $\tmods_0 = \delsym$ then define $w_{0} = \delsym$;
			
		\item 
			If $\tmods_0 = \fix{\mods}$ then consider $\Amc = \apply(\mods, \Amc_0)$ and $\Upsilon$ being the \emph{unique} element in $2^{2^{\props}}$ such that $(\Tmc, \Amc) \models \bigvee_{\Phi \in \Upsilon} \chformula(\Phi)\land \bigwedge_{\Phi \not\in \Upsilon} \lnot\chformula(\Phi)$.
			Therefore, define $w_0 = (\fixsym, \Upsilon)$;
			
		\item 
			If $\tmods_0 = \add{\mods}$ then consider $\Amc = \apply(\mods, \emptyA)$ and $\Upsilon$ being the \emph{unique} element in $2^{2^{\props}}$ such that $(\Tmc, \Amc) \models \bigvee_{\Phi \in \Upsilon} \chformula(\Phi)\land \bigwedge_{\Phi \not\in \Upsilon} \lnot\chformula(\Phi)$.
			Therefore, define $w_0 = (\addsym, \Upsilon)$.
	\end{enumerate}

	Clearly, in all the cases, $w_0$ is the only abstraction of $\tmods_0$.
	
	For the induction case, assume there is a unique abstraction for every TKB modification of length $m$ and consider the TKB-modification $\tmods$ of length $m + 1$.
	First, consider $\tmods' = \tmods_0 \ldots \tmods_m$ be the prefix of $\tmods$ up to $m$.
	By induction hypothesis, let $w'$ be the only abstraction of $\tmods'$ and $\ppath = s_0 \trans{w_0} \ldots \trans{w_{m}} s_{m + 1}$ be the corresponding run on the RT-DFA, with $s_m = (Z, i)$.
	We distinguish three cases:
	
	\begin{enumerate}
		\item 
			If $\tmods_{m + 1} = \del$, then define $w_{m + 1} = \delsym$;
			
		\item 
			If $\tmods_{m + 1} = \fix{\mods}$, then consider $\Amc = \apply(\mods, \Amc_i)$ and $\Upsilon$ being the \emph{unique} element in $2^{2^{\props}}$ such that $(\Tmc, \Amc) \models \bigvee_{\Phi \in \Upsilon} \chformula(\Phi)\land \bigwedge_{\Phi \not\in \Upsilon} \lnot\chformula(\Phi)$.
			Therefore, define $w_{m + 1} = (\fixsym, \Upsilon)$;
			
		\item 
			If $\tmods_{m + 1} = \add{\mods}$, then consider $\Amc = \apply(\mods, \emptyA)$ and $\Upsilon$ being the \emph{unique} element in $2^{2^{\props}}$ such that $(\Tmc, \Amc) \models \bigvee_{\Phi \in \Upsilon} \chformula(\Phi)\land \bigwedge_{\Phi \not\in \Upsilon} \lnot\chformula(\Phi)$.
			Therefore, define $w_{m + 1} = (\addsym, \Upsilon)$;
	\end{enumerate}

	Clearly, also in this case, the abstraction $w = w' \cdot w_{m + 1}$ is the unique abstraction of $\tmods$.
	
\end{proof}



\thmsoundcompl*

\begin{proof}
	First, observe that thanks to Lemma~\ref{lem:abstraction-instantiation} we need to prove only Item~2 of the theorem.
	Indeed, Item~1 would follow from Item~2 and from the fact that every abstraction admits at least an instantiation.
	Analogously, Item~3 would follow from Item~2 and from the fact that every TKB-modification admits a unique abstraction.
	
	We then prove Item~2.
	For technical convenience, we prove a slightly more general result.
	First, for every TKB $\tkb = (\Tmc, \Aseq)$ with $\Aseq = \Amc_0 \cdots \Amc_\ell$ and an index $i$, define $\tkb_{\geq i} = (\Tmc, \Aseq_{\geq i})$ where $\Aseq_{\geq i} = \Amc_i \cdots \Amc_{\ell}$.
	We prove that, for every word $w \in \alphabet^{*}$ and state $s = (Z, i)$ of the automaton $\DFA$, it holds that the run 
	
	\[
	\ppath = s \trans{w_0} s_1 \trans{w_1} \ldots \trans{w_m} s_{m + 1}
	\]
	
	is accepting in $\DFA$ iff, for all instantiations $\tmods$ of $w$ and every trace type $\word$ such that  $\seqI \models \apply(\tmods, \tkb_{\geq i})$ for every $\seqI$ of type $\word$, it holds that the run in $\PA_{\ptquery}$ starting from some $q \in Z$ over $\word$ is accepting.
	
	Item~2 of the theorem then follows by setting $s = (\{q_0, 0\})$ the initial state of $\DFA$, which in turns applies to the TKB $\tkb$.
	
	The proof is by induction on the length $m$ of $w$ and, subsequently, $\tmods$.
	
	As base case, assume $m = 0$, therefore both $m$ and $\tmods$ are the empty sequence.
	This implies that the path $\ppath = s$ is made by just $s = (Z, i)$ and it is accepting iff $s \in F$.
	By construction of $\DFA$, first we have that $i = \ell + 1$ and so that $\tkb_{\geq i} = (\Tmc, \emptyAseq)$.
	Moreover, we have that $Z \cap \Acc(\PA^{\Tmc}_{\ptquery}) = \emptyset$ and so that every state $q$ in $Z$ does not belong to $\Acc(\PA^{\Tmc}_{\ptquery})$.
	By Lemma~\ref{lem:tail}, we have that every trace type $\word$ such that $\seqI \models (\Tmc, \emptyAseq)$ for every $\seqI$ of type $\word$, is such that the run starting from $q$ is accepting in $\PA_{\ptquery}$, which proves the statement.
	
	For the induction case, assume that the property holds for a given $m$, and that $w$ and $\tmods$ are of length $m + 1$.
	Observe that, since $\ppath$ is accepting, then also the path
	
	\[
	\ppath' = s_1 \trans{w_1} \ldots \trans{w_m} s_{m + 1}
	\]
	%
	is accepting, which means that, by induction hypothesis, for every instantiation $\tmods'$ of $w' = w_{1} \cdots w_{m}$, every trace type $\word'$ such that  $\seqI \models \apply(\tmods, \tkb_{\geq 1})$ for every $\seqI'$ of type $\word'$, it holds that the run in $\PA_{\ptquery}$ starting from some $q \in Z_1$, with $s_1 = (Z_1, \iota)$ over $\word'$ is accepting.
	We distinguish three cases:
	
	\begin{itemize}
		\item 
			$w_0 = \delsym$, then $\tmods_{0} = \del$ and every instantiation of $w$ is of the form $\tmods_{0} \cdot \tmods'$.
			We also obtain that $s_1 = \gamma((Z, 0), \delsym) = (Z, 1)$.
			In addition, observe that 
			
			\[
			\apply(\tmods_{0} \cdot \tmods', \tkb_{\geq 0}) = \apply(\del \cdot \tmods', \tkb_{\geq 0}) = \apply(\tmods', \tkb_{\geq 1})
			\]
			which, combined with the induction hypothesis, proves the statement.
			
		\item 
			$w_0 = (\fixsym, \Upsilon)$ and so $\tmods_{0} = \fix{\mods}$ for some $\mods$ such that $(\Tmc, \apply(\mods, \Amc_0)) \models \bigvee_{\Phi \in \Upsilon}\chformula(\Phi)\land
			\bigwedge_{\Phi \not\in \Upsilon}\lnot\chformula(\Phi)$.
			Also, every instantiation of $w$ is of the form $\tmods_{0} \cdot \tmods'$.
			Observe that 
			
			\[
			s_1 = \gamma((Z,0), (\fixsym, \Upsilon)) = (Z_1, 1)
			\]
			
			where $Z_1$ contains only states $q'$ such that $\delta(q,\Phi)=q'$, for some $q \in Z$.
			
			Observe that			
			
			\begin{align*}
				\apply(\tmods_{0} \cdot \tmods', \tkb_{\geq 0}) & =\apply(\fix{\mods} \cdot \tmods', \tkb_{\geq 0}) \\ 
				& =(\Tmc, \apply(\mods, \Amc_0) \cdot \apply(\tmods', \Aseq_{\geq 1}))
			\end{align*}
		
			By induction hypothesis, every trace type $\word'$ such that $\seqI' \models (\Tmc, \apply(\tmods', \Aseq_{\geq 1}))$ for every $\seqI'$ of type $\word'$ has an accepting run in $\PA_{\ptquery}$ starting from $s_1$.
			Moreover, the only types $\Phi_0$ entailed by $(\Tmc, \apply(\mods, \Amc_0))$ must belong to $\Upsilon$ by construction.
			Therefore, the word $\word = \Phi_0 \cdot \word'$ clearly is accepted by $\PA_{\ptquery}$ from some state $q \in Z$ such that $\delta(q, \Phi_0) \in Z_1$ and so the statement holds.
			
			\item 
				$w_0 = (\addsym, \Upsilon)$ and so $\tmods_{0} = \add{\mods}$ for some $\mods$ such that $(\Tmc, \apply(\mods, \emptyA)) \models \bigvee_{\Phi \in \Upsilon}\chformula(\Phi)\land
				\bigwedge_{\Phi \not\in \Upsilon}\lnot\chformula(\Phi)$.
				Also, every instantiation of $w$ is of the form $\tmods_{0} \cdot \tmods'$.
				Observe that 
				
				\[
				s_1 = \gamma((Z,0), (\fixsym, \Upsilon)) = (Z_1, 0)
				\]
%				
				where $Z_1$ contains only states $q'$ such that $\delta(q,\Phi)=q'$, for some $q \in Z$.
%				
				Observe that
				
				\begin{align*}
					\apply(\tmods_{0} \cdot \tmods', \tkb_{\geq 0}) & =\apply(\add{\mods} \cdot \tmods', \tkb_{\geq 0}) \\ 
					& =(\Tmc, \apply(\mods, \emptyA) \cdot \apply(\tmods', \Aseq_{\geq 0}))
				\end{align*}
%				
				By induction hypothesis, every trace type $\word'$ such that $\seqI' \models (\Tmc, \apply(\tmods', \Aseq_{\geq 0}))$ for every $\seqI'$ of type $\word'$ has an accepting run in $\PA_{\ptquery}$ starting from $s_1$.
				Moreover, the only types $\Phi_0$ entailed by $(\Tmc, \apply(\mods, \emptyA))$ must belong to $\Upsilon$ by construction.
				Therefore, the word $\word = \Phi_0 \cdot \word'$ clearly is accepted by $\PA_{\ptquery}$ from some state $q \in Z$ such that $\delta(q, \Phi_0) \in Z_1$ and so the statement holds.
	
	\end{itemize}

	Note that all the arguments in both base and induction cases are equivalences, which means both directions of Item~2 are proved.
\end{proof}

\migcomplexity*
\begin{proof}
Consequence of Def.~\ref{def:mmg}, Lemma~\ref{lem:rtda-complexity}, which states that RT-DFA $\DFA = (\alphabet, S, s_0, \gamma, F)$ for $\tkb$ and $\tquery$ has triply eponential size wrt $\tquery$ and the fact that KB-Alignment is solvable in doubly exponential time (see Thm.~\ref{th:solvability:kb:alignment:no:nominals:CQ}).
\end{proof}


\tkbalgsolv*
\begin{proof}
    Immediate consequence of the fact that the Minimal-instantiation Graph $G$ has triply exponential size and that Shortest Path can be solved in polynomial time on $G$.
\end{proof}



%\thmtkbalign*

%\begin{proof}
%	content
%\end{proof}
%!TEX root = ecai-main.tex

\subsection{Proofs from Section~\ref{sec:kbalign}}

\sigrestr*
%
\begin{proof}
	Let $\Kmc = (\Tmc,\Amc)$, $\mods=\op_0\cdots \op_n$  be an optimal solution to KB Alignment for $(\Kmc,\bquery)$ and let $\Amc' = \apply(\mods,\Amc)$. 
	%
	
	Suppose that $\mods$ contains an ABox operation $\op_i=\rem{\alpha}$ such that $\alpha \not \in \Amc$. In addition, assume that there is no $j, 0 \leq j < i$ such that $\op_j=\op_i$. We make the following case distinction:
	%
	\begin{itemize}
		\item
		$\ins{\alpha}$ does not occur in $\op_0\cdots \op_{i-1}$. Then, removing $\op_i$ from $\mods$ yields an ABox-modification $\mods'$ such that $\Amc' = \apply(\mods',\Amc)$ and $\cost(\mods') < \cost(\mods)$.
		%
		%
		\item
		There is $j, 0 \leq j < i$ such that $m_j=\ins{\alpha}$. In this case, removing $\op_i$ and all ABox operations $m_\ell = \ins{\alpha}$ ($0 \leq \ell < i$) from $\mods$ %has the same effect of the previous case.
		results in a shorter ABox-modification of lesser cost than $\mu$ and that would also result in $\Amc'$ if applied to $\Amc$.
	\end{itemize}
    %    
	Hence, we can assume  without loss of generality that each ABox operation $\rem{\alpha}$ occurring in $\mods$ is such that $\alpha \in \Amc$.
	
	Consider the ABox-modification $\mods^*$ obtained from $\mods$ by removing all operations of the form $\ins{\alpha}$ such that $\alpha$ is not defined over $\sig(\Kmc) \cup \sig(\bquery)$. Two observations regarding $\mods^*$ are in order:
	%
	\begin{itemize}
		\item
		Since every ABox operation $\rem{\alpha}$ in $\mods$ satisfies that $\alpha \in \Amc$ (by assumption), the same is the case for $\mods^*$. Hence, every ABox operation in $\mods^*$ is defined over $\sig(\Kmc) \cup \sig(\bquery)$.
		%
		\item
		By definition of $\mods^*$, we have that $\cost(\mods^*) \leq \cost(\mods')$. In addition, $\Amc^* \subseteq \Amc'$ where $\Amc^* = \apply(\mods^*,\Amc)$.
		%
		%
		Hence, since $(\Tmc,\Amc')$ is consistent, this means that $(\Tmc,\Amc^*)$ is consistent as well.
		%		
	\end{itemize}	
	%
	
	Thus, to conclude the proof of the lemma, it remains to show that $(\Tmc,\Amc^*) \models \bquery$. Let $\Imc$ be a model of $(\Tmc,\Amc^*)$. Since no individual name occurs in $\Tmc$ and assertions removed from $\Amc'$ are defined over concept and role names not contained in $\sig(\Tmc) \cup \sig(\bquery)$, it is not hard to see that: 
	%
	\begin{itemize}
		\item
		$\Imc$ can be extended into a model $\Jmc$ of $(\Tmc,\Amc')$ such that $\Imc$ and $\Jmc$ are identical on $\sig(\Tmc) \cup \sig(\bquery)$ and on the interpretation of the individuals in $\ind(\bquery)$.
		%
		%
		\item
		 Let $\query'$ be a \CQ  occurring in $\bquery$. Since $\Imc$ and $\Jmc$ are identical on $\sig(\query')$ and $\ind(\query')$, it follows that:
		%
		\begin{equation}\label{eq:base:case:signature:lemma}
			\J \models \query'\ \text{iff}\ \ \I \models \query'.
		\end{equation}
		%
		Hence, by using induction on the structure of $\bquery$ one can easily show that $\J \models \bquery$ iff $\I \models \bquery$. The base case follows from \eqref{eq:base:case:signature:lemma}, since it corresponds to the \cq s occurring in $\bquery$.
		%
		Consequently, since $(\Tmc,\Amc') \models \bquery$, it follows that $\Jmc \models \bquery$ and $\Imc\models\bquery$.						
	\end{itemize}
	
	Finally, since $\Imc$ is an arbitrary model of $(\Tmc,\Amc^*)$, we have thus shown that  $(\Tmc,\Amc^*) \models \bquery$. This completes the proof.	
	%
\end{proof}

\algcorrect*
%
\begin{proof}
	Suppose that Algorithm~\ref{alg:kb:alignment} returns an ABox-modification $\mods^*$ on input $\Kmc$ and $\bquery$. The following observations imply that $\mods^*$ is an optimal solution to KB Alignment for $(\Kmc,\bquery)$.
	%
	\begin{itemize}
		\item
		The modification $\mods^*$ is either the trivial modification computed at Line~\ref{alg1:line:trivial}, or a modification in $M$ satisfying the test at Line~\ref{alg1:line:if}. Hence, $\mods^*$ is a solution to KB Alignment for $(\Kmc,\bquery)$, because it satisfies Condition~\ref{prop:cons} and \ref{prop:entail} in Definition~\ref{def:kb-alignment}.
		%
		%
		\item
		% The previous observation tells us 
		Since $\mods^*$ is a solution to KB Alignment for $(\Kmc,\bquery)$, it follows that some optimal solution $\mods'$ to KB Alignment for $(\Kmc,\bquery)$ exists. 
		%
		By Lemma~\ref{lem:signature:restriction}, we can assume that $\mods'$ is defined over $\sig(\Kmc) \cup \sig(\bquery)$. In addition, since $\mods^*$ is a solution, it follows that 
		%
		\begin{equation}\label{eq:fact}
	   	   \cost(\mods') \leq \cost(\mods^*).
		\end{equation}
	%
	If $\cost(\mods') = \cost(\mods^*)$, then $\mods^*$ is obviously optimal. Otherwise, inequation \eqref{eq:fact} implies that $\cost(\mods') < \cost(\mods^*)$. This in turn implies that $\mods'\in M$, since either $\mods^*$ is the trivial modification computed at Line~\ref{alg1:line:trivial} in Algorithm~\ref{alg:kb:alignment} or $\mods^* \in M$.
	%
	Obviously, the \emph{for-loop} would never choose $\mods^*$ if $M$ contains a modification with smaller cost. Hence, $\cost(\mods') < \cost(\mods^*)$ cannot be the case. Therefore, $\mods^*$ must be an optimal solution.
	%
	\end{itemize}
	%
	
	Suppose that Algorithm~\ref{alg:kb:alignment} returns \enquote{no solution}. Due to Line~\ref{alg1:line:nosolution} there is then no ABox $\Amc'$ such that:
	%
	\begin{itemize}
		\item
		$(\Tmc,\Amc')$ is consistent, 
		\item $(\Tmc,\Amc') \models \bquery$, and 
		\item $\sig(\Amc') \subseteq \sig(\Kmc) \cup \sig(\bquery)$.
	\end{itemize}  
     %
     The non-existence of such an ABox $\Amc'$ implies that there is no solution $\mods$ to KB Alignment for $(\Kmc,\bquery)$ such that $\mods$ is defined over $\sig(\Kmc) \cup \sig(\bquery)$. Note that any such ABox-modification $\mods$ would yield an ABox defined over the signature $\sig(\Kmc) \cup \sig(\bquery)$, like $\Amc'$ is.
     %
     Thus, the application of Lemma~\ref{lem:signature:restriction} yields that KB Alignment has indeed no solution for $(\Kmc,\bquery)$.
     %
\end{proof}

\ASigmaenough*

To prove this lemma, we slightly extend the proof arguments given in \cite{BaaderBL16} for the \CQ query language to the more general query  language \bcCQ. To this end, we need to introduce some notions and auxiliary results from \cite{BaaderBL16}.
%
Let us start with the notion of an ABox homomorphism.
%
\begin{definition}
	Let $\Amc$ and $\Amc'$ be two ABoxes. An \emph{ABox homomorphism} from $\Amc$ to $\Amc'$ is a total function $h:\ind(\Amc) \mapsto \ind(\Amc')$ satisfying the following conditions:
	%
	\begin{itemize}
		\item
		$A(a) \in \Amc$ implies $A(h(a)) \in \Amc'$, and
		%
		\item 
		$r(a,a') \in \Amc$ implies $r(h(a),h(a')) \in \Amc'$.
	\end{itemize}
\end{definition}
 %
 The following lemma states an important property of ABox homomorphisms for query answering. It generalizes Lemma~11 in \cite{BaaderBL16} from \CQ to \bcCQ.
 %
\begin{lemma}\label{lem:queries:canonical:abox}
	Let $\Tmc$ be an $\alc$ TBox, $\bquery\in \bcCQ$ with free variables $x_1,\ldots,x_k$, $\Amc$ and $\Amc'$  ABoxes, and $\hphism$ an ABox homomorphism from $\Amc$ to $\Amc'$. If $(\Tmc,\Amc) \models \bquery[(a_1,\ldots,a_k)]$ then $(\Tmc,\Amc') \models \bquery[(\hphism(a_1),\ldots,\hphism(a_k))]$.
\end{lemma}
\begin{proof}
	%
	We assume that $(\Tmc,\Amc) \models \bquery[(a_1,\ldots,a_k)]$ holds.	
	%
	The case where $(\Tmc,\Amc')$ is not consistent is trivial. 
	%
	If $(\Tmc,\Amc')$ is consistent, it has a model. Let $\Imc$ be an arbitrary model of $(\Tmc,\Amc')$. We show that 
	%
	\begin{equation}\label{eq:hom:preserves:entailment}
		\Imc \models \bquery[(\hphism(a_1),\ldots,\hphism(a_k))].
	\end{equation}
	%
	To prove this, we define an interpretation $\Jmc$ from $\Imc$ by reinterpreting the individual names in $\Amc$, i.e
	%
	\begin{itemize}
		\item
		$\Delta^\Jmc := \Delta^\Imc$,
		%
		\item
		$X^\Jmc := X^\Imc$ for all $X \in \cset\cup\rset$, and
		%
		\item
		$a^\Jmc := \hphism(a)^\Imc$ for all $a \in \ind(\Amc)$.
	\end{itemize}
	%
	Let $\query_a$ be a \cq\ occurring in $\bquery[(a_1,\ldots,a_k)]$, and $\bquery_\hphism$ be the \cq\ that results from replacing all occurrences of all $a_i$ ($1 \leq i \leq k$)
	in $\query_a$ by $\hphism(a_i)$. Hence, 	since ${a_i}^\Jmc = \hphism(a_i)^\Imc$ for all $i, 1 \leq i \leq k$, the following holds:
	%
	\begin{equation}\label{eq:base:case}
		\Jmc \models \query_a\ \ \ \text{ iff }\ \ \ \Imc \models \query_\hphism
	\end{equation}
	%
	Therefore, by using induction on the structure of $\bquery[(a_1,\ldots,a_k)]$ and applying \eqref{eq:base:case} to the base case, one can easily show that:
	%
	\begin{equation*}
		\Jmc \models \bquery[(a_1,\ldots,a_k)]\ \text{ iff }\ \Imc \models \bquery[(\hphism(a_1),\ldots,\hphism(a_k))]
	\end{equation*}
	%
	Hence, since $(\Tmc,\Amc) \models \bquery[(a_1,\ldots,a_k)]$, to  show that \eqref{eq:hom:preserves:entailment} holds it is enough to show that $\Jmc \models (\Tmc,\Amc)$.
	%
	\begin{itemize}
		\item
		Since no individual name occurs in $\Tmc$, $\Imc \models \Tmc$ implies that $\Jmc$ is also a model of $\Tmc$. 
		%
		%
		\item
		To see that $\Jmc$ is a model of $\Amc$ as well, consider any concept assertion $A(a) \in \Amc$. The homomorphism $\hphism$ ensures that $A(\hphism(a)) \in \Amc'$. Hence, $\I \models \Amc'$ implies that $\hphism(a)^\I \in A^\I$, and it follows by definition of $\J$ that $a^\J \in A^\J$. The case of role assertions $r(a,b) \in \A$ can be shown analogously.
	\end{itemize}
	%
	Overall, we have just shown that any model $\Imc$ of $(\Tmc,\Amc')$ satisfies \eqref{eq:hom:preserves:entailment}. Thus, $(\Tmc,\Amc') \models \bquery[(\hphism(a_1),\ldots,\hphism(a_k))]$.
	%
\end{proof}

The ABox $\Amc_{\Tmc,\Sigma}$ defined in \cite{BaaderBL16} is \emph{canonical} in the following sense. For any satisfiable $\alc$ TBox $\Tmc$, the following holds:
%
\begin{enumerate}
	\item
	 $(\Tmc,\Amc_{\Tmc,\Sigma})$ is consistent. %(Lemma~10), 
	 %
	 \item\label{claim:BBL16-2}
	 A $\Sigma$-ABox $\Amc$ satisfies that $(\Tmc,\Amc)$ is consistent iff there is an ABox homomorphism from $\Amc$ to $\Amc_{\Tmc,\Sigma}$. %Lemma~12 in \cite{BaaderBL16} 
	 %	 
\end{enumerate}
%
Using these two properties about $\Amc_{\Tmc,\Sigma}$ and Lemma~\ref{lem:queries:canonical:abox}, we are now ready to conclude the proof of Lemma~\ref{lem:ASigmaT:is:enough:for:BbcCO}.
 
 \medskip
 
 \begin{proof}[Proof of Lemma~\ref{lem:ASigmaT:is:enough:for:BbcCO}]
 	The \emph{left-to-right} direction is trivial. For the opposite direction, assume that $\bquery$ is non-empty for $\Sigma$ \wrt $\Tmc$. Then, there exists a $\Sigma$-ABox $\Amc$ such that $\cans{(\Tmc,\Amc)}{\bquery} \neq \emptyset$.
 	%
 	Thus, since there exists an ABox homomorphism from $\Amc$ to $\Amc_{\Tmc,\Sigma}$ (by \ref{claim:BBL16-2}.\ above), Lemma~\ref{lem:queries:canonical:abox} yields that $\cans{(\Tmc,\Amc_{\Tmc,\Sigma})}{\bquery} \neq \emptyset$.
 \end{proof}

\removeindividuals*
%
%We start by recalling the definitions of $\Tmc_p, \Sigma_p$ and $\bquery_p$ stated in the paper. Let $\ind(\bquery)=\{a_1,\ldots,a_m\}$ and $A_1,\ldots,A_m$ be concept names not occurring in $\Tmc, \bquery$ nor $\Sigma$. The elements  $\Tmc_p, \Sigma_p$ and $\bquery_p$ are defined as follows:
%%
%\begin{itemize}
%	\item
%	$\Tmc_p = \Tmc \cup \Tmc_{\disj}$, where $\Tmc_{\disj} =\{A_i \sqcap A_j \sqsubseteq \bot \mid 1 \leq i < j \leq m\}$,
%	%
%	%
%	\item
%	$\bquery_p = \bquery_x \land \query_{\disj}$, where $\bquery_x$ is obtained from $\bquery$ by replacing each $a \in \ind(\bquery)$ by a fresh free variable $x_a$, whereas $\query_{\disj}$ is the \cq\ $\query_{\disj} = A_1(x_{a_1}) \land \ldots \land A_m(x_{a_m})$, and
%	%
%	%
%	\item
%	$\Sigma_p = \Sigma \cup \{A_1,\ldots,A_m\}$.
%\end{itemize}
%%
%We continue with the proof of Lemma~\ref{lem:remove:individuals}. 

\begin{proof}
	Let $x_1,\ldots,x_k$ be the free variables of $\bquery$ and $a_1,\ldots,a_m$ the individual names occurring in $\bquery$. Then, $\bquery_x$ has $k+m$ free variables, i.e.
	%
	\begin{equation*}
		\fvarsq(\bquery_x) =\{x_1,\ldots,x_k\} \cup \bigcup\limits_{i=1}^{m} \{x_{a_i}\},
	\end{equation*}
	%
	where each $x_{a_i}$ is fresh variable not occurring in $\bquery$. We represent the free variables as the ordered tuple $(x_1,\ldots,x_k, x_{a_1},\ldots,x_{a_m})$.
	
	Let us start with Claim 1, by showing the \emph{left-to-right} implication.
	
	($\Rightarrow$) Assume that $\bquery_p$ is non-empty for $\Sigma_p$ \wrt $\Tmc_p$. Hence, there exists a $\Sigma_p$-ABox $\Amc_p$ such that  $(\Tmc_p,\Amc_p)$ is consistent and  $\cans{(\Tmc_p,\Amc_p)}{\bquery_p} \neq \emptyset$. Thus, there exists  a tuple $t=(b_1,\ldots,b_{k+m})$ of individual names of $\iset$ such that $t \in \cans{(\Tmc_p,\Amc_p)}{\bquery_p}$. 
	%
	The following assumptions about $\Amc_p$ are without loss of generality:
	%
	\begin{enumerate}[(a)]
		\item
		 $\Amc_p$ contains no individuals from $\ind(\bquery)$. This can be assumed because neither $\bquery_p$ nor $\Tmc_p$ contain individual names,		
		%
		%
		\item
		$b_{k+1},\ldots,b_{k+m}$ are all distinct individuals of $\Amc_p$. 
		This follows from the fact that $(\Tmc_p,\Amc_p)$ is consistent, the disjointness axioms occurring $\Tmc_{\disj}$ and the form of  $\bquery_{\disj}$.
		%
		%
		\item
		For all $i \in \{1,\ldots,m\}$, $\Amc_p$  contains no assertion of the form $A_i(b)$ with $b \neq b_{k+i}$. Such an assertion would either contradict $t \in \cans{(\Tmc_p,\Amc_p)}{\bquery_p}$, or would not be necessary to ensure that $t$ belongs to $\cans{(\Tmc_p,\Amc_p)}{\bquery_p}$.
	\end{enumerate}
	%
	Let us define a $\Sigma$-ABox $\Amc$ from $\Amc_p$, by taking the following steps.
	%
	\begin{enumerate}
		\item
		Remove all assertions of the form $A_i(b)$ occurring in $\Amc_p$ ($1 \leq i \leq m$).
		%
		\item
		Rename every remaining individual name $b_{k+i}$ as $a_i$ ($1 \leq  i \leq m$).
	\end{enumerate}
     %
     This renaming is well-defined since all individual names $b_{k+1},\ldots,b_{k+m}$  are distinct (recall assumption (b)). For all $i \in \{1,\ldots,k\}$, we write $\overline{b_i}$ to denote the following individual from $\{b_1,\ldots,b_k,a_1,\ldots,a_m\}$:
     %
     \begin{itemize}
     	\item
     	$b_i$, if $b_i \neq b_{k+j}$  for all  $j \in \{1,\ldots,m\}$, or
     	%
     	\item
     	$a_j$, if $b_i = b_{k+j}$ for some  $j \in \{1,\ldots,m\}$.
     \end{itemize}
     %
      %\textcolor{red}{Do all $\overline{b_i}$ occur in $\Amc$?}               
	
	We show that $\Amc$ witnesses non-emptiness of $\bquery$ for $\Sigma$ \wrt $\Tmc$. 
	%
	Clearly, by removing all assertions of the form $A_i(b)$ from $\Amc_p$, we have that $\Amc$ is a $\Sigma$-ABox. To see that $(\Tmc,\Amc)$ is consistent, consider a model $\Imc$ of $(\Tmc _p,\Amc_p)$. We extend $\Imc$ to interpret the individual names $a_i$ as:
	%
	\begin{itemize}
		\item
		$a_i^\Imc = b_{k+i}^\Imc$ ($1 \leq i \leq m$).
	\end{itemize} 
     %
     Since $\Imc \models \Amc_p$, by construction of $\Amc$ it is easy to see that $\Imc \models \Amc$ as well. Further, since no individual name occurs in $\Tmc$, the extended $\Imc$ is still a model of $\Tmc$. Hence, $(\Tmc,\Amc)$ is consistent.
	
	It remains to show that $\cans{(\Tmc,\Amc)}{\bquery} \neq \emptyset$. To this end, we show that $\bar{t} = (\overline{b_1},\ldots,\overline{b_k}) \in \cans{(\Tmc,\Amc)}{\bquery}$. 
	%
	Let $\Jmc$ be an arbitrary model of $(\Tmc,\Amc)$. It is enough to prove that 
	\begin{equation}\label{eq:ent:to:show}
		\Jmc \models \bquery[\:\overline{b_1},\ldots,\overline{b_k}\:].
	\end{equation}
	%
    %
    We use $\Jmc$ to build an interpretation $\Jmc_p$ such that $\Jmc_p \models (\Tmc_p, \Amc_p)$. Since $t \in \cans{(\Tmc_p,\Amc_p)}{\bquery_p}$,  we would have that $\Jmc_p \models \bquery_p[t]$. Recall that $\bquery_p = \bquery_x \land \bquery_d$, $t=(b_1,\ldots,b_{k+m})$,  and $\bquery_x$ has $k+m$ free variables.
    %
    Hence, we would have that:
    %
    \begin{equation}\label{eq:ent:that:holds}
    	\Jmc_p \models \bquery_x[b_1,\ldots,b_k,b_{k+1},\ldots,b_{k+m}].
    \end{equation}
    %
    This will then be used to show that \eqref{eq:ent:to:show} holds.
    
    Let $\Delta^{\Jmc}_0, \Delta^{\Jmc}_1 \ldots, \Delta^{\Jmc}_m$ be disjoint copies of $\Delta^\Jmc$. Given $d \in \Delta^\Jmc$, we denote by $d_i$ the corresponding copy in $\Delta^{\Jmc}_i$ ($0 \leq i \leq m$). The interpretation $\Jmc_p$ is defined as follows:
    %
    \begin{itemize}
    	\item
    	$\Delta^{\Jmc_p} := \Delta^{\Jmc}_0 \cup \cdots \cup \Delta^{\Jmc}_m$,
    	%
    	\item
    	$B^{\Jmc_p} := \{d_i \in \Delta^{\Jmc}_i \mid d \in B^\Jmc \text{ and } 0 \leq i \leq m\}$\\[.2em] \hspace*{4.5cm} for all  $B \in \cset \setminus \Sigma_p$,
    	%
    	%    	
    	\item
    	$r^{\Jmc_p} := \{(d_i,e_j) \in \Delta^{\Jmc}_i \times \Delta^{\Jmc}_j \mid (d,e) \in r^\Jmc \text{ and }$ \\[.2em] 
    	%
    	\hspace*{3cm} $0 \leq i,j \leq m\}$, for all $r\in\rset$,
    	%
    	%
    	\item
    	$A_i^{\Jmc_p} := \{d_i \in \Delta^{\Jmc}_i \mid d =a_i^\Jmc\}$ for all $1 \leq i \leq m$. %\text{ or }\\[.2em]
    	     %
    	      %\hspace*{2.4cm}(d=a_j^\Jmc,  i=k+j \text{ and } 1 \leq j \leq m) \}$,
    	
%    	 $r^{\Jmc_\disj} := \{(d_i,e_i) \in \Delta^{\Jmc}_i \times \Delta^{\Jmc}_i \mid (d,e) \in r^\Jmc\} \: \cup \\[.2em] 
%    	                        %
%    	                        \hspace*{1cm}\{(d_i,e_j) \in \Delta^{\Jmc}_i \times \Delta^{\Jmc}_j \mid d=a_i^\Jmc, e=a_j^\Jmc,$\\[.2em]
%    	                        %
%    	                        \hspace*{1.3cm}$(a_i^\Jmc,a_j^\Jmc)\in r^\Jmc \text{ and } i,j > 0\}$,
    	 %    	 
    \end{itemize}
   %
   The individual names $b_1,\ldots,b_{k+m}$  are interpreted as follows. 
   %
   \begin{itemize}
   	   \item
   	   For all $i \in \{1,\ldots,k\}$ such that $\overline{b_i}\neq a_j$ ($1 \leq j \leq m$):
   	   %
   	   \begin{equation*}
   	   	 b_i^{\Jmc_p} = d_0 \in \Delta_0^{\Jmc_p}, \text{ where } d=b_i^\Jmc.
   	   \end{equation*}
   	   
   	   %
   	   \item
   	   For all $j \in \{1,\ldots,m\}$:
   	   %
   	   \begin{equation*}
   	   	 b_{k+j}^{\Jmc_p} = d_j \in \Delta_j^{\Jmc_p}, \text{ where } d = a_{j}^{\Jmc}.
   	   \end{equation*}
   	    %
   \end{itemize}
  %
  This assignment covers all individuals names in $b_1,\ldots,b_{k+m}$, because if  $\overline{b_i}= a_j$ for some $i \in \{1,\ldots,k\}$ then $b_i=b_{k+j}$ for some $j \in \{1,\ldots,m\}$.
    
  To show that $\Jmc_p \models (\Tmc_p,\Amc_p)$, we use the property that $\alc$ is bisimulation invariant \cite{BaaderHLS17}.

  We start by defining a binary relation $\bis \subseteq \Delta^\Jmc \times \Delta^{\Jmc_p}$ as follows:
	%
	\begin{equation*}
		\bis := \{(d,d_i) \in \Delta^\Jmc \times \Delta^{\Jmc_p} \mid 0 \leq i \leq m\}.
	\end{equation*}
%
It is not hard to show that  $\bis$ is a \emph{bisimulation} between $\Jmc$ and $\Jmc_p$ \wrt the symbols in $(\cset \cup \rset) \setminus \Sigma_p$. 
%\anni{If signatures are finite, then this is not a signature.
%\\[\medskipamount]
%Generally, I don't see why you don't use $\Sigma$ in the proof, but $(\cset \cup \rset)$ instead. It is only necessary to construct one witness for query non-emptiness. That should work for the signature $\Sigma$, doesn't it?
%}
%
Hence, the bisimulation invariance of $\alc$ guarantees the following for all $\alc$ concepts $C$ defined over $(\cset \cup \rset) \setminus \Sigma_p$:
%
\begin{equation}\label{bisimulation:prop}
	d \in C^\Jmc\ \text{ iff }\ d_i \in C^{\Jmc_p},\ \text{for all } i \in \{0,\ldots,m\}.
\end{equation}
	%
Therefore, since no concept name in $\Sigma_p$ occurs in $\Tmc$ and $\Jmc \models \Tmc$, the correspondence in \eqref{bisimulation:prop} implies that $\Jmc_p \models \Tmc$. In addition, the interpretation of $A_1,\ldots,A_m$ implies that $\Jmc_p \models \Tmc_d$. Thus, we have that $\Jmc_p \models \Tmc_p$.
	%	
	Regarding $\Amc_p$, we consider the possible forms of its assertions:
	%
	\begin{itemize}
		\item
		$A_i(b) \in \Amc_p$ for some $i \in \{1,\ldots,m\}$. By assumption (c), we know that $b=b_{k+i}$. The definition of $\Jmc_p$ tells us that $b_{k+i}^{\Jmc_p}=d_i$ where $d=a_i^\Jmc$, and that $A_i^{\Jmc_p}=\{d_i\}$. Thus, $b_{k+i}^{\Jmc_p} \in A_i^{\Jmc_p}$.
		%
		%
		\item
		$A(b) \in \Amc_p$ for $A \in \Sigma$. If $b=b_{k+i}$ for some $i \in \{1,\ldots,m\}$, then $A(a_i) \in \Amc$. Hence, $\Jmc \models \Amc$ implies that $a_i^\Jmc \in A^\Jmc$.
		%		
		Let $a_i^\Jmc=d \in \Delta^\Jmc$. By construction of $\Jmc_p$, we have that $d_i \in A^{\Jmc_p}$ and $b_{k+i}^{\Jmc_p} = d_i$. Thus, it follows that $b_{k+i}^{\Jmc_p} \in A^{\Jmc_p}$.
		%
		
		In case $b\neq b_{k+i}$, we have that $A(b) \in \Amc$. Hence, $b^\Jmc\in A^\Jmc$. It is easy to see from the definition of $\Jmc_p$ that $b^{\Jmc_p} \in A^{\Jmc_p}$.
		%
		\item
		The case of the role inclusions in $\Amc_p$ can be shown similarly, by considering the interpretation of role names in $\Jmc_p$ and the renaming of $b_{k+i}$ into $a_i$.
	\end{itemize}
	%	
	
  Overall, we have shown that $\Jmc_p \models (\Tmc_p,\Amc_p)$. Hence, we have that \eqref{eq:ent:that:holds} holds. From this, we can show that \eqref{eq:ent:to:show} holds in two steps, as follows:
  %
  \begin{enumerate}
  	\item
  	  We show that for each \cq\ $\query^*$ in $\bquery[\:\overline{t}\:]$ and the corresponding one $\query^*_x$ in $\bquery_x[t]$ it holds that:
  	  %
  	  \begin{equation}\label{base:case:induction}
  	  	   \Jmc \models \bquery^*\ \ \text{ iff }\ \ \Jmc_p \models \bquery^*_x.
  	  \end{equation}
      %
      This is a consequence of the definition of $\Jmc_p$ and the fact that $\bquery[\: \overline{t} \:]$ and $\bquery_x [t]$ are Boolean queries that are identical modulo renaming of $a_i$ and $b_{k+i}$.
      %
      \item
     We apply induction on the structure of $\bquery[\: \overline{t} \:]$ to show that \eqref{eq:ent:that:holds} implies \eqref{eq:ent:to:show}, by using \eqref{base:case:induction} as the induction base case.
     %\anni{This needs to be explained in more detail.}
  \end{enumerate}

  This concludes the proof of the \emph{left-to-right} direction of Claim 1 of the lemma.
  %
%  Let $\query^*$ be a \cq\ occurring in $\bquery[\: \overline{t}\:]$ and $\query^*_x$ the corresponding one from $\bquery_x[t]$. We show that $\Jmc \models \bquery^*$ iff $\Jmc_p \models \bquery^*_x$.
%  %
%  \begin{itemize}
%  	  \item
%  	  Suppose that $\Jmc \models \bquery^*$. Then, there exists a match $\match$ for $\bquery^*$ in $\Jmc$, as stated in Definition~\ref{def:sem:cqs}. We define a mapping $\match_p$ into $\Jmc_p$ as follows:
%  	  %
%  	  \begin{itemize}
%  	  	\item
%  	  	for all  $y\in \varsq(\bquery_x^*)$: $\match_p(y)=d_0$, where $d_0$ is the copy in $\Delta_0^\Jmc$ of the domain element $d=\match(y)$.
%  	  	%
%  	  	\item
%  	  	$\match_p(b)=b$ for all $b \in \ind(\bquery_x^*)$.
%  	  \end{itemize} 
%  	  %
%  	  We show that $\match_p$ is a match for $\query_x^*$  in $\Jmc_p$. Let us consider the form of the atoms in $\query_x^*$.
%  	  %
%  	  \begin{itemize}
%  	  	   \item
%  	  	   $A(y)$ for a variable $y$. We know that $A(y)$ is also an atom in $\query^*$. Let $\match(y)=d\in\Delta^\Jmc$. Since $\match$ is a match for $\query^*$ in $\Jmc$, it follows that $d \in A^\Jmc$. Hence, the definition of $\Jmc_p$ yields that $d_0 \in A^\Jmc$.
%  	  	   %
%  	  	   Thus, $\match_p(y) \in A^\Jmc_p$.
%  	  	   %
%  	  	   \item
%  	  	   $A(b)$ for an individual name $b$.
%  	  \end{itemize}
%  \end{itemize}
%     %
    %
	We continue by showing the \emph{right-to-left} direction.
	
	($\Leftarrow$)
	%
	Suppose that $\bquery$ is non-empty for $\Sigma$ \wrt $\Tmc$. This means that there is a $\Sigma$-ABox $\Amc$ and a tuple $t=(b_1,\ldots,b_k)$ of individual names $b_1,\ldots,b_k \in \iset$ such that $(\Tmc,\Amc)$ is consistent and $t \in\cans{(\Tmc,\Amc)}{\bquery}$.
	%
	Since $\Tmc$ contains no occurrences of individual names, there exists a model $\Imc$ of $(\Tmc,\Amc)$ such that $a_i^\Imc \neq a_j^\Imc$ for all $i \neq j, 1 \leq i, j \leq m$. 
	
   Consider the ABox $\Amc_p = \Amc \cup \{A_i(a_i) \mid 1 \leq i \leq m\}$ and the interpretation $\Imc_p$ that extends $\Imc$ by defining $A_i^{\Imc_p} := \{a_i^\Imc\}$ for all $i \in \{1,\ldots m\}$. It is clear that $\Amc_p$ is a $\Sigma_p$-ABox. In addition, it is easy to see that $\Imc_p \models (\Tmc_p, \Amc_p)$. Hence, $(\Tmc_p,\Amc_p)$ is a consistent KB.
    %
    Further, since $\Tmc \subseteq \Tmc_p$ and $\Amc \subseteq \Amc_p$, every model $\Jmc$ of $(\Tmc_p, \Amc_p)$ is a model of $(\Tmc,\Amc)$. This implies that $\Jmc \models \bquery[t]$. Since each individual $a_i$ is replaced by $x_{a_i}$ to obtain $\bquery_x$ from $\bquery$, it follows that $\Jmc \models \bquery_x[(b_1,\ldots,b_k,a_1,\ldots,a_m)]$.
    %
    Moreover, the additional assertions in $\Amc_p$ ensure that $\Jmc \models \query_{\disj}[(a_1,\ldots,a_m)]$. Hence, we have that:
    %
    \begin{equation*}
    	(b_1,\ldots,b_k,a_1,\ldots,a_m) \in \cans{(\Tmc_p,\Amc_p)}{\bquery_x \land \query_{\disj}}.
    \end{equation*} 
    
     Thus, we have shown that $\bquery_p$ is non-empty for $\Sigma_p$ \wrt $\Tmc_p$.\\
     
     To conclude, let us look at Claim 2 of the lemma. Suppose there exists a $\Sigma_p$-ABox $\Amc_p$ such that $(\Tmc_p,\Amc_p)$ is consistent and $\cans{(\Tmc_p,\Amc_p)}{\bquery_p} \neq \emptyset$. In addition, let $t \in \cans{(\Tmc_p,\Amc_p)}{\bquery_p}$.
     %
     In the \emph{left-to-right} direction of the proof of Claim 1, we show how to transform $\Amc_p$ into a $\Sigma$-ABox $\Amc$ that witnesses non-emptiness of $\bquery$ for $\Sigma$ \wrt $\Tmc$.
     %
     In particular, $\Amc$ is obtained by performing two operations:
     %
     \begin{enumerate}
     	\item
     	Removing from $\Amc_p$ all assertions of the form $A_i(b)$ for some individual $b$ and $i \in \{1,\ldots,m\}$, and 
     	%
     	\item
     	Renaming all remaining occurrences of individual names $b_{k+i}$ ($1 \leq i\leq m$). 
     \end{enumerate}  
     %
     This can be done by iterating over all assertions occurring in $\Amc_p$ and checking for the occurrence of $A_i$ and $b_{k+i}$. Since $t$ is a tuple of $k+m$ elements and all individuals names $b_{k+i}$ occur in $t$, this iteration takes polynomial time in the size of $\Amc_p$ and $t$.
\end{proof}

\solvabilitykbalignment*

\begin{proof}
	%In the proof we use ``the size of the input'' to refer to the size of an instance of KB Alignment. Note that this corresponds to the input of Algorithm~\ref{alg:kb:alignment}, i.e., the KB $\Kmc$ and the query $\bquery$.	
	To see that KB Alignment is solvable in \emph{double exponential time}, we analyze the complexity of Algorithm~\ref{alg:kb:alignment}. In the following, we use $\mathcal{N}$ to denote the combined size of the KB $\Kmc$ and the query $\bquery$ in the input of Algorithm~\ref{alg:kb:alignment}.
	%
	Let us continue by analyzing each step of the algorithm.
	%
	\begin{itemize}				
		%
		\item
		As explained in the paper, the initial ABox $\Amc'$  can be selected as $\Amc_{\Tmc,\Sigma}$ when the input query $\bquery$ is pure, where $\Sigma=\sig(\Kmc)\cup\sig(\bquery)$. Then, testing whether $(\Tmc,\Amc')$ entails $\bquery$ can be done in \emph{double exponential time} in $\mathcal{N}$. The reason is that $\Amc_{\Tmc,\Sigma}$  can be of exponential size, and entailment for $\bcBCQ$ in $\alc$ is an ExpTime-complete problem.
		%
		
		In case $\bquery$ is \emph{non-pure}, the algorithm first checks whether $\bquery_p$ is non-empty for $\Sigma_p$ \wrt $\Tmc_p$. This can be done by looking directly at $\Amc_{\Tmc_p,\Sigma_p}$ and checking whether $\cans{(\Tmc_p,\Amc_{\Tmc_p,\Sigma_p})}{\bquery_p} \neq \emptyset$. The latter can be done in \emph{double exponential time} in $\mathcal{N}$: 
		%
		\begin{itemize}
			\item
			Since the increase of the size of $\Tmc_p$ and $\Sigma_p$ \wrt $\Tmc$ and $\Sigma$ is polynomial in the size of the input query $\bquery$, it follows that  the ABox $\Amc_{\Tmc_p,\Sigma_p}$ is of size (at most) exponential and can be computed in exponential time in $\mathcal{N}$. 
			%
			Therefore, checking whether $\cans{(\Tmc_p,\Amc_{\Tmc_p,\Sigma_p})}{\bquery_p} \neq \emptyset$ can be done in \emph{double exponential time} in $\mathcal{N}$.
		\end{itemize}
		
		If the  check $\cans{(\Tmc_p,\Amc_{\Tmc_p,\Sigma_p})}{\bquery_p}\neq \emptyset$ is positive, then there is $t\in \cans{(\Tmc_p,\Amc_{\Tmc_p,\Sigma_p})}{\bquery_p}$, which can be obtained while doing the check. Then, $\Amc'$ can be selected as the ABox that results from applying to $\Amc_{\Tmc_p,\Sigma_p}$ the transformation described in the proof of Lemma~\ref{lem:remove:individuals} \wrt the tuple $t$.
		%
		This transformation is polynomial in the size of $\Amc_{\Tmc_p,\Sigma_p}$ and $t$. Since the arity of $t$ is lineal in the size of the input query $\bquery$, it follows that the obtained ABox  is of size at most exponential in $\mathcal{N}$.
		%
		
		 Overall, we have shown that the initial ABox $\Amc'$ is of size at most exponential in $\mathcal{N}$, and that the first step of Algorithm~\ref{alg:kb:alignment} can be executed in \emph{double exponential time}.
		%		
		\item
		By definition of a trivial modification, the initial modification $\mods^*$ in Algorithm~\ref{alg:kb:alignment} is a sequence of the form 
		%
		\begin{equation*}
			\op_1\cdots \op_k \cdots \op_n,
		\end{equation*}
	%
	where $\op_1 \cdots \op_k$ consists of the removal of all assertions in $\Amc$ and $\op_{k+1} \cdots \op_n$ consists of the insertion of all assertions occurring in $\Amc'$. Hence, $\mods^*$ can be computed in exponential time in $\mathcal{N}$. Further, the magnitude of the number $\cost(\mods^*)$  is at most exponential in $\mathcal{N}$. 
	%
	%
	\item
	Regarding the set $M$, it satisfies the following (as analyzed in Subsetion~\ref{sub:sec:kb:alignment}):
	%
	\begin{itemize}
		\item
		Each ABox-modication $\mods$ in $M$ is defined over the finite signature $\sig(\Kmc) \cup \sig(\bquery)$ and satisfies $\cost(\mods) < \cost(\mods^*)$, and
		%
		%
		\item
		$M$ contains only modifications with no more than $\cost(\mods^*) / c$ ABox-operations, where $c$ is the minimal cost of 
		an ABox operation defined over $\sig(\Kmc) \cup \sig(\bquery)$.		
	\end{itemize}
     %
     Since the magnitude of $\cost(\mods^*)$ is at most exponential in $\mathcal{N}$, each modification $\mods \in M$ contains a number of ABox-operations that is at most exponential in $\mathcal{N}$. Moreover, since each ABox-operation is defined over a symbol in $\sig(\Kmc) \cup \sig(\bquery)$, $M$ can be enumerated in \emph{double exponential time} in the size of $\mathcal{N}$.
     %
     %
     \item
     In the enumeration of $M$, checking whether the conditional at Line~\ref{alg1:line:if} holds, amounts to:
     %
     \begin{itemize}
     	\item
     	applying $\mods$ to the input ABox $\Amc$ to obtain an ABox $\Amc^*$,
     	\item 
     	checking whether $(\Tmc,\Amc^*)$ is consistent and $(\Tmc,\Amc^*) \models \bquery$, and
     	%
     	\item
     	verifying whether $\cost(\mods) < \cost(\mods^*)$, where $\mods^*$ is the modification of minimal cost encountered so far along the enumeration.
     \end{itemize}
       %
       From the discussions in the previous points, $\Amc^*$ is of size at most exponential and can be computed in exponential time in $\mathcal{N}$.
       %
       Moreover, since knowledge base consistency and entailment of $\bcBCQ$ can be checked in exponential time for $\alc$, then checking whether $(\Tmc,\Amc^*)$ is consistent and $(\Tmc,\Amc^*) \models \bquery$ can be done in \emph{double exponential time} in $\mathcal{N}$. Finally, since the magnitude of $\cost(\mods)$ and $\cost(\mods^*)$ are not greater than the initially computed upper bound, checking whether $\cost(\mods) < \cost(\mods^*)$ holds is not more costly than \emph{double exponential time}.
      %              
 	\end{itemize}
    
     Therefore, taking into account the previous analysis, we can conclude that Algorithm~\ref{alg:kb:alignment} runs in \emph{double exponential time} in $\mathcal{N}$. Thus, KB Alignment can be solved in \emph{double exponential time}.
\end{proof}

%\begin{proof}
%	Let $x_1,\ldots,x_k$ be the free variables of $\bquery$ and $a_1,\ldots,a_m$ the individual names occurring in $\bquery$. Then, $\bquery_x$ has $k+m$ free variables, i.e.
%	%
%	\begin{equation*}
%		\fvarsq(\bquery_x) =\{x_1,\ldots,x_k\} \cup \{x_{a_1},\ldots,x_{a_m}\},
%	\end{equation*}
%	%
%	where each $x_{a_i}$ is fresh variable not occurring in $\bquery$.  We represent the free variables as the ordered tuple $(x_1,\ldots,x_k, x_{a_1},\ldots,x_{a_m})$.
%	
%    Let us start by showing the two implications in the first claim of the lemma.
%	
%	($\Rightarrow$) Assume that $\bquery_x$ is non-empty for $\Sigma$ \wrt $\Tmc$. Hence, there exists a $\Sigma$-ABox $\Amc_x$ such that  $(\Tmc,\Amc_x)$ is consistent and  $\cans{(\Tmc,\Amc_x)}{\bquery_x} \neq \emptyset$. This gives us a tuple $t=(b_1,\ldots,b_{k+m})\in \ind(\Amc_x)^{k+m}$ such that $t \in \cans{(\Tmc,\Amc_x)}{\bquery_x}$.
%	%
%	Since neither $\bquery_x$ nor $\Tmc$ contain individual names, we can assume that $\Amc_x$ contains no individuals from $\ind(\bquery)=\{a_1,\ldots,a_m\}$.		
%	%The individuals $b_1,\ldots,b_k$ are assigned to $x_1,\ldots,x_k$, whereas $b_{k+1},\ldots,b_{k+m}$ are assigned to the variables $x_{a_1},\ldots,x_{a_m}$.
%	%
%	
%	
%	
%	
%	Based on this, the goal is to build a $\Sigma$-ABox $\Amc$ that witness non-emptiness of $\bquery$ for $\Sigma$ \wrt $\Tmc$. To this end, we first need to extend $\Amc_x$ into a $\Sigma$-ABox $\Amc'_x$ that also witnesses non-emptiness of $\bquery_x$. We will then use $\Amc'_x$ to define $\Amc$.
%	
%	Let us continue with the construction of $\Amc'_x$, which requires to introduce the following notions. The tuple $t$ defines an equivalence relation $R_t$ over $\{x_{a_1},\ldots,x_{a_m}\}$ as follows:
%	%
%	\begin{equation}\label{equiv:relation:t}
%		R_t := \{(x_{a_i},x_{a_j}) \mid 1 \leq i,j \leq m \land b_{k+i} = b_{k+j}\}.
%	\end{equation}
%  %
%  Further, let $S$ be the set of atoms of the form $r(x_{a_i},x_{a_j})$ occurring in $\bquery_x$ such that $r\in \Sigma$, $i\neq j$ and $(x_{a_i},x_{a_j}) \in R_t$. We select a subset $S_m \subseteq S$ of maximal size satisfying the following property:
%   %
%   \begin{itemize}
%   	   \item
%   	   There is a model $\Imc$ of $(\Tmc,\Amc_x)$ such that:
%   	   %
%   	   \begin{equation}\label{loop:property}
%   	   	\text{ for all } r(x_{a_i},x_{a_j}) \in S_m:\ \ (b_{k+i}^\I,b_{k+i}^\I) \in r^\I.
%   	   \end{equation} 
%   \end{itemize}
%   %
%   Then, the ABox $\Amc'_x$ is defined as:
%   %
%   \begin{equation*}
%   	  \Amc'_x = \Amc_x \cup \{r(b_{k+i},b_{k+i}) \mid r(x_{a_i},x_{a_j}) \in S_m\}
%   \end{equation*}
%   %
%   The following observations follow from the definition of $\Amc'_x$:
%   %
%   \begin{itemize}
%   	  \item
%   	  $\Amc'_x$ is a $\Sigma$-ABox. Note the each new assertion $r(b_{k+i},b_{k+i})$ comes from $r(x_{a_i},x_{a_j}) \in S_m$, which means that $r \in \Sigma$.
%   	  %
%   	  %
%   	  \item
%   	  $(\Tmc,\Amc'_x)$ is consistent. A witness for this is the model $\Imc$ of $(\Tmc,\Amc_x)$ used to select $S_m$. It satisfies the new assertions in $\Amc'_x$, because it satisfies \eqref{loop:property}.
%   	  %
%   	  %
%   	  \item
%   	  $t \in \cans{(\Tmc,\Amc'_x)}{\bquery_x}$. This follows from that  fact that every model of $\Amc'_x$ is a model of $\Amc_x$ and that $t \in \cans{(\Tmc,\Amc_x)}{\bquery_x}$.
%   \end{itemize}
%   %
%   Hence, $\Amc'_x$ also witnesses non-emptiness of $\bquery'$ for $\Sigma$ \wrt $\Tmc$. We are now ready to define the ABox $\Amc$.
%   %
%   Let $B$ be the following subset of individuals of $\Amc'_x$:
%   %
%   \begin{equation*}
%   	   B = \{b \in \ind(\Amc'_x) \mid b \neq b_{k+i} \text{ for all } i \in \{1,\ldots,m\}\}.
%   \end{equation*}
%    %
%    The ABox $\Amc$ is obtained from $\Amc'_x$, $t$ and $R_t$ as follows.
%   	%
%   	\begin{itemize}
%   		\item
%   		The following set of concept assertions:
%   		 %
%   		 \begin{multline*}
%   		 	\{A(b) \mid A(b) \in \Amc'_x, \land\ b \in B\}  \cup \\[.2em]
%   		 	%
%   		 	\{A(a_j) \mid A(b_{k+i}) \in \Amc'_x, (x_{a_i},x_{a_j}) \in R_t \land 1 \leq i,j \leq m\}.
%   		 	%   		 	
%   		 \end{multline*} 
%   		%
%   		\item
%   		The following set of role assertions:
%   		%
%   		\begin{multline*}
%   			\{r(b,c) \mid r(b,c) \in \Amc'_x \land b,c\in B \} \cup \\[.2em]   			
%   			%
%   			\{r(b,a_j) \mid r(b,b_{k+i}) \in \Amc'_x, (x_{a_i},x_{a_j}) \in R_t, b \in B\} \cup\\[.2em]
%   			%
%   			\{r(a_j,b) \mid r(b_{k+i},b) \in \Amc'_x, (x_{a_i},x_{a_j}) \in R_t, b \in B\} \cup\\[.2em]
%   			%
%   			\{r(a_i,a_j) \mid r(b_{k+p},b_{k+\ell}) \in \Amc'_x,  (x_{a_i},x_{a_p}) \in R_t,  (x_{a_j},x_{a_\ell}) \in R_t, \}
%   		\end{multline*}
%   		%
%   	\end{itemize}
%   
%   We show that $\Amc$ witnesses non-emptiness of $\bquery$ for $\Sigma$ \wrt $\Tmc$. To see that $(\Tmc,\Amc)$ is consistent, consider a model $\Imc$ of $(\Tmc,\Amc'_x)$. We extend $\Imc$ to interpret the individual names $a_i$ as:
%  \begin{itemize}
%  		\item
%          $a_i^\Imc = b_{k+i}^\Imc$ ($1 \leq i \leq m$).                    \
%   \end{itemize} 
%    %
%    Note that this does not change the interpretation of the individual names in $\Amc'_x$, since $a_i \not\in \ind(\Amc'_x)$. Further, since no individual name occurs in $\Tmc$, the extended $\Imc$ is still a model of $\Tmc$. 
%    %
%    To see that $\Imc \models \Amc$ as well, we consider the possible forms of the assertions in $\Amc$.
%    %
%    \begin{itemize}
%    	\item
%    	For assertions of the form $A(b) \in \Amc$ for some $b \in B$, we have that $A(b)\in \Amc'_x$. Hence, $\Imc \models \Amc'_x$ implies that $b^\Imc \in A^\Imc$.
%    	%
%    	In case $A(a_j) \in \Amc$, the construction of $\Amc$ yields that $A(b_{k+i}) \in \Amc'_x$ for some $i \in \{1,\ldots,m\}$ such that $(x_{a_i},x_{a_j}) \in R_t$. This means that $b_{k+i}^\I \in A^\Imc$ (since $\Imc \models \Amc'_x$), and that $b_{k+i}=b_{k+j}$ (see \eqref{equiv:relation:t}).
%    	%
%    	Thus, since $a_j^\Imc = b_{k+j}^\Imc$, it follows that $a_j^\Imc \in A^\Imc$.
%    	%
%    	\item
%    	\textcolor{red}{the roles...}
%    \end{itemize}    
%     %
%     Hence, $(\Tmc,\Amc)$ is consistent. 
%     
%     It remains to see that $\cans{(\Tmc,\Amc)}{\bquery} \neq \emptyset$.   
%     %
%     To this end, we show that $(b_1,\ldots,b_k) \in \cans{(\Tmc,\Amc)}{\bquery}$. Let $\Jmc$ be an arbitrary model of $(\Tmc,\Amc)$. It is enough to prove that 
%     	\begin{equation}\label{eq:ent:to:show}
%     		\Jmc \models \bquery[(b_1,\ldots,b_k)].
%     	\end{equation}
%      %
%      For each equivalence class $[x_{a_j}]$ of $R_t$, we define the interpretation in $\Jmc$ of $b_{k+j}$ as 
%      %
%      \begin{equation*}
%      	b_{k+j}^\J = a_j^\J,
%      \end{equation*}
%       %
%       where $x_{a_j}$ is a fixed representative of $[x_{a_j}]$. The rest of the proof consists of showing that $\Jmc$ is a model of $(\Tmc,\Amc'_x)$ and use the fact that $t \in \cans{(\Tmc,\Amc'_x)}{\bquery_x}$ to show that \eqref{eq:ent:to:show} is true.
%       
%       It is obvious that $\Jmc$ remains a model of $\Tmc$. To see that it is a model of $\Amc'_x$, we look at the possible form of the assertions in $\Amc'_x$.
%       %
%       \begin{itemize}
%       	\item
%       	$A(b) \in \Amc'_x$ or $r(b,c) \in \Amc'_x$, where $b,c \in B$. In both cases, the assertions $A(b)$ and $r(b,c)$ belong to $\Amc$. Since $b,c \in B$ and $\Jmc \models (\Tmc,\Amc)$, it follows that $\Jmc$ satisfies these type of assertions.
%       	%
%       	%
%       	\item
%       	$A(b_{k+i}) \in \Amc'_x$ for some $i \in \{1,\ldots,m\}$. Let $x_{a_j}$ be the fixed representative of the equivalence class for $x_{a_i}$. By construction of $\Amc$, we have that $A(a_j) \in \Amc$, since $(x_{a_i},x_{a_j})\in R_t$. 
%       	%
%       	This implies that $a_j^\Jmc \in A^\Jmc$. Thus, since $b_{k+i}=b_{k+j}$ and $b_{k+j}^\Jmc = a_j^\Jmc$, it follows $b_{k+i}^\Jmc \in A^\Jmc$.
%       	%
%       	%
%       	\item
%       	$r(b_{k+p},b_{k+\ell}) \in \Amc'_x$ for some  $p,\ell \in \{1,\ldots,m\}$. Let $x_{a_i}$ and $x_{a_j}$ be the fixed representatives of the equivalence classes for $x_{a_p}$ and $x_{a_\ell}$, respectively. Again, the construction of $\Amc$ tells us that $r(a_i,a_j) \in \Amc$. This implies that $(a_i^\Jmc,a_j^\Jmc) \in r^\Jmc$, since $\Jmc \models \Amc$.
%       	%
%       	Thus, the same argument used in the previous case yields that $(b_{k+p}^\Jmc,b_{k+\ell}^\Jmc) \in r^\Jmc$.
%       	%
%       	%
%       	\item
%       	$r(b,b_{k+i}) \in \Amc'_x$ or $r(b_{k+i},b) \in \Amc'_x$. This case can be shown similarly to the previous cases.
%       \end{itemize}
%       
%       Overall, we have shown that $\Jmc \models (\Tmc,\Amc_x')$. Hence, since $t \in \cans{(\Tmc,\Amc'_x)}{\varphi_x}$, we have that:
%       %
%       \begin{equation*}
%       	\Jmc \models \bquery_x[(b_1,\ldots,b_k,b_{k+1},\ldots,b_{k+m})].
%       \end{equation*}
%      
%\end{proof}
\fi
\end{document}
%%%%%%%%%%%%%%%%%%%%%%%%%%%%%%%%%%%%%%%%%%%%%%%%%%%%%%%%%%%%%%%%%%%%%%
