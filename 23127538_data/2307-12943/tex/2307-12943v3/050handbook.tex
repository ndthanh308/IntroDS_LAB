\global\long\def\vec{\textup{\textsf{vec}}}%
\global\long\def\svec{\textup{\textsf{svec}}}%


\section{Structured densities and constraint families \label{sec:handbook-barrier}}

In order to obtain a mixing-time bound of the $\dw$ for the reduced
problem, a concrete understanding of properties and parameters of
barriers for $K_{i}$ and $K_{j}$ is essential. To this end, we revisit
self-concordant barriers for structured convex constraints and level
sets, examining the required scaling factors which ensure those properties.

\subsection{Linear constraints}

Consider a set of linear constraints given by $K=\Brace{x\in\Rn:Ax\geq b}$
for $A\in\R^{m\times n}$ and $b\in\R^{m}$, where $A$ has no all-zero
rows. We use $s_{x}:=Ax-b$ to denote the slack at $x$ and use $S_{x}:=\Diag(s_{x})$
to denote its diagonalization. Then $A_{x}:=S_{x}^{-1}A$ is the constraint
normalized by the slack.

We now introduce three barriers (and metrics) for handling the linear
constraints.

\paragraph{Logarithmic barrier.}

The logarithmic barrier $\phi_{\log}(x):=-\sum_{i=1}^{m}\log(a_{i}\cdot x-b_{i})$
is the simplest self-concordant barrier for linear constraints. We
refer readers to Section~\ref{proof:linear-log-barrier} for gentle
introduction to the log-barriers. As seen below, we demonstrate that
the metric induced by the logarithmic barrier has $\nu,\onu=m$ and
requires no scaling to achieve SSC, SLTSC, and SASC.
\begin{lem}
[Logarithmic barrier]\label{lem:log-barrier} For a closed convex
$K=\{x\in\Rn:Ax\geq b\}$ with $A\in\R^{m\times n}$ and $b\in\R^{m}$,
let $\phi_{\log}(x)=-\sum_{i=1}^{m}\log(a_{i}^{\top}x-b_{i})$ and
define $g(x):=\hess\phi_{\log}(x)=A_{x}^{\top}A_{x}$.
\begin{itemize}
\item $\nu=m$ (\cite{nesterov1994interior}).
\item SSC along $\rowspace(A)$ and $\onu=m$ (Lemma~\ref{lem:paramsBarrier}).
\item $D^{2}g(x)[h,h]\succeq0$ for any $h\in\Rn$ (so SLTSC) (Claim~\ref{claim:diffLogBarrier}).
\item SASC (Lemma~\ref{lem:logBarrier-SASC}).
\end{itemize}
\end{lem}


\paragraph{Vaidya metric.}

In sampling over a polytope $K$, the number $m$ of constraints is
assumed to be greater than the ambient dimension $n$. Given that
the mixing time of the $\dw$ for uniform sampling is $\otilde{n\onu}=\otilde{nm}$,
a larger $m$ leads to a worse mixing time. Is there a self-concordant
barrier that has a better dependence on $m$ for its self-concordance
and symmetry parameters, without compromising SSC, SLTSC, and SASC?

Let us recall the \emph{leverage score} first and move onto such improved
self-concordant barriers. For a full-rank matrix $A\in\R^{m\times n}$
with $m\geq n$, we recall that $P(A):=A(A^{\top}A)^{-1}A^{\top}$
is the orthogonal projection matrix onto the column space of $A$.
The leverage scores of $A$ is denoted by $\sigma(A):=\diag(P(A))\in\R^{m}$.
We let $\Sigma(A):=\Diag(\sigma(A))=\Diag(P(A))$ and $P^{(2)}(A):=P(A)\circ P(A)$,
where $P(A)\circ P(A)$ is the Hadamard product of size $n\times n$
defined by $(P(A)\circ P(A))_{ij}=P(A)_{ij}^{2}$.

\cite{vaidya1996new} introduced the \emph{volumetric barrier} for
$K$ defined by
\[
\phi_{\vol}=\half\log\det\hess\phi_{\log}=\half\log\det A_{x}^{\top}A_{x}.
\]
Then the Hessian of $\phi_{\vol}$ can be written as 
\[
\hess\phi_{\vol}=A_{x}^{\top}(3\Sigma_{x}-2P_{x}^{(2)})A_{x},
\]
where $\Sigma_{x}=\Diag\Par{\sigma(A_{x})}$ is the diagonalized leverage
scores, and this Hessian satisfies 
\[
A_{x}^{\top}\Sigma_{x}A_{x}\preceq\hess\phi_{\vol}(x)\preceq3A_{x}^{\top}\Sigma_{x}A_{x}.
\]
We refer readers to Section~\ref{proof:linear-volumetric} for details.
In other words, the \emph{approximate} volumetric metric $A_{x}^{\top}\Sigma_{x}A_{x}$
serves as an $O(1)$-approximation of the local metric $\hess\phi_{\vol}$
(i.e., $A_{x}^{\top}\Sigma_{x}A_{x}\asymp\hess\phi_{\vol}(x)$). We
find in Lemma~\ref{lem:paramsBarrier} that the local metric $40\sqrt{m}A_{x}^{\top}\Sigma_{x}A_{x}$
is SSC with $\nu,\,\onu=O(\sqrt{m}n)$, but in some regime of $n$
this parameter leads to a worse mixing time of the $\dw$. In the
same paper, \cite{vaidya1996new} introduced a \emph{regularized}
volumetric metric by adding the Hessian of the logarithmic barriers,
which we call the \emph{Vaidya metric}:

\[
g(x):=\sqrt{\frac{m}{n}}A_{x}^{\top}\Par{\Sigma_{x}+\frac{n}{m}I_{m}}A_{x}.
\]
Note that $g(x)\asymp\hess\Par{\sqrt{\frac{m}{n}}\Par{\phi_{\vol}+\frac{n}{m}\text{\ensuremath{\phi_{\log}}}}}$.
We show that the Vaidya metric is also SSC, SLTSC, and SASC without
additional scaling, while it has a better $\nu$ and $\onu$ than
the logarithmic barrier.
\begin{lem}
[Vaidya metric]\label{lem:vaidya} For a closed convex $K=\{x\in\Rn:Ax\geq b\}$
with $A\in\R^{m\times n}$ and $b\in\R^{m}$, let $g(x)=\sqrt{\frac{m}{n}}A_{x}^{\top}\Par{\Sigma_{x}+\frac{n}{m}I_{m}}A_{x}.$
\begin{itemize}
\item $\nu=O(\sqrt{mn})$ (Theorem 5.2 in \cite{anstreicher1997volumetric}).
\item SSC and $\onu=O(\sqrt{mn})$ (Lemma~\ref{lem:paramsBarrier}).
\item SLTSC (Lemma~\ref{lem:vaidya-SLTSC}) and SASC (Lemma~\ref{lem:vaidya-SASC}).
\end{itemize}
\end{lem}


\paragraph{Lewis weights metric.}

Self-concordance and symmetry parameters of $O(\sqrt{mn})$ is certainly
better than $O(m)$, but can we even achieve an $O\Par{n\log^{O(1)}m}$
bound on those parameters?

Let us recall the $\ell_{p}$-\emph{Lewis weights}. The $\ell_{p}$-Lewis
weight of $A$ is denoted by $w(A)$, the solution $w$ to the equation
$w(A)=\diag\Par{W^{\half-\frac{1}{p}}A\Par{A^{\top}W^{1-\frac{2}{p}}A}^{-1}A^{\top}W^{\half-\frac{1}{p}}}\in\R^{m}$
for $W:=\Diag(w)$. For $W_{x}$ the diagonalized $\ell_{p}$-Lewis
weights of $A_{x}$ with $p\geq2$, the Lewis weight barrier function
is defined by 
\[
\phi_{\lw}(x):=\log\det\Par{A_{x}^{\top}W_{x}^{1-\frac{2}{p}}A_{x}}.
\]
Note that the leverage score and volumetric barrier can be recovered
as a special case of the Lewis weight and barrier by setting $p=2$.
As done for the Vaidya metric, it is natural to consider the Lewis
weight metric with $p=\Theta\Par{\log^{O(1)}m}$, defined as 
\[
g(x):=O(\log^{O(1)}m)A_{x}^{\top}W_{x}A_{x}.
\]
In fact, this metric serves as an $O\Par{\log^{O(1)}m}$-approximation
of $\hess\phi_{\lw}$, as demonstrated in the following relation proven
in Lemma~31 of \cite{lee2019solving}:
\[
A_{x}^{\top}\Sigma_{x}A_{x}\preceq\hess\phi_{\lw}\preceq(1+p)A_{x}^{\top}\Sigma_{x}A_{x}.
\]
Hence, ignoring the logarithmic factors we have $\hess\phi_{\lw}(x)\asymp g(x)$.
Notably, the Lewis-weight metric needs an additional $\sqrt{n}$-scaling
for SLTSC and SASC, unlike the logarithmic barrier and Vaidya metric.
Hence, when combining this with other metrics, one should use $\sqrt{n}g$,
which leads to $\nu,\,\onu=O\Par{n^{3/2}\log^{O(1)}m}$. 
\begin{lem}
[Lewis weight metric]\label{lem:Lewis-weight} For a closed convex
$K=\{x\in\Rn:Ax\geq b\}$ with $A\in\R^{m\times n}$ and $b\in\R^{m}$,
let $g(x)=O(\log^{O(1)}m)A_{x}^{\top}W_{x}A_{x}$.
\begin{itemize}
\item $\nu=O\Par{n\log^{5}m}$ (Theorem 30 in \cite{lee2019solving}).
\item SSC and $\onu=O\Par{n\log^{O(1)}m}$ (Lemma~\ref{lem:paramsBarrier}).
\item $\sqrt{n}g$ is SLTSC (Lemma~\ref{lem:Lw-SLTSC}) and SASC (Lemma~\ref{lem:Lw-SASC}).
\end{itemize}
\end{lem}


\subsubsection{Analysis of self-concordant metrics for linear constraints \label{subsec:analysis-linear-metric}}

\paragraph{Strong self-concordance and symmetry.}

We defer the proofs of two lemmas below to Section~\ref{proof:linear-SSC-symm}.
We study SSC and symmetry of the metrics of the form $A_{x}^{\top}D_{x}A_{x}$
in Lemma~\ref{lem:helper4Diagonal}, where $D_{x}\in\R^{m\times m}$
is a diagonal matrix used to address the constraints of the form $Ax\geq b$
for $A\in\R^{m\times n}$ and $b\in\R^{m}$. Specifically, we relate
the notions of SSC and symmetry to well-studied terms in the field
of optimization, namely $\max_{i}\frac{\sigma\Par{\sqrt{D_{x}}A_{x}}_{i}}{(D_{x})_{ii}}$
and $\norm{DD_{x}[h]}_{D_{x}^{-1}}^{2}$. 
\begin{lem}
\label{lem:helper4Diagonal} For $x\in\inter(K)$, let $g(x)=A_{x}^{\top}D_{x}A_{x}\in\Rnn$
for a diagonal matrix $0\prec D_{x}\in\R^{m\times m}$.
\begin{itemize}
\item For any PSD matrix function $g'$ such that $g'+g$ is invertible
on the domain,
\begin{align*}
 & \norm{(g'(x)+g(x))^{-\half}Dg(x)[h](g'(x)+g(x))^{-\half}}_{F}^{2}\\
 & \qquad\qquad\leq4\max_{i}\Par{\frac{\sigma\Par{\sqrt{D_{x}}A_{x}}_{i}}{\Par{D_{x}}_{ii}}}\cdot\Par{\norm h_{g(x)}^{2}+\sum_{i=1}^{m}\Par{D_{x}^{-1}}_{ii}(DD_{x}[h])_{i}^{2}}
\end{align*}
 
\item $\max_{h:\norm h_{g(x)}=1}\norm{A_{x}h}_{\infty}=\sqrt{\max_{i\in[m]}\frac{\sigma\Par{\sqrt{D_{x}}A_{x}}_{i}}{\Par{D_{x}}_{ii}}}$.
\item $K\cap(2x-K)\subset\dcal_{g}^{\sqrt{\tr\Par{D_{x}}}}(x)$.
\end{itemize}
\end{lem}

Then for each metric we refer to existing bounds on these terms, estimating
the smallest possible scaling required for SSC and symmetry. 
\begin{lem}
[Strong self-concordance and symmetry]\label{lem:paramsBarrier}
Let $A\in\R^{m\times n}$, $\Sigma_{x}=\Diag(\sigma(A_{x}))\in\R^{m\times m}$,
and $W_{x}=\Diag(w_{x})\in\R^{m\times m}$ for the $\ell_{p}$-Lewis
weight $w_{x}$ with $p=O(\log m)$.
\begin{itemize}
\item Logarithmic metric: $g(x)=A_{x}^{\top}A_{x}$ with $D_{x}=I_{m}$
is SSC along $\rowspace(A)$ with $\onu=m$.
\item Approximate volumetric metric: $g(x)=40\sqrt{m}A_{x}^{\top}\Sigma_{x}A_{x}$
with $D_{x}=40\sqrt{m}\Sigma_{x}$ is SSC with $\onu=O(\sqrt{m}n)$.
\item Vaidya metric: $g(x)=22\sqrt{\frac{m}{n}}A_{x}^{\top}\Par{\Sigma_{x}+\frac{n}{m}I_{m}}A_{x}$
with $D_{x}=22\sqrt{\frac{m}{n}}\Par{\Sigma_{x}+\frac{n}{m}I_{m}}$
is SSC with $\onu=O(\sqrt{mn})$.
\item Lewis-weight metric: $\exists$ constants $c_{1}$ and $c_{2}$ such
that $g(x)=c_{1}\Par{\log m}^{c_{2}}A_{x}^{\top}W_{x}A_{x}$ is SSC
and $\onu$-symmetric with $\onu=O^{*}\Par n$.\label{lem:LSmetricStrongandSymmetry}
\end{itemize}
\end{lem}


\paragraph{Strongly lower trace self-concordance}

We show SLTSC of the Vaidya and Lewis-weight metric. Let $g_{2}$
be either Vaidya or Lewis-weight metric, and $g_{1}$ be an arbitrary
PSD matrix function on $K$ such that $g:=g_{1}+g_{2}$ is PD on $\intk$.
Ensuring (S)LTSC of the Vaidya or Lewis-weight metrics is challenging,
as $D^{2}g_{2}[h,h]\succeq0$ is difficult to verify due to complicated
expressions for $D^{2}\Sigma_{x}[h,h]$ and $D^{2}W_{x}[h,h]$. In
the case of the Vaidya metric, we compute higher-order derivatives
of leverage scores and other pertinent matrices in Lemma~\ref{lem:calculusLeverage},
finding succinct formulas by using algebraic properties of the Hadamard
product. Then we use these results to show SLTSC of $g_{2}$ as follows
(see Section~\ref{proof:linear-vaidya-SLTSC} for the proof):
\begin{lem}
[SLTSC of Vaidya]\label{lem:vaidya-SLTSC} $\tr\Par{g^{-1}D^{2}g_{2}(x)[h,h]}\geq-\half\norm h_{g_{2}}^{2}$
for the Vaidya metric $g_{2}$.
\end{lem}

For the Lewis-weights metric, analysis is more involved due to numerous
terms appearing in $D^{2}W_{x}[h,h]$. In order to avoid dealing with
each of the terms, we employ existing bounds on derivatives of $W_{x}$
and other relevant matrices in Section~\ref{proof:linear-LW}. This
approach significantly simplifies the computation but comes at the
cost of an additional scaling of $\sqrt{n}$, which as far as we can
tell might be unavoidable. We refer readers to Section~\ref{proof:linear-Lewis-SLTSC}
for the proof.
\begin{lem}
[SLTSC of Lewis-weight]\label{lem:Lw-SLTSC} $\tr\Par{g^{-1}D^{2}g_{2}(x)[h,h]}\geq-\norm h_{g_{2}}^{2}$,
where $g_{2}(x)=cA_{x}^{\top}W_{x}A_{x}$ with $c=c_{1}(\log m)^{c_{2}}\sqrt{n}$
for some constants $c_{1},c_{2}>0$.
\end{lem}


\paragraph{Strongly average self-concordance.}

Typically, (S)ASC is the most challenging property to verify, often
requiring involved analysis in order to establish it \emph{without}
additional scalings. Since the three metrics are HSC (e.g., see Lemma~\ref{lem:Lw-hsc}
for Lewis-weight metrics), scaling by $n$ leads to SASC by Lemma~\ref{lem:hsc-to-sasc}.
However, for linear constraints one can still achieve SASC without
scaling (or with a smaller scaling) through more sophisticated concentration
techniques.

To sketch this idea, we recall that SASC requires showing that for
small enough $r$
\[
\norm{z-x}_{g(z)}^{2}-\norm{z-x}_{g(x)}^{2}\leq2\veps\frac{r^{2}}{n}.
\]
Taylor's expansion of $\norm{z-x}_{g(z)}^{2}$ at $z=x$ up to second-order
necessitates bounds on
\begin{align*}
Dg(x)\Brack{(z-x)^{\otimes3}} & =\frac{r^{3}}{n^{3/2}}Dg(x)\Brack{h^{\otimes3}}=:P(h),\\
Dg(x')\Brack{(z-x)^{\otimes4}} & =\frac{r^{4}}{n^{2}}D^{2}g(x')\Brack{h^{\otimes4}}
\end{align*}
for some $x'\in[x,z]$ and $h\sim\ncal(0,I_{n})$. Observe that the
first-order term $P(h)$ is a Gaussian polynomial in $h$, and this
is where we can invoke the following concentration phenomenon:
\begin{lem}
[Concentration of Gaussian polynomials] \label{lem:conc-gaussian-poly}
For $n\geq1$, let $P:\Rn\to\R$ be a polynomial of degree $d$. For
any $t\geq(2e)^{d/2}$ and $h\sim\ncal(0,I_{n})$ 
\[
\P\Brack{\Abs{P(h)}\geq t\sqrt{\E\Par{P(h)^{2}}}}\leq\exp\Par{-\frac{d}{2e}t^{2/d}}.
\]
\end{lem}

This concentration inequality necessitates bounding $\E(P(h))^{2}$,
and this is where Stein's lemma comes into play:
\begin{lem}
\label{lem:stein} For $h=(h_{1},\dots,h_{n})\sim\ncal(0,I_{n})$,
it holds that $\E\Par{h_{i}f(h)}=\E\Par{\del_{i}f(h)}$.
\end{lem}

Unlike the first-order term, the second-order term is \emph{not} a
Gaussian polynomial due to $x'$ depending on $z$. To address this
issue, we derive an upper bound (in absolute value) of the quadratic
form. Using coordinate-wise closeness of slacks, leverage scores,
and Lewis weights at two nearby points, we replace every value estimated
at $z$ by those at $x$, removing dependence on $z$ in the quadratic
bound. The resulting quadratic bound is now a Gaussian polynomial,
so we follow the same proof approach as with the first-order term.

This approach was used by \cite{sachdeva2016mixing} for ASC of log-barriers
and by \cite{chen2018fast} for that of Vaidya and Lewis-weight metrics.
We further extend this approach to achieve SASC of those metrics,
going beyond ASC.
\begin{lem}
[SASC of logarithmic barrier] \label{lem:logBarrier-SASC} $g(x)=\hess\phi_{\log}(x)=A_{x}^{\top}A_{x}$
is SASC.
\end{lem}

See Section~\ref{proof:linear-SASC-log} for the proof.
\begin{lem}
[SASC of Vaidya metric] \label{lem:vaidya-SASC} $g(x)=O(1)\sqrt{\frac{m}{n}}A_{x}^{\top}(\Sigma_{x}+\frac{n}{m}I)A_{x}$
is SASC.
\end{lem}

See Section~\ref{proof:linear-SASC-vaidya} for the proof.
\begin{lem}
[SASC of Lewis-weight metric] \label{lem:Lw-SASC} There exists constants
$c_{1}$ and $c_{2}$ such that $g(x)=c_{1}\sqrt{n}\log^{c_{2}}mA_{x}^{\top}W_{x}A_{x}=O^{*}(\sqrt{n})A_{x}^{\top}W_{x}A_{x}$
is SASC.
\end{lem}

See Section~\ref{proof:linear-SASC-Lw} for the proof.

\subsection{Quadratic potentials and constraints}

Suppose that in (\ref{eq:reduced-problem}) we have either $f_{i}(x),\,h_{j}(x)=\norm{x-\mu}_{\Sigma}^{2}$
or $\half x^{\top}Qx+p^{\top}x+l$ for $\mu,p\in\Rn$, $\Sigma\in\pd$,
and $0\neq Q\in\psd$.

\paragraph{Quadratic constraint.}

Consider a second-order region given by $K=\Brace{x\in\Rn:\half x^{\top}Qx+p^{\top}x+l\leq0}$.
\cite{nesterov1994interior} shows that for $f(x):=-\half\norm{x-\mu}_{\Sigma}^{2}$
or $-\Par{\half x^{\top}Qx+p^{\top}x+l}$
\[
\phi(x)=-\log f(x)
\]
is an $1$-self-concordant barrier for $K$. Since $\onu=O(\nu^{2})$
for a self-concordant barrier due to Lemma~\ref{lem:bound-symmetry},
$\phi$ is $O(1)$-symmetric. In case we consider $\norm{x-\mu}_{\Sigma}^{2}$,
the trivial scaling by dimension $n$ implies that $n\phi$ is SSC
and $O(n)$-symmetric.

Moreover, $n\phi$ is SASC by Lemma~\ref{lem:hsc-to-sasc} by HSC
of $\phi$. For HSC of $\phi$, we develop a handy tool for checking
HSC. See Section~\ref{proof:quadratic} for the proof.
\begin{lem}
\label{lem:4th-log} For a real-valued function $f$ on $K\subset\Rn$,
let $\psi$ be a $\nu$-self-concordant barrier of the form $\psi(x)=-\log f(x)$
for $K$. Then 
\[
|D^{4}\psi(x)[h,h,h,h]|\lesssim\nu^{2}\norm h_{\hess\psi(x)}^{2}+\Abs{\frac{D^{4}f(x)[h,h,h,h]}{f(x)}}.
\]
\end{lem}

Using this tool, we can study properties of the barrier for the quadratic
constraints. We provide the proof in Section~\ref{proof:quadratic}.
\begin{lem}
[Quadratic constraint]\label{lem:quadratic-const} For a closed convex
$K=\Brace{x\in\Rn:\half x^{\top}Qx+p^{\top}x+l\leq0}$ with $p\in\R^{n}$
and $0\neq Q\in\psd$, let $\phi(x)=-\log\Par{-l-p^{\top}x-\half x^{\top}Qx}$
and $g(x)=n\hess\phi(x)$.
\begin{itemize}
\item $\nu,\,\onu=O(n)$.
\item SSC when $Q\succ0$, and SASC.
\item $D^{2}g(x)[h,h]\succeq0$ for any $x\in\inter(K)$ and $h\in\Rn$
(so SLTSC).
\end{itemize}
\end{lem}


\paragraph{Gaussian distribution ($f(x)=\protect\half\protect\norm{x-\mu}_{\Sigma}^{2}$).}

Suppose the quadratic term $f(x)=\half\norm{x-\mu}_{\Sigma}^{2}$
appears in a potential of a target distribution. Then its epigraph
is 
\[
\Brace{(x,t)\in\R^{n+1}:\half\norm{x-\mu}_{\Sigma}^{2}-t\leq0},
\]
and clearly $q(x,t)=\half\norm{x-\mu}_{\Sigma}^{2}-t$ is a quadratic
function in $(x,t)$. Hence, this level set admits an $1$-self-concordant
barrier
\[
\phi(x,t)=-\log\Par{t-\half\norm{x-\mu}_{\Sigma}^{2}}.
\]
Our earlier discussion immediately leads to the following result:
\begin{lem}
[Quadratic potential] \label{lem:Gaussian-potential}Consider a closed
convex $K=\Brace{(x,t):\half\norm{x-\mu}_{\Sigma}^{2}\leq t}$ with
$\mu\in\R^{n}$ and $\Sigma\in\pd$, and let $\phi(x)=-\log\Par{t-\half\norm{x-\mu}_{\Sigma}^{2}}$
and $g(x)=n\hess\phi(x)$.
\begin{itemize}
\item $\nu_{g},\,\onu_{g}=O(n)$.
\item SSC and SASC.
\item $D^{2}g(x,t)[h,h]\succeq0$ for any $(x,t)\in\inter(K)$ and $h\in\R^{n+1}$.
\end{itemize}
\end{lem}


\paragraph{Second-order cone ($f(x)=\protect\half\protect\norm{x-\mu}_{\Sigma}$).}

It is common that a potential includes a non-smooth term like $\norm{Ax-b}_{2}$
in many applications, and we can handle such potentials via our framework.
\cite{nesterov1994interior} shows in Lemma~4.3.3 that 
\[
\phi(x,t)=-\log(\underbrace{t^{2}-\norm x^{2}}_{=:F(x,t)})
\]
is a $2$-self-concordant for a level set $K=\Brace{(x,t)\in\R^{n}\times\R:\norm x_{2}\leq t}$
(here we may assume that $\mu=0$ and $\Sigma=I$ due to Lemma~\ref{lem:linear-trans}).
This level set is called a \emph{second-order cone} or Lorentz cone.

Applying Lemma~\ref{lem:4th-log} to $F(x,t)=t^{2}-\norm x^{2}$
with $\nu=2$, we immediately establish HSC of $\phi(x,t)=-\log(t^{2}-\norm x^{2})$.
Thus, $n\phi$ satisfies SLTSC and SASC by Lemma~\ref{lem:hsc-to-sltsc}
and Lemma~\ref{lem:hsc-to-sasc}, respectively.
\begin{lem}
[Second-order cone] \label{lem:soc} Consider a closed convex $K=\Brace{(x,t):\norm{x-\mu}_{\Sigma}\leq t}$
with $\mu\in\R^{n}$ and $\Sigma\in\pd$, and let $\phi(x)=-\log\Par{t^{2}-\norm{x-\mu}_{\Sigma}^{2}}$
and $g(x)=n\hess\phi(x)$.
\begin{itemize}
\item $\nu_{g},\,\onu_{g}=O(n)$.
\item SSC, SASC, and SLTSC.
\end{itemize}
\end{lem}


\subsection{PSD cone}

The function $\phi(X)=-\log\det X$ serves as an $n$-self-concordant
barrier for the PSD cone defined by $\{X\in\Rnn:X\succeq0\}$. While
achieving self-concordance does not require additional scaling, it
turns out that SSC requires a scaling of $\Theta(n)$. Notably, this
scaling is less than the trivial dimension-based scaling of $n(n+1)/2$.
Also, direct computation leads to $D^{4}\phi(X)[H,H]\succeq0$ (so
SLTSC).

As $\phi$ is HSC, scaling by $n(n+1)/2$ ensures SASC. However, we
can achieve ASC with a smaller scaling by $O(n)$ via the random matrix
theory.
\begin{lem}
[PSD cone] \label{lem:psd} For a closed convex $K=\{X\in\Rnn:X\succeq0\}$,
let $\phi(X)=-\log\det X$ and define $g(X)=n\hess\phi(X)$.
\begin{itemize}
\item $\nu=n^{2}$ (\cite{nesterov1994interior}) and $\onu=n^{2}$ (Lemma~\ref{lem:logdet-symm}).
\item SSC (Corollary~\ref{cor:logdet-ssc}).
\item $D^{2}g(X)[H,H]\succeq0$ for any $X\in\intk$ and $H\in\S^{n}$ (Lemma~\ref{lem:logdet-sltsc}).
\item ASC (Lemma~\ref{lem:logdet-asc}), and $\frac{n(n+1)}{2}\hess\phi(X)$
is SASC.
\end{itemize}
\end{lem}


\subsubsection{Formalism via matrix-vector transformations \label{subsec:formalism}}

In analyzing $-\log\det X$, we work in $\R^{d}=\R^{n(n+1)/2}$ and
$\S^{n}$ simultaneously in the sequel, moving back and forth between
them implicitly. We justify this identification as follows.

\paragraph{Measure on $\protect\S^{n}$.}

We can define and work with the Lebesgue measure on $\S^{n}$ by identifying
it with the Lebesgue measure on $\R^{d}$, where each component in
the Lebesgue measure on $\S^{n}$ corresponds to each entry in the
upper triangular part. Hence, with the Lebesgue measure $dX$ on $\S^{n}$
it is straightforward to define a probability distribution on $\S^{n}$
whose probability density function with respect to $dX$ is proportional
to $e^{-f(X)}$ for a function $f:\S^{n}\to\R$. For instance, the
uniform distribution over a region corresponds to $f(X)$ being constant
in the region and infinity outside of the region, and an exponential
distribution to $f(X)=\inner{C,X}=\tr(C^{\top}X)$ for $C\in\S^{n}$.

\paragraph{Directional derivatives.}

A function $\phi:\S^{n}\to\R$ induces its counterpart $\psi:\R^{d}\to\R$
defined by $\psi(x)=\phi(X)$ for $x:=\svec(X)$. For symmetric matrices
$\{H_{i}\}_{i\leq k}$, the $k^{th}$-directional derivative of $\phi$
in directions $H_{1},\dots,H_{k}$ is 
\[
D^{k}\phi(X)[H_{1},\cdots,H_{k}]\defeq\frac{d^{k}}{dt_{k}\cdots dt_{1}}\phi\Par{X+\sum_{i=1}^{k}t_{i}H_{i}}\bigg\vert_{t_{1},\dots,t_{k}=0}.
\]
For $h_{i}:=\svec(H_{i})$, it follows that $\phi\Par{X+\sum_{i=1}^{k}t_{i}H_{i}}=\psi(x+\sum_{i=1}^{k}t_{i}h_{i})$
and thus
\[
D^{k}\phi(X)[H_{1},\cdots,H_{k}]=D^{k}\psi(x)[h_{1},\cdots,h_{k}].
\]
With this identification in hand, since the notion of (symmetric or
strong) self-concordance is formulated in terms of directional derivatives,
we can deal with both representations without having to specify one
of them.

\paragraph{Important operators.}

We introduce three linear operators that enable us to make smooth
transitions between $\S^{n}$ and $\R^{d}$.
\begin{defn}
[\cite{magnus1980elimination}] \label{def:linearOperators} Let
$E_{ij}=e_{i}e_{j}^{\top}\in\Rnn$ be the matrix with a single $1$
in the $(i,j)$ position and zeros elsewhere.
\begin{itemize}
\item $M:\R^{d}\to\R^{n^{2}}$ is the linear operator that maps $\svec(\cdot)$
to $\vec(\cdot)$ (i.e., $M\circ\svec=\vec$). It can be written as
$M=\sum_{i\geq j}\vec(T_{ij})u_{ij}^{\top}$, where $T_{ij}\in\R^{n\times n}$
has all zero entries except for $1$ at $(i,j)$ and $(j,i)$ positions
(i.e., $T_{ij}=E_{ij}+E_{ji}$ if $i\neq j$ and $E_{ij}$ if $i=j$),
and $u_{ij}=\svec(E_{ij})$.
\item $N:\R^{n^{2}}\to\R^{n^{2}}$ is the linear operator that maps $\vec(A)$
to $\vec\Par{\half(A+A^{\top})}$ for a matrix $A\in\R^{n\times n}$.
\item $L:\R^{d}\to\R^{n^{2}}$ is the linear operator that maps $\vec(A)$
to $\svec(A)$ for a matrix $A\in\R^{n\times n}$. It can be written
as $L=\sum_{i\geq j}u_{ij}\vec(E_{ij})^{\top}$. 
\end{itemize}
\end{defn}

\begin{lem}
[\cite{magnus1980elimination}] \label{lem:MNL-properties} Let
$M,N,L$ be matrices in Definition~\ref{def:linearOperators}.
\begin{itemize}
\item (Lemma 2.1) $N=N^{\top}=N^{2}$ and $N(A\otimes A)=(A\otimes A)N$
for any $n\times n$ matrix $A$.
\item (Lemma 3.5)$MLN=N$.
\end{itemize}
\end{lem}


\subsubsection{Analysis of a self-concordant metric for the PSD cone \label{subsec:scBasicMetric}}

We first examine properties of the metric defined by the Hessian of
self-concordant barrier $\phi(X)=-\log\det X$ (see Theorem 4.3.3
in \cite{nesterov2003introductory} for self-concordance). In this
case, its Hessian and inverse have clean formulas. 
\begin{prop}
\label{prop:metricFormula} Let $\grad_{X}^{2}\phi(X)=\grad_{x}^{2}\Par{-\log\det\svec^{-1}(x)}\in\R^{d\times d}$
for $X\in\psd$. Its Hessian and inverse are
\begin{align*}
\hess\phi(X) & =M^{\top}(X^{-1}\otimes X^{-1})M=M^{\top}(X\otimes X)^{-1}M,\\
\Par{\hess\phi(X)}^{-1} & =M^{\dagger}(X\otimes X)M^{\dagger\top}=LN(X\otimes X)NL^{\top},
\end{align*}
where $M^{\dagger}=(M^{\top}M)^{-1}M^{\top}\in\R^{d\times n^{2}}$
is the Moore-Penrose inverse of $M\in\R^{n^{2}\times d}$.
\end{prop}

We defer the proof to Appendix~\ref{app:matrixCalculus}. We remark
that as an immediate corollary to this, the local norm of $h\in\R^{d}$
with metric $\hess\phi(X)$ becomes
\begin{align*}
\norm h_{X}^{2} & =h^{\top}M^{\top}(X^{-1}\otimes X^{-1})Mh=\svec(H)^{\top}M^{\top}(X^{-1}\otimes X^{-1})M\svec(H)\\
 & \underset{\text{(i)}}{=}\tr\Par{HX^{-1}HX^{-1}}=:\norm H_{X}^{2},
\end{align*}
where (i) follows from $\vec=M\circ\svec$ (Definition~\ref{def:linearOperators})
and $\tr\Par{DB^{\top}A^{\top}C}=\vec(A)^{\top}(B\otimes C)\vec(D)$
(Lemma~\ref{lem:Kronecker}). 

\paragraph{Symmetry.}
\begin{lem}
[$\onu$-symmetry] \label{lem:logdet-symm}For $X\in K=\mathbb{S}_{+}^{n}$,
the barrier $\phi(X)=-\log\det X$ is $n$-symmetric.
\end{lem}

\begin{proof}
For $X\in K$, pick any $Y\in K\cap(2X-K)$, and define a symmetric
matrix $H:=Y-X$. Since $Y\in K$ and $2X-Y\in K$, we have $X+H\in K$
and $X-H\in K$. Thus,
\[
-I\preceq X^{-\half}HX^{-\half}\preceq I,
\]
and the magnitude of each eigenvalue $\{\lda_{i}\}_{i=1}^{n}$ of
$X^{-\half}HX^{-\half}$ is bounded by $1$. Hence,
\begin{align*}
\norm H_{X}^{2} & =\tr(X^{-1}HX^{-1}H)=\norm{X^{-\half}HX^{-\half}}_{F}^{2}\leq\sum_{i=1}^{n}\lda_{i}^{2}\leq n.\qedhere
\end{align*}
\end{proof}

\paragraph{}

\paragraph{Convexity of log-determinant of Hessian and SSC.}

Next, the convexity of the log-determinant of $\hess\phi(X)$ can
be checked via properties of Kronecker products. See Section~\ref{proof:psd-convex-ssc}
for the proof.
\begin{prop}
[Convexity of log-determinant of Hessian] \label{prop:convex-logdet}
$\log\det\hess\phi(X)$ is convex in $X$.
\end{prop}

We move onto SSC of $n\phi(X)$.
\begin{lem}
\label{lem:logdet-scaling} For $\psi_{X}:=\sup_{H\in\S^{n}}$$\norm{\Par{\hess\phi(X)}^{-\half}D^{3}\phi(X)[H]\Par{\hess\phi(X)}^{-\half}}_{F}/\norm H_{X}$,
we have
\[
\sqrt{2(n+1)}\leq\psi_{X}\leq2\sqrt{n}.
\]
\end{lem}

We present the proof in Section~\ref{proof:psd-convex-ssc}. This
result informs us of the best possible scaling of $\phi(X)$ which
ensures SSC. Recall that if $g$ satisfies $\norm{g^{-\half}Dg[h]g^{-\half}}_{F}\leq2\alpha\norm h_{g}$
for $\alpha>0$, then $\alpha^{2}g$ is strongly self-concordant.
We remark that the scaling of $n$ is obviously better than the trivial
scaling of $d=\Theta(n^{2})$.
\begin{cor}
[Strong self-concordance] \label{cor:logdet-ssc} For $X\in\psd$,
a function $n\phi(X)=-n\log\det X$ is a strongly self-concordant
barrier for $\psd$. Moreover, the scaling factor of $n$ cannot be
further improved.
\end{cor}


\paragraph{Strongly lower trace self-concordance.}

SLTSC of $\phi$ can be easily checked by noting $g(X)[H,H]=\tr\Par{X^{-1}HX^{-1}H}$
and using the chain rule. See the details in Section~\ref{proof:psd-sltsc}.
\begin{lem}
[SLTSC] \label{lem:logdet-sltsc}$D^{2}g(X)[H,H]\succeq0$ for any
$X\in\intk$ and $H\in\S^{n}$.
\end{lem}


\paragraph{Average self-concordance.}

In establishing ASC, we find an interesting connection to a \emph{Gaussian
orthogonal ensemble} (GOE), one of the main objects studied in the
random matrix theory. We present the proofs of the following lemmas
and explain challenges when extending our arguments to SASC in Section~\ref{proof:psd-asc}.
\begin{lem}
\label{lem:conn-to-goe} For $d=n(n+1)/2$ and $\svec(H)\sim\ncal\Par{0,\frac{r^{2}}{d}g(X)^{-1}}$,
it holds that $\frac{\sqrt{dn}}{r}X^{-\half}HX^{-\half}$ is a Gaussian
orthogonal ensemble.
\end{lem}

\begin{lem}
[ASC] \label{lem:logdet-asc} $-n\log\det X$ is ASC.
\end{lem}


\subsection{Logarithm, exponential, entropy, and $\ell_{p}$-norm (power function)}

\paragraph{Logarithm in potentials.}

Consider $Q_{1}=\{(x,t)\in\R^{2}:-\log x\leq t,x>0\}$. Note that
$f(x)=-\log x$ is convex on $\{x>0\}$ and satisfies the condition
in Lemma~\ref{lem:tool-convex} with $\beta=2$ and $\gamma=6$.
Hence,
\[
F(x,t)=-\log(t+\log x)-36\log x
\]
is a highly $37$-self concordant barrier for $Q_{1}$. Therefore,
$2F$ is SSC and SLTSC with $\onu=O(1)$.
\begin{lem}
[Logarithm] Consider the direct product of level sets
\[
K=\prod_{i=1}^{n}\Brace{(x_{i},t_{i})\in\R^{2}:-\log x_{i}\leq t_{i},\,x_{i}>0},
\]
and let $\phi(x,t)=-\sum_{i=1}^{n}\Par{\log(t_{i}+\log x_{i})+36\log x_{i}}$
and $g(x,t)=2\hess\phi(x,t)$.
\begin{itemize}
\item $\nu,\,\onu=O(n)$.
\item SSC and SLTSC.
\item $n\hess\phi(x,t)$ is SASC.
\end{itemize}
\end{lem}

\begin{proof}
For $i\in[n]$, let $Q_{i}=\{(x_{i},t_{i})\in\R^{2}:-\log x_{i}\leq t_{i},\,y_{i}>0\}$
and $F_{i}(x_{i},t_{i})$ be the self-concordant barrier above. Note
that $2F_{i}$ is SSC and SLTSC. By Lemma~\ref{lem:ssc-direct} and
\ref{lem:sltsc-direct}, the Hessian of $F(x,t):=2\sum_{i=1}^{m}F_{i}(x_{i},t_{i})$
is SSC and SLTSC. The last item on SASC follows from Lemma~\ref{lem:hsc-to-sasc}.
\end{proof}

\paragraph{Exponent in potentials.}

Consider $Q_{2}=\{(x,t)\in\R^{2}:e^{x}\leq t\}=\{(x,t)\in\R^{2}:t>0,\,x\leq\log t\}$.
Note that $f(t)=\log t$ is concave and satisfies the condition in
Lemma~\ref{lem:tool-concave} with $\beta=2$ and $\gamma=6$. Hence,
\[
F(x,t)=-\log(\log t-x)-36\log t
\]
is a highly $37$-self concordant barrier for $Q_{2}$. Therefore,
$2F$ is SSC and SLTSC with $\onu=O(1)$.
\begin{lem}
[Exponential] Consider the direct product of level sets
\[
K=\prod_{i=1}^{n}\Brace{(x_{i},t_{i})\in\R^{2}:e^{x_{i}}\leq t_{i}},
\]
and let $\phi(x,t)=-\sum_{i=1}^{n}\Par{\log(\log t_{i}-x_{i})+36\log t_{i}}$
and $g(x,t)=2\hess\phi(x,t)$.
\begin{itemize}
\item $\nu,\,\onu=O(n)$.
\item SSC and SLTSC.
\item $n\hess\phi(x,t)$ is SASC.
\end{itemize}
\end{lem}

\begin{proof}
For $i\in[n]$, let $Q_{i}=\{(x_{i},t_{i})\in\R^{2}:e^{x_{i}}\leq t_{i}\}$
and $F_{i}(x_{i},t_{i})$ be the self-concordant barrier above. Note
that $2F_{i}$ is SSC and SLTSC. By Lemma~\ref{lem:ssc-direct} and
\ref{lem:sltsc-direct}, the Hessian of $F(x,t):=2\sum_{i=1}^{m}F_{i}(x_{i},t_{i})$
is SSC and SLTSC. The last item on SASC follows from Lemma~\ref{lem:hsc-to-sasc}.
\end{proof}

\paragraph{Entropy in potentials.}

Consider $Q_{3}=\{(x,t)\in\R^{2}:x\geq0,\,t\geq x\log x\}$. Note
that $f(x)=x\log x$ is convex on $\{x>0\}$ and satisfies the condition
in Lemma~\ref{lem:tool-convex} with $\beta=1$ and $\gamma=2$.
Hence,
\[
F(x,t)=-\log(t-x\log x)-36\log x
\]
is a highly $5$-self concordant barrier for $Q_{3}$. Therefore,
$2F$ is SSC and SLTSC with $\onu=O(1)$.
\begin{lem}
[Entropy] Consider the direct product of level sets
\[
K=\prod_{i=1}^{n}\Brace{(x_{i},t_{i})\in\R^{2}:x_{i}\geq0,\,t_{i}\geq x_{i}\log x_{i}},
\]
and let $\phi(x,t)=-\sum_{i=1}^{n}\Par{\log(t_{i}-x_{i}\log x_{i})+36\log x_{i}}$
and $g(x,t)=2\hess\phi(x,t)$.
\begin{itemize}
\item $\nu,\,\onu=O(n)$.
\item SSC and SLTSC.
\item $n\hess\phi(x,t)$ is SASC.
\end{itemize}
\end{lem}

\begin{proof}
For $i\in[n]$, let $Q_{i}=\{(x_{i},t_{i})\in\R^{2}:x_{i}\geq0,\,t_{i}\geq x_{i}\log x_{i}\}$
and $F_{i}(x_{i},t_{i})$ be the self-concordant barrier above. Note
that $2F_{i}$ is SSC and SLTSC. By Lemma~\ref{lem:ssc-direct} and
\ref{lem:sltsc-direct}, the Hessian of $F(x,t):=2\sum_{i=1}^{m}F_{i}(x_{i},t_{i})$
is SSC and SLTSC. The last item on SASC follows from Lemma~\ref{lem:hsc-to-sasc}.
\end{proof}

\paragraph{$\ell_{p}$-norm (power function).}

We start with the power functions. For $p\geq1$, consider $Q_{4}=\{(x,t)\in\R^{2}:t\geq\max(0,x)^{p}\}=\{(x,t)\in\R^{2}:t\geq0,\,x\leq t^{1/p}\}$.
Note that $f(t)=t^{1/p}$ is concave on $t>0$ and satisfies the condition
in Lemma~\ref{lem:tool-concave} with $\beta=2$ and $\gamma=6$.
Hence,
\[
F_{4}(x,t)=-\log\Par{t^{1/p}-x}-36\log t
\]
is a highly $37$-self-concordant barrier for $Q_{4}$. Similarly,
$F_{5}(t,x)=-\log\Par{t^{1/p}+x}-36\log t$ is a highly $37$-self
concordant barrier for the convex set $Q_{5}=\{(x,t)\in\R^{2}:t\geq\max(0,-x)^{p}\}$.
Since the convex set $Q_{6}=\{(x,t)\in\R^{2}:t\geq|x|^{p}\}$ is equal
to $Q_{4}\cap Q_{5}$, the sum of $F_{4}+F_{5}$, which is 
\[
F_{6}(x,t)=-\log\Par{t^{2/p}-x^{2}}-72\log t,
\]
is a highly $72$-self-concordant barrier for $Q_{6}$. Hence, $2F$
is SSC and SLTSC with $\onu=O(1)$.
\begin{lem}
[$\ell_p$-norm] Consider the direct product of level sets $K=\prod_{i=1}^{n}\Brace{(x_{i},t_{i})\in\R^{2}:\Abs{x_{i}}^{p}\leq t_{i}}$,
and let $\phi(x,t)=-\sum_{i=1}^{n}\Par{\log\Par{t_{i}^{2/p}-x_{i}^{2}}+72\log t_{i}}$
and $g(x,t)=2\hess\phi(x,t)$.
\begin{itemize}
\item $\nu,\,\onu=O(n)$.
\item SSC and SLTSC.
\item $n\hess\phi(x,t)$ is SASC.
\end{itemize}
\end{lem}

\begin{proof}
Consider a highly $72$-self-concordant barrier $F_{i}$ above for
$\{(x_{i},t_{i}):|x_{i}|^{p}\leq t_{i}\}$ for $i\in[n]$. Note that
$2F_{i}$ is SSC and SLTSC. By Lemma~\ref{lem:ssc-direct} and \ref{lem:sltsc-direct},
the Hessian of $F(x,t):=2\sum_{i=1}^{m}F_{i}(x_{i},t_{i})$ is SSC
and SLTSC. The last item on SASC follows from Lemma~\ref{lem:hsc-to-sasc}.
\end{proof}

