
\section{Mixing of $\protect\dws$ \label{sec:mixing-Dikin}}

We follow a standard conductance based argument (see e.g., \cite{lovasz1993random,vempala2005geometric}).
A lower bound on the conductance of a Markov chain provides an upper
bound on the mixing time of the Markov chain due to the following
result.
\begin{lem}
[\cite{lovasz1993random}] \label{lem:conductanceBound} Let $\pi_{T}$
be the distribution obtained after $T$ steps of a lazy reversible
Markov chain of conductance at least $\Phi$ with stationary distribution
$\pi$ and initial distribution $\pi_{0}$. 
\begin{itemize}
\item For the warmness parameter $\Lambda=\sup_{\text{\textup{measurable} }S}\frac{\pi_{0}(S)}{\pi(S)}$,
we have $\dtv\Par{\pi_{T},\pi}\leq\sqrt{\Lambda}\Par{1-\frac{\Phi^{2}}{2}}^{T}$.
\item For the $L^{2}$-distance $\Lambda=\norm{\pi_{0}/\pi}=\E_{x\sim\pi_{0}}\frac{d\pi_{0}}{d\pi}(x)$
and any $\veps>0$, we have $\dtv\Par{\pi_{T},\pi}\leq\veps+\sqrt{\frac{\Lambda}{\veps}}\Par{1-\frac{\Phi^{2}}{2}}^{T}$.
\end{itemize}
\end{lem}

A lower bound on the conductance follows from two ingredients: \textbf{(i)}
one-step coupling and \textbf{(ii)} isoperimetry. The first refers
to showing that the one-step distributions of the $\dw$ from two
nearby points have TV-distance bounded away from one. The second is
a purely geometry property about the expansion of the target distribution.
Combining these two leads to a lower bound on the conductance:
\begin{lem}
[\cite{kook2022condition}, Adapted from Proposition 9] \label{lem:conductance}
Let $\pi$ be the stationary distribution of a lazy reversible Markov
chain on $\mcal$ with a transition kernel $P_{x}$. Assume the isoperimetry
$\psi_{\mcal}$ under distance $d(x,y)=\norm{x-y}_{g(x)}$ and the
following one-step coupling: if $\norm{x-y}_{g(x)}\leq\Delta<1$ for
$x,y\in\mcal$, then $\dtv\Par{P_{x},P_{y}}\leq0.9$. Then the conductance
$\Phi$ of the Markov chain is bounded lower by $\Omega\Par{\psi_{\mcal}\Delta}$.
\end{lem}


\subsection{One-step coupling and isoperimetry}

Recall that a $\onu$-Dikin-amenable metric is $\onu$-symmetric,
SSC, LTSC, and ASC. \cite{laddha2020strong} was the first to attempt
characterizing essential properties of $g$ (or $\phi$) that determine
mixing times of $\dws$ for uniform sampling. Their framework necessitates
that $g$ satisfies $\onu$-symmetric, SSC, convexity of $\log\det g(x)$,
and $x\in\dcal_{g}^{r}(z)$ w.h.p. (where $z\sim\text{Unif}\Par{\dcal_{g}^{r}(x)}$). 

However, their framework encounters a challenge when further incorporating
the work of \cite{narayanan2016randomized}, which analyzes the $\dw$
for uniform sampling over a convex region given as the intersection
of various convex sets. The challenge arises from the difficulty of
verifying the convexity of $\log\det\Par{g_{1}+g_{2}}$ when $\log\det g_{i}$
is convex for each $i=1,2$.

To address this challenge and succinctly characterize essential characteristics
of a metric for one-step coupling, we relax the convexity of $\log\det$
to (S)LTSC and introduce the notion of ASC to account for the condition
``$x\in\dcal_{g}^{r}(z)$ w.h.p.''. We show that one-step coupling
lemma below, one of main proof ingredients in obtaining a mixing-time
guarantee of the $\dw$, can be established under Dikin-amenability
of a metric. Our characterization of a metric for achieving one-step
coupling is general and unifies previous work on $\dws$ (\cite{kannan2012random,narayanan2016randomized,chen2018fast,laddha2020strong}).

We now proceed to establish one-step coupling under the relative smoothness
in $\phi$.
\begin{lem}
[One-step coupling]\label{lem:one-step} For convex $K\subset\Rn$,
let $g:\intk\to\pd$ be SSC, ASC, LTSC, and $\phi:\intk\to\R$ be
its function counterpart. Suppose that the potential $f$ of the target
distribution $\pi$ is $\beta$-relatively smooth in $\phi$. Then
there exist constants $s_{1},s_{2}>0$ such that if $\norm{x-y}_{g(x)}\leq s_{1}\frac{r}{\sqrt{n}}$
with $r=s_{2}\min\Par{1,\frac{1}{\sqrt{\beta}}}$ for $x,y\in\intk$,
then $\dtv(P_{x},P_{y})\leq\frac{3}{4}+0.01$. 
\end{lem}

We provide a sketch of the proof (for the full proof, refer to Section~\ref{proof:onestep}).
A main challenge when going beyond uniform distributions is to establish
a lower bound for the ratio $\frac{\exp\Par{f(x)}}{\exp\Par{f(z)}}$
to ensure a sufficiently high acceptance probability. To tackle this
issue, we use the symmetry of the proposal distribution, claiming
$\frac{\exp\Par{f(x)}}{\exp\Par{f(z)}}\geq1$ at the expense of a
$1/2$ probability. However, this $1/2$ probability loss is incompatible
with previous proof techniques based on the triangle inequality: for
a transition kernel $T$ and proposing kernel $P$, 
\[
\dtv\Par{T_{x},T_{y}}\leq\dtv\Par{T_{x},P_{x}}+\dtv\Par{P_{x},P_{y}}+\dtv\Par{P_{y},T_{y}}.
\]
Such an approach bounds the second term by using Pinsker's inequality
and makes it arbitrarily small by taking $r=\Omega(1)$ small enough.
However, this would lead to a bound of $1/2+\veps$ for both $\dtv\Par{T_{x},P_{x}}$
and $\dtv\Par{T_{y},P_{y}}$, making RHS vacuous.

To get around this issue, we work with the exact formula of $\dtv(T_{x},T_{y})$
as follow: for rejection probabilities $r_{x},r_{y}$, and acceptance
probabilities $A_{x}(z)$ and a Gaussian proposal $g_{x}(z)$ from
$x$ to $z$ 
\begin{align*}
\dtv(T_{x},T_{y}) & =\half\Par{r_{x}+r_{y}}+\half\int\Abs{A_{x}(z)g_{x}(z)-A_{y}(z)g_{y}(z)}dz.
\end{align*}

To bound $r_{x}$ and $r_{y}$, we bound $\sqrt{\frac{\det g(z)}{\det g(x)}}$
lower by $1-\veps$ at the cost of $\veps$-probability through SSC,
LTSC, and ASC of $g$, following \cite{laddha2020strong} with convexity
of $\log\det$ replaced by LTSC. As mentioned earlier, we also deduce
$\frac{\exp\Par{f(x)}}{\exp\Par{f(z)}}\geq1$ through the symmetry
of Gaussian distributions at the cost of $1/2$ probability. Combining
these results, we obtain upper bounds of $\half+\veps$ for small
$\veps>0$ on $r_{x}$ and $r_{y}$.

Establishing a bound of $1/4+\veps$ on the second term is a more
involved task. This involves ensuring the closeness of acceptance
probabilities $A_{x}(z)$ and $A_{y}(z)$ as well as the probability
densities $g_{x}(z)$ and $g_{y}(z)$. Such a result can only be achieved
through sophisticated conditioning on high-probability events provided
by ASC, SSC, and symmetry of Gaussian proposals. To be precise, let
us define good events $G_{x}=\cap_{i=0,2,3}B_{x,i}^{c}$ and $G_{y}=\cap_{i=0,2,3}B_{y,i}^{c}$
such that $\P_{\ncal_{g}^{r}(x)}(G_{x}^{c})\leq3\veps$ and $\P_{\ncal_{g}^{r}(y)}(G_{y}^{c})\leq3\veps$,
where 
\begin{align*}
B_{x,0} & =\Brace{\norm{z-x}_{x}\geq cr}\ \text{with }c\geq1+\frac{2}{\sqrt{n}}\log\frac{1}{\veps}\quad\text{(Tail bound for Gaussian)}\\
B_{x,1} & =\Brace{-\inner{\grad f(x),z-x}\leq0}\quad\text{(Symmetry of Gaussian)}\\
B_{x,2} & =\Brace{\norm{z-x}_{z}^{2}-\norm{z-x}_{x}^{2}>2\veps\frac{r^{2}}{n}}\quad\text{(ASC of }g)\\
B_{x,3} & =\Brace{\inner{\grad\vphi(x),z-x}\leq-\frac{r}{\sqrt{n}}\norm{g(x)^{-\half}\grad\vphi(x)}_{2}\cdot2\log\frac{1}{\veps}}\quad\text{(SSC \& tail bound for Gaussian)}
\end{align*}
We further denote $G:=G_{x}\cup G_{y}$ and a partition of $G$ as
follows:
\[
G_{x\backslash y}:=G_{x}\backslash G_{y},\ G_{x,y}:=G_{x}\cap G_{y},\ G_{y\backslash x}:=G_{y}\backslash G_{x}.
\]
Then 
\begin{align*}
\half\int\Abs{A(x,z)g_{x}(z)-A(y,z)g_{y}(z)}dz & \leq3\veps+\underbrace{\half\int_{G_{x\backslash y}}Qdz}_{=:\acal}+\underbrace{\half\int_{G_{y\backslash x}}Qdz}_{=:\bcal}+\underbrace{\half\int_{G_{x,y}}Qdz}_{=:\ccal}
\end{align*}
We can bound $\acal$ and $\bcal$ by $O(\veps)$ by using Pinsker's
inequality and a well-known formula for the KL divergence between
two Gaussians. By conditioning on $B_{x,1}$ and using the triangle
inequality properly, we show
\[
\ccal\leq\frac{1}{4}+2\veps+\half\int_{G_{x}\cap G_{y}\cap B_{x,1}^{c}}\Abs{\min\Par{1,\frac{e^{f(x)}}{e^{f(z)}}\frac{g_{z}(x)}{g_{x}(z)}}-\min\Par{\frac{g_{y}(z)}{g_{x}(z)},\frac{e^{f(y)}}{e^{f(z)}}\frac{g_{z}(y)}{g_{x}(z)}}}g_{x}(z)dz.
\]
For $\textsf{V}:=\frac{g_{y}(z)}{g_{x}(z)}$ and $\text{\textsf{W}:=}\frac{e^{f(y)}}{e^{f(z)}}\frac{g_{z}(y)}{g_{x}(z)}$,
we can show that $\Abs{\log\textsf{V}}\lesssim\veps$ and $\log\textsf{W}\gtrsim-\veps$
conditioned on $G_{x}\cap G_{y}\cap B_{x,1}^{c}$ by using closeness
of SSC (Lemma~\ref{lem:strongSC-closeness}). Once these bound are
achieved, we have 
\[
\int\bigg|\min\bigg(1,\underbrace{\frac{e^{f(x)}}{e^{f(z)}}\frac{g_{z}(x)}{g_{x}(z)}}_{\geq\exp(-4\veps)}\bigg)-\min\bigg(\underbrace{\frac{g_{y}(z)}{g_{x}(z)}}_{\exp(-4\veps)\leq\text{\textsf{V}}\leq\exp(5\veps)},\underbrace{\frac{e^{f(y)}}{e^{f(z)}}\frac{g_{z}(y)}{g_{x}(z)}}_{\textsf{W}\geq\exp(-7\veps)}\bigg)\bigg|g_{x}(z)dz\leq\Par{e^{5\veps}-e^{-4\veps}}
\]
where $\frac{e^{f(x)}}{e^{f(z)}}\frac{g_{z}(x)}{g_{x}(z)}\geq\exp(-4\veps)$
is obtained when bounding $r_{x}$.
\begin{rem}
We further note that $\norm{x-y}_{x}$ can be replaced by $d_{\phi}(x,y)$,
since these two distance are within a constant factor of each other:
\begin{lem}
[\cite{nesterov2002riemannian}, Lemma 3.1] \label{lem:Riemann-Dikin-close}
Let $\phi:\intk\to\R$ be self-concordant, and $x,y\in\intk$ with
$\delta:=\norm{x-y}_{x}<1$. Then 
\[
\delta-\half\delta^{2}\leq d_{\phi}(x,y)\leq-\log(1-\delta).
\]
\end{lem}

\end{rem}

Next, we present two isoperimetric inequalities derived from distinct
sources: the first comes from the symmetry of a barrier, while the
second arises from strong convexity in a local metric.

\paragraph{Isoperimetry via barrier parameters.}

The first one states that isoperimetry of log-concave distributions
under distance $\norm{x-y}_{g(x)}$ or $d_{g}(x,y)$ (due to Lemma~\ref{lem:Riemann-Dikin-close})
is $\Omega\Par{1/\sqrt{\onu}}$. The following lemma is an extension
of \cite{laddha2020strong} from uniform distributions (over a convex
body) to general log-concave distributions. We defer the proof to
Section~\ref{proof:isoperimetry}.
\begin{lem}
\label{lem:symmetry-iso} For a log-concave distribution $\pi$, isoperimetry
$\psi_{\pi}$ under distance $\norm{x-y}_{g(x)}$ is $\Omega\Par{1/\sqrt{\onu}}$.
\end{lem}


\paragraph{Isoperimetry from relative strong convexity.}

Another kind of isoperimetry comes from relative strong-convexity
of the potential of a distribution. For a scalar $\alpha>0$, isoperimetry
of $e^{-\alpha\phi}$ on a Hessian manifold equipped with the metric
$\hess\phi$ is $\Omega\Par{\sqrt{\alpha}}$ if $D^{4}\phi(x)\Brack{h^{\otimes4}}\geq0$
for all $x\in K$ and $h\in\Rn$ (see Lemma~37 in \cite{lee2018convergence}).
Lemma~9 in \cite{gopi2023algorithmic} further generalizes this to
show that if $\phi$ is self-concordant and the potential $f$ is
$\alpha$-relatively strong convex, then its isoperimetry is $\Omega\Par{\sqrt{\alpha}}$.
We can adapt this lemma by restricting this to a convex set $K$ (not
necessarily bounded). See Section~\ref{proof:isoperimetry} for the
proof.
\begin{lem}
[\cite{gopi2023algorithmic}, Adapted from Lemma 9] \label{lem:sc-iso}
For a closed convex set $K\subset\Rn$, let a convex function $\phi:\intk\to\R$
be self-concordant on $K$, $f:\intk\to\R$ $\alpha$-relatively strongly
convex in $\phi$, and $\pi$ a log-concave distribution with $\frac{d\pi}{dx}\propto\exp\Par{-f(x)}\cdot\mathbf{1}_{K}(x)$.
For a partition $\{S_{1},S_{2},S_{3}\}$ of $K$ and the Riemannian
distance $d_{\phi}$ induced by the inner product $\inner{a,b}_{x}=a^{\top}\hess\phi(x)b$,
it holds that 
\[
\pi(S_{3})\gtrsim\sqrt{\alpha}d_{\phi}(S_{1},S_{2})\pi(S_{1})\pi(S_{2}).\qedhere
\]
\end{lem}


\subsection{Mixing time: Proof of Theorem~\ref{thm:Dikin}}

Putting all these components together, we obtain the following mixing-time
bounds for the $\dw$. 

\thmDikin*
\begin{proof}
Lemma~\ref{lem:conductance} ensures that $\Phi\gtrsim\frac{r}{\sqrt{n}}\psi$
due to the one-step coupling in Lemma~\ref{lem:one-step}. Lemma~\ref{lem:symmetry-iso}
leads to $\psi\gtrsim\frac{1}{\sqrt{\onu}}$, while Lemma~\ref{lem:sc-iso}
implies $\psi\gtrsim\sqrt{\alpha}$ due to $\hess\phi\asymp g$. Thus,
\[
\Phi\gtrsim\frac{1}{\sqrt{n}}\max\Par{\frac{1}{\sqrt{\onu}},\sqrt{\alpha}}\min\Par{1,\frac{1}{\sqrt{\beta}}},
\]
and using Lemma~\ref{lem:conductanceBound}, we can enforce $\dtv(\pi_{T},\pi)\leq\veps$
by solving $\sqrt{\Lambda}e^{-T\Phi^{2}/2}\leq\veps$ and $\frac{\veps}{2}+\sqrt{\frac{\Lambda}{\veps/2}}e^{-T\Phi^{2}/2}\leq\veps$
for $T$, which results in 
\[
T\lesssim n\max(1,\beta)\min\Par{\onu,\frac{1}{\alpha}}\log\frac{\Lambda}{\veps}.\qedhere
\]
\end{proof}

