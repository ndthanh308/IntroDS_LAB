\addtocontents{toc}{\setcounter{tocdepth}{2}} 

\section{Proofs \label{sec:proofs}}

We collect deferred proofs in this section.

\subsection{Mixing of $\textsf{Dikin walks}$ (Section \ref{sec:mixing-Dikin})}

\subsubsection{One-step coupling \label{proof:onestep}}

We start with the one-step coupling of the $\dw$ under the setting
$\alpha\hess\phi\preceq\hess f\preceq\beta\hess\phi$ on $\intk$.
Roughly speaking, if $\norm{x-y}_{x}\lesssim\frac{r}{\sqrt{n}}$ with
$r\lesssim\min\Par{1,\frac{1}{\sqrt{\beta}}}$, then $\dtv\Par{P_{x},P_{y}}\leq0.99$.
\begin{proof}
[Proof of Lemma~\ref{lem:one-step}] For $\frac{d\pi}{dx}\propto\exp\Par{-f(x)}\cdot\mathbf{1}_{K}(x)$,
let us denote
\begin{align*}
g_{x}(z) & :=\frac{d\ncal\Par{x,\frac{r^{2}}{n}g(x)^{-1}}}{dx}(z),\\
R(x,z) & =\frac{g_{z}(x)}{g_{x}(z)}\frac{\exp\Par{-f(z)}}{\exp\Par{-f(x)}},\\
A(x,z) & =\min\Par{1,\mathbf{1}\Par{z\in K}\cdot R(x,z)}.
\end{align*}
Then the transition kernel of the $\dw$ started at $x$ can be written
as 
\begin{align*}
P(x,dz) & =\underbrace{\Par{1-\int A(x,z')g_{x}(z')dz'}}_{=:r_{x}}\delta_{x}(dz)+A(x,z)g_{x}(z)dz,\text{ so}\\
P(x,S) & =r_{x}\dx(S)+\int_{S}A(x,z)g_{x}(z)dz\ \text{for any measurable set }S.
\end{align*}
Thus, for $x,y\in\intk$
\begin{align}
\dtv(P_{x},P_{y}) & =\underbrace{\half\Par{r_{x}+r_{y}}}_{\textsf{I}}+\underbrace{\half\int\Abs{A(x,z)g_{x}(z)-A(y,z)g_{y}(z)}dz}_{\textsf{II}}.\label{eq:tv-formula}
\end{align}

Let us define a bad event $B_{0}=\{z\in\Rn:\norm{z-x}_{x}\geq cr\}$
with $c$ determined later. Due to $\norm{z-x}_{x}=\frac{r}{\sqrt{n}}\norm h$
and concentration of the standard Gaussian in a thin shell of radius
$\sqrt{n}$ with annulus $O(1)$\footnote{A standard Gaussian $h\sim\ncal(0,I_{n})$ is concentrated around
a thin sell of radius $\sqrt{n}$ with annulus $O(1)$: For $\lda>0$,
\[
\P_{x\sim\ncal(0,I_{n})}\Par{\norm h_{2}\geq\sqrt{n}+\lda}\leq e^{-\lda/2}.
\]
}, we have $\P_{z\sim\ncal_{g}^{r}(x)}\Par{B_{0}}=\P\Par{\norm h\geq c\sqrt{n}}\leq\exp\Par{-\frac{(c-1)\sqrt{n}}{2}}$.
Hence, $\P\Par{B_{0}}\leq\veps$ for $c\geq1+\frac{2}{\sqrt{n}}\log\frac{1}{\veps}$. 

\paragraph{Rejection probability $r_{x}$ and $r_{y}$ (Term $\textsf{I}$).}

Note that
\[
r_{x}=1-\int A(x,z)g_{x}(z)dz=1-\int\min\bigg(1,\,\underbrace{\mathbf{1}\Par{z\in K}\frac{e^{-f(z)}}{e^{-f(x)}}}_{=:\textsf{A}}\underbrace{\frac{g_{z}(x)}{g_{x}(z)}}_{=:\textsf{B}}\bigg)g_{x}(z)dz.
\]

For $\textsf{A}$, we let $\hess\phi\preceq c_{\phi}g$ for some $c_{\phi}>0$
and use Taylor's expansion at $x\in K\cap B_{0}^{c}$ to show that
for some $x^{*}\in[x,z]$ 
\begin{align*}
f(x)-f(z) & =-\inner{\grad f(x),z-x}-\inner{\hess f(x^{*})(z-x),z-x}\\
 & \geq-\inner{\grad f(x),z-x}-c_{\phi}\beta\inner{g(x^{*})(z-x),z-x}\\
 & \underset{\text{(i)}}{\geq}-\inner{\grad f(x),z-x}-c_{\phi}\beta\norm{z-x}_{x}^{2}\cdot\Par{1+2\norm{x-z}_{x}}^{2}\\
 & \geq-\inner{\grad f(x),z-x}-c_{\phi}\beta c^{2}r^{2}\Par{1+2cr}^{2}\\
 & \underset{\text{(ii)}}{\geq}-\inner{\grad f(x),z-x}-\veps,
\end{align*}
where we used Lemma~\ref{lem:scCloseness} in (i) and took $r\leq r_{1}(\veps)$
in (ii), which is defined so that $\beta c_{\phi}c^{2}r^{2}(1+cr)^{2}\leq\veps$
for any $r\leq r_{1}(\veps)$. Due to symmetry of Gaussian distributions
and $\dcal_{g}^{1}(x)\subset K$, there exists a half-ellipsoid $G$
of $\dcal_{g}^{1}(x)$ in which $-\inner{\grad f(x),z-x}\geq0$. In
summary, on $z\in G$ we have $f(x)-f(z)\geq-\veps$.

Let us define the first bad event by $B_{1}:=G^{c}$. It is clear
that $\P_{\ncal_{g}^{r}(x)}(B_{1})\leq\half+\P_{\ncal_{g}^{r}(x)}\Par{\dcal_{g}^{1}(x)^{c}}$.
Thus, for $z=x+\frac{r}{\sqrt{n}}g(x)^{-1/2}h$ with $h\sim\ncal(0,I_{n})$
\[
\P_{z\sim\ncal_{g}^{r}(x)}\Par{\dcal_{g}^{1}(x)^{c}}=\P\Par{\norm{z-x}_{g(x)}\geq1}=\P_{h\sim\ncal(0,I_{n})}\Par{\norm h\geq\frac{\sqrt{n}}{r}}.
\]
Due to the concentration inequality for $\norm h$, we have $\P_{\ncal_{g}^{r}(x)}(B_{1})\leq\half+\veps$
for any $r\leq r_{2}(\veps):=\Par{1+\frac{2}{\sqrt{n}}\log\frac{1}{\veps}}^{-1}$. 

For $\textsf{B}$, note that for $\vphi(x):=\half\log\det g(x)$
\[
\log\text{\textsf{B}}=-\frac{n}{2r^{2}}\Par{\norm{z-x}_{z}^{2}-\norm{z-x}_{x}^{2}}+\Par{\vphi(z)-\vphi(x)}.
\]
ASC of $\phi$ can control the first term by taking $r_{3}(\veps)$
so that $\P_{\ncal_{g}^{r}(x)}\Par{\norm{z-x}_{z}^{2}-\norm{z-x}_{x}^{2}\leq2\veps\frac{r^{2}}{n}}\geq1-\veps$
for any $r\leq r_{3}(\veps)$. Let the complement of this event be
our second bad event $B_{2}$.

For the second term, Taylor's expansion of $\vphi$ at $x$ results
in 
\[
\vphi(z)-\vphi(x)=\underbrace{\inner{\grad\vphi(x),z-x}}_{=:\textsf{A'}}+\underbrace{\half\inner{\hess\vphi(x^{*})(z-x),z-x)}}_{=:\textsf{B'}}\text{ for some }x^{*}\in[x,z].
\]
For $\textsf{A}'$, we have $\inner{\grad\vphi(x),z-x}=\frac{r}{\sqrt{n}}\inner{g(x)^{-\half}\grad\vphi(x),h}$
for $h\sim\ncal(0,I_{n})$. Using a standard tail bound for a Gaussian
distribution, we have 
\[
\P_{z\sim\ncal_{g}^{r}(x)}\Par{\inner{\grad\vphi(x),z-x}\leq-\frac{r}{\sqrt{n}}\norm{g(x)^{-\half}\grad\vphi(x)}_{2}\cdot2\log\frac{1}{\veps}}\leq\veps.
\]
Call this event $B_{3}$. Conditioned on $B_{3}^{c}$, we have
\begin{align}
\inner{\grad\vphi(x),z-x} & \geq-\frac{r}{\sqrt{n}}\norm{g(x)^{-\half}\grad\vphi(x)}_{2}\cdot2\log\frac{1}{\veps}\label{eq:eq1}
\end{align}
We can further bound $\norm{g(x)^{-\half}\grad\vphi(x)}_{2}$ via
strong self-concordance of $g$ as follows:
\begin{align}
\norm{g(x)^{-\half}\grad\vphi(x)}_{2} & =\sup_{v:\norm v_{2}=1}\grad\vphi(x)^{\top}g(x)^{-\half}v\nonumber \\
 & \underset{\text{(i)}}{=}\sup_{v:\norm v_{2}=1}\tr\Par{g(x)^{-1}Dg(x)\Brack{g(x)^{-\half}v}}\nonumber \\
 & =\sup_{v:\norm v_{2}=1}\tr\Par{g(x)^{-\half}Dg(x)\Brack{g(x)^{-\half}v}g(x)^{-\half}}\nonumber \\
 & \underset{\text{(ii)}}{\leq}\sup_{v:\norm v_{2}=1}\sqrt{n}\norm{g(x)^{-\half}Dg(x)\Brack{g(x)^{-\half}v}g(x)^{-\half}}_{F}\nonumber \\
 & \underset{\text{(iii)}}{\leq}\sup_{v:\norm v_{2}=1}2\sqrt{n}\norm{g(x)^{-\half}v}_{g(x)}=2\sqrt{n},\label{eq:eq2}
\end{align}
where (i) follows from (\ref{eq:gradLogDet}), (ii) is due to $\tr(A)\leq\sqrt{n}\norm A_{F}$
for $A\in\Rnn$ in general, and (iii) is simply the definition of
strong self-concordance. Putting (\ref{eq:eq1}) and (\ref{eq:eq2})
together, and taking $r\leq r_{4}(\veps):=\frac{\veps}{4\log\frac{1}{\veps}}$,
we have
\[
\inner{\grad\vphi(x),z-x}\geq-4r\log\frac{1}{\veps}\geq-\veps.
\]

For $\textsf{B}'$, with $u=z-x$ for $z\in B_{0}^{c}$ we have 
\begin{align}
D^{2}\vphi(x^{*})[u,u] & \underset{\eqref{eq:hessLogDet}}{=}\tr\Par{g(x^{*})^{-1}D^{2}g(x^{*})[u,u]}-\norm{g(x^{*})^{-\half}Dg(x^{*})[u]g(x^{*})^{-\half}}_{F}^{2}\nonumber \\
 & \underset{\text{(i)}}{\geq}-\norm u_{g(x^{*})}^{2}-\norm{g(x^{*})^{-\half}Dg(x^{*})[u]g(x^{*})^{-\half}}_{F}^{2}\nonumber \\
 & \underset{\text{(ii)}}{\geq}-\norm u_{g(x^{*})}^{2}-4\norm u_{g(x^{*})}^{2}\nonumber \\
 & \underset{\text{(iii)}}{\geq}-\frac{5}{\Par{1-\norm{x-x^{*}}_{g(x)}}^{2}}\norm u_{g(x)}^{2}\label{eq:so-taylor-logdet}\\
 & \geq-5\Par{1+2cr}^{2}c^{2}r^{2}\nonumber 
\end{align}
where (i) follows from LTSC, (ii) is SSC, and (iii) follows from Lemma~\ref{lem:scCloseness}.
Hence, $D^{2}\vphi(x^{*})[z-x,z-x]\geq-\veps$ by taking $r\leq r_{5}(\veps)$
so that $5\Par{1+2cr_{5}}^{2}c^{2}r_{5}^{2}=\veps$. 

In summary, conditioned on $z\in G:=\bigcap_{i=0}^{3}B_{i}^{c}$ whose
measure is at least $\half-4\veps$ due to the union bound, we have
\begin{align}
\textsf{A}: & \frac{\exp(f(x))}{\exp(f(z))}\geq\exp(-\veps),\label{eq:fx-ratio}\\
\textsf{B}: & \frac{g_{z}(x)}{g_{x}(z)}\geq\exp(-3\veps),\label{eq:prop-ratio}\\
 & \vphi(z)-\vphi(x)\geq-2\veps\label{eq:vphi-z-x}
\end{align}
Combining these together,
\begin{align*}
r_{x} & =1-\int\min\bigg(1,\,\mathbf{1}\Par{z\in K}\frac{e^{-f(z)}}{e^{-f(x)}}\frac{g_{z}(x)}{g_{x}(z)}\bigg)g_{x}(z)dz\\
 & \leq1-\int_{G}\min\bigg(1,\,e^{-\veps}e^{-3\veps}\bigg)g_{x}(z)dz\\
 & \leq1-e^{-4\veps}\Par{\half-3\veps}\leq\half+5\veps.
\end{align*}
Since we can bound $r_{y}$ in the same way, it follows that $\textsf{I}\leq\half+5\veps$
(see (\ref{eq:tv-formula})).

\paragraph{Overlapping part (Term $\textsf{II}$).}

WLOG, let us assume $f(y)\geq f(x)$. Recall that we can take good
events $G_{x}=\cap_{i=0,2,3}B_{x,i}^{c}$ and $G_{y}=\cap_{i=0,2,3}B_{y,i}^{c}$
such that $\P_{\ncal_{g}^{r}(x)}(G_{x}^{c})\leq3\veps$ and $\P_{\ncal_{g}^{r}(y)}(G_{y}^{c})\leq3\veps$,
where 
\begin{align*}
B_{x,0} & =\Brace{\norm{z-x}_{x}\geq cr}\ \text{with }c\geq1+\frac{2}{\sqrt{n}}\log\frac{1}{\veps},\\
B_{x,2} & =\Brace{\norm{z-x}_{z}^{2}-\norm{z-x}_{x}^{2}>2\veps\frac{r^{2}}{n}},\\
B_{x,3} & =\Brace{\inner{\grad\vphi(x),z-x}\leq-\frac{r}{\sqrt{n}}\norm{g(x)^{-\half}\grad\vphi(x)}_{2}\cdot2\log\frac{1}{\veps}}.
\end{align*}

Let $G=G_{x}\cup G_{y}$, and define a partition of $G$ as follows:
\[
G_{x\backslash y}:=G_{x}\backslash G_{y},\ G_{x,y}:=G_{x}\cap G_{y},\ G_{y\backslash x}:=G_{y}\backslash G_{x}.
\]
Then the term $\textsf{II}$ can be decomposed as follows:
\begin{align*}
 & \half\int\underbrace{\Abs{A(x,z)g_{x}(z)-A(y,z)g_{y}(z)}}_{=Q}dz\\
= & \half\int_{K\backslash G}Qdz+\underbrace{\half\int_{G_{x\backslash y}}Qdz}_{=:\acal}+\underbrace{\half\int_{G_{y\backslash x}}Qdz}_{=:\bcal}+\underbrace{\half\int_{G_{x,y}}Qdz}_{=:\ccal}\\
\leq & \half\Par{\P_{\ncal_{g}^{r}(x)}(K\backslash G)+\P_{\ncal_{g}^{r}(y)}(K\backslash G)}+\acal+\bcal+\ccal\\
\leq & \half\Par{\P_{\ncal_{g}^{r}(x)}(G_{x}^{c})+\P_{\ncal_{g}^{r}(y)}(G_{y}^{c})}+\acal+\bcal+\ccal\\
\leq & 3\veps+\acal+\bcal+\ccal.
\end{align*}

The term $\acal$ can be further decomposed by
\begin{align*}
\half\int_{G_{x\backslash y}}Qdz & \leq\half\int_{G_{x\backslash y}}A(x,z)\Abs{g_{x}(z)-g_{y}(z)}dz+\half\int_{G_{x\backslash y}}\Abs{A(x,z)-A(y,z)}g_{y}(z)dz\\
 & \leq\half\int_{G_{x\backslash y}}\Abs{g_{x}(z)-g_{y}(z)}dz+\half\P_{\ncal_{g}^{r}(y)}\Par{G_{x\backslash y}}\\
 & \leq\half\int_{G_{x\backslash y}}\Abs{g_{x}(z)-g_{y}(z)}dz+\half\underbrace{\P_{\ncal_{g}^{r}(y)}\Par{G_{y}^{c}}}_{\leq3\veps},
\end{align*}
and in the same way the term $\bcal$ can be bounded by
\begin{align*}
\half\int_{G_{y\backslash x}}Qdz & \leq\half\int_{G_{y\backslash x}}\Abs{g_{x}(z)-g_{y}(z)}dz+\half\underbrace{\P_{\ncal_{g}^{r}(x)}\Par{G_{x}^{c}}}_{\leq3\veps}.
\end{align*}
Putting these together,
\begin{align*}
\textsf{II} & \leq3\veps+\acal+\bcal+\ccal\\
 & \leq6\veps+\half\int_{G_{x\backslash y}}\Abs{g_{x}(z)-g_{y}(z)}dz+\half\int_{G_{y\backslash x}}\Abs{g_{x}(z)-g_{y}(z)}dz+\ccal\\
 & \leq6\veps+\half\int\Abs{g_{x}(z)-g_{y}(z)}dz+\ccal=6\veps+\underbrace{\dtv\Par{\ncal_{g}^{r}(x),\ncal_{g}^{r}(y)}}_{\text{Use Pinsker's inequality}}+\ccal\\
 & \leq6\veps+\sqrt{2\kld\Par{\ncal_{g}^{r}(y)\|\ncal_{g}^{r}(x)}}+\ccal.
\end{align*}
It is known that
\[
\kld\Par{\ncal_{g}^{r}(y)\|\ncal_{g}^{r}(x)}=\half\Par{\tr\Par{g(y)^{-1}g(x)}-n+\log\det g(y)g(x)^{-1}+\norm{y-x}_{g(x)}^{2}\frac{n}{r^{2}}}.
\]
By Lemma~\ref{lem:scCloseness},
\begin{equation}
\Par{1-\norm{x-y}_{x}}^{2}I\preceq g(x)^{-\half}g(y)g(x)^{-\half}\preceq(1+2\norm{x-y}_{x})^{2}I.\label{eq:sc-eq}
\end{equation}
Let $\{\lda_{i}\}_{i\in[n]}$ be the eigenvalues of $g(x)^{-\half}g(y)g(x)^{-\half}$.
Then we can write
\begin{align*}
\tr\Par{g(y)^{-1}g(x)}-n+\log\det g(y)g(x)^{-1} & =\sum_{i=1}^{n}\Par{\lda_{i}-1+\log\frac{1}{\lda_{i}}}\\
 & \underset{\text{(i)}}{\leq}\sum_{i=1}^{n}\Par{\lda_{i}+\frac{1}{\lda_{i}}-2}=\sum_{i=1}^{n}\frac{(\lda_{i}-1)^{2}}{\lda_{i}}\\
 & \underset{\text{(ii)}}{\leq}\sum_{i=1}^{n}2\cdot8\norm{x-y}_{x}^{2}\leq16n\cdot\frac{s^{2}r^{2}}{n}\\
 & =16s^{2}r^{2}
\end{align*}
where we used $\log x\leq x-1$ for $x>0$ in (i), and $\half\leq\lda_{i}\leq1+8\norm{x-y}_{x}$
due to (\ref{eq:sc-eq}) in (ii). Thus, $2\kld\Par{\ncal_{g}^{r}(y)\|\ncal_{g}^{r}(x)}\leq s^{2}(16r^{2}+1)$
and 
\begin{align}
\dtv\Par{\ncal_{g}^{r}(x),\ncal_{g}^{r}(y)}\leq\sqrt{2\kld\Par{\ncal_{g}^{r}(y)\|\ncal_{g}^{r}(x)}} & \leq s\sqrt{16r^{2}+1}\leq\veps,\label{eq:TV-by-KL}
\end{align}
where we took $r\leq r_{6}(\veps)$ such that $\sqrt{16r_{6}^{2}+1}\leq2$
and $s\leq s_{1}(\veps):=\veps/2$.

We now move onto the term $\ccal$. Recall $B_{x,1}=\Brace{-\inner{\grad f(x),z-x}\leq0}$
with $\P\Par{B_{x,1}}\leq\half+O(\veps)$. For $Q=\Abs{A(x,z)g_{x}(z)-A(y,z)g_{y}(z)}$,
\begin{align*}
\ccal & =\half\int_{G_{x}\cap G_{y}}Qdz\\
 & =\half\int_{G_{x}\cap G_{y}\backslash B_{x,1}^{c}}Qdz+\half\int_{G_{x}\cap G_{y}\cap B_{x,1}^{c}}Qdz\\
 & \leq\half\int_{B_{x,1}}\underbrace{Q}_{\text{Use the traingle ineq.}}dz+\half\int_{G_{x}\cap G_{y}\cap B_{x,1}^{c}}Qdz\\
 & \leq\half\int_{B_{x,1}}\Abs{A(x,z)-A(y,z)}g_{x}(z)dz+\half\int_{B_{x,1}}A(y,z)\Abs{g_{x}(z)-g_{y}(z)}dz+\half\int_{G_{x}\cap G_{y}\cap B_{x,1}^{c}}Qdz\\
 & \le\half\underbrace{\P_{\ncal_{g}^{r}(x)}\Par{B_{x,1}}}_{\leq\half+\veps}+\underbrace{\dtv\Par{\ncal_{g}^{r}(x),\ncal_{g}^{r}(y)}}_{\leq\veps\ (\ref{eq:TV-by-KL})}+\half\int_{G_{x}\cap G_{y}\cap B_{x,1}^{c}}\Abs{A(x,z)g_{x}(z)-A(y,z)g_{y}(z)}dz\\
 & \leq\frac{1}{4}+2\veps+\half\int_{G_{x}\cap G_{y}\cap B_{x,1}^{c}}\Abs{\min\Par{1,\frac{e^{f(x)}}{e^{f(z)}}\frac{g_{z}(x)}{g_{x}(z)}}-\min\Par{\frac{g_{y}(z)}{g_{x}(z)},\frac{e^{f(y)}}{e^{f(z)}}\frac{g_{z}(y)}{g_{x}(z)}}}g_{x}(z)dz,
\end{align*}
where the last line follows from
\begin{align*}
A(x,z)g_{x}(z)-A(y,z)g_{y}(z) & =\min\Par{1,\frac{e^{f(x)}}{e^{f(z)}}\frac{g_{z}(x)}{g_{x}(z)}}g_{x}(z)-\min\Par{1,\frac{e^{f(y)}}{e^{f(z)}}\frac{g_{z}(y)}{g_{y}(z)}}g_{y}(z)\\
 & =\Par{\min\Par{1,\frac{e^{f(x)}}{e^{f(z)}}\frac{g_{z}(x)}{g_{x}(z)}}-\min\Par{\frac{g_{y}(z)}{g_{x}(z)},\frac{e^{f(y)}}{e^{f(z)}}\frac{g_{z}(y)}{g_{x}(z)}}}\cdot g_{x}(z).
\end{align*}

For $\textsf{V}:=\frac{g_{y}(z)}{g_{x}(z)}$ and $\text{\textsf{W}:=}\frac{e^{f(y)}}{e^{f(z)}}\frac{g_{z}(y)}{g_{x}(z)}$,
we show that $\Abs{\log\textsf{V}}\lesssim\veps$ and $\log\textsf{W}\gtrsim-\veps$
on $z\in G_{x}\cap G_{y}\cap B_{x,1}^{c}$. Recall $\frac{e^{f(x)}}{e^{f(z)}}\geq e^{-\veps}$
and $\frac{g_{z}(x)}{g_{x}(z)}\geq e^{-3\veps}$ due to (\ref{eq:fx-ratio})
and (\ref{eq:prop-ratio}). Observe that for $\vphi(\cdot)=\half\log\det g(\cdot)$
\begin{align}
\log\frac{g_{y}(z)}{g_{x}(z)} & =\underbrace{-\frac{n}{2r^{2}}\Par{\norm{z-y}_{y}^{2}-\norm{z-x}_{x}^{2}}}_{=:\textsf{L}}+\vphi(y)-\vphi(x)\nonumber \\
 & \underset{\text{(i)}}{=}\textsf{L}+\inner{\grad\vphi(x),y-x}+\underbrace{\half\inner{\hess\vphi(x^{*})(y-x),y-x}}_{\text{Use }(\ref{eq:so-taylor-logdet})}\quad\text{for some }x^{*}\in[x,y]\label{eq:bound-vphi}\\
 & \geq\textsf{L}-\norm{g(x)^{-1/2}\grad\vphi(x)}_{2}\norm{y-x}_{x}-5\underbrace{\Par{1+2\norm{x-y}_{x}}^{2}}_{\leq2}\norm{y-x}_{x}^{2}\nonumber \\
 & \geq\textsf{L}-2\sqrt{n}\cdot s\frac{r}{\sqrt{n}}-10s^{2}\frac{r^{2}}{n}\nonumber \\
 & \underset{\text{(ii)}}{\geq}\textsf{L}-\veps,\label{eq:logV-lower}
\end{align}
where we used Taylor's expansion of $\vphi$ at $x$ in (i), and (ii)
follows from $s\leq\frac{\veps}{10}$ and $r\leq r_{7}(\veps):=1$. 

For $\textsf{W}$, observe that 
\begin{align}
\log\frac{e^{f(y)}}{e^{f(z)}}\frac{g_{z}(y)}{g_{x}(z)} & \underset{\text{(i)}}{\geq}\log\frac{e^{f(x)}}{e^{f(z)}}\frac{g_{z}(y)}{g_{x}(z)}\geq\log\Par{e^{-\veps}\frac{g_{z}(y)}{g_{x}(z)}}\nonumber \\
 & =-\veps-\frac{n}{2r^{2}}\Par{\norm{z-y}_{z}^{2}-\norm{z-x}_{x}^{2}}+\vphi(z)-\vphi(x)\nonumber \\
 & \underset{\text{(ii)}}{\geq}-\veps-\frac{n}{2r^{2}}\Par{\norm{z-y}_{y}^{2}+2\veps\frac{r^{2}}{n}-\norm{z-x}_{x}^{2}}-2\veps\nonumber \\
 & =\textsf{L}-4\veps,\label{eq:logW-lower}
\end{align}
where (i) holds due to $f(y)\geq f(x)$, and in (ii) we used $\norm{z-y}_{z}^{2}-\norm{z-y}_{y}^{2}\leq2\veps\frac{r^{2}}{n}$
for $z\in B_{y,2}^{c}$, and $\vphi(z)-\vphi(x)\geq-2\veps$ for $z\in B_{x,3}^{c}$
(see (\ref{eq:vphi-z-x})).

Lastly, we show that $\Abs{\textsf{L}}$ is bounded by $O(\veps)$
with high probability. Due to affine invariance of the algorithm,
we may assume that $x=0$ and $g(x)=I$. Therefore, 
\begin{align*}
\norm{z-y}_{y}^{2}-\norm{z-x}_{x}^{2} & =\norm{z-y}_{y}^{2}-\norm z^{2}\\
 & =\norm z_{y}^{2}-\norm z^{2}-2\inner{z,y}_{y}+\underbrace{\norm y_{y}^{2}}_{\text{Lemma \ref{lem:scCloseness}}}\\
 & \leq z^{\top}\Par{g(y)-I}z-2\inner{z,y}_{y}+\norm y^{2}\underbrace{(1+2\norm y)^{2}}_{\leq2}.
\end{align*}
Using a tail bound for a Gaussian distribution, we have $\P_{z\sim\ncal(0,I)}\Par{\Abs{\inner{z,y}_{y}}\geq\frac{r}{\sqrt{n}}\norm{g(y)y}\cdot2\log\frac{1}{\veps}}\leq\veps$
and denote this event by $C_{1}$. In addition, self-concordance of
$g$ leads to $g(y)\preceq2I$, so $\norm{g(y)y}\leq2\norm y$.

To bound $z^{\top}(g(y)-I)z$, we first note that
\begin{align*}
\norm{g(y)-I}_{F} & \leq(1+2\norm y)^{2}\norm y\leq2s\frac{r}{\sqrt{n}}\quad\text{(Lemma \ref{lem:strongSC-closeness})},\\
\E z^{\top}\Par{g(y)-I}z & =\frac{r^{2}}{n}\tr\Par{g(y)-I}\leq\frac{r^{2}}{n}\sqrt{n}\norm{g(y)-I}_{F}\leq\frac{r^{2}}{n}\cdot2rs.
\end{align*}
By the Hanson-Wright inequality\footnote{\begin{lem*}
[Hanson-Wright; Adapted to Gaussian] Let $h\sim\ncal(0,\sigma^{2}I_{n})$
and $M\in\R^{n\times n}$. Then there exists universal constants $c,K>0$
such that for $t\geq0$
\[
\P\Par{\Abs{\norm h_{A}^{2}-\E\norm h_{A}^{2}}>t}\leq2\exp\Par{-c\min\Par{\frac{t^{2}}{K^{4}\sigma^{4}\norm M_{F}^{2}},\frac{t}{K^{2}\sigma^{2}\norm M_{2}}}}.
\]
\end{lem*}
}, for universal constants $K_{1},K_{2}>0$ and $t\geq0$ it holds
that
\[
\P_{\ncal_{g}^{r}(x)}\Par{\Abs{z^{\top}\Par{g(y)-I}z-\E z^{\top}\Par{g(y)-I}z}\geq t}\leq2\exp\Par{-K_{1}\min\Par{\frac{t^{2}}{K_{2}^{4}\frac{r^{4}}{n^{2}}\norm{g(y)-I}_{F}^{2}},\frac{t}{K_{2}^{2}\frac{r^{2}}{n}\norm{g(y)}_{2}}}}.
\]
By taking $r\leq r_{8}(\veps):=\frac{\sqrt{K_{1}}}{2K_{2}^{2}}$ and
$s\leq s_{2}(\veps):=\frac{\veps}{1+\sqrt{\log\frac{2}{\veps}}}$,
it follows that $\Abs{z^{\top}\Par{g(y)-I}z}\leq\frac{2r^{2}}{n}\veps$
with probability at least $1-\veps$. Let us denote the complement
of this event by $C_{2}$.

Conditioned on $z\in C_{1}^{c}\cap C_{2}^{c}$, we conclude that
\begin{align*}
\Abs{\norm{z-y}_{y}^{2}-\norm{z-x}_{x}^{2}} & \leq\Abs{z^{\top}\Par{g(y)-I}z}+2\Abs{\inner{z,y}_{y}}+2\norm y^{2}\\
 & \leq\frac{2r^{2}}{n}\veps+\frac{2r}{\sqrt{n}}\cdot2\underbrace{\norm y}_{\leq s\cdot\frac{r}{\sqrt{n}}}\cdot2\log\frac{1}{\veps}+2\norm y^{2}\\
 & \leq\frac{2r^{2}}{n}\cdot3\veps,
\end{align*}
where the last lines holds when $s\leq s_{3}(\veps):=\frac{\veps}{4\log\frac{1}{\veps}}$.
Hence, $\Abs{\textsf{L}}\leq3\veps$ on $C_{1}^{c}\cap C_{2}^{c}$.
Putting this into (\ref{eq:logV-lower}) and (\ref{eq:logW-lower})
\[
\frac{g_{y}(z)}{g_{x}(z)}\geq\exp(-4\veps)\quad\&\quad\frac{e^{f(y)}}{e^{f(z)}}\frac{g_{z}(y)}{g_{x}(z)}\geq\exp(-7\veps).
\]

We now show that $\log\textsf{V}\lesssim\veps$. Conditioned on $z\in C_{1}^{c}\cap C_{2}^{c}$,
\[
-\log\frac{g_{y}(z)}{g_{x}(z)}=-\textsf{L}+\vphi(x)-\vphi(y)\geq-3\veps+\vphi(x)-\vphi(y)\geq-5\veps,
\]
since $\vphi(x)-\vphi(y)$ can be lowered bounded by $-2\veps$ just
as in (\ref{eq:bound-vphi}). Hence, $\log\textsf{V}\leq5\veps$.

For $F:=G_{x}\cap G_{y}\cap B_{x,1}^{c}$ and $C:=C_{1}^{c}\cap C_{2}^{c}$,
we can bound
\begin{align*}
 & \int_{F}\Abs{A(x,z)g_{x}(z)-A(y,z)g_{y}(z)}dz\\
\leq & \int_{C^{c}}\Abs{A(x,z)g_{x}(z)-A(y,z)g_{y}(z)}dz+\int_{F\cap C}\Abs{A(x,z)g_{x}(z)-A(y,z)g_{y}(z)}dz\\
\leq & \underbrace{\P_{\ncal_{g}^{r}(x)}\Par{C^{c}}}_{\leq2\veps}+2\underbrace{\dtv\Par{\ncal_{g}^{r}(x),\,\ncal_{g}^{r}(y)}}_{\leq\veps}+\int_{F\cap C}\Abs{A(x,z)g_{x}(z)-A(y,z)g_{y}(z)}dz\\
\leq & 4\veps+\int_{F\cap C}\bigg|\min\bigg(1,\underbrace{\frac{e^{f(x)}}{e^{f(z)}}\frac{g_{z}(x)}{g_{x}(z)}}_{\geq\exp(-4\veps)}\bigg)-\min\bigg(\underbrace{\frac{g_{y}(z)}{g_{x}(z)}}_{\exp(-4\veps)\leq\,\cdot\,\leq\exp(5\veps)},\underbrace{\frac{e^{f(y)}}{e^{f(z)}}\frac{g_{z}(y)}{g_{x}(z)}}_{\geq\exp(-7\veps)}\bigg)\bigg|g_{x}(z)dz\\
\leq & 4\veps+\Par{e^{5\veps}-e^{-4\veps}}\P_{\ncal_{g}^{r}(x)}\Par{F\cap C}\leq4\veps+14\veps=18\veps
\end{align*}
Using this, we can bound $\ccal$ as follows:
\begin{align*}
\ccal & \leq\frac{1}{4}+2\veps+\half\int_{G_{x}\cap G_{y}\cap B_{x,1}^{c}}\Abs{\min\Par{1,\frac{e^{f(x)}}{e^{f(z)}}\frac{g_{z}(x)}{g_{x}(z)}}-\min\Par{\frac{g_{y}(z)}{g_{x}(z)},\frac{e^{f(y)}}{e^{f(z)}}\frac{g_{z}(y)}{g_{x}(z)}}}g_{x}(z)dz\\
 & \leq\frac{1}{4}+11\veps.
\end{align*}
Therefore, $\textsf{II}\leq6\veps+\veps+\ccal\leq\frac{1}{4}+18\veps.$
Combining this with $\textsf{I}\leq\half+5\veps$, we can conclude
that if $r\leq\min_{i}r_{i}(\veps)$ and $s\leq\min_{i}s_{i}(\veps)$,
then 
\[
\dtv(P_{x},P_{y})\leq\frac{3}{4}+16\veps.\qedhere
\]
\end{proof}

\subsubsection{Isoperimetric inequality \label{proof:isoperimetry}}

We now move onto the proof for the isoperimetric inequality coming
from the symmetry of a barrier. To begin with, we recall the \emph{cross-ratio}
\emph{distance} $d_{K}$ defined on a convex body $K$: for $x,y\in\intk$,
suppose that the chord passing through $x,y$ has endpoints $p$ and
$q$ in the boundary $\del K$ (so the order of points is $p,x,y,q$),
then the cross-ratio distance between $x$ and $y$ is defined by
\[
d_{K}(x,y)\defeq\frac{\norm{x-y}_{2}\norm{p-q}_{2}}{\norm{p-x}_{2}\norm{y-q}_{2}}.
\]
The first type of isoperimetric inequality says $\psi_{\pi}\gtrsim\frac{1}{\sqrt{\onu}}$.
\begin{proof}
[Proof of Lemma~\ref{lem:symmetry-iso}] For $r>0$ and a ball $B^{r}(0)$
of radius $r$ centered at the origin, we define a convex body $K_{r}:=K\cap B^{r}(0)$
and denote the restriction of $\pi$ to $K_{r}$ by $\pi_{r}$. Let
$\{S_{1},S_{2},S_{3}\}$ be a partition of $K$, and define $S_{i}^{r}:=S_{i}\cap K_{r}$
for $i\in[3]$. By Theorem~2.5 in \cite{lovasz2007geometry}, we
have 
\[
\pi_{r}(S_{3}^{r})\geq d_{K_{r}}(S_{1}^{r},S_{2}^{r})\pi_{r}(S_{1}^{r})\pi_{r}(S_{2}^{r}),
\]
where $d_{K_{r}}(S_{1}^{r},S_{2}^{r})=\inf_{x\in S_{1}^{r},y\in S_{2}^{r}}d_{K_{r}}(x,y)$.
By Lemma~2.3 in \cite{laddha2020strong} ($d_{K_{r}}(x,y)\geq\norm{x-y}_{x}/\sqrt{\onu}$
for any $x,y\in K_{r}$) and thus 
\[
\pi_{r}(S_{3}^{r})\geq\inf_{x\in S_{1}^{r},\,y\in S_{2}^{r}}\frac{\norm{x-y}_{x}}{\sqrt{\onu}}\pi_{r}(S_{1}^{r})\pi_{r}(S_{2}^{r})\geq\frac{1}{\sqrt{\onu}}\inf_{x\in S_{1},\,y\in S_{2}}\norm{x-y}_{x}\pi_{r}(S_{1}^{r})\pi_{r}(S_{2}^{r})
\]
As $r\to\infty$, the bounded convergence theorem implies $\pi_{r}(S_{i}^{r})\to\pi(S_{i})$
for $i\in[3]$, so 
\[
\frac{\pi(S_{3})/\inf_{x\in S_{1},\,y\in S_{2}}\norm{x-y}_{x}}{\pi(S_{1})\pi(S_{2})}\geq\frac{1}{\sqrt{\onu}}.\qedhere
\]
\end{proof}
%
We also provide the deferred proof for the isoperimetric inequality
$\psi_{\pi}\gtrsim\sqrt{\alpha}$ originating from $\alpha$-relatively
strong-convexity of the potential with respect to $\hess\phi$.
\begin{proof}
[Proof of Lemma~\ref{lem:sc-iso}] The proof essentially follows
\cite{gopi2023algorithmic}. Their first proof ingredient is a modified
localization lemma (Lemma~8 in \cite{gopi2023algorithmic}): Let
$f_{1},f_{2},f_{3},f_{4}$ be non-negative functions on $\Rn$ such
that $f_{1}$ and $f_{2}$ are upper semicontinuous, and $f_{3}$
and $f_{4}$ are lower semicontinuous, and $\phi:\Rn\to\R$ be convex.
Then the following are equivalent:
\begin{itemize}
\item For any density $\pi:\Rn\to\R$ which is $1$-relatively strongly
logconcave in $\phi$,
\[
\int f_{1}\pi\cdot\int f_{2}\pi\leq\int f_{3}\pi\cdot\int f_{4}\pi.
\]
\item $\int_{E}f_{1}e^{-\phi}\cdot\int_{E}f_{2}e^{-\phi}\leq\int_{E}f_{3}e^{-\phi}\cdot\int_{E}f_{4}e^{-\phi}$
for any $a,b\in\Rn$ and $\gamma\in\R$, where $\int_{E}h:=\int_{0}^{1}h((1-t)a+tb)e^{-\gamma t}dt$.
\end{itemize}
First of all, this can be generalized to an extended convex function
$f$ and $\phi$, whose values outside of $\intk$ are set to $\infty$.
Since the density $\pi(x)$ and needles $\exp\Par{\gamma t-\phi((1-t)a+tb)}$
for $\gamma\in\R$ and $a,b\in\Rn$ (induced by the extended $f$
and $\phi$) vanish outside of $\intk$, integrands in their formulations
become zero, and thus the integrals above remain the same.

Following Lemma~9 in \cite{gopi2023algorithmic}, the proof boils
down to the case of $\alpha=1$, and it suffices to show that there
exists a constant $C>0$ such that
\[
C\cdot d_{\phi}(S_{1},S_{2})\int_{S_{1}}e^{-f}\cdot\int_{S_{2}}e^{-f}\leq\int e^{-f}\int_{S_{3}}e^{-f}.
\]
We can replace $S_{i}\gets$ its closure $\bar{S_{i}}$ for $i\in[2]$,
which only increases LHS. Also, we can replace $S_{3}\gets$ an open
set $\intk\backslash\bar{S_{1}}\backslash\bar{S_{2}}$, which does
not change RHS since the boundary of a convex set is a null set (Theorem~1
in \cite{lang1986note}). By taking $f_{i}=\mathbf{1}_{S_{i}}$ for
$i\in[3]$ and $f_{4}=(Cd_{\phi}(S_{1},S_{2}))^{-1}$, we are enough
to show that for some $0\leq c<d\leq1$
\begin{align*}
 & C\cdot d_{\phi}(S_{1},S_{2})\int_{c}^{d}e^{\gamma t-\phi((1-t)a+tb)}\mathbf{1}_{S_{1}}((1-t)a+tb)dt\cdot\int_{c}^{d}e^{\gamma t-\phi((1-t)a+tb)}\mathbf{1}_{S_{2}}((1-t)a+tb)dt\\
\leq & \int_{c}^{d}e^{\gamma t-\phi((1-t)a+tb)}dt\cdot\int_{c}^{d}e^{\gamma t-\phi((1-t)a+tb)}\mathbf{1}_{S_{3}}((1-t)a+tb)dt,
\end{align*}
where $\phi((1-t)a+b)<\infty$ for $t\in(c,d)$. The rest of the proof
is similar to Lemma~9 in \cite{gopi2023algorithmic}.
\end{proof}

\subsection{Sampling IPM (Section \ref{sec:IPM-framework})}

\subsubsection{Well-definedness of sampling IPM \label{proof:IPM-welldefined}}
\begin{prop}
\label{prop:density-bounded} Let $p:\Rn\to\R$ be a log-concave density
with finite second moment. Then $p$ is bounded on $\Rn$.
\end{prop}

\begin{proof}
Let $\mu:=\E_{X\sim p}X$ and $\Sigma:=\E_{X\sim p}(X-\mu)(X-\mu)^{\top}$
be the mean and covariance of the density $p$. Then the pushforward
of $p$ via the map $T:x\mapsto\Sigma^{-1/2}(x-\mu)$ is an isotropic
log-concave distribution. When $q$ is the probability density of
the pushforward, we have $q(y)=q(T(x))=\frac{p(x)}{\Abs{\det T}}$.
Since $q(y)$ is bounded on $\Rn$ due to Theorem 5.14 (e) in \cite{lovasz2007geometry},
$p(x)$ is bounded as well.
\end{proof}
Now we can show that every measures appearing within the sampling
IPM is integrable.
\begin{proof}
[Proof of Proposition~\ref{prop:annealing-welldefined}] Recall
that $e^{-f}$ is integrable over $K$ and $\phi\geq0$. Hence, all
$\mu_{i}$'s in Phase 3 and 4 are integrable due to
\[
\int_{K}\exp\Par{-\Par{f(x)+\frac{\phi(x)}{\sigma_{i}^{2}}}}dx\leq\int_{K}\exp\Par{-f(x)}dx<\infty.
\]
In particular, $\exp\Par{-\Par{f(x)+\frac{\phi(x)}{\nu/n}}}$ is integrable
over $K$ with finite second moment. By Proposition~\ref{prop:density-bounded},
$f(x)+\frac{\phi(x)}{\nu/n}$ achieves a global minimum $d$ in $K$.
As $\sigma_{i}^{2}\leq\sigma_{i_{0}}^{2}=\nu/n$ in Phase 2, we have
\begin{align*}
 & \int_{K}\exp\Par{-\frac{\sigma_{i_{0}}^{2}f(x)+\phi(x)}{\sigma_{i_{0}}^{2}}}dx\\
 & =\int_{K}\exp\Par{-\frac{\sigma_{i_{0}}^{2}f(x)+\phi(x)-\min_{x}\Par{\sigma_{i_{0}}^{2}f(x)+\phi(x)}}{\sigma_{i_{0}}^{2}}-\frac{\min_{x}\Par{\sigma_{i_{0}}^{2}f(x)+\phi(x)}}{\sigma_{i_{0}}^{2}}}dx\\
 & \geq\int_{K}\exp\Par{-\frac{\bar{f}(x)+\phi(x)-\sigma_{i_{0}}^{2}d}{\sigma_{i}^{2}}-\frac{\sigma_{i_{0}}^{2}d}{\sigma_{i_{0}}^{2}}}dx\quad\Par{\because\min_{x}\Par{\sigma_{i_{0}}^{2}f(x)+\phi(x)}=\sigma_{i_{0}}^{2}d\ \&\ \bar{f}=\sigma_{i_{0}}^{2}f}\\
 & =\int_{K}\exp\Par{-\frac{\bar{f}(x)+\phi(x)}{\sigma_{i}^{2}}+\frac{\sigma_{i_{0}}^{2}d}{\sigma_{i}^{2}}-d}dx\\
 & =\int_{K}\exp\Par{-\frac{\bar{f}(x)+\phi(x)}{\sigma_{i}^{2}}}dx\cdot\exp\Par{\Par{\frac{\sigma_{i_{0}}^{2}}{\sigma_{i}^{2}}-1}d}
\end{align*}
Therefore,
\[
\int_{K}\exp\Par{-\frac{\bar{f}(x)+\phi(x)}{\sigma_{i}^{2}}}dx<\infty.\qedhere
\]
\end{proof}

\subsubsection{Closeness of distributions in sampling IPM \label{proof:IPM-closeness}}

We begin with a closeness result of $\ncal\Par{x^{*},\frac{\sigma_{0}^{2}}{1+\nu\beta n}g(x^{*})^{-1}}\cdot\mathbf{1}\Par{\dcal_{g}^{3\sigma_{0}\sqrt{n}}(x^{*})}$
and $\exp\Par{-\frac{\nu f/n+\phi}{\sigma_{0}^{2}}}$ in Phase 1.
\begin{proof}
[Proof of Lemma~\ref{lem:phase1}] Let $\gamma=9$, $r:=(\gamma\sigma^{2}n)^{1/2}<10^{-2}$,
$S:=\Brace{x\in K:\psi(x)\leq\psi(x^{*})+\frac{1}{4}r^{2}}$, and
$\psi:=\bar{f}+\phi$. For $p(x):=\exp\Par{-\frac{\psi(x)}{\sigma^{2}}}\cdot\mathbf{1}_{K}(x)\propto\frac{d\mu_{0}}{dx}$
and $x\in S$, we have $p(x)\geq e^{-\gamma n}p(x^{*}).$ Since $p$
is log-concave, Lemma~\ref{lem:mostMass-logconcave} leads to
\[
\int_{S^{c}}p(x)dx\leq\exp(-\gamma n/3)\int p(x)dx.
\]
Thus, $\int p=\int_{S}p+\int_{S^{c}}p\leq\int_{S}p+e^{-\gamma n/3}\int p$
and
\begin{equation}
\int p\leq(1-e^{-\gamma n/3})^{-1}\int_{S}p\leq(1+2e^{-\gamma n/3})\int_{S}p.\label{eq:intp-intSp}
\end{equation}
By Taylor's expansion of $\psi$ at $x^{*}$, there exists $\bar{x}\in[x^{*},x]$
such that
\begin{align}
\psi(x) & =\psi(x^{*})+\half(x-x^{*})^{\top}\hess\psi(\bar{x})(x-x^{*})\label{eq:psi-taylor}\\
 & \geq\psi(x^{*})+\half(x-x^{*})^{\top}\hess\phi(\bar{x})(x-x^{*}).\nonumber 
\end{align}
As $\psi(x)-\psi(x^{*})\leq\frac{1}{4}r^{2}$ for $x\in S$, it follows
that 
\begin{align*}
\norm{\bar{x}-x^{*}}_{g(\bar{x})}^{2} & \leq\norm{x-x^{*}}_{g(\bar{x})}^{2}\leq2(\psi(x)-\psi(x^{*}))\leq\half r^{2},
\end{align*}
and self-concordance of $\phi$ implies 
\begin{equation}
e^{-3r}\norm{x-x^{*}}_{g(x^{*})}^{2}\leq\norm{x-x^{*}}_{g(\bar{x})}^{2}\leq e^{3r}\norm{x-x^{*}}_{g(x^{*})}^{2}.\label{eq:closenss-initial}
\end{equation}
Thus it follows from these two that $\norm{x-x^{*}}_{g(x^{*})}^{2}\leq r^{2}$
and $S\subset D$.

Due to $\psi(x)-\psi(x^{*})=\half\norm{x-x^{*}}_{\hess\psi(\bar{x})}$
(see (\ref{eq:psi-taylor})) and $\Par{\frac{\nu\alpha}{n}+1}\hess\phi\preceq\hess\psi\preceq\Par{\frac{\nu\beta}{n}+1}\hess\phi$,
(\ref{eq:closenss-initial}) results in 
\begin{equation}
\half e^{-3r}\Par{\frac{\nu\alpha}{n}+1}\norm{x-x^{*}}_{g(x^{*})}^{2}\underset{(*)}{\leq}\psi(x)-\psi(x^{*})\underset{(\#)}{\leq}\half e^{3r}\Par{\frac{\nu\beta}{n}+1}\norm{x-x^{*}}_{g(x^{*})}^{2},\label{eq:approx-psigap}
\end{equation}
and thus for a constant $c:=\frac{\nu\beta}{n}+1$ and $h(x):=-\frac{1}{2\sigma^{2}}\norm{x-x^{*}}_{g(x^{*})}^{2}$,
\begin{align*}
 & \norm{\mu/\mu_{0}}=\E_{\mu}\frac{d\mu}{d\mu_{0}}=\frac{\int_{D}\exp\Par{-\frac{c}{\sigma^{2}}\norm{x-x^{*}}_{g(x^{*})}^{2}+\frac{\psi}{\sigma^{2}}}\int p}{\Par{\int_{D}\exp\Par{-c\frac{1}{2\sigma^{2}}\norm{x-x^{*}}_{g(x^{*})}^{2}}}^{2}}\\
 & \underset{\text{(\ref{eq:intp-intSp})}}{\leq}\frac{1+2e^{-\gamma n/3}}{\Par{\int_{D}\exp\Par{-c\frac{1}{2\sigma^{2}}\norm{x-x^{*}}_{g(x^{*})}^{2}}}^{2}}\int_{D}\exp\bigg(-\frac{c}{\sigma^{2}}\norm{x-x^{*}}_{g(x^{*})}^{2}+\underbrace{\frac{\psi}{\sigma^{2}}}_{\text{Use }(\#)\text{ in (\ref{eq:approx-psigap})}}\bigg)\int_{S}\underbrace{p}_{\text{Use }(*)\text{ in (\ref{eq:approx-psigap})}}\\
 & \lesssim\frac{\int_{D}\exp\Par{-\frac{1}{2\sigma^{2}}\norm{x-x^{*}}_{g(x^{*})}^{2}\Par{2c-e^{3r}\Par{\frac{\nu\beta}{n}+1}}}\int_{D}\exp\Par{-\frac{1}{2\sigma^{2}}\norm{x-x^{*}}_{g(x^{*})}^{2}e^{-3r}\Par{\frac{\nu\alpha}{n}+1}}}{\Par{\int_{D}\exp\Par{-c\frac{1}{2\sigma^{2}}\norm{x-x^{*}}_{g(x^{*})}^{2}}}^{2}}\\
 & \lesssim\underbrace{\frac{\int_{D}\exp\Par{h(x)\Par{2c-e^{3r}\Par{\frac{\nu\beta}{n}+1}}}\int_{D}\exp\Par{h(x)e^{3r}\Par{\frac{\nu\beta}{n}+1}}}{\Par{\int_{D}\exp\Par{ch(x)}}^{2}}}_{=:\text{\textsf{A}}}\underbrace{\frac{\int_{D}\exp\Par{h(x)e^{-3r}\Par{\frac{\nu\alpha}{n}+1}}}{\int_{D}\exp\Par{h(x)e^{3r}\Par{\frac{\nu\beta}{n}+1}}}}_{=:\text{\textsf{B}}}.
\end{align*}
For $\textsf{A}$, Lemma~\ref{lem:adam-logconcave} leads to 
\begin{align*}
\textsf{A} & \leq\Par{\frac{c^{2}}{\Par{2c-e^{3r}\Par{\frac{\nu\beta}{n}+1}}e^{3r}\Par{\frac{\nu\beta}{n}+1}}}^{n}=\Par{\frac{1}{(2-e^{3r})e^{3r}}}^{n}\\
 & =\Par{1+O(r^{2})}^{n}=O(1)
\end{align*}
For $\textsf{B}$, let $c_{1}=e^{-3r}\Par{\frac{\nu\alpha}{n}+1}$
and $c_{2}=e^{3r}\Par{\frac{\nu\beta}{n}+1}$. With the change of
variable $y=\frac{\sqrt{c_{i}}}{\sigma}g(x^{*})^{1/2}(x-x^{*})$ for
$i\in[2]$, it follows that for $r_{i}=r\sqrt{c_{i}}/\sigma=3\sqrt{c_{i}n}\geq3\sqrt{n}$
\begin{align*}
\textsf{B} & =\Par{\frac{c_{2}}{c_{1}}}^{n/2}\frac{\int_{B_{r_{1}}}\exp\Par{-\half\norm y^{2}}dy}{\int_{B_{r_{2}}}\exp\Par{-\half\norm y^{2}}dy}\underbrace{\leq}_{\text{due to }r_{1}\leq r_{2}}\Par{\frac{c_{2}}{c_{1}}}^{n/2}\\
 & \lesssim\Par{\frac{\nu\beta+n}{\nu\alpha+n}}^{n}\cdot(1+O(r))^{n}\lesssim\Par{\frac{\nu\beta+n}{\nu\alpha+n}}^{n}.\qedhere
\end{align*}
\end{proof}
Now we show closeness of two consecutive distributions in Phase 2,
during which we use the update $\sigma^{2}\gets\sigma^{2}\Par{1+\frac{1}{\sqrt{n}}}$.
\begin{proof}
[Proof of Lemma~\ref{lem:phase2}] For $\psi(x)=\bar{f}(x)+\phi(x)=\frac{\nu}{n}f(x)+\phi(x)$
on $x\in K$, observe that
\begin{align*}
\norm{\mu_{i}/\mu_{i+1}} & =\E_{\mu_{i}}\frac{d\mu_{i}}{d\mu_{i+1}}=\frac{\int_{K}\exp\Par{-2\frac{\psi(x)}{\sigma_{i}^{2}}+\frac{\psi(x)}{\sigma_{i+1}^{2}}}dx\int_{K}\exp\Par{-\frac{\psi(x)}{\sigma_{i+1}^{2}}}dx}{\Par{\int_{K}\exp\Par{-\frac{\psi(x)}{\sigma_{i}^{2}}}dx}^{2}}\\
 & =\frac{F\Par{\Par{\frac{2}{\sigma_{i}^{2}}-\frac{1}{\sigma_{i+1}^{2}}}^{-1}}F(\sigma_{i+1}^{2})}{F(\sigma_{i}^{2})^{2}}.
\end{align*}
By Lemma~\ref{lem:adam-logconcave}, the function defined by 
\[
a^{n}F\Par{\frac{\sigma^{2}}{a}}=a^{n}\int_{K}\exp\Par{-a\frac{\psi(x)}{\sigma^{2}}}dx
\]
is log-concave in $a$. Using the definition of log-concavity of $a^{n}F(\frac{\sigma^{2}}{a})$
with endpoints $\frac{2}{\sigma_{i}^{2}}-\frac{1}{\sigma_{i+1}^{2}}$
and $\frac{1}{\sigma_{i+1}^{2}}$, and the middle point $\frac{1}{\sigma_{i}^{2}}$,
we obtain
\begin{align*}
\norm{\mu_{i}/\mu_{i+1}} & =\frac{F\Par{\Par{\frac{2}{\sigma_{i}^{2}}-\frac{1}{\sigma_{i+1}^{2}}}^{-1}}F(\sigma_{i+1}^{2})}{F(\sigma_{i}^{2})^{2}}\leq\Par{\frac{\Par{\frac{1}{\sigma_{i}^{2}}}^{2}}{\Par{\frac{2}{\sigma_{i}^{2}}-\frac{1}{\sigma_{i+1}^{2}}}\frac{1}{\sigma_{i+1}^{2}}}}^{n}\\
 & =\Par{\frac{\Par{1+\frac{1}{\sqrt{n}}}^{2}}{1+\frac{2}{\sqrt{n}}}}^{n}\leq\Par{1+\frac{1}{n}}^{n}\leq e.\qedhere
\end{align*}
\end{proof}
We now establish closeness in Phase 3, during which we use the update
of $\sigma^{2}\gets\sigma^{2}\Par{1+\frac{\sigma}{\sqrt{\nu}}}$.
\begin{proof}
[Proof of Lemma~\ref{lem:phase34}] Let us write the update as $\sigma_{i+1}^{2}=\sigma_{i}^{2}\Par{1+r}$
where $r=\frac{\sigma_{i}}{\sqrt{\nu}}$. For $s:=\frac{r}{1+r}$
and $\sigma:=\sigma_{i}$, we have
\begin{align*}
\norm{\nu_{i}/\nu_{i+1}} & =\frac{F\Par{\Par{\frac{2}{\sigma_{i}^{2}}-\frac{1}{\sigma_{i+1}^{2}}}^{-1}}F(\sigma_{i+1}^{2})}{F(\sigma_{i}^{2})^{2}}=\frac{F\Par{\frac{\sigma^{2}}{1+s}}F\Par{\frac{\sigma^{2}}{1-s}}}{F(\sigma^{2})^{2}}.
\end{align*}
Let $g(t):=\log F\Par{\frac{\sigma^{2}}{t}}$ for $t>0$. Then
\begin{align}
\log\norm{\mu_{i}/\mu_{i+1}} & =g\Par{1+s}+g\Par{1-s}-2g(1)=\int_{0}^{s}\Par{g'(1+t)-g'(1-t)}dt\nonumber \\
 & =\int_{0}^{s}\int_{1-t}^{1+t}g''(q)dqdt\label{eq:L2-bound-phase3}
\end{align}
and
\begin{align*}
g''(q) & =\frac{d^{2}}{dq^{2}}\log\int_{K}\exp\Par{-f(x)-\frac{q\phi(x)}{\sigma^{2}}}dx=-\frac{1}{\sigma^{2}}\frac{d}{dq}\frac{\int_{K}\phi(x)\exp\Par{-f(x)-\frac{q\phi(x)}{\sigma^{2}}}dx}{\int_{K}\exp\Par{-f(x)-\frac{q\phi(x)}{\sigma^{2}}}dx}\\
 & =-\frac{1}{\sigma^{2}}\Par{-\frac{1}{\sigma^{2}}\frac{\int_{K}\phi(x)^{2}\exp\Par{-f(x)-\frac{q\phi(x)}{\sigma^{2}}}dx}{\int_{K}\exp\Par{-f(x)-\frac{q\phi(x)}{\sigma^{2}}}dx}+\frac{1}{\sigma^{2}}\frac{\Par{\int_{K}\phi(x)\cdot\exp\Par{-f(x)-\frac{q\phi(x)}{\sigma^{2}}}dx}^{2}}{\Par{\int_{K}\exp\Par{-f(x)-\frac{q\phi(x)}{\sigma^{2}}}dx}^{2}}}\\
 & =\frac{1}{\sigma^{4}}\Par{\E_{\nu_{q}}\phi(x)^{2}-\Par{\E_{\nu_{q}}\phi(x)}^{2}}=\frac{1}{\sigma^{4}}\var_{\nu_{q}}\phi(x),
\end{align*}
where $d\nu_{q}/dx\propto\exp\Par{-f(x)-\frac{q\phi(x)}{\sigma^{2}}}$.
By the Brascamp-Lieb inequality (\cite{brascamp1976extensions}),
with $V(x):=f(x)+\frac{q\phi(x)}{\sigma^{2}}$
\begin{align*}
\var_{\nu_{q}}\phi(x) & \leq\E_{x\sim\nu_{q}}\grad\phi(x)^{\top}\Par{\hess V(x)}^{-1}\grad\phi(x)\leq\frac{\sigma^{2}}{q}\E_{\nu_{q}}\grad\phi(x)^{\top}\Par{\hess\phi(x)}^{-1}\grad\phi(x)\\
 & \leq\frac{\sigma^{2}\nu}{q},
\end{align*}
and thus $g''(q)\leq\frac{\nu}{q\sigma^{2}}.$ Putting this bound
into (\ref{eq:L2-bound-phase3}), we acquire
\begin{align}
\log\norm{\mu_{i}/\mu_{i+1}} & \leq\frac{\nu}{\sigma^{2}}\int_{0}^{s}\int_{1-t}^{1+t}\frac{1}{q}dqdt=\frac{\nu}{\sigma^{2}}\int_{0}^{s}\Par{\log(1+t)-\log(1-t)}dt\nonumber \\
 & =\frac{\nu}{\sigma^{2}}\Par{(1+s)\log(1+s)+(1-s)\log(1-s)}\nonumber \\
 & \lesssim\frac{\nu}{\sigma^{2}}s^{2}.\label{eq:bound-ph3}
\end{align}
As $s=\frac{r}{1+r}$ and $r=\frac{\sigma}{\sqrt{\nu}}$, it follows
that $\mu_{i}$ is an $O(1)$-warm start for $\mu_{i+1}$.

For Phase 4, observe that for $\sigma^{2}=\nu$ at the end of Phase
3 
\begin{align*}
\norm{\mu/\pi} & =\frac{\int_{K}\exp\Par{-f-\frac{\phi}{\sigma^{2}/2}}\int_{K}\exp\Par{-f}}{\Par{\int_{K}\exp\Par{-f-\frac{\phi}{\sigma^{2}}}}^{2}}\\
 & \underset{\text{(i)}}{=}\lim_{r\to1}\frac{\int_{K}\exp\Par{-f-\frac{\phi}{\frac{\sigma^{2}}{1+r}}}\int_{K}\exp\Par{-f-\frac{\phi}{\frac{\sigma^{2}}{1-r}}}}{\Par{\int_{K}\exp\Par{-f-\frac{\phi}{\sigma^{2}}}}^{2}}=\lim_{r\to1}\frac{F\Par{\frac{\sigma^{2}}{1+r}}F\Par{\frac{\sigma^{2}}{1-r}}}{F\Par{\sigma^{2}}}\\
 & \underset{\text{(ii)}}{\leq}\lim_{r\to1}\exp\Par{O(1)\frac{\nu}{\sigma^{2}}\Par{(1+r)\log(1+r)+(1-r)\log(1-r)}}=\exp\Par{O(1)\frac{\nu}{\sigma^{2}}}\\
 & =\exp\Par{O(1)}.
\end{align*}
where (i) holds due to the monotone convergence theorem, and (ii)
follows from (\ref{eq:bound-ph3}). Therefore, $\mu$ serves as an
$O(1)$-warm start for $\pi$.
\end{proof}
\begin{rem}
[Coupling argument] \label{rem:divine-intervention} The total number
of measures involved in Algorithm~\ref{alg:IPM-sampling} is $m:=O(\sqrt{n})$.
Let $(X_{1},\dots,X_{m})$ be a sequence of samples provided by Algorithm~\ref{alg:IPM-sampling},
and $(\bar{X}_{1},\dots,\bar{X}_{m})$ be a sequence of samples where
each sample is drawn from the \emph{actual} target distributions $\{\mu_{\sigma^{2}}\}$.
Conditioned on events $X_{i}=\bar{X}_{i}$, Algorithm~\ref{alg:IPM-sampling}
ensures that there is a coupling such that $\P\Par{X_{i+1}=\bar{X}_{i+1}\mid X_{i}=\bar{X}_{i}}\geq1-\frac{\veps}{\sqrt{n}}$
due to $\veps/\sqrt{n}$ TV-distance of guarantee. Hence, combining
these couplings, we can obtain
\[
\P\Par{X_{i}=\bar{X_{i}}\,\forall i\in[m]}=\P(X_{1}=\bar{X}_{1})\cdot\prod_{i=2}^{m}\P(X_{i}=\bar{X}_{i}\mid X_{i-1}=\bar{X}_{i-1})\geq1-\veps.
\]
Thus, it leads to a coupling between $X_{m}$ and $\bar{X}_{m}$ such
that $\P(X_{m}=\bar{X}_{m})\geq1-\veps$, so $\text{law}(X_{m})$
is within $\veps$-TV distance to $\pi=\text{law}(\bar{X}_{m})$.
\end{rem}


\subsection{Self-concordance theory (Section \ref{sec:sc-theory-rules})}

\subsubsection{Basic properties: strong self-concordance \label{proof:ssc-basic}}

We show that $2(g_{1}+g_{2})$ is SSC if $g_{1}$ an $g_{2}$ are
SSC.
\begin{proof}
[Proof of Lemma~\ref{lem:ssc-sum}] For fixed $x\in K_{1}\cap K_{2}$
and $h\in\Rn$, let $Dg_{i}:=Dg_{i}(x)[h]$ for $i=1,2$. Note that
\begin{align*}
 & \norm{(g_{1}+g_{2})^{-\half}D(g_{1}+g_{2})(g_{1}+g_{2})^{-\half}}_{F}\\
 & \leq\norm{(g_{1}+g_{2})^{-\half}Dg_{1}(g_{1}+g_{2})^{-\half}}_{F}+\norm{(g_{1}+g_{2})^{-\half}Dg_{2}(g_{1}+g_{2})^{-\half}}_{F}\\
 & =\sqrt{\tr\Par{(g_{1}+g_{2})^{-1}Dg_{1}(g_{1}+g_{2})^{-1}Dg_{1}}}+\sqrt{\tr\Par{(g_{1}+g_{2})^{-1}Dg_{2}(g_{1}+g_{2})^{-1}Dg_{2}}}\\
 & =\sqrt{\tr\bigg(\bigg(\underbrace{I+g_{1}^{-\half}g_{2}g_{1}^{-\half}}_{=:E_{1}}\bigg)^{-1}\underbrace{g_{1}^{-\half}Dg_{1}g_{1}^{-\half}}_{=:T_{1}}\Par{I+g_{1}^{-\half}g_{2}g_{1}^{-\half}}^{-1}g_{1}^{-\half}Dg_{1}g_{1}^{-\half}\bigg)}\\
 & \qquad+\sqrt{\tr\bigg(\bigg(\underbrace{I+g_{2}^{-\half}g_{1}g_{2}^{-\half}}_{=:E_{2}}\bigg)^{-1}\underbrace{g_{2}^{-\half}Dg_{2}g_{2}^{-\half}}_{=:T_{2}}\Par{I+g_{2}^{-\half}g_{1}g_{2}^{-\half}}^{-1}g_{2}^{-\half}Dg_{2}g_{2}^{-\half}\bigg)}\\
 & =\sqrt{\tr\Par{E_{1}^{-1}T_{1}E_{1}^{-1}T_{1}}}+\sqrt{\tr\Par{E_{2}^{-1}T_{2}E_{2}^{-1}T_{2}}}\\
 & \leq\sqrt{\tr\Par{T_{1}E_{1}^{-2}T_{1}}}+\sqrt{\tr\Par{T_{2}E_{2}^{-2}T_{2}}},
\end{align*}
where we used the Cauchy-Schwarz inequality $\tr(A^{2})\leq\tr(A^{\top}A)$
in the last line. Since $I\preceq E_{i}$, it follows that $I\preceq E_{i}^{2}$
and $I\succeq E_{i}^{-2}\succ0$. Therefore, 
\begin{align*}
\sqrt{\tr\Par{T_{1}E_{1}^{-2}T_{1}}}+\sqrt{\tr\Par{T_{2}E_{2}^{-2}T_{2}}} & \leq\sqrt{\tr(T_{1}^{2})}+\sqrt{\tr(T_{2}^{2})}\\
 & =\norm{T_{1}}_{F}+\norm{T_{2}}_{F}\\
 & \leq2\norm h_{g_{1}(x)}+2\norm h_{g_{2}(x)}\\
 & \leq2\sqrt{2}\norm h_{(g_{1}+g_{2})(x)}.
\end{align*}
Putting these together, we conclude that 
\[
\norm{(g_{1}+g_{2})^{-\half}D(g_{1}+g_{2})(g_{1}+g_{2})^{-\half}}_{F}\leq2\sqrt{2}\norm h_{(g_{1}+g_{2})(x)},
\]
and thus $2(g_{1}+g_{2})$ is strongly self-concordant on $K_{1}\cap K_{2}$.
\end{proof}

\subsubsection{Basic properties: lower trace self-concordance \label{proof:ltsc-basic}}

We now show that if $g$ is HSC, then $ng$ is SLTSC.
\begin{proof}
[Proof of Lemma~\ref{lem:hsc-to-sltsc}] We first consider when
$\bar{g}$ is positive definite on $K$. Note that HSC of $\bar{g}$
leads to
\[
-\norm h_{\bar{g}}^{2}\bar{g}\lesssim D^{2}\bar{g}[h,h]\lesssim\norm h_{\bar{g}}^{2}\bar{g},
\]
and thus
\[
-\frac{1}{n}\norm h_{g}^{2}g\lesssim D^{2}g[h,h]\lesssim\norm h_{g}^{2}g.
\]
Hence
\[
-\frac{1}{n}\norm h_{g}^{2}(g'+g)^{-\half}g(g'+g)^{-\half}\lesssim(g'+g)^{-\half}D^{2}g[h,h](g'+g)^{-\half}\lesssim\frac{1}{n}\norm h_{g}^{2}(g'+g)^{-\half}g(g'+g)^{-\half},
\]
and 
\begin{align*}
\tr\Par{(g'+g)^{-1}D^{2}g[h,h]} & \gtrsim-\frac{1}{n}\norm h_{g}^{2}\tr\Par{(g'+g)^{-\half}g(g'+g)^{-\half}}=-\frac{1}{n}\norm h_{g}^{2}\tr\Par{g^{\half}(g'+g)^{-1}g^{\half}}\\
 & \geq-\frac{1}{n}\norm h_{g}^{2}\tr\Par{g^{\half}g^{-1}g^{\half}}=-\norm h_{g}^{2}.
\end{align*}

When $g$ is singular, we consider $\bar{g}_{\veps}=\bar{g}+\frac{\veps}{n}I$
that is invertible for $\veps>0$. Then $\bar{g}_{\veps}$ is HSC,
so for $g_{\veps}=n\bar{g}_{\veps}=g+\veps I$ 
\[
\tr\Par{(g'+g+\veps I)^{-1}D^{2}g[h,h]}\gtrsim-\norm h_{g+\veps I}^{2}.
\]
RHS is continuous in $\veps$, so it converges to $-\norm h_{g}^{2}$
as $\veps\to0$. For LHS, we note that
\[
(g'+g+\veps I)^{-1}=\frac{1}{\det\Par{g'+g+\veps I}}\text{adj}(g'+g+\veps I),
\]
and that $\det(\cdot)$ is continuous in $\veps$ and $\det(g'+g)\neq0$.
Also, $\text{adj}(g'+g+\veps I)$ is a polynomial in $\veps$, so
$(g'+g+\veps I)^{-1}$ converges to $(g'+g)^{-1}$ as $\veps\to0$.
\end{proof}

\subsubsection{Basic properties: average self-concordance \label{proof:asc-basic}}

To prove Lemma \ref{lem:hsc-to-sasc}, we first recall a concentration
bound.
\begin{lem}
[\cite{narayanan2016randomized}, Lemma 4] \label{lem:odd-order-concen}Let
$h$ be drawn from $\S^{n-1}$ uniformly at random. For any odd $k$,
$C^{k}$-smooth $F:\Rn\to\R$, and $\veps>0$,
\[
\P_{h}\Par{\Abs{D^{k}F(x)[h^{\otimes k}]}>k\veps\cdot\sup_{\norm v\leq1}D^{k}F(x)[v^{\otimes k}]}\leq\exp\Par{-\frac{n\veps^{2}}{2}}.
\]
\end{lem}

Using this, we can show that if $g$ is HSC, then $ng$ is SASC.
\begin{proof}
[Proof of Lemma~\ref{lem:hsc-to-sasc}] Let $g=n\hess\phi$ and
consider $g':\intk\to\psd$ such that $\bar{g}=g+g'$ is PD. For fixed
$w\in\Rn$, apply Taylor's expansion to $\vphi(z):=w^{\top}g(z)w$
at $z=x$. There exists $p_{w}\in[x,z]$ such that
\[
w^{\top}g(z)w=w^{\top}g(x)w+Dg(x)[z,w,w]+\half D^{2}g(p_{w})[z,z,w,w].
\]
Putting $z=w$ here,
\[
\Abs{\norm z_{g(z)}^{2}-\norm z_{g(x)}^{2}}\leq\Abs{D^{3}g(x)[z,z,z]}+\half\Abs{D^{2}g(p_{z})[z,z,z,z]}.
\]

Going forward, we can assume that $x=0$ and $\bar{g}(x)=I$ due to
affine invariance. We can represent a proposal $z$ as follows:
\[
z=x+\frac{r}{\sqrt{n}}h=\frac{r}{\sqrt{n}}h\ \text{for }h\sim\ncal(0,I_{n}).
\]
Using a standard tail bound on the standard Gaussian, we have $\P\Par{\norm h\geq\sqrt{n}\cdot2\log\frac{1}{\veps}}\leq\veps.$
Call this event $B_{1}$. In addition, Lemma~\ref{lem:odd-order-concen}
implies that 
\[
\P\Par{\Abs{D^{3}\phi(x)\Brack{\frac{h^{\otimes3}}{\norm h^{3}}}}\geq3\frac{\veps}{\sqrt{n}}\cdot\sup_{\norm v\leq1}D^{3}\phi(x)[v^{\otimes3}]}\leq\veps,
\]
and call this event $B_{2}$. We further note that on $B_{2}^{c}$
\begin{align*}
\Abs{D^{3}\phi(x)\Brack{\frac{h^{\otimes3}}{\norm h^{3}}}} & \leq3\frac{\veps}{\sqrt{n}}\cdot\sup_{\norm v\leq1}D^{3}\phi(x)[v^{\otimes3}]\leq\frac{6\veps}{\sqrt{n}}\sup_{\norm v\leq1}\norm v_{g(x)/n}^{3}\\
 & \leq\frac{6\veps}{n^{2}}\sup_{\norm v\leq1}\norm v_{g(x)}^{3}\underbrace{\leq}_{g(x)\preceq\bar{g}(x)=I}\frac{6\veps}{n^{2}}.
\end{align*}
Hence, conditioned on $z\in B_{1}^{c}\cap B_{2}^{c}$
\begin{align*}
\Abs{D^{3}g(x)[z^{\otimes3}]} & =\frac{r^{3}}{\sqrt{n}}D^{3}\phi(x)\Brack{h^{\otimes3}}\leq\frac{r^{3}}{\sqrt{n}}\cdot\frac{6\veps}{n^{2}}\cdot\norm h^{3}\leq\frac{r^{2}}{n}\cdot48r\veps\Par{\log\frac{1}{\veps}}^{3}.
\end{align*}
By taking $r_{1}(\veps)$ so that $48r_{1}\veps\Par{\log\frac{1}{\veps}}^{3}\leq\veps$,
we can ensure $\Abs{D^{3}g(x)[z^{\otimes3}]}\leq\veps\frac{r^{2}}{n}$
for any $r\leq r_{1}(\veps)$.

For $\Abs{D^{2}g(p_{z})[z,z,z,z]}$, HSC of $\phi$ leads to 
\begin{align*}
\half\Abs{D^{2}g(p_{z})[z^{\otimes4}]} & \leq3n\norm z_{\hess\phi(p_{z})}^{4}\\
 & \underset{\text{Lemma \ref{lem:scCloseness}}}{\leq}\frac{3}{n}\norm z_{\hess\phi(x)}^{4}\Par{1+2\norm z_{\hess\phi(x)}^{2}}^{2}=\frac{3}{n}\norm z_{g(x)}^{4}\Par{1+\frac{2}{n}\norm z_{g(x)}^{2}}^{2}\\
 & \underset{\because\ g\preceq I}{\leq}\frac{3}{n}\norm z^{4}\Par{1+\frac{2}{n}\norm z^{2}}^{2}=\frac{3}{n}\frac{r^{4}}{n^{2}}\norm h^{4}\Par{1+\frac{2r^{2}}{n^{2}}\norm h^{2}}^{2}\\
 & \leq\frac{r^{2}}{n}\cdot3r^{2}\Par{2\log\frac{1}{\veps}}^{4}\Par{1+2r^{2}\Par{2\log\frac{1}{\veps}}^{4}}^{2}.
\end{align*}
By taking $r_{2}(\veps)$ and $r_{3}(\veps)$ so that $\Par{1+2r_{2}^{2}\Par{2\log\frac{1}{\veps}}^{4}}^{2}\leq2$
and $2^{2}\cdot3r_{3}^{2}\Par{2\log\frac{1}{\veps}}^{4}\leq\veps$
respectively, we can ensure that on $B_{1}^{c}\cap B_{2}^{c}$
\[
\half\Abs{D^{2}g(p_{z})[z^{\otimes4}]}\leq\veps\frac{r^{2}}{n}\ \text{for any }r\leq\min r_{i}(\veps).
\]
Putting all these together, it follows that $\Abs{\norm z_{g(z)}^{2}-\norm z_{g(x)}^{2}}\leq2\veps\frac{r^{2}}{n}$
with probability at least $1-2\veps$. By replacing $2\veps\gets\veps$,
the claim follows.
\end{proof}

\subsubsection{Collapse and embedding: well-definedness \label{proof:collapse-embedding-welldefined}}

We start with well-definedness of the notions of collapse and embedding
(Definition~\ref{def:sc-along-subspace}).
\begin{proof}
[Proof of Proposition~\ref{prop:collapse-well-defined}] Let $k:=\dim(W)$,
and $U$ and $V$ be matrices in $\R^{n\times k}$, where the columns
of each matrix form an orthonormal basis of $W$. Let us denote by
$g_{1}:=U^{\top}gU$ and $g_{2}:=V^{\top}gV$ matrices represented
with respect to $U$ and $V$. Clearly, there exists an invertible
matrix $M\in\R^{k\times k}$ such that $U=VM$. Since $U$ and $V$
are full-column rank, if $g_{1}$ is PD, so is $g_{2}$.

Suppose $g$ is SSC along $W$. Then the definition invokes 
\begin{align*}
4\norm h_{g}^{2} & \geq\tr\Par{g_{1}^{-1}Dg_{1}[h]g_{1}^{-1}Dg_{1}}\\
 & =\tr\Par{\Par{U^{\top}gU}^{-1}U^{\top}Dg[h]U\Par{U^{\top}gU}^{-1}U^{\top}Dg[h]U}\\
 & =\tr\Par{\Par{M^{\top}V^{\top}gVM}^{-1}M^{\top}V^{\top}Dg[h]VM\Par{M^{\top}V^{\top}gVM}^{-1}M^{\top}V^{\top}Dg[h]VM}\\
 & =\tr\Par{M^{-1}\Par{V^{\top}gV}^{-1}(M^{\top})^{-1}M^{\top}V^{\top}Dg[h]VMM^{-1}\Par{V^{\top}gV}^{-1}(M^{\top})^{-1}M^{\top}V^{\top}Dg[h]VM}\\
 & =\tr\Par{\Par{V^{\top}gV}^{-1}V^{\top}Dg[h]V\Par{V^{\top}gV}^{-1}V^{\top}Dg[h]V}\\
 & =\norm{g_{2}^{-\half}Dg_{2}[h]g_{2}^{-\half}}_{F}^{2},
\end{align*}
and thus $g_{2}$ also satisfies the definition.
\end{proof}

\subsubsection{Collapse and embedding: affine transformation \label{proof:collap-affine}}

We begin with a barrier version.
\begin{proof}
[Proof of Lemma~\ref{lem:linear-trans}] For the first part, $\psi$
is a $\nu$-self-concordant barrier for $\bar{K}$ by Theorem~4.2.3
in \cite{nesterov2003introductory}, so $\dcal_{\bar{g}}^{1}(x)\subset\bar{K}\cap(2x-\bar{K})$
for $\bar{g}(x):=\hess\psi(x)$ by Lemma~\ref{lem:symmetricLeftpart}.
Now let $z\in\bar{K}\cap(2x-\bar{K})$. Then $Tz\in K$ and $T(2x-z)\in K$,
and the latter implies $2y-Tz\in K$. Thus $Tz\in K\cap(2y-K)$ and
$Tz\in\dcal_{g}^{\sqrt{\onu}}(y)$. Due to 
\begin{align*}
D^{2}\psi(x)[z-x,z-x] & =D^{2}\phi(y)[A(z-x),A(z-x)]=D^{2}\phi(y)[Tz-y,Tz-y]\leq\onu,
\end{align*}
it follows that $\psi$ is also $\onu$-symmetric. 

For the second part, observe that $D^{4}\psi(x)[v,v,h,h]=D^{4}\phi(y)[Av,Av,Ah,Ah]\geq0$
for any $v,h\in\Rn$. The third part can be proven similarly.
\end{proof}
Next is a matrix version.
\begin{proof}
[Proof of Lemma~\ref{lem:linear-trans-matrix}] Let $\phi$ be a
$\nu$-self-concordant function counterpart of $g$. Then $\psi(x):=\phi(Tx)$
defined on $\inter(\bar{K})$ is $\nu$-self-concordant by Lemma~\ref{lem:linear-trans}.
For any $h\in\Rn$ and $y:=Tx$, we have 
\[
D\bar{g}(x)[h]=A^{\top}Dg(y)[Ah]A\preceq2\norm{Ah}_{g(y)}A^{\top}g(y)A=2\norm h_{\bar{g}(x)}\bar{g}(x).
\]
Now consider a sequence $\{x_{n}\}\subset\bar{K}$ converging to a
boundary point $x\in\del\bar{K}$. If $Tx\notin\del K$, then $Tx\in\inter(K)$,
and the continuity of $T$ implies $x$ is also in $\inter(\bar{K})$.
Thus, $Tx\in\del K$ and $\psi(x_{n})=\phi(Tx_{n})\to\phi(Tx)=\infty$.
Lastly, $\hess\phi\asymp g$ leads to $\hess\psi=A^{\top}\hess\phi A\asymp A^{\top}gA=\bar{g}$,
and $\bar{g}$ is $\nu$-self-concordant for $\bar{K}$.

Regarding symmetry, since $\bar{g}$ is self-concordant, the Dikin
ellipsoid of radius $1$ centered at $x\in\bar{K}$ is contained in
$\bar{K}\cap(2x-\bar{K})$ by Lemma~\ref{lem:dikin-in-body}. For
$z\in\bar{K}\cap(2x-\bar{K})$, as $Tz\in K\cap(2Tx-K)$ holds, it
follows that 
\[
\onu\geq\norm{Tz-Tx}_{g(y)}^{2}=\norm{z-y}_{A^{\top}g(y)A}^{2}=\norm{z-y}_{\bar{g}(x)}^{2},
\]
and thus $\bar{g}$ is $\onu$-symmetric.

For the second item, we first show that $\bar{g}$ is collapsed onto
$W=\rowspace(A)$ (i.e., $\bar{g}=P_{W}\bar{g}P_{W}$ for the orthogonal
projection $P_{W}$ onto $W$). To see this, observe that
\begin{align*}
P_{W}\bar{g}P_{W} & =P_{W}A^{\top}gAP_{W}=A^{\top}(AA^{\top})^{\dagger}A\cdot A^{\top}gA\cdot A^{\top}(AA^{\top})^{\dagger}A,
\end{align*}
and due to $AA^{\top}(AA^{\top})^{\dagger}A=AA^{\top}(A^{\top})^{\dagger}A^{\dagger}A=AA^{\dagger}A=A$,
we have $P_{W}\bar{g}P_{W}=A^{\top}gA=\bar{g}$. 

We now show that $\bar{g}$ is indeed SSC along $W$. For $k:=\dim(W)$,
let us take $U\in\R^{n\times k}$ with the columns being an orthonormal
basis of $W$. It suffices to show that $g_{W}:=U^{\top}\bar{g}U=U^{\top}A^{\top}gAU=M^{\top}gM$
for $M:=AU\in\R^{m\times k}$ is SSC. First of all, we can check PDness
of $g_{W}$ as follows: Suppose $g_{W}v=0$ for some $v\in\R^{k}$.
Then $0=\norm v_{g_{W}}=\norm{g^{1/2}Mv}_{2}$ and $AUv=Mv=0$. Since
$Uv\in\rowspace(A)\cap\textsf{ker}(A)$ and $U$ is full-rank, we
have $v=0$. Next, for $h\in\R^{k}$ and $x\in\inter(\bar{K})$
\begin{align*}
 & \tr\Par{g_{W}(x)^{-1}Dg_{W}(x)[h]g_{W}(x)^{-1}Dg_{W}(x)[h]}\\
= & \tr\Par{\Par{g^{\half}M\Par{M^{\top}gM}^{-1}M^{\top}g^{\half}\cdot g^{-\half}Dg(Tx)[Ah]g^{-\half}}^{2}}\\
\underset{\text{(i)}}{\leq} & \tr\Par{\Par{g^{-\half}Dg(Tx)[Ah]g^{-\half}}^{2}}\leq\norm{g^{-\half}Dg(Tx)[Ah]g^{-\half}}_{F}^{2}\\
\leq & 4\norm{Ah}_{g(Tx)}^{2}=4\norm h_{\bar{g}(x)}^{2},
\end{align*}
where in (i) we used $P\Par{g^{\half}M}=g^{\half}M\Par{M^{\top}gM}^{-1}M^{\top}g^{\half}\preceq I$
is the orthogonal projection matrix. Thus, $\bar{g}$ is SSC along
$W=\rowspace(A)$.

The third item immediately follows from $D^{2}\bar{g}(x)[h,h]=A^{\top}D^{2}g(y)[Ah,Ah]A\succeq0$
for any $h\in\Rn$. 

For the fourth item, for any PSD matrix function $g'$ on $\bar{K}$
we have
\begin{align*}
\tr\Par{\Par{g'+\bar{g}}^{-1}D^{2}\bar{g}[h,h]} & =\tr\Par{\Par{g'+A^{\top}gA}^{-1}A^{\top}D^{2}g[Ah,Ah]A}\\
 & =\tr\Par{\Par{A^{-\top}g'A^{-1}+g}^{-1}D^{2}g[Ah,Ah]}\\
 & \geq-\norm{Ah}_{g}^{2}=-\norm h_{\bar{g}}^{2}.
\end{align*}

The last item is straightforward to check by the change of variable.
\end{proof}

\subsubsection{Collapse and embedding: lifting up SSC, SLTSC, and SASC \label{proof:lifting-ssc}}

In passing SSC to the embedding space, the Woodbury matrix identity
is a main technical tool used: for matrices with compatible sizes
\[
\Par{I+UV}^{-1}=I-U\Par{I+VU}^{-1}V.
\]
Using this, we show that if $g\in\pd$ is SSC, then $\bar{g}+\veps I_{m}$
is SSC.
\begin{proof}
[Proof of Lemma~\ref{lem:embedding-ssc}] Fix $\veps>0,y\in\inter(K')$,
and $h\in\R^{m}$. Take a projection matrix $P\in\{0,1\}^{n\times m}$
such that $PP^{\top}=I_{n}$ and $\bar{g}(y)=P^{\top}g(Py)P$ for
$x=Py\in\intk$. Also, by taking an orthonormal basis of $W$ with
$k:=\dim(W)$, there exists a matrix $U\in\R^{n\times k}$ with columns
being those basis such that $U^{\top}U=I_{k}$. Then $\bar{g}(y)=P^{\top}g(Py)P$
and $g(x)=Ug_{W}(x)U$, so for $M:=U^{\top}P\in\R^{k\times m}$
\[
\bar{g}(y)=P^{\top}Ug_{W}(Py)U^{\top}P=M^{\top}g_{W}(Py)M.
\]
 Note that $MM^{\top}=I_{k}$. Thus,
\begin{align*}
 & \norm{\Par{\bar{g}(y)+\veps I}^{-\half}D(\bar{g}+\veps I)(y)[h]\Par{\bar{g}(y)+\veps I}^{-\half}}_{F}^{2}\\
= & \tr\Par{\Par{\bar{g}(y)+\veps I}^{-1}D\bar{g}(y)[h]\Par{\bar{g}(y)+\veps I}^{-1}D\bar{g}(y)[h]}\\
= & \tr\Par{\Par{M^{\top}g_{W}(x)M+\veps I}^{-1}M^{\top}Dg_{W}(x)[Ph]M\Par{M^{\top}g_{W}(x)M+\veps I}^{-1}M^{\top}Dg_{W}(x)[Ph]M}\\
= & \tr\Par{\Par{M\Par{M^{\top}g_{W}(x)M+\veps I}^{-1}M^{\top}Dg_{W}(x)[Ph]}^{2}}.
\end{align*}
Let us show $M\Par{M^{\top}g_{W}(x)M+\veps I}^{-1}M^{\top}=\Par{g_{W}(x)+\veps I_{k}}^{-1}$.
Using the Woodbury matrix identity,
\begin{align*}
\Par{\veps I+M^{\top}g_{W}(x)M}^{-1} & =\frac{1}{\veps}I-\frac{1}{\veps^{2}}M^{\top}g_{W}^{\half}\bigg(I+\frac{1}{\veps}g_{W}^{\half}\underbrace{MM^{\top}}_{=I_{k}}g_{W}^{\half}\bigg)^{-1}g_{W}^{\half}M\\
 & =\frac{1}{\veps}I-\frac{1}{\veps^{2}}M^{\top}g_{W}^{\half}\Par{I+\frac{1}{\veps}g_{W}}^{-1}g_{W}^{\half}M,
\end{align*}
and thus conjugating both sides by $M$ results in 
\begin{align*}
M\Par{M^{\top}g_{W}(x)M+\veps I}^{-1}M^{\top} & =\frac{1}{\veps}I-\frac{1}{\veps}g_{W}^{\half}(g_{W}+\veps I)^{-1}g_{W}^{\half}=\frac{1}{\veps}I-\frac{1}{\veps}(g_{W}+\veps I)^{-1}g_{W}.
\end{align*}
Observe that 
\begin{align*}
\Par{g_{W}+\veps I}\cdot\Par{\frac{1}{\veps}I-\frac{1}{\veps}(g_{W}+\veps I)^{-1}g_{W}} & =\frac{1}{\veps}\Par{g_{W}+\veps I}-\frac{1}{\veps}g_{W}=I
\end{align*}
and thus the claim follows. Putting all these together, we have
\begin{align*}
 & \norm{\Par{\bar{g}(y)+\veps I}^{-\half}D(\bar{g}+\veps I)(y)[h]\Par{\bar{g}(y)+\veps I}^{-\half}}_{F}^{2}\\
= & \tr\Par{\Par{g_{W}(x)+\veps I_{k}}^{-1}Dg_{W}(x)[Ph]\Par{g_{W}(x)+\veps I_{k}}^{-1}Dg_{W}(x)[Ph]}\\
\leq & \tr\Par{g_{W}^{-1}Dg_{W}(x)[Ph]g_{W}^{-1}Dg_{W}(x)[Ph]}=\norm{g_{W}(x)^{-\half}Dg_{W}(x)[Ph]g_{W}(x)^{-\half}}_{F}^{2}\\
\leq & 4\norm{Ph}_{g(x)}^{2}=4\norm h_{\bar{g}(y)}^{2}.
\end{align*}
Therefore, $\bar{g}+\veps I$ is SSC on $K'$.
\end{proof}
In extending SLTSC and SASC, we need two technical lemmas: the inverse
of a block matrix and connection between P(S)Dness and Schur complements.
\begin{lem}
\label{lem:block-inverse} If $D$ and its Schur complement $A-BD^{-1}C$
are invertible, then 
\[
\left[\begin{array}{cc}
A & B\\
C & D
\end{array}\right]^{-1}=\left[\begin{array}{cc}
\Par{A-BD^{-1}C}^{-1} & *\\
* & *
\end{array}\right].
\]
\end{lem}

\begin{lem}
[Schur complement] \label{lem:schur} Let $A\in\R^{n\times n},B\in\R^{n\times m},C\in\R^{m\times m}$
and define a $(m+n)\times(m+n)$ matrix $M$ by 
\[
M=\left[\begin{array}{cc}
A & B\\
B^{\top} & D
\end{array}\right].
\]
Then $M\succ0$ if and only if $A\succ0$ and $C-BA^{-1}B^{\top}\succ0$
if and only $C\succ0$ and $A-B^{\top}C^{-1}B\succ0$.
\end{lem}

Using these, we show that if $g$ is SLTSC and SASC, then $\bar{g}$
is SLTSC and SASC.
\begin{proof}
[Proof of Lemma~\ref{lem:embedding-sltsc}] Take a full row-rank
projection matrix $P\in\R^{n\times m}$ such that $\bar{g}(y)=P^{\top}g(Py)P$,
where the rows of $P$ forms a subset of the canonical basis $\{e_{1},\dots,e_{m}\}$.
We can augment the rows of $P$ with the rest of the canonical basis
so that the augmented matrix $\bar{P}\in\R^{m\times m}$ is an orthonormal
matrix. Then we can represent $\bar{g}$ by 
\[
\bar{g}(y)=\bar{P}^{\top}\left[\begin{array}{cc}
g(Py) & 0\\
0 & 0
\end{array}\right]\bar{P}.
\]

Consider a PSD matrix function $g':\inter(K')\to\S_{+}^{m}$ such
that $g'+\bar{g}$ is PD on $K'$. Representing them in the block
form with $g_{A}\in\R^{n\times n},g_{B}\in\R^{n\times(m-n)},$ and
$g_{C}\in\R^{(m-n)\times(m-n)}$
\[
\bar{g}+g'=\bar{P}^{\top}\Par{\left[\begin{array}{cc}
g & 0\\
0 & 0
\end{array}\right]+\left[\begin{array}{cc}
g_{A} & g_{B}\\
g_{B}^{\top} & g_{C}
\end{array}\right]}\bar{P}=\bar{P}^{\top}\underbrace{\left[\begin{array}{cc}
g+g_{A} & g_{B}\\
g_{B}^{\top} & g_{C}
\end{array}\right]}_{=:g^{*}}\bar{P}.
\]
Since $g^{*}$ is PD, $g_{C}$ and its Schur complement $((g+g_{A})-g_{B}g_{C}^{-1}g_{B}^{\top})$
are PD. Thus by Lemma~\ref{lem:block-inverse}
\[
\left[\begin{array}{cc}
g+g_{A} & g_{B}\\
g_{B}^{\top} & g_{C}
\end{array}\right]^{-1}=\left[\begin{array}{cc}
\Par{g+g_{A}-g_{B}g_{C}^{-1}g_{B}^{\top}}^{-1} & *\\
* & *
\end{array}\right].
\]
Hence,
\begin{align*}
\tr\Par{(\bar{g}+g')^{-1}D^{2}\bar{g}(y)[h,h]} & =\tr\Par{\bar{P}^{\top}\left[\begin{array}{cc}
g+g_{A} & g_{B}\\
g_{B}^{\top} & g_{C}
\end{array}\right]^{-1}\bar{P}\bar{P}^{\top}\left[\begin{array}{cc}
D^{2}g(Py)[Ph,Ph] & 0\\
0 & 0
\end{array}\right]\bar{P}}\\
 & =\tr\Par{\left[\begin{array}{cc}
g+g_{A} & g_{B}\\
g_{B}^{\top} & g_{C}
\end{array}\right]^{-1}\left[\begin{array}{cc}
D^{2}g(Py)[Ph,Ph] & 0\\
0 & 0
\end{array}\right]}\\
 & =\tr\bigg(\bigg(g+\underbrace{g_{A}-g_{B}g_{C}^{-1}g_{B}^{\top}}_{\succeq0}\bigg)^{-1}D^{2}g(Py)[Ph,Ph]\bigg)\\
 & \geq-\norm{Ph}_{g(Py)}^{2}=-\norm h_{\bar{g}(y)}^{2},
\end{align*}
where in the last inequality we could use STLSC of $g$, since $g'\succeq0$
ensures that its Schur complement satisfies $g_{A}-g_{B}g_{C}^{-1}g_{B}^{\top}\succeq0$
by Lemma~\ref{lem:schur}.

For SASC, consider any PSD matrix function $g':\inter(K')\to\S_{+}^{m}$.
For $x=Py$ and $z_{x}=Pz_{y}\in\Rn$ with $z_{y}\sim\ncal\Par{y,\frac{r^{2}}{m}\Par{\bar{g}+g}(y)^{-1}}$,
we have
\[
\norm{z_{y}-y}_{\bar{g}(z_{y})}^{2}-\norm{z_{y}-y}_{\bar{g}(y)}^{2}=\norm{z_{x}-x}_{g(z_{x})}^{2}-\norm{z_{x}-x}_{g(x)}^{2}.
\]
Also, $z_{x}-x=P(z_{y}-y)$ is a Gaussian with zero mean and covariance
\begin{align*}
\frac{r^{2}}{m}P(\bar{g}+g')(y)^{-1}P^{\top} & =\frac{r^{2}}{m}P\bar{P}^{\top}\Par{\left[\begin{array}{cc}
g & 0\\
0 & 0
\end{array}\right]+\left[\begin{array}{cc}
g_{A} & g_{B}\\
g_{B}^{\top} & g_{C}
\end{array}\right]}^{-1}\bar{P}P^{\top}\\
 & =\frac{r^{2}}{m}\left[\begin{array}{cc}
I_{n} & 0_{n\times(m-n)}\end{array}\right]\Par{\left[\begin{array}{cc}
g & 0\\
0 & 0
\end{array}\right]+\left[\begin{array}{cc}
g_{A} & g_{B}\\
g_{B}^{\top} & g_{C}
\end{array}\right]}^{-1}\left[\begin{array}{c}
I_{n}\\
0_{n\times(m-n)}
\end{array}\right]\\
 & =\frac{r^{2}}{m}\bigg(g+g_{A}-g_{B}g_{C}^{-1}g_{B}^{\top}\bigg)^{-1}.
\end{align*}
Since $g_{A}-g_{B}g_{C}^{-1}g_{B}^{\top}\succeq0$ due to  $g'\succeq0$,
it holds that $g_{0}:=\frac{m-n}{n}g+\frac{m}{n}\Par{g_{A}-g_{B}g_{C}^{-1}g_{B}^{\top}}$
on $\intk$ is in $\psd$. Now, it suffices to check that the covariance
matrix above is equal to $\frac{r^{2}}{n}(g+g_{0})^{-1}$: 
\begin{align*}
\frac{n}{r^{2}}\Par{g+g_{0}} & =\frac{n}{r^{2}}\Par{g+\frac{m-n}{n}g+\frac{m}{n}\Par{g_{A}-g_{B}g_{C}^{-1}g_{B}^{\top}}}\\
 & =\frac{n}{r^{2}}\Par{\frac{m}{n}g+\frac{m}{n}\Par{g_{A}-g_{B}g_{C}^{-1}g_{B}^{\top}}}\\
 & =\frac{m}{r^{2}}\Par{g+g_{A}-g_{B}g_{C}^{-1}g_{B}^{\top}}.\qedhere
\end{align*}
\end{proof}

\subsubsection{Direct product: SSC and SLTSC \label{proof:direct-ssc-sltsc}}

We show that if $g_{i}\in\S_{++}^{n_{i}}$ is SSC, then $g=\sum n_{i}\bar{g}_{i}$
is SSC.
\begin{proof}
[Proof of Lemma~\ref{lem:ssc-direct}] Note that $n_{i}g_{i}$ is
SSC for $i=1,\dots,m$. For $x\in\prod E_{i}$ and $h=(h_{1},\dots,h_{m})\in\R^{l}$
with $h_{i}\in\R^{n_{i}}$, we have 
\begin{align*}
 & \norm{g(x)^{-1/2}Dg(x)[h]g(x)^{-1/2}}_{F}^{2}\\
 & =\norm{\left[\begin{array}{ccc}
g_{1}(x_{1})^{-1/2}Dg_{1}(x_{1})[h_{1}]g_{1}(x_{1})^{-1/2}\\
 & \ddots\\
 &  & g_{m}(x_{m})^{-1/2}Dg_{m}(x_{m})[h_{m}]g_{m}(x_{m})^{-1/2}
\end{array}\right]}_{F}^{2}\\
 & =\sum_{i}\norm{g_{i}(x_{i})^{-1/2}Dg_{i}(x_{i})[h_{i}]g_{i}(x_{i})^{-1/2}}_{F}^{2}\leq4\sum_{i}\norm{h_{i}}_{n_{i}g_{i}(x_{i})}^{2}\\
 & =4\sum_{i}\norm h_{\bar{g}_{i}(x)}^{2}=4\norm h_{g(x)}^{2}.
\end{align*}
Therefore, the direct product preserves strong self-concordance.
\end{proof}
If $g_{i}\in\S_{++}^{n_{i}}$ is HSC, then $g=\sum n_{i}\bar{g}_{i}$
is SLTSC.
\begin{proof}
[Proof of Lemma~\ref{lem:sltsc-direct}] We first look into SLTSC.
For $h=(h_{1},\dots,h_{m})$ and any PSD matrix function $g'$, we
have
\begin{align*}
\tr\Par{(g'+g)^{-1}D^{2}g[h,h]} & =\sum_{i}\tr\Par{(g'+(g-n_{i}\bar{g}_{i})+n_{i}\bar{g}_{i})^{-1}D^{2}(n_{i}\bar{g}_{i})[h,h]}\\
 & \gtrsim-\sum_{i}\norm h_{n_{i}\bar{g}_{i}}^{2}=-\sum_{i}\norm h_{g}^{2},
\end{align*}
where we used Lemma~\ref{lem:hsc-to-sltsc} in the inequality.
\end{proof}

\subsubsection{Inverse images under non-linear mappings \label{proof:inverse-non-linear}}
\begin{proof}
[Proof of Lemma~\ref{lem:compatible}] Since $\acal$ is $(R(G),\beta),\gamma)$-compatible
with $\Gamma$, the first two claims immediately follows from Proposition~5.1.7
in \cite{nesterov1994interior}. Let $x\in G^{+}$ and $h\in\Rn$.
Define the following notations:
\begin{align*}
u=D\acal(x)[h], & \quad v=D^{2}\acal(x)[h,h],\quad w=D^{3}\acal(x)[h,h,h],\quad z=D^{4}\acal(x)[h,h,h,h],\\
s=\sqrt{DF(y)[v]}, & \quad\rho=\sqrt{D^{2}\Pi(x)[h,h]},\quad r=\sqrt{D^{2}F(y)[u,u]}.
\end{align*}
From direct computations, we have 
\begin{align*}
D^{2}\Psi(x)[h,h] & =DF(y)[v]+D^{2}F(y)[u,u]+\delta^{2}D^{2}\Pi(x)[h,h]\\
 & =s^{2}+r^{2}+\delta^{2}\rho^{2},\\
D^{3}\Psi(x)[h,h,h] & =DF(y)[w]+3D^{2}F(y)[u,v]+D^{3}F(y)[u,u,u]+\delta^{2}D^{3}\Pi(x)[h,h,h],\\
D^{4}\Psi(x)[h,h,h,h] & =D^{2}F(y)[w,u]+DF(y)[z]+3D^{3}F(y)[u,u,v]+3D^{2}F(y)[v,v]\\
 & \qquad+3D^{2}F(y)[u,w]+D^{4}F(y)[u,u,u,u]+3D^{3}F(y)[u,u,v]+\delta^{2}D^{4}\Pi(x)[h,h,h,h]\\
 & =DF(y)[z]+3D^{2}F(y)[v,v]+4D^{2}F(y)[u,w]\\
 & \qquad+6D^{3}F(y)[u,u,v]+D^{4}F(y)[u,u,u,u]+\delta^{2}D^{4}\Pi(x)[h,h,h,h].
\end{align*}
Highly self-concordance of $F$ and $\Pi$ implies that 
\[
\Abs{D^{4}\Pi(x)[h,h,h,h]}\leq6\rho^{4}\quad\&\quad\Abs{D^{4}F(y)[u,u,u,u]}\leq6r^{4}.
\]
Since $\acal$ is $(K,\beta,\gamma)$-compatible and $K\subset R(G)$,
Lemma~\ref{lem:extension-compatibility}-(1) results in concavity
of $\acal$ with respect to $R(G)$, which means $-v\geq_{R(G)}0$.
Then Corollary~2.3.1 in \cite{nesterov1994interior} implies that
\[
\sqrt{D^{2}F(y)[v,v]}\leq DF(y)[v]=s^{2}.
\]
Hence, $\Abs{3D^{2}F(y)[v,v]}\leq3(DF(y)[v])^{2}=3s^{4}$, and self-concordance
of $F$ results in
\[
\Abs{6D^{3}F(y)[u,u,v]}\leq12r^{2}\sqrt{D^{2}F(y)[v,v]}\leq12r^{2}s^{2}.
\]
Since $\Brace{h:h^{\top}\Pi(x)h\leq1}$ is contained in $\Gamma\cap(2x-\Gamma)$,
compatibility of $\acal$ leads to 
\[
\beta D^{2}\acal(x)\Brack{\Par{\frac{h}{\norm h_{\Pi(x)}}}^{\otimes2}}\leq_{K}D^{3}\acal(x)\Brack{\Par{\frac{h}{\norm h_{\Pi(x)}}}^{\otimes3}}\leq_{K}-\beta D^{2}\acal(x)\Brack{\Par{\frac{h}{\norm h_{\Pi(x)}}}^{\otimes2}},
\]
and thus $\beta\rho v\leq_{K}w\leq_{K}-\beta\rho v$. As $K$ is a
ray, $D^{2}F(y)[w,w]\leq\beta^{2}\rho^{2}D^{2}F(y)[v,v]\leq\beta^{2}\rho^{2}s^{4}$.
Thus,
\[
\Abs{4D^{2}F(y)[u,w]}\leq4\sqrt{D^{2}F(y)[u,u]}\sqrt{D^{2}F(y)[w,w]}\leq4r\beta\rho s^{2}.
\]
Lastly, since $\gamma v\rho^{2}\leq_{K}z\leq_{K}-\gamma v\rho^{2}$
and $K$ is a ray, we have 
\[
\Abs{DF(y)[z]}\leq3\gamma\rho^{2}\Abs{DF(y)[v]}=3\gamma\rho^{2}s^{2}.
\]
Putting these together
\begin{align*}
\Abs{D^{4}\Psi(x)[h,h,h,h]} & \leq3\gamma\rho^{2}s^{2}+4r\beta\rho s^{2}+12r^{2}s^{2}+3s^{4}+6\delta^{2}\rho^{4}+6r^{4}\\
 & \leq6\Par{\delta^{2}\rho^{4}+r^{4}+s^{4}+r^{2}s^{2}+\delta\rho^{2}s^{2}+\delta r\rho s^{2}}\\
 & \leq6\Par{(\delta\rho)^{4}+r^{4}+s^{4}+r^{2}s^{2}+(\delta\rho)^{2}s^{2}+r^{2}s^{2}+(\delta\rho)^{2}s^{2}}\\
 & \leq6\Par{(\delta\rho)^{2}+r^{2}+s^{2}}^{2}\\
 & =6\Par{D^{2}\Psi(x)[h,h]}^{2}.\qedhere
\end{align*}
\end{proof}

\subsection{Main constraints and epigraphs (Section \ref{sec:handbook-barrier})}

\subsubsection{Linear constraints: logarithmic barriers \label{proof:linear-log-barrier}}

Here we collect details used in the paper that involve calculus of
the logarithmic barrier, $\phi_{\log}(X):=-\sum_{i=1}^{m}\log\Par{\inner{A_{i},X}-b_{i}}$
for $X\in\psd$ and $A_{i}\in\R^{n\times n}$. Recall the metric $g$
defined by the Hessian of $\phi_{\log}$ is given by
\begin{align*}
g(X) & =M^{\top}\left[\begin{array}{ccc}
\vec{(}A_{1}) & \cdots & \vec{(}A_{m})\end{array}\right]S_{X}^{-2}\left[\begin{array}{c}
\vec{(}A_{1})^{\top}\\
\vdots\\
\vec{(}A_{m})^{\top}
\end{array}\right]M\\
 & =M^{\top}A^{\top}S_{X}^{-2}AM,
\end{align*}
where $S_{X}=\Diag\Par{\inner{A_{i},X}-b_{i}}\in\R^{m\times m}$ and
$A^{\top}=\left[\begin{array}{ccc}
\vec{(}A_{1}) & \cdots & \vec{(}A_{m})\end{array}\right]\in\R^{n^{2}\times m}$. As we work on $\S^{n}$ and $\R^{d}$ simultaneously in Section~\ref{subsec:PSD-cone-sampling},
we consider its vector version (i.e., $g(x)=A^{\top}S_{x}^{-2}A$
for $x\in\R^{d}$) for simplicity and then translate it into one in
our setting. We recall notations used in our computation:
\begin{itemize}
\item $A_{x}=S_{x}^{-1}A\in\R^{m\times d}$.
\item $s_{x}=\diag(S_{x})\in\R^{m}$.
\item $s_{x,h}=A_{x}h\in\R^{m}$ and $S_{x,h}=\Diag(s_{x,h})\in\R^{m\times m}$.
We drop $x$ if $x$ is clear from the context.
\end{itemize}
Going forward, we use $h$ to denote a vector in $\R^{n}$.
\begin{claim}
\label{claim:1stDiffSlack} $DS_{x}[h]=\Diag(Ah)$ and $DS_{x}^{-1}[h]=-S_{x}^{-1}S_{x,h}$.
\end{claim}

\begin{proof}
The first is obvious from differentiation of $S_{x}=\Diag(Ax-b)$
with respect to $x$. For the second,
\begin{align*}
DS_{x}^{-1}[h] & =-S_{x}^{-1}DS_{x}[h]S_{x}^{-1}=-S_{x}^{-1}\Diag(Ah)S_{x}^{-1}\\
 & =-\Diag(A_{x}h)S_{x}^{-1}=-S_{x}^{-1}\Diag(A_{x}h)\\
 & =-S_{x}^{-1}S_{x,h},
\end{align*}
where $S_{x}^{-1}\Diag(Ah)=\Diag(A_{x}h)$ and $\Diag(A_{x}h)S_{x}^{-1}=S_{x}^{-1}\Diag(A_{x}h)$
hold as all of them are diagonal matrices.
\end{proof}
\begin{claim}
\label{claim:diffLogBarrier} $Dg(x)[h]=-2A_{x}^{\top}S_{x,h}A_{x}$
and $D^{2}g(x)[h,h]=6A_{x}^{\top}S_{x,h}^{2}A_{x}\succeq0$. In the
PSD setting with $A_{X}:=S_{X}^{-1}A$, this becomes $Dg(X)[H]=-2M^{\top}A_{X}^{\top}\Diag\Par{A_{X}\vec{(}H)}A_{X}M$
and $D^{2}g(X)[H,H]=6M^{\top}A_{X}^{\top}\Diag(A_{X}\vec{(}H))^{2}A_{X}M$.
\end{claim}

\begin{proof}
Due to $g(x)=A_{x}^{\top}A_{x}$,
\begin{align*}
Dg(x)[h] & =D(A^{\top}S_{x}^{-2}A)[h]=A^{\top}DS_{x}^{-2}[h]A\\
 & =-2A^{\top}S_{x}^{-3}DS_{x}[h]A=-2A_{x}^{\top}S_{x}^{-1}\Diag(Ah)A_{x}\\
 & =-2A_{x}^{\top}S_{x,h}A_{x}.
\end{align*}

For the second-order directional derivative,
\begin{align*}
Dg^{2}(x)[h,h] & =-2D(A_{x}^{\top}S_{x,h}A_{x})[h]=-2D(A^{\top}S_{x}^{-3}\Diag(Ah)A)[h]\\
 & =6A^{\top}S_{x}^{-4}DS_{x}[h]\Diag(Ah)A\\
 & =6A_{x}^{\top}S_{x,h}^{2}A_{x}.\qedhere
\end{align*}
\end{proof}

\subsubsection{Linear constraints: Volumetric barriers \label{proof:linear-volumetric}}

\cite{vaidya1996new} introduced the \emph{volumetric barrier} for
$Ax\geq b$ defined by 
\[
\phi_{\vol}=\half\log\det\hess\phi_{\log}=\half\log\det A_{x}^{\top}A_{x}.
\]
We collect computational preliminaries regarding the volumetric barrier.
\begin{claim}
$\grad\phi_{\vol}(x)=-A_{x}^{\top}\sigma_{x}$ and $\hess\phi_{\vol}(x)=A_{x}^{\top}\Par{3\Sigma_{x}-2P_{x}^{(2)}}A_{x}$.
\end{claim}

\begin{proof}
Let $g(x)=\hess\phi_{\log}=A_{x}^{\top}A_{x}$. Note that by Claim~\ref{claim:diffLogBarrier}
\begin{align*}
\grad\phi_{\vol}(x)[h] & =-\tr\Par{g^{-1}A_{x}^{\top}S_{x,h}A_{x}}=-\tr\bigg(\underbrace{A_{x}g^{-1}A_{x}^{\top}}_{=P_{x}}S_{x,h}\bigg)\\
 & =-\tr\Par{P_{x}S_{x,h}}=-\tr\Par{P_{x}S_{x,h}I_{m}I_{m}}\\
 & \underset{\text{(i)}}{=}-\bm{1}^{\top}(P_{x}\circ I_{m})S_{x,h}=-1^{\top}\Sigma_{x}A_{x}h\\
 & =-h^{\top}A_{x}^{\top}\sigma_{x},
\end{align*}
where we used Lemma~\ref{lem:Hadamard} in (i). 

For the Hessian of $\phi_{\vol}$,
\[
\hess\phi_{\vol}(x)[h,h]=\half\Par{\tr\Par{g^{-1}D^{2}g[h,h]}-\tr\Par{g^{-1}Dg[h]g^{-1}Dg[h]}}.
\]
For the first term, by Claim~\ref{claim:diffLogBarrier}
\begin{align*}
\half\tr\Par{g^{-1}Dg[h]g^{-1}Dg[h]} & =2\tr\Par{g^{-1}A_{x}^{\top}S_{x,h}A_{x}g^{-1}A_{x}^{\top}S_{x,h}A_{x}}=2\tr\Par{P_{x}S_{x,h}P_{x}S_{x,h}}\\
 & \underset{\text{(i)}}{=}2(A_{x}h)^{\top}\Par{P_{x}\circ P_{x}}(A_{x}h)=2h^{\top}A_{x}^{\top}P_{x}^{(2)}A_{x}h,
\end{align*}
where we used Lemma~\ref{lem:Hadamard} in (i). For the second term,
Claim~\ref{claim:diffLogBarrier} leads to
\begin{align*}
\half\tr\Par{g^{-1}D^{2}g[h,h]} & =3\tr\Par{g^{-1}A_{x}^{\top}S_{x,h}^{2}A_{x}}=3\tr\Par{P_{x}S_{x,h}IS_{x,h}}\\
 & =3h^{\top}A_{x}^{\top}\Par{P_{x}\circ I}A_{x}h=3h^{\top}A_{x}^{\top}\Sigma_{x}A_{x}h.
\end{align*}
Putting these two together, we have
\[
D^{2}\phi_{\vol}(x)[h,h]=h^{\top}A_{x}^{\top}\Par{3\Sigma_{x}-2P_{x}^{(2)}}A_{x}h
\]
and thus
\[
\hess\phi_{\vol}(x)=A_{x}^{\top}(3\Sigma_{x}-2P_{x}^{(2)})A_{x}.\qedhere
\]
\end{proof}
\begin{claim}
\label{claim:schurProjection} $P_{x}^{(2)}\preceq\Sigma_{x}$, so
$A_{x}^{\top}\Sigma_{x}A_{x}\preceq\hess\phi_{\vol}(x)\preceq3A_{x}^{\top}\Sigma_{x}A_{x}$.
\end{claim}

\begin{proof}
Note that $\Sigma_{x}=P_{x}\circ I$. Let us show that $0\leq h^{\top}P_{x}\circ(I-P_{x})h$
for $h\in\Rn$. Since $P_{x}$ and $I-P_{x}$ are both orthogonal
projections matrices, for $C:=P_{x}H(I-P_{x})$ and $H=\Diag(h)$,
\begin{align*}
h^{\top}P_{x}\circ(I-P_{x})h & =\tr\Par{HP_{x}H(I-P_{x})}\\
 & =\tr\Par{(I-P_{x})HP_{x}P_{x}H(I-P_{x})}=\tr(C^{\top}C)\geq0.\qedhere
\end{align*}
\end{proof}
Lastly, we recall computational results on the leverage scores:
\begin{lem}
[\cite{lee2019solving}] \label{lem:usefulFactLeverage} Let $\Sigma_{x}=\Sigma(A_{x})\in\pd,g(x)=A_{x}^{\top}\Sigma_{x}A_{x}$,
and $h\in\Rn$.
\begin{itemize}
\item \textup{(Lemma 26)} $\max_{i\in[m]}\frac{\sigma\Par{\Sigma_{x}^{1/2}A_{x}}_{i}}{\Par{\Sigma_{x}}_{ii}}\leq2m^{\frac{1}{2}}$.
\item \textup{(Lemma 33)} $\norm{A_{x}h}_{\Sigma_{x}}=\norm h_{g(x)}$
and $\norm{A_{x}h}_{\infty}\leq\sqrt{2}m^{\frac{1}{4}}\norm h_{g(x)}$.
\item \textup{(Lemma 34)} $\norm{\Sigma_{x}^{-1}\diag\Par{D\Sigma_{x}[h]}}_{\Sigma_{x}}\leq2\norm h_{g(x)}$.
\end{itemize}
\end{lem}


\subsubsection{Linear constraints: Lewis-weight metric \label{proof:linear-LW}}

We recall the analogue of Lemma~\ref{lem:usefulFactLeverage} for
the $\ell_{p}$-Lewis weight:
\begin{lem}
[\cite{lee2019solving}] \label{lem:usefulFactLewis} Let $W_{x}=\Diag(w_{x}(A_{x}))\in\pd$
be the $\ell_{p}$-Lewis weights and $g(x)=A_{x}^{\top}W_{x}A_{x}$
the Lewis-weights metric, and $h\in\Rn$.
\begin{itemize}
\item \textup{(Lemma 26)} $\max_{i\in[m]}\frac{\sigma\Par{W_{x}^{1/2}A_{x}}_{i}}{\Par{W_{x}}_{ii}}\leq2m^{\frac{2}{p+2}}$.
\item \textup{(Lemma 33)} $\norm{A_{x}h}_{W_{x}}=\norm h_{g(x)}$ and $\norm{A_{x}h}_{\infty}\leq\sqrt{2}m^{\frac{1}{p+2}}\norm h_{g(x)}$.
\item \textup{(Lemma 34)} $\norm{W_{x}^{-1}\diag\Par{W_{x,h}'}}_{W_{x}}\leq p\norm h_{g(x)}$.
\end{itemize}
\end{lem}

Then we recall a directional derivative of the $\ell_{p}$-Lewis weight
of $A_{x}$. 
\begin{lem}
[\cite{lee2019solving}, Lemma 24] \label{lem:DWh} The directional
derivative of the $\ell_{p}$-Lewis weight $W_{x}$ in direction $h\in\Rn$
is
\[
W_{x,h}':=DW_{x}[h]=-2\Diag\Par{\Lambda_{x}G_{x}^{-1}W_{x}s_{x,h}}=-\Diag\Par{W_{x}^{\half}N_{x}W_{x}^{\half}s_{x,h}}
\]
where $\Lambda_{x}\defeq W_{x}-P_{x}^{(2)}$, $\bar{\Lambda}_{x}\defeq W_{x}^{-\half}\Lambda_{x}W_{x}^{-\half}$,
$G_{x}\defeq W_{x}-\Par{1-\frac{2}{p}}\Lambda_{x}$, and $N_{x}\defeq2\bar{\Lambda}_{x}\Par{I-c_{p}\bar{\Lambda}_{x}}^{-1}$.
\end{lem}

It is known that these matrices satisfy
\begin{align}
P_{x}^{(2)}\preceq W_{x}\preceq I,\label{eq:lewisBasic-PWI}\\
\Lambda_{x}\preceq W_{x},\label{eq:lewisBasic-LW}\\
\frac{2}{p}W_{x}\preceq G_{x}\preceq W_{x}, & \text{ which implies }W_{x}^{-1}\preceq G_{x}^{-1}\preceq\frac{p}{2}W_{x}^{-1}\text{ and }I\preceq W_{x}^{\half}G_{x}^{-1}W_{x}^{\half}\preceq\frac{p}{2}I.\label{eq:lewisBasic-WGW}
\end{align}
Next, we recall bounds on the derivatives of matrices relevant to
Lewis weights.
\begin{lem}
[\cite{lee2019solving}] \label{lem:LS-comp-tool} For $P=\Brace{x\in\Rn:Ax\geq b}$
with $A\in\R^{m\times n}$ and $b\in\R^{m}$, let $x\in\inter(P)$
and $h\in\Rn$. For the $\ell_{p}$-Lewis weights matrix $W_{x}$
and $c_{p}:=1-\frac{2}{p}$ with $p>2$, we denote $\bar{\Lambda}_{x}\defeq W_{x}^{-\half}\Lambda_{x}W_{x}^{-\half}=I-W_{x}^{-\half}P_{x}^{(2)}W_{x}^{-\half}$,
$N_{x}\defeq2\bar{\Lambda}_{x}\Par{I-c_{p}\bar{\Lambda}_{x}}^{-1}$
and $\theta_{x}=A_{x}^{\top}W_{x}A_{x}$.
\begin{itemize}
\item \textup{(Lemma 31)} $N_{x}$ is symmetric and $0\preceq N_{x}\preceq pI$.
\item \textup{(Lemma 34)} $\norm{W_{x}^{-1}w_{x,h}}_{\infty}\leq p\Par{\sqrt{2}m^{\frac{1}{p+2}}+\frac{p}{2}}\norm h_{\theta_{x}}$.
\item \textup{(Lemma 37)} $\norm{\Par{I+N_{x}}^{-\half}DN_{x}[h]\Par{I+N_{x}}^{-\half}}_{2}\leq4p^{2}\sqrt{p}\norm h_{\theta_{x}}$.
\end{itemize}
\end{lem}

Lastly, we remind a result about closeness of the Lewis weights at
close-by points.
\begin{lem}
[\cite{lee2019solving}] \label{lem:weight-close} In the same setting
above, let $z_{t,\alpha}\in\R^{m}$ be a vector defined by 
\[
[z_{t,\alpha}]_{i}:=\frac{d}{dt}\log\frac{[w_{t,i}]^{\alpha}}{s_{t,i}}.
\]
Then we have 
\[
\norm{z_{t}}_{\infty}\leq\Par{\sqrt{2}\Par{|\alpha|p+1}m^{\frac{1}{p+2}}+p|\alpha|\max\Par{\frac{p}{2},1}}\norm h_{A_{t}^{\top}W_{t}A_{t}}.
\]
\end{lem}

Now we show highly self-concordance of the Lewis-weight metric.
\begin{lem}
\label{lem:Lw-hsc} $g(x)=c\theta(x)=cA_{x}^{\top}W_{x}A_{x}$ with
$c=c_{1}(\log m)^{c_{2}}\sqrt{n}$ for some constants $c_{1},c_{2}>0$
is highly self-concordant.
\end{lem}

\begin{proof}
For $\theta(x)=A_{x}^{\top}W_{x}A_{x}$ and $h\in\Rn$, we have
\begin{align}
D^{2}\theta[h,h] & =6A_{x}^{\top}S_{x,h}W_{x}S_{x,h}A_{x}-4A_{x}^{\top}W_{x,h}'S_{x,h}A_{x}+A_{x}^{\top}W_{x,h}''A_{x}\nonumber \\
D^{2}\theta[h,h,h,h] & =6s_{x,h}^{\top}S_{x,h}W_{x}S_{x,h}s_{x,h}^{\top}-4s_{x,h}^{\top}W_{x,h}'S_{x,h}s_{x,h}^{\top}+s_{x,h}^{\top}W_{x,h}''s_{x,h}^{\top}\nonumber \\
 & =\tr\Par{6S_{x,h}^{4}W_{x}-4S_{x,h}^{3}W_{x,h}'+S_{x,h}^{2}W_{x,h}''}.\label{eq:LW-fourth-moment}
\end{align}
For the first term, 
\[
\Abs{\tr\Par{S_{x,h}^{4}W_{x}}}\leq\norm{s_{x,h}}_{\infty}^{2}\norm h_{\theta}^{2}.
\]
For the second term,

\begin{align*}
\Abs{\tr\Par{S_{x,h}^{3}W_{x,h}'}} & \leq\norm{s_{x,h}}_{\infty}^{2}\tr\Par{\sqrt{S_{x,h}W_{x,h}'^{2}S_{x,h}}}=\norm{s_{x,h}}_{\infty}^{2}\tr\Par{\sqrt{W_{x,h}'W_{x}^{-1}W_{x,h}'S_{x,h}W_{x}S_{x,h}}}\\
 & =\norm{s_{x,h}}_{\infty}^{2}\tr\Par{\sqrt{W_{x,h}'W_{x}^{-1}W_{x,h}'}\sqrt{S_{x,h}W_{x}S_{x,h}}}\\
 & \underset{\text{(i)}}{\leq}\norm{s_{x,h}}_{\infty}^{2}\sqrt{\tr\Par{W_{x,h}'W_{x}^{-1}W_{x,h}'}}\sqrt{\tr\Par{S_{x,h}W_{x}S_{x,h}}}=\norm{s_{x,h}}_{\infty}^{2}\norm{W_{x}^{-1}w_{x,h}'}_{W_{x}}\norm h_{\theta}\\
 & \underset{\text{(ii)}}{\leq}p\norm{s_{x,h}}_{\infty}^{2}\norm h_{\theta}^{2},
\end{align*}
where we used the Cauchy-Schwarz inequality in (i) and Lemma~\ref{lem:usefulFactLewis}-3
in (ii).

For the third term, let us compute the second-order directional derivate
of $W_{x}$ in direction $h$:
\begin{align*}
W_{x,h}' & =-2\Diag\Par{\Lambda_{x}G_{x}^{-1}W_{x}s_{x,h}}=-2\Diag\Par{\Lambda_{x}\Par{W_{x}-c_{p}\Lambda_{x}}^{-1}W_{x}s_{x,h}}\\
 & =-2\Diag\Par{W_{x}^{\half}\bar{\Lambda}_{x}\Par{I-c_{p}\bar{\Lambda}_{x}}^{-1}W_{x}^{\half}s_{x,h}}=-\Diag\Par{W_{x}^{\half}N_{x}W_{x}^{\half}s_{x,h}}\\
W_{x,h}'' & =-\Diag\Par{\half W_{x}^{-\half}W_{x,h}'N_{x}W_{x}^{\half}s_{x,h}+W_{x}^{\half}N_{x,h}'W_{x}^{\half}s_{x,h}+\half W_{x}^{\half}N_{x}W_{x}^{-\half}W_{x,h}'s_{x,h}}\\
 & \qquad-2\Diag\Par{-\Lambda_{x}G_{x}^{-1}W_{x}S_{x,h}s_{x,h}},
\end{align*}
where $W_{x,h}':=DW_{x}[h]$ and $N_{x,h}':=DN_{x}[h]$. Thus,
\begin{align*}
 & \tr\Par{S_{x,h}^{2}W_{x,h}''}\\
 & =-\half\tr\bigg(S_{x,h}^{2}\Diag\bigg(\underbrace{W_{x}^{-\half}W_{x,h}'N_{x}W_{x}^{\half}s_{x,h}}_{\text{I}}\bigg)\bigg)-\tr\bigg(S_{x,h}^{2}\Diag\bigg(\underbrace{W_{x}^{\half}N_{x,h}'W_{x}^{\half}s_{x,h}}_{\text{II}}\bigg)\bigg)\\
 & \qquad-\half\tr\bigg(S_{x,h}^{2}\Diag\bigg(\underbrace{W_{x}^{\half}N_{x}W_{x}^{-\half}W_{x,h}'s_{x,h}}_{\text{III}}\bigg)\bigg)-2\tr\bigg(S_{x,h}^{2}\Diag\bigg(\underbrace{\Lambda_{x}G_{x}^{-1}W_{x}S_{x,h}s_{x,h}}_{\text{IV}}\bigg)\bigg).
\end{align*}
Each term is of the form $\tr\Par{S_{x,h}^{2}\Diag(v)}$ for a vector
$v\in\R^{m}$, and this can be bounded as follows:
\begin{align*}
\Abs{\tr\Par{S_{x,h}^{2}\Diag(v)}} & =\Abs{\tr\Par{S_{x,h}^{2}W_{x}^{\half}W_{x}^{-\half}\Diag(v)}}\\
 & \leq\sqrt{\tr\Par{W_{x}^{\half}S_{x,h}^{4}W_{x}^{\half}}}\sqrt{\tr\Par{\Diag(v)W_{x}^{-1}\Diag(v)}}\\
 & \leq\norm{s_{x,h}}_{\infty}\norm h_{\theta}\norm v_{W_{x}^{-1}}.
\end{align*}

We bound the local norms of the term (I \textasciitilde{} IV). In
our calculation, the operator $\lesssim$ hides universal constants
and poly-logarithmic factors in $m$:
\begin{align*}
\norm{\text{I}}_{W_{x}^{-1}} & =\norm{W_{x}^{-1}W_{x,h}'N_{x}W_{x}^{\half}s_{x,h}}_{2}\leq\underbrace{\norm{W_{x}^{-1}W_{x,h}'}_{2}}_{\text{Use Lemma \ref{lem:LS-comp-tool}-2}}\underbrace{\norm{N_{x}}_{2}}_{\text{Lemma \ref{lem:LS-comp-tool}-1}}\norm{W_{x}^{\half}s_{x,h}}_{2}\\
 & \lesssim p^{2}m^{\frac{1}{p+2}}\norm h_{\theta}\cdot p\cdot\norm h_{\theta}\\
 & =p^{3}m^{\frac{1}{p+2}}\norm h_{\theta}^{2}.
\end{align*}
For the second term,
\begin{align*}
\norm{\text{II}}_{W_{x}^{-1}} & =\norm{N_{x,h}'W_{x}^{\half}s_{x,h}}_{2}=\norm{\Par{I+N_{x}}^{\half}\Par{I+N_{x}}^{-\half}N_{x,h}'\Par{I+N_{x}}^{-\half}\Par{I+N_{x}}^{\half}W_{x}^{\half}s_{x,h}}_{2}\\
 & \leq\underbrace{\norm{I+N_{x}}_{2}}_{\text{Lemma \ref{lem:LS-comp-tool}-1}}\underbrace{\norm{\Par{I+N_{x}}^{-\half}N_{x,h}'\Par{I+N_{x}}^{-\half}}_{2}}_{\text{Lemma \ref{lem:LS-comp-tool}-3}}\norm{W_{x}^{\half}s_{x,h}}_{2}\\
 & \lesssim p^{3.5}\norm h_{\theta}^{2}.
\end{align*}
For the third term, 
\begin{align*}
\norm{\text{III}}_{W_{x}^{-1}} & =\norm{N_{x}W_{x}^{-\half}W_{x,h}'s_{x,h}}_{2}\leq\underbrace{\norm{N_{x}}_{2}}_{\text{Lemma \ref{lem:LS-comp-tool}-1}}\underbrace{\norm{W_{x}^{-1}W_{x,h}'}_{2}}_{\text{Lemma \ref{lem:LS-comp-tool}-2}}\norm{W_{x}s_{x,h}}_{2}\\
 & \lesssim p^{3}m^{\frac{1}{p+2}}\norm h_{\theta}^{2}.
\end{align*}
 Let us bound the last term:
\begin{align*}
\norm{\text{IV}}_{W_{x}^{-1}}^{2} & =s_{x,h}^{\top}S_{x,h}W_{x}G_{x}^{-1}\underbrace{\Lambda_{x}W_{x}^{-1}\Lambda_{x}}_{\preceq W_{x}\ \text{\eqref{eq:lewisBasic-LW}}}G_{x}^{-1}W_{x}S_{x,h}s_{x,h}\leq s_{x,h}^{\top}S_{x,h}W_{x}\underbrace{G_{x}^{-1}W_{x}G_{x}^{-1}}_{\preceq\frac{p^{2}}{4}W_{x}^{-1}\ \text{\eqref{eq:lewisBasic-WGW}}}W_{x}S_{x,h}s_{x,h}\\
 & \leq p^{2}s_{x,h}^{\top}S_{x,h}W_{x}S_{x,h}s_{x,h}=p^{2}s_{x,h}^{\top}W_{x}^{\half}S_{x,h}^{2}W_{x}^{\half}s_{x,h}\leq p^{2}\norm{s_{x,h}}_{\infty}^{2}\norm h_{\theta}^{2}\\
 & \leq p^{2}m^{\frac{2}{p+2}}\norm h_{\theta}^{4},
\end{align*}
where we used Lemma~\ref{lem:usefulFactLewis}-2 in the last line.

Combining these bounds, (\ref{eq:trGamma}), and (\ref{eq:trGammaBasic})
with $p=O(\log m)$
\begin{align*}
\Abs{\tr\Par{S_{x,h}^{2}W_{x,h}''}} & \lesssim\norm{s_{x,h}}_{\infty}\norm h_{\theta}\norm h_{\theta}^{2}.
\end{align*}
Along with other bounds, we conclude that
\begin{align*}
\Abs{D^{2}\theta[h,h,h,h]} & \lesssim\norm{s_{x,h}}_{\infty}^{2}\norm h_{\theta}^{2}+\norm{s_{x,h}}_{\infty}\norm h_{\theta}^{3}\lesssim\norm h_{\theta}^{4}
\end{align*}
where we used $\norm{s_{x,h}}_{\infty}\leq\sqrt{2}m^{\frac{1}{p+2}}\norm h_{\theta}\lesssim\norm h_{\theta}$
due to Lemma~\ref{lem:usefulFactLewis}-2.
\end{proof}

\subsubsection{Linear constraints: strong self-concordance and symmetry \label{proof:linear-SSC-symm}}

We relate SSC and symmetry to well-studied terms in the field of optimization,
such as $\max_{i}\frac{\sigma\Par{\sqrt{D_{x}}A_{x}}_{i}}{(D_{x})_{ii}}$
and $\norm{D_{x,h}'}_{D_{x}^{-1}}^{2}$.
\begin{proof}
[Proof of Lemma~\ref{lem:helper4Diagonal}] Let us write $g(x)=A_{x}^{\top}D_{x}A_{x}=A^{\top}V_{x}A$
for $V_{x}:=S_{x}^{-1}D_{x}S_{x}^{-1}$. By Claim~\ref{claim:1stDiffSlack}
\begin{align}
Dg(x)[h] & =A^{\top}DV_{x}[h]A\nonumber \\
 & =A^{\top}\Par{-2S_{x}^{-1}S_{x,h}S_{x}^{-1}D_{x}+S_{x}^{-1}DD_{x}[h]S_{x}^{-1}}A\nonumber \\
 & =A^{\top}V_{x}^{1/2}\Par{-2S_{x,h}+D_{x}^{-1}DD_{x}[h]}V_{x}^{1/2}A\label{eq:Dgh}\\
 & =A^{\top}V_{x}^{1/2}\overline{D}_{x}V_{x}^{1/2}A,\nonumber 
\end{align}
where $\overline{D}_{x}:=-2S_{x,h}+D_{x}^{-1}DD_{x}[h]$.

For $P_{x}:=P\big(\underbrace{\sqrt{V_{x}}A}_{=:B_{x}\in\R^{m\times n}}\big)=V_{x}^{\half}A\Par{A^{\top}V_{x}A}^{\dagger}A^{\top}V_{x}^{\half}=B_{x}(B_{x}^{\top}B_{x})^{\dagger}B_{x}^{\top}$,
we have
\begin{align*}
 & \norm{(g'(x)+g(x))^{-\half}Dg(x)[h](g'(x)+g(x))^{-\half}}_{F}^{2}\\
 & =\tr\Par{(g'+g)^{-1}Dg[h](g'+g)^{-1}Dg[h]}\\
 & =\tr\bigg((g'+g)^{-1}A^{\top}V_{x}^{1/2}\overline{D}_{x}\underbrace{V_{x}^{1/2}A(g'+g)^{-1}A^{\top}V_{x}^{1/2}}_{=:P_{x}'}\overline{D}_{x}V_{x}^{1/2}A\bigg)\\
 & =\tr\Par{P_{x}'\overline{D}_{x}P_{x}'\overline{D}_{x}}.
\end{align*}

We show that $P_{x}'\preceq P_{x}$, which is equivalent to $I-P_{x}'\succeq I-P_{x}$.
It holds that $P_{x}'^{2}\preceq P_{x}'$ due to
\begin{align*}
P_{x}'P_{x}' & =B_{x}(g'+B_{x}^{\top}B_{x})^{-1}B_{x}^{\top}B_{x}(g'+B_{x}^{\top}B_{x})^{-1}B_{x}^{\top}\\
 & \preceq B_{x}(g'+B_{x}^{\top}B_{x})^{-1}(g'+B_{x}^{\top}B_{x})(g'+B_{x}^{\top}B_{x})^{-1}B_{x}^{\top}\\
 & =B_{x}(g'+B_{x}^{\top}B_{x})^{-1}B_{x}^{\top}\\
 & =P_{x}',
\end{align*}
and $(I-P_{x}')^{2}\preceq I-P_{x}'$ follows due to
\begin{align*}
(I-P_{x}')(I-P_{x}') & =I+P_{x}'P_{x}'-2P_{x}'\preceq I+P_{x}'-2P_{x}'\\
 & =I-P_{x}'.
\end{align*}
Note that for any $v\in\R^{m}$
\begin{align*}
v^{\top}(I-P_{x}')v & \succeq v^{\top}(I-P_{x}')^{2}v=\norm{(I-P_{x}')v}_{2}^{2}\\
 & \geq\norm{(I-P_{x})v}_{2}^{2}=v^{\top}(I-P_{x})^{2}v\\
 & =v^{\top}(I-P_{x})v,
\end{align*}
where the inequality holds due to $P_{x}'v,\,P_{x}v\in\text{range}(B_{x})$
and $P_{x}v=\arg\min_{w\in\text{range}(B_{x})}\norm{v-w}_{2}^{2}$.
Therefore, $I-P_{x}'\succeq I-P_{x}$ (and thus $P_{x}'\preceq P_{x}$).

Then due to $P_{x}'\succeq0$
\begin{align*}
\tr\Par{P_{x}'\overline{D}_{x}P_{x}'\overline{D}_{x}} & =\tr\Par{P_{x}'^{1/2}\overline{D}_{x}P_{x}'\overline{D}_{x}P_{x}'^{1/2}}\\
 & \leq\tr\Par{P_{x}'^{1/2}\overline{D}_{x}P_{x}\overline{D}_{x}P_{x}'^{1/2}}\leq\tr\Par{P_{x}\overline{D}_{x}P_{x}\overline{D}_{x}}\\
 & \underset{\text{(i)}}{=}\diag\Par{\overline{D}_{x}}^{\top}P_{x}^{(2)}\diag\Par{\overline{D}_{x}}\\
 & \underset{\text{(ii)}}{\leq}\diag\Par{\overline{D}_{x}}^{\top}\Sigma_{x}\diag\Par{\overline{D}_{x}}\\
 & \underset{\text{(iii)}}{\leq}4\sum_{i=1}^{m}\sigma\Par{\sqrt{D_{x}}A_{x}}_{i}\Par{(A_{x}h)_{i}^{2}+(D_{x}^{-1}DD_{x}[h])_{i}^{2}}\\
 & \leq4\max_{i}\Par{\frac{\sigma\Par{\sqrt{D_{x}}A_{x}}_{i}}{(D_{x})_{ii}}}\cdot\sum_{i=1}^{m}(D_{x})_{ii}\Par{(A_{x}h)_{i}^{2}+(D_{x}^{-1}DD_{x}[h])_{i}^{2}}\\
 & \underset{\text{(iv)}}{=}4\max_{i}\Par{\frac{\sigma\Par{\sqrt{D_{x}}A_{x}}_{i}}{(D_{x})_{ii}}}\cdot\Par{\norm h_{g(x)}^{2}+\sum_{i=1}^{m}\Par{D_{x}^{-1}}_{ii}(DD_{x}[h])_{i}^{2}},
\end{align*}
where (i) holds due to $x^{\top}(A\hada B)y=\tr\Par{\Diag(x)A\Diag(y)B^{\top}}$
(Lemma~\ref{lem:Hadamard}), (ii) follows from $P_{x}^{(2)}\preceq\Sigma_{x}$
(Claim~\ref{claim:schurProjection})\footnote{Even though this lemma is proven for leverage scores, the proof there
can be extended to any orthogonal projection matrices.}, (iii) uses $(a+b)^{2}\leq2\Par{a^{2}+b^{2}}$ for $a,b\in\R$ and
$\Sigma_{x}=\Diag(P_{x})=\Diag\Par{\sigma\Par{\sqrt{V_{x}}A}}=\Diag\Par{\sigma\Par{\sqrt{D_{x}}A_{x}}}$,
and (iv) holds due to $\sum_{i=1}^{m}D_{x,i}(A_{x}h)_{i}^{2}=h^{\top}A_{x}^{\top}D_{x}A_{x}h=h^{\top}g(x)h$.

For the second claim,
\begin{align*}
 & \max_{h:\norm h_{g(x)}=1}\norm{A_{x}h}_{\infty}=\max_{h:\norm h_{g(x)}=1}\max_{i\in[m]}\Abs{\frac{a_{i}^{\top}h}{s_{i}}}=\max_{i\in[m]}\max_{u:\norm u_{2}=1}\Abs{\frac{a_{i}^{\top}g(x)^{-1/2}u}{s_{i}}}\\
 & =\max_{i\in[m]}\norm{g(x)^{-1/2}\frac{a_{i}}{s_{i}}}_{2}=\max_{i\in[m]}\sqrt{\frac{1}{s_{i}^{2}}a_{i}^{\top}g(x)^{-1}a_{i}}=\sqrt{\max_{i\in[m]}e_{i}^{\top}A_{x}^{\top}g(x)^{-1}A_{x}e_{i}}\\
 & =\sqrt{\max_{i\in[m]}\frac{\sigma\Par{\sqrt{D_{x}}A_{x}}_{i}}{(D_{x})_{ii}}}.
\end{align*}

For the last claim, for $h\in\Rn$ such that $\norm{A_{x}h}_{\infty}\leq1$
(i.e., $h\in K\cap(2x-K)$ for $K=\{Ax\geq b\}$ due to Lemma~\ref{lem:symmforPolytope})
we have
\begin{align*}
h^{\top}g(x)h & =h^{\top}A_{x}^{\top}D_{x}A_{x}h=\sum_{i=1}^{m}(D_{x})_{ii}(A_{x}h)_{i}^{2}\leq\norm{A_{x}h}_{\infty}^{2}\sum_{i=1}^{m}(D_{x})_{ii}\leq\tr\Par{D_{x}}.\qedhere
\end{align*}
\end{proof}
Using this, we can check SSC and compute the symmetry parameters of
metrics of the form $A_{x}^{\top}D_{x}A_{x}$:
\begin{proof}
[Proof of Lemma~\ref{lem:paramsBarrier}] \textbf{Logarithmic barrier}:
To show that $g$ is SSC along $\rowspace(A)$, consider a self-concordant
matrix $g(y)=S_{y}^{-2}=-\nabla_{y}^{2}\sum_{i=1}^{m}\log(y_{i})$
defined on $\{y\in\R^{m}:y\geq0\}$. By putting $D_{x}=I_{m}$ and
$A_{x}=S_{x}^{-1}$ into Lemma~\ref{lem:helper4Diagonal}-1, 
\[
\norm{g(x)^{-\half}Dg(x)[h]g(x)^{-\half}}_{F}\leq2\sqrt{\max_{i\in[m]}\sigma(A_{x})_{i}}\norm h_{g(x)}.
\]
As $P_{x}$ is the orthogonal projection, $P_{x}\preceq I$ and $\sigma(A_{x})\leq1$.
Thus, 
\[
\norm{g(x)^{-\half}Dg(x)[h]g(x)^{-\half}}_{F}\leq2\norm h_{g(x)},
\]
and it leads to strong self-concordance of the logarithmic barriers.
Through the linear map $Tx=Ax-b=y$, we can recover 
\[
g(x)=\hess\phi_{\log}(x)=A^{\top}S_{y}^{-2}A=A_{x}^{\top}A_{x},
\]
and this is SSC along $\rowspace(A)$ by Lemma~\ref{lem:linear-trans-matrix}.

For the $\onu$-symmetry, we note that the first part (i.e., $\dcal_{g}^{1}(x)\subset K\cap(2x-K)$)
follows from Lemma~\ref{lem:symmetricLeftpart}. The second part
is immediate from $\onu=\tr\Par{I_{m}}=m$ and Lemma~\ref{lem:helper4Diagonal}-3.

\textbf{Approximate volumetric barrier}: Let us set $D_{x}$ to $\Sigma_{x}=\Sigma(A_{x})$
in Lemma~\ref{lem:helper4Diagonal}. By Lemma~\ref{lem:usefulFactLeverage},
we have
\begin{align*}
\max_{i}\Par{\frac{\sigma\Par{\sqrt{D_{x}}A_{x}}_{i}}{(D_{x})_{ii}}} & \leq2\sqrt{m},\\
\sum_{i=1}^{m}\Par{D_{x}^{-1}}_{ii}(DD_{x}[h])_{i}^{2} & =\norm{\Sigma_{x}^{-1}\diag\Par{D\Sigma_{x}[h]}}_{\Sigma_{x}}^{2}\\
 & \leq4\norm h_{g(x)}^{2}.
\end{align*}
Thus,
\begin{align*}
\norm{g(x)^{-\half}Dg(x)[h]g(x)^{-\half}}_{F}^{2} & \leq4\max_{i}\Par{\frac{\sigma\Par{\sqrt{D_{x}}A_{x}}_{i}}{(D_{x})_{ii}}}\cdot\Par{\norm h_{g(x)}^{2}+\sum_{i=1}^{m}\Par{D^{-1}}_{ii}(DD_{x}[h])_{i}^{2}}\\
 & \leq40\sqrt{m}\norm h_{g(x)}^{2}.
\end{align*}
For the symmetry parameter, $\norm{A_{x}(y-x)}_{\infty}\leq\sqrt{\max_{i\in[m]}\frac{\sigma\Par{\sqrt{D_{x}}A_{x}}_{i}}{D_{x,i}}}\leq m^{1/4}$
for $y\in\dcal_{g}^{1}(x)$ by Lemma~\ref{lem:helper4Diagonal}-2.
Also, Lemma~\ref{lem:helper4Diagonal}-3 implies that $y$ with $\norm{A_{x}(y-x)}_{\infty}\leq1$
is contained in $\dcal_{g}^{\sqrt{\tr(D_{x})}}(x)$, where
\[
\tr\Par{D_{x}}=\tr\Par{P_{x}}=\tr\Par{A_{x}(A_{x}^{\top}A_{x})^{-1}A_{x}}=\tr\Par{I_{n}}=n.
\]
Therefore, $\tilde{g}(x):=40\sqrt{m}g(x)=40\sqrt{m}A_{x}^{\top}\Sigma_{x}A_{x}$
is strongly self-concordant with the symmetry parameter $\onu=O(\sqrt{m}n)$.

\textbf{Vaidya metric}: consider the metric without scaling: $g(x):=A_{x}^{\top}D_{x}A_{x}$,
where we set $D_{x}=\Sigma_{x}+\frac{n}{m}I_{m}$. Then
\begin{align}
\max_{i}\Par{\frac{\sigma\Par{\sqrt{D_{x}}A_{x}}_{i}}{D_{x,i}}} & \underset{\text{Lemma \ref{lem:helper4Diagonal}-2}}{=}\Par{\max_{h\in\Rn}\frac{\norm{A_{x}h}_{\infty}}{\norm h_{g(x)}}}^{2}\underset{\text{(i)}}{\leq}\sqrt{\frac{m}{n}},\label{eq:28-1}\\
\sum_{i=1}^{m}\Par{D_{x}^{-1}}_{ii}(DD_{x}[h])_{i}^{2} & \underset{\text{(ii)}}{\leq}\sum_{i=1}^{m}\Par{\Sigma_{x}^{-1}}_{ii}(D\Sigma_{x}[h])_{i}^{2}\nonumber \\
 & \underset{\text{Lemma \ref{lem:usefulFactLeverage}-3}}{\leq}4h^{\top}A_{x}^{\top}\Sigma_{x}A_{x}h\leq4\norm h_{g(x)}^{2},\nonumber 
\end{align}
where (i) follows from (4.5) of \cite{anstreicher1997volumetric}
and (ii) holds due to $\Sigma_{x}\preceq D_{x}$. Putting these back
to Lemma~\ref{lem:helper4Diagonal}-1,
\begin{align*}
\norm{g(x)^{-\half}Dg(x)[h]g^{-\half}}_{F}^{2} & \leq4\max_{i}\Par{\frac{\sigma\Par{\sqrt{D_{x}}A_{x}}_{i}}{D_{x,i}}}\cdot\Par{\norm h_{g(x)}^{2}+\sum_{i=1}^{m}(D_{x}^{-1})_{ii}(DD_{x}[h])_{i}^{2}}\\
 & \leq20\sqrt{\frac{m}{n}}\norm h_{g(x)}^{2}.
\end{align*}
Thus, $\tilde{g}(x):=22\sqrt{\frac{m}{n}}g(x)=22\sqrt{\frac{m}{n}}A_{x}^{\top}\Par{\Sigma_{x}+\frac{n}{m}I_{m}}A_{x}$
is strongly self-concordant.

For the symmetry parameter, Lemma~\ref{lem:helper4Diagonal}-2 implies
that for $y\in\dcal_{g}^{1}(x)$
\[
\norm{A_{x}(y-x)}_{\infty}\leq\norm{y-x}_{g(x)}\sqrt{\max_{i\in[m]}\frac{\sigma\Par{\sqrt{D_{x}}A_{x}}_{i}}{(D_{x})_{ii}}}\underset{\text{\eqref{eq:28-1}}}{\leq}\Par{\frac{m}{n}}^{1/4}.
\]
Also, Lemma~\ref{lem:helper4Diagonal}-3 implies that $y$ with $\norm{A_{x}(y-x)}_{\infty}\leq1$
is contained in $\dcal_{g}^{\sqrt{\tr(D_{x})}}(x)$, where
\[
\tr\Par{D_{x}}=\tr\Par{\Sigma_{x}+\frac{n}{m}I_{m}}=\tr\Par{A_{x}(A_{x}^{\top}A_{x})^{-1}A_{x}}+n=\tr\Par{I_{n}}+n=2n.
\]
Therefore, $\tilde{g}(x)=22\sqrt{\frac{m}{n}}g(x)=22\sqrt{\frac{m}{n}}A_{x}^{\top}\Par{\Sigma_{x}+\frac{n}{m}I_{m}}A_{x}$
ensures
\[
\dcal_{\tilde{g}}^{1}(x)\subset K\cap(2x-K)\subset\dcal_{\tilde{g}}^{\sqrt{44(mn)^{1/2}}}(x),
\]
so $\tilde{g}$ is $O(\sqrt{mn})$-symmetric.

\textbf{Lewis-weight metric}: Consider the unscaled version first:
$g(x)=A_{x}^{\top}W_{x}A_{x}$. By Lemma~\ref{lem:helper4Diagonal}-1
\begin{align*}
\norm{g(x)^{-\half}Dg(x)[h]g(x)^{-\half}}_{F} & \leq2\sqrt{\max_{i}\Par{\frac{\sigma\Par{\sqrt{W_{x}}A_{x}}_{i}}{(W_{x})_{ii}}}\cdot\bigg(\norm h_{g(x)}^{2}+\sum_{i=1}^{m}\Par{W_{x}^{-1}}_{ii}\diag\Par{W_{x,h}'}^{2}\bigg)}\\
 & \underset{\text{(i)}}{\leq}2\sqrt{2m^{\frac{2}{p+2}}}\sqrt{\norm h_{g(x)}^{2}+p^{2}\norm h_{g(x)}^{2}}\\
 & \leq\Par{8m^{\frac{2}{p+2}}(1+p^{2})}^{1/2}\norm h_{g(x)},
\end{align*}
where in (i) we used Lemma~\ref{lem:usefulFactLewis}-1 and 3.

For the first part of $\onu$-symmetry, Lemma~\ref{lem:helper4Diagonal}-2
implies that
\[
\max_{h:\norm h_{g(x)}=1}\norm{A_{x}h}_{\infty}=\sqrt{\max_{i\in[m]}\frac{\sigma\Par{\sqrt{W_{x}}A_{x}}_{i}}{W_{x,i}}}\leq\sqrt{2m^{\frac{2}{p+2}}},
\]
 and Lemma~\ref{lem:helper4Diagonal}-3 leads to $K\cap(2x-K)\subset\dcal_{g}^{\sqrt{n}}(x)$
due to
\[
\tr\Par{W_{x}}=\tr\Par{W_{x}^{\half-\frac{1}{p}}A_{x}\Par{A_{x}^{\top}W_{x}^{1-\frac{2}{p}}A_{x}}^{-1}A_{x}^{\top}W_{x}^{\half-\frac{1}{p}}}=\tr\Par{A_{x}^{\top}W_{x}^{1-\frac{2}{p}}A_{x}\Par{A_{x}^{\top}W_{x}^{1-\frac{2}{p}}A_{x}}^{-1}}=n.
\]
Therefore, $16p^{2}m^{\frac{2}{p+2}}A_{x}^{\top}W_{x}A_{x}$ is strongly
self-concordant with $O\Par{nm^{\frac{2}{p+2}}}$-symmetric by Lemma~\ref{lem:symmforPolytope}.
By setting $p=O(\log m)$, the claim follows.
\end{proof}

\subsubsection{Linear constraints: strongly lower trace self-concordance of Vaidya
\label{proof:linear-vaidya-SLTSC}}

Let $\theta_{1}(x):=A_{x}^{\top}\Sigma_{x}A_{x}$, $\theta_{2}(x):=A_{x}^{\top}A_{x}$,
and $\Gamma_{x}:=\Diag\Par{A_{x}g(x)^{-1}A_{x}^{\top}}\in\R^{m\times m}$.
We define $s_{x,h}:=A_{x}h$ and $S_{x,h}:=\Diag(s_{x,h})$. Recall
that for a matrix function $g(x)$ we use $g_{x,h}'$ and $g_{x,h}''$
to denote $Dg(x)[h]$ and $D^{2}g(x)[h,h]$.

We first derive formulas of derivatives of leverage scores, orthogonal
projections, and so on.
\begin{lem}
\label{lem:calculusLeverage} For $x,h\in\Rn$, let $P_{x}=A_{x}(A_{x}^{\top}A_{x})^{-1}A_{x}^{\top}$,
$\Sigma_{x}=\Diag(P_{x})$, and $\Lambda_{x}=\Sigma_{x}-P_{x}^{(2)}$.
Denote $\theta_{1}(x):=A_{x}^{\top}\Sigma_{x}A_{x}$ and $\theta_{2}(x):=A_{x}^{\top}A_{x}$.
\begin{itemize}
\item $\Sigma_{x,h}'=-2\Diag\Par{\Lambda_{x}s_{x,h}}=2\Par{\Diag\Par{P_{x}S_{x,h}P_{x}}-\Sigma_{x}S_{x,h}}$.
\item $P_{x,h}'=-P_{x}S_{x,h}-S_{x,h}P_{x}+2P_{x}S_{x,h}P_{x}$.
\item $\Lambda_{x,h}'=-2\Diag(\Lambda_{x}s_{x,h})+2P_{x}\circ P_{x}S_{x,h}+2S_{x,h}P_{x}\circ P_{x}-2(P_{x}S_{x,h}P_{x})\circ P_{x}-2P_{x}\circ(P_{x}S_{x,h}P_{x})$.
\item $\Sigma_{x,h}''=6S_{x,h}\Sigma_{x}S_{x,h}+8\Diag\Par{P_{x}S_{x,h}P_{x}S_{x,h}P_{x}}-6\Diag\Par{P_{x}S_{x,h}^{2}P_{x}}-8\Diag\Par{S_{x,h}P_{x}S_{x,h}P_{x}}$.
\item $D\theta_{1}(x)[h]=-2A_{x}^{\top}\Sigma_{x}S_{x,h}A_{x}+A_{x}^{\top}\Sigma_{x,h}'A_{x}$.
\item 
\begin{align*}
D^{2}\theta_{1}(x)[h,h] & =6A_{x}^{\top}S_{x,h}\Sigma_{x}S_{x,h}A_{x}-4A_{x}^{\top}\Sigma_{x,h}'S_{x,h}A_{x}+A_{x}^{\top}\Sigma_{x,h}''A_{x}\\
 & =20A_{x}^{\top}S_{x,h}\Sigma_{x}S_{x,h}A_{x}-16A_{x}^{\top}\Diag\Par{S_{x,h}P_{x}S_{x,h}P_{x}}A_{x}\\
 & \qquad-6A_{x}^{\top}\Diag\Par{P_{x}S_{x,h}^{2}P_{x}}A_{x}+8A_{x}^{\top}\Diag\Par{P_{x}S_{x,h}P_{x}S_{x,h}P_{x}}A_{x}.
\end{align*}
\item $D\theta_{2}(x)[h]=-2A_{x}^{\top}S_{x,h}A_{x}$ and $D^{2}\theta_{2}(x)[h,h]=6A_{x}^{\top}S_{x,h}^{2}A_{x}$.
\end{itemize}
\end{lem}

\begin{proof}
$\Sigma_{x,h}'=-2\Diag\Par{\Lambda_{x}s_{x,h}}$ follows from Lemma
24 in \cite{lee2019solving}. Using the definition of $\Lambda_{x}=\Sigma_{x}-P_{x}^{(2)}$
\begin{align*}
\Diag\Par{\Lambda_{x}s_{x,h}} & =\Diag\Par{\Par{\Sigma_{x}-P_{x}\circ P_{x}}s_{x,h}}\\
 & =\Sigma_{x}\Diag\Par{s_{x,h}}-\Diag\Par{\Par{P_{x}\circ P_{x}}s_{x,h}}\\
 & =\Sigma_{x}S_{x,h}-\Diag\Par{P_{x}S_{x,h}P_{x}},
\end{align*}
where we used Lemma~\ref{lem:Hadamard}-1 in the last line. Thus,
$\Sigma_{x,h}'=2\Par{\Diag\Par{P_{x}S_{x,h}P_{x}}-\Sigma_{x}S_{x,h}}.$
The second items follow from Lemma 49 of \cite{lee2019solving}. From
these formulas and the definition of $\Lambda_{x}$,
\begin{align*}
 & D\Lambda_{x}[h]\\
 & =D\Sigma_{x}[h]-DP_{x}[h]\circ P_{x}-P_{x}\circ DP_{x}[h]\\
 & =-2\Diag(\Lambda_{x}s_{x,h})-\Par{-P_{x}S_{x,h}-S_{x,h}P_{x}+2P_{x}S_{x,h}P_{x}}\circ P_{x}-P_{x}\circ\Par{-P_{x}S_{x,h}-S_{x,h}P_{x}+2P_{x}S_{x,h}P_{x}}\\
 & \underset{\text{(i)}}{=}-2\Diag(\Lambda_{x}s_{x,h})+2P_{x}\circ P_{x}S_{x,h}+2S_{x,h}P_{x}\circ P_{x}-2(P_{x}S_{x,h}P_{x})\circ P_{x}-2P_{x}\circ(P_{x}S_{x,h}P_{x}),
\end{align*}
where in (i) we used $D(A\hada B)=(DA)\circ B=A\hada(DB)$ and $(A\hada B)D=(AD)\hada B=A\circ(BD)$\footnote{This property allows us to write $DA\hada B$ without parenthesis.}
for a diagonal matrix $D\in\Rnn$ (Lemma~\ref{lem:Hadamard}). 

Using the first three formulas
\begin{align*}
 & D^{2}\Sigma_{x}[h,h]\\
 & =-2D\Diag(\Lambda_{x}s_{x,h})[h]\\
 & =-2\Diag(D\Lambda_{x}[h]s_{x,h})+2\Diag\Par{\Lambda_{x}S_{x,h}s_{x,h}}\\
 & =-2\Diag\Par{\Par{-2\Diag(\Lambda_{x}s_{x,h})+2P_{x}\circ P_{x}S_{x,h}+2S_{x,h}P_{x}\circ P_{x}-2(P_{x}S_{x,h}P_{x})\circ P_{x}-2P_{x}\circ(P_{x}S_{x,h}P_{x})}s_{x,h}}\\
 & \qquad+2\Diag\Par{\Lambda_{x}S_{x,h}s_{x,h}}\\
 & =4\Diag\Par{\cred{\Lambda_{x}}s_{x,h}}\cblue{S_{x,h}}-4\Diag\Par{P_{x}\circ P_{x}S_{x,h}s_{x,h}}-4\Diag\Par{S_{x,h}P_{x}\circ P_{x}s_{x,h}}\\
 & \qquad+4\Diag\Par{(P_{x}S_{x,h}P_{x})\circ P_{x}s_{x,h}}+4\Diag\Par{P_{x}\circ(P_{x}S_{x,h}P_{x})s_{x,h}}+2\Diag\Par{\cred{\Lambda_{x}}S_{x,h}s_{x,h}}\\
 & =4\Diag\Par{\cblue{S_{x,h}}\cred{(\Sigma_{x}-P_{x}\circ P_{x})}s_{x,h}}-4\Diag\Par{P_{x}\circ P_{x}S_{x,h}s_{x,h}}-4\Diag\Par{S_{x,h}P_{x}\circ P_{x}s_{x,h}}\\
 & \qquad+4\Diag\Par{(P_{x}S_{x,h}P_{x})\circ P_{x}s_{x,h}}+4\Diag\Par{P_{x}\circ(P_{x}S_{x,h}P_{x})s_{x,h}}+2\Diag\Par{\cred{(\Sigma_{x}-P_{x}\circ P_{x})}S_{x,h}s_{x,h}}\\
 & =\ccyan{4\Diag(S_{x,h}\Sigma_{x}s_{x,h})}-6\Diag\Par{P_{x}\circ P_{x}S_{x,h}s_{x,h}}-8\Diag\Par{S_{x,h}P_{x}\circ P_{x}s_{x,h}}\\
 & \qquad+4\Diag\Par{(P_{x}S_{x,h}P_{x})\circ P_{x}s_{x,h}}+4\Diag\Par{P_{x}\circ(P_{x}S_{x,h}P_{x})s_{x,h}}+\ccyan{2\Diag(\Sigma_{x}S_{x,h}s_{x,h})}\\
 & =\text{\ensuremath{\ccyan{6\Diag(S_{x,h}\Sigma_{x}s_{x,h})}}}-6\Diag\Par{\cblue{P_{x}\circ P_{x}S_{x,h}s_{x,h}}}-8\Diag\Par{\cblue{S_{x,h}P_{x}\circ P_{x}s_{x,h}}}\\
 & \qquad+4\Diag\Par{\cblue{(P_{x}S_{x,h}P_{x})\circ P_{x}s_{x,h}}}+4\Diag\Par{\cblue{P_{x}\circ(P_{x}S_{x,h}P_{x})s_{x,h}}}\\
 & \underset{\text{(i)}}{=}6S_{x,h}\Sigma_{x}\Diag\Par{s_{x,h}}-6\Diag\Par{\diag\Par{P_{x}S_{x,h}(P_{x}S_{x,h})^{\top}}}-8\Diag\Par{\diag\Par{S_{x,h}P_{x}S_{x,h}P_{x}^{\top}}}\\
 & \qquad+4\Diag\Par{P_{x}S_{x,h}P_{x}S_{x,h}P_{x}}+4\Diag\Par{P_{x}S_{x,h}\Par{P_{x}S_{x,h}P_{x}}^{\top}}\\
 & =6S_{x,h}\Sigma_{x}S_{x,h}-6\Diag\Par{P_{x}S_{x,h}^{2}P_{x}}-8\Diag\Par{S_{x,h}P_{x}S_{x,h}P_{x}}+8\Diag\Par{P_{x}S_{x,h}P_{x}S_{x,h}P_{x}},
\end{align*}
where in (i) we applied Lemma~\ref{lem:Hadamard}-1 to the terms
with blue. 

Applying the product rule to $\theta_{1}(x)=A_{x}^{\top}\Sigma_{x}A_{x}=A^{\top}S_{x}^{-2}\Sigma_{x}A,$
\begin{align*}
D\theta_{1}[h] & =-2A^{\top}S_{x}^{-3}\Sigma_{x}\Diag(Ah)A+A^{\top}S_{x}^{-2}D\Sigma_{x}[h]A\\
 & =-2A_{x}^{\top}\Sigma_{x}S_{x,h}A_{x}+A_{x}^{\top}\Sigma_{x,h}'A_{x},\\
D^{2}\theta_{1}[h,h] & =6A_{x}^{\top}S_{x,h}\Sigma_{x}S_{x,h}A_{x}-2A_{x}^{\top}\Sigma_{x,h}'S_{x,h}A_{x}-2A_{x}^{\top}S_{x,h}\Sigma_{x,h}'A_{x}+A_{x}^{\top}\Sigma_{x,h}''A_{x}\\
 & =6A_{x}^{\top}S_{x,h}\Sigma_{x}S_{x,h}A_{x}-4A_{x}^{\top}\Sigma_{x,h}'S_{x,h}A_{x}+A_{x}^{\top}\Sigma_{x,h}''A_{x}.
\end{align*}
By substituting $\Sigma_{x,h}'$ and $\Sigma_{x,h}''$ with our formulas
above, 
\begin{align*}
 & D^{2}\theta_{1}[h,h]\\
 & =6A_{x}^{\top}S_{x,h}\Sigma_{x}S_{x,h}A_{x}-4A_{x}^{\top}\Sigma_{x,h}'S_{x,h}A_{x}+A_{x}^{\top}\Sigma_{x,h}''A_{x}\\
 & =6A_{x}^{\top}S_{x,h}\Sigma_{x}S_{x,h}A_{x}+8A_{x}^{\top}\Par{\Sigma_{x}S_{x,h}-\Diag\Par{P_{x}S_{x,h}P_{x}}}S_{x,h}A_{x}\\
 & \qquad+A_{x}^{\top}\Par{6S_{x,h}\Sigma_{x}S_{x,h}-6\Diag\Par{P_{x}S_{x,h}^{2}P_{x}}-8\Diag\Par{S_{x,h}P_{x}S_{x,h}P_{x}}+8\Diag\Par{P_{x}S_{x,h}P_{x}S_{x,h}P_{x}}}A_{x}\\
 & =20A_{x}^{\top}S_{x,h}\Sigma_{x}S_{x,h}A_{x}-16A_{x}^{\top}\Diag\Par{S_{x,h}P_{x}S_{x,h}P_{x}}A_{x}-6A_{x}^{\top}\Diag\Par{P_{x}S_{x,h}^{2}P_{x}}A_{x}\\
 & \qquad+8A_{x}^{\top}\Diag\Par{P_{x}S_{x,h}P_{x}S_{x,h}P_{x}}A_{x}.
\end{align*}

The derivatives of $\theta_{2}$ simply follow from Claim~\ref{claim:diffLogBarrier}.
\end{proof}
\begin{lem}
\label{lem:HybridGammaNorm} $\norm{\Gamma_{x}}_{\infty}\leq\frac{1}{44}$
for $\Gamma_{x}:=\Diag\Par{A_{x}g(x)^{-1}A_{x}^{\top}}$.
\end{lem}

\begin{proof}
Note that $0\preceq\Gamma_{x}=\Diag\Par{A_{x}g^{-1}A_{x}^{\top}}\preceq\Diag\Par{A_{x}g_{2}^{-1}A_{x}^{\top}}$.
For $\og_{2}:=\theta_{1}+\frac{n}{m}\theta_{2}=\frac{1}{44}\sqrt{\frac{n}{m}}g_{2}$,
we have
\[
\norm{\Diag\Par{A_{x}\og_{2}^{-1}A_{x}^{\top}}}_{\infty}=\max_{i\in[m]}\frac{\sigma\Par{\sqrt{\Sigma_{x}+\frac{n}{m}I}A_{x}}_{i}}{\Par{\Sigma_{x}+\frac{n}{m}I}_{ii}}\underset{\text{\eqref{eq:28-1}}}{\leq}\sqrt{\frac{m}{n}},
\]
and thus
\begin{align*}
\norm{\Gamma_{x}}_{\infty} & \leq\norm{\Diag\Par{A_{x}g_{2}^{-1}A_{x}^{\top}}}_{\infty}=\frac{1}{44}\sqrt{\frac{n}{m}}\norm{\Diag\Par{A_{x}\og_{2}^{-1}A_{x}^{\top}}}_{\infty}\leq\frac{1}{44}.\qedhere
\end{align*}
\end{proof}
Now we are ready to show SLTSC of the Vaidya metric:
\begin{proof}
[Proof of Lemma~\ref{lem:vaidya-SLTSC}] As $D^{2}\theta_{2}(x)[h,h]\succeq0$
by Claim~\ref{claim:diffLogBarrier}, we have 
\[
\tr\Par{g^{-1}D^{2}\theta_{2}(x)[h,h]}=\tr\Par{g^{-\half}D^{2}\theta_{2}(x)[h,h]g^{-\half}}\geq0.
\]
For $\theta_{1}$, Lemma~\ref{lem:calculusLeverage}-6 leads to $D^{2}\theta_{1}[h,h]\succeq-4A_{x}^{\top}\Sigma_{x,h}'S_{x,h}A_{x}+A_{x}^{\top}\Sigma_{x,h}''A_{x}$,
so it suffices to provide
\begin{align*}
\text{Upper bound:} & \ \tr\Par{g^{-1}A_{x}^{\top}\Sigma_{x,h}'S_{x,h}A_{x}}=\tr\Par{A_{x}g^{-1}A_{x}^{\top}\Sigma_{x,h}'S_{x,h}}\underset{\text{(i)}}{=}\tr\Par{\Gamma_{x}\Sigma_{x,h}'S_{x,h}},\\
\text{Lower bound:} & \ \tr\Par{g^{-1}A_{x}^{\top}\Sigma_{x,h}''A_{x}}=\tr\Par{A_{x}g^{-1}A_{x}^{\top}\Sigma_{x,h}''}\underset{\text{(i)}}{=}\tr\Par{\Gamma_{x}\Sigma_{x,h}''},
\end{align*}
where in (i) we used $\tr\Par{AD}=\tr\Par{\Diag(A)D}$ for a diagonal
matrix $D$.

For the upper bound, as diagonal matrices commute,
\begin{align*}
\tr\Par{\Gamma_{x}\Sigma_{x,h}'S_{x,h}} & \leq\norm{\Gamma_{x}}_{\infty}\tr\Par{\Par{S_{x,h}\Sigma_{x,h}'^{2}S_{x,h}}^{\half}}=\norm{\Gamma_{x}}_{\infty}\tr\Par{\Par{\Sigma_{x,h}'\Sigma_{x}^{-1}\Sigma_{x,h}'S_{x,h}\Sigma_{x}S_{x,h}}^{\half}}\\
 & \underset{\text{(i)}}{=}\norm{\Gamma_{x}}_{\infty}\tr\Par{\Par{\Sigma_{x,h}'\Sigma_{x}^{-1}\Sigma_{x,h}'}^{\half}\Par{S_{x,h}\Sigma_{x}S_{x,h}}^{\half}}\\
 & \underset{\text{(ii)}}{\leq}\norm{\Gamma_{x}}_{\infty}\sqrt{\tr\Par{\Sigma_{x,h}'\Sigma_{x}^{-1}\Sigma_{x,h}'}}\sqrt{\tr\Par{S_{x,h}\Sigma_{x}S_{x,h}}}\\
 & =\norm{\Gamma_{x}}_{\infty}\|\Sigma_{x}^{-1}\underbrace{\sigma_{x,h}'}_{:=\diag(D\Sigma_{x}[h])}\|_{\Sigma_{x}}\norm h_{\theta_{1}}\\
 & \underset{\text{(iii)}}{\leq}2\norm{\Gamma_{x}}_{\infty}\norm h_{\theta_{1}}^{2},
\end{align*}
where (i) holds since both $\Sigma_{x,h}'\Sigma_{x}^{-1}\Sigma_{x,h}'$
and $S_{x,h}\Sigma_{x}S_{x,h}$ are PD diagonal matrices, (ii) follows
from the Cauchy-Schwarz inequality, and we used in (iii) $\norm{\Sigma_{x}^{-1}\sigma_{x,h}'}_{\Sigma_{x}}\leq2\norm h_{\theta_{1}}$
(Lemma~\ref{lem:usefulFactLeverage}-3).

For the lower bound, we recall from Lemma~\ref{lem:calculusLeverage}-4
\begin{align*}
\Sigma_{x,h}'' & =6S_{x,h}\Sigma_{x}S_{x,h}+8\Diag\Par{P_{x}S_{x,h}P_{x}S_{x,h}P_{x}}-6\Diag\Par{P_{x}S_{x,h}^{2}P_{x}}-8\Diag\Par{S_{x,h}P_{x}S_{x,h}P_{x}}\\
 & \succeq-6\Diag\Par{P_{x}S_{x,h}^{2}P_{x}}-8\Diag\Par{S_{x,h}P_{x}S_{x,h}P_{x}},
\end{align*}
and thus
\begin{align*}
\tr\Par{\Gamma_{x}\Sigma_{x,h}''} & \geq-6\tr\Par{\Gamma_{x}P_{x}S_{x,h}^{2}P_{x}}-8\tr\Par{\Gamma_{x}S_{x,h}P_{x}S_{x,h}P_{x}}.
\end{align*}
For the first term,
\begin{align}
\tr\Par{\Gamma_{x}P_{x}S_{x,h}^{2}P_{x}} & =\tr\Par{S_{x,h}P_{x}\Gamma_{x}P_{x}S_{x,h}}\leq\norm{\Gamma_{x}}_{\infty}\tr\Par{S_{x,h}P_{x}S_{x,h}}\underset{\text{(i)}}{=}\norm{\Gamma_{x}}_{\infty}s_{x,h}^{\top}\Par{P_{x}\circ I}s_{x,h}\label{eq:trSPS}\\
 & =\norm{\Gamma_{x}}_{\infty}s_{x,h}^{\top}\Sigma_{x}s_{x,h}=\norm{\Gamma_{x}}_{\infty}\norm h_{\theta_{1}}^{2},\nonumber 
\end{align}
where (i) follows from $x^{\top}(A\circ B)y=\tr\Par{\Diag(x)A\Diag(y)B^{\top}}$
(Lemma~\ref{lem:Hadamard}). For the second term,
\begin{align*}
\Abs{\tr\Par{\Gamma_{x}S_{x,h}P_{x}S_{x,h}P_{x}}} & =\Abs{\tr\Par{\Gamma_{x}^{1/2}S_{x,h}P_{x}\cdot S_{x,h}P_{x}\Gamma_{x}^{1/2}}}\\
 & \leq\sqrt{\tr\Par{\Gamma_{x}^{1/2}S_{x,h}P_{x}^{2}S_{x,h}\Gamma_{x}^{1/2}}}\sqrt{\tr\Par{\Gamma_{x}^{1/2}P_{x}S_{x,h}^{2}P_{x}\Gamma_{x}^{1/2}}}\\
 & =\sqrt{\tr\Par{P_{x}S_{x,h}\Gamma_{x}S_{x,h}P_{x}}}\sqrt{\tr\Par{S_{x,h}P_{x}\Gamma_{x}P_{x}S_{x,h}}}\\
 & \leq\norm{\Gamma_{x}}_{\infty}\tr\Par{S_{x,h}P_{x}S_{x,h}}\\
 & =\norm{\Gamma_{x}}_{\infty}\norm h_{\theta_{1}}^{2}.\quad(\text{repeat }(\ref{eq:trSPS}))
\end{align*}
Putting the computations together and using Lemma~\ref{lem:HybridGammaNorm},
\[
\tr\Par{g^{-1}D^{2}\theta_{1}(x)[h,h]}\geq-22\norm{\Gamma_{x}}_{\infty}\norm h_{\theta_{1}}^{2}\geq-\half\norm h_{\theta_{1}}^{2},
\]
and due to $g_{2}=44\sqrt{\frac{m}{n}}\Par{\theta_{1}+\frac{n}{m}\theta_{2}}$,
\[
\tr\Par{g^{-1}D^{2}g_{2}(x)[h,h]}\geq-\half\norm h_{g_{2}}^{2}.\qedhere
\]
\end{proof}

\subsubsection{Linear constraints: strongly lower trace self-concordance of Lewis-weight
\label{proof:linear-Lewis-SLTSC}}

For $\theta(x):=A_{x}^{\top}W_{x}A_{x}$ (i.e., the unscaled version
of $g_{2}$), we write $g_{2}=c\cdot\theta$ for a constant $c$,
which will be set to $c_{1}(\log m)^{c_{2}}\sqrt{n}$ for some constants
$c_{1},c_{2}>0$ later. Going forward, $P_{x}$ indicates the projection
matrix of $W_{x}^{\half-\frac{1}{p}}A_{x}$ (i.e., $P_{x}=P\Par{W_{x}^{\half-\frac{1}{p}}A_{x}}$).

We first bound the largest diagonal entry of $\Gamma_{x}=\Diag\Par{A_{x}g(x)^{-1}A_{x}^{\top}}$.
\begin{lem}
\label{lem:GammaNormLSMetric}$\norm{\Gamma_{x}}_{\infty}\leq2c^{-1}m^{\frac{2}{p+2}}$.
\end{lem}

\begin{proof}
Note that $0\preceq\Gamma_{x}=\Diag\Par{A_{x}g^{-1}A_{x}^{\top}}\preceq c^{-1}\Diag\Par{A_{x}\theta^{-1}A_{x}^{\top}}$.
By Lemma~\ref{lem:usefulFactLewis}-1
\[
\norm{\Diag\Par{A_{x}\theta^{-1}A_{x}^{\top}}}_{\infty}=\max_{i\in[m]}\frac{\sigma\Par{W_{x}^{1/2}A_{x}}_{i}}{\Par{W_{x}}_{ii}}\leq2m^{\frac{2}{p+2}}.\qedhere
\]
\end{proof}
Now we show SLTSC of the Lewis-weight metric:
\begin{proof}
[Proof of Lemma~\ref{lem:Lw-SLTSC}] Repeating the same calculus
done for $A_{x}^{\top}\Sigma_{x}A_{x}$ in Lemma~\ref{lem:calculusLeverage},
we can obtain
\begin{align*}
D^{2}\theta[h,h] & =6A_{x}^{\top}S_{x,h}W_{x}S_{x,h}A_{x}-4A_{x}^{\top}W_{x,h}'S_{x,h}A_{x}+A_{x}^{\top}W_{x,h}''A_{x}\\
 & \succeq-4A_{x}^{\top}W_{x,h}'S_{x,h}A_{x}+A_{x}^{\top}W_{x,h}''A_{x},
\end{align*}
and thus for $\Gamma_{x}=\Diag\Par{A_{x}g^{-1}A_{x}^{\top}}$
\begin{align*}
\tr\Par{g^{-1}D^{2}\theta[h,h]} & =\tr\Par{g^{-\half}D^{2}\theta[h,h]g^{-\half}}\geq\tr\Par{g^{-1}A_{x}^{\top}\Par{W_{x,h}''-4W_{x,h}'S_{x,h}}A_{x}}\\
 & \geq\tr\Par{A_{x}g^{-1}A_{x}^{\top}\Par{W_{x,h}''-4W_{x,h}'S_{x,h}}}=\tr\Par{\Gamma_{x}\Par{W_{x,h}''-4W_{x,h}'S_{x,h}}}\\
 & =-4\tr\Par{\Gamma_{x}W_{x,h}'S_{x,h}}+\tr\Par{\Gamma_{x}W_{x,h}''}.
\end{align*}
For the first term, since diagonal matrices commute,
\begin{align}
\Abs{\tr\Par{\Gamma_{x}W_{x,h}'S_{x,h}}} & \leq\norm{\Gamma_{x}}_{\infty}\tr\Par{\sqrt{S_{x,h}W_{x,h}'^{2}S_{x,h}}}\nonumber \\
 & =\norm{\Gamma_{x}}_{\infty}\tr\Par{\sqrt{W_{x,h}'W_{x}^{-1}W_{x,h}'S_{x,h}W_{x}S_{x,h}}}\nonumber \\
 & =\norm{\Gamma_{x}}_{\infty}\tr\Par{\sqrt{W_{x,h}'W_{x}^{-1}W_{x,h}'}\sqrt{S_{x,h}W_{x}S_{x,h}}}\nonumber \\
 & \underset{\text{(i)}}{\leq}\norm{\Gamma_{x}}_{\infty}\sqrt{\tr\Par{W_{x,h}'W_{x}^{-1}W_{x,h}'}}\sqrt{\tr\Par{S_{x,h}W_{x}S_{x,h}}}\nonumber \\
 & =\norm{\Gamma_{x}}_{\infty}\norm{W_{x}^{-1}w_{x,h}'}_{W_{x}}\norm h_{\theta}\nonumber \\
 & \underset{\text{(ii)}}{\leq}p\norm{\Gamma_{x}}_{\infty}\norm h_{\theta}^{2},\label{eq:boundonFirst}
\end{align}
where we used the Cauchy-Schwarz inequality in (i) and Lemma~\ref{lem:usefulFactLewis}-3
in (ii).

For the second term $\tr\Par{\Gamma_{x}W_{x,h}''}$, let us compute
the second-order directional derivate of $W_{x}$ in direction $h$:
\begin{align*}
W_{x,h}' & =-2\Diag\Par{\Lambda_{x}G_{x}^{-1}W_{x}s_{x,h}}=-2\Diag\Par{\Lambda_{x}\Par{W_{x}-c_{p}\Lambda_{x}}^{-1}W_{x}s_{x,h}}\\
 & =-2\Diag\Par{W_{x}^{\half}\bar{\Lambda}_{x}\Par{I-c_{p}\bar{\Lambda}_{x}}^{-1}W_{x}^{\half}s_{x,h}}=-\Diag\Par{W_{x}^{\half}N_{x}W_{x}^{\half}s_{x,h}}\\
W_{x,h}'' & =-\Diag\Par{\half W_{x}^{-\half}W_{x,h}'N_{x}W_{x}^{\half}s_{x,h}+W_{x}^{\half}N_{x,h}'W_{x}^{\half}s_{x,h}+\half W_{x}^{\half}N_{x}W_{x}^{-\half}W_{x,h}'s_{x,h}}\\
 & \qquad-2\Diag\Par{-\Lambda_{x}G_{x}^{-1}W_{x}S_{x,h}s_{x,h}},
\end{align*}
where $W_{x,h}':=DW_{x}[h]$ and $N_{x,h}':=DN_{x}[h]$.  Thus,
\begin{align}
 & \tr\Par{\Gamma_{x}W_{x,h}''}\nonumber \\
 & =-\half\tr\bigg(\Gamma_{x}\Diag\bigg(\underbrace{W_{x}^{-\half}W_{x,h}'N_{x}W_{x}^{\half}s_{x,h}}_{\text{I}}\bigg)\bigg)-\tr\bigg(\Gamma_{x}\Diag\bigg(\underbrace{W_{x}^{\half}N_{x,h}'W_{x}^{\half}s_{x,h}}_{\text{II}}\bigg)\bigg)\nonumber \\
 & \qquad-\half\tr\bigg(\Gamma_{x}\Diag\bigg(\underbrace{W_{x}^{\half}N_{x}W_{x}^{-\half}W_{x,h}'s_{x,h}}_{\text{III}}\bigg)\bigg)-2\tr\bigg(\Gamma_{x}\Diag\bigg(\underbrace{\Lambda_{x}G_{x}^{-1}W_{x}S_{x,h}s_{x,h}}_{\text{IV}}\bigg)\bigg).\label{eq:trGamma}
\end{align}
Each term is of the form $\tr\Par{\Gamma_{x}\Diag(v)}$ for a vector
$v\in\R^{m}$, and this can be bounded as follows:
\begin{align}
\Abs{\tr\Par{\Gamma_{x}\Diag(v)}} & =\Abs{\tr\Par{\Gamma_{x}W_{x}^{\half}W_{x}^{-\half}\Diag(v)}}\leq\sqrt{\tr\Par{W_{x}^{\half}\Gamma_{x}^{2}W_{x}^{\half}}}\sqrt{\tr\Par{\Diag(v)W_{x}^{-1}\Diag(v)}}\nonumber \\
 & \leq\norm{\Gamma_{x}}_{\infty}\sqrt{\tr\Par{W_{x}}}\norm v_{W_{x}^{-1}}\nonumber \\
 & =\sqrt{n}\norm{\Gamma_{x}}_{\infty}\norm v_{W_{x}^{-1}},\label{eq:trGammaBasic}
\end{align}
where we used $\tr(W_{x})=n$ in the last equality. 

We bound the local norms of the term (I \textasciitilde{} IV). In
our calculation, the operator $\lesssim$ hides universal constants
and poly-logarithmic factors in $m$:
\begin{align*}
\norm{\text{I}}_{W_{x}^{-1}} & =\norm{W_{x}^{-1}W_{x,h}'N_{x}W_{x}^{\half}s_{x,h}}_{2}\leq\underbrace{\norm{W_{x}^{-1}W_{x,h}'}_{2}}_{\text{Use Lemma \ref{lem:LS-comp-tool}-2}}\underbrace{\norm{N_{x}}_{2}}_{\text{Lemma \ref{lem:LS-comp-tool}-1}}\norm{W_{x}^{\half}s_{x,h}}_{2}\\
 & \lesssim p^{2}m^{\frac{1}{p+2}}\norm h_{\theta}\cdot p\cdot\norm h_{\theta}\\
 & =p^{3}m^{\frac{1}{p+2}}\norm h_{\theta}^{2}.
\end{align*}
For the second term,
\begin{align*}
\norm{\text{II}}_{W_{x}^{-1}} & =\norm{N_{x,h}'W_{x}^{\half}s_{x,h}}_{2}=\norm{\Par{I+N_{x}}^{\half}\Par{I+N_{x}}^{-\half}N_{x,h}'\Par{I+N_{x}}^{-\half}\Par{I+N_{x}}^{\half}W_{x}^{\half}s_{x,h}}_{2}\\
 & \leq\underbrace{\norm{I+N_{x}}_{2}}_{\text{Lemma \ref{lem:LS-comp-tool}-1}}\underbrace{\norm{\Par{I+N_{x}}^{-\half}N_{x,h}'\Par{I+N_{x}}^{-\half}}_{2}}_{\text{Lemma \ref{lem:LS-comp-tool}-3}}\norm{W_{x}^{\half}s_{x,h}}_{2}\\
 & \lesssim p^{3.5}\norm h_{\theta}^{2}.
\end{align*}
For the third term,
\begin{align*}
\norm{\text{III}}_{W_{x}^{-1}} & =\norm{N_{x}W_{x}^{-\half}W_{x,h}'s_{x,h}}_{2}\leq\underbrace{\norm{N_{x}}_{2}}_{\text{Lemma \ref{lem:LS-comp-tool}-1}}\underbrace{\norm{W_{x}^{-1}W_{x,h}'}_{2}}_{\text{Lemma \ref{lem:LS-comp-tool}-2}}\norm{W_{x}s_{x,h}}_{2}\\
 & \lesssim p^{3}m^{\frac{1}{p+2}}\norm h_{\theta}^{2}.
\end{align*}
For the last term,
\begin{align*}
\norm{\text{IV}}_{W_{x}^{-1}}^{2} & =s_{x,h}^{\top}S_{x,h}W_{x}G_{x}^{-1}\underbrace{\Lambda_{x}W_{x}^{-1}\Lambda_{x}}_{\preceq W_{x}\ \text{\eqref{eq:lewisBasic-LW}}}G_{x}^{-1}W_{x}S_{x,h}s_{x,h}\\
 & \leq s_{x,h}^{\top}S_{x,h}W_{x}\underbrace{G_{x}^{-1}W_{x}G_{x}^{-1}}_{\preceq\frac{p^{2}}{4}W_{x}^{-1}\ \text{\eqref{eq:lewisBasic-WGW}}}W_{x}S_{x,h}s_{x,h}\\
 & \leq p^{2}s_{x,h}^{\top}S_{x,h}W_{x}S_{x,h}s_{x,h}=p^{2}s_{x,h}^{\top}W_{x}^{\half}S_{x,h}^{2}W_{x}^{\half}s_{x,h}\\
 & \leq p^{2}\norm{s_{x,h}}_{\infty}^{2}\norm h_{\theta}^{2}\\
 & \leq p^{2}m^{\frac{2}{p+2}}\norm h_{\theta}^{4},
\end{align*}
where we used Lemma~\ref{lem:usefulFactLewis}-2 in the last line. 

Combining these bounds, (\ref{eq:trGamma}), and (\ref{eq:trGammaBasic})
for $p=O(\log m)$, we obtain
\begin{align*}
\Abs{\tr\Par{\Gamma_{x}W_{x,h}''}} & \lesssim\sqrt{n}\norm{\Gamma_{x}}_{\infty}\norm h_{\theta}^{2}.
\end{align*}
Along with the bound in (\ref{eq:boundonFirst}), we conclude that
\begin{align*}
\tr\Par{g^{-1}D^{2}\theta[h,h]} & \gtrsim-p\norm{\Gamma_{x}}_{\infty}\norm h_{\theta}^{2}-\sqrt{n}\norm{\Gamma_{x}}_{\infty}\norm h_{\theta}^{2}\\
 & \gtrsim-\sqrt{n}\norm{\Gamma_{x}}_{\infty}\norm h_{\theta}^{2}\\
 & \gtrsim-c^{-1}\sqrt{n}\norm h_{\theta}^{2},
\end{align*}
where the last line follows from Lemma~\ref{lem:GammaNormLSMetric}.
This implies that there exists some positive constants $d_{1}$ and
$d_{2}$ such that $\tr\Par{g^{-1}D^{2}\theta[h,h]}\geq-c^{-1}d_{1}\Par{\log m}^{d_{2}}\sqrt{n}\norm h_{\theta}^{2}$,
which implies
\[
\tr\Par{g^{-1}D^{2}g_{2}[h,h]}\geq-c^{-1}d_{1}\Par{\log m}^{d_{2}}\sqrt{n}\norm h_{g_{2}}^{2}.
\]
By taking $c=d_{1}(\log m)^{d_{2}}\sqrt{n}$, the metric $g_{2}=c\theta=d_{1}(\log m)^{d_{2}}\sqrt{n}A_{x}^{\top}W_{x}A_{x}$
is strongly lower trace self-concordant.
\end{proof}

\subsubsection{Computation of variance of Gaussian polynomials via Stein's lemma}

In this section, we provide computational lemmas used later when establishing
SASC of barriers for linear constraints.
\begin{lem}
\label{lem:variance-1} For $v,w\in\Rn$ and $h\sim\ncal(0,I_{n})$,
\[
\E\Par{v\cdot h}^{3}(w\cdot h)^{3}=9\norm v^{2}\norm w^{2}(v\cdot w)+6(v\cdot w)^{3}.
\]
\end{lem}

\begin{lem}
\label{lem:variance-2} For $p,q,r,s\in\Rn$ and $h\sim\ncal(0,I_{n})$,
\begin{align*}
\E(p\cdot h)^{2}(q\cdot h)(r\cdot h)^{2}(s\cdot h) & =(q\cdot s)\norm p^{2}\norm r^{2}+4(p\cdot r)(p\cdot q)(r\cdot s)\\
 & \quad+2\norm p^{2}(r\cdot q)(r\cdot s)+2\norm r^{2}(p\cdot q)(p\cdot s)\\
 & \quad+2(p\cdot r)^{2}(q\cdot s)+4(p\cdot s)(p\cdot r)(r\cdot q).
\end{align*}
\end{lem}

To prove these, we begin with estimations of expectation of simpler
forms:
\begin{prop}
\label{prop:stein-comp} Let $v,w,p,q,r,s\in\Rn$ and $h\sim\ncal(0,I_{n})$.
\begin{itemize}
\item $\E(v\cdot h)(w\cdot h)^{3}=3\norm w^{2}(v\cdot w)$.
\item $\E(v\cdot h)^{2}(w\cdot h)^{2}=\norm v^{2}\norm w^{2}+2(v\cdot w)^{2}$.
\item $\E(p\cdot h)^{2}(r\cdot h)(s\cdot h)=\norm p^{2}(r\cdot s)+2(p\cdot s)(p\cdot r)$.
\end{itemize}
\end{prop}

\begin{proof}
For the first item,
\begin{align*}
\E(v\cdot h)(w\cdot h)^{3} & \underset{\text{(i)}}{=}\sum_{i}w_{i}\E[h_{i}(v\cdot h)(w\cdot h)^{2}]\\
 & =\sum_{i}w_{i}\Par{v_{i}\E(w\cdot h)^{2}+2w_{i}\E(v\cdot h)(w\cdot h)}\\
 & =(v\cdot w)\norm w^{2}+2\norm w^{2}(v\cdot w)\\
 & =3\norm w^{2}(v\cdot w),
\end{align*}
where in (i) we used Stein's lemma (see Lemma~\ref{lem:stein}) with
$f(h):=(v\cdot h)(w\cdot h)^{2}$.

For the second item,
\begin{align*}
\E(v\cdot h)^{2}(w\cdot h)^{2} & =\sum_{i}v_{i}\E[h_{i}(v\cdot h)(w\cdot h)^{2}]\\
 & \underset{\text{Stein}}{=}\sum_{i}v_{i}\Par{v_{i}\E(w\cdot h)^{2}+2w_{i}\E(v\cdot h)(w\cdot h)}\\
 & =\norm v^{2}\norm w^{2}+2(v\cdot w)^{2}.
\end{align*}

For the third item,
\begin{align*}
\E(p\cdot h)^{2}(r\cdot h)(s\cdot h) & =\sum_{i}p_{i}\E[h_{i}(p\cdot h)(r\cdot h)(s\cdot h)]\\
 & \underset{\text{Stein}}{=}\sum p_{i}\Par{p_{i}\E(r\cdot h)(s\cdot h)+r_{i}\E(p\cdot h)(s\cdot h)+s_{i}\E(p\cdot h)(r\cdot h)}\\
 & =\norm p^{2}(r\cdot s)+(p\cdot r)(p\cdot s)+(p\cdot s)(p\cdot r)\\
 & =\norm p^{2}(r\cdot s)+2(p\cdot s)(p\cdot r).\qedhere
\end{align*}
\end{proof}
Using these estimations, we can easily prove the two computational
lemmas above.
\begin{proof}
[Proof of Lemma~\ref{lem:variance-1}] Using Stein's lemma,
\begin{align*}
\E(v\cdot h)^{3}(w\cdot h)^{3} & =\sum_{i}v_{i}\E[h_{i}(v\cdot h)^{2}(w\cdot h)^{3}]\\
 & =\sum v_{i}\Par{2v_{i}\E(v\cdot h)(w\cdot h)^{3}+3w_{i}\E(v\cdot h)^{2}(w\cdot h)^{2}}\\
 & \underset{\text{(i)}}{=}2\norm v^{2}\cdot3\norm w^{2}(v\cdot w)+3(v\cdot w)\Par{\norm v^{2}\norm w^{2}+2(v\cdot w)^{2}}\\
 & =9\norm v^{2}\norm w^{2}+6(v\cdot w)^{3},
\end{align*}
where in (i) we used Proposition~\ref{prop:stein-comp}-1 and 2.
\end{proof}
%
\begin{proof}
[Proof of Lemma~\ref{lem:variance-2}] Using Stein's lemma,
\begin{align*}
 & \E(p\cdot h)^{2}(q\cdot h)(r\cdot h)^{2}(s\cdot h)\\
 & =\sum_{i}q_{i}\E[h_{i}(p\cdot h)^{2}(r\cdot h)^{2}(s\cdot h)]\\
 & =\sum q_{i}\Par{2p_{i}\E(p\cdot h)(r\cdot h)^{2}(s\cdot h)+2r_{i}\E(p\cdot h)^{2}(r\cdot h)(s\cdot h)+2s_{i}\E(p\cdot h)^{2}(r\cdot h)^{2}}\\
 & \underset{\text{(i)}}{=}2(p\cdot q)\Par{\norm r^{2}(p\cdot s)+2(p\cdot r)(r\cdot s)}+2(r\cdot q)\Par{\norm p^{2}(r\cdot s)+2(p\cdot s)(p\cdot r)}\\
 & \qquad+(q\cdot s)\Par{\norm p^{2}\norm r^{2}+2(p\cdot r)^{2}}\\
 & =(q\cdot s)\norm p^{2}\norm r^{2}+4(p\cdot r)(p\cdot q)(r\cdot s)+2\norm p^{2}(r\cdot q)(r\cdot s)+2\norm r^{2}(p\cdot q)(p\cdot s)\\
 & \qquad+2(p\cdot r)^{2}(q\cdot s)+4(p\cdot s)(p\cdot r)(r\cdot q),
\end{align*}
where in (i) we used Proposition~\ref{prop:stein-comp}-3 to the
first two terms and Proposition~\ref{prop:stein-comp}-2 to the third
term.
\end{proof}

\subsubsection{Linear constraints: strongly average self-concordance of log-barriers
\label{proof:linear-SASC-log}}
\begin{proof}
[Proof of Lemma~\ref{lem:logBarrier-SASC}] Let $g(x)=A_{x}^{\top}A_{x}$
and pick any $g':\intk\to\psd$ such that $\bar{g}:=g+g'\succ0$.
By affine invariance, we may assume $\bar{g}(x)=I$ and $x=0$. Note
that $g(x)\preceq I$. Writing $z=\frac{r}{\sqrt{n}}h$ for $h\sim\ncal(0,I_{n})$
and applying Taylor's expansion to $\norm{z-x}_{g(z)}^{2}$ at $z=x$
(as in the proof of Lemma~\ref{lem:hsc-to-sasc}), for some $p_{z}\in[x,z]$
\begin{align*}
\Abs{\norm{z-x}_{g(z)}^{2}-\norm{z-x}_{g(x)}^{2}} & \leq\frac{r^{3}}{n^{3/2}}\Abs{Dg(x)[h^{\otimes3}]}+\frac{r^{4}}{2n^{2}}\Abs{D^{2}g(p_{z})[h^{\otimes4}]}.\\
 & \leq\frac{r^{2}}{n}\bigg(\underbrace{\frac{r}{\sqrt{n}}\Abs{Dg(x)[h^{\otimes3}]}}_{=:\textsf{A}}+\underbrace{\frac{r^{2}}{2n}\Abs{D^{2}g(p_{z})[h^{\otimes4}]}}_{=:\textsf{B}}\bigg).
\end{align*}
For $S_{x,h}=\Diag(A_{x}h)$ and $p=p_{z}$, Claim~\ref{claim:diffLogBarrier}
leads to
\begin{align*}
Dg(x)[h,h,h] & =-2s_{x,h}^{\top}S_{x,h}s_{x,h}=-2\tr\Par{S_{x,h}^{3}}=:P_{1}(h),\\
D^{2}g(p)[h,h,h,h] & =6s_{p,h}^{\top}S_{p,h}^{4}s_{p,h}=6\tr\Par{S_{p,h}^{4}}=:P_{2}(h).
\end{align*}
Note that $g(x)=A_{x}^{\top}A_{x}\preceq I_{n}$ and $A_{x}A_{x}^{\top}\preceq A_{x}\Par{A_{x}^{\top}A_{x}}^{\dagger}A_{x}^{\top}=P(A_{x})\preceq I_{m}$
(as shown by $P_{x}'\preceq P_{x}$ in the proof of Lemma~\ref{lem:helper4Diagonal}).
Let $a_{i}$ be the $i^{th}$-row of $A_{x}$, which satisfies $\norm{a_{i}}\leq1$
due to $A_{x}^{\top}A_{x}\preceq I$.

\paragraph{Term $\textsf{A}$.}

Using Lemma~\ref{lem:stein}
\begin{align*}
\E P_{1}(h)^{2} & =4\E\Par{\sum_{i=1}^{m}(a_{i}\cdot h)^{3}}^{2}=4\Par{9\sum_{i,j=1}^{m}\norm{a_{i}}^{2}\norm{a_{j}}^{2}(a_{i}\cdot a_{j})+6\sum_{i,j}(a_{i}\cdot a_{j})^{3}}\\
 & =36\cdot1^{\top}\Diag\bigg(\underbrace{A_{x}A_{x}^{\top}}_{\preceq P(A_{x})}\bigg)\underbrace{A_{x}A_{x}^{\top}}_{\preceq I_{m}}\Diag\Par{A_{x}A_{x}^{\top}}1+24\sum_{i,j}(a_{i}\cdot a_{j})^{2}\underbrace{\norm{a_{i}}}_{\leq1}\underbrace{\norm{a_{j}}}_{\leq1}\\
 & \leq36\cdot\tr\Par{\Sigma_{x}^{2}}+24+\sum_{j}\tr\Par{a_{j}^{\top}A_{x}^{\top}A_{x}a_{j}}\\
 & \leq36\cdot\tr\Par{\Sigma_{x}}+24+\underbrace{\sum_{j}\tr\Par{a_{j}^{\top}a_{j}}}_{=\tr\Par{A_{x}A_{x}^{\top}}\leq\tr\Par{\Sigma_{x}}}\\
 & \leq60\rank(A_{x})\\
 & \leq60n.
\end{align*}
Using Lemma~\ref{lem:conc-gaussian-poly} with $t=\max\Par{(2e)^{3/2},\Par{\frac{2e}{3}\log\frac{2}{\veps}}^{3/2}}$
and taking $r_{1}(\veps):=\frac{\veps}{2\sqrt{60}t}$, we have 
\[
\P\Par{\frac{r}{\sqrt{n}}\Abs{Dg(x)[h^{\otimes3}]}\geq\veps}\geq\veps,
\]
 and call this event $B_{1}$.

\paragraph{Term $\textsf{B}$.}

We recall that $\P_{z\sim\ncal_{g+g'}^{r}(x)}\Par{\norm z\geq r\cdot2\log\frac{1}{\veps}}\leq\veps$
and call this event $B_{2}$. We take $r_{2}(\veps)$ so that $1+r_{2}\cdot2\log\frac{1}{\veps}\leq1.1$,
which leads to $\norm z\leq2r$ conditioned on $z\in B_{2}^{c}$ for
$r\leq r_{2}$.

As outline in the proof sketch, we establish coordinate-wise closeness
of slacks at close-by points. Let $x_{t}:=x+t\frac{r}{\sqrt{n}}h$,
and $s_{t}=Ax_{t}-b$. Then for $0\leq t\leq1$
\begin{align*}
\norm{S_{0}^{-1}\frac{d}{dt}s_{t}}_{\infty} & =\frac{r}{\sqrt{n}}\norm{A_{x}h}_{\infty}\leq\frac{r}{\sqrt{n}}\norm h_{A_{x}^{\top}A_{x}}\\
 & \leq\frac{r}{\sqrt{n}}\norm h=\norm z
\end{align*}
and conditioned on $z\in B_{2}^{c}$ we know $\norm z\leq2r\log\frac{1}{\veps}\leq0.1$
for $r\leq r_{2}$. Hence,
\[
\max_{i\in[m]}\Abs{\frac{s_{p,i}-s_{x,i}}{s_{x,i}}}\leq\int_{0}^{1}\norm{S_{0}^{-1}\frac{d}{dt}s_{t}}_{\infty}dt\leq0.1,
\]
and thus $1.2\geq\frac{s_{x,i}}{s_{p,i}}\geq0.9$ for all $i\in[m]$
(i.e., $S_{p}^{-1}\preceq1.2S_{x}^{-1}$).

To use this, we represent $\textsf{B}$ in a quadratic form and replace
$s_{z}$ by $s_{x}$ as follows:
\begin{align*}
\tr\Par{S_{p,h}^{4}} & =\tr\Par{h^{\top}A^{\top}S_{p}^{-1}S_{p,h}^{2}S_{p}^{-1}Ah}=\tr\Par{h^{\top}A^{\top}S_{p,h}S_{p}^{-2}S_{p,h}Ah}\\
 & \leq2\tr\Par{h^{\top}A^{\top}S_{p,h}S_{x}^{-2}S_{p,h}Ah}=2\tr\Par{s_{x,h}^{\top}S_{p,h}^{2}s_{x,h}}=2\tr\Par{S_{x,h}^{2}S_{p,h}^{2}}\\
 & \leq4\tr\Par{S_{x,h}^{4}}.
\end{align*}
Now we establish the concentration of a Gaussian polynomial $\bar{P}_{2}(h):=4\tr\Par{S_{x,h}^{4}}$
(not $P_{2}(h)$):
\begin{align*}
\E\bar{P}_{2}(h)^{2} & =16\E\Par{\tr\Par{S_{x,h}^{4}}}^{2}=16\sum_{i,j}\E(a_{i}\cdot h)^{4}(a_{j}\cdot h)^{4}\\
 & \leq16\sum_{i,j}\Par{\E(a_{i}\cdot h)^{8}\E(a_{j}\cdot h)^{8}}^{\half}\\
 & =\underbrace{16\cdot105}_{=:c_{1}}\sum_{i,j}\norm{a_{i}}^{4}\norm{a_{j}}^{4}=c_{1}\Par{\sum_{i}\norm{a_{i}}^{4}}^{2}\\
 & =c_{1}\Par{\tr\Par{\Par{\Diag\Par{A_{x}A_{x}^{\top}}}^{2}}}^{2}\leq c_{1}\Par{\tr\Par{\Sigma_{x}}}^{2}\\
 & \leq c_{1}n^{2}
\end{align*}
Using Lemma~\ref{lem:conc-gaussian-poly} again with $t=\max\Par{(2e)^{2},\Par{\frac{2e}{4}\log\frac{2}{\veps}}^{3/2}}$
and taking $r_{3}(\veps):=\sqrt{\frac{\veps}{c_{1}t}}$, we have $\P\Par{\frac{r^{2}}{2n}\cdot16\bar{P}_{2}(h)\geq\veps}\geq\veps$
and call this event $B_{3}$. On $B_{3}^{c}$, we clearly have $(0\leq)\frac{r^{2}}{2n}P_{2}(h)\leq\veps$.

Putting all these together, conditioned on $\cap_{i}B_{i}^{c}$ it
holds that for any $r\leq\min_{i}r_{i}(\veps)$ we have with probability
at least $1-3\veps$
\[
\Abs{\norm{z-x}_{g(z)}^{2}-\norm{z-x}_{g(x)}^{2}}\leq2\veps\frac{r^{2}}{n}.
\]
By replacing $3\veps\gets\veps$, the claim follows.
\end{proof}

\subsubsection{Linear constraints: strongly average self-concordance of Vaidya \label{proof:linear-SASC-vaidya}}

Establishing SASC of the Vaidya metric is more involved than that
of the log-barrier.
\begin{proof}
[Proof of Lemma~\ref{lem:vaidya-SASC}] Let $g(x)=\sqrt{\frac{m}{n}}A_{x}^{\top}(\Sigma_{x}+\frac{n}{m}I)A_{x}$
and pick $g':\intk\to\psd$ such that $\bar{g}:=g+g'$ is PD. WMA
$x=0$ and $\bar{g}(x)=I$. For $z\sim\ncal\Par{x,\frac{r^{2}}{n}\bar{g}(x)^{-1}}$
and $h\sim\ncal(0,I_{n})$, this can be written as $z=\frac{r}{\sqrt{n}}h.$
Similar to above, it holds that for some $p_{z}\in[x,z]$
\begin{align*}
\Abs{\norm{z-x}_{g(z)}^{2}-\norm{z-x}_{g(x)}^{2}} & \leq\frac{r^{3}}{n^{3/2}}\Abs{Dg(x)[h^{\otimes3}]}+\frac{r^{4}}{2n^{2}}\Abs{D^{2}g(p_{z})[h^{\otimes4}]}.\\
 & \leq\frac{r^{2}}{n}\bigg(\underbrace{\frac{r}{\sqrt{n}}\Abs{Dg(x)[h^{\otimes3}]}}_{=:\textsf{A}}+\underbrace{\frac{r^{2}}{2n}\Abs{D^{2}g(p_{z})[h^{\otimes4}]}}_{=:\textsf{B}}\bigg).
\end{align*}
We show $\Abs{Dg(x)[h,h,h]}=O(\sqrt{n})$ and $\Abs{D^{2}g(p)[h,h,h,h]}=O(n)$
with high probability.

\paragraph{Term $\textsf{A}$.}

By (\ref{eq:Dgh}), for $g(x)=A_{x}^{\top}D_{x}A_{x}$ and $V_{x}=S_{x}^{-1}D_{x}S_{x}^{-1}$
\begin{align*}
Dg(x)[h,h,h] & =s_{x,h}^{\top}D_{x}\Par{-2S_{x,h}+D_{x}^{-1}D_{x,h}'}s_{x,h}=-2s_{x,h}^{\top}D_{x}S_{x,h}s_{x,h}+s_{x,h}^{\top}D_{x,h}'s_{x,h}
\end{align*}
For $D_{x}=\sqrt{\frac{m}{n}}\Par{\Sigma_{x}+\frac{n}{m}I}$, we have
$D_{x,h}'=\sqrt{\frac{m}{n}}\Sigma_{x,h}'$ and
\begin{align}
Dg(x)[h,h,h] & =-2\sqrt{\frac{m}{n}}s_{x,h}^{\top}\Par{\Sigma_{x}+\frac{n}{m}I}S_{x,h}s_{x,h}+\sqrt{\frac{m}{n}}s_{x,h}^{\top}\Sigma_{x,h}'s_{x,h}\label{eq:vaidya-Dgh3}
\end{align}
Note that $M_{x}:=D_{x}^{1/2}A_{x}$ satisfies $M_{x}^{\top}M_{x}\preceq\bar{g}(x)=I_{n}$
and thus 
\begin{align*}
M_{x}M_{x}^{\top} & =D_{x}^{1/2}A_{x}A_{x}^{\top}D_{x}^{1/2}\preceq P\Par{D_{x}^{1/2}A_{x}}.
\end{align*}

For the first term in (\ref{eq:vaidya-Dgh3}), we define $P_{1}(h):=\sqrt{\frac{m}{n}}s_{x,h}^{\top}\Par{\Sigma_{x}+\frac{n}{m}I}S_{x,h}s_{x,h}=\tr\Par{D_{x}S_{x,h}^{3}}$
and show that $\E(P_{1}(h))^{2}\lesssim n.$ 

We have $\max_{i}\norm{a_{i}}^{2}\leq1$, due to 
\begin{align*}
\norm{\Diag(A_{x}^{\top}A_{x})}_{\infty} & =\max_{i\in[m]}e_{i}^{\top}A_{x}^{\top}A_{x}e_{i}\leq\max_{i\in[m]}e_{i}^{\top}A_{x}^{\top}(A_{x}^{\top}D_{x}A_{x})^{-1}A_{x}e_{i}\\
 & \leq\max_{i}\frac{\sigma_{i}(\sqrt{D_{x}}A_{x})}{(D_{x})_{ii}}\leq1\quad\text{(see \eqref{eq:28-1})}
\end{align*}
By Lemma~\ref{lem:variance-1},
\begin{align*}
\E P_{1}(h)^{2} & =\E\Par{\sum_{i=1}^{m}d_{i}(a_{i}\cdot h)^{3}}^{2}\\
 & \lesssim1^{\top}\Diag\bigg(A_{x}A_{x}^{\top}\bigg)D_{x}^{1/2}\underbrace{D_{x}^{1/2}A_{x}A_{x}^{\top}D_{x}^{1/2}}_{\preceq P(D_{x}^{1/2}A_{x})\preceq I_{m}}D_{x}^{1/2}\Diag\Par{A_{x}A_{x}^{\top}}1+\sum_{i,j}d_{i}d_{j}(a_{i}\cdot a_{j})^{2}\underbrace{\norm{a_{i}}}_{\leq1}\underbrace{\norm{a_{j}}}_{\leq1}\\
 & \lesssim\tr\Par{\Diag\bigg(A_{x}A_{x}^{\top}\bigg)D_{x}\Diag\Par{A_{x}A_{x}^{\top}}}+\sum_{i,j}d_{i}d_{j}(a_{i}\cdot a_{j})^{2}\\
 & \lesssim\tr\Par{\Diag\bigg(A_{x}A_{x}^{\top}\bigg)D_{x}}\norm{\Diag\Par{A_{x}A_{x}^{\top}}}_{\infty}+\sum_{j}\tr\Par{d_{j}a_{j}^{\top}A_{x}^{\top}D_{x}A_{x}a_{j}}\\
 & \leq\tr\Par{A_{x}^{\top}D_{x}A_{x}}+\underbrace{\sum_{j}\tr\Par{d_{j}a_{j}^{\top}a_{j}}}_{=\tr\Par{A_{x}^{\top}D_{x}A_{x}}}\\
 & \leq2n
\end{align*}

For the second term in (\ref{eq:vaidya-Dgh3}), it suffices to show
$\E(\bar{P}_{2}(h))^{2}\lesssim n$ for $\bar{P}_{2}(h):=\sqrt{\frac{m}{n}}s_{x,h}^{\top}\Sigma_{x,h}'s_{x,h}$.
As $\Sigma_{x,h}'=-2\Diag\Par{\Par{\Sigma_{x}-P_{x}^{(2)}}s_{x,h}}$
(Lemma~(\ref{lem:calculusLeverage})), 
\begin{align*}
\Abs{\bar{P}_{2}(h)} & =2\Abs{\sqrt{\frac{m}{n}}\tr\Par{\Diag\Par{\Par{\Sigma_{x}-P_{x}^{(2)}}s_{x,h}}S_{x,h}^{2}}}\\
 & \leq2\sqrt{\frac{m}{n}}\Par{\Abs{\tr\Par{\Sigma_{x}S_{x,h}^{3}}}+\Abs{\tr\Par{\Diag(P_{x}^{(2)}s_{x,h})S_{x,h}^{2}}}}
\end{align*}
We have already bounded the first term when handling $\E P_{1}(h)^{2}$,
so let us focus on the second term. For $\sigma_{x}:=\diag\Par{P_{x}}$
and $\sigma_{x,i,j}:=(P_{x})_{ij}$, it follows from $P_{x}^{2}=P_{x}$
that $\sigma_{x,i}=\sum_{j}\sigma_{x,i,j}^{2}$. Hence 
\begin{align*}
\tr\Par{\Sigma_{x}S_{x,h}^{3}} & =1^{\top}\Sigma_{x}s_{x,h}^{3}=\sum_{i}(s_{x,h})_{i}^{3}\sigma_{x,i}=\sum_{i,j=1}^{m}\sigma_{x,i,j}^{2}(s_{x,h})_{i}^{3},\\
\tr\Par{\Diag(P_{x}^{(2)}s_{x,h})S_{x,h}^{2}} & =\sum_{i,j=1}^{m}\sigma_{x,i,j}^{2}(s_{x,h})_{i}^{2}(s_{x,h})_{j}\underset{\text{symmetry}}{=}\sum_{i,j=1}^{m}\sigma_{x,i,j}^{2}(s_{x,h})_{j}^{2}(s_{x,h})_{i},
\end{align*}
and combining these leads to
\begin{align*}
 & 2\tr\Par{\Sigma_{x}S_{x,h}^{3}}+6\tr\Par{\Diag(P_{x}^{(2)}s_{x,h})S_{x,h}^{2}}\\
 & =\sum_{i,j=1}^{m}\sigma_{x,i,j}^{2}\Par{(s_{x,h})_{i}^{3}+3(s_{x,h})_{i}^{2}(s_{x,h})_{j}+3(s_{x,h})_{i}(s_{x,h})_{j}^{2}+(s_{x,h})_{j}^{3}}=\sum_{i,j=1}^{m}\sigma_{x,i,j}^{2}\Par{(s_{x,h})_{i}+(s_{x,h})_{j}}^{3}.
\end{align*}
Thus it suffices to focus on the concentration of $\sum_{i,j=1}^{m}\sigma_{x,i,j}^{2}\Par{(s_{x,h})_{i}+(s_{x,h})_{j}}^{3}$
instead of that of $\tr\Par{\Diag(P_{x}^{(2)}s_{x,h})S_{x,h}^{2}}$,
as the concentration of $\tr\Par{\Sigma_{x}S_{x,h}^{3}}$ was already
studied.

Denoting the $i^{th}$-row of $A_{x}$ by $a_{i}$,
\begin{align}
 & \E\Par{\sum_{i,j=1}^{m}\sigma_{x,i,j}^{2}\Par{(s_{x,h})_{i}+(s_{x,h})_{j}}^{3}}^{2}\nonumber \\
 & =\E\Par{\sum_{i,j=1}^{m}\sigma_{x,i,j}^{2}\Par{\underbrace{(a_{i}+a_{j})}_{=:c_{ij}}\cdot h}^{3}}^{2}=\sum_{i,j,k,l}\sigma_{x,i,j}^{2}\sigma_{x,k,l}^{2}\E(c_{ij}\cdot h)^{3}(c_{kl}\cdot h)^{3}\nonumber \\
 & =9\sum_{i,j,k,l}\sigma_{x,i,j}^{2}\sigma_{x,k,l}^{2}\norm{c_{ij}}^{2}\norm{c_{kl}}^{2}(c_{ij}\cdot c_{kl})+6\sum_{i,j,k,l}\sigma_{x,i,j}^{2}\sigma_{x,k,l}^{2}(c_{ij}\cdot c_{kl})^{3}.\label{eq:vaidya-cubic-expansion}
\end{align}
For the first term in (\ref{eq:vaidya-cubic-expansion}),
\begin{align*}
 & \sum_{i,j,k,l}\sigma_{x,i,j}^{2}\sigma_{x,k,l}^{2}\norm{c_{ij}}^{2}\norm{c_{kl}}^{2}(c_{ij}\cdot c_{kl})\\
 & =\norm{\sum_{ij}\sigma_{x,i,j}^{2}\norm{c_{ij}}^{2}c_{ij}}^{2}\leq2\norm{\sum_{ij}\sigma_{x,i,j}^{2}\norm{c_{ij}}^{2}a_{i}}^{2}+2\norm{\sum_{ij}\sigma_{x,i,j}^{2}\norm{c_{ij}}^{2}a_{j}}^{2}\\
 & =4\norm{\sum_{ij}\sigma_{x,i,j}^{2}\norm{c_{ij}}^{2}a_{i}}^{2}.
\end{align*}
Then
\begin{align*}
 & \norm{\sum_{ij}\sigma_{x,i,j}^{2}\norm{c_{ij}}^{2}a_{i}}^{2}\\
 & =\norm{\sum_{i}\underbrace{\sum_{j}\sigma_{x,i,j}^{2}\norm{c_{ij}}^{2}}_{=:z_{i}}a_{i}}^{2}=1^{\top}\Diag\Par{A_{x}A_{x}^{\top}}ZA_{x}A_{x}^{\top}Z\Diag(A_{x}A_{x}^{\top})1\quad(Z:=\Diag\Par{(z_{i})_{i\in[m]}}\\
 & \leq1^{\top}\Diag\Par{A_{x}A_{x}^{\top}}ZD_{x}^{-1}Z\Diag\Par{A_{x}A_{x}^{\top}}1\leq\tr\Par{\Diag\Par{A_{x}A_{x}^{\top}}ZD_{x}^{-1}Z}\norm{\Diag\Par{A_{x}A_{x}^{\top}}}_{\infty}\\
 & \lesssim\tr\Par{A_{x}^{\top}ZD_{x}^{-1}ZA_{x}}
\end{align*}
Note that $Z\precsim\Sigma_{x}$ due to
\begin{align*}
z_{i} & =\sum_{j}\sigma_{x,i,j}^{2}\norm{c_{ij}}^{2}\leq2\sum_{j}\sigma_{x,i,j}^{2}(\norm{a_{i}}^{2}+\norm{a_{j}}^{2})\\
 & \lesssim\sigma_{x,i}\norm{a_{i}}^{2}+\sum_{j}\sigma_{x,i,j}^{2}\norm{a_{j}}^{2}\leq\sigma_{x,i}\norm{a_{i}}^{2}+\sigma_{x,i}\\
 & \lesssim\sigma_{x,i}.
\end{align*}
Hence 
\begin{align*}
\tr\Par{A_{x}^{\top}ZD_{x}^{-1}ZA_{x}} & \leq\sqrt{\frac{n}{m}}\tr\Par{A_{x}^{\top}ZA_{x}}\leq\frac{n}{m}\tr\Par{A_{x}^{\top}D_{x}A_{x}}\leq\frac{n^{2}}{m}.
\end{align*}

For the second term in (\ref{eq:vaidya-cubic-expansion}),
\begin{align*}
\sum_{i,j,k,l}\sigma_{x,i,j}^{2}\sigma_{x,k,l}^{2}(c_{ij}\cdot c_{kl})^{3} & \lesssim\sum_{i,j,k,l}\sigma_{x,i,j}^{2}\sigma_{x,k,l}^{2}\Abs{c_{ij}\cdot c_{kl}}^{2}\\
 & \leq\sum_{i,j,k,l}\sigma_{x,i,j}^{2}\sigma_{x,k,l}^{2}\Par{a_{i}\cdot a_{k}+a_{i}\cdot a_{l}+a_{j}\cdot a_{k}+a_{j}\cdot a_{l}}^{2}\\
 & \lesssim\sum_{i,j,k,l}\sigma_{x,i,j}^{2}\sigma_{x,k,l}^{2}(a_{i}\cdot a_{k})^{2}=\sum_{ik}\sigma_{i}\sigma_{k}(a_{i}\cdot a_{k})^{2}\\
 & =\sum_{k}\tr\Par{\sigma_{k}a_{k}^{\top}A_{x}^{\top}\Sigma_{x}A_{x}a_{k}}\\
 & \leq\sqrt{\frac{n}{m}}\sum_{k}\tr\Par{\sigma_{k}a_{k}^{\top}a_{k}}=\sqrt{\frac{n}{m}}\tr\Par{A_{x}^{\top}\Sigma_{x}A_{x}}\\
 & \le\frac{n^{2}}{m}.
\end{align*}

\paragraph{Term $\textsf{B}$.}

Now we control $\Abs{D^{4}g(p_{z})[h,h,h,h]}$. 

We first show that $s_{x}$ and $s_{p_{z}}$ are close, and the same
holds for $\sigma_{x}$ and $\sigma_{p_{z}}$. For $s_{x}$, following
our argument for the log-barrier case, we let $x_{t}:=x+t\frac{r}{\sqrt{n}}h$
and $s_{t}:=Ax_{t}-b$. Then for $0\leq t\leq1$
\begin{align*}
\norm{S_{0}^{-1}\frac{d}{dt}s_{t}}_{\infty} & =\frac{r}{\sqrt{n}}\norm{A_{x}h}_{\infty}\underset{\eqref{eq:28-1}}{\leq}\frac{r}{\sqrt{n}}\norm h_{A_{x}^{\top}D_{x}A_{x}}\leq\frac{r}{\sqrt{n}}\norm h=\norm z
\end{align*}
and conditioned on $z\in B_{2}^{c}$ we know $\norm z\leq2r\log\frac{1}{\veps}\leq0.1$
for $r\leq r_{2}$. Hence,
\[
\max_{i\in[m]}\Abs{\frac{s_{p,i}-s_{x,i}}{s_{x,i}}}\leq\int_{0}^{1}\norm{S_{0}^{-1}\frac{d}{dt}s_{t}}_{\infty}dt\leq0.1,
\]
and thus $1.2\geq\frac{s_{x,i}}{s_{p,i}}\geq0.9$ for all $i\in[m]$
(i.e., $S_{p}^{-1}\preceq1.2S_{x}^{-1}$). For $\sigma_{x}$, as we
have $\Sigma_{x}=\Diag\Par{A_{x}\Par{A_{x}^{\top}A_{x}}^{-1}A_{x}^{\top}}$,
we also have the same closeness between $\sigma_{x,i}$ and $\sigma_{p,i}$
for each $i\in[m]$.

Using the formulas of $D^{2}\theta_{1}[h,h]$ and $D^{2}\theta_{2}[h,h]$
in Lemma~\ref{lem:calculusLeverage}

\begin{align*}
 & \Abs{D^{2}g(p)[(z-x)^{\otimes4}]}=\Abs{\frac{r^{4}}{n^{2}}D^{2}g(p)[h^{\otimes4}]}\\
 & \lesssim\frac{r^{4}}{n^{2}}\sqrt{\frac{m}{n}}\bigg(\tr\Par{\Par{\Sigma_{p}+\frac{n}{m}I}S_{p,h}^{4}}+\underbrace{\tr\Par{S_{p,h}^{2}P_{p}S_{p,h}P_{p}S_{p,h}}}_{\textsf{I}}\\
 & \qquad\qquad\qquad+\tr\Par{S_{p,h}^{2}P_{p}S_{p,h}^{2}P_{p}}+\underbrace{\tr\Par{S_{p,h}P_{p}S_{p,h}P_{p}S_{p,h}P_{p}S_{p,h}}}_{\leq\tr\Par{S_{p,h}^{2}P_{p}S_{p,h}^{2}P_{p}}}\bigg)\\
 & \underset{\text{(i)}}{\lesssim}\frac{r^{4}}{n^{2}}\sqrt{\frac{m}{n}}\bigg(\tr\Par{\Par{\Sigma_{p}+\frac{n}{m}I}S_{p,h}^{4}}+\underbrace{\tr\Par{S_{p,h}^{2}\Sigma_{p}S_{p,h}^{2}}}_{\leq\tr\Par{\Par{\Sigma_{p}+\frac{n}{m}I}S_{p,h}^{4}}}+\underbrace{\tr\Par{S_{p,h}^{2}P_{p}S_{p,h}^{2}P_{p}}}_{\text{Use Lemma \ref{lem:Kronecker}}}\bigg)\\
 & \leq\frac{r^{4}}{n^{2}}\sqrt{\frac{m}{n}}\bigg(\tr\Par{\Par{\Sigma_{p}+\frac{n}{m}I}S_{p,h}^{4}}+s_{p,h}^{2}\cdot P_{p}^{(2)}s_{p,h}^{2}\bigg)\\
 & \leq\frac{r^{4}}{n^{2}}\sqrt{\frac{m}{n}}\bigg(\tr\Par{\Par{\Sigma_{p}+\frac{n}{m}I}S_{p,h}^{4}}+s_{p,h}^{2}\cdot\Sigma_{p}s_{p,h}^{2}\bigg)\\
 & \underset{\text{(ii)}}{\lesssim}\frac{r^{4}}{n^{2}}\sqrt{\frac{m}{n}}\tr\Par{\Par{\Sigma_{x}+\frac{n}{m}I}S_{x,h}^{4}}\\
 & =\frac{r^{4}}{n^{2}}\tr\Par{D_{x}S_{x,h}^{4}}
\end{align*}
where in (i) we used the Cauchy-Schwarz inequality on $\textsf{I}$,
obtaining 
\begin{align*}
\tr\Par{S_{p,h}^{2}P_{p}S_{p,h}P_{p}S_{p,h}} & \leq\sqrt{\tr\Par{S_{p,h}^{2}P_{p}^{2}S_{p,h}^{2}}}\sqrt{\tr\Par{S_{p,h}P_{p}S_{p,h}^{2}P_{p}S_{p,h}}}\\
 & \underset{\text{AM-GM}}{\leq}\frac{\tr\Par{S_{p,h}^{2}P_{p}^{2}S_{p,h}^{2}}+\tr\Par{S_{p,h}P_{p}S_{p,h}^{2}P_{p}S_{p,h}}}{2}\\
 & \leq\half\Par{\tr\Par{S_{p,h}^{2}\Sigma_{p}S_{p,h}^{2}}+\tr\Par{S_{p,h}^{2}P_{p}S_{p,h}^{2}P_{p}}}.
\end{align*}
and in (ii) we used coordinate-wise closeness of $s_{x}\leftrightarrow s_{p}$
and $\sigma_{x}\leftrightarrow\sigma_{p}$. Following our calculation
for the log-barrier case,
\begin{align*}
\E\Par{\tr\Par{D_{x}S_{x,h}^{4}}}^{2} & \lesssim\Par{\tr\Par{A_{x}A_{x}^{\top}\Diag\Par{D_{x}^{\half}A_{x}A_{x}^{\top}D_{x}^{\half}}}}^{2}\\
 & \leq\norm{A_{x}A_{x}^{\top}}_{\infty}^{2}\Par{\tr\Par{\Diag\Par{D_{x}^{\half}A_{x}A_{x}^{\top}D_{x}^{\half}}}}^{2}\\
 & \leq\Par{\tr\Par{A_{x}^{\top}D_{x}A_{x}}}^{2}\leq n^{2}.\qedhere
\end{align*}
\end{proof}

\subsubsection{Linear constraints: strongly average self-concordance of Lewis-weight
\label{proof:linear-SASC-Lw}}
\begin{proof}
[Proof of Lemma~\ref{lem:Lw-SASC}] Let $g(x)=\sqrt{n}A_{x}^{\top}W_{x}A_{x}$
and pick any $g':\intk\to\psd$ such that $\bar{g}:=g+g'$ is PD.
WMA $x=0$ and $\bar{g}(x)=I$. For $z\sim\ncal\Par{x,\frac{r^{2}}{n}\bar{g}(x)^{-1}}$
and $h\sim\ncal(0,I_{n})$, we have $z=\frac{r}{\sqrt{n}}h.$ Similar
to above, it holds that for some $p_{z}\in[x,z]$
\begin{align*}
\Abs{\norm{z-x}_{g(z)}^{2}-\norm{z-x}_{g(x)}^{2}} & \leq\frac{r^{3}}{n^{3/2}}\Abs{Dg(x)[h^{\otimes3}]}+\frac{r^{4}}{2n^{2}}\Abs{D^{2}g(p_{z})[h^{\otimes4}]}.\\
 & \leq\frac{r^{2}}{n}\bigg(\underbrace{\frac{r}{\sqrt{n}}\Abs{Dg(x)[h^{\otimes3}]}}_{=:\textsf{A}}+\underbrace{\frac{r^{2}}{2n}\Abs{D^{2}g(p_{z})[h^{\otimes4}]}}_{=:\textsf{B}}\bigg).
\end{align*}
We show $\Abs{Dg(x)[h,h,h]}=O(\sqrt{n})$ and $\Abs{D^{2}g(p)[h,h,h,h]}=O(n)$
with high probability.

\paragraph{Term $\textsf{A}$.}

By (\ref{eq:28-1}), for $g(x)=A_{x}^{\top}D_{x}A_{x}$ and $V_{x}=S_{x}^{-1}D_{x}S_{x}^{-1}$
\begin{align*}
Dg(x)[h,h,h] & =s_{x,h}^{\top}D_{x}\Par{-2S_{x,h}+D_{x}^{-1}D_{x,h}'}s_{x,h}=-2s_{x,h}^{\top}D_{x}S_{x,h}s_{x,h}+s_{x,h}^{\top}D_{x,h}'s_{x,h}.
\end{align*}
For $D_{x}=\sqrt{n}W_{x}$, we have $D_{x,h}'=\sqrt{n}W_{x,h}'$,
so 
\begin{align}
Dg(x)[h,h,h] & =-2\sqrt{n}s_{x,h}^{\top}W_{x}S_{x,h}s_{x,h}+\sqrt{n}s_{x,h}^{\top}W_{x,h}'s_{x,h}\label{eq:Lw-Dgh3}
\end{align}
Note that $M_{x}:=D_{x}^{1/2}A_{x}$ satisfies $M_{x}^{\top}M_{x}\preceq\bar{g}(x)=I_{n}$
and thus 
\begin{align*}
M_{x}M_{x}^{\top} & =D_{x}^{1/2}A_{x}A_{x}^{\top}D_{x}^{1/2}\preceq P\Par{D_{x}^{1/2}A_{x}}.
\end{align*}

For the first term in (\ref{eq:Lw-Dgh3}), we define $P_{1}(h):=\sqrt{n}s_{x,h}^{\top}W_{x}S_{x,h}s_{x,h}=\tr\Par{D_{x}S_{x,h}^{3}}$
and show that $\E(P_{1}(h))^{2}\lesssim n.$

We have $\max_{i}\norm{a_{i}}^{2}\lesssim\frac{1}{\sqrt{n}}$ due
to
\begin{align*}
\norm{\Diag(A_{x}^{\top}A_{x})}_{\infty} & =\max_{i\in[m]}e_{i}^{\top}A_{x}^{\top}A_{x}e_{i}\leq\max_{i\in[m]}e_{i}^{\top}A_{x}^{\top}(A_{x}^{\top}D_{x}A_{x})^{-1}A_{x}e_{i}\\
 & \leq\max_{i}\frac{\sigma_{i}(\sqrt{D_{x}}A_{x})}{(D_{x})_{ii}}\leq\frac{2m^{\frac{2}{p+2}}}{\sqrt{n}}\quad\text{(Lemma \ref{lem:usefulFactLewis}-1)}\\
 & \lesssim\frac{1}{\sqrt{n}}
\end{align*}
By Lemma~\ref{lem:variance-1}
\begin{align*}
 & \E P_{1}(h)^{2}=\E\Par{\sum_{i=1}^{m}d_{i}(a_{i}\cdot h)^{3}}^{2}\\
 & \lesssim1^{\top}\Diag\bigg(A_{x}A_{x}^{\top}\bigg)D_{x}^{1/2}\underbrace{D_{x}^{1/2}A_{x}A_{x}^{\top}D_{x}^{1/2}}_{\preceq P(D_{x}^{1/2}A_{x})\preceq I_{m}}D_{x}^{1/2}\Diag\Par{A_{x}A_{x}^{\top}}1+\sum_{i,j}d_{i}d_{j}(a_{i}\cdot a_{j})^{2}\underbrace{\norm{a_{i}}}_{\leq n^{-1/4}}\underbrace{\norm{a_{j}}}_{\leq n^{-1/4}}\\
 & \lesssim\tr\Par{\Diag\bigg(A_{x}A_{x}^{\top}\bigg)D_{x}\Diag\Par{A_{x}A_{x}^{\top}}}+\frac{1}{\sqrt{n}}\sum_{i,j}d_{i}d_{j}(a_{i}\cdot a_{j})^{2}\\
 & \lesssim\tr\Par{\Diag\bigg(A_{x}A_{x}^{\top}\bigg)D_{x}}\norm{\Diag\Par{A_{x}A_{x}^{\top}}}_{\infty}+\frac{1}{\sqrt{n}}\sum_{j}\tr\Par{d_{j}a_{j}^{\top}A_{x}^{\top}D_{x}A_{x}a_{j}}\\
 & \leq\frac{1}{\sqrt{n}}\tr\Par{A_{x}^{\top}D_{x}A_{x}}+\frac{1}{\sqrt{n}}\underbrace{\sum_{j}\tr\Par{d_{j}a_{j}^{\top}a_{j}}}_{=\tr\Par{A_{x}^{\top}D_{x}A_{x}}}\\
 & \leq2\sqrt{n}
\end{align*}

For the second term in (\ref{eq:Lw-Dgh3}), we define $\bar{P}_{2}(h):=\sqrt{n}s_{x,h}^{\top}W_{x,h}'s_{x,h}$
and show that $\E(\bar{P}_{2}(h))^{2}\lesssim n.$ Recall from Lemma~\ref{lem:DWh}
that $W_{x,h}'=-\Diag\Par{W_{x}^{\half}N_{x}W_{x}^{\half}s_{x,h}}$,
and thus 
\begin{align*}
\bar{P}_{2}(h) & =\sqrt{n}\tr\Par{\Diag\Par{W_{x}^{\half}N_{x}W_{x}^{\half}s_{x,h}}S_{x,h}^{2}}=\sqrt{n}\tr\Par{\Diag\Par{N_{x}W_{x}^{\half}s_{x,h}}W_{x}^{\half}S_{x,h}^{2}}\\
 & =\sqrt{n}\sum_{i=1}^{m}w_{i}^{1/2}(a_{i}\cdot h)^{2}(b_{i}\cdot h),
\end{align*}
where $b_{i}$ is the $i^{th}$-row of $B:=N_{x}W_{x}^{\half}A_{x}$
for $i=1,\dots,m$. By Lemma~\ref{lem:variance-2}
\begin{align*}
 & \E\Par{\sum_{i=1}^{m}w_{i}^{1/2}(a_{i}\cdot h)^{2}(b_{i}\cdot h)}^{2}\\
 & =\sum_{i,j\in[m]}w_{i}^{1/2}w_{j}^{1/2}\norm{a_{i}}^{2}\norm{a_{j}}^{2}(b_{i}\cdot b_{j})\\
 & \quad+4\sum_{i,j\in[m]}w_{i}^{1/2}w_{j}^{1/2}(a_{i}\cdot a_{j})(a_{i}\cdot b_{i})(a_{j}\cdot b_{j})+4\sum_{i,j\in[m]}w_{i}^{1/2}w_{j}^{1/2}\norm{a_{i}}^{2}(b_{i}\cdot a_{j})(a_{j}\cdot b_{j})\\
 & \quad+2\underbrace{\sum_{i,j\in[m]}w_{i}^{1/2}w_{j}^{1/2}(a_{i}\cdot a_{j})^{2}(b_{i}\cdot b_{j})}_{=:T_{1}}+4\underbrace{\sum_{i,j\in[m]}w_{i}^{1/2}w_{j}^{1/2}(a_{i}\cdot a_{j})(a_{i}\cdot b_{j})(a_{j}\cdot b_{i})}_{=:T_{2}}\\
 & =\underbrace{1^{\top}\Diag\Par{A_{x}A_{x}^{\top}}W^{\half}BB^{\top}W^{\half}\Diag\Par{A_{x}A_{x}^{\top}}1}_{=:N_{1}}\\
 & \quad+4\cdot\underbrace{1^{\top}\Diag\Par{A_{x}B^{\top}}W^{\half}A_{x}A_{x}^{\top}W^{\half}\Diag\Par{A_{x}B^{\top}}1}_{=:N_{2}}+4\cdot\underbrace{1^{\top}\Diag\Par{A_{x}A_{x}^{\top}}W^{\half}BA_{x}^{\top}W^{\half}\Diag\Par{A_{x}B^{\top}}1}_{=:N_{3}}\\
 & \quad+2T_{1}+4T_{2}.
\end{align*}

For $N_{1}$, we note that $BB^{\top}\precsim\frac{1}{\sqrt{n}}I_{n}$
due to $B^{\top}B=A_{x}^{\top}W_{x}^{\half}N_{x}^{2}W_{x}^{\half}A_{x}\leq p^{2}A_{x}^{\top}W_{x}A_{x}\lesssim\frac{1}{\sqrt{n}}I_{n}$.
Hence,
\begin{align*}
N_{1} & \lesssim\frac{1}{\sqrt{n}}\tr\Par{\Diag\Par{A_{x}A_{x}^{\top}}W\Diag\Par{A_{x}A_{x}^{\top}}}\\
 & \leq\frac{1}{\sqrt{n}}\tr\Par{A_{x}^{\top}WA_{x}}\norm{\Diag\Par{A_{x}A_{x}^{\top}}}_{\infty}\leq\frac{1}{n}\tr\Par{\frac{1}{\sqrt{n}}I_{n}}\\
 & =\frac{1}{\sqrt{n}}.
\end{align*}

For $N_{2}$, note that $W^{\half}A_{x}A_{x}^{\top}W^{\half}\preceq\frac{1}{\sqrt{n}}I_{n}$
due to $A_{x}^{\top}W_{x}A_{x}\preceq\frac{1}{\sqrt{n}}I_{n}$. Thus,
\begin{align*}
N_{2} & \lesssim\frac{1}{\sqrt{n}}\tr\Par{\Diag\Par{A_{x}B^{\top}}^{2}}=\frac{1}{\sqrt{n}}\sum_{i\in[m]}(a_{i}\cdot b_{i})^{2}\\
 & \leq\frac{1}{\sqrt{n}}\sum_{i}\norm{a_{i}}^{2}\norm{b_{i}}^{2}\leq\frac{1}{n}\tr\Par{BB^{\top}}\lesssim\frac{1}{n}\tr\Par{\frac{1}{\sqrt{n}}I_{n}}\\
 & =\frac{1}{\sqrt{n}}.
\end{align*}

For $N_{3}$, using the Cauchy-Schwarz inequality
\begin{align*}
N_{3} & \leq\sqrt{1^{\top}\Diag\Par{A_{x}A_{x}^{\top}}W^{\half}BB^{\top}W^{\half}\Diag\Par{A_{x}A_{x}^{\top}}1^{\top}}\sqrt{1^{\top}\Diag\Par{A_{x}A_{x}^{\top}}W^{\half}A_{x}A_{x}^{\top}W^{\half}\Diag\Par{A_{x}B^{\top}}1}\\
 & \leq\sqrt{N_{1}}\sqrt{\frac{1}{\sqrt{n}}\tr\Par{\Diag\Par{A_{x}A_{x}^{\top}}\Diag\Par{A_{x}B^{\top}}}}\\
 & \leq\sqrt{\frac{1}{\sqrt{n}}}\sqrt{\frac{1}{\sqrt{n}}\tr\Par{\Par{\Diag\Par{A_{x}A_{x}^{\top}}}^{2}}\tr\Par{\Par{\Diag\Par{A_{x}B^{\top}}}^{2}}}\\
 & \leq\frac{1}{\sqrt{n}}\sqrt{\norm{\Diag\Par{A_{x}A_{x}^{\top}}}_{\infty}^{2}\tr I_{n}}\sqrt{\sqrt{n}N_{2}}\lesssim\frac{1}{\sqrt{n}}\sqrt{\frac{1}{n}n}\sqrt{1}\\
 & =\frac{1}{\sqrt{n}}.
\end{align*}

For $T_{1}$, using the inequality $2a\cdot b\leq2\norm a\norm b\leq\norm a^{2}+\norm b^{2}$
\begin{align*}
T_{1} & =\sum_{i,j\in[m]}(a_{i}\cdot a_{j})^{2}\Par{(w_{j}^{1/2}b_{i})\cdot(w_{i}^{1/2}b_{j})}\lesssim\sum_{i,j\in[m]}(a_{i}\cdot a_{j})^{2}\Par{w_{j}\norm{b_{i}}^{2}+w_{i}\norm{b_{j}}^{2}}\\
 & \lesssim\sum_{i,j\in[m]}w_{j}(a_{i}\cdot a_{j})^{2}\norm{b_{i}}^{2}=\sum_{i}\norm{b_{i}}^{2}\tr\Par{\sum_{j}w_{j}(a_{i}\cdot a_{j})(a_{i}\cdot a_{j})}\\
 & =\sum_{i}\norm{b_{i}}^{2}\tr\Par{a_{i}^{\top}\Par{\sum_{j}a_{j}w_{j}a_{j}^{\top}}a_{i}}=\sum_{i}\norm{b_{i}}^{2}\tr\Par{a_{i}^{\top}A_{x}^{\top}WA_{x}a_{i}}\\
 & \leq\frac{1}{\sqrt{n}}\sum_{i}\norm{b_{i}}^{2}\norm{a_{i}}^{2}\leq\frac{1}{n}\tr\Par{BB^{\top}}\\
 & \leq\frac{1}{\sqrt{n}}.
\end{align*}

For $T_{2}$, using $(a_{i}\cdot a_{j})\leq\norm{a_{i}}\norm{a_{j}}\leq\frac{1}{\sqrt{n}}$
\begin{align*}
T_{2} & =\sum_{i,j\in[m]}w_{i}^{1/2}w_{j}^{1/2}(a_{i}\cdot a_{j})(a_{i}\cdot b_{j})(a_{j}\cdot b_{i})\leq\frac{1}{\sqrt{n}}\sum_{i,j\in[m]}w_{i}^{1/2}w_{j}^{1/2}(a_{i}\cdot b_{j})(a_{j}\cdot b_{i})\\
 & =\frac{1}{\sqrt{n}}\sum_{i}w_{i}^{1/2}b_{i}^{\top}\sum_{j}a_{j}w_{j}^{1/2}b_{j}^{\top}a_{i}=\frac{1}{\sqrt{n}}\sum_{i}\tr\Par{a_{i}w_{i}^{1/2}b_{i}^{\top}A_{x}^{\top}W^{1/2}B}\\
 & =\frac{1}{\sqrt{n}}\tr\Par{\Par{A_{x}^{\top}W^{1/2}B}^{2}}\underset{\text{CS}}{\leq}\frac{1}{\sqrt{n}}\tr\Par{B^{\top}W^{1/2}A_{x}A_{x}^{\top}W^{1/2}B}\leq\frac{1}{n}\tr\Par{B^{\top}B}\\
 & \leq\frac{1}{\sqrt{n}}
\end{align*}
Putting all the bounds together, we have $\E(\bar{P}_{2}(h))^{2}\lesssim n\cdot\frac{1}{\sqrt{n}}=\sqrt{n}\leq n$.

\paragraph{Term $\textsf{B}$.}

Now we control $\Abs{D^{4}g(p_{z})[h,h,h,h]}$. 

We first show that for any given $\alpha=\Theta(1)$, each coordinate
of $w_{x}/s_{x}^{\alpha}$ and $w_{p_{z}}/s_{p_{z}}^{\alpha}$ is
close. For $0\leq t\le1$, we define $x_{t}:=x+t\frac{r}{\sqrt{n}}h$,
and $s_{t},$ $w_{t}$ in the same fashion. Then for $p=\Theta\Par{\log^{O(1)}m}$
\begin{align*}
\max_{i\in[m]}\Abs{\log\frac{(w_{p_{z},i})^{\alpha}}{s_{p_{z},i}}-\log\frac{(w_{x,i})^{\alpha}}{s_{x,i}}} & \leq\int_{0}^{1}\Abs{\frac{d}{dt}\log\frac{[w_{t,i}]^{\alpha}}{s_{t,i}}}dt\lesssim\frac{r}{\sqrt{n}}\norm h_{A_{x}^{\top}W_{x}A_{x}}\leq\frac{1}{n^{1/4}}\norm z.
\end{align*}
Just as in showing SASC of the Vaidya metric, we can make this bound
arbitrarily small (say $\delta\approx0$) by conditioning on a high-probability
region. Hence,
\begin{equation}
e^{-\delta}\frac{(w_{x,i})^{\alpha}}{s_{x,i}}\leq\frac{(w_{p_{z},i})^{\alpha}}{s_{p_{z},i}}\leq e^{\delta}\frac{(w_{x,i})^{\alpha}}{s_{x,i}}.\label{eq:closeness}
\end{equation}
We remark that this $\Theta(1)$-multiplicative closeness is still
valid without the $\sqrt{n}$-scaling of $A_{x}^{\top}W_{x}A_{x}$.

Using the formulas of $D^{2}\theta[h,h,h,h]$ in (\ref{eq:LW-fourth-moment}),
\begin{align*}
 & \Abs{D^{2}g(p)[(z-x)^{\otimes4}]}\\
 & =\Abs{\frac{r^{4}}{n^{2}}D^{2}g(p)[h^{\otimes4}]}\\
 & \lesssim\frac{r^{4}}{n^{2}}\sqrt{n}\Par{\tr\Par{W_{p}S_{p,h}^{4}}+\Abs{\tr\Par{S_{p,h}^{3}W_{p,h}'}}+\Abs{\tr\Par{S_{p,h}^{2}W_{p,h}''}}}\\
 & =\frac{r^{4}}{n^{2}}\sqrt{n}\bigg(\underbrace{\tr\Par{W_{p}S_{p,h}^{4}}}_{=:T_{1}}+\underbrace{\Abs{\tr\Par{S_{p,h}^{3}\Diag\Par{W_{p}^{\half}N_{p}W_{p}^{\half}s_{p,h}}}}}_{=:T_{2}}+\underbrace{\Abs{\tr\Par{S_{p,h}^{2}W_{p,h}''}}}_{=:T_{3}}\bigg),
\end{align*}
where in the last line we simply used the formula of $W_{p,h}'$ in
Lemma~\ref{lem:DWh}.

For $T_{1}$, using the closeness of $w_{i}/s_{i}^{4}$ for each $i\in[m]$
in (\ref{eq:closeness}), we have $T_{1}\lesssim\tr\Par{W_{x}S_{x,h}^{4}}$.
Using the Cauchy-Schwarz inequality, 
\begin{align}
\E\Par{\tr\Par{W_{x}S_{x,h}^{4}}}^{2} & \lesssim\sum_{i,j\in[m]}w_{i}w_{j}\sqrt{\E(a_{i}\cdot h)^{8}}\sqrt{\E(a_{j}\cdot h)^{8}}\nonumber \\
 & \lesssim\Par{\sum_{i\in[m]}w_{i}\norm{a_{i}}^{4}}^{2}\lesssim\Par{n\cdot\frac{1}{n}}^{2}=1.\label{eq:hpb-ws4}
\end{align}
Thus, $T_{1}\lesssim1$ w.h.p.

For $T_{2}$, using the Cauchy-Schwarz again
\begin{align*}
T_{2} & =\Abs{\tr\Par{S_{p,h}^{3}W_{p}^{\half}\Diag\Par{N_{p}W_{p}^{\half}s_{p,h}}}}\\
 & \leq\sqrt{\tr\Par{S_{p,h}^{3}W_{p}S_{p,h}^{3}}}\sqrt{s_{p,h}^{\top}W_{p}^{\half}N_{p}^{2}W_{p}^{\half}s_{p,h}}\\
 & \underset{\text{(i)}}{\lesssim}\sqrt{s_{p,h}^{3}W_{p}s_{p,h}^{3}}\sqrt{s_{p,h}^{\top}W_{p}s_{p,h}}\\
 & \underset{\text{(ii)}}{\lesssim}\sqrt{s_{x,h}^{3}W_{x}s_{x,h}^{3}}\sqrt{s_{x,h}^{\top}W_{x}s_{x,h}},
\end{align*}
where in (i) we used $N_{x}\preceq p^{2}I$ due to Lemma~\ref{lem:LS-comp-tool},
and in (ii) the closeness of $w_{i}/s_{i}^{6}$ and $w_{i}/s_{i}^{2}$
established in (\ref{eq:closeness}). Note that 
\begin{align*}
\E\Par{s_{x,h}^{3}W_{x}s_{x,h}^{3}}^{2} & \underset{\text{CS}}{\lesssim}\sum_{i,j\in[m]}w_{i}w_{j}\sqrt{\E\Par{a_{i}\cdot h}^{12}}\sqrt{\E\Par{a_{j}\cdot h}^{12}}=\Par{\sum_{i}w_{i}\sqrt{\E\Par{a_{i}\cdot h}^{12}}}^{2}\\
 & \lesssim\Par{\sum_{i}w_{i}\norm{a_{i}}^{6}}^{2}\leq\Par{n\cdot\frac{1}{n^{3/2}}}^{2}\\
 & =\frac{1}{n}.
\end{align*}
As $s_{x,h}^{\top}W_{x}s_{x,h}\preceq\frac{1}{\sqrt{n}}\norm h^{2}$,
the concentration of the standard Gaussian implies that $s_{x,h}^{\top}W_{x}s_{x,h}\lesssim\sqrt{n}$
w.h.p. Therefore, $T_{2}\lesssim1$ w.h.p.

For $T_{3}$, we note that (\ref{eq:trGamma}) with $\Gamma_{p}=S_{p,h}^{2}$
is equal to $T_{3}$. Following (\ref{eq:trGammaBasic}) with I, II,
III, IV defined in (\ref{eq:trGamma}), we have
\begin{align*}
T_{3} & \lesssim\sum_{v=\text{I,II,III,IV}}\sqrt{\tr\Par{S_{p,h}^{2}W_{p}S_{p,h}^{2}}}\norm v_{W_{p}^{-1}}\\
 & \underset{\text{(i)}}{\lesssim}\sqrt{\tr\Par{S_{p,h}^{2}W_{p}S_{p,h}^{2}}}\Par{\tr\Par{S_{p,h}^{2}W_{p}}+\tr\Par{S_{p,h}^{4}W_{p}}}\\
 & \underset{\text{(ii)}}{\lesssim}\sqrt{\tr\Par{S_{x,h}^{4}W_{x}}}\Par{\tr\Par{S_{x,h}^{2}W_{x}}+\tr\Par{S_{x,h}^{4}W_{x}}}
\end{align*}
where (i) follows from our established bounds $\norm v_{W_{p}^{-1}}\lesssim\norm h_{A_{p}^{\top}W_{p}A_{p}}^{2}=\tr\Par{S_{p,h}^{2}W_{p}}$
for $v=$ I, II, and IV, and a bound $\norm{\text{IV}}_{W_{p}^{-1}}\lesssim\tr\Par{S_{p,h}^{4}W_{p}}$,
and (ii) follows from the conditioned event where the closeness of
$w_{i}/s_{i}^{2}$ at $x$ and $z$ holds. Since we established high-probability
bounds $\tr\Par{S_{x,h}^{4}W_{x}}\lesssim1$ due to (\ref{eq:hpb-ws4})
and $\tr\Par{S_{x,h}^{2}W_{x}}\lesssim\sqrt{n}$, combining these
yield $T_{3}\lesssim\sqrt{n}$ w.h.p.

Putting the high-probability bounds on $T_{1},T_{2},T_{3}$ together,
we obtain $\Abs{D^{2}g(p)[h^{\otimes4}]}\lesssim n$ w.h.p.
\end{proof}

\subsubsection{Checking HSC and quadratic constraints \label{proof:quadratic}}

We show that a $\nu$-SC barrier $\psi=-\log f(x)$ satisfies
\[
\Abs{D^{4}\psi(x)[h^{\otimes4}]}\lesssim\nu^{2}\norm h_{\hess\psi(x)}^{2}+\Abs{\frac{D^{4}f(x)[h^{\otimes4}]}{f(x)}}.
\]

\begin{proof}
[Proof of Lemma~\ref{lem:4th-log}] Fix $h\in\Rn$ and $x\in\inter(K)$,
define $\phi(t):=\psi(x+th)$. Then direct computation leads to
\begin{align*}
\phi' & =-\frac{f'}{f},\\
\phi'' & =\Par{\frac{f'}{f}}^{2}-\frac{f''}{f}=(\phi')^{2}-\frac{f''}{f},\\
\phi''' & =2\phi'\phi''-\frac{f'''f-f''f'}{f^{2}}=2\phi'\phi''-\frac{f'''}{f}+\frac{f''f'}{f^{2}}\\
 & =2\phi'\phi''+\phi'(\phi''-(\phi')^{2})-\frac{f'''}{f}\\
 & =3\phi'\phi''-(\phi')^{3}-\frac{f'''}{f},\\
\phi^{(4)} & =3(\phi'')^{2}+3\phi'\phi'''-3(\phi')^{2}\phi''-\frac{f^{(4)}f-f'''f'}{f^{2}}\\
 & =3(\phi'')^{2}+3\phi'\phi'''-3(\phi')^{2}\phi''+\phi'\Par{\phi'''-3\phi'\phi''+(\phi')^{3}}-\frac{f^{(4)}}{f}\\
 & =3(\phi'')^{2}+4\phi'\phi'''-6(\phi')^{2}\phi''+(\phi')^{4}-\frac{f^{(4)}}{f}.
\end{align*}
Recall that $|\phi'''|\leq2(\phi'')^{3/2}$ due to self-concordance
of $\phi$ and $\phi''\geq\frac{1}{\nu}(\phi')^{2}$ due to the definition
of the barrier parameter, which is equivalent to $|\phi'|\leq\sqrt{\nu}(\phi'')^{1/2}$.
Therefore,
\begin{align*}
|\phi^{(4)}| & \leq4\Abs{\phi'\phi'''}+3\Abs{(\phi'')^{2}}+6\Abs{(\phi')^{2}\phi''}+\Abs{(\phi')^{4}}+\Abs{\frac{f^{(4)}}{f}}\\
 & \leq8\sqrt{\nu}\Abs{\phi''}^{2}+3\Abs{\phi''}^{2}+6\nu\Abs{\phi''}^{2}+\nu^{2}\Abs{\phi''}^{2}+\Abs{\frac{f^{(4)}}{f}}\\
 & \lesssim\nu^{2}\Abs{\phi''}^{2}+\Abs{\frac{f^{(4)}}{f}}.\qedhere
\end{align*}
\end{proof}
Using the tool above, we study Dikin-amenability of barriers for quadratic
constraints.
\begin{proof}
[Proof of Lemma~\ref{lem:quadratic-const}] Let us check the last
claim first. By Lemma~\ref{lem:linear-trans}, we may assume that
\[
\phi(x,y)=-\log\Par{l+q^{\top}y-\half\norm x^{2}},
\]
and let $f(x,y)=l+q^{\top}y-\half\norm x^{2}$. For $z=(x,y)\in\intk$
and $u=(u_{x},u_{y})\in\Rn$, we have 
\begin{align}
D\phi(z)[u] & =-\frac{1}{f}(q\cdot u_{y}-x\cdot u_{x})=\frac{x\cdot u_{x}-q\cdot u_{y}}{f},\nonumber \\
D^{2}\phi(z)[u,u] & =\frac{1}{f^{2}}(x\cdot u_{x}-q\cdot u_{y})^{2}+\frac{1}{f}\norm{u_{x}}^{2}.\label{eq:hessian-quadratic}
\end{align}

For the first term, it holds that for $v=(v_{x},v_{y})\in\Rn$ 
\begin{align*}
D\Par{\frac{(x\cdot u_{x}-q\cdot u_{y})^{2}}{f^{2}}}[v] & =\frac{2(x\cdot u_{x}-q\cdot u_{y})(v_{x}\cdot u_{x})}{f^{2}}+2(x\cdot u_{x}-q\cdot u_{y})^{2}\cdot\frac{x\cdot v_{x}-q\cdot v_{y}}{f^{3}}\\
D^{2}\Par{\frac{(x\cdot u_{x}-q\cdot u_{y})^{2}}{f^{2}}}[v,v] & =\frac{2(v_{x}\cdot u_{x})^{2}}{f^{2}}+4\frac{(x\cdot u_{x}-q\cdot u_{y})(v_{x}\cdot u_{x})(x\cdot v_{x}-q\cdot v_{y})}{f^{3}}\\
 & \quad+\frac{4(x\cdot u_{x}-q\cdot u_{y})(v_{x}\cdot u_{x})(x\cdot v_{x}-q\cdot v_{y})+2(x\cdot u_{x}-q\cdot u_{y})^{2}\norm{v_{x}}^{2}}{f^{3}}\\
 & \quad+6\frac{(x\cdot u_{x}-q\cdot u_{y})^{2}(x\cdot v_{x}-q\cdot v_{y})^{2}}{f^{4}}\\
 & =\frac{2(v_{x}\cdot u_{x})^{2}}{f^{2}}+4\frac{(x_{q}\cdot u)(v_{x}\cdot u_{x})(x_{q}\cdot v)}{f^{3}}\\
 & \quad+\frac{4(x_{q}\cdot u)(v_{x}\cdot u_{x})(x_{q}\cdot v)+2(x_{q}\cdot u)^{2}\norm{v_{x}}^{2}}{f^{3}}+6\frac{(x_{q}\cdot u)^{2}(x_{q}\cdot v)^{2}}{f^{4}},
\end{align*}
where $x_{q}:=(x,-q)\in\Rn$.

For the second term, direct computations lead to 
\begin{align*}
D\Par{\frac{\norm{u_{x}}^{2}}{f}}[v] & =\frac{1}{f^{2}}\norm{u_{x}}^{2}(x\cdot v_{x}-q\cdot v_{y}),\\
D^{2}\Par{\frac{\norm{u_{x}}^{2}}{f}}[v,v] & =\frac{2}{f^{3}}\norm{u_{x}}^{2}(x\cdot v_{x}-q\cdot v_{y})^{2}+\frac{1}{f^{2}}\norm{u_{x}}^{2}\norm{v_{x}}^{2}\\
 & =\frac{2}{f^{3}}\norm{u_{x}}^{2}(x_{q}\cdot v)^{2}+\frac{1}{f^{2}}\norm{u_{x}}^{2}\norm{v_{x}}^{2}
\end{align*}
Putting these together, for $u,v\in\Rn$
\begin{align*}
 & D^{4}\phi[u,u,v,v]\\
 & =\frac{1}{f^{2}}\norm{u_{x}}^{2}\norm{v_{x}}^{2}+\underbrace{\frac{2}{f^{2}}(v_{x}\cdot u_{x})^{2}}_{\geq0}+\frac{4}{f^{3}}\Par{\half\norm{u_{x}}^{2}(x_{q}\cdot v)^{2}+2(x_{q}\cdot u)(v_{x}\cdot u_{x})(x_{q}\cdot v)+\frac{(x_{q}\cdot u)^{2}}{2}\norm{v_{x}}^{2}}\\
 & \qquad+\frac{6}{f^{4}}(x_{q}\cdot u)^{2}(x_{q}\cdot v)^{2}\\
 & \geq\frac{4}{f^{3}}\bigg(\underbrace{\half\norm{u_{x}}^{2}(x_{q}\cdot v)^{2}+\frac{1}{2}\norm{v_{x}}^{2}(x_{q}\cdot u)^{2}}_{\text{Use AM-GM}}+2(x_{q}\cdot u)(v_{x}\cdot u_{x})(x_{q}\cdot v)\bigg)\\
 & \qquad+\underbrace{\frac{1}{f^{2}}\norm{u_{x}}^{2}\norm{v_{x}}^{2}+\frac{6}{f^{4}}(x_{q}\cdot u)^{2}(x_{q}\cdot v)^{2}}_{\text{Use AM-GM}}\\
 & \geq\frac{4}{f^{3}}\Par{\norm{u_{x}}\norm{v_{x}}|x_{q}\cdot v||x_{q}\cdot u|-2|x_{q}\cdot u||x_{q}\cdot v|\norm{u_{x}}\norm{v_{x}}}+\frac{2\sqrt{6}}{f^{3}}|x_{q}\cdot u||x_{q}\cdot v|\norm{u_{x}}\norm{v_{x}}\\
 & =\frac{4}{f^{3}}\norm{u_{x}}\norm{v_{x}}|x_{q}\cdot v||x_{q}\cdot u|\Par{\frac{1}{2}\sqrt{6}-1}\\
 & \geq0.\qedhere
\end{align*}
\end{proof}


\subsubsection{PSD: convexity and strongly self-concordance \label{proof:psd-convex-ssc}}

We start with convexity of $\log\det\hess\phi$ for $\phi(X)=-\log\det X$.
\begin{proof}
[Proof of Proposition~\ref{prop:convex-logdet}] Using Lemma~\ref{prop:metricFormula}
and $\det\Par{M^{\top}(A\otimes A)M}=2^{n(n-1)/2}(\det A)^{n+1}$
(Lemma~\ref{lem:Kronecker}) in the first and second equality below,
\begin{align*}
\log\det\hess\phi(X) & =\log\det\Par{M^{\top}(X^{-1}\otimes X^{-1})M}\\
 & =\frac{n(n-1)}{2}\cdot\log2-(n+1)\log\det X.
\end{align*}
Since $-\log\det X$ is convex in $X$ (immediate from (\ref{eq:2ndDiffLogDet})),
the convexity of $\log\det\hess\phi(X)$ also follows.
\end{proof}
Observe from the proof that $\log\det\hess\phi(X)=\text{Const.}+(n+1)\phi(X)$.
Differentiating both sides in direction $H$,
\begin{align}
 & \tr\Par{\Par{\hess\phi(X)}^{-1}D^{3}\phi(X)[H]}=(n+1)D\phi(X)[H]\quad(\because(\ref{eq:gradLogDet}))\nonumber \\
\Longrightarrow & \tr\Par{\Par{\hess\phi(X)}^{-\half}D^{3}\phi(X)[H]\Par{\hess\phi(X)}^{-\half}}=-(n+1)\tr(X^{-1}H).\label{eq:difflogdet}
\end{align}

We are ready to show SSC of $\phi$.
\begin{proof}
[Proof of Lemma~\ref{lem:logdet-scaling}] For $H\in\S^{n}$ and
$t\in\R$, denote $X_{t}:=X+tH$ and $g_{t}:=M^{\top}(X_{t}\otimes X_{t})^{-1}M$.
Note that
\[
\norm{\Par{\hess\phi(X)}^{-\half}D^{3}\phi(X)[H]\Par{\hess\phi(X)}^{-\half}}_{F}^{2}=\tr\Par{g^{-1}\del_{t}g_{t}\vert_{t=0}g^{-1}\del_{t}g_{t}\vert_{t=0}}
\]
and
\begin{align}
\del_{t}g_{t}\vert_{t=0} & \underset{\text{(i)}}{=}\del_{t}\Par{M^{\top}(X_{t}\otimes X_{t})^{-1}M}\bigg|_{t=0}\nonumber \\
 & \underset{\text{(ii)}}{=}-M^{\top}(X\otimes X)^{-1}\del_{t}(X_{t}\otimes X_{t})\vert_{t=0}(X\otimes X)^{-1}M\nonumber \\
 & =-M^{\top}(X^{-1}\otimes X^{-1})\Par{H\otimes X+X\otimes H}(X^{-1}\otimes X^{-1})M\nonumber \\
 & \underset{\text{(iii)}}{=}-M^{\top}\Par{X^{-1}HX^{-1}\otimes X^{-1}+X^{-1}\otimes X^{-1}HX^{-1}}M,\label{eq:18-1}
\end{align}
where (i) is due to Lemma~\ref{prop:metricFormula}, (ii) is from
(\ref{eq:diffInverse}), and (iii) follows from $(A\otimes B)(C\otimes D)=(AC)\otimes(BD)$
(Lemma~\ref{lem:Kronecker}-3).

Recall that positive semidefinite matrices have unique positive semidefinite
square roots, so $(X\otimes X)^{\half}=X^{\half}\otimes X^{\half}$
(due to the fact that $(X^{\half}\otimes X^{\half})\cdot(X^{\half}\otimes X^{\half})=X\otimes X$).
Since $g_{t}=M^{\top}(X_{t}\otimes X_{t})^{-\half}(X_{t}\otimes X_{t})^{-\half}M$,
the corresponding orthogonal projection matrix is 
\[
P_{t}:=P\Par{(X_{t}\otimes X_{t})^{-\half}M}=(X_{t}\otimes X_{t})^{-\half}Mg_{t}^{-1}M^{\top}(X_{t}\otimes X_{t})^{-\half}.
\]
 By substituting $\del_{t}g_{t}\big|_{t=0}$ with (\ref{eq:18-1}),
\begin{align*}
 & \tr\Par{g^{-1}\del_{t}g_{t}\vert_{t=0}g^{-1}\del_{t}g_{t}\vert_{t=0}}\\
 & =\tr\bigg(g^{-1}M^{\top}\Par{X^{-1}HX^{-1}\otimes X^{-1}+X^{-1}\otimes X^{-1}HX^{-1}}M\\
 & \qquad\qquad\cdot g^{-1}M^{\top}\Par{X^{-1}HX^{-1}\otimes X^{-1}+X^{-1}\otimes X^{-1}HX^{-1}}\cblue M\bigg)\\
 & =\tr\bigg(\cblue Mg^{-1}M^{\top}\Par{X^{-1}HX^{-1}\otimes X^{-1}+X^{-1}\otimes X^{-1}HX^{-1}}M\\
 & \qquad\qquad\cdot g^{-1}M^{\top}\Par{X^{-1}HX^{-1}\otimes X^{-1}+X^{-1}\otimes X^{-1}HX^{-1}}\bigg)\\
 & =\tr\Par{\Par{\cred{Mg^{-1}M^{\top}}\Par{X^{-1}HX^{-1}\otimes X^{-1}+X^{-1}\otimes X^{-1}HX^{-1}}}^{2}}\\
 & =\tr\Par{\Par{\cred{(X\otimes X)^{\half}P(X\otimes X)^{\half}}\Par{X^{-1}HX^{-1}\otimes X^{-1}+X^{-1}\otimes X^{-1}HX^{-1}}}^{2}}\\
 & =\tr\bigg(\bigg(P\underbrace{(X\otimes X)^{\half}\Par{X^{-1}HX^{-1}\otimes X^{-1}+X^{-1}\otimes X^{-1}HX^{-1}}(X\otimes X)^{\half}}_{=:S}\bigg)^{2}\bigg)\\
 & =\tr\Par{PSPS}.
\end{align*}
By Lemma~\ref{lem:Kronecker}-3 once again in the second equality,
we can further manipulate $S$ as follows:
\begin{align*}
S & =(X^{\half}\otimes X^{\half})\Par{X^{-1}HX^{-1}\otimes X^{-1}+X^{-1}\otimes X^{-1}HX^{-1}}(X^{\half}\otimes X^{\half})\\
 & =\underbrace{X^{-\half}HX^{-\half}\otimes I}_{=:A}+\underbrace{I\otimes X^{-\half}HX^{-\half}}_{=:B}.
\end{align*}
By the Cauchy-Schwarz inequality along with $P^{\top}P=P^{2}=P$ and
$P\preceq I$,
\begin{align*}
\tr\Par{PSPS} & \leq\tr\Par{\Par{PS}^{\top}PS}=\tr\Par{S^{\top}P^{\top}PS}=\tr\Par{S^{\top}PS}\\
 & \leq\tr\Par{S^{\top}S}=\norm S_{F}^{2}\\
 & \leq\Par{\norm A_{F}+\norm B_{F}}^{2}.
\end{align*}
Using Lemma~\ref{lem:Kronecker}-3, 
\begin{align*}
\norm A_{F}^{2} & =\tr\Par{\Par{X^{-\half}HX^{-\half}\otimes I}\cdot\Par{X^{-\half}HX^{-\half}\otimes I}}\\
 & =\tr(X^{-\half}HX^{-1}HX^{-\half}\otimes I)\\
 & =\tr\Par{X^{-\half}HX^{-1}HX^{-\half}}\tr(I)\\
 & =n\norm H_{X}^{2},
\end{align*}
and similarly $\norm B_{F}^{2}=n\norm H_{X}^{2}$. Therefore, 
\[
\norm{\Par{\hess\phi(X)}^{-\half}D^{3}\phi(X)[H]\Par{\hess\phi(X)}^{-\half}}_{F}\leq\sqrt{\tr(PSPS)}\leq2\sqrt{n}\norm H_{X},
\]
and $\psi_{X}\leq2\sqrt{n}$.

To see the optimality of $\sqrt{n}$, let us recall (\ref{eq:difflogdet}):
\[
\tr\Par{\Par{\hess\phi(X)}^{-\half}D^{3}\phi(X)[H]\Par{\hess\phi(X)}^{-\half}}=-(n+1)\tr(X^{-1}H).
\]
Taking supremum on both sides, 
\begin{align*}
\sup_{H:\norm H_{X}=1}\tr\Par{\Par{\hess\phi(X)}^{-\half}D^{3}\phi(X)[H]\Par{\hess\phi(X)}^{-\half}} & =\sup_{\substack{H\in\S^{n}:\\
\norm{X^{-\half}HX^{-\half}}_{F}=1
}
}-(n+1)\tr(X^{-\half}HX^{-\half})\\
 & =\sup_{S\in\S^{n}:\norm S_{F}=1}(n+1)\tr(S),
\end{align*}
and this objective achieves the maximum at $H=-\frac{1}{\sqrt{n}}X$,
with the supremum being $(n+1)\sqrt{n}$. On the other hand, due to
$\tr(A)\leq\sqrt{d}\norm A_{F}$ for $A\in\R^{d\times d}$,
\begin{align*}
\tr\bigg(\Par{\hess\phi(X)}^{-\half}D^{3}\phi(X)[H] & \Par{\hess\phi(X)}^{-\half}\bigg)\\
 & \leq\sqrt{\frac{n(n+1)}{2}}\cdot\norm{\Par{\hess\phi(X)}^{-\half}D^{3}\phi(X)[H]\Par{\hess\phi(X)}^{-\half}}_{F}\\
 & \leq\sqrt{\frac{n(n+1)}{2}}\cdot\psi_{X}\norm H_{X}
\end{align*}
and thus by taking supremum on both sides over a symmetric matrix
$H$ with $\norm H_{X}=1$, it follows that $(n+1)\sqrt{n}\leq\sqrt{\frac{n(n+1)}{2}}\psi_{X}$
and 
\[
\sqrt{2(n+1)}\leq\psi_{X}.\qedhere
\]
\end{proof}

\subsubsection{PSD: strongly lower trace self-concordance \label{proof:psd-sltsc}}

Direct computation leads to $D^{2}g(X)[H,H]\succeq0$ (so SLTSC).
\begin{proof}
[Proof of Lemma~\ref{lem:logdet-sltsc}] For $g(X)=-\hess\log\det X$,
recall that $g(X)[H,H]=\tr\Par{X^{-1}HX^{-1}H}$, and thus for any
$V\in\S^{n}$
\begin{align*}
Dg(X)[H,H,V] & =-\tr\Par{X^{-1}VX^{-1}\cdot HX^{-1}H}-\tr\Par{X^{-1}H\cdot X^{-1}VX^{-1}\cdot H}\\
 & =-2\tr\Par{X^{-1}VX^{-1}HX^{-1}H},
\end{align*}
and by differentiating again
\begin{align}
 & D^{2}g(X)[H,H,V,V]\nonumber \\
 & =4\tr\Par{X^{-1}VX^{-1}VX^{-1}HX^{-1}H}+2\tr\Par{X^{-1}VX^{-1}HX^{-1}VX^{-1}H}\nonumber \\
 & =4\tr\Par{X^{-\half}HX^{-1}VX^{-1}VX^{-1}HX^{-\half}}+2\tr\Par{X^{-\half}VX^{-1}HX^{-\half}\cdot X^{-\half}VX^{-1}HX^{-\half}}\nonumber \\
 & \underset{\text{(i)}}{\geq}4\tr\Par{X^{-\half}HX^{-1}VX^{-1}VX^{-1}HX^{-\half}}-2\tr\Par{X^{-\half}HX^{-1}VX^{-\half}\cdot X^{-\half}VX^{-1}HX^{-\half}}\nonumber \\
 & =2\tr\Par{X^{-\half}HX^{-1}VX^{-1}VX^{-1}HX^{-\half}}\geq0,\label{eq:D4ph1}
\end{align}
where in (i) we used the Cauchy-Schwarz inequality, and thus $D^{2}g(X)[H,H]\succeq0$.
\end{proof}

\subsubsection{PSD: average self-concordance \label{proof:psd-asc}}

We establish a connection to the Gaussian orthogonal ensemble (GOE):
for $d=n(n+1)/2$ and $\svec(H)\sim\ncal\Par{0,\frac{r^{2}}{d}g(X)^{-1}}$,
we have $\frac{\sqrt{dn}}{r}X^{-\half}HX^{-\half}$ is the GOE.
\begin{proof}
[Proof of Lemma~\ref{lem:conn-to-goe}] For $h_{X}:=\svec\Par{X^{-\half}HX^{-\half}}$
and $h:=\svec(H)$, we have
\[
h_{X}=L(X\otimes X)^{-\half}Mh
\]
 due to $h_{X}=\svec\Par{X^{-\half}HX^{-\half}}=L\vec\Par{X^{-\half}HX^{-\half}}=L(X\otimes X)^{-\half}\vec(H)=L(X\otimes X)^{-\half}Mh$.
As $h\sim\ncal\Par{0,\frac{r^{2}}{d}g(X)^{-1}}$, $h_{X}$ is a Gaussian
with zero mean and covariance
\begin{align*}
 & \frac{r^{2}}{d}L(X\otimes X)^{-\half}Mg(X)^{-1}M^{\top}(X\otimes X)^{-\half}L^{\top}\\
\underset{\text{(i)}}{=} & \frac{r^{2}}{dn}L(X\otimes X)^{-\half}MLN(X\otimes X)N^{\top}L^{\top}M^{\top}(X\otimes X)^{-\half}L^{\top}\\
\underset{\text{(*)}}{=} & \frac{r^{2}}{dn}L(X\otimes X)^{-\half}N(X\otimes X)N^{\top}(X\otimes X)^{-\half}L^{\top}\\
\underset{\text{(*)}}{=} & \frac{r^{2}}{dn}L(X\otimes X)^{-\half}(X\otimes X)N(X\otimes X)^{-\half}L^{\top}\\
\underset{\text{(*)}}{=} & \frac{r^{2}}{dn}LNL^{\top}\\
\underset{\text{(ii)}}{=} & \frac{r^{2}}{dn}\left[\begin{array}{cc}
I_{n}\\
 & \half I_{n(n-1)/2}
\end{array}\right],
\end{align*}
where (i) follows from Proposition~\ref{prop:metricFormula}, ({*})
follows from Lemma~\ref{lem:MNL-properties}, and (ii) follows from
Page 427 of \cite{magnus1980elimination} that $LNL^{\top}$ is a
$d\times d$ diagonal matrix with $n$ times $1$ and $\half n(n-1)$
times $1/2$. Precisely, the entries of $h_{X}\in\R^{d}$ corresponding
to the diagonals of $X^{-1/2}HX^{-1/2}$ are $1$, and its entries
corresponding to off-diagonals is $1/2$. This is exactly the covariance
matrix of a $d$-dimensional GOE, so $X^{-\half}HX^{-\half}\overset{d}{\sim}\frac{r}{\sqrt{dn}}G$
for the GOE $G$.
\end{proof}
Now we show ASC of $n\phi$.
\begin{proof}
[Proof of Lemma~\ref{lem:logdet-asc}] Consider a series expansion
of $\norm{Z-X}_{g(Z)}^{2}$ at $X$ for $Z=X+H$:
\[
\norm{Z-X}_{g(Z)}^{2}-\norm{Z-X}_{g(X)}^{2}=\sum_{k=1}^{\infty}\frac{1}{k!}D^{k}g(X)[H^{\otimes k+2}].
\]
By induction we can check that for $H_{X}:=X^{-\half}HX^{-\half}$
\begin{align*}
Dg(X)[H^{\otimes3}] & =-2n\tr\Par{X^{-1}HX^{-1}HX^{-1}H}=-2\tr\Par{H_{X}^{3}},\\
D^{2}g(X)[H^{\otimes4}] & =3!n\tr\Par{H_{X}^{4}},\\
D^{k}g(X)[H^{\otimes(k+2)}] & =(-1)^{k}(k+1)!n\tr\Par{H_{X}^{k+2}}.
\end{align*}
Putting these back into the series expansion,
\begin{align*}
\norm{Z-X}_{g(Z)}^{2}-\norm{Z-X}_{g(X)}^{2} & =\sum_{k=1}^{\infty}(-1)^{k}(k+1)n\tr\Par{H_{X}^{k+2}}\\
 & =\sum_{k=1}^{\infty}(-1)^{k}(k+1)n\cdot\Par{\frac{r}{\sqrt{dn}}}^{k+2}\tr\Par{H^{k+2}}\\
 & =\frac{r^{2}}{d}\sum_{k=1}^{\infty}(-1)^{k}(k+1)\Par{\frac{r}{\sqrt{dn}}}^{k}\tr\Par{H^{k+2}},
\end{align*}
where $H$ is the GOE.

For ASC, it suffices to show that $\sum_{k=1}^{\infty}(-1)^{k}(k+1)\Par{\frac{r}{\sqrt{dn}}}^{k}\tr\Par{H^{k+2}}$
can be made arbitrarily small. We first control $\sum_{k\geq2}$:
\[
\Abs{\sum_{k\geq2}(-1)^{k}(k+1)\Par{\frac{r}{\sqrt{dn}}}^{k}\tr\Par{H^{k+2}}}\leq\sum_{k\geq2}(k+1)\Par{\frac{r}{\sqrt{dn}}}^{k}n\cdot\norm H_{\text{op}}^{k+2}
\]
By Corollary 4.4.8 in \cite{vershynin2018high}, with high probability
\[
\norm H_{\text{op}}\lesssim\sqrt{n}
\]
and thus
\begin{align*}
\sum_{k\geq2}(k+1)\Par{\frac{r}{\sqrt{dn}}}^{k}n\cdot\norm H_{\text{op}}^{k+2} & \leq\sum_{k\geq2}(k+1)r^{k}\frac{1}{n^{3k/2}}n\cdot n^{\frac{k+2}{2}}\leq\sum_{k\geq2}(k+1)r^{k}n^{2-k}.
\end{align*}
By taking $r=\Omega(1)$ small enough, we can make this series arbitrarily
small.

Now let us bound $\frac{r}{n^{3/2}}\tr\Par{H^{3}}$ ($k=1$ case).
This is a Gaussian polynomial in $\svec(H)$, so it suffices to show
$\E\Par{\tr\Par{H^{3}}}^{2}=O(n^{3})$ in order to use Lemma~\ref{lem:conc-gaussian-poly}
on the concentration bound for Gaussian polynomials. For $H=(H_{ab})\in\S^{n}$,
\[
\Par{\tr\Par{H^{3}}}^{2}=\sum_{ipq}H_{ip}H_{pq}H_{qi}\cdot\sum_{jrs}H_{jr}H_{rs}H_{sj}=\sum_{ipqjrs}H_{ip}H_{pq}H_{qi}H_{jr}H_{rs}H_{sj},
\]
where each $H_{**}$ in the summand is an independent Gaussian with
zero mean and variance $1$ or $1/2$ (as $H$ is the GOE). We can
classify the indices $\{i,p,q,j,r,s\}$ into the following types:
\begin{align*}
6\text{ distinct indices } & \{a,b,c,d,e,f\}\\
5\text{ distinct indices } & \{a,b,c,d,(e,e)\}\\
4\text{ distinct indices } & \{a,b,c,(d,d,d)\},\{a,b,(c,c),(d,d)\}\\
\text{Others } & \dots
\end{align*}
where for example $\{a,b,c,d,e,f\}$ means all indices are different,
and $\{a,b,c,d,(e,e)\}$ means that there appear 5 different indices
$\{a,b,c,d,e\}$ but exists one pair $(e,e)$ of the same index. Note
that $\E H_{ip}H_{pq}H_{qi}H_{jr}H_{rs}H_{sj}=O(1)$ is at most the
sixth moment of a standard Gaussian. It implies that toward our goal
of showing $O(n^{3})$-bound on $\Par{\tr\Par{H^{3}}}^{2}$, it suffices
to look into only three types of indices above. This is because the
terms from other types contribute at most $O(1)n^{3}$ to $\Par{\tr\Par{H^{3}}}^{2}$.

For any term with 6 distinct indices, we can always find an `uncoupled'
$H_{**}$ (for example $H_{ab}$) in the summand that is independent
of all the others, so its expectation of the summand is $0$.

For the terms with $5$-distinct indices $\{a,b,c,d,(e,e)\}$, due
to symmetry (see Figure~\ref{fig:ipq-jrs}) we can further classify
the index $(i,p,q,j,r,s)$ into either $(a,b,c,d,e,e)$ or $(a,b,e,c,d,e)$.
In both cases , $H_{ab}$ has no coupled Gaussian, so the expectations
of the summand are also $0$. 

For $4$-distinct indices, let us first consider $\{a,b,c,(d,d,d)\}$-type
indices. In this case $(i,p,q,j,r,s)$ is of the form either $(a,a,a,b,c,d)$
or $(a,a,b,a,c,d)$ due to symmetry. In both cases, $H_{cd}$ has
no coupled Gaussian. Now consider $\{a,b,(c,c),(d,d)\}$-type indices.
Then $(i,p,q,j,r,s)$ is of the form either $(a,b,c,c,d,d)$ or $(a,c,c,b,d,d)$
or $(a,c,d,b,c,d)$. For each case, $H_{ab},H_{cc},H_{ac}$ are uncoupled
ones. Therefore, $\E[H_{ip}H_{pq}H_{qi}H_{jr}H_{rs}H_{sj}]=0$ whenever
there are at least $4$ distinct indices.
\end{proof}
% Figure environment removed

\begin{rem}
It seems challenging to show that $\phi$ is SASC using the same technique.
When $g$ is given by 
\[
g=n\hess(-\log\det X)+g'
\]
for other PSD matrix function $g'$, we know that $\svec(H_{X})=\svec(X^{-\half}HX^{-\half})$
follows a Gaussian distribution with zero mean and covariance matrix
$M$ satisfying 
\[
M\preceq\left[\begin{array}{cc}
I_{n}\\
 & \half I_{n(n-1)/2}
\end{array}\right].
\]
The challenge in extending our approach to achieving SASC is that
the entries of $h=\svec(H_{X})$ might exhibit dependencies in this
case, making the previous approach infeasible. This arises because
many fundamental results in the random matrix theory often presume
independence of the entries of a random matrix. Moreover, our combinatorial
argument for the $k=1$ case is not feasible in the presence of such
dependencies.
\end{rem}


