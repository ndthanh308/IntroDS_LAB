%% LyX 2.3.6.2 created this file.  For more info, see http://www.lyx.org/.
%% Do not edit unless you really know what you are doing.
\documentclass[10pt,english]{article}
\usepackage{lmodern}
\renewcommand{\sfdefault}{lmss}
\renewcommand{\ttdefault}{lmtt}
\usepackage[T1]{fontenc}
\usepackage[latin9]{inputenc}
\usepackage{geometry}
\geometry{verbose,tmargin=1in,bmargin=1in,lmargin=1in,rmargin=1in}
\usepackage{array}
\usepackage{booktabs}
\usepackage{units}
\usepackage{mathtools}
\usepackage{multirow}
\usepackage{amsmath}
\usepackage{amsthm}
\usepackage{amssymb}
\usepackage[authoryear]{natbib}

\makeatletter

%%%%%%%%%%%%%%%%%%%%%%%%%%%%%% LyX specific LaTeX commands.
\newcommand{\lyxmathsym}[1]{\ifmmode\begingroup\def\b@ld{bold}
  \text{\ifx\math@version\b@ld\bfseries\fi#1}\endgroup\else#1\fi}

%% Because html converters don't know tabularnewline
\providecommand{\tabularnewline}{\\}

%%%%%%%%%%%%%%%%%%%%%%%%%%%%%% Textclass specific LaTeX commands.
\numberwithin{figure}{section}
\numberwithin{equation}{section}
\theoremstyle{definition}
\newtheorem*{problem*}{\protect\problemname}
\theoremstyle{plain}
\newtheorem{thm}{\protect\theoremname}[section]
\theoremstyle{definition}
\newtheorem{defn}[thm]{\protect\definitionname}
\theoremstyle{plain}
\newtheorem*{thm*}{\protect\theoremname}
\theoremstyle{plain}
\newtheorem{lem}[thm]{\protect\lemmaname}
\theoremstyle{remark}
\newtheorem{rem}[thm]{\protect\remarkname}
\theoremstyle{plain}
\newtheorem{prop}[thm]{\protect\propositionname}
\theoremstyle{plain}
\newtheorem{cor}[thm]{\protect\corollaryname}
\theoremstyle{definition}
\newtheorem{example}[thm]{\protect\examplename}
\theoremstyle{plain}
\newtheorem*{lem*}{\protect\lemmaname}
\theoremstyle{remark}
\newtheorem*{acknowledgement*}{\protect\acknowledgementname}
\theoremstyle{remark}
\newtheorem{claim}[thm]{\protect\claimname}

%%%%%%%%%%%%%%%%%%%%%%%%%%%%%% User specified LaTeX commands.
\numberwithin{equation}{section}
\numberwithin{figure}{section}
\theoremstyle{plain}
\newtheorem{thm2}{\protect\theoremname}
\theoremstyle{definition}
\newtheorem{defn2}[thm2]{\protect\definitionname}
\theoremstyle{plain}
\newtheorem*{thm2*}{\protect\theoremname}
\theoremstyle{plain}
\newtheorem{lem2}[thm2]{\protect\lemmaname}
\theoremstyle{plain}
\newtheorem{prop2}[thm2]{\protect\propositionname}
\theoremstyle{plain}
\newtheorem{cor2}[thm2]{\protect\corollaryname}

\usepackage{color}
\definecolor{ForestGreen}{rgb}{0.1333,0.5451,0.1333}
%\usepackage[colorlinks,linkcolor=ForestGreen,citecolor=ForestGreen,bookmarks,bookmarksopen,bookmarksnumbered]{hyperref}
\usepackage{hyperref}
\hypersetup{
  colorlinks,
  linkcolor={red!50!black},
  citecolor={blue!70!black},
  urlcolor={blue!80!black}
}

\usepackage[lined,boxed,ruled,norelsize,linesnumbered,algo2e]{algorithm2e}
\@addtoreset{section}{part}
\usepackage{graphicx}
\usepackage{thmtools}
\usepackage{microtype}
\usepackage{mathrsfs}
\usepackage{ragged2e}
\usepackage{caption}
\usepackage{subcaption}
\usepackage{thm-restate}
\allowdisplaybreaks
\usepackage{natbib}
\bibliographystyle{plainnat}
\bibpunct{(}{)}{;}{a}{,}{,}
\declaretheorem[name=Definition]{defre}
\declaretheorem[name=Theorem,sibling=thm2]{thmre}
\declaretheorem[name=Lemma,sibling=thm2]{lemre}
\declaretheorem[name=Proposition,sibling=thm2]{propre}
\declaretheorem[name=Corollary,sibling=thm2]{corre}
\usepackage{multicol}
\usepackage{enumitem}
\setlist[itemize]{leftmargin=*, itemsep=1pt}

\usepackage{tikz}
\usetikzlibrary{calc}
\usetikzlibrary{positioning, arrows.meta, decorations.pathreplacing}

\makeatother

\usepackage{babel}
\providecommand{\acknowledgementname}{Acknowledgement}
\providecommand{\claimname}{Claim}
\providecommand{\corollaryname}{Corollary}
\providecommand{\definitionname}{Definition}
\providecommand{\examplename}{Example}
\providecommand{\lemmaname}{Lemma}
\providecommand{\problemname}{Problem}
\providecommand{\propositionname}{Proposition}
\providecommand{\remarkname}{Remark}
\providecommand{\theoremname}{Theorem}

\begin{document}
\global\long\def\bw{\mathsf{Ball\ walk}}%
\global\long\def\dw{\mathsf{Dikin\ walk}}%
\global\long\def\sw{\mathsf{Speedy\ walk}}%
\global\long\def\dws{\mathsf{Dikin\ walks}}%
\global\long\def\gcdw{\mathsf{GCDW}}%
\global\long\def\gc{\mathsf{Gaussian\ cooling}}%

\global\long\def\acal{\mathcal{A}}%
\global\long\def\bcal{\mathcal{B}}%
\global\long\def\ccal{\mathcal{C}}%
\global\long\def\dcal{\mathcal{D}}%
\global\long\def\ecal{\mathcal{E}}%
\global\long\def\fcal{\mathcal{F}}%
\global\long\def\gcal{\mathcal{G}}%
\global\long\def\hcal{\mathcal{H}}%
\global\long\def\ical{\mathcal{I}}%
\global\long\def\tcal{\mathbb{\mathcal{T}}}%
\global\long\def\mcal{\mathbb{\mathcal{M}}}%
\global\long\def\pcal{\mathcal{P}}%
\global\long\def\ncal{\mathcal{N}}%
\global\long\def\kcal{\mathcal{K}}%

\global\long\def\O{\mathcal{O}}%
\global\long\def\Otilde{\widetilde{\mathcal{O}}}%

\global\long\def\E{\mathbb{E}}%
\global\long\def\Z{\mathbb{Z}}%
\global\long\def\P{\mathbb{P}}%
\global\long\def\N{\mathbb{N}}%

\global\long\def\R{\mathbb{R}}%
\global\long\def\Rd{\mathbb{R}^{d}}%
\global\long\def\Rdd{\mathbb{R}^{d\times d}}%
\global\long\def\Rn{\mathbb{R}^{n}}%
\global\long\def\Rnn{\mathbb{R}^{n\times n}}%

\global\long\def\psd{\mathbb{S}_{+}^{d}}%
\global\long\def\pd{\mathbb{S}_{++}^{d}}%

\global\long\def\defeq{\stackrel{\mathrm{{\scriptscriptstyle def}}}{=}}%
\global\long\def\ind{\mathds{1}}%

\global\long\def\veps{\varepsilon}%
\global\long\def\lda{\lambda}%
\global\long\def\vphi{\varphi}%
\global\long\def\onu{\bar{\nu}}%
\global\long\def\og{\overline{g}}%
\global\long\def\del{\partial}%

\global\long\def\half{\frac{1}{2}}%
\global\long\def\nhalf{\nicefrac{1}{2}}%
\global\long\def\texthalf{{\textstyle \frac{1}{2}}}%

\global\long\def\kro{\otimes}%
\global\long\def\hada{\circ}%
\global\long\def\chooses#1#2{_{#1}C_{#2}}%

\global\long\def\vol{\textrm{vol}}%
\global\long\def\law{\textup{\textsf{law}}}%

\global\long\def\tr{\textup{\textsf{Tr}}}%
\global\long\def\diag{\textsf{\textup{diag}}}%
\global\long\def\Diag{\textup{\textsf{Diag}}}%
\global\long\def\vec{\textup{\textsf{vec}}}%
\global\long\def\svec{\textup{\textsf{svec}}}%
\global\long\def\inter{\textup{\textsf{int}}}%

\global\long\def\T{\mathsf{T}}%
\global\long\def\e{\mathrm{e}}%

\global\long\def\id{\mathrm{id}}%
\global\long\def\st{\mathrm{s.t.\ }}%
\global\long\def\nnz{\textup{\textsf{nnz}}}%
\global\long\def\lw{\textup{\textsf{Lw}}}%

\global\long\def\intk{\inter(K)}%

\global\long\def\range{\mathrm{Range}}%
\global\long\def\nulls{\mathrm{Null}}%
\global\long\def\spanning{\textup{\textsf{span}}}%
\global\long\def\rowspace{\textup{\textsf{row}}}%
\global\long\def\rank{\mathrm{rank}}%

\global\long\def\bs#1{\boldsymbol{#1}}%
 

\global\long\def\eu#1{\EuScript{#1}}%

\global\long\def\mb#1{\mathbf{#1}}%
 

\global\long\def\mbb#1{\mathbb{#1}}%

\global\long\def\mc#1{\mathcal{#1}}%

\global\long\def\mf#1{\mathfrak{#1}}%

\global\long\def\ms#1{\mathscr{#1}}%

\global\long\def\mss#1{\mathsf{#1}}%

\global\long\def\msf#1{\mathsf{#1}}%

\global\long\def\on#1{\operatorname{#1}}%

\global\long\def\textint{{\textstyle \int}}%
\global\long\def\Dd{\mathrm{D}}%
\global\long\def\D{\mathrm{d}}%
\global\long\def\grad{\nabla}%
 
\global\long\def\hess{\nabla^{2}}%
 
\global\long\def\lapl{\triangle}%
 
\global\long\def\deriv#1#2{\frac{\D#1}{\D#2}}%
 
\global\long\def\pderiv#1#2{\frac{\partial#1}{\partial#2}}%
 
\global\long\def\de{\partial}%
\global\long\def\lagrange{\mathcal{L}}%

\global\long\def\Gsn{\mathcal{N}}%
 
\global\long\def\BeP{\textnormal{BeP}}%
 
\global\long\def\Ber{\textnormal{Ber}}%
 
\global\long\def\Bern{\textnormal{Bern}}%
 
\global\long\def\Bet{\textnormal{Beta}}%
 
\global\long\def\Beta{\textnormal{Beta}}%
 
\global\long\def\Bin{\textnormal{Bin}}%
 
\global\long\def\BP{\textnormal{BP}}%
 
\global\long\def\Dir{\textnormal{Dir}}%
 
\global\long\def\DP{\textnormal{DP}}%
 
\global\long\def\Expo{\textnormal{Expo}}%
 
\global\long\def\Gam{\textnormal{Gamma}}%
 
\global\long\def\GEM{\textnormal{GEM}}%
 
\global\long\def\HypGeo{\textnormal{HypGeo}}%
 
\global\long\def\Mult{\textnormal{Mult}}%
 
\global\long\def\NegMult{\textnormal{NegMult}}%
 
\global\long\def\Poi{\textnormal{Poi}}%
 
\global\long\def\Pois{\textnormal{Pois}}%
 
\global\long\def\Unif{\textnormal{Unif}}%

\global\long\def\bpar#1{{\bigl(#1\bigr)}}%
\global\long\def\Bpar#1{{\Bigl(#1\Bigr)}}%

\global\long\def\snorm#1{{\|#1\|}}%
\global\long\def\bnorm#1{{\bigl\Vert#1\bigr\Vert}}%
\global\long\def\Bnorm#1{{\Bigl\Vert#1\Bigr\Vert}}%

\global\long\def\sbrack#1{{[#1]}}%
\global\long\def\bbrack#1{{\bigl[#1\bigr]}}%
\global\long\def\Bbrack#1{{\Bigl[#1\Bigr]}}%

\global\long\def\sbrace#1{\{#1\}}%
\global\long\def\bbrace#1{\bigl\{#1\bigr\}}%
\global\long\def\Bbrace#1{\Bigl\{#1\Bigr\}}%

\global\long\def\Abs#1{\left\lvert #1\right\rvert }%
\global\long\def\Par#1{\left(#1\right)}%
\global\long\def\Brack#1{\left[#1\right]}%
\global\long\def\Brace#1{\left\{  #1\right\}  }%

\global\long\def\inner#1{\langle#1\rangle}%
 
\global\long\def\binner#1#2{\left\langle {#1},{#2}\right\rangle }%
 

\global\long\def\norm#1{{\|#1\|}}%
\global\long\def\onenorm#1{\norm{#1}_{1}}%
\global\long\def\twonorm#1{\norm{#1}_{2}}%
\global\long\def\infnorm#1{\norm{#1}_{\infty}}%
\global\long\def\fronorm#1{\norm{#1}_{\text{F}}}%
\global\long\def\nucnorm#1{\norm{#1}_{*}}%
\global\long\def\staticnorm#1{\|#1\|}%
\global\long\def\statictwonorm#1{\staticnorm{#1}_{2}}%

\global\long\def\mmid{\mathbin{\|}}%

\global\long\def\otilde#1{\widetilde{\mc O}(#1)}%
\global\long\def\wtilde{\widetilde{W}}%
\global\long\def\wt#1{\widetilde{#1}}%

\global\long\def\KL{\msf{KL}}%
\global\long\def\dtv{d_{\textrm{\textup{TV}}}}%

\global\long\def\cov{\mathrm{Cov}}%
\global\long\def\var{\mathrm{Var}}%

\global\long\def\cred#1{\textcolor{red}{#1}}%
\global\long\def\cblue#1{\textcolor{blue}{#1}}%
\global\long\def\cgreen#1{\textcolor{green}{#1}}%
\global\long\def\ccyan#1{\textcolor{cyan}{#1}}%

\global\long\def\iff{\Leftrightarrow}%
 
\global\long\def\textfrac#1#2{{\textstyle \frac{#1}{#2}}}%

%--------------------------------------------------------------------------------------------------------------------------------
% Common differentials with a small space in front of them
%--------------------------------------------------------------------------------------------------------------------------------
\global\long\def\dee{\mathop{\mathrm{d}\!}}%
 
\global\long\def\dt{\,\dee t}%
 
\global\long\def\ds{\,\dee s}%
 
\global\long\def\dx{\,\dee x}%
 
\global\long\def\dy{\,\dee y}%
 
\global\long\def\dz{\,\dee z}%
  
\global\long\def\dr{\,\dee r}%
 
\global\long\def\dB{\,\dee B}%
 % Brownian motion
\global\long\def\dW{\,\dee W}%
 % Wiener process
\global\long\def\dmu{\,\dee\mu}%
 
\global\long\def\dnu{\,\dee\nu}%
 
\global\long\def\domega{\,\dee\omega}%

%--------------------------------------------------------------------------------------------------------------------------------
% Set notation
%--------------------------------------------------------------------------------------------------------------------------------
\global\long\def\smiddle{\mathrel{}|\mathrel{}}%
 % Well-spaced \middle | symbol
%--------------------------------------------------------------------------------------------------------------------------------
%Text with quads around it
%--------------------------------------------------------------------------------------------------------------------------------
\global\long\def\qtext#1{\quad\text{#1}\quad}%
 % Semidefinite orders
\global\long\def\psdle{\preccurlyeq}%
 
\global\long\def\psdge{\succcurlyeq}%
 
\global\long\def\psdlt{\prec}%
 
\global\long\def\psdgt{\succ}%

%--------------------------------------------------------------------------------------------------------------------------------
% Vectors and matrices
%--------------------------------------------------------------------------------------------------------------------------------
\global\long\def\boldone{\mbf{1}}%
 % Bold 1
\global\long\def\ident{\mbf{I}}%
 % Identity matrix
% \def\v#1{\mbi{#1}} % Vector notation

%--------------------------------------------------------------------------------------------------------------------------------
% Probability and statistics macros
%--------------------------------------------------------------------------------------------------------------------------------
\global\long\def\eqdist{\stackrel{d}{=}}%
 
\global\long\def\todist{\stackrel{d}{\to}}%
 
\global\long\def\eqd{\stackrel{d}{=}}%
 
\global\long\def\independenT#1#2{\mathrel{\rlap{$#1#2$}\mkern4mu  {#1#2}}}%
 %\def\indep{\perp\!\!\!\perp} % conditional independence


\title{Gaussian Cooling and Dikin Walks: The Interior-Point Method for Logconcave
Sampling\date{}\author{Yunbum Kook\\ Georgia Tech\\  \texttt{yb.kook@gatech.edu} \and Santosh S. Vempala\\ Georgia Tech\\ \texttt{vempala@gatech.edu}}}
\maketitle
\begin{abstract}
The connections between (convex) optimization and (logconcave) sampling
have been considerably enriched in the past decade with many conceptual
and mathematical analogies. For instance, the Langevin algorithm can
be viewed as a sampling analogue of gradient descent and has condition-number-dependent
guarantees on its performance. In the early 1990s, Nesterov and Nemirovski
developed the Interior-Point Method (IPM) for convex optimization
based on self-concordant barriers, providing efficient algorithms
for structured convex optimization, often faster than the general
method. This raises the following question: can we develop an analogous
IPM for structured sampling problems?

In 2012, Kannan and Narayanan proposed the Dikin walk for uniformly
sampling polytopes, and an improved analysis was given in 2020 by
Laddha-Lee-Vempala. The Dikin walk uses a local metric defined by
a self-concordant barrier for linear constraints. Here we generalize
this approach by developing and adapting IPM machinery together with
the Dikin walk for poly-time sampling algorithms. Our IPM-based sampling
framework provides an efficient warm start and goes beyond uniform
distributions and linear constraints. We illustrate the approach on
important special cases, in particular giving the fastest algorithms
to sample uniform, exponential, or Gaussian distributions on a truncated
PSD cone. The framework is general and can be applied to other sampling
algorithms.

\pagebreak{}
\end{abstract}
\tableofcontents{}

\section{Introduction}
\label{sec:intro}
Reinforcement learning (RL) has been used successfully to train autonomous agents capable of achieving better than human-level performance in simulated environments like Go~\citep{go_rl} and a suite of Atari games~\citep{atari_rl}. They are increasingly finding new applications across computational analysis~\citep{matrix},  marketing~\citep{marketing},  education~\citep{tutor}, and biomedical research~\citep{protein} and. Recent breakthroughs in large-language models (LLMs)~\citep{gpt4} are primarily attributed to key RL components that have improved the generative capability of state-of-the-art LLMs. With RL frameworks being deployed at scale as well as performing autonomously, it becomes imperative to incorporate explainability in them, resulting in increased user trust in autonomous decision-making. Explaining the decisions of black-box RL agents for a given environment state is non-trivial as it not only involves explaining the final agent action but also includes complex decision-making and planning behind the output action.

A myriad of RL explainability methods with various attribution techniques has recently been proposed~\citep{greydanus2018visualizing,deshmukhexplaining,iyer2018transparency,puri2019explain}. In particular, they focus on identifying input states and past experiences (trajectories) that led the RL agent to learn complex behaviors. While these methods output important input state features (agent's observation) and trajectories, they fail to explain the minimal change in the trained policy leading to a desired outcome or (un)learning of a specific skill. Intuitively, this requires generating \textit{counterfactuals} for a given desired outcome (\ie identifying \textit{what} and \textit{how much} to change a given RL policy to obtain a target return for its current state). While some previous works have explored causal reinforcement learning~\citep{causal_rl}, there is little to no research on systematically explaining the mechanism of the complex policies learned by a given RL agent using counterfactual explanations.

\xhdr{Present Work} We propose \method, a framework for counterfactual analysis of RL policies. In our framework, we generate explanations by asking the question: ``What least change to the current policy would improve or worsen it to a new policy with a specified target return?'' To estimate such counterfactual policies, we present an objective that aims to obtain a new policy with an average performance equal to that of a specified return while limiting its modifications with respect to the given policy. The generated policies provide direct insights into how a policy can be modified to achieve better results as well as what to avoid in order not to deteriorate the performance. Further, we theoretically prove the connection between popular trust region-based optimization methods in RL with \method, bringing a new perspective of looking at RL optimization using a prominent explainability tool. Formally, the \method learns minimal changes in the current policy without changing its general behavior. To optimize counterfactual explanation policies, we specify a novel objective function that can be solved using basic on-policy Monte Carlo policy gradients. In our experiments across diverse RL environments, we show how our algorithm reliably achieves counterfactual for any the set target return for a given policy.

\xhdr{Our Contributions} We present our contributions as follows: 1) We formalize the problem of counterfactual explanation policy for explaining RL policies. 2) We propose \method, an explanatory framework for generating contrastive explanations for RL policies that identify \textit{minimal} changes in the current policy, which would lead to improving/worsening the performance of a given policy. 3) We derive a theoretical equivalence between the \method objective with the widely used trust region-based policy gradient methods. 4) We demonstrate the flexibility of \method through empirical evaluations of explanations generated for five OpenAI gym environments. Qualitative and quantitative results show that \method successfully generates a counterfactual policy for (un)learning skills while keeping close to the original policy.


\section{Mixing of \texorpdfstring{$\dw$}{Dikin walks} \label{sec:mixing-Dikin}}

We follow a standard conductance based argument (see e.g., \citet{lovasz1993random,vempala2005geometric}).
A lower bound on the conductance of a Markov chain provides an upper
bound on the mixing time of the Markov chain due to the following
result.
\begin{lem}
[\citet{lovasz1993random}] \label{lem:conductanceBound} Let $\pi_{T}$
be the distribution obtained after $T$ steps of a lazy reversible
Markov chain of conductance at least $\Phi$ with stationary distribution
$\pi$ and initial distribution $\pi_{0}$. For $\snorm{\pi_{0}/\pi}=\E_{\pi_{0}}\bbrack{\deriv{\pi_{0}}{\pi}}$
and any $\veps>0$, we have $\dtv(\pi_{T},\pi)\leq\veps+\sqrt{\frac{\snorm{\pi_{0}/\pi}}{\veps}}\bpar{1-\frac{\Phi^{2}}{2}}^{T}$.
\end{lem}

A lower bound on the conductance follows from two ingredients: \textbf{(i)}
one-step coupling and \textbf{(ii)} isoperimetry. The first refers
to showing that the one-step distributions of the $\dw$ from two
nearby points have TV-distance bounded away from one. The second is
a purely geometry property about the expansion of the target distribution.
Combining these two leads to a lower bound on the conductance:
\begin{lem}
[\citet{kook2022condition}, Adapted from Proposition 9] \label{lem:conductance}
Let $\pi$ be the stationary distribution of a lazy reversible Markov
chain on $\mc M$ with a transition kernel $P_{x}$. Assume the isoperimetry
$\psi_{\mc M}$ under a Riemannian distance $d_{g}$ and the following
one-step coupling: if $\snorm{x-y}_{g(x)}\leq\Delta<1$ for $x,y\in\mc M$,
then $\dtv(P_{x},P_{y})\leq0.9$. Then the conductance $\Phi$ of
the Markov chain is bounded lower by $\Omega(\psi_{\mc M}\Delta)$.
\end{lem}


\subsection{One-step coupling and isoperimetry}

Recall that a $\onu$-Dikin-amenable metric is $\onu$-symmetric,
SSC, LTSC, and ASC. \citet{laddha2020strong} was the first to attempt
characterizing essential properties of $g$ (or $\phi$) that determine
mixing times of $\dws$ for uniform sampling. Their framework necessitates
that $g$ satisfies $\onu$-symmetric, SSC, convexity of $\log\det g(x)$,
and $x\in\mc D_{g}^{r}(z)$ w.h.p. (where $z\sim\text{Unif}\bpar{\dcal_{g}^{r}(x)}$). 

However, their framework encounters a challenge when further incorporating
the work of \citet{narayanan2016randomized}, which analyzes the $\dw$
for uniform sampling over a convex region given as the intersection
of various convex sets. The challenge arises from the difficulty of
verifying the convexity of $\log\det(g_{1}+g_{2})$ when $\log\det g_{i}$
is convex for each $i=1,2$.

To address this challenge and succinctly characterize essential characteristics
of a metric for one-step coupling, we relax the convexity of $\log\det$
to (S)LTSC and introduce the notion of ASC to account for the condition
``$x\in\mc D_{g}^{r}(z)$ w.h.p.''. We show that one-step coupling
lemma below, one of main proof ingredients in obtaining a mixing-time
guarantee of the $\dw$, can be established under Dikin-amenability
of a metric. Our characterization of a metric for achieving one-step
coupling is general and unifies previous work on $\dws$ (\citet{kannan2012random,narayanan2016randomized,chen2018fast,laddha2020strong}).

We now proceed to establish one-step coupling under the relative smoothness
in $\phi$.
\begin{lem}
[One-step coupling]\label{lem:one-step} For convex $K\subset\Rd$,
let $g:\intk\to\pd$ be SSC, ASC, LTSC, and $\phi:\intk\to\R$ be
its function counterpart. Suppose that the potential $f$ of the target
distribution $\pi$ is $\beta$-relatively smooth in $\phi$. Then
there exist constants $s_{1},s_{2}>0$ such that if $\snorm{x-y}_{g(x)}\leq s_{1}r/\sqrt{d}$
with $r=s_{2}\,(1\wedge\nicefrac{1}{\sqrt{\beta}})$ for $x,y\in\intk$,
then $\dtv(P_{x},P_{y})\leq\frac{3}{4}+0.01$. 
\end{lem}

We provide a sketch of the proof (see \S\ref{proof:onestep} for
the full proof). A key distinction when extending beyond uniform distributions
lies in establishing a lower bound for the ratio $\frac{\exp(f(x))}{\exp(f(z))}$
to ensure a high acceptance probability. To tackle this issue, we
use the symmetry of the proposal distribution, claiming $\nicefrac{\exp(f(x))}{\exp(f(z))}\geq1$
at the expense of $\texthalf$ probability. However, this $\texthalf$
probability loss is incompatible with previous proof techniques based
on the triangle inequality: for a transition kernel $T$ and proposal
kernel $P$, the triangle inequality leads to 
\[
\dtv(T_{x},T_{y})\leq\dtv(T_{x},P_{x})+\dtv(P_{x},P_{y})+\dtv(P_{y},T_{y})\,,
\]
and then bound the second term in the RHS by Pinsker's inequality,
making it arbitrarily small by taking $r=\O(1)$ small enough. However,
this approach yields a bound of $\texthalf+\veps$ for both $\dtv(T_{x},P_{x})$
and $\dtv(T_{y},P_{y})$, making the RHS vacuous.

We instead work with the exact formula for $\dtv(T_{x},T_{y})$: for
the Gaussian $p_{x}=\ncal(x,\frac{r^{2}}{d}g(x)^{-1})$, 
\[
R_{x}(z)=\frac{p_{z}(x)}{p_{x}(z)}\frac{\pi(z)}{\pi(x)}=\sqrt{\frac{\det g(z)}{\det g(x)}}\,\frac{\exp(f(x))}{\exp(f(z))},\qquad A_{x}(z)=\min\bpar{1,R_{x}(z)\,\mathbf{1}_{K}(z)}\,,
\]
the transition kernel $T_{x}$ of the $\dw$ started at $x$ can be
written as 
\[
T_{x}(dz)=\underbrace{\bpar{1-\E_{p_{x}}[A_{x}(\cdot)]}}_{\eqqcolon r_{x}}\,\delta_{x}(\D z)+A_{x}(z)\,p_{x}(\D z)\,.
\]
Then, 
\begin{align*}
\dtv(T_{x},T_{y}) & =\frac{r_{x}+r_{y}}{2}+\half\int|A_{x}(z)\,p_{x}(z)-A_{y}(z)\,p_{y}(z)|\,\D z\,.
\end{align*}

As for $r_{x}$ and $r_{y}$, we bound below $\sqrt{\nicefrac{\det g(z)}{\det g(x)}}$
by $1-\veps$ at the cost of $\veps$-probability through SSC, LTSC,
and ASC of $g$, following \citet{laddha2020strong} with convexity
of $\log\det$ replaced by LTSC. As mentioned earlier, we also deduce
$\nicefrac{\exp(f(x))}{\exp(f(z))}\geq1$ through the symmetry of
Gaussian distributions at the cost of $\texthalf$ probability. Combining
these results, we obtain upper bounds of $\texthalf+\veps$ for small
$\veps>0$ on $r_{x}$ and $r_{y}$.

Establishing a bound of $\nicefrac{1}{4}+\veps$ on the second term
is a more involved task. It requires the closeness of acceptance probabilities
$A_{x}(z)$ and $A_{y}(z)$ as well as the probability densities $g_{x}(z)$
and $g_{y}(z)$. This closeness can only be achieved through sophisticated
conditioning on high-probability events due to ASC, SSC, and symmetry
of Gaussian proposals. To be precise, define good events $G_{x}=\cap_{i=0,2,3}B_{x,i}^{c}$
and $G_{y}=\cap_{i=0,2,3}B_{y,i}^{c}$ such that $\P_{\ncal_{g}^{r}(x)}(G_{x}^{c})\leq3\veps$
and $\P_{\ncal_{g}^{r}(y)}(G_{y}^{c})\leq3\veps$, where 
\begin{align*}
B_{x,0} & =\{\norm{z-x}_{x}\geq cr\}\,\ \text{with }c\geq1+\frac{2}{\sqrt{d}}\,\log\frac{1}{\veps}\,,\quad\text{(Tail bound for Gaussian)}\\
B_{x,1} & =\{-\langle\nabla f(x),x-z\rangle\leq0\}\,,\quad\text{(Symmetry of Gaussian)}\\
B_{x,2} & =\{\snorm{z-x}_{z}^{2}-\snorm{z-x}_{x}^{2}>2\veps\frac{r^{2}}{d}\}\,,\quad\text{(ASC of }g)\\
B_{x,3} & =\bbrace{\langle\grad\vphi(x),z-x\rangle\leq-2\frac{r}{\sqrt{d}}\,\snorm{g(x)^{-1/2}\grad\vphi(x)}_{2}\,\log\frac{1}{\veps}}\,.\quad\text{(SSC \& tail bound for Gaussian)}
\end{align*}
We further denote $G:=G_{x}\cup G_{y}$ and a partition of $G$ by
\[
G_{x\backslash y}:=G_{x}\backslash G_{y},\qquad G_{x,y}:=G_{x}\cap G_{y},\qquad G_{y\backslash x}:=G_{y}\backslash G_{x}\,.
\]
Then,
\begin{align*}
\half\int\underbrace{|A(x,z)\,p_{x}(z)-A(y,z)\,p_{y}(z)|}_{\eqqcolon Q}\,\D z & \leq3\veps+\underbrace{\half\int_{G_{x\backslash y}}Q\,\D z}_{\eqqcolon\mc A}+\underbrace{\half\int_{G_{y\backslash x}}Q\,\D z}_{\eqqcolon\mc B}+\underbrace{\half\int_{G_{x,y}}Q\,\D z}_{\eqqcolon\mc C}\,.
\end{align*}
We can bound $\acal$ and $\bcal$ by $\mc O(\veps)$ by Pinsker's
inequality and a well-known formula for the $\KL$ divergence between
two Gaussians. As for $\mc C$, conditioning on $B_{x,1}$ and using
the triangle inequality lead to

\[
\mc C\leq\frac{1}{4}+2\veps+\half\int_{G_{x}\cap G_{y}\cap B_{x,1}^{c}}\Big|\min\Bpar{1,\underbrace{\frac{\exp f(x)}{\exp f(z)}\,\frac{p_{z}(x)}{p_{x}(z)}}_{\eqqcolon\msf U}}-\min\Bpar{\underbrace{\frac{p_{y}(z)}{p_{x}(z)}}_{\eqqcolon\msf V},\underbrace{\frac{\exp f(y)}{\exp f(z)}\,\frac{p_{z}(y)}{p_{x}(z)}}_{\eqqcolon\msf W}}\Big|\,p_{x}(z)\,\D z\,.
\]
The bound of $\log\msf U\ge-4\veps$ was already obtained when bounding
$r_{x}$. We then show that $\lvert\log\msf V\rvert\le5\veps$ and
$\log\msf W\ge-7\veps$ conditioned on $G_{x}\cap G_{y}\cap B_{x,1}^{c}$
via closeness of SSC (Lemma~\ref{lem:strongSC-closeness}). Using
these, 
\[
\int_{G_{x}\cap G_{y}\cap B_{x,1}^{c}}|1\wedge\msf U-\msf V\wedge\msf W|\,p_{x}(z)\,\D z\leq e^{5\veps}-e^{4\veps}\,,
\]
which results in $\mc C\le1/4+\O(\veps)$. Putting the bounds on $r_{x},r_{y},\mc A,\mc B$,
and $\mc C$ together, we conclude that the TV-distance is bounded
by $3/4+\O(\veps)$.
\begin{rem}
We further note that $\snorm{x-y}_{x}$ can be replaced by the Riemannian
distance $d_{\phi}(x,y)$ with the metric defined by $\hess\phi$,
since these two distance are within a constant factor of each other:
\begin{lem}
[\citet{nesterov2002riemannian}, Lemma 3.1] \label{lem:Riemann-Dikin-close}
Let $\phi:\intk\to\R$ be self-concordant, and $x,y\in\intk$ with
$\delta:=\snorm{x-y}_{x}<1$. Then,
\[
\delta-\half\delta^{2}\leq d_{\phi}(x,y)\leq-\log(1-\delta)\,.
\]
\end{lem}

\end{rem}

Next, we present two isoperimetric inequalities derived from distinct
sources: the first comes from the symmetry of a barrier, while the
second arises from strong convexity in a local metric.

\paragraph{Isoperimetry via barrier parameters.}

The first one states that isoperimetry of log-concave distributions
under distance $d_{g}(x,y)$ (or $\snorm{x-y}_{g(x)}$ due to Lemma~\ref{lem:Riemann-Dikin-close})
is $\Omega(1/\sqrt{\onu})$. The following lemma is an extension of
\citet{laddha2020strong} from uniform distributions (over a convex
body) to general log-concave distributions. We defer the proof to
\S\ref{proof:isoperimetry}.
\begin{lem}
\label{lem:symmetry-iso} Let $\phi$ be self-concordant and $d_{\phi}$
be the Riemannian distance induced by the Hessian metric $\hess\phi$.
For a log-concave distribution $\pi$, isoperimetry $\psi_{\pi}$
under distance $d_{\phi}$ is $\Omega(1/\sqrt{\onu})$.
\end{lem}


\paragraph{Isoperimetry from relative strong convexity.}

Another kind of isoperimetry comes from relative strong-convexity
of the potential of a distribution. For a scalar $\alpha>0$, isoperimetry
of $e^{-\alpha\phi}$ on a Hessian manifold equipped with the metric
$\hess\phi$ is $\Omega(\sqrt{\alpha})$ if $\Dd^{4}\phi(x)\Brack{h^{\otimes4}}\geq0$
for all $x\in K$ and $h\in\Rd$ (see \citet[Lemma 37]{lee2018convergence}).
\citet[Lemma 9]{gopi2023algorithmic} further generalizes this to
show that if $\phi$ is self-concordant and the potential $f$ is
$\alpha$-relatively strong convex, then its isoperimetry is $\Omega(\sqrt{\alpha})$.
We can adapt this lemma by restricting this to a convex set $K$ (not
necessarily bounded). See \S\ref{proof:isoperimetry} for the proof.
\begin{lem}
[\citet{gopi2023algorithmic}, Adapted from Lemma 9] \label{lem:sc-iso}
For a closed convex set $K\subset\Rd$, let a convex function $\phi:\intk\to\R$
be self-concordant on $K$, $f:\intk\to\R$ $\alpha$-relatively strongly
convex in $\phi$, and $\pi$ a log-concave distribution with $\pi\propto\exp(-f)\cdot\mathbf{1}_{K}$.
For a partition $\{S_{1},S_{2},S_{3}\}$ of $K$ and the Riemannian
distance $d_{\phi}$ induced by the inner product $\langle a,b\rangle_{x}:=a^{\T}\hess\phi(x)\,b$,
it holds that 
\[
\pi(S_{3})\gtrsim\sqrt{\alpha}\,d_{\phi}(S_{1},S_{2})\,\pi(S_{1})\,\pi(S_{2})\,.\qedhere
\]
\end{lem}


\subsection{Mixing time: Proof of Theorem~\ref{thm:Dikin}}

Putting all these components together, we obtain the following mixing-time
bounds for the $\dw$. 

\thmDikin*
\begin{proof}
Lemma~\ref{lem:conductance} ensures that $\Phi\gtrsim\frac{r}{\sqrt{d}}\psi$
due to the one-step coupling in Lemma~\ref{lem:one-step}. Lemma~\ref{lem:symmetry-iso}
leads to $\psi\gtrsim\frac{1}{\sqrt{\onu}}$, while Lemma~\ref{lem:sc-iso}
implies $\psi\gtrsim\sqrt{\alpha}$ due to $\hess\phi\asymp g$. Thus,
\[
\Phi\gtrsim\frac{1}{\sqrt{d}}\,\bpar{\sqrt{\alpha}\vee\frac{1}{\sqrt{\onu}}}\bpar{1\vee\frac{1}{\sqrt{\beta}}}\,,
\]
and using Lemma~\ref{lem:conductanceBound}, we can enforce $\dtv(\pi_{T},\pi)\leq\veps$
by solving $\sqrt{\Lambda}e^{-T\Phi^{2}/2}\leq\veps$ and $\frac{\veps}{2}+\sqrt{\frac{\Lambda}{\veps/2}}e^{-T\Phi^{2}/2}\leq\veps$
for $T$, which results in 
\[
T\gtrsim d\,(1\vee\beta)\,\bpar{\onu\wedge\frac{1}{\alpha}}\log\frac{\Lambda}{\veps}\,.\qedhere
\]
\end{proof}




\section{Gaussian cooling on manifolds revisited: IPM framework for sampling
\label{sec:IPM-framework}}

In this section, we derive a sampling analogue of the Interior-Point
Method through comparison with IPM in optimization, by extending \emph{Gaussian
cooling on manifolds} introduced in \cite{cousins2018gaussian,lee2018convergence}.
We demonstrate that combining the sampling IPM framework with the
$\dw$ efficiently generates a warm start, allowing us to sample from
a target distribution $\pi$ with $\frac{d\pi}{dx}\propto e^{-f(x)}\cdot\mathbf{1}_{K}(x)$
and finite second moment.

\subsection{Derivation of sampling IPM \label{subsec:derivation-IPM-sampling}}

We begin by revisiting our setup. Let $K\subset\Rn$ be a closed convex
set, $g:\intk\to\pd$ a $(\nu,\onu)$-self-concordant matrix function,
and $\phi:\intk\to\R$ its (strictly convex) self-concordant counterpart.
We may assume that $\min_{x}\phi(x)=0$ by considering $\phi-\min_{x}\phi(x)$
(here, $\arg\min\phi(x)$ can be efficiently found by the optimization
IPM). We assume that $f$ is $\alpha$-relatively strongly convex
and $\beta$-relatively smooth in $\phi$ for $0\leq\alpha\leq\beta<\infty$,
i.e., $0\preceq\alpha\hess\phi\preceq\hess f\preceq\beta\hess\phi$
on $\inter(K)$. Lastly, we assume that $d\pi/dx\propto e^{-f}\cdot\mathbf{1}_{K}$
is integrable (i.e., $\int_{K}e^{-f(x)}dx<\infty$). We define $\bar{f}(x):=\frac{\nu}{n}f(x)$
and $g_{\phi}(x):=\hess\phi(x)$.

\paragraph{Interior-point method for optimization.}

In solving structural convex optimization problems, we encounter $\min_{x\in K}f(x)$,
where $f:\Rn\to\R$ is a convex function, and $K\subset\Rn$ is a
closed convex set. Also, both $K$ and $\{(x,t):f(x)\leq t\}$ admit
efficiently computable self-concordant barriers denoted by $\phi_{1}$
and $\phi_{2}$, respectively. We can simplify the problem by equivalently
solving $\min_{x\in K,\,\{(x,t):f(x)\leq t\}}t$. Therefore, it suffices
to focus on $\min_{x\in K,\,\{(x,t):f(x)\leq t\}}c\cdot(x,t)$ for
a constant $c\in\R^{n+1}$. 

IPM then regularizes this linear objective function by adding $\frac{1}{\lda}\phi(x,t)=\frac{1}{\lda}\Par{\phi_{1}(x)+\phi_{2}(x,t)}$
for $\lda>0$. This regularization removes the hard constraints of
$K$ and $\{f(x)\leq t\}$, and the resulting formulation becomes
\[
\min_{y=(x,t)\in\R^{n+1}}f_{\lda}(y):=c^{\top}y+\frac{1}{\lda}\phi(y),
\]
where $\phi(y)$ approaches infinity as $y$ approaches the boundary
of the constraints $K\cap\Brace{f(x)\leq t}$. For each fixed $\lda>0$,
there exists a minimum $y_{\lda}$ of the convex function $f_{\lda}(y)$.
Intuitively, as $\lda\to\infty$ the regularization term $\frac{1}{\lda}\phi(y)$
vanishes, so $y_{\lda}$ converges to the optimal point of $\min_{y\in K\cap\{f(x)\le t\}}c^{\top}y$.
The path followed by $\{x_{\lda}\}_{\lda>0}$ is known as the \emph{central
path}, and IPM aims to approximatly follow this central path as $\lda$
increases. 

To be precise, suppose that for $\lda_{1}>0$, an approximation solution
$\bar{y}_{\lda_{1}}$ maintained by IPM is close enough to $y_{\lda_{1}}$.
Then IPM takes an optimization step (e.g., a Newton step), which takes
into account the local geometry induced by the Hessian of the barrier
$\phi$, to find an approximate solution $\bar{y}_{\lda_{2}}$ when
$\lda_{2}>\lda_{1}$. As long as $\bar{y}_{\lda_{1}}$ is sufficiently
close to $y_{\lda_{1}}$, this approximate solution $\bar{y}_{\lda_{1}}$
serves a good starting point for the non-Euclidean optimizer, which
takes $\bar{y}_{\lda_{1}}$ to $\bar{y}_{\lda_{2}}$. IPM alternates
between increasing $\lda$ and updating $\bar{y}_{\lda}$, until $\lda$
reaches $\nu/\veps$. This is described formally as Algorithm~\ref{alg:IPM}.

\begin{algorithm2e}[H]

\caption{Interior-Point Method} \label{alg:IPM}

\SetAlgoLined

\textbf{Input:} A $\nu$-self-concordant barrier $\phi$ for a constraint

\textbf{Output:} $y_{\lda}$

Denote $f_{\lda}(y):=c^{\top}y+\frac{1}{\lda}\phi(y)$.

\tcp{Phase 1: Starting feasible point}

Find $y_{0}=\arg\min\phi(y)$, set $\lda=\frac{1}{6}\norm c_{\hess\phi(y_{0})}^{-1}$,
and $\bar{y}_{\lda}\gets y_{0}$.

\tcp{Phase 2: Increasing $\lda$ until $\lda\leq\frac{\nu+1}{\veps}$}

\While{$\lda\leq\frac{\nu+1}{\veps}$}{

$\bar{y}_{\lda}\gets\bar{y}_{\lda}-\Par{\hess f_{\lda}(\bar{y}_{\lda})}^{-1}\grad f_{\lda}(\bar{y}_{\lda})$
\tcp{``Opt. step'' (e.g., the Newton step)}

$\lda\gets(1+r)\lda$ with $r=\frac{1}{9\sqrt{\nu}}$. \tcp{Increase $\lda$}

}

\end{algorithm2e}

The ideas behind IPM are justified by the following theoretical guarantee:
Algorithm~\ref{alg:IPM} returns $y$ in $O\Par{\sqrt{\nu}\log\Par{\frac{\nu}{\veps}\norm c_{\hess\phi(y_{0})^{-1}}}}$
iterations such that $c^{\top}y\leq c^{\top}y^{*}+\veps$ for $y^{*}=\arg\min_{y\in K\cap\{f(x)\le t\}}c^{\top}y$.
% Figure environment removed


\paragraph{Translation to sampling.}

Now let us adapt each step of IPM into the sampling context with the
conceptual analogy between convex optimization and logconcave sampling
in mind: For convex $K\subset\Rn$ and convex function $f:K\to\R$
\begin{align*}
\min f(x) & \quad\longleftrightarrow\quad\text{sample }e^{-f(x)}\\
\text{s.t. }x\in K & \qquad\qquad\quad\text{s.t. }x\in K
\end{align*}
Similar to the optimization IPM, we first replace $f(x)$ by a new
variable $t$ and the constraint $\{f(x)\leq t\}$ (which is convex
due to convexity of $f$), resulting in the following sampling problem:
sample $(x,t)$ from a distributions whose density is proportional
to $e^{-t}$ subject to $x\in K$ and $\{(x,t)\in\R^{n+1}:f(x)\leq t\}$.
We note that this is indeed an equivalent sampling problem, since
the $x$-marginal of this distribution follows $e^{-f(x)}\cdot\mathbf{1}_{K}(x)$:
\[
\int_{\{(x,t)\in\R^{n+1}:f(x)\leq t\}}e^{-t}\cdot\mathbf{1}_{K}(x)dt=\int_{f(x)}^{\infty}e^{-t}\cdot\mathbf{1}_{K}(x)dt=e^{-f(x)}\cdot\mathbf{1}_{K}(x).
\]
Suppose we have a barrier $\phi$ for $K\cap\{f(x)\leq t\}$. Thus,
this motivates our focus on sampling from distributions of the form
$e^{-c^{\top}y}$ subject to a convex region $K$ with a barrier $\phi$
for $K$, where $y:=(x,t)\in\R^{n+1}$ is the extended variable and
$c\in\R^{n+1}$ is a vector.

Regularizing the potential $c^{\top}y$ of the distribution by adding
$\frac{1}{\sigma^{2}}\phi(y)$ for some $\sigma^{2}>0$, we can ignore
the hard constraint $K$ and obtain the following formulation: for
$f_{\sigma^{2}}(y)=c^{\top}y+\frac{1}{\sigma^{2}}\phi(y)$
\[
\text{sample }y\sim\frac{d\mu_{\sigma^{2}}}{dy}\propto e^{-f_{\sigma^{2}}(y)}=\exp\Par{-\Par{c^{\top}y+\frac{1}{\sigma^{2}}\phi(y)}},
\]
where $\phi(y)$ goes to infinity as it approaches the boundary of
$K$. For each fixed $\sigma^{2}>0$, as $\sigma^{2}\to\infty$ the
regularization term $\frac{1}{\sigma^{2}}\phi$ vanishes, and we can
expect $\mu_{\sigma^{2}}\to\pi$ for $\frac{d\pi}{dy}\propto e^{-c^{\top}y}\cdot\mathbf{1}_{K}(y)$.
Comparing this with the optimization IPM, the path of measures $\{\mu_{\sigma^{2}}\}_{\sigma^{2}>0}$
can be viewed as the central path in the space of measures. In an
ideal scenario, a sampling IPM should closely follow this central
path while increasing $\sigma^{2}$ along the path. To this end, we
update the current distribution $\bar{\mu}_{\sigma^{2}}$, which is
already close to $\mu_{\sigma^{2}}$ on the central path. This update
should leverage a \emph{sampling step} that is aware of the local
geometry induced by $\hess\phi$, which may involve running a non-Euclidean
sampler such as the $\dw$. This update brings $\bar{\mu}_{\sigma^{2}}$
to a new distribution $\bar{\mu}_{\sigma^{2}+\delta}$ that should
be close to $\mu_{\sigma^{2}+\delta}$ for small $\delta>0$, while
$\bar{\mu}_{\sigma^{2}}$ serves a good starting point for this sampling
step to find $\bar{\mu}_{\sigma^{2}+\delta}$. This procedure is repeated
until $\sigma^{2}$ becomes large enough.

To use this sampling IPM, we further refine the framework via \emph{Gaussian
cooling on manifolds}.

\paragraph{Comparison with the Gaussian cooling on manifolds (GCM).}

Gaussian Cooling introduced in \cite{cousins2018gaussian} was extended
to manifolds (GCM) by \cite{lee2018convergence}. It was initially
proposed for volume computation but shares remarkable similarities
with our sampling IPM. In fact, GCM can be recovered as a special
case of the sampling IPM with $c=0$ (i.e., uniform sampling) and
the Riemannian Hamiltonian Monte Carlo employed for the non-Euclidean
sampling step.

Returning to the comparison with the optimization IPM, we note that
two algorithms use different rules for updating $\sigma^{2}$. While
the optimization IPM updates $\sigma^{2}\gets\Par{1+\frac{1}{\sqrt{\nu}}}\sigma^{2}$,
GCM utilizes two distinct annealing schemes: 
\[
\sigma^{2}\gets\begin{cases}
\Par{1+\frac{1}{\sqrt{n}}}\sigma^{2} & \text{if }\sigma^{2}\leq\frac{\nu}{n}\\
\Par{1+\frac{\sigma}{\sqrt{\nu}}}\sigma^{2} & \text{o.w.}
\end{cases}
\]
While the first type of update in the small regime of $\sigma^{2}$
is inspired by a property of logconcavity of regularized distributions,
the second type of update in the large regime of $\sigma^{2}$ is
justified by concentration of measure $\exp\Par{-s\phi}$ in a thin
shell for $s>0$. We note that the second type in fact accelerates
the annealing process.

% Figure environment removed

However, significant challenges remain for the sampling IPM. First,
we need to extend this annealing scheme to exponential distributions
(recall that GCM was proposed for uniform sampling). To be precise,
we must account for the linear term $c^{\top}y$ (in addition to the
$\phi$ term) when designing the annealing scheme. Unfortunately,
the previous update scheme (which is applied only to $\phi$ part)
and analysis do not go through for this purpose. 

To address this issue, we introduce a further generalization of the
GCM annealing scheme in the small regime of $\sigma^{2}$, enabling
us to leverage the property of logconcavity of regularized distributions.
In the large regime of $\sigma^{2}$, we maintain the same annealing
scheme but employ a different analytical approach, utilizing a functional
inequality that avoids the need to quantify the thin-shell phenomenon
of $\exp\Par{-\Par{c^{\top}y+s\phi(y)}}$. 

To discuss another remaining issue, we note that a non-Euclidean sampler
used in the sampling step must have a provable mixing-time guarantee
for a target distribution proportional to $\exp\Par{-\Par{c^{\top}y+s\phi(y)}}$.
We already provided this through Theorem~\ref{thm:Dikin} in Section~\ref{sec:mixing-Dikin}
for the $\dw$, since the target potential is $s$-relatively strongly
convex and $s$-relatively smooth in $\phi$!

\subsection{IPM algorithm for sampling}

Our algorithm consists of four phases, where each phase updates a
current distribution in a different way. For generality, we present
this annealing process for a general potential $f$ instead of linear
functions, where $\alpha\hess\phi\preceq\hess f\preceq\beta\hess\phi$.

\begin{algorithm2e}[H]

\caption{Interior-Point Method for sampling} \label{alg:IPM-sampling}

\SetAlgoLined

\textbf{Input:} Target accuracy $\veps$, local metric $g$, its counterpart
$\phi$, non-Euclidean sampler $\textsf{NE-Sampler}(g,\veps)$, target
distribution $\frac{d\pi}{dy}\propto\exp(-f(y))$.

\textbf{Output:} $x'$

For $\bar{f}(x):=\frac{\nu}{n}f(x)$, let $\mu_{\sigma^{2}}$ be a
probability distribution such that $\frac{d\mu_{\sigma^{2}}}{dx}\propto\exp\Par{-V_{\sigma^{2}}(x)}$,
where
\[
V_{\sigma^{2}}(y):=\begin{cases}
\frac{\bar{f}(y)+\phi(y)}{\sigma^{2}} & \text{if }\sigma^{2}\leq\frac{\nu}{n}\\
f(y)+\frac{1}{\sigma^{2}}\phi(y) & \text{o.w}.
\end{cases}
\]

\tcp{Phase 1: Initial distribution}

Find $x^{*}=\arg\min_{x\in K}(\bar{f}+\phi)$ and let $D:=\dcal_{g}^{3\sigma_{0}\sqrt{n}}(x^{*})$
for $\sigma_{0}^{2}:=10^{-5}/n^{3}$.\label{line:min}

Draw $x_{0}\sim\textsf{NE-Sampler}\Par{g,\frac{\veps}{\sqrt{n}}}$
with initial dist. $\ncal\Par{x^{*},\frac{\sigma_{0}^{2}}{1+\nu\beta/n}g(x^{*})^{-1}}\cdot\mathbf{1}_{D}$
and target dist. $\mu_{\sigma_{0}^{2}}$.

\tcp{Phase 2 \& 3: Annealing until $\sigma^{2}\leq\nu$}

\While{$\sigma^{2}\leq\nu$}{

Update $\sigma^{2}$ by 
\[
\sigma^{2}\gets\begin{cases}
\sigma^{2}\Par{1+\frac{1}{\sqrt{n}}} & \text{if }\sigma^{2}\leq\frac{\nu}{n}\text{ (Phase 2)}\\
\sigma^{2}\Par{1+\frac{\sigma}{\sqrt{\nu}}} & \text{if }\frac{\nu}{n}\leq\sigma^{2}\leq\nu\text{ (Phase 3)},
\end{cases}
\]

Draw $x_{i+1}\sim\textsf{NE-Sampler}\Par{g,\frac{\veps}{\sqrt{n}}}$
started at $x_{i}$ with target dist. $\mu_{\sigma^{2}}$, and increment
$i$.

}

\tcp{Phase 4: Sampling from $e^{-f}$} 

Draw $x'\sim\textsf{NE-Sampler}\Par{g,\frac{\veps}{\sqrt{n}}}$ started
at $x_{i}$ with target dist. $\pi$.

\end{algorithm2e}

\subsubsection{Well-definedness}

Going forward, we use the following notation: for $\bar{f}(x):=\frac{\nu}{n}f(x)$
\begin{align*}
F(\sigma^{2}) & :=\begin{cases}
\int_{K}\exp\Par{-\frac{\bar{f}(x)+\phi(x)}{\sigma^{2}}}dx & \text{if }\sigma^{2}\leq\frac{\nu}{n},\\
\int_{K}\exp\Par{-f(x)-\frac{\phi(x)}{\sigma^{2}}}dx & \text{if }\frac{\nu}{n}\leq\sigma^{2}\leq\nu.
\end{cases}
\end{align*}
We show that $y^{*}=\arg\min_{y\in K}(\bar{f}+\phi)$ exists in Line~\ref{line:min}
of Algorithm~\ref{alg:IPM-sampling} and that all distributions involved
in the algorithm are indeed integrable. We defer the proof to Section~\ref{proof:IPM-welldefined}.
%
\begin{prop}
\label{prop:annealing-welldefined} Each probability density involved
in the algorithm is integrable.
\end{prop}

 

\subsubsection{Closeness of distributions in sampling IPM}

In this section, we demonstrate that within each phase a probability
distribution $\mu_{\sigma_{i}^{2}}$ serves as a good warm start for
sampling the subsequent distribution $\mu_{\sigma_{i+1}^{2}}$. While
Algorithm~\ref{alg:IPM-sampling} uses as an initial distribution
$\bar{\mu}_{\sigma^{2}}$ that is approximately close to $\mu_{\sigma^{2}}$,
we resolve this discrepancy through a coupling argument referred to
as ``divine intervention''. We refer readers to Remark~\ref{rem:divine-intervention}
and to Proof of Lemma~4.2 in \cite{lovasz2006simulated} for fuller
details.

For the first two phases, closeness of consecutive distributions follow
purely from a property of log-concave distributions, which is independent
of local metrics.
\begin{lem}
[\cite{kalai2006simulated}, Lemma 3.2] \label{lem:adam-logconcave}
For a log-concave function $g$, the function $a\mapsto a^{n}\int g(x)^{a}dx$
is log-concave in $a$.
\end{lem}

In Phase 1, we leverage another fundamental property of log-concave
distributions. It allows us to establish that the Gaussian distribution
truncated over a small Dikin ellipsoid in Phase 1 provides an $O\Par{\Par{\frac{\nu\beta+n}{\nu\alpha+n}}^{n}}$-warm
start for $\mu_{\sigma_{0}^{2}}$. Thus, the $\dw$ which has a log-dependency
on the warmness parameter introduces an additional factor of $n$. 
\begin{lem}
[\cite{lovasz2007geometry}, Lemma 5.16] \label{lem:mostMass-logconcave}
Let $X$ be a random point drawn from a log-concave distribution with
a density function $f:\Rn\to\R$. If $\gamma\geq2$, then 
\[
\P\Par{f(X)\leq e^{-\gamma(n-1)}\max f}\leq\Par{e^{1-\gamma}\gamma}^{n-1}.
\]
\end{lem}

\begin{rem}
If we can show that the $\dw$ has a $\log\log$-dependency through
the \emph{blocking conductance} or \emph{Gaussian isoperimetry}, or
if we utilize a non-Euclidean sampler with a double-log dependency,
we can avoid the additional factor of $n$.
\end{rem}

We defer the proofs for closeness to Section~\ref{proof:IPM-closeness}.
\begin{lem}
[Phase 1] \label{lem:phase1} Let $x^{*}=\arg\min_{x\in K}\Par{\bar{f}+\phi}$.
For $\sigma^{2}=10^{-5}/n^{3}$ and $g=\hess\phi$, let $\mu$ be
the Gaussian distribution $\ncal\Par{x^{*},\frac{\sigma^{2}}{1+\nu\beta/n}g(x^{*})^{-1}}$
truncated over $\dcal_{g}^{3\sigma\sqrt{n}}(x^{*})$ and $\mu_{0}$
the initial distribution used in Phase 2 such that $\frac{d\mu_{0}}{dx}\propto\exp\Par{-\frac{\bar{f}(x)+\phi(x)}{\sigma^{2}}}\cdot\mathbf{1}_{K}(x)$.
Then $\norm{\mu/\mu_{0}}\lesssim\Par{\frac{\nu\beta+n}{\nu\alpha+n}}^{n}$.
\end{lem}

In the following lemmas, we show that within each phase of our algorithm
$\mu_{\sigma_{i}^{2}}$ serves as an $O(1)$-warm start for the following
distribution $\mu_{\sigma_{i+1}^{2}}$. In Phase 2, for $1/n^{3}\lesssim\sigma^{2}\leq\nu/n$
the multiplicative update of $(1+1/\sqrt{n})$ allows us to achieve
an $O(1)$-warm start.
\begin{lem}
[Phase 2] \label{lem:phase2} In Phase 2 (i.e., $\sigma_{i}^{2}\leq\nu/n$
with the update $\sigma_{i+1}^{2}=\sigma_{i}^{2}(1+1/\sqrt{n})$),
a previous distribution $\mu_{i}$ serves as an $O(1)$-warm start
for the next distribution $\mu_{i+1}$, i.e., $\norm{\mu_{i}/\mu_{i+1}}=O(1)$.
\end{lem}

In the large regime of $\nu/n\leq\sigma^{2}\leq\nu$ during Phase
3, we leverage the Brascamp-Lieb inequality to show that the accelerated
update of $(1+\sigma/\sqrt{\nu})$ ensures an $O(1)$-warm start.
Moreover, we employ the same technique along with a limiting argument
to show that in Phase 4 the final distribution of $\mu_{\nu}$ is
an $O(1)$-warm start for the target distribution $\pi$.
\begin{lem}
[Phase 3 and 4] \label{lem:phase34} In Phase 3 (i.e., $\nu/n\leq\sigma_{i}^{2}\leq\nu$
with the update $\sigma_{i+1}^{2}=\sigma_{i}^{2}(1+\sigma_{i}/\sqrt{\nu})$,
a previous distribution $\mu_{i}$ serves as an $O(1)$-warm start
for the next distribution $\mu_{i+1}$, i.e., $\norm{\mu_{i}/\mu_{i+1}}=O(1)$.
In Phase 4, the distribution $\mu$ with $\frac{d\mu}{dx}\propto\exp\Par{-\Par{f(x)+\frac{\phi(x)}{\nu}}}\cdot\mathbf{1}_{K}(x)$
is an $O(1)$-warm start for the target distribution $\pi$ with $\frac{d\pi}{dx}\propto\exp\Par{-f(x)}\cdot\mathbf{1}_{K}(x)$.
\end{lem}


\subsubsection{Proof of Theorem \ref{thm:Dikin-annealing}}

We now provide the proof of Theorem~\ref{thm:Dikin-annealing}, Algorithm~\ref{alg:IPM-sampling}
with the $\dw$ employed for the non-Euclidean sampler. 

\thmDikinannealing*
\begin{proof}
By Theorem~\ref{thm:Dikin}, if the potential $V$ of a target distribution
satisfies $\alpha\hess\phi\preceq\hess V\preceq\beta\hess\phi$, the
mixing time of the $\dw$ is $n(1+\beta)\min(\onu,1/\alpha)\log\frac{\Lambda}{\veps}$.
\begin{itemize}
\item Phase 1: When a target distribution is $\exp\Par{-\frac{\bar{f}+\phi}{\sigma^{2}}}$
with $\sigma^{2}=10^{-5}/n^{3}$ 
\begin{align*}
n^{2}\Par{1+\frac{\frac{\nu}{n}\beta+1}{\sigma^{2}}}\min\Par{\onu,\frac{\sigma^{2}}{\frac{\nu\alpha}{n}+1}}\log\Par{\frac{\nu\beta+n}{\nu\alpha+n}} & \leq n^{2}\frac{\nu\beta+n}{\nu\alpha+n}\log\Par{\frac{\nu\beta+n}{\nu\alpha+n}}\\
 & \leq n^{2}(\kappa+1)\log\kappa.
\end{align*}
\item Phase 2 ($1/n^{3}\lesssim\sigma^{2}\leq\nu/n$): note that we need
$O^{*}(\sqrt{n})$-many iterations to double $\sigma^{2}$. Hence,
in this phase the number of iterations of the $\dw$ with a target
$\exp\Par{-\frac{\bar{f}+\phi}{\sigma^{2}}}$ adds up to 
\begin{align*}
n\Par{1+\frac{\frac{\nu}{n}\beta+1}{\sigma^{2}}}\min\Par{\onu,\frac{\sigma^{2}}{\frac{\nu\alpha}{n}+1}}\cdot\sqrt{n} & \leq n^{1.5}\frac{\nu\beta+n}{\nu\alpha+n}+\sqrt{n}\nu\\
 & \leq n^{1.5}(\kappa+1)+\sqrt{n}\nu.
\end{align*}
\item Phase 3 ($\nu/n\leq\sigma^{2}\leq\nu$): we need $O^{*}\Par{1+\frac{\sqrt{\nu}}{\sigma}}$-many
iterations to double $\sigma^{2}$. Hence, in this phase the total
number of iterations of the $\dw$ with a target $\exp\Par{-\Par{f+\frac{\phi}{\sigma^{2}}}}$
is
\begin{align*}
n\Par{1+\beta+\frac{1}{\sigma^{2}}}\min\Par{\onu,\frac{1}{\alpha+\sigma^{-2}}}\cdot\Par{1+\frac{\sqrt{\nu}}{\sigma}} & \leq n(\kappa+\nu).
\end{align*}
\item Phase 4: the $\dw$ takes $O(n\onu)$ iterations.
\end{itemize}
Adding up all iterations in phases, we need $\otilde{n\max\Par{n(\kappa+1),\nu,\onu}}$
iterations of the $\dw$ in total.
\end{proof}




\section{Self-concordance theory for sampling IPM \label{sec:sc-theory-rules}}

Theorem~\ref{thm:Dikin-annealing} shows that $\gcdw$  running with
a $(\nu,\onu$)-Dikin-amenable metric for exponential distributions
mixes in $\otilde{d\max\Par{d,\nu,\onu}}$ iterations. Since every
log-concave sampling problem can be reduced to an exponential sampling
problem (as shown in \eqref{eq:reduced-problem}), Theorem~\ref{thm:Dikin-annealing}
ensures a poly-time mixing algorithm that utilizes local geometry
if we have a $(\nu,\onu)$-Dikin-amenable metric for the reduced sampling
problem.

This poses a natural question of how to construct such an efficiently
computable Dikin-amenable metric for structured sampling problems.
Suppose that the structured sampling problems assume a Dikin-amenable
metric for each constraint and epigraph of potentials. Motivated by
self-concordance theory of the optimization IPM, we consider the sum
of each barrier (and thus, the sum of metrics) as a candidate for
the metric of the reduced sampling problem. In fact, this choice aligns
seamlessly with the $\dw$. However, obtaining a provable guarantee
of the sampling IPM with the $\dw$ necessitates a comprehensive understanding
not only of self-concordance but also of SSC, SLTSC, SASC, and $\onu$-symmetry
under the addition of barriers (or metrics). 

In this section, we develop a ``calculus'' for combining metrics
for multiple constraints and epigraphs, deriving the resulting theoretical
guarantees (Theorem~\ref{thm:IPM-sampling}). This leads to a consistent
analogy with the work of \citet{nesterov1994interior} for the optimization
IPM.

\subsection{Basic properties: Scaling, addition and closeness}

Self-concordance is a central notion in the theory of interior-point
methods for optimization (we refer interested readers to \citet{nesterov1994interior,nesterov2018lectures}).
We first recall basic properties of self-concordance and then investigate
those of strong self-concordance and lower trace self-concordance,
which are crucial to our analysis.

\paragraph{Self-concordance.}
\begin{lem}
[\citet{nesterov2003introductory}] Let $f_{i}$ be a $\nu_{i}$-self-concordant
function on a convex set $K_{i}\subset\Rd$ for $i\in[2]$, and $\alpha>0$
be a scalar.
\begin{itemize}
\item (Theorem 4.1.1 and 4.2.2) $f_{1}+f_{2}$ is $(\nu_{1}+\nu_{2})$-self-concordant
on $K_{1}\cap K_{2}$.
\item (Corollary 4.1.2) $g=\hess(\alpha f_{1})$ satisfies $\snorm{g(x)^{-1/2}\Dd g(x)[h]\,g(x)^{-1/2}}_{2}\leq\frac{2}{\sqrt{\alpha}}\,\snorm h_{g(x)}$
for $x\in\inter(K_{1}\cap K_{2})$ and $h\in\Rd$.
\item If $f_{1}$ is a $\nu$-self-concordant, then $cf_{1}$ is $(c\nu)$-self-concordant
for $c>1$.
\end{itemize}
\end{lem}

We can extend this to self-concordant matrices as well.
\begin{lem}
\label{lem:sc-addition} Let $g_{i}:\inter(K_{i})\to\psd$ be a PSD
matrix function on a convex set $K_{i}\subset\Rd$ for $i\in[2]$,
and $\alpha>0$ be a scalar.
\begin{itemize}
\item $g_{1}+g_{2}$ is $(\nu_{1}+\nu_{2})$-self-concordant on $K_{1}\cap K_{2}$.
\item If $g_{1}$ is self-concordant, then $\alpha g_{1}$ satisfies $\Dd(\alpha g_{1})(x)[h]\preceq\frac{2}{\sqrt{\alpha}}\,\snorm h_{\alpha g_{1}}(\alpha g_{1})$
for $x\in\inter(K_{1}\cap K_{2})$ and $h\in\Rd$.
\item If $g_{1}$ is $\nu$-self-concordant, then $cg_{1}$ is $(c\nu)$-self-concordant
for $c>1$.
\end{itemize}
\end{lem}

\begin{proof}
Let $\phi_{i}$ be a $\nu_{i}$-self-concordant function counterpart
of $g_{i}$ on $K_{i}$ for $i\in[2]$. Then for $x\in\inter(K_{1}\cap K_{2})$
and $h\in\Rd$
\begin{align*}
\Dd(g_{1}+g_{2})(x)[h] & \preceq2\,\bpar{\snorm h_{g_{1}}g_{1}+\snorm h_{g_{2}}g_{2}}\preceq2\,\bpar{\snorm h_{g_{1}+g_{2}}g_{1}+\snorm h_{g_{1}+g_{2}}g_{2}}=2\,\snorm h_{g_{1}+g_{2}}(g_{1}+g_{2})\,.
\end{align*}
Clearly, $\phi_{1}+\phi_{2}$ is a function counterpart of $g_{1}+g_{2}$.
Thus, $g_{1}+g_{2}$ is a $(\nu_{1}+\nu_{2})$-self-concordant matrix
function on $K_{1}\cap K_{2}$.

For $c>1$, if $g_{1}$ is self-concordant, then $\Dd(cg_{1})(x)[h]\preceq\frac{2}{\sqrt{c}}\,\snorm h_{cg_{1}}(cg_{1})\preceq2\,\snorm h_{cg_{1}}(cg_{1})$,
and its function counterpart $c\phi_{1}$ is $(c\nu)$-self-concordant
by Lemma~\ref{lem:sc-addition}. Hence, $cg_{1}$ is $(c\nu)$-self-concordant.
\end{proof}
The following lemma ensures that the $\dw$ stays inside the convex
body. This lemma was proven only for self-concordant function in \citet[Theorem 5.1.5]{nesterov2018lectures},
but it can be straightforwardly extended to self-concordant matrices
as well.
\begin{lem}
\label{lem:dikin-in-body} $\dcal_{g}^{1}(x)\subset K$ for a convex
set $K$ and self-concordant matrix function $g$ on $K$.
\end{lem}

\begin{proof}
Consider a matrix function $g_{\veps}$ from $\intk$ to $\pd$ defined
by $g_{\veps}(x):=g(x)+\veps I$. It is self-concordant with a function
counterpart $\phi(x)+\frac{\veps}{2}\,\snorm x^{2}$, where $\phi:\intk\to\R$
is a function counterpart of $g$. For fixed $x\in\intk$ and $h\in\Rd$,
let us define a function defined by $\psi(t):=\bpar{h^{\T}g_{\veps}(x+th)\,h}^{-1/2}$
for any feasible $t$. Then,
\[
\psi'(t)=-\frac{\Dd g_{\veps}(x+th)[h^{\otimes3}]}{2\snorm h_{g_{\veps}(x+th)}^{3}}\,,
\]
and the definition of self-concordance leads to $|\psi'(t)|\leq1$.
This function can be defined on the interval $\bpar{-\psi(0),\psi(0)}$
due to $\psi(t)\geq\psi(0)-|t|$ (see \citet[Corollary 5.14]{nesterov2018lectures}).
This implies that $K$ contains the set 
\[
\bbrace{x+th:|t|\leq\psi(0)=\snorm h_{g_{\veps}(x)}^{-1}}=\{x+th:\snorm{th}_{g_{\veps}(x)}\leq1\}\,.
\]
By sending $\veps\to0$, the claim follows.
\end{proof}
The following lemma states that self-concordant metrics are similar
for nearby points.
\begin{lem}
[\citet{nesterov2003introductory}, Theorem 4.1.6] \label{lem:scCloseness}
Given any self-concordant matrix function $g$ on $K\subset\Rd$ and
$x,y\in K$ with $\snorm{x-y}_{g(x)}<1$, we have 
\[
(1-\snorm{x-y}_{g(x)})^{2}g(x)\preceq g(y)\preceq(1-\snorm{x-y}_{g(x)})^{-2}g(x)\,.
\]
\end{lem}


\paragraph{Strong self-concordance.}

Strong self-concordance is additive up to a constant scaling. See
\S\ref{proof:ssc-basic} for the proof.
\begin{lem}
\label{lem:ssc-sum} If $g_{i}$ is a SSC matrix function on $K_{i}$
for $i\in[2]$, then $2\,(g_{1}+g_{2})$ is strongly self-concordant
on $K_{1}\cap K_{2}$. 
\end{lem}

Note that if we add $k$-many strongly self-concordant metrics, then
we need the scaling of $2^{\log_{2}k}=k$. We remark that the factor
of $2$ above might be redundant. Next, we recall an analogue of Lemma~\ref{lem:scCloseness}
for strong self-concordance.
\begin{lem}
[\citet{laddha2020strong}, Lemma 1.2] \label{lem:strongSC-closeness}Given
a strongly self-concordant matrix function $g$ on $K$, and any $x,y\in K$
with $\snorm{x-y}_{g(x)}<1$, 
\[
\snorm{g(x)^{-1/2}\bpar{g(y)-g(x)}\,g(x)^{-1/2}}_{F}\leq(1-\snorm{x-y}_{g(x)})^{-2}\snorm{x-y}_{g(x)}\,.
\]
\end{lem}


\paragraph{Symmetry.}

Recall that $\onu$-symmetry requires two-sided inclusion: the first
part is $\dcal_{g}^{1}(x)\subset K\cap(2x-K)$, and the second part
is $K\cap(2x-K)\subset\dcal_{g}^{\sqrt{\onu}}(x)$. The first part
immediately follows when a metric is induced by a self-concordant
function.
\begin{lem}
\label{lem:symmetricLeftpart} If $\phi$ is a self-concordant function
on $K$, then $\dcal_{g}^{1}(x)\subset K\cap(2x-K)$ for $g=\hess\phi$
and $x\in K$.
\end{lem}

\begin{proof}
Lemma~\ref{lem:dikin-in-body} ensures that $y\in K$ whenever $y\in\dcal_{g}^{1}(x)$.
Then $2x-y\in\dcal_{g}^{1}(x)$ and thus $2x-y\in K$. It implies
that $y\in2x-K$.
\end{proof}
When a metric is induced by a self-concordant barrier with a barrier
parameter $\nu$, it holds that $\onu=\O(\nu^{2})$.
\begin{lem}
\label{lem:bound-symmetry} For a self-concordant barrier $\phi$
with a barrier parameter $\nu$ on $K$ and $g=\hess\phi$, it follows
that $\onu=\O(\nu^{2})$.
\end{lem}

\begin{proof}
By \citet[Theorem 4.2.5]{nesterov2003introductory}, for any $x,y\in K$
with $\grad\phi(x)\cdot(y-x)\geq0$ it follows that $\snorm{y-x}_{g(x)}\leq\nu+2\sqrt{\nu}$.
Now, let $x\in K$ and $y\in K\cap(2x-K)$. The latter implies that
$y-x=x-z$ for some $z\in K$. 

If $\grad\phi(x)\cdot(y-x)\geq0$, then $\snorm{y-x}_{g(x)}\leq\nu+2\sqrt{\nu}.$
If $\grad\phi(x)\cdot(y-x)<0$, then $\grad\phi(x)\cdot(z-x)>0$ and
thus $\snorm{y-x}_{g(x)}=\snorm{z-x}_{g(x)}\leq\nu+2\sqrt{\nu}$.
From these two cases, it holds in general that $\snorm{y-x}_{g(x)}\leq\nu+2\sqrt{\nu}$
and thus $K\cap(2x-K)\subset\dcal_{g}^{\nu+2\sqrt{\nu}}(x)$. By Lemma~\ref{lem:symmetricLeftpart},
$\dcal_{g}^{1}(x)\subset K\cap(2x-K)$ and thus $\onu=\mc O(\nu^{2})$.
\end{proof}
For affine constraints $Ax\geq b$, the first inclusion above has
a useful equivalent description as follows:
\begin{lem}
\label{lem:symmforPolytope} Let $x\in K=\{Ax>b\}$. It holds that
$y\in K\cap(2x-K)$ if and only if $\snorm{A_{x}(y-x)}_{\infty}\leq1$.
\end{lem}

\begin{proof}
For $y\in K$, we have $Ay>b$ and thus $s_{x}=Ax-b>A(x-y)$ (elementwise
inequality). As $s_{x}>0$, we have $A_{x}(x-y)\leq1$. When $y\in(2x-K)$,
we can write $y=2x-z$ for some $z\in K$. Note that
\[
A(x-y)=A(z-x)>b-Ax=-s_{x}\,,
\]
and thus $A_{x}(x-y)\geq-1$. Therefore, $\snorm{A_{x}(y-x)}_{\infty}\leq1$.
\end{proof}
\begin{lem}
\label{lem:symmScaling} For $\alpha\geq1$, if $g$ is $\onu$-symmetric,
then $\alpha g$ is $\alpha\onu$-symmetric.
\end{lem}

Symmetry parameters and self-concordance parameters are additive.
\begin{lem}
\label{lem:symmetry-addition} If a PSD matrix function $g_{i}$ is
$\onu_{i}$-symmetric on $K_{i}$ for $i\in[2]$, then $g_{1}+g_{2}$
is $(\onu_{1}+\onu_{2})$-symmetric on $K_{1}\cap K_{2}$.
\end{lem}

\begin{proof}
For $g:=g_{1}+g_{2}$, let $y\in\dcal_{g}^{1}(x)$. It implies $y\in\dcal_{g_{1}}^{1}(x)\cap\dcal_{g_{2}}^{1}(x)$
and so $y\in K_{i}\cap(2x-K_{i})$. Due to $\cap_{i}\bpar{K_{i}\cap(2x-K_{i})}=K\cap(2x-K)$,
we have $y\in K\cap(2x-K)$ and so $\dcal_{g}^{1}(x)\subset K\cap(2x-K)$.

Now let $y\in K\cap(2x-K)$. It is obvious that $y\in K_{i}\cap(2x-K_{i})$
for $i=1,2$, and thus
\[
(y-x)^{\T}g_{1}(x)(y-x)\leq\nu_{1}\,,\qquad\text{and}\qquad(y-x)^{\T}g_{2}(x)(y-x)\leq\nu_{2}\,.
\]
By adding up these two, it follows that $\snorm{y-x}_{g(x)}^{2}\leq\nu_{1}+\nu_{2}$.
\end{proof}

\paragraph{Lower trace self-concordance.}

It readily follows that (strongly) LTSC holds under scaling by a scalar
greater than or equal to $1$. 

We provide a useful sufficient condition under which the sum of PSD
matrix functions is LTSC.
\begin{lem}
\label{lem:sltsc-additive} For a PSD matrix function $g_{i}$ on
$K_{i}$, let $g:=\sum_{i}g_{i}$ be PD on $\bigcap_{i}K_{i}$. If
$g_{i}$ is SLTSC on $K_{i}$, then $g$ is LTSC on $\bigcap_{i}K_{i}$.
\end{lem}

We note that $\Dd^{2}g_{i}(x)[h,h]\succeq0$ is a stronger condition
than $\tr\bpar{g(x)^{-1}\Dd^{2}g_{i}(x)[h,h]}\geq-\snorm h_{g_{i}(x)}^{2}$.
Thus, a special case of the lemma is that if $\Dd^{2}g_{1}[h,h]\succeq0$
and $\Dd^{2}g_{2}[h,h]\succeq0$, then $g_{1}+g_{2}$ is LTSC. Note
that this condition is \emph{additive.}

We also find that highly self-concordance is a handy sufficient condition
by which one can establish strongly lower trace self-concordance,
whose proof is deferred to \S\ref{proof:ltsc-basic}.
\begin{lem}
\label{lem:hsc-to-sltsc} For $K\subset\Rd$, let $\bar{g}:\intk\to\psd$
be a HSC matrix function, and define another matrix function by $g:=d\bar{g}$
on $K$. Then $g$ is SLTSC.
\end{lem}


\paragraph{Average self-concordance.}

Just as (S)LTSC, (S)ASC still holds under scaling by a scalar greater
than or equal to $1$. Also, the definition of SASC immediately leads
to the following additive condition:
\begin{lem}
\label{lem:sasc-additive} For a PSD matrix function $g_{i}$ on $K_{i}$
for $i\in[m]$, let $m=\mc O(1)$ and $g:=\sum_{i=1}^{m}g_{i}$ be
PD on $\bigcap_{i}K_{i}$. If $g_{i}$ is SASC on $K_{i}$, then $g$
is ASC on $\bigcap_{i}K_{i}$.
\end{lem}

\begin{proof}
Fix $\veps>0$. Each $g_{i}$ invokes $r_{i}(\veps)$ such that if
$r\leq r_{i}(\veps/m)$, then 
\[
\P_{z}\Bpar{\snorm{z-x}_{g_{i}(x)}^{2}-\snorm{z-x}_{g_{i}(x)}^{2}\leq\frac{2\veps}{m}\,\frac{r^{2}}{d}}\geq1-\frac{\veps}{m}\,.
\]
If $r\leq\bar{r}(\veps):=\min_{i}\,r_{i}(\veps/m)$, then the union
bound leads to ASC of $\sum g_{i}$ on $\bigcap_{i}K_{i}$.
\end{proof}
When does SASC hold? It is implied in \citet{narayanan2016randomized}
that HSC implies SASC. For completeness, we provide the proof in \S\ref{proof:sasc-basic}.
\begin{lem}
[HSC to SASC] \label{lem:hsc-to-sasc} If $\phi:\intk\to\R$ is HSC,
then $d\phi$ is SASC.
\end{lem}


\subsection{Collapse and embedding: Lifting up SSC, SLTSC, and SASC}

SSC, (S)LTSC, (S)ASC of a local metric do not carry over into an extended
space in the reduced sampling problem. For instance, SSC assumes the
invertibility of the local metric, which may become singular in the
extended space. To address this challenge, we introduce the notions
of \emph{collapse} and \emph{embedding}, based on which we can pass
those properties from the original sampling problem to the reduced
problem.
\begin{defn}
\label{def:sc-along-subspace} Let $K$ and $K'$ be convex sets in
$\Rd$ and in $\R^{m}$ with $d\leq m$, respectively. Let $g:\intk\to\psd$
be a PSD matrix function.
\begin{itemize}
\item We say $g$ is \emph{collapsed onto a linear subspace} $W\subset\Rd$
if $\inner{u,v}_{g(x)}=\inner{P_{W}u,P_{W}v}_{g(x)}$ for any $x\in\intk$
and $u,v\in\Rd$ where $P_{W}$ is the orthogonal projection onto
$W$.
\begin{itemize}
\item In other words, for an orthonormal basis $\{u_{1},\dots,u_{k}\}$
of $W$ there exists the PSD matrix function $g_{W}:\intk\to\mathbb{S}_{+}^{k}$
such that $\inner{e_{i},e_{j}}_{g_{W}(x)}=\inner{u_{i},u_{j}}_{g(x)}$
for $i,j\in[k]$ (i.e., $g_{W}(x)=U^{\T}g(x)U$ where the columns
of $U\in\R^{d\times k}$ are $\Brace{u_{1},\dots,u_{k}}$). 
\end{itemize}
\item For $g$ collapsed onto $W$, we say
\begin{itemize}
\item $g$ is PD along $W$ if $g_{W}$ is PD. In other words, $\norm h_{g(x)}=0$
implies $h\perp W$.
\item $g$ is SSC along $W$ if $g$ is a self-concordant matrix function
and $g_{W}\succ0$ satisfies
\[
\snorm{g_{W}(x)^{-1/2}\Dd g_{W}(x)[h]\,g_{W}(x)^{-1/2}}_{F}\leq2\snorm h_{g}\quad\text{for any }x\in\intk\ \text{and}\ h\in\Rd\,.
\]
\end{itemize}
\item \emph{Embedding} $\bar{g}$ of $g$ into $K'$
\begin{itemize}
\item Let $P:\R^{m}\to\Rd$ be the projection onto the set of coordinates
appearing in the variable $x$ of $g$. The embedding of $g$ onto
$K'$ is a PSD matrix function $\bar{g}(y):\inter(K')\to\mathbb{S}_{+}^{m}$
such that $\inner{u,v}_{\bar{g}(y)}=\inner{Pu,Pv}_{g(P(y))}$.
\end{itemize}
\end{itemize}
\end{defn}

We note that these notions are well-defined independently of the choice
of an orthonormal basis of $W$. The proof can be found in \S\ref{proof:collapse-embedding-welldefined}.
\begin{prop}
\label{prop:collapse-well-defined} Let $K\subset\Rd$ be convex and
$g:\intk\to\psd$ a PSD matrix function collapsed onto a subspace
$W\subset\Rd$. Then PD and SSC along $W$ are well-defined (i.e.,
the condition for each property holds for any orthonormal basis of
$W$).
\end{prop}


\paragraph{Affine transformation.}

Using these notions, we can make it precise that an inverse mapping
of affine transformations preserves SSC. We begin with a barrier version
and subsequently extend it to a matrix-function version. The detailed
proofs are deferred to \S\ref{proof:collap-affine}.
\begin{lem}
\label{lem:linear-trans} Let $T:\Rd\to\R^{m}$ be a linear operator
defined by $T(x)=Ax+b$ for $A\in\R^{m\times d}$ and $b\in\R^{m}$.
Let $\phi(y):\intk\subset\R^{m}\to\R$ be a self-concordant barrier
for $K$ and define $\psi(x):=\phi(T(x))=\phi(y)$ on $\bar{K}:=T^{-1}K\subset\Rd$.
\begin{itemize}
\item If $\phi$ is a $(\nu,\onu)$-self-concordant barrier for $K$, so
is $\psi$ for $\bar{K}$.
\item If $\Dd^{4}\phi(y)[v,v]\succeq0$ for $y\in\intk$ and $v\in\R^{m}$,
then $\Dd^{4}\psi(x)[u,u]\succeq0$ for $x\in\inter(\bar{K})$ and
$u\in\Rd$.
\item If $\phi$ is HSC, so is $\psi$.
\end{itemize}
\end{lem}

\begin{lem}
\label{lem:linear-trans-matrix} Let $g:\intk\subset\R^{m}\to\mathbb{S}_{+}^{m}$
be a self-concordant matrix function and $T(x)=Ax+b$ with $A\in\R^{m\times d}$
and $b\in\R^{m}$ be a linear operator. Let $\bar{g}(x):=A^{\T}g(Tx)A$
be a PSD matrix function from $\bar{K}:=T^{-1}K\subset\Rd$ to $\psd$.
\begin{itemize}
\item If $g$ is $(\nu,\onu)$-self-concordant barrier, so is $\bar{g}$
for $\bar{K}$.
\item If $g$ is SSC, then $\bar{g}$ is SSC along $W=\rowspace(A)$.
\item If $\Dd^{2}g(y)[h,h]\succeq0$ for $y\in\intk$ and $h\in\R^{m}$,
then $\Dd^{2}\bar{g}(x)[\bar{h},\bar{h}]\succeq0$ for $x\in\inter(\bar{K})$
and $\bar{h}\in\Rd$.
\item If $A$ is invertible and $g$ is SLTSC, then $\bar{g}$ is SLTSC.
\item If $A$ is invertible and $g$ is SASC, then $\bar{g}$ is SASC.
\end{itemize}
\end{lem}

Intuitively, embedding should not affect self-concordance and symmetry
parameter, which is indeed the case.
\begin{cor}
\label{cor:embedding-scness} Assume $K\subset\Rd$ is embeddable
into $K'\subset\R^{m}$. If $g:\intk\to\psd$ is a $(\nu,\onu)$-self-concordant
matrix function, then its embedding $\bar{g}:\inter(K')\to\mathbb{S}_{+}^{m}$
is a $(\nu,\onu)$-self-concordant matrix function.
\end{cor}

\begin{proof}
Since $K$ can be embedded into $K'$, there exists a projection matrix
$P\in\{0,1\}^{d\times m}$ such that $\bar{g}(y)=P^{\T}g(Py)P$ with
$x=Py\in\intk$ and $y\in\inter(K')$. As we can view $\bar{g}$ as
a matrix function induced by the inverse of the linear map $x=Py$,
Lemma~\ref{lem:linear-trans-matrix} shows that $\bar{g}$ is a $(\nu,\onu)$-self-concordant
matrix function for $K'=P^{-1}K$. 
\end{proof}

\paragraph{Lifting up SSC, SLTSC, and SASC via embedding.}

In reduction to the exponential sampling problem, passing essential
properties (e.g., SSC, SLTSC, and SASC) of metrics from the original
space to the extended space poses technical issues. We address these
issues in the following two lemmas, whose proofs are deferred to \S\ref{proof:lifting-ssc}.

As mentioned earlier, SSC in the original space does not automatically
imply SSC for its embedding $\bar{g}$, as SSC assumes invertibility.
However, there is a useful method for extending SSC from the original
space to the extended space.
\begin{lem}
\label{lem:embedding-ssc} For convex $K\subset\Rd$, let $g:\intk\to\psd$
be SSC along a subspace $W\subset\Rd$, and assume $K$ is embeddable
into convex $K'\subset\R^{m}$ with $m\geq d$. For the embedding
$\bar{g}:\inter(K')\to\mathbb{S}_{+}^{m}$ of $g$ into $K'$, it
holds that $\bar{g}+\veps I_{m}$ is SSC on $K'$ for any $\veps>0$.
\end{lem}

When extending SLTSC and SASC to the embedding space, we encounter
a different subtlety. The conditions in SLTSC and SASC of $\bar{g}$
consider every PSD matrix functions $g'$ such that $\bar{g}+g'$
is invertible in the extended space $\bar{K}$. However, the embedding
$\bar{g}$ of $g$ is collapsed onto the subspace corresponding to
the original space $K$. As SLTSC and SASC convolve $\bar{g}$ and
$g'$ by considering $(\bar{g}+g')^{-1}$ in their formulations, it
is not evident whether SLTSC and SASC can be transferred to the extended
space $\bar{K}$ from the original space $K$. However, by employing
with Schur complements we can show that these properties can indeed
carry over into the extended space.
\begin{lem}
\label{lem:embedding-sltsc} For convex $K\subset\Rd$, let $g:\intk\to\psd$
is SLTSC, and assume $K$ is embeddable into convex $K'\subset\R^{m}$
with $m\geq d$. Then its embedding $\bar{g}:\inter(K')\to\mathbb{S}_{+}^{m}$
is also SLTSC. The same is true for SASC.
\end{lem}


\subsection{Proof of Theorem \ref{thm:IPM-sampling}}

With our understanding of how to combine properties of barriers for
constraints and epigraphs, we are prepared to prove Theorem~\ref{thm:IPM-sampling}.
Let us revisit the reduced sampling problem in \eqref{eq:reduced-problem}:
\begin{align*}
\text{sample } & y\sim\tilde{\pi}\propto\exp\Bpar{-\inner{(\underbrace{0,\dots,0}_{d\text{ times}},\underbrace{1,\dots,1}_{I\text{ times}}),\cdot}}\\
\text{s.t. } & y\in\bigcap_{i=1}^{I}E_{i}\cap\underbrace{\bigcap_{j=1}^{J}K_{j}}_{\eqqcolon:K}\eqqcolon K'\,,
\end{align*}
where $E_{i}:=\bbrace{y=(x,t_{1},\dots,t_{I})\in\R^{d+I}:f_{i}(x)\leq y_{d+i}}$
for a proper closed convex function $f_{i}$ and $i\in[I]$, and $K_{j}:=\bbrace{y=(x,t_{1},\dots,t_{I})\in\R^{d+I}:h_{j}(x)\leq0}$
for a closed convex function $h_{j}$ and $j\in[J]$, and $K$ has
non-empty interior.

We begin with a useful geometric property of $K'$.
\begin{lem}
If the original sampling problem \eqref{eq:problem} is well-defined,
then the extended convex region $K'$ in the reduced sampling problem
\eqref{eq:reduced-problem} has non-empty interior and no straight
line.
\end{lem}

\begin{proof}
Since $f_{i}$ and $h_{j}$ are closed and convex, $K'$ is convex
and closed. Since $f_{i}$ is continuous on $\inter(K)$ due to convexity
(see \citet[Theorem 10.1]{rockafellar1997convex}), its epigraph has
non-empty interior. Thus, $K'$ has non-empty interior.

Since $K'$ is closed and convex, it can be written as $K'=\bigcap_{i}H_{i}$
where $H_{i}=\{x:a_{i}^{\T}x\geq b_{i}\}$ is any halfspace containing
$K'$. Suppose $K'$ contains a straight line $\ell:=\{p+th:t\in\R\}$
for some $p,h\in\Rd$. Then $\ell\subset H_{i}$ for any $i$, and
thus $\ell$ must be parallel to any halfspace $H_{i}$ (i.e., $h\perp a_{i}$). 

Fix $y\in\inter(K')$. The translated line $\ell_{y}$ of $\ell$
containing $y$ is still included in $H_{i}$ for all $i$. As $y\in\inter(K')$,
the distance from $y$ to $\de H_{i}$ is bounded lower by $\delta>0$
for all $i$. Hence, $\ell_{y}+B_{\delta}$ is fully contained in
$H_{i}$ and thus in $K'$.

Clearly, integration of the exponential distribution along the fiber
$\ell_{y}$ is infinite. Since $K'$ contains the cylinder $\ell_{y}+B_{\delta}$,
integration of the exponential distribution over $K'$ must be infinite,
leading to contradiction.
\end{proof}
The following is the extension of \citet[Theorem 5.1.6]{nesterov2018lectures}
to self-concordant matrix functions, which implies invertibility of
Dikin-amenable metrics in the reduced problem.
\begin{lem}
\label{lem:nondegenerate-no-straightline} For convex $K\subset\Rd$
containing no straight line, a self-concordant matrix function $g:\intk\to\psd$
is non-degenerate on $K$.
\end{lem}

\begin{proof}
Suppose $\snorm h_{g(x)}=0$ for some $0\neq h\in\Rd$ and $x\in\intk$.
Clearly, the line $x+th$ for $t\in\R$ is contained in $\dcal_{g}^{1}(x)$.
As $\dcal_{g}^{1}(x)\subset K$ due to Lemma~\ref{lem:dikin-in-body},
it implies that $K$ contains a straight line $x+th$, which leads
to contradiction. 
\end{proof}
\thmIPMsampling*
\begin{proof}
First of all, $\bar{g}_{i}^{e}$ is $(\nu_{i},\onu_{i})$-self-concordant
(Corollary~\ref{cor:embedding-scness}), and SLTSC and SASC on $K'$
(Lemma~\ref{lem:embedding-sltsc}). For fixed $\veps>0$, $\bar{g}_{i}^{e}+\veps I$
is SSC by Lemma~\ref{lem:embedding-ssc}. We can make similar arguments
for $\bar{g}_{j}^{c}$ regarding self-concordance, symmetry, SLTSC,
SASC, and SSC. Hence, $g+(I+J)\veps I$ is SSC by Lemma~\ref{lem:ssc-sum}.
Since $g$ is self-concordant on $K'$ by Lemma~\ref{lem:sc-addition}
and $K'$ contains no straight line, $g$ is PD by Lemma~\ref{lem:nondegenerate-no-straightline}.
Sending $\veps$ to $0$, we can obtain SSC of $g$. LTSC and ASC
of $g$ follows from Lemma~\ref{lem:sltsc-additive} and \ref{lem:sasc-additive}.
The symmetry parameter of $g$ follows from Lemma~\ref{lem:symmetry-addition}.
\end{proof}

\subsection{Direct product}

For $i\in[m]$ and domain $E_{i}\subset\R^{d_{i}}$, let $g_{i}(x_{i}):\inter(E_{i})\to\mathbb{S}_{++}^{d_{i}}$
be a self-concordant matrix. For $l:=\sum_{i}d_{i}$ and $E:=\prod_{i}E_{i}$,
we define a self-concordant matrix $g$ on $E\subset\R^{l}$ with
block diagonals being $g_{i}$. To be precise, we can write
\begin{align*}
g(x) & =g(x_{1},\dots,x_{m}):=\sum_{i}\bar{g}_{i}(x)\,,
\end{align*}
where $\bar{g}_{i}:\R^{l}\to\mathbb{S}_{+}^{l}$ is a matrix function
whose entry is all zero but the $i$-th block diagonal being $g_{i}$.

When handling the direct product of domains, it is common for each
domain to have an $\mc O(1)$-dimension. In such cases, scaling the
barriers by dimension worsens mixing time at most constant factors
while making the barriers SSC and SLTSC. We defer the proofs to \S\ref{proof:direct-ssc-sltsc}.
\begin{lem}
[SSC  under direct product] \label{lem:ssc-direct} For open $E_{i}\subset\R^{d_{i}}$,
let $g_{i}:E_{i}\to\mathbb{S}_{++}^{d_{i}}$ be SC. Then $g:=\sum d_{i}\bar{g}_{i}$
defined on $\prod E_{i}$ is SSC.
\end{lem}

\begin{lem}
[SLTSC  under direct product] \label{lem:sltsc-direct} For open
$E_{i}\subset\R^{d_{i}}$, let $g_{i}:E_{i}\to\mathbb{S}_{++}^{d_{i}}$
be HSC. Then $g:=\sum d_{i}\bar{g_{i}}$ defined on $\prod E_{i}$
is SLTSC.
\end{lem}


\subsection{Inverse images under non-linear mappings}

\citet{nesterov1994interior} introduced the notion of \emph{compatibility}
with a convex domain while constructing a self-concordant barrier
for a wider class of structured constraints. We generalize this notion
to the fourth order, by which we can easily construct a SSC, SLTSC,
and SASC barrier. For a convex cone $K$, we use $a\leq_{K}b$ to
denote $b-a\in K$.
\begin{defn}
[Compatibility] Let $\beta,\gamma\geq0$. Let $K$ be a convex cone
in $\R^{m}$ and $\Gamma$ be a closed convex domain in $\Rd$. A
mapping $\acal:\inter(\Gamma)\to\R^{m}$ of class $C^{4}$ is called
$(K,\beta,\gamma)$-compatible with the domain $\Gamma$ if 
\begin{itemize}
\item $\acal$ is concave with respect to $K$. That is, $t\acal(x)+(1-t)\,\acal(y)\leq_{K}\acal(tx+(1-t)\,y)$
for all $t\in[0,1]$ and $x,y\in\inter(\Gamma)$. Equivalently, $-\Dd^{2}\acal(x)[h,h]\in K$
for any $x\in\inter(\Gamma)$ and $h\in\R^{m}$.
\item For any $x\in\inter(\Gamma)$, $y\in\Gamma\cap(2x-\Gamma)$, and $h=y-x$,
it holds that 
\begin{align*}
\beta\Dd^{2}\acal(x)[h,h] & \leq_{K}\Dd^{3}\acal(x)[h,h,h]\leq_{K}-\beta\Dd^{2}\acal(x)[h,h]\,,\\
\gamma\Dd^{2}\acal(x)[h,h] & \leq_{K}\Dd^{4}\acal(x)[h,h,h,h]\leq_{K}-\gamma\Dd^{2}\acal(x)[h,h]\,.
\end{align*}
\end{itemize}
\end{defn}

\begin{example}
\label{exa:useful-criteria} An affine mapping is $(\{0\},0,0)$-compatible
with any closed convex domain. We note that a function that is $(\R_{+},\beta,\gamma)$-compatible
with $\R_{+}$ is a $C^{4}$-smooth concave real-valued function $f:(0,\infty)\to\R$
such that for any $t>0$,
\begin{align*}
|f'''(t)| & \leq-\frac{\beta}{t}\,f''(t)\quad\text{and}\quad|f^{(4)}(t)|\leq-\frac{\gamma}{t^{2}}\,f''(t)\,.
\end{align*}
\begin{itemize}
\item Let $0<p\leq1$. Then the function of $f(t)=t^{p}$ is $(\R_{+},2-p,(2-p)\,(3-p))$-compatible
with $\R_{+}$.
\item $f(t)=\log t$ is $(\R_{+},2,6)$-compatible with $\R_{+}$.
\end{itemize}
The following lemma is an extension of \citet[Lemma 5.1.3]{nesterov1994interior}
to our fourth-order compatibility.
\end{example}

\begin{lem}
\label{lem:extension-compatibility} Let $K,K_{1},K_{2}$ be convex
cones in $\R^{m},\R^{m_{1}},\R^{m_{2}}$ respectively.
\begin{itemize}
\item If $\acal:\inter(\Gamma)\to\R^{m}$ is $(K,\beta,\gamma)$-compatible
with $\Gamma$ and $K\subset K'$ is a closed convex cone in $\R^{m}$,
then $\acal$ is $(K',\beta,\gamma)$-compatible with $\Gamma$.
\item If $\acal_{i}:\inter(\Gamma_{i})\to\R^{m_{i}}$ is $(K_{i},\beta_{i},\gamma_{i})$-compatible
with $\Gamma_{i}$ for $i=1,2$, then $\acal:\inter(\Gamma_{1}\times\Gamma_{2})\to\R^{m_{1}}\times\R^{m_{2}}$
mapping $(x,y)\to(\acal_{1}(x),\acal_{2}(y))$ is $(K_{1}\times K_{2},\max(\beta_{1},\beta_{2}),\max(\gamma_{1},\gamma_{2}))$-compatible
with $\Gamma_{1}\times\Gamma_{2}$.
\end{itemize}
\end{lem}

We now introduce a main result in this section (see \S\ref{proof:inverse-non-linear}).
To begin with, we recall that for a closed convex domain $G\subset\Rd$
the \emph{recessive cone} $R(G)$ of $G$ is $\{h\in\Rd:x+th\in G\ \text{for all }x\in G\text{ and }t>0\}$.
\begin{lem}
\label{lem:compatible} Let $G$ be a closed convex domain in $\R^{m}$,
$F$ be a highly $\theta$-self-concordant barrier for $G$, $\Gamma$
be a closed convex domain in $\Rd$, and $\Pi$ be a highly $\nu$-self-concordant
barrier for $\Gamma$. Let $\acal$ be a $(K,\beta,\gamma)$-compatible
with $\Gamma$, where $K$ is a ray contained in the recessive cone
$R(G)$. Assume that $\acal(\inter(\Gamma))\cap G\neq\emptyset$.
\begin{itemize}
\item The set $G^{+}=\overline{\inter(\Gamma)\cap\acal^{-1}\bpar{\inter(G)}}$
is a closed convex domain in $\Rd$.
\item For $\delta=\max\Par{\beta,\gamma,2}$, the function $\Psi(x)=F(\acal(x))+\delta^{2}\,\Pi(x)$
is a $(\theta+\delta^{2}\nu)$-self-concordant barrier for $G^{+}$.
\item $\Psi$ is highly self-concordant.
\end{itemize}
\end{lem}

Using this result, we can obtain a useful tool in establishing lower
trace self-concordance of a barrier for the direct product of structured
sets.
\begin{lem}
\label{lem:tool-concave} Let $f$ be a $C^{4}$ concave function
on $\{t>0\}$ such that $|f'''(t)|\leq\frac{\beta}{t}\,|f''(t)|$
and $|f^{(4)}(t)|\leq\frac{\gamma}{t^{2}}\,|f''(t)|$ for $t>0$.
Then the function 
\[
F(t,x)=-\log\bpar{f(t)-x}-\max(4,\beta^{2},\gamma^{2})\,\log t
\]
is a highly $(1+\max(4,\beta^{2},\gamma^{2}))$-self-concordant barrier
for the two dimensional convex domain
\[
G_{f}=\overline{\{(t,x)\in\R^{2}:t>0,\,x\leq f(t)\}}\,.
\]
\end{lem}

\begin{proof}
From the discussion in Example~\ref{exa:useful-criteria}, the map
$f(t):(0,\infty)\to\R$ is $(\R_{+},\beta,\gamma)$-compatible with
$\R_{+}$. Clearly, the identity map from $\R$ to $\R$ is $(\{0\},0,0)$-compatible
with $\R$. Hence by Lemma~\ref{lem:extension-compatibility}-(2)
implies that the map $\acal:\R_{+}\times\R\to\R^{2}$ defined by $\acal(t,x)=(f(t),x)$
is $(\{0\}\times\R_{+},\beta,\gamma)$-compatible with $\R_{+}\times\R$. 

Now observe that $G_{f}$ can be written as $\acal^{-1}\bpar{\{(t,x):x\leq t\}}$
and that $K=\{0\}\times\R_{+}$ is a ray contained in the recessive
cone $R(G)$ for $G:=\{(t,x):x\leq t\}$. By applying Lemma~\ref{lem:compatible}
to the highly $1$-self-concordant barriers $F(t,x)=-\log(t-x)$ for
$G$ and $\Phi(t,x)=-\log t$ for $\R_{+}\times\R$, it follows that
$F$ is is a highly $(1+\max(4,\beta^{2},\gamma^{2}))$-self-concordant
barrier for $G_{f}$.
\end{proof}
We can prove a similar result for a convex $f$ as follows:
\begin{lem}
\label{lem:tool-convex} Let $f$ be a $C^{4}$ convex function on
$\{x>0\}$ such that $|f'''(x)|\leq\frac{\beta}{x}\,f''(x)$ and $|f^{(4)}(x)|\leq\frac{\gamma}{x^{2}}\,f''(x)$
for $x>0$. Then the function 
\[
F(t,x)=-\log\bpar{t-f(x)}-\max(4,\beta^{2},\gamma^{2})\,\log x
\]
is a highly $(1+\max(4,\beta^{2},\gamma^{2}))$-self-concordant barrier
for the two dimensional convex domain
\[
G_{f}=\overline{\{(t,x)\in\R^{2}:x>0,\,t\geq f(x)\}}\,.
\]
\end{lem}

Its proof follows from applying Lemma~\ref{lem:tool-concave} to
the image of $G_{f}$ under the map $(t,x)\to(-x,t)$.


\global\long\def\vec{\textup{\textsf{vec}}}%
\global\long\def\svec{\textup{\textsf{svec}}}%


\section{Structured densities and constraint families \label{sec:handbook-barrier}}

In order to obtain a mixing-time bound of the $\dw$ for the reduced
problem, a concrete understanding of properties and parameters of
barriers for $K_{i}$ and $K_{j}$ is essential. To this end, we revisit
self-concordant barriers for structured convex constraints and level
sets, examining the required scaling factors which ensure those properties.

\subsection{Linear constraints}

Consider a set of linear constraints: $K=\{x\in\Rd:Ax\geq b\}$ for
$A\in\R^{m\times d}$ and $b\in\R^{m}$, where $A$ has no all-zero
rows. We use $s_{x}:=Ax-b$ to denote the slack at $x$, and $A_{x}:=S_{x}^{-1}A$
to denote the constraints normalized by the slack, where $S_{x}:=\Diag(s_{x})$
is the diagonalization of the slack.

We now introduce three barriers (and metrics) for handling the linear
constraints.

\paragraph{Logarithmic barrier.}

The logarithmic barrier $\phi_{\log}(x):=-\sum_{i=1}^{m}\log(a_{i}^{\T}x-b_{i})$
is the simplest self-concordant barrier for linear constraints. We
refer readers to \S\ref{proof:linear-log-barrier} for gentle introduction
to the log-barriers. As seen below, we demonstrate that the metric
induced by the logarithmic barrier has $\nu,\onu=m$ and requires
no scaling to achieve SSC, SLTSC, and SASC.
\begin{lem}
[Logarithmic barrier]\label{lem:log-barrier} For a closed convex
$K=\{x\in\Rd:Ax\geq b\}$ with $A\in\R^{m\times d}$ and $b\in\R^{m}$,
let $\phi_{\log}(x)=-\sum_{i=1}^{m}\log(a_{i}^{\T}x-b_{i})$ and define
$g(x):=\hess\phi_{\log}(x)=A_{x}^{\T}A_{x}$.
\begin{itemize}
\item $\nu=m$ (\citet{nesterov1994interior}).
\item SSC along $\rowspace(A)$ and $\onu=m$ (Lemma~\ref{lem:paramsBarrier}).
\item $\Dd^{2}g(x)[h,h]\succeq0$ for any $h\in\Rd$ (so SLTSC) (Claim~\ref{claim:diffLogBarrier}).
\item SASC (Lemma~\ref{lem:logBarrier-SASC}).
\end{itemize}
\end{lem}


\paragraph{Vaidya metric.}

In sampling over a polytope $K$, the number $m$ of constraints is
assumed to be greater than the ambient dimension $d$. Given that
the mixing time of the $\dw$ for uniform sampling is $\otilde{d\onu}=\otilde{dm}$,
a larger $m$ leads to a worse mixing time. Is there a self-concordant
barrier that has a better dependence on $m$ for its self-concordance
and symmetry parameters, without compromising SSC, SLTSC, and SASC?

Let us recall the \emph{leverage score} first and move onto such improved
self-concordant barriers. For a full-rank matrix $A\in\R^{m\times d}$
with $m\geq d$, we recall that $P(A)=A(A^{\T}A)^{-1}A^{\T}$ is the
orthogonal projection matrix onto the column space of $A$, and the
leverage scores of $A$ is $\sigma(A)=\diag(P(A))\in\R^{m}$. We let
$\Sigma(A):=\Diag(\sigma(A))=\Diag(P(A))$ and $P^{(2)}(A)=P(A)\circ P(A)$,
where $P(A)\circ P(A)$ is the Hadamard product of size $d\times d$
defined by $(P(A)\circ P(A))_{ij}=[P(A)]_{ij}^{2}$.

\citet{vaidya1996new} introduced the \emph{volumetric barrier} for
$K$ defined by
\[
\phi_{\vol}=\half\,\log\det(\hess\phi_{\log})=\half\,\log\det(A_{x}^{\T}A_{x})\,.
\]
Then the Hessian of $\phi_{\vol}$ can be written as
\[
\hess\phi_{\vol}=A_{x}^{\T}(3\Sigma_{x}-2P_{x}^{(2)})A_{x}\,,
\]
where $\Sigma_{x}=\Diag(\sigma(A_{x}))$ is the diagonalized leverage
scores, and this Hessian satisfies 
\[
A_{x}^{\T}\Sigma_{x}A_{x}\preceq\hess\phi_{\vol}(x)\preceq3A_{x}^{\T}\Sigma_{x}A_{x}\,.
\]
We refer readers to \S\ref{proof:linear-volumetric} for details.
In other words, the \emph{approximate} volumetric metric $A_{x}^{\T}\Sigma_{x}A_{x}$
serves as an $\mc O(1)$-approximation of the local metric $\hess\phi_{\vol}$
(i.e., $A_{x}^{\T}\Sigma_{x}A_{x}\asymp\hess\phi_{\vol}(x)$). We
find in Lemma~\ref{lem:paramsBarrier} that the local metric $40\sqrt{m}A_{x}^{\T}\Sigma_{x}A_{x}$
is SSC with $\nu,\,\onu=\mc O(\sqrt{m}d)$, but in some regime of
$d$ this parameter leads to worse mixing of the $\dw$. In the same
paper, \citet{vaidya1996new} introduced a \emph{regularized} volumetric
metric by adding $\O\bpar{\hess\phi_{\log}}$, which we call the \emph{Vaidya
metric}:

\[
g(x):=\sqrt{\frac{m}{d}}\,A_{x}^{\T}\bpar{\Sigma_{x}+\frac{d}{m}I_{m}}A_{x}\,.
\]
Note that $g(x)\asymp\hess\bpar{\sqrt{\frac{m}{d}}\bpar{\phi_{\vol}+\frac{d}{m}\text{\ensuremath{\phi_{\log}}}}}$.
We show that the Vaidya metric is also SSC, SLTSC, and SASC without
additional scaling, while it has a better $\nu$ and $\onu$ than
the logarithmic barrier.
\begin{lem}
[Vaidya metric]\label{lem:vaidya} For a closed convex $K=\{x\in\Rd:Ax\geq b\}$
with $A\in\R^{m\times d}$ and $b\in\R^{m}$, let $g(x)=\sqrt{\frac{m}{d}}A_{x}^{\T}\bpar{\Sigma_{x}+\frac{d}{m}I_{m}}A_{x}$.
\begin{itemize}
\item $\nu=\mc O(\sqrt{md})$ \citet[Theorem 5.2]{anstreicher1997volumetric}.
\item SSC and $\onu=\mc O(\sqrt{md})$ (Lemma~\ref{lem:paramsBarrier}).
\item SLTSC (Lemma~\ref{lem:vaidya-SLTSC}) and SASC (Lemma~\ref{lem:vaidya-SASC}).
\end{itemize}
\end{lem}


\paragraph{Lewis weights metric.}

Self-concordance and symmetry parameters of $\mc O(\sqrt{md})$ is
certainly better than $\mc O(m)$, but can we even achieve an $\mc O(d\log^{\mc O(1)}m)$
bound on those parameters?

Let us recall the $\ell_{p}$-\emph{Lewis weights}. The $\ell_{p}$-Lewis
weight of $A$ is denoted by $w(A)$, the solution $w$ to the equation
$w(A)=\diag\bpar{W^{\nicefrac{1}{2}-\nicefrac{1}{p}}A(A^{\T}W^{1-\nicefrac{2}{p}}A)^{-1}A^{\T}W^{\nicefrac{1}{2}-\nicefrac{1}{p}}}\in\R^{m}$
for $W:=\Diag(w)$. For $W_{x}=\Diag(w(A_{x}))$ and $p\geq2$, the
Lewis weight barrier function is defined by
\[
\phi_{\lw}(x):=\log\det(A_{x}^{\T}W_{x}^{1-\nicefrac{2}{p}}A_{x})\,.
\]
Note that the leverage score and volumetric barrier can be recovered
as a special case of the Lewis weight and barrier by setting $p=2$.
As done for the Vaidya metric, it is natural to consider the Lewis
weight metric with $p=\Theta(\log^{\mc O(1)}m)$, defined as 
\[
g(x):=\mc O(\log^{\mc O(1)}m)\,A_{x}^{\T}W_{x}A_{x}\,.
\]
In fact, this metric serves as an $\mc O(\log^{\mc O(1)}m)$-approximation
of $\hess\phi_{\lw}$, as demonstrated in the following relation proven
in \citet[Lemma 31]{lee2019solving}:
\[
A_{x}^{\T}\Sigma_{x}A_{x}\preceq\hess\phi_{\lw}\preceq(1+p)\,A_{x}^{\T}\Sigma_{x}A_{x}\,.
\]
Ignoring the logarithmic factors we have $\hess\phi_{\lw}\asymp g$.
Notably, the Lewis-weight metric needs an additional $\sqrt{d}$-scaling
for SLTSC and SASC, unlike the logarithmic barrier and Vaidya metric.
Hence, when combining this with other metrics, one should use $\sqrt{d}g$,
which leads to $\nu,\,\onu=\mc O(d^{3/2}\,\log^{\mc O(1)}m)$. 
\begin{lem}
[Lewis weight metric]\label{lem:Lewis-weight} For a closed convex
$K=\{x\in\Rd:Ax\geq b\}$ with $A\in\R^{m\times d}$ and $b\in\R^{m}$,
let $g(x)=\mc O(\log^{\mc O(1)}m)\,A_{x}^{\T}W_{x}A_{x}$.
\begin{itemize}
\item $\nu=\mc O(d\log^{5}m)$ \citet[Theorem 30]{lee2019solving}.
\item SSC and $\onu=\mc O(d\log^{\mc O(1)}m)$ (Lemma~\ref{lem:paramsBarrier}).
\item $\sqrt{d}g$ is SLTSC (Lemma~\ref{lem:Lw-SLTSC}) and SASC (Lemma~\ref{lem:Lw-SASC}).
\end{itemize}
\end{lem}


\subsubsection{Analysis of self-concordant metrics for linear constraints \label{subsec:analysis-linear-metric}}

\paragraph{Strong self-concordance and symmetry.}

We defer the proofs of two lemmas below to \S\ref{proof:linear-SSC-symm}.
We study SSC and symmetry of the metrics of the form $A_{x}^{\T}D_{x}A_{x}$
in Lemma~\ref{lem:helper4Diagonal}, where $D_{x}\in\R^{m\times m}$
is a diagonal matrix used to address the constraints of the form $Ax\geq b$
for $A\in\R^{m\times d}$ and $b\in\R^{m}$. Specifically, we relate
the notions of SSC and symmetry to well-studied terms in the field
of optimization, namely $\max_{i}\,[\sigma(\sqrt{D_{x}}A_{x})]_{i}/[D_{x}]_{ii}$
and $\snorm{\Dd D_{x}[h]}_{D_{x}^{-1}}^{2}$. 
\begin{lem}
\label{lem:helper4Diagonal} For $x\in\inter(K)$, let $g(x)=A_{x}^{\T}D_{x}A_{x}\in\Rdd$
for a diagonal matrix $0\prec D_{x}\in\R^{m\times m}$.
\begin{itemize}
\item For any PSD matrix function $g'$ such that $g'+g$ is invertible
on the domain,
\begin{align*}
 & \snorm{(g'(x)+g(x))^{-1/2}\Dd g(x)[h]\,(g'(x)+g(x))^{-1/2}}_{F}^{2}\\
 & \qquad\qquad\leq4\max_{i}\frac{[\sigma(\sqrt{D_{x}}A_{x})]_{i}}{[D_{x}]_{ii}}\cdot\bpar{\snorm h_{g(x)}^{2}+\sum_{i=1}^{m}\frac{(\Dd D_{x}[h])_{ii}^{2}}{[D_{x}]_{ii}}}\,.
\end{align*}
 
\item $\max_{h:\norm h_{g(x)}=1}\norm{A_{x}h}_{\infty}=\bpar{\max_{i\in[m]}\frac{[\sigma(\sqrt{D_{x}}A_{x})]_{i}}{[D_{x}]_{ii}}}^{1/2}$.
\item $K\cap(2x-K)\subset\dcal_{g}^{\sqrt{\tr(D_{x})}}(x)$.
\end{itemize}
\end{lem}

Then for each metric we refer to existing bounds on these terms, estimating
the smallest possible scaling required for SSC and symmetry. 
\begin{lem}
[Strong self-concordance and symmetry]\label{lem:paramsBarrier}
Let $A\in\R^{m\times d}$, $\Sigma_{x}=\Diag(\sigma(A_{x}))\in\R^{m\times m}$,
and $W_{x}=\Diag(w_{x})\in\R^{m\times m}$ for the $\ell_{p}$-Lewis
weight $w_{x}$ with $p=\mc O(\log m)$.
\begin{itemize}
\item Logarithmic metric: $g(x)=A_{x}^{\T}A_{x}$ with $D_{x}=I_{m}$ is
SSC along $\rowspace(A)$ with $\onu=m$.
\item Approximate volumetric metric: $g(x)=40\sqrt{m}A_{x}^{\T}\Sigma_{x}A_{x}$
with $D_{x}=40\sqrt{m}\Sigma_{x}$ is SSC with $\onu=\mc O(\sqrt{m}d)$.
\item Vaidya metric: $g(x)=22\sqrt{\frac{m}{d}}A_{x}^{\T}\bpar{\Sigma_{x}+\frac{d}{m}I_{m}}A_{x}$
with $D_{x}=22\sqrt{\frac{m}{d}}\bpar{\Sigma_{x}+\frac{d}{m}I_{m}}$
is SSC with $\onu=\mc O(\sqrt{md})$.
\item Lewis-weight metric: $\exists$ positive constants $c_{1}$ and $c_{2}$
such that $g(x)=c_{1}(\log m)^{c_{2}}A_{x}^{\T}W_{x}A_{x}$ is SSC
and $\onu$-symmetric with $\onu=\mc O^{*}(d)$.\label{lem:LSmetricStrongandSymmetry}
\end{itemize}
\end{lem}


\paragraph{Strongly lower trace self-concordance}

We show SLTSC of the Vaidya and Lewis-weight metric. Let $g_{2}$
be either Vaidya or Lewis-weight metric, and $g_{1}$ be an arbitrary
PSD matrix function on $K$ such that $g=g_{1}+g_{2}$ is PD on $\intk$.
Ensuring (S)LTSC of the Vaidya or Lewis-weight metrics is challenging,
as $\Dd^{2}g_{2}[h,h]\succeq0$ is difficult to verify due to complicated
expressions for $\Dd^{2}\Sigma_{x}[h,h]$ and $\Dd^{2}W_{x}[h,h]$.
As for the Vaidya metric, we compute higher-order derivatives of leverage
scores and other pertinent matrices in Lemma~\ref{lem:calculusLeverage},
finding succinct formulas by using algebraic properties of the Hadamard
product. We then show SLTSC of $g_{2}$ using these results (see \S\ref{proof:linear-vaidya-SLTSC}
for the proof):
\begin{lem}
[SLTSC of Vaidya]\label{lem:vaidya-SLTSC} $\tr\bpar{g^{-1}\Dd^{2}g_{2}(x)[h,h]}\geq-\snorm h_{g_{2}(x)}^{2}/2$
for the Vaidya metric $g_{2}$.
\end{lem}

For the Lewis-weights metric, analysis is more involved due to numerous
terms appearing in $\Dd^{2}W_{x}[h,h]$. In order to avoid dealing
with each of the terms, we employ existing bounds on derivatives of
$W_{x}$ and other relevant matrices in \S\ref{proof:linear-LW}.
This approach significantly simplifies the computation but comes at
the cost of an additional scaling of $\sqrt{d}$, which as far as
we can tell might be unavoidable. We refer readers to \S\ref{proof:linear-Lewis-SLTSC}
for the proof.
\begin{lem}
[SLTSC of Lewis-weight]\label{lem:Lw-SLTSC} $\tr\bpar{g(x)^{-1}\Dd^{2}g_{2}(x)[h,h]}\geq-\snorm h_{g_{2}(x)}^{2}$,
where $g_{2}(x)=cA_{x}^{\T}W_{x}A_{x}$ with $c=c_{1}(\log m)^{c_{2}}\sqrt{d}$
for some constants $c_{1},c_{2}>0$.
\end{lem}


\paragraph{Strongly average self-concordance.}

Typically, (S)ASC is the most challenging property to verify, often
requiring involved analysis in order to establish it \emph{without}
additional scalings. Since the three metrics are HSC (e.g., see Lemma~\ref{lem:Lw-hsc}
for Lewis-weight metrics), scaling by $d$ leads to SASC by Lemma~\ref{lem:hsc-to-sasc}.
However, for linear constraints one can still achieve SASC without
scaling (or with a smaller scaling) through more sophisticated concentration
techniques.

To sketch this idea, we recall that SASC requires showing that for
small enough $r$
\[
\snorm{z-x}_{g(z)}^{2}-\snorm{z-x}_{g(x)}^{2}\leq2\veps\frac{r^{2}}{d}\,.
\]
Taylor's expansion of $\snorm{z-x}_{g(z)}^{2}$ at $z=x$ up to second-order
necessitates bounds on
\[
\Dd g(x)[(z-x)^{\otimes3}]=\frac{r^{3}}{d^{3/2}}\Dd g(x)[h^{\otimes3}]\qquad\text{and}\qquad\Dd g(x')[(z-x)^{\otimes4}]=\frac{r^{4}}{d^{2}}\Dd^{2}g(x')[h^{\otimes4}]\,,
\]
for some $x'\in[x,z]$ and $h\sim\ncal(0,I_{d})$. Observe that the
first-order term $P(h):=\frac{r^{3}}{d^{3/2}}\Dd g(x)[h^{\otimes3}]$
is a Gaussian polynomial in $h$, and this is where we can invoke
the following concentration phenomenon:
\begin{lem}
[Concentration of Gaussian polynomials] \label{lem:conc-gaussian-poly}
For $d\geq1$, let $P:\Rd\to\R$ be a polynomial of degree $n$. For
any $t\geq(2e)^{n/2}$, 
\[
\P_{h\sim\ncal(0,I_{d})}\Bbrack{|P(h)|\geq t\sqrt{\E[P(h)^{2}]}}\leq\exp\bpar{-\frac{n}{2e}\,t^{2/n}}\,.
\]
\end{lem}

This concentration inequality necessitates bounding $\E[P(h)^{2}]$,
and this is where Stein's lemma comes into play:
\begin{lem}
\label{lem:stein} For $h=(h_{1},\dots,h_{d})\sim\ncal(0,I_{d})$,
it holds that $\E[h_{i}f(h)]=\E[\de_{i}f(h)]$.
\end{lem}

Unlike the first-order term, the second-order term is \emph{not} a
Gaussian polynomial due to $x'$ depending on $z$. To address this
issue, we derive an upper bound (in absolute value) of the quadratic
form. Using coordinate-wise closeness of slacks, leverage scores,
and Lewis weights at two nearby points, we replace every value estimated
at $z$ by those at $x$, removing dependence on $z$ in the quadratic
bound. The resulting quadratic bound is now a Gaussian polynomial,
so we follow the same proof approach as with the first-order term.

This approach was used by \citet{sachdeva2016mixing} for ASC of log-barriers
and by \citet{chen2018fast} for that of Vaidya and Lewis-weight metrics.
We further extend this approach to achieve SASC of those metrics,
going beyond ASC.
\begin{lem}
[SASC of logarithmic barrier] \label{lem:logBarrier-SASC} $g(x)=\hess\phi_{\log}(x)=A_{x}^{\T}A_{x}$
is SASC.
\end{lem}

See \S\ref{proof:linear-SASC-log} for the proof.
\begin{lem}
[SASC of Vaidya metric] \label{lem:vaidya-SASC} $g(x)=\mc O\bpar{\sqrt{\frac{m}{d}}}\,A_{x}^{\T}(\Sigma_{x}+\frac{d}{m}I_{m})A_{x}$
is SASC.
\end{lem}

See \S\ref{proof:linear-SASC-vaidya} for the proof.
\begin{lem}
[SASC of Lewis-weight metric] \label{lem:Lw-SASC} There exists constants
$c_{1}$ and $c_{2}$ such that $g(x)=c_{1}\sqrt{d}\log^{c_{2}}m\,A_{x}^{\T}W_{x}A_{x}=\mc O^{*}(\sqrt{d})\,A_{x}^{\T}W_{x}A_{x}$
is SASC.
\end{lem}

See \S\ref{proof:linear-SASC-Lw} for the proof.

\subsection{Quadratic potentials and constraints}

Suppose that in \eqref{eq:reduced-problem} we have either $f_{i}(x),\,h_{j}(x)=\snorm{x-\mu}_{\Sigma}^{2}$
or $\half x^{\T}Qx+p^{\T}x+l$ for $\mu,p\in\Rd$, $\Sigma\in\pd$,
and $0\neq Q\in\psd$.

\paragraph{Quadratic constraint.}

Consider a second-order region given by $K=\{x\in\Rd:\half x^{\T}Qx+p^{\T}x+l\leq0\}$.
\citet{nesterov1994interior} shows that $\phi:=-\log f$ is an $1$-self-concordant
barrier for $K$, when $f(x)=-\half\snorm{x-\mu}_{\Sigma}^{2}$ or
$-(\half x^{\T}Qx+p^{\T}x+l)$. Since $\onu=\mc O(\nu^{2})$ for a
self-concordant barrier due to Lemma~\ref{lem:bound-symmetry}, $\phi$
is $\mc O(1)$-symmetric. In case we consider $\snorm{x-\mu}_{\Sigma}^{2}$,
the trivial scaling by dimension $d$ implies that $d\phi$ is SSC
and $\mc O(d)$-symmetric.

Moreover, $d\phi$ is SASC by Lemma~\ref{lem:hsc-to-sasc} by HSC
of $\phi$. For HSC of $\phi$, we develop a handy tool for checking
HSC. See \S\ref{proof:quadratic} for the proof.
\begin{lem}
\label{lem:4th-log} For a real-valued function $f$ on $K\subset\Rd$,
let $\psi=-\log f$ be a $\nu$-self-concordant barrier for $K$.
Then, 
\[
|\Dd^{4}\psi(x)[h^{\otimes4}]|\lesssim\nu^{2}\snorm h_{\hess\psi(x)}^{2}+\big|\frac{\Dd^{4}f(x)[h^{\otimes4}]}{f(x)}\big|\,.
\]
\end{lem}

Using this tool, we can study properties of the barrier for the quadratic
constraints. We provide the proof in \S\ref{proof:quadratic}.
\begin{lem}
[Quadratic constraint]\label{lem:quadratic-const} For a closed convex
$K=\{x\in\Rd:\half x^{\T}Qx+p^{\T}x+l\leq0\}$ with $p\in\Rd$ and
$0\neq Q\in\psd$, let $\phi(x)=-\log(-l-p^{\T}x-\half x^{\T}Qx)$
and $g=d\,\hess\phi$.
\begin{itemize}
\item $\nu,\,\onu=\mc O(d)$.
\item SSC when $Q\succ0$, and SASC.
\item $\Dd^{2}g(x)[h,h]\succeq0$ for any $x\in\inter(K)$ and $h\in\Rd$
(so SLTSC).
\end{itemize}
\end{lem}


\paragraph{Gaussian distribution ($f(x)=\protect\half\protect\snorm{x-\mu}_{\Sigma}^{2}$).}

Suppose the quadratic term $f(x)=\half\snorm{x-\mu}_{\Sigma}^{2}$
appears in a potential of a target distribution. Then its epigraph
is 
\[
\{(x,t)\in\R^{d+1}:\half\snorm{x-\mu}_{\Sigma}^{2}-t\leq0\}\,,
\]
and clearly $q(x,t)=\half\snorm{x-\mu}_{\Sigma}^{2}-t$ is a quadratic
function in $(x,t)$. Hence, this level set admits an $1$-self-concordant
barrier
\[
\phi(x,t)=-\log(t-\half\snorm{x-\mu}_{\Sigma}^{2})\,.
\]
Our earlier discussion immediately leads to the following result:
\begin{lem}
[Quadratic potential] \label{lem:Gaussian-potential}Consider a closed
convex $K=\{(x,t):\half\snorm{x-\mu}_{\Sigma}^{2}\leq t\}$ with $\mu\in\Rd$
and $\Sigma\in\pd$, and let $\phi(x)=-\log(t-\half\snorm{x-\mu}_{\Sigma}^{2})$
and $g=d\,\hess\phi$.
\begin{itemize}
\item $\nu_{g},\,\onu_{g}=\mc O(d)$.
\item SSC and SASC.
\item $\Dd^{2}g(x,t)[h,h]\succeq0$ for any $(x,t)\in\inter(K)$ and $h\in\R^{d+1}$.
\end{itemize}
\end{lem}


\paragraph{Second-order cone ($f(x)=\protect\half\protect\snorm{x-\mu}_{\Sigma}$).}

It is common that a potential includes a non-smooth term like $\norm{Ax-b}_{2}$
in many applications, and we can handle such potentials via our framework.
\citet[Lemma 4.3.3]{nesterov1994interior} shows that 
\[
\phi(x,t)=-\log(t^{2}-\snorm x^{2})
\]
is a $2$-self-concordant for a level set $K=\{(x,t)\in\Rd\times\R:\snorm x_{2}\leq t\}$
(here we may assume that $\mu=0$ and $\Sigma=I$ due to Lemma~\ref{lem:linear-trans}).
This level set is called a \emph{second-order cone} or Lorentz cone.

Applying Lemma~\ref{lem:4th-log} to $f(x,t)=t^{2}-\norm x^{2}$
with $\nu=2$, we immediately show HSC of $\phi$. Thus, $d\phi$
satisfies SLTSC and SASC by Lemma~\ref{lem:hsc-to-sltsc} and Lemma~\ref{lem:hsc-to-sasc},
respectively.
\begin{lem}
[Second-order cone] \label{lem:soc} Consider a closed convex $K=\{(x,t):\snorm{x-\mu}_{\Sigma}\leq t\}$
with $\mu\in\Rd$ and $\Sigma\in\pd$, and let $\phi(x,t)=-\log(t^{2}-\snorm{x-\mu}_{\Sigma}^{2})$
and $g=d\,\hess\phi$.
\begin{itemize}
\item $\nu_{g},\,\onu_{g}=\mc O(d)$.
\item SSC, SASC, and SLTSC.
\end{itemize}
\end{lem}


\subsection{PSD cone}

The function $\phi(X)=-\log\det X$ serves as an $d$-self-concordant
barrier for the PSD cone $\psd$. While achieving self-concordance
does not require additional scaling, it turns out that SSC requires
a scaling of $\Theta(d)$. Notably, this scaling is less than the
trivial dimension-based scaling of $d_{s}:=d(d+1)/2$. Also, direct
computation leads to $\Dd^{4}\phi(X)[H,H]\succeq0$ (so SLTSC).

As $\phi$ is HSC, scaling by $d_{s}$ ensures SASC. However, we can
achieve ASC with a smaller scaling by $\mc O(d)$ via the random matrix
theory.
\begin{lem}
[PSD cone] \label{lem:psd} On a closed convex $K=\psd$, let $\phi(X)=-\log\det X$
and define $g=d\,\hess\phi$.
\begin{itemize}
\item $\nu=d^{2}$ (\citet{nesterov1994interior}) and $\onu=d^{2}$ (Lemma~\ref{lem:logdet-symm}).
\item SSC (Corollary~\ref{cor:logdet-ssc}).
\item $\Dd^{2}g(X)[H,H]\succeq0$ for any $X\in\intk$ and $H\in\mbb S^{d}$
(Lemma~\ref{lem:logdet-sltsc}).
\item ASC (Lemma~\ref{lem:logdet-asc}), and $d_{s}\,\hess\phi$ is SASC.
\end{itemize}
\end{lem}


\subsubsection{Formalism via matrix-vector transformations \label{subsec:formalism}}

In analyzing $\phi$, we work in $\R^{d_{s}}=\R^{d(d+1)/2}$ and $\mbb S^{d}$
simultaneously in the sequel, moving back and forth between them implicitly.
We justify this identification as follows.

\paragraph{Measure on $\protect\mbb S^{d}$.}

We can define and work with the Lebesgue measure on $\mbb S^{d}$
by identifying it with the Lebesgue measure on $\R^{d_{s}}$, where
each component in the Lebesgue measure on $\mbb S^{d}$ corresponds
to each entry in the upper triangular part. Hence, with the Lebesgue
measure $\D X$ on $\mbb S^{d}$ it is straightforward to define a
probability distribution on $\mbb S^{d}$ whose probability density
function with respect to $\D X$ is proportional to $\exp(-f)$ for
a function $f:\mbb S^{d}\to\R$. For instance, the uniform distribution
over a region corresponds to $f$ being constant in the region and
infinity outside of the region, and an exponential distribution to
$f(X)=\inner{C,X}=\tr(C^{\T}X)$ for $C\in\mbb S^{d}$.

\paragraph{Directional derivatives.}

A function $\phi:\mbb S^{d}\to\R$ induces its counterpart $\psi:\R^{d_{s}}\to\R$
defined by $\psi(x)=\phi(X)$ for $x:=\svec(X)$. For symmetric matrices
$\{H_{i}\}_{i\leq k}$, the $k$-th directional derivative of $\phi$
in directions $H_{1},\dots,H_{k}$ is 
\[
\Dd^{k}\phi(X)[H_{1},\cdots,H_{k}]\defeq\frac{\D^{k}}{\D t_{k}\cdots\D t_{1}}\phi\Bpar{X+\sum_{i=1}^{k}t_{i}H_{i}}\bigg\vert_{t_{1},\dots,t_{k}=0}\,.
\]
For $h_{i}:=\svec(H_{i})$, it follows that $\phi(X+\sum_{i=1}^{k}t_{i}H_{i})=\psi(x+\sum_{i=1}^{k}t_{i}h_{i})$
and thus
\[
\Dd^{k}\phi(X)[H_{1},\cdots,H_{k}]=\Dd^{k}\psi(x)[h_{1},\cdots,h_{k}]\,.
\]
With this identification in hand, since the notion of (symmetric or
strong) self-concordance is formulated in terms of directional derivatives,
we can deal with both representations without having to specify one
of them.

\paragraph{Important operators.}

We introduce three linear operators that enable us to make smooth
transitions between $\mbb S^{d}$ and $\R^{d_{s}}$.
\begin{defn}
[\citet{magnus1980elimination}] \label{def:linearOperators} Let
$E_{ij}=e_{i}e_{j}^{\T}\in\Rdd$ be the matrix with a single $1$
in the $(i,j)$ position and zeros elsewhere.
\begin{itemize}
\item $M:\R^{d_{s}}\to\R^{d^{2}}$ is the linear operator that maps $\svec(\cdot)$
to $\vec(\cdot)$ (i.e., $M\circ\svec=\vec$). It can be written as
$M=\sum_{i\geq j}\vec(T_{ij})u_{ij}^{\T}$, where $T_{ij}\in\Rdd$
has all zero entries except for $1$ at $(i,j)$ and $(j,i)$ positions
(i.e., $T_{ij}=E_{ij}+E_{ji}$ if $i\neq j$ and $E_{ij}$ if $i=j$),
and $u_{ij}=\svec(E_{ij})$.
\item $N:\R^{d^{2}}\to\R^{d^{2}}$ is the linear operator that maps $\vec(A)$
to $\vec\bpar{\half(A+A^{\T})}$ for a matrix $A\in\Rdd$.
\item $L:\R^{d_{s}}\to\R^{d^{2}}$ is the linear operator that maps $\vec(A)$
to $\svec(A)$ for a matrix $A\in\Rdd$. It can be written as $L=\sum_{i\geq j}u_{ij}\vec(E_{ij})^{\T}$. 
\end{itemize}
\end{defn}

\begin{lem}
[\citet{magnus1980elimination}] \label{lem:MNL-properties} Let
$M,N,L$ be matrices in Definition~\ref{def:linearOperators}.
\begin{itemize}
\item (Lemma 2.1) $N=N^{\T}=N^{2}$ and $N(A\otimes A)=(A\otimes A)N$ for
any $d\times d$ matrix $A$.
\item (Lemma 3.5) $MLN=N$.
\end{itemize}
\end{lem}


\subsubsection{Analysis of a self-concordant metric for the PSD cone \label{subsec:scBasicMetric}}

We first examine properties of the metric defined by the Hessian of
self-concordant barrier $\phi(X)=-\log\det X$ (see \citet[Theorem 4.3.3]{nesterov2003introductory}
for self-concordance). In this case, its Hessian and inverse have
clean formulas. 
\begin{prop}
\label{prop:metricFormula} Let $\grad_{X}^{2}\phi(X)=-\grad_{x}^{2}\log\det(\svec^{-1}(x))\in\R^{d_{s}\times d_{s}}$
for $X\in\psd$. Then,
\begin{align*}
\hess\phi(X) & =M^{\T}(X^{-1}\otimes X^{-1})M=M^{\T}(X\otimes X)^{-1}M\,,\\
\bpar{\hess\phi(X)}^{-1} & =M^{\dagger}(X\otimes X)\bpar{M^{\dagger}}^{\T}=LN(X\otimes X)NL^{\T}\,,
\end{align*}
where $M^{\dagger}=(M^{\T}M)^{-1}M^{\T}\in\R^{d_{s}\times d^{2}}$
is the Moore-Penrose inverse of $M\in\R^{d^{2}\times d_{s}}$.
\end{prop}

We defer the proof to Appendix~\ref{app:matrixCalculus}. We remark
that as an immediate corollary to this, the local norm of $h\in\R^{d_{s}}$
with metric $\hess\phi(X)$ is 
\[
\snorm h_{X}^{2}=\svec(H)^{\T}M^{\T}(X^{-1}\otimes X^{-1})M\svec(H)\underset{\text{(i)}}{=}\tr(HX^{-1}HX^{-1})\eqqcolon\snorm H_{X}^{2}\,,
\]
where (i) follows from $\vec=M\circ\svec$ (Definition~\ref{def:linearOperators})
and $\tr(DB^{\T}A^{\T}C)=\vec(A)^{\T}(B\otimes C)\vec(D)$ (Lemma~\ref{lem:Kronecker}). 

\paragraph{Symmetry.}
\begin{lem}
[$\onu$-symmetry] \label{lem:logdet-symm}For $X\in K=\psd$, the
barrier $\phi(X)=-\log\det X$ is $d$-symmetric.
\end{lem}

\begin{proof}
For $X\in K$, pick any $Y\in K\cap(2X-K)$, and define a symmetric
matrix $H:=Y-X$. Since $Y\in K$ and $2X-Y\in K$, we have $X+H\in K$
and $X-H\in K$. Thus,
\[
-I\preceq X^{-1/2}HX^{-1/2}\preceq I\,,
\]
and the magnitude of each eigenvalue $\{\lda_{i}\}_{i=1}^{d}$ of
$X^{-1/2}HX^{-1/2}$ is bounded by $1$. Hence,
\[
\snorm H_{X}^{2}=\tr(X^{-1}HX^{-1}H)=\snorm{X^{-1/2}HX^{-1/2}}_{F}^{2}\leq\sum_{i=1}^{d}\lda_{i}^{2}\leq d\,.\qedhere
\]
\end{proof}

\paragraph{Convexity of log-determinant of Hessian and SSC.}

Next, the convexity of the log-determinant of $\hess\phi$ can be
checked via properties of Kronecker products. See \S\ref{proof:psd-convex-ssc}
for the proof.
\begin{prop}
[Convexity of log-determinant of Hessian] \label{prop:convex-logdet}
$\log\det(\hess\phi(\cdot))$ is convex.
\end{prop}

We move onto SSC of $d\phi(X)$.
\begin{lem}
\label{lem:logdet-scaling} For $\psi_{X}:=\sup_{H\in\mbb S^{d}}\snorm{(\hess\phi(X))^{-1/2}\Dd^{3}\phi(X)[H]\,(\hess\phi(X))^{-1/2}}_{F}/\snorm H_{X}$,
we have
\[
\sqrt{2(d+1)}\leq\psi_{X}\leq2\sqrt{d}\,.
\]
\end{lem}

We present the proof in \S\ref{proof:psd-convex-ssc}. This result
informs us of the best possible scaling of $\phi$ that ensures SSC.
Recall that if $g$ satisfies $\snorm{g^{-1/2}\Dd g[h]g^{-1/2}}_{F}\leq2\alpha\norm h_{g}$
for $\alpha>0$, then $\alpha^{2}g$ is SSC. We remark that the scaling
of $d$ is obviously better than the trivial scaling of $d_{s}=\Theta(d^{2})$.
\begin{cor}
[Strong self-concordance] \label{cor:logdet-ssc} A function $d\phi$
is a strongly self-concordant barrier for $\psd$. Moreover, the scaling
factor of $d$ cannot be further improved.
\end{cor}


\paragraph{Strongly lower trace self-concordance.}

SLTSC of $\phi$ can be easily checked by noting $g(X)[H,H]=\tr(X^{-1}HX^{-1}H)$
and using the chain rule. See the details in \S\ref{proof:psd-sltsc}.
\begin{lem}
[SLTSC] \label{lem:logdet-sltsc}$\Dd^{2}g(X)[H,H]\succeq0$ for
any $X\in\intk$ and $H\in\mathbb{S}^{d}$.
\end{lem}


\paragraph{Average self-concordance.}

In establishing ASC, we find an interesting connection to a \emph{Gaussian
orthogonal ensemble} (GOE), one of the main objects studied in the
random matrix theory. We prove the following lemmas and explain challenges
when extending our arguments to SASC in \S\ref{proof:psd-asc}.
\begin{lem}
\label{lem:conn-to-goe} For $d_{s}=\frac{d(d+1)}{2}$ and $\svec(H)\sim\ncal\bpar{0,\frac{r^{2}}{d_{s}}g(X)^{-1}}$,
$\frac{\sqrt{d_{s}d}}{r}X^{-1/2}HX^{-1/2}$ is a GOE.
\end{lem}

\begin{lem}
[ASC] \label{lem:logdet-asc} $-d\,\log\det X$ is ASC.
\end{lem}


\subsection{Logarithm, exponential, entropy, and $\ell_{p}$-norm (power function)}

\paragraph{Logarithm in potentials.}

Consider $Q_{1}=\{(x,t)\in\R^{2}:-\log x\leq t,x>0\}$. As $f(\cdot)=-\log(\cdot)$
is convex on $\R_{+}$ and satisfies the condition in Lemma~\ref{lem:tool-convex}
with $\beta=2$ and $\gamma=6$,
\[
F(x,t)=-\log(t+\log x)-36\log x
\]
is a highly $37$-self concordant barrier for $Q_{1}$. Therefore,
$2F$ is SSC and SLTSC with $\onu=\mc O(1)$.
\begin{lem}
[Logarithm] Consider the direct product of level sets
\[
K=\prod_{i=1}^{d}\{(x_{i},t_{i})\in\R^{2}:-\log x_{i}\leq t_{i},\,x_{i}>0\}\,,
\]
and let $\phi(x,t)=-\sum_{i=1}^{d}\bpar{\log(t_{i}+\log x_{i})+36\log x_{i}}$
and $g=2\hess\phi$.
\begin{itemize}
\item $\nu,\,\onu=\mc O(d)$.
\item SSC and SLTSC.
\item $d\,\hess\phi$ is SASC.
\end{itemize}
\end{lem}

\begin{proof}
For $i\in[d]$, let $Q_{i}=\{(x_{i},t_{i})\in\R^{2}:-\log x_{i}\leq t_{i},\,y_{i}>0\}$
and $F_{i}(x_{i},t_{i})$ be the self-concordant barrier above. Note
that $2F_{i}$ is SSC and SLTSC. By Lemma~\ref{lem:ssc-direct} and
\ref{lem:sltsc-direct}, the Hessian of $F(x,t):=2\sum_{i=1}^{d}F_{i}(x_{i},t_{i})$
is SSC and SLTSC. The last item on SASC follows from Lemma~\ref{lem:hsc-to-sasc}.
\end{proof}

\paragraph{Exponent in potentials.}

Consider $Q_{2}=\{(x,t)\in\R^{2}:e^{x}\leq t\}=\{(x,t)\in\R^{2}:t>0,\,x\leq\log t\}$.
As $f(t)=\log t$ is concave and satisfies the condition in Lemma~\ref{lem:tool-concave}
with $\beta=2$ and $\gamma=6$,
\[
F(x,t)=-\log(\log t-x)-36\log t
\]
is a highly $37$-self concordant barrier for $Q_{2}$. Therefore,
$2F$ is SSC and SLTSC with $\onu=\mc O(1)$.
\begin{lem}
[Exponential] Consider the direct product of level sets
\[
K=\prod_{i=1}^{d}\{x_{i},t_{i})\in\R^{2}:\exp(x_{i})\leq t_{i}\}\,,
\]
and let $\phi(x,t)=-\sum_{i=1}^{d}(\log(\log t_{i}-x_{i})+36\log t_{i})$
and $g=2\hess\phi$.
\begin{itemize}
\item $\nu,\,\onu=\mc O(d)$.
\item SSC and SLTSC.
\item $d\,\hess\phi$ is SASC.
\end{itemize}
\end{lem}

\begin{proof}
For $i\in[d]$, let $Q_{i}=\{(x_{i},t_{i})\in\R^{2}:e^{x_{i}}\leq t_{i}\}$
and $F_{i}(x_{i},t_{i})$ be the self-concordant barrier above. Note
that $2F_{i}$ is SSC and SLTSC. By Lemma~\ref{lem:ssc-direct} and~\ref{lem:sltsc-direct},
the Hessian of $F(x,t):=2\sum_{i=1}^{d}F_{i}(x_{i},t_{i})$ is SSC
and SLTSC. The last item on SASC follows from Lemma~\ref{lem:hsc-to-sasc}.
\end{proof}

\paragraph{Entropy in potentials.}

Consider $Q_{3}=\{(x,t)\in\R^{2}:x\geq0,\,t\geq x\log x\}$. Note
that $f(x)=x\log x$ is convex on $\{x>0\}$ and satisfies the condition
in Lemma~\ref{lem:tool-convex} with $\beta=1$ and $\gamma=2$.
Hence,
\[
F(x,t)=-\log(t-x\log x)-36\log x
\]
is a highly $5$-self concordant barrier for $Q_{3}$. Therefore,
$2F$ is SSC and SLTSC with $\onu=\mc O(1)$.
\begin{lem}
[Entropy] Consider the direct product of level sets
\[
K=\prod_{i=1}^{d}\{(x_{i},t_{i})\in\R^{2}:x_{i}\geq0,\,t_{i}\geq x_{i}\log x_{i}\}\,,
\]
and let $\phi(x,t)=-\sum_{i=1}^{d}\bpar{\log(t_{i}-x_{i}\log x_{i})+36\log x_{i}}$
and $g=2\hess\phi$.
\begin{itemize}
\item $\nu,\,\onu=\mc O(d)$.
\item SSC and SLTSC.
\item $d\,\hess\phi$ is SASC.
\end{itemize}
\end{lem}

\begin{proof}
For $i\in[d]$, let $Q_{i}=\{(x_{i},t_{i})\in\R^{2}:x_{i}\geq0,\,t_{i}\geq x_{i}\log x_{i}\}$
and $F_{i}(x_{i},t_{i})$ be the self-concordant barrier above. Note
that $2F_{i}$ is SSC and SLTSC. By Lemma~\ref{lem:ssc-direct} and~\ref{lem:sltsc-direct},
the Hessian of $F(x,t):=2\sum_{i=1}^{d}F_{i}(x_{i},t_{i})$ is SSC
and SLTSC. The last item on SASC follows from Lemma~\ref{lem:hsc-to-sasc}.
\end{proof}

\paragraph{$\ell_{p}$-norm (power function).}

We start with the power functions. For $p\geq1$, consider $Q_{4}=\{(x,t)\in\R^{2}:t\geq\max(0,x)^{p}\}=\{(x,t)\in\R^{2}:t\geq0,\,x\leq t^{1/p}\}$.
Note that $f(t)=t^{1/p}$ is concave on $t>0$ and satisfies the condition
in Lemma~\ref{lem:tool-concave} with $\beta=2$ and $\gamma=6$.
Hence,
\[
F_{4}(x,t)=-\log(t^{1/p}-x)-36\log t
\]
is a highly $37$-self-concordant barrier for $Q_{4}$. Similarly,
$F_{5}(t,x)=-\log(t^{1/p}+x)-36\log t$ is a highly $37$-self concordant
barrier for the convex set $Q_{5}=\{(x,t)\in\R^{2}:t\geq\max(0,-x)^{p}\}$.
Since the convex set $Q_{6}=\{(x,t)\in\R^{2}:t\geq|x|^{p}\}$ is equal
to $Q_{4}\cap Q_{5}$, the sum of $F_{4}+F_{5}$, which is 
\[
F_{6}(x,t)=-\log(t^{2/p}-x^{2})-72\log t
\]
is a highly $72$-self-concordant barrier for $Q_{6}$. Hence, $2F$
is SSC and SLTSC with $\onu=\mc O(1)$.
\begin{lem}
[$\ell_p$-norm] Consider the direct product of level sets $K=\prod_{i=1}^{d}\{(x_{i},t_{i})\in\R^{2}:\Abs{x_{i}}^{p}\leq t_{i}\}$,
and let $\phi(x,t)=-\sum_{i=1}^{d}\bpar{\log(t_{i}^{2/p}-x_{i}^{2})+72\log t_{i}}$
and $g=2\hess\phi$.
\begin{itemize}
\item $\nu,\,\onu=\mc O(d)$.
\item SSC and SLTSC.
\item $d\,\hess\phi$ is SASC.
\end{itemize}
\end{lem}

\begin{proof}
Consider a highly $72$-self-concordant barrier $F_{i}$ above for
$\{(x_{i},t_{i}):|x_{i}|^{p}\leq t_{i}\}$ for $i\in[d]$. Note that
$2F_{i}$ is SSC and SLTSC. By Lemma~\ref{lem:ssc-direct} and~\ref{lem:sltsc-direct},
the Hessian of $F(x,t):=2\sum_{i=1}^{d}F_{i}(x_{i},t_{i})$ is SSC
and SLTSC. The last item on SASC follows from Lemma~\ref{lem:hsc-to-sasc}.
\end{proof}



\global\long\def\vec{\textup{\textsf{vec}}}%
\global\long\def\svec{\textup{\textsf{svec}}}%


\section{Examples \label{sec:examples}}

For given constraints and epigraphs, combining metrics for them (according
to the self-concordance theory for sampling developed in Section~\ref{sec:sc-theory-rules})
and employing $\gcdw$ with the combined metric lead to a poly-time
mixing sampling algorithm. Compared to the state-of-the-art poly-time
mixing algorithm, the $\bw$, $\gcdw$ offers several advantages.
First, it does not require any preprocessing (e.g., rounding) due
to affine invariance. Also, it achieves faster mixing by leveraging
inherent geometric information in sampling problems.

The per-step complexity of $\dws$, however, is in general higher
than that of the $\bw$. The primary computational bottleneck lies
in computing the inverse of a local metric. Nevertheless, efficient
implementation of inverse maintenance can significantly reduce the
per-step complexity, improving the total complexity (the number of
iterations needed for mixing times the per-step complexity).

In this section, we illustrate how our framework recovers theoretical
guarantees of previous work on $\dws$ for uniform sampling and extends
beyond uniform sampling. In particular, we show that $\gcdw$ is a
poly-time mixing algorithm capable of sampling uniform, exponential,
or Gaussian distributions on second-order cones or truncated PSD cones.
Additionally, we illustrate an efficient per-step implementation that
yields a faster total complexity when compared to general-purpose
samplers such as the $\bw$.

\subsection{Polytope sampling}

Consider a set of linear constraints given by $K=\Brace{x\in\Rn:Ax\geq b}$
with $A\in\R^{m\times n}$ and $b\in\R^{m}$. 

\paragraph{Uniform sampling.}

\cite{kannan2012random} first studied the $\dw$ for uniformly sampling
a polytope, where a local metric is set to be the Hessian of the logarithmic
barrier, $g=\hess\phi_{\textsf{log}}=A_{x}^{\top}A_{x}.$ They showed
that the $\dw$ with the log-barrier mixes in $O\Par{mn\log\frac{M}{\veps}}$
iterations with a warmness parameter $M$. An immediate consequence
of our work is that $\gcdw$ achieves the mixing time of $\otilde{mn}$
\emph{without a warmness assumption}, as $\onu,\nu=m$ and $g$ is
SSC, LTSC, and ASC by Lemma~\ref{lem:log-barrier}.

\cite{chen2018fast} introduced the $\textsf{Vaidya walk}$ and the
$\textsf{Approximate John walk}$, which are essentially $\dws$ with
the Vaidya metric $\hess\phi_{\textsf{Vaidya}}$ and a version of
the Lewis-weight metric $\sqrt{n}\hess\phi_{\textsf{Lw}}$. Their
work showed that both walks achieves mixing times of $O\Par{\sqrt{m}n^{3/2}\log\frac{M}{\veps}}$
and $O\Par{n^{5/2}\log^{O(1)}m\log\frac{M}{\veps}}$, respectively.
Building upon our analysis of the Vaidya metric and Lewis-weight metric
in Lemma~\ref{lem:vaidya} and \ref{lem:Lewis-weight}, we find that
$\gcdw$ with these metrics achieves the same mixing but without any
warmness assumption.

We note that for the same task the $\bw$ without a warm start requires
$\otilde{n^{3}}$ membership queries due to \cite{kannan1997random,jia2021reducing}.
Given that a membership query involves $O(mn)$ arithmetic operations,
the total complexity of the $\bw$ is $\otilde{mn^{4}}$. In contrast,
the per-step of the $\dw$ with the log-barrier can be run in $O\Par{mn^{\omega-1}}$
operations through the fast matrix multiplication, so the total number
of arithmetic operations is $\otilde{m^{2}n^{\omega}}$. Thus, for
$m$ close to $n$ $\gcdw$ is provably faster than the $\bw$. When
an efficient inverse maintenance proposed in \cite{laddha2020strong}
is employed, the per-step complexity can be improved to $O(n^{2}+\nnz(A))=O(mn)$.
In such cases $\gcdw$ is faster in a broader range of $m$. In particular,
if $A$ is as sparse as $\nnz(A)=O(n^{2})$, then $\gcdw$ is always
faster than the $\bw$. Moreover, $\gcdw$ with the Lewis-weight metric
mixes in $\otilde{n^{2.5}}$ steps with the per-step complexity of
$\otilde{mn^{\omega-1}}$, so it is always faster than the $\bw$
for any $m$.

\paragraph{Exponential and Gaussian sampling.}

The current mixing bound of the $\bw$ for general log-concave sampling
is $\otilde{n^{4}}$ due to \cite{lovasz2007geometry}. On the other
hand, the $\dw$ employed with any metric above for exponential sampling
converges in the same iterations as the $\dw$ for uniform sampling.
Since only difference between two sapling is the additional term of
$\exp(-f(z))/\exp(-f(x))$ in the Metropolis filter, the fast implementation
techniques mentioned earlier can be applied to the context of exponential
sampling. As a result, for the exponential sampling each of the $\dws$
described above surpasses the $\bw$ by a larger margin.

For Gaussian sampling over a polytope, we first reduce it to the exponential
sampling as in (\ref{eq:reduced-problem}): for $y=(x,t)\in\R^{n+1}$
\begin{align*}
\text{sample } & \frac{d\tilde{\pi}}{dy}(y)\propto e^{-t}\\
\text{s.t. } & Ax\geq b,\ \half\norm{x-\mu}_{\Sigma}^{2}\leq t.
\end{align*}
According to our theory, it is natural to use the metric given by
\[
g(x,t)=2\Par{\left[\begin{array}{cc}
\hess_{x}\phi_{\textsf{log}}(x)\\
 & 0
\end{array}\right]+(n+1)\hess_{(x,t)}\phi_{\textsf{Gaussian}}(x,t)},
\]
which is $\Par{O(m+n),O(m+n)}$-Dikin-amenable due to Lemma~\ref{lem:Gaussian-potential}.
Thus, $\gcdw$ needs $\otilde{n(m+n)}$ iterations of the $\dw$.
We note that the log-barrier can be replaced by the Vaidya or Lewis-weight
metrics, and in such cases one can obtain provable guarantees on the
mixing time by computing $\nu$ and $\onu$, referring to Section~\ref{sec:handbook-barrier}
or Table~\ref{tab:scaling-table}.

\subsection{Second-order cone sampling}

We consider a region given by $\norm{x-\mu'}_{\Sigma'}\leq t$ and
$A\left[\begin{array}{c}
x\\
t
\end{array}\right]\leq b$ for $A\in\R^{m\times(n+1)},b\in\R^{m},$ and $\Sigma'\in\pd$.

\paragraph{Uniform and exponential sampling.}

In this case, our self-concordance theory suggests using
\begin{align*}
2\Par{\hess(\phi_{*}+(n+1)\phi_{\textsf{SOC}})}\quad & \text{for }*=\text{log, Vaidya},\\
\text{or }2\Par{\hess(\sqrt{n+1}\phi_{\textsf{Lw}}+(n+1)\phi_{\textsf{SOC}})}\quad
\end{align*}
to deal with the truncated SOC constraint. For the log-barrier case,
this yields an $\Par{O(m+n),O(m+n)}$-Dikin-amenable metric due to
Lemma~\ref{lem:soc}, with which $\gcdw$ requires $\otilde{n(m+n)}$
iterations of the $\dw$.

\paragraph{Gaussian sampling.}

Following the reduction as in the polytope sampling, we should use
\[
g(x,t,t')=3\Par{\left[\begin{array}{cc}
\hess_{(x,t)}\phi_{\textsf{log}}(x,t)\\
 & 0
\end{array}\right]+(n+1)\left[\begin{array}{cc}
\hess_{(x,t)}\phi_{\textsf{SOC}}(x,t)\\
 & 0
\end{array}\right]+(n+2)\hess_{(x,t,t')}\phi_{\textsf{Gaussian}}(x,t,t')},
\]
which is $\Par{O(m+n),O(m+n)}$-Dikin-amenable, and thus $\gcdw$
needs $\otilde{n(m+n)}$ iterations of the $\dw$.

\subsection{PSD cone sampling\label{subsec:PSD-cone-sampling}}

For a matrix $X\in\Rnn$, recall that $\vec(X)\in\R^{n^{2}}$ denotes
the vector obtained by stacking columns of $X$ vertically. Additionally,
we define $A\in\R^{m\times n^{2}}$, $S_{X}\in\R^{m\times m}$, and
$A_{X}\in\R^{m\times n^{2}}$ by 
\[
A:=\left[\begin{array}{ccc}
\vec\Par{A_{1}} & \cdots & \vec\Par{A_{m}}\end{array}\right]^{\top},\,S_{X}:=\Diag\Par{\inner{A_{i},X}-b_{i}},\,A_{X}:=S_{X}^{-1}A,
\]
where we assume $A$ has no all-zero rows and $(S_{X})_{ii}>0$ for
$i\in[m]$. 

\paragraph{Uniform and exponential sampling.}

The metric below comes from the Hessian of the following function:
\[
-2n^{2}\log\det X-2\sum_{i=1}^{m}\log\Par{\inner{A_{i},X}-b_{i}}.
\]
Here the first term, the log-determinant, serves as a barrier for
the PSD cone while the second term is the standard logarithmic barrier
for linear constraints. We note that the $-\log\det X$ is strictly
convex on $x\in\intk$ for $K$ the truncated PSD cone, so all metrics
$g$ introduced in our main results are positive definite. Thus, the
$\dw$ with those $g$ is well-defined.
\begin{prop}
\label{thm:basicPSD} Let $K$ be the truncated PSD cone and $g$
be the local metric such that at each $X\in\intk$, for symmetric
matrices $H_{1},H_{2}$,
\[
g_{X}(H_{1},H_{2})=2n^{2}\tr\Par{X^{-1}H_{1}X^{-1}H_{2}}+2\vec(H_{1})^{\top}A_{X}^{\top}A_{X}\vec(H_{2}).
\]
Then $\gcdw$ needs $\otilde{(n^{3}+m)n^{2}}$ steps of the $\dw$
with the local metric $g$, where each step runs in $O\Par{\min(mn^{\omega}+m^{2}n^{2},n^{2\omega}+mn^{2(\omega-1)})}$
time\footnote{Here $\omega<2.373$ is the current matrix multiplication complexity
exponent (\cite{le2014powers}).}.
\end{prop}

Since $g_{X}$ is $\Par{O(m+n^{3}),O(m+n^{3})}$-Dikin-amenable due
to Lemma~\ref{lem:psd}, $\gcdw$ requires $\otilde{(n^{3}+m)n^{2}}$
iterations of the $\dw$. As mentioned earlier, efficient maintenance
of the inverse of a metric function could lead to a faster per-step
complexity. As an example, we provide such an implementation of Proposition~\ref{thm:basicPSD}
in Section~\ref{subsec:oracle-implementation}. Putting these together,
for an interesting regime of $m=O(1)$ $\gcdw$ is faster than the
$\bw$ by a factor of $n$ in terms of the total complexity.

If we replace the log-barrier by the Vaidya metric, then the dependence
on $m$ is improved to $\sqrt{m}$ as in the polytope sampling:
\begin{prop}
\label{thm:hybridPSD} Let $K$ be the truncated PSD cone and $g$
be the local metric such that at each $X\in\intk$, for symmetric
matrices $H_{1},H_{2}$,
\[
g_{X}(H_{1},H_{2})=2n^{2}\tr\Par{X^{-1}H_{1}X^{-1}H_{2}}+44\sqrt{\frac{m}{n}}\vec(H_{1})^{\top}A_{X}^{\top}\Par{\Sigma_{X}+\frac{n}{m}I_{m}}A_{X}\vec(H_{2}).
\]
Then $\gcdw$ needs $\otilde{\Par{n^{2}+\sqrt{m}}n^{3}}$ steps of
the $\dw$ with the local metric $g$, with each step running in $\otilde{mn^{2(\omega-1)}}$
amortized time.
\end{prop}

\begin{proof}
We set
\begin{align*}
g(X) & =2\Par{n^{2}g_{1}(X)+g_{2}(X)},\ \text{where}\\
g_{1}(X) & =M^{\top}(X\kro X)^{-1}M,\\
g_{2}(X) & =22\sqrt{\frac{m}{n}}M^{\top}A_{X}^{\top}\Par{\Sigma_{X}+\frac{n}{m}I_{m}}A_{X}M.
\end{align*}
Since $n^{2}g_{1}$ and $g_{2}$ are SSC, $g$ is also SSC due to
Lemma~\ref{lem:ssc-sum} and $O(n^{3}+\sqrt{mn^{2}})$-symmetric\footnote{Since the dimension is $d$ in the PSD setting, we should replace
$n$ by $d=O(n^{2})$ when applying Lemma~\ref{lem:paramsBarrier}.} due to Lemma~\ref{lem:symmetry-addition}. As $n^{2}g_{1}$ and
$g_{2}$ is SLTSC and SASC, $g$ is LTSC and ASC. Putting these together,
it follows that $g$ is $\Par{O(n^{3}+\sqrt{mn^{2}}),O(n^{3}+\sqrt{mn^{2}})}$-Dikin-amenable.
Therefore, Theorem~\ref{thm:Dikin-annealing} implies that $\gcdw$
incurs $\otilde{n^{2}\Par{n^{3}+\sqrt{mn^{2}}}}=\otilde{n^{3}\Par{n^{2}+\sqrt{m}}}$
total iterations of the $\dw$ with $g$.

Now we bound the per-step complexity of the $\dw$ (Algorithm~\ref{alg:DikinWalk}).
Recall that it requires (1) the update of the leverage scores, (2)
computation of the matrix function induced by the local metric $g$,
(3) the inverse of the matrix function and (4) its determinant. By
Theorem~46 in \cite{lee2019solving} (with $p=2$ and $n\gets d$
therein), the initialization of the leverage scores at the beginning
takes $\otilde{mn^{2\omega}}$ and their updates takes $\otilde{mn^{2(\omega-1)}}$
time. Since (1) takes $\otilde{mn^{2(\omega-1)}}$, (2) takes $\otilde{n^{4}+mn^{2(\omega-1)}}$,
and (3) and (4) take $O\Par{n^{2\omega}}$, each iteration runs in
$\otilde{n^{2\omega}+mn^{2(\omega-1)}}$ time. Note that even though
the initialization of leverage scores takes $\otilde{mn^{2\omega}}$
time, the amortized per step time complexity (since the mixing rate
is $\otilde{n^{3}(n^{2}+\sqrt{m})}$) goes to $\otilde{n^{2\omega}+mn^{2(\omega-1)}}=\otilde{mn^{2(\omega-1)}}$
time.
\end{proof}
Lastly, the dependence on $m$ can be made poly-logarithmic by working
with the Lewis-weight metric. We remark that for uniform sampling
the total complexity of $\gcdw$ is less than that of the $\bw$ by
the order of $n^{5-2\omega}$.
\begin{prop}
\label{thm:LSPSD} Let $K$ be the truncated PSD cone and $g$ be
the local metric such that at each $X\in\intk$, for symmetric matrices
$H_{1},H_{2}$, 
\[
g_{X}(H_{1},H_{2})=2n^{2}\tr\Par{X^{-1}H_{1}X^{-1}H_{2}}+c_{1}\Par{\log m}^{c_{2}}n\vec(H_{1})^{\top}A_{X}^{\top}W_{X}A_{X}\vec(H_{2}),
\]
where $W_{X}$ is the diagonalized $\ell_{p}$-Lewis weight of $A_{X}$
with $p=O(\log m)$, and $c_{1},c_{2}>0$ are universal constants.
Then $\gcdw$ requires $\otilde{n^{5}}$ steps of the $\dw$, with
each step running in $\otilde{mn^{2(\omega-1)}}$ amortized time.
\end{prop}

\begin{proof}
We set
\begin{align*}
g(X) & =2\Par{n^{2}g_{1}(X)+g_{2}(X)},\ \text{where}\\
g_{1}(X) & =-\hess_{X}\log\det X=M^{\top}(X\kro X)^{-1}M,\\
g_{2}(X) & =c_{1}\Par{\log m}^{c_{2}}nM^{\top}A_{X}^{\top}W_{X}A_{X}M\ \text{for some constants \ensuremath{c_{1},c_{2}>0}.}
\end{align*}
Since $n^{2}g_{1}$ and $g_{2}$ are SSC, $g$ is also SSC due to
Lemma~\ref{lem:ssc-sum} and $O^{*}\Par{n^{3}}$-symmetric due to
Lemma~\ref{lem:symmetry-addition}. As $n^{2}g_{1}$ and $g_{2}$
is SLTSC and SASC, $g$ is LTSC and ASC. Putting these together, it
follows that $g$ is $\Par{O^{*}\Par{n^{3}},O^{*}\Par{n^{3}}}$-Dikin-amenable.
Therefore, Theorem~\ref{thm:Dikin-annealing} implies that $\gcdw$
requires $\otilde{n^{5}}$ iterations of the $\dw$ with $g$. Since
the initialization and update of the Lewis weight takes $\otilde{mn^{2\omega}}$
and $\otilde{mn^{2(\omega-1)}}$ time (Theorem~46 in \cite{lee2019solving}),
the same implementation with Theorem~\ref{thm:hybridPSD} also has
the time complexity of $\otilde{mn^{2(\omega-1)}}$.
\end{proof}

\paragraph{Gaussian sampling.}

Just as in polytope or second-order cone sampling, we introduce a
new variable $t$ by replacing a quadratic term in the potential.
This reduces the Gaussian sampling problem to an exponential sampling
problem. We then work with a local metric 
\[
g(X,t)=3\Par{n\left[\begin{array}{cc}
\hess_{X}\phi_{\textsf{Lw}}(X)\\
 & 0
\end{array}\right]+n^{2}\left[\begin{array}{cc}
\hess_{X}\phi_{\textsf{PSD}}(X)\\
 & 0
\end{array}\right]+n^{2}\hess_{(X,t)}\phi_{\textsf{Gaussian}}(X,t)},
\]
which is $\Par{O^{*}(n^{3}),O^{*}(n^{3})}$-Dikin-amenable. Thus,
$\gcdw$ needs $\otilde{n^{5}}$ iterations of the $\dw$ with the
local metric $g$, and the per-step complexity remains $\otilde{mn^{2(\omega-1)}}$
in amortized time.

\subsubsection{Per-step implementation \label{subsec:oracle-implementation}}

Now we design an oracle that implements each iteration of the $\dw$
(Algorithm~\ref{alg:DikinWalk}). This can be implemented as follows:
when the current point is $x$,
\begin{itemize}
\item Sample $z\sim\ncal\Par{0,\frac{r^{2}}{n}g(x)^{-1}}$.
\item Compute $y=x+g(x)^{-\half}z$ and propose it.
\item Accept $y$ with probability $\min\Par{1,\sqrt{\frac{\det g(y)}{\det g(x)}}\frac{\exp\Par{f(x)}}{\exp\Par{f(y)}}}$.
\end{itemize}
We provide two algorithms with the complexity of $O\Par{mn^{\omega}+m^{2}n^{2}}$
and $O\Par{n^{2\omega}+mn^{2(\omega-1)}}$. We can implement each
iteration in $O\Par{\min\Par{mn^{\omega}+m^{2}n^{2},n^{2\omega}+mn^{2(\omega-1)}}}$
time by using the former for small $m$ and the latter for large $m$.
This completes the second half of Theorem~\ref{thm:basicPSD}. 

\paragraph{Algorithm for small $m$.}

For simplicity here, we ignore the constant factors of $g$, denoting
$g=g_{1}+g_{2}$ for
\begin{align*}
g_{1}(X) & =M^{\top}(X\kro X)^{-1}M=:BB^{\top},\\
g_{2}(X) & =M^{\top}A^{\top}S_{X}^{-2}AM=:UU^{\top},
\end{align*}
where $B:=M^{\top}(X\kro X)^{-1/2}\in\R^{d\times n^{2}}$ and $U:=M^{\top}A^{\top}S_{X}^{-1}\in\R^{d\times m}$.
Letting $u_{i}$ be the $i^{th}$ column of $U$ for $i\in[m]$, we
note that $g_{2}=\sum_{i=1}^{m}u_{i}u_{i}^{\top}$.

We first implement a subroutine for computing $g(X)^{-1}v$ for a
given vector $v\in\R^{d}$ in $O(mn^{\omega}+m^{2}n^{2})$ time.

\begin{algorithm2e}[t]

\caption{Computation of $g(X)^{-1}v$}\label{alg:subroutine}

\SetAlgoLined

\textbf{Input:} $X\in\psd$, vector $v\in\R^{d}$, local metric $g$.

\textbf{Output:} $g(X)^{-1}v$

Prepare the column vectors $u_{i}$ of $U=M^{\top}A^{\top}S_{X}^{-1}$.

For $\bar{g}_{0}:=g_{1}(X)$, compute $\bar{g}_{0}^{-1}v$ and $\bar{g}_{0}^{-1}u_{i}$
for $i\in[m]$.

\For{$i=1,\cdots,m$}{

Compute $\bar{g}_{i}^{-1}v$ and $\bar{g}_{i}^{-1}u_{j}$ for $j\in[m]$,
according to 

\[
\bar{g}_{i}^{-1}w=\bar{g}_{i-1}^{-1}w-\frac{\bar{g}_{i-1}^{-1}u_{i}\cdot u_{i}^{\top}\bar{g}_{i-1}^{-1}w}{1+u_{i}^{\top}\bar{g}_{i-1}^{-1}u_{i}}.
\]

}

Return $\bar{g}_{m}^{-1}v$.

\end{algorithm2e}
\begin{prop}
\label{prop:oracle} Algorithm~\ref{alg:subroutine} computes $g(X)^{-1}v$
in $O(mn^{\omega}+m^{2}n^{2})$ time for a query vector $v\in\R^{d}$.
\end{prop}

\begin{proof}
Let $v\in\R^{d}$ be a given vector, and denote $\bar{g}_{0}:=g_{1}$
and $\bar{g}_{i}:=\bar{g}_{i-1}+u_{i}u_{i}^{\top}$ for $i\in[m]$.
We first prepare the column vectors $u_{i}$'s of $U=M^{\top}A^{\top}S_{X}^{-1}$
in $O(mn^{2})$ time and then initialize $\bar{g}_{0}^{-1}v$ and
$\bar{g}_{0}^{-1}u_{i}$ for $i\in[m]$ in $O(mn^{\omega})$ time.
For $u_{i}$'s, note that $S_{X}$ can be prepared in $O(mn^{2})$
time, and thus $A^{\top}S_{X}^{-1}$ takes $O(mn^{2})$ time due to
$A\in\R^{n^{2}\times m}$. Since each row of $M^{\top}\in\R^{d\times n^{2}}$
has at most two non-zero entries, we can obtain $u_{i}$'s in $O(mn^{2})$
time.

For $\bar{g}_{0}^{-1}v$ and $\bar{g}_{0}^{-1}u_{i}$, we recall from
Lemma~\ref{prop:metricFormula} that for a vector $z\in\R^{d}$ 
\begin{align*}
g_{1}^{-1}z & =M^{\dagger}(X\kro X)M^{\dagger\top}z=LN(X\kro X)NL^{\top}z.
\end{align*}
Since each row of $L^{\top}\in\R^{n^{2}\times d}$ has at most two
non-zero entries, $w:=L^{\top}z\in\R^{n^{2}}$ can be computed in
$O(n^{2})$ time. From the definition of $N$, it follows that $Nw=\vec\Par{\half(W+W^{\top}}$
for $W:=\vec^{-1}(w)\in\R^{n\times n}$, which also can be computed
in $O(n^{2})$ time. For $\overline{W}:=\half\Par{W+W^{\top}}$, it
follows that
\begin{align*}
(X\kro X)Nw & =(X\kro X)\vec\Par{\overline{W}}\underset{\text{Lemma \ref{lem:Kronecker}-1}}{=}\vec\Par{X\overline{W}X},
\end{align*}
which can be computed in $O(n^{\omega})$ time by the fast matrix
multiplication, and in a similar way we can compute $LN\vec\Par{X\overline{W}X}$
in $O(n^{2})$ time. Putting all these together, $\bar{g}_{0}^{-1}v$
can be computed in $O(n^{\omega})$ time, and repeating this for $u_{j}$'s
yields $\Brace{\bar{g}_{0}^{-1}v,\bar{g}_{0}^{-1}u_{1},\dots,\bar{g}_{0}^{-1}u_{m}}$
in $O(mn^{\omega})$ time.

Starting with these initializations, we recursively use the Sherman--Morrison
formula: for a given vector $z\in\R^{d}$
\begin{equation}
\bar{g}_{i}^{-1}z=\bar{g}_{i-1}^{-1}z-\frac{\bar{g}_{i-1}^{-1}u_{i}u_{i}^{\top}\bar{g}_{i-1}^{-1}z}{1+u_{i}^{\top}\bar{g}_{i-1}^{-1}u_{i}}.\label{eq:sherman-morrison}
\end{equation}
Using $\bar{g}_{i-1}^{-1}u_{j}$ and $\bar{g}_{i-1}^{-1}v$ from a
previous iteration, we can compute each of $\bar{g}_{i}^{-1}u_{j}$
and $\bar{g}_{i}^{-1}v$ in the current iteration in $O(n^{2})$ time,
and thus each round for update takes $O(mn^{2})$ time in total. Since
we iterate for $m$ rounds, Algorithm~\ref{alg:subroutine} outputs
$\bar{g}_{m}^{-1}v=g(X)^{-1}v$ in $O(mn^{\omega}+m^{2}n^{2})$ time.
\end{proof}
With this subroutine in hand, we proceed to an efficient implementation
of two tasks -- computation of (1) $g(x)^{-\half}z$ for a given
vector $z\in\R^{d}$ and (2) $\sqrt{\frac{\det g(y)}{\det g(x)}}\frac{\exp\Par{f(x)}}{\exp\Par{f(y)}}$.

\begin{algorithm2e}[t]

\caption{Implementation of the $\dw$}\label{alg:perStep-small-m}

\SetAlgoLined

\textbf{Input:} current point $X\in\psd$, local metric $g$

\tcp{Step 1: Sampling from $\ncal\Par{0,\frac{r^{2}}{n}g(X)^{-1}}$}

Draw $w\sim\ncal\Par{0,I_{n^{2}+m}}$ and $v\gets g(X)^{-1}\left[\begin{array}{cc}
B & U\end{array}\right]w$ by Algorithm~\ref{alg:subroutine}.\

Propose $y\gets\svec(X)+\frac{r}{\sqrt{n}}v$.

\

\tcp{Step 2: Computation of acceptance probability}

Use Algorithm~\ref{alg:subroutine} to prepare $\Brace{\bar{g}_{i}^{-1}u_{1},\dots,\bar{g}_{i}^{-1}u_{m}}_{i=0}^{m}$
at $X$ and $Y:=\svec^{-1}(y)$.\

$\det\bar{g}_{0}(\cdot)\gets2^{n(n-1)/2}(\det(\cdot))^{-(n+1)}$ ($\because$
Lemma~\ref{lem:Kronecker}-7)

\For{$i=1,\cdots,m$}{

$\det(\bar{g}_{i+1})\gets\det\bar{g}_{i}\cdot\Par{1+u_{i+1}^{\top}\bar{g}_{i}^{-1}u_{i+1}}$.

}

Accept $Y$ with probability $\min\Par{1,\sqrt{\frac{\det\bar{g}_{m}(Y)}{\det\bar{g}_{m}(X)}}\frac{\exp\Par{f(X)}}{\exp\Par{f(Y)}}}$.

\end{algorithm2e}
\begin{lem}
\label{lem:perStep-small-m}Algorithm~\ref{alg:perStep-small-m}
implements each iteration of the $\dw$ with complexity of $O\Par{mn^{\omega}+m^{2}n^{2}}$.
\end{lem}

\begin{proof}
Here we provide details of Algorithm~\ref{alg:perStep-small-m} in
two stages -- (1) sampling from $\ncal\Par{0,\frac{r^{2}}{n}g(x)^{-1}}$
and (2) computation of acceptance probability.

\paragraph{(1) Gaussian sampling:}

For simplicity, we ignore $r^{2}/n$ and illustrate how to draw $v\sim\ncal(0,g(X)^{-1})$
without full computation of $g(X)^{-1}$ in $O(mn^{\omega}+m^{2}n^{2})$
time.

Our approach is to compute $v:=g(X)^{-1}\left[\begin{array}{cc}
B & U\end{array}\right]w$ for $w\sim\ncal(0,I_{n^{2}+m})$, which follows the Gaussian distribution
with covariance
\begin{align*}
g(X)^{-1}\left[\begin{array}{cc}
B & U\end{array}\right]\Par{g(X)^{-1}\left[\begin{array}{cc}
B & U\end{array}\right]}^{\top} & =g(X)^{-1}(BB^{\top}+CC^{\top})g(X)^{-1}\\
 & =g(X)^{-1}g(X)g(X)^{-1}\\
 & =g(X)^{-1},
\end{align*}
since $v$ is a linear transformation of the Gaussian random variable
$w$.

Denoting $w=\left[\begin{array}{c}
w_{b}\\
w_{u}
\end{array}\right]$ for $w_{b}\sim\ncal(0,I_{n^{2}})$ and $w_{u}\sim\ncal(0,I_{m})$,
we can show that $\left[\begin{array}{cc}
B & U\end{array}\right]w$ can be computed in $O(n^{\omega}+mn^{2})$ time as follows:
\begin{align*}
\left[\begin{array}{cc}
B & U\end{array}\right]w & =Bw_{b}+Uw_{c}\\
 & =M^{\top}\underbrace{(X\kro X)^{-\half}w_{b}}_{\text{Use Lemma \ref{eq:sherman-morrison}}}+M^{\top}A^{\top}S_{X}^{-1}w_{c}\\
 & =M^{\top}\Par{\vec\Par{X^{-\half}\vec^{-1}(w_{b})X^{-\half}}+A^{\top}S_{X}^{-1}w_{c}},
\end{align*}
where $\vec\Par{X^{-\half}\vec^{-1}(w_{b})X^{-\half}}$ and $A^{\top}S_{X}^{-1}w_{u}$
can be computed in $O(n^{\omega})$ and $O(mn^{2})$ time, respectively.
Since each row of $M^{\top}\in\R^{d\times n^{2}}$ has at most two
non-zero entries, $\left[\begin{array}{cc}
B & U\end{array}\right]w$ can be computed in $O(n^{\omega}+mn^{2})$ time. Using Algorithm~\ref{alg:subroutine},
we obtain $v=g(X)^{-1}\left[\begin{array}{cc}
B & U\end{array}\right]w$ in $O(mn^{\omega}+m^{2}n^{2})$ time.

\paragraph{(2) Computation of acceptance probability. }

We show that this step also takes $O(mn^{\omega}+m^{2}n^{2})$ time.
To compute $\det g(X)$, we use Algorithm~\ref{alg:subroutine} to
prepare $\Brace{\bar{g}_{i}^{-1}u_{1},\dots,\bar{g}_{i}^{-1}u_{m}}_{i=0}^{m}$
at $X$ and $Y=\svec^{-1}(y)$ in $O(mn^{\omega}+m^{2}n^{2})$ time.
Recall the matrix determinant lemma:
\[
\det\Par{A+uu^{\top}}=\det A\cdot\Par{1+u^{\top}A^{-1}u}.
\]
 Using the following recursive formula
\begin{align*}
\det\Par{\bar{g}_{i+1}} & =\det\Par{\bar{g}_{i}+u_{i+1}u_{i+1}^{\top}}=\det\bar{g}_{i}\cdot\Par{1+u_{i+1}^{\top}\bar{g}_{i}^{-1}u_{i+1}},
\end{align*}
we start with $\det\bar{g}_{0}=\det g_{1}=2^{n(n-1)/2}\Par{\det X}^{-(n+1)}$
(see Lemma~\ref{lem:Kronecker}-7), which can be computed in $O(n^{\omega})$
time, and compute $\det g(X)$ (and $\det g(Y)$ in the same way)
in $O(mn^{\omega}+m^{2}n^{2})$ time.
\end{proof}


\paragraph{Algorithm for large $m$.}

The algorithm right above has quadratic dependence on the number $m$
of constraints, which could become expensive for large $m$. In this
regime, we just fully compute the whole matrix function of size $\R^{d\times d}$,
which takes $O(n^{2\omega}+mn^{2(\omega-1)})$ time, and computing
its inverse, square-root, and determinant takes $O(n^{2\omega})$
time. 

\subsubsection{Handling approximate Lewis weights \label{subsec:app-lewis-weight}}

In the implementation of the $\dw$ with the Lewis-weights metric,
we use an approximation algorithm presented in \cite{lee2019solving}
for computing and updating the Lewis weight, which ensures 
\[
(1-\delta)\wtilde_{X}\preceq W_{X}\preceq(1+\delta)\wt W_{X}
\]
for the approximate Lewis weights $\wtilde_{X}$ and a target accuracy
parameter $\delta$ (note that the initialization and update times
of the Lewis weight above hide poly-logarithmic dependence on $\log(1/\delta)$).
Strictly speaking, we should check that these approximate Lewis weights
do not affect the theoretical guarantees above.

To see this, let us define
\begin{align*}
\widetilde{g}(X) & =2\Par{ng_{1}(X)+\widetilde{g_{2}}(X)},\ \text{where}\\
g_{1}(X) & =-n^{2}\hess_{X}\log\det X=n^{2}M^{\top}(X\kro X)^{-1}M,\\
\wt g_{2}(X) & =c_{1}\Par{\log m}^{c_{2}}nM^{\top}A_{X}^{\top}\widetilde{W}_{X}A_{X}M\ \text{for some constants \ensuremath{c_{1},c_{2}>0}.}
\end{align*}
First of all, the $\dw$ with $\widetilde{g}$ still converges to
a target distribution, since the approximation algorithm in \cite{lee2019solving}
is deterministic and thus the condition of detailed balance still
holds under the acceptance probability of $\min\Par{1,\sqrt{\frac{\det\tilde{g}(Y)}{\det\tilde{g}(X)}}\frac{\exp\Par{f(X)}}{\exp\Par{f(Y)}}}$.
For $\widetilde{P}_{X}$ the one-step distribution of the $\dw$ started
at $X$ with $\widetilde{g}$, we can show one-step coupling similar
to Lemma~\ref{lem:one-step}, following the overall proof therein
and taking $\delta=1/\text{poly}(n)$ small enough.
\begin{lem}
[One-step coupling] For convex $K\subset\Rn$, let $g:\intk\to\pd$
be SSC, ASC, LTSC, and $\phi:\intk\to\R$ be its function counterpart.
Suppose that the potential $f$ of the target distribution $\pi$
is $\beta$-relatively smooth in $\phi$. Then there exist constants
$s_{1},s_{2}>0$ such that if $\norm{x-y}_{g(x)}\leq s_{1}\frac{r}{\sqrt{n}}$
with $r=s_{2}\min\Par{1,\frac{1}{\sqrt{\beta}}}$ for $x,y\in\intk$,
then $\dtv\Par{\widetilde{P}_{x},\widetilde{P}_{y}}\leq\frac{3}{4}+0.01$.
\end{lem}

\begin{proof}
We just reproduce the proof of Lemma~\ref{lem:one-step}. For $\frac{d\pi}{dx}\propto\exp\Par{-f(x)}\cdot\mathbf{1}_{K}(x)$,
we denote
\begin{align*}
g_{x}(z) & :=\frac{d\ncal\Par{x,\frac{r^{2}}{n}\tilde{g}(x)^{-1}}}{dx}(z),\\
R(x,z) & =\frac{g_{z}(x)}{g_{x}(z)}\frac{\exp\Par{-f(z)}}{\exp\Par{-f(x)}},\\
A(x,z) & =\min\Par{1,\mathbf{1}\Par{z\in K}\cdot R(x,z)}.
\end{align*}
Then the transition kernel of the $\dw$ started at $x$ can be written
as 
\begin{align*}
\widetilde{P}(x,dz) & =\underbrace{\Par{1-\int A(x,z')g_{x}(z')dz'}}_{=:r_{x}}\delta_{x}(dz)+A(x,z)g_{x}(z)dz,\text{ so}\\
\widetilde{P}(x,S) & =r_{x}\dx(S)+\int_{S}A(x,z)g_{x}(z)dz\ \text{for any measurable set }S.
\end{align*}
Thus, for $x,y\in\intk$
\begin{align*}
\dtv(P_{x},P_{y}) & =\underbrace{\half\Par{r_{x}+r_{y}}}_{\textsf{I}}+\underbrace{\half\int\Abs{A(x,z)g_{x}(z)-A(y,z)g_{y}(z)}dz}_{\textsf{II}}.
\end{align*}

We note that $(1-\delta)\wt g_{2}\preceq g_{2}\preceq(1+\delta)\wt g_{2}$
and thus 
\begin{equation}
(1-\delta)\wt g\preceq g\preceq(1+\delta)\wt g,\label{eq:closeness-approx}
\end{equation}
and this implies $(1-\delta)I\preceq\wt g^{-\half}g\wt g^{-\half}\preceq(1+\delta)I$.
Hence, $(1-\delta)^{n^{2}/2}\leq\sqrt{\frac{\det g}{\det\wt g}}\leq(1+\delta)^{n^{2}/2}$
and
\begin{align}
(1-\delta)^{n^{2}}\sqrt{\frac{\det\wt g(z)}{\det\wt g(x)}} & \leq\sqrt{\frac{\det g(z)}{\det g(x)}}\leq(1+\delta)^{n^{2}}\sqrt{\frac{\det\wt g(z)}{\det\wt g(x)}}.\label{eq:similar-ratio-approx}
\end{align}

With this in mind, recall that 
\[
r_{x}=1-\int A(x,z)g_{x}(z)dz=1-\int\min\bigg(1,\,\underbrace{\mathbf{1}\Par{z\in K}\frac{e^{-f(z)}}{e^{-f(x)}}}_{=:\textsf{A}}\underbrace{\frac{g_{z}(x)}{g_{x}(z)}}_{=:\textsf{B}}\bigg)g_{x}(z)dz.
\]
We can bound $\textsf{A}$ in a similar way by using (\ref{eq:closeness-approx}).
For $\textsf{B}$, 
\[
\log\text{\textsf{B}}=-\frac{n}{2r^{2}}\Par{\norm{z-x}_{z}^{2}-\norm{z-x}_{x}^{2}}+\half\Par{\log\det\widetilde{g}(z)-\log\det\widetilde{g}(x)}.
\]
As in Lemma~\ref{lem:one-step}, the second term can be bounded lower
by $\exp\Par{-3\veps}$ using (\ref{eq:similar-ratio-approx}). The
first term can be lower-bounded by invoking ASC of $g$. To see this,
ignoring the normalization constant of $g_{x}$
\begin{align*}
(*)= & \int\mathbf{1}\Par{\norm{z-x}_{\widetilde{g}(z)}^{2}-\norm{z-x}_{\widetilde{g}(x)}^{2}\leq2\veps\frac{r^{2}}{n}}\sqrt{\Abs{\widetilde{g}(x)}}\exp\Par{-\half\norm{z-x}_{\widetilde{g}(x)}^{2}}dz\\
= & \int\mathbf{1}\Par{\norm{z-x}_{\widetilde{g}(z)}^{2}-\norm{z-x}_{\widetilde{g}(x)}^{2}\leq2\veps\frac{r^{2}}{n}}\sqrt{\Abs{g(x)}}\exp\Par{-\half\norm{z-x}_{g(x)}^{2}}\\
 & \qquad\cdot\sqrt{\Abs{\frac{\widetilde{g}(x)}{g(x)}}}\exp\Par{-\half\Par{\norm{z-x}_{\widetilde{g}(x)}^{2}-\norm{z-x}_{g(x)}^{2}}}dz\\
\leq & \int\mathbf{1}\Par{\norm{z-x}_{\widetilde{g}(z)}^{2}-\norm{z-x}_{\widetilde{g}(x)}^{2}\leq2\veps\frac{r^{2}}{n}}\sqrt{\Abs{g(x)}}\exp\Par{-\half\norm{z-x}_{g(x)}^{2}}\\
 & \qquad\cdot(1+\delta)^{n^{2}/2}\exp\Par{\frac{\delta}{2}\norm{z-x}_{g(x)}^{2}}dz.
\end{align*}
Due to $\norm{z-x}_{g(x)}^{2}\lesssim r^{2}$ w.h.p., taking $\delta=\veps/n^{10}$
leads to 
\[
(*)\leq2\int\mathbf{1}\Par{\norm{z-x}_{\widetilde{g}(z)}^{2}-\norm{z-x}_{\widetilde{g}(x)}^{2}\leq2\veps\frac{r^{2}}{n}}\sqrt{\Abs{g(x)}}\exp\Par{-\half\norm{z-x}_{g(x)}^{2}}dz.
\]
Also, due to 
\begin{align*}
\norm{z-x}_{\widetilde{g}(z)}^{2}-\norm{z-x}_{\widetilde{g}(x)}^{2} & \geq(1-\delta)\norm{z-x}_{g(z)}^{2}-(1+\delta)\norm{z-x}_{g(x)}^{2}\\
 & =(1-\delta)\Par{\norm{z-x}_{g(z)}^{2}-\norm{z-x}_{g(x)}^{2}}-2\delta\norm{z-x}_{g(x)}^{2},
\end{align*}
we have
\begin{align*}
(*) & \leq2\int\mathbf{1}\Par{\norm{z-x}_{g(z)}^{2}-\norm{z-x}_{g(x)}^{2}\leq2\veps(1-\delta)^{-1}\frac{r^{2}}{n}+\veps\frac{r^{2}}{n}}\sqrt{\Abs{g(x)}}\exp\Par{-\half\norm{z-x}_{g(x)}^{2}}dz\leq6\veps
\end{align*}
by invoking ASC of $g$ in the last inequality. Putting these together,
$\textsf{I}\leq\half+O(1)\veps$. For $\textsf{II}$, we can follow
the proof of Lemma~\ref{lem:one-step} to show $\textsf{II}\leq\frac{1}{4}+O(1)\veps$,
and every technical issue can be resolved by repeating the same techniques
above.
\end{proof}



\global\long\def\vec{\textup{\textsf{vec}}}%
\global\long\def\svec{\textup{\textsf{svec}}}%


\section{Proofs \label{sec:proofs}}

We collect deferred proofs in this section.

\subsection{Mixing of \texorpdfstring{$\msf{Dikin\ walk}$}{Dikin walk} ($\S$\ref{sec:mixing-Dikin})}

\subsubsection{One-step coupling \label{proof:onestep}}

We start with the one-step coupling of the $\dw$ under the setting
$\alpha\hess\phi\preceq\hess f\preceq\beta\hess\phi$ on $\intk$.
Roughly speaking, if $\snorm{x-y}_{x}\leq r/\sqrt{d}$ with $r\lesssim1\wedge\nicefrac{1}{\sqrt{\beta}}$,
then $\dtv(P_{x},P_{y})\leq0.99$.
\begin{proof}
[Proof of Lemma~\ref{lem:one-step}] For $\pi\propto\exp(-f)\cdot\mathbf{1}_{K}$
and $z\sim\ncal(x,\frac{r^{2}}{d}g(x)^{-1})$, let us denote
\[
p_{x}=\ncal\Bpar{x,\frac{r^{2}}{d}g(x)^{-1}},\qquad R(x,z)=\frac{p_{z}(x)}{p_{x}(z)}\frac{\pi(z)}{\pi(x)},\qquad A(x,z)=\min\bpar{1,R(x,z)\,\mathbf{1}_{K}(z)}\,.
\]
The transition kernel $P(x,\cdot)$ of the $\dw$ started at $x$
can be written as 
\[
P(x,dz)=\underbrace{\bpar{1-\E_{p_{x}}[A(x,\cdot)]}}_{\eqqcolon r_{x}}\,\delta_{x}(\D z)+A(x,z)\,p_{x}(\D z)\,.
\]
Thus, for $x,y\in\intk$,
\begin{align}
\dtv(P_{x},P_{y}) & =\underbrace{\half(r_{x}+r_{y})}_{\msf I}+\underbrace{\half\int|A(x,z)\,p_{x}(z)-A(y,z)\,p_{y}(z)|\,\D z}_{\msf{II}}\,.\tag{\ensuremath{\msf{TV\text{-}decomposition}}}\label{eq:tv-formula}
\end{align}

Let $h\sim\ncal(0,I_{d})$ and denote a bad event $B_{0}=\{z\in\Rd:\snorm{z-x}_{x}\ge cr\}$
with $c$ determined later. Due to $\snorm{z-x}_{x}=\frac{r}{\sqrt{d}}\snorm h$
(in law) and concentration of the standard Gaussian in a thin shell
of radius $\sqrt{d}$ with annulus $\mc O(1)$\footnote{A standard Gaussian $h\sim\ncal(0,I_{d})$ is concentrated around
a thin sell of radius $\sqrt{d}$ with annulus $\mc O(1)$: For $t>0$,
\[
\P_{h}(\norm h_{2}\geq\sqrt{d}+t)\leq\exp(-t^{2}/2)\,.
\]
}, we have $\P_{z}(B_{0})=\P_{h}(\snorm h\geq c\sqrt{d})\leq\exp\bpar{-(c-1)\sqrt{d}/2}$.
Hence, $\P(B_{0})\leq\veps$ for $c\geq1+\sqrt{\frac{2}{d}\,\log\frac{1}{\veps}}$. 

\paragraph{Rejection probability $r_{x}$ and $r_{y}$ (Term $\protect\msf I$).}

Note that
\[
r_{x}=1-\E_{p_{x}}[A(x,z)]=1-\int\min\Bpar{1,\,\underbrace{\mathbf{1}_{K}(z)\frac{\exp f(x)}{\exp f(z)}}_{\eqqcolon\textsf{A}}\underbrace{\frac{p_{z}(x)}{p_{x}(z)}}_{\eqqcolon\textsf{B}}}p_{x}(\D z)\,.
\]

As for $\textsf{A}$, we let $\hess\phi\preceq c_{\phi}g$ for some
$c_{\phi}>0$ and use Taylor's expansion at $x\in K\cap B_{0}^{c}$
to show that for some $x^{*}\in[x,z]$, 
\begin{align*}
f(x)-f(z) & +\nabla f(x)^{\T}(z-x)=-\snorm{z-x}_{\hess f(x^{*})}^{2}\geq-c_{\phi}\beta\,\snorm{z-x}_{g(x^{*})}^{2}\\
 & \underset{\text{(i)}}{\geq}-c_{\phi}\beta\,\snorm{z-x}_{x}^{2}\cdot(1+2\snorm{x-z}_{x})^{2}\geq-c_{\phi}\beta c^{2}r^{2}(1+2cr)^{2}\underset{\text{(ii)}}{\geq}-\veps\,,
\end{align*}
where we used Lemma~\ref{lem:scCloseness} in (i) and took $r\leq r_{1}(\veps)$
in (ii), which is defined so that $\beta c_{\phi}c^{2}r^{2}(1+cr)^{2}\leq\veps$
for any $r\leq r_{1}(\veps)$. It follows from $\dcal_{g}^{1}(x)\subset K$
and symmetry of $\ncal_{g}^{r}(x)$ that there exists a half-ellipsoid
$G\subset\dcal_{g}^{1}(x)$ in which $\inner{\grad f(x),z-x}\leq0$.
Thus, $f(x)-f(z)\geq-\veps$ holds on $z\in G$.

For a bad event $B_{1}:=G^{c}$, it holds that 
\[
\P_{z}(B_{1})\leq\half+\P_{z}\bpar{\dcal_{g}^{1}(x)^{c}}=\half+\P_{z}(\snorm{z-x}_{x}\geq1)=\half+\P_{h}\Bpar{\norm h\geq\frac{\sqrt{d}}{r}}\leq\half+\veps\,,
\]
where the last inequality follows from concentration of $h$ for any
$r\leq r_{2}(\veps):=\bpar{1+\frac{2}{\sqrt{d}}\,\log\frac{1}{\veps}}^{-1}$. 

As for $\textsf{B}$, for $\vphi(x):=\half\log\det g(x)$ we have
\[
\log\text{\textsf{B}}=-\frac{d}{2r^{2}}\bpar{\snorm{z-x}_{z}^{2}-\snorm{z-x}_{x}^{2}}+\bpar{\vphi(z)-\vphi(x)}\,.
\]
Invoking ASC of $\phi$, we can take $r_{3}(\veps)$ so that $\P_{z}\bpar{\snorm{z-x}_{z}^{2}-\snorm{z-x}_{x}^{2}\leq2\veps r^{2}/d}\geq1-\veps$
for any $r\leq r_{3}(\veps)$ and control the first term. Let the
complement of this event be our second bad event $B_{2}$.

For $\vphi(z)-\vphi(x)$, Taylor's expansion of $\vphi$ at $x$ leads
to 
\[
\vphi(z)-\vphi(x)=\underbrace{\inner{\grad\vphi(x),z-x}}_{\eqqcolon\textsf{A'}}+\underbrace{\half\inner{\hess\vphi(x^{*})(z-x),z-x)}}_{\eqqcolon\textsf{B'}}\text{ for some }x^{*}\in[x,z]\,.
\]
As for $\textsf{A}'$, we have $\inner{\grad\vphi(x),z-x}=\frac{r}{\sqrt{d}}\inner{g(x)^{-1/2}\grad\vphi(x),h}$,
and a standard tail bound for $h$ leads to 
\[
\P_{z}\Bpar{\inner{\grad\vphi(x),z-x}\leq-\frac{r}{\sqrt{d}}\,\snorm{g(x)^{-1/2}\nabla\vphi(x)}_{2}\cdot2\log\frac{1}{\veps}}\leq\veps\,.
\]
We call this event $B_{3}$ and bound $\snorm{g(x)^{-1/2}\nabla\vphi(x)}_{2}$
via SSC of $g$ as follows: omitting $x$ for simplicity,
\begin{align*}
\snorm{g^{-\half}\nabla\vphi}_{2} & =\sup_{v:\norm v_{2}=1}\inner{\grad\vphi,g^{-\half}v}\underset{\text{(i)}}{=}\sup_{v}\tr(g^{-1}\Dd g[g^{-\half}v])=\sup_{v}\tr(g^{-\half}\Dd g[g^{-\half}v]\,g^{-\half})\\
 & \underset{\text{(ii)}}{\leq}\sup_{v}\sqrt{d}\,\snorm{g^{-\half}\Dd g[g^{-\half}v]\,g^{-\half}}_{F}\underset{\text{(iii)}}{\leq}\sup_{v}2\sqrt{d}\snorm{g^{-\half}v}_{x}=2\sqrt{d}\,,
\end{align*}
where (i) follows from \eqref{eq:gradLogDet}, (ii) is due to $\tr(A)\leq\sqrt{d}\snorm A_{F}$
for $A\in\Rdd$, and (iii) is due to SSC. Conditioned on $B_{3}^{c}$,
taking $r\leq r_{4}(\veps):=\veps(4\log\frac{1}{\veps})^{-1}$, we
have
\[
\msf{A'}=\inner{\grad\vphi(x),z-x}\geq-4r\,\log\frac{1}{\veps}\geq-\veps\,.
\]

As for $\textsf{B}'$, denoting $u=z-x$ for $z\in B_{0}^{c}$ 
\begin{align}
\Dd^{2}\vphi(x^{*})[u,u] & \underset{\eqref{eq:hessLogDet}}{=}\tr\bpar{g(x^{*})^{-1}\Dd^{2}g(x^{*})[u,u]}-\snorm{g(x^{*})^{-\half}\Dd g(x^{*})[u]\,g(x^{*})^{-\half}}_{F}^{2}\nonumber \\
 & \underset{\text{(i)}}{\geq}-\snorm u_{x^{*}}^{2}-\snorm{g(x^{*})^{-\half}\Dd g(x^{*})[u]\,g(x^{*})^{-\half}}_{F}^{2}\ge-\snorm u_{x^{*}}^{2}-4\snorm u_{x^{*}}^{2}\nonumber \\
 & \underset{\text{(ii)}}{\geq}-5(1-\snorm{x-x^{*}}_{x})^{-2}\snorm u_{x}^{2}\label{eq:so-taylor-logdet}\\
 & \geq-5(1+2cr)^{2}c^{2}r^{2}\,,\nonumber 
\end{align}
where (i) follows from LTSC and (ii) follows from Lemma~\ref{lem:scCloseness}.
Hence, $\msf{B'}\geq-\veps/2$ by taking $r\leq r_{5}(\veps)$ so
that $5(1+2cr_{5})^{2}c^{2}r_{5}^{2}=\veps$. 

In summary, conditioned on $G:=\bigcap_{i=0}^{3}B_{i}^{c}$ with $\P_{z}(G)\geq\half-4\veps$
due to the union bound, we have
\begin{align}
\textsf{A}: & \,\frac{\exp f(x)}{\exp f(z)}\geq\exp(-\veps)\,,\label{eq:fx-ratio}\\
\textsf{B}: & \,\frac{p_{z}(x)}{p_{x}(z)}\geq\exp(-3\veps)\,,\label{eq:prop-ratio}\\
 & \,\vphi(z)-\vphi(x)\geq-2\veps\,.\label{eq:vphi-z-x}
\end{align}
Combining these together,
\begin{align*}
r_{x} & =1-\int\min\Bpar{1,\mathbf{1}_{K}(z)\frac{\exp f(x)}{\exp f(z)}\,\frac{p_{z}(x)}{p_{x}(z)}}p_{x}(\D z)\leq1-\int_{G}(1\wedge e^{-\veps}e^{-3\veps})\,\P_{z}(G)\leq\half+5\veps\,.
\end{align*}
Bounding $r_{y}$ in the same way, we conclude that $\textsf{I}\leq\half+5\veps$
in \eqref{eq:tv-formula}.

\paragraph{Overlapping part (Term $\protect\msf{II}$).}

WLOG, assume $f(y)\geq f(x)$. We denote good events by $G_{x}=\cap_{i=0,2,3}B_{x,i}^{c}$
and $G_{y}=\cap_{i=0,2,3}B_{y,i}^{c}$ such that $\P_{p_{x}}(G_{x}^{c})\leq3\veps$
and $\P_{p_{y}}(G_{y}^{c})\leq3\veps$, where 
\begin{align*}
B_{x,0} & =\{\snorm{z-x}_{x}\geq cr\}\ \text{with }c\geq1+\frac{2}{\sqrt{d}}\,\log\frac{1}{\veps},\ \text{and}\ B_{x,2}=\Bbrace{\snorm{z-x}_{z}^{2}-\snorm{z-x}_{x}^{2}>\frac{2\veps r^{2}}{d}}\\
B_{x,3} & =\Bbrace{\grad\vphi(x)^{\T}(z-x)\leq-\frac{2r\log\frac{1}{\veps}}{\sqrt{d}}\,\snorm{g(x)^{-\half}\nabla\vphi(x)}_{2}}\,.
\end{align*}
Let $G:=G_{x}\cup G_{y}$, and define a partition of $G$ by
\[
G_{x\backslash y}:=G_{x}\backslash G_{y},\qquad G_{x,y}:=G_{x}\cap G_{y},\qquad G_{y\backslash x}:=G_{y}\backslash G_{x}\,.
\]
Now we decompose the term $\textsf{II}$ as follows: for $Q:=|A(x,z)p_{x}(z)-A(y,z)p_{y}(z)|$,
\begin{align*}
\textsf{II} & =\half\int_{K\backslash G}Q\,\D z+\underbrace{\half\int_{G_{x\backslash y}}Q\,\D z}_{\eqqcolon\acal}+\underbrace{\half\int_{G_{y\backslash x}}Q\,\D z}_{\eqqcolon\bcal}+\underbrace{\half\int_{G_{x,y}}Q\,\D z}_{\eqqcolon\ccal}\\
 & \leq\half\bpar{\P_{p_{x}}(K\backslash G)+\P_{p_{y}}(K\backslash G)}+\acal+\bcal+\ccal\leq\half\bpar{\P_{p_{x}}(G_{x}^{c})+\P_{p_{y}}(G_{y}^{c})}+\acal+\bcal+\ccal\\
 & \leq3\veps+\acal+\bcal+\ccal\,.
\end{align*}
The term $\acal$ can be further decomposed by
\begin{align*}
2\mc A & \leq\int_{G_{x\backslash y}}A(x,z)\,|p_{x}(z)-p_{y}(z)|\,\D z+\int_{G_{x\backslash y}}|A(x,z)-A(y,z)|\,p_{y}(\D z)\\
 & \leq\int_{G_{x\backslash y}}|p_{x}(z)-p_{y}(z)|\,\D z+\P_{p_{y}}(G_{x\backslash y})\leq\int_{G_{x\backslash y}}|p_{x}(z)-p_{y}(z)|\,\D z+\underbrace{\P_{p_{y}}(G_{y}^{c})}_{\leq3\veps}\,,
\end{align*}
and in a similar way $\mc B\leq\half\int_{G_{y\backslash x}}|p_{x}(z)-p_{y}(z)|\,\D z+3\veps/2$.
Combining these together,
\begin{align*}
\mc A+\mc B & \leq3\veps+\half\int_{G_{x\backslash y}\cup G_{y\backslash x}}|p_{x}(z)-p_{y}(z)|\,\D z\leq3\veps+\dtv(p_{x},p_{y})\leq4\veps\,,
\end{align*}
where we used $\dtv(p_{x},p_{y})\leq\veps$; to see this, recall Pinsker's
inequality and a formula for the $\KL$ divergences between two Gaussians:
\[
2[\dtv(p_{x},p_{y})]^{2}\leq\KL(p_{y}\mmid p_{x})=\half\,\Bpar{\tr\bpar{g(y)^{-1}g(x)}-d+\log\det\bpar{g(y)g(x)^{-1}}+\frac{d}{r^{2}}\,\snorm{y-x}_{x}^{2}}\,.
\]
Let $\{\lda_{i}\}_{i\in[d]}$ be the eigenvalues of $g(x)^{-\half}g(y)g(x)^{-\half}$
and $\snorm{x-y}_{x}\leq\frac{sr}{\sqrt{d}}$ with $s>0$ to be determined.
Then, $\half\leq\lda_{i}\leq1+8\norm{x-y}_{x}$ by Lemma~\ref{lem:scCloseness}.
Using this and $\log x\leq x-1$ for $x>0$,
\begin{align*}
2\,\KL(p_{y}\mmid p_{x}) & =\sum_{i=1}^{d}\Bpar{\lda_{i}-1+\log\frac{1}{\lda_{i}}}+\frac{d}{r^{2}}\,\snorm{y-x}_{x}^{2}\le\sum_{i=1}^{d}\frac{(\lda_{i}-1)^{2}}{\lda_{i}}+s^{2}\leq s^{2}\,(128r^{2}+1)\,,
\end{align*}
Taking $s\leq s_{1}(\veps):=\veps$ and $r\leq r_{6}(\veps)$ so that
$\sqrt{128r_{6}^{2}+1}\leq2$, we obtain 
\begin{align}
\dtv(p_{x},p_{y})\leq\sqrt{\half\,\KL(p_{y}\mmid p_{x})} & \leq\frac{s}{2}\sqrt{128r^{2}+1}\leq\veps\,,\label{eq:TV-by-KL}
\end{align}

We now bound $\mc C$. Recall $B_{x,1}=\{\inner{\grad f(x),z-x}\ge0\}$
and $\P_{p_{x}}(B_{x,1})\leq\half+\O(\veps)$. Then,
\begin{align*}
2\mc C & =\int_{(G_{x}\cap G_{y})\backslash B_{x,1}^{c}}Q\,\D z+\int_{G_{x}\cap G_{y}\cap B_{x,1}^{c}}Q\,\D z\leq\int_{B_{x,1}}\underbrace{Q}_{\text{The traingle inequality}}\,\D z+\int_{G_{x}\cap G_{y}\cap B_{x,1}^{c}}Q\,\D z\\
 & \leq\int_{B_{x,1}}|A(x,z)-A(y,z)|\,p_{x}(\D z)+\int_{B_{x,1}}A(y,z)\:|p_{x}(z)-p_{y}(z)|\,\D z+\int_{G_{x}\cap G_{y}\cap B_{x,1}^{c}}Q\,\D z\\
 & \le\underbrace{\P_{p_{z}}(B_{x,1})}_{\leq\half+\veps}+2\underbrace{\dtv(p_{x},p_{y})}_{\leq\veps\ (\ref{eq:TV-by-KL})}+\int_{G_{x}\cap G_{y}\cap B_{x,1}^{c}}|A(x,z)\,p_{x}(z)-A(y,z)\,p_{y}(z)|\,\D z\\
 & \leq\half+2\veps+\int_{G_{x}\cap G_{y}\cap B_{x,1}^{c}}|A(x,z)\,p_{x}(z)-A(y,z)\,p_{y}(z)|\,\D z\,.
\end{align*}
One can check that
\[
|A(x,z)\,p_{x}(z)-A(y,z)\,p_{y}(z)|\,\D z=\Big|\min\Bpar{1,\underbrace{\frac{\exp f(x)}{\exp f(z)}\,\frac{p_{z}(x)}{p_{x}(z)}}_{\eqqcolon\msf U}}-\min\Bpar{\underbrace{\frac{p_{y}(z)}{p_{x}(z)}}_{\eqqcolon\msf V},\underbrace{\frac{\exp f(y)}{\exp f(z)}\,\frac{p_{z}(y)}{p_{x}(z)}}_{\eqqcolon\msf W}}\Big|\,p_{x}(\D z)\,.
\]
Here we note that $\msf U\geq e^{-4\veps}$ due to $\frac{\exp f(x)}{\exp f(z)}\geq e^{-\veps}$
and $\frac{p_{z}(x)}{p_{x}(z)}\geq e^{-3\veps}$ from \eqref{eq:fx-ratio}
and \eqref{eq:prop-ratio}. 

We now show that under additional conditioning, $|\log\textsf{V}|\lesssim\veps$
and $\log\textsf{W}\gtrsim-\veps$ on $z\in G_{x}\cap G_{y}\cap B_{x,1}^{c}$.
For $\vphi(\cdot)=\half\log\det g(\cdot)$ and $\msf L:=-\frac{d}{2r^{2}}(\snorm{z-y}_{y}^{2}-\snorm{z-x}_{x}^{2})$,
\begin{align}
\log\msf V & =-\frac{d}{2r^{2}}(\snorm{z-y}_{y}^{2}-\snorm{z-x}_{x}^{2})+\vphi(y)-\vphi(x)\nonumber \\
 & =\textsf{L}+\inner{\grad\vphi(x),y-x}+\underbrace{\half\inner{\hess\vphi(x^{*})(y-x),y-x}}_{\text{Use }\eqref{eq:so-taylor-logdet}}\quad\text{for some }x^{*}\in[x,y]\label{eq:bound-vphi}\\
 & \geq\textsf{L}-\snorm{g(x)^{-1/2}\nabla\vphi(x)}_{2}\snorm{y-x}_{x}-5\underbrace{(1+2\snorm{x-y}_{x})^{2}}_{\leq2}\snorm{y-x}_{x}^{2}\nonumber \\
 & \geq\textsf{L}-2\sqrt{d}\cdot s\frac{r}{\sqrt{d}}-10s^{2}\frac{r^{2}}{d}\geq\textsf{L}-\veps\,,\label{eq:logV-lower}
\end{align}
where the inequality follows from $s\leq\frac{\veps}{10}$ and $r\leq r_{7}(\veps):=1$. 

As for $\textsf{W}$, due to $f(y)\geq f(x)$ and $\exp(f(x)-f(z))\geq\exp(-\veps)$,
\begin{align}
\log\msf W & \geq\log\Bpar{\frac{\exp f(x)}{\exp f(z)}\frac{p_{z}(y)}{p_{x}(z)}}\geq-\veps-\frac{d}{2r^{2}}(\snorm{z-y}_{z}^{2}-\snorm{z-x}_{x}^{2})+\vphi(z)-\vphi(x)\nonumber \\
 & \underset{\text{(i)}}{\geq}-\veps-\frac{d}{2r^{2}}\Bpar{\snorm{z-y}_{y}^{2}+2\veps\frac{r^{2}}{d}-\snorm{z-x}_{x}^{2}}-2\veps=\textsf{L}-4\veps\,,\label{eq:logW-lower}
\end{align}
where (i) follows from $\snorm{z-y}_{z}^{2}-\snorm{z-y}_{y}^{2}\leq2\veps r^{2}/d$
on $z\in B_{y,2}^{c}$, and $\vphi(z)-\vphi(x)\geq-2\veps$ on $z\in B_{x,3}^{c}$
from \eqref{eq:vphi-z-x}.

Lastly, we show that $|\textsf{L}|$ is bounded by $\mc O(\veps)$
with high probability (w.r.t. $p_{x}$). Due to affine invariance
of the algorithm, we may assume that $x=0$ and $g(x)=I_{d}$ (so
$p_{x}=\ncal(0,I_{d})$). Therefore, 
\begin{align*}
\snorm{z-y}_{y}^{2}-\snorm{z-x}_{x}^{2} & =\snorm{z-y}_{y}^{2}-\snorm z^{2}=\snorm z_{g(y)-I_{d}}^{2}-2\inner{z,y}_{y}+\snorm y_{y}^{2}\,.
\end{align*}
The last term is bounded by $2\norm y^{2}$ due to SC of $g$. Using
a tail bound for Gaussians, we have $\P_{p_{x}}\bpar{|\inner{z,y}_{y}|\geq\frac{r}{\sqrt{d}}\,\snorm{g(y)y}_{2}\cdot2\log\frac{1}{\veps}}\leq\veps$
and call this event $C_{1}$. In addition, SC of $g$ leads to $g(y)\preceq2I_{d}$,
so $\snorm{g(y)y}\leq2\snorm y$.

To bound $\snorm z_{g(y)-I_{d}}^{2}$, we note that $\snorm y=\snorm{y-x}_{x}\leq1/\sqrt{2}$
and so
\begin{align*}
\snorm{g(y)-I_{d}}_{F} & \leq(1+2\snorm y)^{2}\snorm y\leq2s\frac{r}{\sqrt{d}}\,,\quad\text{(Lemma \ref{lem:strongSC-closeness})}\\
\E[\snorm z_{g(y)-I_{d}}^{2}] & =\frac{r^{2}}{d}\tr(g(y)-I_{d})\leq\frac{r^{2}}{d}\sqrt{d}\,\snorm{g(y)-I_{d}}_{F}\leq\frac{r^{2}}{d}\cdot2rs\,.
\end{align*}
By the Hanson-Wright inequality\footnote{\begin{lem*}
[Hanson-Wright; Adapted to Gaussian] Let $h\sim\ncal(0,\sigma^{2}I_{d})$
and $M\in\Rdd$. Then there exists universal constants $c,K>0$ such
that for $t\geq0$
\[
\P\bpar{|\norm h_{A}^{2}-\E[\norm h_{A}^{2}]|>t}\leq2\exp\Bpar{-c\min\Bpar{\frac{t^{2}}{K^{4}\sigma^{4}\norm M_{F}^{2}},\frac{t}{K^{2}\sigma^{2}\norm M_{2}}}}\,.
\]
\end{lem*}
}, for universal constants $K_{1},K_{2}>0$ and $t\geq0$ it holds
that
\[
\P_{z\sim\ncal(0,I_{d})}\bpar{|\norm z_{g(y)-I_{d}}^{2}-\E[\norm z_{g(y)-I_{d}}^{2}]|\geq t}\leq2\exp\Bpar{-K_{1}\Bpar{\frac{t^{2}}{K_{2}^{4}\frac{r^{4}}{d^{2}}\snorm{g(y)-I_{d}}_{F}^{2}}\wedge\frac{t}{K_{2}^{2}\frac{r^{2}}{d}\snorm{g(y)}_{2}}}}\,.
\]
By taking $r\leq r_{8}(\veps):=\frac{\sqrt{K_{1}}}{2K_{2}^{2}}$ and
$s\leq s_{2}(\veps):=\veps(1+\sqrt{\log\frac{2}{\veps}})^{-1}$, it
follows that $\norm z_{g(y)-I_{d}}^{2}\leq\frac{2\veps r^{2}}{d}$
with probability at least $1-\veps$. Denote the complement of this
event by $C_{2}$.

Conditioned on $z\in C_{1}^{c}\cap C_{2}^{c}$, we conclude that
\begin{align*}
|\snorm{z-y}_{y}^{2}-\snorm{z-x}_{x}^{2}| & \leq\snorm z_{g(y)-I_{d}}^{2}+2|\inner{z,y}_{y}|+2\snorm y^{2}\leq\frac{2r^{2}\veps}{d}+\frac{8r\norm y}{\sqrt{d}}\,\log\frac{1}{\veps}+2\norm y^{2}\leq\frac{2r^{2}}{d}\cdot3\veps\,,
\end{align*}
where the last inequality follows from $\norm y\leq\frac{sr}{\sqrt{d}}$
when $s\leq s_{3}(\veps):=\veps\,(4\log\frac{1}{\veps})^{-1}$. Hence,
$|\textsf{L}|\leq3\veps$ on $C_{1}^{c}\cap C_{2}^{c}$. Putting this
into \eqref{eq:logV-lower} and \eqref{eq:logW-lower},
\[
\log\msf V\geq\exp(-4\veps)\qquad\text{and}\qquad\log\msf W\geq\exp(-7\veps)\,.
\]

We can also show $\log\textsf{V}\leq5\veps$. Conditioned on $z\in C_{1}^{c}\cap C_{2}^{c}$,
\[
-\log\msf V=-\textsf{L}+\vphi(x)-\vphi(y)\geq-3\veps+\vphi(x)-\vphi(y)\geq-5\veps\,,
\]
since $\vphi(x)-\vphi(y)$ can be lowered bounded by $-2\veps$ as
in \eqref{eq:bound-vphi}. Hence, $\log\textsf{V}\leq5\veps$.

For $F:=G_{x}\cap G_{y}\cap B_{x,1}^{c}$ and $C:=(C_{1}\cup C_{2})^{c}$,
since $e^{-4\veps}\leq\msf V\leq e^{5\veps}$, $e^{-7\veps}\leq\msf W$,
and $\msf U\geq e^{-4\veps}$,
\begin{align*}
 & \int_{F}|A(x,z)\,p_{x}(z)-A(y,z)\,p_{y}(z)|\,\D z\leq\int_{C^{c}}(\cdot)\,\D z+\int_{F\cap C}(\cdot)\,\D z\\
\leq & \underbrace{\P_{p_{x}}(C^{c})}_{\leq2\veps}+2\underbrace{\dtv(p_{x},p_{y})}_{\leq\veps}+\int_{F\cap C}(\cdot)\,\D z\leq4\veps+\int_{F\cap C}|1\wedge\msf U-\msf V\wedge\msf W|\,p_{x}(\D z)\leq4\veps+(e^{5\veps}-e^{-4\veps})\\
\leq & 18\veps\,.
\end{align*}
Using this, we can bound $\ccal$ by
\begin{align*}
\ccal & \leq\frac{1}{4}+\veps+\half\int_{F}|A(x,z)\,p_{x}(z)-A(y,z)\,p_{y}(z)|\,\D z\leq\frac{1}{4}+10\veps\,.
\end{align*}
Therefore, $\textsf{II}\leq3\veps+\acal+\bcal+\ccal\leq3\veps+4\veps+\frac{1}{4}+10\veps\leq\frac{1}{4}+17\veps.$
Along with $\textsf{I}\leq\half+5\veps$, we can conclude that if
$r\leq\min_{i}r_{i}(\veps)$ and $s\leq\min_{i}s_{i}(\veps)$, then
$\dtv(P_{x},P_{y})\leq\frac{3}{4}+23\veps$.
\end{proof}

\subsubsection{Isoperimetric inequality \label{proof:isoperimetry}}

We now prove an isoperimetric inequality arising from the a SC barrier.
Recall the \emph{cross-ratio} \emph{distance} $d_{K}$ defined on
a convex body $K$: for $x,y\in\intk$, suppose that the chord passing
through $x,y$ has endpoints $p$ and $q$ in the boundary $\de K$
(so the order of points is $p,x,y,q$), then the cross-ratio distance
between $x$ and $y$ is defined by
\[
d_{K}(x,y)\defeq\frac{\snorm{x-y}_{2}\snorm{p-q}_{2}}{\snorm{p-x}_{2}\snorm{y-q}_{2}}\,.
\]
The first type of isoperimetric inequalities says $\psi_{\pi}\gtrsim1/\sqrt{\onu}$.
\begin{proof}
[Proof of Lemma~\ref{lem:symmetry-iso}] For a ball $B_{r}(0)$
of radius $r>0$ centered at the origin, we define a convex body $K_{r}:=K\cap B_{r}(0)$
and use $\pi_{r}$ to denote the truncated distribution of $\pi$
over $K_{r}$. Let $\{S_{1},S_{2},S_{3}\}$ be a partition of $K$
and define $S_{i}^{r}:=S_{i}\cap K_{r}$ for $i\in[3]$. By \citet[Theorem 2.5]{lovasz2007geometry},
we have 
\[
\pi_{r}(S_{3}^{r})\geq d_{K_{r}}(S_{1}^{r},S_{2}^{r})\,\pi_{r}(S_{1}^{r})\,\pi_{r}(S_{2}^{r})\,,
\]
where $d_{K_{r}}(S_{1}^{r},S_{2}^{r})=\inf_{x\in S_{1}^{r},y\in S_{2}^{r}}d_{K_{r}}(x,y)$.
Due to $d_{K_{r}}(x,y)\geq\norm{x-y}_{x}/\sqrt{\onu}$ for any $x,y\in K_{r}$
(see \citet[Lemma 2.3]{laddha2020strong}), 
\[
\pi_{r}(S_{3}^{r})\geq\inf_{x\in S_{1}^{r},\,y\in S_{2}^{r}}\frac{\snorm{x-y}_{x}}{\sqrt{\onu}}\,\pi_{r}(S_{1}^{r})\,\pi_{r}(S_{2}^{r})\geq\frac{1}{\sqrt{\onu}}\inf_{x\in S_{1},\,y\in S_{2}}\snorm{x-y}_{x}\,\pi_{r}(S_{1}^{r})\,\pi_{r}(S_{2}^{r})\,.
\]
As $r\to\infty$, the bounded convergence theorem implies $\pi_{r}(S_{i}^{r})\to\pi(S_{i})$
for $i\in[3]$, completing the proof.
\end{proof}
%
We provide the deferred proof for another isoperimetric inequality,
$\psi_{\pi}\gtrsim\sqrt{\alpha}$, originating from $\alpha$-relatively
strong-convexity of the potential with respect to $\hess\phi$.
\begin{proof}
[Proof of Lemma~\ref{lem:sc-iso}] The proof essentially follows
\citet{gopi2023algorithmic}. Their first proof ingredient is a modified
localization lemma \citet[Lemma 8]{gopi2023algorithmic}; let $f_{1},f_{2},f_{3},f_{4}$
be non-negative functions on $\Rd$ such that $f_{1}$ and $f_{2}$
are upper semicontinuous, and $f_{3}$ and $f_{4}$ are lower semicontinuous,
and $\phi:\Rd\to\R$ be convex. Then the following are equivalent:
\begin{itemize}
\item For any density $\pi:\Rd\to\R$ which is $1$-relatively strongly
logconcave in $\phi$,
\[
\int f_{1}\,\D\pi\cdot\int f_{2}\,\D\pi\leq\int f_{3}\,\D\pi\cdot\int f_{4}\,\D\pi\,.
\]
\item Let $\int_{E}h:=\int_{0}^{1}h((1-t)\,a+tb)e^{-\gamma t}\,\D t$. Then
$\int_{E}f_{1}e^{-\phi}\cdot\int_{E}f_{2}e^{-\phi}\leq\int_{E}f_{3}e^{-\phi}\cdot\int_{E}f_{4}e^{-\phi}$
for any $a,b\in\Rd$ and $\gamma\in\R$.
\end{itemize}
First of all, this can be generalized to an extended convex function
$f$ and $\phi$, whose values outside of $\intk$ are set to $\infty$.
Since the density $\pi$ and a needle $\exp\Par{\gamma t-\phi((1-t)a+tb)}$
for $\gamma\in\R$ and $a,b\in\Rd$ (induced by the extended $f$
and $\phi$) vanish outside of $\intk$, integrands above become zero
on $\intk^{c}$, and thus the integrals above remain the same.

As in \citet[Lemma 9]{gopi2023algorithmic}, the proof boils down
to the case of $\alpha=1$, and it suffices to show that there exists
a constant $C>0$ such that
\[
C\cdot d_{\phi}(S_{1},S_{2})\int_{S_{1}}e^{-f}\cdot\int_{S_{2}}e^{-f}\leq\int e^{-f}\int_{S_{3}}e^{-f}\,.
\]
We can replace $S_{i}\gets$ its closure $\bar{S_{i}}$ for $i\in[2]$,
which only increases the LHS. Also, we can replace $S_{3}\gets$ an
open set $\intk\backslash\bar{S_{1}}\backslash\bar{S_{2}}$, which
does not change the RHS since the boundary of a convex set is a null
set \citet[Theorem 1]{lang1986note}. By taking $f_{i}=\mathbf{1}_{S_{i}}$
for $i\in[3]$ and $f_{4}=(C\,d_{\phi}(S_{1},S_{2}))^{-1}$, we only
need to show that for some $0\leq c<d\leq1$,
\begin{align*}
 & C\cdot d_{\phi}(S_{1},S_{2})\int_{c}^{d}e^{\gamma t-\phi((1-t)\,a+tb)}\mathbf{1}_{S_{1}}((1-t)\,a+tb)\,\D t\cdot\int_{c}^{d}e^{\gamma t-\phi((1-t)\,a+tb)}\mathbf{1}_{S_{2}}((1-t)\,a+tb)\,\D t\\
\leq & \int_{c}^{d}e^{\gamma t-\phi((1-t)\,a+tb)}\,\D t\cdot\int_{c}^{d}e^{\gamma t-\phi((1-t)\,a+tb)}\mathbf{1}_{S_{3}}((1-t)\,a+tb)\,\D t\,,
\end{align*}
where $\phi((1-t)\,a+b)<\infty$ for $t\in(c,d)$. The rest of the
proof is similar to \citet[Lemma 9]{gopi2023algorithmic}.
\end{proof}

\subsection{Sampling IPM ($\S$\ref{sec:IPM-framework})}

\subsubsection{Well-definedness of sampling IPM \label{proof:IPM-welldefined}}
\begin{prop}
\label{prop:density-bounded} Let $p:\Rd\to\R$ be a log-concave density
with finite second moment. Then $p$ is bounded on $\Rd$.
\end{prop}

\begin{proof}
Let $X\sim p$ and denote the mean and covariance of the distribution
$p$ by $\mu:=\E[X]$ and $\Sigma:=\E[(X-\mu)(X-\mu)^{\T}]$. Then
the pushforward $T_{\#}p$ of $p$ via the map $T:x\mapsto y:=\Sigma^{-1/2}(x-\mu)$
is an isotropic log-concave, and satisfy $(T_{\#}p)(y)=\frac{p(x)}{|\det T|}$.
Since $T_{\#}p$ is bounded on $\Rd$ \citet[Theorem 5.14 (e)]{lovasz2007geometry},
$p$ is bounded as well.
\end{proof}
Next, we show that every measure appearing within the sampling IPM
is integrable.
\begin{proof}
[Proof of Proposition~\ref{prop:annealing-welldefined}] Recall
that we may assume $\phi\geq0$. Hence, all $\mu_{i}$'s in Phase
3 and 4 are well-defined
\[
\int_{K}\exp\Bpar{-\bpar{f(x)+\frac{\phi(x)}{\sigma_{i}^{2}}}}\,\D x\leq\int_{K}\exp(-f(x))\,\D x<\infty\,.
\]
In particular, $\exp\bpar{-(f+\frac{\phi}{\nu/d})}$ is integrable
with finite second moment. By Proposition~\ref{prop:density-bounded},
$f(x)+\frac{\phi(x)}{\nu/d}$ achieves a global minimum $m$ in $K$.
As $\sigma_{i}^{2}\leq\sigma_{i_{0}}^{2}=\nu/d$ in Phase 2, we have
\begin{align*}
 & \int_{K}\exp\Bpar{-\frac{\sigma_{i_{0}}^{2}f+\phi}{\sigma_{i_{0}}^{2}}}=\int_{K}\exp\Bpar{-\frac{\sigma_{i_{0}}^{2}f+\phi-\min(\sigma_{i_{0}}^{2}f+\phi)}{\sigma_{i_{0}}^{2}}-\frac{\min(\sigma_{i_{0}}^{2}f+\phi)}{\sigma_{i_{0}}^{2}}}\\
 & \geq\int_{K}\exp\Bpar{-\frac{\bar{f}+\phi-\sigma_{i_{0}}^{2}m}{\sigma_{i}^{2}}-m}=\exp\Bpar{m\bpar{\frac{\sigma_{i_{0}}^{2}}{\sigma_{i}^{2}}-1}}\int_{K}\exp\Bpar{-\frac{\bar{f}+\phi}{\sigma_{i}^{2}}}\,,
\end{align*}
where the inequality holds due to $\min(\sigma_{i_{0}}^{2}f+\phi)=\sigma_{i_{0}}^{2}m$
and $\bar{f}=\sigma_{i_{0}}^{2}f$. Therefore, $\mu_{i}$'s in Phase
2 are also well-defined.
\end{proof}

\subsubsection{Closeness of distributions in sampling IPM \label{proof:IPM-closeness}}

We begin with closeness between $\ncal\bpar{x^{*},\frac{\sigma_{0}^{2}}{1+\nu\beta d^{-1}}g(x^{*})^{-1}}\cdot\mathbf{1}_{\dcal_{g}^{3\sigma_{0}\sqrt{d}}(x^{*})}$
and $\exp\bpar{-\frac{\bar{f}+\phi}{\sigma_{0}^{2}}}$ in Phase 1.
\begin{proof}
[Proof of Lemma~\ref{lem:phase1}] Let $\gamma=9$, $r=(\gamma\sigma_{0}^{2}d)^{1/2}<0.01$,
$\psi:=\bar{f}+\phi$, and $S=\{x\in K:\psi(x)\leq\psi(x^{*})+r^{2}/4\}$.
For $\widetilde{\mu}_{0}=\exp(-\psi/\sigma_{0}^{2})\cdot\mathbf{1}_{K}\propto\mu_{0}$
and $x\in S$, we have $\mu_{0}(x)\geq e^{-\gamma d}\mu_{0}(x^{*}).$
Due to $\mu_{0}(S^{c})\leq\exp(-\gamma d/3)$ (Lemma~\ref{lem:mostMass-logconcave}),
it follows that $1=\mu_{0}(S)+\mu_{0}(S^{c})\leq\mu_{0}(S)+\exp(-\gamma d/3)$
and 
\begin{equation}
1\leq\bpar{1+2\exp(-\gamma d/3)}\,\mu_{0}(S)=\bpar{1+2\exp(-\gamma d/3)}\,\widetilde{\mu}_{0}(S)/\widetilde{\mu}_{0}(\Rd)\,.\label{eq:intp-intSp}
\end{equation}

We show $S\subset D=\dcal_{g}^{3\sigma_{0}\sqrt{d}}(x^{*})$. For
$x\in S$, use Taylor's expansion of $\psi$ at $x^{*}$: for some
$\bar{x}\in[x^{*},x]$
\begin{align}
\psi(x)-\psi(x^{*}) & =\half(x-x^{*})^{\T}\hess\psi(\bar{x})(x-x^{*})\geq\half(x-x^{*})^{\T}\hess\phi(\bar{x})(x-x^{*})\,.\label{eq:psi-taylor}
\end{align}
As $\psi(x)-\psi(x^{*})\leq r^{2}/4$ on $x\in S$, we have $\snorm{\bar{x}-x^{*}}_{\bar{x}}^{2}\leq\snorm{x-x^{*}}_{\bar{x}}^{2}\leq2(\psi(x)-\psi(x^{*}))\leq r^{2}/2$.
Thus, by self-concordance of $\phi$ 
\begin{equation}
\exp(-3r)\,\snorm{x-x^{*}}_{x^{*}}^{2}\leq\snorm{x-x^{*}}_{\bar{x}}^{2}\leq\exp(3r)\,\snorm{x-x^{*}}_{x^{*}}^{2}\,,\label{eq:closenss-initial}
\end{equation}
and it follows that $\snorm{x-x^{*}}_{x^{*}}^{2}\leq r^{2}$, showing
$S\subset D$.

Combining \eqref{eq:psi-taylor}, \eqref{eq:closenss-initial}, and
$(1+\nu\alpha d^{-1})\,\hess\phi\preceq\hess\psi\preceq(1+\nu\beta d^{-1})\,\hess\phi$,
we have 
\begin{equation}
\frac{\exp(-3r)}{2}\Bpar{1+\frac{\nu\alpha}{d}}\,\snorm{x-x^{*}}_{x^{*}}^{2}\underset{(*)}{\leq}\psi(x)-\psi(x^{*})\underset{(\#)}{\leq}\frac{\exp(3r)}{2}\Bpar{1+\frac{\nu\beta}{d}}\,\snorm{x-x^{*}}_{x^{*}}^{2}\,,\label{eq:approx-psigap}
\end{equation}
and thus for a constant $c:=1+\nu\beta d^{-1}$ and function $h(x):=-(2\sigma_{0}^{2})^{-1}\snorm{x-x^{*}}_{x^{*}}^{2}$,
\begin{align*}
 & \norm{\mu/\mu_{0}}=\E_{\mu}\bbrack{\deriv{\mu}{\mu_{0}}}=\frac{\int_{D}\exp\Bpar{-\frac{c}{\sigma_{0}^{2}}\snorm{x-x^{*}}_{x^{*}}^{2}+\frac{\psi}{\sigma_{0}^{2}}}\cdot\widetilde{\mu}_{0}(\Rd)}{\Bbrack{\int_{D}\exp\Bpar{-\frac{c}{2\sigma_{0}^{2}}\,\snorm{x-x^{*}}_{x^{*}}^{2}}}^{2}}\\
 & \underset{\text{(\ref{eq:intp-intSp})}}{\leq}\frac{1}{\Bbrack{\int_{D}\exp(c\cdot h)}^{2}}\int_{D}\exp\Bpar{-\frac{c}{\sigma_{0}^{2}}\snorm{x-x^{*}}_{x^{*}}^{2}+\underbrace{\frac{\psi}{\sigma_{0}^{2}}}_{\text{Use }(\#)\text{ in (\ref{eq:approx-psigap})}}}\bpar{1+2\exp(-\gamma n/3)}\underbrace{\widetilde{\mu}_{0}(S)}_{\text{Use }(*)}\\
 & \lesssim\frac{\int_{D}\exp\Bpar{-\frac{1}{2\sigma_{0}^{2}}\bpar{2c-e^{3r}(1+\nu\beta d^{-1})}\,\snorm{x-x^{*}}_{x^{*}}^{2}}\int_{D}\exp\bpar{-\frac{1}{2\sigma_{0}^{2}}e^{-3r}(1+\nu\alpha d^{-1})\,\snorm{x-x^{*}}_{x^{*}}^{2}}}{\Bbrack{\int_{D}\exp(c\cdot h)}^{2}}\\
 & =\underbrace{\frac{\int_{D}\exp\Bpar{\bpar{2c-c\,e^{3r}}\,h(x)}\cdot\int_{D}\exp\bpar{c\,e^{3r}h(x)}}{\Bbrack{\int_{D}\exp(c\cdot h)}^{2}}}_{=:\text{\textsf{A}}}\,\underbrace{\frac{\int_{D}\exp\bpar{e^{-3r}(1+\nu\alpha d^{-1})\,h(x)}}{\int_{D}\exp\bpar{c\,e^{3r}h(x)}}}_{=:\text{\textsf{B}}}\,.
\end{align*}
As for $\textsf{A}$, Lemma~\ref{lem:adam-logconcave} leads to 
\begin{align*}
\textsf{A} & \leq\Bpar{\frac{c^{2}}{(2c-c\,e^{3r})\,ce^{3r}}}^{d}=\Bpar{\frac{1}{(2-e^{3r})e^{3r}}}^{d}=(1+\mc O(r^{2}))^{d}=\mc O(1)\,.
\end{align*}
As for $\textsf{B}$, let $c_{1}=e^{-3r}\,(1+\nu\alpha d^{-1})$ and
$c_{2}=e^{3r}\,(1+\nu\beta d^{-1})$. With the change of variable
$y=\sigma_{0}^{-1}\sqrt{c_{i}}g(x^{*})^{1/2}(x-x^{*})$ for $i\in[2]$,
it follows that for $r_{i}:=r\sigma_{0}^{-1}\sqrt{c_{i}}(\geq3\sqrt{d})$
\begin{align*}
\textsf{B} & =\Bpar{\frac{c_{2}}{c_{1}}}^{d/2}\frac{\int_{B_{r_{1}}}\exp\bpar{-\half\snorm y^{2}}\,\D y}{\int_{B_{r_{2}}}\exp\bpar{-\half\snorm y^{2}}\,\D y}\leq\Bpar{\frac{c_{2}}{c_{1}}}^{d/2}\lesssim\Bpar{\frac{\nu\beta+d}{\nu\alpha+d}}^{d}\,e^{3rd}\lesssim\Bpar{\frac{\nu\beta+d}{\nu\alpha+d}}^{d}\,.\qedhere
\end{align*}
\end{proof}
Now we show closeness of two consecutive distributions in Phase 2,
i.e., $\sigma_{i+1}^{2}=\sigma_{i}^{2}\bpar{1+\frac{1}{\sqrt{d}}}$.
\begin{proof}
[Proof of Lemma~\ref{lem:phase2}] Observe that for $\psi=\bar{f}+\phi=\frac{\nu}{d}f+\phi$
on $K$ and $F(\sigma^{2})=\int_{K}\exp(-\psi/\sigma^{2})$, 
\begin{align*}
\snorm{\mu_{i}/\mu_{i+1}} & =\E_{\mu_{i}}\bbrack{\deriv{\mu_{i}}{\mu_{i+1}}}=\frac{\int_{K}\exp\bpar{-2\frac{\psi}{\sigma_{i}^{2}}+\frac{\psi}{\sigma_{i+1}^{2}}}\cdot\int_{K}\exp\bpar{-\frac{\psi}{\sigma_{i+1}^{2}}}}{\Par{\int_{K}\exp\bpar{-\frac{\psi}{\sigma_{i}^{2}}}}^{2}}=\frac{F\bpar{\bpar{\frac{2}{\sigma_{i}^{2}}-\frac{1}{\sigma_{i+1}^{2}}}^{-1}}\,F(\sigma_{i+1}^{2})}{F(\sigma_{i}^{2})^{2}}\,.
\end{align*}
By Lemma~\ref{lem:adam-logconcave}, the function $a^{d}F\bpar{\frac{\sigma^{2}}{a}}$
is log-concave in $a$. Using the definition with endpoints $\frac{2}{\sigma_{i}^{2}}-\frac{1}{\sigma_{i+1}^{2}}$
and $\frac{1}{\sigma_{i+1}^{2}}$, and the middle point $\frac{1}{\sigma_{i}^{2}}$,
we obtain
\[
\frac{F\bpar{\bpar{\frac{2}{\sigma_{i}^{2}}-\frac{1}{\sigma_{i+1}^{2}}}^{-1}}\,F(\sigma_{i+1}^{2})}{F(\sigma_{i}^{2})^{2}}\le\Biggl(\frac{\bpar{\frac{1}{\sigma_{i}^{2}}}^{2}}{\bpar{\frac{2}{\sigma_{i}^{2}}-\frac{1}{\sigma_{i+1}^{2}}}\,\frac{1}{\sigma_{i+1}^{2}}}\Biggr)^{d}=\Biggl(\frac{\bpar{1+\frac{1}{\sqrt{d}}}^{2}}{1+\frac{2}{\sqrt{d}}}\Biggr)^{d}\leq\Bpar{1+\frac{1}{d}}^{d}\leq e\,.\qedhere
\]
\end{proof}
We now establish closeness in Phase 3, during which we use the update
of $\sigma_{i+1}^{2}=\sigma_{i}^{2}\bpar{1+\frac{\sigma_{i}}{\sqrt{\nu}}}$.
\begin{proof}
[Proof of Lemma~\ref{lem:phase34}] The update is $\sigma_{i+1}^{2}=\sigma_{i}^{2}\Par{1+r}$
for $r=\frac{\sigma_{i}}{\sqrt{\nu}}$. For $s:=\frac{r}{1+r}$, $\sigma:=\sigma_{i}$,
and $F(\sigma^{2})=\int\exp(-f-\phi/\sigma^{2})$, we have
\begin{align*}
\snorm{\mu_{i}/\mu_{i+1}} & =\frac{F\bpar{\bpar{\frac{2}{\sigma_{i}^{2}}-\frac{1}{\sigma_{i+1}^{2}}}^{-1}}\,F(\sigma_{i+1}^{2})}{F(\sigma_{i}^{2})^{2}}=\frac{F\bpar{\frac{\sigma^{2}}{1+s}}\,F\bpar{\frac{\sigma^{2}}{1-s}}}{F(\sigma^{2})^{2}}\,.
\end{align*}
Let $g(t):=\log F\bpar{\frac{\sigma^{2}}{t}}$ for $t>0$. Then,
\begin{align}
\log\snorm{\mu_{i}/\mu_{i+1}} & =g(1+s)+g(1-s)-2g(1)=\int_{0}^{s}\bpar{g'(1+t)-g'(1-t)}\,\D t=\int_{0}^{s}\int_{1-t}^{1+t}g''(q)\,\D q\,\D t\label{eq:L2-bound-phase3}
\end{align}
and for a probability measure $\nu_{q}\propto\exp\bpar{-f-\frac{q\phi}{\sigma^{2}}}$,
\begin{align*}
g''(q) & =\frac{\D^{2}}{\D q^{2}}\Bbrack{\log\int_{K}\exp\Bpar{-f-\frac{q\phi}{\sigma^{2}}}}=-\frac{1}{\sigma^{2}}\,\frac{\D}{\D q}\Bigg[\frac{\int_{K}\phi\cdot\exp\Bpar{-f-\frac{q\phi}{\sigma^{2}}}}{\int_{K}\exp\Bpar{-f-\frac{q\phi}{\sigma^{2}}}}\Biggr]\\
 & =-\frac{1}{\sigma^{2}}\,\Bigg(-\frac{1}{\sigma^{2}}\,\frac{\int_{K}\phi^{2}\cdot\exp\Bpar{-f-\frac{q\phi}{\sigma^{2}}}}{\int_{K}\exp\Bpar{-f-\frac{q\phi}{\sigma^{2}}}}+\frac{1}{\sigma^{2}}\,\frac{\Bbrack{\int_{K}\phi\cdot\exp\Bpar{-f-\frac{q\phi}{\sigma^{2}}}}^{2}}{\Bbrack{\int_{K}\exp\Bpar{-f-\frac{q\phi}{\sigma^{2}}}}^{2}}\Biggr)\\
 & =\frac{1}{\sigma^{4}}\,\Bpar{\E_{\nu_{q}}[\phi^{2}]-(\E_{\nu_{q}}\phi)^{2}}=\frac{1}{\sigma^{4}}\,\var_{\nu_{q}}\phi\,.
\end{align*}
By the Brascamp-Lieb inequality with $V(\cdot):=f(\cdot)+\frac{q\phi(\cdot)}{\sigma^{2}}$,
\begin{align*}
\var_{\nu_{q}}\phi & \leq\E_{\nu_{q}}\bbrack{(\grad\phi)^{\T}\bpar{\hess V}^{-1}\grad\phi}\leq\frac{\sigma^{2}}{q}\,\E_{\nu_{q}}\snorm{\nabla\phi}_{(\hess\phi)^{-1}}^{2}\leq\frac{\sigma^{2}\nu}{q}\,,
\end{align*}
and thus $g''(q)\leq\frac{\nu}{q\sigma^{2}}.$ Putting this back to
\eqref{eq:L2-bound-phase3}, we acquire
\begin{align}
\log\norm{\mu_{i}/\mu_{i+1}} & \leq\frac{\nu}{\sigma^{2}}\int_{0}^{s}\int_{1-t}^{1+t}\frac{1}{q}\,\D q\,\D t=\frac{\nu}{\sigma^{2}}\int_{0}^{s}\bpar{\log(1+t)-\log(1-t)}\,\D t\nonumber \\
 & =\frac{\nu}{\sigma^{2}}\bpar{(1+s)\,\log(1+s)+(1-s)\,\log(1-s)}\lesssim\frac{\nu s^{2}}{\sigma^{2}}\,.\label{eq:bound-ph3}
\end{align}
It follows from $s=\frac{r}{1+r}$ and $r=\frac{\sigma}{\sqrt{\nu}}$
that $\mu_{i}$ is an $\mc O(1)$-warm start for $\mu_{i+1}$.

For Phase 4, observe that for $\mu\propto\exp(-f-\phi/\sigma^{2})$
with $\sigma^{2}=\nu$,
\begin{align*}
\snorm{\mu/\pi} & =\frac{\int_{K}\exp\bpar{-f-\frac{\phi}{\sigma^{2}/2}}\cdot\int_{K}\exp(-f)}{\Bbrack{\int_{K}\exp\Bpar{-f-\frac{\phi}{\sigma^{2}}}}^{2}}\underset{\text{(i)}}{=}\lim_{r\to1}\frac{F\bpar{\frac{\sigma^{2}}{1+r}}\cdot F\bpar{\frac{\sigma^{2}}{1-r}}}{F(\sigma^{2})}\\
 & \underset{\text{(ii)}}{\leq}\lim_{r\to1}\exp\Bpar{\mc O(1)\frac{\nu}{\sigma^{2}}\,\bpar{(1+r)\,\log(1+r)+(1-r)\,\log(1-r)}}=\exp\Bpar{\mc O(1)\frac{\nu}{\sigma^{2}}}=\exp(\mc O(1))\,.
\end{align*}
where (i) holds due to the monotone convergence theorem, and (ii)
follows from \eqref{eq:bound-ph3}. Therefore, $\mu$ serves as an
$\mc O(1)$-warm start for $\pi$.
\end{proof}
\begin{rem}
[Coupling argument] \label{rem:divine-intervention} The total number
of measures involved in Algorithm~\ref{alg:IPM-sampling} is $m:=\mc O(\sqrt{d})$.
Let $(X_{1},\dots,X_{m})$ be a sequence of samples provided by Algorithm~\ref{alg:IPM-sampling},
and $(\bar{X}_{1},\dots,\bar{X}_{m})$ be a sequence of samples where
each sample is drawn from the \emph{actual} target distributions $\{\mu_{\sigma^{2}}\}$.
Conditioned on events $X_{i}=\bar{X}_{i}$, Algorithm~\ref{alg:IPM-sampling}
ensures that there is a coupling such that $\P(X_{i+1}=\bar{X}_{i+1}\mid X_{i}=\bar{X}_{i})\geq1-\frac{\veps}{\sqrt{d}}$
due to $\veps/\sqrt{d}$ TV-distance guarantee. Combining these couplings,
\[
\P\Par{X_{i}=\bar{X_{i}}\ \forall i\in[m]}=\P(X_{1}=\bar{X}_{1})\cdot\prod_{i=2}^{m}\P(X_{i}=\bar{X}_{i}\mid X_{i-1}=\bar{X}_{i-1})\geq1-\veps\,.
\]
Thus, it leads to a coupling between $X_{m}$ and $\bar{X}_{m}$ such
that $\P(X_{m}=\bar{X}_{m})\geq1-\veps$, so $\law(X_{m})$ is within
$\veps$-TV distance to $\pi=\law(\bar{X}_{m})$.
\end{rem}


\subsection{Self-concordance theory ($\S$\ref{sec:sc-theory-rules})}

\subsubsection{Basic properties: strong self-concordance \label{proof:ssc-basic}}

We show that $2(g_{1}+g_{2})$ is SSC if $g_{1}$ and $g_{2}$ are
SSC.
\begin{proof}
[Proof of Lemma~\ref{lem:ssc-sum}] For fixed $x\in K_{1}\cap K_{2}$
and $h\in\Rd$, let $\Dd g_{i}:=\Dd g_{i}(x)[h]$ for $i=1,2$. Note
that 
\begin{align*}
 & \snorm{(g_{1}+g_{2})^{-\half}\Dd(g_{1}+g_{2})\,(g_{1}+g_{2})^{-\half}}_{F}\\
 & \leq\sum_{i=1}^{2}\snorm{(g_{1}+g_{2})^{-\half}\Dd g_{i}\,(g_{1}+g_{2})^{-\half}}_{F}=\sum_{i=1}^{2}\sqrt{\tr\bpar{(g_{1}+g_{2})^{-1}\Dd g_{i}\,(g_{1}+g_{2})^{-1}\Dd g_{i}}}\\
 & =\Bbrack{\tr\Bpar{\bpar{\underbrace{I+g_{1}^{-\half}g_{2}g_{1}^{-\half}}_{=:E_{1}}}^{-1}\underbrace{g_{1}^{-\half}\Dd g_{1}\,g_{1}^{-\half}}_{=:T_{1}}\bpar{I+g_{1}^{-\half}g_{2}g_{1}^{-\half}}^{-1}g_{1}^{-\half}\Dd g_{1}\,g_{1}^{-\half}}}^{1/2}\\
 & \qquad+\Bbrack{\tr\Bpar{\bpar{\underbrace{I+g_{2}^{-\half}g_{1}g_{2}^{-\half}}_{=:E_{2}}}^{-1}\underbrace{g_{2}^{-\half}\Dd g_{2}\,g_{2}^{-\half}}_{=:T_{2}}\bpar{I+g_{2}^{-\half}g_{1}g_{2}^{-\half}}^{-1}g_{2}^{-\half}\Dd g_{2}\,g_{2}^{-\half}\bigg)}}^{1/2}\\
 & =\sum_{i=1}^{2}\sqrt{\tr(E_{i}^{-1}T_{i}E_{i}^{-1}T_{i})}\leq\sum_{i=1}^{2}\sqrt{\tr(T_{i}E_{i}^{-2}T_{i})}\,,
\end{align*}
where we used the Cauchy-Schwarz inequality $\tr(A^{2})\leq\tr(A^{\T}A)$
in the last line. It follows from $I\preceq E_{i}$ that $I\preceq E_{i}^{2}$
and $I\succeq E_{i}^{-2}\succ0$. Therefore, 
\begin{align*}
\sum_{i=1}^{2}\sqrt{\tr(T_{i}E_{i}^{-2}T_{i})} & \leq\sum_{i=1}^{2}\snorm{T_{i}}_{F}\leq2\sum_{i=1}^{2}\snorm h_{g_{i}(x)}^{2}\leq2\sqrt{2}\norm h_{(g_{1}+g_{2})(x)}\,.
\end{align*}
Putting these together completes the proof.
\end{proof}

\subsubsection{Basic properties: lower trace self-concordance \label{proof:ltsc-basic}}

We now show that if $g$ is HSC, then $dg$ is SLTSC.
\begin{proof}
[Proof of Lemma~\ref{lem:hsc-to-sltsc}] We first consider when
$\bar{g}$ is positive definite on $K$. By HSC of $\bar{g}$, it
holds that $-\snorm h_{\bar{g}}^{2}\,\bar{g}\lesssim\Dd^{2}\bar{g}[h,h]$,
and thus
\[
-\frac{1}{d}\,\snorm h_{g}^{2}\,(g'+g)^{-\half}g\,(g'+g)^{-\half}\lesssim(g'+g)^{-\half}\Dd^{2}g[h,h]\,(g'+g)^{-\half}\,.
\]
Hence,
\begin{align*}
\tr\bpar{(g'+g)^{-1}\Dd^{2}g[h,h]} & \gtrsim-\frac{1}{d}\,\snorm h_{g}^{2}\,\tr\Bpar{(g'+g)^{-\half}g\,(g'+g)^{-\half}}=-\frac{1}{d}\,\snorm h_{g}^{2}\,\tr\bpar{g^{\half}(g'+g)^{-1}g^{\half}}\\
 & \geq-\frac{1}{d}\,\snorm h_{g}^{2}\,\tr(g^{\half}g^{-1}g^{\half})=-\snorm h_{g}^{2}\,.
\end{align*}

When $g$ is singular, we consider $\bar{g}_{\veps}=\bar{g}+\frac{\veps}{d}I\in\pd$
for $\veps>0$. Then $\bar{g}_{\veps}$ is HSC, so for $g_{\veps}=d\bar{g}_{\veps}$
\[
\tr\bpar{(g'+g_{\veps})^{-1}\Dd^{2}g[h,h]}\gtrsim-\snorm h_{g_{\veps}}^{2}\,.
\]
From $(g'+g_{\veps})^{-1}=\frac{1}{\det(g'+g_{\veps})}\,\text{adj}(g'+g_{\veps})$,
the LHS is continuous in $\veps$, and the RHS is too clearly. Sending
$\veps\to0$ completes the proof.
\end{proof}

\subsubsection{Basic properties: strongly average self-concordance \label{proof:sasc-basic}}

To prove Lemma \ref{lem:hsc-to-sasc}, we first recall a concentration
bound.
\begin{lem}
[\citet{narayanan2016randomized}, Lemma 4] \label{lem:odd-order-concen}Let
$h$ be drawn from $\mathbb{S}^{d-1}$ uniformly at random. For any
odd $k$, $C^{k}$-smooth $F:\Rd\to\R$, and $\veps>0$,
\[
\P_{h}\Bpar{|\Dd^{k}F(x)[h^{\otimes k}]|>k\veps\cdot\sup_{\snorm v\leq1}\Dd^{k}F(x)[v^{\otimes k}]}\leq\exp\Bpar{-\frac{d\veps^{2}}{2}}\,.
\]
\end{lem}

We show that if $g$ is HSC, then $dg$ is SASC, using this lemma
and following \citet{narayanan2016randomized}.
\begin{proof}
[Proof of Lemma~\ref{lem:hsc-to-sasc}] Let $g=d\,\hess\phi$ and
consider $g':\intk\to\psd$ such that $\bar{g}=g+g'$ is PD. For fixed
$w\in\Rd$, apply Taylor's expansion to $\vphi(z):=\norm w_{g(z)}^{2}$
at $z=x$, so there exists $p_{w}\in[x,z]$ such that $w^{\T}g(z)w=w^{\T}g(x)w+\Dd g(x)[z,w,w]+\half\,\Dd^{2}g(p_{w})[z,z,w,w].$
Putting $z=w$ here,
\[
|\snorm z_{g(z)}^{2}-\snorm z_{g(x)}^{2}|\leq|\Dd^{3}g(x)[z^{\otimes3}]|+\half|\Dd^{2}g(p_{z})[z^{\otimes4}]|\,.
\]

Going forward, we can assume that $x=0$ and $\bar{g}(x)=I$ due to
affine invariance, and then $z$ equals $rh/\sqrt{d}$ for $h\sim\ncal(0,I_{d})$
in law. Using a standard tail bound on the standard Gaussian, we have
$\P_{h}(\norm h\geq-\sqrt{d}\cdot2\log\veps)\leq\veps.$ Call this
event $B_{1}$. In addition, Lemma~\ref{lem:odd-order-concen} implies
that 
\[
\P\Bpar{\Big|\Dd^{3}\phi(x)\Bbrack{\frac{h^{\otimes3}}{\norm h^{3}}}\Big|\geq3\frac{\veps}{\sqrt{d}}\cdot\sup_{\norm v\leq1}\Dd^{3}\phi(x)[v^{\otimes3}]}\leq\veps\,,
\]
and call this event $B_{2}$. Conditioned on $B_{2}^{c}$,
\begin{align*}
\Big|\Dd^{3}\phi(x)\Bbrack{\frac{h^{\otimes3}}{\norm h^{3}}}\Big| & \leq\frac{3\veps}{\sqrt{d}}\,\sup_{\norm v\leq1}\Dd^{3}\phi(x)[v^{\otimes3}]\leq\frac{6\veps}{\sqrt{d}}\,\sup_{\norm v\leq1}\snorm v_{g(x)/d}^{3}\leq\frac{6\veps}{d^{2}}\,\sup_{\norm v\leq1}\snorm v_{g(x)}^{3}\underbrace{\leq}_{g(x)\preceq I_{d}}\frac{6\veps}{d^{2}}\,.
\end{align*}
Hence, conditioned on $z\in B_{1}^{c}\cap B_{2}^{c}$
\begin{align*}
|\Dd^{3}g(x)[z^{\otimes3}]| & =\frac{r^{3}}{\sqrt{d}}\,\Dd^{3}\phi(x)[h^{\otimes3}]\leq\frac{r^{3}}{\sqrt{d}}\,\frac{6\veps}{d^{2}}\,\snorm h^{3}\leq\frac{r^{2}}{d}\cdot48r\veps\Bpar{\log\frac{1}{\veps}}^{3}\,.
\end{align*}
By taking $r_{1}(\veps)$ so that $-48r_{1}\veps\,(\log\veps)^{3}\leq\veps$,
we can ensure $|\Dd^{3}g(x)[z^{\otimes3}]|\leq\veps r^{2}/d$ for
any $r\leq r_{1}(\veps)$.

As for $|\Dd^{2}g(p_{z})[z^{\otimes4}]|$, HSC of $\phi$ and Lemma~\ref{lem:scCloseness}
lead to 
\begin{align*}
\half\,|\Dd^{2}g(p_{z})[z^{\otimes4}]| & \leq3d\,\snorm z_{\hess\phi(p_{z})}^{4}\le\frac{3}{d}\snorm z_{\hess\phi(x)}^{4}\,(1+2\,\snorm z_{\hess\phi(x)}^{2})^{2}=\frac{3}{d}\,\snorm z_{g(x)}^{4}\,\bpar{1+\frac{2}{d}\,\snorm z_{g(x)}^{2}}^{2}\\
 & \underset{g\preceq I_{d}}{\leq}\frac{3}{d}\snorm z^{4}\bpar{1+\frac{2}{d}\,\snorm z^{2}}^{2}=\frac{3}{d}\,\frac{r^{4}}{d^{2}}\,\snorm h^{4}\Bpar{1+\frac{2r^{2}}{d^{2}}\norm h^{2}}^{2}\\
 & \leq\frac{r^{2}}{d}\cdot3r^{2}\,\bpar{2\log\frac{1}{\veps}}^{4}\Bpar{1+2r^{2}\bpar{2\log\frac{1}{\veps}}^{4}}^{2}\,.
\end{align*}
By taking $r_{2}(\veps)$ and $r_{3}(\veps)$ so that $\Bpar{1+2r_{2}^{2}\bpar{2\log\frac{1}{\veps}}^{4}}^{2}\leq2$
and $2^{2}\cdot3r_{3}^{2}\bpar{2\log\frac{1}{\veps}}^{4}\leq\veps$
respectively, it holds that on $B_{1}^{c}\cap B_{2}^{c}$
\[
\half\,|\Dd^{2}g(p_{z})[z^{\otimes4}]|\leq\veps\frac{r^{2}}{d}\ \text{for any }r\leq\min r_{i}(\veps).
\]
Putting all these together, it follows that $|\norm z_{g(z)}^{2}-\norm z_{g(x)}^{2}|\leq2\veps r^{2}/d$
with probability at least $1-2\veps$. By replacing $2\veps\gets\veps$,
the claim follows.
\end{proof}

\subsubsection{Collapse and embedding: well-definedness \label{proof:collapse-embedding-welldefined}}

We start with well-definedness of the notions of collapse and embedding
(Definition~\ref{def:sc-along-subspace}).
\begin{proof}
[Proof of Proposition~\ref{prop:collapse-well-defined}] Let $k:=\dim(W)$,
and $U$ and $V$ be matrices in $\R^{d\times k}$, where the columns
of each matrix form an orthonormal basis of $W$. Let us denote by
$g_{1}:=U^{\T}gU$ and $g_{2}:=V^{\T}gV$ matrices represented with
respect to $U$ and $V$, and define the invertible matrix $M=V^{-1}U\in\R^{k\times k}$.
Since $U$ and $V$ are full-column rank, if $g_{1}$ is PD, so is
$g_{2}$.

Suppose $g$ is SSC along $W$. Then, 
\begin{align*}
4\norm h_{g}^{2} & \geq\tr(g_{1}^{-1}\Dd g_{1}[h]\,g_{1}^{-1}\Dd g_{1})=\tr\bpar{(U^{\T}gU)^{-1}\cdot U^{\T}\Dd g[h]\,U\cdot(U^{\T}gU)^{-1}\cdot U^{\T}\Dd g[h]\,U}\\
 & =\tr\Bpar{(M^{\T}V^{\T}gVM)^{-1}\cdot M^{\T}V^{\T}\Dd g[h]\,VM\cdot(M^{\T}V^{\T}gVM)^{-1}\cdot M^{\T}V^{\T}\Dd g[h]\,VM}\\
 & =\tr\Bpar{(V^{\T}gV)^{-1}V^{\T}\Dd g[h]\,V\,(V^{\T}gV)^{-1}V^{\T}\Dd g[h]\,V}=\norm{g_{2}^{-\half}\Dd g_{2}[h]\,g_{2}^{-\half}}_{F}^{2}\,,
\end{align*}
and thus $g_{2}$ also satisfies the definition.
\end{proof}

\subsubsection{Collapse and embedding: affine transformation \label{proof:collap-affine}}

We begin with a barrier version.
\begin{proof}
[Proof of Lemma~\ref{lem:linear-trans}] For the first part, $\psi$
is a $\nu$-self-concordant barrier for $\bar{K}$ by \citet[Theorem 4.2.3]{nesterov2003introductory},
so $\dcal_{\bar{g}}^{1}(x)\subset\bar{K}\cap(2x-\bar{K})$ for $\bar{g}(\cdot):=\hess\psi(\cdot)$
by Lemma~\ref{lem:symmetricLeftpart}. Now let $z\in\bar{K}\cap(2x-\bar{K})$.
Then $Tz\in K$ and $T(2x-z)\in K$, and the latter implies $2y-Tz\in K$.
Thus $Tz\in K\cap(2y-K)$ and $Tz\in\dcal_{g}^{\sqrt{\onu}}(y)$.
Due to 
\begin{align*}
\Dd^{2}\psi(x)[(z-x)^{\otimes2}] & =\Dd^{2}\phi(y)[\bpar{A(z-x)}^{\otimes2}]=\Dd^{2}\phi(y)[(Tz-y)^{\otimes2}]\leq\onu\,,
\end{align*}
it follows that $\psi$ is also $\onu$-symmetric. 

For the second part, observe that $\Dd^{4}\psi(x)[v,v,h,h]=\Dd^{4}\phi(y)[Av,Av,Ah,Ah]\geq0$
for any $v,h\in\Rd$. The third part can be proven similarly.
\end{proof}
Next is a matrix version.
\begin{proof}
[Proof of Lemma~\ref{lem:linear-trans-matrix}] Let $\phi$ be a
$\nu$-self-concordant function counterpart of $g$. Then $\psi(x):=\phi(Tx)$
defined on $\inter(\bar{K})$ is $\nu$-self-concordant by Lemma~\ref{lem:linear-trans}.
For any $h\in\Rd$ and $y:=Tx$, we have 
\[
\Dd\bar{g}(x)[h]=A^{\T}\Dd g(y)[Ah]\,A\preceq2\norm{Ah}_{g(y)}\,A^{\T}g(y)A=2\norm h_{\bar{g}(x)}\,\bar{g}(x)\,.
\]

Consider a sequence $\{x_{n}\}\subset\bar{K}$ converging to a boundary
point $x\in\de\bar{K}$. If $Tx\notin\de K$, then $Tx\in\inter(K)$,
and the continuity of $T$ implies $x$ is also in $\inter(\bar{K})$.
Thus, $Tx\in\de K$ and $\psi(x_{n})=\phi(Tx_{n})\to\phi(Tx)=\infty$.
Lastly, $\hess\phi\asymp g$ leads to $\hess\psi=A^{\T}\hess\phi\,A\asymp A^{\T}gA=\bar{g}$,
and $\bar{g}$ is $\nu$-self-concordant for $\bar{K}$.

As for symmetry, since $\bar{g}$ is self-concordant, $\mc D_{\bar{g}}^{1}(x)\subset\bar{K}\cap(2x-\bar{K})$
for $x\in\inter(\bar{K})$ by Lemma~\ref{lem:dikin-in-body}. For
$z\in\bar{K}\cap(2x-\bar{K})$, as $Tz\in K\cap(2Tx-K)$ holds, it
follows that 
\[
\onu\geq\norm{Tz-Tx}_{g(y)}^{2}=\norm{z-y}_{A^{\T}g(y)A}^{2}=\norm{z-y}_{\bar{g}(x)}^{2}\,,
\]
and thus $\bar{g}$ is $\onu$-symmetric.

As for the second item, we first show that $\bar{g}$ is collapsed
onto $W=\rowspace(A)$ (i.e., $\bar{g}=P_{W}\bar{g}P_{W}$ for the
orthogonal projection $P_{W}$ onto $W$). To see this, observe that
\begin{align*}
P_{W}\bar{g}P_{W} & =P_{W}A^{\T}gAP_{W}=A^{\T}(AA^{\T})^{\dagger}A\cdot A^{\T}gA\cdot A^{\T}(AA^{\T})^{\dagger}A\,,
\end{align*}
and due to $AA^{\T}(AA^{\T})^{\dagger}A=AA^{\T}(A^{\T})^{\dagger}A^{\dagger}A=AA^{\dagger}A=A$,
we have $P_{W}\bar{g}P_{W}=A^{\T}gA=\bar{g}$. 

We now show that $\bar{g}$ is SSC along $W$. For $k:=\dim(W)$,
take $U\in\R^{d\times k}$ whose columns form an orthonormal basis
of $W$. It suffices to show that $g_{W}:=U^{\T}\bar{g}U=U^{\T}A^{\T}gAU=M^{\T}gM$
for $M:=AU\in\R^{m\times k}$ is SSC. First of all, we can check PDness
of $g_{W}$ as follows: Suppose $g_{W}v=0$ for some $v\in\R^{k}$.
Then $0=\norm v_{g_{W}}=\norm{g^{1/2}Mv}_{2}$ and $AUv=Mv=0$. Since
$Uv\in\rowspace(A)\cap\textsf{ker}(A)$ and $U$ is full-rank, we
have $v=0$. Next, for $h\in\R^{k}$ and $x\in\inter(\bar{K})$
\begin{align*}
 & \tr\bpar{g_{W}(x)^{-1}\Dd g_{W}(x)[h]\,g_{W}(x)^{-1}\Dd g_{W}(x)[h]}=\tr\Bpar{\bpar{g^{\half}M(M^{\T}gM)^{-1}M^{\T}g^{\half}\cdot g^{-\half}\Dd g(Tx)[Ah]\,g^{-\half}}^{2}}\\
\underset{\text{(i)}}{\leq} & \tr\Bpar{\bpar{g^{-\half}\Dd g(Tx)[Ah]\,g^{-\half}}^{2}}\leq\norm{g^{-\half}\Dd g(Tx)[Ah]\,g^{-\half}}_{F}^{2}\leq4\norm{Ah}_{g(Tx)}^{2}=4\norm h_{\bar{g}(x)}^{2}\,,
\end{align*}
where in (i) we used $P(g^{\half}M)=g^{\half}M(M^{\T}gM)^{-1}M^{\T}g^{\half}\preceq I$.
Thus, $\bar{g}$ is SSC along $W=\rowspace(A)$.

The third item immediately follows from $\Dd^{2}\bar{g}(x)[h,h]=A^{\T}\Dd^{2}g(y)[Ah,Ah]\,A\succeq0$
for any $h\in\Rd$. 

As for the fourth item, for any PSD matrix function $g'$ on $\bar{K}$
we have
\begin{align*}
 & \tr\bpar{(g'+\bar{g})^{-1}\Dd^{2}\bar{g}[h,h]}=\tr\Bpar{(g'+A^{\T}gA)^{-1}A^{\T}\Dd^{2}g[Ah,Ah]\,A}\\
= & \tr\Bpar{(A^{-\T}g'A^{-1}+g)^{-1}\Dd^{2}g[Ah,Ah]}\geq-\norm{Ah}_{g}^{2}=-\norm h_{\bar{g}}^{2}\,.
\end{align*}

The last item is straightforward to check by the change of variable.
\end{proof}

\subsubsection{Collapse and embedding: lifting up SSC, SLTSC, and SASC \label{proof:lifting-ssc}}

In passing SSC to an augmented space, the Woodbury matrix identity
is a main technical tool used: for matrices with compatible sizes
\[
(I+UV)^{-1}=I-U\,(I+VU)^{-1}V\,.
\]
Using this, we show that if $g\in\pd$ is SSC, then $\bar{g}+\veps I_{m}$
is SSC.
\begin{proof}
[Proof of Lemma~\ref{lem:embedding-ssc}] Fix $\veps>0,y\in\inter(K')$,
and $h\in\R^{m}$. Take a projection matrix $P\in\{0,1\}^{d\times m}$
such that $PP^{\T}=I_{d}$ and $\bar{g}(y)=P^{\T}g(Py)P$ for $x=Py\in\intk$.
Also for $k:=\dim(W)$, take a matrix $U\in\R^{d\times k}$ whose
columns form an orthonormal basis of $W$. Then $\bar{g}(y)=P^{\T}g(Py)P$
and $g(x)=Ug_{W}(x)U$, so for $M:=U^{\T}P\in\R^{k\times m}$,
\[
\bar{g}(y)=P^{\T}Ug_{W}(Py)U^{\T}P=M^{\T}g_{W}(Py)M\,.
\]
 Note that $MM^{\T}=I_{k}$. Thus,
\begin{align*}
 & \norm{(\bar{g}(y)+\veps I)^{-\half}\Dd(\bar{g}+\veps I)(y)[h]\,(\bar{g}(y)+\veps I)^{-\half}}_{F}^{2}=\tr\Bpar{\bpar{(\bar{g}(y)+\veps I)^{-1}\Dd\bar{g}(y)[h]}^{2}}\\
= & \tr\Bpar{\bpar{M(M^{\T}g_{W}(x)\,M+\veps I)^{-1}M^{\T}\cdot\Dd g_{W}(x)[Ph]}^{2}}\underset{\text{(i)}}{=}\tr\Bpar{\bpar{(g_{W}(x)+\veps I_{k})^{-1}\Dd g_{W}(x)[Ph]}^{2}}\\
\leq & \norm{g_{W}(x)^{-\half}\Dd g_{W}(x)[Ph]\,g_{W}(x)^{-\half}}_{F}^{2}\leq4\norm{Ph}_{g(x)}^{2}=4\norm h_{\bar{g}(y)}^{2}\,,
\end{align*}
where in (i) we used the identity $M\bpar{M^{\T}g_{W}(x)\,M+\veps I}^{-1}M^{\T}=(g_{W}(x)+\veps I_{k})^{-1}$.
To see this, we use the Woodbury matrix identity to get
\[
(\veps I_{m}+M^{\T}g_{W}M)^{-1}=\frac{1}{\veps}I_{m}-\frac{1}{\veps^{2}}M^{\T}g_{W}^{\half}\bpar{I_{k}+\frac{1}{\veps}g_{W}}^{-1}g_{W}^{\half}M\,,
\]
and thus conjugating both sides by $M$ results in 
\begin{align*}
M\bpar{M^{\T}g_{W}M+\veps I_{m}}^{-1}M^{\T} & =\frac{1}{\veps}I_{k}-\frac{1}{\veps}g_{W}^{\half}(g_{W}+\veps I_{k})^{-1}g_{W}^{\half}=\frac{1}{\veps}I_{k}-\frac{1}{\veps}(g_{W}+\veps I_{k})^{-1}g_{W}\,.
\end{align*}
Then, the identity follows from
\begin{align*}
(g_{W}+\veps I_{k})\cdot\bpar{\frac{1}{\veps}I_{k}-\frac{1}{\veps}\,(g_{W}+\veps I_{k})^{-1}g_{W}} & =\frac{1}{\veps}(g_{W}+\veps I_{k})-\frac{1}{\veps}g_{W}=I_{k}\,.\qedhere
\end{align*}
\end{proof}
In extending SLTSC and SASC, we need two technical lemmas: the inverse
of a block matrix and connection between P(S)Dness and Schur complements.
\begin{lem}
\label{lem:block-inverse} If $D$ and its Schur complement $A-BD^{-1}C$
are invertible, then 
\[
\left[\begin{array}{cc}
A & B\\
C & D
\end{array}\right]^{-1}=\left[\begin{array}{cc}
(A-BD^{-1}C)^{-1} & *\\
* & *
\end{array}\right]\,.
\]
\end{lem}

\begin{lem}
[Schur complement] \label{lem:schur} Let $A\in\Rdd,B\in\R^{d\times m},C\in\R^{m\times m}$
and define a matrix $M\in\R^{(m+d)\times(m+d)}$ by 
\[
M=\left[\begin{array}{cc}
A & B\\
B^{\T} & D
\end{array}\right]\,.
\]
Then $M\succ0$ if and only if $A\succ0$ and $C-BA^{-1}B^{\T}\succ0$
if and only $C\succ0$ and $A-B^{\T}C^{-1}B\succ0$.
\end{lem}

Using these, we show that if $g$ is SLTSC and SASC, then $\bar{g}$
is SLTSC and SASC.
\begin{proof}
[Proof of Lemma~\ref{lem:embedding-sltsc}] Take a full row-rank
projection matrix $P\in\{0,1\}^{d\times m}$ such that $\bar{g}(y)=P^{\T}g(Py)P$,
where the rows of $P$ forms a subset of the canonical basis $\{e_{1},\dots,e_{m}\}$.
We can augment the rows of $P$ with the rest of the canonical basis
so that the augmented matrix $\bar{P}\in\R^{m\times m}$ is an orthonormal
matrix. Then we can represent $\bar{g}$ by 
\[
\bar{g}(y)=\bar{P}^{\T}\left[\begin{array}{cc}
g(Py) & 0\\
0 & 0
\end{array}\right]\bar{P}\,.
\]

Consider a PSD matrix function $g':\inter(K')\to\mbb S_{+}^{m}$ such
that $g'+\bar{g}$ is PD on $K'$. Representing them in the block
form with $g_{A}\in\Rdd,g_{B}\in\R^{d\times(m-d)},$ and $g_{C}\in\R^{(m-d)\times(m-d)}$
\[
\bar{g}+g'=\bar{P}^{\T}\Par{\left[\begin{array}{cc}
g & 0\\
0 & 0
\end{array}\right]+\left[\begin{array}{cc}
g_{A} & g_{B}\\
g_{B}^{\T} & g_{C}
\end{array}\right]}\bar{P}=\bar{P}^{\T}\underbrace{\left[\begin{array}{cc}
g+g_{A} & g_{B}\\
g_{B}^{\T} & g_{C}
\end{array}\right]}_{\eqqcolon g^{*}}\bar{P}\,.
\]
Since $g^{*}$ is PD, $g_{C}$ and its Schur complement $(g+g_{A})-g_{B}g_{C}^{-1}g_{B}^{\T}$
are PD. Thus by Lemma~\ref{lem:block-inverse},
\[
\left[\begin{array}{cc}
g+g_{A} & g_{B}\\
g_{B}^{\T} & g_{C}
\end{array}\right]^{-1}=\left[\begin{array}{cc}
(g+g_{A}-g_{B}g_{C}^{-1}g_{B}^{\T})^{-1} & *\\
* & *
\end{array}\right]\,.
\]
Hence,
\begin{align*}
 & \tr\bpar{(\bar{g}+g')^{-1}\Dd^{2}\bar{g}(y)[h,h]}=\tr\Biggl(\bar{P}^{\T}\left[\begin{array}{cc}
g+g_{A} & g_{B}\\
g_{B}^{\T} & g_{C}
\end{array}\right]^{-1}\bar{P}\bar{P}^{\T}\left[\begin{array}{cc}
\Dd^{2}g(Py)[Ph,Ph] & 0\\
0 & 0
\end{array}\right]\bar{P}\Biggr)\\
= & \tr\Biggl(\left[\begin{array}{cc}
g+g_{A} & g_{B}\\
g_{B}^{\T} & g_{C}
\end{array}\right]^{-1}\left[\begin{array}{cc}
\Dd^{2}g(Py)[Ph,Ph] & 0\\
0 & 0
\end{array}\right]\Biggr)=\tr\bpar{(g+\underbrace{g_{A}-g_{B}g_{C}^{-1}g_{B}^{\T}}_{\succeq0})^{-1}\,\Dd^{2}g(Py)[Ph,Ph]}\\
\ge & -\norm{Ph}_{g(Py)}^{2}=-\norm h_{\bar{g}(y)}^{2}\,,
\end{align*}
where in the last inequality we used STLSC of $g$, since $g'\succeq0$
ensures that its Schur complement satisfies $g_{A}-g_{B}g_{C}^{-1}g_{B}^{\T}\succeq0$
by Lemma~\ref{lem:schur}.

For SASC, consider any PSD matrix function $g':\inter(K')\to\mbb S_{+}^{m}$.
For $x=Py$ and $z_{x}=Pz_{y}\in\Rd$ with $z_{y}\sim\ncal\bpar{y,\frac{r^{2}}{m}\,(\bar{g}+g)(y)^{-1}}$,
we have
\[
\norm{z_{y}-y}_{\bar{g}(z_{y})}^{2}-\norm{z_{y}-y}_{\bar{g}(y)}^{2}=\norm{z_{x}-x}_{g(z_{x})}^{2}-\norm{z_{x}-x}_{g(x)}^{2}\,.
\]
Also, $z_{x}-x=P\,(z_{y}-y)$ is a Gaussian with zero mean and covariance
\begin{align*}
 & \frac{r^{2}}{m}\,P\,(\bar{g}+g')(y)^{-1}P^{\T}=\frac{r^{2}}{m}\,P\bar{P}^{\T}\Par{\left[\begin{array}{cc}
g & 0\\
0 & 0
\end{array}\right]+\left[\begin{array}{cc}
g_{A} & g_{B}\\
g_{B}^{\T} & g_{C}
\end{array}\right]}^{-1}\bar{P}\bar{P}^{\T}\\
= & \frac{r^{2}}{m}\,\left[\begin{array}{cc}
I_{d} & 0_{d\times(m-d)}\end{array}\right]\Par{\left[\begin{array}{cc}
g & 0\\
0 & 0
\end{array}\right]+\left[\begin{array}{cc}
g_{A} & g_{B}\\
g_{B}^{\T} & g_{C}
\end{array}\right]}^{-1}\left[\begin{array}{c}
I_{d}\\
0_{d\times(m-d)}
\end{array}\right]=\frac{r^{2}}{m}\,(g+g_{A}-g_{B}g_{C}^{-1}g_{B}^{\T})^{-1}\,.
\end{align*}
Since $g_{A}-g_{B}g_{C}^{-1}g_{B}^{\T}\succeq0$ due to  $g'\succeq0$,
it holds that $g_{0}:=\frac{m-d}{d}g+\frac{m}{d}(g_{A}-g_{B}g_{C}^{-1}g_{B}^{\T})$
on $\intk$ is PSD. Now, it suffices to check that the covariance
matrix above is equal to $\frac{r^{2}}{d}(g+g_{0})^{-1}$:
\[
\frac{d}{r^{2}}\,(g+g_{0})=\frac{d}{r^{2}}\Bpar{g+\frac{m-d}{d}\,g+\frac{m}{d}\,(g_{A}-g_{B}g_{C}^{-1}g_{B}^{\T})}\frac{m}{r^{2}}\,(g+g_{A}-g_{B}g_{C}^{-1}g_{B}^{\T})\,.\qedhere
\]
\end{proof}

\subsubsection{Direct product: SSC and SLTSC \label{proof:direct-ssc-sltsc}}

We show that if $g_{i}\in\mbb S_{++}^{d_{i}}$ is SC, then $g=\sum d_{i}\bar{g}_{i}$
is SSC.
\begin{proof}
[Proof of Lemma~\ref{lem:ssc-direct}] Note that $d_{i}g_{i}$ is
SSC for $i=1,\dots,m$. For $x\in\prod E_{i}$ and $h=(h_{1},\dots,h_{m})\in\R^{l}$
with $h_{i}\in\R^{d_{i}}$, we have 
\begin{align*}
 & \norm{g(x)^{-\half}\Dd g(x)[h]\,g(x)^{-\half}}_{F}^{2}\\
 & =\left\Vert \left[\begin{array}{ccc}
g_{1}(x_{1})^{-\half}\Dd g_{1}(x_{1})[h_{1}]\,g_{1}(x_{1})^{-\half}\\
 & \ddots\\
 &  & g_{m}(x_{m})^{-\half}\Dd g_{m}(x_{m})[h_{m}]\,g_{m}(x_{m})^{-\half}
\end{array}\right]\right\Vert _{F}^{2}\\
 & =\sum_{i}\norm{g_{i}(x_{i})^{-\half}\Dd g_{i}(x_{i})[h_{i}]\,g_{i}(x_{i})^{-\half}}_{F}^{2}\leq4\sum_{i}\norm{h_{i}}_{d_{i}g_{i}(x_{i})}^{2}=4\norm h_{g(x)}^{2}\,.\qedhere
\end{align*}
\end{proof}
Next, we show that if $g_{i}\in\mbb S_{++}^{d_{i}}$ is HSC, then
$g=\sum d_{i}\bar{g}_{i}$ is SLTSC.
\begin{proof}
[Proof of Lemma~\ref{lem:sltsc-direct}] For $h=(h_{1},\dots,h_{m})$
and any PSD matrix function $g'$, we have
\begin{align*}
\tr\bpar{(g'+g)^{-1}\Dd^{2}g[h^{\otimes2}]} & =\sum_{i}\tr\bpar{(g'+(g-d_{i}\bar{g}_{i})+d_{i}\bar{g}_{i})^{-1}\Dd^{2}(d_{i}\bar{g}_{i})[h^{\otimes2}]}\gtrsim-\sum_{i}\norm h_{d_{i}\bar{g}_{i}}^{2}=-\norm h_{g}^{2}\,,
\end{align*}
where we used Lemma~\ref{lem:hsc-to-sltsc} in the inequality.
\end{proof}

\subsubsection{Inverse images under non-linear mappings \label{proof:inverse-non-linear}}
\begin{proof}
[Proof of Lemma~\ref{lem:compatible}] Since $\acal$ is $(R(G),\beta),\gamma)$-compatible
with $\Gamma$, the first two claims immediately follow from \citet[Proposition 5.1.7]{nesterov1994interior}.
Let $x\in G^{+}$ and $h\in\Rd$. Define the following notations:
\begin{align*}
u=\Dd\acal(x)[h], & \quad v=\Dd^{2}\acal(x)[h^{\otimes2}],\quad w=\Dd^{3}\acal(x)[h^{\otimes3}],\quad z=\Dd^{4}\acal(x)[h^{\otimes4}],\\
s=\sqrt{\Dd F(y)[v]}, & \quad\rho=\sqrt{\Dd^{2}\Pi(x)[h^{\otimes2}]},\quad r=\sqrt{\Dd^{2}F(y)[u^{\otimes2}]}\,.
\end{align*}
From direct computations, we have 
\begin{align*}
\Dd^{2}\Psi(x)[h^{\otimes2}] & =\Dd F(y)[v]+\Dd^{2}F(y)[u^{\otimes2}]+\delta^{2}\Dd^{2}\Pi(x)[h^{\otimes2}]=s^{2}+r^{2}+\delta^{2}\rho^{2}\,,\\
\Dd^{3}\Psi(x)[h^{\otimes3}] & =\Dd F(y)[w]+3\Dd^{2}F(y)[u,v]+\Dd^{3}F(y)[u^{\otimes3}]+\delta^{2}\Dd^{3}\Pi(x)[h^{\otimes3}]\,,\\
\Dd^{4}\Psi(x)[h^{\otimes4}] & =\Dd^{2}F(y)[w,u]+\Dd F(y)[z]+3\Dd^{3}F(y)[u,u,v]+3\Dd^{2}F(y)[v^{\otimes2}]\\
 & \qquad+3\Dd^{2}F(y)[u,w]+\Dd^{4}F(y)[u^{\otimes4}]+3\Dd^{3}F(y)[u,u,v]+\delta^{2}\Dd^{4}\Pi(x)[h^{\otimes4}]\\
 & =\Dd F(y)[z]+3\Dd^{2}F(y)[v^{\otimes2}]+4\Dd^{2}F(y)[u,w]\\
 & \qquad+6\Dd^{3}F(y)[u,u,v]+\Dd^{4}F(y)[u^{\otimes4}]+\delta^{2}\Dd^{4}\Pi(x)[h^{\otimes4}]\,.
\end{align*}
HSC of $F$ and $\Pi$ implies that 
\[
|\Dd^{4}\Pi(x)[h^{\otimes4}]|\leq6\rho^{4}\,,\qquad\text{and}\qquad|\Dd^{4}F(y)[u^{\otimes4}]|\leq6r^{4}\,.
\]
Since $\acal$ is $(K,\beta,\gamma)$-compatible and $K\subset R(G)$,
Lemma~\ref{lem:extension-compatibility}-1 implies concavity of $\acal$
with respect to $R(G)$, which means $-v\geq_{R(G)}0$. Then, \citet[Corollary 2.3.1]{nesterov1994interior}
ensures 
\[
\sqrt{\Dd^{2}F(y)[v^{\otimes2}]}\leq\Dd F(y)[v]=s^{2}\,.
\]
Hence, $|3\Dd^{2}F(y)[v,v]|\leq3(\Dd F(y)[v])^{2}=3s^{4}$, and self-concordance
of $F$ results in
\[
|6\Dd^{3}F(y)[u,u,v]|\leq12r^{2}\sqrt{\Dd^{2}F(y)[v,v]}\leq12r^{2}s^{2}\,.
\]
Since $\{h:h^{\T}\Pi(x)h\leq1\}$ is contained in $\Gamma\cap(2x-\Gamma)$,
compatibility of $\acal$ leads to 
\[
\beta\Dd^{2}\acal(x)\Bbrack{\Bpar{\frac{h}{\norm h_{\Pi(x)}}}^{\otimes2}}\leq_{K}\Dd^{3}\acal(x)\Bbrack{\Bpar{\frac{h}{\norm h_{\Pi(x)}}}^{\otimes3}}\leq_{K}-\beta\Dd^{2}\acal(x)\Bbrack{\Bpar{\frac{h}{\norm h_{\Pi(x)}}}^{\otimes2}}\,,
\]
and thus $\beta\rho v\leq_{K}w\leq_{K}-\beta\rho v$. As $K$ is a
ray, $\Dd^{2}F(y)[w,w]\leq\beta^{2}\rho^{2}\Dd^{2}F(y)[v,v]\leq\beta^{2}\rho^{2}s^{4}$.
Thus,
\[
|4\Dd^{2}F(y)[u,w]|\leq4\sqrt{\Dd^{2}F(y)[u,u]}\sqrt{\Dd^{2}F(y)[w,w]}\leq4r\beta\rho s^{2}\,.
\]
Lastly, since $\gamma v\rho^{2}\leq_{K}z\leq_{K}-\gamma v\rho^{2}$
and $K$ is a ray, we have 
\[
|\Dd F(y)[z]|\leq3\gamma\rho^{2}|\Dd F(y)[v]|=3\gamma\rho^{2}s^{2}\,.
\]
Putting these together,
\begin{align*}
 & \Abs{\Dd^{4}\Psi(x)[h^{\otimes4}]}\leq3\gamma\rho^{2}s^{2}+4r\beta\rho s^{2}+12r^{2}s^{2}+3s^{4}+6\delta^{2}\rho^{4}+6r^{4}\\
\leq & 6(\delta^{2}\rho^{4}+r^{4}+s^{4}+r^{2}s^{2}+\delta\rho^{2}s^{2}+\delta r\rho s^{2})\leq6\bpar{(\delta\rho)^{4}+r^{4}+s^{4}+r^{2}s^{2}+(\delta\rho)^{2}s^{2}+r^{2}s^{2}+(\delta\rho)^{2}s^{2}}\\
\leq & 6\bpar{(\delta\rho)^{2}+r^{2}+s^{2}}^{2}=6\bpar{\Dd^{2}\Psi(x)[h,h]}^{2}\,.\qedhere
\end{align*}
\end{proof}

\subsection{Main constraints and epigraphs ($\S$\ref{sec:handbook-barrier})}

\subsubsection{Linear constraints: strong self-concordance and symmetry \label{proof:linear-SSC-symm}}

We relate SSC and symmetry to well-studied terms in the field of optimization,
such as $\max_{i}\frac{[\sigma(\sqrt{D_{x}}A_{x})]_{i}}{[D_{x}]_{ii}}$
and $\norm{D_{x,h}'}_{D_{x}^{-1}}^{2}$.
\begin{proof}
[Proof of Lemma~\ref{lem:helper4Diagonal}] Let us write $g(x)=A_{x}^{\T}D_{x}A_{x}=A^{\T}V_{x}A$
for $V_{x}:=S_{x}^{-1}D_{x}S_{x}^{-1}$. By Claim~\ref{claim:diffLogBarrier},
\begin{align}
\Dd g(x)[h] & =A^{\T}(-2S_{x}^{-1}S_{x,h}S_{x}^{-1}D_{x}+S_{x}^{-1}\Dd D_{x}[h]\,S_{x}^{-1})A=A^{\T}V_{x}^{1/2}\overline{D}_{x}V_{x}^{1/2}A\,,\label{eq:Dgh}
\end{align}
where $\overline{D}_{x}:=-2S_{x,h}+D_{x}^{-1}\Dd D_{x}[h]$. Using
this,
\begin{align*}
\norm{(g'+g)^{-\half}\Dd g[h]\,(g'+g)^{-\half}}_{F}^{2} & =\tr\bpar{(g'+g)^{-1}A^{\T}V_{x}^{1/2}\overline{D}_{x}\underbrace{V_{x}^{1/2}A(g'+g)^{-1}A^{\T}V_{x}^{1/2}}_{=:P_{x}'}\overline{D}_{x}V_{x}^{1/2}A}\\
 & =\tr(P_{x}'\overline{D}_{x}P_{x}'\overline{D}_{x})\,.
\end{align*}
By Lemma~\ref{lem:matrix-projection}, we have $P_{x}'\preceq P_{x}=P(V_{x}^{1/2}A)=P(D_{x}^{1/2}A_{x})$,
and thus 
\begin{align*}
\tr(P_{x}'\overline{D}_{x}P_{x}'\overline{D}_{x}) & \leq\tr(P_{x}\overline{D}_{x}P_{x}\overline{D}_{x})\underset{\text{(i)}}{=}\diag(\overline{D}_{x})^{\T}P_{x}^{(2)}\,\diag(\overline{D}_{x})\underset{\text{(ii)}}{\leq}\diag(\overline{D}_{x})^{\T}\Sigma_{x}\,\diag(\overline{D}_{x})\\
 & \underset{\text{(iii)}}{\leq}4\sum_{i=1}^{m}[\sigma(D_{x}^{1/2}A_{x})]_{i}\,\bpar{(A_{x}h)_{i}^{2}+(D_{x}^{-1}\Dd D_{x}[h])_{i}^{2}}\\
 & \leq4\max_{i}\frac{[\sigma(D_{x}^{1/2}A_{x})]_{i}}{[D_{x}]_{ii}}\cdot\sum_{i=1}^{m}[D_{x}]_{ii}\,\bpar{(A_{x}h)_{i}^{2}+(D_{x}^{-1}\Dd D_{x}[h])_{i}^{2}}\\
 & \underset{\text{(iv)}}{=}4\max_{i}\frac{[\sigma(D_{x}^{1/2}A_{x})]_{i}}{[D_{x}]_{ii}}\cdot\bpar{\norm h_{g(x)}^{2}+\sum_{i=1}^{m}[D_{x}^{-1}]_{ii}(\Dd D_{x}[h])_{i}^{2}}\,,
\end{align*}
where (i) holds due to $x^{\T}(A\hada B)y=\tr\bpar{\Diag(x)A\Diag(y)B^{\T}}$
(Lemma~\ref{lem:Hadamard}), (ii) follows from $P_{x}^{(2)}\preceq\Sigma_{x}$
(Claim~\ref{claim:schurProjection})\footnote{Even though this lemma is proven for leverage scores, the proof there
can be extended to any orthogonal projection matrices.}, (iii) uses $(a+b)^{2}\leq2\Par{a^{2}+b^{2}}$ for $a,b\in\R$ and
$\Sigma_{x}=\Diag(P_{x})=\sigma(D_{x}^{1/2}A_{x})$, and (iv) holds
due to $\sum_{i=1}^{m}[D_{x}]_{ii}\,(A_{x}h)_{i}^{2}=h^{\T}A_{x}^{\T}D_{x}A_{x}h=h^{\T}g(x)h$.

As for the second claim,
\begin{align*}
 & \max_{h:\norm h_{g(x)}=1}\norm{A_{x}h}_{\infty}=\max_{h}\max_{i\in[m]}\Abs{\frac{a_{i}^{\T}h}{s_{i}}}=\max_{i\in[m]}\max_{u:\norm u_{2}=1}\Abs{\frac{a_{i}^{\T}g(x)^{-1/2}u}{s_{i}}}\\
= & \max_{i\in[m]}\left\Vert g(x)^{-1/2}\frac{a_{i}}{s_{i}}\right\Vert _{2}=\max_{i\in[m]}\sqrt{\frac{1}{s_{i}^{2}}a_{i}^{\T}g(x)^{-1}a_{i}}=\sqrt{\max_{i\in[m]}e_{i}^{\T}A_{x}g(x)^{-1}A_{x}^{\T}e_{i}}=\sqrt{\max_{i\in[m]}\frac{[\sigma(D_{x}^{1/2}A_{x})]_{i}}{[D_{x}]_{ii}}}\,.
\end{align*}

As for the last claim, for $h\in\Rd$ such that $\norm{A_{x}h}_{\infty}\leq1$
(i.e., $h\in K\cap(2x-K)$ for $K=\{Ax\geq b\}$ due to Lemma~\ref{lem:symmforPolytope})
we have
\begin{align*}
h^{\T}g(x)h & =h^{\T}A_{x}^{\T}D_{x}A_{x}h=\sum_{i=1}^{m}(D_{x})_{ii}(A_{x}h)_{i}^{2}\leq\norm{A_{x}h}_{\infty}^{2}\sum_{i=1}^{m}(D_{x})_{ii}\leq\tr(D_{x})\,.\qedhere
\end{align*}
\end{proof}
Now we establish SSC and compute the symmetry parameters of metrics
of the form $A_{x}^{\T}D_{x}A_{x}$:
\begin{proof}
[Proof of Lemma~\ref{lem:paramsBarrier}] \textbf{Logarithmic barrier}:
To show that $g$ is SSC along $\rowspace(A)$, consider a self-concordant
matrix $g(y)=S_{y}^{-2}=-\nabla_{y}^{2}(\sum_{i=1}^{m}\log y_{i})$
defined on $\{y\in\R^{m}:y\geq0\}$. By putting $D_{x}=I_{m}$ and
$A_{x}=S_{x}^{-1}$ into Lemma~\ref{lem:helper4Diagonal}-1, since
$\sigma(A_{x})\leq1$
\[
\norm{g(x)^{-\half}\Dd g(x)[h]\,g(x)^{-\half}}_{F}\leq2\Bpar{\max_{i\in[m]}\sigma(A_{x})_{i}}^{1/2}\,\norm h_{g(x)}\leq2\norm h_{g(x)}\,.
\]
Through the linear map $Tx=Ax-b=y$, we recover $g(x)=\hess\phi_{\log}(x)=A^{\T}S_{y}^{-2}A=A_{x}^{\T}A_{x}$,
which is SSC along $\rowspace(A)$ by Lemma~\ref{lem:linear-trans-matrix}.
For the $\onu$-symmetry, the first part (i.e., $\dcal_{g}^{1}(x)\subset K\cap(2x-K)$)
follows from Lemma~\ref{lem:symmetricLeftpart}. The second part
is immediate from $\onu=\tr(I_{m})=m$ and Lemma~\ref{lem:helper4Diagonal}-3.

\textbf{Approximate volumetric barrier}: For $D_{x}=\Sigma_{x}=\Sigma(A_{x})$,
by Lemma~\ref{lem:usefulFactLewis}-1 and 3 with $p=2$,
\begin{align*}
\max_{i}\frac{[\sigma(D_{x}^{1/2}A_{x})]_{i}}{[D_{x}]_{ii}} & \leq2\sqrt{m}\,,\quad\text{and}\quad\sum_{i=1}^{m}[D_{x}^{-1}]_{ii}\,(\Dd D_{x}[h])_{i}^{2}=\norm{\Sigma_{x}^{-1}\diag(\Dd\Sigma_{x}[h])}_{\Sigma_{x}}^{2}\leq4\norm h_{g(x)}^{2}\,.
\end{align*}
Using Lemma~\ref{lem:helper4Diagonal}-1,
\begin{align*}
\norm{g(x)^{-\half}\Dd g(x)[h]\,g(x)^{-\half}}_{F}^{2} & \leq4\max_{i}\frac{[\sigma(D_{x}^{1/2}A_{x})]_{i}}{[D_{x}]_{ii}}\,\bpar{\norm h_{g(x)}^{2}+\sum_{i=1}^{m}[D_{x}^{-1}]_{ii}(\Dd D_{x}[h])_{i}^{2}}\leq40\sqrt{m}\norm h_{g(x)}^{2}\,.
\end{align*}
For the $\onu$-symmetry, $\norm{A_{x}(y-x)}_{\infty}^{2}\leq\max_{i\in[m]}\frac{[\sigma(D_{x}^{1/2}A_{x})]_{i}}{[D_{x}]_{ii}}\leq2m^{1/2}$
for $y\in\dcal_{g}^{1}(x)$ by Lemma~\ref{lem:helper4Diagonal}-2.
Also, Lemma~\ref{lem:helper4Diagonal}-3 implies that $y$ with $\norm{A_{x}(y-x)}_{\infty}\leq1$
is contained in $\dcal_{g}^{\sqrt{\tr(D_{x})}}(x)$, where $\tr(D_{x})=\tr(P_{x})\leq d$.
Therefore, $\tilde{g}(x):=40\sqrt{m}g(x)=40\sqrt{m}A_{x}^{\T}\Sigma_{x}A_{x}$
is SSC with the symmetry parameter $\onu=\mc O(\sqrt{m}d)$.

\textbf{Vaidya metric}: Consider the metric without scaling: $g(x):=A_{x}^{\T}D_{x}A_{x}$
with $D_{x}=\Sigma_{x}+\frac{d}{m}I_{m}$. Then, using \citet[(4.5)]{anstreicher1997volumetric}
in (i) below
\begin{align}
\max_{i}\frac{[\sigma(D_{x}^{1/2}A_{x})]_{i}}{[D_{x}]_{ii}} & \underset{\text{Lemma \ref{lem:helper4Diagonal}-2}}{=}\Bpar{\max_{h\in\Rd}\frac{\norm{A_{x}h}_{\infty}}{\norm h_{g(x)}}}^{2}\underset{\text{(i)}}{\leq}\sqrt{\frac{m}{d}}\,,\label{eq:28-1}\\
\sum_{i=1}^{m}[D_{x}^{-1}]_{ii}\,(\Dd D_{x}[h])_{i}^{2} & \underset{\text{(ii)}}{\leq}\sum_{i=1}^{m}[\Sigma_{x}^{-1}]_{ii}(\Dd\Sigma_{x}[h])_{i}^{2}\underset{\text{Lemma \ref{lem:usefulFactLewis}-3}}{\leq}4h^{\T}A_{x}^{\T}\Sigma_{x}A_{x}h\leq4\norm h_{g(x)}^{2}\,.\nonumber 
\end{align}
Putting these back to Lemma~\ref{lem:helper4Diagonal}-1,
\begin{align*}
\norm{g(x)^{-\half}\Dd g(x)[h]\,g(x)^{-\half}}_{F}^{2} & \leq4\max_{i}\frac{[\sigma(D_{x}^{1/2}A_{x})]_{i}}{[D_{x}]_{ii}}\,\bpar{\norm h_{g(x)}^{2}+\sum_{i=1}^{m}[D_{x}^{-1}]_{ii}(\Dd D_{x}[h])_{i}^{2}}\leq20\sqrt{\frac{m}{d}}\norm h_{g(x)}^{2}\,.
\end{align*}
Thus, $\tilde{g}(x):=22\sqrt{\frac{m}{d}}g(x)=22\sqrt{\frac{m}{d}}A_{x}^{\T}\bpar{\Sigma_{x}+\frac{d}{m}I_{m}}A_{x}$
is SSC. For the $\onu$-symmetry, Lemma~\ref{lem:helper4Diagonal}-2
implies that for $y\in\dcal_{g}^{1}(x)$,
\[
\norm{A_{x}(y-x)}_{\infty}^{2}\leq\max_{i}\frac{[\sigma(D_{x}^{1/2}A_{x})]_{i}}{[D_{x}]_{ii}}\underset{\text{\eqref{eq:28-1}}}{\leq}\sqrt{\frac{m}{d}}\,.
\]
Also, Lemma~\ref{lem:helper4Diagonal}-3 implies that $y$ with $\norm{A_{x}(y-x)}_{\infty}\leq1$
is contained in $\dcal_{g}^{\sqrt{\tr(D_{x})}}(x)$, where
\[
\tr(D_{x})=\tr\bpar{\Sigma_{x}+\frac{d}{m}I_{m}}=\tr(\Sigma_{x})+d\leq2d\,.
\]
Therefore, $\tilde{g}(x)$ satisfies $\dcal_{\tilde{g}}^{1}(x)\subset K\cap(2x-K)\subset\dcal_{\tilde{g}}^{\sqrt{44(md)^{1/2}}}(x)$,
so $\tilde{g}$ is $\mc O(\sqrt{md})$-symmetric.

\textbf{Lewis-weight metric}: Consider the unscaled version first:
$g(x)=A_{x}^{\T}W_{x}A_{x}$. By Lemma~\ref{lem:helper4Diagonal}-1
\begin{align*}
\norm{g(x)^{-\half}\Dd g(x)[h]\,g(x)^{-\half}}_{F}^{2} & \leq4\max_{i}\frac{[\sigma(W_{x}^{1/2}A_{x})]_{i}}{[W_{x}]_{ii}}\,\bpar{\norm h_{g(x)}^{2}+\sum_{i=1}^{m}[W_{x}^{-1}]_{ii}(\Dd W_{x}[h])_{i}^{2}}\\
 & \underset{\text{(i)}}{\leq}8m^{\frac{2}{p+2}}\bpar{\norm h_{g(x)}^{2}+p^{2}\,\norm h_{g(x)}^{2}}\leq\bpar{8m^{\frac{2}{p+2}}(1+p^{2})}\,\norm h_{g(x)}^{2}\,,
\end{align*}
where in (i) we used Lemma~\ref{lem:usefulFactLewis}-1 and 3.

For the first part of the $\onu$-symmetry, Lemma~\ref{lem:helper4Diagonal}-2
implies that
\[
\max_{h:\norm h_{g(x)}=1}\norm{A_{x}h}_{\infty}=\sqrt{\max_{i}\frac{[\sigma(W_{x}^{1/2}A_{x})]_{i}}{[W_{x}]_{ii}}}\leq\sqrt{2m^{\frac{2}{p+2}}}\,,
\]
 and Lemma~\ref{lem:helper4Diagonal}-3 leads to $K\cap(2x-K)\subset\dcal_{g}^{\sqrt{d}}(x)$
due to
\[
\tr(W_{x})=\tr\bpar{W_{x}^{\half-\frac{1}{p}}A_{x}(A_{x}^{\T}W_{x}^{1-\frac{2}{p}}A_{x})^{-1}A_{x}^{\T}W_{x}^{\half-\frac{1}{p}}}=\tr\bpar{A_{x}^{\T}W_{x}^{1-\frac{2}{p}}A_{x}(A_{x}^{\T}W_{x}^{1-\frac{2}{p}}A_{x})^{-1}}=d\,.
\]
Therefore, $16p^{2}m^{\frac{2}{p+2}}A_{x}^{\T}W_{x}A_{x}$ is SSC
with $\mc O\bpar{dm^{\frac{2}{p+2}}}$-symmetry by Lemma~\ref{lem:symmforPolytope}.
By setting $p=\mc O(\log m)$, the claim follows.
\end{proof}

\subsubsection{Linear constraints: strongly lower trace self-concordance of Vaidya
\label{proof:linear-vaidya-SLTSC}}

Let $\theta_{1}(x):=A_{x}^{\T}\Sigma_{x}A_{x}$, $\theta_{2}(x):=A_{x}^{\T}A_{x}$,
and $\Gamma_{x}:=\Diag\bpar{A_{x}g(x)^{-1}A_{x}^{\T}}$. Recall $g=g_{1}+g_{2}$
for a PSD matrix function $g_{1}$ and the Vaidya metric $g_{2}$.
\begin{lem}
\label{lem:HybridGammaNorm} $\norm{\Gamma_{x}}_{\infty}\leq\frac{1}{44}$.
\end{lem}

\begin{proof}
For $\og_{2}:=\theta_{1}+\frac{d}{m}\theta_{2}=\frac{1}{44}\sqrt{\frac{d}{m}}g_{2}$,
it follows from $g^{-1}\preceq g_{2}^{-1}=\frac{1}{44}\sqrt{\frac{d}{m}}\og_{2}^{-1}$
that
\begin{align*}
44\norm{\Gamma_{x}}_{\infty} & \leq4\sqrt{\frac{d}{m}}\norm{\Diag(A_{x}\og_{2}^{-1}A_{x}^{\T})}_{\infty}=\sqrt{\frac{d}{m}}\max_{i\in[m]}\frac{\bbrack{\sigma\bpar{\sqrt{\Sigma_{x}+\frac{d}{m}I_{m}}A_{x}}}_{i}}{\bbrack{\Sigma_{x}+\frac{d}{m}I_{m}}_{ii}}\underset{\text{\eqref{eq:28-1}}}{\leq}1\,.\qedhere
\end{align*}
\end{proof}
Now we show SLTSC of the Vaidya metric:
\begin{proof}
[Proof of Lemma~\ref{lem:vaidya-SLTSC}]  As $\Dd^{2}\theta_{2}(x)[h,h]\succeq0$
by Claim~\ref{claim:diffLogBarrier}, we have 
\[
\tr\bpar{g^{-1}\Dd^{2}\theta_{2}(x)[h,h]}=\tr\bpar{g^{-\half}\Dd^{2}\theta_{2}(x)[h,h]g^{-\half}}\geq0\,.
\]
As for $\theta_{1}$, by Lemma~\ref{lem:calculusLeverage}-6 $\Dd^{2}\theta_{1}[h,h]\succeq-16A_{x}^{\T}\Diag(S_{x,h}P_{x}S_{x,h}P_{x})A_{x}-6A_{x}^{\T}\Diag(P_{x}S_{x,h}^{2}P_{x})A_{x}$,
so
\[
\tr\bpar{g^{-1}\Dd^{2}\theta_{1}(x)[h,h]}\geq-16\tr(\Gamma_{x}S_{x,h}P_{x}S_{x,h}P_{x})-6\tr(\Gamma_{x}P_{x}S_{x,h}^{2}P_{x})\,.
\]
We first note that $\tr(S_{x,h}P_{x}S_{x,h})=s_{x,h}^{\T}(P_{x}\circ I)s_{x,h}=s_{x,h}^{\T}\Sigma_{x}s_{x,h}=\norm h_{\theta_{1}}^{2}$.
Using this,
\begin{align*}
\tr(\Gamma_{x}S_{x,h}P_{x}S_{x,h}P_{x}) & =\tr(\Gamma_{x}^{1/2}S_{x,h}P_{x}\cdot S_{x,h}P_{x}\Gamma_{x}^{1/2})\leq\sqrt{\tr(\Gamma_{x}^{\half}S_{x,h}P_{x}^{2}S_{x,h}\Gamma_{x}^{\half})\,\tr(\Gamma_{x}^{\half}P_{x}S_{x,h}^{2}P_{x}\Gamma_{x}^{\half})}\\
 & =\sqrt{\tr(P_{x}S_{x,h}\Gamma_{x}S_{x,h}P_{x})}\sqrt{\tr(S_{x,h}P_{x}\Gamma_{x}P_{x}S_{x,h})}=\norm{\Gamma_{x}}_{\infty}\norm h_{\theta_{1}}^{2}\,,\\
\tr(\Gamma_{x}P_{x}S_{x,h}^{2}P_{x}) & =\tr(S_{x,h}P_{x}\Gamma_{x}P_{x}S_{x,h})\leq\norm{\Gamma_{x}}_{\infty}\tr(S_{x,h}P_{x}S_{x,h})\underset{\text{(i)}}{=}\norm{\Gamma_{x}}_{\infty}\norm h_{\theta_{1}}^{2}\,.
\end{align*}
Putting these together and using Lemma~\ref{lem:HybridGammaNorm},
\[
\tr\bpar{g^{-1}\Dd^{2}\theta_{1}(x)[h,h]}\geq-22\norm{\Gamma_{x}}_{\infty}\norm h_{\theta_{1}}^{2}\geq-\half\,\norm h_{\theta_{1}}^{2}\,,
\]
and it follows from $g_{2}=44\sqrt{\frac{m}{d}}\Par{\theta_{1}+\frac{d}{m}\theta_{2}}$
that $\tr\bpar{g^{-1}\Dd^{2}g_{2}(x)[h,h]}\geq-\half\,\norm h_{g_{2}}^{2}$.
\end{proof}

\subsubsection{Linear constraints: strongly lower trace self-concordance of Lewis-weight
\label{proof:linear-Lewis-SLTSC}}

For $\theta(x):=A_{x}^{\T}W_{x}A_{x}$ (i.e., the unscaled version
of $g_{2}$), we write $g_{2}=c\cdot\theta$ for a constant $c$,
which will be set to $c_{1}(\log m)^{c_{2}}\sqrt{d}$ for some constants
$c_{1},c_{2}>0$ later. Going forward, $P_{x}$ indicates the projection
matrix of $W_{x}^{\nicefrac{1}{2}-\nicefrac{1}{p}}A_{x}$ (i.e., $P_{x}=P(W_{x}^{\nicefrac{1}{2}-\nicefrac{1}{p}}A_{x})$).
\begin{lem}
\label{lem:GammaNormLSMetric}$\norm{\Gamma_{x}}_{\infty}\leq2c^{-1}m^{\frac{2}{p+2}}$.
\end{lem}

\begin{proof}
Note that $0\preceq\Gamma_{x}=\Diag(A_{x}g^{-1}A_{x}^{\T})\preceq c^{-1}\Diag(A_{x}\theta^{-1}A_{x}^{\T})$.
By Lemma~\ref{lem:usefulFactLewis}-1,
\[
\norm{\Diag(A_{x}\theta^{-1}A_{x}^{\T})}_{\infty}=\max_{i\in[m]}\frac{\bbrack{\sigma\bpar{W_{x}^{1/2}A_{x}}}_{i}}{\bbrack{W_{x}}_{ii}}\leq2m^{\frac{2}{p+2}}\,.\qedhere
\]
\end{proof}
Now we show SLTSC of the Lewis-weight metric:
\begin{proof}
[Proof of Lemma~\ref{lem:Lw-SLTSC}] From \eqref{eq:LW-second-derv},
$\Dd^{2}\theta[h,h]\succeq-4A_{x}^{\T}W_{x,h}'S_{x,h}A_{x}+A_{x}^{\T}W_{x,h}''A_{x}$.
Thus,
\[
\tr(g^{-1}\Dd^{2}\theta[h,h])\geq\tr\bpar{\Gamma_{x}(W_{x,h}''-4W_{x,h}'S_{x,h})}=-4\tr(\Gamma_{x}W_{x,h}'S_{x,h})+\tr(\Gamma_{x}W_{x,h}'')\,.
\]
As for the first term, $\tr(\Gamma_{x}W_{x,h}'S_{x,h})\leq p\,\norm{\Gamma_{x}}_{\infty}\norm h_{\theta}^{2}$
follows from \eqref{eq:trSW} with $\Gamma_{x}$ replacing $s_{x,h}^{2}$.

As for the second term $\tr(\Gamma_{x}W_{x,h}'')$ (i.e., \eqref{eq:trGamma}
with $\Gamma=\Gamma_{x}$), each term there is of the form $\tr(\Gamma_{x}\Diag(v))$
for $v\in\R^{m}$, which can be bounded as follows:
\begin{align*}
\big|\tr\bpar{\Gamma_{x}\Diag(v)}\big| & =|\tr(\Gamma_{x}W_{x}^{\half}W_{x}^{-\half}\Diag(v))|\leq\sqrt{\tr(W_{x}^{\half}\Gamma_{x}^{2}W_{x}^{\half})}\sqrt{\tr\bpar{\Diag(v)W_{x}^{-1}\Diag(v)}}\\
 & \leq\norm{\Gamma_{x}}_{\infty}\sqrt{\tr(W_{x})}\norm v_{W_{x}^{-1}}=\sqrt{d}\norm{\Gamma_{x}}_{\infty}\norm v_{W_{x}^{-1}}\,.
\end{align*}
Then, we obtain $|\tr(\Gamma_{x}W_{x,h}'')|\lesssim\sqrt{d}\norm{\Gamma_{x}}_{\infty}\norm h_{\theta}^{2}$
for $p=\O(\log m)$ by using this inequality together with the norm
bounds in Lemma~\ref{lem:second-deriv-Lewis}. 

Putting things together, we conclude that
\begin{align*}
\tr(g^{-1}\Dd^{2}\theta[h,h]) & \gtrsim-p\norm{\Gamma_{x}}_{\infty}\norm h_{\theta}^{2}-\sqrt{d}\norm{\Gamma_{x}}_{\infty}\norm h_{\theta}^{2}\gtrsim-c^{-1}\sqrt{d}\norm h_{\theta}^{2}\,,
\end{align*}
where the last line follows from Lemma~\ref{lem:GammaNormLSMetric}.
Therefore, there exists positive constants $d_{1}$ and $d_{2}$ such
that $\tr(g^{-1}\Dd^{2}\theta[h,h])\geq-c^{-1}d_{1}(\log m)^{d_{2}}\sqrt{d}\norm h_{\theta}^{2}$,
which implies
\[
\tr(g^{-1}\Dd^{2}g_{2}[h,h])\geq-c^{-1}d_{1}(\log m)^{d_{2}}\sqrt{d}\norm h_{g_{2}}^{2}\,.
\]
By taking $c=d_{1}(\log m)^{d_{2}}\sqrt{d}$, the metric $g_{2}=c\theta=d_{1}(\log m)^{d_{2}}\sqrt{d}A_{x}^{\T}W_{x}A_{x}$
is SLTSC. 
\end{proof}


\subsubsection{Linear constraints: strongly average self-concordance}

We proceed with a general form of the metric $g(x)=A_{x}^{\T}D_{x}A_{x}$
with a diagonal matrix $0\prec D_{x}\in\R^{m}$. Then we provide computational
lemmas used when proving SASC of barriers for the linear constraints.

We pick any $g':\intk\to\psd$ such that $\bar{g}:=g+g'\succ0$. By
affine invariance, we may assume $\bar{g}(x)=I$ and $x=0$. Note
that $g(x)\preceq I_{d}$, and $z$ equals $rh/\sqrt{d}$ for $h\sim\mc N(0,I_{d})$
in law. Applying Taylor's expansion to $\norm{z-x}_{g(z)}^{2}$ at
$z=x$ (as in the proof of Lemma~\ref{lem:hsc-to-sasc}), for some
$p_{z}\in[x,z]$
\begin{align*}
\big|\norm{z-x}_{g(z)}^{2}-\norm{z-x}_{g(x)}^{2}\big| & \leq\frac{r^{2}}{d}\,\Bpar{\frac{r}{\sqrt{d}}\underbrace{|\Dd g(x)[h^{\otimes3}]|}_{\eqqcolon\textsf{A}}+\frac{r^{2}}{2d}\underbrace{|\Dd^{2}g(p_{z})[h^{\otimes4}]|}_{\eqqcolon\textsf{B}}}\,.
\end{align*}
It suffices to show that $|\Dd g(x)[h^{\otimes3}]|=\mc O(d^{1/2})$
and $|\Dd^{2}g(p_{z})[h^{\otimes4}]|=\mc O(d)$ with high probability.

\paragraph{Term $\textsf{A}$.}

By \eqref{eq:Dgh}, we have $\Dd g(x)[h^{\otimes3}]=-2s_{x,h}^{\T}D_{x}S_{x,h}s_{x,h}+s_{x,h}^{\T}D_{x,h}'s_{x,h}$.
Let $a_{i}$ denote the $i$-th row of $A_{x}$ for $i\in[m]$, and
define two polynomials in $h$ as follows:
\begin{equation}
P_{1}(h):=s_{x,h}^{\T}D_{x}S_{x,h}s_{x,h}=\tr(D_{x}S_{x,h}^{3})=\sum_{i=1}^{m}d_{i}\,(a_{i}^{\T}h)^{3}\,,\quad\text{and}\quad P_{2}(h):=s_{x,h}^{\T}D_{x,h}'s_{x,h}\,.\label{eq:P12}
\end{equation}
By Lemma~\ref{lem:matrix-projection}, $D_{x}^{1/2}A_{x}A_{x}^{\T}D_{x}^{1/2}\preceq P(D_{x}^{1/2}A_{x})$
and thus
\begin{equation}
\max_{i\in[m]}\norm{a_{i}}^{2}=\norm{\Diag(A_{x}A_{x}^{\T})}_{\infty}\leq\max_{i}\frac{[\sigma(D_{x}^{1/2}A_{x})]_{i}}{[D_{x}]_{ii}}\,.\label{eq:max-ai}
\end{equation}
By Lemma~\ref{lem:variance-1},
\begin{align}
\E[P_{1}(h)^{2}] & =\E\Bbrack{\Bbrace{\sum_{i=1}^{m}d_{i}(a_{i}\cdot h)^{3}}^{2}}=9\sum_{i,j=1}^{m}\norm{d_{i}^{1/3}a_{i}}^{2}\norm{d_{j}^{1/3}a_{j}}^{2}\inner{d_{i}^{1/3}a_{i},d_{j}^{1/3}a_{j}}+6\sum_{i,j}\inner{d_{i}^{1/3}a_{i},d_{j}^{1/3}a_{j}}^{3}\nonumber \\
 & =9\cdot1^{\T}\Diag(A_{x}A_{x}^{\T})\,D_{x}^{1/2}\underbrace{D_{x}^{1/2}A_{x}A_{x}^{\T}D_{x}^{1/2}}_{\preceq P(D_{x}^{1/2}A_{x})\preceq I_{m}}D_{x}^{1/2}\,\Diag(A_{x}A_{x}^{\T})\,1+6\sum_{i,j}d_{i}d_{j}(a_{i}\cdot a_{j})^{3}\nonumber \\
 & \lesssim\norm{\Diag(A_{x}A_{x}^{\T})}_{\infty}\,\tr\bpar{\Diag(A_{x}A_{x}^{\T})\,D_{x}}+\max_{i}\norm{a_{i}}^{2}\cdot\sum_{i,j}d_{i}d_{j}(a_{i}\cdot a_{j})^{2}\nonumber \\
 & =\max_{i}\norm{a_{i}}^{2}\,\tr(A_{x}^{\T}D_{x}A_{x})+\max_{i}\norm{a_{i}}^{2}\cdot\sum_{j}\tr(d_{j}a_{j}^{\T}A_{x}^{\T}D_{x}A_{x}a_{j})\nonumber \\
 & \underset{\text{(i)}}{\leq}2\max_{i}\norm{a_{i}}^{2}\,\tr(A_{x}^{\T}D_{x}A_{x})\leq2d\,\max_{i}\norm{a_{i}}^{2}\,,\label{eq:P1_bound}
\end{align}
where (i) follows from $A_{x}^{\T}D_{x}A_{x}\preceq I_{d}$ and $\sum_{j}\tr(d_{j}a_{j}^{\T}A_{x}^{\T}D_{x}A_{x}a_{j})\leq\sum_{j}\tr(d_{j}a_{j}^{\T}a_{j})=\tr(A_{x}^{\T}D_{x}A_{x})$.

Another polynomial $P_{2}(h)$ requires a different strategy for bounding
$\E[P_{2}(h)^{2}]$ for each barrier. This polynomial vanishes for
the log-barrier, while the Vaidya and Lewis-weight metrics requires
rather involved tasks for bounding $\E[P_{2}(h)^{2}]$.

\paragraph{Term $\textsf{B}$.}

Due to \eqref{eq:LW-fourth-moment} (with $W_{x}$ replaced by $D_{x}$),
$|\Dd^{2}g(p_{z})[h^{\otimes4}]|$ consists of three polynomials:
\begin{equation}
\bar{P}_{3}(h):=\tr(D_{p_{z}}S_{p_{z},h}^{4})\,,\quad\bar{P}_{4}(h)=\tr(D_{p_{z},h}'S_{p_{z},h}^{2})\,,\quad\bar{P}_{5}(h)=\tr(D_{p_{z},h}''S_{p_{z},h}^{2})\,.\label{eq:P345}
\end{equation}
For each $i=3,4,5$, we define $P_{i}(h)$ by $\bar{P}_{i}(h)$ with
$p_{z}$ replaced by $x$. For the log-barrier, $\bar{P}_{3}(h)$
only matters since $D_{(\cdot)}=I_{m}$. For the Vaidya metric, $\bar{P}_{4}(h)$
and $\bar{P}_{5}(h)$ can be bounded by multiples of $\bar{P}_{3}(h)$.
For the Lewis-weight metric, each $\bar{P}_{i}$ requires a different
procedure for bounding $\E[\bar{P}_{i}(h)^{2}]$. Moreover, we can
show $\bar{P}_{i}(h)\lesssim P_{i}(h)$ and
\begin{align}
\E[P_{3}(h)^{2}] & =\sum_{i,j\in[m]}\E[d_{i}d_{j}\,(a_{i}\cdot h)^{4}(a_{j}\cdot h)^{4}]\underset{\textup{CS}}{\leq}\sum_{i,j}d_{i}d_{j}\sqrt{\E[(a_{i}\cdot h)^{8}]}\sqrt{\E[(a_{j}\cdot h)^{8}]}\nonumber \\
 & \underset{\text{(i)}}{\lesssim}\Bpar{\sum_{i}d_{i}\norm{a_{i}}^{4}}^{2}\leq\max_{i}\norm{a_{i}}^{4}\,\Bpar{\sum_{i}d_{i}\norm{a_{i}}^{2}}^{2}\underset{\text{(ii)}}{\leq}d^{2}\max_{i}\norm{a_{i}}^{4}\,,\label{eq:P3_bound}
\end{align}
where we used $a_{i}\cdot h\sim\ncal(0,\norm{a_{i}}^{2})$ in (i),
and $\sum_{i}d_{i}\norm{a_{i}}^{2}=\tr(A_{x}^{\T}D_{x}A_{x})\leq\tr(I_{d})$
in (ii).

We now show SASC of the three barriers for linear constraints, using
this proof outline.

\paragraph{SASC of log-barriers.\label{proof:linear-SASC-log}}
\begin{proof}
[Proof of Lemma~\ref{lem:logBarrier-SASC}] Set $g(x)=A_{x}^{\T}A_{x}$
(with $D_{x}=I_{m}$). By \eqref{eq:max-ai}, 
\[
\max_{i\in[m]}\norm{a_{i}}^{2}\leq\max[\sigma(A_{x}^{1/2})]_{i}\leq1\,.
\]

As for the term $\msf A$, it suffices to bound $P_{1}(h)=\tr(S_{x,h}^{3})$.
Since $\E[P_{1}(h)^{2}]\lesssim d$ by \eqref{eq:P1_bound}, by Lemma~\ref{lem:conc-gaussian-poly}
with $t=(2e)^{3/2}\vee\bpar{\frac{2e}{3}\log\frac{2}{\veps}}^{3/2}$
and $r_{1}(\veps):=\veps(2\sqrt{60}t)^{-1}$, we have that for any
$r\leq r_{1}(\veps)$,
\[
\text{Event }B_{1}:\quad\P_{h}\Bpar{\frac{r}{\sqrt{d}}\,|P_{1}(h)|\geq\veps}\leq\veps\,.
\]

As for the term $\msf B$, recall $\P_{z}\bpar{\norm z\geq-r\cdot2\log\veps}\leq\veps$
and call this event $B_{2}$. We take $r_{2}(\veps)$ so that $1-2r_{2}\log\veps\leq1.1$,
which ensures $\norm z\leq2r$ conditioned on $B_{2}^{c}$ for $r\leq r_{2}$.
Next, we establish coordinate-wise closeness of $s_{x}$ at close-by
points. Let $x_{t}=x+\frac{tr}{\sqrt{d}}h$, and $s_{t}=Ax_{t}-b$.
For $t\in[0,1]$,
\begin{align*}
\left\Vert S_{0}^{-1}\,\frac{\D s_{t}}{\D t}\right\Vert _{\infty} & =\frac{r}{\sqrt{d}}\,\norm{A_{x}h}_{\infty}\leq\frac{r}{\sqrt{d}}\,\norm h_{g(x)}\leq\frac{r}{\sqrt{d}}\,\norm h=\norm z\,,
\end{align*}
and conditioned on $z\in B_{2}^{c}$ we know $\norm z\leq2r\log\frac{1}{\veps}\leq0.1$
for $r\leq r_{2}$. Hence,
\[
\max_{i\in[m]}\Big|\frac{s_{p,i}-s_{x,i}}{s_{x,i}}\Big|\leq\int_{0}^{1}\left\Vert S_{0}^{-1}\,\frac{\D s_{t}}{\D t}\right\Vert _{\infty}\,\D t\leq0.1\,,
\]
and thus $1.2\geq s_{x,i}/s_{p,i}\geq0.9$ for all $i\in[m]$ (i.e.,
$S_{p}^{-1}\preceq1.2S_{x}^{-1}$).

Using this, we bound $\bar{P}_{3}(h)=\tr(S_{p,h}^{4})$ by a multiple
of $P_{3}(h)=\tr(S_{x,h}^{4})$ as follows:
\begin{align*}
\tr(S_{p,h}^{4}) & =\tr(h^{\T}A^{\T}S_{p,h}S_{p}^{-2}S_{p,h}Ah)\leq2\tr(h^{\T}A^{\T}S_{p,h}S_{x}^{-2}S_{p,h}Ah)=2\tr(S_{x,h}^{2}S_{p,h}^{2})\leq4\tr(S_{x,h}^{4})\,.
\end{align*}
Hence, $\E[\bar{P}_{3}(h)^{2}]\lesssim\E[P_{3}(h)^{2}]\lesssim d^{2}$
by \eqref{eq:P3_bound}. Using Lemma~\ref{lem:conc-gaussian-poly}
with $t=(2e)^{2}\vee\bpar{\frac{2e}{4}\log\frac{2}{\veps}}^{3/2}$
and taking $r_{3}(\veps):=(\nicefrac{\veps}{c_{1}t})^{1/2}$, we obtain
\[
\text{Event }B_{3}:\quad\P\Bpar{\frac{r^{2}}{2d}\cdot16\bar{P}_{3}(h)\geq\veps}\geq\veps\,,
\]

Combining bounds on $\msf A$ and $\msf B$ conditioned on $\cap_{i}B_{i}^{c}$,
we have with probability at least $1-3\veps$
\[
\big|\norm{z-x}_{g(z)}^{2}-\norm{z-x}_{g(x)}^{2}\big|\leq2\veps\frac{r^{2}}{d}\quad\text{for any }r\leq\min_{i}r_{i}(\veps)\,.
\]
By replacing $3\veps\gets\veps$, the claim follows.
\end{proof}

\paragraph{SASC of Vaidya metric.\label{proof:linear-SASC-vaidya}}
\begin{proof}
[Proof of Lemma~\ref{lem:vaidya-SASC}] Set $g(x)=A_{x}^{\T}D_{x}A_{x}$
with $D_{x}=\sqrt{\frac{m}{d}}(\Sigma_{x}+\frac{d}{m}I_{m})$. By
\eqref{eq:max-ai} and \eqref{eq:28-1},
\[
\max_{i\in[m]}\norm{a_{i}}^{2}\leq\max_{i}\frac{[\sigma(D_{x}^{1/2}A_{x})]_{i}}{[D_{x}]_{ii}}\leq1\,.
\]

\paragraph{Term $\textsf{A}$.}

As $\msf A$ consists of $P_{1}$ and $P_{2}$ (see \eqref{eq:P12}),
we show $\E[P_{i}(h)^{2}]\lesssim d$ for $i\in[2]$, which by Lemma~\ref{lem:conc-gaussian-poly}
implies $|\msf A|\leq\sqrt{d}$ w.h.p. As for $P_{1}(h)=\tr(D_{x}S_{x,h}^{3})$,
we have $\E[P_{1}(h)]^{2}\lesssim d$ from \eqref{eq:P1_bound}.

As for $P_{2}(h)=\tr(D_{x,h}'S_{x,h}^{2})$, our approach is similar
to \citet{chen2018fast}. By Lemma~\ref{lem:calculusLeverage}, 
\begin{align*}
|P_{2}(h)| & =\Big|\sqrt{\frac{m}{d}}\tr\Bpar{\Diag\bpar{(\Sigma_{x}-P_{x}^{(2)})\,s_{x,h}}\,S_{x,h}^{2}}\Big|\leq|P_{1}(h)|+|\tr(S_{x,h}^{3})|+\sqrt{\frac{m}{d}}\,\big|\tr\bpar{\Diag(P_{x}^{(2)}s_{x,h})\,S_{x,h}^{2}}\big|\,.
\end{align*}
Since we already established a high-probability bound for both $|P_{1}(h)|$
and $|\tr(S_{x,h}^{3})|$ (which is $P_{1}(h)$ for the log-barrier),
we focus on the third term in the RHS.

For $\sigma_{x}:=\diag\Par{P_{x}}$ and $\sigma_{x,i,j}:=(P_{x})_{ij}$,
it follows from $P_{x}^{2}=P_{x}$ that $\sigma_{x,i}=\sum_{j}\sigma_{x,i,j}^{2}$.
Hence,
\begin{align*}
\tr(\Sigma_{x}S_{x,h}^{3}) & =1^{\T}\Sigma_{x}s_{x,h}^{3}=\sum_{i}(s_{x,h})_{i}^{3}\sigma_{x,i}=\sum_{i,j=1}^{m}\sigma_{x,i,j}^{2}(s_{x,h})_{i}^{3}\,,\\
\tr\bpar{\Diag(P_{x}^{(2)}s_{x,h})\,S_{x,h}^{2}} & =\sum_{i,j=1}^{m}\sigma_{x,i,j}^{2}(s_{x,h})_{i}^{2}(s_{x,h})_{j}\underset{\text{symmetry}}{=}\sum_{i,j=1}^{m}\sigma_{x,i,j}^{2}(s_{x,h})_{j}^{2}(s_{x,h})_{i}\,.
\end{align*}
Combining these leads to
\begin{align*}
 & 2\,\tr(\Sigma_{x}S_{x,h}^{3})+6\,\tr\bpar{\Diag(P_{x}^{(2)}s_{x,h})S_{x,h}^{2}}\\
 & =\sum_{i,j=1}^{m}\sigma_{x,i,j}^{2}\bpar{(s_{x,h})_{i}^{3}+3(s_{x,h})_{i}^{2}(s_{x,h})_{j}+3(s_{x,h})_{i}(s_{x,h})_{j}^{2}+(s_{x,h})_{j}^{3}}=\sum_{i,j=1}^{m}\sigma_{x,i,j}^{2}\bpar{(s_{x,h})_{i}+(s_{x,h})_{j}}^{3}\,,
\end{align*}
so we handle $\sum_{i,j}\sigma_{x,i,j}^{2}\bpar{(s_{x,h})_{i}+(s_{x,h})_{j}}^{3}$
instead of $\tr\bpar{\Diag(P_{x}^{(2)}s_{x,h})S_{x,h}^{2}}$, as we
already bounded $\sqrt{\frac{m}{d}}\tr(\Sigma_{x}S_{x,h}^{3})=P_{1}(h)-\sqrt{\frac{d}{m}}\tr(S_{x,h}^{3})$.
Due to $(s_{x,h})_{i}+(s_{x,h})_{j}=(a_{i}+a_{j})^{\T}h$, for $c_{ij}:=a_{i}+a_{j}$
\begin{align}
 & \E\Bbrack{\Bbrace{\sum_{i,j\in[m]}\sigma_{x,i,j}^{2}\bpar{(s_{x,h})_{i}+(s_{x,h})_{j}}^{3}}^{2}}=\sum_{i,j,k,l}\sigma_{x,i,j}^{2}\sigma_{x,k,l}^{2}\E[(c_{ij}\cdot h)^{3}(c_{kl}\cdot h)^{3}]\nonumber \\
 & \underset{\text{Lemma \ref{lem:variance-1}}}{=}9\sum_{i,j,k,l}\sigma_{x,i,j}^{2}\sigma_{x,k,l}^{2}\norm{c_{ij}}^{2}\norm{c_{kl}}^{2}(c_{ij}\cdot c_{kl})+6\sum_{i,j,k,l}\sigma_{x,i,j}^{2}\sigma_{x,k,l}^{2}(c_{ij}\cdot c_{kl})^{3}\,.\label{eq:vaidya-cubic-expansion}
\end{align}

As for the first term in \eqref{eq:vaidya-cubic-expansion}, we denote
$z_{i}:=\sum_{j}\sigma_{x,i,j}^{2}\|c_{ij}\|^{2}$ and $Z:=\Diag\bpar{(z_{i})_{i\in[m]}}$.
Then,
\begin{align}
 & \sum_{i,j,k,l}\sigma_{x,i,j}^{2}\sigma_{x,k,l}^{2}\norm{c_{ij}}^{2}\norm{c_{kl}}^{2}(c_{ij}\cdot c_{kl})=\Bnorm{\sum_{ij}\sigma_{x,i,j}^{2}\norm{c_{ij}}^{2}c_{ij}}^{2}\nonumber \\
 & \leq2\Bnorm{\sum_{ij}\sigma_{x,i,j}^{2}\norm{c_{ij}}^{2}a_{i}}^{2}+2\Bnorm{\sum_{ij}\sigma_{x,i,j}^{2}\norm{c_{ij}}^{2}a_{j}}^{2}=4\Bnorm{\sum_{ij}\sigma_{x,i,j}^{2}\norm{c_{ij}}^{2}a_{i}}^{2}=\Bnorm{\sum_{i}z_{i}a_{i}}^{2}\nonumber \\
 & =1^{\T}ZA_{x}A_{x}^{\T}Z\,1\le1^{\T}ZD_{x}^{-1/2}P(D_{x}^{1/2}A_{x})\,D_{x}^{-1/2}Z\,1\le1^{\T}ZD_{x}^{-1}Z\,1\lesssim\sqrt{\frac{d}{m}}\,\tr(Z)\,,\label{eq:trZ-bound}
\end{align}
where the last inequality follows from $Z\precsim\Sigma_{x}\preceq\sqrt{\frac{d}{m}}D_{x}$
due to
\begin{align*}
z_{i} & \leq2\sum_{j}\sigma_{x,i,j}^{2}(\staticnorm{a_{i}}^{2}+\staticnorm{a_{j}}^{2})\lesssim\underbrace{\sigma_{x,i}\staticnorm{a_{i}}^{2}+\sum_{j}\sigma_{x,i,j}^{2}\staticnorm{a_{j}}^{2}}_{\eqqcolon\msf K_{i}}\leq\sigma_{x,i}\|a_{i}\|^{2}+\sigma_{x,i}\lesssim\sigma_{x,i}\,.
\end{align*}
Moreover, using the bound in $\msf K_{i}$ and $\sum_{i,j}\sigma_{x,i,j}^{2}\snorm{a_{j}}^{2}=\sum_{j}\sigma_{x,i}\norm{a_{i}}^{2}$
\[
\tr(Z)\lesssim\sum_{i}(\sigma_{x,i}\staticnorm{a_{i}}^{2}+\sum_{j}\sigma_{x,i,j}^{2}\staticnorm{a_{j}}^{2})=2\tr(A_{x}^{\T}\Sigma_{x}A_{x})\lesssim\sqrt{\frac{d}{m}}\,\tr(A_{x}^{\T}D_{x}A_{x})\leq d\sqrt{\frac{d}{m}}\,.
\]
Putting this into \eqref{eq:trZ-bound}, we obtain $\sum_{i,j,k,l}\sigma_{x,i,j}^{2}\sigma_{x,k,l}^{2}\norm{c_{ij}}^{2}\norm{c_{kl}}^{2}(c_{ij}\cdot c_{kl})\lesssim d^{2}/m$.

As for the second term in \eqref{eq:vaidya-cubic-expansion},
\begin{align*}
 & \sum_{i,j,k,l}\sigma_{x,i,j}^{2}\sigma_{x,k,l}^{2}\,(c_{ij}\cdot c_{kl})^{3}\lesssim\sum_{i,j,k,l}\sigma_{x,i,j}^{2}\sigma_{x,k,l}^{2}\,|c_{ij}\cdot c_{kl}|^{2}\\
 & \leq\sum_{i,j,k,l}\sigma_{x,i,j}^{2}\sigma_{x,k,l}^{2}\,(a_{i}\cdot a_{k}+a_{i}\cdot a_{l}+a_{j}\cdot a_{k}+a_{j}\cdot a_{l})^{2}\lesssim\sum_{i,j,k,l}\sigma_{x,i,j}^{2}\sigma_{x,k,l}^{2}\,(a_{i}\cdot a_{k})^{2}\\
 & =\sum_{ik}\sigma_{i}\sigma_{k}\,(a_{i}\cdot a_{k})^{2}=\sum_{k}\tr(\sigma_{k}a_{k}^{\T}A_{x}^{\T}\Sigma_{x}A_{x}a_{k})\leq\sqrt{\frac{d}{m}}\sum_{k}\tr(\sigma_{k}a_{k}^{\T}a_{k})\\
 & =\sqrt{\frac{d}{m}}\,\tr(A_{x}^{\T}\Sigma_{x}A_{x})\le\frac{d^{2}}{m}\,.
\end{align*}
This establish a high-probability bound of $\O(d^{2}/m)$ on \eqref{eq:vaidya-cubic-expansion},
implying an $\O(\sqrt{d})$-high-probability bound on $\sqrt{\frac{m}{d}}\big|\tr\bpar{\Diag(P_{x}^{(2)}s_{x,h})\,S_{x,h}^{2}}\big|$.

\paragraph{Term $\textsf{B}$.}

We show that $s_{x}$ and $s_{p_{z}}$ are close, and the same holds
for $\sigma_{x}$ and $\sigma_{p_{z}}$. For $s_{x}$, following the
argument for the log-barrier, we let $x_{t}:=x+th\frac{r}{\sqrt{d}}$
and $s_{t}:=Ax_{t}-b$. For $0\leq t\leq1$,
\begin{align*}
\Bnorm{S_{0}^{-1}\deriv{s_{t}}t}_{\infty} & =\frac{r}{\sqrt{d}}\,\norm{A_{x}h}_{\infty}\underset{\eqref{eq:28-1}}{\leq}\frac{r}{\sqrt{d}}\,\norm h_{A_{x}^{\T}D_{x}A_{x}}\leq\frac{r}{\sqrt{d}}\,\norm h=\norm z\,.
\end{align*}
Conditioned on the high-probability bound of $\norm z\leq2r\log\frac{1}{\veps}\leq0.1$
for any $r$ less than some $r(\veps)$,
\[
\max_{i\in[m]}\Big|\frac{s_{p,i}-s_{x,i}}{s_{x,i}}\Big|\leq\int_{0}^{1}\Bnorm{S_{0}^{-1}\deriv{s_{t}}t}_{\infty}\,\D t\leq0.1\,,
\]
and thus $1.2\geq s_{x,i}/s_{p,i}\geq0.9$ for all $i\in[m]$ (i.e.,
$S_{p}^{-1}\preceq1.2S_{x}^{-1}$). For $\sigma_{x}$, as we have
$\Sigma_{x}=\Diag(A_{x}(A_{x}^{\T}A_{x})^{-1}A_{x}^{\T})$, we have
the same closeness between $\sigma_{x,i}$ and $\sigma_{p,i}$ for
each $i\in[m]$.

Using the formulas in Lemma~\ref{lem:calculusLeverage},
\begin{align*}
|\Dd^{2}g(p)[h^{\otimes4}]| & \lesssim\sqrt{\frac{m}{d}}\,\Bigl(\tr\bpar{(\Sigma_{p}+\frac{d}{m}I_{m})S_{p,h}^{4}}+\underbrace{\tr(S_{p,h}^{2}P_{p}S_{p,h}P_{p}S_{p,h})}_{(*)}\\
 & \qquad\qquad\qquad+\tr(S_{p,h}^{2}P_{p}S_{p,h}^{2}P_{p})+\underbrace{\tr(S_{p,h}P_{p}S_{p,h}P_{p}S_{p,h}P_{p}S_{p,h})}_{\leq\tr(S_{p,h}^{2}P_{p}S_{p,h}^{2}P_{p})}\Bigr)\\
 & \underset{\text{(i)}}{\lesssim}\sqrt{\frac{m}{d}}\,\Bpar{\tr\Bpar{(\Sigma_{p}+\frac{d}{m}I_{m})S_{p,h}^{4}}+\tr(S_{p,h}^{2}\Sigma_{p}S_{p,h}^{2})+\underbrace{\tr(S_{p,h}^{2}P_{p}S_{p,h}^{2}P_{p})}_{\text{Use Lemma \ref{lem:Kronecker}}}}\\
 & \underset{\text{(ii)}}{\lesssim}\sqrt{\frac{m}{d}}\,\tr\Bpar{(\Sigma_{p}+\frac{d}{m}I_{m})S_{p,h}^{4}}\underset{\text{(iii)}}{\lesssim}\sqrt{\frac{m}{d}}\,\tr\Bpar{(\Sigma_{x}+\frac{d}{m}I_{m})S_{x,h}^{4}}=P_{3}(h)\,,
\end{align*}
where in (i) we used the Cauchy-Schwarz inequality on $(*)$:
\begin{align*}
 & \tr(S_{p,h}^{2}P_{p}S_{p,h}P_{p}S_{p,h})\leq\sqrt{\tr(S_{p,h}^{2}P_{p}^{2}S_{p,h}^{2})}\sqrt{\tr(S_{p,h}P_{p}S_{p,h}^{2}P_{p}S_{p,h})}\\
 & \underset{\text{AM-GM}}{\leq}\half\bpar{\tr(S_{p,h}^{2}P_{p}^{2}S_{p,h}^{2})+\tr(S_{p,h}P_{p}S_{p,h}^{2}P_{p}S_{p,h})}\leq\half\,\bpar{\tr(S_{p,h}^{2}\Sigma_{p}S_{p,h}^{2})+\tr(S_{p,h}^{2}P_{p}S_{p,h}^{2}P_{p})}\,,
\end{align*}
(ii) follows from $\tr(S_{p,h}^{2}P_{p}S_{p,h}^{2}P_{p})=s_{p,h}^{2}\cdot P_{p}^{(2)}s_{p,h}^{2}\preceq s_{p,h}^{2}\cdot\Sigma_{p}s_{p,h}^{2}\preceq s_{p,h}^{2}\cdot(\Sigma_{p}+\frac{d}{m}I_{m})s_{p,h}^{2}$,
and in (iii) we used coordinate-wise closeness of $s_{x}\leftrightarrow s_{p}$
and $\sigma_{x}\leftrightarrow\sigma_{p}$. By \eqref{eq:P3_bound},
$\E[P_{3}(h)^{2}]\lesssim d^{2}$, and an $\O(d)$-high-probability
bound on $|P_{3}(h)|$ (so on $\msf B$) follows from Lemma~\ref{lem:conc-gaussian-poly}.
\end{proof}

\paragraph{SASC of Lewis-weight. \label{proof:linear-SASC-Lw}}
\begin{proof}
[Proof of Lemma~\ref{lem:Lw-SASC}] Set $g(x)=\sqrt{d}A_{x}^{\T}W_{x}A_{x}$
(with $D_{x}=\sqrt{d}\,W_{x}$). By \eqref{eq:max-ai} and Lemma~\ref{lem:usefulFactLewis}-1,
\[
\max_{i\in[m]}\norm{a_{i}}^{2}\leq\max_{i}\frac{[\sigma(D_{x}^{1/2}A_{x})]_{i}}{[D_{x}]_{ii}}\leq\frac{2m^{\frac{2}{p+2}}}{\sqrt{d}}\lesssim\frac{1}{\sqrt{d}}\,.
\]

\paragraph{Term $\textsf{A}$.}

As done for the Vaidya metric, a high-probability bound on $\msf A$
requires $\E[P_{i}(h)^{2}]\lesssim d$ for $i=1,2$ (see \eqref{eq:P12}).
Note that $\E[P_{1}(h)^{2}]\lesssim\sqrt{d}$ by \eqref{eq:P1_bound}.

As for $P_{2}(h)=\sqrt{d}\,s_{x,h}^{\T}W_{x,h}'s_{x,h}$, we show
$\E[P_{2}(h)^{2}]\lesssim\sqrt{d}$. Due to $W_{x,h}'=-\Diag(W_{x}^{\half}N_{x}W_{x}^{\half}s_{x,h})$
(Lemma~\ref{lem:DWh}), $P_{2}(h)=-\sqrt{d}s_{x,h}^{\T}\Diag(W_{x}^{\half}N_{x}W_{x}^{\half}s_{x,h})s_{x,h}=-\sqrt{d}\tr\bpar{\Diag(W_{x}^{\half}N_{x}W_{x}^{\half}s_{x,h})S_{x,h}^{2}}$.
Thus,
\begin{align*}
P_{2}(h) & =\sqrt{d}\,\tr\bpar{\Diag(N_{x}W_{x}^{\half}s_{x,h})W_{x}^{\half}S_{x,h}^{2}}=\sqrt{d}\sum_{i=1}^{m}w_{i}^{1/2}(a_{i}\cdot h)^{2}(b_{i}\cdot h)\,,
\end{align*}
where $b_{i}$ is the $i$-th row of $B:=N_{x}W_{x}^{\half}A_{x}$
for $i=1,\dots,m$. By Lemma~\ref{lem:variance-2},
\begin{align*}
 & \E\Bbrack{\Bbrace{\sum_{i=1}^{m}w_{i}^{1/2}(a_{i}\cdot h)^{2}(b_{i}\cdot h)}^{2}}\\
 & =\sum_{i,j\in[m]}w_{i}^{1/2}w_{j}^{1/2}\|a_{i}\|^{2}\|a_{j}\|^{2}(b_{i}\cdot b_{j})\\
 & \quad+4\sum_{i,j}w_{i}^{1/2}w_{j}^{1/2}(a_{i}\cdot a_{j})(a_{i}\cdot b_{i})(a_{j}\cdot b_{j})+4\sum_{i,j}w_{i}^{1/2}w_{j}^{1/2}\|a_{i}\|^{2}(b_{i}\cdot a_{j})(a_{j}\cdot b_{j})\\
 & \quad+2\underbrace{\sum_{i,j}w_{i}^{1/2}w_{j}^{1/2}(a_{i}\cdot a_{j})^{2}(b_{i}\cdot b_{j})}_{=:T_{1}}+4\underbrace{\sum_{i,j}w_{i}^{1/2}w_{j}^{1/2}(a_{i}\cdot a_{j})(a_{i}\cdot b_{j})(a_{j}\cdot b_{i})}_{=:T_{2}}\\
 & =\underbrace{1^{\T}\Diag(A_{x}A_{x}^{\T})\,W^{\half}BB^{\T}W^{\half}\,\Diag(A_{x}A_{x}^{\T})\,1}_{\eqqcolon N_{1}}+4\cdot\underbrace{1^{\T}\Diag(A_{x}B^{\T})\,W^{\half}A_{x}A_{x}^{\T}W^{\half}\,\Diag(A_{x}B^{\T})\,1}_{\eqqcolon N_{2}}\\
 & \quad+4\cdot\underbrace{[1^{\T}\Diag(A_{x}A_{x}^{\T})\,W^{\half}B]\cdot[A_{x}^{\T}W^{\half}\,\Diag(A_{x}B^{\T})\,1]}_{\le N_{1}+N_{2}\text{ by Young's inequality}}+2T_{1}+4T_{2}\,.
\end{align*}
As for $N_{1}$, since $B^{\T}B=A_{x}^{\T}W_{x}^{\half}N_{x}^{2}W_{x}^{\half}A_{x}\leq p^{2}A_{x}^{\T}W_{x}A_{x}$
by Lemma~\ref{lem:LS-comp-tool}-1 and thus $B^{\T}B\precsim(d)^{-1/2}I_{d}$,
Lemma~\ref{lem:matrix-projection} ensures $BB^{\T}\precsim\frac{1}{\sqrt{d}}P(B)\preceq\frac{1}{\sqrt{d}}\,I_{m}$.
Hence,
\begin{align*}
N_{1} & \lesssim\frac{1}{\sqrt{d}}\,\tr\bpar{\Diag(A_{x}A_{x}^{\T})\,W\,\Diag(A_{x}A_{x}^{\T})}\leq\frac{1}{\sqrt{d}}\,\tr(A_{x}^{\T}WA_{x})\,\norm{\Diag(A_{x}A_{x}^{\T})}_{\infty}\lesssim\frac{1}{\sqrt{d}}\,.
\end{align*}
As for $N_{2}$, due to $A_{x}^{\T}W_{x}A_{x}\preceq\frac{1}{\sqrt{d}}I_{d}$
we have $W^{\half}A_{x}A_{x}^{\T}W^{\half}\preceq\frac{1}{\sqrt{d}}I_{m}$
by Lemma~\ref{lem:matrix-projection}. Thus,
\begin{align*}
N_{2} & \lesssim\frac{1}{\sqrt{d}}\,\tr\bpar{\{\Diag(A_{x}B^{\T})\}^{2}}=\frac{1}{\sqrt{d}}\sum_{i\in[m]}(a_{i}\cdot b_{i})^{2}\leq\frac{1}{\sqrt{d}}\sum_{i}\|a_{i}\|^{2}\|b_{i}\|^{2}\\
 & \leq\frac{1}{d}\tr(BB^{\T})\lesssim\frac{1}{d^{3/2}}\tr\bpar{P(B)}\le\frac{1}{\sqrt{d}}\,.
\end{align*}
As for $T_{1}$, by Young's inequality (i.e., $2(a\cdot b)\leq\snorm a^{2}+\norm b^{2}$)
\begin{align*}
T_{1} & =\sum_{i,j\in[m]}(a_{i}\cdot a_{j})^{2}\,\bpar{(w_{j}^{1/2}b_{i})\cdot(w_{i}^{1/2}b_{j})}\lesssim\sum_{i,j}(a_{i}\cdot a_{j})^{2}\,(w_{j}\norm{b_{i}}^{2}+w_{i}\norm{b_{j}}^{2})\\
 & =2\sum_{i,j}w_{j}(a_{i}\cdot a_{j})^{2}\norm{b_{i}}^{2}=\sum_{i}\norm{b_{i}}^{2}\cdot\tr\Bpar{a_{i}^{\T}\Bpar{\sum_{j}a_{j}w_{j}a_{j}^{\T}}a_{i}}\\
 & =\sum_{i}\norm{b_{i}}^{2}\tr(a_{i}^{\T}A_{x}^{\T}WA_{x}a_{i})\leq\frac{1}{\sqrt{d}}\sum_{i}\norm{b_{i}}^{2}\norm{a_{i}}^{2}\leq\frac{1}{d}\tr(BB^{\T})\leq\frac{1}{\sqrt{d}}\,.
\end{align*}
As for $T_{2}$, using $(a_{i}\cdot a_{j})\leq\norm{a_{i}}\norm{a_{j}}\lesssim\frac{1}{\sqrt{d}}$
\begin{align*}
T_{2} & =\sum_{i,j\in[m]}w_{i}^{1/2}w_{j}^{1/2}(a_{i}\cdot a_{j})(a_{i}\cdot b_{j})(a_{j}\cdot b_{i})\lesssim\frac{1}{\sqrt{d}}\sum_{i,j\in[m]}w_{i}^{1/2}w_{j}^{1/2}(a_{i}\cdot b_{j})(a_{j}\cdot b_{i})\\
 & =\frac{1}{\sqrt{d}}\sum_{i}w_{i}^{1/2}b_{i}^{\T}\sum_{j}a_{j}w_{j}^{1/2}b_{j}^{\textbackslash T}a_{i}=\frac{1}{\sqrt{d}}\sum_{i}\tr(a_{i}w_{i}^{1/2}b_{i}^{\T}A_{x}^{\T}W^{1/2}B)\\
 & =\frac{1}{\sqrt{d}}\tr\bpar{(A_{x}^{\T}W^{1/2}B)^{2}}\underset{\text{CS}}{\leq}\frac{1}{\sqrt{d}}\tr(B^{\T}W^{1/2}A_{x}A_{x}^{\T}W^{1/2}B)\leq\frac{1}{d}\tr(B^{\T}B)\leq\frac{1}{\sqrt{d}}\,.
\end{align*}
Putting all the bounds together, we have $\E[P_{2}(h)^{2}]\lesssim d\cdot\frac{1}{\sqrt{d}}=\sqrt{d}$.

\paragraph{Term $\textsf{B}$.}

We show that for any given $\alpha=\Theta(1)$, each coordinate of
$w_{x}/s_{x}^{\alpha}$ and $w_{p_{z}}/s_{p_{z}}^{\alpha}$ is close.
For $0\leq t\le1$, we define $x_{t}:=x+\frac{r}{\sqrt{d}}th$, and
$s_{t},$ $w_{t}$ in the same fashion. Then for $p=\O(\log m)$,
\begin{align*}
\max_{i\in[m]}\Big|\log\frac{(w_{p_{z},i})^{\alpha}}{s_{p_{z},i}}-\log\frac{(w_{x,i})^{\alpha}}{s_{x,i}}\Big| & \leq\int_{0}^{1}\Big|\frac{\D}{\D t}\log\frac{[w_{t,i}]^{\alpha}}{s_{t,i}}\Big|\,\D t\lesssim\frac{r}{\sqrt{d}}\,\norm h_{A_{x}^{\T}W_{x}A_{x}}\leq\frac{1}{d^{1/4}}\norm z\,.
\end{align*}
Just as in showing SASC of the Vaidya metric, we can make this bound
arbitrarily small (say $\delta\approx0$) by conditioning on the high-probability
region where $\norm z\leq r\log\frac{1}{\veps}\leq0.01$. Hence,
\begin{equation}
e^{-\delta}\frac{(w_{x,i})^{\alpha}}{s_{x,i}}\leq\frac{(w_{p_{z},i})^{\alpha}}{s_{p_{z},i}}\leq e^{\delta}\frac{(w_{x,i})^{\alpha}}{s_{x,i}}\,.\label{eq:closeness}
\end{equation}
We remark that this $\Theta(1)$-multiplicative closeness is still
valid without the $\sqrt{d}$-scaling of $A_{x}^{\T}W_{x}A_{x}$.

Using the formula for $\Dd^{2}(A_{x}^{\T}W_{x}A_{x})[h^{\otimes4}]$
in \eqref{eq:LW-fourth-moment},
\begin{align*}
 & |\Dd^{2}g(p)[h^{\otimes4}]|\lesssim\bpar{\bar{P}_{3}(h)+|\bar{P}_{4}(h)|+|\bar{P}_{5}(h)|}=\bar{P}_{3}(h)+\sqrt{d}\,\bpar{|\tr(W_{p,h}'S_{p,h}^{3})|+|\tr(W_{p,h}''S_{p,h}^{2})|}\\
 & =\bar{P}_{3}(h)+\sqrt{d}\underbrace{\big|\tr\bpar{S_{p,h}^{3}\Diag(W_{p}^{\half}N_{p}W_{p}^{\half}s_{p,h})}\big|}_{\eqqcolon T_{1}}+\sqrt{d}\underbrace{|\tr(S_{p,h}^{2}W_{p,h}'')|}_{\eqqcolon T_{2}}\,,
\end{align*}
where in the last line we used the formula for $W_{p,h}'$ (Lemma~\ref{lem:DWh}).

Now we show $\E[\bar{P}_{3}(h)^{2}]\lesssim d^{2}$ and $T_{i}\lesssim\sqrt{d}$
w.h.p. for $i=4,5$. As for $\bar{P}_{3}$, we have $\bar{P}_{3}(h)\lesssim P_{3}(h)$
from the closeness \eqref{eq:closeness} of $w_{i}/s_{i}^{4}$ for
each $i\in[m]$, so $\E[P_{3}(h)^{2}]\lesssim d^{2}\cdot d^{-1}=d$
from \eqref{eq:P3_bound}.

As for $T_{1}$, using the Cauchy-Schwarz 
\begin{align*}
T_{1} & =\big|\tr\bpar{S_{p,h}^{3}W_{p}^{\half}\,\Diag(N_{p}W_{p}^{\half}s_{p,h})}\big|\leq\sqrt{\tr(S_{p,h}^{3}W_{p}S_{p,h}^{3})}\sqrt{s_{p,h}^{\T}W_{p}^{1/2}N_{p}^{2}W_{p}^{1/2}s_{p,h}}\\
 & \underset{\text{(i)}}{\lesssim}\sqrt{s_{p,h}^{3}W_{p}s_{p,h}^{3}}\sqrt{s_{p,h}^{\T}W_{p}s_{p,h}}\underset{\text{(ii)}}{\lesssim}\sqrt{s_{x,h}^{3}W_{x}s_{x,h}^{3}}\sqrt{s_{x,h}^{\T}W_{x}s_{x,h}}=\sqrt{s_{x,h}^{3}W_{x}s_{x,h}^{3}}\cdot d^{-1/4}\norm h_{g(x)}\,,
\end{align*}
where in (i) we used $N_{x}\preceq p^{2}I$ (Lemma~\ref{lem:LS-comp-tool}),
and in (ii) the closeness of $w_{i}/s_{i}^{6}$ and $w_{i}/s_{i}^{2}$
established in \eqref{eq:closeness}. As for the first term in the
RHS, 
\begin{align*}
\E[(s_{x,h}^{3}W_{x}s_{x,h}^{3})^{2}] & \underset{\text{CS}}{\lesssim}\sum_{i,j\in[m]}w_{i}w_{j}\sqrt{\E[(a_{i}\cdot h)^{12}]}\sqrt{\E[(a_{j}\cdot h)^{12}]}=\Bpar{\sum_{i}w_{i}\,\bpar{\E[(a_{i}\cdot h)^{12}]}^{2}}^{2}\\
 & \lesssim\Bpar{\sum_{i}w_{i}\norm{a_{i}}^{6}}^{2}\leq\Bpar{\frac{1}{d^{3/2}}\sum_{i}w_{i}}^{2}=\frac{1}{d}\,.
\end{align*}
As for the second term, the concentration of the standard Gaussian
guarantees $\norm h_{g(x)}\leq\norm h\lesssim\sqrt{d}$ w.h.p. Therefore,
$T_{1}\lesssim\sqrt{d}$ w.h.p.

As for $T_{2}$, \eqref{eq:trGamma} with $\Gamma_{p}=S_{p,h}^{2}$
equals $T_{2}$. Following \eqref{eq:last-bound} with I, II, III,
IV defined in \eqref{eq:LW-second-derv},
\begin{align*}
T_{2} & \lesssim\sum_{v=\text{I,II,III,IV}}\sqrt{\tr(W_{p}S_{p,h}^{4})}\norm v_{W_{p}^{-1}}\underset{\text{(i)}}{\lesssim}\sqrt{\tr(W_{p}S_{p,h}^{4})}\,\bpar{\tr(S_{p,h}^{2}W_{p})+\tr(S_{p,h}^{4}W_{p})}\\
 & \underset{\text{(ii)}}{\lesssim}\sqrt{\tr(W_{x}S_{x,h}^{4})}\,\bpar{\tr(S_{x,h}^{2}W_{x})+\tr(S_{x,h}^{4}W_{x})}\,,
\end{align*}
where (i) follows from Lemma~\ref{lem:second-deriv-Lewis} (i.e.,
$\norm v_{W_{p}^{-1}}\lesssim\norm h_{A_{p}^{\T}W_{p}A_{p}}^{2}=\tr(S_{p,h}^{2}W_{p})$
for $v=$ I, II, III, and $\norm{\text{IV}}_{W_{p}^{-1}}\lesssim\tr(S_{p,h}^{4}W_{p})$),
and (ii) follows from the conditioned event where the closeness of
$w_{i}/s_{i}^{2}$ at $x$ and $z$ holds. Since we already established
the high-probability bounds of $d^{-1/2}P_{3}(h)=\tr(S_{x,h}^{4}W_{x})\lesssim1$
and $\tr(S_{x,h}^{2}W_{x})\lesssim\sqrt{d}$, combining these yield
$T_{2}\lesssim\sqrt{d}$ w.h.p.
\end{proof}

\subsubsection{Quadratic constraints \label{proof:quadratic}}

We show that a $\nu$-SC barrier $\psi(\cdot)=-\log f(\cdot)$ satisfies
\[
|\Dd^{4}\psi(x)[h^{\otimes4}]|\lesssim\nu^{2}\norm h_{\hess\psi(x)}^{2}+\Big|\frac{\Dd^{4}f(x)[h^{\otimes4}]}{f(x)}\Big|\,.
\]

\begin{proof}
[Proof of Lemma~\ref{lem:4th-log}] Fix $h\in\Rd$ and $x\in\inter(K)$,
define $\phi(t):=\psi(x+th)$. Then,
\begin{align*}
\phi' & =-\frac{f'}{f}\,,\\
\phi'' & =\Par{\frac{f'}{f}}^{2}-\frac{f''}{f}=(\phi')^{2}-\frac{f''}{f}\,,\\
\phi''' & =2\phi'\phi''-\frac{f'''f-f''f'}{f^{2}}=2\phi'\phi''-\frac{f'''}{f}+\frac{f''f'}{f^{2}}=2\phi'\phi''+\phi'(\phi''-(\phi')^{2})-\frac{f'''}{f}\\
 & =3\phi'\phi''-(\phi')^{3}-\frac{f'''}{f}\,,\\
\phi^{(4)} & =3(\phi'')^{2}+3\phi'\phi'''-3(\phi')^{2}\phi''-\frac{f^{(4)}f-f'''f'}{f^{2}}\\
 & =3(\phi'')^{2}+3\phi'\phi'''-3(\phi')^{2}\phi''+\phi'\Par{\phi'''-3\phi'\phi''+(\phi')^{3}}-\frac{f^{(4)}}{f}\\
 & =3(\phi'')^{2}+4\phi'\phi'''-6(\phi')^{2}\phi''+(\phi')^{4}-\frac{f^{(4)}}{f}\,.
\end{align*}
Since $|\phi'''|\leq2(\phi'')^{3/2}$ (SC of $\phi$) and $\phi''\geq\frac{1}{\nu}(\phi')^{2}$
(the definition of the barrier parameter), which is equivalent to
$|\phi'|\leq\sqrt{\nu}(\phi'')^{1/2}$, we can directly compute as
follows:
\begin{align*}
|\phi^{(4)}| & \leq4\,|\phi'\phi'''|+3\,|(\phi'')^{2}|+6|\,(\phi')^{2}\phi''|+|(\phi')^{4}|+\Big|\frac{f^{(4)}}{f}\Big|\\
 & \leq8\sqrt{\nu}\,|\phi''|^{2}+3\,|\phi''|^{2}+6\nu\,|\phi''|^{2}+\nu^{2}\,|\phi''|^{2}+\Big|\frac{f^{(4)}}{f}\Big|\lesssim\nu^{2}|\phi''|^{2}+\Big|\frac{f^{(4)}}{f}\Big|\,.\qedhere
\end{align*}
\end{proof}
Using this tool, we study Dikin-amenability of barriers for quadratic
constraints.
\begin{proof}
[Proof of Lemma~\ref{lem:quadratic-const}] Let us check the last
claim first. By Lemma~\ref{lem:linear-trans}, we may assume that
\[
\phi(x,y)=-\log(l+q^{\T}y-\half\norm x^{2})\,,
\]
and let $f(x,y)=l+q^{\T}y-\half\,\norm x^{2}$. For $z=(x,y)\in\intk$
and $u=(u_{x},u_{y})\in\Rd$, we have 
\begin{align}
\Dd\phi(z)[u] & =-\frac{1}{f}\,(q\cdot u_{y}-x\cdot u_{x})=\frac{x\cdot u_{x}-q\cdot u_{y}}{f}\,,\nonumber \\
\Dd^{2}\phi(z)[u,u] & =\frac{1}{f^{2}}\,(x\cdot u_{x}-q\cdot u_{y})^{2}+\frac{1}{f}\,\norm{u_{x}}^{2}\,.\label{eq:hessian-quadratic}
\end{align}

As for the first term in the RHS of \eqref{eq:hessian-quadratic},
it holds that for $v=(v_{x},v_{y})\in\Rd$ 
\begin{align*}
\Dd\Bpar{\frac{(x\cdot u_{x}-q\cdot u_{y})^{2}}{f^{2}}}[v] & =\frac{2\,(x\cdot u_{x}-q\cdot u_{y})(v_{x}\cdot u_{x})}{f^{2}}+2\,(x\cdot u_{x}-q\cdot u_{y})^{2}\cdot\frac{x\cdot v_{x}-q\cdot v_{y}}{f^{3}}\,,\\
\Dd^{2}\Bpar{\frac{(x\cdot u_{x}-q\cdot u_{y})^{2}}{f^{2}}}[v,v] & =\frac{2\,(v_{x}\cdot u_{x})^{2}}{f^{2}}+4\frac{(x\cdot u_{x}-q\cdot u_{y})(v_{x}\cdot u_{x})(x\cdot v_{x}-q\cdot v_{y})}{f^{3}}\\
 & \quad+\frac{4\,(x\cdot u_{x}-q\cdot u_{y})(v_{x}\cdot u_{x})(x\cdot v_{x}-q\cdot v_{y})+2\,(x\cdot u_{x}-q\cdot u_{y})^{2}\norm{v_{x}}^{2}}{f^{3}}\\
 & \quad+\frac{6\,(x\cdot u_{x}-q\cdot u_{y})^{2}(x\cdot v_{x}-q\cdot v_{y})^{2}}{f^{4}}\\
 & =\frac{2\,(v_{x}\cdot u_{x})^{2}}{f^{2}}+\frac{4\,(x_{q}\cdot u)(v_{x}\cdot u_{x})(x_{q}\cdot v)}{f^{3}}\\
 & \quad+\frac{4\,(x_{q}\cdot u)(v_{x}\cdot u_{x})(x_{q}\cdot v)+2(x_{q}\cdot u)^{2}\norm{v_{x}}^{2}}{f^{3}}+\frac{6\,(x_{q}\cdot u)^{2}(x_{q}\cdot v)^{2}}{f^{4}}\,,
\end{align*}
where $x_{q}:=(x,-q)\in\Rd$.

As for the second term, direct computations lead to 
\begin{align*}
\Dd\Bpar{\frac{\norm{u_{x}}^{2}}{f}}[v] & =\frac{1}{f^{2}}\,\norm{u_{x}}^{2}(x\cdot v_{x}-q\cdot v_{y})\,,\\
\Dd^{2}\Bpar{\frac{\norm{u_{x}}^{2}}{f}}[v,v] & =\frac{2}{f^{3}}\,\norm{u_{x}}^{2}(x\cdot v_{x}-q\cdot v_{y})^{2}+\frac{1}{f^{2}}\,\norm{u_{x}}^{2}\norm{v_{x}}^{2}\\
 & =\frac{2}{f^{3}}\,\norm{u_{x}}^{2}(x_{q}\cdot v)^{2}+\frac{1}{f^{2}}\,\norm{u_{x}}^{2}\norm{v_{x}}^{2}\,.
\end{align*}
Putting these together, for $u,v\in\Rd$
\begin{align*}
 & \Dd^{4}\phi[u,u,v,v]\\
 & =\frac{1}{f^{2}}\,\norm{u_{x}}^{2}\norm{v_{x}}^{2}+\underbrace{\frac{2}{f^{2}}\,(v_{x}\cdot u_{x})^{2}}_{\geq0}+\frac{4}{f^{3}}\,\Bpar{\half\,\norm{u_{x}}^{2}(x_{q}\cdot v)^{2}+2\,(x_{q}\cdot u)(v_{x}\cdot u_{x})(x_{q}\cdot v)+\frac{(x_{q}\cdot u)^{2}}{2}\,\norm{v_{x}}^{2}}\\
 & \qquad+\frac{6}{f^{4}}\,(x_{q}\cdot u)^{2}(x_{q}\cdot v)^{2}\\
 & \geq\frac{4}{f^{3}}\,\bigg(\underbrace{\half\norm{u_{x}}^{2}(x_{q}\cdot v)^{2}+\frac{1}{2}\norm{v_{x}}^{2}(x_{q}\cdot u)^{2}}_{\text{Use AM-GM}}+2(x_{q}\cdot u)(v_{x}\cdot u_{x})(x_{q}\cdot v)\bigg)\\
 & \qquad+\underbrace{\frac{1}{f^{2}}\,\norm{u_{x}}^{2}\norm{v_{x}}^{2}+\frac{6}{f^{4}}\,(x_{q}\cdot u)^{2}(x_{q}\cdot v)^{2}}_{\text{Use AM-GM}}\\
 & \geq\frac{4}{f^{3}}\,\bpar{\norm{u_{x}}\,\norm{v_{x}}\,|x_{q}\cdot v|\,|x_{q}\cdot u|-2|x_{q}\cdot u|\,|x_{q}\cdot v|\,\norm{u_{x}}\,\norm{v_{x}}}+\frac{2\sqrt{6}}{f^{3}}\,|x_{q}\cdot u|\,|x_{q}\cdot v|\,\norm{u_{x}}\,\norm{v_{x}}\\
 & =\frac{4}{f^{3}}\,\norm{u_{x}}\,\norm{v_{x}}\,|x_{q}\cdot v|\,|x_{q}\cdot u|\,\Bpar{\frac{\sqrt{6}}{2}-1}\geq0\,.\qedhere
\end{align*}
\end{proof}


\subsubsection{PSD: convexity and strongly self-concordance \label{proof:psd-convex-ssc}}

We start with convexity of $\log\det(\hess\phi)$ for $\phi(X)=-\log\det X$.
\begin{proof}
[Proof of Proposition~\ref{prop:convex-logdet}] Using Lemma~\ref{prop:metricFormula}
and $\det\bpar{M^{\T}(A\otimes A)M}=2^{d(d-1)/2}\,(\det A)^{d+1}$
(Lemma~\ref{lem:Kronecker}) in the first and second equality below,
\begin{align*}
\log\det\bpar{\hess\phi(X)} & =\log\det\bpar{M^{\T}(X^{-1}\otimes X^{-1})M}=\frac{d(d-1)}{2}\,\log2-(d+1)\,\log\det X\,.
\end{align*}
Since $-\log\det X$ is convex in $X$ \eqref{eq:2ndDiffLogDet},
the convexity of $\log\det\bpar{\hess\phi(X)}$ also follows.
\end{proof}
Observe from the proof that $\log\det\bpar{\hess\phi(X)}=\text{const.}+(d+1)\,\phi(X)$.
Differentiating both sides in direction $H$, by \eqref{eq:gradLogDet}
$\tr\bpar{[\hess\phi(X)]^{-1}\Dd^{3}\phi(X)[H]}=(d+1)\,\Dd\phi(X)[H]$.
Hence,
\begin{align}
 & \tr\bpar{[\hess\phi(X)]^{-\half}\Dd^{3}\phi(X)[H]\,[\hess\phi(X)]^{-\half}}=-(d+1)\,\tr(X^{-1}H)\,.\label{eq:difflogdet}
\end{align}

We are ready to show SSC of $\phi$.
\begin{proof}
[Proof of Lemma~\ref{lem:logdet-scaling}] For $H\in\mbb S^{d}$
and $t\in\R$, denote $X_{t}:=X+tH$ and $g_{t}:=M^{\T}(X_{t}\otimes X_{t})^{-1}M$.
Note that
\[
\bnorm{[\hess\phi(X)]^{-\half}\Dd^{3}\phi(X)[H]\,[\hess\phi(X)]^{-\half}}_{F}^{2}=\tr(g^{-1}\del_{t}g_{t}\vert_{t=0}\,g^{-1}\del_{t}g_{t}\vert_{t=0})\,,
\]
and
\begin{align}
\del_{t}g_{t}\vert_{t=0} & \underset{\text{(i)}}{=}\del_{t}\bpar{M^{\T}(X_{t}\otimes X_{t})^{-1}M}\Big|_{t=0}\underset{\text{(ii)}}{=}-M^{\T}(X\otimes X)^{-1}\,\del_{t}(X_{t}\otimes X_{t})\vert_{t=0}\,(X\otimes X)^{-1}M\nonumber \\
 & =-M^{\T}(X^{-1}\otimes X^{-1})(H\otimes X+X\otimes H)(X^{-1}\otimes X^{-1})M\nonumber \\
 & \underset{\text{(iii)}}{=}-M^{\T}(X^{-1}HX^{-1}\otimes X^{-1}+X^{-1}\otimes X^{-1}HX^{-1})M\,,\label{eq:18-1}
\end{align}
where (i) follows from Lemma~\ref{prop:metricFormula}, (ii) is due
to \eqref{eq:diffInverse}, and (iii) follows from $(A\otimes B)(C\otimes D)=(AC)\otimes(BD)$
(Lemma~\ref{lem:Kronecker}-3).

Recall that positive semidefinite matrices have unique positive semidefinite
square roots, so $(X\otimes X)^{\half}=X^{\half}\otimes X^{\half}$
(due to $(X^{1/2}\otimes X^{1/2})\cdot(X^{1/2}\otimes X^{1/2})=X\otimes X$).
Since $g_{t}=M^{\T}(X_{t}\otimes X_{t})^{-1/2}(X_{t}\otimes X_{t})^{-1/2}M$,
the corresponding orthogonal projection is 
\[
P_{t}:=P\bpar{(X_{t}\otimes X_{t})^{-\half}M}=(X_{t}\otimes X_{t})^{-\half}Mg_{t}^{-1}M^{\T}(X_{t}\otimes X_{t})^{-\half}\,.
\]
 By substituting $\del_{t}g_{t}\big|_{t=0}$ with \eqref{eq:18-1},
\begin{align*}
 & \tr(g^{-1}\del_{t}g_{t}\vert_{t=0}\,g^{-1}\del_{t}g_{t}\vert_{t=0})\\
 & =\tr\bigl(g^{-1}M^{\T}(X^{-1}HX^{-1}\otimes X^{-1}+X^{-1}\otimes X^{-1}HX^{-1})M\\
 & \qquad\qquad\cdot g^{-1}M^{\T}(X^{-1}HX^{-1}\otimes X^{-1}+X^{-1}\otimes X^{-1}HX^{-1})\cblue M\bigr)\\
 & =\tr\bigl(\cblue Mg^{-1}M^{\T}(X^{-1}HX^{-1}\otimes X^{-1}+X^{-1}\otimes X^{-1}HX^{-1})M\\
 & \qquad\qquad\cdot g^{-1}M^{\T}(X^{-1}HX^{-1}\otimes X^{-1}+X^{-1}\otimes X^{-1}HX^{-1})\bigr)\\
 & =\tr\Bpar{\bbrack{\cred{Mg^{-1}M^{\T}}(X^{-1}HX^{-1}\otimes X^{-1}+X^{-1}\otimes X^{-1}HX^{-1})}^{2}}\\
 & =\tr\Bpar{\bbrack{\cred{(X\otimes X)^{\half}P(X\otimes X)^{\half}}(X^{-1}HX^{-1}\otimes X^{-1}+X^{-1}\otimes X^{-1}HX^{-1})}^{2}}\\
 & =\tr\Bpar{\bbrack{P\underbrace{(X\otimes X)^{\half}(X^{-1}HX^{-1}\otimes X^{-1}+X^{-1}\otimes X^{-1}HX^{-1})(X\otimes X)^{\half}}_{\eqqcolon S}}^{2}}\\
 & =\tr(PSPS)\,.
\end{align*}
Using Lemma~\ref{lem:Kronecker}-3,
\begin{align*}
S & =\underbrace{X^{-\half}HX^{-\half}\otimes I_{d}}_{\eqqcolon A}+\underbrace{I_{d}\otimes X^{-\half}HX^{-\half}}_{\eqqcolon B}\,.
\end{align*}
By the Cauchy-Schwarz inequality along with $P^{\T}P=P^{2}=P$ and
$P\preceq I_{d}$,
\begin{align*}
\tr(PSPS) & \leq\tr((PS)^{\T}PS)\leq\tr(S^{\T}S)=\norm S_{F}^{2}\leq(\norm A_{F}+\norm B_{F})^{2}\,.
\end{align*}
Using Lemma~\ref{lem:Kronecker}-3, 
\begin{align*}
\norm A_{F}^{2} & =\tr\bpar{(X^{-\half}HX^{-\half}\otimes I_{d})\cdot(X^{-\half}HX^{-\half}\otimes I_{d})}\\
 & =\tr(X^{-\half}HX^{-1}HX^{-\half}\otimes I_{d})=\tr(X^{-\half}HX^{-1}HX^{-\half})\,\tr(I_{d})=d\,\norm H_{X}^{2}\,,
\end{align*}
and similarly $\norm B_{F}^{2}=d\,\norm H_{X}^{2}$. Therefore, $\psi_{X}\leq2\sqrt{d}$
follows from
\[
\bnorm{[\hess\phi(X)]^{-\half}\Dd^{3}\phi(X)[H]\,[\hess\phi(X)]^{-\half}}_{F}\leq\sqrt{\tr(PSPS)}\leq2\sqrt{d}\,\norm H_{X}\,.
\]

To see the optimality of $\O(d^{1/2})$, we recall \eqref{eq:difflogdet}:
\[
\tr\bpar{[\hess\phi(X)]^{-\half}\Dd^{3}\phi(X)[H]\,[\hess\phi(X)]^{-\half}}=-(d+1)\,\tr(X^{-1}H)\,.
\]
Taking supremum on both sides,
\begin{align*}
\sup_{H:\norm H_{X}=1}\tr\bpar{[\hess\phi(X)]^{-\half}\Dd^{3}\phi(X)[H]\,[\hess\phi(X)]^{-\half}} & =\sup_{\substack{H\in\mbb S^{d}:\\
\snorm{X^{-1/2}HX^{-1/2}}_{F}=1
}
}-(d+1)\,\tr(X^{-\half}HX^{-\half})\\
 & =\sup_{S\in\mbb S^{d}:\norm S_{F}=1}(d+1)\,\tr(S)\,,
\end{align*}
and this objective achieves the maximum at $H=-d^{-1/2}X$, with the
supremum being $(d+1)\sqrt{d}$. On the other hand, due to $\tr(A)\leq d^{1/2}\,\norm A_{F}$
for $A\in\R^{d\times d}$,
\begin{align*}
 & \tr\bpar{[\hess\phi(X)]^{-\half}\Dd^{3}\phi(X)[H]\,[\hess\phi(X)]^{-\half}}\\
 & \leq\sqrt{\frac{d(d+1)}{2}}\cdot\bnorm{[\hess\phi(X)]^{-\half}\Dd^{3}\phi(X)[H]\,[\hess\phi(X)]^{-\half}}_{F}\leq\sqrt{\frac{d(d+1)}{2}}\cdot\psi_{X}\norm H_{X}\,,
\end{align*}
and thus by taking supremum on both sides over a symmetric matrix
$H$ with $\norm H_{X}=1$, it follows that $(d+1)\sqrt{d}\leq\sqrt{\frac{d(d+1)}{2}}\,\psi_{X}$
and 
\[
\sqrt{2(d+1)}\leq\psi_{X}\,.\qedhere
\]
\end{proof}

\subsubsection{PSD: strongly lower trace self-concordance \label{proof:psd-sltsc}}

Direct computation leads to $\Dd^{2}g(X)[H,H]\succeq0$ (so SLTSC).
\begin{proof}
[Proof of Lemma~\ref{lem:logdet-sltsc}] For $g(X)=-\hess\log\det X$,
recall that $g(X)[H,H]=\tr(X^{-1}HX^{-1}H)$. Thus for any $V\in\mbb S^{d}$,
\begin{align*}
\Dd g(X)[H,H,V] & =-\tr(X^{-1}VX^{-1}\cdot HX^{-1}H)-\tr(X^{-1}H\cdot X^{-1}VX^{-1}\cdot H)\\
 & =-2\,\tr(X^{-1}VX^{-1}HX^{-1}H)\,,
\end{align*}
and differentiating again,
\begin{align}
 & \Dd^{2}g(X)[H,H,V,V]\nonumber \\
 & =4\,\tr(X^{-1}VX^{-1}VX^{-1}HX^{-1}H)+2\,\tr(X^{-1}VX^{-1}HX^{-1}VX^{-1}H)\nonumber \\
 & =4\,\tr(X^{-\half}HX^{-1}VX^{-1}VX^{-1}HX^{-\half})+2\,\tr(X^{-\half}VX^{-1}HX^{-\half}\cdot X^{-\half}VX^{-1}HX^{-\half})\nonumber \\
 & \underset{\text{(i)}}{\geq}4\,\tr(X^{-\half}HX^{-1}VX^{-1}VX^{-1}HX^{-\half})-2\,\tr(X^{-\half}HX^{-1}VX^{-\half}\cdot X^{-\half}VX^{-1}HX^{-\half})\nonumber \\
 & =2\,\tr(X^{-\half}HX^{-1}VX^{-1}VX^{-1}HX^{-\half})\geq0\,,\label{eq:D4ph1}
\end{align}
where in (i) we used the Cauchy-Schwarz inequality. Therefore, $\Dd^{2}g(X)[H,H]\succeq0$.
\end{proof}

\subsubsection{PSD: average self-concordance \label{proof:psd-asc}}

We establish a connection to the Gaussian orthogonal ensemble (GOE):
for $d_{s}=d(d+1)/2$ and $\svec(H)\sim\ncal\bpar{0,\frac{r^{2}}{d_{s}}\,g(X)^{-1}}$,
we have $\frac{\sqrt{d_{s}d}}{r}X^{-\half}HX^{-\half}$ is the GOE.
\begin{proof}
[Proof of Lemma~\ref{lem:conn-to-goe}] Let $h_{X}:=\svec(X^{-1/2}HX^{-1/2})$
and $h:=\svec(H)$. It holds that
\[
h_{X}=L(X\otimes X)^{-\half}Mh
\]
due to $h_{X}=\svec(X^{-\half}HX^{-\half})=L\,\vec(X^{-\half}HX^{-\half})=L(X\otimes X)^{-\half}\vec(H)=L(X\otimes X)^{-\half}Mh$.
As $h\sim\ncal\bpar{0,\frac{r^{2}}{d_{s}}\,g(X)^{-1}}$, $h_{X}$
is a Gaussian with zero mean and covariance
\begin{align*}
 & \frac{r^{2}}{d_{s}}L(X\otimes X)^{-\half}Mg(X)^{-1}M^{\T}(X\otimes X)^{-\half}L^{\T}\\
\underset{\text{(i)}}{=} & \frac{r^{2}}{d_{s}d}L(X\otimes X)^{-\half}MLN(X\otimes X)N^{\T}L^{\T}M^{\T}(X\otimes X)^{-\half}L^{\T}\\
\underset{(*)}{=} & \frac{r^{2}}{d_{s}d}L(X\otimes X)^{-\half}N(X\otimes X)N^{\T}(X\otimes X)^{-\half}L^{\T}\\
\underset{(*)}{=} & \frac{r^{2}}{d_{s}d}L(X\otimes X)^{-\half}(X\otimes X)N(X\otimes X)^{-\half}L^{\T}\underset{\text{(*)}}{=}\frac{r^{2}}{d_{s}d}LNL^{\T}\\
\underset{\text{(ii)}}{=} & \frac{r^{2}}{d_{s}d}\,\left[\begin{array}{cc}
I_{d}\\
 & \half I_{d(d-1)/2}
\end{array}\right]\,,
\end{align*}
where (i) follows from Proposition~\ref{prop:metricFormula}, $(*)$
follows from Lemma~\ref{lem:MNL-properties}, and (ii) follows from
\citet[Page 427]{magnus1980elimination} that $LNL^{\T}$ is a $d_{s}\times d_{s}$
diagonal matrix with $d$ times $1$ and $\half d(d-1)$ times $1/2$.
Precisely, the entries of $h_{X}\in\R^{d_{s}}$ corresponding to the
diagonals of $X^{-1/2}HX^{-1/2}$ are $1$, and its entries corresponding
to off-diagonals is $1/2$. This is exactly the covariance matrix
of a $d_{s}$-dimensional GOE, so $X^{-\half}HX^{-\half}\sim\frac{r}{\sqrt{d_{s}d}}G$
for the GOE $G$.
\end{proof}
Now we show ASC of $d\phi$.
\begin{proof}
[Proof of Lemma~\ref{lem:logdet-asc}] Expand $\norm{Z-X}_{Z}^{2}:=\norm{Z-X}_{g(Z)}^{2}$
at $X$ for $Z=X+H$:
\[
\norm{Z-X}_{Z}^{2}-\norm{Z-X}_{X}^{2}=\sum_{k=1}^{\infty}\frac{1}{k!}\,\Dd^{k}g(X)[H^{\otimes k+2}]\,.
\]
It follows from induction that for $H_{X}:=X^{-\half}HX^{-\half}$
\begin{align*}
\Dd g(X)[H^{\otimes3}] & =-2d\,\tr(X^{-1}HX^{-1}HX^{-1}H)=-2\tr(H_{X}^{3})\,,\\
\Dd^{2}g(X)[H^{\otimes4}] & =3!\,d\,\tr(H_{X}^{4})\,,\\
\Dd^{k}g(X)[H^{\otimes(k+2)}] & =(-1)^{k}(k+1)!\,d\,\tr(H_{X}^{k+2})\,.
\end{align*}
Putting these back into the series expansion, for $H$ the GOE (see
Lemma~\ref{lem:conn-to-goe})
\begin{align*}
 & \norm{Z-X}_{Z}^{2}-\norm{Z-X}_{X}^{2}=\sum_{k=1}^{\infty}(-1)^{k}(k+1)d\,\tr(H_{X}^{k+2})\\
= & \sum_{k=1}^{\infty}(-1)^{k}(k+1)d\cdot\Bpar{\frac{r}{\sqrt{d_{s}d}}}^{k+2}\tr(H^{k+2})=\frac{r^{2}}{d_{s}}\sum_{k=1}^{\infty}(-1)^{k}(k+1)\,\Bpar{\frac{r}{\sqrt{d_{s}d}}}^{k}\tr(H^{k+2})\,.
\end{align*}

As for ASC, it suffices to show that $\sum_{k=1}^{\infty}(-1)^{k}(k+1)\bpar{\frac{r}{\sqrt{d_{s}d}}}^{k}\,\tr(H^{k+2})$
can be made arbitrarily small. We first control $\sum_{k\geq2}$:
\[
\Big|\sum_{k\geq2}(-1)^{k}(k+1)\Bpar{\frac{r}{\sqrt{d_{s}d}}}^{k}\tr(H^{k+2})\Big|\leq\sum_{k\geq2}(k+1)\Bpar{\frac{r}{\sqrt{d_{s}d}}}^{k}d\cdot\norm H_{\text{op}}^{k+2}\,.
\]
By \citet[Corollary 4.4.8]{vershynin2018high}, $\norm H_{\text{op}}\lesssim\sqrt{d}$
holds with high probability, and thus
\begin{align*}
\sum_{k\geq2}(k+1)\Bpar{\frac{r}{\sqrt{d_{s}d}}}^{k}d\cdot\norm H_{\text{op}}^{k+2} & \leq\sum_{k\geq2}(k+1)r^{k}\frac{1}{d^{3k/2}}d\cdot d^{\frac{k+2}{2}}\leq\sum_{k\geq2}(k+1)r^{k}d^{2-k}\,.
\end{align*}
By taking $r=\Omega(1)$ small enough, we can make this series arbitrarily
small.

Now we bound $\frac{r}{d^{3/2}}\tr(H^{3})$ ($k=1$ case). This is
a Gaussian polynomial in $\svec(H)$, so it suffices to show $\E[(\tr(H^{3}))^{2}]=\mc O(d^{3})$;
we then use Lemma~\ref{lem:conc-gaussian-poly} to obtain a high-probability
bound on the Gaussian polynomial $\frac{r}{d^{3/2}}\tr(H^{3})$. For
$H=(H_{ab})\in\mbb S^{d}$, 
\[
\bpar{\tr(H^{3})}^{2}=\sum_{ipq}H_{ip}H_{pq}H_{qi}\cdot\sum_{jrs}H_{jr}H_{rs}H_{sj}=\sum_{ipqjrs}H_{ip}H_{pq}H_{qi}H_{jr}H_{rs}H_{sj}\,,
\]
where each $H_{**}$ in the summand is an independent Gaussian with
zero mean and variance $1$ or $1/2$ (as $H$ is the GOE). We can
classify the indices $\{i,p,q,j,r,s\}$ into the following types:
\begin{align*}
6\text{ distinct indices } & \{a,b,c,d,e,f\}\,,\\
5\text{ distinct indices } & \{a,b,c,d,(e,e)\}\,,\\
4\text{ distinct indices } & \{a,b,c,(d,d,d)\},\{a,b,(c,c),(d,d)\}\,,\\
\text{Others } & \dots\,,
\end{align*}
where for example $\{a,b,c,d,e,f\}$ means all indices are different,
and $\{a,b,c,d,(e,e)\}$ means that there appear 5 different indices
$\{a,b,c,d,e\}$ but exists one pair $(e,e)$ of the same index. Note
that $\E H_{ip}H_{pq}H_{qi}H_{jr}H_{rs}H_{sj}=\mc O(1)$ is at most
the sixth moment of a standard Gaussian. It implies that toward our
goal of showing $\mc O(d^{3})$-bound on $\bpar{\tr(H^{3})}^{2}$,
it suffices to look into only three types of indices above. This is
because the terms from other types contribute at most $\mc O(d^{3})$
to $\bpar{\tr(H^{3})}^{2}$.
% Figure environment removed

For any term with 6 distinct indices, we can always find an `uncoupled'
$H_{**}$ (for example $H_{ab}$) in the summand that is independent
of all the others, so its expectation of the summand is $0$.

For the terms with $5$-distinct indices $\{a,b,c,d,(e,e)\}$, due
to symmetry (see Figure~\ref{fig:ipq-jrs}) we can further classify
the index $(i,p,q,j,r,s)$ into either $(a,b,c,d,e,e)$ or $(a,b,e,c,d,e)$.
In both cases , $H_{ab}$ has no coupled Gaussian, so the expectations
of the summand are also $0$. 

For $4$-distinct indices, let us first consider $\{a,b,c,(d,d,d)\}$-type
indices. In this case $(i,p,q,j,r,s)$ is of the form either $(a,a,a,b,c,d)$
or $(a,a,b,a,c,d)$ due to symmetry. In both cases, $H_{cd}$ has
no coupled Gaussian. Now consider $\{a,b,(c,c),(d,d)\}$-type indices.
Then $(i,p,q,j,r,s)$ is of the form either $(a,b,c,c,d,d)$ or $(a,c,c,b,d,d)$
or $(a,c,d,b,c,d)$. For each case, $H_{ab},H_{cc},H_{ac}$ are uncoupled
ones. Therefore, $\E[H_{ip}H_{pq}H_{qi}H_{jr}H_{rs}H_{sj}]=0$ whenever
there are at least $4$ distinct indices.
\end{proof}
\begin{rem}
\label{rem:challenge-extension-SASC}It seems challenging to show
that $\phi$ is SASC using the same technique. When $g$ is 
\[
g=d\,\hess(-\log\det X)+g'
\]
for other PSD matrix function $g'$, we know that $\svec(H_{X})=\svec(X^{-\half}HX^{-\half})$
follows a Gaussian distribution with zero mean and covariance matrix
$M$ satisfying
\[
M\preceq\left[\begin{array}{cc}
I_{d}\\
 & \half I_{d(d-1)/2}
\end{array}\right]\,.
\]
A main difference in the SASC setting is that the entries of $h=\svec(H_{X})$
might exhibit dependencies, making the previous approach infeasible.
This arises because many fundamental results in the random matrix
theory often presume independence of the entries of a random matrix.
Moreover, our combinatorial argument for the $k=1$ case is not feasible
in the presence of such dependencies.
\end{rem}


\subsection{Examples ($\S$\ref{sec:examples})}

\subsubsection{Algorithms for PSD sampling \label{proof:Algorithms-for-PSD}}
\begin{proof}
[Proof of Proposition~\ref{thm:hybridPSD}] We define $g_{X}=g=2(d^{2}g_{1}+g_{2})$,
where
\[
g_{1}(X)=M^{\T}(X\kro X)^{-1}M\qquad\text{and}\qquad g_{2}(X)=22\sqrt{\frac{m}{d}}\,M^{\T}A_{X}^{\T}\bpar{\Sigma_{X}+\frac{d}{m}I_{m}}A_{X}M\,.
\]
Since $d^{2}g_{1}$ and $g_{2}$ are SSC, $g$ is also SSC due to
Lemma~\ref{lem:ssc-sum} and $\mc O(d^{3}+\sqrt{md^{2}})$-symmetric\footnote{Since the dimension is $d_{s}$ in the PSD setting, we should replace
$d$ by $d_{s}=\O(d^{2})$ when applying Lemma~\ref{lem:paramsBarrier}.} due to Lemma~\ref{lem:symmetry-addition}. As $d^{2}g_{1}$ and
$g_{2}$ is SLTSC and SASC, $g$ is LTSC and ASC. Putting these together,
it follows that $g$ is $\bpar{\mc O(d^{3}+\sqrt{md^{2}}),\mc O(d^{3}+\sqrt{md^{2}})}$-Dikin-amenable.
Therefore, Theorem~\ref{thm:Dikin-annealing} implies that $\gcdw$
incurs $\otilde{d^{2}(d^{3}+\sqrt{md^{2}})}=\otilde{d^{3}(d^{2}+\sqrt{m})}$
total iterations of the $\dw$ with $g$.

Now we bound the per-step complexity of the $\dw$ (Algorithm~\ref{alg:DikinWalk}).
Recall that it requires (1) the update of the leverage scores, (2)
computation of the matrix function induced by the local metric $g$,
(3) the inverse of the matrix function and (4) its determinant. By
\citet[Theorem 46]{lee2019solving} (with $p=2$ and $d\gets d_{s}$
therein), the initialization of the leverage scores at the beginning
takes $\otilde{md^{2\omega}}$ and their updates takes $\otilde{md^{2(\omega-1)}}$
time. Since (1) takes $\otilde{md^{2(\omega-1)}}$, (2) takes $\otilde{d^{4}+md^{2(\omega-1)}}$,
and (3) and (4) take $\mc O\Par{d^{2\omega}}$, each iteration runs
in $\otilde{d^{2\omega}+md^{2(\omega-1)}}$ time. Even though the
initialization of leverage scores takes $\otilde{md^{2\omega}}$ time,
the amortized per-step time complexity becomes $\otilde{d^{2\omega}+md^{2(\omega-1)}}=\otilde{md^{2(\omega-1)}}$
time, as the mixing rate is $\otilde{d^{3}(d^{2}+\sqrt{m})}$.
\end{proof}
%
\begin{proof}
[Proof of Proposition~\ref{thm:LSPSD}] We define $g_{X}=g=2(d^{2}g_{1}+g_{2})$,
where for some constants $c_{1},c_{2}>0$,
\[
g_{1}(X)=M^{\T}(X\kro X)^{-1}M\qquad\text{and}\qquad g_{2}(X)=dc_{1}(\log m)^{c_{2}}M^{\T}A_{X}^{\T}W_{X}A_{X}M\,.
\]
Since $d^{2}g_{1}$ and $g_{2}$ are SSC, $g$ is also SSC due to
Lemma~\ref{lem:ssc-sum} and $\mc O^{*}(d^{3})$-symmetric due to
Lemma~\ref{lem:symmetry-addition}. As $d^{2}g_{1}$ and $g_{2}$
is SLTSC and SASC, $g$ is LTSC and ASC. Putting these together, it
follows that $g$ is $\bpar{\mc O^{*}(d^{3}),\mc O^{*}(d^{3})}$-Dikin-amenable.
Therefore, Theorem~\ref{thm:Dikin-annealing} implies that $\gcdw$
requires $\otilde{d^{5}}$ iterations of the $\dw$ with $g$. Since
the initialization and update of the Lewis weight takes $\otilde{md^{2\omega}}$
and $\otilde{md^{2(\omega-1)}}$ time \citet[Theorem 46]{lee2019solving},
the same implementation with Theorem~\ref{thm:hybridPSD} also has
the time complexity of $\otilde{md^{2(\omega-1)}}$.
\end{proof}

\subsubsection{Efficient implementation \label{proof:eff_implement}}
\begin{proof}
[Proof of Proposition~\ref{prop:oracle}] Let $v\in\R^{d_{s}}$
be a given vector, and denote $\bar{g}_{0}:=g_{1}$ and $\bar{g}_{i}:=\bar{g}_{i-1}+u_{i}u_{i}^{\T}$
for $i\in[m]$. We first prepare the column vectors $u_{i}$'s of
$U=M^{\T}A^{\T}S_{X}^{-1}$ in $\mc O(md^{2})$ time and then initialize
$\bar{g}_{0}^{-1}v$ and $\bar{g}_{0}^{-1}u_{i}$ for $i\in[m]$ in
$\mc O(md^{\omega})$ time. For $u_{i}$'s, note that $S_{X}$ can
be prepared in $\mc O(md^{2})$ time, and thus $A^{\T}S_{X}^{-1}$
takes $\mc O(md^{2})$ time due to $A\in\R^{d^{2}\times m}$. Since
each row of $M^{\T}\in\R^{d_{s}\times d^{2}}$ has at most two non-zero
entries, we can obtain $u_{i}$'s in $\mc O(md^{2})$ time.

For $\bar{g}_{0}^{-1}v$ and $\bar{g}_{0}^{-1}u_{i}$, we recall from
Lemma~\ref{prop:metricFormula} that for a vector $z\in\R^{d_{s}}$
\begin{align*}
g_{1}^{-1}z & =M^{\dagger}(X\kro X)(M^{\dagger})^{\T}z=LN(X\kro X)NL^{\T}z\,.
\end{align*}
Since each row of $L^{\T}\in\R^{d^{2}\times d_{s}}$ has at most two
non-zero entries, $w:=L^{\T}z\in\R^{d^{2}}$ can be computed in $\mc O(d^{2})$
time. From the definition of $N$, it follows that $Nw=\vec\bpar{\half(W+W^{\T})}$
for $W:=\vec^{-1}(w)\in\Rdd$, which also can be computed in $\mc O(d^{2})$
time. For $\overline{W}:=\half(W+W^{\T})$, it follows that
\[
(X\kro X)Nw=(X\kro X)\vec(\overline{W})\underset{\text{Lemma \ref{lem:Kronecker}-1}}{=}\vec(X\overline{W}X)\,,
\]
which can be computed in $\mc O(d^{\omega})$ time by the fast matrix
multiplication, and in a similar way we can compute $LN\,\vec(X\overline{W}X)$
in $\mc O(d^{2})$ time. Putting all these together, $\bar{g}_{0}^{-1}v$
can be computed in $\mc O(d^{\omega})$ time, and repeating this for
$u_{j}$'s yields $\{\bar{g}_{0}^{-1}v,\bar{g}_{0}^{-1}u_{1},\dots,\bar{g}_{0}^{-1}u_{m}\}$
in $\mc O(md^{\omega})$ time.

Starting with these initializations, we recursively use the Sherman--Morrison
formula: for $z\in\R^{d_{s}}$,
\begin{equation}
\bar{g}_{i}^{-1}z=\bar{g}_{i-1}^{-1}z-\frac{\bar{g}_{i-1}^{-1}u_{i}u_{i}^{\T}\bar{g}_{i-1}^{-1}z}{1+u_{i}^{\T}\bar{g}_{i-1}^{-1}u_{i}}\,.\label{eq:sherman-morrison}
\end{equation}
Using $\bar{g}_{i-1}^{-1}u_{j}$ and $\bar{g}_{i-1}^{-1}v$ from a
previous iteration, we can compute each of $\bar{g}_{i}^{-1}u_{j}$
and $\bar{g}_{i}^{-1}v$ in the current iteration in $\mc O(d^{2})$
time, and thus each round for update takes $\mc O(md^{2})$ time in
total. Since we iterate for $m$ rounds, Algorithm~\ref{alg:subroutine}
outputs $\bar{g}_{m}^{-1}v=g(X)^{-1}v$ in $\mc O(md^{\omega}+m^{2}d^{2})$
time.
\end{proof}
%
\begin{proof}
[Proof of Lemma~\ref{lem:perStep-small-m}] Here we provide details
of Algorithm~\ref{alg:perStep-small-m} in two stages -- (1) sampling
from $\ncal\bpar{0,\frac{r^{2}}{d}g(x)^{-1}}$ and (2) computation
of acceptance probability.

\paragraph{(1) Gaussian sampling:}

For simplicity, we ignore $r^{2}/d$ and illustrate how to draw $v\sim\ncal(0,g(X)^{-1})$
without full computation of $g(X)^{-1}$ in $\mc O(md^{\omega}+m^{2}d^{2})$
time.

Our approach is to compute $v:=g(X)^{-1}\left[\begin{array}{cc}
B & U\end{array}\right]w$ for $w\sim\ncal(0,I_{d^{2}+m})$, which follows the Gaussian distribution
with covariance
\begin{align*}
g(X)^{-1}\left[\begin{array}{cc}
B & U\end{array}\right]\Bpar{g(X)^{-1}\left[\begin{array}{cc}
B & U\end{array}\right]}^{\T} & =g(X)^{-1}(BB^{\T}+CC^{\T})g(X)^{-1}g(X)^{-1}\,,
\end{align*}
since $v$ is a linear transformation of the Gaussian random variable
$w$, and $BB^{\T}+CC^{\T}=g(X)$.

Denoting $w=(w_{b},w_{u})$ for $w_{b}\sim\ncal(0,I_{d^{2}})$ and
$w_{u}\sim\ncal(0,I_{m})$, we can show that $\left[\begin{array}{cc}
B & U\end{array}\right]w$ can be computed in $\mc O(d^{\omega}+md^{2})$ time as follows:
\begin{align*}
\left[\begin{array}{cc}
B & U\end{array}\right]w & =Bw_{b}+Uw_{c}=M^{\T}\underbrace{(X\kro X)^{-1/2}w_{b}}_{\text{Use Lemma \ref{eq:sherman-morrison}}}+M^{\T}A^{\T}S_{X}^{-1}w_{c}\\
 & =M^{\T}\Bpar{\vec\bpar{X^{-1/2}\vec^{-1}(w_{b})\,X^{-1/2}}+A^{\T}S_{X}^{-1}w_{c}}\,,
\end{align*}
where $\vec\bpar{X^{-1/2}\,\vec^{-1}(w_{b})\,X^{-1/2}}$ and $A^{\T}S_{X}^{-1}w_{u}$
can be computed in $\mc O(d^{\omega})$ and $\mc O(md^{2})$ time,
respectively. Since each row of $M^{\T}\in\R^{d_{s}\times d^{2}}$
has at most two non-zero entries, $\left[\begin{array}{cc}
B & U\end{array}\right]w$ can be computed in $\mc O(d^{\omega}+md^{2})$ time. Using Algorithm~\ref{alg:subroutine},
we obtain $v=g(X)^{-1}\left[\begin{array}{cc}
B & U\end{array}\right]w$ in $\mc O(md^{\omega}+m^{2}d^{2})$ time.

\paragraph{(2) Computation of acceptance probability. }

We show that this step also takes $\mc O(md^{\omega}+m^{2}d^{2})$
time. To compute $\det g(X)$, we use Algorithm~\ref{alg:subroutine}
to prepare $\{\bar{g}_{i}^{-1}u_{1},\dots,\bar{g}_{i}^{-1}u_{m}\}_{i=0}^{m}$
at $X$ and $Y=\svec^{-1}(y)$ in $\mc O(md^{\omega}+m^{2}d^{2})$
time. Recall the matrix determinant lemma:
\[
\det(A+uu^{\T})=(1+u^{\T}A^{-1}u)\,\det A\,.
\]
 Using the following recursive formula
\begin{align*}
\det(\bar{g}_{i+1}) & =\det(\bar{g}_{i}+u_{i+1}u_{i+1}^{\T})=(1+u_{i+1}^{\T}\bar{g}_{i}^{-1}u_{i+1})\,\det\bar{g}_{i}\,,
\end{align*}
we start with $\det\bar{g}_{0}=\det g_{1}=2^{d(d-1)/2}(\det X)^{-(d+1)}$
(see Lemma~\ref{lem:Kronecker}-7), which can be computed in $\mc O(d^{\omega})$
time, and compute $\det g(X)$ (and $\det g(Y)$ in the same way)
in $\mc O(md^{\omega}+m^{2}d^{2})$ time.
\end{proof}


\subsubsection{Handling approximate Lewis weights \label{proof:Handling-approximate-Lewis}}
\begin{proof}
[Proof of Lemma~\ref{lem:onestep-app-Lewis}] We just reproduce
the proof of Lemma~\ref{lem:one-step}. For $\pi\propto\exp(-f)\cdot\mathbf{1}_{K}$,
we denote 
\[
p_{x}=\ncal\Bpar{x,\frac{r^{2}}{d}g(x)^{-1}},\qquad R_{x}(z)=\frac{p_{z}(x)}{p_{x}(z)}\frac{\pi(z)}{\pi(x)},\qquad A_{x}(z)=\min\bpar{1,R_{x}(z)\,\mathbf{1}_{K}(z)}\,.
\]
Then the transition kernel of the $\dw$ started at $x$ can be written
as 
\begin{align*}
\widetilde{P}(x,dz) & =\underbrace{(1-\E_{p_{x}}[A_{x}(\cdot)])}_{=:r_{x}}\,\delta_{x}(\D z)+A_{x}(z)\,p_{x}(z)\,\D z\,.%\widetilde{P}(x,S)
\end{align*}
Thus, for $x,y\in\intk$ 
\begin{align*}
\dtv(P_{x},P_{y}) & =\underbrace{\frac{r_{x}+r_{y}}{2}}_{\textsf{I}}+\underbrace{\half\int|A_{x}(z)\,p_{x}(z)-A_{y}(z)\,p_{y}(z)|\,\D z}_{\textsf{II}}\,.
\end{align*}
\end{proof}
We note that $(1-\delta)\,\wt g_{2}\preceq g_{2}\preceq(1+\delta)\,\wt g_{2}$
and thus 
\begin{equation}
(1-\delta)\,\wt g\preceq g\preceq(1+\delta)\,\wt g\,,\label{eq:closeness-approx}
\end{equation}
and this implies $(1-\delta)\,I\preceq\wt g^{-1/2}g\wt g^{-1/2}\preceq(1+\delta)\,I$.
Hence, $(1-\delta)^{d^{2}/2}\leq\sqrt{\frac{\det g}{\det\wt g}}\leq(1+\delta)^{d^{2}/2}$
and 
\begin{align}
(1-\delta)^{d^{2}}\sqrt{\frac{\det\wt g(z)}{\det\wt g(x)}} & \leq\sqrt{\frac{\det g(z)}{\det g(x)}}\leq(1+\delta)^{d^{2}}\sqrt{\frac{\det\wt g(z)}{\det\wt g(x)}}\,.\label{eq:similar-ratio-approx}
\end{align}

With this in mind, recall that 
\[
r_{x}=1-\E_{p_{x}}[A_{x}(\cdot)]=1-\int\min\Bpar{1,\,\underbrace{\mathbf{1}_{K}(z)\frac{\exp(-f(z))}{\exp(-f(x))}}_{\eqqcolon\textsf{A}}\underbrace{\frac{p_{z}(x)}{p_{x}(z)}}_{\eqqcolon\textsf{B}}}\,p_{x}(z)\,\D z.
\]
We can bound $\textsf{A}$ in a similar way by using (\ref{eq:closeness-approx}).
As for $\textsf{B}$, 
\[
\log\text{\textsf{B}}=-\frac{d}{2r^{2}}(\snorm{z-x}_{z}^{2}-\snorm{z-x}_{x}^{2})+\half(\log\det\widetilde{g}(z)-\log\det\widetilde{g}(x))\,.
\]
As in Lemma~\ref{lem:one-step}, the second term can be bounded lower
by $\exp\Par{-3\veps}$ using (\ref{eq:similar-ratio-approx}). The
first term can be lower-bounded by invoking ASC of $g$. To see this,
ignoring the normalization constant of $g_{x}$ 
\begin{align*}
(*)= & \int\mathbf{1}\Bpar{\snorm{z-x}_{\widetilde{g}(z)}^{2}-\snorm{z-x}_{\widetilde{g}(x)}^{2}\leq2\veps\frac{r^{2}}{d}}\sqrt{\Abs{\widetilde{g}(x)}}\exp\bpar{-\half\snorm{z-x}_{\widetilde{g}(x)}^{2}}\,\D z\\
= & \int\mathbf{1}\Bpar{\snorm{z-x}_{\widetilde{g}(z)}^{2}-\snorm{z-x}_{\widetilde{g}(x)}^{2}\leq2\veps\frac{r^{2}}{d}}\sqrt{\Abs{g(x)}}\exp\bpar{-\half\snorm{z-x}_{g(x)}^{2}}\\
 & \qquad\cdot\sqrt{\Abs{\frac{\widetilde{g}(x)}{g(x)}}}\exp\bpar{-\half(\snorm{z-x}_{\widetilde{g}(x)}^{2}-\snorm{z-x}_{g(x)}^{2})}\,\D z\\
\leq & \int\mathbf{1}\Bpar{\snorm{z-x}_{\widetilde{g}(z)}^{2}-\snorm{z-x}_{\widetilde{g}(x)}^{2}\leq2\veps\frac{r^{2}}{d}}\sqrt{\Abs{g(x)}}\exp\bpar{-\half\snorm{z-x}_{g(x)}^{2}}\\
 & \qquad\cdot(1+\delta)^{d^{2}/2}\exp\bpar{\frac{\delta}{2}\snorm{z-x}_{g(x)}^{2}}\,\D z\,.
\end{align*}
Due to $\snorm{z-x}_{g(x)}^{2}\lesssim r^{2}$ w.h.p., taking $\delta=\veps/d^{10}$
leads to 
\[
(*)\leq2\int\mathbf{1}\Bpar{\snorm{z-x}_{\widetilde{g}(z)}^{2}-\snorm{z-x}_{\widetilde{g}(x)}^{2}\leq2\veps\frac{r^{2}}{d}}\sqrt{\Abs{g(x)}}\exp\bpar{-\half\snorm{z-x}_{g(x)}^{2}}\,\D z.
\]
Also, due to 
\begin{align*}
\snorm{z-x}_{\widetilde{g}(z)}^{2}-\snorm{z-x}_{\widetilde{g}(x)}^{2} & \geq(1-\delta)\,\snorm{z-x}_{g(z)}^{2}-(1+\delta)\,\snorm{z-x}_{g(x)}^{2}\\
 & =(1-\delta)\,(\snorm{z-x}_{g(z)}^{2}-\snorm{z-x}_{g(x)}^{2})-2\delta\,\snorm{z-x}_{g(x)}^{2}\,,
\end{align*}
we have 
\begin{align*}
(*) & \leq2\int\mathbf{1}\Bpar{\snorm{z-x}_{g(z)}^{2}-\snorm{z-x}_{g(x)}^{2}\leq(2\veps(1-\delta)^{-1}+\veps)\,\frac{r^{2}}{d}}\sqrt{\Abs{g(x)}}e^{-\half\snorm{z-x}_{g(x)}^{2}}\,\D z\leq6\veps
\end{align*}
by invoking ASC of $g$ in the last inequality. Putting these together,
$\msf I\leq\half+\mc O(\veps)$. For $\msf{II}$, we can follow the
proof of Lemma~\ref{lem:one-step} to show $\msf{II}\leq\frac{1}{4}+\mc O(\veps)$,
and every technical issue can be resolved by repeating the same techniques
above.

\begin{acknowledgement*}
This work was supported in part by NSF awards CCF-2007443 and CCF-2134105.
\end{acknowledgement*}
\bibliography{main} 


\appendix

\section{Algebraic identities}

Here we collect useful algebraic identities related to trace, vectorization,
Kronecker product and Hadamard product. 
 
\begin{lem}
[Kronecker product] \label{lem:Kronecker} For $A,B,C,D\in\R^{n\times n}$
and $M$ in Definition~\ref{def:linearOperators},
\begin{itemize}
\item $(A\otimes B)\vec(C)=\tr\Par{BCA^{\top}}$.
\item $\vec(A)^{\top}\Par{B\otimes C}\vec(D)=\tr\Par{DB^{\top}A^{\top}C}$.
\item $(A\otimes B)(C\otimes D)=AC\otimes BD$.
\item $(A\otimes B)^{-1}=A^{-1}\otimes B^{-1}$.
\item $(A\otimes B)^{\top}=A^{\top}\otimes B^{\top}$.
\item $\tr\Par{A\otimes B}=\tr(A)\tr(B)$.
\item $\det(M^{\top}(A\otimes A)M)=2^{n(n-1)/2}\Par{\det A}^{n+1}$.
\end{itemize}
\end{lem}

\begin{lem}
[Hadamard product] \label{lem:Hadamard} Let $A,B,C,D\in\R^{n\times n}$,
$x,y\in\Rn$, and $D_{1},D_{2}\in\Rnn$ be diagonal matrices.
\begin{itemize}
\item $(A\circ B)y=\diag(A\Diag(y)B^{\top})$.
\item $x^{\top}(A\circ B)y=\tr\Par{\Diag(x)A\Diag(y)B^{\top}}$.
\item $D_{1}(A\hada B)=(D_{1}A)\hada B=A\hada(D_{1}B)$.
\item $(A\hada B)D_{2}=(AD_{2})\hada B=A\hada(BD_{2})$.
\item $(A\otimes B)\circ(C\otimes D)=(A\circ C)\otimes(B\circ D)$.
\end{itemize}
\end{lem}


\section{Matrix calculus \label{app:matrixCalculus}}

Let $g(x):\Rn\to\R^{n\times n}$ be a matrix function. Its gradient
at $x$, denoted by $Dg(x)$, is the third-order tensor defined by
$(Dg(x))_{ijk}=\frac{\del g_{ij}(x)}{\del x_{k}}$. Unless specified
otherwise, the multiplication between higher-order tensors and a matrix
of size $n\times n$ is running over $i,j$-entries. For instance,
for a matrix $M\in\R^{n\times n}$ the product $Dg(x)M$ is the third-order
tensor defined by
\[
(Dg(x)M)_{\cdot,\cdot,k}=(Dg(x))_{\cdot,\cdot,k}M\text{ for each }k\in[n].
\]
In the same way, the trace of higher-order tensors is applied to a
matrix spanned by $i,j$-entries, i.e.,
\[
\Par{\tr\Par{Dg(x)}}_{k}=\tr\Par{\Par{Dg(x)}_{\cdot,\cdot,k}}.
\]

Now define $\vphi(x)=\log\det g(x):\Rn\to\R$. Its gradient is
\begin{equation}
\grad\vphi(x)=D\log\det g(x)=\tr\Par{g(x)^{-1}Dg(x)},\label{eq:gradLogDet}
\end{equation}
and this indicates that its directional derivative in $h\in\Rn$ is
$\grad\vphi(x)\cdot h=\tr\Par{g(x)^{-1}Dg(x)[h]}$. To compute the
Hessian of $\vphi$, we note that
\begin{equation}
D(g^{-1})(x)=-g(x)^{-1}Dg(x)g(x)^{-1}.\label{eq:diffInverse}
\end{equation}
Using this and the product rule, we have
\begin{align}
\hess\vphi(x) & =D\tr\Par{g(x)^{-1}Dg(x)}\nonumber \\
 & =-\tr\Par{g(x)^{-1}Dg(x)g(x)^{-1}Dg(x)}+\tr\Par{g(x)^{-1}D^{2}g(x)}\nonumber \\
 & =\tr\Par{g(x)^{-1}D^{2}g(x)}-\norm{g(x)^{-\half}Dg(x)g(x)^{-\half}}_{F}^{2},\label{eq:hessLogDet}
\end{align}
where $D^{2}g(x)$ is the fourth-order tensor defined by $(D^{2}g(x))_{ijkl}=\frac{\del g(x)_{ij}}{\del x_{k}\del x_{l}}$.

We now prove Lemma~\ref{lem:metricFormula}, providing formulas of
the Hessian and its inverse of $\phi(X)=-\log\det X$ for a matrix
$X\in\psd$.
\begin{proof}
[Proof of Lemma \ref{lem:metricFormula}] By setting $g(X)=X$ and
$\phi(X)=-\vphi(X)$ above, (\ref{eq:hessLogDet}) implies that for
a symmetric matrix $H\in\S^{n}$
\begin{align}
\hess\phi(X)[H,H] & =\tr\Par{X^{-1}HX^{-1}H}\label{eq:2ndDiffLogDet}\\
 & =\vec{(}H)^{\top}\Par{X^{-1}\otimes X^{-1}}\vec{(}H)=\vec{(}H)^{\top}\Par{X\otimes X}^{-1}\vec{(}H)\nonumber 
\end{align}
where the last line follows from Lemma~\ref{lem:Kronecker}. When
representing $X$ and $H$ in $\R^{d}$ space with notations $x:=\svec(X)$
and $h:=\svec(H)$, the definition of $M$ (see Definition~\ref{def:linearOperators})
turns the formula above into
\[
\hess\phi(x)[h,h]=h^{\top}M^{\top}(X\otimes X)^{-1}Mh,
\]
and thus $g_{X}:=\nabla_{x}^{2}\phi(x)=\nabla_{X}^{2}\phi(X)$ is
equal to $M^{\top}(X\otimes X)^{-1}M$. The formula of the inverse,
$g_{X}^{-1}=M^{\dagger}(X\otimes X)M^{\dagger\top}$, is immediate
from \cite{magnus1980elimination}, and another part follows from
$M^{\dagger}=LN$ and $N^{\top}=N$ (Lemma 3.6 and Lemma 2.1 in \cite{magnus1980elimination}).
\end{proof}

\section{Remaining proofs}

\subsection{Logarithmic barrier \label{app:subsec:logBarrier}}

Here we collect details used in the paper that involve calculus of
the logarithmic barrier, $\phi_{\log}(X):=-\sum_{i=1}^{m}\log\Par{\inner{A_{i},X}-b_{i}}$.
Recall the metric $g$ defined by the Hessian of $\phi_{\log}$ is
given by
\begin{align*}
g(X) & =M^{\top}\left[\begin{array}{ccc}
\vec(A_{1}) & \cdots & \vec(A_{m})\end{array}\right]S_{X}^{-2}\left[\begin{array}{c}
\vec(A_{1})^{\top}\\
\vdots\\
\vec(A_{m})^{\top}
\end{array}\right]M\\
 & =M^{\top}A^{\top}S_{X}^{-2}AM,
\end{align*}
where $S_{X}=\Diag\Par{\inner{A_{i},X}-b_{i}}\in\R^{m\times m}$ and
$A^{\top}=\left[\begin{array}{ccc}
\vec(A_{1}) & \cdots & \vec(A_{m})\end{array}\right]\in\R^{n^{2}\times m}$.   Since we work on $\S^{n}$ and $\R^{d}$ simultaneously, we
consider its vector version (i.e., $g(x)=A^{\top}S_{x}^{-2}A$ for
$x\in\R^{d}$) for simplicity and then translate it into one in our
setting. We recall notations that appear in our computation:
\begin{itemize}
\item $A_{x}=S_{x}^{-1}A\in\R^{m\times d}$.
\item $s_{x}=\diag(S_{x})\in\R^{m}$.
\item $s_{x,h}=A_{x}h\in\R^{m}$ and $S_{x,h}=\Diag(s_{x,h})\in\R^{m\times m}$.
We drop $x$ if $x$ is clear from the context.
\end{itemize}
Going forward, we use $h$ to denote a vector in $\R^{n}$.
\begin{claim}
\label{claim:1stDiffSlack} $DS_{x}[h]=\Diag(Ah)$ and $DS_{x}^{-1}[h]=-S_{x}^{-1}S_{x,h}$.
\end{claim}

\begin{proof}
The first is obvious from differentiation of $S_{x}=\Diag(Ax-b)$
with respect to $x$. For the second,
\begin{align*}
DS_{x}^{-1}[h] & =-S_{x}^{-1}DS_{x}[h]S_{x}^{-1}=-S_{x}^{-1}\Diag(Ah)S_{x}^{-1}\\
 & =-\Diag(A_{x}h)S_{x}^{-1}=-S_{x}^{-1}\Diag(A_{x}h)\\
 & =-S_{x}^{-1}S_{x,h},
\end{align*}
where $S_{x}^{-1}\Diag(Ah)=\Diag(A_{x}h)$ and $\Diag(A_{x}h)S_{x}^{-1}=S_{x}^{-1}\Diag(A_{x}h)$
hold as all of them are diagonal matrices.
\end{proof}
\begin{claim}
\label{claim:diffLogBarrier} $Dg(x)[h]=-2A_{x}^{\top}S_{x,h}A_{x}$
and $D^{2}g(x)[h,h]=6A_{x}^{\top}S_{x,h}^{2}A_{x}\succeq0$. In the
PSD setting with $A_{X}:=S_{X}^{-1}A$, this becomes $Dg(X)[H]=-2M^{\top}A_{X}^{\top}\Diag\Par{A_{X}\vec(H)}A_{X}M$
and $D^{2}g(X)[H,H]=6M^{\top}A_{X}^{\top}\Diag(A_{X}\vec(H))^{2}A_{X}M$.
\end{claim}

\begin{proof}
Due to $g(x)=A_{x}^{\top}A_{x}$,
\begin{align*}
Dg(x)[h] & =D(A^{\top}S_{x}^{-2}A)[h]=A^{\top}DS_{x}^{-2}[h]A\\
 & =-2A^{\top}S_{x}^{-3}DS_{x}[h]A=-2A_{x}^{\top}S_{x}^{-1}\Diag(Ah)A_{x}\\
 & =-2A_{x}^{\top}S_{x,h}A_{x}.
\end{align*}

For the second-order directional derivative,
\begin{align*}
Dg^{2}(x)[h,h] & =-2D(A_{x}^{\top}S_{x,h}A_{x})[h]=-2D(A^{\top}S_{x}^{-3}\Diag(Ah)A)[h]\\
 & =6A^{\top}S_{x}^{-4}DS_{x}[h]\Diag(Ah)A\\
 & =6A_{x}^{\top}S_{x,h}^{2}A_{x}.\qedhere
\end{align*}
\end{proof}
Note that $D^{2}g(x)[h,h]=6A_{x}^{\top}S_{x,h}^{2}A_{x}\succeq0$.
We now provide the deferred proof of Lemma~\ref{lem:paramsBarrier}-1.
\begin{proof}
[Proof of Lemma \ref{lem:paramsBarrier}-1] By putting $D_{x}=I_{m}$
into Lemma~\ref{lem:helper4Diagonal}-1, we have
\[
\norm{g(x)^{-\half}Dg(x)[h]g(x)^{-\half}}_{F}\leq2\sqrt{\max_{i\in[m]}\sigma(A_{x})_{i}}\norm h_{g(x)}.
\]
As $P_{x}$ is the orthogonal projection, $P_{x}\preceq I$ and $\sigma(A_{x})\leq1$.
Thus, $\norm{g(x)^{-\half}Dg(x)[h]g(x)^{-\half}}_{F}\leq2\norm h_{g(x)}$,
deriving strong self-concordance of the logarithmic barriers. 

For the $\onu$-symmetry, we note that the first part (i.e., $\dcal_{g}^{1}(x)\subset K\cap(2x-K)$)
follows from Lemma~\ref{lem:symmetricLeftpart}. The second part
is immediate from $\onu=\tr\Par{I_{m}}=m$ and Lemma~\ref{lem:helper4Diagonal}-3.
\end{proof}

\subsection{Volumetric barrier \label{app:subsec:volBarrier}}

\cite{vaidya1996new} introduced the \emph{volumetric barrier} for
a convex region $Ax\geq b$ defined by 
\[
\phi_{\vol}=\half\log\det\hess\phi_{\log}=\half\log\det A_{x}^{\top}A_{x}.
\]
We collect computational preliminaries regarding to the volumetric
barrier and then move onto the approximate volumetric barrier. 

\paragraph{Volumetric barrier.}
\begin{claim}
$\grad\phi_{\vol}(x)=-A_{x}^{\top}\sigma_{x}$ and $\hess\phi_{\vol}(x)=A_{x}^{\top}\Par{3\Sigma_{x}-2P_{x}^{(2)}}A_{x}$.
\end{claim}

\begin{proof}
Let $g(x)=\hess\phi_{\log}=A_{x}^{\top}A_{x}$. Note that by Claim~\ref{claim:diffLogBarrier}
\begin{align*}
\grad\phi_{\vol}(x)[h] & =-\tr\Par{g^{-1}A_{x}^{\top}S_{x,h}A_{x}}\\
 & =-\tr\bigg(\underbrace{A_{x}g^{-1}A_{x}^{\top}}_{=P_{x}}S_{x,h}\bigg)\\
 & =-\tr\Par{P_{x}S_{x,h}}=-\tr\Par{P_{x}S_{x,h}I_{m}I_{m}}\\
 & \underset{\text{(i)}}{=}-\bm{1}^{\top}(P_{x}\circ I_{m})S_{x,h}=-1^{\top}\Sigma_{x}A_{x}h\\
 & =-h^{\top}A_{x}^{\top}\sigma_{x},
\end{align*}
where we used Lemma~\ref{lem:Hadamard} in (i). 

For the Hessian of $\phi_{\vol}$,
\[
\hess\phi_{\vol}(x)[h,h]=\half\Par{\tr\Par{g^{-1}D^{2}g[h,h]}-\tr\Par{g^{-1}Dg[h]g^{-1}Dg[h]}}.
\]
For the first term, by Claim~\ref{claim:diffLogBarrier}
\begin{align*}
\half\tr\Par{g^{-1}Dg[h]g^{-1}Dg[h]} & =2\tr\Par{g^{-1}A_{x}^{\top}S_{x,h}A_{x}g^{-1}A_{x}^{\top}S_{x,h}A_{x}}=2\tr\Par{P_{x}S_{x,h}P_{x}S_{x,h}}\\
 & \underset{\text{(i)}}{=}2(A_{x}h)^{\top}\Par{P_{x}\circ P_{x}}(A_{x}h)=2h^{\top}A_{x}^{\top}P_{x}^{(2)}A_{x}h,
\end{align*}
where we used Lemma~\ref{lem:Hadamard} in (i). For the second term,
Claim~\ref{claim:diffLogBarrier} leads to
\begin{align*}
\half\tr\Par{g^{-1}D^{2}g[h,h]} & =3\tr\Par{g^{-1}A_{x}^{\top}S_{x,h}^{2}A_{x}}=3\tr\Par{P_{x}S_{x,h}IS_{x,h}}\\
 & =3h^{\top}A_{x}^{\top}\Par{P_{x}\circ I}A_{x}h=3h^{\top}A_{x}^{\top}\Sigma_{x}A_{x}h.
\end{align*}
Putting these two together, we have
\[
D^{2}\phi_{\vol}(x)[h,h]=h^{\top}A_{x}^{\top}\Par{3\Sigma_{x}-2P_{x}^{(2)}}A_{x}h
\]
and thus
\[
\hess\phi_{\vol}(x)=A_{x}^{\top}(3\Sigma_{x}-2P_{x}^{(2)})A_{x}.\qedhere
\]
\end{proof}
\begin{lem}
\label{lem:schurProjection} $P_{x}^{(2)}\preceq\Sigma_{x}$, so $A_{x}^{\top}\Sigma_{x}A_{x}\preceq\hess\phi_{\vol}(x)\preceq3A_{x}^{\top}\Sigma_{x}A_{x}$.
\end{lem}

\begin{proof}
Note that $\Sigma_{x}=P_{x}\circ I$. Let us show that $0\leq h^{\top}P_{x}\circ(I-P_{x})h$
for $h\in\Rn$. Since $P_{x}$ and $I-P_{x}$ are both orthogonal
projections matrices, for $C:=P_{x}H(I-P_{x})$ and $H=\Diag(h)$,
\begin{align*}
h^{\top}P_{x}\circ(I-P_{x})h & =\tr\Par{HP_{x}H(I-P_{x})}\\
 & =\tr\Par{(I-P_{x})HP_{x}P_{x}H(I-P_{x})}=\tr(C^{\top}C)\geq0.\qedhere
\end{align*}
\end{proof}


\paragraph{Approximate volumetric metric.}

The approximate volumetric metric is defined by
\[
g(x)=A_{x}^{\top}\Sigma_{x}A_{x},
\]
which serves as a good approximation of $\hess\phi_{\vol}$ due to
$\Sigma_{x}\preceq3\Sigma_{x}-2P_{x}^{(2)}\preceq3\Sigma_{x}$. We
begin with recalling computational results on the leverage scores:
\begin{lem}
[\cite{lee2019solving}] \label{lem:usefulFactLeverage} Let $\Sigma_{x}=\Sigma(A_{x})\in\R^{m\times m},g(x)=A_{x}^{\top}\Sigma_{x}A_{x}$,
and $h\in\Rn$.
\begin{itemize}
\item \textup{(Lemma 26)} $\max_{i\in[m]}\frac{\sigma\Par{\Sigma_{x}^{1/2}A_{x}}_{i}}{\Par{\Sigma_{x}}_{ii}}\leq2m^{\frac{1}{2}}$.
\item \textup{(Lemma 33)} $\norm{A_{x}h}_{\Sigma_{x}}=\norm h_{g(x)}$
and $\norm{A_{x}h}_{\infty}\leq\sqrt{2}m^{\frac{1}{4}}\norm h_{g(x)}$.
\item \textup{(Lemma 34)} $\norm{\Sigma_{x}^{-1}\diag\Par{D\Sigma_{x}[h]}}_{\Sigma_{x}}\leq2\norm h_{g(x)}$.
\end{itemize}
\end{lem}

We are now ready to prove the second item of Lemma \ref{lem:paramsBarrier}.
\begin{proof}
[Proof of Lemma~\ref{lem:paramsBarrier}-2] Let us set $D_{x}$
to $\Sigma_{x}=\Sigma(A_{x})$ in Lemma~\ref{lem:helper4Diagonal}.
By Lemma~\ref{lem:usefulFactLeverage}, we have
\begin{align*}
\max_{i}\Par{\frac{\sigma\Par{\sqrt{D_{x}}A_{x}}_{i}}{(D_{x})_{ii}}} & \leq2\sqrt{m},\\
\sum_{i=1}^{m}\Par{D_{x}^{-1}}_{ii}(DD_{x}[h])_{i}^{2} & =\norm{\Sigma_{x}^{-1}\diag\Par{D\Sigma_{x}[h]}}_{\Sigma_{x}}^{2}\\
 & \leq4\norm h_{g(x)}^{2}.
\end{align*}
Thus,
\begin{align*}
\norm{g(x)^{-\half}Dg(x)[h]g(x)^{-\half}}_{F}^{2} & \leq4\max_{i}\Par{\frac{\sigma\Par{\sqrt{D_{x}}A_{x}}_{i}}{(D_{x})_{ii}}}\cdot\Par{\norm h_{g(x)}^{2}+\sum_{i=1}^{m}\Par{D^{-1}}_{ii}(DD_{x}[h])_{i}^{2}}\\
 & \leq40\sqrt{m}\norm h_{g(x)}^{2}.
\end{align*}
For the symmetry parameter, $\norm{A_{x}(y-x)}_{\infty}\leq\sqrt{\max_{i\in[m]}\frac{\sigma\Par{\sqrt{D_{x}}A_{x}}_{i}}{D_{x,i}}}\leq m^{1/4}$
for $y\in\dcal_{g}^{1}(x)$ by Lemma~\ref{lem:helper4Diagonal}-2.
Also, Lemma~\ref{lem:helper4Diagonal}-3 implies that $y$ with $\norm{A_{x}(y-x)}_{\infty}\leq1$
is contained in $\dcal_{g}^{\sqrt{\tr(D_{x})}}(x)$, where
\[
\tr\Par{D_{x}}=\tr\Par{P_{x}}=\tr\Par{A_{x}(A_{x}^{\top}A_{x})^{-1}A_{x}}=\tr\Par{I_{n}}=n.
\]
Therefore, $\tilde{g}(x):=40\sqrt{m}g(x)=40\sqrt{m}A_{x}^{\top}\Sigma_{x}A_{x}$
is strongly self-concordant with the symmetry parameter $\onu=O(\sqrt{m}n)$.
\end{proof}

\subsection{Derivatives of matrices \label{app:subsec:derivativeMatrices}}

In this section, we provide details in computing derivatives of leverage
scores, orthogonal projections, and so on.

\propCalculusLeverage*
\begin{proof}
The first and second items follow from Lemma 2.2 and 2.3 of \cite{gatmiry2023sampling}.
From these formulas and the definition of $\Lambda_{x}$,
\begin{align*}
 & D\Lambda_{x}[h]\\
 & =D\Sigma_{x}[h]-DP_{x}[h]\circ P_{x}-P_{x}\circ DP_{x}[h]\\
 & =-2\Diag(\Lambda_{x}s_{x,h})-\Par{-P_{x}S_{x,h}-S_{x,h}P_{x}+2P_{x}S_{x,h}P_{x}}\circ P_{x}-P_{x}\circ\Par{-P_{x}S_{x,h}-S_{x,h}P_{x}+2P_{x}S_{x,h}P_{x}}\\
 & \underset{\text{(i)}}{=}-2\Diag(\Lambda_{x}s_{x,h})+2P_{x}\circ P_{x}S_{x,h}+2S_{x,h}P_{x}\circ P_{x}-2(P_{x}S_{x,h}P_{x})\circ P_{x}-2P_{x}\circ(P_{x}S_{x,h}P_{x}),
\end{align*}
where in (i) we used $D(A\hada B)=(DA)\circ B=A\hada(DB)$ and $(A\hada B)D=(AD)\hada B=A\circ(BD)$\footnote{This property allows us to write $DA\hada B$ without parenthesis.}
for a diagonal matrix $D\in\Rnn$ (Lemma~\ref{lem:Hadamard}). 

Using the first three formulas
\begin{align*}
 & D^{2}\Sigma_{x}[h,h]\\
 & =-2D\Diag(\Lambda_{x}s_{x,h})[h]\\
 & =-2\Diag(D\Lambda_{x}[h]s_{x,h})+2\Diag\Par{\Lambda_{x}S_{x,h}s_{x,h}}\\
 & =-2\Diag\Par{\Par{-2\Diag(\Lambda_{x}s_{x,h})+2P_{x}\circ P_{x}S_{x,h}+2S_{x,h}P_{x}\circ P_{x}-2(P_{x}S_{x,h}P_{x})\circ P_{x}-2P_{x}\circ(P_{x}S_{x,h}P_{x})}s_{x,h}}\\
 & \qquad+2\Diag\Par{\Lambda_{x}S_{x,h}s_{x,h}}\\
 & =4\Diag\Par{\cred{\Lambda_{x}}s_{x,h}}\cblue{S_{x,h}}-4\Diag\Par{P_{x}\circ P_{x}S_{x,h}s_{x,h}}-4\Diag\Par{S_{x,h}P_{x}\circ P_{x}s_{x,h}}\\
 & \qquad+4\Diag\Par{(P_{x}S_{x,h}P_{x})\circ P_{x}s_{x,h}}+4\Diag\Par{P_{x}\circ(P_{x}S_{x,h}P_{x})s_{x,h}}+2\Diag\Par{\cred{\Lambda_{x}}S_{x,h}s_{x,h}}\\
 & =4\Diag\Par{\cblue{S_{x,h}}\cred{(\Sigma_{x}-P_{x}\circ P_{x})}s_{x,h}}-4\Diag\Par{P_{x}\circ P_{x}S_{x,h}s_{x,h}}-4\Diag\Par{S_{x,h}P_{x}\circ P_{x}s_{x,h}}\\
 & \qquad+4\Diag\Par{(P_{x}S_{x,h}P_{x})\circ P_{x}s_{x,h}}+4\Diag\Par{P_{x}\circ(P_{x}S_{x,h}P_{x})s_{x,h}}+2\Diag\Par{\cred{(\Sigma_{x}-P_{x}\circ P_{x})}S_{x,h}s_{x,h}}\\
 & =\ccyan{4\Diag(S_{x,h}\Sigma_{x}s_{x,h})}-6\Diag\Par{P_{x}\circ P_{x}S_{x,h}s_{x,h}}-8\Diag\Par{S_{x,h}P_{x}\circ P_{x}s_{x,h}}\\
 & \qquad+4\Diag\Par{(P_{x}S_{x,h}P_{x})\circ P_{x}s_{x,h}}+4\Diag\Par{P_{x}\circ(P_{x}S_{x,h}P_{x})s_{x,h}}+\ccyan{2\Diag(\Sigma_{x}S_{x,h}s_{x,h})}\\
 & =\text{\ensuremath{\ccyan{6\Diag(S_{x,h}\Sigma_{x}s_{x,h})}}}-6\Diag\Par{\cblue{P_{x}\circ P_{x}S_{x,h}s_{x,h}}}-8\Diag\Par{\cblue{S_{x,h}P_{x}\circ P_{x}s_{x,h}}}\\
 & \qquad+4\Diag\Par{\cblue{(P_{x}S_{x,h}P_{x})\circ P_{x}s_{x,h}}}+4\Diag\Par{\cblue{P_{x}\circ(P_{x}S_{x,h}P_{x})s_{x,h}}}\\
 & \underset{\text{(i)}}{=}6S_{x,h}\Sigma_{x}\Diag\Par{s_{x,h}}-6\Diag\Par{\diag\Par{P_{x}S_{x,h}(P_{x}S_{x,h})^{\top}}}-8\Diag\Par{\diag\Par{S_{x,h}P_{x}S_{x,h}P_{x}^{\top}}}\\
 & \qquad+4\Diag\Par{P_{x}S_{x,h}P_{x}S_{x,h}P_{x}}+4\Diag\Par{P_{x}S_{x,h}\Par{P_{x}S_{x,h}P_{x}}^{\top}}\\
 & =6S_{x,h}\Sigma_{x}S_{x,h}-6\Diag\Par{P_{x}S_{x,h}^{2}P_{x}}-8\Diag\Par{S_{x,h}P_{x}S_{x,h}P_{x}}+8\Diag\Par{P_{x}S_{x,h}P_{x}S_{x,h}P_{x}},
\end{align*}
where in (i) we applied Lemma~\ref{lem:Hadamard}-1 to the terms
with blue. 

Applying the product rule to $\theta_{1}(x)=A_{x}^{\top}\Sigma_{x}A_{x}=A^{\top}S_{x}^{-2}\Sigma_{x}A,$
\begin{align*}
D\theta_{1}[h] & =-2A^{\top}S_{x}^{-3}\Sigma_{x}\Diag(Ah)A+A^{\top}S_{x}^{-2}D\Sigma_{x}[h]A\\
 & =-2A_{x}^{\top}\Sigma_{x}S_{x,h}A_{x}+A_{x}^{\top}D\Sigma_{x}[h]A_{x},\\
D^{2}\theta_{1}[h,h] & =6A_{x}^{\top}S_{x,h}\Sigma_{x}S_{x,h}A_{x}-2A_{x}^{\top}D\Sigma_{x}[h]S_{x,h}A_{x}-2A_{x}^{\top}S_{x,h}D\Sigma_{x}[h]A_{x}+A_{x}^{\top}D^{2}\Sigma_{x}[h,h]A_{x}\\
 & =6A_{x}^{\top}S_{x,h}\Sigma_{x}S_{x,h}A_{x}-4A_{x}^{\top}D\Sigma_{x}[h]S_{x,h}A_{x}+A_{x}^{\top}D^{2}\Sigma_{x}[h,h]A_{x}.
\end{align*}
The derivatives of $\theta_{2}$ simply follow from Claim \ref{claim:diffLogBarrier}.
\end{proof}


\end{document}
