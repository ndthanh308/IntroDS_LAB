
\section{Mixing of \texorpdfstring{$\dw$}{Dikin walks} \label{sec:mixing-Dikin}}

We follow a standard conductance based argument (see e.g., \citet{lovasz1993random,vempala2005geometric}).
A lower bound on the conductance of a Markov chain provides an upper
bound on the mixing time of the Markov chain due to the following
result.
\begin{lem}
[\citet{lovasz1993random}] \label{lem:conductanceBound} Let $\pi_{T}$
be the distribution obtained after $T$ steps of a lazy reversible
Markov chain of conductance at least $\Phi$ with stationary distribution
$\pi$ and initial distribution $\pi_{0}$. For $\snorm{\pi_{0}/\pi}=\E_{\pi_{0}}\bbrack{\deriv{\pi_{0}}{\pi}}$
and any $\veps>0$, we have $\dtv(\pi_{T},\pi)\leq\veps+\sqrt{\frac{\snorm{\pi_{0}/\pi}}{\veps}}\bpar{1-\frac{\Phi^{2}}{2}}^{T}$.
\end{lem}

A lower bound on the conductance follows from two ingredients: \textbf{(i)}
one-step coupling and \textbf{(ii)} isoperimetry. The first refers
to showing that the one-step distributions of the $\dw$ from two
nearby points have TV-distance bounded away from one. The second is
a purely geometry property about the expansion of the target distribution.
Combining these two leads to a lower bound on the conductance:
\begin{lem}
[\citet{kook2022condition}, Adapted from Proposition 9] \label{lem:conductance}
Let $\pi$ be the stationary distribution of a lazy reversible Markov
chain on $\mc M$ with a transition kernel $P_{x}$. Assume the isoperimetry
$\psi_{\mc M}$ under a Riemannian distance $d_{g}$ and the following
one-step coupling: if $\snorm{x-y}_{g(x)}\leq\Delta<1$ for $x,y\in\mc M$,
then $\dtv(P_{x},P_{y})\leq0.9$. Then the conductance $\Phi$ of
the Markov chain is bounded lower by $\Omega(\psi_{\mc M}\Delta)$.
\end{lem}


\subsection{One-step coupling and isoperimetry}

Recall that a $\onu$-Dikin-amenable metric is $\onu$-symmetric,
SSC, LTSC, and ASC. \citet{laddha2020strong} was the first to attempt
characterizing essential properties of $g$ (or $\phi$) that determine
mixing times of $\dws$ for uniform sampling. Their framework necessitates
that $g$ satisfies $\onu$-symmetric, SSC, convexity of $\log\det g(x)$,
and $x\in\mc D_{g}^{r}(z)$ w.h.p. (where $z\sim\text{Unif}\bpar{\dcal_{g}^{r}(x)}$). 

However, their framework encounters a challenge when further incorporating
the work of \citet{narayanan2016randomized}, which analyzes the $\dw$
for uniform sampling over a convex region given as the intersection
of various convex sets. The challenge arises from the difficulty of
verifying the convexity of $\log\det(g_{1}+g_{2})$ when $\log\det g_{i}$
is convex for each $i=1,2$.

To address this challenge and succinctly characterize essential characteristics
of a metric for one-step coupling, we relax the convexity of $\log\det$
to (S)LTSC and introduce the notion of ASC to account for the condition
``$x\in\mc D_{g}^{r}(z)$ w.h.p.''. We show that one-step coupling
lemma below, one of main proof ingredients in obtaining a mixing-time
guarantee of the $\dw$, can be established under Dikin-amenability
of a metric. Our characterization of a metric for achieving one-step
coupling is general and unifies previous work on $\dws$ (\citet{kannan2012random,narayanan2016randomized,chen2018fast,laddha2020strong}).

We now proceed to establish one-step coupling under the relative smoothness
in $\phi$.
\begin{lem}
[One-step coupling]\label{lem:one-step} For convex $K\subset\Rd$,
let $g:\intk\to\pd$ be SSC, ASC, LTSC, and $\phi:\intk\to\R$ be
its function counterpart. Suppose that the potential $f$ of the target
distribution $\pi$ is $\beta$-relatively smooth in $\phi$. Then
there exist constants $s_{1},s_{2}>0$ such that if $\snorm{x-y}_{g(x)}\leq s_{1}r/\sqrt{d}$
with $r=s_{2}\,(1\wedge\nicefrac{1}{\sqrt{\beta}})$ for $x,y\in\intk$,
then $\dtv(P_{x},P_{y})\leq\frac{3}{4}+0.01$. 
\end{lem}

We provide a sketch of the proof (see \S\ref{proof:onestep} for
the full proof). A key distinction when extending beyond uniform distributions
lies in establishing a lower bound for the ratio $\frac{\exp(f(x))}{\exp(f(z))}$
to ensure a high acceptance probability. To tackle this issue, we
use the symmetry of the proposal distribution, claiming $\nicefrac{\exp(f(x))}{\exp(f(z))}\geq1$
at the expense of $\texthalf$ probability. However, this $\texthalf$
probability loss is incompatible with previous proof techniques based
on the triangle inequality: for a transition kernel $T$ and proposal
kernel $P$, the triangle inequality leads to 
\[
\dtv(T_{x},T_{y})\leq\dtv(T_{x},P_{x})+\dtv(P_{x},P_{y})+\dtv(P_{y},T_{y})\,,
\]
and then bound the second term in the RHS by Pinsker's inequality,
making it arbitrarily small by taking $r=\O(1)$ small enough. However,
this approach yields a bound of $\texthalf+\veps$ for both $\dtv(T_{x},P_{x})$
and $\dtv(T_{y},P_{y})$, making the RHS vacuous.

We instead work with the exact formula for $\dtv(T_{x},T_{y})$: for
the Gaussian $p_{x}=\ncal(x,\frac{r^{2}}{d}g(x)^{-1})$, 
\[
R_{x}(z)=\frac{p_{z}(x)}{p_{x}(z)}\frac{\pi(z)}{\pi(x)}=\sqrt{\frac{\det g(z)}{\det g(x)}}\,\frac{\exp(f(x))}{\exp(f(z))},\qquad A_{x}(z)=\min\bpar{1,R_{x}(z)\,\mathbf{1}_{K}(z)}\,,
\]
the transition kernel $T_{x}$ of the $\dw$ started at $x$ can be
written as 
\[
T_{x}(dz)=\underbrace{\bpar{1-\E_{p_{x}}[A_{x}(\cdot)]}}_{\eqqcolon r_{x}}\,\delta_{x}(\D z)+A_{x}(z)\,p_{x}(\D z)\,.
\]
Then, 
\begin{align*}
\dtv(T_{x},T_{y}) & =\frac{r_{x}+r_{y}}{2}+\half\int|A_{x}(z)\,p_{x}(z)-A_{y}(z)\,p_{y}(z)|\,\D z\,.
\end{align*}

As for $r_{x}$ and $r_{y}$, we bound below $\sqrt{\nicefrac{\det g(z)}{\det g(x)}}$
by $1-\veps$ at the cost of $\veps$-probability through SSC, LTSC,
and ASC of $g$, following \citet{laddha2020strong} with convexity
of $\log\det$ replaced by LTSC. As mentioned earlier, we also deduce
$\nicefrac{\exp(f(x))}{\exp(f(z))}\geq1$ through the symmetry of
Gaussian distributions at the cost of $\texthalf$ probability. Combining
these results, we obtain upper bounds of $\texthalf+\veps$ for small
$\veps>0$ on $r_{x}$ and $r_{y}$.

Establishing a bound of $\nicefrac{1}{4}+\veps$ on the second term
is a more involved task. It requires the closeness of acceptance probabilities
$A_{x}(z)$ and $A_{y}(z)$ as well as the probability densities $g_{x}(z)$
and $g_{y}(z)$. This closeness can only be achieved through sophisticated
conditioning on high-probability events due to ASC, SSC, and symmetry
of Gaussian proposals. To be precise, define good events $G_{x}=\cap_{i=0,2,3}B_{x,i}^{c}$
and $G_{y}=\cap_{i=0,2,3}B_{y,i}^{c}$ such that $\P_{\ncal_{g}^{r}(x)}(G_{x}^{c})\leq3\veps$
and $\P_{\ncal_{g}^{r}(y)}(G_{y}^{c})\leq3\veps$, where 
\begin{align*}
B_{x,0} & =\{\norm{z-x}_{x}\geq cr\}\,\ \text{with }c\geq1+\frac{2}{\sqrt{d}}\,\log\frac{1}{\veps}\,,\quad\text{(Tail bound for Gaussian)}\\
B_{x,1} & =\{-\langle\nabla f(x),x-z\rangle\leq0\}\,,\quad\text{(Symmetry of Gaussian)}\\
B_{x,2} & =\{\snorm{z-x}_{z}^{2}-\snorm{z-x}_{x}^{2}>2\veps\frac{r^{2}}{d}\}\,,\quad\text{(ASC of }g)\\
B_{x,3} & =\bbrace{\langle\grad\vphi(x),z-x\rangle\leq-2\frac{r}{\sqrt{d}}\,\snorm{g(x)^{-1/2}\grad\vphi(x)}_{2}\,\log\frac{1}{\veps}}\,.\quad\text{(SSC \& tail bound for Gaussian)}
\end{align*}
We further denote $G:=G_{x}\cup G_{y}$ and a partition of $G$ by
\[
G_{x\backslash y}:=G_{x}\backslash G_{y},\qquad G_{x,y}:=G_{x}\cap G_{y},\qquad G_{y\backslash x}:=G_{y}\backslash G_{x}\,.
\]
Then,
\begin{align*}
\half\int\underbrace{|A(x,z)\,p_{x}(z)-A(y,z)\,p_{y}(z)|}_{\eqqcolon Q}\,\D z & \leq3\veps+\underbrace{\half\int_{G_{x\backslash y}}Q\,\D z}_{\eqqcolon\mc A}+\underbrace{\half\int_{G_{y\backslash x}}Q\,\D z}_{\eqqcolon\mc B}+\underbrace{\half\int_{G_{x,y}}Q\,\D z}_{\eqqcolon\mc C}\,.
\end{align*}
We can bound $\acal$ and $\bcal$ by $\mc O(\veps)$ by Pinsker's
inequality and a well-known formula for the $\KL$ divergence between
two Gaussians. As for $\mc C$, conditioning on $B_{x,1}$ and using
the triangle inequality lead to

\[
\mc C\leq\frac{1}{4}+2\veps+\half\int_{G_{x}\cap G_{y}\cap B_{x,1}^{c}}\Big|\min\Bpar{1,\underbrace{\frac{\exp f(x)}{\exp f(z)}\,\frac{p_{z}(x)}{p_{x}(z)}}_{\eqqcolon\msf U}}-\min\Bpar{\underbrace{\frac{p_{y}(z)}{p_{x}(z)}}_{\eqqcolon\msf V},\underbrace{\frac{\exp f(y)}{\exp f(z)}\,\frac{p_{z}(y)}{p_{x}(z)}}_{\eqqcolon\msf W}}\Big|\,p_{x}(z)\,\D z\,.
\]
The bound of $\log\msf U\ge-4\veps$ was already obtained when bounding
$r_{x}$. We then show that $\lvert\log\msf V\rvert\le5\veps$ and
$\log\msf W\ge-7\veps$ conditioned on $G_{x}\cap G_{y}\cap B_{x,1}^{c}$
via closeness of SSC (Lemma~\ref{lem:strongSC-closeness}). Using
these, 
\[
\int_{G_{x}\cap G_{y}\cap B_{x,1}^{c}}|1\wedge\msf U-\msf V\wedge\msf W|\,p_{x}(z)\,\D z\leq e^{5\veps}-e^{4\veps}\,,
\]
which results in $\mc C\le1/4+\O(\veps)$. Putting the bounds on $r_{x},r_{y},\mc A,\mc B$,
and $\mc C$ together, we conclude that the TV-distance is bounded
by $3/4+\O(\veps)$.
\begin{rem}
We further note that $\snorm{x-y}_{x}$ can be replaced by the Riemannian
distance $d_{\phi}(x,y)$ with the metric defined by $\hess\phi$,
since these two distance are within a constant factor of each other:
\begin{lem}
[\citet{nesterov2002riemannian}, Lemma 3.1] \label{lem:Riemann-Dikin-close}
Let $\phi:\intk\to\R$ be self-concordant, and $x,y\in\intk$ with
$\delta:=\snorm{x-y}_{x}<1$. Then,
\[
\delta-\half\delta^{2}\leq d_{\phi}(x,y)\leq-\log(1-\delta)\,.
\]
\end{lem}

\end{rem}

Next, we present two isoperimetric inequalities derived from distinct
sources: the first comes from the symmetry of a barrier, while the
second arises from strong convexity in a local metric.

\paragraph{Isoperimetry via barrier parameters.}

The first one states that isoperimetry of log-concave distributions
under distance $d_{g}(x,y)$ (or $\snorm{x-y}_{g(x)}$ due to Lemma~\ref{lem:Riemann-Dikin-close})
is $\Omega(1/\sqrt{\onu})$. The following lemma is an extension of
\citet{laddha2020strong} from uniform distributions (over a convex
body) to general log-concave distributions. We defer the proof to
\S\ref{proof:isoperimetry}.
\begin{lem}
\label{lem:symmetry-iso} Let $\phi$ be self-concordant and $d_{\phi}$
be the Riemannian distance induced by the Hessian metric $\hess\phi$.
For a log-concave distribution $\pi$, isoperimetry $\psi_{\pi}$
under distance $d_{\phi}$ is $\Omega(1/\sqrt{\onu})$.
\end{lem}


\paragraph{Isoperimetry from relative strong convexity.}

Another kind of isoperimetry comes from relative strong-convexity
of the potential of a distribution. For a scalar $\alpha>0$, isoperimetry
of $e^{-\alpha\phi}$ on a Hessian manifold equipped with the metric
$\hess\phi$ is $\Omega(\sqrt{\alpha})$ if $\Dd^{4}\phi(x)\Brack{h^{\otimes4}}\geq0$
for all $x\in K$ and $h\in\Rd$ (see \citet[Lemma 37]{lee2018convergence}).
\citet[Lemma 9]{gopi2023algorithmic} further generalizes this to
show that if $\phi$ is self-concordant and the potential $f$ is
$\alpha$-relatively strong convex, then its isoperimetry is $\Omega(\sqrt{\alpha})$.
We can adapt this lemma by restricting this to a convex set $K$ (not
necessarily bounded). See \S\ref{proof:isoperimetry} for the proof.
\begin{lem}
[\citet{gopi2023algorithmic}, Adapted from Lemma 9] \label{lem:sc-iso}
For a closed convex set $K\subset\Rd$, let a convex function $\phi:\intk\to\R$
be self-concordant on $K$, $f:\intk\to\R$ $\alpha$-relatively strongly
convex in $\phi$, and $\pi$ a log-concave distribution with $\pi\propto\exp(-f)\cdot\mathbf{1}_{K}$.
For a partition $\{S_{1},S_{2},S_{3}\}$ of $K$ and the Riemannian
distance $d_{\phi}$ induced by the inner product $\langle a,b\rangle_{x}:=a^{\T}\hess\phi(x)\,b$,
it holds that 
\[
\pi(S_{3})\gtrsim\sqrt{\alpha}\,d_{\phi}(S_{1},S_{2})\,\pi(S_{1})\,\pi(S_{2})\,.\qedhere
\]
\end{lem}


\subsection{Mixing time: Proof of Theorem~\ref{thm:Dikin}}

Putting all these components together, we obtain the following mixing-time
bounds for the $\dw$. 

\thmDikin*
\begin{proof}
Lemma~\ref{lem:conductance} ensures that $\Phi\gtrsim\frac{r}{\sqrt{d}}\psi$
due to the one-step coupling in Lemma~\ref{lem:one-step}. Lemma~\ref{lem:symmetry-iso}
leads to $\psi\gtrsim\frac{1}{\sqrt{\onu}}$, while Lemma~\ref{lem:sc-iso}
implies $\psi\gtrsim\sqrt{\alpha}$ due to $\hess\phi\asymp g$. Thus,
\[
\Phi\gtrsim\frac{1}{\sqrt{d}}\,\bpar{\sqrt{\alpha}\vee\frac{1}{\sqrt{\onu}}}\bpar{1\vee\frac{1}{\sqrt{\beta}}}\,,
\]
and using Lemma~\ref{lem:conductanceBound}, we can enforce $\dtv(\pi_{T},\pi)\leq\veps$
by solving $\sqrt{\Lambda}e^{-T\Phi^{2}/2}\leq\veps$ and $\frac{\veps}{2}+\sqrt{\frac{\Lambda}{\veps/2}}e^{-T\Phi^{2}/2}\leq\veps$
for $T$, which results in 
\[
T\gtrsim d\,(1\vee\beta)\,\bpar{\onu\wedge\frac{1}{\alpha}}\log\frac{\Lambda}{\veps}\,.\qedhere
\]
\end{proof}

