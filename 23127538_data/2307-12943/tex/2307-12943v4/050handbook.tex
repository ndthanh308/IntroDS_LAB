\global\long\def\vec{\textup{\textsf{vec}}}%
\global\long\def\svec{\textup{\textsf{svec}}}%


\section{Structured densities and constraint families \label{sec:handbook-barrier}}

In order to obtain a mixing-time bound of the $\dw$ for the reduced
problem, a concrete understanding of properties and parameters of
barriers for $K_{i}$ and $K_{j}$ is essential. To this end, we revisit
self-concordant barriers for structured convex constraints and level
sets, examining the required scaling factors which ensure those properties.

\subsection{Linear constraints}

Consider a set of linear constraints: $K=\{x\in\Rd:Ax\geq b\}$ for
$A\in\R^{m\times d}$ and $b\in\R^{m}$, where $A$ has no all-zero
rows. We use $s_{x}:=Ax-b$ to denote the slack at $x$, and $A_{x}:=S_{x}^{-1}A$
to denote the constraints normalized by the slack, where $S_{x}:=\Diag(s_{x})$
is the diagonalization of the slack.

We now introduce three barriers (and metrics) for handling the linear
constraints.

\paragraph{Logarithmic barrier.}

The logarithmic barrier $\phi_{\log}(x):=-\sum_{i=1}^{m}\log(a_{i}^{\T}x-b_{i})$
is the simplest self-concordant barrier for linear constraints. We
refer readers to \S\ref{proof:linear-log-barrier} for gentle introduction
to the log-barriers. As seen below, we demonstrate that the metric
induced by the logarithmic barrier has $\nu,\onu=m$ and requires
no scaling to achieve SSC, SLTSC, and SASC.
\begin{lem}
[Logarithmic barrier]\label{lem:log-barrier} For a closed convex
$K=\{x\in\Rd:Ax\geq b\}$ with $A\in\R^{m\times d}$ and $b\in\R^{m}$,
let $\phi_{\log}(x)=-\sum_{i=1}^{m}\log(a_{i}^{\T}x-b_{i})$ and define
$g(x):=\hess\phi_{\log}(x)=A_{x}^{\T}A_{x}$.
\begin{itemize}
\item $\nu=m$ (\citet{nesterov1994interior}).
\item SSC along $\rowspace(A)$ and $\onu=m$ (Lemma~\ref{lem:paramsBarrier}).
\item $\Dd^{2}g(x)[h,h]\succeq0$ for any $h\in\Rd$ (so SLTSC) (Claim~\ref{claim:diffLogBarrier}).
\item SASC (Lemma~\ref{lem:logBarrier-SASC}).
\end{itemize}
\end{lem}


\paragraph{Vaidya metric.}

In sampling over a polytope $K$, the number $m$ of constraints is
assumed to be greater than the ambient dimension $d$. Given that
the mixing time of the $\dw$ for uniform sampling is $\otilde{d\onu}=\otilde{dm}$,
a larger $m$ leads to a worse mixing time. Is there a self-concordant
barrier that has a better dependence on $m$ for its self-concordance
and symmetry parameters, without compromising SSC, SLTSC, and SASC?

Let us recall the \emph{leverage score} first and move onto such improved
self-concordant barriers. For a full-rank matrix $A\in\R^{m\times d}$
with $m\geq d$, we recall that $P(A)=A(A^{\T}A)^{-1}A^{\T}$ is the
orthogonal projection matrix onto the column space of $A$, and the
leverage scores of $A$ is $\sigma(A)=\diag(P(A))\in\R^{m}$. We let
$\Sigma(A):=\Diag(\sigma(A))=\Diag(P(A))$ and $P^{(2)}(A)=P(A)\circ P(A)$,
where $P(A)\circ P(A)$ is the Hadamard product of size $d\times d$
defined by $(P(A)\circ P(A))_{ij}=[P(A)]_{ij}^{2}$.

\citet{vaidya1996new} introduced the \emph{volumetric barrier} for
$K$ defined by
\[
\phi_{\vol}=\half\,\log\det(\hess\phi_{\log})=\half\,\log\det(A_{x}^{\T}A_{x})\,.
\]
Then the Hessian of $\phi_{\vol}$ can be written as
\[
\hess\phi_{\vol}=A_{x}^{\T}(3\Sigma_{x}-2P_{x}^{(2)})A_{x}\,,
\]
where $\Sigma_{x}=\Diag(\sigma(A_{x}))$ is the diagonalized leverage
scores, and this Hessian satisfies 
\[
A_{x}^{\T}\Sigma_{x}A_{x}\preceq\hess\phi_{\vol}(x)\preceq3A_{x}^{\T}\Sigma_{x}A_{x}\,.
\]
We refer readers to \S\ref{proof:linear-volumetric} for details.
In other words, the \emph{approximate} volumetric metric $A_{x}^{\T}\Sigma_{x}A_{x}$
serves as an $\mc O(1)$-approximation of the local metric $\hess\phi_{\vol}$
(i.e., $A_{x}^{\T}\Sigma_{x}A_{x}\asymp\hess\phi_{\vol}(x)$). We
find in Lemma~\ref{lem:paramsBarrier} that the local metric $40\sqrt{m}A_{x}^{\T}\Sigma_{x}A_{x}$
is SSC with $\nu,\,\onu=\mc O(\sqrt{m}d)$, but in some regime of
$d$ this parameter leads to worse mixing of the $\dw$. In the same
paper, \citet{vaidya1996new} introduced a \emph{regularized} volumetric
metric by adding $\O\bpar{\hess\phi_{\log}}$, which we call the \emph{Vaidya
metric}:

\[
g(x):=\sqrt{\frac{m}{d}}\,A_{x}^{\T}\bpar{\Sigma_{x}+\frac{d}{m}I_{m}}A_{x}\,.
\]
Note that $g(x)\asymp\hess\bpar{\sqrt{\frac{m}{d}}\bpar{\phi_{\vol}+\frac{d}{m}\text{\ensuremath{\phi_{\log}}}}}$.
We show that the Vaidya metric is also SSC, SLTSC, and SASC without
additional scaling, while it has a better $\nu$ and $\onu$ than
the logarithmic barrier.
\begin{lem}
[Vaidya metric]\label{lem:vaidya} For a closed convex $K=\{x\in\Rd:Ax\geq b\}$
with $A\in\R^{m\times d}$ and $b\in\R^{m}$, let $g(x)=\sqrt{\frac{m}{d}}A_{x}^{\T}\bpar{\Sigma_{x}+\frac{d}{m}I_{m}}A_{x}$.
\begin{itemize}
\item $\nu=\mc O(\sqrt{md})$ \citet[Theorem 5.2]{anstreicher1997volumetric}.
\item SSC and $\onu=\mc O(\sqrt{md})$ (Lemma~\ref{lem:paramsBarrier}).
\item SLTSC (Lemma~\ref{lem:vaidya-SLTSC}) and SASC (Lemma~\ref{lem:vaidya-SASC}).
\end{itemize}
\end{lem}


\paragraph{Lewis weights metric.}

Self-concordance and symmetry parameters of $\mc O(\sqrt{md})$ is
certainly better than $\mc O(m)$, but can we even achieve an $\mc O(d\log^{\mc O(1)}m)$
bound on those parameters?

Let us recall the $\ell_{p}$-\emph{Lewis weights}. The $\ell_{p}$-Lewis
weight of $A$ is denoted by $w(A)$, the solution $w$ to the equation
$w(A)=\diag\bpar{W^{\nicefrac{1}{2}-\nicefrac{1}{p}}A(A^{\T}W^{1-\nicefrac{2}{p}}A)^{-1}A^{\T}W^{\nicefrac{1}{2}-\nicefrac{1}{p}}}\in\R^{m}$
for $W:=\Diag(w)$. For $W_{x}=\Diag(w(A_{x}))$ and $p\geq2$, the
Lewis weight barrier function is defined by
\[
\phi_{\lw}(x):=\log\det(A_{x}^{\T}W_{x}^{1-\nicefrac{2}{p}}A_{x})\,.
\]
Note that the leverage score and volumetric barrier can be recovered
as a special case of the Lewis weight and barrier by setting $p=2$.
As done for the Vaidya metric, it is natural to consider the Lewis
weight metric with $p=\Theta(\log^{\mc O(1)}m)$, defined as 
\[
g(x):=\mc O(\log^{\mc O(1)}m)\,A_{x}^{\T}W_{x}A_{x}\,.
\]
In fact, this metric serves as an $\mc O(\log^{\mc O(1)}m)$-approximation
of $\hess\phi_{\lw}$, as demonstrated in the following relation proven
in \citet[Lemma 31]{lee2019solving}:
\[
A_{x}^{\T}\Sigma_{x}A_{x}\preceq\hess\phi_{\lw}\preceq(1+p)\,A_{x}^{\T}\Sigma_{x}A_{x}\,.
\]
Ignoring the logarithmic factors we have $\hess\phi_{\lw}\asymp g$.
Notably, the Lewis-weight metric needs an additional $\sqrt{d}$-scaling
for SLTSC and SASC, unlike the logarithmic barrier and Vaidya metric.
Hence, when combining this with other metrics, one should use $\sqrt{d}g$,
which leads to $\nu,\,\onu=\mc O(d^{3/2}\,\log^{\mc O(1)}m)$. 
\begin{lem}
[Lewis weight metric]\label{lem:Lewis-weight} For a closed convex
$K=\{x\in\Rd:Ax\geq b\}$ with $A\in\R^{m\times d}$ and $b\in\R^{m}$,
let $g(x)=\mc O(\log^{\mc O(1)}m)\,A_{x}^{\T}W_{x}A_{x}$.
\begin{itemize}
\item $\nu=\mc O(d\log^{5}m)$ \citet[Theorem 30]{lee2019solving}.
\item SSC and $\onu=\mc O(d\log^{\mc O(1)}m)$ (Lemma~\ref{lem:paramsBarrier}).
\item $\sqrt{d}g$ is SLTSC (Lemma~\ref{lem:Lw-SLTSC}) and SASC (Lemma~\ref{lem:Lw-SASC}).
\end{itemize}
\end{lem}


\subsubsection{Analysis of self-concordant metrics for linear constraints \label{subsec:analysis-linear-metric}}

\paragraph{Strong self-concordance and symmetry.}

We defer the proofs of two lemmas below to \S\ref{proof:linear-SSC-symm}.
We study SSC and symmetry of the metrics of the form $A_{x}^{\T}D_{x}A_{x}$
in Lemma~\ref{lem:helper4Diagonal}, where $D_{x}\in\R^{m\times m}$
is a diagonal matrix used to address the constraints of the form $Ax\geq b$
for $A\in\R^{m\times d}$ and $b\in\R^{m}$. Specifically, we relate
the notions of SSC and symmetry to well-studied terms in the field
of optimization, namely $\max_{i}\,[\sigma(\sqrt{D_{x}}A_{x})]_{i}/[D_{x}]_{ii}$
and $\snorm{\Dd D_{x}[h]}_{D_{x}^{-1}}^{2}$. 
\begin{lem}
\label{lem:helper4Diagonal} For $x\in\inter(K)$, let $g(x)=A_{x}^{\T}D_{x}A_{x}\in\Rdd$
for a diagonal matrix $0\prec D_{x}\in\R^{m\times m}$.
\begin{itemize}
\item For any PSD matrix function $g'$ such that $g'+g$ is invertible
on the domain,
\begin{align*}
 & \snorm{(g'(x)+g(x))^{-1/2}\Dd g(x)[h]\,(g'(x)+g(x))^{-1/2}}_{F}^{2}\\
 & \qquad\qquad\leq4\max_{i}\frac{[\sigma(\sqrt{D_{x}}A_{x})]_{i}}{[D_{x}]_{ii}}\cdot\bpar{\snorm h_{g(x)}^{2}+\sum_{i=1}^{m}\frac{(\Dd D_{x}[h])_{ii}^{2}}{[D_{x}]_{ii}}}\,.
\end{align*}
 
\item $\max_{h:\norm h_{g(x)}=1}\norm{A_{x}h}_{\infty}=\bpar{\max_{i\in[m]}\frac{[\sigma(\sqrt{D_{x}}A_{x})]_{i}}{[D_{x}]_{ii}}}^{1/2}$.
\item $K\cap(2x-K)\subset\dcal_{g}^{\sqrt{\tr(D_{x})}}(x)$.
\end{itemize}
\end{lem}

Then for each metric we refer to existing bounds on these terms, estimating
the smallest possible scaling required for SSC and symmetry. 
\begin{lem}
[Strong self-concordance and symmetry]\label{lem:paramsBarrier}
Let $A\in\R^{m\times d}$, $\Sigma_{x}=\Diag(\sigma(A_{x}))\in\R^{m\times m}$,
and $W_{x}=\Diag(w_{x})\in\R^{m\times m}$ for the $\ell_{p}$-Lewis
weight $w_{x}$ with $p=\mc O(\log m)$.
\begin{itemize}
\item Logarithmic metric: $g(x)=A_{x}^{\T}A_{x}$ with $D_{x}=I_{m}$ is
SSC along $\rowspace(A)$ with $\onu=m$.
\item Approximate volumetric metric: $g(x)=40\sqrt{m}A_{x}^{\T}\Sigma_{x}A_{x}$
with $D_{x}=40\sqrt{m}\Sigma_{x}$ is SSC with $\onu=\mc O(\sqrt{m}d)$.
\item Vaidya metric: $g(x)=22\sqrt{\frac{m}{d}}A_{x}^{\T}\bpar{\Sigma_{x}+\frac{d}{m}I_{m}}A_{x}$
with $D_{x}=22\sqrt{\frac{m}{d}}\bpar{\Sigma_{x}+\frac{d}{m}I_{m}}$
is SSC with $\onu=\mc O(\sqrt{md})$.
\item Lewis-weight metric: $\exists$ positive constants $c_{1}$ and $c_{2}$
such that $g(x)=c_{1}(\log m)^{c_{2}}A_{x}^{\T}W_{x}A_{x}$ is SSC
and $\onu$-symmetric with $\onu=\mc O^{*}(d)$.\label{lem:LSmetricStrongandSymmetry}
\end{itemize}
\end{lem}


\paragraph{Strongly lower trace self-concordance}

We show SLTSC of the Vaidya and Lewis-weight metric. Let $g_{2}$
be either Vaidya or Lewis-weight metric, and $g_{1}$ be an arbitrary
PSD matrix function on $K$ such that $g=g_{1}+g_{2}$ is PD on $\intk$.
Ensuring (S)LTSC of the Vaidya or Lewis-weight metrics is challenging,
as $\Dd^{2}g_{2}[h,h]\succeq0$ is difficult to verify due to complicated
expressions for $\Dd^{2}\Sigma_{x}[h,h]$ and $\Dd^{2}W_{x}[h,h]$.
As for the Vaidya metric, we compute higher-order derivatives of leverage
scores and other pertinent matrices in Lemma~\ref{lem:calculusLeverage},
finding succinct formulas by using algebraic properties of the Hadamard
product. We then show SLTSC of $g_{2}$ using these results (see \S\ref{proof:linear-vaidya-SLTSC}
for the proof):
\begin{lem}
[SLTSC of Vaidya]\label{lem:vaidya-SLTSC} $\tr\bpar{g^{-1}\Dd^{2}g_{2}(x)[h,h]}\geq-\snorm h_{g_{2}(x)}^{2}/2$
for the Vaidya metric $g_{2}$.
\end{lem}

For the Lewis-weights metric, analysis is more involved due to numerous
terms appearing in $\Dd^{2}W_{x}[h,h]$. In order to avoid dealing
with each of the terms, we employ existing bounds on derivatives of
$W_{x}$ and other relevant matrices in \S\ref{proof:linear-LW}.
This approach significantly simplifies the computation but comes at
the cost of an additional scaling of $\sqrt{d}$, which as far as
we can tell might be unavoidable. We refer readers to \S\ref{proof:linear-Lewis-SLTSC}
for the proof.
\begin{lem}
[SLTSC of Lewis-weight]\label{lem:Lw-SLTSC} $\tr\bpar{g(x)^{-1}\Dd^{2}g_{2}(x)[h,h]}\geq-\snorm h_{g_{2}(x)}^{2}$,
where $g_{2}(x)=cA_{x}^{\T}W_{x}A_{x}$ with $c=c_{1}(\log m)^{c_{2}}\sqrt{d}$
for some constants $c_{1},c_{2}>0$.
\end{lem}


\paragraph{Strongly average self-concordance.}

Typically, (S)ASC is the most challenging property to verify, often
requiring involved analysis in order to establish it \emph{without}
additional scalings. Since the three metrics are HSC (e.g., see Lemma~\ref{lem:Lw-hsc}
for Lewis-weight metrics), scaling by $d$ leads to SASC by Lemma~\ref{lem:hsc-to-sasc}.
However, for linear constraints one can still achieve SASC without
scaling (or with a smaller scaling) through more sophisticated concentration
techniques.

To sketch this idea, we recall that SASC requires showing that for
small enough $r$
\[
\snorm{z-x}_{g(z)}^{2}-\snorm{z-x}_{g(x)}^{2}\leq2\veps\frac{r^{2}}{d}\,.
\]
Taylor's expansion of $\snorm{z-x}_{g(z)}^{2}$ at $z=x$ up to second-order
necessitates bounds on
\[
\Dd g(x)[(z-x)^{\otimes3}]=\frac{r^{3}}{d^{3/2}}\Dd g(x)[h^{\otimes3}]\qquad\text{and}\qquad\Dd g(x')[(z-x)^{\otimes4}]=\frac{r^{4}}{d^{2}}\Dd^{2}g(x')[h^{\otimes4}]\,,
\]
for some $x'\in[x,z]$ and $h\sim\ncal(0,I_{d})$. Observe that the
first-order term $P(h):=\frac{r^{3}}{d^{3/2}}\Dd g(x)[h^{\otimes3}]$
is a Gaussian polynomial in $h$, and this is where we can invoke
the following concentration phenomenon:
\begin{lem}
[Concentration of Gaussian polynomials] \label{lem:conc-gaussian-poly}
For $d\geq1$, let $P:\Rd\to\R$ be a polynomial of degree $n$. For
any $t\geq(2e)^{n/2}$, 
\[
\P_{h\sim\ncal(0,I_{d})}\Bbrack{|P(h)|\geq t\sqrt{\E[P(h)^{2}]}}\leq\exp\bpar{-\frac{n}{2e}\,t^{2/n}}\,.
\]
\end{lem}

This concentration inequality necessitates bounding $\E[P(h)^{2}]$,
and this is where Stein's lemma comes into play:
\begin{lem}
\label{lem:stein} For $h=(h_{1},\dots,h_{d})\sim\ncal(0,I_{d})$,
it holds that $\E[h_{i}f(h)]=\E[\de_{i}f(h)]$.
\end{lem}

Unlike the first-order term, the second-order term is \emph{not} a
Gaussian polynomial due to $x'$ depending on $z$. To address this
issue, we derive an upper bound (in absolute value) of the quadratic
form. Using coordinate-wise closeness of slacks, leverage scores,
and Lewis weights at two nearby points, we replace every value estimated
at $z$ by those at $x$, removing dependence on $z$ in the quadratic
bound. The resulting quadratic bound is now a Gaussian polynomial,
so we follow the same proof approach as with the first-order term.

This approach was used by \citet{sachdeva2016mixing} for ASC of log-barriers
and by \citet{chen2018fast} for that of Vaidya and Lewis-weight metrics.
We further extend this approach to achieve SASC of those metrics,
going beyond ASC.
\begin{lem}
[SASC of logarithmic barrier] \label{lem:logBarrier-SASC} $g(x)=\hess\phi_{\log}(x)=A_{x}^{\T}A_{x}$
is SASC.
\end{lem}

See \S\ref{proof:linear-SASC-log} for the proof.
\begin{lem}
[SASC of Vaidya metric] \label{lem:vaidya-SASC} $g(x)=\mc O\bpar{\sqrt{\frac{m}{d}}}\,A_{x}^{\T}(\Sigma_{x}+\frac{d}{m}I_{m})A_{x}$
is SASC.
\end{lem}

See \S\ref{proof:linear-SASC-vaidya} for the proof.
\begin{lem}
[SASC of Lewis-weight metric] \label{lem:Lw-SASC} There exists constants
$c_{1}$ and $c_{2}$ such that $g(x)=c_{1}\sqrt{d}\log^{c_{2}}m\,A_{x}^{\T}W_{x}A_{x}=\mc O^{*}(\sqrt{d})\,A_{x}^{\T}W_{x}A_{x}$
is SASC.
\end{lem}

See \S\ref{proof:linear-SASC-Lw} for the proof.

\subsection{Quadratic potentials and constraints}

Suppose that in \eqref{eq:reduced-problem} we have either $f_{i}(x),\,h_{j}(x)=\snorm{x-\mu}_{\Sigma}^{2}$
or $\half x^{\T}Qx+p^{\T}x+l$ for $\mu,p\in\Rd$, $\Sigma\in\pd$,
and $0\neq Q\in\psd$.

\paragraph{Quadratic constraint.}

Consider a second-order region given by $K=\{x\in\Rd:\half x^{\T}Qx+p^{\T}x+l\leq0\}$.
\citet{nesterov1994interior} shows that $\phi:=-\log f$ is an $1$-self-concordant
barrier for $K$, when $f(x)=-\half\snorm{x-\mu}_{\Sigma}^{2}$ or
$-(\half x^{\T}Qx+p^{\T}x+l)$. Since $\onu=\mc O(\nu^{2})$ for a
self-concordant barrier due to Lemma~\ref{lem:bound-symmetry}, $\phi$
is $\mc O(1)$-symmetric. In case we consider $\snorm{x-\mu}_{\Sigma}^{2}$,
the trivial scaling by dimension $d$ implies that $d\phi$ is SSC
and $\mc O(d)$-symmetric.

Moreover, $d\phi$ is SASC by Lemma~\ref{lem:hsc-to-sasc} by HSC
of $\phi$. For HSC of $\phi$, we develop a handy tool for checking
HSC. See \S\ref{proof:quadratic} for the proof.
\begin{lem}
\label{lem:4th-log} For a real-valued function $f$ on $K\subset\Rd$,
let $\psi=-\log f$ be a $\nu$-self-concordant barrier for $K$.
Then, 
\[
|\Dd^{4}\psi(x)[h^{\otimes4}]|\lesssim\nu^{2}\snorm h_{\hess\psi(x)}^{2}+\big|\frac{\Dd^{4}f(x)[h^{\otimes4}]}{f(x)}\big|\,.
\]
\end{lem}

Using this tool, we can study properties of the barrier for the quadratic
constraints. We provide the proof in \S\ref{proof:quadratic}.
\begin{lem}
[Quadratic constraint]\label{lem:quadratic-const} For a closed convex
$K=\{x\in\Rd:\half x^{\T}Qx+p^{\T}x+l\leq0\}$ with $p\in\Rd$ and
$0\neq Q\in\psd$, let $\phi(x)=-\log(-l-p^{\T}x-\half x^{\T}Qx)$
and $g=d\,\hess\phi$.
\begin{itemize}
\item $\nu,\,\onu=\mc O(d)$.
\item SSC when $Q\succ0$, and SASC.
\item $\Dd^{2}g(x)[h,h]\succeq0$ for any $x\in\inter(K)$ and $h\in\Rd$
(so SLTSC).
\end{itemize}
\end{lem}


\paragraph{Gaussian distribution ($f(x)=\protect\half\protect\snorm{x-\mu}_{\Sigma}^{2}$).}

Suppose the quadratic term $f(x)=\half\snorm{x-\mu}_{\Sigma}^{2}$
appears in a potential of a target distribution. Then its epigraph
is 
\[
\{(x,t)\in\R^{d+1}:\half\snorm{x-\mu}_{\Sigma}^{2}-t\leq0\}\,,
\]
and clearly $q(x,t)=\half\snorm{x-\mu}_{\Sigma}^{2}-t$ is a quadratic
function in $(x,t)$. Hence, this level set admits an $1$-self-concordant
barrier
\[
\phi(x,t)=-\log(t-\half\snorm{x-\mu}_{\Sigma}^{2})\,.
\]
Our earlier discussion immediately leads to the following result:
\begin{lem}
[Quadratic potential] \label{lem:Gaussian-potential}Consider a closed
convex $K=\{(x,t):\half\snorm{x-\mu}_{\Sigma}^{2}\leq t\}$ with $\mu\in\Rd$
and $\Sigma\in\pd$, and let $\phi(x)=-\log(t-\half\snorm{x-\mu}_{\Sigma}^{2})$
and $g=d\,\hess\phi$.
\begin{itemize}
\item $\nu_{g},\,\onu_{g}=\mc O(d)$.
\item SSC and SASC.
\item $\Dd^{2}g(x,t)[h,h]\succeq0$ for any $(x,t)\in\inter(K)$ and $h\in\R^{d+1}$.
\end{itemize}
\end{lem}


\paragraph{Second-order cone ($f(x)=\protect\half\protect\snorm{x-\mu}_{\Sigma}$).}

It is common that a potential includes a non-smooth term like $\norm{Ax-b}_{2}$
in many applications, and we can handle such potentials via our framework.
\citet[Lemma 4.3.3]{nesterov1994interior} shows that 
\[
\phi(x,t)=-\log(t^{2}-\snorm x^{2})
\]
is a $2$-self-concordant for a level set $K=\{(x,t)\in\Rd\times\R:\snorm x_{2}\leq t\}$
(here we may assume that $\mu=0$ and $\Sigma=I$ due to Lemma~\ref{lem:linear-trans}).
This level set is called a \emph{second-order cone} or Lorentz cone.

Applying Lemma~\ref{lem:4th-log} to $f(x,t)=t^{2}-\norm x^{2}$
with $\nu=2$, we immediately show HSC of $\phi$. Thus, $d\phi$
satisfies SLTSC and SASC by Lemma~\ref{lem:hsc-to-sltsc} and Lemma~\ref{lem:hsc-to-sasc},
respectively.
\begin{lem}
[Second-order cone] \label{lem:soc} Consider a closed convex $K=\{(x,t):\snorm{x-\mu}_{\Sigma}\leq t\}$
with $\mu\in\Rd$ and $\Sigma\in\pd$, and let $\phi(x,t)=-\log(t^{2}-\snorm{x-\mu}_{\Sigma}^{2})$
and $g=d\,\hess\phi$.
\begin{itemize}
\item $\nu_{g},\,\onu_{g}=\mc O(d)$.
\item SSC, SASC, and SLTSC.
\end{itemize}
\end{lem}


\subsection{PSD cone}

The function $\phi(X)=-\log\det X$ serves as an $d$-self-concordant
barrier for the PSD cone $\psd$. While achieving self-concordance
does not require additional scaling, it turns out that SSC requires
a scaling of $\Theta(d)$. Notably, this scaling is less than the
trivial dimension-based scaling of $d_{s}:=d(d+1)/2$. Also, direct
computation leads to $\Dd^{4}\phi(X)[H,H]\succeq0$ (so SLTSC).

As $\phi$ is HSC, scaling by $d_{s}$ ensures SASC. However, we can
achieve ASC with a smaller scaling by $\mc O(d)$ via the random matrix
theory.
\begin{lem}
[PSD cone] \label{lem:psd} On a closed convex $K=\psd$, let $\phi(X)=-\log\det X$
and define $g=d\,\hess\phi$.
\begin{itemize}
\item $\nu=d^{2}$ (\citet{nesterov1994interior}) and $\onu=d^{2}$ (Lemma~\ref{lem:logdet-symm}).
\item SSC (Corollary~\ref{cor:logdet-ssc}).
\item $\Dd^{2}g(X)[H,H]\succeq0$ for any $X\in\intk$ and $H\in\mbb S^{d}$
(Lemma~\ref{lem:logdet-sltsc}).
\item ASC (Lemma~\ref{lem:logdet-asc}), and $d_{s}\,\hess\phi$ is SASC.
\end{itemize}
\end{lem}


\subsubsection{Formalism via matrix-vector transformations \label{subsec:formalism}}

In analyzing $\phi$, we work in $\R^{d_{s}}=\R^{d(d+1)/2}$ and $\mbb S^{d}$
simultaneously in the sequel, moving back and forth between them implicitly.
We justify this identification as follows.

\paragraph{Measure on $\protect\mbb S^{d}$.}

We can define and work with the Lebesgue measure on $\mbb S^{d}$
by identifying it with the Lebesgue measure on $\R^{d_{s}}$, where
each component in the Lebesgue measure on $\mbb S^{d}$ corresponds
to each entry in the upper triangular part. Hence, with the Lebesgue
measure $\D X$ on $\mbb S^{d}$ it is straightforward to define a
probability distribution on $\mbb S^{d}$ whose probability density
function with respect to $\D X$ is proportional to $\exp(-f)$ for
a function $f:\mbb S^{d}\to\R$. For instance, the uniform distribution
over a region corresponds to $f$ being constant in the region and
infinity outside of the region, and an exponential distribution to
$f(X)=\inner{C,X}=\tr(C^{\T}X)$ for $C\in\mbb S^{d}$.

\paragraph{Directional derivatives.}

A function $\phi:\mbb S^{d}\to\R$ induces its counterpart $\psi:\R^{d_{s}}\to\R$
defined by $\psi(x)=\phi(X)$ for $x:=\svec(X)$. For symmetric matrices
$\{H_{i}\}_{i\leq k}$, the $k$-th directional derivative of $\phi$
in directions $H_{1},\dots,H_{k}$ is 
\[
\Dd^{k}\phi(X)[H_{1},\cdots,H_{k}]\defeq\frac{\D^{k}}{\D t_{k}\cdots\D t_{1}}\phi\Bpar{X+\sum_{i=1}^{k}t_{i}H_{i}}\bigg\vert_{t_{1},\dots,t_{k}=0}\,.
\]
For $h_{i}:=\svec(H_{i})$, it follows that $\phi(X+\sum_{i=1}^{k}t_{i}H_{i})=\psi(x+\sum_{i=1}^{k}t_{i}h_{i})$
and thus
\[
\Dd^{k}\phi(X)[H_{1},\cdots,H_{k}]=\Dd^{k}\psi(x)[h_{1},\cdots,h_{k}]\,.
\]
With this identification in hand, since the notion of (symmetric or
strong) self-concordance is formulated in terms of directional derivatives,
we can deal with both representations without having to specify one
of them.

\paragraph{Important operators.}

We introduce three linear operators that enable us to make smooth
transitions between $\mbb S^{d}$ and $\R^{d_{s}}$.
\begin{defn}
[\citet{magnus1980elimination}] \label{def:linearOperators} Let
$E_{ij}=e_{i}e_{j}^{\T}\in\Rdd$ be the matrix with a single $1$
in the $(i,j)$ position and zeros elsewhere.
\begin{itemize}
\item $M:\R^{d_{s}}\to\R^{d^{2}}$ is the linear operator that maps $\svec(\cdot)$
to $\vec(\cdot)$ (i.e., $M\circ\svec=\vec$). It can be written as
$M=\sum_{i\geq j}\vec(T_{ij})u_{ij}^{\T}$, where $T_{ij}\in\Rdd$
has all zero entries except for $1$ at $(i,j)$ and $(j,i)$ positions
(i.e., $T_{ij}=E_{ij}+E_{ji}$ if $i\neq j$ and $E_{ij}$ if $i=j$),
and $u_{ij}=\svec(E_{ij})$.
\item $N:\R^{d^{2}}\to\R^{d^{2}}$ is the linear operator that maps $\vec(A)$
to $\vec\bpar{\half(A+A^{\T})}$ for a matrix $A\in\Rdd$.
\item $L:\R^{d_{s}}\to\R^{d^{2}}$ is the linear operator that maps $\vec(A)$
to $\svec(A)$ for a matrix $A\in\Rdd$. It can be written as $L=\sum_{i\geq j}u_{ij}\vec(E_{ij})^{\T}$. 
\end{itemize}
\end{defn}

\begin{lem}
[\citet{magnus1980elimination}] \label{lem:MNL-properties} Let
$M,N,L$ be matrices in Definition~\ref{def:linearOperators}.
\begin{itemize}
\item (Lemma 2.1) $N=N^{\T}=N^{2}$ and $N(A\otimes A)=(A\otimes A)N$ for
any $d\times d$ matrix $A$.
\item (Lemma 3.5) $MLN=N$.
\end{itemize}
\end{lem}


\subsubsection{Analysis of a self-concordant metric for the PSD cone \label{subsec:scBasicMetric}}

We first examine properties of the metric defined by the Hessian of
self-concordant barrier $\phi(X)=-\log\det X$ (see \citet[Theorem 4.3.3]{nesterov2003introductory}
for self-concordance). In this case, its Hessian and inverse have
clean formulas. 
\begin{prop}
\label{prop:metricFormula} Let $\grad_{X}^{2}\phi(X)=-\grad_{x}^{2}\log\det(\svec^{-1}(x))\in\R^{d_{s}\times d_{s}}$
for $X\in\psd$. Then,
\begin{align*}
\hess\phi(X) & =M^{\T}(X^{-1}\otimes X^{-1})M=M^{\T}(X\otimes X)^{-1}M\,,\\
\bpar{\hess\phi(X)}^{-1} & =M^{\dagger}(X\otimes X)\bpar{M^{\dagger}}^{\T}=LN(X\otimes X)NL^{\T}\,,
\end{align*}
where $M^{\dagger}=(M^{\T}M)^{-1}M^{\T}\in\R^{d_{s}\times d^{2}}$
is the Moore-Penrose inverse of $M\in\R^{d^{2}\times d_{s}}$.
\end{prop}

We defer the proof to Appendix~\ref{app:matrixCalculus}. We remark
that as an immediate corollary to this, the local norm of $h\in\R^{d_{s}}$
with metric $\hess\phi(X)$ is 
\[
\snorm h_{X}^{2}=\svec(H)^{\T}M^{\T}(X^{-1}\otimes X^{-1})M\svec(H)\underset{\text{(i)}}{=}\tr(HX^{-1}HX^{-1})\eqqcolon\snorm H_{X}^{2}\,,
\]
where (i) follows from $\vec=M\circ\svec$ (Definition~\ref{def:linearOperators})
and $\tr(DB^{\T}A^{\T}C)=\vec(A)^{\T}(B\otimes C)\vec(D)$ (Lemma~\ref{lem:Kronecker}). 

\paragraph{Symmetry.}
\begin{lem}
[$\onu$-symmetry] \label{lem:logdet-symm}For $X\in K=\psd$, the
barrier $\phi(X)=-\log\det X$ is $d$-symmetric.
\end{lem}

\begin{proof}
For $X\in K$, pick any $Y\in K\cap(2X-K)$, and define a symmetric
matrix $H:=Y-X$. Since $Y\in K$ and $2X-Y\in K$, we have $X+H\in K$
and $X-H\in K$. Thus,
\[
-I\preceq X^{-1/2}HX^{-1/2}\preceq I\,,
\]
and the magnitude of each eigenvalue $\{\lda_{i}\}_{i=1}^{d}$ of
$X^{-1/2}HX^{-1/2}$ is bounded by $1$. Hence,
\[
\snorm H_{X}^{2}=\tr(X^{-1}HX^{-1}H)=\snorm{X^{-1/2}HX^{-1/2}}_{F}^{2}\leq\sum_{i=1}^{d}\lda_{i}^{2}\leq d\,.\qedhere
\]
\end{proof}

\paragraph{Convexity of log-determinant of Hessian and SSC.}

Next, the convexity of the log-determinant of $\hess\phi$ can be
checked via properties of Kronecker products. See \S\ref{proof:psd-convex-ssc}
for the proof.
\begin{prop}
[Convexity of log-determinant of Hessian] \label{prop:convex-logdet}
$\log\det(\hess\phi(\cdot))$ is convex.
\end{prop}

We move onto SSC of $d\phi(X)$.
\begin{lem}
\label{lem:logdet-scaling} For $\psi_{X}:=\sup_{H\in\mbb S^{d}}\snorm{(\hess\phi(X))^{-1/2}\Dd^{3}\phi(X)[H]\,(\hess\phi(X))^{-1/2}}_{F}/\snorm H_{X}$,
we have
\[
\sqrt{2(d+1)}\leq\psi_{X}\leq2\sqrt{d}\,.
\]
\end{lem}

We present the proof in \S\ref{proof:psd-convex-ssc}. This result
informs us of the best possible scaling of $\phi$ that ensures SSC.
Recall that if $g$ satisfies $\snorm{g^{-1/2}\Dd g[h]g^{-1/2}}_{F}\leq2\alpha\norm h_{g}$
for $\alpha>0$, then $\alpha^{2}g$ is SSC. We remark that the scaling
of $d$ is obviously better than the trivial scaling of $d_{s}=\Theta(d^{2})$.
\begin{cor}
[Strong self-concordance] \label{cor:logdet-ssc} A function $d\phi$
is a strongly self-concordant barrier for $\psd$. Moreover, the scaling
factor of $d$ cannot be further improved.
\end{cor}


\paragraph{Strongly lower trace self-concordance.}

SLTSC of $\phi$ can be easily checked by noting $g(X)[H,H]=\tr(X^{-1}HX^{-1}H)$
and using the chain rule. See the details in \S\ref{proof:psd-sltsc}.
\begin{lem}
[SLTSC] \label{lem:logdet-sltsc}$\Dd^{2}g(X)[H,H]\succeq0$ for
any $X\in\intk$ and $H\in\mathbb{S}^{d}$.
\end{lem}


\paragraph{Average self-concordance.}

In establishing ASC, we find an interesting connection to a \emph{Gaussian
orthogonal ensemble} (GOE), one of the main objects studied in the
random matrix theory. We prove the following lemmas and explain challenges
when extending our arguments to SASC in \S\ref{proof:psd-asc}.
\begin{lem}
\label{lem:conn-to-goe} For $d_{s}=\frac{d(d+1)}{2}$ and $\svec(H)\sim\ncal\bpar{0,\frac{r^{2}}{d_{s}}g(X)^{-1}}$,
$\frac{\sqrt{d_{s}d}}{r}X^{-1/2}HX^{-1/2}$ is a GOE.
\end{lem}

\begin{lem}
[ASC] \label{lem:logdet-asc} $-d\,\log\det X$ is ASC.
\end{lem}


\subsection{Logarithm, exponential, entropy, and $\ell_{p}$-norm (power function)}

\paragraph{Logarithm in potentials.}

Consider $Q_{1}=\{(x,t)\in\R^{2}:-\log x\leq t,x>0\}$. As $f(\cdot)=-\log(\cdot)$
is convex on $\R_{+}$ and satisfies the condition in Lemma~\ref{lem:tool-convex}
with $\beta=2$ and $\gamma=6$,
\[
F(x,t)=-\log(t+\log x)-36\log x
\]
is a highly $37$-self concordant barrier for $Q_{1}$. Therefore,
$2F$ is SSC and SLTSC with $\onu=\mc O(1)$.
\begin{lem}
[Logarithm] Consider the direct product of level sets
\[
K=\prod_{i=1}^{d}\{(x_{i},t_{i})\in\R^{2}:-\log x_{i}\leq t_{i},\,x_{i}>0\}\,,
\]
and let $\phi(x,t)=-\sum_{i=1}^{d}\bpar{\log(t_{i}+\log x_{i})+36\log x_{i}}$
and $g=2\hess\phi$.
\begin{itemize}
\item $\nu,\,\onu=\mc O(d)$.
\item SSC and SLTSC.
\item $d\,\hess\phi$ is SASC.
\end{itemize}
\end{lem}

\begin{proof}
For $i\in[d]$, let $Q_{i}=\{(x_{i},t_{i})\in\R^{2}:-\log x_{i}\leq t_{i},\,y_{i}>0\}$
and $F_{i}(x_{i},t_{i})$ be the self-concordant barrier above. Note
that $2F_{i}$ is SSC and SLTSC. By Lemma~\ref{lem:ssc-direct} and
\ref{lem:sltsc-direct}, the Hessian of $F(x,t):=2\sum_{i=1}^{d}F_{i}(x_{i},t_{i})$
is SSC and SLTSC. The last item on SASC follows from Lemma~\ref{lem:hsc-to-sasc}.
\end{proof}

\paragraph{Exponent in potentials.}

Consider $Q_{2}=\{(x,t)\in\R^{2}:e^{x}\leq t\}=\{(x,t)\in\R^{2}:t>0,\,x\leq\log t\}$.
As $f(t)=\log t$ is concave and satisfies the condition in Lemma~\ref{lem:tool-concave}
with $\beta=2$ and $\gamma=6$,
\[
F(x,t)=-\log(\log t-x)-36\log t
\]
is a highly $37$-self concordant barrier for $Q_{2}$. Therefore,
$2F$ is SSC and SLTSC with $\onu=\mc O(1)$.
\begin{lem}
[Exponential] Consider the direct product of level sets
\[
K=\prod_{i=1}^{d}\{x_{i},t_{i})\in\R^{2}:\exp(x_{i})\leq t_{i}\}\,,
\]
and let $\phi(x,t)=-\sum_{i=1}^{d}(\log(\log t_{i}-x_{i})+36\log t_{i})$
and $g=2\hess\phi$.
\begin{itemize}
\item $\nu,\,\onu=\mc O(d)$.
\item SSC and SLTSC.
\item $d\,\hess\phi$ is SASC.
\end{itemize}
\end{lem}

\begin{proof}
For $i\in[d]$, let $Q_{i}=\{(x_{i},t_{i})\in\R^{2}:e^{x_{i}}\leq t_{i}\}$
and $F_{i}(x_{i},t_{i})$ be the self-concordant barrier above. Note
that $2F_{i}$ is SSC and SLTSC. By Lemma~\ref{lem:ssc-direct} and~\ref{lem:sltsc-direct},
the Hessian of $F(x,t):=2\sum_{i=1}^{d}F_{i}(x_{i},t_{i})$ is SSC
and SLTSC. The last item on SASC follows from Lemma~\ref{lem:hsc-to-sasc}.
\end{proof}

\paragraph{Entropy in potentials.}

Consider $Q_{3}=\{(x,t)\in\R^{2}:x\geq0,\,t\geq x\log x\}$. Note
that $f(x)=x\log x$ is convex on $\{x>0\}$ and satisfies the condition
in Lemma~\ref{lem:tool-convex} with $\beta=1$ and $\gamma=2$.
Hence,
\[
F(x,t)=-\log(t-x\log x)-36\log x
\]
is a highly $5$-self concordant barrier for $Q_{3}$. Therefore,
$2F$ is SSC and SLTSC with $\onu=\mc O(1)$.
\begin{lem}
[Entropy] Consider the direct product of level sets
\[
K=\prod_{i=1}^{d}\{(x_{i},t_{i})\in\R^{2}:x_{i}\geq0,\,t_{i}\geq x_{i}\log x_{i}\}\,,
\]
and let $\phi(x,t)=-\sum_{i=1}^{d}\bpar{\log(t_{i}-x_{i}\log x_{i})+36\log x_{i}}$
and $g=2\hess\phi$.
\begin{itemize}
\item $\nu,\,\onu=\mc O(d)$.
\item SSC and SLTSC.
\item $d\,\hess\phi$ is SASC.
\end{itemize}
\end{lem}

\begin{proof}
For $i\in[d]$, let $Q_{i}=\{(x_{i},t_{i})\in\R^{2}:x_{i}\geq0,\,t_{i}\geq x_{i}\log x_{i}\}$
and $F_{i}(x_{i},t_{i})$ be the self-concordant barrier above. Note
that $2F_{i}$ is SSC and SLTSC. By Lemma~\ref{lem:ssc-direct} and~\ref{lem:sltsc-direct},
the Hessian of $F(x,t):=2\sum_{i=1}^{d}F_{i}(x_{i},t_{i})$ is SSC
and SLTSC. The last item on SASC follows from Lemma~\ref{lem:hsc-to-sasc}.
\end{proof}

\paragraph{$\ell_{p}$-norm (power function).}

We start with the power functions. For $p\geq1$, consider $Q_{4}=\{(x,t)\in\R^{2}:t\geq\max(0,x)^{p}\}=\{(x,t)\in\R^{2}:t\geq0,\,x\leq t^{1/p}\}$.
Note that $f(t)=t^{1/p}$ is concave on $t>0$ and satisfies the condition
in Lemma~\ref{lem:tool-concave} with $\beta=2$ and $\gamma=6$.
Hence,
\[
F_{4}(x,t)=-\log(t^{1/p}-x)-36\log t
\]
is a highly $37$-self-concordant barrier for $Q_{4}$. Similarly,
$F_{5}(t,x)=-\log(t^{1/p}+x)-36\log t$ is a highly $37$-self concordant
barrier for the convex set $Q_{5}=\{(x,t)\in\R^{2}:t\geq\max(0,-x)^{p}\}$.
Since the convex set $Q_{6}=\{(x,t)\in\R^{2}:t\geq|x|^{p}\}$ is equal
to $Q_{4}\cap Q_{5}$, the sum of $F_{4}+F_{5}$, which is 
\[
F_{6}(x,t)=-\log(t^{2/p}-x^{2})-72\log t
\]
is a highly $72$-self-concordant barrier for $Q_{6}$. Hence, $2F$
is SSC and SLTSC with $\onu=\mc O(1)$.
\begin{lem}
[$\ell_p$-norm] Consider the direct product of level sets $K=\prod_{i=1}^{d}\{(x_{i},t_{i})\in\R^{2}:\Abs{x_{i}}^{p}\leq t_{i}\}$,
and let $\phi(x,t)=-\sum_{i=1}^{d}\bpar{\log(t_{i}^{2/p}-x_{i}^{2})+72\log t_{i}}$
and $g=2\hess\phi$.
\begin{itemize}
\item $\nu,\,\onu=\mc O(d)$.
\item SSC and SLTSC.
\item $d\,\hess\phi$ is SASC.
\end{itemize}
\end{lem}

\begin{proof}
Consider a highly $72$-self-concordant barrier $F_{i}$ above for
$\{(x_{i},t_{i}):|x_{i}|^{p}\leq t_{i}\}$ for $i\in[d]$. Note that
$2F_{i}$ is SSC and SLTSC. By Lemma~\ref{lem:ssc-direct} and~\ref{lem:sltsc-direct},
the Hessian of $F(x,t):=2\sum_{i=1}^{d}F_{i}(x_{i},t_{i})$ is SSC
and SLTSC. The last item on SASC follows from Lemma~\ref{lem:hsc-to-sasc}.
\end{proof}

