
\appendix

\section{Algebraic identities}

Here we collect useful algebraic identities related to trace, vectorization,
Kronecker product and Hadamard product. 
 
\begin{lem}
[Kronecker product] \label{lem:Kronecker} For $A,B,C,D\in\R^{n\times n}$
and $M$ in Definition~\ref{def:linearOperators},
\begin{itemize}
\item $(A\otimes B)\vec(C)=\tr\Par{BCA^{\top}}$.
\item $\vec(A)^{\top}\Par{B\otimes C}\vec(D)=\tr\Par{DB^{\top}A^{\top}C}$.
\item $(A\otimes B)(C\otimes D)=AC\otimes BD$.
\item $(A\otimes B)^{-1}=A^{-1}\otimes B^{-1}$.
\item $(A\otimes B)^{\top}=A^{\top}\otimes B^{\top}$.
\item $\tr\Par{A\otimes B}=\tr(A)\tr(B)$.
\item $\det(M^{\top}(A\otimes A)M)=2^{n(n-1)/2}\Par{\det A}^{n+1}$.
\end{itemize}
\end{lem}

\begin{lem}
[Hadamard product] \label{lem:Hadamard} Let $A,B,C,D\in\R^{n\times n}$,
$x,y\in\Rn$, and $D_{1},D_{2}\in\Rnn$ be diagonal matrices.
\begin{itemize}
\item $(A\circ B)y=\diag(A\Diag(y)B^{\top})$.
\item $x^{\top}(A\circ B)y=\tr\Par{\Diag(x)A\Diag(y)B^{\top}}$.
\item $D_{1}(A\hada B)=(D_{1}A)\hada B=A\hada(D_{1}B)$.
\item $(A\hada B)D_{2}=(AD_{2})\hada B=A\hada(BD_{2})$.
\item $(A\otimes B)\circ(C\otimes D)=(A\circ C)\otimes(B\circ D)$.
\end{itemize}
\end{lem}


\section{Matrix calculus \label{app:matrixCalculus}}

Let $g(x):\Rn\to\R^{n\times n}$ be a matrix function. Its gradient
at $x$, denoted by $Dg(x)$, is the third-order tensor defined by
$(Dg(x))_{ijk}=\frac{\del g_{ij}(x)}{\del x_{k}}$. Unless specified
otherwise, the multiplication between higher-order tensors and a matrix
of size $n\times n$ is running over $i,j$-entries. For instance,
for a matrix $M\in\R^{n\times n}$ the product $Dg(x)M$ is the third-order
tensor defined by
\[
(Dg(x)M)_{\cdot,\cdot,k}=(Dg(x))_{\cdot,\cdot,k}M\text{ for each }k\in[n].
\]
In the same way, the trace of higher-order tensors is applied to a
matrix spanned by $i,j$-entries, i.e.,
\[
\Par{\tr\Par{Dg(x)}}_{k}=\tr\Par{\Par{Dg(x)}_{\cdot,\cdot,k}}.
\]

Now define $\vphi(x)=\log\det g(x):\Rn\to\R$. Its gradient is
\begin{equation}
\grad\vphi(x)=D\log\det g(x)=\tr\Par{g(x)^{-1}Dg(x)},\label{eq:gradLogDet}
\end{equation}
and this indicates that its directional derivative in $h\in\Rn$ is
$\grad\vphi(x)\cdot h=\tr\Par{g(x)^{-1}Dg(x)[h]}$. To compute the
Hessian of $\vphi$, we note that
\begin{equation}
D(g^{-1})(x)=-g(x)^{-1}Dg(x)g(x)^{-1}.\label{eq:diffInverse}
\end{equation}
Using this and the product rule, we have
\begin{align}
\hess\vphi(x) & =D\tr\Par{g(x)^{-1}Dg(x)}\nonumber \\
 & =-\tr\Par{g(x)^{-1}Dg(x)g(x)^{-1}Dg(x)}+\tr\Par{g(x)^{-1}D^{2}g(x)}\nonumber \\
 & =\tr\Par{g(x)^{-1}D^{2}g(x)}-\norm{g(x)^{-\half}Dg(x)g(x)^{-\half}}_{F}^{2},\label{eq:hessLogDet}
\end{align}
where $D^{2}g(x)$ is the fourth-order tensor defined by $(D^{2}g(x))_{ijkl}=\frac{\del g(x)_{ij}}{\del x_{k}\del x_{l}}$.

We now prove Lemma~\ref{lem:metricFormula}, providing formulas of
the Hessian and its inverse of $\phi(X)=-\log\det X$ for a matrix
$X\in\psd$.
\begin{proof}
[Proof of Lemma \ref{lem:metricFormula}] By setting $g(X)=X$ and
$\phi(X)=-\vphi(X)$ above, (\ref{eq:hessLogDet}) implies that for
a symmetric matrix $H\in\S^{n}$
\begin{align}
\hess\phi(X)[H,H] & =\tr\Par{X^{-1}HX^{-1}H}\label{eq:2ndDiffLogDet}\\
 & =\vec{(}H)^{\top}\Par{X^{-1}\otimes X^{-1}}\vec{(}H)=\vec{(}H)^{\top}\Par{X\otimes X}^{-1}\vec{(}H)\nonumber 
\end{align}
where the last line follows from Lemma~\ref{lem:Kronecker}. When
representing $X$ and $H$ in $\R^{d}$ space with notations $x:=\svec(X)$
and $h:=\svec(H)$, the definition of $M$ (see Definition~\ref{def:linearOperators})
turns the formula above into
\[
\hess\phi(x)[h,h]=h^{\top}M^{\top}(X\otimes X)^{-1}Mh,
\]
and thus $g_{X}:=\nabla_{x}^{2}\phi(x)=\nabla_{X}^{2}\phi(X)$ is
equal to $M^{\top}(X\otimes X)^{-1}M$. The formula of the inverse,
$g_{X}^{-1}=M^{\dagger}(X\otimes X)M^{\dagger\top}$, is immediate
from \cite{magnus1980elimination}, and another part follows from
$M^{\dagger}=LN$ and $N^{\top}=N$ (Lemma 3.6 and Lemma 2.1 in \cite{magnus1980elimination}).
\end{proof}

\section{Remaining proofs}

\subsection{Logarithmic barrier \label{app:subsec:logBarrier}}

Here we collect details used in the paper that involve calculus of
the logarithmic barrier, $\phi_{\log}(X):=-\sum_{i=1}^{m}\log\Par{\inner{A_{i},X}-b_{i}}$.
Recall the metric $g$ defined by the Hessian of $\phi_{\log}$ is
given by
\begin{align*}
g(X) & =M^{\top}\left[\begin{array}{ccc}
\vec(A_{1}) & \cdots & \vec(A_{m})\end{array}\right]S_{X}^{-2}\left[\begin{array}{c}
\vec(A_{1})^{\top}\\
\vdots\\
\vec(A_{m})^{\top}
\end{array}\right]M\\
 & =M^{\top}A^{\top}S_{X}^{-2}AM,
\end{align*}
where $S_{X}=\Diag\Par{\inner{A_{i},X}-b_{i}}\in\R^{m\times m}$ and
$A^{\top}=\left[\begin{array}{ccc}
\vec(A_{1}) & \cdots & \vec(A_{m})\end{array}\right]\in\R^{n^{2}\times m}$.   Since we work on $\S^{n}$ and $\R^{d}$ simultaneously, we
consider its vector version (i.e., $g(x)=A^{\top}S_{x}^{-2}A$ for
$x\in\R^{d}$) for simplicity and then translate it into one in our
setting. We recall notations that appear in our computation:
\begin{itemize}
\item $A_{x}=S_{x}^{-1}A\in\R^{m\times d}$.
\item $s_{x}=\diag(S_{x})\in\R^{m}$.
\item $s_{x,h}=A_{x}h\in\R^{m}$ and $S_{x,h}=\Diag(s_{x,h})\in\R^{m\times m}$.
We drop $x$ if $x$ is clear from the context.
\end{itemize}
Going forward, we use $h$ to denote a vector in $\R^{n}$.
\begin{claim}
\label{claim:1stDiffSlack} $DS_{x}[h]=\Diag(Ah)$ and $DS_{x}^{-1}[h]=-S_{x}^{-1}S_{x,h}$.
\end{claim}

\begin{proof}
The first is obvious from differentiation of $S_{x}=\Diag(Ax-b)$
with respect to $x$. For the second,
\begin{align*}
DS_{x}^{-1}[h] & =-S_{x}^{-1}DS_{x}[h]S_{x}^{-1}=-S_{x}^{-1}\Diag(Ah)S_{x}^{-1}\\
 & =-\Diag(A_{x}h)S_{x}^{-1}=-S_{x}^{-1}\Diag(A_{x}h)\\
 & =-S_{x}^{-1}S_{x,h},
\end{align*}
where $S_{x}^{-1}\Diag(Ah)=\Diag(A_{x}h)$ and $\Diag(A_{x}h)S_{x}^{-1}=S_{x}^{-1}\Diag(A_{x}h)$
hold as all of them are diagonal matrices.
\end{proof}
\begin{claim}
\label{claim:diffLogBarrier} $Dg(x)[h]=-2A_{x}^{\top}S_{x,h}A_{x}$
and $D^{2}g(x)[h,h]=6A_{x}^{\top}S_{x,h}^{2}A_{x}\succeq0$. In the
PSD setting with $A_{X}:=S_{X}^{-1}A$, this becomes $Dg(X)[H]=-2M^{\top}A_{X}^{\top}\Diag\Par{A_{X}\vec(H)}A_{X}M$
and $D^{2}g(X)[H,H]=6M^{\top}A_{X}^{\top}\Diag(A_{X}\vec(H))^{2}A_{X}M$.
\end{claim}

\begin{proof}
Due to $g(x)=A_{x}^{\top}A_{x}$,
\begin{align*}
Dg(x)[h] & =D(A^{\top}S_{x}^{-2}A)[h]=A^{\top}DS_{x}^{-2}[h]A\\
 & =-2A^{\top}S_{x}^{-3}DS_{x}[h]A=-2A_{x}^{\top}S_{x}^{-1}\Diag(Ah)A_{x}\\
 & =-2A_{x}^{\top}S_{x,h}A_{x}.
\end{align*}

For the second-order directional derivative,
\begin{align*}
Dg^{2}(x)[h,h] & =-2D(A_{x}^{\top}S_{x,h}A_{x})[h]=-2D(A^{\top}S_{x}^{-3}\Diag(Ah)A)[h]\\
 & =6A^{\top}S_{x}^{-4}DS_{x}[h]\Diag(Ah)A\\
 & =6A_{x}^{\top}S_{x,h}^{2}A_{x}.\qedhere
\end{align*}
\end{proof}
Note that $D^{2}g(x)[h,h]=6A_{x}^{\top}S_{x,h}^{2}A_{x}\succeq0$.
We now provide the deferred proof of Lemma~\ref{lem:paramsBarrier}-1.
\begin{proof}
[Proof of Lemma \ref{lem:paramsBarrier}-1] By putting $D_{x}=I_{m}$
into Lemma~\ref{lem:helper4Diagonal}-1, we have
\[
\norm{g(x)^{-\half}Dg(x)[h]g(x)^{-\half}}_{F}\leq2\sqrt{\max_{i\in[m]}\sigma(A_{x})_{i}}\norm h_{g(x)}.
\]
As $P_{x}$ is the orthogonal projection, $P_{x}\preceq I$ and $\sigma(A_{x})\leq1$.
Thus, $\norm{g(x)^{-\half}Dg(x)[h]g(x)^{-\half}}_{F}\leq2\norm h_{g(x)}$,
deriving strong self-concordance of the logarithmic barriers. 

For the $\onu$-symmetry, we note that the first part (i.e., $\dcal_{g}^{1}(x)\subset K\cap(2x-K)$)
follows from Lemma~\ref{lem:symmetricLeftpart}. The second part
is immediate from $\onu=\tr\Par{I_{m}}=m$ and Lemma~\ref{lem:helper4Diagonal}-3.
\end{proof}

\subsection{Volumetric barrier \label{app:subsec:volBarrier}}

\cite{vaidya1996new} introduced the \emph{volumetric barrier} for
a convex region $Ax\geq b$ defined by 
\[
\phi_{\vol}=\half\log\det\hess\phi_{\log}=\half\log\det A_{x}^{\top}A_{x}.
\]
We collect computational preliminaries regarding to the volumetric
barrier and then move onto the approximate volumetric barrier. 

\paragraph{Volumetric barrier.}
\begin{claim}
$\grad\phi_{\vol}(x)=-A_{x}^{\top}\sigma_{x}$ and $\hess\phi_{\vol}(x)=A_{x}^{\top}\Par{3\Sigma_{x}-2P_{x}^{(2)}}A_{x}$.
\end{claim}

\begin{proof}
Let $g(x)=\hess\phi_{\log}=A_{x}^{\top}A_{x}$. Note that by Claim~\ref{claim:diffLogBarrier}
\begin{align*}
\grad\phi_{\vol}(x)[h] & =-\tr\Par{g^{-1}A_{x}^{\top}S_{x,h}A_{x}}\\
 & =-\tr\bigg(\underbrace{A_{x}g^{-1}A_{x}^{\top}}_{=P_{x}}S_{x,h}\bigg)\\
 & =-\tr\Par{P_{x}S_{x,h}}=-\tr\Par{P_{x}S_{x,h}I_{m}I_{m}}\\
 & \underset{\text{(i)}}{=}-\bm{1}^{\top}(P_{x}\circ I_{m})S_{x,h}=-1^{\top}\Sigma_{x}A_{x}h\\
 & =-h^{\top}A_{x}^{\top}\sigma_{x},
\end{align*}
where we used Lemma~\ref{lem:Hadamard} in (i). 

For the Hessian of $\phi_{\vol}$,
\[
\hess\phi_{\vol}(x)[h,h]=\half\Par{\tr\Par{g^{-1}D^{2}g[h,h]}-\tr\Par{g^{-1}Dg[h]g^{-1}Dg[h]}}.
\]
For the first term, by Claim~\ref{claim:diffLogBarrier}
\begin{align*}
\half\tr\Par{g^{-1}Dg[h]g^{-1}Dg[h]} & =2\tr\Par{g^{-1}A_{x}^{\top}S_{x,h}A_{x}g^{-1}A_{x}^{\top}S_{x,h}A_{x}}=2\tr\Par{P_{x}S_{x,h}P_{x}S_{x,h}}\\
 & \underset{\text{(i)}}{=}2(A_{x}h)^{\top}\Par{P_{x}\circ P_{x}}(A_{x}h)=2h^{\top}A_{x}^{\top}P_{x}^{(2)}A_{x}h,
\end{align*}
where we used Lemma~\ref{lem:Hadamard} in (i). For the second term,
Claim~\ref{claim:diffLogBarrier} leads to
\begin{align*}
\half\tr\Par{g^{-1}D^{2}g[h,h]} & =3\tr\Par{g^{-1}A_{x}^{\top}S_{x,h}^{2}A_{x}}=3\tr\Par{P_{x}S_{x,h}IS_{x,h}}\\
 & =3h^{\top}A_{x}^{\top}\Par{P_{x}\circ I}A_{x}h=3h^{\top}A_{x}^{\top}\Sigma_{x}A_{x}h.
\end{align*}
Putting these two together, we have
\[
D^{2}\phi_{\vol}(x)[h,h]=h^{\top}A_{x}^{\top}\Par{3\Sigma_{x}-2P_{x}^{(2)}}A_{x}h
\]
and thus
\[
\hess\phi_{\vol}(x)=A_{x}^{\top}(3\Sigma_{x}-2P_{x}^{(2)})A_{x}.\qedhere
\]
\end{proof}
\begin{lem}
\label{lem:schurProjection} $P_{x}^{(2)}\preceq\Sigma_{x}$, so $A_{x}^{\top}\Sigma_{x}A_{x}\preceq\hess\phi_{\vol}(x)\preceq3A_{x}^{\top}\Sigma_{x}A_{x}$.
\end{lem}

\begin{proof}
Note that $\Sigma_{x}=P_{x}\circ I$. Let us show that $0\leq h^{\top}P_{x}\circ(I-P_{x})h$
for $h\in\Rn$. Since $P_{x}$ and $I-P_{x}$ are both orthogonal
projections matrices, for $C:=P_{x}H(I-P_{x})$ and $H=\Diag(h)$,
\begin{align*}
h^{\top}P_{x}\circ(I-P_{x})h & =\tr\Par{HP_{x}H(I-P_{x})}\\
 & =\tr\Par{(I-P_{x})HP_{x}P_{x}H(I-P_{x})}=\tr(C^{\top}C)\geq0.\qedhere
\end{align*}
\end{proof}


\paragraph{Approximate volumetric metric.}

The approximate volumetric metric is defined by
\[
g(x)=A_{x}^{\top}\Sigma_{x}A_{x},
\]
which serves as a good approximation of $\hess\phi_{\vol}$ due to
$\Sigma_{x}\preceq3\Sigma_{x}-2P_{x}^{(2)}\preceq3\Sigma_{x}$. We
begin with recalling computational results on the leverage scores:
\begin{lem}
[\cite{lee2019solving}] \label{lem:usefulFactLeverage} Let $\Sigma_{x}=\Sigma(A_{x})\in\R^{m\times m},g(x)=A_{x}^{\top}\Sigma_{x}A_{x}$,
and $h\in\Rn$.
\begin{itemize}
\item \textup{(Lemma 26)} $\max_{i\in[m]}\frac{\sigma\Par{\Sigma_{x}^{1/2}A_{x}}_{i}}{\Par{\Sigma_{x}}_{ii}}\leq2m^{\frac{1}{2}}$.
\item \textup{(Lemma 33)} $\norm{A_{x}h}_{\Sigma_{x}}=\norm h_{g(x)}$
and $\norm{A_{x}h}_{\infty}\leq\sqrt{2}m^{\frac{1}{4}}\norm h_{g(x)}$.
\item \textup{(Lemma 34)} $\norm{\Sigma_{x}^{-1}\diag\Par{D\Sigma_{x}[h]}}_{\Sigma_{x}}\leq2\norm h_{g(x)}$.
\end{itemize}
\end{lem}

We are now ready to prove the second item of Lemma \ref{lem:paramsBarrier}.
\begin{proof}
[Proof of Lemma~\ref{lem:paramsBarrier}-2] Let us set $D_{x}$
to $\Sigma_{x}=\Sigma(A_{x})$ in Lemma~\ref{lem:helper4Diagonal}.
By Lemma~\ref{lem:usefulFactLeverage}, we have
\begin{align*}
\max_{i}\Par{\frac{\sigma\Par{\sqrt{D_{x}}A_{x}}_{i}}{(D_{x})_{ii}}} & \leq2\sqrt{m},\\
\sum_{i=1}^{m}\Par{D_{x}^{-1}}_{ii}(DD_{x}[h])_{i}^{2} & =\norm{\Sigma_{x}^{-1}\diag\Par{D\Sigma_{x}[h]}}_{\Sigma_{x}}^{2}\\
 & \leq4\norm h_{g(x)}^{2}.
\end{align*}
Thus,
\begin{align*}
\norm{g(x)^{-\half}Dg(x)[h]g(x)^{-\half}}_{F}^{2} & \leq4\max_{i}\Par{\frac{\sigma\Par{\sqrt{D_{x}}A_{x}}_{i}}{(D_{x})_{ii}}}\cdot\Par{\norm h_{g(x)}^{2}+\sum_{i=1}^{m}\Par{D^{-1}}_{ii}(DD_{x}[h])_{i}^{2}}\\
 & \leq40\sqrt{m}\norm h_{g(x)}^{2}.
\end{align*}
For the symmetry parameter, $\norm{A_{x}(y-x)}_{\infty}\leq\sqrt{\max_{i\in[m]}\frac{\sigma\Par{\sqrt{D_{x}}A_{x}}_{i}}{D_{x,i}}}\leq m^{1/4}$
for $y\in\dcal_{g}^{1}(x)$ by Lemma~\ref{lem:helper4Diagonal}-2.
Also, Lemma~\ref{lem:helper4Diagonal}-3 implies that $y$ with $\norm{A_{x}(y-x)}_{\infty}\leq1$
is contained in $\dcal_{g}^{\sqrt{\tr(D_{x})}}(x)$, where
\[
\tr\Par{D_{x}}=\tr\Par{P_{x}}=\tr\Par{A_{x}(A_{x}^{\top}A_{x})^{-1}A_{x}}=\tr\Par{I_{n}}=n.
\]
Therefore, $\tilde{g}(x):=40\sqrt{m}g(x)=40\sqrt{m}A_{x}^{\top}\Sigma_{x}A_{x}$
is strongly self-concordant with the symmetry parameter $\onu=O(\sqrt{m}n)$.
\end{proof}

\subsection{Derivatives of matrices \label{app:subsec:derivativeMatrices}}

In this section, we provide details in computing derivatives of leverage
scores, orthogonal projections, and so on.

\propCalculusLeverage*
\begin{proof}
The first and second items follow from Lemma 2.2 and 2.3 of \cite{gatmiry2023sampling}.
From these formulas and the definition of $\Lambda_{x}$,
\begin{align*}
 & D\Lambda_{x}[h]\\
 & =D\Sigma_{x}[h]-DP_{x}[h]\circ P_{x}-P_{x}\circ DP_{x}[h]\\
 & =-2\Diag(\Lambda_{x}s_{x,h})-\Par{-P_{x}S_{x,h}-S_{x,h}P_{x}+2P_{x}S_{x,h}P_{x}}\circ P_{x}-P_{x}\circ\Par{-P_{x}S_{x,h}-S_{x,h}P_{x}+2P_{x}S_{x,h}P_{x}}\\
 & \underset{\text{(i)}}{=}-2\Diag(\Lambda_{x}s_{x,h})+2P_{x}\circ P_{x}S_{x,h}+2S_{x,h}P_{x}\circ P_{x}-2(P_{x}S_{x,h}P_{x})\circ P_{x}-2P_{x}\circ(P_{x}S_{x,h}P_{x}),
\end{align*}
where in (i) we used $D(A\hada B)=(DA)\circ B=A\hada(DB)$ and $(A\hada B)D=(AD)\hada B=A\circ(BD)$\footnote{This property allows us to write $DA\hada B$ without parenthesis.}
for a diagonal matrix $D\in\Rnn$ (Lemma~\ref{lem:Hadamard}). 

Using the first three formulas
\begin{align*}
 & D^{2}\Sigma_{x}[h,h]\\
 & =-2D\Diag(\Lambda_{x}s_{x,h})[h]\\
 & =-2\Diag(D\Lambda_{x}[h]s_{x,h})+2\Diag\Par{\Lambda_{x}S_{x,h}s_{x,h}}\\
 & =-2\Diag\Par{\Par{-2\Diag(\Lambda_{x}s_{x,h})+2P_{x}\circ P_{x}S_{x,h}+2S_{x,h}P_{x}\circ P_{x}-2(P_{x}S_{x,h}P_{x})\circ P_{x}-2P_{x}\circ(P_{x}S_{x,h}P_{x})}s_{x,h}}\\
 & \qquad+2\Diag\Par{\Lambda_{x}S_{x,h}s_{x,h}}\\
 & =4\Diag\Par{\cred{\Lambda_{x}}s_{x,h}}\cblue{S_{x,h}}-4\Diag\Par{P_{x}\circ P_{x}S_{x,h}s_{x,h}}-4\Diag\Par{S_{x,h}P_{x}\circ P_{x}s_{x,h}}\\
 & \qquad+4\Diag\Par{(P_{x}S_{x,h}P_{x})\circ P_{x}s_{x,h}}+4\Diag\Par{P_{x}\circ(P_{x}S_{x,h}P_{x})s_{x,h}}+2\Diag\Par{\cred{\Lambda_{x}}S_{x,h}s_{x,h}}\\
 & =4\Diag\Par{\cblue{S_{x,h}}\cred{(\Sigma_{x}-P_{x}\circ P_{x})}s_{x,h}}-4\Diag\Par{P_{x}\circ P_{x}S_{x,h}s_{x,h}}-4\Diag\Par{S_{x,h}P_{x}\circ P_{x}s_{x,h}}\\
 & \qquad+4\Diag\Par{(P_{x}S_{x,h}P_{x})\circ P_{x}s_{x,h}}+4\Diag\Par{P_{x}\circ(P_{x}S_{x,h}P_{x})s_{x,h}}+2\Diag\Par{\cred{(\Sigma_{x}-P_{x}\circ P_{x})}S_{x,h}s_{x,h}}\\
 & =\ccyan{4\Diag(S_{x,h}\Sigma_{x}s_{x,h})}-6\Diag\Par{P_{x}\circ P_{x}S_{x,h}s_{x,h}}-8\Diag\Par{S_{x,h}P_{x}\circ P_{x}s_{x,h}}\\
 & \qquad+4\Diag\Par{(P_{x}S_{x,h}P_{x})\circ P_{x}s_{x,h}}+4\Diag\Par{P_{x}\circ(P_{x}S_{x,h}P_{x})s_{x,h}}+\ccyan{2\Diag(\Sigma_{x}S_{x,h}s_{x,h})}\\
 & =\text{\ensuremath{\ccyan{6\Diag(S_{x,h}\Sigma_{x}s_{x,h})}}}-6\Diag\Par{\cblue{P_{x}\circ P_{x}S_{x,h}s_{x,h}}}-8\Diag\Par{\cblue{S_{x,h}P_{x}\circ P_{x}s_{x,h}}}\\
 & \qquad+4\Diag\Par{\cblue{(P_{x}S_{x,h}P_{x})\circ P_{x}s_{x,h}}}+4\Diag\Par{\cblue{P_{x}\circ(P_{x}S_{x,h}P_{x})s_{x,h}}}\\
 & \underset{\text{(i)}}{=}6S_{x,h}\Sigma_{x}\Diag\Par{s_{x,h}}-6\Diag\Par{\diag\Par{P_{x}S_{x,h}(P_{x}S_{x,h})^{\top}}}-8\Diag\Par{\diag\Par{S_{x,h}P_{x}S_{x,h}P_{x}^{\top}}}\\
 & \qquad+4\Diag\Par{P_{x}S_{x,h}P_{x}S_{x,h}P_{x}}+4\Diag\Par{P_{x}S_{x,h}\Par{P_{x}S_{x,h}P_{x}}^{\top}}\\
 & =6S_{x,h}\Sigma_{x}S_{x,h}-6\Diag\Par{P_{x}S_{x,h}^{2}P_{x}}-8\Diag\Par{S_{x,h}P_{x}S_{x,h}P_{x}}+8\Diag\Par{P_{x}S_{x,h}P_{x}S_{x,h}P_{x}},
\end{align*}
where in (i) we applied Lemma~\ref{lem:Hadamard}-1 to the terms
with blue. 

Applying the product rule to $\theta_{1}(x)=A_{x}^{\top}\Sigma_{x}A_{x}=A^{\top}S_{x}^{-2}\Sigma_{x}A,$
\begin{align*}
D\theta_{1}[h] & =-2A^{\top}S_{x}^{-3}\Sigma_{x}\Diag(Ah)A+A^{\top}S_{x}^{-2}D\Sigma_{x}[h]A\\
 & =-2A_{x}^{\top}\Sigma_{x}S_{x,h}A_{x}+A_{x}^{\top}D\Sigma_{x}[h]A_{x},\\
D^{2}\theta_{1}[h,h] & =6A_{x}^{\top}S_{x,h}\Sigma_{x}S_{x,h}A_{x}-2A_{x}^{\top}D\Sigma_{x}[h]S_{x,h}A_{x}-2A_{x}^{\top}S_{x,h}D\Sigma_{x}[h]A_{x}+A_{x}^{\top}D^{2}\Sigma_{x}[h,h]A_{x}\\
 & =6A_{x}^{\top}S_{x,h}\Sigma_{x}S_{x,h}A_{x}-4A_{x}^{\top}D\Sigma_{x}[h]S_{x,h}A_{x}+A_{x}^{\top}D^{2}\Sigma_{x}[h,h]A_{x}.
\end{align*}
The derivatives of $\theta_{2}$ simply follow from Claim \ref{claim:diffLogBarrier}.
\end{proof}

