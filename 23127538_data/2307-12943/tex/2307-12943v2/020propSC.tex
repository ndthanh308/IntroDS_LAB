
\section{Properties of self-concordance\label{sec:prop-SC}}

Self-concordance is a central notion in the theory of interior-point
methods for optimization (we refer interested readers to \cite{nesterov1994interior,nesterov2003introductory}).
In this section, we first recall basic properties of self-concordance
and then investigate those of strong self-concordance and lower trace
self-concordance, which are crucial to our analysis. In particular,
we establish sufficient conditions for local metrics under which the
sum of the metrics satisfies the assumptions in our general framework
for the mixing rate analysis of the $\dw$ (Theorem~\ref{thm:generalMixing}).

\paragraph{Self-concordance.}

We recall basic properties of self-concordance. Let $f_{1}$ and $f_{2}$
be self-concordant functions on a convex set $K\subset\Rn$. Let $\alpha>0$
be a scalar.
\begin{lem}
[\cite{nesterov2003introductory}] $ $
\begin{itemize}
\item (Theorem 4.1.1) $f_{1}+f_{2}$ is self-concordant on $K$.
\item (Corollary 4.1.2) $g=\hess(\alpha f_{1})$ satisfies $\norm{g(x)^{-\half}Dg(x)[h]g(x)^{-\half}}_{2}\leq\frac{2}{\sqrt{\alpha}}\norm h_{g(x)}$
for $h\in\Rn$ and $x\in K$.
\end{itemize}
\end{lem}

The following lemma ensures that the $\dw$ stays inside the convex
body.
\begin{lem}
[\cite{nesterov2003introductory}, Theorem 4.1.5] \label{lem:dikin-in-body}
$\dcal_{g}^{1}(x)\subset K$ for a convex body $K$ and $g=\hess f_{1}$.
\end{lem}

The following lemma states that self-concordant metrics are similar
for nearby points.
\begin{lem}
[\cite{nesterov2003introductory}, Theorem 4.1.6] \label{lem:scCloseness}
Given any self-concordant matrix function $g$ on $K\subset\Rn$ and
$x,y\in K$ with $\norm{x-y}_{g(x)}<1$, we have 
\[
\Par{1-\norm{x-y}_{g(x)}}^{2}g(x)\preceq g(y)\preceq\frac{1}{\Par{1-\norm{x-y}_{g(x)}}^{2}}g(x).
\]
\end{lem}


\paragraph{Strong self-concordance.}

We first show that strong self-concordance is also additive with a
scaling factor. We remark that the factor of $2$ below might be redundant.
\begin{lem}
\label{lem:sumStrongSC} If $g_{1}$ and $g_{2}$ are strongly self-concordant
matrix functions on $K_{1}$ and $K_{2}$ respectively, then $2(g_{1}+g_{2})$
is strongly self-concordant on $K_{1}\cap K_{2}$. 
\end{lem}

\begin{proof}
For fixed $x\in K_{1}\cap K_{2}$ and $h\in\Rn$, let $Dg_{i}:=Dg_{i}(x)[h]$
for $i=1,2$. Note that
\begin{align*}
 & \norm{(g_{1}+g_{2})^{-\half}D(g_{1}+g_{2})(g_{1}+g_{2})^{-\half}}_{F}\\
 & \leq\norm{(g_{1}+g_{2})^{-\half}Dg_{1}(g_{1}+g_{2})^{-\half}}_{F}+\norm{(g_{1}+g_{2})^{-\half}Dg_{2}(g_{1}+g_{2})^{-\half}}_{F}\\
 & =\sqrt{\tr\Par{(g_{1}+g_{2})^{-1}Dg_{1}(g_{1}+g_{2})^{-1}Dg_{1}}}+\sqrt{\tr\Par{(g_{1}+g_{2})^{-1}Dg_{2}(g_{1}+g_{2})^{-1}Dg_{2}}}\\
 & =\sqrt{\tr\bigg(\bigg(\underbrace{I+g_{1}^{-\half}g_{2}g_{1}^{-\half}}_{=:E_{1}}\bigg)^{-1}\underbrace{g_{1}^{-\half}Dg_{1}g_{1}^{-\half}}_{=:T_{1}}\Par{I+g_{1}^{-\half}g_{2}g_{1}^{-\half}}^{-1}g_{1}^{-\half}Dg_{1}g_{1}^{-\half}\bigg)}\\
 & \qquad+\sqrt{\tr\bigg(\bigg(\underbrace{I+g_{2}^{-\half}g_{1}g_{2}^{-\half}}_{=:E_{2}}\bigg)^{-1}\underbrace{g_{2}^{-\half}Dg_{2}g_{2}^{-\half}}_{=:T_{2}}\Par{I+g_{2}^{-\half}g_{1}g_{2}^{-\half}}^{-1}g_{2}^{-\half}Dg_{2}g_{2}^{-\half}\bigg)}\\
 & =\sqrt{\tr\Par{E_{1}^{-1}T_{1}E_{1}^{-1}T_{1}}}+\sqrt{\tr\Par{E_{2}^{-1}T_{2}E_{2}^{-1}T_{2}}}\\
 & \leq\sqrt{\tr\Par{T_{1}E_{1}^{-2}T_{1}}}+\sqrt{\tr\Par{T_{2}E_{2}^{-2}T_{2}}},
\end{align*}
where we used the Cauchy-Schwartz inequality $\tr(A^{2})\leq\tr(A^{\top}A)$
in the last line. Since $I\preceq E_{i}$, it follows that $I\preceq E_{i}^{2}$
and $I\succeq E_{i}^{-2}\succ0$. Therefore,
\begin{align*}
\sqrt{\tr\Par{T_{1}E_{1}^{-2}T_{1}}}+\sqrt{\tr\Par{T_{2}E_{2}^{-2}T_{2}}} & \leq\sqrt{\tr(T_{1}^{2})}+\sqrt{\tr(T_{2}^{2})}\\
 & =\norm{T_{1}}_{F}+\norm{T_{2}}_{F}\\
 & \leq2\norm h_{g_{1}(x)}+2\norm h_{g_{2}(x)}\\
 & \leq2\sqrt{2}\norm h_{(g_{1}+g_{2})(x)}.
\end{align*}
Putting these together, we conclude that
\[
\norm{(g_{1}+g_{2})^{-\half}D(g_{1}+g_{2})(g_{1}+g_{2})^{-\half}}_{F}\leq2\sqrt{2}\norm h_{(g_{1}+g_{2})(x)},
\]
and thus $2(g_{1}+g_{2})$ is strongly self-concordant on $K_{1}\cap K_{2}$.
\end{proof}
The same scalar-scaling law holds for strong self-concordance since
the operator norm and Frobenius norm behave in the same way for scalar
scaling.

Next, we recall the analogue of Lemma~\ref{lem:scCloseness} for
strong self-concordance.
\begin{lem}
[\cite{laddha2020strong}, Lemma 1.2] \label{lem:strongSC-closeness}Given
a strongly self-concordant matrix function $g$ on $K\subset\Rn$,
and any $x,y\in K$ with $\norm{x-y}_{g(x)}<1$, 
\[
\norm{g(x)^{-\half}\Par{g(y)-g(x)}g(x)^{-\half}}_{F}\leq\frac{\norm{x-y}_{g(x)}}{\Par{1-\norm{x-y}_{g(x)}}^{2}}.
\]
\end{lem}


\paragraph{Symmetry.}

Recall that $\onu$-symmetry requires two-sided inclusion: the first
part is $\dcal_{g}^{1}(x)\subset K\cap(2x-K)$, and the second part
is $K\cap(2x-K)\subset\dcal_{g}^{\sqrt{\onu}}(x)$. The first part
immediately follows when a metric is induced by a self-concordant
function.
\begin{lem}
\label{lem:symmetricLeftpart} If $\phi$ is a self-concordant function
on $K$, then $\dcal_{g}^{1}(x)\subset K\cap(2x-K)$ for $g=\hess\phi$
and $x\in K$.
\end{lem}

\begin{proof}
Lemma~\ref{lem:dikin-in-body} ensures that $y\in K$ whenever $y\in\dcal_{g}^{1}(x)$.
Then $2x-y\in\dcal_{g}^{1}(x)$ and thus $2x-y\in K$. It implies
that $y\in2x-K$.
\end{proof}
When a metric is induced by a self-concordant barrier with a barrier
parameter $\nu$, it holds that $\onu=O(\nu^{2})$. This is mentioned
in \cite{laddha2020strong}, we provide the proof for completeness.
\begin{lem}
For a self-concordant barrier $\phi$ with a barrier parameter $\nu$
on $K$ and $g=\hess\phi$, it follows that $\onu=O(\nu^{2})$.
\end{lem}

\begin{proof}
By Theorem 4.2.5 in \cite{nesterov2003introductory}, for any $x,y\in K$
with $\grad\phi(x)\cdot(y-x)\geq0$ it follows that $\norm{y-x}_{g(x)}\leq\nu+2\sqrt{\nu}$.
Now, let $x\in K$ and $y\in K\cap(2x-K)$. The latter implies that
$y-x=x-z$ for some $z\in K$. 

If $\grad\phi(x)\cdot(y-x)\geq0$, then $\norm{y-x}_{g(x)}\leq\nu+2\sqrt{\nu}.$
If $\grad\phi(x)\cdot(y-x)<0$, then $\grad\phi(x)\cdot(z-x)>0$ and
thus $\norm{y-x}_{g(x)}=\norm{z-x}_{g(x)}\leq\nu+2\sqrt{\nu}$. From
these two cases, it holds in general that $\norm{y-x}_{g(x)}\leq\nu+2\sqrt{\nu}$
and thus $K\cap(2x-K)\subset\dcal_{g}^{\nu+2\sqrt{\nu}}(x)$. By Lemma
\ref{lem:symmetricLeftpart}, $\dcal_{g}^{1}(x)\subset K\cap(2x-K)$
and thus $\onu=O(\nu^{2})$.
\end{proof}
For affine constraints $Ax\geq b$, the first inclusion above has
a useful equivalent description as follows:
\begin{lem}
\label{lem:symmforPolytope} Let $x\in K=\{Ax>b\}$. It holds that
$y\in K\cap(2x-K)$ if and only if $\norm{A_{x}(y-x)}_{\infty}\leq1$.
\end{lem}

\begin{proof}
For $y\in K$, we have $Ay>b$ and thus $s_{x}=Ax-b>A(x-y)$ (elementwise
inequality). As $s_{x}>0$, we have $A_{x}(x-y)\leq1$. When $y\in(2x-K)$,
we can write $y=2x-z$ for some $z\in K$. Note that
\[
A(x-y)=A(z-x)>b-Ax=-s_{x},
\]
and thus $A_{x}(x-y)\geq-1$. Therefore, $\norm{A_{x}(y-x)}_{\infty}\leq1$.
\end{proof}
Lastly, we can check that if $g$ is $\onu$-symmetric, then $\alpha g$
is $\alpha\onu$-symmetric for $\alpha\ge1$, which follows from observation
of $\dcal_{\alpha g}^{r}(x)=\dcal_{g}^{r/\sqrt{\alpha}}(x)$.
\begin{lem}
\label{lem:symmScaling} For $\alpha\geq1$, if $g$ is $\onu$-symmetric,
then $\alpha g$ is $\alpha\onu$-symmetric.
\end{lem}

\begin{lem}
\label{lem:sumSymmetricSC} If $g_{1}$ and $g_{2}$ are symmetric
self-concordant matrices with parameters $\nu_{1}$ and $\nu_{2}$
on $K_{1}$ and $K_{2}$, then $g_{1}+g_{2}$ is a symmetric self-concordant
matrix with parameter $\nu_{1}+\nu_{2}$ on $K_{1}\cap K_{2}$.
\end{lem}

\begin{proof}
For $g:=g_{1}+g_{2}$, let $y\in\dcal_{g}^{1}(x)$. It implies $y\in\dcal_{g_{1}}^{1}(x)\cap\dcal_{g_{2}}^{1}(x)$
and so $y\in K_{i}\cap(2x-K_{i})$. Due to $\cap_{i}\Par{K_{i}\cap(2x-K_{i})}=K\cap(2x-K)$,
we have $y\in K\cap(2x-K)$ and so $\dcal_{g}^{1}(x)\subset K\cap(2x-K)$.

Now let $y\in K\cap(2x-K)$. It is obvious that $y\in K_{i}\cap(2x-K_{i})$
for $i=1,2$, and thus
\begin{align*}
(y-x)^{\top}g_{1}(x)(y-x) & \leq\nu_{1},\\
(y-x)^{\top}g_{2}(x)(y-x) & \leq\nu_{2}.
\end{align*}
By adding up these two, it follows that $\norm{y-x}_{g(x)}^{2}\leq\nu_{1}+\nu_{2}$.
\end{proof}

\paragraph{Lower trace self-concordance.}

We finish this section by providing a sufficient condition under which
the sum of metrics satisfies lower trace self-concordance.
\begin{lem}
\label{lem:additiveCondition}Let $g_{1}$ and $g_{2}$ be matrix
functions from $\R^{n}$ to $\R^{n\times n}$. If
\[
D^{2}g_{1}[h,h]\succeq0\qquad\mbox{ and }\qquad\tr\Par{(g_{1}+g_{2})^{-1}D^{2}g_{2}[h,h]}\geq-\norm h_{g_{1}+g_{2}}^{2},
\]
then $g_{1}+g_{2}$ is lower trace self-concordant.
\end{lem}

Also, a special case of the lemma is that if $D^{2}g_{1}[h,h]\succeq0$
and $D^{2}g_{2}[h,h]\succeq0$, then $g_{1}+g_{2}$ is lower trace
self-concordant. Note that this condition is \emph{additive. }
