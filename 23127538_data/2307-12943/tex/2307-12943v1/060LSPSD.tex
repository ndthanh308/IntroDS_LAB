
\section{Poly-logarithmic dependence on $m$ via the Lewis-weight metric \label{sec:LS-psdSampling}}

The $\dw$ with the Vaidya metric has mixing rate with $\sqrt{m}$-dependence.
A natural question is whether this can be made smaller, potentially
just polylogarithmic in $m$. The $\dw$ on a polytope in $\Rn$ mixes
in $\otilde{n^{2}}$ steps by working with the Lee-Sidford (LS) metric,
$G(x)=(1+p)(1+p^{2})A_{x}^{\top}W_{x}^{1-\frac{2}{p}}A_{x}$ in \cite{laddha2020strong}.
Can we develop an analogous result in the PSD setting (i.e., $\otilde{n^{4}}$
mixing rate) via the LS metric? Unfortunately, it is not straightforward
that the LS metric $G$ is suitable to address the constraints of
$\inner{A_{i},X}\leq b_{i}$ in (\ref{eq:PSDcone}). However, it is
not clear that the LS metric \emph{together with the log-determinant
barrier} has the property that log-determinant of its Hessian is convex.

We find that the $d$-dimensional version of the metric $O^{*}(\sqrt{n})A_{x}^{\top}W_{x}A_{x}$
with $p=O(\log m)$ has a good fit with our analysis framework. To
be precise, we prove Theorem~\ref{thm:LSPSD} that the $\dw$ mixes
in $\otilde{n^{5}}$ steps with the Lewis weights metric defined as
\begin{align}
g(X) & =2\Par{ng_{1}(X)+g_{2}(X)}\label{eq:metricLS}\\
g_{1}(X) & =-\hess_{X}\log\det X=M^{\top}(X\kro X)^{-1}M,\nonumber \\
g_{2}(X) & =c_{1}\Par{\log m}^{c_{2}}\sqrt{n}M^{\top}A_{X}^{\top}W_{X}A_{X}M\ \text{for some constants \ensuremath{c_{1},c_{2}>0},}\nonumber 
\end{align}
where $W_{x}$ is the diagonal matrix with the $\ell_{p}$-Lewis weight
on diagonals and $p=O\Par{\log m}$. We first check strong self-concordance
and symmetry of the Lewis-weight metric in Section~\ref{subsec:sc-and-symm-LS}
and then show lower trace self-concordance in Section~\ref{subsec:lowerSC-LS}.

\subsection{Strong self-concordance and symmetry \label{subsec:sc-and-symm-LS}}

As in the previous section, we work with $x\in\R^{n}$ instead of
$\svec(X)\in\R^{d}=\R^{n(n+1)/2}$ and then replace $n$ by $d$ later
when obtaining results for PSD sampling.

We first state the analogue of Lemma~\ref{lem:usefulFactLeverage}
for the $\ell_{p}$-Lewis weight:
\begin{lem}
[\cite{lee2019solving}] \label{lem:usefulFactLewis} Let $W_{x}=\Diag(w_{x})\in\R^{m\times m}$
be the $\ell_{p}$-Lewis weights of $A_{x}$, $g(x)=A_{x}^{\top}W_{x}A_{x}$
be the Lewis-weight metric, and $h\in\Rn$.
\begin{itemize}
\item \textup{(Lemma 26)} $\max_{i\in[m]}\frac{\sigma\Par{W_{x}^{1/2}A_{x}}_{i}}{\Par{W_{x}}_{ii}}\leq2m^{\frac{2}{p+2}}$.
\item \textup{(Lemma 33)} $\norm{A_{x}h}_{W_{x}}=\norm h_{g(x)}$ and $\norm{A_{x}h}_{\infty}\leq\sqrt{2}m^{\frac{1}{p+2}}\norm h_{g(x)}$.
\item \textup{(Lemma 34)} $\norm{W_{x}^{-1}\diag\Par{W_{x,h}'}}_{W_{x}}\leq p\norm h_{g(x)}$.
\end{itemize}
\end{lem}

We can now show that the Lewis weights metric is strongly self-concordant
and $\onu$-symmetric under appropriate scaling.
\begin{lem}
[Strong self-concordance and symmetry] \label{lem:LSmetricStrongandSymmetry}
Let $g(x)=c_{1}\Par{\log m}^{c_{2}}A_{x}^{\top}W_{x}A_{x}$ be the
Lewis-weight metric with the $\ell_{p}$-Lewis weight $w_{x}$ with
$p=O(\log m)$. There exist constants $c_{1}$ and $c_{2}$ such that
$g$ is strongly self-concordant and $\onu$-symmetric with $\onu=O^{*}\Par n$.
\end{lem}

\begin{proof}
Consider the unscaled version first: $g(x)=A_{x}^{\top}W_{x}A_{x}$.
By Lemma~\ref{lem:helper4Diagonal}-1
\begin{align*}
\norm{g(x)^{-\half}Dg(x)[h]g(x)^{-\half}}_{F} & \leq2\sqrt{\max_{i}\Par{\frac{\sigma\Par{\sqrt{W_{x}}A_{x}}_{i}}{(W_{x})_{ii}}}\cdot\bigg(\norm h_{g(x)}^{2}+\sum_{i=1}^{m}\Par{W_{x}^{-1}}_{ii}\diag\Par{W_{x,h}'}^{2}\bigg)}\\
 & \underset{\text{(i)}}{\leq}2\sqrt{2m^{\frac{2}{p+2}}}\sqrt{\norm h_{g(x)}^{2}+p^{2}\norm h_{g(x)}^{2}}\\
 & \leq\Par{8m^{\frac{2}{p+2}}(1+p^{2})}^{1/2}\norm h_{g(x)},
\end{align*}
where in (i) we used Lemma~\ref{lem:usefulFactLewis}-1 and 3.

For the first part of $\onu$-symmetry, Lemma~\ref{lem:helper4Diagonal}-2
implies that
\[
\max_{h:\norm h_{g(x)}=1}\norm{A_{x}h}_{\infty}=\sqrt{\max_{i\in[m]}\frac{\sigma\Par{\sqrt{W_{x}}A_{x}}_{i}}{W_{x,i}}}\leq\sqrt{2m^{\frac{2}{p+2}}},
\]
 and Lemma~\ref{lem:helper4Diagonal}-3 leads to $K\cap(2x-K)\subset\dcal_{g}^{\sqrt{n}}(x)$
due to
\[
\tr\Par{W_{x}}=\tr\Par{W_{x}^{\half-\frac{1}{p}}A_{x}\Par{A_{x}^{\top}W_{x}^{1-\frac{2}{p}}A_{x}}^{-1}A_{x}^{\top}W_{x}^{\half-\frac{1}{p}}}=\tr\Par{A_{x}^{\top}W_{x}^{1-\frac{2}{p}}A_{x}\Par{A_{x}^{\top}W_{x}^{1-\frac{2}{p}}A_{x}}^{-1}}=n.
\]
Therefore, $4pm^{\frac{1}{p+2}}A_{x}^{\top}W_{x}A_{x}$ is strongly
self-concordant with $O\Par{nm^{\frac{1}{p+2}}}$-symmetric by Lemma~\ref{lem:symmforPolytope}.
By setting $p=O(\log m)$, the claim follows.
\end{proof}

\subsection{Lower trace self-concordance \label{subsec:lowerSC-LS}}

As we did in the previous section, we work with a polytope in $\Rn$
and then adapt our results to the PSD setting to make arguments concise.
For $\theta(x):=A_{x}^{\top}W_{x}A_{x}$ (i.e., the unscaled version
of $g_{2}$), we write $g_{2}=c\cdot\theta$ for a constant $c$,
which will be set to $c_{1}(\log m)^{c_{2}}\sqrt{n}$ for some constants
$c_{1},c_{2}>0$ later. For a given metric $g_{1}$, let $g(x):=g_{1}(x)+c\cdot\theta(x)$.
We begin with a directional derivative of the $\ell_{p}$-Lewis weight
of $A_{x}$.
\begin{lem}
[\cite{gatmiry2023sampling}, Lemma 2.2] The directional derivative
of the $\ell_{p}$-Lewis weight $W_{x}$ in direction $h\in\Rn$ is
\[
W_{x,h}':=DW_{x}[h]=-2\Diag\Par{\Lambda_{x}G_{x}^{-1}W_{x}s_{x,h}},
\]
where $\Lambda_{x}=W_{x}-P_{x}^{(2)}$ and $G_{x}=W_{x}-\Par{1-\frac{2}{p}}\Lambda_{x}$.
\end{lem}

We recall that these matrices satisfy
\begin{align}
P_{x}^{(2)}\preceq W_{x}\preceq I,\label{eq:lewisBasic-PWI}\\
\Lambda_{x}\preceq W_{x},\label{eq:lewisBasic-LW}\\
\frac{2}{p}W_{x}\preceq G_{x}\preceq W_{x}, & \text{ which implies }W_{x}^{-1}\preceq G_{x}^{-1}\preceq\frac{p}{2}W_{x}^{-1}\text{ and }I\preceq W_{x}^{\half}G_{x}^{-1}W_{x}^{\half}\preceq\frac{p}{2}I.\label{eq:lewisBasic-WGW}
\end{align}
We first bound the largest diagonal entry of $\Gamma_{x}=\Diag(A_{x}g(x)^{-1}A_{x}^{\top})$.
\begin{lem}
\label{lem:GammaNormLSMetric}$\norm{\Gamma_{x}}_{\infty}\leq2c^{-1}m^{\frac{2}{p+2}}$.
\end{lem}

\begin{proof}
Note that $0\preceq\Gamma_{x}=\Diag\Par{A_{x}g^{-1}A_{x}^{\top}}\preceq c^{-1}\Diag\Par{A_{x}\theta^{-1}A_{x}^{\top}}$.
By Lemma~\ref{lem:usefulFactLewis}-1
\[
\norm{\Diag\Par{A_{x}\theta^{-1}A_{x}^{\top}}}_{\infty}=\max_{i\in[m]}\frac{\sigma\Par{W_{x}^{1/2}A_{x}}_{i}}{\Par{W_{x}}_{ii}}\leq2m^{\frac{2}{p+2}}.\qedhere
\]
\end{proof}
\begin{lem}
\label{lem:LSLowerSCTrace} $\tr\Par{g^{-1}D^{2}g_{2}(x)[h,h]}\geq-\norm h_{g_{2}}^{2}$,
where $g_{2}(x)=c\theta(x)=cA_{x}^{\top}W_{x}A_{x}$ with $c=c_{1}(\log m)^{c_{2}}\sqrt{n}$
for some constants $c_{1},c_{2}>0$.
\end{lem}

\begin{proof}
Repeating the same calculus done for $A_{x}^{\top}\Sigma_{x}A_{x}$
in Proposition~\ref{prop:calculusLeverage}, we can obtain
\begin{align*}
D^{2}\theta[h,h] & =6A_{x}^{\top}S_{x,h}W_{x}S_{x,h}A_{x}-4A_{x}^{\top}W_{x,h}'S_{x,h}A_{x}+A_{x}^{\top}W_{x,h}''A_{x}\\
 & \succeq-4A_{x}^{\top}W_{x,h}'S_{x,h}A_{x}+A_{x}^{\top}W_{x,h}''A_{x},
\end{align*}
and thus for $\Gamma_{x}=\Diag\Par{A_{x}g^{-1}A_{x}^{\top}}$
\begin{align*}
\tr\Par{g^{-1}D^{2}\theta[h,h]} & =\tr\Par{g^{-\half}D^{2}\theta[h,h]g^{-\half}}\\
 & \geq\tr\Par{g^{-1}A_{x}^{\top}\Par{W_{x,h}''-4W_{x,h}'S_{x,h}}A_{x}}\\
 & \geq\tr\Par{A_{x}g^{-1}A_{x}^{\top}\Par{W_{x,h}''-4W_{x,h}'S_{x,h}}}\\
 & =\tr\Par{\Gamma_{x}\Par{W_{x,h}''-4W_{x,h}'S_{x,h}}}\\
 & =-4\tr\Par{\Gamma_{x}W_{x,h}'S_{x,h}}+\tr\Par{\Gamma_{x}W_{x,h}''}.
\end{align*}
For the first term, since diagonal matrices commute,
\begin{align}
\Abs{\tr\Par{\Gamma_{x}W_{x,h}'S_{x,h}}} & \leq\norm{\Gamma_{x}}_{\infty}\tr\Par{\sqrt{S_{x,h}W_{x,h}'^{2}S_{x,h}}}\nonumber \\
 & =\norm{\Gamma_{x}}_{\infty}\tr\Par{\sqrt{W_{x,h}'W_{x}^{-1}W_{x,h}'S_{x,h}W_{x}S_{x,h}}}\nonumber \\
 & =\norm{\Gamma_{x}}_{\infty}\tr\Par{\sqrt{W_{x,h}'W_{x}^{-1}W_{x,h}'}\sqrt{S_{x,h}W_{x}S_{x,h}}}\nonumber \\
 & \underset{\text{(i)}}{\leq}\norm{\Gamma_{x}}_{\infty}\sqrt{\tr\Par{W_{x,h}'W_{x}^{-1}W_{x,h}'}}\sqrt{\tr\Par{S_{x,h}W_{x}S_{x,h}}}\nonumber \\
 & =\norm{\Gamma_{x}}_{\infty}\norm{W_{x}^{-1}w_{x,h}'}_{W_{x}}\norm h_{\theta}\nonumber \\
 & \underset{\text{(ii)}}{\leq}p\norm{\Gamma_{x}}_{\infty}\norm h_{\theta}^{2},\label{eq:boundonFirst}
\end{align}
where in (i) we used the Cauchy-Schwartz inequality, and in (ii) $\norm{W_{x}^{-1}w_{x,h}'}_{W_{x}}\leq p\norm h_{\theta}$
(Lemma~\ref{lem:usefulFactLewis}-3).

For the second term $\tr\Par{\Gamma_{x}W_{x,h}''}$, let us compute
the second-order directional derivate of $W_{x}$ in direction $h$:
\begin{align*}
W_{x,h}' & =-2\Diag(\Lambda_{x}G_{x}^{-1}W_{x}s_{x,h}),\\
W_{x,h}'' & =-2\Diag\Par{\Lambda_{x,h}'G_{x}^{-1}W_{x}s_{x,h}-\Lambda_{x}G_{x}^{-1}G_{x,h}'G_{x}^{-1}W_{x}s_{x,h}+\Lambda_{x}G_{x}^{-1}W_{x,h}'s_{x,h}-\Lambda_{x}G_{x}^{-1}W_{x}S_{x,h}s_{x,h}},
\end{align*}
where $\Lambda_{x,h}':=D\Lambda_{x}[h]$ and $G_{x,h}':=DG_{x}[h]$.
 Thus,
\begin{align}
 & \tr\Par{\Gamma_{x}W_{x,h}''}\nonumber \\
 & =-2\tr\Par{\Gamma_{x}\Diag\Par{\Lambda_{x,h}'G_{x}^{-1}W_{x}s_{x,h}-\Lambda_{x}G_{x}^{-1}G_{x,h}'G_{x}^{-1}W_{x}s_{x,h}+\Lambda_{x}G_{x}^{-1}W_{x,h}'s_{x,h}-\Lambda_{x}G_{x}^{-1}W_{x}S_{x,h}s_{x,h}}}\nonumber \\
 & =-2\tr\bigg(\Gamma_{x}\Diag\bigg(\underbrace{\Lambda_{x,h}'G_{x}^{-1}W_{x}s_{x,h}}_{\text{I}}\bigg)\bigg)+2\tr\bigg(\Gamma_{x}\Diag\bigg(\underbrace{\Lambda_{x}G_{x}^{-1}G_{x,h}'G_{x}^{-1}W_{x}s_{x,h}}_{\text{II}}\bigg)\bigg)\nonumber \\
 & \qquad-2\tr\bigg(\Gamma_{x}\Diag\bigg(\underbrace{\Lambda_{x}G_{x}^{-1}W_{x,h}'s_{x,h}}_{\text{III}}\bigg)\bigg)+2\tr\bigg(\Gamma_{x}\Diag\bigg(\underbrace{\Lambda_{x}G_{x}^{-1}W_{x}S_{x,h}s_{x,h}}_{\text{IV}}\bigg)\bigg).\label{eq:trGamma}
\end{align}
Each term is of the form $\tr\Par{\Gamma_{x}\Diag(v)}$ for a vector
$v\in\R^{m}$, and thus
\begin{align}
\Abs{\tr\Par{\Gamma_{x}\Diag(v)}} & =\Abs{\tr\Par{\Gamma_{x}W_{x}^{\half}W_{x}^{-\half}\Diag(v)}}\nonumber \\
 & \leq\sqrt{\tr\Par{W_{x}^{\half}\Gamma_{x}^{2}W_{x}^{\half}}}\sqrt{\tr\Par{\Diag(v)W_{x}^{-1}\Diag(v)}}\nonumber \\
 & \leq\norm{\Gamma_{x}}_{\infty}\sqrt{\tr\Par{W_{x}}}\norm v_{W_{x}^{-1}}\nonumber \\
 & =\sqrt{n}\norm{\Gamma_{x}}_{\infty}\norm v_{W_{x}^{-1}},\label{eq:trGammaBasic}
\end{align}
where we used $\tr(W_{x})=n$ in the last equality. 

Now let us state Calabi-type estimates on directional derivatives
of $\Lambda_{x},G_{x},W_{x}$, and then bound $\norm v_{W_{x}^{-1}}$
for each term (I$\sim$IV). 
\begin{lem}
[\cite{gatmiry2023sampling}] \label{lem:calabi} Let $Ax\leq b$
for $A\in\R^{m\times n}$ and $b\in\R^{m}$, and $W_{x}$ the $\ell_{p}$-Lewis
weight of $A_{x}$ with $p=O(\log m)$. There exists universal constants
$c_{1},c_{2}>0$ such that 
\begin{itemize}
\item \textup{(Lemma D.4)} $-c_{1}(\log m)^{c_{2}}\norm{s_{x,h}}_{\infty}W_{x}\preceq G_{x,h}'\preceq c_{1}(\log m)^{c_{2}}\norm{s_{x,h}}_{\infty}W_{x}$.
\item \textup{(Lemma D.4)} $-c_{1}(\log m)^{c_{2}}\norm{s_{x,h}}_{\infty}W_{x}\preceq\Lambda_{x,h}'\preceq c_{1}(\log m)^{c_{2}}\norm{s_{x,h}}_{\infty}W_{x}$.
\item \textup{(Lemma D.13)} $-c_{1}(\log m)^{c_{2}}\norm{s_{x,h}}_{\infty}W_{x}\preceq W_{x,h}'\preceq c_{1}(\log m)^{c_{2}}\norm{s_{x,h}}_{\infty}W_{x}$. 
\end{itemize}
\end{lem}

Using these estimates, we can bound the local norm of the term I.
In our calculation, $\lesssim$ hides universal constants and poly-logarithmic
factors on $m$:
\begin{align*}
\norm{\text{I}}_{W_{x}^{-1}}^{2} & =s_{x,h}^{\top}W_{x}G_{x}^{-1}\Lambda_{x,h}'W_{x}^{-1}\Lambda_{x,h}'G_{x}^{-1}W_{x}s_{x,h}\\
 & =s_{x,h}^{\top}W_{x}G_{x}^{-1}W_{x}^{\half}\bigg(\underbrace{W_{x}^{-\half}\Lambda_{x,h}'W_{x}^{-\half}}_{\precsim\norm{s_{x,h}}_{\infty}I\ \text{(Lemma \ref{lem:calabi}-2)}}\bigg)^{2}W_{x}^{\half}G_{x}^{-1}W_{x}s_{x,h}\\
 & \lesssim\norm{s_{x,h}}_{\infty}^{2}s_{x,h}^{\top}W_{x}G_{x}^{-1}W_{x}G_{x}^{-1}W_{x}s_{x,h}\\
 & =\norm{s_{x,h}}_{\infty}^{2}s_{x,h}^{\top}W_{x}^{\half}\bigg(\underbrace{W_{x}^{\half}G_{x}^{-1}W_{x}^{\half}}_{\preceq\frac{p}{2}I\ \text{\eqref{eq:lewisBasic-WGW}}}\bigg)^{2}W_{x}^{\half}s_{x,h}\\
 & \leq p^{2}\norm{s_{x,h}}_{\infty}^{2}s_{x,h}^{\top}W_{x}s_{x,h}=p^{2}\norm{s_{x,h}}_{\infty}^{2}\norm h_{\theta}^{2}\\
 & \leq2p^{2}m^{\frac{2}{p+2}}\norm h_{\theta}^{4},
\end{align*}
where the last line follows from Lemma~\ref{lem:usefulFactLewis}-2.

For the second term,
\begin{align*}
\norm{\text{II}}_{W_{x}^{-1}}^{2} & =s_{x,h}^{\top}W_{x}G_{x}^{-1}G_{x,h}'G_{x}^{-1}\Lambda_{x}W_{x}^{-1}\Lambda_{x}G_{x}^{-1}G_{x,h}'G_{x}^{-1}W_{x}s_{x,h}\\
 & =s_{x,h}^{\top}W_{x}G_{x}^{-1}G_{x,h}'G_{x}^{-1}W_{x}^{\half}\bigg(\underbrace{W_{x}^{-\half}\Lambda_{x}W_{x}^{-\half}}_{\preceq I\ \text{\eqref{eq:lewisBasic-LW}}}\bigg)^{2}W_{x}^{\half}G_{x}^{-1}G_{x,h}'G_{x}^{-1}W_{x}s_{x,h}\\
 & \leq s_{x,h}^{\top}W_{x}G_{x}^{-1}G_{x,h}'\underbrace{G_{x}^{-1}W_{x}G_{x}^{-1}}_{\preceq\frac{p^{2}}{4}W_{x}^{-1}\ \text{\eqref{eq:lewisBasic-WGW}}}G_{x,h}'G_{x}^{-1}W_{x}s_{x,h}\\
 & \leq p^{2}s_{x,h}^{\top}W_{x}G_{x}^{-1}G_{x,h}'W_{x}^{-1}G_{x,h}'G_{x}^{-1}W_{x}s_{x,h}\\
 & =p^{2}s_{x,h}^{\top}W_{x}G_{x}^{-1}W_{x}^{\half}\bigg(\underbrace{W_{x}^{-\half}G_{x,h}'W_{x}^{-\half}}_{\precsim\norm{s_{x,h}}_{\infty}I\ \text{(Lemma \ref{lem:calabi}-1)}}\bigg)^{2}W_{x}^{\half}G_{x}^{-1}W_{x}s_{x,h}\\
 & \lesssim p^{2}\norm{s_{x,h}}_{\infty}^{2}s_{x,h}^{\top}W_{x}G_{x}^{-1}W_{x}G_{x}^{-1}W_{x}s_{x,h}\\
 & =p^{2}\norm{s_{x,h}}_{\infty}^{2}s_{x,h}^{\top}W_{x}^{\half}\bigg(\underbrace{W_{x}^{\half}G_{x}^{-1}W_{x}^{\half}}_{\preceq\frac{p}{2}I\ \text{\eqref{eq:lewisBasic-WGW}}}\bigg)^{2}W_{x}^{\half}s_{x,h}\\
 & \leq p^{4}\norm{s_{x,h}}_{\infty}^{2}\norm h_{\theta}^{2}\\
 & \leq2p^{4}m^{\frac{2}{p+2}}\norm h_{\theta}^{4},
\end{align*}
where we used Lemma~\ref{lem:usefulFactLewis}-2 in the last line.

For the third term,
\begin{align*}
\norm{\text{III}}_{W_{x}^{-1}}^{2} & =s_{x,h}^{\top}W_{x,h}'G_{x}^{-1}\Lambda_{x}W_{x}^{-1}\Lambda_{x}G_{x}^{-1}W_{x,h}'s_{x,h}\\
 & =s_{x,h}^{\top}W_{x}G_{x}^{-1}G_{x,h}'G_{x}^{-1}W_{x}^{\half}\bigg(\underbrace{W_{x}^{-\half}\Lambda_{x}W_{x}^{-\half}}_{\preceq I\ \eqref{eq:lewisBasic-LW}}\bigg)^{2}W_{x}^{\half}G_{x}^{-1}G_{x,h}'G_{x}^{-1}W_{x}s_{x,h}\\
 & \leq s_{x,h}^{\top}W_{x,h}'\underbrace{G_{x}^{-1}W_{x}G_{x}^{-1}}_{\preceq\frac{p^{2}}{4}W_{x}^{-1}\ \text{\eqref{eq:lewisBasic-WGW}}}W_{x,h}'s_{x,h}\\
 & \leq p^{2}s_{x,h}^{\top}W_{x,h}'W_{x}^{-1}W_{x,h}'s_{x,h}\\
 & =p^{2}s_{x,h}^{\top}W_{x}^{\half}\bigg(\underbrace{W_{x}^{-\half}W_{x,h}'W_{x}^{-\half}}_{\precsim\norm{s_{x,h}}_{\infty}I\ \text{(Lemma \ref{lem:calabi}-3)}}\bigg)^{2}W_{x}^{\half}s_{x,h}\\
 & \lesssim p^{2}\norm{s_{x,h}}_{\infty}^{2}s_{x,h}^{\top}W_{x}s_{x,h}\\
 & \leq p^{2}m^{\frac{2}{p+2}}\norm h_{\theta}^{4},
\end{align*}
where we used Lemma~\ref{lem:usefulFactLewis}-2 in the last line. 

Let us bound the last term:
\begin{align*}
\norm{\text{IV}}_{W_{x}^{-1}}^{2} & =s_{x,h}^{\top}S_{x,h}W_{x}G_{x}^{-1}\underbrace{\Lambda_{x}W_{x}^{-1}\Lambda_{x}}_{\preceq W_{x}\ \text{\eqref{eq:lewisBasic-LW}}}G_{x}^{-1}W_{x}S_{x,h}s_{x,h}\\
 & \leq s_{x,h}^{\top}S_{x,h}W_{x}\underbrace{G_{x}^{-1}W_{x}G_{x}^{-1}}_{\preceq\frac{p^{2}}{4}W_{x}^{-1}\ \text{\eqref{eq:lewisBasic-WGW}}}W_{x}S_{x,h}s_{x,h}\\
 & \leq p^{2}s_{x,h}^{\top}S_{x,h}W_{x}S_{x,h}s_{x,h}\\
 & =p^{2}s_{x,h}^{\top}W_{x}^{\half}S_{x,h}^{2}W_{x}^{\half}s_{x,h}\\
 & \leq p^{2}\norm{s_{x,h}}_{\infty}^{2}\norm h_{\theta}^{2}\\
 & \leq p^{2}m^{\frac{2}{p+2}}\norm h_{\theta}^{4},
\end{align*}
where we used Lemma~\ref{lem:usefulFactLewis}-2 in the last line. 

Combining these bounds, (\ref{eq:trGamma}), and (\ref{eq:trGammaBasic})
for $p=O(\log m)$, we obtain
\begin{align*}
\Abs{\tr\Par{\Gamma_{x}W_{x,h}''}} & \lesssim\sqrt{n}\norm{\Gamma_{x}}_{\infty}\norm h_{\theta}^{2}
\end{align*}
Along with the bound in (\ref{eq:boundonFirst}), we conclude that
\begin{align*}
\tr\Par{g^{-1}D^{2}\theta[h,h]} & \gtrsim-p\norm{\Gamma_{x}}_{\infty}\norm h_{\theta}^{2}-\sqrt{n}\norm{\Gamma_{x}}_{\infty}\norm h_{\theta}^{2}\\
 & \gtrsim-\sqrt{n}\norm{\Gamma_{x}}_{\infty}\norm h_{\theta}^{2}\\
 & \gtrsim-c^{-1}\sqrt{n}\norm h_{\theta}^{2},
\end{align*}
where the last line follows from Lemma~\ref{lem:GammaNormLSMetric}.
This implies that there exists some positive constants $d_{1}$ and
$d_{2}$ such that $\tr\Par{g^{-1}D^{2}\theta[h,h]}\geq-c^{-1}d_{1}\Par{\log m}^{d_{2}}\sqrt{n}\norm h_{\theta}^{2}$,
which implies
\[
\tr\Par{g^{-1}D^{2}g_{2}[h,h]}\geq-c^{-1}d_{1}\Par{\log m}^{d_{2}}\sqrt{n}\norm h_{g_{2}}^{2}.
\]
By taking $c=d_{1}(\log m)^{d_{2}}\sqrt{n}$, the metric $g_{2}=c\theta=d_{1}(\log m)^{d_{2}}\sqrt{n}A_{x}^{\top}W_{x}A_{x}$
is lower trace self-concordant.
\end{proof}
We put together the lemmas above to prove Theorem~\ref{thm:LSPSD}.

\thmLSPSD*
\begin{proof}
As in (\ref{eq:metricLS}), we set
\begin{align*}
g(X) & =2\Par{ng_{1}(X)+g_{2}(X)},\ \text{where}\\
g_{1}(X) & =-\hess_{X}\log\det X=M^{\top}(X\kro X)^{-1}M,\\
g_{2}(X) & =c_{1}\Par{\log m}^{c_{2}}\sqrt{n}M^{\top}A_{X}^{\top}W_{X}A_{X}M\ \text{for some constants \ensuremath{c_{1},c_{2}>0}.}
\end{align*}
Since $ng_{1}$ and $g_{2}$ are strongly self-concordant (see Corollary~\ref{cor:strongSCofLOGDET}
and Lemma~\ref{lem:LSmetricStrongandSymmetry}), $g$ is also strongly
self-concordant due to Lemma~\ref{lem:sumStrongSC} and $O^{*}\Par{n^{3}}$-symmetric\footnote{Since the dimension is $d$ in the setting of (\ref{eq:PSDcone}),
we should replace $n$ by $d=O(n^{2})$ when applying Lemma \ref{lem:paramsBarrier}.} due to Lemma~\ref{lem:symmScaling} and Lemma~\ref{lem:sumSymmetricSC}.

In Lemma~\ref{lem:LSLowerSCTrace}, we set $g_{1}=-2n\hess\log\det$
and note that the proof of Lemma~\ref{lem:LSLowerSCTrace} with $g_{2}=2c_{1}\Par{\log m}^{c_{2}}\sqrt{n}M^{\top}A_{X}^{\top}W_{X}A_{X}M$
leads to $\tr\Par{g^{-1}D^{2}(2g_{2})[h,h]}\geq-\half\norm h_{2g_{2}}^{2}$.
Together with (\ref{eq:D4ph1}) ($D^{2}g_{1}[H,H]\succeq0$ for $H\in\S^{n}$),
Lemma~\ref{lem:additiveCondition} implies that $g$ is lower trace
self-concordant. Therefore, Theorem~\ref{thm:generalMixing} ensures
that the $\dw$ with $g$ mixes in $\otilde{n^{5}}$ steps. Since
the initialization and update of the Lewis weight takes $\otilde{mn^{2\omega}}$
and $\otilde{mn^{2(\omega-1)}}$ time (Theorem~46 in \cite{lee2019solving}),
the same implementation with Theorem~\ref{thm:hybridPSD} also has
the time complexity of $\otilde{mn^{2(\omega-1)}}$.
\end{proof}

\paragraph{Handling approximate Lewis weights.}

In the implementation of the $\dw$ with the Lewis weights metric,
we use an approximation algorithm presented in \cite{lee2019solving}
for computing and updating the Lewis weight, which ensures 
\[
(1-\delta)\wtilde_{X}\preceq W_{X}\preceq(1+\delta)\wt W_{X}
\]
for the approximate Lewis weights $\wtilde_{X}$ and a target accuracy
parameter $\delta$ (note that the initialization and update times
of the Lewis weight above hide poly-logarithmic dependence on $\log(1/\delta)$).
Strictly speaking, we should check that these approximate Lewis weights
do not affect the theoretical guarantees above.

To see this, let us define
\begin{align*}
\widetilde{g}(X) & =2\Par{ng_{1}(X)+\widetilde{g_{2}}(X)},\ \text{where}\\
g_{1}(X) & =-\hess_{X}\log\det X=M^{\top}(X\kro X)^{-1}M,\\
\wt g_{2}(X) & =c_{1}\Par{\log m}^{c_{2}}\sqrt{n}M^{\top}A_{X}^{\top}\widetilde{W}_{X}A_{X}M\ \text{for some constants \ensuremath{c_{1},c_{2}>0}.}
\end{align*}
First of all, the $\dw$ with $\widetilde{g}$ still converges to
the uniform distribution over $K$, since the approximation algorithm
in \cite{lee2019solving} is deterministic and thus the condition
of detailed balance still holds under the acceptance probability of
$\min\Par{1,\sqrt{\frac{\det\tilde{g}(Y)}{\det\tilde{g}(X)}}}$. For
$\widetilde{P}_{X}$ the one-step distribution of the $\dw$ started
at $X$ with $\widetilde{g}$, the triangle inequality leads to 
\[
\dtv(\widetilde{P}_{X},\widetilde{P}_{Y})\leq\dtv(\widetilde{P}_{X},P_{X})+\dtv(P_{X},P_{Y})+\dtv(\widetilde{P}_{Y},P_{Y}).
\]
Since the second term is bounded by a small constant due to Lemma~\ref{lem:one-step},
it suffices to bound $\dtv(\widetilde{P}_{X},P_{X})$ (and $\dtv(\widetilde{P}_{Y},P_{Y})$
similarly) for $\delta=1/\text{poly}(n)$.
\begin{lem}
[One-step coupling] For a convex body $K\subset\Rn$, let $g\in\R^{n\times n}$
be strongly and lower trace self-concordant on $K$. For step size
$r=\frac{1}{2^{12}}$ and $\delta=\frac{10^{-10}}{n^{2}}$, we have
$\dtv(P_{X},\wt P_{X})\leq1/10$ for the transition kernels $P_{X}$
and $\wt P_{X}$ of the $\dw$ with the metrics $g$ and $\wt g$,
respectively.
\end{lem}

\begin{proof}
Let $r_{X}$ and $\wt r_{X}$ be the rejection probabilities of the
one-step of the $\dw$ with $g$ and $\wt g$ started at $X$. Then
the triangle inequality results in
\begin{align*}
\dtv(P_{X},\wt P_{X}) & \leq\dtv\Par{P_{X},U_{g}(X)}+\dtv\Par{U_{g}(X),U_{\tilde{g}}(X)}+\dtv\Par{U_{\tilde{g}}(X),\wt P_{X}}\\
 & =\underbrace{\dtv\Par{U_{g}(X),U_{\tilde{g}}(X)}}_{\text{I}}+\underbrace{r_{X}+\wt r_{X}}_{\text{II}},
\end{align*}
where $U_{g}(X)$ and $U_{\tilde{g}}(X)$ are the uniform distributions
over $\dcal_{g}^{r}(X)$ and $\dcal_{\tilde{g}}^{r}(X)$.

For the overlap (I), we first note that $(1-\delta)\wt g_{2}\preceq g_{2}\preceq(1+\delta)\wt g_{2}$
and thus 
\[
(1-\delta)\wt g\preceq g\preceq(1+\delta)\wt g.
\]
Let us denote the Dikin ellipsoids $\dcal_{g}(X)$ and $\dcal_{\tilde{g}}(X)$
by $D(g)$ and $D(\wt g)$ for simplicity. WLOG, assume $\vol(D(\wt g))\geq\vol(D(g))$.
Repeating the argument in (\ref{eq:eq3}), we have 
\begin{align*}
\text{I} & =\half\int_{\Rn}\Abs{\frac{\bm{1}_{D(g)}(z)}{\vol\Par{D(g)}}-\frac{\bm{1}_{D(\tilde{g})}(z)}{\vol\Par{D(\wt g)}}}dz=1-\frac{\vol\Par{D(g)\cap D(\wt g)}}{\vol\Par{D(\wt g)}}.
\end{align*}
Here, we can assume that $\wt g(X)=I$ due to affine invariance of
the ratio of volumes. Let $g(X)^{-1}=U^{\top}\Diag(\lda)U$ be a spectral
decomposition of $g(X)^{-1}$, where $\{\lda_{i}\}_{i=1}^{d}$ is
the set of eigenvalues of $g(X)^{-1}$ and $\lda:=(\lda_{i})\in\R^{d}$.
Consider a matrix $C\in\R^{d\times d}$ such that $C^{-1}=U^{\top}\Diag(\min(1,\lda))U$,
and by construction $D(C)\subset D(g)\cap D(I)$. Thus,
\begin{align*}
\frac{\vol\Par{D(g)\cap D(\wt g)}}{\vol\Par{D(\wt g)}} & =\frac{\vol\Par{D(g)\cap I}}{\vol\Par I}\\
 & \geq\sqrt{\prod_{i:\lda_{i}\leq1}\lda_{i}}=\sqrt{\prod_{i:\lda_{i}\leq1}(1-(1-\lda_{i}))}\\
 & \geq\sqrt{\exp\Par{-\sum_{i:\lda_{i}\leq1}(1-\lda_{i})}},
\end{align*}
where the last inequality follows from $1-x\geq\exp(-2x)$ for $0\leq x\leq\half$
and the fact that $(1-\delta)I\preceq g(X)^{-1}\preceq(1+\delta)I$
guarantees $\lda_{i}\geq\half$. This fact also implies that $\sum_{\lda_{i}<1}(1-\lda_{i})\leq\delta n^{2}$.
Putting these together, we have that $\text{I}\leq\frac{1}{100}$.

For the rejection probability (II), we have
\begin{align*}
\text{II} & =r_{X}+\wt r_{X}\\
 & =\int\max\Par{0,1-\sqrt{\frac{\det g(Z)}{\det g(X)}}}\frac{1_{D(g)}(Z)}{\vol\Par{D(g)}}dZ+\int\max\Par{0,1-\sqrt{\frac{\det\wt g(Z)}{\det\wt g(X)}}}\frac{1_{D(\tilde{g})}(Z)}{\vol\Par{D(\wt g)}}dZ.
\end{align*}
Since $(1-\delta)\wt g\preceq g\preceq(1+\delta)\wt g$ implies $(1-\delta)I\preceq\wt g^{-\half}g\wt g^{-\half}\preceq(1+\delta)I$,
we have $(1-\delta)^{n^{2}/2}\leq\sqrt{\frac{\det g}{\det\wt g}}\leq(1+\delta)^{n^{2}/2}$
and 
\begin{align*}
(1-\delta)^{n^{2}/2}\vol(D(\wt g)) & \leq\vol\Par{D(g)}\leq(1+\delta)^{n^{2}/2}\vol(D(\wt g)),\\
(1-\delta)^{n^{2}}\sqrt{\frac{\det\wt g(Z)}{\det\wt g(X)}} & \leq\sqrt{\frac{\det g(Z)}{\det g(X)}}\leq(1+\delta)^{n^{2}}\sqrt{\frac{\det\wt g(Z)}{\det\wt g(X)}}.
\end{align*}
Using this, we can further manipulate $\wt r_{X}$ as follows:
\begin{align*}
\wt r_{X} & =\int\max\Par{0,1-\sqrt{\frac{\det\wt g(Z)}{\det\wt g(X)}}}\frac{1_{D(\tilde{g})}(Z)}{\vol\Par{D(\wt g)}}dZ\\
 & \leq\int\frac{1_{D(\tilde{g})\backslash D(g)}(Z)}{\vol\Par{D(\wt g)}}dZ+\int\max\Par{0,1-\sqrt{\frac{\det\wt g(Z)}{\det\wt g(X)}}}\frac{1_{D(g)}(Z)}{\vol\Par{D(\wt g)}}dZ\\
 & \leq\frac{\vol\Par{D(\tilde{g})\backslash D(g)}}{\vol\Par{D(\wt g)}}+(1+\delta)^{n^{2}/2}\int\max\Par{0,1-(1-\delta)^{n^{2}/2}\sqrt{\frac{\det g(Z)}{\det g(X)}}}\frac{1_{D(g)}(Z)}{\vol\Par{D(g)}}dZ.
\end{align*}
Note that the first term is simply the term (I). We can show that
the second term is bounded by $1.1\cdot r_{X}$ for $\delta=10^{-10}/n^{2}$.
As $r_{X}\leq0.014$ due to Lemma~\ref{lem:one-step}, we have $\text{II}\leq0.045$.
\end{proof}


