
\section{$\sqrt{m}$-dependence via the Vaidya metric \label{sec:hybrid-psdSampling}}

The $\dw$ with a hybrid barrier of $-\log\det X$ and the log-barrier
mixes in $\otilde{n^{2}(n^{2}+m)}$ iterations with per-step complexity
$\min\Par{mn^{\omega}+m^{2}n^{2},n^{2\omega}+mn^{2(\omega-1)}}$.
In the regime of $m=O(n^{2})$, the first algorithm for the per-step
implementation is effective, but once $m$ gets larger the second
algorithm should be employed. Thus for large $m$, combined with the
mixing rate of $O(n^{2}m)$ the provable upper bound on the total
time ends up having quadratic dependence on $m$, the number of constraints.

Can we find a better metric that allows the mixing rate of the $\dw$
to have better dependence on $m$ while maintaining the same per-step
complexity of $O\Par{mn^{2(\omega-1)}}$? We show that the $\dw$
can enjoy $\sqrt{m}$-dependence in the mixing rate and the same per-step
complexity by running with the following metric, which is the analog
of the Vaidya metric \cite{vaidya1996new} for the PSD cone: 
\begin{align}
g(X) & =2\Par{ng_{1}(X)+g_{2}(X)},\ \text{where}\label{eq:metricHybrid}\\
g_{1}(X) & =-\hess_{X}\log\det X=M^{\top}(X\kro X)^{-1}M,\nonumber \\
g_{2}(X) & =22\sqrt{\frac{m}{n}}M^{\top}A_{X}^{\top}\Par{\Sigma_{X}+\frac{n}{m}I_{m}}A_{X}M,\nonumber 
\end{align}
where $\Sigma_{X}=\Diag(A_{X}(A_{X}^{\top}A_{X})^{-1}A_{X}^{\top})$
is the diagonal matrix with the leverage scores of $A_{X}$.

\thmHybridPSD*

Notice that the main difference between this metric and the previous
one is that $g_{2}$ used to address constraints of the form $\inner{A_{i},X}\leq b_{i}$.
In the literature of interior-point method, the metric $g_{2}$ was
proposed by \cite{vaidya1996new} and has a better barrier parameter
than the logarithmic barrier; this motivates us to analyze the $\dw$
with this metric. However, we note that upon combining this Vaidya
metric with the barrier for the PSD cone, $-\log\det X$, we should
revisit every step in the previous analysis of the $\dw$ where $g_{2}$
part previously came from the logarithmic barrier. In particular,
$D^{2}g_{2}(X)[H,H]$ is no longer guaranteed to be convex.

We begin by establishing strong self-concordance and then computing
the symmetry parameter of the Vaidya metric $g_{2}$ in Section~\ref{subsec:sc-sym-hybrid}.
 Then in Section~\ref{subsec:lowerSC-hybrid}, we use Lemma~\ref{lem:additiveCondition}
to check lower trace self-concordance of $g=2(ng_{1}+g_{2})$. Given
these, it follows from Theorem~\ref{thm:generalMixing} that the
$\dw$ with this metric mixes in $\otilde{\Par{n+\sqrt{m}}n^{3}}$
iterations with per-step complexity $\otilde{mn^{2(\omega-1)}}$.

\subsection{Strong self-concordance and symmetry \label{subsec:sc-sym-hybrid}}

In the sequel, we work with $x\in\R^{n}$ instead of $\svec(X)\in\R^{d}=\R^{n(n+1)/2}$
and replace $n$ by $d$ later when obtaining results for PSD sampling.

We first attempt to prove helper lemmas for strong self-concordance
and symmetry of the metrics of the form $A_{x}^{\top}D_{x}A_{x}$,
where $D_{x}\in\R^{m\times m}$ is a diagonal matrix used to address
the constraints of the form $Ax\leq b$ for $A\in\R^{m\times n}$
and $b\in\R^{m}$. Then we use these results to look into strong self-concordance
and symmetry of the Vaidya metric.
\begin{lem}
\label{lem:helper4Diagonal} Let $g(x)=A_{x}^{\top}D_{x}A_{x}\in\Rnn$
for a diagonal matrix $0\prec D_{x}\in\R^{m\times m}$.
\begin{itemize}
\item $\norm{g(x)^{-\half}Dg(x)[h]g(x)^{-\half}}_{F}^{2}\leq4\max_{i}\Par{\frac{\sigma\Par{\sqrt{D_{x}}A_{x}}_{i}}{\Par{D_{x}}_{ii}}}\cdot\Par{\norm h_{g(x)}^{2}+\sum_{i=1}^{m}\Par{D_{x}^{-1}}_{ii}(DD_{x}[h])_{i}^{2}}$.
\item $\max_{h:\norm h_{g(x)}=1}\norm{A_{x}h}_{\infty}=\sqrt{\max_{i\in[m]}\frac{\sigma\Par{\sqrt{D_{x}}A_{x}}_{i}}{\Par{D_{x}}_{ii}}}$.
\item $K\cap(2x-K)\subset\dcal_{g}^{\sqrt{\tr\Par{D_{x}}}}(x)$.
\end{itemize}
\end{lem}

\begin{proof}
Let us write $g(x)=A_{x}^{\top}D_{x}A_{x}=A^{\top}V_{x}A$ for $V_{x}:=S_{x}^{-1}D_{x}S_{x}^{-1}$.
By Claim~\ref{claim:1stDiffSlack}
\begin{align*}
Dg(x)[h] & =A^{\top}DV_{x}[h]A\\
 & =A^{\top}\Par{-2S_{x}^{-1}S_{x,h}S_{x}^{-1}D_{x}+S_{x}^{-1}DD_{x}[h]S_{x}^{-1}}A\\
 & =A^{\top}V_{x}^{1/2}\Par{-2S_{x,h}+D_{x}^{-1}DD_{x}[h]}V_{x}^{1/2}A\\
 & =A^{\top}V_{x}^{1/2}\overline{D}_{x}V_{x}^{1/2}A,
\end{align*}
where $\overline{D}_{x}:=-2S_{x,h}+D_{x}^{-1}DD_{x}[h]$. 

For $P_{x}:=P\Par{V_{x}^{\half}A}=V_{x}^{\half}A\Par{A^{\top}V_{x}A}^{-1}A^{\top}V_{x}^{\half}=V_{x}^{\half}Ag^{-1}A^{\top}V_{x}^{\half}$,
\begin{align*}
\norm{g(x)^{-\half}Dg(x)[h]g^{-\half}}_{F}^{2} & =\tr\Par{g^{-1}Dg[h]g^{-1}Dg[h]}\\
 & =\tr\Par{g^{-1}A^{\top}V_{x}^{1/2}\overline{D}_{x}V_{x}^{1/2}Ag^{-1}A^{\top}V_{x}^{1/2}\overline{D}_{x}V_{x}^{1/2}A}\\
 & =\tr\Par{P_{x}\overline{D}_{x}P_{x}\overline{D}_{x}}\\
 & \underset{\text{(i)}}{=}\diag\Par{\overline{D}_{x}}^{\top}P_{x}^{(2)}\diag\Par{\overline{D}_{x}}\\
 & \underset{\text{(ii)}}{\leq}\diag\Par{\overline{D}_{x}}^{\top}\Sigma_{x}\diag\Par{\overline{D}_{x}}\\
 & \underset{\text{(iii)}}{\leq}4\sum_{i=1}^{m}\sigma\Par{\sqrt{D_{x}}A_{x}}_{i}\Par{(A_{x}h)_{i}^{2}+(D_{x}^{-1}DD_{x}[h])_{i}^{2}}\\
 & \leq4\max_{i}\Par{\frac{\sigma\Par{\sqrt{D_{x}}A_{x}}_{i}}{(D_{x})_{ii}}}\cdot\sum_{i=1}^{m}(D_{x})_{ii}\Par{(A_{x}h)_{i}^{2}+(D_{x}^{-1}DD_{x}[h])_{i}^{2}}\\
 & \underset{\text{(iv)}}{=}4\max_{i}\Par{\frac{\sigma\Par{\sqrt{D_{x}}A_{x}}_{i}}{(D_{x})_{ii}}}\cdot\Par{\norm h_{g(x)}^{2}+\sum_{i=1}^{m}\Par{D_{x}^{-1}}_{ii}(DD_{x}[h])_{i}^{2}}
\end{align*}
where (i) holds due to $x^{\top}(A\hada B)y=\tr\Par{\Diag(x)A\Diag(y)B^{\top}}$
(Lemma~\ref{lem:Hadamard}), (ii) follows from $P_{x}^{(2)}\preceq\Sigma_{x}$
(Lemma~\ref{lem:schurProjection}), (iii) uses $(a+b)^{2}\leq2\Par{a^{2}+b^{2}}$
for $a,b\in\R$ and $\Sigma_{x}=\Diag(P_{x})=\Diag\Par{\sigma\Par{\sqrt{V_{x}}A}}=\Diag\Par{\sigma\Par{\sqrt{D_{x}}A_{x}}}$,
and (iv) holds due to $\sum_{i=1}^{m}D_{x,i}(A_{x}h)_{i}^{2}=h^{\top}A_{x}^{\top}D_{x}A_{x}h=h^{\top}g(x)h$.

For the second claim,
\begin{align*}
\max_{h:\norm h_{g(x)}=1}\norm{A_{x}h}_{\infty} & =\max_{h:\norm h_{g(x)}=1}\max_{i\in[m]}\Abs{\frac{a_{i}^{\top}h}{s_{i}}}\\
 & =\max_{i\in[m]}\max_{u:\norm u_{2}=1}\Abs{\frac{a_{i}^{\top}g(x)^{-1/2}u}{s_{i}}}\\
 & =\max_{i\in[m]}\norm{g(x)^{-1/2}\frac{a_{i}}{s_{i}}}_{2}\\
 & =\max_{i\in[m]}\sqrt{\frac{1}{s_{i}^{2}}a_{i}^{\top}g(x)^{-1}a_{i}}\\
 & =\sqrt{\max_{i\in[m]}e_{i}^{\top}A_{x}^{\top}g(x)^{-1}A_{x}e_{i}}\\
 & =\sqrt{\max_{i\in[m]}\frac{\sigma\Par{\sqrt{D_{x}}A_{x}}_{i}}{(D_{x})_{ii}}}.
\end{align*}

For the last claim, for $h\in\Rn$ such that $\norm{A_{x}h}_{\infty}\leq1$
(i.e., $h\in K\cap(2x-K)$ for $K=\{Ax\leq b\}$ due to Lemma~\ref{lem:symmforPolytope})
we have
\begin{align*}
h^{\top}g(x)h & =h^{\top}A_{x}^{\top}D_{x}A_{x}h=\sum_{i=1}^{m}(D_{x})_{ii}(A_{x}h)_{i}^{2}\\
 & \leq\norm{A_{x}h}_{\infty}^{2}\sum_{i=1}^{m}(D_{x})_{ii}\leq\tr\Par{D_{x}}.\qedhere
\end{align*}
\end{proof}
Using this, we can check strong self-concordance and compute the symmetry
parameter of metrics of the form $A_{x}^{\top}D_{x}A_{x}$.
\begin{lem}
\label{lem:paramsBarrier} Let $A\in\R^{m\times n}$ and $\Sigma_{x}=\Diag(\sigma(A_{x}))\in\R^{m\times m}$.
\begin{itemize}
\item Logarithmic metric: $g(x)=A_{x}^{\top}A_{x}$ with $D_{x}=I_{m}$
is strongly self-concordant with $\onu=m$.
\item Approximate volumetric metric: $g(x)=40\sqrt{m}A_{x}^{\top}\Sigma_{x}A_{x}$
with $D_{x}=40\sqrt{m}\Sigma_{x}$ is strongly self-concordant with
$\onu=O(\sqrt{m}n)$. 
\item Vaidya metric: $g(x)=22\sqrt{\frac{m}{n}}A_{x}^{\top}\Par{\Sigma_{x}+\frac{n}{m}I_{m}}A_{x}$
with $D_{x}=22\sqrt{\frac{m}{n}}\Par{\Sigma_{x}+\frac{n}{m}I_{m}}$
is strongly self-concordant with $\onu=O(\sqrt{mn})$. 
\end{itemize}
\end{lem}

We defer the proof of the first and second to Appendix~\ref{app:subsec:logBarrier}
and Appendix~\ref{app:subsec:volBarrier}, respectively.
\begin{proof}
Consider the metric without scaling: $g(x):=A_{x}^{\top}D_{x}A_{x}$,
where we set $D_{x}=\Sigma_{x}+\frac{n}{m}I_{m}$. Then
\begin{align}
\max_{i}\Par{\frac{\sigma\Par{\sqrt{D_{x}}A_{x}}_{i}}{D_{x,i}}} & \underset{\text{Lemma \ref{lem:helper4Diagonal}-2}}{=}\Par{\max_{h\in\Rn}\frac{\norm{A_{x}h}_{\infty}}{\norm h_{g(x)}}}^{2}\underset{\text{(i)}}{\leq}\sqrt{\frac{m}{n}},\label{eq:28-1}\\
\sum_{i=1}^{m}\Par{D_{x}^{-1}}_{ii}(DD_{x}[h])_{i}^{2} & \underset{\text{(ii)}}{\leq}\sum_{i=1}^{m}\Par{\Sigma_{x}^{-1}}_{ii}(D\Sigma_{x}[h])_{i}^{2}\nonumber \\
 & \underset{\text{Lemma \ref{lem:usefulFactLeverage}-3}}{\leq}4h^{\top}A_{x}^{\top}\Sigma_{x}A_{x}h\leq4\norm h_{g(x)}^{2},\nonumber 
\end{align}
where (i) follows from (4.5) of \cite{anstreicher1997volumetric}
and (ii) holds due to $\Sigma_{x}\preceq D_{x}$. Putting these back
to Lemma~\ref{lem:helper4Diagonal}-1,
\begin{align*}
\norm{g(x)^{-\half}Dg(x)[h]g^{-\half}}_{F}^{2} & \leq4\max_{i}\Par{\frac{\sigma\Par{\sqrt{D_{x}}A_{x}}_{i}}{D_{x,i}}}\cdot\Par{\norm h_{g(x)}^{2}+\sum_{i=1}^{m}(D_{x}^{-1})_{ii}(DD_{x}[h])_{i}^{2}}\\
 & \leq20\sqrt{\frac{m}{n}}\norm h_{g(x)}^{2}.
\end{align*}
Thus, $\tilde{g}(x):=22\sqrt{\frac{m}{n}}g(x)=22\sqrt{\frac{m}{n}}A_{x}^{\top}\Par{\Sigma_{x}+\frac{n}{m}I_{m}}A_{x}$
is strongly self-concordant.

For the symmetry parameter, Lemma~\ref{lem:helper4Diagonal}-2 implies
that for $y\in\dcal_{g}^{1}(x)$
\[
\norm{A_{x}(y-x)}_{\infty}\leq\norm{y-x}_{g(x)}\sqrt{\max_{i\in[m]}\frac{\sigma\Par{\sqrt{D_{x}}A_{x}}_{i}}{(D_{x})_{ii}}}\underset{\text{\eqref{eq:28-1}}}{\leq}\Par{\frac{m}{n}}^{1/4}.
\]
Also, Lemma~\ref{lem:helper4Diagonal}-3 implies that $y$ with $\norm{A_{x}(y-x)}_{\infty}\leq1$
is contained in $\dcal_{g}^{\sqrt{\tr(D_{x})}}(x)$, where
\[
\tr\Par{D_{x}}=\tr\Par{\Sigma_{x}+\frac{n}{m}I_{m}}=\tr\Par{A_{x}(A_{x}^{\top}A_{x})^{-1}A_{x}}+n=\tr\Par{I_{n}}+n=2n.
\]
Therefore, $\tilde{g}(x)=22\sqrt{\frac{m}{n}}g(x)=22\sqrt{\frac{m}{n}}A_{x}^{\top}\Par{\Sigma_{x}+\frac{n}{m}I_{m}}A_{x}$
ensures
\[
\dcal_{\tilde{g}}^{1}(x)\subset K\cap(2x-K)\subset\dcal_{\tilde{g}}^{\sqrt{44(mn)^{1/2}}}(x),
\]
so $\tilde{g}$ is $O(\sqrt{mn})$-symmetric.
\end{proof}

\subsection{Lower trace self-concordance \label{subsec:lowerSC-hybrid}}

We demonstrate that the matrix functions $g_{1}\in\R^{n\times n}$
and $g_{2}=44\sqrt{\frac{m}{n}}A_{x}^{\top}(\Sigma_{x}+\frac{n}{m}I)A_{x}$
satisfy
\[
\tr\Par{(g_{1}+g_{2})^{-1}D^{2}g_{2}[h,h]}\geq-\norm h_{g_{2}}^{2}\quad\text{for }h\in\Rn.
\]
Later, we set $g_{1}=-2n\hess\log\det$ so that $g=g_{1}+g_{2}$ is
equal to the metric in (\ref{eq:metricHybrid}), proving Theorem~\ref{thm:hybridPSD}
via our framework.

Let $\theta_{1}(x):=A_{x}^{\top}\Sigma_{x}A_{x}$, $\theta_{2}(x):=A_{x}^{\top}A_{x}$,
and $\Gamma_{x}:=\Diag\Par{A_{x}g(x)^{-1}A_{x}^{\top}}\in\R^{m\times m}$.
We define $s_{x,h}:=A_{x}$ and $S_{x,h}:=\Diag(s_{x,h})$. Recall
that for a matrix function $g(x)$ we use $g_{x,h}'$ and $g_{x,h}''$
to denote $Dg(x)[h]$ and $D^{2}g(x)[h,h]$.

\begin{restatable}{propre}{propCalculusLeverage} \label{prop:calculusLeverage}
For $x,h\in\Rn$, let $P_{x}=A_{x}(A_{x}^{\top}A_{x})^{-1}A_{x}^{\top}$,
$\Sigma_{x}=\Diag(P_{x})$, and $\Lambda_{x}=\Sigma_{x}-P_{x}^{(2)}$.
Denote $\theta_{1}(x):=A_{x}^{\top}\Sigma_{x}A_{x}$ and $\theta_{2}(x):=A_{x}^{\top}A_{x}$.
\begin{itemize}
\item $\Sigma_{x,h}'=-2\Diag\Par{\Lambda_{x}s_{x,h}}$.
\item $P_{x,h}'=-P_{x}S_{x,h}-S_{x,h}P_{x}+2P_{x}S_{x,h}P_{x}$.
\item $\Lambda_{x,h}'=-2\Diag(\Lambda_{x}s_{x,h})+2P_{x}\circ P_{x}S_{x,h}+2S_{x,h}P_{x}\circ P_{x}-2(P_{x}S_{x,h}P_{x})\circ P_{x}-2P_{x}\circ(P_{x}S_{x,h}P_{x})$.
\item $\Sigma_{x,h}''=6S_{x,h}\Sigma_{x}S_{x,h}+8\Diag\Par{P_{x}S_{x,h}P_{x}S_{x,h}P_{x}}-6\Diag\Par{P_{x}S_{x,h}^{2}P_{x}}-8\Diag\Par{S_{x,h}P_{x}S_{x,h}P_{x}}$.
\item $D\theta_{1}(x)[h]=-2A_{x}^{\top}\Sigma_{x}S_{x,h}A_{x}+A_{x}^{\top}D\Sigma_{x}[h]A_{x}$.
\item $D^{2}\theta_{1}(x)[h,h]=6A_{x}^{\top}S_{x,h}\Sigma_{x}S_{x,h}A_{x}-4A_{x}^{\top}D\Sigma_{x}[h]S_{x,h}A_{x}+A_{x}^{\top}D^{2}\Sigma_{x}[h,h]A_{x}$.
\item $D\theta_{2}(x)[h]=-2A_{x}^{\top}S_{x,h}A_{x}$ and $D^{2}\theta_{2}(x)[h,h]=6A_{x}^{\top}S_{x,h}^{2}A_{x}$.
\end{itemize}
\end{restatable}

We defer the proof to Appendix~\ref{app:subsec:derivativeMatrices}.
\begin{lem}
\label{lem:HybridGammaNorm} $\norm{\Gamma_{x}}_{\infty}\leq\frac{1}{44}$. 
\end{lem}

\begin{proof}
Note that $0\preceq\Gamma_{x}=\Diag\Par{A_{x}g^{-1}A_{x}^{\top}}\preceq\Diag\Par{A_{x}g_{2}^{-1}A_{x}^{\top}}$.
For $\og_{2}:=\theta_{1}+\frac{n}{m}\theta_{2}=\frac{1}{44}\sqrt{\frac{n}{m}}g_{2}$,
we have
\[
\norm{\Diag\Par{A_{x}\og_{2}^{-1}A_{x}^{\top}}}_{\infty}=\max_{i\in[m]}\frac{\sigma\Par{\sqrt{\Sigma_{x}+\frac{n}{m}I}A_{x}}_{i}}{\Par{\Sigma_{x}+\frac{n}{m}I}_{ii}}\underset{\text{\eqref{eq:28-1}}}{\leq}\sqrt{\frac{m}{n}},
\]
and thus
\begin{align*}
\norm{\Gamma_{x}}_{\infty} & \leq\norm{\Diag\Par{A_{x}g_{2}^{-1}A_{x}^{\top}}}_{\infty}=\frac{1}{44}\sqrt{\frac{n}{m}}\norm{\Diag\Par{A_{x}\og_{2}^{-1}A_{x}^{\top}}}_{\infty}\leq\frac{1}{44}.\qedhere
\end{align*}
\end{proof}
\begin{lem}
\label{lem:hybridLowerSCTrace} $\tr\Par{g^{-1}D^{2}g_{2}(x)[h,h]}\geq-\half\norm h_{g_{2}}^{2}$.
\end{lem}

\begin{proof}
As $D^{2}\theta_{2}(x)[h,h]\succeq0$ by Claim~\ref{claim:diffLogBarrier},
we have $\tr\Par{g^{-1}D^{2}\theta_{2}(x)[h,h]}=\tr\Par{g^{-\half}D^{2}\theta_{2}(x)[h,h]g^{-\half}}\geq0$.
For $\theta_{1}$, Proposition~\ref{prop:calculusLeverage}-6 leads
to $D^{2}\theta_{1}[h,h]\succeq-4A_{x}^{\top}\Sigma_{x,h}'S_{x,h}A_{x}+A_{x}^{\top}\Sigma_{x,h}''A_{x}$,
so it suffices to provide
\begin{align*}
\text{Upper bound:} & \ \tr\Par{g^{-1}A_{x}^{\top}\Sigma_{x,h}'S_{x,h}A_{x}}=\tr\Par{A_{x}g^{-1}A_{x}^{\top}\Sigma_{x,h}'S_{x,h}}\underset{\text{(i)}}{=}\tr\Par{\Gamma_{x}\Sigma_{x,h}'S_{x,h}},\\
\text{Lower bound:} & \ \tr\Par{g^{-1}A_{x}^{\top}\Sigma_{x,h}''A_{x}}=\tr\Par{A_{x}g^{-1}A_{x}^{\top}\Sigma_{x,h}''}\underset{\text{(i)}}{=}\tr\Par{\Gamma_{x}\Sigma_{x,h}''},
\end{align*}
where in (i) we used $\tr\Par{AD}=\tr\Par{\Diag(A)D}$ for a diagonal
matrix $D$.

For the upper bound, as diagonal matrices commute,
\begin{align*}
\tr\Par{\Gamma_{x}\Sigma_{x,h}'S_{x,h}} & \leq\norm{\Gamma_{x}}_{\infty}\tr\Par{\Par{S_{x,h}\Sigma_{x,h}'^{2}S_{x,h}}^{\half}}\\
 & =\norm{\Gamma_{x}}_{\infty}\tr\Par{\Par{\Sigma_{x,h}'\Sigma_{x}^{-1}\Sigma_{x,h}'S_{x,h}\Sigma_{x}S_{x,h}}^{\half}}\\
 & \underset{\text{(i)}}{=}\norm{\Gamma_{x}}_{\infty}\tr\Par{\Par{\Sigma_{x,h}'\Sigma_{x}^{-1}\Sigma_{x,h}'}^{\half}\Par{S_{x,h}\Sigma_{x}S_{x,h}}^{\half}}\\
 & \underset{\text{(ii)}}{\leq}\norm{\Gamma_{x}}_{\infty}\sqrt{\tr\Par{\Sigma_{x,h}'\Sigma_{x}^{-1}\Sigma_{x,h}'}}\sqrt{\tr\Par{S_{x,h}\Sigma_{x}S_{x,h}}}\\
 & =\norm{\Gamma_{x}}_{\infty}\|\Sigma_{x}^{-1}\underbrace{\sigma_{x,h}'}_{:=\diag(D\Sigma_{x}[h])}\|_{\Sigma_{x}}\norm h_{\theta_{1}}\\
 & \underset{\text{(iii)}}{\leq}2\norm{\Gamma_{x}}_{\infty}\norm h_{\theta_{1}}^{2},
\end{align*}
where (i) holds since both $\Sigma_{x,h}'\Sigma_{x}^{-1}\Sigma_{x,h}'$
and $S_{x,h}\Sigma_{x}S_{x,h}$ are PD diagonal matrices, (ii) follows
from the Cauchy-Schwartz inequality, and we used in (iii) $\norm{\Sigma_{x}^{-1}\sigma_{x,h}'}_{\Sigma_{x}}\leq2\norm h_{\theta_{1}}$
(Lemma~\ref{lem:usefulFactLeverage}-3).

For the lower bound, we recall from Proposition~\ref{prop:calculusLeverage}-4
\begin{align*}
\Sigma_{x,h}'' & =6S_{x,h}\Sigma_{x}S_{x,h}+8\Diag\Par{P_{x}S_{x,h}P_{x}S_{x,h}P_{x}}-6\Diag\Par{P_{x}S_{x,h}^{2}P_{x}}-8\Diag\Par{S_{x,h}P_{x}S_{x,h}P_{x}}\\
 & \succeq-6\Diag\Par{P_{x}S_{x,h}^{2}P_{x}}-8\Diag\Par{S_{x,h}P_{x}S_{x,h}P_{x}},
\end{align*}
and thus
\begin{align*}
\tr\Par{\Gamma_{x}\Sigma_{x,h}''} & \geq-6\tr\Par{\Gamma_{x}P_{x}S_{x,h}^{2}P_{x}}-8\tr\Par{\Gamma_{x}S_{x,h}P_{x}S_{x,h}P_{x}}.
\end{align*}
For the first term,
\begin{align}
\tr\Par{\Gamma_{x}P_{x}S_{x,h}^{2}P_{x}} & =\tr\Par{S_{x,h}P_{x}\Gamma_{x}P_{x}S_{x,h}}\leq\norm{\Gamma_{x}}_{\infty}\tr\Par{S_{x,h}P_{x}S_{x,h}}\underset{\text{(i)}}{=}\norm{\Gamma_{x}}_{\infty}s_{x,h}^{\top}\Par{P_{x}\circ I}s_{x,h}\label{eq:trSPS}\\
 & =\norm{\Gamma_{x}}_{\infty}s_{x,h}^{\top}\Sigma_{x}s_{x,h}=\norm{\Gamma_{x}}_{\infty}\norm h_{\theta_{1}}^{2},\nonumber 
\end{align}
where (i) follows from $x^{\top}(A\circ B)y=\tr\Par{\Diag(x)A\Diag(y)B^{\top}}$
(Lemma~\ref{lem:Hadamard}). For the second term,
\begin{align*}
\Abs{\tr\Par{\Gamma_{x}S_{x,h}P_{x}S_{x,h}P_{x}}} & =\Abs{\tr\Par{\Gamma_{x}^{1/2}S_{x,h}P_{x}\cdot S_{x,h}P_{x}\Gamma_{x}^{1/2}}}\\
 & \leq\sqrt{\tr\Par{\Gamma_{x}^{1/2}S_{x,h}P_{x}^{2}S_{x,h}\Gamma_{x}^{1/2}}}\sqrt{\tr\Par{\Gamma_{x}^{1/2}P_{x}S_{x,h}^{2}P_{x}\Gamma_{x}^{1/2}}}\\
 & =\sqrt{\tr\Par{P_{x}S_{x,h}\Gamma_{x}S_{x,h}P_{x}}}\sqrt{\tr\Par{S_{x,h}P_{x}\Gamma_{x}P_{x}S_{x,h}}}\\
 & \leq\norm{\Gamma_{x}}_{\infty}\tr\Par{S_{x,h}P_{x}S_{x,h}}\\
 & =\norm{\Gamma_{x}}_{\infty}\norm h_{\theta_{1}}^{2}.\quad(\text{repeat }(\ref{eq:trSPS}))
\end{align*}
Putting the computations together and using Lemma~\ref{lem:HybridGammaNorm},
\[
\tr\Par{g^{-1}D^{2}\theta_{1}(x)[h,h]}\geq-22\norm{\Gamma_{x}}_{\infty}\norm h_{\theta_{1}}^{2}\geq-\half\norm h_{\theta_{1}}^{2},
\]
and due to $g_{2}=44\sqrt{\frac{m}{n}}\Par{\theta_{1}+\frac{n}{m}\theta_{2}}$,
\[
\tr\Par{g^{-1}D^{2}g_{2}(x)[h,h]}\geq-\half\norm h_{g_{2}}^{2}.\qedhere
\]
\end{proof}
Combining all the lemmas so far together, we prove Theorem~\ref{thm:hybridPSD}.

\thmHybridPSD*
\begin{proof}
As in (\ref{eq:metricHybrid}), we set
\begin{align*}
g(X) & =2\Par{ng_{1}(X)+g_{2}(X)},\ \text{where}\\
g_{1}(X) & =M^{\top}(X\kro X)^{-1}M,\\
g_{2}(X) & =22\sqrt{\frac{m}{n}}M^{\top}A_{X}^{\top}\Par{\Sigma_{X}+\frac{n}{m}I_{m}}A_{X}M.
\end{align*}
Since $ng_{1}$ and $g_{2}$ are strongly self-concordant (see Corollary~\ref{cor:strongSCofLOGDET}
and Lemma~\ref{lem:paramsBarrier}-3), $g$ is also strongly self-concordant
due to Lemma~\ref{lem:sumStrongSC} and $O(n^{2}+\sqrt{mn^{2}})$-symmetric\footnote{Since the dimension is $d$ in the setting of (\ref{eq:PSDcone}),
we should replace $n$ by $d=O(n^{2})$ when applying Lemma \ref{lem:paramsBarrier}.} due to Lemma~\ref{lem:sumSymmetricSC}. Lastly for lower trace self-concordance
of $g$, we recall that $D^{2}g_{1}[H,H]\succeq0$ as checked in (\ref{eq:D4ph1})
and $\tr\Par{g^{-1}D^{2}g_{2}(x)[h,h]}\geq-\half\norm h_{g_{2}}^{2}$
by Lemma~\ref{lem:hybridLowerSCTrace}, so we can use Lemma~\ref{lem:additiveCondition}-2
to obtain lower trace self-concordance. Therefore, Theorem~\ref{thm:generalMixing}
ensures that the $\dw$ with the Vaidya metric $g$ mixes in $\otilde{n^{2}\Par{n^{2}+\sqrt{mn^{2}}}}=\otilde{n^{3}\Par{n+\sqrt{m}}}$
steps.

Now we bound the time for each step of this $\dw$ (Algorithm~\ref{alg:DikinWalk}).
Recall that it requires (1) the update of the leverage scores, (2)
computation of the matrix function induced by the local metric $g$,
(3) the inverse of the matrix function and (4) its determinant. By
Theorem~46 in \cite{lee2019solving} (with $p=2$ and $n\gets d$
therein), the initialization of the leverage scores at the beginning
takes $\otilde{mn^{2\omega}}$ and their updates takes $\otilde{mn^{2(\omega-1)}}$
time. Since (1) takes $\otilde{mn^{2(\omega-1)}}$, (2) takes $\otilde{n^{4}+mn^{2(\omega-1)}}$,
and (3) and (4) take $O(n^{2\omega})$, each iteration runs in $\otilde{n^{2\omega}+mn^{2(\omega-1)}}$
time. Note that even though the initialization of leverage scores
takes $\otilde{mn^{2\omega}}$ time, the amortized per step time complexity
(since the mixing rate is $\otilde{n^{3}(n+\sqrt{m})}$) goes to $\otilde{n^{2\omega}+mn^{2(\omega-1)}}=\otilde{mn^{2(\omega-1)}}$
time.
\end{proof}

