%% LyX 2.3.6.2 created this file.  For more info, see http://www.lyx.org/.
%% Do not edit unless you really know what you are doing.
\documentclass[11pt,english]{article}
\usepackage{lmodern}
\renewcommand{\sfdefault}{lmss}
\renewcommand{\ttdefault}{lmtt}
\usepackage[T1]{fontenc}
\usepackage[latin9]{inputenc}
\usepackage{geometry}
\geometry{verbose,tmargin=1in,bmargin=1in,lmargin=1in,rmargin=1in}
\usepackage{color}
\usepackage{array}
\usepackage{booktabs}
\usepackage{bm}
\usepackage{multirow}
\usepackage{amsmath}
\usepackage{amsthm}
\usepackage{amssymb}

\makeatletter

%%%%%%%%%%%%%%%%%%%%%%%%%%%%%% LyX specific LaTeX commands.
%% Because html converters don't know tabularnewline
\providecommand{\tabularnewline}{\\}

%%%%%%%%%%%%%%%%%%%%%%%%%%%%%% Textclass specific LaTeX commands.
\numberwithin{figure}{section}
\numberwithin{equation}{section}
\theoremstyle{definition}
\newtheorem*{problem*}{\protect\problemname}
\theoremstyle{plain}
\newtheorem{thm}{\protect\theoremname}
\theoremstyle{definition}
\newtheorem{defn}[thm]{\protect\definitionname}
\theoremstyle{plain}
\newtheorem{lem}[thm]{\protect\lemmaname}
\theoremstyle{plain}
\newtheorem{prop}[thm]{\protect\propositionname}
\theoremstyle{plain}
\newtheorem{cor}[thm]{\protect\corollaryname}
\theoremstyle{remark}
\newtheorem*{acknowledgement*}{\protect\acknowledgementname}
\theoremstyle{remark}
\newtheorem{claim}[thm]{\protect\claimname}

%%%%%%%%%%%%%%%%%%%%%%%%%%%%%% User specified LaTeX commands.
\numberwithin{equation}{section}
\numberwithin{figure}{section}
\theoremstyle{plain}
\newtheorem{thm2}{\protect\theoremname}
\theoremstyle{definition}
\newtheorem{defn2}[thm2]{\protect\definitionname}
\theoremstyle{plain}
\newtheorem*{thm2*}{\protect\theoremname}
\theoremstyle{plain}
\newtheorem{lem2}[thm2]{\protect\lemmaname}
\theoremstyle{plain}
\newtheorem{prop2}[thm2]{\protect\propositionname}
\theoremstyle{plain}
\newtheorem{cor2}[thm2]{\protect\corollaryname}


\usepackage{hyperref}
\hypersetup{
    colorlinks,
    allcolors=magenta
}
\usepackage[lined,boxed,ruled,norelsize,algo2e]{algorithm2e}
\@addtoreset{section}{part}
\usepackage{graphicx}
\usepackage{thmtools}
\usepackage{microtype}
\usepackage{mathrsfs}
\usepackage{ragged2e}
\usepackage{thm-restate}
\usepackage{caption}
\usepackage{subcaption}

\allowdisplaybreaks
\usepackage{natbib}
\bibliographystyle{plainnat}
\bibpunct{(}{)}{;}{a}{,}{,}
\declaretheorem[name=Theorem,sibling=thm2]{thmre}
\declaretheorem[name=Lemma,sibling=thm2]{lemre}
\declaretheorem[name=Proposition,sibling=thm2]{propre}
\declaretheorem[name=Corollary,sibling=thm2]{corre}

\usepackage{tikz}
\usetikzlibrary{calc}
\usetikzlibrary{positioning, arrows.meta, decorations.pathreplacing}

\makeatother

\usepackage{babel}
\providecommand{\acknowledgementname}{Acknowledgement}
\providecommand{\claimname}{Claim}
\providecommand{\corollaryname}{Corollary}
\providecommand{\definitionname}{Definition}
\providecommand{\lemmaname}{Lemma}
\providecommand{\problemname}{Problem}
\providecommand{\propositionname}{Proposition}
\providecommand{\theoremname}{Theorem}

\begin{document}
\global\long\def\defeq{\stackrel{\mathrm{{\scriptscriptstyle def}}}{=}}%

\global\long\def\norm#1{\left\Vert #1\right\Vert }%
\global\long\def\prob#1{\mathbf{P}\left(#1\right)}%
\global\long\def\Par#1{\left(#1\right)}%
\global\long\def\Brack#1{\left[#1\right]}%
\global\long\def\Abs#1{\left|#1\right|}%
\global\long\def\Brace#1{\left\{  #1\right\}  }%
\global\long\def\inner#1{\left\langle #1\right\rangle }%
\global\long\def\otilde#1{\widetilde{O}\Par{#1}}%
\global\long\def\wtilde{\widetilde{W}}%
\global\long\def\wt#1{\widetilde{#1}}%

\global\long\def\ginverse{g^{-1}}%
\global\long\def\ginv#1{g(#1)^{-1}}%
\global\long\def\normg#1{\left\Vert #1\right\Vert _{g}}%
\global\long\def\normginv#1{\left\Vert #1\right\Vert _{g^{-1}}}%

\global\long\def\cred#1{\textcolor{red}{#1}}%
\global\long\def\cblue#1{\textcolor{blue}{#1}}%
\global\long\def\cgreen#1{\textcolor{green}{#1}}%
\global\long\def\ccyan#1{\textcolor{cyan}{#1}}%

\global\long\def\R{\mathbb{R}}%
\global\long\def\Rn{\mathbb{R}^{n}}%
\global\long\def\Rnn{\mathbb{R}^{n\times n}}%
\global\long\def\E{\mathbb{E}}%
\global\long\def\P{\mathbb{P}}%
\global\long\def\S{\mathbb{S}}%
\global\long\def\N{\mathbb{N}}%

\global\long\def\acal{\mathcal{A}}%
\global\long\def\bcal{\mathcal{B}}%
\global\long\def\ccal{\mathcal{C}}%
\global\long\def\dcal{\mathcal{D}}%
\global\long\def\ecal{\mathcal{E}}%
\global\long\def\fcal{\mathcal{F}}%
\global\long\def\gcal{\mathcal{G}}%
\global\long\def\hcal{\mathcal{H}}%
\global\long\def\ical{\mathcal{I}}%
\global\long\def\tcal{\mathbb{\mathcal{T}}}%
\global\long\def\mcal{\mathbb{\mathcal{M}}}%
\global\long\def\pcal{\mathcal{P}}%
\global\long\def\ncal{\mathcal{N}}%
\global\long\def\kcal{\mathcal{K}}%

\global\long\def\veps{\varepsilon}%
\global\long\def\lda{\lambda}%
\global\long\def\vphi{\varphi}%

\global\long\def\half{\frac{1}{2}}%

\global\long\def\bw{\textsf{Ball walk}}%
\global\long\def\dw{\textup{\textsf{Dikin walk}}}%

\global\long\def\tr{\textsf{\textup{Tr}}}%
\global\long\def\diag{\textsf{\textup{diag}}}%
\global\long\def\cov{\mathrm{Cov}}%
\global\long\def\Var{\mathrm{Var}}%
\global\long\def\rank{\mathrm{rank}}%
\global\long\def\range{\mathrm{Range}}%
\global\long\def\nulls{\mathrm{Null}}%
\global\long\def\Diag{\textup{\textsf{Diag}}}%
\global\long\def\vec{\textup{\textsf{vec}}}%
\global\long\def\vol{\textup{\textsf{vol}}}%
\global\long\def\svec{\textup{\textsf{svec}}}%
\global\long\def\id{\mathrm{id}}%
\global\long\def\dtv{d_{\text{TV}}}%
\global\long\def\st{\mathrm{s.t.\ }}%
\global\long\def\nnz{\textup{\textsf{nnz}}}%

\global\long\def\ov{\overline{v}}%
\global\long\def\ox{\overline{x}}%
\global\long\def\og{\overline{g}}%
\global\long\def\onu{\overline{\nu}}%

\global\long\def\tx{\widetilde{x}}%
\global\long\def\tv{\widetilde{v}}%
\global\long\def\tp{\tilde{p}}%

\global\long\def\dv{\delta_{v}}%
\global\long\def\dx{\delta_{x}}%
\global\long\def\del{\mathcal{\partial}}%
\global\long\def\grad{\nabla}%
\global\long\def\hess{\nabla^{2}}%
\global\long\def\psd{\S_{+}^{n}}%

\global\long\def\kro{\otimes}%
\global\long\def\hada{\circ}%

\title{Efficiently Sampling the PSD Cone with the Metric Dikin Walk\author{Yunbum Kook\\ Georgia Tech\\  \texttt{yb.kook@gatech.edu} \and Santosh S. Vempala\\ Georgia Tech\\ \texttt{vempala@gatech.edu}}}
\maketitle
\begin{abstract}
Semi-definite programs represent a frontier of efficient computation.
While there has been much progress on semi-definite optimization,
with moderate-sized instances currently solvable in practice by the
interior-point method, the basic problem of \emph{sampling} semi-definite
solutions remains a formidable challenge. The direct application of
known polynomial-time algorithms for sampling general convex bodies
to semi-definite sampling leads to a prohibitively high running time.
In addition, known general methods require an expensive \emph{rounding
}phase as pre-processing. Here we analyze the Dikin walk, by first
adapting it to general metrics, then devising suitable metrics for
the PSD cone with affine constraints. The resulting mixing time and
per-step complexity are considerably smaller, and by an appropriate
choice of the metric, the dependence on the number of constraints
can be made polylogarithmic. We introduce a refined notion of self-concordant
matrix functions and give rules for combining different metrics. Along
the way, we further develop the theory of interior-point methods for
sampling.
\end{abstract}

\tableofcontents{}


\section{Introduction}

Semi-definite programs are ubiquitous in the theory of complexity
and algorithms, with applications ranging from optimization to compressed
sensing to error-correcting codes to sphere packing to extremal graph
theory to solving unique games, among many others. There has been
much progress on semi-definite optimization over the past half-century,
with moderate-sized instances currently solvable in practice by the
interior-point method. Here we consider the basic problem of \emph{sampling}
semi-definite solutions. This remains a formidable challenge in theory
and in practice, with many applications (to volume computation/integration,
Bayesian inference, systems biology, data privacy etc.). Formally,
we have the following problem.
\begin{problem*}
Given matrices $A_{i}$ of size $n\times n$ and $b_{i}\in\R$, for
$i\in[m]$, 
\begin{align}
\text{uniformly sample \ensuremath{X} subject to } & \inner{A_{i},X}\leq b_{i}\ \forall i\in[m]\text{ and }X\succeq0.\label{eq:PSDcone}
\end{align}
\end{problem*}
There are polynomial-time algorithms for sampling general convex bodies
in the general membership oracle model. The direct application of
these algorithms to semi-definite programs leads to a prohibitively
high running time. In addition, the known general methods require
an expensive \emph{rounding }phase as pre-processing. The $\bw$ takes
$n^{3}$ membership queries for a convex body in $\R^{n}$ for the
first sample, which implies $n^{6}$ membership tests for the PSD
sampling problem since the dimension is effectively $n^{2}$. Each
membership test consists of checking the linear constraints and the
semi-definiteness constraint, leading to an overall complexity of
$n^{6}(n^{w}+mn^{2})$ arithmetic operations (ignoring logarithmic
factors). Besides the high polynomial running time, a more critical
issue is space. The $\bw$ needs to compute and maintain an affine
transformation to achieve its convergence time. For a convex body
in $\R^{n}$, this would be an $n\times n$ matrix. For PSD sampling
it is effectively an $n^{2}\times n^{2}$ matrix, which makes it impractical
for $n$ larger than a few hundred. Although the $\bw$ is now practical
for sampling polytopes in very high dimension, PSD sampling is currently
out of reach even for moderate-sized instances.

How can we sample faster? A natural idea is to look for an affine-invariant
sampling algorithm that would not need to maintain an affine transformation.
The simplest of these is the $\dw$, first analyzed for polytopes
by \cite{kannan2012random}. The $\dw$ is a generalization of the
$\bw$. For each point of the target domain, we define an ellipsoid
and use it to pick a random next point (rather than a fixed ball at
any point, as in the $\bw$). The local ellipsoid is defined in an
affine-invariant manner, typically via the Hessian of a convex function
over the convex set. This approach was used by \cite{kannan2012random}
to show that the mixing time is bounded by $O(mn)$ for a polytope
in $\R^{n}$ defined by $m$ inequalities. Historically, the Dikin
ellipsoid was proposed by \cite{dikin1967iterative} as the main ingredient
of a method for convex optimization and was the first ``interior-point''
method. 
% Figure environment removed

A local metric $g$ is defined at each point $x$ in a set $K\subset\Rn$
by a positive definite matrix $g(x)$ and naturally induces the local
norm as $\norm v_{g(x)}:=\sqrt{v^{\top}g(x)v}$. Using this we can
define the Metric Dikin walk, which is the main object of study in
this paper.

\paragraph{Metric Dikin walk.}

A Dikin ellipsoid of radius $r$ at $x\in\Rn$ with a local metric
$g$, defined by
\[
\dcal_{g}^{r}(x)\defeq\Brace{y\in\Rn:\sqrt{(y-x)^{\top}g(x)(y-x)}=\norm{y-x}_{g(x)}\leq r},
\]
is a ball defined by the local metric. From this perspective, the
$\dw$ defined below is a natural generalization of the $\bw$ to
a local metric setting.

\medskip{}

\begin{algorithm2e}[H]

\caption{$\dw$}\label{alg:DikinWalk}

\SetAlgoLined

\textbf{Input:} Initial distribution $\pi_{0}$, step size $r$, local
metric $g$.

\textbf{Output:} $x_{T}$

Draw an initial point $x_{0}\sim\pi_{0}$ at random. 

\For{$t=1,\cdots,T$}{

Pick $y$ from $\dcal_{g}^{r}(x_{t-1})$ uniformly at random.

Set $x_{t}\gets y$ with probability $\min\Par{1,\frac{\vol\Par{\dcal_{g}^{r}(x_{t-1})}}{\vol\Par{\dcal_{g}^{r}(y)}}}$.
Otherwise, $x_{t}\gets x_{t-1}$.

}

\end{algorithm2e}

\medskip{}

This walk was studied in \cite{kannan2012random,laddha2020strong}
to sample polytopes using the standard logarithmic barrier and by
\cite{laddha2020strong} with a faster weighted version of the log
barrier. Their analysis relied on \emph{self-concordance }properties
of these metrics, which we briefly review next. A principal motivation
of the current paper is to go beyond polytopes, and in particular,
to see if the $\dw$ can be effectively used to sample the PSD cone. 

\paragraph{Self-concordance and $\protect\onu$-symmetry. }

While the $\dw$ can be defined in principle for any local metric,
in order to converge efficiently to a desired target distribution,
it is important for the local metric to change slowly, i.e., the derivative
of the metric should be appropriately bounded. This consideration
is also crucial in interior-point algorithms for optimization and
led to the theory of self-concordant functions and barriers, which
capture this property formally. When the local metric $g$ is \emph{self-concordant}
(defined below) on a convex body $K$, the $\dw$ always remains within
$K$ and converges to the uniform distribution over $K$ (due to reversibility
enforced by the rejection step in the last line). As we will see presently,
the analysis of mixing requires additional properties. In the following
discussion, $D^{i}g(x)[h_{1},\cdots,h_{i}]$ denotes the $i$'th directional
derivative of $g$ at $x$ along directions $h_{1},\cdots,h_{i}\in\Rn$.
% Figure environment removed

\begin{defn}
[Self-concordance] \label{def:sc}Let $K\subset\Rn$ be convex and
$g(x):K\to\R^{n\times n}$ a positive definite matrix function\emph{.}
\begin{itemize}
\item \emph{Self-concordance }(operator, \cite{nesterov1994interior}):\emph{
}$\norm{g(x)^{-\half}Dg(x)[h]g(x)^{-\half}}_{2}\leq2\norm h_{g(x)}$
for any $h\in\Rn$ and $x\in K$. 
\item \emph{Strong self-concordance }(Frobenius, \cite{laddha2020strong}):
$\norm{g(x)^{-\half}Dg(x)[h]g(x)^{-\half}}_{F}\leq2\norm h_{g(x)}$\footnote{In general, $\norm A_{2}\leq\norm A_{F}$ and $\norm A_{F}\leq\sqrt{n}\norm A_{2}$
for a matrix $A\in\Rnn$. Hence, if $g$ is self-concordant, then
$ng$ is strongly self-concordant.} for any $h\in\Rn$ and $x\in K$. 
\item \emph{Lower trace self-concordance}: $\tr\Par{g(x)^{-\half}D^{2}g(x)[h,h]g(x)^{-\half}}\geq-\norm h_{g(x)}^{2}$
for any $h\in\Rn$ and $x\in K$.
\item A function $\phi:K\to\R$ is self-concordant or strongly self-concordant
if the matrix function $\hess\phi(x)$ is self-concordant or strongly
self-concordant, respectively.
\item \emph{Self-concordance parameter}: The self-concordance parameter
$\nu$ of a self-concordant barrier function $\phi$ is defined by
$\nu=\max_{x\in K}\grad\phi(x)\Par{\hess\phi(x)}^{-1}\grad\phi(x)$.
\end{itemize}
\end{defn}

\noindent A convex function $\phi:K\rightarrow\R$ is said to be
a \emph{barrier} for a convex set $K$ if $\phi\rightarrow\infty$
as $x\rightarrow\partial K$. Note that strong self-concordance implies
self-concordance and that if $\norm{g^{-\half}Dg[h]g^{-\half}}_{F}\leq2\alpha\norm h_{g}$
for some $\alpha>0$, then $\alpha^{2}g$ is strongly self-concordant;
in other words, the specific constant of $2$ is convention, and any
constant could be used in its place. Next, we recall a symmetry parameter
of a self-concordant metric.
\begin{defn}
[$\onu$-symmetry] \label{def:symm}For a convex set $K\subset\Rn$,
a positive definite matrix function $g(x):K\to\R^{n\times n}$ is
said to be $\onu$-\emph{symmetric} if $\dcal_{g}^{1}(x)\subseteq K\cap(2x-K)\subseteq\dcal_{g}^{\sqrt{\onu}}(x)$
for any $x\in K$.
\end{defn}

We note that $K\cap(2x-K)$ is the locally symmetrized convex body
with respect to $x$. Hence, $\onu$-symmetry measures how accurately
a Dikin ellipsoid approximates the locally symmetrized body (see Figure~\ref{fig:sc-symm}).
One can show that $\onu=O(\nu^{2})$ for any metric induced by a self-concordant
barrier.

\subsection{Results}

We are now ready to state our main results, summarized in Table~\ref{tab:complexity}.

\begin{table}
\begin{centering}
\begin{tabular}{cccc}
\toprule 
Sampler & Mixing rate & Per-step complexity & Citation\tabularnewline
\midrule
\midrule 
\multicolumn{4}{c}{\textbf{Polytope}}\tabularnewline
\midrule 
{\footnotesize{}BW} & $n^{2}\,(+n^{3})$ & $mn$ & \textcolor{black}{\footnotesize{}\cite{kannan1997random,jia2021reducing}}\tabularnewline
\midrule 
\multirow{2}{*}{{\footnotesize{}DW}} & \multirow{2}{*}{$mn$} & $mn^{\omega-1}$ & \textcolor{black}{\footnotesize{}\cite{kannan2012random}}\tabularnewline
\cmidrule{3-4} \cmidrule{4-4} 
 &  & $n^{2}+\nnz(A)$ & \textcolor{black}{\footnotesize{}\cite{laddha2020strong}}\tabularnewline
\midrule 
{\footnotesize{}DW (Vaidya)} & $\sqrt{m}n^{3/2}$ & $mn^{\omega-1}$ & \textcolor{black}{\footnotesize{}\cite{chen2018fast}}\tabularnewline
\midrule 
{\footnotesize{}DW (Lewis)} & $n^{2}$ & $mn^{\omega-1}$ & \textcolor{black}{\footnotesize{}\cite{laddha2020strong}}\tabularnewline
\midrule 
\multicolumn{4}{c}{\textbf{PSD cone}}\tabularnewline
\midrule 
{\footnotesize{}BW} & $n^{4}\,(+n^{6})$ & $mn^{2}+n^{\omega}$ & (convex body bound)\tabularnewline
\midrule 
{\footnotesize{}DW} & \textcolor{red}{$(n^{2}+m)n^{2}$} & \textcolor{red}{\small{}$\min(mn^{\omega}+m^{2}n^{2},n^{2\omega}+mn^{2(\omega-1)})$} & \textcolor{red}{this paper}\tabularnewline
\midrule 
{\footnotesize{}DW (Vaidya)} & \textcolor{red}{$(n+\sqrt{m})n^{3}$} & \textcolor{red}{$mn^{2(\omega-1)}$} & \textcolor{red}{this paper}\tabularnewline
\midrule 
{\footnotesize{}DW (Lewis)} & \textcolor{red}{$n^{5}$} & \textcolor{red}{$mn^{2(\omega-1)}$} & \textcolor{red}{this paper}\tabularnewline
\bottomrule
\end{tabular}
\par\end{centering}
\caption{\label{tab:complexity}The complexity of uniformly sampling a polytope
in $\protect\Rn$ / PSD cone in $\protect\S^{n}$ with $m$ affine
constraints. The cost in the parenthesis is the number of iterations
needed for isotropic rounding (\cite{jia2021reducing}). The exponent
$\omega<2.373$ indicates the matrix multiplication constant.}
\end{table}

In Section~\ref{sec:mixingDikin}, we provide a general analysis
of the mixing time of the $\dw$ under milder assumptions than those
in \cite{laddha2020strong}. While the latter paper provides a general
guarantee for strongly self-concordant metrics, it relies on a nice
convexity property which is challenging to verify in the PSD setting.
To address this, we introduce a weaker condition. Then we discuss
how our weaker assumptions enable the analysis of the $\dw$ with
a combination of different metrics. This refined analysis turns out
to be crucial to improve over the bounds of the $\bw$. 

\begin{restatable}{thmre}{thmGeneralMixing} \label{thm:generalMixing} 

For a convex body $K\subset\Rn$, let $\pi_{0}$ be an initial distribution,
$\pi$ the uniform distribution over $K$, and $\Lambda=\sup_{S\subset K}\frac{\pi_{0}(S)}{\pi(S)}$
the warmness of the initial distribution $\pi_{0}$. Let $\pi_{T}$
be the distribution obtained after $T$ steps of the $\dw$ starting
from $x_{0}\sim\pi_{0}$ with a $\onu$-symmetric, lower trace self-concordant
and strongly self-concordant matrix function $g$. Then, for any $\veps>0$,
the total variation distance between $\pi_{T}$ and $\pi$ satisfies
$\dtv(\pi_{T},\pi)\leq\veps$ for $T=O\Par{n\onu\log\frac{\Lambda}{\veps}}$.

\end{restatable}

We replace the requirement in \cite{laddha2020strong} that the local
metric $g$ has convex log-determinant by lower trace self-concordance,
which is considerably weaker. The importance of this refinement will
become clear presently. 

In Section~\ref{sec:basic-psdSampling}, we introduce a metric that
satisfies all the assumptions in Theorem~\ref{thm:generalMixing}.
The resulting $\dw$ mixes in $\otilde{(n^{2}+m)n^{2}}$ iterations
for uniformly sampling the PSD cone with $m$ linear constraints.
We also provide an efficient per-step implementation. For a matrix
$X\in\Rnn$, let $\vec(X)\in\R^{n^{2}}$ denote the vector obtained
by stacking columns of $X$ vertically. Additionally, we define $A\in\R^{m\times n^{2}}$,
$S_{X}\in\R^{m\times m}$, and $A_{X}\in\R^{m\times n^{2}}$ by 
\[
A:=\left[\begin{array}{ccc}
\vec(A_{1}) & \cdots & \vec(A_{m})\end{array}\right]^{\top},\,S_{X}:=\Diag\Par{b_{i}-\inner{A_{i},X}},\,A_{X}:=S_{X}^{-1}A.
\]
The metric below comes from the Hessian of the following function:
\[
-2n\log\det X-2\sum_{i=1}^{m}\Par{b_{i}-\inner{A_{i},X}}.
\]
Here the first term, the log-determinant, serves as a barrier for
the PSD cone while the second term is the standard logarithmic barrier
for linear constraints.

% Figure environment removed
\begin{restatable}{thmre}{thmBasicPSD} \label{thm:basicPSD} 

Let $K$ be the constrained PSD cone in (\ref{eq:PSDcone}). Let $g$
be the local metric such that at each $X\in K$, for symmetric matrices
$H_{1},H_{2}$,
\[
g_{X}(H_{1},H_{2})=2n\tr\Par{X^{-1}H_{1}X^{-1}H_{2}}+2\vec(H_{1})^{\top}A_{X}^{\top}A_{X}\vec(H_{2}).
\]
Then the $\dw$ with the local metric $g$ mixes in $\otilde{(n^{2}+m)n^{2}}$
steps, where each step runs in $O(\min(mn^{\omega}+m^{2}n^{2},n^{2\omega}+mn^{2(\omega-1)}))$
time\footnote{Here $\omega<2.373$ is the current matrix multiplication complexity
exponent (\cite{le2014powers}).}.

\end{restatable}

When $m=O(n^{2})$, this improves substantially on existing bounds
(see Table~\ref{tab:complexity}). In particular, for an interesting
regime of $m=O(1)$, the $\dw$ is faster than the $\bw$ by the order
of $n^{2}$.

To improve the mixing time when the number of constraints $m$ is
large, in Section~\ref{sec:hybrid-psdSampling} we use a metric inspired
by a self-concordant barrier due to \cite{vaidya1996new}, which was
used to improve the complexity of the interior-point method for optimization.
The resulting Vaidya metric takes advantage of the \emph{leverage
scores} $\sigma(A_{X})$ of $A_{X}$, the diagonal entries of the
orthogonal projection matrix $P_{X}=A_{X}(A_{X}^{\top}A_{X})^{-1}A_{X}\in\R^{m\times m}$,
i.e., $\sigma(A_{X})_{i}:=\Par{P_{X}}_{i}$ for $i=1,\dots,m$. We
let $\Sigma_{X}$ be the diagonal matrix with $(\Sigma_{X})_{i}=\sigma(A_{X})_{i}$
for $i=1,\dots,m$. The resulting $\dw$ with this metric has $\sqrt{m}$-dependence. 

\begin{restatable}{thmre}{thmHybridPSD} \label{thm:hybridPSD}

Let $K$ be the constrained PSD cone in (\ref{eq:PSDcone}). Let $g$
be the local metric such that at each $X\in K$, for symmetric matrices
$H_{1},H_{2}$,
\[
g_{X}(H_{1},H_{2})=2n\tr\Par{X^{-1}H_{1}X^{-1}H_{2}}+44\sqrt{\frac{m}{n}}\vec(H_{1})^{\top}A_{X}^{\top}\Par{\Sigma_{X}+\frac{n}{m}I_{m}}A_{X}\vec(H_{2}).
\]
Then the $\dw$ with the local metric $g$ mixes in $\otilde{\Par{n+\sqrt{m}}n^{3}}$
steps, with each step running in $\otilde{mn^{2(\omega-1)}}$ amortized
time.

\end{restatable}

% Figure environment removed

See Figure~\ref{fig:betterBarrier}. Unlike the previous theorem,
this metric is not a Hessian of some barrier due to the term $A_{X}^{\top}\Sigma_{X}A_{X}$.
However, this term is close to the Hessian of the \emph{volumetric
barrier} of linear constraints (see Appendix~\ref{app:subsec:volBarrier}).
When $m<n(n+1)/2$, the leverage scores can be generalized as $\sigma(A_{X})=\diag\Par{A_{X}(A_{X}^{\top}A_{X})^{\dagger}A_{X}}$
by replacing the inverse with the Moore-Penrose inverse. If $A$ is
full-rank (i.e., linearly independent rows), then $A_{X}(A_{X}^{\top}A_{X})^{\dagger}A_{X}=A_{X}A_{X}^{\dagger}=I_{m}$,
and thus the second term in $g_{X}$ above becomes the Hessian of
the logarithmic barrier as in Theorem~\ref{thm:basicPSD}.

For very large $m$, this improved dependence is still polynomial
in $m$, and it is natural to ask if the dependence on $m$ can be
removed or made polylogarithmic. In Section~\ref{sec:LS-psdSampling},
we show that the \emph{Lewis weights} of $A_{X}$ enable the $\dw$
to achieve mixing without polynomial dependence on $m$ at the cost
of an additional factor of $n$. The $\ell_{p}$-Lewis weight of $A_{X}$
is the vector $w_{X}\in\R^{m}$ satisfying the implicit equation $w_{X}=\sigma\Par{\Diag(w_{X})^{\half-\frac{1}{p}}A_{X}}$.
Note that the leverage scores can be recovered as the $\ell_{2}$-Lewis
weight of $A_{X}$. By setting $p=O(\log m)$, the $\dw$ with a Lewis-weight
metric achieves poly-logarithmic dependence on $m$.

\begin{restatable}{thmre}{thmLSPSD} \label{thm:LSPSD}

Let $K$ be the constrained PSD cone in (\ref{eq:PSDcone}). Let $g$
be the local metric such that at each $X\in K$, for symmetric matrices
$H_{1},H_{2}$, 
\[
g_{X}(H_{1},H_{2})=2n\tr\Par{X^{-1}H_{1}X^{-1}H_{2}}+c_{1}\Par{\log m}^{c_{2}}\sqrt{n}\vec(H_{1})^{\top}A_{X}^{\top}W_{X}A_{X}\vec(H_{2}),
\]
where $W_{X}$ is the diagonalized $\ell_{p}$-Lewis weight of $A_{X}$
with $p=O(\log m)$, and $c_{1},c_{2}>0$ are universal constants.
Then the $\dw$ with local metric $g$ mixes in $\otilde{n^{5}}$
steps, with each step running in $\otilde{mn^{2(\omega-1)}}$ amortized
time.

\end{restatable}

We show that these theoretical guarantees also hold for approximate
Lewis weights. When $m<\frac{n(n+1)}{2}$ with full-rank $A$, this
metric is also equivalent to the metric in Theorem~\ref{thm:basicPSD}.
Taken together, our results show that the $\dw$ with appropriate
is significantly faster than the state-of-the-art (the $\bw$) for
every value of $m$, the number of affine constraints. 

\subsection{Background and related work}

Uniformly sampling convex bodies is a special case of \emph{logconcave
sampling}: sample from a distribution $\pi$ with density proportional
to $e^{-V(x)}$ for a convex function $V$ on $\Rn$. This problem
has spawned a long line of research in several communities, as it
captures various important distributions, including uniform distributions
over convex bodies and Gaussians.

A large body of recent work in machine learning and statistics makes
the assumption of $0\prec\alpha I\preceq\hess V(x)\preceq\beta I$
for $x\in\Rn$ (i.e., $\alpha$-strong convexity and $\beta$-smoothness
of the potential $V$), where the strong-convexity assumption is sometimes
relaxed to isoperimetry assumptions such as log-Sobolev inequalities
(LSI), Poincar inequality (PI), and Cheeger isoperimetry.  The guarantees
provided on the mixing time of samplers under this assumption have
polynomial dependence on the condition number defined as $\beta/\alpha$
(or $\alpha$ is replaced by the isoperimetric constant). These guarantees
do not apply to constrained sampling. For example, in uniform sampling,
the simplest constrained sampling problem, $V$ is set to be a constant
within the convex body and infinity outside the body, which leads
to discontinuity of $V$ and $\beta=\infty$. The sudden change of
$V$ around the boundary requires special consideration, such as small
step size, use of a Metropolis filter, projection, etc., making it
a more challenging problem.

\paragraph{Uniform sampling.}

Uniform sampling can be done by the $\bw$ (\cite{lovasz1993random,kannan1997random})
and \textsf{Hit-and-Run} (\cite{smith1984efficient}), which require
access to a function proportional to the density. When a convex body
$K\subset\Rn$ satisfies $B_{r}(x_{0})\subset K\subset B_{R}(x_{0})$
for some $x_{0}$, the $\bw$ mixes in $\otilde{n^{2}\Par{R/r}^{2}}$
steps from warm start (\cite{kannan1997random}) and $\textsf{Hit-and-Run}$
mixes in $\otilde{n^{2}\Par{R/r}^{2}}$ steps from any start\footnote{In our paper, \emph{warm start} means polynomial dependence on the
warmness parameter $M$, while \emph{any start} means poly-logarithmic
dependency on $M$. We assume any start unless specified otherwise.} (\cite{lovasz1999hit,lovasz2006hit}). \cite{lovasz2007geometry}
further extended these results to general logconcave distributions.
These algorithms need to use a ``step size'' of $O(1/\sqrt{n})$,
and their mixing is affected by the skewed geometry of the convex
body (i.e., when $R/r\gg1$). The latter can be addressed by first
\emph{rounding} the body, after which the $\bw$ and the $\textsf{Hit-and-Run}$
mix in $\otilde{n^{2}}$ steps from warm start, due to recent progress
on the KLS constant by \cite{chen2021almost,klartag2023logarithmic}
and stochastic localization by \cite{chen2022hit}. The fastest rounding
algorithm by \cite{jia2021reducing} requires $\otilde{n^{3}}$ queries
to a membership oracle, using the $\bw$.


\paragraph{Sampling with local geometry.}

The $\bw$ uses the same radius ball for every point in the convex
body. One might want to use a different radius depending on the distance
to the boundary. This by itself does not work as it simply makes the
current point converge to the boundary. However, replacing balls with
ellipsoids whose shape changes based on the proximity to the boundary
does work! Several sampling algorithms are motivated by the use of
local metrics: the $\dw$ (\cite{kannan2012random}), \textsf{Riemannian Hamiltonian Monte Carlo}
(RHMC), \textsf{Riemannian Langevin algorithm} (\cite{girolami2011riemann}),
etc.

Which local metrics would be suitable candidates? It turns out that
a suitable metric can be derived from self-concordant barriers, a
concept dating back to the development of the interior-point method
in convex-optimization literature (\cite{nesterov1994interior}).
It is well-known that any convex body admits an $n$-self-concordant
barrier such as universal barrier (\cite{nesterov1994interior,lee2021universal})
and entropic barrier (\cite{bubeck2014entropic,chewi2021entropic}),
but these are computationally expensive. Moreover, as noted in \cite{laddha2020strong},
the symmetry parameter of these general barriers is $\Omega(n^{2})$
for $n$-dimensional bodies (even for second-order cones), and so
the resulting complexity for the $\dw$ on (\ref{eq:PSDcone}) is
$\Omega(n^{2}\cdot n^{4})=\Omega(n^{6})$. Thus, there is a need to
find barriers that are more closely aligned with the structure of
sets we wish to sample. 

\paragraph{Polytope sampling.}

Samplers such as the $\bw$ and \textsf{Hit-and-Run} can be used
to sample polytopes, but they do not really use any special properties
of polytopes. In contrast, non-Euclidean samplers based on barriers
can leverage the properties of polytopes, making polytope-sampling
a fertile theoretical and practical ground where rich theory for linear
programming and sampling intersect. 

For polytopes with $m$ linear constraints in $n$-dimension ($m>n$),
the first theoretical result via self-concordant barriers dates back
to \cite{kannan2012random} who proposed the $\dw$ with the $m$-self-concordant
logarithmic barrier and established the mixing rate of $\otilde{mn}$
for uniform sampling. \cite{chen2018fast} revisited the idea of \cite{vaidya1996new}
using the $O(\sqrt{mn})$-self-concordant hybrid barrier, which is
a hybrid of the volumetric barrier and the log barrier and leads to
a faster interior-point method. They presented the $\dw$ with the
hybrid barrier giving an $\otilde{\sqrt{m}n^{3/2}}$-mixing guarantee.
Lastly, \cite{laddha2020strong} proposed the $\dw$ with a variant
of the $O^{*}(n)$-self-concordant LS barrier based on Lewis weights,
developed by \cite{lee2019solving}, and showed a mixing rate of $\otilde{n^{2}}$.

While a next point proposed by all these $\dw$ lies on a straight
line, the \textsf{Geodesic walk} and RHMC use curves (geodesics and
Hamiltonian-preserving curves respectively). \cite{lee2017geodesic}
and \cite{lee2018convergence} showed that for uniform sampling, the
\textsf{Geodesic walk} and RHMC with the log barrier mix in $\otilde{mn^{\frac{3}{4}}}$
and $\otilde{mn^{\frac{2}{3}}}$ steps respectively. \cite{kook2022condition}
extended theoretical analysis of RHMC to truncated exponential distributions
and showed that discretization of Hamilton's equations by practical
numerical integrators maintains a fast mixing rate. \cite{gatmiry2023sampling}
showed that just as the $\dw$ enjoys faster mixing via a barrier
with a better self-concordance parameter, RHMC with a hybrid barrier
consisting of the Lewis-weight and log barrier mixes in $\otilde{m^{\frac{1}{3}}n^{\frac{4}{3}}}$
steps. Their proof is based on developing suitable properties and
algorithmic bounds for Riemannian manifolds.

Extending these methods to PSD sampling, to potentially improve the
complexity of the problem significantly beyond the general bounds
that follow from convex-body sampling, has been an open research direction,
likely to be mathematically and algorithmically interesting in light
of the analogy with convex optimization. 


\subsection{Overview of ideas and contributions}

We refer readers to Section~\ref{subsec:prelim} for definitions
and notation.

\paragraph{Faster uniform sampling from PSD cone.}

Our work presents a polynomial-time sampling algorithm for general
PSD cones. Compared with the $\bw$, the known general method for
uniform sampling from convex bodies, the $\dw$ has two notable advantages:
\textbf{(1)} \emph{affine invariance} that eliminates the need for
preprocessing, and \textbf{(2)} \emph{polynomial-time mixing from
any starting point} within the PSD cone, due to poly-log dependence
on the warmness parameter. In contrast, both the $\bw$ and \textsf{Hit-and-Run}
require an expensive rounding phase. The $\bw$ also needs an $O(1)$-warm
start (i.e., $\Lambda=O(1)$) to achieve polynomial-time mixing. Our
improvements are attained by leveraging the geometric attributes of
the PSD cone, which admits an efficiently computable self-concordant
barrier --- the log-determinant barrier $\phi_{1}$ for semi-definiteness
together with the logarithmic barrier $\phi_{2}$ for linear constraints.

The sum of two metrics, $g=\hess\phi_{1}+\hess\phi_{2}$, serves as
a natural local metric for the $\dw$ to handle two types of constraints.
However, adapting the previous framework from \cite{laddha2020strong}
for the PSD setting presents challenges. It is unclear if $g$ is
strongly self-concordant, and verifying the convexity of $\log\det g$
is difficult. To address this, in Section~\ref{sec:prop-SC} we first
establish additive properties of strong self-concordance (Lemma~\ref{lem:sumStrongSC})
and provide simple sufficient conditions for lower trace self-concordance
of the sum of local metrics (Lemma~\ref{lem:additiveCondition}).
Second, we utilize matrix calculus and properties of the Kronecker
product to show in Section~\ref{subsec:scBasicMetric} that the log-determinant
barrier $\phi_{1}$ must be scaled by $n$ for strong self-concordance,
along with the optimality of  this scaling factor (Theorem~\ref{thm:logdet-scaling}).
Then we estimate the symmetry parameter of $\phi_{1}$ (Lemma~\ref{lem:logdet-symm})
and demonstrate its fourth-order convexity (i.e., $D^{4}\phi_{1}(X)[H,H]\succeq0$
for any $H\in\S^{n}$ via (\ref{eq:D4ph1})). In Lemma~\ref{lem:metricBasic},
we obtain strong and lower trace self-concordance and bound the symmetry
parameter of $g$. Applying our framework developed in Section~\ref{sec:mixingDikin}
to $g$, we obtain the mixing time $\otilde{n^{2}(n^{2}+m)}$ of the
$\dw$ as in Theorem~\ref{thm:basicPSD}.

\paragraph{Analysis of the $\protect\dw$ under trace self-concordance.}

\cite{laddha2020strong} analyzed the $\dw$ in $\Rn$ assuming the
following properties of the metric: strong self-concordance, $\onu$-symmetry,
and convexity of the log-determinant. In the case of the constrained
PSD cone, we intersect the PSD cone with linear constraints, and the
natural efficient metrics arise by adding metrics for the two different
types of constraints. While each one by itself could potentially satisfy
all the above (as verified for the log barrier in earlier work), it
does not follow that the sum of metrics satisfies the last condition.
Self-concordance is additive, and symmetry can be maintained for the
intersection, but verifying the last condition --- convexity of log
determinant --- proves challenging. We replace this restrictive condition
with \emph{lower trace self-concordance}, which is considerably weaker
and easier to check. In Section~\ref{sec:mixingDikin}, we demonstrate
that under this weaker condition, we can achieve the same mixing time
of $O\Par{n\onu\log\frac{\Lambda}{\veps}}$ as stated in Theorem~\ref{thm:generalMixing}.

Its overall proof of this theorem is analogous to \cite{laddha2020strong};
due to \cite{lovasz1993random}, it suffices to establish a lower
bound on the conductance of the $\dw$, which requires \textbf{(1)}
the one-step coupling of the $\dw$ (Lemma~\ref{lem:one-step}) and
\textbf{(2)} the isoperimetry for the Dikin distance (Lemma~\ref{lem:isoperimetry}).
For \textbf{(1)}, when bounding the overlap of the transition kernels
started at two close points, there are two sources of deviation ---
overlap of two Dikin ellipsoids and rejection probability. The former
leverages strong self-concordance and a geometric analysis of overlapping
Dikin ellipsoids. The latter comes down to establishing a constant
lower bound on $\vphi(z)-\vphi(x)$ with $\vphi(\cdot)=\log\det g(\cdot)$
for a proposal $z$ and current point $x$. After using Taylor's expansion
of $\vphi(z)$ at $z=x$, we bound the first-order term, $\tr\Par{g(x)^{-1}Dg(x)[z-x]}$,
via strong self-concordance and the concentration of measure. For
the second-order term, $\tr\Par{g^{-1}D^{2}g[h,h]}-\norm{g^{-\half}Dgg^{-\half}}_{F}^{2}$,
lower trace and strong self-concordance play crucial roles. For \textbf{(2)},
following \cite{laddha2020strong}, we recall the isoperimetric inequality
in terms of the cross-ratio distance $d_{K}$ over a convex body $K$,
which states $\vol(S_{3})\vol(K)\geq d_{K}(S_{1},S_{2})\vol(S_{1})\vol(S_{2})$
for any partition $\{S_{1},S_{2},S_{3}\}$ of $K$. We then use $d_{K}(x,y)\geq\norm{x-y}_{g(x)}/\sqrt{\onu}$
to prove that the isoperimetry of the uniform distribution over $K$
with the Dikin distance $\norm{\cdot}_{g(x)}$ is $\Omega\Par{1/\sqrt{\onu}}$.

\paragraph*{Efficient implementation of each step.}

In Section~\ref{subsec:oracleImplementation}, we propose an efficient
per-step implementation that avoids explicit computation of a local
metric and its inverse of size $\frac{n(n+1)}{2}\times\frac{n(n+1)}{2}$
(Algorithm~\ref{alg:perStep-small-m} and Lemma~\ref{lem:perStep-small-m}).
Each step of the $\dw$ involves \textbf{(1)} uniform sampling from
$Z\sim\dcal_{g}^{r}(X)$ and \textbf{(2)} computing the acceptance
probability, $\min\Par{1,\sqrt{\frac{\det g(Z)}{\det g(X)}}}$. To
achieve efficiency, we first design a subroutine for computing $g(X)^{-1}v$
for a given vector $v$ in $O\Par{mn^{\omega}+m^{2}n^{2}}$ time (Algorithm~\ref{alg:subroutine}
and Proposition~\ref{prop:oracle}). To this end, we find matrices
$B\in\R^{d\times n^{2}}$ and $U\in\R^{d\times m}$ such that $g_{1}=\hess\phi_{1}=BB^{\top}$
and $g_{2}=\hess\phi_{2}=UU^{\top}=\sum_{i=1}^{m}u_{i}u_{i}^{\top}$,
with  $u_{i}$ being the column vectors of $U$. Viewing $g=g_{1}+g_{2}$
as a series of rank-one updates to $g_{1}$ by $u_{i}u_{i}^{\top}$
for $i\in[m]$, we apply the Sherman-Morrison formula to recursively
update $\Par{g_{1}+\sum_{i=1}^{j}u_{i}u_{i}^{\top}}^{-1}v,\Par{g_{1}+\sum_{i=1}^{j}u_{i}u_{i}^{\top}}^{-1}u_{l}$
for $l\in[m]$, starting from $j=1$ to $m$. These updates leverage
the linear transformation defined in Section~\ref{subsec:formalism}
facilitating smooth transitions between $X\in\S^{n}$ and $\vec(X)\in\R^{n(n+1)/2}$.
This approach effectively reduces costly calculations on one side
to simpler ones on another, avoiding the materialization of $g$ and
$g^{-1}$.

With this core subroutine in place, we proceed to implement those
two main steps. For \textbf{(1)}, instead of computing $g(X)^{-\half}v$
with $v\sim\text{Uniform}(B_{r}(0))$, we (i) draw $v\sim\ncal(0,g(X)^{-1})$
without fully computing $g(X)^{-1}$ and (ii) use this to generate
a uniform sample from $\dcal_{g}^{r}(X)$ in $O\Par{mn^{\omega}+m^{2}n^{2}}$
time. For (i), we observe that $g(X)^{-1}\left[\begin{array}{cc}
B & U\end{array}\right]v$ with $v\sim\ncal\Par{0,I_{n^{2}+m}}$ follows $\ncal(0,g(X)^{-1})$.
Utilizing the linear transformation between $X\in\S^{n}$ and $\vec(X)\in\R^{n(n+1)/2}$,
we compute $\left[\begin{array}{cc}
B & U\end{array}\right]v$ efficiently and employ the subroutine to multiply $g(X)^{-1}$ on
the left side. Regarding (ii), we adapt the observation that a uniform
sample from $B_{r}(0)\subset\R^{d}$ can be obtained by $s^{1/d}\zeta/\norm{\zeta}_{2}$
with $s\sim\text{Uniform}([0,r])$ and $\zeta\sim\ncal(0,I_{d})$.
For \textbf{(2)}, we revisit the perspective of viewing $g$ as rank-one
updates to $g_{1}$ by $u_{i}u_{i}^{\top}$, using the matrix determinant
lemma.

\paragraph{Hybrid metrics.}

Despite the challenge of preserving properties like log-determinant
convexity when combining metrics, we demonstrate that the improvement
on $m$-dependency achieved in polytope sampling using better self-concordant
metrics (e.g., log-barriers to Vaidya metrics to Lewis-weight metrics)
can also be realized for PSD sampling, albeit with considerably more
technical challenges. The $\dw$ with the Vaidya metric and Lewis-weight
metric (along with $\hess\phi_{1}$) mixes in $\otilde{(n+\sqrt{m})n^{3}}$
and $\otilde{n^{5}}$ iterations respectively as stated in Theorem~\ref{thm:hybridPSD}
in Section~\ref{sec:hybrid-psdSampling} and Theorem~\ref{thm:LSPSD}
in Section~\ref{sec:LS-psdSampling}, respectively. Notably, by setting
$g_{1}$ to $0$, our framework recovers the mixing time of the \textsf{Vaidya walk}
for polytope sampling in \cite{chen2018fast}, providing a substantially
simpler analysis for this special case. Alongside, by using clean
matrix algebra we identify the appropriate scaling of self-concordant
metrics for linear constraints $Ax\leq b$ with $A\in\R^{m\times n}$
and $b\in\R^{m}$ that ensures strong self-concordance: $\sqrt{m}$
for approximate volumetric metrics, $\sqrt{m/n}$ for Vaidya metrics
(Lemma~\ref{lem:paramsBarrier}), and $(\log m)^{O(1)}$ for Lewis-weight
metrics (Lemma~\ref{lem:LSmetricStrongandSymmetry}). For lower trace
self-concordance, $\sqrt{m/n}$-scaling is sufficient for Vaidya metrics
(Lemma~\ref{lem:hybridLowerSCTrace}), while Lewis-weight metrics
require an additional scaling of $\sqrt{n}$ along with $(\log m)^{O(1)}$
(Lemma~\ref{lem:LSLowerSCTrace}).

In analyzing the Vaidya and Lewis-weight metrics, a main technical
challenge is to find the minimal scaling of these metrics for two
key properties: \textbf{(1)} strong self-concordance and symmetry,
and \textbf{(2)} lower trace self-concordance. For \textbf{(1)}, we
investigate a metric of the form $A_{x}^{\top}D_{x}A_{x}$ for a diagonal
matrix $D_{x}$, which is designed to respect affine constraints.
In Lemma~\ref{lem:helper4Diagonal}, we relate the notions of strong
self-concordance and $\onu$-symmetry to well-studied terms in the
field of optimization, namely $\max_{i}\frac{\sigma\Par{\sqrt{W_{x}}A_{x}}_{i}}{(D_{x})_{ii}}$
and $\norm{DD_{x}[h]}_{D_{x}^{-1}}^{2}$. Then we can refer to existing
bounds on these terms, estimating the smallest possible scaling required
for strong self-concordance and symmetry. For \textbf{(2)}, ensuring
trace self-concordance of $g$ with $g_{2}$ being the Vaidya or Lewis
metrics is challenging, as $D^{2}g_{2}[h,h]\succeq0$ is difficult
to directly verify due to complicated expressions for $D^{2}\Sigma_{x}[h,h]$
and $D^{2}W_{x}[h,h]$. To address this, we aim to show $\tr\Par{(g_{1}+g_{2})^{-1}D^{2}g_{2}[h,h]}\geq-\alpha\norm h_{g}^{2}$
for a small constant $\alpha>0$ and then refer to Lemma~\ref{lem:additiveCondition}
for trace self-concordance of $g$. In the case of the Vaidya metric,
we compute higher-order derivatives of leverage scores and other pertinent
matrices in Proposition~\ref{prop:calculusLeverage}, based on properties
of the Hadamard product, and use these results to derive lower trace
self-concordance of $g$. For the Lewis-weight metric, the situation
is more complicated due to numerous terms appearing in $D^{2}W_{x}[h,h]$.
In order to avoid dealing with each of the terms, we employ Calabi-type
estimates on $W_{x}$ and other relevant matrices. This strategy significantly
simplifies the computation but comes at the cost of an additional
scaling of $\sqrt{n}$, which as far as we can tell might be unavoidable.


\subsection{Preliminaries and notation \label{subsec:prelim}}

\paragraph{Basics.}

For $n\in\mathbb{N}$, let $[n]:=\{1,\cdots,n\}$. The $\widetilde{O}$
complexity notation suppresses poly-logarithmic factors and dependence
on error parameters. For $v\in\Rn$, the Euclidean norm (or $\ell_{2}$-norm)
is denoted by $\norm v_{2}\defeq\sqrt{\sum_{i\in[n]}v_{i}^{2}}$,
and the infinity norm is denoted by $\norm v_{\infty}\defeq\max_{i\in[n]}\Abs{v_{i}}$.
A Gaussian distribution with mean $\mu\in\Rn$ and covariance $\Sigma\in\Rnn$
is denoted by $\ncal(\mu,\Sigma)$.

\paragraph{Matrices.}

We use $\S^{n}$ to denote the set of symmetric matrices of size $n\times n$.
For $X\in\S^{n}$, we call it \emph{positive semidefinite} (PSD) (resp.
\emph{positive definite} (PD)) if $h^{\top}Xh\geq0$ ($>0)$ for any
$h\in\R^{n}$. We use $\psd$ to denote the set of positive definite
matrices of size $n\times n$. Note that their effective dimension
is $d:=n(n+1)/2$ due to symmetry. For a positive (semi) definite
matrix $X$, its \emph{square root} is denoted as $X^{\half}$, and
is the unique positive (semi) definite matrix satisfying $X^{\half}X^{\half}=X$.
For $A,B\in\S^{n}$, we use $A\preceq B$ ($A\prec B$) to indicate
that $B-A$ is PSD (PD). For a matrix $A\in\R^{n\times n}$, its \emph{trace}
is denoted by $\tr\Par A=\sum_{i=1}^{n}A_{ii}$. The \emph{operator
norm} and \emph{Frobenius norm} are denoted by $\norm A_{2}\defeq\sup_{x\in\Rn}\norm{Ax}_{2}/\norm x_{2}$
and $\norm A_{F}\defeq\Par{\sum_{i,j=1}^{n}A_{ij}^{2}}^{\half}=\sqrt{\tr\Par{A^{\top}A}}$,
respectively.

\paragraph{Basic operations.}

For $X\in\S^{n}$, its \emph{vectorization} $\vec{(}X)\in\R^{n^{2}}$
is obtained by stacking each column of $X$ vertically. Its symmetric
vectorization $\svec(X)\in\R^{d}$ is obtained by stacking the lower
triangular part in vertical direction. For a matrix $A\in\R^{n\times n}$
and vector $x\in\Rn$, we use $\diag(A)$ to denote the vector in
$\Rn$ with $(\diag(A))_{i}=A_{ii}$ for $i\in[n]$, $\Diag(A)$ to
denote the diagonal matrix with $(\Diag(A))_{ii}=A_{ii}$ for $i\in[n]$
and $\Diag(x)$ to denote the diagonal matrix in $\R^{n\times n}$
with $(\Diag(x))_{ii}=x_{i}$ for $i\in[n]$.

\paragraph{Matrix operations.}

For matrices $A,B\in\R^{n\times n}$, their inner product is defined
as the inner product of $\vec(A)$ and $\vec(B)$, denoted by $\inner{A,B}=\tr\Par{A^{\top}B}$.
Their \emph{Kronecker product} $A\kro B$ is the matrix of size $n^{2}\times n^{2}$
defined by 
\[
A\otimes B=\left[\begin{array}{ccc}
A_{11}B & \cdots & A_{1n}B\\
\vdots &  & \vdots\\
A_{n1}B & \cdots & A_{nn}B
\end{array}\right],
\]
where $A_{ij}B$ is the block matrix of size $n\times n$ obtained
by multiplying each entry of $B$ by the scalar $A_{ij}$. Their \emph{Hadamard
product} $A\circ B$ is the matrix of size $n\times n$ defined by
$(A\hada B)_{ij}=A_{ij}B_{ij}$ (i.e., obtained by element-wise multiplication).

\paragraph{Projection matrix, Leverage score and Lewis weights.}

For a full-rank matrix $A\in\R^{m\times n}$ with $m\geq n$, we recall
that $P(A):=A(A^{\top}A)^{-1}A^{\top}$ is the orthogonal projection
matrix onto the column space of $A$. The leverage scores of $A$
is denoted by $\sigma(A):=\diag(P(A))\in\R^{m}$. We let $\Sigma(A):=\Diag(\sigma(A))=\Diag(P(A))$
and $P^{(2)}(A):=P(A)\circ P(A)$. The $\ell_{p}$-Lewis weights of
$A$ is denoted by $w(A)$, the solution $w$ to the equation $w(A)=\diag\Par{W^{\half-\frac{1}{p}}A\Par{A^{\top}W^{1-\frac{2}{p}}A}^{-1}A^{\top}W^{\half-\frac{1}{p}}}\in\R^{m}$
for $W:=\Diag(w)$. When $m<n$ or $A$ is not full rank, both leverage
scores and Lewis weights can be generalized via the Moore-Penrose
inverse in place of the inverse in the definitions.

\paragraph{Derivatives.}

For a function $f:\R^{n}\to\R$, let $\grad f(x)\in\R^{n}$ denote
the gradient of $f$ at $x$ (i.e., $\Par{\grad f(x)}_{i}:=\frac{\del f}{\del x_{i}}(x)$)
and $\hess f(x)\in\R^{n\times n}$ denote the Hessian of $f$ at $x$
(i.e., $\Par{\hess f(x)}_{ij}:=\frac{\del^{2}f}{\del x_{i}\del x_{j}}(x)$).
For a matrix function $g:\R^{n}\to\R^{n\times n}$ in $x$, we use
$Dg$ and $D^{2}g$ to denote the third-order and fourth-order tensor
defined by $(Dg(x))_{ijk}=\frac{\del\Par{g(x)}_{ij}}{\del x_{k}}$
and $(D^{2}g(x))_{ijkl}=\frac{\del^{2}\Par{g(x)}_{ij}}{\del x_{k}\del x_{l}}$.
We use the following shorthand notation: $g_{x,h}':=Dg(x)[h]$ and
$g_{x,h}'':=D^{2}g(x)[h,h]$. We let $D^{i}g(x)[h_{1},\cdots,h_{i}]$
denote the $i^{\text{th}}$ directional derivative of $g$ at $x$
in directions $h_{1},\cdots,h_{i}\in\Rn$, i.e.,
\[
D^{i}g(x)[h_{1},\cdots,h_{i}]=\frac{d^{i}}{dt_{1}\cdots dt_{i}}g\Par{x+\sum_{j=1}^{i}t_{j}h_{j}}\bigg|_{t_{1},\cdots,t_{i}=0}.
\]


\paragraph*{Local norm.}

At each point $x$ in a set $K\subset\Rn$, a \emph{local metric}
$g$, denoted as $g_{x}$ or $g(x)$, is a positive-definite inner
product $g_{x}:\R^{n}\times\R^{n}\to\R$, which naturally induces
the local norm as $\norm v_{g(x)}:=\sqrt{g_{x}(v,v)}$. We use $\norm v_{x}$
to refer to $\norm v_{g(x)}$ when the context is clear. When an ambient
space has an orthogonal basis as in our setting (e.g., $\{e_{1},\dots,e_{n}\}$),
the local metric $g_{x}$ can be represented as a positive-definite
matrix of size $n\times n$. With this perspective, the inner product
can be written as $g_{x}(v,w)=v^{\top}g(x)w$. Going forward, we use
$g_{x}=g(x)$ to denote a local metric (or positive definite matrix
of size $\dim(x)\times\dim(x)$) at each point $x\in K$. The local
metric $g$ is assumed to be at least twice differentiable.

\paragraph{Markov chains.}

Many sampling algorithms are based on \emph{Markov chains}. A \emph{transition
kernel} $P:\Rn\times\bcal(\Rn)\to\R_{\geq0}$ (or \emph{one-step distribution})
for the Borel $\sigma$-algebra $\bcal(\Rn)$ quantifies the probability
of the Markov chains transitioning from one point to another measurable
set. The next-step distribution is defined by $P_{x}(A):=P(x,A)$,
which is the probability of a step from $x$ landing in the set $A$.
The transition kernel characterizes the Markov chain in the sense
that if a current distribution is $\mu$, then the distribution after
$n$ steps can be expressed as $\mu P^{(n)}$, where $\mu P^{(i)}:=\int_{\Rn}P(x,\cdot)\cdot(\mu P^{(i-1)})(dx)$
is defined recursively for $i=1,...,n$ with the convention $\mu P^{(0)}(x)=\mu(x)$.
We call $\pi$ a \emph{stationary distribution} of the Markov chain
if $\pi=\pi P$. If the stationary distribution further satisfies
$\int_{A}P(x,B)\pi(dx)=\int_{B}P(x,A)\pi(dx)$ for any two measurable
subsets $A,B$, then the Markov chain is said to be \emph{reversible}
with respect to $\pi$.

It is expected that the Markov chain approaches the stationary distribution.
We measure this with the \emph{total variation distance} (TV-distance):
for two distributions $\mu$ and $\pi$ on $\Rn$, the TV-distance
is defined as $\dtv(\mu,\pi)=\sup_{A\in\bcal(\Rn)}\Abs{\mu(A)-\pi(A)}=\half\int_{\Rn}\Abs{\frac{d\mu}{dx}-\frac{d\pi}{dx}}dx$,
where the last equality holds when the two distributions admit densities
with respect to the Lebesgue measure on $\Rn$. Moreover, the rate
of convergence can be quantified by the \emph{mixing time}: for an
error parameter $\veps\in(0,1)$ and an initial distribution $\pi_{0}$,
the mixing time is defined as the smallest $n\in\N$ such that $\dtv(\pi_{0}P^{(n)},\pi)\leq\veps$.
In this paper, we consider a \emph{lazy} Markov chain, which does
not move with probability $1/2$ at each step, in order to avoid a
uniqueness issue of a stationary distribution. Note that this change
worsens the mixing time by at most a factor of $2$. One of the standard
tools to control progress made by each iterate is the \emph{conductance}
$\Phi$ of the Markov chain with its stationary distribution $\pi$,
defined by $\Phi\defeq\inf_{\text{measurable }S}\frac{\int_{S}P(x,S^{c})\pi(dx)}{\min\Par{\pi(S),\pi(S^{c})}}$.
Another crucial factor affecting the convergence rate is geometry
of the stationary distribution $\pi$, as measured by \emph{Cheeger
isoperimetry} $\psi_{\pi}\defeq\inf_{S}\frac{\lim_{\delta\to0^{+}}\pi\Par{\{x:\,0<d(S,x)\leq\delta\}}/\delta}{\min(\pi(S),\pi(S^{c}))}$
where the infimum is over measurable subsets $S$ and $d(S,x)$ is
some distance between $x$ and the set $S$.


\section{Properties of self-concordance\label{sec:prop-SC}}

Self-concordance is a central notion in the theory of interior-point
methods for optimization (we refer interested readers to \cite{nesterov1994interior,nesterov2003introductory}).
In this section, we first recall basic properties of self-concordance
and then investigate those of strong self-concordance and lower trace
self-concordance, which are crucial to our analysis. In particular,
we establish sufficient conditions for local metrics under which the
sum of the metrics satisfies the assumptions in our general framework
for the mixing rate analysis of the $\dw$ (Theorem~\ref{thm:generalMixing}).

\paragraph{Self-concordance.}

We recall basic properties of self-concordance. Let $f_{1}$ and $f_{2}$
be self-concordant functions on a convex set $K\subset\Rn$. Let $\alpha>0$
be a scalar.
\begin{lem}
[\cite{nesterov2003introductory}] $ $
\begin{itemize}
\item (Theorem 4.1.1) $f_{1}+f_{2}$ is self-concordant on $K$.
\item (Corollary 4.1.2) $g=\hess(\alpha f_{1})$ satisfies $\norm{g(x)^{-\half}Dg(x)[h]g(x)^{-\half}}_{2}\leq\frac{2}{\sqrt{\alpha}}\norm h_{g(x)}$
for $h\in\Rn$ and $x\in K$.
\end{itemize}
\end{lem}

The following lemma ensures that the $\dw$ stays inside the convex
body.
\begin{lem}
[\cite{nesterov2003introductory}, Theorem 4.1.5] \label{lem:dikin-in-body}
$\dcal_{g}^{1}(x)\subset K$ for a convex body $K$ and $g=\hess f_{1}$.
\end{lem}

The following lemma states that self-concordant metrics are similar
for nearby points.
\begin{lem}
[\cite{nesterov2003introductory}, Theorem 4.1.6] \label{lem:scCloseness}
Given any self-concordant matrix function $g$ on $K\subset\Rn$ and
$x,y\in K$ with $\norm{x-y}_{g(x)}<1$, we have 
\[
\Par{1-\norm{x-y}_{g(x)}}^{2}g(x)\preceq g(y)\preceq\frac{1}{\Par{1-\norm{x-y}_{g(x)}}^{2}}g(x).
\]
\end{lem}


\paragraph{Strong self-concordance.}

We first show that strong self-concordance is also additive with a
scaling factor. We remark that the factor of $2$ below might be redundant.
\begin{lem}
\label{lem:sumStrongSC} If $g_{1}$ and $g_{2}$ are strongly self-concordant
matrix functions on $K_{1}$ and $K_{2}$ respectively, then $2(g_{1}+g_{2})$
is strongly self-concordant on $K_{1}\cap K_{2}$. 
\end{lem}

\begin{proof}
For fixed $x\in K_{1}\cap K_{2}$ and $h\in\Rn$, let $Dg_{i}:=Dg_{i}(x)[h]$
for $i=1,2$. Note that
\begin{align*}
 & \norm{(g_{1}+g_{2})^{-\half}D(g_{1}+g_{2})(g_{1}+g_{2})^{-\half}}_{F}\\
 & \leq\norm{(g_{1}+g_{2})^{-\half}Dg_{1}(g_{1}+g_{2})^{-\half}}_{F}+\norm{(g_{1}+g_{2})^{-\half}Dg_{2}(g_{1}+g_{2})^{-\half}}_{F}\\
 & =\sqrt{\tr\Par{(g_{1}+g_{2})^{-1}Dg_{1}(g_{1}+g_{2})^{-1}Dg_{1}}}+\sqrt{\tr\Par{(g_{1}+g_{2})^{-1}Dg_{2}(g_{1}+g_{2})^{-1}Dg_{2}}}\\
 & =\sqrt{\tr\bigg(\bigg(\underbrace{I+g_{1}^{-\half}g_{2}g_{1}^{-\half}}_{=:E_{1}}\bigg)^{-1}\underbrace{g_{1}^{-\half}Dg_{1}g_{1}^{-\half}}_{=:T_{1}}\Par{I+g_{1}^{-\half}g_{2}g_{1}^{-\half}}^{-1}g_{1}^{-\half}Dg_{1}g_{1}^{-\half}\bigg)}\\
 & \qquad+\sqrt{\tr\bigg(\bigg(\underbrace{I+g_{2}^{-\half}g_{1}g_{2}^{-\half}}_{=:E_{2}}\bigg)^{-1}\underbrace{g_{2}^{-\half}Dg_{2}g_{2}^{-\half}}_{=:T_{2}}\Par{I+g_{2}^{-\half}g_{1}g_{2}^{-\half}}^{-1}g_{2}^{-\half}Dg_{2}g_{2}^{-\half}\bigg)}\\
 & =\sqrt{\tr\Par{E_{1}^{-1}T_{1}E_{1}^{-1}T_{1}}}+\sqrt{\tr\Par{E_{2}^{-1}T_{2}E_{2}^{-1}T_{2}}}\\
 & \leq\sqrt{\tr\Par{T_{1}E_{1}^{-2}T_{1}}}+\sqrt{\tr\Par{T_{2}E_{2}^{-2}T_{2}}},
\end{align*}
where we used the Cauchy-Schwartz inequality $\tr(A^{2})\leq\tr(A^{\top}A)$
in the last line. Since $I\preceq E_{i}$, it follows that $I\preceq E_{i}^{2}$
and $I\succeq E_{i}^{-2}\succ0$. Therefore,
\begin{align*}
\sqrt{\tr\Par{T_{1}E_{1}^{-2}T_{1}}}+\sqrt{\tr\Par{T_{2}E_{2}^{-2}T_{2}}} & \leq\sqrt{\tr(T_{1}^{2})}+\sqrt{\tr(T_{2}^{2})}\\
 & =\norm{T_{1}}_{F}+\norm{T_{2}}_{F}\\
 & \leq2\norm h_{g_{1}(x)}+2\norm h_{g_{2}(x)}\\
 & \leq2\sqrt{2}\norm h_{(g_{1}+g_{2})(x)}.
\end{align*}
Putting these together, we conclude that
\[
\norm{(g_{1}+g_{2})^{-\half}D(g_{1}+g_{2})(g_{1}+g_{2})^{-\half}}_{F}\leq2\sqrt{2}\norm h_{(g_{1}+g_{2})(x)},
\]
and thus $2(g_{1}+g_{2})$ is strongly self-concordant on $K_{1}\cap K_{2}$.
\end{proof}
The same scalar-scaling law holds for strong self-concordance since
the operator norm and Frobenius norm behave in the same way for scalar
scaling.

Next, we recall the analogue of Lemma~\ref{lem:scCloseness} for
strong self-concordance.
\begin{lem}
[\cite{laddha2020strong}, Lemma 1.2] \label{lem:strongSC-closeness}Given
a strongly self-concordant matrix function $g$ on $K\subset\Rn$,
and any $x,y\in K$ with $\norm{x-y}_{g(x)}<1$, 
\[
\norm{g(x)^{-\half}\Par{g(y)-g(x)}g(x)^{-\half}}_{F}\leq\frac{\norm{x-y}_{g(x)}}{\Par{1-\norm{x-y}_{g(x)}}^{2}}.
\]
\end{lem}


\paragraph{Symmetry.}

Recall that $\onu$-symmetry requires two-sided inclusion: the first
part is $\dcal_{g}^{1}(x)\subset K\cap(2x-K)$, and the second part
is $K\cap(2x-K)\subset\dcal_{g}^{\sqrt{\onu}}(x)$. The first part
immediately follows when a metric is induced by a self-concordant
function.
\begin{lem}
\label{lem:symmetricLeftpart} If $\phi$ is a self-concordant function
on $K$, then $\dcal_{g}^{1}(x)\subset K\cap(2x-K)$ for $g=\hess\phi$
and $x\in K$.
\end{lem}

\begin{proof}
Lemma~\ref{lem:dikin-in-body} ensures that $y\in K$ whenever $y\in\dcal_{g}^{1}(x)$.
Then $2x-y\in\dcal_{g}^{1}(x)$ and thus $2x-y\in K$. It implies
that $y\in2x-K$.
\end{proof}
When a metric is induced by a self-concordant barrier with a barrier
parameter $\nu$, it holds that $\onu=O(\nu^{2})$. This is mentioned
in \cite{laddha2020strong}, we provide the proof for completeness.
\begin{lem}
For a self-concordant barrier $\phi$ with a barrier parameter $\nu$
on $K$ and $g=\hess\phi$, it follows that $\onu=O(\nu^{2})$.
\end{lem}

\begin{proof}
By Theorem 4.2.5 in \cite{nesterov2003introductory}, for any $x,y\in K$
with $\grad\phi(x)\cdot(y-x)\geq0$ it follows that $\norm{y-x}_{g(x)}\leq\nu+2\sqrt{\nu}$.
Now, let $x\in K$ and $y\in K\cap(2x-K)$. The latter implies that
$y-x=x-z$ for some $z\in K$. 

If $\grad\phi(x)\cdot(y-x)\geq0$, then $\norm{y-x}_{g(x)}\leq\nu+2\sqrt{\nu}.$
If $\grad\phi(x)\cdot(y-x)<0$, then $\grad\phi(x)\cdot(z-x)>0$ and
thus $\norm{y-x}_{g(x)}=\norm{z-x}_{g(x)}\leq\nu+2\sqrt{\nu}$. From
these two cases, it holds in general that $\norm{y-x}_{g(x)}\leq\nu+2\sqrt{\nu}$
and thus $K\cap(2x-K)\subset\dcal_{g}^{\nu+2\sqrt{\nu}}(x)$. By Lemma
\ref{lem:symmetricLeftpart}, $\dcal_{g}^{1}(x)\subset K\cap(2x-K)$
and thus $\onu=O(\nu^{2})$.
\end{proof}
For affine constraints $Ax\geq b$, the first inclusion above has
a useful equivalent description as follows:
\begin{lem}
\label{lem:symmforPolytope} Let $x\in K=\{Ax>b\}$. It holds that
$y\in K\cap(2x-K)$ if and only if $\norm{A_{x}(y-x)}_{\infty}\leq1$.
\end{lem}

\begin{proof}
For $y\in K$, we have $Ay>b$ and thus $s_{x}=Ax-b>A(x-y)$ (elementwise
inequality). As $s_{x}>0$, we have $A_{x}(x-y)\leq1$. When $y\in(2x-K)$,
we can write $y=2x-z$ for some $z\in K$. Note that
\[
A(x-y)=A(z-x)>b-Ax=-s_{x},
\]
and thus $A_{x}(x-y)\geq-1$. Therefore, $\norm{A_{x}(y-x)}_{\infty}\leq1$.
\end{proof}
Lastly, we can check that if $g$ is $\onu$-symmetric, then $\alpha g$
is $\alpha\onu$-symmetric for $\alpha\ge1$, which follows from observation
of $\dcal_{\alpha g}^{r}(x)=\dcal_{g}^{r/\sqrt{\alpha}}(x)$.
\begin{lem}
\label{lem:symmScaling} For $\alpha\geq1$, if $g$ is $\onu$-symmetric,
then $\alpha g$ is $\alpha\onu$-symmetric.
\end{lem}

\begin{lem}
\label{lem:sumSymmetricSC} If $g_{1}$ and $g_{2}$ are symmetric
self-concordant matrices with parameters $\nu_{1}$ and $\nu_{2}$
on $K_{1}$ and $K_{2}$, then $g_{1}+g_{2}$ is a symmetric self-concordant
matrix with parameter $\nu_{1}+\nu_{2}$ on $K_{1}\cap K_{2}$.
\end{lem}

\begin{proof}
For $g:=g_{1}+g_{2}$, let $y\in\dcal_{g}^{1}(x)$. It implies $y\in\dcal_{g_{1}}^{1}(x)\cap\dcal_{g_{2}}^{1}(x)$
and so $y\in K_{i}\cap(2x-K_{i})$. Due to $\cap_{i}\Par{K_{i}\cap(2x-K_{i})}=K\cap(2x-K)$,
we have $y\in K\cap(2x-K)$ and so $\dcal_{g}^{1}(x)\subset K\cap(2x-K)$.

Now let $y\in K\cap(2x-K)$. It is obvious that $y\in K_{i}\cap(2x-K_{i})$
for $i=1,2$, and thus
\begin{align*}
(y-x)^{\top}g_{1}(x)(y-x) & \leq\nu_{1},\\
(y-x)^{\top}g_{2}(x)(y-x) & \leq\nu_{2}.
\end{align*}
By adding up these two, it follows that $\norm{y-x}_{g(x)}^{2}\leq\nu_{1}+\nu_{2}$.
\end{proof}

\paragraph{Lower trace self-concordance.}

We finish this section by providing a sufficient condition under which
the sum of metrics satisfies lower trace self-concordance.
\begin{lem}
\label{lem:additiveCondition}Let $g_{1}$ and $g_{2}$ be matrix
functions from $\R^{n}$ to $\R^{n\times n}$. If
\[
D^{2}g_{1}[h,h]\succeq0\qquad\mbox{ and }\qquad\tr\Par{(g_{1}+g_{2})^{-1}D^{2}g_{2}[h,h]}\geq-\norm h_{g_{1}+g_{2}}^{2},
\]
then $g_{1}+g_{2}$ is lower trace self-concordant.
\end{lem}

Also, a special case of the lemma is that if $D^{2}g_{1}[h,h]\succeq0$
and $D^{2}g_{2}[h,h]\succeq0$, then $g_{1}+g_{2}$ is lower trace
self-concordant. Note that this condition is \emph{additive. }

\section{Mixing of the $\protect\dw$ \label{sec:mixingDikin}}

We prove Theorem~\ref{thm:generalMixing}, carrying out the convergence
analysis of the $\dw$ under weaker assumptions on the local metric
compared to those in \cite{laddha2020strong}. Their framework requires
the convexity of $\log\det g(x)$, i.e., $\hess\log\det g(x)[h,h]\succeq0$
for any $h\in\Rn$. By direct computation using matrix calculus (see
(\ref{eq:hessLogDet})), this condition corresponds to
\[
\tr\Par{g(x)^{-\half}D^{2}g(x)[h,h]g(x)^{-\half}}\geq\norm{g(x)^{-\half}Dg(x)[h,h]g(x)^{-\half}}_{F}^{2}.
\]
The main purpose of this condition is to ensure that the acceptance
probability of each Dikin step (see the last line of Algorithm~\ref{thm:generalMixing})
is bounded below by a constant. We show that this condition can be
relaxed to what we refer to as \emph{lower trace self-concordance}
(see Definition~\ref{def:sc}):
\[
\tr\Par{g(x)^{-\half}D^{2}g(x)[h,h]g(x)^{-\half}}\geq-\norm h_{g(x)}^{2},
\]
which is generally much weaker than the convexity of $\log\det g(x)$.

\paragraph{Proof outline.}

We follow a standard conductance based argument (see e.g., \cite{lovasz1993random,vempala2005geometric}).
A lower bound on the conductance of a Markov chain provides an upper
bound on the mixing time of the Markov chain due to the following
result.
\begin{lem}
[\cite{lovasz1993random}] \label{lem:conductanceBound}Let $\pi_{T}$
be the distribution obtained after $T$ steps of a lazy reversible
Markov chain of conductance at least $\Phi$ with stationary distribution
$\pi$ and initial distribution $\pi_{0}$. For the warmness parameter
$\Lambda=\sup_{\text{\textup{measurable} }S}\frac{\pi_{0}(S)}{\pi(S)}$
of the initial distribution,
\[
\dtv\Par{\pi_{T},\pi}\leq\sqrt{\Lambda}\Par{1-\frac{\Phi^{2}}{2}}^{T}.
\]
\end{lem}

A lower bound on the conductance follows from two ingredients: \textbf{(i)}
one-step coupling and \textbf{(ii)} isoperimetry. The first refers
to showing that the one-step distributions of the $\dw$ from two
nearby points have TV distance bounded away from one. The second is
a purely geometry property about the expansion of the target distribution. 
\begin{prop}
[\cite{kook2022condition}, Adapted from Proposition 9] \label{prop:conductance}Let
$\pi$ be the stationary distribution of a lazy reversible Markov
chain on $\mcal$ with a transition kernel $P_{x}$. Assume the isoperimetry
$\psi_{\mcal}$ under distance $d(x,y)=\norm{x-y}_{g(x)}$ and the
following one-step coupling: if $\norm{x-y}_{g(x)}\leq\Delta<1$ for
$x,y\in\mcal$, then $\dtv\Par{P_{x},P_{y}}\leq0.9$. Then the conductance
$\Phi$ of the Markov chain is bounded lower by $\Omega\Par{\psi_{\mcal}\Delta}$.
\end{prop}

We begin with one-step coupling.
\begin{lem}
[One-step coupling of $\dw$] \label{lem:one-step} For a convex
body $K\subset\Rn$, let $g\in\R^{n\times n}$ be strongly and lower
trace self-concordant on $K$. For step size $r=\frac{1}{2^{12}}$,
if $\norm{x-y}_{g(x)}\leq\frac{r}{\sqrt{n}}$ for $x,y\in K$, then
$\dtv(P_{x},P_{y})\leq3/4$ for the transition kernel $P_{x}$ of
the $\dw$.
\end{lem}

When bounding the overlap of the transition kernels of two close-by
points, there are two sources of deviation --- rejection probability
and overlap of two Dikin ellipsoids. Analyzing the former involves
both strong self-concordance and lower trace self-concordance of the
metric along with the concentration of measure, while the latter only
uses strong self-concordance.
\begin{proof}
Let $q_{x}(z)$ be the probability density of the uniform distribution
$U_{g}^{r}(x)$ over $\dcal_{g}^{r}(x)$. Let $A(x,z)=\min\Par{1,\frac{q_{z}(x)}{q_{x}(z)}}=\min\Par{1,\sqrt{\frac{\det g(z)}{\det g(x)}}}$.
We can write the transition kernel at $x$ as
\[
P_{x}(dz)=\underbrace{\Par{1-\int A(x,z)q_{x}(z)dz}}_{=:r_{x}}\cdot\delta_{x}(dz)+A(x,z)q_{x}(z)dz,
\]
where $\delta_{x}$ is the point mass at $x$. Let $r_{x}=1-\int A(x,z)q_{x}(dz)$
be the rejection probability at $x$. By the triangle inequality,
\begin{align}
\dtv(P_{x},P_{y}) & \leq\dtv\Par{P_{x},U_{g}^{r}(x)}+\dtv\Par{U_{g}^{r}(x),U_{g}^{r}(y)}+\dtv\Par{U_{g}^{r}(y),P_{y}}\nonumber \\
 & =r_{x}+\dtv\Par{U_{g}^{r}(x),U_{g}^{r}(y)}+r_{y}\label{eq:TVfinal}
\end{align}
Going forward, we drop the superscript of the radius $r$ to remove
clutters. We first bound the rejection probabilities $r_{x},r_{y}$
(Part I) and then $\dtv\Par{U_{g}^{r}(x),U_{g}^{r}(y)}$ (Part II).

\paragraph{Part I: $r_{x}$ and $r_{y}$.}

Here we use strong and lower trace self-concordance. Suppose $z$
is drawn from $U_{g}(x)$ at random. Note that the rejection probability
when moving from $x$ to $z$ can be written as 
\[
1-\min\Par{1,\frac{1/\vol\Par{\dcal_{g}(z)}}{1/\vol\Par{\dcal_{g}(x)}}}=1-\min\Par{1,\sqrt{\frac{\det\Par{g(z)}}{\det\Par{g(x)}}}}.
\]
For $\varphi(z):=\log\det g(z)$, it suffices to show that $\vphi(z)-\varphi(x)$
can be made larger than $-\veps$ for any given $\veps>0$. By Taylor's
theorem, there exists $x^{*}\in[x,z]$ such that
\[
\vphi(z)=\vphi(x)+\underbrace{\inner{\grad\vphi(x),z-x}}_{=:\text{I}}+\half\underbrace{D^{2}\vphi(x^{*})\Brack{z-x,z-x}}_{=:\text{II}}.
\]
For I, note that $z'=g(x)^{\half}(z-x)$ follows the uniform distribution
over the ball of radius $r$ centered at $0$. Using a standard concentration
inequality in high-dimensional space (see Theorem~2.7 in \cite{blum2020foundations}\footnote{For $c\ge1$, $n\geq3$, and a fixed vector $v\in\Rn$,
\[
\P_{x\sim\mathrm{Unif}(B_{r}(0))}\Par{x\in\Rn:v^{\top}x\geq-\frac{c}{\sqrt{n-1}}r\norm v_{2}}\geq1-\frac{2}{c}e^{-c^{2}/2}.
\]
}), with probability at least $0.99$ the following holds:
\begin{align}
\inner{\grad\vphi(x),z-x} & =\inner{g(x)^{-\half}\grad\vphi(x),g(x)^{\half}(z-x)}\geq-\frac{4r}{\sqrt{n}}\norm{g(x)^{-\half}\grad\vphi(x)}_{2}.\label{eq:eq1}
\end{align}
We can further bound $\norm{g(x)^{-\half}\grad\vphi(x)}_{2}$ via
strong self-concordance of $g$ as follows:
\begin{align}
\norm{g(x)^{-\half}\grad\vphi(x)}_{2} & =\sup_{v:\norm v_{2}=1}\grad\vphi(x)^{\top}g(x)^{-\half}v\nonumber \\
 & \underset{\text{(i)}}{=}\sup_{v:\norm v_{2}=1}\tr\Par{g(x)^{-1}Dg(x)\Brack{g(x)^{-\half}v}}\nonumber \\
 & =\sup_{v:\norm v_{2}=1}\tr\Par{g(x)^{-\half}Dg(x)\Brack{g(x)^{-\half}v}g(x)^{-\half}}\nonumber \\
 & \underset{\text{(ii)}}{\leq}\sup_{v:\norm v_{2}=1}\sqrt{n}\norm{g(x)^{-\half}Dg(x)\Brack{g(x)^{-\half}v}g(x)^{-\half}}_{F}\nonumber \\
 & \underset{\text{(iii)}}{\leq}\sup_{v:\norm v_{2}=1}2\sqrt{n}\norm{g(x)^{-\half}v}_{g(x)}=2\sqrt{n},\label{eq:eq2}
\end{align}
where (i) follows from (\ref{eq:gradLogDet}), (ii) is due to $\tr(A)\leq\sqrt{n}\norm A_{F}$
for $A\in\Rnn$ in general, and (iii) is simply the definition of
strong self-concordance. Putting (\ref{eq:eq1}) and (\ref{eq:eq2})
together, we have
\[
\inner{\grad\vphi(x),z-x}\geq-\frac{4r}{\sqrt{n}}2\sqrt{n}=-8r
\]
and thus $\inner{\grad\vphi(x),z-x}\geq-1/512$ with probability at
least $0.99$ due to $r=\frac{1}{2^{12}}$. 

For II, with $h=z-x$ we have 
\begin{align*}
D^{2}\vphi(x^{*})[h,h] & \underset{\eqref{eq:hessLogDet}}{=}\tr\Par{g(x^{*})^{-1}D^{2}g(x^{*})[h,h]}-\norm{g(x^{*})^{-\half}Dg(x^{*})[h]g(x^{*})^{-\half}}_{F}^{2}\\
 & \underset{\text{(i)}}{\geq}-\norm h_{g(x^{*})}^{2}-\norm{g(x^{*})^{-\half}Dg(x^{*})[h]g(x^{*})^{-\half}}_{F}^{2}\\
 & \underset{\text{(ii)}}{\geq}-\norm h_{g(x^{*})}^{2}-4\norm h_{g(x^{*})}^{2}\\
 & \underset{\text{(iii)}}{\geq}-\frac{5}{\Par{1-\norm{x-x^{*}}_{g(x)}}^{2}}\norm h_{g(x)}^{2},
\end{align*}
where (i) follows from lower trace self-concordance, (ii) is strong
self-concordance, and (iii) follows from Lemma~\ref{lem:scCloseness}.
Hence, $D^{2}\vphi(x^{*})[z-x,z-x]\geq-6r^{2}\geq-1/512$. Putting
the bounds on I and II together, with probability $0.99$ we have
$\vphi(z)-\vphi(x)\geq-1/256$ and thus 
\begin{equation}
r_{x}\leq0.99\cdot(1-\exp(-1/256))+0.01\leq0.014,\label{eq:rxbound}
\end{equation}
and similarly $r_{y}\leq0.014$.

\paragraph{Part II: $\protect\dtv\protect\Par{U_{g}^{r}(x),U_{g}^{r}(y)}$.}

This part is analogous to \cite{laddha2020strong}, using strong self-concordance
of the metric. In this part, in order to make clear the local metric,
let $E(A,x):=\{y\in\Rn:\sqrt{(y-x)^{\top}A(y-x)}\leq r\}$ and $U_{E}^{r}(x)$
the uniform distribution over $E(A,x)$. By the triangle inequality
\[
\dtv\Par{U_{g}^{r}(x),U_{g}^{r}(y)}\leq\underbrace{\dtv\Par{U_{g(x)}^{r}(x),U_{g(x)}^{r}(y)}}_{\text{(I)}}+\underbrace{\dtv\Par{U_{g(x)}^{r}(y),U_{g(y)}^{r}(y)}}_{\text{(II)}}.
\]
(I) is invariant under the transformation $z\mapsto g(x)^{1/2}z$,
and thus (I) is equal to $\dtv(B_{r}(x),B_{r}(y))$. Since the condition
of $\norm{x-y}_{g(x)}\leq\frac{r}{\sqrt{n}}$ turns into $\norm{x-y}_{2}\leq\frac{r}{\sqrt{n}}$
under the transformation, Lemma~3.2 in \cite{kannan1997random} implies
that $\dtv(B_{r}(x),B_{r}(y))\leq\frac{e}{e+1}$.

For (II), WLOG assume $\vol\Par{E(g(y),y)}\geq\vol\Par{E(g(x),y)}$.
In the following equation, let us use $D(\cdot)$ to denote the Dikin
ellipsoid $E(g(\cdot),y)$ for simplicity (i.e., omit the center $y$):

\begin{align}
\text{II} & =\half\int_{\Rn}\Abs{\frac{\bm{1}_{D(y)}(z)}{\vol\Par{D(y)}}-\frac{\bm{1}_{D(x)}(z)}{\vol\Par{D(x)}}}dz\nonumber \\
 & =\half\Par{\frac{\vol\Par{D(y)\backslash D(x)}}{\vol\Par{D(y)}}+\frac{\vol\Par{D(x)\backslash D(y)}}{\vol\Par{D(x)}}+\frac{\vol\Par{D(y)}-\vol\Par{D(x)}}{\vol\Par{D(y)}\vol\Par{D(x)}}\cdot\vol\Par{D(x)\cap D(y)}}\nonumber \\
 & =\half\Par{1-\frac{\vol\Par{D(x)\cap D(y)}}{\vol\Par{D(y)}}+1-\frac{\vol\Par{D(x)\cap D(y)}}{\vol\Par{D(x)}}+\frac{\vol\Par{D(x)\cap D(y)}}{\vol\Par{D(x)}}-\frac{\vol\Par{D(x)\cap D(y)}}{\vol\Par{D(y)}}}\nonumber \\
 & =1-\frac{\vol\Par{D(x)\cap D(y)}}{\vol\Par{D(y)}}.\label{eq:eq3}
\end{align}
Since the ratio of volumes is invariant under affine transformation,
we may assume that $y=0$ and $g(y)=I$. Let $g(x)^{-1}=U^{\top}\Diag(\lda)U$
be a spectral decomposition of $g(x)^{-1}$, where $\{\lda_{i}\}_{i=1}^{n}$
is the set of eigenvalues of $g(x)^{-1}$ and $\lda:=(\lda_{i})\in\Rn$.
Consider a matrix $C\in\R^{n\times n}$ such that $C^{-1}=U^{\top}\Diag(\min(1,\lda))U$,
and by construction $D(C)\subset D(g(x))\cap D(I)$. Thus,
\begin{align}
\frac{\vol\Par{D(x)\cap D(y)}}{\vol\Par{D(y)}} & =\frac{\vol\Par{D(x)\cap I}}{\vol\Par I}\nonumber \\
 & \geq\sqrt{\prod_{i:\lda_{i}\leq1}\lda_{i}}=\sqrt{\prod_{i:\lda_{i}\leq1}(1-(1-\lda_{i}))}\nonumber \\
 & \geq\sqrt{\exp\Par{-2\sum_{i:\lda_{i}\leq1}(1-\lda_{i})}},\label{eq:eq4}
\end{align}
where the last inequality follows from $1-x\geq\exp(-2x)$ for $0\leq x\leq\half$
and the fact that $\Par{1-\norm{y-x}_{g(x)}}^{2}I\preceq g(x)^{-1}\preceq\Par{1-\norm{y-x}_{g(x)}}^{-2}I$
in Lemma~\ref{lem:scCloseness} guarantees $\lda_{i}\geq\half$.
We further note that 
\begin{align}
\sum_{\lda_{i}<1}(1-\lda_{i}) & \leq\sqrt{n}\sqrt{\sum_{i}(1-\lda_{i})^{2}}=\sqrt{n}\norm{I-g(x)^{-1}}_{F}\nonumber \\
 & \underset{\text{(i)}}{\leq}\sqrt{n}\norm{x-y}_{g(x)}\leq\frac{1}{256},\label{eq:eq5}
\end{align}
where (i) follows from Lemma~\ref{lem:strongSC-closeness}.

Combining (\ref{eq:eq3}), (\ref{eq:eq4}), and (\ref{eq:eq5}) together,
$(\text{II})\leq1-e^{-\frac{1}{512}}$, and along with the bound on
I we obtain 
\[
\dtv\Par{U_{g}^{r}(x),U_{g}^{r}(y)}\leq\frac{e}{e+1}+1-e^{-\frac{1}{512}}.
\]
Putting this and (\ref{eq:rxbound}) back to (\ref{eq:TVfinal}),
we conclude that
\begin{align*}
\dtv(P_{x},P_{y}) & \leq\half\Par{r_{x}+r_{y}}+\dtv\Par{U_{g}^{r}(x),U_{g}^{r}(y)}\\
 & \leq0.028+\frac{e}{e+1}+1-e^{-\frac{1}{512}}\\
 & \leq3/4.\qedhere
\end{align*}
\end{proof}
Next, the isoperimetry of the uniform distribution under the local
metric $\norm{x-y}_{g(x)}$ is guaranteed to be $\Omega\Par{1/\sqrt{\onu}}$,
a fact that is used in \cite{laddha2020strong}. We state it formally
below with a proof. 
\begin{lem}
\label{lem:isoperimetry} For a convex body $K$, the isoperimetry
$\psi$ of the uniform distribution over $K$ under distance $\norm{x-y}_{g(x)}$
is at least $\Omega\Par{1/\sqrt{\onu}}$.
\end{lem}

Recall that the \emph{cross-ratio distance} $d_{K}$ defined on a
convex body $K$: for $x,y\in K$, suppose that the chord passing
through $x,y$ has endpoints $p$ and $q$ in the boundary $\del K$
(so the order of points is $p,x,y,q$), then the cross-ratio distance
between $x$ and $y$ is defined by 
\[
d_{K}(x,y)\defeq\frac{\norm{x-y}_{2}\norm{p-q}_{2}}{\norm{p-x}_{2}\norm{y-q}_{2}}.
\]

\begin{proof}
By Theorem~1 in \cite{lovasz1999hit}, we have that for any partition
$\{S_{1},S_{2},S_{3}\}$ of $K$ 
\[
\textrm{\ensuremath{\vol}}(S_{3})\vol(K)\geq d_{K}(S_{1},S_{2})\vol(S_{1})\vol(S_{2}),
\]
where $d_{K}(S_{1},S_{2})=\inf_{x\in S_{1},y\in S_{2}}d_{K}(x,y)$.
By Lemma~2.3 in \cite{laddha2020strong} ($d_{K}(x,y)\geq\norm{x-y}_{x}/\sqrt{\onu}$
for any $x,y\in K$), we have $\textrm{\ensuremath{\vol}}(S_{3})\vol(K)\geq\inf_{x\in S_{1},y\in S_{2}}\norm{x-y}_{g(x)}\vol(S_{1})\vol(S_{2})/\sqrt{\onu}$
and 
\[
\frac{\vol(S_{3})/\inf_{x\in S_{1},y\in S_{2}}\norm{x-y}_{g(x)}}{\vol(S_{1})\vol(S_{2})/\vol(K)}\geq\frac{1}{\sqrt{\onu}},
\]
leading to $\psi=\Omega\Par{1/\sqrt{\onu}}$.
\end{proof}

Putting the one-step coupling and the isoperimetry together, we can
bound the mixing time of the $\dw$.

\thmGeneralMixing*
\begin{proof}
Proposition~\ref{prop:conductance} ensures that $\Phi=\Omega\Par{1/\sqrt{n\onu}}$
due to the one-step coupling in Lemma~\ref{lem:one-step} and the
isoperimetry in Lemma~\ref{lem:isoperimetry}. Therefore, using Lemma~\ref{lem:conductanceBound},
we can enforce $\dtv(\pi_{T},\pi)\leq\veps$ by solving $\sqrt{\Lambda}e^{-T\Phi^{2}/2}\leq\veps$
for $T$, which results in
\[
T=O\Par{n\onu\log\frac{\Lambda}{\veps}}=\otilde{n\onu}.\qedhere
\]
\end{proof}


\section{Sampling the PSD cone \label{sec:basic-psdSampling}}

With the framework in Section~\ref{sec:mixingDikin} in hand, we
can design and analyze the $\dw$ for sampling the constrained PSD
cone. The framework is formally for $\Rn$, while handling the PSD
cone necessitates reasoning with matrices in $\S^{n}$. We bridge
this gap in Section~\ref{subsec:formalism} to clarify any implicit
transitions between vectors in $\R^{d}$ and matrices in $\S^{n}$.
Then in Section~\ref{subsec:scBasicMetric}, we introduce a local
metric that is both strongly self-concordant and lower trace self-concordant
in the PSD cone; along with a bound on its $\onu$-symmetry, we obtain
the mixing time of the $\dw$ with this metric. We then give an efficient
implementation of each step of this $\dw$ in Section~\ref{subsec:oracleImplementation}.
Theorem~\ref{thm:basicPSD} restated below summarizes these results.

\thmBasicPSD*

\subsection{Formalism via matrix-vector transformations \label{subsec:formalism}}

In dealing with the $\dw$ for PSD sampling, we work in $\R^{d}$
and $\S^{n}$ simultaneously in the sequel (recall $d=n(n+1)/2$),
moving back and forth between them implicitly. We justify this identification
as follows.

\paragraph{Measure on $\protect\S^{n}$.}

We can define and work with the Lebesgue measure on $\S^{n}$ by identifying
it with the Lebesgue measure on $\R^{d}$, where each component in
the Lebesgue measure on $\S^{n}$ corresponds to each entry in the
upper triangular part. Hence, with the Lebesgue measure $dX$ on $\S^{n}$
it is straightforward to define a probability distribution on $\S^{n}$
whose probability density function with respect to $dX$ is proportional
to $e^{-f(X)}$ for a function $f:\S^{n}\to\R$. For instance, the
uniform distribution over a region corresponds to $f(X)$ being constant
in the region and infinity outside of the region, and an exponential
distribution to $f(X)=\inner{C,X}=\tr(C^{\top}X)$ for $C\in\S^{n}$.

\paragraph{Directional derivatives.}

A function $\phi:\S^{n}\to\R$ induces its counterpart $\psi:\R^{d}\to\R$
defined by $\psi(x)=\phi(X)$ for $x:=\svec(X)$. For symmetric matrices
$\{H_{i}\}_{i\leq k}$, the $k^{th}$-directional derivative of $\phi$
in directions $H_{1},\dots,H_{k}$ is 
\[
D^{k}\phi(X)[H_{1},\cdots,H_{k}]\defeq\frac{d^{k}}{dt_{k}\cdots dt_{1}}\phi\Par{X+\sum_{i=1}^{k}t_{i}H_{i}}\bigg\vert_{t_{1},\dots,t_{k}=0}.
\]
For $h_{i}:=\svec(H_{i})$, it follows that $\phi\Par{X+\sum_{i=1}^{k}t_{i}H_{i}}=\psi(x+\sum_{i=1}^{k}t_{i}h_{i})$
and thus
\[
D^{k}\phi(X)[H_{1},\cdots,H_{k}]=D^{k}\psi(x)[h_{1},\cdots,h_{k}].
\]
With this identification in hand, since the notion of (symmetric or
strong) self-concordance is formulated in terms of directional derivatives,
we can deal with both representations without having to specify one
of them.

\paragraph{Important operators.}

We introduce three linear operators that enable us to make smooth
transitions between $\S^{n}$ and $\R^{d}$.
\begin{defn}
[\cite{magnus1980elimination}] \label{def:linearOperators} Let
$E_{ij}=e_{i}e_{j}^{\top}\in\Rnn$ be the matrix with a single $1$
in the $(i,j)$ position and zeros elsewhere.
\begin{itemize}
\item $M:\R^{d}\to\R^{n^{2}}$ is the linear operator that maps $\svec(\cdot)$
to $\vec(\cdot)$ (i.e., $M\circ\svec=\vec$). It can be written as
$M=\sum_{i\geq j}\vec(T_{ij})u_{ij}^{\top}$, where $T_{ij}\in\R^{n\times n}$
has all zero entries except for $1$ at $(i,j)$ and $(j,i)$ positions
(i.e., $T_{ij}=E_{ij}+E_{ji}$ if $i\neq j$ and $E_{ij}$ if $i=j$),
and $u_{ij}=\svec(E_{ij})$.
\item $N:\R^{n^{2}}\to\R^{n^{2}}$ is the linear operator that maps $\vec(A)$
to $\vec\Par{\half(A+A^{\top})}$ for a matrix $A\in\R^{n\times n}$.
Note that $N$ is symmetric.
\item $L:\R^{d}\to\R^{n^{2}}$ is the linear operator that maps $\vec(A)$
to $\svec(A)$ for a matrix $A\in\R^{n\times n}$. It can be written
as $L=\sum_{i\geq j}u_{ij}\vec(E_{ij})^{\top}$. 
\end{itemize}
\end{defn}


\subsection{Self-concordant metric for the PSD cone \label{subsec:scBasicMetric}}

Our goal is to design an algorithm that samples a positive-definite
matrix from (\ref{eq:PSDcone}) uniformly at random (i.e., according
to the Lebesgue measure on $\S^{n}$ truncated over the PSD cone).
We observe that $-\log\det X$ is a self-concordant barrier for the
PSD constraint $X\succeq0$, and $-\sum_{i}\log\Par{b_{i}-\inner{A_{i},X}}$
(the standard log barrier) for linear constraints $\inner{A_{i},X}\leq b_{i}$
for all $i\in[m]$. With our general framework for analyzing the $\dw$
in mind, this observation suggests that the combination of the Hessians
of these self-concordant barriers could be a good candidate as the
local metric of the $\dw$. We show this is indeed the case with appropriate
scaling.

\subsubsection{Analysis of the log-determinant barrier}

We first study the metric defined by the Hessian of self-concordant
barrier $\phi(X):=-\log\det X$ (see Theorem 4.3.3 in \cite{nesterov2003introductory}
for self-concordance). In this case, its Hessian and inverse have
clean formulas. 
\begin{lem}
\label{lem:metricFormula} Let $\grad_{X}^{2}\phi(X)=\grad_{x}^{2}\Par{-\log\det\svec^{-1}(x)}\in\R^{d\times d}$
for $X\in\psd$. Its Hessian and inverse are
\begin{align*}
\hess\phi(X) & =M^{\top}(X^{-1}\otimes X^{-1})M=M^{\top}(X\otimes X)^{-1}M,\\
\Par{\hess\phi(X)}^{-1} & =M^{\dagger}(X\otimes X)M^{\dagger\top}=LN(X\otimes X)NL^{\top},
\end{align*}
where $M^{\dagger}=(M^{\top}M)^{-1}M^{\top}\in\R^{d\times n^{2}}$
is the Moore-Penrose inverse of $M\in\R^{n^{2}\times d}$.
\end{lem}

We defer the proof to Appendix~\ref{app:matrixCalculus}. We remark
that as an immediate corollary to this, the local norm of $h\in\R^{d}$
with metric $\hess\phi(X)$ becomes
\begin{align*}
\norm h_{X}^{2} & =h^{\top}M^{\top}(X^{-1}\otimes X^{-1})Mh\\
 & =\svec(H)^{\top}M^{\top}(X^{-1}\otimes X^{-1})M\svec(H)\\
 & \underset{\text{(i)}}{=}\tr\Par{HX^{-1}HX^{-1}}=:\norm H_{X}^{2},
\end{align*}
where (i) follows from $\vec=M\circ\svec$ (Definition~\ref{def:linearOperators})
and $\tr\Par{DB^{\top}A^{\top}C}=\vec(A)^{\top}(B\otimes C)\vec(D)$
(Lemma~\ref{lem:Kronecker}). 

Next, the convexity of the log-determinant of $\hess\phi(X)$ is immediate
from Lemma~\ref{lem:Kronecker}.
\begin{prop}
[Convexity of log-determinant of Hessian] $\log\det\hess\phi(X)$
is convex in $X$.
\end{prop}

\begin{proof}
Using Lemma~\ref{lem:metricFormula} and $\det\Par{M^{\top}(A\otimes A)M}=2^{n(n-1)/2}(\det A)^{n+1}$
(Lemma~\ref{lem:Kronecker}) in the first and second equality below,
\begin{align*}
\log\det\hess\phi(X) & =\log\det\Par{M^{\top}(X^{-1}\otimes X^{-1})M}\\
 & =\frac{n(n-1)}{2}\cdot\log2-(n+1)\log\det X.
\end{align*}
Since $-\log\det X$ is convex in $X$ (immediate from (\ref{eq:2ndDiffLogDet})),
the convexity of $\log\det\hess\phi(X)$ also follows.
\end{proof}
Observe from the proof that $\log\det\hess\phi(X)=\text{Const.}+(n+1)\phi(X)$.
Differentiating both sides in direction $H$,
\begin{align}
 & \tr\Par{\Par{\hess\phi(X)}^{-1}D^{3}\phi(X)[H]}=(n+1)D\phi(X)[H]\quad(\because(\ref{eq:gradLogDet}))\nonumber \\
\Longrightarrow & \tr\Par{\Par{\hess\phi(X)}^{-\half}D^{3}\phi(X)[H]\Par{\hess\phi(X)}^{-\half}}=-(n+1)\tr(X^{-1}H).\label{eq:difflogdet}
\end{align}
We now show strong self-concordance of $n\phi(X)$.
\begin{thm}
\label{thm:logdet-scaling}For $\psi_{X}:=\sup_{H\in\S^{n}}$$\norm{\Par{\hess\phi(X)}^{-\half}D^{3}\phi(X)[H]\Par{\hess\phi(X)}^{-\half}}_{F}/\norm H_{X}$,
we have $\sqrt{2(n+1)}\leq\psi_{X}\leq2\sqrt{n}$. 
\end{thm}

\begin{proof}
For $H\in\S^{n}$ and $t\in\R$, denote $X_{t}:=X+tH$ and $g_{t}:=M^{\top}(X_{t}\otimes X_{t})^{-1}M$.
Note that
\[
\norm{\Par{\hess\phi(X)}^{-\half}D^{3}\phi(X)[H]\Par{\hess\phi(X)}^{-\half}}_{F}^{2}=\tr\Par{g^{-1}\del_{t}g_{t}\vert_{t=0}g^{-1}\del_{t}g_{t}\vert_{t=0}}
\]
and
\begin{align}
\del_{t}g_{t}\vert_{t=0} & \underset{\text{(i)}}{=}\del_{t}\Par{M^{\top}(X_{t}\otimes X_{t})^{-1}M}\bigg|_{t=0}\nonumber \\
 & \underset{\text{(ii)}}{=}-M^{\top}(X\otimes X)^{-1}\del_{t}(X_{t}\otimes X_{t})\vert_{t=0}(X\otimes X)^{-1}M\nonumber \\
 & =-M^{\top}(X^{-1}\otimes X^{-1})\Par{H\otimes X+X\otimes H}(X^{-1}\otimes X^{-1})M\nonumber \\
 & \underset{\text{(iii)}}{=}-M^{\top}\Par{X^{-1}HX^{-1}\otimes X^{-1}+X^{-1}\otimes X^{-1}HX^{-1}}M,\label{eq:18-1}
\end{align}
where (i) is due to Lemma~\ref{lem:metricFormula}, (ii) is from
(\ref{eq:diffInverse}), and (iii) follows from $(A\otimes B)(C\otimes D)=(AC)\otimes(BD)$
(Lemma~\ref{lem:Kronecker}-3).

Recall that positive semidefinite matrices have unique positive semidefinite
square roots, so $(X\otimes X)^{\half}=X^{\half}\otimes X^{\half}$
(due to the fact that $(X^{\half}\otimes X^{\half})\cdot(X^{\half}\otimes X^{\half})=X\otimes X$).
Since $g_{t}=M^{\top}(X_{t}\otimes X_{t})^{-\half}(X_{t}\otimes X_{t})^{-\half}M$,
the corresponding orthogonal projection matrix is 
\[
P_{t}:=P\Par{(X_{t}\otimes X_{t})^{-\half}M}=(X_{t}\otimes X_{t})^{-\half}Mg_{t}^{-1}M^{\top}(X_{t}\otimes X_{t})^{-\half}.
\]
 By substituting $\del_{t}g_{t}\big|_{t=0}$ with (\ref{eq:18-1}),
\begin{align*}
 & \tr\Par{g^{-1}\del_{t}g_{t}\vert_{t=0}g^{-1}\del_{t}g_{t}\vert_{t=0}}\\
 & =\tr\bigg(g^{-1}M^{\top}\Par{X^{-1}HX^{-1}\otimes X^{-1}+X^{-1}\otimes X^{-1}HX^{-1}}M\\
 & \qquad\qquad\cdot g^{-1}M^{\top}\Par{X^{-1}HX^{-1}\otimes X^{-1}+X^{-1}\otimes X^{-1}HX^{-1}}\cblue M\bigg)\\
 & =\tr\bigg(\cblue Mg^{-1}M^{\top}\Par{X^{-1}HX^{-1}\otimes X^{-1}+X^{-1}\otimes X^{-1}HX^{-1}}M\\
 & \qquad\qquad\cdot g^{-1}M^{\top}\Par{X^{-1}HX^{-1}\otimes X^{-1}+X^{-1}\otimes X^{-1}HX^{-1}}\bigg)\\
 & =\tr\Par{\Par{\cred{Mg^{-1}M^{\top}}\Par{X^{-1}HX^{-1}\otimes X^{-1}+X^{-1}\otimes X^{-1}HX^{-1}}}^{2}}\\
 & =\tr\Par{\Par{\cred{(X\otimes X)^{\half}P(X\otimes X)^{\half}}\Par{X^{-1}HX^{-1}\otimes X^{-1}+X^{-1}\otimes X^{-1}HX^{-1}}}^{2}}\\
 & =\tr\bigg(\bigg(P\underbrace{(X\otimes X)^{\half}\Par{X^{-1}HX^{-1}\otimes X^{-1}+X^{-1}\otimes X^{-1}HX^{-1}}(X\otimes X)^{\half}}_{=:S}\bigg)^{2}\bigg)\\
 & =\tr\Par{PSPS}.
\end{align*}
By Lemma~\ref{lem:Kronecker}-3 once again in the second equality,
we can further manipulate $S$ as follows:
\begin{align*}
S & =(X^{\half}\otimes X^{\half})\Par{X^{-1}HX^{-1}\otimes X^{-1}+X^{-1}\otimes X^{-1}HX^{-1}}(X^{\half}\otimes X^{\half})\\
 & =\underbrace{X^{-\half}HX^{-\half}\otimes I}_{=:A}+\underbrace{I\otimes X^{-\half}HX^{-\half}}_{=:B}.
\end{align*}
By the Cauchy-Schwartz inequality along with $P^{\top}P=P^{2}=P$
and $P\preceq I$,
\begin{align*}
\tr\Par{PSPS} & \leq\tr\Par{\Par{PS}^{\top}PS}=\tr\Par{S^{\top}P^{\top}PS}=\tr\Par{S^{\top}PS}\\
 & \leq\tr\Par{S^{\top}S}=\norm S_{F}^{2}\\
 & \leq\Par{\norm A_{F}+\norm B_{F}}^{2}.
\end{align*}
Using Lemma~\ref{lem:Kronecker}-3, 
\begin{align*}
\norm A_{F}^{2} & =\tr\Par{\Par{X^{-\half}HX^{-\half}\otimes I}\cdot\Par{X^{-\half}HX^{-\half}\otimes I}}\\
 & =\tr(X^{-\half}HX^{-1}HX^{-\half}\otimes I)\\
 & =\tr\Par{X^{-\half}HX^{-1}HX^{-\half}}\tr(I)\\
 & =n\norm H_{X}^{2},
\end{align*}
and similarly $\norm B_{F}^{2}=n\norm H_{X}^{2}$. Therefore, 
\[
\norm{\Par{\hess\phi(X)}^{-\half}D^{3}\phi(X)[H]\Par{\hess\phi(X)}^{-\half}}_{F}\leq\sqrt{\tr(PSPS)}\leq2\sqrt{n}\norm H_{X},
\]
and $\psi_{X}\leq2\sqrt{n}$.

To see the optimality of $\sqrt{n}$, let us recall (\ref{eq:difflogdet}):
\[
\tr\Par{\Par{\hess\phi(X)}^{-\half}D^{3}\phi(X)[H]\Par{\hess\phi(X)}^{-\half}}=-(n+1)\tr(X^{-1}H).
\]
Taking supremum on both sides, 
\begin{align*}
\sup_{H:\norm H_{X}=1}\tr\Par{\Par{\hess\phi(X)}^{-\half}D^{3}\phi(X)[H]\Par{\hess\phi(X)}^{-\half}} & =\sup_{\substack{H\in\S^{n}:\\
\norm{X^{-\half}HX^{-\half}}_{F}=1
}
}-(n+1)\tr(X^{-\half}HX^{-\half})\\
 & =\sup_{S\in\S^{n}:\norm S_{F}=1}(n+1)\tr(S),
\end{align*}
and this objective achieves the maximum at $H=-\frac{1}{\sqrt{n}}X$,
with the supremum being $(n+1)\sqrt{n}$. On the other hand, due to
$\tr(A)\leq\sqrt{d}\norm A_{F}$ for $A\in\R^{d\times d}$,
\begin{align*}
\tr\bigg(\Par{\hess\phi(X)}^{-\half}D^{3}\phi(X)[H] & \Par{\hess\phi(X)}^{-\half}\bigg)\\
 & \leq\sqrt{\frac{n(n+1)}{2}}\cdot\norm{\Par{\hess\phi(X)}^{-\half}D^{3}\phi(X)[H]\Par{\hess\phi(X)}^{-\half}}_{F}\\
 & \leq\sqrt{\frac{n(n+1)}{2}}\cdot\psi_{X}\norm H_{X}
\end{align*}
and thus by taking supremum on both sides over a symmetric matrix
$H$ with $\norm H_{X}=1$, it follows that $(n+1)\sqrt{n}\leq\sqrt{\frac{n(n+1)}{2}}\psi_{X}$
and 
\[
\sqrt{2(n+1)}\leq\psi_{X}.\qedhere
\]
\end{proof}
This result informs us of the best possible scaling of $\phi(X)$
that leads to a strongly self-concordant barrier. Recall that if $g$
satisfies $\norm{g^{-\half}Dg[h]g^{-\half}}_{F}\leq2\alpha\norm h_{g}$
for $\alpha>0$, then $\alpha^{2}g$ is strongly self-concordant.
We remark that the scaling of $\sqrt{n}$ is obviously better than
the trivial scaling of $\sqrt{d}=\Theta(n)$.
\begin{cor}
[Strong self-concordance] \label{cor:strongSCofLOGDET} For $X\in\psd$,
a function $n\phi(X)=-n\log\det X$ is a strongly self-concordant
barrier for $\psd$. Moreover, the scaling factor of $n$ cannot be
further improved.
\end{cor}

\begin{lem}
[$\onu$-symmetry] \label{lem:logdet-symm}For $X\in K=\mathbb{S}_{+}^{n}$,
$n\phi(X)=-n\log\det X$ is $n^{2}$-symmetric.
\end{lem}

\begin{proof}
For $X\in K$, pick any $Y\in K\cap(2X-K)$, and define a symmetric
matrix $H:=Y-X$. Since $Y\in K$ and $2X-Y\in K$, we have $X+H\in K$
and $X-H\in K$. Thus,
\[
-I\preceq X^{-\half}HX^{-\half}\preceq I,
\]
and the magnitude of each eigenvalue $\{\lda_{i}\}_{i=1}^{n}$ of
$X^{-\half}HX^{-\half}$ is bounded by $1$. Hence,
\begin{align*}
\norm H_{X}^{2} & =\tr(X^{-1}HX^{-1}H)=\norm{X^{-\half}HX^{-\half}}_{F}^{2}\leq\sum_{i=1}^{n}\lda_{i}^{2}\leq n.
\end{align*}
As we scale $\phi(X)$ by $n$, the symmetry parameter becomes $\onu\leq n^{2}$
by Lemma~\ref{lem:symmScaling}.
\end{proof}

\subsubsection{Local metric for PSD cone}

To handle both constraints of $X\succeq0$ and $\inner{A_{i},X}\leq b_{i}$
in (\ref{eq:PSDcone}), we use the metric $g(X)$ induced by a barrier
$\phi(X):=-2\Par{n\log\det X+\sum_{i=1}^{m}\log\Par{b_{i}-\inner{A_{i},X}}}$.
To be precise, 
\begin{align*}
g & =2\Par{ng_{1}+g_{2}},\ \text{where}\\
g_{1}(X) & :=M^{\top}(X^{-1}\kro X^{-1})M,\\
g_{2}(X) & :=M^{\top}\left[\begin{array}{ccc}
\vec(A_{1}) & \cdots & \vec(A_{m})\end{array}\right]S_{X}^{-2}\left[\begin{array}{c}
\vec(A_{1})^{\top}\\
\vdots\\
\vec(A_{m})^{\top}
\end{array}\right]M\\
 & =M^{\top}A^{\top}S_{X}^{-2}AM=M^{\top}A_{X}^{\top}A_{X}M,
\end{align*}
where $S_{X}=\Diag(b_{i}-\inner{A_{i},X})\in\R^{m\times m}$, $A^{\top}=\left[\begin{array}{ccc}
\vec(A_{1}) & \cdots & \vec(A_{m})\end{array}\right]\in\R^{n^{2}\times m}$, and $A_{X}=S_{X}^{-1}A\in\R^{m\times n^{2}}$.

We now show that this matrix function $g$ satisfies the assumptions
in Theorem~\ref{thm:generalMixing}: 
\begin{lem}
\label{lem:metricBasic} The matrix function $g$ is $O(n^{2}+m)$-symmetric
and strongly and lower trace self-concordant.
\end{lem}

The discussion above tells us that $-n\log\det X$ is $n^{2}$-symmetric
and strongly self-concordant. On the other hand, we cannot directly
refer to results about the symmetric parameter and strong self-concordance
of $\psi(X):=-\sum_{i=1}^{m}\log\Par{b_{i}-\inner{A_{i},X}}$, since
$m$ could be less than $d$ in (\ref{eq:PSDcone}). It means that
$g_{2}=\hess\psi(x)$ could be singular, so it requires additional
technical work.
\begin{proof}
It suffices to focus on the setting where $\psi$ is the logarithmic
barrier for $P=\{x\in\R^{d}:Ax\leq b\}$ with $A\in\R^{m\times d}$
and $m<d$. We check the symmetry, strong self-concordance, and lower
trace self-concordance of $g$ in order.

\paragraph{Symmetry.}

Let $y\in P\cap(2x-P)$ and $h:=y-x$. We have $\norm{A_{x}h}_{\infty}\leq1$
by Lemma~\ref{lem:symmforPolytope}, so 
\[
h^{\top}A^{\top}\Diag(s_{x})^{-2}Ah=\sum_{i=1}^{m}\Par{\frac{(Ah)_{i}}{\Par{s_{x}}_{i}}}^{2}\leq m,
\]
and thus the symmetry parameter $\onu$ is still $m$ even when $\hess\psi$
is singular. By Lemma~\ref{lem:sumSymmetricSC}, $g$ is $O(n^{2}+m)$-symmetric.

\paragraph{Strong self-concordance.}

As the PSD cone in (\ref{eq:PSDcone}) is bounded, we can simply add
dummy constraints of the form $\ell\leq x\leq u$ on top of $Ax\leq b$
so that the new representation of (\ref{eq:PSDcone}) is still the
same with $K=\{X\succeq0\}\cap\{\inner{A_{i},X}\leq b_{i}\}$. Hence,
we may assume that $\psi_{t}(x):=\psi(x)-t\sum_{i=1}^{l}\log(u_{i}-x_{i})(x_{i}-\ell_{i})$
for $t>0$ and some $l$ has the non-singular Hessian. 

Since $-n\log\det X$ and logarithmic barriers are strongly self-concordant
(see Corollary~\ref{cor:strongSCofLOGDET} and Lemma~\ref{lem:paramsBarrier}-1),
we apply Lemma~\ref{lem:sumStrongSC} to $ng_{1}$ and $\hess\psi_{t}$,
showing that $\phi_{t}:=-2\Par{n\log\det X+\psi_{t}}$ is strongly
self-concordant. Since $\hess\phi_{0}=g\succeq2ng_{1}\succ0$ is invertible,
and both sides in the definition of strongly self-concordance are
continuous in $t$, by sending $t\to0$ we can argue that $g$ is
also strongly self-concordant. 

\paragraph{Lower trace self-concordance.}

By calling upon a special case of Lemma~\ref{lem:additiveCondition},
it suffices to show that each metric $g_{i}$ satisfies $D^{2}g_{i}[h,h]\succeq0$.
For $g_{1}(X)=-\hess\log\det X$, recall that $g_{1}(X)[H,H]=\tr\Par{X^{-1}HX^{-1}H}$,
and thus for any $V\in\S^{n}$
\begin{align*}
Dg_{1}(X)[H,H,V] & =-\tr\Par{X^{-1}VX^{-1}\cdot HX^{-1}H}-\tr\Par{X^{-1}H\cdot X^{-1}VX^{-1}\cdot H}\\
 & =-2\tr\Par{X^{-1}VX^{-1}HX^{-1}H},
\end{align*}
and by differentiating again
\begin{align}
 & D^{2}g_{1}(X)[H,H,V,V]\nonumber \\
 & =4\tr\Par{X^{-1}VX^{-1}VX^{-1}HX^{-1}H}+2\tr\Par{X^{-1}VX^{-1}HX^{-1}VX^{-1}H}\nonumber \\
 & =4\tr\Par{X^{-\half}HX^{-1}VX^{-1}VX^{-1}HX^{-\half}}+2\tr\Par{X^{-\half}VX^{-1}HX^{-\half}\cdot X^{-\half}VX^{-1}HX^{-\half}}\nonumber \\
 & \underset{\text{(i)}}{\geq}4\tr\Par{X^{-\half}HX^{-1}VX^{-1}VX^{-1}HX^{-\half}}-2\tr\Par{X^{-\half}HX^{-1}VX^{-\half}\cdot X^{-\half}VX^{-1}HX^{-\half}}\nonumber \\
 & =2\tr\Par{X^{-\half}HX^{-1}VX^{-1}VX^{-1}HX^{-\half}}\geq0,\label{eq:D4ph1}
\end{align}
where we used the Cauchy-Schwartz inequality in (i), and thus $D^{2}g_{1}(X)[H,H]\succeq0$.
For the Hessian of logarithmic barrier $g_{2}(X)$, direct computation
leads to $D^{2}g_{2}(X)[H,H]\succeq0$ (see Claim~\ref{claim:diffLogBarrier}).
\end{proof}
Putting all these together, it follows that the matrix function
\[
g(X)=-2\hess\Par{n\log\det X+\sum_{i=1}^{m}\log\Par{b_{i}-\inner{A_{i},X}}}
\]
is $O(n^{2}+m)$-symmetric, strongly self-concordant, and lower trace
self-concordant. Therefore, we conclude by Theorem~\ref{thm:generalMixing}
that the $\dw$ with $g(X)$ mixes in $\otilde{n^{2}(n^{2}+m)}$ steps,
which completes the first half of Theorem~\ref{thm:basicPSD}.

\subsection{Implementation of per-step \label{subsec:oracleImplementation}}

Now we design an oracle that implements each iteration of the $\dw$
(Algorithm~\ref{alg:DikinWalk}). This can be implemented as follows:
when the current point is $x$,
\begin{itemize}
\item Sample $z\sim B_{r}(0)$.
\item Compute $y=x+g(x)^{-\half}z$ and propose it.
\item Accept $y$ with probability $\min\Par{1,\sqrt{\frac{\det g(y)}{\det g(x)}}}$.
\end{itemize}
We provide two algorithms with the complexity of $O\Par{mn^{\omega}+m^{2}n^{2}}$
and $O\Par{n^{2\omega}+mn^{2(\omega-1)}}$. We can implement each
iteration in $O\Par{\min\Par{mn^{\omega}+m^{2}n^{2},n^{2\omega}+mn^{2(\omega-1)}}}$
time by using the former for small $m$ and the latter for large $m$.
This completes the second half of Theorem~\ref{thm:basicPSD}. 

\paragraph{Algorithm for small $m$.}

For simplicity here, we ignore the constant factors of $g$, denoting
$g=g_{1}+g_{2}$ for
\begin{align*}
g_{1}(X) & =M^{\top}(X\kro X)^{-1}M=:BB^{\top},\\
g_{2}(X) & =M^{\top}A^{\top}S_{X}^{-2}AM=:UU^{\top},
\end{align*}
where $B:=M^{\top}(X\kro X)^{-1/2}\in\R^{d\times n^{2}}$ and $U:=M^{\top}A^{\top}S_{X}^{-1}\in\R^{d\times m}$.
Letting $u_{i}$ be the $i^{th}$ column of $U$ for $i\in[m]$, we
note that $g_{2}=\sum_{i=1}^{m}u_{i}u_{i}^{\top}$.

We first implement a subroutine for computing $g(X)^{-1}v$ for a
given vector $v\in\R^{d}$ in $O(mn^{\omega}+m^{2}n^{2})$ time.

\begin{algorithm2e}[t]

\caption{Computation of $g(X)^{-1}v$}\label{alg:subroutine}

\SetAlgoLined

\textbf{Input:} $X\in\psd$, vector $v\in\R^{d}$, local metric $g$.

\textbf{Output:} $g(X)^{-1}v$

Prepare the column vectors $u_{i}$ of $U=M^{\top}A^{\top}S_{X}^{-1}$.

For $\bar{g}_{0}:=g_{1}(X)$, compute $\bar{g}_{0}^{-1}v$ and $\bar{g}_{0}^{-1}u_{i}$
for $i\in[m]$.

\For{$i=1,\cdots,m$}{

Compute $\bar{g}_{i}^{-1}v$ and $\bar{g}_{i}^{-1}u_{j}$ for $j\in[m]$,
according to 

\[
\bar{g}_{i}^{-1}w=\bar{g}_{i-1}^{-1}w-\frac{\bar{g}_{i-1}^{-1}u_{i}\cdot u_{i}^{\top}\bar{g}_{i-1}^{-1}w}{1+u_{i}^{\top}\bar{g}_{i-1}^{-1}u_{i}}.
\]

}

Return $\bar{g}_{m}^{-1}v$.

\end{algorithm2e}
\begin{prop}
\label{prop:oracle} Algorithm~\ref{alg:subroutine} computes $g(X)^{-1}v$
in $O(mn^{\omega}+m^{2}n^{2})$ time for a query vector $v\in\R^{d}$.
\end{prop}

\begin{proof}
Let $v\in\R^{d}$ be a given vector, and denote $\bar{g}_{0}:=g_{1}$
and $\bar{g}_{i}:=\bar{g}_{i-1}+u_{i}u_{i}^{\top}$ for $i\in[m]$.
We first prepare the column vectors $u_{i}$'s of $U=M^{\top}A^{\top}S_{X}^{-1}$
in $O(mn^{2})$ time and then initialize $\bar{g}_{0}^{-1}v$ and
$\bar{g}_{0}^{-1}u_{i}$ for $i\in[m]$ in $O(mn^{\omega})$ time.
For $u_{i}$'s, note that $S_{X}$ can be prepared in $O(mn^{2})$
time, and thus $A^{\top}S_{X}^{-1}$ takes $O(mn^{2})$ time due to
$A\in\R^{n^{2}\times m}$. Since each row of $M^{\top}\in\R^{d\times n^{2}}$
has at most two non-zero entries, we can obtain $u_{i}$'s in $O(mn^{2})$
time.

For $\bar{g}_{0}^{-1}v$ and $\bar{g}_{0}^{-1}u_{i}$, we recall from
Lemma~\ref{lem:metricFormula} that for a vector $z\in\R^{d}$ 
\begin{align*}
g_{1}^{-1}z & =M^{\dagger}(X\kro X)M^{\dagger\top}z=LN(X\kro X)NL^{\top}z.
\end{align*}
Since each row of $L^{\top}\in\R^{n^{2}\times d}$ has at most two
non-zero entries, $w:=L^{\top}z\in\R^{n^{2}}$ can be computed in
$O(n^{2})$ time. From the definition of $N$, it follows that $Nw=\vec\Par{\half(W+W^{\top})}$
for $W:=\vec^{-1}(w)\in\R^{n\times n}$, which also can be computed
in $O(n^{2})$ time. For $\overline{W}:=\half\Par{W+W^{\top}}$, it
follows that
\begin{align*}
(X\kro X)Nw & =(X\kro X)\vec\Par{\overline{W}}\underset{\text{Lemma \ref{lem:Kronecker}-1}}{=}\vec\Par{X\overline{W}X},
\end{align*}
which can be computed in $O(n^{\omega})$ time by the fast matrix
multiplication, and in a similar way we can compute $LN\vec\Par{X\overline{W}X}$
in $O(n^{2})$ time. Putting all these together, $\bar{g}_{0}^{-1}v$
can be computed in $O(n^{\omega})$ time, and repeating this for $u_{j}$'s
yields $\Brace{\bar{g}_{0}^{-1}v,\bar{g}_{0}^{-1}u_{1},\dots,\bar{g}_{0}^{-1}u_{m}}$
in $O(mn^{\omega})$ time.

Starting with these initializations, we recursively use the Sherman--Morrison
formula: for a given vector $z\in\R^{d}$
\begin{equation}
\bar{g}_{i}^{-1}z=\bar{g}_{i-1}^{-1}z-\frac{\bar{g}_{i-1}^{-1}u_{i}u_{i}^{\top}\bar{g}_{i-1}^{-1}z}{1+u_{i}^{\top}\bar{g}_{i-1}^{-1}u_{i}}.\label{eq:sherman-morrison}
\end{equation}
Using $\bar{g}_{i-1}^{-1}u_{j}$ and $\bar{g}_{i-1}^{-1}v$ from a
previous iteration, we can compute each of $\bar{g}_{i}^{-1}u_{j}$
and $\bar{g}_{i}^{-1}v$ in the current iteration in $O(n^{2})$ time,
and thus each round for update takes $O(mn^{2})$ time in total. Since
we iterate for $m$ rounds, Algorithm~\ref{alg:subroutine} outputs
$\bar{g}_{m}^{-1}v=g(X)^{-1}v$ in $O(mn^{\omega}+m^{2}n^{2})$ time.
\end{proof}
With this subroutine in hand, we proceed to an efficient implementation
of two tasks -- computation of (1) $g(x)^{-\half}z$ for a given
vector $z\in\R^{d}$ and (2) $\frac{\det g(y)}{\det g(x)}$.

\begin{algorithm2e}[t]

\caption{Per-step implementation of $\dw$}\label{alg:perStep-small-m}

\SetAlgoLined

\textbf{Input:} current point $X\in\psd$, local metric $g$

\tcp{Step 1: Uniform sampling from Dikin ellipsoid $\dcal_{g}^{r}(X)$}

Draw $w\sim\ncal\Par{0,I_{n^{2}+m}}$ and $v\gets g(X)^{-1}\left[\begin{array}{cc}
B & U\end{array}\right]w$ by Algorithm~\ref{alg:subroutine}.\

Draw $s\sim\text{Uniform(}[0,r])$ and $h\gets s^{1/d}v/\norm v_{g(X)}$.\

Propose $y\gets\svec(X)+h$.

\

\tcp{Step 2: Computation of acceptance probability}

Use Algorithm~\ref{alg:subroutine} to prepare $\Brace{\bar{g}_{i}^{-1}u_{1},\dots,\bar{g}_{i}^{-1}u_{m}}_{i=0}^{m}$
at $X$ and $Y:=\svec^{-1}(y)$.\

$\det\bar{g}_{0}(\cdot)\gets2^{n(n-1)/2}(\det(\cdot))^{-(n+1)}$ ($\because$
Lemma~\ref{lem:Kronecker}-7)

\For{$i=1,\cdots,m$}{

$\det(\bar{g}_{i+1})\gets\det\bar{g}_{i}\cdot\Par{1+u_{i+1}^{\top}\bar{g}_{i}^{-1}u_{i+1}}$.

}

Accept $Y$ with probability $\min\Par{1,\sqrt{\frac{\det\bar{g}_{m}(Y)}{\det\bar{g}_{m}(X)}}}$.

\end{algorithm2e}
\begin{lem}
\label{lem:perStep-small-m}Algorithm~\ref{alg:perStep-small-m}
implements each iteration of the $\dw$ with complexity of $O\Par{mn^{\omega}+m^{2}n^{2}}$.
\end{lem}

\begin{proof}
Here we provide details of Algorithm~\ref{alg:perStep-small-m} in
two stages -- (1) uniform sampling from a Dikin ellipsoid and (2)
computation of acceptance probability.

\paragraph{(1) Uniform sampling from Dikin ellipsoid:}

Instead of directly computing $g(X)^{-\half}v$ for $v\sim\text{Uniform}(B_{r}(0))$,
we illustrate how to draw $v\sim\ncal(0,g(X)^{-1})$ without full
computation of $g(X)^{-1}$ and use this to generate a uniform sample
from $\dcal_{g}^{r}(X)$ in $O(mn^{\omega}+m^{2}n^{2})$ time. 

Our approach is to compute $v:=g(X)^{-1}\left[\begin{array}{cc}
B & U\end{array}\right]w$ for $w\sim\ncal(0,I_{n^{2}+m})$, which follows the Gaussian distribution
with covariance
\begin{align*}
g(X)^{-1}\left[\begin{array}{cc}
B & U\end{array}\right]\Par{g(X)^{-1}\left[\begin{array}{cc}
B & U\end{array}\right]}^{\top} & =g(X)^{-1}(BB^{\top}+CC^{\top})g(X)^{-1}\\
 & =g(X)^{-1}g(X)g(X)^{-1}\\
 & =g(X)^{-1},
\end{align*}
since $v$ is a linear transformation of the Gaussian random variable
$w$.

Denoting $w=\left[\begin{array}{c}
w_{b}\\
w_{u}
\end{array}\right]$ for $w_{b}\sim\ncal(0,I_{n^{2}})$ and $w_{u}\sim\ncal(0,I_{m})$,
we can show that $\left[\begin{array}{cc}
B & U\end{array}\right]w$ can be computed in $O(n^{\omega}+mn^{2})$ time as follows:
\begin{align*}
\left[\begin{array}{cc}
B & U\end{array}\right]w & =Bw_{b}+Uw_{c}\\
 & =M^{\top}\underbrace{(X\kro X)^{-\half}w_{b}}_{\text{Use Lemma \ref{eq:sherman-morrison}}}+M^{\top}A^{\top}S_{X}^{-1}w_{c}\\
 & =M^{\top}\Par{\vec\Par{X^{-\half}\vec^{-1}(w_{b})X^{-\half}}+A^{\top}S_{X}^{-1}w_{c}},
\end{align*}
where $\vec\Par{X^{-\half}\vec^{-1}(w_{b})X^{-\half}}$ and $A^{\top}S_{X}^{-1}w_{u}$
can be computed in $O(n^{\omega})$ and $O(mn^{2})$ time, respectively.
Since each row of $M^{\top}\in\R^{d\times n^{2}}$ has at most two
non-zero entries, $\left[\begin{array}{cc}
B & U\end{array}\right]w$ can be computed in $O(n^{\omega}+mn^{2})$ time. Using Algorithm~\ref{alg:subroutine},
we obtain $v=g(X)^{-1}\left[\begin{array}{cc}
B & U\end{array}\right]w$ in $O(mn^{\omega}+m^{2}n^{2})$ time. 

Recall that a uniform sample from $B_{r}(0)\subset\R^{d}$ can be
drawn via $z=s^{1/d}\frac{\zeta}{\norm{\zeta}_{2}}$ for $\zeta\sim\ncal(0,I_{d})$
and $s\sim\text{Uniform(}[0,r])$. Then $g(X)^{-\half}z=s^{1/d}\frac{g(X)^{-\half}\zeta}{\norm{\zeta}_{2}}$
corresponds to a uniform sample from $\dcal_{g(X)}^{r}(0)$. Due to
$\frac{g(X)^{-\half}\zeta}{\norm{\zeta}_{2}}\sim\frac{v}{\norm v_{g(X)}}$
for $v\sim\ncal(0,g(X)^{-1})$, we can generate a uniform sample $s^{1/d}\frac{v}{\norm v_{g(X)}}$
from $\dcal_{g(X)}^{r}(0)$. Note that for $V=\svec^{-1}(v)$ we can
compute $\norm v_{g(X)}^{2}=\tr(VX^{-1}VX^{-1})+v^{\top}M^{\top}A_{X}^{\top}A_{X}Mv$
in $O\Par{n^{\omega}+mn^{2}}$ time. Therefore, we can conclude that
the proposal $y=\svec(X)+s^{1/d}\frac{v}{\norm v_{g(X)}}$ is obtained
in $O(mn^{\omega}+m^{2}n^{2})$ time.

\paragraph{(2) Computation of acceptance probability. }

We show that this step also takes $O(mn^{\omega}+m^{2}n^{2})$ time.
To compute $\det g(X)$, we use Algorithm~\ref{alg:subroutine} to
prepare $\Brace{\bar{g}_{i}^{-1}u_{1},\dots,\bar{g}_{i}^{-1}u_{m}}_{i=0}^{m}$
at $X$ and $Y=\svec^{-1}(y)$ in $O(mn^{\omega}+m^{2}n^{2})$ time.
Recall the matrix determinant lemma:
\[
\det\Par{A+uu^{\top}}=\det A\cdot\Par{1+u^{\top}A^{-1}u}.
\]
 Using the following recursive formula
\begin{align*}
\det\Par{\bar{g}_{i+1}} & =\det\Par{\bar{g}_{i}+u_{i+1}u_{i+1}^{\top}}=\det\bar{g}_{i}\cdot\Par{1+u_{i+1}^{\top}\bar{g}_{i}^{-1}u_{i+1}},
\end{align*}
we start with $\det\bar{g}_{0}=\det g_{1}=2^{n(n-1)/2}\Par{\det X}^{-(n+1)}$
(see Lemma~\ref{lem:Kronecker}-7), which can be computed in $O(n^{\omega})$
time, and compute $\det g(X)$ (and $\det g(Y)$ in the same way)
in $O(mn^{\omega}+m^{2}n^{2})$ time.
\end{proof}


\paragraph{Algorithm for large $m$.}

The algorithm right above has quadratic dependence on the number $m$
of constraints, which could become expensive for large $m$. In this
regime, we just fully compute the whole matrix function of size $\R^{d\times d}$,
which takes $O(n^{2\omega}+mn^{2(\omega-1)})$ time, and computing
its inverse, square-root, and determinant takes $O(n^{2\omega})$
time.



\section{$\sqrt{m}$-dependence via the Vaidya metric \label{sec:hybrid-psdSampling}}

The $\dw$ with a hybrid barrier of $-\log\det X$ and the log-barrier
mixes in $\otilde{n^{2}(n^{2}+m)}$ iterations with per-step complexity
$\min\Par{mn^{\omega}+m^{2}n^{2},n^{2\omega}+mn^{2(\omega-1)}}$.
In the regime of $m=O(n^{2})$, the first algorithm for the per-step
implementation is effective, but once $m$ gets larger the second
algorithm should be employed. Thus for large $m$, combined with the
mixing rate of $O(n^{2}m)$ the provable upper bound on the total
time ends up having quadratic dependence on $m$, the number of constraints.

Can we find a better metric that allows the mixing rate of the $\dw$
to have better dependence on $m$ while maintaining the same per-step
complexity of $O\Par{mn^{2(\omega-1)}}$? We show that the $\dw$
can enjoy $\sqrt{m}$-dependence in the mixing rate and the same per-step
complexity by running with the following metric, which is the analog
of the Vaidya metric \cite{vaidya1996new} for the PSD cone: 
\begin{align}
g(X) & =2\Par{ng_{1}(X)+g_{2}(X)},\ \text{where}\label{eq:metricHybrid}\\
g_{1}(X) & =-\hess_{X}\log\det X=M^{\top}(X\kro X)^{-1}M,\nonumber \\
g_{2}(X) & =22\sqrt{\frac{m}{n}}M^{\top}A_{X}^{\top}\Par{\Sigma_{X}+\frac{n}{m}I_{m}}A_{X}M,\nonumber 
\end{align}
where $\Sigma_{X}=\Diag(A_{X}(A_{X}^{\top}A_{X})^{-1}A_{X}^{\top})$
is the diagonal matrix with the leverage scores of $A_{X}$.

\thmHybridPSD*

Notice that the main difference between this metric and the previous
one is that $g_{2}$ used to address constraints of the form $\inner{A_{i},X}\leq b_{i}$.
In the literature of interior-point method, the metric $g_{2}$ was
proposed by \cite{vaidya1996new} and has a better barrier parameter
than the logarithmic barrier; this motivates us to analyze the $\dw$
with this metric. However, we note that upon combining this Vaidya
metric with the barrier for the PSD cone, $-\log\det X$, we should
revisit every step in the previous analysis of the $\dw$ where $g_{2}$
part previously came from the logarithmic barrier. In particular,
$D^{2}g_{2}(X)[H,H]$ is no longer guaranteed to be convex.

We begin by establishing strong self-concordance and then computing
the symmetry parameter of the Vaidya metric $g_{2}$ in Section~\ref{subsec:sc-sym-hybrid}.
 Then in Section~\ref{subsec:lowerSC-hybrid}, we use Lemma~\ref{lem:additiveCondition}
to check lower trace self-concordance of $g=2(ng_{1}+g_{2})$. Given
these, it follows from Theorem~\ref{thm:generalMixing} that the
$\dw$ with this metric mixes in $\otilde{\Par{n+\sqrt{m}}n^{3}}$
iterations with per-step complexity $\otilde{mn^{2(\omega-1)}}$.

\subsection{Strong self-concordance and symmetry \label{subsec:sc-sym-hybrid}}

In the sequel, we work with $x\in\R^{n}$ instead of $\svec(X)\in\R^{d}=\R^{n(n+1)/2}$
and replace $n$ by $d$ later when obtaining results for PSD sampling.

We first attempt to prove helper lemmas for strong self-concordance
and symmetry of the metrics of the form $A_{x}^{\top}D_{x}A_{x}$,
where $D_{x}\in\R^{m\times m}$ is a diagonal matrix used to address
the constraints of the form $Ax\leq b$ for $A\in\R^{m\times n}$
and $b\in\R^{m}$. Then we use these results to look into strong self-concordance
and symmetry of the Vaidya metric.
\begin{lem}
\label{lem:helper4Diagonal} Let $g(x)=A_{x}^{\top}D_{x}A_{x}\in\Rnn$
for a diagonal matrix $0\prec D_{x}\in\R^{m\times m}$.
\begin{itemize}
\item $\norm{g(x)^{-\half}Dg(x)[h]g(x)^{-\half}}_{F}^{2}\leq4\max_{i}\Par{\frac{\sigma\Par{\sqrt{D_{x}}A_{x}}_{i}}{\Par{D_{x}}_{ii}}}\cdot\Par{\norm h_{g(x)}^{2}+\sum_{i=1}^{m}\Par{D_{x}^{-1}}_{ii}(DD_{x}[h])_{i}^{2}}$.
\item $\max_{h:\norm h_{g(x)}=1}\norm{A_{x}h}_{\infty}=\sqrt{\max_{i\in[m]}\frac{\sigma\Par{\sqrt{D_{x}}A_{x}}_{i}}{\Par{D_{x}}_{ii}}}$.
\item $K\cap(2x-K)\subset\dcal_{g}^{\sqrt{\tr\Par{D_{x}}}}(x)$.
\end{itemize}
\end{lem}

\begin{proof}
Let us write $g(x)=A_{x}^{\top}D_{x}A_{x}=A^{\top}V_{x}A$ for $V_{x}:=S_{x}^{-1}D_{x}S_{x}^{-1}$.
By Claim~\ref{claim:1stDiffSlack}
\begin{align*}
Dg(x)[h] & =A^{\top}DV_{x}[h]A\\
 & =A^{\top}\Par{-2S_{x}^{-1}S_{x,h}S_{x}^{-1}D_{x}+S_{x}^{-1}DD_{x}[h]S_{x}^{-1}}A\\
 & =A^{\top}V_{x}^{1/2}\Par{-2S_{x,h}+D_{x}^{-1}DD_{x}[h]}V_{x}^{1/2}A\\
 & =A^{\top}V_{x}^{1/2}\overline{D}_{x}V_{x}^{1/2}A,
\end{align*}
where $\overline{D}_{x}:=-2S_{x,h}+D_{x}^{-1}DD_{x}[h]$. 

For $P_{x}:=P\Par{V_{x}^{\half}A}=V_{x}^{\half}A\Par{A^{\top}V_{x}A}^{-1}A^{\top}V_{x}^{\half}=V_{x}^{\half}Ag^{-1}A^{\top}V_{x}^{\half}$,
\begin{align*}
\norm{g(x)^{-\half}Dg(x)[h]g^{-\half}}_{F}^{2} & =\tr\Par{g^{-1}Dg[h]g^{-1}Dg[h]}\\
 & =\tr\Par{g^{-1}A^{\top}V_{x}^{1/2}\overline{D}_{x}V_{x}^{1/2}Ag^{-1}A^{\top}V_{x}^{1/2}\overline{D}_{x}V_{x}^{1/2}A}\\
 & =\tr\Par{P_{x}\overline{D}_{x}P_{x}\overline{D}_{x}}\\
 & \underset{\text{(i)}}{=}\diag\Par{\overline{D}_{x}}^{\top}P_{x}^{(2)}\diag\Par{\overline{D}_{x}}\\
 & \underset{\text{(ii)}}{\leq}\diag\Par{\overline{D}_{x}}^{\top}\Sigma_{x}\diag\Par{\overline{D}_{x}}\\
 & \underset{\text{(iii)}}{\leq}4\sum_{i=1}^{m}\sigma\Par{\sqrt{D_{x}}A_{x}}_{i}\Par{(A_{x}h)_{i}^{2}+(D_{x}^{-1}DD_{x}[h])_{i}^{2}}\\
 & \leq4\max_{i}\Par{\frac{\sigma\Par{\sqrt{D_{x}}A_{x}}_{i}}{(D_{x})_{ii}}}\cdot\sum_{i=1}^{m}(D_{x})_{ii}\Par{(A_{x}h)_{i}^{2}+(D_{x}^{-1}DD_{x}[h])_{i}^{2}}\\
 & \underset{\text{(iv)}}{=}4\max_{i}\Par{\frac{\sigma\Par{\sqrt{D_{x}}A_{x}}_{i}}{(D_{x})_{ii}}}\cdot\Par{\norm h_{g(x)}^{2}+\sum_{i=1}^{m}\Par{D_{x}^{-1}}_{ii}(DD_{x}[h])_{i}^{2}}
\end{align*}
where (i) holds due to $x^{\top}(A\hada B)y=\tr\Par{\Diag(x)A\Diag(y)B^{\top}}$
(Lemma~\ref{lem:Hadamard}), (ii) follows from $P_{x}^{(2)}\preceq\Sigma_{x}$
(Lemma~\ref{lem:schurProjection}), (iii) uses $(a+b)^{2}\leq2\Par{a^{2}+b^{2}}$
for $a,b\in\R$ and $\Sigma_{x}=\Diag(P_{x})=\Diag\Par{\sigma\Par{\sqrt{V_{x}}A}}=\Diag\Par{\sigma\Par{\sqrt{D_{x}}A_{x}}}$,
and (iv) holds due to $\sum_{i=1}^{m}D_{x,i}(A_{x}h)_{i}^{2}=h^{\top}A_{x}^{\top}D_{x}A_{x}h=h^{\top}g(x)h$.

For the second claim,
\begin{align*}
\max_{h:\norm h_{g(x)}=1}\norm{A_{x}h}_{\infty} & =\max_{h:\norm h_{g(x)}=1}\max_{i\in[m]}\Abs{\frac{a_{i}^{\top}h}{s_{i}}}\\
 & =\max_{i\in[m]}\max_{u:\norm u_{2}=1}\Abs{\frac{a_{i}^{\top}g(x)^{-1/2}u}{s_{i}}}\\
 & =\max_{i\in[m]}\norm{g(x)^{-1/2}\frac{a_{i}}{s_{i}}}_{2}\\
 & =\max_{i\in[m]}\sqrt{\frac{1}{s_{i}^{2}}a_{i}^{\top}g(x)^{-1}a_{i}}\\
 & =\sqrt{\max_{i\in[m]}e_{i}^{\top}A_{x}^{\top}g(x)^{-1}A_{x}e_{i}}\\
 & =\sqrt{\max_{i\in[m]}\frac{\sigma\Par{\sqrt{D_{x}}A_{x}}_{i}}{(D_{x})_{ii}}}.
\end{align*}

For the last claim, for $h\in\Rn$ such that $\norm{A_{x}h}_{\infty}\leq1$
(i.e., $h\in K\cap(2x-K)$ for $K=\{Ax\leq b\}$ due to Lemma~\ref{lem:symmforPolytope})
we have
\begin{align*}
h^{\top}g(x)h & =h^{\top}A_{x}^{\top}D_{x}A_{x}h=\sum_{i=1}^{m}(D_{x})_{ii}(A_{x}h)_{i}^{2}\\
 & \leq\norm{A_{x}h}_{\infty}^{2}\sum_{i=1}^{m}(D_{x})_{ii}\leq\tr\Par{D_{x}}.\qedhere
\end{align*}
\end{proof}
Using this, we can check strong self-concordance and compute the symmetry
parameter of metrics of the form $A_{x}^{\top}D_{x}A_{x}$.
\begin{lem}
\label{lem:paramsBarrier} Let $A\in\R^{m\times n}$ and $\Sigma_{x}=\Diag(\sigma(A_{x}))\in\R^{m\times m}$.
\begin{itemize}
\item Logarithmic metric: $g(x)=A_{x}^{\top}A_{x}$ with $D_{x}=I_{m}$
is strongly self-concordant with $\onu=m$.
\item Approximate volumetric metric: $g(x)=40\sqrt{m}A_{x}^{\top}\Sigma_{x}A_{x}$
with $D_{x}=40\sqrt{m}\Sigma_{x}$ is strongly self-concordant with
$\onu=O(\sqrt{m}n)$. 
\item Vaidya metric: $g(x)=22\sqrt{\frac{m}{n}}A_{x}^{\top}\Par{\Sigma_{x}+\frac{n}{m}I_{m}}A_{x}$
with $D_{x}=22\sqrt{\frac{m}{n}}\Par{\Sigma_{x}+\frac{n}{m}I_{m}}$
is strongly self-concordant with $\onu=O(\sqrt{mn})$. 
\end{itemize}
\end{lem}

We defer the proof of the first and second to Appendix~\ref{app:subsec:logBarrier}
and Appendix~\ref{app:subsec:volBarrier}, respectively.
\begin{proof}
Consider the metric without scaling: $g(x):=A_{x}^{\top}D_{x}A_{x}$,
where we set $D_{x}=\Sigma_{x}+\frac{n}{m}I_{m}$. Then
\begin{align}
\max_{i}\Par{\frac{\sigma\Par{\sqrt{D_{x}}A_{x}}_{i}}{D_{x,i}}} & \underset{\text{Lemma \ref{lem:helper4Diagonal}-2}}{=}\Par{\max_{h\in\Rn}\frac{\norm{A_{x}h}_{\infty}}{\norm h_{g(x)}}}^{2}\underset{\text{(i)}}{\leq}\sqrt{\frac{m}{n}},\label{eq:28-1}\\
\sum_{i=1}^{m}\Par{D_{x}^{-1}}_{ii}(DD_{x}[h])_{i}^{2} & \underset{\text{(ii)}}{\leq}\sum_{i=1}^{m}\Par{\Sigma_{x}^{-1}}_{ii}(D\Sigma_{x}[h])_{i}^{2}\nonumber \\
 & \underset{\text{Lemma \ref{lem:usefulFactLeverage}-3}}{\leq}4h^{\top}A_{x}^{\top}\Sigma_{x}A_{x}h\leq4\norm h_{g(x)}^{2},\nonumber 
\end{align}
where (i) follows from (4.5) of \cite{anstreicher1997volumetric}
and (ii) holds due to $\Sigma_{x}\preceq D_{x}$. Putting these back
to Lemma~\ref{lem:helper4Diagonal}-1,
\begin{align*}
\norm{g(x)^{-\half}Dg(x)[h]g^{-\half}}_{F}^{2} & \leq4\max_{i}\Par{\frac{\sigma\Par{\sqrt{D_{x}}A_{x}}_{i}}{D_{x,i}}}\cdot\Par{\norm h_{g(x)}^{2}+\sum_{i=1}^{m}(D_{x}^{-1})_{ii}(DD_{x}[h])_{i}^{2}}\\
 & \leq20\sqrt{\frac{m}{n}}\norm h_{g(x)}^{2}.
\end{align*}
Thus, $\tilde{g}(x):=22\sqrt{\frac{m}{n}}g(x)=22\sqrt{\frac{m}{n}}A_{x}^{\top}\Par{\Sigma_{x}+\frac{n}{m}I_{m}}A_{x}$
is strongly self-concordant.

For the symmetry parameter, Lemma~\ref{lem:helper4Diagonal}-2 implies
that for $y\in\dcal_{g}^{1}(x)$
\[
\norm{A_{x}(y-x)}_{\infty}\leq\norm{y-x}_{g(x)}\sqrt{\max_{i\in[m]}\frac{\sigma\Par{\sqrt{D_{x}}A_{x}}_{i}}{(D_{x})_{ii}}}\underset{\text{\eqref{eq:28-1}}}{\leq}\Par{\frac{m}{n}}^{1/4}.
\]
Also, Lemma~\ref{lem:helper4Diagonal}-3 implies that $y$ with $\norm{A_{x}(y-x)}_{\infty}\leq1$
is contained in $\dcal_{g}^{\sqrt{\tr(D_{x})}}(x)$, where
\[
\tr\Par{D_{x}}=\tr\Par{\Sigma_{x}+\frac{n}{m}I_{m}}=\tr\Par{A_{x}(A_{x}^{\top}A_{x})^{-1}A_{x}}+n=\tr\Par{I_{n}}+n=2n.
\]
Therefore, $\tilde{g}(x)=22\sqrt{\frac{m}{n}}g(x)=22\sqrt{\frac{m}{n}}A_{x}^{\top}\Par{\Sigma_{x}+\frac{n}{m}I_{m}}A_{x}$
ensures
\[
\dcal_{\tilde{g}}^{1}(x)\subset K\cap(2x-K)\subset\dcal_{\tilde{g}}^{\sqrt{44(mn)^{1/2}}}(x),
\]
so $\tilde{g}$ is $O(\sqrt{mn})$-symmetric.
\end{proof}

\subsection{Lower trace self-concordance \label{subsec:lowerSC-hybrid}}

We demonstrate that the matrix functions $g_{1}\in\R^{n\times n}$
and $g_{2}=44\sqrt{\frac{m}{n}}A_{x}^{\top}(\Sigma_{x}+\frac{n}{m}I)A_{x}$
satisfy
\[
\tr\Par{(g_{1}+g_{2})^{-1}D^{2}g_{2}[h,h]}\geq-\norm h_{g_{2}}^{2}\quad\text{for }h\in\Rn.
\]
Later, we set $g_{1}=-2n\hess\log\det$ so that $g=g_{1}+g_{2}$ is
equal to the metric in (\ref{eq:metricHybrid}), proving Theorem~\ref{thm:hybridPSD}
via our framework.

Let $\theta_{1}(x):=A_{x}^{\top}\Sigma_{x}A_{x}$, $\theta_{2}(x):=A_{x}^{\top}A_{x}$,
and $\Gamma_{x}:=\Diag\Par{A_{x}g(x)^{-1}A_{x}^{\top}}\in\R^{m\times m}$.
We define $s_{x,h}:=A_{x}$ and $S_{x,h}:=\Diag(s_{x,h})$. Recall
that for a matrix function $g(x)$ we use $g_{x,h}'$ and $g_{x,h}''$
to denote $Dg(x)[h]$ and $D^{2}g(x)[h,h]$.

\begin{restatable}{propre}{propCalculusLeverage} \label{prop:calculusLeverage}
For $x,h\in\Rn$, let $P_{x}=A_{x}(A_{x}^{\top}A_{x})^{-1}A_{x}^{\top}$,
$\Sigma_{x}=\Diag(P_{x})$, and $\Lambda_{x}=\Sigma_{x}-P_{x}^{(2)}$.
Denote $\theta_{1}(x):=A_{x}^{\top}\Sigma_{x}A_{x}$ and $\theta_{2}(x):=A_{x}^{\top}A_{x}$.
\begin{itemize}
\item $\Sigma_{x,h}'=-2\Diag\Par{\Lambda_{x}s_{x,h}}$.
\item $P_{x,h}'=-P_{x}S_{x,h}-S_{x,h}P_{x}+2P_{x}S_{x,h}P_{x}$.
\item $\Lambda_{x,h}'=-2\Diag(\Lambda_{x}s_{x,h})+2P_{x}\circ P_{x}S_{x,h}+2S_{x,h}P_{x}\circ P_{x}-2(P_{x}S_{x,h}P_{x})\circ P_{x}-2P_{x}\circ(P_{x}S_{x,h}P_{x})$.
\item $\Sigma_{x,h}''=6S_{x,h}\Sigma_{x}S_{x,h}+8\Diag\Par{P_{x}S_{x,h}P_{x}S_{x,h}P_{x}}-6\Diag\Par{P_{x}S_{x,h}^{2}P_{x}}-8\Diag\Par{S_{x,h}P_{x}S_{x,h}P_{x}}$.
\item $D\theta_{1}(x)[h]=-2A_{x}^{\top}\Sigma_{x}S_{x,h}A_{x}+A_{x}^{\top}D\Sigma_{x}[h]A_{x}$.
\item $D^{2}\theta_{1}(x)[h,h]=6A_{x}^{\top}S_{x,h}\Sigma_{x}S_{x,h}A_{x}-4A_{x}^{\top}D\Sigma_{x}[h]S_{x,h}A_{x}+A_{x}^{\top}D^{2}\Sigma_{x}[h,h]A_{x}$.
\item $D\theta_{2}(x)[h]=-2A_{x}^{\top}S_{x,h}A_{x}$ and $D^{2}\theta_{2}(x)[h,h]=6A_{x}^{\top}S_{x,h}^{2}A_{x}$.
\end{itemize}
\end{restatable}

We defer the proof to Appendix~\ref{app:subsec:derivativeMatrices}.
\begin{lem}
\label{lem:HybridGammaNorm} $\norm{\Gamma_{x}}_{\infty}\leq\frac{1}{44}$. 
\end{lem}

\begin{proof}
Note that $0\preceq\Gamma_{x}=\Diag\Par{A_{x}g^{-1}A_{x}^{\top}}\preceq\Diag\Par{A_{x}g_{2}^{-1}A_{x}^{\top}}$.
For $\og_{2}:=\theta_{1}+\frac{n}{m}\theta_{2}=\frac{1}{44}\sqrt{\frac{n}{m}}g_{2}$,
we have
\[
\norm{\Diag\Par{A_{x}\og_{2}^{-1}A_{x}^{\top}}}_{\infty}=\max_{i\in[m]}\frac{\sigma\Par{\sqrt{\Sigma_{x}+\frac{n}{m}I}A_{x}}_{i}}{\Par{\Sigma_{x}+\frac{n}{m}I}_{ii}}\underset{\text{\eqref{eq:28-1}}}{\leq}\sqrt{\frac{m}{n}},
\]
and thus
\begin{align*}
\norm{\Gamma_{x}}_{\infty} & \leq\norm{\Diag\Par{A_{x}g_{2}^{-1}A_{x}^{\top}}}_{\infty}=\frac{1}{44}\sqrt{\frac{n}{m}}\norm{\Diag\Par{A_{x}\og_{2}^{-1}A_{x}^{\top}}}_{\infty}\leq\frac{1}{44}.\qedhere
\end{align*}
\end{proof}
\begin{lem}
\label{lem:hybridLowerSCTrace} $\tr\Par{g^{-1}D^{2}g_{2}(x)[h,h]}\geq-\half\norm h_{g_{2}}^{2}$.
\end{lem}

\begin{proof}
As $D^{2}\theta_{2}(x)[h,h]\succeq0$ by Claim~\ref{claim:diffLogBarrier},
we have $\tr\Par{g^{-1}D^{2}\theta_{2}(x)[h,h]}=\tr\Par{g^{-\half}D^{2}\theta_{2}(x)[h,h]g^{-\half}}\geq0$.
For $\theta_{1}$, Proposition~\ref{prop:calculusLeverage}-6 leads
to $D^{2}\theta_{1}[h,h]\succeq-4A_{x}^{\top}\Sigma_{x,h}'S_{x,h}A_{x}+A_{x}^{\top}\Sigma_{x,h}''A_{x}$,
so it suffices to provide
\begin{align*}
\text{Upper bound:} & \ \tr\Par{g^{-1}A_{x}^{\top}\Sigma_{x,h}'S_{x,h}A_{x}}=\tr\Par{A_{x}g^{-1}A_{x}^{\top}\Sigma_{x,h}'S_{x,h}}\underset{\text{(i)}}{=}\tr\Par{\Gamma_{x}\Sigma_{x,h}'S_{x,h}},\\
\text{Lower bound:} & \ \tr\Par{g^{-1}A_{x}^{\top}\Sigma_{x,h}''A_{x}}=\tr\Par{A_{x}g^{-1}A_{x}^{\top}\Sigma_{x,h}''}\underset{\text{(i)}}{=}\tr\Par{\Gamma_{x}\Sigma_{x,h}''},
\end{align*}
where in (i) we used $\tr\Par{AD}=\tr\Par{\Diag(A)D}$ for a diagonal
matrix $D$.

For the upper bound, as diagonal matrices commute,
\begin{align*}
\tr\Par{\Gamma_{x}\Sigma_{x,h}'S_{x,h}} & \leq\norm{\Gamma_{x}}_{\infty}\tr\Par{\Par{S_{x,h}\Sigma_{x,h}'^{2}S_{x,h}}^{\half}}\\
 & =\norm{\Gamma_{x}}_{\infty}\tr\Par{\Par{\Sigma_{x,h}'\Sigma_{x}^{-1}\Sigma_{x,h}'S_{x,h}\Sigma_{x}S_{x,h}}^{\half}}\\
 & \underset{\text{(i)}}{=}\norm{\Gamma_{x}}_{\infty}\tr\Par{\Par{\Sigma_{x,h}'\Sigma_{x}^{-1}\Sigma_{x,h}'}^{\half}\Par{S_{x,h}\Sigma_{x}S_{x,h}}^{\half}}\\
 & \underset{\text{(ii)}}{\leq}\norm{\Gamma_{x}}_{\infty}\sqrt{\tr\Par{\Sigma_{x,h}'\Sigma_{x}^{-1}\Sigma_{x,h}'}}\sqrt{\tr\Par{S_{x,h}\Sigma_{x}S_{x,h}}}\\
 & =\norm{\Gamma_{x}}_{\infty}\|\Sigma_{x}^{-1}\underbrace{\sigma_{x,h}'}_{:=\diag(D\Sigma_{x}[h])}\|_{\Sigma_{x}}\norm h_{\theta_{1}}\\
 & \underset{\text{(iii)}}{\leq}2\norm{\Gamma_{x}}_{\infty}\norm h_{\theta_{1}}^{2},
\end{align*}
where (i) holds since both $\Sigma_{x,h}'\Sigma_{x}^{-1}\Sigma_{x,h}'$
and $S_{x,h}\Sigma_{x}S_{x,h}$ are PD diagonal matrices, (ii) follows
from the Cauchy-Schwartz inequality, and we used in (iii) $\norm{\Sigma_{x}^{-1}\sigma_{x,h}'}_{\Sigma_{x}}\leq2\norm h_{\theta_{1}}$
(Lemma~\ref{lem:usefulFactLeverage}-3).

For the lower bound, we recall from Proposition~\ref{prop:calculusLeverage}-4
\begin{align*}
\Sigma_{x,h}'' & =6S_{x,h}\Sigma_{x}S_{x,h}+8\Diag\Par{P_{x}S_{x,h}P_{x}S_{x,h}P_{x}}-6\Diag\Par{P_{x}S_{x,h}^{2}P_{x}}-8\Diag\Par{S_{x,h}P_{x}S_{x,h}P_{x}}\\
 & \succeq-6\Diag\Par{P_{x}S_{x,h}^{2}P_{x}}-8\Diag\Par{S_{x,h}P_{x}S_{x,h}P_{x}},
\end{align*}
and thus
\begin{align*}
\tr\Par{\Gamma_{x}\Sigma_{x,h}''} & \geq-6\tr\Par{\Gamma_{x}P_{x}S_{x,h}^{2}P_{x}}-8\tr\Par{\Gamma_{x}S_{x,h}P_{x}S_{x,h}P_{x}}.
\end{align*}
For the first term,
\begin{align}
\tr\Par{\Gamma_{x}P_{x}S_{x,h}^{2}P_{x}} & =\tr\Par{S_{x,h}P_{x}\Gamma_{x}P_{x}S_{x,h}}\leq\norm{\Gamma_{x}}_{\infty}\tr\Par{S_{x,h}P_{x}S_{x,h}}\underset{\text{(i)}}{=}\norm{\Gamma_{x}}_{\infty}s_{x,h}^{\top}\Par{P_{x}\circ I}s_{x,h}\label{eq:trSPS}\\
 & =\norm{\Gamma_{x}}_{\infty}s_{x,h}^{\top}\Sigma_{x}s_{x,h}=\norm{\Gamma_{x}}_{\infty}\norm h_{\theta_{1}}^{2},\nonumber 
\end{align}
where (i) follows from $x^{\top}(A\circ B)y=\tr\Par{\Diag(x)A\Diag(y)B^{\top}}$
(Lemma~\ref{lem:Hadamard}). For the second term,
\begin{align*}
\Abs{\tr\Par{\Gamma_{x}S_{x,h}P_{x}S_{x,h}P_{x}}} & =\Abs{\tr\Par{\Gamma_{x}^{1/2}S_{x,h}P_{x}\cdot S_{x,h}P_{x}\Gamma_{x}^{1/2}}}\\
 & \leq\sqrt{\tr\Par{\Gamma_{x}^{1/2}S_{x,h}P_{x}^{2}S_{x,h}\Gamma_{x}^{1/2}}}\sqrt{\tr\Par{\Gamma_{x}^{1/2}P_{x}S_{x,h}^{2}P_{x}\Gamma_{x}^{1/2}}}\\
 & =\sqrt{\tr\Par{P_{x}S_{x,h}\Gamma_{x}S_{x,h}P_{x}}}\sqrt{\tr\Par{S_{x,h}P_{x}\Gamma_{x}P_{x}S_{x,h}}}\\
 & \leq\norm{\Gamma_{x}}_{\infty}\tr\Par{S_{x,h}P_{x}S_{x,h}}\\
 & =\norm{\Gamma_{x}}_{\infty}\norm h_{\theta_{1}}^{2}.\quad(\text{repeat }(\ref{eq:trSPS}))
\end{align*}
Putting the computations together and using Lemma~\ref{lem:HybridGammaNorm},
\[
\tr\Par{g^{-1}D^{2}\theta_{1}(x)[h,h]}\geq-22\norm{\Gamma_{x}}_{\infty}\norm h_{\theta_{1}}^{2}\geq-\half\norm h_{\theta_{1}}^{2},
\]
and due to $g_{2}=44\sqrt{\frac{m}{n}}\Par{\theta_{1}+\frac{n}{m}\theta_{2}}$,
\[
\tr\Par{g^{-1}D^{2}g_{2}(x)[h,h]}\geq-\half\norm h_{g_{2}}^{2}.\qedhere
\]
\end{proof}
Combining all the lemmas so far together, we prove Theorem~\ref{thm:hybridPSD}.

\thmHybridPSD*
\begin{proof}
As in (\ref{eq:metricHybrid}), we set
\begin{align*}
g(X) & =2\Par{ng_{1}(X)+g_{2}(X)},\ \text{where}\\
g_{1}(X) & =M^{\top}(X\kro X)^{-1}M,\\
g_{2}(X) & =22\sqrt{\frac{m}{n}}M^{\top}A_{X}^{\top}\Par{\Sigma_{X}+\frac{n}{m}I_{m}}A_{X}M.
\end{align*}
Since $ng_{1}$ and $g_{2}$ are strongly self-concordant (see Corollary~\ref{cor:strongSCofLOGDET}
and Lemma~\ref{lem:paramsBarrier}-3), $g$ is also strongly self-concordant
due to Lemma~\ref{lem:sumStrongSC} and $O(n^{2}+\sqrt{mn^{2}})$-symmetric\footnote{Since the dimension is $d$ in the setting of (\ref{eq:PSDcone}),
we should replace $n$ by $d=O(n^{2})$ when applying Lemma \ref{lem:paramsBarrier}.} due to Lemma~\ref{lem:sumSymmetricSC}. Lastly for lower trace self-concordance
of $g$, we recall that $D^{2}g_{1}[H,H]\succeq0$ as checked in (\ref{eq:D4ph1})
and $\tr\Par{g^{-1}D^{2}g_{2}(x)[h,h]}\geq-\half\norm h_{g_{2}}^{2}$
by Lemma~\ref{lem:hybridLowerSCTrace}, so we can use Lemma~\ref{lem:additiveCondition}-2
to obtain lower trace self-concordance. Therefore, Theorem~\ref{thm:generalMixing}
ensures that the $\dw$ with the Vaidya metric $g$ mixes in $\otilde{n^{2}\Par{n^{2}+\sqrt{mn^{2}}}}=\otilde{n^{3}\Par{n+\sqrt{m}}}$
steps.

Now we bound the time for each step of this $\dw$ (Algorithm~\ref{alg:DikinWalk}).
Recall that it requires (1) the update of the leverage scores, (2)
computation of the matrix function induced by the local metric $g$,
(3) the inverse of the matrix function and (4) its determinant. By
Theorem~46 in \cite{lee2019solving} (with $p=2$ and $n\gets d$
therein), the initialization of the leverage scores at the beginning
takes $\otilde{mn^{2\omega}}$ and their updates takes $\otilde{mn^{2(\omega-1)}}$
time. Since (1) takes $\otilde{mn^{2(\omega-1)}}$, (2) takes $\otilde{n^{4}+mn^{2(\omega-1)}}$,
and (3) and (4) take $O(n^{2\omega})$, each iteration runs in $\otilde{n^{2\omega}+mn^{2(\omega-1)}}$
time. Note that even though the initialization of leverage scores
takes $\otilde{mn^{2\omega}}$ time, the amortized per step time complexity
(since the mixing rate is $\otilde{n^{3}(n+\sqrt{m})}$) goes to $\otilde{n^{2\omega}+mn^{2(\omega-1)}}=\otilde{mn^{2(\omega-1)}}$
time.
\end{proof}


\section{Poly-logarithmic dependence on $m$ via the Lewis-weight metric \label{sec:LS-psdSampling}}

The $\dw$ with the Vaidya metric has mixing rate with $\sqrt{m}$-dependence.
A natural question is whether this can be made smaller, potentially
just polylogarithmic in $m$. The $\dw$ on a polytope in $\Rn$ mixes
in $\otilde{n^{2}}$ steps by working with the Lee-Sidford (LS) metric,
$G(x)=(1+p)(1+p^{2})A_{x}^{\top}W_{x}^{1-\frac{2}{p}}A_{x}$ in \cite{laddha2020strong}.
Can we develop an analogous result in the PSD setting (i.e., $\otilde{n^{4}}$
mixing rate) via the LS metric? Unfortunately, it is not straightforward
that the LS metric $G$ is suitable to address the constraints of
$\inner{A_{i},X}\leq b_{i}$ in (\ref{eq:PSDcone}). However, it is
not clear that the LS metric \emph{together with the log-determinant
barrier} has the property that log-determinant of its Hessian is convex.

We find that the $d$-dimensional version of the metric $O^{*}(\sqrt{n})A_{x}^{\top}W_{x}A_{x}$
with $p=O(\log m)$ has a good fit with our analysis framework. To
be precise, we prove Theorem~\ref{thm:LSPSD} that the $\dw$ mixes
in $\otilde{n^{5}}$ steps with the Lewis weights metric defined as
\begin{align}
g(X) & =2\Par{ng_{1}(X)+g_{2}(X)}\label{eq:metricLS}\\
g_{1}(X) & =-\hess_{X}\log\det X=M^{\top}(X\kro X)^{-1}M,\nonumber \\
g_{2}(X) & =c_{1}\Par{\log m}^{c_{2}}\sqrt{n}M^{\top}A_{X}^{\top}W_{X}A_{X}M\ \text{for some constants \ensuremath{c_{1},c_{2}>0},}\nonumber 
\end{align}
where $W_{x}$ is the diagonal matrix with the $\ell_{p}$-Lewis weight
on diagonals and $p=O\Par{\log m}$. We first check strong self-concordance
and symmetry of the Lewis-weight metric in Section~\ref{subsec:sc-and-symm-LS}
and then show lower trace self-concordance in Section~\ref{subsec:lowerSC-LS}.

\subsection{Strong self-concordance and symmetry \label{subsec:sc-and-symm-LS}}

As in the previous section, we work with $x\in\R^{n}$ instead of
$\svec(X)\in\R^{d}=\R^{n(n+1)/2}$ and then replace $n$ by $d$ later
when obtaining results for PSD sampling.

We first state the analogue of Lemma~\ref{lem:usefulFactLeverage}
for the $\ell_{p}$-Lewis weight:
\begin{lem}
[\cite{lee2019solving}] \label{lem:usefulFactLewis} Let $W_{x}=\Diag(w_{x})\in\R^{m\times m}$
be the $\ell_{p}$-Lewis weights of $A_{x}$, $g(x)=A_{x}^{\top}W_{x}A_{x}$
be the Lewis-weight metric, and $h\in\Rn$.
\begin{itemize}
\item \textup{(Lemma 26)} $\max_{i\in[m]}\frac{\sigma\Par{W_{x}^{1/2}A_{x}}_{i}}{\Par{W_{x}}_{ii}}\leq2m^{\frac{2}{p+2}}$.
\item \textup{(Lemma 33)} $\norm{A_{x}h}_{W_{x}}=\norm h_{g(x)}$ and $\norm{A_{x}h}_{\infty}\leq\sqrt{2}m^{\frac{1}{p+2}}\norm h_{g(x)}$.
\item \textup{(Lemma 34)} $\norm{W_{x}^{-1}\diag\Par{W_{x,h}'}}_{W_{x}}\leq p\norm h_{g(x)}$.
\end{itemize}
\end{lem}

We can now show that the Lewis weights metric is strongly self-concordant
and $\onu$-symmetric under appropriate scaling.
\begin{lem}
[Strong self-concordance and symmetry] \label{lem:LSmetricStrongandSymmetry}
Let $g(x)=c_{1}\Par{\log m}^{c_{2}}A_{x}^{\top}W_{x}A_{x}$ be the
Lewis-weight metric with the $\ell_{p}$-Lewis weight $w_{x}$ with
$p=O(\log m)$. There exist constants $c_{1}$ and $c_{2}$ such that
$g$ is strongly self-concordant and $\onu$-symmetric with $\onu=O^{*}\Par n$.
\end{lem}

\begin{proof}
Consider the unscaled version first: $g(x)=A_{x}^{\top}W_{x}A_{x}$.
By Lemma~\ref{lem:helper4Diagonal}-1
\begin{align*}
\norm{g(x)^{-\half}Dg(x)[h]g(x)^{-\half}}_{F} & \leq2\sqrt{\max_{i}\Par{\frac{\sigma\Par{\sqrt{W_{x}}A_{x}}_{i}}{(W_{x})_{ii}}}\cdot\bigg(\norm h_{g(x)}^{2}+\sum_{i=1}^{m}\Par{W_{x}^{-1}}_{ii}\diag\Par{W_{x,h}'}^{2}\bigg)}\\
 & \underset{\text{(i)}}{\leq}2\sqrt{2m^{\frac{2}{p+2}}}\sqrt{\norm h_{g(x)}^{2}+p^{2}\norm h_{g(x)}^{2}}\\
 & \leq\Par{8m^{\frac{2}{p+2}}(1+p^{2})}^{1/2}\norm h_{g(x)},
\end{align*}
where in (i) we used Lemma~\ref{lem:usefulFactLewis}-1 and 3.

For the first part of $\onu$-symmetry, Lemma~\ref{lem:helper4Diagonal}-2
implies that
\[
\max_{h:\norm h_{g(x)}=1}\norm{A_{x}h}_{\infty}=\sqrt{\max_{i\in[m]}\frac{\sigma\Par{\sqrt{W_{x}}A_{x}}_{i}}{W_{x,i}}}\leq\sqrt{2m^{\frac{2}{p+2}}},
\]
 and Lemma~\ref{lem:helper4Diagonal}-3 leads to $K\cap(2x-K)\subset\dcal_{g}^{\sqrt{n}}(x)$
due to
\[
\tr\Par{W_{x}}=\tr\Par{W_{x}^{\half-\frac{1}{p}}A_{x}\Par{A_{x}^{\top}W_{x}^{1-\frac{2}{p}}A_{x}}^{-1}A_{x}^{\top}W_{x}^{\half-\frac{1}{p}}}=\tr\Par{A_{x}^{\top}W_{x}^{1-\frac{2}{p}}A_{x}\Par{A_{x}^{\top}W_{x}^{1-\frac{2}{p}}A_{x}}^{-1}}=n.
\]
Therefore, $4pm^{\frac{1}{p+2}}A_{x}^{\top}W_{x}A_{x}$ is strongly
self-concordant with $O\Par{nm^{\frac{1}{p+2}}}$-symmetric by Lemma~\ref{lem:symmforPolytope}.
By setting $p=O(\log m)$, the claim follows.
\end{proof}

\subsection{Lower trace self-concordance \label{subsec:lowerSC-LS}}

As we did in the previous section, we work with a polytope in $\Rn$
and then adapt our results to the PSD setting to make arguments concise.
For $\theta(x):=A_{x}^{\top}W_{x}A_{x}$ (i.e., the unscaled version
of $g_{2}$), we write $g_{2}=c\cdot\theta$ for a constant $c$,
which will be set to $c_{1}(\log m)^{c_{2}}\sqrt{n}$ for some constants
$c_{1},c_{2}>0$ later. For a given metric $g_{1}$, let $g(x):=g_{1}(x)+c\cdot\theta(x)$.
We begin with a directional derivative of the $\ell_{p}$-Lewis weight
of $A_{x}$.
\begin{lem}
[\cite{gatmiry2023sampling}, Lemma 2.2] The directional derivative
of the $\ell_{p}$-Lewis weight $W_{x}$ in direction $h\in\Rn$ is
\[
W_{x,h}':=DW_{x}[h]=-2\Diag\Par{\Lambda_{x}G_{x}^{-1}W_{x}s_{x,h}},
\]
where $\Lambda_{x}=W_{x}-P_{x}^{(2)}$ and $G_{x}=W_{x}-\Par{1-\frac{2}{p}}\Lambda_{x}$.
\end{lem}

We recall that these matrices satisfy
\begin{align}
P_{x}^{(2)}\preceq W_{x}\preceq I,\label{eq:lewisBasic-PWI}\\
\Lambda_{x}\preceq W_{x},\label{eq:lewisBasic-LW}\\
\frac{2}{p}W_{x}\preceq G_{x}\preceq W_{x}, & \text{ which implies }W_{x}^{-1}\preceq G_{x}^{-1}\preceq\frac{p}{2}W_{x}^{-1}\text{ and }I\preceq W_{x}^{\half}G_{x}^{-1}W_{x}^{\half}\preceq\frac{p}{2}I.\label{eq:lewisBasic-WGW}
\end{align}
We first bound the largest diagonal entry of $\Gamma_{x}=\Diag(A_{x}g(x)^{-1}A_{x}^{\top})$.
\begin{lem}
\label{lem:GammaNormLSMetric}$\norm{\Gamma_{x}}_{\infty}\leq2c^{-1}m^{\frac{2}{p+2}}$.
\end{lem}

\begin{proof}
Note that $0\preceq\Gamma_{x}=\Diag\Par{A_{x}g^{-1}A_{x}^{\top}}\preceq c^{-1}\Diag\Par{A_{x}\theta^{-1}A_{x}^{\top}}$.
By Lemma~\ref{lem:usefulFactLewis}-1
\[
\norm{\Diag\Par{A_{x}\theta^{-1}A_{x}^{\top}}}_{\infty}=\max_{i\in[m]}\frac{\sigma\Par{W_{x}^{1/2}A_{x}}_{i}}{\Par{W_{x}}_{ii}}\leq2m^{\frac{2}{p+2}}.\qedhere
\]
\end{proof}
\begin{lem}
\label{lem:LSLowerSCTrace} $\tr\Par{g^{-1}D^{2}g_{2}(x)[h,h]}\geq-\norm h_{g_{2}}^{2}$,
where $g_{2}(x)=c\theta(x)=cA_{x}^{\top}W_{x}A_{x}$ with $c=c_{1}(\log m)^{c_{2}}\sqrt{n}$
for some constants $c_{1},c_{2}>0$.
\end{lem}

\begin{proof}
Repeating the same calculus done for $A_{x}^{\top}\Sigma_{x}A_{x}$
in Proposition~\ref{prop:calculusLeverage}, we can obtain
\begin{align*}
D^{2}\theta[h,h] & =6A_{x}^{\top}S_{x,h}W_{x}S_{x,h}A_{x}-4A_{x}^{\top}W_{x,h}'S_{x,h}A_{x}+A_{x}^{\top}W_{x,h}''A_{x}\\
 & \succeq-4A_{x}^{\top}W_{x,h}'S_{x,h}A_{x}+A_{x}^{\top}W_{x,h}''A_{x},
\end{align*}
and thus for $\Gamma_{x}=\Diag\Par{A_{x}g^{-1}A_{x}^{\top}}$
\begin{align*}
\tr\Par{g^{-1}D^{2}\theta[h,h]} & =\tr\Par{g^{-\half}D^{2}\theta[h,h]g^{-\half}}\\
 & \geq\tr\Par{g^{-1}A_{x}^{\top}\Par{W_{x,h}''-4W_{x,h}'S_{x,h}}A_{x}}\\
 & \geq\tr\Par{A_{x}g^{-1}A_{x}^{\top}\Par{W_{x,h}''-4W_{x,h}'S_{x,h}}}\\
 & =\tr\Par{\Gamma_{x}\Par{W_{x,h}''-4W_{x,h}'S_{x,h}}}\\
 & =-4\tr\Par{\Gamma_{x}W_{x,h}'S_{x,h}}+\tr\Par{\Gamma_{x}W_{x,h}''}.
\end{align*}
For the first term, since diagonal matrices commute,
\begin{align}
\Abs{\tr\Par{\Gamma_{x}W_{x,h}'S_{x,h}}} & \leq\norm{\Gamma_{x}}_{\infty}\tr\Par{\sqrt{S_{x,h}W_{x,h}'^{2}S_{x,h}}}\nonumber \\
 & =\norm{\Gamma_{x}}_{\infty}\tr\Par{\sqrt{W_{x,h}'W_{x}^{-1}W_{x,h}'S_{x,h}W_{x}S_{x,h}}}\nonumber \\
 & =\norm{\Gamma_{x}}_{\infty}\tr\Par{\sqrt{W_{x,h}'W_{x}^{-1}W_{x,h}'}\sqrt{S_{x,h}W_{x}S_{x,h}}}\nonumber \\
 & \underset{\text{(i)}}{\leq}\norm{\Gamma_{x}}_{\infty}\sqrt{\tr\Par{W_{x,h}'W_{x}^{-1}W_{x,h}'}}\sqrt{\tr\Par{S_{x,h}W_{x}S_{x,h}}}\nonumber \\
 & =\norm{\Gamma_{x}}_{\infty}\norm{W_{x}^{-1}w_{x,h}'}_{W_{x}}\norm h_{\theta}\nonumber \\
 & \underset{\text{(ii)}}{\leq}p\norm{\Gamma_{x}}_{\infty}\norm h_{\theta}^{2},\label{eq:boundonFirst}
\end{align}
where in (i) we used the Cauchy-Schwartz inequality, and in (ii) $\norm{W_{x}^{-1}w_{x,h}'}_{W_{x}}\leq p\norm h_{\theta}$
(Lemma~\ref{lem:usefulFactLewis}-3).

For the second term $\tr\Par{\Gamma_{x}W_{x,h}''}$, let us compute
the second-order directional derivate of $W_{x}$ in direction $h$:
\begin{align*}
W_{x,h}' & =-2\Diag(\Lambda_{x}G_{x}^{-1}W_{x}s_{x,h}),\\
W_{x,h}'' & =-2\Diag\Par{\Lambda_{x,h}'G_{x}^{-1}W_{x}s_{x,h}-\Lambda_{x}G_{x}^{-1}G_{x,h}'G_{x}^{-1}W_{x}s_{x,h}+\Lambda_{x}G_{x}^{-1}W_{x,h}'s_{x,h}-\Lambda_{x}G_{x}^{-1}W_{x}S_{x,h}s_{x,h}},
\end{align*}
where $\Lambda_{x,h}':=D\Lambda_{x}[h]$ and $G_{x,h}':=DG_{x}[h]$.
 Thus,
\begin{align}
 & \tr\Par{\Gamma_{x}W_{x,h}''}\nonumber \\
 & =-2\tr\Par{\Gamma_{x}\Diag\Par{\Lambda_{x,h}'G_{x}^{-1}W_{x}s_{x,h}-\Lambda_{x}G_{x}^{-1}G_{x,h}'G_{x}^{-1}W_{x}s_{x,h}+\Lambda_{x}G_{x}^{-1}W_{x,h}'s_{x,h}-\Lambda_{x}G_{x}^{-1}W_{x}S_{x,h}s_{x,h}}}\nonumber \\
 & =-2\tr\bigg(\Gamma_{x}\Diag\bigg(\underbrace{\Lambda_{x,h}'G_{x}^{-1}W_{x}s_{x,h}}_{\text{I}}\bigg)\bigg)+2\tr\bigg(\Gamma_{x}\Diag\bigg(\underbrace{\Lambda_{x}G_{x}^{-1}G_{x,h}'G_{x}^{-1}W_{x}s_{x,h}}_{\text{II}}\bigg)\bigg)\nonumber \\
 & \qquad-2\tr\bigg(\Gamma_{x}\Diag\bigg(\underbrace{\Lambda_{x}G_{x}^{-1}W_{x,h}'s_{x,h}}_{\text{III}}\bigg)\bigg)+2\tr\bigg(\Gamma_{x}\Diag\bigg(\underbrace{\Lambda_{x}G_{x}^{-1}W_{x}S_{x,h}s_{x,h}}_{\text{IV}}\bigg)\bigg).\label{eq:trGamma}
\end{align}
Each term is of the form $\tr\Par{\Gamma_{x}\Diag(v)}$ for a vector
$v\in\R^{m}$, and thus
\begin{align}
\Abs{\tr\Par{\Gamma_{x}\Diag(v)}} & =\Abs{\tr\Par{\Gamma_{x}W_{x}^{\half}W_{x}^{-\half}\Diag(v)}}\nonumber \\
 & \leq\sqrt{\tr\Par{W_{x}^{\half}\Gamma_{x}^{2}W_{x}^{\half}}}\sqrt{\tr\Par{\Diag(v)W_{x}^{-1}\Diag(v)}}\nonumber \\
 & \leq\norm{\Gamma_{x}}_{\infty}\sqrt{\tr\Par{W_{x}}}\norm v_{W_{x}^{-1}}\nonumber \\
 & =\sqrt{n}\norm{\Gamma_{x}}_{\infty}\norm v_{W_{x}^{-1}},\label{eq:trGammaBasic}
\end{align}
where we used $\tr(W_{x})=n$ in the last equality. 

Now let us state Calabi-type estimates on directional derivatives
of $\Lambda_{x},G_{x},W_{x}$, and then bound $\norm v_{W_{x}^{-1}}$
for each term (I$\sim$IV). 
\begin{lem}
[\cite{gatmiry2023sampling}] \label{lem:calabi} Let $Ax\leq b$
for $A\in\R^{m\times n}$ and $b\in\R^{m}$, and $W_{x}$ the $\ell_{p}$-Lewis
weight of $A_{x}$ with $p=O(\log m)$. There exists universal constants
$c_{1},c_{2}>0$ such that 
\begin{itemize}
\item \textup{(Lemma D.4)} $-c_{1}(\log m)^{c_{2}}\norm{s_{x,h}}_{\infty}W_{x}\preceq G_{x,h}'\preceq c_{1}(\log m)^{c_{2}}\norm{s_{x,h}}_{\infty}W_{x}$.
\item \textup{(Lemma D.4)} $-c_{1}(\log m)^{c_{2}}\norm{s_{x,h}}_{\infty}W_{x}\preceq\Lambda_{x,h}'\preceq c_{1}(\log m)^{c_{2}}\norm{s_{x,h}}_{\infty}W_{x}$.
\item \textup{(Lemma D.13)} $-c_{1}(\log m)^{c_{2}}\norm{s_{x,h}}_{\infty}W_{x}\preceq W_{x,h}'\preceq c_{1}(\log m)^{c_{2}}\norm{s_{x,h}}_{\infty}W_{x}$. 
\end{itemize}
\end{lem}

Using these estimates, we can bound the local norm of the term I.
In our calculation, $\lesssim$ hides universal constants and poly-logarithmic
factors on $m$:
\begin{align*}
\norm{\text{I}}_{W_{x}^{-1}}^{2} & =s_{x,h}^{\top}W_{x}G_{x}^{-1}\Lambda_{x,h}'W_{x}^{-1}\Lambda_{x,h}'G_{x}^{-1}W_{x}s_{x,h}\\
 & =s_{x,h}^{\top}W_{x}G_{x}^{-1}W_{x}^{\half}\bigg(\underbrace{W_{x}^{-\half}\Lambda_{x,h}'W_{x}^{-\half}}_{\precsim\norm{s_{x,h}}_{\infty}I\ \text{(Lemma \ref{lem:calabi}-2)}}\bigg)^{2}W_{x}^{\half}G_{x}^{-1}W_{x}s_{x,h}\\
 & \lesssim\norm{s_{x,h}}_{\infty}^{2}s_{x,h}^{\top}W_{x}G_{x}^{-1}W_{x}G_{x}^{-1}W_{x}s_{x,h}\\
 & =\norm{s_{x,h}}_{\infty}^{2}s_{x,h}^{\top}W_{x}^{\half}\bigg(\underbrace{W_{x}^{\half}G_{x}^{-1}W_{x}^{\half}}_{\preceq\frac{p}{2}I\ \text{\eqref{eq:lewisBasic-WGW}}}\bigg)^{2}W_{x}^{\half}s_{x,h}\\
 & \leq p^{2}\norm{s_{x,h}}_{\infty}^{2}s_{x,h}^{\top}W_{x}s_{x,h}=p^{2}\norm{s_{x,h}}_{\infty}^{2}\norm h_{\theta}^{2}\\
 & \leq2p^{2}m^{\frac{2}{p+2}}\norm h_{\theta}^{4},
\end{align*}
where the last line follows from Lemma~\ref{lem:usefulFactLewis}-2.

For the second term,
\begin{align*}
\norm{\text{II}}_{W_{x}^{-1}}^{2} & =s_{x,h}^{\top}W_{x}G_{x}^{-1}G_{x,h}'G_{x}^{-1}\Lambda_{x}W_{x}^{-1}\Lambda_{x}G_{x}^{-1}G_{x,h}'G_{x}^{-1}W_{x}s_{x,h}\\
 & =s_{x,h}^{\top}W_{x}G_{x}^{-1}G_{x,h}'G_{x}^{-1}W_{x}^{\half}\bigg(\underbrace{W_{x}^{-\half}\Lambda_{x}W_{x}^{-\half}}_{\preceq I\ \text{\eqref{eq:lewisBasic-LW}}}\bigg)^{2}W_{x}^{\half}G_{x}^{-1}G_{x,h}'G_{x}^{-1}W_{x}s_{x,h}\\
 & \leq s_{x,h}^{\top}W_{x}G_{x}^{-1}G_{x,h}'\underbrace{G_{x}^{-1}W_{x}G_{x}^{-1}}_{\preceq\frac{p^{2}}{4}W_{x}^{-1}\ \text{\eqref{eq:lewisBasic-WGW}}}G_{x,h}'G_{x}^{-1}W_{x}s_{x,h}\\
 & \leq p^{2}s_{x,h}^{\top}W_{x}G_{x}^{-1}G_{x,h}'W_{x}^{-1}G_{x,h}'G_{x}^{-1}W_{x}s_{x,h}\\
 & =p^{2}s_{x,h}^{\top}W_{x}G_{x}^{-1}W_{x}^{\half}\bigg(\underbrace{W_{x}^{-\half}G_{x,h}'W_{x}^{-\half}}_{\precsim\norm{s_{x,h}}_{\infty}I\ \text{(Lemma \ref{lem:calabi}-1)}}\bigg)^{2}W_{x}^{\half}G_{x}^{-1}W_{x}s_{x,h}\\
 & \lesssim p^{2}\norm{s_{x,h}}_{\infty}^{2}s_{x,h}^{\top}W_{x}G_{x}^{-1}W_{x}G_{x}^{-1}W_{x}s_{x,h}\\
 & =p^{2}\norm{s_{x,h}}_{\infty}^{2}s_{x,h}^{\top}W_{x}^{\half}\bigg(\underbrace{W_{x}^{\half}G_{x}^{-1}W_{x}^{\half}}_{\preceq\frac{p}{2}I\ \text{\eqref{eq:lewisBasic-WGW}}}\bigg)^{2}W_{x}^{\half}s_{x,h}\\
 & \leq p^{4}\norm{s_{x,h}}_{\infty}^{2}\norm h_{\theta}^{2}\\
 & \leq2p^{4}m^{\frac{2}{p+2}}\norm h_{\theta}^{4},
\end{align*}
where we used Lemma~\ref{lem:usefulFactLewis}-2 in the last line.

For the third term,
\begin{align*}
\norm{\text{III}}_{W_{x}^{-1}}^{2} & =s_{x,h}^{\top}W_{x,h}'G_{x}^{-1}\Lambda_{x}W_{x}^{-1}\Lambda_{x}G_{x}^{-1}W_{x,h}'s_{x,h}\\
 & =s_{x,h}^{\top}W_{x}G_{x}^{-1}G_{x,h}'G_{x}^{-1}W_{x}^{\half}\bigg(\underbrace{W_{x}^{-\half}\Lambda_{x}W_{x}^{-\half}}_{\preceq I\ \eqref{eq:lewisBasic-LW}}\bigg)^{2}W_{x}^{\half}G_{x}^{-1}G_{x,h}'G_{x}^{-1}W_{x}s_{x,h}\\
 & \leq s_{x,h}^{\top}W_{x,h}'\underbrace{G_{x}^{-1}W_{x}G_{x}^{-1}}_{\preceq\frac{p^{2}}{4}W_{x}^{-1}\ \text{\eqref{eq:lewisBasic-WGW}}}W_{x,h}'s_{x,h}\\
 & \leq p^{2}s_{x,h}^{\top}W_{x,h}'W_{x}^{-1}W_{x,h}'s_{x,h}\\
 & =p^{2}s_{x,h}^{\top}W_{x}^{\half}\bigg(\underbrace{W_{x}^{-\half}W_{x,h}'W_{x}^{-\half}}_{\precsim\norm{s_{x,h}}_{\infty}I\ \text{(Lemma \ref{lem:calabi}-3)}}\bigg)^{2}W_{x}^{\half}s_{x,h}\\
 & \lesssim p^{2}\norm{s_{x,h}}_{\infty}^{2}s_{x,h}^{\top}W_{x}s_{x,h}\\
 & \leq p^{2}m^{\frac{2}{p+2}}\norm h_{\theta}^{4},
\end{align*}
where we used Lemma~\ref{lem:usefulFactLewis}-2 in the last line. 

Let us bound the last term:
\begin{align*}
\norm{\text{IV}}_{W_{x}^{-1}}^{2} & =s_{x,h}^{\top}S_{x,h}W_{x}G_{x}^{-1}\underbrace{\Lambda_{x}W_{x}^{-1}\Lambda_{x}}_{\preceq W_{x}\ \text{\eqref{eq:lewisBasic-LW}}}G_{x}^{-1}W_{x}S_{x,h}s_{x,h}\\
 & \leq s_{x,h}^{\top}S_{x,h}W_{x}\underbrace{G_{x}^{-1}W_{x}G_{x}^{-1}}_{\preceq\frac{p^{2}}{4}W_{x}^{-1}\ \text{\eqref{eq:lewisBasic-WGW}}}W_{x}S_{x,h}s_{x,h}\\
 & \leq p^{2}s_{x,h}^{\top}S_{x,h}W_{x}S_{x,h}s_{x,h}\\
 & =p^{2}s_{x,h}^{\top}W_{x}^{\half}S_{x,h}^{2}W_{x}^{\half}s_{x,h}\\
 & \leq p^{2}\norm{s_{x,h}}_{\infty}^{2}\norm h_{\theta}^{2}\\
 & \leq p^{2}m^{\frac{2}{p+2}}\norm h_{\theta}^{4},
\end{align*}
where we used Lemma~\ref{lem:usefulFactLewis}-2 in the last line. 

Combining these bounds, (\ref{eq:trGamma}), and (\ref{eq:trGammaBasic})
for $p=O(\log m)$, we obtain
\begin{align*}
\Abs{\tr\Par{\Gamma_{x}W_{x,h}''}} & \lesssim\sqrt{n}\norm{\Gamma_{x}}_{\infty}\norm h_{\theta}^{2}
\end{align*}
Along with the bound in (\ref{eq:boundonFirst}), we conclude that
\begin{align*}
\tr\Par{g^{-1}D^{2}\theta[h,h]} & \gtrsim-p\norm{\Gamma_{x}}_{\infty}\norm h_{\theta}^{2}-\sqrt{n}\norm{\Gamma_{x}}_{\infty}\norm h_{\theta}^{2}\\
 & \gtrsim-\sqrt{n}\norm{\Gamma_{x}}_{\infty}\norm h_{\theta}^{2}\\
 & \gtrsim-c^{-1}\sqrt{n}\norm h_{\theta}^{2},
\end{align*}
where the last line follows from Lemma~\ref{lem:GammaNormLSMetric}.
This implies that there exists some positive constants $d_{1}$ and
$d_{2}$ such that $\tr\Par{g^{-1}D^{2}\theta[h,h]}\geq-c^{-1}d_{1}\Par{\log m}^{d_{2}}\sqrt{n}\norm h_{\theta}^{2}$,
which implies
\[
\tr\Par{g^{-1}D^{2}g_{2}[h,h]}\geq-c^{-1}d_{1}\Par{\log m}^{d_{2}}\sqrt{n}\norm h_{g_{2}}^{2}.
\]
By taking $c=d_{1}(\log m)^{d_{2}}\sqrt{n}$, the metric $g_{2}=c\theta=d_{1}(\log m)^{d_{2}}\sqrt{n}A_{x}^{\top}W_{x}A_{x}$
is lower trace self-concordant.
\end{proof}
We put together the lemmas above to prove Theorem~\ref{thm:LSPSD}.

\thmLSPSD*
\begin{proof}
As in (\ref{eq:metricLS}), we set
\begin{align*}
g(X) & =2\Par{ng_{1}(X)+g_{2}(X)},\ \text{where}\\
g_{1}(X) & =-\hess_{X}\log\det X=M^{\top}(X\kro X)^{-1}M,\\
g_{2}(X) & =c_{1}\Par{\log m}^{c_{2}}\sqrt{n}M^{\top}A_{X}^{\top}W_{X}A_{X}M\ \text{for some constants \ensuremath{c_{1},c_{2}>0}.}
\end{align*}
Since $ng_{1}$ and $g_{2}$ are strongly self-concordant (see Corollary~\ref{cor:strongSCofLOGDET}
and Lemma~\ref{lem:LSmetricStrongandSymmetry}), $g$ is also strongly
self-concordant due to Lemma~\ref{lem:sumStrongSC} and $O^{*}\Par{n^{3}}$-symmetric\footnote{Since the dimension is $d$ in the setting of (\ref{eq:PSDcone}),
we should replace $n$ by $d=O(n^{2})$ when applying Lemma \ref{lem:paramsBarrier}.} due to Lemma~\ref{lem:symmScaling} and Lemma~\ref{lem:sumSymmetricSC}.

In Lemma~\ref{lem:LSLowerSCTrace}, we set $g_{1}=-2n\hess\log\det$
and note that the proof of Lemma~\ref{lem:LSLowerSCTrace} with $g_{2}=2c_{1}\Par{\log m}^{c_{2}}\sqrt{n}M^{\top}A_{X}^{\top}W_{X}A_{X}M$
leads to $\tr\Par{g^{-1}D^{2}(2g_{2})[h,h]}\geq-\half\norm h_{2g_{2}}^{2}$.
Together with (\ref{eq:D4ph1}) ($D^{2}g_{1}[H,H]\succeq0$ for $H\in\S^{n}$),
Lemma~\ref{lem:additiveCondition} implies that $g$ is lower trace
self-concordant. Therefore, Theorem~\ref{thm:generalMixing} ensures
that the $\dw$ with $g$ mixes in $\otilde{n^{5}}$ steps. Since
the initialization and update of the Lewis weight takes $\otilde{mn^{2\omega}}$
and $\otilde{mn^{2(\omega-1)}}$ time (Theorem~46 in \cite{lee2019solving}),
the same implementation with Theorem~\ref{thm:hybridPSD} also has
the time complexity of $\otilde{mn^{2(\omega-1)}}$.
\end{proof}

\paragraph{Handling approximate Lewis weights.}

In the implementation of the $\dw$ with the Lewis weights metric,
we use an approximation algorithm presented in \cite{lee2019solving}
for computing and updating the Lewis weight, which ensures 
\[
(1-\delta)\wtilde_{X}\preceq W_{X}\preceq(1+\delta)\wt W_{X}
\]
for the approximate Lewis weights $\wtilde_{X}$ and a target accuracy
parameter $\delta$ (note that the initialization and update times
of the Lewis weight above hide poly-logarithmic dependence on $\log(1/\delta)$).
Strictly speaking, we should check that these approximate Lewis weights
do not affect the theoretical guarantees above.

To see this, let us define
\begin{align*}
\widetilde{g}(X) & =2\Par{ng_{1}(X)+\widetilde{g_{2}}(X)},\ \text{where}\\
g_{1}(X) & =-\hess_{X}\log\det X=M^{\top}(X\kro X)^{-1}M,\\
\wt g_{2}(X) & =c_{1}\Par{\log m}^{c_{2}}\sqrt{n}M^{\top}A_{X}^{\top}\widetilde{W}_{X}A_{X}M\ \text{for some constants \ensuremath{c_{1},c_{2}>0}.}
\end{align*}
First of all, the $\dw$ with $\widetilde{g}$ still converges to
the uniform distribution over $K$, since the approximation algorithm
in \cite{lee2019solving} is deterministic and thus the condition
of detailed balance still holds under the acceptance probability of
$\min\Par{1,\sqrt{\frac{\det\tilde{g}(Y)}{\det\tilde{g}(X)}}}$. For
$\widetilde{P}_{X}$ the one-step distribution of the $\dw$ started
at $X$ with $\widetilde{g}$, the triangle inequality leads to 
\[
\dtv(\widetilde{P}_{X},\widetilde{P}_{Y})\leq\dtv(\widetilde{P}_{X},P_{X})+\dtv(P_{X},P_{Y})+\dtv(\widetilde{P}_{Y},P_{Y}).
\]
Since the second term is bounded by a small constant due to Lemma~\ref{lem:one-step},
it suffices to bound $\dtv(\widetilde{P}_{X},P_{X})$ (and $\dtv(\widetilde{P}_{Y},P_{Y})$
similarly) for $\delta=1/\text{poly}(n)$.
\begin{lem}
[One-step coupling] For a convex body $K\subset\Rn$, let $g\in\R^{n\times n}$
be strongly and lower trace self-concordant on $K$. For step size
$r=\frac{1}{2^{12}}$ and $\delta=\frac{10^{-10}}{n^{2}}$, we have
$\dtv(P_{X},\wt P_{X})\leq1/10$ for the transition kernels $P_{X}$
and $\wt P_{X}$ of the $\dw$ with the metrics $g$ and $\wt g$,
respectively.
\end{lem}

\begin{proof}
Let $r_{X}$ and $\wt r_{X}$ be the rejection probabilities of the
one-step of the $\dw$ with $g$ and $\wt g$ started at $X$. Then
the triangle inequality results in
\begin{align*}
\dtv(P_{X},\wt P_{X}) & \leq\dtv\Par{P_{X},U_{g}(X)}+\dtv\Par{U_{g}(X),U_{\tilde{g}}(X)}+\dtv\Par{U_{\tilde{g}}(X),\wt P_{X}}\\
 & =\underbrace{\dtv\Par{U_{g}(X),U_{\tilde{g}}(X)}}_{\text{I}}+\underbrace{r_{X}+\wt r_{X}}_{\text{II}},
\end{align*}
where $U_{g}(X)$ and $U_{\tilde{g}}(X)$ are the uniform distributions
over $\dcal_{g}^{r}(X)$ and $\dcal_{\tilde{g}}^{r}(X)$.

For the overlap (I), we first note that $(1-\delta)\wt g_{2}\preceq g_{2}\preceq(1+\delta)\wt g_{2}$
and thus 
\[
(1-\delta)\wt g\preceq g\preceq(1+\delta)\wt g.
\]
Let us denote the Dikin ellipsoids $\dcal_{g}(X)$ and $\dcal_{\tilde{g}}(X)$
by $D(g)$ and $D(\wt g)$ for simplicity. WLOG, assume $\vol(D(\wt g))\geq\vol(D(g))$.
Repeating the argument in (\ref{eq:eq3}), we have 
\begin{align*}
\text{I} & =\half\int_{\Rn}\Abs{\frac{\bm{1}_{D(g)}(z)}{\vol\Par{D(g)}}-\frac{\bm{1}_{D(\tilde{g})}(z)}{\vol\Par{D(\wt g)}}}dz=1-\frac{\vol\Par{D(g)\cap D(\wt g)}}{\vol\Par{D(\wt g)}}.
\end{align*}
Here, we can assume that $\wt g(X)=I$ due to affine invariance of
the ratio of volumes. Let $g(X)^{-1}=U^{\top}\Diag(\lda)U$ be a spectral
decomposition of $g(X)^{-1}$, where $\{\lda_{i}\}_{i=1}^{d}$ is
the set of eigenvalues of $g(X)^{-1}$ and $\lda:=(\lda_{i})\in\R^{d}$.
Consider a matrix $C\in\R^{d\times d}$ such that $C^{-1}=U^{\top}\Diag(\min(1,\lda))U$,
and by construction $D(C)\subset D(g)\cap D(I)$. Thus,
\begin{align*}
\frac{\vol\Par{D(g)\cap D(\wt g)}}{\vol\Par{D(\wt g)}} & =\frac{\vol\Par{D(g)\cap I}}{\vol\Par I}\\
 & \geq\sqrt{\prod_{i:\lda_{i}\leq1}\lda_{i}}=\sqrt{\prod_{i:\lda_{i}\leq1}(1-(1-\lda_{i}))}\\
 & \geq\sqrt{\exp\Par{-\sum_{i:\lda_{i}\leq1}(1-\lda_{i})}},
\end{align*}
where the last inequality follows from $1-x\geq\exp(-2x)$ for $0\leq x\leq\half$
and the fact that $(1-\delta)I\preceq g(X)^{-1}\preceq(1+\delta)I$
guarantees $\lda_{i}\geq\half$. This fact also implies that $\sum_{\lda_{i}<1}(1-\lda_{i})\leq\delta n^{2}$.
Putting these together, we have that $\text{I}\leq\frac{1}{100}$.

For the rejection probability (II), we have
\begin{align*}
\text{II} & =r_{X}+\wt r_{X}\\
 & =\int\max\Par{0,1-\sqrt{\frac{\det g(Z)}{\det g(X)}}}\frac{1_{D(g)}(Z)}{\vol\Par{D(g)}}dZ+\int\max\Par{0,1-\sqrt{\frac{\det\wt g(Z)}{\det\wt g(X)}}}\frac{1_{D(\tilde{g})}(Z)}{\vol\Par{D(\wt g)}}dZ.
\end{align*}
Since $(1-\delta)\wt g\preceq g\preceq(1+\delta)\wt g$ implies $(1-\delta)I\preceq\wt g^{-\half}g\wt g^{-\half}\preceq(1+\delta)I$,
we have $(1-\delta)^{n^{2}/2}\leq\sqrt{\frac{\det g}{\det\wt g}}\leq(1+\delta)^{n^{2}/2}$
and 
\begin{align*}
(1-\delta)^{n^{2}/2}\vol(D(\wt g)) & \leq\vol\Par{D(g)}\leq(1+\delta)^{n^{2}/2}\vol(D(\wt g)),\\
(1-\delta)^{n^{2}}\sqrt{\frac{\det\wt g(Z)}{\det\wt g(X)}} & \leq\sqrt{\frac{\det g(Z)}{\det g(X)}}\leq(1+\delta)^{n^{2}}\sqrt{\frac{\det\wt g(Z)}{\det\wt g(X)}}.
\end{align*}
Using this, we can further manipulate $\wt r_{X}$ as follows:
\begin{align*}
\wt r_{X} & =\int\max\Par{0,1-\sqrt{\frac{\det\wt g(Z)}{\det\wt g(X)}}}\frac{1_{D(\tilde{g})}(Z)}{\vol\Par{D(\wt g)}}dZ\\
 & \leq\int\frac{1_{D(\tilde{g})\backslash D(g)}(Z)}{\vol\Par{D(\wt g)}}dZ+\int\max\Par{0,1-\sqrt{\frac{\det\wt g(Z)}{\det\wt g(X)}}}\frac{1_{D(g)}(Z)}{\vol\Par{D(\wt g)}}dZ\\
 & \leq\frac{\vol\Par{D(\tilde{g})\backslash D(g)}}{\vol\Par{D(\wt g)}}+(1+\delta)^{n^{2}/2}\int\max\Par{0,1-(1-\delta)^{n^{2}/2}\sqrt{\frac{\det g(Z)}{\det g(X)}}}\frac{1_{D(g)}(Z)}{\vol\Par{D(g)}}dZ.
\end{align*}
Note that the first term is simply the term (I). We can show that
the second term is bounded by $1.1\cdot r_{X}$ for $\delta=10^{-10}/n^{2}$.
As $r_{X}\leq0.014$ due to Lemma~\ref{lem:one-step}, we have $\text{II}\leq0.045$.
\end{proof}



\begin{acknowledgement*}
We thank Khashayar Gatmiry for helpful discussions. This work was
supported in part by NSF awards CCF-2007443 and CCF-2134105.
\end{acknowledgement*}
\bibliography{Dikin-PSD} 

\newpage{}


\appendix

\section{Algebraic identities}

Here we collect useful algebraic identities related to trace, vectorization,
Kronecker product and Hadamard product. 
 
\begin{lem}
[Kronecker product] \label{lem:Kronecker} For $A,B,C,D\in\R^{n\times n}$
and $M$ in Definition~\ref{def:linearOperators},
\begin{itemize}
\item $(A\otimes B)\vec(C)=\tr\Par{BCA^{\top}}$.
\item $\vec(A)^{\top}\Par{B\otimes C}\vec(D)=\tr\Par{DB^{\top}A^{\top}C}$.
\item $(A\otimes B)(C\otimes D)=AC\otimes BD$.
\item $(A\otimes B)^{-1}=A^{-1}\otimes B^{-1}$.
\item $(A\otimes B)^{\top}=A^{\top}\otimes B^{\top}$.
\item $\tr\Par{A\otimes B}=\tr(A)\tr(B)$.
\item $\det(M^{\top}(A\otimes A)M)=2^{n(n-1)/2}\Par{\det A}^{n+1}$.
\end{itemize}
\end{lem}

\begin{lem}
[Hadamard product] \label{lem:Hadamard} Let $A,B,C,D\in\R^{n\times n}$,
$x,y\in\Rn$, and $D_{1},D_{2}\in\Rnn$ be diagonal matrices.
\begin{itemize}
\item $(A\circ B)y=\diag(A\Diag(y)B^{\top})$.
\item $x^{\top}(A\circ B)y=\tr\Par{\Diag(x)A\Diag(y)B^{\top}}$.
\item $D_{1}(A\hada B)=(D_{1}A)\hada B=A\hada(D_{1}B)$.
\item $(A\hada B)D_{2}=(AD_{2})\hada B=A\hada(BD_{2})$.
\item $(A\otimes B)\circ(C\otimes D)=(A\circ C)\otimes(B\circ D)$.
\end{itemize}
\end{lem}


\section{Matrix calculus \label{app:matrixCalculus}}

Let $g(x):\Rn\to\R^{n\times n}$ be a matrix function. Its gradient
at $x$, denoted by $Dg(x)$, is the third-order tensor defined by
$(Dg(x))_{ijk}=\frac{\del g_{ij}(x)}{\del x_{k}}$. Unless specified
otherwise, the multiplication between higher-order tensors and a matrix
of size $n\times n$ is running over $i,j$-entries. For instance,
for a matrix $M\in\R^{n\times n}$ the product $Dg(x)M$ is the third-order
tensor defined by
\[
(Dg(x)M)_{\cdot,\cdot,k}=(Dg(x))_{\cdot,\cdot,k}M\text{ for each }k\in[n].
\]
In the same way, the trace of higher-order tensors is applied to a
matrix spanned by $i,j$-entries, i.e.,
\[
\Par{\tr\Par{Dg(x)}}_{k}=\tr\Par{\Par{Dg(x)}_{\cdot,\cdot,k}}.
\]

Now define $\vphi(x)=\log\det g(x):\Rn\to\R$. Its gradient is
\begin{equation}
\grad\vphi(x)=D\log\det g(x)=\tr\Par{g(x)^{-1}Dg(x)},\label{eq:gradLogDet}
\end{equation}
and this indicates that its directional derivative in $h\in\Rn$ is
$\grad\vphi(x)\cdot h=\tr\Par{g(x)^{-1}Dg(x)[h]}$. To compute the
Hessian of $\vphi$, we note that
\begin{equation}
D(g^{-1})(x)=-g(x)^{-1}Dg(x)g(x)^{-1}.\label{eq:diffInverse}
\end{equation}
Using this and the product rule, we have
\begin{align}
\hess\vphi(x) & =D\tr\Par{g(x)^{-1}Dg(x)}\nonumber \\
 & =-\tr\Par{g(x)^{-1}Dg(x)g(x)^{-1}Dg(x)}+\tr\Par{g(x)^{-1}D^{2}g(x)}\nonumber \\
 & =\tr\Par{g(x)^{-1}D^{2}g(x)}-\norm{g(x)^{-\half}Dg(x)g(x)^{-\half}}_{F}^{2},\label{eq:hessLogDet}
\end{align}
where $D^{2}g(x)$ is the fourth-order tensor defined by $(D^{2}g(x))_{ijkl}=\frac{\del g(x)_{ij}}{\del x_{k}\del x_{l}}$.

We now prove Lemma~\ref{lem:metricFormula}, providing formulas of
the Hessian and its inverse of $\phi(X)=-\log\det X$ for a matrix
$X\in\psd$.
\begin{proof}
[Proof of Lemma \ref{lem:metricFormula}] By setting $g(X)=X$ and
$\phi(X)=-\vphi(X)$ above, (\ref{eq:hessLogDet}) implies that for
a symmetric matrix $H\in\S^{n}$
\begin{align}
\hess\phi(X)[H,H] & =\tr\Par{X^{-1}HX^{-1}H}\label{eq:2ndDiffLogDet}\\
 & =\vec{(}H)^{\top}\Par{X^{-1}\otimes X^{-1}}\vec{(}H)=\vec{(}H)^{\top}\Par{X\otimes X}^{-1}\vec{(}H)\nonumber 
\end{align}
where the last line follows from Lemma~\ref{lem:Kronecker}. When
representing $X$ and $H$ in $\R^{d}$ space with notations $x:=\svec(X)$
and $h:=\svec(H)$, the definition of $M$ (see Definition~\ref{def:linearOperators})
turns the formula above into
\[
\hess\phi(x)[h,h]=h^{\top}M^{\top}(X\otimes X)^{-1}Mh,
\]
and thus $g_{X}:=\nabla_{x}^{2}\phi(x)=\nabla_{X}^{2}\phi(X)$ is
equal to $M^{\top}(X\otimes X)^{-1}M$. The formula of the inverse,
$g_{X}^{-1}=M^{\dagger}(X\otimes X)M^{\dagger\top}$, is immediate
from \cite{magnus1980elimination}, and another part follows from
$M^{\dagger}=LN$ and $N^{\top}=N$ (Lemma 3.6 and Lemma 2.1 in \cite{magnus1980elimination}).
\end{proof}

\section{Remaining proofs}

\subsection{Logarithmic barrier \label{app:subsec:logBarrier}}

Here we collect details used in the paper that involve calculus of
the logarithmic barrier, $\phi_{\log}(X):=-\sum_{i=1}^{m}\log\Par{\inner{A_{i},X}-b_{i}}$.
Recall the metric $g$ defined by the Hessian of $\phi_{\log}$ is
given by
\begin{align*}
g(X) & =M^{\top}\left[\begin{array}{ccc}
\vec(A_{1}) & \cdots & \vec(A_{m})\end{array}\right]S_{X}^{-2}\left[\begin{array}{c}
\vec(A_{1})^{\top}\\
\vdots\\
\vec(A_{m})^{\top}
\end{array}\right]M\\
 & =M^{\top}A^{\top}S_{X}^{-2}AM,
\end{align*}
where $S_{X}=\Diag\Par{\inner{A_{i},X}-b_{i}}\in\R^{m\times m}$ and
$A^{\top}=\left[\begin{array}{ccc}
\vec(A_{1}) & \cdots & \vec(A_{m})\end{array}\right]\in\R^{n^{2}\times m}$.   Since we work on $\S^{n}$ and $\R^{d}$ simultaneously, we
consider its vector version (i.e., $g(x)=A^{\top}S_{x}^{-2}A$ for
$x\in\R^{d}$) for simplicity and then translate it into one in our
setting. We recall notations that appear in our computation:
\begin{itemize}
\item $A_{x}=S_{x}^{-1}A\in\R^{m\times d}$.
\item $s_{x}=\diag(S_{x})\in\R^{m}$.
\item $s_{x,h}=A_{x}h\in\R^{m}$ and $S_{x,h}=\Diag(s_{x,h})\in\R^{m\times m}$.
We drop $x$ if $x$ is clear from the context.
\end{itemize}
Going forward, we use $h$ to denote a vector in $\R^{n}$.
\begin{claim}
\label{claim:1stDiffSlack} $DS_{x}[h]=\Diag(Ah)$ and $DS_{x}^{-1}[h]=-S_{x}^{-1}S_{x,h}$.
\end{claim}

\begin{proof}
The first is obvious from differentiation of $S_{x}=\Diag(Ax-b)$
with respect to $x$. For the second,
\begin{align*}
DS_{x}^{-1}[h] & =-S_{x}^{-1}DS_{x}[h]S_{x}^{-1}=-S_{x}^{-1}\Diag(Ah)S_{x}^{-1}\\
 & =-\Diag(A_{x}h)S_{x}^{-1}=-S_{x}^{-1}\Diag(A_{x}h)\\
 & =-S_{x}^{-1}S_{x,h},
\end{align*}
where $S_{x}^{-1}\Diag(Ah)=\Diag(A_{x}h)$ and $\Diag(A_{x}h)S_{x}^{-1}=S_{x}^{-1}\Diag(A_{x}h)$
hold as all of them are diagonal matrices.
\end{proof}
\begin{claim}
\label{claim:diffLogBarrier} $Dg(x)[h]=-2A_{x}^{\top}S_{x,h}A_{x}$
and $D^{2}g(x)[h,h]=6A_{x}^{\top}S_{x,h}^{2}A_{x}\succeq0$. In the
PSD setting with $A_{X}:=S_{X}^{-1}A$, this becomes $Dg(X)[H]=-2M^{\top}A_{X}^{\top}\Diag\Par{A_{X}\vec(H)}A_{X}M$
and $D^{2}g(X)[H,H]=6M^{\top}A_{X}^{\top}\Diag(A_{X}\vec(H))^{2}A_{X}M$.
\end{claim}

\begin{proof}
Due to $g(x)=A_{x}^{\top}A_{x}$,
\begin{align*}
Dg(x)[h] & =D(A^{\top}S_{x}^{-2}A)[h]=A^{\top}DS_{x}^{-2}[h]A\\
 & =-2A^{\top}S_{x}^{-3}DS_{x}[h]A=-2A_{x}^{\top}S_{x}^{-1}\Diag(Ah)A_{x}\\
 & =-2A_{x}^{\top}S_{x,h}A_{x}.
\end{align*}

For the second-order directional derivative,
\begin{align*}
Dg^{2}(x)[h,h] & =-2D(A_{x}^{\top}S_{x,h}A_{x})[h]=-2D(A^{\top}S_{x}^{-3}\Diag(Ah)A)[h]\\
 & =6A^{\top}S_{x}^{-4}DS_{x}[h]\Diag(Ah)A\\
 & =6A_{x}^{\top}S_{x,h}^{2}A_{x}.\qedhere
\end{align*}
\end{proof}
Note that $D^{2}g(x)[h,h]=6A_{x}^{\top}S_{x,h}^{2}A_{x}\succeq0$.
We now provide the deferred proof of Lemma~\ref{lem:paramsBarrier}-1.
\begin{proof}
[Proof of Lemma \ref{lem:paramsBarrier}-1] By putting $D_{x}=I_{m}$
into Lemma~\ref{lem:helper4Diagonal}-1, we have
\[
\norm{g(x)^{-\half}Dg(x)[h]g(x)^{-\half}}_{F}\leq2\sqrt{\max_{i\in[m]}\sigma(A_{x})_{i}}\norm h_{g(x)}.
\]
As $P_{x}$ is the orthogonal projection, $P_{x}\preceq I$ and $\sigma(A_{x})\leq1$.
Thus, $\norm{g(x)^{-\half}Dg(x)[h]g(x)^{-\half}}_{F}\leq2\norm h_{g(x)}$,
deriving strong self-concordance of the logarithmic barriers. 

For the $\onu$-symmetry, we note that the first part (i.e., $\dcal_{g}^{1}(x)\subset K\cap(2x-K)$)
follows from Lemma~\ref{lem:symmetricLeftpart}. The second part
is immediate from $\onu=\tr\Par{I_{m}}=m$ and Lemma~\ref{lem:helper4Diagonal}-3.
\end{proof}

\subsection{Volumetric barrier \label{app:subsec:volBarrier}}

\cite{vaidya1996new} introduced the \emph{volumetric barrier} for
a convex region $Ax\geq b$ defined by 
\[
\phi_{\vol}=\half\log\det\hess\phi_{\log}=\half\log\det A_{x}^{\top}A_{x}.
\]
We collect computational preliminaries regarding to the volumetric
barrier and then move onto the approximate volumetric barrier. 

\paragraph{Volumetric barrier.}
\begin{claim}
$\grad\phi_{\vol}(x)=-A_{x}^{\top}\sigma_{x}$ and $\hess\phi_{\vol}(x)=A_{x}^{\top}\Par{3\Sigma_{x}-2P_{x}^{(2)}}A_{x}$.
\end{claim}

\begin{proof}
Let $g(x)=\hess\phi_{\log}=A_{x}^{\top}A_{x}$. Note that by Claim~\ref{claim:diffLogBarrier}
\begin{align*}
\grad\phi_{\vol}(x)[h] & =-\tr\Par{g^{-1}A_{x}^{\top}S_{x,h}A_{x}}\\
 & =-\tr\bigg(\underbrace{A_{x}g^{-1}A_{x}^{\top}}_{=P_{x}}S_{x,h}\bigg)\\
 & =-\tr\Par{P_{x}S_{x,h}}=-\tr\Par{P_{x}S_{x,h}I_{m}I_{m}}\\
 & \underset{\text{(i)}}{=}-\bm{1}^{\top}(P_{x}\circ I_{m})S_{x,h}=-1^{\top}\Sigma_{x}A_{x}h\\
 & =-h^{\top}A_{x}^{\top}\sigma_{x},
\end{align*}
where we used Lemma~\ref{lem:Hadamard} in (i). 

For the Hessian of $\phi_{\vol}$,
\[
\hess\phi_{\vol}(x)[h,h]=\half\Par{\tr\Par{g^{-1}D^{2}g[h,h]}-\tr\Par{g^{-1}Dg[h]g^{-1}Dg[h]}}.
\]
For the first term, by Claim~\ref{claim:diffLogBarrier}
\begin{align*}
\half\tr\Par{g^{-1}Dg[h]g^{-1}Dg[h]} & =2\tr\Par{g^{-1}A_{x}^{\top}S_{x,h}A_{x}g^{-1}A_{x}^{\top}S_{x,h}A_{x}}=2\tr\Par{P_{x}S_{x,h}P_{x}S_{x,h}}\\
 & \underset{\text{(i)}}{=}2(A_{x}h)^{\top}\Par{P_{x}\circ P_{x}}(A_{x}h)=2h^{\top}A_{x}^{\top}P_{x}^{(2)}A_{x}h,
\end{align*}
where we used Lemma~\ref{lem:Hadamard} in (i). For the second term,
Claim~\ref{claim:diffLogBarrier} leads to
\begin{align*}
\half\tr\Par{g^{-1}D^{2}g[h,h]} & =3\tr\Par{g^{-1}A_{x}^{\top}S_{x,h}^{2}A_{x}}=3\tr\Par{P_{x}S_{x,h}IS_{x,h}}\\
 & =3h^{\top}A_{x}^{\top}\Par{P_{x}\circ I}A_{x}h=3h^{\top}A_{x}^{\top}\Sigma_{x}A_{x}h.
\end{align*}
Putting these two together, we have
\[
D^{2}\phi_{\vol}(x)[h,h]=h^{\top}A_{x}^{\top}\Par{3\Sigma_{x}-2P_{x}^{(2)}}A_{x}h
\]
and thus
\[
\hess\phi_{\vol}(x)=A_{x}^{\top}(3\Sigma_{x}-2P_{x}^{(2)})A_{x}.\qedhere
\]
\end{proof}
\begin{lem}
\label{lem:schurProjection} $P_{x}^{(2)}\preceq\Sigma_{x}$, so $A_{x}^{\top}\Sigma_{x}A_{x}\preceq\hess\phi_{\vol}(x)\preceq3A_{x}^{\top}\Sigma_{x}A_{x}$.
\end{lem}

\begin{proof}
Note that $\Sigma_{x}=P_{x}\circ I$. Let us show that $0\leq h^{\top}P_{x}\circ(I-P_{x})h$
for $h\in\Rn$. Since $P_{x}$ and $I-P_{x}$ are both orthogonal
projections matrices, for $C:=P_{x}H(I-P_{x})$ and $H=\Diag(h)$,
\begin{align*}
h^{\top}P_{x}\circ(I-P_{x})h & =\tr\Par{HP_{x}H(I-P_{x})}\\
 & =\tr\Par{(I-P_{x})HP_{x}P_{x}H(I-P_{x})}=\tr(C^{\top}C)\geq0.\qedhere
\end{align*}
\end{proof}


\paragraph{Approximate volumetric metric.}

The approximate volumetric metric is defined by
\[
g(x)=A_{x}^{\top}\Sigma_{x}A_{x},
\]
which serves as a good approximation of $\hess\phi_{\vol}$ due to
$\Sigma_{x}\preceq3\Sigma_{x}-2P_{x}^{(2)}\preceq3\Sigma_{x}$. We
begin with recalling computational results on the leverage scores:
\begin{lem}
[\cite{lee2019solving}] \label{lem:usefulFactLeverage} Let $\Sigma_{x}=\Sigma(A_{x})\in\R^{m\times m},g(x)=A_{x}^{\top}\Sigma_{x}A_{x}$,
and $h\in\Rn$.
\begin{itemize}
\item \textup{(Lemma 26)} $\max_{i\in[m]}\frac{\sigma\Par{\Sigma_{x}^{1/2}A_{x}}_{i}}{\Par{\Sigma_{x}}_{ii}}\leq2m^{\frac{1}{2}}$.
\item \textup{(Lemma 33)} $\norm{A_{x}h}_{\Sigma_{x}}=\norm h_{g(x)}$
and $\norm{A_{x}h}_{\infty}\leq\sqrt{2}m^{\frac{1}{4}}\norm h_{g(x)}$.
\item \textup{(Lemma 34)} $\norm{\Sigma_{x}^{-1}\diag\Par{D\Sigma_{x}[h]}}_{\Sigma_{x}}\leq2\norm h_{g(x)}$.
\end{itemize}
\end{lem}

We are now ready to prove the second item of Lemma \ref{lem:paramsBarrier}.
\begin{proof}
[Proof of Lemma~\ref{lem:paramsBarrier}-2] Let us set $D_{x}$
to $\Sigma_{x}=\Sigma(A_{x})$ in Lemma~\ref{lem:helper4Diagonal}.
By Lemma~\ref{lem:usefulFactLeverage}, we have
\begin{align*}
\max_{i}\Par{\frac{\sigma\Par{\sqrt{D_{x}}A_{x}}_{i}}{(D_{x})_{ii}}} & \leq2\sqrt{m},\\
\sum_{i=1}^{m}\Par{D_{x}^{-1}}_{ii}(DD_{x}[h])_{i}^{2} & =\norm{\Sigma_{x}^{-1}\diag\Par{D\Sigma_{x}[h]}}_{\Sigma_{x}}^{2}\\
 & \leq4\norm h_{g(x)}^{2}.
\end{align*}
Thus,
\begin{align*}
\norm{g(x)^{-\half}Dg(x)[h]g(x)^{-\half}}_{F}^{2} & \leq4\max_{i}\Par{\frac{\sigma\Par{\sqrt{D_{x}}A_{x}}_{i}}{(D_{x})_{ii}}}\cdot\Par{\norm h_{g(x)}^{2}+\sum_{i=1}^{m}\Par{D^{-1}}_{ii}(DD_{x}[h])_{i}^{2}}\\
 & \leq40\sqrt{m}\norm h_{g(x)}^{2}.
\end{align*}
For the symmetry parameter, $\norm{A_{x}(y-x)}_{\infty}\leq\sqrt{\max_{i\in[m]}\frac{\sigma\Par{\sqrt{D_{x}}A_{x}}_{i}}{D_{x,i}}}\leq m^{1/4}$
for $y\in\dcal_{g}^{1}(x)$ by Lemma~\ref{lem:helper4Diagonal}-2.
Also, Lemma~\ref{lem:helper4Diagonal}-3 implies that $y$ with $\norm{A_{x}(y-x)}_{\infty}\leq1$
is contained in $\dcal_{g}^{\sqrt{\tr(D_{x})}}(x)$, where
\[
\tr\Par{D_{x}}=\tr\Par{P_{x}}=\tr\Par{A_{x}(A_{x}^{\top}A_{x})^{-1}A_{x}}=\tr\Par{I_{n}}=n.
\]
Therefore, $\tilde{g}(x):=40\sqrt{m}g(x)=40\sqrt{m}A_{x}^{\top}\Sigma_{x}A_{x}$
is strongly self-concordant with the symmetry parameter $\onu=O(\sqrt{m}n)$.
\end{proof}

\subsection{Derivatives of matrices \label{app:subsec:derivativeMatrices}}

In this section, we provide details in computing derivatives of leverage
scores, orthogonal projections, and so on.

\propCalculusLeverage*
\begin{proof}
The first and second items follow from Lemma 2.2 and 2.3 of \cite{gatmiry2023sampling}.
From these formulas and the definition of $\Lambda_{x}$,
\begin{align*}
 & D\Lambda_{x}[h]\\
 & =D\Sigma_{x}[h]-DP_{x}[h]\circ P_{x}-P_{x}\circ DP_{x}[h]\\
 & =-2\Diag(\Lambda_{x}s_{x,h})-\Par{-P_{x}S_{x,h}-S_{x,h}P_{x}+2P_{x}S_{x,h}P_{x}}\circ P_{x}-P_{x}\circ\Par{-P_{x}S_{x,h}-S_{x,h}P_{x}+2P_{x}S_{x,h}P_{x}}\\
 & \underset{\text{(i)}}{=}-2\Diag(\Lambda_{x}s_{x,h})+2P_{x}\circ P_{x}S_{x,h}+2S_{x,h}P_{x}\circ P_{x}-2(P_{x}S_{x,h}P_{x})\circ P_{x}-2P_{x}\circ(P_{x}S_{x,h}P_{x}),
\end{align*}
where in (i) we used $D(A\hada B)=(DA)\circ B=A\hada(DB)$ and $(A\hada B)D=(AD)\hada B=A\circ(BD)$\footnote{This property allows us to write $DA\hada B$ without parenthesis.}
for a diagonal matrix $D\in\Rnn$ (Lemma~\ref{lem:Hadamard}). 

Using the first three formulas
\begin{align*}
 & D^{2}\Sigma_{x}[h,h]\\
 & =-2D\Diag(\Lambda_{x}s_{x,h})[h]\\
 & =-2\Diag(D\Lambda_{x}[h]s_{x,h})+2\Diag\Par{\Lambda_{x}S_{x,h}s_{x,h}}\\
 & =-2\Diag\Par{\Par{-2\Diag(\Lambda_{x}s_{x,h})+2P_{x}\circ P_{x}S_{x,h}+2S_{x,h}P_{x}\circ P_{x}-2(P_{x}S_{x,h}P_{x})\circ P_{x}-2P_{x}\circ(P_{x}S_{x,h}P_{x})}s_{x,h}}\\
 & \qquad+2\Diag\Par{\Lambda_{x}S_{x,h}s_{x,h}}\\
 & =4\Diag\Par{\cred{\Lambda_{x}}s_{x,h}}\cblue{S_{x,h}}-4\Diag\Par{P_{x}\circ P_{x}S_{x,h}s_{x,h}}-4\Diag\Par{S_{x,h}P_{x}\circ P_{x}s_{x,h}}\\
 & \qquad+4\Diag\Par{(P_{x}S_{x,h}P_{x})\circ P_{x}s_{x,h}}+4\Diag\Par{P_{x}\circ(P_{x}S_{x,h}P_{x})s_{x,h}}+2\Diag\Par{\cred{\Lambda_{x}}S_{x,h}s_{x,h}}\\
 & =4\Diag\Par{\cblue{S_{x,h}}\cred{(\Sigma_{x}-P_{x}\circ P_{x})}s_{x,h}}-4\Diag\Par{P_{x}\circ P_{x}S_{x,h}s_{x,h}}-4\Diag\Par{S_{x,h}P_{x}\circ P_{x}s_{x,h}}\\
 & \qquad+4\Diag\Par{(P_{x}S_{x,h}P_{x})\circ P_{x}s_{x,h}}+4\Diag\Par{P_{x}\circ(P_{x}S_{x,h}P_{x})s_{x,h}}+2\Diag\Par{\cred{(\Sigma_{x}-P_{x}\circ P_{x})}S_{x,h}s_{x,h}}\\
 & =\ccyan{4\Diag(S_{x,h}\Sigma_{x}s_{x,h})}-6\Diag\Par{P_{x}\circ P_{x}S_{x,h}s_{x,h}}-8\Diag\Par{S_{x,h}P_{x}\circ P_{x}s_{x,h}}\\
 & \qquad+4\Diag\Par{(P_{x}S_{x,h}P_{x})\circ P_{x}s_{x,h}}+4\Diag\Par{P_{x}\circ(P_{x}S_{x,h}P_{x})s_{x,h}}+\ccyan{2\Diag(\Sigma_{x}S_{x,h}s_{x,h})}\\
 & =\text{\ensuremath{\ccyan{6\Diag(S_{x,h}\Sigma_{x}s_{x,h})}}}-6\Diag\Par{\cblue{P_{x}\circ P_{x}S_{x,h}s_{x,h}}}-8\Diag\Par{\cblue{S_{x,h}P_{x}\circ P_{x}s_{x,h}}}\\
 & \qquad+4\Diag\Par{\cblue{(P_{x}S_{x,h}P_{x})\circ P_{x}s_{x,h}}}+4\Diag\Par{\cblue{P_{x}\circ(P_{x}S_{x,h}P_{x})s_{x,h}}}\\
 & \underset{\text{(i)}}{=}6S_{x,h}\Sigma_{x}\Diag\Par{s_{x,h}}-6\Diag\Par{\diag\Par{P_{x}S_{x,h}(P_{x}S_{x,h})^{\top}}}-8\Diag\Par{\diag\Par{S_{x,h}P_{x}S_{x,h}P_{x}^{\top}}}\\
 & \qquad+4\Diag\Par{P_{x}S_{x,h}P_{x}S_{x,h}P_{x}}+4\Diag\Par{P_{x}S_{x,h}\Par{P_{x}S_{x,h}P_{x}}^{\top}}\\
 & =6S_{x,h}\Sigma_{x}S_{x,h}-6\Diag\Par{P_{x}S_{x,h}^{2}P_{x}}-8\Diag\Par{S_{x,h}P_{x}S_{x,h}P_{x}}+8\Diag\Par{P_{x}S_{x,h}P_{x}S_{x,h}P_{x}},
\end{align*}
where in (i) we applied Lemma~\ref{lem:Hadamard}-1 to the terms
with blue. 

Applying the product rule to $\theta_{1}(x)=A_{x}^{\top}\Sigma_{x}A_{x}=A^{\top}S_{x}^{-2}\Sigma_{x}A,$
\begin{align*}
D\theta_{1}[h] & =-2A^{\top}S_{x}^{-3}\Sigma_{x}\Diag(Ah)A+A^{\top}S_{x}^{-2}D\Sigma_{x}[h]A\\
 & =-2A_{x}^{\top}\Sigma_{x}S_{x,h}A_{x}+A_{x}^{\top}D\Sigma_{x}[h]A_{x},\\
D^{2}\theta_{1}[h,h] & =6A_{x}^{\top}S_{x,h}\Sigma_{x}S_{x,h}A_{x}-2A_{x}^{\top}D\Sigma_{x}[h]S_{x,h}A_{x}-2A_{x}^{\top}S_{x,h}D\Sigma_{x}[h]A_{x}+A_{x}^{\top}D^{2}\Sigma_{x}[h,h]A_{x}\\
 & =6A_{x}^{\top}S_{x,h}\Sigma_{x}S_{x,h}A_{x}-4A_{x}^{\top}D\Sigma_{x}[h]S_{x,h}A_{x}+A_{x}^{\top}D^{2}\Sigma_{x}[h,h]A_{x}.
\end{align*}
The derivatives of $\theta_{2}$ simply follow from Claim \ref{claim:diffLogBarrier}.
\end{proof}


\end{document}
