\section{A Construction for REGAs}\label{sec:alternate}

In this section, we give a construction for the case where the group action can only be computed efficiently for a small ``base'' set of group elements. Such group actions are known as ``restricted effective group actions'' (REGAs). 

\subsection{Some additional background}

Before giving the construction, we here provide some additional background that will be necessary for understanding the construction.

\paragraph{Groups.} Let $\G$ be a group (written additively), and $N$ an integer such that $N\times g=0$ for all $g\in\G$. $N=|\G|$ will do. Then $\G$ is a subgroup of $\Z_N^n$ for some positive integer $n$. Let $W$ be the set of vectors in $\Z_N^n$ such that $\wv\cdot g=0\bmod N$ for all $g\in\G$. $W$ is then a group, and we can therefore consider the group $(\Z_N^n)/W$ defined using the equivalence relation $\sim$, where $\uv_1\sim\uv_2$ if $\uv_1-\uv_2\in W$. $(\Z_N^n)/W$ is isomorphic to $\G$; let $\phi:\G\rightarrow(\Z_N^n)/W$ be an isomorphism. Note that for $g\in\G\subseteq\Z_N^n$ and $h\in \G$, $g\cdot\phi(h)\bmod N$ is well-defined by taking any representative $h'\in\phi(h)$ and computing $g\cdot h'\bmod N$.

Under this notation, we can re-define $\chi(g,h)$ as $e^{i2\pi g\cdot\phi(h)/N}$, which is equivalent to the definition in Section~\ref{sec:prelim}.

We associate $\Z_N$ with the interval $[-\lfloor (N-1)/2\rfloor,\lceil (N-1)/2 \rceil]$ in the obvious way, and likewise associate $\Z_N^n$ with the hypercube $[-\lfloor (N-1)/2\rfloor,\lceil (N-1)/2 \rceil]^n$. This gives rise to a notion of norm on $\Z_N^n$ by taking the norm in $\Z^n$.

\begin{lemma}\label{lem:bigelements} Let $\G$ be a subgroup of $\Z_N$. Then the number of elements $g\in\G$ such that $|g|\geq N/4$ is exactly
	$|\G|+1-2\lceil|\G|/4\rceil$.
	In particular, if $\G\neq\{0\}$, then there is at least one element $g\in\G$ has $|g|\geq N/4$.
\end{lemma}
\begin{proof}First, it suffices to consider $|\G|=N$, in other words $\G=\Z_N$: we can then lift to $N=t|\G|$, where $\G$ is embedded into $\Z_N$ by multiplying each element in $\G$ by $t$ (where multiplication is over the integers). Since $N$ is also multiplied by $t$, this preserves the number of elements with $|g|\geq N/4$.
	
	When $\G=\Z_N$, we are then simply asking for the number of elements in $[-\lfloor (|\G|-1)/2\rfloor,\lceil (|\G|-1)/2 \rceil]$ with absolute value at least $|\G|/4$. In other words, it is the combined size of the intervals $[\lceil |\G|/4\rceil,\lceil (|\G|-1)/2 \rceil]$ and $[-\lfloor (|\G|-1)/2\rfloor,-\lceil |\G|/4\rceil]$, giving a total of $\left(\lceil (|\G|-1)/2 \rceil-\lceil |\G|/4\rceil+1\right)+\left(\lfloor (|\G|-1)/2 \lfloor-\lceil |\G|/4\rceil+1\right)=|\G|+1-2\lceil|\G|/4\rceil$.
\end{proof}

\begin{lemma}\label{lem:far}Let $\Am\in\Z_N^{n\times m}$ be a matrix. Let $\G$ be the subgroup of $\Z_N^n$ generated by the columns of $\Am$. Let $B,C$ be positive integers such that $8BCm< N$. Suppose there is a distribution $\Ds$ on $[-B,B]^m$ such that $\Am\cdot \xv$ for $x\gets\Ds$ is negligibly close to uniform in $\G$. Then the function $f:\G\times[-C,C]\rightarrow\Z_N^m$ given by $f(g,\ev)=\Am^T\cdot\phi(g)+\ev$ is injective.
\end{lemma}
\begin{proof}Note that $\Am^T\cdot\phi(g)$ is well defined since it is independent of the representative of $\phi(g)$. Consider a potential collision in $f$: $\Am^T\cdot\phi(g_1)+\ev_1=\Am^T\cdot\phi(g_2)+\ev_2$. By subtracting, this gives a non-zero pair $(g=g_1-g_2,\ev=\ev_1-\ev_2)$ where $\ev\in[-2C,2C]$ such that $\Am^T\cdot\phi(g)+\ev=0$ or equivalently $\Am^T\cdot\phi(g)=-\ev$. Now consider sampling $\xv\gets\Ds$, meaning $\uv=\Am\cdot\xv$ is negligibly close to uniform in $\G$. Then $\uv^T\cdot\phi(g)=\xv^T\cdot\Am^T\cdot\phi(g)=-\xv^T\cdot\ev$. On one hand, $\uv^T\cdot\phi(g)$ is statistically close to uniform in a subgroup $\G'$ of $\Z_N$, and $\G'$ is different from $\{0\}$ since $g\neq 0$. By Lemma~\ref{lem:bigelements}, the probability $|\uv^T\cdot\phi(g)|\geq N/4$ is $|\G'|+1-2\lceil|\G'|/4\rceil>0$ since $|\G'|\geq 2$. On the other hand, $|-\xv^T\cdot\ev|<2mBC\leq N/4$ always. This means the distributions of $\uv^T\cdot\phi(g)$ and $-\xv^T\cdot\ev$ must be non-negligibly far, a contradiction.
\end{proof}




\paragraph{Discrete Gaussians.} The \emph{discrete Gaussian distribution} is the distribution over $\Z$ defined as:
\[\Pr[x]=\Ds_\sigma(x):=C_\sigma e^{2\pi x^2/\sigma^2},\]
where $C_\sigma$ is the normalization constant $C_\sigma=\sum_{x\in\Z}e^{2\pi x^2/\sigma^2}$, so that $\Ds_\sigma$ defined a probability distribution. We will also define a truncated variant, denoted
\[\Ds_{\sigma,B}(x):=\begin{cases}C_{\sigma,B} e^{2\pi x^2/\sigma^2}&\text{ if }|x|\leq B\\0&\text{ otherwise}\end{cases},\]
where again $C_{\sigma,B}$ is an appropriately defined normalization constant. For large $B$, we can treat the truncated and un-truncated Gaussians as essentially the same distribution:
\begin{fact}For $\sigma\geq\omega(\sqrt{\log\lambda})$ and $B\geq \sigma\times\omega(\sqrt{\log\lambda})$, the distributions $\Ds_{\sigma}$ and $\Ds_{\sigma,B}$ are negligibly close
\end{fact}
For a vector $\rv\in\Z^m$, we write $\Ds_{\sigma,B}(\rv)=\prod_{i=1}^m\Ds_{\sigma,B}(r_i)$.

The \emph{discrete Gaussian superposition} is the quantum state
\[|\Ds_\sigma\rangle:=\sum_{x\in\Z}\sqrt{\Ds_\sigma(x)}|x\rangle\]
As we will generally need to restrict to finite-precision, we also consider the truncated variant
\[|\Ds_{\sigma,B}\rangle:=\sum_{x\in[-B,B]}\sqrt{\Ds_{\sigma,B}(x)}|x\rangle\]
Again, for large enough $B$, we can treat the truncated and un-truncated Gaussian superpositions as essentially the same state:
\begin{fact}For $\sigma\geq\omega(\sqrt{\log\lambda})$ and  $B\geq \sigma\times\omega(\sqrt{\log\lambda})$, the $\||\Ds_\sigma\rangle-|\Ds_{\sigma,B}\rangle\|$ is negligible.
\end{fact}
By adapting classicsal lattice sampling algorithms, the states $|\Ds_{\sigma,B}\rangle$ can be efficiently constructed. 


\paragraph{Fourier transform pairs.} Fix an integer $N$. We will associate the set $\Z_N$ with the integers $[-\lfloor (N-1)/2\rfloor,\lceil (N-1)/2\rceil]$. Denote by $\QFT_N$ the Quantum Fourier Transform $\QFT_{\Z_N}$. We now recall some basic facts about quantum Fourier transforms. 

\begin{align*}
	\QFT_N^m\sum_{\rv\in\Z_N^m:\Am\cdot\rv=\sv}|\rv\rangle&=N^{m/2-n}\sum_{\tv\in\Z_N^n}e^{i2\pi \tv\cdot\sv/N }|\Am^T\cdot \tv\rangle\text{ for }\Am\in\Z_N^{n\times m}\\
	\QFT_N^m\sum_{\rv}\alpha_\rv\beta_\rv|\rv\rangle&=\frac{1}{N^{m/2}}\sum_{\tv,\uv}\hat{\alpha}_\tv\hat{\beta}_\uv|\uv+\tv\rangle\text{ for }\begin{subarray}{l}\sum_\tv\hat{\alpha}_\tv|\tv\rangle=\QFT_N^m\sum_\rv\alpha_\rv|\rv\rangle\\\sum_\uv\hat{\beta}_\uv|\uv\rangle=\QFT_N^m\sum_\rv\beta_\rv|\rv\rangle\end{subarray}\\
	\QFT_N |\Ds_{\sigma,\lfloor (N-1)/2\rfloor}\rangle&\approx|\Ds_{N/\sigma,\lfloor (N-1)/2\rfloor}\rangle\text{ for }\begin{subarray}{c}N\geq \sigma\times\omega(\sqrt{\log\lambda})\\ \sigma\geq\omega(\sqrt{\log\lambda})\end{subarray}
\end{align*}
Above, $\approx$ means the two states are negligibly close.


\subsection{The Construction}

Let $\G_\lambda,\Xs_\lambda,*$ be a REGA, and $\Ts=(g_1,\dots,g_m)$ a set such that $*$ can be efficiently computed for $g_i$ and $g_i^{-1}$. We can associate $\G_\lambda$ with a subgroup of $\Z_N^n$ for some integers $N,n$. We can likewise associate the list $\Ts$ with the matrix $\Am=(g_1,\cdots,g_m)\in\Z_N^{n\times m}$.

We will make the following assumption about the structure of $\Ts$, which is typical in the isogeny literature.

\begin{assumption}\label{assump:uniform} There is a polynomial $B$ and a distribution $\Ds^*$ on $[-B,B]^m$ such that for $\xv\gets\Ds$, $\sum_{i=1}^m x_i g_i=\Am\cdot\xv$ is statistically close to a uniform element in $\G$\end{assumption}
%Note that Assumption~\ref{assump:uniform} is typically justified by heuristically treating $\Am$ as a random matrix, in which case it follows provided $\Ts$ is large enough. 
Numerous examples of such $\Ds^*$ have been proposed, such as discrete Gaussians~\cite{EC:DeFGal19}, or uniform vectors in small balls relative to different norms~\cite{AC:CLMPR18,EPRINT:NOTT20}.

Let $C= N/8Bm$, which then satisfies the conditions of Lemma~\ref{lem:far}. Thus, for $\ev$ with entries in $[-C,C]^m$, the map $(g,\ev)\mapsto\Am^T\cdot\phi(g)+\ev$ is injective.

Let $\sigma\geq 16Bm/\epsilon\times\omega(\sqrt{\log\lambda})$ and $B'\geq \sigma\times\omega(\sqrt{\log\lambda})$ be polynomials. We will assume $N\geq 2B'$, which is always possible since we can take $N$ to be arbitrarily large. We will also for simplicity assume $N$ is even. This assumption is not necessary but will simplify some of the analysis, and is moreover without loss of generality since we can always make $N$ larger by multiplying it by arbitrary factors.

\begin{construction}\label{constr:alt}
	$\gen(1^\lambda)$: Initialize quantum registers $\Ss$ (for serial number) and $\Ms$ (for money) to states $|\Ds_{\sigma,B'}\rangle^{\otimes m}_\Ss$ and $|0\rangle_\Ms$, respectively. Then do the following:
	\begin{itemize}
		\item Apply in superposition the map $|\rv\rangle_\Ss|y\rangle_\Ms\mapsto |\rv\rangle_\Ss|y\oplus [(\sum_{i=1}^m r_i g_i)*x_\lambda])\rangle_\Ms$. The joint state of the system $\Ss\otimes\Ms$ is then \[\sum_{\rv\in\Z_N^m}\sqrt{\Ds_{\sigma',B}(\rv)}|\rv\rangle_\Ss|(\sum_{i=1}^m r_i g_i)*x_\lambda\rangle_\Ms=\sum_{g\in\G_\lambda}\left(\sum_{\rv\in\Z_N^m:\Am\cdot\rv=g}\sqrt{\Ds_{\sigma,B'}(\rv)}|\rv\rangle_\Ss\right)|g*x_\lambda\rangle_\Ms\]
		
		\item Apply $\QFT_{\Z_N^m}$ to $\Ss$. Using the QFT rules given above, this yields the state negligibly close to:
		\begin{align*}
			&\frac{1}{N^n}\sum_{g\in\G_\lambda}\left(\sum_{\sv,\ev\in\Z_N^n}\sqrt{\Ds_{N/\sigma,N/2-1}(\ev)}e^{i2\pi (g\cdot \sv)}|\Am^T\cdot\sv+\ev\rangle_\Ss\right)|g*x_\lambda\rangle_\Ms\\
			&=\frac{1}{|\G_\lambda|}\sum_{g\in\G_\lambda}\left(\sum_{h\in\G_\lambda,\ev\in\Z_N^n}\sqrt{\Ds_{N/\sigma,N/2-1}(\ev)}e^{i2\pi (g\cdot\phi(h))}|\Am^T\cdot\phi(h)+\ev\rangle_\Ss\right)|g*x_\lambda\rangle_\Ms\\
			&=\frac{1}{\sqrt{|\G_\lambda|}}\sum_{g\in\G_\lambda}\left(\frac{1}{\sqrt{|\G_\lambda|}}\sum_{h\in\G_\lambda,\ev\in\Z_N^n}\sqrt{\Ds_{N/\sigma,N/2-1}(\ev)}\chi(g,h)|\Am^T\cdot\phi(h)+\ev\rangle_\Ss\right)|g*x_\lambda\rangle_\Ms
		\end{align*}
		\item Measure $\Ss$, giving the serial number $\tv:=\Am^T\cdot\phi(h)+\ev$. $\ev$ is distributed negligibly close to $\Ds_{N/\sigma}$, meaning with overwhelming probability each entry is in $[-N/16Bm,N/16Bm]= [- C/2,C/2]\subseteq [-C,C]$. This means, to within negligible error, $\tv$ uniquely determines $\phi(h)$ and hence $h$.	Therefore, the $\Ms$ register then collapses to a state negligibly close to \[\frac{1}{\sqrt{|\G_\lambda|}}\sum_{g\in\G_\lambda}\chi(g,h)|g*x_\lambda\rangle_\Ms=:|\G_\lambda^h*x_\lambda\rangle\]
		Note that $h$ is unknown. Output $(\tv,|\G_\lambda^h*x_\lambda\rangle)$
	\end{itemize}

	\item $\ver(\tv,\$):$ First verify that the support of $\$$ is contained in $\Xs_\lambda$, by applying the assumed algorithm for recognizing $\Xs_\lambda$ in superposition. Then repeat the following $\lambda$ times:
	\begin{itemize}
		\item Initialize a new register $\Hs$ to $(|0\rangle_\Hs+|1\rangle_\Hs)/\sqrt{2}$. 
		\item Choose a random element $\xv\gets\Ds^*$.
		\item Apply to $\Hs\otimes\Ms$ in superposition the map
		\[{\sf Apply}|b\rangle_\Hs|y\rangle_\Ms\mapsto \begin{cases}|0\rangle_\Hs|y\rangle_\Ms&\text{ if }b=0\\|1\rangle_\Hs|(-\sum_i x_i g_i)*y\rangle_\Ms\enspace&\text{ if }b=1\end{cases}\]
		
		Since the entries of $\xv$ are bounded by $B$ which is polynomial, this step is efficient. 
		\item Measure $\Hs$ in the basis $B_{\tv,\xv}:=\{(|0\rangle_\Hs+e^{i2\pi \xv^T\cdot\tv/N}|1\rangle_\Hs)/\sqrt{2},(|0\rangle_\Hs-e^{i2\pi \xv^T\cdot\tv/N}|1\rangle_\Hs)/\sqrt{2}\}$, giving a bit $b_u\in\{0,1\}$. Discard the $\Hs$ register. 
		\item Accept if at least a fraction $7/8$ of the $b_u=0$ and the support of $\$$ is contained in $\Xs_\lambda$; otherwise reject.
	\end{itemize}
\end{construction}

\subsection{Accepting States of the Verifier}

We now analyze the correctness of the construction. 


\begin{theorem}\label{thm:rejectalternate} Let $|\psi\rangle$ be a state over $\Ms$. Then $\Pr[\ver(h,|\psi\rangle)=1]=\|\langle\psi |\G_\lambda^h*x_\lambda\rangle \|^2(1-2^{-\Omega(\sqrt{\lambda})}+2^{-\Omega(\sqrt{\lambda})}$.
\end{theorem}
\begin{proof}
For simplicity, we analyze the case of $|\psi\rangle=|\G_\lambda^{h'}*x_\lambda$, which form a basis for superpositions over $\Xs_\lambda$. In this case, Theorem~\ref{thm:rejectalternate} states that $|\G_\lambda^h*x_\lambda\rangle$ is accepted with probability $1-2^{\Omega(\sqrt{\lambda})}$, while $|\G_\lambda^{h'}*x_\lambda\rangle$ for $h'\neq h$ is accepted with probability $2^{\Omega(\sqrt{\lambda})}$. By a similar approach as in Theorem~\ref{thm:reject}, we can extend the analysis to all states.
	
If we let $u=\Am\cdot\xv=\sum_i x_i g_i$, then by the same analysis as in Construction~\ref{constr:main}, we have that applying ${\sf Apply}$ to the state $|\G_\lambda^{h'}*x_\lambda\rangle$ results in the state

\begin{align*}
	&\frac{1}{\sqrt{2}}\left(|0\rangle_\Hs+\chi(u,h')|1\rangle_\Hs\right)|\G_\lambda^{h'}*x_\lambda\rangle\\
	&=\frac{1}{\sqrt{2}}\left(|0\rangle_\Hs+e^{i2\pi u\cdot\phi(h')/N}|1\rangle_\Hs\right)|\G_\lambda^{h'}*x_\lambda\rangle\\
	&=\frac{1}{\sqrt{2}}\left(|0\rangle_\Hs+e^{i2\pi \xv^T\cdot\Am^T\cdot\phi(h')/N}|1\rangle_\Hs\right)|\G_\lambda^{h'}*x_\lambda\rangle
\end{align*}
Conditioned on sampling $u$, $\Pr[b_u=0]$ is the inner product squared of $\left(|0\rangle_\Hs+e^{i2\pi \xv^T\cdot\Am^T\cdot\phi(h')/N}|1\rangle_\Hs\right)/\sqrt{2}$ with the basis state $\left(|0\rangle_\Hs+e^{i2\pi \xv\cdot\tv/N}|1\rangle_\Hs\right)/\sqrt{2}$. This is:
\begin{align*}
	\Pr[b_u=0]&=\frac{1}{4}\left\|1+e^{i2\pi (\xv^T\cdot\Am^T\cdot\phi(h')-\xv^T\cdot\tv)/N}\right\|^2\\
	&=\frac{1}{2}\left(1+\cos\left[2\pi (\xv^T\cdot\Am^T\cdot\phi(h')-\xv^T\cdot(\Am^T\cdot\phi(h)+\ev))/N\right]\right)\\
	&=\frac{1}{2}\left(1+\cos\left[2\pi (\xv^T\cdot\Am^T\cdot\phi(h'-h)+\xv^T\cdot\ev)/N\right]\right)
\end{align*}


In the case $h=h'$, $\Pr[b_u=0]=\frac{1}{2}\left(1+\cos\left[2\pi\xv^T\cdot\ev/N\right]\right)$. We have that $|2\pi\xv^T\cdot\ev/N|\leq \pi /8$. Using the fact that $\cos(x)\geq 1-x^2/2$, we therefore have that $\Pr[b_u=0]\geq 1-\pi^2/256=0.9614\ldots=7/8+\Omega(1)$. Then via standard concentration inequalities, after $\lambda$ trials, except with probability $2^{-\Omega(\sqrt{\lambda})}$, at least $7/8$ of the $b_u$ will be 0. Therefore, $\ver$ accepts with probability $1-2^{-\Omega(\sqrt{\lambda})}$.

On the other hand, if $g\neq g'$, then $\xv^T\Am^T$ is statistically close to uniform in $\G_\lambda$, and so $\xv^T\cdot\Am^T\cdot\phi(h'-h)$ is statistically close to uniform in a non-trivial subgroup $\G'$ of $\Z_N$. By Lemma~\ref{lem:bigelements} and our assumption that $N$ is even, at least half of the elements of $\Z_N$ are at least $N/4$ in absolute value. In particular, this means $\Pr[|\xv^T\cdot\Am^T\cdot\phi(h'-h)|\geq N/4]\geq 1/2-\negl$. On the other hand, $|\xv^T\cdot\ev|\leq N/16$ always. This means $\|\xv^T\cdot\Am^T\cdot\phi(h'-h)+\xv^T\cdot\ev\|\geq N/4-N/16$ with probability at least $1/2-\negl$. In this case, we can use that $\cos(\pi/2+x)\leq |x|$ to bound $\cos\left[2\pi (\xv^T\cdot\Am^T\cdot\phi(h'-h)+\xv^T\cdot\ev)/N\right]\leq 2\pi/16=\pi/8$, meaning $\Pr[b_u=0]\leq 1/2+\pi/16$. Averaging over all $u$, we therefore have that: $\Pr[b_u=0]\leq \frac{3}{4}+\pi/32+\negl=0.8481\ldots=7/8-\Omega(1)$. Then via standard concentration inequalities, after $\lambda$ trials, except with probability $2^{-\Omega(\sqrt{\lambda})}$, fewer than $7/8$ of the $b_u$ will be 0. therefore, $\ver$ accepts with probability $2^{-\Omega(\sqrt{\lambda})}$.
\end{proof}

\subsection{Security}

Here, we state the security of Construction~\ref{constr:alt}. 

\paragraph{Assumptions.} We first need to define slight variants of our assumptions, in order to be consistent with the more limited structure of a REGA. For example, in the ordinary Discrete Log assumption (Assumption~\ref{def:dlog}), the challenger computes $y=g*x$ for a random $g$, and adversary produces $g$. But the adversary cannot even tell if it succeeded since it cannot compute the action of $g$ in general. Instead, the adversary is required not to compute $g$, but instead to compute any short $\xv$ such that $g=\sum_i x_i g_i$. The adversary can then check that it has a solution by computing the action of $g$ using its knowledge of $\xv$. We analogously update each of our assumptions to work with the limited ability to compute the group action on REGAs.

As above, let $\G_\lambda,\Xs_\lambda,*$ be a REGA, and $\Ts=(g_1,\dots,g_m)$ a set such that $*$ can be efficiently computed for $g_i$ and $g_i^{-1}$. Let $\Ds^*,B$ be as in Assumption~\ref{assump:uniform}.

\begin{assumption}\label{def:REGAqkgea} The \emph{REGA quantum modified knowledge of group element assumption} (REGA-Q-KGEA) holds on a group action $(\G,\Xs,*)$ if the following is true. For any quantum polynomial time (QPT) adversary $\As$ which performs no measurements except for its final output, there exists a polynomial $C$, a QPT extractor $\Es$ with outputs in $[-C,C]^m$, and negligible $\epsilon$ such that 
	\[\Pr\left[y\in\Xs\wedge y\neq g*x_\lambda:\substack{(y,|\psi\rangle)\gets\As(1^\lambda)\\\xv\gets\Es(y,|\psi\rangle)}\\g\gets\sum_i x_i g_i\right]\leq\epsilon(\lambda)\enspace .\]
\end{assumption}

As with the non-REGA Q-KGEA assumption, we expect the REGA-Q-KGEA assumption is likely false. Certainly it is false on group actions with oblivious sampling. However, we note that it is unclear if our attack from Theorem~\ref{thm:kgeaattack} can be adapted to REGAs. Nevertheless, to mitigate any risks associated with the plain REGA-Q-KGEA assumption, we can likewise define a \emph{modified} REGA KGEA assumption (REGA-Q-mKGEA), in the same spirit as Assumption~\ref{def:qmkgea}.

We next define our REGA analog of Assumption~\ref{def:dlogminimalcdh}.

\begin{assumption}\label{def:REGAdlogminimalcdh} We say that the \emph{REGA Discrete Log with a single minimal CDH query} assumption (REGA-DLog/1-minCDH) assumption holds if the following is true. For any QPT adversary $\As$ playing the following game, parameterized by $\lambda$, there is a negligible $\epsilon$ such that $\As$ wins with probability at most $\epsilon(\lambda)$:
	\begin{itemize}
		\item The challenger, on input $\lambda$, chooses a random $g\in\G_\lambda$. It sends $\lambda$ to $\As$
		\item $\As$ submits a superposition query $\sum_{y\in\Xs,z\in\{0,1\}^*}\alpha_{y,z}|y,z\rangle$. Here, $y$ is a set element that forms the query, and $z$ is the internal state of the adversary when making the query. The challenger responds with $\sum_{y\in\Xs,z\in\{0,1\}^*}\alpha_{y,z}|(-g)*y,z\rangle$. 
		\item The challenger sends $g*x$ to $\As$.
		\item $\As$ outputs a $\xv\in\Z^m$, encoded in unary. It wins if $g=\sum_i x_i g_i$.
	\end{itemize}
\end{assumption}

Note that the challenger in Assumption~\ref{def:REGAdlogminimalcdh} is inefficient on a REGA. However, under Assumption~\ref{assump:uniform}, the challenger can be made efficient by first sampling $\yv\gets\Ds^*$ and then computing $g=\sum_i y_i g_i$.



\begin{theorem}\label{thm:alt} Assuming REGA-DLog/1-minCDH (Assumption~\ref{def:REGAdlogminimalcdh}) and REGA-Q-KGEA (Assumption~\ref{def:REGAqkgea}) (or more generally, REGA-Q-mKGEA) both hold on a group action $(\G,\Xs,*)$, then Construction~\ref{constr:alt} is a quantum lightning scheme. Alternatively, if D2X/min (Assumption~\ref{def:D2X/min}) holds on a group action with $\Xs\subseteq\{0,1\}^m$, then Construction~\ref{constr:alt} is a quantum lightning scheme in the generic group action model $\GGAM_{\G,m'}$ with label length $m'$.
\end{theorem}

%Notice that in the second part, DLog/1-minCDH and DLog/oblivious are for the original non-REGA variants. This is because with non-uniform advice, one can turn a REGA into an EGA, by computing a trapdoor that allows for representing all elements of $\G$ as small linear cominations of the elements of $\Ts$.

We only sketch the proof. Like in the proof of Theorems~\ref{thm:main} and~\ref{thm:main2}, we can assume the adversary wins the quantum lightning experiment with probability $1-\negl(\lambda)$. In order for a supposed note $\$$ to be accepted relative to serial number $\tv$ with overwhelming probability, $\tv$ must have the form $\tv=\Am^T\cdot\phi(h)+\ev$ for ``short'' $\ev$, and $\$$ must be negligibly close to $|\G_\lambda^h*x_\lambda\rangle$. Therefore, a quantum lightning adversary outputs two copies of $|\G_\lambda^h*x_\lambda\rangle$ for some $h$. The security reduction of Theorem~\ref{thm:main} did not rely on knowing $h$, just that the adversary outputted two copies of $|\G_\lambda^h*x_\lambda\rangle$. Hence, a near-identical proof holds for Construction~\ref{constr:alt}. The only difference is that when the extractor $\Es$ outputs a group element, it instead outputs a small linear combination of the $g_i$ giving that group element, and then the DLog/1-minCDH adversary uses this small representation to compute the action by that group element.


