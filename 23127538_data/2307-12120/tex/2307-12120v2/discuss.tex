\section{Further Discussion}\label{sec:discuss}

\subsection{Quantum Group Actions}

Here, we consider a generalization of group actions where set elements are replaced with quantum states.

An quantum (abelian) group action consists of a family of (abelian) groups $\G=(\G_\lambda)_\lambda$ (written additively), a family $\Xs=(\Xs_\lambda)_\lambda$ of sets $\Xs_\lambda$ of quantum states over a system $\Ms_\lambda$, and an operation $*$. We will require that the states in $\Xs_\lambda$ are orthogonal. $*$ is a quantum algorithm that takes as input a group element $g\in\G_\lambda$ and a quantum state $|\psi\rangle$ over $\Ms_\lambda$, and outputs another state over $\Ms_\lambda$. $*$ satisfies the following properties:
\begin{itemize}
	\item {\bf Identity:} If $0\in \G_\lambda$ is the identity element, then $|0\rangle*|\psi\rangle=|\psi\rangle$ for any $|\psi\rangle\in \Xs_\lambda$.
	\item {\bf Compatibility:} For all $g,h\in \G_\lambda$ and $|\psi\rangle\in \Xs_\lambda$, $(g+h)*|\psi\rangle=g*(h*|\psi\rangle)$.
\end{itemize}
We can also relax the above properties to only hold to within negligible error, and/or relax the orthogonality requirement to being near-orthogonal. We will additionally require the following properties:
\begin{itemize}
	\item {\bf Efficiently computable:} There is a pseudodeterministic QPT procedure ${\sf Construct}$ which, on input $1^\lambda$, outputs a description of $\G_\lambda$ and an specific element $|\psi_\lambda\rangle\in\Xs_\lambda$. The operation $*$ is also computable by a QPT algorithm.
	\item {\bf Efficiently Recognizable:} There is a QPT procedure ${\sf Recog}$ which recognizes elements in $\Xs_\lambda$. That is, ${\sf Recog}(1^\lambda,\cdot)$ projects onto the span of the states in $\Xs_\lambda$.
	\item {\bf Regular:} For every $|\phi\rangle\in\Xs_\lambda$, there is exactly one $g\in\G_\lambda$ such that $|\phi\rangle=g*|\psi_\lambda\rangle$. 
\end{itemize}
Again, we can also relax the above properties to only hold to within negligible error.

\paragraph{Cryptographic group actions.} At a minimum, a cryptographically useful quantum group action will satisfy the following discrete log assumption:
\begin{assumption}\label{def:qdlog} The \emph{discrete log assumption} (DLog) holds on a quantum group action $(\G,\Xs,*)$ if, for all QPT adversaries $\As$, there exists a negligible $\lambda$ such that 
	\[\Pr[\As(g*|\psi_\lambda\rangle)=g:g\gets\G_\lambda]\leq\negl(\lambda)\enspace .\]
\end{assumption}

Note that if we do not insist on orthogonality of the states in $\Xs_\lambda$, then it is trivial to construct a quantum group action in which DLog holds: simply have all $|\psi\rangle\in\Xs_\lambda$ be identical, or negligibly close. Then it will be information-theoretically impossible to determine $g$. Orthogonality essentially says that the group action is classical, except that the basis for the set elements is potentially different than the computational basis.


\subsection{Quantum Group Actions From Lattices}

Here, we describe a simple quantum group action from lattices.

The group $\G_{\sf LWE,N,n,m,\sigma}$ will be set to $\Z_N^n$ for some integers $N,n$. We will fix a short wide matrix $\Am\in\Z_N^{n\times m}$; we can think of $\Am$ as being sampled randomly and included in a common reference string. Note that $\G$ is independent of $\sigma$, but we include it for notational consistency.

The set $\Xs_{\sf LWE,N,n,m,\sigma}$ will be the set of states $|\psi_\sv\rangle=\sum_{\ev\in\Z_N^n}\sqrt{\Ds_{\sigma,N/2}(\ev)}|\Am^T\cdot\sv+\ev\rangle$. In other words, we take the discrete Gaussian vector superposition of some width, and add the vector $\Am^T\cdot\sv$.

$\G_{\sf LWE,N,n,m,\sigma}$ acts on $\Xs_{\sf LWE,N,n,m,\sigma}$ in the following obvious way: $\rv*|\psi_\sv\rangle=|\psi_{\rv+\sv}\rangle$, which can be computed by simply adding $\Am^T\cdot\rv$ in superposition.

We have the following theorem:
\begin{theorem}\label{def:qdlogfromlwe} Let $\sigma,\sigma_0$ be non-negative real numbers such that $\sigma/\sigma_0$ is super-logarithmic. Assuming the Learning with Errors problem is hard for noise distribution $\Ds_{\sigma_0}$, discrete logarithms are hard in the group action $(\G_{\sf LWE,N,n,m,\sigma},\Xs_{\sf LWE,N,n,m,\sigma},*)$.
\end{theorem}
\begin{proof}The learning with errors assumption states that it is hard to compute $\sv$ given $\Am^T\cdot\sv+\ev$ with $\ev$ sampled from $\Ds_{\sigma_0}$. We need to show that it is hard to compute $\sv$ given the analogous superposition over $\Am^T\cdot\sv+\ev$, where here $\ev$ comes from the Gaussian superposition $|\Ds_\sigma\rangle$. The idea is a simple application of noise flooding: given $\uv=\Am^T\cdot\sv+\ev$, compute the state $|\psi_\sv'\rangle:=\sum_{\ev'\in\Z_N^n}\sqrt{\Ds_{\sigma,N/2}(\ev')}|\Am^T\cdot\sv+\ev+\ev'\rangle$. Since $\sigma/\sigma_0$ is super-polynomial, $\ev+\ev'$ where $\ev'\gets \Ds_{\sigma,N/2}$ is negligibly close to a Gaussian centered at 0. Therefore, $|\psi_\sv'\rangle$ is negligibly close to $|\psi_\sv\rangle$. Plugging into a supposed DLog adversary then gives $\sv$, breaking LWE.
\end{proof}

Unfortunately, this LWE-based group action is missing a crucial feature: it is not possible to recognize states in $\Xs$. In particular, the states in $\Xs$ are indistinguishable from states of the form $\sum_{\ev\in\Z_N^n}\sqrt{\Ds_{\sigma,N/2}(\ev)}|\vv+\ev\rangle$, where $\vv$ is an arbitrary vector in $\Z_N^m$. As we will see in the next sub-section, the inability to recognize $\Xs$ will prevent us from using this group action to instantiate our quantum money scheme.



\subsection{Relation to Quantum Money Approaches based on Lattices}

Here, we see that our quantum money scheme is conceptually related to a folklore approach to building quantum money from lattices. This approach has not been able to work; in our language, the reason is exactly due to the inability to recognize $\Xs_{\sf LWE,N,n,m,\sigma}$. 

The approach is the following. Let $\Am\in\Z_N^{n\times m}$ be a random short wide matrix over $\Z_n$. To mint a banknote, construct the discrete Gaussian superposition $|\Ds_\sigma\rangle^{\otimes m}$ in register $\Ms$. Then compute and measure $\Am\cdot \xv$ applied to $\Ms$. The result is a vector $\hv\in\Z_N^n$, which will be the serial number, and $\Ms$ collapses to a superposition $|\$_\hv\rangle\propto \sum_{\xv:\Am\cdot\xv=\hv} \sqrt{\Ds_\sigma(\xv)}|\xv\rangle$ of short vectors $\xv$ such that $\Am\cdot\xv=\hv$. This is the banknote. A simple argument shows that it is impossible to construct two copies of $|\$_\hv\rangle$ for the same $\hv$: given such a pair, measure each to get $\xv,\xv'$ such that $\Am\cdot\xv=\Am\cdot\xv'=\hv$. Then subtract to get a short vector $\xv-\xv'$ such that $\Am\cdot(\xv-\xv')=0^n$. We can conclude $\xv-\xv'$ is non-zero with overwhelming probability, since the measurement of $|\$_\hv\rangle$ has high entropy. Such a non-zero short kernel vector would solve the Short Integer Solution (SIS) problem, which is widely believed to be hard and is the foundation of lattice-based cryptography.

Unfortunately, the above approach is broken. The problem is that there is no way to actually verify banknotes. One can verify that a banknote has support on short vectors with $\Am\cdot\xv=\hv$, but it is impossible to verify that the banknote is in superposition. If one could solve the Learning with Errors (LWE) problem, it would be possible to verify banknotes as follows: first perform the QFT to the banknote state. If an honest banknote, the QFT will give a state negligibly close to
\begin{equation}\label{eqn:banknote}|\$_\hv'\rangle:=\frac{1}{N^{n/2}}\sum_{\sv,\ev\in\Z_N^n}\sqrt{\Ds_{N/\sigma}(\ev)}e^{i2\pi \hv\cdot\sv/N}|\Am^T\cdot\sv+\ev\rangle\enspace .\end{equation}
The second step is to simply apply the supposed LWE solver to this state in superposition, ensuring that the state has support on vectors of the form $\Am^T\cdot\sv+\ev$ for small $\ev$. 

Unfortunately, LWE is likely hard. In fact, it is quantumly equivalent to SIS~\cite{STOC:Regev05}, meaning if one could verify banknotes using an LWE solver, then SIS is easy. Not only does this mean we are reducing from an easy problem, but it would be possible to turn such a SIS algorithm into an attack.

Without the ability to verify that banknotes are in supeprosition, the attacker can simply measure a banknote to get $\xv$, and then pass off $|\xv\rangle$ as a fake banknote that will pass verification. Since $\xv$ is trivially copied, this would break security. Interestingly,~\cite{C:LiuZha19} prove that, no matter what efficient verification procedure is used, even if the verification diverged from the LWE-based approach above, this attack works.~\cite{EC:LiuMonZha23} extend this to a variety of potential schemes based on similar ideas, including a recent proposed instantiation of this approach by~\cite{KLS22}.

\medskip

We now see how the above approach is essentially equivalent to our construction of quantum money from group actions, instantiated over our LWE-based quantum group action. The inability to recognize $\Xs$ is the reason this instantiation is insecure, despite natural hardness assumptions presumably holding on the group action.

We consider the quantum group action $(\G_{\sf LWE,N,n,m,N/\sigma},\Xs_{\sf LWE,N,n,m,N/\sigma},*)$, where $\sigma$ is from the folklore construction above. When applied to $(\G_{\sf LWE,N,n,m,N/\sigma},\Xs_{\sf LWE,N,n,m,N/\sigma},*)$, a banknote in our scheme, up to negligibly error from truncating discrete Gaussians, is the state $|\$_\hv'\rangle$ from Equation~\ref{eqn:banknote} above, where the serial number is $\hv$. Thus, we see that our quantum money scheme is simply the folklore construction but moved to the Fourier domain. The attack on the folklore construction can therefore easily be mapped to an attack on our scheme: if the adversary is given $|\$_\hv'\rangle$, it measures in the Fourier domain (which is the primal domain for the folklore construction) to get a short vector $\xv$ such that $\Am\cdot\sv=\hv$. Then it switched back to the primal domain, giving the state
\[\frac{1}{N^{m/2}}\sum_\uv e^{i2\pi\ev\cdot\xv}|\xv\rangle\]
This is a state that lies outside the span of $\Xs$. However, no efficient verification procedure can distinguish it from an honest banknote state.

\medskip

Two features that distinguish isogeny-based group actions from the LWE-based action above. The first is the ability to recognize elements in $\Xs$. Suppose it were possible to recognize elements of $\Xs$ in the LWE-based action, and we had the verifier check to see if the banknote belonged to the span of the elements in $\Xs$. In the language of quantum group actions, this check would prevent the attacker from sending $\frac{1}{N^{m/2}}\sum_\uv e^{i2\pi\ev\cdot\xv}|\xv\rangle$, which lies outside the span of $\Xs$. In the language of the folklore construction, this check would correctly distinguish between an honest banknote and the easily clonable state $|\xv\rangle$ in the attack. If such a check were possible, the proof sketched above would work to base the security of the scheme on SIS. Unfortunately, such a check is computationally intractable under the decision LWE problem, which is equivalent to SIS and most likely hard.

The issue of recognizing set elements is also crucial in our security arguments. Indeed, the first step in our proof was to characterize the states accepted by the verifier, showing that only honest banknote states are accepted. This step in the proof fails in the LWE-based scheme, which would prevent the proof from going through. Thus, even though the scheme based on LWE is broken, it does not contradict our DLog/1-minCDH and Q-KGEA assumptions holding on the LWE-based group action.

The second difference, is that, with the LWE-based group action, taking the QFT of money states gives elements with meaningful structure: short vectors $\xv$ such that $\Am\cdot\xv=\hv$. This structure and it's relation to the original money state are what enables the attack. In contrast, taking the QFT of money states over $\Xs$ coming from isognies will give terms with no discernible structure.

We believe the above perspective adds to the confidence in our proposal. Indeed, in the LWE-based scheme, the key missing piece is recognizing set elements; if not for this missing piece the scheme \emph{could} be proven secure. By switching to group actions based on isogenies, we add the missing piece. The hope is that even though the source of hardness is now from hard problems on isogenies over elliptic curves instead of lattices, by adding the missing piece we can finally obtain a secure scheme.
