\section{Introduction}\label{sec:intro}

Quantum money, first envisioned by Wiesner~\cite{Wiesner83}, is a system of money where banknotes are quantum states. By the no-cloning theorem, such banknotes cannot be copied, leading to un-counterfeitable currency. A critical feature of quantum money, identified by~\cite{CCC:Aaronson09}, is \emph{public verification}, allowing anyone to verify while only the mint can create new banknotes. Such public key quantum money is an important central object in the study of quantum protocols, but unfortunately convincing constructions have remained elusive. See Section~\ref{sec:related} for a more thorough discussion of prior work in the area.

\paragraph{This Work.} We construct public key quantum money from abelian group actions, which can be instantiated by suitable isogenies over ordinary elliptic curves. Group actions, and the isogenies they abstract, are one of the leading contenders for post-quantum secure cryptosystems. Our construction could plausibly even be quantum lightning, a strengthening of quantum money with additional applications. Our construction is arguably the first time group actions have been used to solve a classically-impossible cryptographic task that could not already be solved using other standard tools like LWE. Our construction is sketched in Section~\ref{sec:constroverview} below, and given in detail in Section~\ref{sec:constr}.

While our main construction can be instantiated on a clean abelian group action --- often referred to as an ``effective group action'' (EGA) --- many isogeny-based group actions diverge from this convenient abstraction. We therefore provide an alternative candidate scheme which can be instantiated on so-called ``restricted effective group actions'' (REGAs); see Section~\ref{sec:alternate} for details. We prove the quantum lightning security of our protocols in the generic group action model, under a new but natural strengthening of the discrete log assumption on group actions. Note that generic group actions cannot be used to give unconditional quantum hardness results, so some additional computational assumption is necessary. In order to prove our result, we develop a new toolkit for quantum generic group action proofs; see Section~\ref{sec:ggam}. We believe ours is the first proof of security in the generic group action model. 

Along the way, we explore knowledge assumptions and algebraic group actions in the quantum setting, finding significant limitations of these assumptions/models compared to generic group actions. Specifically, unlike the classical setting where knowledge assumptions typically hold unconditionally against generic attacks, we explain why such statements likely do not hold quantumly. In the specific case of group actions, we indeed show an efficient generic attack on an analog of the ``knowledge of exponent'' assumption. This potentially casts doubt on quantum knowledge assumptions in general. We do give a more complex definition that avoids our attack, but it is unclear if the assumption is sound and more analysis is needed. For completeness, we give an alternative proof of security for our construction under this new knowledge assumption. 

We also discuss an algebraic model for group actions, which can be seen as a variant of the knowledge of exponent assumption. Unlike the classical setting where algebraic models live ``between'' the fully generic and standard models, we find that the algebraic group action model is likely incomparable to the generic group action model, and security proofs in the model are potentially problematic. As these issues do not appear for generic group actions, we therefore propose that generic group actions are the preferred idealized model for analyzing cryptosystems. See Section~\ref{sec:knowledge} for details.

We conclude in Section~\ref{sec:discuss} with a discussion of possible generalizations and relation to approaches for building quantum money from LWE.

\subsection{Our Construction}\label{sec:constroverview}

\paragraph{Abelian Group Actions.} We will use additive group notation for abelian groups. An abelian group action consists of an abelian group $\G$ and a set $\Xs$, such that $\G$ ``acts'' on $\Xs$ through the binary relation $*:\G\times\Xs\rightarrow\Xs$ with the property that $g*(h*x)=(g+h)*x$ for all $g,h\in\G,x\in\Xs$. We will also assume a \emph{regular} group action, which means that for every $x\in\Xs$, the map $g\mapsto g*x$ is a bijection. 

The main group actions used in cryptography are those arising from isogenies over elliptic curves. For example, see~\cite{EPRINT:Couveignes06,EPRINT:RosSto06,AC:CLMPR18,AC:BeuKleVer19,PKC:DFKLMPW23}. %Here, $\Xs$ is a set of elliptic curves, and $\G$ is a class group, acting on $\Xs$ via isogenies. 
Group action cryptosystems rely at a minimum on the assumed hardness of discrete logarithms: given $x,y=g*x\in\Xs$, finding $g$. For isogeny-based actions, this corresponds to the hard problem of computing isogenies between elliptic curves. Other hard problems are possible, such as analogs of computational/decisional Diffie-Hellman, and more.



\paragraph{The QFT.} Our quantum money scheme will utilize the quantum Fourier transform (QFT) over general abelian groups. This is a quantum procedure that maps
\[|g\rangle\mapsto\frac{1}{\sqrt{|\G|}}\sum_{h\in\G}\chi(g,h)|h\rangle\enspace .\]
Here, $\chi$ is some potentially complex phase term. In the case of $\G$ being the additive group $\Z_N$, $\chi(g,h)$ is defined as $e^{i2\pi gh/N}$, with a slightly more complicated definition for non-cyclic groups\footnote{Remember that the group aoperation is $+$, so $gh$ in the exponent is not the group operation, but instead multiplication in the ring $\Z_N$.}. The main property we need from $\chi$ (besides making the QFT unitary) is that it is \emph{bilinear}, in the sense that $\chi(g,h_1+ h_2)=\chi(g,h_1)\cdot\chi(g,h_2)$. It is also symmetric: $\chi(g,h)=\chi(h,g)$.


\paragraph{Our Quantum Money Scheme.} Our quantum money scheme is as follows; see Section~\ref{sec:constr} for additional details.

\begin{itemize}
	\item $\gen$: initialize a register in the state $\frac{1}{\sqrt{|\G|}}\sum_{g\in\G}|g\rangle$, which can be computed by applying the QFT to $|0\rangle$. Let $x\in\Xs$ be arbitrary. Then by computing the group action in superposition, compute $\frac{1}{\sqrt{|\G|}}\sum_{h\in\G}|g\rangle|g*x\rangle$. Next, apply the QFT over $\G$ to the first register. The result is:
	\[\frac{1}{|\G|}\sum_{g,h\in\G} \chi(g,h)|h\rangle|g*x\rangle=\frac{1}{\sqrt{|\G|}}\sum_h |h\rangle|\G^h*x\rangle\]
	Here, $|\G^h*x\rangle$ is the state $\frac{1}{\sqrt{|\G|}}\sum_{g\in\G}\chi(g,h)|g*x\rangle$. Note that $|\G^h*x\rangle$ is, up to an overall phase, independent of $x$. 
	
	Now measure $h$, in which case the second register collapses to $|\G^h*x\rangle$. Output $h$ as the serial number, and $|\G^h*x\rangle$ as the money state.
	\item To verify a banknote $\$$, choose a random $u\in\G$, and initialize a new qubit with $(|0\rangle+|1\rangle)/\sqrt{2}$. Then apply the controlled group action $|b,y\rangle\mapsto\begin{cases}|0,y\rangle&\text{ if }b=0\\|1,u*y\rangle&\text{ if }b=1\end{cases}$. If $\$$ is the honest banknote state, then the state of the system becomes:
	\[\frac{1}{\sqrt{2}}\left(|0\rangle+\chi(u,h)^{-1}|1\rangle\right)|\G^h*x\rangle\]
	We can then measure the first qubit in the basis containing $|0\rangle+\chi(u,h)^{-1}|1\rangle$, which will accept with probability 1 for honest banknote states. We can repeat this process $\lambda$ times, and accept only if all trials accept. It is possible to show that if all $\lambda$ trials accept, the result state is $2^{\Theta(\lambda)}$-close to the honest banknote state.
\end{itemize}

\paragraph{An instantiation using REGAs.} For some isogeny-based group actions such as CSIDH~\cite{AC:CLMPR18}, the operation $*$ is only efficiently computable for a very small set $S\subseteq\G$ of group elements. Such group actions are called ``restricted effective group actions'' (REGAs)~\cite{AC:ADMP20}. Above, however, we see that we need to compute the group action on all possible elements in $\G$, both for minting and for verification. We therefore give a variant of the construction above which only uses the ability to compute $*$ for elements in $S$. We show that we are still able to sample $|\G^h*x\rangle$, but now the serial number has the form $\Am^T h+\ev\bmod N$ for a known matrix $\Am$ and a ``small'' $e\in\Z^n$\enspace\footnote{Here, we are interpreting $h$ a vector in $\Z_N^n$ for some $n,N$, which is possible since $\G$ is abelian.}. Under plausible assumptions, the serial number actually hides $h$\enspace\footnote{This is the Learning with Errors (LWE) problem~\cite{STOC:Regev05} which is widely believed to be hard for \emph{random} $\Am$. In our case, $\Am$ is a fixed matrix that depends on the group action, and LWE may or may not be hard for this $\Am$. However, if LWE is easy for this $\Am$, then we in fact have a plain group action. Indeed, a variant of Regev's quantum reduction between LWE and Short Integer Solution (SIS)~\cite{STOC:Regev05}, outlined by~\cite{FOCS:YamZha22}, shows that if LWE can be solved relative to $\Am$, then SIS can be solved for $\Am$ as well. It is straightforward to adapt this reduction to solve the Inhomogenous SIS (ISIS) problem, which then allows for computing the group action for all of $\G$. In this case we would have a clean group action and would not need this alternate construction.}. We nevertheless show that we can use such a noisy serial number for verification. For details, see Section~\ref{sec:alternate}. The security of our alternate scheme is essentially equivalent to the main scheme.



%\paragraph{An alternate scheme.} We discuss security more thoroughly in Sections~\ref{sec:introknowledge} and~\ref{sec:introsec} below, but here note that there may be some group actions, in particular where $\Xs$ is dense in the set of bit-strings of a certain length, where the above scheme will fail to be secure as quantum \emph{lightning} security~\cite{EC:Zhandry19b}. Quantum lightning requires that it is hard to construct two banknotes with the same serial number, \emph{even for the mint}. We therefore design a different quantum money scheme for such group actions that plausibly overcomes the attack. Details of the attack and alternate scheme are given in Sections~\ref{sec:compserial} and~\ref{sec:alternate}. Note that this attack does not appear to violate the basic quantum money guarantee of the above scheme. Moreover, it does not apply to group actions based on isogenies, which are not dense.


\subsection{The security of our scheme}\label{sec:introsec}

We do not know how to base the security of our schemes on any standard assumptions on isogenies. However, we are able to prove the security of our scheme in a generic group action model (GGAM), an analog of the generic group model~\cite{EC:Shoup97,IMA:Maurer05} adapted to group actions. Generic models for group actions have been considered previously~\cite{AC:MonZha22,EC:BonGuaZha23,EPRINT:OrsZan23,PKC:DHKKLR23}. However, to the best of our knowledge, ours is the first time the model has been used to prove security against quantum attacks.

The challenge with the quantum GGAM is that the query complexity of computing discrete logarithms is actually polynomial~\cite{EttHoy00}. This means we cannot rely on query complexity alone to justify hardness, and must additionally make computational assumptions. This is in contrast to the classical setting, where the generic group (action) model allows for unconditional proofs of security by analyzing query complexity alone. In fact, most if not all generic group model proofs from the classical setting are unconditional query complexity proofs. This means that proofs in the quantum GGAM will look very different than classical proofs in the GGM/GGAM; in particular, proofs will require a reduction from the underlying hard problem. At the same time, in order to take advantage of the generic oracle setting, it would seem that quantum query complexity arguments are still needed. But a priori, it may not be obvious how to leverage query complexity in any useful way, given the preceding discussion.

\paragraph{Our Framework.} In Section~\ref{sec:ggam}, we develop a new framework to help in the task of proving quantum hardness results relative to generic group actions. To illustrate our ideas, we first start with the following task. We want to show that discrete logarithms remain hard, even if the adversary is given quantum oracle access to the function that maps $g*x$ to $(-g)*x$ (for some fixed starting set element $x$). This is an important setting in isogeny-based group actions, as these negation queries correspond to computing twists of elliptic curves. We want to prove generic hardness of this problem, assuming only plausible computational assumptions on a group action where such negation queries are \emph{not} permitted. 

Suppose toward contradiction that there was a generic adversary which could utilize negation queries to solve discrete logarithms. Let $(*,\G,\Xs)$ be a plain group action where negation queries are not allowed. We will define a new group action $(\star,\G,\Xs')$ as follows. First sample a random injection $\Pi:\Xs^2\rightarrow\{0,1\}^m$ whose inputs are \emph{pairs} of set elements. Then define $\Xs'$ as the image under $\Pi$ of pairs of the form $(g*x,(-g)*x)$. $\star$ acts in the natural way: $g\star \Pi(y,z)=\Pi(g*y,(-g)*z)$.

Our reduction will sample a $\Pi$\enspace\footnote{A random injection is exponentially large and cannot be sampled efficient. Instead, the reduction will actually efficiently simulate a random injection $\Pi$ using known techniques. For the purposes of our discussion here, we can ignore this issue.} and run the generic adversary on the new group action, using its knowledge of $\Pi$ and its inverse to implement the action $\star$. Notice now that our reduction also has the ability to compute negations: given $\Pi(y,z)$ where $y=g*x$ and $z=(-g)*x$, the negation of $\Pi(y,z)$ is exactly the element $\Pi(z,y)$ obtained by swapping $y$ and $z$. Thus, our reduction is able to simulate the negation queries, even though the underlying group action does not support efficient negations. This is our main idea, though there are a couple lingering issues to sort out:
\begin{itemize}
	\item The reduction cannot perfectly simulate $(\star,\G,\Xs')$. The issue is that there are elements $\Pi(y,z)$ where $y,z$ do not have the form $y=g*x,z=(-g)*x$ for some $g$. In the group action $(\star,\G,\Xs')$, these elements will be identified as invalid set elements. On the other hand, while our reduction can carry out the correct computation on $y,z$ of the correct form, it will be unable to distinguish such $y,z$ from ones of the incorrect form, and will act on these elements even though they are incorrect. As such, there will be elements that are not in $\Xs'$ that the reduction will nevertheless falsely identify as valid set elements. We resolve this problem by choosing the images of $\Pi$ to be somewhat sparse, by setting the output length $m$ sufficiently large. Our reduction only provides the adversary elements corresponding to valid $y,z$, and we can show, roughly, that the adversary has a negligible chance of computing elements in the image of $\Pi$ that correspond to invalid $y,z$. This follows from standard query complexity arguments. Thus, we are able to simulate with negligible error the correct group action $(\star,\G,\Xs')$.
	\item We have not yet specified what problem the reduction actually solves. The problem we would like to solve is the plain discrete logarithm on $(*,\G,\Xs)$, where the reduction is given $g*x$, and must compute $g$. However, it is unclear what challenge the reduction should give to the adversary. The natural approach is to try to give the adversary $\Pi(g*x,(-g)*x)$, which is just the discrete log instance relative to $(\star,\G,\Xs')$ with the same solution $g$. However, this requires knowing $(-g)*x$, which is presumably hard to compute given just $g*x$ (remember that negation queries are not allowed on $(*,\G,\Xs)$). Our solution is to simply use a slight strengthening of discrete logarithms, where the adversary is given $(g*x,(-g)*x)$ and must compute $g$. Under the assumed hardness of this strengthened discrete log problem (again, in ordinary group actions where negations are presumed hard), we can complete the reduction and prove the generic hardness of discrete logarithms in the presence of negation queries.
\end{itemize}

\paragraph{The security of our money scheme.} We now turn to using our framework to prove the security of our quantum money scheme in the GGAM. Inspired by our negation example above, we will simulate a generic group action $(\star,\G,\Xs')$ using an injection $\Pi$ applied to a vector of set elements. Our goal will be to use a quantum lightning adversary relative to $(\star,\G,\Xs')$ --- in particular, a pair of identical banknotes with the same serial number --- to break some distinguishing problem relative to $(*,\G,\Xs)$. Concretely, our starting assumption gives the adversary $y=u*x$ for a random $u$, and then allows the adversary a single quantum query to $z\mapsto v*x$ for an unknown $v$, where either $v$ is random or $v=2u$. The adversary then has to tell whether $v=2u$ or not. It is straightforward to prove this assumption is true in the classical GGAM. In fact, it is a quantum analog of the classical group-based problem of distinguishing $g,g^a,g^b$ from $g,g^a,g^{a^2}$, a widely used Diffie-Hellman-like assumption.

Our idea is to have $\Xs'$ be elements of the form $\Pi(g*x,g*y)$ where $y=u*x$ is the challenge given by the assumption. Let $X=\Pi(x,y)\in\Xs'$. Now suppose we are given two copies of the banknote $|\G^h\star X\rangle$ relative to $(\star,\G,\Xs')$ for some serial number $h$. We then observe that, in the case where $v=2u$, the following process preserves the banknote (up to phase): map $\Pi(z_1,z_2)$ to $\Pi(z_2,v*z_1)$, where we compute $v*z_1$ from $z_1$ using the challenge oracle. Indeed, if $v=2u$, then
\begin{align*}
	|\G^h*X\rangle&=\frac{1}{\sqrt{|\G|}}\sum_{g\in\G}\chi(g,h)|g\star \Pi(x,y)\rangle=\frac{1}{\sqrt{|\G|}}\sum_{g\in\G}\chi(g,h)|\Pi(g*x,g*y)\rangle\\
	&\mapsto\frac{1}{\sqrt{|\G|}}\sum_{g\in\G}\chi(g,h)|\Pi(g*y,(g+2u)*x)\rangle=\frac{1}{\sqrt{|\G|}}\sum_{g\in\G}\chi(g,h)|\Pi((g+u)*x,(g+2u)*x)\rangle\\
	&=\chi(-u,h)\frac{1}{\sqrt{|\G|}}\sum_{g'\in\G}\chi(g',h)|\Pi(g'*x,g'*y)\rangle=\chi(-u,h)|\G^h*X\rangle
\end{align*}
Above, we used the substitution $g'=g+u$.

On the other hand, if $v\neq 2u$, then the transformation will produce a state whose support is not even on $\Xs'$. In particular, the transformed state would be orthogonal to the original state. So our reduction will apply the above transformation to one copy of $|\G^h\star X\rangle$, leaving the other as is. Then it will perform the SWAP test on the two states. If $v=2u$, the states will be identical and the SWAP test will accept. If $v\neq 2u$, the states will be orthogonal, and the swap test will accept only with probability $1/2$. Thus, we achieve a distinguishing advantage between the two cases, contradicting the assumption.

\medskip

We believe our proof gives convincing evidence that our scheme should be secure on a suitable group action, perhaps even those based on isogenies over elliptic curves. However, our underlying assumption is new, and needs further cryptoanalysis. One limitation of our assumption is that it is interactive, requiring a (quantum) oracle query to the challenger. One may hope instead to use a non-interactive assumption. We do not know how to make non-interactive assumptions work, in general. In particular, if we do not have an oracle that can transform the input for us, it seems like we are limited to strategies that only permute the inputs to $\Pi$. But since the scheme has to be efficient, the inputs to $\Pi$ can only consist of polynomial-length vectors of set elements. Any permutation on a polynomial-length set must have smooth order. On the other hand, the only permutations on $\Xs'$ which preserve $|\G^h\star X\rangle$ seem to have order that divides $|\G|$. Thus, if, say, the order of $\G$ were a large prime, it does not seem that permuting the inputs to $\Pi$ alone will be able to preserve $|\G^h\star X\rangle$.




\subsection{On Knowledge Assumptions and Algebraic Group Actions} 

In Section~\ref{sec:knowledge}, we show a different approach to justifying the security of our scheme, by adapting certain knowledge assumptions~\cite{EC:LiuMonZha23} to the setting of group actions. Despite some high-level similarities to~\cite{EC:LiuMonZha23}, the underlying details are somewhat different. The advantage of this route is that it gives a standard-model security definition (albeit, a non-standard knowledge definition) rather than a generic model proof.

However, we find significant issues with using knowledge assumptions quantumly, that appear not to have been observed before. In particular, the straightforward way to adapt the knowledge assumptions of~\cite{EC:LiuMonZha23} to group actions actually results in \emph{false} assumptions, as we demonstrate. Interestingly, our attack on the assumption is entirely generic. This is quite surprising, as in the classical setting, knowledge assumptions generally trivially hold against generic attacks.

Concretely, we show how to construct a superposition over $\Xs$ where the underlying discrete logarithms are hidden. To accomplish this, we observe that any set element $x$ can be seen as a superposition over all possible banknotes $|\G^h*x\rangle$; the superposition is uniform up to individual phases. Then we show a procedure to compute, given $|\G^h*x\rangle$, the serial number $h$. This allows us to apply individual phases to the various banknotes in the superposition. Certain phases will simply map $x$ to another set element $y$. But other phases will map $x$ to a uniform superposition (up to phases) over $\Xs$. Call this state $|\psi\rangle$.

Any meaningful knowledge assumption, and in particular the result of adapting~\cite{EC:LiuMonZha23} to group actions, would imply that if we were to measure $|\psi\rangle$ to get a set element $y$, then we must also ``know'' $g$ such that $y=g*x$. However, measuring $|\psi\rangle$ simply gives a uniform set element, importantly without any side information about $y$. As such, under the discrete log assumption, computing such a $g$ is hard.

We resolve this particular problem by re-framing knowledge assumptions as follows: instead of saying that any algorithm $A$ which produces a set element $y$ must know $g$ such that $y=g*x$, we say that for any such $A$ solving some task $T$, there is another algorithm $B$ that also solves $T$ such that $B$ knows $g$. Thus, even if the original $A$ is constructed in such a way that it does not know $g$, at least $B$ does, and we can apply any security arguments to $B$ instead of $A$. We demonstrate that this assumption, together with an appropriate generalization of the discrete log assumption, are enough to prove the security of our scheme. However, we are somewhat skeptical of our new knowledge assumption, and it certainly needs more cryptanalysis.

\paragraph{Algebraic Group Actions.} The Algebraic Group Model (AGM)~\cite{C:FucKilLos18} is an important model for studying group-based cryptosystems. It is considered a refinement of the generic group model, meaning that a proof in the model is ``at least as'' convincing as a proof in the generic group model, potentially more convincing. A couple of recent works~\cite{PKC:DHKKLR23,EPRINT:OrsZan23} have considered the group action analog, the Aglebraic Group Action Model (AGAM). Here, any time an adversary outputs a set element $y$, it must ``explain'' $y$ in terms of one of it's input set elements $x_1,\dots,x_n$ by providing a group element $g$ such that $y=g*x_i$.

The AGM can be seen as an idealized model version of the knowledge of exponent assumption, and likewise the AGAM can be seen as an idealized model version of an appropriate knowledge assumption on group actions. After all, a knowledge assumption would say that any time the adversary outputs a $y$, it must ``know'' how it derived $y$ from its inputs. The AGM/AGAM simply require the adversary to output this knowledge.

In Section~\ref{sec:knowledge}, we explore the AGAM in the presence of quantum attackers. We do not prove any formal results, but discuss why, unfortunately, the quantum AGAM appears problematic. For starters, given our attack on quantum knowledge assumptions, we are skeptical about the soundness of the quantum AGAM. In particular, our attack indicates that it is unlikely that the AGAM is a refinement of the generic group action model; rather they are likely incomparable.

Digging a little deeper, the problem with the AGAM is that it requires the adversary to produce extra information, namely the explanation $g$ of any output element $y$. The issue is that if the output is actually a superposition, this $g$ will be entangled with the superposition, meaning the AGAM adversary's output will actually be a different state than if it did not output $g$. For example, if an AGAM adversary had to output a banknote $|\G^h*x\rangle$ (say, as part of the quantum lightning experiment), then if it also ``explained'' the banknote, the entanglement with $g$ would actually cause the banknote state to fail verification. It therefore unclear how to interpret such an adversary. Does it actually break the scheme, even if it does not pass verification? In Section~\ref{sec:knowledge}, we go into more details about this issue as well as pointing out several other issues with the AGAM.

We note that these issues are not present in the generic group action model. Thus, despite classically being a ``worse'' model than the algebraic model, we propose for the quantum setting that the generic group action model is actually \emph{preferred} to the AGAM.





\iffalse

We do not know how to base the security of our schemes on any standard assumptions on isogenies. However, we prove security under non-standard but plausible assumptions, including a knowledge assumption. Specifically, we show how to adapt the security proof of quantum money over walkable invariants~\cite{EC:LiuMonZha23} to our setting, though several important things need to change. Importantly, we note that our scheme \emph{cannot} be seen as an instance of walkable invariants. Here we sketch the proof for our main construction, which is given in full in Section~\ref{sec:security}. The security of our modified construction over REGAs is essentially equivalent, as discussed in Section~\ref{sec:alternate}.

We first define a ``knowledge of group element'' assumption (KGEA), which roughly states that any adversary which can output a set element $y\in\Xs$ must ``know'' a group element $g$ such that $y=g*x$. Here, $x$ is some fixed set element that is provided to all parties. The intuition is that, if $\Xs$ is a sparse set, then perhaps the only way to generate new set elements is to actually use the group action operation. Slightly more formally, we consider a quantum adversary that outputs a set element $y$. We assume that the adversary performed no measurements besides measuring $y$ to get the final output, meaning that the adversary might have some remaining quantum state $|\psi_y\rangle$. Then the knowledge of group element assumption states that given $|\psi_y\rangle$ and $y$, it is possible to efficiently compute $g$ such that $y=g*x$. This assumption is analogous to the ``knowledge of path'' assumption defined by~\cite{EC:LiuMonZha23} for walkable invariants.

We now consider a quantum lightning adversary, which produces two banknotes with the same serial number. We will first purify the adversary, so that it makes no measurements. The state of the adversary can then be written as 
\[\sum_h \alpha_h |\phi_h\rangle |\G^h*x\rangle|\G^h*x\rangle=\frac{1}{|\G|}\sum_{h,g_1,g_2}\alpha_h \chi(h,g_1+g_2) |\phi_h\rangle |g_1*x\rangle|g_2*x\rangle\enspace ,\]
where $\sum_h |\alpha_h|^2=1$, and $|\phi_h\rangle$ is any state of the adversary left over after outputting the two banknotes. Now consider the algorithm which produces this state and then measures the final register, to obtain $g_2*x$, with the remaining state collapsing to
\[|\psi_{g_2*x}\rangle:=\frac{1}{\sqrt{|\G|}}\sum_{g_1,h}\alpha_h \chi(h,g_1+g_2) |\phi_h\rangle |g_1*x\rangle \]

Applying KGEA to this adversary, there is an algorithm $E$ that can compute $g_2$ from $g_2*x$ and $|\psi_{g_2*x}\rangle$. We would like to use $E$ and $|\psi_{g_2*x}\rangle$ to solve the discrete log assumption, reaching a contradiction. While $E$ can solve the discrete log of $g_2*x$, $E$ additionally needs $|\psi_{g_2}*x_\lambda\rangle$, which is correlated with $g_2*x$. Therefore, $E$ does not immediately represent a discrete log adversary. Inspired by~\cite{EC:LiuMonZha23}, we want to apply $E$ to a fresh discrete log challenge $g*x$ to recover $g$. In~\cite{EC:LiuMonZha23}, simply swapping out the measured value for a fresh challenge works, as in their setting the measured output is independent of the remaining state of the adversary; see Section~\ref{sec:related} for a brief explanation of their proof. The problem for us is that $g_2*x$ is correlated with $|\psi_{g_2*x}\rangle$, and $E$ may not work when given $g*x$ and $|\psi_{g_2*x}\rangle$ for $g\neq g_2$; perhaps $E$ only works if given the correct $g_2*x$ for $|\psi_{g_2*x}\rangle$. 

Our solution is to, given $g*x$, transform $|\psi_{g_2*x}\rangle$ into $|\psi_{g*x}\rangle$; then we can apply $E$ to $g*x$ and $|\psi_{g*x}\rangle$ to recover $g$. We first note that we can, using our knowledge of $g_2$ derived from $E$, apply the map $y\mapsto g_2*y$ to the last register in $|\psi_{g_2*x}\rangle$ to obtain

\[\frac{1}{\sqrt{|\G|}}\sum_{g_1,h}\alpha_h \chi(h,g_1+g_2) |\phi_h\rangle |(g_1+g_2)*x\rangle = \frac{1}{\sqrt{|\G|}}\sum_{g_2',h}\alpha_h \chi(h,g_2') |\phi_h\rangle |g_2'*x\rangle = |\psi_{0*x}\rangle=|\psi_{x}\rangle \]

where we used the change of variables $g_1+g_2\mapsto g_2'$. Next, we want to move $|\psi_{x}\rangle$ to $|\psi_{g*x}\rangle$. We could do this if we knew $g$ as well, but this is what we are trying to compute!

Instead we rely on a strengthening of the discrete log assumption, where we give the adversary a call to a computational Diffie-Hellman oracle in order to help it solve the discrete log. That is, we allow the adversary to query on a set element $y$, and obtain $(-g)*y$. We moreover allow this call to happen in superposition. But we only allow a single such query, or possibly two queries, depending on exactly how we model the query\footnote{The number of queries depends on whether $(-g)*y$ replaces the value $y$ (the so-called ``minimal'' oracle~\cite{KKVB02}), or is XORed into a supplied response response register (the ``standard'' oracle). The former requires only a single query, while the latter requires two queries to simulate the former: the first to the function $y\mapsto (-g)*y$, and the second to the function $y\mapsto g*y$.}. Then the adversary is given $g*x$ and must compute $g$. If we apply this oracle to the last register in $|\psi_x\rangle$, we get the state 
\[\frac{1}{\sqrt{|\G|}}\sum_{g_2',h}\alpha_h \chi(h,g_2') |\phi_h\rangle |(g_2'-g)*x\rangle=\frac{1}{\sqrt{|\G|}}\sum_{g_2'',h}\alpha_h \chi(h,g+g_2'') |\phi_h\rangle |g_2''*x\rangle=|\psi_{g*x}\rangle\]
as desired, where we used the change of variables $g_2'-g\mapsto g_2''$. Now we can apply $E$ to $g*x$ and $|\psi_{g*x}\rangle$ to recover $g$, thus solving discrete log and reaching a contradiction.

\medskip

The lingering question is whether our new assumptions hold. For our strengthened discrete log assumption, it appears hard to use one or two queries to such an oracle to actually recover $g$. One possibility is to view the oracle as a CDH oracle, which computes $(r-s)*x$ from $r*x,s*x$\enspace \footnote{CDH would usually be defined as computing $(r+s)*x$, but it is basically equivalent to consider the case where it computes $(r-s)*x$. In the case where we allow $\Omega(|\G|)$ CDH queries, we can use addition queries to implement subtraction queries and vice versa.}, and then use the quantum equivalence of CDH and discrete log for group actions~\cite{GPSV21,AC:MonZha22}. In our case, we only give a very limited CDH oracle which only works if $s*x$ was set to $g*x$. But even if we allowed the full generality of a CDH oracle and tried to apply~\cite{GPSV21,AC:MonZha22}, we will fail to compute the discrete log, since~\cite{GPSV21,AC:MonZha22} require far more than two queries to the oracle. We therefore conjecture our strengthened discrete log problem is hard.

We note that isogenies over elliptic curves will typically allow for sampling certain elements in $\Xs$ without directly computing them via applying $*$ to other elements. Specifically, it is possible sample elements with small discriminant using the complex multiplication method. This means KGEA as described above is technically false on known isogenies. Nevertheless, such directly sampled elements bare no obvious relation to each other, so it is unclear how to use them to actually break the security of our scheme.  Toward rectifying this issue, we give more refined assumptions that avoid this issue while still allowing for proving security. See Section~\ref{sec:oblivious} for details.\fi








\subsection{Further Discussion}

In Section~\ref{sec:discuss}, we generalize group actions to \emph{quantum} group actions, which replace classical set elements with quantum states, but otherwise behave mostly the same as standard group actions. We give a simple quantum group action based on the Learning with Errors (LWE) problem~\cite{STOC:Regev05}, where we can actually prove that the discrete log problem is hard under LWE. Despite this promising result, we expect that the LWE-based quantum group action will be of limited use. In particular, if we instantiate our quantum money construction over this group, the construction is \emph{insecure}. The reason is that, in this group action, it is impossible to recognize the quantum states of the set. Our security proof crucially relies on such recognition, since it allows us to characterize states accepted by the verifier. Moreover, without recognition, there is an attack: it is possible to fool the verifier with dishonest banknotes that are different from the honest ones and moreover are clonable, thereby breaking security.

Interestingly, we explain that this failed instantiation is actually \emph{equivalent} to a folklore approach toward building quantum money from lattices, which has been more-or-less shown impossible to make secure~\cite{C:LiuZha19,EC:LiuMonZha23}. The key missing piece in getting the folklore approach to work has been how to efficiently verify honest banknotes --- if such verification were possible, the scheme could be readily proven secure. Under our equivalence, this missing piece exactly maps to the problem of recognizing set elements in our quantum group action. For details, see Section~\ref{sec:discuss}. We believe this adds to the confidence of our proposal, since in group actions based on isogenies it is possible to recognize set elements, presumably without otherwise compromising hardness.




\subsection{Related Work}\label{sec:related}

\paragraph{Public key quantum money.} In Wiesner's original scheme, the mint is required to verify banknotes, meaning the mint must be involved in any transaction. The involvement of the mint also leads to potential attacks~\cite{Lutomirski10}. Some partial solutions have been proposed, e.g.~\cite{EPRINT:BehSat20,AC:RobZha21}. The dream solution, however, is known as \emph{public key} quantum money~\cite{CCC:Aaronson09}. Here, anyone can verify the banknote, while only the mint can create them.

Unlike Wiesner's scheme which is well-understood, secure public key quantum money has remained elusive. While there have been many proposals for public key quantum money~\cite{CCC:Aaronson09,STOC:AarChr12,ITCS:FGHLS12,Kane18,EC:Zhandry19b,EPRINT:KanShaSil21,KLS22,EC:LiuMonZha23}, they mostly either (1) have been subsequently broken (e.g.~\cite{CCC:Aaronson09,STOC:AarChr12,EC:Zhandry19b,KLS22} which were broken by~\cite{ITCS:LAFGKH10,PDFHP19,EC:Roberts21,EC:LiuMonZha23}), or (2) rely on new cryptographic building blocks that have received little attention from the cryptographic community (e.g.~\cite{ITCS:FGHLS12,Kane18,EPRINT:KanShaSil21} from problems on knots or quaternion algebras). The two exceptions are:
\begin{itemize}
	\item Building on a suggestion of~\cite{BenSat16}, \cite{EC:Zhandry19b} proved that quantum money can be built from post-quantum indistinguishability obfuscation (iO). While iO has received considerable attention and even has a convincing \emph{pre-quantum} instaniation~\cite{STOC:JaiLinSah21}, the post-quantum study of iO has been much less thorough. While some post-quantum proposals have been made~\cite{TCC:GenGorHal15,TCC:BGMZ18,EPRINT:BDGM20b,EC:WeeWic21}, their post-quantum hardness is not well-understood.
	\item \cite{EC:LiuMonZha23} construct quantum money from isogenies over super-singular elliptic curves. However, there is a crucial missing piece to their proposal, namely generating uniform superpositions over super-singular curves, which is currently unknown how to do. This is closely related to the major open question of obliviously sampling super-singular elliptic curves.
\end{itemize}
In light of the above, the existence of public key quantum money is largely considered open.



\paragraph{Cryptography from group actions and isogenies.} Isogenies were first proposed for use in post-quantum cryptography by Couveignes~\cite{EPRINT:Couveignes06} and Rostovtsev and Stolbunov~\cite{EPRINT:RosSto06}. Isogenies give a Diffie-Hellman-like structure, but importantly are immune to Shor's algorithm for discrete logarithms~\cite{FOCS:Shor94} due to a more restricted structure. This restricted structure, while helping preserve security against quantum attacks, also makes the design of cryptosystems based on them more complex. Thus, significant effort has gone into building secure classical cryptosystems from isogenies and understanding their post-quantum security (e.g.~\cite{CJS14,DJP14,AC:CLMPR18,AC:BeuKleVer19,CD20,PKC:DeFMey20,EC:Peikert20,EC:BonSch20,AC:ADMP20,TCC:AlaMalRah22,AC:MonZha22,EPRINT:MaiMar22,EC:CasDec23,EC:BonGuaZha23,EC:Robert23}).

Certain isogenies such as the original proposals of~\cite{EPRINT:Couveignes06,EPRINT:RosSto06} as well as CSIDH and its variants~\cite{AC:CLMPR18,PKC:DFKLMPW23} can be abstracted as abelian group actions. However, many other isogenies (such as SIDH~\cite{DJP14}  and OSIDH~\cite{CD20}) cannot be abstracted as abelian group actions. Even among abelian group actions, we must distinguish between ``effective group actions'' (EGAs) and \emph{restricted} EGAs (REGAs). The former satisfies the notion of a clean group action, whereas in the latter, the group action can only be efficiently computed for a certain small set of group elements. CSIDH could plausibly be a EGA at certain concrete security parameters, though asymptotically it only achieves quasi-polynomial security\footnote{With the state-of-the-art, evaluating CSIDH as an EGA would require time approximately $2^{\sqrt[3]{n}}$ on a quantum computer, while the best quantum attack is time $2^{\sqrt{n}}$. For a thorough discussion, see~\cite{Panny23}. By setting $n=\log^3(\lambda)$, one gets polynomial-time evaluation and the best attack taking time $\lambda^{\sqrt{\log(\lambda)}}$.}. Our alternate construction also works on REGAs, which can plausibly be instantiated even asymptotically by CSIDH using a quantum computer\footnote{In order for CSIDH to be a REGA, one needs to compute the structure of the group. While this is hard classically, it is easy with a quantum computer using Shor's algorithm~\cite{FOCS:Shor94}. Since we always assume a quantum computer in this work, we can therefore treat CSIDH as a REGA.}.

While some non-isogeny abelian group actions have been proposed (e.g.~\cite{Stickel05}), currently all such examples have been broken (e.g.~\cite{Shpilrain08}). For this reason, group actions are largely considered synonymous with isogenies, though this may change if more secure group actions are found.

The vast majority of the isogeny and group action literature has focused on post-quantum cryptography --- classical protocols that are immune to quantum attacks. To the best of our knowledge, only two prior works have used isogenies/group actions to build quantum protocols for tasks that are \emph{impossible} classically. The first is~\cite{TCC:AlaMalRah22}, who build a proof of quantumness~\cite{FOCS:BCMVV18}. We note that proofs of quantumness can also be achieved under several ``standard'' cryptographic tools, such as LWE~\cite{FOCS:BCMVV18} or certain assumptions on hash functions~\cite{FOCS:YamZha22}. In contrast, no prior quantum money protocol could be based on similar standard building blocks. We also note that ~\cite{TCC:AlaMalRah22} currently has no known asymptotic instantiation with better-than-quasi-polynomial security, as it requires a clean group action (EGA). The second quantum protocol based on isogenies is that of~\cite{EC:LiuMonZha23}, who build quantum money from walkable invariants, and propose an instantiation using isogenies over super-singular elliptic curves. However, such isogenies cannot be described as abelian group actions, and even more importantly their proposal is incomplete, as discussed above. Thus, ours is arguably the first application of group actions or isogenies to obtain classically impossible tasks that could not already be achieved under standard tools.



\paragraph{Relation to~\cite{EC:LiuMonZha23}.} Aside from using isogenies, our work has strong conceptual similarities to~\cite{EC:LiuMonZha23}, though also crucial differences that allow us to specify a complete protocol. Here, we give a brief overview of the similarities and differences.

The walkable invariant framework of~\cite{EC:LiuMonZha23} is very general, but here we describe a special case of it that would apply to certain group actions, in order to illustrate the differences with our scheme. Consider a group action that is \emph{not} regular, so that the set $\Xs$ is partitioned into many distinct orbits. For $x,y$ in the same orbit there will exist a unique $g$ such that $y=g*x$, but for $x,y$ in different orbits, there will not exist any group element mapping between them. We will also assume the ability to generate a uniform superposition over $\Xs$. We finally assume an ``invariant'', a unique label for each orbit which can be efficiently computed from any element in the orbit.

The minting process generates the uniform superposition over $\Xs$, and then measure the invariant, which becomes the serial number. The state then collapses to a uniform superposition over a single orbit, which becomes the banknote. This superposition can then be verified as follows. First check that the banknote has support on the right orbit by re-computing the invariant. Then check that the state is in uniform superposition by checking that the state is preserved under action by random group elements; this is accomplished using an analog of the swap test.~\cite{EC:LiuMonZha23} prove the security of their scheme under the certain assumptions which, when mapped to the group action setting above, correspond to the discrete log assumption and a knowledge assumption very similar to ours.

Unfortunately, there are no known instantiations of suitable group actions for their scheme. They propose using the set of ordinary elliptic curves as the set, the number of points on the curve as the invariant, and orbits being sets of curves with the same number of points. Isogenies between curves are then the action\footnote{It is not a proper group action since different orbits will be acted on by different groups.}, which do not change the number of points on the curve. The problem is that in general curves, it is not possible to efficiently compute the action, since the degree will be too high. The action \emph{can} be computed on smooth-order curves, but these are rare and there is no known way to compute a uniform superposition over such smooth-order curves. For reasons we will not get into here,~\cite{EC:LiuMonZha23} propose using instead supersingular curves with non-smooth order, but again these are rare and there is no known way to generate a uniform superposition over such curves.

We resolve the issues with instantiating~\cite{EC:LiuMonZha23}, without needing the ability to compute uniform superpositions over the set. Our key insight is that, if we can compute the group action efficiently (say because we are in an orbit of smooth-order elliptic curves), then this is enough to sample states that \emph{are} uniform over a given orbit, except for certain phase terms: namely the states $|\G^h*x\rangle$ for uniform $h$. Then, rather than the serial number indicating which orbit we are in (which is now useless since we are in a single orbit), the serial number is a description of the phase terms, namely $h$. %Despite these changes we are able to nevertheless prove security under similar assumptions as in~\cite{EC:LiuMonZha23} when specialized to group actions. 

%These changes, however, make adapting the security proof of~\cite{EC:LiuMonZha23} to our setting somewhat non-trivial. When specializing walkable invariants to suitable group actions,~\cite{EC:LiuMonZha23} make two assumptions that essentially correspond to the discrete log and knowledge of group element assumptions. The reason~\cite{EC:LiuMonZha23} can reduce to the plain discrete log assumption is that, when they measure a valid banknote, they get a uniform element in the corresponding orbit, independent of any side information the adversary has. In slightly more detail, their banknotes are uniform superpositions, so when they measure both banknotes from a quantum lightning adversary, they obtain two independent elements in the same orbit. Since the points are uniform and independent of the adversary's state, the discrete log assumption applies. But this directly contradicts the knowledge assumption, which states that any adversary that outputs two points in the same orbit must know the discrete log between them. In our setting, because we have the serial number be the phase instead of the orbit, the banknotes can be entangled with each other through the phase term, and this breaks the straightforward adaptation of the proof from~\cite{EC:LiuMonZha23}. Fortunately, we are able to give a proof under our strengthened discrete log assumption.



\subsection*{Acknowledgments} We thank Hart Montgomery for many helpful discussions about isogenies.

















%Along the way, we investigate how to define knowledge assumptions in the quantum setting. We find, for example, that the ``knowledge of path'' assumption recently proposed by~\cite{EC:LiuMonZha23} for constructing quantum money is likely false, due to the particular way the assumption is formalized. We give an improved formalization which avoids the issues of the prior definition, but appears unsuitable for plugging into the security arguments of~\cite{EC:LiuMonZha23}. We also explore the recent Algebraic Group Action Model (AGAM)~\cite{PKC:DHKKLR23}, an adaptation of the Algebraic Group Model~\cite{C:FucKilLos18} to the setting of group actions and quantum attacks whose validity is closely related to knowledge assumptions. However, we observe that unlike the classical case where the AGM is meaningfully ``between''  groups and standard-model groups\footnote{with some caveats; see Section~\ref{sec:related}.}, the quantum AGAM is incomparable to generic and even standard-model group actions. Specifically, we demonstrate an interactive game that is proved hard in the AGAM, despite being easy in generic/standard model group actions. This casts significant doubt on the utility of the model.







%We now briefly illustrate the key issue. The arguments of~\cite{EC:LiuMonZha23} show that their scheme satisfies the stronger quantum lightning security guarantee. They consider an adversary that outputs two notes with the same serial number, i.e. two uniform superpositions over a common orbit. They then consider measuring both superpositions. Because both superpositions were uniform, the result is two independent uniform set elements in the same orbit. Importantly, these set elements (once conditioned on the orbit) are independent of the rest of the adversary's state. As such, the discrete log assumption implies that it is infeasible to compute the group element mapping between these two set elements. But this then directly contradicts the knowledge of group element assumption.

%In our proof of security from Section~\ref{sec:security}, it turned out that we do not get two uncorrelated elements. This prevented us from reducing to the discrete log problem. However, the correlations have a particular form, which allows us to reduce to the strengthened assumption.

%Now we consider adapting their proof to show that our scheme is quantum lightning. Consider an adversary that can construct the state $\frac{1}{\sqrt{|\G|}}\sum_h |\G^h*x\rangle|\G^h*x\rangle$. This adversary would be able to win the quantum lightning experiment, since both states register have the same serial number\footnote{In the quantum lightning experiment, the serial number is classical, whereas with the state above the serial number is in superposition. However, it is possible to compute $h$ from $|\G^h*x\rangle$. One can then get a classical serial number by measuring $h$ from one of the copies of $|\G^h*x\rangle$. The other will automatically then collapse to another copy of $|\G^h*x\rangle$, thereby breaking the quantum lightning experiment.}. However, this state is also equal to $\frac{1}{\sqrt{|\G|}}\sum_g |g*x,(-g)*x\rangle$. This means measuring the two states gives perfectly correlated set elements in the orbit, quite unlike~\cite{EC:LiuMonZha23} which gave independent set elements. The problem is that by having the serial number encoded in the phase, we allow for much stronger correlations between the two states than were possible before. This breaks the reduction.





\iffalse



\paragraph{An attempted proof.} One possibility is to try to adapt the security proof of quantum money over walkable invariants~\cite{EC:LiuMonZha23} to our setting, though several important things need to change. We here sketch the proof, but note that we will below show that the assumptions needed are most likely false.

We first define a ``knowledge of group element'' assumption (KGEA), which roughly states that any adversary which can output a set element $y\in\Xs$ must ``know'' a group element $g$ such that $y=g*x$. Here, $x$ is some fixed set element that is provided to all parties. The intuition is that, if $\Xs$ is a sparse set, then perhaps the only way to generate new set elements is to actually use the group action operation\footnote{This is technically not true for isogenies over elliptic curves, since there are methods to ``blindly'' generate curves with unknown isogenies with any existing curves. However, we can model these methods as producing random set elements, and then say that any element produced by the adversary must have been derived by applying the group action to either $x$ or one of the random set elements produced by such a blind generation process. For simplicity in this overview, however, we will restrict to the case where such blind sampling is impossible.}. Slightly more formally, we consider a quantum adversary that outputs a vector $\yv\subseteq\Xs$ of set elements. We assume that the adversary performed no measurements besides measuring the final output $\yv$, meaning that the adversary might have some remaining quantum state $|\psi\rangle$. Then the knowledge of group element assumption states that for each $y\in\yv$, we can use $|\psi\rangle$ and $\yv$ to efficiently compute $g$ such that $y=g*x$.

We now consider a quantum lightning adversary, which produces two banknotes with the same serial number. We will first purify the adversary, so that it makes no measurements. The state of the adversary can then be written as $\sum_h \alpha_h |\phi_h\rangle |\G^h*x\rangle|\G^h*x\rangle$, where $\sum_h |\alpha_h|^2=1$. Now we imagine measuring both copies of $|\G^h*x\rangle$, which yields $y_1,y_2$, which are equal to $y_1=g_1*x,y_2=g_2*x$ for some $g_1,g_2$. The remaining state of the adversary collapses to $|\psi\rangle=\sum_h \alpha_h |\phi_h\rangle \chi(h,g_1)\chi(h,g_2)=\sum_h \alpha_h |\phi_h\rangle \chi(h,g_1+g_2)$. The knowledge of group element assumption then says that there is an algorithm $E$ that, given $|\psi\rangle$ and $y_1,y_2$, efficiently computes $g_1,g_2$. The crucial observation is that $|\psi\rangle$ only depends on $y_1,y_2$ through $g_1+g_2$. In particular, if we swap out $y_1,y_2$ with $y_1'=g_1'*x,y_2'=g_2'*x$ such that $g_1'+g_2'=g_1+g_2$, then this will look like a valid input to $E$ and therefore $E(|\psi\rangle,y_1',y_2')=(g_1',g_2')$. 

Such an $E$ then allows for breaking a slight strengthening of the discrete log problem, where the goal is to compute $g$ given $z_1=g*x,z_2=(-g)*x$, as follows. Namely, first run the above to get $|\psi\rangle,y_1,y_2\rangle$. Run $E$ once on $y_1,y_2$ to get $g_1,g_2$. Now compute $y_1'=z_1=g*x,y_2'=(g_1+g_2)*z_2=(g_1+g_2-g)*x$. Then run $E$ once more on $y_1',y_2'$ to get $g_1'=g,g_2'=g_1+g_2-g$, which succeeds since $g_1'+g_2'=g_1+g_2$. Then output $g=g_1'$. Thus, under KGEA and a strengthened discrete log assumption, we can prove security.

\paragraph{The problem with knowledge assumptions.} Unfortunately, we show that KGEA is probably false. Namely, consider the following adversary:
\begin{itemize}
	\item Compute $\frac{1}{|\G|}\sum_{g,h\in\G} \chi(g,h)|h,g*x\rangle$ as in banknote generation.
	\item Apply the group action in superpostion to compute $\frac{1}{|\G|}\sum_{g,h\in\G} \chi(g,h)|h,g*x,h*x\rangle$
	\item Measure the second and third registers, which gives $y_1=g*x,y_2=h*x$. The remaining register collapses to $\chi(g,h)|h\rangle$, which is equivalent to $|h\rangle$. This is $|\psi\rangle$ in the definition of knowledge of group element.
\end{itemize}
Importantly, after measurement, all information about $g$ is lost. Thus, if it were possible to compute $g$ from $|\psi\rangle,y_1,y_2$, then one can solve discrete logs. In other words, the knowledge of group element assumption is incompatible with the hardness of discrete logs, while our justification above required both assumptions to hold simulteneously.

A similar trick can be applied to the analogous knowledge of path and pathfinding assumptions in the walkable invariants of~\cite{EC:LiuMonZha23}. Because of the potentially more relaxed structure, we do not necessarily completely eliminate all information about $g$ in the walkable invariant setting. Thus we cannot conclusively prove that their knowledge assumption is false, though it seems likely that such an $E$ does not exist.


\paragraph{Potential solution: knowledge of only one group element.} The attack above on KGEA seems to inherently require considering adversaries that output two elements. In the attack above, the adversary \emph{can} explain $|h*x\rangle$ since it knows $h$. But it uses the fact that $|h*x\rangle$ gets measured to ``forget'' all knowledge of $g$, which then makes it hard to explain $|g*x\rangle$. More abstractly, we could imagine that it is possible to compute both $g$ and $h$ individually, even if it is not possible to compute them both simultaneously. For example, this would be the case if the computations of $g$ and $h$ were incompatible measurements.

This is certainly the case in the example above: if only $|g*x\rangle$ is measured but \emph{not} $|h*x\rangle$, the it \emph{is} possible to explain $|g*x\rangle$. Indeed, in this case, upon measuring and receiving measurement outcome $g*x$, the final state of the adversary is $\frac{1}{\sqrt{|\G|}}\sum_{h\in\G} \chi(g,h)|h,h*x\rangle$. $E$ can take this state, uncompute $|h*x\rangle$, and then perform the inverse QFT to recover $g$.

This gives rise to a more restricted KGEA, which we call 1-KGEA, where the adversary only outputs a single set element, and $E$ must only explain that one element. This assumptions seems resilient to the above attack. We therefore posit such an assumption.

\medskip

The question now becomes: how do we prove the security of our scheme under 1-KGEA? After all, an adversary for a quantum lightning/quantum money scheme produces two superposiions over set elements. How do we use the ability to explain just one of them to break a discrete log-like assumption? We now give our solution.

Recall that the state outputted by a purified quantum lightning adversary has the form $\sum_h \alpha_h |\phi_h\rangle |\G^h*x\rangle|\G^h*x\rangle$. Consider the algorithm which produces this state and then measures the final register, to obtain $|g_2*x\rangle$, with the remaining state collapsing to
\[|\psi_{g_2}\rangle:=\frac{1}{\sqrt{|\G|}}\sum_{g_1,h}\alpha_h \chi(h,g_1+g_2) |\phi_h\rangle |g_1*x\rangle \]

Applying 1-KGEA to this adversary, there is an algorithm $E$ that can compute $g_2$ from $g_2*x$ and $|\psi_{g_2}\rangle$. We would like to use $|\psi_{g_2}\rangle$ to solve the discrete log assumption, reaching a contradiction. While $E$ can solve the discrete log of $g_2*x$, $g_2*x$ was generated together with $|\psi_{g_2}\rangle$, so it does not immediately represent a discrete log adversary. Inspired by~\cite{EC:LiuMonZha23} and our previous attempt, we want to apply $E$ to a fresh discrete log challenge $g*x$ to recover $g$. The problem is that $E$ may not work on $g*x$ and $|\psi_{g_2}\rangle$; perhaps $E$ only works if given the correct $g_2*x$ for $|\psi_{g_2}\rangle$. 

Our solution is to, given $g*x$, transform $|\psi_{g_2}\rangle$ into $|\psi_g\rangle$. We first note that we can, using our knowledge of $g_2$ derived from $E$, apply the map $y\mapsto g_2*y$ to the last register in $|\psi_{g_2}\rangle$ to obtain

\[\frac{1}{\sqrt{|\G|}}\sum_{g_1,h}\alpha_h \chi(h,g_1+g_2) |\phi_h\rangle |(g_1+g_2)*x\rangle = \frac{1}{\sqrt{|\G|}}\sum_{g_2',h}\alpha_h \chi(h,g_2') |\phi_h\rangle |g_2'*x\rangle = |\psi_0\rangle \]

where we used the change of variables $g_1+g_2\mapsto g_2'$. Here, we used that $E$ can compute $g_2$ without destroying the input state, which is a consequence of the Gentle Measurement Lemma\cite{Winter99}. 

Now, we could similarly move $|\psi_0\rangle$ to $|\psi_g\rangle$ if we new $g$ as well, but this is what we are trying to compute! Instead we rely on a strengthening of the discrete log assumption, where we give the adversary a call to a computational Diffie-Hellman oracle in order to help it solve the discrete log. That is, we allow the adversary to query on a set element $y$, and obtain $(-g)*y$. We moreover allow this call to happen in superposition. But we only allow a single such query, or possibly two queries, depending on exactly how we model the query\footnote{The number of queries depends on whether $(-g)*y$ replaces the value $y$ (the so-called ``minimal'' oracle~\cite{KKVB02}), or is XORed into a supplied response response register (the ``standard'' oracle). The former requires only a single query, while the latter requires two queries to simulate the former: the first to the function $y\mapsto (-g)*y$, and the second to the function $y\mapsto g*y$.}. Then the adversary is given $g*x$ and must compute $g$. If we apply this oracle to the last register in $|\psi_0\rangle$, we get the state 
\[\frac{1}{\sqrt{|\G|}}\sum_{g_2',h}\alpha_h \chi(h,g_2') |\phi_h\rangle |(g_2'-g)*x\rangle=\frac{1}{\sqrt{|\G|}}\sum_{g_2'',h}\alpha_h \chi(h,g+g_2'') |\phi_h\rangle |g_2''*x\rangle=|\psi_g\rangle\]
as desired, where we used the change of variables $g_2'-g\mapsto g_2''$. Now we can apply $E$ to $g*x$ and $|\psi_g\rangle$ to recover $g$, thus solving discrete log. Thus under the assumed KGEA and strengthened discrete log assumption, our scheme is secure.

The lingering question is whether our strengthened discrete log assumption holds. We do not know how to justify the security of our new assumption. Nevertheless, it appears hard to use one or two queries to such an oracle to actually recover $g$. One possibility is to view the oracle as a CDH oracle, which computes $(r-s)*x$ from $r*x,s*x$\footnote{CDH would usually be defined as computing $(r+s)*x$, but it is basically equivalent to consider the case where it computes $(r+s)*x$.}, and then use the quantum equivalence of CDH and discrete log for group actions~\cite{GPSV21,AC:MonZha22}. In our case, the oracle only works if $s*x$ was set to $g*x$. But even if we allowed the full generality of a CDH oracle and try to apply~\cite{GPSV21,AC:MonZha22}, we will fail to compute the discrete log, since~\cite{GPSV21,AC:MonZha22} require far more than two queries to the oracle. We therefore conjecture our strengthened discrete log problem is hard.


















As discussed in Section~\ref{sec:introknowledge}, one approach is to adapt the proof technique of ~\cite{EC:LiuMonZha23} to our setting, but we explain there that the assumptions they use are likely false. However, we here give an informal intuition for why our scheme may be secure. We focus on the isogeny setting, where it appears hard to generate the uniform superposition over $\Xs$. 

For all isogenie-based group actions with efficiently computable actions, the group $\G$ has smooth order. Note that, unlike standard groups where a smooth order means discrete log is easy, in group actions this is not necessarily the case. However, in the quantum setting, smooth-order group actions yield the following potential attack strategy. For a subgroup $\Hg$ of $\G$ and an element $x\in\Xs$, define the state \[|\Hg*x\rangle:=\frac{1}{\sqrt{|\Hg|}}\sum_{g\in\Hg}|g*x\rangle\]
Consider the task of computing $|\Hg*x\rangle$, given (a description of) $\Hg$ and $x$, which we call the \emph{orbit superposition problem}. If this problem were computationally easy, then in smooth-order group actions discrete logarithms are quantumly easy. To see this, let $\G_0=\{1\}\subseteq \G_1\subseteq\cdots\subseteq\G_n=\G$ be a polynomial-length series of subgroups of $\G$ such that $\G_{k}/\G_{k-1}$ is cyclic and polynomial sized. Such a series is guaranteed to exist by smoothness. Let $g_k$ be a generator of $\G_{k}/\G_{k-1}$, and let its order be $p_k$. 

Given $x,y\in\Xs$, then $y=h*x$ where $h$ can be uniquely written as $j_1\cdot g_1+j_2\cdot g_2+\cdots+j_n\cdot g_n$ for some $j_k\in [0,p_k-1]$. We can compute the $j_k$, and hence the discrete log $h$, as follows. First compute $|\G_{n-1}*x\rangle$ as well as $|\G_{n-1}*[(-j\cdot g_n)*y])\rangle$ for each $j\in [p_n]$. For exactly $j=j_n$, $|\G_{n-1}*[(-j\cdot g_n)*y])\rangle=|\G_{n-1}*y\rangle$, and for all other $j$, $|\G_{n-1}*[(-j\cdot g_n)*y]\rangle$ is orthogonal to $|\G_{n-1}*y\rangle$. The correct $j_n$ can be detected by computing several copies of each state and using the swap test. Now consider $y'=((-j_n)\cdot g_{n-1})*y=h'*x$ for $h'=j_1\cdot g_1+j_2\cdot g_2+\cdots+j_{n-1}\cdot g_{n-1}$. If we run the same procedure again except replacing $y$ with $y'$ and $\G_{n-1}$ with $\G_{n-2}$, we can compute $j_{n-1}$. By proceeding in this way, one eventually recovers all the $j_i$.

While this might indicate a weakness in the discrete logarithm problem in smooth-order group actions, another takeaway is that the orbit superposition problem must be hard in such group actions.

We can slightly generalize the above, which we call the \emph{generalized orbit superposition problem} (GOSP). Here, we define states $|\Hg^h*x\rangle$ for $h\in\Gs$ as
\[|\Hg^h*x\rangle:=\frac{1}{\sqrt{|\Hg|}}\sum_{g\in\Hg}\chi(g,h)|g*x\rangle\]
The GOSP problem is to, given $\Hg,x,y$, compute simulteneously $|\Hg^h*x\rangle,|\Hg^h*y\rangle$ for \emph{some} $h$ (which can be chosen by the adversary). The standard OSP problem is equivalent to GOSP except that it requires $h=0$. It is a straightforward adaptation of the above argument that the hardness of discrete logs implies the hardness of GOSP. 

Verification in our scheme is statistically close to accepting exactly the honest banknote state, so we might as well treat it as such. Thus, the security of the quantum money scheme relies on the difficulty of computing two copies of $|\G^h*x\rangle$ for a common $h$. In fact, under this assumption, the scheme obtains the stronger notion of quantum lightning. The intuition for this problem comes from the hardness of GOSP. Unfortunately, GOSP is an \emph{harder} problem (and hence assuming hardness is a milder assumption): our scheme's security is basically the special case where $\Hg=\G$ and $x=y$. Therefore, we cannot directly base quantum lightning hardness on the discrete logarithm problem. But this at least shows the problems are similar, and we therefore conjecture that they have similar hardness on group actions currently being studied.


\subsection{On Quantum Knowledge Assumptions and the AGAM}\label{sec:introknowledge}

\paragraph{A security ``proof.''} We would ideally prove the security of our scheme under concrete hardness assumptions. One possibility is to try to adapt the security proof of quantum money over walkable invariants~\cite{EC:LiuMonZha23} to our setting. Several important things need to change, but the high-level intuition will remain the same. We here sketch the proof, but note that we will below show that the assumptions needed are most likely false.

We first define a ``knowledge of group element'' assumption, which roughly states that any adversary which can output a set element $y\in\Xs$ must ``know'' a group element $g$ such that $y=g*x$. Here, $x$ is some fixed set element that is provided to all parties. The intuition is that, if $\Xs$ is a sparse set, then perhaps the only way to generate new set elements is to actually use the group action operation\footnote{This is technically not true for isogenies over elliptic curves, since there are methods to ``blindly'' generate curves with unknown isogenies with any existing curves. However, we can model these methods as producing random set elements, and then say that any element produced by the adversary must have been derived by applying the group action to either $x$ or one of the random set elements produced by such a blind generation process. For simplicity, however, we will restrict to the case where such blind sampling is impossible, since we anyway show issues with applying such an assumption to quantum money.}. Slightly more formally, we consider a quantum adversary that outputs a vector $\yv\subseteq\Xs$ of set elements. We assume that the adversary performed no measurements besides measuring the final output $\yv$, meaning that the adversary might have some remaining quantum state $|\psi\rangle$. Then the knowledge of group element assumption states that for each $y\in\yv$, we can use $|\psi\rangle$ and $\yv$ to efficiently compute $g$ such that $y=g*x$.

We now consider a quantum lightning adversary, which produces two banknotes with the same serial number. We will first purify the adversary, so that it makes no measurements. The state of the adversary can then be written as $\sum_h |\phi_h\rangle |\G^h*x\rangle|\G^h*x\rangle$. Now we imagine measuring both copies of $|\G^h*x\rangle$, which yields $y_1,y_2$, which are equal to $y_1=g_1*x,y_2=g_2*x$ for some $g_1,g_2$. The remaining state of the adversary collapses to $|\psi\rangle=\sum_h |\phi_h\rangle \chi(h,g_1)\chi(h,g_2)=\sum_h |\phi_h\rangle \chi(h,g_1+g_2)$. The knowledge of group element assumption then says that there is an algorithm $E$ that, given $|\psi\rangle$ and $y_1,y_2$, efficiently computes $g_1,g_2$. The crucial observation is that $|\psi\rangle$ only depends on $y_1,y_2$ through $g_1+g_2$. In particular, if we swap out $y_1,y_2$ with $y_1'=g_1'*x,y_2'=g_2'*x$ such that $g_1'+g_2'=g_1+g_2$, then this will look like a valid input to $E$ and therefore $E(|\psi\rangle,y_1',y_2')=(g_1',g_2')$. 

Such an $E$ then allows for breaking a slight strengthening of the discrete log problem, where the goal is to compute $g$ given $z_1=g*x,z_2=(-g)*x$, as follows. Namely, first run the above to get $|\psi\rangle,y_1,y_2\rangle$. Run $E$ once on $y_1,y_2$ to get $g_1,g_2$. Now compute $y_1'=z_1=g*x,y_2'=(g_1+g_2)*z_2=(g_1+g_2-g)*x$. Then run $E$ once more on $y_1',y_2'$ to get $g_1'=g,g_2'=g_1+g_2-g$, which succeeds since $g_1'+g_2'=g_1+g_2$. Then output $g=g_1'$. Thus, under the knowledge of group element assumption and a strengthened discrete log assumption, we can prove security.

\paragraph{The problem with knowledge assumptions.} Unfortunately, we show that the knowledge of group element assumption is probably false. Namely, consider the following adversary:
\begin{itemize}
	\item Compute $\frac{1}{|\G|}\sum_{g,h\in\G} \chi(g,h)|h,g*x\rangle$ as in banknote generation.
	\item Apply the group action in superpostion to compute $\frac{1}{|\G|}\sum_{g,h\in\G} \chi(g,h)|h,g*x,h*x\rangle$
	\item Measure the second and third registers, which gives $y_1=g*x,y_2=h*x$. The remaining register collapses to $\chi(g,h)|h\rangle$, which is equivalent to $|h\rangle$. This is $|\psi\rangle$ in the definition of knowledge of group element.
\end{itemize}
Importantly, after measurement, all information about $g$ is lost. Thus, if it were possible to compute $g$ from $|\psi\rangle,y_1,y_2$, then one can solve discrete logs. In other words, the knowledge of group element assumption is incompatible with the hardness of discrete logs, while our justification above required both assumptions to hold simulteneously.

A similar trick can be applied to the analogous knowledge of path and pathfinding assumptions in the walkable invariants of~\cite{EC:LiuMonZha23}. Because of the potentially more relaxed structure, we do not necessarily completely eliminate all information about $g$ in the walkable invariant setting. Thus we cannot conclusively prove that their knowledge assumption is false, though it seems likely that such an $E$ does not exist.

\begin{remark}The above does not show any actual vulnerability in our construction or that of~\cite{EC:LiuMonZha23}, just that the security proof is vacuous since it relies on false assumptions. In particular, our knowledge adversary does not correspond to any obvious attack strategy, as it is not even trying to create a pair of quantum lightning states.\end{remark}

\paragraph{A fix that does not work.} We can easily fix the formalization of knowledge assumptions to avoid the issue above. Specifically, rather than giving $E$ the output of the adversary and asking it to compute $g$, we actually have $E$ start ``from scratch'', and simulate the entire view of the adversary together with $g$. So in the attack above, rather than asking $E$ to output $g,h$ given $|h\rangle$, we would just ask that $E$ can create the state $\frac{1}{|\G|}\sum_{g,h\in\G} \chi(g,h)|h,g*x,h*x\rangle\otimes |g,h\rangle$, which looks exactly like the state produced by the original adversary except that it also produces the corresponding $g,h$. Such an $E$ is trivial to construct in this case: first compute $\frac{1}{|\G|}\sum_{g,h\in\G}|g,h\rangle$, perform a phase operation $|g,h\rangle\mapsto\chi(g,h)|g,h\rangle$ and finally the operation $|g,h\rangle\mapsto |h,g*x,h*x\rangle\otimes|g,h\rangle$ using the group action.

Unfortunately, this modification to the knowledge assumption precludes its use in the security proof above. After all, there is no longer any way to use $E$ to break discrete logs, as $E$ takes no input at all.


\paragraph{Generic versus Algebraic Adversaries.} The classical Generic Group Model~\cite{EC:Shoup97,IMA:Maurer05} (GGM) considers \emph{generic} adversaries that only make black-box use of a cryptographic group and do not utilize any specific implementation details\footnote{See Section~\ref{sec:related} for a more precise discussion.}. The Algebraic Group Model (AGM)~\cite{C:FucKilLos18} instead considers \emph{algebraic} adversaries, which may be non-black box but are required to ``explain'' any element they output. This means that the adversary can show how the input elements could be combined using the group operation to arrive at the outputted element. In the classical world, algebraic adversaries are a strictly larger class of adversaries: any generic adversary can be easily converted into an algebraic adversary, computing the required explanation by looking at the history of queries it made to the generic group. Thus, the AGM is ``no worse'' than the the GGM, and the hope is that by capturing more adversaries, the AGM better reflects the real world. We note that the AGM is closely related to knowledge assumptions: under an appropriate knowledge assumption on groups, one can convert any standard model algorithm into an algebraic adversary. Thus, under an appropriate knowledge assumption, the AGM reflects the ``real world.'' See Section~\ref{sec:related} for additional discussion.

Moving to the quantum setting with group actions, we can similarly consider generic and algebraic adversaries, leading respectively to the Generic Group Action Model (GGAM) and Algebraic Group Action Model (AGAM) considered in a couple recent works~\cite{AC:MonZha22,PKC:DHKKLR23,EC:BonGuaZha23}. Specifically, a generic adversary only makes black-box use of the group, and algebraic adversary can make non-black box use, but are required to ``explain'' any outputted element. This means that if the adversary outputs a set element $y$, it must additionally output $(x,g)$ where $x$ is a set element that was inputted to the adversary, and $g$ is a group element such that $y=g*x$.~\cite{PKC:DHKKLR23} consider the AGAM where adversaries are quantum, and moreover where adversaries may make superposition queries to the challenger, and are required to provide a superposition of corresponding explanations. They show, for example, that computational Diffie-Hellman (CDH) and discrete logarithms are equivalent in the AGAM, even if given superposition access to a decisional Diffie-Hellman oracle. 

One may hope to motivate the AGAM and algebraic adversaries over the GGAM and generic adversaries, by an analogous argument that any generic adversary can be compile into an algebraic one\footnote{\cite{PKC:DHKKLR23} who define the AGAM do not make this argument. Indeed, they take a somewhat different view, treating the GGAM as only being used to prove information-theoretic statements by counting query complexity instead of overall (query+time) complexity. In this sense, they argue that the GGAM is rather useless as the query complexity of discrete logarithms in group actions is only polynomial~\cite{EttHoy00}. However, we see no reason not to consider the generic adversaries for group actions, but taking into account query and time complexity. We cannot hope to prove information-theoretic lower bounds, but could still reduce to computational assumptions like the hardness of discrete logarithms, as \cite{PKC:DHKKLR23} argue for doing in the AGAM. Indeed, we will argue that this is actually a \emph{better} reflection of known algorithms for group actions than algebraic adversaries.}. However, we observe that this is false! In order to compile a generic adversary into an algebraic one, one must be able to look at the history of queries of queries to the group action. But for quantum algorithms --- which, importantly, can make quantum queries to the group action operation --- this is not possible in general. Indeed, any attempt to remember or look at quantum queries made to an oracle will perturb them, modifying the behavior of the algorithm. 

We formalize this by showing a game that can be won efficiently (in both time and query complexity) by generic algorithms (and also standard-model algorithms!), but it most likely hard for algebraic algorithms. Our game is simple. We give the adversary the ability to make a single discrete log query, mapping $g*x\mapsto g$ for a known set element $x$, in superposition. Then we ask that the adversary outputs the uniform superposition over $\Xs$.

A generic algorithm can generate the uniform superposition over $\G$, and then use the group action to compute the state \[\frac{1}{\sqrt{|\G|}}\sum_{g\in\G}|g*x,g\rangle\enspace .\]
Then it makes it's discrete log query, setting the left register as the query register, and the right register as the response register. Per the usual modeling of quantum queries, the query output is XORed into the response register, meaning the discrete log oracle maps $|g*x,z\rangle\mapsto |g*x,z\oplus g\rangle$. Applying to the state above gives 
\[\frac{1}{\sqrt{|\G|}}\sum_{g\in\G}|g*x,0\rangle=\left(\frac{1}{\sqrt{|\Xs|}}\sum_{y\in\Xs}|y\rangle\right)|0\rangle\enspace .\]
The first register thus contains the desired output.

Meanwhile, an algebraic adversary has no hope of completing the above task. If it tries to mimic the behavior of the generic algorithm, it will have to produce the state 
\[\frac{1}{\sqrt{|\G|}}\sum_{g\in\G}|g*x,g,(x,g)\rangle\enspace ,\]
where the first two registers are the query/response registers as before, and the last register is the explanation of the set element in the query register.

Now when the discrete log oracle is applied, the result is 
\[\frac{1}{\sqrt{|\G|}}\sum_{g\in\G}|g*x,0,(x,g)\rangle\]
Unlike the generic case, here the query register is entangled with the explanation register, and so the query register is not the required uniform superposition over $\Xs$. We moreover show that any algebraic algorithm for this game can be converted into an algorithm which outputs a uniform superposition over set elements \emph{without making a discrete log query.} As discussed above in the security justification of our quantum money scheme, such an algorithm is likely impossible. We moreover show a slightly more complicated game based on similar ideas where we can formally prove, under the hardness of discrete logs, that there is no algebraic adversary. Yet a simple generic adversary exists.

Thus, we see that tasks that are easy for generic algorithms (and also standard-model algorithms) can be hard for algebraic algorithms. This means that the AGAM is, in some cases, actually ``worse'' than the GGAM. We believe this makes it very unclear if the AGAM is a useful model for understanding the security of group actions.


\begin{remark}Other more subtle issues also arise in the AGAM. For example, suppose the adversary's interaction with the challenger allows the adversary to receive a superposition of set elements, say by making a query to a CDH oracle. Then it is not well-defined what it means to later ``explain'' some element in terms of this superposition. After all, since the superposition of set elements cannot be recorded without altering the adversary's execution, there is nothing that the explanation can be compared against to test its validity.\end{remark}

\fi




%\medskip

%However, we nevertheless provide two arguments as evidence for the security of our scheme.

%The first argument relies on two assumptions. The first assumption is a ``knowledge of group element'' assumption, which very roughly says that any adversary which outputs a set element $y$ must also ``know'' a group element $g$ such that $y=g*x$. This assumption can be seen as a group action version of the ``knowledge of path'' assumption recently proposed by~\cite{EC:LiuMonZha23}, which was applied to certain mathematical structures such as isogenies over \emph{supersingular} curves. The second assumption is a mild strengthening of the discrete log assumption, and stipulates that it is hard to compute $g$ given both $g*x,g^{-1}*x$. 

%The security proof roughly follows a similar flavor of the security proof of quantum money from~\cite{EC:LiuMonZha23}, though several important details are different. Consider an adversary which outputs two copies of $|\G^h*x\rangle$, for some $h$. $h$ itself may be in superposition, and may also be entangled with other registers produced by the adversary. So we can write the overall state of the output of the adversary's output as:
%\begin{align*}
%	\sum_h \alpha_h |\psi_h\rangle |\G^h*x\rangle|\G^h*x\rangle&=\frac{1}{|\G|}\sum_{g,g',h} \alpha_h |\psi_h\rangle \chi(g,h)\chi(g',h)|g*x,g'*x\rangle\\&=\frac{1}{|\G|}\sum_{g,g',h} \alpha_h |\psi_h\rangle \chi(g\times g',h)|g*x,g'*x\rangle
%\end{align*}
%Now we measure $g*x,g'*x$, which collapses the remaining state to $|\tau_{g,g'}\rangle:=\sum_h \alpha_h\chi(g\times g',h)|\psi_h\rangle$. We note two things:
%\begin{itemize}
%	\item Measuring $g*x,g'*x$ uniquely determines $g,g'$. It is possible to show that each of $g,g'$ are individually uniform, though they may be correlated.
%	\item $|\tau_{g,g'}\rangle$ only depends on $u:=g\times g'$
%\end{itemize}
%Now, by the knowledge of group element assumption, the adversary, upon outputting $|\tau_{g,g'}\rangle,g*x,g'*x$, must ``know'' $g,g'$, and hence $u$. In other words, the adversary, can compute 


\iffalse \paragraph{Generic and Algebraic Groups and Actions.} The Generic Group Model (GGM) was first formalized by Shoup~\cite{EC:Shoup97} and Maurer~\cite{IMA:Maurer05}. As shown by~\cite{C:Zhandry22b}, those two works actually defined two \emph{different} generic groups. The key difference between the two is that in Shoup's model, all parties have access to random bit representation of the group elements, whereas in Maurer's model, all parties only receive handles to group elements. Regardless, as shown in~\cite{AC:Dent02,C:Zhandry22b}, both models are uninstantiable, in that there are (contrived) cryptosystems that can be proven secure in the models, but are insecure in \emph{any} concrete group. %A key takeaway from~\cite{C:Zhandry22b}, however, is that if the game defining security satisfies certain properties, then we can treat both models as equivalent. 

The Algebraic Group Model (AGM)~\cite{C:FucKilLos18} attempts to improve on generic groups by giving the adversary non-black-box access to the group, while at the same time leveraging some of the benefits of idealized models by requiring the adversary to explain any element it produces. As hinted at in~\cite{C:FucKilLos18} and advocated for by~\cite{C:Zhandry22b}, in order to avoid certain trivial impossibilities, the AGM should only be applied to security games that are generic in the more restrictive sense of Maurer~\cite{IMA:Maurer05}. By restricting to Maurer-type games, it is indeed the case that the AGM is ``between'' the standard and Maurer-generic group models, though there are still contrived impossibilities~\cite{C:Zhandry22b} in the AGM. We note that a different view was taken by~\cite{AC:ZhaZhoKat22}. Nevertheless, because all known impossibilities are contrived, the AGM remains a plausible model for justifying the security of ``natural'' computational assumptions and protocols. 

The Generic Group Action Model (GGAM), the analog of the GGM for group actions, has been formalized by a few recent works~\cite{AC:MonZha22,PKC:DHKKLR23,EC:BonGuaZha23}. Like the GGM, there are multiple ways to formalize generic group actions. The situation is actually even more complicated for group actions: there are now both a group and a set whose elements can be provided as random bit-strings or handles, and each domain could in principle have a different choice. Moreover, the group could even be given ``in the clear'', with only access to the being black box. This gives rise to several different possible models. We highlight the model of~\cite{EC:BonGuaZha23}, which treats the group as non-black box, but treats the set with a Maurer-type restriction where algorithms are only given handles instead of bit-strings.

\cite{PKC:DHKKLR23} also formulates an Algebraic Group Action Model (AGAM). That work does not discuss which types of games the AGAM should be applied to, but following the case of the AGM, we might restrict to security games that are generic in the Maurer sense. In this case, the AGAM can be easily shown to be ``between'' the standard and the generic group action model of~\cite{EC:BonGuaZha23}.


\paragraph{Knowledge Assumptions.} The knowledge of exponent assumption (KEA)~\cite{C:Damgaard91} is a popular assumption on cryptographic groups which has been used to prove the security of zero knowledge protocols. The assumption is trivially true, but also uninteresting, in the case that discrete logs are easy, such as in the quantum setting. Formalizing knowledge assumptions quantumly is non-trivial, and to the best of our knowledge the first attempt was made only very recently by~\cite{EC:LiuMonZha23} with their knowledge of path assumption.~\cite{STOC:BCPR14} show that certain variants of the KEA assumption are false. These counterexamples do not seem to apply to the knowledge of path assumption as formalized in~\cite{EC:LiuMonZha23}.

Knowledge assumptions can be used to instantiate the AGM~\cite{EPRINT:KasPan19}, the intuition being that an appropriate knowledge assumption lets us map any adversary to an algebraic one. A restricted version of the AGM can even be instantiated under non-knowledge assumptions~\cite{EC:AgrHofKas20}.\fi