\section{Our Quantum Lightning Scheme}\label{sec:constr}

Here, we give our basic quantum lightning construction, which assumes a cryptographic group action.

\begin{construction}\label{constr:main} Let $\gen,\ver$ be the following QPT procedures:
	\begin{itemize}
		\item $\gen(1^\lambda)$: Initialize quantum registers $\Ss$ (for serial number) and $\Ms$ (for money) to states $|0\rangle_\Ss$ and $|0\rangle_\Ms$, respectively. Then do the following:
		\begin{itemize}
			\item Apply $\QFT_{\G_\lambda}$ to $\Ss$, yielding the joint state $\frac{1}{\sqrt{|\G_\lambda|}}\sum_{g\in\G_\lambda}|g\rangle_\Ss|0\rangle_\Ms$.
			\item Apply in superposition the map $|g\rangle_\Ss|y\rangle_\Ms\mapsto |g\rangle_\Ss|y\oplus (g*x_\lambda)\rangle_\Ms$. The joint state of the system $\Ss\otimes\Ms$ is then $\frac{1}{\sqrt{|\G_\lambda|}}\sum_{g\in\G_\lambda}|g\rangle_\Ss|g*x_\lambda\rangle_\Ms$.
			\item Apply $\QFT_{\G_\lambda}$ to $\Ss$ again, yielding $\frac{1}{|\G_\lambda|}\sum_{g,h\in\G_\lambda}\chi(g,h)|h\rangle_\Ss|g*x_\lambda\rangle_\Ms$
			\item Measure $\Ss$, giving the serial number $\sigma:=h$. The $\Ms$ register then collapses to the banknote $\$=|\G_\lambda^h*x_\lambda\rangle:=\frac{1}{\sqrt{|\Gs_\lambda|}}\sum_{g\in\G_\lambda}\chi(g,h)|g*x_\lambda\rangle_\Ms$. Output $(\sigma,\$)$. 
		\end{itemize}
		\item $\ver(\sigma,\$):$ First verify that the support of $\$$ is contained in $\Xs_\lambda$, by applying the assumed algorithm for recognizing $\Xs_\lambda$ in superposition. Then repeat the following $\lambda$ times:
		\begin{itemize}
			\item Initialize a new register $\Hs$ to $(|0\rangle_\Hs+|1\rangle_\Hs)/\sqrt{2}$. 
			\item Choose a random group element $u\in\G_\lambda$.
			\item Apply to $\Hs\otimes\Ms$ in superposition the map
			\[{\sf Apply}|b\rangle_\Hs|y\rangle_\Ms\mapsto \begin{cases}|0\rangle_\Hs|y\rangle_\Ms&\text{ if }b=0\\|1\rangle_\Hs|(-u)*y\rangle_\Ms\enspace\footnotemark&\text{ if }b=1\end{cases}\]
			\footnotetext{Note that we used the ``minimal'' oracle here for the group action computation, having $(-u)*y$ replace $y$, instead of being written to a response register as in the standard quantum oracle. However, since the computation $y\mapsto (-u)*y$ is efficiently reversible (by $y\mapsto u*y$), we can easily implement the minimal oracle efficiently by first computing $|(-u)*y\rangle_{\Ms'}$ in a new register $\Ms'$, then uncomputing $|y\rangle_\Ms$ using the efficient inverse (so it now contains $|0\rangle_\Ms$), and finally swapping $\Ms'$ with $\Ms$.}
			In the case that $\$$ is the correct banknote state $|\G_\lambda^h*x_\lambda\rangle$, the result of applying ${\sf Apply}$ is:
			\begin{align*}&\frac{1}{\sqrt{2|\G_\lambda|}}\left(|0\rangle_\Hs\sum_{g\in\G_\lambda}\chi(g,h)|g*x_\lambda\rangle_\Ms + |1\rangle_\Hs\sum_{g\in\G_\lambda}\chi(g,h)|(g-u)*x_\lambda\rangle_\Ms \right)\\
				&=\frac{1}{\sqrt{2|\G_\lambda|}}\left(|0\rangle_\Hs\sum_{g\in\G_\lambda}\chi(g,h)|g*x_\lambda\rangle_\Ms + |1\rangle_\Hs\sum_{g\in\G_\lambda}\chi(g+u,h)|g*x_\lambda\rangle_\Ms \right)\\
				&=\frac{1}{\sqrt{2|\G_\lambda|}}\left(|0\rangle_\Hs\sum_{g\in\G_\lambda}\chi(g,h)|g*x_\lambda\rangle_\Ms + |1\rangle_\Hs\sum_{g\in\G_\lambda}\chi(g,h)\chi(u,h)|g*x_\lambda\rangle_\Ms \right)\\
				&=\frac{1}{\sqrt{2}}\left(|0\rangle_\Hs+\chi(u,h)|1\rangle_\Hs\right)|\G_\lambda^h*x_\lambda\rangle
			\end{align*}
			\item Measure $\Hs$ in the basis $B_{h,u}:=\{(|0\rangle_\Hs+\chi(u,h)|1\rangle_\Hs)/\sqrt{2},(|0\rangle_\Hs-\chi(u,h)|1\rangle_\Hs)/\sqrt{2}\}$, giving a bit $b_u\in\{0,1\}$. Discard the $\Hs$ register. In the case that $\$$ is the correct banknote state $|\G_\lambda^h*x_\lambda\rangle$, $b_u$ will be 0 with probability 1, and $\Ms$ will be left in the original banknote state.
		\end{itemize}
		If all the $b_u$ are 0 and the support of $\$$ is contained in $\Xs_\lambda$, then accept. If any of the $b_u$ are 1, or if the support is not contained in $\Xs_\lambda$, reject. We see that for the correct banknote, $\ver$ accepts with probability 1.
	\end{itemize}
\end{construction}

\begin{remark}If using a probabilistic setup of the group action, there are two options. The first is to have $\gen$ set up the group action, and have the parameters be included in the serial number. The second is to have a trusted third party set up the group action, and publish the parameters in a common reference string (CRS). If the goal is only quantum money security, then the former option is always possible, since the security experiment uses an honestly generated serial number. If the goal is quantum lightning security, the former option may not be possible, as the adversary computes the serial number; it may be that there are bad choices of parameters for the group action (and hence the CRS inside the serial number) which make it easy to forge banknotes. Therefore, for quantum lightning security, we would expect using a trusted setup to generate a CRS containing the group action parameters.
\end{remark}




\subsection{Accepting States of the Verifier}

Above we showed that honest banknote states are accepted by the verifier. We now prove that, roughly, honest banknote states are the \emph{only} states accepted by the verifier, with overwhelming probability.

\begin{theorem}\label{thm:reject} Let $|\psi\rangle$ be a state over $\Ms$. Then $\Pr[\ver(h,|\psi\rangle)=1]=\|\langle\psi |\G_\lambda^h*x_\lambda\rangle \|^2(1-2^{-\lambda})+2^{-\lambda}$.
\end{theorem}
In other words, we can treat $\ver(h,|\psi\rangle)$ as projecting onto $|\G_\lambda^h*x_\lambda\rangle$, incurring only a negligible error. The remainder of this subsection is devoted to proving Theorem~\ref{thm:reject}.

\begin{lemma}For $h'\neq h$, $\langle \G_\lambda^{h'}*x_\lambda|\G_\lambda^h*x_\lambda\rangle=0$
\end{lemma}
\begin{proof} 
	\begin{align*}
		\langle \G_\lambda^{h'}*x_\lambda|\G_\lambda^h*x_\lambda\rangle&=\frac{1}{|\G_\lambda|}\sum_{g,g'\in\G_\lambda}\chi(g',h')^{-1}\chi(g,h)\langle g'*x_\lambda|g*x_\lambda\rangle\\
		&=\frac{1}{|\G_\lambda|}\sum_{g\in\G_\lambda}\chi(g,h')^{-1}\chi(g,h)=\frac{1}{|\G_\lambda|}\sum_{g\in\G_\lambda}\chi(g,h-h')=0
	\end{align*}
\end{proof}

Let $|\psi\rangle$ be a a state with support on $\Xs$. Since the $|\G^{h'}*x_\lambda\rangle$ are orthogonal and the number of $h'$ equals the size of $\Xs$, the set $\{|\G_\lambda^{h'}*x_\lambda\rangle\}_{h'}$ forms a basis for the set of states with support on $\Xs$. We can then write $|\psi\rangle=\sum_{h'}\alpha_{h'}|\G_\lambda^{h'}*x_\lambda\rangle$ where $\sum_{h'} \|\alpha_{h'}\|^2=1$. We then have $\|\alpha_h\|^2=\|\langle\psi |\G_\lambda^h*x_\lambda\rangle\|^2$.

Consider a single iteration of $\ver$ on serial number $h$, which samples a random $u$, initializes $\Hs$ to $(|0\rangle+|1\rangle)/\sqrt{2}$, applies the map ${\sf Apply}$,
and then measures $\Hs$ is basis $B_{h,u}$ to get outcome $b$. Let  $|\psi'\rangle$ be the post-measurement state of $\Ms$ conditioned on $b=0$.

\begin{lemma} Conditioned on $u$, $p:=\Pr[b_u=0]=\frac{1}{4}\sum_{h'} \|\alpha_{h'}\|^2 \|1+\chi(u,h-h')\|^2$, and \\$|\psi'\rangle=\frac{1}{\sqrt{p}}\sum_{h'}\alpha_{h'} \frac{1+\chi(u,h-h')}{2}|\G_\lambda^{h'}*x_\lambda\rangle_\Ms$.
\end{lemma}
\begin{proof}By adapting the correctness proof above, we see that the state after applying ${\sf Apply}$ (but before measurement) is:
	\[|\phi\rangle=\sum_{h'\in\G_\lambda}\alpha_{h'}\frac{1}{\sqrt{2}}\left(|0\rangle_\Hs+\chi(u,h')|1\rangle_\Hs\right)|\G_\lambda^{h'}*x_\lambda\rangle_\Ms\]
Then $p$ is length squared of the projection of $|\phi\rangle$ onto $(|0\rangle_\Hs+\chi(u,h)|1\rangle_\Hs)/\sqrt{2}$. Therefore, $p=\frac{1}{4}\sum_{h'} \|\alpha_{h'}\|^2 \|1+\chi(u,h')^{-1}\chi(u,h)\|^2=\frac{1}{4}\sum_{h'} \|\alpha_{h'}\|^2 \|1+\chi(u,h-h')\|^2$. Before re-normalization, the state of $\Ms$ conditioned on $b=0$ is then $\sum_{h'}\alpha_h \frac{1+\chi(u,h-h')}{2}|\G_\lambda^{h'}*x_\lambda\rangle_\Ms$. Re-normalization gives $|\psi'\rangle$.
\end{proof}

We now iterate, replacing $\alpha_{h'}$ with $\alpha_{h'} \frac{1+\chi(u,h-h')}{2}/\sqrt{p}$. This means that after $\lambda$ trials, conditioned on trial $i$ using $u_i$ and giving measurement outcome $b_i$, we have that \[p_{\sf final}:=\Pr[b_1=b_2=\cdots=b_\lambda=0]=\frac{1}{4^\lambda}\sum_{h'} \|\alpha_{h'}\|^2\prod_{1=1}^\lambda \|1+\chi(u_i,h-h')\|^2\]

We now average over $u$ to get $\E[p_{\sf final}]$, the overall probability that $\ver$ accepts $|\psi\rangle$. 
\begin{align*}
	\E[p_{\sf final}]&=\frac{1}{(4|\G_\lambda|)^\lambda}\sum_{h',u_1,\cdots,u_\lambda}\|\alpha_{h'}\|^2\prod_{i=1}^\lambda \|1+\chi(u_i,h-h')\|^2\\
	&=\sum_{h'}\|\alpha_{h'}\|^2\prod_{i=1}^\lambda \left(\frac{1}{4|\G_\lambda|}\sum_u \|1+\chi(u,h-h')\|^2\right)\\
	&=\sum_{h'}\|\alpha_{h'}\|^2\prod_{i=1}^\lambda \left(\frac{1}{4|\G_\lambda|}\sum_u 2+\chi(u,h-h')+\chi(u,h-h')^{-1}\right)\\
	&=\|\alpha_h\|^2+2^{\lambda}\sum_{h'\neq h}\|\alpha_{h'}\|^2=\|\alpha_h\|^2+2^{-\lambda}(1-\|\alpha_h\|^2)\\
	&=\|\alpha_h\|^2(1-2^{-\lambda})+2^{-\lambda}
\end{align*}
This completes the proof of Theorem~\ref{thm:reject}.\qed




\subsection{Computing the Serial Number}\label{sec:compserial}

Here, we show that, given a valid banknote $\$=|\G_\lambda^h*x_\lambda\rangle$ with unknown serial number $h$, it is possible to efficiently compute $h$. This result is not needed anywhere else in the paper, but is included in case it may be useful for future work building on our construction.

\begin{theorem}\label{thm:computeh} There exists a QPT algorithm ${\sf Findh}$ and a negligible function $\negl(\lambda)$ such that, on input $|\G_\lambda^h*x_\lambda\rangle$, outputs $h$ with probability at least $1-\negl(\lambda)$.
\end{theorem}
\begin{proof} Recall from the description of $\ver$ that, for a given $u$ and given $|\G_\lambda^h*x_\lambda\rangle$, we can compute the state $|\tau_{u,h}\rangle|\G_\lambda^h*x_\lambda\rangle$ where $|\tau_{u,h}\rangle:=\frac{1}{\sqrt{2}}\left(|0\rangle_\Hs+\chi(u,h)|1\rangle_\Hs\right)$. Since this process still gives us $|\G_\lambda^h*x_\lambda\rangle$, we can repeat the process, computing $|\tau_{u_i,h}\rangle$ for many different $u_i$. 
	
A naive solution is to compute many copies of $|\tau_{u,h}\rangle$ for some $u\in\G_\lambda$, and then do state tomography to recover $\chi(u,h)$. If $\G_\lambda$ were cyclic, then $\chi(u,h)$ will uniquely determine $h$. The problem is that, since $\G_\lambda$ is exponentially large, the distance between $\chi(u,h)$ as $h$ varies will be exponentially small. This means doing state tomography to a sufficiently small error to recover $h$ would require exponentially-many samples and therefore be inefficient. However, by choosing the $u_i$ carefully and being a bit more thoughtful, we can recover $h$ in polynomial time.

Our strategy will still be to compute many copies of $|\tau_{u,h}\rangle$ for some $u$ and do state tomography to recover an estimate $\hat{\chi}(u,h)$ for $\chi(u,h)$. In time $\poly(\lambda,1/\epsilon,\log(1/\delta))$, we can guarantee that $\Pr[\|\hat{\chi}(u,h)-\chi(u,h)\|<\epsilon]\geq 1-\delta$, for any desired inverse-polynomial $\epsilon$ and exponentially-small $\delta$. We then do this for many different carefully chosen $u$, which allows us to correct the errors arising from tomography, as we now explain.

\paragraph{The cyclic case.} Suppose $\G_\lambda$ is cyclic, and is therefore isomorphic to the additive group $\Z_N$. In this case, $\chi(u,h)=e^{i2\pi uh/N}=\omega_N^{uh}$, where $\omega_N=e^{i2\pi/N}$.

Now when we do state tomorgraphy and recover $\hat{\chi}(u,h)$, we learn an estimate of $uh\bmod N$. In more detail, given real number $a$ and real number $R$, we let $a\bmod R$ denote the unique value of $a-Rk$ for integer $k$ that lies in $(-R/2,R/2]$. Since we know $\|\chi(u,h)\|=1$, we can assume, by normalizing if necessary, that $\|\hat{\chi}(u,h)\|$ is also 1. Therefore, $\hat{\chi}(u,h)=e^{i\theta}$ for some $\theta\in(-\pi,\pi]$. Then by the tomography guarantee, we have $|\;[\theta-(2\pi uh/N)]\bmod 2\pi\;|\leq\epsilon$, or equivalently $|\;[N\theta/2\pi-uh]\bmod N\;|\leq \epsilon N/2\pi$, except with negligible probability 


This means we reduce the computation of $h$ to the following classical task: we get to choose arbitrary $u_i\in\Z_N$ for $i=1,\dots,n$. In response, we learn $u_ih+e_i\bmod N$, where $e_i$ is some random variable in $[-\epsilon N/2\pi,\epsilon N/2\pi]$. In vector notation, we can write this as choosing a vector $\uv\in\Z_N^n$, and receiving $h\uv+\ev\bmod N$, where $\ev$ is a vector whose components are independent random variables that are guaranteed to be in $[-\epsilon N/2\pi,\epsilon N/2\pi]$. The goal is to compute $h$.

This looks very similar to a 1-dimensional version of the LWE problem~\cite{STOC:Regev05} (or more accurately, bounded distance decoding) except that in our case we get to choose the vector $\uv$ in whatever way so as to make the task \emph{easy}. We can then use known techniques to find $h$. In particular, we can choose $\uv=(1,2,4,,8,\cdots,2^{n-1})$ where $n=\lceil\log_2 N\rceil$. This is known as the gadget ``matrix''\footnote{In our case the matrix has width 1, whereas in general applications the matrix will have many columns.}. Importantly, $\uv$ has an efficiently computable ``trapdoor''. That is, write $N=\sum_{i=0}^{n-1} 2^i\times N_i$ for bits $N_i$, and let 
\[\Am=\left(\begin{array}{cccccccc}
	2&-1&0&0&0&\cdots&0&0\\
	0&2&-1&0&0&\cdots&0&0\\
	0&0&2&-1&0&\cdots&0&0\\
	\vdots&\vdots&\ddots&\ddots&\ddots&\ddots&\vdots&\vdots\\
	0&0&0&0&0&\cdots&2&-1\\
	N_0&N_1&N_2&N_3&N_4&\cdots&N_{n-2}&N_{n-1}
\end{array}\right)\]
Then $\Am$ is full rank over the integers, but satisfies $\Am\cdot \uv\bmod N=0^n$. Set $\epsilon=\pi/3n$. Thus, given $\vv:=\uv h+\ev\bmod N$, we can compute
\[\Am^{-1}\cdot(\Am\cdot\vv\bmod N)=\Am^{-1}\cdot (\Am\cdot\ev\bmod N)=\Am^{-1}\cdot (\Am\cdot\ev)=\ev\enspace.\]
Above, we used the fact that the entries of $\Am\cdot\ev$ have absolute value at most $n\times \max_{i,j}|\Am_{i,j}|\times\max_j |\ev_j|=n\times 2 \times \epsilon N/2\pi\leq N/3<N/2$, meaning that reduction mod $N$ has no effect.

\medskip

\noindent Once we compute $\ev$, we can then compute $h\uv=\vv-\ev$, and then $h$ is just the first component.


\paragraph{The general case.} We cow consider the case of general groups. Let $\G_\lambda=\Z_{n_1}\times\Z_{n_2}\times\cdots\times\Z_{n_k}$. Write $h=(h_1,\cdots,h_k)$. By choosing $u=(u_1,0,\cdots,0)$, the task of computing $h_1$ reduces to the case where $\G_\lambda=\Z_{n_1}$, which can be solved via the algorithm above. Likewise, we can compute $h_2,\cdots, h_k$, and hence $h$.\end{proof}





\iffalse 
\subsection{An Attack in Certain Group Actions, and an Alternative Scheme}\label{sec:alternate}

As discussed in Section~\ref{sec:intro}, there may be a security vulnerability in the case where $\Xs$ is dense in the set of bit-strings. In particular, for such $\Xs$ it is possible to construct $\sum_{y\in\Xs_\lambda}|y\rangle=|\Gs^0*x\rangle$. In addition to allowing for constructing arbitrary banknotes with the particular serial number $h=0$, this also allows for generating pais of banknotes with closely-related serial numbers, which may indicate a weakness of the protocol as quantum lightning.

In more detail, we can first generate $|\Gs^0*x\rangle$, and then using the CNOT gate construct $\sum_{y\in\Xs_\lambda}|y,y\rangle$. Next, we apply the algorithm guaranteed by Theorem~\ref{thm:computeh} to the first register, which measures the register in the basis $\{|\G_\lambda^h\rangle\}_h$. Call the outcome $h$. The state of the two registers then collapses to the (unnormalized) state:
\begin{align*}
	&\left(|\G_\lambda^h\rangle\langle\G_\lambda^h|\otimes \Id\right)\sum_{y\in\Xs_\lambda}|y,y\rangle
	=\left(|\G_\lambda^h\rangle\langle\G_\lambda^h|\otimes \Id\right)\sum_{g\in\G_\lambda}|g*x_\lambda,g*x_\lambda\rangle\\
	&=\sum_g |\G_\lambda^h\rangle|g*x_\lambda\rangle \langle\G_\lambda^h|g*x_\lambda\rangle
	=|\G_\lambda^h\rangle \sum_g \chi(g,h)^{-1}|g*x_\lambda\rangle\\
	&=|\G_\lambda^h\rangle \sum_g \chi(g,-h)|g*x_\lambda\rangle=\sqrt{|\G|}|\G_\lambda^h\rangle|\G_\lambda^{-h}\rangle
\end{align*}
Thus, after normalizing, we get $|\G_\lambda^h\rangle|\G_\lambda^{-h}\rangle$, two banknotes whose serial numbers are negatives of each other. If $h=-h$, then we obtain two copies of the $|\G_\lambda^h\rangle$ with the same serial number, representing a break of quantum lightning security. We note, however, that for this attack to actually break security with non-negligible probability, the fraction of group elements that have order 2 must be large. In such a group action, however, discrete logarithms are easy anyway due to Simons algorithm~\cite{FOCS:Simon94}.

Nevertheless, in light of this potential attack, we briefly describe an alternative scheme which leverages the attack in the construction to give banknotes that appears resilient to the attack.

\begin{construction}\label{constr:alt} Let $\gen',\ver'$ be the following QPT procedures:
\begin{itemize}
	\item $\gen'(1^\lambda)$: Initialize quantum registers $\Ms_0,\Ms_1$ to states $\frac{1}{\sqrt{|\Xs_\lambda|}}\sum_x|x\rangle_{\Ms_0}$ and $|0\rangle_{\Ms_1}$, respectively. Then do the following:
		\begin{itemize}
			\item Apply the CNOT operation to $\Ms_0,\Ms_1$, resulting in the state $\frac{1}{\sqrt{|\Xs_\lambda|}}|x,x\rangle_{\Ms_0,\Ms_1}$.
			\item Apply the algorithm guaranteed by Theorem~\ref{thm:computeh} to $\Ms_0$ and measure the resulting $h$. Then the joint system of $\Ms_0,\Ms_1$ collapses to $ |\G_\lambda^h\rangle_{\Ms_0}|\G_\lambda^{-h}\rangle_{\Ms_1}$.
			\item Output $(\sigma=h,\$=|\G_\lambda^h\rangle_{\Ms_0}|\G_\lambda^{-h}\rangle_{\Ms_1})$.
		\end{itemize}
	\item $\ver'(\sigma,\$)$. Parse $\sigma$ as $h$ and parse $\$$ as potentially entangled states $\$_0,\$_1$. First, immediately abort and reject if $h=0$. For honest banknotes, $h=0$ with only negligible probability. Then run $\ver(h,\$_0)$ and $\ver(-h,\$_1)$, where $\ver$ is from Construction~\ref{constr:main}. $\ver'$ accepts if and only if $h\neq 0$ and both applications of $\ver$ accept.
\end{itemize}
\end{construction}

Now in order to break quantum lightning security, one must find 4 banknotes of Construction~\ref{constr:main} such that the first two have the same serial number $h\neq 0$, and the last two have the serial number $-h$. Even if $\G_\lambda$ has elements of order 2, this requires generating four banknotes with the same serial number.

If we try to adapt the above attack for generating more than two banknotes, what we get is 4 random banknotes whose serial numbers $h_1,h_2,h_3,h_4$ sum to 0: $h_1+h_2+h_3+h_4=0$. In particular, there does not appear to be any way to generate 4 banknotes satisfying 3 linear constraints, as needed to break quantum lightning security. Hence, for group actions where it is possible to construct the uniform superposition over $\Xs_\lambda$, Construction~\ref{constr:alt} gives a plausibly secure scheme.
\fi










\subsection{Security}\label{sec:security}

Here, we prove the security of our scheme under two new but plausible assumptions on group actions. For now, we assume that it is impossible to obliviously sample set elements. This is false on group actions based on elliptic curves, but our proof here is simpler and provides the main intuition. In Section~\ref{sec:oblivious}, we address the case where oblivious sampling is possible.

\paragraph{The Knowledge of Group Element Assumption (KGEA).} This assumption states, informally, that any algorithm that produces a set element $y$ must ``know'' $g$ such that $y=g*x_\lambda$. We first discuss the case where it is infeasible to sample set elements in the group action. Later, we will discuss how to model the assumption when there is a such a sampling algorithm. In the classical setting, the KGEA assumption would be formalized as follows:

\begin{assumption}\label{def:ckgea} The \emph{classical knowledge of group element assumption} (C-KGEA) holds on a group action $(\G,\Xs,*)$ if the following is true. For any probabilistic polynomial time (PPT) adversary $\As$, there exists a PPT ``extractor'' $\Es$ and a negligible $\epsilon$ such that:
	\[\Pr\left[y\in\Xs\wedge y\neq g*x_\lambda:\substack{y\gets\As(1^\lambda; r)\\g\gets\Es(1^\lambda,r)}\right]\leq\epsilon(\lambda)\enspace .\]
	Above, $r$ are the random coins given to $\As$, which are also given to $\Es$, and the probability is taken over uniform $r$ and any additional randomness of $\Es$.
\end{assumption}
In other words, if $\As$ outputs any set element, it must ``know'' how to derive that set element from $x_\lambda$, since it can compute $g$ such that $y=g*x_\lambda$ using $\Es$ and its random coins. Note that once the random coins are fixed, $\As$ is deterministic. %Also, note that it is possible to consider relaxed versions of this definition, where we only require $\Es$ to output $g$ with non-negligible probability.

As observed by~\cite{EC:LiuMonZha23}, when moving to the quantum setting, the problem with Assumption~\ref{def:ckgea} is that quantum algorithms do not have to flip random coins to generate randomness, and instead their output may be a measurement applied to a quantum state, the result being inherently randomized even if the quantum state is fixed. Thus there is no meaningful way to give the same random coins to $\Es$.

The solution used in~\cite{EC:LiuMonZha23} is to, instead of giving $\Es$ the same inputs as $\As$, give $\Es$ the remaining state of $\As$ at the \emph{end} of the computation. This requires some care, since an algorithm can of course forget any bit of information by simply throwing it away. A more sophisticated way to lose information is to perform other measurements on the state, say measuring in the Fourier basis. The solution in~\cite{EC:LiuMonZha23} is to require that $\As$ makes no measurements at all, \emph{except} for measuring the final output. Note that the Principle of Delayed Measurement implies that it is always possible without loss of generality to move all measurements to the final output. Then $\Es$ is given both the output and the remaining quantum state of $\As$, and tries to compute $g$. Note that in the classical setting, if we restrict to \emph{reversible} $\As$, this formulation of giving $\Es$ the final state of $\As$ is equivalent to given $\Es$ the randomness, since the randomness can be computed by reversing $\As$. Similar to how we can assume a quantum $\As$ makes all its measurements at the end, in we can always assume without loss of generality that a classical $\As$ is reversible. Thus, in the classical setting these two definitions coincide. Adapting to our setting, this approach yields the following assumption:

\begin{assumption}\label{def:qkgea} The \emph{quantum knowledge of group element assumption} (Q-KGEA) holds on a group action $(\G,\Xs,*)$ if the following is true. For any quantum polynomial time (QPT) adversary $\As$ which performs no measurements except for its final output, there exists a QPT extractor $\Es$ and negligible $\epsilon$ such that 
	\[\Pr\left[y\in\Xs\wedge y\neq g*x_\lambda:\substack{(y,|\psi\rangle)\gets\As(1^\lambda)\\g\gets\Es(y,|\psi\rangle)}\right]\leq\epsilon(\lambda)\enspace .\]
\end{assumption}
Above, $y$ is considered as the output of $\As$, and the only measurements applied to $\As$ is the measurement of $y$ to obtain the output. 

\medskip

In group actions based on elliptic curves, it is possible to directly sample set elements. While set elements generated in this way have no obvious relation to other set elements, the ability to generate set elements without applying the group action would technically contradict the KGEA assumptions as defined in Assumptions~\ref{def:ckgea} and~\ref{def:qkgea}. This same issue was present in the knowledge of path assumption in~\cite{EC:LiuMonZha23}. In Section~\ref{sec:oblivious}, we discuss a different approach to remedy this issue that seems more robust. For now we proceed with the basic setting where we assume Q-KGEA as in Assumption~\ref{def:qkgea} is true.


\paragraph{The Discrete Log Assumption, with Help.} We now define a strengthening of the Discrete Log assumption (Assumption~\ref{def:dlog}), which allows the adversary limited query access to a computational Diffie Hellman (CDH) oracle. 

\begin{assumption}\label{def:dlogminimalcdh} We say that the \emph{Discrete Log with a single minimal CDH query} assumption (DLog/1-minCDH) assumption holds if the following is true. For any QPT adversary $\As$ playing the following game, parameterized by $\lambda$, there is a negligible $\epsilon$ such that $\As$ wins with probability at most $\epsilon(\lambda)$:
	\begin{itemize}
		\item The challenger, on input $\lambda$, chooses a random $g\in\G_\lambda$. It sends $\lambda$ to $\As$
		\item $\As$ submits a superposition query $\sum_{y\in\Xs,z\in\{0,1\}^*}\alpha_{y,z}|y,z\rangle$. Here, $y$ is a set element that forms the query, and $z$ is the internal state of the adversary when making the query. The challenger responds with $\sum_{y\in\Xs,z\in\{0,1\}^*}\alpha_{y,z}|(-g)*y,z\rangle$\enspace\footnote{Note that this operation is unitary and efficiently computable since $y\mapsto(-g)*y$ is efficiently computable and efficiently reversible given $g$.}. 
		\item The challenger sends $g*x$ to $\As$.
		\item $\As$ outputs a guess $g'$ for $g$. It wins if $g'=g$.
	\end{itemize}
\end{assumption}

Note that Assumption~\ref{def:dlogminimalcdh} uses a ``minimal'' oracle for the CDH oracle, meaning is replaces $y$ with $(-g)*y$. This is only a possibility because $y\mapsto(-g)*y$ is reversible; otherwise the query would not be unitary. The minimal oracle, however, is somewhat non-standard. So we here define a slightly different assumption which uses ``standard'' oracles:

\begin{assumption}\label{def:dlogstandardcdh} We say that the \emph{Discrete Log with a double standard CDH query} assumption (DLog/2-stdCDH) assumption holds if the following is true. For any QPT adversary $\As$ playing the following game, parameterized by $\lambda$, there is a negligible $\epsilon$ such that $\As$ wins with probability at most $\epsilon(\lambda)$:
	\begin{itemize}
		\item The challenger, on input $\lambda$, chooses a random $g\in\G_\lambda$. It sends $\lambda$ to $\As$.
		\item $\As$ submits a superposition query $\sum_{y\in\Xs,w,z\in\{0,1\}^*}\alpha_{y,w,z}|y,w,z\rangle$. Here, $y$ is a set element that forms the query, $w$ is a string that forms the response register, and $z$ is the internal state of the adversary when making the query. The challenger responds with $\sum_{y\in\Xs,w,z\in\{0,1\}^*}\alpha_{y,w,z}|y,w\oplus[(-g)*y],z\rangle$. 
		\item $\As$ submits a second superposition query $\sum_{y\in\Xs,w,z\in\{0,1\}^*}\alpha_{y,w,z}|y,w,z\rangle$. The challenger responds with $\sum_{y\in\Xs,w,z\in\{0,1\}^*}\alpha_{y,w,z}|y,w\oplus[g*y],z\rangle$. 
		\item The challenger sends $g*x$ to $\As$.
		\item $\As$ outputs a guess $g'$ for $g$. It wins if $g'=g$.
	\end{itemize}
\end{assumption}

\begin{lemma}\label{lem:standard2min} If DLog/2-stdCDH (Assumption~\ref{def:dlogstandardcdh}) holds in a group action, then so does DLog/1-minCDH (Assumption~\ref{def:dlogminimalcdh}).
\end{lemma}
\begin{proof}Consider a supposed adversary $\As$ for DLog/1-minCDH with non-negligible winning probability $\epsilon$. We construct a new adversary $\Bs$ for DLog/2-stdCDH with the same non-negligible winning probability as follows. $\Bs$ runs $\As$ until $\As$ makes its superposition query $\sum_{y\in\Xs,z\in\{0,1\}^*}\alpha_{y,z}|y\rangle_\Ys|z\rangle_\Zs$. $\Bs$ then initializes a new register $\Rs$ with the state $|0\rangle$. $\Bs$ then submits $\sum_{y\in\Xs,z\in\{0,1\}^*}\alpha_{y,z}|y\rangle_\Ys|0\rangle_\Rs|z\rangle_\Zs$ as its first query. In response, it receives $\sum_{y\in\Xs,z\in\{0,1\}^*}\alpha_{y,z}|y\rangle_\Ys|(-g)*y\rangle_\Rs|z\rangle_\Zs$. Now it swaps the roles of $\Rs$ and $\Ys$, and makes its second query on $\sum_{y\in\Xs,z\in\{0,1\}^*}\alpha_{y,z}|(-g)*y\rangle_\Ys|y\rangle_\Rs|z\rangle_\Zs$. In response it receives $\sum_{y\in\Xs,z\in\{0,1\}^*}\alpha_{y,z}|(-g)*y\rangle_\Ys|0\rangle_\Rs|z\rangle_\Zs$. Then it discards the $\Rs$ register, and sends the resulting state $\sum_{y\in\Xs,z\in\{0,1\}^*}\alpha_{y,z}|(-g)*y\rangle_\Ys|z\rangle_\Zs$ to $\As$. 
	
Afterward, when $\Bs$ receives $g*x$, it forwards it to $\As$, and then outputs whatever $g'$ that $\As$ outputs. Thus, we see that $\Bs$ correctly simulates the view of $\As$, and thus the probability $\Bs$ wins is the same as $\As$, namely $\epsilon$. 
\end{proof}

From this point forward, we will use DLog/1-minCDH as our assumption; Lemma~\ref{lem:standard2min} then shows that we could have instead used DLog/2-stdCDH.

\paragraph{The security proof.} We are now ready to formally state and prove security.

\begin{theorem}\label{thm:main} Assuming Q-KGEA (Assumption~\ref{def:qkgea}) and DLog/1-minCDH (Assumption~\ref{def:dlogminimalcdh}) both hold on a group action $(\G,\Xs,*)$, then Construction~\ref{constr:main} is a quantum lightning scheme.
\end{theorem}
\begin{remark}Before proving Theorem~\ref{thm:main}, we briefly discuss how to handle the case of non-uniform attackers, since in this setting quantum lightning is insecure without some modifications. Note that even against non-uniform attackers, DLog/1-minCDH still plausibly holds. However, Q-KGEA certainly does not, as a non-uniform attacker may have a $y$ hard-coded for which it does not know the discrete log with $x_\lambda$. As discussed in Section~\ref{sec:prelim}, there are several possibilities. 
\begin{itemize}
	\item The first is to restrict to non-uniform attackers that only have classical advice. While classical advice does not appear to be useful in breaking Construction~\ref{constr:main}, it still allows for breaking Q-KGEA; thus while our scheme may be secure in this setting, the security proof would be vacuous.
	\item The second is to use a probabilistically generated group action, and define Q-KGEA and DLog/1-minCDH accordingly. For quantum money security, it would suffice to have $\gen$ create the parameters of the group action and then include them in the serial number, since the serial number is generated honestly. For quantum lightning security, we would instead need the parameters to be generated by a trusted third party and then placed in a common random string (CRS). 
	\item The final option is to use the human ignorance approach~\cite{VIETCRYPT:Rogaway06}, where we explicitly state our security theorem as transforming a quantum lightning adversary into a Q-KGEA adversary; while such Q-KGEA adversaries exist in the non-uniform setting without a CRS, they are presumably unknown to human knowledge. As a consequence, a quantum lightning attacker, while existing, would likewise be unknown to human knowledge. 
\end{itemize}
For simplicity, state and prove Theorem~\ref{thm:main} in the uniform setting; either probabilistically generating the group action or using human ignorance would require straightforward modifications. 
\end{remark}

\noindent We now are ready to prove Theorem~\ref{thm:main}.

\begin{proof}Consider a QPT quantum lightning adversary $\As$ which breaks security with non-negligible success probability $\epsilon$. Since an adversary can always tell if it succeeded by running $\ver$, we can run $\As$ multiple times to boost the probability of a successful break. In particular, we can run $\As$ for $\lambda\epsilon$, and at except with probability $1-2^{-\Theta(\lambda)}$, at least one of the runs will succeed. This allows us to conclude without loss of generality that $\As$ has success probability $1-2^{-\Theta(\lambda)}$. By Theorem~\ref{thm:reject}, we also know that if $\As$ outputs a serial number $h$, the states outputted are exponentially close to two copies of $|\G_\lambda^h*x_\lambda\rangle$.

For simplicity in the following proof, we will assume the probability of passing verification is actually 1; it is straightforward to adapt the proof to the case of negligible error.
	
Next, we purify $\As$, and assume that before measurement, $\As$ outputs a pure state $|\psi\rangle$. By our assumption that the success probability is 1, $|\psi\rangle$ will have the form
\[|\psi\rangle=\sum_{h}\alpha_h|\phi_h\rangle|\G_\lambda^h*x_\lambda\rangle|\G_\lambda^h*x_\lambda\rangle=\frac{1}{|\G_\lambda|}\sum_h\alpha_h|\phi_h\rangle\chi(h,g_1+g_2)|g_1*x\rangle_{\Ms_1}|g_2*x\rangle_{\Ms_2}\enspace.\]
Above, $|\phi_h\rangle$ are arbitrary normalized states representing whatever state the adversary contains after outputting its banknotes, and $\sum_h\|\alpha_h\|^2=1$.

Now consider the adversary $\Bs$ which first constructs $|\psi\rangle$, and then measures the register $\Ms_2$ to obtain $y_2=g_2*x$. 
\begin{claim}$g_2$ is uniform in $\G$.
\end{claim}
\begin{proof}Consider additionally measuring $\Ms_1$ in the basis $\{|\G_\lambda^h*x_\lambda\rangle\}$. This this measurement is on a different register than the measurement on $\Ms_2$, measuring $\Ms_1$ does not affect the output distribution of $\Ms_2$ (though the results may be correlated). But the measurement on $\Ms_1$ determines $h$, and conditioned on $h$, $\Ms_2$ collapses to $|\G_\lambda^h*x_\lambda\rangle$. Regardless of what $h$ is, measuring $|\G_\lambda^h*x_\lambda\rangle$ gives a uniformly random element in $\Xs$. Thus, even without measuring $\Ms_1$, the measurement of $\Ms_2$ gives a uniform element in $\Xs$.\end{proof}

Therefore, after measuring $\Ms_2$, the state $|\psi\rangle$ then collapses to \[|\psi_{g_2*x_\lambda}\rangle:=\frac{1}{\sqrt{|\G_\lambda|}}\sum_h\alpha_h|\phi_h\rangle\chi(h,g_1+g_2)|g_1*x\rangle_{\Ms_1}\enspace .\]

\begin{claim}There is a QPT procedure ${\sf Map}$ such that ${\sf Map}(g,|\psi_{y}\rangle)=|\psi_{g*y}\rangle$.
\end{claim}
\begin{proof}${\sf Map}$ simply applies the map $y\mapsto (-g)*y$ to $\Ms_1$ in superposition. Then we have that:
	\begin{align*}
		{\sf Map}(g,|\psi_{g_2*x_\lambda}\rangle)&=\frac{1}{\sqrt{|\G_\lambda|}}\sum_h\alpha_h|\phi_h\rangle\chi(h,g_1+g_2)|(g_1-g)*x\rangle_{\Ms_1}\\
		&=\frac{1}{\sqrt{|\G_\lambda|}}\sum_h\alpha_h|\phi_h\rangle\chi(h,g_1'+g+g_2)|g_1'*x\rangle_{\Ms_1}=|\psi_{(g+g_2)*y}\rangle=|\psi_{g*(g_2*y)}\rangle
	\end{align*}
Above we used the change of variables $g_1'=g_1-g$.
\end{proof}

Now we invoke Q-KGEA (Assumption~\ref{def:qkgea}) on the adversary $\Bs$. Since $\Bs$ always outputs a valid set element, this means there is another QPT algorithm $\Es$ such that 
\[\Pr[\Es(g_2*x_\lambda,|\psi_{g_2*x_\lambda}\rangle)=g_2]\geq 1-\negl(\lambda)\]

Above, the probability is over $g_2*x_\lambda$, as well as any randomness incurred when executing $\Es$. We note by a simple random self-reduction that we can insist the above probability holds for \emph{all} $g_2*x_\lambda$, where the randomness is only over $\Es$. Indeed, given $|\psi_{g_2*x_\lambda}\rangle,g_2*x_\lambda$, we can choose a random $g$ and compute $g_2'*x_\lambda$ as $g*(g_2*x_\lambda)$ where $g_2'=g+g_2$. Likewise, we can compute $|\psi_{g_2'*x_\lambda}\rangle$ as ${\sf Map}(g,|\psi_{g_2*x_\lambda}\rangle)$. This gives a random instance on which to apply $\Es$, giving $g_2'$ with probability $1-\negl(\lambda)$, regardless of $g_2$. Then we can compute $g_2=g_2'-g$. We thus compute $g_2$ with overwhelming probability, even in the worst case. We will therefore assume without loss of generality that this is the case for $\Es$.

For simplicity, we will actually assume that the probability is 1; it is straightforward to handle the case the probability is negligibly close to 1. By the Gentle Measurement Lemma~\cite{Winter99}, $\Es$ can compute $g_2$ without altering the state $|\psi_{g_2*x}\rangle$. Thus, by combining $\Bs$ and $\Es$, we can compute both $|\psi_{g_2*x}\rangle$ and $g_2$ with probability 1. We can then compute ${\sf Map}(-g_2,|\psi_{g_2*x_\lambda}\rangle)=|\psi_{x_\lambda}\rangle$.

We now describe a new algorithm $\Cs$ which breaks DLog/1-minCDH (Assumption~\ref{def:dlogminimalcdh}). $\Cs$ works as follows:
\begin{itemize}
	\item It constructs $|\psi_{x_\lambda}\rangle$ as above.
	\item It makes its query to the DLog/1-minCDH challenger, setting $\Ms_1$ as the query register. This query simulates the operation ${\sf Map}(g,\cdot)$, where $g$ is the group element chosen by the challenger. Thus, at the end of the query, $\Cs$ has $|\psi_{g*x_\lambda}\rangle$.
	\item Now upon receiving $g*x_\lambda$ from the challenger, run $\Es(g*x_\lambda,|\psi_{g*x_\lambda}\rangle)$. By the guarantees of $\Es$, the output will be $g$.
\end{itemize}
Thus we see that $\Cs$ breaks the DLog/1-minCDH assumption. This completes the security proof.\end{proof}





\subsection{Security under existence of obliviously sampled elements}\label{sec:oblivious}

As previously mentioned, the ability to obliviously sample set elements in group actions based on elliptic curves means the KGEA assumption as stated is false. One possible remedy, used in~\cite{EC:LiuMonZha23}, explicitly assumes a probabilistic classical procedure $S()$ for obliviously sampling set elements, and modifies the KGEA assumption so that the extractor either outputs (1) an explanation relative to $x_\lambda$ \emph{or} (2) an explanation relative to some input $y$ together with the random coins $r$ that are fed into $S$ so that $y=S(r)$. The problem with this approach is that the assumption depends on explicitly modeling the oblivious sampling procedure, and if another oblivious sampling procedure is found, it would contradict the assumption.


In order to give a more robust proof, we here devise an alternate solution. Our key idea is to observe that, while obliviously sampling elements strictly speaking violates the KGEA assumption, it does not seem to yield any assistance in actually breaking a quantum money scheme based on group actions, since the elements obliviously sampled will be unrelated to anything else. More generally, we can consider a general cryptographic game that an adversary may play with a challenger. For ``nice'' games (which we will define shortly), in particular games that only use the group action interface and do not themselves obliviously sample elements, it seems that giving the adversary the ability to obliviously sample elements is no help. We therefore postulate that, for any adversary $\As$ that wins such a nice game, there is a different adversary $\As'$ for which the KGEA assumption can be appled, yielding an extractor \emph{for that} $\As'$. Thus, even if the original $\As$ can obliciously sample elements, we essentially assume that $\As'$ cannot, and therefore $\Es$ is possible. We now make this intuition precise.

\paragraph{Generic Group Action Games.} We first introduce the notion of generic group action games. Note that we will only be interested in \emph{games} that are given by generic algorithms; we will always treat the adversary as non-generic.

Briefly, a generic group action game is given by an interactive algorithm (``challenger'') ${\sf Ch}$. ${\sf Ch}$ is limited to only performing group action computations that are ``generic'' and only interacts with the group action through oracles implementing the group action interface. Specifically, a generic algorithm is an oracle-aided algorithm $\Bs$ that has access to oracles ${\sf GA}=({\sf Start},{\sf Act},{\sf Mem})$. Here, ${\sf Start}$ is the oracle that takes as input the empty query, and outputs a string $\tilde{x}$ representing $x_\lambda$. ${\sf Act}$ is the oracle that takes as input a group element $g\in\G_\lambda$ and a string $\tilde{y}$ representing a set element $y$, and outputs a string $\tilde{z}$ representing $z=g*x$. Finally, ${\sf Mem}$ is a membership testing oracle, that tests is a given string $\tilde{x}$ represents an actual set element. From a generic game, we obtain a standard model game by implementing the oracles ${\sf Start},{\sf Act},{\sf Mem}$ with the algorithms for an actual group action: ${\sf Start}$ outputs the actual set element $x_\lambda$, ${\sf Act}$ is the group action $*$, and ${\sf Mem}$ is the membership tester for the set $\Xs_\lambda$. For a concrete group action $(\G,\Xs,*)$, we denote this standard-model game by ${\sf Ch}^{(\G,\Xs,*)}$.

Notice that a generic group action game cannot obliviously sample elements, since it is not given any interface to the group action other than the group action itself.

For any algorithm $\As$, we say the algorithm $\delta(\lambda)$-breaks ${\sf Ch}^{(\G,\Xs,*)}$ if ${\sf Ch}^{(\G,\Xs,*)}(1^\lambda)$ outputs 1 with probability at least $\delta(\lambda)$ when interacting with $\As$.

We say that ${\sf Ch}$ is one-round if it sends a single classical string to $\As$, and then receives a single quantum message from $\As$, before deciding if $\As$ wins.


\paragraph{Our modified KGEA assumption.} We now give our modified KGEA assumption.

\begin{assumption}\label{def:qmkgea} The \emph{quantum modified knowledge of group element assumption} (Q-mKGEA) holds on a group action $(\G,\Xs,*)$ if the following is true. Consider a one-round generic group action game ${\sf Ch}$ and any quantum polynomial time (QPT) adversary $\As$ that $1-\delta$-breaks ${\sf Ch}^{(\G,\Xs,*)}$ for a negligible $\delta$. Write the message from $\As$ to ${\sf Ch}^{(\G,\Xs,*)}$ as $\rho_{1,2}$, as a joint system over two registers $1,2$. Consider measuring the first register, to obtain a set element $y$. Denote this as $(y,|\psi\rangle)\gets\As'(1^\lambda)\Leftrightarrow {\sf Ch}^{(\G,\Xs,*)}(1^\lambda)$. Then for all such $\delta,\As,{\sf Ch}$, there exists another  negligible $\delta'$, a QPT $\As'$ that also $1-\delta'$-breaks ${\sf Ch}^(\G,\Xs,*)$, and moreover there exists a QPT extractor $\Es$ and negligible $\epsilon$ such that 
		\[\Pr\left[y\in\Xs\wedge y\neq g*x_\lambda:\substack{(y,|\psi\rangle)\gets\As'(1^\lambda)\Leftrightarrow {\sf Ch}^{(\G,\Xs,*)}(1^\lambda)\\g\gets\Es(y,|\psi\rangle)}\right]\leq\epsilon(\lambda)\enspace .\]
\end{assumption}

Intuitively, this assumption says that if $\As$ wins some game, we might not be able to apply the KGEA extractor to it. However, there is some other $\As'$ that also wins the game, and that we \emph{can} apply the KGEA extractor to. 

\begin{theorem}\label{thm:main2} Assuming Q-mKGEA (Assumption~\ref{def:qmkgea}) and DLog/1-minCDH (Assumption~\ref{def:dlogminimalcdh}) both hold on a group action $(\G,\Xs,*)$, then Construction~\ref{constr:main} is a quantum lightning scheme.
\end{theorem}
\begin{proof}The proof is the same as the proof of Theorem~\ref{thm:main}, except that we observe that the quantum lightning experiment is a generic group action game. As such, after obtaining an adversary $\As$ that wins the quantum lightning game with probability $1-\negl$, we immediately switch to the adversary $\As'$ that is assumed to exist by Assumption~\ref{def:qmkgea}, and apply the extractor $\Es$ to the output of $\As'$.
\end{proof}




\iffalse

. We use the notation $S(1^\lambda;r)$ to denote running $S$ with random coins $r$. To make our notation simpler, we will assume without loss of generality that $S(1^\lambda;0^\ell)=x_\lambda$. We first give a modified KGEA assumption, which allows for deriving set elements from either $S$ or $x_\lambda$.

\begin{assumption}\label{def:qmkgea} The \emph{quantum modified knowledge of group element assumption} (Q-mKGEA) holds on a group action $(\G,\Xs,*)$ if the following is true. For any quantum polynomial time (QPT) adversary $\As$ which performs no measurements except for its final output, there exists a QPT extractor $\Es$ and negligible $\epsilon$ such that 
	\[\Pr\left[y\in\Xs\wedge y\neq g*S(1^\lambda;r):\substack{(y,|\psi\rangle)\gets\As(1^\lambda)\\(g,r)\gets\Es(y,|\psi\rangle)}\right]\leq\epsilon(\lambda)\enspace .\]
\end{assumption}

We will also need that it is infeasible to compute the discrete log between an obliviously sampled element and $x_\lambda$. This appears true for obliviously sampled set elements on elliptic curves. In particular, if it \emph{is} possible to compute the discrete log between some oblivously sampled element $y$ and $x_\lambda$, then outputting $y$ seems rather useless for meaningfully violating Q-KGEA, as it is anyway possible to explain $y$ in terms of $x_\lambda$.

\begin{assumption}\label{def:dlogobl} We say that the obliviously sampled DLog assumption (DLog/oblivious) holds if, for all QPT adversaries $\As$, there is a negligible $\epsilon$ such that $\Pr[r\neq 0\wedge S(1^\lambda;r)=g*x_\lambda:(g,r)\gets\As(1^\lambda)]\leq\epsilon(\lambda)$.
\end{assumption}

\begin{theorem}\label{thm:main2} Assuming Q-mKGEA (Assumption~\ref{def:qkgea}) holds on a group action $(\G,\Xs,*)$, and both DLog/1-minCDH (Assumption~\ref{def:dlogminimalcdh}) and DLog/oblivious (Assumption~\ref{def:dlogobl}) hold on $(\G,\Xs,*)$ against adversaries with non-uniform quantum advice, then Construction~\ref{constr:main} is a quantum lightning scheme.
\end{theorem}
We note that being secure under quantum non-uniform advice means that hardness holds even if the adversary is supplies some quantum state $\rho_\lambda$ that depends only on the security parameter. While stronger than uniform hardness or hardness under classical non-uniform advice, assuming shardness under quantum non-uniform advice is a standard modeling in the quantum setting.

\begin{proof}The proof follows closely the proof of Theorem~\ref{thm:main}. We assume as before that the quantum lightning adversary first computes a state
	\[|\psi\rangle=\sum_{h}\alpha_h|\phi_h\rangle|\G_\lambda^h*x_\lambda\rangle|\G_\lambda^h*x_\lambda\rangle=\frac{1}{|\G_\lambda|}\sum_h\alpha_h|\phi_h\rangle\chi(h,g_1+g_2)|g_1*x\rangle_{\Ms_1}|g_2*x\rangle_{\Ms_2}\enspace.\]
	and then outputs $\Ms_1,\Ms_2$ as the two banknote states. As before, we will imagine an adversary $\Bs$ which measures $\Ms_2$ to get $g_2*x_\lambda$ for a uniform $g_2\in\G_\lambda$, in which case the state $|\psi\rangle$ then collapses to \[|\psi_{g_2*x_\lambda}\rangle:=\frac{1}{\sqrt{|\G_\lambda|}}\sum_h\alpha_h|\phi_h\rangle\chi(h,g_1+g_2)|g_1*x\rangle_{\Ms_1}\enspace .\]
	
We now invoke Q-mKGEA on $\Bs$ to obtain an extractor $\Es$ such that:

\[\Pr[g_2*x_\lambda= g*S(1^\lambda;r):(g,r)\gets\Es(g_2*x_\lambda,|\psi_{g_2*x_\lambda}\rangle)]\geq1-\negl(\lambda)\enspace .\]
	
We break into two cases. The first is $r=0$, in which case $g=g_2$. Call the probability of this event $\epsilon_=$. The second is $r\neq 0$; call the probability of this even $\epsilon_{\neq}$.	

As in the proof of Theorem~\ref{thm:main}, we will need to obtain $|\psi_{0*x_\lambda}\rangle=|\psi_{x_\lambda}\rangle$. Unlike in the proof of Theorem~\ref{thm:main}, we can no longer use $\Es$ to compute this state. The reason is that $\Es$ in the new Q-mKGEA assumption has many possible outputs for the same $y$, so we cannot rely on the Gentle Measurement Lemma to argue that it computes $g_2$ without disturbing the state $|\psi_{g_2*x_\lambda}\rangle$. Instead, we will simply assume that $|\psi_{x_\lambda}\rangle$ is provided to us non-uniformly.

We now describe an algorithm $\Cs_=$ which breaks DLog/1-minCDH (Assumption~\ref{def:dlogminimalcdh}) with probability $\epsilon_=$. $\Cs_=$ works as follows:
\begin{itemize}
	\item It non-uniformly obtains $|\psi_{x_\lambda}\rangle$.
	\item It makes its query to the DLog/1-minCDH challenger, setting $\Ms_1$ as the query register. This query simulates the operation ${\sf Map}(g,\cdot)$, where $g$ is the group element chosen by the challenger. Thus, at the end of the query, $\Cs$ has $|\psi_{g*x_\lambda}\rangle$.
	\item Now upon receiving $g*x_\lambda$ from the challenger, run $E(g*x_\lambda,|\psi_{g*x_\lambda}\rangle)$. By the guarantees of $\Es$, the output will be $(g,0)$ with probability $\epsilon_=$. Output $g$.
\end{itemize}
Thus we see that, unless $\epsilon_=$ is negligible, $\Cs_=$ breaks the DLog/1-minCDH assumption. 

On the other hand, we can describe an algorithm $\Cs_{\neq}$ which breaks DLog/oblivious (Assumption~\ref{def:dlogobl}) with probability $\epsilon_{\neq}$. $\Cs_{\neq}$ works as follows:
\begin{itemize}
	\item It non-uniformly obtains $|\psi_{x_\lambda}\rangle$.
	\item It chooses a random $g\in\G_\lambda$, and computes $|\psi_{g*x_\lambda}\rangle$ from $|\psi_{x_\lambda}\rangle$.
	\item Now it runs $E(g*x_\lambda,|\psi_{g*x_\lambda}\rangle)$. By the guarantees of $\Es$, the output will be $(g',r),r\neq 0$ such that $g*x_\lambda= g'*S(1^\lambda;r)$ with probability $\epsilon_{\neq}$. Output $(g-g',r)$. Observe that $(g-g')*x_\lambda=S(1^\lambda;r)$.
\end{itemize}
Thus, $\Cs_{\neq}$ breaks the DLog/oblivious assumption with probability $\epsilon_{\neq}$, meaning $\epsilon_{\neq}$ is negligible. But this contradicts the fact that $\epsilon_=+\epsilon_{\neq} \geq1-\negl$. This completes the security proof.\end{proof}
\fi

