\section{Introduction}\label{sec:intro}

Quantum money, first envisioned by Wiesner~\cite{Wiesner83}, is a system of money where banknotes are quantum states. By the no-cloning theorem, such banknotes cannot be copied, leading to un-counterfeitable currency. A critical goal for quantum money, identified by~\cite{CCC:Aaronson09}, is \emph{public verification}, allowing anyone to verify while only the mint can create new banknotes. Such public key quantum money is an important central object in the study of quantum protocols, but unfortunately convincing constructions have remained elusive. See Section~\ref{sec:related} for a more thorough discussion of prior work in the area.

\paragraph{This Work.} We construct public key quantum money from abelian group actions, which can be instantiated by suitable isogenies over ordinary elliptic curves. Group actions, and the isogenies they abstract, are one of the leading contenders for post-quantum secure cryptosystems. Our construction could plausibly even be quantum lightning, a strengthening of quantum money with additional applications. Our construction is arguably the first time group actions have been used to solve a classically-impossible cryptographic task that could not already be solved using other standard tools like LWE. Our construction is sketched in Section~\ref{sec:constroverview} below, and given in detail in Section~\ref{sec:constr}.

While our main construction can be instantiated on a clean abelian group action --- often referred to as an ``effective group action'' (EGA) --- many isogeny-based group actions diverge from this convenient abstraction. We therefore provide an alternative candidate scheme which can be instantiated on so-called ``restricted effective group actions'' (REGAs); see Section~\ref{sec:alternate} for details. We prove the quantum lightning security of our protocols in the generic group action model --- a black box model for group actions --- assuming a new but natural strengthening of the discrete log assumption on group actions. Note that generic group actions cannot be used to give unconditional quantum hardness results, so some additional computational assumption is necessary. In order to prove our result, we develop a new toolkit for quantum generic group action proofs; see Section~\ref{sec:ggam}. We believe ours is the first proof of security in the quantum generic group action model. 

Along the way, we explore knowledge assumptions and algebraic group actions in the quantum setting, finding significant limitations of these assumptions/models compared to generic group actions. Specifically, unlike the classical setting where knowledge assumptions typically hold unconditionally against generic attacks, we explain why such statements likely do not hold quantumly. In the specific case of group actions, we indeed show an efficient generic attack on an analog of the ``knowledge of exponent'' assumption. This potentially casts doubt on quantum knowledge assumptions in general. We do give a more complex definition that avoids our attack, but it is unclear if the assumption is sound and more analysis is needed. For completeness, we give an alternative proof of security for our construction under this new knowledge assumption, which avoids generic group actions. 

We also discuss an algebraic model for group actions, which can be seen as a variant of the knowledge of exponent assumption. Unlike the classical setting where algebraic models live ``between'' the fully generic and standard models, we find that the algebraic group action model is likely incomparable to the generic group action model, and security proofs in the model are potentially problematic. As these issues do not appear for generic group actions, we therefore propose that generic group actions are the preferred quantum idealized model for analyzing cryptosystems, instead of the algebraic group action model as argued for in~\cite{PKC:DHKKLR23}. See Section~\ref{sec:knowledge} for details.

We conclude in Section~\ref{sec:discuss} with a discussion of possible generalizations. In particular, we propose the notion of a \emph{quantum} group action where the set elements are quantum states instead of bit strings. We discuss how instantiating our scheme on quantum group actions is closely related to failed approaches for building quantum money from LWE, but different in key ways that seem to allow our scheme to remain secure while the related LWE approaches failed.

\subsection{Our Construction}\label{sec:constroverview}

\paragraph{Abelian Group Actions.} We will use additive group notation for abelian groups. An abelian group action consists of an abelian group $\G$ and a set $\Xs$, such that $\G$ ``acts'' on $\Xs$ through the efficiently computable binary relation $*:\G\times\Xs\rightarrow\Xs$ with the property that $g*(h*x)=(g+h)*x$ for all $g,h\in\G,x\in\Xs$. We will also assume a \emph{regular} group action, which means that for every $x\in\Xs$, the map $g\mapsto g*x$ is a bijection. 

The main group actions used in cryptography are those arising from isogenies over elliptic curves. For example, see~\cite{EPRINT:Couveignes06,EPRINT:RosSto06,AC:CLMPR18,AC:BeuKleVer19,PKC:DFKLMPW23}. %Here, $\Xs$ is a set of elliptic curves, and $\G$ is a class group, acting on $\Xs$ via isogenies. 
Group action cryptosystems rely at a minimum on the assumed hardness of discrete logarithms: given $x,y=g*x\in\Xs$, finding $g$. In other words, while the map $g\mapsto g^x$ is efficiently computable and has an inverse, the inverse is not efficiently computable. For isogeny-based actions, this corresponds to the hard problem of computing isogenies between elliptic curves. Other hard problems on group actions are possible to consider, such as analogs of computational/decisional Diffie-Hellman, and more.



\paragraph{The QFT.} Our quantum money scheme will utilize the quantum Fourier transform (QFT) over general abelian groups. This is a quantum procedure that maps
\[|g\rangle\mapsto\frac{1}{\sqrt{|\G|}}\sum_{h\in\G}\chi(g,h)|h\rangle\enspace .\]
Here, $\chi$ is some potentially complex phase term. In the case of $\G$ being the additive group $\Z_N$, $\chi(g,h)$ is defined as $e^{i2\pi gh/N}$, with a slightly more complicated definition for non-cyclic groups\footnote{Remember that the group operation is $+$, so $gh$ in the exponent is not the group operation, but instead multiplication in the ring $\Z_N$.}. The main property we need from $\chi$ (besides making the QFT unitary) is that it is \emph{bilinear}, in the sense that $\chi(g,h_1+ h_2)=\chi(g,h_1)\cdot\chi(g,h_2)$. It is also symmetric: $\chi(g,h)=\chi(h,g)$.


\paragraph{Our Quantum Money Scheme.} Our quantum money scheme is as follows; see Section~\ref{sec:constr} for additional details.

\begin{itemize}
	\item $\gen$: initialize a register in the state $\frac{1}{\sqrt{|\G|}}\sum_{g\in\G}|g\rangle$, which can be computed by applying the QFT to $|0\rangle$. Let $x\in\Xs$ be arbitrary. Then by computing the group action in superposition, compute $\frac{1}{\sqrt{|\G|}}\sum_{h\in\G}|g\rangle|g*x\rangle$. Next, apply the QFT over $\G$ to the first register. The result is:
	\[\frac{1}{|\G|}\sum_{g,h\in\G} \chi(g,h)|h\rangle|g*x\rangle=\frac{1}{\sqrt{|\G|}}\sum_h |h\rangle|\G^h*x\rangle\]
	Here, $|\G^h*x\rangle$ is the state $\frac{1}{\sqrt{|\G|}}\sum_{g\in\G}\chi(g,h)|g*x\rangle$. Note that $|\G^h*x\rangle$ is, up to an overall phase, independent of $x$. 
	
	Now measure $h$, in which case the second register collapses to $|\G^h*x\rangle$. Output $h$ as the serial number, and $|\G^h*x\rangle$ as the money state.
	\item To verify a banknote $\$$, we do the following\footnote{In an initial version of this work, we had a more complicated verification. The simplified version here was pointed out to us by Jake Doliskani.}: Initialize a new register in the state $|\phi\rangle:=\frac{1}{\sqrt{|\G|}}\sum_{u\in\G}|u\rangle$. Then apply the map $(u,y)\mapsto(u,(-u)*y)$ to the joint system $|\phi\rangle\times \$$\enspace\footnote{Note that we used the ``minimal'' oracle here for the group action computation, having $(-u)*y$ replace $y$, instead of being written to a response register as in the standard quantum oracle. However, since the computation $y\mapsto (-u)*y$ is efficiently reversible given $u$ (by $y\mapsto u*y$), we can easily implement the minimal oracle efficiently by first computing $|(-u)*y\rangle$, then uncomputing $|y\rangle$ using the efficient inverse, and finally swapping in $|(-u)*y\rangle$.}. In the case where $\$$ is the honest banknote $|\G^h*x\rangle$, the result is
	\begin{align*}\frac{1}{|\G|}\sum_{u\in\G}|u\rangle \sum_{g\in\G}\chi(g,h)|(g-u)*x\rangle&=\frac{1}{|\G|}\sum_{u\in\G}|u\rangle \sum_{g'\in\G}\chi(g'+u,h)|g'*x\rangle\\&=\frac{1}{|\G|}\sum_{u\in\G}\chi(u,h)|u\rangle\sum_{g\in\G}\chi(g',h)|g'*x\rangle\\
	&=\left(\frac{1}{\sqrt{|\G|}}\sum_{u\in\G}\chi(u,h)|u\rangle\right)|\G^h*x\rangle\end{align*}
	where we used the substitution $g'=g-u$. Thus we see that this process preserves the honest banknte state $|\G^h*x\rangle$. Moreover, if we apply the inverse QFT to the first register, the result for honest banknotes is $|h\rangle$, and for any state orthogonal to the honest banknote, the result of the inverse QFT will be something orthogonal to $|h\rangle$. Thus by measuring this register and checking if the result is $h$, we can distinguish the honest banknote state from any other state.
\end{itemize}

\paragraph{An instantiation using REGAs.} In some isogeny-based group actions such as CSIDH~\cite{AC:CLMPR18}, the operation $*$ is only efficiently computable for a very small set $S\subseteq\G$ of group elements. Such group actions are called ``restricted effective group actions'' (REGAs)~\cite{AC:ADMP20}. Above, however, we see that we need to compute the group action on all possible elements in $\G$, both for minting and for verification. We therefore give a variant of the construction above which only uses the ability to compute $*$ for elements in $S$. We show that we are still able to sample $|\G^h*x\rangle$, but now the serial number has the form $\Am^T h+\ev\bmod N$ for a known matrix $\Am$ and a ``small'' $\ev\in\Z^n$\enspace\footnote{Here, we are interpreting $h$ a vector in $\Z_N^n$ for some $n,N$, which is possible since $\G$ is abelian.}. Under plausible assumptions, the serial number actually hides $h$\enspace\footnote{This is the search Learning with Errors (search LWE) problem~\cite{STOC:Regev05} which is widely believed to be hard for \emph{random} $\Am$. In our case, $\Am$ is a fixed matrix that depends on the group action, and LWE may or may not be hard for this $\Am$. However, if LWE is easy for this $\Am$, then we in fact have a plain group action. Indeed, a variant of Regev's quantum reduction between LWE and Short Integer Solution (SIS)~\cite{STOC:Regev05}, outlined by~\cite{EC:Zhandry19b}, shows that if LWE can be solved relative to $\Am$, then SIS can be solved for $\Am$ as well. It is straightforward to adapt this reduction to solve the Inhomogenous SIS (ISIS) problem, which then allows for computing the group action for all of $\G$. In this case we would have a clean group action and would not need this alternate construction.}. We nevertheless show that we can use such a noisy serial number for verification. For details, see Section~\ref{sec:alternate}. The security of our alternate scheme is essentially equivalent to the main scheme.



\subsection{The security of our scheme}\label{sec:introsec}

We do not know how to base the security of our schemes on any standard assumptions on isogenies. However, we are able to prove the security of our scheme in a black box model for group actions called the generic group action model (GGAM), an analog of the generic group model~\cite{EC:Shoup97,IMA:Maurer05} adapted to group actions. Generic models for group actions have been considered previously~\cite{AC:MonZha22,EC:BonGuaZha23,EPRINT:OrsZan23,PKC:DHKKLR23}. While the model is motivated by post-quantum security, to the best of our knowledge ours is the first time the model has been used to actually prove security against quantum attacks.

The challenge with the quantum GGAM is that the query complexity of computing discrete logarithms is actually polynomial~\cite{EttHoy00}. This means we cannot rely on query complexity alone to justify hardness, and must additionally make computational assumptions. This is in contrast to the classical setting, where the generic group (action) model allows for unconditional proofs of security by analyzing query complexity alone. In fact, most if not all generic group model proofs from the classical setting are unconditional query complexity proofs. This means that proofs in the quantum GGAM will look very different than classical proofs in the GGM/GGAM; in particular, proofs will still require a reduction from an underlying hard problem. At the same time, in order to take advantage of the generic oracle setting, it would seem that quantum query complexity arguments are still needed. But a priori, it may not be obvious how to leverage query complexity in any useful way, given the preceding discussion.

\paragraph{Our Framework.} In Section~\ref{sec:ggam}, we develop a new framework to help in the task of proving quantum hardness results relative to generic group actions. To illustrate our ideas, we consider the following warm-up task. An important feature in some isogeny-based group actions are twists, which allow for computing ``negations'': computing $(-g)*x$ from $g*x$. An interesting question is whether this additional structure makes computing discrete logarithms easier. Here, we show that for generic group actions, such negations are unlikely to make discrete logarithms any easier than in group actions without negations. Concretely, we will show that discrete logarithms are generically hard, assuming a plausible computational assumption on some group action where such negation queries are \emph{not} permitted. 

Suppose toward contradiction that there was a generic adversary which could utilize negation queries to solve discrete logarithms. Let $(*,\G,\Xs)$ be a plain group action where negation queries are not allowed. We will define a new group action $(\star,\G,\Xs')$ as follows. First sample a random injection $\Pi:\Xs^2\rightarrow\{0,1\}^m$ whose inputs are \emph{pairs} of set elements. Then define $\Xs'$ as the image under $\Pi$ of pairs of the form $(g*x,(-g)*x)$. $\star$ acts in the natural way: $g\star \Pi(y,z)=\Pi(g*y,(-g)*z)$.

Our reduction will sample a $\Pi$\enspace\footnote{A random injection is exponentially large and cannot be sampled efficiently. Instead, the reduction will actually efficiently simulate a random injection $\Pi$ using known techniques. For the purposes of our discussion here, we can ignore this issue.} and run the generic adversary on the new group action, using its knowledge of $\Pi$ and its inverse to implement the action $\star$. Notice now that our reduction also has the ability to compute negations: given $\Pi(y,z)$ where $y=g*x$ and $z=(-g)*x$, the negation of $\Pi(y,z)$ is exactly the element $\Pi(z,y)$ obtained by swapping $y$ and $z$. Thus, our reduction is able to simulate the negation queries, even though the underlying group action does not support efficient negations. This is our main idea, though there are a couple lingering issues to sort out:
\begin{itemize}
	\item The reduction cannot perfectly simulate $(\star,\G,\Xs')$. The issue is that there are elements $\Pi(y,z)$ where $y,z$ do not have the form $y=g*x,z=(-g)*x$ for some $g$. In the group action $(\star,\G,\Xs')$, these elements will be identified as invalid set elements. On the other hand, while our reduction can carry out the correct computation on $y,z$ of the correct form, it will be unable to distinguish such $y,z$ from ones of the incorrect form, and will act on these elements even though they are incorrect. As such, there will be elements that are not in $\Xs'$ that the reduction will nevertheless falsely identify as valid set elements. We resolve this problem by choosing the images of $\Pi$ to be somewhat sparse, by setting the output length $m$ sufficiently large. Our reduction only provides the adversary elements corresponding to valid $y,z$, and we can show, roughly, that the adversary has a negligible chance of computing elements in the image of $\Pi$ that correspond to invalid $y,z$. This follows from standard query complexity arguments. Thus, we are able to simulate with negligible error the correct group action $(\star,\G,\Xs')$.
	\item We have not yet specified what problem the reduction actually solves. The problem we would like to solve is the plain discrete logarithm on $(*,\G,\Xs)$, where the reduction is given $g*x$, and must compute $g$. However, it is unclear what challenge the reduction should give to the adversary. The natural approach is to try to give the adversary $\Pi(g*x,(-g)*x)$, which is just the discrete log instance relative to $(\star,\G,\Xs')$ with the same solution $g$; the reduction can then simply output whatever the adversary outputs. However, this requires the reduction to know $(-g)*x$, which is presumably hard to compute given just $g*x$ (remember that negation queries are not allowed on $(*,\G,\Xs)$). Our solution is to simply use a slight strengthening of discrete logarithms, where the adversary is given $(g*x,(-g)*x)$ and must compute $g$. Under the assumed hardness of this strengthened discrete log problem (again, in ordinary group actions where negations are presumed hard), we can complete the reduction and prove the generic hardness of discrete logarithms in the presence of negation queries.
\end{itemize}

\paragraph{The security of our money scheme.} We now turn to using our framework to prove the security of our quantum money scheme in the GGAM. Inspired by our negation example above, we will simulate a generic group action $(\star,\G,\Xs')$ using an injection $\Pi$ applied to a vector of set elements. Our goal will be to use two banknotes with the same serial number relative to $(\star,\G,\Xs')$ in order to break some distinguishing problem relative to $(*,\G,\Xs)$. Any quantum money adversary yields such a pair of banknotes, and so if the distinguishing problem is hard, then there can be no such efficient quantum money adversary. This argument in fact shows the scheme attains the stronger notion of quantum lightning~\cite{EC:Zhandry19b}, which has additional applications.

Concretely, our starting assumption gives the adversary $y=u*x$ for a random $u$, and then allows the adversary a single quantum query to $z\mapsto v*z$ for an unknown $v$, where either $v$ is random or $v=2u$. The adversary then has to tell whether $v=2u$ or not. It is straightforward to prove this assumption is true in the classical GGAM. In fact, it is a quantum analog of the classical group-based problem of distinguishing $g^a,g^b$ from $g^a,g^{a^2}$ for a group generator $g$, a widely used Diffie-Hellman-like assumption. Under this analogy, $g$ plays the role of $x$, $a$ plays the role of $u$, and $b$ plays the role of $v$. The main difference from the classical assumption (besides being over group actions instead of groups) is that, instead of receiving $g^b$ or $g^{a^2}$, the adversary receives $h^b$ or $h^{a^2}$ for an adversarially chosen $h$, and we allow the adversary's $h$ to be in superposition.

Our idea is to have $\Xs'$ be elements of the form $\Pi(g*x,g*y)$ where $y=u*x$ is the challenge given by the assumption. Let $X=\Pi(x,y)\in\Xs'$. Now consider the output of a successful adversary, which is two copies of the banknote $|\G^h\star X\rangle$ relative to $(\star,\G,\Xs')$ for some serial number $h$. Now consider applying the following process to, say, the first copy: map any element $\Pi(z_1,z_2)$ in the range of $\Pi$ to $\Pi(z_2,v*z_1)$, where we compute $v*z_1$ from $z_1$ using the challenge oracle. We then observe that if $v=2u$, this process preserves the banknote:
\begin{align*}
	|\G^h\star X'\rangle&=\frac{1}{\sqrt{|\G|}}\sum_{g\in\G}\chi(g,h)|g\star \Pi(x,y)\rangle=\frac{1}{\sqrt{|\G|}}\sum_{g\in\G}\chi(g,h)|\Pi(g*x,g*y)\rangle\\
	&\mapsto\frac{1}{\sqrt{|\G|}}\sum_{g\in\G}\chi(g,h)|\Pi(g*y,(g+2u)*x)\rangle\\
	&=\frac{1}{\sqrt{|\G|}}\sum_{g\in\G}\chi(g,h)|\Pi((g+u)*x,(g+2u)*x)\rangle\\
	&=\chi(-u,h)\frac{1}{\sqrt{|\G|}}\sum_{g'\in\G}\chi(g',h)|\Pi(g'*x,g'*y)\rangle=\chi(-u,h)|\G^h\star X'\rangle
\end{align*}
Above, we used the substitution $g'=g+u$.

On the other hand, if $v\neq 2u$, then the transformation will produce a state whose support is not even on $\Xs'$. In particular, the transformed state would be orthogonal to the original state. So our reduction will apply the above transformation to one copy of $|\G^h\star X\rangle$, leaving the other as is. Then it will perform the SWAP test on the two states. If $v=2u$, the states will be identical and the SWAP test will accept. If $v\neq 2u$, the states will be orthogonal, and the swap test will accept only with probability $1/2$. Thus, we achieve a distinguishing advantage between the two cases, contradicting the assumption.

\medskip

We believe our proof gives convincing evidence that our scheme should be secure on a suitable group action, perhaps even those based on isogenies over elliptic curves. However, our underlying assumption is new, and needs further cryptanalysis. One limitation of our assumption is that it is interactive, requiring a (quantum) oracle query to the challenger. One may hope instead to use a non-interactive assumption. We do not know how to make non-interactive assumptions work, in general. In particular, if we do not have an oracle that can transform the input for us, it seems like we are limited to strategies that only permute the inputs to $\Pi$, like in our negation-query example. But since the scheme has to be efficient, the inputs to $\Pi$ can only consist of polynomial-length vectors of set elements. Any permutation on a polynomial-length set must have smooth order. On the other hand, the only permutations on $\Xs'$ which preserve $|\G^h\star X\rangle$ seem to have order that divides $|\G|$. Thus, if, say, the order of $\G$ were a large prime, it does not seem that permuting the inputs to $\Pi$ alone will be able to preserve $|\G^h\star X\rangle$.




\subsection{On Knowledge Assumptions and Algebraic Group Actions} 

In Section~\ref{sec:knowledge}, we show a different approach to justifying the security of our scheme, by adapting certain knowledge assumptions~\cite{EC:LiuMonZha23} to the setting of group actions. Despite some high-level similarities to~\cite{EC:LiuMonZha23}, the underlying details are somewhat different. The advantage of this route is that it gives a standard-model security proof (albeit, using a non-standard knowledge definition) rather than a generic model proof.

However, we find significant issues with using knowledge assumptions quantumly, that appear not to have been observed before. In particular, the straightforward way to adapt the knowledge assumptions of~\cite{EC:LiuMonZha23} to group actions actually results in \emph{false} assumptions, as we demonstrate. Interestingly, our attack on the assumption is entirely generic. This is quite surprising, as in the classical setting, knowledge assumptions generally trivially hold against generic attacks.

Concretely, we show how to construct a superposition over $\Xs$ where the underlying discrete logarithms are hidden, even to the algorithm creating the superposition. To accomplish this, we observe that any set element $x$ can be seen as a superposition over all possible banknotes $|\G^h*x\rangle$; the superposition is uniform up to individual phases. Then we show a procedure to compute, given $|\G^h*x\rangle$, the serial number $h$. This allows us to apply individual phases to the various banknotes in the superposition. Certain phases will simply map $x$ to another set element $y$. But other phases will map $x$ to a uniform superposition (up to phases) over $\Xs$. Call this state $|\psi\rangle$.

Any meaningful knowledge assumption, and in particular the result of adapting~\cite{EC:LiuMonZha23} to group actions, would imply that if we were to measure $|\psi\rangle$ to get a set element $y$, then we must also ``know'' $g$ such that $y=g*x$. However, measuring $|\psi\rangle$ simply gives a uniform set element, importantly without any side information about $y$. As such, under the discrete log assumption, computing such a $g$ is hard.

We resolve this particular problem by re-framing knowledge assumptions as follows: instead of saying that any algorithm $A$ which produces a set element $y$ must know $g$ such that $y=g*x$, we say that for any such $A$ solving some task $T$, there is another algorithm $B$ that also solves $T$ such that $B$ knows $g$, even if $A$ would not. Thus, even if the original $A$ is constructed in such a way that it does not know $g$, at least $B$ does, and we can apply any security arguments to $B$ instead of $A$. We demonstrate that this assumption, together with an appropriate generalization of the discrete log assumption, are enough to prove the security of our scheme. However, we are somewhat skeptical of our new knowledge assumption, and it certainly needs more cryptanalysis.

\paragraph{Algebraic Group Actions.} The Algebraic Group Model (AGM)~\cite{C:FucKilLos18} is an important classical model for studying group-based cryptosystems. It is considered a refinement of the generic group model, meaning that a proof in the model is ``at least as'' convincing as a proof in the generic group model\footnote{There are some caveats to this classical claim; see~\cite{C:Zhandry22b} for discussion.}, potentially even more convincing. A couple of recent works~\cite{PKC:DHKKLR23,EPRINT:OrsZan23} have considered the group action analog, the Algebraic Group Action Model (AGAM). Here, any time an adversary outputs a set element $y$, it must ``explain'' $y$ in terms of one of its input set elements $x_1,\dots,x_n$ by providing a group element $g$ such that $y=g*x_i$.

The AGM can be seen as an idealized model version of the knowledge of exponent assumption, and likewise the AGAM can be seen as an idealized model version of an appropriate knowledge assumption on group actions. After all, a knowledge assumption would say that any time the adversary outputs a $y$, it must ``know'' how it derived $y$ from its inputs. The AGM/AGAM simply require the adversary to actually output this knowledge.

In Section~\ref{sec:knowledge}, we explore the AGAM in the presence of quantum attackers. We do not prove any formal results, but discuss why, unfortunately, the quantum AGAM appears problematic. For starters, given our attack on quantum knowledge assumptions, we are skeptical about the soundness of the quantum AGAM. In particular, our attack indicates that it is unlikely that the AGAM is a refinement of the generic group action model; rather they are likely incomparable.

Another problem we observe with the AGAM is that it requires the adversary to both solve some task, and also produce some extra information, namely the explanation $g$ of any output element $y$. Classically, if the adversary is able to both solve the task and produce this extra information (which would follow from an appropriate knowledge assumption), then the adversary can do both simultaneously, as required by the AGM/AGAM. However, quantumly, even if we believe the adversary can separately solve the task \emph{or} produce the extra information (provided we believe the knowledge assumption), it may be impossible to do both simulteneously, as required by the AGAM.

This issue manifests in the following way: suppose the output is actually a superposition. Then the information $g$ will be entangled with the superposition, meaning the AGAM adversary's output will actually be a different state than if it did not output $g$. For example, if an AGAM adversary had to output a banknote $|\G^h*x\rangle$ (say, as part of the quantum money/lightning experiment), then if it also ``explained'' the banknote by outputting a group element $g$, the entanglement with $g$ would actually cause the banknote state to fail verification. It therefore unclear how to interpret such an adversary. Does it actually break the scheme, even if it does not pass verification? In Section~\ref{sec:knowledge}, we go into more details about this issue as well as pointing out several other issues with the AGAM.

We note that these issues are not present in the generic group action model. Thus, despite classically being a ``worse'' model than the algebraic model, we propose for the quantum setting that the generic group action model is actually \emph{preferred} to the AGAM.














\subsection{Further Discussion}

In Section~\ref{sec:discuss}, we generalize group actions to \emph{quantum} group actions, which replace classical set elements with quantum states, but otherwise behave mostly the same as standard group actions. We give a simple quantum group action based on the Learning with Errors (LWE) problem~\cite{STOC:Regev05}, where we can actually prove that the discrete log problem is hard under LWE. Despite this promising result, we expect that the LWE-based quantum group action will be of limited use. In particular, if we instantiate our quantum money construction over this group, the construction is \emph{insecure}. The reason is that, in this group action, it is impossible to recognize the quantum states of the set. Our security proof crucially relies on such recognition in order to characterize states accepted by the verifier. Moreover, without recognition, there is an attack which fools the verifier with dishonest --- and importantly, clonable --- banknotes that are different from the honest ones, breaking security.

Interestingly, we explain that this failed instantiation is actually \emph{equivalent} to a folklore approach toward building quantum money from lattices, which has been more-or-less shown impossible to make secure~\cite{C:LiuZha19,EC:LiuMonZha23}. The \emph{only} missing piece in the folklore approach has been how to efficiently verify honest banknotes. Under our equivalence, this missing piece exactly maps to the problem of recognizing set elements in our quantum group action. For details, see Section~\ref{sec:discuss}. We believe this adds to the confidence of our proposal, since in group actions based on isogenies it is possible to recognize set elements, presumably without otherwise compromising hardness.




\subsection{Related Work}\label{sec:related}

\paragraph{Public key quantum money.} In Wiesner's original scheme, the mint is required to verify banknotes, meaning the mint must be involved in any transaction. The involvement of the mint also leads to potential attacks~\cite{Lutomirski10}. Some partial solutions have been proposed, e.g.~\cite{EPRINT:BehSat20,AC:RobZha21}. The dream solution, however, is known as \emph{public key} quantum money~\cite{CCC:Aaronson09}. Here, anyone can verify the banknote, while only the mint can create them.

Unlike Wiesner's scheme which is well-understood, secure public key quantum money has remained elusive. While there have been many proposals for public key quantum money~\cite{CCC:Aaronson09,STOC:AarChr12,ITCS:FGHLS12,Kane18,EC:Zhandry19b,EPRINT:KanShaSil21,KLS22,EC:LiuMonZha23}, they mostly either (1) have been subsequently broken (e.g.~\cite{CCC:Aaronson09,STOC:AarChr12,EC:Zhandry19b,KLS22} which were broken by~\cite{ITCS:LAFGKH10,PDFHP19,EC:Roberts21,EC:LiuMonZha23}), or (2) rely on new cryptographic building blocks that have received little attention from the cryptographic community (e.g.~\cite{ITCS:FGHLS12,Kane18,EPRINT:KanShaSil21} from problems on knots or quaternion algebras). The two exceptions are:
\begin{itemize}
	\item Building on a suggestion of~\cite{BenSat16}, \cite{EC:Zhandry19b} proved that quantum money can be built from post-quantum indistinguishability obfuscation (iO). iO has received considerable attention and even has a convincing \emph{pre-quantum} instantiation~\cite{STOC:JaiLinSah21}. Yet the post-quantum study of iO is much less thorough. While some post-quantum proposals have been made~\cite{TCC:GenGorHal15,TCC:BGMZ18,EPRINT:BDGM20b,EC:WeeWic21}, their post-quantum hardness is not well-understood.
	\item \cite{EC:LiuMonZha23} construct quantum money from isogenies over super-singular elliptic curves. While isogenies have garnered significant attention from cryptographers, there is a crucial missing piece to their proposal: generating uniform superpositions over super-singular curves, which is currently unknown how to do. This is closely related to the major open question of obliviously sampling super-singular elliptic curves.
\end{itemize}
In light of the above, the existence of public key quantum money is largely considered open.



\paragraph{Cryptography from group actions and isogenies.} Isogenies were first proposed for use in post-quantum cryptography by Couveignes~\cite{EPRINT:Couveignes06} and Rostovtsev and Stolbunov~\cite{EPRINT:RosSto06}. Isogenies give a Diffie-Hellman-like structure, but importantly are immune to Shor's algorithm for discrete logarithms~\cite{FOCS:Shor94} due to a more restricted structure. This restricted structure, while helping preserve security against quantum attacks, also makes the design of cryptosystems based on them more complex. Thus, significant effort has gone into building secure classical cryptosystems from isogenies and understanding their post-quantum security (e.g.~\cite{CJS14,DJP14,AC:CLMPR18,AC:BeuKleVer19,CD20,PKC:DeFMey20,EC:Peikert20,EC:BonSch20,AC:ADMP20,TCC:AlaMalRah22,AC:MonZha22,EPRINT:MaiMar22,EC:CasDec23,EC:BonGuaZha23,EC:Robert23}).

Certain isogenies such as the original proposals of~\cite{EPRINT:Couveignes06,EPRINT:RosSto06} as well as CSIDH and its variants~\cite{AC:CLMPR18,PKC:DFKLMPW23} can be abstracted as abelian group actions. However, many other isogenies (such as SIDH~\cite{DJP14}  and OSIDH~\cite{CD20}) cannot be abstracted as abelian group actions. Even among abelian group actions, we must distinguish between ``effective group actions'' (EGAs) and \emph{restricted} EGAs (REGAs). The former satisfies the notion of a clean group action, whereas in the latter, the group action can only be efficiently computed for a certain small set of group elements. CSIDH could plausibly be a EGA at certain concrete security parameters, though asymptotically it only achieves quasi-polynomial security\footnote{With the state-of-the-art, evaluating CSIDH as an EGA would require time approximately $2^{\sqrt[3]{n}}$ on a quantum computer, while the best quantum attack is time $2^{\sqrt{n}}$. For a thorough discussion, see~\cite{Panny23}. By setting $n=\log^3(\lambda)$, one gets polynomial-time evaluation and the best attack taking time $\lambda^{\sqrt{\log(\lambda)}}$.}. Our alternate construction also works on REGAs, which can plausibly be instantiated even asymptotically by CSIDH using a quantum computer\footnote{In order for CSIDH to be a REGA, one needs to compute the structure of the group. While this is hard classically, it is easy with a quantum computer using Shor's algorithm~\cite{FOCS:Shor94}. Since we always assume a quantum computer in this work, we can therefore treat CSIDH as a REGA.}.

While some non-isogeny abelian group actions have been proposed (e.g.~\cite{Stickel05}), currently all such examples have been broken (e.g.~\cite{Shpilrain08}). For this reason, (abelian) group actions are largely considered synonymous with isogenies, though this may change if more secure group actions are found.

The vast majority of the isogeny and group action literature has focused on post-quantum cryptography --- classical protocols that are immune to quantum attacks. To the best of our knowledge, only two prior works have used isogenies/group actions to build quantum protocols for tasks that are \emph{impossible} classically. The first is~\cite{TCC:AlaMalRah22}, who build a proof of quantumness~\cite{FOCS:BCMVV18}. We note that proofs of quantumness can also be achieved under several ``standard'' cryptographic tools, such as LWE~\cite{FOCS:BCMVV18} or under certain assumptions on hash functions~\cite{FOCS:YamZha22}. In contrast, no prior quantum money protocol could be based on similar standard building blocks. We also note that ~\cite{TCC:AlaMalRah22} currently has no known asymptotic instantiation with better-than-quasi-polynomial security, as it requires a clean group action (EGA). The second quantum protocol based on isogenies is that of~\cite{EC:LiuMonZha23}, who build quantum money from walkable invariants, and propose an instantiation using isogenies over super-singular elliptic curves. However, such isogenies cannot be described as abelian group actions, and even more importantly their proposal is incomplete, as discussed above. Thus, ours is arguably the first application of group actions or isogenies to obtain classically impossible tasks that could not already be achieved under standard tools.



\paragraph{Relation to~\cite{EC:LiuMonZha23}.} Aside from using isogenies, our construction has some conceptual similarities to~\cite{EC:LiuMonZha23}, though also crucial differences that allow us to specify a complete protocol, and our idealized-model analysis is completely new. Here, we give a brief overview of the similarities and differences.

The walkable invariant framework of~\cite{EC:LiuMonZha23} is very general, but here we describe a special case of it that would apply to certain group actions, in order to illustrate the differences with our scheme. Consider a group action that is \emph{not} regular, so that the set $\Xs$ is partitioned into many distinct orbits. For $x,y$ in the same orbit there will exist a unique $g$ such that $y=g*x$, but for $x,y$ in different orbits, there will not exist any group element mapping between them. We will also assume the ability to generate a uniform superposition over $\Xs$. We finally assume an ``invariant'', a unique label for each orbit which can be efficiently computed from any element in the orbit.

The minting process generates the uniform superposition over $\Xs$, and then measures the invariant, which becomes the serial number. The state then collapses to a uniform superposition over a single orbit, which becomes the banknote. This superposition can then be verified as follows. First check that the banknote has support on the right orbit by re-computing the invariant. Then check that the state is in uniform superposition by checking that the state is preserved under action by random group elements; this is accomplished using an analog of the swap test.~\cite{EC:LiuMonZha23} prove the security of their scheme under the certain assumptions which, when mapped to the group action setting above, correspond to the discrete log assumption and a knowledge assumption very similar to ours.

Unfortunately, there are no known instantiations of suitable group actions for their scheme. One possibility is to use the set of ordinary elliptic curves as the set, the number of points on the curve as the invariant, and orbits being sets of curves with the same number of points. Isogenies between curves are then the action\footnote{It is not a proper group action since different orbits will be acted on by different groups.}, which do not change the number of points on the curve. The problem is that in general curves, it is not possible to efficiently compute the action, since the degree of the isogenies will be too high. The action \emph{can} be computed on smooth-degree isogenies, but these are rare and there is no known way to compute a uniform superposition over curves supporting smooth-degree isogenies. For reasons we will not get into here,~\cite{EC:LiuMonZha23} propose using instead supersingular curves with non-smooth order, but again these are rare and there is no known way to generate a uniform superposition over such curves.

We resolve the issues with instantiating~\cite{EC:LiuMonZha23}, without needing the ability to compute uniform superpositions over the set. Our key insight is that, if we can compute the group action efficiently (say because we are using isogenies of smooth degree), then this is enough to sample states that \emph{are} uniform over a given orbit, except for certain phase terms: namely the states $|\G^h*x\rangle$ for uniform $h$. Then, rather than the serial number indicating which orbit we are in (which is now useless since we are in a single orbit), the serial number is a description of the phase terms, namely $h$. 

\subsection*{Acknowledgments} We thank Hart Montgomery for many helpful discussions about isogenies. We thank Jake Doliskani for pointing out a simpler procedure for verifying banknotes and computing the serial number of banknotes.