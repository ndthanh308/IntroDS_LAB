\section*{Appendix A -- Proofs}
\label{appendix:a}

In this appendix, we report all the proofs of the results stated in the main body of the paper.


\begin{proof} (\emph{Lemma \ref{property}})
It follows from the following chain of inequalities

\begin{align*}
    \sum_{i=1}^b w_i x_i &= \sum_{i=1}^a w_i x_i + \sum_{i=a+1}^b w_i x_i\\
    &\geq \sum_{i=1}^a w_i x_i + w_a \sum_{i=a+1}^b  x_i\\
&= \sum_{i=1}^a w_i x_i + w_a  a- w_a \sum_{i=1}^a  x_i \\
&= \sum_{i=1}^a \left( w_i x_i + w_a -   w_a x_i\right)\\
&= \sum_{i=1}^a \left( (w_a - w_i)(1- x_i) +    w_i\right)\geq \sum_{i=1}^a   w_i.
\end{align*}
\end{proof}


\begin{proof}(\emph{Theorem \ref{lemma-equivelance}})
It is easy to see that the problems \eqref{lower-level-problem-primal_0} and \eqref{lower-level-problem-primal} are equivalent to the following Linear Programming problems
\begin{equation} \label{lower-level-problem-primal-equ}
\arraycolsep=1.4pt\def\arraystretch{1.25}
\begin{array}{lrl} 
\forall~j\in[m],~~& Y^*_{:,j}\in\underset{Y_{:,j} }{\rm argmin}&~\langle  W_{:,j}, Y_{:,j}\rangle,\\
&{\rm s.t.}~&~\sum_{i\in [n]}  Y_{i,j} = U_{i},\\
&\quad &0\le Y_{i,j} \leq Z_{i,j},
\end{array} 
\end{equation}
and the following Integer Linear Programming problems
\begin{equation} \label{lower-level-problem-relax-equ}
\arraycolsep=1.4pt\def\arraystretch{1.25}
\begin{array}{lrl} 
\forall~j\in[m],~~& \widehat{Y}_{:,j}\in\underset{Y_{:,j} }{\rm argmin}&~\langle  W_{:,j}, Y_{:,j}\rangle,\\
&{\rm s.t.}~&~\sum_{i\in [n]} Y_{i,j} = U_j,\\
&\quad &Y_{i,j} \leq Z_{i,j},\\
&\quad &Y_{:,j}\in \mathbb{B}_{1,m},
\end{array} 
\end{equation}
respectively. 
% 
Therefore, to show that $\langle  W, Y^*\rangle=\langle  W, \widehat Y\rangle$, we prove $\langle  W_{:,j}, Y^*_{:,j}\rangle=\langle  W_{:,j}, \widehat Y_{:,j}\rangle$ for every $j\in[m]$. Without loss of generality, we only consider the case $j=1$.
% 
Furthermore, we assume $Z_{i,1}=1$ for every $i\in [n]$, this can be done without loss of generality since otherwise, it suffices to restrict the sets of indices.

Since the feasible region of \eqref{lower-level-problem-primal-equ} is a superset of the feasible set of \eqref{lower-level-problem-relax-equ}, it follows $\langle  W_{:,1}, Y^*_{:,1}\rangle\geq\langle  W_{:,1}, \widehat Y_{:,1}\rangle$.
% 
To conclude the proof, we show the other inequality.
% 
If $\widehat{Y}_{:,1}\in  \B_{n,1}$, we conclude the proof, therefore, we assume $\widehat{Y}_{:,1}\notin  \B_{n,1}$. 
% 
Let us now define
\begin{equation}
    \label{S1-S2}
S: = \Big\{i\in[n]~ :~ 0< \widehat Y_{i,1} <1\Big\}
\end{equation}
and
\begin{equation}
    K: = \Big\{i\in[n]~ : ~ \widehat Y_{i,1} \in\{0,1\}\Big\}.
\end{equation}
% 
By assumption, $S\neq \emptyset$ and, by construction, we have
\begin{equation}
    \label{S1S2t}
 U_1=\sum_{i\in [n]}\widehat Y_{i,1}= \underset{i\in S}\sum \widehat Y_{i,1} +   \underset{i\in K}\sum  \widehat Y_{i,1} =: \underset{i\in S}\sum  \widehat Y_{i,1} +t.
\end{equation}
% 
Since $\widehat Y_{i,1} \in\{0,1\}, ~\forall i \in K$, we get that $t$ is an integer number. 
% 
Let us now define $b:=|S|$ and let $a=U_1-t>0$. From equation \eqref{S1S2t} and $ 0< \widehat Y_{1,j} <1$, we infer that $b>a$.
% 
Now, we increasingly reorder the elements in $\{W_{i,1}:i\in S\}$, so that 
\begin{equation} 
\label{cond-0}
{W}_{i_1,1} \leq {W}_{i_2,1} \leq\cdots\leq{W}_{i_a,1}\leq \cdots \leq  {W}_{i_b,1},
\end{equation} 
where $\{i_1,\cdots,i_b\}=S$. Finally, we define $\overline{Y}$ as 
\begin{eqnarray}
\label{YYYYYYY}
\overline Y_{i,1}=\left\{
\arraycolsep=1.4pt\def\arraystretch{1.5}
 \begin{array}{llllll} 
 \widehat Y_{i,1},&~~~~ \text{for}~ i\in K,\\
 1,&~~~~ i\in \{i_1,i_2,\cdots,i_a\},\\
 0,&~~~~ \text{otherwise} 
.\end{array}
\right.
\end{eqnarray}
A simple computation shows that

 \begin{eqnarray} 
\label{cond-1}&& \sum_ {i\in S} \overline Y_{i,1} = \sum_ {i = i_1} ^{i_a} 1 =a,\\ 
\label{cond__2}&&\sum_ {i\in S} {W}_{i,1} \overline Y_{i,1}  =\sum_ {i = i_1} ^{i_a} {W}_{i,1} \\
\nonumber &&\quad\quad\quad\quad\quad\quad\leq \sum_ {i = i_1} ^{i_b}   {W}_{i,1} \widehat Y_{i,1} \leq  \sum_ {j\in S}  {W}_{i,1} \widehat Y_{i,1},
\end{eqnarray} 

which, in conjunction with relationships \eqref{YYYYYYY}, \eqref{cond__2}, and \eqref{S1S2t}, allows us to conclude

\begin{align*}
    \sum_{i\in [n]} \overline{Y}_{i,1}&= \underset{i\in S}\sum \overline{Y}_{i,1} + \underset{i\in K}\sum\overline{Y}_{i,1}\\
    &= \underset{i\in S}\sum  \overline Y_{i,1} + \underset{i\in K}\sum  \widehat Y_{i,1}= a+t= U_1,
\end{align*}

% 
which means that $ \overline Y$ is feasible. Moreover,
% 

 \begin{eqnarray} \label{cond-5}
\langle {W}_{:,1}, \overline Y_{:,1} \rangle&=&   \sum_ {i\in S} {W}_{i,1} \overline Y_{i,1} +  \sum_ {i\in K} {W}_{i,1} \overline Y_{i,1}    \nonumber \\
&\leq& \sum_ {i\in S} {W}_{i,1} \widehat Y_{i,1} +  \sum_ {i\in K} {W}_{i,1} \widehat Y_{i,1} ~~~~~(\text{due to  \eqref{cond__2})}\nonumber\\
 &=& \langle {W}_{:,1}, \widehat Y_{:,1} \rangle,
\end{eqnarray} 
hence $\langle {W}_{1 :}, \overline Y_{1 :} \rangle = \langle {W}_{1 :}, \widehat Y_{1 :} \rangle$.
% 
The uniqueness result follows from the uniqueness of the non-decreasing ordering of the values $\{W_{i,1}:i\in S\}$.
% 
\end{proof}

\begin{proof}(\emph{Theorem \ref{prop:prob_simp}})
First of all, we notice that, since we are considering a minimization problem, imposing the constraint $\sum_{i\in [m]}Y_{i,j}=U_{j}$ for every $j\in [m]$ is equivalent to impose $\sum_{i\in [m]}Y_{i,j}\ge U_{j}$ for every $j\in [m]$.
% 
If $Y$ minimizes the effort of the reviewers component-wise, it also minimizes the total effort of the reviewers.
% 
Toward a contradiction, assume that $Y$ minimizes the total effort of the reviewers, but does not minimize the effort of a reviewer, namely $j$.
% 
From Theorem \ref{lemma-equivelance}, there exists a binary vector $\tilde{Y}_j$ that minimizes the effort of reviewer $j$.
% 
Let us now define the matrix $\bar{Y}$ as it follows: every column different from the $j$-th is equal to the respective column of $Y$, while the $j$-th column is equal to $\tilde{Y}_j$.
% 
It is easy to see that $\bar{Y}$ is still feasible and that the reviewers' effort of $\bar{Y}$ is lower than the effort of $Y$, which contradicts the hypothesis.
\end{proof}


\begin{proof}(\emph{Theorem \ref{prop:feasibility}})
Since the set of feasible triplets for the BP problem is discrete, if we show that it is also not empty, we infer that there exists an optimal solution, which concludes the proof.
% 
Since the ILP problem is feasible, let $X$ be a solution to Problem \ref{pr:classic}.
% 
Since we have that $\max_{j\in [m]}\;\phi_j + 2\max_{j\in [m]} \; U_j\le n$, we can find a $Z$ such that $X\le E-Z$.
% 
Notice that, if the editor proposes $Z$, the matching $X$ is a feasible allocation for the ULp regardless of what the solution to the LLp $Y$ is.
% 
Finally, from Theorem \ref{lemma-equivelance}, we have that there exists an optimal $Y$ that solves the LLp.
% 
Therefore $(Z,Y,X)$ is a feasible triplet for the BP problem, which concludes the proof.
\end{proof}






\begin{proof}(\emph{Proposition \ref{prop:dots}})
Let $(X,Y,Z)$ be a perfect solution, so that $X\le Y\le Z$ holds.
% 
Moreover, let us fix $\Phi:=(\phi_1,\dots,\phi_m)$.
% 
Since $Z$ has at least $U_j+\phi_j$ non-null entries in every row $j\in [m]$, and since $X_{:,j}\le Z_{:,j}$ we infer that the allocation $X$ is $\Phi$-burden-free, which proves the first half of the proposition.
% 

% 
We now focus the second half of the proof.
% 
Let $(X,Y,Z)$ be a perfect solution such that $X$ maximizes the quality of the matching.
% 
Since $X$ maximizes the quality, then $X$ is a solution of the ILP defined in Problem \ref{pr:classic}.
% 
Since we have already proven that $X$ is also $\Phi$-burden-free, this concludes the proof.
% 
To prove the other implication, let $X$ be an optimal and $\Phi$-burden-free solution to Problem \ref{pr:classic}. 
% 
Let $T_j$ be a set that contains $\phi_j$ papers that are worse, according to reviewer $j$, than the ones $j$ is allocated with by assignment $X$.
% 
Since the solution is $\Phi$-burden-free, the set $T_j$ exists for every reviewer $j\in [m]$.
% 
Finally, let us consider a feasible $Z$ such that $Z_{:,j}\ge X_{:,j}+T_j$.
% 
Regardless of the LLp solution $Y$, it holds $X\le Y$, which concludes the second part of the proof.
\end{proof}

\begin{proof}(\emph{Proposition \ref{prop:dict}})
% 
First, we prove that every solution to the classic ILP problem induces at least a solution to the BP problem, and then we show that also the reverse implication holds.
% 
Let us consider $X$ a solution to the classic ILP model in Problem \eqref{pr:classic} and let $Z$ be a binary matrix such that $X\le Z$ and $\sum_{i\in [n]}Z_{i,j}=U_j$ for every $j\in [m]$.
% 
Since every solution to Problem \eqref{pr:classic} satisfies $\sum_{i\in [n]}X_{i,j}\le U_j$ for every $j\in [m]$, such $Z$ always exists.
% 

% 
Let us now define the triplet $(X,Y,Z)$, where $Y=Z$.
% 
Since $X\le Z=Y$, we infer that $\langle X,Y \rangle =\langle X,X \rangle$, hence $AC(X,Y,Z)=1$.
% 
Moreover, since $Y=Z$ is the only feasible matrix of the LLp, it is also optimal for the LLp.
% 
To conclude, notice that $X$ maximizes, by construction, both the functional $\langle W_E, X\rangle$ and $\langle Y,X \rangle$, therefore it is optimal for the BP problem.
% 
Let us now consider a solution to \eqref{problem:bilevel}, namely $(X,Y,Z)$.
% 
If $X$ is not a solution to the ILP problem, we can construct a better solution to the BP problem by using the procedure described above, which contradicts the optimality of $(X,Y,Z)$.
% 
\end{proof}

% 

\begin{proof}(\emph{Theorem \ref{thm:feas_cond}})
    We prove the Theorem by showing that there exists at least one feasible matching for the ULp when we fix $Z=Z_g$ and $Y=Y_g$.
    % 
    To do so, we show that there exists a feasible matching $X$ such that $X\le E-Z_g$.
    % 

    % 
    Given the bipartite graph $G=([n]\times [m], E-Z_g)$, we build the auxiliary graph $G'$ as it follows.
    % 
    For every element $i\in [n]$, we generate $l_i$ copies of $i$, we denote this set of papers with $\mathfrak{P}'$.
    % 
    Similarly, for every $j\in [m]$, we generate $U_j$ copies of $j$, we denote this set of reviewers with $\mathfrak{R}'$.
    % 
    The edge set of the bipartite graph, namely $E'$, is as it follows,: $e':=(i',j')\in E'$ if and only if $e=(i,j)\in E$, where $i'$ is a copy of $i$ and $j'$ is a copy of $j$. 
    % 
    To conclude the proof, it suffice to show that there exists a perfect matching of $\mathfrak{P}'$ into $\mathfrak{R}'$.
    % 
    First notice that $\mathfrak{P}'$ contains $L=\sum_{i\in [n]}l_i$ elements.
    % 
    This follows from the Hall's theorem \cite{kierstead1983effective}, since, by hypothesis and construction, every paper in $\mathfrak{P}'$ is connected to at least $L$ elements of $\mathfrak{R}'$.
\end{proof}



\begin{proof}(\emph{Theorem \ref{them:heur_estimate}})
It is easy to see that $$\langle W_E+ Y_{Z_g},X_{Z_g}\rangle\le\langle W_E+ Y^*,X^*\rangle,$$ where $(X^*,Y^*,Z^*)$ is the optimal solution of the BP problem.
% 
By rearranging the terms of the latter inequality, we have
% 
\[
AC(X_g,Y_g,Z_g)-AC(X^*,Y^*,Z^*)\le \langle W_E, X^*\rangle - \langle W_E, X_g \rangle.
\]
Since $AC(X_g,Y_g,Z_g)=1$ and $AC(X^*,Y^*,Z^*)\le 1$, we infer that the left-hand side of the equation is positive, hence
\[
0\le\langle W_E, X^*\rangle - \langle W_E, X_g \rangle \quad \to \quad \langle W_E, X_g \rangle\le\langle W_E, X^*\rangle.
\]
% 
In particular, the quality obtained by the heuristic final assignment $X_{X_g,Y}$ gives a lower bound on the quality attained by the optimal solution of the BP problem. 
\end{proof}
