\documentclass[a4paper]{jpconf}
\usepackage{graphicx}
\usepackage[english]{babel}   
\usepackage[usenames]{color}   
\usepackage{colortbl} 
\usepackage[normalem]{ulem}  % 

\pagestyle{plain}

\begin{document}
\title{Formation of spiral dwarf galaxies: observational data and results of numerical simulation
\footnote{\color{blue}\underline{\bf Published}: Khrapov S.S., Khoperskov A.V., Zaitseva N.A., Zasov A.V., Titov A.V., Formation of spiral dwarf galaxies: observational data and results of numerical simulation // St. Petersburg State Polytechnical University Journal. Physics and Mathematics, 2023, v.16 (1.2), pp.395--402. DOI: https://doi.org/10.18721/JPM.161.260  https://physmath.spbstu.ru/en/article/2023.64.60/}}

\author{Sergey Khrapov$^1$, Alexander Khoperskov$^1$, Natalia Zaitseva$^3$, Anatoly~Zasov$^{2,3}$, Alexander Titov$^1$} 
\address{$^1$ Institute of Mathematics and Information Technology, Volgograd State University, Volgograd, Russia \\
         $^2$ Lomonosov Moscow State University, Faculty of Physics, Moscow, Russia \\
         $^3$ Lomonosov Moscow State University, Sternberg Astronomical Institute, Moscow, Russia
        }

\ead{khrapov@volsu.ru, khoperskov@volsu.ru}

\begin{abstract}
Recent studies show the possibility of the formation of fairly regular and global spiral patterns in dwarf galaxies (dS type).
Our sample of observed dwarf objects of this class also includes galaxies with a central stellar bar.
The analysis of the observational data provides a small rotation velocity and a small disk component mass for dS galaxies, which is in poor agreement with the spiral structure generation mechanism in isolated dwarfs due to the development of disk gravitational instability.
Numerical simulation of the stellar–gaseous disks self-consistent dynamics imposes restrictions on the stellar disk thickness and the maximum gas rotation velocity, at which the gravitational mechanism of spiral formation can still be effective. 
\end{abstract}

\section{Introduction}
Dwarf galaxies are small in size and mass compared to classical spiral galaxies (S or SB types) and are usually considered structureless, irregular objects (Irr).
Some late-type dwarfs (Sd~--~Sm types) have a rotating stellar disk without any regular and developed spiral structure.
Such galaxies exhibit flocculent type of spirals, which are discontinuous and consist of short regions~\cite{Mondal-etal-2021}.
Gravitational instability in large massive disks is able to provide a regular spiral pattern covering the entire disk, so Grand Design spiral structure is common for Sa -- Sc galaxy types \cite{Kormendy-2011rev, khrapov-etal-2021galaxies, Griv-etal-2017, Dobbs-Baba-2014spiral, Buta-2013rev}. 

Only a small part of dwarf galaxies shows a global, relatively regular spiral pattern in their disk, and such objects belong to the fairly rare dS type \cite{Zasov-etal-2021dwarf, Magana-Serrano-etal-2020dwarf-dS, Edmunds-Roy-1993-dS}. The observations comparative analysis of normal S- and dS-galaxies represents that such dwarfs are more than just smaller copies of large objects, since their spectral characteristics are similar to Irr galaxies \cite{Hidalgo-Gamez-2004-dS}.
The small size and mass of dwarf galaxies appear to be a theoretical problem for the formation of extended spirals due to gravitational instability \cite{Dobbs-Baba-2014spiral, Buta-2013rev}.

Here we consider the observed properties of the sample of dS galaxies compared to dwarf galaxies without a regular spiral structure \cite{Zasov-etal-2021dwarf}.
The numerical simulation of the dynamics of dwarfs stellar disks with a rich gaseous component makes it possible to determine the conditions under which gravitational instability can generate sufficiently extended spiral patterns in dS galaxies, which morphology is similar to Grand Design galaxies.

% Figure environment removed

\section{Sample of dS-galaxies and its properties}


Our sample is limited to objects (usually type Sc -- Irr) with the absolute magnitude $M_B > -18^{m}$, the optical diameter $D_{25} < 12$\,kpc, $m_B < 15^m$, the inclination angle $i< 75^\circ$ \cite{Zasov-etal-2021dwarf}. 
It is also important to have images in different spectral ranges for deeper study (Figures \ref{fig:ImageObservation1}, \ref{fig:ImageObservation2}).
The logarithm of isolation index $\log(ii)$ characterizes the degree of environmental influence \cite{Makarov-Karachentsev-2011interactions} and we do not consider both Virgo, UMa, Fornax clusters and peculiar galaxies with signs of strong interaction.

% Figure environment removed

Our sample of spiral dwarf galaxies includes 43 objects, which are compared with the sample of dwarfs without spirals (119 objects of Sm and Irr types, see detailed description in \cite{Zasov-etal-2021dwarf}).
Figure~\ref{fig:ImageObservation1} shows dwarf galaxies with bars and two distinct arms.
The images of SDSS\,9 (the Sloan Digital Sky Survey), DECaLS (the Dark Energy Camera Legacy Survey), DSS (the Digitized Sky Survey), PanSTARRS (the Panoramic Survey Telescope and Rapid Response System), LEGA (the DESI Legacy Imaging Surveys) \cite{Dey-etal-2019LEGA} characterize the distributions of the stellar components.
The bottom rows in the fig.~\ref{fig:ImageObservation1},~\ref{fig:ImageObservation2} show the distributions of gas and young stars according to GALEX data (the Galaxy Evolution Explorer).
Figure \ref{fig:ImageObservation2} demonstrates galaxies with a more complex spiral structure, where the yellow lines highlight the positions of the spiral arms. Moreover, the spirals in the stellar and gaseous components are in good agreement with each other. Three-arm patterns indicate a rather massive dark halo.

% Figure environment removed

Each galaxy in our two samples is characterized by the systemic velocity $V_{sys}$, the diameter $D_{25}$, the maximum rotation velocity $V_{rot}$, the HI mass $M_{HI}$, the estimate of the total gravitating mass $M_{dyn}$, the luminosities $L_K$ according to the K-magnitudes of 2MASS catalog \cite{Huchra-etal-2012-2mass}. Statistical analysis gives the following conclusions (See also \cite{Zasov-etal-2021dwarf}).

\noindent --- Dwarf galaxies with developed spiral structure are the most massive objects in the sample.

\noindent --- The distributions of dS galaxies and objects without spirals indicate the absence of very significant differences for various pairs of parameters, for example, $L_K - M_{dyn}$, $L_K - M_{HI}$, $D_{25} - M_{HI}$, $M_{dyn} - M_{HI}$, $V_{rot} - M_{HI}$ and others (Figure \ref{fig:ObservationStatist}).

\noindent --- The HI mass in dS galaxies is, on average, about two times less than in dwarf non-spiral galaxies, although the dynamical and photometric parameters are close for both samples.

\noindent --- The central stellar bar is found both in dS-type objects and in non-spiral galaxies.

\noindent --- Tidal influence is apparently not an essential factor of the  formation in considered galaxies.

\noindent --- The proportion of baryonic matter in spiral dwarfs is, on average, lower than in objects with irregular structure and in giant spiral galaxies, which may indicate the influence of the dark halo on the formation of the spiral patterns in dS-dwarfs.

\noindent --- Remote dwarf galaxies follow the same Tully-Fisher relation as Local Volume dwarfs, but their physical parameters are determined with a larger uncertainties due to the low brightness of these objects, which significantly increases the points scatter on the diagram.

% Figure environment removed

\section{Numerical modeling of the galactic disk}

The numerical models are based on the self-consistent dynamics of the N-body gravitating system for the stellar disk and the gaseous component, which is described by the hydrodynamic equations \cite{khrapov-etal-2021galaxies, Zasov-etal-2021dwarf}.
 %Мы используем прямой метод вычисления сил самогравитации, поскольку он явдяется самым точным \cite{Khrapov-2018chel}. Использование GPUs позволяет проводить достаточно быстрые численные эксперименты с $N = 2^{20}-2^{23}$.
We used direct method to calculate self-gravity forces (each particle interacts with each other), which is the most accurate modeling approach \cite{Khrapov-2018chel}. The GPUs application for parallel code makes it possible to perform fairly fast numerical experiments with $N = 2^{20}-2^{23}$.
 %Численная модель должна обеспечивать бесстолкновительность звездной компоненты \cite{Smirnov-Sotnikova-2018relax, Zasov-etal-2021dwarf}, что достигается обрезанием ньютоновского потенциала на малых радиусах $r_c$. 
The numerical model should ensure the collisionlessness of the stellar component \cite{Zasov-etal-2021dwarf, Smirnov-Sotnikova-2018relax}, which is achieved by cutting off the Newtonian potential at small radii $r_c$.
%Поверхностная плотность звездного экспоненциального диска $\sigma(r) = \sigma_0 \, \exp(-r/r_d)$ характеризуется радиальной шкалой $r_d$. Мы используем следующие безразмерные параметры длины $r=1 \rightarrow 4$\,kpc, скорости $V=1 \rightarrow 47$\,km\,sec$^{-1}$ и времени $t=1 \rightarrow 80 Myr$.
Surface density of the stellar exponential disk $\sigma(r) = \sigma_0 \, \exp(-r/r_d)$ characterized by the radial scale $r_d$. We use dimensionless units of length $r=1 \rightarrow 4$\,kpc, velocity $V=1 \rightarrow 47$\,km\,sec$^{-1}$ and time $t=1 \rightarrow 80$\,Myr.
 
 
 %Рисунок \ref{fig:Relaxation} демонстрирует наличие или отсутствие разогрева звездного диска при соответствующих значениях $r_c$ и $N = N^{(*)} + N^{(h)}$ ($N^{(*)}$ is число частиц в звездном диске, $N^{(h)}$ is число частиц, образующих темное живое гало).
Figure \ref{fig:Relaxation} demonstrates the presence or absence of the stellar disk heating for the corresponding values of $r_c$ and $N = N^{(*)} + N^{(h)}$ (where $N^{(*)} $ is the number of particles in the stellar disk, $N^{(h)}$~is the number of particles that form dark live halo). 

% Мы видим заметное линейное увеличение дисперсии вертикальных скоростей при очень малом радиусе обрезания из-за отсутствия бесстолкновительности в такой модели.Значение $r_c = 0.004$ обеспечивает практически стационарное поведение дисперсий скоростей (green lines). Разогрев более сильный при меньших значениях числа частиц $N^{(*)}$ и $N^{(h)}$ (сравнение рисунков Figure \ref{fig:Relaxation}\textcolor{a} и Figure \ref{fig:Relaxation}\textcolor{b}). 
 We see a noticeable linear increase in vertical velocity dispersion at a very small cutoff radii due the absence of collisionlessness in such a model. The value $r_c = 0.004$ ensures the almost stationary behavior of the velocity dispersions (green lines). 
The heating is stronger for a smaller values of the number of particles $N^{(*)}$ and $N^{(h)}$ (comparison of Figure \ref{fig:Relaxation}a and Figure \ref{fig:Relaxation}b).


% Figure environment removed

 %Рисунок \ref{fig:NumericalModels} показывает формирование различных спиральных структур в моделях карликовых галактик. Мы имеем возможность получать разную морфологию звездного и газового дисков в численных моделях с различными вкладами массы звезд, газа и темной массы в гало. Модельные структуры в газовых и звездных дисках близки к наблюдаемым узорам соответствующих галактик (See Figure~\ref{fig:ImageObservation1}, \ref{fig:ImageObservation2}). Другие примеры расчетов приведены в \cite{Zasov-etal-2021dwarf}.
The figure \ref{fig:NumericalModels} shows various spiral structures in dwarf galaxies models.
We obtain different stellar and gaseous disks morphology in numerical models by varying the relative masses of stars, gas, and dark halo, as well as the radial and vertical scales that determine the distributions of subsystem parameters. The model structures in gaseous and stellar disks are close to the observed patterns of the corresponding galaxies (See Figures~\ref{fig:ImageObservation1}, \ref{fig:ImageObservation2}). Other calculation examples are given in \cite{Zasov-etal-2021dwarf}.

\section{Conclusion} 

% Figure environment removed

 %Фотометрические и кинематические данные наблюдений не позволяют уверенно выделить факторы, обеспечивающие формирование глобального спирального узора в некоторых карликовых галактиках, который является достаточно редким феноменом. Имеется только некоторый дефицит газа в dS-галактиках по сравнению с объектами dIrr. 
Photometric and kinematic observational data do not allow us to confidently identify the factors that ensure the formation of a global spiral patterns in dwarf galaxies, which are a rather rare phenomenon.
There is only some gas deficit in dS galaxies compared to dIrr objects.

 %Мы изучили возможность образования глобальной спиральной структуры в численных моделях изолированных карликовых галактик за счет развития гравитационной неустойчивости в звездном диске с богатым содержанием газа.
We have studied the possibility of the global spiral structure formation in numerical models of isolated dwarf galaxies due to the development of gravitational instability in the stellar disk rich in gas.
 %Возможность генерации спиралей накладывает некоторые ограничения на толщину дисков, радиальные профили дисперсий скоростей в звездном диске, скорости звука в газе, газовой плотности. 
The presence of a spiral pattern in dwarf models imposes some restrictions on the disks thickness, the radial velocity dispersion profiles in stellar disk, the sound speed in gas and the gas density. 
 %Результаты численного моделирования показывают, что максимальная скорость вращения газа должна быть выше 60 км/сек для возбужения спиральных волн существенной амплитуды. Более толстый звездный диск требует больше газа для формирования спирального узора.
The results of numerical simulations show that the maximum gas rotation velocity must be higher than 60 km\,sec$^{-1}$ in order to excite spiral waves with significant amplitude.
Thicker stellar disk requires more gas to form the spiral pattern.

\subsection*{Acknowledgments}
This research has made use of ``Aladin sky atlas'' developed at CDS, Strasbourg Observatory, France. The work was supported by the Ministry of Science and Higher Education of the Russian Federation (state assignment, project No. 0633-2020-0003, implementation of all numerical simulations) and by Russian Foundation for Basic Research (project 20-02-00080~A, observational data analysis).

\section*{References}
\begin{thebibliography}{99}


\bibitem{Mondal-etal-2021} 
Mondal C, Subramaniam A, George K, Postma J E, Subramanian S and Barway S. 2021 \emph{Astrophysical Journal} \textbf{909} 203
% Tracing Young Star-forming Clumps in the Nearby Flocculent Spiral Galaxy NGC 7793 with UVIT Imaging

\bibitem{Kormendy-2011rev} 
Kormendy J 2011 \emph{Secular evolution in disk galaxies} ed J Falcón-Barroso and J H Knapen (Cambridge: Cambridge University Press)

\bibitem{khrapov-etal-2021galaxies} 
Khrapov S, Khoperskov A and Korchagin V 2021 \emph{Galaxies} \textbf{9} 29
% Modeling of Spiral Structure in a Multi-Component Milky Way-Like Galaxy

\bibitem{Griv-etal-2017}
Griv E, Jiang I G and Hou L G 2017 \emph{Monthly Notices of the Royal Astronomical Society} \textbf{468} 3361–-67
%  The nearby spiral density-wave structure of the galaxy

\bibitem{Dobbs-Baba-2014spiral}
Dobbs C and Baba J 2014 \emph{Publications of the Astronomical Society of Australia} \textbf{31} e035
% Dawes review 4: Spiral structures in disc galaxies

\bibitem{Buta-2013rev} 
Buta R 2013 Galaxy morphology \emph{Secular Evolution of Galaxies} ed J Falcón-Barroso and J H Knapen (Cambridge: Cambridge University Press) pp 155--258

\bibitem{Zasov-etal-2021dwarf}  
Zasov A V, Khoperskov A V, Zaitseva N A and Khrapov S S 2021 \emph{Astronomy Reports} \textbf{65} 1215--32 
%On the Formation of Spiral Arms in Dwarf Galaxies

\bibitem{Magana-Serrano-etal-2020dwarf-dS} 
Magana-Serrano M A, Hidalgo-Gamez A M,
 Vega-Acevedo I and  Castaneda H O 2020 \emph{Revista Mexicana de Astronomia y Astrofisica} \textbf{56} 39--54
 %Star formation rate in late-type galaxies: I-The H$\alpha$ and FUV integrated values

\bibitem{Edmunds-Roy-1993-dS} 
Edmunds M G and Roy J R 1993 \emph{Monthly Notices of the Royal Astronomical Society} \textbf{261} L17-L19 
% The co-existence of spiral structure and abundance gradients

\bibitem{Hidalgo-Gamez-2004-dS} 
Hidalgo-Gamez A M 2004 \emph{Revista Mexicana de Astronomía y Astrofísica} \textbf{40} 37--51
% Dwarf and Normal Spiral Galaxies: are they Self-Similar?

\bibitem{Makarov-Karachentsev-2011interactions} 
 Makarov D and Karachentsev I 2011 \emph{Monthly Notices of the Royal Astronomical Society} \textbf{412} 2498--2520.
% Galaxy groups and clouds in the local ($z ∼ 0.01$) Universe

\bibitem{Dey-etal-2019LEGA} 
Dey A  et al. 2019 \emph{Astronomical Journal} \textbf{157} 168  
% Overview of the DESI Legacy Imaging Surveys 

\bibitem{Lisenfeld_etal_07}
Lisenfeld  U et al. 2007 \emph{Astronomy and Astrophysics} \textbf{462} 507--523
% he AMIGA sample of isolated galaxies. III. IRAS data and infrared diagnostics

\bibitem{McGaugh_Schombert}
McGaugh S S and Schombert J M 2014 \emph{The Astronomical Journal} \textbf{148} 77
% Color-Mass-to-light-ratio Relations for Disk Galaxies

\bibitem{Karachentsev_etal_17}
Karachentsev I D, Kaisina E I, Kashibadze N and Olga G 2017 \emph{The Astronomical Journal} \textbf{153} 6
%The Local Tully-Fisher Relation for Dwarf Galaxies

\bibitem{Huchra-etal-2012-2mass} 
Huchra J P et al 2012 \emph{Astrophys. J. Suppl. Ser.} \textbf{199} 26

\bibitem{Khrapov-2018chel} 
Khrapov S S, Khoperskov S A and Khoperskov A V 2018 \emph{Bulletin of the South Ural State University, Series: Mathematical Modelling, Programming and Computer Software} \textbf{11} 124–-136
% New features of parallel implementation of N-body problems on GPU

\bibitem{Smirnov-Sotnikova-2018relax} 
Smirnov A A and Sotnikova N Ya 2018 \emph{Astronomical Society} \textbf{481} 4058--76
% hat determines the flatness of X-shaped structures in edge-on galaxies?



























%1
%2

%3

%4

%5

%6

%7

%8

%9
%10

%11

%12


%13

%14

%15

%16

%17
\end{thebibliography}
\end{document}%###############################################################################################################################################


\begin{table}[h]
\caption{\label{tabone}A simple example produced using the standard table commands 
and $\backslash${\tt lineup} to assist in aligning columns on the 
decimal point. The width of the 
table and rules is set automatically by the 
preamble.} 

\begin{center}
\lineup
\begin{tabular}{*{7}{l}}
\br                              
$\0\0A$&$B$&$C$&\m$D$&\m$E$&$F$&$\0G$\cr 
\mr
\0\023.5&60  &0.53&$-20.2$&$-0.22$ &\01.7&\014.5\cr
\0\039.7&\-60&0.74&$-51.9$&$-0.208$&47.2 &146\cr 
\0123.7 &\00 &0.75&$-57.2$&\m---   &---  &---\cr 
3241.56 &60  &0.60&$-48.1$&$-0.29$ &41   &\015\cr 
\br
\end{tabular}
\end{center}
\end{table}

\begin{center}
\begin{table}[h]
\centering
\caption{\label{jfonts}Font styles for a reference to a journal article.} 
\begin{tabular}{@{}l*{15}{l}}
\br
Element&Style\\
\mr
Authors&Roman type\\
Date&Roman type\\
Article title (optional)&Roman type\\
Journal title&Italic type\\
Volume number&Bold type\\
Page numbers&Roman type\\
\br
\end{tabular}
\end{table}
\end{center}