\section{Conclusion} \label{sec:conclusion}
% We present CryptoMask, a practical privacy-preserving face recognition protocol based on homomorphic encryption and secure multi-party computation techniques. The novel designed encoding strategy enables an efficient enrolment process, allowing DP to add more face vectors to the CS. We construct an efficient matrix computation for distance computation based on the encoding method. Different from the state-of-the-art leaks the distance to the verifier, we protect the intermediate results using a secure result-revealing protocol. The experiments show that CryptoMask outperforms the state-of-the-art both in computation and communication. 

We introduce CryptoMask, a practical privacy-preserving face recognition protocol that leverages homomorphic encryption and secure multi-party computation techniques. Our encoding strategy facilitates an efficient enrollment process, enabling DP to add more face vectors to CS. We construct an efficient matrix computation for distance calculation, based on our encoding method. Unlike existing state-of-the-art techniques that reveal the computed distance to the verifier, we protect intermediate results using a secure result-revealing protocol. Our experiments show that CryptoMask outperforms existing approaches in both computation and communication.

% The relied new encoding approach allows us to compress the ciphertext which saves plenty of storage overhead. A more efficient inner product computation method employed from Cheetah~\cite{huang2022cheetah} enables us to improve the computation overhead by avoiding expensive rotation and homomorphic addition. Different from prior work exposing the relationship of the query and each record in the database, we introduce a secure comparison between the CS and the client, which ensures the client can only learn how many faces are similar to hers. The experiment results show that our protocol significantly improves the performance compared with the existing literature.


