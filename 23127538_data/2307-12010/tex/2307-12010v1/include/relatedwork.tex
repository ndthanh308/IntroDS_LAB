%shao hang%

\section{Related Work} \label{sec:relatedwork}
For the protection methods of biometric security authentication in the past, we can roughly divide them into protection based on traditional cryptographic algorithms, protection based on pattern recognition and protection based on privacy computing:\par
    
\textbf{Protection based on traditional cryptographic algorithm.} In 1998, Soutar et al. ~\cite{soutar1998biometric}first proposed to apply encryption technology to fingerprint authentication. For example, visual password ~\cite{naor1994visual} can be applied to biometric authentication such as fingerprint identification ~\cite{rao2008fingerprint}, iris identification ~\cite{revenkar2010secure} and face recognition ~\cite{ross2010visual}. In addition, fuzzy value ~\cite{juels2006fuzzy} technology is used to achieve fingerprint authentication ~\cite{uludag2005fuzzy} and iris authentication ~\cite{lee2008new}. Although this method can achieve biometric authentication, However, biometric data may be exposed during authentication and matching, and the matching accuracy is not high.

\textbf{Protection based on pattern recognition.} Unlike the security provided by the encryption algorithm, biometric authentication based on pattern recognition is another security authentication method. For example, Davida et al. ~\cite{davida1998enabling} and Radha et al. ~\cite{ratha2001enhancing} use non invertible transformation functions to protect biometrics. Patel et al. ~\cite{patel2015cancelable} and Jin et al. ~\cite{jin2017ranking} propose a security authentication model of revocability and diversity of templates, This method requires a trade-off between security and matching performance, and is not practical.  
        
\textbf{Protection based on privacy computing.} Shai et al. ~\cite{avidan2006efficient} used encryption in secure multi-party computing to protect features, but the computing speed was slow, and then introduced machine learning to speed up computing efficiency; Shashank et al. ~\cite{shashank2008private} realized image retrieval using the technology of oblivious transfer (OT); Maneesh et al. ~\cite{upmanyu2009efficient} implemented a video surveillance system based on the secret sharing technology constituted by the Chinese Residual Theorem (CRT), which can effectively detect and track personnel; Andy et al. ~\cite{abadi2016deep} used differential privacy to infer and learn privacy data, and then applied it to face recognition systems. Homomorphic encryption can directly calculate the distance of feature vectors (Euclidean distance or cosine distance) in the encryption domain because it can calculate ciphertext directly, and then complete feature matching. Most of the original schemes use partial homomorphic encryption (PHE). For example, Maneesh et al. ~\cite{upmanyu2009efficient2} proposed a biometric matching scheme using the additive homomorphism of Pailllier algorithm, which requires repeated interaction between the client and the data end to calculate the similarity, Therefore, the communication and computing costs are large. By using fully homomorphic encryption(FHE) technology, arbitrary computation can be completely performed between ciphertext, which can effectively reduce the traffic. Troncoso et al. ~\cite{troncoso2013fully} designed a face recognition system based on the early STOC09 ~\cite{gentry2011implementing} fully homomorphic encryption scheme, but its efficiency is low due to the expensive homomorphic computation; In 2018, Vishnu et al. ~\cite{boddeti2018secure} designed a face recognition protection scheme, in which the fully homomorphism encryption of the BFV scheme was used to calculate the cosine distance, and the batch processing (SIMD technology) and dimension reduction technology were used to improve the computing efficiency. For 512 dimensional facial features, it only took 16KB of storage space for each encrypted feature and 0.02s of time to match a pair of encrypted features. In 2022, they proposed a new face recognition framework HERS ~\cite{engelsma2022hers} on the basis of ~\cite{boddeti2018secure}, which optimized the batch processing technology to achieve 1: m matching. Compared with using ciphertext rotation to calculate the inner product, it improved the calculation efficiency. The matching time in 1 million encrypted images was 10 minutes, and the loss of accuracy was controlled within $2\%$. It can be seen that homomorphic encryption is a feasible and practical solution for accurate face matching in the encryption domain, which not only protects user privacy, but also prevents information leakage.  

On the basis of ~\cite{boddeti2018secure} and ~\cite{engelsma2022hers}, our protocol uses a more efficient method to calculate the inner product, and proposes a new encoding method, which compresses data when encoding, further reducing communication and improving computing efficiency. Experiments show that the protocol has better performance and is more suitable for scalable databases\notesh{It can supplement the experimental results}.