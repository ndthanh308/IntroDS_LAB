% SKETCH

% 1.Face recognition has received tremendous attention in the pattern recognition and biometrics communities. Data owners upload considerable facial feature vectors to a database and later a client intends to know whether a face image is in the database without learning any other information.  

% 2.Most of existing work~\cite{boddeti2018secure}~\cite{engelsma2022hers} encrypt the face features by Fully Homomorphic Encryption~(FHE). They optimize the computation by employing SIMD primitive based on specific encoding scheme. The state-of-the-art~\cite{engelsma2022hers} shows the required homomorphic multiplication is $\lceil \frac{m}{n} \rceil d$ where $d$ is feature dimension, $m$ is feature size and $n$ is a the degree of polynomial. 

% 3.We use a different encoding scheme based on~\cite{huang2022cheetah}. In our privacy-preserving face recognition, we only require $\lceil \frac{m}{n} \rceil$ homomorphic multiplication which is more efficient than the-state-of-the-art. Besides, we avoid expensive rotations thanks to Cheetah~\cite{huang2022cheetah}. 

% 4.Most of the HE-based face recognition protocols reveal all the distance to the client which violates the owners privacy. Different from previous work, in our protocol, the client cannot learn anything besides whether a face image is in the database. We carefully combine homomorphic encryption with OT as~\cite{huang2022cheetah} and also give optimizations for comparison which is of independent interest(if have).

% 5.The experiments show that our protocol reduces $10 \times$ computation compared with~\cite{engelsma2022hers}. 

\section{Introduction}\label{sec:introduction}
% In recent years, biometrics authentication has become increasingly important in various applications. 
% In this work, we focus on face recognition, a technique that identifies or confirms a person's identity with their facial characteristics.
% Due to its simplicity and convenience, face recognition has attracted more and more attention from real applications like the surveillance of public places (\eg,~streets, airports, etc.)~\cite{parmar2014face}, social media~\cite{cherepanova2021lowkey} and corporate punch card supervision~\cite{haigh2001chromium}, to name some.

Biometric authentication has become increasingly vital in various applications in recent years. This work focuses on face recognition, which identifies or verifies a person's identity based on their facial features. Due to its ease of use and convenience, face recognition has gained significant traction in real-world applications such as public place surveillance (\eg, streets, airports, etc.)~\cite{parmar2014face}, social media~\cite{cherepanova2021lowkey}, and corporate punch card supervision~\cite{haigh2001chromium}.

% However, potential privacy risks come with the wide adoption of face recognition techniques. 
% In a typical face recognition system, a server stores face images from a collection of verified users. A verifier who holds a user's face image queries whether the image is in the database by measuring the similarity or distance between the queried image and the images in the database, thus knowing whether the user is verified. 
% Typically, the verifier must directly upload the face image to the server for recognition. However, in many cases, disclosing the face image of the users to the server may not be permitted due to privacy concerns or the possibility of human rights abuses~\cite{bowyer2004face}. \textit{Thus, it is a natural desire to build a privacy-preserving face recognition protocol that protects data privacy while maintaining its efficiency}. 

% As face recognition systems become more widespread, concerns about privacy have grown. In a typical system, a server stores face images belonging to verified users. 
As face recognition systems become more widespread, concerns about privacy have grown. In a typical system, a server stores face images belonging to users who are registered. When a verifier, who possesses a user's face image, queries the server to check if the user is verified, the system measures the similarity or distance between the queried image and the images in the database. However, in many cases, it may not be permissible to disclose users' face images to the server due to privacy concerns or the possibility of human rights abuses~\cite{bowyer2004face}. \textit{Therefore, it is essential to develop privacy-preserving face recognition protocols that protect data privacy while maintaining efficient recognition}.


% A straightforward way is to encrypt all the images and perform face recognition over encrypted data. 
% Some of the literature employ a promising encryption scheme called Homomorphic Encryption (HE), which was first proposed in~\cite{rivest1978data} and realized in~\cite{gentry2009fully}. 
% HE allows us to perform computation in the encrypted domain without decryption. However, ease of use comes with a price. The HE-based privacy-preserving face recognition protocol~\cite{boddeti2018secure} is tested several orders of magnitudes slower than the original one even when they utilize the Single-Instruction-Multiple-Data~(SIMD) technique~\cite{smart2014fully} to amortize the cost of homomorphic operations. 
% The approach proposed in~\cite{engelsma2022hers} explores the encoding method on the database of images, which reduces the number of required homomorphic multiplications and rotations and improves the computation efficiency. 

% Encrypting all images and performing face recognition over encrypted data is a straightforward approach to ensure data privacy.
Encrypting pre-processed images (\eg,~extracted face vectors) and performing face recognition over encrypted data is a straightforward approach to ensure data privacy. Homomorphic Encryption (HE) is a promising encryption scheme for this purpose, which was first proposed in \cite{rivest1978data} and realized in \cite{gentry2009fully}. HE allows computation in the encrypted domain without decryption. However, HE-based privacy-preserving face recognition protocols, such as the one proposed in \cite{boddeti2018secure}, are several orders of magnitudes slower than the original method, even when utilizing the Single-Instruction-Multiple-Data (SIMD) technique \cite{smart2014fully} to amortize the cost of homomorphic operations. To overcome this, the approach proposed in \cite{engelsma2022hers} explores encoding methods on the image database, reducing the number of homomorphic multiplications and rotations required and improving computation efficiency.
Moreover, previous works \cite{boddeti2018secure,engelsma2022hers} in this field fail to protect the private information of the database, as they allow the verifier to learn sensitive distance or similarity information and the number of face images close to the queried one.

% In this paper, we further improve the performance by designing a novel encoding method and employing an efficient matrix multiplication method over the encrypted domain. Besides, we observe that existing works fail to protect the privacy of the database because they allow the verifier to learn sensitive distance or similarity information directly; the verifier can additionally learn how many face images in the database are close to the given one. We solve this problem by designing a secure result-revealing protocol. In our work, the verifier only learns 1 bit of information indicating whether the given face image is in the database. We name our privacy-preserving face recognition protocol as CryptoMask. 
% Table~\ref{table::comparison} compares the existing distance-based privacy-preserving biometric (face recognition) schemes constructed via HE in terms of the computation and storage overhead, and information leakage. We can see that our system \sys~requires the least number of HE multiplications and additions and has the minimum leakage.

In this paper, we propose Cryptomask, an efficient privacy-preserving face recognition protocol that only reveals a single bit of information to the verifier, indicating whether the queried face image is present in the database. 
%by introducing a novel encoding method and an efficient matrix multiplication method over encrypted data.  To address this issue, we design a secure result-revealing protocol 
We propose a novel encoding method to encrypt the database in a compact manner, resulting in improved performance. For distance computation, we use efficient matrix multiplication techniques that avoid expensive homomorphic rotations. Additionally, we ensure the privacy of distance calculations by designing a secure result-revealing protocol and optimizing its efficiency. 
CryptoMask outperforms existing distance-based privacy-preserving biometric schemes constructed via HE in terms of computation and storage overhead, and information leakage. Table~\ref{table::comparison} provides a comparison of different schemes, showing that our approach requires the least number of HE multiplications and additions and has minimal information leakage.
We implement CryptoMask and compare its performance with existing works~\cite{boddeti2018secure} and~\cite{engelsma2022hers}.
In the case of a database with 100 million face images, CryptoMask outperforms others up to~${\thicksim}5 \times$ and~${\thicksim}144~\times$ in computation and communication, respectively.  
%We propose CryptoMask, an efficient privacy-preserving face recognition protocol. CryptoMask enables the verifier to check if the requested face vector is in the database without learning anything about the database meanwhile, the server cannot know the face vector from the verifier. 
% \vspace{-5mm}
\begin{table*}[!ht]
    \footnotesize  
	\centering 
 \caption{S{\upshape ummary of existing privacy-preserving face recognition protocols}.}\label{table::comparison}
	\begin{threeparttable}%for tablefootnote, should include #usepakage threeparttable
	\begin{tabular}{c|c|c|c|c|c}
		% \hline
		\toprule
		\textbf{Protocol} & 
        % \textbf{Round} & 
        \textbf{Multiplication} & \textbf{Addition} & \textbf{Rotation} & \textbf{Memory} &
        \textbf{Leakage}\\
		% \hline
		% \hline
        \hline
        % na{\"\i}ve & $mC_{ip}^{plain}$ & $d\ell$ & 1 & $\boldsymbol{A},\boldsymbol{b},\boldsymbol{D},r$ & NULL & $md$ & $m(d-1)$ & 0 & $O(md\ell)$\\
        Na{\"\i}ve & $md$ & $m(d-1)$ & 0 & $O(md\ell)$& $\boldsymbol{A},\boldsymbol{b},\boldsymbol{\mathsf{d}},r$\\
        \hline
        Hu \etal~\cite{hu2018outsourced} & $md^3$ & $md^2(d-1)$ & 0 & $O(md^2)$ & $\boldsymbol{\mathsf{d}},m$ \\
        \hline
        Pradel \etal~\cite{pradel2021privacy} & $md$ & $m(d-1)$ & 0 & $O(mdN)$ & $\boldsymbol{\mathsf{d}},m$ \\
        \hline
        Boddeti \etal~\cite{boddeti2018secure} & $m$ & $m$log$_2d$ & $m$log$_2d$ & $O(mN)$ & $\boldsymbol{\mathsf{d}},m$\\
        \hline
        HERS~\cite{engelsma2022hers} & $\lceil \frac{m}{N}\rceil d $ & $\lceil \frac{m}{N}\rceil (d-1) $ & 0 & $O(dN\lceil \frac{m}{N} \rceil)$  & $\boldsymbol{\mathsf{d}},m$ \\
        \hline
        Erkin \etal~\cite{erkin2009privacy} & $m(d+2)$ & $2m(d-1)$ & 0 & $O(mdN)$ & $m$ \\
        \hline
        \textbf{CryptoMask} & $\lceil \frac{m}{N-d}\rceil d $ & $\lceil \frac{m}{N-d}\rceil d $ & 0 & $O( dN \lceil\frac{m}{N-d}\rceil)$ & $m$ \\
		\bottomrule
	\end{tabular}
	\footnotesize{$\boldsymbol{A}:$ database containing face vectors; $\boldsymbol{b}:$ queried face vector; $m:$ database size; $d:$ dimension of each face vector; $N:$ HE plaintext polynomial degree; $l:$ length of each element in face vector;   $\boldsymbol{\mathsf{d}}:$ distance vector; $r:$ face recognition result. The notation $\lceil x \rceil$ denotes rounding up to the nearest integer of $x$. na{\"\i}ve represents the face recognition performed in plaintext. % scenario \refscui{original face recognition}{unclear}.
	}
    \end{threeparttable}
    \vspace{-3mm}
\end{table*}
% \vspace{-5mm}
% \vspace{-5mm}
% \begin{table*}[!ht]
%     \scriptsize  
% 	\centering 
%  \caption{S{\upshape ummary of existing privacy-preserving face recognition protocols}.}\label{table::comparison}
% 	\begin{threeparttable}%for tablefootnote, should include #usepakage threeparttable
% 	\begin{tabular}{c|c|c|c|c|c|c}
% 		% \hline
% 		\toprule
% 		\textbf{Protocol} & 
%         % \textbf{Round} & 
%         \textbf{Multiplication} & \textbf{Addition} & \textbf{Rotation} & \textbf{Memory} &
%         \textbf{Leakage} & \textbf{Technique}\\
% 		% \hline
% 		% \hline
%         \hline
%         % na{\"\i}ve & $mC_{ip}^{plain}$ & $d\ell$ & 1 & $\boldsymbol{A},\boldsymbol{b},\boldsymbol{D},r$ & NULL & $md$ & $m(d-1)$ & 0 & $O(md\ell)$\\
%         na{\"\i}ve & $md$ & $m(d-1)$ & 0 & $O(md\ell)$& $\boldsymbol{A},\boldsymbol{b},\boldsymbol{\mathsf{d}},r$ & NULL\\
%         \hline
%         Hu \etal~\cite{hu2018outsourced} & $md^3$ & $md^2(d-1)$ & 0 & $O(md^2)$ & $\boldsymbol{\mathsf{d}},m$ & Symmetric-key\\
%         \hline
%         Pradel \etal~\cite{pradel2021privacy} & $md$ & $m(d-1)$ & 0 & $O(mdN)$ & $\boldsymbol{\mathsf{d}},m$  & TFHE\\
%         \hline
%         Boddeti \etal~\cite{boddeti2018secure} & $m$ & $m$log$_2d$ & $m$log$_2d$ & $O(mN)$ & $\boldsymbol{\mathsf{d}},m$ & BFV\\
%         \hline
%         HERS~\cite{engelsma2022hers} & $\lceil \frac{m}{N}\rceil d $ & $\lceil \frac{m}{N}\rceil (d-1) $ & 0 & $O(dN\lceil \frac{m}{N} \rceil)$  & $\boldsymbol{\mathsf{d}},m$ & BFV\\
%         \hline
%         Erkin \etal~\cite{erkin2009privacy} & $m(d+2)$ & $2m(d-1)$ & 0 & $O(mdN)$ & $m$ & Paillier, DGK\\
%         \hline
%         CryptoMask & $\lceil \frac{m}{N-d}\rceil d $ & $\lceil \frac{m}{N-d}\rceil d $ & 0 & $O( dN \lceil\frac{m}{N-d}\rceil)$ & $m$ & BFV,SS,OT\\
% 		\bottomrule
% 	\end{tabular}

% 	\footnotesize{$m:$ database size; $d:$ dimension of each face vector; $N:$ HE plaintext polynomial degree; $l:$ length of each element in face vector;   $\boldsymbol{\mathsf{d}}:$ distance vector; $r:$ face recognition result. Paillier~\cite{paillier1999public}, DGK~\cite{damgaard2007efficient}, OT~\cite{ishai2003extending}, TFHE\cite{chillotti2020tfhe}. The notation $\lceil x \rceil$ denotes rounding up to the nearest integer of $x$. na{\"\i}ve represents the face recognition performed in plaintext. % scenario \refscui{original face recognition}{unclear}.
% 	}
%     \end{threeparttable}
% \end{table*}
% \vspace{-5mm}

%$^*$: We only count the computation on the server side. $m$ is the number of facial records in the database. $C_{ip}^{plain}$ is the computation cost for one inner product under plaintext data.

\subsection{Related Work}
The early work given in~\cite{ross2010visual} relies on secret sharing to authenticate face recognition. However, it cannot ensure the privacy of face images. 
There are some similar works~\cite{rao2008fingerprint,uludag2005fuzzy,lee2008new} working for biometric authentication. Another line is employing pattern recognition to protect the queried database~\cite{patel2015cancelable,jin2017ranking}. However, this method also fails to ensure the security of the database and the queried face image. Some works~\cite{shashank2008private,upmanyu2009efficient} employ secure multi-party computation (MPC)~\cite{yao1986generate} to achieve the privacy-preserving goals, yet they are communication costly due to multiple interactions between the participants. Homomorphic encryption~\cite{rivest1978data} allows computations to be performed over encrypted data without first decrypting it. Many face recognition protocols~\cite{upmanyu2009efficient2,erkin2009privacy,troncoso2013fully,boddeti2018secure,engelsma2022hers} based on HE have been proposed. Unfortunately, they either result in heavy computation~\cite{erkin2009privacy,troncoso2013fully,boddeti2018secure} or cannot provide full secrecy~(\eg~leakage of distance similarity)~\cite{boddeti2018secure,engelsma2022hers}. We fill this gap by employing HE to perform distance computations and utilizing MPC to do a secure result-revealing process. Compared with the state-of-the-art~\cite{engelsma2022hers}, our work reduces both the computation and communication while maintaining the privacy of not only inputs and outputs but also intermediate data.
  
        
