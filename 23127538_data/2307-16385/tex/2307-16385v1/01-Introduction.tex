% 1 - SoRo real-life realization gap
% [VV] Remove 1st sstance
% [VV] Put a short paragraph about the gap and then continue with the challenges
% [VV] - "To best of our knowledge" at beginning or at the end. In the middle, it would get lost.
% [VV] - Merge gaits, and their outcomes as figure* rather than multiple figures
% [VV] - Other gait citations in bibtex
Since the advent of the soft Mckibben actuator, soft materials have been envisioned to be an integral part of next generation robots, including for terrestrial environments. This can be attributed to their ability to adapt and interact with the environment. Over the past few decades there has been active research in discovery of novel soft materials, actuators and sensors \cite{LaEtAl_Science[2016],RusEtAl_Nature[2015]}. Correspondingly, there has been advancement in modeling and control techniques for soft systems like manipulators \cite{ArmEtAl_TRO[2023]}. Additionally, researchers have demonstrated terrestrial locomotion of soft robots using ad-hoc or intuitive gaits \cite{SuEtAl_CurrRobRprts[2021]}. However, in comparison to their rigid counterparts, path planning and navigation of soft robot locomotors is understudied and rarely implemented. This can be primarily attributed to challenges related to modeling of soft materials and their actuation, plus robot-environment interaction. Furthermore, unlike rigid robots, soft terrestrial robots generate lower magnitudes of force when interacting with the environment. This has multiple consequences, including higher variance in locomotive gait displacements (translation and rotation) and higher sensitivity to small environmental changes. 
The research focus is to mitigate the sources of discrepancy between the potential of soft terrestrial robots and their real-life realization by developing a real-time, closed-loop locomotion controller with localization feedback. 

% 2 - Gait Synthesis
 Finite-element modeling and different reduced-order models have been explored by researchers for creating locomotion models \cite{ChangEtAl_RAS[2020],chirikjian1995kinematics,ChangEtAl_IROS[2021],CoEtAl_RoboSoft[2019],BernEtAl_RSS[2019]}. However, the accuracy and predictability of these approaches remains limited.  Models for the friction and sliding of soft materials over a substrate have proven inadequate to capture the robot-environment interaction. Additionally, soft robots are sensitive to manufacturing inaccuracies and defects. Consequently, most locomotion control strategies for soft terrestrial robots rely on biomimetic, intuitive approaches, or trial-and-error \cite{SuEtAl_CurrRobRprts[2021],MaEtAl_JrnofBionicEng[2014]}. More recently, environment-centric, data-driven model-free approaches have been implemented to synthesize gaits \cite{VikasEtAl_IROS[2015],FreemanEtAl_TRO[2023]}. This paper utilizes the gaits synthesized by these approaches as briefly discussed later in the paper.

% 3 - Feedback control and Experimental Setup
Tethered and untethered soft locomotors are actuated using various methods, e.g., pneumatic, shape memory alloys (SMAs), dielectric elastomers (DEA), and motor-tendon actuators\cite{SuEtAl_CurrRobRprts[2021]}. While most use open-loop control strategies, closed-loop control has been performed by few researchers. Patterson et al use a reactive strategy to perform closed-loop control of an SMA-actuated soft swimming robot \cite{PaEtAl_IROS[2020]}; Liu et al \cite{LiuEtAl_IROS[2021]} use a reactive planner for control of a pneumatically actuated soft robot with predetermined gaits; Hamill et al \cite{HaEtAl_ICRA[2019]} perform gait-based path planning using temporal logic; Lu et al \cite{LuEtAl_FrontRobAI[2020]} apply bidirectional A$^*$ with a time varying bounding box. For pose recovery, soft robotics researchers typically rely on economically expensive motion capture systems, e.g., VICON and Optitrack, limiting their more widespread study.
%as it is influenced by factors like marker occlusion. 
This research employs a lattice-based trajectory planner; robot gait models inform the design of of controlled trajectories that move the robot from start to goal while avoiding obstacles. It also details an experimental setup that uses two inexpensive overhead webcams and a localization algorithm that compensates for marker occlusion.

{
Multi-modal motion planning for traditional rigid-body mobile robots typically entails a solution search, through the robot configuration space, for a state sequence (and accompanying control) to accomplish an objective while respecting both task and feasibility constraints. Though articulated mobile robots (e.g. humanoids, multi-legged mechanisms) are often natively described by high-dimensional configuration spaces, planning exploits dimensional-reduction strategies that specialize the search space to modes and configurations relevant to a particular task \cite{DoEtAl_ICRA[2018], BaKaLo_ICRA[2013]_HierApproach_Manip, HaHiGo_ISRR[2011]}; these manifest in conjoinments of several graph-based representations that together capture the multi-modal search space, and where sampling approaches may be utilized to construct graphs or graph-to-graph (i.e. mode-to-mode) transitions. 
Similarly, graph-based representations have been used to synthesize motion plans for vehicles capable of multiple geographically-dependent modalities (e.g. swimming, driving, flying); graphs are synthesized using a sampling-based approach, with edges valued according to modal cost of transport. Dijkstra's algorithm may then be used to identify the optimal multi-modal solution when traveling between locations \cite{SuEtAl_IROS[2020]}.
For a hyper-redundant snake-like robot, a hybrid-optimization approach employs mixed integer programming with model predictive control (MPC) to guide step climbing; the former enforces a particular sequence of discrete modes to be traversed, while the latter designs reference trajectories within each mode \cite{KoTaTa_TCST[2016]}. 
This work focuses on the \MSoRo~soft mobile robot, capable of several locomotion gaits; trajectory planning and re-planning entails a lattice-based search through the space of possible gait sequences, for guiding the robot to desired goal locations within obstacle-strewn environments and in the face of locomotion uncertainty. 
}
% % \hl{System Integration}
% \begin{enumerate}
%     % \item SoRos were supposed to be the next generation robots, even for terrestrial environments. However, this gap has not shrunk. [citation]
%     \item This is "low-force regime" work. [citation]
%     \item Unlike pneumatic, MTAs are low-force generators [citation].
%     \item Challenge specific to SoRos : they rotate with translation, and their localization is often influenced by marker occlusion. The experimental setup compensates for that.
% \end{enumerate}


\textbf{Contributions.} This research, to best of our knowledge, is the first instance of real-time path planning and closed-loop control of a non-pneumatic  soft robot. %The research proposes a generic, closed-loop path planning framework for terrestrial soft robots. 
The experimental setup involves system integration of high-level (online path planning and offline gait synthesis) control, low-level control (\MSoRo~actuation), and real-time pose estimation that uses parallel architecture to process visual feedback. %To the best of our knowledge, this is the first instance of such system integration and experimental validation of closed loop path planning for motor-tendon actuated soft robots. 
Trajectory planning is accomplished using a lattice-based, greedy breadth-first search through the robot's gait control space; motion models of available robot gaits inform the design of controlled locomotion trajectories that move the robot from start to goal while avoiding obstacles. A rotate-then-translate motion control paradigm is adopted to both simplify the procedure and aid tractability of the planning problem. The planned trajectory is re-computed when the position error between the estimated and the actual path exceeds a prescribed threshold.

\textbf{Organization.}
%The paper is structured as follows: 
\Sec \ref{Sec:ProblemFormulation} formulates the navigation problem, details the robot, and provides an overview of data-driven gait discovery and selection. Next, \Sec \ref{Sec:Methodology} discusses the closed-loop control framework, architecture and  path-planning methodology. \Sec \ref{Sec:Experiments} contains the experimental setup, methodology, tracking algorithms, and results. \Sec \ref{Sec:Conclusion} concludes the paper and discusses future work.