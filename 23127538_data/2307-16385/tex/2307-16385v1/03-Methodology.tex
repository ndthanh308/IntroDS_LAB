
The control framework architecture comprises a high-level controller in \MATLAB~that communicates with the low-level \Arduino~microprocessor to result in locomotion of the robot as summarized in \Fig \ref{Fig:ControlArchitecture}. Localization is achieved by analyzing the webcam output and tracking the center of the robot. In the offline state, the Gait Synthesizer uses the motion data from the Euler cycle experiments to build the Gait Library. The gaits therein are then experimentally validated to store the expected motion data to feed into the path planner. In the online state, the experiment world (which maps the obstacles and initial robot pose) and the gait library are used to initialize the path planner. Real-time control is then achieved by comparing the expected pose from the path planner and the instantaneous experiment pose to inform the low-level controller and re-planning is performed as necessary.
% 
\subsection{Localization}
\label{Subsec:Localization}
%The experimental setup consists of the the \MSoRo~on a rubber garage mat, and two overhead webcams. Low-level robot control is performed by the \Arduino and is integrated with high-level planning and localization that is performed in MATLAB\textregistered. The first webcam is used to capture HD video of the experiments. The second webcam has its properties (e.g., contrast, brightness, etc.) adjusted to allow for better image segmentation of the four neon markers on the robot hub to facilitate tracking. Similarly, the obstacles, and goal are identified. The real-time robot pose is estimated as described in \Sec \ref{Subsec:Localization}. The Gait Library described in \Sec \ref{Subsec:GaitSynthesizer} informs path planning described in Sec. \ref{Subsec:TrajectoryPlanning}.

The feedback to the robot controller plays a critical role in path re-planning. As can be seen in the experimental setup, \Fig \ref{Fig:ControlArchitecture}b, two overhead webcams are used. One webcam is used to record HD video; the other is used for localization and has its properties (e.g., contrast, brightness, etc.) adjusted to facilitate image segmentation of the four neon markers on the robot hub. Both webcam videos are processed in parallel using the \MATLAB~Parallel Processing Toolbox\textsuperscript{\texttrademark}.
%The experiment performs localization of the robot using two web-cameras that are processed in parallel using the \MATLAB Parallel Processing Toolbox. The web-cameras have specific tasks - one is dedicated for localization purposes, while the other records HD video. Each of the video stream is processed by separate cores on the microprocessor. 
The ``localization core" video is processed through a image mask that highlights the markers located on the robot (blue markers on orange robot), obstacles (pink) and the target (green cross), as seen in \Fig \ref{Fig:ControlArchitecture}. This video is stored and a pollable data queue accesses the data for localization at appropriate times. The pose estimation is performed using Arun's method \cite{arun1987least}, which finds the least-squares solution to the pose by using singular value decomposition. Occlusion of the markers by the tether is managed by identifying the marker in the next time frame using the nearest neighbor. The occluded markers are reconstructed using the estimated pose. 
\textadded{Camera capture occurs at $\sim 30$Hz. 
The robot pose estimation provides feedback at $6$Hz on an Intel\textsuperscript{\textregistered} Xeon\textsuperscript{\textregistered} E5-1650 v4, as dictated by the computational complexity and the robot's average motion profile.}
%every 5th frame out of the 30 frames per second is been processed for real time robot’s pose estimation.}

% \begin{algorithm}[h]
% \caption{Robot pose estimation}\label{Alg:PoseEstimation}
% \KwData{Two sets of markers $\bm{p}_i,\bm{p}_i'$ related by\\$\displaystyle \bm{p}_i'=Rp_i+\bm{t}+\bm{n}_i,~\forall i=1,2,\cdots, N$}
% \KwResult{$R \in SO(3), ~\bm{t}\in\Re^{3\times 1}$}
% { $\displaystyle \bm{p} \gets {\frac{1}{N}\sum_{i=1}^N \bm{p}_i}, \quad \bm{p}' \gets {\frac{1}{N}\sum_{i=1}^N\bm{p}_i'}$\;
% $\displaystyle \bm{q}_i \gets {\left(\bm{p}_i-\bm{p}\right)}, \quad  \bm{q}_i' \gets {\left(\bm{p}_i'-\bm{p}'\right)}$\;
% $\displaystyle H \gets {\frac{1}{N}\sum_{i=1}^N \bm{q}_i \bm{q}_i'}$. 
% Find SVD of $H =U\Lambda V^T$\; 
% $R \gets {VU^T}, \quad \bm{t} \gets {\bm{p}'-R \bm{p}}$\;
% }
% \end{algorithm}
% Figure environment removed
%%%%%%%%%%%
\subsection{Gait Library}
\label{Subsec:GaitSynthesizer}
Gaits are synthesized as described in \Sec \ref{SubSec:GaitSynthesis}. For this research, five gaits are chosen - a rotation dominant gait $R$ and a translation dominant gait $T_1$ with its permutations $T_2$, $T_3$, and $T_4$. The limb actuation patterns are shown in \Fig \ref{Fig:GaitLibrary}a. %below:
% \begin{center}
% % Figure removed
% \end{center}
% The red indicate the actuated limb, while dotted black are the unactuated limbs. The numbers in bracket in front of each gait indicate the edge that is activated, e.g., `Gait G: [3 15 1]' implies that the simple cycle representing Gait G traverses edges $\{e_3,e_{15},e_1\}$ and only the $3,15,1$ elements of the gait vector $\mathbf{z}_G$ are non-zero. %
%%%%%%%%%

Each of the gaits are run for 120 gait cycles and the mean locomotion twist $\xi(\bm{p_i},\theta_i)$ is obtained that can be used by the path planner. The mean translation and rotation for each of these gaits is visually shown in \Fig \ref{Fig:GaitLibrary}. These plots highlight a key observation: the translation-dominant gaits $T_1$, $T_2$, $T_3$, and $T_4$ are offset from each other by 90 degrees as expected \textadded{(due to the limbs being positioned at 90 degree offsets)}. As the robot is fabricated to be rotationally symmetric, \textadded{we anticipated} that the twist magnitudes of these gaits would be identical; after all, they are the same gait \textadded{with actuation initiated by a different limb}. However, this is not true and highlights the sensitivity of soft robots to small manufacturing inaccuracies/non-uniformities. \textadded{While control parameters could potentially be adjusted to reduce these differences in behavior, it would be unlikely to eliminate them completely as the frictional effects appear to dominate.} Moreover, soft robots are sensitive to small changes in the environment (e.g., small bumps in the substrate) as suggested by the error bars in \Fig \ref{Fig:GaitLibrary}b. \textadded{The data-driven gait discovery and the path planning strategy accommodate these asymmetries and variations by treating each starting limb option as a distinct gait with individually (experimentally-) derived behaviors; feedback-based re-planning additionally serves to mitigate undesirable effects.}

%  % Figure environment removed
% % Figure environment removed


\subsection{Trajectory Planning}
\label{Subsec:TrajectoryPlanning}
Trajectory synthesis is computed on a binary image representation of an obstacle scenario. This input image is transcribed to a grid world cost map, $C(x, y): \mathbb{R}^2 \to \mathbb{R}$, where $\left(x, y\right)$ denote grid coordinates and $C(x, y)$ denotes the cost associated with each grid location. Obstacle locations are characterized by high costs, which fall off radially with distance. 
A tunable parameter governs how quickly cost decays and is used to effectively dilate obstacles.
Grid locations are classified as either \textit{occupied} or \textit{free} based on a pre-configured cost threshold. This results in a configuration space suitable for a point representation of the robot. Robot presence at any \textit{occupied} grid location constitutes an obstacle collision.
For a prescribed goal position $\bm{x}_{\rm goal} \in E(2)$, cost-to-go $C_{\rm go}(x, y; \bm{x}_{\rm goal}): \mathbb{R}^2 \to \mathbb{R}^+$ across the grid world is quickly computed using an expanding wavefront approach. The cost-to-go allows us to discern the relative value of potential trajectory destinations within the world.
Gait-based controlled trajectories are then designed within this grid world representation of the locomotion scenario. 

\smallskip \noindent{\bf Gait Models.}
Trajectory plans adhere to a rotate-then-translate motion paradigm. Synthesis entails searching for a sequence of these rotate-then-translate pairs, in order to move the robot from a starting pose, $g_0 \in SE(2)$, to a desired goal location, $\bm{x}_{\rm goal}$. This conceptually simplifies the trajectory planning problem; planning becomes an iterative process of: \textbf{(1)} `aiming'  (i.e. rotation), then \textbf{(2)} traveling in that direction (i.e. translation). This procedure terminates when the trajectory plan reaches  a pre-defined radius $\delta_{\rm goal}$ of the goal position  $\bm{x}_{\rm goal}$.

From the set of gaits that \MSoRo~ is able to accomplish, we select a subset to be used in trajectory planning and control. 
A single gait is selected to accomplish rotationally-dominant motion. We denote this gait as $R$; its behavior is characterized by a time-averaged body velocity twist $\xi^{\rm R} \in \mathfrak{se}(2)$, and a gait periodicity of $Q_{\rm R}$ seconds. 
The remaining $d$ gaits are characterized by translationally-dominant motion, and denoted as the set $\left\{ T_1, T_2, \ldots, T_d  \right\}$. Each translational gait $T_i$ exhibits a time-averaged body velocity twist $\xi^{\rm T_i} \in \mathfrak{se}(2)$ and gait periodicity $Q_{\rm T_i}$, where $i \in \left\{1 \ldots d \right\}$. 

% Figure environment removed
\smallskip \noindent{\bf Trajectory Synthesis.}
The planning strategy presented here employs a greedy breadth-first search through the space of possible \MSoRo~ gait sequences. The approach successively expands sets of neighboring, collision-free trajectory destinations that may be reached by a single rotation-translation gait sequence. The neighboring destination with the smallest cost-to-go is then selected for subsequent expansion. \Fig \ref{Fig:TrajPlanner} illustrates trajectory exploration and synthesis, through an obstacle-strewn environment. This strategy produced feasible trajectories, and the corresponding gait sequences needed to accomplish them, facilitating the robot's intelligent traversal through a variety of obstacle arrangements.

Beginning at an initial expansion pose $g^{\rm expand} \in SE(2)$, the planner first samples trajectory end points that may be reached by the rotational profile $\xi^{\rm R}$, over durations of $n_{\rm R} = 1 \ldots N_{\rm R}$ rotational gait periods. $N_{\rm R} \in \mathbb{Z}^{+}$ is a fixed limit and computed such that $\xi^{\rm R}_{\omega} \cdot N_{\rm R} < \frac{\pi}{2}$. Reachable robot poses, using the rotational gait $R$, are denoted $g^{\rm R}_{n_{\rm R}} \in SE(2)$ and expressed relative to the initial expansion pose $g^{\rm expand}$. These poses are checked for collisions; if a collision is identified, the pose is discarded and removed from consideration going forward. Beginning from each collision-free $g^{\rm R}_{n_{\rm R}}$, trajectory end points are forward sampled for each translational motion profile $\xi^{\rm T_i}$ where $i = 1 \ldots d$, and durations of $n_{\rm T_i} = 1 \ldots N_{\rm T}$ translational gait periods; $N_{\rm T} \in \mathbb{Z}^{+}$ is pre-configured and denotes the maximum number of consecutive cycles a translational gait may be run. The poses of these trajectory end points, relative to $g^{\rm R}_{n_{\rm R}}$, are denoted $g^{\rm T_ i}_{n_{\rm T_i}}$. Their poses, relative to the initial expansion pose, are computed as $g^{\rm R}_{n_{\rm R}} \cdot g^{\rm T_i}_{n_{\rm T_i}}$; their corresponding spatial poses are $g{\left(n_{\rm R},i,n_{\rm T_i}\right)} = g^{\rm expand} \cdot g^{\rm R}_{n_{\rm R}} \cdot g^{\rm T_i}_{n_{\rm T_i}}$. The trajectory end points, as well as sub-sampled points along each trajectory segment, are tested for collisions; a trajectory expansion is discarded if a collision is detected. Cost-to-go $C_{\rm go}$ is evaluated at the ($x, y)$ coordinates associated to each $g{\left(n_{\rm R},i,n_{\rm T_i}\right)} \in SE(2)$. The motion sequence, described by $\{ n_{\rm R},i,n_{\rm T_i}\}$, and the corresponding trajectory destination $g{\left(n_{\rm R},i,n_{\rm T_i}\right)}$ that are associated with the lowest cost-to-go, are selected as the optimal control and subsequent node to be expanded, respectively.
This procedure iterates until the selected pose $g{\left(n_{\rm R},i,n_{\rm T_i}\right)}$ falls within a threshold distance $\delta_{\rm goal}$ of the goal position $\bm{x}_{\rm goal}$.


\subsection{Path Recalculation}
As observed, the locomotion gaits have both rotation and translation associated with them. As the robot performs gait cycles and switches gaits, the pose error does not always monotonically increase. Generally, the position error reaches some maximum value and then decreases. This observation is illustrated in \Fig \ref{Fig:GatiSwitchingError}. Consequently, path recalculation should be performed when the position error exceeds a user-defined error threshold upon completion of a gait sequence and/or at user-defined intervals (i.e., every $n$ gait cycles). 
%Consequently, the path-recalculation is performed upon completion of a gait sequence and then comparing the position error with the user-defined threshold error.
% Figure environment removed


% % Figure environment removed
% % Figure environment removed
