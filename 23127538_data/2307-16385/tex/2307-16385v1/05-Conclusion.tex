% The research proposes a generic, closed-loop control framework for terrestrial soft robots. The experimental setup involves system integration of high-level (online path planning and offline gait synthesizer) control, low-level control (\MSoRo~actuation), and real-time pose estimation that uses parallel architecture to process visual feedback. To the best of our knowledge, this is the first instance of such system integration and experimental validation of closed loop path planning for motor-tendon actuated soft robots. {The path planning is performed using a lattice-based algorithm using the robot gaits as the inputs. The greedy breadth-first strategy generate trajectories comprise of rotate-then-translate gait pairs to iteratively solve the problem. The path is re-planned when the position error between the estimated and the executed path are above a pre-defined threshold.

For the first time, this research presents successful closed-loop path planning and obstacle avoidance of the four-limb motor-tendon actuated soft robot \MSoRo. The experiment uses real-time visual feedback from low-cost webcams to perform localization, while the lattice-based path planner generates collision-free trajectories using a greedy breadth-first approach. As soft robots are more sensitive to factors like  changes in the environment and manufacturing uncertainties, the locomotion gaits are synthesized using a data-driven environment-centric framework. Conceptually, this approach discretizes the factors dominating the environment-robot interaction and synthesizes the tracked motion of these interactions to find translation and rotation gaits. The path planner generates sequences of gaits that are pairs of rotation-then-translation. The synthesized gaits for the \MSoRo~have coupled translation and rotation. Consequently, the path is recalculated when the position error exceeds a threshold after completion of a gait sequence, or at a user-defined interval. The framework is validated on complex world scenarios with obstacles that require the robot to perform challenging maneuvers. The non-uniform nature of the surface/environment and the intentional slipping action of the robot limbs further complicate this challenge.

Future work of this research involves extending the framework to a larger gait library that allows for locomotion on different surfaces. Furthermore, work will be done to adapt the path planner to incorporate the probabilistic nature of locomotion gaits.


% \begin{itemize}
%     % \item this was a successful closed-loop, low-cost, generic planner.
%     % \item Re-planning cannot be purely based on translation error because of the coupled rotation/translation . 
%     % \item Because soft robots are more sensitive to changes in the environment, manufacturing defects, it is important to have a data-driven 
%     % \item sources of variation in the gait (tether, surface defects, actuation variance, nonlinear properties, not considering movements of transitions between gaits)
%     % \item robustness of method has been tested on a particularly difficult example: coupled rotation/translation, slipping, complicated geometry/friction
%     \item future works: adding probability into the planner
% \end{itemize}