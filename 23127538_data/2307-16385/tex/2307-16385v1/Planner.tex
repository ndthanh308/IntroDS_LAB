Trajectory synthesis is computed on a binary image representation of an obstacle scenario. This input image is transcribed to a grid world cost map, $C(x, y): \mathbb{R}^2 \to \mathbb{R}$, where $\left(x, y\right)$ denote grid coordinates and $C(x, y)$ denotes the cost associated with each grid location. Obstacle locations are characterized by high costs, which fall off radially with distance. 
A tunable parameter governs how quickly cost decays and is used to effectively dilate obstacles.
Grid locations are classified as either \textit{occupied} or \textit{free} based on a pre-configured cost threshold. This results in a configuration space suitable for a point representation of the robot. Robot presence at any \textit{occupied} grid location constitutes an obstacle collision.
For a prescribed goal position $\bm{x}_{\rm goal} \in E(2)$, cost-to-go $C_{\rm go}(x, y; \bm{x}_{\rm goal}): \mathbb{R}^2 \to \mathbb{R}^+$ across the grid world is quickly computed using an expanding wavefront approach. The cost-to-go allows us to discern the relative value of potential trajectory destinations within the world.
Gait-based controlled trajectories are then designed within this grid world representation of the locomotion scenario. 

\smallskip \noindent{\bf Gait Models.}
Trajectory plans adhere to a rotate-then-translate motion paradigm. Synthesis entails searching for a sequence of these rotate-then-translate pairs, in order to move the robot from a starting pose, $g_0 \in SE(2)$, to a desired goal location, $\bm{x}_{\rm goal}$. This conceptually simplifies the trajectory planning problem; planning becomes an iterative process of: \textbf{(1)} `aiming'  (i.e. rotation), then \textbf{(2)} traveling in that direction (i.e. translation). This procedure terminates when the trajectory plan reaches  a pre-defined radius $\delta_{\rm goal}$ of the goal position  $\bm{x}_{\rm goal}$.

From the set of gaits that \MSoRo~ is able to accomplish, we select a subset to be used in trajectory planning and control. 
A single gait is selected to accomplish rotationally-dominant motion. We denote this gait as $R$; its behavior is characterized by a time-averaged body velocity twist $\xi^{\rm R} \in \mathfrak{se}(2)$, and a gait periodicity of $Q_{\rm R}$ seconds. 
The remaining $d$ gaits are characterized by translationally-dominant motion, and denoted as the set $\left\{ T_1, T_2, \ldots, T_d  \right\}$. Each translational gait $T_i$ exhibits a time-averaged body velocity twist $\xi^{\rm T_i} \in \mathfrak{se}(2)$ and gait periodicity $Q_{\rm T_i}$, where $i \in \left\{1 \ldots d \right\}$. 

% Figure environment removed
\smallskip \noindent{\bf Trajectory Synthesis.}
The planning strategy presented here employs a greedy breadth-first search through the space of possible \MSoRo~ gait sequences. The approach successively expands sets of neighboring, collision-free trajectory destinations that may be reached by a single rotation-translation gait sequence. The neighboring destination with the smallest cost-to-go is then selected for subsequent expansion. \Fig \ref{Fig:TrajPlanner} illustrates trajectory exploration and synthesis, through an obstacle-strewn environment. This strategy produced feasible trajectories, and the corresponding gait sequences needed to accomplish them, facilitating the robot's intelligent traversal through a variety of obstacle arrangements.

Beginning at an initial expansion pose $g^{\rm expand} \in SE(2)$, the planner first samples trajectory end points that may be reached by the rotational profile $\xi^{\rm R}$, over durations of $n_{\rm R} = 1 \ldots N_{\rm R}$ rotational gait periods. $N_{\rm R} \in \mathbb{Z}^{+}$ is a fixed limit and computed such that $\xi^{\rm R}_{\omega} \cdot N_{\rm R} < \frac{\pi}{2}$. Reachable robot poses, using the rotational gait $R$, are denoted $g^{\rm R}_{n_{\rm R}} \in SE(2)$ and expressed relative to the initial expansion pose $g^{\rm expand}$. These poses are checked for collisions; if a collision is identified, the pose is discarded and removed from consideration going forward. Beginning from each collision-free $g^{\rm R}_{n_{\rm R}}$, trajectory end points are forward sampled for each translational motion profile $\xi^{\rm T_i}$ where $i = 1 \ldots d$, and durations of $n_{\rm T_i} = 1 \ldots N_{\rm T}$ translational gait periods; $N_{\rm T} \in \mathbb{Z}^{+}$ is pre-configured and denotes the maximum number of consecutive cycles a translational gait may be run. The poses of these trajectory end points, relative to $g^{\rm R}_{n_{\rm R}}$, are denoted $g^{\rm T_ i}_{n_{\rm T_i}}$. Their poses, relative to the initial expansion pose, are computed as $g^{\rm R}_{n_{\rm R}} \cdot g^{\rm T_i}_{n_{\rm T_i}}$; their corresponding spatial poses are $g{\left(n_{\rm R},i,n_{\rm T_i}\right)} = g^{\rm expand} \cdot g^{\rm R}_{n_{\rm R}} \cdot g^{\rm T_i}_{n_{\rm T_i}}$. The trajectory end points, as well as sub-sampled points along each trajectory segment, are tested for collisions; a trajectory expansion is discarded if a collision is detected. Cost-to-go $C_{\rm go}$ is evaluated at the ($x, y)$ coordinates associated to each $g{\left(n_{\rm R},i,n_{\rm T_i}\right)} \in SE(2)$. The motion sequence, described by $\{ n_{\rm R},i,n_{\rm T_i}\}$, and the corresponding trajectory destination $g{\left(n_{\rm R},i,n_{\rm T_i}\right)}$ that are associated with the lowest cost-to-go, are selected as the optimal control and subsequent node to be expanded, respectively.
This procedure iterates until the selected pose $g{\left(n_{\rm R},i,n_{\rm T_i}\right)}$ falls within a threshold distance $\delta_{\rm goal}$ of the goal position $\bm{x}_{\rm goal}$.
