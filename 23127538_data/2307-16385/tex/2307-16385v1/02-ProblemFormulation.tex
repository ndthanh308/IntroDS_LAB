\subsection{\MSoRo: \MSoRoLong}

\MSoRo~is a four-limb terrestrial soft robot  actuated using motor-tendon actuators (MTAs). The robot is powered using external power with a low-level controller. The high-level controller and path planner is located off-board on a desktop computer that communicates with the webcam for localization (described in \Sec \ref{Sec:Methodology}). The robot design is the result of topology optimization to allow six identical robots to reconfigure into a sphere\textadded{, whose details are more fully explored in }\cite{FreemanEtAl_SoRo[2021],FreemanEtAl_JCND[2023]_SoRoDesign}. As a result, the limbs are designed for complex geometrical curling and not optimized for any particular locomotion modes. \textadded{The emphasis is on implementing a path planning strategy for individual planar locomotion of the soft robot and does not address reconfigurability.}

Robot fabrication involves integration of soft material limbs, control and actuation payload (motors, electronics), and routing of the tendons through the limb as shown in \Fig \ref{Fig:robotcasting}. The modular fabrication process involves mixing two liquid silicone components (Smooth-On Dragon Skin Part A and B - Shore Hardness 20A) degassed in vacuo. The tendon paths are cast by threading a thick wire through the rigid 3D printed mold as shown in Fig. \ref{Fig:robotcasting} which is removed upon curing of the cast. The central hub is 3D printed with flexible filament (Shore Harness 85A, placed in the mold for casting; the casting is repeated for the other limbs. %
% Figure environment removed
Rapid curling and uncurling of the flexible limbs (\textadded{$450 \text{ ms}$}/transition) is achieved through motor-tendon actuation. Four DC motors with 3D printed PLA spools are placed in the hub and secured using zip ties. Teflon tubing is inserted into the individual fins of each limb to prevent tear caused by the difference in stiffness between the silicone and the fishing line tendon, \Fig \ref{Fig:robotcasting}. Finally, threaded fishing line attached to the spool is routed through each fin and anchored at the end with a fishing hook. A slip ring is incorporated into the tether connector to reduce any effects of built-up torsion in the tether. %Design parameters of the fins of each limb (separation, height, thickness, etc.) are experimentally determined to maximize curling while still permitting uncurling upon relaxing the motor.

% However, all possible motion primitives are quasistatic and all robot configurations are statically stable, satisfying the assumptions of the MFC framework. Hence, this robot can be used to highlight and validate this gait synthesis procedure as (1) simple actuation produces complex movement with coupled translation and rotation, (2) the stick-slip nature of the limb movement complicates traditional gait synthesis and increases the dependence of the robot behavior on the environment, and (3) the exploration of a diverse set of gaits can be used for future planned open-loop path planning. %
%%%%%%%%%%%%%%%%%%%%%%%%%%%%%%%%%%%%
\subsection{Gait Synthesis}
\label{SubSec:GaitSynthesis}
The gaits for \MSoRo~are synthesized (discovered) using an  environment-centric framework that discretizes the factors dominating the robot-environment interaction. The procedure of synthesizing locomotion gaits is described briefly.\footnote{The reader may refer to \cite{VikasEtAl_IROS[2015],FreemanEtAl_TRO[2023]} for detailed analysis, and to  \cite{bollobas_modern_2013} for the graph theory terminology and concepts used.}

\smallskip \noindent{\bf Environment-centric Framework.} Conceptually, locomotion results from optimization of forces acting at different parts of the body that ultimately effect change in inertia \cite{Radhakrishnan_PNAS[1998]}. In that context, \textit{robot states} are defined as discrete physical states (e.g., postures, shapes) where the forces acting on the robot body in each state are considerably different. % Concisely, they discretize the factors dominating the robot-environment interaction. (this is repeated a few sentences earlier)
\textit{Motion primitives} refer to the possible transitions between these robot states. These transitions result in motion of the robot (rotation and translation). A weighted digraph is effective in modeling the robot states, motion primitives, resulting motion, and their inter-dependencies. The robot states and motion primitives correspond to the digraph's $n$ vertices $V(G)$ and the $m$ directional edges $E(G)$, respectively. For \MSoRo, robot states correspond to permutations of the four limbs being curled (actuated) or uncurled (un-actuated) as illustrated in \Fig \ref{Fig:RobotStates}. For this robot, actuation is binary (on/ off) and all possible permutations (states) are statically stable. The weight associated with each edge $e_i$ is the resulting motion of the motion primitive, i.e., the translation $\bm{p}_i\in\Re^{2\times 1}$ and rotation $\theta_i$ measured in the coordinate system of the initial vertex of the edge. For this discussion, the edge weight $\bm{w}_i$ is modeled as a normal distribution with mean $\bm{\mu}_{i} \in \Re^{3\times 1}$ and covariance matrix ${\Sigma}_{i} \in \Re^{3 \times 3}$:

\begin{align} 
% \bm{w}_{i} = \vGaussian{\mu_{i}}{\Sigma_{i}} \label{Eqn:ArcWeight}
\bm{w}_i = \mathcal{N}\left(\bm{\mu}_i,\Sigma_i\right)
,  %
\bm{\mu}_i(e_i) = \begin{bmatrix}\bm{p}_i\\ \theta_i
\end{bmatrix},  
{\Sigma}_i(e_i) = \begin{bmatrix}
{\Sigma}_{pp} &\Sigma_{p\theta}\\
\Sigma_{\theta p} & \Sigma_{\theta\theta}
\end{bmatrix}.
\label{Eqn:ArcWeight}
\end{align}

In summary, we define the following matrices with elements corresponding to edges $e_i$.
\begin{center}
\begin{tabular}{|c|c|c|}
\hline
    & Description & Dim \\ \hline \hline
     $P(E)$& mean displacement matrix, $P(e_i)=\bm{p}_i$ & $\Re^{2\times m}$\\\hline
     $\Theta(E)$& mean rotation matrix, $\Theta(e_i)=\theta_i$ & $\Re^{1\times m}$\\\hline
     %
    \multirow{2}{*}{$S_p(E)$}& translation covariance trace matrix & \multirow{2}{*}{$\Re^{2\times m}$}\\
    & where $\displaystyle S_p(e_i) = \mathrm{tr}\left(\Sigma_{pp}(e_i)\right)$ &\\ \hline
    \multirow{2}{*}{$S_\theta(E)$} & rotation covariance matrix & \multirow{2}{*}{$\Re^{1\times m}$} \\ & where $S_\theta(e_i) = \Sigma_{\theta \theta}(e_i)$&\\  \hline
\end{tabular}
\end{center}

% Figure environment removed

\textit{Learning of the environment} is equivalent to learning the graph edge weights. Experimentally, this is achieved by traversing all the edges of the graph without repeating any and recording the resulting motion. This traversal sequence, referred to as the Euler cycle, is repeated multiple (five) times with randomized starting positions and path orders to learn the probabilistic weights of the graph.

\smallskip \noindent{\bf Translation and Rotation Gaits.}  %A closed walk consists of a sequence of vertices starting and ending at the same vertex, where consecutive vertices in the sequence are connected by a directed edge. A simple cycle is a closed walk where no vertices or directed edges are repeated, other than the start and the end vertex. 
Locomotion gaits are defined here as simple cycles that are transformation invariant. The transformation invariance principle implies that the distance and rotation of the robot are preserved irrespective of starting vertex as it traverses through all edges of the simple cycle. It has been proven that under this definition, there will exist two types of planar gaits: translation and rotation \cite{FreemanEtAl_TRO[2023]}. The former is the gait when the cumulative rotation of the simple cycle is zero. The latter corresponds to the simple cycle when translation of all the edges is zero. Hence, we individually optimize for these two type of gaits with different cost functions and constraints.

The \textit{gait library} comprises the synthesized translation and rotation gaits discovered using the discussed data-driven approach (learning of the graph and searching for optimal gaits). The cost functions, $J_t,J_\theta$ linearly weight the locomotion, variance and  gait length while assuming small rotations of the motion primitives.
\begin{equation}
\begin{gathered}
    J_t(\mathbf{z}) = % \bm{\alpha}_t^T\underbrace{P\mathbf{z}}_{\mathrm{translation}} + \beta_t \underbrace{S_{p}^T\mathbf{z}}_{\mathrm{variance}} + \gamma_t \underbrace{{1}_{1\times m}\mathbf{z}}_{\mathrm{length}}\\
    \left(\bm{\alpha}_t^T P + \beta_t S_{p} + \gamma_t 1_{1\times m} \right)\mathbf{z}\\
    J_\theta(\mathbf{z}) = \left({\alpha}_\theta \Theta + \beta_\theta S_{\theta} + \gamma_\theta 1_{1\times m} \right)\mathbf{z}
\end{gathered}
\label{Eqn:costfuncs}
\end{equation}
where $\{\bm{\alpha}_t,\beta_t,\gamma_t,\alpha_\theta,\beta_\theta,\gamma_\theta\}$ are the linear weights, and the binary vector $\mathbf{z}\in \{0,1\}^m$ is the mathematical representation of a gait. Consequently, the gait synthesis is formulated as a \underline{B}inary \underline{I}nteger \underline{L}inear Programming (BILP) optimization problem with linear constraints that can be solved using optimization solvers, e.g., MATLAB\textsuperscript{\textregistered}. The translation $\mathbf{z}_t$ and rotation $\mathbf{z}_\theta$ gaits are synthesized using
\begin{align}
\begin{gathered}
    \mathbf{z}_t = \min_{\mathbf{z}} J_t(\mathbf{z}) \quad \mathrm{s.t.} \quad\Theta \mathbf{z}\leq \varepsilon_\Theta\\
    \mathbf{z}_\theta = \min_{\mathbf{z}} J_\theta(\mathbf{z}) \mathrm{~s.t.~} |P\mathbf{z}| \leq \varepsilon_p \forall z_i=1\\
    \mathrm{Gait~constraints:} B\mathbf{z}=0, B^i\mathbf{z}\leq 1, z_i\in\{0,1\} \forall i,\\
    \nexists \mathbf{z}_1, \mathbf{z}_2 \mathrm{~s.t.~} \mathbf{z} = \mathbf{z}_1+\mathbf{z}_2,~B\mathbf{z}_1=B\mathbf{z}_2=0
\end{gathered}
\label{Eq:constraints}
\end{align}
where $B$ is the incidence matrix, $B^i$ is the positive elements of $B$, and the gait constraints mathematically ensure that vector $\mathbf{z}$ is a simple cycle. 

Once the gait library has been created and contains a synthesized rotation gait and translation gait, symmetry of the robot can be assumed to expand the library to improve path planning capabilities. The permutations of the translation gait w.r.t. each limb are considered for control purposes to achieve change in orientation of 90 degrees. Such behavior is observed in biology (e.g., brittle star) \cite{As_JrnlofExpBio[2012]} where the animal can change their leading limb to change the direction of their translation. While it is assumed that these permutations will have similar motion in four different directions, each of these gaits are tested to characterize their distinct twists.
% Figure environment removed
%%%%%%%%%%%%%%%%%%%%%%%%%%%%%%%%%%%%
\subsection{Problem Statement}
%Way-point navigation between two points with obstacle-avoidance with multiple (finite) gait library, where each gait has a probabilistic twist (mean and variance). 
% \hl{Soft mobile robots, such as \MSoRo, manipulate their deformable body composition to accomplish distinctly useful locomotion, relative to more traditional rigid-body robots. In particular, locomotion results from unique interactions between their soft body material composition and the surrounding environment. This allowance comes with distinct challenges; locomotion outcomes are heavily coupled to manufacturing and material variabilities, factors that often are difficult to control. Data-driven techniques are demonstrably effective for: (1) discovering useful gaits indigenous to a particular manufacturing instance, and (2) generate motion models that characterize these gaits for planning and control. Progression of these mobile robots toward autonomous field deployment entails an ability to both plan viable trajectories through an environment as well as accomplish some form of feedback-based control to track synthesized plans.  }

% \hl{We employ a data-driven approach to synthesize gaits and predictive motion models for \MSoRo; these inform a lattice-based planner that designs gait sequences, and accompanying locomotion trajectories, that move \MSoRo~ to prescribed goal locations within an obstacle-strewn environment. Feedback takes the form of trajectory re-planning, when tracking error exceeds a pre-defined threshold.}

Soft mobile robots, such as \MSoRo, manipulate their deformable body to accomplish distinctly useful locomotion, relative to more traditional rigid-body robots. In particular, locomotion results from unique interactions between their soft body material composition and the surrounding environment. This allowance comes with distinct challenges; locomotion outcomes are heavily coupled to manufacturing and material variabilities, factors that often are difficult to control. Data-driven techniques are demonstrably effective for: (1) discovering useful gaits indigenous to a particular manufacturing instance, and (2) generating motion models that characterize these gaits for planning and control. Progression of these mobile robots toward autonomous field deployment entails an ability to both plan viable trajectories through an environment as well as accomplish some form of feedback-based control to track synthesized plans.

We employ a data-driven approach to synthesize gaits and predictive motion models for \MSoRo; these inform a lattice-based planner whose gait sequence outputs, and accompanying locomotion trajectories, move \MSoRo~to prescribed goal locations within an obstacle-strewn environment. Feedback takes the form of trajectory re-planning when tracking error exceeds a pre-defined threshold.
