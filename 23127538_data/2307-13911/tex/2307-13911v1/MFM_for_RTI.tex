% This is file JFM2esam.tex
% first release v1.0, 20th October 1996
%       release v1.01, 29th October 1996
%       release v1.1, 25th June 1997
%       release v2.0, 27th July 2004 
%       release v3.0, 16th July 2014
%   (based on JFMsampl.tex v1.3 for LaTeX2.09)
% Copyright (C) 1996, 1997, 2014 Cambridge University Press

\documentclass[]{jfm}

\usepackage{graphicx}
%\usepackage{epstopdf,epsfig}
\usepackage{newtxtext}
\usepackage{newtxmath}
\usepackage{natbib}
\usepackage{hyperref}
\hypersetup{
    colorlinks = true,
    urlcolor   = blue,
    citecolor  = black,
}
\newtheorem{lemma}{Lemma}
\newtheorem{corollary}{Corollary}
\newcommand{\RomanNumeralCaps}[1]
\linenumbers

% \usepackage{amssymb,amsmath}
\usepackage{subcaption}
\usepackage{float}

% \shorttitle{Nonlocality of Mean Scalar Transport in 2D RTI using MFM}
% \shortauthor{D. L. O. Lavacot, J. Liu, H. Williams, B. E. Morgan, A. Mani}

% \title{Towards Construction of a Nonlocal RANS Model for Rayleigh-Taylor Instability Using the Macroscopic Forcing Method}
\title{Nonlocality of Mean Scalar Transport in Two-Dimensional Rayleigh-Taylor Instability Using the Macroscopic Forcing Method}

\author{D. L. O.-L. Lavacot\aff{1, 2}
  \corresp{\email{dlol@stanford.edu}},
  J. Liu\aff{1},
  H. Williams\aff{1, 2},
  B. E. Morgan\aff{2},
  \and A. Mani\aff{1}
  }

\affiliation{\aff{1}Department of Mechanical Engineering, Stanford University, Stanford, CA
\aff{2}Lawrence Livermore National Laboratory, Livermore, CA}

\begin{document}

\maketitle

\begin{abstract}
The importance of nonlocality of mean scalar transport in 2D Rayleigh-Taylor Instability (RTI) is investigated.
The Macroscopic Forcing Method (MFM) is utilized to measure spatio-temporal moments of the eddy diffusivity kernel representing passive scalar transport in the ensemble averaged fields.
Presented in this work are several studies assessing the importance of the higher-order moments of the eddy diffusivity, which contain information about nonlocality, in models for RTI.
First, it is demonstrated through a comparison of leading-order models that a purely local eddy diffusivity is insufficient in capturing the mean field evolution of the mass fraction in RTI.
Therefore, higher-order moments of the eddy diffusivity operator are not negligible.
Models are then constructed by utilizing the measured higher-order moments. 
It is demonstrated that an explicit model based on the Kramers-Moyal expansion of the eddy diffusivity kernel is insufficient. 
An implicit model construction that matches the measured moments is shown to offer improvements relative to the local model in a converging fashion.
\end{abstract}

% \begin{keywords}
% Authors should not enter keywords on the manuscript, as these must be chosen by the author during the online submission process and will then be added during the typesetting process (see http://journals.cambridge.org/data/\linebreak[3]relatedlink/jfm-\linebreak[3]keywords.pdf for the full list)
% \end{keywords}


\section{Introduction}
Rayleigh-Taylor Instability (RTI) is a phenomenon that occurs when a heavy fluid is accelerated into a light fluid.
Specifically, RTI occurs when the following are present: 1) a density gradient, 2) an acceleration (associated with the body force) in the direction opposite that of the density gradient, and 3) a perturbation at the interface of the two fluids.
RTI is present in many scientific and engineering applications such as supernovae \citep{gull1975} and inertial confinement fusion (ICF) \citep{zhou2017,lindl1995}.
In the case of ICF, RTI occurs when a perturbation forms between the outer heavy ablator and the inner light deuterium gas, which causes premature mixing in the target, thereby greatly reducing the efficiency of the process.
% causing perturbations between the less-dense outer layer and the more-dense inner layer. 
% The subsequent implosion accelerates the light fluid through the perturbed interface into the heavy fluid, resulting in RTI.
% This causes premature mixing in the target, thereby greatly reducing the efficiency of the process.
Thus, RTI is of great interest to scientists and engineers, especially in the context of ICF.

% For a good understanding of the role of RTI in ICF, simulations of the phenomenon are carried out during the iterative experiment design process.
During a typical ICF experiment design process, a Reynolds-Averaged Navier-Stokes (RANS) approach is often utilized to model the role of hydrodynamic instabilities such as RTI.
This is despite the fact that RTI can be more accurately predicted using high-fidelity methods like direct numerical simulations (DNS) \citep{youngs1994, cookdimotakis2001, cookzhou2002, cabotcook2006, mueschkeschilling2009} and large eddy simulations (LES) \citep{darlington2002, cookcabotmiller2004, cabot2006}.
Motivation for development of RANS models for various engineering applications like ICF can be understood by considering the computational cost of each method.
DNS requires resolution of the smallest turbulent scales, and LES the energy-containing scales, which are still much smaller than the macroscopic physics (i.e., averaged fields) of engineering interest.
On the other hand, by design, RANS must only resolve macroscopic scales, thereby requiring much lower computational cost.
Thus, RANS models are commonly used in engineering practice, especially in design optimization, where hundreds of thousands of simulations are often performed.
Such is especially the case in designing targets for ICF experiments \citep{casey2014, khan2016}.
Due to the utility of RANS in such applications, the need for predictive RANS models remains salient.

% RANS model design involves validation against higher-fidelity codes.
% LES is a popular candidate for model validation, since it is computationally less expensive than DNS but still provides accurate results.
%%% LES HISTORY %%%

Models of varying complexities have been applied to the RTI problem.
Among the most commonly-used types are two-equation models.
One such model is the ubiquitously-used $k$-$\varepsilon$ model \citep{launder1974}.
% which has been modified and tuned by several authors (\cite{gauthierbonnet1990},\cite{}) to be used in ICF-related RTI simulations.
Particularly, \citet{gauthierbonnet1990} introduced algebraic relations for some closures to satisfy realizability constraints for the model to be valid under the strong gradients of RTI.
Another popular two-equation model is the $k$-$L$ model; a version was introduced by \citet{dimontetipton2006} for RTI.
One appeal of the $k$-$L$ model is its inclusion of a transport equation for turbulence lengthscale $L$ (in place of the transport equation for $\varepsilon$ in $k$-$\varepsilon$) that can be related to the initial interface perturbation.
% The $k$-$L$ model involves transport equations for the turbulent kinetic energy $k$ and turbulent lengthscale $L$ and utilizes the self-similarity of turbulent RTI to set the model coefficients.
The self-similarity of turbulent RTI is leveraged to set the model coefficients.
% By construction, the $k$-$L$ model predicts growth rates of RTI mixing well, but it is inaccurate in other aspects of RTI flows: the $k$-$L$ model predicts a linear concentration profile and a quadratic turbulent kinetic energy profile, which are both inaccurate.
% and ultimately overpredicts mixing.

% This is due to its closure of the turbulent mass flux term in the species mass equation using the gradient diffusion approximation, which models the turbulent mass flux as a local eddy diffusivity times the scalar gradient.
% It has been shown that the turbulent mass flux can look very different from the scalar gradient, so a simple local coefficient may not be enough to scale the scalar gradient to model turbulent mass flux (\cite{}).

These two-equation models rely on the gradient diffusion approximation for the turbulent mass flux closure.
The gradient diffusion approximation rests on the assumption that turbulence transports quantities in a manner similar to Fickian diffusion.
Importantly, this approximation implies purely local dependence of the mean turbulent flux on the mean gradient, ignoring history effects and gradients at nearby points in space.
%is a local one, in that it relates the turbulent flux to the mean gradient by a local coefficient that depends only at the single point in space and time at which the turbulent scalar flux is measured.
However, this approximation may not be valid for mean scalar transport.
% Specifically, it has been shown that the turbulent mass flux can look very different from the scalar gradient, so a simple local coefficient may not be enough to scale the scalar gradient to model turbulent mass flux (\cite{}).
Specifically, the turbulent mass flux contains features that the gradient diffusion approximation cannot capture \citep{morgangreenough2015, denissen2014}, so a local coefficient may not be enough to scale the mean gradient to model turbulent mass flux.

Nonlocality in RTI has been studied in experiments and simulations.
\citet{clark1997} analyzed data from turbulent RTI experiments and compared the pressure-strain correlation and pressure production due to turbulent mass flux, suggesting spatial nonlocality of pressure effects.
% DNS studies by \cite{ristorcelliclark2004} suggest temporal nonlocality, as mixing layer growth depends largely on the initial perturbations; this is also examined and substantiated by experimental measurements of \cite{mueschke2006}.
DNS studies by \citet{ristorcelliclark2004} and experiments by \citet{mueschke2006} have also examined nonlocality of RTI in the context of two-point correlations.
% The Besnard-Harlow-Rauenzen (BHR) model (\cite{besnard1992}) attempts to address this issue by including a transport equation for the mass flux $a$ (along with equations for turbulence length scale, Reynolds stress, and density-specific volume correlation).
% However, BHR models require different sets of coefficients for RTI and Richtmeyer-Meshkov Instability (RMI)---which is shock-driven RTI---flows.
% The $k$-$L$-$a$ model (\cite{morgan2018}) was introduced as a unifying model that captures the growth rate of both RTI and RMI flows without needing separate coefficients for each.
% Similarly, inspired by the BHR model, the $k$-$L$-$a$ model (\cite{morganwickett2015}) replaces the gradient diffusion approximation for the turbulent mass flux in the $k$-$L$ model with a transport equation for $a$.
% While these works address the inadequacy of the gradient diffusion model, they do not directly address its inherent locality compared to the possible nonlocal nature of RTI.
Thus, the nonlocal nature of RTI is well-known, and work has been done to capture this nonlocality in models.
For example, two-point closures to account for nonlocality in RTI have been developed by several authors for RANS \citep{steinkamp_i_1999, steinkamp_ii_1999} and LES \citep{parishduraisamy2017}.
While these works attempt to address the effects of nonlocality in RTI, they do so without directly studying the form of the nonlocal operator.

Several authors have studied ways to directly measure the nonlocal eddy diffusivity in other canonical flows.
One such approach involves application of the Green's function.
The Green's function approach starts from analytical derivations of relations between turbulent fluxes and mean gradients, which was done by \citet{kraichnan1987}.
\cite{hamba1995} then introduced a reformulation of these relations appropriate for numerical computation of nonlocal eddy diffusivities, which has been applied to study channel flow \citep{hamba_2004} and, most recently, homogeneous isotropic turbulence (HIT) \citep{hamba2022}.
% \citet{kraichnan1987} derived an exact expression for the turbulent scalar flux using a Green's function approach, but it is difficult to solve in that form without approximation.
% Using a modified Green's function, \citet{hamba1995} determined an explicit and exact solution for the turbulent scalar flux.
% \citet{hamba_2004} used this method to compute nonlocal eddy diffusivity kernels for turbulent shear flow; \citet{hamba2005} extended the formulation to the Reynolds stress.

A different approach to determining nonlocal eddy diffusivities is the Macroscopic Forcing Method (MFM) by \citet{manipark2021}.
% \citet{manipark2021} demonstrated a numerical procedure for measuring closure operators called the Macroscopic Forcing Method (MFM).
In contrast to the Green's function approach, MFM is derived by considering arbitrary forcing added directly to the transport equations with its formulation rooted in linear algebra.
Additionally, MFM offers extensions to the Green's function approach by utilization of forcing functions that are not of the form of a Dirac delta.
Harmonic forcing has been utilized to derive analytical fits to nonlocal operators in Fourier space \citep{shirian2022}.
Additionally, forcing polynomial mean fields using inverse MFM offers a computationally economical path for determination of spatio-temporal moments of the eddy diffusivity operator in conjunction with the Kramers-Moyal expansion as opposed to computation of the moments from a full MFM analysis through post-processing \citep{manipark2021}.
% MFM is similar to the Green's function method for computing the eddy diffusivity, but the former offers more flexibility.
% That is, MFM allows for specification of forcing to compute closure operators as well as choice of mean field (through inverse MFM) to determine moments of eddy diffusivity.
Previous works using MFM have revealed nonlocal operators for a variety of flows, including HIT \citep{shirian2022}, channel flow \citep{park2021}, and a separated boundary layer \citep{parkliumani2021}.

It is with a motivation towards RANS model improvement that the present work seeks to understand nonlocality of closure operators governing turbulent scalar flux transport in RTI using MFM.
% In this work, MFM is used to measure the generalized eddy diffusivity described by \citet{romanof1985} and \citet{kraichnan1987}.
% Specifically, the moments of the nonlocal eddy diffusivity are measured, which have been studied in other flows in works including \citet{manipark2021}, \citet{liu2021}, \citet{kraichnan1987}, \citet{hamba1995}, and \citet{hamba_2004}.
% The present work seeks to apply MFM in order to directly measure from DNS nonlocality of the governing closure operators in RTI, particularly operators governing turbulent scalar flux transport.
Direct measurement of nonlocal closure operators has not yet been done in RTI.
This new knowledge of nonlocality of the mean scalar transport closure operator in RTI will aid in the development of improved RANS models used for studying ICF.

It must be noted that the focus of this work is on passive scalar mixing by RTI.
While important, coupling of the scalar field with the momentum field is not included in this analysis for simplicity.
The aim of this work is to provide a preliminary analysis of nonlocality of mean scalar transport in RTI that will inform future works with additional complexities, including coupling with momentum. 

% \hl{
% As MFM is a relatively novel and evolving technique, it must be emphasized that this work is intended to be a proof of principle.
% As it is presented in this work, MFM should be considered a diagnostic tool for assessing RANS models rather than one that produces models.
% The analyses presented here are not intended to supplant existing RANS models for RTI.
% Instead, they reveal what characteristics RANS models should attain to achieve more accurate predictions of RTI mixing.

% }



% organization:
% LES
% k-eps
% k-L
% BHR & k-L-a
% RSMs
% transition: none of reviewed models incorporate nonlocality. to authors' knowledge, literature is lacking in assessments of nonlocality in RTI.



% ***Following paragraph is under construction; will be rewritten as paper is reorganized***
This work is organized as follows.
First, an overview of RTI is covered briefly in \S \ref{sec:rti_physics}.
Next, \S \ref{sec:math_methods} gives an overview of the mathematical methods used in this work, including: 1) the generalized eddy diffusivity and its approximation via a Karmers-Moyal expansion; 2) MFM and its application for finding the eddy diffusivity moments; 3) self-similarity analysis.
Simulation details, including the governing equations and the computational approach, are given in \S \ref{sec:sim_details}.
Finally, results of several studies on the importance of higher-order eddy diffusivity moments as well as assessments of suggested model forms incorporating nonlocality of the eddy diffusivity for mean scalar transport in RTI are presented in \S \ref{sec:results}.
% Finally, model forms incorporating nonlocality of the eddy diffusivity for mean scalart transport are suggested and assessed in \S \ref{sec:improve_RANS}.

% First, in \S \ref{sec:ged}, we refer to a generalized eddy diffusivity kernel that is nonlocal in space and time and approximate the turbulent scalar flux with moments of this eddy diffusivity kernel.
% This approximation involves higher-order eddy diffusivity moments that characterize the nonlocality of the eddy diffusivity.
% We then measure these moments for RTI using the Macroscopic Forcing Method (MFM), described in \S \ref{sec:MFM}.
% Details about the simulations used in MFM are in \S \ref{sec:sim_details}, and the resulting measurements are presented in \S \ref{sec:edm}.
% Next, we assess the importance of these moments in the self-similar regime of RTI. The self-similarity analysis is outlined in \S \ref{sec:self-sim}.
% We present assessments via comparison of leading-order models (\S \ref{sec:lom}) and reconstructions of the turbulent scalar flux (\S \ref{sec:exp_mod}).
% Finally, we suggest methods for improving RANS models using what is learned from the higher-order moments. 

% \section{Physics of Rayleigh Taylor instability}
% 

\section{Brief overview of RTI}
\label{sec:rti_physics}
RTI is characterized by spikes (heavy fluid moving into light fluid) and bubbles (light fluid into heavy fluid).
The mixing widths of these spikes and bubbles are denoted as $h_s$ and $h_b$, respectively, and the mixing half-width is defined as $h=\frac{1}{2}(h_s+h_b)$.
% As RTI transitions to turbulence, the flow enters a self-similar regime.
The behaviors of these quantities in RTI are dependent on 
% $A$.
the Atwood number, defined as
\begin{align}
    A=\frac{\rho_H-\rho_L}{\rho_H+\rho_L}.
\end{align}
Here, $\rho_H$ and $\rho_L$ are the densities of the heavy and light fluids, respectively.
In the limit of low-Atwood number and late time, the mixing layer width is expected to reach a self-similar state of growth that scales quadratically with time:
% That is, the mixing widths $h_s$ and $h_b$ of the spikes and bubbles, respectively, become independent of time when scaled by $Agt^2$, where $t$ is time, $g$ is gravitational acceleration, and $A$ is the Atwood number:
% At low $A$, the half-width $h=\frac{1}{2}(h_s+h_b)$ is quadratic with time:
\begin{align}
    h\approx\alpha Agt^2,
\end{align}
where $\alpha$ is the mixing width growth rate.
The mixing width growth rate can also be viewed as the net mass flux through the midplane \citep{cookcabotmiller2004}.
In this case, $\alpha$ can also be written as
\begin{align}
    \alpha = \frac{\dot{h}^2}{4Agh}.
\end{align}
In the limit of self-similarity, these two definitions of $\alpha$ are expected to converge to the same value.

In a simulation, $h$ can be measured as
\begin{align}
    h \equiv 4\int\overline{Y_H\left(1-Y_H\right)}dy,
\label{eq:h_meas}
\end{align}
where $Y_H$ is the mass fraction of the heavy fluid (therefore, $Y_L=1-Y_H$ is the mass fraction of the light fluid).
% Alternatively, the mixing width can be defined as the thickness of fluid in which the species are perfectly homogenized (\cite{cabotcook2006}, \cite{morgan2017}):
An alternative definition used in works such as \citet{cabotcook2006} and \citet{morgan2017} is
\begin{align}
    h_\text{hom} \equiv 4\int\overline{Y_H}\left(1-\overline{Y_H}\right)dy.
    \label{eq:h_hom}
\end{align}
This definition is particularly useful, since it allows $h$ to be determined solely based on the RANS field.
That is, there is no closure problem in determining $h$ with this definition.
Thus, this is the $h$ reported in this work.

From these two definitions, a mixedness parameter $\phi$ can be defined, which can be interpreted as the ratio of mixed to entrained fluid \citep{youngs1994, morgan2017}:
\begin{align}
    \phi \equiv \frac{h}{h_\text{hom}}=1-4\frac{\int\overline{Y_H'Y_H'}dy}{h_\text{hom}}.
\end{align}
% When $\phi$ converges to a steady-state value, self-similarity has been achieved.
In the limit of self-similarity, $\phi$ is expected to approach a steady-state value.

A metric for turbulent transition is the Taylor Reynolds number:
\begin{align}
    Re_T = \frac{k^{1/2}\lambda}{\nu},
\end{align}
where $k=\frac{1}{2}\overline{u_i'u_i'}$ is the turbulence kinetic energy, and $\lambda$ is the effective Taylor microscale, approximated by
\begin{align}
    \lambda = \sqrt{\frac{10\nu L}{k^{1/2}}}.
    \label{eq:ReT}
\end{align}
Here, the turbulent lengthscale $L$ can be approximated as $\frac{1}{5}$ the mixing layer width \citep{morgan2017}.
The large-scale Reynolds number can also be examined \citep{cabotcook2006}:
\begin{align}
    Re_L = \frac{h_{99}\dot{h}_{99}}{\nu},
\end{align}
where $h_{99}$ is the mixing width based on $1\%$-$99\%$ mass fraction.
\citet{dimotakis2000} determined that the criterion for turbulent transition is when $Re_T>100$ or $Re_L>10,000$.


\section{Mathematical methods}
\label{sec:math_methods}

\subsection{Model problem}

In this work, a two-dimensional (2D), nonreacting flow with two species---a heavy fluid over a light fluid---is considered, with gravity pointing in the negative $y$-direction. 
It must be noted that the behavior of 2D RTI is significantly different from three-dimensional (3D) RTI, the latter of which is more relevant to problems of engineering interest.
It is well known that while 2D RTI is unsteady and chaotic, it is not strictly turbulent, since turbulence is a characteristic of 3D flows.
In addition, 2D RTI has a faster late-time growth rate, develops larger structures, and is ultimately less well-mixed.
These differences have been studied in RTI by \citet{cabot2006} and \citet{young_tufo_dubey_rosner_2001} and in Richtmeyer-Meshkov instability by \citet{olson2014}.

For this study, 2D RTI is chosen as the model problem instead of 3D RTI, since it is a good simplified setting for understanding nonlocality in RTI through the lens of MFM.
Specifically, 2D RTI simulations are much less computationally expensive than those of 3D RTI, and MFM requires many simulations to attain statistical convergence.
Thus, 2D RTI remains the focus of this work, with the hope that the understanding of nonlocality in this flow could be extended to nonlocality in 3D RTI.

In this 2D problem, $x$ is the homogenous direction.
In addition, there is no surface tension, the Atwood and Mach ($Ma$) numbers are finite but small, and the Peclet ($Pe$) number is finite but large.

\subsection{Generalized eddy diffusivity and higher-order moments}
In this work, the effect of nonlocality on mean scalar transport is of interest, so analysis begins with the scalar transport equation under the assumption of incompressibility:
\begin{align}
    \frac{\partial Y_H}{\partial t} + \nabla\cdot(\mathbf{u}Y_H) = D_H\nabla^2Y_H,
    \label{eq:ste}
\end{align}
where $\mathbf{u}$ is the velocity vector and $D_H$ is the molecular diffusivity of the heavy fluid.

After Reynolds decomposition and averaging and assumptions of large $Pe$ and small $A$ (which implies small mean velocity), this becomes
\begin{align}
    \frac{\partial \overline{Y_H}}{\partial t} = -\frac{\partial\overline{v'Y_H'}}{\partial y}
    \label{eq:ste_avg}.
\end{align}
Here, $\overline{*}$ denotes averaging over ensembles and the homogenous direction $x$.
% We arrive at this simplified equation due to small $Ma$ and large $Pe$.
The term $\overline{v'Y_H'}$ is the turbulent scalar flux, and this is the unclosed term that needs to be modeled.

As mentioned previously, one reason the gradient diffusion approximation used to model this term is inaccurate is that it assumes locality of the eddy diffusivity.
% We would like to remove this assumption completely and assess whether a nonlocal eddy diffusivity instead would result in a more predictive model.
This assumption can be removed by instead considering a generalized eddy diffusivity that is nonlocal in space and time, as demonstrated by \citet{romanof1985} and  \citet{kraichnan1987}.
% \begin{align}
%     -\overline{u_i'c'}(\mathbf{x}) = \int D_{ij}(\mathbf{x},\mathbf{y}) \left.\frac{\partial \overline{c}}{\partial x_j}\right|_\mathbf{y} \mathbf{dy}
%     \label{eq:gen_eddy_diff}
% \end{align}
% Here, $\mathbf{x}$ is the spatial and temporal location where the turbulent scalar flux is measured, and $\mathbf{y}$ is all spatial and temporal locations in the domain.
For 2-D RTI, such a model reduces to
\begin{align}
    -\overline{v'Y_H'}(y,t) = \int\int D(y,y',t,t') \left.\frac{\partial \overline{Y_H}}{\partial y}\right|_{y',t'} dy' dt'
    \label{eq:gen_eddy_diff_2d}.
\end{align}
Here, $y$ is the spatial coordinate in averaged space and $t$ is the time at which the turbulent scalar flux is measured,  $y'$ is all points in averaged space, and $t'$ is all points in time.

The eddy diffusivity kernel can be approximated by Taylor-series-expanding the scalar gradient locally about $y$ and $t$, which results in the following Kramers-Moyal-like expansion for the turbulent scalar flux as done by \citet{kraichnan1987}, \citet{hamba1995}, and \citet{hamba_2004}:
\begin{align}
    -\overline{v'Y_H'}(y,t) = D^{00}\frac{\partial \overline{Y_H}}{\partial y} +
    D^{10}\frac{\partial^2 \overline{Y_H}}{\partial y^2} +
    D^{01}\frac{\partial^2 \overline{Y_H}}{\partial t \partial y} +
    D^{20}\frac{\partial^3 \overline{Y_H}}{\partial y^3} \dots
    \label{eq:explicit}
\end{align}
\begin{align}
    D^{00} &= \int \int D(y,y',t,t')dy' dt', 
    \\
    D^{10} &= \int \int (y'-y)D(y,y',t,t')dy' dt',
    \\
    D^{01} &= \int \int (t'-t)D(y,y',t,t')dy' dt',
    \\
    D^{20} &= \int \int \frac{(y'-y)^2}{2}D(y,y',t,t')dy' dt'.
\end{align}
Here, $D^{mn}$ are the eddy diffusivity moments; the first index, $m$, denotes order in space, while the second, $n$, denotes order in time. 
This is the form presented in \citet{manipark2021} and \citet{liu2021}.
% $D^{00}$ is purely local, while the higher-order moments characterize the nonlocality of the eddy diffusivity kernel.

When the eddy diffusivity kernel is purely local, 
\begin{align}
    D(y,y',t,t')=D^{00}\delta(y-y')\delta(t-t').
\end{align}
In this case, $D^{00}$ is the only surviving moment, while all higher-order moments in space and time are zero.
Any non-zero higher-order moment therefore characterizes the nonlocality of the eddy diffusivity kernel.
Thus, this expansion implies explicitly a model form for the turbulent scalar flux that incorporates nonlocality of the eddy diffusivity.
Truncating the expansion provides an approximation of $\overline{v'Y_H'}$ but with the caveat that the expansion may not converge.
This will be discussed in more detail later in \S \ref{sec:exp_model_form}.

Each $D^{mn}$ provides more information about the eddy diffusivity kernel with increasing order.
For example, $D^{00}$ represents the volume of the kernel in space-time. 
The coefficient corresponding to one higher-order in space, $D^{10}$, provides information about the centroid of the kernel in space.  
$D^{20}$ contains information about the moment of inertia of the kernel in space, $D^{01}$ contains information about the centroid of the kernel in time, and so on.

\subsection{The Macroscopic Forcing Method}
% \begin{itemize}
%     \item MFM details: GMT, donor/receiver
% \end{itemize}

% Figure environment removed

MFM is a method for numerically determining closure operators in turbulent flows \citep{manipark2021}.
Much like a rheometer measures the molecular viscosity of a fluid by imposing a shear force on the flow, MFM forces the transport equation in a turbulent flow and extracts the closure operator from its response. 

Specifically, MFM can be used to determine the RANS closure operator, as shown in the pipeline diagram in figure \ref{fig:MFM_diagram}.
In MFM, two simulations are run at once: the donor and the receiver simulation.
% In this work, the receiver simulation numerically solves the scalar transport equation (Eq. \ref{eq:ste}).
In this work, the donor simulation numerically solves the multicomponent Navier-Stokes equations in equations \ref{eq:gov_eq1} - \ref{eq:gov_eq2}.
% \begin{gather}
%     \frac{\partial u_i}{\partial t}
%     + \frac{\partial u_ju_i}{\partial x_j}
%     = -\frac{\partial q}{\partial x_i} + \frac{1}{\text{Re}}\frac{\partial^2u_i}{\partial x_j\partial x_j} + r_i,
%     \label{eq:NS}\\
%     \frac{\partial u_j}{\partial x_j} = 0
% \end{gather}
The receiver simulation ``receives" $u_i$ from the donor simulation and uses it to solve the scalar transport equation with a forcing $s$:
\begin{align}
    \rho\frac{\partial Y_H}{\partial t} + \rho \frac{\partial}{\partial x_i}\left(u_iY_H\right)= \frac{\partial}{\partial x_i}\left(\rho D_H\frac{\partial}{\partial x_i}Y_H\right) + s.
    \label{eq:receiver}
\end{align}
Ultimately, forcings on the receiver simulation effect a response from the flow, and measuring this response allows for determination of the eddy diffusivity.
For details, see \citet{manipark2021}.

In actuality, the Inverse Macroscopic Forcing Method (IMFM) is used to determine eddy diffusivity moments.
That is, instead of the forcings being chosen, certain mean mass fraction fields are chosen.
The forcing needed to maintain the chosen $\overline{Y_H}$ is determined implicitly along the process and is not directly used in the analysis.
% Each $D^{mn}$ can be probed depending on the choice of mean mass fraction field.
As an illustration, the measurement of $D^{00}$ can be considered.
According to equation \ref{eq:explicit}, choosing $\overline{Y_H}=y$ (for $y$ between $-1/2$ and $1/2$) results in $\frac{\partial\overline{Y_H}}{\partial y}=1$, and all other higher-order derivatives are zero.
Thus, choosing this $\overline{Y_H}$ results in the measurement $-\overline{v'Y_H'}=D^{00}$.

Measurement of higher-order moments involves similar choices of $\overline{Y_H}$ but requires information from lower order moments.
For example, measuring $D^{10}$ involves choosing $\overline{Y_H}=y^2$, which results in $-\overline{v'Y_H'}=yD^{00}+D^{10}$.
Here, $D^{00}$ comes from the simulation using $\overline{Y_H}=y$.
Thus, $D^{10}$ is computed by subtracting $yD^{00}$ from the $\overline{v'Y_H'}$ measurement from the simulation using $\overline{Y_H}=y^2$.

Specifically, the following desired mean mass fractions are used for each moment for $y$ between $-1/2$ and $1/2$:
\begin{align}
    \overline{Y_H}&=y \Rightarrow D^{00}\label{eq:des_YH_1},\\
    \overline{Y_H}&=\frac{1}{2}y^2 \Rightarrow D^{10},\\
    \overline{Y_H}&=yt \Rightarrow D^{01},\\
    \overline{Y_H}&=\frac{1}{6}y^3+\frac{1}{48} \Rightarrow D^{20}\label{eq:des_YH_2}.
\end{align}
% Note that the choices for $\overline{Y_H}$ are designed so that they are spatially symmetric about $y=0$.
From these $\overline{Y_H}$, the needed forcing in each timestep is numerically determined:
\begin{align}
    s^k = \frac{\overline{Y_H}_\text{desired}^k - \overline{Y_H}^{k-1}}{\Delta t}
\end{align}
where the superscript $k$ denotes the timestep number, $\overline{Y_H}_\text{desired}$ is the mean mass fraction desired as outlined in equations \ref{eq:des_YH_1} - \ref{eq:des_YH_2}, and $\Delta t$ is the timestep size.

To determine $D^{00}$, $D^{10}$, $D^{01}$, and $D^{20}$, four separate simulations are needed.
For each of these simulations, the moments can be calculated using measurements of the turbulent scalar flux as follows:
\begin{align}
    D^{00}&=F^{00},\\
    D^{10}&=F^{10}-yD^{00},\\
    D^{01}&=F^{01}-tD^{00},\\
    D^{20}&=F^{20}-yD^{10}-\frac{1}{2}y^2D^{00}.
\end{align}
% \begin{align}
%     D^{00}&=\left.\overline{v'Y_H'}\right|_{\overline{Y_H}=y}\\
%     D^{10}&=\left.\overline{v'Y_H'}\right|_{\overline{Y_H}=\left(y-\frac{1}{2}\right)^2}-\left(y-\frac{1}{2}\right)D^{00}\\
%     D^{01}&=\left.\overline{v'Y_H'}\right|_{\overline{Y_H}=yt}-tD^{00}\\
%     D^{20}&=\left.\overline{v'Y_H'}\right|_{\overline{Y_H}=\frac{1}{6}\left(y-\frac{1}{2}\right)^3}-\left(y-\frac{1}{2}\right)D^{10}-\left(\frac{1}{2}y-\frac{1}{2}\right)^2D^{00}
% \end{align}
where $F^{mn}$ denotes the $-\overline{v'Y_H'}$ measured from the receiver simulation using the forcing corresponding the moment $D^{mn}$.

% It is useful to consider the physical meaning of each moment.
% $D^{00}$ is the integral of the eddy diffusivity kernel; $D^{10}$...

\subsection{Self-similarity analysis}
% \begin{itemize}
%     \item Similarity coordinate eta
%     \item Normalization of D’s \& <v’YH’> \& self-similar collapse --> APPENDIX
%     \begin{itemize}
%         \item Based on DNS measurements
%         \item Emphasize difference bet. kL and MFM/DNS normalizations
%     \end{itemize}
%     \item Transformation of explicit model into self similar coord.
%     \begin{itemize}
%         \item MMI transformation in MMI section later
%     \end{itemize}
%     \item Algebraic fits: comment in main body that fitting is done, explained in APPENDIX; no additional notation to keep text clean; potential effects in sensitivity analysis (as seen in plots in main body, but details in Appendix)
%     \begin{itemize}
%         \item Explanation: to not have to rely on noisy measured data
%     \end{itemize}
% \end{itemize}

We perform our analysis in the self-similar regime.
First, we define a self-similar coordinate:
\begin{align}
    \eta = \frac{y}{h(t)},
\end{align}
so that $\overline{Y_H}$ is only a function of $\eta$.
% \begin{align}
%     \overline{Y_H}=f(\eta).
% \end{align}
Note that $\eta$ requires a definition of $h(t)$.
From the previous discussion on the self-similarity of RTI, an appropriate definition is $h(t)=\alpha Agt^2$.

% Thus, we transform equation \ref{eq:ste_avg} with the explicit model for the turbulent scalar flux (as defined in equation \ref{eq:explicit}):
% \begin{align}
%     2\eta f'(\eta)=\frac{d}{d\eta}\left[\left(\widehat{D^{00}}+\widehat{D^{01}}\right)f'(\eta) + \left(\eta\widehat{D^{01}}+\widehat{D^{10}}\right)f''(\eta)+\widehat{D^{20}}f'''(\eta)\right].
% \end{align}
% Here, $\widehat{*}$ denotes normalized quantities.
% These normalizations are determined through self-similarity analysis; details can be found in the Appendix.
Through self-similar analysis of equation \ref{eq:explicit}, the eddy diffusivity moments and turbulent scalar flux can be normalized.
Details of this process can be found in the Appendix.
The resulting normalizations are
\begin{align}
    \widehat{\overline{v'Y_H'}}=&\frac{\overline{v'Y_H'}}{\alpha Agt}\label{eq:nondim1},\\
    \widehat{D^{00}}=&\frac{D^{00}}{\alpha^2 A^2g^2t^3},\\
    \widehat{D^{10}}=&\frac{D^{10}}{\alpha^3 A^3g^3t^5},\\
    \widehat{D^{01}}=&\frac{D^{01}}{\alpha^2 A^2g^2t^4},\\
    \widehat{D^{20}}=&\frac{D^{20}}{\alpha^4 A^4g^4t^7}\label{eq:nondim2}.
\end{align}

\subsection{Algebraic fit to $h$}
% Recall that MFM can precisely determine the eddy diffusivity moments through DNS.
% In MFM, multiple DNS (with different initial conditions) are run for each moment determination and averaged to achieve statistical convergence.
% With a reasonable number of realizations, the MFM measurements will still have some statistical noise that can impede the analysis.
% All measurements of quantities from our simulations carry some statistical error that impede the analysis.
% To remove the effect of this statistical error, we construct algebraic fits to the quantities.
% Details on fits of all quantities are included in the Appendix, but in this section, we specifically discuss the fit for $h(t)$.

% In our simulations (both DNS and RANS), $h$ is computed in each timestep using Eq. \ref{eq:h_meas}.
Recall that $h(t)=\alpha Agt^2$ is used in the self-similarity analysis.
This is valid only for late time, so the subsequent analyses in this work are all done in this self-similar timeframe.
Usually, $\alpha$ can be determined from $\frac{h(t)}{Agt^2}$, where $h(t)$ is computed from the simulation via equation \ref{eq:h_hom}.
However, due to the convergence and statistical errors as well as the existence of a virtual time origin, $\alpha Agt^2$ is not a good representation of $h(t)$ measured in the DNS.
% Thus, this computed $\alpha$ is not used in this analysis.
Instead, a fitting coefficient $\alpha^*$ and virtual time origin $t^*$ are determined to make a shifted quadratic fit to $h(t)$ from the simulation:
\begin{align}
    h_\text{fit}(t)=\alpha^*Ag(t-t^*)^2.
\end{align}
With this fit, the normalizations of the turbulent scalar flux and moments become
\begin{align}
    \widehat{\overline{v'Y_H'}}=&\frac{\overline{v'Y_H'}}{\alpha^* Ag(t-t^*)}\label{eq:nondim11},\\
    \widehat{D^{00}}=&\frac{D^{00}}{{\alpha^*}^2 A^2g^2(t-t^*)^3},\label{eq:nondimD00}\\
    \widehat{D^{10}}=&\frac{D^{10}}{{\alpha^*}^3 A^3g^3(t-t^*)^5},\\
    \widehat{D^{01}}=&\frac{D^{01}}{{\alpha^*}^2 A^2g^2(t-t^*)^4},\\
    \widehat{D^{20}}=&\frac{D^{20}}{{\alpha^*}^4 A^4g^4(t-t^*)^7}\label{eq:nondim21}.
\end{align}

For exact self-similarity, plots of the measured $\widehat{D^{mn}}$ against $\eta$ must be independent of time.
This expectation sets a criterion to assess the extent to which ideal self-similarity is achieved.
Plots and assessment of the self-similar collapse of the measurements presented in this work are in the Appendix.

% Note that $\widehat{h(t)}$ is different for each different kind of simulation.
% For example, the DNS $\widehat{h(t)}$ is different from the $k$-$L$ RANS model $\widehat{h(t)}$. 
% The biggest difference between simulations is $t_0$; $\alpha^*$ remains about the same between simulations, and it is close to the theoretical $\alpha$.
% We have detailed these points in order to emphasize that the correct $\widehat{h(t)}$ must be used in normalizing quantities measured in each simulation to properly compare them.


\section{Simulation details}
\label{sec:sim_details}
% \begin{itemize}
%     \item Description of Ares code
%     \item Examine the DNS a bit on its own to establish confidence
%     \begin{itemize}
%         \item Grashof number = 1 should keep us in the “DNS” regime
%         \item Show that it reaches a self-similar state
%         \item Report steady-state value of “alpha”
%     \end{itemize}
% \end{itemize}

\subsection{Governing equations}
The governing equations solved in this work are the compressible multicomponent Navier-Stokes equations, which involve equations for continuity, diffusion of mass fraction $Y_\alpha$ of species $\alpha$ (characterized by its binary molecular diffusivity $D_\alpha$), momentum transport, and transport of specific internal energy $e$:
\begin{align}
    \frac{D\rho}{Dt}&=-\rho\frac{\partial u_i}{\partial x_i}\label{eq:gov_eq1},\\
    \rho\frac{DY_\alpha}{Dt}&=\frac{\partial }{\partial x_i}\left(\rho D_\alpha\frac{\partial Y_\alpha}{\partial x_i}\right),\\
    \rho\frac{Du_j}{Dt}&=-\frac{\partial }{\partial x_i}\left(p\delta_{ij}+\sigma_{ij}\right)+\rho g_j,\\
    \rho\frac{De}{Dt}&=-p\frac{\partial u_i}{\partial x_i}+\frac{\partial }{\partial x_i}\left(u_i\sigma_{ij}-q_j\right)\label{eq:gov_eq2}.
\end{align}
Here, $\frac{D}{Dt}$ is the material derivative $\frac{\partial}{\partial t} + u_i\frac{\partial}{\partial x_i}$, $\rho$ is density, $u$ is velocity, and $p$ is pressure.
The viscous stress tensor $\sigma_{ij}$ and heat flux vector $q_j$ are respectively defined as
\begin{align}
    \sigma_{ij} &= \mu\left(\frac{\partial u_i}{\partial x_j}+\frac{\partial u_j}{\partial x_i}\right)-\mu\frac{2}{3}\frac{\partial u_k}{\partial x_k}\delta_{ij},\\
    q_j &= -\kappa\frac{\partial T}{\partial x_j} - \sum^N_{\alpha=1}h_\alpha\rho D_\alpha\frac{\partial Y_\alpha}{\partial x_j}.
\end{align}
Here, $\mu$ is the dynamic viscosity, $\kappa$ is the thermal conductivity, $T$ is temperature, and $h_\alpha$ is the specific enthalpy of species $\alpha$.

Component pressures and temperatures of each species are determined using ideal gas equations of state.
Under the assumption of pressure and temperature equilibrium, an iterative process is performed to determine volume fractions $v_\alpha$ that allow for computation of partial densities and energies.
% \begin{align}
%     p_\alpha = \left(\gamma_\alpha-1\right)\rho_\alpha e_\alpha,
%     \label{eq:palpha}
% \end{align}
% where $\gamma_\alpha$ is the ratio of specific heat of species $\alpha$ and $e_\alpha$ is the specific energy of species $\alpha$.
% % Total pressure and specific energy are sums of partial quantities:
% % \begin{align}
% %     p&=\sum^N_{\alpha=1}\nu_\alpha p_\alpha,\\
% %     e&=\sum^N_{\alpha=1}Y_\alpha e_\alpha,
% % \end{align}
% % where $\nu_\alpha$ is the volume fraction of species $\alpha$.
% To obtain volume fractions $v_\alpha$, an iterative process is performed.
% For each guess of $v_\alpha$, component density is computed as
% \begin{align}
%     \rho_\alpha=\frac{Y_\alpha\rho}{v_\alpha}
% \end{align}
% Equilibrium pressure is assumed ($p$=$p_\alpha$), and $p$ is determined by substituting equation \ref{eq:palpha} into the following:
% \begin{align}
%     e&=\sum^N_{\alpha=1}Y_\alpha e_\alpha.
% \end{align}
% Finally, total pressure is determined:
% \begin{align}
%     p&=\sum^N_{\alpha=1}\nu_\alpha p_\alpha.
% \end{align}
More details on the hydrodynamics equations and computation of component quantities can be found in \cite{morganolson2018}.

Finally, total pressure is determined as the weighted sum of component pressures:
\begin{align}
    p&=\sum^N_{\alpha=1}v_\alpha p_\alpha.
\end{align}

\subsection{Computational approach}

Simulations for 2D RTI are run using the Ares code, a hydrodynamics solver developed at Lawrence Livermore National Laboratory (LLNL) \citep{morgangreenough2015, bender2021}.
Ares employs an arbitrary Lagrangian-Eulerian (ALE) method based on the one by \citet{sharp_barton_1981}, in which the governing equations (equations \ref{eq:gov_eq1} to \ref{eq:gov_eq2}) are solved in a Lagrangian frame and then remapped to an Eulerian mesh through a second-order scheme.
The spatial discretization is a second-order non-dissipative finite element method, and time advancement is a second-order explicit predictor-corrector scheme.

The Reynolds number (more specifically, the kinematic viscosity $\nu$) is set through a numerical Grashof number, such that
\begin{align}
    \nu = \sqrt{ \frac{-2gA\Delta^3}{Gr}}.
\end{align}
Here, $\Delta$ is the grid spacing; in the simulations, a uniform mesh is used, and $\Delta=\Delta x=\Delta y$.
To ensure that the unsteady structures are properly resolved and for the simulation to appropriately be considered a DNS, $Gr$ should be kept small.
A $Gr$ that is too large results in a simulation with dissipation dominated by numerics rather than the physics.
\citet{morganblack2020} found that past $Gr\approx12$ in the Ares code, numerical diffusivity dominates molecular diffusivity.
For our simulations, we use $A=0.05$ and $Gr=1$, the latter of which is in line with the DNS by \citet{cabotcook2006}.

The Mach number, $Ma=\frac{u}{c}$, where $c$ is the speed of sound, characterizes compressibility effects of the flow.
$Ma$ is set by the specific heat ratio $\gamma$, which is $5/3$ in the simulations im this work.
The maximum $Ma$ is measured at the last timestep to be approximately $0.03$, which is ascertained to be small enough to assume incompressibility.

The Peclet number $Pe$ characterizes the advection versus diffusion rate and is defined as $ReSc$, where $Sc$ is the Schmidt number.
Here, a $Pe_L$ and a $Pe_T$ are reported, which use a large-scale $Re_L$ and the Taylor Reynolds number $Re_T$, respectively.
In the presented simulations, $Sc=1$.
The two $Pe$ are computed in post-processing: $Pe_L$ is approximately $8,000$, and $Pe_T$ is approximately $54$.
Both are below the criterion established by \citet{dimotakis2000}, suggesting that the simulated flow is transitional or pre-transitional.
% may not be fully turbulent.
% For the purposes of this work, which is mainly intended to be a proof of concept of MFM for RTI, it is acceptable that the Reynolds number criterion for turbulence is not reached, especially since 2D RTI is examined.

The number of cells in each simulation is $2049\times2049$.
The width $L_x$ of the domain is $1$, and the height $L_y$ is $1$.
The boundary conditions are periodic in $x$ and no slip and no penetration in $y$.

Initially, the velocity field is zero, temperature is 293.15 K, and pressure is 1 atm. 
A tophat perturbation based on the ones used by \citet{morgangreenough2015} and \citet{morgan2022} is imposed on the density field at the interface of the heavy and light fluids:
\begin{align}
    \xi(x) &=
    % \sum^{\kappa_\text{max}}_{j=\kappa_\text{min}}
    \sum^{\kappa_\text{max}}_{k=\kappa_\text{min}}
    \frac{\Delta}{\kappa_\text{max}-\kappa_\text{min}+1}
    \left(\cos\left(2\pi k x+\phi_{1,k}\right)
    +\sin\left(2\pi k x+\phi_{2,k}\right)\right),\\
    \rho(x,y) &= \rho_L+ \frac{\rho_H-\rho_L}{2}\left(1+\tanh\left(\frac{y-L_y/2+\xi}{2\Delta}\right)\right),
\end{align}
where $\phi_{1,k}$ and $\phi_{2,k}$ are phase shift vectors randomly taken from a uniform distribution, and $L_y$ is the length in $y$ of the domain.
Here, the minimum and maximum wavenumbers are set to  $\kappa_\text{min}=8$ and $\kappa_\text{max}=256$, respectively.

The stop condition of the simulations is when $h$ is approximately half the domain size in $y$.
This corresponds to the nondimensional simulation time $\tau$ of $30.84$.
$\tau$ is defined as $\frac{t}{t_0}$, where $t_0=\sqrt{\frac{h_0}{Ag}}$ and $h_0$ is the dominant lengthscale determined by the peak of the initial perturbation spectrum.


% Figure environment removed

Before the MFM analysis was conducted, the results of the donor simulations were examined.
% The figures presented in this work use a nondimensional time 
Figure \ref{subfig:donor_phi} shows mixedness over time for one donor simulation. 
After around $\tau = 5$, mixedness stops oscillating and converges to around $0.6$, indicating that the flow has entered a self-similar state.
Figure \ref{subfig:donor_alpha} shows the two definitions of $\alpha$ over time.
Similar to the behavior of $\phi$, we see that after sufficient time, $\alpha$ becomes approximately independent of time, again confirming the flow has entered a self-similar state.
Furthermore, the two definitions of $\alpha$ appear to be converging to about the same value, as theoretically expected.

% Figure environment removed

% Figure environment removed

To further ensure the simulations are working as desired, the flow fields of the donor and receiver simulations can be examined qualitatively.
The $Y_H$ contours at the last timesteps of each simulation are shown in figure \ref{fig:donorvsreceiver}.
The receiver simulation shown is the one used to compute $D^{00}$ (where $\overline{Y_H}=y$).
Self-similar RTI turbulent mixing is observed at this timestep, where the characteristic heavy spikes are sinking into the lighter fluid and the light bubbles rise into the heavier fluid.
Both simulations have the same velocity fields, since the receiver simulation ``receives'' the velocity field from the donor simulation.
In contrast with the donor simulation, which has a stark black-and-white difference between the heavy and light fluids, there is a grey gradient of density in the receiver simulation due to the imposed mean scalar gradient.
The fluctuations of $Y_H$ in each simualtion are also compared in figure \ref{fig:donorvsreceiver_YHp}.
The $Y_H'$ contours are not identical but are qualitatively very similar.
In both simulations, $Y_H'$ is constrained to the mixing layer.
Based on these observations, it is concluded that the simulations are visually working as intended.



\section{Results}
\label{sec:results}
\subsection{Eddy diffusivity moments}
\label{sec:moments}
% \begin{itemize}
%     \item Show measurements
%     \item Note on nondimensionalization
%     \item Explain physical meaning of moments (e.g., D00 is area under kernel, D01 is asymmetry in space, D10 is asymmetry in time, D02 is MOI)
%     \item Point out notable characteristics: symmetry of D00, always positive; antisymmetry of D01; symmetry of D10, alsways negative; symmstry of D02, always positive
%     \item Note on statistical convergence --> explanation of difficulty w/ convergence of D10?
% \end{itemize}

% Figure environment removed

Figure \ref{fig:moments} shows normalized MFM measurements of the eddy diffusivity moments $D^{00}$, $D^{10}$, $D^{01}$, and $D^{20}$ averaged over $1,000$ realizations and the homogeneous direction $x$.
Some expected characteristics of the measured moments are observed: 
\begin{enumerate}
    \item $D^{00}$ is symmetric and always positive.
    \item $D^{10}$ is antisymmetric. This is expected, since distance from the centroid is positive with positive $\eta$ and negative with negative $\eta$.
    \item $D^{01}$ is symmetric and always negative. The latter must be true for the flow to depend on its history (it does not violate causality).
    \item $D^{20}$ is symmetric and always positive, as is characteristic of moment of inertia of a positive kernel.
\end{enumerate}

Based on the magnitudes of the normalized moments, some initial observations on importance of each moment can be made.
$D^{00}$ has the highest magnitude of all the moments, which is expected since it is the leading-order moment.
The magnitudes of $D^{10}/h$ and $D^{01}/t$ are on the order of $10\%$ of the magnitude of $D^{00}$, which suggests that they are non-negligible.
On the other hand, the magnitude $D^{20}/h^2$ is on the order of $1\%$ that of $D^{00}$, so $D^{20}$ is likely not an important moment to include in modeling RTI.
More robust studies will be presented in the following sections to determine importance of each of the eddy diffusivity moments.

% Include statistical convergenc plots


% Figure environment removed


It is also observed that there is statistical error in the measurements.
Due the chaotic nature of RTI, the moment measurements contain statistical error, but this error can be reduced by averaging many realizations.
To demonstrate statistical convergence of the measurements, plots of $D^{00}$ averaged over different numbers of realizations are included in figure \ref{fig:stat_conv}.
As the number of realizations increases, the plots become smoother, and it is found that after $1,000$ realizations, the rate of reduction in statistical error slows down significantly.
Averaging over this number of realizations results in a smooth $D^{00}$ measurement and higher-order moment measurements with acceptably less statistical error.

It is observed that the higher the order of the moment, the slower its rate of statistical convergence. 
Recall that determination of higher-order moments requires information from lower-order moments.
For example, in determining $D^{01}$, $tD^{00}$ is subtracted from $F^{01}$, the turbulent scalar flux measurement in the simulation associated with $D^{01}$.
Naturally, there is statistical error associated with both $D^{01}$ and $D^{00}$.
However, the error in $D^{00}$ is amplified by $t$, so the overall statistical error of $D^{01}$ increases with time.
This statistical error ``leakage'' occurs for all higher-order moments.
The higher the order of the moment, the worse the statistical error, since information from more lower-order moments is needed, and so more statistical error is accumulated and amplified.
The relatively high statistical error of the higher-order moments makes it challenging to study their importance.
Particularly, taking derivatives of quantities with high statistical error amplifies the error, so smoother measurements are desired.
In this work, the moment measurements are smoothed using a Savitzky-Golay filter function in Matlab with a polynomial order of unity and window size of $191$.
These smoothed moments are shown in figure \ref{fig:sm_moments}.
% Ideally, smoother moment measurements would be obtained by modifying the MFM equations to increase the rate of statistical convergence of the moment measurements, but this is left for a future work \citep[see][]{lavacot2022}.
While it is possible to design an alternative formulation of MFM that removes leakage of statistical error from low-order moments to higher-order moments \citep[see][]{lavacot2022}, for this 2D study and for the order of moments considered here, the statistical convergence is sufficient. 

% Figure environment removed
% Remedy of this statistical error will be included in future work and involves analytically subtracting the ``leftover" moment terms to avoid error increasing with time.
% One remedy for this statistical error involves derivation of new receiver equations that eliminates the numerical subtraction of ``leftover" moment terms.
% For now, as explained in \S \ref{sec:alg_fit}, we obviate any effects from statistical error by making algebraic fits to these measurements.


\subsection{Assessment of importance of nonlocal effects}



\subsubsection{Comparison of terms in turbulent scalar flux expansion}
\label{sec:term_comp}

% Figure environment removed

To aid in the determination of which moments are important for a RANS model, a comparison of the terms in the expansion of the turbulent scalar flux (equation \ref{eq:explicit}) is presented.
% In computation of the terms, a $\overline{Y_H}$ fitted to the DNS calculation is used and substituted into equation \ref{eq:explicit}.
These terms involve gradients of $\overline{Y_H}$.
Instead of using $\overline{Y_H}$ directly from the DNS, a fit to $\overline{Y_H}$  is used, since the statistical error in the raw measurement gets amplified by derivatives in $\eta$.
That is, the quantities of interest are sufficiently converged for plotting but not for operations involving derivatives.
Thus, an analytical fit to $\overline{Y_H}$ is obtained as follows:
\begin{gather}
    \overline{Y_H}^* = \begin{cases}
    0 & \text{if } \eta<-a\\
    \int_{-a}^\eta \frac{1}{\left(a^2-{\eta'}^2\right)^2}\exp\left(\frac{1}{B\left(a^2-{\eta'}^2\right)}\right)d{\eta'} & \text{if } -a\leq\eta\leq a\\
    1 & \text{if } \eta>a
    \end{cases},\\
    \overline{Y_H} = \frac{\overline{Y_H}^*}{\overline{Y_H}^*_\text{max}},
\end{gather}
where the integral is determined numerically, and $a$ and $B$ are fitting coefficients.
The coefficients $a^2=1.2$ and $B=0.36$ are found to give good agreement to the mean concentration profile from DNS, as shown in figure \ref{fig:YH_fit}.


% Figure environment removed
The terms on the right hand side of equation \ref{eq:explicit} are plotted against the DNS measurement of the turbulent scalar flux in figure \ref{fig:exp_terms}.
Clearly, the $\widehat{D^{00}}$ term is not enough to capture the turbulent scalar flux.
It is observed that the $\widehat{D^{01}}$ term is significant in magnitude in the middle of the domain, and superposing this term on the $\widehat{D^{00}}$ term may increase its magnitude to better match the DNS measurement.
However, $\widehat{D^{01}}$ would not be enough to match the shape of the turbulent scalar flux: at the edges of the domain, this term becomes negative, which would make the modeled flux skinnier than the actual flux.
At the edges, it can be seen that the $\widehat{D^{10}}$ term carries importance, so this moment must also be included.
Lastly, the $\widehat{D^{20}}$ term is smaller in magnitude than the other terms, suggesting that this moment is least important.
% With this knowledge, we move on to constructing improved models using these higher-order moments.

\subsubsection{Comparison of leading-order model against a local model}
% \begin{itemize}
%     \item Comparison of h(t): there is a difference between h from kL and MFM
%     \item Comparison of D00: k-L>MFM
%     \item Comparison of YH
%     \begin{itemize}
%         \item Using diff. forms of D00: not much variance, so form of fit not that important
%         \item K-L matches slope of DNS better, but linear (not correct shape). MFM has more accurate shape but doesn’t match slope
%         \item k-L D00 is bigger to compensate for lack of higher-order terms --> <YH> is good in middle, but incorrect shape
%         \item error in MFM is due to missing higher-order terms. MFM D00 not compensating for anything; it is < k-L D00, so slope not matched
%         \item takeaway: need nonlocal terms for more accuracy
%     \end{itemize}
% \end{itemize}

% Figure environment removed

To demonstrate the shortcomings of models using purely local coefficients, an MFM-based leading-order model and the $k$-$L$ RANS model are compared.
In particular, a 1D $k$-$L$ simulation is run, and the eddy diffusivity and mean concentration profiles are extracted from the results to be compared to those of the MFM-based model using the measured $D^{00}$ that was presented in \S \ref{sec:moments}.
The $k$-$L$ simulation used in this section is implemented in Ares, and details of the implementation are in \citet{morgangreenough2015} and \citet{morgan2018response}.
Note that the $k$-$L$ simulation is used here for illustration purposes and should not be confused with the 2D DNS simulations used to obtain our MFM moments.

The MFM-measured $D^{00}$ is used for the leading-order MFM-based model:
\begin{align}
    -\overline{v'Y_H'} = D^{00}_\text{MFM}\frac{\partial\overline{ Y_H}}{\partial y}.
    \label{eq:MFM_leading_order}
\end{align}
To solve this, $D^{00}_\text{MFM}$ is obtained from the smoothed MFM measurements and transformed to self-similar coordinates.
The resulting $\widehat{D^{00}_\text{MFM}}$ is a function of $\eta=\frac{y}{h_\text{DNS}}$, where $h_\text{DNS}=\alpha^*_\text{DNS}Ag(t-t^*_\text{DNS})^2$ is an algebraic fit to the mixing width from the DNS.
The equation is then solved semi-analytically in conjunction with the mean mass fraction evolution equation in self-similar coordinates:
\begin{align}
    -2\eta\frac{d\overline{Y_H}}{d\eta}=\frac{d}{d\eta}\left(-\widehat{\overline{v'Y_H'}}\right),\\
    -\widehat{\overline{v'Y_H'}} = \widehat{D^{00}_\text{MFM}}\frac{d\overline{Y_H}}{d\eta}\label{eq:MFM_leading_order_ss}.
\end{align}
% The obtained $\overline{Y_H}$ is then substituted back into equation \ref{eq:MFM_leading_order_ss} to reconstruct $-\widehat{\overline{v'Y_H'}}$.

The $k\text{-}L$ model uses the gradient diffusion approximation for the turbulent flux:
\begin{align}
    -\overline{v'Y_H'} = \frac{\mu_t}{\overline{\rho}N_Y}\frac{\partial\overline{ Y_H}}{\partial y}\label{eq:kL_grad_diff} = D^{00}_{k\text{-}L}\frac{\partial\overline{ Y_H}}{\partial y},
\end{align}
where $\mu_t=C_\mu\overline{\rho}L\sqrt{2k}$.
$N_Y$ is a model coefficient set by similarity constraints derived by \citet{dimontetipton2006}.
In this work, $C_\mu$ is unity, $N_Y$ is $2.47$.
Again, $D^{00}_{k\text{-}L}$ is purely local, and it is obtained through post-processing results of an RTI simulation using the implementation of the $k\text{-}L$ model in Ares.
While equation \ref{eq:kL_grad_diff} is solved in spatio-temporal coordinates, the resulting $D^{00}_{k\text{-}L}$ is transformed to $\widehat{D^{00}_{k\text{-}L}}$ according to the self-similar coordinate $\xi=\frac{y}{h_{k\text{-}L}}$, where $h_{k\text{-}L}=\alpha^*_{k\text{-}L}Ag(t-t^*_{k\text{-}L})^2$, for a meaningful comparison with the MFM-based model.
It must be noted that the $h$ fitting coefficients $\alpha^*$ and $t^*$ are not the same between the DNS and $k$-$L$ solutions.
In this work, $\alpha^*_\text{DNS}=0.046$, $t^*_\text{DNS}=-1600\ s$, $\alpha^*_{k\text{-}L}=0.04$, and $t^*_{k\text{-}L}=1250\ s$ ($t^*_{k\text{-}L}$ is positive due to the relaxation time to the self-similar profiles in the beginning of the $k$-$L$ simulation). 

% The following results are all determined in self-similar space, so normalized versions of the model coefficients are presented.
Plots comparing $\widehat{D^{00}_\text{MFM}}$ and $\widehat{D^{00}_{k\text{-}L}}$ are in figure \ref{fig:klvsmfm_D00}. 
% In these plots, each model coefficient is plotted against the appropriate $\eta$.
% That is, we plot $\widehat{D^{00}_\text{MFM}} $ and $\widehat{D^{00}_{k\text{-}L}}$ against $\eta_\text{MFM}=\frac{y}{h_\text{MFM}}$ and $\eta_{k\text{-}L}=\frac{y}{h_{k\text{-}L}}$, respectively.
% $h_\text{MFM}$ and $h_{k\text{-}L}$ are each algebraic fits to $h$ measured from each corresponding simulation (the donor DNS for the former, and the Ares $k\text{-}L$ model simulation for the latter).
% This way, the two leading-order model coefficients can be compared meaningfully.
It is observed that $\widehat{D^{00}_{k\text{-}L}}$ is greater in magnitude than $\widehat{D^{00}_\text{MFM}}$.
The reason for this can be understood by comparing $\overline{Y_H}$ computed using each of the two models, shown in figure \ref{fig:klvsmfm_YH}.
% Figure \ref{fig:klvsmfm_YH} shows the $\overline{Y_H}$ computed using the $k$-$L$ and MFM-based leading-order models.
% Specifically, the $k$-$L$ $\overline{Y_H}$ is retrieved from the Ares $k$-$L$ simulation, and the MFM-based $\overline{Y_H}$ is obtained by solving equations \ref{eq:MFM_leading_order} and \ref{eq:ste_avg} in self-similar space.
% Details for the solution of the MFM-based leading-order model can be found later in \S \ref{sec:improve_RANS}.
% In this study, we solve for $\overline{Y_H}$ using each of the above models substituted into Eq. \ref{eq:ste_avg} and transformed to self-similar space.
% Note that this means that the $\overline{Y_H}$ reported for the $k\text{-}L$ model is not the one computed using the ARES implementation but instead this 1-D equation in self-similar space using $D^{00}_{k\text{-}L}$.
% The resulting plots are in figure \ref{fig:klvsmfm_YH}, which also includes the DNS measurement of $\overline{Y_H}$.
At first glance, it seems that the $k\text{-}L$ model actually performs fairly well, matching the slope of the DNS measurement.
However, it gives a linear profile for $\overline{Y_H}$, which diverges from the DNS in the profile tails.
% As work will show in the following sections, 
This is because the $k\text{-}L$ model is compensating for error in model form by using a model coefficient of higher magnitude than what is measured through MFM.
On the other hand, the leading-order MFM-based model gives a more realistic profile that is smooth at the edges of the mixing region, but it does not match the slope of the DNS measurement, which is not surprising since this is a truncated model form.
As results will show shortly, the gap between the leading-order MFM-based model $\overline{Y_H}$ and the DNS measurement would be bridged by inclusion of higher-order moments, which would introduce information about the nonlocality of the eddy diffusivity.


\subsection{Assessment of nonlocal model forms}
\label{sec:improve_RANS}
In this section, two RANS model forms using information about the nonlocality of the eddy diffusivity are presented.
These are the explicit and implicit model forms; the former is a truncation of the turbulent scalar flux expansion \ref{eq:explicit}, and the latter will be presented shortly. 
It must be stressed that the intention of the following studies is not to propose a new RANS model.
Ultimately, a RANS model should not depend on direct MFM measurements that can only be retrieved from impractically many DNS.
Instead, these studies are performed to further assess the importance of each of the eddy diffusivity moments, determine which combinations of moments best enhance the performance of a RANS model, and examine the differences between the explicit and implicit model \textit{forms}.
The aim of these studies is to inform development of more predictive RANS models for RTI, not to suggest that these are the exact models that should be used.

In addition, $D^{20}$ will not be included in the following studies.
This is mainly due to the high statistical error in the measurement that makes it difficult to ascertain whether errors in the results are due to this statistical error or solely the addition of the moment to the model.
From the comparison of terms in \S \ref{sec:term_comp}, it is expected that $D^{20}$ is not as important as $D^{10}$ and $D^{01}$ to include in a RANS model.
This should be tested in future work when a more statistically-converged measurement is achieved for $D^{20}$, ideally in a 3D analysis.

\subsubsection{Explicit model form}
\label{sec:exp_model_form}
% \begin{itemize}
%     \item Define explicit model as truncated taylor series
%     \item Taylor series doesn’t converge
%     \item Comparison of terms in explicit model
%     \item A priori study: mass flux reconstruction
%     \item TRANSITION: explain shortcomings of explicit model (implementation for explicit model is difficult), need for implicit model
% \end{itemize}

The explicit model form is a truncation of the expansion of the turbulent scalar flux, as defined in equation \ref{eq:explicit}.
\citet{hamba1995} and \citet{hamba_2004} have examined this form in the context of shear flows.
Transformation of this truncated expansion to self-similar coordinates and substitution into \ref{eq:ste_avg} results in
\begin{align}
    2\eta \frac{d\overline{Y_H}}{d\eta}=\frac{d}{d\eta}\left[\left(\widehat{D^{00}}+\widehat{D^{01}}\right)\frac{d\overline{Y_H}}{d\eta} + \left(\eta\widehat{D^{01}}+\widehat{D^{10}}\right)\frac{d^2\overline{Y_H}}{d\eta^2}\right],
    \label{eq:explicit_model}
\end{align}
which can be solved numerically for $\overline{Y_H}$.
The $\widehat{D^{mn}}$ used in the numerical solve are the smoothed, normalized moments.
To determine which eddy diffusivity moments are important in constructing RANS models for RTI, different combinations of $\widehat{D^{mn}}$ terms are kept in equation \ref{eq:explicit_model}, and the results are compared to DNS.
In the numerical solve, equation \ref{eq:explicit_model} is discretized on a staggered mesh, and derivatives are computed using central finite differences. 
A matrix-vector equation is assembled and solved for $\overline{Y_H}$ with Dirichlet boundary conditions.


% Figure environment removed
% Figure environment removed

Figure \ref{fig:exp_aposteriori_tsf} shows the turbulent scalar fluxes computed using the explicit model form, and figure \ref{fig:exp_aposteriori_YH} shows the corresponding mean concentration profiles.
Again, it is apparent that the leading-order moment is not enough to capture the turbulent scalar flux.
% Adding $D^{10}$ improves the shape of the turbulent scalar flux slightly, but still does not match the DNS-measured turbulent scalar flux very well.
% Only adding the first temporal moment $D^{01}$ results in oscillations near the tail ends of the turbulent scalar flux, indicating that this also is insufficient for a good RANS model.
The combination using $D^{00}$, $D^{10}$, and $D^{01}$---the moments deemed most important in \S \ref{sec:term_comp}---gives the best match to the DNS measurement.

It is particularly remarkable that a converged turbulent scalar flux can be obtained using $D^{00}$, $D^{10}$, and $D^{01}$.
As mentioned previously, it is known that equation \ref{eq:explicit} may not converge.
That is, the expansion must be taken to infinite terms to remove error; truncating the expansion can result in significant error.
This is analogous to a Kramers-Moyal expansion, which cannot be approximated adequately by more than two terms, after which it requires infinite terms for convergence \citep{pawula1967,mauri1991}. 
To understand how adding terms to equation \ref{eq:explicit} can result in greater error, one can consider the eddy diffusivity kernel associated with each term.
The leading-order moment is associated with a delta function kernel, as it is purely local.
% Thus, higher-order moments are simply superpositions of derivatives of this delta function.
However, when equation \ref{eq:gen_eddy_diff_2d} is replaced by equation \ref{eq:explicit}, an integral operator is replaced with a high-order differential operator.
This means that the nonlocal effects are approximated by derivatives of delta functions; see \citet{liu2021} for more details.
It has been shown that, in general, the eddy diffusivity kernel is not a superposition of finite delta functions, as it is smooth \citep{manipark2021, liu2021}.
Therefore, truncation of the expansion does not match the shape of the eddy diffusivity kernel, leading to errors in prediction of the turbulent scalar flux.

Another issue with the explicit model form is its numerical implementation.
% In this work, we implement the explicit model in self-similar space, and this is more convenient than spatio-temporal space, which is what would be done in practice.
In spatio-temporal space, some terms associated with higher-order moments involve mixed derivatives (e.g., the term $D^{01}\frac{\partial^2}{\partial t\partial y}$), which would undergo another spatial gradient when substituted into equation \ref{eq:ste_avg}.
Such terms are difficult to handle numerically.
In this work, the model is implemented in the more convenient self-similar space, but, ultimately, a spatio-temporal model would be developed, as it is more practical. 
It is thus pertinent to work towards a better method to incorporate nonlocal information in a RANS model that does not encounter the Kramers-Moyal-like convergence issue and is easier to implement.
% MORE

% % Figure environment removed

\subsubsection{Implicit model form and the Matched Moment Inverse (MMI)}
% \begin{itemize}
%     \item Overview \& definitions
%     \item Transformation of implicit model to self similar coord.
%     \item Investigation of zero-crossings of det(MMI matrix)
%     \begin{itemize}
%         \item MMI is a tool that tells us whether or not our assumed implicit model form is “good”
%         \item Compare “good” model (D00, D01, D10) vs “bad” model (D00, D10)
%     \end{itemize}
%     \item A priori study: comparisons of reconstructed mass flux
% \end{itemize}

In this section, an implicit model form is introduced as a solution to both the increasing error when adding terms from the turbulent scalar flux expansion and implementation challenges associated with the explicit model form.
Recall that the explicit model form fails to match the shape of the eddy diffusivity kernel without infinite terms of the turbulent scalar flux expansion.
In this implicit model form, the aim is to match the shape of the eddy diffusivity kernel, instead of using the truncated expansion for the turbulent scalar flux.
Using the four moments that have been measured, this model form is
\begin{align}
    \left[1+a^{01}\frac{\partial}{\partial t}+a^{10}\frac{\partial}{\partial y}+a^{20}\frac{\partial^2}{\partial y^2}\right](-\overline{v'Y_H'})=a^{00}\frac{\partial \overline{Y_H}}{\partial y},
\label{eq:imp_model_form}
\end{align}
where $a^{mn}(y,t)$ are model coefficients fitted corresponding to each of the eddy diffusivity moments $D^{mn}$ measured using MFM.
The bracketed operator on the left hand side is the Matched Moment Inverse (MMI) operator.
The way this model form is designed to match the eddy diffusivity kernel shape is detailed in \cite{liu2021}.
% In addition, this form is significantly easier to implement numerically, as it only involves construction of an operator matrix that can be inverted to solve the equations.
In addition, this form is significantly easier to implement numerically in spatio-temporal space, since it can be directly time-integrated using explicit methods.
In this way, it is also easy to add more terms with higher-order moments, as it simply requires extension of the operator.

In self-similar coordinates, this becomes
\begin{align}
    \left[1+\widehat{a^{01}}\left(1-2\eta\frac{d}{d\eta}\right)+\widehat{a^{10}}\frac{d}{d \eta}+\widehat{a^{20}}\frac{d^2}{d \eta^2}\right](-\widehat{\overline{v'Y_H'}} )=\widehat{a^{00}}\frac{d \overline{Y_H}}{d \eta},
    \label{eq:MMI_selfsim}
\end{align}
where it is found through self-similar analysis that
\begin{align}
    \widehat{a^{00}}&=\frac{1}{{\alpha^*}^2 A^2g^2(t-t^*)^3}a^{00},\\
    \widehat{a^{01}}&=\frac{1}{t-t^*}a^{01},\\
    \widehat{a^{10}}&=\frac{1}{\alpha^* Ag(t-t^*)^2}a^{10},\\
    \widehat{a^{20}}&=\frac{1}{{\alpha^*}^2 A^2g^2(t-t^*)^4}a^{20}.
\end{align}

The coefficients are determined through a process illustrated as follows in spatio-temporal coordinates for simplicity.
If one wants to construct a model in the form of equation \ref{eq:imp_model_form}, four equations must be formulated to determine the four coefficients.
This is done by using measurements from the four simulations used to determine the four moments $D^{00}$, $D^{10}$, $D^{01}$, and $D^{20}$.
For example, the first equation results from substitution of $F^{00}$ for $-\overline{v'Y_H'}$ and the associated desired $\frac{\partial\overline{Y_H}}{\partial y}$; the remaining three equations follow, using the other three moments:
\begin{align}
    \left[1+a^{10}\frac{\partial}{\partial y}+a^{01}\frac{\partial}{\partial t}+a^{20}\frac{\partial^2}{\partial y^2}\right]F^{00}&=a^{00},\\
    \left[1+a^{10}\frac{\partial}{\partial y}+a^{01}\frac{\partial}{\partial t}+a^{20}\frac{\partial^2}{\partial y^2}\right]F^{10}&=a^{00} \left(y-\frac{1}{2}\right),\\
    \left[1+a^{10}\frac{\partial}{\partial y}+a^{01}\frac{\partial}{\partial t}+a^{20}\frac{\partial^2}{\partial y^2}\right]F^{01}&=a^{00}t,\\
    \left[1+a^{10}\frac{\partial}{\partial y}+a^{01}\frac{\partial}{\partial t}+a^{20}\frac{\partial^2}{\partial y^2}\right]F^{20}&=a^{00}\frac{1}{2}\left(y-\frac{1}{2}\right)^2.
\end{align}
This system of equations is then rearranged into a matrix equation $M_\text{MMI}\mathbf{a}=\mathbf{b}$, which is solved for the coefficients in vector $\mathbf{a}=(a^{00},a^{10},a^{01},a^{20})^T$.
Note that this matrix equation is constructed over every point in space and time, so $\mathbf{a}=\mathbf{a}(y,t)$.
In this work, analysis is done in self-similar coordinates, in which $\textbf{a}=\textbf{a}(\eta)$.
If one wishes to construct a model with different moments, the MMI operator and equations must be modified accordingly.
For example, a model using only $D^{00}$ and $D^{10}$ would have an MMI operator of the form $1+a_{10}\frac{\partial}{\partial y}$ and use only the first two equations (with the $a^{01}$ and $a^{20}$ terms removed).
Thus, models using different combinations of moments would use different MMI coefficients $a^{mn}$.

To summarize, for this implicit model form, the following system of equations is solved in self-similar coordinates:
\begin{align}
    \mathcal{L}_\text{MMI}\left\{-\widehat{\overline{v'Y_H'}} \right\}&=\widehat{a^{00}}\frac{d \overline{Y_H}}{d \eta},\\
    \frac{d}{d\eta}\left(-\widehat{\overline{v'Y_H'}}\right) &= -2\eta\frac{d\overline{Y_H}}{d\eta},
\end{align}
where $\mathcal{L}_\text{MMI}$ is the MMI operator constructed using some combination of moments, such as in equation \ref{eq:MMI_selfsim}.
% \begin{align}
%     \left[1+\widehat{a^{01}}\left(1-2\eta\frac{d}{d\eta}\right)+\widehat{a^{10}}\frac{d}{d \eta}+\widehat{a^{20}}\frac{d^2}{d \eta^2}\right](-\widehat{\overline{v'Y_H'}} )&=\widehat{a^{00}}\frac{d \overline{Y_H}}{d \eta},\\
%     \frac{d}{d\eta}\left(-\widehat{\overline{v'Y_H'}}\right) &= -2\eta\frac{d\overline{Y_H}}{d\eta}.
% \end{align}
Numerically, the following system is solved:
\begin{align}
    P(-\widehat{\overline{v'Y_H'}} )&=\widehat{a^{00}}\mathcal{D_\eta}\overline{Y_H},\\
    \mathcal{D_\eta}\left(-\widehat{\overline{v'Y_H'}}\right) &= -2\eta\mathcal{D_\eta}\overline{Y_H},
    \label{eq:apost_eqs}
\end{align}
where $P$ is the matrix representing the numerical MMI operator, and $\mathcal{D_\eta}$ is the matrix representing the numerical derivative with respect to $\eta$.
This can be rewritten as a block matrix-vector multiplication:
\begin{align}
    M \textbf{x}= 
    \begin{bmatrix}
    P & -\widehat{a^{00}}\mathcal{D_\eta}\\
    \mathcal{D_\eta} & 2\eta\mathcal{D_\eta}
    \end{bmatrix} 
    \begin{bmatrix}
    -\widehat{\overline{v'Y_H'}}\\
    \overline{Y_H}
    \end{bmatrix}
    = \textbf{b},
\end{align}
where $\textbf{b}$ is a vector representing the right-hand side of equations \ref{eq:apost_eqs}, with the proper boundary conditions enforced.
In this study, zero gradient boundary conditions are used for the turbulent scalar flux, and Dirichlet boundary conditions are used for the mean concentration.
The system is solved using finite differences on a staggered mesh.


% Figure environment removed

% Figure environment removed


Presented in this work are the determinants of the MMI matrix and resulting $a_{mn}$ for two different combinations of moments. Figure \ref{fig:MMI_good} shows that with the combination of $D^{00}$, $D^{10}$, and $D^{01}$, the determinant of the MMI matrix is positive for all $\eta$, so $a_{mn}$ are all well-behaved.
This is indicative of good model form.
On the other hand, figure \ref{fig:MMI_singularity} shows that with the combination of $D^{00}$ and $D^{01}$, the determinant of the MMI matrix crosses zero, so $a_{mn}$ contain singularities which effect poor RANS predictions (observable in plots presented later).
Singular matrices arising in the MMI solve for a certain form of the implicit model form may indicate that form is poor, in the sense that the combination of moments does not make a good RANS model.
Since MMI appears to be sensitive to the information it takes in to determine the implicit model coefficients, one must take special care and choose a model form that avoids this issue.
It is found that MMI determinant zero-crossings do not occur for any of the moment combinations tested in this work other than the $D^{00}$ and $D^{01}$ combination, but it may happen with combinations of other higher-order moments not measured here.

% Figure environment removed

% Figure environment removed

Turbulent scalar fluxes computed using the implicit model form are shown in figure \ref{fig:imp_aposteriori_tsf}.
The implicit model form's turbulent scalar flux prediction using just $D^{00}$ is identical to that of the explicit model form, by construction.
It is apparent that adding either $D^{10}$ or $D^{01}$ alone is insufficient.
As noted earlier, adding $D^{01}$ leads to a particularly poor prediction due to singular MMI matrices at some $\eta$.
The best match to DNS is attained using the combination of $D^{00}$, $D^{10}$, and $D^{01}$.
In fact, it is evident that the implicit model form using $D^{00}$, $D^{10}$, and $D^{01}$ predicts the turbulent scalar flux more accurately than the explicit model form using the same moments.
This is because the implicit model form is designed to match the shape of the eddy diffusivity kernel, and 
the explicit model form may not be accomplishing this.


% Figure environment removed

These trends in the explicit and implicit model forms can be observed again in the predictions of the mean concentration profile, shown in figure \ref{fig:imp_aposteriori_YH}.
Particularly, the implicit model form using $D^{00}$, $D^{10}$, and $D^{01}$ gives a very good prediction of the mean concentration that nearly overlaps the DNS measurement.
For a clearer comparison of the explicit and implicit model forms using these moments, figure \ref{fig:expvsimp} shows the derivatives of the DNS- and model-computed $\overline{Y_H}$.
The implicit model form predicts a magnitude and shape closer to the DNS measurement than the explicit model form does.
In particular, the implicit model form captures the shape of the tails much better than the explicit model form.



\section{Conclusion}
% \begin{itemize}
%     \item Future work
%     \begin{itemize}
%         \item 3D RTI
%         \item Potentially explore other moments, e.g., D11
%         \item Improved RANS model is work in progress
%         \item Compressible flow
%     \end{itemize}
% \end{itemize}

In this assessment, it is determined that nonlocality must be considered in developing more predictive models for RTI.
The studies presented in this work are facilitated using MFM, a numerical tool for precisely measuring closure operators.
Four of the eddy diffusivity moments of RTI ($D^{00}$, $D^{10}$, $D^{01}$, and $D^{20}$) are measured, and it is demonstrated that the higher-order moments, which contain information about the nonlocality of the eddy diffusivity kernel, should not be neglected when constructing models for RTI.

Specifically, it is determined that $D^{00}$, $D^{10}$, and $D^{01}$ are the most important moments for constructing a model for RTI.
Two methods for constructing RANS models using these moments are presented.
First, an explicit model form, based on a Kramers-Moyal-like expansion derived by taking the Taylor series expansion of the scalar gradient in the generalized eddy diffusivity, is described and tested.
While incorporation of higher-order moments in the explicit model form results in more accurate predictions than a leading-order model, there exist several issues.
One problem is that the expansion used for the explicit model form may not converge, so addition of higher-order moments leads to less accurate predictions.
Another problem is that the explicit model form is difficult to implement numerically.

Thus, an implicit model form is presented to address these issues with the explicit model form.
Since an implicit model form involves an invertible matrix operator, it is easier to implement than an explicit model form.
In addition, the proposed implicit model form is designed to match the shape of the eddy diffusivity kernel via the MMI operator, in contrast to the explicit model form, which truncates a non-converging Kramer-Moyal expansion.
It is shown that the implicit model form exhibits a marked improvement in predictions over the explicit model form.

One obstacle encountered in these studies is the inherent statistical error in the DNS computations.
% In an attempt to circumvent this, fits were applied to $h$ and $\overline{Y_H}$, and the moments were smoothed before they were used in the models.
The higher-order moments particularly contain high statistical error due to buildup of error from the lower-order moments on which they depend.
Because of this, it is admittedly difficult to draw definite conclusions about the effect of higher-order moments beyond the first-order moments.
That is, due to the statistical error, it is currently unclear if inclusion of moments beyond first-order in a RANS model would significantly improve its predictions, or if just the first-order temporal and spatial moments along with the leading-order moment are sufficient.
This motivates development of a technique to accelerate the statistical convergence of these higher-order moments.
Such a method could also be used to study the effect of other higher-order moments that were not measured in this present work, since they would have suffered from high statistical error with the current method.
% This may involve derivation of new MFM equations that mitigate leakage of error from low-order moments to higher-order moments.

% In addition, it would be useful to examine the effect of other higher-order moments not measured in this present work.
% Specifically, it is unknown whether addition of $D^{20}$ to the RANS model must be accompanied by other higher-order moments in either time and space for good predictive power.
% With the current method, such higher-order moments would suffer from high statistical error, further motivating the need for an MFM with faster statistical convergence.

% We note several areas in need of improvement.
% First, it would be useful to achieve faster statistical convergence for our MFM measurements.
% With the method presented in this work, we required 1,000 high-fidelity simulations for statistical convergence.
% Future work will involve developing a method to reduce this requirement significantly.
% Second, while the implicit model is more desirable than the explicit model in several ways, the model coefficient solve is highly sensitive to model form.
% Specifically, the implicit model results in poor predictions when an inadequate model form is used (i.e., an inadequate combination of eddy diffusivity moments is utilized in the model).
% This requires further investigation and exploration of different implicit model forms.

Through this work, an understanding of nonlocality in 2D RTI has been developed, and similar trends in 3D are expected, but potential differences between 2D and 3D should also be thoroughly examined in future work.
It has been shown that incorporation of information about the nonlocality of the eddy diffusivity may greatly improve the accuracy of a RANS model.
This work demonstrates this by testing models using MFM measurements of the nonlocal eddy diffusivity.
In practice, a RANS model for RTI would not have to rely on these MFM measurements directly; one would not have to perform many MFM simulations to construct a model.
% Instead, a RANS model would be constructed based on the findings presented here.
In other words, MFM should be seen as a diagnostic tool rather than the means for building the actual model.
The ultimate goal is to develop an improved, more predictive model for RTI by incorporating nonlocal information, which the present work has demonstrated to be significant for accurate prediction of mean scalar transport in 2D RTI.

% Now that we have developed an understanding of nonlocality in RTI for this 2D problem, we aim to continue these studies for 3D RTI, which is more relevant to engineering applications, like ICF.
% While 3D simulations would require more mesh points, we expect statistical convergence to be achieved over fewer simulations, since we would have access to a third homogenous direction over which we could take averages, and 3D RTI is less chaotic than 2D RTI.
% In this future study, we will also measure more higher moments that were not investigated in this work, such as $D^{11}$ and $D^{20}$, so that we can assess their importance in constructing models for RTI.
% Our ultimate goal is to develop an improved, more predictive model for RTI by incorporating nonlocal information, which the present work has demonstrated to be significant for accurate prediction of mean scalar transport.

\textbf{Acknowledgements.} 
This work was performed under the auspices of the US Department of Energy by Lawrence Livermore National Laboratory under Contract No. DE-AC52-07NA27344.
D.L. was additionally supported by the Charles H. Kruger Stanford Graduate Fellowship.
J.L. was additionally supported by the Burt and Deedee McMurtry Stanford Graduate Fellowship.

\textbf{Declaration of Interests.} The authors report no conflict of interest.


\appendix
\section{Nondimensionalizations}

To determine the nondimensionalizations in equations \ref{eq:nondim1} - \ref{eq:nondim2}, a self-similarity analysis is performed.
The following self-similarity coordinate is used:
\begin{align}
    \eta = \frac{y}{{h_\text{fit}}(t)} = \frac{y}{\alpha^*Ag\left(t-t^*\right)^2}.
\end{align}
To perform transformations to this self-similar space, all derivatives are written in terms of $\eta$:
\begin{align}
    \frac{\partial}{\partial t} &= -\frac{2\eta}{t-t^*}\frac{d}{d \eta},\\
    \frac{\partial}{\partial y} &= \frac{1}{\alpha^*Ag\left(t-t^*\right)^2}\frac{d}{d \eta},\\
    \frac{\partial^2}{\partial t \partial y} &= -\frac{2}{\alpha^*Ag\left(t-t^*\right)^3}\left(\frac{\partial}{d \eta} + \frac{d^2}{d \eta^2}\right).
\end{align}
% First, the nondimensionalization of the eddy diffusivity moments is demonstrated here.
To nondimensionalize the eddy diffusivity moments, equation \ref{eq:explicit} is substituted into equation \ref{eq:ste_avg}:
\begin{align}
    \frac{\partial \overline{Y_H}}{\partial t} = \frac{\partial}{\partial y}\left(D^{00}\frac{\partial \overline{Y_H}}{\partial y} +
    D^{10}\frac{\partial^2 \overline{Y_H}}{\partial y^2} +
    D^{01}\frac{\partial^2 \overline{Y_H}}{\partial t \partial y} +
    D^{20}\frac{\partial^3 \overline{Y_H}}{\partial y^3}+\hdots\right).
\end{align}
The equation is then transformed to self-similar space:
\begin{align}
    -\frac{2\eta}{t-t^*}\frac{d \overline{Y_H}}{d \eta} =
    \frac{1}{\alpha^*Ag\left(t-t^*\right)^2}\frac{d}{d \eta}&\left[
    \frac{1}{\alpha^*Ag\left(t-t^*\right)^2} D^{00}\frac{d \overline{Y_H}}{d \eta} \right.\nonumber\\
    &+\frac{1}{{\alpha^*}^2A^2g^2\left(t-t^*\right)^4} D^{10}\frac{d^2 \overline{Y_H}}{d \eta^2} \nonumber\\
    &-\frac{2}{\alpha^*Ag\left(t-t^*\right)^3} D^{01}\left(\frac{d \overline{Y_H}}{d \eta} + \eta\frac{d^2 \overline{Y_H}}{d \eta^2}\right) \nonumber\\
    &\left.+\frac{1}{{\alpha^*}^3A^3g^3\left(t-t^*\right)^6} D^{20}\frac{d^3 \overline{Y_H}}{d \eta^3}\hdots
    \right].
\end{align}
Rearranging,
\begin{align}
    -2\eta\frac{d \overline{Y_H}}{d \eta} = \frac{d}{d \eta}&\left[
    \frac{1}{{\alpha^*}^2A^2g^2\left(t-t^*\right)^3} D^{00}\frac{d \overline{Y_H}}{d \eta} \right.\nonumber\\
    &+\frac{1}{{\alpha^*}^3A^3g^3\left(t-t^*\right)^5} D^{10}\frac{d^2 \overline{Y_H}}{d \eta^2} \nonumber\\
    &-\frac{2}{{\alpha^*}^2A^2g^2\left(t-t^*\right)^4} D^{01}\left(\frac{d \overline{Y_H}}{d \eta} + \eta\frac{d^2 \overline{Y_H}}{d \eta^2}\right) \nonumber\\
    &\left.+\frac{1}{{\alpha^*}^4A^4g^4\left(t-t^*\right)^7} D^{20}\frac{d^3 \overline{Y_H}}{d \eta^3}\hdots
    \right].
\end{align}
This reveals nondimensionalizations for the eddy diffusivity moments.
The prefactors to the derivatives of $\overline{Y_H}$ on the right hand side are denoted as the normalized eddy diffusivity moments $\widehat{D^{mn}}$.

The turbulent scalar flux scales with the leading-order term in equation \ref{eq:explicit}.
Substitution of the nondimensionalization for $D^{00}$ (equation \ref{eq:nondimD00}) into the leading-order term in equation \ref{eq:explicit} and transformation to self-similar coordinates gives the scaling for the turbulent scalar flux:
\begin{align}
    -\overline{v'Y_H'} \sim {\alpha^*}Ag\left(t-t^*\right)\widehat{D^{00}}\frac{d\overline{Y_H}}{d \eta}.
\end{align}


% Figure environment removed

% Figure environment removed

% Figure environment removed
Figure \ref{fig:raw_moms} shows the unscaled moments measured directly from the MFM simulations.
The profiles are taken from the portion of the simulation where the flow is self-similar ($\tau\gtrapprox22$).
It is obvious that without normalizing the moments as described above there is no self-similar collapse.
The moments are scaled and plotted against $\eta$ in figure \ref{fig:selfsim_moms} to demonstrate the self-similar collapse.
The normalized turbulent scalar flux is shown in figure \ref{fig:selfsim_tsf}, also showing a self-similar collapse.

% \section{Algebraic Fits}
% % \begin{itemize}
% %     \item algebraic fit form
% %     \item sensitivity analysis: alg fit coefficient range
% % \end{itemize}

% % Figure environment removed

% In this section, we denote the algebraic fits to quantities as $\widetilde{*}$.
% Note that we drop this notation in the main body of this text to reduce clutter and confusion.
% The algebraic fits for the moments are in the following form:
% \begin{align}
%     \widetilde{\widehat{D^{00}}} = B_{00}\left(1-\left(\frac{\eta}{a}\right)^2\right)^2,\\
%     \widetilde{\widehat{D^{10}}} = B_{01}\left(1-\left(\frac{\eta}{a}\right)^2\right)^2,\\
%     \widetilde{\widehat{D^{01}}} = B_{10}\eta\left(1-\left(\frac{\eta}{a}\right)^2\right)^2,\\
%     \widetilde{\widehat{D^{20}}} = B_{20}\left(1-\left(\frac{\eta}{a}\right)^2\right)^2 - g(\eta).
% \end{align}

% $B_{ij}$ essentially controls the magnitude of the resulting fit.
% Table \ref{tab:bij} lists the $B^{mn}$ chosen for each fit (under the ``Nominal'' column).
% $a$ is the width of the fit; outside $\pm a$, the resulting fit is zero. 
% For our fits, we use $a=1.1$.
% $g(\eta)$ is a function used only in the fit for $D^{20}$ that is chosen to produce the dip at $\eta=0$.
% We use a Gaussian for $g(\eta)$:
% \begin{align}
%     g(\eta) = 0.007\exp{\left(-\frac{\eta^2}{0.045}\right)}
% \end{align}



% \begin{table}
%     \centering
%     \begin{tabular}{c c c c c}
%          $i$ & $j$ & Nominal $B^{mn}$ & Upper $B^{mn}$ & Lower $B^{mn}$\\
%          $0$ & $0$ & $0.37$ & $0.355$ & $0.385$ \\
%          $1$ & $0$ & $-0.038$  & $-0.035$ & $-0.041$ \\
%     \end{tabular}
%     \caption{Fitting coefficients chosen for algebraic fits of eddy diffusivity moments.}
%     \label{tab:bij}
% \end{table}

% \section{Uncertainty Analysis}

% To quantify the error due to uncertainty of the algebraic fits, we tested out a range of combinations of fit coefficients.
% Upper and lower values of each fitting coefficient were determined, and 100 random combinations of each of the coefficients (i.e., $B^{mn}$ and $a$) were used to produce the red lines in the figures in \S \ref{sec:results}.
% For $a^2$, we chose upper and lower values of $1.0$ and $1.2$, respectively;
% The upper and lower $B^{mn}$ were chosen such that they were the maximum and minimum values for the fitting coefficient that would produce reconstructed turbulent scalar fluxes closest to the latest and earliest time measurements in the DNS, respectively.

% It was found that results are not as sensitive to algebraic fits of $\widehat{D^{10}}$ and $\widehat{D^{20}}$ compared to those of $\widehat{D^{00}}$ and $\widehat{D^{01}}$, so the former two are not included in the uncertainty analysis presented here.
% The upper and lower values of $B^{mn}$ for the latter two are in table \ref{tab:bij}.

% We also completed uncertainty analysis including the outer exponent of the algebraic fits (i.e., testing $1$, $2$, and $3$ for this exponent), but we found that the results are not as sensitive to the exponent as they are to the other fitting coefficients.


% \section{How to submit to the \textbf{\textit{Journal of Fluid Mechanics}}}
% Authors must submit using the online submission and peer review system, Scholar One Manuscripts (formerly Manuscript Central) at http://mc.manuscriptcentral.com/jfm. If visiting the site for the first time, users must create a new account by clicking on `register here'. Once logged in, authors should click on the `Corresponding Author Centre', from which point a new manuscript can be submitted, with step-by-step instructions provided. Authors must at this stage specify whether the submission is a standard paper, or a {\it JFM Rapids} paper (see \S\ref{sec:filetypes} for more details). In addition, authors must specify an editor to whom the paper will be submitted from the drop-down list provided. Note that all editors exclusively deal with either standard papers or {\it JFM Rapids} (clearly indicated on the list), so please ensure that you choose an editor accordingly. Corresponding authors must provide a valid ORCID ID in order to submit a manuscript, either by linking an existing ORCID profile to your ScholarOne account or by creating a new ORCID profile. Once your submission is completed you will receive an email confirmation.  Book reviews should not be submitted via the online submission site and should instead be submitted by email to c.p.caulfield@damtp.cam.ac.uk.
 
% \section{Rules of submission}\label{sec:rules_submission}
% Submission of a paper implies a declaration by the author that the work has not previously been published, that it is not being considered for publication elsewhere and that it has not already been considered by a different editor of the Journal. Note that a report on a conference must be submitted within 3 months of the meeting.

% \section{Types of paper}\label{sec:types_paper}
% \subsection{Standard papers}
% Regular submissions to JFM are termed `standard papers'. Note that such papers must contain original research, as review articles are not published in JFM. Papers should be written in a concise manner; though JFM has no page limit, each paper will be judged on its own merits, and those deemed excessive in length will be rejected or will require significant revision.  
% \subsection{JFM Rapids}
% {\it JFM Rapids} is devoted to the rapid publication of short, high-impact papers across the full range of fluid mechanics. Manuscripts submitted as {\it Rapids}  must be strictly 10 or fewer printed pages, and must be submitted in {\LaTeX} using the jfm.cls class file, so as to ensure that they meet the page limit and to expedite their production.  The principal and over-riding objective is to publish papers of the highest scientific quality. 

% Once a paper is submitted, referees are asked to provide reports with a short turnaround.  In order to be accepted for publication in {\it JFM Rapids}, such papers must require only minor revisions to improve clarity, usually without recourse to a second round of refereeing. In this case, and at the discretion of the editor, some additional pages may be allowed to address specific points raised by the referees, such as the addition of an extra figure or some explanatory text.  

% In cases where the editor, guided by the referees, judges that a paper has merit but requires substantial revision that will require significant refereeing, a decision of `revise and resubmit' will be given. On re-submission, such papers will be handled as standard JFM papers and if accepted will not subsequently appear as {\it JFM Rapids}.

% {\it Rapids} will be published online within one month of final acceptance.  They will appear within a designated section on the Journal of Fluid Mechanics website.  Each {\it Rapid} will be cited and indexed as a JFM article but with a distinctive {\it Rapids} identifier, and will be assigned to a JFM volume. {\it Rapids} will not be included in the regular print versions of JFM, and accordingly they incur no colour figure charges.
 
% \section{File types}\label{sec:filetypes}
% Authors are strongly encouraged to compose their papers in {\LaTeX}, using the jfm.cls style file and supporting files provided at\\ https://www.cambridge.org/core/journals/journal-of-fluid-mechanics/information/\\ instructions-contributors, with the jfm-instructions.tex file serving as a template (note that this is mandatory for {\it JFM Rapids}). A PDF of the {\LaTeX} file should then be generated and submitted via the submission site. Please note that PDFs larger than 10MB are not acceptable for review. There is no need to submit the {\LaTeX} source file alongside the PDF, but upon provisional acceptance of the paper, the {\LaTeX} source file, along with individual figure files and a PDF of the final version, will need to be submitted for typesetting purposes. 
% Authors may also compose standard papers in Word, though this will lead to the paper spending a longer period in production. If using Word, please note that equations must NOT be converted to picture format and the file must be saved with the option `make equation editable'. 
% All submitted video abstract files should be formatted as MP4 (H.264). MP4 has full compatibility across commonly used browsers, whereas other video formats will only work on selected browsers. This will ensure the greatest possible dissemination of this work. 
% \section{Preparing your manuscript}
% Authors should write their papers clearly and concisely in English, adhering to JFM's established style for notation, as provided in \S\ref{notstyle}. We encourage the submission of online supplementary material alongside the manuscript where appropriate (see section \ref{online}). Metric units should be used throughout and all abbreviations must be defined at first use, even those deemed to be well known to the readership. British spelling must be used, and should follow the \textit{Shorter Oxford English Dictionary}.


% \subsection{figures}
% All authors need to acquire the correct permissions and licences to reproduce figures. figures should be as small as possible while displaying clearly all the information required, and with all lettering readable. Every effort should be taken to avoid figures that run over more than one page. figures submitted in colour will appear online in colour but, with the exception of {\it JFM Rapids}, all figures will be printed in black and white unless authors specify during submission that figures should be printed in colour, for which there is a charge of \pounds200 plus VAT per figure (i.e. \pounds240) with a cap of \pounds1000 plus VAT per article (i.e. \pounds1200) (colour is free for {\it JFM Rapids}).  Note that separate figures for online and print will {\bf not} be accepted and it is the author's responsibility to ensure that if a figure is to appear in colour online only, that same figure will still be meaningful when printed in black and white (for example, do not rely upon colours to distinguish lines if those colours will just appear as similar shades of grey when printed). If using colour, authors should also ensure that some consistency is applied within the manuscript. For review purposes figures should be embedded within the manuscript. Upon final acceptance, however, individual figure files will be required for production. These should be submitted in EPS or high-resolution TIFF format (1200 dpi for lines, 300 dpi for halftone and colour in CMYK format, and 600 dpi for a mixture of lines and halftone). The minimum acceptable width of any line is 0.5pt. Each figure should be accompanied by a single caption, to appear beneath, and must be cited in the text. figures should appear in the order in which they are first mentioned in the text and figure files must be named accordingly (`figure 1.eps', `figure 2a.tiff', etc) to assist the production process (and numbering of figures should continue through any appendices). The word \textit {figure} is only capitalized at the start of a sentence. For example see figures \ref{fig:ka} and \ref{fig:kd}. Failure to follow figure guidelines may result in a request for resupply and a subsequent delay in the production process. Note that {\em all} figures are relabelled by the typesetter, therefore the files supplied should have editable labelling. Please ensure all figure labels are carefully checked against your originals when you receive your proofs. 

% % Figure environment removed

% % Figure environment removed

% \subsection{Tables}
% Tables, however small, must be numbered sequentially in the order in which they are mentioned in the text. The word \textit {table} is only capitalized at the start of a sentence. See table \ref{tab:kd} for an example.

% \begin{table}
%   \begin{center}
% \def~{\hphantom{0}}
%   \begin{tabular}{lccc}
%       $a/d$  & $M=4$   &   $M=8$ & Callan \etal \\[3pt]
%       0.1   & 1.56905 & ~~1.56~ & 1.56904\\
%       0.3   & 1.50484 & ~~1.504 & 1.50484\\
%       0.55  & 1.39128 & ~~1.391 & 1.39131\\
%       0.7   & 1.32281 & ~10.322 & 1.32288\\
%       0.913 & 1.34479 & 100.351 & 1.35185\\
%   \end{tabular}
%   \caption{Values of $kd$ at which trapped modes occur when $\rho(\theta)=a$}
%   \label{tab:kd}
%   \end{center}
% \end{table}

% \subsection{Online supplementary material}\label{online}
% Relevant material which is not suitable for print production, such as movies or numerical simulations/animations, can be uploaded as part of the initial submission. Movies should be designated as `Movie' and each individual file must be accompanied by a separate caption and a suitable title (eg Movie 1). Accepted formats are .mov, .mpg, .mp4, and .avi, though they should be archived as a .zip or .tar file before uploading. Each movie should be no more than 10MB. Upon publication these materials will then be hosted online alongside the final published article. Likewise, should there be detailed mathematical relations, tables or figures which are likely to be useful only to a few specialists or take up excessive space in the printed journal, these can also be published online as supplementary material [designated as `Other supplementary material']. Note that supplementary material is published `as is', with no further production performed.

% \section{Editorial decisions}
% \subsection{Revision}
% If a revision is requested, you should upload revised files following the same procedure as for submitting a new paper. You begin by clicking on `Manuscripts with decision' in your Corresponding Author Center, and then on `Create a revision'. (Note that if you abandon the process before completing the submission, to continue the submission, you must click on `Revised manuscripts in draft'.) There is a new first page showing the decision letter and a space for your reply to the referee's/editor's comments. You also have the opportunity at this stage to upload your reply to the comments as a separate file. All the values filled in on original submission are displayed again. The ID number of the paper will be appended `.R1'. Also note that if a manuscript is submitted as a {\it JFM Rapid}, but requires substantial revision, it will be re-designated as a standard paper, and the ID and paper type will be amended to reflect this.

% \subsection{Provisional acceptance}
% If the paper is accepted as suitable for publication you will be sent a provisional acceptance decision. This enables you to upload the final files required for production:
% (1) the final PDF or word version of the paper, designated as a `main document';
% (2) any source files (see section \ref{sec:filetypes}) which must be designated as `production (not for review)' and uploaded as a single .zip or .tar file.

% \subsection{Acceptance}
% On receipt of the production files you will be sent an email indicating completion of the acceptance process.

% \section{Publication process}
% Once a paper has been accepted for publication and the source files have been uploaded, the manuscript will be sent to Cambridge University Press for copyediting and typesetting, and will be assigned a digital object identifier (doi). When the proof is ready, authors will receive an email alert containing a link to the PDF of the proof, and instructions for its correction and return. It is imperative that authors check their proofs closely, particularly the equations and figures, which should be checked against the accepted file, as the production schedule does not allow for corrections at a later stage. Once ready for printing, papers will be published online on Cambridge Journals Online in the current `open' volume. {\it JFM Rapids} will also be assigned to the current open volume, but will receive an article number in place of traditional pagination, as they will not appear in the print version of that volume. Each volume will be kept open for approximately two weeks, at which point it will be considered `closed', and then printed and distributed. The following volume will be immediately opened. Note that the PDF published online is the Version of Record, matching the print version, and no further alterations/corrections to this document will be allowed. The corresponding author is emailed a link to the published article when it is first published online.

% \section{Corrigenda}
% The Journal will publish corrigenda that alter significant conclusions made in a paper.  Such corrigenda should be submitted to an associate editor, who will consider the submission similarly to a new paper and may consult referees if appropriate.  When published, corrigenda are clearly linked with the original articles to which they refer, and the articles to them.

% The Journal does not normally publish corrigenda to amend typographical errors, so it is extremely important that authors make very careful checks of their manuscript at every stage, including the reading of proofs, prior to publication.



% \section{Obtaining help}
% Technical support for the online submission system is available by clicking on the `Get Help Now' link at the top-right corner of each page of the submission site. Any other questions relating to the submission or publication process should be directed to the JFM Editorial Assistant, Mrs Amanda Johns, at ajohns@cambridge.org.

% \section{Cambridge Language Editing Service} 
% We suggest that authors whose first language is not English have their manuscripts checked by a native English speaker before submission. This is optional but will help to ensure that any submissions that reach peer review can be judged exclusively on academic merit. We offer a Cambridge service which you can find out more about at https://www.cambridge.org/core/services/authors/language-services, and suggest that authors make contact as appropriate. Please note that use of language editing services is voluntary and at the author’s own expense. Use of these services does not guarantee that the manuscript will be accepted for publication nor does it restrict the author to submitting to a Cambridge-published journal


% \section{Notation and style}\label{notstyle}
% Generally any queries concerning notation and journal style can be answered by viewing recent pages in the Journal. However, the following guide provides the key points to note. It is expected that Journal style will be followed, and authors should take care to define all variables or entities upon first use. Also note that footnotes are not normally accepted.

% \subsection{Mathematical notation}
% \subsubsection{Setting variables, functions, vectors, matrices etc}
% {\bf Italic font} should be used for denoting variables, with multiple-letter symbols avoided except in the case of dimensionless numbers such as $\Rey$, $\Pran$ and $\Pen$ (Reynolds, Prandtl, and P\'eclet numbers respectively, which are defined as \verb}\Rey}, \verb}\Pran} and {\verb}\Pen} in the template).\\
% {\bf Upright Roman font} (or upright Greek where appropriate) should be used for:\\
% Operators: sin, log, d, $\Delta$, e etc.\\
% Constants: i ($\sqrt{-1}$), $\upi$ (defined as \verb}\upi}), etc.\\
% Functions: $\Ai$, $\Bi$ (Airy functions, defined as \verb}\Ai} and \verb}\Bi}), $\Real$ (real part, defined as \verb}\Real}), $\Imag$ (imaginary part, defined as \verb}\Imag}), etc.\\
% Physical units: cm, s, etc\\
% Abbreviations: c.c. (complex conjugate), h.o.t. (higher-order terms), DNS, etc.\\
% {\bf Bold italic font} (or bold sloping Greek) should be used for:\\
% Vectors (with the centred dot for a scalar product also in bold): $\boldsymbol{i \cdot j}$\\
% {\bf Bold sloping sans serif font}, defined by the \verb}\mathsfbi} macro, should be used for:\\
% Tensors and matrices: $\mathsfbi{D}$ \\
% {\bf Script font} (for example $\mathcal{G}$, $\mathcal{R}$) can be used as an alternative to italic when the same letter denotes a different quantity (use \verb}\mathcal} in \LaTeX)\\
% The product symbol ($\times$) should only be used to denote multiplication where an equation is broken over more than one line, to denote a cross product, or between numbers (the $\cdot$ symbol should not be used, except to denote a scalar product specifically).\\ 

% \subsubsection{Other symbols}
% A centred point should be used only for the scalar product of vectors.
% Large numbers that are not scientific powers should not include commas, but have the
% form 1600 or 16 000 or 160 000.
% Use \textit{O} to denote `of the order of', not the \LaTeX\ $\mathcal{O}$.

% \section{Citations and references}
% All papers included in the References section must be cited in the article, and vice versa. Citations should be included as, for example ``It has been shown \citep{Rogallo81} that...'' (using the {\verb}\citep} command, part of the natbib package) ``recent work by \citet{Dennis85}...'' (using {\verb}\citet}).
% The natbib package can be used to generate citation variations, as shown below.\\
% \verb#\citet[pp. 2-4]{Hwang70}#:\\
% \citet[pp. 2-4]{Hwang70} \\
% \verb#\citep[p. 6]{Worster92}#:\\
% \citep[p. 6]{Worster92}\\
% \verb#\citep[see][]{Koch83, Lee71, Linton92}#:\\
% \citep[see][]{Koch83, Lee71, Linton92}\\
% \verb#\citep[see][p. 18]{Martin80}#:\\
% \citep[see][p. 18]{Martin80}\\
% \verb#\citep{Brownell04,Brownell07,Ursell50,Wijngaarden68,Miller91}#:\\
% \citep{Brownell04,Brownell07,Ursell50,Wijngaarden68,Miller91}\\
% The References section can either be built from individual \verb#\bibitem# commands, or can be built using BibTex. The BibTex files used to generate the references in this document can be found in the zip file at http://journals.cambridge.org/\linebreak[3]data/\linebreak[3]relatedlink/\linebreak[3]jfm-ifc.zip.\\
% Where there are up to ten authors, all authors' names should be given in the reference list. Where there are more than ten authors, only the first name should appear, followed by et al.\\

% Acknowledgements should be included at the end of the paper, before the References section or any appendicies, and should be a separate paragraph without a heading. Several anonymous individuals are thanked for contributions to these instructions.

% \appendix
% \section{}\label{appA}
% This appendix contains sample equations in the JFM style. Please refer to the {\LaTeX} source file for examples of how to display such equations in your manuscript.

% \begin{equation}
%   (\nabla^2+k^2)G_s=(\nabla^2+k^2)G_a=0
%   \label{Helm}
% \end{equation}

% \begin{equation}
%   \bnabla\bcdot\boldsymbol{v} = 0,\quad \nabla^{2}P=
%     \bnabla\bcdot(\boldsymbol{v}\times \boldsymbol{w}).
% \end{equation}

% \begin{equation}
%   G_s,G_a\sim 1 / (2\upi)\ln r
%   \quad \mbox{as\ }\quad r\equiv|P-Q|\rightarrow 0,
%   \label{singular}
% \end{equation}

% \begin{equation}
% \left. \begin{array}{ll}  
% \displaystyle\frac{\p G_s}{\p y}=0
%   \quad \mbox{on\ }\quad y=0,\\[8pt]
% \displaystyle  G_a=0
%   \quad \mbox{on\ }\quad y=0,
%  \end{array}\right\}
%   \label{symbc}
% \end{equation}


% \begin{equation}
%   -\frac{1}{2\upi} \int_0^{\infty} \gamma^{-1}[\mathrm exp(-k\gamma|y-\eta|)
%   + \mathrm exp(-k\gamma(2d-y-\eta))] \cos k(x-\xi)t\:\mathrm{d} t,
%   \qquad 0<y,\quad \eta<d,
% \end{equation}

% \begin{equation}
%   \gamma(t) = \left\{
%     \begin{array}{ll}
%       -\mathrm{i}(1-t^2)^{1/2}, & t\le 1 \\[2pt]
%       (t^2-1)^{1/2},         & t>1.
%     \end{array} \right.
% \end{equation}

% \[
%   -\frac{1}{2\upi}
%   \pvi B(t)\frac{\cosh k\gamma(d-y)}{\gamma\sinh k\gamma d}
%   \cos k(x-\xi)t\:\mathrm{d} t
% \]

% \begin{equation}
%   G = -\frac{1}{4}\mathrm{i} (H_0(kr)+H_0(kr_1))
%     - \frac{1}{\upi} \pvi\frac{\mathrm{e}^{-\kgd}}%
%     {\gamma\sinh\kgd} \cosh k\gamma(d-y) \cosh k\gamma(d-\eta)
% \end{equation}

% Note that when equations are included in definitions, it may be suitable to render them in line, rather than in the equation environment: $\boldsymbol{n}_q=(-y^{\prime}(\theta),
% x^{\prime}(\theta))/w(\theta)$.
% Now $G_a=\squart Y_0(kr)+\Gat$ where
% $r=\{[x(\theta)-x(\psi)]^2 + [y(\theta)-y(\psi)]^2\}^{1/2}$ and $\Gat$ is
% regular as $kr\ttz$. However, any fractions displayed like this, other than $\thalf$ or $\squart$, must be written on the line, and not stacked (ie 1/3).
 
% \begin{eqnarray}
%   \ndq\left(\frac{1}{4} Y_0(kr)\right) & \sim &
%     \frac{1}{4\upi w^3(\theta)}
%     [x^{\prime\prime}(\theta)y^{\prime}(\theta)-
%     y^{\prime\prime}(\theta)x^{\prime}(\theta)] \nonumber\\
%   & = & \frac{1}{4\upi w^3(\theta)}
%     [\rho^{\prime}(\theta)\rho^{\prime\prime}(\theta)
%     - \rho^2(\theta)-2\rho^{\prime 2}(\theta)]
%     \quad \mbox{as\ }\quad kr\ttz . \label{inteqpt}
% \end{eqnarray}

% \begin{equation}
%   \frac{1}{2}\phi_i = \frac{\upi}{M} \sumjm\phi_j K_{ij}^a w_j,
%   \qquad i=1,\,\ldots,\,M,
% \end{equation}
% where
% \begin{equation}
%   K_{ij}^a = \left\{
%     \begin{array}{ll}
%       \p G_a(\theta_i,\theta_j)/\p n_q, & i\neq j \\[2pt]
%       \p\Gat(\theta_i,\theta_i)/\p n_q
%       + [\rho_i^{\prime}\rho_i^{\prime\prime}-\rho_i^2-2\rho_i^{\prime 2}]
%       / 4\upi w_i^3, & i=j.
%   \end{array} \right.
% \end{equation}


% \refstepcounter{equation}
% $$
%   \rho_l = \lim_{\zeta \rightarrow Z^-_l(x)} \rho(x,\zeta), \quad
%   \rho_{u} = \lim_{\zeta \rightarrow Z^{+}_u(x)} \rho(x,\zeta)
%   \eqno{(\theequation{\mathit{a},\mathit{b}})}\label{eq35}
% $$

% \begin{equation}
%   (\rho(x,\zeta),\phi_{\zeta\zeta}(x,\zeta))=(\rho_0,N_0)
%   \quad \mbox{for}\quad Z_l(x) < \zeta < Z_u(x).
% \end{equation}


% \begin{subeqnarray}
%   \tau_{ij} & = &
%     (\overline{\overline{u}_i \overline{u}_j}
%     - \overline{u}_i\overline{u}_j)
%     + (\overline{\overline{u}_iu^{SGS}_j
%     + u^{SGS}_i\overline{u}_j})
%     + \overline{u^{SGS}_iu^{SGS}_j},\\[3pt]
%   \tau^\theta_j & = &
%     (\overline{\overline{u}_j\overline{\theta}}
%     - \overline{u}_j \overline{\theta})
%     + (\overline{\overline{u}_j\theta^{SGS}
%     + u^{SGS}_j \overline{\theta}})
%     + \overline{u^{SGS}_j\theta^{SGS}}.
% \end{subeqnarray}

% \begin{equation}
% \setlength{\arraycolsep}{0pt}
% \renewcommand{\arraystretch}{1.3}
% \slsQ_C = \left[
% \begin{array}{ccccc}
%   -\omega^{-2}V'_w  &  -(\alpha^t\omega)^{-1}  &  0  &  0  &  0  \\
%   \displaystyle
%   \frac{\beta}{\alpha\omega^2}V'_w  &  0  &  0  &  0  &  \mathrm{i}\omega^{-1} \\
%   \mathrm{i}\omega^{-1}  &  0  &  0  &  0  &  0  \\
%   \displaystyle
%   \mathrm{i} R^{-1}_{\delta}(\alpha^t+\omega^{-1}V''_w)  &  0
%     & -(\mathrm{i}\alpha^tR_\delta)^{-1}  &  0  &  0  \\
%   \displaystyle
%   \frac{\mathrm{i}\beta}{\alpha\omega}R^{-1}_\delta V''_w  &  0  &  0
%     &  0  & 0 \\
%   (\mathrm{i}\alpha^t)^{-1}V'_w  &  (3R^{-1}_{\delta}+c^t(\mathrm{i}\alpha^t)^{-1})
%     &  0  &  -(\alpha^t)^{-2}R^{-1}_{\delta}  &  0  \\
% \end{array}  \right] .
% \label{defQc}
% \end{equation}

% \begin{equation}
% \etb^t = \skew2\widehat{\etb}^t \exp [\mathrm{i} (\alpha^tx^t_1-\omega t)],
% \end{equation}
% where $\skew2\widehat{\etb}^t=\boldsymbol{b}\exp (\mathrm{i}\gamma x^t_3)$. 
% \begin{equation}
% \mbox{Det}[\rho\omega^2\delta_{ps}-C^t_{pqrs}k^t_qk^t_r]=0,
% \end{equation}

% \begin{equation}
%  \langle k^t_1,k^t_2,k^t_3\rangle = \langle
% \alpha^t,0,\gamma\rangle  
% \end{equation}

% \begin{equation}
% \boldsymbol{f}(\theta,\psi) = (g(\psi)\cos \theta,g(\psi) \sin \theta,f(\psi)).
% \label{eq41}
% \end{equation}

% \begin{eqnarray}
% f(\psi_1) = \frac{3b}{\upi[2(a+b \cos \psi_1)]^{{3}/{2}}}
%   \int^{2\upi}_0 \frac{(\sin \psi_1 - \sin \psi)(a+b \cos \psi)^{1/2}}%
%   {[1 - \cos (\psi_1 - \psi)](2+\alpha)^{1/2}}\mathrm{d}x,
% \label{eq42}
% \end{eqnarray}
% \begin{eqnarray}
% g(\psi_1) & = & \frac{3}{\upi[2(a+b \cos \psi_1)]^{{3}/{2}}}
%   \int^{2\upi}_0 \left(\frac{a+b \cos \psi}{2+\alpha}\right)^{1/2}
%   \left\{ \astrut f(\psi)[(\cos \psi_1 - b \beta_1)S + \beta_1P]
%   \right. \nonumber\\
% && \mbox{}\times \frac{\sin \psi_1 - \sin \psi}{1-\cos(\psi_1 - \psi)}
%   + g(\psi) \left[\left(2+\alpha - \frac{(\sin \psi_1 - \sin \psi)^2}
%   {1- \cos (\psi - \psi_1)} - b^2 \gamma \right) S \right.\nonumber\\
% && \left.\left.\mbox{} + \left( b^2 \cos \psi_1\gamma -
%   \frac{a}{b}\alpha \right) F(\frac{1}{2}\upi, \delta) - (2+\alpha)
%   \cos\psi_1 E(\frac{1}{2}\upi, \delta)\right] \astrut\right\} \mathrm{d} \psi,
% \label{eq43}
% \end{eqnarray}
% \begin{equation}
% \alpha = \alpha(\psi,\psi_1) = \frac{b^2[1-\cos(\psi-\psi_1)]}%
%   {(a+b\cos\psi) (a+b\cos\psi_1)},
%   \quad
%   \beta - \beta(\psi,\psi_1) = \frac{1-\cos(\psi-\psi_1)}{a+b\cos\psi}.
% \end{equation}


% \begin{equation}
% \left. \begin{array}{l}
% \displaystyle
% H(0) = \frac{\epsilon \overline{C}_v}{\tilde{v}^{{1}/{2}}_T
% (1- \beta)},\quad H'(0) = -1+\epsilon^{{2}/{3}} \overline{C}_u
% + \epsilon \skew5\widehat{C}_u'; \\[16pt]
% \displaystyle
% H''(0) = \frac{\epsilon u^2_{\ast}}{\tilde{v}^{{1}/{2}}
% _T u^2_P},\quad H' (\infty) = 0.
% \end{array} \right\}
% \end{equation}

% \begin{lemma}
% Let $f(z)$ be a trial \citet[][pp.~231--232]{Batchelor59} function defined on $[0,1]$.  Let $\varLambda_1$ denote
% the ground-state eigenvalue for $-\mathrm{d}^2g/\mathrm{d} z^2=\varLambda g$,
% where $g$ must satisfy $\pm\mathrm{d} g/\mathrm{d} z+\alpha g=0$ at $z=0,1$
% for some non-negative constant~$\alpha$.  Then for any $f$ that is not
% identically zero we have
% \begin{equation}
% \frac{\displaystyle
%   \alpha(f^2(0)+f^2(1)) + \int_0^1 \left(
%   \frac{\mathrm{d} f}{\mathrm{d} z} \right)^2 \mathrm{d} z}%
%   {\displaystyle \int_0^1 f^2\mathrm{d} z}
% \ge \varLambda_1 \ge
% \left( \frac{-\alpha+(\alpha^2+8\upi^2\alpha)^{1/2}}{4\upi} \right)^2.
% \end{equation}
% \end{lemma}

% \begin{corollary}
% Any non-zero trial function $f$ which satisfies the boundary condition
% $f(0)=f(1)=0$ always satisfies
% \begin{equation}
%   \int_0^1 \left( \frac{\mathrm{d} f}{\mathrm{d} z} \right)^2 \mathrm{d} z.
% \end{equation}
% \end{corollary}

\bibliographystyle{jfm}
% Note the spaces between the initials
\bibliography{MFM_for_RTI}
    
\end{document}
