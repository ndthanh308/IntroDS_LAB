% ****** Start of file aipsamp.tex ******
%
%   This file is part of the AIP files in the AIP distribution for REVTeX 4.
%   Version 4.1 of REVTeX, October 2009
%
%   Copyright (c) 2009 American Institute of Physics.
%
%   See the AIP README file for restrictions and more information.
%
% TeX'ing this file requires that you have AMS-LaTeX 2.0 installed
% as well as the rest of the prerequisites for REVTeX 4.1
% 
% It also requires running BibTeX. The commands are as follows:
%
%  1)  latex  aipsamp
%  2)  bibtex aipsamp
%  3)  latex  aipsamp
%  4)  latex  aipsamp
%
% Use this file as a source of example code for your aip document.
% Use the file aiptemplate.tex as a template for your document.
\documentclass[%
 aip,
% jmp,
% bmf,
% sd,
% rsi,
 amsmath,amssymb,
preprint,%
 %reprint,%
%author-year,%
%author-numerical,%
% Conference Proceedings
]{revtex4-1}



\usepackage[ruled, linesnumbered]{algorithm2e}
%\usepackage{algorithm}% http://ctan.org/pkg/algorithms
\usepackage{algorithmicx}
\usepackage{algcompatible}
\usepackage{algpseudocode}% http://ctan.org/pkg/algorithmicx
\usepackage{amsmath}
\usepackage{bm}% bold math
\usepackage{color}
%\usepackage{dcolumn}% Align table columns on decimal point
\usepackage{etoolbox}
\usepackage{float}
\usepackage[T1]{fontenc}
\usepackage{graphicx}% Include figure files
%\usepackage[mathlines]{lineno}% Enable numbering of text and display math
%\linenumbers\relax % Commence numbering lines
\usepackage[utf8]{inputenc}
\usepackage{mathptmx}
\usepackage{multirow}
\usepackage{stmaryrd}  % to denote jump in formulations
\usepackage[caption = false]{subfig}

%% Apr 2021: AIP requests that the corresponding 
%% email to be moved after the affiliations
\makeatletter
\def\@email#1#2{%
 \endgroup
 \patchcmd{\titleblock@produce}
  {\frontmatter@RRAPformat}
  {\frontmatter@RRAPformat{\produce@RRAP{*#1\href{mailto:#2}{#2}}}\frontmatter@RRAPformat}
  {}{}
}%
\makeatother

\begin{document}

\preprint{AIP/123-QED}

\title[Macroscopic data extraction from microscopic fields]{A method to extract macroscopic interface data from microscale rough/porous wall flow fields}
% Force line breaks with \\
\author{Vedanth N Kuchibhotla}
\affiliation{ 
School of Mechanical Sciences, Indian Institute of Technology Goa, Farmagudi, Goa -- 403 401.
}%
\author{Sujit Kumar Sahoo}
\email{sujit@iitgoa.ac.in}
\affiliation{ 
School of Electrical Sciences, Indian Institute of Technology Goa, Farmagudi, Goa -- 403 401.
}%
\author{Y. Sudhakar}%
 \email{sudhakar@iitgoa.ac.in}
\affiliation{ 
School of Mechanical Sciences, Indian Institute of Technology Goa, Farmagudi, Goa -- 403 401.
}%

\date{\today}% It is always \today, today,
             %  but any date may be explicitly specified

\begin{abstract}
Performing geometry-resolved simulations of flows over rough and porous walls is highly expensive due to their multiscale characteristics. Effective models that circumvent this difficulty are often used to investigate the interaction between the free-fluid and such complex walls. These models, by construction, employ an intrinsic averaging process and capture only macroscopic physical processes. However, physical experiments or direct simulations yield micro- and macroscale information, and isolating the macroscopic effect from them is crucial for rigorously validating the accuracy of effective models. Despite the increasing use of effective models, this aspect received the least attention in the literature. This paper presents an efficient averaging technique to extract macroscopic interface data from the flowfield obtained via direct simulations or physical experiments. Compared to the ensemble averaging, the proposed method, while retaining accuracy, is cost-effective for rough and porous walls. To the best of our knowledge, this is the only averaging method that works for poroelastic walls, for which the ensemble averaging fails. Moreover, it applies equally to viscous- and inertia-dominated flows over irregular surfaces.

%The definition of the averaging mentioned in equation~\eqref{eqn:decomp} is a delicate one; it can dictate both the computational effort and the accuracy of obtaining macroscopic data. The objective of this work is to devise an efficient method,  based on the knowledge from signal processing, to extract $\overline{Q}$ from $Q$. While the ensemble averaging, the only procedure reported in the literature for multidimensional flows over irregular surfaces, requires a large number of samples, the present method can yield accurate macroscopic data by using only 2 samples.

\end{abstract}

\maketitle

\section{Introduction}
\label{sec:intro}
%Surfaces found in biological systems are covered with porous or rough elements, whose function varies from providing thermal insulation [Ref] to drag reduction [Ref] and lift enhancement~\cite{kovalev2008}. Owing to the interest in developing biomimetic applications, understanding interaction between such irregular surfaces and the surrounding fluid flow has received considerable research attention in recent years.  A noteworthy example is riblets found on shark skins; the rough features help sharks to reduce the skin-friction drag induced by the surrounding fluid flow. A riblet-inspired drag reduction mechanism has been successfully developed, which yielded a X\% drag reduction on a passenger airplane.  For a comprehensive overview of surface features and their biomimetic applications,  readers can refer to the review article by Bottaro~\cite{bottaro2019}.

%The potential application of these complex surfaces motivated various experimental [ref], analytical [ref] and numerical [ref] studies aimed at either understanding or developing predicting models for transport phenomena associated with them.  Performing geometry-resolved simulations of fluid flows in such configurations is formidable because of the associated multiscale nature.

%\textcolor{blue}{This scale disparity, given by the scale separation parameter $\epsilon(=l/H)\ll 1$,  and the subsequent prohibitive computational costs have led to the development of effective or homogenized models. Such models aim at capturing the macroscopic behaviour of the system by intrinsically averaging out of the microscale variation.  This means the model captures only those events occurring on the scale of $H$ by discarding the physical processes on the scale of $l$.}

Fluid flows over rough, porous and poroelastic walls are prevalent in nature~\cite{bottaro2019}. Understanding heat and fluid transport associated with such irregular walls is essential for developing biomimetic engineering devices. A well-known example is the development of skin-friction-reducing devices, called riblets, inspired by the denticles found on shark skins~\cite{garcia2011}. Optimal riblet geometries can yield skin-friction drag reduction of up to 10\% in turbulent flows~\cite{bechert1997}.

The characteristic feature of the aforementioned irregular surfaces is that the geometric microscale $l$ can be orders of magnitude smaller than the macroscale $H$ relevant for the surrounding fluid flow. This multiscale nature of the configuration, quantified by the scale separation parameter $\eta(=l/H)\ll 1$, makes geometry-resolved numerical simulations (called DNS from hereafter) practically impossible. This paper uses the terminology of a complex or irregular surface to denote a solid wall coated with rough, porous or poroelastic features.

Due to the practical difficulty of performing DNS, effective models are widely employed to study the flow physics associated with rough/porous walls~\cite{rosti2018,khorasani2022}.  Beavers and Joseph~\cite{beavers1967}, based on their experiments on a channel flow over a porous block, proposed an empirical shear-dependent slip velocity boundary condition at the fluid-porous interface. This is one of the first effective models proposed. Due to the practical importance of flows over irregular walls, various improved models are rigorously derived based on  multiscale homogenisation, volume-averaging, or other frameworks.
%While most of the effective models are applicable only for Stokes flow over irregular surfaces, some models also consider the inertial effects. 

Effective models are the only viable means of numerically studying fluid flows over irregular surfaces.  They describe the interaction between the irregular surface and the surrounding fluid flow via coupling conditions specified at a chosen \textit{nominal} interface.  The accuracy of an effective model is dictated by how closely these interface conditions capture the underlying transport phenomena across the interface.  A crucial step in developing an effective model is to test the accuracy of interface conditions by comparing the model predictions against results obtained from DNS or physical experiments.  A major challenge in carrying out such a validation study is as follows: the effective models capture only the macroscopic effects, while DNS/experiments provide data that contain microscopic fields in addition to macroscopic effects.  As highlighted in section~\ref{sec:problem}, this mandates that an appropriate averaging procedure be applied to the DNS/experimental data to isolate the macroscopic effects.  Despite the recent surge of interest in this subject, this aspect has received the least attention in the literature. This paper presents an efficient approach to address this issue. 

The averaging process has received the least attention in the literature because most existing studies considered a laminar~\cite{jager2001,goyeau2003, tachie2004,breugem2005,deng2005, agelinchaab2006,chandesris2006,zhang2009,liu2011,carraro2013,wu2018,lu2019,terzis2019,lu2020a,lu2020b,strohbeck2023} or turbulent~\cite{chandesris2009,kuwata2017,suga2018,chen2021,chu2021}  channel flow over a rough/porous medium.  In such a configuration, the macroscopic flow is unidirectional because the averaged velocity in the interface-normal direction is zero everywhere.  From the microscopic point of view, what happens in a single interface periodic unit is repeated for all units along the interface.  From the macroscopic point of view, the flow is homogeneous along the interface. The averaging procedure in such a configuration is straightforward: performing a \textit{single} DNS over a domain consisting of a \textit{single} roughness element and averaging the pressure/velocity along the flow direction is sufficient to obtain macroscopic fields in a laminar flow; for turbulent flows, the time-averaged fields can be averaged along the interface, due to the spatial homogeneity, to obtain macroscopic fields.

Extracting macroscopic data from DNS/experimental results for multidimensional flows over irregular surfaces is not straightforward. The work of Ugis \& Shervin~\cite{lacis2016} is one of the first to compare DNS and a homogenized model for two-dimensional flows over a porous bed. Due to the non-availability of a reliable averaging process, they reported a comparison of the model against non-averaged DNS results. Another study on the flow over poroelastic media~\cite{lacis2017} also presented such comparisons. Due to the presence of microscopic oscillations in DNS,  rigorous quantification of the accuracy of the homogenized model was not possible from such comparisons.

A few studies have employed different procedures to obtain macroscopic fields from DNS.  Breugem \& Boersma~\cite{breugem2005} presented a comprehensive analysis of the turbulent flow over a permeable wall and compared the DNS results against those of an effective model. A volume averaging technique involving a weight function is used to obtain the macroscopic mean and turbulent fluctuating quantities.  Chandesris et al.~\cite{chandesris2013} used a similar averaging but adopted a weight function with increased regularity.
An investigation by Nair et al.~\cite{nair2018} focused on laminar flows over a porous flat plate employed averaging over a representative volume element~(REV).  The size of the REV equal to $20D$, where $D$ is the diameter of the circular solid inclusions used to represent the porous medium, was chosen to obtain macroscopic results from experiments and simulations.  Experimental studies on porous media~\cite{terzis2019,yang2019} used surface and volume averaging, respectively, to obtain macroscopic interface and interior quantities.
A recent study of natural convection flow over a rough vertical wall used a moving averaging technique\cite{ahmed2022}.  

Ugis et al.~\cite{lacis2020} utilized an ensemble averaging procedure to extract averaged fields from DNS results. This method differs from the volume averaging technique or surface averaging over the interface, used earlier. Samples needed for the ensemble averaging are obtained by displacing the solid skeleton parallel to the interface by a uniform distance. This ensures that the generated samples cover the span of one microscopic length scale. This procedure's main feature is that it directly applies to multidimensional flows. This averaging method is subsequently used in validating the effective models formulated in Cartesian~\cite{sudhakar2021} and Polar~\cite{jain2022} coordinates. Although this averaging process yielded smooth macroscopic fields even for multidimensional flows over irregular surfaces, generating samples for the ensemble averaging is daunting because each sample data requires a full DNS. This proves that the averaging is a computationally costly procedure. Our tests indicate that at least ten samples for each interface dimension are needed to eliminate the microscopic oscillations completely. While addressing three-dimensional flows over a complex surface, the computational cost becomes prohibitive because 100 expensive geometry-resolved simulations would be required. Inertia-dominated flows add additional complexities to this averaging procedure since they require longer time and more computational resources, owing to the underlying nonlinear nature of the governing PDEs.

In this paper, we present a computationally feasible procedure to extract macroscopic data at a nominal interface between an irregular surface and the surrounding two-dimensional free fluid flow.  We make use of the concepts of signal processing and polynomial interpolation to develop this method.
In contrast to the large number of samples required for ensemble averaging, the present method requires only two samples to provide accurate results. Thus the method is efficient when compared to the ensemble averaging. As shown in a later section, the proposed method works well for flows over poroelastic surfaces exhibiting large deformations, for which the ensemble averaging can not be applied. Moreover, we provide the source code and relevant data reported in this work in a public repository~\cite{bitbucket}.

This paper is organized as follows: In section~\ref{sec:problem}, we describe the appearance of microscopic oscillations in DNS,  and the importance of the averaging process for flows over complex walls. The methodology and the algorithm are detailed in \S~\ref{sec:method}. In section~\ref{sec:results}, for flows over rough, porous and poroelastic surfaces, results on extracting macroscopic data from DNS using the present method are presented; results obtained from our approach are compared to those from the ensemble averaging technique. Conclusions drawn from our study are summarised in \S~\ref{sec:conc}.
%===============================================================
\section{Problem definition}
\label{sec:problem}
This section details the appearance of microscopic oscillations in DNS and the necessity of performing the averaging process to extract the macroscopic data. Let us consider a steady, laminar flow over an ordered porous medium, as shown in figure~\ref{fig:probdef}(a). The circular inclusions represent the solid portion of the porous domain. Here, the microscale $l$ denotes the length scale characterising the pore structure, and the characteristic length scale of the free-flow is $H$ (denoted in this paper as macroscale). The microscopic description can be obtained by performing DNS or conducting physical experiments. Such a description will yield both macroscopic as well as microscopic fields. This is explained as follows.

% Figure environment removed

Let us assume that we are releasing a massless particle P in the domain at a point close to the interface between the free-fluid and the porous part. When it travels and approaches the fore surface of a circular inclusion, the particle experiences a positive vertical velocity (point A in figure~\ref{fig:probdef}a). The velocity component becomes negative when the particle moves past the aft surface of the inclusion (point B in figure~\ref{fig:probdef}a). The particle must have a zero vertical velocity between these two points. The vertical velocity's ``cyclic'' behaviour is repeated as it flows over each solid inclusion. This is the source of the appearance of the microscale oscillations, whose wavelength is always $l$. In addition, the particle itself travels within the macroscopic velocity field generated in the free fluid flow. Hence, a measurement of the vertical velocity of the particle in the microscopic description will contain both macroscopic and microscopic contributions, as shown in figure~\ref{fig:probdef}(b). Such an observation is valid for velocity components, pressure, and other variables in a fluid flow. Although the above illustration is presented for a porous wall, it also directly applies to rough and poroelastic surfaces.

Geometry-resolved simulations of flows over rough/porous walls are costly due to the large disparity of length scales quantified by the scale separation parameter. Volume-averaging or homogenization-based effective models, as shown in figure~\ref{fig:probdef}(e), are widely used to overcome this difficulty. Contrary to DNS, these models do not resolve the pore-scale geometry in the simulations. The fluid flow through the porous medium is represented by upscaled models~\cite{lage1998} like the Darcy, Darcy-Brinkmann, Forchheimer, or power law formulation. The interaction between the porous medium and the surrounding free-fluid region is modelled by specifying appropriate conditions at a nominal interface, as shown in figure~\ref{fig:probdef}(e). Effective models employ an intrinsic averaging procedure, and therefore by construction, they can capture only the macroscopic behaviour: the vertical velocity of the same particle mentioned above will appear as a smooth curve without microscopic variations, as shown in~figure~\ref{fig:probdef}(d).

The accuracy of any effective model is dictated by how accurately the respective interface conditions capture the transport phenomena across the nominal interface. Existing studies~\cite{lacis2020,sudhakar2021,jain2022} quantify the accuracy by comparing the interface quantities predicted by the model with those from DNS. Due to the widespread use of effective models, investigating their validity is crucial before using them in practical applications. In order to aid this validation, it is of utmost importance to devise a procedure to extract the macroscopic data (figure~\ref{fig:probdef}d) from the microscopic fields obtained via DNS/experiments (figure~\ref{fig:probdef}b).

The data obtained from DNS/experiments is of complex nature that they contain microscopic oscillations superimposed on the macroscopic field data. We can decompose a microscopic quantity~($Q$) into macroscopic~($\overline{Q}$) and fluctuations~($Q^\prime$), as shown below
\begin{equation}
Q(x,X) = \overline{Q}(X) + Q^\prime (x,X)\ \ \ \ \ \ \ \textrm{with\ \ \ \ \ }\overline{Q^\prime}=0,
\label{eqn:decomp}
\end{equation}
which is analogous to the Reynolds decomposition used in turbulent flows~\cite{davidson2015}.  However, contrary to turbulent fluctuations that are disordered and chaotic, the microscopic fluctuations appearing here are ordered in nature. Here, $Q$ can be any variable associated with the fluid flow. $x$ and $X$ represent fast and slow scales, respectively, in the context of multiscale homogenization~\cite{meibook}. $\overline{Q}$ contains the effect of microscale surface features on the length scale~($H$) of our interest, while details at the much smaller scales~($l$) are averaged out. Another point worthy of mentioning is the following. While the wavelength of the microscale field is $l$, its magnitude depends on the flow details. This explains why $Q^\prime$ is a function of both $x$ and $X$. Figure~\ref{fig:probdef}(c) illustrates the decomposition and underlines the above points.

The definition of the averaging mentioned in equation~\eqref{eqn:decomp} is delicate; it can dictate both the computational effort and the accuracy of obtaining macroscopic data. This work aims to devise an efficient method,  based on the knowledge from signal processing, to extract $\overline{Q}$ from $Q$. While ensemble averaging, the only procedure reported in the literature for multidimensional flows over irregular surfaces, requires many samples, the present method can yield accurate macroscopic data using only two samples.
%===============================================================
\section{Methodology}
\label{sec:method}
In the previous section, we have introduced the decomposition of the microscopic quantity~($Q$) into macroscopic~($\overline{Q}$) and fluctuations~($Q^\prime$), as given by equation~\eqref{eqn:decomp}. In this work, we make use of additional observations to develop a simple averaging methodology and an associated algorithm to isolate the macroscopic effect from the data obtained via either DNS or experiments. These details are elaborated on in this section.

We can notice from figures~\ref{fig:probdef} and \ref{fig:porousres} that $\overline{Q}$ is slowly varying over space. However,  $Q$ carries high-frequency oscillations~($Q^\prime$) of varying amplitude riding on top of $\overline{Q}$. We observed in all our simulations, and as explained in the previous section,  $Q^\prime$ is periodic. Further, it can be expressed as the following multiplicative decomposition
\begin{equation}
Q^\prime (x,X)=w(X)t(x),
\end{equation}
where $w(X)$ is a smooth window that shapes the amplitude of these oscillating functions $t(x)$. Since $t(x)$ is a periodic function over space, we can decompose it as fundamental sinusoidal and its harmonics using the Fourier series.
\begin{equation}
t(x)=\sum_{k=1}^{\infty}\left[a_k\sin (k\omega_0x)+b_k\cos (k\omega_0x)\right],
\end{equation}
where $a_k$ and $b_k$ are the real Fourier series coefficients that set the amplitude of the respective oscillating sinusoidal. 
%The amplitude of these oscillations is further shaped by a window function $w(x)$. %The following figure 1 that plot of one simulation sample clarifies the proposed model. The unwanted osculations happen through a periodic waveform at a repetition of 1 cycle per l, which we can observe from figure 2. Thus the fundamental oscillation frequency is $\omega_0 = 1/l~m^{-1}$. Figure 3 shows the window shape that modulates the amplitude of the pure oscillation.

% Figure environment removed

In order to quantify the accuracy of the present method, when compared to the ensemble averaging, we compare the minimum tangential velocity and the maximum wall-normal velocity at the interface. These quantities were used to describe the accuracy of effective models in earlier studies~\cite{lacis2020,sudhakar2021,jain2022}. These values, reported in Table~\ref{tab:porous}, clearly show that the present method gives the same values as the ensemble averaging, with only a negligible difference. Thus, we can conclude that the technique reported in this paper produces as accurate results as that of the expensive ensemble averaging.

\begin{table*}
\caption{\label{tab:porous} Comparison of the minimum tangential velocity and the maximum wall-normal velocity, for the lid-driven cavity with the porous bed,  obtained using the present method and the ensemble averaging.}
\begin{ruledtabular}
\begin{tabular}{cccc}
\multicolumn{2}{c}{minimum $u_1$} & \multicolumn{2}{c}{maximum $u_2$} \\\cline{1-2} \cline{3-4}
ensemble avg. & Present & ensemble avg. & Present \\
$-1.563433\times 10^{-2}$ & $-1.563502\times 10^{-2}$ & $9.8724\times 10^{-4}$ & $9.8681\times 10^{-4}$ \\
\end{tabular}
\end{ruledtabular}
\end{table*}

Although we presented only the results for an isotropic porous medium here,  we verified that the averaging process described in this paper also works for anisotropic porous inclusions.
%-----------------------------
\subsection{Flow over rough walls}
Having demonstrated the applicability of the present method to flow over a porous medium, in this section,  we consider the flow within a lid-driven cavity in which the bottom wall is covered with ordered roughness elements, as shown in figure~\ref{fig:roughgeom}. In the previous section, we investigated only Stokes flow. This section aims to verify the applicability of the proposed method to flows in which the inertial effects are non-negligible. Within this rough-wall cavity, we perform simulations at a Reynolds number ($Re=U_0H/\nu$) of 0~(Stokes flow), 100 and 1000. Semi-major and semi-minor axes of the roughness elements are 0.25$l$ and $l$, respectively. The scale separation parameter, $\eta=0.1$. We choose the interface height, $\delta=0.1l$, to report the results.

% Figure environment removed
\end{figure}

% Figure environment removed

\begin{table*}
\caption{\label{tab:rough} Comparison of the minimum tangential velocity and the maximum wall-normal velocity, for the lid-driven cavity with rough wall, obtained using the present method and the ensemble averaging.}
\begin{ruledtabular}
\begin{tabular}{ccccc}
$Re$ & \multicolumn{2}{c}{minimum $(u_1/U_0)$} & \multicolumn{2}{c}{maximum $(u_2/U_0)$} \\\cline{2-3} \cline{4-5}
& ensemble avg. & Present & ensemble avg. & Present \\
0        & $1.127388\times 10^{-2}$ & $1.127491\times 10^{-2}$  & $4.155904\times 10^{-4}$    &  $4.162649\times 10^{-4}$ \\
100   & $1.211736\times 10^{-2}$ & $1.211848\times 10^{-2}$  & $5.028988\times 10^{-4}$   &  $5.032383\times 10^{-4}$\\
1000 & $6.202675\times 10^{-2}$ & $6.202663\times 10^{-2}$  & $5.337153\times 10^{-3}$ & $5.334945\times 10^{-3}$\\
\end{tabular}
\end{ruledtabular}
\end{table*}

The DNS produced velocity and pressure fields that contain macro- and micro-scale variations, similar to the flow over the porous medium reported in the previous section.  Since we discussed this point already, we present only the averaged results in this section.  In figure~\ref{fig:probdef}, the wall-normal velocity obtained for $Re=1000$ simulation is used to describe the decomposition and the averaging process.  
%Except for small oscillations produced near the ends, the velocity curves are reproduced very accurately by the present method.  

% Figure environment removed

The tangential and wall-normal interface velocities for the Stokes flow, $Re=100$ and 1000 are presented in figure~\ref{fig:roughres}. As can be seen from the figure, for all simulations, the proposed method produces results as accurate as the ensemble averaging. Additional confirmation of the accuracy is provided by comparing the minimum tangential and the maximum wall-normal interface velocity. These values presented in table~\ref{tab:rough} clearly show that the difference between the present and the ensemble averaging method is negligible. From these results, we can conclude that the method presented in this paper works well for both viscous-dominate and inertial-dominated laminar flows over rough surfaces.

Similar to the porous wall, the ensemble averaging required approximately ten samples to fully eliminate the microscale oscillations. The present method, by construction, only requires two samples. The reduction in computational time is more significant for $Re=1000$ than for the Stokes flow. This is because the Picard iteration or the Newton-Raphson method requires more iterations to converge for a large $Re$, while the governing equations are linear for the Stokes flow. 

Before concluding this section,  we relate the velocity profiles at the interface to the flowfield occurring within the cavity. The non-averaged streamline plots for the Stokes flow and at $Re=1000$ are presented in figure~\ref{fig:roughstreamline}. Due to linearity, the Stokes flow is symmetric with respect to the vertical line passing through the cavity centre. As a result, the velocity profiles are also symmetric, as shown in figure~\ref{fig:roughres}(a) and (b). Moreover, the main vortex formed within the cavity induces negative shear along the whole interface for Stokes flow. Hence, the tangential velocity is also negative, as explained by the well-known slip-velocity model~\cite{beavers1967}. In contrast, the flowfield is unsymmetric for $Re=1000$, hence the velocity profiles~(figure~\ref{fig:roughres}e and f). One of the significant differences at $Re=1000$ is the appearance of corner vortices, which induce a positive shear along a portion of the interface, and corresponding positive tangential velocity regions at either end of the interface. A larger $u_1$ at the right end indicates that the respective corner vortex is stronger. In recent works~\cite{lacis2020,sudhakar2021}, considering Stokes flow in the interface region, the following model for wall-normal velocity is proposed
\begin{equation}
u_2\propto-\frac{\partial u_1}{\partial x_1}.
\end{equation}
As can be seen from figure~\ref{fig:roughres}(e) and (f), $u_2$ is positive when $\frac{\partial u_1}{\partial x_1}<0$, and vice versa along the entire interface.  This points out that although the above model is obtained theoretically by eliminating inertial effects,  for the considered example, this relation is valid even when the interface Reynolds number, $Re_s=u_1^\textrm{max}l/\nu>6$.
%-----------------------------
\subsection{Flow over a poroelastic wall}
Results presented in the previous section for porous and rough walls indicate that the approach proposed in this paper can efficiently extract macroscopic data from DNS. ensemble averaging, although expensive, works well for these examples. This section presents a test case for which the ensemble averaging fails, but the present method works without any issues.

% Figure environment removed

The configuration considered is presented in figure~\ref{fig:hairygeom}(a). The lid-driven cavity problem is considered in which the bottom wall is coated with rod-like hairy filaments. The filaments are elastic, and hence they deform due to the pressure and viscous forces acting on them. We consider Stokes flow within the cavity. The parameters considered are $l=1$~m, $H=10$~m,  dimension of the hairy filament $0.5\times 0.1$~m$^2$; the ratio of solid to fluid density is 10. Neo-Hookean material model is used for the solid, and the interface height $\delta=0.1l$.  These parameters lead to a large deformation of the filaments, and should be modelled using a two-way coupled fluid-structure interaction technique. We simulated this problem with the COMSOL software using a partitioned coupling approach. We performed a quasi-steady FSI simulation, implying that the elastic filament attains an equilibrium position due to the fluid loads exerted. 

As discussed earlier,  moving the microscale geometry by one microlength scale is a prerequisite to applying the ensemble averaging.
%As discussed earlier, ensemble averaging requires moving the micro-geometry at equal intervals for each sample so that the total samples cover one microscopic length scale.  
This requirement is challenging to satisfy, for the poroelastic problem, due to the following two factors: (i)~during sampling, the filament can reach close to the cavity walls, and due to the subsequent fluid-induced deformation,  the filament may touch or even move out of the domain boundary as shown in figure~\ref{fig:hairygeom}(b), and (ii)~while moving the filament by equal distances between samples, only a part of the filament may lie on one wall and the remaining filament will be associated with the opposite cavity wall as illustrated in figure~\ref{fig:hairygeom}(c). The latter case is not an exception and happens for rough and porous walls discussed in earlier sections. Due to the elastic deformation of the filament, these scenarios are challenging to handle. Both these challenges stem from the large deformation of filaments, and they limit the applicability of the ensemble averaging only to small deformation cases.

As mentioned earlier, the configuration considered leads to large deformations of the filaments. Due to the abovementioned challenges associated with the ensemble averaging, 102 samples are generated that cover 51\% % of the microscopic length scale. The averaged tangential and wall-normal velocity along the interface are presented in figure~\ref{fig:flexres}. Since the complete microscopic length scale is not covered, the averaging does not extract the macroscopic fields from DNS results, and the microscopic oscillations prevail even after performing averaging. As shown in figure~\ref{fig:flexres}, while the oscillations in the tangential velocity are negligible, the wall-normal velocity exhibits significant microscale oscillations.  

% Figure environment removed

The proposed method, with only two samples, is able to recover the smooth macroscopic data as shown in figure~\ref{fig:flexres}. This is also true for wall-normal velocity that exhibited large oscillations with ensemble averaging. Even if ensemble averaging would have worked, it would have required performing at least ten DNS and in this particular case, ten multiphysics geometry-resolved simulations. Performing such simulations is computationally very expensive. This highlights the potential of the proposed method in extracting macroscopic data from geometry-resolved simulations, even in cases for which ensemble averaging fails.

% Figure environment removed

The results presented above lead to the conclusion that the proposed method can also apply to non-ordered rough/porous surfaces. To explain this, let us consider the results depicting the deformed configuration of the filaments, as shown in figure~\ref{fig:hairycont}. Parameters that characterise the macroscopic description of this problem are the slip length~($\mathcal{L}$) for the tangential velocity and the transpiration length~($\mathcal{M}$) for the wall-normal velocity~\cite{lacis2020}. If the filaments were rigid, both $\mathcal{L}$ and $\mathcal{M}$ would be constant along the entire interface length for viscous-dominated flows considered here. However, in case of the flexible filaments, the local fluid flow dictates the deformed configuration of each of them. The amount of elastic bending is different for each filament, and hence $\mathcal{L}$ and $\mathcal{M}$ are non-constant functions along the interface. This is illustrated in figure~\ref{fig:rigidflex}, which shows the velocity components along the interface for flexible and rigid filaments. The difference between these two indicates the spatial variation of $\mathcal{L}$ and $\mathcal{M}$, even in the absence of inertial effects. Since such variation is a characteristic feature of non-ordered rough/porous surfaces, we envisage that our method works equally well for non-ordered surfaces. 

% Figure environment removed

In order to verify the above claim, we simulated a Stokes flow within a lid-driven cavity in which the bottom wall is covered with non-ordered random rough elements. The configuration is depicted in figure~\ref{fig:randconf}, which also shows contours of vertical velocity and streamlines within the cavity. Although the shape of the elements changes, we maintain the same microscopic length $l$ for each element. The comparison of ensemble averaged and macroscopic velocity components obtained from the present method is shown in figure~\ref{fig:randvelo}. The nominal interface is chosen at a vertical distance of 0.1$l$ from the tallest roughness element.

% Figure environment removed


We can clearly see that the matching between both methods is good, except for a minor phase shift as shown in figure~\ref{fig:randvelo}. Notably, the present method yields a smooth variation of velocity components with only two samples, while the ensemble averaging used ten samples. The reason for the phase shift is currently unclear. As the treatment of random surfaces is beyond the scope of the present work,  it will be addressed in a future study.
% Figure environment removed
%===============================================================
\section{Conclusion}
\label{sec:conc}
This paper described an efficient method to extract macroscopic interface fields from the microscopic flow fields obtained for flows over irregular surfaces using geometry-resolved simulations. The proposed method requires only two samples, even for inertia-dominated flows. Hence, it drastically reduces the amount of computational effort in getting the macroscopic data compared to the ensemble averaging technique. Results presented for different configurations proved that the method can produce accurate results for flows over rough, porous, and poroelastic media. In future, we will extend this method to compute averaged results using only a single DNS data, and to handle problems with velocity boundary conditions other than Dirichlet type. Even in the current form,  to our knowledge, the present method is the only available technique applicable for flows over poroelastic surfaces exhibiting large deformation. This feature can aid in developing effective models for such a practically important multiscale surface.

%Extracting the macroscopic information is crucial for validating the effective models that are the only viable means of investigating flows over multiscale surfaces.

%In this paper, we described a method to extract average macroscopic interface fields from the data obtained for flows over irregular surfaces using geometry-resolved simulations or by conducting physical experiments. The method requires only two samples even for inertia-dominated flows, and hence it drastically reduces the amount of computational effort in getting the macroscopic data, when compared to the ensemble averaging technique. Results presented for different configurations proved that the method is capable of producing accurate results for flows over rough, porous, and poroelastic media. While ensemble averaging, although expensive, works for rough and porous walls, it can't handle poroelastic medium. However, the present method, without any modifications, worked well for poroelasic cases. Extracting the macroscopic information is crucial for validating the effective models that are the only viable means of investigating flows over multiscale surfaces.

% It will make it possible to extract data of, for example dispersive stresses, in turbulent flows. Computationally, the channel flow is preferred because of homogeneity in x-direction. With our work, it will be possible to extract such data also for non-homogeneous data. It is not known before hand in ensemble averaging how many number of samples are needed. However, irrespective of the Re, here we need only 2 simulations/experiments. It can be used to study non-homogeneous macroscopic flows (for example turbulent flow over a flat plate with porous inclusions.); Applicable for creeping and inertia dominated flows.

%===============================================================
\begin{acknowledgments}
We acknowledge the financial support provided by the DST-SERB Ramanujan fellowship~(sanction order no. SB/S2/RJN-037/2018) and SERB MATRICS Grant (MTR/2021/000841).
\end{acknowledgments}

\section*{Data Availability Statement}
The source code to perform averaging using the present method, and data for rough, porous, and poroelastic walls are made available in a public repository~\cite{bitbucket}.
%%============Start of appendix====================================================
%\appendix
%\section{Appendixes}
%Our aim is to extract $\overline{Q}$ from $Q$ obtained from DNS or experiments., with minimal effort.  This is an essential step to devise reference validation data from DNS/exp. Moreover this step is inevitable to investigate the flow physics.  This can be best understood by relating it to turbulent flows; although DNS or physical experiments are performed, to understand the key physical details of the turbulent flows, averaged quantities are always employed [Pope's book]. Not only to serve as reference data for validation, macroscopic details are necessary to understanding the associated flow physics.
%
%We consider ordered features. Although we have shown only the flow quantities, it can be directly applied to average out thermal effect [Bottaro's recent JFM] and other passive scalars as well. Applicable for creeping as well as inertia-dominated flows.
%
%
%Studies that performed averaging~\cite{wu2018,lu2019,lu2020a,lu2020b}. For turbulent flows, since channel is concerned, the time averaged fields can be obtained above the interface region and due to the spatial homogeneity, averaging can be performed. Surface averaged velocity at the interface~\cite{agelinchaab2006} because they consider a 3D geometry in experiments.
%
%Experimental~\cite{agelinchaab2006,carotenuto2011}.
%\cite{carotenuto2011} used simple shear flow over a porous medium.
%===============================================================
\nocite{*}
\bibliography{avg}
\end{document}
%
% ****** End of file aipsamp.tex ******
