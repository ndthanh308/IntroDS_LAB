%------------------------------------------------------------------------
% Header file for latex
%   importing latex packages
%   define macros for latex
%------------------------------------------------------------------------
%\usepackage[dvipsnames]{xcolor}
%------------------------------------------------------------------------
% import packages:
%------------------------------------------------------------------------
%\usepackage{lipsum}     % package for generating dummy text
\usepackage{amsthm}
\usepackage{algorithm}
\usepackage{algpseudocode}
\usepackage{amsmath}
% \usepackage{amssymb}
\usepackage{bm}
\usepackage{color}
%\usepackage[compress,sort,comma,numbers]{natbib}
%\usepackage{etex}
\usepackage{qtree}
\usepackage{graphicx}
%\usepackage{named}
\usepackage{multirow}
\usepackage{multicol}
%\usepackage{stmaryrd}
\usepackage{subfigure}
% \usepackage{subcaption}
\usepackage{url}
\usepackage{thmtools}

\usepackage{tabularx}
\usepackage{balance}    % balance the two columns of the paper in the last page
\usepackage{booktabs}   % professional tables. See http://cs.brown.edu/about/system/managed/latex/doc/booktabs.pdf
\usepackage{graphics}   % image files in figure (e.g., pdf, eps files)
\usepackage{url}        % formating a web url using \url{...}
\usepackage{pifont}     % adding special characters using \ding{...}. See http://willbenton.com/wb-images/pifont.pdf

\usepackage{xcolor}      % color text
%\usepackage{hyperref}   % add a hyper link to each reference
%TG:added---------------------------------------------------------------------
\usepackage{xspace}
\usepackage{balance}
\usepackage{enumitem}
% to display page number (first page is not working yet.)
%\pagestyle{plain}
%------------------------------------------------------------------------
% macros:
%------------------------------------------------------------------------
\newcommand{\ie}[0]{\textit{i.e.},\ }   % i.e., meaning "which is/means "
\newcommand{\eg}[0]{\textit{e.g.},\ }   % e.g., meaning "for example, "
\newcommand{\etc}[0]{\textit{etc.}\ }   % etc. meaning "and so on."


\newcommand{\modelCategory}{\text{multi-branch dynamic network}\xspace}
\newcommand{\modelname}{\textsc{EPNet}\xspace}
\newcommand{\tian}[1]{{\color{red}{ Tian: #1}}}

\usepackage[labelfont={bf,small},textfont={bf,small}]{caption}
% \usepackage[labelformat=simple]{subcaption}
% \renewcommand\thesubfigure{(\alph{subfigure})}

% \usepackage{amssymb}% http://ctan.org/pkg/amssymb
\usepackage{pifont}% http://ctan.org/pkg/pifont
\newcommand{\cmark}{\ding{51}}%
\newcommand{\xmark}{\ding{55}}%

\usepackage{pgfplots}
% \pgfplotsset{width=8cm,compat=1.9}

% We will externalize the figures
% \usepgfplotslibrary{external}
% \tikzexternalize
% \tikzexternalize[prefix=tikz/,optimize command away=\includepdf]
%\usepackage{ctable}
% \usepackage{tikz}
% \def\checkmark{\tikz\fill[scale=0.4](0,.35) -- (.25,0) -- (1,.7) -- (.25,.15) -- cycle;}



\newcommand{\para }[1]{\smallskip \noindent  {\bf \emph{#1}}} 
\newcommand{\1}{{\em (i)}}
\newcommand{\2}{{\em (ii)}}
\newcommand{\3}{{\em (iii)}}
\newcommand{\4}{{\em (iv)}}
\newcommand{\5}{{\em (v)}}

\newcommand{\eat}[1]{}

%%%%%%%%%%%%%%%%%%%%%%%%%%
% Comments from Kong
% if you want to temporarily disable the comments in the pdf, you could comment the first line and un-comment the second line of code:

\newcommand{\kong}[1]{\textcolor{red}{[Kong: #1]}}
%\newcommand{\kong}[1]{}
%%%%%%%%%%%%%%%%%%%%%%%%%%
\newcommand{\yao}[1]{{\color{blue}{ [Yao: #1]}}}

% \newenvironment{myitemize}
% { \begin{itemize}
%     \setlength{\itemsep}{0pt}
%     \setlength{\parskip}{0pt}
%     \setlength{\parsep}{0pt} 
%     \setlength{\leftmargin}{*}}
% { \end{itemize}}   