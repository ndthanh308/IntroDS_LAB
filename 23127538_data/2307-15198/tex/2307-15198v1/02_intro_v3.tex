\vspace{-6pt}
\section{Introduction}
\label{sec:intro}
\textbf{Background.} Brain extraction (\emph{a.k.a.} skull stripping), registration and segmentation serve as preliminary yet indispensable steps in many neuroimaging studies, such as anatomical and functional analysis~\cite{bai2017unsupervised,wang2017structural,yang2015structural,papalexakis2014good}, brain networks discovering~\cite{liu2017collective, liu2017unified, lee2017identifying, yin2018coherent, yin2020gaussian,dai2020recurrent}, multi-modality fusion~\cite{cai2018deep, 10.1145/3534678.3539301}, diagnostic assistance~\cite{sun2009mining,huang2011brain}, and brain region studies~\cite{chen2018voxel,lee2020deep}.
The brain extraction targets the removal of non-cerebral tissues (\eg skull, dura, and scalp) from a patient's head scan; the registration step aims to align the extracted brain with a standard brain template; the segmentation step intends to label the anatomical brain regions in the raw imaging scan. These three tasks serve as crucial preprocessing steps in many neuroimaging studies.
For example, in brain functional and anatomical analysis, upon extracting and aligning the brain, the interference of non-cerebral tissues, imaging modalities, and viewpoints can be eliminated, thereby enabling accurate quantification of shifts in the shape, size, and signal; and labeled anatomical brain regions (\eg frontal lobe, cerebellum, etc.) can be used to guide the structural diagnosis.
In Alzheimer's disease diagnosis, the brain across subjects needs to be first extracted from raw brain imaging scans and then aligned with a standard template to counteract inter-individual variations and perform brain function analysis (\eg discovering the brain network connectivity). Meanwhile, the intra-individual structural lesions (\eg brain atrophy) across different pathological stages need to be monitored in anatomical analysis (\eg identify the corresponding anatomical brain region and measure its alteration of brain volume). These essential processing steps help doctors make a comprehensive and accurate diagnosis.

\textbf{State-of-the-Art.} 
The literature extensively explores brain extraction, registration, and segmentation problems~\cite{kleesiek2016deep,lucena2019convolutional,sokooti2017nonrigid, dai2020dual, akkus2017deep, chen2019learning, kamnitsas2017efficient}. Conventional approaches primarily emphasize the development of separate methods for extraction~\cite{kleesiek2016deep,lucena2019convolutional}, registration~\cite{sokooti2017nonrigid, dai2020dual}, and segmentation~\cite{akkus2017deep, chen2019learning, kamnitsas2017efficient} under supervised settings.
However, within the domain of medical studies,  the process of obtaining annotations for brain location, image transformations, and segmentation is often accompanied by significant expenses, necessitating expertise, and substantial time, especially when dealing with high-dimensional neuroimages (\eg 3D MRI).
To overcome this limitation, recent works~\cite{smith2002fast,cox1996afni,shattuck2002brainsuite,segonne2004hybrid,balakrishnan2018unsupervised,zhao2019recursive} introduce a three-step approach for one-shot extraction, registration and segmentation by using automated brain extraction tools~\cite{smith2002fast,cox1996afni,shattuck2002brainsuite,segonne2004hybrid}, unsupervised registration and segmentation models with direct warping~\cite{balakrishnan2018unsupervised,zhao2019recursive,jaderberg2015spatial}, as shown in Figure~\ref{fig:family 2}. However, these approaches often rely on manual quality control to correct intermediate results before performing subsequent tasks, which is time-consuming, labor-intensive, and subject to variability, thus hampering overall efficiency and performance.
More recently, joint extraction-registration method~\cite{su2022ernet} and joint registration-segmentation methods~\cite{qiu2021u, xu2019deepatlas,he2020deep} are introduced to solve the problem in a two-stage design, as shown in Figure~\ref{fig:family 3} and Figure~\ref{fig:family 4}. However, partial joint learning neglects the potential relationship among all tasks and negatively impacts overall performance.

\textbf{Problem Definition.} This paper investigates the problem of one-shot joint brain extraction, registration, and segmentation, as shown in Figure~\ref{fig:intro}. The goal is to capture the connections among three tasks to mutually boost their performance in a one-shot training scenario. Notably, the extraction, registration and segmentation labels of the raw image are not available. We expect to perform the three tasks simultaneously with only one labeled template. 

\textbf{Challenges.} Despite its value and significance, the problem of one-shot joint extraction, registration and segmentation has not been studied before and is very challenging due to its unique characteristics listed below:

\textbullet  \  
\textit{Lack of labels for extraction:} Traditional learning-based extraction methods rely on a substantial number of training samples with accurate ground truth labels. However, collecting voxel-level labels for high-dimensional neuroimaging data is a resource-intensive and time-consuming endeavor.

\textbullet  \  
\textit{Lack of labels for registration:} Obtaining the accurate ground truth transformation between raw and template images poses significant challenges. While unsupervised registration methods~\cite{balakrishnan2018unsupervised,zhao2019recursive} optimize transformation parameters by maximizing image similarity, their effectiveness is contingent upon the prior removal of non-brain tissue from the raw image. Failing to do this may lead to erroneous transformations, rendering the registration invalid.


\textbullet  \  
\textit{Lack of labels for segmentation:} Collecting the voxel-level segmentation labels is also difficult. Although we provide a template with its segmentation labeled (in template image space), the segmentation (in raw image space) of the raw image is not available.



\textbullet  \  
\textit{Dependencies among extraction, registration and segmentation:} Conventional research typically treats extraction, registration, and segmentation tasks separately. However, these tasks exhibit a high degree of interdependency. The accuracy of registration and segmentation tasks heavily relies on the extraction task. The registration process assists the extraction task in capturing cerebral/non-cerebral information from raw and template images, and providing segmentation labels for guiding the segmentation task. The segmentation task can inversely force the extraction and registration tasks to provide precise results. Thus, a holistic solution is required to effectively manage the interdependencies among these tasks.


%---------------------
% Figure 2
\input{fig_family}
%---------------------


\textbf{Proposed Method.} To address the aforementioned challenges, we propose JERS, a unified end-to-end framework for joint brain extraction, registration, and segmentation. Figure~\ref{fig: family} showcases a comparison between our method and state-of-the-art approaches. Specifically, JERS contains a group of extraction, registration and segmentation modules, where the extraction module gradually eliminates the non-brain tissue from the raw image, producing an extracted brain image; the registration module incrementally aligns the extracted image with the template and warps the template's segmentation label in the raw image space to guide the segmentation module; the segmentation module generates a segmentation label for the raw image and provides feedback to extraction and registration modules. These three modules help each other to boost extraction, registration and segmentation performance simultaneously. By bridging these three modules end-to-end, we achieve a joint optimization with only one labeled template assistance. 


%Extensive experiments on public brain MRI datasets demonstrate that our proposed method significantly outperforms state-of-the-art approaches in extraction, registration, and segmentation accuracy.

Upon evaluation on public brain MRI datasets, our proposed method significantly outperforms state-of-the-art techniques in extraction, registration, and segmentation accuracy.