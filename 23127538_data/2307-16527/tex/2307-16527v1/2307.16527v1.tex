\documentclass[11pt, a4paper]{article}

\title{On asymptotic stability  on a center hypersurface at the solition for even solutions of the NLKG  when $2\ge p> \frac{5}{3}$ }




\author{ Scipio Cuccagna  \and  Masaya Maeda  \and   Federico Murgante  \and  Stefano Scrobogna }



\usepackage[a4paper]{geometry}
\geometry{
 a4paper,
 total={160mm,257mm},
 left=20mm,
 right=20mm,
 top=20mm,
 bottom=20mm
 }

 \newcommand\hmmax{0}
\newcommand\bmmax{0}


%           Pacchetti vari ed eventuali
\usepackage{empheq}
\usepackage{times}
%\usepackage{stix}
\usepackage{stmaryrd}
\usepackage[notref]{showkeys}
%\usepackage{fourier}
\usepackage[T1]{fontenc}
\usepackage{amssymb, amsmath,  bm, mathrsfs}
\usepackage{amsfonts}
\usepackage[english]{babel}
\usepackage{amssymb,amsthm}
\usepackage{mathtools}
\usepackage{accents}
\usepackage{mathalfa}
%\usepackage{showkeys}
\DeclareMathAlphabet{\mathcal}{OMS}{cmsy}{m}{n}
\DeclareFontFamily{U}{mathc}{}
\DeclareFontShape{U}{mathc}{m}{it}%
{<->x*[1.03] mathc10}{}
\DeclareMathAlphabet{\mathscr}{U}{mathc}{m}{it}


\DeclareMathAlphabet{\mathpzc}{OT1}{pzc}{m}{it}


\usepackage{hyperref}
\usepackage[capitalise]{cleveref}
\usepackage{cite}
\allowdisplaybreaks[1]
%\usepackage[bbgreekl]{mathbbol}
%\DeclareSymbolFontAlphabet{\mathbb}{AMSb}
%\DeclareSymbolFontAlphabet{\mathbbl}{bbold}
\usepackage{xfrac}
\usepackage{soul}
\usepackage[utf8]{inputenc}
\usepackage[T1]{fontenc}
\usepackage{enumerate}
\usepackage{accents}
\usepackage{bbm}
\usepackage{thmtools}
\usepackage[author={Max Schlepzig}]{pdfcomment}

\usepackage{todonotes}
\newcommand{\todoSS}[1]{\todo{\footnotesize SS: #1}}
%\newcommand{\todog}[1]{\todo[color=green]{#1}}
%\newcommand{\todor}[1]{\todo[color=red]{#1}}
%\newcommand{\red}[1]{\textcolor{red}{#1}}
%\newcommand{\orange}[1]{\textcolor{orange}{#1}}
\usepackage[makeroom]{cancel}

\usepackage{tikz}
\usetikzlibrary{arrows, automata, backgrounds, shadows, patterns, calc, hobby, plotmarks, shapes}
\usetikzlibrary{shapes.misc}

\tikzset{cross/.style={cross out, draw=black, minimum size=2*(#1-\pgflinewidth), inner sep=0pt, outer sep=0pt},
%default radius will be 1pt.
cross/.default={1pt}}
\allowdisplaybreaks

\makeatletter
\@ifpackageloaded{stix}{%
}{%
  \DeclareFontEncoding{LS2}{}{\noaccents@}
  \DeclareFontSubstitution{LS2}{stix}{m}{n}
  \DeclareSymbolFont{stix@largesymbols}{LS2}{stixex}{m}{n}
  \SetSymbolFont{stix@largesymbols}{bold}{LS2}{stixex}{b}{n}
  \DeclareMathDelimiter{\lBrace}{\mathopen} {stix@largesymbols}{"E8}%
                                            {stix@largesymbols}{"0E}
  \DeclareMathDelimiter{\rBrace}{\mathclose}{stix@largesymbols}{"E9}%
                                            {stix@largesymbols}{"0F}
}
\makeatother

\DeclareSymbolFontAlphabet{\amsmathbb}{AMSb}%



%------------------------------------
%            CODE FORMATTING
%------------------------------------


\usepackage{listings}
\usepackage{color}

\definecolor{dkgreen}{rgb}{0,0.6,0}
\definecolor{gray}{rgb}{0.5,0.5,0.5}
\definecolor{mauve}{rgb}{0.58,0,0.82}

\lstset{frame=tb,
  language=Java,
  aboveskip=3mm,
  belowskip=3mm,
  showstringspaces=false,
  columns=flexible,
  basicstyle={\small\ttfamily},
  numbers=none,
  numberstyle=\tiny\color{gray},
  keywordstyle=\color{blue},
  commentstyle=\color{dkgreen},
  stringstyle=\color{mauve},
  breaklines=true,
  breakatwhitespace=true,
  tabsize=3
}

%       SIZE OF LABEL IN EQUATIONS
\makeatletter
\def\maketag@@@#1{\hbox{\m@th\normalfont\normalsize#1}}
\makeatother

%------------------------------------
%             MACRO SCIPIO
%------------------------------------

\newcommand{\R}{{\mathbb R}}
\newcommand{\U}{{\mathcal U}}
\newcommand{\Q}{{\mathbb Q}}
\newcommand{\Z}{{\mathbb Z}}

\newcommand{\N}{{\mathbb N}}
\newcommand{\In}{{\mathbb Z}}
\newcommand{\Tr}{{\mathcal T}}

\def\im{{\rm i}}

\newcommand{\C}{\mathbb{C}}
\newcommand{\T}{\mathbb{T}}

\newcommand{\E}{\mathcal{E}}
%\newcommand{\cH}{\mathcal{H}}

\newcommand{\bG}{\mathbf{G}}
\newcommand{\bO}{\mathbf{O}}
\newcommand{\bc}{\mathbf{c}}

\def\({\left(}
\def\){\right)}
\def\<{\left\langle}
\def\>{\right\rangle}
\newcommand{\sech}{{\mathrm{sech}}}
\newcommand{\even}{{\mathrm{even}}}
\newcommand{\odd}{{\mathrm{odd}}}
\newcommand{\supp}{{\mathrm{supp}\ }}



%------------------------------------
%             MACRO STEFANO
%------------------------------------
%\newcommand{\dx}{\textnormal{d}{x}}
%\newcommand{\ds}{\textnormal{d}{x}}
\newcommand{\dd}{\textnormal{d}}
%\newcommand{\dive}{\textnormal{div}}
%\renewcommand{\div}{\textnormal{div}}
%\newcommand{\sgn}{\textnormal{sgn}}
%\newcommand{\Res}{\textnormal{Res}}
%\newcommand{\Vol}{\textnormal{Vol}}
%\newcommand{\fine}{\hfill$\blacklozenge$}
\newcommand{\pare}[1]{\left( #1 \right)}
%\newcommand{\co}[1]{\left[ #1 \right)}
\newcommand{\angles}[1]{\left\langle #1 \right\rangle}
%\newcommand{\Pare}[1]{\Big( #1 \Big)}
\newcommand{\norm}[1]{\left\| #1 \right\|}
\newcommand{\av}[1]{\left| #1 \right|}
\newcommand{\bra}[1]{\left[ #1 \right]}
\newcommand{\bbra}[1]{\left\llbracket #1 \right\rrbracket}
%\newcommand{\pbra}[2]{\left\lbrace\!\!\left[ #1, #2 \right]\!\!\right\rbrace} lBrace
\newcommand{\pbra}[2]{\left\lBrace #1 \ , \  #2 \right\rBrace}
\newcommand{\comm}[2]{\bra{ #1 \  , \   #2  }}
%\newcommand{\Ad}{\textnormal{Ad}}
\newcommand{\xfrac}[2]{\left. #1 \middle/ #2 \right. }
\newcommand{\set}[1]{\left\{ #1 \right\}}
\newcommand{\system}[1]{\left\{ #1 \right.}
%\newcommand{\floor}[1]{\left\lfloor #1 \right\rfloor}
%\newcommand{hyp}{Hypothesis}
\newcommand{\interior}[1]{\accentset{\circ}{#1}}
%\newcommand{\OpBW}[1]{\textnormal{Op}^{BW}\bra{#1}}
%\newcommand{\OpB}[1]{\textnormal{Op}^{B}\bra{#1}}
%\newcommand{\Op}[1]{\textnormal{Op}\bra{#1}}
%\newcommand{\supp}{\textnormal{supp}}
\newcommand{\ddt}{\frac{\textnormal{d}}{\textnormal{d}t}}
%\newcommand{\pv}{\textnormal{p.v.}}
%\newcommand{\cst}{\textnormal{cst}}
\newcommand{\Id}{\textnormal{Id}}
%\newcommand{\cCs}{\cC_{\textnormal{small}}}
\newcommand{\defeq}{\vcentcolon=}
\newcommand{\eqdef}{=\vcentcolon}


\newcommand{\spp}{\sigma_{\textnormal{pp}}}
\newcommand{\Span}[1]{\textnormal{Span}\set{#1}}



\def\fint{\mathop{\,\rlap{--}\!\!\!\int}\nolimits}
\newcommand{\vect}[2]{ \pare{ \begin{array}{c} #1 \\ #2 \end{array} } }
%\def\fint{\mathop{\mathchoice{\,\rlap{-}\!\!\int}
%              {\rlap{\raise.15em{\scriptstyle -}}\kern-.2em\int}
%              {\rlap{\raise.09em{\scriptscriptstyle -}}\!\int}
%              {\rlap{-}\!\int}}\nolimits}

%\def\fiint{\mathop{\,\rlap{---}\!\!\!\iint}\nolimits}
\renewcommand{\Re}{\textnormal{Re}}
\renewcommand{\Im}{\textnormal{Im}}
%\newcommand{\tj}{\triangle_j}
%\newcommand{\D}{\angles{D}}
%\newcommand{\vv}{\mathsf{v}}
\newcommand{\RN}[1]{%
  \textup{\uppercase\expandafter{\romannumeral#1}}%
}


\newcommand{\bmz}{{\bm z}}
\newcommand{\bmR}{{\bm R}}
\newcommand{\bmPhi}{{\bm \Phi}}

\newcommand{\bmtilde}[1]{\accentset{\sim}{\bm{#1}}}




\newcommand{\cA}{\mathcal{A}}
\newcommand{\cC}{\mathcal{C}}
\newcommand{\cI}{\mathcal{I}}
\newcommand{\cL}{\mathcal{L}}
\newcommand{\cT}{\mathcal{T}}
\newcommand{\cB}{\mathcal{B}}
\newcommand{\pB}{\mathpzc{B}}
\newcommand{\pR}{\mathpzc{R}}
\newcommand{\cQ}{\mathcal{Q}}
\newcommand{\cF}{\mathcal{F}}
\newcommand{\cJ}{\mathcal{J}}
\newcommand{\cD}{\mathcal{D}}
\newcommand{\cE}{\mathcal{E}}
\newcommand{\cS}{\mathcal{S}}
\newcommand{\cR}{\mathcal{R}}
\newcommand{\bR}{\mathbb{R}}
\newcommand{\bN}{\mathbb{N}}
\newcommand{\bM}{\mathbb{M}}
\newcommand{\bS}{\mathbb{S}}
\newcommand{\cG}{\mathcal{G}}
\newcommand{\cH}{\mathcal{H}}
\newcommand{\cO}{\mathcal{O}}
\newcommand{\cK}{\mathcal{K}}
\newcommand{\sG}{\mathsf{G}}
\newcommand{\sL}{\mathsf{L}}
\newcommand{\cV}{\mathcal{V}}
%\newcommand{\fV}{\mathfrak{V}}
\newcommand{\fb}{\mathfrak{b}}
\newcommand{\fc}{\mathfrak{c}}
\newcommand{\fB}{\mathfrak{B}}
\newcommand{\cY}{\mathcal{Y}}
\newcommand{\cP}{\mathcal{P}}
\newcommand{\cX}{\mathcal{X}}
\newcommand{\cN}{\mathcal{N}}
\newcommand{\cM}{\mathcal{M}}
\newcommand{\bP}{\mathbb{P}}
\newcommand{\bZ}{\mathbb{Z}}
\newcommand{\bC}{\mathbb{C}}
\newcommand{\bJ}{\mathbb{J}}
\newcommand{\bK}{\mathbb{K}}
\newcommand{\bT}{\mathbb{T}}
\newcommand{\bX}{\mathbb{X}}
\newcommand{\bL}{\mathbb{L}}
\newcommand{\bV}{\mathbb{V}}
\newcommand{\cf}{\mathpzc{f}}
\newcommand{\cg}{\mathpzc{g}}
\newcommand{\cw}{\mathpzc{w}}
\newcommand{\cm}{\mathpzc{m}}
\newcommand{\ccC}{\mathscr{C}}
\newcommand{\sS}{\mathsf{S}}
\newcommand{\sT}{\mathsf{T}}
\newcommand{\sH}{\mathsf{H}}
\newcommand{\sK}{\mathsf{K}}
\newcommand{\sR}{\mathsf{R}}
\newcommand{\sM}{\mathsf{M}}
\newcommand{\sx}{\mathsf{x}}
\newcommand{\sy}{\mathsf{y}}
%\newcommand{\st}{\mathsf{t}}
\newcommand{\sY}{\mathsf{Y}}
\newcommand{\sA}{\mathsf{A}}
\newcommand{\sa}{\mathsf{a}}
\newcommand{\sC}{\mathsf{C}}
\newcommand{\fI}{\mathfrak{I}}

\newcommand{\ii}{\mathrm{i}}



\newcommand{\ps}[2]{\pare{ \left. #1 \ \right| \ #2 }}
\newcommand{\psc}[2]{\left\langle  #1 \ \middle\vert \ #2 \right\rangle}

\newcommand{\hra}{\hookrightarrow}
\newcommand{\rhu}{\rightharpoonup}
\newcommand{\loc}{\textnormal{loc}}

%\newcommand{\jump}[1]{\left\llbracket #1 \right\rrbracket}

%\def\jump#1{{[\hspace{-2pt}[#1]\hspace{-2pt}]}}


%%%% MACRO MAX
%
%\newcommand{\be}{\begin{equation}}
%\newcommand{\ee}{\end{equation}}
%\newcommand{\vr}{\varrho}
%\newcommand{\N}{\mathbb N}
%\newcommand{\R}{\mathbb R}
%\newcommand{\T}{\mathbb T}
%\newcommand{\C}{\mathbb C}
%\newcommand{\Z}{\mathbb Z}
%\newcommand{\di}{{\rm d}}
%\newcommand{\im}{{\rm i}}
%\newcommand{\cU}{\mathcal{U}}
%\newcommand{\ov}{\overline}
%\def\wt{\widetilde}
%\renewcommand{\a}{\alpha}
%\newcommand{\la}{\langle}
%\newcommand{\ra}{\rangle}
%\newcommand{\x}{\xi}
%\newcommand{\s}{\sigma}
%\newcommand{\pa}{\partial}
%\newcommand{\opw}{{{\rm Op}^{\mathrm{W}}}}
%\newcommand{\opbw}{{{\rm Op}^{{\scriptscriptstyle{\mathrm BW}}}}}
%\newcommand{\sign}{{\rm sign}}
%\newcommand{\Opbw}[1]{{{{\rm Op}^{{\scriptscriptstyle{\mathrm BW}}}}\left(#1\right)}}
%\newcommand{\mM}{\mathcal{M}}
%\newcommand{\mR}{\mathcal{R}}
%\newcommand{\Gt}[2]{{\tilde{\Gamma}^{#1}_{#2}}}
%\newcommand{\Hds}[1]{{{H}_0^{#1}}}
%\newcommand{\Tu}{{{\mathbb T}^1}}
%\newcommand{\Ucal}{{\mathcal U}}
%\newcommand{\Pin}[1]{\Pi_{n_{#1}}}
%\newcommand{\Gr}[2]{{{\Gamma}^{#1}_{#2}[r]}}
%\newcommand{\Br}[2]{{B^{#1}_{\sigma_{#2}}(I,r)}}
%\newcommand{\Lcal}{{\mathcal L}}
%\newcommand{\Gcals}[3]{{\Gcal^{#1}_{#2}(#3,t)}}
%\newcommand{\Gcalsm}[3]{{\Gcal^{#1}_{#2}(#3)}}
%\newcommand{\Gcal}{{\mathcal G}}
%\providecommand{\nnorm}[1]{\vert\!\vert\!\vert#1\vert\!\vert\!\vert}
%\newcommand{\Gra}[2]{{{\Gamma}^{#1}_{#2}[r,\mathrm{aut}]}}
%\newcommand{\Rra}[2]{{{\Rcal}^{#1}_{#2}[r,\mathrm{aut}]}}
%\newcommand{\Rcal}{{\mathcal R}}
%\newcommand{\sg}[3]{{{\Sigma\Gamma}^{#1}_{#2}[r,#3]}}
%\newcommand{\sr}[3]{{{\Sigma\Rcal}^{#1}_{#2}[r,#3]}}
%\newcommand{\sGM}[3]{{{\Sigma\Gamma}^{#1}_{#2}[r,#3]\otimes \Mcal_2(\C)}}
%\newcommand{\sRM}[3]{{{\Sigma\Rcal}^{#1}_{#2}[r,#3]\otimes \Mcal_2(\C)}}
%\newcommand{\Mcal}{{\mathcal M}}
%\newcommand{\CKHR}[2]{{C_{*\R}^K(I,\Hds{#1}(\Tu,#2))}}
%\newcommand{\sFR}[2]{{{\Sigma\Fcal^{\R}_{#1}[r,#2]}}}
%\newcommand{\Fcal}{{\mathcal F}}
%\newcommand{\sF}[2]{{{\Sigma\Fcal}_{#1}[r,#2]}}
%\newcommand{\sFa}[2]{{{\Sigma\Fcal}_{#1}[r,#2,\mathrm{aut}]}}
%\newcommand{\sGa}[3]{{{\Sigma\Gamma}^{#1}_{#2}[r,#3,\mathrm{aut}]}}

%------------------------------------
%            TEOREMI E STRUTTURA
%------------------------------------


\theoremstyle{theorem}
\newtheorem{theorem}{Theorem}[section]
\newtheorem*{Theorem}{Main Theorem}
\newtheorem*{theorem*}{Theorem}
\newtheorem{prop}[theorem]{Proposition}
\newtheorem{proposition}[theorem]{Proposition}
\newtheorem{lemma}[theorem]{Lemma}
\newtheorem{cor}[theorem]{Corollary}



\theoremstyle{definition}
\newtheorem{definition}[theorem]{Definition}
\newtheorem{example}[theorem]{Example}
\newtheorem{rem}[theorem]{Remark}
\newtheorem{remark}[theorem]{Remark}
\newtheorem{assumption}[theorem]{Assumption}
\newtheorem{notation}[theorem]{Notation}
\newtheorem*{question}{Question}
\newtheorem{step}{Step}
\makeatletter
\@addtoreset{step}{subsection}
\makeatother
\newtheorem{substep}{Substep}
\makeatletter
\@addtoreset{substep}{step}
\makeatother
\newtheorem{proofpart}{Part}
\makeatletter
\@addtoreset{proofpart}{theorem}
\makeatother
\newtheorem{proofsubpart}{Subpart}
\makeatletter
\@addtoreset{proofsubpart}{part}
\makeatother
\newtheorem{case}{Case}
\makeatletter
\@addtoreset{case}{theorem}
\makeatother
\newtheorem{hyp}{Hypothesis}
\newenvironment{claim}{\textbf{ \textsc{ Claim:}}}{\hfill$\dagger$}
\numberwithin{equation}{section}


%
%\declaretheoremstyle[
%spaceabove=6pt, spacebelow=6pt,
%headfont=\normalfont\bfseries,
%notefont=\mdseries, notebraces={(}{)},
%bodyfont=\normalfont,
%postheadspace=1em,
%qed=$ \blacklozenge $
%]{mystyle1}
%\declaretheorem[style=mystyle1]{remark}
%\theoremstyle{remark}



%\newcommand{\commentb}[1]{{\color{blue}\textnormal{\textsf{#1}}}}
%\newcommand{\commentr}[1]{{\color{red}\textnormal{\textsf{#1}}}}

%\usepackage[author={Max Schlepzig}]{pdfcomment}



%--------------------------------------------------------------------------


%--------------------------------------------------------------------------

%% \usepackage[utf8]{inputenc} % allow utf-8 input
%\usepackage[T1]{fontenc}    % use 8-bit T1 fonts
%\usepackage{hyperref}       % hyperlinks
\usepackage{url}            % simple URL typesetting
\usepackage{booktabs}       % professional-quality tables
\usepackage{multirow}    
\usepackage{amsfonts}       % blackboard math symbols
\usepackage{nicefrac}       % compact symbols for 1/2, etc.
\usepackage{microtype}      % microtypogrhy
% \usepackage{natbib}
\usepackage{enumerate}
%\usepackage{enumitem}
\usepackage{hhline}
\usepackage{makecell}
\usepackage{pifont}

% use Times
%\usepackage{times}
% For figures
\usepackage{graphicx} % more modern
%\usepackage{epsfig} % less modern
%\usepackage{subfigure}
\usepackage{caption}
\usepackage{subcaption}
% For citations
\usepackage{amsmath}
\usepackage{amsthm}
\usepackage{amssymb}
\usepackage{tikz}
\usepackage{xcolor}
\usetikzlibrary{arrows}

\allowdisplaybreaks

%for fonts
\usepackage{mathrsfs}

% For algorithms
\usepackage{algorithm}
\usepackage{algorithmic}
% \usepackage{algpseudocode}
% \usepackage[noend]{algpseudocode}
\usepackage{hyperref}
\usepackage{bm}
%\usepackage{todonotes}

%For theorems
\allowdisplaybreaks

%for convinience
\newcommand{\RR}{\mathbb{R}}
\newcommand{\vct}{\boldsymbol }
%\newcommand{\mat}{\mathbf}
\newcommand{\rnd}{\mathsf}
\newcommand{\ud}{\mathrm d}
\newcommand{\nml}{\mathcal{N}}
\newcommand{\loss}{\mathcal{L}}
\newcommand{\hinge}{\mathcal{R}}
\newcommand{\kl}{\mathrm{KL}}
\newcommand{\cov}{\mathrm{cov}}
\newcommand{\dir}{\mathrm{Dir}}
\newcommand{\mult}{\mathrm{Mult}}
\newcommand{\err}{\mathrm{err}}
\newcommand{\sgn}{\mathrm{sgn}}
%\renewcommand{\span}{\mathrm{span}}
% \newcommand{\argmin}{\mathrm{argmin}}
% \newcommand{\argmax}{\mathrm{argmax}}
\newcommand{\poly}{\mathrm{poly}}
% \newcommand{\rank}{\mathrm{rank}}
% \newcommand{\conv}{\mathrm{conv}}
%\newcommand{\E}{\mathbb{E}}
% \newcommand{\diag}{\mat{diag}}
\newcommand{\acc}{\mathrm{acc}}

\newcommand{\labs}{\left\vert}
\newcommand{\rabs}{\right\vert}
\newcommand{\lnorm}{\left\Vert}
\newcommand{\rnorm}{\right\Vert}

\newcommand{\aff}{\mathrm{aff}}
% \newcommand{\range}{\mathrm{Range}}
\newcommand{\Sgn}{\mathrm{sign}}

\newcommand{\hit}{\mathrm{hit}}
\newcommand{\cross}{\mathrm{cross}}
\newcommand{\Left}{\mathrm{left}}
\newcommand{\Right}{\mathrm{right}}
\newcommand{\Mid}{\mathrm{mid}}
\newcommand{\bern}{\mathrm{Bernoulli}}
\newcommand{\ols}{\mathrm{ols}}
\newcommand{\tr}{\operatorname{tr}}
\newcommand{\opt}{\mathrm{opt}}
%\newcommand{\ridge}{\mathrm{ridge}}
\newcommand{\unif}{\mathrm{Unif}}
\newcommand{\Image}{\mathrm{im}}
\newcommand{\Kernel}{\mathrm{ker}}
\newcommand{\supp}{\mathrm{supp}}
\newcommand{\pred}{\mathrm{pred}}
\newcommand{\distequal}{\stackrel{\mathbf{P}}{=}}
%\newcommand{\gege}{\textcircled{1}}
\newcommand{\gege}{{A(\vect{w},\vect{w}_*)}}
\newcommand{\gele}{{A(\vect{w},-\vect{w}_*)}}
\newcommand{\lele}{{A(-\vect{w},-\vect{w}_*)}}
\newcommand{\lege}{{A(-\vect{w},\vect{w}_*)}}
\newcommand{\firstlayer}{\mathbf{W}}
\newcommand{\firstlayerWN}{v}
\newcommand{\secondlayer}{a}
\newcommand{\inputvar}{\vect{x}}
\newcommand{\anglemat}{\mathbf{\Phi}}
\newcommand{\holder}{H\"{o}lder }
\newcommand{\real}{\mathbb{R}}
\newcommand{\approxerr}{\delta}

\def\R{\mathbb{R}}
\def\Z{\mathbb{Z}}
\def\cA{\mathcal{A}}
\def\cB{\mathcal{B}}
\def\cD{\mathcal{D}}
\def\cE{\mathcal{E}}
\def\cF{\mathcal{F}}
\def\cG{\mathcal{G}}
\def\cH{\mathcal{H}}
\def\cS{\mathcal{S}}
\def\cI{\mathcal{I}}
\def\cL{\mathcal{L}}
\def\cM{\mathcal{M}}
\def\cN{\mathcal{N}}
\def\cP{\mathcal{P}}
\def\cS{\mathcal{S}}
\def\cT{\mathcal{T}}
\def\cV{\mathcal{V}}
\def\cW{\mathcal{W}}
\def\cZ{\mathcal{Z}}
\def\SS{\mathbb{S}}
\def\NN{\mathbb{N}}
\def\bP{\mathbf{P}}
\def\TV{\mathrm{TV}}
\def\MSE{\mathrm{MSE}}

\def\vw{\mathbf{w}}
\def\va{\mathbf{a}}
\def\vZ{\mathbf{Z}}

\newcommand{\mat}[1]{#1}
\newcommand{\vect}[1]{#1}
\newcommand{\norm}[1]{\left\|#1\right\|}
\newcommand{\normop}[1]{\left\|#1\right\|_{\mathrm{op}}}
\newcommand{\simplex}{\triangle}
\newcommand{\abs}[1]{\left|#1\right|}
\newcommand{\expect}{\mathbb{E}}
\newcommand{\prob}{\mathbb{P}}
\newcommand{\proj}{\gP}
% \newcommand{\prox}[2]{\textbf{Prox}_{#1}\left\{#2\right\}}
\newcommand{\event}[1]{\mathscr{#1}}
\newcommand{\set}[1]{#1}
\newcommand{\diff}{\text{d}}
\newcommand{\difference}{\triangle}
\newcommand{\inputdist}{\mathcal{Z}}
\newcommand{\indict}{\mathbb{I}}
\newcommand{\rotmat}{\mathbf{R}}
\newcommand{\normalize}[1]{\overline{#1}}
\newcommand{\vectorize}[1]{\text{vec}\left(#1\right)}
\newcommand{\vclass}{\mathcal{G}}
\newcommand{\pclass}{\Pi}
\newcommand{\qclass}{\mathcal{Q}}
\newcommand{\rclass}{\mathcal{R}}
\newcommand{\classComplexity}[2]{N_{class}(#1,#2)}
\newcommand{\cclass}{\mathcal{F}}
\newcommand{\gclass}{\mathcal{G}}
\newcommand{\pthres}{p_{thres}}
\newcommand{\ethres}{\epsilon_{thres}}
\newcommand{\eclass}{\epsilon_{class}}
\newcommand{\states}{\mathcal{S}}
\newcommand{\trans}{P}
\newcommand{\lowprobstate}{\psi}
\newcommand{\actions}{\mathcal{A}}
\newcommand{\contexts}{\mathcal{X}}
\newcommand{\edges}{\mathcal{E}}
\newcommand{\variance}{\text{Var}}
\newcommand{\params}{\vect{w}}

\newcommand{\relu}[1]{\sigma\left(#1\right)}
\newcommand{\reluder}[1]{\sigma'\left(#1\right)}
\newcommand{\act}[1]{\sigma\left(#1\right)}

\newtheorem{thm}{Theorem}
% \newtheorem{thm}{Theorem}
\newtheorem{lem}{Lemma}
% Thm -> corollary 
\newtheorem{cor}{Corollary}
\newtheorem{prop}{Proposition}
\newtheorem{asmp}{Assumption}
\newtheorem{defn}{Definition}
\newtheorem{oracle}{Oracle}
\newtheorem{fact}{Fact}
\newtheorem{conj}{Conjecture}
\newtheorem{rem}{Remark}
\newtheorem{example}{Example}
\newtheorem{condition}{Condition}
\newtheorem{exercise}{Exercise}
\newtheorem{mess}{Message}
\newtheorem{claim}{Claim}
\newtheorem{ec}{Empirical Conclusion}






\usepackage[capitalize,noabbrev]{cleveref}
% \usepackage{cleveref}
\crefname{thm}{Theorem}{Theorems}
\crefname{lem}{Lemma}{Lemmas}
\crefname{cor}{Corollary}{Corollaries}
\crefname{prop}{Proposition}{Propositions}
\crefname{asmp}{Assumption}{Assumptions}
\crefname{defn}{Definition}{Definitions}
\crefname{oracle}{Oracle}{Oracles}
\crefname{fact}{Fact}{Facts}
\crefname{conj}{Conjecture}{Conjectures}
\crefname{rem}{Remark}{Remarks}
\crefname{claim}{Claim}{Claims}
\crefname{ec}{Empirical Observation}{Empirical Observations}


\renewcommand{\algorithmicrequire}{\textbf{Input:}}
\renewcommand{\algorithmicensure}{\textbf{Output:}}


\definecolor{red}{rgb}{1, 0, 0}
\newcommand{\RED}[1]{{\color{red} #1}}

\definecolor{green}{rgb}{0, 1, 0}
\definecolor{darkgreen}{rgb}{0.0, 0.2, 0.13}
\definecolor{darkseagreen}{rgb}{0.56, 0.74, 0.56}
\definecolor{officegreen}{rgb}{0.0, 0.5, 0.0}


\newcommand{\GREEN}[1]{{\color{green} #1}}

\definecolor{blue}{rgb}{0, 0, 1}
\newcommand{\BLUE}[1]{{\color{blue} #1}}

\definecolor{orange}{rgb}{1, 0.4, 0.0}
\newcommand{\ORANGE}[1]{{\color{orange} #1}}



%%&LaTeX
% \documentclass[draft]{article}
%
%
%\usepackage{amsmath,amsfonts,amsthm,amssymb,amscd,cancel,color}
%        \usepackage[notref]{showkeys}
%%\usepackage{enumitem}
%\usepackage{enumerate}
%\usepackage{verbatim}
%\usepackage[dvipdfmx]{graphicx}
%\usepackage{ulem}
%%\usepackage{hyperref}
%\newcommand\red[1]{ {#1}}
%\usepackage{cleveref}
%
%%\setlength{\topmargin}{-0.5in}
%\setlength{\textheight}{8in}
%\setlength{\oddsidemargin}{-0.1in}
%%\setlength{\evensidemargin}{0.in}
%\setlength{\textwidth}{6in}
%%\setlength{\headsep}{1.0cm}
%\setlength{\parindent}{0.75cm}
%
%\binoppenalty=9999 \relpenalty=9999
%
%\renewcommand{\Re}{\mathop{\rm Re}\nolimits}
%\renewcommand{\Im}{\mathop{\rm Im}\nolimits}
%\def\S{\mathhexbox278}
%\def\lan{\langle} \def\ran{\rangle}
%\def\ra{\rightharpoonup} \def\vf{\varphi}
% \newcommand{\Ph}{{\mathcal P}}
%
%\theoremstyle{plain}
%\newtheorem{theorem}{Theorem}[section]
%\newtheorem{lemma}[theorem]{Lemma}
%\newtheorem{proposition}[theorem]{Proposition}
%\newtheorem{corollary}[theorem]{Corollary}
%\theoremstyle{definition}
%\newtheorem{definition}[theorem]{Definition}
%\theoremstyle{remark}
%\newtheorem{remark}[theorem]{Remark}
%\newtheorem{notation}[theorem]{Notation}
%\newtheorem{example}[theorem]{Example}
%\newtheorem{probl}[theorem]{Problem}
%\newtheorem*{examples}{Examples}
%\newtheorem{assumption}[theorem]{Assumption}
%\newtheorem{claim}[theorem]{Claim}
%
%
%\newcommand{\R}{{\mathbb R}}
%\newcommand{\U}{{\mathcal U}}
%\newcommand{\Q}{{\mathbb Q}}
%\newcommand{\Z}{{\mathbb Z}}
%
%\newcommand{\N}{{\mathbb N}}
%\newcommand{\In}{{\mathbb Z}}
%\newcommand{\Tr}{{\mathcal T}}
%
%\def\im{{\rm i}}
%
%\newcommand{\C}{\mathbb{C}}
%\newcommand{\T}{\mathbb{T}}
%
%\newcommand{\E}{\mathcal{E}}
%\newcommand{\cH}{\mathcal{H}}
%
%\newcommand{\bG}{\mathbf{G}}
%\newcommand{\bO}{\mathbf{O}}
%\newcommand{\bc}{\mathbf{c}}
%
%\def\({\left(}
%\def\){\right)}
%\def\<{\left\langle}
%\def\>{\right\rangle}
%\newcommand{\sech}{{\mathrm{sech}}}
%\newcommand{\even}{{\mathrm{even}}}
%\newcommand{\odd}{{\mathrm{odd}}}
%\newcommand{\supp}{{\mathrm{supp}\ }}
%
%\numberwithin{equation}{section}
%
%\setcounter{section}{0}
%
%
%
%
%
%%
%%               MACRO STEFANO
%%
%
%\newcommand{\pare}[1]{\left(#1\right)}
%\newcommand{\bra}[1]{\left[#1\right]}
%\newcommand{\norm}[1]{\left\| #1\right\|}
%\newcommand{\av}[1]{\left| #1\right|}
%
%
%\newcommand{\set}[1]{\left\lbrace #1\right\rbrace}
%\newcommand{\spp}{\sigma_{\textnormal{pp}}}
%\newcommand{\Span}[1]{\textnormal{Span}\set{#1}}

















\begin{document}

\maketitle

\begin{abstract} We extend the result of   Kowalczyk, Martel and Munoz \cite{KMM2022} on the existence, in the context of spatially even solutions,  of asymptotic stability on a center hypersurface at the soliton of the defocusing power  Nonlinear Klein Gordon Equation with $p>3$,  to the case $2\ge p> \frac{5}{3}$.
\end{abstract}


\begin{footnotesize}
\tableofcontents
\end{footnotesize}

\section{Introduction}


We consider the Nonlinear-Klein Gordon Equation (NLKG) on the line
\begin{align}\label{eq:nlkg1}&
  \partial _t^2 u_1- \partial _x^2 u_1+u_1 - f(u_1)
  =0,
\end{align}
  where $  f(u_1)\defeq
  |u_1|^{p-1}u_1 $ with $ p_3\defeq2\ge p>p_4\defeq\frac{5}{3} $.
As in      Krieger et al.  \cite{KNS2012},
 Kowalczyk et al. \cite{KMM2022} and    Li  and   L\" uhrmann \cite {LL2023}, we  consider only even solutions, eliminating translations  and simplifying the problem. Translations pose  difficulties,
 see  the complications in the virial inequalities  in  Kowalczyk et al. \cite{KMMV2021}.

\noindent Setting ${\bm u}\defeq(u_1,u_2) ^ \intercal  \defeq(u_1,\dot u_1)^ \intercal  $, where in the sequel $\dot u=\partial _t u$ and $u'=\partial _x u$, we obtain the system
\begin{align}\label{NLKG}
\dot{{\bm u}}=\mathbf{J} \begin{pmatrix}
- \partial _x^2   +1 & 0 \\ 0 & 1
\end{pmatrix} {\bm u}+  {f}(u_1) \mathbf{j}, \text{ with } \mathbf{J}\defeq\begin{pmatrix}
0 & 1 \\ -1 & 0
\end{pmatrix}
\end{align}
and with $\mathbf{j}=(0,1)^\intercal$, later also $\mathbf{i}=(0,1)^\intercal$ (not to be confused with the imaginary unit $\im$).
We denote by $Q$ the standing wave, given, see for example in \cite[formula (3.1)]{CGN2007}, by
\begin{align}\label{eq:solit1}
 Q(x)\defeq \left( \frac{p+1}{2}  \right) ^{\frac{1}{p-1}}  \sech    ^{\frac{2}{p-1}}\left(  \frac{p-1}{2}x   \right) .
\end{align}
The system \eqref{NLKG} is a  Hamiltonian system for the symplectic form
\begin{align} &
\Omega({\bm u},{\bm v})\defeq\<\mathbf{J}^{-1}{\bm u},{\bm v}\>,      \label{eq:inner1} \\
&  \label{eq:inner0}
\<{\bm u},{\bm v}\> \defeq  \pare{ {\bm u},\overline{{\bm v}}} ,
&&
  \pare{
 {\bm u},  {\bm v} }  \defeq\int _{\R} {{\bm u}}(x) ^ \intercal \ {\bm v}(x) dx,
\end{align}
and  Hamiltonian,  or  energy function, given by
\begin{align}\label{eq:energy}
E({\bm u})=\frac{1}{2} \left( \norm{ u'_1} ^2 _{L^2\left( \R \right) }+ \norm{ u _2 } ^2_{L^2\left( \R \right) }  \right)-\int_{\R}\dfrac{\av{u_1}^{p+1}}{p+1}\,dx .
\end{align}
We consider the  space of even functions
\begin{equation*}
\bm{\mathcal{H}  }   ^{1}_{\even}
=
\set{{\bf u}=\pare{u_1, u_2} ^ \intercal  \ \middle| \ {\bf u} \in \boldsymbol{\mathcal{H}  }^1 \ \text{and} \ {\bf u}\pare{x} = {\bf u}\pare{-x} }
\end{equation*}
  with norm \begin{align}\label{eq:ennrom}
 \|{\bm u}\|_{\boldsymbol{\mathcal{H}  }   ^{1}   }^2=\|u_1\|_{H^1 }^2+\|u_2\|_{L^2 }^2.
\end{align}
We recall that Kowalczyk et al. \cite{KMM2022} proved, for $p>3$, that there exists an orbitally stable central hypersurface   in $\boldsymbol{\mathcal{H}  }   ^{1}_{\even}$. With minor modifications, their proof   extends also to our case  $2\ge p > \frac{5}{3}$, so we assume it.  Furthermore, Kowalczyk et al. \cite{KMM2022} proved that
$Q$ is asymptotically stable for solutions on this hypersurface.   Li  and   L\" uhrmann \cite {LL2023} consider    $p=p_3=2$,  with $f(u_1)=u_1^2$.
 In  Li  and   L\" uhrmann \cite {LL2023}  the  threshold resonance $\mu _3=1$  is potentially an obstruction to the proof, but since the corresponding generalized eigenfunctions  are odd in $x$, they are naturally orthogonal to the elements of $\boldsymbol{\mathcal{H}  }   ^{1}_{\even}$.  So the threshold resonance $\mu _3=1$
does not affect the argument, in line with the discussion in Kowalczyk and Martel \cite{KM22} on the asymptotic stability of odd perturbations of the standing kink of the $\phi ^4$ model, where the threshold resonance is \textit{even}, and so does not affect odd solutions.  In the case $p=2$, for the non smooth nonlinearity $f(u_1)=|u_1| u_1 $ treated here,  the same will happen in our proof.



  Notice that the case   $2<p<3$  is simpler.  Case $p=3$ is different and still open, because of a  \textit{even} threshold resonance,   certainly affecting even solutions of \eqref{eq:nlkg1}. For a partial result, framed with   very different techniques, see  L\" uhrmann and Schlag \cite{LS2023}. Finally, as $p\rightarrow 1 ^+$, besides the cases with resonances, there is also an increase of the number of eigenvalues, with some converging to 0. The lack of regularity  of the nonlinearity becomes an issue since normal forms arguments like in \cite{CMS2023} require   a sufficient amount of regularity of the nonlinearity.


\subsection{On the linearization of \Cref{NLKG} around $ Q $} Here we give some information about the linearization of   \Cref{NLKG} around $ Q $, cfr. \cite{CGN2007}.
Key to the analysis is the operator (often denoted, in the literature, by $L_+$),
\begin{align}\label{eq:lin1}
  L_0\defeq - \partial _x^2   +1-p Q ^{p-1} .
\end{align}
  It is well known, see Grillakis et  al. \cite[\S 6]{GSS1},  that $Q$ is orbitally unstable.  Indeed, if we set
\begin{align}\label{eq:lin1.1}
\mathbf{L}_0\defeq\begin{pmatrix}
L_0 & 0 \\ 0 & 1
\end{pmatrix},
\end{align}
the linearization
    of \eqref{NLKG}   at $Q \mathbf{i}$ is
  \begin{equation*}
  \mathbf{J}\mathbf{L}_0
  \defeq
  \pare{
\begin{array}{cc}
0&1 \\ -L_0 & 0
\end{array}
  } ,
  \end{equation*}
   admits   point spectrum
\begin{equation*}
\spp \pare{\mathbf{J}\mathbf{L}_0 } = \set{ \nu \defeq \pm \sqrt{-\mu} \ \middle| \ \mu \in \spp \pare{L_0}
}.
\end{equation*}
In particular the elements of $ \spp \pare{L_0} $   as functions of $ p $ are
\begin{align}
\label{eq:pM_kj}
p_N \defeq  \begin{cases}\frac{N+1}{N-1} \text{  for } N > 1,\\  \infty \text{ if } N=1, \end{cases}
&&
 k_j \defeq \frac{p+1}{2}-\frac{j(p-1)}{2} , \quad j\in\mathbb{N},
\end{align}
if $ p_{N+1}  < p<p_N$   then  $ \spp\pare{L_0} $     has  $ N+1 $ elements    and they   are explicitly  given by
\begin{align} \label{eq:eigenv} \mu _m \defeq 1-k^2_m,
&& m=0, \ldots , N.
   \end{align}
%Let us thus remark, from \cref{eq:eigenv}, that when $ p>1 $ we have that if $ m=0 $ then $ \mu_0 < 0 $. This implies that $ {\bf J}\mathbf{L}_0 $, has a positive, and thus spectrally and also dynamically unstable eigenvalue $\nu _0=\sqrt{-\mu _0}>0$. Moreover it is known (see \cite{DT1979}) that the operator $ L_0  $ has  continuous spectrum $ \sigma_{\text{c}}\pare{L_0} = [1, \infty) $ and generalized eigenfunctions $ \pare{ \mathsf{e}\pare{\xi} }_{\xi \neq 0} $ \todoSS{Please Scipio, check}\begin{equation}\label{eq:geneigenfLQ}L_0 \ \mathsf{e}\pare{\xi} = \pare{1+\xi^2} \mathsf{e}\pare{\xi}.\end{equation}Let us notice that defining\begin{equation}\label{eq:geneigenfJLQ}\bm{\mathsf{e}}_{\pm}\pare{\xi} \defeq \binom{1}{\pm \ii \sqrt{1+\xi^2}} \mathsf{e}\pare{\xi},\end{equation}we have\begin{equation*}\mathbf{JL}_0 \  \bm{\mathsf{e}}_{\pm}\pare{\xi} = \pm \ii \sqrt{1+\xi^2}\  \bm{\mathsf{e}}\pare{\xi}.\end{equation*}



\noindent When $ p=p_N $ for some $ N\geq 1 $, then from \cref{eq:eigenv} we have $ \mu_N =1 $ and the linear operator $ {\bf J}\mathbf{L}_0 $ has a threshold  resonance at the boundary of the continuous spectrum. Here we will consider \begin{align}\label{def:valueN}
  N =\begin{cases}
 3   \text{   if }  p_3=2> p>p_4=\frac{5}{3}  \\
 2 \text{   if }  p=2
\end{cases}
\end{align}
  and    we  will use
 \begin{align}\label{eq:valk}
   k_0 = \frac{p+1}{2}, &&  k_1=1  , && k_2= \frac{3-p}{2} , && k_3= 2-p,  &&  k_4= \frac{5-3p}{2}.
 \end{align}
   Like in   Krieger et al.  \cite{KNS2012},
 Kowalczyk et al. \cite{KMM2022} and    Li  and   L\" uhrmann \cite {LL2023}, we will prove the   asymptotically stability of $Q$ in the central hypersurface of Kowalczyk et al. \cite{KMM2022}. The constraint  $2\ge p> \frac{5}{3}$  renders the nonlinearity more problematic  compared  \cite{KNS2012,KMM2022,LL2023},  in terms of lower   differentiability and bigger strength. %We will denote with $ D_0 $ and $ \mathbf{D}_0 $ the unperturbed operators\begin{align*}D_0 \defeq -\partial_x^2 + 1,&&{\bf D}_0 \defeq\begin{pmatrix}- \partial _x^2   +1 & 0 \\ 0 & 1\end{pmatrix}  . \end{align*}
For % \footnote{The operators $ S_j $ here correspond to the operators $ A_j $ in \cite[Proposition 1.11]{CMS2023}}
\begin{align}\label{eq:defSj}
   S_j \defeq\partial_{x}-k_j \tanh \left(  \frac{p-1}{2}x   \right)
 \end{align}
and for the eigenvalues $ \mu_j $   in \cref{eq:eigenv} ,
 we have $
 \ker \pare{L_0-\mu_j} = \Span{\varphi_j} ,$
 with
 \begin{align}
 \label{eq:eigC}
% \begin{aligned}
 \varphi_0 \defeq Q ^{k_0} ,  &&
   \varphi_1 \defeq S^\ast_0 Q^{k_1} =   \frac{p-1}{2} \ Q'  ,  && \varphi_2 \defeq  S_0^*S_1^* Q ^{k_2} , && \varphi_3 \defeq S_0^*S_1^* S_2^* Q ^{k_3}.
% \end{aligned}
 \end{align}
We emphasize that the
$ S^\ast_j $ change the symmetry of a function,  so that we have the very general fact that $ \varphi_0 $ and $ \varphi_2 $ are even while $ \varphi_1 $ and $ \varphi_3 $ are odd.  It is elementary and well known \cite{DT1979}, that
 \begin{equation}
 \label{eq:asympt_eigenf}
 \av{\varphi_j\pare{x}}\sim e^{-k_j\av{x}} \text{ as }   \av{x}\to \infty  .
 \end{equation}
Using   \eqref{eq:eigC}, for
\begin{align}\label{eq:matr_eig1}
{\mathbf{Y}}_{\pm }\defeq\begin{pmatrix}
\varphi_0\\ \pm  \nu _0\varphi_0
\end{pmatrix} ,
&&
{\mathbf{Z}}_{\pm }\defeq\begin{pmatrix}
\varphi_0\\ \pm  \nu _0 ^{-1}\varphi_0
\end{pmatrix} ,
\end{align}
we have
 \begin{align*}
\mathbf{J}\mathbf{L}_0  {\mathbf{Y}}_{\pm }   = \pm \nu _0   {\mathbf{Y}}_{\pm }
 .
\end{align*}
We set
\begin{align}\label{eq:matr_eig1}
\boldsymbol{\Phi}_0\defeq \frac{  {\mathbf{Y}}_{+ }- \im {\mathbf{Y}}_{- }}{2} = \frac{1}{2}\binom{1-\ii}{\pare{1+\ii}\nu_0} \varphi_0 \  .
\end{align}
Then,
\begin{align}\label{eq:matr_eig12}
\mathbf{J}\mathbf{L}_0 \boldsymbol{\Phi}_0 =  \nu _0 \overline{\boldsymbol{\Phi}}_0.
\end{align}
Similarly, setting $ \lambda  = \sqrt{\mu _2}>0 $ we define
\begin{align}\label{eq:matr_eig2}
\boldsymbol{\Phi}_2\defeq\begin{pmatrix}
 1 \\ \im \lambda
\end{pmatrix} \ \varphi_2 \  ,
\end{align}
so that
\begin{align}\label{eq:matr_eig3}
\mathbf{J}\mathbf{L}_0 \boldsymbol{\Phi}_2=\im \lambda  \boldsymbol{\Phi}_2   &&
 \mathbf{J}\mathbf{L}_0 \overline{\boldsymbol{\Phi}}_2=-\im \lambda \overline{\boldsymbol{\Phi}}_2.
\end{align}
%Let us notice that since $ \varphi_1 $ and $ \varphi_3 $ are odd they shall play no role in the forthcoming analysis of the problem. \\
Notice that for \begin{align}\label{def:calA}
 \mathcal{A} =S_0\cdots S_N   , \text{ for $N$  as in \eqref{def:valueN},}
%&& N =\begin{cases} 3   \text{   if }  p_3=2> p>p_4=\frac{5}{3}  \\ 2 \text{   if }  p=2\end{cases} ,
\end{align}
 we have
 \begin{align}\label{eq:DarConj2}
\mathcal{A}  L_0=L_{N+1}\mathcal{A} ,
\end{align}
where
\begin{align}\label{eq:opLj}
 L_{j}\defeq  - \partial _x^2   +1-k_{j-1}k_{j }\frac{2}{p-1}Q ^{p-1}.
\end{align}
The procedure outlined in \Cref{eq:opLj,eq:DarConj2,def:calA} transforms the linearization of \cref{NLKG} around $ Q $ from
$
\dot U = {\bf J} \mathbf{L}_0 U ,
$
into the new linear equation, in the auxiliary variable $ V : = \mathcal{A} U $,
\begin{align}
\label{eq:linearNLKG_2}
\dot V = {\bf J } {\bf L}_{ N+1} V ,
&&
{\bf L}_{ N+1}
: =
\left(
\begin{array}{cc}
L_{N+1} & 0
\\
0 & 1
\end{array}
\right) ,
\end{align}
%the advantage of \cref{eq:linearNLKG_2} compared to \cref{eq:linearNLKG} is that
where for  $ p_3 >  p>p_4 $ the operator $L_{N +1}=L_4$ has a repulsive potential, in the sense that
\begin{align}\label{eq:repuls}
  -k_{3}k_{4 }\frac{2}{p-1}   x \left(Q ^{p-1} \right) ' <0 \text{  for all } x\neq 0,
\end{align}
while for $p=2$ we have $L_{N +1}=L_3=- \partial _x^2   +1$  but fortunately the parity of the solutions obviates to the lack of this repulsivity. \\

 \noindent Given a constant $\kappa >0$, we consider the spaces defined by the following norms,
\begin{align}\label{eq:enwei}
 \| {\bm u}\|_{ \boldsymbol{\mathcal{H}}^{1 }_{-\kappa }}\defeq \| \sech \left( \kappa x \right)   {\bm u}\|_{ \boldsymbol{\mathcal{H}}^{1 }  }  , &&
  \| {\bm u}\|_{ L^{2 }_{-\kappa }}\defeq \| \sech \left( \kappa x \right)   {\bm u}\|_{ L^2  }.
\end{align}
In the sequel   $\kappa>0$ is fixed and smaller than a certain finite set of positive numbers coming up in the proof. For ${\bm z}=(z_1,z_2) ^\intercal \in \C^2$, we set
\begin{align}\label{eq:rp}&
 \boldsymbol{\Phi}[{\bm z}] \defeq Q \mathbf{i} +  \widetilde{\boldsymbol{\Phi}}[{\bm z}] ,
&&
 \widetilde{\boldsymbol{\Phi}}[{\bm z}] \defeq z_1 \boldsymbol{\Phi}_0+z_2 \boldsymbol{\Phi}_2+  \overline{z}_1 \overline{\boldsymbol{\Phi}}_0+\overline{z}_2 \overline{\boldsymbol{\Phi}}_2 .
\end{align}
We set
\begin{align}\label{eq:cont_mod}
 \boldsymbol{\mathcal{H}}^{1 }_{c} \defeq \set{ \boldsymbol{\eta} \in \boldsymbol{\mathcal{H}  }   ^{1}_{\even}(\R , \C ^2 ) \ \middle| \  \left \langle  \mathbf{J}\boldsymbol{\eta} , D_{\bm z} \boldsymbol{\Phi}[{\bm z}] \boldsymbol{\Theta} \right  \rangle  =0 \text{ for all }\boldsymbol{\Theta}\in \C^2} .
\end{align}
Our main result is the following.

\begin{theorem}\label{thm:main1}
Assume \ref{ass:FGR}.
for any $a>0$ and $ \epsilon >0$
  there exists $\delta_0>0$ such that for  $\delta \in (0,\delta_0)$ if
  \begin{align}\label{eq:cond_sta}
    \sup _{t\ge 0} \norm{ {\bm u} (t) -  Q \mathbf{i} }  _{\boldsymbol{\mathcal{H}  }   ^{1}_{\even}(\R , \R ^2 )}<\delta
  \end{align}
    then there exists ${\bm z}\in C ^1\pare{ \R , \C ^{2} } $ and  $ \boldsymbol{\eta }\in C^0 \pare{ \R ,  \boldsymbol{\mathcal{H}}^{1 }_{c} }$ such that we have a global representation
 \begin{align}
 {\bm u}(t) =     {\boldsymbol{\Phi}}\bra{ {\bm z}(t) }   + \boldsymbol{\eta }
   (t)
 \label{eq:main1}
\end{align}
such that, for  $I=[0,+\infty)$,
\begin{align}
 \int _{I }  \|     \boldsymbol{\eta }  (t) \| ^2 _{\boldsymbol{\mathcal{H}}^{1} _{-a}(\R )} \dd t   \le \epsilon,     \label{eq:main2}
\end{align}
   and
\begin{align}&
\lim_{t\to  \infty} {\bm z} (t) =0     \text{  .  }   \label{eq:main3}
\end{align}
\end{theorem}


For  $\delta _0>0$ we  set
\begin{align}& \mathcal{B}_0\defeq \set{    \boldsymbol{\varepsilon } \in  \boldsymbol{\mathcal{H}  }   ^{1}_{\even}(\R , \R ^2 )  \ \middle| \   \|     \boldsymbol{\varepsilon}    \| ^2 _{\boldsymbol{\mathcal{H}}^{1}  }  <\delta _0   \text{  and }   \<   \boldsymbol{\varepsilon } , \mathbf{Z} _+\> =0} .  \label{eq:stbm}
\end{align}
The following   existence and uniqueness of a stable central hypersurface, can be proved like   Theorem 2 in  Kowalczyk et al. \cite{KMM2022}.
\begin{theorem}\label{thm:main2}
There exist  $C,\delta _0>0$ and a Lipschitz function $ h: \mathcal{B}_0 \to \R$
 with $h(0)=0$ and $|h(\boldsymbol{\varepsilon })|\le  C \| \boldsymbol{\varepsilon } \|  _{\boldsymbol{\mathcal{H}}^{1}  } ^{ \frac{p+1}{2}} $ such that, for
 \begin{align}\label{eq:stabman}
     \mathcal{N}\defeq\set{ Q \mathbf{i} + \boldsymbol{\varepsilon }+ h(\boldsymbol{\varepsilon }) \mathbf{Y}_+  \ \Big| \  \boldsymbol{\varepsilon }\in      \mathcal{B}_0             } ,
  \end{align}
 we have the following:
 \begin{description}
   \item[a]  if ${\bm u}_0\in   \mathcal{N}$, then the corresponding solution of \eqref{NLKG} is  defined  for  all $t\ge 0$   and  \begin{align}
  \sup _{t\ge 0} \| {\bm u} (t) - Q \mathbf{i}  \| _{\boldsymbol{\mathcal{H}  }   ^{1}_{\even}(\R , \R ^2 )}\le C  \| {\bm u}_0 - Q \mathbf{i}  \| _{\boldsymbol{\mathcal{H}  }   ^{1}_{\even}(\R , \R ^2 )} .
 \label{eq:main21}
\end{align}
   \item[b] if a a solution ${\bm u} (t)$ defined  for  all $t\ge 0$  satisfies   \begin{align}
    \| {\bm u} (t) - Q \mathbf{i}  \| _{\boldsymbol{\mathcal{H}  }   ^{1}_{\even}(\R , \R ^2 )}   < \frac{\delta _0}{2}
 \label{eq:main22}
 \end{align} then ${\bm u} (t) \in  \mathcal{N}$ for all $t\ge 0$.


 \end{description}

\end{theorem} \qed



As a corollary, it follows that  the conclusions of  Theorem  \ref{thm:main1}   are true for all the solutions of \eqref{NLKG} in the stable center hypersurface $\mathcal{N}$.   This paper is  focused uniquely on the proof of Theorem  \ref{thm:main1} and, for dispersion, uses the framework of Kowalczyk et al., even though for technical results   often refers to \cite{CMS2023,CM2019, CM2022, CM2023}. The gist of the argument and the intuition behind this paper is that, while obviously $(\av{x} ^{p-1}x)''  $  is singular at $x=0$, we  can expand the nonlinearity  by an appropriate partitioning of spacetime, for example with the distinction in \cref{case:one} and \cref{case:two} in the proof of \Cref{lem:rpcorr}. Furthermore, to get estimates like \eqref{eq:Ijest2}, and specifically the fact that $ Q ^{p-2}\varphi ^2_2\sim e^{-|x|} $, we neutralize the growth of  $ Q ^{p-2}$ as $x\to  \infty$    with the decay of $\varphi ^2$.  For $p_5= \frac{3}{2}<p<p_4= \frac{5}{3}$ then we  have an additional term
$ Q ^{p-2}\varphi ^2_4 \sim  e^{ (2p-3)|x|}  $, unbounded as $x\to  \infty$, and so our proof breaks down.

There is a large literature, beyond the above references, like the influential  Kowalczyk et al. \cite{KMM2017} and   papers with a different framework and origin,  involving dispersive estimates, like \cite{CLL2020,CP2022,DM2020,GP2022,GPZ2023, LP2022} and   references therein.












\subsection{Refined profile and Fermi Golden Rule }\label{sec:refprof}

 In our earlier work \cite{CMS2023}, we referred to the notion of \textit{Refined Profile}. Our problem here requires only a special case of this notion.
We introduce the notation
\begin{align}\label{eq:rp--1}&  \tilde{{\bm z}}_0=\tilde{{\bm z}}_0[{\bm z}]=(\nu _0 \overline{z}_1, \im \lambda z_2) .
\end{align}
Then $ \boldsymbol{\Phi}[{\bm z}] = \pare{ \boldsymbol{\Phi}[{\bm z}]_1, \boldsymbol{\Phi}[{\bm z}]_2  }^\intercal $, satisfies, using \cref{eq:rp--1,eq:rp,eq:matr_eig3,eq:matr_eig12}, the equation
\begin{equation}\label{eq:LQonPhitilde}
D_{\bm z}\boldsymbol{\Phi}[{\bm z}]\tilde{{\bm z}}_0 = {\bf J}\mathbf{L}_0 \tilde {\bm \Phi}\bra{z}.
\end{equation}
Thus using the fact that $ Q {\bf i} $ is a stationary solution of \cref{NLKG} %and the fact that the potential in \cref{eq:lin1} is $ - f'\pare{Q} $
we obtain
\begin{align}\label{eq:rp1}&
D_{\bm z}\boldsymbol{\Phi}[{\bm z}]\tilde{{\bm z}}_0 = \mathbf{J}
\begin{pmatrix}
	-\partial_x^2+1 & 0\\ 0 &1
\end{pmatrix}
%\mathbf{L}_0
 \boldsymbol{\Phi}[{\bm z}] + f(\boldsymbol{\Phi}[{\bm z}]_1) \mathbf{j} + \hat{\bm R}[{\bm z}], \text{ with}
\\ \label{eq:errp1}&
\hat{\bm R}[{\bm z}] := -\left(  f(\boldsymbol{\Phi}[{\bm z}]_1)   - f(Q)-f' (Q)  \widetilde{\boldsymbol{\Phi}}[{\bm z}]_1     \right)  \mathbf{j} .
\end{align}
It is useful to improve the remainder $ \hat{\bm R}[{\bm z}]$  as follows.


\begin{lemma} \label{lem:rpcorr} There exist constants $C_1,\delta _1 >0$ and  a map $ \tilde{{\bm z}}_R : D _{\C^2} (0, \delta _1) \to \C^2 $ s.t. the following   holds:
\begin{enumerate}[\bf i)]

\item \label{item:rem_better_3} for
\begin{equation}
\label{eq:errp2}  {\bm R}[{\bm z}] : = D_{\bm z}\bm{\Phi}[{\bm z}] \ \tilde{{\bm z}}_R \bra{{\bm z}}
+\hat{\bm R}[{\bm z}],
\end{equation}
we have
\begin{align} \label{R:orth} &
 \left< \mathbf{J}{\bm R}[{\bm z}]  , D _{{\bm z}}\boldsymbol{\Phi}[{\bm z}]   \Theta  \right>  =0 \text{ for all   $\Theta \in \C ^2 $  and all $ {\bm z} \in D _{\C^2} (0, \delta _1)$;}
\end{align}


\item  \label{item:rem_better_20} we have
 \begin{equation}\label{item:rem_better_1}
 	\av{ \tilde{{\bm z}}_R ({\bm z})} \le C _1 |{\bm z}|^2   \text{  for any ${\bm z}\in D _{\C^2} (0, \delta _1)$;}
 \end{equation}
\item \label{item:rem_better_1.1} $\tilde{{\bm z}}_R \in C^1  \pare{ D _{\C^2} (0, \delta _1), \C^2  }$;
\item \label{item:rem_better_2} $\tilde{{\bm z}}_R$ is twice differentiable in 0.


\end{enumerate}

\end{lemma}

\begin{proof}
We prove  \Cref{item:rem_better_3}.
  First of   all,  \eqref{R:orth}  is equivalent to
\begin{align}\label{R:orth1} &
 \<   \mathbf{J} D _{{\bm z}}\boldsymbol{\Phi}[{\bm z}]  \tilde{{\bm z}}_R  , D _{{\bm z}}\boldsymbol{\Phi}[{\bm z}]   \Theta  \>  =-\< \mathbf{J}\widehat{{\bm R}}[{\bm z}]   , D _{{\bm z}}\boldsymbol{\Phi}[{\bm z}]   \Theta  \> , \text{     for all   $\Theta \in \C ^2 $  and all $ {\bm z} \in D _{\C^2} (0, \delta _1)$.}
\end{align}
By the definition  of $\boldsymbol{\Phi}[{\bm z}]$ in \eqref{eq:rp}, the left hand side  in \eqref{R:orth1} is constant in ${\bm z}$. Hence, varying $\Theta$ in a $\R$--basis of $\C ^2$, the above is a linear and solvable system in $\tilde{{\bm z}}_R$.\\


 We prove \Cref {item:rem_better_20}.
 Let us take  $\Theta = \Theta_j  $ s.t.
$ D _{{\bm z}}\boldsymbol{\Phi}[{\bm z}]  \Theta_j   =\boldsymbol{\Phi}_j $.  We have
\begin{align}
 | \< \mathbf{J}\widehat{{\bm R}}[{\bm z}]    , \boldsymbol{\Phi}_j   \> |\lesssim  & \   \int     |I \varphi  _j|  \dd x   \text{  with }
   I  \defeq  & \   f(Q+ \widetilde{\boldsymbol{\Phi}}[{\bm z}]_1)   - f(Q)-f' (Q)  \widetilde{\boldsymbol{\Phi}}[{\bm z}]_1     . \label{eq:defI}
\end{align}
Notice that
\begin{align} \label{eq:esttildeph}
  \left |   \widetilde{\boldsymbol{\Phi}}[{\bm z}]_1\right | \lesssim |z_1| \ |\varphi _0|  + |z_2| \ |\varphi _2| ,
\end{align}
 with $|\varphi_0|\sim e ^{-\frac{p+1}{2}|x|}$ and $|\varphi_2|\lesssim e ^{-k_2|x|}=e ^{-     \frac{3-p}{2}|x|} $ by \cref{eq:asympt_eigenf}.
  Now we estimate pointwise $I $.

\begin{case}\label{case:one}
Let $|\widetilde{\boldsymbol{\Phi}}[{\bm z}]_1| \leq  Q/2$.   By $Q\sim e^{-|x|}$,  $k_2=\frac{3-p}{2} $ and $ \av{f''\pare{Q+\tau \tilde{\bm{\Phi}}\bra{{\bm z}}_1 }} \sim Q^{p-2} $, for $ \tau\in \bra{0, 1} $ we have
\begin{multline}
 |I|  \lesssim  \sup_{\tau\in\bra{0, 1}}\av{f''\pare{Q+\tau \tilde{\bm{\Phi}}\bra{{\bm z}}_1 }} \  \pare{  |z_1|^2 |\varphi_0|^2 + |z_2|^2 |\varphi_2|^2 }   \lesssim  |Q|^{p-2} \pare{ |z_1|^2 |\varphi_0|^2 + |z_2|^2 |\varphi_2|^2 }    \\
  \lesssim  Q ^{p-2} |{\bm z}|^2 e^{-2k_2|x|}\sim  |{\bm z}|^2 e^{|x| \pare{  2-p - (3-p) }  } = |{\bm z}|^2 e^{-|x| } .\label{eq:Ijest}
\end{multline}
\end{case}


\begin{case}\label{case:two}
Let $ |\widetilde{\boldsymbol{\Phi}}[{\bm z}]_1| >  Q/2 $.   Then
\begin{align}\label{eq:dicot1}
    |z_1| \ |\varphi_0|  + |z_2| \ |\varphi_2| \gtrsim |\widetilde{\boldsymbol{\Phi}}[{\bm z}]_1|\gtrsim Q && \Longrightarrow &&  |z_2| \ |\varphi_2|  \gtrsim Q
\end{align}
  by $|\varphi_0|\sim e ^{-\frac{p+1}{2}|x|}$, $Q\sim e ^{- |x|} $ and $|z_1|\lesssim \delta \ll 1$, which obviously yield  $ |z_1| \ |\varphi_0|\ll Q$  always. Then
\begin{equation}
\begin{aligned}
   |I|\lesssim   \left |  \widetilde{\boldsymbol{\Phi}}[{\bm z}]_1  \right  |^p         \lesssim |z_2 \varphi_2| ^p   \lesssim |z_2|^p e ^{- p \frac{3-p}{2} |x|} &=   |z_2|^p   e^{ - (2-p) \frac{ p-1}{2}  |x|}        e ^{ 2^{-1}\left( (2-p)  ( p-1)- p (3-p)    \right)|x|}
    \\&=   |z_2|^p   e^{ - (2-p) \frac{ p-1}{2}  |x|}      e^{-|x| } .  \label{eq:Ijest1}
\end{aligned}
\end{equation}
It is easy to conclude from \eqref{eq:dicot1}, that $|z_2| e ^{-\frac{3-p}{2}|x|} \gtrsim e^{-|x|}$, and so $|z_2|  \gtrsim e^{ -   \frac{ p-1}{2}  |x|}$ or $|z_2| ^{2-p} \gtrsim e^{ - (2-p) \frac{ p-1}{2}  |x|}$. This proves that $ \av{I}\lesssim \av{{\bm z}}^2 e^{-\av{x}} $ when $ |\widetilde{\boldsymbol{\Phi}}[{\bm z}]_1| >  Q/2 $.
\end{case}
 Cases \ref{case:one} and \ref{case:two}  yield
\begin{align}
   \av{I}\lesssim      |{\bm z}|^2    e^{-|x| }  \text{ for all $x\in \R$}.  \label{eq:Ijest2}
\end{align}
     {  This yields    \eqref{item:rem_better_1}
and, by dominated convergence,   $\tilde{{\bm z}}_R \in C^0 \pare{ D _{\C^2} (0, \delta _1), \C^2 }$, proving \Cref{item:rem_better_20}.}  \\

We prove \Cref{item:rem_better_1.1}. It suffices to prove that the right hand side in \eqref{R:orth1} is in $  C^1 \pare{ D _{\C^2} (0, \delta _1), \C^2 }$. We have
\begin{equation}
\label{eq:first_differential_hatR}
D_{{\bm z}}\hat{ {\bm R}} \bra{{\bm z}} \ \bm{\zeta} = \pare{f'\pare{\bm{\Phi}\bra{{\bm z}}_1} - f'\pare{Q}} D_{{\bm z}} \bm{\Phi}\bra{{\bm z}}_1  \bm{\zeta} \ \mathbf{j}.
\end{equation}
 For $ |\widetilde{\boldsymbol{\Phi}}[{\bm z}]_1|\ll  Q $ we have
\begin{align}\label{eq:estder1}
  |f' (Q)   -  f'(\boldsymbol{\Phi}[{\bm z}]_1)|\lesssim \left |f'' (Q)   \widetilde{\boldsymbol{\Phi}}[{\bm z}]_1 \right |  \lesssim  |{\bm z}| e^{|x| \left( 2-p- \frac{3-p}{2}\right) } =  |{\bm z}| e^{-\frac{p-1}{2}|x|}
\end{align}
while  for  $ |\widetilde{\boldsymbol{\Phi}}[{\bm z}]_1|\gtrsim   Q $, by $|z_2|\gtrsim e^{- \frac{p-1}{2} |x|}$, and so, by   $|z_2| ^{2-p} \gtrsim e^{ - (2-p) \frac{ p-1}{2}  |x|}$,
 we have
\begin{align}\label{eq:estder2}
  |f' (Q)   -  f'(\boldsymbol{\Phi}[{\bm z}]_1)| &\lesssim    |     \widetilde{\boldsymbol{\Phi}}[{\bm z}]_1   | ^{p-1} \lesssim |z_2|^{p-1}
   e^{-(3-p)\frac{p-1}{2}|x|} =|z_2|^{p-1}
   e^{-(2-p)\frac{p-1}{2}|x|}    e^{- \frac{p-1}{2}|x|} \\& \lesssim  |{\bm z}| e^{-\frac{p-1}{2}|x|}.\nonumber
\end{align}
So, since the differentiation in \cref{eq:first_differential_hatR} is well defined and is uniformly bounded by an integrable function,
    by dominated convergence   we can commute derivative and integral and conclude
\begin{align*}
 D_{{\bm z}}\< \mathbf{J} \hat{{\bm R}}[{\bm z}]   , D _{{\bm z}}\boldsymbol{\Phi}[{\bm z}]   \Theta  \> \bm{\zeta} = \< \mathbf{J} D_{\bm z}\widehat{{\bm R}}[{\bm z}]  \bm{\zeta} , D _{{\bm z}}\boldsymbol{\Phi}[{\bm z}]   \Theta  \>
\end{align*}
with, furthermore, the differential continuous in $\mathbf{z}$. So, from  \cref{R:orth1} we obtain
  $\tilde{{\bm z}}_R \in C^1 \pare{  D _{\C^2} (0, \delta _1), \C^2 }$.\\


{  We prove  \Cref{item:rem_better_2}.    From \cref{eq:first_differential_hatR} and the linearity in $\mathbf{z}$ of $ \widetilde{ \bmPhi}\bra{{\bm z} }$ ,
\begin{equation*}
 D^2_{\bm z} \hat{\bm R}\bra{\bm z} \pare{{\bm \zeta}^1, {\bm \zeta}^2 }
 =
 f''\pare{\bmPhi \bra{\bmz}_1}  \widetilde{ \bmPhi}\bra{{\bm \zeta}^1}_1  \   \widetilde{\bmPhi}\bra{{\bm \zeta}^2}  _1    \mathbf{ j} .
\end{equation*}
%so that, by $Q(x) \sim e^{-|x|}$, \eqref{eq:valk}  and \eqref{eq:asympt_eigenf}, we have \begin{align*}\av{D^2_{\bmz} \hat {\bm R}\bra{0} \pare{{\bm \zeta}^1, {\bm \zeta}^2 }}&\lesssim \av{f''\pare{Q}} \pare{\av{\varphi_0}^2 + \av{\varphi_2}^2}  \av{\bm \zeta^1}\av{\bm \zeta^2}  \lesssim e^{-|x|} \av{\bm \zeta^1}\av{\bm \zeta^2}.\end{align*}
Now, for fixed $A=R,I$, $z_{a R}:=\Re z_a$, $z_{a I}:=\Im z_a$ and $a=1,2$, we have
\begin{align}\label{eq:2diff}& \< \mathbf{J}  \( \partial _{z_{a A}}\widehat{{\mathbf{R}}}[\mathbf{z}] -  D_\bmz  \partial _{z_{a A}}\widehat{{\mathbf{R}}}[0]  \bmz   \)    , D _{\mathbf{z}}\boldsymbol{\Phi}[\mathbf{z}]   \Theta  \>   =
- \<   II ,1 \>  \\&  \text{with }II:= \( f'(\boldsymbol{\Phi}[\mathbf{z}]_1) - f'(Q)- f''(Q)  \widetilde{\boldsymbol{\Phi}}[\mathbf{z}] _1          \)    \partial _{z_{a A}}\widetilde{\boldsymbol{\Phi}}[\mathbf{z}]_1      \overline{D _{\mathbf{z}}\boldsymbol{\Phi}[\mathbf{z}] _1  \Theta }.\nonumber
\end{align}
To show that the above expression is $o(|\mathbf{z}|)$ we bound $II$. For   $ |\widetilde{\boldsymbol{\Phi}}[\mathbf{z}]_1|\ll  Q $, we have
\begin{align*} &
  |II| \lesssim  \left |f'''(Q) \right | \  \left | \widetilde{\boldsymbol{\Phi}}[\mathbf{z}] _1   e^{- k_2|x|} \right | ^{2} \lesssim
  |\mathbf{z}|^{2} e^{ \( 3-p - 2 (3-p)\)   |x| } =  |\mathbf{z}|^{2} e^{-   ( 3-p   )  |x|  } .
\end{align*}
  For $ |\widetilde{\boldsymbol{\Phi}}[\mathbf{z}]_1|\gtrsim  Q $,
like in \eqref{eq:estder2} we have  %, which completes the proof of the 2nd order differentiability of $\widetilde{\mathbf{z}} _R$ at $\mathbf{z}=0$,
\begin{align*}
  |II| &\lesssim     \left | \widetilde{\boldsymbol{\Phi}}[\mathbf{z}] _1   \right | ^{p-1} e^{- 2k_2|x|}\sim   \left | z_2 \psi _2   \right | ^{p-1} e^{- 2k_2|x|} \lesssim   |z_2 | ^{p  }   |z_2 | ^{ -1 } e ^{-\frac{(3-p) ^2}{2}|x|}\\&  \lesssim  | \mathbf{z} | ^{p  }  e^{  \frac{1}{2}\(  {p-1}   -(3-p) ^2 \) |x|} \le  | \mathbf{z} | ^{p  } e^{-|x|} .
   \end{align*}
So  we can use dominated convergence obtaining
\begin{align*}
   \lim _{\bm z\to 0}   \frac{1}{\av{\bm z}} \< \mathbf{J}  \( \partial _{z_{a A}}\widehat{{\mathbf{R}}}[\mathbf{z}] -  D_\bmz  \partial _{z_{a A}}\widehat{{\mathbf{R}}}[0]  \bmz   \)    , D _{\mathbf{z}}\boldsymbol{\Phi}[\mathbf{z}]   \Theta  \> =    \lim _{\bm z\to 0}   O\(  | \mathbf{z} | ^{p -1 }\) =0    .
\end{align*}
This shows that  the function in \eqref{eq:2diff} is differentiable in $\bm z=0$ for any $a=1,2$ and any $A=R,I$ or, equivalently, that the right hand side in \eqref{R:orth1}, and so also $\tilde{{\bm z}}_R $, is twice differentiable in $\bm z=0$. }

   \end{proof}



We set now
\begin{align}\label{eq:tildez}& \tilde{\bm z}\defeq  \tilde{{\bm z}}_0 + \tilde{{\bm z}}_R
\end{align}
and  consider
\begin{align}\label{eq:G0}  z_2^2\mathbf{G}_2 (x)+ \overline{z}_2^2\overline{\mathbf{G}}_2 (x) &&  \text{  where}
&&
\mathbf{G}_2=  D_{\bm z}\boldsymbol{\Phi}[{\bm z}] \left . \partial ^2_2\tilde{{\bm z}}_R \right |_{{\bm z}=0} -f''\pare{Q} \varphi _2^2 \mathbf{j},
\end{align}
where we remark that $f''\pare{Q} \varphi _2^2 =-p (p-1) Q ^{p-2} \varphi _2^2 \sim  -e^{ -       |x|      }$ for $x\to \infty$.
From \eqref{eq:rp1} and \eqref{eq:errp2} we have
\begin{align}\label{eq:rp2}
	D_{\mathbf{z}}\boldsymbol{\Phi}[\mathbf{z}]\widetilde{\mathbf{z}}= \mathbf{J}
	\begin{pmatrix}
		-\partial_x^2+1 & 0\\ 0 &1
	\end{pmatrix}
	\boldsymbol{\Phi}[\mathbf{z}] + f(\boldsymbol{\Phi}[\mathbf{z}]_1) \mathbf{j} + \mathbf{R}[\mathbf{z}].
\end{align}

 We assume the following.
\begin{assumption}[Fermi Golden Rule]\label{ass:FGR}
There exists a function $\mathbf{g}_{2}\in L^\infty\pare{\bR} $ which is a solution  of $\mathbf{J}\mathbf{L}_0 \mathbf{g}_{2}=\im 2\lambda  \mathbf{g}_{2}$ s.t.
\begin{align}\label{ass:FGR1}
\< \mathbf{J} \mathbf{G}_{2},\mathbf{g}_{2}\> =\gamma > 0.
\end{align}
\end{assumption}
\begin{remark}\label{rem:FGR2}
   Obviously, in \eqref{ass:FGR1} we only need $\gamma \neq 0$, since then    \eqref{ass:FGR1} follows multiplying
   $\mathbf{g}_{2}$ by a constant.
 Notice that  $ \mathbf{g}_{2} =   {  ( 1, 2 \im \lambda ) ^ \intercal } \  c  \ g$ with $ L_0g= 4\lambda ^2 g$ and $c\in \C \backslash \{  0\}$, so that $g$ is proportional to a distorted plain wave associated to $L_0$, cfr. \cite{DT1979}.  This and the definition of $\boldsymbol{\Phi}[\mathbf{z}]$  gives a cancellation and  leads  to
\begin{align*}
  \< \mathbf{J} \mathbf{G}_{2},\mathbf{g}_{2}\> =  - \overline{c}\ p\ (p-1) \<Q ^{p-2}\varphi _2^2  ,g \> .
\end{align*}
   Assumption \ref{ass:FGR} is proved   for $p=2$ by   Li  and   L\" uhrmann \cite [Lemma 3.1] {LL2023} and  for   $p $ close to 2 remains true by continuous dependence on $p$. Since this dependence is also analytic,    Assumption \ref{ass:FGR}  is true for all but finitely many $p\in  ( 5/3, 2] $.
\end{remark}

\section{Main estimates and proof of Theorem \ref{thm:main1}.} \label{sec:modul}



Consider a solution ${\bm u}\in C^0 \pare{ [0,+\infty ) ; \boldsymbol{\mathcal{H}}^{1}  _{\even} } $  gravitating around $Q\mathbf{i}$  like in \eqref{eq:cond_sta}.
From the spectral decomposition of ${\bm u}-Q\mathbf{i}$, we have ${\bm u}=\boldsymbol{\Phi}[{\bm z}]+\boldsymbol{\eta} $ with   $\boldsymbol{\eta}\in C^0([0,+\infty ) , \boldsymbol{\mathcal{H}}^{1} _{c} )$ and  the NLKG \eqref{NLKG} rewrites  \begin{align}\nonumber
\dot{\boldsymbol{\eta}}+D_{\mathbf{z}}\boldsymbol{\Phi}[\mathbf{z}](\dot{\mathbf{z}}-\widetilde{\mathbf{z}})+\xcancel{ D_{\mathbf{z}}\boldsymbol{\Phi}[\mathbf{z}] \widetilde{\mathbf{z}}} =\mathbf{J}
 \mathbf{L}_0\boldsymbol{\eta} +\xcancel{ \mathbf{J}
 \mathbf{L}_0\boldsymbol{\Phi}[\mathbf{z}]} + \xcancel{ f(\boldsymbol{\Phi}[\mathbf{z}]_1) \mathbf{j}} + \xcancel{\mathbf{R}[\mathbf{z}] }
  - \mathbf{R}[\mathbf{z}] \\   + \underbrace{\left(  f'\pare{ \boldsymbol{\Phi}[{\bm z}]_1 } - f'(Q) \right) \eta _1 \mathbf{j}  + \left( f\left( \boldsymbol{\Phi}[{\bm z}]_1 +\eta _1 \right)-f(\boldsymbol{\Phi}[{\bm z}]_1)- f'(\boldsymbol{\Phi}[{\bm z}]_1)\eta _1  \right) \mathbf{j}} _{\defeq \bm{F}_Q\pare{ \bmz, \bm{\eta}}}  ,\label{eq:modeq}
\end{align}
for $\mathbf{R}[\mathbf{z}]$ defined in \eqref{eq:errp1} and \eqref{eq:errp2} and where the canceling follows from \eqref{eq:errp1}.
Equivalently, we have also
\begin{equation}\label{nuovaequi}
	\dot{\boldsymbol{\eta}}+D_{{\bm z}}\boldsymbol{\Phi}[{\bm z}](\dot{{\bm z}}-\tilde{{\bm z}})
	=\mathbf{J}
	\begin{pmatrix}
		-\partial_{x}^2+ 1 &0\\0&1
	\end{pmatrix}\boldsymbol{\eta}
	- {\bm R}[{\bm z}]
	 + \left( f\left( \boldsymbol{\Phi}[{\bm z}]_1 +\eta _1 \right)-f(\boldsymbol{\Phi}[{\bm z}]_1) \right) \mathbf{j}.
\end{equation}
 In the sequel we will consider constants  $A, B,\varepsilon >0$ satisfying
 \begin{align}\label{eq:relABg}
\log(\delta ^{-1})\gg\log\pare{ \epsilon ^{-1} } \gg  A\gg    B^2\gg B \gg  \exp \left( \varepsilon ^{-1} \right) \gg 1.
 \end{align}
 Exploiting  the double differentiability of $\tilde{{\bm z}}_R$ in ${\bm z}=0$, we   impose  a further condition on $A$, that is
\begin{align}\label{eq:142bis}
   A  \av{\tilde \bmz _R\bra{\bmz} - \frac{D_\bmz^2 \tilde\bmz_R\bra{0}\pare{\bmz, \bmz}}{2} } =  o\pare{ \delta ^2 }  \text{  for } \av{{\bm z}} \lesssim \delta.
\end{align}
   We will denote by    $o_{\alpha}(1)$  constants depending on $\alpha >0$ such that
 \begin{align}\label{eq:smallo}
o_{\alpha}(1) \xrightarrow {\alpha  \to 0^+   }0.
 \end{align}
 We will consider the norms
\begin{align}\label{eq:normA}&
\norm{ \boldsymbol{\eta} }_{ \boldsymbol{ \Sigma }_A} \defeq\left \| \sech \pare{\frac{2}{A} x} \eta_1'\right \|_{L^2} +A^{-1}\left \|    \sech \pare{\frac{2}{A} x} \boldsymbol{\eta}\right\|_{L^2}  \\
%-----------------------------
&  \norm{ \boldsymbol{\eta} }_{ \boldsymbol{L }^2_{-\kappa} } \defeq\left \| \sech \pare{ \kappa  x}  \boldsymbol{\eta}\right \|_{L^2}.\label{eq:normk}
\end{align}
We will prove the following  continuation argument.
 \begin{proposition}\label{prop:contreform}   Under the Assumption  \ref{ass:FGR},
  for any small $\epsilon>0 $
there exists  a    $\delta _0 \defeq \delta _0(\epsilon )  \ll \epsilon $ such that if a solution of \eqref{NLKG} satisfying \eqref{eq:cond_sta},
if   in  $I=[0,T]$ we have
\begin{align}&
\|\dot {{\bm z}} - \tilde{{\bm z}}\|_{L^2(I)}^2+\| {z} _1\|_{L^2(I)}^2 +\| {z} _2\|_{L^4(I)} ^4+ \|   \boldsymbol{\eta}  \|_{L^2(I, \boldsymbol{ \Sigma }_A  \cap   \boldsymbol{L }^2_{-\kappa})}^2\le   \epsilon ^2  \label{eq:main11}
\end{align}
then  for $\delta  \in (0, \delta _0)$
     inequality   \eqref{eq:main11} holds   for   $\epsilon^2$ replaced by $ o_{\varepsilon}(1)   \epsilon ^2$    where $o_{\varepsilon}(1) \xrightarrow {\varepsilon  \to 0^+   }0 $.
\end{proposition}
% Notice that Proposition \ref{prop:contreform} implies by standard continuation arguments Theorem \ref{thm:main}.
It is elementary that     Proposition \ref{prop:contreform} follows  from the following
  \Cref{prop:modp,prop:FGR,prop:1stvirial,prop:2ndvirial}, where we always assume that our solution satisfies the stability condition \eqref{eq:cond_sta}.


\begin{proposition}\label{prop:modp}
 Under the assumptions of \Cref{prop:contreform} there exists a $ \kappa > 0 $ such that
\begin{align}
\|\dot{{\bm z}}-\tilde{{\bm z}}\|_{L^2(I)}^2=
\delta^{2(p-1)} \norm{  \boldsymbol{\eta}  }_{L^2 \pare{ I ;   \boldsymbol{L }^2_{-\kappa} } }^2      . \label{eq:lem:estdtz}
\end{align}
\end{proposition}

 \begin{proposition}[Radiation Damping] \label{prop:FGR}
We have \begin{align}\label{eq:FGRint}
 \norm{ {z} _1}_{L^2(I)}^2 +\norm{ {z} _2 }_{L^4(I)} ^4=  o(\epsilon ^2) .
 \end{align}
 \end{proposition}


\begin{proposition}[1st virial estimate]\label{prop:1stvirial}
We have
\begin{align}
 \|   \boldsymbol{\eta}  \|_{L^2(I, \boldsymbol{ \Sigma }_A  )}^2  \lesssim \delta^2 +  \|  \boldsymbol{\eta}  \|_{L^2(I,  \boldsymbol{L }^2_{-\kappa})}^2  + \| \mathbf{{z} } \|_{L^4(I)} ^4.\label{eq:1stvInt}
\end{align}

\end{proposition}

\begin{proposition}[2nd virial  estimate]\label{prop:2ndvirial}
We have
\begin{align}\label{eq:2ndv}
 \|  \boldsymbol{\eta}  \|_{L^2(I,  \boldsymbol{L }^2_{-\kappa})}^2  \lesssim  {\delta}^2 +o _\varepsilon (1)  \|   \boldsymbol{\eta}  \|_{L^2(I, \boldsymbol{ \Sigma }_A  )}^2  +\| {\bm z}  \|_{L^4(I)}^4   .
\end{align}
\end{proposition}

\begin{proof}[Proof of Theorem \ref{thm:main1}]
By continuity,  Proposition \ref{prop:contreform} implies that inequality \eqref{eq:main11} is valid with $I=\R _+$. This implies \eqref{eq:main2} (adjusting $\epsilon$).  From  \eqref{eq:cond_sta},  \eqref{eq:tildez} and  \eqref{lem:modbound1} below,   we have $\dot {{\bm z}}\in L^\infty (\R , \C ^2 )$. By ${\bm z} \in L^4 (\R )$   we get \eqref{eq:main3}.
\end{proof}



 \section{Proof of Proposition \ref{prop:modp}%: bounds for the approximate time-derivative of the discrete modes
 }
\label{sec:propdmodes1}


\begin{notation}\label{notation:chi}
We fix an even function $\chi\in C_0^\infty(\R , [0,1])$ satisfying
$1_{[-1,1]}\leq \chi \leq 1_{[-2,2]}$ and $x\chi'(x)\leq 0$ and set $\chi_A\defeq\chi(\cdot/A)$.
\end{notation}




\begin{lemma}\label{lem:estF}
For $ \delta > 0 $, $ \bmz \in D_{\mathbb{C}^2}\pare{0, \delta}$ and  $ \bm{\eta} \in D_{ \cH^1_c}\pare{0, \delta} $   then, for any $ A> 0 $ and $ \kappa >0$,
\begin{align}   \label{eq:lem:estF1}
\left \|    \sech \pare{  \kappa x}    \pare{  f'(\boldsymbol{\Phi}[{\bm z}]_1) - f'(Q) } \eta _1\right \|_{L^2} \lesssim  \delta ^{p-1} \left \|    \sech \pare{  \pare{\kappa +  (p-1)k _2} x} \eta_1\right \| _{L^2}, \\
%----------------------------------------------------------------
\norm{\chi_A  \pare{  f'(\boldsymbol{\Phi}[{\bm z}]_1) - f'(Q) } \eta _1}_{L^1} \lesssim {  \delta^{p-1}} \norm{      \sech \pare{   \frac{2}{A}x}     \eta_1}_{L^2},    \label{eq:lem:estF2}\\
%----------------------------------------------------------------
 \norm{   \sech \pare{  \kappa x}    \pare{ f\pare{ \boldsymbol{\Phi}[{\bm z}]_1 +\eta _1 }-f(\boldsymbol{\Phi}[{\bm z}]_1)- f'(\boldsymbol{\Phi}[{\bm z}]_1)\eta _1  }}_{L^2}
  \lesssim  \delta ^{p-1} \norm{        \sech \pare{  \kappa x}    \eta_1}_{L^2}, \label{eq:lem:estF3} \\
 %----------------------------------------------------------------
\norm{ \chi_A \pare{ f\pare{ \boldsymbol{\Phi}[{\bm z}]_1 +\eta _1 }-f(\boldsymbol{\Phi}[{\bm z}]_1)- f'(\boldsymbol{\Phi}[{\bm z}]_1)\eta _1  } }_{L^1} \lesssim A^{1-\tfrac{p}{2}} { \delta ^{p-1}}\norm{     \sech \pare{   \frac{2}{A}x}     \eta_1 }_{L^2}.    \label{eq:lem:estF4}
\end{align}
\end{lemma}
\begin{proof}
\begin{step}[Proof of \eqref{eq:lem:estF1} and \eqref{eq:lem:estF2}]
First of all in the points where $|\widetilde{\boldsymbol{\Phi}}[{\bm z}]_1|\ll Q$,
\begin{align*}
  \left | f'(\boldsymbol{\Phi}[{\bm z}]_1) - f'(Q) \right | \lesssim  \left |    f''(Q)\widetilde{ \boldsymbol{\Phi}}[{\bm z}]_1 \right | \sim Q ^{p-2}\left |     \widetilde{ \boldsymbol{\Phi}}[{\bm z}]_1 \right | ^{1-(2-p)+(2-p)}\le \left |     \widetilde{ \boldsymbol{\Phi}}[{\bm z}]_1 \right | ^{p-1}.
\end{align*}
If instead $|\widetilde{\boldsymbol{\Phi}}[{\bm z}]_1|\gtrsim Q$ then
\begin{align*}
  \left | f'(\boldsymbol{\Phi}[{\bm z}]_1) - f'(Q) \right | \lesssim  \left |    f'  \pare{\widetilde{ \boldsymbol{\Phi}}[{\bm z}]_1  }\right |
  \sim   \left |     \widetilde{ \boldsymbol{\Phi}}[{\bm z}]_1 \right | ^{p-1}.
\end{align*}
Thus, by \eqref{eq:asympt_eigenf} we have
\begin{align}\label{eq:pointwise_bound}
  \left |  \pare{  f'(\boldsymbol{\Phi}[{\bm z}]_1) - f'(Q) } \eta _1 \right | \lesssim  \left |    {\bm z} \right | ^{p-1}  \sech \pare{ (p-1) k _2 x}  |\eta _1|  ,
\end{align}
which yields \eqref{eq:lem:estF1}. We prove now \eqref{eq:lem:estF2}. Using \eqref{eq:pointwise_bound} and   $ \sech\pare{\frac{2}{A} x} \sim 1 $ on $ \bra{-A, A} $,
\begin{align*}
\norm{\chi_A  \pare{  f'(\boldsymbol{\Phi}[{\bm z}]_1) - f'(Q) } \eta _1}_{L^1}\lesssim & \  \av{\bm z}  ^{p-1}\int _{-A}^{A}      \sech \pare{ (p-1) k _2 x}  |\eta _1\pare{x}|  \dd x
\\
\lesssim  & \ \delta^{p-1} \norm{\sech\pare{\frac{2}{A} \ x} \eta_1}_{L^2}.
\end{align*}
\end{step}

 \begin{step}[Proof of \eqref{eq:lem:estF3} and \eqref{eq:lem:estF4}]
If $|\eta _1|\gtrsim |\boldsymbol{\Phi}[{\bm z}]_1|$  we have
\begin{align}\label{eq:buuuuh1}
 \left |  f\pare{ \boldsymbol{\Phi}[{\bm z}]_1 +\eta _1 }-f(\boldsymbol{\Phi}[{\bm z}]_1)- f'(\boldsymbol{\Phi}[{\bm z}]_1)\eta _1  \right | \lesssim | \eta _1 | ^{p}.
\end{align}
In the points where $|\eta _1|\ll |\boldsymbol{\Phi}[{\bm z}]_1|$, for some $t \in (0,1)$ we have
\begin{align} \label{eq:buuuuh2}
 \left |  f\pare{ \boldsymbol{\Phi}[{\bm z}]_1 +\eta _1 }-f(\boldsymbol{\Phi}[{\bm z}]_1)- f'(\boldsymbol{\Phi}[{\bm z}]_1)\eta _1  \right | = \left |  \pare{f'\pare{ \boldsymbol{\Phi}[{\bm z}]_1 +t\eta _1 } - f'(\boldsymbol{\Phi}[{\bm z}]_1) } \eta _1  \right |
 %--------------------------------------
 \\ \lesssim  \left |   f''\pare{ \boldsymbol{\Phi}[{\bm z}]_1   }   \eta _1 ^2  \right | \sim |\boldsymbol{\Phi}[{\bm z}]_1 | ^{p-2}   \eta _1 ^2 \le  | \eta _1 | ^{p}. \nonumber
\end{align}
Inequality \eqref{eq:lem:estF3}  follows immediately.
 By  $  \sech \pare{   \frac{2}{A}x} \sim 1$ in $\mathrm{supp} \chi_A$,  \eqref{eq:cond_sta} and H\"older inequality, we get
\begin{align*}
\|\chi_A | \eta _1 | ^{p}\|_{L^1}& \ \lesssim  A^{1-\tfrac{p}{2}}  \norm{\sech\pare{\frac{2}{A} \ x} \eta_1}_{L^{ \frac{2}{p}}}   \lesssim  A^{1-\tfrac{p}{2}} \delta ^{p-1} \norm{\sech\pare{\frac{2}{A} \ x} \eta_1}_{L^2},
\end{align*}
which proves \eqref{eq:lem:estF4}.
 \end{step}
\end{proof}


\begin{lemma}
\label{lem:modbound}
With the same hypothesis of \Cref{lem:estF} and for any $\kappa\in\pare{0, k_2} $ the following bound holds true
\begin{align}\label{lem:modbound1}
\av{ \dot{{\bm z}}\pare{t}-\tilde{{\bm z}}\pare{t}}   \lesssim  \delta ^{p-1}   \norm{     \sech \pare{  \kappa x} \boldsymbol{\eta}\pare{t} } _{L^2\pare{\mathbb{R}}}.
\end{align}
\end{lemma}

\begin{proof}
 We apply $\< \mathbf{J} \bullet \ , D_{\bm z}\boldsymbol{\Phi}[{\bm z}]\Theta \>$ to equation \cref{eq:modeq,eq:modeq}. By $ \< \mathbf{J} \dot {\bm \eta} \ , D_{\bm z}\boldsymbol{\Phi}[{\bm z}]\Theta \> = 0 $, due to $ {\bm \eta}\in \boldsymbol{\mathcal{H}}^1_c $, and $  D_{\bm z}\boldsymbol{\Phi}[{\bm z}] $ independent of $ \bm z $, we get
\begin{multline*}
  \< \mathbf{J} D_{{\bm z}}\boldsymbol{\Phi}[{\bm z}](\dot{{\bm z}}-\tilde{{\bm z}}) , D_{\bm z}\boldsymbol{\Phi}[{\bm z}]\Theta \>
  =- \cancel{\<   \mathbf{J} \boldsymbol{\eta} ,  \mathbf{J}\mathbf{L}_0 D_{\bm z}\boldsymbol{\Phi}[{\bm z}]\Theta \> }-\cancel{\< \mathbf{J} {\bm R}[{\bm z}] , D_{\bm z}\boldsymbol{\Phi}[{\bm z}]\Theta \> } \\
  %---------------------------------------------------
   +\< \mathbf{J}\pare{  f'(\boldsymbol{\Phi}[{\bm z}]_1) - f'(Q) } \eta _1 \mathbf{j}  \ , D_{\bm z}\boldsymbol{\Phi}[{\bm z}]\Theta \> %\\
    +\< \mathbf{ J}\pare{ f\pare{ \boldsymbol{\Phi}[{\bm z}]_1 +\eta _1 }-f(\boldsymbol{\Phi}[{\bm z}]_1)- f'(\boldsymbol{\Phi}[{\bm z}]_1)\eta _1  } \mathbf{j}  \ , D_{\bm z}\boldsymbol{\Phi}[{\bm z}]\Theta \> ,
\end{multline*}
where the second cancelation  follows from \eqref{R:orth} and the first, by \eqref{eq:rp}, from (cf, \cref{eq:rp--1})
\begin{align*}
  \mathbf{J}\mathbf{L}_0 D_{\bm z}\boldsymbol{\Phi}[{\bm z}]\Theta = D_{\bm z}\boldsymbol{\Phi}[{\bm z}] \  \tilde{{\bm z}}_0 \bra{ \Theta}  ,
\end{align*}
 and the fact that $ \bm{\eta}\in\cH^1_c $.
Using  Lemma \ref{lem:estF}, we obtain
\begin{align*}
\av{\dot{{\bm z}}-\tilde{{\bm z}}}  \lesssim \norm{   \sech \pare{ \kappa x}    \pare{  f'(\boldsymbol{\Phi}[{\bm z}]_1) - f'(Q) } \eta _1}_{L^2} \\
%------------------------------------------------------------
+  \norm{   \sech \pare{  \kappa x}    \pare{ f\pare{ \boldsymbol{\Phi}[{\bm z}]_1 +\eta _1 }-f(\boldsymbol{\Phi}[{\bm z}]_1)- f'(\boldsymbol{\Phi}[{\bm z}]_1)\eta _1  } }_{L^2}  \lesssim  \delta ^{p-1} \norm{       \sech \pare{   \kappa x} \eta_1 }_{L^2} .
\end{align*}
\end{proof}
The proof   of \Cref{prop:modp} follows from \Cref{lem:modbound} setting $ \kappa \defeq \xfrac{k_2}{2} $ and  integrating-in-time.















\section{Proof of  Proposition \ref{prop:FGR}: the Fermi Golden Rule} \label{sec:FGR}


Here the Fermi Golden Rule does not refer any more to the condition in \eqref{ass:FGR1}, but rather to the proof  of dissipation of $z_2$ due, ultimately,  to the nonlinear interaction of $z_2$ with the continuous modes of ${\bm u}$.  There is an old history, starting from \cite{Sigal1993,BP1995},
with many significant contributions, like \cite{SW1999,MR2006}. Here we use the argument  of  Kowalczyk and   Martel \cite{KM22}. \\



%{\color{blue} Let us give at first an heuristic idea on the procedure we adopt in the present section. Let us recall that $ \bm{\mathsf{e}}_{\pm} $ defined in \cref{eq:geneigenfJLQ} are the generalized eigenfunctions of the operator $ \mathbf{JL}_0 $, noticing that $ \pare{ \mathbf{JL}_0 }^\ast = -\mathbf{L}_0 \mathbf{J} $ we can deduce that\begin{equation*}\pare{ \mathbf{JL}_0 }^\ast \mathbf{J} \bm{\mathsf{e}}_{\pm}\pare{\xi} = \mp \ii \angles{\xi} \ \mathbf{J} \bm{\mathsf{e}}_{\pm}\pare{\xi} ,\end{equation*}i.e. $ \pare{\mathbf{J} \bm{\mathsf{e}}_{\pm}\pare{\xi} }_{\xi\neq 0} $ are generalized eigenfunctions for the operator $ \pare{ \mathbf{JL}_0 }^\ast $. As such we can define the distorted Fourier transform\begin{equation*}\interior{ \eta }_{\pm} \pare{\xi} \defeq \psc{\eta}{\mathbf{J}  \bm{\mathsf{e}}_{\mp}\pare{\xi} }.\end{equation*}By construction $ \interior{ \eta }_{\pm} \pare{\xi} $ oscillates with frequency $ \pm \angles{\xi} $, hence since $ \dot{z}_2\approx i\lambda \ z_2 $ the quantity\begin{equation*}z_2^2 \ \interior{ \eta }_{\pm} \pare{\xi}\end{equation*}oscillates, approximately, with a frequency $ 2\lambda \pm \angles{\xi} $. Namely we have\begin{equation*}\ddt\pare{z_2^2 \interior{\eta}_{\pm}\pare{\xi}} = \ii \pare{2\lambda \pm \angles{\xi}} \ z_2^2 \interior{\eta}\pare{\xi} + \text{lower order terms}.\end{equation*}It is thus clear that selecting $ \xi=\xi_\lambda \defeq\sqrt{\pare{2\lambda}^2 -1} $ then the quantity $ z_2^2 \  \interior{\eta}_{-}\pare{\xi_\lambda} $ does not present first-order oscillations. \\}




To prove Proposition \ref{prop:FGR},    for the    $\mathbf{g}_{2}$ in Assumption \ref{ass:FGR},    we consider
\begin{align}\label{eq:FGRfunctional}
\mathcal{J}_{\mathrm{FGR}}\defeq\< \mathbf{J} \boldsymbol{\eta},\chi_A \pare{  {z}^{2}_2\mathbf{g}_{2} + \overline{{z}}^{2}_2\overline{\mathbf{g}}_{2}}      \> .
\end{align}

%Computing the time derivative of $\mathcal{J}_{\mathrm{FGR}}$, we have the following estimate.
\begin{lemma}\label{lem:FGR1}
We have
\begin{equation}
	|z_2|^4 \lesssim - \dot{\mathcal{J}}_{\mathrm{FGR}}+ \frac{\dd}{\dd t} \Re  \<  {2^{-1}\lambda  ^{-2}} z_2^4\mathbf{J}\mathbf{G}_2 , \overline{\mathbf{g}}_{2} \>+\delta^2|z_1|^2 +
	A^{-1/2} \pare{ \|\boldsymbol{\eta}\|_{\boldsymbol{ \Sigma }_A}^2 +   \| \sech  ( k_2 x) \boldsymbol{\eta}\|_{L^2}^2 }
 \label{eq:lem:FGR11}
\end{equation}
\end{lemma}

\begin{proof}
Differentiating $\mathcal{J}_{\mathrm{FGR}}$  and using \cref{eq:modeq}, we have
\begin{align*}%\label{KMFGR1}
\dot{\mathcal{J}}_{\mathrm{FGR}}=&
\< \mathbf{J} \dot{\boldsymbol{\eta}}, \chi_A \pare{  {z}^{2}_2\mathbf{g}_{2} + \overline{{z}}^{2}_2\overline{\mathbf{g}}_{2}}\>
+\< \mathbf{J}  \boldsymbol{\eta},\chi_A     D_{\bm z} \pare{  {z}^{2}_2\mathbf{g}_{2} + \overline{{z}}^{2}_2\overline{\mathbf{g}}_{2}} \widetilde{{\bm z} }_0        \> +\< \mathbf{J}  \boldsymbol{\eta},\chi_A     D_{\bm z} \pare{  {z}^{2}_2\mathbf{g}_{2} + \overline{{z}}^{2}_2\overline{\mathbf{g}}_{2}} \widetilde{{\bm z} } _R       \>
\\&+\< \mathbf{J}   \boldsymbol{\eta},\chi_A D_{\bm z} \pare{  {z}^{2}_2\mathbf{g}_{2} + \overline{{z}}^{2}_2\overline{\mathbf{g}}_{2}} \pare{\dot{{\bm z}}-\tilde{{\bm z}}}  \> =:A_1+A_2+A_3+A_4.\nonumber
\end{align*}
By  \cref{lem:modbound},  \eqref{eq:relABg} and the following elementary inequality,
\begin{equation}
\label {eq:lem:rhoequiv}
 \| \sech \pare{  \kappa x}  \eta_1\|_{L^2}    \le     \norm{ \sech   \pare{   \frac{2}{A}x}  \eta_1}_{L^2} \le A \norm{\bm{\eta}}_{{\bm \Sigma}_A},
\end{equation}
  we have
\begin{align*}
|A_4|&\lesssim
 \|\boldsymbol{\eta}\chi_A\|_{L^1}\delta ^2 \av{\dot{{\bm z}}-\tilde{{\bm z}} } _{\C^2}
\lesssim
 \delta  ^{p+1}  A ^{\frac{1}{2}}  \norm{ \sech \pare{ \frac{2}{A}x}\boldsymbol{\eta} }_{L^2}  \norm{ \sech \pare{  k_2 x} {\eta}_1 }_{L^2}
\\&\lesssim
  \delta ^{p+1}   A ^{\frac{5}{2}}    \|    \boldsymbol{\eta}\|_{\boldsymbol{ \Sigma }_A}^2 \lesssim A^{-1/2}\|    \boldsymbol{\eta}\|_{\boldsymbol{ \Sigma }_A}^2.
\end{align*}
We remark that
\begin{align}\label{eq:fgr11}
  A_2= \< \mathbf{J}  \boldsymbol{\eta},\chi_A     \pare{  2\im \lambda  {z}^{2}_2\mathbf{g}_{2} -  2\im \lambda \overline{{z}}^{2}_2\overline{\mathbf{g}}_{2} } \> .
\end{align}
By   \cref{eq:modeq} and with the cancelation in the 1st line   due to \eqref{eq:fgr11}  and $\mathbf{J} \mathbf{L}_1\mathbf{g}_{2}  =2\im \lambda \mathbf{g}_{2} $,
\begin{align*}&
A_1+A_2= -\< \mathbf{J}  D_{{\bm z}}\boldsymbol{\Phi}[{\bm z}](\dot{{\bm z}}-\tilde{{\bm z}}),\chi_A\pare{  {z}^{2}_2\mathbf{g}_{2} + \overline{{z}}^{2}_2\overline{\mathbf{g}}_{2}} \>%\label{KMFGR2}
-\xcancel{
\< \mathbf{J}
	\boldsymbol{\eta},\chi_A  \mathbf{J} \mathbf{L}_0 \pare{  {z}^{2}_2\mathbf{g}_{2} + \overline{{z}}^{2}_2\overline{\mathbf{g}}_{2}}\> }  + \xcancel{A_2}\\
	& -\<
	\boldsymbol{\eta} , \comm{ \mathbf{L}_0}{\chi_A}     \pare{   {z}^{2}_2\mathbf{g}_{2} + \overline{{z}}^{2}_2\overline{\mathbf{g}}_{2} } \>  +\< \mathbf{J}\bm{F}_Q\pare{ \bmz, \bm{\eta}} , \chi_A \pare{   {z}^{2}_2\mathbf{g}_{2} + \overline{{z}}^{2}_2\overline{\mathbf{g}}_{2} } \>
\\&
-\<  \mathbf{J} {\bm R}[{\bm z}],\chi_A \pare{   {z}^{2}_2\mathbf{g}_{2} + \overline{{z}}^{2}_2\overline{\mathbf{g}}_{2} }  \>
=:A_{11}+A_{12}+A_{13}+A_{14} .
\end{align*}
By  \Cref{lem:modbound}, \eqref{eq:relABg} and \eqref{eq:lem:rhoequiv}, we have
\begin{align*}
|A_{11}|&\lesssim
|\dot{{\bm z}}-\tilde{{\bm z}} |  |z_2 |^2 \lesssim
\delta  ^{p-1} \pare{|{\bm z} |^4+ \norm{\sech \pare{ \xfrac{k _2 x}{2}}\boldsymbol{\eta} }_{L^2}^2  } \le A^{-1/2}\pare{ |{\bm z} |^4+\|\boldsymbol{\eta}\|_{\boldsymbol{ \Sigma }_A}^2}.
\end{align*}
By $ \bra{\mathbf{L}_0,\chi_A}  =\begin{pmatrix}
-\chi_A''-2\chi_A'\partial_x & 0 \\ 0 & 0
\end{pmatrix}$,   we have
\begin{align*}
|A_{12}|
&\lesssim  |z_2 |^2  \pare{\norm{\chi_A'' \eta_1}_{L^1}+\norm{\chi_A' \eta_1'}_{L^1}}\\
&\lesssim |z_2 |^2 \pare{ A^{-3/2}\norm{\sech \pare{ \frac{2}{A} x} \eta_1}_{L^2}+A^{-1/2}\norm{\sech \pare{ \frac{2}{A} x} \eta_1'}_{L^2} }\\
&
\lesssim A^{-1/2}\pare{ |{\bm z} |^4 +\|\sech \pare{ \frac{2}{A} x}\eta_1'\|_{L^2}^2+A^{-2}\|\sech \pare{ \frac{2}{A} x} \eta_1\|_{L^2}^2} .
\end{align*}
By   \Cref{lem:estF}, \eqref{eq:lem:estF2} and  \eqref{eq:lem:estF4},   we have
\begin{align*}
|A_{13}|&\lesssim   |z_2 |^2   \norm{  \chi_A \bm{F}_Q\pare{\bm{\eta}, \bmz} }_{L^1}  \ \lesssim {  A \delta ^{p-1}} \ \pare{ \av{\bmz}^4 +  \norm{\bm{\eta}}_{\bm{\Sigma}_A}^2 }.
\end{align*}
The key term is the following,
\begin{align*}
A_{14}= & \  -2^{-1}\<  \mathbf{J}  D^2_{\bm z} {\bm R}[0] {\bm z}^2,\chi_A\pare{  {z}^{2}_2\mathbf{g}_{2} + \overline{{z}}^{2}_2\overline{\mathbf{g}}_{2}} \> \\
%-----------------------------------------------------
& +\<  \mathbf{J}  \pare{ \frac{1}{2}  D^2_{\bm z} {\bm R}[0]    \pare{\bmz, \bmz} - {\bm R}[{\bm z}]   },\chi_A\pare{  {z}^{2}_2\mathbf{g}_{2} + \overline{{z}}^{2}_2\overline{\mathbf{g}}_{2}} \>
=:    A_{141} + A_{142}
\end{align*}
We bound the term $ A_{142} $.
 We have
\begin{equation*}
\begin{aligned}
  D^2_{\bm z} {\bm R}[0]   \pare{\bmz, \bmz} = & \  \left. D^2_{\bm z} \left [  D_{\bm z}\bm{\Phi}[{\bm z}] \ \tilde{{\bm z}}_R \bra{{\bm z}}
 -\left(  f(\boldsymbol{\Phi}[{\bm z}]_1)   - f(Q)-f' (Q)  \widetilde{\boldsymbol{\Phi}}[{\bm z}]_1     \right)  \mathbf{j}   \right ] \right | _{{\bm z}=0} \pare{\bmz, \bmz}
 \\
 %----------------------------------------------------------------
 = & \   D_{\bm z}\bm{\Phi}[{\bm z}] D^2_{\bm z}\tilde{{\bm z}}_R \bra{0} \pare{\bmz, \bmz} - f''\pare{Q}\mathbf{j} \  \tilde{\bmPhi}\bra{\bmz}^2 ,
\end{aligned}
\end{equation*}
so that
\begin{align}
\label{eq:Taylor_second_order_Remainder}
\frac{1}{2}  D^2_{\bm z} {\bm R}[0]    \pare{\bmz, \bmz} - {\bm R}[{\bm z}]
&=
D_{\bmz}\bmPhi\bra{\bmz} \pare{ \frac{1}{2} D^2_\bmz \tilde{\mathbf{z}}_R \bra{0}\pare{\bmz, \bmz}  - \tilde{\bmz}_R }
\\&
%---------------------------------------------
-\left(     f(Q) + f' (Q)  \widetilde{\boldsymbol{\Phi}}[{\bm z}]_1   + \frac{f''\pare{Q}}{2} \tilde{\bmPhi}\bra{z}^2 - f(\boldsymbol{\Phi}[{\bm z}]_1)  \right)  \mathbf{j} . \nonumber
\end{align}
In $\supp \chi (\cdot /A) \subseteq [-2A,2A]$,  we have $\left |   \boldsymbol{\Phi} [{\bm z}]_1\right |\ll Q$,  so that by \eqref{eq:Taylor_second_order_Remainder}, $ \av{D_{\bmz}\bmPhi\bra{\bmz}\pare{x} \cdot w}\lesssim e^{-k_2\av{x}}\av{w} $  and  \eqref{eq:142bis},   we have
\begin{equation*}
 \begin{aligned}
  |A_{142}|   \lesssim & \  |{\bm z}| ^2  \int _{-2A}^{2A} \set{ \av{ \frac{1}{2} D^2_\bmz \tilde{\mathbf{z}}_R \bra{0}\pare{\bmz, \bmz}  - \tilde{\bmz}_R }
  e^{-k_2|x|}  + \left|     f(Q) + f' (Q)  \widetilde{\boldsymbol{\Phi}}[{\bm z}]_1   + \frac{f''\pare{Q}}{2} \tilde{\bmPhi}\bra{z}^2 - f(\boldsymbol{\Phi}[{\bm z}]_1)  \right|} \dd x
   \\
%--------------------------------------------------------
    \lesssim  & \    o \pare{ \av{{\bm z}} ^4 }  + |{\bm z}| ^5 \norm{  f'''(Q) e^{-3k_2 |x|} }_{L^1(\R )} ,
 \end{aligned}
 \end{equation*}
  so that, by   $\av{ f'''(Q\pare{x}) e^{-3k_2 |x|}}\sim e^{-k_2\av{x}} $, we obtain
 \begin{equation}
 \label{eq:142}
  |A_{142}|   \lesssim   o \pare{ \av{{\bm z}} ^4 }.
 \end{equation}
 Next we  write
\begin{align*}
  A_{141} &= -2^{-1}\<  \mathbf{J}\pare{ z_2^2\mathbf{G}_2  + \overline{z}_2^2\overline{\mathbf{G}}_2  } ,   {z}^{2}_2\mathbf{g}_{2} + \overline{{z}}^{2}_2\overline{\mathbf{g}}_{2} \> \\& -2^{-1}\<  \mathbf{J}\pare{ z_2^2\mathbf{G}_2  + \overline{z}_2^2\overline{\mathbf{G}}_2   } ,(1-\chi_A)\pare{  {z}^{2}_2\mathbf{g}_{2} + \overline{{z}}^{2}_2\overline{\mathbf{g}}_{2}} \> +O\pare{  z_1 ^4   } +  \delta _2    O\pare{  z_2 ^4  }
\end{align*}
for a   $\delta _2 >0$ arbitrarily small but fixed.  The first term in the second line, can be absorbed in the third.  So,
\begin{align*}
  A_{142} &= - |z_2| ^4 \<  \mathbf{J}\mathbf{G}_2 , \mathbf{g}_{2} \> - 2\Re \<  z _2 ^4\mathbf{J}\mathbf{G}_2 , \overline{\mathbf{g}}_{2} \> +O\pare{  z_1 ^4   } +  \delta _2    O\pare{  z_2 ^4  }.
\end{align*}
By elementary computation,    for $\tilde{{\bm z}} =:((\tilde{{\bm z}})_1,(\tilde{{\bm z}})_2) ^ \intercal $ and using \eqref{eq:rp--1}  and \eqref{eq:tildez},    we have
\begin{align*}
   \frac{d}{\dd t}z_2 ^4&= 4\im \lambda (\tilde{{\bm z}})_2 z_2 ^3 +4\im \lambda \pare{ \dot z_2- (\tilde{{\bm z}})_2 } z_2 ^3\\&  =  -4\lambda ^2z_2 ^4+4\im \lambda (\tilde{{\bm z}}_{R})_2 z_2 ^3 +4\im \lambda \pare{ \dot z_2- (\tilde{{\bm z}})_2 } z_2 ^3.
\end{align*}
Hence we conclude the following, which with \eqref{lem:modbound1} yields the statement,
\begin{align*}
  A_{142} &= - |z_2| ^4 \<  \mathbf{J}\mathbf{G}_2 , \mathbf{g}_{2} \>  + \frac{d}{\dd t}\Re \<  2^{-1}\lambda ^{-2} z _2 ^4\mathbf{J}\mathbf{G}_2 , \overline{\mathbf{g}}_{2} \>
   +O\pare{  z_1 ^4   } +  \delta _2    O\pare{  z_2 ^4  }  + O\pare{  |\dot {{\bm z}} -\tilde{{\bm z}} |^2} .
\end{align*}
\end{proof}


 \begin{lemma}\label{lem:esz1L4} We have
 \begin{equation}\label{eq:esz1L41}
 	 |z_1| ^2  \lesssim   \frac{d}{\dd t} \Re z_1 ^2  +    |z_2| ^4 +   \delta^{2(p-1)}     \| \sech  ( k_2 x) \boldsymbol{\eta}\|_{L^2}^2   .
 \end{equation}
 \end{lemma}
 \begin{proof} Using \eqref{eq:rp--1} and \eqref{eq:tildez}, we have
    \begin{align*}
       \frac{d}{\dd t}z_1^2 = 2\nu _0  |z_1| ^2 + 2z_1 (\dot z_1 - \nu _0 \overline{z}_1) =       2\nu _0  |z_1| ^2 +      2z_1 (\dot z_1 - ( \tilde{{\bm z}} )_1) +  2z_1 ( \tilde{{\bm z}}_R )_1.
    \end{align*}
Taking the real part of the above equation and using   Young inequality, for any $\theta>0$   we get
\begin{align*}
2\nu _0 |z_1| ^2\leq &\frac{d}{\dd t} \Re z_1 ^2+ 2\theta |z_1|^2 + \frac{1}{\theta} | \dot{{\bm z}}-\tilde{{\bm z}}|^2+ \frac{1}{\theta}| \tilde{{\bm z}}_R|^2  \\
\stackrel{\eqref{item:rem_better_1}}{\leq} & \frac{d}{\dd t} \Re z_1 ^2+ 2\theta |z_1|^2 + \frac{1}{\theta} | \dot{{\bm z}}-\tilde{{\bm z}}|^2+ \frac{2 C_1^2}{\theta}  |{\bm z_1}|^4+ \frac{2 C_1^2}{\theta}  |{\bm z_2}|^4\\
{\leq} & \frac{d}{\dd t} \Re z_1 ^2+ (2\theta+\frac{2 C_1^2}{\theta}  \delta^2) |z_1|^2 + \frac{1}{\theta} | \dot{{\bm z}}-\tilde{{\bm z}}|^2+ \frac{2 C_1^2}{\theta}  |{\bm z_2}|^4
\end{align*}
  Choosing $ \theta\defeq \frac{\nu_0}{4}$ and using $ \delta\ll 1$, we get the following, which proves \eqref{eq:esz1L41},
    \begin{equation*}
 |z_1| ^2\lesssim \frac{1}{\nu_0^2} | \dot{{\bm z}}-\tilde{{\bm z}}|^2+\frac{1}{\nu_0} \frac{d}{\dd t} \Re z_1 ^2+\frac{1}{\nu_0^2}|{\bm z_2}|^4\lesssim \frac{\delta ^{2(p-1)}}{\nu_0^2}    \norm{     \sech \pare{  \kappa x} \boldsymbol{\eta}\pare{t} } _{L^2\pare{\mathbb{R}}}^2+\frac{1}{\nu_0} \frac{d}{\dd t} \Re z_1 ^2+\frac{1}{\nu_0^2}|{\bm z_2}|^4.
    \end{equation*}

 \end{proof}


\begin{proof}[Proof of Proposition \ref{prop:FGR}]
	Integrating \eqref{eq:lem:FGR11}  for $t \in I$ and by \eqref{eq:main11},
\begin{align}\nonumber
  \| z_2  \|  ^{4}_{L^4(0,t)} &\lesssim  \left |  \left .\pare{ {\mathcal{J}}_{\mathrm{FGR}}-  \Re  \<  {2^{-1}\lambda  ^{-2}} z _2 ^4\mathbf{J}\mathbf{G}_2 , \overline{\mathbf{g}}_{2} \> }  \right ] _{0} ^{t}\right |  +\delta ^2 \| z_1  \|  ^{2}_{L^2(0,t)} +
A^{-1/2}    \|   \boldsymbol{\eta}  \|_{L^2((0,t), \boldsymbol{ \Sigma }_A  \cap   \boldsymbol{L }^2_{-\kappa})}^2\\& = o _\epsilon  (1)      \epsilon ^2  .\label{eq:esz1L421}
\end{align}
Integrating \eqref{eq:esz1L41}   for $t \in I$ and by \eqref{eq:main11} we obtain the following, which completes the proof,
 \begin{align*}
      \| z_1  \|  ^{2}_{L^2(0,t)}&\lesssim  \left |  \left . \Re z_1 ^2 \right ] _{0} ^{t}\right |  + \delta _2 ^{-1}\| z_2  \|  ^{4}_{L^4(0,t)}  +
A^{-1/2}    \| \sech  ( k_2 x) \boldsymbol{\eta}\|_{L^2 ((0,t),L^2)}^2= o _\epsilon  (1)      \epsilon ^2.
\end{align*}


\end{proof}


\section{Proof of Proposition \ref{prop:1stvirial}: the first virial estimate.}
\label{sec:1virial}




\begin{notation}
For the $\chi$ in \Cref{notation:chi} and for $C>0$, we set
\begin{align}\label{def:zetaphi}
\zeta_C(x)\defeq\exp\left(-\frac{|x|}{C}(1-\chi(x))\right), && \varphi_C(x)\defeq\int_0^x \zeta_C^2(y)\,\dd y\  &&  S_C\defeq\frac{1}{2}\varphi_C'+\varphi_C\partial_x.
\end{align}
\end{notation}
We define functionals
\begin{align*}
&\mathcal{I}_{\mathrm{1st},1}\defeq\frac{1}{2}\< \mathbf{J} \boldsymbol{\eta},S_A\boldsymbol{\eta}\>, \\
& \mathcal{I}_{\mathrm{1st},2}\defeq\frac{1}{2}\< \mathbf{J}  \boldsymbol{\eta},\sigma_3\zeta_A^4 \boldsymbol{\eta}\> && \text{where} && \sigma_3\defeq \begin{pmatrix}
	1 & 0 \\ 0 & -1
\end{pmatrix}.
\end{align*}


\begin{lemma}\label{lem:1stV1}
There exist $\delta_0>0$, $A_0>0$ and $ 0 < \kappa_0 \ll k_2 $ s.t.  if $\delta\in\pare{0,  \delta_0 }$, for any $ \kappa\in\pare{0, \kappa_0} $ and $A>A_0$, we have
\begin{equation}
\norm{    \sech  \left(   \frac{2}{A}x\right)\eta_1'}_{L^2}^2 + A^{-2}\norm{    \sech  \left(   \frac{2}{A}x\right) \eta_1}_{L^2}^2   \lesssim \dot{\mathcal{I}}_{\mathrm{1st},1}+ \|   \sech  \left(  \kappa x\right) \boldsymbol{\eta}\|_{L^2}^2+    |{\bm z} |^4. \label{eq:lem:1stV1}
\end{equation}
\end{lemma}

\begin{proof}
Using   \eqref{nuovaequi} and the fact that $ S_A$ is skew-adjoint ,  we have
\begin{equation}
\label{eq:1V1}
\begin{aligned}
\dot{\mathcal{I}}_{\mathrm{1st},1}&=\< \mathbf{J} \dot {\boldsymbol{\eta}},S_A\boldsymbol{\eta}\> =    \< \mathbf{J}S_A  D\boldsymbol{\Phi}[{\bm z}](\dot{{\bm z}}-\tilde{{\bm z}}),\boldsymbol{\eta}\> -\<\partial_x^2 \eta_1  ,S_A\eta_1 \>
\\
%----------------------------------------
& \ + \<\left( f\left( \boldsymbol{\Phi}[{\bm z}]_1 +\eta _1 \right)-f(\boldsymbol{\Phi}[{\bm z}]_1) \right),S_A\eta_1\>  -\< \mathbf{J}  {\bm R}[{\bm z}],S_A\boldsymbol{\eta}\>
 \eqdef   \
 B_1+B_2+B_3+B_4 .
\end{aligned}
\end{equation}
 We have
\begin{equation*}
B_2 = \frac{1}{2} \left <  \comm{S_A}{\partial_x^2}\eta_1 , \eta_1  \right >  ,
\end{equation*}
and, by an elementary   computation  which shows that $ \comm{S_A}{\partial_x^2} = -2 \partial_x \pare{\varphi_A' \partial_x} - \frac{\varphi_A'''}{2} $,  we get
\begin{equation}\label{B2}
B_2=-\<\partial_x^2\eta_1,S_A\eta_1\> =\|\zeta_A\eta_1'\|_{L^2}^2- \frac14\<\varphi'''_A(x) \eta_1,\eta_1\> .
%+B_{21}+B_{22} \defeq\\&=-\|(\zeta_A\eta_1)'\|_{L^2}^2+\frac{p}{2}\int \varphi_A  \left( Q^{p-1} \right) ' \eta_1^2\,dx-\frac{1}{2}\int A^{-1}\left(\chi'' |x|+2\chi' \frac{x}{|x|}\right) \zeta_A^2\eta_1 ^{2} \,dx ,
\end{equation}
We have
\begin{equation}\label{bound_B2}
\big| \<\varphi'''_A(x) \eta_1,\eta_1\> \big|\lesssim \frac{1}{A^2}\| \sech\big(\frac{2}{A} x\big) \eta_1\|_{L^2}^2 .
\end{equation}
%where, $   | x  \left(  Q ^{p-1}\right)'|\lesssim | x e ^{-(p-1)|x|}|$    implies
%\begin{align*}
%|B_{21}|&\lesssim \|   \sech  \left(  \kappa x\right) \eta_1\|_{L^2}^2.
%\end{align*}
%By \Cref{notation:chi}
%\begin{align*}
%|B_{22}|&\lesssim A^{-1}\|   \sech  \left(  \kappa x\right)\eta_1\|_{L^2}^2.
%\end{align*}
Using  Lemma \ref{lem:modbound} to bound $| \dot {{\bm z}} -\tilde{{\bm z}}|$, we have, for $\kappa<k_2$,
\begin{equation}\label{B1}
|B_{1}| \le  | \dot {{\bm z}} -\tilde{{\bm z}}|  \| \sech(\kappa x)  \boldsymbol{\eta}\| _{L^2} \lesssim \delta ^{p-1}\| \sech(\kappa x)  \boldsymbol{\eta}\| _{L^2}^2.
\end{equation}
 We   decompose
 \begin{equation}\label{B4}
 	B_4=   \< \mathbf{J}S_A D_{\bm z}\boldsymbol{\Phi}[{\bm z}]\tilde{{\bm z}}_R ,\boldsymbol{\eta}\>+ \<      f(\boldsymbol{\Phi}[{\bm z}]_1)   - f(Q)-f' (Q)  \widetilde{\boldsymbol{\Phi}}[{\bm z}]_1      ,S_A {\eta}_1\> =: B_{41}+ B_{42}.
 \end{equation}
By \Cref{lem:rpcorr}       we have
\begin{align}
 | B_{41}| = |\< \mathbf{J}S_A D_{\bm z}\boldsymbol{\Phi}[{\bm z}]\tilde{{\bm z}}_R ,\boldsymbol{\eta}\> |
  \lesssim  & |\tilde{{\bm z}}_R| \  \| \sech(\kappa x)\boldsymbol{\eta} \| _{L^2 } \notag \\
   \lesssim    |{\bm z}|^2 \|\sech(\kappa x) \boldsymbol{\eta} \| _{L^2}
    \le &  |{\bm z}|^4+ \|\sech(\kappa x) \boldsymbol{\eta} \| _{L^2}^2.\label{bound_B41}
 \end{align}
By \eqref{eq:Ijest2}  and for $\delta_2>0$ small fixed,
\begin{align}
 | B_{42}| \lesssim & |{\bm z}|^2 \|  e^{-|x| }\varphi _A \cosh \pare{ \frac{2}{A}x} \| _{L^2} \| \sech \pare{ \frac{2}{A}x} \eta _1 '\| _{L^2}  + |{\bm z}|^2\|\sech(\kappa x)  {\eta}_1 \| _{L^2}
 \nonumber \\ \le &   \frac{2}{\delta_2} |{\bm z}|^4+ \delta_2\|\sech(\kappa x)  {\eta} _1\| _{L^2}^2 +\delta_2 \| \sech \pare{ \frac{2}{A}x} \eta _1 '\| _{L^2}^2     .\label{bound_B42}
 \end{align}
We decompose
 \begin{equation}\label{B3}
   B_3=   \< f\left( \boldsymbol{\Phi}[{\bm z}]_1 +\eta _1 \right)-f(\boldsymbol{\Phi}[{\bm z}]_1)- f\left(  \eta _1 \right) ,S_A {\eta}_1\> + \<  f\left(  \eta _1 \right) ,S_A {\eta}_1\>=:B_{31}+B_{32}.
 \end{equation}
For the pure non-linear term $B_{32}$, we have
\begin{align}
B_{32}=& \frac{1}{p+1}\int_{\R}  \partial_x\big(|\eta_1|^{p+1}\big) \varphi_A(x)\, {\rm d}x\, + \frac12 \int_{\R} \zeta_A^2(x) |\eta_1|^{p+1}\,{\rm d} x\notag \\
=&\frac{p-1}{2(p+1)} \int_{\R} \zeta_A^2(x) |\eta_1|^{p+1}\,{\rm d} x\lesssim \delta^{p-1}\| \sech\big(\frac{2}{A}x\big) \eta_1\|_{L^2}^2. \label{bound_B32}
\end{align}
Next, since for the concave function $t\to |t| ^{p-1}$ for $1<p\le 2$ we have
$$
\big||y+h|^{p-1}-|y|^{p-1}\big|\leq |h|^{p-1} \quad \text{for any} \quad y,h\in \C,
$$
we conclude
\begin{align}
|f\left( \boldsymbol{\Phi}[{\bm z}]_1 +\eta _1 \right)-f(\boldsymbol{\Phi}[{\bm z}]_1)- f\left(  \eta _1 \right)|\leq &\int_0^1 \big|f'(\boldsymbol{\Phi}[{\bm z}]_1+ \tau \eta _1)- f'(\tau \eta_1)\big|{\rm d} \tau\, \cdot  | \eta_1|\notag  \\
\leq & p \av{\boldsymbol{\Phi}[{\bm z}]_1}^{p-1} |\eta_1|\lesssim \sech( k_2(p-1) x ) | \eta_1|.\label{bonina}
\end{align}
Then, for $ \tfrac{1}{A}<\kappa< k_2 (p-1)/2$, we have
\begin{align}
  |B_{31}| \lesssim \int_{\R} \sech^2(\kappa x ) | \eta_1| | \eta_1'| +\int_{\R} \sech^2( \kappa x ) | \eta_1|^2\leq (1+\tfrac{1}{\delta_2})\|\sech(\kappa x ) \eta_1\|_{L^2}^2+ \delta_2\| \sech\big(\frac{x}{A}\big) \eta_1'\|_{L^2}^2. \label{B31}
\end{align}
Collecting all the terms in \cref{eq:1V1,B2,B4,B3}  and their bounds in \cref{bound_B2,B1,bound_B41,bound_B42,bound_B32,B31}  and, fixing $\delta_2>0$ small, we get
\begin{align*}
&\|\sech\left(\frac{x}{A}\right) \eta_1'\|_{L^2}^2\lesssim   -\dot{\mathcal{I}}_{\mathrm{1st},1}
 +\|\sech  \left(  \kappa x\right)\boldsymbol{\eta}\|_{L^2}^2+  |{\bm z} |^4+ \delta^{p-1}\|\sech \left(\frac{x}{A}\right) \eta_1\|_{L^2}^2 .
\end{align*}
Finally, we use the following, which is analogous to  (19) of \cite{KM22} and is proved in \cite{CMS2023},
\begin{equation}\label{eq:KM19}
\| \sech  \left(   \frac{2}{A}x\right) \eta_1\|_{L^2}^2\lesssim A^2 \| \sech  \left(   \frac{2}{A}x\right) \eta_1'\|_{L^2}^2 + A\|\sech  \left(  \kappa x\right)\eta_1\|_{L^2}^2.
\end{equation}
This, using that $A^2\delta^{p-1}\ll 1$, yields \eqref{eq:lem:1stV1}.
\end{proof}

\begin{lemma}\label{lem:1stV2}
With the same hypothesis of \Cref{lem:1stV1}, we have
\begin{align}
\norm{\sech  \left(   \frac{2}{A}x\right)\eta_2}_{L^2}^2  \lesssim  \dot{\mathcal{I}}_{\mathrm{1st},2}+\norm{\sech  \left(   \frac{2}{A}x\right) \eta_1' }_{L^2}^2+ \norm{ \sech  \left(   \frac{2}{A}x\right)\eta_1}_{L^2}^2+ \norm{\boldsymbol{\eta} }_{L^2_{-\kappa}}^2+ |{\bm z} |^4.\label{eq:lem:1stV2}
\end{align}

\end{lemma}
\begin{proof}
We have, from \cref{nuovaequi}
\begin{align*}
\dot{\mathcal{I}}_{\mathrm{1st},2} & =-\< \mathbf{J} D_{\mathbf{z}}\boldsymbol{\Phi}[{\bm z}](\dot{{\bm z}}-\tilde{{\bm z}}),\sigma_3\zeta_A^4 \boldsymbol{\eta} \>
-\< \begin{pmatrix}
	-\partial_{x}^2+ 1 &0\\0&1
\end{pmatrix} \boldsymbol{\eta}
,
\sigma_3\zeta_A^4 \boldsymbol{\eta}\>
\\
& \qquad +  \<   \left( f\left( \boldsymbol{\Phi}[{\bm z}]_1 +\eta _1 \right)-f(\boldsymbol{\Phi}[{\bm z}]_1) \right),
\zeta_A^4  {\eta}_1\> -\< \mathbf{J}  {\bm R}[{\bm z}],
\sigma_3\zeta_A^4 \boldsymbol{\eta}\>\\
&=:  \  C_1+C_2+C_3+C_4
\end{align*}
For the main term $C_2$, we have
\begin{equation}\label{eq:C2_bound1}
C_2 =- \<(-\partial_{x}^2+ 1) \eta_1, \zeta_A^4\eta_1\> + \norm{\zeta_A^2\eta_2} _{L^2}^2,
\end{equation}
with
\begin{align}
\label{eq:C2_bound2}
|\<(-\partial_{x}^2+ 1)\eta_1,\zeta_A^4\eta_1\>|\lesssim \norm{\sech  \left(   \frac{2}{A}x\right) \eta_1'}_{L^2}^2+\norm{\sech  \left(   \frac{2}{A}x\right) \eta_1}_{L^2}^2.
\end{align}
We consider  the remainder terms, starting with $ C_1 $. By \Cref{lem:modbound} and
\begin{equation}\label{scemetta}
	\av{D_\bmz \bmPhi\bra{\bmz}(w)\zeta_A^4}\lesssim  \sech\left(  \frac{4}{A}x\right)\sech  \left(  \kappa x\right) \av{w}, \quad \forall w \in \C^2,
\end{equation} we obtain
\begin{align}\label{eq:C1_bound}
|C_1|&\lesssim |\dot{{\bm z}}-\tilde{{\bm z}}|\norm{\sech  \left(  \kappa x\right)\boldsymbol{\eta}}_{L^2}\lesssim \delta ^{p-1} \norm{\boldsymbol{\eta}}_{L^2_{-\kappa}}^2 .
\end{align}
We next focus on $C_4$, with
\begin{equation}
\label{eq:C4_decomposition}
 \begin{aligned}
 C_4 & \ = C_{41}+ C_{42} , \\
 C_{41} & \ \defeq -\< \mathbf{J} D_{\bm z}\boldsymbol{\Phi}[{\bm z}]\tilde{{\bm z}}_R ,  \sigma_3\zeta_A^4 \boldsymbol{\eta} \>
 \\
  C_{42}& \ \defeq  \<{      f(\boldsymbol{\Phi}[{\bm z}]_1)   - f(Q)-f' (Q)  \widetilde{\boldsymbol{\Phi}}[{\bm z}]_1}, {\sigma_3\zeta_A^4 \boldsymbol{\eta}} \>
 \end{aligned}
\end{equation}
Combining  \eqref{scemetta} and \eqref{item:rem_better_1}, we get
\begin{align}\label{eq:C41_bound}
 | C_{41}| \le \av{\bm{z}}^2\norm{\boldsymbol{\eta}}_{L^2_{-\kappa}}\le  |{\bm z}|^4+ \| \boldsymbol{\eta} \| _{L^2_{- \kappa}}^2.
 \end{align}
By \eqref{eq:Ijest2}, we have
\begin{equation}\label{eq:C42_bound}
  | C_{42}|\leq \av{\bm{z}}^2\norm{\boldsymbol{\eta}}_{L^2_{-\kappa}} \le      |{\bm z}| ^4 +   \norm{\boldsymbol{\eta}}^2_{L^2_{-\kappa}} ,
\end{equation}
so that \cref{eq:C4_decomposition,eq:C42_bound,eq:C41_bound} give that
\begin{equation} \label{eq:C4_bound}
|C_4| \lesssim \av{\bmz}^4
+ \norm{   \boldsymbol{\eta}  } ^2 _{L^2_{-\kappa}}.
\end{equation}
To bound $C_3$, we notice that by
$$
 \av{f\left( \boldsymbol{\Phi}[{\bm z}]_1 +\eta _1 \right)-f(\boldsymbol{\Phi}[{\bm z}]_1)}\leq p \av{\boldsymbol{\Phi}[{\bm z}]_1}^{p-1} | \eta_1| + \av{\eta_1}^p
$$
 we obtain \begin{equation}\label{loveroC3}
	|C_3| \lesssim \| \sech\left(\kappa x\right)\eta_1\|_{L^2}^2+ \delta^{p-1}\| \sech\left(\frac{2}{A} x\right)\eta_1\|_{L^2}^2.
\end{equation}
Collecting the estimates in \cref{eq:C2_bound1,eq:C2_bound2,eq:C1_bound,loveroC3}, we have the conclusion.
\end{proof}

\begin{proof}[Proof of Proposition \ref{prop:1stvirial}]
It is immediate from  $|\mathcal{I}_{\mathrm{1st},1}|\lesssim A\delta^2$, $|\mathcal{I}_{\mathrm{1st},2}|\lesssim \delta^2$,   Lemmas \ref{lem:1stV1} and \ref{lem:1stV2} and the definition \eqref{eq:normA} of
$\| \cdot \| _{ \boldsymbol{ \Sigma }_A}.$
\end{proof}




\section{Technical estimates on the Darboux transform}\label{sec:tech}

 For the $N$  in \eqref{def:valueN}   and  the  $ \mathcal{A}$ in \eqref{def:calA}, we consider
\begin{align} \label{def:Tg} &
\mathcal{T}\defeq\<\im \varepsilon \partial_x\>^{- \pare{1+N} }\mathcal{A}^\ast.
\end{align}
We will use the following   formula proved in \cite{CM2022}, with $L^2_c(L_0)$ the continuous spectrum component in $L^2(\R , \C)$  of $L_0$ and $P_c$ the corresponding orthogonal projection,
\begin{align}\label{eq:Tinverse}
  {\bm u}=\prod_{j=0}^{N}R _{L_0}(\mu _j ) P_c \mathcal{A} \<   \im \varepsilon\partial_x\>^{N+1}  \mathcal{T} {\bm u} && \forall \ {\bm u}\in L^2_c(L_0).
\end{align}
In  \cite{CM2022}  the following result  is proved.
\begin{lemma}\label{lem:coer6A}
There exists a $ 0 < \kappa_0 \ll k_2  $ such that for any $ \kappa\in\pare{0, \kappa_0} $ there exists a constant $C_\kappa $ such that  for all  $0<\varepsilon\le  1$ and $\bm{w} \in L ^{2}_{-\frac{\kappa}{2}} $
\begin{align}\label{eq:coer6A1} \norm{
   \prod_{j=0}^{N}R _{L_0}(\mu _j ) P_c \mathcal{A}^* \<   \im \varepsilon\partial_x\>^{N+1}   \bm{w}  } _{L ^{2}_{-\kappa}} \le  C_\kappa    \|
     \bm{w}  \| _{L ^{2}_{-\frac{\kappa}{2}}} .
\end{align}
\end{lemma}
\qed


In \cite[Lemma 7.5]{CMS2023}, the following is proved.

\begin{lemma} \label{lem:KM1}
there exists $ A_0, \varepsilon_0 > 0 $ such that for any $ A > A_0 $ and $ \varepsilon\in\pare{0, \varepsilon_0} $ and for any $u\in H^1$ we have
\begin{align}\label{eq:KM1}
\norm{ \sech \left(   \frac{4}{A}    x \right)    \mathcal{T} u}_{L^2}\lesssim & \
 \varepsilon^{-\pare{1+N}}\norm{\sech \left(    \frac{2}{A}    x\right)u}_{L^2},\\ \label{eq:KM2}
\norm{ \sech \left(    \frac{4}{A}    x  \right) \partial_x\mathcal{T} u }_{L^2}\lesssim & \
 \varepsilon^{-\pare{1+N}}\norm{ \sech \left(  \frac{2}{A}    x\right)u'}_{L^2}+\norm{ u}_{L^2_{-\kappa}}.
\end{align}

\end{lemma}
\qed


%Let $ N=3 $ and
%\begin{align}\label{eqdefVN1}
%  V_{4} = \begin{cases}
%  -k_{3}k_{4 }\frac{2}{p-1}Q ^{p-1} >0 \text{ if    $2>p>5/3$}
%\\  0 \text{ if    $p=2$.}
%\end{cases}
%\end{align}
In \cite[Lemma 7.6]{CMS2023}   the following, obvious for $ V_{N+1} =0$,  is proved.

\begin{lemma} \label{lem:KM2}
For any $u\in H^1$,
\begin{align}  \label{eq:KM3}
\norm{     \comm{\<\im \varepsilon \partial_x\>^{-\pare{1+N}}}{V_{N+1}}\mathcal{A} u }_{L^2}& \ \lesssim
 \varepsilon \norm{ \mathcal{T}u} _{L^2_{-\kappa}}, \\
   \label{eq:KM3b} \norm{  \cosh \left( \frac{\kappa}{2} x\right)   \comm{\<\im \varepsilon \partial_x\>^{-\pare{1+N}}}{V_{N+1}} \mathcal{A} u}_{L^2}& \ \lesssim
 \varepsilon \norm{ \mathcal{T}u } _{L^2_{-\frac{\kappa}{2}}}.
\end{align}
\end{lemma} \qed

%\section{Proof of Proposition \ref{prop:2ndvirial}: Virial estimates for the transformed equation} \label{proof:2dnv}
%
%Using the operator $\mathcal{T}$  in  \eqref{def:Tg} with $ N=3 $, we consider the transformed variable
%\begin{align}   \label{def:vBg}
%{\bm v}\defeq\mathcal{T}\boldsymbol{\eta}.
%\end{align}
%Here notice that ${\bm v}$ is even for $5/3<p<2$ but is odd for $p=2$.
%For ${\bf L}_{ N+1}$ (cf \cref{eq:opLj,eq:linearNLKG_2}) the variable ${\bm v}$ satisfies
%\begin{multline}
%\label{eq:vBg}
%\dot{{\bm v}}=
%-\mathcal{T}D\boldsymbol{\Phi}[{\bm z}](\dot{{\bm z}}-\tilde{{\bm z}})
%+\mathbf{J}\left({\bf L}_{ N+1}{\bm v}
%+\begin{pmatrix}
%\comm{\< \ii\varepsilon \partial_x\>^{-\pare{N+1}}}{V_4} & 0 \\ 0 & 0
%\end{pmatrix}
%\mathcal{A}^*\boldsymbol{\eta}\right)\\
%+        \mathcal{T} \left  [\left(  f'(\boldsymbol{\Phi}[{\bm z}]_1) - f'(Q) \right) \eta _1 \mathbf{i}  + \left( f\left( \boldsymbol{\Phi}[{\bm z}]_1 +\eta _1 \right)-f(\boldsymbol{\Phi}[{\bm z}]_1)- f'(\boldsymbol{\Phi}[{\bm z}]_1)\eta _1  \right) \mathbf{i}
% - \mathbf{J }{\bm R}[{\bm z}]\right ]   ,
%\end{multline}
%where $ V_4$ is the potential in \eqref {eqdefVN1}.
%From Lemma \ref{lem:coer6A}, we have
%\begin{align}
%\|      \sech ( \kappa x)\boldsymbol{\eta}\|_{L^2}\lesssim \norm{ \sech \pare{  \frac{\kappa}{2} \  x }{\bm v}}_{L^2}.\label{eq:key1}
%\end{align}
%Set, for $\varphi_B$ defined in \eqref{def:zetaphi},
%\begin{align*}
%\psi_{A,B}=\chi_A^2 \varphi_B,
%&&
%  \tilde{S}_{A,B}=\frac{1}{2}\psi_{A,B}'+\psi_{A,B}\partial_x,
%\end{align*}
%and consider the functionals
%\begin{align*}
%\mathcal{I}_{\mathrm{2nd},1}\defeq\frac{1}{2}\< \mathbf{J} {\bm v},\tilde{S}_{A,B}{\bm v}\> ,
%&&
% \mathcal{I}_{\mathrm{2nd},2}\defeq\frac{1}{2}\< \mathbf{J}  {\bm v},\sigma_3 e^{-\kappa \<x\>}{\bm v}\> .
%\end{align*}
%
%\begin{lemma}\label{lem:2v1}
%	We have
%\begin{align} &
%\norm{v_1'}_{L^2_{-\frac{\kappa}{2}}}^2+\norm{v_1}_{L^2_{-\frac{\kappa}{2}}}^2+\dot{\mathcal{I}}_{\mathrm{2nd},1}
%\lesssim   \left(\varepsilon^{-\pare{N+1}}A^2\delta+A^{-1/2}\right)\|\boldsymbol{\eta}\|  _{\boldsymbol{ \Sigma }_A}^2
%+  |{\bm z} |^4.
%\label{eq:lem:2v1} \end{align}
%\end{lemma}
%
%\begin{proof}
%By \eqref{eq:vBg}, the time derivative of $\mathcal{I}_{\mathrm{2nd},1}$ reads
%\begin{align*}
%\dot{\mathcal{I}}_{\mathrm{2nd},1}= & \  -\< \mathcal{T}D\boldsymbol{\phi}[{\bm z}](\dot{{\bm z}}-\tilde{{\bm z}}),\tilde{S}_{A,B}{\bm v}\> -
%\<{\bf L}_{ N+1}{\bm v},\tilde{S}_{A,B}{\bm v}\> \\& +\<\begin{pmatrix}
%\comm{\< \ii\varepsilon \partial_x\>^{-\pare{N+1}}}{V_4} & 0 \\ 0 & 0
%\end{pmatrix}\mathcal{A}^*\boldsymbol{\eta},\tilde{S}_{A,B}{\bm v}\>
%  \\&+\<\mathcal{T} \left( \left(  f'(\boldsymbol{\Phi}[{\bm z}]_1) - f'(Q) \right) \eta _1   + \left( f\left( \boldsymbol{\Phi}[{\bm z}]_1 +\eta _1 \right)-f(\boldsymbol{\Phi}[{\bm z}]_1)- f'(\boldsymbol{\Phi}[{\bm z}]_1)\eta _1  \right)   \right)
% ,\tilde{S}_{A,B} {v}_1\>  \\& +\<\mathcal{T}  {\bm R}[{\bm z}] ,\tilde{S}_{A,B}{\bm v}\>
% \\
%=: & \ D_1+D_2+D_3+D_4+D_5.
%\end{align*}
%
%\begin{step}[Coercivity of the main contribution, $D_2$]
%First of all we have
%$$
%D_2=\<L_{N+1}v_1,\widetilde{S}_{A,B}v_1\>=\frac12 \<\big[L_{N+1},\widetilde{S}_{A,B}\big]v_1,v_1\>=\frac12 \<\big[-\partial_x^2 ,\widetilde{S}_{A,B}\big]v_1,v_1\>+\frac12 \<\big[V_4 ,\widetilde{S}_{A,B}\big]v_1,v_1\>
%$$
%The first term, as for \eqref{B2}, reads
%\begin{align*}
%\frac12 \<\big[-\partial_x^2 ,\widetilde{S}_{A,B}\big]v_1,v_1\>=&\int_{\R}  \psi_{A,B}' (x) (v_1')^2\, {\rm d}x- \frac14 \int_{\R} \psi_{A,B}'''(x) v_1^2(x)\, {\rm d}x
%%\\
%%=&\| \chi_A(x) \zeta_B(x) v_1'\|_{L^2}^2+\int_{\R}  2\chi_A(x) \chi'_A(x)\varphi_B(x) (v_1')^2 - \frac14 \int_{\R} \psi_{A,B}'''(x) v_1^2(x)\, {\rm d}x
%\end{align*}
%while the second term gives the following positive contribution
%\begin{align}
%\frac12 \<\comm{V_4}{\widetilde{S}_{A,B}}v_1,v_1\>=  \int_{\R} W_{A,B}(x) v_1^2, &&
%  W_{A,B}(x):= - \frac12 \chi_A^2(x)\varphi_B(x) V_4'(x) \geq 0
%\end{align}
%
%
%The following Lemma ensure the coercivity of $D_2$.
%
%\begin{lemma}
%Let $ p\in \left( \frac{5}{3}, 2 \right] $, there exists a $ \kappa_0, \varepsilon_0,  A_0, B_0 > 0 $ such that for any $ \kappa\in\pare{0, \kappa_0} , \ \varepsilon\in\pare{0, \varepsilon_0}, \  A>A_0, \ B>B_0 $
%	We have
%	\begin{multline*}
%	D_2 = \int_{\R}  \psi_{A,B}' (x) \pare{ v_1' }^2\, {\rm d}x- \frac14 \int_{\R} \psi_{A,B}'''(x) v_1^2(x)\, {\rm d}x+\int_{\R} W_{A,B}(x) v_1^2\, {\rm d} x\gtrsim \\
%	 \norm{  v_1'}_{L^2_{-\frac{\kappa}{2}}}^2+  \norm{ v_1 }_{L^2_{-\frac{\kappa}{2}}}^2
% + A^{-\tfrac12}	 \norm{ \bm{\eta}}_{\Sigma_A}^2
%%	 + R\pare{ \bm{\eta} }
%	\end{multline*}
%%	where $R\pare{\bm{\eta}}$ is  a quadratic remainder satisfying
%%	\begin{equation}\label{eq:restogenerico}
%%		\av{R\pare{\bm{\eta}}}\lesssim A^{-\tfrac12} \left( \norm{ \bm{\eta}}_{\Sigma_A}^2+ \varepsilon^{-\pare{N+1}} \norm{ \eta_1 }_{L^2_{-\frac{\kappa}{2}}}^2\right)	.
%%			\end{equation}
%\end{lemma}
%
%%\begin{notation}\label{notation:restogenerico}
%%We denote with $ R\pare{\bm{\eta}} $ any function satisfying \cref{eq:restogenerico}, whose explicit expression may implicitly vary from line to line.
%%\end{notation}
%
%\begin{proof}
%We shall only outline the proof of the above lemma and we shall refer to the detailed computations performed in \cite[Section 5]{KM22}. Notice that from \cref{eq:opLj,eq:valk} we have $ V_4 = - \pare{2-p}\pare{\frac{5-3p}{2}} Q^{p-1} $ hence it is immediate that there exists a $ 0< \kappa_0 \ll 1 $ such that $ \sech\pare{\frac{\kappa}{2} \ x}^2 \lesssim_p W_{A,B}\pare{x}  $ for all $ \kappa\in\pare{0, \kappa_0} $ and so we obtain that
%\begin{equation*}
%\norm{v_1}_{L^2_{-\frac{\kappa}{2}}}^2 \lesssim _p \int_{\R} W_{A,B}(x) v_1^2\, {\rm d} x.
%\end{equation*}
%Next, notice that $ \psi_{A,B}'= \zeta_B^2 + \pare{ \chi_A^2-1 } \zeta_B^2 + 2\chi_A \chi_A'\varphi_B $ so that
%\begin{equation*}
%\int_{\R}  \psi_{A,B}'  \pare{ v_1' }^2\, {\rm d}x
%=
% \int_\bR  \zeta_B^2   \pare{ v_1' }^2  \dd x
%+
%\int_\bR  \pare{ \chi_A^2-1 } \zeta_B^2    \pare{ v_1' }^2  \dd x
%+
%\int_\bR  2\chi_A \pare{ \chi_A }'\varphi_B    \pare{ v_1' }^2  \dd x.
%\end{equation*}
%It is immediate that
%\begin{equation*}
% \int_\bR  \zeta_B^2   \pare{ v_1' }^2  \dd x \sim \norm{ \sech\pare{\frac{x}{B} } v_1'}_{L^2}^2 \geq  \norm{v_1'}_{L^2_{-\xfrac{\kappa}{2}}}^2,
%\end{equation*}
%provided that $ B\geq \xfrac{4}{\kappa} $. From \cref{eq:relABg} and the fact that $ \av{\pare{ \chi_A^2-1 } \zeta_B^2 } \lesssim e^{-B}\sech\pare{\xfrac{8x}{A}} $ and \cref{eq:KM2,eq:key1} we obtain that
%\begin{equation*}
%\av{\int_\bR  \pare{ \chi_A^2-1 } \zeta_B^2    \pare{ v_1' }^2  \dd x}
%\lesssim e^{-B} \norm{v_1'}_{L^2_{-4/A}}\lesssim e^{-B} \pare{\varepsilon^{-\pare{N+1}}\norm{\bm{\eta}}_{\Sigma_A} + \norm{\bm{\eta}}_{L^2_{-\kappa}}}
%\end{equation*}
%
% and $ \av{\chi_A \pare{ \chi_A }' \varphi_B} \lesssim \xfrac{B}{A} \sech\pare{\xfrac{8x}{A}} $ and \cref{eq:KM1,eq:KM2} we obtain that
%\begin{equation*}
%\int_\bR  \pare{ \chi_A^2-1 } \zeta_B^2    \pare{ v_1' }^2  \dd x
%+
%\int_\bR  2\chi_A \pare{ \chi_A }'\varphi_B    \pare{ v_1' }^2  \dd x
%=
%R\pare{\bm{\eta}},
%\end{equation*}
%as per \Cref{notation:restogenerico}. At last we control the term $ - \frac14 \int_{\R} \psi_{A,B}'''(x) v_1^2(x)\, {\rm d}x $. Notice that
%\begin{multline*}
%-\frac14 \int_{\R} \psi_{A,B}'''(x) v_1^2(x)\, {\rm d}x
%=
%- \frac14 \int_{\R}  \pare{\zeta_B^2}'' v_1^2 \, \dd x
%\\
%+ \frac14 \int_{\R} \pare{1- \chi_A^2 } \pare{\zeta_B^2}'' v_1^2 \, \dd x
%- \frac14 \int_{\R} \pare{\sum_{j=1}^3 \binom{3}{j}\partial_x^j\pare{\chi_A^2}\partial_x^{3-j}\varphi_B} v_1^2 \, \dd x.
%\end{multline*}
%since $ \av{\pare{\zeta_B^2}''}\sim \frac{ e^{-\frac{2}{B}  \angles{x}}}{B^2} \geq B^{-2}\sech\pare{\frac{\kappa}{2} \ x} $ we obtain that
%\begin{equation*}
%\av{- \frac14 \int_{\R}  \pare{\zeta_B^2}'' v_1^2 \, \dd x}\gtrsim \frac{\norm{v_1}_{L^2_{- \frac{\kappa}{2}}}^2}{B^2},
%\end{equation*}
%moreover since $ \av{ \pare{1- \chi_A^2 } \pare{\zeta_B^2}''} \ll \pare{ \xfrac{e^{-B/4}}{B^2} }\sech\pare{8x/A} $ we obtain that
%\begin{equation*}
% \frac14 \int_{\R} \pare{1- \chi_A^2 } \pare{\zeta_B^2}'' v_1^2 \, \dd x
%= R\pare{\bm{\eta}},
%\end{equation*}
%via \cref{eq:KM1,eq:KM2}, while similar computations to the ones performed above, which can be found in \cite[Lemma 4]{KM22}, show that
%\begin{equation*}
%- \frac14 \int_{\R} \pare{\sum_{j=1}^3 \binom{3}{j}\partial_x^j\pare{\chi_A^2}\partial_x^{3-j}\varphi_B} v_1^2 \, \dd x
%= R\pare{\bm{\eta}}.
%\end{equation*}
%Collecting the estimates above we obtain the desired bound.
%\end{proof}
%
%
%\end{step}
%
%
%
%
%
%
%
%
%
%
%
%
%
%
%
%
%
%%\textcolor{red}{CONTINUA QUI}
%% Following \cite[Section 4.3]{KMM2022} (see as well \cite[Section 5]{KM22}),  we have
%%\begin{align}\label{def:xi}
%%D_2=\<L_{N+1}v_1,\widetilde{S}_{A,B}v_1\>=-\int \left(  \xi _1 ^{\prime \prime2}+V_B \xi _1 \right)\,dx+D_{21}  , && \xi _1\defeq \chi_A\zeta_B v_1,
%%\end{align}
%%and where
%%\begin{align}\label{eq:defvb}&
%% V_B = \frac{1}{2} B^{-1}\left(\chi'' |x|+2\chi' \frac{x}{|x|}\right)  -\frac{1}{2}\  \frac{\varphi _B}{\zeta _B^2}V'_{N+1} \text{ and}\\&
%%D_{21}=\frac{1}{4}\int (\chi_A^2)'(\zeta_B^2)'v_1^2+\frac{1}{2}\int \left(3(\chi_A')^2+\chi_A''\chi_A\right)\zeta_B^2v_1^2-\int (\chi_A^2)'\varphi_B(v_1')^2+\frac{1}{4}\int (\chi_A^2)''' \varphi_B v_1^2. \nonumber
%%\end{align}
%%We claim that
%%\begin{align}\label{eq:KMlemma3}
%%\int ( \xi _1 ^{\prime  2}+V_B \xi _1 )\,dx  +A^{-1}\|\boldsymbol{\eta}\|_{\boldsymbol{ \Sigma }_A}^2\gtrsim \left(\norm{\sech \left( \frac{\kappa}{2} x\right) v_1'}_{L^2}^2+\norm{\sech \left( \frac{\kappa}{2} x\right) v_1}^2\right).
%%\end{align}
%%The proof is like in \cite[Lemma 3]{KM22}.  We have, by $0<\zeta  _B\le 1$,
%%\begin{align*} & \int _{|x|\le A}\sech \left(  \kappa x\right)v_1^2\le  \int _{|x|\le A}\sech \left(  \frac{\kappa}{2} x\right) \zeta ^2_B   v_1^2\le  \int _{|x|\le A}\sech \left(  \frac{\kappa}{2} x\right) \xi_1^2.
%%\end{align*}
%%We have
%%\begin{align*} & \int _{|x|\le A}\sech \left(  \kappa x\right)v_1 ^{\prime 2}\le  \int _{|x|\le A}\sech \left(  \frac{\kappa}{2} x\right) \left( \xi _1' - \zeta '_Bv_1\right) ^2
%%\lesssim  \int _{|x|\le A}\sech \left(  \frac{\kappa}{2} x\right)   (  \xi _1 ^{\prime 2} + \xi _1 ^{  2}    )    .
%%\end{align*}
%%We have, thanks to \cref{eq:KM1}
%%\begin{align*}
%% \int _{|x|\ge A}\sech \left(  \kappa x\right) \left(  v_1 ^{\prime 2}+     v_1^2\right) &\le  \sech \left(  \frac{\kappa}{2} A\right)\int _{\R }\sech \left( \frac{8}{A} x\right) \left(  v_1 ^{\prime 2}+     v_1^2\right) dx \\
%%   &\lesssim \sech \left(  \frac{\kappa}{2} A\right) \varepsilon ^{-\pare{N+1}}\int _{\R }\sech \left( \frac{4}{A} x\right) \left(  \eta_1 ^{\prime 2}+     \eta_1^2\right) dx\le A ^{-1}\|\boldsymbol{\eta}\|_{\boldsymbol{ \Sigma }_A}^2.
%%\end{align*}
%%Finally,  we claim the following, which  completes the proof of \eqref {eq:KMlemma3},
%%\begin{align} \label{eq:coercVB} & \int _{\R } \sech \left(  \frac{\kappa}{2} x\right)   (  \xi _1 ^{\prime 2} + \xi _1 ^{  2}    )  \lesssim  \int _{\R }( \xi _1 ^{\prime  2}+V_B \xi _1^{  2}   )\,dx .
%%\end{align}
%%In the case $5/3<p<2$, the above inequality  is true for all $\xi _1\in H ^{1}    \left( \R \right)$ and
%%follows easily from the fact  for $B$ large the potential $V_B$ is a small perturbation of the  positive potential  $- \  \frac{\varphi _B}{\zeta _B^2}V'_{N+1}
%%  >0$ for $x\neq 0$. In the case $p=2$, then  $V _{N+1}=V_3=0$, and so $V_B$ is a small potential.  Then for $B$ large, the coercivity in \eqref{eq:coercVB} is true
%% only for $\xi _1\in H ^{1} _{\odd}  \left( \R \right)$.    Fortunately, from \eqref{def:xi} we see that $\xi _1$ has the same parity of $v_1$ and, as we remarked right under  \eqref{def:vBg},   $v_1$ is odd in the case $p=2$.
%%
%%
%%\noindent We next have the following, which is \cite[Lemma 4]{KM22} in the case $ N=3 $ (see also \cite{CMS2023}),
%%\begin{align}\label{eq:KMlemma4}
%%|D_{21}|\lesssim A^{-1/2}\left(\|\boldsymbol{\eta}\|    _{\boldsymbol{ \Sigma }_A}^2+\|\sech \left( \kappa x\right)\eta_1\|_{L^2}^2\right) \lesssim A^{-1/2}\left(\|\boldsymbol{\eta}\|    _{\boldsymbol{ \Sigma }_A}^2+   \varepsilon ^{-\pare{N+1}}\norm{\sech \left( \frac{\kappa}{2} x\right)\eta_1}_{L^2}^2\right) .
%%\end{align}
%
%
%
%\begin{step}[Analysis of the quadratic terms $ D_1 $ and $ D_3 $]
%  By Lemma \ref{lem:modbound}, the estimate $ \av{\tilde{S}_{A,B}^\ast \cT D_\bmz\bmPhi\bra{\bmz}} \sim e^{-\kappa\av{x}} $ which holds for $ 0< \kappa \ll k_2 $ and \cref{eq:KM1} we have that
%  \begin{equation}
%  \label{eq:D1}
%  \begin{aligned}
%|D_1|&\lesssim |\dot{{\bm z}}-\tilde{{\bm z}}| \|  \sech \left( 2\kappa x\right)    {\bm v}\|_{L^2}\\
%& \lesssim \delta ^{p-1} \|\sech \left(  \kappa x\right) {\eta}_1\|_{L^2} \|\sech \left( 2\kappa x\right) {\bm v}\|_{L^2} \\
%& \lesssim \delta ^{p-1} \varepsilon ^{-\pare{N+1}} \|\sech \left(  \kappa x\right) \boldsymbol{\eta} \|_{L^2}^2 .
%\end{aligned}
%  \end{equation}
%Like in \cite{CMS2023},  we have (obviously here  we care only of case $2>p>5/3$ where $V_4\neq 0$)
%\begin{equation}
%\label{eq:D3}
%\begin{aligned}
%|D_3|  & \ =\av{\< \comm{\<\im \varepsilon \partial_x\>^{-\pare{N+1}}}{V_4} \mathcal{A}^*\eta _1, \widetilde{S} _{A,B}v_1\>}  \\&
%\le \norm{ \cosh \left( \frac{\kappa}{2} x\right)\comm{\<\im \varepsilon \partial_x\>^{-\pare{N+1}}}{V_4} \mathcal{A}^*\eta _1 } _{L^2} \norm{ \sech \left( \frac{\kappa}{2} x\right) \widetilde{S} _{A,B}v_1} _{L^2}
%\\&
%\leq   \varepsilon   \norm{ \sech \left( \frac{\kappa}{2} x\right)v_1}_{L^2}  \left(\norm{\sech \left( \frac{\kappa}{2} x\right) v_1'}_{L^2}+\norm{\sech \left( \frac{\kappa}{2} x\right) v_1}_{L^2}\right)\\&
%\lesssim\varepsilon \left(\norm{v_1'}_{L^2_{-\frac{\kappa}{2}}}^2+\norm{v_1}_{L^2_{-\frac{\kappa}{2}}}^2\right) ,
%\end{aligned}
%\end{equation}
%where the upper bound can be absorbed inside the left hand side of \eqref{eq:lem:2v1}.
%\end{step}
%
%
%
%
%
%
%
%
%
%
%
%
%\begin{step}[Analysis of the purely nonlinear terms, $ D_4 $ and $ D_5 $]
%  We now write $D_4= D _{41}+ D _{42}$, with \footnote{Notice that in the estimate of the term $D_4$ in \cite[Lemma 8.1]{CMS2023} the operator $\mathcal{T}$ is omitted by mistake. The proof can be corrected proceeding like here.}
% \begin{align*}&
%D _{41} \defeq\<\mathcal{T}   \left(  f'(\boldsymbol{\Phi}[{\bm z}]_1) - f'(Q) \right) \eta _1
% ,\tilde{S}_{A,B} {v}_1\>    \\& D _{42} \defeq\<\mathcal{T}   \left(  \left( f\left( \boldsymbol{\Phi}[{\bm z}]_1 +\eta _1 \right)-f(\boldsymbol{\Phi}[{\bm z}]_1)- f'(\boldsymbol{\Phi}[{\bm z}]_1)\eta _1  \right)   \right)
% ,\tilde{S}_{A,B} {v}_1\>  .
%\end{align*}
% By  $\sech \left( \frac{4}{A}  x \right)\sim 1$ on               $\supp \psi_{A,B}\subseteq [-2A,2A] $,  by \cref{eq:estder1,eq:estder2}
% and by \cref{eq:KM1,eq:KM2},
% \begin{equation}
% \label{eq:D41}
% \begin{aligned}
%   |D _{41}|& \le \norm{    \sech \left( \frac{4}{A}  x \right) \mathcal{T} \left(  f'(\boldsymbol{\Phi}[{\bm z}]_1) - f'(Q) \right) \eta _1 } _{L^2}
%    \left(  \norm{\sech \left(  \frac{4}{A} x \right) v_1}_{L^2}
%+ \norm{\sech \left(  \frac{4}{A} x \right)v_1'}_{L^2}\right)  \\&  \lesssim \varepsilon ^{-2\pare{N+1}}  \norm{    \left(  f'(\boldsymbol{\Phi}[{\bm z}]_1) - f'(Q) \right) \eta _1 } _{L^2} \left(  \norm{\sech \left(  \frac{2}{A} x \right) \eta_1}_{L^2}
%+ \norm{\sech \left(  \frac{2}{A} x \right)\eta_1'}_{L^2}\right)
%\\& \lesssim   \varepsilon ^{-2\pare{N+1}}A |{\bm z}|   \norm{    e^{-\frac{p-1}{2}|x|} \eta _1 }_{L^2} \norm{\boldsymbol{\eta}}_{\boldsymbol{ \Sigma }_A}
% \lesssim \varepsilon ^{-2\pare{N+1}}A ^2 \delta   \norm{\boldsymbol{\eta}}_{\boldsymbol{ \Sigma }_A}^2 .
%\end{aligned}
%\end{equation}
%We have similarly
%   \begin{align*}
%   |D _{42}|& \le  \varepsilon ^{-2\pare{N+1}}A \norm{   \sech \left( \frac{2}{A}  x \right) \left( f\left( \boldsymbol{\Phi}[{\bm z}]_1 +\eta _1 \right)-f(\boldsymbol{\Phi}[{\bm z}]_1)- f'(\boldsymbol{\Phi}[{\bm z}]_1)\eta _1  \right)  }_{L^2}
%     \norm{ \boldsymbol{\eta}}_{\boldsymbol{ \Sigma }_A}
%\end{align*}
%
%Invoking the estimates in \cref{eq:buuuuh1,eq:buuuuh2}
%  \begin{align}\label{eq:D42}
%   |D _{42}|& \lesssim   \varepsilon ^{-2\pare{N+1}}A^2 \|  \eta _1\| _{L^\infty}^{p-1}
%     \|\boldsymbol{\eta}\|    _{\boldsymbol{ \Sigma }_A}^2 \lesssim \varepsilon ^{-2\pare{N+1}}A^2    \delta^{p-1}
%     \|\boldsymbol{\eta}\|    _{\boldsymbol{ \Sigma }_A}^2.
%\end{align}
%Hence \cref{eq:D41,eq:D42} give that
%\begin{equation}
%\label{eq:D4}
%\av{D_4} \lesssim \pare{ \frac{A}{\varepsilon^4} }^2    \delta^{p-1}
%     \|\boldsymbol{\eta}\|    _{\boldsymbol{ \Sigma }_A}^2.
%\end{equation}
%Finally,  we consider
%\begin{align*}
% D_5 =
% \< \mathcal{T} D_{\bm z}\boldsymbol{\Phi}[{\bm z}]\tilde{{\bm z}}_R ,\tilde{S}_{A,B}{\bm v}\> -\< \mathcal{T} \left(  f(\boldsymbol{\Phi}[{\bm z}]_1)   - f(Q)-f' (Q)  \widetilde{\boldsymbol{\Phi}}[{\bm z}]_1  \right) ,\tilde{S}_{A,B} {v}_1\> =:D_{51}+D_{52}.
%\end{align*}
%\todoSS{This part is changed, check}
%{\color{red} We have
%\begin{align*}
% D_{51} = & \  D_{511}+ D_{512} \\
%   = & \  - \< \partial _x \mathcal{T} D_{\bm z}\boldsymbol{\Phi}[{\bm z}]\tilde{{\bm z}}_R , \psi_{A,B} {\bm v}\>   - \frac{1}{2} \< \mathcal{T}  D_{\bm z}\boldsymbol{\Phi}[{\bm z}]\tilde{{\bm z}}_R , \psi_{A,B}' {\bm v}\>
%\end{align*}
% We have
%\begin{align*}
% | D_{511}|     \le     \norm{ \cosh \left(  \kappa x \right) \partial _x \mathcal{T} D_{\bm z}\boldsymbol{\Phi}[{\bm z}]\tilde{{\bm z}}_R } _{L^2}
%    \| \sech \left(  \kappa x \right) {\bm v} \| _{L^2}   .
% \end{align*}
% Explicit computations show that $ \av{\partial _x \mathcal{T} D_{\bm z}\boldsymbol{\Phi}[{\bm z}]} \sim e^{-k_2\av{x}} $, hence if $ 0 <  \kappa \ll k_2 $
% \begin{align*}
%  D_{511} \lesssim |\tilde{{\bm z}}_R|.
% \end{align*}
%The same argument yields the same bounds for $D_{512}$ so
%  we conclude that
% \begin{align}\label{eq:D51}
% \av{ D_{51}}      \lesssim    |\tilde{{\bm z}}_R| \
%    \|  {\bm v} \| _{L^2_{-\kappa }}  \lesssim \delta _2^{-1}|{\bm z}|^4 + \delta _2 \|  {\bm v} \| _{L^2_{-\frac{\kappa}{2}}}^2.
% \end{align}}
% {\color{red} \todoSS{Completamente misterioso}Turning to $D _{52}$, we have
% \begin{align*}&
% | D_{52}|     \le      \|  \mathbf{C}  \| _{L^2 (|x|\le 2A)}
%    \| \sech \left( \kappa x \right)  \left(   2^{-1}\psi_{A,B}' v_1+\psi_{A,B}v_1'    \right)   \| _{L^2}   \text{  where } \\&\mathbf{C}
%\defeq
%  \cosh \left(  \kappa x \right) \mathcal{T} \left(  f(\boldsymbol{\Phi}[{\bm z}]_1)   - f(Q)-f' (Q)  \widetilde{\boldsymbol{\Phi}}[{\bm z}]_1  \right) .
% \end{align*}
% For a cutoff $chi$ like in \Cref{notation:chi} ,  we write  $ \mathbf{C } =\mathbf{C }_{1}+\mathbf{C }_{2}$, with
% \begin{align*}
%   & \mathbf{C }_{1}\defeq  \cosh \left( \kappa x \right) \mathcal{T}   \chi (x-\cdot ) \left(  f(\boldsymbol{\Phi}[{\bm z}]_1)   - f(Q)-f' (Q)  \widetilde{\boldsymbol{\Phi}}[{\bm z}]_1  \right) ,\\&  \mathbf{C }_{2}\defeq  \cosh \left( \kappa x \right) \mathcal{T}   \chi _1(x-\cdot ) \left(  f(\boldsymbol{\Phi}[{\bm z}]_1)   - f(Q)-f' (Q)  \widetilde{\boldsymbol{\Phi}}[{\bm z}]_1  \right) \text{ with }\chi _1=1-\chi .
% \end{align*}
% For the $I$ in \eqref{eq:defI},  by    \eqref{eq:Ijest2}  we have
% \begin{align}\label{eq:estbfc2}
%   \|  \mathbf{C} _2 \| _{L^2  } \lesssim \| \cosh \left(  \kappa x \right) \int   e^{-\frac{|x-y|}{3\varepsilon}} |I (y)| dy\| _{L^2  } \lesssim  |{\bm z}|^2\| \cosh \left(  \kappa x \right) \int   e^{-\frac{|x-y|}{3\varepsilon}} e^{-|y| } dy\| _{L^2  }  \lesssim  |{\bm z}|^2,
% \end{align}
%   where we used  \cite[Lemma 5.6]{CM2022}   which states that there exist constants $C_0>0$ and $ \varepsilon _0>0$ such that for   $\varepsilon  \in (0, \varepsilon _0)$   and for $K_{\varepsilon}(x,y) \in \mathcal{D}'(\R \times \R)$   the Schwartz kernel of  $\mathcal{T} $, then  we have
%\begin{align} \nonumber
%| K_{\varepsilon}(x,y)    | \le C_0  e^{-\frac{|x-y|}{3\varepsilon}}  \text{  for all $x,y$ with $|x-y|\ge 1$.}
%\end{align}
% We bound
% \begin{align*}&
%  \|  \mathbf{C} _2 \| _{L^2 (|x|\le 2A)  } \\&\lesssim  \int _{\R} d\xi   | \widehat{\chi}(\xi) |\sum _{j=0 }^{ \log (10A)}
%     2^{ \kappa j} \|    \varphi _j    \mathcal{T}   e ^{\im \xi \sqcup }    \widetilde{\varphi} _j   \left(  f(\boldsymbol{\Phi}[{\bm z}]_1)   - f(Q)-f' (Q)  \widetilde{\boldsymbol{\Phi}}[{\bm z}]_1  \right) \| _{L^2 (|x|\le 2A)}
% \end{align*}
% for appropriate Paley--Littlewood decompositions  (in $x$ and $y$ space) with $ \supp  \varphi _j $ and $ \supp  \widetilde{\varphi} _j $ in $|x|\sim 2 ^{j}$ for $j\ge 1$ and   $|x|\lesssim 1$ for $j=0$.
%  Since for $|y|\lesssim 2 ^ j$  with $j\le   \log (10A)$ we have
% \begin{align} \label{eq:estbfc2-}&
%   f(\boldsymbol{\Phi}[{\bm z}]_1)   - f(Q)-f' (Q)  \widetilde{\boldsymbol{\Phi}}[{\bm z}]_1  = 2^{-1} f'' (Q)  \left( \widetilde{\boldsymbol{\Phi}}[{\bm z}]_1\right) ^2 +  {E} \text{  with}\\& |E|\lesssim \left | f''' (Q)  \left( \widetilde{\boldsymbol{\Phi}}[{\bm z}]_1\right) ^3 \right | \lesssim |{\bm z}|^3 e^{-\frac{3-p}{2} |x|  } ,\nonumber
% \end{align}
% we accordingly split  $\mathbf{C} _2= \mathbf{C} _{21}+\mathbf{C} _{22} $. We have
% \begin{align*}&
%  \|  \mathbf{C} _{21} \| _{L^2 (|x|\le 2A)  } \lesssim    \int _{\R} d\xi   | \widehat{\chi}(\xi) |\sum _{j=0 }^{ \log (10A)}
%     2^{ \kappa j} \|   \< \im \varepsilon \partial _x \> ^{-\pare{N+1}}    \mathcal{A }^*    e ^{\im \xi \sqcup }    \widetilde{\varphi} _j   f'' (Q)  \left( \widetilde{\boldsymbol{\Phi}}[{\bm z}]_1\right) ^2 \| _{L^2 (|x|\le 2A)}\\& \lesssim  \|  \< \xi \>  ^{N+1}  \widehat{\chi}  \|_{L^1\left(\R _\xi \right)}
%     \sum _{j=0 }^{ \log (10A)} 2^{ \kappa j} \|  \widetilde{\varphi} _j   f'' (Q)  \left( \widetilde{\boldsymbol{\Phi}}[{\bm z}]_1\right) ^2 \| _{H^{N+1}(\R )}  \lesssim |{\bm z}|^2.
% \end{align*}
% We have
% \begin{align*}&
%  \|  \mathbf{C} _{22} \| _{L^2 (|x|\le 2A)  } \lesssim   \int _{\R} d\xi   | \widehat{\chi}(\xi) |\sum _{j=0 }^{ \log (10A)}
%     2^{\kappa  j} \|   \< \im \varepsilon \partial _x \> ^{-\pare{N+1}}    \mathcal{A }^*    e ^{\im \xi \sqcup }    \widetilde{\varphi} _j  E  \| _{L^2 (|x|\le 2A)}\\& \lesssim  \|     \widehat{\chi}  \|_{L^1 \left(\R _\xi \right)} \varepsilon^{-\pare{N+1}}
%     \sum _{j=0 }^{ \log (10A)} 2^{  \kappa  j} \| E  \| _{L^2(\R )}  \lesssim  A \varepsilon^{-\pare{N+1}}  |{\bm z}| ^3 \lesssim  A \varepsilon^{-\pare{N+1}} \delta  |{\bm z}| ^2.
% \end{align*}
% Summing up, we have
%\begin{align}\label{eq:D52}
%   \av{ D_{52}}  \lesssim \delta _2^{-1}|{\bm z}|^4 + \delta _2 \|  {\bm v} \| _{L^2_{-\frac{\kappa}{2}}}^2 +   \delta _2 \norm{ \sech \left(  \frac{\kappa}{2} x\right)   {v}'_1 } _{L^2  } ^2
%\end{align}}
%\Cref{eq:D51,eq:D52} give that
%\begin{equation}\label{eq:D5}
%\av{ D_{5}}  \lesssim \frac{\av{\bm z}^4}{\delta _2} + \delta _2 \|  {\bm v} \| _{L^2_{-\frac{\kappa}{2}}}^2 +   \delta _2 \norm{ \sech \left(  \frac{\kappa}{2} x\right)   {v}'_1 } _{L^2  } ^2
%\end{equation}
%\end{step}
%Collecting the estimates in \cref{eq:D1,eq:D3,eq:D4,eq:D5,eq:KMlemma4,eq:KMlemma3,def:xi} and bootstrapping, we deduce the estimate \eqref{eq:lem:2v1}.
%\end{proof}
%
%\begin{lemma}\label{lem:2v2}
%	We have
%	\begin{align}\label{eq:lem:2v2}
%		\|e^{-\kappa\<x\>/2}v_2\|_{L^2} + \dot{\mathcal{I}}_{\mathrm{2nd},2}\lesssim& \|e^{-\kappa \<x\>/2}v_1'\|_{L^2}^2+
%\|e^{-\kappa \<x\>/2}v_1\|_{L^2}^2+|{\bm z}|^4+
%   o_{\varepsilon}(1)
%   \| \boldsymbol{\eta} \|_{ \boldsymbol{ \Sigma }_A}^2.
%	\end{align}
%\end{lemma}
%
%\begin{proof}
%Differentiating $\mathcal{I}_{\mathrm{2nd},2}$, we have
%\begin{align*}&
%\dot{\mathcal{I}}_{\mathrm{2nd},2}=    -\< \mathcal{T}D\boldsymbol{\phi}[{\bm z}](\dot{{\bm z}}-\tilde{{\bm z}}),\sigma_3 e^{-\kappa \<x\>}{\bm v}\> \\& -
%\<{\bf L}_{ N+1}{\bm v},\tilde{S}_{A,B}{\bm v}\> +\<\begin{pmatrix}
%\comm{\< \ii\varepsilon \partial_x\>^{-\pare{N+1}}}{V_4} & 0 \\ 0 & 0
%\end{pmatrix}\mathcal{A}^*\boldsymbol{\eta},\sigma_3 e^{-\kappa \<x\>}{\bm v}\>
%  \\&+\<\mathcal{T} \left( \left(  f'(\boldsymbol{\Phi}[{\bm z}]_1) - f'(Q) \right) \eta _1   + \left( f\left( \boldsymbol{\Phi}[{\bm z}]_1 +\eta _1 \right)-f(\boldsymbol{\Phi}[{\bm z}]_1)- f'(\boldsymbol{\Phi}[{\bm z}]_1)\eta _1  \right)   \right)
% ,  e^{-\kappa \<x\>}  {v}_1\>  \\& +\<\mathcal{T}  {\bm R}[{\bm z}] ,\sigma_3 e^{-\kappa \<x\>}{\bm v}\>
%=:E_1+E_2+E_3+E_4+E_5.
%\end{align*}
%Like in \cite{CMS2023} we have
%\begin{align*}&
%	E_2=-\|e^{-\kappa\<x\>/2}v_2\|_{L^2}^2+\<L_{N+1}v_1, e^{-\kappa\<x\>}v_1\>=-\|e^{-\kappa\<x\>/2}v_2\|_{L^2}^2+E_{21}, \text{  with}\\&
%	|E_{21}|\lesssim  \|e^{-\kappa \<x\>/2}v_1'\|_{L^2}^2+\|e^{-\kappa \<x\>/2}v_1\|_{L^2}^2.
%\end{align*}
%By Lemma \ref{lem:modbound}, we have
%\begin{align*}
%	|E_1|\lesssim \delta ^{p-1}\|e^{-\kappa\<x\>/2}{\bm v}\|_{L^2}  \|e^{-\kappa\<x\>}\boldsymbol{\eta}\|_{L^2}
% \lesssim \delta^{p-1} \varepsilon ^{-N}\|e^{-\kappa\<x\>/2}{\bm v}\|_{L^2}^2.
%\end{align*}
%By \eqref{eq:KM3b}, we have
%\begin{align*}
%	&|E_3|  = |\<  [\<\im\varepsilon\partial_x\>^{-N},V_4]\mathcal{A}^*\eta_1,\sigma_3e^{-\kappa\<x\>}v_1\>| \lesssim
% \varepsilon \|e^{-\frac{\kappa}{2}\<x\> }v_1\|_{L^2} \|e^{-\kappa\<x\> }v_1\|_{L^2}    \le  \varepsilon \|e^{-\frac{\kappa}{2}\<x\> }v_1\|_{L^2}^2.
%\end{align*}
%Proceeding like in Lemma \ref{lem:2v1} we write
%$E_4= E _{41}+ E _{42}$, with \footnote{Also in the estimate of the term $E_4$ in \cite[Lemma 8.2]{CMS2023}, like in the estimate of
%$D_4$ in \cite[Lemma 8.1]{CMS2023},
%the operator $\mathcal{T}$ has been omitted by mistake. The proof can be corrected proceeding like here.}
% \begin{align*}&
%E _{41} \defeq\<\mathcal{T}   \left(  f'(\boldsymbol{\Phi}[{\bm z}]_1) - f'(Q) \right) \eta _1
% ,e^{-\kappa \<x\>} {v}_1\>    \\& E _{42} \defeq\<\mathcal{T}   \left(  \left( f\left( \boldsymbol{\Phi}[{\bm z}]_1 +\eta _1 \right)-f(\boldsymbol{\Phi}[{\bm z}]_1)- f'(\boldsymbol{\Phi}[{\bm z}]_1)\eta _1  \right)   \right)
% ,e^{-\kappa \<x\>} {v}_1\>  .
%\end{align*}
% By  an  analogue of \eqref{eq:key1}
% and by \eqref{eq:KM1}--\eqref{eq:KM2},
% \begin{align*}
%   |E _{41}|& \le \varepsilon ^{- N-1}  \|      \left(  f'(\boldsymbol{\Phi}[{\bm z}]_1) - f'(Q) \right) \eta _1 \| _{L^2}
%       \|e^{-\kappa \<x\>} {v}_1\|_{L^2}   \\&   \lesssim   \varepsilon ^{-\pare{N+1}}A |{\bm z}|   \|    e^{-\frac{p-1}{2}|x|} \eta _1 \| _{L^2} \|e^{-\kappa \<x\>} {v}_1\|_{L^2}
% \lesssim   \varepsilon ^{-N}A \delta   \|    e^{-\frac{p-1}{4}|x|} v _1 \| _{L^2} \|e^{-\kappa \<x\>} {v}_1\|_{L^2}  \\&   \lesssim     \varepsilon ^{- N-1}A  \delta  \|e^{-\kappa \<x\>} {v}_1\|_{L^2} ^2 .
%\end{align*}
%In analogy to estimates for  $D_{41} $ and $D_{42} $ in Lemma \ref{lem:2v1}, but using $\sech \left( \frac{4}{A}  x \right)e^{-\frac{\kappa }{2} \<x\>}\lesssim e^{-\kappa \<x\>}$ along with
%  \eqref{eq:KM1}--\eqref{eq:KM2},
% \begin{align*}
%   |E _{42}|& \le \|   \sech \left( \frac{4}{A}  x \right) \mathcal{T} \left(   f\left( \boldsymbol{\Phi}[{\bm z}]_1 +\eta _1 \right)-f(\boldsymbol{\Phi}[{\bm z}]_1)- f'(\boldsymbol{\Phi}[{\bm z}]_1)\eta _1 \right)  \| _{L^2}
%      \|  e^{-\frac{\kappa }{2} \<x\>} v_1\|_{L^2} \\&
%        \lesssim   \varepsilon ^{-2\pare{N+1}}A  \|   \sech \left( \frac{2}{A}  x \right) \left( f\left( \boldsymbol{\Phi}[{\bm z}]_1 +\eta _1 \right)-f(\boldsymbol{\Phi}[{\bm z}]_1)- f'(\boldsymbol{\Phi}[{\bm z}]_1)\eta _1  \right)  \| _{L^2} \|e^{-\kappa \<x\>} {v}_1\|_{L^2}
% \\& \lesssim   \varepsilon ^{-2\pare{N+1}}A \delta ^{p-1}  \|\boldsymbol{\eta}\|    _{\boldsymbol{ \Sigma }_A} \|  e^{-\frac{\kappa }{2} \<x\>} v_1\|_{L^2}    \le      \varepsilon ^{-2\pare{N+1}}A \delta ^{p-1} \left(  \|\boldsymbol{\eta}\|    _{\boldsymbol{ \Sigma }_A} ^2+  \|  e^{-\frac{\kappa }{2} \<x\>} v_1\|_{L^2} ^2    \right).
%\end{align*}
% Finally,  we consider
%\begin{align*}
% E _5 =
% \< \mathcal{T} D_{\bm z}\boldsymbol{\Phi}[{\bm z}]\tilde{{\bm z}}_R ,\sigma_3 e^{-\kappa \<x\>}{\bm v}\> -\< \mathcal{T} \left(  f(\boldsymbol{\Phi}[{\bm z}]_1)   - f(Q)-f' (Q)  \widetilde{\boldsymbol{\Phi}}[{\bm z}]_1  \right) ,  e^{-\kappa \<x\>}  {v}_1\> =:E_{51}+E_{52}.
%\end{align*}
%We have
%The last three terms can be bounded similarly. We have
%\begin{align*}
% | E_{51}|     \le     \|  \mathcal{T} D_{\bm z}\boldsymbol{\Phi}[{\bm z}]\tilde{{\bm z}}_R \| _{L^2}
%    \|e^{-\kappa \<x\>} {\bm v} \| _{L^2}  \lesssim  \|   \mathcal{A}^* D_{\bm z}\boldsymbol{\Phi}[{\bm z}]\tilde{{\bm z}}_R \| _{L^2}
%    \|e^{-\kappa \<x\>} {\bm v} \| _{L^2} \lesssim    \delta _2^{-1}|{\bm z}|^4 + \delta _2 \| e^{-\frac{\kappa}{2} \<x\>} {\bm v} \| _{L^2 }^2.
% \end{align*}
% Turning to $E _{52}$, we have
% \begin{align*}&
% | E_{52}|     \le      \|  \mathbf{E}  \| _{L^2  }
%    \| e^{-\frac{\kappa}{2} \<x\>}v_1         \| _{L^2}   \text{  where } \\&\mathbf{E}
%\defeq
%  e^{-\frac{\kappa}{2} \<x\>} \mathcal{T} \left(  f(\boldsymbol{\Phi}[{\bm z}]_1)   - f(Q)-f' (Q)  \widetilde{\boldsymbol{\Phi}}[{\bm z}]_1  \right) .
% \end{align*}
% Let us write  $ \mathbf{E } =\mathbf{E }_{1}+\mathbf{E}_{2}$ with
% \begin{align*}
%   & \mathbf{E}_{1}\defeq  e^{-\frac{\kappa}{2} \<x\>}\mathcal{T}   \chi (x-\cdot ) \left(  f(\boldsymbol{\Phi}[{\bm z}]_1)   - f(Q)-f' (Q)  \widetilde{\boldsymbol{\Phi}}[{\bm z}]_1  \right) ,\\&  \mathbf{E }_{2}\defeq  e^{-\frac{\kappa}{2} \<x\>} \mathcal{T}   \chi _1(x-\cdot ) \left(  f(\boldsymbol{\Phi}[{\bm z}]_1)   - f(Q)-f' (Q)  \widetilde{\boldsymbol{\Phi}}[{\bm z}]_1  \right) \text{ with }\chi _1=1-\chi .
% \end{align*}
% Like for \eqref{eq:estbfc2},   we have
% \begin{align*}
%   \|  \mathbf{E} _2 \| _{L^2  } \lesssim \|  e^{-\frac{\kappa}{2} \<x\>} \int   e^{-\frac{|x-y|}{3\varepsilon}} |I (y)|dy\| _{L^2  } \lesssim  |{\bm z}|^2\|  e^{-\frac{\kappa}{2} \<x\>} \int   e^{-\frac{|x-y|}{3\varepsilon}} e^{-|y| } dy\| _{L^2  }  \lesssim  |{\bm z}|^2.
% \end{align*}
% In analogy to the bound of $\mathbf{C} _2$   in Lemma \ref{lem:2v1}, we bound
% \begin{align*}&
%  \|  \mathbf{E} _2 \| _{L^2 } \lesssim  \int _{\R} d\xi   | \widehat{\chi}(\xi) |\sum _{j=0 }^{ \infty}
%     2^{ -\frac{\kappa}{2} j} \|    \varphi _j    \mathcal{T}   e ^{\im \xi \sqcup }    \widetilde{\varphi} _j   \left(  f(\boldsymbol{\Phi}[{\bm z}]_1)   - f(Q)-f' (Q)  \widetilde{\boldsymbol{\Phi}}[{\bm z}]_1  \right) \| _{L^2  }
% \end{align*}
% for appropriate Paley--Littlewood decompositions with $ \supp  \varphi _j $ and $ \supp  \widetilde{\varphi} _j $ in $|x|\sim 2 ^{j}$ for $j\ge 1$ and   $|x|\lesssim 1$ for $j=0$.
% Next we split $  \|  \mathbf{E} _2 \| _{L^2  } \le A _{2l}+ A _{2h}$, with $A _{2l}$ involving sum only for
%   $j\le  \log (10A)$   and  $A _{2l}$ involving sum only for
%   $j>    \log (10A)$.   Now, using the same argument in  Lemma \ref{lem:2v1}, we have
%    \begin{align*}
%     A _{2l} \lesssim |{\bm z}|^2
%   \end{align*}
%  Next, using also \eqref{eq:Ijest} and  \eqref{eq:Ijest2}
%  \begin{align*}
%     A _{2h}&  \lesssim  \int _{\R} d\xi   | \widehat{\chi}(\xi) |\sum _{j> \log (10A) }
%     2^{ -\frac{\kappa}{2} j} \|         \mathcal{T}   e ^{\im \xi \sqcup }    \widetilde{\varphi} _j   \left(  f(\boldsymbol{\Phi}[{\bm z}]_1)   - f(Q)-f' (Q)  \widetilde{\boldsymbol{\Phi}}[{\bm z}]_1  \right) \| _{L^2  } \\& \lesssim  \varepsilon ^{-\pare{N+1}} A ^{-1}   \|        f(\boldsymbol{\Phi}[{\bm z}]_1)   - f(Q)-f' (Q)  \widetilde{\boldsymbol{\Phi}}[{\bm z}]_1   \| _{L^2  } \lesssim \varepsilon ^{-\pare{N+1}} A ^{-1} |{\bm z}|^2.
%   \end{align*}
%   Summing up
%   \begin{align*}&  | E_{52}|     \le      \|  \mathbf{E}  \| _{L^2  }
%    \| e^{-\frac{\kappa}{2} \<x\>}v_1         \| _{L^2}  \delta _2^{-1}|{\bm z}|^4 + \delta _2 \| e^{-\frac{\kappa}{2} \<x\>}v_1 \| _{L^2 }^2
% \end{align*}
%Collecting the estimates, we have the conclusion.
%\end{proof}
%
%Combining Lemmas \ref{lem:2v1} and \ref{lem:2v2}, we have
%\begin{lemma}\label{lem:2v3}
%	For any $\mu>0$, we have
%	\begin{align*}
%		&\int_0^T\left(\| \sech \left( \frac{\kappa}{2}x \right)v_1'\|_{L^2}^2+\|\sech \left( \frac{\kappa}{2}x \right){\bm v}\|_{L^2}^2\right)
%\lesssim  \delta   + o_\varepsilon (1)\int_0^T\|\boldsymbol{\eta}\|_A^2+ \|{\bm z} \|_{L^4(0,T)}^4.
%	\end{align*}
%\end{lemma}
%\begin{proof}
%	The claim follows from Lemmas  \ref{lem:2v1} and \ref{lem:2v2} and
%	\begin{align*}
%		|\mathcal{I}_{\mathrm{2nd},1}|\lesssim B\varepsilon^{-\pare{N+1}}\delta^2,\
%		|\mathcal{I}_{\mathrm{2nd},2}|\lesssim \varepsilon^{-\pare{N+1}}\delta^2.
%	\end{align*}
%\end{proof}
%
%\begin{proof}[Proof of Proposition \ref{prop:2ndvirial}]
%	It  is a consequence of Lemma \ref{lem:2v3}     and inequality \eqref{eq:key1}.
%\end{proof}




\section{Proof of Proposition \ref{prop:2ndvirial}%: virial estimates for the transformed equation
} \label{proof:2dnv}

Using the $\mathcal{T}$  in  \eqref{def:Tg} with $ N=3 $ for $ p\in\pare{\frac{5}{3}, 2} $ and $ N=2 $ for $ p=2 $, we consider the new   variable
\begin{align}   \label{def:vBg}
{\bm v}\defeq\mathcal{T}\boldsymbol{\eta}.
\end{align}
Here   ${\bm v}$ is even for $5/3<p<2$ but, crucially,  is odd for $p=2$.
For ${\bf L}_{ N+1}$, cfr. \cref{eq:linearNLKG_2},    ${\bm v}$ satisfies
\begin{multline}
\label{eq:vBg}
\dot{{\bm v}}=
-\mathcal{T}D\boldsymbol{\Phi}[{\bm z}](\dot{{\bm z}}-\tilde{{\bm z}})
+\mathbf{J}\left({\bf L}_{ N+1}{\bm v}
+\begin{pmatrix}
\comm{\< \ii\varepsilon \partial_x\>^{-\pare{N+1}}}{V_{N+1}} & 0 \\ 0 & 0
\end{pmatrix}
\mathcal{A}^*\boldsymbol{\eta}\right)\\
%--------------------------------------------------------
+        \mathcal{T} \left  [\left(  f'(\boldsymbol{\Phi}[{\bm z}]_1) - f'(Q) \right) \eta _1 \mathbf{i}  + \left( f\left( \boldsymbol{\Phi}[{\bm z}]_1 +\eta _1 \right)-f(\boldsymbol{\Phi}[{\bm z}]_1)- f'(\boldsymbol{\Phi}[{\bm z}]_1)\eta _1  \right) \mathbf{i}
 -  {\bm R}[{\bm z}]\right ]   .
\end{multline}
From Lemma \ref{lem:coer6A}, we have
\begin{align}
\|      \sech ( \kappa x)\boldsymbol{\eta}\|_{L^2}\lesssim \norm{ \sech \pare{  \frac{\kappa}{2} \  x }{\bm v}}_{L^2}.\label{eq:key1}
\end{align}
%Therefore, to bound the local norm of $\boldsymbol{\eta}$, it suffices to bound a local norm of ${\bm v}$.
Set, for the $\varphi_B$ defined in \eqref{def:zetaphi},
\begin{align*}
\psi_{A,B}=\chi_A^2 \varphi_B,
&&
  \tilde{S}_{A,B}=\frac{1}{2}\psi_{A,B}'+\psi_{A,B}\partial_x,
\end{align*}
and consider the functionals
\begin{align*}
\mathcal{I}_{\mathrm{2nd},1}\defeq\frac{1}{2}\< \mathbf{J} {\bm v},\tilde{S}_{A,B}{\bm v}\> ,
&&
 \mathcal{I}_{\mathrm{2nd},2}\defeq\frac{1}{2}\< \mathbf{J}  {\bm v},\sigma_3 e^{-\kappa \<x\>}{\bm v}\> .
\end{align*}

\begin{lemma}\label{lem:2v1}
	We have
\begin{align} &
\|v_1'\|_{L^2_{-\frac{\kappa}{2}}}^2+\|v_1\|_{L^2_{-\frac{\kappa}{2}}}^2+\dot{\mathcal{I}}_{\mathrm{2nd},1}
\lesssim   \left(\varepsilon^{-\pare{N+1}}A^2\delta+A^{-1/2}\right)\|\boldsymbol{\eta}\|  _{\boldsymbol{ \Sigma }_A}^2
+  |{\bm z} |^4.
\label{eq:lem:2v1} \end{align}
\end{lemma}

\begin{proof}
By \cref{eq:vBg}, we have
\begin{align*}
\dot{\mathcal{I}}_{\mathrm{2nd},1}= & \  -\< \mathcal{T}D\boldsymbol{\phi}[{\bm z}](\dot{{\bm z}}-\tilde{{\bm z}}),\tilde{S}_{A,B}{\bm v}\> -
\<{\bf L}_{ N+1}{\bm v},\tilde{S}_{A,B}{\bm v}\> \\& +\<\begin{pmatrix}
\comm{\< \ii\varepsilon \partial_x\>^{-\pare{N+1}}}{V_4} & 0 \\ 0 & 0
\end{pmatrix}\mathcal{A}^*\boldsymbol{\eta},\tilde{S}_{A,B}{\bm v}\>
  \\&+\<\mathcal{T} \left( \left(  f'(\boldsymbol{\Phi}[{\bm z}]_1) - f'(Q) \right) \eta _1   + \left( f\left( \boldsymbol{\Phi}[{\bm z}]_1 +\eta _1 \right)-f(\boldsymbol{\Phi}[{\bm z}]_1)- f'(\boldsymbol{\Phi}[{\bm z}]_1)\eta _1  \right)   \right)
 ,\tilde{S}_{A,B} {v}_1\>  \\& -\<\mathcal{T}  {\bm R}[{\bm z}] ,\tilde{S}_{A,B}{\bm v}\>
=:   \ D_1+D_2+D_3+D_4+D_5.
\end{align*}
 Following \cite[Section 5]{KM22},  for the main term $D_2$  we have
\begin{align}\label{def:xi}
D_2=\<L_{N+1}v_1,\widetilde{S}_{A,B}v_1\>=-\int \left(  \xi _1 ^{\prime  2}+V_B \xi _1 \right)\,dx+D_{21}  , && \xi _1\defeq \chi_A\zeta_B v_1,
\end{align}
  where
\begin{align}\label{eq:defvb}&
 V_B = \frac{1}{2} B^{-1}\left(\chi'' |x|+2\chi' \frac{x}{|x|}\right)  -\frac{1}{2}\  \frac{\varphi _B}{\zeta _B^2}V'_{N+1} \text{ and}\\&
D_{21}=\frac{1}{4}\int (\chi_A^2)'(\zeta_B^2)'v_1^2+\frac{1}{2}\int \left(3(\chi_A')^2+\chi_A''\chi_A\right)\zeta_B^2v_1^2-\int (\chi_A^2)'\varphi_B(v_1')^2+\frac{1}{4}\int (\chi_A^2)''' \varphi_B v_1^2. \nonumber
\end{align}
We claim that
\begin{align}\label{eq:KMlemma3}
\int ( \xi _1 ^{\prime  2}+V_B \xi _1 )\,dx  +A^{-1}\|\boldsymbol{\eta}\|_{\boldsymbol{ \Sigma }_A}^2\gtrsim \left(\norm{\sech \left( \frac{\kappa}{2} x\right) v_1'}_{L^2}^2+\norm{\sech \left( \frac{\kappa}{2} x\right) v_1}^2\right).
\end{align}
The proof is like in \cite[Lemma 3]{KM22}.  We have, by $0<\zeta  _B\le 1$,
\begin{align*} & \int _{|x|\le A}\sech \left(  \kappa x\right)v_1^2\le  \int _{|x|\le A}\sech \left(  \frac{\kappa}{2} x\right) \zeta ^2_B   v_1^2\le  \int _{|x|\le A}\sech \left(  \frac{\kappa}{2} x\right) \xi_1^2.
\end{align*}
We have
\begin{align*} & \int _{|x|\le A}\sech \left(  \kappa x\right)v_1 ^{\prime 2}\le  \int _{|x|\le A}\sech \left(  \frac{\kappa}{2} x\right) \left( \xi _1' - \zeta '_Bv_1\right) ^2
\lesssim  \int _{|x|\le A}\sech \left(  \frac{\kappa}{2} x\right)   (  \xi _1 ^{\prime 2} + \xi _1 ^{  2}    )    .
\end{align*}
We have, thanks to \eqref{eq:KM1},
\begin{multline*}
 \int _{|x|\ge A}\sech \left(  \kappa x\right) \left(  v_1 ^{\prime 2}+     v_1^2\right) \le  \sech \left(  \frac{\kappa}{2} A\right)\int _{\R }\sech \left( \frac{8}{A} x\right) \left(  v_1 ^{\prime 2}+     v_1^2\right) dx \\
%----------------------------------------
   \lesssim \sech \left(  \frac{\kappa}{2} A\right) \varepsilon ^{-\pare{N+1}}\int _{\R }\sech \left( \frac{4}{A} x\right) \left(  \eta_1 ^{\prime 2}+     \eta_1^2\right) dx\le A ^{-1}\|\boldsymbol{\eta}\|_{\boldsymbol{ \Sigma }_A}^2.
\end{multline*}
Finally,  we claim the following, which  completes the proof of \eqref {eq:KMlemma3},
\begin{align} \label{eq:coercVB} & \int _{\R } \sech \left(  \frac{\kappa}{2} x\right)   (  \xi _1 ^{\prime 2} + \xi _1 ^{  2}    )  \lesssim  \int _{\R }( \xi _1 ^{\prime  2}+V_B \xi _1^{  2}   )\,dx .
\end{align}
In the case $5/3<p<2$, the above inequality  is true for all $\xi _1\in H ^{1}    \left( \R \right)$ and
follows easily from the fact  for $B$ large the potential $V_B$ is a small perturbation of the  positive potential  $- \  \frac{\varphi _B}{\zeta _B^2}V'_{N+1}
  >0$ for $x\neq 0$. In the case $p=2$, then  $V _{N+1}=V_3=0$, and so $V_B$ is a small potential.  Then for $B$ large, the coercivity in \eqref{eq:coercVB} is true
 only for $\xi _1\in H ^{1} _{\odd}  \left( \R \right)$.    Fortunately, from \eqref{def:xi} we see that $\xi _1$ has the same parity of $v_1$ and, as we remarked right under  \eqref{def:vBg},   $v_1$ is odd if  $p=2$.


\noindent We next have the following, which is \cite[Lemma 4]{KM22}  (see also \cite{CMS2023}),
\begin{align}\label{eq:KMlemma4}
|D_{21}|\lesssim A^{-1/2}\left(\|\boldsymbol{\eta}\|    _{\boldsymbol{ \Sigma }_A}^2+\|\sech \left( \kappa x\right)\eta_1\|_{L^2}^2\right) \lesssim A^{-1/2}\left(\|\boldsymbol{\eta}\|    _{\boldsymbol{ \Sigma }_A}^2+   \varepsilon ^{-\pare{N+1}}\norm{\sech \left( \frac{\kappa}{2} x\right)\eta_1}_{L^2}^2\right) .
\end{align}
  By Lemma \ref{lem:modbound}, $ \av{\tilde{S}_{A,B}^\ast \cT D_\bmz\bmPhi\bra{\bmz}} \sim e^{-\kappa\av{x}} $ and \eqref{eq:KM1} we have that
  \begin{equation}
  \label{eq:D1}
  \begin{aligned}
|D_1|&\lesssim |\dot{{\bm z}}-\tilde{{\bm z}}| \|  \sech \left( 2\kappa x\right)    {\bm v}\|_{L^2}\\
& \lesssim \delta ^{p-1} \|\sech \left(  \kappa x\right) {\eta}_1\|_{L^2} \|\sech \left( 2\kappa x\right) {\bm v}\|_{L^2}   \lesssim \delta ^{p-1} \varepsilon ^{-\pare{N+1}} \|\sech \left(  \kappa x\right) \boldsymbol{\eta} \|_{L^2}^2 .
\end{aligned}
  \end{equation}
Like in \cite{CMS2023},  we have (obviously here  we care only of case $2>p>5/3$ where $V_{N+1}\neq 0$)
\begin{equation}
\label{eq:D3}
\begin{aligned}
|D_3|  & \ =\av{\< \comm{\<\im \varepsilon \partial_x\>^{-\pare{N+1}}}{V_4} \mathcal{A}^*\eta _1, \widetilde{S} _{A,B}v_1\>}  \\&
%------------------------------------------------
\le \norm{ \cosh \left( \frac{\kappa}{2} x\right)\comm{\<\im \varepsilon \partial_x\>^{-\pare{N+1}}}{V_4} \mathcal{A}^*\eta _1 } _{L^2} \norm{ \sech \left( \frac{\kappa}{2} x\right) \widetilde{S} _{A,B}v_1} _{L^2}
\\&
\leq   \varepsilon   \norm{ \sech \left( \frac{\kappa}{2} x\right)v_1}_{L^2}  \left(\norm{\sech \left( \frac{\kappa}{2} x\right) v_1'}_{L^2}+\norm{\sech \left( \frac{\kappa}{2} x\right) v_1}_{L^2}\right)\\&
\lesssim\varepsilon \left(\norm{ \sech \left( \frac{\kappa}{2} x\right)v_1'}_{L^2}^2+\norm{\sech \left( \frac{\kappa}{2} x\right) v_1}_{L^2}^2\right) ,
\end{aligned}
\end{equation}
where the upper bound can be absorbed inside the left hand side of \eqref{eq:lem:2v1}.


 \noindent   We now write $D_4= D _{41}+ D _{42}$, with \footnote{Notice that in the estimate of the term $D_4$ in \cite[Lemma 8.1]{CMS2023} the operator $\mathcal{T}$ is omitted by mistake. The proof can be corrected proceeding like here.}
 \begin{align*}&
D _{41} \defeq\<\mathcal{T}   \left(  f'(\boldsymbol{\Phi}[{\bm z}]_1) - f'(Q) \right) \eta _1
 ,\tilde{S}_{A,B} {v}_1\>    \\& D _{42} \defeq\<\mathcal{T}   \left(  \left( f\left( \boldsymbol{\Phi}[{\bm z}]_1 +\eta _1 \right)-f(\boldsymbol{\Phi}[{\bm z}]_1)- f'(\boldsymbol{\Phi}[{\bm z}]_1)\eta _1  \right)   \right)
 ,\tilde{S}_{A,B} {v}_1\>  .
\end{align*}
 By  $\sech \left( \frac{4}{A}  x \right)\sim 1$ on               $\supp \psi_{A,B}\subseteq [-2A,2A] $,  by \cref{eq:estder1,eq:estder2}
 and by \cref{eq:KM1,eq:KM2},
 \begin{equation}
 \label{eq:D41}
 \begin{aligned}
   |D _{41}|& \le \norm{    \sech \left( \frac{4}{A}  x \right) \mathcal{T} \left(  f'(\boldsymbol{\Phi}[{\bm z}]_1) - f'(Q) \right) \eta _1 } _{L^2}
    \left(  \norm{\sech \left(  \frac{4}{A} x \right) v_1}_{L^2}
+ \norm{\sech \left(  \frac{4}{A} x \right)v_1'}_{L^2}\right)  \\&  \lesssim \varepsilon ^{-2\pare{N+1}}  \norm{    \left(  f'(\boldsymbol{\Phi}[{\bm z}]_1) - f'(Q) \right) \eta _1 } _{L^2} \left(  \norm{\sech \left(  \frac{2}{A} x \right) \eta_1}_{L^2}
+ \norm{\sech \left(  \frac{2}{A} x \right)\eta_1'}_{L^2}\right)
\\& \lesssim   \varepsilon ^{-2\pare{N+1}}A |{\bm z}|   \norm{    e^{-\frac{p-1}{2}|x|} \eta _1 }_{L^2} \norm{\boldsymbol{\eta}}_{\boldsymbol{ \Sigma }_A}
 \lesssim \varepsilon ^{-2\pare{N+1}}A ^2 \delta   \norm{\boldsymbol{\eta}}_{\boldsymbol{ \Sigma }_A}^2 .
\end{aligned}
\end{equation}
We have similarly
   \begin{align*}
   |D _{42}|& \le  \varepsilon ^{-2\pare{N+1}}A \norm{   \sech \left( \frac{2}{A}  x \right) \left( f\left( \boldsymbol{\Phi}[{\bm z}]_1 +\eta _1 \right)-f(\boldsymbol{\Phi}[{\bm z}]_1)- f'(\boldsymbol{\Phi}[{\bm z}]_1)\eta _1  \right)  }_{L^2}
     \norm{ \boldsymbol{\eta}}_{\boldsymbol{ \Sigma }_A}
\end{align*}
% In the region with $\left |  \boldsymbol{\Phi}[{\bm z}]_1  \right | \gg |\eta _1| $ we have
% \begin{align*}
%   \left | f\left( \boldsymbol{\Phi}[{\bm z}]_1 +\eta _1 \right)-f(\boldsymbol{\Phi}[{\bm z}]_1)- f'(\boldsymbol{\Phi}[{\bm z}]_1)\eta _1  \boldsymbol{\Phi}[{\bm z}]_1  \right | \lesssim \left |  \boldsymbol{\Phi}[{\bm z}]_1  \right |  ^{p-2} |\eta _1| ^2\le  |\eta _1| ^p
% \end{align*}
%and in the region $\left |  \boldsymbol{\Phi}[{\bm z}]_1  \right | \lesssim |\eta _1| $ we have
% \begin{align*}
%   \left | f\left( \boldsymbol{\Phi}[{\bm z}]_1 +\eta _1 \right)-f(\boldsymbol{\Phi}[{\bm z}]_1)- f'(\boldsymbol{\Phi}[{\bm z}]_1)\eta _1  \boldsymbol{\Phi}[{\bm z}]_1  \right | \lesssim \left |  f\left(  \eta _1 \right)  \right |  = |\eta _1| ^p.
% \end{align*}
% So, in all cases
By the estimates in \cref{eq:buuuuh1,eq:buuuuh2}
  \begin{align}\label{eq:D42}
   |D _{42}|& \lesssim   \varepsilon ^{-2\pare{N+1}}A^2 \|  \eta _1\| _{L^\infty}^{p-1}
     \|\boldsymbol{\eta}\|    _{\boldsymbol{ \Sigma }_A}^2 \lesssim \varepsilon ^{-2\pare{N+1}}A^2    \delta^{p-1}
     \|\boldsymbol{\eta}\|    _{\boldsymbol{ \Sigma }_A}^2.
\end{align}
Hence \cref{eq:D41,eq:D42} yield
\begin{equation}
\label{eq:D4}
\av{D_4} \lesssim \pare{ \frac{A}{\varepsilon^4} }^2    \delta^{p-1}
     \|\boldsymbol{\eta}\|    _{\boldsymbol{ \Sigma }_A}^2.
\end{equation}
Finally,  we consider
\begin{align*}
 D_5 =
 \< \mathcal{T} D_{\bm z}\boldsymbol{\Phi}[{\bm z}]\tilde{{\bm z}}_R ,\tilde{S}_{A,B}{\bm v}\> -\< \mathcal{T} \left(  f(\boldsymbol{\Phi}[{\bm z}]_1)   - f(Q)-f' (Q)  \widetilde{\boldsymbol{\Phi}}[{\bm z}]_1  \right) ,\tilde{S}_{A,B} {v}_1\> =:D_{51}+D_{52}.
\end{align*}
 We have
\begin{align*}
 D_{51} = & \  D_{511}+ D_{512} \\
%-----------------------------------------
   = & \  - \< \partial _x \mathcal{T} D_{\bm z}\boldsymbol{\Phi}[{\bm z}]\tilde{{\bm z}}_R , \psi_{A,B} {\bm v}\>   - \frac{1}{2} \< \mathcal{T}  D_{\bm z}\boldsymbol{\Phi}[{\bm z}]\tilde{{\bm z}}_R , \psi_{A,B}' {\bm v}\>
%   - \<        \< \im \varepsilon \partial _x \> ^{-\pare{N+1}} [\partial _x, \mathcal{A}^*]    D_{\bm z}\boldsymbol{\Phi}[{\bm z}]\tilde{{\bm z}}_R , \psi_{A,B} {\bm v}\>
    .
\end{align*}
 We have
\begin{align*}
 | D_{511}|     \le     \norm{ \cosh \left(  \kappa x \right) \partial _x \mathcal{T} D_{\bm z}\boldsymbol{\Phi}[{\bm z}]\tilde{{\bm z}}_R } _{L^2}
    \| \sech \left(  \kappa x \right) {\bm v} \| _{L^2}   .
 \end{align*}
 In \cite[Lemma 5.6]{CM2022}   it is shown  that there exist constants $C_0>0$ and $ \varepsilon _0>0$ such that for   $\varepsilon  \in (0, \varepsilon _0)$   and for $K_{\varepsilon}(x,y) \in \mathcal{D}'(\R \times \R)$   the Schwartz kernel of  $\mathcal{T} $, then  we have
\begin{align}  \label{eq:scker}
| K_{\varepsilon}(x,y)    | \le C_0  e^{-\frac{|x-y|}{3\varepsilon}}  \text{  for all $x,y$ with $|x-y|\ge 1$.}
\end{align}
 Using this fact, it  is elementary to show that $ \av{\partial _x \mathcal{T} D_{\bm z}\boldsymbol{\Phi}[{\bm z}]} \sim e^{-k_2\av{x}} $. Hence, if $ 0 <  \kappa \ll k_2 $.
 \begin{align*}
 \av{ D_{511} }\lesssim |\tilde{{\bm z}}_R|.
 \end{align*}
The same argument yields the same bounds for $D_{512}$ so
  we conclude that
 \begin{align}\label{eq:D51}
 \av{ D_{51}}      \lesssim    |\tilde{{\bm z}}_R| \
    \|  {\bm v} \| _{L^2_{-\kappa }}  \lesssim \delta _2^{-1}|{\bm z}|^4 + \delta _2 \|  {\bm v} \| _{L^2_{-\frac{\kappa}{2}}}^2.
 \end{align}
 { Turning to $D _{52}$, we have
 \begin{align*}&
 | D_{52}|     \le      \|  \mathbf{C}  \| _{L^2 (|x|\le 2A)}
    \| \sech \left( \kappa x \right)  \left(   2^{-1}\psi_{A,B}' v_1+\psi_{A,B}v_1'    \right)   \| _{L^2}   \text{  where } \\&\mathbf{C}
\defeq
  \cosh \left(  \kappa x \right) \mathcal{T} \left(  f(\boldsymbol{\Phi}[{\bm z}]_1)   - f(Q)-f' (Q)  \widetilde{\boldsymbol{\Phi}}[{\bm z}]_1  \right) .
 \end{align*}
 For a cutoff $\chi$ like in \Cref{notation:chi},  we write  $ \mathbf{C } =\mathbf{C }_{1}+\mathbf{C }_{2}$, with
 \begin{align*}
   & \mathbf{C }_{1}\defeq  \cosh \left( \kappa x \right) \mathcal{T}   \chi (x-\cdot ) \left(  f(\boldsymbol{\Phi}[{\bm z}]_1)   - f(Q)-f' (Q)  \widetilde{\boldsymbol{\Phi}}[{\bm z}]_1  \right) ,\\&  \mathbf{C }_{2}\defeq  \cosh \left( \kappa x \right) \mathcal{T}   \chi _1(x-\cdot ) \left(  f(\boldsymbol{\Phi}[{\bm z}]_1)   - f(Q)-f' (Q)  \widetilde{\boldsymbol{\Phi}}[{\bm z}]_1  \right) \text{ with }\chi _1:=1-\chi .
 \end{align*}
 For the $I$ in \eqref{eq:defI},  by    \eqref{eq:Ijest2} and using  \eqref{eq:scker},  we have
 \begin{align}\label{eq:estbfc2}
   \|  \mathbf{C} _2 \| _{L^2  } \lesssim \| \cosh \left(  \kappa x \right) \int   e^{-\frac{|x-y|}{3\varepsilon}} |I (y)| dy\| _{L^2  } \lesssim  |{\bm z}|^2\| \cosh \left(  \kappa x \right) \int   e^{-\frac{|x-y|}{3\varepsilon}} e^{-|y| } dy\| _{L^2  }  \lesssim  |{\bm z}|^2,
 \end{align}
 We bound
 \begin{align*}&
  \|  \mathbf{C} _2 \| _{L^2 (|x|\le 2A)  } \\&\lesssim  \int _{\R} d\xi   | \widehat{\chi}(\xi) |\sum _{j=0 }^{ \log (10A)}
     2^{ \kappa j} \|    \varphi _j    \mathcal{T}   e ^{\im \xi \sqcup }    \widetilde{\varphi} _j   \left(  f(\boldsymbol{\Phi}[{\bm z}]_1)   - f(Q)-f' (Q)  \widetilde{\boldsymbol{\Phi}}[{\bm z}]_1  \right) \| _{L^2 (|x|\le 2A)}
 \end{align*}
 for appropriate Paley--Littlewood decompositions  (in $x$ and $y$ space) with $ \supp  \varphi _j $ and $ \supp  \widetilde{\varphi} _j $ in $|x|\sim 2 ^{j}$ for $j\ge 1$ and   $|x|\lesssim 1$ for $j=0$.
  Since for $|y|\lesssim 2 ^ j$  with $j\le   \log (10A)$ we have
 \begin{align} \label{eq:estbfc2-}&
   f(\boldsymbol{\Phi}[{\bm z}]_1)   - f(Q)-f' (Q)  \widetilde{\boldsymbol{\Phi}}[{\bm z}]_1  = 2^{-1} f'' (Q)  \left( \widetilde{\boldsymbol{\Phi}}[{\bm z}]_1\right) ^2 +  {E} \text{  with}\\& |E|\lesssim \left | f''' (Q)  \left( \widetilde{\boldsymbol{\Phi}}[{\bm z}]_1\right) ^3 \right | \lesssim |{\bm z}|^3 e^{-\frac{3-p}{2} |x|  } ,\nonumber
 \end{align}
 we accordingly split  $\mathbf{C} _2= \mathbf{C} _{21}+\mathbf{C} _{22} $.  With the function   $ \( e ^{\im \xi \sqcup }\) (x):=  e ^{\im \xi x }$, we have
 \begin{align*}&
  \|  \mathbf{C} _{21} \| _{L^2 (|x|\le 2A)  } \lesssim    \int _{\R} d\xi   | \widehat{\chi}(\xi) |\sum _{j=0 }^{ \log (10A)}
     2^{ \kappa j} \|   \< \im \varepsilon \partial _x \> ^{-\pare{N+1}}    \mathcal{A }^*    e ^{\im \xi \sqcup }    \widetilde{\varphi} _j   f'' (Q)  \left( \widetilde{\boldsymbol{\Phi}}[{\bm z}]_1\right) ^2 \| _{L^2 (|x|\le 2A)}\\& \lesssim  \|  \< \xi \>  ^{N+1}  \widehat{\chi}  \|_{L^1\left(\R _\xi \right)}
     \sum _{j=0 }^{ \log (10A)} 2^{ \kappa j} \|  \widetilde{\varphi} _j   f'' (Q)  \left( \widetilde{\boldsymbol{\Phi}}[{\bm z}]_1\right) ^2 \| _{H^{N+1}(\R )}  \lesssim |{\bm z}|^2.
 \end{align*}
 We have
 \begin{align*}&
  \|  \mathbf{C} _{22} \| _{L^2 (|x|\le 2A)  } \lesssim   \int _{\R} d\xi   | \widehat{\chi}(\xi) |\sum _{j=0 }^{ \log (10A)}
     2^{\kappa  j} \|   \< \im \varepsilon \partial _x \> ^{-\pare{N+1}}    \mathcal{A }^*    e ^{\im \xi \sqcup }    \widetilde{\varphi} _j  E  \| _{L^2 (|x|\le 2A)}\\& \lesssim  \|     \widehat{\chi}  \|_{L^1 \left(\R _\xi \right)} \varepsilon^{-\pare{N+1}}
     \sum _{j=0 }^{ \log (10A)} 2^{  \kappa  j} \| E  \| _{L^2(\R )}  \lesssim  A \varepsilon^{-\pare{N+1}}  |{\bm z}| ^3 \lesssim  A \varepsilon^{-\pare{N+1}} \delta  |{\bm z}| ^2.
 \end{align*}
 Summing up, we have
\begin{align}\label{eq:D52}
   \av{ D_{52}}  \lesssim \delta _2^{-1}|{\bm z}|^4 + \delta _2 \|  {\bm v} \| _{L^2_{-\frac{\kappa}{2}}}^2 +   \delta _2 \norm{ \sech \left(  \frac{\kappa}{2} x\right)   {v}'_1 } _{L^2  } ^2
\end{align}}
\Cref{eq:D51,eq:D52} give that
\begin{equation}\label{eq:D5}
\av{ D_{5}}  \lesssim \frac{\av{\bm z}^4}{\delta _2} + \delta _2 \|  {\bm v} \| _{L^2_{-\frac{\kappa}{2}}}^2 +   \delta _2 \norm{ \sech \left(  \frac{\kappa}{2} x\right)   {v}'_1 } _{L^2  } ^2
\end{equation}
Collecting   \cref{eq:D1,eq:D3,eq:D4,eq:D5,eq:KMlemma4,eq:KMlemma3,def:xi} and bootstrapping, we deduce   \eqref{eq:lem:2v1}.
\end{proof}

\begin{lemma}\label{lem:2v2}
	We have
	\begin{align}\label{eq:lem:2v2}
		\|e^{-\kappa\<x\>/2}v_2\|_{L^2} + \dot{\mathcal{I}}_{\mathrm{2nd},2}\lesssim& \|e^{-\kappa \<x\>/2}v_1'\|_{L^2}^2+
\|e^{-\kappa \<x\>/2}v_1\|_{L^2}^2+|{\bm z}|^4+
   o_{\varepsilon}(1)
   \| \boldsymbol{\eta} \|_{ \boldsymbol{ \Sigma }_A}^2      .%+\varepsilon^{-1}A^2\delta\|e^{-\kappa\<x\>}\eta_1\|_{L^2}
	\end{align}
\end{lemma}

\begin{proof}
Differentiating $\mathcal{I}_{\mathrm{2nd},2}$, we have
\begin{align*}&
\dot{\mathcal{I}}_{\mathrm{2nd},2}=    -\< \mathcal{T}D\boldsymbol{\phi}[{\bm z}](\dot{{\bm z}}-\tilde{{\bm z}}),\sigma_3 e^{-\kappa \<x\>}{\bm v}\> \\& -
\<{\bf L}_{ N+1}{\bm v},\tilde{S}_{A,B}{\bm v}\> +\<\begin{pmatrix}
\comm{\< \ii\varepsilon \partial_x\>^{-\pare{N+1}}}{V_4} & 0 \\ 0 & 0
\end{pmatrix}\mathcal{A}^*\boldsymbol{\eta},\sigma_3 e^{-\kappa \<x\>}{\bm v}\>
  \\&+\<\mathcal{T} \left( \left(  f'(\boldsymbol{\Phi}[{\bm z}]_1) - f'(Q) \right) \eta _1   + \left( f\left( \boldsymbol{\Phi}[{\bm z}]_1 +\eta _1 \right)-f(\boldsymbol{\Phi}[{\bm z}]_1)- f'(\boldsymbol{\Phi}[{\bm z}]_1)\eta _1  \right)   \right)
 ,  e^{-\kappa \<x\>}  {v}_1\>  \\& -\<\mathcal{T}  {\bm R}[{\bm z}] ,\sigma_3 e^{-\kappa \<x\>}{\bm v}\>
=:E_1+E_2+E_3+E_4+E_5.
\end{align*}
Like in \cite{CMS2023} we have
\begin{align*}&
	E_2=-\|e^{-\kappa\<x\>/2}v_2\|_{L^2}^2+\<L_{N+1}v_1, e^{-\kappa\<x\>}v_1\>=-\|e^{-\kappa\<x\>/2}v_2\|_{L^2}^2+E_{21}, \text{  with}\\&
	|E_{21}|\lesssim  \|e^{-\kappa \<x\>/2}v_1'\|_{L^2}^2+\|e^{-\kappa \<x\>/2}v_1\|_{L^2}^2.
\end{align*}
By Lemma \ref{lem:modbound}, we have
\begin{align*}
	|E_1|\lesssim \delta ^{p-1}\|e^{-\kappa\<x\>/2}{\bm v}\|_{L^2}  \|e^{-\kappa\<x\>}\boldsymbol{\eta}\|_{L^2}
 \lesssim \delta^{p-1} \varepsilon ^{-N}\|e^{-\kappa\<x\>/2}{\bm v}\|_{L^2}^2.
\end{align*}
By \eqref{eq:KM3b}, we have
\begin{align*}
	&|E_3|  = \av{\<  [\<\im\varepsilon\partial_x\>^{-N},V_4]\mathcal{A}^*\eta_1,\sigma_3e^{-\kappa\<x\>}v_1\> } \lesssim
 \varepsilon \|e^{-\frac{\kappa}{2}\<x\> }v_1\|_{L^2} \|e^{-\kappa\<x\> }v_1\|_{L^2}    \le  \varepsilon \|e^{-\frac{\kappa}{2}\<x\> }v_1\|_{L^2}^2.
\end{align*}
Proceeding like in Lemma \ref{lem:2v1}, we write
$E_4= E _{41}+ E _{42}$, with \footnote{Also in the estimate of the term $E_4$ in \cite[Lemma 8.2]{CMS2023}, like in the estimate of
$D_4$ in \cite[Lemma 8.1]{CMS2023},
the operator $\mathcal{T}$ has been omitted by mistake. The proof can be corrected proceeding like here.}
 \begin{align*}&
E _{41} \defeq\<\mathcal{T}   \left(  f'(\boldsymbol{\Phi}[{\bm z}]_1) - f'(Q) \right) \eta _1
 ,e^{-\kappa \<x\>} {v}_1\>    \\& E _{42} \defeq\<\mathcal{T}   \left(  \left( f\left( \boldsymbol{\Phi}[{\bm z}]_1 +\eta _1 \right)-f(\boldsymbol{\Phi}[{\bm z}]_1)- f'(\boldsymbol{\Phi}[{\bm z}]_1)\eta _1  \right)   \right)
 ,e^{-\kappa \<x\>} {v}_1\>  .
\end{align*}
 By  an  analogue of \eqref{eq:key1}
 and by \eqref{eq:KM1}--\eqref{eq:KM2},
 \begin{align*}
   |E _{41}|& \le \varepsilon ^{- N-1}  \|      \left(  f'(\boldsymbol{\Phi}[{\bm z}]_1) - f'(Q) \right) \eta _1 \| _{L^2}
       \|e^{-\kappa \<x\>} {v}_1\|_{L^2}   \\&   \lesssim   \varepsilon ^{-\pare{N+1}}A |{\bm z}|   \|    e^{-\frac{p-1}{2}|x|} \eta _1 \| _{L^2} \|e^{-\kappa \<x\>} {v}_1\|_{L^2}
 \lesssim   \varepsilon ^{-N}A \delta   \|    e^{-\frac{p-1}{4}|x|} v _1 \| _{L^2} \|e^{-\kappa \<x\>} {v}_1\|_{L^2}  \\&   \lesssim     \varepsilon ^{- N-1}A  \delta  \|e^{-\kappa \<x\>} {v}_1\|_{L^2} ^2 .
\end{align*}
In analogy to estimates for  $D_{41} $ and $D_{42} $ in Lemma \ref{lem:2v1}, but using $\sech \left( \frac{4}{A}  x \right)e^{-\frac{\kappa }{2} \<x\>}\lesssim e^{-\kappa \<x\>}$ along with
  \eqref{eq:KM1}--\eqref{eq:KM2},
 \begin{align*}
   |E _{42}|& \le \|   \sech \left( \frac{4}{A}  x \right) \mathcal{T} \left(   f\left( \boldsymbol{\Phi}[{\bm z}]_1 +\eta _1 \right)-f(\boldsymbol{\Phi}[{\bm z}]_1)- f'(\boldsymbol{\Phi}[{\bm z}]_1)\eta _1 \right)  \| _{L^2}
      \|  e^{-\frac{\kappa }{2} \<x\>} v_1\|_{L^2} \\&
        \lesssim   \varepsilon ^{-2\pare{N+1}}A  \|   \sech \left( \frac{2}{A}  x \right) \left( f\left( \boldsymbol{\Phi}[{\bm z}]_1 +\eta _1 \right)-f(\boldsymbol{\Phi}[{\bm z}]_1)- f'(\boldsymbol{\Phi}[{\bm z}]_1)\eta _1  \right)  \| _{L^2} \|e^{-\kappa \<x\>} {v}_1\|_{L^2}
 \\& \lesssim   \varepsilon ^{-2\pare{N+1}}A \delta ^{p-1}  \|\boldsymbol{\eta}\|    _{\boldsymbol{ \Sigma }_A} \|  e^{-\frac{\kappa }{2} \<x\>} v_1\|_{L^2}    \le      \varepsilon ^{-2\pare{N+1}}A \delta ^{p-1} \left(  \|\boldsymbol{\eta}\|    _{\boldsymbol{ \Sigma }_A} ^2+  \|  e^{-\frac{\kappa }{2} \<x\>} v_1\|_{L^2} ^2    \right).
\end{align*}
 Finally,  we consider
\begin{align*}
 E _5 =
 \< \mathcal{T} D_{\bm z}\boldsymbol{\Phi}[{\bm z}]\tilde{{\bm z}}_R ,\sigma_3 e^{-\kappa \<x\>}{\bm v}\> -\< \mathcal{T} \left(  f(\boldsymbol{\Phi}[{\bm z}]_1)   - f(Q)-f' (Q)  \widetilde{\boldsymbol{\Phi}}[{\bm z}]_1  \right) ,  e^{-\kappa \<x\>}  {v}_1\> =:E_{51}+E_{52}.
\end{align*}
We have
The last three terms can be bounded similarly. We have
\begin{align*}
 | E_{51}|     \le     \|  \mathcal{T} D_{\bm z}\boldsymbol{\Phi}[{\bm z}]\tilde{{\bm z}}_R \| _{L^2}
    \|e^{-\kappa \<x\>} {\bm v} \| _{L^2}  \lesssim  \|   \mathcal{A}^* D_{\bm z}\boldsymbol{\Phi}[{\bm z}]\tilde{{\bm z}}_R \| _{L^2}
    \|e^{-\kappa \<x\>} {\bm v} \| _{L^2} \lesssim    \delta _2^{-1}|{\bm z}|^4 + \delta _2 \| e^{-\frac{\kappa}{2} \<x\>} {\bm v} \| _{L^2 }^2.
 \end{align*}
 Turning to $E _{52}$, we have
 \begin{align*}&
 | E_{52}|     \le      \|  \mathbf{E}  \| _{L^2  }
    \| e^{-\frac{\kappa}{2} \<x\>}v_1         \| _{L^2}   \text{  where } \\&\mathbf{E}
\defeq
  e^{-\frac{\kappa}{2} \<x\>} \mathcal{T} \left(  f(\boldsymbol{\Phi}[{\bm z}]_1)   - f(Q)-f' (Q)  \widetilde{\boldsymbol{\Phi}}[{\bm z}]_1  \right) .
 \end{align*}
 Let us write  $ \mathbf{E } =\mathbf{E }_{1}+\mathbf{E}_{2}$ with
 \begin{align*}
   & \mathbf{E}_{1}\defeq  e^{-\frac{\kappa}{2} \<x\>}\mathcal{T}   \chi (x-\cdot ) \left(  f(\boldsymbol{\Phi}[{\bm z}]_1)   - f(Q)-f' (Q)  \widetilde{\boldsymbol{\Phi}}[{\bm z}]_1  \right) ,\\&  \mathbf{E }_{2}\defeq  e^{-\frac{\kappa}{2} \<x\>} \mathcal{T}   \chi _1(x-\cdot ) \left(  f(\boldsymbol{\Phi}[{\bm z}]_1)   - f(Q)-f' (Q)  \widetilde{\boldsymbol{\Phi}}[{\bm z}]_1  \right) \text{ with }\chi _1=1-\chi .
 \end{align*}
 Like for \eqref{eq:estbfc2} and using \eqref{eq:Ijest2},   we have
 \begin{align*}
   \|  \mathbf{E} _2 \| _{L^2  } \lesssim \|  e^{-\frac{\kappa}{2} \<x\>} \int   e^{-\frac{|x-y|}{3\varepsilon}} |I (y)|dy\| _{L^2  } \lesssim  |{\bm z}|^2\|  e^{-\frac{\kappa}{2} \<x\>} \int   e^{-\frac{|x-y|}{3\varepsilon}} e^{-|y| } dy\| _{L^2  }  \lesssim  |{\bm z}|^2.
 \end{align*}
 In analogy to the bound of $\mathbf{C} _2$   in Lemma \ref{lem:2v1}, we bound
 \begin{align*}&
  \|  \mathbf{E} _2 \| _{L^2 } \lesssim  \int _{\R} d\xi   | \widehat{\chi}(\xi) |\sum _{j=0 }^{ \infty}
     2^{ -\frac{\kappa}{2} j} \|    \varphi _j    \mathcal{T}   e ^{\im \xi \sqcup }    \widetilde{\varphi} _j   \left(  f(\boldsymbol{\Phi}[{\bm z}]_1)   - f(Q)-f' (Q)  \widetilde{\boldsymbol{\Phi}}[{\bm z}]_1  \right) \| _{L^2  }
 \end{align*}
 for appropriate Paley--Littlewood decompositions with $ \supp  \varphi _j $ and $ \supp  \widetilde{\varphi} _j $ in $|x|\sim 2 ^{j}$ for $j\ge 1$ and   $|x|\lesssim 1$ for $j=0$.
 Next we split $  \|  \mathbf{E} _2 \| _{L^2  } \le A _{2l}+ A _{2h}$, with $A _{2l}$ involving the sum   for
   $j\le  \log (10A)$   and  $A _{2l}$ involving the sum   for
   $j>    \log (10A)$.   Now, using the same argument in  Lemma \ref{lem:2v1}, we have
    \begin{align*}
     A _{2l} \lesssim |{\bm z}|^2.
   \end{align*}
  Next, using also \eqref{eq:Ijest} and  \eqref{eq:Ijest2}
  \begin{align*}
     A _{2h}&  \lesssim  \int _{\R} d\xi   | \widehat{\chi}(\xi) |\sum _{j> \log (10A) }
     2^{ -\frac{\kappa}{2} j} \|         \mathcal{T}   e ^{\im \xi \sqcup }    \widetilde{\varphi} _j   \left(  f(\boldsymbol{\Phi}[{\bm z}]_1)   - f(Q)-f' (Q)  \widetilde{\boldsymbol{\Phi}}[{\bm z}]_1  \right) \| _{L^2  } \\& \lesssim  \varepsilon ^{-\pare{N+1}} A ^{-1}   \|        f(\boldsymbol{\Phi}[{\bm z}]_1)   - f(Q)-f' (Q)  \widetilde{\boldsymbol{\Phi}}[{\bm z}]_1   \| _{L^2  } \lesssim \varepsilon ^{-\pare{N+1}} A ^{-1} |{\bm z}|^2.
   \end{align*}
   Summing up
   \begin{align*}&  | E_{52}|     \le      \|  \mathbf{E}  \| _{L^2  }
    \| e^{-\frac{\kappa}{2} \<x\>}v_1         \| _{L^2} \lesssim  \delta _2^{-1}|{\bm z}|^4 + \delta _2 \| e^{-\frac{\kappa}{2} \<x\>}v_1 \| _{L^2 }^2
 \end{align*}
Collecting the estimates and bootstrapping,  we obtain the conclusion \eqref{eq:lem:2v2}.
\end{proof}

Combining Lemmas \ref{lem:2v1} and \ref{lem:2v2}, we have
\begin{lemma}\label{lem:2v3}
	For any $\mu>0$, we have
	\begin{align*}
		&\int_0^T\left(\| \sech \left( \frac{\kappa}{2}x \right)v_1'\|_{L^2}^2+\|\sech \left( \frac{\kappa}{2}x \right){\bm v}\|_{L^2}^2\right)
\lesssim  \delta   + o_\varepsilon (1)\int_0^T\| \boldsymbol{\eta} \|_{ \boldsymbol{ \Sigma }_A}^2+ \|{\bm z} \|_{L^4(0,T)}^4.
	\end{align*}
\end{lemma}
\begin{proof}
	The claim follows from Lemmas  \ref{lem:2v1} and \ref{lem:2v2} and
	\begin{align*}
		|\mathcal{I}_{\mathrm{2nd},1}|\lesssim B\varepsilon^{-\pare{N+1}}\delta^2,\
		|\mathcal{I}_{\mathrm{2nd},2}|\lesssim \varepsilon^{-\pare{N+1}}\delta^2.
	\end{align*}
\end{proof}

\begin{proof}[Proof of Proposition \ref{prop:2ndvirial}]
	It  is a consequence of Lemma \ref{lem:2v3}     and inequality \eqref{eq:key1}.
\end{proof}






\section*{Acknowledgments}
S.C. was supported   by the Prin 2020 project \textit{Hamiltonian and Dispersive PDEs} N. 2020XB3EFL.
M.M. was supported by the JSPS KAKENHI Grant Number 19K03579 and G19KK0066A.
S.S. was partially supported by Gnampa, Indam and the PRIN 2022 project {\it Turbulent Effects vs Stability in Equations from Oceanography}, acronym TESEO.

	\begin{footnotesize}
%		\bibliography{references}
%		\bibliographystyle{plain}
\begin{thebibliography}{10}

\bibitem{BP1995}
V.~S. Buslaev and G.~S. Perelman.
\newblock On the stability of solitary waves for nonlinear {S}chr\"{o}dinger
  equations.
\newblock In {\em Nonlinear evolution equations}, volume 164 of {\em Amer.
  Math. Soc. Transl. Ser. 2}, pages 75--98. Amer. Math. Soc., Providence, RI,
  1995.

\bibitem{CGN2007}
S.~Chang, S.~Gustafson, K.~Nakanishi, and T.~Tsai.
\newblock Spectra of linearized operators for {NLS} solitary waves.
\newblock {\em SIAM J. Math. Anal.}, 39(4):1070--1111, 2007/08.

\bibitem{CLL2020}
G.~Chen, J.~Liu, and B.~Lu.
\newblock Long--time asymptotics and stability for the sine-{G}ordon equation.
\newblock preprint arXiv:2009.04260.

\bibitem{CP2022}
G.~Chen and F.~Pusateri.
\newblock The 1-dimensional nonlinear {S}chr\"{o}dinger equation with a
  weighted {$L^{1}$} potential.
\newblock {\em Anal. PDE}, 15(4):937--982, 2022.

\bibitem{CM2019}
S.~Cuccagna and M.~Maeda.
\newblock On stability of small solitons of the 1-{D} {NLS} with a trapping
  delta potential.
\newblock {\em SIAM J. Math. Anal.}, 51(6):4311--4331, 2019.

\bibitem{CM2022}
S.~Cuccagna and M.~Maeda.
\newblock On selection of standing wave at small energy in the 1{D} cubic
  {S}chr\"{o}dinger equation with a trapping potential.
\newblock {\em Comm. Math. Phys.}, 396(3):1135--1186, 2022.

\bibitem{CM2023}
S.~Cuccagna and M.~Maeda.
\newblock Asymptotic stability of kink with internal modes under odd
  perturbation.
\newblock {\em NoDEA Nonlinear Differential Equations Appl.}, 30(1):Paper No.
  1, 47, 2023.

\bibitem{CMS2023}
S.~Cuccagna, M.~Maeda, and Stefano Scrobogna.
\newblock Small energy stabilization for 1{D} nonlinear {K}lein {G}ordon
  equations.
\newblock {\em J. Differential Equations}, 350, 2023.

\bibitem{DT1979}
P.~Deift and E.~Trubowitz.
\newblock Inverse scattering on the line.
\newblock {\em Comm. Pure Appl. Math.}, 32(2):121--251, 1979.

\bibitem{DM2020}
J.-M. Delort and N.~Masmoudi.
\newblock {\em Long-time dispersive estimates for perturbations of a kink
  solution of one-dimensional cubic wave equations}, volume~1 of {\em Memoirs
  of the European Mathematical Society}.
\newblock EMS Press, Berlin, [2022] \copyright 2022.

\bibitem{GP2022}
P.~Germain and F.~Pusateri.
\newblock Quadratic {K}lein-{G}ordon equations with a potential in one
  dimension.
\newblock {\em Forum Math. Pi}, 10:Paper No. e17, 172, 2022.

\bibitem{GPZ2023}
P.~Germain, F.~Pusateri, and K.~Zhang.
\newblock On 1d quadratic {K}lein-{G}ordon equations with a potential and
  symmetries.
\newblock {\em Arch. Ration. Mech. Anal.}, 247(2):Paper No. 17, 39, 2023.

\bibitem{GSS1}
M.~Grillakis, J.~Shatah, and W.~Strauss.
\newblock Stability theory of solitary waves in the presence of symmetry. {I}.
\newblock {\em J. Funct. Anal.}, 74(1):160--197, 1987.

\bibitem{KM22}
M.~Kowalczyk and Y.~Martel.
\newblock Kink dynamics under odd perturbations for $(1+1)$-scalar field models
  with one internal mode.
\newblock preprint arXiv:2203.04143.

\bibitem{KMM2017}
M.~Kowalczyk, Y.~Martel, and C.~Mu\~{n}oz.
\newblock Kink dynamics in the {$\phi^4$} model: asymptotic stability for odd
  perturbations in the energy space.
\newblock {\em J. Amer. Math. Soc.}, 30(3):769--798, 2017.

\bibitem{KMM2022}
M.~Kowalczyk, Y.~Martel, and C.~Mu\~{n}oz.
\newblock Soliton dynamics for the 1{D} {NLKG} equation with symmetry and in
  the absence of internal modes.
\newblock {\em J. Eur. Math. Soc. (JEMS)}, 24(6):2133--2167, 2022.

\bibitem{KMMV2021}
M.~Kowalczyk, Y.~Martel, C.~Mu\~{n}oz, and H.~Van Den~Bosch.
\newblock A sufficient condition for asymptotic stability of kinks in general
  {$(1+1)$}-scalar field models.
\newblock {\em Ann. PDE}, 7(1):Paper No. 10, 98, 2021.

\bibitem{KNS2012}
J.~Krieger, K.~Nakanishi, and W.~Schlag.
\newblock Global dynamics above the ground state energy for the one-dimensional
  {NLKG} equation.
\newblock {\em Math. Z.}, 272(1-2):297--316, 2012.

\bibitem{LP2022}
T~L\'{e}ger and F.~Pusateri.
\newblock Internal mode-induced growth in 3d nonlinear {K}lein-{G}ordon
  equations.
\newblock {\em Atti Accad. Naz. Lincei Rend. Lincei Mat. Appl.},
  33(3):695--727, 2022.

\bibitem{LL2023}
Y.~Li and J.~L\"{u}hrmann.
\newblock Soliton dynamics for the 1{D} quadratic {K}lein-{G}ordon equation
  with symmetry.
\newblock {\em J. Differential Equations}, 344:172--202, 2023.

\bibitem{LS2023}
J.~L\"uhrmann and W.~Schlag.
\newblock On codimension one stability of the soliton for the 1{D} focusing
  cubic {K}lein--{G}ordon equation.
\newblock preprint \url{https://arxiv.org/pdf/2302.05273.pdf}.

\bibitem{MR2006}
F.~Merle and P.~Raphael.
\newblock On a sharp lower bound on the blow-up rate for the {$L^2$} critical
  nonlinear {S}chr\"{o}dinger equation.
\newblock {\em J. Amer. Math. Soc.}, 19(1):37--90, 2006.

\bibitem{Sigal1993}
I.~M. Sigal.
\newblock Nonlinear wave and {S}chr\"{o}dinger equations. {I}. {I}nstability of
  periodic and quasiperiodic solutions.
\newblock {\em Comm. Math. Phys.}, 153(2):297--320, 1993.

\bibitem{SW1999}
A.~Soffer and M.~I. Weinstein.
\newblock Resonances, radiation damping and instability in {H}amiltonian
  nonlinear wave equations.
\newblock {\em Invent. Math.}, 136(1):9--74, 1999.

\end{thebibliography}
	\end{footnotesize}



Department of Mathematics and Geosciences,  University
of Trieste, via Valerio  12/1  Trieste, 34127  Italy.
{\it E-mail Address}: {\tt scuccagna@units.it}


Department of Mathematics and Informatics,
Graduate School of Science,
Chiba University,
Chiba 263-8522, Japan.
{\it E-mail Address}: {\tt maeda@math.s.chiba-u.ac.jp}



Department of Mathematics and Geosciences,  University
of Trieste, via Valerio  12/1  Trieste, 34127  Italy.
{\it E-mail Address}: {\tt FEDERICO.MURGANTE@units.it}


Department of Mathematics and Geosciences,  University
of Trieste, via Valerio  12/1  Trieste, 34127  Italy.
{\it E-mail Address}: {\tt STEFANO.SCROBOGNA@units.it}

\end{document}     