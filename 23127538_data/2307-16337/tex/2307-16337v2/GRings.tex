\documentclass[12pt]{amsart}

\input{"Preamble"}

%\usepackage{standalone} 
%\def\inmain{1}

\title{Fully dualizable categories over commutative rings}

\author{G. Stefanich}

\externaldocument{Introduction}
\externaldocument{Preliminaries}
\externaldocument{Filtered}
\externaldocument{Semisimple}

\date{}

\begin{document}

%%%%%%%%%%%%%%%%%%%%%%%%%%%%%%%%%%%%%%%%%%%%%%%%%%%%%%%%%%%%%%%%%%%%%%%%
%%%%%%%%%%%%%%%%%%%%%%%%%%%%%%%%%%%%%%%%%%%%%%%%%%%%%%%%%%%%%%%%%%%%%%%%
%%%%%%%%%%%%%%%%%%%%%%%%%%%%%%%%%%%%%%%%%%%%%%%%%%%%%%%%%%%%%%%%%%%%%%%%
%%%%%%%%%%%%%%%%%%%%%%%%%%%%%%%%%%%%%%%%%%%%%%%%%%%%%%%%%%%%%%%%%%%%%%%%
%%%%%%%%%%%%%%%%%%%%%%%%%%%%%%%%%%%%%%%%%%%%%%%%%%%%%%%%%%%%%%%%%%%%%%%%
%%%%%%%%%%%%%%%%%%%%%%%%%%%%%%%%%%%%%%%%%%%%%%%%%%%%%%%%%%%%%%%%%%%%%%%%

\section{Fully dualizable categories over commutative rings}\label{section G rings}

The main goal of this section is to supply proofs of theorems \ref{theorem principal 2 introduction}, \ref{teo introduction spectral}, and their variants. We may summarize our strategy in the $(1,1)$-categorical case as follows:
\begin{enumerate}[\normalfont (1)]
\item Using the results from \ref{subsection compactly assembled}, we have that if $\ccal$ is a fully dualizable $R$-linear cocomplete category then $\ccal$ is automatically Grothendieck abelian and has exact products. The main task is to show that $\ccal$ admits a compact projective generator \'etale locally on $\Spec(R)$.
\item  We first address the case when $R$ is a local Artinian commutative ring. This is done by induction on the length of $R$ as an $R$-module.   The case when $R$ has length $1$ (i.e., is a field) follows from the results of section \ref{section invertible semisimple}.  For the inductive step one shows that compact projective generators may be deformed along an elementary extension $R \rightarrow S$ of local Artinian commutative rings (see definition  \ref{definition elementary extension}).
\item  We then address the case when $R$ is a complete local Noetherian commutative ring with maximal ideal $\mathfrak{m}$. In this case one uses the result from step (2) to construct a compatible sequence of compact projective generators for $\Ccal \otimes_R R/\mathfrak{m}^n$, and in the limit an object $X$ of $\Ccal$. The full dualizability of $\ccal$ is used to show that $X$ is a generator for $\ccal$.  The Gabriel-Popescu theorem then shows that $\Ccal$ is a left exact localization of the category of modules over the endomorphism algebra of $X$, and we finally proceed  to show that this localization has trivial kernel.
\item The general case is proven using Popescu's smoothing theorem together with the result from step (3) and our main theorem from section \ref{section compactly assembled}.
\end{enumerate}

The $(1,1)$-categorical theorem is used to deduce the $(\infty,1)$-categorical theorem in the case of (classical) G-rings, by showing that any fully dualizable $R$-linear Grothendieck prestable category is the connective derived category of its heart. The general case is then proven by deforming compact generators along the Postnikov tower of $R$, with the clutching theorems from \cite{SAG} section 16.2 being applied for the inductive step.

As discussed in the introduction to the paper, we will deduce our main theorems from a general result that applies to graded categories as well. In this case, instead of working with $R$-linear cocomplete categories, we consider instead cocomplete categories linear over a base symmetric monoidal $R$-linear Grothendieck abelian or prestable category, which we assume to be rigid, generated under colimits by compact projective objects, proper over $R$, and with semisimple fibers on a dense subset of $\Spec(R)$.

 We begin this section in \ref{subsection proper} with some basic facts concerning proper $R$-linear Grothendieck abelian and prestable categories. In \ref{subsection completion} we collect a few results on the procedure of completion with respect to an ideal for objects of $R$-linear Grothendieck prestable categories, that will be needed when carrying out step (3) in our proof. 
 
 The proof of our main results is given in \ref{subsection fully dualizable 11} and \ref{subsection fully dualizable infty}. We then show in \ref{subsection rings of def} that these results extend beyond G-rings under additional compact generation hypothesis on $\ccal$ and $\ccal^\vee$.  We finish the section in \ref{subsection invertible stable over R} with a proof of theorem \ref{teo principal stable introduction}.

%%%%%%%%%%%%%%%%%%%%%%%%%%%%%%%%%%%%%%%%%%%%%%%%%%%%%%%%%%%%%%%%%%%%%%%%
%%%%%%%%%%%%%%%%%%%%%%%%%%%%%%%%%%%%%%%%%%%%%%%%%%%%%%%%%%%%%%%%%%%%%%%%
%%%%%%%%%%%%%%%%%%%%%%%%%%%%%%%%%%%%%%%%%%%%%%%%%%%%%%%%%%%%%%%%%%%%%%%%
%%%%%%%%%%%%%%%%%%%%%%%%%%%%%%%%%%%%%%%%%%%%%%%%%%%%%%%%%%%%%%%%%%%%%%%%
%%%%%%%%%%%%%%%%%%%%%%%%%%%%%%%%%%%%%%%%%%%%%%%%%%%%%%%%%%%%%%%%%%%%%%%%
%%%%%%%%%%%%%%%%%%%%%%%%%%%%%%%%%%%%%%%%%%%%%%%%%%%%%%%%%%%%%%%%%%%%%%%%

\subsection{Proper \texorpdfstring{$R$}{R}-linear categories}\label{subsection proper}

We will be interested in this section in properness for $R$-linear Grothendieck abelian and Grothendieck prestable categories generated under colimits by compact projective objects (corresponding to the cases $n = 1$ and $n = \infty$ of remark \ref{remark properness when comp gen}).  It follows from remark \ref{remark base change proper} that if $\ccal$ is a proper $R$-linear Grothendieck prestable category then $\ccal^\heartsuit$ is a proper $\pi_0(R)$-linear Grothendieck abelian category. The following proposition shows that the procedure of deriving Grothendieck abelian categories also preserves properness:

\begin{proposition}\label{prop proper clasica vs prestable}
Let $R$ be a commutative ring and let $\ccal$ be an $R$-linear Grothendieck abelian category. Assume that $\ccal$ is generated by compact projective objects and is proper over $R$. Then $\der(\ccal)_{\geq 0}$ is proper over $R$.
\end{proposition}
\begin{proof}
Let $X, Y$ be a pair of compact projective objects of $\ccal$. Then for each compact projective $R$-module $Z$ we have
\begin{align*}
\Hom_{\Mod_R^\cn}(Z, \Hom_{\der(\ccal)_{\geq 0}}^\enh(X, Y)) & = \Hom_{\der(\ccal)}(Z \otimes X, Y) \\ &= \Hom_{\ccal}(Z \otimes X, Y) 
\\ &= \Hom_{\Mod_R^\heartsuit}(Z, \Hom_{\ccal}^\enh(X, Y)).
\end{align*}
The above equivalence is functorial in $Z$ and therefore induces an isomorphism of $R$-modules $\Hom_{\der(\ccal)_{\geq 0}}^\enh(X, Y) = \Hom_{\ccal}^\enh(X, Y)$. The proof finishes by observing that an object of $\Mod_R^\heartsuit$ is compact projective if and only if it is compact projective inside $\Mod_R^\cn$.
\end{proof}

\begin{remark}\label{remark proper is derived}
Let $R$ be a commutative ring and let $\ccal$ be an $R$-linear Grothendieck prestable category. Assume that $\ccal$ is generated under colimits by compact projective objects and is proper over $R$. Then $\Hom_\ccal(X, Y)$ is $0$-truncated for each pair of compact projective objects of $\ccal$, and in particular we see that every compact projective object of $\ccal$ is $0$-truncated. It follows from this that there is an $R$-linear equivalence $\ccal = \der(\ccal^\heartsuit)$.
\end{remark}

In the presence of properness projective objects are automatically flat:

\begin{proposition}\label{prop flatness of compact projectives}
Let $R$ be a connective commutative ring spectrum and let $\ccal$ be an $R$-linear Grothendieck prestable category. Assume that $\ccal$ is generated under colimits compact projective objects and proper over $R$. Then every projective object of $\ccal$ is flat over $R$.
\end{proposition}
\begin{proof}
Since every projective object is a retract of a direct sum of compact projective objects, and flatness is preserved by retracts and direct sums it is enough to show that every compact projective object $X$ in $\ccal$ is flat over $R$. This amounts to showing that $M \otimes_R X$ is $0$-truncated for every $0$-truncated $R$-module $M$. To prove this it is enough to show that $\Hom^\enh_\ccal(Y, M \otimes_R X)$ is a $0$-truncated $R$-module for every compact projective object $Y$. This follows from the fact that  $\Hom^\enh_\ccal(Y, M \otimes_R X) = M \otimes_R \Hom^\enh_\ccal(Y, X)$ combined with the fact that dualizable $R$-modules are flat.
\end{proof}

We focus in the remainder of this section in the case where $R$ is a (classical) commutative ring. We will formulate our result for proper $R$-linear Grothendieck abelian categories generated by compact projective objects; it follows from remark \ref{remark proper is derived} that the Grothendieck prestable setting does not present extra generality in this case.

  We start by formulating a version of Nakayama's lemma:

\begin{proposition}\label{proposition nakayama}
Let $R$ be a local commutative ring with residue field $k$ and let $\ccal$ be an $R$-linear Grothendieck abelian category generated by compact projective objects and proper over $R$. Let $X$ be a finitely generated object of $\Ccal$. If  $X \otimes_R k = 0$ then $X = 0$.
\end{proposition}
\begin{proof}
Pick  an epimorphism $f: Y \rightarrow X$ with $Y$ compact projective. Passing to $R$-modules of maps from $Y$ we obtain an epimorphism
\[
\Hom^\enh_\ccal(Y, Y) \rightarrow \Hom^\enh_\ccal(Y, X) .
\]
 In particular, since the left hand side is a compact projective $R$-module we have that $\Hom^\enh_\ccal(Y, X)$ is a finitely generated $R$-module. We now have
 \[
\Hom^\enh_\ccal(Y, X) \otimes_R k   =  \Hom^\enh_\ccal(Y, X \otimes_R k) = 0.
\]
An application of Nakayama's lemma shows that $\Hom^\enh_\ccal(Y, X) = 0$. Hence $f = 0$ and therefore $X = 0$, as desired.
\end{proof}

\begin{corollary}\label{coro check epi on fiber}
Let $R$ be a local commutative ring with residue field $k$ and let $\ccal$ be an $R$-linear Grothendieck abelian category generated by compact projective objects and proper over $R$. Let $f: X \rightarrow Y$ be a morphism in $\ccal$, with $Y$ finitely generated. If $f \otimes_R k$ is an epimorphism then $f$ is an epimorphism.
\end{corollary}
\begin{proof}
Follows from proposition \ref{proposition nakayama} since $\operatorname{Coker}(f) \otimes_R k = \operatorname{Coker}(f \otimes_R k) = 0$.
\end{proof}

\begin{corollary}\label{coro lift equivalences of comp projectives}
Let $R$ be a local commutative ring with residue field $k$ and let $\ccal$ be an $R$-linear Grothendieck abelian category. Assume that $\ccal$ is generated by compact projective objects and proper over $R$. If $X$ and $Y$ are compact projective objects such that $X \otimes_R k$ and $Y \otimes_R k$ are isomorphic, then $X$ and $Y$ are isomorphic. Furthermore, any morphism $f: X \rightarrow Y$ such that $f \otimes_R k$ is an isomorphism, is itself an isomorphism.
\end{corollary}
\begin{proof}
Since $X$ is projective any isomorphism between $X \otimes_R k$ and $Y \otimes_R k$ may be lifted to a morphism $f: X \rightarrow Y$. It remains to show that in this case $f$ is an isomorphism. The map $f$ is an epimorphism by corollary \ref{coro check epi on fiber}. In particular, the kernel of $f$ is finitely generated as well. By proposition \ref{prop flatness of compact projectives} we have that $\Ker(f) \otimes_R k = \Ker(f \otimes_R k) = 0$. Another application of proposition \ref{proposition nakayama} shows that $f$ is a monomorphism, and therefore an isomorphism.
\end{proof}

We now prove that compact projective objects in fibers extend to \'etale neighborhoods.

\begin{proposition}\label{prop extend compact projectives}
Let $R$ be a commutative ring and let $\ccal$ be an $R$-linear Grothendieck abelian category. Assume that $\ccal$ is generated   by compact projective objects and proper over $R$. Let $\mathfrak{p}$ be a prime ideal of $R$ with residue field $k$ and let $X$ be a compact projective object of $\ccal \otimes_R k$.
\begin{enumerate}[\normalfont (1)]
\item If $R$ is complete local and $\mathfrak{p}$ is maximal then there exists a compact projective object in $\ccal$ whose image in $\ccal \otimes_R k$ recovers $X$.
\item There exists an \'etale morphism $R \rightarrow R'$ with $R' \otimes_R k \neq 0$ and a compact projective object in $\ccal \otimes_R R'$ whose image in $\ccal \otimes_R (k \otimes_R R')$ recovers $X \otimes_k (k \otimes_R R')$.
\end{enumerate}
\end{proposition}
\begin{proof}
We first address part (1). Denote $\ccal_k = \ccal \otimes_R k$. Since $\ccal_k$ is generated under colimits by objects of the form $Y \otimes_R k$ we may pick a compact projective object $Y$ in $\ccal$ such that  $X$ is a retract of $Y \otimes_R k$. Let $r$ be an idempotent endomorphism on $Y \otimes_R k$ with image $X$. The lemma will follow if we are able to show that $r$ admits a lift to an idempotent endomorphism of $Y$.

 Since $Y$ is compact projective the extension of scalars map $\End^\enh_\Ccal(Y) \rightarrow \End^\enh_{\Ccal_k}(Y_k)$ is equivalent to the  canonical map of $R$-algebras $\End^\enh_\ccal(Y) \rightarrow \End^\enh_\ccal(Y) \otimes_R k$. Set $A = \End^\enh_\ccal(Y)$, so that our map is given by the quotient $A \rightarrow A/\mathfrak{m}$.   We may identify $r$ with an idempotent element of $A/\mathfrak{m}A$, and our task is to show that this lifts to an idempotent element of $A$. Consider the sequence of square zero extensions $A/\mathfrak{m}A \leftarrow A/\mathfrak{m}^2A \leftarrow A/\mathfrak{m}^3 A \leftarrow \ldots$. We may lift $r$ to a compatible sequence of idempotents $r = r_1, r_2, \ldots$. Since $A$ is a compact projective $R$-module we have an isomorphism $A = \lim A/\mathfrak{m}^nA$ and hence the sequence induces in the limit the desired idempotent in $A$ lifting $r$.
 
 We now prove part (2).  As before, pick a compact projective object $Y$ in $\ccal$ such that $X$ is a retract of $Y \otimes_R k$ and let $A$ be the $R$-algebra of endomorphisms of $Y$. Then we have an $R$-linear fully faithful embedding $\LMod_A(\Mod^\heartsuit_R) \rightarrow \ccal$. Since $X$ belongs to $\LMod_A(\Mod^\heartsuit_R) \otimes_R k$ it is enough to prove the proposition in the case $\ccal = \LMod_A(\Mod^\heartsuit_R)$.  Write $R$ as a filtered colimit of a diagram of commutative rings $R_\alpha$ of finite type over $\ZZ$. Since the $R$-module underlying $A$ is dualizable, there exists an index $\alpha$ and an $R_\alpha$-algebra $A_\alpha$ whose underlying $R_\alpha$-module is dualizable, and such that $A = R \otimes_{R_\alpha} A_\alpha$.
 
 We assume without loss of generality that $\alpha$ is initial. For each index $\beta$ let $A_\beta =R_\beta \otimes_{R_\alpha} A_\alpha$, and let $k_\beta$ be the residue field of $R_\beta$ at the preimage of $\mathfrak{p}$ under the map $R_\beta \rightarrow R$. We have $k = \colim k_\beta$, so the idempotent of $A \otimes_R k$ with image $x$ lifts to an idempotent of $A_\beta \otimes_{R_\beta} k_\beta$ for some $\beta$. Replacing $R$ by $R_\beta$ and $\ccal$ by $\LMod_{A_\beta}(\Mod_{R_\beta}^\heartsuit)$ we may now assume that $R$ is of finite type over $\ZZ$, and in particular a G-ring.
 
By part (1) we may find a compact projective object in $\ccal \otimes_R R^\wedge_{\mathfrak{p}}$ whose image in $\ccal \otimes_R k$ recovers $X$. Combining theorem \ref{theorem lift strongly compacts} with Popescu's smoothing theorem we may find a smooth $R$-algebra $S$ such that $S \otimes_R k \neq 0$ and a compact projective object $Z$ in $\ccal \otimes_R S$ whose image in $\ccal \otimes_R (S \otimes_R k)$ recovers $X \otimes_k (S \otimes_R k)$. Pick a morphism of commutative rings $S \rightarrow R'$ such that the induced map $R \rightarrow R'$ is \'etale and $R' \otimes_R k \neq 0$.  Then $Z \otimes_S R'$ is a compact projective object of $\ccal \otimes_R R'$ with the desired property.
\end{proof}

\begin{corollary}\label{corollary extend equivalence etale locally}
Let $R$ be a commutative ring and let $\ccal, \ccal'$ be $R$-linear Grothendieck abelian categories. Assume that $\ccal$ and $\ccal'$ are generated   by compact projective objects and fully dualizable over $R$. Let $\mathfrak{p}$ be a prime ideal of $R$ with residue field $k$. If $\ccal \otimes_R k = \ccal' \otimes_R k$ then there exists an \'etale $R$-algebra $R'$ such that $R' \otimes_R k \neq 0$ and an equivalence $\ccal \otimes_R R' = \ccal' \otimes_R R'$.
\end{corollary}
\begin{proof}
We have that $\ccal$ and $\ccal'$ are dualizable over $R$, and furthermore $\Funct(\ccal, \ccal')$ and $\Funct(\ccal', \ccal)$ are Grothendieck abelian categories generated by compact projective objects and proper over $R$. Let $F: \ccal \otimes_R k \rightarrow \ccal' \otimes_R k$ and $G: \ccal' \otimes_R k \rightarrow \ccal \otimes_R k$ be inverse equivalences. Then $F$ and $G$ define compact projective objects in $\Funct(\ccal, \ccal') \otimes_R k$ and $\Funct(\ccal', \ccal) \otimes_R k$, respectively. An application of proposition \ref{prop extend compact projectives} shows that, after replacing $R$ with an \'etale $R$-algebra if necessary, we may find compact projective lifts $\overline{F}: \ccal \rightarrow \ccal'$ and $\overline{G}: \ccal' \rightarrow \Ccal$ for $F$ and $G$. The fact that $\ccal$ and $\ccal'$ are proper implies that $\overline{F} \circ \overline{G}$ and $\overline{G} \circ \overline{F}$ are also compact projective. Since $F \circ G$ and $G \circ F$ are identities, it follows from corollary \ref{coro lift equivalences of comp projectives} that $\overline{F} \circ \overline{G}$ and $\overline{G} \circ \overline{F}$ become isomorphic to identities after passing to a localization of $R$.
\end{proof}

\begin{corollary}\label{coro smooth and proper locally trivial}
Let $R$ be a commutative ring and let $A$ be a smooth and proper algebra in $\Mod^\heartsuit_R$. Let $\mathfrak{p}$ be a prime ideal of $R$ with residue field $k$. Then there exists an \'etale $R$-algebra $R'$ such that $R' \otimes_R k \neq 0$ with the property that $A \otimes_R R'$ is Morita equivalent to  a finite product of copies of $R'$.
\end{corollary}
\begin{proof}
Follows from corollary \ref{corollary extend equivalence etale locally} together with the fact that every separable algebra over $k$ becomes Morita equivalent to a finite product of copies of $k$ after passage to a finite separable field extension.
\end{proof}

\begin{remark}
The conclusion of proposition \ref{prop extend compact projectives} does not hold if we replace \'etale morphisms with Zariski morphisms: consider for instance $\ccal = \Mod^\heartsuit_{\ZZ[x,x^{-1}]}$ as a category linear over $\ZZ[x^2, x^{-2}]$.
\end{remark}

In what follows we will be working with proper $R$-linear categories with semisimple fibers. We will need the following:

\begin{proposition}\label{prop subcat closed under passage to subobjects}
Let $R$ be a commutative ring and let $\ccal$ be an $R$-linear Grothendieck abelian category. Assume that $\ccal$ is generated by compact projective objects, proper over $R$, and that the set of points $x$ in $\Spec(R)$ such that $\ccal \otimes_R \kappa(x)$ is semisimple is dense in the Zariski topology. Let $X$ be a compact projective object of $\ccal$. Then the full subcategory of $\ccal$ generated under colimits by $X$ is closed under passage to subobjects.
\end{proposition}

The proof of proposition \ref{prop subcat closed under passage to subobjects} requires some preliminary lemmas.

\begin{lemma}\label{lemma split after cover}
Let $R$ be a commutative ring and let $\ccal$ be an $R$-linear Grothendieck abelian category. Assume that $\ccal$ is generated by compact projective objects, proper over $R$, and fix a point in $\Spec(R)$ with residue field $k$ such that $\ccal \otimes_R k$ is semisimple. Let $X$ be a compact projective object of $\ccal$. Then there exists an \'etale morphism $R \rightarrow R'$ such that $R' \otimes_R k = k'$ is a field and a finite sequence of compact projective objects $Y_i$ in $\ccal \otimes_R R'$  such that $X \otimes_R R' = \bigoplus Y_i$ and $Y_i \otimes_R' k'$ is a simple object of $\ccal \otimes_R k'$ for every $i$.
\end{lemma}
\begin{proof}
For every finite separable field extension $F$ of $k$ let $n_F$ be the number of simple summands of $X \otimes_R F$ in $\ccal \otimes_R F$. This is bounded by the dimension of $\End^\enh_{\ccal \otimes_R F}(X \otimes_R F)$, which itself agrees with the dimension of $\End^\enh_{\ccal \otimes_R k}(X \otimes_R k)$. We may therefore choose $F$ such that $n_F$ is maximal. This extension has the property that the simple summands of $X \otimes_R F$ remain simple under further separable finite extensions.  Let $R \rightarrow S$ be an \'etale morphism such that $S \otimes_R k = F$. Write $X \otimes_R F$ as a sum of simple objects $S_i$. Applying proposition \ref{prop extend compact projectives} we may find an \'etale morphism $S \rightarrow R'$ with the property that $R' \otimes_R k \neq 0$, and a sequence of compact projective objects $Y_i$ in $\ccal \otimes_R R'$ such that $Y_i \otimes_R k = S_i \otimes_{F} (F \otimes_S R')$. Passing to a localization of $R'$ we may further assume that $R' \otimes_R k = k'$ is a finite separable field extension of $k$. Note that then $X \otimes_R k' = \bigoplus Y_i \otimes_{R} k $ is a decomposition in simples. It now follows from corollary \ref{coro lift equivalences of comp projectives} that after replacing $R'$ with a localization if necessary we have $X \otimes_R R' = \bigoplus Y_i$.
\end{proof}

\begin{lemma}\label{lemma epi after cover}
Let $R$ be a commutative ring and let $\ccal$ be an $R$-linear Grothendieck abelian category. Assume that $\ccal$ is generated by compact projective objects, proper over $R$, and fix a point in $\Spec(R)$ with residue field $k$ such that $\ccal \otimes_R k$ is semisimple. Let $X$ be a compact projective object of $\ccal$ and let $Y$ be a finitely generated subobject of $X$. Then there exists an \'etale morphism $R \rightarrow R'$ such that $R' \otimes_R k \neq 0$ with the property that that $Y \otimes_R R'$ receives an epimorphism from a finite direct sum of copies of $X \otimes_R R'$.
\end{lemma}
\begin{proof}
By lemma \ref{lemma split after cover} we may assume (after passing to a cover if necessary) that $X = \bigoplus X_i$ for a finite sequence of compact projective objects such that $S_i = X_i \otimes_R k $ is simple for all $i$. Pick an epimorphism $\rho: Z \rightarrow Y$ with $Z$ compact projective. Write $Z \otimes_R k = Q \oplus \bigoplus S_{i_\alpha}$ where $\Hom_{\ccal \otimes_R k}(S_i, Q) = 0$ for all $i$. Let $W = \bigoplus X_{i_\alpha}$. Since $W \otimes_R k$ is a direct summand of $Z \otimes_R k$ and $Z$ and $W$ are projective we may find maps $f:Z \rightarrow W$ and $g: W \rightarrow Z$ such that $(f \circ g) \otimes_R k$ is the identity. By corollary \ref{coro lift equivalences of comp projectives}, after passing to a localization of $R$ we may assume that $f \circ g$ is an isomorphism. We may therefore write $Z = W \oplus Z'$, where $Z' \otimes_R k = Q$. Note that $\Hom^\enh_\ccal(Z', X) \otimes_R k = \Hom^\enh_{\ccal \otimes_R k}(Q, X \otimes_R k) = 0$. After passing to a localization of $R$ we may assume that $\Hom^\enh_\ccal(Z', X) = 0$.  We conclude that the image of $\rho$ is equivalent to the image of $\rho \circ g$, which receives an epimorphism from a direct sum of copies of $X$ since $W$ does.
\end{proof}

\begin{proof}[Proof of proposition \ref{prop subcat closed under passage to subobjects}]
Let $\ccal'$ be the full subcategory of $\ccal$ generated by $X$. Let $Y$ be an object of $\ccal'$ and let $Z$ be a subobject of $Y$ inside $\ccal$. Our task is to show that $Z$ belongs to $\ccal'$. Since Grothendieck abelian categories satisfy \'etale descent, it suffices to prove this after passage to an \'etale cover of $\Spec(R)$. 

Since $Z$ is a colimit of finitely generated subobjects we may reduce to the case when $Z$ is finitely generated. Pick an epimorphism $S \otimes X \rightarrow Y$ where $S$ is a set. Then $Z$ receives an epimorphism from a subobject $Z'$ of $T \otimes X$, where $T$ is a finite subset of $S$. Replacing $X$ with $T \otimes X$ we may reduce to the case where $Z'$ is a subobject of $X$. By lemma \ref{lemma epi after cover} we may, after passing to a cover, assume that $Z'$ admits an epimorphism from a finite direct sum of copies of $X$. Replacing $X$ with a finite direct sum of copies of $X$ we may assume the existence of an epimorphism $X \rightarrow Z'$, which implies that there exists an epimorphism $X \rightarrow Z$. Let $W$ be its kernel. Write $W = \colim W_\alpha$ where $W_\alpha$ are finitely generated subobjects of $W$. Then  $Z = \colim X/W_\alpha$, so it suffices to prove that $X/W_\alpha$ belongs to $\ccal'$ for each $\alpha$. Replacing $Z$ by $X/W_\alpha$ and $W$ by $W_\alpha$ we may now assume that $W$ is finitely generated. Passing to a cover if necessary we may, thanks to lemma \ref{lemma epi after cover}, find an epimorphism $U \otimes X \rightarrow W$ for some finite set $U$. Then $Z$ is the cokernel of the map $U \otimes X \rightarrow X$, so it belongs to $\ccal'$, as desired.
\end{proof}

\begin{corollary}\label{coro C is locally noetherian}
Let $R$ be a Noetherian commutative ring and let $\ccal$ be an $R$-linear Grothendieck abelian category. Assume that $\ccal$ is generated by compact projective objects, proper over $R$, and that the set of points $x$ in $\Spec(R)$ such that $\ccal \otimes_R \kappa(x)$ is semisimple is dense in the Zariski topology. Then $\ccal$ is locally Noetherian. In particular, every finitely generated object of $\ccal$ admits a resolution by compact projective objects.
\end{corollary}
\begin{proof}
It suffices to prove that every compact projective object $X$ in $\ccal$ is Noetherian. By proposition \ref{prop subcat closed under passage to subobjects} it is enough to show that $X$ is a Noetherian object of the full subcategory $\ccal'$ of $\ccal$ generated  by $X$. This follows from the fact that the functor $\Hom^\enh_\ccal(X, -): \ccal \rightarrow \Mod_R^\heartsuit$ is conservative when restricted to $\ccal'$ and sends subobjects of $X$ to subobjects of the Noetherian $R$-module $\Hom^\enh_\ccal(X, X)$.
\end{proof}

%%%%%%%%%%%%%%%%%%%%%%%%%%%%%%%%%%%%%%%%%%%%%%%%%%%%%%%%%%%%%%%%%%%%%%%%
%%%%%%%%%%%%%%%%%%%%%%%%%%%%%%%%%%%%%%%%%%%%%%%%%%%%%%%%%%%%%%%%%%%%%%%%
%%%%%%%%%%%%%%%%%%%%%%%%%%%%%%%%%%%%%%%%%%%%%%%%%%%%%%%%%%%%%%%%%%%%%%%%
%%%%%%%%%%%%%%%%%%%%%%%%%%%%%%%%%%%%%%%%%%%%%%%%%%%%%%%%%%%%%%%%%%%%%%%%
%%%%%%%%%%%%%%%%%%%%%%%%%%%%%%%%%%%%%%%%%%%%%%%%%%%%%%%%%%%%%%%%%%%%%%%%
%%%%%%%%%%%%%%%%%%%%%%%%%%%%%%%%%%%%%%%%%%%%%%%%%%%%%%%%%%%%%%%%%%%%%%%%

\subsection{Completions and prestable categories}\label{subsection completion}

We now discuss some properties of the procedure of completion of objects of $R$-linear Grothendieck prestable categories.

\begin{proposition}\label{prop commute geom realiz and inverse limits}
Let $\Ccal$ be a Grothendieck prestable category, separated and such that products in $\Sp(\Ccal)$ are t-exact. Let $\ccal^{\NN^\op}$ be the category of inverse sequences in $\ccal$. Let $(X_{n, \bullet})$ be a semisimplicial object in $\ccal^{\NN^\op}$, where here we denote by $n$ the sequence index and by $\bullet$  the semisimplicial direction. Assume that for every $n, m \geq 0$ the map $X_{n+1, m} \rightarrow X_{n, m}$ induces an epimorphism on $H_0$. Then the limit functor  $\ccal^{\NN^\op} \rightarrow \ccal$ preserves the colimit of $(X_{n, \bullet})$.
\end{proposition}
\begin{proof}
We wish to show that the map $\xi: | \lim X_{n, \bullet} | \rightarrow \lim |X_{n, \bullet}|$ is an isomorphism. Since $\ccal$ is separated we may reduce to showing that $\xi$ induces an isomorphism on homologies. Fix $t \geq 0$. We will show that $\xi$ induces an isomorphism on $H_s$ for all $s < t-1$. 

Let $\Delta_{\text{inj}}$ be the wide subcategory of $\Delta$ on the injective maps, and let $\Delta_{\text{inj}, \leq t}$ be the full subcategory of $\Delta_{\text{inj}}$ on the simplices of dimension at most $t$. We have a commutative square
\[
\begin{tikzcd}
\colim_{\Delta^\op_{\text{inj}, \leq t}} \lim X_{n, \bullet} \arrow{r}{\xi_t} \arrow{d}{\nu_1} & \arrow{d}{\nu_2} \lim \colim_{\Delta^\op_{\text{inj}, \leq t}} X_{n,\bullet}\\
{| \lim X_{n, \bullet} |} \arrow{r}{\xi}& \lim |X_{n, \bullet}|.
\end{tikzcd}
\]
Since the inclusion $\Delta_{\text{inj}, \leq t} \rightarrow \Delta_{\text{inj}}$ is $t$-initial, we have that $\nu_1$ is an isomorphism on $H_s$ for all $s < t$. Furthermore, $\nu_2$ is an inverse limit of maps with that property, and since products in $\Sp(\ccal)$ are t-exact we have that $\nu_2$ is an isomorphism on $H_s$ for all $s < t-1$. To prove our claim it now suffices to show that $\xi_t$ is an isomorphism.

Our assumptions imply that the maps $ \colim_{\Delta^\op_{\text{inj}, \leq t}} X_{n+1,\bullet} \rightarrow  \colim_{\Delta^\op_{\text{inj}, \leq t}} X_{n,\bullet}$ induce epimorphisms on $H_0$. It follows from this together with the fact that products in $\Sp(\ccal)$ are t-exact that the limits $\lim X_{n,\bullet}$ and $\lim \colim_{\Delta^\op_{\text{inj}, \leq t}} X_{n,\bullet}$ are preserved  by the inclusion into $\Sp(\Ccal)$. The fact that $\xi_t$ is an isomorphism is now a consequence of the fact that $\Delta^\op_{\text{inj}, \leq t}$ is a finite category.
\end{proof}

\begin{corollary}\label{coro tensor preserves}
Let $R$ be a connective $E_\infty$-ring and let $\Ccal$ be an $R$-linear Grothendieck prestable category, separated and such that products in $\Sp(\Ccal)$ are t-exact. Let $X_0 \leftarrow X_1 \leftarrow X_2 \leftarrow \ldots$ be a sequence in $\ccal$ whose transitions induce epimorphisms on $H_0$. Let $M$ be an almost finitely presented  connective $R$-module. Then the map $M \otimes (\lim X_n) \rightarrow \lim (M \otimes X_n)$ is an isomorphism. 
\end{corollary}
\begin{proof}
Since $M$ is almost finitely presented there exists a simplicial resolution $M_\bullet$ of $M$ by compact projective $R$-modules. The result follows from proposition \ref{prop commute geom realiz and inverse limits} applied to the simplicial sequence $(M_\bullet \otimes X_n)$.
\end{proof}
 
 \begin{remark}
Let $R_0 \leftarrow R_1 \leftarrow R_2 \leftarrow \ldots $ be a sequence of connective $E_\infty$-rings and let $R$ be a connective $E_\infty$-ring equipped with a map $R \rightarrow \lim R_n$. Let $\Ccal$ be an $R$-linear Grothendieck prestable category. Then we have a sequence of $R$-linear Grothendieck prestable categories
\[
\Ccal \rightarrow \ldots \Ccal \otimes_R R_2 \rightarrow \Ccal \otimes_R R_1 \rightarrow \Ccal \otimes_R R_0
\]
which induces an $R$-linear functor $p: \Ccal \rightarrow \lim \Ccal \otimes_R R_n$. We think about objects in $\lim \Ccal \otimes_R R_n$ as compatible sequences $(X_n)$ with $X_n$ in $\ccal \otimes_R R_n$ for all $n \geq 0$. The functor $p$ has a right adjoint $p^R:  \lim \Ccal \otimes_R R_n \rightarrow \Ccal$ that sends a sequence $(X_n)$ to $\lim X_n$ (where here we regard each $X_n$ as an object of $\ccal$ via restriction of scalars). Note that in general $p^R$ does not preserve colimits and does not commute with the action of $\Mod_R^\cn$.
\end{remark}

\begin{corollary}\label{coro completes right adjoint 1}
Let $R_0 \leftarrow R_1 \leftarrow R_2 \leftarrow \ldots $ be a sequence of connective $E_\infty$-rings and let $R$ be a connective $E_\infty$-ring equipped with a map $R \rightarrow \lim R_n$. Assume that for every $n \geq 0$ the transition $R_{n+1} \rightarrow R_n$ induces an epimorphism on $\pi_0$. Let $\Ccal$ be an $R$-linear Grothendieck prestable category, separated and such that products in $\Sp(\Ccal)$ are t-exact. Then the right adjoint to the functor $p: \Ccal \rightarrow \lim \Ccal \otimes_R  R_n$ preserves geometric realizations and commutes with the action of the full subcategory of $\Mod_R^\cn$ on the almost finitely presented connective $R$-modules.
\end{corollary}
\begin{proof}
This is a direct consequence of proposition \ref{prop commute geom realiz and inverse limits} and corollary \ref{coro tensor preserves}.
\end{proof}

\begin{proposition}\label{prop completes ffff}
Let $R_0 \leftarrow R_1 \leftarrow R_2 \leftarrow \ldots $ be a sequence of connective $E_\infty$-rings and let $R$ be a connective $E_\infty$-ring equipped with a map $R \rightarrow \lim R_n$.  Assume the following:
\begin{itemize}
\item For every $n \geq 0$ the transition $R_{n+1} \rightarrow R_n$ induces an epimorphism on $\pi_0$.
\item The map from $R$ to the pro-$E_\infty$-ring defined by the sequence $(R_n)$ is an epimorphism in the category of pro-$E_\infty$-rings.
\item $R_n$ is almost finitely presented as an $R$-module for all $n$.
\end{itemize}
 Let $\ccal$ be an $R$-linear Grothendieck prestable category, separated and such that products in $\Sp(\Ccal)$ are t-exact. Then the right adjoint to the functor $p: \ccal \rightarrow \lim \ccal \otimes_R R_n$ is fully faithful.
\end{proposition}
\begin{proof}
We will prove the lemma by showing that the counit of the adjunction is an isomorphism. Let $(X_n)$ be an object of $\lim \ccal \otimes_R R_n$. Our goal is to prove that for every $s \geq 0$ the map $(\lim X_n) \otimes_R R_s \rightarrow X_s$ is an isomorphism. Our assumptions imply that $R_s$ is the limit  in the category of pro-$E_\infty$-rings of the sequence $R_s \otimes_R R_n$. It follows that the pro-object associated to the sequence $(R_s \otimes_R R_n)$ is equivalent to the constant pro-object on $R_s$, and therefore the projection $q: \ccal \otimes_R R_s \rightarrow \lim \ccal \otimes_R R_n \otimes_{R} R_s$ is an isomorphism. 

Consider now the object $(X_n \otimes_{R} R_s)$ in $\lim \ccal \otimes_R R_n \otimes_{R} R_s$. The fact that $q$ is an isomorphism implies that  the map $(\lim (X_n \otimes_{R} R_s)) \otimes_{R_s} (R_s \otimes_R R_s) \rightarrow X_s \otimes_{R} R_s$ is an isomorphism.  Since $R_s$ is almost finitely presented as an $R$-module there exists a simplicial $R$-module with colimit $R_s$  which is levelwise compact projective. Combining this with corollary \ref{coro tensor preserves} we obtain an equivalence $(\lim X_n) \otimes_{R} R_s = \lim (X_n \otimes_R R_s)$, so it follows that the map $(\lim X_n) \otimes_R (R_s \otimes_R R_s) \rightarrow X_s \otimes_R R_s$ is an isomorphism. Tensoring with $R_s$ over $R_s \otimes_R R_s$ we conclude that the map $(\lim X_n) \otimes R_s \rightarrow X_s$ is an isomorphism, as desired.
\end{proof}

\begin{proposition}\label{prop factors through pi0}
Let $R$ be a commutative ring and $x_1, \ldots, x_t$ be a finite sequence of elements of $R$. Consider for each $n \geq 1$ the commutative ring spectrum 
\[
R_n = R \otimes_{\ZZ[x_1, \ldots, x_t]}  \ZZ[x_1, \ldots, x_t]/(x_1^n,x_2^n \ldots, x_t^n).
\] Then for each $n \geq 1$ the canonical map $f_{2n, n}: R_{2n} \rightarrow R_n$ factors through $\pi_0(R_{2n})$.
\end{proposition}
\begin{proof}
We argue by induction on $t$. We consider first the case when $t = 1$, so that the sequence consists of a single element $x$. Let $F$ be the free commutative ring spectrum on $\Sigma (\pi_1(R_{2n}))$. Then the map $R_{2n} \rightarrow \pi_0(R_{2n})$ factors as a composition
\[
R_{2n} \rightarrow R_{2n}\otimes_{F} \SS \rightarrow \pi_0(R_{2n})
\]
where here $\SS$ denotes the sphere spectrum and the map $F \rightarrow \SS$ is zero on generators. The second map above is $2$-connective, and since $R_n$ is $1$-truncated it is enough to prove that $f_{{2n}, n}$ factors through $R_{2n}\otimes_{F} \SS$. This amounts to showing that $f_{2n, n}$ induces the zero map on $\pi_1$. Unwinding the definitions, we have that $\pi_1(f_{2n, n})$ is given by the map
\[
\Ker(x^{2n}: R \rightarrow R) \xrightarrow{x^n} \Ker(x^{n}: R \rightarrow R) 
\]
which is zero, as desired.

Assume now that $t > 1$ and that the proposition is known for all $s < t$. For each pair of positive integers $n, m$ let 
\[
R_{n, m} =  R \otimes_{\ZZ[x_1, \ldots, x_t]}  \ZZ[x_1, \ldots, x_t]/(x_1^n, x_2^n \ldots, x_{t-1}^n, x_t^m)
\]
and
\[
R'_{n, m} = R/(x_1^n, \ldots, x_{t-1}^n) \otimes_{\ZZ[x_t]} \ZZ[x_t]/(x_t^m).
\]
 We have a commutative square of commutative ring spectra
\[
\begin{tikzcd}
R_{2n, 2n} \arrow{d}{} \arrow{r}{} & \arrow{d}{} R_{n, 2n}  \\
R_{2n, n} \arrow{r}{} & R_{n, n} .
\end{tikzcd}
\]
Our task is to show that the diagonal map factors through $\pi_0(R_{2n, 2n})$.  Applying our inductive hypothesis to the sequence $x_1, \ldots, x_{t-1}$ we obtain a commutative diagram
\[
\begin{tikzcd}
R_{2n, 2n} \arrow{d}{} \arrow{r}{} & R'_{2n, 2n} \arrow{d}{} \arrow{r}{} &  \arrow{d}{} R_{n, 2n}  \\
R_{2n, n} \arrow{r}{} & R'_{2n, n} \arrow{r}{} & R_{n, n} .
\end{tikzcd}
\]
Our result now follows from another application of the inductive hypothesis to show that the map $R'_{2n, 2n} \rightarrow R'_{2n, n}$ factors through $\pi_0(R'_{2n, 2n}) = \pi_0(R_{2n, 2n})$.
\end{proof}

\begin{corollary}\label{coro fundamental completion}
Let $R$ be a commutative ring and let $I$ be an ideal in $R$ generated by elements $x_1, \ldots, x_t$. Then the pro-$E_\infty$-ring spectrum defined by the sequence $(R_n)$ from proposition \ref{prop factors through pi0} is equivalent to the one defined by the sequence $(R/I^n)$.
\end{corollary}

\begin{remark}\label{remark base change pro object}
Let $R \rightarrow S$ be a morphism of commutative rings and let $I$ be a finitely generated ideal of $R$. Then it follows from corollary \ref{coro fundamental completion} that the pro-$E_\infty$-ring spectrum defined by the sequence $(S \otimes_R R/I^n)$ is equivalent to the one defined by the sequence $(S/(SI)^n)$. In other words, this assignment of pro-$E_\infty$-ring spectra to pairs of a ring and a finitely generated ideal is preserved by base change.

Specializing this to the case $S = R/I^m$ for some $m \geq 1$ shows that the pro-$E_\infty$-ring defined by the sequence $(R/I^m \otimes_R R/I^n)$ is equivalent to $R/I^m$. It follows from this that the morphism $R \rightarrow \lim R/I^n$ induces an epimorphism of pro-$E_\infty$-ring spectra. In particular, if we assume that $R/I^n$ is an almost finitely presented $R$-module for all $n$ (which for instance holds whenever $R$ is Noetherian) then the sequence $R_n = R/I^n$ satisfies the conditions of proposition \ref{prop completes ffff}.
\end{remark}

If $R$ is a Noetherian commutative ring and $I \subseteq R$ is an ideal, then it follows from corollary \ref{coro completes right adjoint 1} that $M \otimes_R^L R^\wedge_I = \lim M \otimes^L_R R/I^n$ for every finitely generated $R$-module $M$. Combined with the fact that $R^\wedge_I$ is a flat $R$-module this implies that $\lim \Tor^R_s(M, R/I^n) = 0$ for all $s \geq 1$. A stronger claim is in fact true: the pro-$R$-module defined by the sequence $\Tor^R_s(M, R/I^n)$ vanishes for all $s \geq 1$. The following proposition generalizes this fact:
 
 \begin{proposition}\label{prop pro object 0 truncated}
 Let $R$ be a Noetherian commutative ring and $I\subseteq R$ be an ideal. Let $\ccal$ be an $R$-linear Grothendieck abelian category. Assume that $\ccal$ is generated by compact projective objects, proper over $R$, and that the set of points $x$ in $\Spec(R)$ such that $\ccal \otimes_R \kappa(x)$ is semisimple is dense in the Zariski topology. Let $X$ be a finitely generated object of $\ccal$. Then for every $s \geq 1$ the pro-object of $\ccal$ defined by the sequence $(\Tor_s(R/I^n, X))$ vanishes.
  \end{proposition}
\begin{proof}
Pick an epimorphism $Y \rightarrow X$ with $Y$ compact projective, and let $\Ccal'$ be the full subcategory of $\ccal$ generated by $Y$. Recall from proposition \ref{prop subcat closed under passage to subobjects} that $\ccal'$ is closed under passage to subobjects in $\ccal$. In particular the inclusion $\ccal' \rightarrow \ccal$ is left exact, so it commutes with $\Tor$. Replacing $\ccal$ with $\ccal'$ we may now assume that $Y$ is a compact projective generator for $\ccal$. In this case it is sufficient to prove that for every $s \geq 1$ the pro-$R$-module defined by the sequence 
\[
\Hom^\enh_\ccal(Y, \Tor_s(R/I^n, X)) = \Tor^R_s(R/I^n, \Hom^\enh_\ccal(Y, X))
\]
vanishes.  Replacing $\ccal$ by $\Mod_R^\heartsuit$ and $X$ by $\Hom^\enh_\ccal(Y, X)$ we may now reduce to the case when $\ccal = \Mod_R^\heartsuit$. The property that our pro-object vanishes is preserved by extensions in $X$, so it is enough to consider the case when $X = R/J$ for some ideal $J$. In this case the claim follows from remark \ref{remark base change pro object}.
\end{proof} 

%%%%%%%%%%%%%%%%%%%%%%%%%%%%%%%%%%%%%%%%%%%%%%%%%%%%%%%%%%%%%%%%%%%%%%%%
%%%%%%%%%%%%%%%%%%%%%%%%%%%%%%%%%%%%%%%%%%%%%%%%%%%%%%%%%%%%%%%%%%%%%%%%
%%%%%%%%%%%%%%%%%%%%%%%%%%%%%%%%%%%%%%%%%%%%%%%%%%%%%%%%%%%%%%%%%%%%%%%%
%%%%%%%%%%%%%%%%%%%%%%%%%%%%%%%%%%%%%%%%%%%%%%%%%%%%%%%%%%%%%%%%%%%%%%%%
%%%%%%%%%%%%%%%%%%%%%%%%%%%%%%%%%%%%%%%%%%%%%%%%%%%%%%%%%%%%%%%%%%%%%%%%
%%%%%%%%%%%%%%%%%%%%%%%%%%%%%%%%%%%%%%%%%%%%%%%%%%%%%%%%%%%%%%%%%%%%%%%%
 
\subsection{Fully dualizable \texorpdfstring{$(1,1)$}{(1,1)}-categories}\label{subsection fully dualizable 11}

The following is our main theorem concerning fully dualizable $R$-linear $(1,1)$-categories:

\begin{theorem}\label{theo abelian con coefficients}
Let $R$ be a G-ring and let $\acal$ be a symmetric monoidal $R$-linear Grothendieck abelian category. Assume the following:
\begin{itemize}
\item $\acal$ is rigid and generated by compact projective objects.
\item $\acal$ is proper over $R$.
\item The set of points $x$ in $\Spec(R)$ such that $\acal \otimes_R \kappa(x)$ is semisimple is dense in the Zariski topology.
\end{itemize}
Let $\ccal$ be a fully dualizable $\acal$-linear cocomplete category. Then there exists a faithfully flat \'etale morphism of commutative rings $R \rightarrow R'$ and a smooth and proper algebra $A$ in $\acal \otimes_R R'$  such that $\Ccal \otimes_{R} R'$ is equivalent to $\LMod_A(\acal \otimes_R R')$ as an $\acal \otimes_R R'$-linear category.
\end{theorem}

Before going into the proof, we record a few consequences.

\begin{corollary}\label{coro etale locally trivial}
Let $R$ be a  G-ring and let $\Ccal$ be an invertible $\Mod^\heartsuit_R$-linear cocomplete category. Then there exists a faithfully flat \'etale morphism of  commutative rings  $R \rightarrow R'$ such that $\Ccal \otimes_{R} R'$ is equivalent to $\Mod^\heartsuit_{R'}$ as an $R'$-linear category.
\end{corollary}
\begin{proof}
By theorem \ref{theo abelian con coefficients} we may after passing to a faithfully flat \'etale $R$-algebra assume that $\ccal$ is the category of left modules over an Azumaya $R$-algebra $A$. The corollary now follows from the fact that Azumaya $R$-algebras are \'etale locally Morita equivalent to the unit (see corollary \ref{coro smooth and proper locally trivial}).
\end{proof}

\begin{notation}\label{notation twist mod}
Let $\mathcal{L}: \operatorname{CRing}\rightarrow \Spc$ be the functor that associates to each commutative ring $R$ the space of $R$-linear Grothendieck abelian categories which are \'etale locally on $\Spec(R)$ equivalent to $\Mod_R^\heartsuit$. This is a sheaf for the \'etale topology, with a pointing given by the object $\Mod_\ZZ^\heartsuit$ in $\mathcal{L}(\ZZ)$. The resulting pointed object is equivalent to $B^2\GG_m$. If $\mathcal{G}$ is a $\GG_m$-gerbe  on $\Spec(R)$ we denote by $\Mod^\heartsuit_{R, \mathcal{G}}$ the induced point in $\mathcal{L}(\Spec(R))$.
\end{notation}

\begin{corollary}\label{coro exists gerbe invertible} 
Let $R$ be a  G-ring and let $\Ccal$ be an invertible $\Mod^\heartsuit_R$-linear cocomplete category. Then there exists a $\GG_m$-gerbe $\mathcal{G}$ on $\Spec(R)$ and an $R$-linear equivalence ${\ccal = \Mod^\heartsuit_{R, \mathcal{G}}}$
\end{corollary}
\begin{proof}
This is a direct consequence of corollary  \ref{coro etale locally trivial} and the definitions.
\end{proof}

\begin{corollary}\label{coro classify fully dualizables abelian} 
Let $R$ be a G-ring and let $\Ccal$ be a fully dualizable $\Mod^\heartsuit_R$-linear cocomplete category. Then there exists a finite \'etale $R$-algebra $\tilde{R}$, a $\GG_m$-gerbe $\mathcal{G}$ on $\Spec(\tilde{R})$ and an $R$-linear equivalence $\ccal = \Mod^\heartsuit_{\tilde{R}, \mathcal{G}} $.
\end{corollary}
\begin{proof}
Let $\tilde{R} = \End^\enh_{\Funct_R(\ccal, \ccal)}(\id_\ccal)$ be the $R$-linear center of $\ccal$. Then $\tilde{R}$ is a commutative $R$-algebra, and $\ccal$ may be equipped with a canonical $\tilde{R}$-linear structure. Since $\ccal$ is dualizable the formation of $\Funct_{R}(\ccal, \ccal)$ commutes with base change, and since $\id_\ccal$ is compact projective the formation of its endomorphisms commutes with base change as well. The fact that $\ccal$ is fully dualizable implies that $\tilde{R}$ is a dualizable $R$-module. The assertion that $\tilde{R}$ is \'etale may then be reduced by base change to the case where $R = k$ is an algebraically closed field. In this case theorem \ref{theo abelian con coefficients} implies that $\ccal$ is the category of left modules over a finite product of copies of $k$, which has  \'etale center.

It remains to show that $\ccal =  \smash{\Mod^\heartsuit_{\tilde{R}, \mathcal{G}}}$ for some gerbe $\mathcal{G}$ on $\Spec(\tilde{R})$.  By corollary \ref{coro exists gerbe invertible} it suffices to show that $\ccal$ is invertible over $\tilde{R}$.  This can be checked \'etale locally by virtue of the fact that $\ccal$ is a Grothendieck abelian category (corollary \ref{coro properties dualizable 11}). By an application of theorem \ref{theo abelian con coefficients} we may reduce to the case where  $\ccal$ is the category of left modules over a smooth and proper algebra $A$ in $\Mod_R^\heartsuit$. Applying corollary \ref{coro smooth and proper locally trivial} we may further reduce to the case when $A$ is a finite product of copies of $R$, in which case the assertion is clear.
\end{proof}

We devote the remainder of this section to the proof of theorem \ref{theo abelian con coefficients}.

\begin{definition}\label{definition elementary extension}
Let $f: R \rightarrow S$ be a morphism of local Artinian commutative rings, and let $\mathfrak{m}$ be the maximal ideal of $R$. We say that $f$ is an elementary extension if it is surjective and $\Ker(f)$ is isomorphic as an $R$-module to $R/\mathfrak{m}$.
\end{definition}

\begin{remark}
Let $f: R \rightarrow S$ be a morphism of local Artinian commutative rings and let $\mathfrak{m}$ be the maximal ideal of $R$. Then $f$ is an elementary extension if and only if it induces an isomorphism $S = R/Rx$ for some nonzero nonunit element $x$ in $R$ such that $x\mathfrak{m} = 0$.
\end{remark}

\begin{lemma}\label{lemma quotient is sequence of elementaries}
Let $f: R \rightarrow S$ be a surjective morphism of local Artinian commutative rings. Then there exists a sequence of elementary extensions of local Artinian commutative rings $R = R_0 \rightarrow R_1 \rightarrow R_2 \ldots \rightarrow R_n = S$.
\end{lemma}
\begin{proof}
Since $R$ is Noetherian any sequence of quotients of $R$ stabilizes. To prove the lemma it will therefore suffice to show that if we have a sequence of local Artinian commutative rings $R = R_0 \rightarrow R_1 \rightarrow \rightarrow R_2 \rightarrow \ldots \rightarrow R_k \rightarrow S$ such that $R_i \rightarrow R_{i+1}$ is an elementary extension for all $0 \leq i < k$, then either $R_k = S$ or there exists a factorization $R_k \rightarrow R_{k+1} \rightarrow S$ where the first map is an elementary extension. Replacing $R$ with $R_k$ we may reduce to showing that if $f$ is not an isomorphism then we have a factorization $R \rightarrow T \rightarrow S$ where the first map is an elementary extension. In this case $\Ker(f)$ is nonzero and since the only prime ideal of $R$ is $\mathfrak{m}$ and $R$ is Noetherian we see that $\mathfrak{m}$ is an associated prime of $\Ker(f)$. Let $x$ in $\Ker(f)$ be an element with annihilator $\mathfrak{m}$. Then the proof finishes by setting $T = R/Rx$.
\end{proof}

\begin{remark}\label{remark properties CS}
Let $f: R \rightarrow S$ be a surjective map of commutative rings, and let $\Ccal$ be an $R$-linear Grothendieck abelian category. Let $\Ccal_S = \Ccal \otimes_R S$. Then the extension of scalars functor $\Ccal \rightarrow \Ccal_S$ admits a fully faithful right adjoint $\iota$. The unit of this localization is given by tensoring with $f$, so that an object $X$ in $\Ccal$ belongs to the image of $\iota$ if and only if the map $R \otimes X \rightarrow S \otimes X$ is an isomorphism. We will often identify $\ccal_S$ with its image under $\iota$.

 For every object $X$ in $\Ccal$ we have an exact sequence
\[
\Ker(f) \otimes X \rightarrow R \otimes X \rightarrow S \otimes X \rightarrow 0
\]
so that $X$ belongs to $\ccal_S$ if and only if the first map above is zero. It follows that $x_\alpha$ is a set of generators for the $R$-module $\Ker(f)$ then $X$ belongs to $\ccal_S$ if and only if $x_\alpha: X \rightarrow X$ is zero for all $\alpha$.

It follows from the above description that $\ccal_S$ is closed under limits, colimits and passage to subobjects inside $\ccal$ (however note that it is not closed under passage to extensions in general). Furthermore, $\iota$ admits a right adjoint which sends each object $X$ to the intersection over all $\alpha$ of the kernel of $x_\alpha: X \rightarrow X$. In particular, if $\Ker(f)$ is finitely generated then the right adjoint to $\iota$ preserves filtered colimits, which implies that $\iota$ sends compact objects to compact objects.
\end{remark}

\begin{lemma}\label{lemma exist filtration}
Let $R$ be a local Artinian commutative ring with residue field $k$. Let $\ccal$ be an $R$-linear Grothendieck abelian category and let $X$ be an object of $\ccal$. Then there exists a finite filtration $0 = X_0 \subseteq X_1 \subseteq \ldots \subseteq X_t = X$ such that $X_{i+1} / X_i$ belongs to $\Ccal \otimes_R k$ for all $i$.
\end{lemma}
\begin{proof}
By lemma \ref{lemma quotient is sequence of elementaries} we may pick a sequence of elementary extensions of local Artinian commutative rings $R = R_0 \rightarrow R_1 \rightarrow \ldots \rightarrow R_n = k$. We will prove that the lemma holds for the rings $R_s$ by reverse induction on $s$. The case $s = n$ is clear. Assume now that $s < n$ and that the lemma is known for $R_{s+1}$. Let $x$ be a generator for the kernel of $R_s \rightarrow R_{s+1}$. Then we have an exact sequence $0 \rightarrow X' \rightarrow X \rightarrow X'' \rightarrow 0$ where $X' = \Ker(x: X \rightarrow X)$ and $X'' = \operatorname{Im}(x: X \rightarrow X)$. Since $x^2 = 0$ we have that both $X'$ and $X''$ belong to $\ccal \otimes_{R_{s}} R_{s+1}$.  The inductive hypothesis allows us to construct filtrations for $X'$ and $X''$ whose associated graded pieces belong to $\Ccal \otimes_R k$. We now obtain the desired filtration on $X$ by putting together these two filtrations.
\end{proof}

\begin{lemma}\label{lemma construct non split extension}
Let $R$ be a local Artinian commutative ring with residue field $k$.  Let $\Ccal$ be an $R$-linear Grothendieck abelian category such that $\ccal \otimes_R k$ is semisimple, and let $X$ be an object of $\ccal$.  Then one of the following two happen:
\begin{enumerate}[\normalfont (a)]
\item $X$ is projective.
\item There exists a non-split extension of $X$ by a simple object of $\ccal \otimes_R k$.
\end{enumerate}
\end{lemma}
\begin{proof}
Assume that $X$ is not  projective in $\ccal$. We will show that (b) holds. Choose  $M$ in $\Ccal$ such that $\Ext_\ccal^1(X, M) \neq 0$. By lemma \ref{lemma exist filtration} we may pick a filtration $0 = M_0 \subseteq M_1 \subseteq \ldots \subseteq M_t = M$ with $M_{i+1}/M_i$ in $\ccal_k$ for all $i$. Let $j$ be the smallest index such that $\Ext_\ccal^1(M, M_j) \neq 0$. Then $\Ext_\ccal^1(X, M_{j}/M_{j-1}) \neq 0$. Replacing $M$ by $M_{j}/M_{j-1}$ if necessary we may assume that $M$ belongs to $\ccal_k$. 

Since $\ccal$ is semisimple we may write $M$ as a direct sum of a family of simple objects $S_i$. Then $M$ is a direct summand of $\prod_i S_i$, and therefore $\Ext_\ccal^1(X, \prod_i S_i) \neq 0$. Set $W = \prod_i S_i$, and let $Z$ be the product of the family $S_i$ computed in the derived category $\der(\ccal)$. Then $W = \tau_{\geq 0}Z$, so we have an exact sequence 
\[
\Ext^0_{\der(\ccal)}(X, \tau_{\leq -1}Z) \rightarrow \Ext^1_{\der(\ccal)}(X, W) \rightarrow \Ext^1_{\der(\ccal)}(X, Z).
\]
Here the first term vanishes, and since the middle term is nonzero we conclude that the third term is nonzero. This is the same as $\prod_i \Ext_\ccal^1(X, S_i)$, so   we have that $\Ext_\ccal^1(X, S_i) \neq 0$ for some $i$, and therefore a non-split extension of $X$ by $S_i$ exists, as desired.
\end{proof}

\begin{lemma}\label{lemma flat into local artinian}
Let $R$ be a local Artinian commutative ring with residue field $k$ and let $\Ccal$ be an $R$-linear Grothendieck abelian category. Let $X$ be an object of $\ccal$. The following are equivalent:
\begin{enumerate}[\normalfont(1)]
\item $X$ is flat over $R$.
\item The map $\mu: \mathfrak{m} \otimes X \rightarrow R \otimes X = X$ induced from the inclusion $\mathfrak{m} \rightarrow R$ is a monomorphism.
\item $\Tor_1(k, X) = 0$.
\end{enumerate}
\end{lemma}
\begin{proof}
The fact that (1) implies (2) follows directly from the definitions. The equivalence of (2) and (3) follows  from the fact that we have an exact sequence
\[
0 = \Tor_1(R, X) \rightarrow \Tor_1(k, X) \xrightarrow{} \Tor_0(\mathfrak{m}, X) \rightarrow \Tor_0(R, X).
\]
Assume now that (3) holds. Since the property that $\Tor_1(Y, X) = 0$ is preserved under passage to extensions and filtered colimits and $\Mod^\heartsuit_{R}$ is generated by $k$ under extensions and filtered colimits we see that $\Tor_1(Y, X) = 0$ for all $Y$ in $\Mod^\heartsuit_{R}$. Assume now given a monomorphism $i: Z \rightarrow Z'$ in $\Mod^\heartsuit_{R}$. Then the kernel of $i \otimes \id_X : Z \otimes X \rightarrow Z' \otimes X$ receives an epimorphism from $\Tor_1(Z'/Z, X)$, and is therefore $0$. This proves that (1) holds.
\end{proof}

\begin{lemma}\label{lemma construct flat generator}
Let $f: R \rightarrow S$ be an elementary extension of local Artinian commutative rings with residue field $k$. Let $\Ccal$ be an $R$-linear Grothendieck abelian category such that $\ccal \otimes_R k$ is semisimple. Assume given a non-split extension
\[
0 \rightarrow M \rightarrow U \rightarrow X \rightarrow 0
\]
in $\ccal$, where $M$ is a simple object of $\Ccal \otimes_R k$ and $X$ is a projective object of $\ccal \otimes_R S$, flat over $S$, and such that $X \otimes_R k$ is simple. Then $U$ is projective, flat over $R$, and $U \otimes_R S = X$.
\end{lemma}
\begin{proof}
Let $\mathfrak{m}$ be the maximal ideal of $R$. Set $\Ccal_S = \Ccal \otimes_R S$ and $\ccal_k = \ccal \otimes_R k$.  Fix a generator $x$ for $\Ker(f)$. Since $X$ is projective in $\ccal_S$ we see that $U$ cannot belong to $\ccal_S$, and hence $x: U \rightarrow U$ is nonzero.  Since $x$ acts by zero on $X$ we have that $x: U \rightarrow U$ factors through $M$. Its image is a nonzero subobject of $M$, and since $M$ is simple we conclude that the image of $x: U \rightarrow U$ is equal to $M$. In particular, we claim:
\begin{itemize}
\item [$(\star)$] If  $N$ is an $R$-module on which $x$ acts by zero, the map $N \otimes M \rightarrow N \otimes U$ obtained by tensoring $N$ with the inclusion $M \rightarrow U$ is zero.
\end{itemize}
To see this, note that it is enough to show that $\id \otimes x : N \otimes U \rightarrow N \otimes U$ vanishes, which follows from the fact that this map is equivalent to $x \otimes \id: N \otimes U \rightarrow N \otimes U$.

By lemma \ref{lemma flat into local artinian}, to show that $U$ is flat over $R$ it suffices to show that the map $j: \mathfrak{m} \otimes U \rightarrow R \otimes U = U$ obtained by tensoring the inclusion $\mathfrak{m} \rightarrow R$ with $U$ is a monomorphism. Let $\mathfrak{m}_S = \mathfrak{m}/Rx$. We have a commutative diagram in $\Ccal$ with exact rows as follows:
 \[
 \begin{tikzcd}
& Rx \otimes U \arrow{d}{j_1} \arrow{r}{} & \mathfrak{m} \otimes U \arrow{r}{} \arrow{d}{j} & \mathfrak{m}_S \otimes U \arrow{d}{j_2} \arrow{r}{} & 0 \\
0 \arrow{r}{} & M \arrow{r}{} & U \arrow{r}{} & X \arrow{r}{} & 0
\end{tikzcd}
\]
To show that $j$ is a monomorphism it will suffice to show that $j_1$ and $j_2$ are monomorphisms.

We first consider $j_1$. Our assumptions guarantee that $Rx$ is isomorphic to $k$ as an $R$-module, so that $Rx \otimes U$ belongs to $\Ccal \otimes_R k$. The same holds for $M$. To show that $j_1$ is a monomorphism it will suffice to prove the following two assertions:
\begin{enumerate}[(a)]
\item $j_1$ is nonzero.
\item $Rx \otimes U$ is simple.
\end{enumerate}
We begin by addressing (a). To show that $j_1$ is nonzero it suffices to show that the map $Rx \otimes U \rightarrow R \otimes U = U$ obtained by tensoring the inclusion $Rx \rightarrow R$ with $U$ is nonzero. To do so it suffices to show that the map
\[
U = R \otimes U \xrightarrow{x \otimes \id_U} R \otimes U = U
\]
is nonzero . This is the same as the action of $x$ on $U$, which we have already shown to be nonzero.

It remains to address (b). Consider the exact sequence
\[
Rx \otimes M \rightarrow Rx \otimes U \rightarrow Rx \otimes X \rightarrow 0.
\]
Here $Rx \otimes X = k \otimes X$ is simple by our assumption on $X$. We may thus reduce to proving that the map $Rx \otimes M \rightarrow Rx \otimes U$ is zero. This follows from $(\star)$ since $x$ acts by zero on $Rx$. 

We now show that $j_2$ is a monomorphism. This is the composition of the map $\mu_1: \mathfrak{m}_S \otimes U \rightarrow \mathfrak{m}_S \otimes X$ obtained by tensoring $\mathfrak{m}_S$ with the projection $U \rightarrow X$, and the map $\mu_2: \mathfrak{m}_S \otimes X \rightarrow S \otimes X = X$ obtained by tensoring the inclusion $\mathfrak{m}_S \rightarrow S$ with $X$. The kernel of $\mu_1$ is the image of the map $\mathfrak{m}_S \otimes M \rightarrow \mathfrak{m}_S \otimes U$ obtained by tensoring $\mathfrak{m}_S$ with the inclusion $M \rightarrow U$. It now follows from $(\star)$ that $\mu_1$ is a monomorphism, since $x$ acts by $0$ on $\mathfrak{m}_S$. The map $\mu_2$ is a monomorphism by virtue of our assumption that $X$ is flat over $S$. We conclude that $j_2$ is a monomorphism, as desired.


We now show that $U$ is projective. We will do so by proving that $\Ext^1_\ccal(U, Y) = 0$ for all $Y$ in $\ccal$. By lemma \ref{lemma exist filtration} it suffices to address the case when $Y$ belongs to $\ccal_k$. Then we have
\[
\Ext^1_\ccal(U, Y) = \Ext^1_{\der(\ccal)}(U, Y) = \Ext^1_{\der(\ccal)\otimes_R k}( U \otimes_R k, Y) = \Ext^1_{\ccal \otimes_R k}(U \otimes_R k, Y)
\]
where here we use flatness of $U$ to identify $U \otimes_R^L k$ with $U \otimes_R k$. Recall from our proof of (a) that $U \otimes_R k = X \otimes_R k$. The fact that the above group vanishes is now a consequence of the fact that $X \otimes_R k$ is projective in $\ccal \otimes_R k$.

It remains to prove that $U \otimes_R S = X$. Consider the exact sequence
\[
S \otimes M \rightarrow S \otimes U \rightarrow S \otimes X \rightarrow 0.
\]
It suffices to show that the first map is zero. This follows from an application of $(\star)$, since $x$ acts by zero on $S$. 
\end{proof}

\begin{lemma}\label{lemma defo is compact}
Let $f: R \rightarrow S$ be a  surjective morphism of local Artinian commutative rings and let $\ccal$ be an $R$-linear Grothendieck abelian category. Let $X$ be a projective object of $\ccal$ such that $X \otimes_R S$ is compact in $\ccal \otimes_R S$. Then $X$ is compact in $\ccal$.
\end{lemma}
\begin{proof}
Applying lemma \ref{lemma quotient is sequence of elementaries} we may reduce to the case when $f$ is an elementary extension. Let $x$ be a generator of $\Ker(f)$. Let $Y_\alpha$ be a filtered diagram of objects of $\ccal$. For each $\alpha$ let $Y_\alpha' = \Ker(x: Y_\alpha \rightarrow Y_\alpha)$ and $Y_\alpha'' = \operatorname{Im}(x:Y_\alpha \rightarrow Y_\alpha)$. Then we have a commutative diagram of $R$-modules with exact rows
\[
\begin{tikzcd}[column sep = small]
0 \arrow{r}{} & \colim \Hom^\enh_\ccal(X, Y_\alpha') \arrow{r}{} \arrow{d}{} & \colim \Hom^\enh_\ccal(X, Y_\alpha) \arrow{r}{} \arrow{d}{} & \colim \Hom^\enh_\ccal(X, Y_\alpha')  \arrow{r}{} \arrow{d}{} & 0 \\
0 \arrow{r}{} &  \Hom^\enh_\ccal(X, \colim Y_\alpha')  \arrow{r}{} &  \Hom^\enh_\ccal(X,\colim Y_\alpha) \arrow{r}{}  &  \Hom^\enh_\ccal(X, \colim Y_\alpha'') \arrow{r}{} & 0.
\end{tikzcd}
\]
We wish to prove that the middle vertical arrow is an isomorphism. This will follow if we can prove that the other two vertical arrows are isomorphisms. This follows from the fact that $X \otimes_R S$ is compact in $\ccal \otimes_R S$, since $Y_\alpha'$ and $Y_\alpha''$ belong to $\ccal \otimes_R S$.
\end{proof}

\begin{lemma}\label{lemma deform generators abelian}
Let $f: R \rightarrow S$ be surjective map of local Artinian commutative rings with residue field $k$. Let $\acal$ be a symmetric monoidal $R$-linear Grothendieck abelian category. Assume that $\acal$ is rigid, generated by compact projective objects, proper over $R$, and that $\acal \otimes_R k$ is semisimple. Let $\Ccal$ be a fully dualizable $\acal$-linear Grothendieck abelian category. Assume given a finite family $\lbrace X_t \rbrace$ of objects of $\ccal \otimes_R S$ with the following properties:
\begin{itemize}
\item $X_t$ is compact projective and flat over $S$ for all $t$.
\item $X_t \otimes_S k$ is simple for all $t$.
\item $\bigoplus X_t$ is an $\acal \otimes_R S$-generator for $\ccal \otimes_R S$.
\end{itemize}
Then there exist a family of objects $X'_t$ of $\ccal$ with the following properties:
\begin{itemize}
\item $X'_t \otimes_R S = X_t$ for all $t$.
\item $X'_t$ is compact projective and flat over $R$ for all $t$.
\item  $\bigoplus X'_t$ is an $\acal$-generator for $\ccal$.
\end{itemize}
\end{lemma}
\begin{proof}
Applying lemma \ref{lemma quotient is sequence of elementaries} we may reduce to the case when $f$ is an elementary extension. Let $\ccal_S$, $\ccal_k$, $\acal_S$ and $\acal_k$ the base changes of $\ccal$ and $\acal$. By theorem \ref{theorem abelian} combined with remark \ref{remark separable are semisimple} we see that $\ccal_k$ is semisimple. Define for each $t$ an object $X'_t$ of $\ccal$, as follows:
\begin{itemize}
\item If $X_t$ is projective in $\ccal$ then $X'_t = X_t$.
\item If $X_t$ is not projective in $\ccal$, then using lemma \ref{lemma construct non split extension} construct a non-split extension $0 \rightarrow M_t \rightarrow U_t \rightarrow X_t \rightarrow 0$ in $\ccal$ where $M_t$ is simple in $\ccal_t$, and set $X'_t = U_t$. By lemma \ref{lemma construct flat generator} we have that $X'_t$ is  flat over $R$, projective, and satisfies $X'_t \otimes_R S = X_t$.
\end{itemize}

Lemma \ref{lemma defo is compact} implies that $X'_t$ is in fact compact projective for every $t$. We claim that  $X' = \bigoplus X'_t$ is an $\acal$-generator for $\ccal$. Let $Y$ be an arbitrary object of $\ccal$. By lemma \ref{lemma exist filtration} we may pick a filtration $0 = Y_0 \subseteq Y_1 \subseteq \ldots \subseteq Y_n = Y$ with successive quotients in $\ccal_k$. Pick for each $i$ an object $Z_i$ in $\acal_k$ and an epimorphism $Z_i \otimes (X' \otimes_R k) \rightarrow Y_{i+1}/Y_i$ in $\ccal_k$ (which exists since $X' \otimes_R S = \bigoplus X_t$ is an $\acal_S$-generator for $\ccal_S$). For each $i$ pick an epimorphism $Z'_i \rightarrow Z_i$ in $\acal$ with $Z'_i$ projective. Since $X'$ is projective the induced maps $Z'_i \otimes X' \rightarrow Y_{i+1}/Y_i$ may be lifted to a sequence of morphisms $Z'_i \otimes X' \rightarrow Y$. The resulting map $\bigoplus Z'_i \otimes X'\rightarrow Y$ is then an epimorphism. Since $Y$ was arbitrary we conclude that $X'$ is an $\acal$-generator for $\ccal$, as desired.

To prove the proposition it now suffices to show that for all $t$ the object $X_t$ is not projective in $\ccal$ (so that $X'_t = U_t$ is flat over $R$ for all $t$).  Assume for the sake of contradiction that there is an index $t$ such that $X_t$ is projective in $\ccal$. Since $\ccal$ admits a compact projective $\acal$-generator, it is generated by compact projective objects. Let $A$ be the algebra of endomorphisms of $X_t$ in $\acal$. Applying remark \ref{remark properness when comp gen} we see that $A$ is compact projective as an object of $\acal$, and therefore the $R$-module $\Hom^\enh_\acal(1_\acal, A)$ is compact projective.  This coincides with the $R$-module of endomorphisms of $X_t$, which is an $S$-module since $X_t$ belongs to $\ccal_S$. It follows that $\Hom^\enh_\acal(1_\acal, A) = 0$ and hence $X_t = 0$. This contradicts the fact that $X_t \otimes_S k$ is simple (and in particular nontrivial).
\end{proof}

\begin{lemma}\label{lemma compactness y completion}
Let $R$  complete local Noetherian commutative ring with maximal ideal $\mathfrak{m}$ and residue field $k$. Let $\ccal$ be an $R$-linear Grothendieck abelian category, generated by compact projective objects and proper over $R$. Let $X$ be an object of $\ccal$.
\begin{enumerate}[\normalfont(1)]
\item If $X$ is compact projective then $X = \lim X \otimes_R R/\mathfrak{m}^n$.
\item If $X = \lim X \otimes^L_R R_n$ and $X \otimes^L_R k$ is compact projective in $\der(\ccal)_{\geq 0} \otimes_R k$, then $X$ is compact projective.
\end{enumerate}
\end{lemma}
\begin{proof}
We first prove part (1). Set $R_n = R/\mathfrak{m}^n$ for all $n \geq 1$.  Let $\eta: X \rightarrow \lim X \otimes_R R_n$ be the canonical map. We wish to show that $\eta$ is an isomorphism. It suffices for this to prove that $\Hom^\enh_\ccal(Y, \eta)$ is an isomorphism for all compact projective objects $Y$ in $\ccal$. This is equivalent to the canonical map
\[
\Hom^\enh_\ccal(Y, X) \rightarrow \lim \Hom^\enh_{\ccal \otimes_R R_n}(Y \otimes_R R_n, X \otimes_R R_n)
\]
which is, in turn, equivalent to the map
\[
\Hom^\enh_\ccal(Y, X) \rightarrow \lim \Hom^\enh_{\ccal}(Y, X) \otimes_R R_n.
\]
The above is an equivalence since $\Hom_{\Mcal}^\enh(Y, X)$ is a dualizable $R$-module.

We now prove part (2).  By  proposition \ref{prop extend compact projectives} we may pick a compact projective object $X'$ in $\ccal$ whose image in $\ccal \otimes_R k$ is isomorphic to $X \otimes_R k$.  The fact that $X'$ is projective allows us to lift this isomorphism to a morphism $f: X' \rightarrow X$ in $\ccal$. Let $Z$ be the cofiber of $f$ inside $\der(\ccal)_{\geq 0}$, and note that $Z \otimes^L_R k = 0$. It follows from this that $Z \otimes^L_R R_n = 0$ for all $n \geq 1$. Combining this with an application of part (1) to $X'$, we see that $\Ext^1_{\der(\ccal)}(Z, X') = 0$. Hence $Z$ is a retract of  $X$, which implies $Z = \lim Z \otimes^L_R R_n = 0$. We conclude that $f$ is an isomorphism, and therefore $X$ is compact projective, as desired.
\end{proof}

\begin{lemma}\label{lemma deal with completes classical}
Let $R$ be a complete local Noetherian ring with residue field $k$ and let $\acal$ be a symmetric monoidal $R$-linear Grothendieck abelian category. Assume that $\acal$ is rigid, generated by compact projective objects, proper over $R$, and that $\acal \otimes_R k$ is semisimple. Let $\ccal$ be a fully dualizable $\acal$-linear Grothendieck abelian category. Then $\ccal$ admits a compact projective $\acal$-generator.
\end{lemma}
\begin{proof}
Let $\mathfrak{m}$ be the maximal ideal of $R$. For each $n \geq 1$ set $R_n = R/\mathfrak{m}^n$, $\acal_n = \acal \otimes_R R_n$ and $\Ccal_n = \Ccal \otimes_R R_n$. By theorem \ref{theorem abelian} combined with remark \ref{remark separable are semisimple} we see that $\ccal_1$ is semisimple and admits a compact projective $\acal_1$-generator $X_1$. Since $X_1$ is a finite sum of simple objects we may apply lemma \ref{lemma deform generators abelian} inductively to find a compatible sequence of compact projective $\acal_n$-generators $X_n$ in $\ccal_n$, flat over $R_n$. 

Set $X = \lim X_n$.  Embed $\ccal$ inside $\der(\ccal)_{\geq 0}$, and note that we have for each $n$ an equivalence $(\der(\ccal)_{\geq 0} \otimes_R R_n)^\heartsuit = \ccal_n$. Since $X_n$ is flat over $R_n$ we see that its image inside $\der(\ccal)_{\geq 0} \otimes_R R_n$ is also flat over $R_n$. Hence the sequence $(X_n)$ defines an object of $\lim \der(\ccal)_{\geq 0}  \otimes_R R_n$. We may identify $X$ with the image of $(X_n)$ under the right adjoint to the projection $p: \der(\ccal)_{\geq 0} \rightarrow \lim \der(\ccal)_{\geq 0}  \otimes_R R_n$. Applying corollary \ref{coro properties dualizable 11} we see that products in $\ccal$ are exact, and therefore by proposition \ref{prop completes ffff} we have that $X \otimes_R^L R_n = X_n$ for all $n \geq 1$. In particular, $X \otimes_R R_n = X_n$ for all $n$.

We will show that $X$ is a compact projective $\acal$-generator for $\ccal$. We begin by proving that it is an $\acal$-generator. Changing the roles of $\Ccal$ and $\ccal^\vee$ in the previous arguments we see that  $\ccal^\vee$ contains an object $Y$ such that $Y_n = Y \otimes_R R_n$  is a compact projective $\acal_n$-generator for $\ccal^\vee_n$ for all $n$.  Consider the object $X \otimes Y$ inside $\Ccal \otimes_\acal \ccal^\vee$, and note that $(X \otimes Y) \otimes_R R_n = X_n \otimes Y_n$ is a compact projective $\acal_n$-generator for  $\ccal_n \otimes_{\acal_n} \ccal^\vee_n$ for all $n$.

Let $\delta$ be the image of the unit $1_\acal$ under the unit map $\eta: \acal \rightarrow \ccal \otimes_\acal \ccal^\vee$, and note that $\delta$ is compact projective since $\ccal$ is smooth. For each $n \geq 1$ set $\delta_n = \delta \otimes_R R_n$. By lemma \ref{lemma compactness y completion} applied to $\acal$ we have $1_\acal = \lim 1_{\acal} \otimes_R R_n$.  Since $\ccal$ is fully dualizable the map $\eta$ admits a left adjoint, and in particular preserves limits. It follows that $\delta = \lim \delta_n$. Fix an epimorphism $Z \otimes (X_1 \otimes Y_1) \rightarrow  \delta_1$, where $Z$ is a compact projective object of $\acal_1$. By proposition \ref{prop extend compact projectives} we may find a compact projective object $Z'$ in $\acal$ whose image in $\acal_1$ recovers $Z$.  Replacing $Y$ with $Y \oplus Z' \otimes Y$ we may now assume the existence of an epimorphism $\rho_1 : X_1 \otimes Y_1 \rightarrow \delta_1$. Using the fact that $X_n \otimes Y_n$ is projective for all $n$ we may construct inductively a compatible sequence of maps $\rho_n:X_n \otimes Y_n \rightarrow \delta_n$, which in the limit defines a morphism $\rho: X \otimes Y  \rightarrow \delta$. 

We claim that $\rho$ admits a section. To prove this it suffices to show that the induced map 
\[
\rho_*: \Hom^\enh_{\ccal \otimes_\acal \ccal^\vee}(\delta, X \otimes Y ) \rightarrow \Hom^\enh_{\ccal \otimes_\acal \ccal^\vee}(\delta, \delta)
\] is an epimorphism in $\acal$. Using corollary \ref{coro check epi on fiber} we may reduce to proving that $\rho_* \otimes_R k$ is an epimorphism in $\acal_1$. Since $\delta$ is compact projective we have that $\rho_* \otimes_R k$ is equivalent to the map 
\[
(\rho_1)_* : \Hom^\enh_{\ccal_1 \otimes_{\acal_1} \ccal_1^\vee}(\delta_1, X_1 \otimes Y_1 ) \rightarrow \Hom^\enh_{\ccal_1 \otimes_{\acal_1} \ccal_1^\vee}(\delta_1, \delta_1)
\]
which is an epimorphism since $\delta_1$ is projective and $\rho_1$ is an epimorphism.

Let $\ccal'$ be the smallest subcategory of $\ccal$ closed under colimits, the action of $\acal$, and containing $X$. Then the induced map $\ccal' \otimes_\acal \ccal^\vee \rightarrow \ccal \otimes_\acal \ccal^\vee$ is fully faithful. Its image contains $X \otimes Y$ and is closed under retracts, so it also contains $\delta$. It now follows that the identity of $\ccal$ belongs to the image of the functor $\Funct_{\acal}(\ccal, \ccal') \rightarrow \Funct_{\acal}(\ccal, \ccal)$ of composition with the inclusion $i: \ccal' \rightarrow \ccal$. Therefore $i$ admits a section, which implies that $\ccal' = \ccal$. This concludes that proof that $X$ is an $\acal$-generator for $\ccal$.

It remains to show that $X$ is compact projective. Let $A$ be (the opposite of) the endomorphism algebra of $X$ inside $\acal$. By proposition \ref{prop lex localization prestable} the functor of tensoring with $X$ yields an $\acal$-linear left exact localization $q: \LMod_A(\acal) \rightarrow \ccal$, which is an equivalence if and only if $X$ is compact projective. We will finish the proof by showing that $q$ is an equivalence. Since $q$ is left exact it is enough to prove that if $M$ is a $0$-truncated left $A$-module such that $q(M) = 0$ then $M = 0$. Since $\LMod_A(\acal)$ is generated by compact projective objects it is enough to show that if $N$ is a finitely generated subobject of $M$ then $N = 0$. The fact that $q$ is left exact implies that $q(N) = 0$. Replacing $M$ by $N$ we may now reduce to the case when $M$ is finitely generated as a left $A$-module.

For each $n$ let $A_n$ be (the opposite of) the endomorphism algebra of $X_n$ inside $\acal_n$, and observe that we have an equivalence of algebras $A = \lim A_n$ (where here we regard $A_n$ as an algebra in $\acal$ via restriction of scalars). The fact that $X_n$ is compact projective for all $n$ implies that the sequence of algebras $A_n$ is compatible with base changes. Since $\LMod_{A_n}(\acal_n)$ is a fully dualizable $\acal_n$-linear category, we see that $A_n$ is proper, and in particular flat. It follows from this that $(A_n)$ defines an object in $\lim \der(\acal_n)_{\geq 0}$. An application of  proposition \ref{prop completes ffff} now shows that $A \otimes^L_R R_n = A_n$ for all $n \geq 1$. By virtue of part (2) of lemma \ref{lemma compactness y completion} we have that $A$ is compact projective as an object of $\acal$, and therefore  $M$ is finitely generated as an object of $\acal$. 

Consider now the commutative square of categories
\[
\begin{tikzcd}
\LMod_A(\acal) \arrow{d}{q} \arrow{r}{f^*} & \LMod_{A_1}(\acal_1) \arrow{d}{q_1} \\
\Ccal \arrow{r}{f^*} & \Ccal_1
\end{tikzcd}
\]
obtained from $q$ by tensoring with $\Mod_R^\heartsuit \rightarrow \Mod_{R_1}^\heartsuit$. Observe that this is horizontally right adjointable. Denote by $f_*$ the right adjoints to the horizontal arrows. Then $f_*q_1(M \otimes_R k) = q f_* (M \otimes_R k) = q(M)\otimes_R k = 0$, and since restriction of scalars is conservative we have $q_1(M \otimes_R k) = 0$. We may identify the functor $q_1$ with the functor of tensoring with the right $A_1$-module $X_1$ in $\Ccal_1$, which is an equivalence by virtue of the fact that $X_1$ is a compact projective $\acal_1$-generator for $\ccal_1$. It follows that $M \otimes_R k = 0$. The fact that $M = 0$ now follows from an application of proposition  \ref{proposition nakayama}.
\end{proof}

\begin{lemma}\label{lemma deal with filtered colimits classical}
Let $R_\alpha$ be a filtered diagram of commutative rings with colimit $R$. Assume given an index $\alpha_0$, a symmetric monoidal $R_{\alpha_0}$-linear Grothendieck abelian category $\acal_{\alpha_0}$ rigid and generated by compact projective objects, and a smooth $\acal_{\alpha_0}$-linear Grothendieck abelian category $\Ccal_{\alpha_0}$. If $\Ccal_{\alpha_0} \otimes_{R_{\alpha_0}} R$ has a compact projective $\acal_{\alpha_0} \otimes_{R_{\alpha_0}} R$-generator then there exists a transition $\alpha_0 \rightarrow \alpha$ such that $\Ccal_{\alpha_0} \otimes_{R_{\alpha_0}} R_\alpha$ has a compact projective $\acal_{\alpha_0} \otimes_{R_{\alpha_0}} R_\alpha$-generator.
\end{lemma}
\begin{proof}
Without loss of generality we assume that $\alpha_0$ is an initial index. For each $\alpha$ let $\acal_\alpha$ and $\ccal_\alpha$ be the base changes of $\acal_{\alpha_0}$ and $\ccal_{\alpha_0}$ to $R_\alpha$, and similarly denote by $\acal$ and $\ccal$ the base changes to $R$. Observe that we have $\Mod_R^\heartsuit = \colim \Mod_{R_\alpha}^\heartsuit$ in $\Pr^L$ (see the argument from \cite{HA} lemma 7.3.5.12), and therefore $\acal = \colim \acal_\alpha$ and $\ccal = \colim \ccal_\alpha$. 

Since $\ccal_\alpha$ is a dualizable $\acal_\alpha$-linear category for all $\alpha$ we see that the categories $\ccal_\alpha$ are $1$-strongly compactly assembled. Similarly, $\ccal$ is $1$-strongly compactly assembled. Let $X$ be a compact projective $\acal$-generator for $\ccal$. Applying theorem \ref{theorem lift strongly compacts} we may pick an index $\alpha$ and a compact projective lift $X_\alpha$ of $X$ to $\ccal_\alpha$. Restricting our diagram to the undercategory of $\alpha$ if necessary we may without loss of generality assume that $\alpha = \alpha_0$ is an initial index.

For each $\beta$ let $X_\beta = X_\alpha \otimes_{R_\alpha} R_\beta$. We will finish the proof by showing that there exists an index $\beta$ such that $X_\beta $ is an $\acal_\beta$-generator for $\ccal_\beta$. For each $\beta$ let $\dcal_{\beta}$ be the smallest subcategory of $\ccal_\beta$ containing $X_\beta$ and closed under colimits and the action of $\acal_\beta$. Then the action of $\acal_{\beta}$ on $\ccal_\beta$ restricts to give $\dcal_{\beta}$ an $\acal_\beta$-linear structure. Note that for every transition $\beta \rightarrow \beta'$ we have $\dcal_{\beta'} = \dcal_{\beta} \otimes_{R_\beta} R_{\beta'}$. Furthermore, the base change of these to $R$ recovers $\ccal$, and in particular we have $\ccal \otimes_\acal \ccal^\vee = \colim \dcal_\beta \otimes_{\acal_\beta} \ccal_\beta^\vee$. Let $\delta$ be the image of $1_\acal$ under the unit map $\acal \rightarrow \ccal \otimes_\acal \ccal^\vee$, and for each $\beta$ let $\delta_\beta$ be the image of $1_{\acal_\beta}$ under the unit map $\acal_\beta \rightarrow \ccal_\beta \otimes_{\acal_\beta} \ccal_\beta^\vee$. 
 Another application of theorem  \ref{theorem lift strongly compacts} shows that there exists $\beta$ and a compact projective lift $Y_\beta$ of $\delta$ to $ \dcal_\beta \otimes_{\acal_\beta} \ccal_\beta^\vee$. The image of $Y_\beta$ inside $\ccal_\beta \otimes_{\acal_\beta} \ccal_\beta^\vee$ is a compact projective object whose image in $\ccal \otimes_\acal \ccal^\vee$ agrees with the image of $\delta_\beta$. A final application of theorem \ref{theorem lift strongly compacts} shows that (changing $\beta$ if necessary) we may assume that $Y_\beta = \delta_\beta$. Hence  $\dcal_\beta \otimes_{\acal_\beta} \ccal_\beta^\vee$ contains $\delta_\beta$, which implies that the image of the inclusion $\Funct_{\acal_\beta}(\ccal_\beta, \dcal_\beta) \rightarrow \Funct_{\acal_\beta}(\ccal_\beta, \ccal_\beta)$ contains the identity. It follows that the inclusion $\dcal_\beta \rightarrow \ccal_\beta$ has a section, and therefore $\ccal_\beta = \dcal_\beta$. This shows that $X_\beta$ is an $\acal_\beta$-generator for $\ccal_\beta$, as desired.
\end{proof} 
 
  \begin{lemma}\label{lemma deal with products}
Let $\lbrace R_i \rbrace$ be a finite family of connective $E_\infty$-rings with product $R$. Let $\acal$ be a symmetric monoidal $R$-linear Grothendieck abelian category, rigid and generated by compact projective objects. Let $\ccal$ be an $\acal$-linear Grothendieck abelian category. Assume that  $\ccal \otimes_R R_i$ admits a compact projective $\acal\otimes_R R_i$-generator for all $i$.  Then there exists an algebra $A$ in $\acal  $  such that $\Ccal  $ is equivalent to $\LMod_A(\acal  )$ as an $\acal $-linear category.
 \end{lemma}
 \begin{proof}
 By proposition \ref{prop lex localization prestable} it suffices to show that $\Ccal $ admits a compact projective $\acal  $-generator. Zariski descent for Grothendieck abelian categories implies that the $\acal  $-linear functors $p^*_i: \Ccal  \rightarrow \Ccal \otimes_R R_i$ form a product diagram. For each $i$ the functor $p_i^*$ admits an $\acal  $-linear right adjoint $(p_i)_*$, which is also left adjoint to $p_i^*$. Pick for each $i$  a compact projective $\acal\otimes_R R_i$-generator  $X_i$ for $\ccal \otimes_R R_i$. Then $\bigoplus (p_i)_* X_i$ is a compact projective $\acal $-generator for $\ccal  $. 
  \end{proof}
 
\begin{proof}[Proof of theorem \ref{theo abelian con coefficients}]
By corollary \ref{coro properties dualizable 11} we have that $\ccal$ and $\ccal^\vee$ are Grothendieck abelian categories with exact products. By lemma \ref{lemma deal with products} it suffices to show that for every point in $\Spec(R)$ with residue field $k$ such that $\acal \otimes_R k$ is semisimple there exists an \'etale morphism $R \rightarrow R'$ such that $R' \otimes_R k \neq 0$ having  the property that $\ccal \otimes_R R'$ admits a compact projective $\acal \otimes_R R'$-generator.  Applying lemma \ref{lemma deal with completes classical} we see that $\ccal \otimes_R R_{\mathfrak{p}}^\wedge$ admits a compact projective $\acal \otimes_R  R_{\mathfrak{p}}^\wedge$-generator. It now follows from lemma \ref{lemma deal with filtered colimits classical} together with Popescu's smoothing theorem that there exists a smooth $R$-algebra $S$ with the property that $S \otimes_R k \neq 0$ and $\ccal \otimes_R S$ admits a compact projective $\acal \otimes_R S$-generator.  Pick a morphism of commutative rings $S \rightarrow R'$  such that the induced map $R \rightarrow R'$ is \'etale and $R' \otimes_R k \neq 0$. Then $R'$ has the desired property.
\end{proof}

%%%%%%%%%%%%%%%%%%%%%%%%%%%%%%%%%%%%%%%%%%%%%%%%%%%%%%%%%%%%%%%%%%%%%%%%
%%%%%%%%%%%%%%%%%%%%%%%%%%%%%%%%%%%%%%%%%%%%%%%%%%%%%%%%%%%%%%%%%%%%%%%%
%%%%%%%%%%%%%%%%%%%%%%%%%%%%%%%%%%%%%%%%%%%%%%%%%%%%%%%%%%%%%%%%%%%%%%%%
%%%%%%%%%%%%%%%%%%%%%%%%%%%%%%%%%%%%%%%%%%%%%%%%%%%%%%%%%%%%%%%%%%%%%%%%
%%%%%%%%%%%%%%%%%%%%%%%%%%%%%%%%%%%%%%%%%%%%%%%%%%%%%%%%%%%%%%%%%%%%%%%%
%%%%%%%%%%%%%%%%%%%%%%%%%%%%%%%%%%%%%%%%%%%%%%%%%%%%%%%%%%%%%%%%%%%%%%%%

\subsection{Fully dualizable \texorpdfstring{$(\infty,1)$}{(∞,1)}-categories}\label{subsection fully dualizable infty}

The following is our main theorem concerning fully dualizable $R$-linear categories:

\begin{theorem}\label{theo prestable con coefficients}
Let $R$ be an $E_\infty$-ring such that $\pi_0(R)$ is a G-ring, and let $\Mcal$ be a symmetric monoidal $R$-linear Grothendieck prestable category. Assume the following:
\begin{itemize}
\item $\Mcal$ is rigid and generated under colimits compact projective objects.
\item $\Mcal$ is proper over $R$.
\item The set of points $x$ in $\Spec(R)$ such that $(\Mcal \otimes_R \kappa(x))^\heartsuit$ is semisimple is dense in the Zariski topology.
\end{itemize}
Let $\ccal$ be a fully dualizable $\Mcal$-linear cocomplete category. Then there exists a faithfully flat \'etale morphism of connective $E_\infty$-rings $R \rightarrow R'$ and a smooth and proper algebra $A$ in $\Mcal \otimes_R R'$  such that $\Ccal \otimes_{R} R'$ is equivalent to $\LMod_A(\Mcal \otimes_R R')$ as an $\Mcal \otimes_R R'$-linear category.
\end{theorem}

Before going into the proof, we record a few consequences.

\begin{corollary}\label{coro etale locally trivial prestable}
Let $R$ be an $E_\infty$-ring such that $\pi_0(R)$ is a G-ring and let $\Ccal$ be an invertible $\Mod^\cn_R$-linear cocomplete category. Then there exists a faithfully flat \'etale morphism of connective $E_\infty$-rings $R \rightarrow R'$ such that $\Ccal \otimes_{R} R'$ is equivalent to $\Mod^\cn_{R'}$ as an $R'$-linear category.
\end{corollary}
\begin{proof}
By theorem \ref{theo prestable con coefficients} we may after passing to a faithfully flat \'etale $R$-algebra assume that $\ccal$ is the category of left modules over a connective Azumaya $R$-algebra $A$. The corollary now follows from the fact that connective Azumaya $R$-algebras are \'etale locally Morita equivalent to the unit (\cite{SAG} theorem 11.5.7.11).
\end{proof}

\begin{notation}
Let $\mathscr{L}: \CAlg(\Sp) \rightarrow \Spc$ be the functor from connective $E_\infty$-rings into spaces which associates to each connective $E_\infty$-ring $R$ the space of $R$-linear Grothendieck prestable categories which are \'etale locally on $\Spec(R)$ equivalent to $\Mod_R^\cn$. Then $\mathscr{L}$ is a sheaf for the \'etale topology, with a pointing given by the object $\Mod^\cn_{\SS}$ in $\mathscr{L}(\SS)$.  The resulting pointed object is equivalent to $B^2\operatorname{GL}_1$. For each $\operatorname{GL}_1$-gerbe $\mathcal{G}$ on $\Spec(R)$ we will denote by $\Mod^\cn_{R, \mathcal{G}}$ the associated twist of $\Mod^\cn_{R}$.
\end{notation}

\begin{corollary}\label{coro exists gerbe invertible prestable} 
Let $R$ be an $E_\infty$-ring such that $\pi_0(R)$ is a G-ring and let $\Ccal$ be an invertible $\Mod^\cn_R$-linear cocomplete category. Then there exists a $\operatorname{GL}_1$-gerbe $\mathcal{G}$ on $\Spec(R)$ and an $R$-linear equivalence ${\ccal = \Mod^\cn_{R, \mathcal{G}}}$
\end{corollary}
\begin{proof}
This is a direct consequence of corollary  \ref{coro etale locally trivial prestable} and the definitions.
\end{proof}

\begin{corollary}\label{coro classify fully dualizables prestable} 
Let $R$ be an  $E_\infty$-ring such that $\pi_0(R)$ is a G-ring and let $\Ccal$ be a fully dualizable $\Mod^\cn_R$-linear cocomplete category. Then there exists a finite \'etale $R$-algebra $\tilde{R}$, a $\operatorname{GL}_1$-gerbe $\mathcal{G}$ on $\Spec(\tilde{R})$ and an $R$-linear equivalence $\ccal = \Mod^\cn_{\tilde{R}, \mathcal{G}} $.
\end{corollary}
\begin{proof}
Let $\tilde{R} = \End^\enh_{\Funct_R(\ccal, \ccal)}(\id_\ccal)$ be the $R$-linear center of $\ccal$. Then $\tilde{R}$ is an $E_2$ $R$-algebra, and $\ccal$ may be equipped with a canonical $\tilde{R}$-linear structure. Since $\ccal$ is dualizable the formation of $\Funct_{R}(\ccal, \ccal)$ commutes with base change, and since $\id_\ccal$ is compact projective the formation of its endomorphisms commutes with base change as well. The fact that $\ccal$ is fully dualizable implies that $\tilde{R}$ is a dualizable $R$-module.  The assertion that $\tilde{R}$ is \'etale may then be reduced by base change to the case where $R = k$ is an algebraically closed field. In this case theorem \ref{theo prestable con coefficients} implies that $\ccal$ is the category of left modules over a finite product of copies of $k$, which has \'etale center. 

We note that since $\tilde{R}$ is $E_\infty$ then $\tilde{R}$ admits a unique enhancement to an $E_\infty$ $R$-algebra. It remains to show that $\ccal =  \smash{\Mod^\cn_{\tilde{R}, \mathcal{G}}}$ for some $\operatorname{GL}_1$-gerbe $\mathcal{G}$ on $\Spec(\tilde{R})$. By corollary \ref{coro exists gerbe invertible} it suffices to show that $\ccal$ is invertible over $\tilde{R}$. This can be checked \'etale locally by virtue of the fact that $\ccal$ is a Grothendieck prestable category (corollary \ref{coro properties dualizable infty}). By an application of theorem \ref{theo prestable con coefficients} we may reduce to the case where  $\ccal$ is the category of left modules over a smooth and proper algebra $A$ in $\Mod_R^\cn$. Applying corollary \ref{coro smooth and proper locally trivial} we may further reduce to the case when $\pi_0(A)$ is a finite product of copies of $\pi_0(R)$. Since $A$ is flat over $R$ we in fact have that $A$ is a finite product of copies of $R$. In this case the claim is clear.
\end{proof}

We devote the remainder of this section to the proof of theorem \ref{theo prestable con coefficients}.

\begin{notation}\label{notation sq zero}
Let $R$ be an $E_\infty$-ring. For each $R$-module $M$ we denote by $R \oplus M$ the corresponding split square zero extension of $R$ by $M$. Recall that if $M$ is an $R$-module then an $M[1]$-valued derivation on $R$ is a morphism of $E_\infty$-rings $\delta: R \rightarrow R \oplus M[1]$ whose composition with the projection $R \oplus M[1] \rightarrow R$ recovers the identity on $R$. Such a derivation defines a square zero extension of $R$ by $M$ which we will denote by $R \oplus^\delta M$. In other words, $R \oplus^\delta M$ is defined by the pullback
\[
\begin{tikzcd}
R \oplus^\delta M \arrow{d}{} \arrow{r}{} & R \arrow{d}{(\id, 0)} \\
R \arrow{r}{\delta} & R \oplus M[1] .
\end{tikzcd}
\]
\end{notation}

\begin{lemma}\label{lemma deal with square zero isomorphisms}
Let $R$ be a connective $E_\infty$-ring. Let $M$ be a connective $R$-module and $\delta: R \rightarrow R \oplus M[1]$ be an $M[1]$-valued derivation. Let $\Ccal$ be a $R \oplus^\delta M$-linear Grothendieck prestable category and let $X, Y$ be compact projective objects of $\Ccal$. If $X \otimes_{R \oplus^\delta M} R$ is isomorphic to $Y \otimes_{R \oplus^\delta M} R$ (as objects of $\Ccal \otimes_{R \oplus^\delta M} R)$ then $X$ is isomorphic to $Y$.
\end{lemma}
\begin{proof}
Let $\Ccal_R = \Ccal \otimes_{R \oplus^\delta M} R$ and $X_R, Y_R$ be the images of $X, Y$ in $\Ccal_R$. Let $f: X_R \rightarrow Y_R$ be an isomorphism and $g: Y_R \rightarrow X_R$ be an inverse. Let $U: \Ccal_R \rightarrow \Ccal$ be the forgetful functor and $\eta_X: X \rightarrow U(X_R)$, $\eta_Y:Y \rightarrow U(X_Y)$ the unit maps. We have that $\eta_Y: Y \rightarrow U(Y_R)$ is obtained by tensoring $Y$ with  the projection $R \oplus^\delta M \rightarrow R$ and therefore induces an epimorphism on $H_0$. The projectivity of $X$ allows us to pick a lift $\overline{f}: X \rightarrow Y$ of $U(f) \circ \eta_X$ against $\eta_Y$. Similarly, we may pick a lift $\overline{g}: Y \rightarrow X$ of $U(g) \circ \eta_Y$ against $\eta_X$. Note that the images of $\overline{f}$ and $\overline{g}$ under the map $\Ccal \rightarrow \Ccal_R$ recover $f$ and $g$, respectively. 

We claim that   $\overline{f}$ and $\overline{g}$ are inverses. By symmetry it suffices to prove that $\overline{g}$ is a left inverse to $\overline{f}$. Let $h = \overline{g} \circ \overline{f}$. Then $h: X \rightarrow X$ is a lift of the identity on $X_R$. Let $A$ be the $R \oplus^\delta M$-algebra of endomorphisms of $X$, and let $A_R = A \otimes_{R \oplus^\delta M} R$. The fact that $X$ is compact projective implies that $A_R$ is the $R$-algebra of endomorphisms of $X_R$. The endomorphism $h$ defines an element $[h]$ in $\pi_0(A)$ whose image in $\pi_0(A_R)$ is the unit. We regard $\pi_0(A)$ as a classical $\pi_0(R \oplus^\delta M)$-algebra, so that $\pi_0(A_R) = \pi_0(A)/K\pi_0(A)$ where $K$ is the kernel of the map $\pi_0(R \oplus^\delta M) \rightarrow \pi_0(R)$. Since $K$ is square zero we see that $\pi_0(A)$ is a square zero extension of $\pi_0(A_R)$, and therefore $[h]$ is invertible. This shows that $h$ is an isomorphism, as desired.
\end{proof}

\begin{lemma}\label{lemma deal with square zero previa}
Let $R$ be a connective $E_\infty$-ring. Let $M$ be a connective $R$-module and  $\delta: R \rightarrow R \oplus M[1]$ be an $M[1]$-valued derivation. Let $\Ccal$ be a separated $R \oplus^\delta M$-linear Grothendieck prestable category. Then every compact projective object of $\Ccal \otimes_{R \oplus^\delta M} R$ admits a lift to a compact projective object of $\Ccal$.
\end{lemma}
\begin{proof}
Let $\Ccal_{R} =\Ccal \otimes_{R \oplus^\delta M} R$ and $\Ccal_{R \oplus M[1]} =\Ccal \otimes_{R \oplus^\delta M} (R \oplus M[1])$. Fix a compact projective object $X$ of $\Ccal_R$. Applying \cite{SAG} proposition 16.2.2.1 to the commutative square from notation \ref{notation sq zero} we obtain a pullback square of categories
\[
\begin{tikzcd}
\Ccal \arrow{r}{} \arrow{d}{} & \Ccal_R \arrow{d}{0^*} \\
\Ccal_R \arrow{r}{\delta^*} & \Ccal_{R \oplus M[1]}.
\end{tikzcd}
\] 
We have that both $\delta^*X$ and $0^*X$ have image $X$ under extension of scalars along $R \oplus M[1] \rightarrow R$. It follows from lemma \ref{lemma deal with square zero isomorphisms} that there exists an isomorphism $\delta^*X = 0^*X$, so that we may identify both objects with a (compact projective) object of $\Ccal_{R \oplus M[1]}$ which we denote $X_{R \oplus M[1]}$. The triple $(X, X_{R \oplus M[1]}, X)$ defines an object of $\Ccal$ which we will denote $\overline{X}$. This is a lift of $X$, so to prove the lemma it will suffice to show that $\overline{X}$ is compact projective.

We first show that $\overline{X}$ is compact. Assume given a filtered diagram $Y_\alpha$ in $\Ccal$ and denote by $(Y_\alpha)_R$ and $(Y_\alpha)_{R \oplus M[1]}$ its base changes. We want to show that the  map $\colim \Hom_{\ccal}(\overline{X}, Y_\alpha) \rightarrow \Hom_{\ccal}(\overline{X}, \colim Y_\alpha)$ is an isomorphism. This follows from the compactness of $X$ and $X_{R\oplus M[1]}$, since this map is the pullback of the map  
\[
\colim \Hom_{\ccal_R}(X, (Y_\alpha)_R) \rightarrow  \Hom_{\ccal_R}(X,\colim  (Y_\alpha)_R )
\]
with itself over
\[
\colim \Hom_{\ccal_{R\oplus M[1]}}(X_{R\oplus M[1]}, (Y_\alpha)_{R\oplus M[1]}) \rightarrow  \Hom_{\ccal_{R\oplus M[1]}}(X_{R\oplus M[1]},\colim  (Y_\alpha)_{R\oplus M[1]} ).
\]

It remains to prove that $\overline{X}$ is projective. This amounts to showing that if $Y$ is an object of $\Ccal$ then $\pi_0 \Hom_{\ccal}(\overline{X}, \Sigma Y) = 0$. Let $Y_R = Y \otimes_{R \oplus^\delta M} R$ and $Y_{R \oplus M[1]} = Y \otimes_{R \oplus^\delta M} (R \oplus M[1])$. We have a pullback square
\[
\begin{tikzcd}
 \Hom_{\ccal}(\overline{X}, \Sigma Y)\arrow{r}{} \arrow{d}{} & \Hom_{\ccal_R}(X, \Sigma Y_R) \arrow{d}{0^*} \\
  \Hom_{\ccal_R}(X, \Sigma Y_R)  \arrow{r}{\delta^*} & \Hom_{\Ccal_{R \oplus M[1]}}(X_{R \oplus M[1]},\Sigma Y_{{R \oplus M[1]}}).
\end{tikzcd}
\]
The projectivity of $X$ and $X_{R \oplus M[1]}$ guarantees that 
\[
\pi_0(\Hom_{\ccal_R}(X, \Sigma Y_R)) = \pi_0( \Hom_{\Ccal_{R \oplus M[1]}}(X_{R \oplus M[1]},\Sigma Y_{{R \oplus M[1]}})) = 0.
\] We may thus reduce to showing that the right vertical map is a surjection on $\pi_1$. This is the same as showing that the map $0^*: \Hom_{\ccal_R}(X,  Y_R) \rightarrow \ \Hom_{\Ccal_{R \oplus M[1]}}(X_{R \oplus M[1]}, Y_{{R \oplus M[1]}})$ is a surjection on $\pi_0$. In other words, we have to prove that any map $X_{R \oplus M[1]} \rightarrow Y_{{R \oplus M[1]}}$ lifts to a map $ X \rightarrow Y_R$. This is the same as showing that the composite map $X \rightarrow X_{R \oplus M[1]} \rightarrow Y_{R \oplus M[1]}$ admits a factorization through $Y_R \rightarrow Y_{R \oplus M[1]}$ (where here we identify $ X_{R \oplus M[1]}$ and $ Y_{R \oplus M[1]}$ with their restriction of scalars along $(\id, 0): R \rightarrow R \oplus M[1]$).  This follows from the projectivity of $X$.
\end{proof}

\begin{lemma}\label{lemma deal with square zero para prestable}
Let $R$ be a connective $E_\infty$-ring. Let $M$ be a connective $R$-module and  $\delta: R \rightarrow R \oplus M[1]$ be an $M[1]$-valued derivation. Let $\Ccal$ be a separated $R \oplus^\delta M$-linear Grothendieck prestable category. Let $X_\alpha$ be a family of compact projective objects of $\ccal$ whose image in  $\Ccal \otimes_{R \oplus^\delta M} R$ is a generating family. Then the family $X_\alpha$ generates $\Ccal$ under colimits.
\end{lemma}
\begin{proof}
Let $\Ccal'$ be the full subcategory of $\Ccal$ generated under colimits by the family $X_\alpha$. We want to show that $\ccal' = \ccal$. Let $\ccal_R, \ccal_{R \oplus M[1]}, \ccal'_R, \ccal'_{R \oplus M[1]}$ be the base changes of $\ccal$ and $\ccal'$.  Applying \cite{SAG} proposition 16.2.2.1 to the commutative square from notation \ref{notation sq zero} we see that $\Ccal = \Ccal_R \times_{\Ccal_{R \oplus M[1]}} \Ccal_R$ and $\Ccal' = \Ccal'_R \times_{\Ccal'_{R \oplus M[1]}} \Ccal'_R$. We may thus reduce to showing that $\ccal'_R = \Ccal_R$. We note that $\ccal'_R$ is a full subcategory of $\ccal_R$ closed under colimits and containing the image of the composite functor $\ccal' \rightarrow \Ccal \rightarrow \ccal_R$. In particular, $\ccal'_R$ contains the objects $X_\alpha \otimes_{R \oplus^\delta M} R$.  Our claim now follows from the fact that this family was assumed to generate $\ccal_R$ under colimits.
\end{proof}

\begin{lemma}\label{lemma completes equiv postnikov}
Let $R$ be a connective $E_\infty$-ring and let $\Ccal$ be an $R$-linear Grothendieck prestable category. Assume that $\ccal$ is separated and that products in $\Sp(\ccal)$ are t-exact. Then the functor $p:  \Ccal \rightarrow \lim \Ccal \otimes_R \tau_{\leq n}R$ is an equivalence.
\end{lemma}
\begin{proof}
For each $n \geq 0$ let $R_n = \tau_{\leq n} R$.  We first prove that $p^R$ is fully faithful, by showing that the counit $p p^R(M_n) \rightarrow (M_n)$ is an isomorphism for all sequences $(M_n)$. This amounts to showing that the canonical map $\mu: (\lim M_n) \otimes_R R_s \rightarrow M_s$ is an isomorphism for all $s \geq 0$. This map factors, for each $t \geq s$, as a composition 
\[
(\lim M_n) \otimes_R R_s \xrightarrow{\mu_1} M_t \otimes_R R_s \xrightarrow{\mu_2} M_s
\]
where $\mu_1$ is induced from the projection $\lim M_n \rightarrow M_t$, and $\mu_2$ is induced from the $R$-linear map $M_t \rightarrow M_s$.

For each $n \geq 0$ the transition $M_{n+1} \rightarrow M_n$ is obtained by tensoring the $(n+1)$-connective map $R_{n+1} \rightarrow R_n$ with $M_{n+1}$. In particular, it is itself $(n+1)$-connective. Since products in $\Sp(\ccal)$ are t-exact we have that the map $(\lim M_n) \rightarrow M_t$ is $(t+1)$-connective and therefore $\mu_1$ is $(t+1)$-connective as well. 

The map $\mu_2$ is equivalent to the induction along $R_t \rightarrow R_s$ of the map $\mu_2': M_t \otimes_R R_t \rightarrow M_t$ induced from the identity on $M_t$. The map $\mu_2'$ is obtained by tensoring $M_t$ with the $(t+1)$-connective map $R_t \otimes_R R_t \rightarrow R_t$, and is therefore $(t+1)$-connective. It follows that $\mu_2$ is also $(t+1)$-connective. Now $\mu$ is a composition of $(t+1)$-connective maps so it is $(t+1)$-connective. Letting $t \rightarrow \infty$ we see that $\mu$ is $\infty$-connective. The fact that $\ccal$ is separated now implies that $\mu$ is an equivalence, as desired.

To prove that $p$ is an equivalence it now suffices to show that it is conservative. Since every morphism in $\Mod_R^\cn$ is the fiber of its cofiber it is enough to prove that if $X$ is an object of $\ccal$ such that $p(X) = 0$ then $X = 0$. Since $\ccal$ is separated it is enough for this to prove that $H_t(X) = 0$ for all $t \geq 0$. Arguing by induction on $t$, we may assume that $H_s(X) = 0 $ for all $s < t$. Replacing $X$ by $\Omega^t(X)$ we may reduce to the case $t = 0$. Since $p(X) = 0$ we have in particular that $X \otimes_R \pi_0(R) = 0$. The fact that $H_0(X) = 0$ now follows from the fact that the map $X \rightarrow X \otimes_R \pi_0(R)$ is $1$-connective.
\end{proof}

\begin{lemma}\label{lemma check compact projective by tensoring}
Let $R$ be a connective $E_\infty$-ring and let $\Ccal$ be an $R$-linear Grothendieck prestable category. Assume that $\ccal$ is separated and that products in $\Sp(\ccal)$ are t-exact. Let $X$ be an object of $\ccal$. Then $X$ is compact projective if and only if $X \otimes_R \tau_{\leq n} R$ is a compact projective object of $\Ccal \otimes_R \tau_{\leq n} R$ for all $n \geq 0$.
\end{lemma}
\begin{proof}
For each $n \geq 0$ set $R_n = \tau_{\leq n} R$, $\ccal_n =\ccal \otimes_R R_n$, and $X_n = X \otimes_R R_n$. The only if direction follows from the fact that the extension of scalars functors $\ccal \rightarrow \ccal_n$ admit colimit preserving right adjoints. It remains to prove the if direction.

We first show that $X$ is projective. To prove this it suffices to show that if $Y$ is an object of $\ccal$ then $\Hom_\ccal(X, \Sigma Y)$ is connected. Set $Y_n = Y \otimes_R R_n$ for all $n \geq 0$. Applying lemma \ref{lemma completes equiv postnikov} we  have $\Hom_\ccal(X, \Sigma  Y) = \lim \Hom_{\ccal_n}(X_n, \Sigma  Y_n)$. Since $X_n$ is projective for all $n$ each of the spaces $\Hom_{\ccal_n}(X_n, \Sigma  Y_n)$ is connected. We may therefore reduce to proving that the transitions $\Hom_{\ccal_{n+1}}(X_{n+1}, \Sigma  Y_{n+1}) \rightarrow \Hom_{\ccal_n}(X_n, \Sigma  Y_n)$ induce surjections on $\pi_1$. This is the same as showing that the transitions $\Hom_{\ccal_{n+1}}(X_{n+1}, Y_{n+1}) \rightarrow \Hom_{\ccal_n}(X_n,  Y_n)$ induce surjections on $\pi_0$. In other words, we have to show that every map $X_{n+1} \rightarrow Y_n$ in $\ccal_{n+1}$ factors through $Y_{n+1}$. This is a consequence of the fact that $X_{n+1}$ is projective and the morphism $Y_{n+1} \rightarrow Y_n$ is $0$-connective.

We now show that $X$ is compact. Assume given a filtered diagram $Y_\alpha$ in $\ccal$. We wish to prove that $\colim \Hom_{\ccal}(X, Y_\alpha) = \Hom_{\ccal}(X, \colim Y_\alpha)$. We will do so by proving that it induces an isomorphism on $\pi_t$ for all $t \geq 0$. Using the fact that $X$ is projective we obtain equivalences 
\[
\pi_t(\Hom_{\ccal}(X, Y_\alpha)) = \pi_t(\Hom_{\ccal}(X, \Sigma^t H_t(Y_\alpha)))
\]
 and 
 \[\pi_t(\Hom_{\ccal}(X, \colim Y_\alpha)) = \pi_t(\Hom_{\ccal}(X, \colim \Sigma^t  H_t(Y_\alpha))).
\]
We may therefore reduce to proving that we have an equivalence $\colim \Hom_\Ccal(X, \Sigma^t H_t(Y_\alpha)) = \Hom_\ccal(X, \colim \Sigma^t H_t(Y_\alpha))$. Replacing $Y_\alpha$ with $\Sigma^t H_t(Y_\alpha)$ we may now reduce to the case when $Y_\alpha$ is $t$-truncated for all $\alpha$. To prove this it suffices to show that $\tau_{\leq t}(X)$ is a compact object of $\ccal_{\leq t}$. Since the functor $\Mod_R^\cn \rightarrow \Mod_{R_t}^\cn$ induces an equivalence on $t$-truncated objects we have that the functor $\ccal \rightarrow \ccal_t$ induces an equivalence on $t$-truncated objects as well. Hence we may reduce to proving that $\tau_{\leq t}(X_t)$ is a compact object of $(\ccal_t)_{\leq t}$. This is a consequence of the fact that $X_t$ is compact in $\ccal_t$.
\end{proof}

\begin{lemma}\label{lemma deal with postnikov}
Let $R$ be a connective $E_\infty$-ring and let $\Mcal$ be a symmetric monoidal $R$-linear Grothendieck prestable category. Assume that $\Mcal$ is rigid and generated under colimits by compact projective objects. Let $\ccal$ be a separated $\Mcal$-linear Grothendieck prestable category such that products in $\Sp(\ccal)$ are t-exact. If $\ccal \otimes_R \pi_0(R)$ admits a compact projective $\Mcal\otimes_R \pi_0(R)$-generator then $\ccal$ admits a compact projective $\Mcal$-generator.
\end{lemma}
\begin{proof}
For each $n \geq 0$ set $R_n = \tau_{\leq n} R$, $\Mcal_n = \Mcal \otimes_R R_n$ and $\Ccal_n = \Ccal \otimes_R R_n$.  Fix a compact projective $\Mcal\otimes_R \pi_0(R)$-generator $X_0$ for $\ccal_0$. Applying lemma \ref{lemma deal with square zero previa} inductively we may construct a compatible sequence of compact projective objects $X_n$ in $\ccal_n$.  An inductive application of lemma \ref{lemma deal with square zero para prestable} implies that the family of objects obtained by tensoring $X_n$ with a compact projective object of $\Mcal$ generates $\ccal_n$ under colimits. It follows that $X_n$ is a compact projective $\Mcal_n$-generator of $\ccal_n$ for every $n \geq 0$. 


By lemma \ref{lemma completes equiv postnikov} there exists an object $X$ in $\Ccal$ such that $X \otimes_R R_n = X_n$ for all $n$. It follows from lemma \ref{lemma check compact projective by tensoring} that $X$ is compact projective. It remains to show that $X$ is an $\Mcal$-generator for $\ccal$. Let $\ccal'$ be the smallest full subcategory of $\ccal$ containing $X$ and closed under colimits and the action of $\Mcal$. Then $\ccal'$ is a $\Mcal$-linear Grothendieck prestable category generated under colimits by compact projective objects. We wish to show that the inclusion $\ccal' \rightarrow \ccal$ is an equivalence. By lemma \ref{lemma completes equiv postnikov} it is enough to prove that $\ccal' \otimes_R R_n = \ccal_n$ for all $n \geq 0$. Note that $\ccal' \otimes_R R_n$ is a full subcategory of $\ccal_n$  closed under colimits and the action of $\Mcal_n$, and containing $X_n$. Our claim now follows from the fact that $X_n$ is a $\Mcal_n$-generator for $\ccal_n$.
\end{proof}

\begin{proof}[Proof of theorem \ref{theo prestable con coefficients}]
By corollary \ref{coro properties dualizable infty} we have that $\ccal$ is a separated Grothendieck prestable category and products in $\Sp(\ccal)$  are t-exact. Applying theorem \ref{theo abelian con coefficients} we may assume, after changing base to a faithfully flat \'etale $R$-algebra, that $\ccal^\heartsuit$ admits a compact projective $\Mcal^\heartsuit$-generator.   By proposition \ref{prop lex localization prestable} it suffices to show that $\ccal$ admits a compact projective $\Mcal$-generator. Applying lemma \ref{lemma deal with postnikov} we may reduce to the case when $R$ is $0$-truncated. It now suffices to show that $\ccal$ is the connective derived category of its heart.  For this it is enough to prove that $\ccal^\heartsuit$ generates $\ccal$ under colimits.

Let $\ccal'$ be the full subcategory of $\ccal$ generated under colimits by $\ccal^\heartsuit$, and note that $\ccal'$ inherits an $\Mcal$-linear structure from $\ccal$. Let $\delta$ be the image of $1_{\Mcal}$ under the unit map $\Mcal \rightarrow \ccal \otimes_\Mcal \ccal^\heartsuit$. The fact that $\ccal$ is fully dualizable implies that $\delta$ is compact projective and $0$-truncated. Since $\ccal \otimes_\Mcal \ccal^\heartsuit$ is generated by the objects of the form $X \otimes Y$ with $X$ in $\ccal$ and $Y$ in $\ccal^\heartsuit$, we may find finite sequences of objects $X_i$ in $\ccal$ and $Y_i$ in $\ccal^\vee$ and a morphism $f: \bigoplus (X_i \otimes Y_i) \rightarrow \delta$ which induces an epimorphism on $H_0$. The fact that $\delta$ is $0$-truncated implies that $f$ factors through $\bigoplus (H_0(X_i) \otimes Y_i)$, and since $\delta$ is projective we see that $\delta$ is a direct summand of $\bigoplus (H_0(X_i) \otimes Y_i)$. It follows in particular that $\delta$ belongs to $\ccal' \otimes_\Mcal \ccal^\vee$, which implies that the identity on $\ccal$ belongs to the image of the inclusion $\Funct_\Mcal(\ccal, \ccal') \rightarrow \Funct_\Mcal(\ccal, \ccal)$. In other words, the inclusion $\ccal' \rightarrow \ccal$ admits a section, which implies that $\ccal' = \ccal$, as desired.
\end{proof}

%%%%%%%%%%%%%%%%%%%%%%%%%%%%%%%%%%%%%%%%%%%%%%%%%%%%%%%%%%%%%%%%%%%%%%%%
%%%%%%%%%%%%%%%%%%%%%%%%%%%%%%%%%%%%%%%%%%%%%%%%%%%%%%%%%%%%%%%%%%%%%%%%
%%%%%%%%%%%%%%%%%%%%%%%%%%%%%%%%%%%%%%%%%%%%%%%%%%%%%%%%%%%%%%%%%%%%%%%%
%%%%%%%%%%%%%%%%%%%%%%%%%%%%%%%%%%%%%%%%%%%%%%%%%%%%%%%%%%%%%%%%%%%%%%%%
%%%%%%%%%%%%%%%%%%%%%%%%%%%%%%%%%%%%%%%%%%%%%%%%%%%%%%%%%%%%%%%%%%%%%%%%
%%%%%%%%%%%%%%%%%%%%%%%%%%%%%%%%%%%%%%%%%%%%%%%%%%%%%%%%%%%%%%%%%%%%%%%%

\subsection{Rings of definition of fully dualizable categories}\label{subsection rings of def}

Our next goal is to discuss how to extend the above results beyond the context of G-rings, under an additional compact generation hypothesis. The basic mechanism is given by the following:

\begin{proposition}\label{prop lift fully dualizables}
Let $\Mcal_\alpha$ be a filtered diagram of commutative algebras in $\Pr^L$ with compact transition functors. Assume that for every $\alpha$ the category $\Mcal_\alpha$ is generated under colimits by dualizable objects and has compact unit.  Let $\ccal$ be a fully dualizable $\Mcal$-linear cocomplete category, and assume that $\ccal$ and $\ccal^\vee$ are generated under colimits by compact objects. Then there exists an index $\alpha$, a fully dualizable $\Mcal_\alpha$-linear cocomplete category $\ccal_\alpha$, and an equivalence $\ccal_\alpha \otimes_{\Mcal_\alpha} \Mcal =\ccal$. Furthermore, if $\ccal$ is invertible then $\ccal_\alpha$ may be chosen to be invertible as well.
\end{proposition}
\begin{proof}
 Let $\Pr^L_\omega$ be the subcategory of $\Pr^L$ on the compactly generated categories and compact functors. Note that $\Pr^L_\omega$ is a compactly generated presentable category and the inclusion into $\Pr^L$ preserves colimits. Furthermore, the symmetric monoidal structure on $\Pr^L$ restricts to a symmetric monoidal structure on $\Pr^L_\omega$, which in turn restricts to a symmetric monoidal structure on its full subcategory $(\Pr^L_\omega)^\omega$ on the compact objects. 
 
 Our assumptions imply that $\Mcal_\alpha$ is a commutative algebra in $\Pr^L_\omega$ for every $\alpha$, and therefore $\Mcal$ is also a commutative algebra in $\Pr^L_\omega$. We now have $\Mod_{\Mcal}(\Pr^L_\omega) = \colim \Mod_{\Mcal_\alpha}(\Pr^L_\omega)$ which implies, after passing to compact objects, that $\Mod_{\Mcal}(\Pr^L_\omega)^\omega = \colim \Mod_{\Mcal_\alpha}(\Pr^L_\omega)^\omega$.
 
 
By proposition \ref{proposition dualizable is presentable} we have that $\ccal$ and its dual are presentable. Since $\Mcal_\alpha$ is generated under colimits by dualizable objects for all $\alpha$ the same thing holds for $\Mcal$, which implies that the action functors $\Mcal \times \ccal \rightarrow \ccal$ and $\Mcal \times \ccal^\vee \rightarrow \ccal^\vee$ preserve compact objects. It follows that $\ccal$ and $\ccal^\vee$ belong to $\Mod_{\Mcal}(\Pr^L_\omega)$. Combining this with the fact that $\ccal$ is fully dualizable in $\Mod_\Mcal(\Pr^L)$ we see that $\ccal$ is a fully dualizable object of $\Mod_{\Mcal}(\Pr^L_\omega)$. Since the unit in $\Mod_{\Mcal}(\Pr^L_\omega)$ is compact we have that $\ccal$ is a fully dualizable object of $\Mod_{\Mcal}(\Pr^L_\omega)^\omega$. The proposition now follows from the fact that $\Mod_{\Mcal}(\Pr^L_\omega)^\omega = \colim \Mod_{\Mcal_\alpha}(\Pr^L_\omega)^\omega$, since the functor that sends each symmetric monoidal $2$-category to its space of fully dualizable (resp. invertible) objects preserves filtered colimits.
\end{proof}
 
 \begin{corollary}\label{coro rings of definitions 11}
Let $\acal$ be a symmetric monoidal Grothendieck abelian category, rigid, generated by compact projective objects, and proper over $\ZZ$. Assume that for every field $k$ the Grothendieck abelian category $\acal \otimes k$ is semisimple. Let $R$ be a commutative ring and let $\ccal$ be a fully dualizable  $\Acal \otimes R$-linear cocomplete category. Assume that $\ccal$ and $\ccal^\vee$ are compactly generated. Then there exists a subalgebra $S \subseteq R$ of finite type over $\ZZ$, a fully dualizable $\acal \otimes S$-linear cocomplete category $\dcal$, and an $\acal \otimes R$-linear equivalence $\dcal \otimes_S R = \ccal$. Furthermore, if $\ccal$ is invertible then $\dcal$ may be chosen to be invertible as well.
 \end{corollary}
 \begin{proof}
 Specialize proposition \ref{prop lift fully dualizables} to the filtered diagram indexed by the poset of finite type subalgebras of $R$, that sends a subalgebra $S$ to $\acal \otimes S$.
 \end{proof}

\begin{corollary}\label{coro los coros classicos extienden}
Corollaries \ref{coro etale locally trivial},  \ref{coro exists gerbe invertible}, and \ref{coro classify fully dualizables abelian} hold over an arbitrary commutative ring, provided that $\ccal$ and $\ccal^\vee$ are compactly generated.
\end{corollary} 
\begin{proof}
Follows directly from the case $\acal = \Ab$ of corollary \ref{coro rings of definitions 11} together with the fact that finite type commutative rings are G-rings.
\end{proof}

In the same way, we have:

\begin{corollary}
Corollaries \ref{coro etale locally trivial prestable}, \ref{coro exists gerbe invertible prestable} and \ref{coro classify fully dualizables prestable} hold over an arbitrary connective $E_\infty$-ring, provided that $\ccal$ and $\ccal^\vee$ are compactly generated.
\end{corollary}

\begin{remark}
Corollary \ref{coro etale locally trivial prestable} was proven for arbitrary connective $E_\infty$-rings in \cite{SAG} theorem 11.5.7.11, under the condition that $\ccal$ and $\ccal^\vee$ are compactly generated Grothendieck prestable categories.
\end{remark}

%%%%%%%%%%%%%%%%%%%%%%%%%%%%%%%%%%%%%%%%%%%%%%%%%%%%%%%%%%%%%%%%%%%%%%%%
%%%%%%%%%%%%%%%%%%%%%%%%%%%%%%%%%%%%%%%%%%%%%%%%%%%%%%%%%%%%%%%%%%%%%%%%
%%%%%%%%%%%%%%%%%%%%%%%%%%%%%%%%%%%%%%%%%%%%%%%%%%%%%%%%%%%%%%%%%%%%%%%%
%%%%%%%%%%%%%%%%%%%%%%%%%%%%%%%%%%%%%%%%%%%%%%%%%%%%%%%%%%%%%%%%%%%%%%%%
%%%%%%%%%%%%%%%%%%%%%%%%%%%%%%%%%%%%%%%%%%%%%%%%%%%%%%%%%%%%%%%%%%%%%%%%
%%%%%%%%%%%%%%%%%%%%%%%%%%%%%%%%%%%%%%%%%%%%%%%%%%%%%%%%%%%%%%%%%%%%%%%%

\subsection{Invertible stable categories}\label{subsection invertible stable over R}

We finish with a classification of invertible stable categories over truncated connective $E_\infty$-rings.

\begin{theorem}\label{theorem stable over R}
Let $R$ be a truncated connective $E_\infty$-ring and let $\Ccal$ be an invertible $R$-linear cocomplete stable category. Then $\Ccal = \LMod_A(\Mod_R)$ for some Azumaya algebra $A$ in $\Mod_R$.
\end{theorem}

The remainder of this section is devoted to a proof of theorem \ref{theorem stable over R}.

\begin{lemma}\label{lemma conservative pullback}
Let 
\[
\begin{tikzcd}
R \arrow{r}{} \arrow{d}{} & R_0 \arrow{d}{} \\
R_1 \arrow{r}{} & R_{01}
\end{tikzcd}
\]
be a pullback diagram of $E_\infty$-rings. Assume given a morphism $f: \Ccal \rightarrow \Dcal$ in $\Mod_{\Mod_R}(\Pr^L)$. If the induced functors $\Ccal \otimes_{R} R_0 \rightarrow \Dcal \otimes_R R_0$ and $\Ccal \otimes_R R_1 \rightarrow \Dcal \otimes_R R_1$ are equivalences then $f$ is an equivalence.
\end{lemma}
\begin{proof}
Let $\Ccal_0 = \Ccal \otimes_R R_0$, $\Ccal_1 = \Ccal \otimes_R R_1$ and $\Ccal_{01} = \Ccal \otimes_R R_{01}$, and define $\Dcal_0, \Dcal_1, \Dcal_{01}$ similarly. We have a commutative square of categories
\[
\begin{tikzcd}
\Ccal \arrow{d}{f} \arrow{r}{} & \Ccal_0 \times_{\Ccal_{01}} \Ccal_1 \arrow{d}{} \\
\Dcal \arrow{r}{} & \Dcal_0 \times_{\Dcal_{01}} \Dcal_1
\end{tikzcd}
\]
where the right vertical arrow is an isomorphism by our hypothesis. The horizontal arrows in the above square are fully faithful by \cite{SAG} proposition 16.2.1.1, and hence $f$ is fully faithful. To finish the proof it will suffice to show that $\Dcal/\Ccal = 0$. We have $(\Dcal/\Ccal)\otimes_{R} R_0 = \Dcal_0/\Ccal_0 = 0$ and $(\Dcal/\Ccal) \otimes_R R_1 = \Dcal_1 / \Ccal_1 = 0$. Hence the projection $\Dcal/\Ccal \rightarrow 0$ induces equivalences after tensoring with $R_0$ and $R_1$. It follows that the projection $\Dcal/\Ccal \rightarrow 0$ is fully faithful, and hence it is an isomorphism, as desired.
\end{proof}

\begin{lemma}\label{lemma deal with square zero para stable}
Let $R$ be a connective $E_\infty$-ring, let $M$ be a connective $R$-module and $\delta: R \rightarrow R \oplus M[1]$ be an $M[1]$-valued derivation. Let $\Ccal$ be an $R \oplus^\delta M$-linear presentable stable category and assume given an $R$-linear equivalence $\varphi: \Ccal \otimes_{R \oplus^\delta M} R  = \Mod_R$. Then there is an $R \oplus^\delta M$-linear equivalence $\Ccal = \Mod_{R \oplus^\delta M}$ which recovers $\varphi$ after tensoring with $R$.
\end{lemma}
\begin{proof}
We have a commutative diagram of categories with invertible vertical arrows
\[
\begin{tikzcd}[column sep = small]
\Ccal \otimes_{R \oplus^\delta M } R \arrow{r}{}  \arrow{d}{\varphi} & \Ccal \otimes_{R \oplus^\delta M } R \otimes_R ( R \oplus M[1] ) \arrow{r}{=} \arrow{d}{} &  \arrow{d}{} \Ccal \otimes_{R \oplus^\delta M } R \otimes_R ( R \oplus M[1] ) & \arrow{l}{} \Ccal \otimes_{R \oplus^\delta M} R  \arrow{d}{\varphi}\\
\Mod_R \arrow{r}{} & \Mod_{R \oplus M[1]} \arrow{r}{\psi} & \Mod_{R \oplus M[1]} & \arrow{l}{} \Mod_R.
\end{tikzcd}
\]
where the leftmost horizontal arrows are given by induction along $\delta$, the rightmost horizontal arrows are given by induction along $(\id,0)$, and the top middle horizontal arrow is induced from the commutativity of the square in notation \ref{notation sq zero}. The equivalence $\psi$ is $R \oplus M[1]$-linear and is therefore given by tensoring with an invertible $R \oplus M[1]$-module $L$. Since $\delta$ is an isomorphism on $\pi_0$, there exists an invertible $R$-module $L'$ whose extension of scalars along $\delta$ recovers $L$. We may now extend the above commutative diagram as follows:
\[
\begin{tikzcd}[column sep = small]
\Ccal \otimes_{R \oplus^\delta M } R \arrow{r}{}  \arrow{d}{\varphi} & \Ccal \otimes_{R \oplus^\delta M } R \otimes_R ( R \oplus M[1] ) \arrow{r}{=} \arrow{d}{} &  \arrow{d}{} \Ccal \otimes_{R \oplus^\delta M } R \otimes_R ( R \oplus M[1] ) & \arrow{l}{} \Ccal \otimes_{R \oplus^\delta M} R  \arrow{d}{\varphi}\\
\Mod_R \arrow{r}{} \arrow{d}{-\otimes_R L'} & \Mod_{R \oplus M[1]} \arrow{r}{\psi} \arrow{d}{- \otimes_{R \oplus M[1]} L} & \Mod_{R \oplus M[1]} \arrow{d}{\id} & \arrow{l}{} \Mod_R \arrow{d}{\id} \\
\Mod_R \arrow{r}{} & \Mod_{R \oplus M[1]} \arrow{r}{\id} & \Mod_{R \oplus M[1]} & \arrow{l}{} \Mod_R 
\end{tikzcd}
\]
Passing to pullbacks of the first and third rows we obtain an $R \oplus^\delta M$-linear equivalence
\[
\xi:( \Ccal \otimes_{R \oplus^\delta M} R) \times_{\Ccal \otimes_{R \oplus^\delta M} (R \oplus M[1])} ( \Ccal \otimes_{R \oplus^\delta M} R)  = \Mod_R \times_{\Mod_{R \oplus M[1]}} \Mod_R.
\]
Denote by 
\[
\iota: \Mod_{R \oplus^\delta M} \rightarrow \Mod_R \times_{\Mod_{R \oplus M[1]}} \Mod_R
\]
 and 
 \[
 \iota_{\Ccal}: \Ccal \rightarrow ( \Ccal \otimes_{R \oplus^\delta M} R) \times_{\Ccal \otimes_{R \oplus^\delta M} (R \oplus M[1])} ( \Ccal \otimes_{R \oplus^\delta M} R) 
 \]
 the canonical functors. By \cite{SAG} proposition 16.2.1.1 both $\iota$ and $\iota_\Ccal$ are fully faithful.
 
 We claim that $\xi \circ \iota_\Ccal$ factors through $\iota$. Let $X$ be an object of $\Ccal$. We wish to show that $\xi \iota_\Ccal(X)$ belongs to the image of $\iota$. We have 
 \[
 X = (X \otimes_{R \oplus^\delta M} R) \times_{X \otimes_{R \oplus^\delta M} (R \oplus M[1])} (X \otimes_{R \oplus^\delta M} R)
 \]
 so it suffices to prove that $\xi \iota_\Ccal( X \otimes_{R \oplus^\delta M} R)$ and $\xi \iota_\Ccal(X \otimes_{R \oplus^\delta M} (R \oplus M[1]))$ belong to the image of $\iota$. In other words, we may reduce to the case when $X$ is obtained by restriction of scalars along $R \oplus^\delta M \rightarrow R$.  Since $\Ccal \otimes_{R \oplus^\delta M} R$ is equivalent to $\Mod_R$, which is generated under colimits and shifts by $R$, we may further reduce to the case when $X$ is given by restriction of scalars of $\varphi^{-1}(R)$. In this case the image of $X$ in $\Ccal \otimes_{R \oplus^\delta M} R$ is given by $\varphi^{-1}(R \otimes_{R \oplus^\delta M} R)$. Hence $\xi \iota_\Ccal(X)$ is an object of $\Mod_R \times_{\Mod_{R \oplus M[1]}} \Mod_R$ whose coordinates are given by 
 \[
 (R \otimes_{R \oplus^\delta M} R \otimes_R L', R \otimes_{R \oplus^\delta M} (R\oplus M[1]), R \otimes_{R \oplus^\delta M} R).
\]
It follows that $\xi \iota_\Ccal(X)$ is a shift of an object in $\Mod_R^\cn \times_{\Mod^\cn_{R \oplus M[1]}} \Mod_R^\cn$. The fact that it belongs to the image of $\iota$ now follows from \cite{SAG} theorem 16.2.0.2.

We now have a well defined $R \oplus^\delta M$-linear functor $f: \Ccal \rightarrow \Mod_{R \oplus^\delta M}$ with the property that it recovers the equivalence $\varphi$ after tensoring with $R$. The proof concludes by an application of lemma \ref{lemma conservative pullback}.
\end{proof}

\begin{lemma}\label{lemma deal with filtered colimits stable}
Let $R_\alpha$ be a filtered diagram of commutative rings with colimit $R$. Assume given an index $\alpha_0$ and a smooth $R_{\alpha_0}$-linear presentable stable category $\ccal_{\alpha_0}$. If $\ccal_{\alpha_0} \otimes_{R_{\alpha_0}} R$ has a compact generator then there exists a transition $\alpha_0 \rightarrow \alpha$ such that $\ccal_{\alpha_0} \otimes_{R_{\alpha_0}} R_\alpha$ has a compact generator.
\end{lemma}
\begin{proof}
Analogous to the proof of lemma \ref{lemma deal with filtered colimits classical}.
\end{proof}

\begin{lemma}\label{lemma postnikov compact generator}
Let $R$ be a truncated connective $E_\infty$-ring and let $\ccal$ be an invertible $R$-linear presentable stable category. Assume that $\ccal \otimes_R \pi_0(R)$ admits a compact generator. Then $\ccal$ admits a compact generator.
\end{lemma}
\begin{proof}
By \cite{AGBrauer} theorem 5.11 we may find a faithfully flat \'etale $\pi_0(R)$ algebra $S_0$ such that $\ccal \otimes_{R} S_0$ is equivalent to $\Mod_{S_0}$ as an $S_0$-linear category. Let $S$ be a faithfully flat \'etale $R$-algebra such that $S \otimes_R \pi_0(R) = S_0$. We have $\pi_0(S) = S_0$, and hence $(\ccal \otimes_R S) \otimes_S \pi_0(S)$ is equivalent to $\Mod_{\pi_0(S)}$ as a $\pi_0(S)$-linear category. An inductive application of lemma \ref{lemma deal with square zero para stable} shows that $\ccal \otimes_R S$ is equivalent to $\Mod_S$. The lemma now follows from \cite{AGBrauer} theorem 6.16. 
\end{proof}

\begin{notation}
Let $R$ be a commutative ring and let $\ccal$ be an $R$-linear presentable stable category. Let $x$ be an element of $R$. We denote by $\ccal_{x\normalfont{\text{-nil}}}$ the kernel of the extension of scalars functor $\ccal \rightarrow \ccal \otimes_R R[x^{-1}]$.
\end{notation}

\begin{lemma}\label{lemma compact generator on completion}
Let $R$ be a commutative ring and $\ccal$ be an invertible $R$-linear presentable stable category. Let $x$ be an element of $R$ and assume that $\ccal \otimes_R R/Rx$ admits a compact generator. Then $\ccal_{x\normalfont{\text{-nil}}}$ admits a compact generator.
\end{lemma}
\begin{proof}
By \cite{AGBrauer} theorem 5.11 we may find a faithfully flat \'etale $R/Rx$ algebra $S_0$ such that $\ccal \otimes_R S_0$ is equivalent to $\Mod_{S_0}$ as an $S_0$-linear category. Let $S$ be a faithfully flat \'etale $R$-algebra such that $S \otimes_R R/Rx = S_0$. By \cite{AGBrauer} theorem 6.16. it suffices to show that $\ccal_{x\normalfont{\text{-nil}}} \otimes_R S$ admits a compact generator.  This category is equivalent to $(\ccal \otimes_R S)_{y\normalfont{\text{-nil}}}$ where $y$ is the image of $x$ in $S$. Replacing $R$ by $S$ we may now assume that $\ccal \otimes_R R/Rx$ is equivalent to $\Mod_{R/Rx}$ as an $R/Rx$-linear category.

An iterated application of lemma \ref{lemma deal with square zero para stable} identifies the inverse system of categories
\[
\ccal \otimes_R R/Rx \leftarrow \ccal \otimes_R R/Rx^2 \leftarrow \ccal \otimes_R R/Rx^3 \leftarrow \ldots
\]
with the inverse system
\[
\Mod_{R/Rx} \leftarrow \Mod_{R/Rx^2} \leftarrow \Mod_{R/Rx^3} \leftarrow \ldots.
\]
Passing to limits, we obtain an equivalence
\[
\ccal_{x\normalfont{\text{-nil}}} = (\Mod_R)_{x\normalfont{\text{-nil}}}
\]
and the lemma now follows from the fact that the right hand side admits a compact generator (namely, the cofiber of $x: R \rightarrow R$).
\end{proof}

\begin{lemma}\label{lemma handle extensions}
Let $R$ be a commutative ring and $\ccal$ be an invertible $R$-linear presentable stable category. Let $x$ be an element of $R$ and assume that both $\ccal \otimes_R R/Rx$ and $\ccal \otimes_R R[x^{-1}]$ admit a compact generator. Then $\ccal$ admits a compact generator. 
\end{lemma}
\begin{proof}
We have a short exact sequence of stable categories
\[
0 \rightarrow \ccal_{x\text{-nil}} \rightarrow \ccal \rightarrow \ccal \otimes_R R[x^{-1}] \rightarrow 0
\]
where the functors admit colimit preserving right adjoints. The third category is assumed to be compactly generated, while the first category is compactly generated by lemma \ref{lemma compact generator on completion}. By \cite{Efimov} proposition 3.3 we have that $\ccal$ is compactly generated. The fact that $\ccal$ admits a compact generator now follows from \cite{AGBrauer} lemma 3.9.
\end{proof}

\begin{proof}[Proof of theorem \ref{theorem stable over R}]
By proposition \ref{proposition dualizable is presentable} we see that $\Ccal$ and $\Ccal^\vee$ are presentable. Our goal is to show that $\ccal$ admits a compact generator. By lemma \ref{lemma postnikov compact generator} we may reduce to the case when $R$ is $0$-truncated.

 Assume for the sake of contradiction that $\ccal$ does not admit a compact generator. Consider the poset $P$ consisting of those ideals $I$ of $R$ with the property that $\ccal \otimes_R R/I$ does not admit a compact generator (where we order the ideals by inclusion). It follows from lemma \ref{lemma deal with filtered colimits stable} that $P$ is closed under filtered colimits inside the poset of all ideals of $R$. Since $P$ is nonempty (as it contains $0$) we deduce that $P$ has a maximal element $I$. Replacing $R$ by $R/I$ we may reduce to the case when $I = 0$.  In other words, we may assume that  $\ccal \otimes_R R/J$ admits a compact generator for every nonzero ideal $J$. 

We claim that $R$ is reduced. Let $x$ be an element of $R$ such that $x^2 = 0$. An application of lemma \ref{lemma handle extensions} shows that $\ccal \otimes_{R} R/{Rx}$ does not admit a compact object. It follows that $Rx = 0$, so that $x = 0$ and $R$ is reduced, as claimed.

If $R$ is the zero ring the desired assertion is clear, so suppose now that $R$ is nonzero. We claim that $R$ is an integral domain. Let $x$ be a nonzero element of $R$, and suppose given another element $y$ such that $xy = 0$. By lemma \ref{lemma handle extensions} we see that $\ccal \otimes_R R[x^{-1}]$ does not admit a compact generator. Since the map $R \rightarrow R[x^{-1}]$ factors through $R/Ry$ we deduce that $\ccal \otimes_R R/Ry$ does not admit a compact generator. Hence $y = 0$, so that $R$ is an integral domain, as desired.

Let $F$ be the fraction field of $R$. By theorem \ref{theorem stable} we have that $\ccal \otimes_R F$ admits a compact generator. Applying lemma \ref{lemma deal with filtered colimits stable} we deduce the existence of a nonzero element $x$ of $R$ such that $\ccal \otimes_R R[x^{-1}]$ admits a compact generator. An application of lemma \ref{lemma handle extensions} shows that $\ccal$ admits a compact generator, which is a contradiction.
\end{proof}

%%%%%%%%%%%%%%%%%%%%%%%%%%%%%%%%%%%%%%%%%%%%%%%%%%%%%%%%%%%%%%%%%%%%%%%%
%%%%%%%%%%%%%%%%%%%%%%%%%%%%%%%%%%%%%%%%%%%%%%%%%%%%%%%%%%%%%%%%%%%%%%%%
%%%%%%%%%%%%%%%%%%%%%%%%%%%%%%%%%%%%%%%%%%%%%%%%%%%%%%%%%%%%%%%%%%%%%%%%
%%%%%%%%%%%%%%%%%%%%%%%%%%%%%%%%%%%%%%%%%%%%%%%%%%%%%%%%%%%%%%%%%%%%%%%%
%%%%%%%%%%%%%%%%%%%%%%%%%%%%%%%%%%%%%%%%%%%%%%%%%%%%%%%%%%%%%%%%%%%%%%%%
%%%%%%%%%%%%%%%%%%%%%%%%%%%%%%%%%%%%%%%%%%%%%%%%%%%%%%%%%%%%%%%%%%%%%%%%

\ifx\inmain\undefined
\bibliographystyle{myamsalpha2}
\bibliography{References}
\fi
\end{document}
