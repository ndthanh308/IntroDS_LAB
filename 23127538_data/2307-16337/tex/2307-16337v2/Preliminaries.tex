\documentclass[12pt]{amsart}

\input{"Preamble"}

%\usepackage{standalone} 
%\def\inmain{1}

\title{Linear categories}

\author{G. Stefanich}

\externaldocument{Introduction}
\externaldocument{Semisimple}
\externaldocument{GRings}

\date{}

\begin{document}

%%%%%%%%%%%%%%%%%%%%%%%%%%%%%%%%%%%%%%%%%%%%%%%%%%%%%%%%%%%%%%%%%%%%%%%%
%%%%%%%%%%%%%%%%%%%%%%%%%%%%%%%%%%%%%%%%%%%%%%%%%%%%%%%%%%%%%%%%%%%%%%%%
%%%%%%%%%%%%%%%%%%%%%%%%%%%%%%%%%%%%%%%%%%%%%%%%%%%%%%%%%%%%%%%%%%%%%%%%
%%%%%%%%%%%%%%%%%%%%%%%%%%%%%%%%%%%%%%%%%%%%%%%%%%%%%%%%%%%%%%%%%%%%%%%%
%%%%%%%%%%%%%%%%%%%%%%%%%%%%%%%%%%%%%%%%%%%%%%%%%%%%%%%%%%%%%%%%%%%%%%%%
%%%%%%%%%%%%%%%%%%%%%%%%%%%%%%%%%%%%%%%%%%%%%%%%%%%%%%%%%%%%%%%%%%%%%%%%

\section{Linear categories}

This section contains preliminary material on the theory of linear categories that will be used throughout the paper. We begin in \ref{subsection linear cats} with a review of the notion of cocomplete category linear over a commutative algebra in $\catl$.  This recovers in particular the notion of cocomplete category linear over a (connective) commutative ring spectrum. We include here a proof of the fact that dualizable categories linear over a presentable base are automatically presentable, which forms the first step in the proof of the main theorems of this paper.

In \ref{subsection abelian} we study the theory of Grothendieck abelian categories linear over a base symmetric monoidal Grothendieck abelian category $\Acal$. We show that some basic aspects of the theory of Grothendieck abelian categories (tensor products, Gabriel-Popescu theorem) hold in this relative context as long as we require $\Acal$ to be generated by compact projective objects and rigid. We then discuss the notion of flatness for objects in an $\Acal$-linear Grothendieck abelian category, which will be needed in section \ref{section G rings}.

In \ref{subsection spectral categories} we review the notions of spectral and semisimple categories, and prove an $\acal$-linear version of a basic structure result from \cite{Spectral} that relates spectral categories to self-injective von Neumann regular algebras. This will be used in our classification of smooth categories over a rigid semisimple base in section \ref{section invertible semisimple}.

Finally, in \ref{subsection prestables} we review the theory of Grothendieck prestable categories from \cite{SAG}, and discuss a version relative to a base symmetric monoidal Grothendieck prestable category. For the most part, the material here is parallel to  that of \ref{subsection abelian}. We also include a general discussion of how linearity interacts with the passage to derived categories.

\subsection{General notions}\label{subsection linear cats}

We begin with some  background on the notion of linear category.

\begin{definition}
Let $\Mcal$ be a commutative algebra in $\catl$. An $\Mcal$-linear cocomplete category is an $\Mcal$-module in $\catl$. An $\Mcal$-linear colimit preserving functor is a morphism of $\Mcal$-modules in $\catl$.
\end{definition}

\begin{example}
Let $\Mcal$ be a commutative algebra in $\catl$. Then $\Mcal$ has a structure of   $\Mcal$-linear cocomplete category. For every   $\Mcal$-linear cocomplete category $\ccal$, evaluation at the unit induces an equivalence between  the category of   $\mcal$-linear colimit preserving functors $\Mcal \rightarrow \ccal$ and $\ccal$. The inverse to this  equivalence associates to each object $X$ in $\ccal$ an $\Mcal$-linear enhancement of the functor $- \otimes X: \Mcal \rightarrow \ccal$. We may summarize this by saying that $\Mcal$ is the free $\Mcal$-linear cocomplete category on one object.
\end{example}

\begin{example}\label{example LMod}
Let $\Mcal$ be a commutative algebra in $\catl$ and let $A$ be an algebra in $\Mcal$. Then the category $\LMod_A(\Mcal)$ of left $A$-modules in $\Mcal$ has a structure of $\Mcal$-linear cocomplete category. Thinking about $A$ as a left $A$-module we obtain an object of $\LMod_A(\Mcal)$, which itself admits a right $A$-module structure. In fact $A$ is the universal algebra in $\Mcal$ equipped with a right action on the left $A$-module $A$: in other words, $A$ is (the opposite of) the algebra of endomorphisms of the left $A$-module $A$.

Assume now given another   $\Mcal$-linear  cocomplete category $\Ccal$. Then for each  $\Mcal$-linear colimit preserving functor $f: \LMod_A(\Mcal) \rightarrow \Ccal$ we obtain a right $A$-module $f(A)$ in $\Ccal$. The assignment $f \mapsto f(A)$ turns out to induce an equivalence between the category of  $\Mcal$-linear colimit preserving functors $\LMod_A(\Mcal) \rightarrow \Ccal$ and the category of right $A$-modules in $\ccal$ (\cite{HA} theorem 4.8.4.1). The inverse to this equivalence maps a right $A$-module $M$ in $\Ccal$ to an $\Mcal$-linear enhancement of  the relative tensor product functor $M \otimes_A - : \LMod_A(\Mcal) \rightarrow \Ccal$. We may summarize this by saying that $\LMod_A(\Mcal)$ is the universal   $\Mcal$-linear cocomplete  category on a right $A$-module.
\end{example}

If $\Mcal$ is a commutative algebra in $\catl$ then $\Mod_{\Mcal}(\catl)$ inherits a closed symmetric monoidal structure from $\catl$. It makes sense in particular to consider dualizable and invertible objects in $\Mod_{\Mcal}(\catl)$. The following proposition shows that as long as $\Mcal$ is presentable, dualizability automatically implies presentability.

\begin{proposition}\label{proposition dualizable is presentable}
Let $\Mcal$ be a presentable symmetric monoidal category and let $\ccal$ be a dualizable object of $\Mod_\Mcal(\catl)$. Then $\ccal$ is presentable.
\end{proposition}
\begin{proof}
Let $\kappa$ be the smallest large cardinal. It is proven in \cite{Pres} section 5.1 (in particular, proposition 5.1.7 and corollary 5.1.15) that $\Mod_\Mcal(\catl)$ is a very large presentable category $\kappa$-compactly generated by those $\Mcal$-modules which belong to $\Pr^L$. Since the symmetric monoidal structure on $\Mod_\Mcal(\catl)$ is compatible with large colimits and the unit is $\kappa$-compact, we have that every dualizable object is $\kappa$-compact and therefore presentable.
\end{proof}

For each commutative algebra $\Mcal$ in $\catl$ we will denote by $- \otimes_{\Mcal}-$ the tensor product on $\Mod_\Mcal(\catl)$, and by $\Funct_{\Mcal}(-,-)$ the internal Hom. If $\Mcal$ is presentable, then these bifunctors restrict to $\Mod_{\Mcal}(\Pr^L)$.

\begin{example}\label{example tensor with amod}
Let $\Mcal$ be a commutative algebra in $\catl$. Let $A$ be an algebra in $\Mcal$ and let $\Ccal$ be an   $\Mcal$-linear cocomplete category. Then we have an $\Mcal$-bilinear functor 
\[
\LMod_A(\Mcal) \times \Ccal \rightarrow \LMod_A(\Ccal)
\]
that sends a pair $(M, X)$ to $M \otimes X$.  This induces an equivalence 
\[
\LMod_A(\Mcal) \otimes_\Mcal \Ccal = \LMod_A(\Ccal)
\]
 (see \cite{HA} theorem 4.8.4.6).  In particular, if $\Ccal$ is the category $\RMod_B(\Mcal)$ of right modules over some algebra $B$ in $\Mcal$ we obtain an equivalence 
\[
\LMod_A(\Mcal) \otimes_\Mcal \RMod_B(\Mcal) = \LMod_A(\RMod_B(\Mcal)) = {}_A\kr\BMod_B(\Mcal).
\]
\end{example}

\begin{example}
Let $\Mcal$ be a commutative algebra in $\catl$ and let $A$ be an algebra in $\Mcal$. Then by example \ref{example tensor with amod} we have an equivalence 
\[
\LMod_A(\Mcal) \otimes_\Mcal \RMod_A(\Mcal) = \LMod_A(\RMod_B(\Mcal)) = {}_A\kr\BMod_A(\Mcal).
\]
 The diagonal bimodule for $A$ defines an object in $\LMod_A(\Mcal) \otimes_\Mcal \RMod_A(\Mcal)$, which then extends uniquely to an $\Mcal$-linear colimit preserving functor 
 \[
 \eta: \Mcal \rightarrow \LMod_A(\Mcal) \otimes_\Mcal \RMod_A(\Mcal).
 \]
 As discussed in \cite{HA} remark 4.8.4.8, the map $\eta$ exhibits $\LMod_A(\Mcal)$ and $\RMod_A(\Mcal)$ as dual objects in $\Mod_{\Mcal}(\catl)$. 
\end{example}

\begin{remark}
Let $\Mcal$ be a commutative algebra in $\catl$. Then we have a symmetric monoidal colimit preserving functor $\Cat \rightarrow \Mod_\Mcal(\catl)$ obtained by composing the free cocompletion functor $\Cat \rightarrow \catl$ with the free module functor $\catl \rightarrow \Mod_\mcal(\catl)$. It follows from this that $\Mod_\Mcal(\catl)$ has a structure of symmetric monoidal $2$-category, with the Hom category between two objects $\ccal$ and $\dcal$ being given by the category underlying $\Funct_{\Mcal}(\ccal, \dcal)$.

It makes sense in particular to consider adjunctions in $\Mod_\Mcal(\catl)$. Given an $\Mcal$-linear colimit preserving functor $f: \ccal \rightarrow \dcal$, we have that $f$ admits a right (resp. left) adjoint in $\Mod_\mcal(\catl)$ if and only if it admits a colimit preserving right (resp. left) adjoint as a functor of categories, which commutes strictly with the action of $\Mcal$.
\end{remark}

Assume given a morphism $f: \Mcal \rightarrow \Mcal'$ of commutative algebras in $\catl$. Then we obtain a symmetric monoidal extension of scalars functor 
\[
- \otimes_\Mcal \Mcal' : \Mod_{\Mcal}(\catl) \rightarrow \Mod_{\Mcal'}(\catl),
\] and a restriction of scalars right adjoint to it. If $\Ccal$ is an $\Mcal$-linear cocomplete category then the unit of the adjunction provides an $\Mcal$-linear colimit preserving functor $\Ccal \rightarrow \Ccal \otimes_\Mcal \Mcal'$ which we call extension of scalars along $f$. A right adjoint to it (which automatically exists if we work with presentable categories) is called restriction of scalars along $f$.

\begin{example}
Let $f: \Mcal \rightarrow \Mcal'$ be a morphism of commutative algebras in $\catl$. Let $A$ be an algebra in $\Mcal$ and consider the algebra $f(A)$ in $\Mcal'$. We may regard $\LMod_{f(A)}(\Mcal')$ as a cocomplete $\Mcal$-linear category by restriction of scalars long $f$. Then $f(A)$ becomes a right $A$-module in $\LMod_{f(A)}(\Mcal')$, so it induces an $\Mcal$-linear colimit preserving functor $\LMod_A(\Mcal) \rightarrow \LMod_{f(A)}(\Mcal')$, as discussed in example \ref{example LMod}. It follows from the universal properties of $\LMod_A(\Mcal)$ and $\LMod_{f(A)}(\Mcal')$  that this functor induces an equivalence 
\[
\LMod_A(\Mcal) \otimes_{\Mcal} \Mcal' = \LMod_{f(A)}(\Mcal').
\]
\end{example}

\begin{example}\label{example stabilize linear}
 Let $\Mcal$ be a commutative algebra in $\catl$ and let $\Mcal' = \Mcal \otimes \Sp$ be the stabilization of $\Mcal$. Then for each  $\Mcal$-linear cocomplete category $\ccal$ the extension of scalars functor $\Ccal \rightarrow \Ccal \otimes_\Mcal \Mcal'$ presents $\Ccal \otimes_\Mcal \Mcal'$ as the stabilization of $\Ccal$. In other words, the stabilization of an $\Mcal$-module is automatically a module over the stabilization of $\Mcal$. It follows in particular that the functor of restriction of scalars $\Mod_{\Mcal'}(\catl) \rightarrow \Mod_{\Mcal}(\catl)$ is fully faithful, and its image consists of those $\Mcal$-modules which are stable.
\end{example}

\begin{example}\label{example truncate linear}
Let $\Mcal$ be  a commutative algebra in $\catl$. Fix an $n \geq 1$ and  let $\Mcal' = \Mcal \otimes \Spc_{\leq n-1}$. Then for each $\Mcal$-linear cocomplete category $\ccal$ the extension of scalars functor $\Ccal \rightarrow \Ccal \otimes_{\Mcal} \Mcal'$ induces an equivalence $\Ccal \otimes_{\Mcal} \Mcal' = \Ccal \otimes \Spc_{\leq n-1}$. It follows in particular that the functor of restriction of scalars $\Mod_{\Mcal'}(\catl) \rightarrow \Mod_{\Mcal}(\catl)$ is fully faithful, and its image consists of those $\Mcal$-modules which are $(n,1)$-categories.
\end{example}


We now specialize the above discussion to obtain a notion of cocomplete categories linear over a base connective commutative ring spectrum.

\begin{definition}\label{def R linear cocomplete}
Let $R$ be a connective commutative ring spectrum. An $R$-linear cocomplete category is a $\Mod_R^\cn$-linear cocomplete category. An $R$-linear colimit preserving functor is a morphism in $\Mod_{\Mod_R^\cn}(\catl)$.
\end{definition} 

\begin{remark}
Let $R$ be a connective commutative ring spectrum. Then every $R$-linear cocomplete category is automatically additive. 
\end{remark}

\begin{remark}
Let $R$ be a connective commutative ring spectrum and let $\Ccal$ be an $R$-linear cocomplete category. The action of $\Mod_R^\cn$ on $\Ccal$ provides a monoidal functor $\Mod_R^\cn \rightarrow \Funct(\Ccal, \Ccal)$, which after passing to endomorphisms of the identity yields an $E_2$-map from $R$ into the center of $\Ccal$. In particular, given an element $x$ in $R$ and an object $X$ in $\Ccal$ we have an endofunctor of $X$ given by
\[
X = R \otimes X \xrightarrow{x \otimes \id} R \otimes X = X
\]
which we usually denote by $x: X \rightarrow X$ and call the action of $x$ on $X$.
\end{remark}

\begin{remark}
If $R$ is a (non necessarily connective) commutative ring spectrum then one may consider $\Mod_R$-linear cocomplete categories. We call these $R$-linear cocomplete stable categories. In the case when $R$ is connective, it follows from example \ref{example stabilize linear} that an $R$-linear cocomplete stable category in this sense is the same as an $R$-linear cocomplete category in the sense of definition \ref{def R linear cocomplete} which is in addition stable.
\end{remark}

\begin{remark}
For every connective commutative ring spectrum $R$ one may consider $(\Mod^\cn_R)_{\leq n-1}$-linear categories. We call these $R$-linear cocomplete $(n,1)$-categories. It follows from example \ref{example truncate linear} that an $R$-linear cocomplete $(n,1)$-category in this sense is the same as an $R$-linear cocomplete category in the sense of definition \ref{def R linear cocomplete} which is in addition an $(n,1)$-category. This will frequently be used in the case when $n = 0$ and $R$ is a (classical) commutative ring: in this case one obtains a notion of classical $R$-linear cocomplete category, which is simply a $\Mod^\heartsuit_R$-linear category.
\end{remark}

\begin{example}
Let $R$ be a connective commutative ring spectrum and let $\ccal$ be an $R$-linear cocomplete category. Then specializing examples \ref{example stabilize linear} and \ref{example truncate linear} yields the following:
\begin{itemize}
\item The stabilization $\ccal \otimes \Sp$ has a structure of $R$-linear cocomplete stable category.
\item For each $n \geq 0$ the $(n,1)$-category $\ccal \otimes \Spc_{\leq n-1}$ as a structure of $\tau_{\leq n-1}(R)$-linear cocomplete $(n,1)$-category.
\end{itemize}
\end{example}

\begin{remark}
Let $f: R \rightarrow R'$ be a morphism of connective commutative ring spectra. Then we obtain a symmetric monoidal extension of scalars functor $f^*: \Mod^\cn_R \rightarrow \Mod^\cn_{R'}$. For each $R$-linear cocomplete category $\Ccal$ we will denote by $\Ccal \otimes_R R'$ its extension of scalars along $f^*$, and by $- \otimes_R R': \Ccal \rightarrow \Ccal \otimes_R R'$ the corresponding extension of scalars functor. The composition $\Ccal \rightarrow \Ccal \otimes_R R' \rightarrow \Ccal$ is given by tensoring with the $R$-module $R'$, while the unit of the adjunction is given by tensoring with the map of $R$-modules $R \rightarrow R'$. Similar considerations apply to the case when $R$ and $R'$ are non necessarily connective commutative ring spectra and we extend scalars along $f^*: \Mod_R  \rightarrow \Mod_{R'} $.
\end{remark}

%%%%%%%%%%%%%%%%%%%%%%%%%%%%%%%%%%%%%%%%%%%%%%%%%%%%%%%%%%%%%%%%%%%%%%%%
%%%%%%%%%%%%%%%%%%%%%%%%%%%%%%%%%%%%%%%%%%%%%%%%%%%%%%%%%%%%%%%%%%%%%%%%
%%%%%%%%%%%%%%%%%%%%%%%%%%%%%%%%%%%%%%%%%%%%%%%%%%%%%%%%%%%%%%%%%%%%%%%%
%%%%%%%%%%%%%%%%%%%%%%%%%%%%%%%%%%%%%%%%%%%%%%%%%%%%%%%%%%%%%%%%%%%%%%%%
%%%%%%%%%%%%%%%%%%%%%%%%%%%%%%%%%%%%%%%%%%%%%%%%%%%%%%%%%%%%%%%%%%%%%%%%
%%%%%%%%%%%%%%%%%%%%%%%%%%%%%%%%%%%%%%%%%%%%%%%%%%%%%%%%%%%%%%%%%%%%%%%%

\subsection{Grothendieck abelian categories}\label{subsection abelian}

We now proceed with some recollections on the theory of Grothendieck abelian categories.

\begin{definition}\label{def grothendieck abelian}
A Grothendieck abelian category is an abelian category $\Ccal$ which is presentable and such that filtered colimits in $\Ccal$ are exact. We denote by $\Groth_1$ the category of Grothendieck abelian categories and colimit preserving functors.
\end{definition}

Definition \ref{def grothendieck abelian} is equivalent to the (perhaps more common) definition where presentability of $\Ccal$ is replaced by the requirement that $\Ccal$ is locally small, admits small colimits, and admits a generator. The following result provides an ample source of examples.

\begin{proposition}[\cite{SAG} proposition 10.6.3.1]
Let $\Ccal, \Dcal$ be presentable categories and assume given a functor $G: \Dcal \rightarrow \Ccal$ which is conservative and preserves small limits\footnote{In fact only preservation of finite limits is necessary.} and colimits. If $\Ccal$ is a Grothendieck abelian category, then so is $\Dcal$.
\end{proposition}

\begin{corollary}
Let $\Acal$ be a Grothendieck abelian category equipped with a monoidal structure compatible with colimits. Let $A$ be an algebra in $\Acal$. Then the category $\LMod_A(\Acal)$ of left $A$-modules in $\Acal$ is a Grothendieck abelian category.
\end{corollary}

Limits of Grothendieck abelian categories exist along left exact colimit preserving functors:

\begin{proposition}[\cite{SAG} proposition C.5.4.21]\label{prop limits along left exact}
Let $F: \Ical \rightarrow \Groth_1$ be a diagram whose transition maps are left exact. Then $F$ admits a limit which is preserved by the inclusion of $\Groth_1$ inside $\Pr^L$.
\end{proposition}

\begin{corollary}
The category $\Groth_1$ admits small products, and these are preserved by the inclusions $\Groth_1 \rightarrow \Pr^L \rightarrow \cathat$.
\end{corollary}

We can also form colimits of diagrams of right adjointable diagrams:

\begin{proposition}\label{proposition colimits of right adjointable}
Let $F: \Ical \rightarrow \Groth_1$ be a diagram whose transition maps admit colimit preserving right adjoints. Then $F$ admits a colimit which is preserved by the inclusion of $\Groth_1$ inside $\Pr^L$.
\end{proposition}
\begin{proof}
It suffices to show that the colimit of $F$ in $\Pr^L$ is Grothendieck abelian. This colimit agrees with the limit of the diagram $F^R : \Ical^\op \rightarrow \cathat$ obtained by passing to right adjoints of the morphisms in $F$. This a diagram of Grothendieck abelian categories and left exact colimit preserving functors. The fact that the limit is Grothendieck abelian now follows from proposition \ref{prop limits along left exact}.
\end{proof}

\begin{corollary}
The category $\Groth_1$ admits small direct sums, and these are preserved by the inclusion $\Groth_1 \rightarrow \Pr^L$. In particular, small direct sums and small direct products agree in $\Groth_1$.
\end{corollary}

There is a good theory of tensor products of Grothendieck abelian categories:

\begin{theorem}[\cite{SAG} theorem C.5.4.16, \cite{Tensor} theorem 5.4]\label{teo tensor product abelian}
Let $\Ccal, \Dcal$ be Grothendieck abelian categories. Then their tensor product $\Ccal \otimes \Dcal$ (formed in $\Pr^L$) is Grothendieck abelian. In particular, the symmetric monoidal structure on the category $\Mod_{\Ab}(\Pr^L)$ of presentable additive $(1,1)$-categories and colimit preserving functors restricts to a symmetric monoidal structure on $\Groth_1$.
\end{theorem}

In particular, it makes sense to consider commutative algebras in $\Groth_1$. We call these symmetric monoidal Grothendieck abelian categories. Note that this terminology leaves implicit the fact that the tensor operation commutes with colimits in each variable.

\begin{definition}
Let $\acal$ be a symmetric monoidal Grothendieck abelian category. An $\acal$-linear Grothendieck abelian category is an object of $\Mod_\Acal(\Groth_1)$.
\end{definition}

In other words, an $\acal$-linear Grothendieck abelian category  is an $\acal$-linear presentable category (in the sense of section \ref{subsection linear cats}) which is in addition a Grothendieck abelian category. In the case when $\acal = \Mod_R^\heartsuit$ is the category of classical modules over a connective $E_\infty$-ring $R$, we call these $R$-linear Grothendieck abelian categories.

Since the class of colimits that we have available in $\Groth_1$ is relatively restricted, one has to be careful when forming relative tensor products. There is however a class of commutative algebras that admit a well behaved theory of relative tensor products.

\begin{definition}
Let $\acal$ be a commutative algebra in $\groth$. Assume that $\acal$ is generated by compact projective objects. We say that $\acal$ is rigid if compact projective and dualizable objects of $\acal$ coincide.
\end{definition}

\begin{example}
Let $R$ be a commutative ring. Then $\Mod_R(\Ab)$ is rigid.
\end{example}

We fix for the remainder of this section a base symmetric monoidal Grothendieck abelian category $\acal$, generated by compact projective objects and rigid.

\begin{remark}\label{remark tensor compact projectives}
 Let $\Ccal$ be an $\Acal$-linear Grothendieck abelian category. Then if $X$ is a compact projective object of $\Acal$ the functor $X \otimes - : \Ccal \rightarrow \Ccal$ admits both a left and a right adjoint, given by $X^\vee \otimes -$. In particular, $X \otimes -$ admits a colimit preserving right adjoint, and therefore it maps compact projective objects of $\Ccal$ to compact projective objects.
\end{remark}

\begin{remark}
Let $f: \Ccal \rightarrow \Dcal$ be a morphism in $\Mod_\Acal(\catl)$ and assume that $f$ admits a right adjoint $f^R$ (as a functor of categories, ignoring the $\Acal$-action). Then $f^R$ commutes laxly with the action of $\Acal$, and strictly with the action of the dualizable objects in $\Acal$. In particular, since $\Acal$ is generated under colimits by dualizable objects, we see that if $f^R$ is colimit preserving then it commutes strictly with the action of $\Acal$.
\end{remark}

\begin{proposition}\label{proposition action has colimit preserving adjoint}
Let $\ccal$ be an $\acal$-module in $\catl$. Then the action map $\acal \otimes \ccal \rightarrow \ccal$ admits a colimit preserving right adjoint.
\end{proposition}
\begin{proof}
We first prove the proposition in the case when $\Ccal = \Acal \otimes \Ccal'$ is a free $\Acal$-module. In this case the action map is obtained by tensoring the multiplication map $\mu: \Acal \otimes \Acal \rightarrow \Acal$ with the identity on $\Ccal'$. We may thus reduce to showing that $\mu$ admits a colimit preserving right adjoint. This is a consequence of the fact that $\Acal$ is generated by compact projective objects and that such objects are preserved by tensor products.

We now prove the general case. Consider the Bar resolution $\Acal^{\otimes \bullet + 1} \otimes \Ccal$ of $\Ccal$. Then the action map $\Acal \otimes \Ccal \rightarrow \Ccal$ is the colimit of the action maps $\Acal \otimes (\Acal^{\otimes \bullet + 1} \otimes \Ccal) \rightarrow \Acal^{\otimes \bullet + 1} \otimes \Ccal$. Since the Bar resolution is levelwise free we see that each of these maps admits a colimit preserving right adjoint. To prove the proposition it remains to show that for each face map $\sigma: [n] \rightarrow [n+1]$ in $\Delta$ the induced commutative square
\begin{equation}
\begin{tikzcd}
\Acal \otimes (\Acal^{\otimes n+2} \otimes \Ccal) \arrow{d}{} \arrow{r}{} & \Acal \otimes (\Acal^{\otimes n+1} \otimes \Ccal) \arrow{d}{} \\
\Acal^{\otimes n+2} \otimes \Ccal \arrow{r}{}  & \Acal^{\otimes n+1} \otimes \Ccal
\end{tikzcd}
\end{equation}
is vertically right adjointable. If $\sigma$ is not the $0$-th face then the above square is a tensor product of the square
\[
\begin{tikzcd}
\Acal \otimes \Acal \arrow{d}{\mu} \arrow{r}{\id} & \Acal \otimes \Acal \arrow{d}{\mu} \\
\Acal \arrow{r}{\id} & \Acal 
\end{tikzcd}
\] 
with
\[
\begin{tikzcd}
\Acal^{\otimes n+1} \otimes \Ccal \arrow{d}{\id} \arrow{r}{} & \Acal^{\otimes n} \otimes \Ccal \arrow{d}{\id} \\
\Acal^{\otimes n+1} \otimes \Ccal \arrow{r}{}  & \Acal^{\otimes n} \otimes \Ccal
\end{tikzcd}
\]
and our claim follows from the fact that these two are vertically right adjointable. It remains to analyze the case when $\sigma$ is the $0$-th face. In this case the square (1) is obtained by tensoring the square
\[
\begin{tikzcd}
\Acal \otimes \Acal \otimes \Acal \arrow{d}{\mu \otimes \id} \arrow{r}{\id \otimes \mu} & \Acal \otimes \Acal \arrow{d}{\mu} \\
\Acal \otimes \Acal \arrow{r}{\mu} & \Acal
\end{tikzcd}
\]
with $\Acal^{\otimes n} \otimes \Ccal$.  We may thus reduce to showing that the above is vertically right adjointable. This amounts to showing that the right adjoint to the multiplication map $\Acal \otimes \Acal \rightarrow \Acal$ is $\Acal$-linear. This follows from the fact that $\Acal$ is generated under colimits by dualizable objects.
\end{proof}

\begin{corollary}\label{coro tensor products over A}
The full subcategory of $\Mod_\acal(\Pr^L)$ on the $\acal$-linear Grothendieck abelian categories is closed under tensor products. In other words, $\Mod_\acal(\groth_1)$ admits a symmetric monoidal structure that makes the inclusion $\Mod_\acal(\groth_1) \rightarrow \Mod_\acal(\Pr^L)$ symmetric monoidal.
\end{corollary}
\begin{proof}
Let $\Ccal$ and $\Dcal$ be a pair of $\acal$-linear Grothendieck abelian categories. The relative tensor product $\ccal \otimes_\acal \Dcal$ in $\Pr^L$  is the geometric realization of the Bar construction $\ccal \otimes \acal^{\otimes \bullet} \otimes \Dcal$. By virtue of proposition \ref{proposition colimits of right adjointable}, to show that $\ccal \otimes_\acal \Dcal$ is Grothendieck abelian, it suffices to show that the face maps in the Bar construction admit colimit preserving right adjoints. This follows from proposition \ref{proposition action has colimit preserving adjoint}.
\end{proof}

\begin{corollary}
Let $\ccal$ and $\dcal$ be a pair of $\acal$-linear Grothendieck abelian categories. Then the functor $\ccal \otimes \dcal \rightarrow \ccal \otimes_\acal \dcal$ admits a colimit preserving right adjoint. In particular, for each pair of compact projective objects $X$ in $\ccal$ and $Y$ in $\dcal$, the object $X \otimes Y$ in $\ccal \otimes_\acal \dcal$ is compact projective.
\end{corollary}
\begin{proof}
Follows directly from the fact that the functor $\ccal \otimes \dcal \rightarrow \ccal \otimes_\acal \dcal$ arises from the geometric realization of a simplicial diagram whose face maps have colimit preserving right adjoints.
\end{proof}

We will frequently use the following relative variant of the notion of generator:

\begin{definition}\label{definition A generator}
Let $\ccal$ be an $\acal$-linear Grothendieck abelian category. An object $G$ in $\ccal$ is said to be an $\acal$-generator if $\ccal$ is generated by the family of objects $X \otimes G$ with {$X$ in $\acal$.} 
\end{definition}

\begin{remark}
Let $\Ccal$ be an $\acal$-linear Grothendieck abelian category. Then an object $G$ in $\ccal$ is an $\acal$-generator if and only if $\ccal$ is generated by the family of objects $X \otimes G$ with $X$ a compact projective object of $\acal$.
\end{remark}

The following is an $\acal$-linear version of the Gabriel-Popescu theorem:

\begin{proposition}\label{prop lex localization}
Let $\Ccal$ be an $\Acal$-linear Grothendieck abelian category and let $G$ be an $\acal$-generator for $\Ccal$. Let $A$ be the opposite of the algebra of endomorphisms of $G$ associated to the action of $\Acal$ on $\Ccal$. Then the functor
\[
G \otimes_A - : \LMod_A(\Acal) \rightarrow \Ccal 
\]
is an $\Acal$-linear left exact localization. Furthermore, it is an equivalence if and only if $G$ is compact projective.
\end{proposition}
\begin{proof}
Let $\Ccal_0$ be the full subcategory of $\ccal$ on the objects of the form $X \otimes G$ where $X$ is a compact projective object of $\Acal$, and let $\Dcal$ be the full subcategory of $\LMod_A(\Acal)$ on the objects of the form $X \otimes A$ where $X$ is a compact projective object of $\Acal$. If $X, Y$ are compact projective objects of $\Acal$ then we have
\begin{align*}
\Hom_{\LMod_A(\Acal)}(X \otimes A, Y \otimes A) &= \Hom_{\LMod_A(\Acal)}(X \otimes Y^\vee \otimes A, A) \\ &= \Hom_{\Ccal}(X \otimes Y^\vee \otimes G, G)  \\ &= \Hom_{\Ccal}(X \otimes G, Y \otimes G)
\end{align*}
and therefore $G \otimes_A -$ restricts to an equivalence $\Dcal \rightarrow \Ccal_0$. We now have a commutative square of  categories
\[
\begin{tikzcd}
\Pcal^1_\Sigma(\Dcal) \arrow{d}{} \arrow{r}{} & \Pcal^1_\Sigma(\Ccal_0) \arrow{d}{} \\
\LMod_A(\Acal) \arrow{r}{ G \otimes_A -} & \Ccal
\end{tikzcd}
\]
where the categories on the top row are the $(1,1)$-categories obtained from $\Dcal$ and $\Ccal_0$ by freely adjoining sifted colimits, and the vertical arrows are the unique sifted colimit preserving extensions of the inclusions $\Dcal \rightarrow \LMod_A(\Acal)$ and $\Ccal_0 \rightarrow \Ccal$. The upper horizontal arrow is an equivalence since $G \otimes_A -$ restricts to an equivalence $\Dcal \rightarrow \Ccal_0$. Furthermore, $\Dcal$ is a generating family of compact projective objects of $\LMod_A(\Acal)$, and therefore the left vertical arrow is an equivalence as well. 

The fact that $G \otimes_A -$ is a left exact localization now follows from the fact that the functor $\Pcal^1_\Sigma(\Ccal_0) \rightarrow \Ccal$ is a left exact localization, due to the many object version of the classical Gabriel-Popescu theorem (see \cite{Kuhn} theorem 2.1, or theorem C.2.2.1 of \cite{SAG}). It remains to show that $G \otimes_A -$ is an equivalence if and only if $G$ is compact projective. The only if direction follows from the fact that $A$ is a compact projective left $A$-module. To prove the if direction, we observe that if $G$ is compact projective then $\Ccal_0$ is a generating family of compact projective objects of $\Ccal$, so that the functor $\Pcal^1_\Sigma(\Ccal_0) \rightarrow \Ccal$ is an equivalence.
\end{proof}

We finish this section with a discussion of flatness in the context of $\Acal$-linear Grothendieck abelian categories.

\begin{definition}\label{definition flat abelian}
Let $\Ccal$ be an $\Acal$-linear Grothendieck abelian category. We say that an object $X$ in $\Ccal$ is flat over $\acal$ if the functor $- \otimes X : \Acal \rightarrow \Ccal$ is left exact. In cases when the base $\acal$ is clear from the context we simply  say that $X$ is flat.
\end{definition}

\begin{example}
Let $R$ be a commutative ring and let $\Ccal$ be an $R$-linear Grothendieck abelian category. Since every monomorphism of $R$-modules is a transfinite composition of pushouts of inclusions of ideals into $R$, we have that an object $X$ in $\Ccal$ is flat if and only if the morphism
\[
I \otimes X \xrightarrow{i \otimes \id} R \otimes X = X
\]
is a monomorphism for all inclusions of ideals $i: I \rightarrow R$.
\end{example}

As a particular case of definition \ref{definition flat abelian} we obtain a notion of flatness for objects of $\acal$. These are characterized by the following variant of Lazard's theorem:

\begin{proposition}\label{prop lazard classico}
Let $X$ be an object of $\Acal$. Then $X$ is flat if and only if it is a filtered colimit of compact projective objects.
\end{proposition}
\begin{proof}
We first show that if $X$ is a filtered colimit of compact projective objects then it is flat. Since filtered colimits in $\Acal$ are left exact it suffices to consider the case when $X$ is compact projective. This follows from the fact that the functor $- \otimes X : \Acal \rightarrow \Acal$ has a left adjoint.

Assume now that $X$ is flat. Let $\acal^\cp$ be the full subcategory of $\acal$ on the compact projective objects and consider the functor $F(-): (\acal^\cp)^\op \rightarrow \Spc$ represented by $X$. We wish to show that this functor defines an ind-object of $\acal^\cp$. Let $D: \acal^\cp \rightarrow (\acal^\cp)^\op$ be the dualization equivalence. We will prove that $F(D(-)): \acal^\cp \rightarrow \Spc$ defines a pro-object of $\acal^\cp$.

Let $p: \Ecal \rightarrow \acal$ be the left fibration associated to the functor $\Hom_{\acal}(1_\acal, - \otimes X)$. Then the base change of $p$ to $\acal^\cp$ is the left fibration classifying $F(D(-))$. We have to show that every finite diagram $G: \Ical \rightarrow \Ecal \times_\acal \acal^\cp$ admits a left cone. The fact that $X$ is flat implies that the functor $\Hom_{\acal}(1_\acal, - \otimes X)$ is left exact, and therefore $G$ extends to a left cone $G^\lhd: \Ical^\lhd \rightarrow \Ecal$. Let $\overline{Y} = (Y, \rho: 1_\acal \rightarrow Y \otimes X)$ be the value of $G^\lhd$ at the cone point. To show that $G$ extends to a left cone in $\Ecal \times_\acal \acal^\cp$ it is enough to prove that $\overline{Y}$ receives a map from an object in $\Ecal \times_\acal \acal^\cp$. This amounts to showing that there exists a map $Y' \rightarrow Y$ from a compact projective object with the property that $\rho$ factors through $Y' \otimes X$. This follows from the fact that $1_\acal$ is compact projective.
\end{proof}

We now study the behavior of flatness under tensor products.

\begin{proposition}\label{prop tensoring left exact}
Let $f: \Ccal \rightarrow \Ccal'$ and $g: \Dcal \rightarrow \Dcal'$   morphisms in $\Mod_{\Acal}(\Groth_1)$. If $f$ and $g$ are left exact then $f \otimes_{\Acal} g : \Ccal \otimes_\acal \Dcal \rightarrow \Ccal' \otimes_\Acal \Dcal'$ is left exact.
\end{proposition}
\begin{proof}
Since $f \otimes_\Acal g$ is the composition of $f \otimes_\Acal \id_{\Dcal}$ and $\id_{\Ccal'} \otimes_\Acal g$, it suffices to prove that these two functors are left exact. Changing the role of $f$ and $g$ we may reduce to showing that $f \otimes_\Acal \id_{\Dcal}$ is left exact. Pick an algebra $B$ in $\Acal$ and an $\Acal$-linear left exact localization $q: \LMod_B(\Acal) \rightarrow \Dcal$. We have a commutative square of categories
\[
\begin{tikzcd}
\Ccal\otimes_\Acal \LMod_B(\Acal) \arrow{r}{f \otimes  \id} \arrow{d}{\id \otimes  q} & \Ccal' \otimes_\Acal \LMod_B(\Acal) \arrow{d}{\id \otimes  q} \\
\Ccal \otimes_\Acal \Dcal \arrow{r}{f \otimes  \id} & \Ccal' \otimes_\Acal \Dcal.
\end{tikzcd}
\]
Here the upper horizontal arrow is equivalent to the functor $\LMod_B(\Ccal) \rightarrow \LMod_B(\Ccal')$ induced by $f$, and is therefore left exact. To prove the proposition it will suffice to show that the left vertical arrow is a left exact localization, and that the right vertical arrow is left exact. Changing the role of $\Ccal$ and $\Ccal'$ we see that it suffices to show that the left vertical arrow is a left exact localization.

Pick an algebra $A$ in $\Acal$ and an $\Acal$-linear left exact localization $p: \LMod_A(\Acal) \rightarrow \Ccal$. We now have a commutative square of categories
\[
\begin{tikzcd}
\LMod_A(\Acal) \otimes_\acal \LMod_B(\Acal) \arrow{r}{p \otimes  \id} \arrow{d}{\id \otimes  q} & \Ccal \otimes_\Acal \LMod_B(\Acal) \arrow{d}{\id \otimes   q} \\
\LMod_A(\Acal) \otimes_\acal  \Dcal \arrow{r}{p \otimes \id} & \Ccal \otimes_\Acal \Dcal.
\end{tikzcd}
\]
The upper horizontal arrow is equivalent to the functor $\LMod_B(\LMod_A(\Acal))  \rightarrow \LMod_B(\Ccal)$ induced by $p$, and is therefore a left exact localization. Similarly, the left vertical arrow is a left exact localization. To prove the proposition it will suffice to show that the diagonal map $p \otimes  q$ is a left exact localization as well.

We have that $p \otimes q$ is an epimorphism in $\Pr^L$, with the property that a map 
\[
f: \LMod_A(\Acal) \otimes_\Acal \LMod_B(\Acal)  \rightarrow \Ecal
\]
factors through $\Ccal \otimes_\Acal \Dcal$ if and only if its restriction to $\LMod_A(\Acal) \otimes \LMod_B(\Acal)$ factors through $\Ccal \otimes \Dcal$. Similarly, the upper horizontal arrow (resp. left vertical arrow) is an epimorphism with the property that $f$ factors through it if and only if the restriction of $f$ to $\LMod_A(\Acal) \otimes \LMod_B(\Acal)$  factors through $\Ccal \otimes \LMod_{B}(\Acal)$ (resp. $\LMod_A(\Acal) \otimes \Dcal$). It follows that $f$ factors through $p \otimes  q$ if and only if it factors through both $p \otimes  \id$ and $\id \otimes  q$, so that $p \otimes  q$ is a localization at the union of the class of arrows inverted by the latter two maps. The fact that $p \otimes  q$ is left exact now follows from \cite{SAG} lemma C.4.3.1.
\end{proof}

\begin{corollary}\label{coro tensor flats}
Let $\Ccal, \Dcal$ be $\Acal$-linear Grothendieck abelian categories, and let $X, Y$ be flat objects of $\Ccal$ and $\Dcal$ respectively. Then the object $X \otimes Y$ in $\Ccal \otimes_\Acal \Dcal$ is flat.
\end{corollary}
\begin{proof}
Let $F: \Acal \rightarrow \Ccal$ (resp. $G: \Acal \rightarrow \Dcal$) be the unique $\Acal$-linear colimit preserving functor sending the unit to $X$ (resp. $Y$). Then $X \otimes Y$ is the image of the unit under the composite functor
\[
\Acal = \Acal \otimes_\Acal \Acal \xrightarrow{F \otimes G} \Ccal \otimes_\Acal \Dcal.
\]
To show that $X \otimes Y$ is flat we must show that the above functor is left exact. This is a direct consequence of proposition \ref{prop tensoring left exact}.
\end{proof}

%%%%%%%%%%%%%%%%%%%%%%%%%%%%%%%%%%%%%%%%%%%%%%%%%%%%%%%%%%%%%%%%%%%%%%%%
%%%%%%%%%%%%%%%%%%%%%%%%%%%%%%%%%%%%%%%%%%%%%%%%%%%%%%%%%%%%%%%%%%%%%%%%
%%%%%%%%%%%%%%%%%%%%%%%%%%%%%%%%%%%%%%%%%%%%%%%%%%%%%%%%%%%%%%%%%%%%%%%%
%%%%%%%%%%%%%%%%%%%%%%%%%%%%%%%%%%%%%%%%%%%%%%%%%%%%%%%%%%%%%%%%%%%%%%%%
%%%%%%%%%%%%%%%%%%%%%%%%%%%%%%%%%%%%%%%%%%%%%%%%%%%%%%%%%%%%%%%%%%%%%%%%
%%%%%%%%%%%%%%%%%%%%%%%%%%%%%%%%%%%%%%%%%%%%%%%%%%%%%%%%%%%%%%%%%%%%%%%%

\subsection{Spectral categories}\label{subsection spectral categories}

We now review the notion of spectral categories from \cite{Spectral}.

\begin{definition}
Let $\Ccal$ be a Grothendieck abelian category. We say that $\Ccal$ is spectral if every exact sequence in $\Ccal$ splits.
\end{definition}

Various finiteness conditions become equivalent for objects in a spectral category:

\begin{proposition}\label{prop equiv finiteness}
Let $\Ccal$ be a spectral category and let $X$ be an object in $\Ccal$. The following are equivalent:
\begin{enumerate}[\normalfont(a)]
\item $X$ is a finite direct sum of simple objects.
\item $X$ is compact.
\item $X$ is finitely generated.\footnote{An object $X$ in a Grothendieck abelian category is said to be finitely generated if $X$ is compact as an object in its poset of subobjects.}
\end{enumerate}
\end{proposition}
\begin{proof}
Condition (b) clearly implies (c). Assume now that $X$ is finitely generated. Since every subobject of $X$ is a direct summand of $X$, we see that every subobject of $X$ is also finitely generated. Hence $X$ is Noetherian. Assume now given a decreasing sequence of subobjects $X_n$ of $X$. Using the fact that $\Ccal$ is spectral we may inductively construct a sequence of complements $X_n^c$ for $X_n$ with the property that $X_n^c \subseteq X_{n+1}^c$ for all $n$. Since $X$ is Noetherian we have that the sequence $X_n^c$ is eventually constant, and hence $X_n$ is eventually constant as well. We conclude that $X$ is also Artinian, and so it has finite length. The fact that $X$ is spectral now implies that $X$ is a finite direct sum of simple objects. Thus we see that (c) implies (a).

It remains to show that (a) implies (b). For this it suffices to show that if $S$ is a simple object in $\Ccal$, then $S$ is compact. We will do so by showing that  $\Hom^\enh_\Ccal(S, -) : \Ccal \rightarrow \Ab$ preserves colimits. The fact that it is right exact follows from the fact that $\Ccal$ is spectral. We may therefore reduce to showing that $\Hom^\enh_\Ccal(S, -)$ preserves infinite direct sums. 

Let $Y_\alpha$ be a family of objects of $\Ccal$ indexed by a set $\Lambda$. We need to show that the map
\[
\bigoplus_{\alpha \in \Lambda} \Hom^\enh_\Ccal(S, Y_\alpha) \rightarrow \Hom^\enh_\Ccal(S, \bigoplus_{\alpha \in \Lambda} Y_\alpha)
\]
is an isomorphism. The fact that the above is a monomorphism is a general fact about Grothendieck abelian categories (and does not use the simplicity of $S$). It remains to show that every morphism $S \rightarrow \bigoplus_{\alpha \in \Lambda} Y_\alpha$ factors through $\bigoplus_{\alpha \in \Lambda'} Y_\alpha$ for some finite subset $\Lambda' \subseteq \Lambda$. This follows from the fact that $S$ is finitely generated, since $\bigoplus_{\alpha \in \Lambda}Y_\alpha$ is the filtered union of the subobjects $\bigoplus_{\alpha \in \Lambda'} Y_\alpha$ over all finite $\Lambda'$.
\end{proof}

The most familiar spectral categories are the semisimple ones, which admit a number of equivalent characterizations:

\begin{proposition}\label{prop equivalences semisimple}
Let $\Ccal$ be a Grothendieck abelian category. The following are equivalent:
\begin{enumerate}[\normalfont (a)]
\item $\Ccal$ is locally finitely generated\footnote{A Grothendieck abelian category $\Ccal$ is said to be locally finitely generated if it is generated by its finitely generated objects.} and spectral.
\item $\Ccal$ is generated by compact projective objects and spectral.
\item Every object of $\Ccal$ is a direct sum of simple objects.
\end{enumerate}
\end{proposition}
\begin{proof}
The fact that (a) and (b) are equivalent, and the fact that (c) implies these, are both consequences of proposition \ref{prop equiv finiteness}. It remains to show that if (a) and (b) hold then every object of $\Ccal$ is a direct sum of simple objects. Let $X$ be an object of $\Ccal$. We construct a strictly increasing transfinite sequence of subobjects $X_\alpha$ of $X$ by induction as follows:
\begin{itemize}
\item Let $X_0 = 0$.
\item If $\alpha$ is a limit ordinal then we let $X_\alpha = \colim_{\beta < \alpha} X_\beta$.
\item Assume $\alpha = \beta + 1$ is a successor ordinal and $X_\alpha \neq X$. Choose a complement $Y$ for $X_\beta$ inside $X$. Since $\Ccal$ is assumed to be locally finitely generated we may pick a nonzero finitely generated subobject $Y'$ of $Y$. Since $\Ccal$ is spectral, an application of proposition \ref{prop equiv finiteness} shows that $Y'$ contains a simple subobject $S$. We let $X_\alpha = X_\beta \oplus S$.
\end{itemize}
The above construction ends whenever it reaches a small cardinal $\alpha$ with $X_\alpha = X$. The proposition now follows from the fact that $X_\alpha$ is a direct sum of simple objects for all $\alpha$.
\end{proof}

\begin{definition}
Let $\Ccal$ be a Grothendieck abelian category. We say that $\ccal$ is semisimple if it satisfies the equivalent conditions of proposition \ref{prop equivalences semisimple}.
\end{definition}

We will be interested in understanding spectral categories linear over a base. We fix for the remainder of this section a symmetric monoidal Grothendieck abelian category $\acal$, rigid and generated by compact projective objects.  We will need the notion of von Neumann regularity for algebras in $\acal$. Before introducing this notion we recall some basic concepts from ring theory in the relative context:

\begin{definition}
Let $A$ be an algebra in $\Acal$. 
\begin{itemize}
\item A left ideal of $A$ is a subobject of $A$ in $\LMod_A(\Acal)$.
\item A left $A$-module $M$ is said to be flat if the functor $- \otimes_A M : \RMod_A(\acal) \rightarrow \acal$ is left exact.
\item We say that $R$ is left self-injective if $R$ is an injective object of $\LMod_A(\acal)$.
\end{itemize}
We also define the right variants of the above notions in a similar way.
\end{definition}

\begin{lemma}\label{lemma equivalences dualizable}
Let $A$ be an algebra in $\Acal$ and let $M$ be a left $A$-module. The following are equivalent:
\begin{enumerate}[\normalfont(a)]
\item The left module $M$ admits a left dual.
\item The functor $- \otimes_A M : \RMod_A(\acal) \rightarrow \acal$   preserves limits.
\item $M$ is finitely generated projective.
\item $M$ is a retract of a free left $A$-module on a dualizable object of $\Acal$.
\item $M$ is finitely presented and flat.
\end{enumerate}
\end{lemma}
\begin{proof}
The existence of a left dual to $M$ is equivalent to the existence of an $\Acal$-linear left adjoint to $-\otimes_A M$. Since $\Acal$ is generated under colimits by its dualizable objects any left adjoint to $- \otimes_A M$ is automatically $\Acal$-linear. The equivalence of (a) and (b) now follows directly from the adjoint functor theorem.

The fact that (c) implies (d) is a direct consequence of the fact that $\LMod_A(\Acal)$ is generated by free modules on dualizable objects. The fact that (d) implies (c) follows from the fact that the property of being compact projective is stable under retracts and passage to free modules. 

We now show that (a) implies (d). If $M$ admits a left dual $M^\vee$ then  we have 
\[
\Hom_{\RMod_A(\Acal)}(M^\vee, -) = \Hom_\Acal(1_\acal, - \otimes_A M)
\] which preserves sifted colimits since the unit in $\Acal$ is compact projective. It follows that in this case $M^\vee$ is a compact projective right $A$-module. Since $\RMod_A(\Acal)$ is generated by objects of the form $A \otimes V$ with $V$ a dualizable object of $\Acal$ we conclude that $M^\vee$ is a retract of $V \otimes A$ for some dualizable $V$. The right $A$-module $V \otimes A$ admits a right dual given by $V^\vee \otimes A$. It follows that $M$ is a retract of $V^\vee \otimes A$, so that (d) holds.

Assume now that (d) holds, so that $M$ is retract of $A \otimes V$ for some dualizable $V$. Then $- \otimes_A M$ is a retract of the composite functor
\[
\RMod_A(\Acal) \xrightarrow{ - \otimes_A A}  \Acal \xrightarrow{ \otimes V} \Acal.
\]
Each of the two functors above preserve limits. It follows that $- \otimes_A M$ preserves limits, so that (b) holds.

It remains to show that properties (a) through (d) are equivalent to (e). Assume first that (a) through (d) hold. The flatness of $M$ is then a consequence of (b), while the fact that $M$ is finitely presented follows from (d).
 
We finish the proof by showing that (e) implies (b). Since $M$ is flat, it is enough to show that $- \otimes_A M$ preserves products. Pick an exact sequence $P' \rightarrow P \rightarrow M \rightarrow 0$ with $P, P'$ finitely generated projective left $A$-modules. Let $N_\alpha$ be a family of right $A$-modules. Then we have a commutative diagram
\[
\begin{tikzcd}
(\prod_\alpha N_\alpha) \otimes_A P' \arrow{d}{} \arrow{r}{} & (\prod_\alpha N_\alpha) \otimes_A P \arrow{r}{} \arrow{d}{} & (\prod_\alpha N_\alpha) \otimes_A M \arrow{r}{} \arrow{d}{} & 0  \\
\prod_\alpha N_\alpha \otimes_A P' \arrow{r}{} &\prod_\alpha N_\alpha \otimes_A P \arrow{r}{} & \prod_\alpha N_\alpha \otimes_A M \arrow{r}{}& 0.
\end{tikzcd}
\]
The upper row is evidently exact, and the bottom row is exact since $\Acal$ is generated by compact projective objects (and thus products are exact in $\acal$). We now finish by observing that the left and middle vertical arrows are isomorphisms, by applying the equivalence of (b) and (c) for the modules $P$ and $P'$.
\end{proof}

\begin{proposition}\label{prop equivalences VN}
Let $A$ be an algebra in $\Acal$. The following are equivalent:
\begin{enumerate}[\normalfont(a)]
\item Every finitely generated submodule of a finitely generated projective left $A$-module is a direct summand.
\item Every finitely generated submodule of a finitely generated projective right $A$-module is a direct summand.
\item Every finitely presented left $A$-module is projective.
\item Every finitely presented right $A$-module is projective.
\item Every left $A$-module is flat.
\item Every right $A$-module is flat.
\end{enumerate}
\end{proposition}
\begin{proof}
Let $N \subseteq M$ be a finitely generated submodule of a finitely generated projective left $A$-module. Then $M/N$ is a finitely presented left $R$-module. Conversely, every finitely presented left $A$-module may be written in such a way. It follows from this that (a) and (c) are equivalent.

Since every left $A$-module is a filtered colimit of finitely presented left $A$-modules, and filtered colimits preserve flatness, we see that (c) implies (e). The fact that (e) implies (c) is a direct consequence of lemma \ref{lemma equivalences dualizable}.

The same arguments applied to $A^\op$ prove the equivalence of (b), (d), and (f). To finish it suffices to show that the left versions imply the right versions. Assume that (a) holds. Let $i: M' \rightarrow M$ be an inclusion of left $A$-modules and let $N$ be a right $A$-module. We will show that $N \otimes_A i$ is a monomorphism.

Write $M$ as a filtered colimit of a family of finitely presented left $A$-modules $M_\alpha$. Then $i$ is a filtered colimit of the induced inclusions $i_\alpha: M' \times_{M} M_\alpha \rightarrow M_\alpha$. It suffices to show that $N \otimes_A i_\alpha$ is a monomorphism for all $\alpha$. In other words, we may reduce to the case when $M$ is finitely presented.

Write $M'$ as a filtered union of finitely generated subobjects $M'_\beta$. Then $i$ is the filtered colimit of the inclusions $i_\beta: M'_\beta \rightarrow M$, and it suffices to show that $N \otimes_A i_\beta$ is a monomorphism for all $\beta$. In other words, we may further reduce to the case when $M'$ is finitely generated. This now follows from the fact that $i$ is the inclusion of a summand.
\end{proof}

\begin{definition}
Let $A$ be an algebra in $\Acal$. We say that $A$ is von Neumann regular if it satisfies the equivalent conditions of proposition \ref{prop equivalences VN}.
\end{definition}

The following is an $\acal$-linear version of \cite{Spectral} theorem 2.1:

\begin{proposition}\label{prop classif spectral}
Let $\Ccal$ be an $\Acal$-linear spectral category. Then there exists a left self-injective von Neumann regular algebra $A$ in $\Acal$ and an $\Acal$-linear left exact localization 
\[
\LMod_A(\Acal) \rightarrow \Ccal.
\]
\end{proposition}
\begin{proof}
Let $G$ be an $\acal$-generator for $\Ccal$ and let $A$ be the opposite to the algebra of endomorphisms of $G$. We will show that $A$ is left self-injective von Neumann regular.

Denote by $q: \LMod_A(\Acal) \rightarrow \Ccal$ the functor of tensoring with $G$ and by $i$ its right adjoint. Since $q$ is left exact and every object of $\Ccal$ is injective, we see that every left $A$-module in the image of $i$ is injective. In particular this holds for $i(G) = A$, and so $A$ is left self-injective.

It remains to show that $A$ is von Neumann regular. We will do so by showing that condition (a) in proposition \ref{prop equivalences VN} holds. Let $M$ be a finitely generated projective left $A$-module and let $N$ be a finitely generated submodule of $M$. We may write $N$ as the image of a map $\alpha: M' \rightarrow M$ of finitely generated projective left $A$-modules. Each of $M$ and $M'$ is a direct summand of a left $A$-module of the form $A \otimes X$ with $X$ a dualizable object of $\Acal$. Since $A \otimes X = i(G \otimes X)$ and $\Ccal$ is idempotent complete, we see that $M$ and $M'$, and therefore also $\alpha$, belong to the image of $i$. Since $\Ccal$ is spectral, every morphism in $\Ccal$ may be written as the composition of a retraction followed by a section. Hence $\alpha$ is a composition of a retraction followed by a section. This is necessarily equivalent to the image factorization for $\alpha$, so we deduce that the inclusion $N \rightarrow M$ is a section, as desired.
\end{proof}

%%%%%%%%%%%%%%%%%%%%%%%%%%%%%%%%%%%%%%%%%%%%%%%%%%%%%%%%%%%%%%%%%%%%%%%%
%%%%%%%%%%%%%%%%%%%%%%%%%%%%%%%%%%%%%%%%%%%%%%%%%%%%%%%%%%%%%%%%%%%%%%%%
%%%%%%%%%%%%%%%%%%%%%%%%%%%%%%%%%%%%%%%%%%%%%%%%%%%%%%%%%%%%%%%%%%%%%%%%
%%%%%%%%%%%%%%%%%%%%%%%%%%%%%%%%%%%%%%%%%%%%%%%%%%%%%%%%%%%%%%%%%%%%%%%%
%%%%%%%%%%%%%%%%%%%%%%%%%%%%%%%%%%%%%%%%%%%%%%%%%%%%%%%%%%%%%%%%%%%%%%%%
%%%%%%%%%%%%%%%%%%%%%%%%%%%%%%%%%%%%%%%%%%%%%%%%%%%%%%%%%%%%%%%%%%%%%%%%

\subsection{Grothendieck prestable categories} \label{subsection prestables}

We now review the theory of Grothendieck prestable categories, as introduced in \cite{SAG} appendix C.

\begin{definition}\label{def prestable}
A Grothendieck prestable category is a presentable category $\Ccal$ satisfying the following properties:
\begin{enumerate}[\normalfont (a)]
\item The initial and final objects of $\Ccal$ agree (that is, $\Ccal$ is pointed).
\item Every cofiber sequence in $\Ccal$ is also a fiber sequence.
\item Every map in $\Ccal$ of the form $f: X \rightarrow \Sigma(Y)$ is the cofiber of its fiber.
\item Filtered colimits and finite limits commute in $\Ccal$.
\end{enumerate}

We denote by $\Groth_\infty$ the full subcategory of $\Pr^L$ on the Grothendieck prestable categories.
\end{definition}

\begin{remark}\label{remark groth prestable vs t structure}
Let $\Ccal$ be a Grothendieck prestable category. Then the functor $\Ccal \rightarrow \Ccal \otimes \Sp = \Sp(\Ccal)$ is fully faithful, and identifies $\Ccal$ with the connective half of t-structure on $\Sp(\Ccal)$. In particular, $\Ccal$ is additive, and moreover it makes sense to consider for each nonnegative integer $n$ the homology functor $H_n = \Omega^n: \Ccal \rightarrow \Ccal^\heartsuit = \Ccal_{\leq 0}$. Note that $\Ccal^\heartsuit$ is a Grothendieck abelian category.
\end{remark}

It turns out that the assignment $\Ccal \mapsto \Sp(\Ccal)$ provides a one to one correspondence between Grothendieck prestable categories and presentable stable categories equipped with a with right complete t-structure compatible with filtered colimits. One virtue of working with Grothendieck prestable categories instead of t-structures is that being Grothendieck prestable is a property of a category (as opposed to a t-structure on a presentable stable category which is a piece of structure).

\begin{example}
Let $\Ccal$ be a Grothendieck abelian category. Then the derived category $\der(\Ccal)$ is a presentable stable category with a right complete t-structure compatible with filtered colimits. By virtue of remark \ref{remark groth prestable vs t structure} the connective half of this t-structure is Grothendieck prestable. We denote this category by $\der(\Ccal)_{\geq 0}$.
\end{example}

The following result provides an ample source of Grothendieck prestable categories:

\begin{proposition}[\cite{SAG} proposition 10.4.3.1]
Let $\Ccal, \Dcal$ be presentable categories and assume given a functor $G: \Ccal \rightarrow \Dcal$ which is conservative and preserves small limits\footnote{Only preservation of finite limits is necessary.} and colimits. If $\Ccal$ is a Grothendieck prestable category then so is $\Dcal$.
\end{proposition}

\begin{corollary}
Let $\Acal$ be a Grothendieck prestable category equipped with a monoidal structure compatible with colimits. Let $A$ be an algebra in $\Acal$. Then the category $\LMod_A(\Acal)$ of left $A$-modules in $\Acal$ is a Grothendieck prestable category.
\end{corollary}

The notion of projective object in a Grothendieck abelian category admits a version in the setting of Grothendieck prestable categories:

\begin{definition}
Let $\Ccal$ be a Grothendieck prestable category. We say that an object $P$ in $\Ccal$ is projective if every map $X \rightarrow P$ in $\Ccal$ which is an epimorphism on $H_0$ admits a section.
\end{definition}

\begin{remark}
Let $\Ccal$ be a Grothendieck prestable category. Then an object $P$ in $\Ccal$ is projective if and only if $\Hom_\Ccal(P, - ): \Ccal \rightarrow \Spc$ preserves geometric realizations. In other words, if and only if $P$ is projective in the sense of \cite{HTT} section 5.5.8.
\end{remark}

We may think about Grothendieck prestable categories generated under colimits by compact projective objects as many object versions of connective ring spectra. In that setting there is a close relation between projective modules over a connective ring spectrum $R$ and projective modules over $\pi_0(R)$ (see \cite{HA} corollary 7.2.2.19). The following proposition is an extension of that relation:

\begin{proposition}\label{proposition projectives vs heart}
Let $\Ccal$ be a Grothendieck prestable category generated under colimits by compact projective objects. Then
\begin{enumerate}[\normalfont (1)]
\item The truncation functor $H_0 : \ccal \rightarrow \ccal^\heartsuit$ sends projective objects to projective objects and compact objects to compact objects.
\item The $0$-truncations of the compact projective objects of $\ccal$ provide a family of compact projective generators for $\ccal^\heartsuit$.
\item The functor $\Ho(H_0): \Ho(\ccal) \rightarrow \Ho(\ccal^\heartsuit) = \ccal^\heartsuit$ induced at the level of homotopy categories restricts to an equivalence between the full subcategories on the projective objects, which in turn restricts to an equivalence on the full subcategories on the compact projective objects.
\end{enumerate}
\end{proposition}
\begin{proof}
We first prove (1). The fact that $H_0$ sends compact objects to compact objects follows directly from the fact that the inclusion $\ccal^\heartsuit \rightarrow \ccal$ preserves filtered colimits. The fact that $H_0$ sends projective objects to projective objects follows from the fact that the inclusion $\ccal^\heartsuit \rightarrow \ccal$ maps epimorphisms to morphisms which induce epimorphisms on $H_0$. 

Item (2) follows directly from (1) together with the fact that $H_0$ is a localization. It  remains to establish (3). We first prove fully faithfulness. Let $X, Y$ be a pair of projective objects of $\ccal$. Then the map $\Hom_{\ccal}(X, Y) \rightarrow \Hom_{\ccal^\heartsuit}(H_0(X), H_0(Y))$ induced by $H_0$ is equivalent to the map $\eta_*: \Hom_{\ccal}(X, Y) \rightarrow \Hom_{\ccal}(X, H_0(Y))$ of composition with the unit $\eta: Y \rightarrow H_0(Y)$. The fact that $X$ is projective and $\eta$ induces an equivalence on $H_0$ implies that $\eta_*$ is an effective epimorphism. Its fiber is given by $\Hom_{\ccal}(X, \tau_{\geq 1}(Y))$ which is connected since $X$ is projective. We conclude that $\eta_*$ induces an equivalence on $\pi_0$, and therefore $\Ho(H_0)$ is fully faithful on the full subcategory on the projective objects.

It remains to prove surjectivity. In other words, we have to show that every (compact) projective object of $\ccal^\heartsuit$ is the image under $H_0$ of a (compact) projective object of $\ccal$. We establish the case of compact projective objects, the proof in the projective case being similar. Let $Y$ be a compact projective object of $\ccal^\heartsuit$.  Applying (2) we may find a compact projective object $X$ in $\ccal$ such that $Y$ is a retract of $H_0(X)$. Let $r: H_0(X) \rightarrow H_0(X)$ be the induced retraction. The fully faithfulness part of (3) allows us to lift $r$ to an idempotent endomorphism $\rho$ of the image of $X$ inside $\Ho(\ccal)$. Let $X'$ be a representative in $\ccal$ of the image of $\rho$. Then $X'$ is a direct summand of $X$ and therefore it is compact projective. The proof finishes by observing that $H_0(X') = \operatorname{Im}(r) = Y$. 
\end{proof}

There is a good theory of tensor products of Grothendieck prestable categories:

\begin{theorem}[\cite{SAG} theorem C.4.2.1]\label{teo tensor product prestable}
Let $\Ccal, \Dcal$ be Grothendieck prestable categories. Then their tensor product $\Ccal \otimes \Dcal$ (formed in $\Pr^L$) is Grothendieck prestable. In particular, the symmetric monoidal structure on the category $\Mod_{\Sp^\cn}(\Pr^L)$ of presentable additive categories and colimit preserving functors restricts to a symmetric monoidal structure on $\Groth_\infty$.
\end{theorem}

In particular, it makes sense to consider commutative algebras in $\Groth_\infty$. We call these symmetric monoidal Grothendieck prestable categories. Note that this terminology leaves implicit the fact that the tensor operation commutes with colimits in each variable.

\begin{definition}
Let $\Mcal$ be a commutative algebra in $\Groth_\infty$. An $\Mcal$-linear Grothendieck prestable category is an object of $\Mod_\Mcal(\Groth_\infty)$.
\end{definition}

In other words, an $\Mcal$-linear Grothendieck prestable category is an $\Mcal$-linear presentable category (in the sense of section \ref{subsection linear cats}) which is in addition a Grothendieck prestable category. In the case when $\Mcal = \Mod^\cn_R$ is the category of connective modules over a connective $E_\infty$-ring $R$ we call these $R$-linear Grothendieck prestable categories.

Since the class of colimits that we have available in $\Groth_\infty$ is relatively restricted, one has to be careful when forming relative tensor products. There is however a class of commutative algebras that admit a well behaved theory of relative tensor products.

\begin{definition}
Let $\Mcal$ be a commutative algebra in $\Groth_\infty$. Assume that $\Mcal$ is generated under colimits by compact projective objects. We say that $\Mcal$ is rigid if compact projective and dualizable objects of $\Mcal$ coincide.
\end{definition}

\begin{example}
Let $R$ be a connective commutative ring spectrum. Then $\Mod_R^\cn$ is rigid.
\end{example}

We fix for the remainder of this section a symmetric monoidal Grothendieck prestable category $\Mcal$ generated under colimits by compact projective objects and rigid. We begin with the observation that $\Mcal$-linear Grothendieck prestable categories are closed under tensor products:

\begin{proposition}
The full subcategory of $\Mod_\Mcal(\Pr^L)$ on the $\Mcal$-linear Grothendieck prestable categories is closed under tensor products. In other words, $\Mod_\Mcal(\Groth_\infty)$ admits a symmetric monoidal structure that makes the inclusion $\Mod_\Mcal(\Groth_\infty) \rightarrow \Mod_\Mcal(\Pr^L)$ symmetric monoidal.
\end{proposition}
\begin{proof}
Completely analogous to the proof of corollary \ref{coro tensor products over A}. One first shows that for any $\Mcal$-module $\Ccal$ in $\Pr^L$ the action map  $\Mcal \otimes \Ccal \rightarrow \Ccal$ admits a colimit preserving right adjoint, imitating the proof of proposition \ref{proposition action has colimit preserving adjoint}. The role of proposition \ref{proposition colimits of right adjointable} is then played by \cite{SAG} remark C.3.5.4.
\end{proof}

We now formulate a Grothendieck prestable version of proposition \ref{prop lex localization}.

\begin{definition}\label{definition generator in groth prestable}
Let $\ccal$ be an $\Mcal$-linear Grothendieck prestable category. We say that an object $G$ in $\ccal$ is an $\Mcal$-generator if for every object $Y$ in $\ccal$ there exists an object $X$ in $\Mcal$ and a morphism $X \otimes G \rightarrow Y$ inducing an epimorphism on $H_0$.
\end{definition}

In the context of Grothendieck prestable categories some attention needs to be paid to the distinction between generators and colimit generators. Unlike the situation with Grothendieck abelian categories, it is possible for an object $G$ in $\ccal$ to be an $\Mcal$-generator in the sense of definition \ref{definition generator in groth prestable} and the family of objects $X \otimes G$ not generate $\ccal$ under colimits (for instance, $0$ is always an $\Mcal$-generator whenever $\ccal$ is stable). As shown in \cite{SAG} theorem 2.1.6, the distinction disappears when $\ccal$ is assumed to be separated:

\begin{definition}
Let $\Ccal$ be a Grothendieck prestable category. We say that $\Ccal$ is separated if it contains no nonzero $\infty$-connective objects.\footnote{An object of $\Ccal$ is $\infty$-connective if all its homology objects vanish.}
\end{definition}

\begin{proposition}\label{prop lex localization prestable}
Let $\Ccal$ be a separated $\Mcal$-linear Grothendieck prestable category and let $G$ be an $\Mcal$-generator for $\Ccal$. Let $A$ be the opposite of the algebra of endomorphisms of $G$ associated to the action of $\Mcal$ on $\Ccal$. Then the functor
\[
G \otimes_A - : \LMod_A(\Mcal) \rightarrow \Ccal 
\]
is an $\Mcal$-linear left exact localization. Furthermore, it is an equivalence if and only if $G$ is compact projective.
\end{proposition}
\begin{proof}
Completely analogous to the proof of proposition \ref{prop lex localization}, where the role of the classical many object Gabriel-Popescu theorem  is played by \cite{SAG} theorem C.2.1.6.
\end{proof}

We now turn to a discussion of flatness in the context of $\Mcal$-linear Grothendieck prestable categories.

\begin{definition}
Let $\Ccal$ be an $\Mcal$-linear Grothendieck prestable category. We say that an object $X$ in $\Ccal$ is flat over $\mcal$ if the functor $- \otimes X : \Mcal \rightarrow \Ccal$ is left exact. In cases when the base $\mcal$ is clear from the context we simply say that $X$ is flat.
\end{definition}

\begin{example}
Let $R$ be a connective $E_\infty$-ring and let $\Ccal$ be an $R$-linear Grothendieck prestable category. It follows from \cite{SAG} proposition C.3.2.1 that an object $X$ in $\Ccal$ is flat if and only if $M \otimes X$ is $0$-truncated for all $0$-truncated $R$-modules $M$. Since every $0$-truncated $R$-module is a filtered colimit of cyclic $\pi_0(R)$-modules, we see that $X$ is flat if and only if $M \otimes X$ is $0$-truncated for every cyclic $\pi_0(R)$-module $M$.
\end{example}

\begin{proposition}\label{prop lazard prestable}
Let $X$ be an object of $\Mcal$. Then $X$ is flat if and only if it is a filtered colimit of compact projective objects.
\end{proposition}
\begin{proof}
Analogous to the proof of proposition \ref{prop lazard classico}.
\end{proof}

\begin{proposition}\label{prop tensoring left exact prestable}
Let $f: \Ccal \rightarrow \Ccal'$ and $g: \Dcal \rightarrow \Dcal'$ be  morphisms in $\Mod_{\Mcal}(\Groth_\infty)$. If $f$ and $g$ are left exact then $f \otimes_{\Mcal} g : \Ccal \otimes_\Mcal \Dcal \rightarrow \Ccal' \otimes_\Mcal \Dcal'$ is left exact.
\end{proposition}
\begin{proof}
Analogous to the proof of proposition \ref{prop tensoring left exact}.
\end{proof}
\begin{corollary}\label{coro tensor flats prestable}
Let $\Ccal, \Dcal$ be $\Mcal$-linear Grothendieck prestable categories, and let $X, Y$ be flat objects of $\Ccal$ and $\Dcal$ respectively. Then the object $X \otimes Y$ in $\Ccal \otimes_\Mcal \Dcal$ is flat.
\end{corollary}
\begin{proof}
Analogous to the proof of corollary \ref{coro tensor flats}.
\end{proof}

In the presence of flatness, projectivity of objects may be checked after passing to $H_0$:

\begin{proposition}\label{prop check compact projective on heart}
Let $\ccal$ be an $\Mcal$-linear Grothendieck prestable category. Assume that $\ccal$ is generated under colimits by compact projective objects and that compact projective objects in $\ccal$ are flat. Then an object $X$ in $\ccal$ is projective if and only if it is flat and $H_0(X)$ is projective in $\ccal^\heartsuit$.
\end{proposition}
\begin{proof}
Since every projective object is a retract of a direct sum of compact projective objects, and flat objects are closed under retracts and direct sums, we see that every projective object of $\ccal$ is flat. The fact that $H_0$ sends projective objects to projective objects was already observed in proposition \ref{proposition projectives vs heart}. This finishes the proof of the only if direction.

 Assume now that $X$ is flat and $H_0(X)$ is projective. Applying proposition \ref{proposition projectives vs heart} we may find a projective object $X'$ in $\ccal$ and an isomorphism $H_0(X') = H_0(X)$. The fact that $X'$ is projective allows us to lift this isomorphism to a morphism $f: X' \rightarrow X$.   We claim that $f$ is an isomorphism. To do so it suffices to prove that $f \otimes 1_{\mcal}$ is an isomorphism. Since both $X$ and $X'$ are flat and $\ccal$ is separated we may reduce to proving that $H_0(f \otimes H_n(1_{\mcal}))$ is an isomorphism for all $n \geq 0$. This agrees with $H_0(H_0(f) \otimes H_n(1_\mcal))$, which is an isomorphism by virtue of the fact that $H_0(f)$ is an isomorphism.
\end{proof}

We now discuss the operation of passage to derived categories for linear Grothendieck abelian categories. Fix for the remainder of this section a symmetric monoidal Grothendieck abelian category $\acal$, rigid and generated by compact projective objects.

\begin{construction}\label{construction smon structure on der 1}
Let $\Acal^\cp$ be the full subcategory of $\Acal$ on the compact projective objects and equip $\acal^\cp$ with the symmetric monoidal structure restricted from $\acal$. Note that   $\der(\Acal)_{\geq 0}$ is  obtained by freely adjoining colimits to $\Acal^\cp$. We equip $\der(\Acal)_{\geq 0}$  with the unique colimit preserving extension of the existing symmetric monoidal structure on $\acal^{cp}$. Since $\der(\acal)$ is the stabilization of $\der(\acal)$, we may further extend our symmetric monoidal structure uniquely to a symmetric monoidal structure compatible with colimits on $\der(\acal)$.
\end{construction}

\begin{remark}
Construction \ref{construction smon structure on der 1} makes $\der(\Acal)_{\geq 0}$ into a rigid commutative algebra in $\Groth_\infty$. It is in fact the unique way to equip  $\der(\Acal)_{\geq 0}$ with a rigid commutative algebra structure making the truncation functor $\der(\Acal)_{\geq 0} \rightarrow \acal$ symmetric monoidal.
\end{remark}

A variant of construction \ref{construction smon structure on der 1} allows us to give the connective derived category of an $\acal$-linear Grothendieck abelian category the structure of a  $\der(\acal)_{\geq 0}$-linear Grothendieck prestable category:

\begin{construction}\label{construction smon structure on der 2}
Let $\ccal$ be an $\acal$-linear Grothendieck abelian category. We view the $\acal$-linear structure as a monoidal finite coproduct preserving functor $f: \acal^\cp \rightarrow \Funct^L_{\lex}(\ccal, \Ccal)$ where the target is the category of left exact colimit preserving endofunctors of $\ccal$. Passing to derived functors provides a monoidal equivalence 
\[
\der(-): \Funct^L_{\lex}(\ccal, \Ccal) \rightarrow \Funct^L_{\lex}(\der(\Ccal)_{\geq 0}, \der(\Ccal)_{\geq 0}).
\]
 We equip $\der(\ccal)_{\geq 0}$ with the $\der(\acal)_{\geq 0}$-linear structure arising from the  monoidal finite coproduct preserving functor $\der(-) \circ f : \acal^\cp \rightarrow \Funct^L_{\lex}(\der(\Ccal)_{\geq 0}, \der(\Ccal)_{\geq 0})$. We equip $\der(\ccal) = \Sp(\der(\ccal)_{\geq 0})$ with the induced $\der(\acal) = \Sp(\der(\acal)_{\geq 0})$-linear structure.
\end{construction}

\begin{remark}
Let $\Ccal$ be an $\acal$-linear Grothendieck abelian category. Then the $\der(\acal)_{\geq 0}$-linear structure on $\der(\ccal)_{\geq 0}$ from construction \ref{construction smon structure on der 2} is the unique such structure making the truncation functor $\der(\ccal)_{\geq 0} \rightarrow \ccal$ into a $\der(\ccal)_{\geq 0}$-linear functor.
\end{remark}

\begin{notation}
The inclusion $\acal \rightarrow \der(\acal)$ is generally not symmetric monoidal. When we wish to emphasize the distinction between both symmetric monoidal structures we will write $\otimes^L$ for the tensor product in $\der(\acal) $, and $\otimes$ for the tensor product in $\acal$. If it is clear from the context which operation is being used, we will simply write $\otimes$ instead of $\otimes^L$. The same considerations apply to the case of the action of $\der(\acal)$ on the   derived category of an $\acal$-linear Grothendieck abelian category.
\end{notation}

The following proposition relates the notion of flatness  of objects in $\acal$-linear Grothendieck abelian categories from section \ref{subsection abelian} with the notion studied in this section.

\begin{proposition}\label{prop flat in C vs DC}
Let $\ccal$ be a separated $\der(\acal)_{\geq 0}$-linear Grothendieck prestable category and let $X$ be an object of $\ccal$. The following are equivalent:
\begin{enumerate}[\normalfont(1)]
\item  $X$ is flat over $\der(\acal)_{\geq 0}$.
\item $X$ is $0$-truncated and $H_1(Y \otimes X) = 0$ for all $Y$ in $\acal$.
\item  $X$ is $0$-truncated and flat over $\acal$ (as an object of $\ccal^\heartsuit$).
\end{enumerate}
\end{proposition}
\begin{proof}
We first show that (1) implies (2). Since $X$ is flat and the unit $1_\acal$ is $0$-truncated, we have that $X = 1_\acal \otimes X$ is $0$-truncated. Furthermore, appealing once again to the flatness of $X$ we have that for every $Y$ in $\acal$ the object $Y \otimes X$ is $0$-truncated, and hence $H_1(Y \otimes X) = 0$, as desired. To see that (2) implies (3) we must show that if  $i: Z \rightarrow Z'$ is a monomorphism in $\Acal$ then $H_0(i \otimes X)$ is a monomorphism in $\ccal$. Indeed, the kernel of $H_0(i \otimes X)$ receives an epimorphism from $H_1(\coker(i)\otimes X)$, which vanishes.

It remains to prove that (3) implies (1). Let $Y$ be a $0$-truncated object of $\der(\acal)_{\geq 0}$. Pick a resolution $Y_\bullet$ of $Y$ by compact projective objects. Then $Y \otimes X$ is the realization of $Y_\bullet \otimes X$. Since $Y_\bullet$ is levelwise compact projective we have that $Y_\bullet \otimes X$ is a diagram of $0$-truncated objects. The fact that $X$ is a flat object of $\ccal^\heartsuit$ now implies that $Y_\bullet \otimes X$ is a resolution of $H_0(Y \otimes X)$. Since $\ccal$ is separated we have that $Y \otimes X = H_0(Y \otimes X)$ is $0$-truncated. Since $Y$ was arbitrary we conclude that $X$ is flat, as desired.
\end{proof}

\begin{remark}
Let $\ccal$ be an $\acal$-linear Grothendieck abelian category. For each $n \geq 0$ we denote by $\Tor_n(-,-): \acal \times \ccal \rightarrow \ccal$ the composite functor
\[
\acal \times \ccal = \der(\acal)^\heartsuit \times \der(\ccal)^\heartsuit \hookrightarrow \der(\acal)_{\geq 0} \times \der(\ccal)_{\geq 0} \xrightarrow{-\otimes-} \der(\ccal)_{\geq 0} \xrightarrow{H_n(-)} \ccal.
\]
Specializing proposition \ref{prop flat in C vs DC} we see that an object $X$ in $\der(\ccal)_{\geq 0}$ is flat over $\der(\acal)_{\geq 0}$ if and only if it is $0$-truncated and $\Tor_1(Y, X) = 0$ for all $Y$ in $\acal$.
\end{remark}

%%%%%%%%%%%%%%%%%%%%%%%%%%%%%%%%%%%%%%%%%%%%%%%%%%%%%%%%%%%%%%%%%%%%%%%%
%%%%%%%%%%%%%%%%%%%%%%%%%%%%%%%%%%%%%%%%%%%%%%%%%%%%%%%%%%%%%%%%%%%%%%%%
%%%%%%%%%%%%%%%%%%%%%%%%%%%%%%%%%%%%%%%%%%%%%%%%%%%%%%%%%%%%%%%%%%%%%%%%
%%%%%%%%%%%%%%%%%%%%%%%%%%%%%%%%%%%%%%%%%%%%%%%%%%%%%%%%%%%%%%%%%%%%%%%%
%%%%%%%%%%%%%%%%%%%%%%%%%%%%%%%%%%%%%%%%%%%%%%%%%%%%%%%%%%%%%%%%%%%%%%%%
%%%%%%%%%%%%%%%%%%%%%%%%%%%%%%%%%%%%%%%%%%%%%%%%%%%%%%%%%%%%%%%%%%%%%%%%

\ifx\inmain\undefined
\bibliographystyle{myamsalpha2}
\bibliography{References}
\fi
\end{document}
