\documentclass[12pt]{amsart}

\input{"Preamble"}

%\usepackage{standalone} 
%\def\inmain{1}

\title{Compact assembly, duality, and filtered colimits of categories}

\author{G. Stefanich}

\externaldocument{"Introduction"}
\externaldocument{"Preliminaries"}
\externaldocument{"GRings"}

\date{}

\begin{document}

%%%%%%%%%%%%%%%%%%%%%%%%%%%%%%%%%%%%%%%%%%%%%%%%%%%%%%%%%%%%%%%%%%%%%%%%
%%%%%%%%%%%%%%%%%%%%%%%%%%%%%%%%%%%%%%%%%%%%%%%%%%%%%%%%%%%%%%%%%%%%%%%%
%%%%%%%%%%%%%%%%%%%%%%%%%%%%%%%%%%%%%%%%%%%%%%%%%%%%%%%%%%%%%%%%%%%%%%%%
%%%%%%%%%%%%%%%%%%%%%%%%%%%%%%%%%%%%%%%%%%%%%%%%%%%%%%%%%%%%%%%%%%%%%%%%
%%%%%%%%%%%%%%%%%%%%%%%%%%%%%%%%%%%%%%%%%%%%%%%%%%%%%%%%%%%%%%%%%%%%%%%%
%%%%%%%%%%%%%%%%%%%%%%%%%%%%%%%%%%%%%%%%%%%%%%%%%%%%%%%%%%%%%%%%%%%%%%%%

\section{Compact assembly, duality, and filtered colimits of categories}\label{section compactly assembled}

In \cite{JJcont}, Johnstone and Joyal introduced the notion of a continuous $(1,1)$-category as a categorification of the more classical notion of continuous poset, with the purposes of characterizing exponentiable Grothendieck $(1,1)$-topoi. In \cite{SAG}, an $(\infty,1)$-categorical generalization of this notion was developed, under the name of compactly assembled categories. In the appendix D, Lurie proves a fundamental result that shows that a presentable stable category is dualizable if and only if it is compactly assembled. 

We begin this section in \ref{subsection compactly assembled} by reviewing Lurie's results, and formulating a variant that we call $n$-strong compact assembly, which plays a similar role in classifying dualizable presentable additive $(n,1)$-categories for each $1 \leq n \leq \infty$. This is of fundamental importance in our paper: together with   proposition \ref{proposition dualizable is presentable}, it implies that dualizable cocomplete  additive  $(1,1)$-categories (resp. $(\infty,1)$-categories) are automatically Grothendieck abelian (resp. separated Grothendieck prestable) and have exact products.

In the same way that compact assembly is a weakening of the notion of generation by compact objects, $n$-strong compact assembly is a weakening of the notion of generation by $n$-strongly compact objects, where an object is said to be $n$-strongly compact if the functor it corepresents commutes with $(n,1)$-categorical sifted colimits. For categories which are compactly generated, a natural class of functors to consider between them are those which preserve compact objects. In the case of compactly assembled categories, this condition may be weakened to yield the notion of compact functor. We explore this in \ref{subsection colimits}, together with the related notion of $n$-strongly compact functor. 

Finally, in \ref{subsection lifting} we prove the main result of this section, that guarantees that the functor that assigns to each ($n$-strongly) compactly assembled category its full subcategory on the ($n$-strongly) compact objects preserves filtered colimits of diagrams with ($n$-strongly) compact transitions.  This will be of fundamental importance in section \ref{section G rings}, allowing us deduce the general case of theorem \ref{theorem principal 2 introduction} from the case of complete local Noetherian rings.

\subsection{Compact assembly and strong compact assembly}\label{subsection compactly assembled}

We begin with a review of the notion of compactly assembled category.

\begin{notation}
We denote by $\catomega$ the subcategory of $\cathat$ on those large categories admitting small filtered colimits and functors which preserve small filtered colimits. We will denote by $\Ind:  \cathat \rightarrow \catomega$ the left adjoint to the forgetful functor. In other words, $\Ind$ is the functor that freely adjoins small filtered colimits.
\end{notation}

\begin{definition}\label{def compactly assembled}
Let $\Ccal$ be a category admitting small filtered colimits. We say that $\Ccal$ is compactly assembled if the functor $\Ind(\Ccal) \rightarrow \Ccal$ which ind-extends the identity on $\Ccal$, admits a left adjoint.
\end{definition}

\begin{remark}
The notion of compactly assembled category was introduced, in the $(1,1)$-categorical context, in \cite{JJcont} under the name of continuous category. The $\infty$-categorical notion is explored in \cite{SAG} section 21.1.2. Note that our definition is slightly different from that of \cite{SAG} in that we do not require accessibility of $\ccal$. This allows us to treat $\ccal$ and $\Ind(\ccal)$ on equal footing: if $\ccal$ is a category with small filtered colimits then $\Ind(\ccal)$ is compactly assembled according to definition \ref{def compactly assembled}, but usually not accessible.
\end{remark}

\begin{remark}\label{remark compactly assembled presentable}
Let $\ccal$ be a presentable category, and assume that $\ccal$ is compactly assembled. Let $p: \Ind(\ccal) \rightarrow \ccal$ be the projection and $i: \ccal \rightarrow \Ind(\ccal)$ be its left adjoint. For each regular cardinal $\kappa$ let $\ccal^\kappa$ be the full subcategory of $\ccal$ on the $\kappa$-compact objects. Then $\Ind(\ccal) = \colim \Ind(\ccal^\kappa)$, and since $\ccal$ is presentable we have that $i$ factors through $\Ind(\ccal^\kappa)$ for some $\kappa$. The restriction of $p$ to $\Ind(\ccal^\kappa)$ is a limit preserving localization between presentable categories.
\end{remark}

The following is the content of \cite{SAG} theorem 21.1.2.10 (with appropriate modifications to remove the accessibility conditions):

\begin{theorem}\label{theorem comp assembled iff retract}
Let $\Ccal$ be a category admitting small filtered colimits. Then $\Ccal$ is compactly assembled if and only if it is a retract in $\catomega$ of a category which is generated under small filtered colimits by compact objects.
\end{theorem}

The following result explains the relevance of the notion of compact assembly for our purposes:

\begin{theorem}[\cite{SAG} proposition D.7.3.1 and corollary D.7.7.3] \label{theorem dualizable iff comp assembled}
Let $\Mcal$ be a compactly generated presentable symmetric monoidal stable category. Assume that compact and dualizable objects in $\mcal$ coincide (in other words, $\Mcal$ is rigid). Then an object of $\Mod_{\Mcal}(\Pr^L)$ is dualizable if and only if the underlying category is compactly assembled.
\end{theorem}

We now discuss a variant of the notion of compact assembly, where filtered colimits are replaced by sifted colimits. We fix throughout a constant $1 \leq n \leq \infty$. We begin with a preliminary discussion on the notion of sifted colimits in $(n,1)$-categories.

\begin{proposition}\label{prop equivalences initial}
Let $F: \Ical \rightarrow \Jcal$ be a functor of $(n,1)$-categories. The following are equivalent:
\begin{enumerate}[\normalfont (1)]
\item For every $(n,1)$-category $\Ccal$ and every limit diagram $\Jcal^\lhd \rightarrow \Ccal$, the induced diagram $\Ical^\lhd \rightarrow \Ccal$ is a limit diagram.
\item For every object $X$ in $\Jcal$, the geometric realization of the category $\Ical \times_{\Jcal} \Jcal_{/X}$ is $n$-connective.
\end{enumerate}
\end{proposition}
\begin{proof}
Assume first that (1) holds and let $X$ be an object in $\Jcal$. Then the projection $\Ical \times_{\Jcal} \Jcal_{/X} \rightarrow \Ical$ is a right fibration which classifies the functor $\Hom_{\Jcal}(-, X)|_{\Ical^\op}$. The geometric realization of $\Ical \times_{\Jcal} \Jcal_{/X}$ is therefore equivalent to the colimit of $\Hom_{\Jcal}(-, X)|_{\Ical^\op}: \Ical^\op \rightarrow \Spc$, which in turn agrees with the limit of the functor $\Hom_{\Jcal}(-, X)|_{\Ical}: \Ical \rightarrow \Spc^\op$. To prove (2) it suffices to show that the limit of the functor $\Hom_{\Jcal}(-, X)|_{\Ical}: \Ical \rightarrow (\Spc_{\leq n-1})^\op$ is the terminal space. Applying (1) we reduce to showing that the limit of the functor $\Hom_{\Jcal}(-, X): \Jcal \rightarrow (\Spc_{\leq n-1})^\op$ is the terminal space. As before, we may identify this with the $(n-1)$-truncation of the geometric realization of the right fibration classifying the functor $\Hom_{\Jcal}(-, X): \Jcal^\op \rightarrow \Spc$. The fact that (2) holds now follows from the fact that this right fibration admits a terminal object and is therefore contractible.

Assume now that (2) holds. Since the inclusion $\ccal \rightarrow \Pcal(\ccal)$ of $\ccal$ into its free cocompletion preserves all limits we may after replacing $\ccal$ with $\Pcal(\ccal)_{\leq n-1}$ assume that $\ccal$ is presentable. Let $\Dcal = (\Pcal(\ccal^\op)_{\leq n-1})^\op$ be the $(n,1)$-category obtained from $\ccal$ by freely adjoining small limits. Since the projection $\dcal \rightarrow \ccal$ preserves limits, it is enough to show that restriction along $F$ preserves limits valued in $\dcal$. Note that $\dcal$ is equivalent to the opposite of the category of accessible functors $\ccal \rightarrow \Spc_{\leq n-1}$, and in particular it has a limit preserving embedding into $\Funct(\ccal^\op, (\Spc_{\leq n-1})^\op)$. It is therefore enough to show that restriction along $F$ preserves limits valued in $\Funct(\ccal^\op, \Spc_{\leq n-1}^\op)$. Since the evaluation functors create limits we may now reduce to the case $\ccal = \Spc_{\leq n-1}^\op$. In this case every functor $G: \Jcal \rightarrow \Spc_{\leq n-1}^\op$ is the limit of corepresentable functors. Since the property that the limit of $G$ is preserved by restriction along $F$ is preserved by limits in $G$, it is enough to consider the case where $G$ is corepresentable. The arguments from the first paragraph of the proof show that this case is implied by (2).
\end{proof}

\begin{definition}
We say that a functor $F: \Ical \rightarrow \Jcal$ of $(n,1)$-categories is initial if it satisfies the equivalent conditions of proposition \ref{prop equivalences initial}. We say that an $(n,1)$-category $\Ical$ is $n$-sifted if the diagonal $\Ical^\op \rightarrow \Ical^\op \times  \Ical^\op$ is initial.
\end{definition}

\begin{definition}
Let $\Ccal$ be an $(n,1)$-category admitting small colimits indexed by $n$-sifted $(n,1)$-categories and let $X$ be an object in $\Ccal$. We say that $X$ is $n$-strongly compact if the functor $\Hom_\Ccal(X, -): \Ccal \rightarrow \Spc_{\leq n-1}$ preserves small $n$-sifted colimits.
\end{definition}

\begin{remark}
In the case $n = \infty$, our notion of $\infty$-strongly compact object agrees with the notion of compact projective object from \cite{HTT} section 5.5.8.
\end{remark}

\begin{notation}
Let $\cathat_n$ be the category of large $(n,1)$-categories. We denote by $\cathat^\Sigma_n$ the subcategory of $\cathat_n$ on the large $(n,1)$-categories admitting small $n$-sifted colimits, and functors which preserve small sifted colimits. The forgetful functor $\cathat_n^\Sigma \rightarrow \cathat_n$ admits a left adjoint, which we denote by $\Pcal_\Sigma^n$. In other words, $\Pcal_\Sigma^n$ is the functor that freely adjoins small $n$-sifted colimits to an $(n,1)$-category. If $n = \infty$ we denote this simply by $\Pcal_\Sigma$.
\end{notation}

\begin{remark}
For each $(n,1)$-category $\Ccal$ we have a Yoneda embedding $\Ccal \rightarrow \Pcal_\Sigma^n(\Ccal)$. Its image consists of $n$-strongly compact objects of $\Pcal_\Sigma^n(\Ccal)$, and generates it under $n$-sifted colimits. These properties characterize the categories of the form $\Pcal_{\Sigma}^n(\Ccal)$: if $\Dcal$ is an $(n,1)$-category with $n$-sifted colimits, generated under $n$-sifted colimits by $n$-strongly compact objects, then $\Dcal = \Pcal_{\Sigma}^n(\Ccal)$ where $\Ccal$ is the full subcategory of $\Dcal$ on the $n$-strongly compact objects.
\end{remark}

With the notion of $n$-sifted colimits at hand, we may define $n$-strong compact assembly in a way which is completely analogous to compact assembly.

\begin{definition}
Let $\Ccal$ be an $(n,1)$-category admitting small $n$-sifted colimits. We say that $\Ccal$ is $n$-strongly compactly assembled if the unique $n$-sifted colimit preserving extension  $\Pcal_{\Sigma}^n(\Ccal) \rightarrow \Ccal$  of the identity on $\Ccal$ admits a left adjoint.
\end{definition}

\begin{remark}\label{remark presentable strongly compactly assembled}
As in remark \ref{remark compactly assembled presentable}, one shows that if $\ccal$ is an $n$-strongly compactly assembled presentable category then $\ccal$ is a limit preserving localization of $\Pcal^n_\Sigma(\ccal^\kappa)$ for some regular cardinal $\kappa$.
\end{remark}

In the additive context, $n$-strongly compactly assembled categories have particularly good behavior:

\begin{proposition}\label{prop properties strongly compactly assembled} 
\hspace{1cm}
\begin{enumerate}[\normalfont (1)]
\item Let $\ccal$ be a $1$-strongly compactly assembled presentable additive  $(1,1)$-category. Then $\ccal$ is a Grothendieck abelian category with exact products.
\item Let $\ccal$ be an $\infty$-strongly compactly assembled presentable additive category. Then $\ccal$ is a separated Grothendieck prestable category such that products in $\Sp(\ccal)$ are exact.
\end{enumerate}
\end{proposition}
\begin{proof}
The property of being a Grothendieck abelian category with exact products (resp. a separated Grothendieck prestable category with exact products) is preserved under passage to limit preserving accessible localizations. The proposition now follows from remark \ref{remark presentable strongly compactly assembled}. 
\end{proof}

The following is a variant of theorem \ref{theorem comp assembled iff retract}:

\begin{proposition}\label{prop characterize strongly comp assembled}
Let $\Ccal$ be an $(n,1)$-category admitting $n$-sifted colimits. Then $\Ccal$ is $n$-strongly compactly assembled if and only if it is a retract in $\cathat_n^\Sigma$ of a category which is generated under $n$-sifted colimits by $n$-strongly compact objects.
\end{proposition}
\begin{proof}
If $\ccal$ is $n$-strongly compactly assembled then $\ccal$ is a retract in $\cathat_n^\Sigma$ of $\Pcal_\Sigma^n(\ccal)$. Conversely, assume that $\ccal$ is a retract in $\cathat_n^\Sigma$ of a category $\dcal$ which is generated under $n$-sifted colimits by $n$-strongly compact objects. Then the projection $p_\Ccal: \Pcal^n_\Sigma(\ccal) \rightarrow \ccal$ is a retract of the projection $p_\Dcal: \Pcal^n_\Sigma(\Dcal) \rightarrow \Dcal$. The functor $p_\Dcal$ admits a left adjoint which sends each $n$-strongly compact object to its image under the Yoneda embedding. The fact that $p_\Ccal$ has a left adjoint now follows from \cite{SAG} lemma 21.1.2.14. 
\end{proof}

We also have the following variant of theorem \ref{theorem dualizable iff comp assembled}:

\begin{theorem}\label{theorem dualizables over truncated connective}
Let $\Mcal$ be a presentable symmetric monoidal additive $(n,1)$-category. Assume that $\Mcal$ is generated under colimits by $n$-strongly compact objects, and that $n$-strongly compact objects and dualizable objects coincide in $\Mcal$ (in other words, $\Mcal$ is rigid). Then an object of $\Mod_{\Mcal}(\Pr^L)$ is dualizable if and only if the underlying $(n,1)$-category is $n$-strongly compactly assembled.
\end{theorem}

\begin{remark}\label{remark description duality comp}
Let $\ccal$ be a presentable additive $(n,1)$-category  generated under colimits by $n$-strongly compact objects. Denote by $\ccal^\Sigma$ the full subcategory of $\ccal$ on the $n$-strongly compact objects. Then there is a colimit preserving functor
\[
\ccal \otimes \Pcal_\Sigma^n((\ccal^\Sigma)^\op) \rightarrow \Sp_{\leq n-1}
\]
with target the category of $(n-1)$-truncated spectra, defined by the property that its restriction to $\ccal \times (\ccal^\Sigma)^\op$ recovers the restriction of the functor $\Hom^\enh_\ccal(-,-): \ccal^\op \times \ccal \rightarrow \Sp_{\leq n-1}$. The above functor is  the counit of a duality in $\Mod_{\Sp_{\leq n-1}}(\Pr^L)$. In particular, we see that the dual to $\ccal$ is also generated under colimits by $n$-strongly compact objects.
\end{remark}

\begin{proof}[Proof of theorem \ref{theorem dualizables over truncated connective}]
The assumptions on $\Mcal$ imply that the tensoring functor $\Mcal \otimes \Mcal \rightarrow \Mcal$ admits an $\Mcal$-linear colimit preserving right adjoint, and its image generates $\Mcal$ under colimits. It follows from this that $\Mcal$ is equivalent to the category of modules over the endomorphism algebra of $1_\Mcal$ in $\Mcal \otimes \Mcal$, and therefore it is dualizable as a module over $\Mcal \otimes \Mcal$.  Combined with remark \ref{remark description duality comp} we see that $\Mcal$ is smooth and proper as an algebra in $\Mod_{\Sp_{\leq n-1}}(\Pr^L)$, so that an $\Mcal$-module is dualizable if and only if its image in $\Mod_{\Sp_{\leq n-1}}(\Pr^L)$ is dualizable.

It remains to show that an object $\ccal$ in  $\Mod_{\Sp_{\leq n-1}}(\Pr^L)$ is dualizable if and only if it is $n$-strongly compactly assembled. Assume first that $\ccal$ is $n$-strongly compactly assembled. It follows from remark \ref{remark presentable strongly compactly assembled} that $\ccal$ is a retract in $\Mod_{\Sp_{\leq n-1}}(\Pr^L)$ of a category which is generated under colimits by $n$-strongly compact objects. The fact that $\ccal$ is dualizable now follows from remark \ref{remark description duality comp} together with the fact that $\Mod_{\Sp_{\leq n-1}}(\Pr^L)$ is idempotent complete.

Assume now that $\ccal$ is a dualizable object of $\Mod_{\Sp_{\leq n-1}}(\Pr^L)$. Pick a regular cardinal $\kappa$ such that $\ccal$ is $\kappa$-accessible. Then the colimit preserving functor $p: \Pcal_\Sigma^n(\ccal^\kappa) \rightarrow \ccal$ that extends the inclusion $\ccal^\kappa \rightarrow \ccal$ is a localization. It follows that $\id_{\ccal^\vee} \otimes p$ is also a localization, which implies in particular that the functor $\Funct^L(\ccal, \Pcal_\Sigma^n(\ccal^\kappa)) \rightarrow \Funct^L(\ccal, \ccal)$ of composition with $p$ is surjective. Therefore the identity on $\ccal$ admits a colimit preserving lift along $p$, which means that $\ccal$ is a retract in $\Pr^L$ of $\Pcal_\Sigma^n(\ccal^\kappa)$. The fact that $\ccal$ is $n$-strongly compactly assembled now follows from proposition \ref{prop characterize strongly comp assembled}.
\end{proof}

\begin{corollary}\label{coro properties dualizable 11}
Let $\acal$ be a symmetric monoidal Grothendieck abelian category, rigid and generated by compact projective objects. Let $\ccal$ be a dualizable   $\acal$-linear cocomplete category. Then $\ccal$ is a Grothendieck abelian category with exact products.
\end{corollary}
\begin{proof}
Combine propositions \ref{proposition dualizable is presentable} and \ref{prop properties strongly compactly assembled} with theorem \ref{theorem dualizables over truncated connective}.
\end{proof}

\begin{corollary}\label{coro properties dualizable infty}
Let $\Mcal$ be a symmetric monoidal Grothendieck prestable category, rigid and generated under colimits by compact projective objects. Let $\ccal$ be a dualizable   $\Mcal$-linear cocomplete category. Then $\ccal$ is a separated Grothendieck prestable category and products are t-exact in $\Sp(\ccal)$.
\end{corollary}
\begin{proof}
Combine propositions \ref{proposition dualizable is presentable} and \ref{prop properties strongly compactly assembled} with theorem \ref{theorem dualizables over truncated connective}.
\end{proof}

\begin{remark}\label{remark description duality over M}
Let $\Mcal$ be as in theorem \ref{theorem dualizables over truncated connective}. Let $\ccal$ be a dualizable $\Mcal$-module and let $\epsilon: \Ccal \otimes_\Mcal \ccal^\vee \rightarrow \Mcal$ be the counit of the duality. Then the composite map
\[
\ccal \otimes \ccal^\vee \rightarrow \ccal \otimes_{\Mcal} \ccal^\vee \xrightarrow{\epsilon} \Mcal \xrightarrow{\Hom^\enh_{\Mcal}(1_\Mcal, -)} \Sp_{\leq n-1}
\]
is the counit of a duality between $\ccal$ and $\ccal^\vee$ in the symmetric monoidal category of presentable additive $(n,1)$-categories. 

Assume now that $\ccal$ is generated under colimits by $n$-strongly compact objects. Then combining with remark \ref{remark description duality comp} we obtain an identification $\ccal^\vee = \Pcal_\Sigma^n((\ccal^\Sigma)^\op)$, such that $\epsilon$ is a colimit preserving $\Mcal$-linear functor with the property that its restriction to $\ccal \times  (\ccal^\Sigma)^\op$ recovers the restriction of the functor $\Hom^\enh_\ccal(-,-): \ccal^\op \times \ccal \rightarrow \Mcal$.
\end{remark}

The following proposition shows that passage to derived categories preserves the property of being strongly compactly assembled:

 \begin{proposition}\label{lemma derive strong compact assembly}
 Let $\ccal$ be a $1$-strongly compactly assembled Grothendieck abelian category. Then $\der(\ccal)_{\geq 0}$ is $\infty$-strongly compactly assembled.
 \end{proposition}
 \begin{proof}
 By remark \ref{remark presentable strongly compactly assembled}, we may pick a regular cardinal $\kappa$ such that the canonical functor $p : \Pcal_\Sigma^1(\ccal^\kappa) \rightarrow \ccal$ is a limit preserving localization. In particular, passing to derived categories it induces a left exact colimit preserving functor $p': \Pcal_\Sigma(\ccal^\kappa) \rightarrow \der(\ccal)_{\geq 0}$. It follows from the universal property of derived categories that $p'$ is the universal left exact colimit preserving functor from $\Pcal_\Sigma(\ccal^\kappa)$ into a separated  Grothendieck prestable category such that $p'(X) = 0$ for all $X$ in $\Pcal_\Sigma^1(\ccal^\kappa) $ such that $p(X) = 0$. Therefore $p'$ exhibits $\der(\ccal)_{\geq 0}$ as the left exact localization of $\Pcal_\Sigma(\ccal^\kappa)$ at the localizing class given by those objects $X$ such that $p(H_n(X)) = 0$ for all $n$ (see \cite{SAG} proposition C.5.2.7). 
 
 To prove that $\ccal$ is $\infty$-strongly compactly assembled it will suffice to show that $p'$ admits a left adjoint. By the adjoint functor theorem, it is enough to show that $p'$ preserves limits. Since $p'$ is already known to preserve finite limits, we may reduce to showing that $p'$ preserves products. Let $X_\alpha$ be a set-indexed family of objects of $\Pcal_\Sigma(\ccal^\kappa)$. We wish to show that $p'(\prod X_\alpha) = \prod p'(X_\alpha)$. By proposition \ref{prop properties strongly compactly assembled} it suffices to show that $H_n(p'(\prod X_\alpha)) = \prod H_n(p'(X_\alpha))$ for all $n \geq 0$. Since $p'$ is left exact and colimit preserving we may reduce to proving that $p'(H_n(\prod X_\alpha)) = \prod p'(H_n(X_\alpha))$. This follows from the fact that $p$ preserves products and that products are exact in $\Pcal_\Sigma(\ccal^\kappa)$.
  \end{proof}

\begin{remark}
Let $\ccal$ be an $\infty$-strongly compactly assembled Grothendieck prestable category. It follows from remark \ref{remark presentable strongly compactly assembled} that $\ccal^\heartsuit$ is a limit preserving localization of a Grothendieck abelian category generated by compact projective objects. Applying proposition \ref{prop characterize strongly comp assembled} we see that $\ccal^\heartsuit$ is $1$-strongly compactly assembled. Similarly, $\Sp(\ccal)$ is  a limit preserving localization of a compactly generated presentable stable category, and therefore $\Sp(\ccal)$ is compactly assembled.
\end{remark}

We finish by discussing two families of examples of categories which are (strongly) compactly assembled but not (strongly) compactly generated.

\begin{example}\label{example almost modules}
Let $R$ be a valuation ring with value group $\ZZ[1/2]$ and let $\mathfrak{m}$ be the maximal ideal of $R$. Then $\mathfrak{m}$ is idempotent and flat. It follows that the functor $- \otimes \mathfrak{m} : \Mod_R^\heartsuit \rightarrow \Mod_R^\heartsuit$ is a colimit preserving left exact colocalization. Its image is the Grothendieck abelian category $\operatorname{aMod}_{R}^\heartsuit$ of almost $R$-modules from \cite{GabberRamero}. This is a retract of $\Mod^\heartsuit_R$ in $\Pr^L$, and in particular it is $1$-strongly compactly generated and a dualizable Grothendieck abelian category. Any projective object of  $\operatorname{aMod}_R^\heartsuit$ defines a projective $R$-module $P$ with the property that $\mathfrak{m} \otimes_R P = P$. Since any nonzero projective $R$-module contains a copy of $R$ and $\mathfrak{m} \otimes_R R \neq R$, we have that $P$ is necessarily zero. In particular we have that  $\operatorname{aMod}_R^\heartsuit$ is not generated by $1$-strongly compact objects.

The category $\operatorname{aMod}_R^\heartsuit$ inherits an $R$-linear structure from $\operatorname{Mod}_R^\heartsuit$. If $R$ is an algebra over a field $k$ then $\operatorname{aMod}_R^\heartsuit$ gives an example of a nonzero dualizable $k$-linear Grothendieck abelian category without nonzero projective objects.

This example extends to the derived context. Namely, $- \otimes \mathfrak{m}: \Mod_R \rightarrow \Mod_R$ is a colimit preserving t-exact colocalization. Its image is the presentable stable category $\operatorname{aMod}_R$ of almost $R$-modules. It inherits a t-structure from $\Mod_R$, and is equivalent to the derived category of $\operatorname{aMod}_R^\heartsuit$. For each $1 \leq n \leq \infty$ the full subcategory $(\operatorname{aMod}^\cn_R)_{\leq n-1}$  on the $(n-1)$-truncated connective objects is $n$-strongly compactly assembled and has no nontrivial projective objects.

The presentable stable category $\operatorname{aMod}_R$ is compactly assembled by the same reasoning as above. We claim that it contains no nontrivial compact objects. This is equivalent to the assertion that if $M$ is a compact $R$-module such that $\mathfrak{m} \otimes_R M = M$ then $M = 0$. Assume for the sake of contradiction that $M \neq 0$, and let $n$ be such that $H_n(M) \neq 0$. Since $R$ is coherent we have that $H_n(M)$ is coherent. We may then write $H_n(M)$ as a finite direct sum of modules of the form $R/Rx$ for $x$ a nonunit in $R$.  The fact that $\mathfrak{m} \otimes_R M = M$ then implies that $\mathfrak{m} \otimes_R R/Rx = R/Rx$ for some nonunit $x$.   We now derive a contradiction from the fact that $\mathfrak{m} \otimes_R R/Rx$ is not finitely generated.
\end{example}

\begin{example}\label{example non torsion}
Let $R$ be a connected commutative ring such that the \'etale cohomology group $H^2(\Spec(R), \GG_m)$ is non-torsion (see \cite{Grothendieck} remark 1.11b, or chapter 7 in \cite{BrauerBook} for examples). Let $\mathcal{G}$ be a non-torsion $\GG_m$-gerbe on $\Spec(R)$, and let $\Mod^\heartsuit_{R, \mathcal{G}}$ be the associated twist of $\Mod^\heartsuit_R$  (see notation \ref{notation twist mod}). Then $\Mod^\heartsuit_{R, \mathcal{G}}$ is an invertible $R$-linear Grothendieck abelian category, and in particular it is $1$-strongly compactly assembled. 

We claim that the only compact projective object of $\Mod^\heartsuit_{R, \mathcal{G}}$ is zero. Assume that $X$ is a compact projective object, and let $\Dcal$ be the full subcategory of $\smash{\Mod^\heartsuit_{R, \mathcal{G}}}$ generated under colimits by $X$.  Since $\mathcal{G}$ is non-torsion, we have that $\Mod^\heartsuit_{R, \mathcal{G}}$  is not the category of modules over an Azumaya $R$-algebra, and therefore $\dcal \neq \smash{\Mod^\heartsuit_{R, \mathcal{G}}}$. Pick a faithfully flat \'etale $R$-algebra $R'$ such that the pullback of $\mathcal{G}$ along the projection $\Spec(R') \rightarrow \Spec(R)$ is trivial. Then we may identify $X \otimes_R R'$ with a finitely generated free $R'$-module. The subcategory of $\Mod_{R'}^\heartsuit$ generated by $X \otimes_R R'$ is $\dcal \otimes_R R'$, so we see that $X \otimes_R R'$ is not a generator for $\Mod_{R'}^\heartsuit$. The fact that $R$ is connected now implies that $X$ vanishes.

Taking $R$ to be a commutative algebra over a field $k$ yields an example of a dualizable $k$-linear Grothendieck abelian category which has no nontrivial compact projective objects. We note that this example extends to the connective derived context as well: for each $1 \leq n \leq \infty$ the full subcategory $(\der(\Mod^\heartsuit_{R, \mathcal{G}})_{\geq 0})_{\leq n-1}$ of the derived category of $\Mod^\heartsuit_{R, \mathcal{G}}$ on the connective $(n-1)$-truncated objects is $n$-strongly compactly assembled and has no nontrivial compact projective objects. Note that the situation is different once one stabilizes: the full derived category $\der(\Mod^\heartsuit_{R, \mathcal{G}})$ is compactly generated, as explained in \cite{ToenAzumaya}.
\end{example}

%%%%%%%%%%%%%%%%%%%%%%%%%%%%%%%%%%%%%%%%%%%%%%%%%%%%%%%%%%%%%%%%%%%%%%%%
%%%%%%%%%%%%%%%%%%%%%%%%%%%%%%%%%%%%%%%%%%%%%%%%%%%%%%%%%%%%%%%%%%%%%%%%
%%%%%%%%%%%%%%%%%%%%%%%%%%%%%%%%%%%%%%%%%%%%%%%%%%%%%%%%%%%%%%%%%%%%%%%%
%%%%%%%%%%%%%%%%%%%%%%%%%%%%%%%%%%%%%%%%%%%%%%%%%%%%%%%%%%%%%%%%%%%%%%%%
%%%%%%%%%%%%%%%%%%%%%%%%%%%%%%%%%%%%%%%%%%%%%%%%%%%%%%%%%%%%%%%%%%%%%%%%
%%%%%%%%%%%%%%%%%%%%%%%%%%%%%%%%%%%%%%%%%%%%%%%%%%%%%%%%%%%%%%%%%%%%%%%%

\subsection{Compact and strongly compact functors}\label{subsection colimits}

Our next goal is to study certain classes of filtered colimits in $\catl$ which are preserved by the forgetful functor to $\catomega$.  The relevant condition that we will need to impose on our diagrams is that the transition maps be compact functors. In the presence of right adjoints, compactness of functors is equivalent to the right adjoints preserving filtered colimits. In general, one has to pass to a suitable completion of the categories to be able to define this right adjoint.

\begin{notation}\label{notation right adjoint after completing}
For each object $\Ccal$ in $\catomega$ we denote by $\widehat{\Ccal}$ the (very large) category obtained by adjoining large colimits to $\Ccal$ while preserving the filtered small colimits in existence. Given a functor $F: \ccal \rightarrow \dcal$ in $\catomega$ we denote by $\widehat{F}: \widehat{\ccal} \rightarrow \widehat{\Dcal}$ the induced functor. Observe that $\widehat{F}$ is a colimit preserving functor between (very large) presentable categories, and it therefore admits a right adjoint.
\end{notation}

\begin{definition}
Let $F: \ccal \rightarrow \Dcal$ be a morphism in $\catomega$. We say that $F$ is compact if the right adjoint to the functor $\widehat{F}$ from notation \ref{notation right adjoint after completing} preserves large colimits.
\end{definition}

\begin{example}\label{example compact is compact}
Let $\Dcal$ be a category with small filtered colimits, and $F: \Delta^0 \rightarrow \Dcal$ be a functor that picks out an object $X$ in $\Dcal$. Then $F$ is compact if and only if $X$ is compact.
\end{example}

\begin{remark}
Let $F: \ccal \rightarrow \Dcal$ be a morphism in $\catomega$ and assume that $F$ admits a right adjoint $F^R$. Then $F$ is compact if and only if $F^R$ preserves small filtered colimits.
\end{remark}

\begin{remark}\label{remark compact sends compact to compact}
Compact functors are closed under compositions. In particular, it follows from example \ref{example compact is compact} that if a functor $F: \Ccal \rightarrow \Dcal$ in $\catomega$ is compact then it sends compact objects to compact objects. The converse is true provided that $\Ccal$ is generated under filtered colimits by compact objects.
\end{remark}

\begin{remark}\label{remark colimit of compacts}
Let $(\Ccal_\alpha)$ be a diagram in $\catomega$ with compact transition maps and colimit $\Ccal$. Denote by $F_{\alpha, \beta}: \Ccal_\alpha \rightarrow \ccal_\beta$ the transition maps, and $F_\alpha: \Ccal_\alpha \rightarrow \Ccal$ the induced functors. Then the functors $(\widehat{F_\alpha})^R$ exhibit $\widehat{\Ccal}$ as the limit of the categories $\widehat{\ccal_\alpha}$ along the transition maps $(\widehat{F_{\alpha, \beta}})^R$. It follows from this that $F_\alpha$ is compact for all $\alpha$.
\end{remark}

\begin{remark}
Let $\Ccal$ be an object of $\catl$. Then we may consider the (very large) category $\widetilde{\Ccal}$ obtained from $\Ccal$ by freely adjoining large colimits while preserving the small colimits in existence. Alternatively, $\widetilde{\Ccal}$ is obtained from $\Ccal$ by freely adjoining large $\kappa$-filtered colimits, where $\kappa$ is the smallest large cardinal. The category $\widehat{\Ccal}$ is then obtained from $\widetilde{\Ccal}$ by freely adjoining large colimits while preserving the large filtered colimits in existence.

Assume now given a functor $F: \Ccal \rightarrow \Dcal$ in  $\catl$. Then the induced functor $\widetilde{F}: \widetilde{\ccal} \rightarrow \widetilde{\dcal}$ is a colimit preserving functor between very large presentable categories, and it therefore has a right adjoint. The functor $F$ is compact if and only if the right adjoint to $\widetilde{F}$ preserves large filtered colimits.  In particular, similar arguments as in remark \ref{remark colimit of compacts} show that if $\Ccal$ is a colimit of a diagram $(\Ccal_\alpha)$ in $\catl$ with compact transition maps, then the functors $\Ccal_\alpha \rightarrow \Ccal$ are compact.
\end{remark}

\begin{example}
Let $\Dcal$ be a cocomplete category and $F: \Spc \rightarrow \Dcal$ be a colimit preserving functor. Then $F$ is compact if and only if $F(\Delta^0)$ is a compact object of $\Dcal$.
\end{example}

\begin{proposition}\label{prop filtered colimits preserved}
The forgetful functor $\catl \rightarrow \catomega$ preserves colimits of filtered diagrams with compact transitions.
\end{proposition}

The proof of proposition  \ref{prop filtered colimits preserved} works by reduction to the case of compactly generated categories:

\begin{lemma}\label{lemma colim of ind completions}
The forgetful functor $\catl \rightarrow \catomega$ preserves colimits of small filtered diagrams of compactly generated presentable categories and compact functors.
\end{lemma}
\begin{proof}
Let $(\Ccal_\alpha)$ be a small filtered diagram of compactly generated presentable categories and compact functors, and let $\Ccal$ be its colimit in $\catl$. Then $\Ccal$ is the colimit of the diagram $(\Ccal_\alpha)$ on the category $\Pr^L_\omega$ of compactly generated presentable categories and compact functors. Passing to compact objects induces an equivalence between this and the category $\Cat^{\rex, \id}$ of finitely cocomplete idempotent complete categories and right exact functors, so that $\Ccal^\omega$ is the colimit of $(\Ccal_\alpha^\omega)$ in $\Cat^{\rex,\id}$. Since the forgetful functor $\Cat^{\rex, \id} \rightarrow \Cat$ preserves filtered colimits we furthermore have that $\Ccal^\omega$ is the colimit of the diagram $(\Ccal_\alpha^\omega)$ in $\Cat$. The lemma now follows from the fact that the functor $\Cat \rightarrow \catomega$ of ind-completion is colimit preserving.
\end{proof}

\begin{proof}[Proof of proposition \ref{prop filtered colimits preserved}]
Assume given a filtered system $(\Ccal_\alpha)$ in $\catl$ with compact transitions, and let $\Ccal$ be its colimit. We must show that $\Ccal$ is also the colimit of this diagram in $\catomega$. Replacing $\Ccal_\alpha$ and $\Ccal$ with $\widetilde{\Ccal_\alpha}$ and $\widetilde{\Ccal}$ and changing the universe, we may assume that all the categories involved are in fact presentable, and the diagram is small.

Let $\kappa$ be a  regular cardinal such that all the categories $\Ccal_\alpha$ are $\kappa$-accessible for all $\alpha$ and all the transition functors and their right adjoints are $\kappa$-accessible. For each $\alpha$ the inclusion $\Ccal_\alpha^\kappa \rightarrow \Ccal_\alpha$ of the subcategory of $\kappa$-compact objects admits an ind-extension $p_\alpha: \Ind(\Ccal_\alpha^\kappa) \rightarrow \Ccal_\alpha$ which is   an accessible localization. Let $p: \Dcal \rightarrow \Ccal$ be the colimit of the maps $p_\alpha$, computed in $\Pr^L$. For each $\alpha$ denote by $g_\alpha: \Ind(\Ccal^\kappa_\alpha) \rightarrow \Dcal$ the canonical map, and note that $p$ exhibits $\Ccal$ as an accessible localization of $\Dcal$, at the union over all $\alpha$ of the image under $g_\alpha$ of the class of $p_\alpha$-local morphisms.

Assume now given an object $\Ecal$ in $\widehat{\Cat}^\omega$. We have a commutative square of spaces
\[
\begin{tikzcd}
\lim \Hom_{\catomega}(\Ind(\Ccal_\alpha^\kappa), \Ecal) & \arrow{l}{}  \Hom_{\catomega}(\Dcal, \Ecal) \\
\lim \Hom_{\catomega}(\Ccal_\alpha, \Ecal) \arrow{u}{\lim p_\alpha^*} & \arrow{l}{} \arrow{u}{p^*} \Hom_{\catomega}(\Ccal, \Ecal).
\end{tikzcd}
\]

By lemma \ref{lemma colim of ind completions}, the upper horizontal arrow is an equivalence. Furthermore, since $p$ and the maps $p_\alpha$ are all localizations, the vertical arrows are inclusions. Hence the bottom horizontal arrow is an inclusion. 

To prove our proposition it remains to show that the bottom horizontal arrow is surjective. In other words, we must show that if $F: \Dcal \rightarrow \Ecal$ is a filtered colimit preserving functor whose restriction to $\Ind(\Ccal^\kappa_\alpha)$ factors through $\Ccal_\alpha$ for all $\alpha$, then $F$ factors through $\Ccal$. It suffices for this to show that every $p$-local map is a filtered colimit of a diagram consisting of maps each of which is a image under $g_\alpha$ of a $p_\alpha$-local morphism for some $\alpha$.

For each $\alpha$ denote by $g_\alpha^R$ the right adjoint to $g_\alpha$. Since the identity of $\Dcal$ is the filtered colimit of the endofunctors $g_\alpha g_\alpha^R$, it suffices to show that $g_\alpha^R$ maps $p$-local maps to $p_\alpha$-local maps for each $\alpha$. This would follow if we are able to show that for every $\alpha$ the square
\[
\begin{tikzcd}
\Ind(\Ccal^\kappa_\alpha) \arrow{d}{p_\alpha} \arrow{r}{g_\alpha} & \Dcal \arrow{d}{p} \\
\Ccal_\alpha \arrow{r}{} & \Ccal
\end{tikzcd}
\]
is horizontally right adjointable. For this it is enough to prove that for each transition map the square
\[
\begin{tikzcd}
\Ind(\Ccal^\kappa_\alpha) \arrow{d}{p_\alpha} \arrow{r}{} & \Ind(\Ccal^\kappa_\beta) \arrow{d}{p_\beta} \\
\Ccal_\alpha \arrow{r}{} & \ccal_\beta 
\end{tikzcd}
\]
is horizontally right adjointable. This is a consequence of the fact that the right adjoints to the transition maps preserve filtered colimits.
\end{proof}

We now discuss a variant of the above where compactness is replaced by strong compactness. In what follows we fix a constant $1 \leq n  \leq \infty$.

\begin{notation}\label{notation extend F}
For each object $\Ccal$ in $\cathat_n^\Sigma$ we denote by $\widehat{\Ccal}$ the (very large) $(n,1)$-category obtained by adjoining large colimits to $\Ccal$ while preserving the $n$-sifted small colimits in existence. Given a functor $F: \Ccal \rightarrow \Dcal$ in $\cathat_n^\Sigma$ we denote by $\widehat{F}: \widehat{\Ccal} \rightarrow \widehat{\Dcal}$ the induced functor. Observe that $\widehat{F}$ is a colimit preserving functor between (very large) presentable categories, and it therefore admits a right adjoint.
\end{notation}

\begin{definition}
Let $F: \Ccal \rightarrow \Dcal$ be a morphism in $\cathat_n^\Sigma$. We say that $F$ is $n$-strongly compact if the right adjoint to the functor $\widehat{F}$ from notation \ref{notation extend F} preserves large colimits.
\end{definition}

\begin{example}\label{example n strongly compact from delta0}
Let $\Dcal$ be an object of $\cathat_n^\Sigma$ and $F: \Delta^0 \rightarrow \Dcal$ be a functor that picks out an object $X$ in $\Dcal$. Then $F$ is $n$-strongly compact if and only if $X$ is $n$-strongly compact.
\end{example}

\begin{remark}
Let $F: \Ccal \rightarrow \Dcal$ be a morphism in $\cathat_n^\Sigma$ and assume that $F$ admits a right adjoint $F^R$. Then $F$ is $n$-strongly compact if and only if $F^R$ preserves $n$-sifted colimits.
\end{remark}

\begin{remark}
Compositions of $n$-strongly compact functors are $n$-strongly compact. In particular, it follows from example \ref{example n strongly compact from delta0} that if a functor $F: \Ccal \rightarrow \Dcal$ in $\cathat_n^\Sigma$ is $n$-strongly compact then it maps  $n$-strongly compact objects to $n$-strongly compact objects. The converse is true provided that $\Ccal$ is generated under $n$-sifted colimits by $n$-strongly compact objects.
\end{remark}

The proof of proposition \ref{prop filtered colimits preserved} adapts to this setting, yielding the following:

\begin{proposition}
Let $\cathat^L_n$ be the subcategory of $\cathat$ on the large $(n,1)$-categories with small colimits and colimit preserving functors. Then the forgetful functor $\cathat^L_n \rightarrow \cathat_n^\Sigma$ preserves colimits of filtered diagrams with $n$-strongly compact transitions.
\end{proposition}

%%%%%%%%%%%%%%%%%%%%%%%%%%%%%%%%%%%%%%%%%%%%%%%%%%%%%%%%%%%%%%%%%%%%%%%%
%%%%%%%%%%%%%%%%%%%%%%%%%%%%%%%%%%%%%%%%%%%%%%%%%%%%%%%%%%%%%%%%%%%%%%%%
%%%%%%%%%%%%%%%%%%%%%%%%%%%%%%%%%%%%%%%%%%%%%%%%%%%%%%%%%%%%%%%%%%%%%%%%
%%%%%%%%%%%%%%%%%%%%%%%%%%%%%%%%%%%%%%%%%%%%%%%%%%%%%%%%%%%%%%%%%%%%%%%%
%%%%%%%%%%%%%%%%%%%%%%%%%%%%%%%%%%%%%%%%%%%%%%%%%%%%%%%%%%%%%%%%%%%%%%%%
%%%%%%%%%%%%%%%%%%%%%%%%%%%%%%%%%%%%%%%%%%%%%%%%%%%%%%%%%%%%%%%%%%%%%%%%

\subsection{Lifting of (strongly) compact objects}\label{subsection lifting}

We are now ready to discuss to what extent the operation of passing to compact objects commutes with taking filtered colimits of categories. The following is our main result on this topic:

\begin{theorem}\label{theorem compacts in filtered colimit}
Let $(\Ccal_\alpha)$ be a filtered diagram in $\catomega$ with compact transition maps, and let $\ccal$ be its colimit. Assume that $\Ccal_{\alpha}$ is compactly assembled for all $\alpha$. Then $\Ccal$ is compactly assembled and the induced functor $\colim \Ccal_\alpha^\omega \rightarrow \Ccal^\omega$ is an equivalence.
\end{theorem}

The proof of theorem \ref{theorem compacts in filtered colimit} requires some preliminary lemmas.
\begin{lemma}\label{lemma filtered colimit of cg}
Let $(\Ccal_\alpha)$ be a filtered diagram in $\catomega$ with compact transition maps, and let $\ccal$ be its colimit. Assume that $\ccal_\alpha$ is generated under filtered colimits by compact objects, for all $\alpha$. Then  $\Ccal$ is generated under filtered colimits by compact objects, and the induced map $\colim \Ccal_\alpha^\omega \rightarrow \Ccal^\omega$ is an equivalence.
\end{lemma}
\begin{proof}
By remark \ref{remark colimit of compacts} we have that the functors $\Ccal_\alpha \rightarrow \Ccal$ are all compact, and in particular they send compact objects to compact objects (remark \ref{remark compact sends compact to compact}). The category $\Ccal$ is generated under filtered colimits by the images of the functors $\Ccal_{\alpha}$. Since $\Ccal_{\alpha}$ is generated under filtered colimits by compact objects for all $\alpha$, we conclude that $\Ccal$ is generated under filtered colimits by compact objects. It remains to show that the map $\colim \Ccal_\alpha^\omega \rightarrow \Ccal^\omega$  is an equivalence.

Since the categories $\Ccal_\alpha$ and $\ccal$ admit filtered colimits, they are in particular idempotent complete, and hence $\ccal_\alpha^\omega$ and $\ccal^\omega$ are all idempotent complete. By \cite{HTT} corollary 4.4.5.21, we have that $\colim \ccal^\omega_\alpha$ is also idempotent complete. Hence the map $\colim \Ccal_\alpha^\omega \rightarrow \Ccal^\omega$ can be recovered by passing to compact objects of the induced functor $\Ind(\colim \Ccal_\alpha^\omega) \rightarrow \Ind(\Ccal^\omega)$. It thus suffices to show that the latter functor is an equivalence. 

Since $\Ind$ is a left adjoint, we have that $\Ind(\colim \Ccal_\alpha^\omega) = \colim \Ind(\ccal_\alpha^\omega)$ (where here the second colimit is computed in $\catomega$). Thus we must show that $\Ind(\ccal^\omega)$ is the filtered colimit of the categories $\Ind(\ccal_\alpha^\omega)$. This follows from the fact that $\Ccal$ and $\Ccal_\alpha$ are all generated under filtered colimits by compact objects.
\end{proof}

\begin{lemma}\label{lemma vertically left adj}
Let $F: \Ccal \rightarrow \Dcal$ be a compact functor in $\catomega$, where $\Ccal$ and $\Dcal$ are compactly assembled. Then the commutative square
\[
\begin{tikzcd}
\Ind(\Ccal) \arrow{r}{\Ind(F)} \arrow{d}{p_\Ccal} & \Ind(\Dcal) \arrow{d}{p_\Dcal} \\
\Ccal \arrow{r}{F} & \Dcal
\end{tikzcd}
\]
is vertically left adjointable. 
\end{lemma}
\begin{proof}
The fact that $\Ccal$ and $\Dcal$ are compactly assembled guarantees that the vertical arrows admit left adjoints, so we only need to verify that the square obtained by passage to left adjoints of the vertical arrows is strictly commutative. Consider the commutative cube
\[
\begin{tikzcd}
& \widehat{\Ind(\Ccal)} \arrow{rr}{\widehat{\Ind(F)}} \arrow[dd, pos=0.7, "\widehat{p_\Ccal}"] & & \widehat{\Ind(\Dcal)} \arrow{dd}{\widehat{p_\Dcal}} \\ 
\Ind(\Ccal) \arrow[rr, pos=0.7, "\Ind(F)"] \arrow{dd}{p_\Ccal} \arrow{ur}{} &  & \Ind(\Dcal) \arrow[dd, pos=0.7, "p_\Dcal"] \arrow{ur}{} & \\
 & \widehat{\Ccal} \arrow[rr,"\widehat{F}", pos=0.6] & & \widehat{\Dcal} \\
\Ccal \arrow{rr}{F} \arrow{ur}{} &  & \Dcal \arrow{ur}{} & 
\end{tikzcd}
\]
where the back face is obtained from the front face by adjoining all large colimits while preserving the small filtered colimits in existence. Here the diagonal arrows are fully faithful, and the left and right face are vertically left adjointable. We may thus reduce to showing that the square
\[
\begin{tikzcd}
\widehat{\Ind(\Ccal)} \arrow{r}{\widehat{\Ind(F)}} \arrow{d}{\widehat{p_\Ccal}} & \widehat{\Ind(\Dcal)} \arrow{d}{\widehat{p_\Dcal}} \\
\widehat{\Ccal} \arrow{r}{\widehat{F}} & \widehat{\Dcal}
\end{tikzcd}
\]
is vertically left adjointable. Since the vertical arrows admit left adjoints it will suffice to show that the above square horizontally right adjointable. The horizontal arrows in the above square are colimit preserving functors between very large presentable categories, so they admit right adjoints $(\widehat{F})^R$ and $\smash{(\widehat{\Ind(F)})^R}$ by the adjoint functor theorem. To finish the proof we must show that the induced natural transformation 
\[
\mu: \widehat{p_\Ccal} (\widehat{\Ind(F)})^R  \rightarrow (\widehat{F})^R \widehat{p_\Dcal} 
\]
is an isomorphism.

Since $p_\Ccal$ admits a fully faithful left adjoint it is a colocalization in $\catomega$, and hence $\widehat{p_\Ccal}$ is a colocalization of categories with large colimits. The adjoint functor theorem implies that $\widehat{p_\Ccal}$ also has a right adjoint $(\widehat{p_\Ccal})^R$, which is then necessarily fully faithful. Similarly, we have that $\widehat{p_\Dcal}$ admits a fully faithful right adjoint $(\widehat{p_\Dcal})^R$. It follows that $\mu$ is an isomorphism when restricted to the image of $(\widehat{p_\Dcal})^R$. 

We claim that the image of $(\widehat{p_\Dcal})^R$ contains $\Dcal$. Assume given an object $X$ in $\Dcal$. Then we may regard $X$ as an object of $\widehat{\Dcal}$, and its image under $(\widehat{p_\Dcal})^R$ is an object $Y$ in $\smash{\widehat{\Ind(\Dcal)}}$ with the property that 
\[
\Hom_{\widehat{\Ind(\Dcal)}}(-, Y)|_{\Dcal^\op} = \Hom_{\Dcal}(-, X) = \Hom_{\widehat{\Ind(\Dcal)}}(-, X)|_{\Dcal^\op}.
\] 
Since $\widehat{\Ind(\Dcal)}$ is the free category with large colimits on $\Dcal$ the inclusion $\Dcal \rightarrow \widehat{\Ind(\Dcal)}$ is dense, and hence $Y$ is equivalent to $X$. This shows that every object in $\Dcal$ belongs to the image of $(\widehat{p_\Dcal})^R$, as claimed. In particular, $\mu$ is an isomorphism when restricted to $\Dcal$.

The fact that $F$ and $\Ind(F)$ are compact implies that $(\widehat{F})^R$ and $(\widehat{\Ind(F)})^R$ preserve large colimits. It follows that the source and target of $\mu$ are colimit preserving, and hence $\mu$ is an isomorphism when restricted to the colimit closure of $\Dcal$. The lemma now follows from the fact that $\widehat{\Ind(\Dcal)}$ is generated under large colimits by $\Dcal$.
\end{proof}

\begin{proof}[Proof of theorem \ref{theorem compacts in filtered colimit}]
Denote by $g_{\alpha, \beta}: \Ccal_\alpha \rightarrow \ccal_\beta$ the transition functors. For each $\alpha$ denote by $p_\alpha: \Ind(\ccal_\alpha) \rightarrow \ccal_\alpha$ the projection, and by $i_\alpha$ its (fully faithful) left adjoint.

For each transition we have a commutative square
\[
\begin{tikzcd}[column sep = large]
\Ind(\ccal_\alpha) \arrow{d}{p_\alpha} \arrow{r}{\Ind(g_{\alpha, \beta})} & \Ind(\ccal_\beta) \arrow{d}{p_\beta}\\
\ccal_\alpha \arrow{r}{g_{\alpha, \beta}} & \ccal_\beta
\end{tikzcd}
\]
which is vertically left adjointable by lemma \ref{lemma vertically left adj}. Passing to the colimit in $\catomega$ we conclude that the induced map $p: \colim \Ind(\Ccal_\alpha) \rightarrow \ccal$ admits a fully faithful left adjoint, which is obtained as the colimit of the maps $i_\alpha$. We denote this left adjoint by $i$. 

It follows from lemma \ref{lemma filtered colimit of cg} that $\colim \Ind(\Ccal_\alpha)$ is compactly generated, and in particular it is compactly assembled. Since $\Ccal$ is a retract of it we conclude that $\Ccal$ is compactly assembled as well.

Passing to compact objects we obtain a commutative square
\[
\begin{tikzcd}
\colim (\Ind(\Ccal_\alpha)^\omega) \arrow{r}{} & (\colim \Ind(\Ccal_\alpha))^\omega \\
\colim \ccal_\alpha^\omega \arrow{u}{\colim {i_\alpha}} \arrow{r}{} & \ccal^\omega \arrow{u}{i} 
\end{tikzcd}
\]
where the colimits on the left column take place in $\cathat$ and the colimit on the top right corner takes place in $\catomega$. Here the vertical arrows are fully faithful, and the top horizontal arrow is an equivalence by lemma \ref{lemma filtered colimit of cg}. Hence the bottom horizontal arrow is fully faithful.

It only remains to show surjectivity. Let $X$ be a compact object in $\Ccal$. Since the top horizontal arrow in the above commutative square is an equivalence, there exists an $\alpha$ and a compact object $Y_\alpha$ in $\Ind(\ccal_\alpha)$ whose image in $\colim \Ind(\Ccal_\alpha)$ is given by $i(X)$. Let $\epsilon_\alpha: i_\alpha p_\alpha(Y_\alpha) \rightarrow Y_\alpha$ be the counit of the adjunction.

The image of $\epsilon_\alpha$ in $\colim \Ind(\ccal_\alpha)$ is given by the counit map $\epsilon: ipi(X) \rightarrow i(X)$, which is an isomorphism. Let $\nu: i(X) \rightarrow ipi(X)$ be an inverse to $\epsilon$. The compactness of $Y_\alpha$ implies that there exists a transition map $g_{\alpha, \beta}$ and a map $\nu_\beta: \Ind(g_{\alpha, \beta})(Y_\alpha) \rightarrow i_\beta p_\beta(\Ind(g_{\alpha, \beta})(Y_\alpha))$ whose image in $\colim(\Ind(\ccal_\alpha))$ is $\nu$. Replacing $\alpha$ with $\beta$ we may in fact assume that $\alpha = \beta$, so that $\nu_\alpha: Y_\alpha \rightarrow i_\alpha p_\alpha(Y_\alpha)$ lifts $\nu$.

The image of $\epsilon_\alpha \nu_\alpha$ in $\colim \Ind(\ccal_\alpha)$ is the identity on $i(X)$. As before, since $Y_\alpha$ is compact we may assume, after replacing $\alpha$ with some other index $\beta$ if necessary, that $\epsilon_\alpha \nu_\alpha$ is the identity on $Y_\alpha$. In other words, $Y_\alpha$ is a retract of $i_\alpha p_\alpha(Y_\alpha)$. Since $\ccal_\alpha$ is idempotent complete, we see that $Y_\alpha$ belongs to the image of $i_\alpha$. Write $Y_\alpha = i_\alpha(X_\alpha)$. Then $X_\alpha$ is a compact object in $\ccal_\alpha$ whose image in $\ccal$ is given by $X$.
\end{proof}

We also have a variant of theorem \ref{theorem compacts in filtered colimit} that applies to $n$-strong compactness for each $1 \leq n \leq \infty$.

\begin{notation}
For each object $\ccal$ in  $\cathat^\Sigma_n$ we denote by $\Ccal^\Sigma$ the full subcategory of $\Ccal$ on the $n$-strongly compact objects.
\end{notation}

\begin{theorem}\label{theorem lift strongly compacts}
Let $(\Ccal_\alpha)$ be a filtered diagram in $\cathat^\Sigma_n$ with $n$-strongly compact transition maps, and let $\Ccal$ be its colimit. Assume that $\Ccal_\alpha$ is $n$-strongly compactly assembled for all $\alpha$. Then $\Ccal$ is $n$-strongly compactly assembled and the induced functor $\colim \Ccal_\alpha^\Sigma \rightarrow \Ccal^\Sigma$ is an equivalence.
\end{theorem}

The proof of theorem \ref{theorem lift strongly compacts} is completely analogous to that of theorem \ref{theorem compacts in filtered colimit}.

%%%%%%%%%%%%%%%%%%%%%%%%%%%%%%%%%%%%%%%%%%%%%%%%%%%%%%%%%%%%%%%%%%%%%%%%
%%%%%%%%%%%%%%%%%%%%%%%%%%%%%%%%%%%%%%%%%%%%%%%%%%%%%%%%%%%%%%%%%%%%%%%%
%%%%%%%%%%%%%%%%%%%%%%%%%%%%%%%%%%%%%%%%%%%%%%%%%%%%%%%%%%%%%%%%%%%%%%%%
%%%%%%%%%%%%%%%%%%%%%%%%%%%%%%%%%%%%%%%%%%%%%%%%%%%%%%%%%%%%%%%%%%%%%%%%
%%%%%%%%%%%%%%%%%%%%%%%%%%%%%%%%%%%%%%%%%%%%%%%%%%%%%%%%%%%%%%%%%%%%%%%%
%%%%%%%%%%%%%%%%%%%%%%%%%%%%%%%%%%%%%%%%%%%%%%%%%%%%%%%%%%%%%%%%%%%%%%%%

\ifx\inmain\undefined
\bibliographystyle{myamsalpha}
\bibliography{References}
\fi

\end{document}
