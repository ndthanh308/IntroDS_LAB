\documentclass[12pt]{amsart} 
\usepackage[utf8]{inputenc}
\usepackage[T1]{fontenc}
%\usepackage{lipsum}% for line spread
%\linespread{1.2} %line spacing
\usepackage{lmodern}
\usepackage{amsmath, amsthm, amssymb, amsfonts}
\usepackage{latexsym}
\usepackage{amsxtra} %for sptilde
\usepackage[all]{xy}
\usepackage{nicefrac,mathtools}
\usepackage[shortlabels, inline]{enumitem}
\usepackage{microtype}
\usepackage{hyperref}
\usepackage{graphicx,calc}%for importing images and pdf
\newlength\myheight%pdf embedd
\newlength\mydepth%for pdf embedd
\settototalheight\myheight{Xygp}
\settodepth\mydepth{Xygp}
\usepackage[all]{xy}
\usepackage{xcolor}
\usepackage{tikz-cd}
\usepackage{tikz}
\usepackage[lite]{amsrefs}
\usepackage[T1]{fontenc}
\usepackage[utf8]{inputenc}
\usepackage{thmtools}
%\usepackage{refcheck}%to check unused labels.see the log file
% \usepackage[displaymath,mathlines=]{lineno}
% \linenumbers
% \end{document}
% \renewcommand\linenumberfont{\normalfont\bfseries\small}
%%%%%%%%%%%%%%%%%%%%%%%%%%%%%%%%%%%%%%%%%%%%%%%%
\usepackage{geometry}
\geometry{left=3cm,right=3cm,top=2.5cm,bottom=2.5cm}
%%%%%%%%%%%%%%%%%%%%%%%%%%%%%%%%%%%%%%%%%%%%

\numberwithin{equation}{section}
\theoremstyle{plain}
\newtheorem{theorem}[equation]{Theorem}

\newtheorem{Lemma}[equation]{Lemma}
\newtheorem{lemdef}[equation]{Lemma and definition}
\newtheorem{proposition}[equation]{Proposition}
\newtheorem{propdef}[equation]{Proposition and Definition}
\newtheorem{deflem}[equation]{Definition and Lemma}
\newtheorem{corollary}[equation]{Corollary}
\newtheorem{claim}[equation]{Claim}


\theoremstyle{definition}
\newtheorem{definition}[equation]{Definition}
\newtheorem{notation}[equation]{Notation}

\theoremstyle{remark} \newtheorem{remark}[equation]{Remark}
\theoremstyle{remark} \newtheorem{observation}[equation]{Observation}
\newtheorem{remarks}[equation]{Remarks} \newtheorem*{remark*}{Remark}
\newtheorem*{remarks*}{Remark} \newtheorem{example}[equation]{Example}
\newtheorem{obs}[equation]{Observation}
\newtheorem{ills}[equation]{Illustration}


\newcommand{\Cost}{\text{Cost}}
\newcommand{\R}{\mathbb{R}}
%\newcommand{\S1}{\mathbb{S}^1}
\newcommand{\Z}{\mathbb{Z}}
\newcommand{\RP}{\mathbb{R}^+}
\newcommand{\N}{\mathbb{N}}
\newcommand{\Q}{\mathbb{Q}}
\newcommand{\homeo}{\approx}
\newcommand{\iso}{\simeq}
\DeclarePairedDelimiter{\abs}{\lvert}{\rvert}% absolute value

%other shortforms
\newcommand*{\nb}{\nobreakdash}
\newcommand{\recheck}[1]{\textcolor{red}{#1}}
\newcommand{\comment}[1]{\marginpar{\small{#1}}}
\newcommand*{\defeq}{\mathrel{\vcentcolon=}}
\newcommand{\embeddpdf}[3]{\raisebox{#1\mydepth}{% Figure removed}}

\title[Cost Function]{On The Cost Function associated with Legendrian Knots}
\author{Dheeraj Kulkarni}\email{dheeraj@iiserb.ac.in}\address{Department of Mathematics, Indian Institute of Science Education and Research, Bhopal, India}
\author{Tanushree Shah} \email{tanushree@renyi.hu}\address{Alfr\'ed R\'enyi Institute, Budapest, Hungary}
\author{Monika Yadav} \email{monika18@iiserb.ac.in}\address{Department of Mathematics, Indian Institute of Science Education and Research, Bhopal, India}
%\author{}
%\email{monika18@iiserb.ac.in}

%\address{Department
	%of Mathematics, Indian Institute of Science Education and Research
	%Bhopal, Bhopal Bypass Road, Bhauri, Bhopal 462 066, Madhya Pradesh,
	%India.}

\keywords{Knot theory, Contact topology, Legendrian knots, connect sum } \thanks{\emph{Subjclass[2020]}: 57K10, 57K14, 57K33 }

\begin{document}
\begin{abstract} In this article, we introduce a non-negative integer-valued function that measures 
the obstruction for converting topological isotopy between two Legendrian knots into a Legendrian 
isotopy. We refer to this function as Cost function. We show that the Cost function induces a metric on the set of topologically isotopic Legendrian knots. Hence, the set of topologically isotopic 
Legendrian knots can be seen as a graph with path-metric given by the Cost function. Legendrian simple 
knot types are shown to be characterized using the Cost function. We also get a quantitative version of Fuchs-Tabachnikov's Theorem that says any two Legendrian knots in $(\mathbb{S}^3,\xi_{std})$ in the same topological knot type become Legendrian isotopic after sufficiently many stabilizations \cite{FT}. 
We compute the Cost function for torus knots (which are Legendrian simple) and some twist knots (which are Legendrian non-simple). We investigate the behavior of the Cost function under the 
connect sum operation. We conclude with some questions about the Cost function, its relation with the standard contact structure and the topological knot type.
\end{abstract}


\maketitle
% \setcounter{tocdepth}{1}
\tableofcontents

\section{Introduction}

A \emph{Legendrian} knot in $\R^3 $ with the standard contact structure $\xi_{st} = \text{ker}(dz-ydx) $ is a smooth
embedding of $\mathbb{S}^1 $ in $\R^3 $ such that its image is tangent to the contact plane at each point. Two
Legendrian knots are called \emph{Legendrian} isotopic if there is a smooth 1-parameter family of Legendrian knots starting with one and ending with the other. 
The classification of Legendrian knots up to Legendrian isotopy is not completely understood. It is easy to see that given a topological embedding of $\mathbb{S}^1$ in $\R^3 $, denoted by $K$, can be approximated by a Legendrian knot in an arbitrarily small $C^0 $-neighborhood of $K$.
Hence, there are infinitely many Legendrian knots approximating $K$. All of them are topologically isotopic to $K$. We refer to a Legendrian knot that approximates $ K$ as a \emph{Legendrian representative} of the knot type of $K$.

The properties of the contact structure $\xi_{st} $ reflect deeply in the classification problem of Legendrian representatives of a given knot. Sometimes, the classical invariants given by the Thurston-Bennequin number ($tb$) and the rotation number ($rot$) completely classify Legendrian representatives upto Legendrian isotopy.
On the other hand, there are knot types where classical invariants are not sufficient to tell the Legendrian representatives apart (see \cite{C1},\cite{C2}). Deep mathematical ideas based on counting J-holomorphic curves and packaging the information into differential graded algebras (DGAs) were developed by Eliashberg-Chekanov (\cite{EF}) and Ng (\cite{LNg}) to classify Legendrian representatives. 

In contrast, there have not been many attempts to understand the \emph{space} of all Legendrian representatives of a given knot type. Whether the set of all Legendrian representatives of a given knot type admits a meaningful and natural structure is the question in focus. Mountain ranges formed by Legendrian representatives based on $tb $ and $rot $ have been useful in a limited manner (\cite{E}). The limitation of mountain ranges arises from the fact that $tb$ and $rot$ are not sufficient for many families of knot types.

In this article, we introduce a function that assigns a non-negative integer to a pair of Legendrian representatives of a given knot type. Roughly speaking, this function measures the obstruction for converting a topological isotopy between two Legendrian representatives into a Legendrian isotopy. The output of this function, an integer, tells us the minimum cost we have to pay to upgrade a topological isotopy into a Legendrian isotopy. Hence we refer it as the `Cost function'. We show that the Cost function gives a metric (see Section \ref{Costgraph}) on the set Legendrian representatives of a given knot type turning it into a metric space. Therefore, we get a graph of Legendrian representatives where path metric is given by the Cost function.

We investigate the properties of the Cost function. We give bounds on the Cost function in terms of $tb$
and $rot$ (Lemma \ref{CostL9}). We obtain a characterization of Legendrian simple knot types in terms of an explicit formula for the Cost function. We give a quantitative version of Fuchs-Tabachnikov's Theorem (\cite{FT}). We study the effect of Legendrian connect-sum operation on the Cost function (\cite{C}). We give examples of Legendrian twist knots and torus knots together with values of Cost function. Finally, we raise a few questions for further understanding of the behavior of the Cost function from theoretical as well as computational viewpoints. \\

\noindent {\bf Acknowledgements:} The second author was supported by "Singularities and Low Dimensional Topology" semester at Alfr\'ed R\'enyi Institute, Hungary.

The third author is supported by the CSIR grant 09/1020(0152)/2019-EMR-I, DST,
Government of India.
\section{Preliminaries} 

A \textit{topological knot} $K$ in $\mathbb{R}^3$ is a smooth embedding of $\mathbb{S}^1$ in it. A \textit{regular projection} of $K$ is its projection onto a plane such that all the double points intersect transversely and a maximum of two points on $K$ project to the same point. A regular projection of $K$ with the information of undercrossing and overcrossing is called a \textit{knot diagram}.

\begin{definition}
    Knots $K_1$ and $K_2$ are said to be ambient isotopic if there exists a smooth map $H:\mathbb{R}^3\times[0,1]\rightarrow \mathbb{R}^3$ such that $F(\cdot,0)$ is $Id_{\mathbb{R}^3}$ and $F(\cdot,1)$ maps $K_1$ to $K_2$
\end{definition}

\begin{theorem}[Reidemeister,\cite{R}]
    Let $K_1$ and $K_2$ be knots. Let $D_1$ and $D_2$ be their knot diagrams, respectively. Then $K_1$ is ambient isotopic to $K_2$ if $D_1$ can be converted in $D_2$ by a finite sequence of local moves shown in Figure \ref{fig:Rmoves} and rotations of diagrams.
\end{theorem}

% Figure environment removed 

   \begin{definition}
       A contact manifold is a pair $(M,\xi)$, where $M$ is an odd-dimensional smooth manifold and $\xi$ is a maximally non-integrable hyperplane field on $M$. We call $\xi$ a contact structure on $M$.
   \end{definition}

Locally, $\xi$ can be seen as kernel of some one form $\alpha$ on $M$ satisfying $\alpha\wedge(d\alpha)^n\neq 0$. Throughout this paper, we will be working with the contact manifold $(\mathbb{R}^3,\xi_{st})$, where $\xi_{st}:=ker(dz-ydx)$. 

\begin{definition}
    A Legendrian knot in $(\mathbb{R}^3,\xi_{st})$ is a smooth embedding $\gamma:\mathbb{S}^1\rightarrow \mathbb{R}^3$ such that $\gamma'(t)\in \xi_{st}(\gamma(t))\forall t \in \mathbb{S}^1.$ Such an embedding is called a Legendrian embedding.
\end{definition}

\begin{definition}
    A front projection of a Legendrian knot in $(\mathbb{R}^3,\xi_{st})$ is its projection onto the $xz$-plane.
\end{definition}

Let $\gamma(t)=(x(t),y(t),z(t))$ be a parametrization of a Legendrian knot $K$. If $x'(t_0)=0$ for some $t_0\in \mathbb{S}^1$, then $(x(t_0),z(t_0))$ in front projection is called a cusp. A Legendrian projection is said to be generic if it has isolated cusps. Due to the contact condition, we have $z'(t)-y(t)x'(t) \forall t \in \mathbb{S}^1$. Thus the $y$ coordinate of a point on the $K$ away from cusps can be recovered from the front projection by $y(t)=z'(t)/x'(t)$. In a neighborhood of a cusp, the knot can be perturbed to get semi-cubical cusps as ( refer \cite{E}) shown in Figure \ref{fig:CTF}. The $y$ coordinate at a cusp is zero. At a crossing $(x(t),z(t))$, the undercrossing is the one with a more positive slope since $y(t)=z'(t)/x'(t)$. Going forward, by a Legendrian knot we will be meaning a generic Legendrian knot.

% Figure environment removed

    A knot diagram $D$ can be converted into a front projection of a Legendrian knot in the following way:  replace all the vertical tangencies with cusps and whenever at a crossing, the slope of overcrossing is more positive than the slope of undercrossing modify it by adding two cusps as shown in Figure \ref{fig:CTF}.

    \begin{definition}
         Legendrian knots $K_1$ and $K_2$ in $(\mathbb{R}^3,\xi_{st})$ are said to be Legendrian isotopic if there exists a smooth map $F:\mathbb{S}^1\times [0,1]\rightarrow\mathbb{S}^1$ such that, $F(\cdot,t)$ is a Legendrian embedding for all $t\in [0,1] $, $F(\cdot,0)$ is $K_1$ and $F(\cdot,1)$ is $K_2$.
    \end{definition}

We have adopted a version of the Reidemeister theorem for Legendrian knots from \cite{SJ}.
    \begin{theorem}[\cite{SJ}]
        Let $K_1$ and $K_2$ be Legendrian knots. Then $K_1$ is Legendrian isotopic to $K_2$ if and only if their front projections are related by regular homotopy and a sequence of moves shown in Figure \ref{fig:LRmoves}.
     \end{theorem}
     % Figure environment removed


\begin{definition}

In an oriented link diagram a crossing of the type $\raisebox{-3\mydepth}{% Figure removed}$ is called a right-handed crossing and a crossing of the type $\raisebox{-3\mydepth}{% Figure removed}$ is called a left-handed crossing.
\end{definition}

\begin{definition}
    Let $D$ be an oriented link diagram. Let $RHC(D)$ denote the number of right-handed crossings in $D$ and let $LHC(D)$ denote the number of left-handed crossings in $D$. Then \textit{Writhe} of a knot diagram $D$ is defined as the difference $RHC(D)-LHC(D)$ and is denoted by $\omega(D)$.
\end{definition}
Next, we give a combinatorial definition of the Thurston-Bennequin number and rotation number involving front projections.
\begin{definition}
    Let $F$ be a front projection of a nullhomologous Legendrian knot $K$. Then the Thurston-Bennequin number of $K$ is defined as $$\omega(F)-\#\frac{cusps(F)}{2}.$$ It is denoted by $tb(K)$. 
\end{definition}


\begin{definition}
     Let $F$ be a front projection of an oriented Legendrian knot $K$. The rotation number of $K$ is defined as 
     $$\frac{D(F)-U(F)}{2},$$ where $D(F)$ and $U(F)$ denote the number of downward and upward cusps in $F$, respectively. We denote the rotation number of $K$ by $rot(K)$.
\end{definition} 

An oriented arc in a front projection $F$ can be replaced by an arc with two downwards (upwards) cusps as shown in Figure \ref{fig:stabilization}. This process is called a \textit{positive (negative) stabilization} of $F$. A \textit{positive (negative) stabilization} of an oriented knot $K$ is given by an oriented Legendrian knot whose front projection is obtained by a positive (negative) stabilization on a front projection of $K$. We use the notation $S^{\pm}(K)$ and $S^{\pm}(F)$ to denote a positive (negative) stabilization of $K$ and $F$ respectively. A stabilization is independent of its locations in the front projection upto Legendrian isotopy. Adding a stabilization changes $tb$ by $-1$ and rotation number by $\pm 1$. 
 % Figure environment removed
    
A (\textit{Legendrian}) \textit{topological knot type} is an equivalence class of (Legendrian) topological knots upto (Legendrian) topological isotopy. For a topological knot type $\mathcal{K}$, we use the notation $\mathcal{L(K)}$ to denote the set of all Legendrian representatives of $\mathcal{K}$.

\begin{definition}
    Let $K$ be a topological knot. Then the knot type of $K$ is said to be Legendrian simple if all its Legendrian representatives are classified by the pair $(tb,\:rot)$ upto the Legendrian isotopy.
\end{definition}

\section{Connected Sum of Legendrian knots}
Let $K_1$ and $K_2$ be oriented Legendrian knots in oriented contact manifold $(\R^3,\xi_{st})$. Let $p_i$ be a point on $K_i$ and let $B_i$ be an open $3$-ball around $p_i$ such that $B_i\cap K_i$ is an unknotted arc $\alpha_i$ which lies on the $x$-axis, $i=1,2$. Choose $B_1$ and $B_2$ such that $B_1$ and $B_2$ are disjoint from $K_2$ and $K_1$ respectively. Let $\phi:\partial(\R^3-B_1)\rightarrow \partial(\R^3-B_2)$ be an orientation reversing diffeomorphism. Then the connected sum  $\R^3\#\R^3:=(\R^3-B_1)\cup_{\phi} (\R^3-B_2)$ is diffeomorphic to $\R^3$. The contact structure induced on $(\R^3-B_1)\cup_{\phi} (\R^3-B_2)$ is tight and the connected sum is independent of the points $p_1,p_2$, the balls $B_1, B_2$ and the map $\phi$, \cite{C}.


    Under the connected sum operation of contact manifolds done above, the Legendrian knots $K_1$ and $K_2$ will be mapped to another Legendrian knot $(K_1-\alpha_1)\cup (K_2-\alpha_2)$. One can choose $\phi$ such that it maps $\partial(B_1\cap K_1)$ to $\partial(B_2\cap K_2)$ such that there is a coherent orientation on $(K_1-\alpha_1)\cup (K_2-\alpha_2).$ Define $K_1\#K_2$ to be the Legendrian knot $(K_1-\alpha_1)\cup (K_2-\alpha_2).$

     This connected sum operation on Legendrian knots can be seen using the front projections as shown in Figure \ref{fig:LCS}.
% Figure environment removed


\begin{Lemma}[\cite{EH}]
    The connected sum $K_1\#K_2$ is independent of $p_1,p_2,B_1$ and $B_2$.
\end{Lemma}

\begin{remark}
    It follows from a simple observation that if $K_1\#K_2$ is nullhomologous then $tb(K_1\#K_2)=tb(K_1)+tb(K_2)+1$ and $rot(K_1\#K_2)=rot(K_1)+rot(K_2)$, \cite{EH}.
\end{remark}

\begin{remark}
    Legendrian prime decomposition of a Legendrian knot is unique upto moving stabilizations from one component to another and possible permutations of components (\cite{EH}). 
\end{remark}

\begin{remark}
    Section 4 onward all the knots are oriented and a (Legendrian) topological knot type is considered upto the (Legendrian) isotopy of oriented (Legendrian) topological knots.
\end{remark}
\section{Cost Function}

Fuchs-Tabachnikov \cite{FT} proved that any two Legendrian knots that are topologically isotopic can be stabilized sufficiently many times to obtain Legendrian isotopic knots. In the light of this theorem, we define the cost function which measures the obstruction for the two Legendrian knots in the same topological type from becoming Legendrian isotopic. First, we will show that any R move between two front projections can be converted into an LR move by stabilizing the fronts. We start with two front projections $F_0$ and $F_1$ which differ in a disc by an R move and are identical outside the disc, then we add the minimum number of cusps required to get a Legendrian isotopy supported in the disc. 



\begin{theorem}\label{CostL1}
    Let $F_0$ and $F_1$ be front projections that differ by an R move or a local rotation in a disc $Q$ and are identical outside $Q$. Then $F_0$ and $F_1$ can be stabilized sufficiently within $Q$ to obtain Legendrian isotopic knots and this Legendrian isotopy is supported in $Q$.
\end{theorem}
\begin{proof}
   We consider each R move and local rotations one by one. 
   
   \noindent  \textit{(i) Conversion of R1 move into a sequence of LR moves.}
   
   Let $F_0$ and $F_1$ be connected by an R1 move in a disc $Q$ and are identical outside $Q$. Let $x$ be the new crossing in $F_1$ created by the R1 move. Then $x$ is either a left-handed crossing or a right-handed crossing. Let us assume that $x$ is a left-handed crossing, the case of a right-handed crossing is similar to this one. Figure \ref{fig:CostLR1} shows two examples of such front projections $F_0$ and $F_1$ in the disc $Q$. All the other possible cases are obtained from these by stabilizing arcs $\alpha$, $\beta$, and $\gamma$.
    % Figure environment removed
   Since stabilization can be moved across crossings without changing the Legendrian knot type, it is enough to discuss cases shown in Figure \ref{fig:CostLR11} and Figure \ref{fig:CostLR12}. For the case shown in Figure \ref{fig:CostLR11}, then the front projection $F$ obtained by adding two cusps to $F_0$ is connected to $F_1$ by an LR2 move followed by an LR1 move. This Legendrian isotopy connecting $F$ and $F_1$ is supported within $Q.$ In the second case when $F_0$ and $F_1$ are connected by R1 move as shown in Figure \ref{fig:CostLR12}, the Legendrian isotopy supported within $Q$ is obtained by stabilizing $F_0$ three times. 
    % Figure environment removed
    
    % Figure environment removed

    
 \noindent  \textit{(ii) Conversion of R2 move into a sequence of LR moves.}
 
 Consider front projections $F_0$ and $F_1$ that are connected by an R2 move in a disc $Q$ and are identical outside $Q$. We consider two cases as shown in Figure \ref{fig:CostLR21} and Figure \ref{fig:CostLR22}. All possible front projections $F_0$ and $F_1$ that are connected by a horizontal R2 move can be obtained by stabilizing $F_0$ and $F_1$ shown in Figure \ref{fig:CostLR22} and the cases for the vertical R2 move can be exhausted by stabilizing the fronts shown in Figure \ref{fig:CostLR21}. Both of these figures prove the existence of Legendrian isotopies between $F_1$ and $F$ supported within $Q$.
    
% Figure environment removed

    % Figure environment removed

 \noindent  \textit{(iii) Conversion of R3 move into a sequence of LR moves.}
 
Let $F_0$ and $F_1$ be front projections that are connected by R3 move in $Q$ and are identical outside $Q$. One can construct all possible front projections for $F_0$ and $F_1$ inside $Q$ in the following steps: (1) consider any two topological knot diagrams connected by R3 move inside a disc $Q$, (2) convert these diagrams into front projections inside $Q$ by adding minimum cusps, (3) on the boundary of $Q$, keep the slope of the corresponding strands same during the conversion of knot diagram to a front projection, (4) rest of the possible cases of such front projections $F_0$ and $F_1$ are obtained by stabilizing the front projections obtained by last three steps. Figure \ref{fig:CostLR31} and Figure \ref{fig:CostLR32} illustrate two such cases of $F_0$ and $F_1$. The front projections $F_0$ and $F_1$ shown in Figure \ref{fig:CostLR31} are connected by a sequence of LR moves within the disc. Also, it can be checked that there exists a Legendrian isotopy supported within $Q$ connecting $F_0$ and $F_1$ shown in Figure \ref{fig:CostLR32}.

 \noindent  \textit{(iv) Conversion of a local rotation into a sequence of LR moves.}
 
 Now consider the case when front projections $F_0$ and $F_1$ are connected by a rotation in a disc $Q$ and identical outside $Q$. All front projections connected by such local planar isotopy can be obtained by stabilizing the front projections shown in Figure \ref{fig:CostP}. The bottom two front projections shown in Figure \ref{fig:CostP} are connected by a sequence of LR2 moves within the disc.
% Figure environment removed

    % Figure environment removed
    
% Figure environment removed

Hence, $F_0$ and $F_1$ can be stabilized sufficiently within $Q$ to obtain Legendrian isotopic knots and this Legendrian isotopy is supported in $Q$.
    
\end{proof}

\begin{remark}
    Note that the two front projections that are connected by a global rotation of knot diagrams can be seen as a composition of planar rotation near crossings and translations of the diagram as illustrated in Figure \ref{fig:costglob1} and Figure \ref{fig:costglob2}.
    % Figure environment removed
   
   % Figure environment removed
\end{remark}

\begin{remark}
    Let $K_0$ and $K_1$ be Legendrian knots with front projections $F_0$ and $F_1$ respectively. The notations $K_0=K_1$ and $F_0=F_1$ will be used to denote that $K_0$ and $K_1$ are of the same Legendrian knot type.
\end{remark}
Let $S^{\pm}_{Q}(F_0)$ denote the front projection obtained by positively/negatively stabilizing $F_0$ within the disc $Q$. Then we give the following definition for the cost of two front projections in the same topological knot type.

\begin{definition}
    Let $F_0$ and $F_1$ be front projections that differ by an R move within a disc $Q$ and are identical outside $Q$. Then define the cost function $\Cost(F_0,F_1)$ as
    \begin{align*}
        \min  \{m_0+n_0+m_1+n_1:  (S^+_{Q})^{m_0}(S^-_{Q})^{n_0}(F_0)
       =(S^+_{Q})^{m_1}(S^-_{Q})^{n_1}(F_1)\}.
    \end{align*}
        
\end{definition}

\begin{Lemma}\label{CostL2}
    Let $F$ and $\Tilde{F}$ be front projections of Legendrian knots of the same topological knot type. Then there exist a finite sequence $SF=\left(F=F_0,F_1,\cdots,F_n,F_{n+1}=\Tilde{F}\right)$ of front projections and discs $\Q_i$'s such that $F_i$ and $F_{i+1}$ are identical outside $Q_i$ and are connected by an R move inside $Q_i\: \forall 0\leq i \leq n,$ for some positive integer $n$.
\end{Lemma}
\begin{proof}
    There exists a sequence of R moves and local rotations connecting $F$ to $\Tilde{F}$, since $F$ and $\Tilde{F}$ are in the same topological knot type. Fix $\mathcal{S}$, a sequence of R moves connecting $F$ to $\Tilde{F}$. 

 Let $\mathcal{S}$ consists of $n$ R moves and let $D_0=F, D_1,\cdots,D_n=\Tilde{F}$ be the sequence of topological knot diagrams obtained by applying $\mathcal{S}$ on $F$. The knot diagram $D_0$ is identical to $F$. Let $Q_i$ be the disc where $D_i$  and $D_{i+1}$ differ by an R move and are identical outside $Q_i\: \forall 0\leq i\leq n-1$. 

Using $D_0,D_1,\cdots,D_n$, construct a sequence $F_0,F_1,\cdots,F_n,F_{n+1}=\Tilde{F}$ such that $F_{i}$ and $F_{i+1}$ are identical outside disc $Q_{i}$ and identical outside $Q_{i}\: \forall 0\leq i \le n$. Construct $F_1$, from $D_0$ and $D_1$ as follows: Fix $F_1$ to be identical to $D_0$ outside $Q_0$. Inside $Q_0$, fix $F_1$ to be a front projection obtained by converting $D_1$ into a front projection without changing the topological knot type as discussed earlier. Note that inside $Q_1$, we are making a choice for $F_1$ as there are many ways to convert a topological knot diagram into a front projection. The front projection $F_1$ obtained by this process is identical to $F_0$ outside $Q_0$ and connected by an R move inside $Q_0$. Repeat the process inside and outside the disc $Q_1$ to construct $F_2$ from $F_1$ and $D_2$. Repeating this process $n$ times we get a desired sequence $F_0, F_1,\cdots, F_n$. The front projection $F_n$ is identical to $F_{n-1}$ outside $Q_{n-1}$ and differs from it by an R move inside $Q_{n-1}$. Thus, $F_n$ differs from $\Tilde{F}$ only in terms of stabilizations. After moving all the stabilizations in $F_n$ and $\Tilde{F}$ to a common disc $Q_{n+1}$, we can assume that $F_n$ and $\Tilde{F}$ are identical outside $Q_{n+1}$ and differ by some stabilizations in $Q_{n+1}$. Hence, we have the desired sequence $F_0,F_1,\cdots,F_n,F_{n+1}=\Tilde{F}$ of front projections. 
   
\end{proof}

Let $\mathcal{SF}(F,\Tilde{F})$ denote the set of all the sequences of front projections connecting $F$ and $\Tilde{F}$ with the properties mentioned in Lemma \ref{CostL2}. Then we give the following definition: 
\begin{definition}
 Let $F$ and $\Tilde{F}$ be front projections of Legendrian knots of the same topological knot type. Let $SF=\left(F=F_0,F_1,\cdots,F_{n+1}=\Tilde{F}\right)$ be an element from $ \mathcal{SF}(F,\Tilde{F})$. Define

 \begin{align*}
     \Cost(F,\Tilde{F},SF)&:=\sum_{i=0}^{n}\Cost(F_{i},F_{i+1}), \quad \text{and}\\
     \Cost(F,\Tilde{F}):=\min\{&\Cost(F,\Tilde{F},SF):SF\in \mathcal{SF}(F,\Tilde{F})\}.
 \end{align*}
\end{definition}
\begin{remark}\label{re:Cost1}
    From the definition it can be seen that the Cost function is symmetric, that is, $\Cost(F,\Tilde{F})=\Cost(\Tilde{F},F)$ for all front projections $F$ and $\Tilde{F}$ in the same topological knot type.
\end{remark}

%\begin{remark}\label{CostRe1}
 %  Since minimum is taken over a discrete set bounded below by zero, $\Cost(F,\Tilde{F})$ is attained.
%\end{remark}

\begin{Lemma}\label{CostL3}
   Let $F$ and $\Tilde{F}$ be front projections of Legendrian knots of the same topological knot type. Let $F_1$ and $\Tilde{F}_1$ be another front projections Legendrian isotopic to $F$ and $\Tilde{F}$, respectively. Then $\Cost(F,\Tilde{F})=\Cost(F_1,\Tilde{F_1})$.
\end{Lemma}
\begin{proof}
    Let $S$ and $\Tilde{S}$ sequences of LR moves connecting $F_1$ to $F$ and $\Tilde{F}$ to $\Tilde{F}_1$, respectively. Then $S\in \mathcal{SF}(F_1, F)$ and $S_1\in\mathcal{SF}(\Tilde{F},\Tilde{F}_1)$. Let $SF$ be an arbitrary element of $ \mathcal{SF}(F,\Tilde{F})$. Let $X=\left(S_1,SF,S_2\right)$, then $X\in\mathcal{SF}(F_1,\Tilde{F}_1)$ and $\Cost(F_1,\Tilde{F}_1,X)=\Cost(F,\Tilde{F},SF)$. Thus we have $\Cost(F_1,\Tilde{F}_1)\leq \Cost(F,\Tilde{F},SF) \: \forall SF\in \mathcal{SF}(F,\Tilde{F})$. Hence, $\Cost(F_1,\Tilde{F}_1)\leq \Cost(F,\Tilde{F})$.

    Similarly, it can be shown that $\Cost(F,\Tilde{F})\leq \Cost(F_1,\Tilde{F}_1)$. Hence , $\Cost(F,\Tilde{F})=\Cost(F_1,\Tilde{F}_1)$.
\end{proof}

\begin{definition}
    Let $K$ and $\Tilde{K}$ be Legendrian knots of the same topological knot type. Define $\Cost(K,\Tilde{K}):= \Cost(F,\Tilde{F}),$ where $F$ and $\Tilde{F}$ are front projections of $K$ and $\Tilde{K}$, respectively.
\end{definition}


\begin{example}
Let $F_0$, $F_1$ and $F_2$ be the Legendrian unknots as shown in Figure \ref{CU}, then $\Cost(F_0,F_1)=1,$ $\Cost(F_1,F_2)=1$ and $\Cost(F_0,F_2)=2.$
    
    % Figure environment removed
\end{example}

\begin{theorem}\label{CostL4}
Cost$(K,\Tilde{K})=0$ if and only if $K$ is Legendrian isotopic to $\Tilde{K}$.
\end{theorem}
\begin{proof}
     Let $F$ and $\Tilde{F}$ denote the front projections of  $K$ and $\Tilde{K}$, respectively. If $K$ and $\Tilde{K}$ are Legendrian isotopic then there exists a sequence $SF_0$ of LR moves connecting $F$ to $\Tilde{F}$. Therefore, $\Cost(F,\Tilde{F})\leq \Cost(F,\Tilde{F},SF_0)=0$. Hence, $\Cost(K,\Tilde{K})=0$.

     Now, assume $\Cost(K,\Tilde{K})=0$. Then there exists a sequence $SF_0\in \mathcal{SF}(F,\Tilde{F})$ given by $(F=F_0,F_1,\cdots,F_n,F_{n+1}=\Tilde{F})$ such that the $\Cost(F,\Tilde{F},SF_0)=0$. Since, $\Cost(\cdot,\cdot)$ is a non-negative integer-valued function, we get $\Cost(F_{i},F_{i+1})=0 \: \forall 0\leq i\leq n$. Thus $F_i$ is Legendrian isotopic to $F_{i+1}$ $\forall 0\leq i\leq n$. Hence, $F$ is Legendrian isotopic to $\Tilde{F}.$ 
\end{proof}

\begin{Lemma}\label{CostL5}
     Let $F$ and $\Tilde{F}$ be front projections of the same topological knot type. Then there exist positive integers $p,q$ such that  $\Cost(F,\Tilde{F})=p+q$ and $(S)^p(F)=(S)^q(\Tilde{F})$. 
\end{Lemma}   
\begin{proof}
  Let $\Cost(F,\Tilde{F})=\Cost(F,\Tilde{F},SF_0)$ for some $SF_0\in \mathcal{SF}(F,\Tilde{F})$. Let $SF_0$ be given by $\left(F=F_0,F_1,\cdots,F_n,F_{n+1}=\Tilde{F}\right)$. Let $Q_{i}$ be the disc such that $F_i$ and $F_{i+1}$ differ by an R move or local planar isotopy within $Q_i$, $ 0\leq i \leq n$. Then there exist positive integers $p_i,q_i$ such that $\Cost(F_i,F_{i+1})=p_{i}+q_{i}$ and $S_{Q_i}^{p_{i}}(F_i)=S_{Q_i}^{q_{i}}(F_{i+1})$ for all $ 0\leq i \leq n.$ Then $\Cost(F,\Tilde{F})=\sum_{i=0}^n(q_{i}+p_{i})$. Thus
  \begin{align*}
      S^{p_n}S^{p_{n-1}}\cdots S^{p_1}S^{p_{0}}(F)=&S^{p_{n}}S^{p_{n-1}}\cdots S^{p_{1}} S^{q_0}(F_1)\\
      =&S^{q_0}S^{p_{n}}S^{p_{n-1}}\cdots S^{p_{1}}(F_1)\\
     = &S^{q_0}S^{p_{n}}S^{p_{n-1}}\cdots S^{q_{1}}(F_2)\\
     & \vdots\\
      =&S^{q_{0}}S^{q_1} \cdots S^{q_n}(\Tilde{F}).
  \end{align*}
  Then $p=\sum_{i=0}^np_i$ and $q=\sum_{i=0}^nq_i$ are the desired integers.
\end{proof}
\begin{remark}
    The above lemma gives a \emph{quantitative} version of the theorem of Fuchs and Tabachnikov.
\end{remark}
\begin{Lemma}\label{CostL6}
    Let $F$ and $\Tilde{F}$ be front projections of the same topological knot type. Let $p,n,\Tilde{p},\Tilde{n}$ be positive integers such that $(S^+)^p(S^-)^n(F)$ is Legendrian isotopic to  $(S^+)^{\Tilde{p}}(S^-)^{\Tilde{n}}(\Tilde{F})$. Then $\Cost(F,\Tilde{F})\leq p+n+\Tilde{p}+\Tilde{n}.$
\end{Lemma}
\begin{proof}
    Let $F_1$ be a front projection obtained by applying $p$ positive and $n$ negative stabilization on an arc $\alpha$ of $F$ in a disc $Q_0$. Let $F_2$  be a front projection obtained by applying $\Tilde{p}$ positive and $\Tilde{n}$ negative stabilization on an arc $\beta$ of $\Tilde{F}$ in a disc $Q_2$. Then $\Cost(F,F_1)=p+n$ and $\Cost(F_2,\Tilde{F})=\Tilde{p}+\Tilde{n}$ since $(S^+_{Q_0})^{p}(S^-_{Q_0})^n(F)= F_1$ and $F_2= (S^+_{Q_2})^{\Tilde{p}}(S^-_{Q_2})^{\Tilde{n}}(\Tilde{F})$. Since stabilization is independent of its location in a front projection $F_1$ is Legendrian isotopic to $(S^+)^p(S^-)^n(F)$ and $F_2$ is Legendrian isotopic to $(S^+)^{\Tilde{p}}(S^-)^{\Tilde{n}}(\Tilde{F})$ which implies that $F_1$ and $F_2$ are Legendrian isotopic. Let $X$ be a sequence of LR moves connecting $F_1$ and $F_2$.  Then $SF_0:=\left(F,F_1,X,F_2,\Tilde{F}\right) \in \mathcal{SF}(F,\Tilde{F})$ and 
    \begin{align*}
\Cost(F,\Tilde{F},SF_0)&=\Cost(F,F_1)+\Cost(F_2,\Tilde{F})+\Cost(F_1,F_2,X)\\
&=p+n+\Tilde{p}+\Tilde{n}+0.
    \end{align*}
    Hence, $\Cost(F,\Tilde{F})\leq p+n+\Tilde{p}+\Tilde{n}. $ 
\end{proof}


\begin{remark}
    From the Lemmas \ref{CostL5}, \ref{CostL6} it follows that $\Cost(F,\Tilde{F})=\min\{m+n:S^m(F)\text{ is Legendrian isotopic to }S^n(\Tilde{F})\}$.
\end{remark}


\begin{Lemma}\label{CostL7}
If $tb(K)=tb(\Tilde{K})$, then $\Cost(K,\Tilde{K})$ is even. 
\end{Lemma}
\begin{proof}
   Let $n$ and $\Tilde{n}$ be the minimum number of stabilizations required on $K$ and $\Tilde{K}$ to obtain Legendrian isotopic knots $K'$ and $\Tilde{K}'$ respectively. Then $tb(K)-n =tb(K')=tb(\Tilde{K}')=tb(\Tilde{K})-\Tilde{n}$. Hence, $n=\Tilde{n}$ and $\Cost(K,\Tilde{K})=2n.$
\end{proof}


\begin{Lemma}\label{CostL8}
    Let $K$ and $\Tilde{K}$ be Legendrian knots of the same topological knot type. Then $$\Cost(K,\Tilde{K})\geq \max\{\abs{tb(K)-tb(\Tilde{K})},\abs{rot(K)-rot(\Tilde{K})}\}.$$
\end{Lemma}
\begin{proof}
    Let $\Cost(K,\Tilde{K})=p+n+\Tilde{p}+\Tilde{n}$, where $(S^+)^{p}(S^-)^{n}(K)=(S^+)^{\Tilde{p}}(S^-)^{\Tilde{n}}(\Tilde{K})$. Then $tb(K)-p-n=tb(\Tilde{K})-\Tilde{p}-\Tilde{n}$ and $rot(K)+p-n=rot(\Tilde{K})+\Tilde{p}-\Tilde{n}$. Since $p,n,\Tilde{p},$ and $ \Tilde{n}$ are non-negative integers, $p+n+\Tilde{p}+\Tilde{n}\geq |(p+n)-(\Tilde{p}+\Tilde{n})|=\abs{tb(K)-tb(\Tilde{K})}$ and $p+n+\Tilde{p}+\Tilde{n}\geq |(\Tilde{p}+n)-(p+\Tilde{n})|=\abs{rot(K)-rot(\Tilde{K})}$. Hence, $$\Cost(K,\Tilde{K})\geq \max\{\abs{tb(K)-tb(\Tilde{K})},\abs{rot(K)-rot(\Tilde{K})}\}.$$

\end{proof}
\begin{Lemma}\label{CostL9}
    Let $\mathcal{K}$ be a topological knot type which is Legendrian simple. Let $K$ and $\Tilde{K}$ be Legendrian representatives of $\mathcal{K}$. Then 
$$\Cost(K,\Tilde{K})=\max\{\abs{tb(K)-tb(\Tilde{K})},\abs{rot(K)-rot(\Tilde{K})}\}.$$
\end{Lemma}
    
\begin{proof}
 Let $t$ and $\Tilde{t}$ be Thurston-Bennequin numbers of $K$ and $\Tilde{K}$, respectively. Let $r,\Tilde{r}$ be the rotation number invariants of $K$ and $\Tilde{K}$, respectively. Let $p=\abs{t-\Tilde{t}}$ and $m=\abs{r-\Tilde{r}}$.
Using Lemma \ref{CostL8}, we get $\Cost(K,\Tilde{K})\geq \{p,m\}$. Now we need to prove the reverse inequality. Note that
\begin{equation}\label{Costeq3}
    p+m=0 \text{ mod }2,
\end{equation} 
since $tb(K)+rot(K)=1 \text{mod }2$ for every Legendrian knot $K$. Without loss of generality, we can assume that $t\geq\Tilde{t}$ and $r\leq \Tilde{r}.$  Now we consider the following cases.

\noindent \textbf{Case 1:} $p=0$.

From Equation \ref{Costeq3}, we have $m=2n$ for some integer $n.$

\noindent \textbf{Subcase 1:} $m=0$.

Since $\mathcal{K}$ is a Legendrian simple knot type, $K$ is Legendrian isotopic to $\Tilde{K}$ and $\Cost(K,\Tilde{K})=0.$

\noindent \textbf{Subcase 2:} $m>0$.
 Let $K'$ and $\Tilde{K}'$ be the Legendrian knots obtained by $n$ positive stabilization on $K$ and $n$ negative stabilization on $\Tilde{K}$, respectively. Then $tb(K')=tb(\Tilde{K}')$ and $rot(K')=rot(\Tilde{K}')$. Therefore, $K'$ is Legendrian isotopic to $\Tilde{K}'$. Hence, $\Cost(K,\Tilde{K})\leq 2n=m$.

\noindent \textbf{Case 2:} $p\ne 0$.

\noindent \textbf{Subcase 1:} $m=0$.

Using Equation \ref{Costeq3}, we have $p=2d$ for some positive integer $d$. Let $\hat{K}$ be the Legendrian knot obtained by stabilizing $K$ positively and negatively $d$ times each. Then $tb(\hat{K})=\Tilde{t}, rot(\hat{K})=\Tilde{r}$. Thus $\hat{K}$ is Legendrian isotopic to $\Tilde{K}$ since $\mathcal{K}$ is a Legendrian simple knot type. Hence, $\Cost(K,\Tilde{K})\leq 2d=p$.



\noindent \textbf{Subcase 2:} $p=m$.

 Let $K_2$ be the Legendrian knot obtained after positively stabilizing $K$, $m$ times. Then $rot(K_2)=r+m=\Tilde{r}$ and $tb(K_2)=t-m=\Tilde{t}$. Hence, $K_2$ is Legendrian isotopic to $\Tilde{K}$ and $\Cost(K,\Tilde{K})\leq m$. 


\noindent \textbf{Subcase 3:} $p>m>0$. 

 Let $K_3$ be the Legendrian knot obtained by $m+\frac{p-m}{2}$ positive stabilizations and $\frac{p-m}{2}$ negative stabilizations on $K$. Then $tb(K_3)=t-p=\Tilde{t}$ and $rot(K_3)=r+m=\Tilde{r}$. Thus $K_3$ is Legendrian isotopic to $\Tilde{K}$. Hence, $\Cost(K,\Tilde{K})\leq p$. 


\textbf{Subcase 4:} $m>p>0$.

Let $K_4$ be the Legendrian knot obtained by $ p+\frac{m-p}{2}$ positive stabilizations on $K$. Let $K_5$ be the Legendrian knot obtained by $\frac{m-p}{2}$ negative stabilizations on $\Tilde{K}$. Then $tb(K_4)=t-p-\frac{m-p}{2}=\Tilde{t}-\frac{m-p}{2}=tb(K_5)$ and $rot(K_4)=r+p+\frac{m-p}{2}=\Tilde{r}-\frac{m-p}{2}=rot(K_5)$. Thus $K_4$ and $K_5$ are Legendrian isotopic. Hence, $\Cost(K,\Tilde{K})\leq p+\frac{m-p}{2}+\frac{m-p}{2}=m.$ 

\end{proof}



\section{Cost Associated With a Family of Legendrian Twist Knots}

In this section, we consider a family of oriented Legendrian knots $E_{k,l}$ whose front projection has $k\ge 1$ crossings on the left and $l\geq 1$ on the right as shown in Figure \ref{fig:ekl}. 

% Figure environment removed

\begin{Lemma}\label{CostL10}
    Let $k,l\:\geq1$ and $n\geq4$ be positive integers with $k+l=n$. Then $\Cost(E_{k,l},E_{k+1,l-1})=2$ if $k\neq l-1$ and $0$ otherwise.
\end{Lemma}
     
\begin{proof}
It is proved in \cite{NEV} that $E_{k,l}$ and $E_{k+1,l-1}$ are Legendrian isotopic if and only if $k=l-1$. Then $\Cost(E_{k,l},E_{k+1,l-1})=0 $, if $k=l-1$ and nonzero, otherwise. Below we have a sequence of LR moves on one-time stabilized $E_{k,l}$ and $E_{k+1,l-1}$. Hence, $\Cost(E_{k,l},E_{k+1,l-1})=2$.

    % Figure environment removed
\end{proof}

\section{Cost Under the Connected Sum}
In \cite{EH}, the examples of Legendrian non-simple knot types are produced by taking the connected sum of negative torus knots. Lemma \ref{CostL11}, \ref{CostL12}, \ref{CostL14}, \ref{CostL15} and Proposition \ref{CostL16}, \ref{CostL17} are the results inspired by the techniques used in \cite{EH}.

\begin{Lemma}\label{CostL11}
    Let $K'$ and $K''$ be prime topological knots with the property that every Legendrian representative of $K'$ destabilizes to a unique Legendrian knot $\Bar{K'}$ and every Legendrian representative of $K''$ destabilizes to a unique Legendrian knot $\Bar{K''}$. Then $K'\#K''$ is a Legendrian simple knot type and each Legendrian representative of $K'\#K''$ destabilizes to $\Bar{K'}\#\Bar{K''}$.
\end{Lemma}

\begin{proof}
    Let $K_1,K_2$ be Legendrian knots in $\mathcal{L}(K'\#K'')$ with the same $tb$ and $rot$. Then there exist $K_1', K_2'\in \mathcal{L}(K')$ and $K_1'', K_2'' \in \mathcal{L}(K'')$ such that $K_1=K_1'\#K_1''$ and $K_2=K_2'\#K_2''$. There exist positive integers $p_1,p_2,q_1,q_2,n_1,n_2,m_1$ and $m_2$ such that $K_i'=(S^+)^{p_i}(S^-)^{n_i}(\bar{K'})$ and $K_i''=(S^+)^{q_i}(S^-)^{m_i}(\bar{K''})$ for $i=1,2.$ Then 
 \begin{align*}
        K_1&=(S^+)^{p_1}(S^-)^{n_1}(\bar{K'})\#(S^+)^{q_1}(S^-)^{m_1}(\bar{K''})\\
        &=\Bar{K'}\#(S^+)^{p_1+q_1}(S^-)^{n_1+m_1}(\Bar{K''}) \quad \text{ and }\\
        K_2&=(S^+)^{p_2}(S^-)^{n_2}(\bar{K'})\#(S^+)^{q_2}(S^-)^{m_2}(\bar{K''})\\
        &=\Bar{K'}\#(S^+)^{p_2+q_2}(S^-)^{n_2+m_2}(\Bar{K''}).&&
    \end{align*}

We get $p_1+q_1+n_1+m_1=p_2+q_2+m_2+n_2$ and $p_1+q_1-n_1-m_1=p_2+q_2-n_2-m_2$ since $tb(K_1)=tb(K_2)$ and $rot(K_1)=rot(K_2)$. Solving these two equations we get $p_1+q_1=p_2+q_2$ and $n_1+m_1=n_2+m_2$. Thus, $(S^+)^{p_1+q_1}(S^-)^{n_1+m_1}(\Bar{K''})=(S^+)^{p_2+q_2}(S^-)^{n_2+m_2}(\Bar{K''}).$ Hence, $K_1$ is Legendrian isotopic to $K_2$.

     For the proof of the second part, let $K\in \mathcal{L}(K'\#K'')$ be any Legendrian knot. Then $K=K_1'\#K_1''$ for some $K_1'\in \mathcal{L}(K')$ and $K_1''\in \mathcal{L}(K'')$. We know that $K_1'=(S^+)^{p}(S^-)^{n}(\Bar{K'})$ and $K_1''=(S^+)^{q}(S^-)^{m}(\Bar{K''})$ for some $p,n,q$ and $m$. Thus
     \begin{align*}
         K=(S^+)^{p}(S^-)^{n}(\Bar{K_1'})\#(S^+)^{q}(S^-)^{m}(\Bar{K_1''})=(S^+)^{p+q}(S^-)^{m+n}(\Bar{K'}\#\Bar{K''}).
     \end{align*}
\end{proof}

\begin{corollary}
     The connected sum of two positive torus knots is Legendrian simple and every Legendrian representative destabilizes to a unique Legendrian knot with the maximal Thurston-Bennequin number.
\end{corollary}
The following result concerns the unique prime decomposition of a given Legendrian knot. In \cite{EH} a similar result for Legendrian knots having maximal tb is given. We generalize their result for the nondestabilizable Legendrian knots.
\begin{Lemma}\label{CostL12}
   If $K$ is a nondestabilizable Legendrian knot then it admits a unique prime decomposition upto possible permutations.
\end{Lemma}

\begin{proof}
    Let $K=K_1\#K_2\cdots \#K_n$ be a prime decomposition unique utpo stabilization and possible permutations. If none of the $K_i$s destabilizes, then we are done. Let if possible $K_i$ is obtained by stabilizing $\Bar{K_i}$, that is $K_1=(S^+)^p(S^-)^m(\Bar{K}_i)$ for some positive integer $p$ and $m$. Then 
    \begin{align*}
        K=&\Bar{K_1}\#\cdots (S^+)^p(S^-)^m(\Bar{K}_i) \cdots\# K_n\\
        =&(S^+)^p(S^-)^m(K_1\#\cdots \Bar{K}_i\cdots \#K_n).
    \end{align*}
    Thus we arrive at a contradiction.
\end{proof}

\begin{Lemma}\label{CostL14}
    Let $K=K_1\#\cdots\#K_n$ be a prime decomposition. If every Legendrian knot in $\mathcal{L}(K)$ destabilizes to a unique Legendrian knot $\Bar{K}$ then for each $i \in \{1,2,\cdots,n\}$, there exists a unique Legendrian representative $\Bar{K_i}$ of $K_i$ such that every Legendrian knot in $\mathcal{L}(K_i)$ destabilizes to $\Bar{K_i}$.
\end{Lemma}

\begin{proof}
    Let $\Bar{K}_i$ be a maximal $tb$ Legendrian knot in $\mathcal{L}(K_i) \: \forall 1\leq i \leq n.$ If possible, let $\hat{K_i}\in \mathcal{L}(K_i)$ be another maximal $tb$ Legendrian knot. Then $\Bar{K_1}\#\cdots \Bar{K_i}\#\cdots \Bar{K_n}$ and $\Bar{K_1}\#\cdots\hat{K_i}\#\cdots \Bar{K_n}$ are Legendrian knots in $\mathcal{L}(K)$, with maximal $tb$. Since $\Bar{K}$ is the unique maximal tb knot, $\Bar{K_1}\#\cdots \Bar{K_i}\#\cdots \Bar{K_n}$ and $\Bar{K_1}\#\cdots\hat{K_i}\#\cdots \Bar{K_n}$ are Legendrian isotopic to $\Bar{K}$. Thus $\Bar{K_1}\#\cdots\hat{K_i}\#\cdots \Bar{K_n}=\Bar{K_1}\#\cdots \Bar{K_i}\#\cdots \Bar{K_n}$ which implies $\Bar{K}_i=\hat{K}_i$. Therefore the exists a unique maximal $tb$ representative of $K_i$ $\forall 1\leq i\leq n$

    If possible, let $K_i'\in \mathcal{L}(K_i)$ be a nonmaximal $tb$ and nondestabilizable Legendrian knot for some $i\in \{1,2,\cdots,n\}$. Now consider $\Bar{K}_1\#\cdots K_i'\cdots \#\Bar{K}_n\in \mathcal{L}(K)$. Since every Legendrian knot in $\mathcal{L}(K)$ destabilizes to $\Bar{K}$, for some $p,m\in \mathbb{Z}^+$ we get
    \begin{align*}
        \Bar{K}_1\#\cdots K_i'\cdots \#\Bar{K}_n&=(S^+)^{p}(S^-)^{m}(\Bar{K})\\
        &=\Bar{K_1}\#\cdots(S^+)^{p}(S^-)^m(\Bar{K}_i)\cdots\#{K_n}.
    \end{align*}
    Thus $K_i'=(S^+)^{p}(S^-)^{m}(\Bar{K}_i)$, which is a contradiction to the fact that $K_i'$ is non-destabilizable Legendrian knot. Hence, every Legendrian representative of $K_i$ destabilzes to $\Bar{K}_i \: \forall 1\leq i\leq n.$
\end{proof}

\begin{Lemma}\label{CostL15}
Let $K'$ and $K''$ be topological knots with the property that every Legendrian representative of $K'$ destabilizes to a unique Legendrian knot $\Bar{K'}$ and every Legendrian representative of $K''$ destabilizes to a unique Legendrian knot $\Bar{K''}$. Then $K'\#K''$ is a Legendrian simple knot type and every Legendrian representative of $K'\#K''$ destabilizes to $\Bar{K'}\#\Bar{K''}$.
\end{Lemma}
\begin{proof}
    The proof follows from Lemmas \ref{CostL11}, \ref{CostL14}.
\end{proof}

\begin{proposition}\label{CostL16}
     Let $K'$ and $K''$ be prime knots of different topological knot types. If there exist $\Bar{K_1'},\Bar{K_2'}\in \mathcal{L}(K')$ and $\Bar{K_1''},\Bar{K_2''}\in \mathcal{L}(K'')$ maximal $tb$ Legendrian representatives such that $rot(K_1')+rot(\Bar{K_1''})=rot(K_2')+rot(\Bar{K_2''})$, then $K'\#K''$ is Legendrian non-simple knot type.
\end{proposition}
   
\begin{proof}
   
The Legendrian knots $K_1'\#\Bar{K_1''}$ and $K_2'\#\Bar{K_2''}$ are maximal $tb$ knots, thus their prime decomposition is unique upto possible permutations. Hence, $K_1'\#\Bar{K_1''}$ is not Legendrian isotopic to $K_2'\#\Bar{K_2''}$. Hence, $K'\#K''$ is Legendrian non-simple knot type, since $tb(K_1'\#\Bar{K_1''})=tb(K_2'\#\Bar{K_2''})$ and $rot(K_1'\#\Bar{K_1''})=rot(K_2'\#\Bar{K_2''})$.
\end{proof}

\begin{proposition}\label{CostL17}
    Let $K$ be a topological prime knot. If there exist maximal $tb$ Legendrian knots $K_0,K_1,K_2$ and $K_3$ with $rot(K_i)=rot(K_0)+i$ for $i=1,2,3$, then $\mathcal{K}\#\mathcal{K}$ is a Legendrian non-simple knot type.
\end{proposition}

\begin{proof}
    Consider Legendrian knots $K_0\#K_3$ and $K_1\#K_2$. Then $tb(K_0\#K_3)=tb(K_1\#K_2)$
 and $r(K_0\#K_3)=r(K_1\#K_2)$. Since $K_0\#K_3$ and $K_1\#K_2$ are maximal $tb$ knots, their prime decomposition is unique upto possible permutations. Hence, $K_0\#K_3$ is not Legendrian isotopic to $K_1\#K_2$.
 \end{proof}


\begin{Lemma}\label{CostL18}
    Let $K_1,L_1, K_2$ and $L_2$ be Legendrian knots of same topological knot type, then $$\Cost(K_1\#K_2,L_1\#L_2)\leq \min\{\Cost(K_1,L_1)+\Cost(K_2,L_2),\Cost(K_1,L_2)+\Cost(K_2,L_1)\}.$$
\end{Lemma}
\begin{proof}
 Let $\Cost(K_i,L_j)=p_{ij}+n_{ij}+q_{ij}+m_{ij}$ with $$(S^+)^{p_{ij}}(S^-)^{n_{ij}}(K_i)=(S^+)^{q_{ij}}(S^-)^{m_{ij}}(L_j), \: i,j=1,2.$$ Then, 

    \begin{align}
        \nonumber (S^+)^{p_{11}+p_{22}}(S^-)^{n_{11}+n_{22}}(K_1\#K_2)&=(S^+)^{p_{11}}(S^-)^{n_{11}}(K_1)\#(S^+)^{p_{22}}(S^-)^{n_{22}}(K_2)\\
       \nonumber &=(S^+)^{q_{11}+q_{22}}(S^-)^{m_{11}+m_{22}}(L_1\#L_2).
        \end{align}
        Thus,
        \begin{align}
        \nonumber \Cost(K_1\#K_2,L_1\#L_2)&\leq \sum_{i=1}^2p_{ii}+n_{ii}+q_{ii}+m_{ii}\\
        &=\Cost(K_1,L_1)+\Cost(K_2,L_2).\label{eq:Costeq1}
    \end{align}
    Now,
     \begin{align}
       \nonumber (S^+)^{p_{12}+p_{21}}(S^-)^{n_{12}+n_{21}}(K_1\#K_2)&=(S^+)^{p_{12}}(S^-)^{n_{12}}(K_1)\#(S^+)^{p_{21}}(S^-)^{n_{21}}(K_2)\\
       \nonumber &=(S^+)^{q_{12}}(S^-)^{m_{12}}(L_2)\#(S^+)^{q_{21}}(S^-)^{m_{21}}(L_1)\\
       \nonumber &=(S^+)^{q_{12}+q_{21}}(S^-)^{m_{21}+m_{21}}(L_2\#L_1).
       \end{align}
       Thus,
       \begin{align}
        \Cost(K_1\#K_2,L_1\#L_2)&\leq \Cost(K_1,L_2)+\Cost(K_2,L_1).\label{eq:Costeq2}
    \end{align}

   Using equations \ref{eq:Costeq1} and \ref{eq:Costeq2}, we have the $\Cost(K_1\#K_2,L_1\#L_2)$ 
   \begin{align*}
       \leq \min\{\Cost(K_1,L_1)+\Cost(K_2,L_2),\Cost(K_1,L_2)+\Cost(K_2,L_1)\}.
   \end{align*}
   
    \end{proof}

\begin{Lemma}\label{CostL19}
    Let $K_1,L_1$ and $K_2,L_2$ be Legendrian knots of two different topological knot types, then $\Cost(K_1\#K_2,L_1\#L_2)\leq \Cost(K_1,L_1)+\Cost(K_2,L_2)$.
\end{Lemma}
\begin{proof}
   The proof follows from the first part of the proof of Lemma \ref{CostL18}.
\end{proof}

\begin{Lemma}\label{CostL20}
    Let $K_1,L_1$ and $K_2,L_2$ be Legendrian knots of two different topological knot types such that $K_1, K_2$ destabilizes to $ L_1,L_2$, respectively. Then $\Cost(K_1\#K_2,L_1\#L_2)=\Cost(K_1,L_1)+\Cost(K_2,L_2)$.  
\end{Lemma}

\begin{proof}
 As $K_1, K_2$ destabilizes to $ L_1,L_2$, respectively, there exist non negative integers $p,n,q$ and $m$ such that $K_1=(S^+)^p(S^-)^n(L_1),$ and $ K_2=(S^+)^{q}(S^-)^m(L_2)$. Then the lemma follows from the observation
 \begin{align*}
     K_1\#K_2&=(S^+)^p(S^-)^n(L_1)\#(S^+)^{q}(S^-)^m(L_2)\\
     &=(S^+)^{p+q}(S^-)^{n+m}(L_1\#L_2).
 \end{align*}

\end{proof}

\begin{Lemma}\label{CostL21}
Let $K_1,L_1$ and $K_2,L_2$ be Legendrian knots of two different topological knot types such that $K_1=(S^+)^p(S^-)^n(L_1)$ and $ L_2=(S^+)^q(S^-)^{m}(K_2)$. Then $$\Cost(K_1\#K_2,L_1\#L_2)=|p-q|+|n-m|.$$
\end{Lemma}
\begin{proof}
We will prove this Lemma in cases. 
    
    \noindent \textbf{Case I:} $p\geq q,n\geq m$.
    
  \noindent We prove $K_1\#K_2$ destabilizes to $L_1\#L_2$ as follows:
   \begin{align*}
       K_1\#K_2&=(S^+)^p(S^-)^n(L_1\#K_2)\\
       &=(S^+)^{p-q}(S^-)^{n-m}(L_1\#(S^+)^q(S^-)^{m}(K_2))\\
       &=(S^+)^{p-q}(S^-)^{n-m}(L_1\#L_2).&&
   \end{align*} 
   Thus, 
   \begin{align*}
       \Cost(K_1\#K_2,L_1\#L_2)&=p-q+n-m\\
       &=|\Cost(K_1,L_1)-\Cost(K_2,L_2)|.
   \end{align*}
   
 \noindent \textbf{Case II:} $p\leq q,n\leq m$.

   \noindent Using the similar technique as used in Case I, we prove that $K_1\#K_2$ destabilizes to $L_1\#L_2$.
   \begin{align*}
       L_1\#L_2&=(S^+)^q(S^-)^{m}(L_1\#K_2)\\
       &=(S^+)^{q-p}(S^-)^{m-n}((S^+)^p(S^-)^n(L_1)\#K_2)\\
       &=(S^+)^{q-p}(S^-)^{m-n}(K_1\#K_2).&&
   \end{align*} 
   Therefore, 
   \begin{align*}
       \Cost(K_1\#K_2,L_1\#L_2)&=q-p+m-n\\
       &=|\Cost(K_1,L_1)-\Cost(K_2,L_2)|.
   \end{align*}

   \noindent \textbf{Case III:} $p\geq q,n\leq m$.

\noindent Since $ rot(K_1)=rot(L_1)-n+p$, and $,rot(L_2)=rot(K_2)-m+q$, using Lemma \ref{CostL8}, we get
\begin{align*}
    \Cost(K_1\#K_2,L_1\#L_2)&\geq |rot(K_1\#K_2)-rot(L_1\#L_2)|\\
    &=p-q+m-n.
\end{align*}
The reverse inequality follows from the computations done below. 

   \begin{align*}
       (S^-)^{m-n}(K_1\#K_2)&=(S^-)^{m-n}((S^+)^p(S^-)^n(L_1)\#K_2)\\
       &=(S^+)^p(S^-)^{m}(L_1\#K_2)\\
       &=(S^+)^{p-q}(L_1\#(S^+)^q(S^-)^m(K_2))\\
       &=(S^+)^{p-q}(L_1\#L_2).
   \end{align*}

   
   \noindent \textbf{Case IV:} $q\geq p,n\geq m$.

   \noindent The proof of this case is similar to Case III. 

   Hence, we conclude from the above cases that $$\Cost(K_1\#K_2,L_1\#L_2)=|p-q|+|n-m|.$$
\end{proof}

\begin{Lemma}\label{CostL22}
    Let $\mathcal{K}_1$ and $\mathcal{K}_2$ be different topological knot types which are Legendrian simple. Let $K_1,L_1\in \mathcal{K}_1$ and $K_2,L_2\in\mathcal{K}_2$ be Legendrian knots with maximal $tb$, then 
    
    \noindent (i) If $rot(K_i)\leq rot(L_i),i=1,2$, then $$\Cost(K_1\#K_2,L_1\#L_2)=\Cost(K_1,L_1)+\Cost(K_2,L_2).$$
       
\noindent (ii) If $rot(K_1)\leq rot(L_1)$ and $rot(K_2)\geq rot(L_2)$, then
        \begin{align*}
            |\Cost(K_1,L_1)-\Cost(K_2,L_2)|&\leq \Cost(K_1\#K_2,L_1\#L_2)\\
            &\leq \max\{Cost(K_1,L_1),Cost(K_2,L_2)\}.&&
        \end{align*}
    
\end{Lemma}

\begin{proof}

By Lemma \ref{CostL9}, we have $(S^+)^{r_i}(K_i)=(S^-)^{r_i}(L_i)$, where $$2r_i=|rot(K_i)-rot(L_i)|=\Cost(K_i,L_i),\:i=1,2.$$  

 \noindent(i)  $rot(K_i)\leq rot(L_i),i=1,2$.
    
Using Lemma \ref{CostL19}, we have
\begin{align*}
    \Cost(K_1\#K_2,L_1\#L_2)&\leq \Cost(K_1,L_1)+\Cost(K_2,L_2)\\
    &=2(r_1+r_2).
\end{align*}

The reverse inequality can be derived using Lemma \ref{CostL8} in the following way.
\begin{align*}
    \Cost(K_1\#K_2,L_1\#L_2)&\geq |rot(K_1\#K_2)-rot(L_1\#L_2)|\\
    &=(rot(L_1)-rot(K_1))+(rot(L_2)-rot(K_2))\\
    &=2(r_1+r_2). 
\end{align*}
Hence, $ \Cost(K_1\#K_2,L_1\#L_2)=\Cost(K_1,L_1)+\Cost(K_2,L_2)$.

   \noindent (ii) $rot(K_1)\leq rot(L_1)$ and $rot(K_2)\geq rot(L_2)$.

    Observe that,
    \begin{align*}
        rot(K_1\#K_2)&=rot(K_1)+rot(K_2)\\
        &=rot(K_1)+rot(L_2)+2r_2,\: \text{ and }\\
        rot(L_1\#L_2)&=rot(L_1)+rot(L_2)\\
       & =rot(K_1)+rot(L_2)+2r_1.
    \end{align*}
    Then, $|rot(K_1\#K_2)-rot(L_1\#L_2)|=2r$, where $r=|r_1-r_2|$. Using Lemma \ref{CostL8}, we have
   $$\Cost(K_1\#K_2,L_1\#L_2)\geq 2r=|\Cost(K_1,L_1)-\Cost(K_2,L_2)|.$$ 
   Now, we consider two cases to prove 
     $$\Cost(K_1\#K_2,L_1\#L_2)\leq \max\{Cost(K_1,L_1),Cost(K_2,L_2)\}.$$ 
     First consider $r_2\leq r_1$. Then
 \begin{align*}
       (S^+)^{r_1}(K_1\#K_2)&=(S^-)^{r_1}(L_1\#K_2)\\
       &=(S^-)^{r}(L_1)\#(S^-)^{r_2}(K_2)\\
       &=(S^-)^{r}(L_1)\#(S^+)^{r_2}(L_2)\\
       &=(S^-)^{r}(S^+)^{r_2}(L_1\#L_2).
   \end{align*}
       
     Thus, $\Cost(K_1\#K_2,L_1\#L_2)\leq r+r_1+r_2=2r_1.$

    Now consider the case with $r_2\geq r_1$. Then
    \begin{align*}
        (S^-)^{r_2}(K_1\#K_2)&=(S^+)^{r_2}(K_1\#L_2)\\
        &=(S^+)^{r_1}(K_1)\#(S^+)^{r}(L_2)\\
        &=(S^-)^{r_1}(L_1)\#(S^+)^{r}(L_2)\\
        &=(S^-)^{r_1}(S^+)^{r}(L_1\#L_2).
    \end{align*}
     Thus, $\Cost(K_1\#K_2,L_1\#L_2)\leq r+r_1+r_2=2r_2.$ Hence,
     $$\Cost(K_1\#K_2,L_1\#L_2)\leq \max\{Cost(K_1,L_1),Cost(K_2,L_2)\}.$$
     
   \end{proof}

   \begin{example}
       Let $K_1$ and $K_2$ be oriented left-handed and right-handed Legendrian trefoils as shown in Figure \ref{fig:CostOLH}. Let $\Bar{K_1}$ and $\Bar{K_2}$ be the Legendrian knots obtained by reversing the orientations of $K_1$ and $K_2$ respectively. Note that $tb(K_1)=-6$ and $tb(K_2)=1$ and $rot(K_1)=-1, rot(K_2)=0$. Then $\Cost(K_1,\Bar{K_1})=|rot(K_1)-rot(\Bar{K_2})|=2$, since left-handed trefoil is a Legendrian simple knot type. The Legendrian knots $K_1\#K_2$ and $\Bar{K_1}\#K_2$ are of the same topological knot type since left-handed trefoil knot is invertible. Then by Lemma \ref{CostL22}
       $ \Cost(K_1\#K_2,\Bar{K_1}\#K_2)= 2$.
       % Figure environment removed
   \end{example}
   
\section{Conclusion}\label{Costgraph}

For a topological knot type $\mathcal{K}$, let $\mathcal{L(K)}/\sim$ denote the set of equivalence classes of Legendrian representatives of $\mathcal{K}$ upto Legendrian isotopy. The Cost function gives a metric on the set $\mathcal{L(K)}/\sim$ in the following way. Then the map $d_C: (\mathcal{L(K)}/\sim)\times (\mathcal{L(K)}/\sim)\rightarrow \Z^+\cup\{0\}$ defined as $d_C([K],[\Tilde{K}])=\Cost(K,\Tilde{K})$ for all $[K],[\Tilde{K}]\in \mathcal{L(K)}/\sim$ is a metric. The map $d_C$ satisfies the first two properties of a metric (see Theorem \ref{CostL4} and Remark \ref{re:Cost1}). For the triangle inequality, consider $[K_1],[K_2],[K_3]\in \mathcal{L(K)}/\sim$. Let $F_i$ be a front projection of $K_i,\: i=1,2,3$. Let $SF_{1}\in \mathcal{SF}(F_1,F_2)$ and $SF_{2}\in \mathcal{SF}(F_2,F_3)$ be such that $\Cost(K_1,K_2)=\Cost(F_1,F_2,SF_{1})$ and $\Cost(K_2,K_3)=\Cost(F_2,F_3,SF_{2})$. Therefore the sequence $(SF_{1},SF_{2})\in \mathcal{SF}(F_1,F_3)$. Thus \begin{align*}
    \Cost(K_1,K_3)&\leq \Cost(F_1,F_3,(SF_{1},SF_{2}))\\
    &=\Cost(F_1,F_2,SF_1)+\Cost(F_2,F_3,SF_2)\\
    &=\Cost(K_1,K_2)+\Cost(K_2,K_3).
\end{align*}
Hence, $d_C$ is a metric.

Using the Cost function we define a graph invariant of the topological knot. We associate a graph $G_{\mathcal{K}}$ to a topological knot type $\mathcal{K}$ in the following way. Fix the set of vertices for $G_{\mathcal{K}}$ to be $\mathcal{L(K)}/\sim$. Two vertices $[K],[\Tilde{K}]\in \mathcal{L(K)}$ are joined by an edge if and only if $\Cost(K,\Tilde{K})=1$. We refer to this graph as the graph of \emph{Legendrian Representatives} of ${\mathcal{K}}$. The Cost function graph for unknot is shown in Figure \ref{fig:Costgraph}. 

% Figure environment removed
\begin{remark}
     For Legendrian simple knot types, the Cost function graph is the same as the mountain range defined in \cite{E}.
\end{remark}

We now raise a few questions, given below, that the authors are currently investigating.
\begin{enumerate}
    \item Does the graph $G_{\mathcal{K}}$ together with the value of maximal $tb$ detect unknot? More explicitly, if $G_{\mathcal{K}}$ is isomorphic to that of unknot as a graph and maximal $tb$ of $\mathcal{K}$ equals $-1$ then does it necessarily follow that $\mathcal{K}$ is unknot?
    \item Is there a relation between the values of the Cost function for a given knot type and Eliashberg-Chekanov DGAs of corresponding Legendrian representatives?
    \item From a computational viewpoint, is there an algorithm or a procedure to generate the graph $G_{\mathcal{K}}$ for a given knot type $\mathcal{K}$?
\end{enumerate}




\bibliographystyle{plain}
\bibliography{references.bib}
\vspace{10pt}
\end{document}
