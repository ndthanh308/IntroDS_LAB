\subsection{When $n=2\ell$ and $i=\ell$}
\label{sec:main_thm}


In this subsection, we will focus on extending 
Lemma \ref{lem:inequality} to the case when 
$n= 2\ell$ and when $i = \ell$.  Our first lemma
shows that the $\alpha_{2\ell,k,\ell}$'s are differences 
of succesive coefficients of the trinomial coefficients. 
Let  $p_{\ell,k}$ denote the coefficient of $x^k$ in 
$(1+x+x^2)^{\ell}$ and to save one subscript, let 
$\last_{\ell,k} =\alpha_{2\ell,k,\ell}$.
Thus, $\last_{\ell,k}=0$ when $k>\ell$ and when $k < 0.$


 \begin{lemma}
\label{lem:ank:rec:1}
Fix positive integers $\ell, k$ with $\ell \geq 2$ and with 
$0 \leq k \leq \ell$.  Then, we have
\begin{enumerate}
%	\item \label{lem:ank:rec:1:1} For positive integers $n \geq 3$, when $i < \nhalf$, we have $\alpha_{n,k,i}=\alpha_{n-1,k,i}+\alpha_{n-1,k-1,i}.$
\item  \label{lem:ank:rec:1:2} 
%$a_{n,k} = a_{n-1,k} + a_{n-1,k-1} + a_{n-1, k-2}$. 
$\last_{\ell,k} = \last_{\ell-1,k} + \last_{\ell-1,k-1} + \last_{\ell-1, k-2}$. 
\item  \label{lem:ank:rec:1:3} 
$p_{\ell,k} = p_{\ell-1,k} + p_{\ell-1,k-1}+p_{\ell-1,k-2}.$
\item  \label{lem:ank:rec:1:4} 
%$a_{n,k} = p_{n,k} - p_{n,k-1}$ is a  positive integer.
$\alpha_{2 \ell, k, \ell} =  \last_{\ell,k} = p_{\ell,k} - p_{\ell,k-1}$ is a  positive integer.
\end{enumerate}
 \end{lemma}
 \begin{proof}
 	\begin{enumerate}
% 		\item When $j\leq i < \nhalf$, from Murnaghan Nakayama lemma it is simple to check that $\chi_{n,k}(j)=\chi_{n-1,k}(j)+\chi_{n-1,k-1}(j).$ Therefore 
% 		$$\alpha_{n,k,i}  = \frac{1}{2^i} \sum_{j=0}^{i}[\chi_{n-1,k}(j)+\chi_{n-1,k-1}(j)]\binom{i}{j}=\alpha_{n-1,k,i}+\alpha_{n-1,k-1,i}.$$
\item This follows from the Murnaghan Nakayama lemma, see
\cite[Lemma 2.1]{chan-lam-binom-coeffs-char}.
\item  By the definition of $p_{\ell,k}$, we have 
\begin{align*}
p_{\ell,k} &= \mbox{ Coeff. of } x^k \mbox{ in } (1+x+x^2)^{\ell}\\
&= \mbox{ Coeff. of } x^k \mbox{ in } (1+x+x^2)^{\ell-1}(1+x+x^2)=p_{\ell-1,k} 
+ p_{\ell-1,k-1}+p_{\ell-1,k-2}.
\end{align*}
\item We induct on $\ell$.  When  $\ell=2,3$, it is easy to see that 
$\last_{\ell,k} = p_{\ell,k} - p_{\ell,k-1}$ is a  positive 
integer (also see Table \ref{tab:last_l,k}). Let the 
result be true for all positive integers less than $\ell$. From part (1) 
we have  
\begin{align*}
\last_{\ell,k} & = \last_{\ell-1,k} + \last_{\ell-1,k-1} + \last_{\ell-1, k-2} \\
& = p_{\ell-1,k} - p_{\ell-1,k-1}+p_{\ell-1,k-1} - p_{\ell-1,k-2}
+p_{\ell-1,k-2} - p_{\ell-1,k-3} \\
& = p_{\ell-1,k} +p_{\ell-1,k-1}+p_{\ell-1,k-2}- p_{\ell-1,k-1} - 
p_{\ell-1,k-2} - p_{\ell-1,k-3} \\
& = p_{\ell,k}-p_{\ell,k-1}.
\end{align*}
%The above equality follows from part (3).
\end{enumerate}
The proof is complete.
\end{proof}
 


We need a couple of inequalities which we see in the next few
lemmas.  Our  next lemma is an extension of Lemma \ref{lem:ineq}.


\begin{lemma}
\label{lem:frac:3}
For $1 \le i \le 4$, let $a_i, b_i$ be positive integers with 
$\dfrac{a_1}{b_1} \le \dfrac{a_2}{b_2} \le \dfrac{a_3}{b_3} 
\le \dfrac{a_4}{b_4}$. Then, 
$$\dfrac{a_1 + a_2 + a_3}{b_1 + b_2 + b_3} \le 
\dfrac{ a_2 + a_3+a_4}{ b_2 + b_3+b_4}.$$ 
\end{lemma}
\begin{proof}
We know  
$\dfrac{a_1}{b_1} \le \dfrac{a_2}{b_2} \le 
\dfrac{a_3}{b_3} \le \dfrac{a_4}{b_4}$.  Lemma \ref{lem:ineq}
implies that 
$\dfrac{a_1}{b_1} \le \dfrac{a_2}{b_2} \le 
\dfrac{a_2+a_3}{b_2+b_3} \le \dfrac{a_3}{b_3} \le \dfrac{a_4}{b_4}$. 
Applying Lemma \ref{lem:ineq} again, we get 
$$\dfrac{a_1}{b_1} \le   \dfrac{a_1+ a_2+a_3}{b_1+ b_2+b_3} 
\le \dfrac{a_2 + a_3}{b_2 + b_3} \le 
\dfrac{ a_2 + a_3+a_4}{ b_2 + b_3+b_4} \le  \dfrac{a_4}{b_4}, \mbox{ completing the proof.}$$ 
\end{proof}
 
We next compare the ratio $\last_{\ell,k+1}/\last_{\ell-1,k}$
as $k$ decreases.


\begin{lemma}
\label{lem:ank:rec:2}
Fix positive integers $\ell, k $ with $1 \le k \le \ell-1$ 
and $\ell \geq 3$. Then, 
$$\dfrac{\last_{\ell,k+1}}{\last_{\ell-1,k} } \le \dfrac{\last_{\ell,k}}{\last_{\ell-1,k-1} }.$$ 
\end{lemma}
\begin{proof}
We use induction on $\ell$.  The result is easily verified 
when $\ell = 3,4$ (see Table \ref{tab:last_l,k} in 
Example \ref{example:tab_last} below). 
Let the result be true when $N \le \ell$ and when $ 1 \le k \le \ell-1$.
We then show that it holds for $\ell+1$. Thus, we need to show that 
$\dfrac{\last_{\ell+1,k+1}}{\last_{\ell,k} } \le 
\dfrac{\last_{\ell+1,k}}{\last_{\ell,k-1} }$.  By Lemma~\ref{lem:ank:rec:1}, 
this is equivalent to showing that 
$$\dfrac{\last_{\ell,k+1} + \last_{\ell,k} + \last_{\ell, k-1}}
{\last_{\ell-1,k} + \last_{\ell-1,k-1} + \last_{\ell-1, k-2} }  
\le 
\dfrac{\last_{\ell,k} + \last_{\ell,k-1} + \last_{\ell, k-2}}
{\last_{\ell-1,k-1} + \last_{\ell-1,k-2} + \last_{\ell-1, k-3} }.$$
By the induction hypothesis, we know that 
$$\dfrac{\last_{\ell,k+1}}{\last_{\ell-1,k} } \le 
\dfrac{\last_{\ell,k}}{\last_{\ell-1,k-1} } \le  
\dfrac{\last_{\ell,k-1}}{\last_{\ell-1,k-2} } \le  
\dfrac{\last_{\ell,k-2}}{\last_{\ell-1,k-3} }.$$
The proof is complete by applying Lemma~\ref{lem:frac:3}.
 \end{proof}



By Lemma \ref{lem:ank:rec:1}, 
$\last_{\ell,k}$ is a positive integer. 
Rearranging terms a bit, as an immediate consequence of 
Lemma~\ref{lem:ank:rec:2}, we obtain the following corollary.
\begin{corollary}
\label{cor:ank:rec}
Fix positive integers $\ell, k $ with $ 1 \le k \le \ell-1$ 
and with $\ell\geq 3$. Then, we have
$$\dfrac{\last_{\ell,k+1}}{\last_{\ell,k} } \le \dfrac{\last_{\ell-1,k}}{\last_{\ell-1,k-1} }.$$ 
In general, we have
$$\dfrac{\last_{\ell,k+1}}{\last_{\ell,k}} \le 
\dfrac{\last_{\ell - r,k+1-r}}{\last_{\ell - r, k - r}}, 
\mbox{ when } 1 \le r \le k.$$
\end{corollary}

\begin{example}
	\label{example:tab_last}
	We illustrate Lemma \ref{lem:ank:rec:2} and Corollary 
	\ref{cor:ank:rec} when
	$\ell,k\in\{0,1,\ldots,9\}$  in Table 
\ref{tab:last_l,k} which contains $\last_{\ell,k}$'s. 
From the table, one can verify 
	Lemma \ref{lem:ank:rec:2} when $3\leq \ell\leq 9$ and can also check 
	when $\ell=2$ and $k=1$ that we have $\last_{\ell-1,k}=0$.
	Thus, Lemma \ref{lem:ank:rec:2} is not true when $\ell=2$.
	\begin{table}
		$ \displaystyle \begin{array}{r|c|c|c|c|c|c|c|c|c|c|c|} 
			& k=0  & k=1  &  k=2  & k=3  & k=4   & k=5  & k=6   & k=7   & k=8  & k=9    \\ \hline
			\ell=0  &  1 &    &   &    &   &    &   &   &   &      \\  \hline
			\ell=1    &  1 &   0 &   &    &   &    &   &   &    &   \\  \hline
			\ell=2      & 1  &  1  &  1 &    &   &    &   &     &   &   \\  \hline
			\ell=3        &1   &  2  &  3 &  1  &   &    &     &   &   &   \\  \hline  
			\ell=4          & 1  &   3 & 6  &  6  &  3 &      &  &   &   &   \\   \hline
			\ell=5            &1   &   4 & \fya{10}  & \fby{15}   & \fyc{15}     & 6  &   &   &   &   \\   \hline
			\ell=6             &  1 &  5  &  15 &  \fya{29}    &  \fby{40}  & \fyc{36}  & 15  &   &   &   \\   \hline
			\ell=7               & 1  & 6   & 21    &49   & 84   & 105  & 91  & 36  &   &   \\   \hline
			\ell=8                  &1   &   7  &  28   & 76  &  154  & 238  & 280  & 232  & 91  &   \\  \hline
			\ell=9                   &1    &  8  & 36   & 111  & 258   & 468  & 672  & 750  & 603  & 232  \\
			\hline
		\end{array}$
		\caption{The values of $\last_{\ell,k}$.}
		\label{tab:last_l,k}
		
	\end{table}
\end{example}

With this preparation, we can prove the following result.
 
\begin{lemma}
\label{lem:ank:rec}
Fix positive integers $\ell, k $ with $ 1 \le k \le \ell-1$ and 
with $\ell\geq 3$. Then, we have
\begin{equation}\label{eqn:ank:rec}
\dfrac{ \last_{\ell, k+1} }{ {{2\ell} \choose {k+1}} - {{2\ell} \choose {k}}} \le 
\dfrac{ \last_{\ell,k} }{ {{2\ell} \choose {k}} - {{2\ell} \choose {k-1}} }.
\end{equation}
\end{lemma}
\begin{proof}
Note that ${{2\ell} \choose {k}} $ equals the coefficient of 
$x^k$ in the expansion of $$(1+x)^{2\ell} =   (1 + x + x^2 + x)^{\ell} = 
\sum_{r=0}^{\ell} {\ell \choose r} x^r (1 + x + x^2)^{\ell- r}.$$
Thus, ${ {2 \ell} \choose {k}} = \sum_{r=0}^{\ell} {\ell \choose r} p_{\ell-r, k-r},$
and hence
\begin{eqnarray*}
 {{2\ell} \choose {k+1}} - {{2\ell} \choose {k}} &=& 
\sum_{r=0}^{\ell} {\ell \choose r} p_{\ell-r, k+1-r} - 
\sum_{r=0}^{\ell} {\ell \choose r} p_{\ell-r, k-r} \\
% &=& \sum_{r=0}^{\ell} {\ell \choose r} \big( p_{\ell-r, k+1-r}  
%- p_{\ell-r, k-r}\big) \\
 &=& \sum_{r=0}^{\ell} {\ell \choose r} \last_{\ell-r, k+1-r}  
 = \last_{\ell,k+1} + \sum_{r=1}^{\ell}{\ell \choose r} 
\last_{\ell-r, k+1-r}.
%\big(p_{\ell-r, k+1-r}  - p_{\ell-r, k-r}\big) .
\end{eqnarray*}
 Hence, 
\begin{equation}
\label{eqn:pnk+1}
\dfrac{{{2\ell} \choose {k+1}} - {{2\ell} \choose {k}}}{\last_{\ell,k+1} }  = 1 + \sum_{r=1}^{\ell} {\ell \choose r} \dfrac{ \last_{\ell-r, k+1-r} }{ \last_{\ell,k+1} }.
\end{equation} 
 Similarly, 
\begin{equation}\label{eqn:pnk} 
\dfrac{{{2\ell} \choose {k}} - {{2\ell} \choose {k-1}}}{\last_{\ell,k}}  
= 
1 + \sum_{r=1}^{\ell} 
{\ell \choose r} \dfrac{ \last_{\ell-r, k-r} }{\last_{\ell,k} }.
\end{equation}
Thus, to obtain our required result, we need to show that 
$$\dfrac{\last_{\ell-r, k+1-r}  }{\last_{\ell,k+1}} \ge 
\dfrac{ \last_{\ell-r, k-r} }{\last_{\ell,k} } 
\Leftrightarrow 
\dfrac{\last_{\ell,k+1} }{ \last_{\ell,k} } \le 
\dfrac{\last_{\ell-r, k+1-r} }{\last_{\ell-r, k-r}  }.$$
By Corollary~\ref{cor:ank:rec}, 
$\dfrac{\last_{\ell,k+1}}{\last_{\ell,k}} \le 
\dfrac{\last_{n - r,k+1-r}}{\last_{n - r, k - r}}$ 
for $1 \le r \le k$, completing the proof.
\end{proof}

%A rearrangement of Equation~\eqref{eqn:ank:rec} implies that 
%$$\dfrac{p_{\ell,k+1} - p_{\ell,k}}{{{2\ell} \choose {k+1}} - {{2\ell} \choose {k}}} \le 
%  \dfrac{p_{\ell,k} - p_{\ell,k-1}}{{{2\ell} \choose {k}} - {{2\ell} \choose {k-1}}},$$ giving us Conjecture~$5$.
