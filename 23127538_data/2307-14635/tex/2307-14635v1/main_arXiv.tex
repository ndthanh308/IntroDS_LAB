\documentclass[aps,prl,reprint,superscriptaddress,preprintnumbers]{revtex4-2}
\usepackage{amsfonts, amssymb, amsmath,graphicx}
%usepackage{caption}
%\usepackage{subcaption}
\usepackage{multirow}
\usepackage{dcolumn}
\usepackage{bm}
\usepackage{color}
\usepackage[colorlinks=true,allcolors=blue]{hyperref}

\begin{document}

\preprint{RIKEN-iTHEMS-Report-23}
\title{Two dimensional lattice with an imaginary magnetic field}

\author{Tomoki Ozawa}
\affiliation{Advanced Institute for Materials Research (WPI-AIMR), Tohoku University, Sendai 980-8577, Japan}
\author{Tomoya Hayata}
\affiliation{Departments of Physics, Keio University, 4-1-1 Hiyoshi, Kanagawa 223-8521, Japan}
\affiliation{RIKEN iTHEMS, RIKEN, Wako 351-0198, Japan}

\date{\today}

\newcommand{\tom}[1]{{\color{red} #1}}

\begin{abstract}
We explore gauge-independent properties of two-dimensional non-Hermitian lattice systems with an imaginary magnetic field. 
We find that the energy spectrum under the open boundary conditions is an example of such gauge-independent properties. We discuss how to obtain the asymptotic continuum energy spectrum upon increasing length of one side using the framework of the non-Bloch band theory. We also find an analog of the Aharonov-Bohm effect; the net change of the norm of the wavefunction upon adiabatically forming a closed path is determined by the imaginary magnetic flux enclosed by the path.
\end{abstract}

\maketitle

Physics of a charged particle in an external magnetic field has been of fundamental importance in condensed matter physics. In two dimensions, a charged particle in a magnetic field forms the equally spaced energy spectrum called the Landau level, which is directly responsible for phenomena such as the Landau diamagnetism, de Haas-van Alphen effect~\cite{AshcroftMermin}, and the integer and fractional quantum Hall effects~\cite{Klitzing:1980PRL,Tsui:1982PRL}. A charged particle on a two-dimensional lattice under a magnetic field is described by the Harper-Hofstadter model~\cite{Harper:1955PPS,Hofstadter:1976PRB}, which is the paradigmatic model of the Chern insulator~\cite{Thouless:1982PRL}. 

Recently, there is an increasing interest in non-Hermitian physics~\cite{Ashida:2020AdvPhys,Bergholtz:2021RMP}. Also in non-Hermitian quantum mechanics, the effect of vector potentials has played significant roles. For example, the Hatano-Nelson model is the one dimensional lattice model under an imaginary vector potential~\cite{Hatano:1996PRL,Hatano:1997PRB}, and has been of fundamental importance showing the non-Hermitian skin effect and non-trivial point gap topology~\cite{Kawabata:2019PRX,Okuma:2020PRL}.
The imaginary vector potential has also been crucial in understanding the Landau-Zener transition~\cite{PhysRevB.58.16051,PhysRevB.81.033103}.

With the recent experimental development of non-Hermitian quantum mechanics, one can now realize a variety of non-Hermitian models under control, and there is an increasing interest in experimentally realizing two or higher dimensional non-Hermitian models~\cite{Yamauchi:2020arXiv,Zou:2021NatComm,Palacios:2021NatComm,Shang:2022AdvSci}.
Albeit this rapid progress in non-Hermitian physics and the important role of magnetic fields played in condensed matter physics, there is little study on properties of the imaginary magnetic fields in two dimensions, namely, properties of non-Hermitian lattice systems where magnetic fields are imaginary, analogous to the imaginary vector potential in the Hatano-Nelson model.

In this paper, we explore basic properties of two-dimensional lattices with an imaginary magnetic field. We first elucidate the meaning of gauge invariance in non-Hermitian settings, in order to distinguish between properties which are intrinsically due to the imaginary magnetic field and gauge-dependent properties depending on specific realizations and setups. We find that certain spectral properties are gauge invariant, and discuss that the asymptotic energy spectrum as one makes the length of the system larger, fixing the other length, can be nicely understood within the framework of the non-Bloch band theory.
Especially, even though the system is non-Hermitian, the asymptotic spectrum under the open boundary condition can be related to the spectrum under the periodic boundary condition satisfying certain conditions.
We also find an analog of the Aharonov-Bohm effect for imaginary magnetic fields. Upon making a wavepacket move to form a closed trajectory in real space, the overall change of the norm of the wavefunction is related to the imaginary magnetic flux enclosed by the trajectory. Our work lays foundation to understand gauge invariant properties in the setup of imaginary magnetic fields, generalizing the concept of magnetic fields to two-dimensional non-Hermitian settings.

\textit{Model.---}
We consider a two-dimensional square lattice with an imaginary magnetic field. We label the lattice sites by coordinates $(x,y)$, where $x$ and $y$ are both integers. We let $\psi_{x,y}$ denote the amplitude of the wavefunction at site $(x,y)$. The Schr\"odinger equation governing the dynamics of the system is
\begin{align}
    i\frac{d \psi_{x,y}}{dt}
    =&
    J\left( e^{i\theta_X (x-1,y)}\psi_{x-1,y} + e^{-i\theta_X (x,y)}\psi_{x+1,y} \right.
    \notag \\
    &\left. + e^{i\theta_Y (x,y-1)}\psi_{x,y-1} + e^{-i\theta_Y (x,y)}\psi_{x,y+1}\right),
\end{align}
where $t$, and $J$ are the time, and hopping parameter, respectively.
In this paper, we consider two gauge choices: the Landau gauge and the symmetric gauge. 
The Landau gauge is defined by $(\theta_X, \theta_Y) = (0, Bx)$, and the symmetric gauge is defined by $(\theta_X, \theta_Y) = (-By/2, Bx/2)$. When $B$ is real, these gauges correspond to the ordinary Landau and symmetric gauges with a real magnetic field. In this paper, however, we take $B$ to represent a purely imaginary magnetic field, $B = i\mathcal{B}$, with $\mathcal{B}$ being a real number. We note that, when $B$ is imaginary, the factors $e^{i\theta_X (x,y)}$ and $e^{i\theta_Y (x,y)}$ can have modulus different from one, implying non-Hermiticity. 

\textit{Gauge transformation.---}
Landau and symmetric gauges are equivalent for a real magnetic field because they are related by a gauge transformation. We first review the gauge transformation in Hermitian setups, and then extend the concept to non-Hermitian settings. The gauge transformation is to consider a state which is related to the original state by a position-dependent phase factor $\psi^\prime_{x,y} = e^{i\chi (x,y)}\psi_{x,y}$, where $\chi (x,y)$ is a real function. This transformation amounts to applying a local unitary transformation to wavefunction. If we take the Hamiltonian $H$ in the Hermitian Landau gauge, by choosing the gauge transformation $e^{i \chi (x,y)} = e^{iBxy/2}$, the Hamiltonian $H^\prime$ transformed under the gauge transformation is in the Hermitian symmetric gauge. 

Now we extend the concept of gauge transformations to non-Hermitian setups of imaginary magnetic fields. An important feature of imaginary magnetic fields is that the Landau gauge and the symmetric gauge cannot be connected via an ordinary gauge transformation $e^{i\chi (x,y)}$ determined by a real function $\chi (x,y)$. Instead, it is more appropriate to consider a generalized gauge transformation, $\psi^\prime_{x,y} = f(x,y)\psi_{x,y}$ with $f(x,y)$ being a nonzero complex function, which does not just multiply a phase factor but also allows scale change for the wavefunction. The Hamiltonian changes under this generalized gauge transformation, not by a unitary transformation but by a local (diagonal) similarity transformation. The Landau and symmetric gauges are related via the generalized gauge transformation $f(x,y) = e^{iBxy/2}= e^{-\mathcal{B}xy/2}$.
Since this generalized gauge transformation is a similarity transformation, the energy spectrum is invariant. Furthermore, upon the generalized gauge transformation, the product of hopping amplitudes as one goes around a plaquette of the square lattice does not change, implying that the imaginary magnetic field is also invariant. 

There are various realizations of non-Hermitian Hamiltonians, and what is observable depends on individual system that one works on.
Upon studying properties of imaginary magnetic fields, one should thus make clear distinction between what are universal properties of imaginary magnetic fields and what are gauge- and system-specific features which depend on particular realizations. We consider properties intrinsic to imaginary magnetic fields to be those invariant under the generalized gauge transformation.

\textit{Energy spectrum.---}
As well known in the study of non-Hermitian Hamiltonians, energy spectrum under periodic and open boundary conditions can take drastically different values~\cite{Hatano:1996PRL,Hatano:1997PRB,Yao:2018PRL,Okuma:2020PRL}. We should therefore analyze the energy spectrum together with the boundary conditions. We first note that, unlike the case of real magnetic fields, lattice models with imaginary magnetic fields cannot be made periodic in both $x$ and $y$ directions. For the Landau gauge, we can make the lattice periodic in the $y$-direction, but not in the $x$ direction, and for the symmetric gauge we cannot make the Hamiltonian periodic in either direction. In this paper, we call the Landau gauge with the periodic boundary condition in the $y$ direction a \textit{cylindrical configuration}.

We first consider the open boundary conditions.
As we have seen, the energy spectrum under the Landau and symmetric gauges are the same because they are related by the generalized gauge transformation. We also note that the energy spectrum is invariant upon the change of the origin of the coordinate: the spectrum is invariant under changing $x$ to $x + x_0$ and $y$ to $y+y_0$ in the hopping factors $\theta_X$ and $\theta_Y$. This invariance can be shown, for example for the symmetric gauge, by noting that the shift $x \to x+x_0$ can be realized by $f(x,y) = e^{-\mathcal{B} x_0 y/2}$ and the shift $y \to y+y_0$ by $f(x,y) = e^{-\mathcal{B} x y_0/2}$. Since we do not want the imaginary magnetic fields and their properties to depend on the origin of the coordinates, these transformation properties are desirable. 

In Fig.~\ref{fig:first}(a,b), we plot the energy spectrum in the complex plane for a lattice of size $N_x \times N_y$ with $N_x = N_y = 40$ and the imaginary magnetic field of $B = 0.001i$, and $0.01i$.
The spread of the energy spectrum along the real axis is from $-4J$ to $4J$ similar to the case of real magnetic fields. On the other hand, the spread of the energy along imaginary axis varies with the strength of the imaginary magnetic fields as well as the size of the system. 

% Figure environment removed

We next consider the case of the periodic boundary condition along the $y$-direction under the Landau gauge, namely the cylindrical configuration.
The energy spectrum can be obtained by performing the Fourier transformation in the $y$ direction and diagonalizing the Hamiltonian for each momentum separately. Writing $\psi_{x,y} = \psi_x e^{ik}$, the equation to solve is
\begin{align}
    E \psi_x = J \left\{ \psi_{x-1} + \psi_{x+1} + \left( e^{-x\mathcal{B}-ik} + e^{x\mathcal{B}+ik} \right) \psi_x \right\}. \label{eq:nonblocheig}
\end{align}
We note that this is an analog of the Harper equation for the imaginary magnetic field~\cite{Harper:1955PPS}.
In Fig.~\ref{fig:first}(c,d), we plot the energy spectrum in the cylindrical configuration with $B = 0.01i$ and two different ways to choose the origin of $x = 0$. 
We find an unexpected feature that the energy spectrum depends on the origin of the coordinates. Under the open boundary conditions, we saw that shifting of $x \to x + x_0$ is achieved by the generalized gauge transformation of $f(x,y) = e^{-\mathcal{B}x_0 y}$. However, this gauge transformation is not periodic in the $y$ direction and thus it is not compatible with the cylindrical configuration.

\textit{Asymptotic spectrum and non-Bloch band theory.---}
We now examine properties of the asymptotic energy spectrum under the open boundary conditions. 
We consider fixing the magnetic field and increasing the system size. We first take $N_x = N_y = N$, namely keeping the same length in both directions, and making $N$ large. We find that the energy spectrum does not converge as $N$ becomes large.
(Details are given in Appendix.)
This is in stark contrast to any two-dimensional Hermitian system with a periodic structure in which increasing the system size makes the energy spectrum converge to a continuous band structure. The origin of the non-convergence of our energy spectrum is because, even though the imaginary magnetic field is fixed and constant over the entire lattice, the hopping strength such as $e^{-\mathcal{B}x}$ keeps increasing in the $x$ direction, and the energy spectrum is not bounded in the imaginary direction.

Even though the spectrum does not converge keeping $N_x = N_y$, we find that the spectrum does converge as one fixes the size of one side and makes the other side become longer. In Fig.~\ref{fig:third}, we plot the energy spectrum under the open boundary conditions when $B = 0.01i$ fixing $N_x = 40$ and choosing $N_y = 50$, 100, 200. Together with the spectrum when $N_y = 40$ plotted in Fig.~\ref{fig:first}(a), one sees that the overall shape tends to stabilize as $N_y$ becomes large. We can understand this asymptotic energy spectrum in the limit of large $N_y$ by means of the non-Bloch band theory~\cite{Yokomizo:2019PRL,Yokomizo:2023PRB}. The non-Bloch band theory is a formalism to obtain the continuous energy spectrum of non-Hermitian systems under the open boundary condition. To understand the asymptotic behavior of fixing $N_x$ and making $N_y \to \infty$, we now regard the index $x$ to be an internal index of a one dimensional system elongated along the $y$ direction. 

% Figure environment removed

In order to apply the non-Bloch band theory, we perform the Fourier transformation along the $y$ direction, as done in Eq.~(\ref{eq:nonblocheig}) above. In the non-Bloch band theory, we replace $e^{ik}$ by a general complex number $\beta$, and solve the above eigenvalue equation for a given value of $E$. Writing the above eigenvalue equation as $E\vec{\psi}_X = H_X(\beta) \vec{\psi}_X$, where $\vec{\psi}_X$ is a vector whose element is $\psi_x$, solutions to the eigenvalue equation for a given value of $E$ are given by the solutions of $\det [H_X(\beta) - E] = 0$.
This equation is an algebraic equation for $\beta$ with degree $2N_x$, and thus we generally have $2N_x$ solutions of $\beta$. Writing $2N_x$ solutions of $\beta$ in the ascending order of their magnitudes and labeling them as $\beta_1$, $\beta_2$, $\cdots$, the eigenvalue $E$ belongs to a continuum of energy band if and only if $|\beta_{N_x}| = |\beta_{N_x+1}|$~\cite{Yokomizo:2019PRL}. The corresponding values of $\beta_{N_x}$ and $\beta_{N_x+1}$ form the generalized Brillouin zone in the complex plane. We find that the generalized Brillouin zone coincides with the ordinary Brillouin, namely $\beta = e^{ik}$ for real $k$, when the $x$ coordinate is labeled so that $x = 0$ is in the center of the system. (See Appendix for detailed derivation.) This implies that the solutions of Eq.~(\ref{eq:nonblocheig}) for real $k$, which are nothing but the energy spectrum of the cylindrical configuration, are the asymptotic spectrum when fixing $N_x$ and making $N_y$ large under the open boundary conditions. 
The fact that the generalized Brillouin zone coincides with the ordinary Brillouin zone implies that there is no non-Hermitian skin effect. This absence of the non-Hermitian skin effect is related to the $\mathcal{PT}$-symmetry present in the system~\cite{Yi:2020PRL}.

In Fig.~\ref{fig:third}(d), we show the continuum bands obtained from the energy spectrum of a cylindrical configuration, taking $x = 0$ to be at the center. We see that the spectra in Fig.~\ref{fig:third}(a-c) indeed approaches that of Fig.~\ref{fig:third}(d).
With different values of $\mathcal{B}$ and $N_x$, we find that there is a general structure of a continuous spectrum along the real axis and several oval structures spread along the imaginary direction, but the exact number of ovals and the spread along the imaginary direction depend on specific values of the parameters. 
We stress that the energy spectrum under the open boundary conditions does not depend on how the coordinates are chosen. Nevertheless, the asymptotic spectrum coincides with the energy spectrum in the cylindrical configuration where the coordinates are chosen in a symmetric manner.

\textit{Aharonov-Bohm effect for imaginary magnetic fields.---}
We now discuss an effect analogous to the Aharonov-Bohm effect~\cite{Aharonov:1959PR} for imaginary magnetic fields.
The non-Hermitian Aharonov-Bohm effect in a parameter space has been experimentally observed for synthetic mechanical metamaterials~\cite{Anandwade:2021arXiv,Singhal:2022arXiv}, but, to our knowledge, it has never been observed in real space.
We consider the setup where we start from a wavepacket around the center of the lattice, and then add external forces to make the wavepacket move. As the trajectory of the wavepacket forms a closed path, the change of the magnitude of the wavefunction is precisely related to the imaginary magnetic flux enclosed by the path. We now numerically demonstrate the effect.

% Figure environment removed

As an initial state, we choose a normalized Gaussian wavepacket $\psi_{x,y} \propto e^{-\{(x-x_0)^2 + (y-y_0)^2)\}/(2\sigma^2)}$ centered around the point $(x_0, y_0)$ with the spread $\sigma = 5$.
We apply a force changing sinusoidally in time, created by a potential $V_{x,y} = E_x \sin (2\pi t/T) x + E_y \sin (2\pi t/T) y$,  so that the wavepacket makes a rectangular trajectory either in the counter-clockwise or the clockwise direction. For the counter-clockwise trajectory, we apply $(E_x,E_y) = (2,0)$, $(0,1)$, $(-2,0)$, and $(0,-1)$ for $0 \le t \le T$, $T \le t \le 2T$, $2T \le t \le 3T$, and $3T \le t \le 4T$, respectively.
For the clockwise trajectory, we instead apply $(E_x,E_y) = (0,1)$, $(2,0)$, $(0,-1)$, and $(-2,0)$ for $0 \le t \le T$, $T \le t \le 2T$, $2T \le t \le 3T$, and $3T \le t \le 4T$, respectively.
We use $T = 5/J$ in the numerical simulation. 

In the following numerical simulation, we choose a lattice of size $N_x = N_y = 50$ with an imaginary magnetic field $B = 0.001i$ under the open boundary conditions. We use coordinates $x = 0, 1, 2, \cdots, 49$ and $y = 0, 1, 2, \cdots, 49$ for numerical calculation. Starting from a wavepacket centered around $(x_0, y_0) = (25,25)$, and evolving in time until $t = 4T$, the center of the wavepacket forms rectangles as plotted in Fig.~\ref{fig:fourth}(a) for the counter-clockwise trajectory. We performed simulation in both Landau and symmetric gauges, and the trajectory of the center of the wavepacket slightly differs for the two gauges. 
In Fig.~\ref{fig:fourth}(b), we plot the modulus of the wavefunction, $|\vec{\psi}| \equiv \sqrt{\sum_{x,y}|\psi_{x,y}|^2}$, as a function of time for both gauges for the counter-clockwise trajectory. The same quantity for the clockwise trajectory is also plotted in Fig.~\ref{fig:fourth}(c). We see that during the time evolution $|\vec{\psi}|$ is generally different between the two gauges, but after the closed trajectory is formed, they coincide. The decay of the norm of the wavefunction is related to the Aharonov-Bohm factor times the so-called dynamical phase. The Aharonov-Bohm factor is determined by $e^{iB A_\mathrm{Area}} = e^{-\mathcal{B}A_\mathrm{Area}} \approx 0.973$ for the counter-clockwise trajectory, where we used that the area enclosed by the trajectory is $A_\mathrm{Area} \approx 27.4$ for both gauges. There is another contribution to the change of $|\vec{\psi}|$, which is the dynamical phase factor depending on the growth/decay of the wavefunction due to the complex instantaneous eigenvalues. It turns out that, for the situation corresponding to our situation the dynamical phase factor is almost negligible, and the final values of Fig.~\ref{fig:fourth}(b,c) agree well with the Aharonov-Bohm factor. To be more precise, in order to extract solely the effect of the Aharonov-Bohm factor, we calculate the ratio of the counter-clockwise and clockwise trajectories, in which the dyamical phase contribution should cancel. We plot the result in Fig.~\ref{fig:fourth}(d), where we see a perfect agreement between the final value and the Aharonov-Bohm factor $e^{i2B A_\mathrm{Area}} = e^{-2\mathcal{B}A_\mathrm{Area}} \approx 0.947$.

\textit{Conclusion.---}
We have studied spectral and geometrical properties of two-dimensional lattices under a uniform imaginary magnetic field. Our results unveil features of imaginary magnetic fields which are intrinsically different from real magnetic fields, such as impossibility to take periodic boundary condition in both directions and non-convergence of the energy spectrum in the limit when both sides are taken large. On the other hand, there also are similarities to the real magnetic field, such as description in terms of the Harper equation and the analog of the Aharonov-Bohm effect. Although we focused on the cases of purely imaginary magnetic fields, general results presented in the paper, such as the non-Bloch band theory when increasing the length of one direction and the non-Hermitian Aharonov-Bohm effect, should be valid also for more general complex magnetic fields including both real and imaginary components.

Our results provide a starting point toward the research field of non-Hermitian magnetic fields.
We have focused on physics on lattices; properties under an imaginary magnetic field in continuous two dimensional systems is also an open field of study. Understanding properties under more general gauge fields such as complex electromagnetic fields and non-Abelian gauge fields (e.g. spin-orbit coupling)  is also left for future study.

\textit{Acknowledgements.---}
The authors would like to thank Shuichi Murakami for helpful discussion on the non-Bloch band theory.
This work is supported by JSPS KAKENHI Grant No. JP20H01845, Grant No. JP21H01007, Grant No. JP21H01084, and JST CREST Grant No.JPMJCR19T1.

\bibliography{bibliography.bib}

%\clearpage
\begin{widetext}
\appendix
\section{Appendix}
\section{I. Asymptotic energy spectrum as both sides become longer}
We examine the energy spectrum when $N_x = N_y = N$ as $N$ becomes longer under the open boundary conditions.
We plot the energy spectrum in Fig.~\ref{fig:app1} for $N = 20$, 40, 60, and 80 for $B = 0.001i$ and $B = 0.01i$. We see that the spread of the energy spectrum in the imaginary direction keeps increasing with more and more complicated structures appearing as $N$ increases, and the spectrum does not converge in the $N\rightarrow\infty$ limit. 

% Figure environment removed

\section{II. Details of the non-Bloch band theory}
Here we give a detailed discussion on the non-Bloch band theory applied to two-dimensional lattices with imaginary magnetic fields. 
We want to look for the generalized Brillouin zone and the continuum of energy bands by solving $\det [H_X(\beta) - E] = 0$.
Remembering that the spectrum under the open boundary conditions is independent of the choice of the origin of $x = 0$, we can choose the coordinates most convenient for our purpose. It turns out to be particularly useful to take a coordinate system where $x = 0$ is in the center. Namely, if $N_x$ is an odd number, we write $N_x = 2p + 1$ with a positive integer $p$ and take the $x$ coordinate to be $x = -p, -p + 1, \cdots, -1, 0, 1, \cdots, p-1, p$. If $N_x$ is an even number, we write $N_x = 2p$ with a positive integer $p$ and take $x = -p + \frac{1}{2}, \cdots, -\frac{3}{2}, -\frac{1}{2}, \frac{1}{2}, \frac{3}{2}, \cdots, p - \frac{1}{2}$.

Let us first consider the case where $N_x$ is an odd number: $N_x = 2p + 1$. We take the $x$ coordinate label as above so that $x = 0$ is in the center. Then,
\begin{align}
    H_X(\beta)
    =
    \begin{pmatrix}
        e^{-p\mathcal{B}}\beta + e^{p\mathcal{B}}\frac{1}{\beta} & 1 \\
        1 & e^{(-p+1)\mathcal{B}}\beta + e^{(p-1)\mathcal{B}}\frac{1}{\beta} & \ddots \\
        & \ddots & \ddots \\
        & & & e^{-\mathcal{B}}\beta + e^{\mathcal{B}\frac{1}{\beta}} & 1 \\
        & & & 1 & \beta + \frac{1}{\beta} & 1\\
        & & & & 1 & e^{\mathcal{B}}\beta + e^{-\mathcal{B}}\frac{1}{\beta} & \ddots \\
        & & & & & \ddots & \ddots & 1\\
        & & & & & & 1 & e^{p\mathcal{B}}\beta + e^{-p\mathcal{B}}\frac{1}{\beta}
    \end{pmatrix} ,
\end{align}
in units of $J$.
One thus sees that if $\beta$ is a solution of $\det [H_X(\beta) - E] = 0$, so is $1/\beta$, namely $\det [H_X(1/\beta) - E] = 0$.
The same holds for even $N_x$.

This implies that, upon writing $2N_x$ solutions of $\det [H_X(\beta) - E] = 0$ in an ascending order we should have $\beta_{N_x + 1} = 1/\beta_{N_x}$. From this condition together with the condition that $|\beta_{N_x}| = |\beta_{N_x + 1}|$ for $E$ to be in the continuum of energy bands~\cite{Yokomizo:2019PRL}, we arrive at the condition $|\beta_{N_x}| = |\beta_{N_x + 1}| = 1$. This implies that $\beta$ belonging to the generalized Brillouin zone can be written as $\beta = e^{ik}$ with a real number $k$, namely the generalized Brillouin zone coincides with the ordinary Brillouin zone with quasimomentum $0 \le k < 2\pi$. 

Conversely, for a given value of $\beta = e^{ik}$ with a real $k$, all $N_x$ eigenvalues of $H_X(e^{ik})$ belong to the continuum of energy bands. This is because any eigenvalue $E$ of $H_X(e^{ik})$ satisfies $\det [H_X(e^{ik}) - E] = 0$, and when writing $2N_x$ solutions of $\beta$ in an ascending order, $\beta = e^{\pm ik}$ must appear at $N_x$-th and $N_x+1$-th position. We have thus shown that the continuum of energy bands obtained upon fixing $N_x$ and making $N_y$ large is nothing but the energy spectrum in the cylindrical configuration where coordinates in the $x$ direction are taken so that $x = 0$ is placed at the center.

\end{widetext}
\end{document}