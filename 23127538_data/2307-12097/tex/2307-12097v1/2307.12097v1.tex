%%%%%%%%%%%%%%%%%%%%%%%%%%%%%%%%%%%%%%%%%%%%%%%%%%%%%%%%%%%%%%%%%%%%%%
%% Zariski Dense Orbits
%% "Joseph Silverman" <joseph_silverman@brown.edu>,
%% Hector Pasten <hpasten@gmail.com>,
%% %% . ~/.bash_profile
%%%%%%%%%%%%%%%%%%%%%%%%%%%%%%%%%%%%%%%%%%%%%%%%%%%%%%%%%%%%%%%%%%%%%%


\documentclass[12pt,reqno]{amsart}

\RequirePackage{pict2e}
\usepackage{amssymb,amscd,url,amsmath,amscd}
\usepackage{xcolor}    %% allows colors, used for edit comments
\usepackage{bbm}       %% Blackboard bold math font, such as \mathbbm{1}
\usepackage{extarrows}
%% \usepackage{lscape}    %%  \begin{landscape} Some text \end{landscape}.
\usepackage{constants} %% \C is next constant, \Cl{name} is labeled constant, \Cr{name} references old constant
\usepackage{stackrel}
\usepackage{tikz}
\usetikzlibrary{arrows.meta}
\usepackage{centernot}


%%%%%%%%%%%%%%%%%%%%%%%%%%%%%%%%%%%%%%%%%%%%%%%%%%%%%%%%%%%%%%%%%%%%%%
%%  For hyperref, uncomment these lines. Works in Overleaf
\RequirePackage[pdfa, psdextra]{hyperref}  
\hypersetup{
    colorlinks=true,
    linkcolor=blue,
    urlcolor=cyan,
    }
%%%%%%%%%%%%%%%%%%%%%%%%%%%%%%%%%%%%%%%%%%%%%%%%%%%%%%%%%%%%%%%%%%%%%%
\providecommand{\texorpdfstring}[2]{#1}  %% Prevent error if hyperref not loaded.


%%%%%%%%%%%%%%%%%%%%%%%%%%%%%%%%%%%%%%%%%%%%%%%%%%%%%%%%%%%%%%%%%%%%%%
%% Typeset using double-spaced output
%% \renewcommand\baselinestretch{1.67}

\begin{document}

\allowdisplaybreaks

%%%%%%%%%%%%%%%%%%%%%%%%%%%%%%%%%%%%%%%%%%%%%%%%%%%%%%%%%%%%%%%%%%%%%%
%% Title and Author Information

\title[Propagation of Zariski Dense Orbits]
{Propagation of Zariski Dense Orbits}
\date{\today}

\author[H. Pasten]{Hector Pasten}
\email{hpasten@gmail.com}
\address{Facultad de Mathm{\'a}ticas, Pontificia Universidad Católica de Chile, Vicu{\~n}a Mackenna 4860, Macul, Chile}

\author[J.H. Silverman]{Joseph H. Silverman}
\email{joseph\_silverman@.brown.edu}
\address{Department of Mathematics, Box 1917
  Brown University, Providence, RI 02912 USA.
  ORCID: 0000-0003-3887-3248}

% 37P05 Arithmetic dynamical systems involving polynomial and rational maps
% 37P15 Dynamical systems over global ground fields
% 37P30 Height functions
% 37P55 Arithmetic dynamics on general algebraic varieties
\subjclass[2010]{Primary: 37P15; Secondary: 37P05, 37P30, 37P55}
\keywords{arithmetic dynamics, orbit propagation}
\thanks{
Pasten's research was supported by ANID FONDECYT Regular Grant 1230507 from Chile. Silverman's research was supported by Simons Collaboration Grant \#712332. Both authors' research was supported by the National Science Foundation under Grant No.\ 1440140 while the authors were in residence at the Mathematical Sciences Research Institute in Berkeley, California, during the spring of 2023. 
}




%%%%%%%%%%%%%%%%%%%%%%%%%%%%%%%%%%%%%%%%%%%%%%%%%%%%%%%%%%%%%%%%%%%%%%

\newcommand{\JOE}[1]{{\textup{\color{blue} $\bigstar$ \textsf{{\bf Joe:} [#1]}}}}
\newcommand{\HECTOR}[1]{{\textup{\color{blue} $\bigstar$ \textsf{{\bf Hector:} [#1]}}}}

\hyphenation{ca-non-i-cal semi-abel-ian}

%%%%%%%%%%%%%%%%%%%%%%%%%%%%%%%%%%%%%%%%%%%%%%%%%%%%%%%%%%%%%%%%%%%%%%
% Theorem environments

\newtheorem{theorem}{Theorem}[section]
\newtheorem{lemma}[theorem]{Lemma}
\newtheorem{sublemma}[theorem]{Sublemma}
\newtheorem{conjecture}[theorem]{Conjecture}
\newtheorem{protoconjecture}[theorem]{Proto-Conjecture}
\newtheorem{proposition}[theorem]{Proposition}
\newtheorem{corollary}[theorem]{Corollary}

\theoremstyle{definition}
\newtheorem*{claim}{Claim}
\newtheorem{definition}[theorem]{Definition}
\newtheorem*{intuition}{Intuition}
\newtheorem{example}[theorem]{Example}
\newtheorem{remark}[theorem]{Remark}
\newtheorem{question}[theorem]{Question}

\theoremstyle{remark}
% The * surpresses numbering
\newtheorem*{acknowledgement}{Acknowledgements}
\newtheorem{exercise}[theorem]{exercise}


%%%%%%%%%%%%%%%%%%%%%%%%%%%%%%%%%%%%%%%%%%%%%%%%%%%%%%%%%%%%%%%%%%%%%%

%%%%%%%% Set Up Environment for Notation %%%%%%%%%%%%%%
% This is currently set to allow quite wide items to be defined
\newenvironment{notation}[0]{%
  \begin{list}%
    {}%
    {\setlength{\itemindent}{0pt}
     \setlength{\labelwidth}{4\parindent}
     \setlength{\labelsep}{\parindent}
     \setlength{\leftmargin}{5\parindent}
     \setlength{\itemsep}{0pt}
     }%
   }%
  {\end{list}}

%%%%%%%% Set Up Environment for Parts in Theorems %%%%%%%%%%%%%%
\newenvironment{parts}[0]{%
  \begin{list}{}%
    {\setlength{\itemindent}{0pt}
     \setlength{\labelwidth}{1.5\parindent}
     \setlength{\labelsep}{.5\parindent}
     \setlength{\leftmargin}{2\parindent}
     \setlength{\itemsep}{0pt}
     }%
   }%
  {\end{list}}
% Use \Part{(a)}, instead of \item[(a)], to ensure upright font
\newcommand{\Part}[1]{\item[\upshape#1]}

%%%%%%%% Set Up Macro for Cases and Claims %%%%%%%%%%%%%%
\newcommand{\Case}[2]{\paragraph{\textbf{\boldmath Case #1: #2.}}\hfil\break\ignorespaces}
\newcommand{\Claim}[2]{\paragraph{\textbf{\boldmath Claim #1: #2}}\hfil\break\ignorespaces}


%%%%%%%%%%%%%%%%%%
% Greek Alphabet %
%%%%%%%%%%%%%%%%%%
\renewcommand{\a}{\alpha}
\newcommand{\bfalpha}{{\boldsymbol{\alpha}}}
\renewcommand{\b}{\beta}
\newcommand{\bfbeta}{{\boldsymbol{\beta}}}
\newcommand{\g}{\gamma}
\renewcommand{\d}{\delta}
\newcommand{\e}{\epsilon}
\newcommand{\f}{\varphi}
\newcommand{\bfphi}{{\boldsymbol{\f}}}
\renewcommand{\l}{\lambda}
\renewcommand{\k}{\kappa}
\newcommand{\lhat}{\hat\lambda}
\newcommand{\m}{\mu}
\newcommand{\bfmu}{{\boldsymbol{\mu}}}
\renewcommand{\o}{\omega}
\newcommand{\bfpi}{{\boldsymbol{\pi}}}
\renewcommand{\r}{\rho}
\newcommand{\bfrho}{{\boldsymbol{\rho}}}
\newcommand{\rbar}{{\bar\rho}}
\newcommand{\s}{\sigma}
\newcommand{\sbar}{{\bar\sigma}}
\renewcommand{\t}{\tau}
\newcommand{\z}{\zeta}
\newcommand{\D}{\Delta}
\newcommand{\G}{\Gamma}
\newcommand{\F}{\Phi}
\renewcommand{\L}{\Lambda}

%%%%%%%%%%%%%%%%%%%%
% Fraktur Alphabet %
%%%%%%%%%%%%%%%%%%%%
\newcommand{\ga}{{\mathfrak{a}}}
\newcommand{\gA}{{\mathfrak{A}}}
\newcommand{\gb}{{\mathfrak{b}}}
\newcommand{\gB}{{\mathfrak{B}}}
\newcommand{\gc}{{\mathfrak{c}}}
\newcommand{\gM}{{\mathfrak{M}}}
\newcommand{\gn}{{\mathfrak{n}}}
\newcommand{\gp}{{\mathfrak{p}}}
\newcommand{\gP}{{\mathfrak{P}}}
\newcommand{\gS}{{\mathfrak{S}}}
\newcommand{\gq}{{\mathfrak{q}}}

%%%%%%%%%%%%%%%%%%%
% Barred Alphabet %
%%%%%%%%%%%%%%%%%%%
\newcommand{\Abar}{{\bar A}}
\newcommand{\Ebar}{{\bar E}}
\newcommand{\kbar}{{\bar k}}
\newcommand{\Kbar}{{\bar K}}
\newcommand{\Pbar}{{\bar P}}
\newcommand{\Sbar}{{\bar S}}
\newcommand{\Tbar}{{\bar T}}
\newcommand{\gbar}{{\bar\gamma}}
\newcommand{\lbar}{{\bar\lambda}}
\newcommand{\ybar}{{\bar y}}
\newcommand{\phibar}{{\bar\f}}
\newcommand{\nubar}{{\overline\nu}}

%%%%%%%%%%%%%%%%%%%%%%%%%
% Calligraphic Alphabet %
%%%%%%%%%%%%%%%%%%%%%%%%%
\newcommand{\Acal}{{\mathcal A}}
\newcommand{\Bcal}{{\mathcal B}}
\newcommand{\Ccal}{{\mathcal C}}
\newcommand{\Dcal}{{\mathcal D}}
\newcommand{\Ecal}{{\mathcal E}}
\newcommand{\Fcal}{{\mathcal F}}
\newcommand{\Gcal}{{\mathcal G}}
\newcommand{\Hcal}{{\mathcal H}}
\newcommand{\Ical}{{\mathcal I}}
\newcommand{\Jcal}{{\mathcal J}}
\newcommand{\Kcal}{{\mathcal K}}
\newcommand{\Lcal}{{\mathcal L}}
\newcommand{\Mcal}{{\mathcal M}}
\newcommand{\Ncal}{{\mathcal N}}
\newcommand{\Ocal}{{\mathcal O}}
\newcommand{\Pcal}{{\mathcal P}}
\newcommand{\Qcal}{{\mathcal Q}}
\newcommand{\Rcal}{{\mathcal R}}
\newcommand{\Scal}{{\mathcal S}}
\newcommand{\Tcal}{{\mathcal T}}
\newcommand{\Ucal}{{\mathcal U}}
\newcommand{\Vcal}{{\mathcal V}}
\newcommand{\Wcal}{{\mathcal W}}
\newcommand{\Wbar}{{\overline{\mathcal W}}}
\newcommand{\Xcal}{{\mathcal X}}
\newcommand{\Ycal}{{\mathcal Y}}
\newcommand{\Zcal}{{\mathcal Z}}

%%%%%%%%%%%%%%%%%%%%%%%%%%%%
% Blackboard Bold Alphabet %
%%%%%%%%%%%%%%%%%%%%%%%%%%%%
\renewcommand{\AA}{\mathbb{A}}
\newcommand{\BB}{\mathbb{B}}
\newcommand{\CC}{\mathbb{C}}
\newcommand{\FF}{\mathbb{F}}
\newcommand{\EE}{\mathbb{E}}
\newcommand{\GG}{\mathbb{G}}
\newcommand{\NN}{\mathbb{N}}
\newcommand{\PP}{\mathbb{P}}
\newcommand{\QQ}{\mathbb{Q}}
\newcommand{\RR}{\mathbb{R}}
\newcommand{\TT}{\mathbb{T}}
\newcommand{\ZZ}{\mathbb{Z}}

%%%%%%%%%%%%%%%%%%%%%%%%%%
% Boldface Math Alphabet %
%%%%%%%%%%%%%%%%%%%%%%%%%%
\newcommand{\bfa}{{\boldsymbol a}}
\newcommand{\bfb}{{\boldsymbol b}}
\newcommand{\bfc}{{\boldsymbol c}}
\newcommand{\bfd}{{\boldsymbol d}}
\newcommand{\bfe}{{\boldsymbol e}}
\newcommand{\ee}{{\boldsymbol{e}}} %% exp(2 \pi i .)
\newcommand{\bff}{{\boldsymbol f}}
\newcommand{\bfg}{{\boldsymbol g}}
\newcommand{\bfi}{{\boldsymbol i}}
\newcommand{\bfj}{{\boldsymbol j}}
\newcommand{\bfk}{{\boldsymbol k}}
\newcommand{\bfm}{{\boldsymbol m}}
\newcommand{\bfn}{{\boldsymbol n}}
\newcommand{\bfp}{{\boldsymbol p}}
\newcommand{\bfr}{{\boldsymbol r}}
\newcommand{\bfs}{{\boldsymbol s}}
\newcommand{\bft}{{\boldsymbol t}}
\newcommand{\bfu}{{\boldsymbol u}}
\newcommand{\bfv}{{\boldsymbol v}}
\newcommand{\bfw}{{\boldsymbol w}}
\newcommand{\bfx}{{\boldsymbol x}}
\newcommand{\bfy}{{\boldsymbol y}}
\newcommand{\bfz}{{\boldsymbol z}}
\newcommand{\bfA}{{\boldsymbol A}}
\newcommand{\bfF}{{\boldsymbol F}}
\newcommand{\bfB}{{\boldsymbol B}}
\newcommand{\bfD}{{\boldsymbol D}}
\newcommand{\bfG}{{\boldsymbol G}}
\newcommand{\bfI}{{\boldsymbol I}}
\newcommand{\bfM}{{\boldsymbol M}}
\newcommand{\bfP}{{\boldsymbol P}}
\newcommand{\bfQ}{{\boldsymbol Q}}
\newcommand{\bfT}{{\boldsymbol T}}
\newcommand{\bfU}{{\boldsymbol U}}
\newcommand{\bfX}{{\boldsymbol X}}
\newcommand{\bfY}{{\boldsymbol Y}}
\newcommand{\bfzero}{{\boldsymbol{0}}}
\newcommand{\bfone}{{\boldsymbol{1}}}

%%%%%%%%%%%%%%%%%%%%%%%%%%%%%%
% Miscellaneous New Commands %
%%%%%%%%%%%%%%%%%%%%%%%%%%%%%%
\newcommand{\Aut}{\operatorname{Aut}}
\newcommand{\Berk}{{\textup{Berk}}}
\newcommand{\Birat}{\operatorname{Birat}}
\newcommand{\characteristic}{\operatorname{char}}
\newcommand{\CircleNum}[1]{\raisebox{.5pt}{\textcircled{\raisebox{-.9pt} {\small#1}}}}
\newcommand{\Closure}{\operatorname{\textsf{Cl}}}
\newcommand{\codim}{\operatorname{codim}}
\newcommand{\Count}{\operatorname{\textsf{N}}}
\newcommand{\CountSet}{\operatorname{\mathfrak{N}}}
\newcommand{\Crit}{\operatorname{Crit}}
\newcommand{\crit}{{\textup{crit}}}
\newcommand{\critwt}{\operatorname{critwt}} % valency of a portrait
\newcommand{\Cycle}{\operatorname{Cycles}}
\newcommand{\diag}{\operatorname{diag}}
\newcommand{\dimEnd}{{M}}  % Dimension of End_d^N
\newcommand{\dimpadic}{\dim_{\textup{$p$-adic}}}
\newcommand{\dimKrull}{\dim_{\textup{Krull}}}
\newcommand{\Disc}{\operatorname{Disc}}
\newcommand{\Div}{\operatorname{Div}}
\newcommand{\Df}{{Df}}  % adjust spacing?
\newcommand{\Dom}{\operatorname{Dom}}
\newcommand{\dyn}{{\textup{dyn}}}
\newcommand{\End}{\operatorname{End}}
\newcommand{\PortEndPt}{{\textup{endpt}}} 
\newcommand{\END}{\smash[t]{\overline{\operatorname{End}}}\vphantom{E}}
\newcommand{\EndPoint}{E}  % endpoint of a component with no cycle
\newcommand{\Expectation}{\operatornamewithlimits{\mathbb{E}}} %% Variance
\newcommand{\ExtOrbit}{\mathcal{EO}} %% Extended orbit
\newcommand{\Fbar}{{\bar{F}}}
\newcommand{\fib}{{\textup{fib}}}
\newcommand{\Fix}{\operatorname{Fix}}
\newcommand{\Fiber}{\operatorname{Fiber}}
\newcommand{\FOD}{\operatorname{FOD}}
\newcommand{\FOM}{\operatorname{FOM}}
\newcommand{\Frame}{\operatorname{Fr}}
\newcommand{\Gal}{\operatorname{Gal}}
\newcommand{\genus}{\operatorname{genus}}
\newcommand{\GITQuot}{/\!/}
\newcommand{\GL}{\operatorname{GL}}
\newcommand{\Gp}{\Gcal}
\newcommand{\GR}{\operatorname{\mathcal{G\!R}}}
\newcommand{\grand}{{\textup{grand}}} % for grand orbit
%% \newcommand{\full}{{\textup{full}}} % for full orbit
\newcommand{\full}{\pm} % for full orbit
\newcommand{\hhat}{{\hat h}}
\newcommand{\hplus}{h^{\scriptscriptstyle+}}
\newcommand{\Hom}{\operatorname{Hom}}
\newcommand{\Index}{\operatorname{Index}}
\newcommand{\Image}{\operatorname{Image}}
\newcommand{\Isog}{\operatorname{Isog}}
\newcommand{\Isom}{\operatorname{Isom}}
\newcommand{\Jac}{\operatorname{Jac}}
\newcommand{\Ker}{{\operatorname{Ker}}}
\newcommand{\Ksep}{K^{\text{sep}}}  %% separable closure of K
\newcommand{\Length}{\operatorname{Length}}
\newcommand{\Lift}{\operatorname{Lift}}
\newcommand{\limstar}{\lim\nolimits^*}
\newcommand{\limstarn}{\lim_{\hidewidth n\to\infty\hidewidth}{\!}^*{\,}}
\def\LS#1#2{{\genfrac{(}{)}{}{}{#1}{#2}}} % Legendre symbol
\newcommand{\Mat}{\operatorname{Mat}}
\newcommand{\maxplus}{\operatornamewithlimits{\textup{max}^{\scriptscriptstyle+}}}
\newcommand{\MOD}[1]{~(\textup{mod}~#1)}
\newcommand{\Model}{\operatorname{Model}}
\newcommand{\Mor}{\operatorname{Mor}}
\newcommand{\Moduli}{\mathcal{M}}
\newcommand{\MODULI}{\overline{\mathcal{M}}}
\newcommand{\Mult}{\operatorname{\textup{\textsf{Mult}}}}
\newcommand{\Norm}{{\operatorname{\mathsf{N}}}}
\newcommand{\notdivide}{\nmid}
\newcommand{\normalsubgroup}{\triangleleft}
\newcommand{\NS}{\operatorname{NS}}
\newcommand{\onto}{\twoheadrightarrow}
\newcommand{\ord}{\operatorname{ord}}
\newcommand{\Orbit}{\mathcal{O}}
\newcommand{\Pcase}[3]{\par\noindent\framebox{$\boldsymbol{\Pcal_{#1,#2}}$}\enspace\ignorespaces}
\newcommand{\Per}{\operatorname{Per}}
\newcommand{\Perp}{\operatorname{Perp}}
\newcommand{\PrePer}{\operatorname{PrePer}}
\newcommand{\PGL}{\operatorname{PGL}}
\newcommand{\Pic}{\operatorname{Pic}}
\newcommand{\prim}{\textup{prim}}
\newcommand{\Prob}{\operatorname{Prob}}
\newcommand{\Proj}{\operatorname{Proj}}
\newcommand{\Qbar}{{\bar{\QQ}}}
\newcommand{\QR}[2]{\left( \dfrac{#1}{#2} \right) }
\newcommand{\rank}{\operatorname{rank}}
\newcommand{\Rat}{\operatorname{Rat}}
\newcommand{\Resultant}{\operatorname{Res}}
\newcommand{\Residue}{\operatorname{Residue}} %% residue
\renewcommand{\setminus}{\smallsetminus}
\newcommand{\sgn}{\operatorname{sgn}}
\newcommand{\SL}{\operatorname{SL}}
\newcommand{\Sing}{\operatorname{Sing}} %% singular locus
\newcommand{\Span}{\operatorname{Span}}
\newcommand{\Spec}{\operatorname{Spec}}
\renewcommand{\ss}{{\textup{ss}}}
\newcommand{\stab}{{\textup{stab}}}
\newcommand{\Stab}{\operatorname{Stab}}
\newcommand{\Support}{\operatorname{Supp}}
\newcommand{\Sym}{\operatorname{Sym}}  %% Symmetric group
\newcommand{\tors}{{\textup{tors}}}
\newcommand{\Trace}{\operatorname{Trace}}
\newcommand{\trianglebin}{\mathbin{\triangle}} % symmetric set difference
\newcommand{\tr}{{\textup{tr}}} % for K/k trace
\newcommand{\UHP}{{\mathfrak{h}}}    % Upper half plane
\newcommand{\val}{\operatorname{val}} % valency of a portrait
\newcommand{\Var}{\operatornamewithlimits{Var}} %% Variance
\newcommand{\wt}{\operatorname{wt}} %% weight of a portrait
\newcommand{\<}{\langle}
\renewcommand{\>}{\rangle}

\newcommand{\pmodintext}[1]{~\textup{(mod}~#1\textup{)}}
\newcommand{\ds}{\displaystyle}
\newcommand{\longhookrightarrow}{\lhook\joinrel\longrightarrow}
\newcommand{\longonto}{\relbar\joinrel\twoheadrightarrow}
\newcommand{\SmallMatrix}[1]{%
  \left(\begin{smallmatrix} #1 \end{smallmatrix}\right)}

\begin{abstract}
Let $X/K$ be a smooth projective variety defined over a number field, and let $f:X\to{X}$ be a morphism defined over $K$. We formulate a number of statements of varying strengths asserting, roughly, that if there is at least one point $P_0\in{X(K)}$ whose $f$-orbit $\mathcal{O}_f(P_0):=\bigl\{f^n(P):n\in\mathbb{N}\bigr\}$ is Zariski dense, then there are many such points.  For example, a weak conclusion would be that $X(K)$ is not the union of finitely many (grand) $f$-orbits, while a strong conclusion would be that any set of representatives for the Zariski dense grand $f$-orbits is Zariski dense. We prove statements of this sort for various classes of varieties and maps, including projective spaces, abelian varieties, and surfaces.
\end{abstract}

\maketitle

\tableofcontents

%%%%%%%%%%%%%%%%%%%%%%%%%%%%%%%%%%%%%%%%%%%%%%%%%%%%%%%%%%%%%%%%%%%%%%
\section{Introduction}
\label{section:introduction}
%%%%%%%%%%%%%%%%%%%%%%%%%%%%%%%%%%%%%%%%%%%%%%%%%%%%%%%%%%%%%%%%%%%%%%
Let~$K$ be a number field, and let
\[
f:X\to{X}
\]
be an endomorphism of a smooth projective variety defined over~$K$. The theme of this article is that if there is even a single point~$P_0\in{X(K)}$ whose forward $f$-orbit
\[
\Orbit_f(P_0) := \bigl\{ f^n(P_0) : n \ge 0 \bigr\}
\]
is Zariski dense in~$X$, then~$X(K)$ contains lots of intrinsically (large) different orbits. There are many ways to turn this vague idea, which we call an \emph{orbit propagation principle}, into a precise statement.  We briefly describe two orbit propagation principles, and we refer the reader to Section~\ref{section:notdefconj}, and especially Table~\ref{table:propogationstatements}, for many others. In order to state these principles, we recall that the \emph{grand orbit} of a point~$P\in{X(\Kbar)}$ is the set of points whose orbits eventually merge with the orbit of~$P$, i.e.,
\[
\Orbit_f^\grand(P) 
:= \bigl\{ Q\in X(\Kbar) : \Orbit_f(P)\cap\Orbit_f(Q)\ne\emptyset\bigr\}.
\]

\begin{conjecture}[Weak Orbit Propagation Principles (B1) and (C1)]
\label{conjecture:weakorbitpropagationprinciple}
%\leavevmode\newline
Assume that~$X(K)$ has at least one Zariski dense~$f$-orbit. Then there is a finite extension~$K'/K$ such that for all~$P_1,\ldots,P_r\in{X(K')}$\textup:
\begin{parts}
\Part{(a)}
$X(K') \setminus \bigl( \Orbit_f(P_1) \cup\cdots\cup \Orbit_f(P_r) \bigr)$
is Zariski dense in $X$.
\Part{(b)}
$X(K') \setminus \bigl( \Orbit_f^\grand(P_1) \cup\cdots\cup \Orbit_f^\grand(P_r) \bigr)$
is Zariski dense in $X$.
\end{parts}
\textup{Intuition for Conjecture~\ref{conjecture:weakorbitpropagationprinciple}:} A finite union of \textup(grand\textup) orbits can cover only a small portion of the $K'$-rational points.
\end{conjecture}

\begin{conjecture}[Strong Orbit Propagation Principle (C3$\forall$)]
\label{conjecture:strongorbitpropagationprinciple}
\leavevmode\newline
Assume that~$X(K)$ has at least one Zariski dense~$f$-orbit. Then there is a finite extension~$K'/K$ with the following property. If~$\Qcal\subset{X(K')}$ is a set of points having the following boxed property, then~$\Qcal$ is Zariski dense in~$X$.
\[
\framebox{
\begin{tabular}{l}
For every $P\in{X(K')}$ such that~$\Orbit_f(P)$ is Zariski dense,\\
there exists a $Q\in\Qcal$ such that $\Orbit_f^\grand(Q)=\Orbit_f^\grand(P)$,\\
i.e., $\Qcal$ contains a set of representatives for the grand\\
$f$-orbits that contain a point of~$X(K')$.
\end{tabular}
}
\]
\textup{Intuition for Conjecture~\ref{conjecture:strongorbitpropagationprinciple}:} 
The existence of one Zariski dense forward orbit implies that there are lots of distinct Zariski dense grand orbits, and that any set of representatives for these grand orbits must be a widely scattered set of points.
\end{conjecture}

To what extent are propagation principles of varying strengths true? We have no general answer, but the bulk of this article is devoted to proving orbit propagation results  for various classes of varieties and maps. We state here some exemplary results, and we refer the reader to Theorem~\ref{theorem:orbitpropagationprovencases} for a complete description of the results in this paper and to Sections~\ref{section:projectivespace}--\ref{section:surfaces} for the proofs. 

\begin{theorem}
\label{theorem:exampleresults}
\leavevmode
\begin{parts}
\Part{(a)}
The Weak Orbit Propagation Principle \textup(Conjecture~\textup{\ref{conjecture:weakorbitpropagationprinciple}(a)}\textup) is true
for endomorphisms of smooth projective surfaces
and \textup(Conjecture~\textup{\ref{conjecture:weakorbitpropagationprinciple}(b)}\textup) for endomorphisms of projective space~$\PP^N$.
\Part{(b)}
The Strong Orbit Propagation Principle \textup(Conjecture~\textup{\ref{conjecture:strongorbitpropagationprinciple}}\textup) is true for  semisimple linear endomorphisms of~$\PP^N$, for all linear endomorphisms of~$\PP^2$, and for endomorphisms of geometrically simple abelian varieties.
\end{parts}
\end{theorem}
\begin{proof}
(a)\enspace
See Theorem~\ref{theorem:surfaces} for surfaces and Theorem~\ref{theorem:projectivespacedeg2}(a) for~$\PP^N$.
\par\noindent(b)\enspace
See Theorem~\ref{theorem:projectivespacedeg2}(d) for~$\PP^N$, Theorem~\ref{theorem:autsofP2} for~$\PP^2$, and Theorem~\ref{theorem:simpleabelianvariety} for abelian varieties.
\end{proof}


\begin{remark}
We note that although we are only able to prove the Strong Orbit Propagation Principle (Conjecture~\ref{conjecture:strongorbitpropagationprinciple}) in a limited number of cases, we do not know of an example of even a single endomorphism of a smooth projective variety for which it fails to be true.
\end{remark}

\begin{remark}
Our motivation for an orbit propagation principle arose from a~$35$-year old conjecture~\cite{MR1009803} of the first author. Very roughly, the conjecture says that if~$X(K)$ has infinitely many points, then ignoring ``error terms,'' the height counting function of~$X(K)$ should grow like a power of~$T$ or a power of~$\log(T)$. (See Section~\ref{section:motivation} for a precise statement.) On the other hand, a conjecture of Kawaguchi and the second author~\cite{MR3456169} suggests that if~$f$ is sufficiently dynamically complicated (formally, if the dynamical degree of~$f$ satisfies $\d(f)>1$), then the height counting function of~$\Orbit_f(P)$ of the orbit of a point grows no faster than~$\log\log(T)$. Hence~$X(K)$ should have lots of different orbits. This vague idea led first to various weak conjectures such as~(B1) and~(C1) in Table~\ref{table:propogationstatements}, and eventually to stronger statements culminating in the strong orbit propagation principle described in Conjecture~\ref{conjecture:strongorbitpropagationprinciple}.
\end{remark}

\begin{remark}
In this paper we only consider orbit propagation principles for self-morphisms~$f$ of smooth projective varieties~$X$. One might ask to what extent similar statements might hold for rational maps of non-smooth quasi-projective varieties, but we have not considered such questions.
\end{remark}

We now describe the structure of this paper. We start in Section~\ref{section:notdefconj} with definitions, notation, the description of a number of different orbit propagation principles, and a statement of our main results. Section~\ref{section:motivation} describes in more detail the motivation that led to the idea of orbit propagation. Then Sections~\ref{section:projectivespace}--\ref{section:surfaces} contain the proofs of our main results, using a variety of techniques and tools that include height counting functions, $p$-adic methods, algebro-geometric techniques, and deep theorems of Faltings et al.\ on the intersection of subvarieties of abelian varieties with subgroups of finite type. We include two appendices. Appendix~\ref{appendix:conditionalresults} proves stronger propagation results for non-linear endomorphisms of~$\PP^N$ that are conditional on various other arithmetic dynamical conjectures; see Theorem~\ref{theorem:projectivespacedeg2conditional}. Appendix~\ref{appendix:elementaryresults} contains a number of auxiliary results, including elementary properties of orbits (Lemma~\ref{lemma:orbitelemproperties}), elementary implications relating the various orbit propagation properties (Proposition~\ref{proposition:propagationimplications} and Lemma~\ref{lemma:C2forallZDimpliesC3forall}), and a weak height counting estimate for~$\PP^N(K)$ (Lemma~\ref{lemma:ctYKTleCTN}).

\begin{acknowledgement}
We would like to thank Brendan Hassett and Yohsuke Matsuzawa for their assistance. 
\end{acknowledgement}


%%%%%%%%%%%%%%%%%%%%%%%%%%%%%%%%%%%%%%%%%%%%%%%%%%%%%%%%%%%%%%%%%%%%%%
\section{Notation, Definitions, Conjectures and Main Results}
\label{section:notdefconj}
%%%%%%%%%%%%%%%%%%%%%%%%%%%%%%%%%%%%%%%%%%%%%%%%%%%%%%%%%%%%%%%%%%%%%%

% \newcommand{\feq}{\stackrel[\scriptscriptstyle\textup{$f$-orbit}]
% {\scriptscriptstyle\textup{full}}{\equiv}}
% \newcommand{\fgeq}{\stackrel[\scriptscriptstyle\textup{$f$-orbit}]
% {\scriptscriptstyle\textup{grand}}{\equiv}}
\newcommand{\fgeq}{\mathbin{\equiv_f}}
\newcommand{\dense}{{\textup{dense}}}

\begin{definition}
\label{definition:orbits}
Throughout this article, we fix the following notation:
\begin{align*}
    X&\quad\text{a smooth projective variety with $\dim(X)\ge1$.} \\
    f&\quad\text{an endomorphism $f:X\to X$.}
\end{align*}
Let~$P\in{X}$ be a geometric point of~$X$. We define various sorts of orbits of~$P$.\footnote{Our set of natural numbers~$\NN$ includes~$0$, so~$\Orbit_f(P)$ includes the point~$P=f^0(P)$.}
\begin{description}
\item[(Forward) $f$-orbit]
\[
\Orbit_f(P) = \Orbit^+_f(P) := \bigl\{ f^n(P) : n\in\NN \bigr\}.
\]
\item[Backward $f$-orbit]
\[
% \Orbit^-_f(P) := \bigl\{ Q\in X : P=f^n(Q)~\text{for some}~n\in\NN \bigr\}.
\Orbit^-_f(P) := \bigl\{ Q\in X : P\in\Orbit_f(Q) \bigr\}.
\]
\item[Full $f$-orbit]
\[
\Orbit^\full_f(P) := \Orbit_f^+(P) \cup \Orbit_f^-(P).
\]
\item[Grand $f$-orbit]
\[
\Orbit_f^\grand(P) = \bigl\{ Q\in X : \Orbit_f(P)\cap\Orbit_f(Q)\ne\emptyset \bigr\}.
\]
\end{description}
\end{definition}

\begin{definition}
Let~$P,Q\in{X}$. If 
\[
\Orbit_f(P)\cap\Orbit_f(Q)\ne\emptyset,
\]
then we say that~$P$ and~$Q$ are
\[
\text{\emph{grand $f$-orbit equivalent},\quad and we write\quad $P\fgeq{Q}$.}
\]
\end{definition}

\begin{definition}
We denote the set of points with Zariski dense $f$-orbit by
\[
X_f^\dense := \bigl\{ P\in X : \overline{\Orbit_f(P)}=X \bigr\}.
\]
\end{definition}

We refer the reader to Section~\ref{appendix:elementaryresults} for a proof that 
% $f$-orbit equivalance and 
grand $f$-orbit equivalence is an equivalence relation on the set of points of~$X$, and for various other elementary properties of the different types of orbits.

\begin{conjecture}[\textup{Orbit Propagation}]
\label{conjecture:propagation}
Let~$\Xcal$ be a set of \textup(smooth projective connected\textup) varieties defined over~$\Qbar$, and for each~$X\in\Xcal$, let~$\Fcal_X$ be a collection of~$\Qbar$-morphisms~$X\to{X}$. We say that~$(\Xcal,\Fcal)$ has an \emph{orbit propagation property} if for every~$X\in\Xcal$ and every~$f\in\Fcal_X$, the following implication holds\textup:
\[
\left(\begin{tabular}{@{}l@{}}
$X(\Qbar)$ contains at\\
least one Zariski\\
dense $f$-orbit\\
\end{tabular}\right)
\quad\Longrightarrow\quad
\left(
\text{\parbox{.5\hsize}{\raggedright
there is a number field $K/\QQ$
that is a field of definition
for $X$ and $f$ such that $X(K)$
contains ``many large'' $f$-orbits
}}\right).
\]
The words ``many large'' may be quantified using the various orbit statements in Table~\textup{\ref{table:propogationstatements}}. So we say that~$(\Xcal,\Fcal)$ has an  \emph{orbit propagation property of a specified type} if for every~$X\in\Xcal$ and every~$f\in\Fcal_X$, it satisfies\textup:
% \[
% X_f^\dense(\Qbar)\ne\emptyset
% \quad\Longrightarrow\quad
% \left(\begin{tabular}{@{}l@{}}
% there is a number field $K/\QQ$\\
% that is a field of definition\\
% for $X$ and $f$ such that \\
% a specific \textup{(B)} or \textup{(C)} state-\\
% ment in Table~\textup{\ref{table:propogationstatements}} is valid for $X(K)$\\
% \end{tabular}\right).
% \]
\[
X_f^\dense(\Qbar)\ne\emptyset
\quad\Longrightarrow\quad
\left(
\text{\parbox{.6\hsize}{\raggedright
there is a number field~$K/\QQ$ that is a field of definition for~$X$ and~$f$ such that one of the~\textup{(B)} or~\textup{(C)} statements in Table~\textup{\ref{table:propogationstatements}} is valid for~$X(K)$
}}
\right).
\]
\end{conjecture}

\begin{remark}
\label{remark:(A)implies}
When we refer to Table~\ref{table:propogationstatements} and make an assertion such as
\[
\text{(A)} \quad\Longrightarrow\quad \text{(C1)},
\]
what we always mean is that if there is a number field~$K$ such that~(A) is true, then possibly after replacing~$K$ with a finite extension, the statement~(C1) is also true.
\end{remark}
\begin{remark}
It is conjectured that $X_f^\dense(\Qbar)\ne\emptyset$ if and only if~$f$ is not rationally fibered over~$\PP^1$, i.e., if and only if~$\Qbar(X)$ contains no non-constant~$f$-invariant functions. See~\cite[Conjecture~1.2]{MR2862064},~\cite[Conjecture~5.10]{arxiv0901.2352} and~\cite{MR2408228}.
% There's a nice summary of the Zariski dense orbit conjecture (of various flavors) in Section 11 of Sheng Meng and De-Qi Zhang's article for the Simons Symposium: ``Advances In The Equivariant Minimal Model Program And Their Applications In Complex And Arithmetic Dynamics.''  They call it a ``long-standing conjecture'' without attributing it to anyone.
\end{remark}

\begin{table}[t]
\framebox{\vbox{%
% \small
\begin{parts}
\Part{(A)\enspace}
There is at least one Zariski dense $f$-orbit in
$X(K)$, i.e., $$X_f^\dense(K)\ne\emptyset.$$
\par\vspace{-7pt}\hbox to.98\hsize{\hrulefill}
% \Part{(B0)\enspace}
% For any finite collection of~$f$-orbits~$\G_1,\ldots,\G_r$, the set
% \[
% X(K) \setminus (\G_1\cup\cdots\cup\G_r)
% \quad\text{is non-empty.}
% \]
\Part{(B1)\enspace}
For any finite collection of~$f$-orbits~$\G_1,\ldots,\G_r$, the set
\[
X(K) \setminus (\G_1\cup\cdots\cup\G_r)
\quad\text{is Zariski dense in $X$.}
\]
\Part{(B1$\infty$)\enspace}
For every proper Zariski closed set $Y\subsetneq{X}$, the set
\[
X(K) \setminus \bigcup_{P\in Y(K)} \Orbit_f(P)
\quad\text{is Zariski dense in $X$.}
\]
\par\vspace{0pt}\hbox to.98\hsize{\hrulefill}
% \Part{(C0)\enspace}
% For any finite collection of grand~$f$-orbits~$\G_1,\ldots,\G_r$, the set
% \[
% X(K) \setminus (\G_1\cup\cdots\cup\G_r)
% \quad\text{is non-empty.}
% \]
\Part{(C1)\enspace}
For any finite collection of grand~$f$-orbits~$\G_1,\ldots,\G_r$, the set
\[
X(K) \setminus (\G_1\cup\cdots\cup\G_r)
\quad\text{is Zariski dense in $X$.}
\]
\Part{(C1$\infty$)\enspace}
For every proper Zariski closed set $Y\subsetneq{X}$, the set
\[
X(K) \setminus \bigcup_{P\in Y(K)} \Orbit_f^\grand(P)
\quad\text{is Zariski dense in $X$.}
\]
\Part{(C2$\exists$)\enspace}
There exists a Zariski dense set of representatives in $X(K)$ for 
\[
X(K)/\fgeq.
\]
\Part{(C2$\forall$)\enspace}
Every complete set of representatives for 
\[
\text{$X(K)/\fgeq$\quad is Zariski dense in $X$.}
\]
\Part{(C3$\exists$)\enspace}
There exists a Zariski dense set of representatives in $X_f^\dense(K)$ for
\[
X_f^\dense(K)/\fgeq.
\]
\Part{(C3$\forall$)\enspace}
$X_f^\dense(K)\ne\emptyset$ and every complete set of representatives for 
\[
\text{$X_f^\dense(K)/\fgeq$\quad is Zariski dense in $X$.}
\]
\end{parts}
}}
\caption{Orbit propagation statements}
\label{table:propogationstatements}
\end{table}



The many orbit propagation statements in Table~\ref{table:propogationstatements} are related by a number of straightforward implications, which we have described pictorially in Table~\ref{table:propagationimplications} and proven in Proposition~\ref{proposition:propagationimplications}. In particular, we note that statement (C3$\forall$) has place of honor as the ``one orbit propagation property to rule them all,'' so the strongest form of orbit propagation says that~(A) implies~(C3$\forall$). We do not know any smooth connected projective variety~$X/\Qbar$ with an endomorphism~$f:X\to{X}$ that satisfies~(A), but does not satisfy~(C3$\forall$). We are not able to prove this strong implication in general, but in a number of  situations we are able to prove weaker orbit propagation implications. Theorem~\ref{theorem:orbitpropagationprovencases} summarizes our results.

% \begin{table}
% \begin{tikzpicture}[%
%   node distance=2cm,
%   auto]
%   \node (C3forall) {(C3$\forall$)};
%   \node [right of=C3forall] (C2forall) {(C2$\forall$)};
%   \node [right of=C2forall, node distance=3cm] (C1infty) {(C1$\infty$)};
%   \node [right of=C1infty] (C1) {(C1)};
%   \node [below of=C3forall] (C3exists) {(C3$\exists$)};
%   \node [right of=C3exists] (C2exists) {(C2$\exists$)};
%   \node [right of=C2exists, node distance=3cm] (B1infty) {(B1$\infty$)};
%   \node [right of=B1infty] (B1) {(B1)};
%   \node [below of=C3exists] (A) {(A)};

%   \draw[-{Implies},double] (C3forall) to node {} (C2forall);
%   \draw[-{Implies},double] (C2forall) to node {$\sim$} (C1infty);
%   \draw[-{Implies},double] (C1infty) to node {} (C2forall);
%   \draw[-{Implies},double] (C1infty) to node {} (C1);
%   \draw[-{Implies},double] (C3exists) to node {} (C2exists);
%   \draw[-{Implies},double] (B1infty) to node {} (B1);
%   \draw[-{Implies},double] (C3forall) to node {} (C3exists);
%   \draw[-{Implies},double] (C2forall) to node {} (C2exists);
%   \draw[-{Implies},double] (C1infty) to node {} (B1infty);
%   \draw[-{Implies},double] (C1) to node {} (B1);
%   \draw[-{Implies},double] (C3exists) to node {} (A);
 
%   % \node (A1) [right of=A, node distance=3cm, above of=A, node distance=2cm] {(B2$\forall$)};
%   % \node [right of=A1] (B1) {(B2$\exists$)};
%   % \node [below of=A1] (C1) {(B3$\forall$)};
%   % \node [right of=C1] (D1) {(B3$\exists$)};
%   % \node [right of=B1] (E1) {(B1$\infty$)};
%   % \node [right of=E1] (F1) {(B1)};
%   % \node [below of=F1, node distance=2cm] (G1) {(A)};
  
% \end{tikzpicture}
% \caption{Elementary implications relating the orbit propagation properties in Table~\ref{table:propogationstatements}.  See Proposition~\ref{proposition:propagationimplications}.}
% \label{table:propagationimplications}
% \end{table}

\begin{table}
\framebox{\vbox{\hbox{%
$
\def\P#1#2{\textup{(#1$#2$)}}
\def\L{\Longrightarrow}
\def\D{\Big\Downarrow}
\begin{array}{*7c}
\P{C3}{\forall} & \L & \P{C2}{\forall} 
& \Longleftrightarrow & \P{C1}{\infty} & \L & \P{C1}{} \\[1.5\jot]
\D && \D && \D && \D \\[2.5\jot]
\P{C3}{\exists} & \L & \P{C2}{\exists} & & \P{B1}{\infty} & \L & \P{B1}{} \\[1.5\jot]
\D \\[2.5\jot]
\P{A}{} \\
\end{array}
$
}}}
\caption{\strut Elementary implications relating the orbit propagation properties in Table~\ref{table:propogationstatements}.  See Proposition~\ref{proposition:propagationimplications}.}
\label{table:propagationimplications}
\end{table}

\begin{theorem}
\label{theorem:orbitpropagationprovencases}
Orbit propagation holds as indicated for the following sets of varieties and maps, with the proviso indicated in Remark~\textup{\ref{remark:(A)implies}:}
\begin{parts}
    \Part{(1)}
    Projective space, $\deg(f)\ge2$\textup: \\
    \textup{{(A) $\Rightarrow$ (C1)}} and \textup{{(A) $\Rightarrow$ (B1$\infty$)}}  ---
    \textup{Theorem \ref{theorem:projectivespacedeg2}(a,b)}
    \Part{(2)}
    Projective space, $\deg(f)=1$\textup: \\
    \textup{{(A) $\Rightarrow$ (C2$\forall$)}} ---
    \textup{Theorem \ref{theorem:projectivespacedeg2}(c)}
    \Part{(3)}
    Projective space, $\deg(f)=1$ and either 
    $f$ semisimple or $N=2$\textup: \\
    \textup{{(A) $\Rightarrow$ (C3$\forall$)}} ---
    \textup{Theorems \ref{theorem:projectivespacedeg2}(d) and \ref{theorem:autsofP2}}
   \Part{(4)}
    K3 Surfaces\textup: \\
    \textup{{(A) $\Rightarrow$ (C1)}} ---
    \textup{Theorem \ref{theorem:K3}}
    \Part{(5)}
    Geometrically simple abelian varieties\textup: \\
    \textup{{(A) $\Rightarrow$ (C3$\forall$)}} ---
    \textup{Theorem \ref{theorem:simpleabelianvariety}}
    \Part{(6)}
    Smooth rational varieties and {\'e}tale $f$\textup: \\
    \textup{{(A) $\Rightarrow$ (B1)}} ---
    \textup{Corollary~\ref{corollary:CoroRat}}
    \Part{(7)}
    {\'E}tale quotients of abelian varieties\textup: \\
    \textup{{(A) $\Rightarrow$ (B1)}} ---
    \textup{Corollary~\ref{corollary:B1foretaleabelianquotients}}
    \Part{(8)}
    Smooth projective surfaces\textup: \\
    \textup{{(A) $\Rightarrow$ (B1)}} ---
    \textup{Theorem~\ref{theorem:surfaces}}
\end{parts}
\end{theorem}

We also prove two conditional orbit propagation results for projective space.

\begin{theorem}
The following orbit propagation statements are true for non-linear endomorphisms~$f:\PP^N\to\PP^N$, conditional on the indicated conjectures\textup:
\begin{parts}
    \Part{(1)}
    Assume the cancellation conjecture 
    \textup(Conjecture~\textup{\ref{conjecture:cancellation}}\textup). Then\\
    \textup{{(A) $\Rightarrow$ (C2$\exists$)}} ---
    \textup{Theorem~\ref{theorem:projectivespacedeg2conditional}(a)}
    \Part{(2)}
    Assume the cancellation conjecture 
    \textup(Conjecture~\textup{\ref{conjecture:cancellation}}\textup)
    and the strong preimages conjecture
    \textup(Conjecture~\textup{\ref{conjecture:strongpreimage}}\textup). Then\\
    \textup{{(A) $\Rightarrow$ (C2$\forall$)}} ---
    \textup{Theorem~\ref{theorem:projectivespacedeg2conditional}(b)}
\end{parts} 
\end{theorem}


%%%%%%%%%%%%%%%%%%%%%%%%%%%%%%%%%%%%%%
\section{Motivation for the Propagation Conjecture}
\label{section:motivation}
%%%%%%%%%%%%%%%%%%%%%%%%%%%%%%%%%%%%%%
Our motivation for formulating some sort of propagation conjecture rests on an older, highly speculative, conjecture of the second author. That conjecture says roughly that, up to lower order terms, the height counting function for the integral points on an algebraic variety can have only one of three possible growth rates. The precise formulation requires some care balancing extending the field and discarding Zariski closed sets having fast growth rates. The following is a precise statement for projective varieties and $K$-rational poins; see the cited reference for a more general formulation for quasi-projective varieties and~$S$-integral points.


\newcommand{\arithord}{\mathsf{A}}

\begin{definition}
\label{definition:arithmeticorder}
Let~$X/K$ be a smooth projective variety, and let
\[
H:X(\Kbar)\to[1,\infty)
\]
be a Weil height function associated to an ample divisor. We say that~$X$ has \emph{arithmetic order~$\arithord(X)$} if there is an integer~$m\ge2$ and a non-empty Zariski open subset~$U_0\subseteq{X}$ such that for every non-empty Zariski open subset~$U\subseteq{U_0}\subseteq{X}$ there is a finite extension~$L_0/K$ such that for every finite extension~$L/L_0$,
\begin{equation}
\label{eqn:arithorderlimit}
\lim_{T\to\infty} \frac{ \log^{(m)} \#\bigl\{ P\in U(L) : H(P) \le T\bigr\} }
{ \log^{(m+\arithord(X))} T } = 1.
\end{equation}
The notation~$\log^{(m)}$ denotes the~$m$-fold iterate, and by convention we set~$\arithord(X)=\infty$ if the limit is~$0$ for all~$U$ and all~$L$.
\end{definition}

Since the chain of logic in Definition~\ref{definition:arithmeticorder} is somewhat complicated, we note that it may be written succinctly using logical notation as
\[
\exists m\ge2,\;
\exists U_0\subseteq X,\;
\forall U\subseteq U_0,\;
\exists L_0/K,\;
\forall L/L_0,\;
\text{\eqref{eqn:arithorderlimit} is true.}
\]


\begin{conjecture}
\label{conjecture:Ulm}
\textup{\cite[Silverman 1987]{MR1009803}}
Let~$X/K$ be a smooth projective variety defined over a number field. Then the arithmetic order~$\arithord(X)$ exists and satisfies
\[
\arithord(X)=0 \quad\text{or}\quad \arithord(X)=1 \quad\text{or}\quad \arithord(X)=\infty.
\]
\end{conjecture}

\begin{remark}
We note that it is tempting to simply set~$m=2$ in Definition~\ref{eqn:arithorderlimit}, so we refer the reader to~\cite{MR1009803} for an example that suggests why it may be necessary in some cases to take~$m\ge3$. In any case, the conclusion of Conjecture~\ref{conjecture:Ulm} says roughly that one of the following is true, where we are being very coarse about ignoring error terms:
\begin{equation}
\label{eqn:growthtrichotomy}
\#\bigl\{ P\in X(K) : H(P) \le T \bigr\}
\quad
\begin{cases}
\text{grows like a power of $T$,} \\
\text{grows like a power of $\log T$,} \\
\text{is bounded at $T\to\infty$.} \\
\end{cases}
\end{equation}
\end{remark}

We next observe that in many situations, if $f:X\to{X}$ is an endomorphism of a smooth projective variety defined over a number field~$K$, and if~$P\in{X(P)}$, then the logarithmic height of the points in $f$-orbit of~$P$ tend to grow exponentially. Indeed, if the dynamical degree of~$f$ satisfies~$\d(f)>1$, and if~$\Orbit_f(P)$ is Zariski dense, then it is conjectured~\cite{MR3456169} that
\[
\lim_{n\to\infty} \sqrt[n]{\log H\bigl(f^n(P)\bigr)} = \d(f).
\]
If this is true, then the height counting function for the points in the orbit satisfy
\begin{equation}
\label{eqn:QHQTllloglogT}
\#\bigl\{ Q\in \Orbit_f(P) : H(Q) \le T \bigr\} \ll \log\log T,
\end{equation}
where the implied constant depends on~$f$ and~$P$, but is independent of~$T$. Thus if~$X(K)$ were to be the union of a finite number of $f$-orbits, at least one of which is Zariski dense, then its height counting function would not be bounded, yet would increase too slowly to satisfy the other growth conditions in Conjecture~\ref{conjecture:Ulm}. (This can be seen more clearly, albeit less precisely, by comparing~\eqref{eqn:growthtrichotomy} and~\eqref{eqn:QHQTllloglogT}.) Hence Conjecture~\ref{conjecture:Ulm} suggests that if~$X(K)$ contains at least one Zariski dense $f$-orbit, then (possibly after extending~$K$), it must contain infinitely many non-overlapping $f$-orbits. 
\par
We acknowledge that making a new conjecture on the basis of an older, not widely known, conjecture is somewhat dubious. Further,  using Conjecture~\ref{conjecture:Ulm} as a starting point, a natural conjecture would be a relatively weak statement such as the following:

\begin{protoconjecture}
Let~$K$ be a number field, let~$X/K$ be a smooth projective variety, let $f:X\to{X}$ be an endomorphism defined over~$K$ with dynamical degree~$\d(f)>1$. If~$\#X(K)=\infty$, then~$X(K)$ is not the union of finitely many~$f$-orbits.
\end{protoconjecture}

But as we explored this proto-conjecture, we realized that we were unable to rule out even much stronger statements, including dropping the~$\d(f)>1$ requirement, looking at grand orbits, requiring orbits to be Zariski dense, and changing the ``not a union of finitely many~$f$-orbits'' to a statement that complete sets of representatives for the~$f$-orbits must be Zariski dense. This led us to the plethora of orbit propagation properties listed in Table~\ref{table:propogationstatements}. The remainder of this paper is devoted to proving, for certain classes of varieties and maps, various versions of the statement that one Zariski dense $f$-orbit leads to many such orbits. 

%%%%%%%%%%%%%%%%%%%%%%%%%%%%%%%%%%%%%%
\section{Projective Space}
\label{section:projectivespace}
%%%%%%%%%%%%%%%%%%%%%%%%%%%%%%%%%%%%%%


\begin{theorem}
\label{theorem:projectivespacedeg2}
Let $N\ge1$, let $f:\PP^N\to\PP^N$ be an endomorphism defined over~$\Qbar$, and assume that there is a point~$P_0\in\PP^N(\Qbar)$ whose orbit~$\Orbit_f(P_0)$ is Zariski dense in~$\PP^N$. There exists a number field~$K$ that is a common field of definition for~$f$ and~$P_0$ so that the following hold\textup:
\begin{parts}
\Part{(a)}
\framebox{$\deg(f)\ge2$}
For every finite collection of grand $f$-orbits 
\[
\G_1,\ldots,\G_r\subset\PP^N(\Qbar),
\]
the set
\[
{\PP^N(K) \setminus (\G_1\cup\cdots\cup\G_r)} 
\quad\text{is Zariski dense in $\PP^N$.}
\]
In the terminology of Table~\textup{\ref{table:propogationstatements}}, non-linear endomorphisms of~$\PP^N$ satisfy the orbit propagation statement
\[
\textup{(A)} \quad\Longrightarrow\quad \textup{(C1)}.
\]
\Part{(b)}
\framebox{$\deg(f)\ge2$}
For every proper Zariski closed set $Y\subsetneq{\PP^N}$, the set
\[
\PP^N(K) \setminus \bigcup_{P\in Y(K)} \Orbit_f(P)
\quad\text{is Zariski dense in $\PP^N$.}
\]
In the terminology of Table~\textup{\ref{table:propogationstatements}}, non-linear endomorphisms of~$\PP^N$ of degree at least~$2$ satisfy the orbit propagation statement
\[
\textup{(A)} \quad\Longrightarrow\quad \textup{(B1$\infty$)}.
\]
\Part{(c)}
\framebox{$\deg(f)=1$}
Every complete set of representatives in~$\PP^N(K)$ for $\PP^N(K)/\fgeq$ is Zariski dense in~$\PP^N$. 
In the terminology of Table~\textup{\ref{table:propogationstatements}}, linear endomorphisms of~$\PP^N$ satisfy the orbit propagation statement
\[
\textup{(A)} \quad\Longrightarrow\quad \textup{(C2$\forall$)}.
\]
\Part{(d)}
\framebox{$\deg(f)=1$ and $f$ is semisimple}
Every complete set of representatives in $(\PP^N)_f^\dense(K)$ for $(\PP^N)_f^\dense(K)/\fgeq$ is Zariski dense in~$\PP^N$. 
In the terminology of Table~\textup{\ref{table:propogationstatements}}, semisimple\footnote{A map $f\in\PGL_{N+1}(\Qbar)$ is \emph{semisimple} if some, equivalently every, matrix representing~$f$ is diagonalizable.} linear endomorphisms of~$\PP^N$ satisfy the orbit propagation statement
\[
\textup{(A)} \quad\Longrightarrow\quad \textup{(C3$\forall$)}.
\]
\end{parts}
\end{theorem}

\begin{remark}
We are also able to prove that~(A) implies~(C3$\forall$) for degree~$1$ maps of~$\PP^2$ without the semisimplicity assumption; see Theorem~\ref{theorem:autsofP2}.
\end{remark}

\begin{remark}
For maps~$f:\PP^N\to\PP^N$ of degree at least~$2$, the proof that~(A) implies~(B1$\infty$) uses the fact that forward~$f$-orbits have very few points when ordered by height. It would be more difficult to deal with grand orbits, since for example, for every~$n$, all of the points in~$f^{-n}\bigl(f^n(P)\bigr)$ have the same $f$-canonical height, and thus their Weil heights differ by a bounded amount. We overcome this difficulty conditionally in Theorem~\ref{theorem:projectivespacedeg2conditional} by assuming various arithmetic dynamical conjectures.
\end{remark}

\begin{proof}[Proof of Theorem \textup{\ref{theorem:projectivespacedeg2}}]
We remark that the proof of Theorem~\ref{theorem:projectivespacedeg2} uses two lemmas whose statements and proofs we defer until the end of this section.
\par\noindent(a,b)\enspace\framebox{$\deg(f)\ge2$}\enspace
The assumption that $\deg(f)\ge2$ means that there is a canonical height function~\cite{MR1255693}
\[
\hhat_f:\PP^N(\Kbar)\longrightarrow\RR_{\ge0}
\]
satisfying
\begin{align}
\label{eqn:canht1}
\hhat_f\circ f&=d\cdot\hhat_f,\\
\label{eqn:canht2}
|\hhat_f - h| &\le \Cl{hfh}(f),\\
\label{eqn:canht3}
\hhat_f(P)=0&\quad\Longleftrightarrow\quad \#\Orbit_f(P)<\infty.
\end{align}
\par
Combining~\eqref{eqn:canht2} and~\eqref{eqn:canht3} with the fact that~$\PP^N(K)$ has only finitely many points of bounded height, we see that~$\hhat_f$ takes on a minimal positive value, which we denote by
\begin{equation}
\label{eqn:hhatfminPNK}
\hhat_f^{\min}(\PP^N,K) 
:= \inf_{\substack{P\in\PP^N(K)\\\#\Orbit_f(P)=\infty\\}} \hhat_f(P)
> 0. 
\end{equation}
\par
Let~$\G$ be an $f$-orbit or a grand $f$-orbit such that~$\G\cap\PP^N(K)$ is non-empty. We define
\[
\hhat_f^{\min}(\G,K) = \inf\bigl\{ \hhat_f(Q) : Q\in\G\cap\PP^N(K) \bigr\}.
\]
Using~\eqref{eqn:canht2} and the fact that there are only finitely many points in~$\PP^N(K)$ of bounded Weil height, we see that there exists a point
\[
Q_{\G,K}\in\G\cap\PP^N(K)
\]
(not necessarily unique if~$\G$ is a grand orbit) such that
\[
\hhat_f(Q_{\G,K}) = \hhat_f^{\min}(\G,K).
\]
Further, we see that 
\begin{equation}
\label{eqn:hhatfminGKgt0}
\hhat_f^{\min}(\G,K)>0
\quad\Longleftrightarrow\quad
\text{$\G$ contains an $f$-wandering point}.
\end{equation}
(This is equivalent to every point in~$\G$ being $f$-wandering.)
%%%%%%%%%%%%%%%%%%%%%%%%%%%
\par\noindent(a)\enspace
%%%%%%%%%%%%%%%%%%%%%%%%%%%
Let~$\G$ be a grand $f$-orbit. Then
\begin{align*}
Q & \in\G\cap\PP^N(K) \\
&\quad\Longrightarrow\quad
f^i(Q)=f^j(Q_{\G,K})\quad\text{for some $i,j\in\NN$,} \\
&\quad\Longrightarrow\quad
d^i\cdot\hhat_f(Q)=d^j\cdot\hhat_f(Q_{\G,K})
\quad\text{for some $i,j\in\NN$,} \\
&\quad\Longrightarrow\quad
d^i\cdot\hhat_f(Q)=d^j\cdot\hhat_f(Q_{\G,K})
\quad\begin{tabular}[t]{@{}l@{}}
with $j\ge i$, 
since $\hhat_f(Q_{\G,K})$\\
is the smallest
value of~$\hhat_f$\\ for the points in~$\G$,\\
\end{tabular} \\
&\quad\Longrightarrow\quad
\hhat_f(Q) \in d^\NN \cdot\hhat_f^{\min}(\G,K)
\quad \text{since $\hhat_f(Q_{\G,K})=\hhat_f^{\min}(\G,K)$,} \\
&\quad\Longrightarrow\quad
\smash[b]{ h(Q) \in \bigcup_{n\in\NN} }
\Bigl[d^n\hhat_f^{\min}(\G,K)-\Cr{hfh}(f),d^n\hhat_f^{\min}(\G,K)+\Cr{hfh}(f)\Bigr] \\*
&\omit\hfill\quad\text{since $|\hhat_f-h|\le\Cr{hfh}(f)$.}
\end{align*}
\par
We now suppose that~$\G_1,\ldots,\G_r$ are grand orbits, and to ease notation, we let
\[
\Cl{G1Gr}(i) := \Cr{G1Gr}(f,\G_i,K) = \hhat_f^{\min}(\G_i,K).
\]
The above calculation shows that
\[
\bigcup_{i=1}^r \G_i\cap\PP^N(K)
\subseteq
\bigcup_{i=1}^r \bigcup_{n\in\NN}
\Bigl\{ Q\in\PP^N(K) : \bigl| h(Q) - d^n\Cr{G1Gr}(i) \bigr| \le \Cr{hfh}(f) \Bigr\}.
\]
Hence taking heights, we find that the set of heights of the points in the union of the \text{$\G_i\cap\PP^N(K)$} is contained in a union of intervals,
\begin{multline*}
\bigcup_{i=1}^r 
\Bigl\{ h(Q) : Q\in\G_i\cap\PP^N(K) \Bigr\}  \\
\subseteq \bigcup_{i=1}^r \bigcup_{n\in\NN}
\Bigl[ d^n\Cr{G1Gr}(i) - \Cr{hfh} ,\; d^n\Cr{G1Gr}(i) + \Cr{hfh} \Bigr].
\end{multline*}
An elementary estimate shows that the double union on the right-hand side omits intervals in~$\RR_{\ge0}$ of arbitrarily large length; see Lemma~\ref{lemma:missedinterval} for a more precise result. In particular, we can find infinitely many intervals of length~$1$ that are omitted, say
\begin{multline*}
0\le t_1 < t_2 < t_3 < \cdots\quad\text{satisfying}\quad
t_{i+1}>t_i+1\quad\text{and}\\
\biggl( \bigcup_{i\ge1} [t_i,t_i+1] \biggr) \;\cap\;
\biggl( \bigcup_{i=1}^r 
\Bigl\{ h(Q) : Q\in\G_i\cap\PP^N(K) \Bigr\} \biggr) = \emptyset.
\end{multline*}
Every interval of length~$1$ contains a number of the form~$\log(a)$ with~$a\in\NN$, so we can find a sequence of distinct positive integers~$a_1,a_2,\ldots$ satisfying
\[
\log(a_j) \notin \bigcup_{i=1}^r \Bigl\{ h(Q) : Q\in\G_i\cap\PP^N(K) \Bigr\}
\quad\text{for all $j\ge1$.}
\]
We consider the set of points
\[
\Acal := \bigcup_{j\ge1} \left\{ [a_j,b_1,\cdots,b_N]\in\PP^N(\QQ) : 
\begin{array}{@{}l@{}}
b_1,\ldots,b_N\in\ZZ,\\
0\le |b_1|,\ldots,|b_N|\le a_j\\
\end{array}\right\} .
\]
The heights of the points in~$\Acal$ are all of the form~$\log(a_j)$, so they are not heights of~$K$-rational points in any of the grand orbits~$\G_1,\ldots,\G_r$. This proves that
\[
\Acal\; \cap \; \bigcup_{i=1}^r \Bigl(\G_i\cap\PP^N(K) \Bigr)
= \emptyset.
\]
On the other hand, it is clear that~$\Acal$ is Zariski dense in~$\PP^N$. 
%%%%%%%%%%%%%%%%%%%%%%%%%%%%%%%%%%%%%%%%%%%%
% Proof of this ``clear'' assertion: Suppose that $F(X_0,\ldots,X_N)\in\CC[X_0,\ldots,X_N]$ is a homogeneous polynomial that vanishes at every point of~$\Acal$. For any $\bfb=(b_1,\ldots,b_N)\in\ZZ^N$, let $F_\bfb(X)=F(X,b_1,\ldots,b_N)$. For any fixed~$\bfb$, there are infinitely many~$a\in\NN$ such that~$[a,\bfb]\in\Acal$. More precisely, we have~$[a_j,\bfb]\in\Acal$ if~$a_j\ge\max|\bfb|$. It follows that the polynomial~$F_\bfb(X)$ has infinitely many roots, and hence that it is identically~$0$. In particular, $F_\bfb(a)=0$ for every~$a\in\ZZ$, and since this is true for all~$\bfb\in\ZZ^N$, we conclude that~$F$ vanishes at every point in~$\PP^N(\QQ)$. Hence~$F$ is identically~$0$.
% Addendum: If we're careful and use the quantitative result in Lemma~\ref{lemma:missedinterval}, we can probably prove something like $\Count(\Acal,T)\gg{T^{N+1}}/\log(T)$, which is stronger than simply proving that~$\Acal$ is Zariski dense. 
%%%%%%%%%%%%%%%%%%%%%%%%%%%%%%%%%%%%%%%%%%%%
Hence $\PP^N(K)$ contains a Zariski dense set of points not lying in any of the grand orbits~$\G_1,\ldots,\G_r$, which completes the proof of orbit propagation property~(C1).
%%%%%%%%%%%%%%%%%%%%%%%%%%%
\par\noindent(b)\enspace
%%%%%%%%%%%%%%%%%%%%%%%%%%%
To simplify formulas, we are going to use Weil and canonical heights relative to the field~$K$. For any subset~$\Pcal\subseteq\PP^N(K)$, we define a counting function
\begin{equation}
\label{eqn:countingfunction}
\Count(\Pcal,T) := \#\bigl\{P\in\Pcal : H(P) \le T\bigr\}.
\end{equation}
With our height normalization, Lemma~\ref{lemma:ctYKTleCTN}(a) tells us that\footnote{Schanual's formula~\cite{MR557080} gives a formula, with error term, for~$\Count\bigl(\PP^N(K),T\bigr)$, but we will not need anything that precise.}
\begin{equation}
\label{eqn:countPNKT}
\Cl{PNKT}(K,N) T^{N+1} \le \Count\bigl(\PP^N(K),T\bigr) \le \Cl{PNKTup}(K,N) T^{N+1}.
\end{equation}
\par
Let~$P\in{X(K)}$ be an $f$-wandering point. Then
\begin{align}
\Count\bigl(\Orbit_f(P),T\bigr)
&= \#\bigl\{ n\ge0 : H\bigl(f^n(P)\bigr) \le T \bigr\} \notag \\
&\le \#\Bigl\{ n\ge0 : \hhat_f\bigl(f^n(P)\bigr) \le \log(T)+\Cr{hfh}(f) \Bigr\}
\quad\text{from \eqref{eqn:canht2},} \notag \\
%% \label{eqn:ctfororb1}
&\le \#\Bigl\{ n\ge0 : d^n\hhat_f(P) \le \log(T)+\Cr{hfh}(f) \Bigr\} \notag \\
&\le 1 + \log_d \left( \frac{\log T + \Cr{hfh}(f)}{\hhat_f(P)} \right) \notag \\
&\le 1 + \log_d \left( \frac{\log T + \Cr{hfh}(f)}{\hhat_f^{\min}(\PP^N,K)} \right) 
\quad\text{from \eqref{eqn:hhatfminPNK},} \notag \\
\label{eqn:ctfororb2}
&\le \Cl{Oplus}(K,N,f)\cdot\log\log(T).
\end{align}
\par
We also note that
\begin{align*}
h(P) & > \log(T) + 2\Cr{hfh}(f)\\
&\quad\Longrightarrow\quad
\hhat_f(P) > \log(T) + \Cr{hfh}(f)
\quad\text{from \eqref{eqn:canht2},} \\
&\quad\Longrightarrow\quad
d^n\hhat_f(P) > d^n\log(T) + d^n\Cr{hfh}(f)
\quad\text{for all $n\in\NN$,} \\
&\quad\Longrightarrow\quad
\hhat_f\bigl(f^n(P)\bigr) > d^n\log(T) + d^n\Cr{hfh}(f)
\quad\text{for all $n\in\NN$, from \eqref{eqn:canht1},} \\
&\quad\Longrightarrow\quad
h\bigl(f^n(P)\bigr) > d^n\log(T) + (d^n-1)\Cr{hfh}(f)
\quad\text{for all $n\in\NN$, from \eqref{eqn:canht2},} \\
&\quad\Longrightarrow\quad
h\bigl(f^n(P)\bigr) > \log(T) 
\quad\text{for all $n\in\NN$.} 
\end{align*}
Hence
\begin{equation}
\label{eqn:ctfororb3}
h(P) > \log(T)+2\Cr{hfh}(f)
\quad\Longrightarrow\quad
\Count\bigl(\Orbit_f(P),T\bigr)=0.
\end{equation}
\par
The Zariski closed set~$Y$ consists of a finite number of irreducible subvarieties of~$\PP^N$ of dimension at most~$N-1$. It follows Lemma~\ref{lemma:ctYKTleCTN}(b) that
\begin{equation}
\label{eqn:countYKT}   
\Count\bigl( Y(K), T \bigr) 
\le \Cl{countPKT}(K,Y) \cdot T^N.
\end{equation}
\par
We estimate
\begin{align}
\label{eqn:contYKTorb}
\Count \biggl( & \bigcup_{P\in Y(K)} \Orbit_f(P),\, T \biggr) \notag \\
&\le \sum_{P\in Y(K)} \Count\bigl(\Orbit_f(P),T\bigr) \notag \\
&= \sum_{\substack{P\in Y(K)\\ h(P) \le \log(T)+2\Cr{hfh}(f)\\ }}
\Count\bigl(\Orbit_f(P),T\bigr) 
\quad\text{from \eqref{eqn:ctfororb3},} \notag \\
&\le \sum_{\substack{P\in Y(K)\\ h(P) \le \log(T)+2\Cr{hfh}(f)\\ }}
\Cr{Oplus}(K,N,f)\cdot\log\log(T)
\quad\text{from \eqref{eqn:ctfororb2},} \notag \\
&= \Count\bigl( Y(K), \Cl{hfh2}(f)\cdot T \bigr) \cdot \Cr{Oplus}(K,N,f)\cdot\log\log(T) \notag \\
&\omit\hfill\quad\text{where $\Cr{hfh2}=\exp\bigl(2\Cr{hfh}(f)\bigr)$,} \notag \\
&\le 
\Cr{countPKT}(K,Y) \cdot \bigl(\Cr{hfh2}(f)\cdot T\bigr)^N
\cdot \Cr{Oplus}(K,N,f)\cdot\log\log(T)
\quad\text{from \eqref{eqn:countYKT},} \notag \\
&\le \Cl{countPKTY2}(K,Y,N,f) \cdot T^N \cdot \log\log(T).
\end{align}
\par
Combining~\eqref{eqn:countPNKT} and~\eqref{eqn:contYKTorb} yields
\begin{multline*}
\Count\biggl( \PP^N(K) \setminus \bigcup_{P\in Y(K)} \Orbit_f(P),\, T \biggr) \\
\ge \Cr{PNKT}(K,N) T^{N+1} - \Cr{countPKTY2}(K,Y,N,f) \cdot T^N \cdot \log\log(T).
\end{multline*}
This function grows faster than any multiple of~$T^N$, so~\eqref{eqn:countYKT} tells us that
\[
\PP^N(K) \setminus \bigcup_{P\in Y(K)} \Orbit_f(P)
\quad\text{\parbox[t]{.4\hsize}{is not contained in a Zariski closed subset of~$\PP^N$.}}
\]
This completes the proof of~(b).
%%%%%%%%%%%%%%%%%%%%%%%%%%%
\par\noindent(c)\enspace\framebox{$\deg(f)=1$}\enspace
%%%%%%%%%%%%%%%%%%%%%%%%%%%
The assumption that~$\deg(f)=1$ tells us that~$f\in\PGL_{N+1}(\Qbar)$. Making a change of coordinates, we may assume that~$f$ is represented by a matrix~$A_f\in\GL_{N+1}(\Qbar)$ in Jordan normal form, say
\[
A_f = \begin{pmatrix}
\l_{0} & *      & 0      &\cdots & 0 \\
0      & \l_{1} & *      &\cdots & 0 \\
0      & 0      & \l_{2} &\cdots & 0 \\
\vdots & \vdots & \vdots &\ddots & \vdots \\
0       & 0     & 0      & \cdots & \l_{N} \\
\end{pmatrix} = \L + \Theta,
\]
where the stars are~$0$ or~$1$. 
We let~$\L$ denote the diagonal matrix with entries~$\l_0,\ldots,\l_N$, and we let~$\Theta=A_f-\L$ be the nilpotent matrix containing the off-diagonal entries of~$A_f$. In particular, we have
\[
\L\Theta=\Theta\L\quad\text{and}\quad\Theta^N=0.
\]
It follows that for all~$n\in\ZZ$, 
\begin{equation}
\label{eqn:Afnj0N1}
A_f^n = \sum_{j=0}^{N-1} \binom{n}{j} \L^{n-j} \Theta^j.    
\end{equation}
We note that~\eqref{eqn:Afnj0N1} holds for all integers~$n$, using the usual definition of~$\binom{n}{j}$ for~$n<0$, and that it holds for~$0\le{n}<N$, since~$\binom{n}{j}=0$ for~$j>n$.
\par
We claim that our assumption that~$f$ has propagation property~(A) implies in particular that none of the eigenvalues, i.e., none of the diagonal entries, are~$0$. To see this, suppose that the contrary. Then due to the configuration of Jordan normal form, one of the rows of~$A_f$ is~$0$, say the~$k$th row, where~$k$ is some value between~$0$ and~$N$. It follows that for any point~$P\in\PP^N$, the orbit of~$P$ is not Zariski dense; more precisely,
\[
\Orbit_f(P) \subset \{x_k=0\} \subsetneq\PP^N.
\]
This contradicts our assumption that there is at least one Zariski dense orbit, which completes the proof of the claim that the~$\l_i$ are all non-zero.
\par
We set the following notation:
\[
\begin{array}{c@{\quad}l}
K/\QQ & \text{a number field containing all of the~$\l_{ij}$} \\[1\jot]
S & \text{\parbox[t]{.85\hsize}{a set of places of $K$, including all archimedean places,
such that $\l_{ii}\in R_S^*$ for all $i$ and $\l_{ij}\in R_S$ for all $i,j$}} \\[5\jot]
R_S & \text{the ring of $S$-integers of $K$} \\
\end{array}
\]
For~$n\in\ZZ$ we write the corresponding power of~$A_f$ as
\[
A_f^n = \begin{pmatrix}
\mu^{(n)}_{00} & \mu^{(n)}_{01} & \cdots & \mu^{(n)}_{0N} \\
0       & \mu^{(n)}_{11} & \cdots & \mu^{(n)}_{1N} \\
\vdots  & \vdots  & \ddots & \vdots \\
0       & 0       & \cdots & \mu^{(n)}_{NN} \\
\end{pmatrix}.
\]
We note that
\[
\mu^{(n)}_{ii} = \l_{i}^n \in R_S^*~\text{for all $i$,}
\quad\text{and}\quad
\mu^{(n)}_{ij} \in R_S~\text{for all $i,j$.}
\]
\par
Let~$\Qcal\subset\PP^N(K)$ be a complete set of representatives for $\PP^N(K)/\fgeq$. In order to prove that~$f$ has propagation property~(C3$\forall$), we need to show that~$\Qcal$ is Zariski dense in~$\PP^N$. So we suppose that~$\overline\Qcal$ is not Zariski dense and derive a contradiction. Under this assumption, we can find a hypersurface~$Y\subsetneq\PP^N$ containing~$\overline\Qcal$. We let~$\f(\bfx)$ be a homogeneous polynomials in~$K[\bfx]$ that is an equation for~$Y$, so
\begin{equation}
\label{eqn:QYinphix0}
    \Qcal \subseteq \overline\Qcal \subseteq Y = \bigl\{\f(\bfx)=0\bigr\} \subseteq \PP^N.
\end{equation}
We let~$d=\deg(\f)$, and we write~$\f$ explicitly as
\[
\f(\bfx) = \sum_{k_0+k_1+\cdots+k_N=d} a(k_0,k_1,\ldots,k_N)x_0^{k_0}x_1^{k_1}\cdots x_N^{k_N}
= \sum_{|\bfk|=d} a(\bfk)\bfx^\bfk.
\]
We adjoin finitely many additional primes to the set~$S$ so that the non-zero coefficients of~$\f$ are~$S$-units, i.e., 
\[
a(\bfk)\ne0 \quad\Longrightarrow\quad a(\bfk)\in R_S^*.
\]
\par
We note that our assumptions imply that 
\[
\PP^N(K)
= \bigcup_{Q\in\Qcal} \Orbit_f^\grand(Q)
\subseteq \bigcup_{Q\in Y(K)} \Orbit_f^\grand(Q).
\]
Hence for all $P\in\PP^N(K)$ there exists a point $Q_P\in\PP^N(K)$ and an integer $n_P\in\ZZ$ such that
\[
\f(Q_P)=0 \quad\text{and}\quad P = f^{-n_P}(Q_P).
\]
(Since~$n_P\in\ZZ$, it's notationally convenient to insert a negative sign here.) Note that we have complete freedom in our choice of~$P\in(\PP^N(K)$. We fix a prime~$\gp$ of~$R_K$ with~$\gp\notin{S}$, and we let~$\pi\in{R_K}$ be a uniformizer for~$\gp$.  Then for each choice of
\[
\bfr=(r_0,r_1,\ldots,r_N)\in\NN^{N+1},
\quad\text{we set}\quad
P(\bfr) = [\pi^{-r_0},\pi^{-r_1},\ldots,\pi^{-r_N}].
\]
For the moment we  assume that the~$r_i$ are positive and decreasing,
\begin{equation}
\label{eqn:r0gtr1gtrNge1}
r_0 > r_1 > \cdots > r_N \ge 1.
\end{equation}
Later we will specify them more precisely.
\par
We let~$0\le{i}\le{N}$, and for~$n\in\ZZ$ we consider the~$\gp$-adic valuation of $i$th coordinate of~$f^n\bigl(P(\bfr)\bigr)$. (We adopt the convenient standard notation that $\bfv[i]$ denotes the $i$th coordinate of point~$\bfv=[v_0,\ldots,v_N]\in\PP^N$ having specified homogeneous coordinates.) Thus
\begin{align*}
\ord_\gp\Bigl(f^n\bigl(P(\bfr)\bigr)[i]\Bigr)
&=
\ord_\gp\biggl( \sum_{j=i}^N \mu_{ij}^{(n)} \cdot \pi^{-r_j} \biggr) \\
&=
\ord_\gp\biggl( \l_i^n \pi^{-r_i} + \sum_{j=i+1}^N \mu_{ij}^{(n)} \cdot \pi^{-r_j} \biggr).
\end{align*}
The facts that~$\gp\notin{S}$ and~$\l_i\in{R_S^*}$ and $\mu_{ij}^{(n)}\in{R_S}$ yield
\[
\ord_\gp( \l_i^n \pi^{-r_i} ) = -r_i
\quad\text{and}\quad
\ord_\gp( \mu_{ij}^{(n)} \cdot \pi^{-r_j} ) \ge -r_j.
\]
It follows from~\eqref{eqn:r0gtr1gtrNge1} that we have a strict inequality
\[
\ord_\gp( \l_i^n \pi^{-r_i} ) < \ord_\gp( \mu_{ij}^{(n)} \cdot \pi^{-r_j} )
\quad\text{for all $i<j$,} 
\]
and hence the non-archimedean triangle inequality yields
\begin{equation}
\label{eqn:ordpfnPri}
\ord_\gp\Bigl(f^n\bigl(P(\bfr)\bigr)[i]\Bigr) 
= \ord_\gp( \l_i^n \pi^{-r_i} ) = -r_i.
\end{equation}
\par
We compute
\begin{align}
\label{eqn:0fQPr}
0 &= \f(Q_{P(\bfr)}) \notag\\
&= \f\Bigl(f^{n_{P(r)}} \bigl(P(\bfr)\bigr) \Bigr)  \notag\\
&= \sum_{|\bfk|=d} a(\bfk) \Bigl( f^{n_{P(\bfr)}}\bigl(P(\bfr)\bigr) \Bigr)^\bfk \notag\\
&= \sum_{|\bfk|=d} a(\bfk) \prod_{i=0}^N f^{n_{P(\bfr)}}\bigl(P(\bfr)\bigr)[i]^{k_i} .
\end{align}
\par
We use~\eqref{eqn:ordpfnPri} to compute the valuation of the non-zero monomials appearing in~\eqref{eqn:0fQPr}. Thus if~$a(\bfk)\ne0$, then
\begin{align}
\label{eqn:ordpaiLnPpi2}
\ord_\gp\Bigl( a(\bfk) &  \Bigl( f^{n_{P(\bfr)}}\bigl(P(\bfr)\bigr) \Bigr)^\bfk \notag\\
&= 
\ord_\gp\bigl(a(\bfk)\bigr) + 
\sum_{i=0}^N  \ord_\gp\Bigl( f^{n_{P(\bfr)}}\bigl(P(\bfr)\bigr)[i]^{k_i} \Bigr) \notag\\
&=
\sum_{i=0}^N -r_i k_i.
\end{align}
The last equality follows from~\eqref{eqn:ordpfnPri} and the fact that the non-zero~$a(\bfk)$ are~$S$-units.
\par
We now set the~$r_i$ to equal
\[
r_i = (d+1)^{N-i} \quad\text{for $0\le i\le N$.}
\]
It then follows from~\eqref{eqn:ordpaiLnPpi2} and Lemma~\ref{lemma:kmapstordotkinjective} that the non-zero monomials appearing in the expansion~\eqref{eqn:0fQPr} have distinct negative $\gp$-adic valuations. 
% Hence the non-archimedean triangle inequality implies that
% \[
% \ord_\gp\bigl( \f(Q_{P(\bfr)}) \bigr)
% = \min \bigl\{ -\bfr\cdot\bfk : 
% \text{$\bfk\in\NN^{N+1}$ satisfies $|\bfk|=d$ and $a(\bfk)\ne0$ } \bigr\}. 
% \]
Thus if~$\f(\bfx)$ has any non-zero monomials, then the non-archimedean triangle inequality implies that cannot sum to~$0$. Since $\f(Q_{P(\bfr)})=0$ by construction, it follows that the polynomial~$\f$ has no non-zero monomials, i.e., we have proven that~$\f=0$. This contradicts the assumption that~$Y=\{\f=0\}$ is a proper Zariski closed subset of~$\PP^N$, which completes the proof that~$\Qcal$ is Zariski dense in~$\PP^N$, and thus the proof that the map~$f$ has property~(C3$\forall$).
\par\noindent(d)\enspace
The semisimplicity assumption on~$f$ says that~$A_f$ is a diagonal matrix, so
\[
f\bigl([x_0,x_1,\ldots,x_n]\bigr) = [\l_0 x_0,\l_1x_1,\ldots,\l_Nx_N].
\]
We are given that there exists a point~$P_0=[b_0,b_1,\ldots,b_N]\in\PP^N$ whose orbit
\[
\Orbit_f(P_0) = \bigl\{ [\l_0^n b_0,\l_1^nb_1,\ldots,\l_N^nb_N] : n\in\NN \bigl\}
\]
is Zariski dense in~$\PP^N$. If some~$b_i=0$, then~$\Orbit_f(P_0)$ would be contained in the hyperplane~$x_i=0$, contradicting the Zariski density of~$\Orbit_f(P_0)$. Hence~$b_0,\ldots,b_N$ are non-zero. We write
\[
\TT = 
\GG_m^{N+1}/\GG_m^{\vphantom{N}} \cong \bigl\{[x_0,x_1,\ldots,x_N]\in\PP^N:x_0\cdots x_N\ne0\bigr\}
\]
for the standard torus contained in~$\PP^N$. We observe that coordinate-wise multiplication gives~$\TT$ the structure of a group, that~$\TT$ is an~$f$-invariant subset of~$\PP^N$, and that the map~$f:\TT\to\TT$ is multiplication by the point\footnote{We remark that the existence of a Zariski dense orbit easily implies that~$\l_0,\ldots,\l_N$ are homogeneously multiplicatively independent. The converse is also true, but not as easy to prove. We will not need either of these facts.}
\[
L = [\l_0,\l_1,\ldots,\l_N] \in\TT.
\]
\par
Let~$P\in\TT$ be an arbitrary point. Then
\begin{align*}
\Orbit_f(P_0) 
&= \{L^n\cdot P_0:n\in\NN\}\\
&= \{ L^n\cdot P_0\cdot P^{-1}\cdot P : n\in\NN\}\\
&= \{ P_0\cdot P^{-1}\cdot L^n\cdot P : n\in\NN\} 
\quad\text{since~$\TT$ is commutative,} \\
&= P_0\cdot P^{-1} \cdot \Orbit_f(P).
\end{align*}
The orbit~$\Orbit_f(P_0)$ is Zariski dense by assumption, so this shows that the orbit~$\Orbit_f(P)$ is Zariski dense for every~$P$ in~$\TT(K)$. And as noted earlier, if~$P\notin\TT(K)$, then~$x_i(P)=0$ for some~$i$, which implies that~$\Orbit_f(P)\subset\{x_i=0\}$ is not Zariski dense. We have thus proven that
\begin{equation}
\label{eqn:OfPZDiffPinTTx}
\overline{\Orbit_f(P)} = \PP^N \quad\Longleftrightarrow\quad P\in\TT;
\end{equation}
or equivalently\footnote{We remark that it is not hard to show that the equality $(\PP^N)_f^\dense=\TT$ induces an isomorphism $(\PP^N)_f^\dense/\fgeq=\TT/L^\ZZ$, but we will not need this stronger statement.}
%%%%%%%%%%%%%%
% *** Justification for statement in the footnote ***
% Take two points~$P,Q\in\TT$ and compute
% \begin{align*}
% \Orbit_f^\grand(P) &= \Orbit_f^\grand(Q) \ne \emptyset \\
% &\quad\Longleftrightarrow\quad
% \Orbit_f(P)  \cap \Orbit_f(Q) \ne \emptyset \\
% &\quad\Longleftrightarrow\quad
% f^m(P) = f^n(Q)  \quad\text{for some $m,n\in\NN$,} \\
% &\quad\Longleftrightarrow\quad
% \L^m\cdot P = \L^n\cdot Q  \quad\text{for some $m,n\in\NN$,} \\
% &\quad\Longrightarrow\quad
% P\cdot Q^{-1} \in \L^\ZZ.
% \end{align*}
% This proves that
% \begin{equation}
% \label{eqn:PgeqQiffQLZZP}
% P \fgeq Q \quad\Longleftrightarrow\quad Q \in \L^\ZZ\cdot P.
% \end{equation}
% It follows from~\eqref{eqn:OfPZDiffPinTTx} and~\eqref{eqn:PgeqQiffQLZZP} that the natural inclusion $\TT(K)\hookrightarrow\PP^N(K)$ induces a bijection
% \begin{equation}
% \label{eqn:PNdensefKTKLZ}
% (\PP^N)^\dense_f(K) / \fgeq \; \longleftrightarrow \; \TT(K)/\L^\ZZ.
% \end{equation}
%%%%%%%%%%%%%%%%%%%%
\[
(\PP^N)_f^\dense=\TT
\]
In particular, we see that
\[
\text{$(\PP^N)_f^\dense(K)=\TT(K)$ is Zariski dense in $\PP^N$.}
\]
Since we have assumed~(A) and have proven~(C2$\forall$), it follows from Lemma~\ref{lemma:C2forallZDimpliesC3forall} that~(C3$\forall$) is true.
\end{proof}




We conclude this section with two elementary results. The first describes intervals contained within other intervals that was used in the proof of Theorem~\ref{theorem:projectivespacedeg2}(a). The second is an injectivity result for dot products of sequences.

\begin{lemma}
\label{lemma:missedinterval}
Let $\a_1,\ldots,\a_r>0$ and $\b\ge0$ and $d>1$. For~$1\le{i}\le{r}$ and~$n\in\NN$, define real intervals
\[
I_{i,n} := [\a_i d^n - \b\,, \a_i d^n + \b]
\quad\text{and}\quad I := \bigcup_{1\le i\le r} \bigcup_{n\in\NN} I_{i,n}.
\]
There are constants $\Cl{aibde}$ and $\Cl{aibde7}>0$, depending only on~$\a_1,\ldots,\a_r,\b,d$, so that
\[
T \ge \Cr{aibde}
\quad\Longrightarrow\quad
\left(\begin{array}{@{}l@{}}
\text{$[0,T]\setminus I$ contains an interval} \\
\text{of length at least ${\Cr{aibde7}T}/{\log(T)}$.}
\end{array}\right).
\]
\end{lemma}
\begin{proof}
Fix~$T$. In the following calculation, all constants are independent of~$T$. The number of intervals~$I_{i,n}$ that have a point in common with the interval~$[0,T]$ is bounded by
\begin{align*}
\#\bigl\{(i,n)\in[r]\times\NN & : I_{i,n}\cap[0,T]\ne\emptyset \bigr\}\\
&\le \sum_{i=1}^r \#\bigl\{n \in \NN : \a_id^n-\b \le T \bigr\} \\
&\le \sum_{i=1}^r \left( \log_d\left(\frac{T+\b}{\a_i}\right) + 2 \right) \\
&\le \Cl{aibde2}\log_d(T) \quad\text{for $T\ge\Cl{aibde3}$.}
\end{align*}
Suppose that $[0,T]\setminus{I}$ contains no intervals of length at least~$B$. This implies that if we lengthen each~$I_{i,n}$ by~$B/2$ on each side, then~$\{I_{i,n}\}$ will cover all of~$[0,T]$. In other words, if we define
\[
I_{i,n}(B) := [\a_i d^n - \b - B/2\,, \a_i d^n + \b + B/2],
\]
then the assumption that~$[0,T]\setminus{I}$ contains no intervals of length at least~$B$ implies that
\[
[0,T] \subseteq \bigcup_{\substack{i\in[r],\,n\in\NN\\ I_{i,n}\cap[0,T]\ne\emptyset\\}} I_{i,n}(B). 
\]
Hence
\begin{align*}
T 
&\le \sum_{\substack{i\in[r],\,n\in\NN\\ I_{i,n}\cap[0,T]\ne\emptyset\\}} 
\operatorname{Length}\bigl(I_{i,n}(B)\bigr) \\
&\le \sum_{\substack{i\in[r],\,n\in\NN\\ I_{i,n}\cap[0,T]\ne\emptyset\\}} 
(2\b+B) \\
&\le (2\b+B)\cdot \bigl( \Cr{aibde2} \log_d(T) \bigr)
\quad\text{for $T\ge\Cr{aibde3}$.} 
\end{align*}
Rearranging this inequality, we have proven that if~$T\ge\Cr{aibde3}$ and if $[0,T]\setminus{I}$ contains no intervals of length at least~$B$, then
\[
B \ge \frac{T}{\Cr{aibde2}\log_d(T)} - 2\b.
\]
Adjusting constants, it follows that if~$T\ge\Cl{aibde4}$, then~$[0,T]\setminus{I}$ will contain an interval of length $\Cl{aibde5}{T}/\log(T)$.
\end{proof}

\begin{lemma}
\label{lemma:kmapstordotkinjective}
Define a list of integers
\begin{equation}
\label{eqn:rjdj1d1x}
r_i = (d+1)^{N-i} \quad\text{for $0\le i\le N$.} 
\end{equation}
Then the map
\begin{equation}
\label{eqn:idtorix}
\bigl\{ \bfk\in\NN^{N+1} : |\bfk|=d \bigr\} \longrightarrow \NN,\quad
\bfk \longmapsto \bfr\cdot\bfk =\sum_{i=0}^N r_i k_i
\end{equation}
is injective.
\end{lemma}
\begin{proof}
We suppose that 
\[
\bfj\ne\bfk 
\quad\text{and}\quad
|\bfj|=|\bfk|=d
\quad\text{and}\quad
\bfr\cdot\bfj=\bfr\cdot\bfk
\]
and derive a contradiction. The assumption that~$\bfj\ne\bfk$ implies that there is an index~$0\le\ell\le{N}$ such that
\begin{equation}
\label{eqn:iellnekell2}
j_\ell\ne k_\ell \quad\text{and}\quad j_m=k_m~\text{for $0\le m<\ell$.}
\end{equation}
We now compute 
\begin{align*}
\bfr\cdot\bfj &=\bfr\cdot\bfk \\
&\quad\Longrightarrow\quad
\sum_{i=\ell}^N r_i j_i = \sum_{i=\ell}^N r_i k_i
\quad\text{from \eqref{eqn:iellnekell2},} \\
&\quad\Longrightarrow\quad
\bigl| r_\ell (j_\ell-k_\ell) \bigr|
\le \sum_{i=\ell+1}^N \bigl| r_i (j_i-k_i) \bigr|
\quad\text{triangle inequality,}\\
&\quad\Longrightarrow\quad
(d+1)^{N-\ell}  |j_\ell-k_\ell|
\le \sum_{i=\ell+1}^N (d+1)^{N-i} |j_i-k_i| 
\quad\text{from \eqref{eqn:rjdj1d1x},} \\
&\quad\Longrightarrow\quad
(d+1)^{N-\ell} \le \sum_{i=\ell+1}^N (d+1)^{N-i} |j_i-k_i|  
\quad\text{from \eqref{eqn:iellnekell2},} \\
&\quad\Longrightarrow\quad
(d+1)^{N-\ell} \le (d+1)^{N-\ell-1} \cdot d + (d+1)^{N-\ell-2} \cdot d \\
&\omit\hfill\text{\parbox{.6\hsize}{since $|\bfj|=|\bfk|=d$, so the right-hand side is 
maximized by taking $j_{\ell+1}=k_{\ell+2}=d$,}} \\
&\quad\Longrightarrow\quad
(d+1)^2 \le (d+1)\cdot d + d = (d+1)^2-1 \\
&\omit\hfill\text{\parbox{.6\hsize}{dividing both sides by $(d+1)^{N-\ell-2}$.}}
\end{align*}
This contradiction proves that the map~\eqref{eqn:idtorix} is injective for the choice~\eqref{eqn:rjdj1d1x} of~$r_0,\ldots,r_N$. 
\end{proof}


%%%%%%%%%%%%%%%%%%%%%%%%%%%%%%%%%%%%%%
\section{K3 Surfaces}
\label{section:K3surfaces}
%%%%%%%%%%%%%%%%%%%%%%%%%%%%%%%%%%%%%%



\begin{theorem}
\label{theorem:K3}
Let~$X/\Qbar$ be a smooth projective~K3 surface, let~$f:X\to{X}$ be an endomorphism, and assume that there is a point~$P_0\in{X(\Qbar)}$ whose $f$-orbit~$\Orbit_f(P_0)$ is Zariski dense in~$X$. Then there is a common field of definition~$K$ for~$X$,~$f$ and~$P_0$ such that for any finite collection of grand $f$-orbits $\G_1,\ldots,\G_r\subset{X}(\Qbar)$, we have
\begin{equation}
\label{eqn:K3XminusGi}
\overline{X(K) \setminus (\G_1\cup\cdots\cup\G_r)} = X.
\end{equation}
In the terminology of Table~\textup{\ref{table:propogationstatements}}, endomorphisms of~$X$ satisfy the orbit propagation statement
\[
\textup{(A)} \quad\Longrightarrow\quad \textup{(C1)}.
\]
\end{theorem}
\begin{proof}
% [1] MR3586372
% [2] MR2917181
We first recall the well-known fact that all dominant endomorphisms of a smooth algebraic~K3 surface~$X$ are automorphisms; cf.~\cite[Section1.2]{MR2597304}. To see this, we start with the general formula
\[
f^*K_X = K_X + \text{(Ramification divisor of $f$)}.
\]
Since~K3 surfaces have trivial canonical bundle~$K_X=0$, we conclude that~$f$ is unramified. But~$X(\CC)$ is simply connected, and hence~$f$ is an automorphism.
%%%%%%%%%%%%%%%%%%%%%%%%%%%%%%%%%%%%%
% An Introduction To K3 Surfaces And Their Dynamics,
% Simion Filip May 2019
% \url{https://math.uchicago.edu/~sfilip/public_files/lectures_k3_dynamics.pdf}
% This is a nice summary of the geometry and dynamics of automorphisms of K3 
% surfaces. But I do not see any discussion of endomorphisms of K3 surfaces.
%%%%%%%%%%%%%%%%%%%%%%%%%%%%%%%%%%%%%
\par
We note that the existence of a point with Zariski dense~$f$-orbit implies in particular that~$f$, and hence also~$f^{-1}$, have infinite order.
Further, as noted in Lemma~\ref{lemma:orbitelemproperties}, the fact that~$f$ is an automorphism implies that the grand $f$-orbit of a point~$P$ is the union of the $f$-orbit of~$P$ and the~$f^{-1}$-orbit of~$P$. Hence it suffices to prove~\eqref{eqn:K3XminusGi} for~$f$ under the assumption that the~$\G_i$ are forward $f$-orbits, since we can then apply the same reasoning to~$f^{-1}$. 
\par
\par
We say that a curve~$C\subset{X}$ is \emph{$f$-periodic} if~$f^n(C)={C}$ for some $n\ge1$.\footnote{We note that since~$f$ is an automorphism, a curve is $f$-periodic if and only if it is~$f$-preperiodic.} Otherwise we say that~$C$ is
\emph{$f$-wandering}. Our first step is to produce an $f$-wandering rational curve on~$X$. To do this, we use the following two results.
%%%%%%%%%%%
\begin{lemma}
\label{lemma:21}
\cite[Theorem~2.1]{MR2917181}
Let~$D$ be a non-trivial effective divisor on~$X$. Then~$D$ is linearly equivalent to a sum of rational curves.
\end{lemma}
%%%%%%%%%%%
\begin{lemma}
\label{lemma:22}
\cite[Chapter~5,~Corollary~3.5]{MR3586372}
Let~$D$ be an ample divisor on~$X$, and let~$g:X\to{X}$ be an automorphism such that $g_*D\sim{D}$. Then~$g$ is of finite order.
\end{lemma}
%%%%%%%%%%%
Resuming the proof of Theorem~\ref{theorem:K3}, we suppose that all rational curves in~$X$ are $f$-periodic and derive a contradiction. Let~$D$ be an ample effective divisor on~$X$. By Lemma~\ref{lemma:21}, there are rational curves~$Y_1,\ldots,Y_m$ and integers~$a_1,\ldots,a_m$ such that
\[
D \sim a_1Y_1+\cdots+a_mY_m.
\]
The rational curves~$Y_j$ are assumed to be $f$-periodic. Let $p\ge1$ be a positive integer such that $f^p(Y_j)=Y_j$ for all~$1\le{j}\le{m}$. Since~$f$ is an automorphism, this implies that $f_*^pY_j=Y_j$ as divisors. Hence
\[
f_*^pD \sim a_1f_*^pY_1+\cdots+a_mf_*^pY_m=a_1Y_1+\cdots+a_mY_m\sim D.
\]
Since~$D$ is ample by assumption, Lemma~\ref{lemma:22} implies that~$f^p$ has finite order, which contradicts the assumption that there exists a Zariski dense $f$-orbit.
\par
We let~$Y$ be an $f$-wandering rational curve on~$X$. Replacing~$K$ with a finite extension, which by abuse of notation we continue to call~$K$, we may assume that~$Y$ is defined over~$K$ and that~$Y(K)$ is Zariski dense in~$Y$.
\par
Let~$\Fcal\subset{X(K)}$ be a finite set of points, and let
\[
\Orbit_f^\grand(\Fcal) := \bigcup_{Q\in\Fcal} \Orbit_f^\grand(Q)
= \bigcup_{Q\in\Fcal} \Bigl( \Orbit_f(Q) \cup \Orbit_{f^{-1}}(Q) \Bigr)
\]
be the union of the $f$-orbits of the points in~$\Fcal$. 
Our goal is to prove that~$X(K)$ has a Zariski dense set that is disjoint from~$\Orbit_f^\grand(\Fcal)$. 
% The fact that the curve~$Y$ is~$f$-wandering tells us that the union of curves
% \begin{equation}
% \label{eqn:fnRZDinX}
% \bigcup_{n\ge0} f^n(Y)\quad\text{is Zariski dense in $X$.}
% \end{equation}
\par
We know from earlier that~$Y(K)$ is Zariski dense in~$Y$. Further, the dynamical Mordell--Lang conjecture is true for \'etale maps~\cite{MR2766180}, so in particular for automorphisms such as~$f$ and~$f^{-1}$.  Hence 
\begin{equation}
\label{eqn:MLforetale}
\left(\begin{tabular}{@{}l@{}}
$\Orbit_f^\grand(Q)\cap C$ is finite for every point \\
$Q\in X(K)$ and every  curve~$C\subset{X}$ \\
\end{tabular}
\right).
\end{equation}
In particular, since~$f^n(Y)$ is a curve for every~$n\ge0$, and since~$\Orbit_f^\grand(\Fcal)$ is a finite union of $f$-orbits and~$f^{-1}$-orbits, it follows
from~\eqref{eqn:MLforetale} that
\begin{equation}
\label{eqn:fnRZDinX2}
\underbrace{f^n\bigl(Y(K)\bigr)\setminus\Orbit_f^\grand(\Fcal)}_{\hidewidth\text{This is an infinite subset of the curve $f^n(Y)$.}\hidewidth}\quad\text{is Zariski dense in $f^n(Y)$.}
\end{equation}
But the fact that~$Y$ is wandering implies that the union of curves
\begin{equation}
\label{eqn:fnRZDinX3}
\bigcup_{n\ge0} f^n(Y)\quad\text{is Zariski dense in $X$,}
\end{equation}
and combining~\eqref{eqn:fnRZDinX2} and~\eqref{eqn:fnRZDinX3}, we conclude that 
\[
\bigcup_{n\ge0} \Bigl( f^n\bigl(Y(K)\bigr)\setminus\Orbit_f^\grand(\Fcal) \Bigr)
\quad\text{is Zariski dense in $X$.}
\]
Hence
\[
X(K) \setminus \Orbit_f^\grand(\Fcal)
\supseteq 
\biggl( \bigcup_{n\ge0} \Bigl( f^n\bigl(Y(K)\bigr) \biggr)
 \setminus \Orbit_f^\grand(\Fcal),
\]
is also Zariski dense in~$X$, which completes the proof that~(A) implies~(C1) for K3 surfaces.
\end{proof}




%%%%%%%%%%%%%%%%%%%%%%%%%%%%%%%%%%%%%%
\section{Abelian Varieties}
\label{section:abelianvariety}
%%%%%%%%%%%%%%%%%%%%%%%%%%%%%%%%%%%%%%



\begin{theorem}
\label{theorem:simpleabelianvariety}
Let~$X/\Qbar$ be a geometrically simple abelian variety, let~$f:X\to{X}$ be an endomorphism of~$X$ \textup(as an abstract variety\textup), and let~$P_0\in{X}(\Qbar)$ be a point such that~$\Orbit_f(P_0)$ is Zariski dense in~$X$. Then there is a number field~$K/\QQ$ such that~$X$ and~$f$ are defined over~$K$ and such that every complete set of representatives for 
\[
\text{$X_f^\dense(K)/\fgeq$\quad  is Zariski dense in $X$.}
\]
In the terminology of Table~\textup{\ref{table:propogationstatements}}, endomorphisms of geometrically simple abelian varieties satisfy the orbit propagation statement
\[
\textup{(A)} \quad\Longrightarrow\quad \textup{(C3$\forall$)}.
\]
\end{theorem}

\begin{proof}
We start by fixing a field of definition~$K$ for~$X$,~$f$, and~$P_0$, but we may at times replace~$K$ with a finite extension. Every map between abelian varieties is the composition of a homomorphism and a translation~\cite[Section~4, Corollary~1]{MR2514037}, say
\[
f(x) = \f(x) + Q_0,\;
\text{with $Q_0\in X(K)$ and $\f:X\to X$ a homomorphism.}
\]
The assumption that~$P_0\in{X}(K)$ satisfies
\[
\overline{\Orbit_f(P_0)} = X
\]
implies, in particular, that~$f$ is dominant, and thus that~$\f$ is a non-zero isogeny. 
\par
The assumption that~$X$ is geometrically simple implies that the  kernel of~$\f-1$ is either a finite subgroup of~$X$ or all of~$X$. We consider these two case in turn.
\Case{1}{$\Ker(\f-1)$ is finite}
In this case~$\f-1$ is an isogeny, i.e., the map~$\f-1:X\to{X}$ is a finite surjective map. Hence we can find a point~$R_0\in{X(\Kbar)}$ satisfying
\[
(\f-1)(R_0) = Q_0.
\]
Writing~$T_P$ in general for the translation-by-$P$ map, we have
\begin{align*}
T_{R_0}\circ f\circ T_{-R_0}(x)
&= f(x-R_0) + R_0\\
&= \f(x-R_0)+Q_0+R_0\\
&= \f(x)-\f(R_0)+Q_0+R_0\\
&= \f(x).
\end{align*}
Thus~$f$ and~$\f$ are conjugate via the automorphism~$T_{R_0}\in\Aut(X)$, so they have the same dynamics. Hence in Case~1, it suffices to consider the case that~$f$ itself is an isogeny.
\par
Let~$\Mcal$ be an infinite set of positive integers with the following properties:
\begin{align}
\label{eqn:Mgcdfm1}
\gcd\bigl(\deg(f),m\bigr)&=1~\text{for all $m\in\Mcal$.} \\
\label{eqn:Mgcdm1m2}\gcd(m_1,m_2)&=1~\text{for all distinct $m_1,m_2\in\Mcal$.} 
\end{align}
For example, we could take~$\Mcal$ to be the set of all primes not dividing~$\deg(f)$. Then for~$m_1,m_2\in\Mcal$ we compute
\begin{align*}
    \Orbit_f&(m_1P_0)  \cap \Orbit_f(m_2P_0) \ne \emptyset \\
    &\quad\Longleftrightarrow\quad
    f^{n_1}(m_1P_0) = f^{n_2}(m_2P_0) \quad\text{for some $n_1,n_2\in\NN$,} \\
    &\quad\Longrightarrow\quad
    (f^{n_1}\circ m_1 -  f^{n_2}\circ m_2)(P_0)=0 \quad\text{for some $n_1,n_2\in\NN$,} \\
    &\quad\Longrightarrow\quad
   f^{n_1}\circ m_1 -  f^{n_2}\circ m_2 = 0 \quad
   \text{\parbox[t]{.4\hsize}{in $\ZZ[f]\subseteq\Isog(X)$ for some $n_1,n_2\in\NN$,
   since $P_0$ is non-torsion because its $f$-orbit is Zariski dense,}} \\
    &\quad\Longrightarrow\quad
   (\deg f)^{n_1}\cdot m_1^{2g} = \deg(f)^{n_2}\cdot m_2^{2g} 
   \quad\text{for some $n_1,n_2\in\NN$,} \\
    &\quad\Longrightarrow\quad
    m_1=m_2 \quad\text{from \eqref{eqn:Mgcdfm1} and \eqref{eqn:Mgcdm1m2}.}
\end{align*}
Taking the contrapositive, we have proven that
\[
\text{$m_1,m_2\in\Mcal_f$ and $m_1\ne m_2$}
\quad\Longrightarrow\quad
\Orbit_f(m_1P_0)  \cap\Orbit_f(m_2P_0)=\emptyset.
\]
Hence the grand $f$-orbits generated by the points in~$\{mP_0:m\in\Mcal_f\}$ are distinct.
\par
We claim that the $f$-orbits of these points are also Zariski dense. To prove this, we note that since~$f$ is an isogeny, it commutes with multiplication-by-$m$, and since~$f$ is a finite map of a proper variety, it sends closed sets to closed sets. Hence
\begin{align}
\label{eqn:OfmPZC}
\overline{\Orbit_f(mP_0)} &= \overline{m\cdot\Orbit_f(P_0)}
\quad\text{since $fm=mf$,} \notag\\
&= m\cdot\overline{\Orbit_f(P_0)}
\quad\text{since $m$ is finite and $X$ proper.}
\end{align}
It follows immediately from~\eqref{eqn:OfmPZC} that\footnote{The converse also holds, since if~$\Orbit_f(P_0)$ is not Zariski dense, then its Zariski closure is a subvariety whose dimension is strictly smaller than~$\dim(X)$, and multiplication-by-$m$ preserves the dimension, so~$\Orbit_f(mP_0)$ is also not Zariski dense.}
\[
\overline{\Orbit_f(P_0)} = X\quad\Longrightarrow\quad \overline{m\Orbit_f(P_0)} = X.
\]
Since the $f$-orbit of~$P_0$ is Zariski dense by assumption, the same is true of the~$f$-orbit of~$mP_0$ for all positive integers~$m$.
\par
We let
\[
\Mcal\cdot P_0 := \{mP_0:m\in\Mcal\},
\]
and we consider the following statements:
\begin{parts}
\Part{(1)}
Every point in $\Mcal\cdot{P_0}$ has a Zariski dense $f$-orbit.
\Part{(2)}
The points in $\Mcal\cdot{P_0}$ generate distinct grand $f$-orbits.
\Part{(3)}
The set $\Mcal\cdot{P_0}$ is Zariski dense.
\end{parts}
We have proven~(1) and~(2), and we now prove~(3). We first show that~$\Mcal\cdot{P_0}$ is an infinite set. Since~$\Mcal$ is an infinite set of integers, it suffices to show that~$P_0$ is a non-torsion point. To see this, we assume that~$mP_0=0$ for some non-zero integer and derive a contradiction. The fact that~$f$ is an isogeny implies that~$f$ commutes with the multiplication-by-$m$ map, so we have
\[
\Orbit_f(P_0) \subseteq \ZZ[f]\cdot P_0 = (\ZZ[f]/m\ZZ[f])\cdot P_0.
\]
The ring~$\ZZ[f]/m\ZZ[f]$ is finite, so the $f$-orbit~$\Orbit_f(P_0)$ is finite, i.e., the point~$P_0$ is~$f$-preperiodic. This contradicts the assumption that~$\Orbit_f(P_0)$ is Zariski density, and completes the proof that~$\Mcal\cdot{P_0}$ is an infinite set of points.
\par
The Zariski closure $\overline{\Mcal\cdot{P_0}}$ contains the infinite set~$\Mcal\cdot{P_0}$, which in turn is a subset of the finitely generated (indeed, rank~$1$) subgroup~$\ZZ\cdot{P_0}$ of~$X$. It follows from Faltings's theorem~\cite{MR1109353} (originally the Mordell--Lang conjecture) that~$\overline{\Mcal\cdot{P_0}}$ contains a translate of an abelian subvariety of~$X$, necessarily positive dimensional since~$\#(\Mcal\cdot{P_0})=\infty$. The assumed simplicity of~$X$ tells us that the only such abelian subvariety is~$X$ itself. Hence $\overline{\Mcal\cdot{P_0}}=X$, which completes the proof of~(3).
\par
We can restate~(1) as the inclusion~$\Mcal\cdot{P_0}\subseteq{X_f^\dense}$, and then~(3) tells us that~$X_f^\dense(K)$ contains a Zariski dense set of points. We have thus proven the following useful fact:
\begin{equation}
\label{eqn:Xfdenseisdenseabvar}
\overline{X_f^\dense(K)} = X.
\end{equation}
% \par
% Finally, we note that~(1)--(3) gives an explicit proof of the orbit propagation property~(C3$\exists$), since we have shown that~$\Mcal\cdot{P_0}$ is a Zariski dense set of points in~$X(K)$ such that each point in~$\Mcal\cdot{P_0}$ has a Zariski dense~$f$-orbit and such the points in~$\Mcal\cdot{P_0}$ lie in distinct grand~$f$-orbits. 
\par
Our next goal is to prove that~$f$ has propagation property~(C2$\forall$). So we let $\Qcal\subset{X(K)}$ be a set of representatives for the grand $f$-orbits, i.e., a set satisfying
\begin{equation}
\label{eqn:QbijectXKfgeqabvar}
\Qcal \xleftrightarrow{\;\text{bijective}\;} X(K)/\fgeq,
\end{equation}
and we need to show that~$\Qcal$ is Zariski dense. We let
\[
Y = \overline\Qcal \subseteq X,
\]
and our goal is to prove that~$Y=X$.
\par
We proved earlier that the set~$\Mcal\cdot{P_0}$ is Zariski dense and that each point in~$\Mcal\cdot{P_0}$ is in a distinct grand~$f$-orbit. The choice~\eqref{eqn:QbijectXKfgeqabvar} of~$\Qcal$ says that every grand $f$-orbit containing a~$K$-rational point will contain a unique point of~$\Qcal$. This allows us to  define an injection
\[
\psi : \Mcal\cdot{P_0} \longhookrightarrow\Qcal,
\quad
\left(
\text{\begin{tabular}{@{}l@{}}
$\psi(mP_0)$ is the unique point~$Q\in\Qcal$\\
such that $\Orbit_f(Q)^\grand=\Orbit_f(mP_0)^\grand$\\
\end{tabular}}
\right).
% \text{\parbox{.55\hsize}{$\psi(mP_0)$ is the unique point~$Q\in\Qcal$ such that
% $\Orbit_f(Q)^\grand=\Orbit_f(mP_0)^\grand$}}\;\right).
% \text{\parbox{.5\hsize}{$\psi(mP_0)$ is the unique point~$Q\in\Qcal$ such that there
% exist~$n_1,n_2\ge0$ satisfying $f^{n_1}(Q)=f^{n_2}(mP_0)$}}\;\right).
\]
The existence of Zariski dense $f$-orbit~$\Orbit_f(P_0)$ tells us, in particular, that~$f\ne0$, so there exists an isogeny $g:X\to{X}$ satisfying
\[
f\circ g = g\circ f = q \in \ZZ_{>0}.
\]
%%%%%%%%%%%%%%%%%%%%%
% Justification for the assertion that $g$ exists.
% There is a reduced norm map $\Norm:\Isog(X)\to\ZZ$ such 
% that $\Norm(\a)$ is a product of elements of~$\Isog(X)$, 
% one of which is~$\a$. Further, we have $\Norm(\a)=0$ 
% if and only if $\a=0$. So we can let~$g=\Norm(f)/f$.
%%%%%%%%%%%%%%%%%%%%%%
The condition that $\Orbit_f(Q)^\grand=\Orbit_f(mP_0)^\grand$ with $Q=\psi(mP_0)$
says that there are integers~$n_1,n_2\ge0$ satisfying 
\begin{equation}
\label{fn1Qfn2mP0abvar}
f^{n_1}(Q)=f^{n_2}(mP_0)
\end{equation}
Applying~$g^{n_1}$ to both sides of~\eqref{fn1Qfn2mP0abvar}, we find that for $mP_0\in\Mcal$ there are non-negative integers~$n_1$ and~$n_2$ satisfying
\[
q \psi(mP_0) = g^{n_1}\circ f^{n_1}\bigl(\psi(mP_0)\bigr) 
= g^{n_1}\circ f^{n_2}(mP_0) \in \Isog(X)\cdot P_0.
\]
Hence\footnote{We use that notation that for any subgroup~$Z\subset{X(\Kbar)}$, the division subgroup associated to~$Z$ is
$
Z^{\operatorname{div}} 
:= \bigl\{ P \in X(\Kbar): \text{$nP\in Z$ for some integer $n\ge1$} \bigr\}. 
$
For example, $\{0\}^{\operatorname{div}}=X(\Kbar)_\tors$.}
\[
\psi(mP_0) \in \Bigl( \Isog(X)\cdot P_0 \Bigr)^{\operatorname{div}}
\quad\text{for all $m\in\Mcal$.}
\]
To recapitulate, we have proven that
\begin{equation}
\label{eqn:YcapIsgoXP01}
\psi(\Mcal\cdot P_0)
\subseteq \Qcal \cap \Bigl( \Isog(X)\cdot P_0 \Bigr)^{\operatorname{div}}
\subseteq Y \cap \Bigl( \Isog(X)\cdot P_0 \Bigr)^{\operatorname{div}} .
\end{equation}
\par
The ring of isogenies~$\Isog(X)$ is a finitely generated~$\ZZ$-module. Indeed, it is a classical result that~$\Isog(X)$ is an order in a central simple algebra of rank at most~$2\dim(X)$; see for example~\cite[section~19]{MR2514037}. Hence the subgroup~$\Isog(X)\cdot{P}$ is a finitely generated subgroup of~$X$. We now apply~\cite{MR1323985}, which represents a culmination of fundamental work of Faltings, Vojta, Raynaud, Hindry, and others on the intersection of a subvariety of a semi-abelian variety with a subgroup of finite type. This result tells us that
\begin{equation}
\label{eqn:YcapIsgoXP02}
\overline{Y \cap \Bigl( \Isog(X)\cdot P_0 \Bigr)^{\operatorname{div}}}
= 
\left(\;\text{\parbox{.5\hsize}{the union of finitely many translates of abelian subvarieties of $X$}}\;\right).
\end{equation}
It follows from~\eqref{eqn:YcapIsgoXP01} that the set~\eqref{eqn:YcapIsgoXP02} contains~$\psi(\Mcal\cdot{P_0})$. We know that~$\psi$ is injective, and we proved earlier that~$\Mcal\cdot{P_0}$ is an infinite set. It follows that at least one of the abelian subvarieties appearing in the set~\eqref{eqn:YcapIsgoXP02} is positive dimensional. On the other hand, since~$Y$ is Zariski closed, it contains the set~\eqref{eqn:YcapIsgoXP02}, so we conclude that~$Y$ contains a translate of a positive dimensional abelian subvariety of~$X$. Our assumption that~$X$ is geometrically simple implies~$X$ has no non-trival positive dimensional abelian subvarieties. Hence~$Y=X$, which concludes our proof that~$f$ has propagation property~(C2$\forall$).
\par
We know that~$f$ has propagation property~(A) by assumption, we have proven that~$f$ has propagation property~(C2$\forall$), and we have proven~\eqref{eqn:Xfdenseisdenseabvar} that~$X_f^\dense(K)$ is Zariski dense in~$X$. It follows from Lemma~\ref{lemma:C2forallZDimpliesC3forall} that~$f$ has propagation property~(C3$\forall$).
\Case{2}{$\Ker(\f-1)=X$}
In this case~$\f=1$, so~$f$ is a pure translation
\[
f(x)=x+Q_0.
\]
Hence for all $n\in\NN$, we have~$f^n(x)=x+nQ_0$, which shows that
\begin{equation}
\label{eqn:OfxxplusNQ0}
\Orbit_f(x) = x + \NN\cdot Q_0.
\end{equation}
It follows that for any~$P\in{X}$ we have
\[
\Orbit_f(P) = P+\NN\cdot Q_0 = (P-P_0)+(P_0+\NN\cdot Q_0) = (P-P_0)+\Orbit_f(P_0),
\]
i.e., all~$f$-orbits are translates of one another. Hence the fact that~$\Orbit_f(P_0)$ is Zariski dense in~$X$ implies that every~$f$-orbit is Zariski dense in~$X$. This tells us that
\begin{equation}
\label{eqn:abvartransXfdenseX}    
\overline{X_f^\dense(K)} = \overline{X(K)} = X,
\end{equation}
where the second equality follows from the fact that~$\Orbit_f(P_0)$ is a Zariski dense subset of~$X(K)$.
\par
We next replace~$K$ with a finite extension, which by abuse of notation we again denote by~$K$, such that~$\rank{X(K)}\ge2$. This is always possible, since more generally it is known~\cite[Theorem~10.1]{MR337997} that~$\rank{X(\Kbar)}=\infty$. We let~$Q_1\in{X(K)}$ be a point such that~$Q_0$ and~$Q_1$ are~$\ZZ$-linearly independent. 
We claim that~$\NN\cdot{Q_1}$ has the following three properties:
%  , which suffice to prove that~(C3$\exists$) is true. 
\begin{parts}
\Part{(1)}
Every point in $\NN\cdot{Q_1}$ has a Zariski dense $f$-orbit.
\Part{(2)}
The points in $\NN\cdot{Q_1}$ generate distinct grand $f$-orbits.
\Part{(3)}
The set $\NN\cdot{Q_1}$ is Zariski dense in~$X$.
\end{parts}
The verification of~(1) is immediate, since as already noted, every point in~$X$ has Zariski dense $f$-orbit. For~(3), we note that since~$Q_1$ is a non-torsion point, any subvariety~$Y$ of~$X$ containing~$\NN\cdot{Q_1}$ would have infinite intersection with the finitely generated subgroup~$\ZZ\cdot{Q_1}$ of~$X$. Faltings's theorem~\cite{MR1109353} and the assumed simplicity of~$X$ then implies that~$Y=X$, which proves that~$\NN\cdot{Q_1}$ is Zariski dense.\footnote{We probably don't need the full strength of Faltings's theorem, since the assumption that~$Y$ contains all of~$\NN\cdot{Q_1}$ is much stronger than the assumption that $\#\bigl(Y\cap(\NN\cdot{Q_1})\bigr)=\infty$. It would suffice to show that the Zariski closure of $\NN\cdot{Q_1}$ is an algebraic subgroup of $X$.}
\par
It remains to prove~(2). For~$mQ_1,m'Q_1\in\NN\cdot{Q_1}$ we compute
\begin{align*}
\Orbit_f&(mQ_1)^\grand \cap \Orbit_f(m'Q_1)^\grand \ne \emptyset\\
&\quad\Longleftrightarrow\quad
f^n(mQ_1) = f^{n'}(m'Q_1) \quad\text{for some $n,n'\in\NN$,} \\
&\quad\Longleftrightarrow\quad
mQ_1+nQ_0 = m'Q_1+n'Q_0 \quad\text{for some $n,n'\in\NN$,} \\
&\quad\Longleftrightarrow\quad
(m-m')Q_1= (n'-n)Q_0 \quad\text{for some $n,n'\in\NN$,} \\
&\quad\Longrightarrow\quad
m=m'\quad\text{since $Q_0$ and $Q_1$ are $\ZZ$-linearly independent.}
\end{align*}
This completes the proof that distinct points in~$\NN\cdot{Q_1}$ have distinct grand~$f$-orbits.
\par
The properties~(1),~(2),~(3) that we proved for the set~$\NN\cdot{Q_1}$ are exactly that same as the properties~(1),~(2),~(3) that we proved for the set~$\Mcal\cdot{P_0}$ while proving Case~1. The remainder of the proof that~$f$ has propagation property~(C3$\forall$) in Case~2 is the same, \emph{mutatis mutandis}, as the proof in Case~1 with the set~$\Mcal\cdot{P_0}$ in Case~1 replaced with the set~$\NN\cdot{Q_1}$ for Case~2.
\end{proof}

%%%%%%%%%%%%%%%%%%%%%%%%%%%%%%%%%%%%%%
\section{\texorpdfstring
{{\'E}tale Maps and $p$-adic Dimension}
{{\'E}tale Maps and p-adic Dimension}
} 
\label{section:etalepadic}
%%%%%%%%%%%%%%%%%%%%%%%%%%%%%%%%%%%%%%
In this section we use $p$-adic arguments to prove orbit propagation
results for {\'e}tale maps. 

\begin{definition}
We need to explain what we mean by the \emph{$p$-adic dimension} of an
algebraic variety. For a smooth variety, we use the same analytic
definition as for~$\RR$ and~$\CC$, namely we use local charts and
locally convergent power series, and we declare that~$\AA^n(\QQ_p)$
has $p$-adic dimension~$n$. For a singular variety, we write the variety as a union of smooth varieties by inductively removing singular loci, and
then the $p$-adic dimension of~$X(\QQ_p)$ is the maximum of the
$p$-adic dimensions of the smooth varieties appearing in the
decomposition. And since we will be using two notions of dimension in 
the section, we write~$\dimpadic$ for the $p$-adic dimension of a $p$-adic set,
and we write~$\dimKrull$ for the usual Krull, or algebraic, dimension of an algebraic variety.
\end{definition}

\begin{lemma}
\label{lemma:padicdimen}
\begin{parts}
\Part{(a)}
Let~$X/\QQ_p$ be a smooth algebraic variety. Then
\[
X(\QQ_p)\ne\emptyset \quad\Longrightarrow\quad
\dimpadic X(\QQ_p) = \dimKrull X.
\]
\Part{(b)}
Let~$K$ be a number field, and let~$X/K$ be a smooth algebraic variety.
Then for all but finitely many primes~$v$ of~$K$ satisfying~$K_v=\QQ_v$, we have
\[
\dimpadic X(K_v) = \dimKrull X.
\]
\Part{(c)}
Let~$X/\QQ_p$ be a smooth algebraic variety, and let~$Z\subsetneq{X}$
be a proper \textup(not necessarily smooth\textup) algebraic subset of~$Z$ defined over~$\QQ_p$. Then
\[
\dimpadic Z(\QQ_p) < \dimKrull X.
\]
\Part{(d)}
Let $K$ be a number field, let $X/K$ be a smooth projective variety with~$X(K)\ne\emptyset$, and suppose that there are infinitely many degree~$1$ primes~$v$ of~$K$ such that~$X(K)$ is $v$-adically dense in~$X(K_v)$. Then~$X(K)$ is Zariski dense in~$X$.
\end{parts}
\end{lemma}
\begin{proof}
(a)\enspace
Let~$n=\dimKrull X$ and~$Q\in{X}(\QQ_v)$. 
The smoothness of~$X$ and the $v$-adic implicit function theorem~\cite[page~73]{MR2179691} says that there is neighborhood of~$Q$ that is $v$-adic analytically
isomorphic to a neighborhood of the origin in~$\AA^n(\QQ_v)$. 
%% \cite[Theorem~7.1]{MR4247874} is a much fancier inverse function theorem
\par\noindent(b)\enspace
We choose a model~$\Xcal$ for~$X$ over the ring of integers of~$K$.
The smoothness of~$X$ tells us that the reduction of~$\Xcal$ modulo~$v$
is smooth for all but finitely many primes~$v$. If $\operatorname{char}(v)$ is sufficiently large, then the standard Lang--Weil estimate~\cite{MR65218} shows that~$X(\FF_v)\ne\emptyset$, and then smoothness and Hensel's lemma show that~$X(K_v)\ne\emptyset$. Restricting to primes with~$K_v=\QQ_v$, the desired result follows from~(a).
\par\noindent(c)\enspace  
We can check the assertion on Zariski open sets, and by Noetherian induction
on the dimension, we are reduced to proving that if~$V/\QQ_p$ is a smooth
(affine) variety, then
\[
\dimpadic V(\QQ_p) \le \dimKrull V.
\]
If~$V(\QQ_p)=\emptyset$, this is vacuously true. Otherwise we may apply~(a).
\par\noindent(d)\enspace  
We write~$\Closure_v$ for the $v$-adic closure of a set and~$\Closure_Z$ for the Zariski closure. So our assumption is that there are infinitely many degree~$1$ primes~$v$ with~$\Closure_v\bigl(X(K)\bigr)=X(K_v)$. We use this assumption and~(b) to deduce that there is a degree~$1$ prime~$v$, say of characteristic~$p\ge3$, such that
\begin{equation}
\label{eqn:dpCXKdpXKvdKX1}
\dimpadic \Closure_v\bigl(X(K)\bigr)
= \dimpadic X(K_v) = \dimKrull X.
\end{equation}
On the other hand, if $\Closure_Z\bigl(X(K)\bigr)$ is a proper subset of~$X$, then~(c) says that
\begin{equation}
\label{eqn:dpCXKdpXKvdKX2}
\dimpadic \bigl(\Closure_Z(X(K))\bigr)(K_v) < \dimKrull X
\quad\text{(strict inequality).}
\end{equation}
Since~\eqref{eqn:dpCXKdpXKvdKX1} and~\eqref{eqn:dpCXKdpXKvdKX2} are incompatible, we conclude that~$\Closure_Z\bigl(X(K)\bigr)$ is equal to~$X$.
\end{proof}

\begin{proposition} 
\label{proposition:schemeoverZp}
Let~$p\ge 3$ be prime, let~$\pi:\Xcal\to\Spec \ZZ_p$ be a smooth projective morphism with geometrically irreducible fibres, and let~$F:\Xcal\to \Xcal$ be a finite {\'e}tale~$\ZZ_p$-morphism. Let~$X=\Xcal\times_{\ZZ_p}\QQ_p$ be the generic fibre of~$\pi$, and let~$f:X\to X$ be the restriction of~$F$ to the generic fiber. Let~$x\in X(\QQ_p)$. Then the~$p$-adic closure of~$\Orbit_f(x)$ in~$X(\QQ_p)$ consists of finitely many~$p$-adic arcs, so in particular it has~$p$-adic dimension~$1$.
\end{proposition}
\begin{proof} 
We let~$X'=\Xcal\times_{\ZZ_p}\FF_p$ be the special fiber of~$\pi$, and 
we let~$n$ be the relative dimension $n$ of~$\pi$, so~$X$ and~$X'$ are smooth projective varieties of dimension~$n$. By properness,  the point $x\in{X(\QQ_p)}$ extends to a section $\s_x:\Spec\ZZ_p\to\Xcal$ of $\pi$, and the reduction of~$\s_x$ to the special fiber gives a point $x'\in X'(\F_p)$. Since $X(\F_p)$ is finite, there is no loss of generality if we assume that $f(x')=x'$.
%%%%%%%%%%%%%%%%%%%%%%
% Details of this WLOG assertion: Suppose $f^i(x')=f^j(x')$ with
% $i>j\ge0$. We may discard finitely many points from the orbit of~$x$
% without affected the conclusion, so starting with~$x_1:=f^j(x)$ in
% place of~$x$, we get $f^{i-j}(x_1')=x_1'$. So we are reduced to the
% case that~$f^k(x')=x'$ for some $k\ge1$. It then suffices to replace
% $f$ with $f^k$ and to prove the result for each of the starting points
% $x,f(x),\ldots,f^{k-1}(x)$.
%%%%%%%%%%%%%%%%%%%%%%%
\par
Expanding $f$ locally on the residue disk of $x'$ centered along $\sigma_x$, we see that $f$ is given by an $n$-tuple of convergent power series $g_1,...,g_n\in \ZZ_p[\![t_1,...,t_n]\!]$, where the $t_j$ are local parameters along $\sigma$. As in~\cite[Proposition~2.2]{MR2766180}, we may choose the $t_j$ so that the $g_j$ are congruent to linear polynomials modulo $p$. Since $f$ is \'etale, the Jacobian of the $g_j$ reduces modulo $p$ to a matrix in $\GL_n(\FF_p)$ that is necessarily of finite order. Replacing $f$ by a suitable iterate, we get that 
\[
(g_1,...,g_n) \equiv (t_1,...,t_n) \pmod{p}.
\]
It now follows from~\cite{MR3210707} that the map
\[
\NN\longrightarrow X(\QQ_p),\quad
n\longmapsto f^n(x),
\]
extends to a locally $p$-adic analytic map
\begin{equation}
\label{eqn:ZptoXQp}
\ZZ_p\longrightarrow X(\QQ_p). 
\end{equation}
Hence the $p$-adic closure of~$\Orbit_f(x)$ is contained in the image of~\eqref{eqn:ZptoXQp}.
\end{proof}

% For the convenience of the reader, we include a proof of the following elementary result in point-set topology.

% \begin{lemma}
% \label{lemma:ClAnegBsupClAnegClB}
% Let~$A$ and~$B$ be subsets of a topological space. Then
% \[
% \Closure(A\setminus B) \supseteq \Closure(A)\setminus\Closure(B).
% \]
% \end{lemma}
% \begin{proof}
% Let $\a\in\Closure(A)\setminus\Closure(B)$. The fact that~$\a\in\Closure(A)$ says that we can find a sequence~$(\a_i)\subset{A}$ such that~$\a_i\to\a$. Suppose that
% \[
% \#\bigl( \{\a_i\} \cap B \bigr) = \infty.
% \]
% Then taking the subsequence consisting of the~$\a_i$ that are in~$B$, we could obtain a sequence~$(\b_i)\subset{B}$ satisfying~$\b_i\to\a$. That contradicts the assumption that~$\a\notin\Closure(B)$. This proves that~\text{$\{\a_i\}\cap{B}$} is a finite set, so discarding finitely many of the~$\a_i$, we may assume that \text{$\{\a_i\}\cap{B}=\emptyset$} Then~$\{\a_i\}\subset(A\setminus{B})$ satisfies~$\a_i\to\a$, which completes the proof that~\text{$\a\in\Closure(A\setminus{B})$}.
% \end{proof}


\begin{theorem} 
\label{theorem:XQpdim2}
Let $K$ be a number field, let $X/K$ be a smooth projective variety of dimension at least~$2$, and let $f:X\to X$ be a finite {\'e}tale morphism. Suppose that there are infinitely many degree~$1$ primes $v$ of $K$, i.e., primes satisfying $K_v=\QQ_v$, such that~$X(K)$ is $v$-adically dense in~$X(\QQ_v)$. Then~\textup{(B1)} in Table~$\ref{table:propogationstatements}$ is true for~$(X,f)$, i.e., if~$\G_1,\ldots,\G_r\subset{X(K)}$ are~$f$-orbits, then
\[
\text{$X(K)\setminus(\G_1\cup\cdots\cup\G_r)$
is Zariski dense in $X$.}
\]
% \[
% \text{\textup{(A)} for $(X,f)$}
% \quad\Longrightarrow\quad
% \text{\textup{(B1)} for $(X,f)$.}
% \]
\end{theorem}

Before proving Theorem~\ref{theorem:XQpdim2}, we state and prove two corollaries, one of which will require a preiminary lemma.

\begin{corollary}
\label{corollary:CoroRat}
Let~$X/K$ be a smooth rational variety defined over a number field, and let~$f:X\to{X}$ be a finite {\'e}tale morphism defined over~$K$. Then~\textup{(B1)} in Table~$\ref{table:propogationstatements}$ is true for~$(X,f)$. In particular,~\textup{(B1)} is true for automorphisms of rational varieties.
\end{corollary}

\begin{corollary}
\label{corollary:B1foretaleabelianquotients}
Let~$X$ be an {\'e}tale quotient of an abelian variety,\footnote{In other words, there is an abelian variety~$A$ and a finite group of automorphisms~$G\subset\Aut(A)$ (not necessarily isogenies) so that $X\cong{A}/G$ and $A\to{X}$ is {\'e}tale.} and let $f:X\to{X}$ a morphism. Then~\textup{(B1)} in Table~$\ref{table:propogationstatements}$ is true for~$(X,f)$. In particular,~\textup{(B1)} is true for abelian varieties and for bielliptic surfaces.
\end{corollary}

\begin{proof}[Proof of Corollary~$\ref{corollary:CoroRat}$]
Possibly after replacing~$K$ with a finite extension, we can find a birational map $\f:\PP^N\dashrightarrow{X}$ defined over~$K$, and we let~$U\subseteq\PP^N$ be a non-empty Zariski open subset defined over~$K$ such that~$\f|_U$ is an isomorphism onto its image. The fact that~$U$ is an open subset of~$\PP^N$ implies that~$U(K)$ is $v$-adically dense in~$U(K_v)$ for all places of~$K$. (This follows simply from the fact that~$K$ is $v$-adically dense in~$K_v$.) It follows that~$\f\bigl(U(K)\bigr)=\f(U)(K)$ is~$v$-adically dense in~$\f(U)(K_v)$,
and then since~$\f(U)$ is a Zariski open subset of the irreducible variety~$X$, we see that~$\f(U)(K)$ is $v$-adically dense in~$X(K_v)$. Hence the large set~$X(K)$ is $v$-adically dense in~$X(K_v)$, and the desired conclusion follows from Theorem~\ref{theorem:XQpdim2}.
\end{proof}

\begin{lemma}
\label{lemma:etalequotabelianvariety}
Let~$A$ be abelian variety, let $\pi:A\to B$ be an {\'e}tale map, and let $f:B\to B$ be a surjective morphism. Then $f$ is {\'e}tale.  
\end{lemma}
\begin{proof}
We let~$X$ be the fiber product of~$A$ and~$B$ relative to the map~$\pi$ and~$f$, so we have a commutative diagram
\[
\begin{CD}
  X @>h>> A \\
  @V p VV @V \pi VV \\
  B @>f>> B \\
\end{CD}
\]
The assumption that~$\pi$ is {\'e}tale implies that~$X$ is smooth and~$p$ is {\'e}tale.
\par
The fact that~$p$ is {\'e}tale tells us that~$\kappa(X)=0$ and that~$\Omega^1_X$ is semiample, since~$B$ has these properties, so~\cite{MR1183561} allows us to conclude that~$X$ is the quotientof an abelian variety~$C$. Considering the composite ma $C\to{X}\to{A}$, we conclude that~$X\to{A}$ is {\'e}tale, and thus that $f:B\to{B}$ is also {\'e}tale.
\end{proof}

\begin{proof}[Proof of Corollary $\ref{corollary:B1foretaleabelianquotients}$]
Lemma~\ref{lemma:etalequotabelianvariety} says that it is enough to check the case that~$X$ is an abelian variety. Since every map between dominant map~$X\to{X}$ of an abelian variety is {\'e}tale, Theorem~\ref{theorem:XQpdim2} says that it suffices to check that for all but finitely many places~$v$ of~$K$, the set of~$K$-rational points~$X(K)$ is $v$-adically dense in~$X(K_v)$. This will be true provided that the rank of the Mordell-Weil group of each simple isogeny factor of~$X(K)$ is sufficiently large with respect to the dimension of~$X$; see~\cite{MR2793038}. And we can make these ranks sufficiently large by taking a finite extension of~$K$.
\end{proof}



\begin{proof}[Proof of Theorem~$\ref{theorem:XQpdim2}$]
We choose a model~$\Xcal\to\Spec(\Ocal_K)$ of~$X$ over the ring of integers~$\Ocal_K$ of~$K$, and then~$f$ extends to a rational map~$F:\Xcal\dashrightarrow\Xcal$. For any finite set of places~$S$ of~$K$ that includes the archimedean places, we denote the ring of~$S$-integers by~$\Ocal_{K,S}$. We choose such a set~$S$ having the property that the scheme~$\Xcal_S=\Xcal\times_{\Ocal_K}\Ocal_{K,S}$ is smooth and proper over~$\Ocal_{K,S}$ with geometrically irreducible fibers and such that the map~$F_S:\Xcal_S\to\Xcal_S$ is a dominant {\'e}tale morphism. Since the degree~$1$ primes have positive density, the assumption of Theorem~\ref{theorem:XQpdim2} and the finiteness of~$S$ imply that we can find a degree~$1$ prime~$v\notin{S}$ of characteristic $p\ge3$ such that $X(K)$ is $p$-adically dense in $X(\QQ_p)$. Using the smoothness of~$X$, the Lang--Weil estimate~\cite{MR65218}, and Hensel's lemma, we may also assume that~$X(\QQ_p)\ne\emptyset$. 
\par
Using the closure notation from the proof of Lemma~\ref{lemma:padicdimen}(d), we have
\begin{align}
\label{eqn:2ledimKrullX1}
2 &\le \dimKrull X
&&\text{by assumption,} \notag\\
&= \dimpadic X(\QQ_p)
&&\text{from Lemma~\ref{lemma:padicdimen}(a),} \\
\label{eqn:2ledimKrullX2}
&= \dimpadic \Closure_p\bigl(X(K)\bigr)
&&\begin{tabular}[t]{@{}l}
since $\Closure_p\bigl(X(K)\bigr)=X(\QQ_p)$\\ by assumption.\\
\end{tabular}
\end{align}
\par
On the other hand, for~$x\in{X(K)}$, it follows from Proposition~\ref{proposition:schemeoverZp} that 
\[
\dimpadic \Closure_p\bigl( \Orbit_f(x) \bigr) \le 1.
\]
To ease notation, we let
\[
\G := \G_1\cup\cdots\cup\G_r
\]
be the finite union of orbits in the statement of the Corollary. Then, since the dimension of a finite union of sets is the maximum of the dimensions of the components,  we have
\begin{equation}
\label{eqn:QpdimGammai}
\dimpadic \Closure_p(\G) \le 1.
\end{equation}
Thus~\eqref{eqn:2ledimKrullX1} says that $\Closure_p\bigl(X(K)\bigr)$ has $p$-adic dimension at least~$2$, while~\eqref{eqn:QpdimGammai} says that~$\Closure_p(\G))$ has $p$-adic dimension at most~$1$, so the complement satisfies
\[
\dimpadic\Bigl(\Closure_p\bigl(X(K)\bigr) \setminus \Closure_p(\G)) \Bigr)
= \dimpadic \Closure_p\bigl(X(K)\bigr).
\]
Applying~\eqref{eqn:2ledimKrullX2} gives
\begin{equation}
\label{eqn:dimpCpXKnegCpGdimK}
\dimpadic\Bigl(\Closure_p\bigl(X(K)\bigr) 
\setminus \Closure_p(\G) \Bigr)
= \dimKrull X.
\end{equation}
Elementary topology\footnote{It is an exercise to check that if~$U$ and~$V$ are subsets of a topological space, then their closures satisfy
$\Closure(U\setminus V)\supseteq\Closure(U)\setminus\Closure(V)$. See for example~\cite{MSE493989}.} tells us that
\[
\Closure_p\Bigl(X(K) \setminus \G\Bigr)
\supseteq
\Closure_p\bigl(X(K)\bigr) 
\setminus \Closure_p\bigl(\G\bigr),
\]
so taking $p$-adic dimension and using~\eqref{eqn:dimpCpXKnegCpGdimK} yields
\begin{equation}
\label{eqn:dimCpXKnegGgedimKX}
\dimpadic\Bigl( \Closure_p\bigl(X(K) \setminus \G\bigr) \Bigr) \ge \dimKrull X.
\end{equation}
\par
Let
\[
Z := \Closure_Z\bigl( X(K)\setminus\G \bigr). 
\]
We note that
\begin{align*}
\dimpadic & Z(\QQ_p)  \\
&\ge \dimpadic \Closure_p\bigl( X(K)\setminus\G \bigr)
\quad\begin{tabular}[t]{@{}l}
since $Z$ is a Zariski closed\\ set that contains $X(K)\setminus\G$,\\
\end{tabular}
\\
&\ge \dimKrull X
\quad\text{from \eqref{eqn:dimCpXKnegGgedimKX}.}
\end{align*}
On the other hand, if~$Z\subsetneq{X}$ is a proper subset of~$X$, then Lemma~\ref{lemma:padicdimen}(c) says that
\[
\dimpadic Z(\QQ_p) < \dimKrull X.
\]
Hence~$Z=X$, which is the desired conclusion.
\end{proof}


%%%%%%%%%%%%%%%%%%%%%%%%%%%%%%%%%%%%%%
\section{Surfaces}
\label{section:surfaces}
%%%%%%%%%%%%%%%%%%%%%%%%%%%%%%%%%%%%%%

Our goal in this section is to prove an orbit propagation result for all smooth projective surfaces. We note that we have already proven a number of cases that apply in particular to surfaces. These results are summarized in Table~\ref{table:surfaceorbitpropagation}.

\begin{table}[ht]
\begin{center}
\begin{tabular}{|lc@{\quad}l|} \hline
$\PP^2$, $\deg(f)\ge2$ 
& $\text{(A)}\Longrightarrow\text{(C1) and (B1$\exists$)}$
& Theorem~\ref{theorem:projectivespacedeg2}(a,b) \\ \hline
%%%
\begin{tabular}{@{}l} $\PP^2$, $\deg(f)=1$\\  \end{tabular}
& $\text{(A)}\Longrightarrow\text{(C3$\exists$)}$ 
& Theorem~\ref{theorem:projectivespacedeg2}(c)\\ \hline
%%%
\begin{tabular}{@{}l} $\PP^2$, $\deg(f)=1$\\ $f$ semisimple\\ \end{tabular}
& $\text{(A)}\Longrightarrow\text{(C3$\forall$)}$ 
& Theorem~\ref{theorem:projectivespacedeg2}(d)\\ \hline
%%%
\begin{tabular}{@{}l} Geometrically simple\\ abelian surfaces\\ \end{tabular}
& $\text{(A)}\Longrightarrow\text{(C3$\forall$)}$
& Theorem~\ref{theorem:simpleabelianvariety} \\ \hline
%%%
K3 surfaces 
& $\text{(A)}\Longrightarrow\text{(C1)}$ 
& Theorem~\ref{theorem:K3} \\ \hline
%%%
\begin{tabular}{@{}l} Rational surface\\ $f$ \'etale\\ \end{tabular}
& (B1)
& Corollary~\ref{corollary:CoroRat} \\ \hline
%%%
\begin{tabular}{@{}l} \'etale quotient of\\ an abelian surface\\ \end{tabular}
& (B1)
& Corollary~\ref{corollary:B1foretaleabelianquotients} \\ \hline
\end{tabular}
\end{center}
\caption{Orbit propagation results that apply to surfaces} 
\label{table:surfaceorbitpropagation}
\end{table}

\begin{theorem}
\label{theorem:surfaces}
Let~$K$ be a number field and let~$X/K$ be a smooth projective surface. Then possibly after replacing~$K$ by a finite extension, for every finite collection of $f$-orbits $\G_1,\ldots,\G_r\subset{X}(K)$, we have
\[
\overline{X(K) \setminus (\G_1\cup\cdots\cup\G_r)} = X.
\]
In the terminology of Table~\textup{\ref{table:propogationstatements}}, smooth projective surfaces satisfy the orbit propagation statement
\[
\textup{(A)} \quad\Longrightarrow\quad \textup{(B1)}.
\]
\end{theorem}

Before starting the proof of Theorem~\ref{theorem:surfaces}, we
give some lemmas that will be used in the proof.

\begin{lemma} 
\label{lemma:Lemmakmi} 
Let $Y$ be a minimal surface with $\kappa(Y)=-\infty$. Let $X$ be obtained from $Y$ by a sequence of blow-ups. If $f:X\to X$ is a non-invertible surjective morphism, then some iterate of $f$ is induced from an endomorphism of $Y$. In particular, if $\kappa(X)=-\infty$, then Theorem~$\ref{theorem:surfaces}$ for all minimal surfaces $Y$ and and all non-invertible maps implies Theorem~$\ref{theorem:surfaces}$ for $X$ with a non-invertible map $f$.
\end{lemma}
\begin{proof}
See \cite[Lemma~4.1]{MR3871505}.
\end{proof}

\begin{lemma} 
\label{lemma:negonecurves}
Let $X$ be a smooth projective complex surface that is not birational to $\PP^2$. Then $X$ contains at most finitely many $(-1)$-curves.
\end{lemma}
\begin{proof}
This is well-known. See for example~\cite{MathOverflow267708}.
\end{proof}

We next consider a version of Lemma~\ref{lemma:Lemmakmi} for non-rational surfaces.

\begin{lemma}
\label{lemma:LemmaNonMin} 
Let $Y$ be a minimal surface that is not birational to $\PP^2$. Let $X$ be obtained from $Y$ by a sequence of blow-ups. If $f:X\to X$ is a  surjective morphism, then some iterate of $f$ is induced from an endomorphism of $Y$. 
In particular, Theorem~$\ref{theorem:surfaces}$ for $Y$ implies Theorem~$\ref{theorem:surfaces}$ for $X$.
\end{lemma}
\begin{proof} 
If~$f$ is an automorphism, then it maps exceptional curves into exceptional curves and we are done. So we assume that~$f$ is non-invertible and consider two cases. First, if $\kappa(X)\ge 0$, then~\cite[Lemma~4.2]{MR3871505} tells us that $X=Y$. Second, if $\kappa(X)=-\infty$, then Lemma~\ref{lemma:Lemmakmi} gives the desired result.
\end{proof}

\begin{proof}[Proof of Theorem~$\ref{theorem:surfaces}$]
We note from Lemma~\ref{lemma:LemmaNonMin} that Theorem~\ref{theorem:surfaces} for minimal surfaces implies Theorem~\ref{theorem:surfaces} for all surfaces, except possibly in the case that~$X$ is a rational surface and $f:X\to X$ is an automorphism. But that case is covered by Corollary~\ref{corollary:CoroRat}. So we have reduced the proof of Theorem~\ref{theorem:surfaces} to the case of minimal surfaces. The proof of Theorem~\ref{theorem:surfaces} is now a case-by-case analysis via the classical classification of surfaces~\cite[Section~V.6]{MR0463157}.
\par
We remark that all of the implications in Table~\ref{table:surfaceorbitpropagation} are at least as strong as the assertion of Theorem~\ref{theorem:surfaces}, since each of the properties~(C1),~(C3$\exists$), and~(C3$\forall$) imply~(B1); cf.\ Table~\ref{table:propagationimplications}. Hence Theorem~\ref{theorem:surfaces} is true for the surfaces listed in Table~\ref{table:surfaceorbitpropagation}.
\par
\Case{1}{$\kappa(X)=-1$, $X$ is rational}
Since we are assuming that~$X$ is minimal, this implies that~$X\cong\PP^2$, so Theorem~\ref{theorem:projectivespacedeg2}(a,b,c) give something stronger than the desired result.
%%%%%%%%
\Case{2}{$\kappa(X)=-1$, $X$ is ruled}
Let~$\pi:X\to{B}$ be a surjective map from~$X$ to a curve~$B$ whose fibers are~$\Kbar$-isomorphic to~$\PP^1$. If~$g(B)\ge2$, then~$B(K)$ is finite, so there are no dense orbits. We may thus assume that~$g(B)$ is~$0$ or~$1$.
\par
Possibly after replacing $f$ by $f^2$ (cf.\ \cite[Lemma~5.4]{MR3871505}), we may assume that $f$ is semi-conjugate to $\pi$, i.e, there is a map $h:B\to B$ 
and a commutative diagram
\begin{equation}
\label{eqn:ruledsurface}
\begin{CD}
X @>f>> X \\
@V \pi VV @V \pi VV \\
B @>h>> B \\
\end{CD}
\end{equation}
Furthermore, $\pi$ has a section $\sigma:B\to X$; see \cite[Lemma~5.1]{MR3871505}. Replacing $K$ by a finite extension,  we may assume that everything is defined over $K$. Then for every finite extension~$L/K$, we have
\[
\pi\bigl(X(L)\bigr) \supseteq \pi\bigl(\s\bigl(B(L)\bigr)\bigr) = B(L).
\]
Thus$\pi\bigl(X(L)\bigr)=B(L)$, where~$B$ is either~$\PP^1$ or an elliptic curve. The implication (A)$\Rightarrow$(B1) is true for the curve~$B$ by the dimension~$1$ cases of Theorem~\ref{theorem:projectivespacedeg2} and~\ref{theorem:simpleabelianvariety}, where for the latter we note that an elliptic curve is always geometrically simple.
\par
We want to prove that~(A) implies~(B1) for~$X$, so we assume that there is a point~$x_0\in{X(K)}$ whose orbit~$\Orbit_f(x_0)$ is Zariski dense in~$X$. It follows from the commutative diagram~\eqref{eqn:ruledsurface} that~$\Orbit_h\bigl(\pi(x_0)\bigr)$ is Zariski dense in~$B$, so as noted in the previous paragraph, we know that~(B1) is true for~$B$.
\par
Let~$\G_1,\ldots,\G_r$ be $f$-orbits in~$X(K)$. Then~$\pi(\G_1),\ldots,\pi(\G_r)$ are~$h$ orbits in~$B$, so the validity of~(B1) for~$B$ tells us that
\[
B^\circ(K) := B(K) \setminus \bigl( \pi(\G_1),\ldots,\pi(\G_r) \bigr)
\quad\text{is Zariski dense in $B$.}
\]
The fibers of~$\pi$ over the points in~$B^\circ(K)$ are $K$-isomorphism to~$\PP^1$, since the fiber over~$b\in{B^\circ(K)}$ is a rational curve $X_b:=\pi^{-1}(b)$ that contains the $K$-rational point~$\s(b)$. Hence~$X_b(K)$ is Zariski dense in~$X$, so the density of~$B^\circ(K)$ in~$B$ implies that
\[
\bigcup_{b\in B^\circ(K)} X_b(K)\quad\text{is Zariski dense in $X$.}
\]
This set is disjoint from~$\G_1\cup\cdots\cup\G_r$ be construction, which completes the proof that~$X(K)\setminus(\G_1\cup\cdots\cup\G_r)$ is Zariski dense in~$X$, and thus the proof that~(B1) holds for~$X$.
%%%%%%%%
\Case{3}{$\kappa(X)=0$, $X$ is a K3 surface}
Theorem~\ref{theorem:K3} proves  $\text{(A)}\Longrightarrow\text{(C1)}$ for~K3 surfaces, which is stronger than the desired result.
%%%%%%%%
\Case{4}{$\kappa(X)=0$, $X$ is an Enriques surface}
There is an \'etale quotient map $\pi:X'\to X$ with $X'$ a $K3$ surface, and the endomorphism $f:X\to X$ lifts to $X'$ because~K3 surfaces are simply connected. 
Hence the desired result for~$X$ follows from the fact that it is true the~K3 surface~$X'$.
%%%%%%%%
\Case{5}{$\kappa(X)=0$, $X$ is an abelian surface}
This is covered by Corollary~\ref{corollary:B1foretaleabelianquotients}.
%%%%%%%%
\Case{6}{$\kappa(X)=0$, $X$ is a bielliptic surface}
This is covered by Corollary~\ref{corollary:B1foretaleabelianquotients}.
%%%%%%%%
% \Case{7}{$\kappa(X)=1$, $X$ is an elliptic surface}
% \Case{8}{$\kappa(X)=2$, $X$ is of general type}
\Case{7}{\begin{tabular}[t]{@{}l@{}}
$\kappa(X)=1$, $X$ is an elliptic surface\\
$\kappa(X)=2$, $X$ is of general type.\\
\end{tabular}}
In general, if~$\kappa(X)>0$, then we can use Nakayama and Zhang~\cite[Theorem~A]{MR2551469}. They prove that any dominant rational self-map of a smooth projective variety of positive Kodaira dimension factors over a positive dimensional base, and hence does not have any Zariski dense orbits. See also~\cite[Corollary~8.2]{MR3871505} for a more elementary proof of the case that we need. Thus all versions of orbit propagation are vacuously true, since there are no Zariski dense orbits. (We mention that varieties of general type cannot even have endomorphisms of infinite order, so orbits are necessarily finite in that case, making orbit propagation even more vacuous.)
\end{proof}

We conclude this section with a maximally strong orbit propagation result for automorphisms of~$\PP^2$. The somewhat ad hoc nature of the proof illustrates the difficulties inherent in handling non-semi-simple endomorphisms of~$\PP^N$ for~$N\ge3$.

\begin{theorem}
\label{theorem:autsofP2}
Let $f:\PP^2\to\PP^2$ be an automorphism, i.e.,~$f$ is a map of degree~$1$. Then~$(\PP^2,f)$ has the property that~\textup{(A)} implies~\textup{(C3$\forall$)}.
\end{theorem}
\begin{proof}
If~$f$ is a semi-simple automorphism, then we proved the desired result for~$\PP^N$ in Theorem~\ref{theorem:projectivespacedeg2}(d). So we are reduced to the case that~$f$ is not semi-simple
\par
Taking a field~$K$ containing the eigenvalues of~$f$, using the fact that the matrices representing~$f$ are only defined up to scalar multiplication, and permuting the coordinates if necessary, we are reduced to considering the case that~$f$ is given by a matrix of one of the following two forms:\footnote{More precisely, we start with an arbitrary matrix~$A$ representing~$f$. If the Jordan form of~$A$ consists of a~$2$-dimensional block with eigenvalue~$\l$ and a~$1$-dimensional block, we replace~$A$ with~$\l^{-1}A$, and then the Jordan form looks like~$A_\l$; and if the Jordan form of~$A$ consists of a single~$3$-dimensional  block with eigenvalue~$\l$, we replace~$A$ with~$\l^{-1}A$, and then the Jordan form looks like~$B$.}
\[
A_\l = \begin{pmatrix}
1 & 1 & 0 \\ 0 & 1 & 0 \\ 0 & 0 & \l \\
\end{pmatrix}\quad\text{with $\l\ne0$},
\qquad
B = \begin{pmatrix}
1 & 1 & 0 \\ 0 & 1 & 1 \\ 0 & 0 & 1 \\
\end{pmatrix}.
\]
Then the iterates of~$f$ are given by the formulas
\[
A_\l^n = \begin{pmatrix}
1 & n & 0 \\ 0 & 1 & 0 \\ 0 & 0 & \l^n \\
\end{pmatrix},
\qquad
B^n = \begin{pmatrix}
1 & n & \frac12(n^2-n) \\ 0 & 1 & n \\ 0 & 0 & 1 \\
\end{pmatrix}.
\]
\par
The $B$-orbit of an arbitrary point $[a,b,c]\in\PP^2$ is 
\[
\Orbit_{B}\bigl([a,b,c]\bigr)
= \Bigl\{ \left[a+(b-\tfrac12c)n+\tfrac12cn^2,\, b+cn,\, c\right] : n\in\NN \Bigr\}.
\]
If~$c=0$, then the orbit $\Orbit_{B}\bigl([a,b,c]\bigr\}$ is contained in the line~$\{z=0\}$. And if~$c\ne0$, then eliminating~$n$ from the coordinates, we find that
\[
\Orbit_{B}\bigl([a,b,c]\bigr\}
\subset 
\Bigl\{ [x,y,z] : 2 c^2 x z - c^2 y^2 + c^2 y z - (2 a c + b c + b^2) z^2 = 0 \Bigr\}.
\]
Hence the map~$f$ induced by the matrix~$B$ has no Zariski dense orbits, contrary to assumption.
\par
The situation for~$A_\l$ is a bit subtler. If~$\l$ is a root of unity, say $\l^r=1$, then
\begin{align*}
\Orbit_{A_\l}\bigl([a,b,c]\bigr) 
&= \Bigl\{ [a+bn, b, c\l^n] : n\in\NN \Bigr\} \\
&\subseteq \Bigl\{  [a+bn, b, c\z] : \z\in\bfmu_r \Bigr\} \\
&\subseteq \biggl\{ \prod_{\z\in\bfmu_r} (c y \z - b z) = 0 \biggr\}.
\end{align*}
This shows that $\Orbit_{A_\l}\bigl([a,b,c]\bigr)$ is not Zariski dense if either~$b$ or~$c$ is non-zero, while $\Orbit_{A_\l}\bigl([1,0,0]\bigr)$ consists of the single point~$[1,0,0]$. Hence the assumption that~$f$ has at least one Zariski dense orbit tells us that~$\l$ is not a root of unity.
\par
Let~$[a,b,c]\in\PP^2$ be a point with $abc\ne0$, and let~$F(x,y,z)$ be a homomgeneous polynomial~$F$ that vanishes at every point in $\Orbit_{A_\l}\bigl([a,b,c]\bigr)$. Our goal is to prove that~$F=0$, which will show that the orbit is Zariski dense. So we start with the assumption that
\[
F(a+bn, b, c\l^n) = 0 \quad\text{for all $n\in\NN$.}
\]
We let~$d=\deg(F)$, and we write
\begin{equation}
\label{eqn:Gxyzsum}
F(x,y,z) = \sum_{i+j+k=d} a_{ij} x^iy^jz^{k}
= \sum_{k=0}^d \sum_{i+j=k} a_{ij} x^iy^jz^{d-k}.
\end{equation}
Then for all $n\in\NN$, we have
\begin{equation}
\label{eqn:0Glnnln11}
0 = F(a+bn,b,c\l^n)
= \sum_{k=0}^d \biggl( \sum_{i+j=k} a_{ij} (a+bn)^i b^j  \biggr) c^{d-k} \l^{n(d-k)}.
\end{equation}
Since~$\l$ is not a root of unity, Kronecker's theorem~\cite[Theorem~3.8]{MR2316407} says that we can find an absolute value~$\|\,\cdot\,\|$ on~$K$ satisfying \text{$\|\l\|>1$}. We then observe that:
\[
\parbox{.8\hsize}{For distinct $0\le k<d$, the absolute values $\|\l^{n(d-k)}\|$ grow at different exponential rates as $n\to\infty$.}
\]
It follows from~\eqref{eqn:0Glnnln11} and the assumption that~$c\ne0$ that all of the inner sums vanish, i.e., 
\[
\sum_{i+j=k} a_{ij} (a+bn)^i b^j = 0 \quad\text{for all $0\le k\le d$ and all $n\ge0$.}
\]
Thus for all $0\le{k}\le{d}$, the polynomial
\[
G_k(t) := \sum_{i=0}^k a_{i,k-i} b^{k-i} t^i
\quad\text{vanishes at all $t\in a+b\NN$.}
\]
Since~$b\ne0$, this forces the polynomial~$G_k(t)$ to be identically~$0$. We have now proven that
\[
\text{$a_{i,k-i}=0 $ for all $0\le k\le d$ and all $0\le i\le k$.}
\]
Hence~$F(x,y,z)=0$, which completes the proof that $\Orbit_{A_\l}\bigl([a,b,c]\bigr)$ is Zariski dense in~$\PP^N$. 
\par
To recapitulate, we've proven that the following holds for all non-semisimple degree~$1$ maps of~$\PP^2$, after an appropriate change of coordinates:
\[
\overline{\Orbit_f\bigl([a,b,c]\bigr)} = \PP^2
\quad\Longleftrightarrow\quad
abc\ne0.
\]
Since the same is true for semisimple linear maps of~$\PP^N$ in all dimensions, cf.\ \eqref{eqn:OfPZDiffPinTTx} in the proof of Theorem~\ref{theorem:projectivespacedeg2}(d), it follows that for all degree~$1$ maps of~$\PP^2$, we have
\[
\overline{(\PP^2)_f^\dense(K)} = \PP^2.
\]
Since we are assuming~(A), and since Theorem~\ref{theorem:projectivespacedeg2}(c) gives propagation property~(C2$\forall$) for all linear maps of~$\PP^N$, we may appeal to Lemma~\ref{lemma:C2forallZDimpliesC3forall} to conclude that~$f$ has propagation property~(C3$\forall$).
\end{proof}



%%%%%%%%%%%%%%%%%%%%%%%%%%%%%%%%%%%%%%%%%%%%%%%%%%%%%%%%%%%%%%%%%%%%%%
% \section{Speculations and Conjectures}
% \label{section:spectulations}
%%%%%%%%%%%%%%%%%%%%%%%%%%%%%%%%%%%%%%%%%%%%%%%%%%%%%%%%%%%%%%%%%%%%%%

%%%%%%%%%%%%%%%%%%%%%%%%%%%%%%%%%%%%%%%%%%%%%%%%
\appendix
%%%%%%%%%%%%%%%%%%%%%%%%%%%%%%%%%%%%%%%%%%%%%%%%

%%%%%%%%%%%%%%%%%%%%%%%%%%%%%%%%%%%%%%%%%%%%%%%%
\section{Conditional Results}
\label{appendix:conditionalresults}
%%%%%%%%%%%%%%%%%%%%%%%%%%%%%%%%%%%%%%%%%%%%%%%%

We prove some propagation results that are conditional on various other dynamical conjectures.

\begin{conjecture}[Cancellation Conjecture]
\label{conjecture:cancellation}
\textup{(Bell--Matsuzawa--Satriano \cite[Conjecture 1.5]{MR4574395})}
Let $f:X\to X$ be a surjective self-morphism of a projective variety~$X$ defined over a
number field~$K$. Then there exists an integer~$n_0=n_0(K,X,f)$ such that for all~$P,Q\in{X(K)}$,
\[
\text{$f^n(P)=f^n(Q)$ for some $n\ge0$}
\quad\Longrightarrow\quad
f^{n_0}(x)=f^{n_0}(y)
\]
\end{conjecture}

\begin{remark}
\label{remark:preimageconj}
The authors of~\cite{MR4543765} pose a preimages question; see also~\cite[Question~1.1]{MR4574395}. Their question deals with an  endomorphism~$f:X\to{X}$ and a subvariety~$Y\subseteq{X}$ satisfying~$f(Y)=Y$.
If we drop the requirement that~$Y$ is~$f$-invariant, then their question may have a negative answer, for example if~$f$ is an automorphism and~$Y$ is a non-preperiodic rational variety. However, if we impose further conditions, for example if we assume that~$f$ is polarized and is not an automorphism (we thank Yohsuke Matsuzawa for this suggestion), then possibly the question will still have an affirmative answer. We formulate a conjecture of this nature for projective space.
\end{remark}

\begin{conjecture}[Strong Preimages Conjecture for $\PP^N$]
\label{conjecture:strongpreimage}
Let~$f:\PP^N\to\PP^N$ be an endomorphism of degree~$d\ge2$, and let~$Y\subseteq\PP^N$ be a subvariety, with~$f$ and~$Y$ defined over a number field~$K$. Then there exists an $n_1=n_1(K,N,f,Y)$ with the property that for all~$n\ge{n_1}$,
\begin{multline*}
\text{$P\in \PP^N(K)$ and $f^n(P)\in Y(K)$ for some $n$} \\*
\Longrightarrow\quad
\text{$f^{n}(x)\in Y(K)$ for some $n\le n_1$.}
\end{multline*}
\end{conjecture}

We use Conjectures~\ref{conjecture:cancellation} and~\ref{conjecture:strongpreimage} to prove strong orbit propagation results for projective space.

\begin{theorem} 
\label{theorem:projectivespacedeg2conditional}
Let $N\ge1$, let $f:\PP^N\to\PP^N$ be an endomorphism defined over~$\Qbar$, and assume that there is a point~$P_0\in\PP^N(\Qbar)$ whose orbit~$\Orbit_f(P_0)$ is Zariski dense in~$\PP^N$. 
\begin{parts}
\Part{(a)}
Assume that Conjecture~\textup{\ref{conjecture:cancellation}} is true for~$\PP^N$ and~$f$.  Then there exists a number field~$K$ that is a common field of definition for~$f$ and~$P_0$ and a complete set of representatives~$\Qcal\subset(\PP^N)_f^\grand(K)$ for $(\PP^N)_f^\grand(K)/\fgeq$ such that~$\Qcal$ is Zariski dense in~$\PP^N$. In the terminology of Table~\textup{\ref{table:propogationstatements}}, non-linear endomorphisms of~$\PP^N$ satisfy the orbit propagation statement
\[
\textup{(A)} \quad\Longrightarrow\quad \textup{(C2$\exists$)}.
\]
\Part{(b)}
Assume that Conjecture~\textup{\ref{conjecture:cancellation}} is true for~$\PP^N$ and~$f$ and that Conjecture~\textup{\ref{conjecture:strongpreimage}} is true for~$f$. Then there exists a number field~$K$ that is a common field of definition for~$f$ and~$P_0$ so that every complete set of representatives in $(\PP^N)_f^\grand(K)$ for $(\PP^N)_f^\grand(K)/\fgeq$ is Zariski dense in~$\PP^N$. In the terminology of Table~\textup{\ref{table:propogationstatements}}, non-linear endomorphisms of~$\PP^N$ satisfy the orbit propagation statement
\[
\textup{(A)} \quad\Longrightarrow\quad \textup{(C2$\forall$)}.
\]
\end{parts}
\end{theorem}
\begin{proof}
The proof uses a counting argument similar to the proof of \text{(A)${}\Rightarrow{}$(B1$\infty$)} in Theorem~\ref{theorem:projectivespacedeg2}(b), but we use the cancellation and preimage conjectures to replace orbits with grand orbits. We start with a complete set of representatives~$\Qcal\subset\PP^N(K)$ for $(\PP^N)_f(K)/\fgeq$, which by definition gives a (disjoint) union decomposition
\begin{equation}
\label{eqn:PNKcupQOfgrQK}
\PP^N(K) = \bigcup_{Q} \Orbit_f^\grand(Q)(K)
\end{equation}
We let
\[
Y=\overline{\Qcal}
\]
be the Zariski closure of~$\Qcal$ in~$\PP^N$, and our goal is to show that~$Y=\PP^N$.
\par
Later, when we prove~(a), we will impose a further condition on~$\Qcal$, but first we prove some general consequences of Conjecture~\ref{conjecture:cancellation} that will be used for both parts of the proof.
\par
We recall from~\eqref{eqn:countingfunction} that for a subset~$\Pcal\subseteq\PP^N(K)$, we defined a counting function
\begin{equation}
\label{eqn:countingfunctionx}
\Count(\Pcal,T) := \#\bigl\{P\in\Pcal : h(P) \le \log(T)\bigr\},
\end{equation}
where~$h=\log(H)$ is the logarithmic Weil height relative to~$K$. It will be convenient to let
\[
\CountSet(\Pcal,T):= \bigl\{ P\in \Pcal : h(P)\le\log(T) \bigr\}
\]
be the associated set, so~$\Count(\Pcal,T)=\#\CountSet(\Pcal,T)$.
\par
In general, the grand orbit of a point~$Q\in\PP^N$ is the union of the backward orbits of the forward $f$-iterates of~$Q$,
\begin{equation}
\label{eqn:grandorbbackorbs}
\Orbit_f^\grand(Q) = \bigcup_{n\ge0} \Orbit_f^-\bigl(f^n(Q)\bigr),
\end{equation}
so we start with a subsidiary calculations on backward orbits.
\par
Let~$Q\in\PP^N(K)$ be an $f$-wandering point.  Then
\begin{align}
P & \in \Orbit_f^-\bigl(f^n(Q)\bigr) \setminus \Orbit_f^-\bigl(f^{n-1}(Q)\bigr) \notag\\
&\quad\Longrightarrow\quad
\left[\text{\parbox{.7\hsize}{there is an integer $m(P)\ge0$ satisfying\\
\hspace*{0em}
$f^{m(P)}(P)=f^n(Q)$ and $f^{m(P)-1}(P)\ne f^{n-1}(Q)$.
}} \right]
\label{eqn:fmPPeqfnQneqneg1x}
\end{align}
\par
We suppose further that~$P\in\PP^N(K)$, we let~$n_0=n_0(K,N,f)$ be the integer that appears in Conjecture~\ref{conjecture:cancellation} for~$\PP^N$, and we assume that~$n>n_0$. We are going to prove that~$m(P)\le{n_0}$ by assuming the contrary and deriving a contradiction. This assumption allows us to rewrite the equality in~\eqref{eqn:fmPPeqfnQneqneg1x} as
\begin{equation}
\label{eqn:fn01circfmPn01x}
f^{n_0+1}\bigl( f^{m(P)-n_0-1}(P)\bigr)=f^{n_0+1}\bigl( f^{n-n_0-1}(Q)\bigr)
\end{equation}
Conjecture~\ref{conjecture:cancellation} tells us that if~$R,R'\in\PP^N(K)$ are points that satisfy $f^{n_0+1}(R)=f^{n_0+1}(R')$, then they satisfy~$f^{n_0}(R)=f^{n_0}(R')$. Applying this to~\eqref{eqn:fn01circfmPn01x}  yields
\[
f^{n_0}\circ f^{m(P)-n_0-1}(P)=f^{n_0}\circ f^{n-n_0-1}(Q).
\]
Hence~$f^{m(P)-1}(P)=f^{n-1}(Q)$, which contradicts the inequality in~\eqref{eqn:fmPPeqfnQneqneg1x}. This contradiction completes the proof that
\begin{equation}
\label{eqn:Mplen0}
m(P) \le n_0.
\end{equation}
Written out more fully, we have shown that
\begin{multline*}
P   \in \Bigl(\Orbit_f^-\bigl(f^n (Q)\bigr)\setminus \Orbit_f^-\bigl(f^{n-1}(Q)\bigr)\Bigr)(K)
\quad\text{for some $n>n_0$}\\
\quad\Longrightarrow\quad
f^{m(P)}(P)=f^n(Q)\quad\text{for some $0\le m(P)\le n_0(K,N,f)$.}
\end{multline*}
The crucial information is that~$m(P)$ is bounded by a quantity~$n_0(K,N,f)$ that is independent of~$n$ and~$Q$.
\par
Continuing with the assumption that~$n>n_0$, we conclude that
\begin{align}
\label{eqn:Pinbackorbitsetminus}
P \in  & \Bigl(\Orbit_f^-\bigl(f^n  (Q)\bigr)
\setminus \Orbit_f^-\bigl(f^{n-1}(Q)\bigr)\Bigr)(K) \notag \\
&\quad\Longrightarrow\quad
\hhat_f\bigl( f^{m(P)}\bigr)=\hhat_f\bigl(f^n(Q)\bigr)
\quad\text{for some $0\le m(P)\le n_0$.} \notag \\
&\quad\Longrightarrow\quad
d^{m(P)} \hhat_f(P) = d^n \hhat_f(Q)
\quad\text{from \eqref{eqn:canht1},}\notag \\
&\quad\Longrightarrow\quad
n = m(P) + \log_d\bigl(\hhat_f(P)/\hhat_f(Q)\bigr) \notag \\
&\quad\Longrightarrow\quad
n \le n_0(K,N,f) + \log_d\left(\frac{h(P)+\Cr{hfh}(f)}{\hhat_f^{\min}(\PP^N,K)} \right) \notag \\
&\omit\hfill\quad\text{from \eqref{eqn:canht2} and \eqref{eqn:hhatfminPNK},} \notag \\
&\quad\Longrightarrow\quad
n \le  \log\bigl(h(P)\bigr) + \Cl{cc2}(K,N,f).
\end{align}
We stress that~$\Cr{cc2}$ depends on~$K$,~$N$, and~$f$, but not on~$P$ or~$Q$. We also note that~\eqref{eqn:Pinbackorbitsetminus} remains true for~$n\le{n_0}$, since we are allowed to increase the constant so that it satisfies~$\Cr{cc2}(K,N,f)\ge{n_0}(K,N,f)$.
\par
We set the notation
\begin{equation}
\label{eqn:LTLKPNf}
L = L(K,N,f;T) := \bigl\lfloor \log_d(T)+\Cr{cc2}(K,N,f) \bigr\rfloor,
\end{equation}
where it is important to keep in mind that~$L$ is a function of~$T$. Then~\eqref{eqn:Pinbackorbitsetminus} tells us that
\begin{equation}
\label{eqn:ngtLTimpliesCount0}
n > L \;\Longrightarrow\;
\CountSet\Bigl(
\Orbit_f^-\bigl(f^n(Q)\bigr)(K)  \setminus \Orbit_f^-\bigl(f^{n-1}(Q)\bigr)(K),\, T
\Bigr) = \emptyset.
\end{equation}
\par
We next write the points of bounded height in~$\PP^N(K)$ as a union of subsets. Thus
\begin{align}
\CountSet & \bigl(\PP^N(K),T\bigr) \notag\\
&=
\smash[b]{
\bigcup_{Q\in\Qcal} \CountSet\Bigl( \Orbit_f^\grand(Q)(K),\,T \Bigr) 
} \notag\\
&\omit\hfill\text{\parbox{.6\hsize}{from \eqref{eqn:PNKcupQOfgrQK}, which says that every point in~$\PP^N(K)$ is in the grand orbit of a point in~$\Qcal$,}} \notag\\
% &\subseteq
% \bigcup_{Q\in Y(K)} \CountSet\Bigl( \Orbit_f^\grand(Q)(K),\,T \Bigr) 
% \quad\text{since $\Qcal\subseteq\overline\Qcal\cap\PP^N(K)=Y(K)$,} \notag\\
&=
\smash[b]{ 
\bigcup_{Q\in \Qcal} 
\bigcup_{n=0}^\infty
}
\CountSet\biggl(
\Bigl(\Orbit_f^-\bigl(f^n(Q)\bigr)  \setminus \Orbit_f^-\bigl(f^{n-1}(Q)\bigr)\Bigr)(K),\; T
\biggr) 
\hspace*{2em}
\notag\\*
&\omit\hfill\text{\parbox{.6\hsize}{
from \eqref{eqn:grandorbbackorbs}, where for the $n=0$ term we set 
$\Orbit_f^-\bigl(f^{-1}(Q)\bigr)=\emptyset$ by convention,}} \notag\\
&=
\smash[b]{
\bigcup_{Q\in \Qcal} 
\bigcup_{n=0}^{L}
}
\CountSet\biggl(
\Bigl(\Orbit_f^-\bigl(f^n(Q)\bigr)  \setminus \Orbit_f^-\bigl(f^{n-1}(Q)\bigr)\Bigr)(K),\; T
\biggr) 
\notag\\*
&\omit\hfill\text{\parbox{.6\hsize}{
from \eqref{eqn:ngtLTimpliesCount0}, which says that the
set is empty if $n>L$,}} \notag\\
&= 
\bigcup_{Q\in \Qcal} 
\CountSet\Bigl(
\Orbit_f^-\bigl(f^{L}(Q)\bigr)(K),\; T
\Bigr) 
\notag\\*
&=
\smash[b]{
\bigcup_{Q\in \Qcal}  \bigcup_{m=0}^\infty
\CountSet\Bigl( f^{-m}\bigl(f^L(Q)\bigr)(K),\, T\Bigr) 
}\notag\\
&\omit\hfill\text{by definition of $\Orbit_f^-\bigl(f^L(Q)\bigr)$,}\notag \\
% &= \bigcup_{Q\in \Qcal}\bigcup_{m=0}^L
% \CountSet\Bigl( f^{-m}\bigl(f^L(Q)\bigr)(K),\, T\Bigr) 
% \notag\\
% &\hspace*{4em} \cup 
% \bigcup_{Q\in \Qcal} \bigcup_{m=L+1}^\infty
% \CountSet\Bigl( f^{-m}\bigl(f^L(Q)\bigr)(K),\, T\Bigr) 
% \notag\\
% &= \bigcup_{\substack{Q\in \Qcal\\ h(Q)\le\log(T)+2\Cr{hfh}}}
&= \bigcup_{Q\in\Qcal}
\bigcup_{m=0}^L
\CountSet\Bigl( f^{-m}\bigl(f^L(Q)\bigr)(K),\, T\Bigr) 
\label{eqn:QYmlen} \\
&\hspace*{4em} \cup 
\bigcup_{Q\in \Qcal} \bigcup_{m=L+1}^\infty
\CountSet\Bigl( f^{-m}\bigl(f^L(Q)\bigr)(K),\, T\Bigr) 
\label{eqn:QYmgen} \\
&\omit\hfill\text{from \eqref{eqn:mleLCSfnegmfLQKTnees}.}
\notag
\end{align}
\par
We next show that if~$h(Q)$ is sufficiently large, then the set~\eqref{eqn:QYmlen} is empty. Thus a given~$Q\in\Qcal$ and integers~$0\le{m}\le{L}$, we compute
\begin{align*}
P \in \CountSet\Bigl( f^{-m}\bigl(f^L(Q)\bigr) & (K),\, T\Bigr) \\
&\quad\Longrightarrow\quad
\text{$f^m(P)=f^L(Q)$ and $h(P)\le\log(T)$} \\
&\quad\Longrightarrow\quad
\text{$d^m\hhat_f(P)=d^L\hhat_f(Q)$ and $h(P)\le\log(T)$} \\
&\quad\Longrightarrow\quad
\begin{aligned}[t]
h(Q)
&\le \hhat_f(Q) + \Cr{hfh}\\
&= d^{m-L} \hhat_f(P) + \Cr{hfh}(f)(f)\\
&\le \hhat_f(P) + \Cr{hfh}(f) \quad\text{since $m\le L$,}\\
&\le h(P) + 2\Cr{hfh}(f) \\
&\le \log(T) + 2\Cr{hfh}(f). \\
\end{aligned}
\end{align*}
Hence
\begin{multline}
\label{eqn:mleLCSfnegmfLQKTnees}
\text{$m\le L$ and $\CountSet\Bigl( f^{-m}\bigl(f^L(Q)\bigr)(K),\, T\Bigr)\ne\emptyset$} \\
\Longrightarrow\quad h(Q) \le \log(T) + 2\Cr{hfh}(f).
\end{multline}
\par
We can further estimate the size of the sets appearing~\eqref{eqn:QYmlen} by computing
\begin{align*}
m\le L&~\text{and}~f^m(P)=f^L(Q) \\
&\quad\Longrightarrow\quad
f^m(P) = f^m\bigl(f^{L-m}(Q)\bigr) \\
&\quad\Longrightarrow\quad
f^{n_0}(P) = f^{n_0}\bigl(f^{L-m}(Q)\bigr) 
\quad\text{from Conjecture~\ref{conjecture:cancellation},} \\
&\quad\Longrightarrow\quad
P \in (f^{n_0})^{-1} \bigl( f^{n_0+L-m}(Q) \bigr).
\end{align*}
It follows that the sets appearing in~\eqref{eqn:QYmlen} have cardinality bounded by~$d^{n_0}$, where we note that the quantity~$n_0=n_0(K,N,f)$ from Conjecture~\ref{conjecture:cancellation} does not depend on~$Q$. Hence for all~$m\le{L}$ we have
\begin{align}
\label{eqn:ctfnegmflQKT}
\Count\Bigl( f^{-m}\bigl(f^L(Q)\bigr)(K),\, T\Bigr)
&\le \#\Bigl( (f^{n_0})^{-1} \bigl( f^{n_0+L-m}(Q) \bigr) \Bigr) \notag\\
&\le d^{n_0} \notag\\
&= \Cl{dn0cond}(K,N,f).
\end{align}
\par
We now have the tools to deal with the set~\eqref{eqn:QYmlen}, where we note that up to this point, we have used only Conjecture~\ref{conjecture:cancellation}. Thus
\begin{align}
\label{eqn:numpts50}
\bigl(\text{\# of }&\text{points in the set~\eqref{eqn:QYmlen}}\bigr) \notag\\
&\le
\sum_{\substack{Q\in \Qcal\\ h(Q)\le \log(T)+2\Cr{hfh}(f)}}
\sum_{m=0}^L
\Count\Bigl( f^{-m}\bigl(f^L(Q)\bigr)(K),\, T\Bigr)
\quad\text{from \eqref{eqn:mleLCSfnegmfLQKTnees},} \notag\\
&\le
\sum_{\substack{Q\in \Qcal\\ h(Q)\le \log(T)+2\Cr{hfh}(f)}}
\sum_{m=0}^L
\Cr{dn0cond}(K,N,f)
\quad\text{from \eqref{eqn:ctfnegmflQKT},} \notag\\
&\le
\Count\bigl( \Qcal,\, \Cl{hfh3}(f)\cdot T \bigr)
\cdot(L+1) \cdot \Cr{dn0cond}(K,N,f) \notag\\
&\le
\Count\bigl( Y(K),\, \Cr{hfh3}(f)\cdot T \bigr)
\cdot \log(T) \cdot \Cl{dn0cond2}(K,N,f) \notag\\
&\omit\hfill\text{by definition~\eqref{eqn:LTLKPNf} of $L=L(T)$
and $\Qcal\subseteq Y(K)$,} \notag\\
&\le \Cr{dn0cond2}(K,N,f,\Qcal) \cdot T^{1+\dim(Y)} \cdot \log(T) \notag\\
&\omit\hfill\quad\text{from Lemma~\ref{lemma:ctYKTleCTN}(b)}.\phantom{(000)}
\end{align}
\par
The remainder of the proofs of~(a) and~(b) now bifurcates in how it deals with the set~\eqref{eqn:QYmgen}.
%%%%%%%%%%%%%%%%%%%%%%%%%%%%%%%%%%%%%%%%%%%%%%%%%%%
\par\noindent(a)\enspace
For this part we are trying to prove the existence propagation property~(C2$\exists$), so we are free to choose a paticular set~$\Qcal$. As we shall see, a judicious choice of~$\Qcal$ allows us to eliminate the set~\eqref{eqn:QYmgen}.
\par
We start with an arbitrary complete set of representatives~$\Qcal\subset\PP^N(K)$ for $(\PP^N)_f^\grand(K)/\fgeq$, and we replace each point~$Q\in\Qcal$ with a $K$-rational point of minimal canonical $f$-height in the grand $f$-orbit of~$Q$. This is possible since~$\PP^N(K)$ has only finitely many points of bounded~$f$-canonical height. Our new~$\Qcal\subset\PP^N(K)$ is thus a complete set of representatives for $(\PP^N)_f^\grand(K)/\fgeq$ having the property that
\begin{equation}
\label{eqn:htsmallest}
Q\in\Qcal \quad\Longrightarrow\quad
\hhat_f(Q) = \min\Bigl\{ \hhat_f(P) : P\in\Orbit_f^\grand(Q)(K) \Bigr\}.
\end{equation}
\par
We next observe that for $Q\in\Qcal$ and~$P\in\PP^N(K)$, we have
\begin{align}
\label{eqn:fmPeqfnQimplyngem}
f^m & (P)=f^L(Q) \notag \\
&\quad\Longrightarrow\quad
P\in\Orbit_f^\grand(Q)(K)~\text{and}~
d^m\hhat_f(P) = d^L\hhat_f(Q) 
\hspace*{2em}\notag \\
&\omit\hfill\text{\parbox{.6\hsize}{
by definition of grand orbit and applying~\eqref{eqn:canht1},}} \notag\\
&\quad\Longrightarrow\quad
d^{L-m}\hhat_f(Q) = \hhat_f(P) \ge \hhat_f(Q) \notag \\
&\omit\hfill\text{\parbox{.6\hsize}{
since \eqref{eqn:htsmallest} says $\hhat_f(Q)$ is smallest in $Q$'s grand orbit,}} \notag\\
&\quad\Longrightarrow\quad
L \ge m.
\end{align}
It follows that the set~\eqref{eqn:QYmgen} is the empty set. Hence
\begin{align*}
\Cr{PNapp2}(K,N) & \cdot T^{1+\dim(\PP^N)} \\
&\le \Count\bigl(\PP^N(K),\,T\bigr)
\quad\text{from Lemma~\ref{lemma:ctYKTleCTN}(a)} \\
&\le
\left(\begin{tabular}{@{}l@{}}
\# of points in the set \eqref{eqn:QYmlen}, \\
since~\eqref{eqn:QYmgen} is the empty set \\
\end{tabular}
\right) \\
&\le \Cr{dn0cond2}(K,N,f,\Qcal) \cdot T^{1+\dim(Y)} \cdot \log(T)
\quad\text{from \eqref{eqn:numpts50}.}
\end{align*}
Letting~$T\to\infty$ forces
\[
\dim(\PP^N) \le \dim(Y),
\]
so the fact that~$Y$ is a Zariski closed subset of~$\PP^N$ implies that~$Y=\PP^N$.
%%%%%%%%%%%%%%%%%%%%%%%%%%%%%%%%%%%%%%%%%%%%%%%%%%%
\par\noindent(b)\enspace
In the proof of~(a), we used our choice of a set~$\Qcal$ with the special property~\eqref{eqn:htsmallest} to show that the set~\eqref{eqn:QYmgen} was empty. To prove~(b), where we make no assumptions on the set of representatives~$\Qcal$, we will use Conjecture~\ref{conjecture:strongpreimage} as a substitute for~\eqref{eqn:htsmallest}.
\par
We define two algebraic subsets of~$\PP^N$,
\begin{align*}
Z &= Z(K,N,f,Y) := f^{-n_0(K,N,f)}(Y), \\
W &= W(K,N,f,Y) := \bigcup_{n=0}^{n_1(K,N,f,Z)}  f^{-n}(Z).
\end{align*}
Here the constants~$n_0$ and~$n_1$ respectively come from Conjectures~\ref{conjecture:cancellation} and~\ref{conjecture:strongpreimage}. We then note that if~$P$ is an element of the set~\eqref{eqn:QYmgen} for a given~$Q\in\Qcal$ and~$m>L$, then
\begin{align*}
f^m(P)&=f^L(Q)\quad\text{from the definition of the set~\eqref{eqn:QYmgen},}\\
&\quad\Longrightarrow\quad
f^L\bigl(f^{m-L}(P)\bigr) = f^L(Q)
\quad\text{since $m>L$,} \\
&\quad\Longrightarrow\quad
f^{n_0}\bigl(f^{m-L}(P)\bigr) = f^{n_0}(Q) \\
&\omit\hfill\text{where $n_0=n_0(K,N,f)$ is as in Conjecture~\ref{conjecture:cancellation},} \\
&\quad\Longrightarrow\quad
f^{m-L}(P) \in f^{-n_0}\bigl(f^{n_0}(Q)\bigr)
\subseteq \bigl(f^{-n_0} (Y)\bigr)(K) = Z(K) \\
&\omit\hfill\text{\parbox{.6\hsize}{
since $Q\in\Qcal\subseteq{Y}$ and $P\in\PP^N(K)$, and by definition of $Z$,}} \\
&\quad\Longrightarrow\quad
f^n(P) \in Z(K) 
\quad\text{for some $n\le n_1(K,N,f,Z)$,} \\
&\omit\hfill\text{
where $n_1(K,N,f,Z)$ is as in Conjecture~\ref{conjecture:strongpreimage},} \\
&\quad\Longrightarrow\quad
P\in W(K)\quad\text{by definition of $W$.}
\end{align*}
We have thus shown that the set~\eqref{eqn:QYmgen} is a subset of~$W(K)$, and hence
\begin{align}
\label{eqn:num51cond}
\text{(\# of elements }&\text{in \eqref{eqn:QYmgen})} \notag\\
&\le
\Count\bigl(W(K),T\bigr) \notag\\
&\le 
\Cl{WKT1}(K,W)\cdot T^{1+\dim(W)} 
\quad\text{from Lemma~\ref{lemma:ctYKTleCTN}(b),} \notag\\ 
&= 
\Cl{WKT2}(K,N,f,\Qcal) \cdot T^{1+\dim(Y)} \\
&\omit\hfill\text{\parbox[t]{.6\hsize}{
by definition of $W$, since the finiteness of~$f$ tells us that every component of~$W$ has dimension at most~$\dim(Y)$, and where we note that~$W$ is determined by~$N$,~$f$, and~$Y$, and that~$Y$ is determined by~$\Qcal$.
}} \notag
\end{align}
\par
We can now argue similarly to the end of the proof of~(a). Thus
\begin{align*}
\Cr{PNapp2}(K,N)  \cdot T^{1+\dim(\PP^N)} 
&\le \Count\bigl(\PP^N(K),\,T\bigr)
\quad\text{from Lemma~\ref{lemma:ctYKTleCTN}(a)} \\
&\le
\left(\begin{tabular}{@{}l@{}}
\# of points in the sets \eqref{eqn:QYmlen} and~\eqref{eqn:QYmgen}, \\
\end{tabular}
\right) \\
&\le \Cr{dn0cond2}(K,N,f,\Qcal) \cdot T^{1+\dim(Y)} \cdot \log(T) \\
&\hspace*{3em} + \Cr{WKT2}(K,N,f,\Qcal) \cdot T^{1+\dim(Y)} \\
&\omit\hfill
\quad\text{from \eqref{eqn:numpts50} and \eqref{eqn:num51cond}.}
\end{align*}
Letting~$T\to\infty$ forces
\[
\dim(\PP^N) \le \dim(Y),
\]
so the fact that~$Y$ is a Zariski closed subset of~$\PP^N$ implies that~$Y=\PP^N$.
\end{proof}

%%%%%%%%%%%%%%%%%%%%%%%%%%%%%%%%%%%%%%%%%%%%%%%%
% \section{Basic and Auxiliary Results}
\section{Additional Results}
\label{appendix:elementaryresults}
%%%%%%%%%%%%%%%%%%%%%%%%%%%%%%%%%%%%%%%%%%%%%%%%

% *** We can include all of the material in this section in the ArXiv version. How much should we include in the journal version?

The following lemma describes some elementary properties of orbits.

\begin{lemma}
\label{lemma:orbitelemproperties}
\begin{parts}
\Part{(a)}
Grand $f$-orbit equivalence is an equivalence relation.
\Part{(b)}
Let~$\G_1$ and~$\G_2$ be~$f$-grand orbits. Then either
\[
\G_1=\G_2 \quad\text{or}\quad \G_1\cap\G_2=\emptyset.
\]
Hence for $P,Q\in X$, we have
\[
P\fgeq{Q} \quad\Longleftrightarrow\quad Orbit_f^\grand(P)=\Orbit_f^\grand(Q).
\]
\Part{(c)}
If~$\G$ is a grand orbit, then either all of its points are preperiodic, or all of its points are wandering. 
\Part{(d)}
Let~$\G$ be a grand orbit. If~$\G$ contains a point with Zariski dense forward orbit, then every point in~$\G$ has Zariski dense forward orbit. 
\Part{(e)}
If~$f$ is an automorphism, then the $f$-grand orbit of~$Q$ is the union of orbits of~$f$ and it's inverse,
\[
\Orbit_f(Q)^\grand = \Orbit_f(Q) \cup \Orbit_{f^{-1}}(Q).
\]
\Part{(f)}
If~$f$ is an automorphism, then~$P$ is~$f$-wandering if and only if~$P$ is~$f^{-1}$-wandering.
\Part{(g)}
Let $P,Q\in{X}$. Then
\[
P\fgeq Q \quad\Longrightarrow\quad 
\Bigl( P\in X^\dense \;\Longleftrightarrow\; Q\in X^\dense \Bigr).
\]
\end{parts}
\end{lemma}


\begin{proof}[Proof of Lemma \textup{\ref{lemma:orbitelemproperties}}]
(a)\enspace
We first note that
\[
\Orbit_f(P)\cap\Orbit_f(P)\supseteq\{P\}\ne\emptyset
\quad\Longrightarrow\quad P\fgeq P.
\]
Next, since the definitions of grand $f$-orbit equivalence is symmetric in~$P$ and~$Q$, we have
\[
P\fgeq Q \quad \Longrightarrow\quad Q\fgeq P.
\]
It remains to check transitivity. We compute
\begin{align*}
P & \fgeq Q \quad\text{and}\quad Q\fgeq R \\
&\quad\Longrightarrow\quad
\Orbit_f^\grand(P)=\Orbit_f^\grand(Q) 
\quad\text{and}\quad
\Orbit_f^\grand(Q)=\Orbit_f^\grand(R) \\
&\quad\Longrightarrow\quad
\Orbit_f^\grand(P)=\Orbit_f^\grand(R) \\
&\quad\Longrightarrow\quad
P\fgeq R.
\end{align*}
\par\noindent(b)\enspace
Let $\G_1=\Orbit_f^\grand(P_1)$ and  $\G_2=\Orbit_f^\grand(P_2)$, and assume that
$\G_1\cap\G_2\ne\emptyset$. Let $Q\in\G_1\cap\G_2$. This implies that there are iterates of~$f$ satisfying
\[
f^{n_1}(P_1) = f^{m_1}(Q)
\quad\text{and}\quad
f^{n_2}(P_2) = f^{m_2}(Q).
\]
Now let~$R\in\G_1$ be an arbitrary point in~$\G_1$. Our goal is to show that~$R\in\G_2$. The assumption that~$R\in\G_1\Orbit_f^\grand(P_1)$ tells us that there are iterates of~$f$ satisfying
\[
f^p(P_1) = f^q(R).
\]
This allows us to compute
\[
f^{m_2+n_1+q}(R) = f^{m_2+n_1+p}(P_1) = f^{m_2+m_1+p}(Q) = f^{n_2+m_1+p}(P_2.)
\]
This shows that~$\Orbit_f(R)\cap\Orbit_f(P_2)\ne\emptyset$, and hence by definition
$R\in\Orbit_f^\grand(P_2)=\G_2$. This completes the proof that~$\G_1\subseteq\G_2$, and the proof of the opposite inclusion follows by symmetry. Hence~$\G_1=\G_2$.
\par\noindent(c,d)\enspace
Let~$P,Q\in\G$. By definition of grand orbit, the forward orbits of~$P$ and~$Q$ eventually merge, so their set difference\footnote{We use the standard set theory notation that the set difference of sets~$A$ and~$B$ is $A\operatorname{\scriptstyle\triangle}{B}:=(A\cup{B})\setminus(A\cap{B})$, i.e., it is the set of points in exactly one of~$A$ and~$B$.}
\[
\Orbit_f(P) \operatorname{\scriptstyle\triangle} \Orbit_f(Q)
% \Bigl(\Orbit_f(P) \cup \Orbit_f(Q)\Bigr) 
% \setminus \Bigl(\Orbit_f(P) \cap \Orbit_f(Q)\Bigr)
\quad\text{is a finite set.}
\]
This implies first that
\[
\#\Orbit_f(P) - \#\Orbit_f(Q)\quad\text{is finite,}
\]
which shows that~$P$ and~$Q$ are either both preperiodic or both wandering, which completes the proof of~(c). Second, it implies that
\[
\overline{\Orbit_f(P)} \setminus \overline{\Orbit_f(Q)} \quad\text{is finite,}
\]
and similarly with~$P$ and~$Q$ reversed. Hence~$\Orbit_f(P)$ is Zariski dense if and only if~$\Orbit_f(Q)$ is Zariski dense, which completes the proof of~(d).
\par\noindent(e)\enspace
We compute
\begin{align*}
P \in \Orbit_f(Q)^\grand
&\quad\Longleftrightarrow\quad
f^n(P) = f^m(Q) \quad\text{for some $n,m\in\NN$,} \\
&\quad\Longrightarrow\quad
\begin{cases}
P = f^{m-n}(Q) \in \Orbit_f(Q) &\text{if $m\ge n$,} \\
P = (f^{-1})^{n-m}(Q) \in \Orbit_{f^{-1}}(Q) &\text{if $n\ge m$.} \\
\end{cases} \\
&\quad\Longrightarrow\quad
P \in \Orbit_f(Q) \cup \Orbit_{f^{-1}}(Q).
\end{align*}
This gives the inclusion
\[
\Orbit_f(Q)^\grand \subseteq \Orbit_f(Q) \cup \Orbit_{f^{-1}}(Q).
\]
For the opposite inclusion, we note that the forward orbit of~$Q$ is clearly contained in the grand orbit of~$Q$. Finally, if~$P\in\Orbit_{f^{-1}}(Q)$, then~$P=(f^{-1})^n(Q)$ for some~$n\in\NN$, and hence~$f^n(P)=Q=f^{0}(Q)$. Therefore~$P\in\Orbit_f(Q)^\grand$.
\par\noindent(f)\enspace
We compute
\begin{align*}
\text{$P$ is}&\text{~not $f$-wandering} \\
&\quad\Longleftrightarrow\quad
P\in\PrePer(f) \\
&\quad\Longleftrightarrow\quad
f^n(P)=f^m(P)\quad\text{for some $n>m\ge0$,} \\
&\quad\Longleftrightarrow\quad
(f^{-1})^{m}(P)  = (f^{-1})^{n}(P)  
\quad\text{applying $(f^{-1})^{m+n}$,} \\
&\quad\Longleftrightarrow\quad
P\in\PrePer(f^{-1}) \\
&\quad\Longleftrightarrow\quad
\text{$P$ is not $f^{-1}$-wandering.}
\end{align*}
\par\noindent(g)\enspace
Two points that are grand $f$-orbit equivalent have forward orbits that differ by only finitely many points, since their forward orbits eventually merge. Hence
\[
\Bigl( \Orbit_f(P)\cup\Orbit_f(Q) \Bigr)
\setminus \Bigl( \Orbit_f(P)\cap\Orbit_f(Q) \Bigr) \quad\text{is a finite set.}
\]
Therefore~$\Orbit_f(P)$ is Zariski dense in~$X$ if and only if~$\Orbit_f(Q)$ is Zariski dense in~$X$.
\end{proof}

\begin{proposition}
\label{proposition:propagationimplications}
The following implications hold for the various orbit propagation statements in Table~\textup{\ref{table:propogationstatements}}, with the convention described in Remark~\textup{\ref{remark:(A)implies}}. These implications are illustrated in Table~\textup{\ref{table:propagationimplications}}.%
\newcommand{\Imply}[3]
{\label{imply:#1} \textup{(#2)}&\;\Longrightarrow\textup{(#3)}}
\newcommand{\NotImply}[3]
{\label{imply:#1} \textup{(#2)}&\;\centernot\Longrightarrow\textup{(#3)}}
\begin{align}
\NotImply{C1A}{C1}{A} \\
\Imply{B1inftyB1}{B1\text{$\infty$}}{B1} \\
\Imply{C1inftyC1}{C1\text{$\infty$}}{C1} \\
\Imply{C2forallC1infty}{C2\text{$\forall$}}{C1\text{$\infty$}} \\
\Imply{C2forallC2exists}{C2\text{$\forall$}}{C2\text{$\exists$}} \\
\Imply{C1B1}{C1}{B1} \\
\Imply{C1inftyB1infty}{C1\text{$\infty$}}{B1\text{$\infty$}} \\
\Imply{C3existsA}{C3\text{$\exists$}}{A} \\
\Imply{C3forallC3exists}{C3\text{$\forall$}}{C3\text{$\exists$}} \\
\Imply{C3forallC2forall}{C3\text{$\forall$}}{C2\text{$\forall$}} \\
\Imply{C3existsC2exists}{C3\text{$\exists$}}{C2\text{$\exists$}} \\
\Imply{C1inftyC2forall}{C1\text{$\infty$}}{C2\text{$\forall$}} 
\end{align}
\end{proposition}
\begin{proof}
%%%%%%%%%%%%%
\newcommand{\Imply}[3]{\par\noindent\eqref{imply:#1}\enspace
\framebox{$\textup{(#2)}\;\Longrightarrow\textup{(#3)}$}\enspace}
\newcommand{\NotImply}[3]{\par\noindent\eqref{imply:#1}\enspace
\framebox{$\textup{(#2)}\;\centernot\Longrightarrow\textup{(#3)}$}\enspace}
%%%%%%%%%%%%%
\NotImply{C1A}{C1}{A}
If the map~$f$ admits a non-trivial fibration, then~(C1) is true, since any finite set of grand $f$-orbits will be contained in a finite number of fibers, but~(A) is not true, since every grand orbits is contained in a fiber, so there are no Zariski dense grand orbits. We note that it is not clear if~(B1$\infty$) or~(C1$\infty$) implies (A), but neither do we see an obvious counterexample.
%%%%%%%%%%%%%
\Imply{B1inftyB1}{B1$\infty$}{B1}
Let~$\G_1,\ldots,\G_r$ be a collection of $f$-orbits, say $\G_i=\Orbit_f(P_i)$. Then $Y=\{P_1,\ldots,P_r\}$ is a proper Zariski closed subset of~$X$, so applying~(B1$\infty$) to~$Y$ gives~(B1).
%%%%%%%%%%%%%
\Imply{C1inftyC1}{C1$\infty$}{C1}
Let~$\G_1,\ldots,\G_r$ be a collection of grand $f$-orbits, say $\G_i=\Orbit_f^\grand(P_i)$. Then $Y=\{P_1,\ldots,P_r\}$ is a proper Zariski closed subset of~$X$, so applying~(C1$\infty$) to~$Y$ gives~(C1).
%%%%%%%%%%%%%
\Imply{C2forallC1infty}{C2$\forall$}{C1$\infty$}
Let $Y\subsetneq{X}$ be a proper Zariski closed set, and let~$\Qcal\subset{X(K)}$ be any complete set of representatives for~$X(K)/\fgeq$. Thus every grand $f$-orbit~$\G$ that contains a point of~$X(K)$ has the property that $\#(\G\cap\Qcal)=1$. We create a modified version of~$\Qcal$, which we denote by~$\Qcal'$, as follows: For each~$Q\in\Qcal$, if
\[
\Orbit_f^\grand(Q)\cap{Y(K)}\ne\emptyset,
\]
then we replace~$Q$ with one of the points in~\text{$\Orbit_f^\grand(Q)\cap{Y(K)}$}. This modified set~$\Qcal'$ is still a complete set of representatives for~$X(K)/\fgeq$. Our assumption that~(C2$\forall$) is true tells us that~$\Qcal'$ is Zariski dense in~$X$. Since~$Y$ is not Zariski dense in~$X$, it follows that
\begin{equation}
\label{eqn:QcalprimeminusY1}
\text{$\Qcal'\setminus{Y}$ is Zariski dense in~$X$.}
\end{equation}
On the other hand, the construction of~$\Qcal'$ ensures that
\begin{equation}
\label{eqn:QcalprimeminusY2}
Q\in\Qcal'\cap\bigcup_{P\in Y(K)} \Orbit_f^\grand(P)
\quad\Longrightarrow\quad
Q\in Y. 
\end{equation}
Combining~\eqref{eqn:QcalprimeminusY1} and~\eqref{eqn:QcalprimeminusY2}, we deduce that
\[
\Qcal'\setminus \biggl(\bigcup_{P\in Y(K)} \Orbit_f^\grand(P)\biggr)
\supset Q\setminus{Y}\quad\text{is Zariski dense in $X$,}
\]
and since~$\Qcal'$ is a subset of~$X(K)$, we conclude that~(C1$\infty$) is true.
\par
We remark that if we use the same argument to try to prove that (C2$\exists$) implies (C1$\infty$), we run into the problem that the replacement procedure used to create~$\Qcal'$ could potentially replace every point of~$\Qcal$ with a point in~$Y$, which prevents us from concluding that~$\Qcal'$ is Zariski dense in $X$.
%%%%%%%%%%%%%
\Imply{C2forallC2exists}{C2$\forall$}{C2$\exists$}
There certainly exists at least one set of representatives~$\Qcal\subset{X(K)}$ for
\[
X(K)/\fgeq.
\]
Our assumption that~(C2$\forall$) holds tells us that~$\Qcal$ is Zariski dense. Hence there exists at least one Zariski dense set of representatives, so~(C2$\exists$) is true.
%%%%%%%%%%%%%
\Imply{C1B1}{C1}{B1}
Let $\G_1,\ldots,\G_r$ be $f$-orbits, and let~$\L_1,\ldots,\L_s$ be the grand $f$-orbits generated by the points in~$\G_1,\ldots,\G_r$. Then each~$\G_i$ is contained in a (unique)~$\L_j$, so we have the following inclusion:
\[
\overbrace{X(K)\setminus(\L_1\cup\cdots\cup\L_s)}^{\hidewidth\text{(C1) tells us that this set is Zariski dense.}\hidewidth}
\subseteq
\underbrace{X(K)\setminus(\G_1\cup\cdots\cup\G_r) }_{\hidewidth\text{Hence this set is also Zariski dense.}\hidewidth}
\]
%%%%%%%%%%%%%
\Imply{C1inftyB1infty}{C1$\infty$}{B1$\infty$}
We prove these simultaneously.
Every $f$-orbit is contained in a (unique) grand $f$-orbit, since for every point~$P$ we have
\[
\Orbit_f(P) \subseteq \Orbit_f^\grand(P).
\]
This implies that if~$\Pcal\subset{X(K)}$ is any set of points, then
\[
\biggl( X(K)\setminus\bigcup_{P\in\Pcal} \Orbit_f(P) \biggr)
\supseteq \biggl( X(K)\setminus\bigcup_{P\in\Pcal} \Orbit_f^\grand(P) \biggr).
\]
Taking~$\Pcal$ to be a finite set of points gives \text{(C1)}$\;\longrightarrow\;$\text{(B1)}, and taking~$\Pcal=Y(K)$ gives \text{(C1$\infty$)}$\;\longrightarrow\;$\text{(B1$\infty$)}.
%%%%%%%%%%%
\Imply{C3existsA}{C3$\exists$}{A}
We start by extending~$K$ so that~$X(K)\ne\emptyset$. The assumption that~(C3$\exists$) is true says that, after replacing~$K$ by a further finite extension, there is a set of points $\Qcal\subseteq{X(K)}$ such that~$\Qcal$ is Zariski dense in~$X$ and such that~$\Qcal$ is a complete set of representatives for~$X_f^\dense(K)/\fgeq$. Let~$Q\in\Qcal$. Lemma~\ref{lemma:orbitelemproperties}(j) tells us that~$Q\in{X_f^\dense}$, and we know~$Q\in{X(K)}$, so we have found a point~$Q\in{X_f^\dense(K)}$. By definition, the orbit~$\Orbit_f(Q)$ is Zariski dense in~$X$, which proves that~(A) is true.
%%%%%%%%%%%
\Imply{C3forallC3exists}{C3$\forall$}{C3$\exists$}
By assumption, there is a point $Q\in{X_f^\dense}(K)$. It follows that
$X_f^\dense(K)/\fgeq$ is non-empty, so there exists a set~$\Qcal\subset{X_f^\dense(K)}$ of  representatives for $X_f^\dense(K)/\fgeq$. The assumption~(C3$\forall$) tells us that~$\Qcal$ is Zariski dense in~$X$, so the existence of~$\Qcal$ implies that~(C3$\exists$) is true.
%%%%%%%%%%%
\Imply{C3forallC2forall}{C3$\forall$}{C2$\forall$}
Let~$\Qcal\subset{X(K)}$ be a complete set of representatives for $X(K)/\fgeq$. Then $\Qcal\cap{X^\dense_f(K)}$ is a complete set of representative for~$X_f^\dense(K)/\fgeq$, where we are using Lemma~\ref{lemma:orbitelemproperties}(j). The assumption that~(v3$\forall$) is true tells us that~$\Qcal\cap{X^\dense_f(K)}$ is Zariski dense in~$X$, from which it is clear that~$\Qcal$ is Zariski dense in~$X$.
%%%%%%%%%%%
\Imply{C3existsC2exists}{C3$\exists$}{C2$\exists$}
We are given that there exists a set~$\Qcal\subset{X_f^\dense(K)}$ such that~$\Qcal$ is Zariski dense in~$X$ and such that~$\Qcal$ is a complete set of representatives for~$X_f^\dense(K)/\fgeq$. Let~$\Qcal'\subset{(X\setminus{X}_f^\dense)(K)}$ be a complete set of representatives for
\[
(X\setminus X_f^\dense)(K)/\fgeq.
\]
Lemma~\ref{lemma:orbitelemproperties}(j) tells us that~$\Qcal\cup\Qcal'$ is a complete set of representatives for~$X(K)/\fgeq$, and that~$\Qcal\cup\Qcal'$ is Zariski dense in~$X$, since it contains~$\Qcal$. Hence~(C2$\exists$) is true.
%%%%%%%%%%%%%
\Imply{C1inftyC2forall}{C1$\infty$}{C2$\forall$} 
Let~$\Qcal\subset{X(K)}$ be a complete set of representatives for~$X(K)/\fgeq$, and let
\[
Y = \overline{\Qcal} = \text{the Zariski closure of $\Qcal$.}
\]
We suppose that~$Y\ne{X}$, and our goal is to show that~(C1$\infty$) is false for this~$Y$. The choice of~$\Qcal$ tells us that
\begin{equation}
\label{eqn:XKcupQQcalgrand}
X(K) = \bigcup_{Q\in\Qcal} \Bigl( \Orbit_f^\grand(Q)\cap X(K) \Bigr).
\end{equation}
(Indeed, it even tells us that the union is a disjoint union, but we will not need this fact.) The facts~$Y=\overline\Qcal$ and~$\Qcal\subseteq{X(K)}$ imply that
\begin{equation}
\label{eqn:XksetminuscupYQcal}
Y(K)\supseteq\Qcal.
\end{equation}
Hence
\begin{align}
\label{eqn:XknegPinYKgrandorb}
X(K) \setminus \biggl( \bigcup_{P\in Y(K)} & \Orbit_f^\grand(P) \biggr) \\
&\subseteq X(K) \setminus 
\smash[t]{ \biggl( \bigcup_{Q\in\Qcal} \Orbit_f^\grand(Q) \biggr) }
\quad\text{from \eqref{eqn:XksetminuscupYQcal},} \notag \\
&= \emptyset \quad\text{from \eqref{eqn:XKcupQQcalgrand}.} \notag
\end{align}
This certainly contradicts~(C1$\infty$), since~(C1$\infty$) would imply that the set~\eqref{eqn:XknegPinYKgrandorb} is Zariski dense in~$X$.
\end{proof}

The next result describes another relation among the various propagation properties, but it has a slightly different flavor from the results in Proposition~\ref{proposition:propagationimplications}.

\begin{lemma}
\label{lemma:C2forallZDimpliesC3forall}
Let $f:X\to{X}$.  Then
\[
\text{\textup{(A)}\quad and\quad \textup{(C2$\forall$)}\quad
and\quad $\overline{X_f^\dense(K)} = X$}
\quad\Longrightarrow\quad
\textup{(C3$\forall)$}.
\]
\end{lemma}
\begin{proof}
Let
\[
\Qcal\subset X_f^\dense(K)
\quad\text{such that}\quad
\Qcal \xleftrightarrow{\;\text{bijective}\;}
X_f^\dense(K)/\fgeq.
\]
Our goal is to prove that~$\Qcal$ is Zariski dense in~$X$. To do this, we consider the complement of~$X_f^\dense$, and we choose a set
\[
\Qcal'\subset (X\setminus X_f^\dense)(K)
\quad\text{such that}\quad
\Qcal' \xleftrightarrow{\;\text{bijective}\;}
(X\setminus X_f^\dense)(K)\fgeq.
\]
Then the union satisfies
\[
\Qcal\cup\Qcal'
\xleftrightarrow{\;\text{bijective}\;} X(K)/\fgeq,
\]
so the assumption that~(C2$\forall$) is true tells us that~$\Qcal\cup\Qcal'$ is Zariski dense in~$X$. However, the set~$\Qcal'$ is not Zariski dense in~$X$, since
\[
\Qcal' \subset X\setminus X_f^\dense,
\]
and the Zariski density of~$X_f^\dense$ implies that its complement is not Zariski dense. Hence~$\Qcal$ is Zariski dense in~$X$, which completes the proof that~$f$ has propagation property~(C3$\forall$).
%%%%%%%%%%%%%%%%%%%%
% *** Justification that $\Qcal$ is Zariski dense:
% We can work on affine neighborhoods. Let $g$ be a non-zero regular function vanishing on~$\overline{\Qcal'}$. The for any regular~$f$ that vanishes on~$\Qcal$, the product~$fg$ vanishes on the union~$\Qcal\cup\Qcal'$. We know that$\overline{\Qcal\cup\Qcal'}=X$, so~$fg$ vanishes identically, and since~$g=0$, we conclude that~$f=0$. Hence~$\Qcal$ is Zariski dense in~$X$.
%%%%%%%%%%%%%%%%%%%%
\end{proof}

We will apply the next lemma, which is undoubtedly well known, to the case of an automorphism. But since it is no harder to handle the case of arbitrary backward branches, we formulate it in that generality.

\begin{lemma}
\label{lemma:zariskidensebranch}
Let~$P_0\in{X}$ be a point whose forward orbit~$\Orbit_f(P_0)$ is Zariski dense in~$X$, and let
\[
\Bcal = \{P_0,P_1,P_2,\ldots\}\subset X(\Kbar)
\]
be a complete backward $f$-branch of~$P_0$, i.e., a sequence of points in~$X(\Kbar)$ satisfying
\[
f(P_{i+1}) = P_i\quad\text{for all $i\ge0$.}
\]
Then~$\Bcal$ is Zariski dense in~$X$.
\end{lemma}
\begin{proof}
Our first observation is that~$P_0,P_1,\ldots$ are distinct. To see why, suppose that~$P_i=P_j$ for some $i>j$. Then
\[
f^{i-j}(P_0) = f^{i-j}\bigl(f^j(P_j)\bigr) 
= \underbrace{f^i(P_j) = f^i(P_i)}_{\text{since $P_i=P_j$}} = P_0,
\]
which implies that~$P_0$ is periodic and thus contradicts the assumed Zariski density of~$\Orbit_f(P_0)$. (Note we always assume that $\dim(X)\ge1$.)
\par
Our goal is to prove that the Zariski closure~$\overline\Bcal$ of~$\Bcal$ is equal to~$X$. 
For each integer~$m\ge0$, we let
\[
\Bcal_m := \{P_m,P_{m+1},P_{m+2},\ldots\}
\]
denote the branch with it's first~$m$ points omitted. The actions of~$f$ on these sets and their closures satisfy
\begin{equation}
\label{eqn:fBmplus1toBm}
f : \Bcal_{m+1} \longrightarrow \Bcal_m
\quad\text{and}\quad
f : \overline{\Bcal_{m+1}} \longrightarrow \overline{\Bcal_m}.
\end{equation}
\par
We now suppose that
\begin{equation}
\label{eqn:PmnotinBmplus1}
P_m\notin\overline{\Bcal_{m+1}}\quad\text{for all $m\ge0$,}
\end{equation}
and derive a contradiction. Since~$\Bcal_m=\{P_m\}\cup\Bcal_{m+1}$, it would follow from~\eqref{eqn:PmnotinBmplus1} that the single point~$\{P_m\}$ is an irreducible component of~$\overline{\Bcal_m}$; and hence, since~$\Bcal\setminus\Bcal_m$ is a finite set, it would follow that~$\{P_m\}$ is an irreducible component of~$\overline\Bcal$. The Zariski closed set~$\overline\Bcal$ has only finitely many irreducible components, and we proved earlier that the~$P_i$ are distinct, which completes our proof that~\eqref{eqn:PmnotinBmplus1} is false. 
\par
Therefore, there exists an index~$\ell\ge0$ such that
\begin{equation}
\label{eqn:PellinZCBellplus1}
P_\ell \in \overline{\Bcal_{\ell+1}}.
\end{equation}
This implies that
\[
\overline{\Bcal_{\ell}} 
= \overline{\{P_\ell\}\cup\Bcal_{\ell+1}}
= \{P_\ell\}\cup\overline{\Bcal_{\ell+1}}
= \overline{\Bcal_{\ell+1}},
\]
and then~\eqref{eqn:fBmplus1toBm} tells us that~$f$ induces a map
\begin{equation}
\label{eqn:fBellplus1toitself}
f : \overline{\Bcal_{\ell+1}} \longrightarrow \overline{\Bcal_{\ell+1}}.
\end{equation}
We know from~\eqref{eqn:PellinZCBellplus1} that the point~$P_\ell$ is in~$\Bcal_{\ell+1}$, so~\eqref{eqn:fBellplus1toitself} implies that
\begin{equation}
\label{eqn:orbitfPell}
\Orbit_f(P_\ell) \subset \overline{\Bcal_{\ell+1}}.
\end{equation}
We compute
\begin{align*}
X &= \overline{\Orbit_f(P_0)} 
&&\text{by assumption,} \\
&\subseteq \overline{\Orbit_f(P_\ell)} 
&&\text{since $f^\ell(P_\ell)=P_0$,} \\
&\subseteq \overline{\Bcal_{\ell+1}}
&&\text{from \eqref{eqn:orbitfPell},} \\
&= \overline{ \Bcal \setminus \{P_0,\ldots,P_{\ell}\} } 
&&\text{by definition of $\Bcal_m$,} \\
&\subseteq \overline\Bcal.
\end{align*}
Hence~$\overline\Bcal=X$, which completes the proof that~$\Bcal$ is Zariski dense in~$X$.
\end{proof}

If~$f:X\to{X}$ is an automorphism, then the existence of a Zariski dense orbit automatically implies a weak propagation property, as in the following result. The proof follows from the fact that~$\Orbit_f(P)$ is Zariski dense if and only if~$\Orbit_{f^{-1}}(P)$ is Zariski dense. This allows us to use points in the backward orbit to prove that~(A) implies propagation property~(B1); but this is a bit of a cheat, and in any case the idea cannot be used to prove a grand orbit propagation property such as~(C1).

\begin{proposition}
\label{proposition:AimpliesB1forautomorphisms}
Let $f:X\to{X}$ be an automorphism. Then
\[
\textup{(A)} \quad\Longrightarrow\quad \textup{(B1)}
\]
\end{proposition}
\begin{proof}
We are given that there is a point~$P_0\in{X(K)}$ whose $f$-orbit $\Orbit_f(P_0)$ is Zariski dense. The forward~$f^{-1}$-orbit of~$P_0$ is a backward $f$-branch of~$P_0$ in the sense of Lemma~\ref{lemma:zariskidensebranch}, so Lemma~\ref{lemma:zariskidensebranch} tells us that~$\Orbit_{f^{-1}(P_0}$ is Zariski dense in~$X$.
\par
We now commence the proof of propagation property~(B1). Let $\G_1,\ldots,\G_r$ be a collection of~$f$-orbits. The fact that the~$\G_i$ are forward $f$-orbits implies that they have only finitely many points in common with the $f^{-1}$-orbit~$\Orbit_{f^{-1}}(P_0)$.\footnote{Suppose that~$\Orbit_f(Q)\cap\Orbit_{f^{-1}}(P_0)\ne\emptyset$. Then~$Q=f^{-m}(P_0)$ for some~$m\ge0$, so $\Orbit_f(Q)\cap\Orbit_{f^{-1}}(P_0)=\bigl\{f^n(P_0):-m\le{n}\le0\bigr\}$ is a finite set.} Hence
\begin{align*}
X(K)\setminus(\G_1\cup\cdots\cup\G_r) 
&\supseteq \Orbit_{f^{-1}}(P_0) \setminus (\G_1\cup\cdots\cup\G_r) \\
&= \underbrace{\Orbit_{f^{-1}}(P_0) \setminus \{\text{finite set}\}}_{\text{Zariski dense in $X$.}}.
\end{align*}
This completes the proof that~$f$ has propagation property~(B1).
\end{proof}

\begin{remark}
We note that if we assume the dynamical Mordell--Lang conjecture for~$f^{-1}$, then we can strengthen Proposition~\ref{proposition:AimpliesB1forautomorphisms} to the statement that~(A) implies that~$f$ has propagation property~(B1$\infty$). Briefly, if a Zariski closed subset~$Y\subseteq{X}$ contains~$f^{-n}(P_0)$ for infinitely many~$n\ge0$, then Mordell--Lang says that~$Y$ contains an~$f^{-1}$-invariant subvariety~$Z$ that contains those points. In particular, there is a point~$f^{-m}(P_0)\in{Z}$. But then~$Z$ is also~$f$-invariant, so it contains~$f^m(f^{-m}(P_0))=P_0$. Hence~$\Orbit_f(P_0)\subset{Z}$, and then the Zariski density of~$\Orbit_f(P_0)$ forces~$Z=X$, and thus also $Y=X$.
\end{remark}

There are many stronger results that imply the following height counting lemma, but we include it to indicate that the estimate that we need can be obtained in an elementary manner.

\begin{lemma}
\label{lemma:ctYKTleCTN}
Let~$K$ be a number field, and let~$N\ge1$, and compute counting functions using the multiplicative $K$-height on $\PP^N(K)$; see~\textup{\cite[Section~B.2]{MR1745599}}.
%% Hindry-Silverman DGI
\begin{parts}
\Part{(a)}
There are constants~$\Cl{PNapp1}(K,N)>0$ and~$\Cl{PNapp2}(K,N)>0$ such that
\[
\Cr{PNapp1}(K,N) T^{N+1} \le \Count\bigl(\PP^N(K),T\bigr) \le \Cr{PNapp2}(K,N) T^{N+1}.
\]
\Part{(b)}
Let $Y\subsetneq\PP^N$ be a Zariski closed set. 
There is a constant~$\Cl{YinPNlemma}(K,Y)$ such that
\begin{equation}
\label{eqn:CtYKTCTdimy1}
\Count\bigl(Y(K),T\bigr) \le \Cr{YinPNlemma}(K,Y)T^{1+\dim(Y)},
\end{equation}
where the dimension of an algebraic subset of~$\PP^N$ is defined to be the maximum of the dimensions of its geometrically irreducible components. In particular, if~$Y\subsetneq\PP^N$ is a proper subset, then
\[
\Count\bigl(Y(K),T\bigr) \le \Cr{YinPNlemma}(K,Y)T^N.
\]
\end{parts}
\end{lemma}
\begin{proof}
(a)\enspace
Schanual's formula~\cite{MR557080} gives a precise asymptotic formula for~$\Count\bigl(\PP^N(K),T\bigr)$, but it is relatively elementary to obtain the weak estimate that we have cited in~(a).
\par\noindent(b)\enspace
We prove~\eqref{eqn:CtYKTCTdimy1} by induction on~$\dim(Y)$.
If~$\dim(Y)=0$, then~$Y$ is a finite set of points, so its counting function is bounded and~\eqref{eqn:CtYKTCTdimy1} is trivially true.
\par
Suppose now that~$m\ge1$, that we know~\eqref{eqn:CtYKTCTdimy1} for all algebraic sets of dimension at most~$m-1$, and that~$Y\subseteq\PP^N$ is a Zariski closed subset with~$\dim(Y)=m$. (Of course, we must have $m\le{N}$.) Let~$Y_1,\ldots,Y_r$ be the irreducible components of~$Y$ of dimension~$m$. By the induction hypothesis, it suffices to prove~\eqref{eqn:CtYKTCTdimy1} for the union of the~$Y_i$, which reduces us to the case that~$Y$ is irreducible and of dimension~$m$. A generic projection of~$Y$ onto a linear subspace of dimension~$m$ yields a quasi-finite rational map
\[
\f : Y \dashrightarrow \PP^m.
\]
Let~$Y^\circ\subseteq{Y}$ be a non-empty Zariski open subset on which~$\f$ is a quasi-finite morphism of degree~$d$. Then
\begin{align}
\label{eqn:CtYoKT}
\Count\bigl( Y^\circ(K), T\bigr) 
&\le (\deg\f)\cdot\Count\bigl(\PP^m(K),T\bigr)\notag\\
&\le (\deg\f)\cdot \Cl{CtYoKT}(m,K) T^{1+m} quad\text{from (a)}\notag\\
&= \Cr{CtYoKT}(K,Y) T^{1+\dim(Y)}.
\end{align}
We next observe that~$Y\setminus{Y^\circ}$ is a Zariski closed subset of~$\PP^N$ of dimension at most~$m-1$, so our induction hypothesis gives
\begin{equation}
\label{eqn:CtYminusYoKT}
\Count\bigl( (Y\setminus Y^\circ)(K),T\bigr) 
\le \Cl{CtYminusYoKT}(K,Y) T^{1+\dim(Y\setminus Y^\circ)}
\le \Cr{CtYminusYoKT}(K,Y) T^{\dim(Y)}.
\end{equation}
Combining~\eqref{eqn:CtYoKT} and~\eqref{eqn:CtYminusYoKT} yields~\eqref{eqn:CtYKTCTdimy1} for~$Y$, which completes our induction proof that~\eqref{eqn:CtYKTCTdimy1} is true for all algebraic subsets of~$\PP^N$. In particular, if~$Y\subsetneq\PP^N$ is a proper Zariski closed subset, then its dimension is at most~$N-1$. This completes the proof of Lemma~\ref{lemma:ctYKTleCTN}(b).
\end{proof}

%%%%%%%%%%%%%%%%%%%%%%%%%%%%%%%%%%%%%%%%%%%%%%%%%%%%%%%%%%%%%%%%%%%%%%%%%%%%
%% Bibliography - References
%%%%%%%%%%%%%%%%%%%%%%%%%%%%%%%%%%%%%%%%%%%%%%%%%%%%%%%%%%%%%%%%%%%%%%%%%%%%

\begin{thebibliography}{10}

\bibitem{MR2862064}
E.~Amerik, F.~Bogomolov, and M.~Rovinsky.
\newblock Remarks on endomorphisms and rational points.
\newblock {\em Compos. Math.}, 147(6):1819--1842, 2011.

\bibitem{MR2766180}
J.~P. Bell, D.~Ghioca, and T.~J. Tucker.
\newblock The dynamical {M}ordell-{L}ang problem for \'{e}tale maps.
\newblock {\em Amer. J. Math.}, 132(6):1655--1675, 2010.
\newblock MR2766180.

\bibitem{MR4574395}
Jason~P. Bell, Yohsuke Matsuzawa, and Matthew Satriano.
\newblock On {D}ynamical {C}ancellation.
\newblock {\em Int. Math. Res. Not. IMRN}, 8:7099--7139, 2023.

\bibitem{MR1255693}
Gregory~S. Call and Joseph~H. Silverman.
\newblock Canonical heights on varieties with morphisms.
\newblock {\em Compositio Math.}, 89(2):163--205, 1993.
\newblock MR1255693.

\bibitem{MR2597304}
Thomas Dedieu.
\newblock Severi varieties and self-rational maps of {$K3$} surfaces.
\newblock {\em Internat. J. Math.}, 20(12):1455--1477, 2009.

\bibitem{MR1109353}
Gerd Faltings.
\newblock Diophantine approximation on abelian varieties.
\newblock {\em Ann. of Math. (2)}, 133(3):549--576, 1991.

\bibitem{MR337997}
Gerhard Frey and Moshe Jarden.
\newblock Approximation theory and the rank of abelian varieties over large
  algebraic fields.
\newblock {\em Proc. London Math. Soc. (3)}, 28:112--128, 1974.

\bibitem{MR1183561}
Tsuyoshi Fujiwara.
\newblock Varieties of small {K}odaira dimension whose cotangent bundles are
  semiample.
\newblock {\em Compositio Math.}, 84(1):43--52, 1992.

\bibitem{MR0463157}
Robin Hartshorne.
\newblock {\em Algebraic geometry}.
\newblock Graduate Texts in Mathematics, No. 52. Springer-Verlag, New
  York-Heidelberg, 1977.

\bibitem{MR1745599}
Marc Hindry and Joseph~H. Silverman.
\newblock {\em Diophantine geometry}, volume 201 of {\em Graduate Texts in
  Mathematics}.
\newblock Springer-Verlag, New York, 2000.
\newblock An introduction.

\bibitem{MR3586372}
Daniel Huybrechts.
\newblock {\em Lectures on {K}3 surfaces}, volume 158 of {\em Cambridge Studies
  in Advanced Mathematics}.
\newblock Cambridge University Press, Cambridge, 2016.
\newblock MR3586372.

\bibitem{MR3456169}
Shu Kawaguchi and Joseph~H. Silverman.
\newblock Dynamical canonical heights for {J}ordan blocks, arithmetic degrees
  of orbits, and nef canonical heights on abelian varieties.
\newblock {\em Trans. Amer. Math. Soc.}, 368(7):5009--5035, 2016.
\newblock MR3456169.

\bibitem{MR65218}
Serge Lang and Andr\'{e} Weil.
\newblock Number of points of varieties in finite fields.
\newblock {\em Amer. J. Math.}, 76:819--827, 1954.

\bibitem{MR2917181}
Jun Li and Christian Liedtke.
\newblock Rational curves on {K}3 surfaces.
\newblock {\em Invent. Math.}, 188(3):713--727, 2012.
\newblock MR2917181.

\bibitem{MR4543765}
Yohsuke Matsuzawa, Sheng Meng, Takahiro Shibata, and De-Qi Zhang.
\newblock Non-density of points of small arithmetic degrees.
\newblock {\em J. Geom. Anal.}, 33(4):Paper No. 112, 41, 2023.

\bibitem{MR3871505}
Yohsuke Matsuzawa, Kaoru Sano, and Takahiro Shibata.
\newblock Arithmetic degrees and dynamical degrees of endomorphisms on
  surfaces.
\newblock {\em Algebra Number Theory}, 12(7):1635--1657, 2018.
\newblock MR3871505.

\bibitem{MR1323985}
Michael McQuillan.
\newblock Division points on semi-abelian varieties.
\newblock {\em Invent. Math.}, 120(1):143--159, 1995.

\bibitem{arxiv0901.2352}
Alice Medvedev and Thomas Scanlon.
\newblock Polynomial dynamics, 2009.
\newblock \url{arXiv:0901.2352}.

\bibitem{MR2514037}
David Mumford.
\newblock {\em Abelian varieties}, volume~5 of {\em Tata Institute of
  Fundamental Research Studies in Mathematics}.
\newblock Published for the Tata Institute of Fundamental Research, Bombay; by
  Hindustan Book Agency, New Delhi, 2008.
\newblock With appendices by C. P. Ramanujam and Yuri Manin, Corrected reprint
  of the second (1974) edition, MR2514037.

\bibitem{MR2551469}
Noboru Nakayama and De-Qi Zhang.
\newblock Building blocks of \'{e}tale endomorphisms of complex projective
  manifolds.
\newblock {\em Proc. Lond. Math. Soc. (3)}, 99(3):725--756, 2009.

\bibitem{MathOverflow267708}
Francesco Polizzi.
\newblock Infinitely many exceptional curves on ruled surfaces.
\newblock MathOverflow.
\newblock \url{https://mathoverflow.net/q/267708} (version: 2020-06-15).

\bibitem{MR3210707}
Bjorn Poonen.
\newblock {$p$}-adic interpolation of iterates.
\newblock {\em Bull. Lond. Math. Soc.}, 46(3):525--527, 2014.

\bibitem{MR557080}
Stephen~Hoel Schanuel.
\newblock Heights in number fields.
\newblock {\em Bull. Soc. Math. France}, 107(4):433--449, 1979.

\bibitem{MSE493989}
Brian~M. Scott.
\newblock Difference of closures and closure of difference.
\newblock Mathematics Stack Exchange.
\newblock \url{https://math.stackexchange.com/q/493989} (version: 2013-09-15).

\bibitem{MR2179691}
Jean-Pierre Serre.
\newblock {\em Lie algebras and {L}ie groups}, volume 1500 of {\em Lecture
  Notes in Mathematics}.
\newblock Springer-Verlag, Berlin, 2006.
\newblock 1964 lectures given at Harvard University, Corrected fifth printing
  of the second (1992) edition.

\bibitem{MR1009803}
Joseph~H. Silverman.
\newblock Integral points on curves and surfaces.
\newblock In {\em Number theory (Ulm, 1987)}, volume 1380 of {\em Lecture Notes
  in Math.}, pages 202--241. Springer, New York, 1989.

\bibitem{MR2316407}
Joseph~H. Silverman.
\newblock {\em The arithmetic of dynamical systems}, volume 241 of {\em
  Graduate Texts in Mathematics}.
\newblock Springer, New York, 2007.

\bibitem{MR2793038}
Michel Waldschmidt.
\newblock On the {$p$}-adic closure of a subgroup of rational points on an
  {A}belian variety.
\newblock {\em Afr. Mat.}, 22(1):79--89, 2011.

\bibitem{MR2408228}
Shou-Wu Zhang.
\newblock Distributions in algebraic dynamics.
\newblock In {\em Surveys in differential geometry. {V}ol. {X}}, volume~10 of
  {\em Surv. Differ. Geom.}, pages 381--430. Int. Press, Somerville, MA, 2006.

\end{thebibliography}


%% \bibliographystyle{plain}
%% \bibliography{Propagation}


\end{document} 

