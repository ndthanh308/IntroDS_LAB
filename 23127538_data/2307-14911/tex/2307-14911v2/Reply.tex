%%%%%%%%%%%%%%%%%%%%%%%%%%%%%%%%%%%%%%%%%%%%%%%%%%%%%%%%%%%%%%%%%%%%%%%%%%%%%%%%
\documentclass[
preprint,          % 12pt, single-column
superscriptaddress,% authors with affiliations via superscripts
amsmath,           % add AMS-Latex features
amssymb,           % add extra AMS symbols, including amsfonts
aps,               % aps or aip
prl,               % prl, pra, prb, prc, prd, pre, prstab
notitlepage,       % control appearance of title page
longbibliography,  % show  article titles in the bibliography
floatfix,          % process floats as early as possible
nofootinbib,
onecolumn,
]{revtex4-1}

\usepackage{bm}         % \bm{<text>} Bold math symbols
\usepackage{amsmath}
\usepackage{graphicx}   % include figures
\usepackage[
colorlinks=true,        % color link
citecolor=blue,         % cite color
linkcolor=blue,         % link color
urlcolor=blue           % url color
]{hyperref}             % create hyperlinks
\usepackage{color}    
\newcommand{\nc}{\newcommand*} 
\nc{\fpbh}{f_{\mathrm{pbh}}}    % f_pbh
\nc{\Mmin}{{M_{\mathrm{min}}}}
\nc{\Mf}{{M_\mathrm{f}}}
\nc{\al}{\alpha}
\newcommand{\red}[1]{\textcolor{red}{#1}}

\newcommand{\cR}{\mathcal{R}}

\begin{document}

\noindent Response to Report of the Referee \\

We would like to thank the referee for the valuable comments and suggestions. We have considered all of them and revised the manuscript accordingly. 
Our responses to the comments are given below.


\begin{enumerate}
\item \red{The authors compute PBH overproduction limits assuming threshold statistics. The threshold for collapse, which depends on the Eos of the universe at the time of collapse, was chosen to be the upper bound from Ref.~144. However, as shown in Ref.~144, figure 3, this fails to describe the correct scaling of the threshold with w (drastically dropping close to $w\rightarrow0$) seen in dedicated numerical simulations (e.g. https://arxiv.org/pdf/1201.2379.pdf figure 8). I would suggest the authors to reconsider their choice of threshold and describe better the motivation behind this choice.}

We kindly thank the referee's suggestions. We do agree that the computation of PBH abundance deeply relies on the threshold for collapse which depends on the Eos of the Universe at the time of collapse, and there is a large uncertainty for the choice of threshold. However, we want to stress that it is no strange that the choice of threshold in our paper is inconsistent with Ref.~\cite{Musco:2012au}, because we assume different scenarios. For the same EoS parameter $w$, we consider that the early Universe was dominated by a canonical scalar field with $c_s^2=1$ while Ref.~\cite{Musco:2012au} assumes an adiabatic perfect fluid with $c_s^2=w$. Especially, when $w\rightarrow0$, the case of our paper is still that the early Universe was dominated by a canonical scalar field while the case of Ref.~\cite{Musco:2012au} is the dust-like stage.  We think the choice of threshold in our paper is suitable and well-motivated. Let's explain. It is important to note that the fluctuations of the scalar field propagate at the speed of sound squared, making the formation of PBHs more challenging compared to an adiabatic perfect fluid. Therefore,  we adopt the upper limit of the density threshold as suggested by Ref.~\cite{Harada:2013epa}. In fact, our choice of threshold is also adopted in Refs.~\cite{Domenech:2020ers,Balaji:2023ehk}.

In the revised version, we have added the motivation and discussion of our choice below Eq.~(2.5).


\item \red{It is interesting to see the negative correlation between w and A observed in the reconstructed posterior. I would suggest the author to explain more in detail why this is observed in the posterior, as it is crucial to avoid PBH overproduction.}

We sincerely appreciate the valuable suggestion provided by the referee. We have added a discussion at the end of \textit{Data analyses and results} section as\\
\textit{``The negative correlations observed between $w$ and $A$ in Fig.~3 can be explained as follows. Referring to Eq.~(2.2), with fixed values of $k$, $k_*$ and $\Delta$, we have $\Omega_{\rm GW,0}h^2 \propto \Omega^\delta_{\rm GW,0}h^2$. By further evaluating Eq.~(2.4), we have $\Omega^\delta_{\rm GW,0}h^2 \propto A^2\, (k/k_\mathrm{rh})^{-2b}\, \frac{\mathcal{T}(\frac{k_*}{k}, \frac{k_*}{k}, w)}{(k/k_*)^2} \Theta(2-\frac{k}{k_*})$, where $\Theta$ is Heaviside step function. For the case where $k_\mathrm{rh}\ll k \ll k_*$ and $w>0$, it becomes evident that an increase in $w$ results in a corresponding increase in $\Omega^\delta_{\rm GW,0}h^2$, thus necessitating a smaller value of $A$ to compensate for it."
}



\item \red{Finally, the authors write in the abstract that the posterior over w points towards a period of condensate domination. While this is certainly allowed by the data, it seems radiation domination falls within the 90$\%$ CI, therefore no clear preference is observed. I would suggest the authors to rephrase this statement in the abstract and the paper.}

Following the referee's suggestion, we have placed additional emphasis on the consistency of the data with radiation domination in both the \textit{Abstract} and the \textit{Summary and discussion} section.
In the \textit{Abstract}, we have rephrased the sentence as\\
\textit{``..., indicating a period of condensate domination at the production of SIGWs is allowed by the data. Moreover, the data also supports radiation domination ($w=1/3$) within the $90\%$ credible interval."}\\

In the \textit{Summary and discussion} section, we have added a sentence as\\
\textit{``Nevertheless, we should emphasize that the radiation domination ($w=1/3$) is consistent with the PTA data within the $90\%$ credible interval."}

\end{enumerate}

We have highlighted the changes in red fonts in the manuscript. 
We hope these changes will address the referee's concerns. Please reconsider our paper for publication in JACP. \\

\noindent Sincerely,\\
Lang Liu, Zu-Cheng Chen, Qing-Guo Huang


%%%%%%%%%%%%%%%%%%%%%%%%%%%%%%%%%%%%%%%%%%%%%%%%%%%%%%%%%%%%%%%%%%%%%%%%%%%%%%%%
%%%%%%%%%%%%%%%%%%%%%%%%%%%%%%% %%% references %%%%%%%%%%%%%%%%%%%%%%%%%%%%%%%%%%
\bibliography{refs}
%%%%%%%%%%%%%%%%%%%%%%%%%%%%%%%%%%%%%%%%
\end{document}

%%%%%%%%%%%%%%%%%%%%%%%%%%%%%%%%%%%%%%%%