%-------------------------------------------------------------------------------
\section{Use of OIDC² in Applications}
\label{sec:applications}
%-------------------------------------------------------------------------------
% Überblick über diese Section:
We explain how end-to-end authentication is currently implemented in the communication applications video conferencing, instant messaging and  email and how it can be improved by OIDC².
In addition, we recommend validity periods for ICTs depending on these applications.

\subsection{Video Conferences}
\label{sec:vc}
%-----------------------------------
% Überblick über die Section:
In many videoconferencing systems, users must rely on the identities of their communication partners provided by the service's IdP.
As an incident \cite{Oltermann2022} demonstrated in 2022, new technologies such as deep fakes show that relying on identifying a communication partner in a video stream does not suffice.
We explain how video conferencing services use OAuth 2.0 and OpenID Connect and how they can benefit from OIDC².

\subsubsection{End-to-End Authentication in Video Conferences}
% Einführung wie Authentifizierung in VCs mit OAuth 2.0 und OIDC funktioniert:
In video conferencing (VC), users log in to the service provider's OAuth 2.0 Authorization Server (AS) either directly with their credentials or through the OP with their OIDC identity, as explained in \sect{sso}.
After authentication, the VC service provider's AS gets an ID Token from the OP.
After authorization, the Client gets an Access Token (AT) from the AS.
The Client uses the AT to prove its authorization to the VC server.
The AT contains the EU's VC account ID, which the VC server uses to identify the EU's profile that the VC server provides to the communication partner.

% Einführung in die E2EE und Vertrauensproblematik:
A communication partner, aka the AU, identifies the EU using the identity claims and its AP's public key $K^+$ provided by the VC server.
Based on this public key, the client of the AU, aka the AP, establishes an end-to-end encrypted communication channel with the Client of the EU.
If the AU does not trust the VC server, he cannot trust the identity of the EU and therefore cannot be sure with whom he is establishing a secure channel.

\subsubsection{End-to-End Authentication with OIDC²}
% Grobe übersicht; Verweis auf Grafik:
We propose that the EU authenticates to the AU using an ICT obtained directly from the EU's OP.
After a mutual ICT exchange, the Client and the RP use the contained verified public keys to establish a secure channel, as shown in \fig{oidc2_vc}.

% Figure environment removed

% Verbindungsanfrage Client -> AP:
First, Client A generates an ephemeral key pair $K^\pm_A$ and contacts the OP of the EU's choice to obtain an ICT containing its public signing key $K^+_A$ (1).
The Client signs this ICT and a unique session identifier with its private key $K^+_A$ and sends it to the AP via the VC server (2).
The session identifier is required to prevent relaying attacks.
% Verbindungsannahme AP -> Client:
If the AU trusts the EU's OP, Client B generates its own ephemeral key pair $K^\pm_B$ (3) and requests an ICT containing its public signing key $K^+_B$ from the AU's OP (4).
The AP signs its ICT and the session identifier with its private signing key $K^-_B$ and responds to the Client via the VC server (5).
% Verbindungsaufbau:
If the EU trusts the AU's OP (7), then the Client and AP have successfully performed mutual authentication enabling them to establish a secure e2e encrypted and authenticated channel (7).

\subsubsection{Discussion}
% OIDC² kann auch im Kontext von OAuth und OIDC eingesetzt werden, um sicheren Kanal aufzubauen.
We explained how OIDC² can be used in the context of OAuth 2.0 and OpenID Connect.
OIDC² can also be used to establish a secure channel based on an untrusted communication channel.
% PoP ist hier das ICT und die Session ID:
The PoP is the signature over a unique session identifier and the ICT.
This session identifier must not be reused and should therefore contain unique components such as a timestamp and an identifier of the communication partner.
% Validity Period of ICT:
Since starting a videoconference takes only a few seconds, the validity of an ICT can be very short, e.g., 5 minutes, to avoid time synchronization problems.
When the ICT expires, an active secure channel remains valid until it is closed.

\subsection{Instant Messaging}
\label{sec:im}
%-----------------------------------
% Section-Übersicht:
We suggest how the instant messaging (IM) service Signal \cite{signal} could benefit from OIDC².

\subsubsection{End-to-End Authentication in Signal}
% Einführung in Ende-zu-Ende Authentifizierung in Signal und In-Person Verifikation mit QR Codes:
In the Signal protocol \cite{WhatsappWhitepaper}, users are identified by their phone number and public key $K^+$, both of which are verified and published by the service provider.
Two users establish an end-to-end encrypted communication channel and authenticate using digital signatures with their respective private key $K^-$, which remain on their primary devices.
To prove his identity to a communication partner and detect man-in-the-middle attacks, a user must present his public key via QR code.
The partner's client then verifies that the scanned public key $K^+$ matches the public key $K'^+$ used to authenticate the channel.
This is a strong but cumbersome verification mechanism that requires either a face-to-face meeting or a secure side channel.

\subsubsection{End-to-End Authentication with OIDC²}
% Einführung; Verweis auf Figure.
We propose an end-to-end authentication method for instant messaging based on OIDC², illustrated in \fig{e2e_im_authentication}.
% Figure environment removed
% Kurze Erklärung, wie Kanal mit OIDC² nachträglich authentifiziert wird:
Assume that two IM clients have already established a secure channel and know each other's public key $K'^+$ provided by the service provider.
Using OIDC², the IM Client requests an ICT from its EU's OP for this public key $K^+$ and sends the ICT over the secure channel to the AP.
If the AU trusts the EU's OP, the AP verifies the received ICT and compares the contained public key $K^+$ with the assumed public key $K'^+$.

\subsubsection{Discussion}
% Erklärung, dass der Channel der PoP ist:
This example shows that an ICT can also be used to authenticate an established secure channel.
Therefore, the ICT must be issued for the public key $K^+$ used to authenticate the channel.
Being able to send messages through this secure channel serves as implicit PoP for the corresponding private key $K^-_C$.

% ICT ersetzt secure side channel; Empfehlung für ICT validity time:
The ICT signed by an OP trusted by the AU thus replaces the need for a face-to-face meeting or a secure side channel.
This requires that the ICT is valid when the AP receives it and adds a timestamp to it.
After that, the AP can verify the ICT at any time later, so the AU does not need to immediately confirm trust to the OP.
Since IM services deliver their messages to the receiving client very quickly, we recommend a validity period of 5 minutes for ICTs in this context.
If the ICT is transmitted when the AP is offline, the verification process must be repeated.

% Zusammenfassung:
This approach does not use any Signal-specific features and can therefore be applied to any other IM service.
It shows that existing key management systems like Signal's can be extended with OIDC² as an authentication layer.
It also shows that OIDC² can be used without any OAuth 2.0 Authorization Server involved in the communication protocol.

\subsection{Email}
\label{sec:email}
%-----------------------------------
% Überblick über diese section:
For the past three decades, S/MIME and PGP have been state-of-the-art standards for secure end-to-end authenticated and optionally encrypted email communication.
But with $2.8 \%$ of signed and $0.06 \%$ of encrypted emails \cite{Stransky2022}, neither of these technologies has taken off, probably due to their complex key generation and management.
We briefly describe email authentication with PGP and S/MIME, propose a variant using OIDC², and explain its advantages.

\subsubsection{End-to-End Authentication with PGP and S/MIME}
% Kurze Einführung in PGP und S/MIME:
The user generates a long-term PGP key pair $K^\pm_{PGP}$ and imports it into his email client.
When sending an email, the client attaches the public PGP key $K^+_{PGP}$ and signs the whole email with the private PGP key $K^-_{PGP}$.
The recipient of the email then verifies the signature using the provided public key $K^+_{PGP}$.
To authenticate the sender, the receiver must know whether the public key $K^+_{PGP}$ belongs to the sender.
This requires a cumbersome Web of Trust-based approach in which people must often meet in person to sign each other's key pairs or exchange public key fingerprints.
Email authentication with S/MIME works similarly, but with a PKI-based approach using S/MIME certificates instead of the Web of Trust approach.
The drawbacks have been discussed in \sect{idp}.

\subsubsection{End-to-End Authentication with OIDC²}
% Erklärung warum PGP verwendet wird; Referenz auf Grafik:
While S/MIME benefits from the trust layer of a PKI, PGP lacks such a layer.
This is where OIDC² can help, as shown in \fig{e2e_email_authentication}.
% Figure environment removed
% Email Signieren und versenden:
For each email, the EU's Client requests a unique ICT for a uniquely generated ephemeral public PGP key $K^+_{PGP}$.
This requires the EU to log in to his OP and authorize the issuance of the ICT for the email context.
The Client then attaches the ICT and PGP-related attachments to the email, e.g. the public PGP key $K^+_{PGP}$ for normal PGP compatibility.
Before sending, the Client signs the entire email with its private PGP key $K^-_{PGP}$.

% Email empfangen und verifizieren:
The receiving Client uses the attached PGP public key $K^+_{PGP}$ to verify the signature as usual in PGP.
To authenticate the sender, the receiving Client, aka Authenticating Party (AP), verifies the ICT using the OP's public key $K^+_{OP}$ and compares its contained public key to the attached public PGP key $K^+_{PGP}$.
To verify the trust level of the OP, the AP can use preconfigured policies or ask his user, aka the Authenticating User (AU).
If the integrity of the ICT is verified and the OP is trusted, the AU can rely on the identity claims that identify the EU.

\subsubsection{Discussion}
% Erklärung, dass email der PoP ist:
The email including attachments combined with timestamps is considered unique.
Thus, the signed email serves as Proof of Possession (PoP) of the corresponding private key $K^-_{PGP}$.
If an attacker mutates the email or replaces the ICT, it will be detected when verifying the signature.

% Erklärung, dass bei vertrauenswürdigem Posteingangszeitstempel die Verifikation zurückdatiert werden kann:
The AU can rely on the inbox timestamp added to the email by his trusted email server when verifying the PoP and ICT.
Therefore, a validity period of 1 hour is sufficient as most emails are delivered to the server within this period.
% Funktioniert aber nicht, wenn dem Email Server nicht vertraut wird:
However, if the AU does not trust his email server, his trusted email client must receive the email within this period.
Otherwise the ICT will expire and the Client cannot trust the key pair and therefore cannot trust the email.

% Limitation von OIDC² durch short-term key und Lösung durch Kombination mit long-term key.
This shows the limitations of short-term keys in OIDC².
However, as described in \sect{ict_validity_period}, the EU could choose a long-term key such as a normal PGP key instead of the ephemeral key.
The OP must then add the URL of the key server that publishes the PGP key revocation to the ICT.
When verifying the ICT, the AP must also verify that the PGP key has not been revoked.
