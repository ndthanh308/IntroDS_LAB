%-------------------------------------------------------------------------------
\section{Introduction}
\label{sec:introduction}
%-------------------------------------------------------------------------------
% Einführung, warum E2E Authentication notwendig ist:
In most communication services, users identify each other through account profiles in which they provide their own identity information.
To make these profiles more trustworthy, social network operators such as Meta and Twitter offer identity verification services for an additional fee that can only be used within their ecosystem.
However, identity verification is often a cumbersome process that users may not want to repeat for each of their service platforms.
In addition, users must still trust the service provider to sufficiently verify identities and not impersonate them.
End-to-end user authentication mechanisms attempt to solve this problem, but they often lack adoption due to poor usability.
% Vorstellung OIDC und SSO; Hinweis darauf, dass E2E Authentication fehlt:
Therefore, reusing an account for a verified identity would be desirable.
With modern single sign-on (SSO) services, users can reuse their existing accounts to log in to other services.
The OpenID Connect (OIDC) protocol, which is based on the OAuth 2.0 authorization framework, is widely used for this purpose.
However, OIDC is designed for user-to-service authentication and does not address the purpose of end-to-end user authentication.

% Vorstellung ICT und OIDC²:
In this paper, we define a new Identity Certification Token (ICT) for OIDC.
It is similar to the ID Token which holds identity claims about a user, but also contains a verified public key of the user's client.
As such, it can be thought of as a JSON-based, short-lived user certificate without the need for a revocation mechanism.
The use of an ICT differs significantly from the use of an ID Token.
A user requests an ICT from his OpenID Provider (OP) and presents it to another user's client to authenticate himself.
If the other user trusts the issuing OP, his client verifies the integrity and validity of the ICT and authenticates the user using his client's public key contained in the ICT.
As the OP certifies the identity of the user, we call this concept Open Identity Certification with OpenID Connect (OIDC²).
It facilitates mutual authentication of users if they trust each other's OP.

% Notwendigkeit von Identitätsverifizierung und Relevanz von OIDC:
While most OPs have a rather superficial identity verification process for their accounts, some practice a more thorough verification.
In particular, new players such as banks and government institutions that perform rigorous identity verification for their accounts are becoming OPs.
With OIDC², unknown users can be reliably authenticated if they have an OIDC account at a trusted OP.
% Kritik an PKI und SSI:
Some services already provide strong user authentication, but these methods are difficult to use.
Many instant messaging services support the exchange of public keys between users when they meet in person.
Public key infrastructures (PKIs) require certificate management by users and reliable revocation list checking.
PGP or S/MIME have long been proposed for email authentication, but are rarely used \cite{Stransky2022}.
Self-Sovereign Identity (SSI) technology is currently emerging and solves this problem with device-specific long-term keys in a wallet app.
However, this requires not only revocation mechanisms, but also recovery mechanisms in case the phone with the wallet app is lost or stolen.
% Was OIDC² besser macht und wie:
OIDC² provides a more user-friendly alternative for end-to-end authentication.
The ICT is short-lived, eliminating the need for cumbersome key revocation mechanisms, which improves security.
OIDC² avoids complex key management across devices by simply requesting a new ICT from the OP whenever needed.
Using trusted OPs that verify the identity of their users also eliminates the need for face-to-face key exchange.
Thus, a trusted OP can be compared with a trusted certification authority in a PKI or a trusted issuer in the SSI context.
However, OIDC² is only a lightweight extension for end-to-end authentication with existing OIDC accounts.
It is not intended to replace PKIs or SSIs.

% Vorstellung der Paper-Struktur:
The paper is structured as follows.
In \sect{background}, we revisit basics of OAuth 2.0 and OIDC, and in \sect{related_technologies}, we review related authentication technologies.
\sect{oidc2} introduces the concept of OIDC² and proposes the extension to the OIDC protocol.
Trust relationships in OIDC², a classification of OPs, authentication with multiple ICTs, and validity periods of ICTs are discussed in \sect{security}.
In \sect{applications}, we explain how OIDC² can be applied to video conferencing, instant messaging, and email.
To test OIDC², we provide a simple extension to the OIDC server software in \sect{implementation}, which we evaluate in \sect{evaluation}.
\sect{conclusion} concludes our findings.
