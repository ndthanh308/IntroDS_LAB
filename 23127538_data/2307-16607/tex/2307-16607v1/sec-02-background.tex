%-------------------------------------------------------------------------------
\section{Introduction to OAuth 2.0 and OIDC}
\label{sec:background}
%-------------------------------------------------------------------------------
% Überblick über Struktur der Section:
We introduce basics of OAuth 2.0 and OIDC, as they are the underlying technologies for OIDC².
We discuss their trust relationship and explain how they facilitate single sign-on (SSO).
In the following, we capitalize technical terms from the official OAuth 2.0 and OIDC terminology.
For ease of understanding, we omit non-essential steps in the protocols and refer to the authoritative standards for details.

\subsection{OAuth 2.0}
\label{sec:oauth2}
%-----------------------------------
% Einführung OAuth; Begriffsdefinitionen:
The OAuth 2.0 authorization framework, as defined in \rfc{6749}, is based on HTTP (\rfc{7231}) and the JavaScript Object Notation (JSON) \rfc{8259}.
It allows a user to grant his Client scoped access to his resources on a server, e.g., to only read emails.
A Client can be a web application, or a native email client application.
In OAuth 2.0, this server is called Resource Server (RS) because it protects the user's Protected Resources (PR); the user is called the Resource Owner (RO).

% Vorteil von OAuth gegenüber Username/Password Autorisierung; Einführung Access Token:
Without OAuth 2.0, the RO would leave his credentials with his Client to log in directly to the RS.
With OAuth 2.0, the RO logs in to an Authorization Server (AS) and tells the AS to authorize the Client to access a scoped set of PRs.
To do this, the AS issues an Access Token (AT) to the Client.
This AT allows the Client to access the PRs on the RS.
In this way, OAuth 2.0 improves the security by granting Clients only a scoped access to the user's account without exposing the user's credentials to any of the Clients.

% Wie das Access Token abgerufen wird:
\fig{oauth_authorization_flow} shows a simplified authorization request where the RO authorizes his Client to read email.
% Figure environment removed
First, the Client requests access to the Scope \texttt{read\_email}, which authorizes read-only access to the RO's emails (1).
Then, the AS authenticates the RO (2) and the RO authorizes the Client for the requested Scope (3).
Finally, the AS issues the AT and optionally a Refresh Token (RT) (4).
This AT contains the authorized Scopes with a short validity period.
It is signed with the AS's private key $K^-_{AS}$.
The RT is like a revocable session ID that authorizes the Client to refresh an expired AT without further user interaction.

% Wie das Access Token einegsetzt wird um Protected Resources abzurufen:
\fig{oauth_resource_request} describes a Resource Request where the Client uses the AT to access PRs on the RS.
First, the Client requests the PRs and provides the AT to prove authorization (1).
Then, the RS verifies the AT for a sufficient Scope, its expiration date, and the validity of its signature with the AS's public key $K^+_{AS}$ (2).
Finally, the RS responds with the PRs (3).

\subsection{OpenID Connect (OIDC)}
\label{sec:oidc}
%-----------------------------------
% Einführung OpenID Connect; Abgrenzung zu OAuth:
OpenID Connect (OIDC) \cite{OidcCore} is an authentication framework that allows users to be authenticated with an SSO identity through a third-party service, such as an email service.
It extends OAuth 2.0 for this purpose.
Unlike the example in \sect{oauth2}, the SSO identity has no relationship to the third-party service.

% Figure environment removed

% Begriffsdefinitionen mit Abgrenzung zu OAuth:
In OIDC, an End-User (EU) is authenticated by an OpenID Provider (OP).
The EU grants OIDC-specific Scopes to the EU's intended service, e.g., to his email client, which is called the Relying Party (RP).
This communication flow is supported by OAuth 2.0, where the EU corresponds to a RO, the OP to an AS, and the RP to a Client.
The OP issues an ID Token (IDT) to the RP.
This IDT contains claims about the authentication event which typically includes information about the EU, such as his name or address.
Since this Authorization Request is for authentication, it is called an Authentication Request in OIDC.
With this mechanism, an EU can be authenticated with his SSO identity by different services without providing his credentials.
Instead, the OP passes profile information about the EU to the RP as identity claims in the IDT, but the EU controls which information is passed.

% Wie das ID Token abgerufen wird:
\fig{oidc_authentication_flow} describes a simplified Authentication Request where the EU is authenticated by the RP via the OP.
First, the RP requests access to the \texttt{profile} Scope.
If the EU grants this Scope, the RP is authorized to access the EU's profile information (1).
Then, the EU is authenticated by the OP (2) and authorizes the RP for the requested Scope (3).
Finally, the OP issues an IDT in addition to the AT and an optional RT (4).
This IDT contains the identity claims related to the authorized \texttt{profile} Scope, such as the EU's name, profile picture, or date of birth, and is signed with the OP's private key $K^-_{OP}$.
The RP can verify the signature of the identity claims with the OP's public key $K^+_{OP}$.

\subsection{Trust Relationship}
\label{sec:trust_relationship}
%-----------------------------------
% Einführung in Trust-Relationship; Referenz auf Abbildung:
The following describes the resulting trust relationship between entities in OAuth 2.0 and OIDC as shown in \fig{oidc_trust_relationship}.
% % Figure environment removed

% Trust-Verhältnis Client/RP <-> AS/OP:
The Client/RP never sees any credentials from the RO/EU because the authentication process is performed solely by the AS/OP.
Therefore, the Client/RP must rely on the AS/OP to correctly verify the identity of the RO/EU and that the AS/OP will issue the correct AT/IDT of the authenticated RO/EU.
% Trust-Verhältnis RO/EU <-> Client/RP:
Once the Client/RP of the RO/EU receives the tokens, the RO/EU may not be able to verify what it is doing with them.
The Client/RP may even leak the tokens, so the RO/EU must trust that it is working as intended.
To minimize this risk, the RO/EU restricts the Client/RP's access to only the necessary PRs and identity claims.
% Trust-Verhältnis AS/OP <-> RO/EU:
The RO/EU must also trust the AS/OP to protect his identity.
This includes a secure login process and secure credential storage, but also that the AS/OP will not impersonate his account.
Such impersonation would not even require any credentials since the AS/OP needs only its private key $K^-_{AS}$/$K^-_{OP}$ to sign an AT/IDT.

\subsection{Single Sign-On with OAuth 2.0 and OIDC}
\label{sec:sso}
%-----------------------------------
% Einführung in Single Sign-On; Referenz auf Abbildung:
Today, many services require dedicated accounts, forcing users to remember multiple service-specific credentials.
With SSO systems, users only need to remember the credentials for one account.
They can use this SSO identity to log in to multiple service accounts.
Logging in to a service account with this SSO identity is typically solved with a combination of OAuth 2.0 and OIDC, as depicted simplified in \fig{oidc_sso}.
% Figure environment removed

% Beschreibung des OAuth Ablaufs in Abbildung
First, the Client initiates an OAuth Authorization Request to the service-specific AS (1).
Instead of using service account credentials, the RO chooses to log in with his SSO identity via OIDC.
To do this, the AS acts as a RP and initiates an OIDC Authentication Request to the OP (2).
The EU is then authenticated by the OP with his credentials (3) and consents to the OP providing the RP with access to his profile information (4).
Technically, this consent is an authorization in the sense of OAuth 2.0.
The OP then responds with an IDT to the RP (5), which authenticates the EU to the AS and completes the OIDC-based authentication process.
Now the authenticated RO authorizes the requested Scopes of the Client (6).
Finally, the AS issues an AT and an optional RT to the Client (7).

% Benutzer muss jetzt auch OP vertrauen, welcher dadurch mächtiger wird als AS:
In an SSO environment, the trust relationship changes slightly.
While the user has to trust the AS not to impersonate his service account, he also has to trust the OP not to impersonate any of his service accounts.
This makes the OP very powerful because it could impersonate any of the user's service accounts.
Therefore, EUs should only choose trusted OPs.
