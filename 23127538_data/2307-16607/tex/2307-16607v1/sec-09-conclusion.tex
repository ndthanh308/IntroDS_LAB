%-------------------------------------------------------------------------------
\section{Conclusion and Future Work}
\label{sec:conclusion}
%-------------------------------------------------------------------------------
This paper introduced Open Identity Certification with OpenID Connect (OIDC²), which allows End-Users (EUs) to request a verifiable Identity Certification Token (ICT) from an OpenID Provider (OP).
An ICT contains claims about an EU and a public key chosen by the EU.
Authenticating Users can authenticate EUs with an ICT if they trust his issuing OP.
%
We compared OIDC² to existing end-to-end authentication methods and found that OIDC² is easier to use and improves security by eliminating the need for revocation lists.
We suggested how OIDC² can be implemented based on the OIDC framework.
We discussed security considerations for and general improvements with OIDC²: the trust relationship among its entities, a classification of OPs and their utilization with OIDC², authentication with multiple ICTs to increase the likelihood of successful authentication, as well as appropriate (short) validity periods for ICTs.
%
Furthermore, we proposed how OIDC² can be used for simple and user-friendly end-to-end authentication for video conferences, email, and instant messaging.
Finally, we provided a simple, open-source extension for OIDC server software to support OIDC² for testing purposes.
We proved its compatibility with Authentik and Keycloak and the performance of the new \texttt{/ict} Endpoints is comparable to or better than the performance of the existing \texttt{/token} Endpoints.

To demonstrate the feasibility of OIDC² for end-to-end authentication, we plan to integrate OIDC² for video conferences based on the open WebRTC protocol, for instant messaging based on the open Matrix protocol, and for email communication based on the PGP standard.
