%-------------------------------------------------------------------------------
\section{OIDC²: Open Identity Certification with OIDC}
\label{sec:oidc2}
%-------------------------------------------------------------------------------
% Überblick über Section:
This section describes the OIDC² concept in more detail and proposes a simple OIDC protocol extension to support it.

\subsection{Concept of OIDC²}
\label{sec:oidc2_concept}
%-----------------------------------
% Überblick wie wir OIDC² erklären:
We define new terminology, introduce the Identity Certification Token (ICT), and explain how to use it.

\subsubsection{Terminology}
\label{sec:oidc2_terminology}
% Einführung Terminologie EU, Client, AU, AP, OP und Keys:
Consider a user of one application authenticating to a user of another application.
The user authenticating himself is called the End-User (EU), his application is called the Client.
The other user is called the Authenticating User (AU), and his application is called the Authenticating Party (AP).
% Erklärung der Herkunft der Terminologie; Klarstellung, dass dies nicht heißt, dass OAuth und OIDC für OIDC² erforderlich sind:
We also assume that the EU has an SSO identity provided by an OpenID Provider (OP) trusted by the AU.
The terminology used for the EU, Client, and OP is consistent with the combined OAuth 2.0 and OIDC scenario described in \sect{sso}.
However, OIDC² does not require this scenario.

\subsubsection{Identity Certification Token (ICT)}
\label{sec:ict}
% Einführung ICT und abgrenzung von ID Token:
We introduce the ICT, which addresses the end-to-end authentication use case.
The ICT contains the Client's verified public key $K^+_C$, an application-specific Scope, an expiration date, and a unique identifier of the EU's SSO identity.
It may also contain other claims about the user which are not necessarily verified by the OP.

\subsection{ICT Request}
\label{sec:obtain_ict}
%-----------------------------------
% Wann ein ICT abgerufen wird; Verweis auf Grafik:
The Client uniquely requests an ICT from the OP for each end-to-end (e2e) authentication process.
\fig{oidc2_ict_request} simplifies the ICT request.
% Figure environment removed

% Abruf von Access Token:
First, the Client performs an OAuth 2.0 Authorization Request as described in \sect{oauth2} (1-4) to obtain an Access Token (AT) for the ICT Request.
For this purpose, the AT requires a Scope sufficient to access the EU's profile information, e.g., \texttt{profile}, and an e2e Scope, e.g., \texttt{e2e\_auth\_email}.
% Abruf von Identity Certification Token; Einführung Client + OP Key Pairs:
The Client then uses the AT to authorize an OAuth 2.0 Resource Request for an ICT (5) from the OP, called an ICT Request.
For this purpose, the Client uniquely generates a new public key $K^+_C$ and presents it to the OP.
The Client also presents a Proof of Possession (PoP) of the corresponding private key $K^-_C$, e.g., by signing a unique nonce.
The OP verifies the validity of the AT and the PoP (6).
If valid, the OP signs the ICT with its private key $K^-_{OP}$ corresponding to its published public key $K^+_{OP}$ and responds with the ICT (7).
When the ICT expires and a new ICT is required, the Client repeats steps (5) to (7) to request a new ICT for a new key pair.

\subsection{E2E Authentication with ICT}
\label{sec:ict_usage}
%-----------------------------------
% Einführung; Verweis auf Grafik:
The Client uses the ICT to authenticate its EU to the AP's AU as shown in \fig{oidc2_e2ea}.
% Transfer von ICT; Erklärung PoP Möglichkeiten:
First, the Client passes the ICT containing its public key $K^+_C$ to the AP and provides a PoP for the corresponding private key $K^-_C$ (1).
To do this, the Client signs either a unique nonce provided by the AP or a unique session-specific identifier.
Alternatively, the Client can prove the possession by establishing a secure channel based on the private key $K^-_C$.
In \sect{applications}, we show and explain use cases that take advantage of these three options.
% Verifikation von ICT und PoP:
The AP then verifies the Client's PoP (2) using the public key $K^+_C$ from the ICT and verifies the AU's trust relationship with the OP (3).
This may require user interaction or the use of whitelists, discussed further in \sect{trust_relationship}.
If the AU trusts the OP, the AP checks the expiration date and verifies the signature of the ICT using the OP's public key $K^+_{OP}$ (4).
If successful, the EU has proven its SSO identity to the AU (5).
