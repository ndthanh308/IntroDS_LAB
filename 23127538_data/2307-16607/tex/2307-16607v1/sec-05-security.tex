%-------------------------------------------------------------------------------
\section{Security Considerations}
\label{sec:security}
%-------------------------------------------------------------------------------
% Überblick über Section:
First, we discuss how OIDC² shifts the burden of thorough authentication from service providers to identity providers.
Then, we analyze the trust relationship between OIDC² entities and propose a trust classification for OPs.
Finally, we propose authentication with multiple ICTs and discuss the correlation between the validity of an ICT and its corresponding key pair.

\subsection{Service Provider vs. OpenID Provider}
\label{sec:sp_vs_op}
%-----------------------------------
% Vorteile für Nutzer (e2e authentication); Vorteile für service provider (kaum Implementierungsaufwand, kein OP Betrieb nötig):
In most communication services, users must rely on the identity claims of their communication partners provided by the service provider, with no way to verify them.
OIDC² allows users to verify each other's identities without having to trust the service provider.
This only requires the Client to implement OIDC² and the protocol to provide a way to exchange the ICTs.
The service provider does not need to implement OAuth 2.0 for the Client or provide an OP.
This improves the overall security of the service and prevents privacy issues by eliminating the need for the service provider to collect sensitive information about its users.

\subsection{Trust Relationship}
\label{sec:oidc2_trust_relationship}
%-----------------------------------
% Verweis auf Abbildung:
\fig{oidc2_trust_relationship} shows an overview of the trust relationship between the entities of the OIDC² protocol.
% Figure environment removed

% Vertrauen auf Proving Side: EU -> OP; EU -> Client:
On the proving side, the End-User (EU) trusts his OP to protect his identity from impersonation attacks and not to impersonate him.
This includes that the OP will only issue authorized ICTs.
Furthermore, the EU trusts that his Client will operate as intended.
This means that the Client will protect its private key $K^-_C$ from third parties and use the ICT only for the intended authentication processes.
To limit potential misuse by the Client, the ICT is scoped to a specific context.
For example, this prevents an email client from misusing the ICT to sign contracts on behalf of the EU.

% Vertrauen auf Authenticating Side: AU -> OP; AU -> AP:
On the authentication side, the Authenticating User (AU) trusts the OP to protect the EU's identity and to sufficiently verify the Client's possession of its private key $K^-_C$.
The AU also trusts the OP to certify sufficiently trustworthy identity claims with the issued ICT, which we will discuss in more detail in \sect{classification}.
To ensure that the authentication process is intended by the EU, the AU trusts the OP to issue only EU-authorized ICTs.
The AU must also trust his Authenticating Party (AP) to correctly verify the received ICT and PoP.

% Lösungsansätze, wie Vertrauen zwischen AU und OP hergestellt werden kann:
The AU needs to trust the OP.
We offer two solutions that can be combined.
First, the AP trusts a trusted identity federation such as the Global Assured Identity Network (GAIN) \cite{GainWhitepaper}, which consists of international OPs such as banks, insurance companies, or government institutions, all of which manage fully verified real-world identities.
Second, the AU maintains his own whitelist of OPs, such as social media platforms or his business partners.
Not every OP has the same level of trustworthiness, so we classify them in the next section.

\subsection{Classification of OpenID Providers}
\label{sec:classification}
%-----------------------------------
When working with OIDC², we suggest three classes of OPs to consider.

\subsubsection{Insecure OpenID Providers}
\label{sec:insecure_op}
% Beschreibung von nicht vertrauenswürdigem OP:
OPs can be considered insecure for a variety of reasons.
They may not be able to sufficiently protect their users' credentials, or they may be untrustworthy for political or economic reasons.
For example, they may certify potentially false or insufficiently verified claims.
If an AU considers an OP insecure, his AP will not accept any ICTs issued by that OP.

\subsubsection{Authoritative OpenID Providers (AOP)}
\label{sec:authoritative_op}
% AOP schützt accounts hinreichend und ist Autorität für bestimmte claims:
We classify an OP as an Authoritative OP (AOP) for specific claims, if the AU accepts the OP as an authority for those claims and trusts the OP to protect managed identities.
For example, an email server's OP is authoritative for email addresses within its own domain.
Because an OP issues a unique subject identifier for each SSO identity by specification, an OP is always authoritative for its associated \texttt{sub} claim.

% Beispiel und Limitations:
This makes AOPs sufficient for scenarios where an EU wants to be authenticated with specific claims.
For example, if the AU knows the EU's email address, the EU uses an ICT issued by his email provider's OP to authenticate on a social media platform.
However, AOPs are only sufficient to certify identity claims for which they are an authority.
To certify real-world identity claims such as names or addresses, the AOP must typically be the OP of a trusted government organization.

\subsubsection{Verifying OpenID Providers (VOP)}
\label{sec:verifying_op}
% Ein VOP schützt seine Accounts hinreichen und verifiziert bestimmte claims hinreichend:
There is not always an AOP for every claim the EU wants to be authenticated with.
Instead, the EU can use a third-party service that the AU trusts to sufficiently verify his identity claims and protect his account.
We call the OP of this third-party service a Verifying OpenID Provider (VOP).
This VOP could check the EU's passport to verify his name, or send him a verification code via SMS to verify his phone number.

% Beispiele, Limitations und Hybrid-Modelle:
There are already OpenID Providers such as banks or insurance companies that are required by law to verify their customers' claims.
However, such verification processes are often costly, which is why VOPs often do not verify all claims or offer it as an optional service, such as the social media platforms Facebook and Twitter.
Both can be AOPs at the same time.
For example, banks are VOPs for the name of an EU, but also AOPs for bank account numbers.

\subsection{Authentication with Multiple ICTs}
\label{sec:multi_ict}
%-----------------------------------
% Diskussion über Vor- und Nachteile der Authentifizierung mit mehreren ICTs.
The classification of an OP is up to the AU, i.e., the AU may not accept ICTs from certain OPs.
Since an EU may not know the AU's classification in advance, the EU can present ICTs from different OPs and the corresponding PoPs to increase the likelihood of successful authentication by the AU.
However, this requires more work for the EU as he has to log in to all these OPs to receive ICTs.
If the AP receives multiple ICTs, it presents them to the AU, which then selects the most trusted issuer or rejects them all.
Furthermore, the EU must be aware that presenting multiple ICTs also exposes all his presented accounts to the AU.

\subsection{Validity of ICTs and Client Key Pairs}
\label{sec:ict_validity_period}
%-----------------------------------
% Einführung, wie das ICT mit dem Key Pair des Clients zusammenhängt:
An ICT contains the public key $K^+_C$ of the Client.
By issuing this ICT, the OP certifies that the corresponding EU authorized the Client for e2e authentication with the contained identity claims.
% By issuing an ICT, the OP certifies that the Client whose public key $K^+_C$ the ICT contains is authorized by the EU to perform e2e authentication with the contained identity claims.
An attacker trying to impersonate the EU needs the corresponding private key of the ICT.
% Whoever proves to be in possession of the corresponding verified private key $K^-_C$ is also authorized -- even attackers.
% Einschränkung von ICT durch validity period und scope:
We minimize this potential misuse of the ICT by a leaked private key by making the ICT short-lived and limited in scope.
Since a few minutes are sufficient for most use cases (see \sect{applications}), we recommend setting the ICT validity period to no more than 1 hour.

% ICT Validity = Key Pair Validity:
We propose that an ephemeral and unique key pair $K^\pm_C$ expires along with its associated ICT, eliminating the need for key revocation mechanisms.
% Ausnahme ist long-term key pair; Erfordert Verweis auf key revocation list:
However, Sections \ref{sec:im} and \ref{sec:email} show that long-term key pairs are useful in some cases.
Therefore, we further propose that an ICT may also contain a long-term public key, which must be indicated by providing the key revocation server of the key.
% For security reasons, long-term key pairs must be distinguishable from short-term key pairs.
% Therefore, we suggest that this information either comes from the application context or a link to the key revocation server must be provided in the ICT.
Such a key is valid until revoked and is associated with the claims in the ICT.
Some services control the lifetime of public keys by associating them with user profiles.
An example of this approach is the Signal protocol (see \sect{im}).
In such applications, a user can be authenticated with a public key received from an ICT as long as the public key contained in it is associated with the profile.
% Either way, the AP must verify the expiration of a long-term key pair when verifying the ICT.
% Session darf auch nach key expiry gültig bleiben:
In any case, an active session can remain valid even after the underlying key pair $K^\pm_C$ expires (see \sect{vc}).

% Einführung, wie das ICT mit dem Key Pair des Clients zusammenhängt:
% An ICT certifies the validity of its contained public key $K^+_C$ for the identity of the EU indicated by the contained identity claims.
% This assures APs trusting the OP that the Client of the EU has the corresponding private key $K^-_C$.
% Since the ICT can be assumed to be public, anyone in possession of the private key $K^-_C$ can be authenticated as the EU of the ICT -- even attackers who have obtained that private key.
% Erklärung, wie kurzlebig ICT ist und dass Gültigkeit auf key pair übertragbar ist:
% To minimize the risk of impersonation attacks and to eliminate the need for ICT revocation mechanisms, the Client requests a unique ICT for each authentication process, with a very short validity period, just long enough to perform the PoP.
% While our findings in \sect{applications} show that a validity period of 5 minutes often works well, we recommend a maximum of 1 hour to avoid time synchronization problems.
% Strictly speaking, this validity period only applies to the ICT.
% However, if the corresponding key pair $K^\pm_C$ is ephemeral and uniquely generated for the ICT, the validity period is transferred to it.
% This means that when the ICT expires, so does the key pair, eliminating the need for key revocation mechanisms.

% Zusammenhang zwischen Gültigketi der Session und des ICTs/Key Pairs:
% This does not necessarily mean that the session associated with an expired key pair will also expire.
% For example, in \sect{vc}, the ICT must be valid when a videoconferencing session is authenticated.
% However, the session remains valid even after the ICT expires because compromising the key pair after session authentication does not compromise the session itself.
% Auch Long-term Key Pairs dürfen verwendet werden, aber mit Vorsicht:
% Since some applications require EU authentication with a long-lived key pair, we want to cover this use case as well.
% However, we do not recommend it because it requires a key-specific revocation mechanism.
% If unavoidable, the public key $K^+_C$ of an ICT must be explicitly marked as long-lived and the AP must verify its validity.
