%-------------------------------------------------------------------------------
\section{Related Technologies}
\label{sec:related_technologies}
%-------------------------------------------------------------------------------
% Überblick über diese Section:
We review related technologies for \ac{e2e} authentication and compare them to \ac{oidc²}.

\subsection{Identity Providers and Certificates}
\label{sec:idp}
%-----------------------------------
% Einführung PKI, X.509, S/MIME und Nachteile:
In a \ac{pki} \cite{Weise2001}, a \ac{ca} verifies that an entity's real-world identity and long-term public key $K^+_E$ belong together, records them in a document, signs it, and issues it in the common X.509 certificate format (\rfc{5280}).
Such X.509 certificates are used, e.g.\jp{,} in the \ac{smime} standard (\rfc{8551}) to authenticate and encrypt email.
However, \jp{the identity verification is cumbersome and users have to manage their own certificate files, an authentication concept that many people are still unfamiliar with, which is why} only $2.50 \%$ of over $81$ million emails examined in a study \cite{Stransky2022} were signed with \ac{smime}.

% Einführung SAML2-basierter Cisco Draft:
To simplify this process, Cisco proposed an expired Internet draft \cite{I-D.biggs-acme-sso} where an \ac{idp} issues X.509 certificates to its users.
According to their white paper \cite{WebexWhitepaper}, \jp{these certificates are used for challenge/response-based \cite{Kushwaha2021}} \ac{e2e} user authentication in the Webex video conferencing service \jp{where session partners trust each other's \ac{op}}.
The draft \cite{I-D.biggs-acme-sso} is designed for the \ac{saml2} authentication standard \cite{saml2core}, but \ac{oidc} performs better for mobile devices and cloud computing \cite{Naik2016}.
This may be one reason why the design has not been adopted by other applications and \acp{idp}.
% Abgrenzung von OIDC² zu Cisco draft:
Conceptually, the presented approach is similar to \ac{oidc²}; we continue with the differences.
X.509 is a binary format limited to a small set of standardized identity-related fields \cite{RFC8551}\jp{, while \ac{oidc²} uses the more flexible JSON-based \ac{jwt} (\rfc{7519}) format} to represent claims about the user.
\jp{These claims are signed by the \ac{idp} and their formats are standardized in the \ac{oidc} Core \cite{OidcCore} and eKYC \cite{OidcEkyc} specifications.}
\jp{In addition, long-lived user certificates require key revocation mechanisms and error-prone manual key management by the user, whereas short-lived \acp{ict} are issued spontaneously and only require the user to log in.}
\jp{This reduces security issues and improves the user experience}.

\subsection{Self-Sovereign Identity (SSI)}
\label{sec:ssi}
%-----------------------------------
% Einführung in SSI; Erklärung der Verwendung und Speicherung von public/private key:
In \ac{ssi} \cite{Muehle2018}, participating entities generate their own asymmetric key pairs $K^\pm$.
Entities are identified by their \ac{did}, which is linked to at least one public key $K^+$.
Entities store their private key $K^-$ in their digital wallet, e.g.\jp{,} an app on their smartphone.
This can be used for \ac{e2e} authentication with the key pair $K^\pm$.

% Begriffsdefinition Holder, Issuer, Verifier, Verifiable Credential, Verifiable Presentation:
\ac{ssi} describes three entities: the \ac{issuer}, the \ac{holder}, and the \ac{verifier}.
The issuer knows or verifies the \jp{holder's credentials} and issues them to the holder as a \ac{vc}.
This \ac{vc} is signed by the issuer with his private key $K^-_{\ac{issuer}}$; it contains the issuer's ${DID}_{\ac{issuer}}$ and the credentials and ${DID}_{\ac{holder}}$ of the holder.
The holder holds this \ac{vc} in his wallet and presents it to a verifier as a \ac{vp}.
This \ac{vp} is signed by the holder with his private key $K^-_{\ac{holder}}$; it contains the \ac{vc} and the verifier's ${DID}_{\ac{verifier}}$.
The verifier verifies this \ac{vp} by checking the issuer's signature on the \ac{vp} and the issuer's signature on the \ac{vc}.
If the verifier accepts the issuer as a trusted authority for issuing the holder's credentials, then the verifier trusts that these credentials belong to the holder.

% Implementierungen mit Blockchain und OIDC; Beispiel US Mobile Driving License:
Early implementations of \ac{ssi} made use of blockchain technology \cite{Ferdous2019} and used a \jp{distributed public} ledger \cite{Ioini2018} to store the mapping of a {\ac{did}} to its associated public keys.
Modern approaches are based on OAuth 2.0 and \ac{oidc}, such as the mobile driving license in the United States standardized in ISO/IEC 18013-5:2021 \cite{ISO18013-5}.
This approach implements the \ac{siopv2} \cite{siopv2} draft in the wallet app for key management.
Driving license offices provide OAuth 2.0 based interfaces defined in the \ac{openid4vci} draft \cite{openid4vci} to issue driving licenses as \acp{vc} in the \ac{w3c} format \cite{w3cVc}.
Drivers present these \acp{vc} as \acp{vp} to police officers using OAuth 2.0 based interfaces between smartphones defined in the \ac{openid4vp} draft \cite{openid4vp}.
Another \ac{oidc} draft describes the issuance of identity claims of the \ac{idt} as a \ac{vc} \cite{userinfoVc}.
This is similar to our approach, but requires the full \ac{openid4vci} infrastructure to be deployed, which is currently rare.

% Probleme an SSI Lösungen: Long-term keys, die revokierbar sein müssen oder nicht übertragen werden können.
\jp{Although \ac{ssi} is now being adopted for some government use cases, there are still issues with usability \cite{Sartor2022}\cite{Zaeem2021} and identity recovery \cite{Zhou2019}.}
\jp{These stem from manual key management by users who are unaware of their responsibilities, and the entirely new concept of operation.}
Since the private key is a long-term key that could be leaked during its lifetime, the system requires a key revocation list.
But as argued by Ronald L. Rivest more \jp{than} two decades ago \cite{Rivest1998}, revocation lists should be avoided for security reasons.
Modern technologies such as \ac{hsm} or \ac{tpm} address this problem by protecting the private key inside the hardware.
Here, the private key cannot be exported and can only be used for signing after the platform integrity has been verified and the user has been authenticated.
This creates problems when a user wants to use \acp{vc} from other devices.
\jp{Additionally}, if the device is lost or broken, the user needs a recovery method for the private key and \ac{did} that must be configured in advance.

% Was OIDC² besser macht:
\ac{oidc²} does not have these problems.
It uses short-lived ephemeral key pairs and \acp{ict} \jp{that do not require a} specific hardware or software platform.
\jp{It also leverages existing account recovery capabilities and the familiar sign-in user authentication concept}.
Compared to \ac{ssi} approaches, it does not require currently rarely deployed frameworks such as installed wallet apps, issued \acp{vc}, and a huge amount of implemented new standards.
Instead, \ac{oidc²} requires a small extension of \acp{op} to use existing \ac{oidc} accounts.
% All it takes is an existing OIDC-based user account and implemented OAuth 2 and OIDC features plus our tiny extension.
In contrast, the \ac{ict} may also contain claims that the issuing \ac{op} is not a trusted source of, which will be discovered in \sect{trust_relationship}.

% \vspace{-0.2cm}
\subsection{OpenPubkey}
\label{sec:openpk}
% \vspace{-0.1cm}
%-----------------------------------
% Vorstellung von OpenPubkey:
BastionZero has developed OpenPubkey \cite{openpubkey} which is very similar to \ac{oidc²}.
The \ac{rp} of an \ac{eu} can create a \ac{cic} that contains, among others, the \ac{rp}'s public key $K^+_C$.
When requesting an \ac{idt} (see \fig{oidc_authentication_flow}), the \ac{rp} can optionally provide a nonce in the Authentication Request (1), which we omitted in \sect{oidc}.
The \ac{op} will then insert this nonce into the \ac{idt} before issuing it (4).
With OpenPubkey, the \ac{rp} offers its hashed \ac{cic} as a nonce to be inserted into the \ac{idt}.
After receiving the \ac{idt}, the \ac{rp} appends the \ac{cic} and signs it with its private key $K^-_C$, resulting in a \ac{pk} Token.
\jp{The \ac{rp} can use this \ac{pk} Token to be authenticated with the  \ac{eu}'s identity.}

However, from our point of view, this approach \jp{has some security-relevant drawbacks.}
In an \ac{sso} context, the \ac{rp} is often a login service (see the \ac{as} in \fig{oidc_sso}) that the \ac{eu} usually authorizes to access his profile information.
\jp{This \ac{rp} may act malicious and} request a \ac{pk} Token with its own public key $K^+_C$ to impersonate the \ac{eu} without his knowledge.
The authors' solution to this problem is to have the authenticating user only accept \ac{e2e} authentications \jp{from trusted \acp{rp}}, identified by \jp{their} Client ID contained in the \ac{pk} Token.
First, this \jp{induces} a high burden on the user, which is unacceptable since it is difficult for the user to identify trusted \acp{rp}.
Second, the \ac{eu}'s trust in a service, such as an online store, may be sufficient to be authenticated by that store, but it may not be sufficient to allow the store to impersonate him.
Third, in open communication systems such as email, there are many clients, and it is unlikely that all of them are trusted.
This limits the use of OpenPubkey to a small set of explicitly trusted services and clients.
We believe that these three problems \jp{may cause security vulnerabilities in the future}.
In contrast, with \ac{oidc²}, the \ac{eu} does not risk being impersonated when logging in to a malicious service.

% Was OIDC² besser macht:
\ac{oidc²} solves this problem by introducing \jp{the new \ac{ict}} that can only be requested by an \ac{rp} with sufficient scope for \jp{a specific} \ac{e2e} authentication \jp{context}.
This means that an \ac{eu} can control whether to issue only an \ac{idt} or also an \ac{ict}.
