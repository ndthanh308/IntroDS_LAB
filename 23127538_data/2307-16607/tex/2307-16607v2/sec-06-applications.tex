%-------------------------------------------------------------------------------
% \vspace{-0.2cm}
\section{Use of \ac{oidc²} in Applications}
\label{sec:applications}
% \vspace{-0.2cm}
%-------------------------------------------------------------------------------
% Überblick über diese Section:
\jp{We explain how \ac{e2e} authentication is currently implemented in video conferencing, \ac{im}\jp{,} and email, and how it can be improved by \ac{oidc²}.}
\jp{The use of \ac{oidc²} is not limited to these applications, they were just selected to illustrate different application patterns.}
In addition, we recommend validity periods for \acp{ict} depending on these applications.

% \vspace{-0.2cm}
\subsection{Video Conferencing}
\label{sec:vc}
% \vspace{-0.1cm}
%-----------------------------------
% Überblick über die Section:
\jp{Most video conferencing systems do not utilize any \ac{e2e} authentication.}
\jp{Instead,} users must rely on the identities of their communication partners provided by the service's \ac{idp}.
\jp{Identifying a communication partner just by its video does not suffice.}
\jp{New technologies like deep fakes make this unreliable as demonstrated by an incident in 2022 \cite{Oltermann2022}.}
We explain how video conferencing services use OAuth 2.0 and \ac{oidc} and how they can benefit from \ac{oidc²}.

% \vspace{-0.2cm}
\subsubsection{\jp{\ac{e2e}} Authentication in Video Conferencing}
% Einführung wie Authentifizierung in VCs mit OAuth 2.0 und OIDC funktioniert:
In video conferencing, users log in to the service provider's OAuth 2.0 \ac{as} either directly with their credentials\jp{,} or through the \ac{op} with their \ac{oidc} identity, as explained in \sect{sso}.
After authentication, the video conferencing service provider's \ac{as} \jp{obtains} an \ac{idt} from the \ac{op}.
After authorization, the Client gets an \ac{at} from the \ac{as}.
The Client \jp{proves} its authorization to the video conferencing server \jp{with the \ac{at}}.
\jp{The video conferencing server retrieves the \ac{eu}'s service account ID from the \ac{at} and provides the corresponding service profile to the communication partner.}
\jp{Note that the \ac{eu}'s service profile may be different from his \ac{oidc} identity.}

% Einführung in die E2EE und Vertrauensproblematik:
\jp{In addition, the clients of both communication partners generate an ephemeral asymmetric key pair.}
\jp{They sign their public keys and key negotiation messages and exchange them via the video conferencing server.}
\jp{This enables an \ac{e2e} encrypted communication channel, but users cannot rely on their communication partner’s identity for two reasons.}
\jp{First, the service profile may not reflect the partner's real-world identity.}
\jp{Second, the service provider may provide a fake profile.}

% \vspace{-0.2cm}
\subsubsection{\jp{\ac{e2e}} Authentication with \ac{oidc²}}
% Grobe übersicht; Verweis auf Grafik:
We propose that the \ac{eu} \jp{is authenticated by his communication partner (\ac{au}) with his \ac{ict} that the \ac{eu}'s Client requested from the \ac{op}}.
After a mutual \ac{ict} exchange, the Client and the \jp{\ac{ap}} use the contained verified public keys to establish a secure channel, as shown in \fig{oidc2_vc}.

% Figure environment removed

% Verbindungsanfrage Client -> AP:
First, Client A generates an ephemeral key pair $K^\pm_A$ and \jp{requests an \ac{ict} for its public key $K^+_A$ from the \ac{eu}'s \ac{op}} (1).
\jp{The Client signs this \ac{ict} and some unique session context, e.g., including timestamps and ephemeral public keys, with its private key $K^-_A$.}
\jp{The latter serves as a \ac{pop}.}
\jp{The Client sends the \ac{ict} and the \ac{pop}} to the \ac{ap} via the video conferencing server (2).
% Verbindungsannahme AP -> Client:
If the \ac{au} trusts the \ac{eu}'s \ac{op}, \jp{it validates the \ac{ict} and verifies the \ac{pop}.}
\jp{Then} Client B generates its own ephemeral key pair $K^\pm_B$ (3) and requests an \ac{ict} \jp{for its public key $K^+_B$} from the \ac{au}'s \ac{op} (4).
The \ac{ap} signs its \ac{ict} and the \jp{unique session context with its private key $K^-_B$ (\ac{pop})} and responds to \jp{Client A} via the video conferencing server (5).
% Verbindungsaufbau:
If the \ac{eu} trusts the \ac{au}'s \ac{op} (\jp{6}) \jp{an the validation of the \ac{ict} and the \ac{pop} are successful}, then the Client and \jp{the} \ac{ap} have successfully performed mutual authentication\jp{, which enables} them to establish a \jp{securely \ac{e2e} authenticated and encrypted channel} (7).

% \vspace{-0.4cm}
\subsubsection{Discussion}
% Größter Vorteil von OIDC².
\jp{When video conferencing is improved with \ac{oidc²}, users do no longer need to trust their service provider to display the correct service profiles of the participants in a video session.}
\jp{While the service profiles may not reflect the true identities of the participants, the users can rely that verifying \acp{op} have thoroughly checked the participants' identity when they registered.}
% Validity Period of ICT:
\jp{We recommend a validity period of 5 minutes for video conferencing services as starting} a video conference takes only a few seconds.
\jp{Within that time a user has joined a call, and that duration is also long enough to compensate for clock drifts on end systems.}
When the \ac{ict} expires, an active secure channel remains valid until it is closed.

% \vspace{-0.3cm}
\subsection{\jp{\acf{im}}}
\label{sec:im}
% \vspace{-0.1cm}
%-----------------------------------
% Section-Übersicht:
\jp{Secure \ac{e2e} authentication in \ac{im} requires a prior secure exchange of public keys which is mostly achieved by face-to-face meetings.}
%\jp{We suggest how the Signal \cite{signal} \ac{im} service could benefit from \ac{oidc²}, eliminating the need for face-to-face meetings.}
\jp{We suggest an \ac{oidc²}-based authentication method for \ac{im} which does not require such a prior secure exchange of public keys.}

% \vspace{-0.4cm}
\subsubsection{\jp{\ac{e2e}} Authentication in Signal}
% Einführung in Ende-zu-Ende Authentifizierung in Signal und In-Person Verifikation mit QR Codes:
In the Signal protocol \cite{WhatsappWhitepaper}, users are identified by their phone number and public key $K^+$\jp{.}
\jp{Both phone number and public key $K^+$ are verified and published by Signal while the corresponding long-term private key $K^-$ remains on the user's device.}
\jp{To establish an \ac{e2e} encrypted communication channel, a user requests from Signal the public key of the communication partner.}
\jp{The public keys of both users are leveraged for an authenticated Diffie-Hellman key exchange to establish an encrypted communication channel between the partner.}
\jp{Over this channel, they may authenticate each other with their public keys.}
\jp{However, at this point the public keys are not yet reliable as this method requires trust in Signal and its verification method for phone numbers and public keys.}
\jp{When the partners meet in presence, they can mutually verify each other's public keys by exchanging them via a secure side channel, e.g., by presenting them as a \ac{qr} code in a face-to-face meeting.}
%\jp{This manual process eliminates the need for trust in Signal.} 
\jp{When the partners communicate again, they can rely on the verified public keys and trust in Signal is no longer needed.}
\jp{This mechanism is a strong but also cumbersome.}

% \vspace{-0.4cm}
\subsubsection{\jp{\ac{e2e}} Authentication with \ac{oidc²}}
% Einführung; Verweis auf Figure.
We propose an \ac{e2e} authentication method for \ac{im} based on \ac{oidc²}.
\jp{It is} illustrated in \fig{e2e_im_authentication}.
% Kurze Erklärung, wie Kanal mit OIDC² nachträglich authentifiziert wird:
\jp{We assume that the \ac{im} clients have already established an \ac{e2e} encrypted channel that is mutually authenticated by each other's public keys $K'^+$.}
\jp{The mutual authentication method based on \ac{oidc²} works as follows.}
\jp{The} \ac{im} \jp{client} requests an \ac{ict} from its \ac{eu}'s \ac{op} for his public key $K^+$ and sends the \ac{ict} over the secure channel to the \ac{ap}.
If the \ac{au} trusts the \ac{eu}'s \ac{op}, the \ac{ap} verifies the received \ac{ict} and compares the contained public key $K^+$ with the \jp{one that was used to establish the secure channel.}
\jp{The implicit \ac{pop} consists of the fact that both users can communicate over the secure channel, i.e., they possess the corresponding private keys.}

% Figure environment removed

% \vspace{-0.4cm}
\subsubsection{Discussion}
\jp{The presented approach shows that existing key management systems like the one of Signal can be extended with \ac{oidc²} as an authentication layer.}
\jp{It} does not use any Signal-specific features and can therefore be applied to any other \ac{im} service \jp{while preserving all security-related features such as ratchet-based \ac{e2e} encryption or forward secrecy.}
\jp{As a particularity, the \ac{ict}'s key is not ephemeral but an existing long-term key. Moreover, the method demonstrates that a secure channel may be set up with non-verified public keys with subsequent key verification based on \ac{oidc²}. Probably, this pattern can also be applied to other applications.}
% Validity Time:
\jp{We recommend a validity period of 5 minutes for \acp{ict} in an \ac{im} context because messages are delivered to the receiving client very quickly.}
\jp{If the \ac{ict} is transmitted when the \ac{ap} is offline, the verification process must be repeated.}

\subsection{Email}
\label{sec:email}
% \vspace{-0.2cm}
%-----------------------------------
% Übersicht über Section:
\jp{\ac{smime} and \ac{pgp} are common standards for email authentication for almost three decades.}
\jp{However, probably due to their complex key management, signed emails are still the exception \cite{Stransky2022} with $2.8 \%$.}
\jp{We briefly describe \ac{smime} and \ac{pgp} and their shortcomings, explain how \ac{pgp} can be enhanced with \ac{oidc²} for better authentication, and the limitations of that approach.}

% \vspace{-0.5cm}
\subsubsection{\jp{\ac{e2e} Authentication with \ac{smime} and \ac{pgp}}}
% Einführung Email-Signaturen:
\jp{\ac{smime} and \ac{pgp} utilize a long-term asymmetric key pair $K^\pm$ to sign emails.}
% Einführung S/MIME:
\jp{\ac{smime} leverages X.509 certificates issued by a \ac{ca} so that receivers of a signed email can validate its signature with the enclosed public key after validation of the public key and checking its associated \ac{crl}.}
\jp{Obtaining such a certificate may be a cumbersome and expensive process for the user unless it is provided by his employer.}
% Einführung PGP:
\jp{\ac{pgp} requires that the receiver of an email has obtained a fingerprint of the sender's public key via some side channel.}
\jp{Mostly, the communication partners have exchanged the fingerprints of their public keys in a face-to-face meeting.}
\jp{This is also a cumbersome process and does not allow for signed communication before having known the receiver of the email.}
\jp{To facilitate revocation of a key pair, the public key is published on a key server.}
\jp{The key server maintains a key revocation list which needs to be checked by the receiver of an email.}
% Thus, the receiver of an email must have met with the sender before, identify the sender by the received public key, and check the \ac{crl} on the corresponding key server.

% Einführung Email Verschlüsselung:
\jp{\ac{smime} and \ac{pgp} also support email encryption.}
\jp{To that end, the sender encrypts an email with a symmetric key, encrypts this key with the receiver's public key $K^+$, and attaches it to the email.}
\jp{The receiver uses his private key $K^-$ to decrypt the symmetric key for decryption of the email.}

% Zusammenfassung der Probleme:
\jp{Both \ac{smime} and \ac{pgp} suffer from the fact that they require complex key management.}
\jp{First, the user must protect his private key $K^-$.}
\jp{Second, the user must securely synchronize his key pair $K^\pm$ across his devices if he wants to use them all for signed and encrypted email communication.}
\jp{And third, the sender must revoke compromised key pairs $K^\pm$, and the receiver must check \acp{crl} to validate the validity of a received public key $K^+$.}

% \vspace{-0.4cm}
\subsubsection{\jp{\ac{e2e} Authentication with \jp{\ac{pgp} and} \ac{oidc²}}}
% Vorstellung von PGP + OIDC²:
\jp{We propose to combine \ac{pgp} with the key verification method of \ac{oidc²}, but do not touch any security properties of \ac{pgp}.}
\jp{We have implemented this concept in a prototype which is published on GitHub\footnote{\url{https://anonymous.4open.science/r/oidc2-demo/}}.}

% Signieren und Versenden der Email:
\jp{\fig{e2e_email_authentication} shows how emails are sent with \ac{pgp} and \ac{oidc²}.}
\jp{The sender (\ac{eu}) authorizes his email client (Client) for \ac{e2e} authentication in the \texttt{email} context.}
\jp{Then the Client requests an $\ac{ict}_{\ac{eu}}$ for its stored \ac{pgp} key $K^\pm_{\ac{pgp}}$, attaches both the public \ac{pgp} key $K^+_{\ac{pgp}}$ and the obtained $\ac{ict}_{\ac{eu}}$ to the email, and signs the email with the private \ac{pgp} key $K^-_{\ac{pgp}}$.}
% Empfangen und verifizieren der Email:
\jp{The receiver (\ac{au}) opens the email with his client (\ac{ap}).}
\jp{The \ac{ap} first validates the $\ac{ict}_{\ac{eu}}$, i.e., it checks whether it trusts the issuing \ac{op} of the \ac{ict} and it validates the \ac{ict}'s signature.}
\jp{This can be done offline as the public keys of trusted \acp{op} are typically cached for a moderate time.}
\jp{Then, the \ac{ap} verifies the email's integrity by verifying its signature with the \ac{ict}'s contained public key.}
% Erklärung von PoP:
\jp{The integrity of the mail serves as a \ac{pop} to prove that the sender of the email is also the owner of the key in the $\ac{ict}_{\ac{eu}}$.}
\jp{Now, the sender of the email is identified with the identity provided in the $\ac{ict}_{\ac{eu}}$.}

% Figure environment removed

% \vspace{-0.5cm}
\subsubsection{Discussion}
% Vorteil von OIDC²: Kein Secure Side Channel mehr notwendig
\jp{The email is sufficiently authenticated when it was validated before the expiration of the $\ac{ict}_{\ac{eu}}$.}
\jp{Then, checking a possibly available \ac{crl} of the sender's PGP key is not needed, as the validity of the key is already given by the $\ac{ict}_{\ac{eu}}$.}
\jp{When receiving signed emails with conventional \ac{pgp} or \ac{smime}, the key's \ac{crl} must be checked so that the \ac{crl} must be reachable.}
\jp{Conversely, \ac{oidc²}-based \ac{pgp} requires that the \ac{op} is reachable to issue the $\ac{ict}_{\ac{eu}}$ when sending the email.}

\jp{The \ac{pgp} key pair in the presented method can be either the sender's long-term \ac{pgp} key pair, or a short-term key pair just generated by the Client.}
% Diskussion zu long-term key pair:
\jp{We first assume that a long-term public \ac{pgp} key is attached to the email and the contained $\ac{ict}_{\ac{eu}}$ references a key server that is reliable in the sense that it adds information reliably to the \ac{crl}.}
\jp{The information about the public key on the key server equals the one in the $\ac{ict}_{\ac{eu}}$ and that this information predates the issuing time of the $\ac{ict}_{\ac{eu}}$.}
\jp{These preconditions seem comprehensive but are mostly fulfilled.}
\jp{We can argue that the \ac{op} confirmed the mapping of key and identity, which was equal to the information on the key server at that time, so that this information on the key server is also reliable as long as the key is valid, i.e., until it expires or it is revoked.}
\jp{Therefore we can retain two advantages of long-term keys as long as the key is valid.}
\jp{First, the message may be even validated after the $\ac{ict}_{\ac{eu}}$ expired.}
\jp{Second, the key may be stored as an authenticated public \ac{pgp} key of the sender.}
\jp{Thus, the proposed procedure may be used to securely exchange a long-term \ac{pgp} key, and this key may be used to send encrypted emails to the key's owner.}

% Diskussion zu short-term key pair:
\jp{We now consider the use of short-lived keys.}
\jp{They do not need to be stored so that no complex key management is required.}
\jp{This is unlike for long-term keys and greatly improves the usability of email authentication.}
\jp{However, short-lived keys come with two drawbacks.}
\jp{First, sending encrypted emails is not possible without long-term keys.}
\jp{Second, an email can be successfully verified only as long as the $\ac{ict}_{\ac{eu}}$ is valid, which may be too short when the email is received late by the \ac{ap}.}
\jp{This problem can be solved if the \ac{eu} can trust his mail server so that the \ac{ap} can rely on the inbox timestamp.}
\jp{Then an email can be considered valid if it was received within the validity period of the attached \ac{ict}.}
\jp{As most emails are received within a few minutes by the inbox on the mail server, we recommend an expiration date of at most one hour.}
