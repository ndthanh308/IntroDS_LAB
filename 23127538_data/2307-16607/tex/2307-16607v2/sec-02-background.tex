%-------------------------------------------------------------------------------
% \vspace{-0.2cm}
\section{Introduction to OAuth 2.0 and \ac{oidc}}
\label{sec:background}
% \vspace{-0.2cm}
%-------------------------------------------------------------------------------
% Überblick über Struktur der Section:
We introduce basics of OAuth 2.0 and \ac{oidc}, as they are the underlying technologies for \ac{oidc²}.
We discuss their trust relationship and explain how they facilitate \ac{sso}.
In the following, we capitalize technical terms from the official OAuth 2.0 and \ac{oidc} terminology.
For ease of understanding, we omit non-essential steps in the protocols and refer to the authoritative standards for details.

% \vspace{-0.2cm}
\subsection{OAuth 2.0}
\label{sec:oauth2}
% \vspace{-0.1cm}
%-----------------------------------
% Einführung OAuth; Begriffsdefinitionen:
The OAuth 2.0 authorization framework, as defined in \rfc{6749}, is based on \ac{http} (\rfc{7231}) and \jp{JSON (\rfc{8259})}.
It allows a user to grant his Client scoped access to his resources on a server, e.g., to only read emails.
A Client can be a web application, or a native email client application.
In OAuth 2.0, this server is called \ac{rs} because it protects the user's \acp{pr}; the user is called the \ac{ro}.

% Vorteil von OAuth gegenüber Username/Password Autorisierung; Einführung Access Token:
Without OAuth 2.0, the \ac{ro} would leave his credentials with his Client to log in directly to the \ac{rs}.
With OAuth 2.0, the \ac{ro} logs in to an \ac{as} and tells the \ac{as} to authorize the Client to access a scoped set of \acp{pr}.
To do this, the \ac{as} issues an \ac{at} to the Client.
This \ac{at} allows the Client to access the \acp{pr} on the \ac{rs}.
In this way, OAuth 2.0 improves the security by granting Clients only a scoped access to the user's account without exposing the user's credentials to any of the Clients.

% Wie das Access Token abgerufen wird:
\fig{oauth_authorization_flow} shows a simplified authorization request where the \ac{ro} authorizes his Client to read email.
% Figure environment removed
First, the Client requests access to the Scope \texttt{read\_email}, which authorizes read-only access to the \ac{ro}'s emails (1).
Then, the \ac{as} authenticates the \ac{ro} (2) and the \ac{ro} authorizes the Client for the requested Scope (3).
Finally, the \ac{as} issues the \ac{at} and optionally a \ac{rt} (4).
This \ac{at} contains the authorized Scopes with a short validity period.
It is signed with the \ac{as}'s private key $K^-_{\ac{as}}$.
The \ac{rt} is like a revocable session ID that authorizes the Client to refresh an expired \ac{at} without further user interaction.

% Wie das Access Token einegsetzt wird um Protected Resources abzurufen:
\fig{oauth_resource_request} describes a Resource Request where the Client uses the \ac{at} to access \acp{pr} on the \ac{rs}.
First, the Client requests the \acp{pr} and provides the \ac{at} to prove authorization (1).
Then, the \ac{rs} verifies the \ac{at} for a sufficient Scope, its expiration date, and the validity of its signature with the \ac{as}'s public key $K^+_{\ac{as}}$ (2).
Finally, the \ac{rs} responds with the \acp{pr} (3).

% \vspace{-0.2cm}
\subsection{OpenID Connect (OIDC)}
\label{sec:oidc}
% \vspace{-0.1cm}
%-----------------------------------
% Einführung OpenID Connect; Abgrenzung zu OAuth:
\ac{oidc} \cite{OidcCore} is an authentication framework that allows users to be authenticated with an \ac{sso} identity through a third-party service, such as an email service.
It extends OAuth 2.0 for this purpose.
Unlike the example in \sect{oauth2}, the \ac{sso} identity has no relationship to the third-party service.

% \vspace{-0.8cm}
% Figure environment removed

% Begriffsdefinitionen mit Abgrenzung zu OAuth:
In \ac{oidc}, an \ac{eu} is authenticated by an \ac{op}.
The \ac{eu} grants \ac{oidc}-specific Scopes to the EU's intended service, e.g., to his email client, which is called the \ac{rp}.
This communication flow is supported by OAuth 2.0, where the \ac{eu} corresponds to a \ac{ro}, the \ac{op} to an \ac{as}, and the \ac{rp} to a Client.
The \ac{op} issues an \ac{idt} to the \ac{rp}.
This \ac{idt} contains claims about the authentication event which typically includes information about the \ac{eu}, such as his name or address.
Since this Authorization Request is for authentication, it is called an Authentication Request in \ac{oidc}.
With this mechanism, an \ac{eu} can be authenticated with his \ac{sso} identity by different services without providing his credentials.
Instead, the \ac{op} passes profile information about the \ac{eu} to the \ac{rp} as identity claims in the \ac{idt}, but the \ac{eu} controls which information is passed.

% Wie das ID Token abgerufen wird:
\fig{oidc_authentication_flow} describes a simplified Authentication Request where the \ac{eu} is authenticated by the \ac{rp} via the \ac{op}.
First, the \ac{rp} requests access to the \texttt{profile} Scope.
If the \ac{eu} grants this Scope, the \ac{rp} is authorized to access the \ac{eu}'s profile information (1).
Then, the \ac{eu} is authenticated by the \ac{op} (2) and authorizes the \ac{rp} for the requested Scope (3).
Finally, the \ac{op} issues an \ac{idt} in addition to the \ac{at} and an optional \ac{rt} (4).
This \ac{idt} contains the identity claims related to the authorized \texttt{profile} Scope, such as the \ac{eu}'s name, profile picture, or date of birth, and is signed with the \ac{op}'s private key $K^-_{\ac{op}}$.
The \ac{rp} can verify the signature of the identity claims with the \ac{op}'s public key $K^+_{\ac{op}}$.

% \vspace{-0.2cm}
\subsection{Trust Relationship}
\label{sec:trust_relationship}
% \vspace{-0.1cm}
%-----------------------------------
% Einführung in Trust-Relationship; Referenz auf Abbildung:
The following describes the resulting trust relationship between entities in OAuth 2.0 and \ac{oidc} as shown in \fig{oidc_trust_relationship}.

% Trust-Verhältnis Client/RP <-> AS/OP:
The Client/\ac{rp} never sees any credentials from the \ac{ro}/\ac{eu} because the authentication process is performed solely by the \ac{as}/\ac{op}.
Therefore, the Client/\ac{rp} must rely on the \ac{as}/\ac{op} to correctly verify the identity of the \ac{ro}/\ac{eu} and that the \ac{as}/\ac{op} will issue the correct \ac{at}/\ac{idt} of the authenticated \ac{ro}/\ac{eu}.
% Trust-Verhältnis RO/EU <-> Client/RP:
Once the Client/\ac{rp} of the \ac{ro}/\ac{eu} receives the tokens, the \ac{ro}/\ac{eu} may not be able to verify what it is doing with them.
The Client/\ac{rp} may even leak the tokens, so the \ac{ro}/\ac{eu} must trust that it is working as intended.
To minimize this risk, the \ac{ro}/\ac{eu} restricts the Client/\ac{rp}'s access to only the necessary \acp{pr} and identity claims.
% Trust-Verhältnis AS/OP <-> RO/EU:
The \ac{ro}/\ac{eu} must also trust the \ac{as}/\ac{op} to protect his identity.
This includes a secure login process and secure credential storage, but also that the \ac{as}/\ac{op} will not impersonate his account.
Such impersonation would not even require any credentials since the \ac{as}/\ac{op} needs only its private key $K^-_{\ac{as}}$/$K^-_{\ac{op}}$ to sign an \ac{at}/\ac{idt}.

% \vspace{-0.2cm}
\subsection{Single Sign-On with OAuth 2.0 and OIDC}
\label{sec:sso}
% \vspace{-0.1cm}
%-----------------------------------
% Einführung in Single Sign-On; Referenz auf Abbildung:
Today, many services require dedicated accounts, forcing users to remember multiple service-specific credentials.
With \ac{sso} systems, users only need to remember the credentials for one account.
They can use this \ac{sso} identity to log in to multiple service accounts.
Logging in to a service account with this \ac{sso} identity is typically solved with a combination of OAuth 2.0 and \ac{oidc}, as depicted simplified in \fig{oidc_sso}.

% Beschreibung des OAuth Ablaufs in Abbildung
First, the Client initiates an OAuth Authorization Request to the service-specific \ac{as} (1).
Instead of using service account credentials, the \ac{ro} chooses to log in with his \ac{sso} identity via \ac{oidc}.
To do this, the \ac{as} acts as a \ac{rp} and initiates an \ac{oidc} Authentication Request to the \ac{op} (2).
The \ac{eu} is then authenticated by the \ac{op} with his credentials (3) and consents to the \ac{op} providing the \ac{rp} with access to his profile information (4).
Technically, this consent is an authorization in the sense of OAuth 2.0.
The \ac{op} then responds with an \ac{idt} to the \ac{rp} (5), which authenticates the \ac{eu} to the \ac{as} and completes the \ac{oidc}-based authentication process.
Now the authenticated \ac{ro} authorizes the requested Scopes of the Client (6).
Finally, the \ac{as} issues an \ac{at} and an optional \ac{rt} to the Client (7).

% Figure environment removed

% Benutzer muss jetzt auch OP vertrauen, welcher dadurch mächtiger wird als AS:
In an \ac{sso} environment, the trust relationship changes slightly.
While the user has to trust the \ac{as} not to impersonate his service account, he also has to trust the \ac{op} not to impersonate any of his service accounts.
This makes the \ac{op} very powerful because it could impersonate any of the user's service accounts.
Therefore, \acp{eu} should only choose trusted \acp{op}.
