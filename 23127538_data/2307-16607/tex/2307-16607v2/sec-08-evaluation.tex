%-------------------------------------------------------------------------------
% \vspace{-0.2cm}
\section{Evaluation}
\label{sec:evaluation}
% \vspace{-0.2cm}
%-------------------------------------------------------------------------------
We evaluate the performance of the provided \texttt{/ict} Endpoint, written in Go, compared to the \texttt{/token} Endpoint of the Keycloak~22.0.1 and Authentik~2023.6.1 \ac{oidc} server software \jp{to estimate the additional costs for an \ac{op}}.
% They are written in Java and Go, respectively.

We conduct the following two experiments.
(A) A Client sends a Refresh Token to the \texttt{/token} Endpoint of the \ac{oidc} server and obtains an \ac{idt}, an \ac{rt}, and an \ac{at}.
(B) A Client generates a \ac{pop}, sends an \ac{at} to the new \texttt{/ict} Endpoint, and obtains an \ac{ict}.
Both experiments are conducted over one minute, i.e., a token is requested, returned, and then the next request is sent.
We ran each experiment 20 times and computed mean requests per minute including confidence intervals with a confidence level of 95\% (${CI}_{0.95}$) using the Student's t-distribution.
We automate this process with the help of a web application\footnote{The application is programmed in Angular~15 and its code is available on GitHub \url{https://github.com/oidc2/benchmark}}. 

The \ac{oidc} server, its user database based on PostgreSQL~15.2, and the new \texttt{/ict} Endpoint run in separate Docker containers\footnote{\url{https://github.com/oidc2/op-extension/blob/main/docker-compose.yaml}}.
The host is a Lenovo ThinkPad T14s with \jp{a} 2.1 GHz AMD Ryzen 5 PRO 4650U processor, 16 GB RAM, and a 512 GB SSD with Windows~11 22H2 x64, and running the Docker engine\footnote{\url{https://www.docker.com/}}~24.0.2 in WSL~2\footnote{\url{https://learn.microsoft.com/en-us/windows/wsl/}}.
While Authentik can import and export any private keys, Keycloak cannot export private keys and it can import only RSA keys.
Therefore, we chose RS256 for signatures, i.e., a 2048 bit RSA key with the SHA-256 hashing algorithm to make experiments with different server software comparable.

With Keycloak, a mean request rate of 994.00 \acp{idt}\jp{/min} (A) (${CI}_{0.95}$: [992.97; 995.03]) and 988.20 \acp{ict}\jp{/min} (B) (${CI}_{0.95}$: [986.72; 989.68]) could be served\footnote{The values per experiment run are available here: \url{https://github.com/oidc2/benchmark/blob/main/results}.}.
In contrast, with Authentik, 190.95 \acp{idt}\jp{/min} (A) (${CI}_{0.95}$: [190.35; 191.35]) and 891.65 \acp{ict}\jp{/min} (B) (${CI}_{0.95}$: [886.04; 897.26]) could be served.
Thus, the tested version of Keycloak is more efficient than the tested version of Authentik.
Moreover, the provided \texttt{/ict} Endpoint is as efficient as the built-in \texttt{/token} Endpoint or even more efficient.

We compare the work done by the \texttt{/token} Endpoint and the \texttt{/ict} Endpoint. 
%
(A) The \texttt{/token} Endpoint validates the \ac{rt}, creates an \ac{idt}, and signs the \ac{at} and the \ac{idt} with its private key.
The integrity of the \ac{rt} is secured differently\footnote{Authentik uses a nonce for the \ac{rt} stored in the database while Keycloak secures the \ac{rt} with an HMAC.}.
%
(B) The \texttt{/ict} Endpoint validates the \ac{pop} for the Client's public key, and requests user information using an \ac{at} from the \texttt{/userinfo} Endpoint, which validates the \ac{at}.
Then the \texttt{/ict} Endpoint creates and signs the \ac{ict}.

The effort for creating and signing an \ac{idt} in (A) and an \ac{ict} in (B) is possibly similar, as both require \ac{rt}/\ac{at} validation, a database request, and a token signature.
Thus, creating an \ac{rt} and \ac{at}, and signing the \ac{at} in (A) is apparently equal or more time consuming than creating the \ac{pop} at the Client and validating the \ac{pop} at the \texttt{/ict} Endpoint in (B).

\jp{The cost of providing \acp{ict} scales with the frequency \acp{ict} are requested, which depends on the adoption of \ac{oidc²} by applications and by \acp{eu}.}
\jp{However, \acp{ict} are typically requested by \acp{eu}' Clients before sending an email, when being authenticated by a new \ac{im} communication partner, and when joining a video conferencing session.}
\jp{Such user-triggered actions may not exceed 10 \ac{ict} requests per hour in normal workloads.}
\jp{Inactive \acp{eu} do not request any \acp{ict}.}
\jp{In contrast, \acp{at} are recommended to be renewed every hour to comply with \rfc{6749}'s expiration time guidance.}
\jp{This is needed for every running Client, even if the \ac{eu} is inactive.}
\jp{Therefore, our predicted usage of \acp{ict} after full adoption is in a similar order of magnitude as the recommended need for \acp{at}.}
