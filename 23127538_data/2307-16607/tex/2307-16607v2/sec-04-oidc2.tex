%-------------------------------------------------------------------------------
\section{OIDC²: Open Identity Certification with \ac{oidc}}
\label{sec:oidc2}
% \vspace{-0.2cm}
%-------------------------------------------------------------------------------
% Überblick über Section:
This section describes the \ac{oidc²} concept in more detail and proposes a simple \ac{oidc} protocol extension to support it.

% \vspace{-0.2cm}
\subsection{Concept of \ac{oidc²}}
\label{sec:oidc2_concept}
% \vspace{-0.1cm}
%-----------------------------------
% Überblick wie wir OIDC² erklären:
We define new terminology, introduce the \acf{ict}, and explain how to use it.

% \vspace{-0.2cm}
\subsubsection{Terminology}
\label{sec:oidc2_terminology}
% Einführung Terminologie EU, Client, AU, AP, OP und Keys:
Consider a user of one application authenticating to a user of another application.
The user authenticating himself is called the \acf{eu}, his application is called the Client.
The other user is called the \acf{au}, and his application is called the \acf{ap}.
% Erklärung der Herkunft der Terminologie; Klarstellung, dass dies nicht heißt, dass OAuth und OIDC für OIDC² erforderlich sind:
We also assume that the \ac{eu} has an \ac{sso} identity provided by an \acf{op} trusted by the \ac{au}.
The terminology used for the \ac{eu}, Client, and \ac{op} is consistent with the combined OAuth 2.0 and \ac{oidc} scenario described in \sect{sso}.
However, \ac{oidc²} does not require this scenario.

% \vspace{-0.2cm}
\subsubsection{\acf{ict}}
\label{sec:ict}
% Einführung ICT und abgrenzung von ID Token:
We introduce the \ac{ict}, which addresses the \ac{e2e} authentication use case.
The \ac{ict} contains the Client's verified public key $K^+_C$, an application-specific Scope, an expiration date, and a unique identifier of the \ac{eu}'s \ac{sso} identity.
It may also contain other claims about the user which are not necessarily verified by the \ac{op}.

% \vspace{-0.2cm}
\subsection{\ac{ict} Request}
\label{sec:obtain_ict}
% \vspace{-0.1cm}
%-----------------------------------
% Wann ein ICT abgerufen wird; Verweis auf Grafik:
The Client uniquely requests an \ac{ict} from the \ac{op} for each \ac{e2e} authentication process.
\fig{oidc2_ict_request} simplifies the \ac{ict} request.
% Figure environment removed

% Abruf von Access Token:
First, the Client performs an OAuth 2.0 Authorization Request as described in \sect{oauth2} (1-4) to obtain an \ac{at} for the \ac{ict} Request.
For this purpose, the \ac{at} requires a Scope sufficient to access the \ac{eu}'s profile information, e.g., \texttt{profile}, and an \ac{e2e} Scope, e.g., \texttt{e2e\_auth\_email}.
% Abruf von Identity Certification Token; Einführung Client + OP Key Pairs:
The Client then uses the \ac{at} to authorize an OAuth 2.0 Resource Request for an \ac{ict} (5) from the \ac{op}, called an \ac{ict} Request.
For this purpose, the Client uniquely generates a new public key $K^+_C$ and presents it to the \ac{op}.
The Client also presents a \ac{pop} of the corresponding private key $K^-_C$, e.g., by signing a unique nonce.
The \ac{op} verifies the validity of the \ac{at} and the \ac{pop} (6).
If valid, the \ac{op} signs the \ac{ict} with its private key $K^-_{\ac{op}}$ corresponding to its published public key $K^+_{\ac{op}}$ and responds with the \ac{ict} (7).
When the \ac{ict} expires and a new \ac{ict} is required, the Client repeats steps (5) to (7) to request a new \ac{ict} for a new key pair.

% \vspace{-0.2cm}
\subsection{\ac{e2e} Authentication with \ac{ict}}
\label{sec:ict_usage}
% \vspace{-0.1cm}
%-----------------------------------
% Einführung; Verweis auf Grafik:
The Client uses the \ac{ict} to authenticate its \ac{eu} to the \ac{ap}'s \ac{au} as shown in \fig{oidc2_e2ea}.
% Transfer von ICT; Erklärung PoP Möglichkeiten:
First, the Client passes the \ac{ict} containing its public key $K^+_C$ to the \ac{ap} and provides a \ac{pop} for the corresponding private key $K^-_C$ (1).
To do this, the Client signs either a unique nonce provided by the \ac{ap} or a unique session-specific identifier.
Alternatively, the Client can prove the possession by establishing a secure channel based on the private key $K^-_C$.
In \sect{applications}, we show and explain use cases that take advantage of these three options.
% Verifikation von ICT und PoP:
The \ac{ap} then verifies the Client's \ac{pop} (2) using the public key $K^+_C$ from the \ac{ict} and verifies the \ac{au}'s trust relationship with the \ac{op} (3).
This may require user interaction or the use of whitelists, discussed further in \sect{trust_relationship}.
If the \ac{au} trusts the \ac{op}, the \ac{ap} checks the expiration date and verifies the signature of the \ac{ict} using the \ac{op}'s public key $K^+_{\ac{op}}$ (4).
If successful, the \ac{eu} has proven its \ac{sso} identity to the \ac{au} (5).
