%-------------------------------------------------------------------------------
% \vspace{-0.2cm}
\section{Security Considerations}
\label{sec:security}
% \vspace{-0.2cm}
%-------------------------------------------------------------------------------
% Überblick über Section:
First, we discuss how \ac{oidc²} shifts the burden of thorough authentication from service providers to identity providers.
Then, we analyze the trust relationship between \ac{oidc²} entities and propose a trust classification for \acp{op}.
Finally, we propose authentication with multiple \acp{ict} and discuss the correlation between the validity of an \ac{ict} and its corresponding key pair.

% \vspace{-0.2cm}
\subsection{Service Provider vs. OpenID Provider}
\label{sec:sp_vs_op}
% \vspace{-0.2cm}
%-----------------------------------
% Vorteile für Nutzer (e2e authentication); Vorteile für service provider (kaum Implementierungsaufwand, kein OP Betrieb nötig):
In most communication services, users must rely on the identity claims of their communication partners provided by the service provider, with no way to verify them.
\ac{oidc²} allows users to verify each other's identities without having to trust the service provider.
This only requires the Client to implement \ac{oidc²} and the protocol to provide a way to exchange the \acp{ict}.
The service provider does not need to implement OAuth 2.0 for the Client or provide an \ac{op}.
This improves the overall security of the service and prevents privacy issues by eliminating the need for the service provider to collect sensitive information about its users.

\subsection{Trust Relationship}
\label{sec:oidc2_trust_relationship}
% \vspace{-0.2cm}
%-----------------------------------
% Verweis auf Abbildung:
\fig{oidc2_trust_relationship} shows an overview of the trust relationship between the entities of the \ac{oidc²} protocol.
% Figure environment removed

% Vertrauen auf Proving Side: EU -> OP; EU -> Client:
On the proving side, the \ac{eu} trusts his \ac{op} to protect his identity from impersonation attacks and not to impersonate him.
This includes that the \ac{op} will only issue authorized \acp{ict}.
Furthermore, the \ac{eu} trusts that his Client will operate as intended.
This means that the Client will protect its private key $K^-_C$ from third parties and use the \ac{ict} only for the intended authentication processes.
To limit potential misuse by the Client, the \ac{ict} is scoped to a specific context.
For example, this prevents an email client from misusing the \ac{ict} to sign contracts on behalf of the \ac{eu}.

% Vertrauen auf Authenticating Side: AU -> OP; AU -> AP:
On the authentication side, the \ac{au} trusts the \ac{op} to protect the \ac{eu}'s identity and to sufficiently verify the Client's possession of its private key $K^-_C$.
The \ac{au} also trusts the \ac{op} to certify sufficiently trustworthy identity claims with the issued \ac{ict}, which we will discuss in more detail in \sect{classification}.
To ensure that the authentication process is intended by the \ac{eu}, the \ac{au} trusts the \ac{op} to issue only \ac{eu}-authorized \acp{ict}.
The \ac{au} must also trust his \ac{ap} to correctly verify the received \ac{ict} and \ac{pop}.

% Lösungsansätze, wie Vertrauen zwischen AU und OP hergestellt werden kann:
The \ac{au} needs to trust the \ac{op}.
We offer two solutions that can be combined.
First, the \ac{ap} trusts a trusted identity federation such as the \ac{gain} \cite{GainWhitepaper}, which consists of international \acp{op} such as banks, insurance companies, or government institutions, all of which manage fully verified real-world identities.
Second, the \ac{au} maintains his own whitelist of \acp{op}, such as social media platforms or his business partners.
Not every \ac{op} has the same level of trustworthiness, so we classify them in the next section.

% \vspace{-0.4cm}
\subsection{Classification of OpenID Providers}
\label{sec:classification}
% \vspace{-0.1cm}
%-----------------------------------
When working with \ac{oidc²}, we suggest three classes of \acp{op} to consider.

% \vspace{-0.4cm}
\subsubsection{\jp{\ac{iop}}}
\label{sec:insecure_op}
% Beschreibung von nicht vertrauenswürdigem OP:
\acp{op} can be considered insecure for a variety of reasons.
They may not be able to sufficiently protect their users' credentials, or they may be untrustworthy for political or economic reasons.
For example, they may certify potentially false or insufficiently verified claims.
If an \ac{au} considers an \ac{op} insecure, his \ac{ap} will not accept any \acp{ict} issued by that \ac{op}.

% \vspace{-0.4cm}
\subsubsection{\jp{\ac{aop}}}
\label{sec:authoritative_op}
% AOP schützt accounts hinreichend und ist Autorität für bestimmte claims:
We classify an \ac{op} as an \ac{aop} for specific claims, if the \ac{au} accepts the \ac{op} as an authority for those claims and trusts the \ac{op} to protect managed identities.
For example, an email server's \ac{op} is authoritative for email addresses within its own domain.
Because an \ac{op} issues a unique subject identifier for each \ac{sso} identity by specification, an \ac{op} is always authoritative for its associated \texttt{sub} claim.

% Beispiel und Limitations:
This makes \acp{aop} sufficient for scenarios where an \ac{eu} wants to be authenticated with specific claims.
For example, if the \ac{au} knows the \ac{eu}'s email address, the \ac{eu} uses an \ac{ict} issued by his email provider's \ac{op} to authenticate on a social media platform.
However, \acp{aop} are only sufficient to certify identity claims for which they are an authority.
To certify real-world identity claims such as names or addresses, the \ac{aop} must typically be the \ac{op} of a trusted government organization.

% \vspace{-0.4cm}
\subsubsection{\jp{\ac{vop}}}
\label{sec:verifying_op}
% Ein VOP schützt seine Accounts hinreichen und verifiziert bestimmte claims hinreichend:
There is not always an \ac{aop} for every claim the \ac{eu} wants to be authenticated with.
Instead, the \ac{eu} can use a third-party service that the \ac{au} trusts to sufficiently verify his identity claims and protect his account.
We call the \ac{op} of this third-party service a \jp{\ac{vop}}.
This \ac{vop} could check the \ac{eu}'s passport to verify his name, or send him a verification code via SMS to verify his phone number.

% Beispiele, Limitations und Hybrid-Modelle:
There are already \jp{\acs{op}} such as banks or insurance companies that are required by law to verify their customers' claims.
However, such verification processes are often costly, which is why \acp{vop} often do not verify all claims or offer it as an optional service, such as the social media platforms Facebook and \jp{X}.
Both can be \acp{aop} at the same time.
For example, banks are \acp{vop} for the name of an \ac{eu}, but also \acp{aop} for bank account numbers.

% \vspace{-0.3cm}
\subsection{Authentication with Multiple \acp{ict}}
\label{sec:multi_ict}
% \vspace{-0.1cm}
%-----------------------------------
% Diskussion über Vor- und Nachteile der Authentifizierung mit mehreren ICTs.
The classification of an \ac{op} is up to the \ac{au}, i.e., the \ac{au} may not accept \acp{ict} from certain \acp{op}.
Since an \ac{eu} may not know the \ac{au}'s classification in advance, the \ac{eu} can present \acp{ict} from different \acp{op} and the corresponding \acp{pop} to increase the likelihood of successful authentication by the \ac{au}.
However, this requires more work for the \ac{eu} as he has to log in to all these \acp{op} to receive \acp{ict}.
If the \ac{ap} receives multiple \acp{ict}, it presents them to the \ac{au}, which then selects the most trusted issuer or rejects them all.
Furthermore, the \ac{eu} must be aware that presenting multiple \acp{ict} also exposes all his presented accounts to the \ac{au}.

% \vspace{-0.3cm}
\subsection{Validity of \acp{ict} and Client Key Pairs}
\label{sec:ict_validity_period}
% \vspace{-0.1cm}
%-----------------------------------
% Einführung, wie das ICT mit dem Key Pair des Clients zusammenhängt:
An \ac{ict} contains the public key $K^+_C$ of the Client.
By issuing this \ac{ict}, the ac{op} certifies that the corresponding \ac{eu} authorized the Client for \ac{e2e} authentication with the contained identity claims.
% By issuing an ICT, the OP certifies that the Client whose public key $K^+_C$ the ICT contains is authorized by the EU to perform e2e authentication with the contained identity claims.
An attacker trying to impersonate the \ac{eu} needs the corresponding private key of the \ac{ict}.
% Whoever proves to be in possession of the corresponding verified private key $K^-_C$ is also authorized -- even attackers.
% Einschränkung von ICT durch validity period und scope:
We minimize this potential misuse of the \ac{ict} by a leaked private key by making the \ac{ict} short-lived and limited in scope.
Since a few minutes are sufficient for most use cases (see \sect{applications}), we recommend setting the \ac{ict} validity period to no more than 1 hour.

% ICT Validity = Key Pair Validity:
We propose that an ephemeral and unique key pair $K^\pm_C$ expires along with its associated \ac{ict}, eliminating the need for key revocation mechanisms.
% Ausnahme ist long-term key pair; Erfordert Verweis auf key revocation list:
However, Sections \ref{sec:im} and \ref{sec:email} show that long-term key pairs are useful in some cases.
Therefore, we further propose that an \ac{ict} may also contain a long-term public key, which must be indicated by providing the key revocation server of the key.
Such a key is valid until revoked and is associated with the claims in the \ac{ict}.
Some services control the lifetime of public keys by associating them with user profiles.
An example of this approach is the Signal protocol (see \sect{im}).
In such applications, a user can be authenticated with a public key received from an \ac{ict} as long as the public key contained in it is associated with the profile.
% Session darf auch nach key expiry gültig bleiben:
In any case, an active session can remain valid even after the underlying key pair $K^\pm_C$ expires (see \sect{vc}).
