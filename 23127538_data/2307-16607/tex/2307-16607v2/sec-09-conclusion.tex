%-------------------------------------------------------------------------------
% \vspace{-0.2cm}
\section{Conclusion and Future Work}
\label{sec:conclusion}
% \vspace{-0.2cm}
%-------------------------------------------------------------------------------
This paper introduced \ac{oidc²}, which allows \acp{eu} to request a verifiable \ac{ict} from an \ac{op}.
An \ac{ict} contains claims about an \ac{eu} and a public key chosen by the \ac{eu}.
\acp{au} can authenticate \acp{eu} with an \ac{ict} if they trust his issuing \ac{op}.
%
We compared \ac{oidc²} to existing \ac{e2e} authentication methods and found that \ac{oidc²} is easier to use and improves security by eliminating the need for revocation lists.
We suggested how \ac{oidc²} can be implemented based on the \ac{oidc} framework.
We discussed security considerations for and general improvements with \ac{oidc²}: the trust relationship among its entities, a classification of \acp{op} and their utilization with \ac{oidc²}, authentication with multiple \acp{ict} to increase the likelihood of successful authentication, as well as appropriate (short) validity periods for \acp{ict}.
%
Furthermore, we proposed how \ac{oidc²} can be used for simple and user-friendly \ac{e2e} authentication for video conferencing, email, and \ac{im}.
Finally, we provided a simple, open-source extension for \ac{oidc} server software to support \ac{oidc²} for testing purposes.
We proved its compatibility with Authentik and Keycloak and the performance of the new \texttt{/ict} Endpoints is comparable to or better than the performance of the existing \texttt{/token} Endpoints.

To demonstrate the feasibility of \ac{oidc²} for \ac{e2e} authentication, we plan to integrate \ac{oidc²} for video conferencing based on the open WebRTC protocol, for \ac{im} based on the open Matrix protocol, and for email communication based on the \ac{pgp} standard.
