%-------------------------------------------------------------------------------
% \vspace{-0.2cm}
\section{Introduction}
\label{sec:introduction}
% \vspace{-0.2cm}
%-------------------------------------------------------------------------------
% Einführung, warum E2E Authentication notwendig ist:
In most communication services, users identify each other through account profiles in which they provide their own identity information.
To make these profiles more trustworthy, social network operators such as Meta and \jp{X} offer identity verification services for an additional fee that can only be used within their ecosystem.
However, identity verification is often a cumbersome process that users may not want to repeat for each of their service platforms.
\jp{Furthermore}, users must still trust the service provider to sufficiently verify identities and not impersonate them.
\ac{e2e} user authentication mechanisms \jp{like X.509 certificates or \ac{pgp}} attempt to solve this problem, but often lack adoption due to poor usability \jp{caused by the necessity of manual key management}.
% Vorstellung OIDC und SSO; Hinweis darauf, dass E2E Authentication fehlt:
Therefore, reusing an account \jp{with} a verified identity \jp{for \ac{e2e} user authentication} would be desirable.
With modern \ac{sso} services, users can reuse their existing accounts to log in to other services.
The \ac{oidc} protocol, which is based on the OAuth 2.0 authorization framework, is widely used for this purpose.
However, \ac{oidc} is designed for user-to-service authentication and does not address the purpose of \ac{e2e} user authentication.

% Vorstellung ICT und OIDC²:
\jp{In this paper, we developed an \ac{oidc} extension for \ac{e2e} user authentication which is (1) easy to implement and (2) easy to use.}
\jp{Therefore}, we define a new \ac{ict} for \ac{oidc}.
It is similar to the \ac{idt} which holds identity claims about a user, but also contains a verified public key of the user's client.
As such, it can be thought of as a JSON-based, short-lived user certificate without the need for a revocation mechanism.
The use of an \ac{ict} differs significantly from the use of an \ac{idt}.
A user requests an \ac{ict} from his \ac{op} and presents it to another user's client to authenticate himself.
If the other user trusts the issuing \ac{op}, his client verifies the integrity and validity of the \ac{ict} and authenticates the user using his client's public key contained in the \ac{ict}.
As the \ac{op} certifies the identity of the user, we call this concept \acp{oidc²}.
It facilitates mutual authentication of users if they trust each other's \ac{op}.

% Notwendigkeit von Identitätsverifizierung und Relevanz von OIDC:
\jp{Although} most \acp{op} have a rather superficial identity verification process for their accounts, some practice a more thorough verification.
In particular, new players such as banks and government institutions that perform rigorous identity verification for their accounts are becoming \acp{op}.
With \ac{oidc²}, unknown users can be reliably authenticated if they have an \ac{oidc} account at a trusted \ac{op}.
% Kritik an PKI und SSI:
Some services already provide strong user authentication, but these methods are difficult to use.
Many \ac{im} services support the exchange of public keys between users when they meet in person.
\acp{pki} require certificate management by users and reliable revocation list checking.
\ac{pgp} or \ac{smime} have long been proposed for email authentication, but are rarely used \cite{Stransky2022}.
\ac{ssi} technology is currently emerging and solves this problem with device-specific long-term keys in a wallet app.
However, this requires not only revocation mechanisms, but also recovery mechanisms in case the phone with the wallet app is lost or stolen.
% Was OIDC² besser macht und wie:
\ac{oidc²} provides a more user-friendly alternative for \ac{e2e} authentication.
The \ac{ict} is short-lived, eliminating the need for cumbersome key revocation mechanisms, which improves security.
\ac{oidc²} avoids complex key management across devices by simply requesting a new \ac{ict} from the \ac{op} whenever needed.
Using trusted \acp{op} that verify the identity of their users also eliminates the need for face-to-face key exchange.
Thus, a trusted \ac{op} can be compared with a trusted certification authority in a \ac{pki} or a trusted issuer in the \ac{ssi} context.
However, \ac{oidc²} is only a lightweight extension for \ac{e2e} authentication with existing \ac{oidc} accounts.
It is not intended to replace \jp{X.509} \acp{pki} or \acp{ssi}.

% Vorstellung der Paper-Struktur:
The paper is structured as follows.
In \sect{background}, we revisit \jp{the} basics of OAuth 2.0 and \ac{oidc}, and in \sect{related_technologies}, we review related authentication technologies.
\sect{oidc2} introduces the concept of \ac{oidc²} and proposes the extension to the \ac{oidc} protocol.
Trust relationships in \ac{oidc²}, a classification of \acp{op}, authentication with multiple \acp{ict}, and validity periods of \acp{ict} are discussed in \sect{security}.
In \sect{applications}, we explain how \ac{oidc²} can be applied to video conferencing, \ac{im}, and email.
To test \ac{oidc²}, we provide a simple extension to the \ac{oidc} server software in \sect{implementation}, which we evaluate in \sect{evaluation}.
\sect{conclusion} concludes our findings.
