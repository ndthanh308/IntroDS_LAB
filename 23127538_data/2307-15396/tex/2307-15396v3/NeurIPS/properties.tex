
\section{Omitted Proof from Section \ref{sec:characterizing}}

Recall the definition of the affine functions $g_1,\dots, g_{n-1}$ in Section \ref{sec:characterizing}. Also, recall the slope values $\delta_0,\dots, \delta_{n}$ and the sign of the discrete curvature $\curv(x_i)$ at any point $x_i$. As mentioned, \citet{boursier2023penalising} showed Lemma \ref{lem:char}, which says that there is a unique minimizer of \eqref{eq:main_problem}, which is piecewise linear and has at most one kink in the range $[x_i,x_{i+1})$. Due to this, we have only one degree of freedom between any two points; the solution can be completely characterized by variables $s^*=\{s^*_i\}_{i=1}^{n}$ where $s^*_i$ is the slope of the line incoming to point $(x_i,y_i).$ Formally, $s^*_i= \lim_{\eps \rightarrow 0^+} D \hat{f}(x_i-\eps)$.
 
\citet{boursier2023penalising} (Lemma 3) proved the following lemma which upper and lower bounds the value of $s^*_i$ in terms of $\delta_{i-1}$ and $\delta_i$.
\begin{lemma}[\citet{boursier2023penalising}]\label{lem:slopes-bound-left-right}
 For any $i\in [n]$, $s^*_i \in [ \min(\delta_{i-1},\delta_i), \max(\delta_{i-1},\delta_i)]$ where $\delta_0=\delta_1$ and $\delta_n=\delta_{n-1}$.   
\end{lemma}
Having the above lemma at our disposal, we will now show Lemma \ref{lem:complete-description-property-main_text}, which characterizes the worst-case behavior of formation spikes.
\begin{proof}[Proof of Lemma \ref{lem:complete-description-property-main_text}]
    It is easy to see that the function $\hat{f}_S$ cannot have any kink on $(-\infty, x_1)$. This just follows from Lemma \ref{lem:char}. Moreover, the slope of this straight line incoming to $(x_1,y_1)$ is given by $s^*_1 \in [\min\{\delta_0,\delta_1\},\max\{\delta_0,\delta_1\}]=\delta_1$ by Lemma \ref{lem:slopes-bound-left-right}. Therefore, $\hat{f}_S(x)$ in the interval $(-\infty, x_1)$ must be $g_1(x)$. Now, again by Lemma \ref{lem:char} we can have at most one kink between $[x_1, x_2)$. Since $g_1(x)$ is a unique line joining the points $(x_1,y_1)$ and $(x_2,y_2)$, having one kink and changing the line at some point $x\in [x_1, x_2)$ will not pass through the point $(x_2,y_2)$. But since $\hat{f}_S(x_2)=y_2$, we must have that $\hat{f}_S(x)=g_1(x)$ in the entire $(-\infty, x_2)$. 

    Similarly, by Lemma \ref{lem:char}, we don't have any kink from $[x_n,\infty)$. Moreover, the slope incoming to the point $(x_n,y_n)$ is $s^*_n=\delta_{n-1}$ since $s_n^* \in [ \min\{\delta_{n-1},\delta_n\}, \max\{\delta_{n-1}, \delta_n\}]$ and $\delta_{n-1}=\delta_n$ by Lemma \ref{lem:slopes-bound-left-right}. Thus, it must be that $\hat{f}_S(x)=g_{n-1}(x)$ in $[x_n,\infty)$. Moreover, $\hat{f}_S(x)$ has at most one kink in $[x_{n-1},x_n)$. This kink cannot belong to $(x_{n-1},x_n)$ since $g_{n-1}(x)$ is the unique line joining the points $(x_{n-1},y_{n-1})$ and $(x_n,y_n)$. Combining, we can say that $\hat{f}_S(x)=g_{n-1}(x)$ in $[x_{n-1},\infty)$.

    We now consider $[x_i, x_{i+1})$ for any $i \in \{2, \dots, n-2\}$. We prove the lemma under three different cases.
    \begin{enumerate}
        \item \textbf{Case 1: } $\curv(x_i)=\curv(x_{i+1})=-1$\\
        First of all, since $\delta_{i-1}>\delta_i$ and $g_{i-1}(x_i)=g_{i}(x_i)=y_i$, we have $g_{i-1}(x) \geq g_i(x)$ in $x\in [x_i,x_{i+1})$. Similarly, since $\delta_{i+1}<\delta_i$ even $g_{i+1}(x) \geq g_i(x)$ for $x \in [x_i, x_{i+1})$. Using the same argument, since $s^*_i \in [\delta_i, \delta_{i-1}]$ and $s^*_{i+1} \in [\delta_{i+1},\delta_i]$ by Lemma \ref{lem:slopes-bound-left-right}, we say that $\hat{f}_S$ lies above $g_i(x)$ in $[x_i,x_{i+1})$. Therefore, it only remains to show that $\hat{f}_S(x) \leq \min\{g_{i-1}(x),g_{i+1}(x)\}$. 
        
        Let us assume that it is not true. Let $x^* \in [x_i,x_{i+1})$ be the intersection point of $g_{i-1}(x)$ and $g_{i+1}(x)$. If $\hat{f}_S(x) \geq \min\{g_{i-1}(x),g_{i+1}(x)\}$ for some $x \in [x_i,x_{i+1})$, then it is easy to observe that it must be true at the location of the kink which is $x'$, i.e. $\hat{f}_S(x') \geq \min\{g_{i-1}(x'),g_{i+1}(x')\}$.% is where the maximum value of $\hat{f}_S$ is (in the considered interval). 
        
        %Then $\hat{f}_S(x)>\min\{g_3(x),g_4(x)\}$. 
        
        Now we have two possibilities; the first one is $x'<x^*$, in which case $$s^*_i=\frac{\hat{f}_S(x')-y_{i}}{x'-x_i} > \frac{g_{i-1}(x')-y_{i}}{x'-x_i} = \delta_{i-1},$$ contradicting Lemma \ref{lem:slopes-bound-left-right}. If $x'>x^*$ then 
        $$s^*_{i+1}=\frac{y_{i+1}-\hat{f}_S(x')}{x_{i+1}-x'} < \frac{y_{i+1}-g_{i+1}(x')}{x_{i+1}-x'} = \delta_{i+1},$$ again contradicting Lemma \ref{lem:slopes-bound-left-right}. In either case, we get a contradiction and therefore, we must have
        $$ g_i(x) \leq \hat{f}_S(x) \leq \min\{g_{i-1}(x),g_{i+1}(x)\}.$$

        \item \textbf{Case 2:} $\curv(x_i)=\curv(x_{i+1})=+1$\\
        Using a similar strategy, one can also show the desired result in this case. More formally, flip the label signs and apply exactly the same argument but on $-\hat{f}_S,-g_i,-g_{i-1},-g_{i+1}$ instead. Then using the above argument, we achieve
        $$ -g_i(x) \leq  -\hat{f}_S(x) \leq \min\{-g_{i-1}(x), -g_{i+1}(x)\}.$$ 
        This implies
        $$ g_i(x) \geq  \hat{f}_S(x) \geq \max\{g_{i-1}(x), g_{i+1}(x)\}.$$
        \item \textbf{Case 3:} Otherwise, we may have several situations. We consider them one by one. If $\curv(x_i)=0$. Then $\delta_{i-1}=\delta_i$. Then by Lemma \ref{lem:slopes-bound-left-right} we must have that $s^*_i=\delta_i$. Also, we have either one or no kink in $[x_i,x_{i+1})$ by Lemma \ref{lem:char}. Since $g_i(x)$ is the unique line joining the points $(x_i,y_i)$ and $(x_{i+1},y_{i+1})$, we cannot have any kink and $\hat{f}_S=g_i(x)$. Similarly, if $\curv(x_{i+1})=0$ then $\delta_i=\delta_{i+1}$ and $s^*_{i+1}=\delta_i$, which gives us that $\hat{f}_S(x)=g_i(x)$ using the uniqueness.

        The only remaining possibilities are $\curv(x_i)=+1$ but $\curv(x_{i+1})=-1$ or $\curv(x_i)=-1$ but $\curv(x_{i+1})=+1$. W.l.o.g., we consider the former situation. Then $\delta_{i-1}< \delta_i$ and $\delta_i > \delta_{i+1}$. Also, by Lemma \ref{lem:char} there is at most one kink in $[x_i,x_{i+1})$. Therefore, if we show that the kink is at $x_i$ only, it is sufficient. Let us assume that it is not the case, namely, the kink is at $x' \in (x_i,x_{i+1})$. If $\hat{f}_S(x')>g_i(x)$, then the slope of the line incoming at $x_i$:
        $$s^*_i= \frac{\hat{f}_S(x')-y_i}{x'-x_i}>\frac{g_i(x')-y_i}{x'-x_i}=\delta_i,$$
        contradicting Lemma \ref{lem:slopes-bound-left-right}. On the other hand, if $\hat{f}_S(x') < g_i(x)$, then the slope of the line incoming at $x_{i+1}$:
        $$s^*_{i+1}= \frac{y_{i+1}-\hat{f}_S(x')}{x_{i+1}-x'}>\frac{y_{i+1}-g_i(x')}{x_{i+1}-x'}=\delta_i,$$
        contradicting Lemma \ref{lem:slopes-bound-left-right}.
    \end{enumerate}
    Therefore, the lemma is true in all the cases. 
    
        %First of all, there is exactly one kink in $[x_i,x_{i+1})$. If it happens at $x_i$, then clearly, there can be no other kink an $\hat{f}_S(x)$ must be $g_i(x)$ in $[x_i,x_{i+1})$. In this case, the claim clealry holds
\end{proof}
As a corollary, we get the following lemma.
\begin{lemma}\label{lem:bound on |f^-f*|}
    Fix any $i\in \{2,\dots,n-2\}$, and consider $x\in [x_{i},x_{i+1})$:
\[
 |\hat{f}_S(x)-f^*(x)| \leq \max\left\{ |g_i(x)-f^*(x)|, \min\{|g_{i+1}(x)-f^*(x)|, |g_{i-1}(x)-f^*(x)|\}\right\}~,
 \]
and for $x\in [0,x_2)$ 
 \[
 |\hat{f}_S(x)-f^*(x)| = |g_1(x)-f^*(x)|,
 \]
 and for $x \in [x_{n-1},1]$
 \[ 
 |\hat{f}_S(x)-f^*(x)| = |g_{n-1}(x)-f^*(x)|~.
 \]
\end{lemma}
\begin{proof}
    The above lemma is clearly true for $x\in [0,x_2)$, since in that range, $\hat{f}_S(x)=g_1(x)$ by Lemma \ref{lem:complete-description-property-main_text}. Similarly, for $x\in [x_{n-1},1]$ we have $\hat{f}_S(x)=g_{n-1}(x)$ by Lemma \ref{lem:complete-description-property-main_text}. This implies that $|\hat{f}_S(x)-f^*(x)|=|g_{n-1}(x)-f^*(x)|$.

    Also, by applying Lemma \ref{lem:complete-description-property-main_text}, for any $i\in \{2,\dots,n-2)$ and for any $x\in [x_i,x_{i+1})$, one can say that unless $\curv(x_i)=\curv(x_{i+1})=-1$ or $\curv(x_i)=\curv(x_{i+1})=+1$, we have
    $$ \hat{f}_S(x)=|g_i(x)-f^*(x)|,$$
    and the lemma holds. If $\curv(x_i)=\curv(x_{i+1})=-1$. Then by Lemma \ref{lem:complete-description-property-main_text}, we have $ g_i(x) \leq \hat{f}_S(x) \leq \min\{g_{i-1}(x),g_{i+1}(x)\}$. Let $x^* \in [x_{i},x_{i+1})$, where $g_{i-1}(x)$ and $g_{i+1}(x)$ meet. 

    For $x\in [x_i,x^*]$: $g_i(x)\leq \hat{f}_S(x) \leq g_{i-1}(x) \leq g_{i+1}(x)$. Therefore,
    $$ \underbrace{g_i(x)-f^*(x)}_{:=(1)} \leq \hat{f}_S(x)-f^*(x) \leq \underbrace{g_{i-1}(x)-f^*(x)}_{:=(2)} \leq \underbrace{g_{i+1}(x)-f^*(x)}_{:=(3)}.$$
    If $(1)$ is non-negative, then the claim is clearly true. Because even $(2)$ and $(3)$ are positive. If $(1)$ is negative then the only way the $|\hat{f}_S(x)-f^*(x)|$ can be greater than $|g_i(x)-f^*(x)|$ is when $g_{i-1}(x)-f^*(x)$ is positive. And thus $g_{i+1}(x)-f^*(x)$ is also positive.
    Therefore we have
     $$-|g_i(x)-f^*(x)|\leq \hat{f}_S(x)-f^*(x) \leq |g_{i-1}(x)-f^*(x)| \leq |g_{i+1}(x)-f^*(x)|,$$
     which immediately implies the desired result. Similarly, for $x \in (x^*,x_{i+1})$: $g_i(x) \leq \hat{f}_S(x) \leq g_{i+1}(x) \leq g_{i-1}(x)$. Therefore,
    $$ \underbrace{g_i(x)-f^*(x)}_{:=(1)} \leq \hat{f}_S(x)-f^*(x) \leq \underbrace{g_{i+1}(x)-f^*(x)}_{:=(2)} \leq \underbrace{g_{i-1}(x)-f^*(x)}_{:=(3)}.$$
    Again if $(1)$ is non-negative, then the claim is clearly true. If $(1)$ is negative then the only way the $|\hat{f}_S(x)-f^*(x)|$ can be greater than $|g_i(x)-f^*(x)|$ is when $g_{i+1}(x)-f^*(x)$ is positive. And thus $g_{i-1}(x)-f^*(x)$ is also positive.
    Therefore we have
     $$-|g_i(x)-f^*(x)|\leq \hat{f}_S(x)-f^*(x) \leq |g_{i+1}(x)-f^*(x)| \leq |g_{i-1}(x)-f^*(x)|,$$
     and the lemma follows. The proof is exactly symmetric for when $\curv(x_i)=\curv(x_i)=+1$; the only difference is that we apply the same argument on $-g_i,-g_{i-1},-g_{i+1}$ and $-\hat{f}_S$ and add the function $f^*(x)$ instead of subtracting. 
    
\end{proof}

We now recall Definition \ref{def:cruv-changed} of special points. \citet{boursier2023penalising} proved that if two special points occur within two points. Then we get a spike. Our Lemma \ref{lem:sparsity-main_text} is a mild generalization of the lemma.
\begin{proof}[Proof of Lemma \ref{lem:sparsity-main_text}]
\citep[Lemma 8]{boursier2023penalising} already showed that if $n_{k+1}=n_k+2$ then $\hat{f}_S$ has exactly one kink in $(x_{n_k-1},x_{n_{k+1}})$. Consider four consecutive points $(x_{n_k-1}, y_{n_k-1})$, $(x_{n_k},y_{n_k})$,  $(x_{n_k+1},y_{n_k+1}),$ and $(x_{n_k+2},x_{n_k+2})$. The first three are non-collinear and even the last three are non-colinear since $\curv(x_{n_k})=\curv(x_{n_k+1})=-1$. Therefore, the only way to interpolate them with 2 linear pieces is to extend $g_{n_k-1}$ and $g_{n_k+1}$ until they intersect. This immediately implies that $\hat{f}_S(x)=\min\{g_{n_k-1},g_{n_k+1}\}$.
\end{proof} 
%Moreover, we identify special points from left to right recursively where the sign of the curvature changes.
%\begin{definition}\label{def:cruv-changed}
%    We define $n_1=1$. Having defined $n_1,\dots, n_{i-1}$, we recursively define 
%    $$n_i=\arg \min \{ j>n_{i-1} : \curv(x_{j}) \neq \curv (x_{n_i})\}.$$
%    If there is no such $n_{i-1} < j \leq n$ such that $\curv(x_{j}) \neq \curv (x_{n_i})$, then $n_{i-1}$ is the location of the last special point.
%\end{definition}
    %\begin{lemma}\label{lem:sparsity}
   %     If for any $k \geq 1$, if $\delta_{n_k -1 }  \neq  \delta_{n_k}$ and $n_{k+1}= n_k + 2$, then $\hat{f}$ has exactly one kink between $(x_{{n_k}-1}, x_{n_{k+1}})$.
    %\end{lemma}
