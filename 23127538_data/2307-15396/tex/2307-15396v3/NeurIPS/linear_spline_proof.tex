\section{Proof of Theorem \ref{thm:linear-splines}}
\begin{proof}[Proof of Theorem \ref{thm:linear-splines}]
Let $G<\infty$ be the Lipschitz constant of $f^*$. We sample $S\sim \dist^n$ and we number points $(x_i,y_i)$ such that:
$$0 < x_1 < x_2 < \dots < x_n < 1.$$ 
We will also denote $x_0=0$ and $x_{n+1} = 1$ for simplicity of exposition. Our goal is to analyze the population and reconstruction errors of linear splines $\hat{g}_S(x)$, as defined in \eqref{eq:linear-spline}. 
\begin{align}
  \mathcal{L}_p(\hat{g}_S) &= \Exp_{(x,y)\sim \dist} \left[ | \hat{g}_S(x)-y |^p\right] \nonumber \\  
  &= \Exp_{x\sim \unif([0,1]),\eps} \left[ |\hat{g}_{S}(x)-f^*(x)-\eps|^p\right] \nonumber\\
  & \leq 2^{p-1} \Exp_{x\sim \unif([0,1]),\eps} \left[ |\hat{g}_S(x)-f^*(x)|^p+|\eps|^p\right] \nonumber\\
  &= 2^{p-1} \left( \Exp_{x\sim \unif([0,1])} \left[ |\hat{g}_S(x)-f^*(x)|^p \right] + \Exp_{\eps} \left[ |\eps|^p\right] \right) \nonumber\\
  &= 2^{p-1} ( \mathcal{R}_p(\hat{g}_S)+ \mathcal{L}_p(f^*) ) \label{eq:L_p-R_p-splines}~.
\end{align}
Therefore, it boils down to analyzing $\mathcal{R}_p(\hat{g}_S)$. We define the risk in the interval $[x_i,x_{i+1}]$ as the random variable $R_i$. In particular, for $i\in \{0,1,\dots,n\}$ as
\begin{equation}\label{eq:lin-spline-sum-of-risk}
   R_i:=\int_{x_i}^{x_{i+1}} |\hat{g}_S(x)-f^*(x)|^p \, dx  \, , \quad \text{ and } \quad \mathcal{R}_p(\hat{g}_S(x))=\sum_{i=0}^{n} R_i. 
\end{equation}
The entire range from $[0,1]$ is divided into $n+1$ intervals. We denote their length by $\ell_0,\dots, \ell_n$. In particular, $\ell_i:= x_{i+1}-x_{i}.$ Recall the joint distribution of $(\ell_0,\dots,\ell_n)$ in \eqref{eq:ell_distribution}.

We first show that the sum of the risks in the first and the last intervals is vanishing as $n\rightarrow \infty$.
\begin{lemma}\label{lem:R_0+R_n-spline}
    For any $\gamma>0$, we have $\lim_{n \rightarrow \infty} \Pr_S[ R_0 +R_n \leq \gamma]=1.$
\end{lemma}

All the helper lemmas, including the above, are proved at the end of the proof of the theorem. We now focus on bounding the remaining $R_i$'s. Define 
\begin{equation}\label{eq:def-R-splines}
    R:=\sum_{i=1}^{n-1}R_i~,
\end{equation}
then we are interested in bounding $R$. For any $i\in[n-1]$ and $x\in [x_i,x_{i+1}]$,
\begin{align*}
 |\hat{g}_{S}(x)-f^*(x)| =& |g_i(x)-f^*(x)| \\
 =&\left| y_{i}+ \frac{(y_{i+1}-y_{i})}{(x_{i+1}-x_{i})} (x-x_{i})-f^*(x)\right|\\
 =& \left| f^*(x_{i})+\eps_{i} + \left(\frac{f^*(x_{i+1})+\eps_{i+1}-f^*(x_{i})-\eps_{i}}{x_{i+1}-x_{i}}\right)(x-x_{i}) -f^*(x)\right| \\
 \leq & \left| f^*(x_{i}) -f^*(x)\right|+ |\eps_{i}|+  \frac{G|x_{i+1}-x_{i}|+|\eps_{i+1}-\eps_{i}|}{x_{i+1}-x_{i}}   (x-x_{i})  \\
 \leq &  G(x-x_{i}) + |\eps_{i}|+ \left( \frac{G(x_{i+1}-x_{i})+|\eps_{i+1}|+|\eps_{i}|}{x_{i+1}-x_{i}} \right)  (x-x_{i})  \\
 \leq & G \cdot \ell_i + |\eps_{i}|+ \left( \frac{G \cdot \ell_i +|\eps_{i+1}|+|\eps_{i}|}{\ell_i} \right) \cdot \ell_i\\
 = & 2G \cdot \ell_i + |\eps_{i+1}|+2 |\eps_{i}|.
\end{align*}
Therefore, for $i\in [n-1]$
\begin{align*}
  R_i=\int_{x_i}^{x_{i+1}} |\hat{g}_{S}(x)-f^*(x)|^p \, dx \, & \leq \ell_i (2G \cdot \ell_i + |\eps_{i+1}|+2 |\eps_{i}|)^p\\
  & \leq 3^{p-1} \ell_i ((2G)^p \ell_i^p+|\eps_{i+1}|^{p}+2^p |\eps_{i}|^p)\\
  &\leq 3^{p-1}(2G)^p \ell_i^{p+1}+3^{p-1} \ell_i |\eps_{i+1}|^{p}  + 3^{p-1} 2^p \ell_i |\eps_{i}|^p  :=\hat{R}_i  
\end{align*}
Therefore, if we define $\hat{R}:=\sum_{i=1}^{n-1} \hat{R}_i$ then it serves as upper bound for $R$, i.e. $R \leq \hat{R}$. Since the $\ell_i$'s are mildly dependent random variables; we will try to re-express $\hat{R}$ as the sum of independent random variables. We now define random variables $\Tilde{\ell}_0, \dots, \Tilde{\ell}_n \iid \text{Exp}(1)/(n+1)$. More specifically, $(\Tilde{\ell}_0,\dots,\Tilde{\ell}_n)=(X_0,\dots,X_n)/(n+1)$. Using these random variables, define random variables similar to $\hat{R}_1,\dots, \hat{R}_{n-1}$, but replace $\ell_i$ with $\Tilde{\ell}_i$
$$ \Tilde{R}_i:= 3^{p-1}(2G)^p \Tilde{\ell}_i^{p+1}+3^{p-1} \Tilde{\ell}_i |\eps_{i+1}|^{p}  + 3^{p-1} \cdot 2^p \Tilde{\ell}_i |\eps_{i}|^p,  \quad \text{and} \quad \Tilde{R}=\sum_{i=1}^{n-1} \Tilde{R}_i.$$
Then, the following lemma (whose proof we include after the proof of the theorem) establishes the almost sure convergence between $\hat{R}$ and $\Tilde{R}$. Therefore, it suffices to bound the latter.
\begin{lemma}\label{lem:poison-equivalence-splines}
    As $n \rightarrow \infty$, we have $\Tilde{R} - \hat{R} \xrightarrow{\textnormal{a.s.}} 0$.
\end{lemma}


Still on looking at $\Tilde{R}$, any two consecutive $\Tilde{R}_i$ and $\Tilde{R}_{i+1}$ are dependent since it shares $\eps_{i+1}$ in its definition. Due to this, we split $\Tilde{R}$ into two sums $\Tilde{R}_{\text{odd}}$ (and $\Tilde{R}_{\text{even}}$ ) containing odd-numbered terms (and even-numbered terms respectively).
$$\Tilde{R}_{\text{odd}} := \hspace{-5mm} \sum_{i \in [n-1],i\%2=1} \hspace{-5mm}\Tilde{R}_i, \quad \Tilde{R}_{\text{even}} := \hspace{-5mm}\sum_{i \in [n-1],i\%2=0} \hspace{-5mm}\Tilde{R}_i, \quad \text{and } \quad \Tilde{R}=\Tilde{R}_{\text{odd}}+\Tilde{R}_{\text{even}}$$
Now, $\Tilde{R}_{\text{odd}}$ is the sum of $\ceil{(n-1)/2}$ i.i.d. random variables. Similarly, $\Tilde{R}_{\text{even}}$ is the sum of $\floor{(n-1)/2}$ i.i.d. random variables. Let us calculate the expectation of these identically distributed random variables, which are $\Tilde{R}_i$'s. For any $i \in [n-1]$,
Further simplifying:
\begin{align}
    \Tilde{R}_{\text{odd}} &=\sum_{i \in [n-1], i \%2=1} \left( \frac{3^{p-1} (2G)^p X_i^{p+1}}{(n+1)^{p+1}}+\frac{3^{p-1} |\eps_{i+1}|^p X_i}{(n+1)} + \frac{3^{p-1} \cdot 2 ^p \cdot |\eps_{i}|^p X_i}{(n+1)} \right) \label{eq:expanding_R_odd}
\end{align}
By the strong law of large numbers (LLN), we can say that as $n \rightarrow \infty$
$$ \frac{1}{\ceil{(n-1)/2}} \sum_{i \in [n-1], i \%2=1} X_i^{p+1} \xrightarrow{\text{\text{a.s.}}} \Exp [\text{Exp(1)}^{p+1}]=\Gamma(p+2)$$
Therefore, as $n\rightarrow \infty$, for $p\geq 1$ the first term of \eqref{eq:expanding_R_odd}
\begin{equation}\label{eq:15}
 \sum_{i \in [n-1], i \%2=1}  \frac{3^{p-1} (2G)^p X_i^{p+1}}{(n+1)^{p+1}} \xrightarrow{\text{\text{a.s.}}} 0.   
\end{equation}
Similarly, using the strong LLN
\begin{align*}
    \frac{1}{\ceil{(n-1)/2}} \sum_{i \in [n-1], i \%2=1} \hspace{-5mm}3^{p-1}|\eps_{i+1}|^p X_i+ 3^{p-1} \cdot 2^p |\eps_i|^p X_i \xrightarrow{\text{\text{a.s.}}} & \,3^{p-1} \Exp \left[|\eps_{i+1}|^p X_i \right] + 3^{p-1} \cdot 2^p \Exp[ |\eps_i|^p X_i],\\
    &= 3^{p-1} \mathcal{L}_p(f^*) + 3^{p-1} \cdot 2^p \mathcal{L}_p(f^*).
\end{align*} 
Therefore, as $n \rightarrow \infty$, the second and third terms of \eqref{eq:expanding_R_odd} converge almost surely as follows.
\begin{equation}\label{eq:16}
 \sum_{i \in [n-1], i \%2=1} \left( \frac{3^{p-1} |\eps_{i+1}|^p X_i}{(n+1)} + \frac{3^{p-1} \cdot 2 ^p \cdot |\eps_{i}|^p X_i}{(n+1)} \right) \xrightarrow{\text{\text{a.s.}}} \frac{3^{p-1} \mathcal{L}_p(f^*) + 3^{p-1} \cdot 2^p \mathcal{L}_p(f^*)}{2}.   
\end{equation}
Therefore, combining \eqref{eq:15} and \eqref{eq:16} and substituting in \eqref{eq:expanding_R_odd}, as $n\rightarrow \infty$
$$\Tilde{R}_{\text{odd}} \xrightarrow{\text{a.s.}} \frac{3^{p-1} \mathcal{L}_p(f^*) + 3^{p-1} \cdot 2^p \mathcal{L}_p(f^*)}{2}.$$
Exactly following a similar argument,
$$\Tilde{R}_{\text{even}} \xrightarrow{\text{a.s.}} \frac{3^{p-1} \mathcal{L}_p(f^*) + 3^{p-1} \cdot 2^p \mathcal{L}_p(f^*)}{2}.$$
Therefore,
$$\Tilde{R} \xrightarrow{\text{\text{a.s.}}} 3^{p-1} \mathcal{L}_p(f^*) + 3^{p-1} \cdot 2^p \mathcal{L}_p(f^*)~.$$

Using $ \Tilde{R} - \hat{R}\hspace{1mm} \xrightarrow{\text{\text{a.s.}}} 0$ by Lemma \ref{lem:poison-equivalence-splines}, as $n\rightarrow \infty$, we have
$\hat{R} \xrightarrow{\text{\text{a.s.}}} 3^{p-1} \mathcal{L}_p(f^*) + 3^{p-1} \cdot 2^p \mathcal{L}_p(f^*)~$.
Finally, using the fact that $R \leq \hat{R}$, we obtain the following:
\begin{equation}\label{eq:R-bound-splines}
   \lim_{n \rightarrow \infty} \Pr_S\left[ R \leq (3^{p-1}\,(2^p + 1) +1) \, \mathcal{L}_p(f^*) \right] =1. 
\end{equation}

   Recall the definition of $R$ in \eqref{eq:def-R-splines} and $\mathcal{R}_p(\hat{g}_S)$ in \eqref{eq:lin-spline-sum-of-risk}. Having a probabilistic bound on $R$ gives us a bound on $\mathcal{R}_p(\hat{g}_S)$ when using Lemma \ref{lem:R_0+R_n-spline}. In particular, we get
\begin{equation}\label{eq:R_p(ghat)}
    \lim_{n\rightarrow \infty} \Pr_S \left[ \mathcal{R}_p(\hat{g}_S) \leq \, (3^{p-1}\,(2^p + 1) +2) \, \mathcal{L}_p(f^*) \right]=1.
\end{equation}
Finally, combining this with \eqref{eq:L_p-R_p-splines}:
$$ \lim_{n \rightarrow \infty} \Pr_S \left[ \mathcal{L}_p(\hat{g}_S) \leq \, C_p \, \mathcal{L}_p(f^*) \right]=1,$$
where $C_p:=2^{p-1}[3^{p-1}((2^{p}+1)+2)+1]$. 
 \end{proof}
We now prove Lemmas \ref{lem:R_0+R_n-spline} and \ref{lem:poison-equivalence-splines} in order.

\begin{proof}[Proof of Lemma \ref{lem:R_0+R_n-spline}]
    For any $x \in [0,x_1]$, we have $\hat{g}_S(x)=y_1$. Therefore,
    \begin{align*}
        |\hat{g}_S(x)-f^*(x)|&=|y_1-f^*(x)|=|f^*(x_1)+\eps_1-f^*(x)| \leq G|x_1-x| +|\eps_1| \leq G \ell_0+|\eps_1|.
    \end{align*}
    The implies that
        \begin{align*}
             R_0 & =\int_0^{x_1} |\hat{g}_S(x)-f^*(x)|^p \, dx \leq (G\ell_0+|\eps_1|)^p \ell_0 \leq 2^{p-1}G^p \ell_0^{p+1} + 2^{p-1} |\eps_1|^p \ell_0. 
    \end{align*}
 Similarly, for $x \in [x_n,1]$
      \begin{align*}
        |\hat{g}_S(x)-f^*(x)|&=|y_n-f^*(x)|=|f^*(x_n)+\eps_n-f^*(x)| \leq G|x_n-x|+|\eps_n| \leq G \ell_n+|\eps_n|.
    \end{align*}
    Therefore
        \begin{align*}
        R_n & = \int_{x_n}^{1} |\hat{g}_S(x)-f^*(x)|^p \, dx \leq (G \ell_n+|\eps_n|)^p \ell_n \leq 2^{p-1} G^p \ell_n^{p+1}+2^{p-1} |\eps_n|^p \ell_n. 
    \end{align*}
    Combining the two we get
    \begin{align*}
        \Exp_S [R_0+R_n] & \leq 2^{p-1} G^p \cdot (\Exp[\ell_0^{p+1}]+\Exp[\ell_n^{p+1}])+2^{p-1} (\Exp[|\eps_1|^p \ell_0]+\Exp[|\eps_n|^p\ell_n])\\
        &= 2^p G^p \Exp[\ell_0^{p+1}] + 2^p \mathcal{L}_p(f^*) \Exp[\ell_0] = 2^p G^p \Exp[\ell_0^{p+1}] + \frac{2^p \mathcal{L}_p(f^*)}{n+1} \\
        & \leq 2^p G^p \cdot \Exp \left[ \frac{X_0^{2}}{(X_1+\dots+X_n)^{2}} \right]+\frac{2^p \mathcal{L}_p(f^*)}{n+1} \\
        &=2^p G^p \cdot \Exp[\text{Exp}(1)^{2}] \cdot \Exp \left[ \frac{1}{\Gamma(n,1)^{2}} \right] + o_n(1)\\
        &= 2^p G^p \cdot 2 \cdot \int_{0}^{\infty} \frac{1}{z^2} \cdot \frac{ z^{n-1}\cdot e^{-z} }{\Gamma(n)} \, dz +o_n(1)=  \frac{2^{p+1} G^p}{\Gamma(n)} \cdot \int_{0}^{\infty} z^{n-3} \cdot e^{-z} \, dz +o_n(1)\\
        &=  \frac{2^{p+1}G^p \Gamma(n-2)}{\Gamma(n)} +o_n(1)= \frac{2^{p+1} G^p}{(n-1)(n-2)}+o_n(1) = o_n(1).\\
    \end{align*} 
Therefore, applying Markov's inequality yields that for any $\gamma>0$,
$$\Pr_S[R_0+R_n >\gamma] \leq \hspace{2mm} \frac{\Exp[R_0+R_n]}{\gamma} \leq \hspace{2mm} o_n(1),$$
and the lemma follows. 
\end{proof}
\begin{proof}[Proof of Lemma \ref{lem:poison-equivalence-splines}]
    By \eqref{eq:ell_distribution}, we have $\left( \ell_0,\dots,\ell_n \right) \sim  \left( \frac{X_0}{X}, \dots, \frac{X_n}{X} \right)$, where $X_0,\dots,X_n \iid \text{Exp}(1)$ and $X:=\sum_{i=0}^{n} X_i$. Also, we have $\tilde{\ell}_i = \frac{X_i}{n+1}$ for all $i$. 
    
    For every $p \geq 1$ we have
    \begin{align*}
        \sum_{i=1}^{n-1} \left( \tilde{\ell}_i^{p+1} - \ell_i^{p+1}  \right)
        &= \sum_{i=1}^{n-1} \left( \frac{X_i^{p+1}}{(n+1)^{p+1}} -  \frac{X_i^{p+1}}{X^{p+1}} \right)
        \\
        &= \sum_{i=1}^{n-1} \frac{X_i^{p+1}}{(n+1)^{p+1}} \left( 1 - \frac{(n+1)^{p+1}}{X^{p+1}} \right)
        \\
        &= \frac{n-1}{(n+1)^{p+1}} \cdot \frac{ \sum_{i=1}^{n-1} X_i^{p+1}}{n-1} \left[ 1 - \left(\frac{n+1}{X}\right)^{p+1} \right]~,
    \end{align*}
    and by the strong law of large numbers we have $\frac{ \sum_{i=1}^{n-1} X_i^{p+1}}{n-1} \xrightarrow{\text{a.s.}} \Exp X_i^{p+1} < \infty$ and $\left(\frac{n+1}{X}\right)^{p+1} \xrightarrow{\text{a.s.}} 1$, and thus 
    \[
        \sum_{i=1}^{n-1} \left( \tilde{\ell}_i^{p+1} - \ell_i^{p+1}\right) \xrightarrow{\text{a.s.}} 0~.
    \]

    Moreover, we have
    \begin{align*}
        \sum_{i=1}^{n-1} \left( \tilde{\ell}_i |\epsilon_i|^p - \ell_i |\epsilon_i|^p  \right)
        &= \sum_{i=1}^{n-1} \left( \frac{X_i |\epsilon_i|^p}{n+1} -  \frac{X_i |\epsilon_i|^p}{X} \right)
        \\
        &= \sum_{i=1}^{n-1} \frac{X_i |\epsilon_i|^p}{n+1} \left( 1 - \frac{n+1}{X} \right)
        \\
        &= \frac{n-1}{n+1} \cdot \frac{ \sum_{i=1}^{n-1} X_i |\epsilon_i|^p}{n-1} \left( 1 - \frac{n+1}{X} \right)~,
    \end{align*}
    and by the strong law of large numbers we have $\frac{ \sum_{i=1}^{n-1} X_i|\epsilon_i|^p}{n-1} \xrightarrow{\text{a.s.}} \Exp X_i|\epsilon_i|^p < \infty$ and $\frac{n+1}{X} \xrightarrow{\text{a.s.}} 1$, and thus 
    \[
        \sum_{i=1}^{n-1} \left( \tilde{\ell}_i |\epsilon_i|^p - \ell_i |\epsilon_i|^p  \right) \xrightarrow{\text{a.s.}} 0~.
    \]
    Similarly, we also have 
    \[
        \sum_{i=1}^{n-1} \left( \tilde{\ell}_i |\epsilon_{i+1}|^p - \ell_i |\epsilon_{i+1}|^p  \right) \xrightarrow{\text{a.s.}} 0~.
    \]
    Overall, we get
    \begin{align*}
        &\tilde{R} - \hat{R} 
        \\
        &= \sum_{i=1}^{n-1} \left(3^{p-1} (2G)^p \tilde{\ell}_i^{p+1} + 3^{p-1} \tilde{\ell}_i |\eps_{i+1}|^p + 3^{p-1}\cdot 2^p \cdot \tilde{\ell}_i |\eps_{i}|^p 
        \right.
        \\
        &\quad\quad\quad
        \left.
        - 3^{p-1} (2G)^p \ell_i^{p+1} - 3^{p-1} \ell_i |\eps_{i+1}|^p - 3^{p-1} \cdot 2^p \cdot \ell_i |\eps_{i}|^p \right)       \\
        &= 3^{p-1} (2G)^p \sum_{i=1}^{n-1} \left(\tilde{\ell}_i^{p+1} - \ell_i^{p+1} \right) + 3^{p-1} \sum_{i=1}^{n-1} \left(\tilde{\ell}_i |\eps_{i+1}|^p - \ell_i |\eps_{i+1}|^p\right) + 3^{p-1} \cdot 2^{p} \sum_{i=1}^{n-1} \left( \tilde{\ell}_i |\eps_{i}|^p - \ell_i |\eps_{i}|^p \right)
        \\
        &\xrightarrow{\text{a.s.}} 0~.
    \end{align*}
    
\end{proof}
    

