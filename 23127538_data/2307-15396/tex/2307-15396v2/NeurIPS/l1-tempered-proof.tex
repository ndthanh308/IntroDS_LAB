\section{Proof of Theorem \ref{thm:tempered-for-ell_1}}
 \begin{proof}[Proof of Theorem \ref{thm:tempered-for-ell_1}] 
Let $f^*$ be $G$-Lipschitz. We sample $S\sim \dist^n$ and number points $(x_i,y_i)$ from left to right based on the $x$-coordinate. Again, we denote $x_0:=0$ and $x_{n+1}:=1$ for notational convenience (they are not used in determining $\hat{f}_S$). The domain $[0,1]$ is divided into $n+1$ intervals of length $\ell_0,\ell_1, \dots,\ell_n$, where $\ell_i=x_{i+1}-x_i$. We want to analyze the population $L_p$ loss of the min-norm interpolator $\hat{f}_S$ as defined in \eqref{eq:main_problem} for $p\in [1,2)$. Exactly following the step for the derivation of \eqref{eq:L_p-R_p-grid}, we get
    \begin{align}
  \mathcal{L}_p(\hat{f}_S) 
  & \leq  2^{p-1} ( \mathcal{R}_p(\hat{f}_S)+ \mathcal{L}_p(f^*) ), \label{eq:L_p-R_p-p(1,2)}
\end{align}
where $\mathcal{R}_p(\hat{f}_S)$ is defined and simplified as the following.
\begin{align}
    \mathcal{R}_p(\hat{f}_S) &= \int_{x_0=0}^{x_{n+1}=1} \hspace{-2mm} |\hat{f}_S(x)-f^*(x)|^p \, dx \nonumber\\
    & = \sum_{i=0}^{n} \int_{x_i}^{x_{i+1}} \hspace{-2mm} |\hat{f}_S(x)-f^*(x)|^p \, dx \nonumber\\
    & \leq \int_{0}^{x_2} |g_1(x)-f^*(x)|^p \, dx + \int_{x_{n-1}}^{1} |g_{n-1}(x)-f^*(x)|^p \, dx \nonumber\\ & \hspace{10mm} + \sum_{i=2}^{n-2} \int_{x_i}^{x_{i+1}} \hspace{-2mm} \max\{|g_i(x)-f^*(x)|^p,\min\{|g_{i-1}(x)-f^*(x)|^p, |g_{i+1}(x)-f^*(x)|^p\} \} \, dx \nonumber \tag{by Lemma \ref{lem:bound on |f^-f*|}} \\
    & \leq \int_{0}^{x_2} |g_1(x)-f^*(x)|^p \, dx + \int_{x_{n-1}}^{1} |g_{n-1}(x)-f^*(x)|^p \, dx  \nonumber\\ & \hspace{10mm} + \sum_{i=2}^{n-2} \int_{x_i}^{x_{i+1}} \hspace{-2mm} |g_i(x)-f^*(x)|^p+\min\{|g_{i-1}(x)-f^*(x)|^p, |g_{i+1}(x)-f^*(x)|^p\}  \, dx \nonumber \\
     & \leq \underbrace{\int_{0}^{x_1} |g_1(x)-f^*(x)|^p \, dx}_{:=R_0} + \underbrace{\int_{x_{n}}^{1} |g_{n-1}(x)-f^*(x)|^p \, dx}_{:=R_n} + \sum_{i=1}^{n-1} \int_{x_i}^{x_{i+1}} \hspace{-2mm} |g_i(x)-f^*(x)|^p \, dx \nonumber\\ & \hspace{10mm} + \underbrace{\sum_{i=2}^{n-2} \int_{x_i}^{x_{i+1}} \min\{|g_{i-1}(x)-f^*(x)|^p, |g_{i+1}(x)-f^*(x)|^p\} \, dx}_{:=R}  \nonumber \\
     &\leq R_0+R_n+\mathcal{R}_p(\hat{g}_S)+R.  \label{eq:linearspline+R}
\end{align}
The following lemma, which we prove after the theorem, establishes that $R_0+R_n$ is vanishing with high probability.
\begin{lemma}\label{lem:R_0+R_n: ell 1-2}
    For any $\gamma>0$, we have $\lim_{n \rightarrow \infty} \Pr_S[ R_0 +R_n \leq \gamma]=1.$
\end{lemma}
Moreover, we have already bounded $\mathcal{R}_p(\hat{g}_S)$ in \eqref{eq:R_p(ghat)}; in fact for any $p \geq 1$. When $p\in[1,2)$, it reduces to:
\begin{equation}\label{eq:22}
    \lim_{n \rightarrow \infty} \Pr_S \left[ \mathcal{R}_p(\hat{g}_S) \leq 17 \, \mathcal{L}_p(f^*) \right] =1.
\end{equation}
We now focus on bounding $R$. Intuitively, $R$ is the risk term caused due to spike formation. This is only bounded for $p\in [1,2)$ and grows as $p$ approaches 2. For $i\in \{2,\dots, n-2\}$, define:
$$R_i:= \int_{x_i}^{x_{i+1}} \hspace{-2mm} \min \{ |g_{i-1}(x)-f^*(x)|^p,|g_{i+1}(x)-f^*(x)|^p\} \, dx.$$
Then $R=\sum_{i=2}^{n-2} R_i$. Moreover, each $R_i$ can be simplified as the following. For any $i\in \{2,\dots,n-2\}$ and any $x\in [x_{i},x_{i+1}]$, 
\begin{align}
    |g_{i-1}(x)-f^*(x)|=&\left| y_{i}+ \frac{(y_{i}-y_{i-1})}{x_{i}-x_{i-1}} (x-x_{i})-f^*(x)\right| \nonumber\\
 =& \left| f^*(x_{i})+\eps_{i} + \left(\frac{f^*(x_{i})+\eps_{i}-f^*(x_{i-1})-\eps_{i-1}}{x_{i}-x_{i-1}}\right)(x-x_{i}) -f^*(x)\right| \nonumber\\
 \leq & \left| f^*(x_{i}) -f^*(x)\right|+ |\eps_{i}|+  \frac{G(x_{i}-x_{i-1})+|\eps_{i}-\eps_{i-1}|}{x_{i}-x_{i-1}}   |(x-x_{i})|  \nonumber\\
 \leq &  G(x-x_{i}) + |\eps_{i}|+ \left( \frac{G(x_{i}-x_{i-1})+|\eps_{i}|+|\eps_{i-1}|}{x_{i}-x_{i-1}} \right)  (x-x_{i})  \nonumber\\
 \leq & G \cdot \ell_i + |\eps_{i}|+ \left( \frac{G \cdot \ell_{i-1} +|\eps_{i}|+|\eps_{i-1}|}{\ell_{i-1}} \right) \cdot \ell_i \nonumber\\
 = & 2G \cdot \ell_i + |\eps_i|+ ( |\eps_{i}|+|\eps_{i-1}|) \frac{\ell_i}{\ell_{i-1}} \label{eq:left-side-bound}
\end{align}

\begin{align}
    |g_{i+1}(x)-f^*(x)|=&\left| y_{i+1}+ \frac{(y_{i+2}-y_{i+1})}{x_{i+2}-x_{i+1}} (x-x_{i+1})-f^*(x)\right| \nonumber\\
 =& \left| f^*(x_{i+1})+\eps_{i+1} + \left(\frac{f^*(x_{i+2})+\eps_{i+2}-f^*(x_{i+1})-\eps_{i+1}}{x_{i+2}-x_{i+1}}\right)(x-x_{i+1}) -f^*(x)\right| \nonumber\\
 \leq & \left| f^*(x_{i+1}) -f^*(x)\right|+ |\eps_{i+1}|+  \frac{G(x_{i+2}-x_{i+1})+|\eps_{i+2}-\eps_{i+1}|}{x_{i+2}-x_{i+1}}   |x-x_{i+1}|  \nonumber\\
 \leq &  G|x-x_{i+1}| + |\eps_{i+1}|+ \left( \frac{G(x_{i+2}-x_{i+1})+|\eps_{i+2}|+|\eps_{i+1}|}{x_{i+2}-x_{i+1}} \right)  |x-x_{i+1}|  \nonumber\\
 \leq & G \cdot \ell_i + |\eps_{i+1}|+ \left( \frac{G \cdot \ell_{i+1} +|\eps_{i+2}|+|\eps_{i+1}|}{\ell_{i+1}} \right) \cdot \ell_i \nonumber\\
 = & 2G \cdot \ell_{i} + |\eps_{i+1}|+( |\eps_{i+1}|+|\eps_{i+2}|) \frac{\ell_i}{\ell_{i+1}} \label{eq:right-side-bound}
\end{align}

\begin{align}
    R_i&= \int_{x_i}^{x_{i+1}} \hspace{-2mm} \min \{ |g_{i-1}(x)-f^*(x)|^p,|g_{i+1}(x)-f^*(x)|^p\} \, dx \nonumber\\
    & \leq \int_{x_i}^{x_{i+1}} \min \left\{ 2G \cdot \ell_i + |\eps_i|+(|\eps_{i}|+|\eps_{i-1}|) \frac{\ell_i}{\ell_{i-1}}, 2G \cdot \ell_i + |\eps_{i+1}|+(|\eps_{i+1}|+|\eps_{i+2}|) \frac{\ell_i}{\ell_{i+1}} \right\}^p \, dx \tag{using \eqref{eq:left-side-bound} and \eqref{eq:right-side-bound}}\\
    & \leq \left( 2G\ell_i+ |\eps_i|+|\eps_{i+1}|+(|\eps_i|+|\eps_{i+1}|+|\eps_{i-1}|+|\eps_{i+2}|) \frac{\ell_i}{\max\{\ell_{i-1},\ell_{i+1}\}} \right)^p \ell_i, \nonumber\\
    & \leq 4^{p-1} \left( (2G\ell_i)^p+ |\eps_{i}|^p+|\eps_{i+1}|^p+ \left( (2|\eps_i|+2|\eps_{i+1}|+|\eps_{i-1}|+|\eps_{i+2}|) \frac{\ell_i}{\max\{\ell_{i-1},\ell_{i+1}\}} \right)^p \right) \ell_i \nonumber\\
    & = 2^{3p-2} G^p \ell_i^{p+1} + 4^{p-1} |\eps_i|^p  \ell_i+ 4^{p-1} |\eps_{i+1}|^p \ell_i+4^{p-1} (|\eps_i|+|\eps_{i+1}|+|\eps_{i-1}|+|\eps_{i+2}|)^p \frac{\ell_i^{p+1}}{\max\{\ell_{i-1},\ell_{i+1}\}^p}  \nonumber
    \\
    &:=\hat{R}_i\nonumber\\
\end{align}
Then $\hat{R}:=\sum_{i=2}^{n-2} \hat{R}_i$ is an upper bound on $R$. The random variables $\ell_i$ s are mildly dependent. To achieve independence, we denote $\Tilde{\ell}_0, \Tilde{\ell}_1, \dots, \Tilde{\ell}_n \iid \text{Exp}(1)/(n+1)$; in particular, $\Tilde{\ell}_i=\nicefrac{X_i}{n+1}$. We replace $\ell_i$ with $\Tilde{\ell}_i$ and define random variables $\Tilde{R}_i$.
\begin{align*}
    &\Tilde{R}_i= 2^{3p-2} G^p \Tilde{\ell}_i^{p+1} + 4^{p-1}(|\eps_i|^p+|\eps_{i+1}|^p) \Tilde{\ell}_i+4^{p-1} (|\eps_i|+|\eps_{i+1}|+|\eps_{i-1}|+|\eps_{i+2}|)^p \frac{{\Tilde{\ell}_i}^{p+1}}{\max\{\Tilde{\ell}_{i-1},\Tilde{\ell}_{i+1}\}^p},
    \text{ and} \\
    &\Tilde{R}=\sum_{i=2}^{n-2} \Tilde{R}_i.
\end{align*}
We claim that:
\begin{lemma} \label{lem:Rhat-Rtilde-as-ell1}
    As $n \rightarrow \infty$, we have $\hat{R}-\Tilde{R}\xrightarrow{\textnormal{a.s.}} $0\, .
\end{lemma}
The proof is similar to Lemma \ref{lem:poison-equivalence-splines}. Therefore, it suffices to give a bound on $\Tilde{R}$. Still, any four consecutive $\Tilde{R}_i$'s are dependent. Therefore, we re-express $\Tilde{R}$ into four disjoint sums such that each individual of them is the sum of only i.i.d. random variables. We divide the indices into four sets $\mathcal{I}_j$ for $0 \leq j \leq 3$. Define $\mathcal{I}_j=\{ i\%4=j: 2 \leq i \leq (n-2) \}$ for $0 \leq j \leq 3$ and 
$$\Tilde{R}^{(j)}:= \sum_{i \in \mathcal{I}_j} \Tilde{R}_i. \quad \text{Then } \quad \Tilde{R}=\sum_{j=0}^{3} \Tilde{R}^{(j)}.$$ 
Then for any $0 \leq j \leq 3$, further simplifying
\begin{align}
    \Tilde{R}^{(j)}&=\sum_{i \in \mathcal{I}_j} 2^{3p-2} G^p \Tilde{\ell}_i^{p+1} + 4^{p-1}(|\eps_i|^p+|\eps_{i+1}|^p) \Tilde{\ell}_i+ 4^{p-1} (|\eps_i|+|\eps_{i+1}|+|\eps_{i-1}|+|\eps_{i+2}|)^p \frac{{\Tilde{\ell}_i}^{p+1}}{\max\{\Tilde{\ell}_{i-1},\Tilde{\ell}_{i+1}\}^p} \nonumber \\
    & =  \underbrace{\frac{1}{(n+1)^{p+1}}\sum_{i \in \mathcal{I}_j} 2^{3p-2} G^p X_i^{p+1}}_{:=T_1} + \underbrace{\frac{1}{(n+1)}\sum_{i \in \mathcal{I}_j} 4^{p-1} ( |\eps_i|^p+|\eps_{i+1}|^p) X_i }_{:=T_2}\nonumber\\
    & \hspace{2mm}+ \underbrace{\frac{1}{(n+1)}\sum_{i \in \mathcal{I}_j} 4^{p-1} (|\eps_i|+|\eps_{i+1}|+|\eps_{i-1}|+|\eps_{i+2}|)^p \frac{X_i^{p+1}}{\max\{X_{i-1},X_{i+1}\}^p}}_{:=T_3} \label{eq:stepbeforeLLN}
\end{align}
Now, each $T_1, T_2$ and $T_3$ is the average of i.i.d. random variables (up to scaling). Thus, as $n\rightarrow \infty$, by the strong LLN,
$$ \frac{1}{\card{\mathcal{I}_j}} \sum_{i \in \mathcal{I}_j} 2^{3p-2} G^p X_i^{p+1} \xrightarrow{\text{a.s.}} 2^{3p-2} G^p \Exp[X_i^{p+1}]= 2^{3p-2} G^p \Gamma(p+2) < \infty.$$
Therefore, using the fact $\lim_{n \rightarrow \infty} \frac{\card{\mathcal{I}_j}}{(n+1)}= \frac{1}{4}$ and since we are considering that $p \geq 1$ we get that the first term of \eqref{eq:stepbeforeLLN} converges to 0 almost surely, i.e. as $n \rightarrow \infty$
\begin{equation}\label{eq:0-convergence}
  T_1=  \frac{1}{(n+1)^{p+1}}\sum_{i \in \mathcal{I}_j} 2^{3p-2} G^p X_i^{p+1} \xrightarrow{\text{a.s.}} 0.
\end{equation}
Similarly, since $\Exp[X_i]=1$ and $\Exp[|\eps|^p]=\mathcal{L}_p(f^*)<\infty$ even $T_2$ (up to scaling) is the average i.i.d. random variables, with finite expectation. Using the strong law of large numbers, as $n \rightarrow \infty$ the second term of \eqref{eq:stepbeforeLLN}
\begin{equation}\label{eq:t2-convergence}
  T_2=  \frac{1}{(n+1)}\sum_{i \in \mathcal{I}_j} 4^{p-1} (|\eps_i|^p+|\eps_{i+1}|^p)X_i  \xrightarrow{\text{a.s.}} \frac{1}{4} \cdot 4^{p-1}\Exp[(|\eps_i|^p+|\eps_{i+1}|^p)X_i]= \frac{4^{p-1}}{2} \mathcal{L}_p(f^*).
\end{equation}
Finally, $T_3$ is also the average of i.i.d random variables (up to scaling). Before applying the strong LLN, we must verify if each summand has a finite expectation. Since $\Exp[|\eps|^p]=\mathcal{L}_p(f^*)<\infty$ and $\Exp[X_i^{p+1}]=\Gamma(p+2)<\infty$, it is easy to observe that it boils down to verifying if $\Exp[\frac{1}{\max\{X_{i-1}, X_{i+1}\}^p}]$ has a finite expectation. The following claim verifies this.
\begin{claim}\label{clm:maxA,B}
 Let $A,B \iid \text{Exp}(1)$, then $\Exp\left[ \frac{1}{\max\{A,B\}^p}\right] \leq \frac{2^p}{(2-p)} $.
\end{claim} 
Therefore, applying the strong LLN in exactly the same way:
\begin{align}
T_3=&\frac{1}{(n+1)}\sum_{i \in \mathcal{I}_j} 4^{p-1} (|\eps_i|+|\eps_{i+1}|+|\eps_{i-1}|+|\eps_{i+2}|)^p \frac{X_i^{p+1}}{\max\{X_{i-1},X_{i+1}\}^p}  \xrightarrow{\text{a.s.}} \nonumber\\
& \frac{1}{4} \cdot \Exp\left[ 4^{p-1} (|\eps_i|+|\eps_{i+1}|+|\eps_{i-1}|+|\eps_{i+2}|)^p \frac{X_i^{p+1}}{\max\{X_{i-1},X_{i+1}\}^p}\right] \nonumber\\ 
 & \leq \frac{4^{p-1} \, 4^{p-1} (\Exp[|\eps_i|^p+|\eps_{i+1}|^p+|\eps_{i-1}|^p+|\eps_{i+2}|^p)]}{4}  \, \Gamma(p+2) \cdot \Exp \left[ \frac{1}{\max\{X_{i-1},X_{i+1}\}^p}\right] \nonumber\\
& \leq \frac{2^{4p-4} \mathcal{L}_p(f^*)\Gamma(p+2) 2^p}{(2-p)} ,\label{eq:expectation-TildeRi}
\end{align}
where in the last inequality, we use Claim \ref{clm:maxA,B}. Using \eqref{eq:expectation-TildeRi}, \eqref{eq:0-convergence} and \eqref{eq:t2-convergence}, we get a high probability bound on $\Tilde{R}^{(j)}$ (recall definition in \eqref{eq:stepbeforeLLN}). In particular, for $0 \leq j \leq 3$

$$\lim_{n\rightarrow \infty} \Pr_S \left[ \Tilde{R}^{(j)} \leq \left(\frac{4^{p-1}}{2}+ \frac{2^{4p-4} \cdot 2^p \cdot \Gamma(p+2)}{2-p}+1\right) \mathcal{L}_p(f^*) \right]=1 \, .$$
Using the fact that we are considering $p \in [1,2)$, we get
$$\lim_{n\rightarrow \infty} \Pr_S \left[ \Tilde{R}^{(j)} \leq  \left( 2 + \frac{384}{2-p}+1 \right) \mathcal{L}_p(f^*) \right]=1 \, ,$$
$$\implies \lim_{n\rightarrow \infty} \Pr_S \left[ \Tilde{R}^{(j)} \leq   \frac{387\mathcal{L}_p(f^*)}{2-p}  \right]=1 \, . $$
Therefore, since $\Tilde{R}=\sum_{j=0}^{3} \Tilde{R}^{(j)}$ we obtain 
$$\lim_{n\rightarrow \infty} \Pr_S \left[ \Tilde{R} \leq \frac{1548\, \mathcal{L}_p(f^*)}{(2-p)} \right]=1 \, .$$
Using this along with Lemma \ref{lem:Rhat-Rtilde-as-ell1} and the fact that $R \leq \hat{R}$ gives us $$\lim_{n\rightarrow \infty} \Pr_S \left[ R \leq \frac{1549\, \mathcal{L}_p(f^*)}{(2-p)} \right]=1 \,.$$
Further substituting this in the bounds from \eqref{eq:linearspline+R} and using Lemma \ref{lem:R_0+R_n: ell 1-2} and \eqref{eq:22}
$$\lim_{n \rightarrow \infty} \Pr_S \left[ \mathcal{R}_p(\hat{f}_S) \, \leq \frac{1567 \mathcal{L}_p(f^*)}{(2-p)} \right]=1.$$
Finally, using \eqref{eq:L_p-R_p-p(1,2)} with the above
$$\lim_{n \rightarrow \infty} \Pr_S \left[ \mathcal{L}_p(\hat{f}_S) \, \leq \frac{3136 \mathcal{L}_p(f^*)}{(2-p)} \right]=1.$$
As such, by letting $C=3136$, the theorem follows.\qed{ Theorem \ref{thm:tempered-for-ell_1}}

\end{proof}
We now prove the lemmas and claims we used in the above proof. 
\begin{proof}[Proof of Lemma \ref{lem:R_0+R_n: ell 1-2}]
    For any $x \in [0,x_1]$, we have
 \begin{align*}
|g_1(x)-f^*(x)|
     =&\left| y_{1}+ \frac{(y_{2}-y_{1})}{x_{2}-x_{1}} (x-x_{1})-f^*(x)\right|\\
     =& \left| f^*(x_{1})+\eps_{1} + \left(\frac{f^*(x_{2})+\eps_{2}-f^*(x_{1})-\eps_{1}}{x_{2}-x_{1}}\right)(x-x_{1}) -f^*(x)\right| \\
     \leq & \left| f^*(x_{1}) -f^*(x)\right|+ |\eps_{1}|+ \left| \frac{G(x_{2}-x_{1})+|\eps_{2}-\eps_{1}|}{x_{2}-x_{1}} \right|  |x-x_{1}|  \\
    \leq &  G|x-x_{1}| + |\eps_{1}|+ \left( \frac{G(x_{2}-x_{1})+|\eps_{2}|+|\eps_{1}|}{x_{2}-x_{1}} \right)  |x-x_{1}|  \\
    \leq & G \cdot \ell_0 + |\eps_{1}|+ \left( \frac{G \cdot \ell_1 +|\eps_{2}|+|\eps_{1}|}{\ell_1} \right) \cdot \ell_0\\
    = & 2G \cdot \ell_0 + |\eps_1|+(|\eps_{2}| + |\eps_{1}|) \frac{\ell_0}{\ell_1}.
    \end{align*}
    Therefore,
    \begin{align}
        R_0 &= \int_{0}^{x_1} \hspace{-2mm} |g_1(x)-f^*(x)|^p \, dx  \leq \ell_0 \left[ 2G \cdot \ell_0 +  |\eps_1|+(|\eps_{2}| + |\eps_{1}|) \frac{\ell_0}{\ell_1}\right]^p \nonumber\\
        &\leq 4^{p-1}\cdot (2G)^p \cdot \ell_0^{p+1} + 4^{p-1} |\eps_1|^p \ell_0 +4^{p-1}|\eps_2|^p \frac{\ell_0^{p+1}}{\ell_1^p} + 4^{p-1} \cdot |\eps_1|^p  \cdot \frac{\ell_0^{p+1}}{\ell_1^p} \nonumber\\
        &= 4^{p-1}\cdot (2G)^p \cdot \frac{X_0^{p+1}}{(X_0+\dots+X_n)^{p+1}}+ 4^{p-1} |\eps_1|^p \frac{X_0}{(X_0+\dots+X_n)} \nonumber \\
        & \hspace{5mm}+(4^{p-1}\cdot |\eps_2|^p +4^{p-1} \cdot |\eps_1|^p)  \cdot \frac{X_0^{p+1}}{X_1^p \cdot (X_0+\dots+X_n)} \label{eq:high-probability-bound-splines}
    \end{align}
We now provide probabilistic upper bounds on $X_0$, $|\eps_2|^p$ and $|\eps_1|^p$, and lower bounds on $X_1$ and $X_0+\dots+X_n$, which suffices to further upper bound $R_0$.
\begin{align*}
    \Pr[X_0 \leq 3 \ln n] = 1-\Pr[X_0 > 3 \ln n]=1-\int_{3\ln n}^{\infty} e^{-z} \, dz = 1-\frac{1}{n^3}.
\end{align*}
Using Markov's inequality,
\begin{align*}
    \Pr[ |\eps_1|^p \leq \mathcal{L}_p(f^*)\, \ln n  ] = 1-\Pr[ |\eps_1|^p > \mathcal{L}_p(f^*)\, \ln n ] \geq 1-\frac{\Exp[|\eps_1|^p]}{\mathcal{L}_p(f^*) \ln n} = 1-\frac{1}{\ln n}.
\end{align*}
Similarly, 
\begin{align*}
    \Pr[ |\eps_2|^p \leq \mathcal{L}_p(f^*)\, \ln n  ] \geq 1-\frac{1}{\ln n}.
\end{align*}
\begin{align*}
    \Pr[ X_1 \geq \frac{1}{\ln n}] = \int_{\frac{1}{\ln n}}^{\infty} e^{-z} \, dz = e^{-\frac{1}{\ln n}} \geq 1-\frac{1}{\ln n},
\end{align*}
where the last inequality follows from the Taylor expansion of $e^z$.
Finally, as $n \rightarrow \infty$, by the strong Law of Large Numbers, $ \frac{1}{n} \sum_{i=0}^{n}X_i \xrightarrow{\text{a.s.}} 1$. This implies
$$\Pr\left[ \sum_{i=0}^{n} X_i \geq \frac{n}{2}\right]=1-o_n(1).$$
Doing a union bound over these events, and substituting these probabilistic bounds in \eqref{eq:high-probability-bound-splines}, we get that with probability $1-o_n(1)$,
\begin{align*}
    R_0 \leq \frac{4^{p-1} \cdot (2G)^p (3\ln n)^{p+1}  }{(\nicefrac{n}{2})^{p+1}}+ \frac{4^{p-1} \mathcal{L}_p(f^*) \ln n \cdot 3 \ln n}{(\nicefrac{n}{2})}+\frac{2 \cdot 4^{p-1} \mathcal{L}_p(f^*) \ln n \cdot (3 \ln n)^{p+1}}{ \left(\nicefrac{1}{\ln^p n}\right) \cdot (\nicefrac{n}{2})}=o_n(1).
\end{align*}
Following the exact same steps,  one can say that with probability $1-o_n(1)$, we have $R_n=o_n(1)$. Therefore, doing the union bound and combining both, we get that with probability $1-o_n(1)$ we have $R_0+R_n=o_n(1)$. This implies, for any $\gamma>0$,
$$\lim_{n\rightarrow \infty} \Pr_S \left[ R_0+R_n \leq \gamma \right]=1. \qed{ \text{ Lemma } \ref{lem:R_0+R_n: ell 1-2}}$$

\end{proof}
\begin{proof}[Proof of Lemma \ref{lem:Rhat-Rtilde-as-ell1}]
    By \eqref{eq:ell_distribution}, we have $\left( \ell_0,\dots,\ell_n \right) \sim  \left( \frac{X_0}{X}, \dots, \frac{X_n}{X} \right)$, where $X_0,\dots,X_n \iid \text{Exp}(1)$ and $X:=\sum_{i=0}^{n} X_i$. Also, we have $\tilde{\ell}_i = \frac{X_i}{n+1}$ for all $i$. 

Therefore,
\begin{align*}
   & \sum_{i=2}^{n-2} 4^{p-1} (|\eps_i|+|\eps_{i+1}|+|\eps_{i-1}|+|\eps_{i+2}|)^p \left( \frac{{\ell_i}^{p+1}}{\max\{\ell_{i-1},\ell_{i+1}\}^p}- \frac{{\Tilde{\ell}_i}^{p+1}}{\max\{\Tilde{\ell}_{i-1},\Tilde{\ell}_{i+1}\}^p} \right) \nonumber\\
   &= \sum_{i=2}^{n-2} 4^{p-1} (|\eps_i|+|\eps_{i+1}|+|\eps_{i-1}|+|\eps_{i+2}|)^p \cdot  \frac{{X_i}^{p+1}}{\max\{X_{i-1},X_{i+1}\}^p} \left(\frac{1}{X}-\frac{1}{n+1} \right) \nonumber\\
  &= \underbrace{\frac{1}{(n+1)}\sum_{i=2}^{n-2} 4^{p-1} (|\eps_i|+|\eps_{i+1}|+|\eps_{i-1}|+|\eps_{i+2}|)^p \cdot  \frac{{X_i}^{p+1}}{\max\{X_{i-1},X_{i+1}\}^p}}_{:=T} \left(\frac{n+1}{X}-1 \right)
\end{align*}
   By the strong law of large numbers, as $n \rightarrow \infty$, $\frac{n+1}{X} \xrightarrow{\text{a.s.}} 1$. Thus, $(\frac{n+1}{X}-1) \xrightarrow{\text{a.s.}}0$. Also, $T$ converges 
   almost surely 
   to a finite quantity
   as $n \rightarrow \infty$. To see this, we split $T$ into four disjoint sums, one each over indices in $\mathcal{I}_j=\{i : i\%4=j\}$ for $0 \leq j \leq 3 $. And each of them 
   converges almost surely to a
   finite quantity by \eqref{eq:expectation-TildeRi}, and hence also the overall sum $T$. Therefore, as $n \rightarrow \infty$
   \begin{equation} \label{eq:a.s._lemma_app_c_1}
    \sum_{i=2}^{n-2} 4^{p-1} (|\eps_i|+|\eps_{i+1}|+|\eps_{i-1}|+|\eps_{i+2}|)^p \left( \frac{{\ell_i}^{p+1}}{\max\{\ell_{i-1},\ell_{i+1}\}^p}- \frac{{\Tilde{\ell}_i}^{p+1}}{\max\{\Tilde{\ell}_{i-1},\Tilde{\ell}_{i+1}\}^p} \right) \xrightarrow{\text{a.s.}} 0~.
   \end{equation}

   Next, we have
   \begin{align*}
       \sum_{i=2}^{n-2} 2^{3p-2} G^p \left(\ell_i^{p+1}-\Tilde{\ell}_i^{p+1}\right)
       &= \sum_{i=2}^{n-2} 2^{3p-2} G^p X_i^{p+1} \left(\frac{1}{X^{p+1}} - \frac{1}{(n+1)^{p+1}}\right)
       \\
       &= 2^{3p-2} G^p \cdot \frac{\sum_{i=2}^{n-2}  X_i^{p+1}}{n-3} \cdot \frac{n-3}{(n+1)^{p+1}} \cdot \left(\frac{(n+1)^{p+1}}{X^{p+1}} - 1\right)~,
   \end{align*}
   and by the strong law of large numbers we have $\frac{ \sum_{i=2}^{n-2} X_i^{p+1}}{n-3} \xrightarrow{\text{a.s.}} \Exp X_i^{p+1} < \infty$ and $\left(\frac{n+1}{X}\right)^{p+1} \xrightarrow{\text{a.s.}} 1$, and thus
   \begin{equation} \label{eq:a.s._lemma_app_c_2}
        \sum_{i=2}^{n-2} 2^{3p-2} G^p \left(\ell_i^{p+1}-\Tilde{\ell}_i^{p+1}\right) 
        \xrightarrow{\text{a.s.}} 0~.   
   \end{equation}

   Moreover,
   \begin{align*}
       \sum_{i=2}^{n-2} 4^{p-1}|\eps_i|^p \left(\ell_i - \Tilde{\ell}_i \right)
       &= 4^{p-1} \sum_{i=2}^{n-2} \left(\frac{|\eps_i|^p X_i}{X} - \frac{|\eps_i|^p X_i}{n+1}\right) 
       \\
       &= 4^{p-1} \cdot \frac{\sum_{i=2}^{n-2} |\eps_i|^p X_i}{n-3} \cdot \frac{n-3}{n+1} \left(\frac{n+1}{X} - 1\right)~,
   \end{align*}
   and by the strong law of large numbers we have $\frac{\sum_{i=2}^{n-2} |\eps_i|^p X_i}{n-3} \xrightarrow{\text{a.s.}} \Exp |\eps_i|^p X_i < \infty$ and $\left(\frac{n+1}{X}\right) \xrightarrow{\text{a.s.}} 1$, and thus
   \begin{equation} \label{eq:a.s._lemma_app_c_3}
        \sum_{i=2}^{n-2} 4^{p-1}|\eps_i|^p \left(\ell_i - \Tilde{\ell}_i \right)
        \xrightarrow{\text{a.s.}} 0~.
   \end{equation}
   By a similar argument, we also have
   \begin{equation} \label{eq:a.s._lemma_app_c_4}
         \sum_{i=2}^{n-2} 4^{p-1}|\eps_{i+1}|^p \left(\ell_i - \Tilde{\ell}_i \right)
        \xrightarrow{\text{a.s.}} 0~.
   \end{equation}

   Combining \eqref{eq:a.s._lemma_app_c_1}, (\ref{eq:a.s._lemma_app_c_2}),~(\ref{eq:a.s._lemma_app_c_3}) and~(\ref{eq:a.s._lemma_app_c_4}), we get 
   \begin{align*}
       &\hat{R} - \Tilde{R}
       = \sum_{i=2}^{n-2} 2^{3p-2} G^p \left(\ell_i^{p+1} - \Tilde{\ell}_i^{p+1}\right) + \sum_{i=2}^{n-2} 4^{p-1} |\eps_i|^p  \left(\ell_i - \Tilde{\ell}_i\right) + \sum_{i=2}^{n-2} 4^{p-1} |\eps_{i+1}|^p \left( \ell_i - \Tilde{\ell}_i\right) 
       \\
       &+ \sum_{i=2}^{n-2} 4^{p-1} (|\eps_i|+|\eps_{i+1}|+|\eps_{i-1}|+|\eps_{i+2}|)^p \left( \frac{\ell_i^{p+1}}{\max\{\ell_{i-1},\ell_{i+1}\}^p} - \frac{\Tilde{\ell}_i^{p+1}}{\max\{\Tilde{\ell}_{i-1},\Tilde{\ell}_{i+1}\}^p} \right) 
       \xrightarrow{\text{a.s.}} 0~ \qed{ \text{ Lemma \ref{lem:Rhat-Rtilde-as-ell1}}}
   \end{align*}
   
\end{proof}

\begin{proof}[Proof of Claim \ref{clm:maxA,B}]
We first show that $\max\{A,B\}\overset{d}{=} A+\frac{B}{2}.$ Both $A,B \iid \text{Exp}(1)$. We will show that the CDF of $\max\{A, B\}$ is the same as the CDF of $A+\frac{B}{2}$. Let $F_{\max\{A,B\}}(.)$ and $F_{A+\frac{B}{2}}(.)$ be their CDFs respectively. Then for any $z \geq 0$.
    \begin{align*}
        F_{A+\frac{B}{2}}(z) =& \Pr\left(A + \frac{B}{2} \leq z\right) = \int_0^z e^{-x} \Pr \left(B \leq 2(z-x)\right)\ dx = \int_0^z e^{-x} \left(1-e^{-2(z-x)}\right)\ dx\\
        =& \int_0^z e^{-x}\ dx - e^{-2z}\int_0^z e^x\ dx= 1 - e^{-z} - e^{-2z}(e^z-1)= 1 - 2e^{-z} + e^{-2z}\\
        =& \left(1-e^{-z}\right)^2=  \Pr (A \leq z, B \leq z)=\Pr(\max\{A , B\} \leq z)\\
        =& F_{\max\{A,B\}}(z). 
    \end{align*}
   Using this, we can further simplify the expectation as follows. 
\begin{align*}
    \Exp \left[ \frac{1}{\max\{A,B\}^p}\right] &=   \Exp \left[ \frac{1}{(A+\frac{B}{2})^p}\right]  \leq \Exp \left[ \frac{2^p}{(A+B)^p}\right] = 2^p \Exp \left[\frac{1}{\Gamma(2,1)^p} \right] \\
    &= 2^p \int_{0}^{\infty} \frac{1}{z^p} \cdot z e^{-z} \, dz \\    
    &= 2^p \int_{0}^{\infty} z^{1-p} e^{-z} \, dz \\
    &=2^p \Gamma(2-p)    \\
    & \leq \frac{2^p}{2-p},
\end{align*} 
where the last inequality follows from Claim \ref{clm:Gamma(z)} and the fact that $1\leq p < 2$. \qed{\text{ Claim \ref{clm:maxA,B}}}

\end{proof}
\begin{claim}\label{clm:Gamma(z)}For $z\in (0,1]$, we have $ \frac{1}{2z}\leq \Gamma(z) \leq \frac{1}{z}$.
\end{claim}
\begin{proof}
    For $z>0$,
    \begin{align}
      \Gamma(z)&=\int_{0}^{\infty} w^{z-1} e^{-w} \, dw = \frac{w^z}{z} \cdot e^{-w} \Big\vert_0^{\infty} - \int_{0}^{\infty} -e^{-w} \frac{w^z}{z} \, dw =0 + \frac{1}{z} \cdot \int_{0}^{\infty} e^{-w} w^z \, dw \nonumber\\
      &= \frac{\Gamma(1+z)}{z} \label{eq:Gammaz-Gammaz+1}
    \end{align}
Now for $0 <  z \leq 1$, we have $1 < 1+z \leq 2$. Therefore, 
$$\frac{1}{2} \leq \Gamma(1+z) \leq 1 .$$ 
The upper bound follows from the fact that $\Gamma(.)$ is unimodal (with first decreasing and then increasing). Therefore, the maximum value of $\Gamma(.)$ in $[1,2]$ is $\max\{\Gamma(1),\Gamma(2)\}=1$. The lower bound follows from \citet{deming1935minimum}; the minimum value is approximately 0.8856032, which is at least $ 1/2$. Putting this back in \eqref{eq:Gammaz-Gammaz+1}, for $0 < z \leq  1$
$$ \frac{1}{2z} \leq \Gamma(z) \leq \frac{1}{z}. \qed{ \text{ Claim } \ref{clm:Gamma(z)}}.$$

\end{proof}
