\documentclass[lettersize,journal]{IEEEtran}
\usepackage{amsmath,amsfonts}
\usepackage{algorithmic}
\usepackage{algorithm}
\usepackage{array}
\usepackage[caption=false,font=normalsize,labelfont=sf,textfont=sf]{subfig}
\usepackage{textcomp}
\usepackage{stfloats}
\usepackage{url}
\usepackage{verbatim}
\usepackage{graphicx}
\usepackage{cite}
\hyphenation{op-tical net-works semi-conduc-tor IEEE-Xplore}

\usepackage{placeins}

% \usepackage{lineno}
\usepackage{amssymb}
\usepackage{amstext}
\usepackage{amsmath}
\usepackage{multicol, multirow}
\usepackage{dsfont}
\usepackage{pifont}
\usepackage[dvipsnames]{xcolor}
\usepackage{overpic}
\usepackage{tikz}
\usepackage{soul}
\usepackage{url}
\usepackage{listings}
\usepackage{lipsum}
\usepackage{placeins}
\usepackage{booktabs,hyphenat} % for professional tables
%\usepackage[ruled]{algorithm2e}
\usepackage{cite,hyperref}
% \usepackage{enumitem}
% \usepackage{enumitem}% http://ctan.org/pkg/enumitem

\usepackage{adjustbox}
\def\imagetop#1{\vtop{\null\hbox{#1}}}

\usepackage[resetlabels,labeled]{multibib}
\newcites{R}{References}
\usepackage[switch]{lineno}

% updated with editorial comments 8/9/2021
\newcommand{\Yiming}[1]{{\color{magenta} {[Yiming]  #1}}}
\newcommand{\Youjie}[1]{{\color{red} {[Youjie] #1}}}
\newcommand{\gmei}[1]{{\color{brown} {[GF] #1}}}

\newcommand{\mname}[0]{\mbox{FusionRAFT}}
% \usepackage{changes}
% fabio
\usepackage{soul}
\usepackage{todonotes}
\usepackage{bm}
\newcommand{\fabiocomment}[1]{\todo[color=purple!20, inline, author=Fabio]{#1}}
\newcommand{\fabio}[1]{\textbf{\textcolor{purple}{#1}}}
% \newcommand{\gmei}[1]{\textbf{\textcolor{red}{#1}}}

\begin{document}



%%%%%%%%%%%%%%%%%%%%%%%%%%%%%%%%%%%%%%%%%%%%%%%%%%%%%%%%%%%%%%%%%%%%
%%%%%%%%%%%%%%%%%%%%%%%%%%%%%%%%%%%%%%%%%%%%%%%%%%%%%%%%%%%%%%%%%%%%
% \documentclass[11pt,a4paper]{article}

\usepackage[colorlinks,pdfstartview=FitH,citecolor=red,linkcolor=red,urlcolor=red]{hyperref}

\usepackage{cite}
\usepackage{tikz}\usetikzlibrary{matrix,decorations.pathreplacing,positioning}
\usepackage{hyperref}
\usepackage{amsmath,amsthm,amssymb,bm,relsize}
\usepackage{calrsfs}
\usepackage{scalerel}

\usepackage{float}

\usepackage{pgfplots}
\pgfplotsset{compat=1.12}

\usepackage{setspace}
\setstretch{1.02}

\usepackage{multirow}
\usepackage{array}
\newcolumntype{x}[1]{>{\centering\arraybackslash\hspace{0pt}}p{#1}}

\usepackage{empheq}
\usepackage{wasysym}

\usepackage{caption}
\usepackage{subcaption}

\usetikzlibrary{arrows.meta}
\newcommand{\midarrow}{\tikz \draw[-triangle 90] (0,0) -- +(4,0);}
\usetikzlibrary{knots}
\newcommand{\wt}{\textnormal{wt}}
\usetikzlibrary{spath3}
\usetikzlibrary{hobby}



\usepackage{enumitem}
\setitemize{itemsep=-1pt}
\setenumerate{itemsep=-1pt}


\usepackage{titling}
\setlength{\droptitle}{-0.5cm}

\usepackage[margin=2.8cm]{geometry}
\usepackage{pgfplots}
\pgfplotsset{width=10cm,compat=1.9}


\definecolor{myg}{RGB}{220,220,220}

\newcommand{\bigzero}{\mbox{\normalfont\Large\bfseries 0}}
\newcommand{\rvline}{\hspace*{-\arraycolsep}\vline\hspace*{-\arraycolsep}}

\usepackage[capitalise]{cleveref}

\theoremstyle{definition}
\newtheorem{theorem}{Theorem}[section]
\newtheorem{corollary}[theorem]{Corollary}
\newtheorem{proposition}[theorem]{Proposition}
\newtheorem{lemma}[theorem]{Lemma}
\newtheorem{conjecture}[theorem]{Conjecture}
\newtheorem{definition}[theorem]{Definition}
\newtheorem{example}[theorem]{Example}
\newtheorem{notation}[theorem]{Notation}
\newtheorem{remark}[theorem]{Remark}
\newtheorem{terminology}[theorem]{Terminology}
\newtheorem{oproblem}[theorem]{Open Problem}

\newlength{\mynodespace}
\setlength{\mynodespace}{6.5em}



\newcommand{\ABK}[1]{\textcolor{orange}{#1}}
\newcommand{\todo}[1]{\textcolor{red}{To be added: #1}}
\newcommand{\AN}[1]{\textcolor{blue}{#1}}
\newcommand{\RP}[1]{\textcolor{green}{#1}}


\newcommand{\red}[1]{\textcolor{red}{#1}}
\newcommand{\numberset}{\mathbb}
\newcommand{\C}{\numberset{C}}
\newcommand{\F}{\numberset{F}}
\newcommand{\R}{\numberset{R}}
\newcommand{\Z}{\numberset{Z}}
\newcommand{\mA}{\mathcal{A}}
\newcommand{\mC}{\mathcal{C}}
\newcommand{\mD}{\mathcal{D}}
\newcommand{\mF}{\mathcal{F}}


\newcommand{\tub}{\textnormal{tub}}

\newcommand{\mat}{\F_q^{n \times m}}

\newcommand{\Ker}{\textnormal{Ker}}

\newcommand{\rk}{\textnormal{rk}}
\newcommand{\wts}{\textnormal{wts}}

\newcommand{\dH}{d^\textnormal{H}}

\newcommand{\gaussbinom}{\genfrac{[}{]}{0pt}{}}


\newcommand{\Mod}[1]{\ (\mathrm{mod}\ #1)}


\title{\textbf{Knot Theory and Error-Correcting Codes}}

\usepackage{authblk}


\author{Altan B. K\i l\i\c{c}, Anne Nijsten, Ruud Pellikaan, Alberto Ravagnani \\ 
Department of Mathematics and Computer Science, Eindhoven University of Technology, the Netherlands
\thanks{A. B. K\i l\i\c{c} is supported by the Dutch Research Council through grant VI.Vidi.203.045. and A. Ravagnani is supported by the Dutch Research Council through grants VI.Vidi.203.045, 
OCENW.KLEIN.539, 
and by the Royal Academy of Arts and Sciences of the Netherlands.}
\thanks{Emails: a.b.kilic@tue.nl, a.nijsten@live.nl, g.r.pellikaan@outlook.com, a.ravagnani@tue.nl.}}



\date{}

\begin{document}

\maketitle


\vspace{ 1 cm}
\begin{abstract}

This paper builds a novel bridge
between algebraic coding theory and mathematical knot theory, with applications in both directions. We give methods to construct error-correcting codes starting from the colorings of a knot, describing through a series of results how the properties of the knot translate into code parameters. We show that 
knots can be used 
to obtain error-correcting codes
with prescribed parameters 
and an efficient decoding algorithm. 
\end{abstract}



\medskip

\section*{Introduction}

The theory of error-correcting codes and their properties has been classically investigated in connection with several other areas of discrete mathematics, including finite geometry, enumerative combinatorics, algebraic combinatorics, algebraic and arithmetic geometry, matroid theory, ring theory, symbolic dynamics, and lattice theory to mention a few \cite{lind2021introduction, blake2014introduction,oxley2006matroid, stichtenoth2009algebraic, ball2015finite,ebeling2013lattices}. 

Studying codes in relation to other mathematical objects is 
an interesting and well-established 
research direction, which over the decades offered a new perspective on various classical problems. For example, deciding over which fields MDS codes exist is equivalent to deciding over which fields the uniform matroid is representable and is linked to the famous \textit{MDS Conjecture}~\cite{segre1955curve}. 

In this paper, we initiate the study of error-correcting codes in connection with mathematical knot theory, establishing a link between these two research domains. To our best knowledge, our paper is the first attempt to systematically and rigorously bridge coding theory with knot theory, except for the BSc and MSc theses of the second author of this paper~\cite{nijsten2019knots,nijsten2022knots}.

The way we associate codes to knots is via (Fox, Dehn or Alexander-Briggs) colorings of the \textit{knot diagram}. A knot diagram is a planar representation of a knot that can be divided into \textit{strands}, \textit{regions} and \textit{crossings}. These can be assigned \textit{colors}, which are elements of a commutative ring $R$ and where the coloring rules depend on some invertible element $t \in R$. The code is then constructed by using the \textit{coloring matrix} as a \textit{parity check matrix}; see Sections~\ref{sec:1} and~\ref{sec:2} for the definitions.

The paper then investigates how properties of knots translate into properties of the associated error-correcting code. To do so, we also establish some new properties of knot colorings. Most of our results focus on the length and the dimension of the associated code, but we are also able to 
prove some properties of the minimum distance (whose study appears to be a challenging task).

In our paper, we pay particular attention to two families of knots and their error-correcting codes.
These are \textit{torus knots} and their iterations, and \textit{pretzel knots}. We compute the parameters of the corresponding codes
in several instances. We also study the connected sum of knots and how the corresponding codes behave. We investigate the natural question of when the dual of a Fox knot code is a Fox knot code, and provide partial answers. 

\paragraph{Outline.} The remainder of this paper is organized as follows. In Section~\ref{sec:1}
we briefly review the preliminaries of knot and coding theory that are needed for this paper.
Section~\ref{sec:2} is about knot colorings and their algebra.
In Section~\ref{sec:3} we show how one can associate a code to a knot and investigate how the knot properties translate into code parameters. Section~\ref{sec:4} is devoted to torus knots, pretzel knots, and their associated codes. In Section \ref{sec:graph}, we study codes from graphs of Tait diagram of knots. Sections~\ref{sec:5} and~\ref{sec:6} conclude the paper and are about 
the connected sum of knots and the dual of Fox knot codes, respectively.
The paper also contains an appendix for the needed commutative algebra background.







\section{Knots and Codes}
\label{sec:preliminaries}
\label{sec:1}
In this section we give preliminary definitions and results on knot and coding theory that will be used throughout the paper. 
Since these two research areas are almost disjoint, we review the very basic concepts and include a selection of standard references. 
We assume that the reader is familiar with  
elementary concepts from algebra and topology; see~\cite{lang2012algebra} and~\cite{munkres} as standard references, among many others. 




\subsection{Knot Theory}
\label{subsec:knot}
We start with the definition of a mathematical knot, following to various degrees~\cite{crowell2012introduction,livingston1993knot,kosniowski1980first,murasugi1996knot}. 

\begin{definition}
\label{def:knot}
A (\textbf{mathematical}) \textbf{knot} $K$ is a topological subspace of the Euclidean space~$\R^3$ that is homeomorphic to the unit circle $S^1 \subseteq \R^2$, endowed with the induced Euclidean topology. An \textbf{oriented knot} is the image of the unit circle under this map whose orientation is induced by the orientation of $S^1$ (clockwise or counterclockwise).
Knots $K_1, K_2 \subseteq \R^3$ are 
\textbf{equivalent} if there exists an orientation-preserving homeomorphism
$f: \R^3 \to \R^3$ such that
$f(K_1)=K_2$. A knot $K$ is called \textbf{trivial}~(or \textbf{unknotted}) if it is equivalent to the knot $$\{(x_1,x_2,0) \mid x_1, x_2 \in \R, \, x_1^2+x_2^2=1\} \subseteq \R^3.$$
\end{definition}

Making the notions of orientation
and orientation-preserving map
rigorous is 
a non-trivial task that is best accomplished by homology theory in algebraic topology; see e.g.~\cite[Chapter~22]{greenberg2018algebraic}.
Intuitively (and not rigorously), a homeomorphism $\R^3 \to \R^3$ is orientation-preserving if it sends a right-hand frame into a right-hand frame.
It can be shown (see~\cite[page~212]{kosniowski1980first}) that knots $K_1, K_2 \subseteq \R^3$ are equivalent if and only if there exists a homeomorphism~$f: \R^3 \to \R^3$ and a real number $\xi >0$ such that 
$f(K_1)=K_2$ and $f(x)=x$ for all~$x \in \R^3$ with $\left\| x\right\| \ge \xi$. The latter can be taken as an elementary, but fully rigorous, definition of a knot equivalence.



A trivial knot is also called an \textbf{unknot}. An unknot is depicted in Figure \ref{fig:unknot} and a \textbf{figure-eight} knot
is depicted in Figure~\ref{fig:fig8}. The latter is a non-trivial knot as we will explain later via colorings, see Figure \ref{fig:foxdehn}.


% Figure environment removed

A knot can sometimes be seen as an entangled polygon in a three-dimensional space. To make this formal, we give the following definition.

\begin{definition}
\label{def:polygonal}
A knot is called \textbf{polygonal} if it is a union of finite number of line segments. These line segments are the \textbf{edges} and their endpoints are the \textbf{vertices} of the knot.
\end{definition}

Note that the drawings of Figure \ref{fig:exampleknots} are smooth, but can be seen as polygonal knots with smoothened vertices. 
A knot that is equivalent to a polygonal knot is called \textbf{tame}. A knot that is not tame is called \textbf{wild}; see~\cite[Chapter~I]{crowell2012introduction}.



\begin{terminology} \label{term}
    In this paper, a knot will always mean an oriented, polygonal knot, unless otherwise stated. We will omit information about the orientation when it is not relevant.    
    See Remark~\ref{rem:explain} for the reason of restricting ourselves to this specific family of knots.
    Throughout this paper, $K$ always denotes a knot, unless otherwise stated.
\end{terminology} 

% Figure environment removed


The knots that lie on the surface of an unknotted torus are of particular interest and will be used later in Section \ref{subsec:torus}. 

\begin{definition} \label{def:torusknot}
Consider the embedding of the torus $S^1\times S^1$  in $\R^3$ given by the implicit equation $$\left(\sqrt{x^2+y^2}-2\right)^2+z^2=1.$$
Let $(a,b)$ be a pair of nonzero integers that are relatively prime.
The $(a,b)$\textbf{-torus knot} $T(a,b)$ 
is the image of the map $S^1 \rightarrow \R^3$, lying on the torus, given by
$$
(\cos(t),\sin(t)) \longmapsto (\cos(at)(2+\cos(bt)),\ \sin(at)(2+\cos(bt)),\ \sin(bt));
$$ 
see e.g.~\cite[Chapter~7]{murasugi1996knot}.
The knot ``turns'' $a$ times meridionally and $b$ times longitudinally.
\end{definition}



\begin{example}
\label{ex:trefoil}
The {torus knot} $T(2,3)$ is more commonly known as the \textbf{trefoil knot}.  
It is depicted in Figure~\ref{fig:trefoil}. For any nonzero integer $a$, the torus knot
$T(a,\pm 1)$ is a trivial knot. The torus knots are completely classified; see \cite[Theorem 7.4.3]{murasugi1996knot}.


Figure \ref{fig:torus-poly}  depicts
the trefoil knot of Figure \ref{fig:trefoil} as an entangled polygon in a three-dimensional space, and as a knot that lie on the surface of a torus.

% Figure environment removed
\end{example}



As in Figures \ref{fig:exampleknots} and \ref{fig:trefoil}, to visualize knots their two-dimensional projections are used. These are called knot \textit{diagrams} and  are defined as follows. We refer to~\cite{murasugi1996knot} for a complete treatment.


\begin{definition}
\label{def:knotdiag}
Let $p:\R^3 \rightarrow \R^3$ be defined by $p(x,y,z) = (x,y,0)$. 
The 
\textbf{projection}
of a polygonal knot $K$ is 
$p(K)$, together with the orientation inherited by
$K$, if $K$ was oriented.
The projection is called \textbf{regular} if it satisfies the following three conditions:
\begin{enumerate}
    \item $p(K)$ has at most a finite number of points of intersection, where $Q$ is a point of intersection of $p(K)$ if $|p^{-1}(Q)| > 1.$ 
    \item If $Q$ is point of intersection of $p(K)$, then $K \cap p^{-1}(Q)$ has exactly two points. Such a point is called a \textbf{double point} of $p(K)$.
    \item A vertex of $K$ is not mapped to a double point of $p(K)$.
\end{enumerate}

At a double point of a projection, to distinguish whether the
knot passes over or under itself, we draw the projection
so that it appears to have been cut; see for example Figure~\ref{fig:trefoil}. Such an altered projection is called a \textbf{diagram} of $K$. 
\end{definition}





From now on, we assume that the diagrams of knots we use in the paper are regular. This can be justified by the following theorem.


\begin{theorem} [see \cite{crowell2012introduction}]
\label{thm:reg}
Any polygonal knot $K$ is equivalent, under an arbitrarily small rotation of $\R^3$, to a polygonal knot $K'$ for which $p(K')$ is regular.
\end{theorem}

Thus, for a given polygonal knot there exists an equivalent knot with a regular projection. Combining with the definition of a tame knot, we have that every tame knot is equivalent to a polygonal knot with a regular diagram.

\begin{remark}
\label{rem:explain}
In knot theory, knots are studied up to equivalence. Most knot theory references focus solely on tame knots; see \cite{crowell2012introduction}. One of the reasons is that some very natural invariants are not necessarily defined for wild knots.
\end{remark}

Although we work with polygonal knots, their diagrams are depicted with smooth vertices, since one can think of a polygonal knot as a union of a large number of  edges. In the next definition, we introduce some terminology of knot diagrams.


\begin{definition} 
Each double point of a regular projection is the image of two different points of the knot, and a such a point is called a \textbf{crossing} of a diagram. To distinguish edges that cross each other in a diagram, the lower edge in the crossing is drawn with a break. The resulting separate edges are called \textbf{strands}. At each crossing, the strands that are separated by the break are called the \textbf{understrands} and the other strand is called the \textbf{overstrand}. The connected components of the complement of $p(K)$ in the $z=0$ plane are called the \textbf{regions}. 

\end{definition}

As an example, the diagram depicted in Figure \ref{fig:fig8} has 4 crossings and 4 strands, and the diagram of Figure \ref{fig:trefoil} has 3 crossings and 3 strands. It is not a coincidence that the number of crossings is equal to the number of strands. We now give a simple but fundamental lemma which will play an important role in the next section, where we explain knot colorings. The result can be found in~\cite{alexander1928topological}.

\begin{lemma}
\label{lem:knot_diag}
Let $D$ be knot diagram with $n$ crossings. Then it has $n$ strands and $n+2$ regions.
\end{lemma}

Elementary knot moves lead to changes in knot diagrams. However, it is possible to
restrict only to the following moves. 




\begin{definition}
\label{def:reidemeister}
Consider the following three \textbf{Reidemeister moves}:
\begin{enumerate}[label= \Roman*.]
\item The \textbf{twist} move: This move twists or untwists a part of a strand in either direction, and is called a move of type I.
\item  The \textbf{poke} move: This move takes a strand and moves it completely over another (thus adding 2 crossings) or vice versa (thus removing 2 crossings), and is called a move of type II. 
\item  The \textbf{slide} move: This move slides a strand from one side of a crossing to the other side of the same crossing, and is called a move of type III.
\end{enumerate}
\end{definition}

The Reidemeister moves are depicted in Figure~\ref{fig:reidemeistermoves} and they are used to define equivalence of diagrams.

\begin{definition}
Two diagrams $D$ and $D'$ are called \textbf{equivalent} if $D$ can be transformed into~$D'$ by using a finite sequence of Reidemeister moves. We denote this by~$D \approx D'$.
\end{definition}

Reidemeister proved the following crucial result in \cite{reidemeister1927}. In this paper we use the statement of~\cite[Theorem 4.1.1]{murasugi1996knot}.

\begin{theorem}\label{thm:reidemeistermoves}
Let $D$ and $D'$ be the diagrams of two knots $K$ and $K'$, respectively. Then~$K \approx K'$ if and only if $D \approx D'$.
\end{theorem}

% Figure environment removed


Using the Reidemeister moves, one can show the equivalence of knots by applying Theorem~\ref{thm:reidemeistermoves}. For example, in Figure \ref{fig: amphichiral} we show that the figure-eight knot of Figure \ref{fig:fig8} is equivalent to its mirrored image. In the last step, no Reidemeister moves are used, but the position of the strands are changed slightly. The colors indicate how the strands are moved in the last step.
% Figure environment removed



In this paper, we will also use the concepts of a reduced and alternating 
knot diagrams.
These are defined as follows.


\begin{definition}
\label{def:reduced_alt}
A knot diagram is called \textbf{alternating} if the understands and overstrands are alternating in a fixed orientation. A knot diagram is called \textbf{reduced} if there are no crossings that can be removed via the twist move.
\end{definition}


\subsection{Coding Theory}
\label{subsec:coding}

We now turn to the coding theory fundamentals, that are also 
needed to understand the rest of the paper. Let $n \in \Z_{\ge 1}$, $q$ be a prime power, and $\F_q$ be the finite field with $q$ elements. 
General coding theory references are \cite{macwilliams1977theory,huffman2010fundamentals,pellikaan2018}.


\begin{definition}
A (\textbf{linear}, \textbf{error-correcting}) \textbf{code} of
\textbf{length} $n$ is
an $\F_q$-linear subspace~$\mC \subseteq \F_q^n$.
The \textbf{dimension} of $\mC$, denoted by $\dim(\mC)$, is its 
dimension as a vector space over~$\F_q$.
The quantity $\dim(\mC)/n$ is the \textbf{rate} of $\mC$, and denoted by $R(\mC)$.
The \textbf{dual} of $\mC$ 
is the code~$\mC^\perp = \{x \in \F_q^n \mid xy^\top =0 \mbox{ for all } y \in \mC\}$. Note that $\dim(\mC^\perp) = n - \dim(\mC)$.
A code $\mD \subseteq \mC$ is called a \textbf{subcode} of $\mC$.
\end{definition}



A code is most often represented by matrices.




\begin{definition}
\label{def:parity}
We say that a matrix $G \in \F_q^{k \times n}$ is a \textbf{generator matrix} of a code $\mC$ (and that $\mC$ is \textbf{generated} by $G$) if $\mC$ is the row-space of $G$. 
A \textbf{parity check matrix} $H$ of~$\mC \subseteq \F_q^n$ is a matrix such that
$$\mC=\{c \in \F_q^n \mid cH^T=0\}.$$
\end{definition}

Note that, in contrast with some coding theory references, we do not require $G$ and $H$ to have full rank in this paper. 

The performance of an 
error-correcting code is measured by its rate and its \textit{minimum Hamming distance}, defined below. Ideally, these parameters should both be as large as possible.


\begin{definition}
\label{def:weight_distance}
The \textbf{support} of a vector $x \in \F_q^n$ is
$\sigma(x) = \{i \in \{1,\ldots,n\} \mid \ x_i \neq 0\}$. The \textbf{Hamming weight} of a vector $x \in \F_q^n$ is the number of its nonzero entries, i.e.,~$\wt(x)=|\sigma(x)|$. 
The \textbf{minimum} (\textbf{Hamming}) \textbf{distance} of a code $\mC $
is 
$$d(\mC)=\min\left\{\wt(x) \mid x \in \mC, \, x \neq 0\right\},$$
where the code $\{0\} \subseteq \F_q^n$ has minimum distance $\infty$ by definition. The quantity $d(\mC)/n$ is the \textbf{relative minimum distance} of $\mC$, and denoted by $\delta(\mC)$.


The \textbf{weight enumerator} of $\mC $ is given by  
$W_{\mC}(t)=\sum_{w=0}^n a_w(\mC)t^w$, where $a_w(\mC)$ is the number of codewords of $\mC$ of weight $w$. 
Lastly, we let $\wt(\mC) = \{\wt(c) \mid c \in \mC \}$. 

\end{definition}

\begin{example}
\label{ex:repetition}
The $q$-ary \textbf{$n$-repetition code} is $\{(a,\ldots,a) \in \F_q^n \mid a \in \F_q\}$. It has dimension 1, minimum distance $n$, and rate $1/n$.


\end{example}


We write that $\mC$ is an $[n,k,d]_q$ code
if $\mC \subseteq \F_q^n$ has dimension $k$ and minimum distance~$d$. 

One of the best known results in coding theory establishes a trade-off between the dimension and the minimum distance of a code of a given length. In particular, they cannot be both arbitrarily large.

\begin{theorem}[\textbf{Singleton Bound}; see~\cite{singleton1964maximum}] \label{thm:sbound}
Let $\mC \neq \{0\}$ be an $[n,k,d]_q$ code. We have~$k \le n-d+1$.
\end{theorem}
Another very famous bound is the Gilbert-Varshamov bound.
\begin{theorem}[see~\cite{gilbert1952comparison,varshamov1957estimate}] \label{thm:gvbound}
Let $\mC \neq \{0\}$ be an $[n,k,d]_q$ code. We have 
$$q^{n-k} \le \sum_{i=0}^{d-1} \binom{n}{i}(q-1)^i.$$
\end{theorem}


Next, we give three definitions of code equivalence.
\begin{definition}
\label{d-mono-perm-pm-equiv}
Two $\F_q$-linear codes are called \textbf{permutation equivalent} if one is obtained from the other by permuting the coordinates.
They are called \textbf{monomial equivalent} if one is obtained from the other by permuting the coordinates and by multiplying the coordinates 
with nonzero elements of the field $\F_q$, see \cite{huffman2010fundamentals} and \cite[Definition 1.1.15]{pellikaan2018}.
They are called \textbf{($\pm1$)-permutation equivalent} if one is obtained from the other by permuting the coordinates and by multiplying the coordinates with $\pm 1$.
\end{definition}

Over $\F_2$, the three equivalences defined in \ref{d-mono-perm-pm-equiv} are the same.
Next, we define two classes of codes that will arise later in our paper. These two classes of codes are examples of well-known families from classical coding theory that can be obtained as knot codes, and thus have particular interest for us.

\begin{definition}
    \label{def:ldpc}
    A code that has a parity check matrix in which every row has Hamming weight $r$ and every column has Hamming weight $c$, is called a \textbf{$(r,c)$-doubly-regular low-density parity check (LDPC) code}. If the rows or  the columns of the matrix have a fixed Hamming weight $w$, then the LDPC code is called \textbf{right} or \textbf{left $w$-regular}, respectively.
\end{definition}

LDPC codes, first introduced in \cite{gallager1962low}, have efficient decoding algorithms, see for example~\cite{mackay1997near,mackay1999good, luby2001improved} among many others.

\begin{definition}
\label{def:lcd}
The \textbf{hull} of a code $\mC$ denoted by $H(\mC)$ is the intersection of the code with its dual: 
 $H(\mC)=\mC \cap \mC^\perp$. A code $\mC$ is called \textbf{linear complementary dual (LCD)} if $H(\mC)=\{ 0 \}$. See \cite{massey1992}.
\end{definition}

 LCD codes have been widely applied in data storage, communications systems, consumer electronics, and cryptography \cite{carlet2016}. 

\begin{definition}
\label{def:asympgood}
A sequence of linear codes $(\mC_j)^{\infty}_{j=1}$ where each $\mC_j$ has parameters $[n_j,k_j,d_j]$ is called \textbf{asymptotically good} if the following hold:
\begin{enumerate}
    \item $\lim_{j \to \infty}n_j = \infty$,
    \item $\liminf_{j\to\infty} R(\mC_j) > 0$,
    \item $\liminf_{j\to\infty} \delta(\mC_j) > 0$.
\end{enumerate}

\end{definition}

\section{Knot Colorings}
\label{subsec:color}
\label{sec:2}

In this section we explain three types of knot colorings. 
Fox coloring and Dehn coloring are colorings of the strands and crossings, respectively, see Figure \ref{fig:Fox-Dehn}. For the third, the Alexander-Briggs coloring, we first define the \textit{Tait diagram} of an oriented knot, see Figure \ref{fig:Tait}. 
We refer to Appendix~\ref{commut-alg} for the necessary background in commutative algebra needed for this section.

% Figure environment removed


\subsection{Fox Coloring}

The Fox coloring is the coloring of the strands of the knot diagrams. In this section we introduce the concept of \textit{Fox $(R,t)$-coloring}, where $R$ is a Noetherian commutative ring with an identity and $t$ is an invertible element of this ring. 
We start with the definition of \textit{Fox tricolorability}, that is where $R=\Z/(3)$ and $t=-1$.

\begin{definition}
\label{def:3color}
A \textbf{Fox tricoloring} of a  knot diagram is a coloring of the strands with three colors such that at each crossing, the colors of the strands that meet at that crossing are either all the same or all different. If we take as colors $0$, $1$ and $2$, then this rules amounts to the linear equation $a+b+c \equiv 0 \Mod{3}$, where $a,b$ and $c$ are the colors of the three strands that come together at a crossing. Moreover, a Fox tricoloring is called \textbf{trivial} if all strands have the same color. A knot diagram is called \textbf{Fox tricolorable} if it has a non-trivial tricoloring.

\end{definition}
Tricolorability is another invariant of a knot\cite{przytycki}. This already allows us to distinguish the unknot and trefoil knot, as the latter is tricolorable and the former is not, see Figure \ref{fig:3color_trefoil}.

% Figure environment removed

Generalizing Definition \ref{def:3color} to colors $0,1, \ldots, n-1$ gives the equation $a+c \equiv 2b \Mod{n}$, where $a,b$ and $c$ are again the colors of the three strands that come together at a crossing with $b$ being the overstrand, and $n \in \Z_{>0}$. This can also be generalized further.

\begin{definition}
\label{def:FoxRt}
A \textbf{Fox $(R,t)$-coloring} of a knot diagram is a coloring of its strands with colors that are elements of $R$ and for each crossing it holds that
\begin{equation}
\label{eq:Fox}
c=ta +(1-t)b,
\end{equation}
where $t$ is a fixed invertible element in the ring $R$, the strand with color $b$ is the overstrand and the strands colored with $a$ and $c$ are understands such that the rotation from $b$ to $c$ around the crossing is counter clockwise; see Figure \ref{fig:Fox_coloring}.
A coloring is called trivial if all the colors are the same. The knot diagram is called \textbf{Fox $(R,t)$-colorable} if there is a non-trivial Fox $(R,t)$-coloring; see \cite{fox1970}.
\end{definition}

In particular, every Fox $(R,t)$-coloring with $R=\Z/(2)$ is trivial, since $t=1$ is the only invertible element of $R=\Z/(2)$. We also have the following result.


\begin{proposition}\label{p-(R,1)-col}
All Fox $(R,1)$-colorings of a knot diagram are trivial.
\end{proposition}

\begin{proof}
Up to a permutation, we may assume that the strands~$\{x_1,\ldots,x_n\}$ are numbered in such a way that $x_{j+1}$ comes after $x_j$ for a given choice of the orientation of the knot diagram.
Then the equations \eqref{eq:Fox} with $t=1$ become $x_{j+1}=x_j$ for all $j=1, \ldots n-1$. Hence every Fox $(R,1)$-coloring is trivial.
\end{proof}

Lemma \ref{lem:knot_diag} allows us to place the coefficients in the system of equations that has to hold for a Fox coloring of a diagram with $n$ strands into an $n \times n$ matrix. Before defining this matrix, we remark the following.

\begin{remark}\label{r-alex}
The definition of the \textit{Alexander matrix} of a knot diagram is usually given by means of the free calculus of a presentation of the fundamental group of the complement of the knot; see \cite[Chapter III]{crowell2012introduction}. From this approach one readily gets that the distinct presentations of the fundamental group of the knot give equivalent (see Definition \ref{d-equiv-matrix}) Alexander matrices. Hence the equivalence class of the Alexander matrix of a knot diagram is in fact an invariant of the knot.
\end{remark}

We give the following \textit{ad hoc} definition of the Alexander matrix of a knot diagram and show that it is an invariant under Reidemeister moves.

\begin{definition}
\label{def:coloringMat}
The \textbf{Alexander matrix} (or the \textbf{Fox coloring matrix}) of an oriented knot diagram with $n$ crossings $\{c_1,\ldots,c_n\}$ and strands~$\{x_1,\ldots,x_n\}$ is defined as the 
matrix~$M(t)$ with entries in $\Z[t,t^{-1}]$ with
$$
M_{ij}(t) = 
\begin{cases}
1 - t & \text{if} \ \ x_j \textnormal{ is an overstrand at } c_i,\\
-1 & \text{if} \ \ x_j \textnormal{ is an understrand at } c_i \textnormal{ at the left side of the overstrand},\\
t & \text{if} \ \ x_j \textnormal{ is an understrand at } c_i \textnormal{ at the right side of the overstrand},\\
0 & \text{otherwise}
\end{cases}
$$ for $1 \le i,j \le n$ with $i,j \in \Z_{>0}.$
\end{definition}

The matrix of Definition \ref{def:coloringMat} is called the Alexander matrix by Crowell and Fox \cite{crowell2012introduction} and it is different than the matrix Alexander defined in his paper~\cite{alexander1928topological}. 
Note that we write $M_{ij}(t)$ instead of 
$M(t)_{ij}$ and, for ease 
of notation, we omit the knot diagram in the symbol $M_{ij}(t)$.

\begin{definition}
\label{def:Foxmodule}
The \textbf{module of Fox $(R,t)$-colorings} of a knot diagram is the kernel of the matrix $M(t)$, that is, the $R$-module of 
$x \in R^n$ such that $M(t) x^T=0$.
\end{definition}


The sum of the entries in any row of $M(t)$ is zero. So, the columns of $M(t)$ are dependent, and thus the determinant of $M(t)$ is zero. 
Hence $E_0(M(t))=0$ (see Definition~\ref{d-elem-id}) and there is a non-trivial solution of the system of equations \eqref{eq:Fox}. Trivial colorings correspond to the solutions of this system of equations where all the (nonzero) elements are the same. 
By taking any $(n-1)$-minor of the Alexander matrix, we get another knot invariant; see~\cite{alexander1928topological}. 


\begin{proposition}\label{p-pincipalM}
Let $M_{ij}^*(t)$ be the matrix obtained by deleting the $i$-th row and the $j$-th column of $M(t)$. 
Then the determinants $\det (M_{ij}^*(t))$ are equal to each other up to a factor~$\pm t^s$, where $s$ is an integer. 
In particular, $E_1(M(t))$ (see Definition \ref{d-elem-id}) is a principal ideal generated by $\det (M_{ij}^*(t))$ for any choice of the indices $1 \le i, j \le n$.
\end{proposition}

\begin{proof}
See \cite[Chapter VI (1.3)]{crowell2012introduction} and \cite[Chapter VIII (3.7)]{crowell2012introduction}.
\end{proof}


\begin{proposition}\label{p-det=pm1}
Let $M(t)$ be the Alexander matrix of a knot diagram with $n$ crossings. We have $\det (M_{ij}^*(1))=\pm1$ for all $1 \le i \le j \le n.$
\end{proposition} 
\begin{proof} 
The equations \eqref{eq:Fox} with $t=1$ become $x_{j+1}=x_j$ for all $j=1, \ldots,n-1$ as we have seen in the proof of Proposition \ref{p-(R,1)-col}. The matrix that is obtained by deleting the last column and last row is upper triangular with ones on the diagonal, so it has determinant one. The result follows from Proposition \ref{p-pincipalM}.
\end{proof}

Another important result is the following.

 
\begin{proposition}\label{p-ReidemesterM}
If $D_1 \approx D_2$, 
then the corresponding Alexander matrices~$M_1(t)$ and~$M_2(t)$ are equivalent; see Definition \ref{d-equiv-matrix}. 
\end{proposition}

\begin{proof}
See \cite[Chapter 2, Theorem 3]{livingston1993knot} in the case $t=-1$ and $R=\F_p$ for a prime $p$. The general case is proved similarly.
\end{proof}


$E_1(M(t))$ is a principal ideal in $\Z[t,t^{-1}]$ by Proposition \ref{p-pincipalM}, that is generated by a nonzero element by Proposition \ref{p-det=pm1}.
Hence there exists an integer $s$ such that multiplication of a generator of $E_1(M(t))$ by the invertible element $\pm t^s$ gives a polynomial with a constant term that is positive. 

\begin{definition}
\label{def:alex}
Let $K$ be a knot. The \textbf{Alexander polynomial} of $K$, denoted by $\Delta_K(t)$, 
is the generator of $E_1(M(t))$ which is the polynomial with a constant term that is positive. Moreover, the value $|\Delta_K(-1)|$ is called the \textbf{determinant} of $K$. 
\end{definition}

The Alexander polynomial is a knot invariant by Proposition \ref{p-ReidemesterM}. However it is important to note that although the elementary ideals $E_0(M(t))=0$ and $E_1(M(t))$ of a knot are principal ideals, the other elementary ideals $E_k(M(t))$ are not necessarily principal for~$k>1$, as the following example shows.

\begin{example}
\label{ex:not-principal-ideal}
Figures 50 and 51 of \cite{crowell2012introduction} have both $\Delta_K(t)=2t^2-5t+2$ as their Alexander polynomial, but they have distinct elementary ideals.
The Stevedore's knot depicted in Figure~50 has $E_k=(1)$ for all $k\geq 2$, but the knot of Figure~51 has $E_2=(2-t,1-2t)$, which is not principal.
\end{example}

We work out an example to show how the Alexander polynomial of a knot is computed. Note that it does not depend on the chosen submatrix or the chosen knot diagram.


\begin{example}
\label{ex:alexander}
The Alexander matrix of the diagram of the trefoil knot $K$ depicted in Figure \ref{fig:trefoil} is given by 
\begin{equation*}
M(t) = \begin{pmatrix}
1-t & t & -1 \\
-1 & 1-t & t \\
t & -1 & 1-t
\end{pmatrix}.
\end{equation*}
We have $\det(M^*_{11}(t)) = t^2 -t + 1$ and $\det(M^*_{12}(t)) = -t^2 +t - 1$. Following Definition~\ref{def:alex}, we observe that the polynomial $\det(M^*_{11}(t))$ has a positive constant term, and thus~$\Delta_K(t) = t^2-t+1$.
\end{example}


We now turn our attention to the invariant factors of  the Alexander matrix and the invariant factors of the module of Fox $(R,t)$-colorings, see Definition \ref{def:Foxmodule}.

\begin{proposition}\label{p-elem-ideals-Alex-pol}
Let $R$ be a principal ideal domain with invertible element $t$. 
Let~$(d_1)\subseteq (d_2)\subseteq \cdots \subseteq (d_l)$ be the  invariant factors of the matrix $M(t)$ and~$E_k(M(t))$ be generated by $\Delta_k$, see Corollary \ref{c-struct-pid}. Then $\Delta_0=0$, $d_1=0$, and $\Delta_K(t)=\Delta_1=\prod_{j=2}^ld_j$. 
\end{proposition}

\begin{proof}
The determinant of $M(t)$ is zero. So $E_0(M(t))=(0)$ and $\Delta_0=0$.
Now~$\Delta_K(t)=\Delta_1=\prod_{j=2}^ld_j$ by Corollary \ref{c-struct-pid} and $\Delta_K(t)\not=0$ by Proposition \ref{p-det=pm1}. 
So $\Delta_0=d_1\Delta_1$ by Corollary \ref{c-struct-pid}. This implies that $d_1=0$, since $R$ is an integral domain.
\end{proof}



Slightly abusing the notation, denote the localization of $\Z$ at a nonzero $t\in \Z$ by $Z_t$, and the localization of $\F_p[T]$ at a nonzero $t$ of $\F_p[T]$ by $\F_p[T]_t$. So $Z_t=\{ n/t^e \mid n,e \in\Z \}$ and~$\F_p[T]_t=\{ f/t^e \mid f\in \F_p[T],\ e \in\Z \}$. The next two propositions will be used later in Subsection \ref{subsec:dim} to bound the dimension of so-called \textit{Fox knot codes}. 


\begin{proposition}\label{p-number-color-Z}
Let $D$ be a  knot diagram of a knot $K$.
Let $d, t\in \Z$ such that $0<t<d$ and $ \gcd(d,t)=1$. Let $R=\Z_t$ and and $\overline{R}=R/(d)$.
Let  $(d_1)\subseteq (d_2)\subseteq \cdots \subseteq (d_l) $ be the invariant factors of the matrix $M(t)$ of Fox $(R,t)$-colorings.
Let $a_i=\gcd (d,d_i)$ and~$\overline{x}=x+(d) \in R/(d)$ for $x\in R$.
Then $\overline{t}$ is an invertible element in $\overline{R}$ and $d\prod_{i=2}^n a_i$ is the number of Fox $(\overline{R},\overline{t})$-colorings of $D$.
\end{proposition}
\begin{proof}


The element $\overline{t}$ is invertible in $R/(d)$, since $ \gcd(d,t)=1$. 
Hence~$\Z/(d)\cong \Z_t/(d)=\overline{R}$.
The module of $(\overline{R},\overline{t})$-colorings of $D$ is equal to the $Ker(M(\overline{t}) )$ which is isomorphic to~$ \overline{R}/(\overline{a_1}) \oplus \overline{R}/(\overline{a_2}) \oplus  \cdots \oplus  \overline{R}/(\overline{a_l})$ by Proposition~\ref{p-struct-ker-im}. We have $d_1=0$ by Proposition~\ref{p-elem-ideals-Alex-pol}, and $|\overline{R}|=d$. Furthermore $\overline{R}/(\overline{a_i})\cong \Z/(a_i)$, and thus $|\overline{R}/(\overline{a_i})|=a_i$ for all $2 \leq i\leq n$.
Hence~$d\prod_{i=2}^n a_i$ is the number of Fox $(\overline{R},\overline{t})$-colorings of $D$.
\end{proof}


\begin{proposition}\label{p-number-color-prime}
Let $D$ be a  knot diagram of a knot $K$.
Let $p\in \Z$  be a prime number. Let $d,t \in \F_p[T]$ such that $\gcd(d,t)=1$. Let $R=\F_p[T]_t$ and let $\overline{R}=R/(d)$.
Let~$(d_1)\subseteq (d_2)\subseteq \cdots \subseteq (d_l) $ be the invariant factors of the matrix $M(t)$ of Fox $(R,t)$-colorings.
Let~$a_i=\gcd (d,d_i)$. Let $\delta = \deg (d)$ and $\alpha_i = \deg (a_i)$.
Then $\overline{t}$ is an invertible element of $\overline{R}$ and~$p^{\delta+\sum_{i=2}^n \alpha_i}$ is the number of Fox $(\overline{R},\overline{t})$-colorings of $D$.
\end{proposition}
\begin{proof}
The proof is verbatim the same as for Proposition \ref{p-number-color-Z}, except for the final part. The element $\overline{t}$ is invertible in $R/(d)$, since $ \gcd(d,t)=1$. 
Hence $\overline{R}\cong \F_p[T]/(d)$ which has $p^{\deg(d)}$ elements, and 
$\overline{R}/(\overline{a_i})\cong \F_p[T]/(a_i)$ which has $p^{\deg(d_i)}$ elements.
\end{proof}


The Alexander polynomial~$\Delta_K(t)$ plays a direct role in determining whether a knot diagram is Fox $(R,t)$-colorable or not.

\begin{proposition}
\label{prop:determinant}
Let $R=\Z$ or $R=\F_q[T]$. 
Let $d,t \in R$ such that $d$ is not invertible in~$R$ and $\gcd(d,t)=1$.
Let $\overline{R}= R/(d)$ and $K$ be a knot.
Then the following statements are equivalent:\\
(1) A knot diagram of $K$ is Fox $(\overline{R},t)$-colorable;\\
(2) $\gcd(d,\Delta_K(t))\not=1$ in $R$;\\
(3) $\Delta_K(t)=0$ in $\overline{R}$ or  $\Delta_K(t)$ is a zero-divisor of $\overline{R}$.
\end{proposition}
\begin{proof}
The proof  Fox $(\F_p,-1)$-colorability for $p$ a prime is given in \cite[Chapter 3, Theorem 4]{livingston1993knot} and \cite[Proposition 2.1]{kauffman2018}. The knot $K$ is Fox $(\overline{R},t)$-colorable if and only if 
$\gcd (d,d_i)=\overline{d_i}\not=1$ for some $i$, $2 \leq i\leq n$, where
$(d_1)\subseteq (d_2)\subseteq \cdots \subseteq (d_l)$ are the invariant factors in $R$ 
by Propositions~\ref{p-number-color-Z} and~\ref{p-number-color-prime}.
But $\Delta_K(t)=\prod_{i=2}^l d_i$ by Proposition \ref{p-elem-ideals-Alex-pol}.
So $K$ is Fox $(\overline{R},t)$-colorable if and only if $\gcd(d,\Delta_K(t))\not=1$ in $R$ if and only if
$\Delta_K(t)=0$ in $\overline{R}$ or  $\Delta_K(t)$ is a zero-divisor of $\overline{R}$.
\end{proof}


Next, we will show that the trefoil knot is Fox $(R,t)$-colorable for several choices of the ring $R$ and the invertible element $t$.
\begin{example}
\label{ex:Z4F4}
We have seen that the trefoil knot is tricolorable, which is in agreement with Proposition \ref{prop:determinant}, since $\Delta_K(t) = t^2 - t + 1$ and $\Delta_K(-1) =3$. It is also $(\Z/(d),-1)$-colorable for all positive integers $d$ that are a multiple of $3$ with the colors $0, d/3, 2d/3$ assigned to the three strands.

Consider the Fox colorings for the pairs $(\Z/(4), -1)$, $(\mathbb{F}_4, \alpha)$ and $(\mathbb{F}_7, 3)$ of the trefoil knot~$K$, where $\mathbb{F}_4 = \{0, 1, \alpha, \alpha^2 \}$ and $\alpha$ is a root of the irreducible polynomial $x^2+x+1$ over~$\F_2[x]$. We find that $\Delta_K(-1) = 3 \neq 0$ over $\Z/(4)$ and $\Delta_K(\alpha) = \alpha^2 + 1 - \alpha$, which is 0 over~$\mathbb{F}_4$ and $\Delta_K(3) = 7 = 0$ over $\mathbb{F}_7$. Therefore, the trefoil knot has only trivial Fox colorings when~$(R,t) = (\Z/(4), -1)$, but it has a non-trivial Fox coloring when $(R,t) \in \{(\mathbb{F}_4, \alpha),(\mathbb{F}_7, 3)\}$.
\end{example}

\subsection{Dehn Coloring}

In this subsection, we study another way to color knot diagrams, called Dehn colorings. The \textit{Dehn coloring} is the coloring of the regions of a knot diagram. Similar to Definition~\ref{def:FoxRt}, we give the definition of a Dehn coloring as follows. 

\begin{definition}
\label{def:DehnRt}

A coloring of the regions of a knot diagram with $n$ crossings is called a \textbf{Dehn $(R,t)$-coloring} 
if the regions are colored via colors that are elements of $R$ 
and at each crossing~$c_m$ for $1 \le m \le n$ with an overstrand $x$, it holds that
\begin{equation*}
U_i - tU_j = U_k - tU_l,
\end{equation*}
where $t$ is a fixed invertible element in the ring $R$, the regions $U_i,U_j,U_k$ and $U_l$ are regions that have $c_m$ on their border in a way that $U_i$ and $U_k$ are on the left side of~$x$ and~$U_j$ and~$U_l$ are on the right side of $x$ with respect to the orientation of the diagram; see Figure~\ref{fig:Dehn_coloring}. Following the convention, the color $0$ is assigned to the unbounded outside region. 
\end{definition}


Analogous to Definition \ref{def:coloringMat}, we define the following matrix for Dehn colorings.

\begin{definition}
\label{def:DehnMat}
Let $D$ be a knot diagram with $n$ crossings. At each crossing $c_m$ for~$1 \le m \le n$ with an overstrand $x$ such that the regions $U_i,U_j,U_k$ and $U_l$ are regions that have $c_m$ on their border in a way that $U_i$ and $U_k$ are on the left side of $x$ and $U_j$ and $U_l$ are on the right side of $x$ with respect to the orientation of the diagram, the \textbf{Dehn coloring matrix}~$N(t)$ of $D$ is defined as
$$
N_{ms}(t) = 
\begin{cases}
1 & \text{if} \ \ s=i,\\
-t & \text{if} \ \ s=j,\\
-1 & \text{if} \ \ s=k,\\
t & \text{if} \ \  s=l,\\
0 & \text{otherwise},
\end{cases}
$$ for $1 \le m \le n$ and $1 \le s \le n+2$ with $m,s \in \Z_{>0}.$
\end{definition}

The matrix $N(t)$ is the one defined by Alexander \cite{alexander1928topological} as remarked after Definition \ref{def:coloringMat}.  

\begin{remark}\label{r-dehn}
Dehn \cite{dehn1910} gave a less known presentation of the fundamental group of the complement of a knot with generators $U_i$
and relations $U_1=1$ and $U_iU_j^{-1}=U_kU_l^{-1}$ for all crossings as in Figure \ref{fig:Dehn_coloring}.
The free calculus of this presentation gives the matrix $N(t)$ with the first column deleted, see \cite{kauffman1983}.
 
\end{remark}




Analogous to Proposition \ref{p-ReidemesterM}, we have the following result that is proven in \cite{alexander1928topological}.

\begin{proposition}\label{p-ReidemesterN}
If $D_1 \approx D_2$, 
then the corresponding Dehn coloring matrices~$N_1(t)$ and~$N_2(t)$ are equivalent.
\end{proposition}


Similar to the module of Fox $(R,t)$-colorings of Definition \ref{def:Foxmodule}, one can define the module of Dehn $(R,t)$-colorings.

\begin{definition}

The \textbf{module of Dehn $(R,t)$-colorings} of the knot diagram is given by the kernel of the matrix $N(t)$, 
that is the $R$-module of all $ x\in R^{n+2}$  such that $N(t)x^T=0.$
\end{definition}

Fox and Dehn colorings can be obtained from each other. 
The following proposition is a generalization of the relation between Fox and Dehn colorings. We slightly abuse notation: the color of a region $U$ will also be denoted by $U$. Similarly, the color of a strand $x$ is also denoted by $x$.

\begin{proposition}
\label{prop:fox2dehn}
Let $D$ be an oriented knot diagram with $m$ regions and $n$ strands. 
Consider the map $\varphi : R^m \rightarrow R^n$ such that  $\varphi (U)=x$ 
gives the colors of the strands $x$ for a given coloring $U$ of the regions
such that $x_r =U_i-tU_j$ is the color of the stand $x_r$ where~$U_i$ and $U_j$ are the colors of the regions next to the strand~$x_r$, with $U_i$ on the left side of $x_r$  and $U_j$ on the right side of $x_r$. 
Then this map is a well-defined morphism of $R$-modules when restricted to the module of Dehn $(R,t)$-colorings, and Dehn colorings are mapped to Fox  $(R,t)$-colorings. 
Furthermore $\varphi$ is surjective onto the module of Fox  $(R,t)$-colorings, its kernel is isomorphic to $R$, 
and it is an isomorphism when $\varphi$ is restricted to the submodule of Dehn $(R,t)$-colorings where a fixed region gets the value $0$.
\end{proposition}



\begin{proof}
It is a straightforward generalization of the proofs given in \cite{carter2014three,traldi2017} for $t=-1$.
\end{proof}


In Figure \ref{fig:foxdehn} an example of a Fox $(\F_5,-1)$-coloring and Dehn $(\F_5,-1)$-coloring that are constructed via these steps can be found with the value $0$ for the outside region. 

% Figure environment removed


\begin{remark}\label{r-index}Consider the diagram of an oriented knot. 
The \textbf{index} of a region is an integer and defined in \cite[pp. 277]{alexander1928topological} 
such that the index of a given region is chosen at random, and the indices of the remaining regions 
are uniquely defined by the property that if a region with index $e$ is on the left of a strand, 
then the region on the right of that strand has index $e-1$.
\end{remark}

\begin{definition}\label{def:checkerboard}
The \textbf{checkerboard coloring} of a knot diagram is a coloring of the regions with two colors (white and black), in such a way that 
the regions with even index are colored white and the regions with odd index are colored black.
\end{definition}

\begin{remark}\label{rm:checkerboard}
In a checkerboard coloring of a knot diagram the two regions adjacent to a strand have distinct colors. 
So at each crossing, two regions have the same color if and only if they are not adjacent.
Notice that the checkerboard coloring does not depend on the orientation of the knot, 
but it does depend on the random value of of the index  of the given region in Definition \ref{r-index} 
in such a way that the colors black and white are interchanged when the random value is changed from even to odd and vice versa.
So a knot diagram has two checkerboard colorings which can be obtained from each other by interchanging the colors white and black.
\end{remark}



\begin{remark}\label{r-trivial-Dehn-col}
Let $x$ be the trivial Fox $(R,t)$-coloring where all the strands have the same color. Then for a fixed region $U_1$ with a fixed color, there exists Dehn $(R,t)$-coloring $U$ such that $\varphi (U)=x$ by Proposition \ref{prop:fox2dehn}. In this way we get all the so called \textbf{trivial} Dehn $(R,t)$-colorings which constitute a free summand $R^2$ of the module of Dehn $(R,t)$-colorings.
In particular, if $t\not=1$, $U_1=1$ and $x=0$, then all the regions $U_i$ have color $t^{e_i}$ where $e_i$ is the index of region $U_i$.
If $t=-1$, then the \textbf{trivial} Dehn $(R,t)$-colorings are such that all white regions have the same color and all the black regions have the same color.
A knot diagram is called \textbf{Dehn $(R,t)$-colorable} if it has a non-trivial coloring. Note that with these steps, trivial Fox colorings will transform into trivial Dehn colorings and the other way around, as well. Hence a knot diagram is Dehn $(R,t)$-colorable if and only it is Fox $(R,t)$-colorable.
\end{remark}



The next result allows us to compare some properties of the Alexander matrix and the Dehn coloring matrix of the same knot diagram.


\begin{proposition}\label{p-elem-id-dehn}
Let $N(t)$ be the Dehn $(R,t)$-coloring matrix of a knot diagram of a knot $K$,  
then $E_1(N(t))=0$ and $E_2(N(t))$ is a principal ideal generated by $\Delta_K(t)$.
\end{proposition}

\begin{proof}Choose two columns that correspond to two regions that have index $e$ and $e+1$ for some $e$. See Remark \ref{r-trivial-Dehn-col}.
Let $N''(t)$ be the $(n+2) \times n$ matrix  that is obtained from $N(t)$ by replacing the two chosen columns by zero columns.
Let $N_0(t)$ be the $n \times n$ matrix  that is obtained from $N(t)$ by deleting the two chosen columns.
The matrix $N(t)$ is equivalent to the matrix $N''(t)$. See \cite[pp. 280]{alexander1928topological}. 
So $E_1(N(t))=0$ and~$E_2(N(t))=E_2(N''(t))=E_0(N_0(t))$ by Proposition \ref{p-free-matrix}, which is a principal ideal generated by $\det (N_0(t))$.


Let $N_1(t)$ be the $(n+1) \times n$ matrix  that is obtained from $N(t)$ by deleting the chosen column of index $e$. In order to show that $E_2(N(t))$ is generated by $\Delta_K(t)$ we need to refer 
to the fact that the matrix $N_1(t)$ is obtained by the free calculus of 
the Dehn representation of the fundamental group of the complement of the knot by Remark \ref{r-dehn}, 
and $M(t)$ is the Alexander matrix obtained by the free calculus of 
another representation of the same fundamental group by Remark \ref{r-alex}. 
Therefore, these matrices are equivalent and have the same elementary ideals. 
See \cite[Chapter VII (4.5)]{crowell2012introduction}.
\end{proof}

We conclude this subsection with an example verifying that $E_1(M(t))$ and $E_2(N(t))$ are both generated by $\Delta_K(t)$.

\begin{example}
Consider the diagram of the trefoil knot whose Fox coloring matrix is given in Example \ref{ex:alexander}, and its Alexander polynomial is computed as $\Delta_K(t) = t^2-t+1$. Its Dehn coloring matrix is 
\begin{equation*}
N(t) = \begin{pmatrix}
1& -t & -1 & t & 0 \\
1 & -1 & 0 & t & -t \\
1 & 0 & -t & t & -1
\end{pmatrix}.
\end{equation*}

Let $N_{ij}^*(t)$ be the matrix obtained by deleting the $i$-th and the $j$-th column of $N(t)$ for~$i \neq j$. We have $M_{ij}^*(t) \in \{\pm (t^2 -t +1)\}$ and~$N_{ij}^*(t) \in \{ 0, \pm (t^3-t^2+t), \pm (t^2-t+1), t^3+1\}$. One can check that they are both generated by $\Delta_K(t) = t^2-t+1$, since~$t^3+1 = (t^2-t+1)(t+1).$
\end{example}

\subsection{Alexander-Briggs Coloring}
In this subsection, we study a third way to color knot diagrams, called Alexander-Briggs (AB) colorings. The \textit{AB coloring} is the coloring of the vertices of the Tait diagrams.
% Figure environment removed

\begin{definition}
\label{def:Tait}
The \textbf{Tait diagram} of a knot is a diagram of that knot with a chosen orientation, with two additional dots at every crossing beside the left 
hand side of an overstrand such that one is placed just before and the other is placed just after the understrand, 
together with a chosen checkerboard coloring of the regions, see Figure \ref{fig:Tait}.
\end{definition}

The ``vertices" in knot diagrams are changed into ``crossings" in Tait diagrams which fits with the subsequent part of the paper where on the graph of a Tait diagram we have vertices and edges that are in fact the regions and the crossings, respectively, of the Tait diagram.


\begin{definition}
\label{def:AB-coloring}
Consider a Tait diagram of a knot. Define
$$
\wt (U) = \sum_{v \in \partial U} \wt (U,v) v
$$
where $U$ is a region of the diagram and $v$ is a vertex in the boundary $\partial U$ of $U$ and
$$
\wt (U,v) = 
\begin{cases}
t & \text{if there is a dot in } U \textnormal{ near } v,\\
1 & \text{otherwise}. 
\end{cases}
$$
is defined as the \textbf{weight} of $U$ at $v$. An \textbf{Alexander-Briggs (AB) $(R,t)$-coloring} is a coloring of the crossings with colors that are elements of $R$ 
in such a way that $\wt(U) = 0$ for all the regions $U$ of the the Tait diagram. A Tait diagram is called \textbf{Alexander-Briggs $(R,t)$-colorable} if it has a nonzero Alexander-Briggs $(R,t)$-coloring.  
\end{definition}

As an example, for an Alexander-Briggs $(R,t)$-coloring of the diagram in Figure \ref{fig:Tait} it is necessary that  $\wt(U_{11}) = 0$, that is,
$$
v_1 + v_2 + v_3 + tv_4 + tv_5 = 0,
$$
according to Definition \ref{def:AB-coloring}.

Consider a knot diagram with $n$ crossings. By Lemma \ref{lem:knot_diag}, we let $v_1,v_2, \ldots ,v_n$ be an enumeration of the crossings, and $x_1,x_2, \ldots ,x_n$ an enumeration of the strands and $U_1,U_2, \ldots , U_{n+2}$ an enumeration of the regions in the sequel. With a slight abuse of notation, we denote the colors assigned to these crossings, strands or regions with the same notation as their enumeration. Analogous to Definition \ref{def:DehnMat}, we define the following matrix for AB colorings.

\begin{definition}
\label{def:ABmatrix}
The \textbf{Alexander-Briggs $(R,t)$-coloring matrix} $P(t)$ of a Tait diagram with $n$ crossings is defined by $P_{rs}(t) = \wt(U_r,v_s)$ for $1 \le r \le n+2$ and $1 \le s \le n$ with $r,s \in \Z_{>0}.$
\end{definition}

Consider the morphism $R^n \rightarrow R^{n+2}$ of $R$-modules
given by the matrix $P(t)$. The \textbf{module of Alexander-Briggs $(R,t)$ colorings} of the Tait diagram is given by the kernel of this morphism, that is the solution space of the set of equations:
$$
\sum_{s=1}^n \wt(U_r,v_s) v_s \textnormal{ for } r=1,2,\ldots , n+2.
$$

\begin{proposition}
\label{p-PN}
Let $D_{\pm}$ be the $(n+2)\times (n+2)$ diagonal matrix with $1$ at entry $(i,i)$ if the region $U_i$ is white, and $-1$ if the region $U_i$ is black. Then $P(t)^T = N(t)D_{\pm}$ where the matrices $N(t)$ and $P(t)$ are as in \ref{def:DehnMat} and \ref{def:ABmatrix}, respectively.
\end{proposition}

\begin{proof}
The definition of $N(t)$ and $P(t)$ are such that the entries of $N_{ij}(t)$ and $P_{ji}(t)$ are the same up to a sign, 
and this sign is $+1$ if the region $U_j$ is white and $-1$ if the region $U_j$ is black.
\end{proof}

The next result shows that the module of AB colorings is invariant under Reidemeister moves.

\begin{corollary}
\label{cor:ReidemesterP}
If $D_1$ and $D_2$ are two equivalent Tait diagrams of knots, 
then the corresponding Alexander-Briggs $(R,t)$-coloring matrices~$P_1(t)$ and~$P_2(t)$ are equivalent.
\end{corollary}
\begin{proof}
This is a direct consequence of Propositions \ref{p-ReidemesterN} and \ref{p-PN}.
\end{proof}


\begin{corollary}
\label{cor:DAB}
Let $R$ be a field and $t$ a nonzero element of $R$. Then the dimension of the space of Dehn $(R,t)$-colorings is $2$ more than the dimension of the space of Alexander-Briggs $(R,t)$-colorings.
\end{corollary}

\begin{proof}
Let $D_{\pm}$ be as in \ref{p-PN} of size $n+2$. The ranks of $P(t)$ and $N(t)$ are the same by Proposition \ref{p-PN} since $D_{\pm}$ is an invertible matrix. The dimension of the module of Dehn $(R,t)$-colorings is equal to $n+2- rank(P(t))$.
The dimension of the module of Alexander-Briggs $(R,t)$-colorings is equal to $n- rank(N(t))$. 
\end{proof}

We conclude the section with a key remark that connects the three notions of colorability, showing that a Tait diagram is Alexander-Briggs $(R,t)$-colorable if and only if it is Dehn $(R,t)$-colorable if and only if it is Fox $(R,t)$-colorable. 

\begin{remark}\label{r-morphDtoAB}
One can generalize \cite[Theorem 3.1]{carter2014three} to show that 
there is a surjective morphism from the module of Dehn $(R,t)$-colorings to the module of Alexander-Briggs $(R,t)$-colorings 
that has as kernel a free $R$-module of rank $2$ consisting of the trivial Dehn $(R,t)$-colorings.
Hence a Tait diagram is Alexander-Briggs $(R,t)$-colorable if and only if it is  Dehn  $(R,t)$-colorable.
We saw already in Remark \ref{r-trivial-Dehn-col} that  a diagram is Fox $(R,t)$-colorable if and only if it is  Dehn  $(R,t)$-colorable.
Hence the three notions of colorability of a diagram coincide.
\end{remark}


\section{Codes from Knots and Their Properties}
\label{sec:codes}
\label{sec:3}


This section explains how 
one can construct a code
starting from 
a knot with its diagram and coloring. 
We also establish a series of results illustrating how the properties of knots determine those of codes via the said constructions.
We essentially regard
the three possible colorings of a knot diagram as a linear code over a finite field $\F_q$ with $q$ elements, that is $R = \F_q$. 
\begin{definition}
\label{def:knotcode}
Let $D$ be a knot diagram that is Fox $(\F_q,t)$-colored.
The \textbf{Fox code} associated with $D$ (or the \textbf{Fox knot code} of $D$) with  coloring matrix $M$ is 
$$
\mF_{D,t} = \{x \in \F_q^n \mid Mx^T = 0 \}.
$$
If $t=-1$, we denote this code by $\mF_{D}$. 
Similarly we define the \textbf{Dehn code}  and the \textbf{Alexander-Briggs code} of $D$ by
$$
\mD_{D,t} = \{x \in \F_q^n \mid Nx^T = 0 \} \ \mbox{ and } \ \mA_{D,t} = \{x \in \F_q^n \mid Px^T = 0 \}, 
$$
respectively where $N$ and $P$ are as in Definition \ref{def:DehnMat} and Definition \ref{def:ABmatrix}. The codes $\mF_{D,-1}$, $\mD_{D,-1}$ and $\mA_{D,-1}$ are abbreviated by $\mF_{D}$, $\mD_{D}$ and $\mA_{D}$, respectively.
\end{definition}

Note that the coloring matrix is interpreted as the parity check matrix of a code; see Definition \ref{def:parity}. 
A relabeling of the strands in the knot diagram will result in a possibly different code. 
Such a code is permutation equivalent to the original one 
and thus has the same dimension and minimum distance.
However, the Fox code of a knot diagram is not a knot invariant, as the following example illustrates.





\begin{example}
Let $q=19$. In Figure \ref{fig:76_flype} two diagrams of the same knot $K$ is given with~$\Delta_K(-1) = 19$. The Fox $(\F_{19},-1)$-coloring matrices of the knot diagrams depicted in Figure~\ref{fig:76_flype}, denoted by $H_a$ and $H_b$ respectively, are the parity check matrices of the corresponding knot codes.

% Figure environment removed

We apply row operations to transform the parity check matrices in the form $\bigl[-A^T \mid I \bigr]$, for some matrix $A$ of suitable size. 
This results in the generator matrices 
$$
G_a = \begin{pmatrix}
1 & 0 & 6 & 15 & 16 & 3 & 10\\
0 & 1 & 14 & 5 & 4 & 17 & 10
\end{pmatrix}
$$
and 
$$
G_b = \begin{pmatrix}
1 & 0 & 6 & 15 & 3 & 9 & 10\\
0 & 1 & 14 & 5 & 17 & 11 & 10
\end{pmatrix}.
$$
It is easy to see that the codes generated by them are not monomial equivalent. Therefore, the Fox code of a knot diagram is not a knot invariant. 
\end{example}

In the theory of knot colorings one is interested in the (minimum) number of colors used in a coloring. This number cannot be translated in results about the weight of the coloring, that is, the number of nonzero colors. In 1999, Kauffman and Harary conjectured the following \cite{harary1999} and it was proven in 2009 by Mattman and Solis in \cite{mattman2009proof}.


\begin{theorem}
\label{thm:conjecture}
Let $D$ be a reduced, alternating knot diagram (see Definition \ref{def:reduced_alt}) of $K$ with~$|\Delta_K(-1)|=p$, where $p$ is prime. Then, every non-trivial Fox $(\F_p,-1)$-coloring of $D$ assigns different colors to different strands of the diagram.
\end{theorem}


Note that Theorem \ref{thm:conjecture} is not true if the determinant of the knot is not a prime. The alternating knot diagram $7_7$ in \cite[Figure 25]{harary1999} has non-prime determinant $21$ and has a Fox $(\F_7,-1)$-coloring with $6$ colors such that two strands have the same color.


The Kauffman-Harary conjecture of Theorem \ref{thm:conjecture} motivates the following result.


\begin{proposition}
Let $D$ be a reduced, alternating knot diagram (see Definition \ref{def:reduced_alt}) of~$K$ with $n$ strands such that $|\Delta_K(-1)|=p$, with $p$ prime. Then, the Fox knot code of $D$ is an $[n,2,n-1]_p$ code over $\F_p$.
\end{proposition}



\begin{proof}
By Theorem \ref{thm:conjecture}, every non-trivial coloring of $D$ assigns different colors to different strands. This implies that the minimum distance of the Fox knot code is $n-1$ which is attained by~$d(c,c')$ where $c$ is any non-trivial coloring and $c'$ is a trivial coloring where all strands have color $c_i$ for some $i \in \{1,\ldots,n\}$.  By Proposition \ref{prop:determinant}, we have that $D$ is Fox $(\F_p,-1)$-colorable since $p \mid \Delta_K(-1)=p$. Non-trivial colorability implies that the dimension of the Fox knot code is at least $2$. The only possible code parameters are $[n,2,n-1]_p$ by the Singleton bound of Theorem \ref{thm:sbound}.
\end{proof}

Note that the Fox knot codes with Fox coloring matrix as their parity check matrices are right $3$-regular LDPC codes. If a knot diagram is alternating, it gives a $(3,3)$-doubly-regular LDPC code. Moreover, if one considers the Dehn colorings, then the corresponding code is a right $4$-regular LDPC code. For the rest of this section, when we say coloring matrix, we mean the Alexander matrix of Definition \ref{def:coloringMat}.

Regarding the minimum distance of a Fox knot code, one can obtain the following, rather simple, result.

\begin{proposition}
\label{prop:minimumdistance}
A Fox code of a knot diagram of a non-trivial knot has minimum distance at least 2.
\end{proposition}
\begin{proof}
Suppose there exists a Fox code of a knot diagram with minimum distance 1. Then this code contains a codeword of weight 1, which corresponds to a coloring of the knot diagram in which only one of the strands is colored with a color $c \in R\backslash\{0\}$. In case this strand is an overstrand at a crossing in the diagram, it is also an understrand at another crossing in the diagram, unless it is the trivial knot. Then, there exists a crossing for which it should hold that $0 - 0 = t(c - 0)$ or $c - 0 = t(0 - 0)$, depending on which understrand of the crossing is colored. It follows that $c = 0$ should hold as $t$ is invertible over $R$. From this contradiction, it follows that the minimum distance of the code is at least 2.
\end{proof}

We will return to the minimum distance of Fox knot codes in Remark~\ref{rem:minimumdist} and in Subsection~\ref{subsec:pretzel}. We can already disclose that it is not a knot invariant; see Remark \ref{rem:minimumdist} for the details. 



\subsection{Dimension of Codes from Knot Diagrams}
\label{subsec:dim}


In this subsection we investigate
the dimension of a Fox code of a knot diagram. In particular, we prove that the dimension of a Fox code of a knot diagram is a knot invariant. We start with an observation.

\begin{remark}
\label{rem:dim1}
The $n$-repetition code of Example \ref{ex:repetition} is always a subcode of the Fox code of a knot diagram with $n$ strands, as trivial colorings are always possible. Therefore, the dimension of the Fox code of a knot diagram is at least 1. Conversely, when the dimension of the Fox code of a knot diagram is larger than 1, the knot has a non-trivial coloring.
\end{remark}



We directly start with one of the main theorems of the subsection.


% Figure environment removed


\begin{theorem}
\label{thm:dimreidemeister}
Let $D$ and $D'$ be equivalent knot diagrams. Then 
$\mF_{D,t}$ and $\mF_{D',t}$ have the same dimension.
\end{theorem}

\begin{proof}
Let $D$ and $D'$ have $n$ strands. Denote by $\mF_{D,t}$ and $\mF_{D',t}$ the respective codes related to the diagrams. One locally investigate what happens when performing each Reidemeister move, see 
Figure~\ref{fig:fox_reidemeister}. Suppose $D'$ is obtained from $D$ by twisting a strand $x_1$ (Reidemeister move of type I), then the twist results in two strands and a crossing in this part of the diagram, where both strands are the understrands and one of the strands is the overstrand. For a Fox coloring it then follows that the colors assigned to both strands must be the same. 
    Let 
    $$G = 
    \begin{pmatrix}
     | & | & | \\
    \dots & x_1 & \dots \\
    | & | & |
    \end{pmatrix}
    $$ 
    be a full rank $k \times n$ generator matrix of $\mF_{D,t}$.
    Then    $$G' = 
    \begin{pmatrix}
    | & | & | & |\\
    \dots & x_1 & x_1 & \dots\\
    | & | & | & | 
    \end{pmatrix}
    $$ 
    is a $k \times (n+1)$ generator matrix for $\mF_{D'}$, which has the same rank since the added column is a duplicate of another column.

    The other moves can be investigated
    in a similar manner and we omit the proof here. By Theorem \ref{thm:reidemeistermoves},
any two diagrams of a knot can be transformed into each other using Reidemeister moves. It follows that $\mF_{D,t}$ and $\mF_{D',t}$ have the same dimension.
\end{proof}



The next result derives an upper bound for the dimension of Fox knot codes. 

\begin{theorem}
\label{thm:dimension-ineq}
Let $D$ be a knot diagram with $n$ strands and let $\mF_{D,t}$ be the corresponding Fox knot code over $\F_q$. We have
$$
1 \le \dim(\mF_{D,t}) \le \frac{n+1}{2} .
$$ 
\end{theorem}
\begin{proof}
By Theorem \ref{thm:dimreidemeister}, the Reidemeister moves do not affect the dimension of a Fox code of a knot diagram. Therefore, without loss of generality, let $D$ be a knot diagram that does not have any crossings which can be removed via the twist move, see Definition \ref{def:reidemeister} 
(that is, there exists no crossing in $D$ such that the overstrand and one of the understrands of the crossing are the same strand).  
Label the strands of $D$ as $x_1, x_2, ..., x_n$ by following the knot in one direction. Denote by $c_i$ the crossing where $x_i$ and $x_{i+1}$ are understrands and $x_{i'}$ is the overstrand, and with $x_{i+1}$ being the strand leaving the crossing with respect to the chosen orientation. Note that $i' \in \{1, \ldots, n\}$ depends on $i$. Since $D$ does not contain twists, we have that either $i' < i$ or $i' > i+1$. By going through the crossings $c_i$ with $1 \le i \le n-1$, we define the sets $L = \{ c_i \mid 1 \leq i \le n-1,\  i' < i\}$ and~$U= \{ c_i \mid 1 \leq i \le n-1,\  i' > i+1\}$. As $L \cap U = \emptyset$ and $L \cup U = \{1, \ldots, n-1\},$ we find that either $|L| \geq \frac{n-1}{2}$ or $|U| \geq \frac{n-1}{2}$, since $|L \cup U| = n-1.$ 
Then, the parity check matrix $H$ of $\mF_{D,t}$ can be constructed like in Definition~\ref{def:coloringMat} as follows. 

Let 
$$
H_{ij}(t) = 
\begin{cases}
x & \text{if} \ \ j=i,\\
y & \text{if} \ \ j \equiv i+1 \pmod n,\\
1-t & \text{if} \ \ j=i',\\
0 & \text{otherwise},
\end{cases}
$$ where $(x,y) \in \{(-1,t),(t,-1)\}$ depending on the diagram as in Definition~\ref{def:coloringMat}. 

Throughout the rest of the proof, we show that $\rk(H) \ge \frac{n-1}{2}$. When~$|U| \geq \frac{n-1}{2}$, take the submatrix $H'$ of $H$ consisting of the rows corresponding to the crossings in $U$. 
Then $H'$ is in row echolon form since the $i$-th row of $H$ such that $c_i \in U$ only has nonzero entries at positions $i$, $i + 1$ and $i'$ with $i' > i+1$. Thus, $\rk(H') = |U|$. This implies that $\rk(H) \ge |U|$ and consequently $\mF_{D,t}$ has dimension at most $n - |U|$. Similarly, if $|L| \geq \frac{n-1}{2}$ then the submatrix $H''$ consisting of the rows of $H$ corresponding to the crossings in $L$ is in column echelon form since the $i$-th row of $H$ such that $c_i \in L$ only has nonzero entries at positions $i$, $i + 1$ and $i'$ with $i' < i$. Therefore, we have $\dim(\mF_{D,t}) \leq n - \frac{n-1}{2} = \frac{n+1}{2}$.
\end{proof}

In addition to Theorem \ref{thm:dimension-ineq}, the following two results hold 
about the dimension of Fox knot codes.


\begin{proposition}
\label{prop:dimension}
Let $D$ be a knot diagram with $n$ strands and let $\mF_{D,t}$ be the corresponding Fox knot code over $\F_q$. 
Let $\dim(\mF_{D,t}) = k$. Then $k$ is the smallest integer with the property that~$E_k(M(t))=\F_q$.
\end{proposition}

\begin{proof}
Let $\dim(\mF_{D,t}) = k$. Then $M(t)$ has rank $r=n-k$, since $\mF_{D,t}$ is the null space of the matrix $M(t)$. 
Proposition \ref{p-elem} states that $E_l(M(t))=\F_q$ if $l\geq n-r=k$ and $E_l(M(t))=0$ if $l<k$.
Hence $k$ is the smallest integer such that $E_k(M(t))=\F_q$.
\end{proof}

Note that Proposition~\ref{prop:dimension} is also stated in \cite[Corollary 12]{traldi2018}, where it has a longer proof. 

\begin{proposition}\label{p-dim-prime-e}
Let $p$ be a prime number and let $t$ an integer such that $1\leq t<p$.
Let~$D$ be a knot diagram and let
$\mF_{D,t}$ be the corresponding Fox knot code over $\F_p$. 
Let $e$ the largest integer such that $p^e$ divides $\Delta_K(t)$ in $\Z$. 
Then $\dim(\mF_{D,t}) \leq e+1$, and equality holds if $e=1$.
\end{proposition}

\begin{proof}
Our proof uses Proposition \ref{p-number-color-Z} with $R=\Z_t$, the localization of $\Z$ at $t$, and $d=p$. In particular,  $\overline{R}=\F_p$ and 
$t$ is an invertible element in $\F_p$.
Let $(d_1)\subseteq (d_2)\subseteq \cdots \subseteq (d_l)$ be the invariant factors of the module of Fox $(\Z_t,t)$-colorings.
Then $\Delta_K(t)=\prod_{i=2}^n d_i$ and $d_1=0$ by Proposition \ref{p-elem-ideals-Alex-pol}.
Let $\overline{d_i}=\gcd (p,d_i)$ and let $\overline{e}$ be the number of integers $i$ with $2 \leq i \leq l$ and $\overline{d_i}=p$. 
Furthermore, $p^{\overline{e}}$ divides $\Delta_K(t)$ in $\Z$, hence
$\overline{e} \leq e$.
Then $p\prod_{i=2}^n\overline{d_i}$ is the number of Fox~$(\F_p,t)$-colorings of $D$ by Proposition~\ref{p-number-color-Z}. So $\dim(\mF_{D,t})=1 + \overline{e}$
and $\dim(\mF_{D,t})\leq 1 + e$. If~$e=1$, then the underlying knot, say $K$, is Fox $(\overline{R},t)$-colorable by Proposition \ref{prop:determinant}. 
So~$1<\dim(\mF_{D,t})\leq 1 + e$
and $\dim(\mF_{D,t})=2$.
\end{proof}

We finish this section with the following observation.


\begin{remark}
\label{rem:minimumdist}
The minimum distance of codes of knots is not a knot invariant. This can be seen, for example, from the generator matrix of the first Reidemeister move in the proof of Theorem~\ref{thm:dimreidemeister}.
\end{remark}


\section{Two Families of Fox Knot Codes}
\label{sec:families}
\label{sec:4}

This section is devoted to the study of two families of knots and their codes, namely \textit{torus knots around other knots} and \textit{pretzel knots}.
These can be both used to construct codes with 
interesting parameters and will be treated in dedicated subsections. In the sequel, for ease of notation we will write
$E_k(t)$
instead of
$E_k(M(t))$,
where $M(t)$ is the coloring matrix of the knot diagram at hand.


\subsection{Torus Knots}
\label{subsec:torus}
The notion of a torus knot was already introduced in Definition \ref{def:torusknot}. In this subsection we investigate the properties of these knots and their generalizations. We then study the dimension of codes of knot diagrams of these knots and show how to construct codes of arbitrary dimension.

\begin{remark}\label{tubular-nbhd}
There exits a closed \textbf{tubular neighbourhood} of $K$, denoted by $\tub(K)$, 
such that $\tub(K)$ is homeomorphic to $S^1\times D^2$ via a homeomorphism $h$ where $K$ is mapped to $S^1\times \{0\}$,
and the boundary of $\tub(K)$ is homeomorphic to the torus $S^1\times S^1$. See \cite{hirsch1968} for more details.
\end{remark}

Generalizations of torus knots are defined as follows; see \cite{burau1934knots,le1972knots}.

\begin{definition}
\label{def:cableknot}
Let $\tub(K)$ be a {tubular neighbourhood} of~$K$.
Let~$a,b$ be positive integers that are relatively prime.
Then the curve on $S^1\times S^1$ given by the parametrization $\varphi (t)=at$, $\theta (t)=bt$, is mapped via $h^{-1}$ of Remark \ref{tubular-nbhd} to a knot on the
boundary of the tubular neighbourhood $\tub(K)$. 
This knot is called the $(a,b)$-\textbf{torus knot around} $K$ and is denoted by $K(a,b)$. 
By induction, we can repeat this procedure for some integer $m \in \Z_{\ge 2}$ to obtain the $(a_1,b_1,\ldots ,a_m,b_m)$-\textbf{iterated torus knot} $K(a_1,b_1,\ldots ,a_m,b_m)$ \textbf{around} $K$, where
the pairs $(a_i,b_i)$ are relatively prime  and $K(a_1,b_1,\ldots ,a_i,b_i)$ is the $(a_i,b_i)$-torus knot around
$K(a_1,b_1,\ldots ,a_{i-1}, b_{i-1})$ for all~$i \in \{1,\ldots,m\}$. 
\end{definition}


\begin{remark}
    It can be seen that Definition~\ref{def:cableknot} generalizes torus knots, in the sense that the torus knot $T(a,b)$ is the $(a,b)$-torus knot around the trivial knot or unknot $U$.
\end{remark}

The Alexander polynomial of a torus knot has a rather simple expression.
 
\begin{proposition}\label{p-alex-torus}

Let~$a,b$ be positive integers that are relatively prime.
The Alexander polynomial of the torus knot $T(a,b)$ is given by 
$$
\Delta_{T(a,b)}(t) =\frac{(t^{ab}-1)(t-1)}{(t^a-1)(t^b-1)}.
$$
Moreover, the $k$-th elementary ideal of $T(a,b)$ is $\Z[t,t^{-1}]$ for all $k \geq 2$.
\end{proposition}
\begin{proof}
See  \cite[Chapter 3]{burau1934knots}, \cite[VIII Exercise 3]{crowell2012introduction}, 
and \cite[Theorem 7.3.2]{murasugi1996knot}.
\end{proof}

We give an example to show how Proposition \ref{p-alex-torus} can be used.



\begin{example}\label{ex:torus}

Let~$a,b$ be positive integers that are relatively prime. Let $D$ be a diagram of $T(a,b)$. Proposition \ref{prop:determinant} implies the following.
\begin{enumerate}
    \item If $a$ and $b$ are odd, then $\Delta_{T(a,b)}(-1)=1$ and there are only trivial Fox $(\F_p,-1)$-colorings of $D$.
    \item If $a$ is odd and $b$ is even, then $\Delta_{T(a,b)}(-1)=a$, and $D$ is Fox $(\F_p,-1)$-colorable if and only if $p$ divides $a$.
    \item If $b$ is odd and $a$ is even, then $\Delta_{T(a,b)}(-1)=b$, and $D$ is Fox $(\F_p,-1)$-colorable if and only if $p$ divides $b$.
\end{enumerate}

The dimension over $\F_p$ of $\mF_D$ is 1 in the first case and 2 in the second and the third case by Proposition \ref{prop:dimension}, since the second elementary ideal is the whole ring by Proposition \ref{p-alex-torus}. If $ab$ divides $q-1$, then there exists an element $t$ in $\F_q^*$ of order $ab$. So $\Delta_{T(a,b)}(t)=0$, $D$ is $(\F_q,t)$-colorable, and the dimension over $\F_q$ of $\mF_{D,t}$ is 2.
\end{example}



\begin{remark}\label{r-p-order-det} 
The inequality in Proposition \ref{p-dim-prime-e} is in general not an equality (see \cite[Chapter 3, §4, Exercise 4.6]{livingston1993knot}), 
contrary to what is stated in \cite[Theorem 23]{henrich2022}. This can also be seen by taking $K=T(2,9)$. We have~$\Delta_K(-1)=9$, and thus the largest integer $e$ such that $3^e$ divides 9 is $e=2$ in this case. However, the dimension of the code over $\F_3$ is equal to $2$, showing that the bound of Proposition \ref{p-dim-prime-e} is not sharp in general. 
\end{remark}


We can determine the elementary ideals of the knot $K(a,b)$ in terms of the elementary ideals of the knot $K$ and the Alexander polynomial of the torus knot $T(a,b)$.

\begin{proposition}\label{p-elem-torus-around-K}
Let~$a,b$ be nonzero integers that are relatively prime. We have
$$
\tilde{E}_k(t) = \Delta_{T(a,b)}(t)E_k(t^b) + E_{k-1}(t^b).
$$
where $E_k(t)$ denotes the $k$-th elementary ideal of $K$ and $\tilde{E}_k(t)$ denotes the $k$-th elementary ideal  of the knot~$K(a,b)$.
\end{proposition}
\begin{proof}
See \cite{le1972knots} and \cite[Proposition 10.5]{pellikaan1981knopen}.
\end{proof}

We have the following two corollaries of Proposition \ref{p-elem-torus-around-K}.

\begin{corollary}\label{c-alex-torus-around-K}
Let~$a,b$ be positive integers that are relatively prime. Then the Alexander polynomial of the $(a,b)$-torus knot around $K$ is given by
$$
\Delta_{K(a,b)}(t)=\Delta_{T(a,b)}(t)\Delta_K(t^b).
$$
\end{corollary}
\begin{proof}
$\Delta_{K(a,b)}(t)$ is a generator of the principal ideal $\tilde{E}_1(t)$,
$\Delta_K(t)$ is a generator of the principal ideal $E_1(t)$ in Proposition \ref{p-elem-torus-around-K}, and 
$\tilde{E}_1(t) = \Delta_{T(a,b)}(t)E_1(t^b) + E_0(t^b)$. This gives the desired result since $E_0(t) =(0)$.
\end{proof}

\begin{corollary}\label{c-dim-torus-around-K}
Let~$a,b$ be nonzero integers that are relatively prime such that $a$ is even and $b$ is odd. Let $p$ be a prime divisor of $b$.
Let $K$ be a knot, $D$ a diagram of $K$, and let~$k$ denote the dimension of $\mF_{D,t}$ over $\F_p$.
Then the code $\mF_{\tilde{D},t}$ of a diagram $\tilde{D}$ of $K(a, b)$ has dimension $k + 1$ over $\F_p$.
\end{corollary}
\begin{proof}
Let $E_l(t)$ be the $l$-th elementary ideal of $K$. Then  $E_l(-1)=\F_p$ if $l\geq k$, and $E_l(-1)=0$ if $l<k$ by Proposition \ref{prop:dimension}.
Since $p$ is a prime that divides~$b$, $a$ is even and~$b$ is odd, we have $\Delta_{T(a,b)}(-1)=0$ in $\F_p$ by Example \ref{ex:torus} and $(-1)^b=-1$.
Let $\tilde{E}_l(t)$ be the $l$-th elementary ideal of $(a,b)$-torus knot around~$K$.
Then  $\tilde{E}_l(-1)=\F_p$ if $l\geq k+1$, and $\tilde{E}_l(-1)=0$ if $l<k+1$, by Proposition \ref{p-elem-torus-around-K}.
Hence~$\mF_{\tilde{D}}$ has dimension $k+1$ by Proposition~\ref{prop:dimension}, as claimed.
\end{proof}


In the next example we show how to build codes using 
iterated torus knots around other knots.

\begin{example}
Let $p$ be an odd prime. A diagram of the iterated torus knot $K(2,p,\ldots ,2,p)$, where $K$ is the unknot and $(2,p,\ldots ,2,p)$ is the $m$-fold
repetition of $(2,p)$, gives a code over~$\F_p$ of dimension $m+1$ by Corollary \ref{c-dim-torus-around-K}. 
The recursive formula of the length of the code of~$K(2,p,\ldots ,2,p)$ is given by $n_1=3$, $n_{m+1} =4n_m+p$.
\end{example}

We conclude this subsection with the following crucial example.

\begin{example}\label{ex:min-dist-fox-dehn}
Let $b=2l+1$ be a positive odd integer for some $l$ and let $T(2,b)$ be the torus knot as given in Definition \ref{def:torusknot}. Consider its diagram depicted in Figure \ref{fig:2btorus}. This is a diagram with $b$ crossings where the upper left understrand is connected with the lower left overstrand, and the upper right overstrand is connected with the lower right understrand.
Denote the upper left understrand by $x_1$ and the upper right overstrand by $y_1$. 
Denote the strands by following the knot's orientation from 
the upper left understrand $x_1$ to the lower right understrand by~$x_1, x_2, \ldots ,x_{l+2}$, respectively.
Denote the strands following the knot's orientation from the
upper right overstrand $y_1$ to the lower left overstrand by $y_1, y_2, \ldots ,y_{l+1}$, respectively.
Then~$x_{l+2}=y_1$ and $y_{l+1}=x_1$, see again Figure \ref{fig:2btorus}.

% Figure environment removed

Let $p$ be a prime and suppose that the strands are Fox $(\F_p,-1)$ colored. 
Then we get by induction that $x_{i+1}=2iy_1-(2i-1)x_1$ and $y_{i+1}=(2i+1)y_1-2ix_1$.
So $x_{l+2}=y_1$ and~$y_{l+1}=x_1$ imply $by_1-bx_1=0$ in $\F_p$.
Hence $T(2,b)$ is  Fox $(\F_p,-1)$-colorable  if and only if $b$ is divisible by $p$.

If $b=p$, then we get a non-trivial coloring with~$x_{i+1}=2i$ and $y_{i+1}=2i+1$.
So all the strands have mutually distinct colors, which is in agreement with Theorem \ref{thm:conjecture}, since $\Delta_{T(2,p)} (-1)= p$ by Proposition \ref{p-alex-torus}. Furthermore, the Fox colorings have weight $1$ (all strands have color $0$), or $p$ (all strands have the same nonzero color), or $p-1$ for a non-trivial coloring, when the colorings are viewed as codewords as in Section \ref{sec:codes}. A checkerboard coloring (see Definition~\ref{def:checkerboard}) of the regions has~$p$ regions with color~$0$ (white in Figure \ref{fig:toruscheckerboard}), where the unbounded region is colored white, and two regions of nonzero color~(black in the figure). This gives a Dehn $(\F_p,-1)$-coloring of the diagram  of weight $2$. So the isomorphism of modules as mentioned in Proposition \ref{prop:fox2dehn} sends a word of weight $2$ to a word of weight $p$. Hence the isomorphism is not an isometry if $p>3$. 
\end{example}

\subsection{Pretzel Knots}
\label{subsec:pretzel}

In this subsection we prove that also pretzel knots codes can be used to construct codes with prescribed dimension. Moreover, we study the error correction capability of these codes. Starting from knots, one can create larger objects called \textit{links}.


\begin{definition}
\label{def:link}
Let $n \in \Z_{\ge 1}$. A \textbf{link} $L = \{K_1,\ldots, K_n\}$ is a finite collection of knots such that $K_i \cap K_j = \emptyset$ for all  $i,j \in \{1,\ldots,n\}$ with $i \neq j$. Each of the constituent knots is a \textbf{component} of the link. In particular, a \textbf{polygonal link} is a link each of whose component is a polygonal knot. 
\end{definition}

Since we only consider polygonal knots in this paper, we only consider polygonal links and simply write \textbf{link} for those.


% Figure environment removed



Next, we informally define pretzel links following \cite{kawauchi1996survey}.


\begin{definition}
\label{def:pretzelink}
A \textbf{twist} is a part of a knot diagram consisting of two strands and at least a crossing such that all the crossings are obtained using both strands together, and it is of the form depicted in in Figure~\ref{fig:twists}. A twist with $|b| \in \Z_{>0}$ crossings is called a $(|b|)$-crossing twist if the top right strand is an overstrand, and is called a $(-|b|)$-crossing twist if the top right strand is an understrand. Let $p_1,\ldots,p_m$ be nonzero integers for some $m \in Z_{> 0}$. A \textbf{pretzel link} is a link with its diagram depicted as in Figure~\ref{fig:pretzellink}, where each rectangle denotes a twist with $|p_i|$ crossings. We denote this object by $P(p_1,\ldots,p_m)$. It is obtained when multiple twists are placed next to each other, where for each pair of neighboring strands the top and bottom right strands of the left twist are connected to the top and bottom left strands of the right twist, respectively, and the the top and bottom left strands of the leftmost twist are connected to the top and bottom right strands of the rightmost twist, respectively. 
\end{definition}

% Figure environment removed


The sufficient and necessary condition when a pretzel link is a knot is proven in \cite{kawauchi1996survey}. 
\begin{proposition}
\label{lem:knot}
A pretzel link $P(p_1,\ldots,p_m)$ is a knot if and only if $m$ and $p_i$ are odd integers for all $i \in \{1,\ldots,m\}$, 
or $m \ge 1$ and exactly one of the the $p_i$ is even.
\end{proposition}

Next, we show that codes with any dimension can be constructed using pretzel knots.


\begin{theorem}(see \text{\cite[Theorem 17]{kolay2023}})
\label{thm:pretzeldim}
Let $D$ be a diagram of the pretzel knot $P(p_1, p_2, ..., p_m)$.
Let $q$ be a power of the prime $p$.
\begin{enumerate}
    \item If $p_i$ is coprime with $q$ for all $i \in \{1,\ldots,m\}$, 
then the dimension of a Fox knot code $\mF_D$ over~$\F_q$ is given by
$$
\dim(\mF_D) = 
\begin{cases}
2 & \text{if} \ \ p \mid \Delta_K(-1),\\
1 & \text{otherwise}.
\end{cases}
$$
\item If there exists a~$p_i$ that is not coprime with $q$ for some $i \in \{1,...,m\}$, then the dimension of $\mF_D$ over $\F_q$ is 
$|\{i \mid \gcd(p_i, q) \neq 1,\, i \in \{1, \ldots, m\} \}|$.
\end{enumerate}

\end{theorem}



We give an example to show an application of Theorem \ref{thm:pretzeldim}.


% Figure environment removed


\begin{example}
\label{ex:pretzel}
The $P(3,2,3,5)$ pretzel knot is depicted in Figure \ref{fig:3235pretzel} has determinant $123 = 3 \cdot 41$, which means it is non-trivially colorable over $\mathbb{F}_3$ and $\mathbb{F}_{41}$ by Proposition \ref{prop:determinant}. By the second part of Theorem~\ref{thm:pretzeldim}, we then find that the code obtained from the colorings over $\mathbb{F}_3$ has dimension~2. By the first part of Theorem~\ref{thm:pretzeldim}, we also have that the code obtained from the colorings over $\mathbb{F}_{41}$ has dimension 2, as well.
\end{example}



For some special family of pretzel knots, we can determine the exact code parameters of the corresponding code. Computing the minimum distance in the general case seems to be a hard task. 


\begin{proposition}
\label{prop:pretzmind}
Let $p$ be an odd prime and $D$ be a diagram of $P(p_1,\ldots ,p_m)$ with $p_i=p$ for all $i \in \{1,\ldots,m\}$. Then,  
$\mF_D$ is a $[pm,m,2p-2]_p$ code.
\end{proposition}

\begin{proof}
The statement about the dimension is already shown in Theorem \ref{thm:pretzeldim}.
Consider the numbering of the strands of the diagram $D$ analogous to Example \ref{ex:min-dist-fox-dehn}.
Let $p=2l+1$ for some $l \in \Z$.
For $i \in \{1,\ldots,m\}$, number the strands in the $i$-th block as follows. 
Following the knot's orientation, label the strands from the upper left $x_{i,1}$ to the lower right by $x_{i,1}, x_{i,2}, \ldots ,x_{i,l+2}$.
Following the knot's orientation, label
the strands from the
upper right $y_{i,1}$ to the lower left by $y_{i,1}, y_{i,2}, \ldots ,y_{i,l+1}$.
The strand $y_{i,1}$ is equal to $x_{i+1,1}$ for all~$i \in \{1,\ldots,m-1\}$, and $y_{m,1}$ is equal to $x_{1,1}$.
The strand $y_{i,l+1}$ is equal to $x_{i-1,l+2}$ for all~$i \in \{2,\ldots,m\}$, and $y_{1,l+1}$ is equal to $x_{m,l+2}$.

The $i$-th block consist of $2l+1=p$ strands
$\smash{x_{i,1}, x_{i,2}, \ldots ,x_{i,l+2}}$ and $\smash{y_{i,2}, \ldots ,y_{i,l}}$, since $\smash{y_{i,1}=x_{i+1,1}}$ and $\smash{y_{i,l+1}=x_{i-1,l+2}}$.
The values of $\smash{x_{1,1}, x_{2,1}, \ldots ,x_{m,1}}$ determine all the values of the other strands due the our assigning above.
Now $x_{i,1}=0$ and $x_{i+1,1}=0$ if and only if all the strands in the~$i$-th block have value zero.
If $x_{i,1}\not=0$, then at least $p-1$ strands of the $i$-th block and~$p-1$ strands of the $(i-1)$-th block have nonzero value as shown in Example~\ref{ex:min-dist-fox-dehn}.
Hence the weight of a nonzero codeword is at least $2(p-1)$.

Choosing $x_{1,1}=1$ and $x_{i,1}=0$ for all $i\neq 1$ gives a codeword of weight $2(p-1)$.
Hence~$\mF_D$ has indeed minimum distance $2(p-1)$ and rate $R=m/pm= 1/p$.
\end{proof}



\section{Knot Graphs and Their Codes}
\label{sec:graph}

Starting from Tait diagram of knots one can construct graphs, see \cite{kauffman1983,kauffman1987,kauffman1988,kauffman1989,kauffman1991,kauffman2012}. We assume that the reader is familiar with basic concepts in graph theory, see \cite{west2001introduction} as a reference. 

\begin{definition}
\label{def:signedgraph}
Let $D$ be a Tait diagram of a knot, and $D^*$ be equal to $D$ with the interchanged checkerboard coloring. The \textbf{black graph} of $D$ is the planar graph $\Gamma _D$ whose vertices are the black regions of $D$. There is an edge between two vertices if the black regions in the Tait diagram corresponding to these vertices have a crossing in their common boundaries. Similarly, $\Gamma _{D^*}$ is called the \textbf{white graph} of $D$. The graphs can be made directed by choosing the direction from the region without a dot to the region that has a dot near the crossing in their common boundary. See Figure \ref{fig:signedgraph} for illustration.
\end{definition}


% Figure environment removed


We define \textit{graph codes} from such directed graphs using their incidence matrices.

\begin{definition}
\label{def:graphcode}
Let $\Gamma$ be a directed graph, $v_1,\ldots,v_m$ be an enumeration of the vertices of the graph, and $e_1,\ldots ,e_n$ be an enumeration of the edges of the graph. Let $R$ be a ring and let $t\in R$ be an invertible element of $R$. Define $A(t)$ be the $m \times n$ matrix with entries:
$$
A(t)_{ij} = 
\begin{cases}
1 & \text{if} \ \ e_j \textnormal{ is an outgoing edge of } v_i,\\
t & \text{if} \ \ e_j \textnormal{ is an ingoing edge of } v_i,\\
0 & \text{otherwise}.
\end{cases}
$$
Then, the $R$-linear code with parity check matrix $A(t)$ is denoted by $\mC_{\Gamma,t}$.
\end{definition}

\begin{remark}\label{r-A(t)}
If $t=-1$ in Definition \ref{def:graphcode}, then $A(-1)$ is the incidence matrix of $\Gamma$ and has the property that the sum of the rows is the all-zero vector.
So, deleting a row of $A(-1)$ gives a matrix that is still a parity check matrix of $\mC_{\Gamma,-1}$.
The code $\mC_{\Gamma,-1}$ is abbreviated by $\mC_{\Gamma}$ and is called the \textbf{cycle code} of $\Gamma $,
and its dual is called its \textbf{graph code}.
Sometimes the cycle code is called graphic or cographic, see \cite[\S 8.1.2]{pellikaan2018}. 
\end{remark}

We note that the graph code is denoted by $C_{\Gamma}$ in \cite[\S 8.1.2]{pellikaan2018}. It corresponds to the notation $\mC_{\Gamma}^\perp $ in this paper.


\begin{remark}\label{r-param-cyclecode}
Let $\Gamma$ be a connected graph with $V$ vertices and $n$ edges.
The cycle code $\mC_{\Gamma}$ of $\Gamma $ is an $[n,k,d]$ code,
where $k=n-V+1$ and $d$ is the girth, the length of the smallest cycle, of $\Gamma$, see \cite[Proposition 8.1.22]{pellikaan2018}. 
\end{remark}

Throughout this section, we point out whether the defined codes are asymptotically good or not, and propose an open question at the end, see Definition \ref{def:asympgood}. Graphic and cographic codes are not asymptotically good \cite{kashyap2008}.

Definition \ref{def:signedgraph} motivates the following definition.

\begin{definition}
\label{def:blackcode}
Let $R$ be a ring and let $t\in R$ be an invertible element of $R$.
Let $\Gamma _D$ and $\Gamma_{D^*}$ be the black and white directed graphs of the Tait diagram $D$ of a knot. 
The codes $\mC_{\Gamma_D,t}$ and $\mC_{\Gamma_{D^*},t}$ of these graphs are called \textbf{black code} and \textbf{white code}, 
and denoted by $\mC_{D,t}$ and $\mC_{D^*,t}$, respectively.
And $\mC_{D,-1}$ and $\mC_{D^*,-1}$ are abbreviated by $\mC_{D}$ and $\mC_{D^*}$, respectively.
\end{definition}

Let $B_D$ be the incidence matrix of the black directed graph $\Gamma_D$ of $D$ and let  $W_D$ be the incidence matrix of the white directed graph $\Gamma_{D^*}$ of $D$.
Let $\mathbf{b}$ be a row of $B_D$ corresponding to the black region $B$. Then, the entries of $\mathbf{b}$ correspond to edges of $\Gamma _D$ which correspond to crossings of $D$. The entry is $0$ if the crossing is not in the boundary of $B$, it is $1$ if the crossing is in the boundary of $B$ and there is no dot in $B$ near that crossing, and it is $-1$ if the crossing is in the boundary of $B$ and there is no dot in $B$ near that crossing. Similarly, we do the same for $W_D$. These matrices are the parity check matrices of the black and white codes, respectively.


 \begin{theorem}
 \label{p-dual-black-white}
 Let $D$ be a reduced Tait diagram of a knot.
 If the characteristic is $2$ or the diagram is alternating, 
 then the black and white codes are dual to each other, i.e., $\mC_D^\perp =\mC_{D^*}$.
 \end{theorem}
 
 \begin{proof}
 Let $v$ be a crossing in the the intersection of the boundaries of a black and a white region of $D$.
 Then $v$ is a crossing of $D$ and it lies on a piece of a strand, call it $e$, between $v$ and another crossing $v'$ and 
 that is in the boundary of both a black and a white region.
 Then $v\not=v'$, otherwise $e$ can be deformed such that one get a loop that it is not self-intersecting and is in the interior of one the regions except $v$. So we get an unknot that intersects the diagram $D$ in exactly $v$, that means that $v$ is a reducible crossing which contradicts the assumption that $D$ is reduced. Hence $e$ is not a loop and there is a unique crossing $v'$ which is distinct from $v$ and is incident to $e$. In particular, in the the intersection of the boundaries of a black and a white region of $D$ the number of crossings is even.
 
 Let $\mathbf{b}_i$ be a row $B_D$ corresponding to the black region $B_i$ and $\mathbf{w}_j$ a row of $W_D$ corresponding to the white region $W_j$. If the characteristic is $2$, then
 \begin{equation*}
 \mathbf{b}_i \cdot \mathbf{w}_j =  \sum_{v \in \partial B_i \cap \partial W_j}1 = 0
 \end{equation*}
 is equal to 0 since $|\partial B_i \cap \partial W_j |$ is even, proving the result.
 Now, suppose that the diagram is alternating. 
 If $\mathbf{b}_i \cdot \mathbf{w}_j $ has a nonzero contribution at a crossing $v$ in the summation, 
 then the crossing is in the intersection of the boundaries $B_i$ and $W_j$. 
 The crossings appear in pairs, so  there are distinct crossings $v$ and $v'$ that are endpoints of the piece of a strand $e$ that is contained  
 $ \partial B_i \cap \partial W_j$. 
 Since the diagram is alternating, we may assume that $e$ is part of an overcrossing at $v'$ and of an undercrossing at $v$.
 Suppose that $B_i$ is on the right-hand side of $e$ and $W_j$ is on the left-hand side of $e$. (Similar reasoning follows if it is the other way around.) Then the entry of $ \mathbf{b}_i$ at $v'$ is $1$ and the entry of $ \mathbf{w}_j$ is $-1$, since $e$ is part of an overcrossing at $v'$. 
 So the contribution to the inner product is $1\cdot (-1)=-1$. 
 The entries of $ \mathbf{b}_i$ and $\mathbf{w}_j$ at $v$ are both $1$ or both $-1$, since $e$ is part of an undercrossing at $v$.
 So the contribution to the inner product is $1$ in that case. Hence, the nonzero contributions to $ \mathbf{b}_i \cdot \mathbf{w}_j$ appear in pairs of $\pm 1$,
 and they sum up to zero. Therefore $\mC_D \perp \mC_{D^*}$.
 
 Suppose that the diagram $D$ consists of $n$ crossings and $b$ black regions, then $\Gamma_D$ is a graph with $n$ edges and $b$ vertices.
 Hence $\mC_D $ has length $n$ and dimension $n-b+1$ by Remark \ref{r-param-cyclecode}. 
 The total number of regions is $n+2$ by Lemma \ref{lem:knot_diag}. So the number of white regions is $n+2-b$.
 Hence $\Gamma_{D^*}$ is a graph with $n$ edges and $n+2-b$ vertices. 
 Therefore $\mC_{D^*}$ has length $n$ and dimension $n-(n+2-b)+1=b-1$ by Remark \ref{r-param-cyclecode}. 
 Hence the codes $\mC_D$ and $\mC_{D^*}$ have complementary dimensions. Therefore $\mC_D^\perp =\mC_{D^*}$, concluding the proof.
 \end{proof}

Theorem \ref{p-dual-black-white} does not generalize to the case of arbitrary $t$, since in general $\mC_{D,t}$ and $\mC_{D^*,t}$ do not have complementary dimensions, and they are not perpendicular to each other. The fact that the proof of Theorem \ref{p-dual-black-white}  works for $t=-1$ boils down to two facts:
\begin{enumerate}
    \item The sum of rows of the parity check matrix of the black graph is the all-zero vector. The same holds for the white graph. So the corresponding codes have complementary dimensions,
    \item The inner product of a row of the parity check matrix of the black graph with a row of the parity check matrix of the white graph is zero.
\end{enumerate}


\begin{proposition}\label{p-AB-hull}
 The code $\mC_{D,t} \cap \mC_{D^*,t}$ is equal to the Alexander-Briggs code $\mA_{D,t}$. 
 If $t=-1$, then $\mA_{D}$ is equal to the hull of $\mC_{D}$.
 \end{proposition}

\begin{proof}
The Alexander-Briggs code $\mA_{D,t}$ is defined by the parity checks defined by both the black and white regions.
Hence $\mA_{D,t}=\mC_{D,t} \cap \mC_{D^*,t}$. If $t=-1$, then $\mC_{D^*}= \mC_{D^*,-1}=\mC_{D}^\perp $ by Theorem \ref{p-dual-black-white}.
Hence $\mA_{D}$ is the hull of $\mC_{D}$.
\end{proof}

Combining Theorem \ref{p-dual-black-white} and Proposition \ref{p-AB-hull}, we get the next result related to LCD codes.


\begin{corollary}
\label{c-dual}
Let $D$ be a reduced Tait diagram of a knot.
If the characteristic is $2$ or the knot is alternating, then the Alexander-Briggs code $\mA_{D}$ (when $t=-1$) is LCD.
\end{corollary} 

We add the next remark about LCD codes and whether graph codes of Tait diagrams of knots can lead to ``good" LCD codes.

 
\begin{remark}
If $\mC$ and $\mD$ are $(\pm1)$-permutation equivalent codes, then their hulls (see Definition \ref{def:lcd}) are also $(\pm1)$-permutation equivalent. This is not true for monomial equivalent codes. If $q>3$, then every linear code is monomial equivalent to an LCD code \cite{carlet2018}. 
So the question about the existence of LCD codes is the same as the question about the existence of linear codes in the case of $q>3$. 
However, the cases $q=2$ and $q=3$ need separate attention, see \cite{dougherty2017}. 
It was shown that that LCD codes are asymptotically good \cite{massey1992}, in fact they attain the Gilbert-Varshamov bound \cite{sendrier2004}. However, the graph codes of Tait diagrams of knots cannot give ``good" LCD codes since cycle codes are not asymptotically good as mentioned before. 
\end{remark}

We conclude the section with an open problem. 

\begin{oproblem}
Do Alexander-Briggs codes of knots give asymptotically good
codes?
\end{oproblem}


\section{Connected Sum of Knot Diagrams}
\label{sec:connected}
\label{sec:5}

Using the \textit{connected sum} operation, two knot diagrams form a new knot diagram. This will give us a way 
of constructing Fox knot codes
with arbitrary dimension. 
This section is devoted to studying how the codes of two knot diagrams are related to the code of their connected sum.

\begin{definition}
\label{def:knotsum}
The \textbf{connected sum}
of oriented knots $K_1$ and $K_2$
is the oriented knot $K_1 \# K_2$ 
whose diagram is obtained by taking an arc from a strand of each knot and connecting the open ends with two new arcs, in such a way that the orientation is preserved in the sum; see Figure \ref{fig:knotsum}. In this way we get a diagram $D_1 \# D_2$ of $K_1 \# K_2$, where $D_1$ and $D_2$ are the diagrams of $K_1$ and $K_2$, respectively. \end{definition}

It can be shown that the 
connected sum of knots indeed
does not depend on the choice of the strands.

% Figure environment removed





The following concepts naturally arise from Definition \ref{def:knotsum}.

\begin{definition}
\label{def:prime}
A knot that cannot be written as the sum of two non-trivial knots is called a \textbf{prime} knot, otherwise it is called a \textbf{composite} knot.
\end{definition}

All composite knots have a unique decomposition into prime knots \cite{schubert2013}. Determining whether a knot is composite or not is generally a hard task.

We establish the notation for the rest of this section. 

\begin{notation}
\label{not:sum}
In the sequel we let
$D_1$ and $D_2$ be knot diagrams of (oriented, polygonal) knots $K_1$ and~$K_2$ with strands $x_1, ..., x_n$ and $y_1, ..., y_m$, respectively. 
We let $\mF_1$ and $\mF_2$ be their respective codes in~$\mathbb{F}_{q}^n$ and~$\mathbb{F}_{q}^m$, as in Definition~\ref{def:knotcode},
where $p$ is prime and $a$ is a positive integer. Moreover, we let 
$$\mF_1' = \{ c \in \mF_1 \mid c_n = 0 \}, \ \ \mF_2' = \{ d \in \mF_2 \mid d_m = 0 \}.$$
\end{notation}

The following 
result provides an explicit description of
the connected sum of knot diagrams. 

\begin{lemma}\label{lem:colorknotsum}
The Fox code of the sum $K_1\#K_2$ taken by connecting the knots diagrams~$D_1$ and~$D_2$, respectively over strands $x_n$ and $y_m$ is 
$$\mF_1\# \mF_2 = \{(c,d) \mid c \in \mF_1, d \in \mF_2, c_n = d_m\}.$$
\end{lemma}

\begin{proof}
A Fox coloring for $D_{K_1\#K_2}$ consists of a Fox coloring of $D_1$ and a Fox coloring of $D_2$ where the colors of the strands $x_n$ and $y_m$ that have been connected have the same color.
\end{proof}

Lemma \ref{lem:colorknotsum} implies that if $\mF_1$ and $\mF_2$ have parity check matrices $H_{\mF_1}$ and $H_{\mF_2}$, then $\mF_1\# \mF_2$ has parity check matrix 


\begin{equation}
\label{eq:paritycheck}
  H_{\mF_1 \# \mF_2} = \left(
\begin{array}{cccc|cccc}
& H_{\mF_1} & & & & 0_{(n - \dim(\mF_1)) \times m} &  & \\
& 0_{(m - \dim(\mF_2)) \times n} & & & & H_{\mF_2} & & \\
0 & \dots & 0 & 1 & 0 & \dots & 0 & -1
\end{array} \right).  
\end{equation}
The last line of the matrix consists only of zeroes, except for a 1 on the $n$-th column and -1 on the $(n+m)$-th column.

The following proposition shows that the connected sum gives us another method, besides pretzel knots (see Theorem \ref{thm:pretzeldim}), to construct codes with any dimension.

\begin{proposition}
\label{prop:dimsum}
We have
$$\dim(\mF_1 \# \mF_2) = \dim(\mF_1) + \dim(\mF_2) -1.$$
\end{proposition}
\begin{proof}

The parity check matrices $H_{\mF_1}$ and $H_{\mF_2}$ of $\mF_1$ and $\mF_2$ are of size $(n - \dim(\mF_1)) \times n$ and $(m - \dim(\mF_2)) \times m$, respectively. Using the above construction from matrix \eqref{eq:paritycheck} we then get a parity check matrix $H$ for $\mF_1\# \mF_2$ of size~$(n+m - (\dim(\mF_1) + \dim(\mF_2) - 1)) \times (n + m)$ such that the first $n + m - (\dim(\mF_1) + \dim(\mF_2))$ rows are linearly independent. 

Towards a contradiction, assume that the last row can be written as a linear combination of the other rows of $H$. That would mean that there exists a linear combination of the rows of $H_{\mF_1}$ equal to $(0,...,0,1)$, which means that the strand~$x_m$ should always be colored with 0. The possible trivial colorings contradict this, as these include vectors with the same nonzero element on each position. So we find that the last row of $H_{\mF_1\# \mF_2}$ is independent from the other rows. Therefore, the rank of the matrix is $n + m - (\dim(\mF_1) + \dim(\mF_2) - 1)$. This proves the 
desired result.
\end{proof}

The diagram of the $m$-fold sum construction of the trefoil knot gives a code over $\F_3$ of length $3m$ and dimension $m+1$. 
Hence its rate is $R=(m+1)/3m \approx 1/3$. 

As one expects, the Alexander polynomials of two knots and their knot sum are also related; see~\cite[Theorem~6.3.5]{murasugi1996knot}. 


\begin{proposition}
\label{prop:detsum}
We have 
$$
\Delta_{K_1\#K_2}(t) = \Delta_{K_1}(t)\Delta_{K_2}(t)
.$$
\end{proposition}



Next, we give an example of a connected sum of two knot diagrams and compute the determinant using Proposition \ref{prop:detsum}.




\begin{example}
\label{ex:detsum}
In Figure \ref{fig:sum3141}, the diagrams of the trefoil knot, figure-eight knot and their connected sum are depicted. 

% Figure environment removed

\noindent Using the matrix in \eqref{eq:paritycheck}, their coloring matrices are derived as follows:
$$
\begin{pmatrix}
 1 & 1 & -2 \\ 
 -2 & 1 & 1 \\
  1 & -2 & 1 \\
\end{pmatrix} \textnormal{, }
\begin{pmatrix}
  1 & 1 & -2 & 0 \\
  0 & 1 & 1 & -2 \\
  -2 & 0 & 1 & 1 \\
  1 & -2 & 0 & 1 \\
  \end{pmatrix},
  \textnormal{ and }
$$
\begin{equation}\label{eq: first matrix}
  \begin{pmatrix}
  1 & 1 & -2 & 0 & 0 & 0 & 0 \\
  -2 & 1 & 1 & 0 & 0 & 0 & 0 \\
  0 & 0 & 1 & 1 & -2 & 0 & 0 \\
  0 & 0 & 0 & 1 & 1 & -2 & 0 \\
  0 & 0 & -2 & 0 & 1 & 1 & 0 \\
  0 & 0 & 0 & -2 & 0 & 1 & 1 \\
  1 & -2 & 0 & 0 & 0 & 0 & 1
\end{pmatrix},  
\end{equation}
which are the parity check matrices of $\mF_1$, $\mF_2$ and $\mF_1\# \mF_2$, respectively. These knots have determinants 3, 5 and 15 by Proposition \ref{prop:detsum}, respectively. 
\end{example}

Lastly, we prove that the minimum distance of a code of the diagram of a connected sum is determined by the weight distributions of the codes of the constituent knot diagrams. 


\begin{theorem}
\label{prop:sumknotmindist}
Let $\mF_1$, $\mF_1'$, $\mF_2$, and $\mF_2'$
and $\mF_1 \# \mF_2$ be as in Notation \ref{not:sum} and Lemma \ref{lem:colorknotsum} and 
let $d(\mF_1)$, $d(\mF_2)$ and $d(\mF_1 \# \mF_2)$ their respective minimum distances.  
The minimum distance of $\mF_1 \# \mF_2$ is equal to 
\begin{equation}
    \min \{ \ d(\mF_1'),\ d(\mF_2'),\ v+w \ \mid \  v \in  \wt(\mF_1 \setminus \mF_1'),\ w \in \wt(\mF_2 \setminus \mF_2') \ \}.
\end{equation}
\end{theorem}

\begin{proof}
As the codewords of $\mF_1$ and the codewords of $\mF_2$ only affect the weight of the codewords of $\mF_1 \# \mF_2$ at the first $n$ positions and the last $m$ positions, respectively, we look how minimum weight codewords of $\mF_1 \# \mF_2$ can be constructed by concatenating codewords of $\mF_1$ and $\mF_2$. 
Now $x \in \mF_1 \# \mF_2$ if and only if $x=(c,d) $ with $c \in \mF_1$ and $d \in \mF_2$  and $c_n = d_m$. Then $\wt(x)=\wt(c)+ \wt(d)$. We investigate two cases to finish the proof.
\begin{enumerate}
    \item Let $c_n = d_m = 0$, that is $c \in \mF_1'$ and $d \in \mF_2'$. 
In this case $\min\{ d(\mF_1'), d(\mF_2') \}$ is the smallest nonzero weight and is obtained by means of $(c, 0^m)$ or $(0^n,d)$ with the all-zeros codeword $0^n \in \mF_1$ and the all-zeros codeword $0^m \in \mF_2$.
    \item Let $c_n = d_m \neq 0$, that is $c \in \mF_1 \setminus \mF_1'$ and $d \in \mF_2 \setminus  \mF_2'$.
If $v =\wt (c)$, then $v \in \wt(\mF_1 \setminus \mF_1')$. Similarly, if $w =\wt (d)$, then $w \in \wt(\mF_2 \setminus \mF_2')$.
Conversely, if $v \in \wt(\mF_1 \setminus \mF_1') $, then there exists a  $c \in \mF_1 \setminus \mF_1'$ with $v =\wt (c)$. 
Similarly, if $w \in \wt(\mF_2 \setminus \mF_2')$, then there exists a  $d \in \mF_2 \setminus \mF_2'$ with $w =\wt (d)$. 
Hence, $\min \{ v+w \mid v \in  \wt(\mF_1 \setminus \mF_1'),\ w \in \wt(\mF_2 \setminus \mF_2')\}$ 
is the smallest weight of a nonzero codeword of $\mF_1 \# \mF_2$ obtained in this case. \qedhere
\end{enumerate}
\end{proof}

We give the following two remarks related to Theorem \ref{prop:sumknotmindist}.

\begin{remark}
If $\mF_1' =\{0\}$ and $\mF_2' =\{0\}$, then $K$ and $L$ have only trivial colorings, so $ \mF_1 \# \mF_2$ has only trivial colorings and $d(\mF_1 \# \mF_2)= n+ m$.
This is in agreement with the statement in Theorem \ref{prop:sumknotmindist}, since we defined the minimum distance of the zero code to be $\infty$ in Definition~\ref{def:weight_distance}.

If $\mF_1' =\{0\}$ and $\mF_2' \not=\{0\}$, then $d(\mF_1 \# \mF_2)=\min \{ \   d(\mF_2'),\ n+w \ \mid \ w \in \wt(\mF_2 \setminus \mF_2')\  \}$.\\
And a similar formula holds in case $\mF_1' \not=\{0\}$ and $\mF_2' =\{0\}$.
\end{remark}


\begin{remark}
Theorem \ref{prop:sumknotmindist} also follows from~\cite[Proposition 6.1.1]{nijsten2022knots}, 
where a formula for the weight enumerator of $\mF_1 \# \mF_2$ is given in terms of the weight enumerators of $\mF_1 $, $\mF_2$, $\mF_1'$, and~$\mF_2'$.
The formula is: 
$$
W_{\mF_1 \# \mF_2}(t) = W_{\mF_1'}(t)\cdot W_{\mF_2'}(t) + \frac{1}{q - 1} (W_{\mF_1}(t) - W_{\mF_1'}(t))(W_{\mF_2}(t) - W_{\mF_2'}(t)).
$$
This is in agreement with Theorem \ref{prop:sumknotmindist}, since $a_v(\mF_1') < a_v(\mF_1)$ if and only if $ v\in \wt(\mF_1 \setminus \mF_1')$,
and $a_w(\mF_2') < a_w(\mF_2)$ if and only if $w \in \wt(\mF_2 \setminus \mF_2')$.
\end{remark}

The next example shows applications of some of our results in this section.
\begin{example}
Let $q=3$ and $\mF=\mF_D$ where $D$ is the knot diagram of the trefoil knot depicted in Figure \ref{fig:trefoil}. By Example \ref{ex:detsum}, its parity check matrix is all-ones matrix and thus has rank 1. Thus, $\mF$ is a $[3,2,2]_3$ MDS code such that
\begin{align*}
    \mF &= \{(0,0,0),(0,1,2),(0,2,1),(1,0,2),(1,2,0),(1,1,1),(2,0,1),(2,1,0),(2,2,2)\}, \\
    \mF' &= \{(0,0,0),(1,2,0),(2,1,0)\}.
\end{align*}
We have $d(\mF')=2$ and $\wt(\mF \setminus \mF') = \{2,3\}$. By Proposition \ref{prop:dimsum} and Theorem \ref{prop:sumknotmindist} one can create a code with parameters
$$[n+m, \dim(\mF_2)+1, \min\{2,d(\mF_2')\}]_3$$
where $\mF_2$ is a code a knot diagram of some knot with $m$ strands. For example, if $\mF_2 = \mF_1$, then one gets a $[6,3,2]_3$ code which is a non-MDS. \end{example}


For the remaining part of this section, we focus on cycle codes.


\begin{definition}\label{d-sum-graph}
Let $\Gamma$ and $\Sigma$ be two (directed) graphs.
The \textbf{disjoint sum} of $\Gamma$ and $\Sigma$ is denoted by $\Gamma \sqcup\Sigma$
and has as nodes the disjoint union of the nodes of $\Gamma$ and $\Sigma$, 
and has as edges the disjoint union of the edges of $\Gamma$ and $\Sigma$.

Let $p$ be a node of $\Gamma$, 
and let $q$ be a node of $\Sigma$.
Then $(\Gamma \sqcup\Sigma )/ (p,q)$ is the graph
$\Gamma \sqcup\Sigma$ where the the node $p$ is identified with $q$.
\end{definition}

\begin{proposition}\label{p-sum-code-graph-pol}
Let $\Gamma_1$ and $\Gamma_2$ be two directed graphs. Let $p_1$ be a node of $\Gamma_1 $ and let $p_2$ be a node of $\Gamma_2 $.
Let $\Gamma =(\Gamma_1 \sqcup \Gamma_2 )/ (p_1,p_2)$.
Then 
$$
C_{\Gamma}=C_{\Gamma_1} \oplus C_{\Gamma_2}
$$  
\end{proposition}
\begin{proof}
Let $A_1$, $A_2$ and $A$ be the matrices of the directed graphs $\Gamma_1$,  $\Gamma_2$ and $\Gamma$, respectively as given in 
Definition \ref{def:graphcode} for $t=-1$. Then $A_1$, $A_2$ and $A$ are parity check matrices of the cycle codes 
 $C_{\Gamma_1}$, $ C_{\Gamma_2}$ and $C_{\Gamma}$, respectively by definition.
Let $A_1'$ be the matrix obtained from $A_1$ by deleting the row corresponding to $p_1$.
Let $A_2'$ be the matrix obtained from $A_2$ by deleting the row corresponding to $p_2$.
Let $A'$ be the matrix obtained from $A$ by deleting the row corresponding to $p_1=p_2$.
Then $A_1'$, $A_2'$ and $A'$ are also parity check matrices of the cycle codes 
$C_{\Gamma_1}$, $ C_{\Gamma_2}$ and $C_{\Gamma}$, respectively by Remark \ref{r-A(t)}, since $t=-1$. 
Now 
 $$
 A'=
 \left(
\begin{array}{c c}
 A_1' & 0 \\
 0    & A_2' \\
 \end{array}
 \right)
 $$
This proves the proposition.
\end{proof}

The graph $\Gamma =(\Gamma_1 \sqcup \Gamma_2 )/ (p_1,p_2)$ varies with the choices of the nodes $p_1$ and $p_2$, that is they are in general not isomorphic. But its graph code is independent of the choices of $p_1$ and $p_2$. 


\begin{proposition}\label{p-prod-sum-knots}
Let the black regions of the constituent knots $K_1$ and $K_2$, and their Tait diagrams $D_1$ and $D_2$, respectively 
be such that their unbounded regions are white. 
Let $D_1\# D_2$ be the Tait diagram of $K_1\# K_2$ where the regions $B_1$ and $B_2$ of $D_1$ and $D_2$, respectively, are glued together.
Let $\Gamma_1$ and $\Gamma_2$ be the graphs of the black regions of $D_1$ and $D_2$, respectively.
Then $ (\Gamma_1 \sqcup \Gamma_2 )/ (B_1,B_2)$ is the graph of the black regions of the diagram of the connected sum $K_1\# K_2$.
\end{proposition}
\begin{proof}
This is a direct consequence of the definitions.  
\end{proof}

As a result of Propositions \ref{p-sum-code-graph-pol} and \ref{p-prod-sum-knots} we see that the cycle code of the connected sum of two knots 
does not depend on the choice of the strands and regions where the constituent knots are glued together.


\section{Dual of Fox Knot Codes}
\label{sec:dual}
\label{sec:6}


It is a standard problem in coding theory to understand how properties of a code determine or influence properties of the dual code.
In this short section, we ask ourselves 
if the dual of a Fox code of a knot diagram is also a Fox code of a knot diagram. 
We start by proving a necessary but not sufficient condition for a dual code to be a knot diagram.
\begin{proposition}
\label{prop:pdividesn}
Let $\mF$ be a Fox code of a knot diagram with $n$ strands over $\F_q.$ Then~$q$ divides~$n$ if $\mF^\perp$ is a Fox code of a knot diagram. 
\end{proposition}
\begin{proof}
By Remark \ref{rem:dim1}, the Fox code $\mF$ of a knot diagram with $n$ strands has the $n$-times repetition code $$\langle \underbrace{(1,1,...,1)}_n \rangle$$ as a subcode. If  $\mF^\perp$ is a code of some knot diagram, then it should also have the $n$-times repetition code as a subcode. 
We have that $$\underbrace{(a,a,...,a)}_{n}\underbrace{(a,a,...,a)}_{n} {^\top} = na^2$$ for all $a \in \F_q^n$. So in order for both a code and its dual to have the $n$ times repetition code as their subcode, it must be 
that $na^2 = 0$ for all $a \in \F_q^n$. Therefore, we must have that $n$ is divisible by $q$, as desired.
\end{proof}

Using results we obtained on the dimension of a Fox code of a knot diagram in Subsection~\ref{subsec:dim}, we can obtain information about the dual code as well.


\begin{proposition}
\label{prop:duallimit}
Let $\mF$ be the Fox code of a knot diagram. If $\dim(\mF) < \frac{n-1}{2}$, then~$\mF^\perp$ is not monomial equivalent to the Fox code of a knot diagram.
\end{proposition}
\begin{proof}
If $\dim(\mF) < \frac{n-1}{2}$ then $\dim(\mF^\perp) > \frac{n+1}{2}$.
The result then follows from Theorem~\ref{thm:dimension-ineq}.
\end{proof}


This result can be used on composite knot diagrams to determine whether the duals of their codes are codes of knot diagrams.
\begin{proposition}
\label{prop:4componentdual}
Let $\mF = \mF_1 \# \mF_2 \# \cdots \# \mF_i$ be the Fox code of a diagram of $i$ composed knots~$K = K_1 \# K_2 \# \cdots \# K_i$, where each $K_j$ has $n_j$ strands in their corresponding knot diagrams. If $i \geq 4$, then $\mF^\perp$ is not a Fox code of a knot diagram. 
\end{proposition}
\begin{proof}
Let $n = n_1 + n_2 + \cdots + n_i$.
Using Proposition \ref{prop:dimsum} we find that 
\begin{align*}
    \dim(\mF_1 \# \mF_2 \#\cdots \# \mF_i) &= \dim(\mF_1) + \dim(\mF_2) + \cdots + \dim(\mF_i) - i + 1 \\
    &\leq \frac{n_1 + 1}{2} + \frac{n_2 + 1}{2} + \cdots + \frac{n_i + 1}{2} - i + 1 \\
    &= \frac{n - i}{2} + 1.
\end{align*}
Therefore $\dim(\mF) < \frac{n-1}{2}$ if $i \geq 4$ and the result follows from Proposition \ref{prop:duallimit}.
\end{proof}


\subsection*{Data Availability}
There is no data associated with this article. This article is self-contained. 
\subsection*{Competing Interest}
The authors have no conflicts of interest that could potentially influence or bias this article. 

\bigskip

\bibliographystyle{abbrv}
\bibliography{ADV}

\bigskip

\appendix

\section{Commutative Algebra} \label{commut-alg} 

For the basic definitions and properties of commutative algebra such as modules and morphisms we refer to \cite{atiyah2018introduction,eisenbud1995,lang2012algebra}.
In this paper, a ring will always mean a Noetherian, commutative ring with a unit element $1$. 
So, the ideals of a ring are finitely generated. Furthermore, all modules will be assumed to be finitely generated.

\begin{remark}\label{r-row-column}
In this appendix, we adopt the usual convention in commutative algebra to consider the elements of $R^{(n)}$ as column vectors of length $n$ with entries in $R$,
contrary to the rest of this paper where we align to the convention in coding theory where the elements of $R^n$ are row vectors of length $n$ with entries in $R$.
So this difference is stressed by the notation $R^{(n)}$ for column vectors and $R^n$ for row vectors.

The set of $m \times n$ matrices with entries in the ring $R$ is denoted by $R^{m \times n}$.
The matrix~$A \in R^{m \times n}$ gives a morphism of $R$-modules $R^{(n)} \rightarrow R^{(m)}$ 
defined by $x \mapsto Ax$ for $ x \in R^{(n)}$.
The \textbf{kernel} of $A \in R^{m \times n}$ is $\Ker (A) =\{ x\in R^n \mid Ax^T =0 \}$.
\end{remark}

To define equivalence of matrices, row/column operations are used.


\begin{definition}\label{d-matrix-equiv}
The \textbf{elementary row operations} on a matrix with entries in a ring are:
\begin{enumerate}
    \item interchanging rows,
\item adding a row to another row,
\item multiplying a row with an invertible element of the ring.
\end{enumerate}
\end{definition}

One defines \textbf{elementary column operations} similarly.
If $A$  is the $m \times n$ matrix in the left upper submatrix of the $(m+1)\times (n+1)$ matrix $B$ such that the entries of the last row and column of $B$ are all zero, except a pivot $1$ at the entry corresponding to the last row and last column, then we say that $B$ is obtained from $A$ by \textbf{adding a pivot}, and $A$ from $B$ by \textbf{deleting a pivot}.

\begin{definition}\label{d-equiv-matrix}
Matrices are called \textbf{equivalent} if they can be obtained from each other by a sequence of
\begin{itemize}
    \item elementary row and column operations,
    \item adding and deleting a zero row, 
    \item adding and deleting a pivot.
\end{itemize}
\end{definition}

Definition \ref{d-equiv-matrix} is taken from \cite[Chapter VII \S 4]{crowell2012introduction} and is more general than the one given in  \cite[Chapter II]{newman1972}, where  equivalent matrices must have the same size.



\begin{proposition}\label{p-module-equiv}
Let $A$ and $B$ be  matrices with entries in $R$.
If $A$ and $B$ are equivalent, then $\Ker (A) \cong \Ker (B)$ as $R$-modules.
\end{proposition}

Given a matrix, one defines ideals generated by the determinant of all submatrices of some fixed size.

\begin{definition}\label{d-elem-id}
Let $A \in R^{m \times n}$ and $k \in \Z_{\geq 0}$. 
Let $E_k(A)$ denote the $k$-th \textbf{elementary} (or \textbf{Fitting}) \textbf{ideal} of $A$, that is the ideal generated by determinants of all $(n-k)\times (n-k)$ submatrices of $A$ if $0<n-k\leq m$, $E_k(A)=0$ if $n-k>m$, and $E_k(A)=R$ if $n-k\leq 0$. 
\end{definition}

Elementary ideals of equivalent matrices are the same. Moreover this fact can be slightly refined,
as the following two propositions formalize.

\begin{proposition}\label{p-elem-id-1}
Let $A \in R^{m \times n}$ and $k \in \Z_{\geq 0}$. The elementary ideals $E_k(A)$ form an increasing sequence of ideals with respect to inclusion. If $A$ and $B$ are equivalent matrices, then $E_k(A)=E_k(B)$.
\end{proposition}
\begin{proof}
See \cite[Chapter VII (4.1)]{crowell2012introduction}.
\end{proof}

\begin{proposition}\label{p-free-matrix}
Let $A \in R^{m \times n}$ and $B \in R^{m \times (n+l)}$ be matrices such that $B$ is equivalent to $\smash{(A \mid O_{m\times l})}$, where 
 $0_{m\times l} \in R^{m \times l}$ is the matrix with all zero entries.
Then $E_{k}(B)=E_{k-l}(A)$ for all $k$.
\end{proposition}
\begin{proof}
The result follows directly from the definitions if $B=(A \mid O_{m\times l})$, and from Proposition \ref{p-elem-id-1} otherwise.
\end{proof}

Adding zero rows to a matrix does not change its elementary ideals. Thus, we have the following result that is independent of the number of columns of the matrix.

\begin{proposition}\label{p-elem}
Let $R$ be a field and let $A \in R^{m \times n}$. 
If $A$ has rank $r$, then  $E_k(A)=R$ if~$k\geq n-r$, and $E_k(A)=0$ otherwise.
\end{proposition}

\begin{proof}
If $A$ has rank $r$, then one can transform $A$ by elementary row and column operations into a matrix $B$ that has the $r \times r$ identity matrix $I_r$ as a submatrix and entries equal to zero outside that identity matrix. 
Deleting the $r$ rows and columns corresponding to the pivots of the  matrix gives the $(m-r) \times (n-r)$ matrix with zeros as entries.
The elementary ideals remain the same under these transformations by Proposition \ref{p-elem-id-1}.
Hence $E_k(A)=R$ if $k\geq n-r$ and $E_k(A)=0$ otherwise. 
\end{proof}

\begin{proposition}\label{p-elem-id-2}
Let $\varphi : R \rightarrow S$ be a morphism of rings and let $A$ be a matrix with entries $a_{ij}$ in $R$.
Denote by $\varphi (A)$ the matrix with entries $\varphi (a_{ij})$ in $S$. 
If $\varphi $ is surjective, then~$E_k (\varphi (A)) = \varphi (E_k(A))$.
\end{proposition}

\begin{proof}
See \cite[Chapter VII (4.3)]{crowell2012introduction}
\end{proof}


For the rest of the appendix, we focus on the principal ideals of a principal ideal domain~$R$ and its relations with the elementary ideals of a matrix whose entries are coming from $R$.

\begin{proposition}[{\bf Smith Normal Form}]\label{p-struct-pid}
Let $R$ be a principal ideal domain and let~$A$ be a matrix with entries in~$R$.
Then there is an increasing sequence of principal ideals~$(d_1)\subseteq (d_2)\subseteq \cdots \subseteq (d_l)\not= R $ 
such that $A$ is equivalent to a diagonal square matrix with~$(d_1, d_2, \ldots ,d_l) $ on the diagonal.
\end{proposition}
\begin{proof}
See \cite[Theorem II.9]{newman1972}.
\end{proof}

The principal ideals  $(d_i)$ in the previous proposition are called  \textbf{invariant factors} of the matrix $A$.
A generator of $(d_i)$ is unique up to an unit and the invariant factors are unique.
Note that the
principal ideals might be zero. 
Let $r$ be the smallest non-negative integer such that $d_{r}=0$ and $d_{r+1} \not= 0$,
where
$d_0=0$ and $d_{l+1}=1$. 
Then the smallest non-negative integer $r$ such that $d_{r}=0$ and $d_{r+1} \not= 0$
is called the \textbf{rank} of the matrix $A$.


\begin{corollary}\label{c-struct-pid}
Let $R$ be a principal ideal domain. Let $M$ be a matrix with entries in $R$
and invariant factors $(d_1)\subseteq (d_2)\subseteq \cdots \subseteq (d_l)$.
Then $E_k(M)$ is generated by $$\Delta_k:=\prod_{j=k+1}^l d_j.$$
Conversely, let $E_k(M)=(\Delta_k)$. Then $\Delta_{k-1}$ is divisible by  $\Delta_k$ and $d_k =\Delta_{k-1}/\Delta_k$ is the $k$-th invariant factor of $M$.
\end{corollary}
\begin{proof}
See \cite[Chapter II \S 15 and \S 16]{newman1972}.
\end{proof}


We conclude this appendix with the following proposition which in this paper is used in the principal ideal domains $R=\Z$ and $R=\F_p[T]$, and in their localizations; see Propositions~\ref{p-number-color-Z} and~\ref{p-number-color-prime}.


\begin{proposition}\label{p-struct-ker-im}
Let $R$ be a principal ideal domain. 
Let $A$ be a matrix with entries in~$R$ and invariant factors $(d_1)\subseteq (d_2)\subseteq \cdots \subseteq (d_l) $. 
Let $d$ be a nonzero element of $R$ and let~$a_i=\gcd (d,d_i)$ and $a_ib_i=d$.
Let $\overline{R}=R/(d)$ and $\overline{x}=x+(d) \in \overline{R}$ for $x\in R$. 
Then
$$
\Ker (\overline{A})  \cong \overline{R}/(\overline{a_1}) \oplus \overline{R}/(\overline{a_2}) \oplus  \cdots \oplus  \overline{R}/(\overline{a_l}).
$$
\end{proposition}
\begin{proof}
The matrix $A$ is equivalent to the diagonal matrix $B$ that has $(d_1,d_2, \ldots ,d_l)$ on its diagonal by Proposition \ref{p-struct-pid}.
Hence $\Ker (A ) \cong \Ker (B)$  by Proposition~\ref{p-module-equiv}.
To prove the result it is enough to show it separately for each $d_i$ on the diagonal. Notice that $\overline{d_i}=\overline{a_i}$, since $a_i=\gcd (d,d_i)$.
Consider the sequence of $\overline{R}$-modules:
$$
0\rightarrow (\overline{b_i})\overline{R} \rightarrow \overline{R} \rightarrow \overline{R} \rightarrow (\overline{b_i})\overline{R} \rightarrow 0
$$
where the map $(\overline{b_i})\overline{R} \rightarrow \overline{R}$ is an inclusion, and $\overline{R} \rightarrow \overline{R}$ is given by multiplication by $\overline{a_i}$,
and the surjective map  $\overline{R} \rightarrow (\overline{b_i})\overline{R} $ is given by multiplication by $\overline{b_i}$. 
This sequence is a chain complex, that is, the composition of two consecutive maps is zero, since~$a_ib_i=d \equiv 0 \pmod d$.
But it is in fact an exact sequence: Consider the kernel of the multiplication by~$\overline{a_i}$ and suppose that $\overline{x}\overline{a_i}=0$. Then $xa_i \equiv 0 \pmod d$, and thus $xa_i = yd$ for some $y\in R$. So~$xa_i = ya_ib_i$, and consequently $x = yb_i$ since $R$ is an integral domain. Therefore $\overline{x} \in (\overline{b_i})\overline{R}$. 

On the right hand, we have the sequence $ \overline{R} \rightarrow \overline{R} \rightarrow (\overline{b_i})\overline{R}$, 
which is exact at the middle by a similar reasoning as before.
The cokernel of the multiplication by $\overline{a_i}$ is by definition equal to $\overline{R}/(\overline{a_i})$.
Hence $\overline{R}/(\overline{a_i})$ isomorphic to $(\overline{b_i})\overline{R}$. Therefore the statement on $\Ker (\overline{A})$ follows, as desired.
\end{proof}

\end{document}


% \bibliographystyleR{IEEEtran}
% \bibliographyR{revision/refe1}
% \newpage
% \linenumbers
% \pagenumbering{gobble}
% \setcounter{figure}{0}
% \setcounter{table}{0}

% \runninglinenumbers
%%%%%%%%%%%%%%%%%%%%%%%%%%%%%%%%%%%%%%%%%%%%%%%%%%%%%%%%%%%%%%%%%%%%
%%%%%%%%%%%%%%%%%%%%%%%%%%%%%%%%%%%%%%%%%%%%%%%%%%%%%%%%%%%%%%%%%%%%
\title{Attentive Multimodal Fusion for \\Optical and Scene Flow}

\author{Youjie Zhou$^1$, Guofeng Mei$^2$, Yiming Wang$^2$, Fabio Poiesi$^2$, Yi Wan$^1$
% \thanks{Manuscript received: March, 5, 2023; Revised June, 14, 2023; Accepted July, 12, 2023. }%Use only for final RAL version 
% \thanks{*This paper was recommended for publication by Editor Pascal Vasseur upon evaluation of the Associate Editor and Reviewers' comments. }
\thanks{This work was supported by the China government project (2019JZZY010112), (2020JMRH0202), (YC-KYXM-07-2021) and by the PNRR project FAIR - Future AI Research (PE00000013), under the NRRP MUR program funded by the NextGenerationEU.}
%
\thanks{$^{1}$Youjie Zhou and Yi Wan are with the School of Mechanical Engineering, Shandong University, China,
and the Key Laboratory of High Efficiency and Clean Mechanical Manufacture of Ministry of Education, Shandong University, China. {\tt 202020511@mail.sdu.edu.cn,<wanyi>@sdu.edu.cn}. Yi Wan is the corresponding author.}%
%
\thanks{$^{2}$Guofeng Mei, Yiming Wang, and Fabio Poiesi are with Fondazione Bruno Kessler, Italy {\tt <gmei,ywang,poiesi>@fbk.eu}.}
% \thanks{Digital Object Identifier (DOI): see top of this page.}
}

\markboth{IEEE Robotics and Automation Letters. Preprint Version. July, 2023}%
{Zhou \MakeLowercase{\textit{et al.}}: Attentive Multimodal Fusion for Optical and Scene Flow}

\maketitle



%%%%%%%%%%%%%%%%%%%%%%%%%%%%%%%%%%%%%%%%%%%%%%%%%%%%%%%%%%%%%%%
%%%%%%%%%%%%%%%%%%%%%%%%%%%%%%%%%%%%%%%%%%%%%%%%%%%%%%%%%%%%%%%
%%%%%%%%%%%%%%%%%%%%%%%%%%%%%%%%%%%%%%%%%%%%%%%%%%%%%%%%%%%%%%%
\begin{abstract}
This paper presents an investigation into the estimation of optical and scene flow using RGBD information in scenarios where the RGB modality is affected by noise or captured in dark environments. Existing methods typically rely solely on RGB images or fuse the modalities at later stages, which can result in lower accuracy when the RGB information is unreliable. To address this issue, we propose a novel deep neural network approach named {\mname}, which enables early-stage information fusion between sensor modalities (RGB and depth). Our approach incorporates self- and cross-attention layers at different network levels to construct informative features that leverage the strengths of both modalities. Through comparative experiments, we demonstrate that our approach outperforms recent methods in terms of performance on the synthetic dataset Flyingthings3D, as well as the generalization on the real-world dataset KITTI. We illustrate that our approach exhibits improved robustness in the presence of noise and low-lighting conditions that affect the RGB images. We release the code, models and dataset at \url{https://github.com/jiesico/FusionRAFT}.
\end{abstract}


\begin{IEEEkeywords}
Optical and scene flow, multimodal fusion, self- and cross-attention.
\end{IEEEkeywords}


\section{Introduction}
Current quantum hardware is unable to carry out universal quantum computations due to the buildup of errors that occur during the computation. 
The magnitude of the individual error is currently above the value that the Threshold Theorem requires in order to kick-start quantum error correction and fault-tolerant quantum computation~\cite[Section 10.6]{nielsen_chuang_2010}. 
Although the experimentally achieved fidelity rates are promising and the error bounds are inching closer to the required threshold, we will have to work for the foreseeable future with quantum hardware with errors that build-up during the computation.  This implies that we can only do a limited number of steps before the output of the computation has become completely uncorrelated with the intended one.

For fault-tolerant quantum computing, we repeat four steps: 
1) We apply a number of single and two-qubit quantum gates, in parallel whenever possible; 
2) We perform a syndrome measurement on a subset of the qubits; 
3) We perform fast classical computations to determine which errors have occurred and how to correct them; 
and, 4) We apply correction terms based on the classical computations.
We then repeat these four steps with a next sequence of gates. 
These four steps are essential to fault-tolerant quantum computing. 


The starting point of this work is to use the four steps outlined above, not to carry out error correction and fault-tolerant computation, but to enhance short, constant-depth, {\em uncorrected} quantum circuits that perform single qubit gates and {\em nearest-neighbor} two qubit gates. 
Since in the long run we will have to implement error-correction and fault-tolerant computation anyhow, and this is done by such a four-step process, why not make other use of this architecture? Moreover, on some of the quantum hardware platforms, these operations are already in place.
Embracing this idea we naturally arrive at the question: what is the computational power of \textit{low-depth} quantum-classical circuits organized as in the four steps outlined above? 
We thus investigate circuits that execute a small, ideally constant, number of stages, where at each stage we may apply, in parallel, single qubit gates and {\em nearest-neighbor} two qubit gates, followed by measurements, followed by low-depth classical computations of which the outcome can control quantum gates in later stages. 
It is not clear, at first, whether such circuits, especially with constant depth, can do anything remotely useful. 
But we will see that this is indeed the case: many quantum computations can be done by such circuits in constant depth. 
By parallelizing quantum computations in this way, we improve the overall computational capabilities of these circuits, as we do not incur errors on qubits that are idle, simply because qubits are not idle for a very long time. 
Furthermore, reducing the depth of quantum circuits, at the cost of increasing width, allows the circuit to be run faster even if errors occur.

The first usage of such a four-step layout, not to do error correction, but to perform computations, can be found in the paradigm of measurement-based quantum computing~\cite{gottesman1999demonstrating,raussendorf2001one,jozsa2006introduction,clark2007generalised}: 
A universal form of quantum computing where a quantum state is prepared and operations are performed by measuring qubits in different bases, depending on previous measurements and intermediate measurements.

\citeauthor{PhamSvore2013} were the first to formalize the four-step protocol for performing computations~\cite{PhamSvore2013}. They included specific hardware topologies by considering two-dimensional graphs for imposing constraints on qubit interactions. In their model, they develop circuits for particularly useful multi-qubit gates, including specifying costs in the width, number of qubits, depth, number of concurrent time steps, size, and total number of non-Identity operations.
As a result, they find an algorithm that factors integers in polylogarithmic depth.
\citeauthor{Browne:2011} showed that the main tool in the work by \citeauthor{PhamSvore2013}, the fan-out gate, can also be replaced by additional log-depth classical computations in the measurement-based quantum computing setting~\cite{Browne:2011}.

More recently, \citeauthor{Cirac:2021} introduced a scheme to implement unitary operations involving quantum circuits combined with Local Operations and Classical Communication ($\mathsf{LOCC}$) channels: $\mathsf{LOCC}$-assisted quantum circuits~\cite{Cirac:2021}. Similarly to the four-step scheme we just described, they allow for a short depth circuit to be run on the qubits, followed by one round of $\mathsf{LOCC}$, in which ancilla qubits are measured and local unitaries are applied based on the measurement outcomes. They show that in this model any 1D transitionally invariant matrix-product state (MPS) with fixed bond dimension is in the same phase of matter as the trivial state. Similar ideas can be found in~\cite{TVV_NonAbelianTopologicalOrder_2022, tantivasadakarn2021long}.

In this work, we introduce a new model, called \textit{Local Alternating Quantum-Classical Computations} ($\LAQCC$). In this model we alternate between running quantum circuits (constrained by locality), ending in the measurement of a subset of qubits, and fast classical computations based on the measurement results. The outcome of the classical computations are then used to control future quantum circuits. We allow for flexibility in this model, by giving different constraints to the power of both the quantum circuits and the classical circuits as well as the number of alternations between them. 
Most attention will be given to $\LAQCC$ containing quantum circuits of constant depth, classical circuits of logarithmic depth and at most a constant number of alternations between them. 
Any circuit constructed in this model is considered to be of constant depth. 
We restrict ourselves to logarithmic depth classical computations, as this is the first natural and non-trivial extension beyond constant-depth classical computations. 
Constant-depth classical computations do however also have an equivalent constant-depth quantum implementation.

The definition of $\LAQCC$ sharpens the original definition of \citeauthor{PhamSvore2013} by adding constraints to the intermediate classical computations. This allows us to bound the power of $\LAQCC$ from above. 

The main result of \citeauthor{Cirac:2021}, that 1D translational invariant MPS with fixed bond dimension can be prepared by $\mathsf{LOCC}$-assisted circuits, relies on local symmetries of the MPS. These symmetries allow them to prepare local states (on a constant number of qubits) and glue them together by doing one round of the appropriate entangling measurement and corrections, after which they run a round of local unitaries to get the desired result. This general scheme for preparing states that exhibit an MPS description with the appropriate local symmetries requires only geometrically local unitaries and one round of measurement and corrections an therefore is accessible in $\LAQCC$. Studying different local symmetries, known as Symmetry Protected Topological (SPT) phases of matter, to find measurement-based constant depth circuits for states is a broad ongoing field of research~\cite{TVV_NonAbelianTopologicalOrder_2022, tantivasadakarn2021long, smith2023deterministic}. 
All these schemes have a $\LAQCC$ implementation.

%$\LAQCC$-circuits also exist for general schemes of preparing local states, based on the local tensors, and gluing them together using one round of entangled measurement and corrections, based on the local symmetry. 
%The main result of \citeauthor{Cirac:2021}, that 1D translational invariant MPS with fixed bond dimension can be prepared by $\mathsf{LOCC}$-assisted circuits, relies heavily on local symmetries of the MPS and as a result also has an equivalent $\LAQCC$ implementation. 
%The corrections applied after the measurement round are local unitaries depending on the local symmetries of the MPS. 

 

%This general scheme of preparing local states, based on the local tensors, and gluing it together by doing one round of entangled measurement and corrections, based on the local symmetry, is accessible in $\LAQCC$.
Note however that \citeauthor{Cirac:2021} also suggest a circuit for the $W$-state.
This circuit uses sequentially and dependent measurement-based corrections of the ancilla qubits. 
These dependent measurements translate to sequential alternations between the quantum and classical circuits and therefore increase the total depth to linear depth, exceeding the constant-depth constraints imposed by $\LAQCC$-circuits. 

We study the power of the $\LAQCC$ model with respect to state preparation, showing that even with only constant quantum-depth and logarithmic classical depth it remains possible to prepare states with long-range entanglement.
Another surprising result is that it is unlikely that $\LAQCC$ circuits are classically simulatable. We show that any instantaneous quantum polynomial-time (IQP) circuit~\cite{Bremner2010,Shepherd2009} has an $\LAQCC$ implementation.
Classical simulation of IQP circuits implies the collapse of the polynomial hierarchy to the third level, which is not believed to be true~\cite{Bremner2017}. Therefore, we expect that $\LAQCC$ circuits are unlikely to be classically simulatable. We bound the power of $\LAQCC$ by showing that it is contained in $\QNC^1$, the class of polynomial-size, log-depth circuits.

Next, we also study the power that intermediate classical calculations can add to quantum computations, by considering a new model that alternates between polynomially many polynomial-depth quantum circuits and unbounded classical computations
We study this model by doing a complexity theoretical analysis, where we draw inspiration from the notions of complexity given by \citeauthor{RosenthalYuen:2022}, \citeauthor{MetgerYuen:2023}, and \citeauthor{Aaronson:2004}.
All three complexity notions are based on the notion of state preparation, instead of more traditional definition of complexity such as the decidability of a computational problem. 
The first two consider classes based on sequences of quantum states preparable by a polynomial-sized quantum circuit, where the circuits are uniformly generated by a computational class, for instance, the class $\mathsf{PSPACE}$, which results in the complexity class $\mathsf{StatePSPACE}$~\cite{RosenthalYuen:2022,MetgerYuen:2023}.
The third notion considers a relative complexity, where the complexity is measured between two given states, and is measured by the number of gates, from a given gate-set, required to transform one state in another state~\cite{Aaronson:2004}. 
For our definition of state preparation complexity, we drop the uniformity constraint from~\cite{RosenthalYuen:2022,MetgerYuen:2023} and define a class as $\mathsf{StateX}$, which refers to states preparable by circuits of type $\mathsf{X}$. 
As an example, if $\mathsf{X} = \QNC^0$, this results in the class $\mathsf{StateQNC^0}$, which is the set of states preparable from the $\ket{0}^n$ state by poly-size constant-depth circuits. 
This notion is similar to the relative complexity from~\cite{Aaronson:2004}, where one state is the  $\ket{0}^n$ state and instead of counting the number of gates we consider the set of states preparable by a fixed number of gates. Using this notion of complexity we show that any state preparable by an $\LAQCC^*$ circuit is also preparable by a $\mathsf{PostQPoly}$ circuit, the class of circuits of polynomial depth with an additional post-selection gate. 

All Clifford circuits have a constant-depth $\LAQCC$ implementation, implying that any stabilizer state can be implemented by a constant-depth $\LAQCC$ circuit, see Section~\ref{sec:clifford_circuits} for a proof of this statement. 
Efficient circuits for stabilizer states have been known already through measurement-based quantum computing. Therefore this paper focuses on the preparation of non-stabilizer states, and as a surprising result we find novel constant-depth protocols for four very natural classes of non-stabilizer states.
Despite the extensive research into these four classes of non-stabilizer states and the many applications of them, no efficient constant- or low-depth state preparation protocols are known yet. We specifically consider these four classes as they are all often used as initial states in other algorithms.

The first state is a uniform superposition over an arbitrary number of states. 
This state finds applications in many quantum algorithms, as they often start with a uniform superposition over multiple states. 
This superposition is often achieved by applying Hadamard gates to every qubit due to its simplicity to prepare. 
Yet, the analysis of many algorithms, such as Shor's algorithm~\cite{Shor:1997}, would benefit from a different initial superposition. 
The circuit to prepare the uniform superposition over an arbitrary number of states uses an exact version of Grover search as a subroutine, that turns a probabilistic circuit, with a known constant probability of success, into a deterministic circuit. 
We use the circuit for preparing a uniform superposition over an arbitrary number of states as a subroutine in the next two quantum state preparation protocols. 

The second state is the $W$-state, the uniform superposition over all computational basis states of Hamming-weight~$1$, a natural long-ranged entangled state that displays a fundamentally nonequivalent type of entanglement from the Greenberger–Horne–Zeilinger state~\cite{WState:2000}, for which $\LAQCC$-type constant-depth circuits were previously known~\cite{PhamSvore2013, Cirac:2021}. 
The $W$-state is often used as benchmark for new quantum hardware~\cite{Haffner2005,Neeley2010,GarciaPerez:2021}. 
A novel way to prepare the $W$-state therefore gives a new way to benchmark different quantum devices with each other. 
A circuit for preparing the $W$-state was given in~\cite{Cirac:2021}, but this implementation requires sequentially alternating measurements followed by local unitaries, which in the $\LAQCC$ model is not considered to be of constant depth. 
We improve this protocol by giving an $\LAQCC$ implementation of the $W$-state, based on a compress-uncompress method that links the one-hot and binary encoding of integers.

The third state considered is the Dicke state, a generalization of the $W$-state, a superposition over all computational basis states with Hamming-weight $k$~\cite{Dicke:1954}. 
Dicke states have relevance in various practical settings.
For instance, for quantum game theory~\cite{zdemir2007}, quantum storage~\cite{Bacon_Compress:2006,Plesch:2010}, quantum error correction~\cite{ouyang2014permutation}, quantum metrology~\cite{toth2012multipartite}, and quantum networking~\cite{prevedel2009experimental}. 
Dicke states have been used as a starting state for variational optimization algorithms, most notably Quantum Alternating Operator Ansatz (QAOA)~\cite{Hadfield2019}, to find solutions to problems such as Maximum k-vertex Cover~\cite{Brandhofer2022,cook2020quantum}.
The ground states of physical Hamiltonians describing one-dimensional chains tend to show a resemblance to Dicke states such as states resulting from the Bethe ansatz, making them an ideal starting state when investigating the ground state behavior of these Hamiltonians~\cite{TDL_BetheAnsatzDerivation:2010,B_ExcitedStateQuantumPhaseTransitions:2013,DickeTransitions:2021}. 
For instance, the algorithm by \citeauthor{van2021preparing}, who give an algorithm to prepare the Bethe ansatz eigenstates of the spin-1/2 XXZ spin chain, starts by first preparing a Dicke state~\cite{van2021preparing}. 
A Dicke-state preparation protocol based on the compress-uncompress methodology used in the $W$-state furthermore finds applications in entanglement distillation, where the entanglement of a large state is concentrated on only a few qubits. 
Efficient deterministic circuits for preparing Dicke states have been proposed by \citeauthor{bartschi2019deterministic}~\cite{bartschi2019deterministic, bartschi2022deterministic_short_depth}. 
They provide a quantum circuit of depth $\mathO(k \log(\frac{n}{k}))$, allowing arbitrary connectivity, to prepare a Dicke state, which they conjecture to be optimal when $k$ is constant. 
In this work, we provide a constant-depth $\LAQCC$ circuit below their conjectured bound already for constant $k$. 
However, this does not directly disprove their conjecture, as we allow for intermediate measurements and classical computations. 
More significantly, we even construct constant-depth $\LAQCC$ circuits for $k = \mathO(\sqrt{n})$ greatly improving their bound.
This construction extends the compress-uncompress method for the $W$-state combined with additional subroutines. 

We continue with a log-depth state preparation protocol for the Dicke-state for arbitrary $k$. 
This protocol implements an efficient transformation between the factoradic number representation and the combinatorial number representation of a positive integer. 
The combinatorial number representation relates directly to the Dicke state. 
The provided efficient transformation between number representation systems might be of independent interest. 

We conclude by modifying our protocol for preparing a Dicke-state to a protocol that prepares quantum many-body scar states in constant-depth. 
These states have low entanglement and longer coherence times than states with similar energy density.
These characteristics make many-body scar states interesting to analyze and relevant within physics.
Many-body scar states appear for instance in the AKLT model~\cite{AKLT:1987,MRBAR:2018,MRB:2018} and different spin models~\cite{SI:2019,MOBFR:2020}.
Known methods for preparing these states have polynomial-depth~\cite{Gustafson:2023}, whereas our circuit has constant depth. 

% We conclude by studying the power that intermediate classical calculations can add to quantum computations. 
% In this study, we define a new model that relaxes constant-depth quantum circuits to polynomial depth quantum circuits, log-depth classical calculations to unbounded classical computations and a constant number of alternations to a polynomial number of alternations. 
% We call this model $\LAQCC^*$. 
% We study this model by doing a complexity theoretical analysis, where we draw inspiration from the notions of complexity given by \citeauthor{RosenthalYuen:2022}, \citeauthor{MetgerYuen:2023}, and \citeauthor{Aaronson:2004}.
% All three complexity notions are based on the notion of state preparation, instead of more traditional definition of complexity such as the decidability of a computational problem. 
% The first two consider classes based on sequences of quantum states preparable by a polynomial-sized quantum circuit, where the circuits are uniformly generated by a computational class, for instance, the class $\mathsf{PSPACE}$, which results in the complexity class $\mathsf{StatePSPACE}$~\cite{RosenthalYuen:2022,MetgerYuen:2023}.
% The third notion considers a relative complexity, where the complexity is measured between two given states, and is measured by the number of gates, from a given gate-set, required to transform one state in another state~\cite{Aaronson:2004}. 
% For our definition of state preparation complexity, we drop the uniformity constraint from~\cite{RosenthalYuen:2022,MetgerYuen:2023} and define a class as $\mathsf{StateX}$, which refers to states preparable by circuits of type $\mathsf{X}$. 
% As an example, if $\mathsf{X} = \QNC^0$, this results in the class $\mathsf{StateQNC^0}$, which is the set of states preparable from the $\ket{0}^n$ state by poly-size constant-depth circuits. 
% This notion is similar to the relative complexity from~\cite{Aaronson:2004}, where one state is the  $\ket{0}^n$ state and instead of counting the number of gates we consider the set of states preparable by a fixed number of gates. Using this notion of complexity we show that any state preparable by an $\LAQCC^*$ circuit is also preparable by a $\mathsf{PostQPoly}$ circuit, the class of circuits of polynomial depth with an additional post-selection gate. 

\paragraph{Summary of results}
\begin{itemize}
    \item We give a new definition of a computational model that captures the power of the four step process: applying a constant number of layers of one- and two-qubit gates; performing a syndrome measurement; perform a fast classical computation determining corrections; apply corrections. We call this model \emph{Local Alternating Quantum Classical Computations}, or $\LAQCC$ for short. In this model we bound the allowed quantum operations, intermediate classical calculations, and number of rounds separately. In Section~\ref{sec:LAQCC_model} we define this model and give a list of operations based on results from literature contained in this computational model. In some of these operations we explicitly use that we allow for multiple, but at most constant, rounds  of corrections.
    \item  We show show that there exist $\LAQCC$ circuits that can not be weakly simulated in Section~\ref{sec:IQP_in_LAQCC}. We further show that for every $\LAQCC$ circuit there exists a $\QNC^1$ circuit simulating it perfectly, in Section~\ref{sec:LAQCC_in_QNC1}.
    \item We introduce a new type computational complexity for preparing states and show that the extension of $\LAQCC$ where we allow a polynomial number of rounds and unbounded classical computation, is contained in $\mathsf{PostQPoly}$, the class of polynomial circuits with post-selection, in Section~\ref{sec:Complexity results}.
    \item We show a protocol to prepare the uniform superposition state of size $q$ in $\LAQCC$ using $\mathO(\ceil{\log_2(q)}^2)$ qubits in Section~\ref{sec:superposition_modulo_q}. 
    \item We show a protocol to prepare the $W_n$ state in $\LAQCC$ using $\mathO(n\log(n))$ qubits in Section~\ref{sec:W_state_in_LAQCC}.
    \item We show two ways of preparing the Dicke-$(n,k)$ state. The first method is in $\LAQCC$, works up to $k = \mathO(\sqrt{n})$, uses $\mathO(n^2\log(n))$ qubits, and is found in Section~\ref{sec:dicke:small_k}. The second method is in $\LAQCC\text{-}\mathsf{LOG}$ (an extension of $\LAQCC$ allowing for logarithmic number of alterations instead of constant), works for any $k$, uses $\mathO(\text{poly}(n))$ qubits, and is found in Section~\ref{sec:Dicke_in_LAQCC_LOG}. 
    \item We extend on our $\LAQCC$ method of generating Dicke-$(n,k)$ states for $k = \mathO(\sqrt{n})$ and show a protocol to generate many-body scar states for a particular Hamiltonian in $\LAQCC$ (Section~\ref{sec:many_body_scar}). 
\end{itemize}
Summarized in a table, we provide the following state generation protocols:
\begin{table}[htb]
\centering
\begin{tabular}{l|l|l|l}
\textbf{State description} & \textbf{Width} & \textbf{Depth} & \textbf{Implementation}\\
\hline 
Uniform superposition mod $q$: $\frac{1}{\sqrt{q}} \sum_{i = 0}^{q-1}\ket{i}$ & $\mathO(\ceil{\log^2 q})$ & $\mathO(1)$ & Section~\ref{sec:superposition_modulo_q}\\

$W$-state: $\frac{1}{\sqrt{n}}\sum_{i = 0}^{n-1}\ket{e_i}$ & $\mathO(n \log n)$ & $\mathO(1)$ & Section~\ref{sec:W_state_in_LAQCC}\\

Dicke-$(n,k)$, $k = \mathO(\sqrt{n})$: $\binom{n}{k}^{-1/2}\sum_{x \in \{0,1\}^n: |x| = k} \ket{x}$ &  $\mathO(n^2\log n)$ & $\mathO(1)$ 
&Section~\ref{sec:dicke:small_k}\\

Dicke-$(n,k)$: $\binom{n}{k}^{-1/2}\sum_{x \in \{0,1\}^n: |x| = k} \ket{x}$ & $\mathO(\text{poly}(n))$ & $\mathO(\log n)$ &Section~\ref{sec:Dicke_in_LAQCC_LOG}\\

QMBS: $\ket{S_k} = \frac{1}{k! \sqrt{\mathcal N(n,k)}}(Q^\dagger)^k \ket{\Omega}$ &  $\mathO(n^2\log n)$ & $\mathO(1)$  &  Section~\ref{sec:many_body_scar}
\end{tabular}
\caption{Summary of state preparation protocols given in this paper.}
\label{tab:sate_prep}
\end{table}
In the entry for the quantum many-body scar state $Q$ denotes the raising operator and $\mathcal N(n,k)=\binom{n-k-1}{k}$. 
Section~\ref{sec:many_body_scar} will provide more details on the variables and the implementation. 

\paragraph{Organization of the paper}
\noindent We first introduce relevant preliminaries in Section~\ref{sec:preliminaries}. 
In Section~\ref{sec:LAQCC_model} we formally define the class of Local Alternating Quantum-Classical Computations ($\LAQCC$). We also show that any Clifford circuit can be implemented in constant depth $\LAQCC$ (a result based on a result from measurement-based quantum computing~\cite{jozsa2006introduction}). 
This result allows us to give many useful multi-qubit gates and routines in Section~\ref{sec:gates_created_in_LAQCC}. 
Beyond that we show that constant depth $\LAQCC$ circuits are contained in $\QNC^1$ and that any $\mathsf{IQP}$ circuit has an $\LAQCC$ implementation.
We conclude this section with an analysis of a more powerful instantiation of $\LAQCC$ and show an inclusion with respect to the class $\mathsf{PostQPoly}$, which is the class of circuits of polynomial depth with one additional post-selection gate. 
In Section~\ref{sec:state_prep_in_LAQCC} we give $\LAQCC$ circuit implementations for preparing the uniform superposition over an arbitrary number of states, the $W$-state and the Dicke state up to $k = \mathO(\sqrt{n})$. We furthermore give a log-depth circuit implementation for preparing the Dicke state for any $k$. We conclude by showing a $\LAQCC$ circuit for generating many body scar states of a particular type of Hamiltonian.


\section{Related Work}
%\subsection{Cost Volume based Deep Stereo Matching}
%Stereo matching is a typical problem that has been studied for decades and a well-known four-step pipeline \cite{scharstein2002taxonomy} has been established, where cost volume construction is an indispensable step. Current state-of-the-art stereo matching methods are all cost volume based methods and they can be categorized into two types. Typically, a cost volume is a 4D tensor of height, width, disparity, and features. The first category just uses a full correlation to generate a single-feature cost volume. Such methods are usually efficient but lose much information because of the decimation of feature channels. Many previous work, including Dispnet \cite{dispnet}, MADNet \cite{madnet}, IResNet \cite{iresnet} and AANet \cite{aanet}, belong to this category. The second category usually uses concatenation \cite{gcnet} or group-wise correlation \cite{gwcnet} to generate a multi-feature 4D cost volume. Such a method can achieve better performance while requiring higher computational complexity and memory consumption. Actually, a majority of the top-performing networks in public leaderboards belong to this category, such as GANet \cite{ganet}, CSPN \cite{cspn} and ACFNet \cite{acfnet}. These methods generally employ multiple 3D convolution layers to constantly regularize the 4D cost volume and then apply softmax over the disparity dimension to produce a discrete disparity probability distribution. The final predicted disparity is obtained by softly weighting indices according to their probability, which is also called soft argmin in GCNet \cite{gcnet}. However, soft argmin leaves the output susceptible to multi-modal disparity probability distributions. ACFNet \cite{acfnet} observes this problem and proposes to directly supervise the cost volume with unimodal ground truth distributions. In contrast, we define an uncertainty estimation to quantify the degree to which the cost volume tends to be multi-modal distribution, higher implies the higher possibility of estimation error.

\subsection{Multi-scale Cost Volume based Stereo Matching}
Cost volume construction is an indispensable step in the well-known four-step pipeline for stereo matching \cite{scharstein2002taxonomy, pamisurvey1, pamisurvey2}. Typically, current state-of-the-art stereo matching methods can be categorized into two types of cost volume-based methods, where the cost volume is a 4D tensor of height, width, disparity, and features. The first category usually uses the single-feature 3D cost volume generated by full correlation, which is efficient while losing much information due to the decimation of feature channels. Many real-time methods, such as Dispnet \cite{dispnet}, MADNet \cite{madnet, madnet_pami} and AANet \cite{aanet}, belongs to the category. Moreover, two-stage refinement \cite{mcvmfc} and pyramidal towers \cite{madnet} are commonly applied in the single-feature cost volume based network to construct multi-scale cost volume. The second category usually uses the multi-feature 4D cost volume generated by concatenation \cite{gcnet} or group-wise correlation \cite{gwcnet}, which can achieve better performance with higher computational complexity and memory consumption. Most top-performing networks, including GANet \cite{ganet}, CSPN \cite{cspn} and ACFNet \cite{acfnet} belong to this category. 
% In these methods, the 4D cost volume is constantly regularized by multiple 3D convolution layers and then a discrete disparity probability distribution can be produced by softmax. Next, the final predicted disparity can be obtained by softly weighting indices according to their probability \cite{gcnet}. However, such output is susceptible to multimodal disparity probability distributions and ACFNet \cite{acfnet} gives a solution by directly supervising the cost volume with unimodal ground truth distributions to alleviate this problem. 
Recently, to alleviate the high computational complexity and memory consumption when employing multi-feature 4D cost volumes, \cite{cvpmvsnet, cascade, uscnet} propose to use cascade cost volume representation in multi-view stereo. These methods usually first predict an initial disparity at the coarsest resolution of the image and then gradually refine the disparity by narrowing down the disparity search space. More closely related to our approach is Casstereo \cite{cascade}, which first extended such representation to stereo matching. It selected to uniform sample a pre-defined range to generate the next stage’s disparity search range. Instead, we employ pixel-level uncertainty estimation to adaptively adjust the next stage disparity searching range and generate pseudo-labels for subsequent domain adaptation. Our method also shares similarities with UCSNet \cite{uscnet}, which constructs uncertainty-aware cost volume in multi-view stereo while it doesn’t employ uncertainty estimation to generate pseudo-labels.

%\subsection{Multi-scale Cost Volume based Deep Stereo Matching} 
% \subsection{Multi-scale Cost Volume based Stereo Matching} 
%Multi-scale cost volume firstly was applied in the single-feature cost volume based network with the form of two-stage refinement \cite{mcvmfc} and pyramidal towers \cite{madnet}. Recently, cascade cost volume representation \cite{cvpmvsnet, cascade, uscnet} was proposed in multi-view stereo to alleviate the high computational complexity and memory consumption when employing multi-feature 4D cost volumes. These methods generally predict an initial disparity at the coarsest resolution of the image. Then, they will narrow down the disparity search space and gradually refine the disparity. More closely related to our approach is Casstereo \cite{cascade}, which first extended such representation to stereo matching. It selected to uniform sample a pre-defined range to generate the next stage’s disparity search range. Instead, we employ uncertainty estimation to adaptively adjust the next stage pixel-level disparity searching range and push the next stage's cost volume to be predominantly unimodal.

% The single-feature cost volume based network with the form of two-stage refinement \cite{mcvmfc} and pyramidal towers \cite{madnet} first employ multi-scale cost volume for stereo matching. Recently, to alleviate the high computational complexity and memory consumption when employing multi-feature 4D cost volumes, \cite{cvpmvsnet, cascade, uscnet} propose to use cascade cost volume representation in multi-view stereo, which generally predict an initial disparity at the coarsest resolution of the image. Then, the disparity search space is narrowed down and the disparity is gradually refined. More closely related to our approach is Casstereo \cite{cascade}, which first extended such representation to stereo matching. It selected to uniform sample a pre-defined range to generate the next stage’s disparity search range. Instead, we employ uncertainty estimation to adaptively adjust the next stage pixel-level disparity searching range and push the next stage's cost volume to be predominantly unimodal.

% Figure environment removed

\subsection{Robust Stereo Matching} 
There exist three categories of generalization definitions for robust stereo matching. 1) Cross-domain Generalization: the network’s ability to perform well on unseen scenes (cannot see the image pairs of the target domain in advance). Towards this end, Jia et al \cite{sungeneralizaiton} propose to incorporate scene geometry priors into an end-to-end network. Zhang et al \cite{dsmnet} introduce a domain normalization and a trainable non-local graph-based filter to construct a domain-invariant stereo matching network. 2) Adapt Generalization: the network’s ability to adapt pre-trained models to the new domain with unlabeled target data. Previous work usually pre-trains the models on synthetic data and then adapts it to new target domains with Graph Laplacian regularization \cite{zoom}, non-adversarial progressive color transfer \cite{adastereo}, and Knowledge Reverse Distillation \cite{aohnet}. More closely related to our approach are \cite{aohnet, unsuperviseddomainadaptation} in stereo matching and Monoresmatch \cite{monoresmatch} in monocular depth estimation, which also proposes to generate a pseudo-label for domain adaptation. However, these methods all select to employ classical stereo matching methods \cite{sgm} alongside with confidence estimators, e.g., left-right consistency check to generate pseudo-labels. That is all these methods need an independent method to generate corresponding pseudo-labels. Instead, the proposed method is an end-to-end network that can generate the predicted disparity map, corresponding uncertainty map and pseudo-labels jointly, which is a more simple, yet efficient way. 
% Instead, our proposed method can employ pixel-level and area-level uncertainty estimation to self-distill the predicted disparity maps of our pre-training model and generate sparse while reliable pseudo-labels to align the domain gap, which is a more simple, yet efficient way. 
3) Joint Generalization: the network’s ability to perform well on a variety of datasets with the same model parameters. MCV-MFC \cite{mcvmfc} introduces a two-stage finetuning scheme to achieve a good trade-off between generalization and fitting capability on multiple datasets. However, it doesn’t touch the inner difference between diverse datasets, e.g, the unbalanced disparity distribution. To further address this problem, we propose a cascade cost volume to adaptively the next stage disparity searching space, where the pixel-level uncertainty estimation is at the core.

% \subsection{Monocular Depth Estimation}
% Monocular depth estimation aims to estimate depth values from a single image, instead of stereo images or multiple frames in a video. This problem is ill-posed because of the ambiguity of object sizes. However, humans could estimate the depth from a single image with prior knowledge of the scenes. Recently, learning based methods were explored to learn depth values by supervised or unsupervised learning. Eigen et al. first employed Convolutional Neural Networks (CNN) to predict depth in a coarse-to-fine manner and further improved its performance by multi-task learning. Liu et al. presented deep convolutional neural fields model by combining deep model with continuous CRF. Li et al. [22] refined deep CNN outputs with a hierarchical CRF. Multi-scale continuous CRF was formulated into a deep sequential network by Xu et al. [45] to refine depth estimation. Unsupervised methods tried to train monocular depth estimation with stereo
% image pairs or image sequences and test on single images. Garg et al. [9] used novel image view synthesis loss to train a depth estimation network in an unsupervised way. Godard et al. [11] introduced left-right consistency regularization to improve the performance of view synthesis loss. Recently, some work also propose to use the stereo matching network as a proxy to learn depth from synthetic data or directly employ traditional stereo matching methods to distill proxies labels from the target domain, which proves the feasibility of distilling stereo matching networks to learn monocular depth estimation.



In this section, we describe how to learn repair strategies from the  unsafe programs and edits collected in Section~\ref{sec:data}. We define a \dsl (Section~\ref{subsec:dsl}) to express repair strategies that take an \pdg of an unsafe program  as input and generate a safe program as output. The DSL is expressive and can even express bad strategies that don't generalize well to programs in the wild. We provide examples of such bad strategies and good strategies that generalize well  (Section~\ref{subsec:examples}). We learn good repair strategies  in a data-driven manner using an example-based synthesis algorithm (Section~\ref{subsec:synthesis}). %Finally, given a new unsafe program and a set of learned repair strategies, we apply these strategies and generate  candidate repairs (Section~\ref{subsec:applying}).


%Our goal is to use the collected data to learn high-level general repair strategies. We learn these repair strategies over a joint representation of the \astree with the annotations inferred from the \sa tool (the representation referred to as \pdg ahead).  These inferred \sa tool annotations allow us to take the advantage of rich semantic information while performing \unsure{repairs}. Figure ~\ref{fig:example1-pdg} shows an example \pdg corresponding to the unsafe code shown in Figure ~\ref{fig:unsafememberex}. We develop a powerful \dsl that can utilize the annotations in the \pdg structure and learns repair strategies using a deductive synthesis algorithm. More specifically, strategies in this \dsl operate over the \pdg structure of unseen code-snippets and suggest appropriate edits correspondingly. \aksays{The following sentence can be removed if space becomes a constraint.} Section~\ref{subsec:dsl} describes the \dsl, Section~\ref{subsec:synthesis} talks about the synthesis algorithm, and Section~\ref{subsec:applying} demonstrates strategies in this \dsl can be applied. 


\subsection{\dsl for repair strategies}
\label{subsec:dsl}
We introduce a novel \dsl to express repair strategies in Figure~\ref{fig:fixing-dsl}.
%that use the knowledge of program semantics annotated on \pdg instead of just using the syntactic program structure and in-turn are more expressive and generalize better. These strategies take the an unsafe-program as input and return candidate repair programs by performing tree-edit-operations.
At a high level, the strategies define a three-step process where  they provide a computation to identify the edit-location node \editloc, a computation to identify the child index $\editindex$ of \editloc where repair happens, and a computation to generate the AST that must be placed at  index $\editindex$ of \editloc for the repair. The main part of these computations involve traversing paths of the input unsafe program \prog.
%The edit-operation can either be inserting a syntactic-child at \editloc (\insertsc) or replacing a syntactic-child with another tree at \editloc (\replace). The index at the \astree-node $\editloc$ where the insertion or replacement occurs is called the edit-index (\editindex). The tree that is inserted or replaces another existing tree at the \editloc is materialized hierarchically for the given example by defining abstract program structure using a combination of constant structure and references to \astree-nodes in the existing program \prog. These \astree-nodes are called reference-locations (\refloc). To find these locations (\editloc, \refloc) in a given program, the strategies abstractly store \textit{traversals} which materialize into a \textit{concrete} \astree-node in the given programs.  

%\naman{todo - talk about traversal in the introduction, background etc.}
\input{dsl}

%We present our \dsl in Figure ~\ref{fig:fixing-dsl}. 
%The DSL is a list of definitions for various non-terminals in the grammar. For each non-terminal, we define a corresponding type and a set of production rules. Each production rule is either a fixed expression, or an operator applied to other non-terminals or fixed-expressions in the grammar.  
The top-level production rule of the DSL defines strategies, \strategy, with type \newtextsc{Strategy}. 
%A \node is either the source node (\prog.source) or an application of \traversal on another \node. 
\gettraversal, \getclauses, and \getindex are all functions that take a \node $n$ as input and return a \node, \bool, and \integer as output respectively. The edit-AST, \eastree, is similar to a syntactic variant of \astree (i.e. no semantic edges) which we define in Section~\ref{sec:data} with one addition. It has reference nodes that, when applying the strategy to the input \pdg of \prog, are materialized from sub-trees of this \pdg, where the root nodes of these sub-trees are identified by traversing paths in the input. 
%Finally, edge-type (denoted by \edgetype) is an enumeration describing the type of edge, i.e. syntactic or semantic, and parent or child, as defined in Section~\ref{sec:data}. 

% Given these types, we now define the operators used in our \dsl. 
The strategy \strategy is of two types, \insertsc or \replace. \DMethod{Insert}{\I{L}}{\I{I}}{\I{O}}\ declaratively expresses the computation that computes the edit-location \editloc by traversing the path supplied in \I{L}, then computes \editindex, the index of edit-location,  by evaluating \I{I}(\editloc), and inserts the materialization of \I{O} as a syntactic-child \astree at index \editindex of the edit-location \editloc. \DMethod{Replace}{\I{L}}{\I{I}}{\I{O}}\ is similar and performs a replacement instead of an insertion.
%computes the \editloc and \editindex, and replaces syntactic-child of \editloc at \editindex with \I{O}.
%The insertion and replacement operations modify the nodes $\mathcal{N}$ and edges $\mathcal{E}$ of the \astree (Figure~\ref{fig:astsyntax}) appropriately. 

\node (\I{L}) is either the node corresponding to the source of vulnerability (\prog.source) or the target of the path corresponding to the traversal \DMethod{ApplyTraversal}{L}{\I{F}$_k\ o\ $\I{F}$_{k-1}\ o\ \cdots$\I{F}$_0$}. 
Here, each \I{F}$_i$ 
is a function that takes a node 
$n$ as input, performs a traversal from $n$, and returns the traversal's target node $n'$. 
Thus, \T{ApplyTraversal} can be recursively defined as \DMethod{ApplyTraversal}{\I{F}$_0$(L)}{\I{F}$_k\ o\ $\I{F}$_{k-1}\ o\ \cdots$\I{F}$_1$}\ if $k>0$ and \I{F}$_0$(L) otherwise. 

\newtextsc{GetTraversal} (\I{F}) defines a function that takes a node $n$ and returns a node $n'$ reachable from $n$ and can be of two types. Given $n$, the \DMethod{GetEdge}{\I{ET}}{\I{I}}\ operator first finds the possible single-edge traversals of type \I{ET} and indexes it using \I{I}. Specifically, if edge type \I{ET} is a parent then it returns the parent of $n$. Otherwise, 
it finds a set of $N$ of nodes that are connected with $n$ via the edge type \I{ET}, i.e., $N = \mathcal{E}(n, \I{ET})$, and returns the node $N[I(\I{n})]$ at the index given by $I$. In contrast, $\DMethod{GetKleeneStar}{\I{ET}}{\I{C}}(n)$  performs a \newtextsc{KleeneStarTraversal} that iteratively traverses edges of type \I{ET}, staring from input node $n$, until it reaches an edge whose target  node $n^{i}$ satisfies the condition defined by the clause \I{C}. Formally, \newtextsc{KleeneStarTraversal} can be defined recursively as $KE(n_1,ET,C) = \I{C}(n_1)? n_1 : \left(let\ t\in\mathcal{E}(n_1,ET)\ in\ KE(t,ET,C)\right)$. Here, the node $t$, which is target of an edge with source $n_1$ and type $ET$,  is chosen non-deterministically and our implementation resolves this non-determinism through a breadth-first search.
%A \traversal is a relation between nodes $n_1$ and $n_2$ such that there is an edge or a sequence of edges between them. 

\newtextsc{GetIndex} (\I{I}) defines a function that takes a node $n$ and returns a \integer. It is either a constant function that returns a fixed integer $z$ or a \DMethod{GetOffsetIndex}{\I{L}, \I{z}}. \DMethod{GetOffsetIndex}{\I{L}, \I{z}}\ takes a node $n$ as input and returns an integer $DO(n,L)+z$, where $DO(n_1,n_2)$ returns the index of syntactic child of $n_2$ who is a syntactic ancestor of $n_1$. 

\eastree (\I{O}) defines the edit \astree with reference nodes which, given an input program \prog, are materialized to a concrete \astree. The \eastree can either be a \T{ConstantAST} or a \T{ReferenceAST}. Specifically, \DMethod{ConstantAST}{$\tau$}{\I{value}}{\I{O}$_1$}{\I{O}$_2$}{$\cdots$}{{\I{O}$_k$}}\ returns an \eastree that has a type $\tau$, string representation \I{value}, and is recursively constructed with sub-trees \I{O}$_1 \cdots$ \I{O}$_k$ as syntactic children, each of which can either be a \T{ConstantAST} or a \T{ReferenceAST}. The \DMethod{ReferenceAST}{\I{L}}, when applying the strategy, finds a node $n$ in \prog by traversing the path described in \I{L} and returns a copy of the (syntactic) sub-tree of \prog rooted at $n$. %Next, we show examples of strategies written in this DSL and how to learn them automatically.

% Finally, note that the traversals can be composed by applying multiple \T{ApplyTraversal} operators sequentially. We use this key insight into developing our learning from examples setup. 
\newcommand{\newwrapbox}[2]{\adjustbox{margin=1pt 1.3pt, bgcolor=white, frame=1pt, cframe=#1, color=#1}{#2}}
% Figure environment removed


\lstMakeShortInline[columns=fixed]@
\subsection{Example of strategies in our \dsl}
\label{subsec:examples}
Figure~\ref{fig:repair-strategy-ex1} describes   two possible repair strategies that are sufficient to repair the motivating example in Figure~\ref{fig:vulnerabilty-example1}. We first describe the good strategy in Figure~\ref{fig:strat1}, referred to as \strategyone,  and then compare it with the bad strategy \strategytwo in Figure~\ref{fig:strat2}. 

Given the program \prog in Figure~\ref{fig:vulnerabilty-example1}(a) as input, the strategy \strategyone
performs a replacement at index \I{I} of edit-location $L_e$ with the materialization of \I{O} (line 20 of \strategyone).
This process requires first finding the "semantic location" node \semloc. %The semantic location for \prog is shown in red in Figure~\ref{fig:example1-pdg}.
To this end, the strategy 
first  traverses a path from the node annotated as \T{source} by \sa  using \DMethod{GetKleeneStar}\ in Line~\ref{lst:line:semkleene} of \strategyone.  This \newtextsc{KleeneStarTraversal} starts from \T{source}, traverses semantic dataflow edges, and stops at a node 
corresponding to an identifier being used as the function name in a function call. 
 For the input program $P$, the traversal takes the semantic-child-edges 1-7 (Figure~\ref{fig:example1-pdg}) and stops at @foo@ in Line~\ref{lst:line:callerId-sink} of Figure~\ref{fig:vulnerabilty-example1}(a). Next, to reach the edit-location $L_e$, the strategy uses a \newtextsc{KleeneStarTraversal} that starts from \semloc, traverses syntactic parent edges,  and stops when it reaches a \blockstmt. For $P$, this traversal sets $L_e$  as the node corresponding to the \blockstmt between Lines~\ref{lst:line:handlers-run} and ~\ref{lst:line:handlers-run-end} of Figure~\ref{fig:vulnerabilty-example1}(a). Next, in Line~\ref{lst:line:offseteditindex} of \strategyone, the index \I{I} is set to the index corresponding to the  syntactic child of the edit-location $L_e$ who is an ancestor of the semantic location $L_s$ . For $P$, this index  materializes into $13$; the edge  outgoing from blue \blockstmt in Figure~\ref{fig:example1-pdg} to an ancestor of semantic location (shown in red) has label \T{ch:13}. Next, we materialize the \eastree defined in Line~\ref{lst:line:eastree} of \strategyone by  materializing the  reference-nodes. The \eastree \I{O} serializes into @if (REF1.hasOwnProperty(REF2)) { REF3 } @ where @REF1@, @REF2@, and @REF3@ correspond to \T{ReferenceAST} operators with locations as \reflocone, \refloctwo, and \reflocthree. \refloctwo traverses semantic-parent edges  from \semloc (Line~\ref{lst:line:goodref}) and materialize into @data.id@. Similarly, \reflocone and \reflocthree traverse syntactic children edges and materialize into @handlers@ and @foo(data);@ respectively. Thus, the \eastree \I{O} materializes  into @if (handlers.hasOwnProperty(data.id)) { foo(data); }@, which is the required repair. 

%When \strategyone is given the program \prog in Figure~\ref{fig:vulnerabilty-example1}(a) as input, then it first  traverses a path from the source node to the "semantic location"  \semloc using \DMethod{GetKleeneStar}{"SemChild"}{\DMethod{GetClause}{"Expr"}}\ in Line~\ref{lst:line:semkleene}. This leads to a \newtextsc{KleeneStarTraversal} with the stopping condition $\lambda n.\mathcal{T}[n] = \text{"CallExpr"}$. For the input program, the traversal skips through the semantic-child-edges 1-7 and reaches @foo@ in Line~\ref{lst:line:callerId-sink}. Next, it applies another \newtextsc{KleeneStarTraversal} starting from \semloc to reach \editloc in Line~\ref{lst:line:synkleene}. This traversal skips over syntactic-parent-edges and reaches the \blockstmt between Lines~\ref{lst:line:handlers-run} and ~\ref{lst:line:handlers-run-end}. Next, in Line~\ref{lst:line:offseteditindex}, the index \I{I} is computed as \DMethod{GetOffsetIndex}{Ls, 0}\ which means to pick the child-index of \editloc that has \semloc as its descendent. For our example, this index would materialize into the statement number in the block statement containing @foo@, which turns out to be $13$. Next, we instantiate the \eastree in Line~\ref{lst:line:eastree} which hierarchically defines the children-nodes or reference-nodes. The \eastree \I{O} deserializes into @if (REF1.hasOwnProperty(REF2)) { REF3 } @ where @REF1@, @REF2@, and @REF3@ correspond to \T{ReferenceAST} operators with locations as \reflocone, \refloctwo, and \reflocthree. \reflocone and \refloctwo use the semantic-parent edge traversals from \semloc (Line~\ref{lst:line:goodref}) and materialize into @handlers@ and @data.id@. \reflocthree performs a syntactic-child edge traversal from \editloc and materializes into @foo(data);@ thus materializing the entire \eastree \I{O} into @if (handlers.hasOwnProperty(data.id)) { foo(data); }@, i.e. the required repair. 

Now consider the repair strategy \strategytwo in Figure~\ref{fig:strat2}. This strategy shares a similar structure with the earlier strategy but differs in the way traversals and the index $\I{I}$ are computed. There are four key differences
\begin{enumerate}
    \item In order to reach \semloc from \prog.source, \strategytwo performs the \T{EdgeTraversal} using semantic-child edge seven times in Line~\ref{lst:line:nosemkleene}. The number of semantic edges varies widely across programs and prevents generalization to other scenarios. \T{KleeneStarTraversal} operator instead uses \newtextsc{Clauses} over nodes to find the edit-location.
    \item To reach $L_e$ from \semloc, \strategytwo performs the \T{EdgeTraversal} using syntactic-parent edge seven times in Line~\ref{lst:line:nosynkleene}. Consider a program that instead assigns output of the function-call @let out = foo(data)@. \strategytwo will find \assignexpr as the edit-location and fail to generalize whereas \strategyone will appropriately adjust and take four parent steps.
    \item In order to compute the index at which replacement needs to occur, \strategytwo uses a \DMethod{ConstantIndex}{13}\ in Line~\ref{lst:line:consteditindex} of Figure~\ref{fig:strat2}, which effectively assumes that replacement should always occur at 13$^{th}$ child of $L_e$ and again doesn't generalize. \strategyone on the other hand uses of \T{GetOffsetIndex} operator to instead compute index dynamically for a given input program
    \item In order to materialize reference nodes, \strategytwo uses syntactic edge traversals (Line~\ref{lst:line:badref} of Figure~\ref{fig:strat2}) which assume definite structure about the structure of the program (@GetConstant(7)@ used as syntactic child index to solve a long-ranged-dependency). \strategyone instead uses semantic-parent edges to capture the semantics here and produces a better generalizing repair.
\end{enumerate} 

\lstDeleteShortInline@

\noindent These programs highlight that our \dsl is expressive enough to perform complicated non-local repairs in a generic manner. At the same time, while many strategies can repair a given program, all applicable strategies are not equally good. A key realization is that we \emph{prefer shorter traversal functions} (\newtextsc{KleeneStarTraversal}\ over a long sequence of \newtextsc{EdgeTraversal}). Similarly, we \textit{prefer the traversals with none or small constants}. For example, we prefer \DMethod{GetOffsetIndex}{\semloc}{0}\  over \DMethod{GetConstant}{13}\ and semantic-parent traversal over syntactic-parent traversal with index \DMethod{GetConstant}{7}. %Finally, we also \emph{prefer strategies that share traversals across localizing \editloc and \refloc}.
We use these insights to guide the search in our synthesis algorithm.

% The strategy (\strategy) is defined by 
% performs this localization using an edit path (\editpath). We define a path (\genpath) in the strategy as a sequence of edges in the \prog. An edge is either a syntactic \astree edge or a semantic \taintpropedge in either direction (i.e. towards parent or child). 
% The localized node in the \prog is called edit location (\editloc). Next, the strategy either inserts or replaces a child of the edit location with a new \astree. This new \astree can either be a constant node or reference a node in the original \prog using a reference path (\rfpath). Figure ~\ref{fig:approach-notations} summarizes the notations 

% This \dsl was created so we can use the \sa annotations seamlessly and is guided by how humans fix such vulnerabilities. A line of previous works~\cite{} manually write repair patterns for fixing code. Our \dsl-based approach is strictly more general as it can perform various kinds of repairs and the exact repair strategies are learned from data. Moreover, we make effective use of high-level patterns and domain insights, and annotations. So while these other approaches tend to be simplistic and \textbf{either do not generalize well or over-generalize (generate a large number of false positives)}, concrete instantiations of strategies in our \dsl are better at capturing the high-level repair intent better. Following we describe the terminologies used in the repair \dsl.

% \lstMakeShortInline[columns=fixed]@
% The top level rule in our \dsl defines the Repair Strategy (denoted by \strategy). It is parameterized by the type of vulnerability the strategy fixes and the edit \edit. We consider two kinds of edits, either an insert operation or a replace operation. This means that the edit \edit either inserts an \astree child or replaces an \astree child with another \astree. Since we are fixing taint-flow vulnerabilities, we found these two operations to be sufficient. However, our \dsl can be expanded to also handle delete operations \aksays{Why can't we say that delete is replacement with an empty tree?}. In Figure ~\ref{fig:vulnerabilty-example1}, the fix applied in replaces the \astree corresponding to @handlers[callerId](data)@ (line ~\ref{lst:line:callerId-sink}, Figure ~\ref{fig:unsafememberex}) with the if statement in lines ~\ref{lst:line:fix-start}-\ref{lst:line:fix-end}, (Figure ~\ref{fig:safememberex}) and depicts a replace edit. 

% %\paragraph{Edit (\edit)} Since we are solving source-sink-sanitizer vulnerabilities, our \edit either inserts child \astree at edit-locations (denoted by \editloc) or replaces a child with another \astree (at \editloc). %This \editloc is a node in the \pdg which is reachable from the vulnerability source (as provided by the \astree) by traversing syntactic (\astree) or semantic edges in \pdg. %Once the \editloc is found, the new \astree (either replacing the existing child being inserted as a child) can be   

% Notice that in the \pdg, @handlers[callerId](data)@ is a child of the \blockstmt (parenthesis block between lines ~\ref{lst:line:handlers-run}-\ref{lst:line:handlers-run-end} and marked in blue in Figure ~\ref{fig:example1-pdg}). So while applying the fix, we replace the \textit{$k^{th}$} child of \blockstmt with the \ifstmt. We call the node in the \pdg where the edit operation applies as the edit location (\editloc). When a strategy applies, it has to determine this edit location based on the \pdg structure. Our \dsl defines an edit path (denoted by \editpath) to find edit location. In Figure ~\ref{fig:example1-pdg}, starting from the source-node @event@ (marked in orange), we take 7 semantic edges (reaching @callerId@) and then after hopping four synactic parent edges we reach the edit location \aksays{The notion of semantic edges should be defined and explained before this.}. This sequence of edge traversal defines our edit path. More generally, our \dsl considers the \editpath to be a set of semantic edges followed by a set of syntactic edges. This constraint on the paths allows expressivity to learn general strategies while also keeping the search space small. The semantic edges in \editpath allows navigating to ``somewhere close'' to sink location. Next once semantic edges are traversed, \editloc is reached by traversing a set of syntactic edges. Additionally, since the number of semantic edges might vary across examples, our \dsl allows a powerful  operator that navigates an indefinite number of semantic edges. This formulation helps our strategies to generalize well across widely different sets of programs. 

% %\paragraph{Edit Location (\editloc)} Edit location is the node in the \pdg where the edit operation (i.e. insertion or replacement of a child node) applies. \editloc is reachable from the vulnerability source found by traversing the edit-path (\editpath) in the \pdg.  So, in our running example, "$\dots$ function (data){$\dots$}" node (marked in blue) is the edit root and is reachable from the source via first traversing the semantic edges followed by traversing to "syntactic parent" four times. 
% Our kleene-star operator navigates indefinite semantic child edges until a ``stopping node'' (parameterized by a stopping condition) is reached. This stopping-condition is defined by a set of predicates applied on a \aksays{an} \astree node. We find that simplistic predicates about \newtextsc{ASTType} or \newtextsc{ASTValue} of \astree node and its neighbours suffice in locating this stopping node. For our running example, the stopping condition is the conjunction of the predicates @ASTType(node.parent) = IndexExpr@, @ASTType(node.parent.parent) = MethodCallExpr@. The stopping node lies on the taint-flow path from source to sink and therefore is quite relevant to the insert or replace operations (being a proxy for the semantic information of the vulnerability). Therefore, we call the stopping node as the semantic location (\semloc). In our running example, @calledId@, the sink-node is also the \semloc.

% As described above, our \dsl either inserts a new \astree, or it replaces an existing \astree with another \astree. An \astree is defined by three properties -- type, value and an array of children \astree. One can construct such an \astree by concretely initializing it using specific types and values for the tree and its descendants. However, a constant \astree cannot generalize because the fix depends on existing variables in the source code. Therefore, in addition to a constant \astree, our \dsl also allows referring to any existing node in the \astree. This referral is computed by traversing a path from the semantic location (\semloc defined above) to the node to reference \aksays{What is node to reference?}. The corresponding path is known as reference path. For e.g. the condition @handers.hasOwnProperty(callerId)@ which is used in the fix refers to @handlers@ and @callerId@ nodes in the tree and combines them in a constant \callexpr \astree. 

% We described edit paths and reference paths above. More generally a path is a sequence of edges in the \pdg where the edge can be one of syntactic or semantic or ancestral. When selecting a child edge, we also need to store \textit{which child} to select and it is determined by an index. Figure ~\ref{fig:repair-strategy-ex1} shows the entire strategy that fixes the example in Figure ~\ref{fig:vulnerabilty-example1}.
% \lstDeleteShortInline@

% \paragraph{Semantic Location (\semloc)} The node at which Kleene-Star traversal of semantic edges stops is called semantic location. This node is a key component in the repair since this node is a proxy for the semantically important values that would be necessary for making the edit. The sink node, callerId, is also the \semloc in our example. %Additionally, the \editloc is near this node and 

% \paragraph{Paths (\dslpath)} A path is described as a sequence of edges in the \pdg. The edges can be syntactic parent or child edges, semantic child edges, and ancestor edges in the \pdg.

% \paragraph{Index (\dslindex)}
% While selecting a child edge or while determining where the \concinsertcode needs to be inserted or replaced with, we need some index of which child to follow. Generally, this index is a constant value however it can be computed as an offset from the \semloc descendent direction as computed with the \newtextsc{OffsetFrom} operator. \naman{explain the requirement of offset with example}

%\paragraph{\astree} \astree is the tree-representation of the editcode that will replace some existing child of \editloc or will be inserted at some child indices of \editloc. One possible way to construct this \astree is to concretely initialize it using specific values and types. However, a constant \astree cannot generalize because the inserted code almost always depends on specific variables and the structures in the code. Therefore, in addition to a constant structure, the \astree can also refer to existing elements in the \pdg. This referral is again found using a path (\dslpath) traversal from the \semloc, the intuition being that it is a good \textit{proxy for the semantics of the vulnerability} and necessary variables to refer would be close to it.  


\subsection{Synthesizing \dsl strategies from examples}
\label{subsec:synthesis}
%\naman{The discussion about anti-unification would go in related work I presume?}

Given this high-level \dsl, we will now describe our example-based synthesis algorithm. 
We take as input a set of unsafe programs and edits generated as output at the end of data collection step (Section~\ref{sec:data}). 
Let $\{(\prog_{1},\edit_1),(\prog_{2},\edit_2),\dots$ $,(\prog_{n},\edit_n)\}$. 
Here, $\prog_{i}$ is the $i^{th}$ unsafe program and $\edit_i$ is the corresponding edit. Edit ($\edit$) contains the \astree-node of the edit-location ($\edit$.loc), the \textit{concrete} \astree of the edit-program ($\edit$.editprog), and the type of edit i.e. \insertsc or \replace ($\edit$.type). We use these to learn high-level repair strategies in our \dsl. 
\lstMakeShortInline[columns=fixed]@
Our goal is to combine specific paths, learned over examples that share similar repairs in different semantic and syntactic contexts, to obtain general strategies. Our repair strategies abstractly learn the following:
% \begin{enumerate}
%     \item the traversal for localizing edit-locations (\editloc) and reference-locations (\refloc). 
%     \item template-repair-program-representations using the reference-traversals . 
% \end{enumerate}
\begin{enumerate}
    \item Traversals for localizing edit-locations (\editloc) and reference-locations (\refloc). For example, @Ls@ in Line~\ref{lst:line:semkleene} (Strategy \strategyone) depicts a \T{KleeneTraversal} abstraction we can learn from examples having a variable number of semantic-edges. Similarly, @I@ in Line~\ref{lst:line:offseteditindex} (of \strategyone) depicts a generalized index expression we can learn from examples.
    \item \eastree which use reference-traversals. For example, @O@ in Line~\ref{lst:line:goodstratO} demonstrates templated-program-structure that we can learn from examples (say by generalizing from the witnessed guards @handlers.has(data)@  and @events.storage.has(event.name)@).
\end{enumerate}
\lstDeleteShortInline@


% In particular, we wish to abstract over examples that share similar repairs in different semantic or syntactic contexts. Consider the example abstractions below: 
% \begin{enumerate}
%     \item  
%     \item Line~\ref{lst:line:eastree} in Figure~\ref{fig:strat1} depicts 
%     the guard condition in Figure~\ref{fig:safememberex} @handlers.hasOwnProperty(data.id)@ can be abstracted with another guard @eventHandlers._storage.hasOwnProperty(event.name)@ into an abstract template @REF1.hasOwnProperty(REF2)@ where @REF1@ and @REF2@ are \T{ReferenceAST} have use traversals
% \end{enumerate}
% For example, . 
%Consider the example in Figures~\ref{fig:static-witnessing-1},~\ref{fig:static-witnessing-2}, and ~\ref{fig:static-witnessing-3}. They describe three different kinds of syntactic repairs and are not candidates to merge. Instead, 

We depict our synthesis algorithm in Figure~\ref{fig:strategy-learning}. At a high-level, our synthesis algorithm, first pre-processes the inputs, storing the required \textit{concrete} traversals. Next, it performs ranked pair-wise merging over the processed edits to synthesize strategies.
%We merge non-terminals recursively by deductively choosing production rules to enumerate and merging the non-terminals appearing in the productions. %During this recursion, it learns the \textit{traversals} and program templates abstractly. 

\noindent \textbf{Pre-processing.} In this step, given the programs and edits, we store the concrete traversals required for learning \editloc and \refloc (Line~\ref{algo:line:preprocess}). Naively computing all such traversals is very expensive and also leads to \textit{bad strategies}. Here, based on the insights from Section~\ref{subsec:examples}, we only compute the traversals that lead to shorter  
traversals
%\textit{abstract traversals}
which generalize better. In addition, we also share traversals between between \editloc and \refloc. Pre-processing has following three key steps:
\begin{enumerate}
    \item \textbf{Edit Traversals.} We compute the traversals between \prog.source and \editloc (Line~\ref{algo:line:conceditloc} of Figure~\ref{fig:strategy-learning}) that have the form of a sequence of semantic-edges followed by a sequence of syntactic-edges. This allows abstracting variable-length sequences of semantic-edge traversals as a \kleeneedge (corresponding to an abstract \newtextsc{KleeneTraversal}). We implement this using \newtextsc{BiDirecBFS} method at Line ~\ref{algo:line:bidirecbfs}. For every edit-traversal ($\I{T}_e$), we define {\em semantic-location} (\semloc for brevity) as the last-node on the semantic (dataflow) traversal before traversing the syntactic-edges.
    \item \textbf{Compressing Edit Traversals.} We compress these edit-traversals using the \newtextsc{Compress} method in Line~\ref{algo:line:compress}. It takes in a sequence of (syntactic or semantic) edges as input, greedily combines the consecutive edges with the same edge-type (\edgetype) into a \kleeneedge. The \kleeneedge is constructed using the edge type \edgetype, and a set of clauses $\clause_i$ that satisfy the target node of \kleeneedge. These clauses are either $\lambda n. \mathcal{T}(n) = \tau$ that check the type  or $\lambda n. \mathcal{T}(F_i(n)) = \nu$ that check the type of a neighbor. \newtextsc{Compress} returns a sequence of edges or \kleeneedges as output. 
    \item \textbf{Reference Traversals.} For every node of the edit-program, we locate nodes in the \pdg with the same \textit{value} using a \newtextsc{LevelOrderBFS} until a max-depth (Line~\ref{algo:line:maxlevel}). We perform this traversal from \semloc (defined in (1) above). We thus share parts of traversals between locating \editloc and \refloc which helps in learning \textit{better strategies}. The motivation behind using \semloc is that the expressions necessary for repair will be close to \semloc as it lies on the information-flow path. 
\end{enumerate}
%Specifically, for \editloc, we find the traversals between \prog.source and \editloc (Line~\ref{algo:line:conceditloc}) that first navigate a set of semantic-edges followed by a set of syntactic-edges. We implement this using the \newtextsc{BiDirecBFS} function at Line ~\ref{algo:line:bidirecbfs}. %It traverses semantic-edges from the source, syntactic-edges from the edit-location, and returns the intersecting traversals. 
%For every edit-traversal ($\I{T}_e$), we define semantic-location (\semloc for brevity) as the last-node on the semantic (dataflow) traversal before navigating a syntactic-edge. 
%Next, we compress the traversals greedy by combining consecutive edges of the same edge-type (\edgetype) into a \kleeneedge using the \newtextsc{Compress} method in Line~\ref{algo:line:compress}. Every \kleeneedge stores the \edgetype, and a set of clauses $\clause_i : i \in {1,\dots,n}$ that satisfy the end-node of \kleeneedge. These clauses are either $\lambda n. \mathcal{T}(n) = \nu$, i.e. a clause on the type of \semloc or $\lambda n. \mathcal{T}(F_i(n)) = \nu$, i.e. a clause on the type of a syntactic-neighbour of \semloc. We then compute traversals for finding reference locations. Here, instead of computing traversals from the source-node, we instead compute traversals from the semantic-locations. The expressions referenced in repairs are usually close to the \semloc (as it lies on the information-flow path and thus is affiliated with variables likely necessary for building the repair). This traversal-sharing optimizes the search and generalization of our strategies.

\noindent \textbf{Strategy Synthesis.} Given the edits and the associated traversal meta-data, we synthesize the strategy by pair-wise merging  (Line~\ref{algo:line:callmerge}). \newtextsc{MergeEdits}, the top-level synthesis method, takes a pair of edits as inputs and returns a list of strategies satisfying the example edits. We synthesize the strategies recursively using a deductive search over the non-terminals of the DSL (Figure~\ref{fig:fixing-dsl}). Specifically, to synthesize an expression corresponding to a non-terminal, we deduce which production to use and recursively synthesize the non-terminals given by its production-rule. This has the following key components: 
\begin{enumerate}
    \item \newtextsc{MergeEdits}: It takes pairs of edits as inputs and returns the strategy. It recursively synthesizes the traversal (for \editloc), index, and \eastree. It combines and returns them using the edit-type. 
    \item \newtextsc{MergeTraversal}: It takes two concrete traversals (sequence of edges or \kleeneedges) as inputs and returns the abstracted traversal. by merging elements in the sequence.
    \item \newtextsc{MergeEdge}: It takes two edges or \kleeneedges as inputs and returns a \T{GetKleeneTraversal} or \T{GetEdgeTraversal}. We combine two \kleeneedges using their edge-types and intersecting the clauses stored during pre-processing. We combine two edges using their edge-types, and recursively combining their indices.
    \item \newtextsc{MergeIndex}: It takes two integer indices as inputs and returns an abstracted index. If the two input indices are equal, we return a \T{GetConstant} operator with the input index value. Otherwise, we compute offset as the difference between input-index and index of child of $n$ which has \semloc as descendent (computed by $DO(n, \semloc)$). We return this offset if they are equal and an empty-list otherwise. 
    \item \newtextsc{MergeProg}: It takes two programs as input and returns a list of \eastree, where each list element can materialize into the input programs. If the top-level node in the programs have equal values and types, we combine them as a \T{ConstantAST}. Otherwise, we recursively combine their children. Finally, we merge the reference-traversals corresponding to the input programs and combine them into \T{ReferenceAST}.
\end{enumerate}

Our synthesis procedure is inspired by anti-unification~\cite{anti-unification} and we abstract the paths and edit-programs across different examples. Specifically, our \T{KleeneTraversal} and \T{OffsetIndex} functions allow generalization across paths having different number of edges and different indices where naive abstractions fail. Similarly, \eastree also resemble anti-unification over tree-edits. However, again we use traversals over syntactic and longer-context semantic-edges, for better generalizations  and repairs. 

Finally, note that while we perform pair-wise merges over the edits, the strategy synthesis algorithm can be extended to merge bigger cluster of edits together as well. However, from our experience, we find that the pair-wise merging performs well and is sufficient for our experiments. 
%by recursively synthesizing values corresponding to the non-terminals in our \dsl. 
%Thus, to synthesize a strategy, we synthesize traversals for the \editloc. To synthesize a traversal, we check if the two traversals contain an equal number of edges. Next, we try to merge the corresponding pairs of edges on each traversal. To merge an edge, if it is simply a syntactic or semantic edge, we instantiate a \T{GetEdge} operator appropriately. However, if the two edges are \kleeneedge, then we instantiate a \T{GetKleeneEdge} operator where the clauses are constructed by intersecting the clauses computed in \newtextsc{Compress} step. It additionally ensures that the number of clauses after the intersection is more than $1$ to prevent over-generalizing strategies. In order to merge strategies, we next merge the edit-programs (\edit.editprog). We first combine the programs as a \T{ConstantAST} by checking whether the type and value match and then recursively merge the children and finally take the cartesian product of children. Next, based on the concrete reference traversals, we merge them and construct \T{ReferenceAST} \eastree. 



%Note that in order to learn repair strategies, we need to synthesize paths corresponding to all locations (\location) used in the strategy. We use \location in three production rules in our \dsl (\locationone, \locationtwo, \locationthree in Rules~\ref{dslrule:strategy},~\ref{dslrule:index},~\ref{dslrule:node}). Naively, trying to synthesize these paths is very expensive and will lead to non-generalizing strategies.
%Here, based on common fix-patterns for these vulnerabilities, we reduce the search space by enforcing structure over the paths we learn. Additionally, we also share paths between the three locations. 

%We find that performing these 

%At a high-level, our synthesis algorithm takes these unsafe-programs and edits as inputs, preprocesses them to store relevant meta-data, clusters them and then recursively enumerates over the non-terminals in the \dsl. How


% \aksays{This section is very dense. It would be good to take a small example and illustrate the key steps visually.}
\input{synthesisalgo2}

% Given this high-level \dsl, we will now describe our synthesis algorithm. We build a top-down synthesis algorithm that learns strategies in this \dsl through a \pbe approach. We receive a set of unsafe codes and concrete edits \unsure{the terminology of concrete edits can be confusing for the paper. Basically, anologous to all things in \dsl, we have concrete edits, paths, etc.}. Let $\{(\prog_{1},\concedit_1),(\prog_{2},\concedit_2),\dots$ $,(\prog_{n},\concedit_n)\}$ be the data we collect from our data collection step where $\prog_{i}$ is the ith unsafe code snippets and $\concedit_i$ is the corresponding concrete edit. $\concedit_i$ contains the concrete edit-location $\conceditloc_i$, the \astree of the editcode $\concinsertcode_i$, and the type of edit i.e. \insertsc or \replace.

% Given this data, we instantiate our algorithm to learn repair strategies. Our algorithm takes in these set of examples and learns a set of ($\{\strategy_1,\strategy_1,\dots,\strategy_k\}$) that are supposed to cover the training examples. Later, these repair strategies, when given an unsafe program \pdg will generate the edit that needs to be applied. The sketch of our synthesis and learning algorithm is presented in Figure~\ref{fig:strategy-learning}.

% %%O := \{\(\strategy\sb{1},\strategy\sb{2},\dots,\strategy\sb{k}\)\}
\HUGE DONT USE THIS - USE synthesisalgo2
% Figure environment removed

% \unsure{\textbf{Terminology for reference!!:} Letters with overlines are concrete elements ($\concedit$ is a concrete edit, $\conceditpath$ is a concrete edit path) while the letters without lines are abstract elements that can generalize over examples ($\edit$ is an abstract edit, $\editpath$ is an abstract edit path). Following, we again define the various abbreviations used in the algorithm
% \begin{itemize}
%     \item \prog is the \pdg containing \sa annotations
%     \item \concedit is the concrete edit which itself contains editcode, edit-location, edit-type, and indices
%     \item \concinsertcode is the concrete editcode that is either inserted or replaces some existing region in the unsafe code. \concinsertcode itself can be represented as an \astree
%     \item \conceditloc is the concrete edit location i.e. where the edit takes place in a \prog
%     \item \concpath is a concrete path as a sequence of edges
%     \item \conceditpath is a concrete edit path from source to \conceditloc
%     \item semLoc or semantic location is the stopping node of \conceditpath
%     \item \concrefpath is a concrete reference path from semantic location to a particular node matching value of a \astree node in \concinsertcode
%     \item edit-type refers to whether edit is \insertsc or \replace
% \end{itemize}}

% \spsays{Consider breaking these into subsections and have a running example. for instance, first section can be running bidirectional bfs, other could be pairwise merging, and so on..} 

% Our top-level \newtextsc{Learn} method receives the programs and concrete edits as inputs (Line~\ref{algo:line:learn}). This method first invokes the \newtextsc{PreProcessConcEdit} method which computes edit-paths and reference-paths in the $\concedit$. Next, \newtextsc{Learn} method ranks edits based on the similarity of their $\concinsertcode$ and in that order tries to combine edits together in a pairwise manner using the \newtextsc{MergeEdit} method.

% \newtextsc{PreProcessConcEdit} method (in Line~\ref{algo:line:preprocess}) stores the relevant paths in the $\concedit$ structure that will be useful during pairwise-merging. It first computes a set of edit paths from source to sink using a bi-directional breadth-first search (\newtextsc{BiDirectBFS}) (Line~\ref{algo:line:conceditloc}) and stores it in the edit. Note that during this \newtextsc{BiDirectBFS}, it traverses only semantic edges from the source and only syntactic edges from the edit location. This naturally leads to paths that follow the required pattern of a set of semantic edges followed by a set of syntactic edges. Note that edit-paths also contain the semantic-locations in the semLoc field. Then for every edit-node in the edit code and every semantic location in the edit path, it stores reference paths from semantic locations to nodes in the \astree having the same value as edit-node (Line~\ref{algo:line:concrefpath}. These paths are computed using a \newtextsc{MaxLevelBFS} until a certain depth and storing the satisfying nodes. Note that during implementation, we memoize the path-finding steps to avoid repeating computations.

% \newtextsc{PairSimilar} method computes a score for every pair of edits in the edit set. To compute the similarity for a given pair of edits it performs \newtextsc{ASTSimilarity} on their editcodes. 

% \newtextsc{MergeEdit} is the top-level method of our deductive top-down synthesis algorithm. The merging procedure is intuitive. For every element in the edit, it recursively calls \newtextsc{Merge} operation on the elements and then assembles an edit using their outputs. Specifically, this method first ensures that edits are of the same type (\insertsc or \replace. Then it obtains a set of candidate edit-paths by calling the \newtextsc{MergeEditPaths} method on the concrete edit paths stored in the input concrete edits. Next for every candidate edit-path, it finds a candidate editcode using the \newtextsc{MergeEditCode} method. Finally, for every editpath and editcode pair, it assembles the final edit (Line~\ref{algo:line:assembleedit})
    
% \newtextsc{MergeEditPath} method tries to merge two concrete editpaths. It first computes a set of intersecting predicates over the semantic-locations of the two paths. Using the predicates, it builds the \semkleene edge. Finally, over the remaining set of edges in the edit-path, it calls the \newtextsc{MergePath} method which inturn merges all the edges successively (Line~\ref{algo:line:mergepath})


% \subsection{Applying the learned strategies}
% \label{subsec:applying}
% Given an unsafe-program and a set of repair strategies, we apply each strategy to the program to generate candidate repairs. 
% To apply a single repair strategy, we use the definitions of operators described in Section~\ref{subsec:dsl} to generate candidate repair programs. We find that, in practice, we obtain a few distinct repairs and we return them as the output of our system. 

%Once these high level repair strategies are learnt, applying them is natural. For an unsafe program, strategy \strategy ingests the \pdg of the program. Then it tries to build an edit, by first locating the edit location (using the edit path \editpath) and building the \astree recursively depending on whether it is a constant or reference tree. If edit location and \astree are generated, the edit operation is applied appropriately based on whether it is an \insertsc or \replace edit. Otherwise, if either of location or \astree is not generated then no edit is applied. 
% \subsection{Scribblings}
% Each node $n$ of the AST has an identifier $\mathit{id}\in\mathbb{N}$. The AST is characterized by a set $\mathcal{N}$ of node $\mathit{id}$s, i.e., $\mathcal{N}=\{\mathit{id}_0,\ldots\mathit{id}_k\}$. We have a map $\mathcal{T}$ from nodes to their types, i.e., $\mathcal{T}(n)=\tau$, where the primitive types $\tau$ include {\sc MethodCallExpr}, {\sc IndexExpr}, etc. We also have a set $\mathcal{E}$ of edges, where each edge is $(n_1,n_2,ET,z)$. Here, $n_1$ is a source node, $n_2$ is a target node, $ET$ is the type of edge (syntactic parent, syntactic child, semantic parent, or semantic child), and $z$ is a child's index (set to $-1$ if the edge is a parent edge). 

% The strategy $S$ is of two types, insert an AST $O$ at index $I$ of location $L$, $\mathit{Insert}(L(\mathit{source}),I,O)$, and replace the AST at index $I$ of location $L$ with $O$, $\mathit{Replace}(L(\mathit{source}),I,O)$. Each $L(n)$ takes a node $n$ and traverses a path to reach a location, i.e.,  $L(n)=\mathit{ApplyPath}(n,F_k\circ\ldots\circ F_0)$, where the output node is $F_k(F_{k-1}(\ldots F_0(n)\ldots)$. Each $F$ instantiates the edge traversal function $TE[I,ET,C]$ with an index $I$, an edge type $ET$, and an optional clause $C$ (relevant for KleeneEdge). We define $TE[I,ET](n)$ as $let\ i=I(n)\ in\ let\ N=\mathcal{E}(n,ET)\ in\ N[i]$, which gets an integer index $i$ of children of $n$, gets a set $N$ of nodes by dereferencing edges of type $ET$ from $n$ and returns the $i^{th}$ child of $n$.
% $F$ can also be $TE[ET,C](n)\equiv KE(n,ET,C)$.
%  The KleeneEdge $KE$ keeps dereferencing edges of type $ET$ till it hits a node where a clause $C$ holds, i.e., $KE(n,ET,C)$ is defined as $C(n)? n : \left(let\ t=\mathcal{E}(n,ET)\ in\ KE(t,ET,C)\right)$. The node $t$ which is target of an edge with type $ET$ and $n$ as a source here is chosen non-deterministically and our implementation resolves this non-determinism through a breadth-first search. A clause is a conjunction of predicates of the form $\lambda n. \mathcal{T}(L(n))=\tau$. The index $I$ is either an integer $z$ or of the form $\lambda n.DO(n,L(\mathit{source}))+z$, where $DO(n_1,n_2)$ returns the index of syntactic child of $n_2$ who is a syntactic ancestor of $n_1$. 
 
%  A strategy can fail to apply if in $DO(n_1,n_2)$ there is no path from $n_2$ to $n_1$,
%%%%%%%%%%%%%%%%%%%%%%%%%%%%%%%%%%%%%%%%%%%%%%%%%%%%%%%%%%%%%%%
%%%%%%%%%%%%%%%%%%%%%%%%%%%%%%%%%%%%%%%%%%%%%%%%%%%%%%%%%%%%%%%
%%%%%%%%%%%%%%%%%%%%%%%%%%%%%%%%%%%%%%%%%%%%%%%%%%%%%%%%%%%%%%%
\section{Experiments}

We compare {\mname} against state-of-the-art approaches on the FlyingThings3D~\cite{flythings} and KITTI~\cite{kitti} datasets. 
We design two experimental settings to mimic corrupted RGB images and poor lighting condition scenarios.
We also evaluate on data we acquired with a RGBD sensor in various lighting conditions.
We report both quantitative and qualitative results, and carry out ablation studies.





%%%%%%%%%%%%%%%%%%%%%%%%%%%%%%%%%%%%%%%%%%%%%%%%%%%%%%%%%%%%%%%
%%%%%%%%%%%%%%%%%%%%%%%%%%%%%%%%%%%%%%%%%%%%%%%%%%%%%%%%%%%%%%%
\subsection{Experimental setup}



%%%%%%%%%%%%%%%%%%%%%%%%%%%%%%%%%%%%%%%%%%%%%%%%%%%%%%%%%%%%%%%
\noindent \textbf{Datasets.}
FlyingThings3D~\cite{flythings} is split into \textit{clean} and \textit{final} sets containing dynamic synthetic scenes.
The former is composed of 27K RGBD images including changing lighting and shading effects, while the latter is an augmented version of the former with simulated challenging motions and blurs.
Each set contains train and test splits. 
Previous methods~\cite{Raft,Raft-3D,CamLiFlow} exclude samples containing fast-moving objects during the evaluation. However, as such visual challenges is of interests to our problem, we use the \textit{whole} training set of FlyingThings3D and sample 1K RGBD image pairs from the \textit{whole} test set for the evaluation.
KITTI consists of real-world scenes captured from vehicles in urban scenarios.
Because the original dataset does not provide depth data, we use the disparity estimated by GA-Net~\cite{Ga-net} as in~\cite{Raft-3D}.
We exploit KITTI to assess the ability of our model and the compared ones in generalizing from synthetic to real data, without training or finetuning using any of the KITTI’s sequences. 
We use the training set of KITTI as our evaluation set since KITTI's test set is not publicly available.
To further validate the performance of {\mname} in real-world scenarios, we collect an RGBD dataset using a Realsense D415 camera in an indoor office with moving people under three lighting setups, named Bright, Dimmed, and Dark.
The Bright setting features bright lighting, where the moving objects are clearly visible.
The Dimmed setting features dimmed lighting, where the moving objects can be observed with a lower visual quality.
The Dark setting features very low lighting where the moving objects can be barely seen.
We only qualitatively evaluate this dataset because we could not produce optical flow ground truth. 






%%%%%%%%%%%%%%%%%%%%%%%%%%%%%%%%%%%%%%%%%%%%%%%%%%%%%%%%%%%%%%%
\noindent \textbf{Evaluation metrics.}
We quantify the optical and scene flow results using conventional evaluation metrics \cite{Raft, CRaft, Raft-3D}:
for the optical flow we use $\rm AEPE_{2D}$(pixel), $\rm {ACC}_{1px}$(\%) and $\rm Fl^{all}_{2D}$(\%), 
for the scene flow we use $\rm AEPE_{3D}$(m), $\rm {ACC}_{0.05m}$(\%), $\rm {ACC}_{0.10m}$(\%) and $\rm Fl^{all}_{3D}$(\%).
$\rm AEPE_{2D}$ measures the average end-point error (EPE)~\cite{Raft}, which is an average value of all the 2D flow errors.
$\rm AEPE^{epe{\textless{100}}}_{2D}$ measures the average end-point error (EPE) among the 2D flow errors that are less than 100 pixels. 
$\rm AEPE_{3D}$ is the average of euclidean distance (EPE for 3D) between the ground-truth 3D scene flow and the predicted results.
$\rm AEPE^{epe{\textless{1}}}_{3D}$ measures the average end-point error (EPE) among the 3D flow errors that are less than 1 meter.
$\rm {ACC}_{1px}$ \cite{Raft-3D} measures the portion of errors that are within a threshold of one pixel.
$\rm {ACC}_{0.05m}$ \cite{Raft-3D} measures the portion of errors that are within a threshold of 0.05 meters, while $\rm {ACC}_{0.10m}$ \cite{Raft-3D} measures the portion of errors that are within a threshold of 0.10 meters.
$\rm MEAN_{AEPE}$ and $\rm MEAN_{ACC}$ are the average values of $\rm AEPE^{all}_{2D}$ and $\rm ACC_{1px}$, respectively, calculated over FlyingThings3D-clean and FlyingThings3D-final.
$\rm Fl^{all}_{2D}$ \cite{CRaft} is the percentage of outlier pixels whose end-point error is $>3$ pixels or $5\%$ of the ground-truth flow magnitude.
$\rm Fl^{all}_{3D}$~\cite{Flownet3d++} is the percentage of outlier pixels whose 3D Euclidean distance between the ground-truth 3D scene flow and the predicted one is $>0.3$ m or $5\%$ of the ground-truth flow magnitude.




%%%%%%%%%%%%%%%%%%%%%%%%%%%%%%%%%%%%%%%%%%%%%%%%%%%%%%%%%%%%%%%
\noindent \textbf{Evaluation settings.}
Environments with poor light conditions lead to weak texture information that can compromise the stability of feature representation.
Also additive Gaussian noises can affect optical and scene flow estimation.
To assess the robustness, we design three experimental settings on the public FlyingThings3D and KITTI datasets:
\emph{Standard}: we use the original version of the dataset;
\emph{AGN}: we apply Additive Gaussian Noise on RGB images;
\emph{Dark}: we darken RGB images.
In AGN we randomly sample noise values ($\alpha$) from a normal distribution centered in zero with a standard deviation equal to 35.
In Dark we divide pixel values by a random factor $\beta  \sim \mbox{U}(\{ 1,2, \cdots,9\} )$.








%%%%%%%%%%%%%%%%%%%%%%%%%%%%%%%%%%%%%%%%%%%%%%%%%%%%%%%%%%%%%%%
\noindent \textbf{Implementation details.}
We implemented \mname{} in PyTorch with all modules initialized with random weights.
We train our network for 100K iterations with the batch size of 6 on 3 Nvidia 3090 GPUs.
During training, we set the initial learning rate at $1.25\cdot10^{-4}$ and use linear decay. 
We apply MMTM sequentially with N {=} 3 times as suggested in the original paper~\cite{MMTM}.
We set $\gamma {=} 0.8$ in Eq.~\eqref{eq:loss} as in RAFT~\cite{Raft}.






%%%%%%%%%%%%%%%%%%%%%%%%%%%%%%%%%%%%%%%%%%%%%%%%%%%%%%%%%%%%%%%
%%%%%%%%%%%%%%%%%%%%%%%%%%%%%%%%%%%%%%%%%%%%%%%%%%%%%%%%%%%%%%%
\subsection{Comparisons}\label{sec:exp:comparisons}
%
We compare \mname{} against RGB methods for 2D optical flow estimation, i.e.~RAFT~\cite{Raft}, GMA~\cite{GMA}, CRAFT~\cite{CRaft}, and Separable flow~\cite{Separable_flow}, and against methods for 3D scene flow estimation, i.e.~RAFT-3D~\cite{Raft-3D} and CamLiRAFT~\cite{camliraft2023}.
See Sec.~\ref{sec:related_work} for the description of these methods.




\begin{table*}[t]
    % \tabcolsep 3pt
    \centering
    \caption{
    Optical flow estimation in the Standard, AGN, and Dark settings on FlyingThings3D-clean, FlyingThings3D-final, and KITTI. 
    All models are trained with FlyingThings3D, without fine-tuning on KITTI.
    Bold font indicates the best-performing method.}
    \label{tab:setting1_results}
    \vspace{-.2cm}
    \resizebox{\linewidth}{!}{%
    \begin{tabular}{clc|ccc|ccc|cc}
        \toprule
        & \multirow{2}{*}{Method} & \multirow{2}{*}{Input} & \multicolumn{3}{c|}{FlyingThings3D-clean}  &  \multicolumn{3}{c|}{FlyingThings3D-final}&  \multicolumn{2}{c}{KITTI-Train} \\
        & & & $\rm ACC_{1px}$&  $\rm AEPE^{epe{\textless{100}}}_{2D}$& $\rm AEPE^{all}_{2D}$ &$\rm ACC_{1px}$&  $\rm AEPE^{epe{\textless{100}}}_{2D}$& $\rm AEPE^{all}_{2D}$ &$\rm AEPE^{all}_{2D}$&$\rm Fl^{all}_{2D}$\\
        % \texttt{\footnotesize FlowNet2.0 \cite{Flow_Net2.0}} & RGB &  &  &  &  \\
        % \texttt{\footnotesize PWC-Net \cite{PWC-Net}} &RGB &  &  & &  \\
        % \multicolumn{10}{c}{Standard setting}\\
        \midrule
        \multirow{8}{*}{\rotatebox[origin=c]{90}{Standard setting}} & \texttt{\footnotesize RAFT} \cite{Raft} &RGB & 77.06 &2.65 & 4.69 & 76.91&2.67 &4.39&6.76& 20.99 \\
        & \texttt{\footnotesize GMA} \cite{GMA} &RGB & 78.81& 2.58& 4.43 & 78.66& 2.57 &4.20& 6.10& 20.47 \\
        & \texttt{\footnotesize Separable flow} \cite{Separable_flow} & RGB & 75.39&2.88 & 4.57 &75.29 &2.84 &4.29&6.40& 20.66\\
        & \texttt{\footnotesize CRAFT} \cite{CRaft}& RGB & 77.90&2.79 & 4.85 & 77.70& 2.77& 4.66&6.82&21.95 \\
        & \texttt{\footnotesize {\mname}-2D} & RGBD & \textbf{80.37}&\textbf{2.17} & \textbf{3.52} & \textbf{80.21}&\textbf{2.22}  & \textbf{3.42}&\textbf{5.49}&\textbf{18.05}\\
        \cmidrule{2-11}
        % \texttt{\footnotesize CamLiFlow \cite{CamLiFlow}} & RGB+PCD &  &  &  &  \\
        & \texttt{\footnotesize RAFT-3D} \cite{Raft-3D}& RGBD & 86.01& 1.79& 3.58&85.97& 1.76& 3.57&5.91&17.80\\
        & \texttt{\footnotesize CamLiRAFT} \cite{camliraft2023}& RGB+LiDAR & 83.59& 1.81& 3.03&83.26& 1.80& 2.84& 4.84&14.76 \\
        % \texttt{\footnotesize CamLiFlow \cite{CamLiFlow}} & RGB+PCD &  &  &  & \\
        & \texttt{\footnotesize {\mname}-3D} & RGBD & \textbf{87.45}&\textbf{1.57} & \textbf{2.58} & \textbf{87.39}&\textbf{1.58} & \textbf{2.69}&\textbf{4.70}&\textbf{12.36}\\
        % \multicolumn{10}{c}{AGN setting}\\
        \midrule
        % \texttt{\footnotesize FlowNet2.0 \cite{Flow_Net2.0}} & RGB &  &  &  &  \\
        % \texttt{\footnotesize PWC-Net \cite{PWC-Net}} &RGB &  &  & &  \\
        \multirow{8}{*}{\rotatebox[origin=c]{90}{AGN setting}} & \texttt{\footnotesize RAFT} \cite{Raft} &RGB & 71.42&2.98 & 4.89 & 71.01&2.96 & 4.64& 7.23&23.45 \\
        & \texttt{\footnotesize GMA} \cite{GMA} &RGB & 72.63& 2.99& 5.23 & 72.20& 2.94& 5.24&7.01& 23.52 \\
        & \texttt{\footnotesize Separable flow} \cite{Separable_flow} & RGB & 68.95&3.15  & 5.32 &68.57 &3.18 & 5.23&8.26&25.79 \\
        & \texttt{\footnotesize CRAFT} \cite{CRaft} & RGB & 73.30&2.86 & 4.65 & 72.81& 2.88& 4.67& 7.45&23.65 \\
        & \texttt{\footnotesize {\mname}-2D} & RGBD & \textbf{77.24}&\textbf{2.24} &\textbf{3.50}  & \textbf{76.77} &\textbf{2.30}  & \textbf{3.38}&\textbf{5.47}&\textbf{19.15}\\
        \cmidrule{2-11}
        % \texttt{\footnotesize DeepLiDARFlow \cite{DeepLiDARFlow}} & RGB+PCD &  &  &  &  \\
        & \texttt{\footnotesize RAFT-3D} \cite{Raft-3D} & RGBD & 84.59 & 1.79 & 3.14&84.26  & 1.84 &3.21&5.50&17.94\\
        & \texttt{\footnotesize CamLiRAFT} \cite{camliraft2023} & RGB+LiDAR & 76.98 & 2.23 & 3.98&76.31 & 2.33 &3.71&5.26&16.98\\
        % \texttt{\footnotesize CamLiFlow \cite{CamLiFlow}} & RGB+PCD &  &  &  & \\
        & \texttt{\footnotesize {\mname}-3D} & RGBD &\textbf{86.75} &\textbf{1.55}&\textbf{2.63} &\textbf{86.71} & \textbf{1.53}  &\textbf{2.60}&\textbf{4.53}&\textbf{11.57} \\
        % \multicolumn{10}{c}{Dark setting}\\
        \midrule
        \multirow{8}{*}{\rotatebox[origin=c]{90}{Dark setting}} & \texttt{\footnotesize RAFT} \cite{Raft} &RGB & 60.26&4.00 & 8.15 & 60.36& 4.01& 7.85& 11.75&31.81 \\
        & \texttt{\footnotesize GMA} \cite{GMA} &RGB &63.36&4.50 & 9.96 & 62.10&4.62 & 10.34&9.65&27.87 \\
        & \texttt{\footnotesize Separable flow} \cite{Separable_flow}& RGB & 68.20& 4.84& 7.96 &68.03 &4.86 & 7.79&10.09&28.41 \\
        & \texttt{\footnotesize CRAFT} \cite{CRaft} & RGB & 70.07&4.84 & 8.46 & 69.77& 4.87&8.44& 11.10&29.47  \\
        & \texttt{\footnotesize {\mname}-2D} & RGBD & \textbf{76.70}&\textbf{2.39} & \textbf{3.65} & \textbf{76.57}& \textbf{2.38} &\textbf{3.66}&\textbf{8.29}&\textbf{23.87} \\
        \cmidrule{2-11}
        & \texttt{\footnotesize RAFT-3D} \cite{Raft-3D} & RGBD & 81.03& 2.20&3.78 & 80.96&2.20  &3.56&15.14&32.08 \\
        & \texttt{\footnotesize CamLiRAFT} \cite{camliraft2023} & RGB+LiDAR & 74.80& 2.54&4.54 &74.73&2.64  &4.11&7.44 & \textbf{16.97}\\
        & \texttt{\footnotesize {\mname}-3D} & RGBD & \textbf{87.11}&\textbf{1.55} & \textbf{2.91} & \textbf{87.03}& \textbf{1.58} &\textbf{2.84}&\textbf{7.26}&20.07 \\
        \bottomrule 
    \end{tabular}
    }
\end{table*}
\begin{table*}[!h]
    % \tabcolsep 3pt
    \centering
    \caption{
    Scene flow estimation in the Standard, AGN, and Dark settings on FlyingThings3D-clean, FlyingThings3D-final, and KITTI.
    All models are trained with FlyingThings3D, without fine-tuning on KITTI.
    Bold font indicates the best-performing method.}
    \vspace{-.2cm}
    \label{tab:4}  
    \resizebox{\linewidth}{!}{%
    \begin{tabular}{lc|cccc|cccc|cc}
        \toprule
        \multirow{2}{*}{Method} & \multirow{2}{*}{Setting} & \multicolumn{4}{c|}{FlyingThings3D-clean}  &  \multicolumn{4}{c|}{FlyingThings3D-final}&  \multicolumn{2}{c}{KITTI-Train}  \\
         & & $\rm ACC_{0.05m}$& $\rm ACC_{0.10m}$&  $\rm AEPE^{epe{\textless{1}}}_{3D}$& $\rm AEPE^{all}_{3D}$& $\rm ACC_{0.05m}$ &$\rm ACC_{0.10m}$& $\rm AEPE^{epe{\textless{1}}}_{3D}$& $\rm AEPE^{all}_{3D}$&$\rm AEPE^{all}_{3D}$&$\rm Fl^{all}_{3D}$\\
        \midrule
        \texttt{\footnotesize RAFT-3D} \cite{Raft-3D}& Standard & 74.01& 81.22&0.064 & 0.186&74.25  &81.43&0.064&0.180&0.136 &5.20 \\
        \texttt{\footnotesize CamLiRAFT} \cite{camliraft2023} & Standard & 76.83& \textbf{87.98}&\textbf{0.049} & 0.104&\textbf{76.87} &\textbf{88.20}&\textbf{0.049}&0.102& \textbf{0.121}& 7.13\\
        \texttt{\footnotesize {\mname}-3D} & Standard & \textbf{77.04}&83.74 & 0.056 & \textbf{0.100}& 76.80 &83.58&0.057&\textbf{0.101}&0.134&\textbf{4.90} \\
        \midrule
        \texttt{\footnotesize RAFT-3D}\cite{Raft-3D} & AGN & 75.05& 81.77&0.065 & 0.193&74.75  &81.56&0.066&0.144&0.134&5.36\\
        
        \texttt{\footnotesize CamLiRAFT} \cite{camliraft2023}& AGN & 73.83& \textbf{86.59}&\textbf{0.055} & 0.116&73.38 &\textbf{86.31}&\textbf{0.055}&0.126& \textbf{0.122}& 7.81\\
        
        \texttt{\footnotesize {\mname}-3D} & AGN & \textbf{76.60}&82.71 & 0.061 & \textbf{0.104}& \textbf{76.35} &82.55&0.062&\textbf{0.107}&0.134&\textbf{5.52} \\
        \midrule
        \texttt{\footnotesize RAFT-3D}\cite{Raft-3D} & Dark & 71.21& 79.33&0.071& 0.203&71.00  &79.24&0.072&0.145&0.145&9.39 \\
        \texttt{\footnotesize CamLiRAFT}\cite{camliraft2023} & Dark & 66.70& 82.24&0.068& \textbf{0.141}&66.65  &82.01&0.069&0.127&0.171 &9.39 \\
        \texttt{\footnotesize {\mname}-3D} & Dark & \textbf{76.72}&\textbf{83.62} & \textbf{0.057} & 0.175& \textbf{76.58} &\textbf{83.51}&\textbf{0.057}&\textbf{0.112}&\textbf{0.136}&\textbf{6.20} \\
        \bottomrule 
    \end{tabular}
    }
\end{table*}




%%%%%%%%%%%%%%%%%%%%%%%%%%%%%%%%%%%%%%%%%%%%%%%%%%%
\subsubsection{Quantitative results}



Tab.~\ref{tab:setting1_results} (top) reports optical flow results in Standard setting.
{\mname}-2D outperforms GMA by $+1.56\%$ and $+1.55\%$ in terms of $\rm ACC_{1px}$, and $+0.91$ and $+0.78$ in terms of $\rm AEPE^{all}_{2D}$ in FlyingThings3D-clean and FlyingThings3D-final, respectively.
{\mname}-3D outperforms RAFT-3D by $+1.44\%$ and $+1.42\%$ in terms of $\rm ACC_{1px}$, and $+1.00$ and $+0.88$ in terms of $\rm AEPE^{all}_{2D}$. 
While RAFT-3D extracts features only from RGB images, our MFF encoder extracts features from both RGB and depth, producing more informative internal representations.
{\mname}-3D outperforms CamLiRAFT by $+3.86\%$ and $+4.13\%$ in terms of $\rm ACC_{1px}$, and $+0.45$ and $+0.15$ in terms of $\rm AEPE^{all}_{2D}$.

\input{figures/figure_of_comparison_ft}
\input{figures/figure_of_comparison_kitti}
\input{figures/figure_of_comparison_ours}

Tab.~\ref{tab:setting1_results} (middle) reports optical flow results in AGN setting.
{\mname}-2D outperforms CRAFT by $+3.94\%$ and $+3.96\%$ in terms of $\rm ACC_{1px}$, and $+1.15$ and $+1.29$ in terms of $\rm AEPE^{all}_{2D}$ in FlyingThings3D-clean and FlyingThings3D-final, respectively.
{\mname}-3D outperforms RAFT-3D by $+2.16\%$ and $+2.45\%$ in in terms of $\rm ACC_{1px}$ and $+0.51$ and $+0.61$ in terms of $\rm AEPE^{all}_{2D}$. 
$\rm AEPE^{all}_{2D}$ of RAFT-3D is lower than that of the Standard setting.
This is because $\rm AEPE^{all}_{2D}$ computes the average of all errors, and the average is known to be sensitive to outliers. 
In fact by computing the median (less sensitive to outliers), the performance of RAFT-3D in AGN setting is worse than the Standard setting: 
e.g.~RAFT-3D in FlyingThings3D-clean achieves a median $\rm EPE^{all}_{2D}$ of 0.127 and 0.130 in the Standard and AGN settings, respectively.


Tab.~\ref{tab:setting1_results} (bottom) reports optical flow results in Dark setting.
RGB methods perform worse than the previous settings, whereas our {\mname} methods outperform RGB methods, RAFT-3D, and CamLiRAFT.
Tab.~\ref{tab:setting1_results} also reports $\rm AEPE^{all}_{2D}$ and $\rm Fl^{all}_{2D}$ on KITTI without fine-tuning the models.
Although CamLiRAFT demonstrates a better generalization capability than RAFT-3D, {\mname} outperforms almost all the other methods in all three settings, demonstrating its robustness and adaptability in real-world scenarios.




Tab.~\ref{tab:4} reports scene flow results.
{\mname}-3D outperforms RAFT-3D on both FlyingThings3D-clean and FlyingThings3D-final.
CamLiRAFT scores on par with {\mname}-3D in all three settings.
{\mname}-3D performs better in terms of $\rm ACC_{0.05m}$, while CamLiRAFT performs better in terms of $\rm ACC_{0.10m}$. 
This suggests that {\mname}-3D produces more small flow errors than CamLiRAFT.
In general, {\mname}-3D performs stably across all three settings, while the performance of RAFT-3D and CamLiRAFT degrades in the AGN and Dark settings.
On KITTI, {\mname}-3D is the best-performing method on all three settings in terms of $\rm Fl^{all}_{3D}$. 
In terms of $\rm AEPE_{3D}^{all}$, CamLiRAFT performs better in the Standard and AGN settings, while {\mname}-3D scores the best in the Dark setting.






%%%%%%%%%%%%%%%%%%%%%%%%%%%%%%%%%%%%%%%%%%%%%%%%%%%%%%%%%%%%%%%
%%%%%%%%%%%%%%%%%%%%%%%%%%%%%%%%%%%%%%%%%%%%%%%%%%%%%%%%%%%%%%%
\subsubsection{Qualitative results}

We provide examples of qualitative optical flow results indicated with their corresponding $\rm AEPE^{all}_{2D}$.
We visualize the errors with respect to the ground-truth: the stronger the magenta, the higher the error.
Fig.~\ref{fig:FlyingThings3D} and Fig.~\ref{fig:KITTI} show the results of optical flow errors on FlyingThings3D and KITTI, respectively.
Both {\mname}-2D and {\mname}-3D consistently produce smaller $\rm AEPE^{all}_{2D}$ values than the other methods, which can also be visually verified with less magenta areas produced by our models.
Fig.~\ref{fig:realtest} shows the flow estimation on our acquired indoor dataset with RAFT, GMA, RAFT-3D, CamLiRAFT, and {\mname}.
In the Bright setting (top), all compared methods produce good-quality results.
In the Dimmed setting (middle), RAFT, GMA, and CamLiRAFT show low-quality results, which we can observe from the poor edges produced by the moving objects.
In the Dark setting, {\mname} is the only method that produces results where the moving objects are distinguishable.

%%%%%%%%%%%%%%%%%%%%%%%%%%%%%%%%%%%%%%%%%%%%%%%%%%%%%%%%%%%%%%%
%%%%%%%%%%%%%%%%%%%%%%%%%%%%%%%%%%%%%%%%%%%%%%%%%%%%%%%%%%%%%%%
\subsection{Ablation study}
Tab.~\ref{tab:ablation_setting1} reports the ablation study on self-attention (SA), cross-attention (CA), and Multimodal Transfer Module (MMTM) on the FlyingThings3D dataset in both Standard and Dark settings. 
Overall, we can observe that all the components we added provide an incremental contribution to improve the quality of the output optical flow compared to the RGB baseline.
SA and CA consistently improve performance (see Exp 3 vs 6 vs 8, 4 vs 7 vs 9, and similarly for the Dark setting).
The SA applied to both depth and RGB is better than applying it to the RGB branch only (see Exp 5 vs 6 for the Standard setting, and 14 vs 15 for the Dark setting).
MMTM fusion consistently outperforms the simple concatenation of RGB and depth branches in the Dark setting (see Exp 12 vs 13, 15 vs 16, 17 vs 18).
There is one case in the Standard setting where this last does not occur (see Exp 6 vs 7).
In general, SA focuses on intra-modality relationships while CA focuses on inter-modality relationships. MMTM further exchanges information across modalities at a deeper level. The best performance is achieved when all the modules are activated.

\renewcommand{\arraystretch}{.9}
%+++++++++++++++++++++++++
\begin{table*}[!t]
    \centering
    \caption{Ablation study in Standard and Dark settings on FlyingThings3D.
    $\rm MEAN_{ACC}$ and $\rm MEAN_{AEPE}$ are the mean of $\rm ACC_{1px}$ and $\rm AEPE^{all}_{2D}$, respectively, on FlyingThings3D-clean and FlyingThings3D-final. \checkmark(RGB): self-attention is performed on the RGB.}
    \vspace{-.2cm}
    \label{tab:ablation_setting1}
    \resizebox{1\linewidth}{!}{%
    \begin{tabular}{cllcccccccccc}
        \toprule
        & \multirow{2}{*}{Exp} & \multirow{2}{*}{RGB} & \multirow{2}{*}{depth}  &  \multirow{2}{*}{SA}&  \multirow{2}{*}{CA} &  \multirow{2}{*}{Fusion} &\multicolumn{2}{c}{FlyingThings3D-clean}  &  \multicolumn{2}{c}{FlyingThings3D-final}&\multirow{2}{*}{$\rm MEAN_{ACC}$}&\multirow{2}{*}{$\rm MEAN_{AEPE}$} \\
        & & & & & & & $\rm ACC_{1px}$& $\rm AEPE^{all}_{2D}$&$\rm ACC_{1px}$&$\rm AEPE^{all}_{2D}$\\
        \midrule
        \multirow{9}{*}{\rotatebox[origin=c]{90}{Standard setting}} & 1 & \texttt \checkmark& - & - & -& - & 77.06 &4.69& 76.91&4.39&76.99 &4.54\\
        & 2 & \texttt \checkmark& - & \checkmark & - & - &77.81& 4.50 & 77.72 &4.26&77.77&4.38\\
        & 3 & \texttt \checkmark&\checkmark & - & -& concat &79.21 & 3.74&79.06 &3.71&79.14 &3.73 \\
        & 4 & \texttt \checkmark&\checkmark & - & -& MMTM &79.43& 3.79&79.27 &3.69& 79.35&3.74 \\
        & 5 & \texttt \checkmark&  \checkmark& \checkmark(RGB) & - & concat & 79.35 & 3.72 & 79.18 & 3.66& 79.27&3.69\\
        & 6 & \texttt \checkmark&  \checkmark& \checkmark & - & concat & 79.71 & 3.81 & 79.55 & 3.63& 79.63&3.72\\
        & 7 & \texttt \checkmark&  \checkmark& \checkmark & - & MMTM & 78.97 & 3.73 & 78.77 & 3.65& 78.87&3.69 \\
        & 8 & \texttt \checkmark&  \checkmark & \checkmark &  \checkmark & concat & 80.14 & 3.56 & 79.95& 3.53&80.05&3.55\\
        & 9 & \texttt \checkmark&  \checkmark & \checkmark & \checkmark & MMTM & 80.37  & 3.52 &80.21&3.42&80.29&3.47 \\
        \midrule
        \multirow{9}{*}{\rotatebox[origin=c]{90}{Dark setting}} & 10 & \texttt \checkmark& - & - & -& - & 60.26 & 8.15& 60.36& 7.85&60.31&8.00\\
        & 11 & \texttt \checkmark& - & \checkmark & - & - &67.92& 8.01 & 67.80 &7.77&67.86&7.89\\
        & 12 & \texttt \checkmark&\checkmark & - & -& concat &75.33 & 4.06&75.17 &4.05&75.25&4.06 \\
        & 13 & \texttt \checkmark&\checkmark & - & -& MMTM &75.56 & 3.97&75.40&3.99&75.48&3.98 \\
        & 14 & \texttt \checkmark&\checkmark & \checkmark(RGB) & -& concat &75.47& 3.95&75.32&3.94&75.40&3.95 \\
        & 15 & \texttt \checkmark&  \checkmark& \checkmark & - & concat & 75.69 & 3.96 & 75.57 & 3.90&75.63&3.93\\
        & 16 & \texttt \checkmark&  \checkmark & \checkmark & - & MMTM & 75.75 & 3.81 & 75.60& 3.76&75.68&3.79\\
        & 17 & \texttt \checkmark&  \checkmark & \checkmark & \checkmark & concat &76.43  & 3.79 &76.26 &3.72&76.35&3.76 \\
        & 18 & \texttt \checkmark&  \checkmark & \checkmark & \checkmark & MMTM & 76.70  & 3.65 &76.57 &3.66&76.64&3.66 \\
        \bottomrule 
    \end{tabular}
    }
\end{table*}
\renewcommand{\arraystretch}{1}



%%%%%%%%%%%%%%%%%%%%%%%%%%%%%%%%%%%%%%%%%%%%%%%%%%%%%%%%%%%%%%%
\subsection{Computation analysis}
We measure the number of parameters, Floating-Point Operations (FLOPs), and inference time of all compared methods using FlyingThings3D.
We conducted the experiments with a Nvidia 3090 GPU (24G) and I9-10900 CPUs, and reported the results in Tab.~\ref{tab:cost}.
Despite {\mname}-2D has the second-largest number of parameters, its number of FLOPs and inference time are in-between the other methods for optical flow estimation. The inference time of {\mname}-3D is slightly higher than that of CamLiRAFT, although our number of parameters is one order of magnitude larger than CamLiRAFT.
From the per-component analysis of {\mname}-2D in Tab.~\ref{tab:costabla}, we can observe that Self-attention and Cross-attention have a higher computational cost than MMTM and the two-branch encoder. The most time-consuming component is \textit{Others} which includes all the other modules to compute the optical flow.

\renewcommand{\arraystretch}{.9}
\begin{table}[t]
    \centering
    \caption{Comparative computational analysis by using 960$\times$540-sized images on a Nvidia 3090.}
    \vspace{-.3cm}
    \label{tab:cost}  
    \resizebox{1\linewidth}{!}{%
    \begin{tabular}{lccc}
        \toprule
        Models & Params [M] & FLOPs [T] & Inference time [ms] \\
        \midrule
        RAFT \cite{Raft} & 5.31 & 0.78 & 99\\
        GMA \cite{GMA} & 5.88 & 0.59 & 87 \\
        Separable flow \cite{Separable_flow} & 8.35 & 0.50 & 639 \\
        CRAFT \cite{CRaft} & 6.31 & 0.99 & 302 \\
        {\mname}-2D& 8.13 & 0.68 & 219\\
        \midrule
        RAFT-3D \cite{Raft-3D} & 44.50 & 0.51 & 277 \\
        CamLiRAFT \cite{camliraft2023} & 8.41 & 0.67 & 312 \\
        {\mname}-3D & 86.32 & 0.76 & 398 \\  
        \bottomrule 
    \end{tabular}
    }
\end{table}

\begin{table}[t]
    \centering
    \caption{Ablations of computational performance on {\mname}-2D by using 960$\times$540-sized images on a Nvidia 3090.}
    \vspace{-.3cm}
    \label{tab:costabla}  
    \resizebox{1\linewidth}{!}{%
    \begin{tabular}{lccc}
        \toprule
        Models & Params [M] & FLOPs [T] & Inference time [ms] \\
        \midrule
        Image encoder & 2.33 & 0.11 & 13\\
        Depth encoder & 2.33 & 0.11 & 13\\
        Self-attention & 0.16 & 0.06 & 63 \\
        Cross-attention & 0.09 & 0.04 & 43 \\
        MMTM & 0.24 & 2M &1\\
        Others & 2.98 & 0.36 & 86\\        
        \midrule
        Total & 8.13 & 0.68 & 219\\
        \bottomrule 
    \end{tabular}
    }
\end{table}
\renewcommand{\arraystretch}{1}











%% -*- mode: LaTeX; fill-column: 78; -*-

\section{Concluding Remarks}
\label{sec:conclusions}

In this paper, we presented a novel SMC algorithm, \EventDPOR, tailored to the
characteristics of event-driven multi-threaded programs running under the SC
semantics. The algorithm was proven correct and optimal for event-driven
programs in which the variable accesses of events do not depend on how their
execution is interleaved with other threads.

We have implemented \EventDPOR in the \Nidhugg tool, and we will open-source
our implementation.
%
With a wide range of event-driven programs, we have shown that \EventDPOR
incurs only a moderate constant overhead over its baseline implementation
(\OptimalDPOR), it is exponentially faster than existing state-of-the-art SMC
algorithms in time and number of traces examined on programs where events'
actions do not conflict, and does not suffer from performance degradation
caused by having to examine
% a significant number of
non-serializable executions.
%
%% \bjcom{Should we include:
%% Moreover, in our benchmarks, also those that are not non-branching,
%% \EventDPOR explores only the optimal number of executions, and never
%% had to resort to a potentially expensive decision procedure.}

\EventDPOR assumes that handlers can process their events in arbitrary order.
Directions for future work include to retarget \EventDPOR for event-driven
programs with other policies (e.g., FIFO), and for specific event-driven
execution models.


% \FloatBarrier

\IEEEpubidadjcol

\bibliographystyle{IEEEtran}
\bibliography{refe}

\end{document}


