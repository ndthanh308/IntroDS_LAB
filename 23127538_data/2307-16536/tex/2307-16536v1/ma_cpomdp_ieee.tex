%% bare_jrnl_comsoc.tex
%% V1.4b
%% 2015/08/26
%% by Michael Shell
%% see http://www.michaelshell.org/
%% for current contact information.
%%
%% This is a skeleton file demonstrating the use of IEEEtran.cls
%% (requires IEEEtran.cls version 1.8b or later) with an IEEE
%% Communications Society journal paper.
%%
%% Support sites:
%% http://www.michaelshell.org/tex/ieeetran/
%% http://www.ctan.org/pkg/ieeetran
%% and
%% http://www.ieee.org/

%%*************************************************************************
%% Legal Notice:
%% This code is offered as-is without any warranty either expressed or
%% implied; without even the implied warranty of MERCHANTABILITY or
%% FITNESS FOR A PARTICULAR PURPOSE! 
%% User assumes all risk.
%% In no event shall the IEEE or any contributor to this code be liable for
%% any damages or losses, including, but not limited to, incidental,
%% consequential, or any other damages, resulting from the use or misuse
%% of any information contained here.
%%
%% All comments are the opinions of their respective authors and are not
%% necessarily endorsed by the IEEE.
%%
%% This work is distributed under the LaTeX Project Public License (LPPL)
%% ( http://www.latex-project.org/ ) version 1.3, and may be freely used,
%% distributed and modified. A copy of the LPPL, version 1.3, is included
%% in the base LaTeX documentation of all distributions of LaTeX released
%% 2003/12/01 or later.
%% Retain all contribution notices and credits.
%% ** Modified files should be clearly indicated as such, including  **
%% ** renaming them and changing author support contact information. **
%%*************************************************************************


% *** Authors should verify (and, if needed, correct) their LaTeX system  ***
% *** with the testflow diagnostic prior to trusting their LaTeX platform ***
% *** with production work. The IEEE's font choices and paper sizes can   ***
% *** trigger bugs that do not appear when using other class files.       ***                          ***
% The testflow support page is at:
% http://www.michaelshell.org/tex/testflow/


%\documentclass[journal,onecolumn]{IEEEtran}
% \documentclass[technote,10.5pt,onecolumn]{IEEEtran}
%\documentclass[journal]{IEEEtran}
%\documentclass[journal,twoside]{IEEEtran}
% \documentclass[journal,11pt,draftclsnofoot,onecolumn]{IEEEtran}
\documentclass[journal,final]{IEEEtran}
%
% If IEEEtran.cls has not been installed into the LaTeX system files,
% manually specify the path to it like:
%\documentclass[journal,comsoc]{IEEEtran}

%---------------------------------------------%
%--------------- MY CODE START ---------------%
%---------------------------------------------%
\usepackage[noadjust]{cite}
\usepackage{amsmath,amssymb,amsfonts}
\usepackage{graphicx}
\usepackage{textcomp}
\usepackage{xcolor}
\usepackage{hyperref}
\usepackage{amsmath}
\usepackage{accents}
\makeatletter
\newcommand{\sbullet}{%		
\hbox{\fontfamily{lmr}\fontsize{.4\dimexpr(\f@size pt)}{0}\selectfont\textbullet}}		
\DeclareRobustCommand{\mathbullet}{\accentset{\sbullet}}

% \usepackage{enumitem}
\interdisplaylinepenalty=2500
% -----SET UP ALGORITHM ENVIRONMENT START ------%
% \usepackage{algorithm,algorithmic}
% \renewcommand{\algorithmicrequire}{\textbf{Input:}}
% \renewcommand{\algorithmicensure}{\textbf{Output:}}
\usepackage[linesnumbered,ruled,vlined,algo2e]{algorithm2e}
\newcommand\mycommfont[1]{\footnotesize\ttfamily\textcolor{blue}{#1}}
\SetCommentSty{mycommfont}
\let\oldnl\nl% Store \nl in \oldnl
\newcommand{\nonl}{\renewcommand{\nl}{\let\nl\oldnl}}% Remove line number for one line
\SetKwInput{KwInput}{Input}                % Set the Input
\SetKwInput{KwOutput}{Output}              % set the Output
\SetKwInOut{Parameter}{Parameter}
\SetKwRepeat{Do}{do}{while}%
% -----SET UP ALGORITHM ENVIRONMENT END ------%

\usepackage{array}
\usepackage{dblfloatfix}
\usepackage{url}
\usepackage{tikz}
\usetikzlibrary{positioning}
\usepackage{cuted} %standard LATEX will only switch between \onecolumn and \twocolumn at the top of a page; the commands themselves clear the previous page. This package does away with the restriction, and allows you to mix one and two column modes on the same page.

\def\BibTeX{{\rm B\kern-.05em{\sc i\kern-.025em b}\kern-.08em
		T\kern-.1667em\lower.7ex\hbox{E}\kern-.125emX}}

%%%%%%%%%%%%% Additional Commands %%%%%%%%%%%%%%%
% \usepackage[english]{babel} %DONT USE IT IT WILL CHANGE THE CAPTION OF FIGURES AND TABLES
\usepackage{enumerate}
\usepackage{array}
\usepackage[utf8]{inputenc}
\usepackage{amsmath}
\usepackage{amsthm, amssymb}
\usepackage{nccmath}
\usepackage{mathtools}
\usepackage{bbm}
\usepackage[makeroom]{cancel}
\usepackage{amssymb}
\graphicspath{{./figures/}}
\usepackage{fixltx2e}
\usepackage{soul}
\usepackage{xcolor}
\usepackage{ upgreek }
\usepackage{dblfloatfix}
%\usepackage[font=small,labelfont=bf]{caption}
\usepackage[colorinlistoftodos]{todonotes}
\usepackage{listings}
\usepackage{color}
\usepackage{cuted}
\usepackage{url}
\usepackage{hyperref}
%\usepackage[numbers,sort,compress]{natbib}

\usepackage{mathrsfs}
\usepackage{amsfonts} 
\usepackage{epstopdf}
%\usepackage{relsize}
%\setlength{\footskip}{20pt}
\usepackage{epsf,epsfig,verbatim,amssymb,amsmath,array,cite,amsthm,multicol,multirow}  
\usepackage{psfrag,bm,xspace}
\usepackage{hhline}
% \usepackage{mathabx}
\usepackage{physics}

%-----------------------------------------------------------------------------------------
\newtheorem{thm}{Theorem}
\newtheorem{prop}{Proposition}%[thm]
\newtheorem{dfn}{Definition}
\newtheorem{rem}{Remark}
\newtheorem{lem}[thm]{Lemma}
\newtheorem{cor}{Corollary}[thm]
\newtheorem{fact}{Fact}
\newtheorem{assumption}{Assumption}
\newtheorem{claim}{Claim}

\theoremstyle{definition}
\newtheorem{example}{Example}
\newtheorem{conjecture}{Conjecture}
\newtheorem{question}{Question}
\newtheorem{problem}{Problem}
\newtheorem*{prob*}{Problem}

\theoremstyle{remark}
\newtheorem{remark}{Remark}
\newtheorem{rems}{Remarks}
\newtheorem{para}{}
\newtheorem{summary}{Summary}

%-----------------------------------------------------------------------------------------
\newcounter{relctr} %% <- counter for relations
\everydisplay\expandafter{\the\everydisplay\setcounter{relctr}{0}} %% <- reset every eq
\renewcommand*\therelctr{\alph{relctr}} %% <- label format

\newcommand\labelrel[2]{%
  \begingroup
    \refstepcounter{relctr}%
    \stackrel{\textnormal{(\alph{relctr})}}{\mathstrut{#1}}%
    \originallabel{#2}%
  \endgroup
}
\AtBeginDocument{\let\originallabel\label} %% <- store original definition

\allowdisplaybreaks % makes automatic splits of equations
\global\long\def\11{\mathbbm{1}}
\newcommand{\du}[1]{ \uuline{ #1 } }
\newcommand\numberthis{\addtocounter{equation}{1}\tag{\theequation}}
\newcommand{\ra}{\rightarrow}
\newcommand{\la}{\leftarrow}
\newcommand{\mcl}{\mathcal}
\newcommand{\mbb}{\mathbb}
\newcommand{\mbf}{\mathbf}
\newcommand{\eps}{\epsilon}
\newcommand{\ov}{\overline}
\newcommand{\udl}{\underline}
\newcommand{\uda}{\underaccent}
\newcommand{\ulbar}[1]{ \uda{\bar}{\bar{#1}} }
\newcommand{\bsl}{\boldsymbol}
\def \defeq{\overset{\Delta}{=}}
\def \dg{\dagger}
\def \bsl{\boldsymbol}
\def \wc{\widecheck}
\def \wh{\widehat}
\def \wt{\widetilde}
\def \pr{\mbb{P}}
\def \xlrarrow{\xleftrightarrow}
\def \l{\left}
\def \r{\right}
\newcommand{\argmax}{\operatorname{argmax}}
\newcommand{\argmin}{\operatorname{argmin}}
\newcommand\mymathop[1]{\mathop{\operatorname{#1}}}



\section{NOTATION AND BACKGROUND}

We denote the set of real numbers by $\R$ and the set of natural numbers by $\N$. The set $\{1, \dots, a\} \subset \N$ is denoted by $[a]$ for $a \in \N$, and similarly ${a, \dots, b} \subset \N$ is denoted by $[a \sdots b]$. For a pair of boolean variables $x$ and $y$, the notation $\wedge$ denotes the ``and'' operator while $\vee$ denotes ``or.'' For a set of boolean variables $\{ x_1, x_2, \ldots, x_n \}$, the notations $\bigwedge_{i=1}^n x_i$ and $\bigvee_{i=1}^n x_i$ denote $x_1 \wedge x_2 \wedge \ldots \wedge x_n$ and $x_1 \vee x_2 \vee \ldots \vee x_n$, respectively. The logical negation of a boolean variable or vector $x$ is denoted by $\neg x$. We denote the identically zero function on a domain by $\zf$, and we write $f(\cdot) \not \equiv \zf$ to mean that $f(\cdot)$ is not equivalent to the zero function over its argument---i.e., there exists an input where $f$ is nonzero.


\subsection{Measure theory and probability}

For a random variable $X$, we introduce the notation ${\Pdi{x} \in M(X)}$ to represent a probability measure over the values $x$ in the domain of $X$, contained in the space of measures $M(X)$. The uniform measure over an interval $[a,b] \subset \R$ is denoted by $\cU(a,b)$. For two measures $\mu$ and $\nu$, we say that $\nu$ is absolutely continuous with respect to $\mu$ if for every $\mu$-measurable set $A$, $\mu(A) = 0$ implies $\nu(A) = 0$. If $\nu$ is absolutely continuous with respect to $\mu$, we let $d\nu / d\mu$ denote the Radon-Nikodym derivative of $\nu$ with respect to $\mu$. The standard Lebesgue measure on $\R$ is denoted $\lambda$. For a measure $\mu$ which is absolutely continuous with respect to $\lambda$, we define its $L_1$ norm in the typical manner
\begin{align*}
    \Pnormi{\mu} \defeq \int \biggl \lvert \frac{d \mu}{d \lambda} \biggr \rvert d \lambda,
\end{align*}
which we take to be the default norm in the Banach space of measures on $\R$. We denote independence between two random variables using $\PI$ and its negation by $\nPI$. 


\subsection{Causal graphs and structural causal models}

We denote a directed acyclic graph by $\G$, with the presence of a direct edge between nodes $X$ and $Y$ denoted $X \cau Y$. For a given node $X$ in $\G$, we let $\G_{\underline{X}}$ denote the graph obtained by deleting outgoing edges from $X$. We denote sets of nodes in a graph using bold font (e.g., $\bZ$). The set of parents of a node $X$ in a graph is denoted by $\pa{X}$. A path between two nodes $X$ and $Y$ can consist of arbitrarily directed edges and is said to be blocked by a set of nodes $\bZ$ if the path contains any of the following \cite{Pearl09}:
\begin{itemize}
    \item A chain $I \cau M \cau J$ with $M \in \bZ$.
    \item A fork $I \leftarrow M \cau J$ with $M \in \bZ$.
    \item A collider $I \cau M \leftarrow J$ such that $M \not \in \bZ$ and no descendent of $M$ is in $\bZ$.
\end{itemize}
     Two nodes $X$ and $Y$ are said to be d-separated by $\bZ$ if $\bZ$ blocks every path between $X$ and $Y$. We call a path with all edges oriented the same direction a directed path.

We leverage Pearl's structural causal model (SCM) formalism \cite{Pearl09}. An SCM $\M = \SCM$ consists of endogenous variables $\bV$, exogenous variables $\bU$, and structural equations $\cF$. Each $V \in \bV$ is represented by a node in the causal graph $\G$ and associated with an independently distributed exogenous variable $U_V \in \bU$. The structural equations $f_V \in \cF$ assign values of a particular node $V \in \bV$ as a function $V \defeq f_V(\pa{V}, U_V)$ of its parents and associated exogenous variable. The SCM $\M$ induces a joint distribution $\Pd{\bv}$ over the endogenous variables $\bV$. We say that an SCM $\M$ is faithful to its causal graph $\G$ if the distribution $\Pd{\bv}$ induced by $\M$ contains only the pairwise conditional independencies implied by $\G$; i.e. $X \PI Y \mid \bZ$ in the joint distribution from $\M$ iff $X$ and $Y$ are d-separated by $\bZ$ in $\G$ \cite{Spirtes2000}. As a notable special case, if $\bZ$ is empty and there exists a path from $X$ to $Y$ with no colliders then $X \nPI Y$. 

We define an intervention on a particular node $V$ to be a reassignment of the associated structural equation $f_V$. This intervention can take the form of a constant intervention $V \defeq v$, which we denote by $\doc(V = v)$ for a constant $v$ and may abbreviate to $\doc(v)$. We also define a distributional intervention, denoted by $\doc(V \sim \Pdint{v})$, where we assign $V$ to be drawn from a specified distribution $\Pdint{v}$. We denote the post-intervention SCM by $\Mi$, with an associated causal graph $\Gi$ identical to $\G$ but with incoming edges to $V$ removed. Note that reassigning the associated structural equation for any particular node $V$ induces a new distribution generated by $\Mi$ over the set of all endogenous variables $\bV$, which we denote by $\Pdi{\bv \mid \doc(V = v)}$ or $\Pdi{\bv \mid \doc(V \sim \Pdint{v})}$.


\subsection{Behavior cloning} \label{sec: imitation_learning}
Behavior cloning uses expert trajectories to train an imitating policy. For the system of interest, we use $\stdim$, $\imdim$, $\obdim$, and $\acdim$ to denote the dimensionality of the bounded state space $\stspace \subseteq \R^{\stdim}$, raw image observation space $\imspace \subseteq \R^{\imdim}$, disentangled observation space $\obspace \subseteq \R^{\obdim}$, and action space $\acspace \subseteq \R^{\acdim}$. Let $\St$, $\Im$, $\Ob$, and $\Ac$ be vector random variables taking on values in $\stspace$, $\imspace$, $\obspace$, and $\acspace$, respectively, for a discrete time step $t \in \N$. States variables $\St$ represent the intrinsic low-dimensional dynamics of the system (e.g. simulator variables) while observations $\Ob$ are distilled using a VAE-style framework from high-dimensional image measurements $\Im$, with $\imdim \gg \obdim$. The system dynamics assume that $\St[t+1]$ is strictly a function of $\St$ and $\Ac$. Lower-case script letters $\si \in [\stdim]$, $\oi \in [\obdim]$, and $\ai \in [\acdim]$ denote specific indices in the state, observation, and action vectors. For example, $\St[1][\si]$ refers to the real-valued random variable corresponding to the $\si \tth$ state variable at the first time step. We model $\Hi[1] \sim \cU(a,b)$ to be an unobserved variable capturing uncontrolled and unknown initialization stochasticity (i.e. a random ``seed'').

The collection of states, observations, and actions, along with $\Hi[1]$, comprise endogenous variables in an SCM defining our system. We denote the system SCM by $\Ms$ and denote the corresponding faithful causal graph by $\Gs$. Note that the SCM depends on the choice of policy. Since we aim to infer causalities regarding the expert policy, we generally let any causal relationships refer to the $\Ms$ and $\Gs$ induced by the expert policy unless otherwise stated. We pair the system SCM and causal graph with the tuple $\sys$. Although nodes in $\Gs$ are individual elements in our vector-valued random variables (i.e., $\St[t][\si]$ is a node, not $\St[t]$), with some abuse of notation, we let the edge symbol $\St[t] \cau X$ signify that $\St[t][\si] \cau X$ for some $\si \in [\stdim]$. Similarly, $X \cau \St[t]$ denotes that $X \cau \St[t][\si]$ for some $\si$.

% Figure environment removed

This work evaluates the importance of interventionally assigning the initial state to a particular distribution ${\St[1] \sim \Pdint{\st[1]}}$. This intervention yields a modified SCM $\Msi$ with a corresponding (not necessarily faithful) causal graph $\Gsi$, which removes the edge $\Hi[1] \to \St[1]$ in $\Gs$ (Figure~\ref{fig: structure}). We collect $N$ arbitrary-length expert trajectories from $\Msi$. The collection of all such trajectories is denoted $\trajs$. Among these $N$ trajectories, the $i^{\textrm{th}}$ trajectory consists of the tuple
\[
	\traj = \langle \st[1], \dots, \st[T];\ \im[1], \dots, \im[T];\ \ob[1], \dots, \ob[T];\ \ac[1], \dots, \ac[T] \rangle,
\]
where lowercase letters represent a concrete random variable value (to avoid confusion with indices, we use $\im$ to denote a value of $\Im$). Implicit in this definition is the existence of an \emph{encoder} $\enc: \imspace \to \obspace$ mapping each image $\im$ to a disentangled observation $\ob$. We characterize trajectories as containing observations for simplicity; our environment only provides the images $\im$, and the extraction of disentangled observations $\ob$ is method-dependent.

When training agents on $\trajs$, we parameterize policies as a neural network $\net: \imspace^L \to \acspace$. The neural policy maps some history of observations to an action $a_t$ via
\begin{align} \label{eqn: policy}
    \ac[t] = \net(\im[t], \im[t-1], \dots, \im[t-L+1]).
\end{align}
We then train $\net$ via standard behavior cloning by randomly sampling batches of images and expert actions from $\trajs$ and performing supervised regression.


\subsection{Statistical independence tests} \label{sec: hoeffding}
\newcommand{\Nhoeff}{N_{\textrm{Hoeff}}}

Our method relies on identifying whether two random variables are statistically dependent. While this is a challenging problem with a rich literature \cite{sheskin2020handbook}, in this paper, we only briefly introduce a well-known independence test for continuous distributions based on Hoeffding's D statistic \cite{hoeffding1994non, even2020independence}. Consider two real-valued random variables $X$ and $Y$ with a joint cumulative distribution function ${F(x, y) = \Pd{X \leq x, Y \leq y}}$. Hoeffding's D statistic operates on $\Nhoeff$ independent pairs of observations $\{(X_1, Y_1), \dots (X_{\Nhoeff}, Y_{\Nhoeff})\}$ and outputs a real number $D$ in the range $[-0.5, 1]$, with $D > 0$ indicating dependence. The computational complexity of calculating this statistic is $\mathcal{O} (\Nhoeff \log \Nhoeff)$. For absolutely continuous joint distributions, the D statistic is unbiased and consistent as $\Nhoeff \to \infty$, meaning that the dependence is correctly represented with probability arbitrarily close to $1$. Subsequent variations of the D statistic maintain consistency even for non-absolutely continuous joint distributions \cite{blum1961distribution}, although these complications are outside the scope of our work. We refer to the independence test based on the Hoeffding's D statistic as Hoeffding's independence test.

%%%%%%%%% End of Additional Commands %%%%%%%%%%%%
\usepackage{float,layouts}
\usepackage[justification=centering]{caption}
%---------------------------------------------%
%--------------- MY CODE END -----------------%
%---------------------------------------------%
\usepackage[T1]{fontenc}% optional T1 font encoding

% Some very useful LaTeX packages include:
% (uncomment the ones you want to load)

% *** MISC UTILITY PACKAGES ***
%
%\usepackage{ifpdf}
% Heiko Oberdiek's ifpdf.sty is very useful if you need conditional
% compilation based on whether the output is pdf or dvi.
% usage:
% \ifpdf
%   % pdf code
% \else
%   % dvi code
% \fi
% The latest version of ifpdf.sty can be obtained from:
% http://www.ctan.org/pkg/ifpdf
% Also, note that IEEEtran.cls V1.7 and later provides a builtin
% \ifCLASSINFOpdf conditional that works the same way.
% When switching from latex to pdflatex and vice-versa, the compiler may
% have to be run twice to clear warning/error messages.

% *** CITATION PACKAGES ***
%
\usepackage{cite}
\usepackage{hyperref}
% cite.sty was written by Donald Arseneau
% V1.6 and later of IEEEtran pre-defines the format of the cite.sty package
% \cite{} output to follow that of the IEEE. Loading the cite package will
% result in citation numbers being automatically sorted and properly
% "compressed/ranged". e.g., [1], [9], [2], [7], [5], [6] without using
% cite.sty will become [1], [2], [5]--[7], [9] using cite.sty. cite.sty's
% \cite will automatically add leading space, if needed. Use cite.sty's
% noadjust option (cite.sty V3.8 and later) if you want to turn this off
% such as if a citation ever needs to be enclosed in parenthesis.
% cite.sty is already installed on most LaTeX systems. Be sure and use
% version 5.0 (2009-03-20) and later if using hyperref.sty.
% The latest version can be obtained at:
% http://www.ctan.org/pkg/cite
% The documentation is contained in the cite.sty file itself.

% *** GRAPHICS RELATED PACKAGES ***
%
% \ifCLASSINFOpdf
%    \usepackage[pdftex]{graphicx}
%   % declare the path(s) where your graphic files are
%   % \graphicspath{{../pdf/}{../jpeg/}}
%   \graphicspath{{./Figures/}}
%   % and their extensions so you won't have to specify these with
%   % every instance of \includegraphics
%   % \DeclareGraphicsExtensions{.pdf,.jpeg,.png}
% \else
%   % or other class option (dvipsone, dvipdf, if not using dvips). graphicx
%   % will default to the driver specified in the system graphics.cfg if no
%   % driver is specified.
%   % \usepackage[dvips]{graphicx}
%   % declare the path(s) where your graphic files are
%   % \graphicspath{{../eps/}}
%   % and their extensions so you won't have to specify these with
%   % every instance of \includegraphics
%   % \DeclareGraphicsExtensions{.eps}
% \fi
% graphicx was written by David Carlisle and Sebastian Rahtz. It is
% required if you want graphics, photos, etc. graphicx.sty is already
% installed on most LaTeX systems. The latest version and documentation
% can be obtained at: 
% http://www.ctan.org/pkg/graphicx
% Another good source of documentation is "Using Imported Graphics in
% LaTeX2e" by Keith Reckdahl which can be found at:
% http://www.ctan.org/pkg/epslatex
%
% latex, and pdflatex in dvi mode, support graphics in encapsulated
% postscript (.eps) format. pdflatex in pdf mode supports graphics
% in .pdf, .jpeg, .png and .mps (metapost) formats. Users should ensure
% that all non-photo figures use a vector format (.eps, .pdf, .mps) and
% not a bitmapped formats (.jpeg, .png). The IEEE frowns on bitmapped formats
% which can result in "jaggedy"/blurry rendering of lines and letters as
% well as large increases in file sizes.
%
% You can find documentation about the pdfTeX application at:
% http://www.tug.org/applications/pdftex

% *** MATH PACKAGES ***
%
\usepackage{amsmath}
% A popular package from the American Mathematical Society that provides
% many useful and powerful commands for dealing with mathematics.
% Do NOT use the amsbsy package under comsoc mode as that feature is
% already built into the Times Math font (newtxmath, mathtime, etc.).
% 
% Also, note that the amsmath package sets \interdisplaylinepenalty to 10000
% thus preventing page breaks from occurring within multiline equations. Use:
\interdisplaylinepenalty=2500
% after loading amsmath to restore such page breaks as IEEEtran.cls normally
% does. amsmath.sty is already installed on most LaTeX systems. The latest
% version and documentation can be obtained at:
% http://www.ctan.org/pkg/amsmath


% Select a Times math font under comsoc mode or else one will automatically
% be selected for you at the document start. This is required as Communications
% Society journals use a Times, not Computer Modern, math font.
%\usepackage[cmintegrals]{newtxmath}
% The freely available newtxmath package was written by Michael Sharpe and
% provides a feature rich Times math font. The cmintegrals option, which is
% the default under IEEEtran, is needed to get the correct style integral
% symbols used in Communications Society journals. Version 1.451, July 28,
% 2015 or later is recommended. Also, do *not* load the newtxtext.sty package
% as doing so would alter the main text font.
% http://www.ctan.org/pkg/newtx
%
% Alternatively, you can use the MathTime commercial fonts if you have them
% installed on your system:
%\usepackage{mtpro2}
%\usepackage{mt11p}
%\usepackage{mathtime}


%\usepackage{bm}
% The bm.sty package was written by David Carlisle and Frank Mittelbach.
% This package provides a \bm{} to produce bold math symbols.
% http://www.ctan.org/pkg/bm

% *** SPECIALIZED LIST PACKAGES ***
%
\usepackage{algorithm,algorithmic}
% algorithmic.sty was written by Peter Williams and Rogerio Brito.
% This package provides an algorithmic environment fo describing algorithms.
% You can use the algorithmic environment in-text or within a figure
% environment to provide for a floating algorithm. Do NOT use the algorithm
% floating environment provided by algorithm.sty (by the same authors) or
% algorithm2e.sty (by Christophe Fiorio) as the IEEE does not use dedicated
% algorithm float types and packages that provide these will not provide
% correct IEEE style captions. The latest version and documentation of
% algorithmic.sty can be obtained at:
% http://www.ctan.org/pkg/algorithms
% Also of interest may be the (relatively newer and more customizable)
% algorithmicx.sty package by Szasz Janos:
%\usepackage{algorithmicx}
% http://www.ctan.org/pkg/algorithmicx

\renewcommand{\algorithmicrequire}{\textbf{Input:}}
\renewcommand{\algorithmicensure}{\textbf{Output:}}

% *** ALIGNMENT PACKAGES ***
%
\usepackage{array}
% Frank Mittelbach's and David Carlisle's array.sty patches and improves
% the standard LaTeX2e array and tabular environments to provide better
% appearance and additional user controls. As the default LaTeX2e table
% generation code is lacking to the point of almost being broken with
% respect to the quality of the end results, all users are strongly
% advised to use an enhanced (at the very least that provided by array.sty)
% set of table tools. array.sty is already installed on most systems. The
% latest version and documentation can be obtained at:
% http://www.ctan.org/pkg/array

% IEEEtran contains the IEEEeqnarray family of commands that can be used to
% generate multiline equations as well as matrices, tables, etc., of high
% quality.

% *** SUBFIGURE PACKAGES ***
\ifCLASSOPTIONcompsoc
  \usepackage[caption=false,font=normalsize,labelfont=sf,textfont=sf]{subfig}
\else
  \usepackage[caption=false,font=footnotesize]{subfig}
\fi
% subfig.sty, written by Steven Douglas Cochran, is the modern replacement
% for subfigure.sty, the latter of which is no longer maintained and is
% incompatible with some LaTeX packages including fixltx2e. However,
% subfig.sty requires and automatically loads Axel Sommerfeldt's caption.sty
% which will override IEEEtran.cls' handling of captions and this will result
% in non-IEEE style figure/table captions. To prevent this problem, be sure
% and invoke subfig.sty's "caption=false" package option (available since
% subfig.sty version 1.3, 2005/06/28) as this is will preserve IEEEtran.cls
% handling of captions.
% Note that the Computer Society format requires a larger sans serif font
% than the serif footnote size font used in traditional IEEE formatting
% and thus the need to invoke different subfig.sty package options depending
% on whether compsoc mode has been enabled.
%
% The latest version and documentation of subfig.sty can be obtained at:
% http://www.ctan.org/pkg/subfig

% *** FLOAT PACKAGES ***
%
%\usepackage{fixltx2e}
% fixltx2e, the successor to the earlier fix2col.sty, was written by
% Frank Mittelbach and David Carlisle. This package corrects a few problems
% in the LaTeX2e kernel, the most notable of which is that in current
% LaTeX2e releases, the ordering of single and double column floats is not
% guaranteed to be preserved. Thus, an unpatched LaTeX2e can allow a
% single column figure to be placed prior to an earlier double column
% figure.
% Be aware that LaTeX2e kernels dated 2015 and later have fixltx2e.sty's
% corrections already built into the system in which case a warning will
% be issued if an attempt is made to load fixltx2e.sty as it is no longer
% needed.
% The latest version and documentation can be found at:
% http://www.ctan.org/pkg/fixltx2e

%\usepackage{stfloats}
% stfloats.sty was written by Sigitas Tolusis. This package gives LaTeX2e
% the ability to do double column floats at the bottom of the page as well
% as the top. (e.g., "\begin{figure*}[!b]" is not normally possible in
% LaTeX2e). It also provides a command:
%\fnbelowfloat
% to enable the placement of footnotes below bottom floats (the standard
% LaTeX2e kernel puts them above bottom floats). This is an invasive package
% which rewrites many portions of the LaTeX2e float routines. It may not work
% with other packages that modify the LaTeX2e float routines. The latest
% version and documentation can be obtained at:
% http://www.ctan.org/pkg/stfloats
% Do not use the stfloats baselinefloat ability as the IEEE does not allow
% \baselineskip to stretch. Authors submitting work to the IEEE should note
% that the IEEE rarely uses double column equations and that authors should try
% to avoid such use. Do not be tempted to use the cuted.sty or midfloat.sty
% packages (also by Sigitas Tolusis) as the IEEE does not format its papers in
% such ways.
% Do not attempt to use stfloats with fixltx2e as they are incompatible.
% Instead, use Morten Hogholm'a dblfloatfix which combines the features
% of both fixltx2e and stfloats:
%
\usepackage{dblfloatfix}
% The latest version can be found at:
% http://www.ctan.org/pkg/dblfloatfix

%\ifCLASSOPTIONcaptionsoff
%  \usepackage[nomarkers]{endfloat}
% \let\MYoriglatexcaption\caption
% \renewcommand{\caption}[2][\relax]{\MYoriglatexcaption[#2]{#2}}
%\fi
% endfloat.sty was written by James Darrell McCauley, Jeff Goldberg and 
% Axel Sommerfeldt. This package may be useful when used in conjunction with 
% IEEEtran.cls'  captionsoff option. Some IEEE journals/societies require that
% submissions have lists of figures/tables at the end of the paper and that
% figures/tables without any captions are placed on a page by themselves at
% the end of the document. If needed, the draftcls IEEEtran class option or
% \CLASSINPUTbaselinestretch interface can be used to increase the line
% spacing as well. Be sure and use the nomarkers option of endfloat to
% prevent endfloat from "marking" where the figures would have been placed
% in the text. The two hack lines of code above are a slight modification of
% that suggested by in the endfloat docs (section 8.4.1) to ensure that
% the full captions always appear in the list of figures/tables - even if
% the user used the short optional argument of \caption[]{}.
% IEEE papers do not typically make use of \caption[]'s optional argument,
% so this should not be an issue. A similar trick can be used to disable
% captions of packages such as subfig.sty that lack options to turn off
% the subcaptions:
% For subfig.sty:
% \let\MYorigsubfloat\subfloat
% \renewcommand{\subfloat}[2][\relax]{\MYorigsubfloat[]{#2}}
% However, the above trick will not work if both optional arguments of
% the \subfloat command are used. Furthermore, there needs to be a
% description of each subfigure *somewhere* and endfloat does not add
% subfigure captions to its list of figures. Thus, the best approach is to
% avoid the use of subfigure captions (many IEEE journals avoid them anyway)
% and instead reference/explain all the subfigures within the main caption.
% The latest version of endfloat.sty and its documentation can obtained at:
% http://www.ctan.org/pkg/endfloat
%
% The IEEEtran \ifCLASSOPTIONcaptionsoff conditional can also be used
% later in the document, say, to conditionally put the References on a 
% page by themselves.

% *** PDF, URL AND HYPERLINK PACKAGES ***
%
\usepackage{url}
% url.sty was written by Donald Arseneau. It provides better support for
% handling and breaking URLs. url.sty is already installed on most LaTeX
% systems. The latest version and documentation can be obtained at:
% http://www.ctan.org/pkg/url
% Basically, \url{my_url_here}.

% *** Do not adjust lengths that control margins, column widths, etc. ***
% *** Do not use packages that alter fonts (such as pslatex).         ***
% There should be no need to do such things with IEEEtran.cls V1.6 and later.
% (Unless specifically asked to do so by the journal or conference you plan
% to submit to, of course. )

\usepackage{color-edits}
%\usepackage[suppress]{color-edits}
\addauthor{vj}{red}

% correct bad hyphenation here
\hyphenation{op-tical net-works semi-conduc-tor}

\allowdisplaybreaks

\begin{document}
% \setlength{\baselineskip}{15pt}%use for manually setting line spacing, comment for submission
%
% paper title
% Titles are generally capitalized except for words such as a, an, and, as,
% at, but, by, for, in, nor, of, on, or, the, to and up, which are usually
% not capitalized unless they are the first or last word of the title.
% Linebreaks \\ can be used within to get better formatting as desired.
% Do not put math or special symbols in the title.
\title{Cooperative Multi-Agent Constrained POMDPs: Strong Duality and Primal-Dual Reinforcement Learning with Approximate Information States
}
%
%
% author names and IEEE memberships
% note positions of commas and nonbreaking spaces ( ~ ) LaTeX will not break
% a structure at a ~ so this keeps an author's name from being broken across
% two lines.
% use \thanks{} to gain access to the first footnote area
% a separate \thanks must be used for each paragraph as LaTeX2e's \thanks
% was not built to handle multiple paragraphs
%
% 
\author{Nouman~Khan,~\IEEEmembership{Member,~IEEE,}
       % Mehrdad~Moharrami,~\IEEEmembership{Member,~IEEE,}
       and~Vijay~Subramanian,~\IEEEmembership{Senior Member,~IEEE}% <-this % stops a space
% \thanks{A short version of this work appeared in \textit{IEEE INFOCOM 2020 - IEEE Conference on Computer Communications}, 2020, pp. 1153-1162.}% <-this % stops a space
%\thanks{J. Doe and J. Doe are with Anonymous University.}% <-this % stops a space
%\thanks{Manuscript received April 19, 2005; revised August 26, 2015.}
}

% note the % following the last \IEEEmembership and also \thanks - 
% these prevent an unwanted space from occurring between the last author name
% and the end of the author line. i.e., if you had this:
% 
% \author{....lastname \thanks{...} \thanks{...} }
%                     ^------------^------------^----Do not want these spaces!
%
% a space would be appended to the last name and could cause every name on that
% line to be shifted left slightly. This is one of those "LaTeX things". For
% instance, "\textbf{A} \textbf{B}" will typeset as "A B" not "AB". To get
% "AB" then you have to do: "\textbf{A}\textbf{B}"
% \thanks is no different in this regard, so shield the last } of each \thanks
% that ends a line with a % and do not let a space in before the next \thanks.
% Spaces after \IEEEmembership other than the last one are OK (and needed) as
% you are supposed to have spaces between the names. For what it is worth,
% this is a minor point as most people would not even notice if the said evil
% space somehow managed to creep in.

% The paper headers
%\markboth{Journal of \LaTeX\ Class Files,~Vol.~14, No.~8, August~2015}%
%{Shell \MakeLowercase{\textit{et al.}}: Bare Demo of IEEEtran.cls for IEEE Journals}
% The only time the second header will appear is for the odd numbered pages
% after the title page when using the twoside option.
% 
% *** Note that you probably will NOT want to include the author's ***
% *** name in the headers of peer review papers.                   ***
% You can use \ifCLASSOPTIONpeerreview for conditional compilation here if
% you desire.

% If you want to put a publisher's ID mark on the page you can do it like
% this:
%\IEEEpubid{0000--0000/00\$00.00~\copyright~2015 IEEE}
% Remember, if you use this you must call \IEEEpubidadjcol in the second
% column for its text to clear the IEEEpubid mark.

% use for special paper notices
%\IEEEspecialpapernotice{(Invited Paper)}

% make the title area
\maketitle
% As a general rule, do not put math, special symbols or citations
% in the abstract or keywords.
\begin{abstract}
We study the problem of decentralized constrained POMDPs in a team-setting where the multiple non-strategic agents have asymmetric information. Strong duality is established for the setting of infinite-horizon expected total discounted costs when the observations lie in a countable space, the actions are chosen from a finite space, and the immediate cost functions are bounded. Following this, connections with the common-information and approximate information-state approaches are established. The approximate information-states are characterized independent of the Lagrange-multipliers vector so that adaptations of the multiplier (during learning) will not necessitate new representations. Finally, a primal-dual multi-agent reinforcement learning (MARL) framework based on centralized training distributed execution (CTDE) and three time-scale stochastic approximation is developed with the aid of recurrent and feed-forward neural-networks as function-approximators.
\end{abstract}

% Note that keywords are not normally used for peerreview papers.
\begin{IEEEkeywords}
Planning and Learning in Multi-Agent POMDP with Constraints, Strong Duality, Lower Semi-continuity, A Minimax Theorem for Functions with Positive Infinity, Tychonoff's theorem, Common Information, Approximate Information State, Dynamic Programming, Centralized Training Distributed Execution, Stochastic Approximation.
\end{IEEEkeywords}

% For peer review papers, you can put extra information on the cover
% page as needed:
% \ifCLASSOPTIONpeerreview
% \begin{center} \bfseries EDICS Category: 3-BBND \end{center}
% \fi
%
% For peerreview papers, this IEEEtran command inserts a page break and
% creates the second title. It will be ignored for other modes.
\IEEEpeerreviewmaketitle


%----------------------%
%------ SECTIONS ------%
%----------------------%
% \onecolumn
\section{Introduction}
Current quantum hardware is unable to carry out universal quantum computations due to the buildup of errors that occur during the computation. 
The magnitude of the individual error is currently above the value that the Threshold Theorem requires in order to kick-start quantum error correction and fault-tolerant quantum computation~\cite[Section 10.6]{nielsen_chuang_2010}. 
Although the experimentally achieved fidelity rates are promising and the error bounds are inching closer to the required threshold, we will have to work for the foreseeable future with quantum hardware with errors that build-up during the computation.  This implies that we can only do a limited number of steps before the output of the computation has become completely uncorrelated with the intended one.

For fault-tolerant quantum computing, we repeat four steps: 
1) We apply a number of single and two-qubit quantum gates, in parallel whenever possible; 
2) We perform a syndrome measurement on a subset of the qubits; 
3) We perform fast classical computations to determine which errors have occurred and how to correct them; 
and, 4) We apply correction terms based on the classical computations.
We then repeat these four steps with a next sequence of gates. 
These four steps are essential to fault-tolerant quantum computing. 


The starting point of this work is to use the four steps outlined above, not to carry out error correction and fault-tolerant computation, but to enhance short, constant-depth, {\em uncorrected} quantum circuits that perform single qubit gates and {\em nearest-neighbor} two qubit gates. 
Since in the long run we will have to implement error-correction and fault-tolerant computation anyhow, and this is done by such a four-step process, why not make other use of this architecture? Moreover, on some of the quantum hardware platforms, these operations are already in place.
Embracing this idea we naturally arrive at the question: what is the computational power of \textit{low-depth} quantum-classical circuits organized as in the four steps outlined above? 
We thus investigate circuits that execute a small, ideally constant, number of stages, where at each stage we may apply, in parallel, single qubit gates and {\em nearest-neighbor} two qubit gates, followed by measurements, followed by low-depth classical computations of which the outcome can control quantum gates in later stages. 
It is not clear, at first, whether such circuits, especially with constant depth, can do anything remotely useful. 
But we will see that this is indeed the case: many quantum computations can be done by such circuits in constant depth. 
By parallelizing quantum computations in this way, we improve the overall computational capabilities of these circuits, as we do not incur errors on qubits that are idle, simply because qubits are not idle for a very long time. 
Furthermore, reducing the depth of quantum circuits, at the cost of increasing width, allows the circuit to be run faster even if errors occur.

The first usage of such a four-step layout, not to do error correction, but to perform computations, can be found in the paradigm of measurement-based quantum computing~\cite{gottesman1999demonstrating,raussendorf2001one,jozsa2006introduction,clark2007generalised}: 
A universal form of quantum computing where a quantum state is prepared and operations are performed by measuring qubits in different bases, depending on previous measurements and intermediate measurements.

\citeauthor{PhamSvore2013} were the first to formalize the four-step protocol for performing computations~\cite{PhamSvore2013}. They included specific hardware topologies by considering two-dimensional graphs for imposing constraints on qubit interactions. In their model, they develop circuits for particularly useful multi-qubit gates, including specifying costs in the width, number of qubits, depth, number of concurrent time steps, size, and total number of non-Identity operations.
As a result, they find an algorithm that factors integers in polylogarithmic depth.
\citeauthor{Browne:2011} showed that the main tool in the work by \citeauthor{PhamSvore2013}, the fan-out gate, can also be replaced by additional log-depth classical computations in the measurement-based quantum computing setting~\cite{Browne:2011}.

More recently, \citeauthor{Cirac:2021} introduced a scheme to implement unitary operations involving quantum circuits combined with Local Operations and Classical Communication ($\mathsf{LOCC}$) channels: $\mathsf{LOCC}$-assisted quantum circuits~\cite{Cirac:2021}. Similarly to the four-step scheme we just described, they allow for a short depth circuit to be run on the qubits, followed by one round of $\mathsf{LOCC}$, in which ancilla qubits are measured and local unitaries are applied based on the measurement outcomes. They show that in this model any 1D transitionally invariant matrix-product state (MPS) with fixed bond dimension is in the same phase of matter as the trivial state. Similar ideas can be found in~\cite{TVV_NonAbelianTopologicalOrder_2022, tantivasadakarn2021long}.

In this work, we introduce a new model, called \textit{Local Alternating Quantum-Classical Computations} ($\LAQCC$). In this model we alternate between running quantum circuits (constrained by locality), ending in the measurement of a subset of qubits, and fast classical computations based on the measurement results. The outcome of the classical computations are then used to control future quantum circuits. We allow for flexibility in this model, by giving different constraints to the power of both the quantum circuits and the classical circuits as well as the number of alternations between them. 
Most attention will be given to $\LAQCC$ containing quantum circuits of constant depth, classical circuits of logarithmic depth and at most a constant number of alternations between them. 
Any circuit constructed in this model is considered to be of constant depth. 
We restrict ourselves to logarithmic depth classical computations, as this is the first natural and non-trivial extension beyond constant-depth classical computations. 
Constant-depth classical computations do however also have an equivalent constant-depth quantum implementation.

The definition of $\LAQCC$ sharpens the original definition of \citeauthor{PhamSvore2013} by adding constraints to the intermediate classical computations. This allows us to bound the power of $\LAQCC$ from above. 

The main result of \citeauthor{Cirac:2021}, that 1D translational invariant MPS with fixed bond dimension can be prepared by $\mathsf{LOCC}$-assisted circuits, relies on local symmetries of the MPS. These symmetries allow them to prepare local states (on a constant number of qubits) and glue them together by doing one round of the appropriate entangling measurement and corrections, after which they run a round of local unitaries to get the desired result. This general scheme for preparing states that exhibit an MPS description with the appropriate local symmetries requires only geometrically local unitaries and one round of measurement and corrections an therefore is accessible in $\LAQCC$. Studying different local symmetries, known as Symmetry Protected Topological (SPT) phases of matter, to find measurement-based constant depth circuits for states is a broad ongoing field of research~\cite{TVV_NonAbelianTopologicalOrder_2022, tantivasadakarn2021long, smith2023deterministic}. 
All these schemes have a $\LAQCC$ implementation.

%$\LAQCC$-circuits also exist for general schemes of preparing local states, based on the local tensors, and gluing them together using one round of entangled measurement and corrections, based on the local symmetry. 
%The main result of \citeauthor{Cirac:2021}, that 1D translational invariant MPS with fixed bond dimension can be prepared by $\mathsf{LOCC}$-assisted circuits, relies heavily on local symmetries of the MPS and as a result also has an equivalent $\LAQCC$ implementation. 
%The corrections applied after the measurement round are local unitaries depending on the local symmetries of the MPS. 

 

%This general scheme of preparing local states, based on the local tensors, and gluing it together by doing one round of entangled measurement and corrections, based on the local symmetry, is accessible in $\LAQCC$.
Note however that \citeauthor{Cirac:2021} also suggest a circuit for the $W$-state.
This circuit uses sequentially and dependent measurement-based corrections of the ancilla qubits. 
These dependent measurements translate to sequential alternations between the quantum and classical circuits and therefore increase the total depth to linear depth, exceeding the constant-depth constraints imposed by $\LAQCC$-circuits. 

We study the power of the $\LAQCC$ model with respect to state preparation, showing that even with only constant quantum-depth and logarithmic classical depth it remains possible to prepare states with long-range entanglement.
Another surprising result is that it is unlikely that $\LAQCC$ circuits are classically simulatable. We show that any instantaneous quantum polynomial-time (IQP) circuit~\cite{Bremner2010,Shepherd2009} has an $\LAQCC$ implementation.
Classical simulation of IQP circuits implies the collapse of the polynomial hierarchy to the third level, which is not believed to be true~\cite{Bremner2017}. Therefore, we expect that $\LAQCC$ circuits are unlikely to be classically simulatable. We bound the power of $\LAQCC$ by showing that it is contained in $\QNC^1$, the class of polynomial-size, log-depth circuits.

Next, we also study the power that intermediate classical calculations can add to quantum computations, by considering a new model that alternates between polynomially many polynomial-depth quantum circuits and unbounded classical computations
We study this model by doing a complexity theoretical analysis, where we draw inspiration from the notions of complexity given by \citeauthor{RosenthalYuen:2022}, \citeauthor{MetgerYuen:2023}, and \citeauthor{Aaronson:2004}.
All three complexity notions are based on the notion of state preparation, instead of more traditional definition of complexity such as the decidability of a computational problem. 
The first two consider classes based on sequences of quantum states preparable by a polynomial-sized quantum circuit, where the circuits are uniformly generated by a computational class, for instance, the class $\mathsf{PSPACE}$, which results in the complexity class $\mathsf{StatePSPACE}$~\cite{RosenthalYuen:2022,MetgerYuen:2023}.
The third notion considers a relative complexity, where the complexity is measured between two given states, and is measured by the number of gates, from a given gate-set, required to transform one state in another state~\cite{Aaronson:2004}. 
For our definition of state preparation complexity, we drop the uniformity constraint from~\cite{RosenthalYuen:2022,MetgerYuen:2023} and define a class as $\mathsf{StateX}$, which refers to states preparable by circuits of type $\mathsf{X}$. 
As an example, if $\mathsf{X} = \QNC^0$, this results in the class $\mathsf{StateQNC^0}$, which is the set of states preparable from the $\ket{0}^n$ state by poly-size constant-depth circuits. 
This notion is similar to the relative complexity from~\cite{Aaronson:2004}, where one state is the  $\ket{0}^n$ state and instead of counting the number of gates we consider the set of states preparable by a fixed number of gates. Using this notion of complexity we show that any state preparable by an $\LAQCC^*$ circuit is also preparable by a $\mathsf{PostQPoly}$ circuit, the class of circuits of polynomial depth with an additional post-selection gate. 

All Clifford circuits have a constant-depth $\LAQCC$ implementation, implying that any stabilizer state can be implemented by a constant-depth $\LAQCC$ circuit, see Section~\ref{sec:clifford_circuits} for a proof of this statement. 
Efficient circuits for stabilizer states have been known already through measurement-based quantum computing. Therefore this paper focuses on the preparation of non-stabilizer states, and as a surprising result we find novel constant-depth protocols for four very natural classes of non-stabilizer states.
Despite the extensive research into these four classes of non-stabilizer states and the many applications of them, no efficient constant- or low-depth state preparation protocols are known yet. We specifically consider these four classes as they are all often used as initial states in other algorithms.

The first state is a uniform superposition over an arbitrary number of states. 
This state finds applications in many quantum algorithms, as they often start with a uniform superposition over multiple states. 
This superposition is often achieved by applying Hadamard gates to every qubit due to its simplicity to prepare. 
Yet, the analysis of many algorithms, such as Shor's algorithm~\cite{Shor:1997}, would benefit from a different initial superposition. 
The circuit to prepare the uniform superposition over an arbitrary number of states uses an exact version of Grover search as a subroutine, that turns a probabilistic circuit, with a known constant probability of success, into a deterministic circuit. 
We use the circuit for preparing a uniform superposition over an arbitrary number of states as a subroutine in the next two quantum state preparation protocols. 

The second state is the $W$-state, the uniform superposition over all computational basis states of Hamming-weight~$1$, a natural long-ranged entangled state that displays a fundamentally nonequivalent type of entanglement from the Greenberger–Horne–Zeilinger state~\cite{WState:2000}, for which $\LAQCC$-type constant-depth circuits were previously known~\cite{PhamSvore2013, Cirac:2021}. 
The $W$-state is often used as benchmark for new quantum hardware~\cite{Haffner2005,Neeley2010,GarciaPerez:2021}. 
A novel way to prepare the $W$-state therefore gives a new way to benchmark different quantum devices with each other. 
A circuit for preparing the $W$-state was given in~\cite{Cirac:2021}, but this implementation requires sequentially alternating measurements followed by local unitaries, which in the $\LAQCC$ model is not considered to be of constant depth. 
We improve this protocol by giving an $\LAQCC$ implementation of the $W$-state, based on a compress-uncompress method that links the one-hot and binary encoding of integers.

The third state considered is the Dicke state, a generalization of the $W$-state, a superposition over all computational basis states with Hamming-weight $k$~\cite{Dicke:1954}. 
Dicke states have relevance in various practical settings.
For instance, for quantum game theory~\cite{zdemir2007}, quantum storage~\cite{Bacon_Compress:2006,Plesch:2010}, quantum error correction~\cite{ouyang2014permutation}, quantum metrology~\cite{toth2012multipartite}, and quantum networking~\cite{prevedel2009experimental}. 
Dicke states have been used as a starting state for variational optimization algorithms, most notably Quantum Alternating Operator Ansatz (QAOA)~\cite{Hadfield2019}, to find solutions to problems such as Maximum k-vertex Cover~\cite{Brandhofer2022,cook2020quantum}.
The ground states of physical Hamiltonians describing one-dimensional chains tend to show a resemblance to Dicke states such as states resulting from the Bethe ansatz, making them an ideal starting state when investigating the ground state behavior of these Hamiltonians~\cite{TDL_BetheAnsatzDerivation:2010,B_ExcitedStateQuantumPhaseTransitions:2013,DickeTransitions:2021}. 
For instance, the algorithm by \citeauthor{van2021preparing}, who give an algorithm to prepare the Bethe ansatz eigenstates of the spin-1/2 XXZ spin chain, starts by first preparing a Dicke state~\cite{van2021preparing}. 
A Dicke-state preparation protocol based on the compress-uncompress methodology used in the $W$-state furthermore finds applications in entanglement distillation, where the entanglement of a large state is concentrated on only a few qubits. 
Efficient deterministic circuits for preparing Dicke states have been proposed by \citeauthor{bartschi2019deterministic}~\cite{bartschi2019deterministic, bartschi2022deterministic_short_depth}. 
They provide a quantum circuit of depth $\mathO(k \log(\frac{n}{k}))$, allowing arbitrary connectivity, to prepare a Dicke state, which they conjecture to be optimal when $k$ is constant. 
In this work, we provide a constant-depth $\LAQCC$ circuit below their conjectured bound already for constant $k$. 
However, this does not directly disprove their conjecture, as we allow for intermediate measurements and classical computations. 
More significantly, we even construct constant-depth $\LAQCC$ circuits for $k = \mathO(\sqrt{n})$ greatly improving their bound.
This construction extends the compress-uncompress method for the $W$-state combined with additional subroutines. 

We continue with a log-depth state preparation protocol for the Dicke-state for arbitrary $k$. 
This protocol implements an efficient transformation between the factoradic number representation and the combinatorial number representation of a positive integer. 
The combinatorial number representation relates directly to the Dicke state. 
The provided efficient transformation between number representation systems might be of independent interest. 

We conclude by modifying our protocol for preparing a Dicke-state to a protocol that prepares quantum many-body scar states in constant-depth. 
These states have low entanglement and longer coherence times than states with similar energy density.
These characteristics make many-body scar states interesting to analyze and relevant within physics.
Many-body scar states appear for instance in the AKLT model~\cite{AKLT:1987,MRBAR:2018,MRB:2018} and different spin models~\cite{SI:2019,MOBFR:2020}.
Known methods for preparing these states have polynomial-depth~\cite{Gustafson:2023}, whereas our circuit has constant depth. 

% We conclude by studying the power that intermediate classical calculations can add to quantum computations. 
% In this study, we define a new model that relaxes constant-depth quantum circuits to polynomial depth quantum circuits, log-depth classical calculations to unbounded classical computations and a constant number of alternations to a polynomial number of alternations. 
% We call this model $\LAQCC^*$. 
% We study this model by doing a complexity theoretical analysis, where we draw inspiration from the notions of complexity given by \citeauthor{RosenthalYuen:2022}, \citeauthor{MetgerYuen:2023}, and \citeauthor{Aaronson:2004}.
% All three complexity notions are based on the notion of state preparation, instead of more traditional definition of complexity such as the decidability of a computational problem. 
% The first two consider classes based on sequences of quantum states preparable by a polynomial-sized quantum circuit, where the circuits are uniformly generated by a computational class, for instance, the class $\mathsf{PSPACE}$, which results in the complexity class $\mathsf{StatePSPACE}$~\cite{RosenthalYuen:2022,MetgerYuen:2023}.
% The third notion considers a relative complexity, where the complexity is measured between two given states, and is measured by the number of gates, from a given gate-set, required to transform one state in another state~\cite{Aaronson:2004}. 
% For our definition of state preparation complexity, we drop the uniformity constraint from~\cite{RosenthalYuen:2022,MetgerYuen:2023} and define a class as $\mathsf{StateX}$, which refers to states preparable by circuits of type $\mathsf{X}$. 
% As an example, if $\mathsf{X} = \QNC^0$, this results in the class $\mathsf{StateQNC^0}$, which is the set of states preparable from the $\ket{0}^n$ state by poly-size constant-depth circuits. 
% This notion is similar to the relative complexity from~\cite{Aaronson:2004}, where one state is the  $\ket{0}^n$ state and instead of counting the number of gates we consider the set of states preparable by a fixed number of gates. Using this notion of complexity we show that any state preparable by an $\LAQCC^*$ circuit is also preparable by a $\mathsf{PostQPoly}$ circuit, the class of circuits of polynomial depth with an additional post-selection gate. 

\paragraph{Summary of results}
\begin{itemize}
    \item We give a new definition of a computational model that captures the power of the four step process: applying a constant number of layers of one- and two-qubit gates; performing a syndrome measurement; perform a fast classical computation determining corrections; apply corrections. We call this model \emph{Local Alternating Quantum Classical Computations}, or $\LAQCC$ for short. In this model we bound the allowed quantum operations, intermediate classical calculations, and number of rounds separately. In Section~\ref{sec:LAQCC_model} we define this model and give a list of operations based on results from literature contained in this computational model. In some of these operations we explicitly use that we allow for multiple, but at most constant, rounds  of corrections.
    \item  We show show that there exist $\LAQCC$ circuits that can not be weakly simulated in Section~\ref{sec:IQP_in_LAQCC}. We further show that for every $\LAQCC$ circuit there exists a $\QNC^1$ circuit simulating it perfectly, in Section~\ref{sec:LAQCC_in_QNC1}.
    \item We introduce a new type computational complexity for preparing states and show that the extension of $\LAQCC$ where we allow a polynomial number of rounds and unbounded classical computation, is contained in $\mathsf{PostQPoly}$, the class of polynomial circuits with post-selection, in Section~\ref{sec:Complexity results}.
    \item We show a protocol to prepare the uniform superposition state of size $q$ in $\LAQCC$ using $\mathO(\ceil{\log_2(q)}^2)$ qubits in Section~\ref{sec:superposition_modulo_q}. 
    \item We show a protocol to prepare the $W_n$ state in $\LAQCC$ using $\mathO(n\log(n))$ qubits in Section~\ref{sec:W_state_in_LAQCC}.
    \item We show two ways of preparing the Dicke-$(n,k)$ state. The first method is in $\LAQCC$, works up to $k = \mathO(\sqrt{n})$, uses $\mathO(n^2\log(n))$ qubits, and is found in Section~\ref{sec:dicke:small_k}. The second method is in $\LAQCC\text{-}\mathsf{LOG}$ (an extension of $\LAQCC$ allowing for logarithmic number of alterations instead of constant), works for any $k$, uses $\mathO(\text{poly}(n))$ qubits, and is found in Section~\ref{sec:Dicke_in_LAQCC_LOG}. 
    \item We extend on our $\LAQCC$ method of generating Dicke-$(n,k)$ states for $k = \mathO(\sqrt{n})$ and show a protocol to generate many-body scar states for a particular Hamiltonian in $\LAQCC$ (Section~\ref{sec:many_body_scar}). 
\end{itemize}
Summarized in a table, we provide the following state generation protocols:
\begin{table}[htb]
\centering
\begin{tabular}{l|l|l|l}
\textbf{State description} & \textbf{Width} & \textbf{Depth} & \textbf{Implementation}\\
\hline 
Uniform superposition mod $q$: $\frac{1}{\sqrt{q}} \sum_{i = 0}^{q-1}\ket{i}$ & $\mathO(\ceil{\log^2 q})$ & $\mathO(1)$ & Section~\ref{sec:superposition_modulo_q}\\

$W$-state: $\frac{1}{\sqrt{n}}\sum_{i = 0}^{n-1}\ket{e_i}$ & $\mathO(n \log n)$ & $\mathO(1)$ & Section~\ref{sec:W_state_in_LAQCC}\\

Dicke-$(n,k)$, $k = \mathO(\sqrt{n})$: $\binom{n}{k}^{-1/2}\sum_{x \in \{0,1\}^n: |x| = k} \ket{x}$ &  $\mathO(n^2\log n)$ & $\mathO(1)$ 
&Section~\ref{sec:dicke:small_k}\\

Dicke-$(n,k)$: $\binom{n}{k}^{-1/2}\sum_{x \in \{0,1\}^n: |x| = k} \ket{x}$ & $\mathO(\text{poly}(n))$ & $\mathO(\log n)$ &Section~\ref{sec:Dicke_in_LAQCC_LOG}\\

QMBS: $\ket{S_k} = \frac{1}{k! \sqrt{\mathcal N(n,k)}}(Q^\dagger)^k \ket{\Omega}$ &  $\mathO(n^2\log n)$ & $\mathO(1)$  &  Section~\ref{sec:many_body_scar}
\end{tabular}
\caption{Summary of state preparation protocols given in this paper.}
\label{tab:sate_prep}
\end{table}
In the entry for the quantum many-body scar state $Q$ denotes the raising operator and $\mathcal N(n,k)=\binom{n-k-1}{k}$. 
Section~\ref{sec:many_body_scar} will provide more details on the variables and the implementation. 

\paragraph{Organization of the paper}
\noindent We first introduce relevant preliminaries in Section~\ref{sec:preliminaries}. 
In Section~\ref{sec:LAQCC_model} we formally define the class of Local Alternating Quantum-Classical Computations ($\LAQCC$). We also show that any Clifford circuit can be implemented in constant depth $\LAQCC$ (a result based on a result from measurement-based quantum computing~\cite{jozsa2006introduction}). 
This result allows us to give many useful multi-qubit gates and routines in Section~\ref{sec:gates_created_in_LAQCC}. 
Beyond that we show that constant depth $\LAQCC$ circuits are contained in $\QNC^1$ and that any $\mathsf{IQP}$ circuit has an $\LAQCC$ implementation.
We conclude this section with an analysis of a more powerful instantiation of $\LAQCC$ and show an inclusion with respect to the class $\mathsf{PostQPoly}$, which is the class of circuits of polynomial depth with one additional post-selection gate. 
In Section~\ref{sec:state_prep_in_LAQCC} we give $\LAQCC$ circuit implementations for preparing the uniform superposition over an arbitrary number of states, the $W$-state and the Dicke state up to $k = \mathO(\sqrt{n})$. We furthermore give a log-depth circuit implementation for preparing the Dicke state for any $k$. We conclude by showing a $\LAQCC$ circuit for generating many body scar states of a particular type of Hamiltonian.



\vspace{-0.15cm}

\section{Strategy Templates}\label{sec:templates}

In this section, we introduce a formalization of player \Odd strategies in \Odd-fair parity games via \emph{strategy templates}.
% 
In contrast to player \Even, player \Odd winning strategies are no longer positional in \Odd-fair parity games, as illustrated by the following example. %that requires the same number of symbolic steps as the algorithm computing winning strategies for \Even in \enquote{normal} parity games.
% \vspace{-0.5em}
\begin{example}\label{ex:strategytemplates}
Consider the three different parity games depicted in Fig.~\ref{fig:Oddstrategies1}. %, three \Odd-fair parity games are depicted, with circles indicating \Ve and squares indicating \Vo. Edges in $E^\ell$ are shown by dashed lines. All nodes are labeled with their priorities.
   In all three games, \Odd has a winning strategy from all vertices, i.e., $\mathcal{W}_{Odd}=V$. %The red-colored edges indicate \Odd's strategy: if \Odd takes the red edges alternatingly from the source nodes, it wins from all nodes. 
  However, in order to win, the vertex $3$ has to be seen infinitely often in game (a) and (b), which forces \Odd to use its live edge\textbackslash s infinitely often. This prevents the existence of a positional strategy for \Odd in games (a) and (b): In (a) it needs to somehow alternate between (it's only) live edge to $4$ and a \enquote{normal} edge to $7$ (both indicated in red) in order to win, and in (b) it needs to somehow alternate between all its live edges (also indicated in red). In the game (c), \Odd can win by 'escaping' its live vertex $3$ to a \enquote{normal} vertex $5$, and thereby has a positional strategy. % (again indicated in red).
   
  Now consider the subgraph of each game formed by all colored edges (red and blue), which include the strategy choices from \Vo and \emph{all} outgoing edges from \Ve. As we have seen that \Odd needs to play all red edges repeatably, this subgraph represents the paths that \emph{can} be seen in the game depending on the \Even strategy. Hence, a node $v\in\Vl\subseteq\Vo$ can be seen infinitely often in a play (compliant with \Odd's strategy), if it lies on a cycle in this subgraph. We observe that, in games (a) and (b), node $3$ lies on cycles in this subgraph, whereas in game (c), it does not. 
  We further see that whenever a vertex  $v\in\Vl$ lies on a cycle, \Odd needs to take all its outgoing live edges (as for vertex $3$ in example (b)) and possibly one more edge (as for vertex $3$ in example (a)), for all other vertices in $\Vo$ a positional strategy suffices (as for vertex $5$ in all examples, and for vertex $3$ in example (c)). This shows that \Odd strategies are intuitively still \enquote{almost positional}.
% 
\end{example}

% Figure environment removed


\vspace*{-0.2cm}

The intuitions conveyed by Ex.~\ref{ex:strategytemplates} are formalized by the following definitions. % for \Odd strategy templates.


\begin{definition}[\Odd Strategy Template]\label{def:Oddstrategytemplate}
 Given an \Odd-fair parity game $\mathcal{G}^\ell = \ltup{\mathcal{G}, E^\ell}$ with \newline $\mathcal{G} = \langle V, \Ve, \Vo, E, \chi\rangle$, an \Odd \emph{strategy template} $\mathcal{S}$ over $\mathcal{G}^\ell$ is a subgraph of $\mathcal{G}$ given as follows: $\mathcal{S}:=\tup{V',E'}$ where $V'\subseteq V$ and $E'\subseteq E \cap (V' \times V')$ such that the following hold,
\begin{compactitem}\label{item:Oddstrtemprules}
 \item if $v \in \Vo \cap V'$ does not lie on a cycle in $(V',E')$, then $|E'(v)|=1$,
 \item if $v \in \Vo \cap V'$ lies on a cycle in $(V',E')$ then $E^\ell(v) \subseteq E'(v)$ and  $1\leq |E'(v)|\leq |E^\ell(v)| + 1$,
 \item if $v \in \Ve \cap V'$, then  $E'(v) = E(v)$.
\end{compactitem}
\end{definition}
%
\begin{definition}\label{def:compliantstrat}
 Let  $\mathcal{G}^\ell = \ltup{\mathcal{G}, E^\ell}$ be an \Odd-fair parity game with \Odd strategy template $\mathcal{S}=\tup{V',E'}$, and $V'_\Odd := V' \cap V_\Odd$. Then an
\Odd strategy $\rho$ is said to be \textbf{compliant} with $\mathcal{S}$ if  
it is a winning strategy in the game $\ltup{\gamegraph,\alpha'}$ where $\gamegraph= \tup{V,\Ve,\Vo,E}$ and 
\begin{subequations}
 \begin{align}
 \alpha':= &\textstyle\bigwedge_{v\in\Vo'}(\,\square\, (\,v \implies \bigvee_{(v,w)\in E'} \bigcirc\, w\,))\,\label{equ:alpha:a}\\
 & \textstyle\wedge \bigwedge_{v\in\Vo'} (\,\square \,\diamondsuit\, v \implies \bigwedge_{(v,w)\in E'}\square\, \diamondsuit\, (\,v \wedge \bigcirc \,w\,)).\label{equ:alpha:b}
\end{align}
\end{subequations}
\end{definition}

Intuitively, for all \Odd vertices in $\mathcal{S}$, the strategy $\rho$ compliant with $\mathcal{S}$ takes only their outgoing edges in $\mathcal{S}$ \eqref{equ:alpha:a}, and if a play visits an \Odd node $v$ infinitely often, then $\rho$ takes each of $v$'s outgoing edges in $\mathcal{S}$ infinitely often \eqref{equ:alpha:b}.
% 
For an \Odd strategy template $\mathcal{S}$, if $v \in V'_\Odd$ lies on a cycle in $\mathcal{S}$, then by Def. \ref{def:Oddstrategytemplate}, $\mathcal{S}$ contains all live outgoing edges of $v$. By \eqref{equ:alpha:b} any \Odd strategy $\rho$ compliant with $\mathcal{S}$ satisfies the fairness condition in \eqref{eq:fairness-ltl} for $v$. 
On the other hand, if $v \in V'_\Odd$ does not lie on a cycle in $\mathcal{S}$, then by \eqref{equ:alpha:a} any such $\rho$ sees $v$ at most once. Thus $\rho$ trivially satisfies \eqref{eq:fairness-ltl} for $v$. 
This observation is stated in the following proposition.


\begin{proposition}
 Given the premisses of Def.~\ref{def:compliantstrat} let $\pi$ be a play starting from a node in $V'$ that complies with $\rho$. Then $\pi \models \alpha$ where $\alpha$ if the LTL formula in~\eqref{eq:fairness-ltl}.%\vspace{-2mm}
\end{proposition}

Next, we define \Even strategy templates. Each \Even strategy template encodes a unique \Even positional strategy, which is known to exist in \Odd-fair parity games \cite{banerjee2022fast}, due to the lack of fair edges defined on \Even vertices. %, \Even strategy templates are very simple\footnote{In fact, \Even strategy templates simply encode a positional strategy and are only re-defined to make further arguments more symmetric for both players.}.
\begin{definition}\label{def:Evenstrategytemplate}
    Given an \Odd-fair parity game $\mathcal{G}^\ell = \ltup{\mathcal{G}, E^\ell}$ with \newline $\mathcal{G} = \langle V, \Ve, \Vo, E, \chi\rangle$, an \Even \emph{strategy template} $\mathcal{S}$ over $\mathcal{G}^\ell$ is a subgraph of $\mathcal{G}$ given as $\mathcal{S}:=\tup{V', E'}$ where $V'\subseteq V$ and $E'\subseteq E \cap (V' \times V')$ such that,    \begin{compactitem}\label{item:Evenstrtemprules}
     \item if $v \in \Ve \cap V'$, then $|E'(v)|=1$,
     \item if $v \in \Vo \cap V'$, then  $E'(v) = E(v)$.
    \end{compactitem}
\end{definition}

\vspace*{-0.1cm}

An \Even strategy $\rho$ is compliant with the \Even strategy template $\mathcal{S} = \tup{V', E'}$ if for all $v \in V'_\Even$, $\rho(v) = E'(v)$. In other words, $\rho$ is the positional strategy defined by $\mathcal{S}$.

Let $\rho$ be an \Odd (\Even) strategy, compliant with the \Odd (\Even) strategy template $\mathcal{S}$ and let $\pi$ be a play compliant with $\rho$. Then we call $\pi$ a play \emph{compliant with $\mathcal{S}$}.

\vspace*{-0.1cm}

\begin{definition}
An \Odd (\Even) strategy template $\mathcal{S}=\ltup{V', E'}$ is \emph{winning} in the \Odd-fair parity game $\mathcal{G}^\ell$ if all \Odd (\Even) strategies $\rho$ compliant with $\mathcal{S}$ are winning for player \Odd (\Even) in $\mathcal{G}^\ell$ from $V'$. A winning \Odd (\Even) strategy template $\mathcal{S}$ is called \emph{maximal} if $V'=\Wo$ ($\We$).%\vspace{-2mm}
\end{definition}

\vspace*{-0.2cm}
We note that maximal winning \Odd (\Even) strategy templates $\mathcal{S}$ immediately imply that for every vertex $v\in \Wo$ ($\We$) there exists a winning strategy for player \Odd (\Even) from $v$ that is compliant with $\mathcal{S}$.
% 
The existence of maximal winning \Even strategy templates follows from the existence of positional \Even strategies \cite{banerjee2022fast}. 
% 
The first main contribution of this paper is a constructive proof showing the existence of maximal winning \Odd strategy templates given in the next section. 
This result is then used in Sec.~\ref{sec:zielonka} to prove the correctness of \Odd-fair Zielonka's algorithm, which is introduced there.



 % \onecolumn
\section{Characterization of Strong Duality}\label{sec:strongduality}
\begin{comment}
\textit{Remark}: The proof here works for finite-horizon setting as well.
\end{comment}

To solve \eqref{eq:macpomdp}, let us define the Lagrangian function $L : \uspace\times \mcl{Y} \mapsto \mbb{R} \cup \{\infty\}$ as follows.
\begin{align*}
\lags{u}{\lambda} &= \lag{u}{\lambda} 
\defeq \fullccosts{u} + \dotp{\lambda}{\fulldcosts{u} -\constraintv}\\
&= \fullccosts{u} + \sum_{k=1}^{K} \lambda_k \l(D_{k}\l(u\r) - \constraintv_k \r)
,\numberthis\label{eq:lagrangian}
\end{align*}
Here, $\mcl{Y} \defeq \{ \lambda \in \mbb{R}^K : \lambda \ge 0\}$ is the set of tuples of $K$ non-negative real-numbers, each commonly known as a Lagrange-multiplier. Our first result shows that the the solution $\udl{C}$ satisfies
\begin{align*}
\optcosts &= \infsup{u\in \uspace }{\lambda\in \mcl{Y}} \lags{u}{\lambda}
,\numberthis\label{eq:optccost:infsup}
\end{align*}
and that the inf and sup can be interchanged, i.e.,
\begin{align*}
\optcosts &= \supinf{\lambda\in \mcl{Y}}{u\in \uspace } \lags{u}{\lambda}
.\numberthis\label{eq:optccost:supinf}
\end{align*}

\begin{thm}[Strong Duality and Existence of Saddle Point]\label{thm:strongduality}
Under Assumptions \ref{assmp:boundedcosts}(a) and \ref{assmp:boundedcosts}(c), the following statements hold.
\begin{enumerate}
\item[(a)] The optimal value satisfies 
\begin{align*}
\optcosts = \infsup{u\in\uspace}{\lambda\in \mcl{Y}} \lags{u}{\lambda}
.\numberthis   
\end{align*}
\item[(b)] A policy-profile $u^\star \in \uspace$ is optimal if and only if $\optcosts = \sup_{\lambda\in \mcl{Y}} \lags{u^\star}{\lambda}$.
\item[(c)] Strong duality holds for \eqref{eq:macpomdp}, i.e.,
\begin{align*}
\optcosts &= \infsup{u\in\uspace}{\lambda\in \mcl{Y}} \lags{u}{\lambda} 
= \supinf{\lambda\in \mcl{Y}}{u\in\uspace} \lags{u}{\lambda}.\numberthis
\end{align*}
Moreover, there exists a $u^\star \in \uspace$ such that $\optcosts = \sup_{\lambda \in \mcl{Y}} \lags{u^\star}{\lambda} $ and $u^\star$ is optimal for \eqref{eq:macpomdp}. 
\item[(d)] If Assumption \ref{assmp:slatercondition} holds, then there also exists $\lambda^\star \in \mcl{Y}$ such that the following saddle-point condition holds for all $(u,\lambda)\in \uspace \times \mcl{Y}$,
\begin{align*}
\lags{u^\star}{\lambda} \le \lags{u^\star}{\lambda^\star} = \optcosts \le \lags{u}{\lambda^\star}.
\numberthis\label{eq:saddlepointconditions} 
\end{align*}
i.e., $u^\star$ minimizes $\lags{\cdot}{\lambda^\star}$ and $\lambda^\star$ maximizes $\lags{u^\star}{\cdot}$. In addition to this, the primal dual pair $\l( u^\star, \lambda^\star \r)$ satisfies the complementary-slackness condition:
\begin{align*}\label{eq:compslack}
\dotp{\lambda^\star}{\fulldcosts{u^\star}-\constraintv} = 0.\numberthis
\end{align*}
\end{enumerate}
\end{thm}

\begin{proof}
\begin{enumerate}
\item[(a)] If $u \in \uspace$ is feasible (i.e., it satisfies $\fulldcosts{u} \le \constraintv$), then the $\sup$ is obtained by choosing $\lambda = 0$, so
\begin{align*}
\sup_{\lambda\in\mcl{Y}} \lags{u}{\lambda} &= \fullccosts{u}.
\numberthis\label{eq:ufeasible}
\end{align*}
If $u \in \uspace$ is not feasible, then %it is easily seen that
\begin{align*}
\sup_{\lambda\in \mcl{Y}} \lags{u}{\lambda} = \infty.
\numberthis\label{eq:unotfeasible}
\end{align*}
Indeed, suppose WLOG that the $k^{th}$ constraint is violated, i.e., $\fulldkcosts{k}{u} > \constraintv_k$, then $\infty$ can be obtained by choosing $\lambda_k$ arbitrarily large and setting other $\lambda_k$'s to 0). From \eqref{eq:ufeasible}, \eqref{eq:unotfeasible}, and our convention that $\optcosts = \infty$ whenever the feasible-set is empty, it follows that
\begin{align*}
\optcosts = \infsup{u\in\uspace}{\lambda\in\mcl{Y} } \lags{u}{\lambda}.
\numberthis\label{eq:fullccostisinfsup}
\end{align*}

\item[(b)] By our convention on the value of $\optcosts$ (when there is no feasible policy-profile), $u^\star$ is optimal if and only if $\fullccosts{u^\star} = \optcosts $, i.e., $\sup_{\lambda\in \mcl{Y} } \lags{u^\star}{\lambda} = \optcosts$.

\item[(c)] To establish strong duality, we use a Minimax Theorem (see Proposition \ref{prop:sionminimax}) which requires $\uspace$ and $\mcl{Y}$ to be convex\footnote{Convexity is a set property rather than a topological property. In the rest of the paper, by a ``convex topological space'', we mean convexity of the set on which the topology is defined.} topological spaces (with $\uspace$ being compact also). It is clear that $\mcl{Y}$ is convex and we can endow it with the usual subspace topology of $\mbb{R}^K$. 
For $\uspace$ however, we need to endow it with a suitable topology in which it is both convex and compact. To achieve compactness, we can use the finiteness of joint-action space $\aspace$ and the countability of joint-observation space $\ospace$ to associate $\uspace$ with a product of compact sets that are parameterized by (countable number of) all possible histories. Tychonoff's theorem (see Proposition \ref{prop:tychonoff}) then helps establish compactness under the product topology. 
(Convexity holds trivially). 
Now, we make this idea precise. For $t \in \mbb{N}$ and $n\in[0,N]_{\mbb{Z}}$, let $\hstnspace{t}{n}$ denote the set of all possible realizations of $\Hstn{t}{n}$. Then, by countability of observation and action spaces, the sets
\begin{align}
\begin{split}\label{eq:hthnandh}
\hstspace{t} &\defeq \prod_{n=0}^{N} \hstnspace{t}{n},\\
\hsnspace{n} &\defeq \bigcup_{t= 1}^{\infty} \hstnspace{t}{0} \times \hstnspace{t}{n}, \text{ and }\\
\hsspace &\defeq \bigcup_{t=1}^{\infty} \hstspace{t},
\end{split}
\end{align}
are countable. Here, $\hstspace{t}$ is the set of all possible joint-histories at time $t$, $\hsnspace{n}$ is the set of all possible histories of agent $n$, and $\hsspace$ is the set of all possible joint-histories. With this in mind, one observes that $\uspace$ is in one-to-one correspondence with the set $\xuspace \defeq \prod_{n=1}^{N} \xuspacen{n}$, where
\begin{align*}
\xuspacen{n} \defeq \prod_{h \in \hsnspace{n}} \m{\anspace{n}; h},\numberthis\label{eq:xuspace}
\end{align*}
and $\m{\anspace{n}; h}$ is a copy of $\m{\anspace{n}}$ dedicated for agent-$n$'s history $h$. For example, a given policy $u$ would correspond to a point $x\in \xuspace$ such that $x_{n, \l( \hstn{h}{t}{0}, \hstn{h}{t}{n}\r) } = \utn{u}{t}{n} \l( \cdot | \hstn{h}{t}{0}, \hstn{h}{t}{n} \r) $, and similarly, vice versa. 

\hspace{5pt} Since $\anspace{n}$ is a complete separable (compact) metric space, by Prokhorov's Theorem (see Proposition \ref{prop:prokhorov}), each $\m{\anspace{n}; h}$ is a compact (and convex\footnote{Convexity of $\m{\anspace{n}}$ is trivial.}) metric space (with the topology of weak-convergence). Therefore, endowing $\xuspacen{n}$ and $\xuspace$ with the product topology makes each a compact (and convex) metric space via Tychonoff's theorem (see Proposition \ref{prop:tychonoff}), which is also metrizable (via Proposition \ref{prop:metrizability}). Given the one-to-one correspondence, \textbf{from now onward, we assume that $\uspacen{n}$ and $\uspace$ have the same topology as that of $\xuspacen{n}$ and $\xuspace$ respectively}. Henceforth, we will consider $C$, $D_k$, and $L$ as functions on topological spaces. Furthermore, since  $\uspacen{n}$'s and $\uspace$ have been shown to be compact metric spaces (hence, also complete and separable), we can also define $\borel{\uspacen{n}}$, $\borel{\uspace} = \otimes_{n=1}^{N} \borel{\uspacen{n}}$\footnote{For separable metric spaces $\mcl{W}_1, \mcl{W}_2, \ldots$, $\borel{\mcl{W}_1 \times \mcl{W}_2 \times \ldots } = \borel{\mcl{W}_1} \otimes \borel{\mcl{W}_2} \otimes \ldots$. See \cite{kallenberg2002foundations}[Lemma 1.2].}, and $\m{\uspace}$, where $\m{\uspace}$ is compact (and convex) metrizable space by Prokhorov's theorem (see Proposition \ref{prop:prokhorov}).

\hspace{5pt} To establish part (c), it will be helpful to work with (decentralized) mixtures of behavioral policy-profiles -- wherein each agent first uses a measure $\mun{n} \in \m{\uspacen{n}} $ to choose its policy-profile $\un{u}{n}$ and then proceeds with it from time 1 onward. We denote this set of mixtures by $\uspacemix \defeq \prod_{n=1}^{N} \m{\uspacen{n}}$, whose typical element, denoted by $\mu \defeq \mymathop{\times}_{n=1}^{N} \mun{n} $, is a factorized measure on $\uspace$, i.e., $\mun{n}\in \m{\uspacen{n}}$. Since $\uspacemix \subseteq \m{\uspace}$, we endow it with the same metric as that of $\m{\uspace}$. Now, we can extend the definitions of $C$, $D$, and $L$ to $\wh{C} : \uspacemix \ra \mbb{R} \cup \{\infty\}$, $\wh{D}: \uspacemix \ra \mbb{R}^K$, and $\wh{L}: \uspacemix \times \mcl{Y} \ra \mbb{R} \cup \{\infty\}$ as follows:		
\begin{align}		
\begin{split}\label{eq:lagrangianmix}		
\wh{C} (\mu) &= \wh{C}_{P_1}(\mu) \defeq \E{\mu}{P_1} \l[ \sum_{t=1}^{\infty} \alpha^{t-1} c(S_t, A_t) \r], \\		
\wh{D} (\mu) &= \wh{D}_{P_1}(\mu) \defeq \E{\mu}{P_1} \l[ \sum_{t=1}^{\infty} \alpha^{t-1} d(S_t, A_t) \r], \text{ and }\\		
\lagsmix{\mu}{\lambda} &= \lagmix{\mu}{\lambda} = \wh{C}(\mu) + \dotp{\lambda}{\wh{D}(u)}.		
\end{split}		
\end{align}
In Lemma \ref{lem:dominance}, it is shown that any $\mu \in \uspacemix$ can be replicated by a behavioral policy-policy $u \in \uspace$. Corollary \ref{cor:lbar_and_l} then shows that
\begin{align}
\begin{split}\label{eq:lag_and_lagmix}
\infsup{u\in\uspace}{\lambda\in\mcl{Y}} \lags{u}{\lambda} &= \infsup{\mu \in\uspacemix}{\lambda\in\mcl{Y}} \lagsmix{\mu}{\lambda}, \text{ and } \\
\supinf{\lambda\in\mcl{Y}}{u\in\uspace} \lags{u}{\lambda} &= \supinf{\lambda\in\mcl{Y}}{u\in\uspacemix} \lagsmix{\mu}{\lambda}.
\end{split}
\end{align}
In light of \eqref{eq:lag_and_lagmix}, it suffices to prove part (c) for $\wh{L}$. By definition, $\wh{L}$ is affine and thus trivially concave in $\lambda$. Proposition \ref{prop:integral_linearity} implies that $\wh{L}$ is convex in $\mu$ and Lemma \ref{lem:lsc2} shows that $\wh{L}$ is lower semi-continuous\footnote{For definition of lower semi-continuity, see Definition \ref{dfn:lsc}.} in $\mu$. From Proposition \ref{prop:sionminimax}, it then follows that
\begin{align*}
\infsup{u\in\uspacemix}{\lambda\in \mcl{Y}} \lagsmix{u}{\lambda} = \supinf{\lambda\in \mcl{Y}}{\uspacemix} \lagsmix{u}{\lambda},
\end{align*}
and that there exists $\mu^\star \in \uspacemix$ such that
\begin{align*}	
\sup_{\lambda\in \mcl{Y}} \lagsmix{\mu^\star}{\lambda} = \infsup{u\in\uspacemix}{\lambda\in \mcl{Y}} \lagsmix{u}{\lambda}.	
\end{align*}
The optimality of $\mu^\star$ is implied by parts (b) and (a).
\item[(d)] This follows from Lagrange-multiplier theory.
\end{enumerate}
This concludes the proof.
\end{proof}


\begin{lem}[Lower Semi-Continuity of $\wh{L}$ on $\uspacemix$]\label{lem:lsc2}
Under Assumptions \ref{assmp:boundedcosts}(a) and \ref{assmp:boundedcosts}(c), $\wh{L}$ is lower semi-continuous on $\uspacemix$.
\end{lem}
\begin{proof}
Fix $\lambda\in\mcl{Y}$ and $\mu\in \uspacemix$. Let $\l\{ \mu_i \r\}_{i=1}^{\infty}$ be a sequence of (factorized) measures in $\uspacemix$ that converges to $\mu \in \uspacemix$. Since $\uspacemix \subseteq \m{\uspace}$ and has the same metric as $\m{\uspace}$, it means that $\l\{ \mu_i \r\}_{i=1}^{\infty}$ also converges to $\mu$ in $ \m{\uspace}$. We want to show
\begin{align*}
\liminf_{i\ra\infty} \E{U \sim \mu_i}{P_1} \l[  \lags{U}{\lambda} \r] \ge \E{U \sim \mu}{P_1} \l[  \lags{U}{\lambda} \r].
\end{align*}
By Lemma \ref{lem:lsc}, $L$ is point-wise lower semi-continuous on $\uspace$. Therefore, Proposition \ref{prop:lsc} applies on $\m{\uspace}$ and the above inequality follows.
\end{proof}


\begin{lem}[Lower Semi-Continuity of $L$ on $\uspace$]\label{lem:lsc}
Under Assumptions \ref{assmp:boundedcosts}(a) and \ref{assmp:boundedcosts}(c), the functions $C$ and $D_k$'s are lower semi-continuous on $\uspace$. Hence, $L$ is lower semi-continuous on $\uspace$. 
\end{lem}
\begin{proof}
We will prove the statement for $C$. The proof of lower semi-continuity of $D_k$'s is similar. For brevity, let 
\begin{align*}
\pruphsts{u}{t}{\hst{h}{t}, \at{t}} &= \pruphst{u}{t}{\hst{h}{t}, \at{t}} \defeq \prup{u}{P_1}\l(\Hst{t} = \hst{h}{t}, \At{t} = \at{t}\r),
\\
\zuphsts{u}{t}{\hst{h}{t}, \at{t}} &= \zuphst{u}{t}{\hst{h}{t}, \at{t}}\\
&\hspace{-25pt} \defeq \pruphsts{u}{t}{\hst{h}{t}, \at{t}} \mbb{E}_{P_1}\l[ \cCost | \Hst{t} = \hst{h}{t}, \At{t} = \at{t} \r],
\end{align*}
where we use the convention $0 \cdot \infty = 0$. Then,
\begin{align*}
\fullccosts{u} &= \E{u}{P_1}\l[ \sum_{t=1}^{\infty} \alpha^{t-1} c(S_t, A_t) \r] \\
&= \E{u}{P_1}\l[ \sum_{t=1}^{\infty} \alpha^{t-1} \l( c(S_t, A_t) - \udl{c} \r) \r] + \sum_{t=1}^{\infty} \alpha^{t-1} \udl{c}\\
&\labelrel{=}{eqr:cp1u:a} \sum_{t=1}^{\infty} \alpha^{t-1} \E{u}{P_1} \l[ c(S_t, A_t) - \udl{c} \r] + \sum_{t=1}^{\infty} \alpha^{t-1} \udl{c}\\
&= \sum_{t=1}^{\infty} \alpha^{t-1} \E{u}{P_1} \l[ c(S_t, A_t) \r] \\
&\labelrel{=}{eqr:cp1u:b} \sum_{t=1}^{\infty} \alpha^{t-1} \E{u}{P_1} \l[ \mbb{E}_{P_1} \l[ c(S_t, A_t) | \Hst{t}, \At{t} \r] \r]\\
&= \sum_{t=1}^{\infty} \sum_{\hst{h}{t} \in \hstspace{t}} \sum_{\at{t}\in \aspace} \alpha^{t-1} \zuphsts{u}{t}{\hst{h}{t}, \at{t}}.
\end{align*}
Here, \eqref{eqr:cp1u:a} follows from applying the Monotone-Convergence Theorem to the (non-decreasing and non-negative) sequence $\{ \sum_{t=1}^{i} \alpha^{t-1} \l( c\l( \Stt{t}, \At{t} \r) - \udl{c} \r) \}_{i=1}^{\infty}$ (see Proposition \ref{prop:mct}); and \eqref{eqr:cp1u:b} uses the tower property of conditional expectation.\footnote{The conditional expectations $\mbb{E}_{P_1} \l[ c(S_t, A_t) | \Hst{t}, \At{t} \r]$ exist and are unique because $c(\cdot, \cdot)$ is bounded from below.}


Let $\l\{ \useq{i}{u} \r\}_{i=1}^{\infty}$ be a sequence in $\uspace$ that converges to $u$. By Fatou's Lemma (see Proposition \ref{prop:fatou}),
\begin{align*}
\liminf_{i\ra \infty} \fullccosts{\useq{i}{u}} \ge \sum_{t=1}^{\infty} \sum_{\hst{h}{t} \in \hstspace{t}} \sum_{\at{t}\in \aspace} \alpha^{t-1} \liminf_{i\ra\infty} \zuphsts{\useq{i}{u}}{t}{\hst{h}{t}, \at{t}}.\numberthis\label{eq:step1}
\end{align*}
Following Lemma \ref{lem:puth}, $\pruphsts{\useq{i}{u}}{t}{\hst{h}{t}, \at{t}} $ converges to $\pruphsts{u}{t}{\hst{h}{t}, \at{t}}$. Therefore,
\begin{align*}
\lim_{i\ra\infty} \zuphsts{\useq{i}{u}}{t}{\hst{h}{t}, \at{t}} = \zuphsts{u}{t}{\hst{h}{t}, \at{t}}.\numberthis\label{eq:step2}
\end{align*}
From \eqref{eq:step1} and \eqref{eq:step2}, it follows that
\begin{align*}
\liminf_{i\ra \infty} \fullccosts{\useq{i}{u}} \ge \fullccosts{u},
\end{align*}
which establishes the lower semi-continuity of $\fullccosts{u}$. 
\end{proof}
\section{Planning Using Common-Information Approach and Approximate-Information States}\label{sec:history_embedding}
Theorem \ref{thm:strongduality} provides firm theoretical support for primal-dual type planning and learning algorithms for a given MA-C-POMDP. Indeed, given the optimal Lagrange-multipliers vector $\lambda^\star$, MA-C-POMDP simply reduces to a MA-POMDP\footnote{Except the fact that randomization amongst equally valuable actions cannot be ignored, in general.}, so essentially all MA-POMPD algorithms apply for gradient descent in the primal space $\uspace$. However, one must find %the correct Lagrangian multipliers 
$\lambda^\star$. In light of the first inequality in \eqref{eq:saddlepointconditions}, we can do this by a projected gradient ascent in the dual space $\mcl{Y}$ -- on a slower time-scale so that it sees the minimization over the primal space $\uspace$ as having essentially equilibrated. In this section, we will assume that Assumption \ref{assmp:boundedcosts} ((a)-(c)) holds.

In this section, we shall first review the common-information approach \cite{nayyar13,nayyar14} that transforms a given MA-POMDP into an equivalent SA-POMDP. We will then use insights from the resulting SA-POMDP in order to derive a compression-framework for approximately solving the original MA-C-POMDP. This framework will be an extension of \cite{hsu22} the details of which will be left as an exercise for the reader. Nevertheless, an important goal will be achieved via this exercise: the approximation criteria of the compression-framework will be independent of the Lagrange-multipliers vector $\lambda$. This property will be essential in the learning context where we would like the learning of the compression-mapping to be independent of $\lambda$ (since $\lambda$ needs to be learned as well). Note that if we, instead, directly followed the approach of \cite{hsu22}, then for each value of the Lagrange-multipliers vector $\lambda$, we would have to find a new compression-mapping, and then adapt it as $\lambda$ is changed.

To achieve optimality gaps for the said compression-framework, we will first consider \eqref{eq:macpomdp} over a finite-horizon $T<\infty$, and then (with the aid of Assumption \ref{assmp:boundedcosts}) let $T$ go to infinity. Before, we proceed further, we present a simple lemma that uses Assumption \ref{assmp:slatercondition} to get an upper-bound on $\lambda^\star$ -- the existence and search of which can therefore be restricted to a compact cube in $\mbb{R}^K$. (This shall enable us to get $\lambda$-independent optimality-gaps for the compression-framework: see Remark \ref{rem:universalbound_optimalitygap}).  

\begin{lem}\label{lem:upperbound_lambda}
Under Assumptions \ref{assmp:boundedcosts}(a) and \ref{assmp:slatercondition}, the optimal Lagrange-multipliers vector $\lambda^\star$ is upper-bounded as follows:
\begin{align*}
\| \lambda^{\star} \|_{\infty} \le \| \lambda^{\star} \|_1 \le \frac{1}{\zeta} \l(  \fullccosts{\ov{u}} - \frac{\udl{c}}{1-\alpha} \r).
\end{align*}
\end{lem}

\begin{proof}
We have the following sequence of inequalities:
\begin{align*}
\sum_{t=1}^{\infty} \alpha^{t-1} \udl{c} &\labelrel{\le}{eqr:lamda:assmp:boundedcosts}  \optcosts = \lags{u^\star}{\lambda^\star} \\
&\hspace{-0pt} \labelrel{\le}{eqr:lamda:saddlepoint} \lags{\ov{u}}{\lambda^\star}\\
&\hspace{-0pt} = \fullccosts{\ov{u}} + \dotp{\lambda^\star}{\fulldcosts{\ov{u}}-\constraintv}\\
&\hspace{-0pt} \labelrel{\le}{eqr:lamda:assmp:slater} \fullccosts{\ov{u}} + \dotp{\lambda^\star}{-\zeta1}\\
&\hspace{-0pt} \labelrel{=}{eqr:lamda:positive} \fullccosts{\ov{u}} - \zeta \|\lambda^\star\|_1 \\
&\hspace{-0pt}\le \fullccosts{\ov{u}} - \zeta \|\lambda^\star\|_{\infty}.
\end{align*}
Here, \eqref{eqr:lamda:assmp:boundedcosts} uses Assumption \ref{assmp:boundedcosts}; \eqref{eqr:lamda:saddlepoint} uses the second inequality in \eqref{eq:saddlepointconditions}; \eqref{eqr:lamda:assmp:slater} uses Assumption \ref{assmp:slatercondition}; and \eqref{eqr:lamda:positive} uses the fact that $\lambda^{\star}$ is non-negative. %\textcolor{red}{Move proof to Appendix.}
\end{proof}


\subsection{Common-Information Approach}
The common-information approach \cite{nayyar13,nayyar14} shows that a (cooperative) MA-POMDP can be converted into an equivalent SA-POMDP---called the \textit{coordinated-system}. In this system, the single agent, called the \emph{coordinator}, is a virtual entity that has access to the common observations $\Otn{t}{0}$, and does not get to see the agents' private observations and actions $( \Otn{t}{1:N}, \Atn{t}{1:N} ) \setminus \Otn{t}{0}$. Therefore, from the perspective of the coordinator, the unknown state is the environment's state combined with the private histories of all agents, i.e., $( \Stt{t}, \Hstn{t}{1:N} )$. 

At time $t \in \mbb{N}$, via a \emph{coordination policy} (to be defined later), the coordinator uses the common-history $\Hstn{t}{0}$ to \udl{deterministically} chooses an action $ \Gt{t} = \Gtn{t}{1:N} $\footnote{As $\Gt{t}$ depends deterministically on the common-history, all agents can replicate it with consensus.}, where $\Gtn{t}{n}$ maps $ \hstnspace{t}{n}$ to $ \m{\anspace{n}} $ and is designed to be an enforcing \textit{prescription} for agent-$n$---agent-$n$ applies $\Gtn{t}{n}$ to its private history $\Hstn{t}{n}$ and then draws its action $\Atn{t}{n} \sim \Gtn{t}{n} ( \Hstn{t}{n})$. We use $\gtspace{t}$ to denote the set of all possible prescriptions at time $t$, i.e., $\gtspace{t} = \prod_{n=1}^{N} \gtnspace{t}{n}$ and $\gtnspace{t}{n} \defeq \{ \gtn{t}{n} : \hstnspace{t}{n} \ra \m{\anspace{n}} \}$. We note that $\gtspace{t}$ is in one-to-one correspondence with the set
\begin{align*}
    \xtspace{\gtspace{t}} \defeq \prod_{n=1}^{N} \prod_{ \hstn{h}{t}{n} \in \hstnspace{t}{n} } \m{\anspace{n}; \hstn{h}{t}{n} }.\numberthis\label{eq:xgtspace}
\end{align*}
which is a compact (and convex) space by Tychonoff's theorem (see Proposition \ref{prop:tychonoff}), and is also metrizable (see Proposition \ref{prop:metrizability}). Henceforth, we will assume $\gtspace{t}$ to have the same topology as $\xtspace{\gtspace{t}}$.
\begin{rem}
To help achieve equivalence between the coordinated system and MA-C-POMDP (which uses decentralized policy-profiles only), we have restricted the coordinator to choose prescriptions in a deterministic manner---no randomization over the elements of 
% the spaces of prescriptions 
$\xtspace{\gtspace{t}}$'s. 
\end{rem}
Let $\wtHstn{t}{0}$ denote the prescription-observation history of the coordinator, i.e.,
\begin{align}
\begin{split}\label{eq:aoh:coordinator}
\wtHstn{1}{0} &\defeq \Otn{1}{0} \text{ and } \\
\wtHstn{t}{0} &\defeq \l( \wtHstn{t-1}{0}, \Gt{t-1}, \Otn{t}{0} \r) \text{ for all } t\in[2,\infty]_{\mbb{Z}}.
\end{split}
\end{align}
Then we can define a coordination policy $v$ as a tuple $\ut{v}{1:\infty} \in \vvspace$ where $\ut{v}{t}$ maps $\wthstnspace{t}{0}$ to $\gtspace{t}$ and where agent $n$ draws its action $\Atn{t}{n}$ according to the distribution $\Gtn{t}{n}( \Hstn{t}{n} ) = ( \ut{v}{t}( \wtHstn{t}{0} ) )^{(n)}( \Hstn{t}{n} )$. 
%With this setup, of deterministic coordination policy (with stochastic prescriptions), 
Note that in this setup, each $\wtHstn{t}{0}$ is some deterministic function of $\Hstn{t}{0}$. Therefore, we can replace $\ut{v}{t} ( \wtHstn{t}{0} )$ by $\ut{\wt{v}}{t} ( \Hstn{t}{0} )$. One then establishes equivalence between the coordinated system and MA-C-POMDP by setting 
\begin{align*}
\utn{u}{t}{n} \l( \hstn{h}{t}{0}, \hstn{h}{t}{n} \r) 
&= \l( \ut{v}{t} \l(  \hstn{\wt{h}}{t}{0} \r) \r)^{(n)} \l( \hstn{h}{t}{n} \r) \\
&= \l( \wt{v}_t  \l(  \hstn{h}{t}{0} \r) \r)^{(n)} \l( \hstn{h}{t}{n} \r).
\end{align*}
In light of the above equivalence and the strong duality result of Theorem \ref{thm:strongduality}, from now onward, we will restrict our focus to deriving optimal and approximately optimal coordination policies for coordinated-systems whose immediate-costs are parametrized by 
%the Lagrange-multipliers vector 
$\lambda\in\mcl{Y}$, namely $\un{l}{\lambda}: \sspace \times \aspace \ra \mbb{R}$, where
\begin{align*}
\un{l}{\lambda}\l(s,a\r) &\defeq c\l(s,a\r) + \dotp{\lambda}{d\l(s,a\r) -\constraintv}.\numberthis\label{eq:l_lamda}
\end{align*}
Also, in line with our aforementioned plan, we parametrize the aggregate discounted costs by horizons $T \in \mbb{N} \cup \{ \infty \}$, namely $L_T: \vvspace \times \mcl{Y} \ra \mbb{R}$, where
\begin{align*}
    % L_T &: \vvspace \times \mcl{Y} \ra \mbb{R},\\ 
    L_T \l( v, \lambda \r) = L_T^{(P_1, \alpha)} \l( v, \lambda \r) &\defeq \E{v}{P_1} \l[ \sum_{t=1}^{T} \alpha^{t-1} \lCost \r].\numberthis\label{eq:L_T}
\end{align*}
With the above setup in place, for a (finite) horizon $T \in \mbb{N}$ and a fixed Lagrange-multiplier $\lambda\in\mcl{Y}$, one has the following dynamic program (Algorithm \ref{alg:cidecomposition}) to find the optimal coordination policy for the objective $L_T$ (see \cite{nayyar13,nayyar14}). We note that the usage of $\min$ and $\argmin$ in \eqref{eq:V_tT_lamda} and \eqref{eq:v_tT_lamda_star} is justified due to compactness of $\gtspace{t}$, and the choice of prescription in \eqref{eq:v_tT_lamda_star} is arbitrary. 

%---------------------------------------------------------%
%-----ALGORITHM FOR FULL COMMON AND PRIVATE HISTORIES-----%
%---------------------------------------------------------%
\begin{algorithm}[H]
\DontPrintSemicolon
\KwInput{Time-horizon $T \in \mbb{N}$, Discount-factor $\alpha\in(0,1]$, MA-C-POMDP Model (see Section \ref{sec:problem}).}
%%%%%%%%%%%%%%%%%%%%%%%%%%%%%%%%%%%%%%
\Parameter{Lagrange-multipliers vector $\lambda \in \mcl{Y}$.}
\KwOutput{$\utn{v}{1:T}{\lambda,\star}$ determined by $\l\{ \Qtl{t,T}{\lambda} : \wthstnspace{t}{0} \times \gtspace{t} \ra \mbb{R} \r\}_{t=1}^{T}$.
}

%%%%%%%%%%%%%%%%%%%%%%%%%%%%%%%%%%%%%%
\nonl $V_{T+1,T}^{(\lambda)} \equiv 0$.

\nonl \For{$t = T, T-1, \dots, 1$}{
\nonl \begin{align*}
% \nonl
% $    
&\Qtl{t,T}{\lambda} \l( \hstn{\wt{h}}{t}{0}, \gt{t} \r) 
= \mbb{E} \l[ \lCost \r. \\
&\hspace{10pt} \l. 
+ \alpha \Vtl{t+1,T}{\lambda} \l( \wtHstn{t+1}{0} \r)  \mid \wtHstn{t}{0} = \hstn{\wt{h}}{t}{0}, \Gt{t} = \gt{t} \r].\numberthis\label{eq:Q_tT_lamda}\\
&\Vtl{t,T}{\lambda} \l( \hstn{\wt{h}}{t}{0} \r) = \min_{\gt{t} \in \gtspace{t} } \Qtl{t,T}{\lambda} \l( \hstn{\wt{h}}{t}{0}, \gt{t} \r).\numberthis\label{eq:V_tT_lamda}\\
&\utn{v}{t}{\lambda, \star} \l( \hstn{\wt{h}}{t}{0} \r) \in \argmin_{\gt{t} \in \gtspace{t}} \Qtl{t,T}{\lambda} \l( \hstn{\wt{h}}{t}{0}, \gt{t} \r).\numberthis\label{eq:v_tT_lamda_star}
% $
\end{align*}
}
\caption{Dynamic Programming with Full Common and Private Histories in Finite-Horizon setting.}\label{alg:cidecomposition}
\end{algorithm}
%---------------------------------------------------------%
%-----ALGORITHM FOR FULL COMMON AND PRIVATE HISTORIES-----%
%---------------------------------------------------------%
\begin{rem}\label{rem:necessaryonly}
    Akin to SA-C-MDP and SA-C-POMDPs, even when $\lambda^\star$ is known, finding an optimal policy for the unconstrained objective $L_T\l( \cdot, \lambda^*\r)$ does not necessarily imply solving the (finite-horizon version of the) original constrained optimization problem---because the coordination policy $\utn{v}{1:T}{\lambda, \star} $ obtained from \eqref{eq:v_tT_lamda_star} may not satisfy the constraints. However, from Theorem \ref{thm:strongduality}, we are guaranteed that an optimal coordination policy exists and it is also true that any such policy must choose a prescription from the set in the right-hand-side of \eqref{eq:v_tT_lamda_star}. It is hard to characterize how the optimal policy randomizes between these prescriptions, thus our choice of arbitrary selection in \eqref{eq:v_tT_lamda_star}. Having said that, this issue shall remain somewhat innocuous in the learning context where $\lambda$ will be continuously updated based on constraint violations. A similar remark would apply to the (approximate) dynamic program in Algorithm \ref{alg:dp_approximatestate}.
\end{rem}


From dynamic programming theory, it is known that $\Vtl{t,T}{\lambda} : \wthstnspace{t}{0} \ra \mbb{R}$ (see \eqref{eq:V_tT_lamda}) satisfies 
\begin{align*}
\Vtl{t,T}{\lambda} \l( \hstn{\wt{h}}{t}{0} \r) = \inf_{v\in\vvspace} \E{v}{P_1} \l[ \sum_{\tau=t}^{T} \alpha^{\tau - t} \un{l}{\lambda}\l(\Stt{\tau}, \At{\tau}\r) \Big| \wtHstn{t}{0} = \hstn{\wt{h}}{t}{0} \r].
\end{align*}
Therefore, the (finite-horizon) coordination policy $\utn{v}{1:T}{\lambda, \star}$ (see \eqref{eq:v_tT_lamda_star}) minimizes the (finite-horizon) objective $L_T$, i.e.,
\begin{align*}
    L_T\l( \utn{v}{1:T}{\lambda, \star}, \lambda \r) &= \inf_{v\in\vvspace} L_T\l( v, \lambda \r).
\end{align*}
\begin{rem}
The first argument of $L_T$, by definition (see \eqref{eq:L_T}), should be an element of $\vvspace$. However, since the specification of the policy for times $T+1$ and onward does not matter, the above equation is consistent (with slight abuse of notation). 
\end{rem}

Using Assumption \ref{assmp:boundedcosts}, we can now compare $\Vtl{t,T}{\lambda}$ (the optimal performance on a finite-horizon $T\ge t$) with the optimal performance on the infinite-horizon. Let us define value-functions $\{ \Vtl{t}{\lambda} : \wthstnspace{t}{0} \ra \mbb{R} \}_{t=1}^{\infty}$,
\begin{align*}
&\Vtl{t}{\lambda} \l( \hstn{\wt{h}}{t}{0} \r) \\
&\defeq \inf_{v\in \vvspace} 
\E{v}{P_1} \l[ \sum_{\tau=t}^{\infty} \alpha^{\tau-t} \un{l}{\lambda}\l(\Stt{\tau}, \At{\tau}\r) \Big| \wtHstn{t}{0} = \hstn{\wt{h}}{t}{0} \r],\numberthis\label{eq:V_t_lamda}
\end{align*}
and the corresponding prescription-value-functions $\{ \Qtl{t}{\lambda} : \wthstnspace{t}{0} \times \gtspace{t} \ra \mbb{R} \}_{t=1}^{\infty}$,
\begin{align*}
&\Qtl{t}{\lambda} \l( \hstn{\wt{h}}{t}{0}, \gt{t} \r) \defeq \mbb{E}_{P_1} \l[ \un{l}{\lambda}\l(\Stt{t}, \At{t}\r) \r. \\
&\hspace{10pt} \l. + \alpha \Vtl{t+1}{\lambda} \l( \wtHstn{t+1}{0} \r) \Big| \wtHstn{t}{0} = \hstn{\wt{h}}{t}{0}, \Gt{t} = \gt{t} \r].\numberthis\label{eq:Q_t_lamda}
\end{align*}
Then, the following bound on $\Vtl{t}{\lambda}$ with respect to $\Vtl{t,T}{\lambda}$ ($T\ge t$) holds.
\begin{prop}\label{prop:VtT_vs_Vt}
    Fix $\lambda\in\mcl{Y}$, $\alpha\in(0,1)$, and $t \in \mbb{N}$. Suppose Assumption \ref{assmp:boundedcosts} holds and consider a (finite) horizon $T\in [t, \cdot]$. Then, for any $\hstn{\wt{h}}{t}{0} \in \wthstnspace{t}{0}$, the following relation holds between $\Vtl{t,T}{\lambda} \l( \hstn{\wt{h}}{t}{0} \r) $ and $\Vtl{t}{\lambda} $.
    \begin{align*}
        &\Vtl{t,T}{\lambda} \l( \hstn{\wt{h}}{t}{0} \r) + \frac{\alpha^{T-t+1}}{1-\alpha} \un{\udl{l}}{\lambda} \le \Vtl{t}{\lambda} \l( \hstn{\wt{h}}{t}{0} \r) \\
        &\hspace{20pt} \le \Vtl{t,T}{\lambda} \l( \hstn{\wt{h}}{t}{0} \r) + \frac{\alpha^{T-t+1}}{1-\alpha} \un{\ov{l}}{\lambda},
    \end{align*}
    where
    \begin{align*}\numberthis\label{eq:lower_and_upper_lcost}
        \un{\udl{l}}{\lambda} \defeq \udl{c} + \dotp{\lambda}{\udl{d} - \constraintv } \text{ and }
        \un{\ov{l}}{\lambda} \defeq \ov{c} + \dotp{\lambda}{\ov{d} - \constraintv }
    \end{align*}
\end{prop}
\begin{proof}
    The proof can be established by backward induction. For brevity, it is omitted.
\end{proof}
With Proposition \ref{prop:VtT_vs_Vt}, we have a bound on the gap between $\Vtl{t}{\lambda}$ and $\Vtl{t,T}{\lambda}$ ($T\ge t$), that decays exponentially with $T$ (due to discounting). Therefore, for the time being, let us focus on the finite-horizon setup for which an optimal coordination policy can be found by Algorithm \ref{alg:cidecomposition}. There are, however, two key issues with Algorithm \ref{alg:cidecomposition}. Firstly, the domain of $\wtHstn{t}{0}$ grows exponentially in time and while one may compress $\wtHstn{t}{0}$ to a belief-state $\Pi_t \defeq \pr_{P_1} ( \Stt{t}, \Hstn{t}{1:N} \mid \wtHstn{t}{0} )  $ (without loss of optimality), 
% which is the conditional distribution of $\l( \Stt{t}, \Hstn{t}{1:N} \r) $ given the coordinator's prescription-observation history $\wtHstn{t}{0}$
the update of $\Pi_t$ requires knowledge of $\mcl{P}_{tr}$ (the transition-law) and $P_1$ (joint distribution of initial state and initial joint-observation) which are not available in the learning context. Secondly, and more importantly, the domain of the unobserved state $\Stt{t}, \Hstn{t}{1:N}$ grows exponentially with time and the number of agents---which leads to doubly-exponential growth of the coordinator's prescriptions. Due to these issues, Algorithm \ref{alg:cidecomposition} is generally not implementable, and thus remains conceptual. 

One means to address the above challenges is to compress the increasing common and private histories in such a way that when policies are restricted to take actions based on the compressed images of these histories, the performance is still approximately good. This is the goal in \cite{hsu22} which proposes one such compression-framework, and then characterizes its optimality-gap (as a function of the compression attributes). In the remainder of the section, we will extend the notions of \cite{hsu22} to the constrained setting, and as mentioned earlier, we will do so in a manner that the learning of the compression mapping shall remain independent of the Lagrange-multipliers vector $\lambda$. Also, it will be helpful to keep in mind that our ultimate goal will be to get a bound on the gap between $\Vtl{t,T}{\lambda}$ and $\whVtl{t,T}{\lambda}$ where $\whVtl{t,T}{\lambda}$ will be the optimal cost-to-go when coordination policies are restricted to use a prespecified compression framework.

\subsection{Approximate State Representations (Finite-Horizon)}
The framework in \cite{hsu22} uses a three-step process. It first compresses private histories from $\Hstn{t}{1:N} $ to an \textit{approximate sufficient private state (ASPS)}, which we will denote by $\Zhtn{t}{1:N} %\in \zhtnspace{t}{n} 
$. Compressing private histories to ASPS then changes the set of coordinator's prescriptions to a reduced set, the elements of which we will denote by $\lamdaht{t}$. Finally, the coordinator's prescription-observation history (now with the set of reduced prescriptions) is compressed from $\wtHstn{t}{0}$ to an \textit{approximate sufficient common state (ASCS)}, which we will denote by $\Zhtn{t}{0}$. With these three steps in mind, we now give formal definitions of ASPS and ASCS. (Note that ASCS relies on having an ASPS beforehand).

%---------------------------------------------------------%
%-----ASPS DEFINITION-----%
%---------------------------------------------------------%
\begin{dfn}[Finite-Horizon ASPS Generator]\label{dfn:asps_generator_finite_horizon}
Let $T\in\mbb{N}$ and let $\zhtnspace{1:T}{1:N}$ be a pre-specified collection of topological spaces.\footnote{%A complete normed vector space. 
We can assume each $\zhtnspace{t}{n}$ is a Euclidean space of some fixed dimension that can vary with $t$ and $n$, but a time-invariant domain is best in practice.} A collection $ \varthetahtn{1:T}{1:N} $ of compression-functions where $\varthetahtn{t}{n}$ maps $\hstnspace{t}{0} \times \hstnspace{t}{n}$ to $\zhtnspace{t}{n}$ is called a $T$-horizon $\l(\eps_{p,1}, \eps_{p,2}, \delta_p \r)$-ASPS generator if the process $\{ \Zhtn{t}{1:N} \}_{t=1}^{T}$, with $\Zhtn{t}{n} = \varthetahtn{t}{n}( \Hstn{t}{0}, \Hstn{t}{n} ) $ almost-surely, satisfies the following properties.
\begin{itemize}%[leftmargin=*]
\item[] \textbf{(ASPS1)} It evolves in a recursive manner: for each $n \in [N] $,
\begin{align*}
\Zhtn{t}{n}=\phihtn{t}{n} \l( \Zhtn{t-1}{n}, %\Gt{t-1}, 
\Otn{t}{0}, \l( \Otn{t}{n}, \Atn{t-1}{n}\r)\setminus \Otn{t}{0} \r).
\end{align*}

\item[] \textbf{(ASPS2.1)} It suffices for approximate prediction of objective-cost: for all $\l(\hstn{\wt{h}}{t}{0}, \hstn{h}{t}{1:N}, \at{t}\r) \in \wthstnspace{t}{0} \times \prod_{n=1}^{N} \hstnspace{t}{n} \times \aspace$,
\begin{align*}
&\l| \mbb{E}_{P_1} \l[  c\l( \Stt{t}, \At{t} \r) \Big| \hstn{\wt{h}}{t}{0}, \hstn{h}{t}{1:N}, \at{t} \r] - \r. \\
&\hspace{40pt} \l. \mbb{E}_{P_1} \l[ c\l( \Stt{t}, \At{t} \r) \Big| \hstn{\wt{h}}{t}{0}, \zhtn{t}{1:N}, \at{t} \r]  \r| \le \frac{\eps_{p,1}}{4}.
\end{align*}

\item[] \textbf{(ASPS2.2)} It suffices for approximate prediction of constraint cost: for all $\l(\hstn{\wt{h}}{t}{0}, \hstn{h}{t}{1:N}, \at{t}\r) \in \wthstnspace{t}{0} \times \prod_{n=1}^{N} \hstnspace{t}{n} \times \aspace$,
\begin{align*}
&\l\| \mbb{E}_{P_1} \l[  d\l( \Stt{t}, \At{t} \r) \Big| \hstn{\wt{h}}{t}{0}, \hstn{h}{t}{1:N}, \at{t} \r] - \r. \\
&\hspace{40pt} \l. \mbb{E}_{P_1} \l[ d\l( \Stt{t}, \At{t} \r) \Big| \hstn{\wt{h}}{t}{0}, \zhtn{t}{1:N}, \at{t} \r]  \r\|_{\infty} \le \frac{\eps_{p,2}}{4}.
\end{align*}

\item[] \textbf{(ASPS3)} It suffices for approximate prediction of observations: for all $\l(\hstn{\wt{h}}{t}{0}, \hstn{h}{t}{1:N}, \at{t}\r) \in \wthstnspace{t}{0} \times \prod_{n=1}^{N} \hstnspace{t}{n} \times \aspace$,
\begin{align*}
&\kappa \l( 
\mbb{P}_{P_1} \l( \Otn{t+1}{0:N} \mid \hstn{\wt{h}}{t}{0}, \hstn{h}{t}{1:N}, \at{t} \r), \r. \\ 
&\hspace{30pt} \l. \mbb{P}_{P_1} \l( \Otn{t}{0:N} \mid \hstn{\wt{h}}{t}{0}, \zhtn{t}{1:N}, \at{t} \r) 
\r)  \leq \frac{\delta_p}{8},
\end{align*}
where $\kappa\l(\cdot, \star \r)$ is the total variation distance between the two probability measures.
\end{itemize}
\end{dfn}

%---------------------------------------------------------%
%-----ASPS DEFINITION-----%
%---------------------------------------------------------%
Let us denote the range of $\varthetahtn{t}{n}$ by $\zhtnspaces{t}{n}$, i.e., $\zhtnspaces{t}{n} \defeq \varthetahtn{t}{n} \l( \hstnspace{t}{0} \times \hstnspace{t}{n} \r)$. Then the above definition induces a compression in the coordinator's prescriptions from $\Gt{t}$ to $\Lamdaht{t} = \Lamdahtn{t}{1:N}$ where $\Lamdahtn{t}{n}$ maps $\zhtnspaces{t}{n}$ (instead of $\hstnspace{t}{n}$) to $\m{\anspace{n}}$. We use $\lamdahtspace{t}$ to denote the set of all possible reduced prescriptions at time $t$, i.e., $\lamdahtspace{t} = \lamdahtnspace{t}{1:N}$ where $ \lamdahtnspace{t}{n} \defeq \{ \lamdahtn{t}{n} : \zhtnspaces{t}{n} \mapsto \m{\anspace{n}} \}$. We note that $\lamdahtspace{t}$ is in one-to-one correspondence with the set
\begin{align*}
    \xtspace{\lamdahtspace{t}} \defeq \prod_{n=1}^{N} \prod_{ \zhtn{t}{n} \in \zhtnspaces{t}{n} } \m{\anspace{n}; \zhtn{t}{n}}.\numberthis\label{eq:xlamdahtspace}
\end{align*}
which is a compact (and convex) space by Tychonoff's theorem (see Proposition \ref{prop:tychonoff}), and is also metrizable. From hereon, we will assume $\lamdahtspace{t}$ to have the same topology as $\xtspace{\lamdahtspace{t}}$.

Having detailed the compression of \textit{i}) private histories to ASPS, and \textit{ii}) private-history based prescriptions to ASPS-based prescriptions, we now proceed to formally characterizing the compression of the common history to ASCS.

%---------------------------------------------------------%
%-----ASCS DEFINITION-----%
%---------------------------------------------------------%
\begin{dfn}[Finite-Horizon ASCS Generator]\label{dfn:ascs_generator_finite_horizon}
Let $T\in\mbb{N}$ and let $\zhtnspace{1:T}{0}$ be a collection of topological spaces.\footnote{%A complete normed vector space. 
For our purposes, we can assume each $\zhtnspace{t}{0}$ is a Euclidean space of some fixed dimension that can vary with $t$.} For a given $T$-horizon $\l(\eps_{p,1}, \eps_{p,2}, \delta_p \r)$-ASPS generator, a collection $ \varthetahtn{1:T}{0} $ of compression-functions where $\varthetahtn{t}{0}$ maps $\wthstnspace{t}{0}$ to $\zhtnspace{t}{0}$ is called the corresponding $T$-horizon $\l(\eps_{c,1}, \eps_{c,2}, \delta_c \r)$-ASCS generator if the process $\{ \Zhtn{t}{0} \}_{t=1}^{T}$, with $\Zhtn{t}{0} = \varthetahtn{t}{0}( \wtHstn{t}{0} ) $ almost-surely, satisfies the following properties.
\begin{enumerate}
\item \textbf{(ASCS1)} It evolves in a recursive manner:
\begin{align*}
\Zhtn{t}{0}=\phihtn{t}{0} \l(\Zhtn{t-1}{0}, \Lamdaht{t-1}, \Otn{t}{0} \r).
\end{align*}

\item \textbf{(ASCS2.1)} It suffices for approximate prediction of objective-cost: for all $\l(\hstn{\wt{h}}{t}{0}, \lamdaht{t} \r) \in \wthstnspace{t}{0} \times \lamdahtspace{t}$,
\begin{align*}
&\l| \mbb{E}_{P_1} \l[  c\l( \Stt{t}, \At{t} \r) \Big| \hstn{\wt{h}}{t}{0}, \lamdaht{t} \r] - \r. \\
&\hspace{40pt} \l. \mbb{E}_{P_1} \l[ c\l( \Stt{t}, \At{t} \r) \Big| \zhtn{t}{0}, \lamdaht{t} \r]  \r| \le \frac{\eps_{c,1}}{4}.
\end{align*}

\item \textbf{(ASCS2.2)} It suffices for approximate prediction of constraint-cost: for all $\l(\hstn{\wt{h}}{t}{0}, \lamdaht{t} \r) \in \wthstnspace{t}{0} \times \lamdahtspace{t}$,
\begin{align*}
&\l\| \mbb{E}_{P_1} \l[  d\l( \Stt{t}, \At{t} \r) \Big| \hstn{\wt{h}}{t}{0}, \lamdaht{t} \r] - \r. \\
&\hspace{40pt} \l. \mbb{E}_{P_1} \l[ d\l( \Stt{t}, \At{t} \r) \Big| \zhtn{t}{0}, \lamdaht{t} \r]  \r\|_{\infty} \le \frac{\eps_{c,2}}{4}.
\end{align*}

\item \textbf{(ASCS3)} It suffices for approximate prediction of common-observations: for all $\l(\hstn{\wt{h}}{t}{0}, \lamdaht{t} \r) \in \wthstnspace{t}{0} \times \lamdahtspace{t}$,
\begin{align*}
&\kappa \l( 
\mbb{P}_{P_1} \l( \Otn{t+1}{0} \mid \hstn{\wt{h}}{t}{0}, \lamdaht{t} \r), \mbb{P}_{P_1} \l( \Otn{t}{0} \mid \zhtn{t}{0}, \lamdaht{t} \r) 
\r)  \leq \frac{\delta_c}{8}.
\end{align*}
\end{enumerate}
\end{dfn}
\begin{rem}
ASPS-1 and ASCS-1 are important for designing implementable algorithms; the recursive nature of $\Zhtn{t}{n}$'s obviates the requirement of storing the full histories.
\end{rem}

%---------------------------------------------------------%
%-----ASCS DEFINITION-----%
%---------------------------------------------------------%
Let us denote the range of $\varthetahtn{t}{0}$ by $\zhtnspaces{t}{0}$, i.e., $\zhtnspaces{t}{0} \defeq \varthetahtn{t}{0} \l( \wthstnspace{t}{0} \r)$. Then combining ASCS with ASPS (and ASPS-based prescriptions) leads to a reduction in the policy search space of the coordinator from $\vvspace$ to $\wh{\vvspace}( \varthetahtn{1:T}{0:N} )$. A typical element $\wh{v}$ of $\wh{\vvspace}$ is a tuple $\wh{v}_{1:\infty}$, where for all $t \in [T]$, $\wh{v}_t$ maps $\zhtnspaces{t}{0}$ to $\lamdahtspace{t}$, such that to take action $\Atn{t}{n}$, agent $n$ uses the distribution $$\Lamdahtn{t}{n}\l(\Zhtn{t}{n} \r) = \l[ \wh{v}_t \l(\Zhtn{t}{0}\r) \r]^{(n)} \l( \Zhtn{t}{n} \r), $$ 
which is almost-surely the same as 
$$ \l[ \wh{v}_t \l( \varthetahtn{t}{0}\l(\wtHstn{t}{0}\r) \r) \r]^{(n)} \l( \varthetahtn{t}{n} \l( \Hstn{t}{0}, \Hstn{t}{n} \r)  \r).$$

Given $T$-horizon ASPS and ASCS generators $\varthetahtn{1:T}{0:N}$, we can find an optimal-policy in $\wh{\vvspace}$ (thus approximately optimal in $\vvspace$) using the (approximate) dynamic program in Algorithm \ref{alg:dp_approximatestate}.

%---------------------------------------------------------%
%--ALGORITHM FOR COMPRESSED COMMON AND PRIVATE HISTORIES--%
%---------------------------------------------------------%
\begin{algorithm}[H]
\DontPrintSemicolon
\KwInput{Time-horizon $T \in \mbb{N}$, Discount-factor $\alpha\in(0,1]$, MA-C-POMDP Model (see Section \ref{sec:problem}).}
%%%%%%%%%%%%%%%%%%%%%%%%%%%%%%%%%%%%%%
\Parameter{Lagrange-multipliers vector $\lambda \in \mcl{Y}$.}
\KwOutput{$\utn{\wh{v}}{1:T}{\lambda,\star}$ determined by $\l\{ \whQtl{t,T}{\lambda} : \zhtnspaces{t}{0} \times \lamdahtspace{t} \ra \mbb{R} \r\}_{t=1}^{T}$.
}

%%%%%%%%%%%%%%%%%%%%%%%%%%%%%%%%%%%%%%
\nonl $\wh{V}_{T+1,T}^{(\lambda)} \equiv 0$.

\nonl \For{$t = T, T-1, \dots, 1$}{
\nonl \begin{align*}
% \nonl
% $
&\whQtl{t,T}{\lambda}\l(\zhtn{t}{0}, \lamdaht{t}\r) = 
\mbb{E} \l[
\lCost \r. \\
&\hspace{10pt} \l. + \alpha \whVtl{t+1,T}{\lambda}  \l( \Zhtn{t+1}{0} \r)  \mid \Zhtn{t}{0} = \zhtn{t}{0}, \Lamdaht{t} = \lamdaht{t} \r].\numberthis\label{eq:whQ_tT_lamda}\\
&\whVtl{t,T}{\lambda}\l( \zhtn{t}{0} \r) = \min_{\lamdaht{t} \in \lamdahtspace{t} } \whQtl{t,T}{\lambda}\l( \zhtn{t}{0}, \lamdaht{t} \r)\numberthis\label{eq:whV_tT_lamda}\\
&\utn{\wh{v}}{t}{\lambda,\star} \l(\zhtn{t}{0} \r) = \argmin_{\lamdaht{t} \in \lamdahtspace{t}} \whQtl{t,T}{\lambda}\l( \zhtn{t}{0}, \lamdaht{t} \r).\numberthis\label{eq:whv_tT_lamda_star}
% .$
\end{align*}
}
\caption{%(Approximate) 
Dynamic Programming with Compressed Common and Private Histories in Finite-Horizon setting.}\label{alg:dp_approximatestate}
\end{algorithm}
%---------------------------------------------------------%
%--ALGORITHM FOR COMPRESSED COMMON AND PRIVATE HISTORIES--%
%---------------------------------------------------------%
Like before, the usage of $\min$ and $\argmin$ in Algorithm \ref{alg:dp_approximatestate} is justified due to compactness of $\lamdahtspace{t}$, and the choice of the reduced prescription in \eqref{eq:whv_tT_lamda_star} is arbitrary. Since ``$\Zhtn{t}{0} = \varthetahtn{t}{0} \l( \wtHstn{t}{0}  \r)$'' and ``$\Zhtn{t}{n} = \varthetahtn{t}{n} \l( \Hstn{t}{0}, \Hstn{t}{n} \r) \forall\  n\in[N]$'' hold almost-surely, from standard results in dynamic programming theory, it follows that $\whVtl{t,T}{\lambda} : \wthstnspace{t}{0} \ra \mbb{R} $ (see \eqref{eq:whV_tT_lamda}) satisfies 
\begin{align*}
&\whVtl{t,T}{\lambda} \l( \varthetahtn{t}{0} \l( \hstn{\wt{h}}{t}{0} \r) \r) \\
&\hspace{10pt} = \inf_{v\in\wh{\vvspace}} \E{\wh{v}}{P_1} \l[ \sum_{\tau=t}^{T} \alpha^{\tau - t} \lCost \Big| \wtHstn{t}{0} = \hstn{\wt{h}}{t}{0} \r].
\end{align*}
Therefore, the (finite-horizon) coordination policy $\utn{\wh{v}}{1:T}{\lambda, \star}$ (see \eqref{eq:whv_tT_lamda_star}) minimizes the (finite-horizon) objective \eqref{eq:L_T} amongst all coordination policies that are restricted to draw actions based on ASCS, ASPS, and ASPS-based prescriptions over the horizon $[T]$, i.e.,
\begin{align*}
    L_T\l( \utn{\wh{v}}{1:T}{\lambda, \star}, \lambda \r) &= \inf_{\wh{v}\in\wh{\vvspace}( \varthetahtn{1:T}{0:N} )} L_T\l( v, \lambda \r).
\end{align*}
A natural question is to estimate the finite-horizon optimality gap of the \emph{ASPS-ASCS coordination policy} obtained from Algorithm \ref{alg:dp_approximatestate} from the one obtained via Algorithm \ref{alg:cidecomposition}.


%---------------------------------------------------------%
%--OPTIMALITY GAP--%
%---------------------------------------------------------%
\begin{prop}[Optimality Gap for Finite-Horizon]\label{prop:optimalitygap_T}
Fix $ \lambda \in \mcl{Y}$, $\alpha\in(0,1]$, $T \in \mbb{N}$,  
and $T$-horizon ASPS and ASCS generators $\varthetahtn{1:T}{0:N}$ (see Definitions \ref{dfn:asps_generator_finite_horizon} and \ref{dfn:ascs_generator_finite_horizon}). 
Suppose that Assumption \ref{assmp:boundedcosts} holds. Then, for any $t\in[T]$ and any $\hstn{\wt{h}}{t}{0} \in \wthstnspace{t}{0}$ with $\gt{t}^\star \in \argmin_{\gt{t} \in \gtspace{t} } \Qtl{t,T}{\lambda} \l( \hstn{\wt{h}}{t}{0}, \gt{t} \r)$, there exists $\lamdaht{t} \in \lamdahtspace{t}$ such that
\begin{align*}
&\whQtl{t,T}{\lambda} \l( \varthetahtn{t}{0}\l( \hstn{\wt{h}}{t}{0} \r), \lamdaht{t} \r) - \Qtl{t,T}{\lambda}\l( \hstn{\wt{h}}{t}{0}, \gt{t}^\star \r) \\
&\hspace{80pt} \le M_c\l(t; \alpha, T \r) + M_p \l(t; \alpha, T \r),\\
&\whVtl{t,T}{\lambda} \l( \varthetahtn{t}{0}\l( \hstn{\wt{h}}{t}{0} \r) \r) - \Vtl{t,T}{\lambda}\l( \hstn{\wt{h}}{t}{0} \r) \\
&\hspace{80pt} \le M_c\l(t; \alpha, T \r) + M_p\l(t; \alpha, T \r),
\end{align*}
where,
\begin{align*}
%----------
%---------- common part
%----------
&M_c \l(t; \alpha, T \r) = M_c^{(\eps_{c,1}, \eps_{c,2}, \delta_c, \lambda, \ulbar{c}, \ulbar{d}, 
\constraintv)} \l( t; \alpha, T \r) \\
&\hspace{10pt} \defeq \l( \eps_{c,1} + \|\lambda\|_1 \eps_{c,2} \r) \\ 
&\hspace{20pt} + \alpha \l( \sum_{\tau = t+1}^{T} \alpha^{T-\tau} \r) \l[ \l( \eps_{c,1} + \eps_{c,2} \| \lambda \|_1 + N\l( \alpha, T\r) \delta_c \r) \r],\numberthis\label{eq:M_c}\\
%----------
%------------- private part
%----------
&M_p \l(t; \alpha, T \r) = M_p^{(\eps_{p,1}, \eps_{p,2}, \delta_p, \lambda, \ulbar{c}, \ulbar{d}, 
\constraintv)} \l(t; \alpha, T \r) \\
&\hspace{10pt} \defeq \l( \eps_{p,1} + \|\lambda\|_1 \eps_{p,2} \r)\l( \sum_{\tau=t}^{T} \alpha^{T-\tau} \r) \\
&\hspace{20pt} + \alpha \l( \sum_{i=0}^{T-t-1} \sum_{j=0}^{T-t-1-i} \alpha^{i+j} \r) \l[ \l( \eps_{p,1} + \|\lambda\|_1 \eps_{p,2} \r. \r. \\
&\hspace{150pt} \l. \l. + N \l( \alpha, T \r)\delta_p \r)  \r],\numberthis\label{eq:M_p}\\
%------------ definition of N(.)
&N\l( \alpha, T \r) = N^{(\lambda,\ulbar{c},\ulbar{d},\constraintv)} \l( \alpha, T \r) \\
&\hspace{0pt} \defeq  \sum_{\tau=1}^{T} \alpha^{\tau-1} \l[ \ulbar{c} + \|\lambda\|_1 \l( \ulbar{d} + \frac{1}{2} \l( \max_k \constraintv_k - \min_k \constraintv_k \r) \r)  \r].\numberthis\label{eq:N_alpha_T}
\end{align*}
\end{prop}
\begin{proof}
The proof follows the same methodology as in \cite{hsu22}[Theorem 7]. For brevity, it is omitted. It is helpful to note that the main result of \cite{hsu22} (namely Theorem 7) follows as a special case of this proposition (take $\alpha=1$ and $\lambda = 0$).
\end{proof}
%---------------------------------------------------------%
%--OPTIMALITY GAP--%
%---------------------------------------------------------%
Propositions \ref{prop:VtT_vs_Vt} and \ref{prop:optimalitygap_T} yield the following corollary.
\begin{cor}\label{cor:optimalitygap_T}
    Fix $\lambda\in\mcl{Y}$, $\alpha\in(0,1)$, $T \in \mbb{N}$, and $T$-horizon ASPS and ASCS generators $\varthetahtn{1:T}{0:N}$ (see Definitions \ref{dfn:asps_generator_finite_horizon} and \ref{dfn:ascs_generator_finite_horizon}). Suppose that Assumption \ref{assmp:boundedcosts} holds. Then, for any $t\in[T]$ and any $\hstn{\wt{h}}{t}{0} \in \wthstnspace{t}{0}$, the following relation holds between $\whVtl{t,T}{\lambda} \l( \varthetahtn{t}{0}\l(\hstn{\wt{h}}{t}{0}\r) \r) $ and $\Vtl{t}{\lambda} \l( \hstn{\wt{h}}{t}{0} \r) $.
    \begin{align*}
        &\whVtl{t,T}{\lambda} \l( \varthetahtn{t}{0}\l(\hstn{\wt{h}}{t}{0}\r) \r) + \frac{\alpha^{T-t+1}}{1-\alpha} \un{\udl{l}}{\lambda} \\
        &\hspace{30pt} - M_c\l(t;\alpha, T \r) - M_p\l(t;\alpha, T \r) \le \Vtl{t}{\lambda} \l( \hstn{\wt{h}}{t}{0} \r) \\
        &\hspace{90pt} \le \whVtl{t,T}{\lambda} \l( \varthetahtn{t}{0}\l(\hstn{\wt{h}}{t}{0}\r) \r) + \frac{\alpha^{T-t+1}}{1-\alpha} \un{\ov{l}}{\lambda}.
    \end{align*}
\end{cor}
\begin{rem}\label{rem:universalbound_optimalitygap}
Assumption \ref{assmp:slatercondition} can be combined with Corollary \ref{cor:optimalitygap_T} to get $\lambda$-independent bounds on the optimality gap. 
\end{rem}

\subsection{Approximate State Representations (Infinite-Horizon)}
We will now extend the notions from the finite-horizon setting to the infinite-horizon case. In doing so, synonymous to the standard MDP literature, our goal is to identify an ASPS-ASCS based fixed-point iteration scheme that can approximate the optimal value function $\Vtl{t}{\lambda}$ and return a policy that depends, in a time-homogeneous way, on the approximate-state processes $\{ \Zhtn{t}{0:N}\}_{t=1}^{\infty}$.

\begin{dfn}[Infinite-Horizon ASPS-Generator]\label{dfn:asps_generator_infinite_horizon}
Let $\zhnspace{1:N}$ be a collection of topological spaces. A collection $ \varthetahn{1:N} $ of compression-functions where $\varthetahn{n}$ maps $\cup_{t=1}^{\infty} \hstnspace{t}{0} \times \hstnspace{t}{n}$ to $\zhnspace{n}$ is called an infinite-horizon $\l(\eps_{p,1}, \eps_{p,2}, \delta_p \r)$-ASPS generator if the process $\{ \Zhtn{t}{1:N} \}_{t=1}^{\infty}$, with $\Zhtn{t}{n} = \varthetahn{n}( \Hstn{t}{0}, \Hstn{t}{n} ) $ almost-surely, satisfies ASPS-1, ASPS-2.1, ASPS-2.2, and ASPS-3 along with the addition that the evolution functions do not depend on time, i.e., $\phihtn{t}{n} = \phihtn{t'}{n}$.
\end{dfn}

\begin{dfn}[Infinite-Horizon ASCS-Generator]\label{dfn:ascs_generator_infinite_horizon}
Let $\zhnspace{0}$ be a topological space. A compression-function $ \varthetahn{0} $ where $\varthetahn{0}$ maps $\cup_{t=1}^{\infty}\wthstnspace{t}{0}$ to $\zhnspace{0}$ is called an infinite-horizon $\l(\eps_{c,1}, \eps_{c,2}, \delta_c \r)$-ASCS generator if the process $\{ \Zhtn{t}{0} \}_{t=1}^{\infty}$, with $\Zhtn{t}{0} = \varthetahn{0}( \wtHstn{t}{0} ) $ almost-surely, satisfies ASCS-1, ASCS-2.1, ASCS-2.2, and ASCS-3, along with the addition that 
\begin{enumerate}
    \item the evolution functions do not depend on time, i.e., $\phihtn{t}{0} = \phihtn{t'}{0}$; and
    \item the conditional expectations $ \mbb{E}_{P_1} \l[ c\l( \Stt{t}, \At{t} \r) \Big| \zhtn{t}{0}, \lamdaht{t} \r] $ and $ \mbb{E}_{P_1} \l[ d\l( \Stt{t}, \At{t} \r) \Big| \zhtn{t}{0}, \lamdaht{t} \r] $ do not depend on time.
\end{enumerate}

\end{dfn}
Let us denote the range of $\varthetahn{n}$ by $\zhnspaces{n}$, i.e., $\zhnspaces{0} \defeq \varthetahn{0} \l( \cup_{t=1}^{\infty} \wthstnspace{t}{0} \r)$ and $\zhnspaces{n} \defeq \varthetahn{n} \l( \cup_{t=1}^{\infty} \hstnspace{t}{0} \times \hstnspace{t}{n} \r)$ for all $n \in [N]$. Then the above definition leads to a time-invariant ASPS-based prescription space 
$\lamdahspace = \lamdahnspace{1:N}$ where $ \lamdahnspace{n} \defeq \{ \lamdahn{n} : \zhnspaces{n} \mapsto \m{\anspace{n}} \}$. We note that $\lamdahspace$ is in one-to-one correspondence with the set
\begin{align*}
    \xtspace{\lamdahspace} \defeq \prod_{n=1}^{N} \prod_{ \zhn{n} \in \zhnspaces{n} } \m{\anspace{n}; \zhn{n}}.\numberthis\label{eq:xlamdahspace}
\end{align*}
which is a compact (and convex) space by Tychonoff's theorem (see Proposition \ref{prop:tychonoff}), and is also metrizable (see Proposition \ref{prop:metrizability}). From hereon, we will assume $\lamdahspace$ to have the same topology as $\xtspace{\lamdahspace}$.

Definitions \ref{dfn:asps_generator_infinite_horizon} and \ref{dfn:ascs_generator_infinite_horizon} 
naturally lead to an approximate Bellman-type operator $\wh{B} : \{ \zhnspaces{0} \ra \mbb{R} \} \ra \{\zhnspaces{0} \ra \mbb{R}\} $ defined as follows: for a uniformly bounded function $\whVl{\lambda} : \zhnspaces{0} \ra \mbb{R}$,
\begin{align*}
    \l[ \wh{B} \whVl{\lambda} \r] \l( \zhn{0} \r) &\defeq \min_{\lamdah \in \lamdahspace} \mbb{E}_{P_1} \l[ \lCost \r. \\
    &\hspace{-30pt} \l. + \alpha \whVl{\lambda}\l( \Zhtn{t+1}{0} \r) \Big| \Zhtn{t}{0} = \zhtn{t}{0}, \Lamdaht{t} = \lamdah \r].\numberthis\label{eq:whB_whV}
\end{align*}
Note that with the definitions of infinite-horizon ASPS and ASCS (and the time-homogeneous prescriptions), the expectation on the right-hand-side in \eqref{eq:whB_whV} does not depend on time, and due to discounting ($\alpha\in(0,1)$), the operator $\wh{B}$ is a contraction under the supremum norm. Therefore, under Assumption \ref{assmp:boundedcosts}, the fixed point equation $\whVl{\lambda} = \wh{B} \whVl{\lambda}$ has a unique bounded solution according to the Banach fixed-point theorem, (the vector space of uniformly bounded functions with the supremum norm is a complete metric space, see Proposition \ref{prop:banach}).

\begin{thm}[Optimality Gap for Infinite-Horizon]\label{thm:optimalitygap}
Fix $\alpha\in(0,1)$ and $\lambda\in\mcl{Y}$, and infinite-horizon ASPS and ASCS generators $\varthetahn{0:N}$ (see Definitions \ref{dfn:asps_generator_infinite_horizon} and \ref{dfn:ascs_generator_infinite_horizon}). 
Suppose that Assumption \ref{assmp:boundedcosts} holds. Consider the fixed point equation \eqref{eq:whB_whV} which we rewrite as follows:
\begin{align*}
     \whVl{\lambda} \l( \zhn{0} \r) &\defeq \min_{\lamdah \in \lamdahspace} \whQl{\lambda} \l( \zhn{0}, \lamdah \r),\numberthis \label{eq:whVl} \\ 
    \whQl{\lambda} \l( \zhn{0}, \lamdah \r) &\defeq \mbb{E}_{P_1} \l[ \lCost \r. \\
    &\hspace{-30pt} \l. + \alpha \whVl{\lambda}\l( \Zhtn{t+1}{0} \r) \Big| \Zhtn{t}{0} = \zhtn{t}{0}, \Lamdaht{t} = \lamdah \r].\numberthis\label{eq:whQl}
\end{align*}
Let $ \whVl{\lambda,\star}$ denote the fixed-point of \eqref{eq:whB_whV} and $\whQl{\lambda, \star}$ denote the corresponding prescription-value function. Then, for any $t\in\mbb{N}$ and any $\hstn{\wt{h}}{t}{0} \in \wthstnspace{t}{0}$ with $\gt{t}^\star \in \argmin_{\gt{t} \in \gtspace{t} } \Qtl{t}{\lambda} \l( \hstn{\wt{h}}{t}{0}, \gt{t} \r)$, there exists $\lamdah \in \lamdahspace$ such that
\begin{align*}
&\whQl{\lambda,\star}\l( \varthetahn{0} \l( \hstn{\wt{h}}{t}{0} \r), \lamdah \r) - M_c \l( \alpha, \infty \r) - M_p\l( \alpha, \infty \r) \\
&\hspace{40pt} \le \Qtl{t}{\lambda}\l( \hstn{\wt{h}}{t}{0}, \gt{t}^\star \r) \le \whQl{\lambda,\star}\l( \varthetahn{0} \l( \hstn{\wt{h}}{t}{0} \r), \lamdah \r),\\
&\whVl{\lambda}\l( \varthetahn{0} \l( \hstn{\wt{h}}{t}{0} \r) \r) - M_c\l( \alpha, \infty \r) - M_p\l( \alpha, \infty \r) \\
&\hspace{40pt} \le \Vtl{t}{\lambda}\l( \hstn{\wh{h}}{t}{0} \r) \le \whVl{\lambda}\l( \varthetahn{0} \l( \hstn{\wt{h}}{t}{0} \r) \r),
\end{align*}
where $\Vtl{t}{\lambda}$ and $\Qtl{t}{\lambda}$ were defined in \eqref{eq:V_t_lamda} and \eqref{eq:Q_t_lamda}, respectively.
\end{thm}

\begin{proof}
Consider the following sequence of value-functions: $^{0}\whVl{\lambda} \equiv 0$ and for $i\in \mbb{N}$, $^{i+1}\whVl{\lambda} = \wh{B} \l(^{i}\whVl{\lambda}\r)$. Let $T \in [t,\cdot]_{\mbb{Z}}$; by construction of $\l\{^{i}\whVl{\lambda}\r\}_{i=0}^{\infty}$, it follows that $^{T-t+1}\whVl{\lambda} = \whVtl{t,T}{\lambda} $. Therefore, from Corollary 5.1, 
\begin{align*}
    &^{T-t+1}\whVl{\lambda} \l( \varthetahtn{t}{0}\l(\hstn{\wt{h}}{t}{0}\r) \r) + \frac{\alpha^{T-t+1}}{1-\alpha} \un{\udl{l}}{\lambda} \\
    &\hspace{20pt} - M_c\l( \alpha, T \r) - M_p\l( \alpha, T \r) \le \Vtl{t}{\lambda} \l( \hstn{\wt{h}}{t}{0} \r)\\
    &\hspace{40pt} \le ^{T-t+1}\whVl{\lambda} \l( \varthetahtn{t}{0}\l(\hstn{\wt{h}}{t}{0}\r) \r) + \frac{\alpha^{T-t+1}}{1-\alpha} \un{\ov{l}}{\lambda}.
    \end{align*}
Taking the limit $T\ra\infty$, we get
\begin{align*}
    &\whVl{\lambda,\star} \l( \varthetahtn{t}{0}\l(\hstn{\wt{h}}{t}{0}\r) \r) - M_c\l(\alpha, \infty\r) - M_p\l(\alpha, \infty\r)  \\
    &\hspace{20pt} \le \Vtl{t}{\lambda} \l( \hstn{\wt{h}}{t}{0} \r) \le \whVl{\lambda, \star} \l( \varthetahtn{t}{0}\l(\hstn{\wt{h}}{t}{0}\r) \r).
\end{align*}
By choosing $\lamdah \in \lamdahspace$ to be the minimizing prescription in the corresponding prescription-value function $\whQl{\lambda,\star}$, we get the inequality in \eqref{eq:whQl}.
\end{proof}

Theorem \ref{thm:optimalitygap} gives us a time-homogeneous ASPS-ASCS based coordination policy that is approximately optimal for the $\lambda$-parametrized infinite-horizon objective $L_{\infty}\l( \cdot, \lambda \r)$. Specifically, let $\ut{\wh{v}} = \ut{\wh{v}}{1:\infty} $ be such that for all $t\in\mbb{N}$, $\ut{\wh{v}}{t}\l( \varthetahn{0} \l( \hstn{\wt{h}}{t}{0} \r) \r) \in \argmin_{\lamdah \in \lamdahspace} \whQl{\lambda,\star} \l( \varthetahn{0} \l( \hstn{\wt{h}}{t}{0} \r) \r)$. Then,
\begin{align*}
L_\infty \l( \wh{v}, \lambda \r) - \inf_{v\in \vvspace} L_\infty \l( v, \lambda \r) \le M_c\l( \alpha, \infty \r) + M_p\l(\alpha, \infty \r).
\end{align*}
\section{Primal Dual Type Reinforcement Learning Using Approximate Information States}\label{sec:marl}
% Figure environment removed
% \textcolor{red}{Suppress for conference version?} 
In this section, we use the notions of ASPS and ASCS to develop an algorithmic framework for reinforcement learning in (cooperative) MA-C-POMDPs. This framework will involve recurrent and feed-forward neural-networks as function-approximators, and will be based on \emph{centralized training, distributed execution (CTDE)} wherein the training will be performed in a three time-scale stochastic approximation setup\footnote{See \cite{borkar08_sabook} for details.} as follows: first, the (time-invariant) approximate-state generators will be learnt on fast time-scale (using loss-functions motivated by ASPS-2.1/2.2/3 and ASCS-2.1/2.2/3), then an ASPS-ASCS based coordination policy will be learnt on medium time-scale (using off-the-shelf policy-gradient algorithms), and finally the optimal Lagrange-multipliers vector will be learnt on the slowest time-scale (using projected gradient-ascent).

\subsection{Function-Approximators}
We use the following types of %neural-network based 
function-approximators (see Figure \ref{fig:marl1}):
\begin{enumerate}
    \item \emph{Coordinator's ASCS Network}, modelled by a recurrent neural network (RNN), has inputs $\Otn{t}{0}, \Lamdaht{t-1}$, internal state $\Zhtn{t-1}{0} $, and output $\Zhtn{t}{0}$. It is denoted by $\rho^{(0)}$.
    \item \emph{Agent-$n$'s ASPS Network}, modelled by an RNN with inputs $\Otn{t}{0}, ( \Otn{t}{n}, \Atn{t-1}{n} ) \setminus \Otn{t}{0} $, internal state $\Zhtn{t-1}{n} $, and output $\Zhtn{t}{n}$. It is denoted by $\rho^{(n)}$. 
    \item \emph{Coordinator's Prescription Network}, modelled by a feed-forward neural network (FNN) with input $\Zhtn{t}{0} $ and output $\Lamdahtn{t}{1:N}$. It is denoted by $\varphin{0}$.
    \item \emph{Agent-$n$'s Prescription-Applier Network}, modelled by an FNN (with softmax as its last layer), has inputs $\Zhtn{t}{n}, \Lamdahtn{t}{n}$, and outputs a distribution on $\anspace{n}$.% (agent-$n$'s action-set).
    \item \emph{Coordinator's Prediction Network}, modelled by an FNN, has inputs $ \Zhtn{t}{0}, \Lamdahtn{t}{1:N}$, and outputs $\utn{\wh{C}}{t}{0}, \utn{\wh{D}}{t}{0}$ and $\un{\wh{\pr}}{0}\l( \cdot \r) \in \m{\onspace{0}}$. It is denoted by $\psi^{(0)}$. Here, $\utn{\wh{C}}{t}{0}$ and $\utn{\wh{D}}{t}{0}$ respectively serve as estimates of the conditional expectation of $\cCost$ and $\dCost$ given $\Zhtn{t}{0}, \Lamdahtn{t}{1:N}$ (see ASCS-2.1/2.2) and $\un{\wh{\pr}}{0}\l( \cdot \r)$ serves as an estimate of the conditional distribution of $\Otn{t+1}{0}$ given $\Zhtn{t}{0}, \Lamdahtn{t}{1:N}$ (see ASCS-3).
    \item \emph{Supervisor's Prediction Network}, modelled by an FNN, has inputs $ \Zhtn{t}{0:N}, \Atn{t}{1:N}$, and outputs $\utn{\wh{C}}{t}{S}, \utn{\wh{D}}{t}{S}$ and $\un{\wh{\pr}}{S}\l( \cdot \r) \in \m{\onspace{0:N}}$. It is denoted by $\psi^{(S)}$. Here, $\utn{\wh{C}}{t}{S}$ and $\utn{\wh{D}}{t}{S}$ respectively serve as estimates of the conditional expectation of $\cCost$ and $\dCost$ given $\wtHstn{t}{0}, \Zhtn{t}{1:N}, \Atn{t}{1:N}$ (see ASPS-2.1/2.2), and $\un{\wh{\pr}}{S}\l( \cdot \r)$ serves as an estimate of the conditional distribution of $\Ot{t+1}$ given $\wtHstn{t}{0}, \Zhtn{t}{1:N}, \Atn{t}{1:N}$ (see ASCS-3).
    
    Here, \emph{supervisor} is a omniscient entity that can access the union of the information of all agents and is needed because the approximation criteria in ASPS-2.1/2.2/3 require the knowledge of the tuples $\Zhtn{t}{0:N}$ and $\Atn{t}{1:N}$.
\end{enumerate}

\subsection{Centralized Training Distributed Execution}
The architectural setup of the aforementioned function-approximators is shown pictorially in \figurename \ref{fig:marl1}. Here, the networks $\rho \defeq \rhon{0:N}$ and $ \varphi \defeq \varphin{0:N}$ collectively define a parametrized ASPS-ASCS coordination policy which we denote by $\ut{\wh{v}}{\rho, \varphi}$. The state networks $\rho$ collectively generate ASPS and ASCS $\Zhtn{t}{0:N}$. The coordinator's prescription network $\varphin{0}$ takes the generated ASCS $\Zhtn{t}{0}$ as input and outputs a \emph{pseudo-prescription} tuple $\Lamdaht{t}$.\footnote{We use the term \emph{pseudo-prescription} because the original interpretation of a prescription is lost.} Then, each agent-$n$'s prescription-applier network $\varphin{n}$ takes $\Zhtn{t}{n}$ and %the pseudo-prescription 
$\Lamdahtn{t}{n}$ as input and outputs a distribution on $\anspace{n}$. This distribution is used by the agent to draw its action $\Atn{t}{n}$. 
Finally, the environment generates the next observations in response to the joint-action $\At{t}$. 

\begin{algorithm}[t]
\DontPrintSemicolon

\KwInput
{%\\
% \nonl \hspace{9pt} 
Initial weights of all neural-networks 
$\l( \utn{\rho}{0}{0:N}, \utn{\varphi}{0}{0:N}, \utn{\psi}{0}{0}, \utn{\psi}{0}{S} \r) $.}
%%%%%%%%%%%%%%%%%%%%%%%%%%%%%%%%%%%%%%
\Parameter{
Batch-size ($B$), 
simulated-episodes' horizon ($T$), 
% ignored-tail's length ($W$), 
Learning schedules $\{ \delta_{1,i}, \delta_{2,i}, \delta_{3,i} \}_{i=1}^{\infty}$ satisfying \eqref{eq:stepsizes}, $maxiters$.
}
\KwOutput{%\\
%\nonl \hspace{9pt}
Learned weights of all neural-networks.
}
%%%%%%%%%%%%%%%%%%%%%%%%%%%%%%%%%%%%%%
$i \la 0$.\\
\Repeat{convergence or $i > maxiters$.}{
Collect a set of $B$ trajectories namely $\l\{ \tau_j \r\}_{j=1}^{B}$, each of horizon $T$, by running the ASPS-ASCS based coordination policy $\ut{\wh{v}}{\rho_i,\varphi_i}$. (See \eqref{eq:tauj}).\label{line:alg3:trajectories}%\\

% ----- Coordinator's Backprop start ----- %
\tcc{Coordinator's Backprop}
Compute $R_{ \un{\rho}{0}, \un{\psi}{0} } \l( \l\{ \tau_j \r\}_{j=1}^{B} \r)$ according to \eqref{eq:R_rho0_psi0}.\label{line:alg3:R_rho0_psi0}

\nonl \begin{align*}
\begin{bmatrix}
\un{\rho_{i+1}}{0}\\
\un{\psi_{i+1}}{0}
\end{bmatrix} \la 
\begin{bmatrix}
\un{\rho_{i}}{0}\\
\un{\psi_{i}}{0}
\end{bmatrix}
- \delta_{1,i} \nabla_{\un{\rho}{0}, \un{\psi}{0} } R_{\un{\rho}{0}, \un{\psi}{0}} \big|_{\un{\rho_i}{0}, \un{\psi_i}{0}} . 
\end{align*}
% ----- Coordinator's Backprop end ----- %

% ----- Supervisor's Backprop start ----- %
\tcc{Supervisor's Backprop}
\nl Compute $R_{ \rho, \un{\psi}{S} } \l( \l\{ \tau_j \r\}_{j=1}^{B} \r)$ according to \eqref{eq:R_rho_psiS}.\label{line:alg3:R_rho_psiS}
% \begin{align*}
% L_{ \un{\rho}{0:N}, \un{\psi}{S} }  \la  \frac{1}{B} \sum\limits_{j=1}^{B} \sum\limits_{t=1}^{T} l_{\un{\rho}{0},\un{\psi}{C} } 
% \l( c_{j, t}, d_{j, t}, o_{j, t+1}^{(0:N)}, \wh{c}_{j,t}^{(S)}, \wh{d}_{j,t}^{(S)}, \un{\nu_{j, t}}{S}  \r).
% \end{align*}
\nonl \begin{align*}
&\begin{bmatrix}
\un{\rho_{i+1}}{0}\\
\un{\rho_{i+1}}{1:N}\\
\un{\psi_{i+1}}{S}
\end{bmatrix} \la 
\begin{bmatrix}
\un{\rho_{i+1}}{0}\\
\un{\rho_{i}}{1:N}\\
\un{\psi_{i}}{S}
\end{bmatrix}
\\
&\hspace{60pt} - \delta_{1, i} \nabla_{\rho, \un{\psi}{S} } R_{\rho, \un{\psi}{S}} \big|_{\un{\rho_{i+1}}{0}, \un{\rho_i}{1:N}, \un{\psi_i}{S} } . 
\end{align*}
% ----- Supervisor's Backprop end ----- %

%------policy network updates start-------%
\tcc{Gradient-descent in the primal}
\nl Compute cost-to-go terms $\{ \{ g_{j,t} \}_{t=1}^{T} \}_{j=1}^{B}$ given by \eqref{eq:gjt}.\label{line:alg3:gjt}

Compute $\wh{\nabla_{\varphi}L_{\infty}}\l(\ut{\wh{v}}{\rho,\varphi} ,\lambda  \r) \big|_{\varphi_i}$ according to \eqref{eq:pg_estimate_reinforce}.\label{line:alg3:pg_estimate}

\nonl \begin{align*}
    \begin{bmatrix}
    \un{\varphi_{i+1}}{0}\\
    \un{\varphi_{i+1}}{1:N}
    \end{bmatrix}
    &\la
    \begin{bmatrix}
    \un{\varphi_{i}}{0}\\
    \un{\varphi_{i}}{1:N}
    \end{bmatrix}
    - \delta_{2, i}
    \wh{\nabla_{\varphi}L_{\infty}}\l(\ut{\wh{v}}{\rho,\varphi} ,\lambda  \r) \big|_{\varphi_i}.
\end{align*}
%------policy network updates end-------%

%------Lagrange Multipliers' Update start-------%
\tcc{Gradient-ascent in the dual}
\nl Compute $\wh{\nabla_{\lambda}L_{\infty}}\l(\ut{\wh{v}}{\rho,\varphi} ,\lambda  \r) \big|_{\lambda_i}$ according to \eqref{eq:lg_estimate}.\label{line:alg3:lg_estimate}

\nonl \begin{align*}
    \lambda_{i+1} \la \lambda_{i} + \delta_{3, i} \wh{\nabla_{\lambda}L_{\infty}}\l(\ut{\wh{v}}{\rho,\varphi} ,\lambda  \r) \big|_{\lambda_i}.
\end{align*}
%------Lagrange Multipliers' Update end-------%
$i \la i+1$.
}
\caption{Pseudo-code for History-Embedding Based Reinforcement Learning in constrained Multi-Agent POMDPs.}\label{alg:reinforce_macpomdp}
\end{algorithm}


In the above framework, the prediction-networks $\psi^{(0)}$ and $\psi^{(S)}$ are used (during the training phase) to generate estimates of the conditional expectations of immediate-costs and conditional distributions of observations. These estimates are compared with ground-truth realizations to drive the learning of the state-generators $\rho$. As concerns the learning of the coordination policy, the basic idea (synonymous to single-agent learning settings) is to get sample-paths based estimates of the policy-gradient, 
\begin{align*}
\nabla_{\varphi} L_{\infty} \l(\ut{\wh{v}}{\rho, \varphi},\lambda  \r) = \nabla_{\varphi} \E{\ut{\wh{v}}{\rho, \varphi}}{P_1} \l[ \sum_{t=1}^{\infty} \alpha^{t-1} \lCost \r].\numberthis\label{eq:policy_gradient}
\end{align*}
For learning the Lagrange-multipliers vector, sample-bath estimates of the below gradient are used.
\begin{align*}
\nabla_{\lambda} L_{\infty}\l(\ut{\wh{v}}{\rho, \varphi},\lambda \r) = \nabla_{\lambda}  \E{\ut{\wh{v}}{\rho, \varphi}}{P_1} \l[ \sum_{t=1}^{\infty} \alpha^{t-1} \lCost \r].\numberthis\label{eq:lagrangian_gradient}
\end{align*}
Once the training is complete, the execution phase is distributed -- the prediction networks $\psi^{(0)}$ and $\psi^{(S)}$ are no longer needed and (as mentioned earlier) the coordinator's networks can be instantiated by all agents. In the remainder of this section, we give a concrete instance of this framework that is based on multi-agent extension of the (single-agent) REINFORCE algorithm. Extending other policy-gradient algorithms such as actor-critic methods is %also possible and is 
left as an exercise.

\subsection{Three Time-scale Stochastic Approximation - Example Instantiation Based on REINFORCE \cite{sutton98} Algorithm}
For simplicity, we assume that the observation-sets $\onspace{n}$'s are finite. Also, with slight abuse of notation, we will denote the weights of a neural-network by the same letter that is used to denote it. Let $\l\{ \delta_{1,i} \r\}_{i=1}^{\infty}, \l\{ \delta_{2,i} \r\}_{i=1}^{\infty}, \l\{ \delta_{3,i} \r\}_{i=1}^{\infty} $ be three sequences of step-sizes that satisfy the standard three time-scale stochastic approximation conditions \cite{borkar08_sabook}, namely,
\begin{align}
\begin{split}\label{eq:stepsizes}
    \sum_{i=1}^{\infty} \delta_{1, i} = \sum_{i=1}^{\infty} \delta_{2, i} = \sum_{i=1}^{\infty} \delta_{3, i} = \infty,\\
    \sum_{i=1}^{\infty} \delta_{1, i}^2 + \sum_{i=1}^{\infty} \delta_{2, i}^2 + \sum_{i=1}^{\infty} \delta_{3, i}^2 < \infty,\\
    \frac{\delta_{3, i} }{ \delta_{2, i}} , \frac{\delta_{2, i} }{ \delta_{1, i}} \xrightarrow{i\ra\infty} 0.
    \end{split}
\end{align}
In our setup, $\delta_{1,i}$ will be used to learn the approximate-state generator networks, $\delta_{2,i}$ will be used to update the parameters of the ASPS-ASCS based coordination policy, and $\delta_{3,i}$ will be used to update the Lagrange-multipliers vector.

\begin{rem}
In practical implementations, the above conditions are rarely satisfied.
\end{rem}

\subsubsection{Learning the Approximate-State Generator Networks}
To drive the learning of the approximate-state generators, a few definitions are in order.
\begin{enumerate}
    \item To minimize the epsilons in the definitions of ASCS and ASPS, we define the loss-function,
    \begin{align*}
        l_2 &: \mbb{R}\times \mbb{R} \ra \mbb{R}_{\ge 0}\\
        l_2\l(\cdot, \star \r) &= \operatorname{smoothL1}\l( \cdot - \star \r)\\
        &\defeq \begin{cases}
        \frac{1}{2} \l( \cdot - \star \r)^2, &|\cdot-\star|<1,\\
        |\cdot - \star| - \frac{1}{2}, &\text{otherwise}.
        \end{cases}\numberthis\label{eq:l2}
    \end{align*}
    \item To minimize $\delta_c$ in ASCS-3, we define the negative log-likelihood loss-function,
    \begin{align*}
        l_{c,3} : \onspace{0} \times \m{\onspace{0}} &\ra \mbb{R}_{\ge 0} %\cup \{ \infty \}
        ,\\
        l_{c,3} \l( \on{0}, \wh{\pr}^{(0)}\l( \cdot \r) \r) &\defeq - \log \l(  \wh{\pr}^{(0)}\l( \on{0} \r) + \eta \r).\numberthis\label{eq:lc3}
    \end{align*}
    \item And similarly, to minimize $\delta_p$ in ASPS-3, we define
    \begin{align*}
        l_{p,3} : \ospace \times \m{\ospace} &\ra \mbb{R}_{\ge 0} %\cup \{ \infty \}
        ,\\
        l_{p,3}\l( o, \wh{\pr}^{(S)}\l(\cdot\r) \r) &\defeq - \log \l( \wh{\pr}^{(S)}\l(o\r) + \eta \r).\numberthis\label{eq:lp3}
    \end{align*}
\end{enumerate}
In \eqref{eq:lc3} and \eqref{eq:lp3}, $\eta$ is a sufficiently-small hyper-parameter to avoid indefinite gradients.

Consider a set of $B$ finite-horizon trajectories of length $T$ generated (independently) using coordination policy $\ut{\wh{v}}{\rho, \varphi} $. We denote this set by $\l\{ \tau_j \r\}_{j=1}^{B}$, where $\tau_j$ is given by
\begin{align*}
\tau_j &\defeq \l\{ o_{j,t}, a_{j,t}, c_{j,t}, d_{j,t}, \zhtn{j,t}{0:N}, \lamdaht{j,t}, \r. \\
&\hspace{20pt} \l. \wh{c}_{j,t}^{(0)}, \wh{d}_{j,t}^{(0)}, \wh{c}_{j,t}^{(S)}, \wh{d}_{j,t}^{(S)}, \wh{\pr}^{(0)}_{j,t}\l( \cdot \r), \wh{\pr}^{(S)}_{j,t}\l( \cdot \r) \r\}_{t=1}^{T}.\numberthis\label{eq:tauj}
\end{align*}
Here, $\star_{j,t}$ denotes the realization of the corresponding variable at time $t$ of the $j^{th}$ trajectory and $T$ denotes the common length of all the $B$ trajectories.\footnote{For the case of infinite-horizon total expected discounted costs, $T$ should preferably be on the order of $\frac{1}{1-\alpha}$.} For learning of ASCS-generator $\rho^{(0)}$ (coupled with the learning of the prediction network $\psin{0}$), we can use the loss-function: 
\begin{align*}
    &l_{\rhon{0}, \psin{0}} \l( c_{j, t}, d_{j, t}, o_{j, t+1}^{(0)}, \wh{c}_{j,t}^{(0)}, \wh{d}_{j,t}^{(0)}, \wh{\pr}_{j,t}^{(0)}\l(\cdot \r) \r) \\
    &\defeq \beta_0 l_2\l( \ut{c}{j,t}, \ut{\wh{c}}{j,t} \r) + \sum_{k=1}^{K} \beta_k l_2 \l( \ut{\l(d_k\r)}{j,t}, \ut{\l(\wh{d}_k\r)}{j,t} \r) \\
    &\hspace{10pt} + \beta_{K+1}l_{c,3} \l( \otn{j,t}{0}, \wh{\pr}_{j,t}^{(0)}\l( \cdot \r) \r).\numberthis\label{eq:l_rho0_psi0}
\end{align*}
Similarly, for learning of the ASPS-generator (coupled with the learning of $\rhon{0}$ and $\psin{S}$), we can use the loss-function:
\begin{align*}
    l_{\rho, \psin{S}} &\defeq \beta'_0 l_2\l( \ut{c}{j,t}, \ut{\wh{c}}{j,t} \r) + \sum_{k=1}^{K} \beta'_k l_2 \l( \ut{\l(d_k\r)}{j,t}, \ut{\l(\wh{d}_k\r)}{j,t} \r) \\
    &\hspace{30pt} + \beta'_{K+1}l_{p,3} \l( \ot{j,t}, \wh{\pr}_{j,t}^{(S)}\l( \cdot \r) \r).\numberthis\label{eq:l_rho_psiS}
\end{align*}
In \eqref{eq:l_rho0_psi0} and \eqref{eq:l_rho0_psi0}, the weights $\beta_k, \beta'_k$ are hyper-parameters that satisfy $\beta_k, \beta'_k > 0, \text{ and } \sum_{k=0}^{K+1} \beta_k = \sum_{k=0}^{K+1}\beta'_k = 1$. The above loss-functions lead to the following empirical risk functions whose gradients can be used to update the networks $\rho, \psin{0}$, and $\psin{S}$.
\begin{align*}
    &R_{ \un{\rho}{0}, \un{\psi}{0} } \l( \l\{ \tau_j \r\}_{j=1}^{B} \r)
    \defeq  \frac{1}{BT} \sum_{j=1}^{B} \sum_{t=1}^{T} l_{\un{\rho}{0},\un{\psi}{0} } 
    \l( \cdot \r),\numberthis\label{eq:R_rho0_psi0}\\
    &R_{ \rho, \un{\psi}{S} } 
    \l( \l\{ \tau_j \r\}_{j=1}^{B} \r)
    \defeq  \frac{1}{BT} \sum_{j=1}^{B} \sum_{t=1}^{T} l_{\un{\rho}{0},\un{\psi}{0} } 
    \l( \cdot \r).\numberthis\label{eq:R_rho_psiS}
\end{align*}
These risk functions can be used to update the state-generator and prediction networks $\rho, \psi^{(0)}, \psi^{(S)}$ (See Lines \ref{line:alg3:R_rho0_psi0} and \ref{line:alg3:R_rho_psiS} in Algorithm \ref{alg:reinforce_macpomdp}).

\subsubsection{Learning Prescription and Prescription-Applier Networks}
Based on the collected trajectories, we define cost-to-go terms,
\begin{align*}
    g_{j,t} \defeq \sum_{t'=t}^{T} \alpha^{t'-t} \l( c_{j,t} + \dotp{\lambda}{d_{j,t}-\constraintv} \r).\numberthis\label{eq:gjt}
\end{align*}
Then, the sample-paths based estimate of $\nabla_{\varphi} L_{\infty} \l(\ut{\wh{v}}{\rho,\varphi},\lambda \r)$, in flavor of the REINFORCE algorithm \cite{sutton98}, is given by
\begin{align*}
&\wh{\nabla_{\varphi} L_{\infty}}\l(\ut{\wh{v}}{\rho,\varphi},\lambda \r) \\
&\hspace{5pt} \defeq \nabla_{\varphi} \frac{1}{B} \sum_{j=1}^{B} \sum_{t=1}^{T}  g_{j, t} \l[ \sum_{n=1}^{N} \log \l( \varphi^{(n)}\l( \atn{j,t}{n}, \zhtn{j,t}{n}, \r. \r. \r.\\
&\hspace{100pt} \l. \l. \l. \proj{n}{ \un{\varphi}{0}\l( \zhtn{j,t}{0} \r) } 
\r)   \r) \r],\numberthis\label{eq:pg_estimate_reinforce}
\end{align*}
which can be used to update the prescription and prescription-applier networks $\varphi$. (See Lines \ref{line:alg3:gjt} and \ref{line:alg3:pg_estimate} in Algorithm \ref{alg:reinforce_macpomdp}).

\subsubsection{Learning the Lagrange-Multiplier}
The sample-paths based estimate of $\nabla_{\lambda} L_{\infty} \l(\ut{\wh{v}}{\rho,\varphi},\lambda \r) $ is simply given by
\begin{align*}
    % \lambda \la \lambda + 
    \wh{\nabla_{\lambda} L_{\infty}} \l( \ut{\wh{v}}{\rho,\varphi} ,\lambda\r) \defeq \l[ 
    \l( \frac{1}{B} \sum_{j=1}^{B} \sum_{t=1}^{T} \alpha^{t-1} d_{j,t}\r) - \constraintv
    \r]^{+},\numberthis\label{eq:lg_estimate}
\end{align*}
which is used to project the updated $\lambda$ onto $\mcl{Y}$ (see Line \ref{line:alg3:lg_estimate}) in Algorithm \ref{alg:reinforce_macpomdp}).
\section{Conclusion and Future Work}
In this work, I design corruption-robust algorithms for the Lipschitz contextual search problem. I present the \emph{agnostic checking} technique and demonstrate its effectiveness in designing corruption-robust algorithms. There are several open problems for future research. First, in the algorithm I propose for pricing loss, the schedule for agnostic checks is fixed upfront. Can the learner design an adaptive checking schedule for the pricing loss? Second, this work assumes the learner has knowledge of the Lipschitz constant $L$. Can the learner design efficient no-regret algorithms without knowledge of $L$? 
% The \appendix command is used to start a single appendix.
% An optional argument can be used to specify a title:
% \appendix[Proof of the Zonklar Equations]
% After issuing \appendix, the \section command will be
% disabled and any attempt to use \section will be ignored
% and will cause a warning message to be generated. (The
% single appendix marks the end of the enumerated sections
% and the section counter is fixed at zero—one does not state
% “see Appendix A” when there is only one appendix, instead
% “see the Appendix” is used.) However, all lower \subsecti
% on commands and the \section* form will work as normal
% as these may still be needed for things like acknowledgments.
\appendices
% is used when there is more than one appendix
% section. \section is then used to declare each appendix:
% \section{Proof of the First Zonklar Equation}

% The mandatory argument to section can be left blank (\sect
% ion{}) if no title is desired. It is important to remember to
% declare a section before any additional subsections or labels
% that refer to section (or subsection, etc.) numbers. As with \appendix, the \section* command and the lower \subsection commands will still work as usual.

% Some authors prefer to have the appendix number to be part
% of equation numbers for equations that appear in an appendix.
% This can be accomplished by redefining the equation numbers
% as
\renewcommand{\theequation}{\thesection.\arabic{equation}}
% before the first appendix equation. For a single appendix, the
% constant “A” should be used in place of \thesection.

%------------------------------------------%	
%- INTERMEDIARY RESULTS FOR THEOREM 1 ----%	
%------------------------------------------%
\section{Intermediary Results for Theorem \ref{thm:strongduality}}\label{sec:appendix:intermediary_results}
\begin{lem}[Equivalence between Behavioral Policy-Profiles and their (decentralized) Mixtures]\label{lem:dominance}
Fix a (factorized) measure $\mu \in \uspacemix$. Then there exists a behavioral policy-profile $\udl{u}=\udl{u}(\mu) \in \uspace$, such that for any $t \in \mbb{N}$, $\hst{h}{t} \in \hstspace{t}$, and $\at{t} \in \aspace$,
\begin{align*}
    p\l( \mu, t, \hst{h}{t}, \at{t} \r) = p\l( u, t, \hst{h}{t}, \at{t} \r),
\end{align*}
where, for brevity and with slight abuse of notation,
\begin{align*}
p\l( \cdot, t, \hst{h}{t}, \at{t} \r) &= \prup{\cdot}{P_1}\l(\Hst{t} = \hst{h}{t}, \At{t} = \at{t} \r), \text{ and}\\
p\l( \cdot, t, \hst{h}{t} \r) &= \prup{\cdot}{P_1}\l(\Hst{t} = \hst{h}{t} \r).
\end{align*}
\end{lem}
\begin{proof}
Define $\udl{u}=\udl{u}(\mu) \in \uspace$ such that
\begin{align*}
    &\ut{\udl{u}}{t}\l( \at{t} | \hst{h}{t}\ \r) = \prod_{n=1}^{N} 
    \utn{\udl{u}}{t}{n} 
    \l( \atn{t}{n} | \hstn{h}{t}{0}, \hstn{h}{t}{n} \r)  \\
    &= \begin{cases}
    \frac{ p\l( \mu, t, \hst{h}{t}, \at{t} \r)
    }{p\l( \mu, t, \hst{h}{t}\r)}, &\text{if } p\l(\mu, t, \hst{h}{t} \r) \ne 0,\\
    \prod_{n=1}^{N} \frac{1}{|\anspace{n}|}, &\text{otherwise}.
    \end{cases}\numberthis\label{eq:dominance:u}
\end{align*}
The above assignment is correct because the right-hand-side of \eqref{eq:dominance:u} is a fully-factorized function of $\atn{t}{n}$'s.
\begin{align*}
    &p\l( \mu, t, \hst{h}{t}, \at{t} \r) = \int_{U} \mu\l( du \r) \prup{u}{P_1} \l( \Hst{t} = \hst{h}{t}, \At{t} = \at{t} \r)\\
    &= \int_{\uspace} P_1\l( \sspace, \hst{h}{1} \r) \prod_{t'=2}^{t} \pr_{P_1} \l( \ot{t'} | \hst{h}{t'-1}, \at{t'-1} \r)\\
    &\hspace{10pt} \times \prod_{n=1}^{N} \prod_{t'=1}^{t} \utn{u}{t'}{n}\l( \atn{t'}{n} |  \hstn{h}{t'}{0}, \hstn{h}{t'}{n} \r) \mu\l( du \r)
    % \l( \l( \mymathop{\times}_{n=1}^{N} \mun{n}\r)(du) \r)
    \\
    &=P_1\l( \sspace, \hst{h}{1} \r) \prod_{t'=1}^{t} \pr_{P_1} \l( \ot{t'} | \hst{h}{t-1}, \at{t'-1} \r)\\
    &\hspace{10pt} \times \prod_{n=1}^{N} \int_{\uspacen{n}}  \prod_{t'=1}^{t} \utn{u}{t'}{n}\l( \atn{t'}{n} |  \hstn{h}{t'}{0}, \hstn{h}{t'}{n} \r) \mun{n}\l( d\un{u}{n}\r),
\end{align*}
where the last equality follows from Tonneli's Theorem (see Proposition \ref{prop:tonneli}). We will now prove, by forward induction, that for all $t\in\mbb{N}$, $\udl{u}$ and $\mu$ induce the same distribution on the pair $\l( \Hst{t}, \At{t} \r)$.

\begin{enumerate}
    \item \textbf{Base Case}: For time $t=1$, let $\ot{1} \in \hstspace{1} = \ospace$ and $\at{1} \in \aspace$. We have
    \begin{align*}
        p\l( \mu, 1, \ot{1}, \at{1}\r) &= P_1\l( \sspace, \ot{1} \r) \int_{\uspace} \mu\l( du \r) \ut{u}{1}\l( \at{1} | \ot{1} \r),
    \end{align*}
    and
    \begin{align*}
        p\l( \udl{u}, 1, \ot{1}, \at{1}\r) &= P_1\l( \sspace, \ot{1} \r) \ut{\udl{u}}{1}\l( \at{1} | \ot{1} \r) \\
        &\hspace{-30pt}= P_1\l( \sspace, \ot{1} \r) \frac{p\l( \mu, 1, \ot{1}, \at{1}\r)}{p\l( \mu, 1, \ot{1}\r)}\\ 
        &%\hspace{-30pt}= P_1\l( \sspace, \ot{1} \r) \frac{p\l( \mu, 1, \ot{1}, \at{1}\r)}{P_1\l( \sspace, \ot{1} \r) } 
        \hspace{-30pt}=p\l( \mu, 1, \ot{1}, \at{1}\r),
    \end{align*}
    where the last equality follows from $p\l( \mu, 1, \ot{1}\r) = P_1\l( \sspace, \ot{1} \r) $.

    \item \textbf{Induction Step}. Assuming that the statement is true for time $t$, we show that it is true for time $t+1$. Let $\hst{h}{t+1} = \l( \ot{1:t+1}, \at{1:t} \r) = \l( \hst{h}{t}, \at{t}, \ot{t+1} \r) \in \hstspace{t+1}$ and $\at{t+1} \in \aspace$. We have
    \begin{align*}
        p\l(\mu, t+1, \hst{h}{t+1} \r) &= \int_{\uspace} \mu\l( du \r) \prup{u}{P_1} \l( \Hst{t+1} = \hst{h}{t+1} \r)\\
        &\hspace{-60pt} = \int_{\uspace} \mu\l( du \r) \prup{u}{P_1} \l( \Hst{t} = \hst{h}{t}, \At{t} = \at{t}, \Ot{t+1} = \ot{t+1} \r)\\
        &\hspace{-60pt}= \int_{\uspace} \mu\l( du \r) \prup{u}{P_1} \l( \Hst{t} = \hst{h}{t}, \At{t} = \at{t}, \r) \\
        &\hspace{-40pt} \times \prup{u}{P_1} \l( \Ot{t+1} = \ot{t+1} | \Hst{t} = \hst{h}{t}, \At{t} = \at{t} \r)\\
        &\hspace{-60pt}= \int_{\uspace} \mu\l( du \r) \prup{u}{P_1} \l( \Hst{t} = \hst{h}{t}, \At{t} = \at{t}, \r) \\
        &\hspace{-40pt} \times \pr_{P_1} \l( \Ot{t+1} = \ot{t+1} | \Hst{t} = \hst{h}{t}, \At{t} = \at{t} \r)\\
        &\hspace{-60pt}= p\l(\mu, t, \hst{h}{t}, \at{t}\r) \pr_{P_1} \l( \Ot{t+1} = \ot{t+1} | \Hst{t} = \hst{h}{t}, \At{t} = \at{t} \r)\\
        &\hspace{-60pt}\labelrel{=}{eqr:dominance:ind} p\l(\udl{u}, t, \hst{h}{t}, \at{t}\r) \pr_{P_1} \l( \Ot{t+1} = \ot{t+1} | \Hst{t} = \hst{h}{t}, \At{t} = \at{t} \r)\\
        &\hspace{-60pt}=p\l(\udl{u}, t+1, \hst{h}{t+1} \r),
    \end{align*}
where \eqref{eqr:dominance:ind} uses the inductive hypothesis. The above work implies 
\begin{align*}
&p\l(\udl{u}, t+1, \hst{h}{t+1}, \at{t+1} \r) \\
&\hspace{0pt} = p\l(\udl{u}, t+1, \hst{h}{t+1} \r) \cdot\ \ut{\udl{u}}{t+1}\l(\at{t+1} | \hst{h}{t+1} \r) \\
&\hspace{0pt} = p\l(\mu, t+1, \hst{h}{t+1} \r) \frac{p\l(\mu, t+1, \hst{h}{t+1}, \at{t+1} \r)}{p\l(\mu, t+1, \hst{h}{t+1} \r)}\\
&\hspace{0pt} = p\l(\mu, t+1, \hst{h}{t+1}, \at{t+1} \r).
\end{align*}
\end{enumerate}
This completes the proof.
\end{proof}


\begin{cor}\label{cor:lbar_and_l}	
Fix $\lambda \in \mcl{Y}$. For any $\mu \in \uspacemix$, there exists $u = u(\mu) \in \uspace$ such that $\lags{u}{\lambda} = \lagsmix{\mu}{\lambda}$.	
\end{cor}	
\begin{proof}	
One notes that $\wh{C}(\mu)$ and $\wh{D}(\mu)$ can be written as:	
\begin{align*}	
\wh{C}(\mu) &= \sum_{t=1}^{\infty} \alpha^{t-1} \E{\mu}{P_1} \l[ \mbb{E}_{P_1} \l[ c\l( \Stt{t}, \At{t} \r)  \r] | \Hst{t}, \At{t} \r],\\	
\wh{D}(\mu) &= \sum_{t=1}^{\infty} \alpha^{t-1} \E{\mu}{P_1} \l[ \mbb{E}_{P_1} \l[ d\l( \Stt{t}, \At{t} \r)  \r] | \Hst{t}, \At{t} \r],	
\end{align*}	
and the result follows.	
\end{proof}


\begin{lem}\label{lem:puth}[Limit Probabilities for a converging sequence of policy-profiles]
Let $\l\{ \useq{i}{u} \r\}_{i=1}^{\infty}$ be a sequence in $\uspace$ that converges to $u$. Then, for any $t \in \mbb{N}$, $ \hst{h}{t} \in \hstspace{t} $, and $\at{t} \in \mcl{A}$,
\begin{align*}
\lim_{i\ra \infty}  \pruphsts{\useq{i}{u}}{t}{\hst{h}{t}, \at{t}} = \pruphsts{u}{t}{\hst{h}{t}, \at{t}},
\end{align*}
where $\pruphsts{\cdot}{t}{\hst{h}{t}, \at{t}} = \prup{\cdot}{P_1} \l( \Hst{t} = \hst{h}{t}, \At{t} = \at{t} \r)$. In other words, for every $t \in \mbb{N}$, the sequence of measures $\l\{ \pruphsts{ \useq{i}{u}}{t}{\cdot, \cdot} \r\}_{i=1}^{\infty}$ converges weakly to $\pruphsts{u}{t}{\cdot, \cdot}$.
\end{lem}
\begin{proof}
Given that $\useq{i}{u}$ converges to $u$, by the definition of convergence in product topology, for every $n \in [N]$, $\useq{i}{\utn{u}{t}{n}} (\hstn{h}{t}{0}, \hstn{h}{t}{n} )$ converges weakly to $ \utn{u}{t}{n} ( \hstn{h}{t}{0}, \hstn{h}{t}{n} ) $. Since $\mcl{A}^n$ is finite, this means that for each $\an{n}\in \anspace{n}$, $\useq{i}{\utn{u}{t}{n}} ( \an{n} | \hstn{h}{t}{0}, \hstn{h}{t}{n} )$ converges to $\utn{u}{t}{n} ( \an{n} | \hstn{h}{t}{0}, \hstn{h}{t}{n} )$, which further implies that for all $a \in \aspace$, $ \useq{i}{\ut{u}{t}} ( a | \hst{h}{t}) $ converges to $ \ut{u}{t} ( a | \hst{h}{t}) $. Now, we use forward induction to prove the statement. 
\begin{enumerate}
\item \textbf{Base Case}:  For time $t=1$, let $\ot{1} \in \hstspace{1} = \ospace$ and $\at{1} \in \mcl{A}$. We have
\begin{align*}
\pruphsts{\useq{i}{u}}{1}{\ot{1}, \at{1}}
=P_1\l( \sspace, o \r) \useq{i}{\ut{u}{1}} \l( \at{1} | \ot{1} \r) 
\ra \pruphsts{u}{1}{\ot{1}, \at{1}}.
\end{align*}

\item \textbf{Induction Step}: Assuming that the statement is true for time $t$, we show that it is true for time $t+1$. Let $\hst{h}{t+1} = \l( \ot{1:t+1}, \at{1:t} \r) = \l( \hst{h}{t}, \at{t}, \ot{t+1} \r) \in \hstspace{t+1}$ and $\at{t+1} \in \aspace$. We have
\begin{align*}
&\pruphsts{\useq{i}{u}}{t+1}{\hst{h}{t+1}, \at{t+1}} \\
&\hspace{0pt} =  
\pruphsts{\useq{i}{u}}{t}{\hst{h}{t}, \at{t}} \useq{i}{\ut{u}{t+1}} \l( \at{t+1} | \hst{h}{t+1} \r) \\
&\hspace{5pt} \times \pr_{P_1} \l( \Ot{t+1} = \ot{t+1} | \Hst{t} = \hst{h}{t}, \At{t} = \at{t} \r).
\end{align*}
By inductive hypothesis, $\pruphsts{\useq{i}{u}}{t}{\hst{h}{t}, \at{t}} $ converges to $\pruphsts{u}{t}{\hst{h}{t}, \at{t}}$, and $ \useq{i}{\ut{u}{t}}\l( \at{t+1} | \hst{h}{t+1}\r) $ converges to $ \ut{u}{t} \l( \at{t+1} | \hst{h}{t+1}\r) $ by assumption. We conclude that $\pruphsts{\useq{i}{u}}{t+1}{\hst{h}{t+1}, \at{t+1}}$ converges to $\pruphsts{u}{t+1}{\hst{h}{t+1}, \at{t+1}}$.
\end{enumerate}
This completes the proof.
\end{proof}


%---------------------------------------------%
%-------------- HELPFUL FACTS ----------------%
%---------------------------------------------%
\section{Helpful Facts and Results}\label{sec:appendix:helpful_facts}
\begin{dfn}[Semi-continuity]\label{dfn:lsc}
A function $f : \mcl{X} \mapsto [-\infty, \infty]$ on a topological space $\mcl{X}$ is called \emph{lower semi-continuous} if for every point $x_0 \in \mcl{X}$,
\begin{align*}
\liminf\limits_{x\ra x_0} f(x) \ge f(x_0).
\end{align*}
We call $f$ as an upper semi-continuous function $-f$ is lower semi-continuous.
\end{dfn}

\begin{prop}[Monotone Convergence Theorem]\label{prop:mct}
    Let $\l(X, \mcl{M}, \mu \r)$ be a measure-space. Let $\l\{ f_i \r\}_{i=1}^{\infty}$ be an increasing sequence of measurable functions which are uniformly bounded-from-below. Then, 
    \begin{align*}
        &\int_{X} \lim_{i\ra\infty} f_i(x) \mu(dx) = \lim_{i\ra\infty} \int_{X} f_i(x) \mu(dx). 
    \end{align*}
\end{prop}


\begin{prop}[Tonneli's Theorem]\label{prop:tonneli}
    Let $f$ be a measurable function on the cartesian product of two $\sigma$-finite measure spaces $(X, \mcl{M}, \mu)$ and $(Y, \mcl{N}, \nu)$ which is bounded from below. Then, 
    \begin{align*}
        &\int_{X\times Y} f(x,y) (\mu \times \nu) (d(x,y))\\ 
        &\hspace{10pt} =\int_X \l( \int_Y f(x,y) \nu(dy)\r) \mu(dx)\\
        &\hspace{10pt} =\int_Y \l(\int_X f(x,y) \mu(dx)\r) \nu(dy).
    \end{align*}
\end{prop}

\begin{prop}[Fatou's Lemma]\label{prop:fatou}
    Let  $(X, \mcl{M}, \mu)$ be a measure-space and let $\{ f_i \}_{i=1}^{\infty}$ be a sequence of measurable functions which are uniformly bounded from below. Then,
    \begin{align*}
        & \liminf_{i\ra\infty} \int f_i(x) \mu (dx)\ge \int \liminf_{i\ra\infty} f_i(x) \mu(dx).
    \end{align*}
\end{prop}

\begin{prop}[Tychonoff's Theorem]\label{prop:tychonoff}
Product of countable number of compact spaces is compact under the product topology.
\end{prop}

\begin{prop}[Metrizability of Product Topology on Countable Product of Metric Spaces]\label{prop:metrizability}
Product of countable number of metric spaces, when endowed with the product topology, is metrizable.
\end{prop}

% \begin{prop}[Liminf of a Product of two Sequences]\label{prop:liminfproduct}
% Let $\l\{ a_i \r\}_{i=1}^{\infty}$ and $\l\{ b_i \r\}_{i=1}^{\infty}$ be two sequences such that $\lim_{i\ra\infty} a_i = a \ge 0$ and $ \l\{ b_i \r\}_{i=1}^{\infty} $ is bounded, i.e., $|b_i| \le \eta < \infty$. Then
% \begin{align*}
%     \liminf_{i\ra \infty} a_i b_i = a \liminf_{i\ra \infty} b_i.
% \end{align*}
% \end{prop}

\begin{prop}[Prokhorov's Theorem]\label{prop:prokhorov}
Let $\l( \mcl{X}, \metric{\mcl{X}} \r)$ be a complete separable metric space with distance metric $\metric{\mcl{X}}$ and let $\borel{\mcl{X}}$ denote the Borel $\sigma$-algebra generated by $\metric{\mcl{X}}$. Let $\m{\mcl{X}}$ be the set of all probability measures on $\borel{\mcl{X}}$ and let $\tau$ denote the topology of weak-convergence on $\m{\mcl{X}}$. Then,  
\begin{enumerate}
    \item The topological space $ \l(\m{\mcl{X}} , \tau\r)$ is completely-metrizable. That is, there exists a complete metric $\metric{\m{\mcl{X}}}$ on $ \m{\mcl{X}}$ that induces the same topology as $ \tau $.
    \item An arbitrary collection $W \subseteq \m{\mcl{X}}$ of probability measures in $ \m{\mcl{X}}$ is tight iff its closure in $\tau $ is compact (i.e., $W$ is precompact in $\tau$).
\end{enumerate}
\end{prop}


% \begin{prop}[Generalized Dominated Convergence Theorem]\label{prop:generalizeddct}
% Let $(X, \salgebra)$ be a measurable space and $\l( \mu_i \r)_{i=1}^{\infty}$ a sequence of measures on $\salgebra$ that converge set-wise to a measure $\mu$. Let $\{f_i\}_{i=1}^{\infty}$ and $\{g_i\}_{i=1}^{\infty}$ be a sequence of measurable functions that converge point-wise to $f$ and $g$ respectively. Suppose that $|f_i| \le g_i$ (almost surely) and that $\lim_{i\ra \infty} \int g_i(x) \mu_i(dx) = \int g(x) \mu(dx) < \infty$. Then,
% \begin{align*}
% \lim_{i \ra \infty} \int f_i(x) \mu_i(dx) = \int f(x) \mu(dx).
% \end{align*}
% \end{prop}
% \begin{proof}
%     See \cite{}.
% \end{proof}

\begin{prop}[Hyperplane Separation Theorem]\label{prop:separation_theorem}
Let $M$ be a non-empty convex subset of 
$\mbb{R}^n$. If $x_0 \in \mbb{R}^n$ does not belong to $M$, there exists $\rho \in \mbb{R}^n$ such that
\begin{align*}
\rho \neq 0 \text { and } \inf_{x \in M} \dotp{p}{x} \geq \dotp{p}{x_0}.
\end{align*}
\end{prop}


\begin{prop}[Integral of Bounded-from-Below function with respect to Convex Combination of Non-negative Measures]\label{prop:integral_linearity}
Let $\l(X, \mcl{M}\r)$ be a measure-space. Let $f : X \ra \mbb{R} \cup \{ \infty \}$ be a measurable function that is bounded from below, and let $\mu, \nu$ be two non-negative measures on $\mcl{M}$. Then, for any $\theta \in [0,1]$,
\begin{align*}
    &\int f(x) \l(\theta \mu + (1-\theta) \nu \r)(dx) \\
    &\hspace{10pt} = \theta \int f(x) \mu(dx) + (1-\theta) \int f(x)\nu(dx). 
\end{align*}
    
\end{prop}


\begin{prop}[Behavior of Integrals of a Bounded-from-Below and Lower Semi-Continuous Function]\label{prop:lsc}
Let $(\mcl{X}, \metric{\mcl{X}})$ be a complete separable metric space with distance metric $\metric{\mcl{X}}$ and let $\borel{\mcl{X}}$ denote the Borel $\sigma$-algebra generated by $\metric{\mcl{X}}$. Let $\l( \m{\mcl{X}} , \metric{\m{\mcl{X}}} \r)$ be the complete metric space of all probability measures on $\borel{\mcl{X}}$ with the topology of weak-convergence.\footnote{Prokhorov's theorem (see Proposition \ref{prop:prokhorov}) ensures completeness and metrizability of $\m{\mcl{X}}$.} Let $\mu \in \m{\mcl{X}}$ and let $f : \mcl{X} \ra \mbb{R} \cup \l\{ \infty\r\} $ be a function that is lower semi-continuous $\mu$-amost-everywhere\footnote{Lower semi-continuity of $f$ ensures that it is measurable.} and is bounded from below. Then, the function
\begin{align*}
H : \m{\mcl{X}} \mapsto \mbb{R} \cup \l\{ \infty \r\},\  %\\
H(\mu') \defeq \int f(x) \mu'(dx)
\end{align*}
is lower semi-continuous at $\mu$. In particular, if $f$ is point-wise lower semi-continuous, then $H$ is also point-wise lower semi-continuous (on $\m{\mcl{X}}$).
\end{prop}
\begin{proof}
% The proof is omitted due to space reasons.
% \begin{comment}
Define $f' : \mcl{X} \ra \mbb{R} \cup \{\infty \}$ as $f'(x) \defeq f(x) \wedge \liminf_{y\ra x} f(y)$. Then, $f'$ minorizes $f$\footnote{That is, $f'(x) \le f(x)$.}, is lower semi-continuous, and coincides with $f$ at $x$ if and only if $f$ is lower semi-continuous at $x$. Also, $f'$ is bounded from below (since $f$ is). By Proposition \ref{prop:lsc3}, $f'$ can be written as the point-wise limit of increasing sequence of uniformly bounded-from-below continuous functions from $\mcl{X}$ into $\mbb{R} \cup \{ \infty \}$, say $\l\{ g_i %: \mcl{X} \ra \mbb{R} \cup \{\infty\}
\r\}_{i=1}^{\infty} $, i.e., $f'(x) = \lim_{i\ra \infty} g_i(x)$. Then, for every $\mu' \in \m{\mcl{X}}$,
\begin{align*}
\int f'(x)\mu'(dx) = \int \lim_{i\ra \infty} g_i(x) \mu'(dx) = \lim_{i\ra \infty} \int g_i(x)\mu'(dx),
\end{align*}
where the last equality follows from the Montone Convergence Theorem (see Proposition \ref{prop:mct}). The above equality shows that the function $H' : \m{\mcl{X}} \ra \mbb{R} \cup \{ \infty\}$ such that $H'(\mu') = \int f'(x) \mu'(dx)$, is the point-wise limit of an increasing sequence of uniformly bounded-from-below continuous functions. Therefore, by Proposition \ref{prop:lsc3}, $H'$ is lower semi-continuous. Now, if $f$ is lower semi-continuous $\mu$-almost-everywhere, then $f = f'$ $\mu-$almost-everywhere. This gives,
\begin{align*}
H(\mu) &= \int f(x) \mu(dx) \\
&= \int f'(x) \mu(dx) \\
&\labelrel{=}{eqr:lsc:H2islsc} \liminf_{\mu'\ra\mu} H'(\mu') \\
&\labelrel{\le}{eqr:lsc:H2minorizesH} \liminf_{\mu'\ra\mu} H(\mu'),
\end{align*}
Here, \eqref{eqr:lsc:H2islsc} uses lower semi-continuity of $H'$ and \eqref{eqr:lsc:H2minorizesH} follows from the fact that $H'$ minorizes $H$ (since $f'$ minorizes $f$). The inequality $H(\mu) \le \liminf_{\mu'\ra\mu} H(\mu')$ is the definition of lower semi-continuity at $\mu$. 
% \end{comment}
\end{proof}

\begin{prop}[Equivalent Characterization of a Bounded-from-Below Lower Semi-Continuous Function]\label{prop:lsc3}
Let $\l( \mcl{X}, \metric{\mcl{X}} \r)$ be a metric space. Then, a function $f : \mcl{X} \ra \mbb{R} \cup \{ \infty \}$ is a bounded-from-below lower semi-continuous function if and only if it can be written as the point-wise limit of an increasing sequence of uniformly bounded-from-below continuous functions from $\mcl{X}$ into $\mbb{R} \cup \{ \infty \}$. 
\end{prop}
\begin{proof}
% The proof is omitted due to space reasons.
% \begin{comment}
\textbf{Necessity}: Define $f_n : \mcl{X} \ra \mbb{R} \cup \{ \infty \}$ as follows:
\begin{align*}
f_n\l( x \r) &\defeq \inf_{y\in\mcl{X}} \l\{ f(y) + n \metric{\mcl{X}} \l(x, y\r) \r\}.
\end{align*}
\begin{enumerate}
\item \textit{Increasing}: 
\begin{align*}
f_{n+1}\l( x \r) = \inf_{y\in\mcl{X}} \l\{ f(y) + (n+1)\metric{\mcl{X}}\l( x,y\r) \r\} \ge f_n(x).
\end{align*}
\item \textit{Uniformly Bounded-from-Below}: Since $f_n\l(x \r) \ge \inf_{y\in\mcl{X}} \l\{ f(y) \r\}$ and $f$ is bounded-from-below, the functions $\l\{ f_n\r\}_{n=1}^{\infty}$ are uniformly bounded-from-below.
\item \textit{Continuity}: By triangle-inequality,
\begin{align*}
f(y) + n\metric{\mcl{X}}\l(y, z\r) \le
f(y) + n\metric{\mcl{X}}\l(y, w\r) +  n\metric{\mcl{X}}\l(w, z\r),
\end{align*}
and therefore, taking the infimum over $y$ on both sides gives $ f_n\l( z \r) - f_n\l( w \r) \le n\metric{\mcl{X}}\l( w, z\r) $. Similarly, we can get $ f_n\l( w \r) - f_n\l( z \r) \le n\metric{\mcl{X}}\l( w, z\r) $, and so
\begin{align*}
|f_n\l( z \r) - f_n\l( w \r)| \le n \metric{\mcl{X}} \l(w, z\r).
\end{align*}
The above relation shows that $f_n$ is Lipschitz and thus continuous.
\item \textit{Point-wise Convergence to $f$}: Fix $x_0 \in \mcl{X}$ and $\eps>0$. We would like to show that there exists a positive integer $n' = n'(x_0, \eps)$ such that, for all $ n \ge n'$, $| f_n\l(x_0\r) - f\l(x_0\r) | < \eps$. Since $f$ is lower semi-continuous at $x_0$, there exists $\delta = \delta(x_0, \eps) > 0$ such that
\begin{align*}
\metric{\mcl{X}}\l( x_0, y\r) < \delta \implies f(y) >  f(x_0) -\eps.\numberthis\label{eq:lsc2:implication}
\end{align*}
Since $f$ is bounded-from-below (and $\delta>0$), there exists a positive integer $n'=n'(\delta(x_0,\eps))$ such that
\begin{align*}
&\metric{\mcl{X}}\l( x_0, y\r) \ge \delta\\
&\hspace{0pt} \implies \forall\  n\ge n', f(y) + n\metric{\mcl{X}}\l( x_0, y\r) > f(x_0)\\
&\hspace{0pt} \implies \forall\  n\ge n', \inf_{\metric{\mcl{X}}\l( x_0, y\r) \ge \delta } \l\{ f(y) + \metric{\mcl{X}}(x_0, y) \r\}\ge f\l(x_0\r).
\end{align*}
So, for all $n\ge n'$, we have
\begin{align*}
f(x_0) \ge f_n\l( x_0 \r) &= \inf_{\metric{\mcl{X}}\l( x_0, y\r) \le \delta } \l\{ f(y) + n\metric{\mcl{X}}(x_0, y) \r\}\\
&\ge\inf_{\metric{\mcl{X}}\l( x_0, y\r) \le \delta } \l\{ f(y) \r\}\\
&\labelrel{>}{eqr:lsc2:1}\inf_{\metric{\mcl{X}}\l( x_0, y\r) \le \delta } \l\{ f(x_0) - \eps \r\}\\
&=f(x_0) - \eps.
\end{align*}
where \eqref{eqr:lsc2:1} uses \eqref{eq:lsc2:implication}.
\end{enumerate}
\hspace{5pt} \textbf{Sufficiency}: Let $\l\{ f_n \r\}_{n=1}^{\infty} $ be an increasing sequence of uniformly bounded-from-below continuous functions from $\mcl{X}$ into $\mbb{R} \cup \l\{ \infty \r\}$. Since the sequence is monotonic, it has a point-wise-limit $f : \mcl{X} \ra \mbb{R} \cup \l\{ \infty \r\}$ which is bounded-from-below because all the functions in the sequence are uniformly bounded-from-below. We need to show that $f$ is lower semi-continuous. 

Fix $x_0 \in \mcl{X}$ and $\eps>0$. We would like to show that there exists $\delta = \delta(x_0,\eps)>0$ such that $\metric{\mcl{X}}\l( x_0, y\r) < \delta \implies f(y) >  f(x_0) -\eps $. Since  $\l\{ f_n \r\}_{n=1}^{\infty} $ is increasing (and converges point-wise to $f$), there exists a positive integer $n'=n'(x_0, \eps)$ such that, for all $n\ge n'$, $f(x_0) \ge f_n(x_0) \ge f(x_0) - \frac{\eps}{2}$. Since $f_{n'}$ is lower semi-continuous, there exists $\delta=\delta(n'(x_0, \eps)) > 0$ such that $\metric{\mcl{X}}\l( x_0, y\r)<\delta \implies f(y) \ge f_{n'}(y) > f_{n'}(x_0) - \frac{\eps}{2} \ge f(x_0) - \eps$. 
\end{proof}



\begin{prop}[Banach Fixed-Point Theorem]\label{prop:banach}
    Let $\l(\mcl{X}, \metric{\mcl{X}}\r)$ be a (non-empty) complete metric space with a contraction mapping $T: \mcl{X} \ra \mcl{X}$. Then $T$ admits a unique fixed-point $x^\star$ in $\mcl{X}$ (i.e. $T\l(x^\star\r)=x^\star$ ). Furthermore, $x^\star$ can be found as follows: start with an arbitrary element $x_0 \in \mcl{X}$ and define a sequence $\l(x_i\r)_{i \in \mbb{N}}$ by $x_i = T\l(x_{i-1}\r)$ for $i \in \mbb{N}$. Then, $\lim _{i \ra \infty} x_i=x^\star$.
\end{prop}

%---------------------------------------------%
%------------- MINIMAX THEOREM ---------------%
%---------------------------------------------%
\section{A Minimax Theorem for Functions with Positive Infinity
}\label{sec:appendix:minimax}

\begin{prop}[A Minimax Theorem For Functions with Positive Infinity]\label{prop:sionminimax}
Let $\mcl{X}$ and $\mcl{Y}$ be convex topological spaces where $\mcl{X}$ is also compact. Consider a function $f : \mcl{X} \times \mcl{Y} \ra \mbb{R} \cup \{ \infty \} $ such that
\begin{enumerate}
\item for each $y \in \mcl{Y}$, $f\l(\cdot, y \r)$ is convex and lower semi-continuous.
\item for each $x \in \mcl{X}$, $f\l(x, \cdot \r)$ is concave.
\item If $f (x, y) = \infty$, then $f(x, y') = \infty$ for all $y'\in\mcl{Y}$.
\end{enumerate}
Then, there exists $x^\star \in \mcl{X}$ such that
\begin{align*}
\sup_{y\in \mcl{Y}} f\l( x^\star, y \r) &=
\inf_{x \in \mcl{X}} \sup_{y \in \mcl{Y}} f\l( x, y \r)\\
&=\sup_{y \in \mcl{Y}} \inf_{x \in \mcl{X}} f(x, y).
% &\hspace{0pt}
\end{align*}
\end{prop}

Proposition \ref{prop:sionminimax} is a mild adaptation of the Minimax theorem presented in \cite{aubin_book_2002}[Theorem 8.1] where a real-valued function is considered. In the MA-C-POMDP model described in Section \ref{sec:problem}, it is possible that $\fullccosts{u}$ and hence $\lags{u}{\lambda}$ is $\infty$ for all $\lambda \in \mcl{Y}$. We will use the same methodology as in \cite{aubin_book_2002}[Propositions 8.2 and 8.3] to prove Proposition \ref{prop:sionminimax}. In particular, the entire proof remains the same except that in Lemma \ref{lem:lem8.2:aubin:modified}, the compactness of $\mcl{X}$ is used together with Assumption 3). 

Define
\begin{align*}
f^{\sharp}(x) & :=\sup_{y \in \mcl{Y}} f(x, y), & & v^{\sharp}:=\inf_{x \in \mcl{X}} \sup_{y \in \mcl{Y}} f(x, y) \numberthis\\
f^b(y) & :=\inf_{x \in \mcl{X}} f(x, y), & & v^{\flat}:=\sup_{y \in \mcl{Y}} \inf_{x \in \mcl{X}} f(x, y).\numberthis
\end{align*}
To show the equality of $v^{\sharp}$ and $v^{\flat}$, we will introduce an intermediate value $v^{\natural}$ ($v$ natural) and prove successively that $v^{\natural}=v^{\sharp}$ and that $v^{\natural}=v^{\flat}$. 

We denote the family of finite subsets $J$ of $\mcl{Y}$ by $\mcl{J}$. We set
$$
v_J^{\sharp}:=\inf_{x \in \mcl{X}} \sup_{y \in J} f(x, y)
$$
and
$$
v^{\natural}:=\sup_{J \in \mcl{J}} v_J^{\sharp}=\sup_{J \in \mcl{J}} \inf_{x \in \mcl{X}} \sup_{y \in J} f(x, y).
$$
Since every point $y$ of $\mcl{Y}$ may be identified with the finite subset $\{y\} \in \mcl{J}$, we note that $v_{\{y\}}^{\sharp}=f^b(y)$ and consequently, $v^{\flat}=\sup_{y \in \mcl{Y}} v_{\{y\}}^{\sharp} \leq \sup_{J \in \mcl{J}} v_J^{\sharp} = v^{\natural}$. Also, since $\sup_{y \in J} f(x, y) \leq \sup_{y \in \mcl{Y}} f(x, y)$, we deduce that $v_J^{\sharp} \leq v^{\sharp}$, and hence $v^{\natural} \leq v^{\sharp}$. In summary, we have shown that
\begin{align*}
v^{\flat} \leq v^{\natural} \leq v^{\sharp} .
\end{align*}
Lemma \ref{lem:prop8.2:aubin} shows that $v^{\natural} = v^{\sharp} $ and Lemma \ref{lem:prop8.3:aubin} shows that $v^{\flat} = v^{\natural}$. This concludes the proof.

%---------------------------------------------%
%-------------- PROP 8.2 AUBIN ---------------%
%---------------------------------------------%
\begin{lem}\label{lem:prop8.2:aubin}
Consider a function $f : \mcl{X} \times \mcl{Y} \mapsto \mbb{R} \cup \{ \infty \} $ such that $\mcl{X}$ is compact and for each $
y \in \mcl{Y}$, $f(\cdot, y)$ is lower semi-continuous. Then, there exists $x^\star \in \mcl{X}$ such that
$$
\sup_{y \in \mcl{Y}} f(x^\star, y)=v^{\sharp}
$$
and
$$
v^{\natural}=v^{\sharp} .
$$
\end{lem}

\begin{rem}
Since the functions $f(\cdot, y)$ are lower semi-continuous, the same is true of the function $f^{\sharp}$.\footnote{Supremum of arbitrary collection of lower semi-continuous functions is lower semi-continuous.} Since $\mcl{X}$ is compact, Weierstrass's theorem implies the existence of $x^\star \in \mcl{X}$ which minimises $f^{\sharp}$. Following (3), this may be written as
\begin{align*}
&\sup_{y \in \mcl{Y}} f(x^\star, y) = f^{\sharp}(x^\star) = \inf_{x \in \mcl{X}} f^{\sharp}(x) \\
&\hspace{50pt} = \inf_{x \in \mcl{X}} \sup_{y \in \mcl{Y}} f(x, y)=v^{\sharp}.
\end{align*}
In comparison to this, Lemma \ref{lem:prop8.2:aubin} proves that $v^{\natural} = v^{\sharp} $.
\end{rem}

\begin{proof}
It suffices to show that there exists $x^\star \in \mcl{X}$ such that
\begin{align*}
\sup_{y \in \mcl{Y}} f(x^\star, y) \leq v^{\natural}.\numberthis\label{eq:vsharp<=vnatural}
\end{align*}
Since $v^{\sharp} \leq \sup_{y \in \mcl{Y}} f(x^\star, y)$ and $v^{\natural} \leq v^{\sharp}$, we shall deduce that $v^{\natural}=v^{\sharp}$.
We set
$$
S_{y}:=\l\{x \in \mcl{X} \mid f(x, y) \leq v^{\natural}\r\}.
$$
The inequality \eqref{eq:vsharp<=vnatural} is equivalent to the inclusion
\begin{align*}
x^\star \in \bigcap_{y \in \mcl{Y}} S_{y}.\numberthis\label{eq:nonemptyintersection}
\end{align*}
Thus, we must show that this intersection is non-empty.
For this, we shall prove that the $S_{y}$ are closed sets (inside the compact set $\mcl{X}$) with the finite-intersection property.\footnote{The intersection of an arbitrary collection of closed sets that lie inside a compact set and satisfy the finite-intersection property, is non-empty.}

If $v^{\natural} = \infty$, then every $S_y$ equals $\mcl{X}$ and the intersection is trivially non-empty. Therefore, WLOG, assume that $v^{\natural}$ is finite. Then the set $S_{y}$ is a lower section of the lower semi-continuous function $f(\cdot, y)$ and is thus closed.\footnote{The lower section of a lower semi-continuous function is closed. For every $\eta \in \mbb{R}$, the corresponding lower section is defined as $\{x \in \mcl{X} : f(x) \le \eta \}$.}

We show that for any finite sequence $J :=\l\{y_{1}, y_{2}, \ldots, y_{n}\r\} \in \mcl{J}$ of $\mcl{Y}$, the finite intersection
$$
\bigcap_{i \in [n]} S_{y_i} \neq \emptyset
$$
is non-empty. In fact, since $\mcl{X}$ is compact, and since $\max_{y \in J} f(\cdot, y) $ is lower semi-continuous, it follows that there exists $\hat{x} \in \mcl{X}$ which minimises this function. Such an $\hat{x} \in \mcl{X}$ satisfies
\begin{align*}
\max_{y \in J} f(\hat{x}, y) &= \inf_{x \in \mcl{X}} \max_{y \in J} f(x, y) \\
&\leq \sup_{J \in \mcl{J}} \inf_{x \in \mcl{X}} \max_{y \in J} f(x, y)=v^{\natural} .
\end{align*}
Since $\mcl{X}$ is compact, the intersection of the closed sets $S_{y}$ is non-empty and there exists $x^\star \in \mcl{X}$ satisfying \eqref{eq:nonemptyintersection} and thus \eqref{eq:vsharp<=vnatural}.
\end{proof}

%---------------------------------------------%
%-------------- PROP 8.3 AUBIN ---------------%
%---------------------------------------------%
\begin{lem}\label{lem:prop8.3:aubin}
Consider a function $f : \mcl{X} \times \mcl{Y} \mapsto \mbb{R} \cup \{ \infty \} $ such that $\mcl{X}$ and $\mcl{Y}$ are convex sets, (i) for each $y \in \mcl{Y}$, $f(\cdot, y)$ is convex, and (ii) for each $x \in \mcl{X}$, $f(x, \cdot)$ is concave. Then, $v^{\flat}=v^{\natural}$.
\end{lem}
\begin{proof}
We set $M_J:=\l\{\lambda \in \mbb{R}_{\ge 0}^{|J|} \mid \sum_{i=1}^n \lambda_i=1\r\}$. With any finite (ordered) subset $J \defeq = \l\{y_1, y_2, \ldots, y_n\r\}$, we associate the mapping $\phi_J$ from $\mcl{X}$ to $\l( \mbb{R} \cup \{ \infty \} \r)^{|J|}$ defined by
$$
\phi_J(x):=\l(f\l(x, y_1\r), \ldots, f\l(x, y_n\r)\r)
$$
We also set
$$
w_J:=\sup_{\lambda \in M_J} \inf_{x \in \mcl{X}} \dotp{\lambda}{\phi_J(x)}
$$
We prove successively that
\begin{enumerate}
    % \item $\phi_J(\mcl{X}_J)+\mbb{R}_{\ge 0}^{|J|}$ is a convex subset.
    \item $\sup_{J\in\mcl{J}} w_J \leq v^{\flat}$ (Lemma \ref{lem:lem8.3:aubin}).
    \item $\sup_{J\in\mcl{J}} v^{\sharp}_{J}  \le \sup_{J\in\mcl{J}} w_J$ (Lemma \ref{lem:lem8.2:aubin:modified}).
\end{enumerate}
Hence, the inequalities
\begin{align*}
v^{\natural} = \sup_{J \in \mcl{J}} v_J^{\sharp} \leq \sup_{J \in \mcl{J}} w_J \leq v^{\flat} \leq v^{\natural}
\end{align*}
imply the desired equality 
 $v^{\flat}=v^{\natural}$.
\end{proof}




\begin{lem}\label{lem:lem8.3:aubin}
Consider a function $f : \mcl{X} \times \mcl{Y} \mapsto \mbb{R} \cup \{ \infty \} $ such that $\mcl{Y}$ is convex and for each $x \in \mcl{X}$, $f(x, \cdot)$ is concave. Then, for any finite subset $J$ of $\mcl{Y}$, we have $w_J \leq v^{\flat}$. Hence, $$\sup_{J\in\mcl{J}} w_J \le v^{\flat}.$$    
\end{lem}
\begin{proof}
With each $\lambda \in M_J$, we associate the point $y_\lambda:=\sum_{i=1}^n \lambda_i y_i$ which belongs to $\mcl{Y}$ since $\mcl{Y}$ is convex. The concavity of the functions $\l\{ f(x, \cdot)\r\}_{x\in\mcl{X}}$ implies that
\begin{align*}
\forall x \in \mcl{X}, \quad \sum_{i=1}^n \lambda_i f\l(x, y_i\r) \leq f\l(x, y_\lambda\r).
\end{align*}
Consequently,
\begin{align*}
\inf_{x \in \mcl{X}} \sum_{i=1}^n \lambda_i f\l(x, y_i\r) &\leq \inf_{x \in \mcl{X}} f\l(x, y_\lambda\r) \\
&\leq \sup_{y \in \mcl{Y}} \inf_{x \in \mcl{X}} f(x, y) \defeq v^{\flat}.
\end{align*}
The proof is completed by taking the supremum over $M_J$.
\end{proof}




\begin{lem}\label{lem:lem8.2:aubin:modified}
Consider a function $f : \mcl{X} \times \mcl{Y} \mapsto \mbb{R} \cup \{ \infty \} $ such that $\mcl{X}$ is a convex compact topological space, for each $y \in \mcl{Y}$, $f(\cdot, y)$ is convex and lower semi-continuous, and $f (x, y) = \infty$ implies $f(x, y') = \infty$ for all $y'\in\mcl{Y}$.
Then, 
\begin{align*}
v^{\natural} \defeq \sup_{J \in \mcl{J}} v_J^{\sharp} \leq \sup_{J \in \mcl{J}} w_J .
\end{align*}
\end{lem}
\begin{proof}
WLOG we assume that $\sup_{J \in \mcl{J}} w_J < \infty$. In this case, we can rewrite $w_J$ as $\supinf{\lambda\in M_J}{x\in \mcl{X}_J} \dotp{\lambda}{\phi_J(x)}$ where 
$$\mcl{X}_J \defeq \bigcap_{y\in J} dom f(\cdot, y).$$
To see this, note that $\dotp{\lambda}{\phi_J(x)} $ is a lower semi-continuous function on the compact space $\mcl{X}$. By Weierstrass theorem, $\dotp{\lambda}{\phi_J(x)} $ achieves its minimum in $\mcl{X}$ and we can write $w_J = \sup_{\lambda \in M_J} \dotp{\lambda}{\phi_J(\hat{x}(\lambda))}$. Suppose that $\hat{x}(\lambda) \in \mcl{X} \setminus \mcl{X}_J$, i.e., there exists $y \in J$ such that $\hat{x}(\lambda) \notin dom f(\cdot, y)$. This implies that $\hat{x}(\lambda) \notin dom f(\cdot, y')$ for all $y' \in J$. This renders $w_J$ to be infinity which contradicts our assumption $\sup_{J\in\mcl{J}} w_J < \infty $.

Therefore, now onward, we assume each $w_J = \sup_{\lambda\in M_J} \inf_{x \in \mcl{X}_J} \dotp{\lambda }{ \phi_J(x) }$. To prove the lemma, it suffices to show that $v_J^{\sharp} \le w_J $. Let $\eps>0$ and denote $\mbf{1} \defeq (1, \ldots, 1)$. We shall show that
\begin{align*}
\l( w_J + \eps \r) \mbf{1} \in \phi_J(\mcl{X}_J) + \mbb{R}_{\ge 0}^n .\numberthis\label{eq:wj_in_convex_set}
\end{align*}
Suppose that this is not the case. Since $\phi_J(\mcl{X}_J)+\mbb{R}_{\ge 0}^n$ is a convex set in $\mbb{R}^n $, following Lemma \ref{lem:lem8.1:aubin:modified}, we may use the hyperplane separation theorem (see Proposition \ref{prop:separation_theorem}), via which there exists $\rho \in \mbb{R}^n$, $\rho \neq 0$, such that
\begin{align*}
\sum_{i=1}^n \rho_i \l( w_J + \eps \r) &=\dotp{\rho}{\l(w_J + \eps\r) \mbf{1}} \\
&\leq \inf_{v \in \phi_J(\mcl{X}_J)+\mbb{R}_{\ge 0}^n} \dotp{\rho}{v}\\
& =\inf_{x \in \mcl{X}_J} \dotp{\rho}{ \phi_J(x)} + \inf_{u \in \mbb{R}_{\ge 0}^n} \dotp{\rho}{u}.
\end{align*}
Then $\inf_{u \in \mbb{R}_{\ge 0}^n} \dotp{\rho}{u}$ is bounded below and consequently, $\rho$ belongs to $\mbb{R}_{\ge 0}^n$ and $\inf_{u \in \mbb{R}_{\ge 0}^n} \dotp{\rho}{u}$ is equal to 0. Since $\rho$ is non-zero, $\sum_{i=1}^n \rho_i$ is strictly positive. We set $\bar{\lambda} =\rho / \sum_{i=1}^n \rho_i \in M_J$ and %in case $\lambda > 0$ 
deduce that
\begin{align*}
w_J + \eps &\leq \inf_{x \in \mcl{X}_J } \dotp{\bar{\lambda}} {\phi_J(x)} \\
&\leq \sup_{\substack{\lambda \in M_J}} \inf_{x \in \mcl{X}_J} \dotp{\lambda}{ \phi_J(x) }= w_J.
\end{align*}
This is impossible and thus \eqref{eq:wj_in_convex_set} is established, which implies that there exist $x_{\eps} \in \mcl{X}_J$ and $u_{\eps} \in \mbb{R}_{\ge 0}^n$ such that $\l(w_J+\eps\r) \mathbf{1}=$ $\phi_J\l(x_{\eps}\r)+u_{\eps}$.
From the definition of $\phi_J$, we deduce that
\begin{align*}
\forall i=1, \ldots, n, \quad f\l(x_{\eps}, y_i\r) \leq w_J+\eps,
\end{align*}
and hence 
\begin{align*}
v_J^{\sharp} \leq \max _{i=1, \ldots, n} f\l(x_{\eps}, y_i\r) \leq w_J+\eps.
\end{align*}
We complete the proof of the lemma by letting $\eps$ tend to 0.   
\end{proof}




\begin{lem}\label{lem:lem8.1:aubin:modified}
Consider a function $f : \mcl{X} \times \mcl{Y} \mapsto \mbb{R} \cup \{ \infty \} $ such that $\mcl{X}$ is convex and for each $y \in \mcl{Y}$, $f(\cdot, y)$ is convex. Then, $\phi_J(\mcl{X}_J) + \mbb{R}_{\ge 0}^n$ is a convex set in $\mbb{R}^n$.    
\end{lem}
\begin{proof} 
Take any convex combination $\alpha_1 \l(\phi_J (x_1) + u_1 \r) + \alpha_2 \l(\phi_J(x_2) + u_2\r)$ where $\alpha_1, \alpha_2 \geq 0$, $\alpha_1 + \alpha_2 = 1$, $x_1$ and $x_2$ are in $\mcl{X}_J$, and $u_1$ and $u_2$ are in $\mbb{R}_{\ge 0}^n$. Let $x = \alpha_1 x_1 + \alpha_2 x_2$. For each $y \in J$, the function $f(\cdot, y)$ is convex, therefore $\phi_J(x) \le \alpha_1 \phi_J(x_1) + \alpha_2 \phi_J(x_2) < \infty$ (latter by definition of $\mathcal{X}_J$). Hence, $x \in \mcl{X}_J$. We can write the convex combination in the form $\phi_J (x) + u$ where $u = \alpha_1 u_1 + \alpha_2 u_2 + \alpha_1 \phi_J(x)+\alpha_2 \phi_J(y)-\phi_J(x)$. Note that $u \in \mbb{R}_{\ge 0}^{n}$ because $\phi_J(x) \le \alpha_1 \phi_J(x_1) + \alpha_2 \phi_J(x_2)$. Consequently, $\alpha_1\l(\phi_J\l(x\r)+u_1\r)+\alpha_2\l(\phi_J\l(y\r)+u_2\r)=\phi_J(x)+u$ belongs to $\phi_J(\mcl{X}_J)+\mbb{R}_{\ge 0}^n$.
\end{proof}



%---------------------------------------------%
%------------- NOTATION SUMMARY ---------------%
%---------------------------------------------%
\section{List of Symbols}\label{sec:appendix:notation}
\begin{itemize}
\setlength\itemsep{2pt}
\item MDP: Markov Decision Process.
\item POMDP: Partially Observable Markov Decision Process.
\item SA-MDP: Single-Agent MDP.
\item SA-POMDP: Single-Agent POMDP.
\item SA-C-MDP: Single-Agent Constrained MDP.
\item SA-C-POMDP: Single-Agent Constrained POMDP.
\item MA-POMDP: Multi-Agent POMDP.
\item MA-C-POMDP: Multi-Agent Constrained POMDP.
\item MARL: Multi-Agent Reinforcement Learning.
\item CTDE: Centralized Training Distributed Execution.
\item ASPS: Approximate Sufficient Private State.
\item ASCS: Approximate Sufficient Common State.
\item RNN: Recurrent Neural Network.
\item FNN: Feed-forward Neural Network.
\item $N$: Number of agents.
\item $\sspace$: State space.
\item $\onspace{0}$: Space of common observations of all agents.
\item $\onspace{n}$: Space of private observations of agent-$n$.
\item $\ospace$: Joint-observation space, given by $\ospace = \prod_{n=0}^{N} \onspace{n}$.
\item $\anspace{n}$: Space of actions of agent $n$.
\item $\aspace$: Joint-action space, given by $\aspace = \prod_{n=1}^{N} \anspace{n}$.
\item $M_1(\cdot)$: Set of all probability measures on topological space $\cdot$ endowed with the topology of weak convergence. 
\item $\mcl{P}_{tr}$: Transition-law. See \eqref{eq:transitionlaw}.
\item $P_{saBo}$: Probability that the next state is in the measurable set $B$ and the next joint-observation is $o$ given action $a$ is taken at current state $s$. See \eqref{eq:psabo}.
\item $c, d$: Immediate-costs: $c$ is the immediate objective cost and $d$ is the immediate constraint cost.
\item $\udl{c}, \ov{c}, \udl{d}, \ov{d}$: Upper and lower bounds on immediate costs. See Assumption \ref{assmp:boundedcosts}.
\item $\alpha$: Discount factor.
\item $P_1$: Initial distribution on the initial state and joint-observation. See \eqref{eq:initialdistribution}.
\item $\uspacen{n}$: Space of policies of agent $n$.
\item $\un{u}{n}$: Used to denote a policy of agent $n$. (in $\uspacen{n}$).
\item $\uspace$: Space of (decentralized) policy-profiles, $\prod_{n=1}^{N} \uspacen{n}$.
\item $u$: Used to denote a policy-profile (in $\uspace$). %See \eqref{eq:uah}.
\item $\uspacemix$: Space of (decentralized) mixtures of policy-profiles in $\uspace$, i.e., $\prod_{n=1}^{N} \m{\uspacen{n}}$.
\item $\mu$: Used to denote a typical element of $\uspacemix$,  given by $\mymathop{\times}_{n=1}^{N} \mun{n}$.
\item $\prup{u}{P_1}, \E{u}{P_1}$: Probability measure and expectation operator corresponding to policy-profile $u\in \uspace$ and initial-distribution $P_1$.
\item $C$: Infinite-horizon expected total discounted objective cost. See \eqref{eq:C}.
\item $D$: Infinite-horizon expected total constraint cost. See \eqref{eq:D}.
\item $\Hstn{t}{0}$: Common history of all agents at time $t$.
\item $\Hstn{t}{n}$: Private history of agent $n$ at time $t$.
\item $\Hst{t}$: Joint history at time $t$, given by $\Hstn{t}{0:N}$.
\item $\hstnspace{t}{n}, \hstspace{t}, \hsspace$: See \eqref{eq:hthnandh}.
\item $\optcosts$: Optimal solution of \eqref{eq:macpomdp}. See \eqref{eq:optccost:infsup}.
\item $\zeta$: Slack of feasible policy-profile $\ov{u}$ in Assumption \ref{assmp:slatercondition}. See \eqref{eq:slatercondition}.
\item $\mcl{Y}$: Space of non-negative Lagrange-multipliers, $\l\{ \lambda \in \mbb{R}^K: \lambda\ge 0\r\}$.
\item $L$: Lagrangian function for \eqref{eq:macpomdp}. See \eqref{eq:lagrangian}.
\item $\wh{L}$: Extended Lagrangian function for \eqref{eq:macpomdp}. See \eqref{eq:lagrangianmix}.
\item $\xuspace$: $\prod_{n=1}^{N}\xtspace{\uspacen{n}}$, also see \eqref{eq:xuspace}.
\item $\pruphsts{u}{t}{\hst{h}{t}, \at{t}}$: 
%Shorthand for 
$\prup{u}{P_1}\l( \Hst{t} = \hst{h}{t}, \At{t} = \at{t} \r)$.
\item $\zuphsts{u}{t}{\hst{h}{t}, \at{t}}$: 
%Shorthand for 
$\E{u}{P_1}\l[ \cCost \mid \Hst{t} = \hst{h}{t}, \At{t} = \at{t} \r] $.
\item $^i u$: $i^{th}$ policy-profile in the sequence $\l\{^i u\r\}_{i=1}^{\infty}$.
\item $\Gt{t} = \Gtn{t}{1:N}$: Coordinator's prescriptions at time $t$.
% \item $\gt{t} = \gtn{t}{1:N}$: Realization of $\Gt{t}$
\item $\gtspace{t}$: Set of all possible prescriptions at time $t$.
\item $ \xtspace{\gtspace{t}}$: See \eqref{eq:xgtspace}.
\item $\wtHstn{t}{0}$: Prescription-observation history of coordinator.
\item $\un{l}{\lambda}$: Immediate-cost in unconstrained version of \eqref{eq:macpomdp} parametrized by Lagrange-multiplier $\lambda\in\mcl{Y}$. See \eqref{eq:l_lamda}.
\item $L_T$: See \eqref{eq:L_T}.
\item $\Qtl{t,T}{\lambda}$: See \eqref{eq:Q_tT_lamda}. 
\item $\Vtl{t,T}{\lambda}$: Optimal cost-to-go from time $t\in[T]$ onward in a finite-horizon $T$. See \eqref{eq:V_tT_lamda}.
\item $\utn{v}{1:T}{\lambda, \star}$: See \eqref{eq:v_tT_lamda_star}.
\item $\Vtl{t}{\lambda}$: See \eqref{eq:V_t_lamda}.
\item $\Qtl{t}{\lambda}$: See \eqref{eq:Q_t_lamda}.
\item $ \un{\udl{l}}{\lambda}$: Lower bound on $\un{l}{\lambda}$. See \eqref{eq:lower_and_upper_lcost}.
\item $\un{\ov{l}}{\lambda} $: Upper bound on $\un{l}{\lambda}$. See \eqref{eq:lower_and_upper_lcost}.
\item $ \Pi_t$: Conditional distribution of $\Stt{t}, \Hstn{t}{1:N}$ given $\wtHstn{t}{0}$, $\pr_{P_1} \l( \Stt{t}, \Hstn{t}{1:N} \mid \wtHstn{t}{0} \r)$.
\item $\Zhtn{t}{1:N}$: ASPS at time $t$. See Definitions \ref{dfn:asps_generator_finite_horizon} and \ref{dfn:asps_generator_infinite_horizon}. 
\item $\Lamdaht{t}$: ASPS-based prescription at time $t$.
\item $\Zhtn{t}{0}$: ASCS at time $t$. See Definitions \ref{dfn:ascs_generator_finite_horizon} and \ref{dfn:ascs_generator_infinite_horizon}.
\item $\varthetahtn{t}{1:N}$: $t$-th component of (finite-horizon) ASPS-generator. See Definition \ref{dfn:asps_generator_finite_horizon}.
\item $\eps_{p,1}, \eps_{p,2}, \delta_{p}$: Attributes of ASPS-generator.
\item $\phihtn{t}{1:N}$: $t$-th component of evolution functions of (finite-horizon) ASPS-generator. See ASPS-1.
\item $\kappa \l(\cdot, \star \r)$: Total variation distance between probability measures $\cdot$ and $\star$. 
% \item $\zhtnspaces{t}{n}$:
\item $\lamdahtnspace{t}{n}$: Set of all possible ASPS-ASCS based prescriptions for agent $n$ at time $t$
\item $\lamdahtspace{t}$. Set of all possible ASPS-ASCS based prescriptions at time $t$, $\prod_{n=1}^{N}\lamdahtnspace{t}{n} $.
\item $\xtspace{\lamdahtspace{t}}$: See \eqref{eq:xlamdahtspace}.
\item $\varthetahtn{t}{0}$: $t$-th component of (finite-horizon) ASCS-generator.
\item $\eps_{c,1}, \eps_{c,2}, \delta_{c}$: Attributes of ASCS-generator.
\item $\phihtn{t}{0}$: $t$-th component of evolution functions of (finite-horizon) ASCS-generator. See ASCS-1.
\item $\whVtl{t,T}{\lambda}$ See \eqref{eq:whV_tT_lamda}.
\item $\whQtl{t,T}{\lambda}$: See \eqref{eq:whQ_tT_lamda}. 
\item $\utn{\wh{v}}{1:T}{\lambda,\star}$: See \eqref{eq:whv_tT_lamda_star}.
\item $M_c\l( \cdot; \alpha, T \r)$: See \eqref{eq:M_c}.
\item $M_p\l( \cdot; \alpha, T\r)$: See \eqref{eq:M_p}.
\item $N\l( \alpha, T\r)$: See \eqref{eq:N_alpha_T}.
% \item $\zhnspace{1:N} $:
\item $\varthetahn{1:N}$: Infinite-horizon ASPS-generator. See Definition \ref{dfn:asps_generator_infinite_horizon}.
% \item $\zhnspace{0}$:
\item $\varthetahn{0}$: Infinite-horizon ASCS-generator. See Definition \ref{dfn:ascs_generator_infinite_horizon}.
\item $\lamdahspace$: Set of all possible ASPS-ASCS based prescriptions when infinite-horizon ASPS and ASCS generators are used.
\item $\xtspace{\lamdahspace}$: See \eqref{eq:xlamdahspace}.
\item $\wh{B}$: See \eqref{eq:whB_whV}.
\item $\whVl{\lambda}, \whQl{\lambda}$: See \eqref{eq:whVl} and \eqref{eq:whQl}.
\item $\rhon{0} $: RNN to serve as (infinite-horizon) ASCS-generator.
\item $\rhon{1:N}$: RNNs to collectively serve as (infinite-horizon) ASPS-generator.
\item $\varphin{0}$: Coordinator's prescription network.
\item $\varphin{1:N}$: Prescription-applier networks of all agents.
\item $\psin{0}$: Coordinator's prediction network.
\item $\psin{S}$: Supervisor's prediction network.
\item $\nabla_{\varphi}L_{\infty}\l( \ut{\wh{v}}{\rho, \varphi} \r)$: Policy-gradient. See \eqref{eq:policy_gradient}.
\item $\delta_{1,i}, \delta_{2,i}$, $\delta_{3,i}$: Sequences of time-steps that satisfy three time-scale stochastic approximation conditions. See \eqref{eq:stepsizes}.
\item $l_2, l_{c,3}, l_{p,3}$: See \eqref{eq:l2}, \eqref{eq:lc3}, and \eqref{eq:lp3}.
\item $\eta$: Used as a placeholder in \eqref{eq:lc3} and \eqref{eq:lp3}.
\item $B$: Batch-size of trajectories.
\item $\tau_j$: $j^{th}$ trajectory. See \eqref{eq:tauj}.
\item $l_{\rhon{0}, \psin{0}}$: See \eqref{eq:l_rho0_psi0}.
\item $l_{\rho, \psin{S}}$: See \eqref{eq:l_rho_psiS}.
\item $R_{\rhon{0}, \psin{0}}$: See \eqref{eq:R_rho0_psi0}.
\item $R_{\rho, \psin{S}}$: See \eqref{eq:R_rho_psiS}.
\item $g_{j,t}$: Cost-to-go at time $t$ in $j^{th}$ trajectory. See \eqref{eq:gjt}.
\item $\wh{\nabla_{\varphi}L_{\infty} }\l( \ut{\wh{v}}{\rho, \varphi} \r)$: REINFORCE-estimate of policy-gradient $\nabla_{\varphi} L_{\infty} \l( \ut{\wh{v}}{\rho,\varphi} \r)$. See \eqref{eq:pg_estimate_reinforce}.

\end{itemize}
% use section* for acknowledgment
% \twocolumn
\section*{Acknowledgment}
This work was funded in part, by NSF via grants ECCS2038416, EPCN1608361, EARS1516075, CNS1955777, and CCF2008130 for V. Subramanian, and grants EARS1516075, CNS1955777, and CCF2008130 for N. Khan. The authors would also like to thank Hsu Kao for helpful discussions. 
%----------------------%
%------ SECTIONS ------%
%----------------------%



%-------------------------%
%----- BIBLIOGRAPHY ------%
%-------------------------%
% Can use something like this to put references on a page
% by themselves when using endfloat and the captionsoff option.
\ifCLASSOPTIONcaptionsoff
  \newpage
\fi

% trigger a \newpage just before the given reference
% number - used to balance the columns on the last page
% adjust value as needed - may need to be readjusted if
% the document is modified later
%\IEEEtriggeratref{8}
% The "triggered" command can be changed if desired:
%\IEEEtriggercmd{\enlargethispage{-5in}}

% references section

% can use a bibliography generated by BibTeX as a .bbl file
% BibTeX documentation can be easily obtained at:
% http://mirror.ctan.org/biblio/bibtex/contrib/doc/
% The IEEEtran BibTeX style support page is at:
% http://www.michaelshell.org/tex/ieeetran/bibtex/
\bibliographystyle{ieeetr}
% argument is your BibTeX string definitions and bibliography database(s)
\bibliography{ma_cpomdp_ieee}
%
% <OR> manually copy in the resultant .bbl file
% set second argument of \begin to the number of references
% (used to reserve space for the reference number labels box)
% \begin{thebibliography}{1}

% \bibitem{IEEEhowto:kopka}
% H.~Kopka and P.~W. Daly, \emph{A Guide to \LaTeX}, 3rd~ed.\hskip 1em plus
%   0.5em minus 0.4em\relax Harlow, England: Addison-Wesley, 1999.

% \end{thebibliography}
%-------------------------%
%----- BIBLIOGRAPHY ------%
%-------------------------%








%-----------------------------------%
%----------- BIOGRAPHY -------------%
%-----------------------------------%
% biography section
% 
% If you have an EPS/PDF photo (graphicx package needed) extra braces are
% needed around the contents of the optional argument to biography to prevent
% the LaTeX parser from getting confused when it sees the complicated
% \includegraphics command within an optional argument. (You could create
% your own custom macro containing the \includegraphics command to make things
% simpler here.)
%\begin{IEEEbiography}[{% Figure removed}]{Michael Shell}
% or if you just want to reserve a space for a photo:

% \begin{comment}
\begin{IEEEbiography}
%[{% Figure removed}]{Nouman Khan} 
[{% Figure removed}]{Nouman Khan} (Member, IEEE) is a Ph.D candidate in the department of Electrical Engineering and Computer Science (EECS) at the University of Michigan, Ann Arbor, MI, USA. He received the B.S. degree in Electronic Engineering from the GIK Institute of Engineering Sciences and Technology, Topi, KPK, Pakistan, in 2014 and the M.S. degree in Electrical and Computer Engineering from the University of Michigan, Ann Arbor, MI, USA in 2019. His research interests include stochastic systems and their analysis and control.
\end{IEEEbiography}

\begin{IEEEbiography}[{% Figure removed}]{Vijay Subramanian} (Senior Member, IEEE) received the Ph.D. degree in electrical engineering from the University of Illinois at Urbana-Champaign, Champaign, IL, USA, in 1999. He was a Researcher with Motorola Inc., and also with Hamilton Institute, Maynooth, Ireland, for a few years following which he was a Research Faculty with the Electrical Engineering and Computer Science (EECS) Department, Northwestern University, Evanston, IL, USA. In 2014, he joined the University of Michigan, Ann Arbor, MI, USA, where he is currently an Associate Professor with the EECS Department. His research interests are in stochastic analysis, random graphs, game theory, and mechanism design with applications to social, as well as economic and technological networks. 
\end{IEEEbiography}
% \end{comment}
%-----------------------------------%
%----------- BIOGRAPHY -------------%
%-----------------------------------%
\end{document}