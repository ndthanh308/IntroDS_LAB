\usepackage{mathrsfs}
\usepackage{amsfonts} 
\usepackage{epstopdf}
%\usepackage{relsize}
%\setlength{\footskip}{20pt}
\usepackage{epsf,epsfig,verbatim,amssymb,amsmath,array,cite,amsthm,multicol,multirow}  
\usepackage{psfrag,bm,xspace}
\usepackage{hhline}
% \usepackage{mathabx}
\usepackage{physics}

%-----------------------------------------------------------------------------------------
\newtheorem{thm}{Theorem}
\newtheorem{prop}{Proposition}%[thm]
\newtheorem{dfn}{Definition}
\newtheorem{rem}{Remark}
\newtheorem{lem}[thm]{Lemma}
\newtheorem{cor}{Corollary}[thm]
\newtheorem{fact}{Fact}
\newtheorem{assumption}{Assumption}
\newtheorem{claim}{Claim}

\theoremstyle{definition}
\newtheorem{example}{Example}
\newtheorem{conjecture}{Conjecture}
\newtheorem{question}{Question}
\newtheorem{problem}{Problem}
\newtheorem*{prob*}{Problem}

\theoremstyle{remark}
\newtheorem{remark}{Remark}
\newtheorem{rems}{Remarks}
\newtheorem{para}{}
\newtheorem{summary}{Summary}

%-----------------------------------------------------------------------------------------
\newcounter{relctr} %% <- counter for relations
\everydisplay\expandafter{\the\everydisplay\setcounter{relctr}{0}} %% <- reset every eq
\renewcommand*\therelctr{\alph{relctr}} %% <- label format

\newcommand\labelrel[2]{%
  \begingroup
    \refstepcounter{relctr}%
    \stackrel{\textnormal{(\alph{relctr})}}{\mathstrut{#1}}%
    \originallabel{#2}%
  \endgroup
}
\AtBeginDocument{\let\originallabel\label} %% <- store original definition

\allowdisplaybreaks % makes automatic splits of equations
\global\long\def\11{\mathbbm{1}}
\newcommand{\du}[1]{ \uuline{ #1 } }
\newcommand\numberthis{\addtocounter{equation}{1}\tag{\theequation}}
\newcommand{\ra}{\rightarrow}
\newcommand{\la}{\leftarrow}
\newcommand{\mcl}{\mathcal}
\newcommand{\mbb}{\mathbb}
\newcommand{\mbf}{\mathbf}
\newcommand{\eps}{\epsilon}
\newcommand{\ov}{\overline}
\newcommand{\udl}{\underline}
\newcommand{\uda}{\underaccent}
\newcommand{\ulbar}[1]{ \uda{\bar}{\bar{#1}} }
\newcommand{\bsl}{\boldsymbol}
\def \defeq{\overset{\Delta}{=}}
\def \dg{\dagger}
\def \bsl{\boldsymbol}
\def \wc{\widecheck}
\def \wh{\widehat}
\def \wt{\widetilde}
\def \pr{\mbb{P}}
\def \xlrarrow{\xleftrightarrow}
\def \l{\left}
\def \r{\right}
\newcommand{\argmax}{\operatorname{argmax}}
\newcommand{\argmin}{\operatorname{argmin}}
\newcommand\mymathop[1]{\mathop{\operatorname{#1}}}


