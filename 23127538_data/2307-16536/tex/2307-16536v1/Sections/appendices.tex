% The \appendix command is used to start a single appendix.
% An optional argument can be used to specify a title:
% \appendix[Proof of the Zonklar Equations]
% After issuing \appendix, the \section command will be
% disabled and any attempt to use \section will be ignored
% and will cause a warning message to be generated. (The
% single appendix marks the end of the enumerated sections
% and the section counter is fixed at zero—one does not state
% “see Appendix A” when there is only one appendix, instead
% “see the Appendix” is used.) However, all lower \subsecti
% on commands and the \section* form will work as normal
% as these may still be needed for things like acknowledgments.
\appendices
% is used when there is more than one appendix
% section. \section is then used to declare each appendix:
% \section{Proof of the First Zonklar Equation}

% The mandatory argument to section can be left blank (\sect
% ion{}) if no title is desired. It is important to remember to
% declare a section before any additional subsections or labels
% that refer to section (or subsection, etc.) numbers. As with \appendix, the \section* command and the lower \subsection commands will still work as usual.

% Some authors prefer to have the appendix number to be part
% of equation numbers for equations that appear in an appendix.
% This can be accomplished by redefining the equation numbers
% as
\renewcommand{\theequation}{\thesection.\arabic{equation}}
% before the first appendix equation. For a single appendix, the
% constant “A” should be used in place of \thesection.

%------------------------------------------%	
%- INTERMEDIARY RESULTS FOR THEOREM 1 ----%	
%------------------------------------------%
\section{Intermediary Results for Theorem \ref{thm:strongduality}}\label{sec:appendix:intermediary_results}
\begin{lem}[Equivalence between Behavioral Policy-Profiles and their (decentralized) Mixtures]\label{lem:dominance}
Fix a (factorized) measure $\mu \in \uspacemix$. Then there exists a behavioral policy-profile $\udl{u}=\udl{u}(\mu) \in \uspace$, such that for any $t \in \mbb{N}$, $\hst{h}{t} \in \hstspace{t}$, and $\at{t} \in \aspace$,
\begin{align*}
    p\l( \mu, t, \hst{h}{t}, \at{t} \r) = p\l( u, t, \hst{h}{t}, \at{t} \r),
\end{align*}
where, for brevity and with slight abuse of notation,
\begin{align*}
p\l( \cdot, t, \hst{h}{t}, \at{t} \r) &= \prup{\cdot}{P_1}\l(\Hst{t} = \hst{h}{t}, \At{t} = \at{t} \r), \text{ and}\\
p\l( \cdot, t, \hst{h}{t} \r) &= \prup{\cdot}{P_1}\l(\Hst{t} = \hst{h}{t} \r).
\end{align*}
\end{lem}
\begin{proof}
Define $\udl{u}=\udl{u}(\mu) \in \uspace$ such that
\begin{align*}
    &\ut{\udl{u}}{t}\l( \at{t} | \hst{h}{t}\ \r) = \prod_{n=1}^{N} 
    \utn{\udl{u}}{t}{n} 
    \l( \atn{t}{n} | \hstn{h}{t}{0}, \hstn{h}{t}{n} \r)  \\
    &= \begin{cases}
    \frac{ p\l( \mu, t, \hst{h}{t}, \at{t} \r)
    }{p\l( \mu, t, \hst{h}{t}\r)}, &\text{if } p\l(\mu, t, \hst{h}{t} \r) \ne 0,\\
    \prod_{n=1}^{N} \frac{1}{|\anspace{n}|}, &\text{otherwise}.
    \end{cases}\numberthis\label{eq:dominance:u}
\end{align*}
The above assignment is correct because the right-hand-side of \eqref{eq:dominance:u} is a fully-factorized function of $\atn{t}{n}$'s.
\begin{align*}
    &p\l( \mu, t, \hst{h}{t}, \at{t} \r) = \int_{U} \mu\l( du \r) \prup{u}{P_1} \l( \Hst{t} = \hst{h}{t}, \At{t} = \at{t} \r)\\
    &= \int_{\uspace} P_1\l( \sspace, \hst{h}{1} \r) \prod_{t'=2}^{t} \pr_{P_1} \l( \ot{t'} | \hst{h}{t'-1}, \at{t'-1} \r)\\
    &\hspace{10pt} \times \prod_{n=1}^{N} \prod_{t'=1}^{t} \utn{u}{t'}{n}\l( \atn{t'}{n} |  \hstn{h}{t'}{0}, \hstn{h}{t'}{n} \r) \mu\l( du \r)
    % \l( \l( \mymathop{\times}_{n=1}^{N} \mun{n}\r)(du) \r)
    \\
    &=P_1\l( \sspace, \hst{h}{1} \r) \prod_{t'=1}^{t} \pr_{P_1} \l( \ot{t'} | \hst{h}{t-1}, \at{t'-1} \r)\\
    &\hspace{10pt} \times \prod_{n=1}^{N} \int_{\uspacen{n}}  \prod_{t'=1}^{t} \utn{u}{t'}{n}\l( \atn{t'}{n} |  \hstn{h}{t'}{0}, \hstn{h}{t'}{n} \r) \mun{n}\l( d\un{u}{n}\r),
\end{align*}
where the last equality follows from Tonneli's Theorem (see Proposition \ref{prop:tonneli}). We will now prove, by forward induction, that for all $t\in\mbb{N}$, $\udl{u}$ and $\mu$ induce the same distribution on the pair $\l( \Hst{t}, \At{t} \r)$.

\begin{enumerate}
    \item \textbf{Base Case}: For time $t=1$, let $\ot{1} \in \hstspace{1} = \ospace$ and $\at{1} \in \aspace$. We have
    \begin{align*}
        p\l( \mu, 1, \ot{1}, \at{1}\r) &= P_1\l( \sspace, \ot{1} \r) \int_{\uspace} \mu\l( du \r) \ut{u}{1}\l( \at{1} | \ot{1} \r),
    \end{align*}
    and
    \begin{align*}
        p\l( \udl{u}, 1, \ot{1}, \at{1}\r) &= P_1\l( \sspace, \ot{1} \r) \ut{\udl{u}}{1}\l( \at{1} | \ot{1} \r) \\
        &\hspace{-30pt}= P_1\l( \sspace, \ot{1} \r) \frac{p\l( \mu, 1, \ot{1}, \at{1}\r)}{p\l( \mu, 1, \ot{1}\r)}\\ 
        &%\hspace{-30pt}= P_1\l( \sspace, \ot{1} \r) \frac{p\l( \mu, 1, \ot{1}, \at{1}\r)}{P_1\l( \sspace, \ot{1} \r) } 
        \hspace{-30pt}=p\l( \mu, 1, \ot{1}, \at{1}\r),
    \end{align*}
    where the last equality follows from $p\l( \mu, 1, \ot{1}\r) = P_1\l( \sspace, \ot{1} \r) $.

    \item \textbf{Induction Step}. Assuming that the statement is true for time $t$, we show that it is true for time $t+1$. Let $\hst{h}{t+1} = \l( \ot{1:t+1}, \at{1:t} \r) = \l( \hst{h}{t}, \at{t}, \ot{t+1} \r) \in \hstspace{t+1}$ and $\at{t+1} \in \aspace$. We have
    \begin{align*}
        p\l(\mu, t+1, \hst{h}{t+1} \r) &= \int_{\uspace} \mu\l( du \r) \prup{u}{P_1} \l( \Hst{t+1} = \hst{h}{t+1} \r)\\
        &\hspace{-60pt} = \int_{\uspace} \mu\l( du \r) \prup{u}{P_1} \l( \Hst{t} = \hst{h}{t}, \At{t} = \at{t}, \Ot{t+1} = \ot{t+1} \r)\\
        &\hspace{-60pt}= \int_{\uspace} \mu\l( du \r) \prup{u}{P_1} \l( \Hst{t} = \hst{h}{t}, \At{t} = \at{t}, \r) \\
        &\hspace{-40pt} \times \prup{u}{P_1} \l( \Ot{t+1} = \ot{t+1} | \Hst{t} = \hst{h}{t}, \At{t} = \at{t} \r)\\
        &\hspace{-60pt}= \int_{\uspace} \mu\l( du \r) \prup{u}{P_1} \l( \Hst{t} = \hst{h}{t}, \At{t} = \at{t}, \r) \\
        &\hspace{-40pt} \times \pr_{P_1} \l( \Ot{t+1} = \ot{t+1} | \Hst{t} = \hst{h}{t}, \At{t} = \at{t} \r)\\
        &\hspace{-60pt}= p\l(\mu, t, \hst{h}{t}, \at{t}\r) \pr_{P_1} \l( \Ot{t+1} = \ot{t+1} | \Hst{t} = \hst{h}{t}, \At{t} = \at{t} \r)\\
        &\hspace{-60pt}\labelrel{=}{eqr:dominance:ind} p\l(\udl{u}, t, \hst{h}{t}, \at{t}\r) \pr_{P_1} \l( \Ot{t+1} = \ot{t+1} | \Hst{t} = \hst{h}{t}, \At{t} = \at{t} \r)\\
        &\hspace{-60pt}=p\l(\udl{u}, t+1, \hst{h}{t+1} \r),
    \end{align*}
where \eqref{eqr:dominance:ind} uses the inductive hypothesis. The above work implies 
\begin{align*}
&p\l(\udl{u}, t+1, \hst{h}{t+1}, \at{t+1} \r) \\
&\hspace{0pt} = p\l(\udl{u}, t+1, \hst{h}{t+1} \r) \cdot\ \ut{\udl{u}}{t+1}\l(\at{t+1} | \hst{h}{t+1} \r) \\
&\hspace{0pt} = p\l(\mu, t+1, \hst{h}{t+1} \r) \frac{p\l(\mu, t+1, \hst{h}{t+1}, \at{t+1} \r)}{p\l(\mu, t+1, \hst{h}{t+1} \r)}\\
&\hspace{0pt} = p\l(\mu, t+1, \hst{h}{t+1}, \at{t+1} \r).
\end{align*}
\end{enumerate}
This completes the proof.
\end{proof}


\begin{cor}\label{cor:lbar_and_l}	
Fix $\lambda \in \mcl{Y}$. For any $\mu \in \uspacemix$, there exists $u = u(\mu) \in \uspace$ such that $\lags{u}{\lambda} = \lagsmix{\mu}{\lambda}$.	
\end{cor}	
\begin{proof}	
One notes that $\wh{C}(\mu)$ and $\wh{D}(\mu)$ can be written as:	
\begin{align*}	
\wh{C}(\mu) &= \sum_{t=1}^{\infty} \alpha^{t-1} \E{\mu}{P_1} \l[ \mbb{E}_{P_1} \l[ c\l( \Stt{t}, \At{t} \r)  \r] | \Hst{t}, \At{t} \r],\\	
\wh{D}(\mu) &= \sum_{t=1}^{\infty} \alpha^{t-1} \E{\mu}{P_1} \l[ \mbb{E}_{P_1} \l[ d\l( \Stt{t}, \At{t} \r)  \r] | \Hst{t}, \At{t} \r],	
\end{align*}	
and the result follows.	
\end{proof}


\begin{lem}\label{lem:puth}[Limit Probabilities for a converging sequence of policy-profiles]
Let $\l\{ \useq{i}{u} \r\}_{i=1}^{\infty}$ be a sequence in $\uspace$ that converges to $u$. Then, for any $t \in \mbb{N}$, $ \hst{h}{t} \in \hstspace{t} $, and $\at{t} \in \mcl{A}$,
\begin{align*}
\lim_{i\ra \infty}  \pruphsts{\useq{i}{u}}{t}{\hst{h}{t}, \at{t}} = \pruphsts{u}{t}{\hst{h}{t}, \at{t}},
\end{align*}
where $\pruphsts{\cdot}{t}{\hst{h}{t}, \at{t}} = \prup{\cdot}{P_1} \l( \Hst{t} = \hst{h}{t}, \At{t} = \at{t} \r)$. In other words, for every $t \in \mbb{N}$, the sequence of measures $\l\{ \pruphsts{ \useq{i}{u}}{t}{\cdot, \cdot} \r\}_{i=1}^{\infty}$ converges weakly to $\pruphsts{u}{t}{\cdot, \cdot}$.
\end{lem}
\begin{proof}
Given that $\useq{i}{u}$ converges to $u$, by the definition of convergence in product topology, for every $n \in [N]$, $\useq{i}{\utn{u}{t}{n}} (\hstn{h}{t}{0}, \hstn{h}{t}{n} )$ converges weakly to $ \utn{u}{t}{n} ( \hstn{h}{t}{0}, \hstn{h}{t}{n} ) $. Since $\mcl{A}^n$ is finite, this means that for each $\an{n}\in \anspace{n}$, $\useq{i}{\utn{u}{t}{n}} ( \an{n} | \hstn{h}{t}{0}, \hstn{h}{t}{n} )$ converges to $\utn{u}{t}{n} ( \an{n} | \hstn{h}{t}{0}, \hstn{h}{t}{n} )$, which further implies that for all $a \in \aspace$, $ \useq{i}{\ut{u}{t}} ( a | \hst{h}{t}) $ converges to $ \ut{u}{t} ( a | \hst{h}{t}) $. Now, we use forward induction to prove the statement. 
\begin{enumerate}
\item \textbf{Base Case}:  For time $t=1$, let $\ot{1} \in \hstspace{1} = \ospace$ and $\at{1} \in \mcl{A}$. We have
\begin{align*}
\pruphsts{\useq{i}{u}}{1}{\ot{1}, \at{1}}
=P_1\l( \sspace, o \r) \useq{i}{\ut{u}{1}} \l( \at{1} | \ot{1} \r) 
\ra \pruphsts{u}{1}{\ot{1}, \at{1}}.
\end{align*}

\item \textbf{Induction Step}: Assuming that the statement is true for time $t$, we show that it is true for time $t+1$. Let $\hst{h}{t+1} = \l( \ot{1:t+1}, \at{1:t} \r) = \l( \hst{h}{t}, \at{t}, \ot{t+1} \r) \in \hstspace{t+1}$ and $\at{t+1} \in \aspace$. We have
\begin{align*}
&\pruphsts{\useq{i}{u}}{t+1}{\hst{h}{t+1}, \at{t+1}} \\
&\hspace{0pt} =  
\pruphsts{\useq{i}{u}}{t}{\hst{h}{t}, \at{t}} \useq{i}{\ut{u}{t+1}} \l( \at{t+1} | \hst{h}{t+1} \r) \\
&\hspace{5pt} \times \pr_{P_1} \l( \Ot{t+1} = \ot{t+1} | \Hst{t} = \hst{h}{t}, \At{t} = \at{t} \r).
\end{align*}
By inductive hypothesis, $\pruphsts{\useq{i}{u}}{t}{\hst{h}{t}, \at{t}} $ converges to $\pruphsts{u}{t}{\hst{h}{t}, \at{t}}$, and $ \useq{i}{\ut{u}{t}}\l( \at{t+1} | \hst{h}{t+1}\r) $ converges to $ \ut{u}{t} \l( \at{t+1} | \hst{h}{t+1}\r) $ by assumption. We conclude that $\pruphsts{\useq{i}{u}}{t+1}{\hst{h}{t+1}, \at{t+1}}$ converges to $\pruphsts{u}{t+1}{\hst{h}{t+1}, \at{t+1}}$.
\end{enumerate}
This completes the proof.
\end{proof}


%---------------------------------------------%
%-------------- HELPFUL FACTS ----------------%
%---------------------------------------------%
\section{Helpful Facts and Results}\label{sec:appendix:helpful_facts}
\begin{dfn}[Semi-continuity]\label{dfn:lsc}
A function $f : \mcl{X} \mapsto [-\infty, \infty]$ on a topological space $\mcl{X}$ is called \emph{lower semi-continuous} if for every point $x_0 \in \mcl{X}$,
\begin{align*}
\liminf\limits_{x\ra x_0} f(x) \ge f(x_0).
\end{align*}
We call $f$ as an upper semi-continuous function $-f$ is lower semi-continuous.
\end{dfn}

\begin{prop}[Monotone Convergence Theorem]\label{prop:mct}
    Let $\l(X, \mcl{M}, \mu \r)$ be a measure-space. Let $\l\{ f_i \r\}_{i=1}^{\infty}$ be an increasing sequence of measurable functions which are uniformly bounded-from-below. Then, 
    \begin{align*}
        &\int_{X} \lim_{i\ra\infty} f_i(x) \mu(dx) = \lim_{i\ra\infty} \int_{X} f_i(x) \mu(dx). 
    \end{align*}
\end{prop}


\begin{prop}[Tonneli's Theorem]\label{prop:tonneli}
    Let $f$ be a measurable function on the cartesian product of two $\sigma$-finite measure spaces $(X, \mcl{M}, \mu)$ and $(Y, \mcl{N}, \nu)$ which is bounded from below. Then, 
    \begin{align*}
        &\int_{X\times Y} f(x,y) (\mu \times \nu) (d(x,y))\\ 
        &\hspace{10pt} =\int_X \l( \int_Y f(x,y) \nu(dy)\r) \mu(dx)\\
        &\hspace{10pt} =\int_Y \l(\int_X f(x,y) \mu(dx)\r) \nu(dy).
    \end{align*}
\end{prop}

\begin{prop}[Fatou's Lemma]\label{prop:fatou}
    Let  $(X, \mcl{M}, \mu)$ be a measure-space and let $\{ f_i \}_{i=1}^{\infty}$ be a sequence of measurable functions which are uniformly bounded from below. Then,
    \begin{align*}
        & \liminf_{i\ra\infty} \int f_i(x) \mu (dx)\ge \int \liminf_{i\ra\infty} f_i(x) \mu(dx).
    \end{align*}
\end{prop}

\begin{prop}[Tychonoff's Theorem]\label{prop:tychonoff}
Product of countable number of compact spaces is compact under the product topology.
\end{prop}

\begin{prop}[Metrizability of Product Topology on Countable Product of Metric Spaces]\label{prop:metrizability}
Product of countable number of metric spaces, when endowed with the product topology, is metrizable.
\end{prop}

% \begin{prop}[Liminf of a Product of two Sequences]\label{prop:liminfproduct}
% Let $\l\{ a_i \r\}_{i=1}^{\infty}$ and $\l\{ b_i \r\}_{i=1}^{\infty}$ be two sequences such that $\lim_{i\ra\infty} a_i = a \ge 0$ and $ \l\{ b_i \r\}_{i=1}^{\infty} $ is bounded, i.e., $|b_i| \le \eta < \infty$. Then
% \begin{align*}
%     \liminf_{i\ra \infty} a_i b_i = a \liminf_{i\ra \infty} b_i.
% \end{align*}
% \end{prop}

\begin{prop}[Prokhorov's Theorem]\label{prop:prokhorov}
Let $\l( \mcl{X}, \metric{\mcl{X}} \r)$ be a complete separable metric space with distance metric $\metric{\mcl{X}}$ and let $\borel{\mcl{X}}$ denote the Borel $\sigma$-algebra generated by $\metric{\mcl{X}}$. Let $\m{\mcl{X}}$ be the set of all probability measures on $\borel{\mcl{X}}$ and let $\tau$ denote the topology of weak-convergence on $\m{\mcl{X}}$. Then,  
\begin{enumerate}
    \item The topological space $ \l(\m{\mcl{X}} , \tau\r)$ is completely-metrizable. That is, there exists a complete metric $\metric{\m{\mcl{X}}}$ on $ \m{\mcl{X}}$ that induces the same topology as $ \tau $.
    \item An arbitrary collection $W \subseteq \m{\mcl{X}}$ of probability measures in $ \m{\mcl{X}}$ is tight iff its closure in $\tau $ is compact (i.e., $W$ is precompact in $\tau$).
\end{enumerate}
\end{prop}


% \begin{prop}[Generalized Dominated Convergence Theorem]\label{prop:generalizeddct}
% Let $(X, \salgebra)$ be a measurable space and $\l( \mu_i \r)_{i=1}^{\infty}$ a sequence of measures on $\salgebra$ that converge set-wise to a measure $\mu$. Let $\{f_i\}_{i=1}^{\infty}$ and $\{g_i\}_{i=1}^{\infty}$ be a sequence of measurable functions that converge point-wise to $f$ and $g$ respectively. Suppose that $|f_i| \le g_i$ (almost surely) and that $\lim_{i\ra \infty} \int g_i(x) \mu_i(dx) = \int g(x) \mu(dx) < \infty$. Then,
% \begin{align*}
% \lim_{i \ra \infty} \int f_i(x) \mu_i(dx) = \int f(x) \mu(dx).
% \end{align*}
% \end{prop}
% \begin{proof}
%     See \cite{}.
% \end{proof}

\begin{prop}[Hyperplane Separation Theorem]\label{prop:separation_theorem}
Let $M$ be a non-empty convex subset of 
$\mbb{R}^n$. If $x_0 \in \mbb{R}^n$ does not belong to $M$, there exists $\rho \in \mbb{R}^n$ such that
\begin{align*}
\rho \neq 0 \text { and } \inf_{x \in M} \dotp{p}{x} \geq \dotp{p}{x_0}.
\end{align*}
\end{prop}


\begin{prop}[Integral of Bounded-from-Below function with respect to Convex Combination of Non-negative Measures]\label{prop:integral_linearity}
Let $\l(X, \mcl{M}\r)$ be a measure-space. Let $f : X \ra \mbb{R} \cup \{ \infty \}$ be a measurable function that is bounded from below, and let $\mu, \nu$ be two non-negative measures on $\mcl{M}$. Then, for any $\theta \in [0,1]$,
\begin{align*}
    &\int f(x) \l(\theta \mu + (1-\theta) \nu \r)(dx) \\
    &\hspace{10pt} = \theta \int f(x) \mu(dx) + (1-\theta) \int f(x)\nu(dx). 
\end{align*}
    
\end{prop}


\begin{prop}[Behavior of Integrals of a Bounded-from-Below and Lower Semi-Continuous Function]\label{prop:lsc}
Let $(\mcl{X}, \metric{\mcl{X}})$ be a complete separable metric space with distance metric $\metric{\mcl{X}}$ and let $\borel{\mcl{X}}$ denote the Borel $\sigma$-algebra generated by $\metric{\mcl{X}}$. Let $\l( \m{\mcl{X}} , \metric{\m{\mcl{X}}} \r)$ be the complete metric space of all probability measures on $\borel{\mcl{X}}$ with the topology of weak-convergence.\footnote{Prokhorov's theorem (see Proposition \ref{prop:prokhorov}) ensures completeness and metrizability of $\m{\mcl{X}}$.} Let $\mu \in \m{\mcl{X}}$ and let $f : \mcl{X} \ra \mbb{R} \cup \l\{ \infty\r\} $ be a function that is lower semi-continuous $\mu$-amost-everywhere\footnote{Lower semi-continuity of $f$ ensures that it is measurable.} and is bounded from below. Then, the function
\begin{align*}
H : \m{\mcl{X}} \mapsto \mbb{R} \cup \l\{ \infty \r\},\  %\\
H(\mu') \defeq \int f(x) \mu'(dx)
\end{align*}
is lower semi-continuous at $\mu$. In particular, if $f$ is point-wise lower semi-continuous, then $H$ is also point-wise lower semi-continuous (on $\m{\mcl{X}}$).
\end{prop}
\begin{proof}
% The proof is omitted due to space reasons.
% \begin{comment}
Define $f' : \mcl{X} \ra \mbb{R} \cup \{\infty \}$ as $f'(x) \defeq f(x) \wedge \liminf_{y\ra x} f(y)$. Then, $f'$ minorizes $f$\footnote{That is, $f'(x) \le f(x)$.}, is lower semi-continuous, and coincides with $f$ at $x$ if and only if $f$ is lower semi-continuous at $x$. Also, $f'$ is bounded from below (since $f$ is). By Proposition \ref{prop:lsc3}, $f'$ can be written as the point-wise limit of increasing sequence of uniformly bounded-from-below continuous functions from $\mcl{X}$ into $\mbb{R} \cup \{ \infty \}$, say $\l\{ g_i %: \mcl{X} \ra \mbb{R} \cup \{\infty\}
\r\}_{i=1}^{\infty} $, i.e., $f'(x) = \lim_{i\ra \infty} g_i(x)$. Then, for every $\mu' \in \m{\mcl{X}}$,
\begin{align*}
\int f'(x)\mu'(dx) = \int \lim_{i\ra \infty} g_i(x) \mu'(dx) = \lim_{i\ra \infty} \int g_i(x)\mu'(dx),
\end{align*}
where the last equality follows from the Montone Convergence Theorem (see Proposition \ref{prop:mct}). The above equality shows that the function $H' : \m{\mcl{X}} \ra \mbb{R} \cup \{ \infty\}$ such that $H'(\mu') = \int f'(x) \mu'(dx)$, is the point-wise limit of an increasing sequence of uniformly bounded-from-below continuous functions. Therefore, by Proposition \ref{prop:lsc3}, $H'$ is lower semi-continuous. Now, if $f$ is lower semi-continuous $\mu$-almost-everywhere, then $f = f'$ $\mu-$almost-everywhere. This gives,
\begin{align*}
H(\mu) &= \int f(x) \mu(dx) \\
&= \int f'(x) \mu(dx) \\
&\labelrel{=}{eqr:lsc:H2islsc} \liminf_{\mu'\ra\mu} H'(\mu') \\
&\labelrel{\le}{eqr:lsc:H2minorizesH} \liminf_{\mu'\ra\mu} H(\mu'),
\end{align*}
Here, \eqref{eqr:lsc:H2islsc} uses lower semi-continuity of $H'$ and \eqref{eqr:lsc:H2minorizesH} follows from the fact that $H'$ minorizes $H$ (since $f'$ minorizes $f$). The inequality $H(\mu) \le \liminf_{\mu'\ra\mu} H(\mu')$ is the definition of lower semi-continuity at $\mu$. 
% \end{comment}
\end{proof}

\begin{prop}[Equivalent Characterization of a Bounded-from-Below Lower Semi-Continuous Function]\label{prop:lsc3}
Let $\l( \mcl{X}, \metric{\mcl{X}} \r)$ be a metric space. Then, a function $f : \mcl{X} \ra \mbb{R} \cup \{ \infty \}$ is a bounded-from-below lower semi-continuous function if and only if it can be written as the point-wise limit of an increasing sequence of uniformly bounded-from-below continuous functions from $\mcl{X}$ into $\mbb{R} \cup \{ \infty \}$. 
\end{prop}
\begin{proof}
% The proof is omitted due to space reasons.
% \begin{comment}
\textbf{Necessity}: Define $f_n : \mcl{X} \ra \mbb{R} \cup \{ \infty \}$ as follows:
\begin{align*}
f_n\l( x \r) &\defeq \inf_{y\in\mcl{X}} \l\{ f(y) + n \metric{\mcl{X}} \l(x, y\r) \r\}.
\end{align*}
\begin{enumerate}
\item \textit{Increasing}: 
\begin{align*}
f_{n+1}\l( x \r) = \inf_{y\in\mcl{X}} \l\{ f(y) + (n+1)\metric{\mcl{X}}\l( x,y\r) \r\} \ge f_n(x).
\end{align*}
\item \textit{Uniformly Bounded-from-Below}: Since $f_n\l(x \r) \ge \inf_{y\in\mcl{X}} \l\{ f(y) \r\}$ and $f$ is bounded-from-below, the functions $\l\{ f_n\r\}_{n=1}^{\infty}$ are uniformly bounded-from-below.
\item \textit{Continuity}: By triangle-inequality,
\begin{align*}
f(y) + n\metric{\mcl{X}}\l(y, z\r) \le
f(y) + n\metric{\mcl{X}}\l(y, w\r) +  n\metric{\mcl{X}}\l(w, z\r),
\end{align*}
and therefore, taking the infimum over $y$ on both sides gives $ f_n\l( z \r) - f_n\l( w \r) \le n\metric{\mcl{X}}\l( w, z\r) $. Similarly, we can get $ f_n\l( w \r) - f_n\l( z \r) \le n\metric{\mcl{X}}\l( w, z\r) $, and so
\begin{align*}
|f_n\l( z \r) - f_n\l( w \r)| \le n \metric{\mcl{X}} \l(w, z\r).
\end{align*}
The above relation shows that $f_n$ is Lipschitz and thus continuous.
\item \textit{Point-wise Convergence to $f$}: Fix $x_0 \in \mcl{X}$ and $\eps>0$. We would like to show that there exists a positive integer $n' = n'(x_0, \eps)$ such that, for all $ n \ge n'$, $| f_n\l(x_0\r) - f\l(x_0\r) | < \eps$. Since $f$ is lower semi-continuous at $x_0$, there exists $\delta = \delta(x_0, \eps) > 0$ such that
\begin{align*}
\metric{\mcl{X}}\l( x_0, y\r) < \delta \implies f(y) >  f(x_0) -\eps.\numberthis\label{eq:lsc2:implication}
\end{align*}
Since $f$ is bounded-from-below (and $\delta>0$), there exists a positive integer $n'=n'(\delta(x_0,\eps))$ such that
\begin{align*}
&\metric{\mcl{X}}\l( x_0, y\r) \ge \delta\\
&\hspace{0pt} \implies \forall\  n\ge n', f(y) + n\metric{\mcl{X}}\l( x_0, y\r) > f(x_0)\\
&\hspace{0pt} \implies \forall\  n\ge n', \inf_{\metric{\mcl{X}}\l( x_0, y\r) \ge \delta } \l\{ f(y) + \metric{\mcl{X}}(x_0, y) \r\}\ge f\l(x_0\r).
\end{align*}
So, for all $n\ge n'$, we have
\begin{align*}
f(x_0) \ge f_n\l( x_0 \r) &= \inf_{\metric{\mcl{X}}\l( x_0, y\r) \le \delta } \l\{ f(y) + n\metric{\mcl{X}}(x_0, y) \r\}\\
&\ge\inf_{\metric{\mcl{X}}\l( x_0, y\r) \le \delta } \l\{ f(y) \r\}\\
&\labelrel{>}{eqr:lsc2:1}\inf_{\metric{\mcl{X}}\l( x_0, y\r) \le \delta } \l\{ f(x_0) - \eps \r\}\\
&=f(x_0) - \eps.
\end{align*}
where \eqref{eqr:lsc2:1} uses \eqref{eq:lsc2:implication}.
\end{enumerate}
\hspace{5pt} \textbf{Sufficiency}: Let $\l\{ f_n \r\}_{n=1}^{\infty} $ be an increasing sequence of uniformly bounded-from-below continuous functions from $\mcl{X}$ into $\mbb{R} \cup \l\{ \infty \r\}$. Since the sequence is monotonic, it has a point-wise-limit $f : \mcl{X} \ra \mbb{R} \cup \l\{ \infty \r\}$ which is bounded-from-below because all the functions in the sequence are uniformly bounded-from-below. We need to show that $f$ is lower semi-continuous. 

Fix $x_0 \in \mcl{X}$ and $\eps>0$. We would like to show that there exists $\delta = \delta(x_0,\eps)>0$ such that $\metric{\mcl{X}}\l( x_0, y\r) < \delta \implies f(y) >  f(x_0) -\eps $. Since  $\l\{ f_n \r\}_{n=1}^{\infty} $ is increasing (and converges point-wise to $f$), there exists a positive integer $n'=n'(x_0, \eps)$ such that, for all $n\ge n'$, $f(x_0) \ge f_n(x_0) \ge f(x_0) - \frac{\eps}{2}$. Since $f_{n'}$ is lower semi-continuous, there exists $\delta=\delta(n'(x_0, \eps)) > 0$ such that $\metric{\mcl{X}}\l( x_0, y\r)<\delta \implies f(y) \ge f_{n'}(y) > f_{n'}(x_0) - \frac{\eps}{2} \ge f(x_0) - \eps$. 
\end{proof}



\begin{prop}[Banach Fixed-Point Theorem]\label{prop:banach}
    Let $\l(\mcl{X}, \metric{\mcl{X}}\r)$ be a (non-empty) complete metric space with a contraction mapping $T: \mcl{X} \ra \mcl{X}$. Then $T$ admits a unique fixed-point $x^\star$ in $\mcl{X}$ (i.e. $T\l(x^\star\r)=x^\star$ ). Furthermore, $x^\star$ can be found as follows: start with an arbitrary element $x_0 \in \mcl{X}$ and define a sequence $\l(x_i\r)_{i \in \mbb{N}}$ by $x_i = T\l(x_{i-1}\r)$ for $i \in \mbb{N}$. Then, $\lim _{i \ra \infty} x_i=x^\star$.
\end{prop}

%---------------------------------------------%
%------------- MINIMAX THEOREM ---------------%
%---------------------------------------------%
\section{A Minimax Theorem for Functions with Positive Infinity
}\label{sec:appendix:minimax}

\begin{prop}[A Minimax Theorem For Functions with Positive Infinity]\label{prop:sionminimax}
Let $\mcl{X}$ and $\mcl{Y}$ be convex topological spaces where $\mcl{X}$ is also compact. Consider a function $f : \mcl{X} \times \mcl{Y} \ra \mbb{R} \cup \{ \infty \} $ such that
\begin{enumerate}
\item for each $y \in \mcl{Y}$, $f\l(\cdot, y \r)$ is convex and lower semi-continuous.
\item for each $x \in \mcl{X}$, $f\l(x, \cdot \r)$ is concave.
\item If $f (x, y) = \infty$, then $f(x, y') = \infty$ for all $y'\in\mcl{Y}$.
\end{enumerate}
Then, there exists $x^\star \in \mcl{X}$ such that
\begin{align*}
\sup_{y\in \mcl{Y}} f\l( x^\star, y \r) &=
\inf_{x \in \mcl{X}} \sup_{y \in \mcl{Y}} f\l( x, y \r)\\
&=\sup_{y \in \mcl{Y}} \inf_{x \in \mcl{X}} f(x, y).
% &\hspace{0pt}
\end{align*}
\end{prop}

Proposition \ref{prop:sionminimax} is a mild adaptation of the Minimax theorem presented in \cite{aubin_book_2002}[Theorem 8.1] where a real-valued function is considered. In the MA-C-POMDP model described in Section \ref{sec:problem}, it is possible that $\fullccosts{u}$ and hence $\lags{u}{\lambda}$ is $\infty$ for all $\lambda \in \mcl{Y}$. We will use the same methodology as in \cite{aubin_book_2002}[Propositions 8.2 and 8.3] to prove Proposition \ref{prop:sionminimax}. In particular, the entire proof remains the same except that in Lemma \ref{lem:lem8.2:aubin:modified}, the compactness of $\mcl{X}$ is used together with Assumption 3). 

Define
\begin{align*}
f^{\sharp}(x) & :=\sup_{y \in \mcl{Y}} f(x, y), & & v^{\sharp}:=\inf_{x \in \mcl{X}} \sup_{y \in \mcl{Y}} f(x, y) \numberthis\\
f^b(y) & :=\inf_{x \in \mcl{X}} f(x, y), & & v^{\flat}:=\sup_{y \in \mcl{Y}} \inf_{x \in \mcl{X}} f(x, y).\numberthis
\end{align*}
To show the equality of $v^{\sharp}$ and $v^{\flat}$, we will introduce an intermediate value $v^{\natural}$ ($v$ natural) and prove successively that $v^{\natural}=v^{\sharp}$ and that $v^{\natural}=v^{\flat}$. 

We denote the family of finite subsets $J$ of $\mcl{Y}$ by $\mcl{J}$. We set
$$
v_J^{\sharp}:=\inf_{x \in \mcl{X}} \sup_{y \in J} f(x, y)
$$
and
$$
v^{\natural}:=\sup_{J \in \mcl{J}} v_J^{\sharp}=\sup_{J \in \mcl{J}} \inf_{x \in \mcl{X}} \sup_{y \in J} f(x, y).
$$
Since every point $y$ of $\mcl{Y}$ may be identified with the finite subset $\{y\} \in \mcl{J}$, we note that $v_{\{y\}}^{\sharp}=f^b(y)$ and consequently, $v^{\flat}=\sup_{y \in \mcl{Y}} v_{\{y\}}^{\sharp} \leq \sup_{J \in \mcl{J}} v_J^{\sharp} = v^{\natural}$. Also, since $\sup_{y \in J} f(x, y) \leq \sup_{y \in \mcl{Y}} f(x, y)$, we deduce that $v_J^{\sharp} \leq v^{\sharp}$, and hence $v^{\natural} \leq v^{\sharp}$. In summary, we have shown that
\begin{align*}
v^{\flat} \leq v^{\natural} \leq v^{\sharp} .
\end{align*}
Lemma \ref{lem:prop8.2:aubin} shows that $v^{\natural} = v^{\sharp} $ and Lemma \ref{lem:prop8.3:aubin} shows that $v^{\flat} = v^{\natural}$. This concludes the proof.

%---------------------------------------------%
%-------------- PROP 8.2 AUBIN ---------------%
%---------------------------------------------%
\begin{lem}\label{lem:prop8.2:aubin}
Consider a function $f : \mcl{X} \times \mcl{Y} \mapsto \mbb{R} \cup \{ \infty \} $ such that $\mcl{X}$ is compact and for each $
y \in \mcl{Y}$, $f(\cdot, y)$ is lower semi-continuous. Then, there exists $x^\star \in \mcl{X}$ such that
$$
\sup_{y \in \mcl{Y}} f(x^\star, y)=v^{\sharp}
$$
and
$$
v^{\natural}=v^{\sharp} .
$$
\end{lem}

\begin{rem}
Since the functions $f(\cdot, y)$ are lower semi-continuous, the same is true of the function $f^{\sharp}$.\footnote{Supremum of arbitrary collection of lower semi-continuous functions is lower semi-continuous.} Since $\mcl{X}$ is compact, Weierstrass's theorem implies the existence of $x^\star \in \mcl{X}$ which minimises $f^{\sharp}$. Following (3), this may be written as
\begin{align*}
&\sup_{y \in \mcl{Y}} f(x^\star, y) = f^{\sharp}(x^\star) = \inf_{x \in \mcl{X}} f^{\sharp}(x) \\
&\hspace{50pt} = \inf_{x \in \mcl{X}} \sup_{y \in \mcl{Y}} f(x, y)=v^{\sharp}.
\end{align*}
In comparison to this, Lemma \ref{lem:prop8.2:aubin} proves that $v^{\natural} = v^{\sharp} $.
\end{rem}

\begin{proof}
It suffices to show that there exists $x^\star \in \mcl{X}$ such that
\begin{align*}
\sup_{y \in \mcl{Y}} f(x^\star, y) \leq v^{\natural}.\numberthis\label{eq:vsharp<=vnatural}
\end{align*}
Since $v^{\sharp} \leq \sup_{y \in \mcl{Y}} f(x^\star, y)$ and $v^{\natural} \leq v^{\sharp}$, we shall deduce that $v^{\natural}=v^{\sharp}$.
We set
$$
S_{y}:=\l\{x \in \mcl{X} \mid f(x, y) \leq v^{\natural}\r\}.
$$
The inequality \eqref{eq:vsharp<=vnatural} is equivalent to the inclusion
\begin{align*}
x^\star \in \bigcap_{y \in \mcl{Y}} S_{y}.\numberthis\label{eq:nonemptyintersection}
\end{align*}
Thus, we must show that this intersection is non-empty.
For this, we shall prove that the $S_{y}$ are closed sets (inside the compact set $\mcl{X}$) with the finite-intersection property.\footnote{The intersection of an arbitrary collection of closed sets that lie inside a compact set and satisfy the finite-intersection property, is non-empty.}

If $v^{\natural} = \infty$, then every $S_y$ equals $\mcl{X}$ and the intersection is trivially non-empty. Therefore, WLOG, assume that $v^{\natural}$ is finite. Then the set $S_{y}$ is a lower section of the lower semi-continuous function $f(\cdot, y)$ and is thus closed.\footnote{The lower section of a lower semi-continuous function is closed. For every $\eta \in \mbb{R}$, the corresponding lower section is defined as $\{x \in \mcl{X} : f(x) \le \eta \}$.}

We show that for any finite sequence $J :=\l\{y_{1}, y_{2}, \ldots, y_{n}\r\} \in \mcl{J}$ of $\mcl{Y}$, the finite intersection
$$
\bigcap_{i \in [n]} S_{y_i} \neq \emptyset
$$
is non-empty. In fact, since $\mcl{X}$ is compact, and since $\max_{y \in J} f(\cdot, y) $ is lower semi-continuous, it follows that there exists $\hat{x} \in \mcl{X}$ which minimises this function. Such an $\hat{x} \in \mcl{X}$ satisfies
\begin{align*}
\max_{y \in J} f(\hat{x}, y) &= \inf_{x \in \mcl{X}} \max_{y \in J} f(x, y) \\
&\leq \sup_{J \in \mcl{J}} \inf_{x \in \mcl{X}} \max_{y \in J} f(x, y)=v^{\natural} .
\end{align*}
Since $\mcl{X}$ is compact, the intersection of the closed sets $S_{y}$ is non-empty and there exists $x^\star \in \mcl{X}$ satisfying \eqref{eq:nonemptyintersection} and thus \eqref{eq:vsharp<=vnatural}.
\end{proof}

%---------------------------------------------%
%-------------- PROP 8.3 AUBIN ---------------%
%---------------------------------------------%
\begin{lem}\label{lem:prop8.3:aubin}
Consider a function $f : \mcl{X} \times \mcl{Y} \mapsto \mbb{R} \cup \{ \infty \} $ such that $\mcl{X}$ and $\mcl{Y}$ are convex sets, (i) for each $y \in \mcl{Y}$, $f(\cdot, y)$ is convex, and (ii) for each $x \in \mcl{X}$, $f(x, \cdot)$ is concave. Then, $v^{\flat}=v^{\natural}$.
\end{lem}
\begin{proof}
We set $M_J:=\l\{\lambda \in \mbb{R}_{\ge 0}^{|J|} \mid \sum_{i=1}^n \lambda_i=1\r\}$. With any finite (ordered) subset $J \defeq = \l\{y_1, y_2, \ldots, y_n\r\}$, we associate the mapping $\phi_J$ from $\mcl{X}$ to $\l( \mbb{R} \cup \{ \infty \} \r)^{|J|}$ defined by
$$
\phi_J(x):=\l(f\l(x, y_1\r), \ldots, f\l(x, y_n\r)\r)
$$
We also set
$$
w_J:=\sup_{\lambda \in M_J} \inf_{x \in \mcl{X}} \dotp{\lambda}{\phi_J(x)}
$$
We prove successively that
\begin{enumerate}
    % \item $\phi_J(\mcl{X}_J)+\mbb{R}_{\ge 0}^{|J|}$ is a convex subset.
    \item $\sup_{J\in\mcl{J}} w_J \leq v^{\flat}$ (Lemma \ref{lem:lem8.3:aubin}).
    \item $\sup_{J\in\mcl{J}} v^{\sharp}_{J}  \le \sup_{J\in\mcl{J}} w_J$ (Lemma \ref{lem:lem8.2:aubin:modified}).
\end{enumerate}
Hence, the inequalities
\begin{align*}
v^{\natural} = \sup_{J \in \mcl{J}} v_J^{\sharp} \leq \sup_{J \in \mcl{J}} w_J \leq v^{\flat} \leq v^{\natural}
\end{align*}
imply the desired equality 
 $v^{\flat}=v^{\natural}$.
\end{proof}




\begin{lem}\label{lem:lem8.3:aubin}
Consider a function $f : \mcl{X} \times \mcl{Y} \mapsto \mbb{R} \cup \{ \infty \} $ such that $\mcl{Y}$ is convex and for each $x \in \mcl{X}$, $f(x, \cdot)$ is concave. Then, for any finite subset $J$ of $\mcl{Y}$, we have $w_J \leq v^{\flat}$. Hence, $$\sup_{J\in\mcl{J}} w_J \le v^{\flat}.$$    
\end{lem}
\begin{proof}
With each $\lambda \in M_J$, we associate the point $y_\lambda:=\sum_{i=1}^n \lambda_i y_i$ which belongs to $\mcl{Y}$ since $\mcl{Y}$ is convex. The concavity of the functions $\l\{ f(x, \cdot)\r\}_{x\in\mcl{X}}$ implies that
\begin{align*}
\forall x \in \mcl{X}, \quad \sum_{i=1}^n \lambda_i f\l(x, y_i\r) \leq f\l(x, y_\lambda\r).
\end{align*}
Consequently,
\begin{align*}
\inf_{x \in \mcl{X}} \sum_{i=1}^n \lambda_i f\l(x, y_i\r) &\leq \inf_{x \in \mcl{X}} f\l(x, y_\lambda\r) \\
&\leq \sup_{y \in \mcl{Y}} \inf_{x \in \mcl{X}} f(x, y) \defeq v^{\flat}.
\end{align*}
The proof is completed by taking the supremum over $M_J$.
\end{proof}




\begin{lem}\label{lem:lem8.2:aubin:modified}
Consider a function $f : \mcl{X} \times \mcl{Y} \mapsto \mbb{R} \cup \{ \infty \} $ such that $\mcl{X}$ is a convex compact topological space, for each $y \in \mcl{Y}$, $f(\cdot, y)$ is convex and lower semi-continuous, and $f (x, y) = \infty$ implies $f(x, y') = \infty$ for all $y'\in\mcl{Y}$.
Then, 
\begin{align*}
v^{\natural} \defeq \sup_{J \in \mcl{J}} v_J^{\sharp} \leq \sup_{J \in \mcl{J}} w_J .
\end{align*}
\end{lem}
\begin{proof}
WLOG we assume that $\sup_{J \in \mcl{J}} w_J < \infty$. In this case, we can rewrite $w_J$ as $\supinf{\lambda\in M_J}{x\in \mcl{X}_J} \dotp{\lambda}{\phi_J(x)}$ where 
$$\mcl{X}_J \defeq \bigcap_{y\in J} dom f(\cdot, y).$$
To see this, note that $\dotp{\lambda}{\phi_J(x)} $ is a lower semi-continuous function on the compact space $\mcl{X}$. By Weierstrass theorem, $\dotp{\lambda}{\phi_J(x)} $ achieves its minimum in $\mcl{X}$ and we can write $w_J = \sup_{\lambda \in M_J} \dotp{\lambda}{\phi_J(\hat{x}(\lambda))}$. Suppose that $\hat{x}(\lambda) \in \mcl{X} \setminus \mcl{X}_J$, i.e., there exists $y \in J$ such that $\hat{x}(\lambda) \notin dom f(\cdot, y)$. This implies that $\hat{x}(\lambda) \notin dom f(\cdot, y')$ for all $y' \in J$. This renders $w_J$ to be infinity which contradicts our assumption $\sup_{J\in\mcl{J}} w_J < \infty $.

Therefore, now onward, we assume each $w_J = \sup_{\lambda\in M_J} \inf_{x \in \mcl{X}_J} \dotp{\lambda }{ \phi_J(x) }$. To prove the lemma, it suffices to show that $v_J^{\sharp} \le w_J $. Let $\eps>0$ and denote $\mbf{1} \defeq (1, \ldots, 1)$. We shall show that
\begin{align*}
\l( w_J + \eps \r) \mbf{1} \in \phi_J(\mcl{X}_J) + \mbb{R}_{\ge 0}^n .\numberthis\label{eq:wj_in_convex_set}
\end{align*}
Suppose that this is not the case. Since $\phi_J(\mcl{X}_J)+\mbb{R}_{\ge 0}^n$ is a convex set in $\mbb{R}^n $, following Lemma \ref{lem:lem8.1:aubin:modified}, we may use the hyperplane separation theorem (see Proposition \ref{prop:separation_theorem}), via which there exists $\rho \in \mbb{R}^n$, $\rho \neq 0$, such that
\begin{align*}
\sum_{i=1}^n \rho_i \l( w_J + \eps \r) &=\dotp{\rho}{\l(w_J + \eps\r) \mbf{1}} \\
&\leq \inf_{v \in \phi_J(\mcl{X}_J)+\mbb{R}_{\ge 0}^n} \dotp{\rho}{v}\\
& =\inf_{x \in \mcl{X}_J} \dotp{\rho}{ \phi_J(x)} + \inf_{u \in \mbb{R}_{\ge 0}^n} \dotp{\rho}{u}.
\end{align*}
Then $\inf_{u \in \mbb{R}_{\ge 0}^n} \dotp{\rho}{u}$ is bounded below and consequently, $\rho$ belongs to $\mbb{R}_{\ge 0}^n$ and $\inf_{u \in \mbb{R}_{\ge 0}^n} \dotp{\rho}{u}$ is equal to 0. Since $\rho$ is non-zero, $\sum_{i=1}^n \rho_i$ is strictly positive. We set $\bar{\lambda} =\rho / \sum_{i=1}^n \rho_i \in M_J$ and %in case $\lambda > 0$ 
deduce that
\begin{align*}
w_J + \eps &\leq \inf_{x \in \mcl{X}_J } \dotp{\bar{\lambda}} {\phi_J(x)} \\
&\leq \sup_{\substack{\lambda \in M_J}} \inf_{x \in \mcl{X}_J} \dotp{\lambda}{ \phi_J(x) }= w_J.
\end{align*}
This is impossible and thus \eqref{eq:wj_in_convex_set} is established, which implies that there exist $x_{\eps} \in \mcl{X}_J$ and $u_{\eps} \in \mbb{R}_{\ge 0}^n$ such that $\l(w_J+\eps\r) \mathbf{1}=$ $\phi_J\l(x_{\eps}\r)+u_{\eps}$.
From the definition of $\phi_J$, we deduce that
\begin{align*}
\forall i=1, \ldots, n, \quad f\l(x_{\eps}, y_i\r) \leq w_J+\eps,
\end{align*}
and hence 
\begin{align*}
v_J^{\sharp} \leq \max _{i=1, \ldots, n} f\l(x_{\eps}, y_i\r) \leq w_J+\eps.
\end{align*}
We complete the proof of the lemma by letting $\eps$ tend to 0.   
\end{proof}




\begin{lem}\label{lem:lem8.1:aubin:modified}
Consider a function $f : \mcl{X} \times \mcl{Y} \mapsto \mbb{R} \cup \{ \infty \} $ such that $\mcl{X}$ is convex and for each $y \in \mcl{Y}$, $f(\cdot, y)$ is convex. Then, $\phi_J(\mcl{X}_J) + \mbb{R}_{\ge 0}^n$ is a convex set in $\mbb{R}^n$.    
\end{lem}
\begin{proof} 
Take any convex combination $\alpha_1 \l(\phi_J (x_1) + u_1 \r) + \alpha_2 \l(\phi_J(x_2) + u_2\r)$ where $\alpha_1, \alpha_2 \geq 0$, $\alpha_1 + \alpha_2 = 1$, $x_1$ and $x_2$ are in $\mcl{X}_J$, and $u_1$ and $u_2$ are in $\mbb{R}_{\ge 0}^n$. Let $x = \alpha_1 x_1 + \alpha_2 x_2$. For each $y \in J$, the function $f(\cdot, y)$ is convex, therefore $\phi_J(x) \le \alpha_1 \phi_J(x_1) + \alpha_2 \phi_J(x_2) < \infty$ (latter by definition of $\mathcal{X}_J$). Hence, $x \in \mcl{X}_J$. We can write the convex combination in the form $\phi_J (x) + u$ where $u = \alpha_1 u_1 + \alpha_2 u_2 + \alpha_1 \phi_J(x)+\alpha_2 \phi_J(y)-\phi_J(x)$. Note that $u \in \mbb{R}_{\ge 0}^{n}$ because $\phi_J(x) \le \alpha_1 \phi_J(x_1) + \alpha_2 \phi_J(x_2)$. Consequently, $\alpha_1\l(\phi_J\l(x\r)+u_1\r)+\alpha_2\l(\phi_J\l(y\r)+u_2\r)=\phi_J(x)+u$ belongs to $\phi_J(\mcl{X}_J)+\mbb{R}_{\ge 0}^n$.
\end{proof}



%---------------------------------------------%
%------------- NOTATION SUMMARY ---------------%
%---------------------------------------------%
\section{List of Symbols}\label{sec:appendix:notation}
\begin{itemize}
\setlength\itemsep{2pt}
\item MDP: Markov Decision Process.
\item POMDP: Partially Observable Markov Decision Process.
\item SA-MDP: Single-Agent MDP.
\item SA-POMDP: Single-Agent POMDP.
\item SA-C-MDP: Single-Agent Constrained MDP.
\item SA-C-POMDP: Single-Agent Constrained POMDP.
\item MA-POMDP: Multi-Agent POMDP.
\item MA-C-POMDP: Multi-Agent Constrained POMDP.
\item MARL: Multi-Agent Reinforcement Learning.
\item CTDE: Centralized Training Distributed Execution.
\item ASPS: Approximate Sufficient Private State.
\item ASCS: Approximate Sufficient Common State.
\item RNN: Recurrent Neural Network.
\item FNN: Feed-forward Neural Network.
\item $N$: Number of agents.
\item $\sspace$: State space.
\item $\onspace{0}$: Space of common observations of all agents.
\item $\onspace{n}$: Space of private observations of agent-$n$.
\item $\ospace$: Joint-observation space, given by $\ospace = \prod_{n=0}^{N} \onspace{n}$.
\item $\anspace{n}$: Space of actions of agent $n$.
\item $\aspace$: Joint-action space, given by $\aspace = \prod_{n=1}^{N} \anspace{n}$.
\item $M_1(\cdot)$: Set of all probability measures on topological space $\cdot$ endowed with the topology of weak convergence. 
\item $\mcl{P}_{tr}$: Transition-law. See \eqref{eq:transitionlaw}.
\item $P_{saBo}$: Probability that the next state is in the measurable set $B$ and the next joint-observation is $o$ given action $a$ is taken at current state $s$. See \eqref{eq:psabo}.
\item $c, d$: Immediate-costs: $c$ is the immediate objective cost and $d$ is the immediate constraint cost.
\item $\udl{c}, \ov{c}, \udl{d}, \ov{d}$: Upper and lower bounds on immediate costs. See Assumption \ref{assmp:boundedcosts}.
\item $\alpha$: Discount factor.
\item $P_1$: Initial distribution on the initial state and joint-observation. See \eqref{eq:initialdistribution}.
\item $\uspacen{n}$: Space of policies of agent $n$.
\item $\un{u}{n}$: Used to denote a policy of agent $n$. (in $\uspacen{n}$).
\item $\uspace$: Space of (decentralized) policy-profiles, $\prod_{n=1}^{N} \uspacen{n}$.
\item $u$: Used to denote a policy-profile (in $\uspace$). %See \eqref{eq:uah}.
\item $\uspacemix$: Space of (decentralized) mixtures of policy-profiles in $\uspace$, i.e., $\prod_{n=1}^{N} \m{\uspacen{n}}$.
\item $\mu$: Used to denote a typical element of $\uspacemix$,  given by $\mymathop{\times}_{n=1}^{N} \mun{n}$.
\item $\prup{u}{P_1}, \E{u}{P_1}$: Probability measure and expectation operator corresponding to policy-profile $u\in \uspace$ and initial-distribution $P_1$.
\item $C$: Infinite-horizon expected total discounted objective cost. See \eqref{eq:C}.
\item $D$: Infinite-horizon expected total constraint cost. See \eqref{eq:D}.
\item $\Hstn{t}{0}$: Common history of all agents at time $t$.
\item $\Hstn{t}{n}$: Private history of agent $n$ at time $t$.
\item $\Hst{t}$: Joint history at time $t$, given by $\Hstn{t}{0:N}$.
\item $\hstnspace{t}{n}, \hstspace{t}, \hsspace$: See \eqref{eq:hthnandh}.
\item $\optcosts$: Optimal solution of \eqref{eq:macpomdp}. See \eqref{eq:optccost:infsup}.
\item $\zeta$: Slack of feasible policy-profile $\ov{u}$ in Assumption \ref{assmp:slatercondition}. See \eqref{eq:slatercondition}.
\item $\mcl{Y}$: Space of non-negative Lagrange-multipliers, $\l\{ \lambda \in \mbb{R}^K: \lambda\ge 0\r\}$.
\item $L$: Lagrangian function for \eqref{eq:macpomdp}. See \eqref{eq:lagrangian}.
\item $\wh{L}$: Extended Lagrangian function for \eqref{eq:macpomdp}. See \eqref{eq:lagrangianmix}.
\item $\xuspace$: $\prod_{n=1}^{N}\xtspace{\uspacen{n}}$, also see \eqref{eq:xuspace}.
\item $\pruphsts{u}{t}{\hst{h}{t}, \at{t}}$: 
%Shorthand for 
$\prup{u}{P_1}\l( \Hst{t} = \hst{h}{t}, \At{t} = \at{t} \r)$.
\item $\zuphsts{u}{t}{\hst{h}{t}, \at{t}}$: 
%Shorthand for 
$\E{u}{P_1}\l[ \cCost \mid \Hst{t} = \hst{h}{t}, \At{t} = \at{t} \r] $.
\item $^i u$: $i^{th}$ policy-profile in the sequence $\l\{^i u\r\}_{i=1}^{\infty}$.
\item $\Gt{t} = \Gtn{t}{1:N}$: Coordinator's prescriptions at time $t$.
% \item $\gt{t} = \gtn{t}{1:N}$: Realization of $\Gt{t}$
\item $\gtspace{t}$: Set of all possible prescriptions at time $t$.
\item $ \xtspace{\gtspace{t}}$: See \eqref{eq:xgtspace}.
\item $\wtHstn{t}{0}$: Prescription-observation history of coordinator.
\item $\un{l}{\lambda}$: Immediate-cost in unconstrained version of \eqref{eq:macpomdp} parametrized by Lagrange-multiplier $\lambda\in\mcl{Y}$. See \eqref{eq:l_lamda}.
\item $L_T$: See \eqref{eq:L_T}.
\item $\Qtl{t,T}{\lambda}$: See \eqref{eq:Q_tT_lamda}. 
\item $\Vtl{t,T}{\lambda}$: Optimal cost-to-go from time $t\in[T]$ onward in a finite-horizon $T$. See \eqref{eq:V_tT_lamda}.
\item $\utn{v}{1:T}{\lambda, \star}$: See \eqref{eq:v_tT_lamda_star}.
\item $\Vtl{t}{\lambda}$: See \eqref{eq:V_t_lamda}.
\item $\Qtl{t}{\lambda}$: See \eqref{eq:Q_t_lamda}.
\item $ \un{\udl{l}}{\lambda}$: Lower bound on $\un{l}{\lambda}$. See \eqref{eq:lower_and_upper_lcost}.
\item $\un{\ov{l}}{\lambda} $: Upper bound on $\un{l}{\lambda}$. See \eqref{eq:lower_and_upper_lcost}.
\item $ \Pi_t$: Conditional distribution of $\Stt{t}, \Hstn{t}{1:N}$ given $\wtHstn{t}{0}$, $\pr_{P_1} \l( \Stt{t}, \Hstn{t}{1:N} \mid \wtHstn{t}{0} \r)$.
\item $\Zhtn{t}{1:N}$: ASPS at time $t$. See Definitions \ref{dfn:asps_generator_finite_horizon} and \ref{dfn:asps_generator_infinite_horizon}. 
\item $\Lamdaht{t}$: ASPS-based prescription at time $t$.
\item $\Zhtn{t}{0}$: ASCS at time $t$. See Definitions \ref{dfn:ascs_generator_finite_horizon} and \ref{dfn:ascs_generator_infinite_horizon}.
\item $\varthetahtn{t}{1:N}$: $t$-th component of (finite-horizon) ASPS-generator. See Definition \ref{dfn:asps_generator_finite_horizon}.
\item $\eps_{p,1}, \eps_{p,2}, \delta_{p}$: Attributes of ASPS-generator.
\item $\phihtn{t}{1:N}$: $t$-th component of evolution functions of (finite-horizon) ASPS-generator. See ASPS-1.
\item $\kappa \l(\cdot, \star \r)$: Total variation distance between probability measures $\cdot$ and $\star$. 
% \item $\zhtnspaces{t}{n}$:
\item $\lamdahtnspace{t}{n}$: Set of all possible ASPS-ASCS based prescriptions for agent $n$ at time $t$
\item $\lamdahtspace{t}$. Set of all possible ASPS-ASCS based prescriptions at time $t$, $\prod_{n=1}^{N}\lamdahtnspace{t}{n} $.
\item $\xtspace{\lamdahtspace{t}}$: See \eqref{eq:xlamdahtspace}.
\item $\varthetahtn{t}{0}$: $t$-th component of (finite-horizon) ASCS-generator.
\item $\eps_{c,1}, \eps_{c,2}, \delta_{c}$: Attributes of ASCS-generator.
\item $\phihtn{t}{0}$: $t$-th component of evolution functions of (finite-horizon) ASCS-generator. See ASCS-1.
\item $\whVtl{t,T}{\lambda}$ See \eqref{eq:whV_tT_lamda}.
\item $\whQtl{t,T}{\lambda}$: See \eqref{eq:whQ_tT_lamda}. 
\item $\utn{\wh{v}}{1:T}{\lambda,\star}$: See \eqref{eq:whv_tT_lamda_star}.
\item $M_c\l( \cdot; \alpha, T \r)$: See \eqref{eq:M_c}.
\item $M_p\l( \cdot; \alpha, T\r)$: See \eqref{eq:M_p}.
\item $N\l( \alpha, T\r)$: See \eqref{eq:N_alpha_T}.
% \item $\zhnspace{1:N} $:
\item $\varthetahn{1:N}$: Infinite-horizon ASPS-generator. See Definition \ref{dfn:asps_generator_infinite_horizon}.
% \item $\zhnspace{0}$:
\item $\varthetahn{0}$: Infinite-horizon ASCS-generator. See Definition \ref{dfn:ascs_generator_infinite_horizon}.
\item $\lamdahspace$: Set of all possible ASPS-ASCS based prescriptions when infinite-horizon ASPS and ASCS generators are used.
\item $\xtspace{\lamdahspace}$: See \eqref{eq:xlamdahspace}.
\item $\wh{B}$: See \eqref{eq:whB_whV}.
\item $\whVl{\lambda}, \whQl{\lambda}$: See \eqref{eq:whVl} and \eqref{eq:whQl}.
\item $\rhon{0} $: RNN to serve as (infinite-horizon) ASCS-generator.
\item $\rhon{1:N}$: RNNs to collectively serve as (infinite-horizon) ASPS-generator.
\item $\varphin{0}$: Coordinator's prescription network.
\item $\varphin{1:N}$: Prescription-applier networks of all agents.
\item $\psin{0}$: Coordinator's prediction network.
\item $\psin{S}$: Supervisor's prediction network.
\item $\nabla_{\varphi}L_{\infty}\l( \ut{\wh{v}}{\rho, \varphi} \r)$: Policy-gradient. See \eqref{eq:policy_gradient}.
\item $\delta_{1,i}, \delta_{2,i}$, $\delta_{3,i}$: Sequences of time-steps that satisfy three time-scale stochastic approximation conditions. See \eqref{eq:stepsizes}.
\item $l_2, l_{c,3}, l_{p,3}$: See \eqref{eq:l2}, \eqref{eq:lc3}, and \eqref{eq:lp3}.
\item $\eta$: Used as a placeholder in \eqref{eq:lc3} and \eqref{eq:lp3}.
\item $B$: Batch-size of trajectories.
\item $\tau_j$: $j^{th}$ trajectory. See \eqref{eq:tauj}.
\item $l_{\rhon{0}, \psin{0}}$: See \eqref{eq:l_rho0_psi0}.
\item $l_{\rho, \psin{S}}$: See \eqref{eq:l_rho_psiS}.
\item $R_{\rhon{0}, \psin{0}}$: See \eqref{eq:R_rho0_psi0}.
\item $R_{\rho, \psin{S}}$: See \eqref{eq:R_rho_psiS}.
\item $g_{j,t}$: Cost-to-go at time $t$ in $j^{th}$ trajectory. See \eqref{eq:gjt}.
\item $\wh{\nabla_{\varphi}L_{\infty} }\l( \ut{\wh{v}}{\rho, \varphi} \r)$: REINFORCE-estimate of policy-gradient $\nabla_{\varphi} L_{\infty} \l( \ut{\wh{v}}{\rho,\varphi} \r)$. See \eqref{eq:pg_estimate_reinforce}.

\end{itemize}