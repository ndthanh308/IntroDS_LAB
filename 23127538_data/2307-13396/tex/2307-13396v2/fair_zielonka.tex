
\section{Zielonka's Algorithm for \Odd-Fair Parity Games}\label{sec:zielonka}
In this section, we construct a Zielonka-like algorithm that solves \Odd-fair parity games. We call this algorithm \emph{\Odd-fair Zielonka's algorithm}. We first recall Zielonka's original algorithm in Sec.~\ref{sec:zielonka:orig} and outline the changes imposed for our new \Odd-fair version in Sec.~\ref{sec:zielonka:fair}. We then discuss the correctness of this new algorithm in Sec.~\ref{sec:zielonka:correct}.

% and preview it for player \Even and \Odd, respectively, in Alg.~\ref{algo:fair-zielonka-odd} and Alg.~\ref{algo:fair-zielonka-even}.
% The differences between Zielonka's algorithm and \Odd-fair Zielonka's algorithm very minor and will be illustrated in Sec.~\ref{}.

% are small in sight. Namely, if one slightly changes the definition of $\SafeReach^f_\Even$ in Alg.~\ref{alg:fair-zielonka} and 
% replace the $\SafeReach^f_\Odd(V, G, \mathcal{G}^\ell)$ with $G$ on the last line of procedure $\SOLVE_\Odd(l, \mathcal{G}^\ell)$, one gets Zielonka's original algorithm.
% 
% The changes in the algorithm are actually small enough not to alter its worst-case computational complexity at all. However, the proof of one recursive call  of $\SOLVE_\Odd$ becomes substantially more complex with the involvement of live edges. 
% Due to space concerns, we only try to convey the main idea of the proof here and give the full proof in the appendix (App.~\ref{app:zielonka-proof}).


%  \vspace{0.5cm}
From now on we take $\mathcal{G}^\ell = \ltup{(V, V_\Even, V_\Odd, E, \chi), E^\ell}$ to be an \Odd-fair parity game. 

\subsection{Zielonka's Original Algorithm}\label{sec:zielonka:orig}
Intuitively, Zielonka's algorithm consists of two nested recursive functions, 
$\SOLVE_{\Even}(n,\mathcal{G})$ and $\SOLVE_{\Odd}(n,\mathcal{G})$ which compute \We and \Wo in a given parity game $\mathcal{G}$ with, respectively, even or odd upper bound priority $n$. Both functions recursively call each other on a sequence of sub-games that is constructed during the run of the algorithm. 

The main difference between Zielonka's original algorithm \cite{Zielonka98} and our new \Odd-fair version in Alg.~\ref{algo:fair-zielonka-bb} %(note that they are symmetrical except for the different \textbf{return} conditions) 
is the computation of the safe reachability set, denoted by $\SafeReach^f_\bb$ within the algorithms. 
%\footnote{Note that Alg.~\ref{algo:fair-zielonka-odd} and Alg.~\ref{algo:fair-zielonka-even} are symmetrical except for the different \textbf{return} conditions. In Alg.~\ref{algo:fair-zielonka-even} the last line can be changed to `\textbf{return} $\SafeReach_\Even^f(V, X, \mathcal{G}^\ell)$' to make the algorithms symmetrical, since it is equal to $X$ in this algorithm. However, we did not want to make this visual change to emphasize that the algorithm for \Even is slightly simpler than the one for \Odd, and to make it more visible that it mimics the original Zielonka's algorithm except for the definition of safe reachability sets.}. 
Intuitively, the safe reachability set of player \bb %in a (normal) game $G$ with safety set $S\subseteq V$ and reach set $R\subseteq V$, 
is the set of vertices from which \bb has a strategy to force the game into the reach set $R\subseteq V$, while staying in the safety set $S\subseteq V$. 
% 
In a (normal) parity game $\mathcal{G}$ (without live edges), this set %the safe reachability set $\Xsr_\bb$ for player \bb 
can be computed via the single-nested fixed-point formula
\begin{equation}\label{equ:Xsr1}
 \Xsr_\bb:=\mu X~.~(S \cap (R \cup \Cpre_\bb(X))).
\end{equation}
If one interpretes Alg.~\ref{algo:fair-zielonka-bb} over (normal) parity games $\mathcal{G}$, defines $\SafeReach^f_\bb$ via \eqref{equ:Xsr1} for the respective player, and replaces $\SafeReach^f_\Odd(\cdot,X,\cdot)$ in the last return statement with $X$ (so, the algorithm returns $X$ for any $\Lambda$), one gets exactly Zielonka's algorithm for parity games. 

%\begin{algorithm}
%    \centering
%    \caption{\Odd-Fair-Zielonka's Algorithm for player \Odd}\label{algo:fair-zielonka-odd}
%    \begin{algorithmic}\fontsmall
%  \Procedure{$\SOLVE_{\Odd}$}{$n$, $\mathcal{G}^\ell$}
%    \State $X \gets V$
%    \State $Z_\Even \gets G$ \label{lineo:Z_Odd_initialize}
%    \While{$Z_\Even \neq \emptyset$}\label{lineo:while_start}
%    \State $N \gets \{ v \mid v \in X \text{ with } \chi(v) = n\}$ \label{lineo:N}
%    \State $Z \gets X \setminus \SafeReach^f_\Odd(X, N, \mathcal{G}^\ell)$ \label{lineo:Z}
%    \State $Z_\Even \gets \SOLVE_{\Even}(n-1,\mathcal{G}^\ell[Z])$ \label{lineo:Z_Odd}
%    \State $X \gets X \setminus \SafeReach^f_\Even(X, Z_\Even, \mathcal{G}^\ell)$\label{lineo:G_update}
%    \EndWhile \label{lineo:while_end}
%    \State \Return $\SafeReach_\Odd^f(V, X, \mathcal{G}^\ell)$
%    \EndProcedure
%    \end{algorithmic}
%\end{algorithm}
%\begin{algorithm}
%    \centering
%    \caption{\Odd-Fair-Zielonka's Algorithm for player \Even}
%    \label{algo:fair-zielonka-even}
%    \begin{algorithmic} \fontsmall
%    \Procedure{$\SOLVE_{\Even}$}{$n$, $\mathcal{G}^\ell$}
%    \State $X \gets V$
%    \If{$n = 0$} \Return $\emptyset$\EndIf
%    \State $Z_\Odd \gets X$ \label{line:Z_Odd_initialize}
%    \While{$Z_\Odd \neq \emptyset$}\label{line:while_start}
%    \State $N \gets \{ v \mid v \in X \text{ with } \chi(v) = n\}$ \label{line:N}
%    \State $Z \gets X \setminus \SafeReach^f_\Even(X, N, \mathcal{G}^\ell)$ \label{line:Z}
%    \State $Z_\Odd \gets \SOLVE_{\Odd}(n-1,\mathcal{G}^\ell[Z])$ \label{line:Z_Odd}
%    \State $X \gets X \setminus \SafeReach^f_\Odd(X, Z_\Odd, \mathcal{G}^\ell)$\label{line:G_update}
%    \EndWhile \label{line:while_end}
%    \State \Return $X$
%    \EndProcedure
%    \end{algorithmic}
%\end{algorithm}

%ANNE'S Version
%    \begin{algorithmic}\fontsmall
%  \Procedure{$\SOLVE_{\bb}$}{$n$, $\mathcal{G}^\ell$}
%    \State $X \gets V$
%    \State $Z_{\neg \bb} \gets G$ \label{lineo:Z_bb_initialize}
%   \While{$Z_{\neg \bb} \neq \emptyset$}\label{lineo:while_start}
%    \State $N \gets \{ v \mid v \in X \text{ with } \chi(v) = n\}$ \label{lineo:N}
%    \State $Z \gets X \setminus \SafeReach^f_\bb(X, N, \mathcal{G}^\ell)$ \label{lineo:Z}
%    \State $Z_{\neg \bb} \gets \SOLVE_{\neg \bb}(n-1,\mathcal{G}^\ell[Z])$ \label{lineo:Z_bb}
%    \State $X \gets X \setminus \SafeReach^f_{\neg \bb}(X, Z_{\neg \bb}, \mathcal{G}^\ell)$\label{lineo:G_update}
%    \EndWhile \label{lineo:while_end}
%    \State \Return $\SafeReach_\bb^f(V, X, \mathcal{G}^\ell)$
%    \EndProcedure
%    \end{algorithmic}
 %\end{minipage}
%\begin{minipage}{0.4\textwidth}
%    % Figure removed
%\end{minipage}

\vspace*{-0.5cm}

% Figure environment removed

 %  \vspace*{-0.5cm}

\subsection{The \Odd-fair Zielonka's Algorithm}\label{sec:zielonka:fair}
We are now considering an \Odd-fair parity game $\mathcal{G}^\ell$. % with live edges on \Odd player vertices.
As discussed before, the main difference of the \Odd-fair Zielonka's algorithm from the original one lies in the construction of the safe reachability sets denoted by $\SafeReach^f_\bb$ in Alg.~\ref{algo:fair-zielonka-bb}. We therefore start by discussing its computation for both players.

\smallskip
\noindent\textbf{The \Odd Player.}
The first, somehow surprising, observation is that for player \Odd in \Odd-fair parity game $\mathcal{G}^\ell$, the safe reachability set $\Xsr_\Odd$ can still be computed via \eqref{equ:Xsr1}. This is due to the fact that $R$ only needs to be visited once, and 
\Even vertices do not have live outgoing edges that might prevent player \Odd from forcing a visit to $R$. 

In addition, we can extract a \emph{partial strategy template} for player \Odd from the iterative computation of \eqref{equ:Xsr1} via a similar, but much simpler ranking argument as used in Sec.~\ref{sec:strat-templates}. Here, $\rank{v} = 1$ for $v \in R$ and for the remaining vertices, 
$\rank{v}$ is the minimum integer $j$ for which $v \in X^j:=(S \cap (R \cup \Cpre_\Odd(X^{j-1})))$ where $X^0=\emptyset$. The positional strategy of \bb is then to take the minimum ranked successor from each \Odd node. 

Another way to think about this strategy is in the form of an acyclic subgraph of $\mathcal{G}^\ell$ on $\Xsr_\Odd$, where nodes in $R$ have no outgoing edges,
and for the remaining nodes, \Odd nodes have one outgoing edge and \Even nodes have all their outgoing edges. This is because if $v \in X^j\cap \Ve$, all outgoing edges achieve positive progress towards $R$, i.e. for all $(v, w) \in E$, $w \in X^{j-1}$.
Now it is easy to see that this subgraph almost defines a strategy template, i.e., on $\Xsr_\Odd\setminus R$, \Even nodes have all their outgoing edges in the subgraph, no \Odd node lies on a cycle and all of them have one outgoing edge. However, vertices in $R$ are dead-ends. We therefore call the strategy template induced by  \eqref{equ:Xsr1} \emph{partial} and denote it by $sr$. %\AKS{I think we need to properly formalize this template to use it in the next section}


\smallskip
\noindent\textbf{The \Even Player.}
It follows from the results of Banerjee et. al.~\cite{banerjee2022fast} that the safe reachability set $\Xsr_\Even$ of player \Even in \Odd-fair parity games requires the 2-nested fixed-point formula $\nu Y.\mu X.S \cap (R \cup \Apre(Y,X))$, which (via the operators defined in Sec.~\ref{sec:assump:prelim}) equals
%  
\begin{equation}\label{equ:Xsr2}
 \Xsr_\Even: =~\nu Y~.~\mu X~.~S \cap (R \cup (\Cpre_\Even(X) \cup (\Lpre^{\exists}(X) \cap \Pre_\Odd^{\forall}(Y))))
\end{equation}%=&~\nu Y~.~\mu X~.~S \cap (R \cup \Apre(Y,X))\\
% \vspace{-0.1cm}
%with all predecessor operators defined in Sec.~\ref{sec:assump:prelim}.
% We denote this formula by $\SafeReach^l(S, R, \mathcal{G}^\ell)$. 
Intuitively, the necessity of a 2-nested formula arises from the following lack of information: we do not know in advance, which \Odd nodes need to lie on a cycle on a strategy template required for \Odd to win. If any positional strategy that lets \Odd win (i.e., to avoid $R$ or leave $S$) from a $v\in V^\ell$
requires $v$ to lie on a cycle, then \Odd has to take $v$'s live outgoing edges as well, and thus, it can enter $\Xsr_\Even$ and lose.
The calculation of \eqref{equ:Xsr2} starts with $Y^0 := V$, resulting in $\Pre_\Odd^{\forall}(V)=V$, hence
% 
% \vspace{-0.5cm}
\begin{equation}\label{equ:Xsr2a}
 Y^{1}:=\mu X~.~S\cap (R \cup \Cpre_\Even(X) \cup \Lpre^\exists(X)).
\end{equation}
% 
% \vspace{-0.1cm}
Due to the disappearence of $\Pre^\forall_\Odd(Y)$ in this iteration, intuitively all $v \in V^\ell$ are treated as if they do not have any positional winning \Odd strategy on them, so as if all \Odd strategies have to take all the live edges in the game. 
%This is due to the triviality of $\Pre^\forall_\Odd(Y) = V$ in this iteration. 
%When  $\Pre^\forall_\Odd(Y)$ vanishes from the equation, 
$Y^1$ includes any \Odd vertex that progresses towards $R$ while staying in $S$ with using either all its edges (due to $\Cpre_\Even(X)$) or through one live edge (due to $\Lpre^\exists(X)$). Thus, any vertex that manages to stay in $V \setminus Y^1$ does so due to being won by \Odd even if \Even could force all the live outgoing edges to be taken. 
Note that due to the monotonicity of fixed-point operators, for all $j$, $V \setminus Y^1 \subseteq V \setminus Y^j$.

Throughout the calculation, $ V \setminus Y^j$ keeps track of the nodes that have managed to escape $S$ or avoid $R$ in the previous iteration, so are `already' won by \Odd in the first $j$ iterations. The inner fixed-point calculation in the $(j+1)^{th}$ iteration treats $V \setminus Y^j$ as a subset of \Odd's winning region and it deems any node that can be forced by \Odd to reach $V \setminus Y^j$, lost by \Even.
When the algorithm saturates, $Y^\infty$ contains only those \Odd nodes that cannot be forced by \Odd to reach $V \setminus Y^\infty$, i.e., are won by \Even. Here it is important to observe that, $V \setminus Y^\infty$ contains some \Odd nodes that are not $V \setminus Y^1$. Since they are in $Y^1$, these nodes inductively %\todo{IS:iteratively? any other word?} 
reach \Even winning vertices through live edges. %(or, reach some vertices that have this property\IS{how to put this?}). 
This reveals that, all nodes in $V \setminus Y^j$ but not in $V \setminus Y^1$ win due to a positional \Odd strategy that reaches $V \setminus Y^{j-1}$. 
Iteratively, this reveals that all such nodes have positional \Odd strategies that make them reach $V \setminus Y^1$.%, and this is the key observation we will explot in the next section. 

%The key observation from the previous discussion is the following: if $v$ is won by \Odd even under the assumption that all the live edges in the game have to be taken by \Odd (which is an assumption that increases \Even's power), then $v$, or any node that can be forced by \Odd to reach $v$, is won by \Odd. Also, note that these nodes are exactly the nodes in $V \setminus Y^1$ as explained above.
The above alternative interpretation of the computation of $\Xsr_\Even$ in \eqref{equ:Xsr2} is the key insight that we utilize to define our new \Odd-fair Zielonka's algorithm, as discussed next.
% This is the key observation we will utilize within our new \Odd-fair Zielonka's algorithm as discussed next. %, and show that it carries over from \Odd-fair safe-reachability games to \Odd-fair parity games.
% \todo{IS: maybe we can get rid of the 'and show that it carries over...' part.}
%We utilize this key observation within our new \Odd-fair Zielonka's algorithm as discussed next, and thereby show that it carries over from \Odd-fair safe-reachability games to \Odd-fair parity games.

%As Banerjee et. al.~\cite{banerjee2022fast} have shown that this intuition carries over from safe-reachability games to different $\omega$-regular games under strong transition fairness, including Rabin and Parity games, we can utilize this observation for our \Odd-fair Zielonka's algorithm as follows.

\smallskip
\noindent\textbf{The \Odd-fair Zielonka's Algorithm.} %\todo{IS:I made some changes starting from here - until the end of this subsection}
% It remains to formally define the safe-reachability procedures used in Alg.~\ref{algo:fair-zielonka-odd} and Alg.~\ref{algo:fair-zielonka-even}. 
% 
Following up on the previous discussion, we use the following insight within the construction of the \Odd-fair Zielonka's algorithm. We assume the existence of a core subset $\Wo' \subseteq \Wo$ %\todo{this was in the text, I am not sure if it should be..: (actually corresponding to $ V \setminus Y^j$ in the above argumentation for \Odd-fair safe-reachability games)} 
that player \Odd can force all nodes in $\Wo$ %(i.e., the winning region of \Odd in the \Odd-fair parity game $\mathcal{G}^\ell$) 
to, that is winning for \Odd even under the assumption that \Even can force all the live edges in the game to be taken. %\footnote{We note that this is a similar insight used in the proof of winning strategy templates discussed in Sec.~\ref{sec:strat-templates}.}.  
% 
% We exploit this observation as follows: 
% \emph{Any $v \in \Wo$ can be made by \Odd to reach a core subset $\Wo' \subseteq \Wo$, that is won by \Odd even if all $v \in V^\ell \cap \Wo'$ have to take all their live edges.} 
% In $\SafeReach^l(S, R, \mathcal{G}^\ell)$, the first iteration of the $Y$ variable calculates this $\Wo'$. 
Since Zielonka's algorithm solves parity games by a sequence of nested safe-reachability calculations for alternating players, we apply the following trick:
Instead of computing $\Xsr_\Even$ via \eqref{equ:Xsr2} in each recursive call of Alg.~\ref{algo:fair-zielonka-bb}, we only compute $Y^1$ via \eqref{equ:Xsr2a} and use it as an \emph{overapproximation} of $\Xsr_\Even$ (which is indeed the case due to the monotonicity of \eqref{equ:Xsr2} in $Y$). 
That is, while we take the \Odd safe reachability set $\SafeReach^f_\Odd$ as the original (linear) \Odd safe reachability computation known for these games (given in~\eqref{equ:Xsr1}), we do not take \Even safe reachability formula $\SafeReach^f_\Even$ to be the (quadratic) \Even safe reachability computation known for these games (given in~\eqref{equ:Xsr2}),
but we instead take it as its (linear) subformula given in~\eqref{equ:Xsr2a} and arrive at an overapproximation of the \Even safe reachability region at the end of each $\SafeReach^f_\Even$ calculation. We finalize the recursive call $\SOLVE_\Odd$ by an extra call of $\SafeReach^f_\Odd$ applied to the (thus) underapproximated \Odd winning region in the sub-game, therefore expanding the returned \Odd winning region of the sub-game.  

By this, it turns out that the recursive call of $\SOLVE_\Odd(n, \mathcal{G}^\ell)$ actually computes $\Wo'$ as the set $X$ and we ensure that $\Wo$ is returned by the additional (linear) computation of $\SafeReach^f_\Odd$ over $X$ in the last return statement of Alg.~\ref{algo:fair-zielonka-bb}.
% 
This instantiation of the safe-reachability computations is formalized next.

\begin{definition}\label{def:safereach}
 Given an \Odd-fair parity game $\mathcal{G}^\ell=\ltup{(V, V_\Even, V_\Odd, E, \chi), E^\ell}$ the safe-reachability procedures $\SafeReach^f_\Odd(S, R, \mathcal{G}^\ell)$ and $\SafeReach^f_\Even(S, R, \mathcal{G}^\ell)$ in Alg.~\ref{algo:fair-zielonka-bb} denote the iterative fixed-point computations in \eqref{equ:Xsr1} for \Odd and \eqref{equ:Xsr2a} for \Even.
\end{definition}
% 
% \noindent\textbf{The reason behind the computational advantage of Alg~\ref{algo:fair-zielonka-bb}.}\todo{One of the reviewers suggested that we highlight this part, since we have this part as the 'main achievement' of this technique. I think maybe we can add a small title like this one, or maybe some other title like just "computational advantage", or "reason of efficiency" or "computational succinctness" or something like this. }

\subsection{Complexity of the \Odd-fair Zielonka's Algorithm}\label{sec:zielonka:complexity}
The safe-reachability computations defined in Def.~\ref{def:safereach} have the same complexity as their computations via \eqref{equ:Xsr1} in Zielonka's original algorithm. The only difference is in the number of calculated $\Pre$ operations: while $\SafeReach_\Even$ from Zielonka's original algorithm~\eqref{equ:Xsr1} require the calculation of only one $\Pre$ operator, $\SafeReach_\Even^f$ from~\eqref{equ:Xsr2a} requires the calculation of 2 $\Pre$ operators. The additional final call of $\SafeReach^f_\Odd$ in $\SOLVE_\Odd$ procedure also has linear complexity and requires one $\Pre$ calculation. 
Therefore, not only the worst-case time complexity of Alg.~\ref{algo:fair-zielonka-bb} is equivalent to that of Zielonka's original algorithm (which would be the case even if we used the quadratic safe reachability formula from~\eqref{equ:Xsr2} for \Even since the overall complexity of the algorithm is exponential) but we create almost no additional computational overhead in the algorithm by introducing the fairness assumptions.

We further remark that Alg.~\ref{algo:fair-zielonka-bb} is not a straight-forward interpretation of the nested fixed-point in~\eqref{eq:fp-odd}, and its negation (see (14) in App. A.1 of~\cite{extended}) in the form of Zielonka's algorithm. 
\begin{comment}
%Such a straight forward approach would increase the number of $\Pre$ calls in each recursive step polynomially in the number of priorities, whereas in Alg.~\ref{algo:fair-zielonka-bb} we have at most 2 times (plus 1 extra for $\SOLVE_\Odd$) as many $\Pre$ calls compared to Zielonka's original algorithm.
Such a conversion would require the nested fixed-point formula to be turned into a fixed-point parity game \cite{}, which would basically be equivalent to the gadget-enhanced parity game from \cite{} and the state space of the resulting parity game would be polynomially increased with respect to the state space of the \Odd-fair parity game. More precisely, an efficient fixed-point parity game would have $\frac{3}{2}\cdot l \cdot|V|$-many states where $l$ is the number of priorities and $|V|$ is the number of states of the original \Odd-fair parity game, while keeping the same number of priorities. This approach would not only require a precomputation, but would also require a polynomial increase in the total number of $\Pre$ calculations performed by the resulting Zielonka's algorithm. However, in Alg.~\ref{algo:fair-zielonka-bb} we require at most 2 times (plus 1 extra for $\SOLVE_\Odd$) as many $\Pre$ calls compared to Zielonka's original algorithm at each recursive call.
\end{comment}
%\begin{comment}
%Such a conversion would be non-trivial due to $\Apre$ and $\Npre$ taking 2 different variables from 2 different iterations of the fixed-point calculation. 
Firstly, such a straightforward approach is non-trivial due to $\Apre$ and $\Npre$ operators taking two variables from two different iterations of the fixed-point calculation. 
Furthermore, 
at each \Even safe-reachability call of Alg.~\ref{algo:fair-zielonka-bb}, as mentioned we compute 2 $\Pre$ operators (equation~\ref{equ:Xsr2a}), whereas in each such corresponding step in the fixed-point iteration, we would have to compute 3 $\Pre$ operators due to the expansion of $\Apre$~\eqref{equ:apre} and $\Npre$~\eqref{equ:npre}.
%\end{comment}

%This is slightly surprising given that even the algorithm by Banerjee et. al. in \cite{banerjee2022fast}  %introduces a small overhead from $|V|^n$ (the complexity of \enquote{normal} parity fixed-point) to $|V|^{n+1}$ (the complexity of \Odd-fair parity fixed-point) where $n$ is the number of colors.
%\AKS{we should give the actual complexity here.}
% 
% As in the original Zielonka's algorithm, the Alg.~\ref{algo:fair-zielonka-odd} and Alg.~\ref{algo:fair-zielonka-even}
% 
% We first recall that the determinacy of \Odd-fair parity games follows from the results of Banerjee et. al.~\cite{banerjee2022fast}. 
% 
% We use two recursive calls to calculate \We and \Wo in an \Odd-fair parity game.
% $\SOLVE_\Odd(n,\mathcal{G}^\ell)$ takes an \Odd-fair parity game $\mathcal{G}^\ell$ with an \emph{odd} upper bound $n$ on the priorities of $V$ and returns $\Wo$ of $\mathcal{G}^\ell$.
% $\SOLVE_\Even(n,\mathcal{G}^\ell)$ returns \We for a $\mathcal{G}^\ell$ and an \emph{even} priority upper bound $n$ on $V$. Two calls recursively call each other, from a subgame $\mathcal{G}^\ell[X]$ that has priorities at most $ n - 1$.
% Since in the base cases $n = 0$ and $G = \emptyset$, the calls correctly return $\emptyset$, in order to prove the correctness of the algorithm, it is enough to 
% prove the correctness of each procedure, assuming the correctness of the other. 

It remains to show that \Odd-fair Zielonka's algorithm solves \Odd-fair parity games. 


%This captures the assumption that 'all the live edges in the game have to be taken by \Odd'. The correctness of the new Zielonka's algorithm then relies on the fact that 

% $\SafeReach_\Even(S, R, \mathcal{G})$ in Zielonka's original algorithm, we use a formula $\SafeReach^f_\Even(S, R, \mathcal{G}^\ell) = \mu X. p_s \wedge (p_r \vee \Cpre_\Even(X) \vee \Lpre^\exists(X))$ that corresponds to the first iteration of $\SafeReach^l_\Even(S, R, \mathcal{G}^\ell)$.
% This calculation does not give us the \Even safe reachability set, but an over-approximation of it. We leave $\SafeReach_\Odd$ calculations the same by assigning $\SafeReach^f_\Odd(S, R, \mathcal{G}^\ell) = \SafeReach_\Odd(S,R,\mathcal{G})$.





% By using $\SafeReach^f_\Even(S, R, \mathcal{G})$, at each recursive call of $\SOLVE_\Odd(n, \mathcal{G}^\ell)$ we make the algorithm calculate \Wo' as $G$. Then by taking $\SafeReach^f_\Odd(V, \Wo', \mathcal{G}^\ell)$ we reach \Wo.

%In other words, we can use only the calculation from the first iteration of $\SafeReach^l(S, R, \mathcal{G}^\ell)$, i.e. $\mu X. p_s \wedge (p_r \vee \Cpre_\Even(X) \vee \Lpre^\exists(X))$ as the safe reachability set for \Even.
%This calculation will not give us a precise set, but an over approximation of the \Even winning region everytime we use it. 
%However, 


%safe reachability set computation for player \Even, called $ \SafeReach_\Even(S, R, \mathcal{G}^\ell)$. The computation of the safe reachability set for player $\Odd$ is the same as the computation of it in "normal" parity games. However, 
%the safe reachability set computation for \Even player is different. For this computation, we use the 2-nested fixed-point formula from \cite{banerjee2022fast}.
%\begin{align}
%    & \SafeReach^f_\Even(S, R, \mathcal{G}^\ell) := \mu X. (p_s \wedge (p_r \vee \Cpre_\Odd(X) \vee \Lpre^\exists(X))) \label{eq:safereacheven}\\
%    & \SafeReach^f_\Odd(S, R, \mathcal{G}^\ell) := \mu X. (p_s \wedge (p_r \vee \Cpre_\Odd(X))) \notag
%\end{align}
%\vspace{-0.2cm}

%\vspace*{-0.3cm}

\subsection{Correctness of the \Odd-fair Zielonka's Algorithm}\label{sec:zielonka:correct}
% 
% 
% Now we will try to convey the idea of the correctness proof of Alg.~\ref{alg:fair-zielonka}. 
We first recall that \Odd-fair parity games are determined. %the results of Banerjee et. al.~\cite{banerjee2022fast}. Given an \Odd-fair parity game $\mathcal{G}^\ell=\ltup{(V, V_\Even, V_\Odd, E, \chi), E^\ell}$, we therefore know that \We and \Wo partition $V$. Following the original Zieloka's algorithms proof, it therefore remains to show that $\SOLVE_{\Even}(n,\mathcal{G}^\ell)$ and $\SOLVE_{\Odd}(n,\mathcal{G}^\ell)$ as in Alg.~\ref{algo:fair-zielonka-odd} and Alg.~\ref{algo:fair-zielonka-even} actually compute \We and \Wo, respectively.
Next, we prove the correctness of the algorithm by induction on $n$. Since in the base case $n = 0$ the calls correctly return $\emptyset$, it suffices to prove the correctness of each function, assuming the correctness of the other. This is formalized next. %for $\SOLVE_{EVEN}$ (Alg.~\ref{algo:fair-zielonka-even}), where $\SOLVE_{ODD}$ (Alg.~\ref{algo:fair-zielonka-odd}) follows from a symmetrical argument with odd $n$.
\begin{comment}
\begin{theorem}[Correctness of $\SOLVE_{\bb}$, Alg.~\ref{algo:fair-zielonka-bb}]\label{thm:solvebb}
Let $\mathcal{G}^\ell=\ltup{(V, V_\Even, V_\Odd, E, \chi), E^\ell}$ be an \Odd-fair parity game with \textsf{parity(\bb)}\footnote{\textsf{parity(\Odd)} is odd and  \textsf{parity(\Even)} is even.} upper bound priority $n$. Further, assume that for any \Odd-fair parity game $\mathcal{G'}^\ell$ with  \textsf{parity($\bb$)} upper bound priority $n'<n$ holds that 
 $\mathcal{W}_\bb[\mathcal{G'}^\ell]=\SOLVE_{\bb}(n',\mathcal{G'}^\ell)$ and for any  \Odd-fair parity game $\mathcal{G''}^\ell$ with \textsf{parity($\neg \bb$)} upper bound priority $n''<n$ holds that 
 $\mathcal{W}_{\neg \bb}[\mathcal{G''}^\ell]=\SOLVE_{\neg \bb}(n'',\mathcal{G''}^\ell)$. Then, $\mathcal{W}_\bb[\mathcal{G}^\ell]=\SOLVE_{\bb}(n,\mathcal{G}^\ell)$.
\end{theorem}
\end{comment}

\begin{theorem}[Correctness of $\SOLVE_{\bb}$, Alg.~\ref{algo:fair-zielonka-bb}]\label{thm:solvebb}
Assume that for any \Odd-fair parity game $\mathcal{G}'^\ell$ where $n' < n$ is an odd (resp. even) upper bound on the priorities of the game, $SOLVE_\Odd(n', \mathcal{G}')^\ell$ correctly returns the \Odd winning region (resp. $\SOLVE_\Even(n', \mathcal{G}')^\ell$ correctly returns the \Even winning region) in $\mathcal{G}'^\ell$. Then $SOLVE_\bb(n, \mathcal{G}^\ell)$ correctly returns the winning region of player $\bb$ where $n$ is even if $\bb= \Even$ and odd if $\bb = \Odd$.
\end{theorem}

%While the proof of Alg.~\ref{algo:fair-zielonka-odd} follows essentially the proof by Ralf K{\"u}sters \cite{Kuesters2002} of the original Zielonka's algorithm \cite{zielonkas-alg}, the proof of Alg.~\ref{algo:fair-zielonka-even} formalized by Thm.~\ref{thm:solveodd} becomes substantially more complex. First, our instantiation of $\SafeReach^f_\Even(S, R, \mathcal{G}^\ell)$ via \eqref{equ:Xsr2} only computes an \emph{overapproximation} of the safe reachability set $\Xsr_\Even$, and second, we must use \Odd \emph{winning strategy templates} instead of positional winning strategies, to prove a vertex to be winning. While the complete correctness proofs of both algorithms can be found in App.~\ref{app:zielonka-proof}, we give the intuition of Thm.~\ref{thm:solveodd} here, as this is the main contribution of this section. % In order to do so, we first define some preliminaries in Sec.~\ref{sec:zielonka:correctness:prelim}.
% We follow the notation of Ralf K{\"u}sters proof \cite{Kuesters2002} of the original Zielonka's algorithm \cite{Zielonka98}.% Let us first set up some preliminaries.

% \vspace{0.2cm}

\noindent\textbf{Notation.}
We follow the notation of K{\"u}sters' proof \cite{Kuesters2002} of Zielonka's original algorithm \cite{Zielonka98}.
%For the remainer of this section, take $\mathcal{G}^\ell = \ltup{(V, V_\Even, V_\Odd, E, \chi), E^\ell}$. 
Recall that $\mathcal{G}^\ell$ has no dead-ends. For some $X \subseteq V$, we call $\mathcal{G}^\ell[X] = \ltup{(X, X \cap V_\Even, X \cap V_\Odd, X \times X \subseteq E, \chi \mid_X), X \times X \subseteq E^\ell }$ 
a \emph{subgame} of $\mathcal{G}^\ell$ if it has no dead-ends. Here, $\chi\mid_X$ is the priority function $\chi : V \to \mathbb{N}$ restricted to domain $X$. Let $n$ be an upper bound on the priorities in $V$. If the parity of $n$ is even, set $\bb$ to $\Even$; if it's odd, set $\bb$ to \Odd. 

\vspace{0.2cm}

\noindent\textbf{\bb-trap and \bb-paradise.}
A $\bb$-trap is a subset $T \subseteq V$ for $\bb \in \{\Even, \Odd\}$ such that,
$\forall v \in T \cap V_{\nb},\,\, \exists (v, w)\in E \,\,\text{ with } w \in T$ and $\forall v \in T \cap V_{\bb},\,\, (v, w) \in E \implies w \in T$. 
A $\bb$-paradise in $\mathcal{G}^\ell$ is a subset $T \subseteq V$ which is a $\nb$-trap in $V$ and there exists a winning $\bb$ strategy template $(T, \e)$ in $\mathcal{G}^\ell$. %\AKS{don't we need that the vertices contained in this template are precisely $T$.}

\vspace{0.2cm}

The recursive calls of $\SOLVE_\bb$ and $\SOLVE_{\neg \bb}$ on subgames within Alg.~\ref{algo:fair-zielonka-bb} induce a characteristic partition of the game graph. For the correctness proof, 
we need to remember a series of these subgames that are constructed through previous recursive calls. The partition of these subsets is illustrated in Fig.~\ref{fig:kuesters-figure-extended} and formalized as follows.
% 
% % Figure environment removed

\vspace{-0.4cm}

\begin{align}\label{equ:seriesZielonka}
    &X_\bb^i := V \setminus X_\nb^i \quad \quad \quad &&N^i:= \{v \in X^i_\bb \mid \chi(v) = n\}\\
    &Z^i:= X^i_\bb \setminus \SafeReach^f_\bb(X^i_\bb, N^i, \mathcal{G}^\ell) \quad &&X^{i+1}_\nb :=  \SafeReach^f_\nb(V, X_\nb^{i} \cup Z_\nb^{i}, \mathcal{G}^\ell)\nonumber % X_\Even^{i} \cup \SafeReach_\Even^f(X^{i}_\Odd, Z_\Even^{i}, \mathcal{G}^\ell) )%\text{\todo{IS: I know the equality is not completely justified. The first one is cheaper for an algorithm pov, whereas the second one is easier to justify that $X^i_\Odd$ is an \Even-trap.}} 
\end{align}

%\vspace{-0.3cm}

where, in addition $Z_\bb^i$ is the \bb winning region in the subgame $\mathcal{G}^\ell[Z^i]$. Intuitively, the sets constructed in \eqref{equ:seriesZielonka} correspond to the sets with the same name within Alg.~\ref{algo:fair-zielonka-bb}.
    
We collect the following observations on these sets, which are proven in App.~\ref{app:zielonka-proof}. %and mimic the corresponding properties in the proof of the original Zielonka's proof \cite{Kuesters2002}.
\begin{enumerate}\label{it:zlk-observations}
\item  \textbf{(App. - Obs.~\ref{app-obs:traps-subgames})} $X^i_\nb$ is an \bb-trap, $X^i_\bb$, $Z^i$ and $Z_\bb^i$ are \nb-traps in $V$. $Z^i$ is in \nb-trap in $X_\bb$ and $Z_\nb^i, Z_\bb^i$ are \bb- and \nb-traps in $Z^i$, respectively. Therefore, $\mathcal{G}^\ell[Y]$ is a subgame of $\mathcal{G}^\ell$ with $Y$ being any of these sets.\label{it:obs1} %(see Obs.~\ref{obs:traps-subgames} in App.~\ref{}).
 \item \textbf{(App. - Lem.~\ref{app-lem:X_nb-equivalence})} $X_\nb^{i} \cup \SafeReach_\nb^f(X^{i}_\bb, Z_\nb^{i}, \mathcal{G}^\ell) =  \SafeReach_\nb^f(V, X_\nb^{i} \cup Z_\nb^{i}, \mathcal{G}^\ell)$.\label{it:obs2}%(see Lem.~\ref{lem:X_nb-equivalence} in App.~\ref{}).
 \item \textbf{(App.  - Cor.~\ref{app-cor:increasing-decreasing-sequences})} As a consequence of the previous item, $\{X_\nb^{i}\}_{i\in \mathbb{N}}$ is an increasing sequence. Consequently, $\{X_\bb^{i}\}_{i\in \mathbb{N}}$ is a decreasing sequence. As $V$ is finite, this immediately implies that these sequences reach a saturation value for some, and in fact the same, $k$. \label{it:obs3}
 \item \textbf{(App.  - Lem.~\ref{app-lem:safe-reach-Odd-paradise})} If $R \subseteq V$ is an \Odd-paradise in $\mathcal{G}^\ell$, then $\SafeReach^f_\Odd(V, R, \mathcal{G}^\ell)$ is also an \Odd-paradise in $\mathcal{G}^\ell$.\label{it:obs4}
 \item \textbf{(App.  - Lem.~\ref{app-lem:safereacheven-noliveedges})} The set $U \setminus \SafeReach_\bb(U, R, \mathcal{G}^\ell)$ is a $\bb$-trap in $U$. \label{it:obs5}
\end{enumerate}

\vspace{0.1cm}

In contrast to Zielonka's original algorithm, the proof of the procedures $\SOLVE_\Even$ and $\SOLVE_\Odd$ is not identical in \Odd-fair Zielonka's algorithm. This is due to the different safe-reachability set constructions used. Next we sketch the correctness proof of Thm.~\ref{thm:solvebb} for $\bb:=\Odd$, corresponding to the correctness of procedure $\SOLVE_\Odd$. The proof for $\bb:=\Even$ is left to the appendix, as it resembles the proof Zielonka's original algorithm more.
% % \subsubsection{Correctness of $\SOLVE_\Odd$ --  Thm.~\ref{thm:solvebb}}\label{sec:zielonka:correctness:odd}

\begin{proposition}\label{prop:n-odd}
Given the premisses of Thm.~\ref{thm:solvebb} for $\bb = \Odd$, if $Z_\Even^k = \emptyset$ then $\SafeReach^f_\Odd(V, X^k_\Odd, \mathcal{G}^\ell)$ is an \Odd-paradise and $V \setminus \SafeReach^f_\Odd(V, X^k_\Odd, \mathcal{G}^\ell)$ is an \Even-paradise in $\mathcal{G}^\ell$.
\end{proposition}

Within Prop.~\ref{prop:n-odd}, the fact that $Z_\Even^k = \emptyset$ refers to the termination of the recursive call in Alg.~\ref{algo:fair-zielonka-bb} which results in the saturation of the sequence $\{X_\Odd^i\}_{i\in \mathbb{N}}$ with $X_\Odd^k$. This implies that $\SOLVE_\Odd$ returns 
$ T:=\SafeReach^f_\Odd(V, X^k_\Odd, \mathcal{G}^\ell) $, which is an \Odd-paradise and $V \setminus T$ an \Even-paradise. With this, Thm.~\ref{thm:solvebb} follows from Prop.~\ref{prop:n-odd} for $\bb=\Odd$. % Alg.~\ref{algo:fair-zielonka-bb} for $\bb = \Odd$.
We now give a proof sketch of Prop.~\ref{prop:n-odd}.

We first recall from observation~\ref{it:obs1} that $T$ and $V\setminus T$ are \Even- and \Odd-traps in $V$, respectively. In order to prove Prop.~\ref{prop:n-odd}, it remains to show that there exists an \Odd (resp. \Even) strategy template which is winning in $\mathcal{G}^\ell$ and maximal on $T$ (resp. $V\setminus T$). We next give the construction of these templates and a high-level intuition on why they are actually \emph{winning}. 

\smallskip
\noindent\textbf{Winning \Odd Strategy Templates.} 
As $X^k_\Odd$ is known to be an \Even-trap, it can be proven to be an \Odd-paradise by constructing a winning maximal strategy template on it. It then follows from observation~\ref{it:obs4} that $T$ is also an \Odd-paradise.

Towards a construction of a maximal winning \Odd strategy template on $X_\Odd$, we first observe that $X^k_\Odd=Z_\Odd^k\cup \SafeReach^f_\Odd(X^k_\Odd, N^k, \mathcal{G}^\ell)$ (as $Z_\Even^k=\emptyset$). % Now we first consider $Z^k = Z_\Odd^k$ (as $Z_\Even^k=\emptyset$)\todo{IS: why do we 'consider' this, isn't this given for $k$?}
 Then there exists a maximal winning \Odd strategy template $z$ on $Z^k = Z_\Odd^k$ in game $\mathcal{G}^\ell[Z^k]$. % and the definition of $Z_\Odd^k$. 
 Any play $\pi$ compliant with $z$ that starts and stays in $Z^k$ is clearly \Odd winning.
However, $z$ is not necessarily an \Odd strategy template in $\mathcal{G}^\ell$ since there are possibly some $(v,w) \in E$ with $v \in Z^k \cap V_\Even$  and $w \not \in Z^k$.
For all such edges, $w \in \SafeReach^f_\Odd(X^k_\Odd, N^k, \mathcal{G}^\ell)$ since $X^k_\Odd$ is an \Even-trap in $V$.
% 
For the state set $\Xsr_\Odd:=\SafeReach^f_\Odd(X^k_\Odd, N^k, \mathcal{G}^\ell)$, recall from Sec.~\ref{sec:zielonka:fair} that there exists partial strategy template $sr$ defined on $\Xsr_\Odd$ with dead ends in $N^k$.

Using the templates $z$ and $sr$, we can construct a maximal candidate \Odd strategy template on $X^k_\Odd$. Following the intuition behind the construction of $\mathcal{S}^{\mathcal{G}^\ell}$ in Def.~\ref{def:S}, we first define a base subgraph $(X^k_\Odd,\e)$ with $\e\subseteq E$ s.t.\
 $(v,w) \in E $ is in $\e$ if either (i) $(v,w) \in z \cup sr$, (ii) $v \in V_\Even \cap X^k_\Odd$, or (iii) $v \in N^k \cap V_\Odd$ and $w = v_r$
where $v_r$ is a random fixed successor of $v$, that is in $X^k_\Odd$.
Such a successor is guaranteed to exist since $X^k_\Odd$ is an \Even-trap.
We now extend the subgraph $(X^k_\Odd,\e)$ to an \Odd strategy template by adding all live edges originating in vertices  $X^k_\Odd\cap V^\ell$ that lie on a cycle in $\e$, similar to Def.~\ref{const:S} (S3)-(S4). This results in a subgraph $\Sc=(X^k_\Odd,\overline{\e})$ %where $\overline{e}$ is defined to be the saturation value of the sequence $\overline{e}^j = \overline{e}^{j-1} \cup \{(v, w) \in V^\ell \mid v \text{ lies on a cycle in } (X_\Odd^k, \overline{e}^{j-1})\}$ where $\overline{e}^0 = e$.
that is a maximal \Odd strategy template. %\AKS{Why don't we need $\Sc$ to be maximal on $T$?}
% 
The underlying idea behind $\mathcal{S}$ being winning %(formally proven in App.~\ref{app:zielonka-proof}) 
is the following: Any play that starts in $X_\Odd^k$ either stays in $Z^k$ after some point and is won by $\mathcal{S}$ collapsing to $z$, or sees a newly added cycle (one that is not in $z \cup sr$) infinitely often. All such cycles contain a newly added edge. An analysis of newly added edges reveal that, 
all of them -- when seen infinitely often -- eventually drag a play towards $N^i$. Thus, every play that sees a new cycle infinitely often sees $n$ infinitely often, and thus won by \Odd.

\smallskip

\noindent\textbf{Winning \Even Strategy Templates.} 
Here we show that $V \setminus T$ is an \Even-paradise in $\mathcal{G}^\ell$. 
We first define $\Xsr^i_\Even:=\SafeReach^f_\Even(X_\Odd^i, Z_\Even^i, \mathcal{G}^\ell)$ and denote by $sr^i$ the partial \Even strategy template defined on $\Xsr^i_\Even$. We further denote the winning \Even strategy on $Z_\Even^i$ in game $\mathcal{G}^\ell[Z^i]$ by $z^i$. 
% 
We can now construct the \Even strategy template $\mathcal{S} = (V \setminus T, \e)$ where $\e$ is the combination of edges in $sr^i \cup z^i$ with $\{(v,w) \in E \mid v \in V_\Odd \cap (V \setminus T)\}$.
Since $V\setminus T$ is an \Odd-trap by observation~\ref{it:obs5}, the edge set $\e$ stays within $V \setminus T$, i.e. $\e \subseteq V\setminus T \times V \setminus T$. Then clearly, $\mathcal{S}$ is an \Even strategy template.
To see $\mathcal{S}$ is winning we first observe that each $v \in V \setminus T$ there exists a unique $i<k$ such that $v \in \Xsr^i_\Even$. Let $\pi = v_1 v_2 \ldots$ be a play compliant with $\mathcal{S}$ and let $s = \Xsr_1 \Xsr_2 \ldots$ be the sequence such that $v_i \in \Xsr$.
(1) If $v_j \in Z_\Even^i$, $v_{j+1} \in Z_\Even^i \cup \{\Xsr_\Even^r \mid r < i\}$. This follows from $Z_\Even^i$ being an \Odd-trap in $X_\Odd^i$.
(2) If $\pi$ visits $v \in \Xsr^i$ infinitely often, $\pi$ visits $Z_\Even^i$ infinitely often: This is because $\pi$ visits the $(v,w)$ in $\mathcal{S}$ that makes positive progress towards $Z_\Even^i$ infinitely often as well. 
% 
Let $i$ be the minimum index such that $\Xsr_\Even^i$ is seen infinitely often in $s$. By (1), $\pi$ visits $Z_\Even^i$ infinitely often and by (1) and the minimality of $i$, it should eventually stay in $Z_\Even^i$.
Thus $\mathcal{S}$ eventually collapses to $z_\Even^i$ on $\pi$ and the play is won by \Even.

\vspace{-0.2cm}
%\smallskip
%\noindent \textbf{The Algorithm.} Observe that $\SOLVE_\Odd(n, \mathcal{G}^\ell)$ (Alg.~\ref{algo:fair-zielonka-odd}) calculates the sets as given in the construction(Fig.~\ref{fig:kuesters-figure-extended}) where $X$ holds the value of $X_\Odd^i$ at the end of the $i^{th}$ iteration of it's \emph{while} loop. $\{X_\Odd^i\}_{i \in \mathbb{N}}$ is a decreasing sequence which saturates at some $X_\Odd^k$ where $Z_\Even^k = \emptyset$.
%$\SOLVE_\Odd(n, \mathcal{G}^\ell)$ returns  $\SafeReach(V, X_\Odd^k, \mathcal{G}^\ell)$, which is shown to be equal to \Wo by Prop.~\ref{prop:n-odd}. 

%Similarly, $X$ in $\SOLVE_\Even(n, \mathcal{G}^\ell)$ (Alg.~\ref{algo:fair-zielonka-even}) holds the value of $X_\Even^i$ after the $i^{th}$ iteration. The constructive proof of Thm.~\ref{thm:solveeven} (App.~\ref{app:zielonka-proof}) shows that the saturation value $X^{k'}_\Even$ (where $Z^{k'}_\Odd = \emptyset$) is equal to \We, and this is exacly the value $\SOLVE_\Even(n, \mathcal{G}^\ell)$ returns.
%

