\subsection{Proof of the Fixed-point Formula for \Wo}\label{app:fp-proof}
It was recently shown in \cite{banerjee2022fast} that the winning region $\We$ for \Even in an \Odd-fair parity game $\mathcal{G}^\ell$ with least even upper bound priority $l\geq 0$ can be computed by the fixed-point algorithm 
% 
\begin{align}\label{eq:fp-even}
 \We = &\nu {Y_l}.~ \mu X_{l-1}.~ \ldots \nu{Y_2}.~ \mu{X_1}.~ \bigcup_{j \in \ev{2}{l}} \A_j \quad \\
 & \text{ where, } \quad \A_j := \left(C_j \cap Cpre_\Even(Y_j)\right) \cup \left(\left(\textstyle\bigcup_{i \in [1,j-1]}C_i\right) \cap \Apre(Y_j, X_{j-1})\right)\nonumber
\end{align}

As  \Odd-fair parity games are determined, we can simply compute the winning region for player $\Odd$ by negating \eqref{eq:fp-even}, which leads to Prop.~\ref{prop: W_Odd}. For the sake of self-containment, we restate Prop.~\ref{prop: W_Odd} here.

% restating prop 2
\begingroup
\def\theproposition{\ref{prop: W_Odd}}
\begin{proposition}
    Given an \Odd-fair parity game $\mathcal{G}^\ell = (\ltup{V, \Ve, \Vo, E, \chi}, E^\ell)$ with least even upper bound $l\geq 0$ and
\begin{align}\label{eq:fp-odd-app}
    Z := &\mu {Y_l}.~  \nu {X_{l-1}}.~  \ldots \mu{Y_2}.~  \nu{X_1}.~  \bigcap_{j \in \ev{2}{l}} \B_j, \\
    &\text{ where} \quad
    \B_j := \left(\textstyle\bigcup_{i \in [j+1,l]} C_i\right) \cup \left(\overline{C_j} \cap \Npre(Y_j, X_{j-1}) \right) \cup \left(C_j \cap \Cpre_\Odd(Y_j)\right)\nonumber
\end{align}
then $\Phi=\Wo$.
Further, it takes $\mathcal{O}(n^{l+1})$ symbolic steps to compute $\Wo$ via \eqref{eq:fp-odd}.
\end{proposition}
\addtocounter{proposition}{-1} % decrease the counter that holds proposition numbers, so that the previous restated proposition is not seen.
\endgroup

\begin{proof}
We use the negation rule of the $\mu$-calculus, i.e., $\neg (\mu X~.~F(X))=\nu X~.~\neg F(\neg X)$, to negate \eqref{eq:fp-even}. Using the equivalences in \eqref{equ:Preseq} and \eqref{equ:cpre_equal} and common De-Morgan laws, we get 
\begin{subequations}
\begin{align}
 \neg \A_j(\neg Y_j,\neg X_{j-1})=&\left(\overline{C_j} \cup \Cpre_\Odd(Y_j)\right) \cap \left(\left(\textstyle\bigcup_{i \in [j,l]} C_i\right) \cup \Npre(Y_j, X_{j-1})\right)\\
 =&\left(\textstyle\bigcup_{i \in [j+1, l]} C_i\right) \cup \left(\overline{C_j} \cap \Npre(Y_j, X_{j-1})\right) \nonumber\\
 &\cup \left(C_j \cap \Cpre_\Odd(Y_j)\right) \cup \left(\Cpre_\Odd(Y_j) \cap \Npre(Y_j, X_{j-1})\right)\label{eq2}\\
 =&\left(\textstyle\bigcup_{i \in [j+1,l]} C_i\right) \cup \left(\overline{C_j} \cap \Npre(Y_j, X_{j-1}) \right) \cup \left(C_j \cap \Cpre_\Odd(Y_j)\right)
\end{align}\end{subequations}
where the last equivalence follows from the observation that the last term of \eqref{eq2} is redundant since it is a subset of both $ \Npre(Y_j, X_{j-1})$ and $\Cpre_\Odd(Y_j)$: If a $v$ is in the last term, it either has priority $j$, in which case it is already in $C_j \cap \Cpre_\Odd(Y_j)$, or it has a different priority, in which case it is already in $\Npre(Y_j, X_{j-1})$. %As all formal variables can have arbitrary symbols, we just rename them to their non-overlined versions but swap the preceding $\mu$/$\nu$ operators. This yields \eqref{eq:fp-odd} from negating \eqref{eq:fp-even}.
\end{proof}
