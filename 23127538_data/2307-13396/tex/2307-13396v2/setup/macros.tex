% custom environments
% \newtheorem{proposition}{Proposition}[section]
% \newtheorem{corollary}{Corollary}
 
% todo setup

\newcommand{\unsure}{\todo[linecolor=brown,backgroundcolor=brown!35,bordercolor=brown]}
\newcommand{\change}{\todo[linecolor=blue,backgroundcolor=blue!40,bordercolor=blue]}
\newcommand{\info}{\todo[linecolor=green!90!yellow,backgroundcolor=green!30!yellow!40,bordercolor=green!90!yellow]}
\newcommand{\improvement}{\todo[linecolor=yellow,backgroundcolor=yellow!25,bordercolor=yellow]}
\newcommand{\add}{\todo[linecolor=red,backgroundcolor=red!25,bordercolor=red	]}
\newcommand{\exacttodo}{\todo[fancyline]}
\newcommand{\thiswillnotshow}{\todo[disable]}
 
% Fonts
\newcommand{\fontsmall}{\fontsize{7pt}{6pt}\selectfont}

% Comments
\newcommand{\IS}[1]{\textcolor{red!50}{[\textbf{IS:} #1]}}
\newcommand{\AKS}[1]{\textcolor{violet!50}{[\textbf{AKS:} #1]}}
\newcommand{\new}[1]{{\color{cyan} #1}}
%\newcommand{\todo}[1]{\textcolor{orange}{\textit{TODO:} #1}}

% Colors
\newcommand{\red}[1]{{\color{red} #1}}
\newcommand{\blue}[1]{{\color{Turquoise} #1}}
\newcommand{\comm}[1]{{\color{lightgray} $\diamond$ #1}}
\newcommand{\green}[1]{{\color{Green} #1}}

\newcommand{\ev}[2]{\llbracket {#1}, {#2} \rrbracket}
\newcommand{\1}{\mathbb{I}}
\newcommand{\te}{\mathbf{t}}

% JG 
\newcommand{\justs}{$(newpri, J)$ }
\newcommand{\just}{$(newpri, J, MW)$ }
\newcommand{\justi}{$(newpri^{i,j,fin}, J^{i,j,fin}, MW^{i,j,fin})$ }
\newcommand{\justioa}{$(newpri^{i,j,fin}_{oa}, J^{i,j,fin}_{oa}, MW^{i,j,fin}_{oa})$}

\newcommand{\dummy}{$\overline{\mathcal{G}}^\ell_{X_\Odd}$}
\newcommand{\SR}{$\mathbf{SR}$}
\newcommand{\SRT}{$\mathbf{SRT}$}
\newcommand{\SRSRT}{\mathbf{SR}\setminus \mathbf{SRT}}
\newcommand{\dummySR}{$\overline{\mathcal{G}}^\ell_{\mathbf{SR}}$}
\newcommand{\dummySRSRT}{$\overline{\mathcal{G}}^\ell_{\mathbf{SR}\setminus \mathbf{SRT}}$}


\newcommand{\Lpre}{\mathsf{Lpre}}
\newcommand{\Apre}{\mathsf{Apre}}
\newcommand{\Cpre}{\mathsf{Cpre}}
\newcommand{\Npre}{\mathsf{Npre}}
\newcommand{\Pre}{\mathsf{Pre}}
\newcommand{\SOLVE}{\mathsf{SOLVE}}
\newcommand{\SafeReach}{\mathsf{SafeReach}}
\newcommand{\e}{E'}

\newcommand{\Even}{\ensuremath{\textsf{Even}}\xspace}
\newcommand{\Odd}{\ensuremath{\textsf{Odd}}\xspace}
\newcommand{\Ve}{\ensuremath{V_{\Even}}\xspace}
\newcommand{\Vo}{\ensuremath{V_{\Odd}}\xspace}
\newcommand{\We}{\ensuremath{\mathcal{W}_{\Even}}\xspace}
\newcommand{\Wo}{\ensuremath{\mathcal{W}_{\Odd}}\xspace}
\newcommand{\bb}{\ensuremath{\Lambda}\xspace}
\newcommand{\nb}{\ensuremath{{\neg\bb}}\xspace}
\newcommand{\Xsr}{\ensuremath{{\mathcal{X}}}\xspace}

\newcommand{\sem}[1]{{\color{blue}\mathbf{#1}}}
\newcommand{\id}{\mathbbm{1}}
\newcommand{\CondPre}{\mathrm{CondPre}}

% \newcommand{\p}[1]{\ensuremath{\text{Player}~#1}}

\DeclareMathOperator{\Paths}{\mathsf{PATH}}


%  \newtheorem{problem}{Problem}
% \newtheorem{assumption}{Assumption}
% \newtheorem{observation}{Observation}

% Generic macros
\newcommand{\Ker}{T_{\mathfrak s}}
\newcommand{\half}{\sfrac{1}{2}}
\newcommand{\fairsyn}{\textsf{Fairsyn}\xspace}

%% Algorithm
%\renewcommand{\algorithmicrequire}{\textbf{Input:}}
%\renewcommand{\algorithmicensure}{\textbf{Output:}}

%Mathematical notations and symbols
\newcommand{\N}{\mathbb{N}}
\newcommand{\Q}{\mathbb{Q}}
\newcommand{\Z}{\mathbb{Z}}
\newcommand{\Prop}{Prop}
\newcommand{\Var}{Var}


% \ifthenelse{\boolean{changePowerNotation}}{
% 	\newcommand{\powset}[1]{2^{#1}}
% }{%else
% 	\newcommand{\powset}[1]{2^{#1}} %change powerset notation here
% }

\newcommand{\bs}{\backslash}


% mu-calculus
\newcommand{\all}{\left[~\right]}
\newcommand{\exi}{<>}
\newcommand{\mean}[1]{{\llbracket #1 \rrbracket}}

%LTL notations
\newcommand{\LTLnext}{\bigcirc}
\newcommand{\LTLeventually}{\lozenge}
\newcommand{\LTLalways}{\square}
\newcommand{\LTLuntil}{\mathbf{U}}

% \ifthenelse{\boolean{changeLTLnotation}}{
% 	\renewcommand{\bigcirc}{\mathbf{X}}
% 	\renewcommand{\lozenge}{\mathbf{E}}
% 	\renewcommand{\square}{\mathbf{G}}
% 
% }{%else
% }

%Games
% \newcommand{\game}{\mathcal{G}}
\newcommand{\El}{E^{\ell}}
\newcommand{\Vl}{V^{\ell}}
\newcommand{\Gl}{\game^{\ell}}
\newcommand{\twohalf}{2\sfrac{1}{2}}
\newcommand{\onehalf}{1\sfrac{1}{2}}
\newcommand{\Inf}{\mathrm{Inf}}
% \newcommand{\win}{\mathcal{W}}
\newcommand{\wino}{\win_0}
\newcommand{\wini}{\win_1}

% Probability, MC, IMC
\newcommand{\f}{\mathcal{F}}
\newcommand{\pp}{\mathbb{P}}
\newcommand{\ee}{\mathbb{E}}
%\newcommand{\tker}[3]{\ifthenelse{\isempty{#1}}{\ensuremath{T(#2\mid #3)}}{\ensuremath{T^{#1}(#2\mid #3)}}}
\newcommand{\tden}[2]{\ensuremath{t(#1 \mid #2)}}
\newcommand{\intr}{\mathrm{Interval}}
\newcommand{\mulb}{\mu_{lb}}
\newcommand{\muub}{\mu_{ub}}
\newcommand{\supp}{\mathsf{supp}}
\newcommand{\cdf}{\Phi}
\newcommand{\Den}{\mathcal{D}}
\newcommand{\dist}{\mathit{Dist}}
\newcommand{\Dr}{\mathit{Derand}}

% Related to controlled markov processes
\newcommand{\Sys}{\mathfrak S}
\newcommand{\cell}[1]{\llbracket #1\rrbracket}
\newcommand{\Xs}{\mathcal S}
\newcommand{\xs}{x}
\newcommand{\Us}{\mathcal U}
\newcommand{\us}{u}
\newcommand{\reach}{\ensuremath{\Phi}}
\newcommand{\tker}{T_{\mathfrak s}}

% Abstraction
\newcommand{\Abs}{\mathcal{A}}
\newcommand{\Xh}{\widehat{\Xs}}
\newcommand{\xh}{\widehat{\xs}}
\newcommand{\yh}{\widehat{y}}
\newcommand{\Fo}{\overline{F}}
\newcommand{\Fu}{\underline{F}}
\newcommand{\Trh}{Tr}
\newcommand{\Bh}{\underline{B}}
\newcommand{\Bho}{\overline{B}}

% Different types of predecessors
\newcommand{\cpre}{\mathrm{Cpre}}
\newcommand{\pre}{\mathrm{Pre}}
\newcommand{\apre}{\mathrm{Apre}}
\newcommand{\upre}{\mathrm{Upre}}
\newcommand{\epre}{\mathrm{Pre}^{\exists}}
\newcommand{\elpre}{\mathrm{Lpre}^{\exists}}
\newcommand{\alpre}{\mathrm{Lpre}^{\forall}}
\newcommand{\eapre}{\mathrm{Pre}^{\forall}}

% Controller
\newcommand{\Cont}{\mathcal{C}}
\newcommand{\Conth}{\widehat{\Cont}}
\newcommand{\M}{\mathrm{M}}

% winning regions
\newcommand{\sw}{\mathsf{SW}}
\newcommand{\asw}{\mathsf{ASW}}
\newcommand{\pw}{\mathsf{PW}}
\newcommand{\slo}{\mathsf{SL}}
\newcommand{\W}{\mathsf{WinDom}}
\newcommand{\LD}{\mathsf{LoseDom}}

% ATL
\newcommand{\atldia}[1]{\ensuremath{\langle\langle #1\rangle\rangle}}
\newcommand{\atlsqr}[1]{\ensuremath{[[#1]]}}

% Misc
% \newcommand{\set}[1]{\lbrace #1 \rbrace}
% \newcommand{\tup}[1]{\langle #1 \rangle}
\newcommand{\dom}{\mathsf{dom}}
\newcommand{\jac}{\mathbb{J}}
\newcommand{\tup}[1]{\left( #1\right)}
\newcommand{\ltup}[1]{\langle #1\rangle}

\newcommand{\safe}{\mathsf{Safe}}
\newcommand{\unsafe}{\mathsf{Unsafe}}
\newcommand{\computeInv}{\mathrm{ComputeInvariant}}
\newcommand{\upperAbs}{\mathrm{UpperAbsorbing}}
\newcommand{\lowerAbs}{\mathrm{LowerAbsorbing}}
\newcommand{\las}[1]{\mathrm{l.t.a.s.}(#1)}
\newcommand{\ulas}[1]{\mathrm{u.l.a.s.}(#1)}
\newcommand{\olas}[1]{\mathrm{o.l.a.s.}(#1)}
\newcommand{\norm}[1]{\left\Vert #1\right\Vert}
\newcommand{\T}{\mathrm{TRUE}}
\newcommand{\F}{\mathrm{FALSE}}
% \newcommand{\flag}{\mathsf{badState}}
%\newcommand{\kern}{\mathcal{K}}

%Specifications/Games
\newcommand{\FR}{\mathcal{R}}
\newcommand{\FS}{\mathcal{S}}
\newcommand{\FgR}{\widetilde{\mathcal{R}}}

\newcommand{\FP}{\mathcal{C}}
\newcommand{\FB}{B}
\newcommand{\FGB}{\mathcal{B}}
\newcommand{\FGRo}{\mathcal{A},\Fc}
\newcommand{\Fc}{\mathcal{F}}
\newcommand{\Tc}{\mathcal{T}}


\newcommand{\Gc}{\mathbf{G}}
\newcommand{\Rc}{\mathcal{R}}

\newcommand{\vl}{\alpha}
\newcommand{\vR}{\varphi}
\newcommand{\vS}{\varphi(\FS)}
\newcommand{\vP}{\varphi}
\newcommand{\vB}{\varphi}
\newcommand{\vGB}{\varphi}

\newcommand{\vgR}{\varphi}
\newcommand{\vGRo}{\varphi}

\newcommand{\WR}{\mathcal{W}}
% \newcommand{\WR}{\mathcal{W}^{\mathrm{R}}}
\newcommand{\WlR}{\mathcal{W}}
\newcommand{\WgR}{\mathcal{W}^{\mathrm{gR}}}
\newcommand{\WlgR}{\mathcal{W}^{\mathrm{gR}}}

 \newcommand{\ps}[1]{\ensuremath{~{}^{#1}\!}}
\newcommand{\OpInf}[1]{\mathop{\mathrm{Inf}(#1)}}
\newcommand{\rank}[1]{\mathop{\mathsf{rank}\ifthenelse{\isempty{#1}}{}{(#1)}}}
\newcommand{\ranko}[1]{\mathop{\overline{\mathrm{rank}}\ifthenelse{\isempty{#1}}{}{(#1)}}}

\newcommand{\Ct}{\widetilde{\mathcal{C}}}
\newcommand{\Pt}[1]{\widetilde{P}_{\setminus #1}}
\newcommand{\Xt}{\widetilde{X}}
\newcommand{\Yt}{\widetilde{Y}}
\newcommand{\Zt}{\widetilde{Z}}
\newcommand{\Xo}{\overline{X}}
\newcommand{\Yo}{\overline{Y}}
\newcommand{\Zo}{\overline{Z}}
\newcommand{\Xc}{\check{X}}
\newcommand{\Yc}{\check{Y}}
\newcommand{\Zc}{\check{Z}}
\newcommand{\Dt}{\widetilde{D}}
\newcommand{\delt}{\widetilde{\delta}}
\newcommand{\dt}{\widetilde{d}}


\newcommand{\Ro}{\overline{R}}
\newcommand{\To}{\overline{T}}
\newcommand{\Qo}{\overline{Q}}


% Chain recurrence
\newcommand{\CR}{\mathcal{CR}}

% Transition systems
\newcommand{\tro}{\overline{R}}
\newcommand{\tru}{\underline{R}}

% Winning regions
\newcommand{\wu}{\underline{W}}
\newcommand{\wo}{\overline{W}}
\newcommand{\lo}{\overline{L}}

% References
\newcommand{\REFlem}[1]{\text{Lem.}~\ref{#1}}
\newcommand{\REFthm}[1]{\text{Thm.}~\ref{#1}}
\newcommand{\REFdef}[1]{Def.~\ref{#1}}
\newcommand{\REFalg}[1]{Alg.~\ref{#1}}
\newcommand{\REFrem}[1]{Rem.~\ref{#1}}
\newcommand{\REFexp}[1]{Expl.~\ref{#1}}
\newcommand{\REFsec}[1]{Sec.~\ref{#1}}
\newcommand{\REFsubsec}[1]{Subsec.~\ref{#1}}
\newcommand{\REFprop}[1]{Prop.~\ref{#1}}
\newcommand{\REFfig}[1]{Fig.~\ref{#1}}
\newcommand{\REFass}[1]{Assump.~\ref{#1}}
\newcommand{\REFapp}[1]{App.~\ref{#1}}
\newcommand{\REFcor}[1]{Cor.~\ref{#1}}
\newcommand{\REFtab}[1]{Table~\ref{#1}}
\newcommand{\REFex}[1]{Ex.~\ref{#1}}
\newcommand{\REFprob}[1]{Prob.~\ref{#1}}



% \newcommand{\FR}{\mathcal{F}^{\mathrm{R}}}
% \newcommand{\Fc}{\mathcal{F}}
% \newcommand{\OpInf}[1]{\mathop{\mathrm{Inf}(#1)}}
%  \newcommand{\rank}[1]{\mathop{\mathrm{rank}(#1)}}


% -----------------------------------------------------------------------------
% comma seperated lists
% -----------------------------------------------------------------------------
\makeatletter 
\newif\ifFIRST
\newif\ifSECOND
\let\LISTOP\relax
\newcommand{\List}[4][\;]{#3#1%
        \FIRSTtrue
        \@for\i:=#2\do{%
        \ifFIRST\LISTOP{\i}\FIRSTfalse\else,\LISTOP{\i}\fi%
        }%
        #1#4%
        \let\LISTOP\relax
}
\makeatother

\makeatletter 
\newcommand{\TRUE}{\ensuremath{\mathtt{True}}}
\newcommand{\FALSE}{\ensuremath{\mathbf{False}}}

\newcommand{\propNeg}{\@ifstar\propNegStar\propNegNoStar}
\newcommand{\propNegStar}[1]{\ensuremath{\left(\propNegNoStar{#1}\right)}}
\newcommand{\propNegNoStar}[2][\cdot]{\ensuremath{\neg\ifthenelse{\isempty{#2}}{#1}{#2}}}

\newcommand{\propConj}{\@ifstar\propConjStar\propConjNoStar}
\newcommand{\propConjStar}[2]{\ensuremath{\left(\propConjNoStar{#1}{#2}\right)}}
\newcommand{\propConjNoStar}[3][\cdot]{\ensuremath{\ifthenelse{\isempty{#2}}{#1}{#2}\wedge\ifthenelse{\isempty{#3}}{#1}{#3}}}

\newcommand{\propDisj}{\@ifstar\propDisjStar\propDisjNoStar}
\newcommand{\propDisjStar}[2]{\ensuremath{\left(\propDisjNoStar{#1}{#2}\right)}}
\newcommand{\propDisjNoStar}[3][\cdot]{\ensuremath{\ifthenelse{\isempty{#2}}{#1}{#2}\vee\ifthenelse{\isempty{#3}}{#1}{#3}}}

\newcommand{\propImp}{\@ifstar\propImpStar\propImpNoStar}
\newcommand{\propImpStar}[2]{\ensuremath{\left(\propImpNoStar{#1}{#2}\right)}}
\newcommand{\propImpNoStar}[3][\cdot]{\ensuremath{\ifthenelse{\isempty{#2}}{#1}{#2}\Rightarrow\ifthenelse{\isempty{#3}}{#1}{#3}}}

\newcommand{\propAequ}{\@ifstar\propAequStar\propAequNoStar}
\newcommand{\propAequStar}[2]{\ensuremath{\left(\propAequNoStar{#1}{#2}\right)}}
\newcommand{\propAequNoStar}[3][\cdot]{\ensuremath{\ifthenelse{\isempty{#2}}{#1}{#2}\Leftrightarrow\ifthenelse{\isempty{#3}}{#1}{#3}}}

% \newcommand{\propXOR}{\@ifstar\propXORStar\propXORNoStar}
% \newcommand{\propXORStar}[2]{\ensuremath{\left(\propXORNoStar{#1}{#2}\right)}}
% \newcommand{\propXORNoStar}[3][\cdot]{\ensuremath{\ifthenelse{\isempty{#2}}{#1}{#2}\oplus\ifthenelse{\isempty{#3}}{#1}{#3}}}

% \newcommand{\propSemEQ}{\ensuremath{\equiv}}

% -----------------------------------------------------------------------------
% predicate logic
% -----------------------------------------------------------------------------
\newcommand{\AllQ}{\@ifstar\AllQStar\AllQNoStar}
\newcommand{\AllQStar}[3][\;]{\ensuremath{\left(\forall #2#1.#1#3\right)}}
\newcommand{\AllQNoStar}[3][\;]{\ensuremath{\forall #2#1.#1#3}}
\newcommand{\AllQu}{\@ifstar\AllQuStar\AllQuNoStar}
\newcommand{\AllQuStar}[3][\;]{\ensuremath{\left(\forall^{\infty} #2#1.#1#3\right)}}
\newcommand{\AllQuNoStar}[3][\;]{\ensuremath{\forall^{\infty} #2#1.#1#3}}

\newcommand{\ExQ}{\@ifstar\ExQStar\ExQNoStar}
\newcommand{\ExQStar}[3][\;]{\ensuremath{\left(\exists #2#1.#1#3\right)}}
\newcommand{\ExQNoStar}[3][\;]{\ensuremath{\exists #2#1.#1#3}}

\newcommand{\NExQ}{\@ifstar\NExQStar\NExQNoStar}
\newcommand{\NExQStar}[3][\;]{\ensuremath{\left(\nexists #2#1.#1#3\right)}}
\newcommand{\NExQNoStar}[3][\;]{\ensuremath{\nexists #2#1.#1#3}}

\newcommand{\UniqueQ}{\@ifstar\UniqueQStar\UniqueQNoStar}
\newcommand{\UniqueQStar}[3][\;]{\ensuremath{\left(\exists! #2#1.#1#3\right)}}
\newcommand{\UniqueQNoStar}[3][\;]{\ensuremath{\exists! #2#1.#1#3}}

% \newcommand{\PredSemImp}{\ensuremath{\Rightarrow}}
% \newcommand{\PredSemImpB}{\ensuremath{\Leftarrow}}
% \newcommand{\PredSemEQ}{\ensuremath{\Leftrightarrow}}


\newenvironment{propConjA}{\left(\def\unionAtest{1}\begin{array}{@{\if\unionAtest1\gdef\unionAtest{0}\phantom{\wedge}\else\wedge\fi}l@{}}}{\end{array}\right)}
\newenvironment{propImpA}{\left(\def\propImpAtest{1}\begin{array}{@{\if\propImpAtest1\gdef\propImpAtest{0}\phantom{\rightarrow}\else\rightarrow\fi}l@{}}}{\end{array}\right)}
\newenvironment{propDisjA}{\left(\def\unionAtest{1}\begin{array}{@{\if\unionAtest1\gdef\unionAtest{0}\phantom{\vee}\else\vee\fi}l@{}}}{\end{array}\right)}
\newenvironment{propUnionA}{\def\unionAtest{1}\begin{array}[t]{@{\if\unionAtest1\gdef\unionAtest{0}\else\cup\fi}l@{}}}{\end{array}}



  \newlength{\SFS@HEIGHT}
  \newlength{\SFS@WIDTH}
  \newcommand{\SplitX}[2]{
            \settoheight{\SFS@HEIGHT}{$#2$}
            \settowidth{\SFS@WIDTH}{$#2$}
            \mbox{\begin{tikzpicture}[baseline=(current bounding box.center)]
            \node[] (E) at (0,0) {$#1$};
            \node[inner sep=0pt] (F) at ($(E.south west)+(1ex,-1ex)+(3ex+.5\SFS@WIDTH,-\SFS@HEIGHT)$) {$#2$};
            \node[] (E) at (0,0) {\phantom{$#1$}};
            \draw[fill] ($(E.east)+(1ex,0ex)$) circle (.2ex);
            \draw[-] ($(E.east)+(1ex,0ex)$) -- ($(E.south east)+(1ex,-0.5ex)$) -- ($(E.south west)+(1ex,-0.5ex)$) -- ($(E.south west)+(1ex,-1ex)-(0,\SFS@HEIGHT)$) -- ($(E.south west)+(2.5ex,-1ex)-(0,\SFS@HEIGHT)$);
            \draw[fill] ($(E.south west)+(2.5ex,-1ex)-(0,\SFS@HEIGHT)$) circle (.2ex);
            \end{tikzpicture}}}
  \newcommand{\SplitS}[2]{
            \settoheight{\SFS@HEIGHT}{$#2$}
            \settowidth{\SFS@WIDTH}{$#2$}
            \mbox{\begin{tikzpicture}[baseline=(current bounding box.center)]
            \node[] (E) at (0,0) {$#1$};
            \node[inner sep=0pt] (F) at ($(E.south west)+(1ex,0.5ex)+(0ex+.5\SFS@WIDTH,-\SFS@HEIGHT)$) {$#2$};
            \end{tikzpicture}}}     

                     \newcommand{\SetComp}[3][]{\{#1#2#1\mid#1#3#1\}} 
  \newcommand{\AllQSplit}[2]{\SplitX{\forall\;#1\;.}{#2}}
    \newcommand{\AllQuSplit}[2]{\SplitX{\forall^\infty\;#1\;.}{#2}}
  \newcommand{\SetCompSplit}[2]{\left\{\SplitS{#1\mid}{#2}\right\}}
  \newcommand{\ExQSplit}[2]{\SplitX{\exists\;#1\;.}{#2}}
  \newcommand{\nExQSplit}[2]{\SplitX{\neg\exists\;#1\;.}{#2}}
  \newcommand{\THESplit}[2]{\SplitX{\iota\;#1\;.}{#2}}
  \newcommand{\LetSplit}[2]{\SplitX{\text{let $#1$ in}}{#2}}
  \newcommand{\propImpSplitS}[2]{\SplitS{#1\;\Rightarrow\;}{#2}}
  \newcommand{\propImpSplit}[2]{\SplitX{#1\;\Rightarrow\;}{#2}}
  \newcommand{\propAequSplit}[2]{\SplitX{#1\;\Leftrightarrow\;}{#2}}
 \newcommand{\propEquSplit}[2]{\SplitS{#1\;=\;}{#2}}
  \newcommand{\DERIVESplit}[3]{\SplitX{#2\;\DERIVE{#1}\;}{#3}}
  \newcommand{\SubseteqSplit}[2]{\SplitS{#1\;\subseteq\;}{#2}}
%   
%   

% =============================================================================
% <<< logics
% =============================================================================

% =============================================================================
% >>> sets, tuples, strings
% =============================================================================

% -----------------------------------------------------------------------------
% shortcuts for sets, tuples, strings
% -----------------------------------------------------------------------------
% \newcommand{\Set}[2][]{\List[#1]{#2}{\{}{\}}}
\newcommand{\VSet}[2][]{\let\LISTOP\val\List[#1]{#2}{\{}{\}}}
\newcommand{\SetL}[2][]{\List[#1]{#2}{\{}{}}
\newcommand{\SetR}[2][]{\List[#1]{#2}{}{\}}}
\newcommand{\Tuple}[2][]{\List[#1]{#2}{(}{)}}
\newcommand{\TupleA}[2][]{\List[#1]{#2}{\langle}{\rangle}}
\newcommand{\VTuple}[2][]{\let\LISTOP\val\List[#1]{#2}{(}{)}}
% \newcommand{\EqClass}[2][]{\List[#1]{#2}{[}{]}}
% \newcommand{\EqClassL}[2][]{\List[#1]{#2}{[}{}}
% \newcommand{\EqClassR}[2][]{\List[#1]{#2}{}{]}}
\newcommand{\EMPTYSTRING}{\lambda}
\newcommand{\vEMPTYSTRING}{\vlambda}


% -----------------------------------------------------------------------------
% operations on sets
% -----------------------------------------------------------------------------
\newcommand{\DISJOINT}[2]{\INTERSECT{#1}{#2}=\0}

\newcommand{\UNION}{\@ifstar\UNIONStar\UNIONNoStar}
\newcommand{\UNIONStar}[2]{\ensuremath{\left(\UNIONNoStar{#1}{#2}\right)}}
\newcommand{\UNIONNoStar}[2]{\ensuremath{\ifthenelse{\isempty{#1}}{\cdot}{#1}\cup\ifthenelse{\isempty{#2}}{\cdot}{#2}}}

\newcommand{\UNIOND}{\@ifstar\UNIONDStar\UNIONDNoStar}
\newcommand{\UNIONDStar}[2]{\ensuremath{\left(\UNIONDNoStar{#1}{#2}\right)}}
\newcommand{\UNIONDNoStar}[2]{\ensuremath{\ifthenelse{\isempty{#1}}{\cdot}{#1}\uplus\ifthenelse{\isempty{#2}}{\cdot}{#2}}}

\newcommand{\SETMINUS}{\@ifstar\SETMINUSStar\SETMINUSNoStar}
\newcommand{\SETMINUSStar}[2]{\ensuremath{\left(\SETMINUSNoStar{#1}{#2}\right)}}
\newcommand{\SETMINUSNoStar}[2]{\ensuremath{\ifthenelse{\isempty{#1}}{\cdot}{#1}\setminus\ifthenelse{\isempty{#2}}{\cdot}{#2}}}

\newcommand{\INTERSECT}{\@ifstar\INTERSECTStar\INTERSECTNoStar}
\newcommand{\INTERSECTStar}[2]{\ensuremath{\left(\INTERSECTNoStar{#1}{#2}\right)}}
\newcommand{\INTERSECTNoStar}[2]{\ensuremath{\ifthenelse{\isempty{#1}}{\cdot}{#1}\cap\ifthenelse{\isempty{#2}}{\cdot}{#2}}}

\newcommand{\CARTPROD}{\@ifstar\CARTPRODStar\CARTPRODNoStar}
\newcommand{\CARTPRODStar}[2]{\ensuremath{\left(\CARTPRODNoStar{#1}{#2}\right)}}
\newcommand{\CARTPRODNoStar}[2]{\ensuremath{\ifthenelse{\isempty{#1}}{\cdot}{#1}\times\ifthenelse{\isempty{#2}}{\cdot}{#2}}}

\newcommand{\0}{\ensuremath{\emptyset}}

\newcommand{\FINCOUNT}{\@ifstar\FinCountStar\FinCountNoStar}
\newcommand{\FinCountStar}[1]{\ensuremath{\#(\ifthenelse{\isempty{#1}}{\cdot}{#1})}}
\newcommand{\FinCountNoStar}[1]{\ensuremath{\#\left(\ifthenelse{\isempty{#1}}{\cdot}{#1}\right)}}

\newcommand{\CARD}[1]{\ensuremath{\ON{card}(#1)}}

\newcommand{\CONCAT}[4]{#1\wedge^{#2}_{#3}#4}

\makeatother 

\newcommand{\incone}[2]{[#1;#2]}
\newcommand{\inctwo}[2]{[#1\mathrel{{.}{.}}\nobreak#2]}
