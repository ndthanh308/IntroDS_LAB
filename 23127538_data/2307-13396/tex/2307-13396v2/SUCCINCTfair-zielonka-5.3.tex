\subsection{Correctness of the \Odd-fair Zielonka's Algorithm}\label{sec:zielonka:correct}
% 
% 
% Now we will try to convey the idea of the correctness proof of Alg.~\ref{alg:fair-zielonka}. 
We first recall that \Odd-fair parity games are determined. %the results of Banerjee et. al.~\cite{banerjee2022fast}. Given an \Odd-fair parity game $\mathcal{G}^\ell=\ltup{(V, V_\Even, V_\Odd, E, \chi), E^\ell}$, we therefore know that \We and \Wo partition $V$. Following the original Zieloka's algorithms proof, it therefore remains to show that $\SOLVE_{\Even}(n,\mathcal{G}^\ell)$ and $\SOLVE_{\Odd}(n,\mathcal{G}^\ell)$ as in Alg.~\ref{algo:fair-zielonka-odd} and Alg.~\ref{algo:fair-zielonka-even} actually compute \We and \Wo, respectively.
Next, we prove the correctness of the algorithm by induction on $n$. Since in the base case $n = 0$ the calls correctly return $\emptyset$, it suffices to prove the correctness of each function, assuming the correctness of the other. This is formalized next. %for $\SOLVE_{EVEN}$ (Alg.~\ref{algo:fair-zielonka-even}), where $\SOLVE_{ODD}$ (Alg.~\ref{algo:fair-zielonka-odd}) follows from a symmetrical argument with odd $n$.
\begin{comment}
\begin{theorem}[Correctness of $\SOLVE_{\bb}$, Alg.~\ref{algo:fair-zielonka-bb}]\label{thm:solvebb}
Let $\mathcal{G}^\ell=\ltup{(V, V_\Even, V_\Odd, E, \chi), E^\ell}$ be an \Odd-fair parity game with \textsf{parity(\bb)}\footnote{\textsf{parity(\Odd)} is odd and  \textsf{parity(\Even)} is even.} upper bound priority $n$. Further, assume that for any \Odd-fair parity game $\mathcal{G'}^\ell$ with  \textsf{parity($\bb$)} upper bound priority $n'<n$ holds that 
 $\mathcal{W}_\bb[\mathcal{G'}^\ell]=\SOLVE_{\bb}(n',\mathcal{G'}^\ell)$ and for any  \Odd-fair parity game $\mathcal{G''}^\ell$ with \textsf{parity($\neg \bb$)} upper bound priority $n''<n$ holds that 
 $\mathcal{W}_{\neg \bb}[\mathcal{G''}^\ell]=\SOLVE_{\neg \bb}(n'',\mathcal{G''}^\ell)$. Then, $\mathcal{W}_\bb[\mathcal{G}^\ell]=\SOLVE_{\bb}(n,\mathcal{G}^\ell)$.
\end{theorem}
\end{comment}

\begin{theorem}[Correctness of $\SOLVE_{\bb}$, Alg.~\ref{algo:fair-zielonka-bb}]\label{thm:solvebb}
Assume that for any \Odd-fair parity game $\mathcal{G}'^\ell$ where $n' < n$ is an odd (resp. even) upper bound on the priorities of the game, $SOLVE_\Odd(n', \mathcal{G}')^\ell$ correctly returns the \Odd winning region (resp. $\SOLVE_\Even(n', \mathcal{G}')^\ell$ correctly returns the \Even winning region) in $\mathcal{G}'^\ell$. Then $SOLVE_\bb(n, \mathcal{G}^\ell)$ correctly returns the winning region of player $\bb$ where $n$ is even if $\bb= \Even$ and odd if $\bb = \Odd$.
\end{theorem}

%While the proof of Alg.~\ref{algo:fair-zielonka-odd} follows essentially the proof by Ralf K{\"u}sters \cite{Kuesters2002} of the original Zielonka's algorithm \cite{zielonkas-alg}, the proof of Alg.~\ref{algo:fair-zielonka-even} formalized by Thm.~\ref{thm:solveodd} becomes substantially more complex. First, our instantiation of $\SafeReach^f_\Even(S, R, \mathcal{G}^\ell)$ via \eqref{equ:Xsr2} only computes an \emph{overapproximation} of the safe reachability set $\Xsr_\Even$, and second, we must use \Odd \emph{winning strategy templates} instead of positional winning strategies, to prove a vertex to be winning. While the complete correctness proofs of both algorithms can be found in App.~\ref{app:zielonka-proof}, we give the intuition of Thm.~\ref{thm:solveodd} here, as this is the main contribution of this section. % In order to do so, we first define some preliminaries in Sec.~\ref{sec:zielonka:correctness:prelim}.
% We follow the notation of Ralf K{\"u}sters proof \cite{Kuesters2002} of the original Zielonka's algorithm \cite{Zielonka98}.% Let us first set up some preliminaries.

% \vspace{0.2cm}

\noindent\textbf{Notation.}
We follow the notation of K{\"u}sters' proof \cite{Kuesters2002} of Zielonka's original algorithm \cite{Zielonka98}.
%For the remainer of this section, take $\mathcal{G}^\ell = \ltup{(V, V_\Even, V_\Odd, E, \chi), E^\ell}$. 
Recall that $\mathcal{G}^\ell$ has no dead-ends. For some $X \subseteq V$, we call $\mathcal{G}^\ell[X] = \ltup{(X, X \cap V_\Even, X \cap V_\Odd, X \times X \subseteq E, \chi \mid_X), X \times X \subseteq E^\ell }$ 
a \emph{subgame} of $\mathcal{G}^\ell$ if it has no dead-ends. Here, $\chi\mid_X$ is the priority function $\chi : V \to \mathbb{N}$ restricted to domain $X$. Let $n$ be an upper bound on the priorities in $V$. If the parity of $n$ is even, set $\bb$ to $\Even$; if it's odd, set $\bb$ to \Odd. 

\vspace{0.2cm}

\noindent\textbf{\bb-trap and \bb-paradise.}
A $\bb$-trap is a subset $T \subseteq V$ for $\bb \in \{\Even, \Odd\}$ such that,
$\forall v \in T \cap V_{\nb},\,\, \exists (v, w)\in E \,\,\text{ with } w \in T$ and $\forall v \in T \cap V_{\bb},\,\, (v, w) \in E \implies w \in T$. 
A $\bb$-paradise in $\mathcal{G}^\ell$ is a subset $T \subseteq V$ which is a $\nb$-trap in $V$ and there exists a winning $\bb$ strategy template $(T, \e)$ in $\mathcal{G}^\ell$. %\AKS{don't we need that the vertices contained in this template are precisely $T$.}

\vspace{0.2cm}

The recursive calls of $\SOLVE_\bb$ and $\SOLVE_{\neg \bb}$ on subgames within Alg.~\ref{algo:fair-zielonka-bb} induce a characteristic partition of the game graph. For the correctness proof, 
we need to remember a series of these subgames that are constructed through previous recursive calls. The partition of these subsets is illustrated in Fig.~\ref{fig:kuesters-figure-extended} and formalized as follows.
% 
% % Figure environment removed

\vspace{-0.4cm}

\begin{align}\label{equ:seriesZielonka}
    &X_\bb^i := V \setminus X_\nb^i \quad \quad \quad &&N^i:= \{v \in X^i_\bb \mid \chi(v) = n\}\\
    &Z^i:= X^i_\bb \setminus \SafeReach^f_\bb(X^i_\bb, N^i, \mathcal{G}^\ell) \quad &&X^{i+1}_\nb :=  \SafeReach^f_\nb(V, X_\nb^{i} \cup Z_\nb^{i}, \mathcal{G}^\ell)\nonumber % X_\Even^{i} \cup \SafeReach_\Even^f(X^{i}_\Odd, Z_\Even^{i}, \mathcal{G}^\ell) )%\text{\todo{IS: I know the equality is not completely justified. The first one is cheaper for an algorithm pov, whereas the second one is easier to justify that $X^i_\Odd$ is an \Even-trap.}} 
\end{align}

%\vspace{-0.3cm}

where, in addition $Z_\bb^i$ is the \bb winning region in the subgame $\mathcal{G}^\ell[Z^i]$. Intuitively, the sets constructed in \eqref{equ:seriesZielonka} correspond to the sets with the same name within Alg.~\ref{algo:fair-zielonka-bb}.
    
We collect the following observations on these sets, which are proven in App.~\ref{app:zielonka-proof}. %and mimic the corresponding properties in the proof of the original Zielonka's proof \cite{Kuesters2002}.
\begin{enumerate}\label{it:zlk-observations}
\item  \textbf{(App. - Obs.~\ref{app-obs:traps-subgames})} $X^i_\nb$ is an \bb-trap, $X^i_\bb$, $Z^i$ and $Z_\bb^i$ are \nb-traps in $V$. $Z^i$ is in \nb-trap in $X_\bb$ and $Z_\nb^i, Z_\bb^i$ are \bb- and \nb-traps in $Z^i$, respectively. Therefore, $\mathcal{G}^\ell[Y]$ is a subgame of $\mathcal{G}^\ell$ with $Y$ being any of these sets.\label{it:obs1} %(see Obs.~\ref{obs:traps-subgames} in App.~\ref{}).
 \item \textbf{(App. - Lem.~\ref{app-lem:X_nb-equivalence})} $X_\nb^{i} \cup \SafeReach_\nb^f(X^{i}_\bb, Z_\nb^{i}, \mathcal{G}^\ell) =  \SafeReach_\nb^f(V, X_\nb^{i} \cup Z_\nb^{i}, \mathcal{G}^\ell)$.\label{it:obs2}%(see Lem.~\ref{lem:X_nb-equivalence} in App.~\ref{}).
 \item \textbf{(App.  - Cor.~\ref{app-cor:increasing-decreasing-sequences})} As a consequence of the previous item, $\{X_\nb^{i}\}_{i\in \mathbb{N}}$ is an increasing sequence. Consequently, $\{X_\bb^{i}\}_{i\in \mathbb{N}}$ is a decreasing sequence. As $V$ is finite, this immediately implies that these sequences reach a saturation value for some, and in fact the same, $k$. \label{it:obs3}
 \item \textbf{(App.  - Lem.~\ref{app-lem:safe-reach-Odd-paradise})} If $R \subseteq V$ is an \Odd-paradise in $\mathcal{G}^\ell$, then $\SafeReach^f_\Odd(V, R, \mathcal{G}^\ell)$ is also an \Odd-paradise in $\mathcal{G}^\ell$.\label{it:obs4}
 \item \textbf{(App.  - Lem.~\ref{app-lem:safereacheven-noliveedges})} The set $U \setminus \SafeReach_\bb(U, R, \mathcal{G}^\ell)$ is a $\bb$-trap in $U$. \label{it:obs5}
\end{enumerate}

\vspace{0.1cm}

In contrast to Zielonka's original algorithm, the proof of the procedures $\SOLVE_\Even$ and $\SOLVE_\Odd$ is not identical in \Odd-fair Zielonka's algorithm. This is due to the different safe-reachability set constructions used. Next we sketch the correctness proof of Thm.~\ref{thm:solvebb} for $\bb:=\Odd$, corresponding to the correctness of procedure $\SOLVE_\Odd$. The proof for $\bb:=\Even$ is left to the appendix, as it resembles the proof Zielonka's original algorithm more.
% % \subsubsection{Correctness of $\SOLVE_\Odd$ --  Thm.~\ref{thm:solvebb}}\label{sec:zielonka:correctness:odd}

\begin{proposition}\label{prop:n-odd}
Given the premisses of Thm.~\ref{thm:solvebb} for $\bb = \Odd$, if $Z_\Even^k = \emptyset$ then $\SafeReach^f_\Odd(V, X^k_\Odd, \mathcal{G}^\ell)$ is an \Odd-paradise and $V \setminus \SafeReach^f_\Odd(V, X^k_\Odd, \mathcal{G}^\ell)$ is an \Even-paradise in $\mathcal{G}^\ell$.
\end{proposition}

Within Prop.~\ref{prop:n-odd}, the fact that $Z_\Even^k = \emptyset$ refers to the termination of the recursive call in Alg.~\ref{algo:fair-zielonka-bb} which results in the saturation of the sequence $\{X_\Odd^i\}_{i\in \mathbb{N}}$ with $X_\Odd^k$. This implies that $\SOLVE_\Odd$ returns 
$ T:=\SafeReach^f_\Odd(V, X^k_\Odd, \mathcal{G}^\ell) $, which is an \Odd-paradise and $V \setminus T$ an \Even-paradise. With this, Thm.~\ref{thm:solvebb} follows from Prop.~\ref{prop:n-odd} for $\bb=\Odd$. % Alg.~\ref{algo:fair-zielonka-bb} for $\bb = \Odd$.
We now give a proof sketch of Prop.~\ref{prop:n-odd}.

We first recall from observation~\ref{it:obs1} that $T$ and $V\setminus T$ are \Even- and \Odd-traps in $V$, respectively. In order to prove Prop.~\ref{prop:n-odd}, it remains to show that there exists an \Odd (resp. \Even) strategy template which is winning in $\mathcal{G}^\ell$ and maximal on $T$ (resp. $V\setminus T$). We next give the construction of these templates and a high-level intuition on why they are actually \emph{winning}. 

\smallskip
\noindent\textbf{Winning \Odd Strategy Templates.} 
As $X^k_\Odd$ is known to be an \Even-trap, it can be proven to be an \Odd-paradise by constructing a winning maximal strategy template on it. It then follows from observation~\ref{it:obs4} that $T$ is also an \Odd-paradise.

Towards a construction of a maximal winning \Odd strategy template on $X_\Odd$, we first observe that $X^k_\Odd=Z_\Odd^k\cup \SafeReach^f_\Odd(X^k_\Odd, N^k, \mathcal{G}^\ell)$ (as $Z_\Even^k=\emptyset$). % Now we first consider $Z^k = Z_\Odd^k$ (as $Z_\Even^k=\emptyset$)\todo{IS: why do we 'consider' this, isn't this given for $k$?}
 Then there exists a maximal winning \Odd strategy template $z$ on $Z^k = Z_\Odd^k$ in game $\mathcal{G}^\ell[Z^k]$. % and the definition of $Z_\Odd^k$. 
 Any play $\pi$ compliant with $z$ that starts and stays in $Z^k$ is clearly \Odd winning.
However, $z$ is not necessarily an \Odd strategy template in $\mathcal{G}^\ell$ since there are possibly some $(v,w) \in E$ with $v \in Z^k \cap V_\Even$  and $w \not \in Z^k$.
For all such edges, $w \in \SafeReach^f_\Odd(X^k_\Odd, N^k, \mathcal{G}^\ell)$ since $X^k_\Odd$ is an \Even-trap in $V$.
% 
For the state set $\Xsr_\Odd:=\SafeReach^f_\Odd(X^k_\Odd, N^k, \mathcal{G}^\ell)$, recall from Sec.~\ref{sec:zielonka:fair} that there exists partial strategy template $sr$ defined on $\Xsr_\Odd$ with dead ends in $N^k$.

Using the templates $z$ and $sr$, we can construct a maximal candidate \Odd strategy template on $X^k_\Odd$. Following the intuition behind the construction of $\mathcal{S}^{\mathcal{G}^\ell}$ in Def.~\ref{def:S}, we first define a base subgraph $(X^k_\Odd,\e)$ with $\e\subseteq E$ s.t.\
 $(v,w) \in E $ is in $\e$ if either (i) $(v,w) \in z \cup sr$, (ii) $v \in V_\Even \cap X^k_\Odd$, or (iii) $v \in N^k \cap V_\Odd$ and $w = v_r$
where $v_r$ is a random fixed successor of $v$, that is in $X^k_\Odd$.
Such a successor is guaranteed to exist since $X^k_\Odd$ is an \Even-trap.
We now extend the subgraph $(X^k_\Odd,\e)$ to an \Odd strategy template by adding all live edges originating in vertices  $X^k_\Odd\cap V^\ell$ that lie on a cycle in $\e$, similar to Def.~\ref{const:S} (S3)-(S4). This results in a subgraph $\Sc=(X^k_\Odd,\overline{\e})$ %where $\overline{e}$ is defined to be the saturation value of the sequence $\overline{e}^j = \overline{e}^{j-1} \cup \{(v, w) \in V^\ell \mid v \text{ lies on a cycle in } (X_\Odd^k, \overline{e}^{j-1})\}$ where $\overline{e}^0 = e$.
that is a maximal \Odd strategy template. %\AKS{Why don't we need $\Sc$ to be maximal on $T$?}
% 
The underlying idea behind $\mathcal{S}$ being winning %(formally proven in App.~\ref{app:zielonka-proof}) 
is the following: Any play that starts in $X_\Odd^k$ either stays in $Z^k$ after some point and is won by $\mathcal{S}$ collapsing to $z$, or sees a newly added cycle (one that is not in $z \cup sr$) infinitely often. All such cycles contain a newly added edge. An analysis of newly added edges reveal that, 
all of them -- when seen infinitely often -- eventually drag a play towards $N^i$. Thus, every play that sees a new cycle infinitely often sees $n$ infinitely often, and thus won by \Odd.

\smallskip

\noindent\textbf{Winning \Even Strategy Templates.} 
Here we show that $V \setminus T$ is an \Even-paradise in $\mathcal{G}^\ell$. 
We first define $\Xsr^i_\Even:=\SafeReach^f_\Even(X_\Odd^i, Z_\Even^i, \mathcal{G}^\ell)$ and denote by $sr^i$ the partial \Even strategy template defined on $\Xsr^i_\Even$. We further denote the winning \Even strategy on $Z_\Even^i$ in game $\mathcal{G}^\ell[Z^i]$ by $z^i$. 
% 
We can now construct the \Even strategy template $\mathcal{S} = (V \setminus T, \e)$ where $\e$ is the combination of edges in $sr^i \cup z^i$ with $\{(v,w) \in E \mid v \in V_\Odd \cap (V \setminus T)\}$.
Since $V\setminus T$ is an \Odd-trap by observation~\ref{it:obs5}, the edge set $\e$ stays within $V \setminus T$, i.e. $\e \subseteq V\setminus T \times V \setminus T$. Then clearly, $\mathcal{S}$ is an \Even strategy template.
To see $\mathcal{S}$ is winning we first observe that each $v \in V \setminus T$ there exists a unique $i<k$ such that $v \in \Xsr^i_\Even$. Let $\pi = v_1 v_2 \ldots$ be a play compliant with $\mathcal{S}$ and let $s = \Xsr_1 \Xsr_2 \ldots$ be the sequence such that $v_i \in \Xsr$.
(1) If $v_j \in Z_\Even^i$, $v_{j+1} \in Z_\Even^i \cup \{\Xsr_\Even^r \mid r < i\}$. This follows from $Z_\Even^i$ being an \Odd-trap in $X_\Odd^i$.
(2) If $\pi$ visits $v \in \Xsr^i$ infinitely often, $\pi$ visits $Z_\Even^i$ infinitely often: This is because $\pi$ visits the $(v,w)$ in $\mathcal{S}$ that makes positive progress towards $Z_\Even^i$ infinitely often as well. 
% 
Let $i$ be the minimum index such that $\Xsr_\Even^i$ is seen infinitely often in $s$. By (1), $\pi$ visits $Z_\Even^i$ infinitely often and by (1) and the minimality of $i$, it should eventually stay in $Z_\Even^i$.
Thus $\mathcal{S}$ eventually collapses to $z_\Even^i$ on $\pi$ and the play is won by \Even.

\vspace{-0.2cm}
%\smallskip
%\noindent \textbf{The Algorithm.} Observe that $\SOLVE_\Odd(n, \mathcal{G}^\ell)$ (Alg.~\ref{algo:fair-zielonka-odd}) calculates the sets as given in the construction(Fig.~\ref{fig:kuesters-figure-extended}) where $X$ holds the value of $X_\Odd^i$ at the end of the $i^{th}$ iteration of it's \emph{while} loop. $\{X_\Odd^i\}_{i \in \mathbb{N}}$ is a decreasing sequence which saturates at some $X_\Odd^k$ where $Z_\Even^k = \emptyset$.
%$\SOLVE_\Odd(n, \mathcal{G}^\ell)$ returns  $\SafeReach(V, X_\Odd^k, \mathcal{G}^\ell)$, which is shown to be equal to \Wo by Prop.~\ref{prop:n-odd}. 

%Similarly, $X$ in $\SOLVE_\Even(n, \mathcal{G}^\ell)$ (Alg.~\ref{algo:fair-zielonka-even}) holds the value of $X_\Even^i$ after the $i^{th}$ iteration. The constructive proof of Thm.~\ref{thm:solveeven} (App.~\ref{app:zielonka-proof}) shows that the saturation value $X^{k'}_\Even$ (where $Z^{k'}_\Odd = \emptyset$) is equal to \We, and this is exacly the value $\SOLVE_\Even(n, \mathcal{G}^\ell)$ returns.
%
