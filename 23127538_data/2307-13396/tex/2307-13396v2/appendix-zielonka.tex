\subsection{Zielonka's Algorithm for \Odd-Fair Parity Games}\label{app:zielonka-proof}

% This section is intended to provide a detailed proof of Thm.~\ref{thm:solvebb}.
% 
% We refer the reader to Sec.~\ref{sec:zielonka} for the introduction and motivation of the proofs. We will not repeat the definitions of safe reachability sets and partial strategy templates in this section. Once more we refer the reader to Sec.~\ref{sec:zielonka} for these preliminaries. 
% That said,  we will not follow the same lay-out on the proofs as in Sec.~\ref{sec:zielonka}. Moreover, we will present new definitions and lemmas liberally. So this section should be perceived as \emph{somehow} stand-alone, with the exception of the aforementioned dependencies. 
% The proof roughly follows the foot steps of ~\cite{Kuesters2002}.% We will cite some lemmas from this work and leave them unproven if the proof is unaffected by the fairness condition. 

This section provides a detailed proof of Thm.~\ref{thm:solvebb}.
% 
However, we will not follow the lay-out given for this proof in Sec.~\ref{sec:zielonka} but rather follow the foot steps of the correctness proof of the \enquote{normal} Zielonka's algorithm from  ~\cite{Kuesters2002}. Hence, this section should be perceived as stand-alone, with the exception of the definitions of safe reachability sets and partial strategy templates, which can be found in Sec.~\ref{sec:zielonka}. While we do not follow the same lay-out, the motivation and intuition given for the proof in Sec.~\ref{sec:zielonka} still carries over to this section.


% \subsubsection{Preliminaries} 
% 
%We start by extending the strategy template definition given in section 4 to \Even strategies (just \Even and \Odd are swapped in the definition). Note that, since \Even nodes do not have live outgoing edges, for all $v \in V'_\Odd$, $E(v) \subseteq E'$ and for all $v \in V'_\Even$, $|E'(v)| = 1$. Consequently, a positional \Even strategy $\rho$ is equal to the unique strategy compliant with the strategy template $(V', V'_\Even, V'_\Odd, E')$ for which $E'(v) = \rho(v)$ for all $v \in V'_\Even$. We use a positional \Even strategy $\rho$ to define a strategy template for \Even.
%To distinguish strategy templates defining \Odd strategies from the ones defining \Even strategies, we denote \Odd strategy templates by $\mathcal{S}^\Odd$ and \Even ones with $\mathcal{S}^\Even$. 
% 
%Furthermore, we obtain a partial strategy template by taking a strategy template and a subset $V'$ of its vertex set, and removing all the outgoing edges of all vertices in $V'$. \IS{?}
%Let $a \in \{\Even, \Odd\}$ and $\mathcal{S}^a = (V', V'_\Even, V'_\Odd, E')$ be an $a-$strategy template. We obtain a partial strategy template as follows: Let $V'' \subseteq V'$ be such that, for all $v \in V''$, $v$ does not lay on a cycle in $E'$. Now for all $v \in V''$, remove all outgoing edges of $v$ from $E'$ to obtain $E''$. Then $(V', V'_\Even, V'_\Odd, E'')$ is a partial strategy template for $a$. We denote a partial strategy template of $a$ by $\mathcal{P}^a$.

\subsubsection{Preliminaries} 
% This section proves important observations w.r.t.\ the \bb paradises in \odd-fair parity games. In order to do so, we restate the definitions of an $\bb-$trap (Def.~\ref{def:atrap}) and a  \bb paradise (Def.~\ref{}) from Sec.~\ref{sec:zielonka}. We also estate three lemmas (Lem.~\ref{app-lem:Kuesters6.2}, \ref{app-lem:Kuesters6.3} and \ref{app-lem:Kuesters6.4}) exactly as they appear in \cite{Kuesters2002} and omit the proofs since the statements of these lemmas are only concerned with the properties of the subsets of $V$, and are therefore unaffected by the fairness condition.
We emphasize again that we assume the underlying game graph of the fair parity game $\mathcal{G}^\ell$ to be deadend-free.

\smallskip
\noindent\textbf{Subgames.}
For some $U \subseteq V$ we denote by $E \mid_U = \{(v, w) \in E \mid v, w \in U \}$ and by $\chi\mid_U$ we denote the restriction of the function $\chi$ to the domain $U$. %With this, we formally define subgames as follows.

%We call a $\mathcal{P} = (V', E')$ a \emph{partial strategy template in $\mathcal{G}^\ell$} if it is a strategy template, with the exception of some dead-ends. That is, $\mathcal{P}$ is a partial \Odd (\Even) strategy template if for each $v \in V'$, either $E'(v) = \emptyset$ or $v$ satisfies~\ref{item:Oddstrtemprules} (\ref{item:Evenstrtemprules}).

\begin{definition}[Subgames]
    Let $U \subseteq V$. The subgraph of $\mathcal{G}^\ell$ induced by $U$ is shown as $\mathcal{G}^\ell[U]$ and is the restriction of the game graph to $U$, i.e.
    $\mathcal{G}^\ell[U] = \ltup{ \langle U, \Ve \cap U, \Vo \cap U, E|_U, \chi|_U\rangle, E^\ell|_U} $. $\mathcal{G}^\ell[U]$ is a subgame of $\mathcal{G}^\ell$ if and only if $\mathcal{G}^\ell[U]$ is deadend-free.
\end{definition}

\begin{lemma}[\cite{Kuesters2002}, Lemma 6.2]\label{app-lem:Kuesters6.2}
    If $U, U' \subseteq V$ where $\mathcal{G}^\ell[U]$ is a subgame of $\mathcal{G}^\ell$ and $(\mathcal{G}^\ell[U])[U']$ is a subgame of $\mathcal{G}^\ell[U]$, then $\mathcal{G}^\ell[U']$ is a subgame of $\mathcal{G}^\ell$.
\end{lemma}

The above lemma (as well as the following two lemmas  \ref{app-lem:Kuesters6.3} and \ref{app-lem:Kuesters6.4}) are restated exactly as they appear in \cite{Kuesters2002}. We omit their proofs since the statements of these lemmas are only concerned with the properties of the subsets of $V$, and are therefore unaffected by the fairness condition.

%\begin{definition}[Safe Reachability Game] A safe reachability game on game $\mathcal{G}$ is given by the following LTL formula
%    $$ \alpha^{sr} = \square \,\, p_s \,\wedge \,\diamond \,\,p_r $$
%    where $S$ and $R$ are two subsets of $V$, called the safety and reachabilty sets respectively. For $v \in V$, $v \models p_s$ iff $v\in S$ and $v \models p_r$ iff $v\in R$. 
%    $\SafeReach_\Even(S, R, \mathcal{G})$ denotes the winning region of \Even in this game and  $\SafeReach_\Odd(S, R, \mathcal{G})$ denotes the winning region of \Odd.
%\end{definition}
%Given a game graph $G = (V, V_\Even, V_\Odd, E)$, reachability set $R\subseteq V$ and safety set $S \subseteq V$, the following fixed-point formula computes the winning region for player $\bb$ where $\bb \in \{\Even, \Odd\}$ \cite{banerjee2022fast}:
%$$ \SafeReach_\bb(S, R, \mathcal{G}) = \mu X. (p_s \wedge (p_r \vee \\Cpre_\bb(X))) $$
%Note that there is an obvious positional strategy for player $\bb$, which makes it progress towards $R$ while staying in $S$ from each vertex in $\SafeReach_\bb(S, R, \mathcal{G}) \setminus R$. This
%strategy can be obtained by assigning a ranking (where
%\newline $rank: \SafeReach_\bb(S, R, \mathcal{G}) \to \mathbb{N}^+$ to each node $ v \in \SafeReach_\bb(S, R, \mathcal{G})$ that represents the minimum iteration of the least fixed-point variable $X$, in which they got added into the formula. That is, 
%$rank(v) = k $ iff $ v \in X^k \setminus X^{k-1}$. 
%Then, the strategy of player $\bb$ is simply to take the minimum ranked successor from each $v \in V_\bb \cap \SafeReach_\bb(S, R, \mathcal{G}) \setminus R$. Note that $rank(v) = 1$ iff $v\in R$.
%This strategy defines a partial strategy template $\mathcal{P}^\bb = \tup{V', E' }$ where $V' = SafeReach_\bb(S, R, \mathcal{G})$ and $v \in R$ are dead-ends. For all $v \in V'_\bb$, $E'(v)$ only contains a minimum ranked successor of $v$ and for all $v \in V'_{\nb}$, $E'(v) = E(v)$. Note that this partial strategy template gives out a minimum-rank strategy for player \bb. Additionally, $\mathcal{P}^\bb$ contains no cycles, since in a play $\pi = v_0 v_1 \ldots v_k$ compliant with $\mathcal{P}^\bb$, where $v_0 \in SafeReach_\bb(S,R,\mathcal{G}^\ell) \setminus R$ and $v_k \in R$; $rank(v_{i+1})< rank(v_i)$ for all $i \in [0,k-1]$ in order for the play to be able to eventually reach $R$. This is the case for all strategies of $\nb$, whereas $\bb$ nodes only have one outgoing edge. This is only possible in case $\mathcal{P}^\bb$ is cycle-free.

%On a fair game graph $\mathcal{G}^\ell := \ltup{\mathcal{G}, E^\ell}$, the safe reachability set of \Odd is calculated by the same formula as in regular games; hewever, the safe reachability set of \Even requires a 2-nested fixed-point formula:
%\begin{align*}
%& SafeReach^l_\Odd(S, R, \mathcal{G}^\ell) = \mu X. (p_s \wedge (p_r \vee \\Cpre_\Odd(X)))\\
%& SafeReach^l_\Even(S, R, \mathcal{G}^\ell) = \nu Y \mu X. (p_s  \wedge (p_r \vee \Apre(Y,X)))\\
%\end{align*}
%The change is due to the increased power of \Even in \Odd-fair games. Namely, \Even can force a $v\in V_\Odd$ into a region $R$ in the next step not only if all the outgoing edges of $v$ land in $R$, but also if $v$ has a live edge to $R$ and the \Odd strategy has a cycle passing through $v$.
%$SafeReach^l_\Even(S, R, \mathcal{G}^\ell)$ does the following with it's nested fixed-point calculation:
%It begins by setting $Y^0:= V$ and $X^{0,0} := \emptyset$. Then calculates all nodes that can be forced to reach $R$, assuming all $v \in V^\ell$ have a cycle passing through them (this is because $\\Apre(V,X)$ adds $v \in V^\ell$ to $X^{0, i}$ if it has a live outgoing edge to $X^{0, i-1}$).
%The saturation value of $X^{0,k}$ is set to $Y^1$. $Y^1$ contains elements that can reach $R$, if all $V^\ell$ had lied on a cycle, and therefore had to eventually take their live edges. In the next iteration, the formula calculates the set of nodes that can be forced by \Even to reach $R$, assuming that for all $v \in V^\ell$, if $v$ has an outgoing edge to
%$V \setminus Y^1$, it takes this outgoing edge and avoids $R$ (so it wins by a 1-step strategy from $v$ and therefore, $v$ does not lie on a cycle the \Odd strategy), and if it does not have such an edge, then we assume it lies on a cycle and it is included in $X^i$ if it has a live outgoing edge leading to $X^{i-1}$.
%The saturation value of $X^{1, k}$ is set to $Y^2$ and the calculation is repeated until the saturation of the $Y$ variable. Once $Y$ has saturated, all nodes that are included in $Y$ are those that can be forced by \Even to reach $R$. If $v \in V^\ell \cap Y$ then 
% all outgoing edges of $v$ lead to $Y$. \IS{Anne: Maybe a better explanation instead of the last sentence?}

% In \Odd-fair Zielonka algorithm, we will use the safe reachability sets for \Even and \Odd denoted by $SafeReach^f_\Even$ and $SafeReach^f_\Odd$. The \Odd safe reachability set will remain as before, i.e. $SafeReach^f_\Odd(S, R, \mathcal{G}^\ell) := SafeReach^l_\Odd(S, R, \mathcal{G}^\ell)$.
%However, we will use a simpler variant of the \Even safe reachability set. In this variant intuitively, we act as if all the $v \in V^\ell$ have a cycle passing through them, and add them to the \Even safe reachability set if there is a live edge $(v,w)$ that has positive progress towards $R$.
% \begin{equation}\label{app-eq:SafeReachEven}
% SafeReach^f_\Even(S, R, \mathcal{G}^\ell) = \mu X. (p_s  \wedge (p_r \vee \Cpre_\Even(X) \vee Lpre^\exists(X)))
%% \end{equation}
% That is, $v \in X^i$ if $\exists (v,w)$ with $w \in X^{i-1}$. Using this fixed-point formula to calculate the \Even reachability sets allows us to calculate an over-approximation, say $W$, of the \Even winning region $\We$ at the end of the algorithm. The complement of $W$ is an under approximation of $\Wo$ that contains no live edges leading to $W$.
% Then we show that $\Wo$ is exactly $SafeReach^f_\Odd(V, V \setminus W, \mathcal{G}^\ell)$ and $\We$ is the complement of this set. 
%\IS{some connecting sentence}
 %Let's give some more preliminaries before moving to the proof.
 
\smallskip
\noindent\textbf{\bb-Trap.} We restate the definition of a $\bb$-trap from Sec.~\ref{sec:zielonka}. and subsequently show important observations w.r.t.\ $\bb$-traps in \Odd-fair parity games.
 
\begin{definition}[\bb-trap]\label{def:atrap}
A $\bb$-trap is a subset $T \subseteq V$ for $\bb \in \{\Even, \Odd\}$ such that,
\begin{align*} &\forall v \in T \cap V_{\nb}, \quad \exists (v, w)\in E \text{ with } w \in T, \\
   & \forall v \in T \cap V_{\bb}, \quad \, (v, w) \in E \implies w \in T.
 \end{align*}
\end{definition}

%So any strategy $\rho$ that is compliant with a strategy template $\mathcal{S}^\Odd$, for which 

% Now we will give some lemmas, the proofs of which can be found in \cite{Kuesters2002} and the fairness condition do not change the proofs. For the following, let $\mathcal{G}^\ell = \tup{\mathcal{G}, E^\ell}$ be an \Odd-Fair Parity game. 

\begin{lemma}[\cite{Kuesters2002} Lemma 6.3]\label{app-lem:Kuesters6.3}
    \begin{enumerate}
        \item For every $\bb$-trap $U$ in $\mathcal{G}^\ell$, $\mathcal{G}^\ell[U]$ is a subgame.
        \item If $X$ is a $\bb$-trap in $\mathcal{G}^\ell$ and $Y\subseteq X$ is a $\bb$-trap in  $\mathcal{G}^\ell[X]$, then $Y$ is a $\bb$-trap in $\mathcal{G}^\ell$.
    \end{enumerate}
\end{lemma}

%\noindent For the following, let $U \subseteq V$  such that $\mathcal{G}^\ell[U]$  is a subgame of $\mathcal{G}^\ell$.
\begin{lemma}[\cite{Kuesters2002}, Lemma 6.4 -- Sec.~\ref{sec:zielonka:correct} Obs.~\ref{it:obs5}]\label{app-lem:Kuesters6.4}
 The set $U \setminus \SafeReach^f_\bb(U, R, \mathcal{G}^\ell)$ is a $\bb$-trap in $U$.
\end{lemma}
\begin{lemma}\label{app-lem:safereacheven-noliveedges} 
    Let $W = U \setminus \SafeReach^f_\Even(U, R, \mathcal{G}^\ell)$. There exists no $(v,w) \in E^\ell$ with $v \in W$ and $w \in \SafeReach^f_\Even(U, R, \mathcal{G}^\ell)$.
\end{lemma}
\begin{proof}
%(1.) For $a= \Odd$, the proof is the same as the original proof. Let $a= \Even$. Assume $V \setminus \SafeReach^f_\Even(U, R, \mathcal{G}^\ell)$ is not an $\Even-$trap. Then there is either a $v \in V_\Even|_U \setminus \SafeReach^f_\Even(U, R, \mathcal{G}^\ell) $ such that for all $(v,w) \in E|_U$, $w \in \SafeReach^f_\Even(U, R, \mathcal{G}^\ell)$;
%or, there exists a $v \in V_\Odd|_U \setminus \SafeReach^f_\Even(U, R, \mathcal{G}^\ell) $ such that there exists a $(v, w) \in E|_U$ like that. For both of these cases, $v \models (p_U \vee \Cpre_\Even(\mathbf{X}))$ where $\mathbf{X} = \SafeReach^f_\Even(U, R, \mathcal{G}^\ell)$, as it is the value it gets in its saturation. But this requires $v$ to be in $\SafeReach^f_\Even(U, R, \mathcal{G}^\ell)$ as well, through the definition of $\Apre$ and the fixed-point calculation. Thus, no such $v$ exists and the set is an $a-$trap.\\
%(2.) For $a= \Even$, the proof  is the same as the original proof. Let $a= \Odd$. Assume that the given set is not an \Odd-trap. Then either there exists a $v \in V_\Odd|_{U} \cap \SafeReach^f_{Even}(U, R, \mathcal{G}^\ell)$ such that $\exists (v,w) \in E|_{U}$ with $w \in U \setminus \SafeReach^f_\Even(U, R, \mathcal{G}^\ell)$, or there exists a $v \in V_\Even |_U \cap \SafeReach^f_{Even}(U, R, \mathcal{G}^\ell)$ such that all $(v,w) \in E|_{U}$ satisfy that. For both of these cases, $w \not \in \mathbf{X}, \mathbf{Y} = ^f_\Even(U, R, \mathcal{G}^\ell)$, for the saturation values of the fixed-point variables. Thus, $v \not \models \Apre(\mathbf{Y}, \mathbf{X})$. I.e. $v \not \in \SafeReach^f_\Even(U, R, \mathcal{G}^\ell)$.\\
A node $v \in U \setminus \SafeReach^f_\bb(U, R, \mathcal{G}^\ell) \cap V_\bb$ cannot have an edge that leads to $\SafeReach^f_\bb(U, R, \mathcal{G}^\ell)$, since then $v$ itself must be in this set.
Similarly a node  $v \in U \setminus \SafeReach^f_\bb(U, R, \mathcal{G}^\ell) \cap V_{\nb}$ must have an edge that leads to $ U \setminus \SafeReach^f_\bb(U, R, \mathcal{G}^\ell)$, or else $v$ would be in  $\SafeReach^f_\bb(U, R, \mathcal{G}^\ell)$.
\end{proof}

\begin{lemma}\label{app-lem:SafeReachOdd_of_an_even_trap_is_an_even_trap}
    If $R$ is an \Even-trap in $U$, then so is $\SafeReach^f_{\Odd}(U, R, \mathcal{G}^\ell)$.
\end{lemma}
\begin{proof}
    This is easy to observe from the definition of a partial strategy template $sr_\Odd$ on $\SafeReach^f_\Odd(U, R, \mathcal{G}^\ell)$.
    All $(v, w) \in E$ with $v \in V_\Even \cap \SafeReach^f_\Odd(U, R, \mathcal{G}^\ell)\setminus R$, are in $sr_\Odd$. That is, $w \in \SafeReach^f_\Odd(U, R, \mathcal{G}^\ell)$. For all $v \in V_\Even \cap R$, all $(v,w) \in E\subseteq U \times U$ are in $R$ since $R$ is an \Even-trap in $U$. 
    Thus for all \Even nodes in $\SafeReach^f_\Odd(U, R, \mathcal{G}^\ell)$, all their successors in $U$ are in the set again. 
    We can similarly observe that for all $v \in V_\Odd \cap \SafeReach^f_\Odd(U, R, \mathcal{G}^\ell)$ they have at least one successor in the set. 
    Thus this set is an \Even-trap in $U$. 
\end{proof}

\smallskip
\noindent\textbf{\bb-Paradise.} We restate the definition of a $\bb$-paradise from Sec.~\ref{sec:zielonka} and subsequently show important observations w.r.t.\ $\bb-$paradises in \Odd-fair parity games.

\begin{definition}[$\bb$-paradise]\
A $\bb$-paradise of an \Odd-fair parity game $\mathcal{G}^\ell$ is a region $P \subseteq V$ from which player $\nb$ cannot escape (i.e. $P$ is a $\nb$-trap) and player $\bb$ has a strategy to win from all $v\in P$. As we have proven in section 5, this implies that there exists a strategy template $\mathcal{S}^\bb$ with the vertex set $P$ such that all player $\bb$ strategies compliant with $\mathcal{S}^\bb$ are winning for player $\bb$.

Formally $P \subseteq V$  is a $\bb$-paradise if:
\begin{itemize}
\item $P$ is a $\neg \bb$-trap and, 
\item There exists a winning \bb strategy template $\mathcal{S}^\bb = \ltup{P, E'}$ on $\mathcal{G}^\ell$.
\end{itemize}
\end{definition}
Note that if $P$ is a $\bb$-paradise, and play $\pi$ starting in $P$ and is compliant with $\mathcal{S}^a$, stays in $P$ and is won by \bb.

\begin{lemma}[Sec.~\ref{sec:zielonka:correct} Obs.~\ref{it:obs4}]\label{app-lem:safe-reach-Odd-paradise}
    If $R\subseteq V$ is an \Odd-paradise in $\mathcal{G}^\ell$, then $\SafeReach_\Odd^f(V, R, \mathcal{G}^\ell)$ is also an \Odd-paradise in $\mathcal{G}^\ell$.
\end{lemma}
\begin{proof}
Due to Lem.~\ref{app-lem:SafeReachOdd_of_an_even_trap_is_an_even_trap}, $\SafeReach_\Odd^f(V, R, \mathcal{G}^\ell)$ is an \Even-trap in $V$.
The winning \Odd strategy template on it is just a combination of the winning \Odd strategy template $\mathcal{S}$ on $R$ and the partial \Odd strategy template $sr_\Odd$ on $\SafeReach^f_\Odd(V, R, \mathcal{G}^\ell)$, on which nodes in $R$ are dead-ends and  
all $v \in \SafeReach^f_\Odd(V, R, \mathcal{G}^\ell) \setminus R$ are guaranteed to reach $R$ in finitely many steps. %Remember that $sr_\Odd$ is acyclic. 
Let $\e$ be the combination of edges in $sr_\Odd$ and $\mathcal{S}$. 
Since $R$ is an \Even-trap in $V$, all outgoing edges of \Even nodes in $R$ stay in $R$. All outgoing edges of \Even nodes in 
$\SafeReach^f_\Odd(V, R, \mathcal{G}^\ell) \setminus R$ are in $sr_\Odd$. Therefore all outgoing edges of \Even nodes in $\SafeReach^f_\Odd(V, R, \mathcal{G}^\ell)$ are in $\e$.
It's easy to see $\e$ introduces no new cycles to $sr_\Odd \cup \mathcal{S}$. Therefore $\mathcal{S}' = 
(\SafeReach^f_\Odd(V, R, \mathcal{G}^\ell), \e)$ is an \Odd strategy template in $\mathcal{G}^\ell$.
$\mathcal{S}'$ is winning because any play starting in $\SafeReach^f_\Odd(V, R, \mathcal{G}^\ell)\setminus R$ reaches $R$ in finitely many steps and from there on stays in $R$. 
Since from that point on $\mathcal{S}'$ collapses to $\mathcal{S}$, the game is won by \Odd. 

\end{proof}
\begin{corollary}\label{app-cor:determinacy}
    For an \Odd-fair parity game $\mathcal{G}^\ell$, $V$ is partitoned into an \Even-paradise and an \Odd-paradise. 
\end{corollary}
The corollary follows from the fixed-point equations~\eqref{eq:fp-even} and~\eqref{eq:fp-odd}. Winning region of player $\bb$ is by definition a $\bb$-paradise. \We is the \Even-paradise with the strategy template defined by the positional strategy acquired from the fixed-point formula in~\eqref{eq:fp-even}. The calculation of the positional strategy is closely related to the ranking function and strategy template computation in Sec.~\ref{sec:strat-templates}, and a brief introduction of the calculation can be found in \cite{banerjee2022fast}.
$\Wo = V \setminus \We$ is the \Odd-paradise. The calculation of the strategy template for \Odd is given in Section 5. 


\subsubsection{Computing Winning Regions $\mathcal{W}_\bb$}
Now we will give a construction to calculate $\Wo$ and $\We$ in $\mathcal{G}^\ell$. The construction corresponds to the \Odd-fair Zielonka's algorithm given in Alg.~\ref{algo:fair-zielonka-bb}.
We will give the construction in two parts. First we will take an \Odd-fair parity game $\mathcal{G}^\ell$ and an \emph{odd} integer $n$ where $n$ is an upper bound on the priorities seen in the vertex set of $\mathcal{G}^\ell$. Then we will show how to obtain $\Wo$ and $\We$ in $\mathcal{G}^\ell$ in the existence of a procedure
that can do the same on a subgame $\mathcal{G}^\ell[X]$ of $\mathcal{G}^\ell$ where $n-1$ is an upper bound of the priorities seen in $\mathcal{G}^\ell[X]$. 
In the second part we will show the same for $\mathcal{G}^\ell$ with an \emph{even} $n$. The combination of these two procedures with a base case, will give the recursive algorithm we need to solve \Odd-fair parity games. 
We will count on strategy templates in the proof of both parts. However, the second part of the algorithm follows roughly the same principles in Zielonka's original algorithm, whereas the
 the first part requires an essential change in reasoning, due to the adoption of $\SafeReach^f_\Even$. Even though the reasoning required to prove the first part is fairly different than Zielonka's original algorithm, 
a computationally cheap addition to the original algorithm is sufficient to get the correct computation for the \Odd-fair variant. Surprisingly, the trick is cheap enough not to alter the complexity of the original algorithm at all!

\smallskip
\noindent\textbf{Subsets and Sequences.}
Let $n$ be an upper bound on the priorities seen in $V$. If $n$ is \Even, set $\bb:=\Even$, otherwise $\bb:=\Odd$.
Further, we construct a decreasing series of subsets of $V$, $\{X_\bb^i\}_{i\in \mathbb{N}}$
by assigning the following sets (see Fig.~\ref{fig:kuesters-figure-extended} for an illustration): % V =: &X_\Odd^0 \supsetneq X_\Odd^1 \supsetneq \ldots \supsetneq  X_\Odd^k =  X_\Odd^{k+1}
\vspace{0.3cm}

\noindent Initially set $X^0_\nb = \emptyset$. For all $i \in \mathbb{N}$, set 
\begin{subequations}\label{equ:seriesZielonka}
    \begin{align*}
   &X_\bb^i := V \setminus X_\nb^i \quad \quad \quad &N^i:= \{v \in X^i_\bb \mid \chi(v) = n\}\\
       &Z^i:= X^i_\bb \setminus \SafeReach^f_\bb(X^i_\bb, N^i, \mathcal{G}^\ell) \quad &X^{i+1}_\nb :=  \SafeReach^f_\nb(V, X_\nb^{i} \cup Z_\nb^{i}, \mathcal{G}^\ell) % X_\Even^{i} \cup \SafeReach_\Even^f(X^{i}_\Odd, Z_\Even^{i}, \mathcal{G}^\ell) )%\text{\todo{IS: I know the equality is not completely justified. The first one is cheaper for an algorithm pov, whereas the second one is easier to justify that $X^i_\Odd$ is an \Even-trap.}} 
   \end{align*}
   \end{subequations}
where $Z_\nb^i$ is the \nb winning region in the subgame $\mathcal{G}^\ell[Z^i]$, assuming it is a subgame. 
First let's show that these sets are well-defined.
\begin{lemma}
The sets $X_\bb^i, X_\nb^i, N^i, Z^i, Z_\nb^i$ and $Z_\bb^i$ are well defined for all $i \in \mathbb{N}$.
\end{lemma}
\begin{proof} We will prove this by induction. For the base case $i = 0$, 
$X_\bb^0 = V$ is trivially an \nb-trap in $V$ and $\mathcal{G}^\ell[X^0_\bb]$ is trivially a subgame of $\mathcal{G}^\ell$. 
By Lem.~\ref{app-lem:Kuesters6.4}, $Z^0$ is an \bb-trap in $X^0_\bb$, and thus by Lem.~\ref{app-lem:Kuesters6.3}-1, $\mathcal{G}^\ell[Z^0]$ is a subgame of $\mathcal{G}^\ell$. 
Due to Corollary~\ref{app-cor:determinacy}, we know $\mathcal{G}^\ell[Z^0]$ is divided into an \bb-paradise and \nb-paradise. Therefore,  
$Z^0_\bb$ and $Z^0_\nb$ are also well-defined.  

By induction on $i$, we get by Lem.~\ref{app-lem:Kuesters6.4} that $X^i_\bb$ is an \nb-trap in $V$, and by Lem.~\ref{app-lem:Kuesters6.3}-1 $\mathcal{G}^\ell[X_\bb^i]$ is a subgame of $\mathcal{G}^\ell$. $Z^i$ is an \bb-trap in $\mathcal{G}^\ell[X^i_\bb]$, and thus by Lem.~\ref{app-lem:Kuesters6.2}, $\mathcal{G}^\ell[Z^i]$ is a subgame in $\mathcal{G}^\ell$.
Therefore $Z_\nb^i$ and $Z_\bb^i$ are well-defined.
\end{proof}
We also derived the following observations from the proof:

\begin{observation}[Sec.~\ref{sec:zielonka:correct} Obs.~\ref{it:obs1}]\label{app-obs:traps-subgames}
    $X^i_\nb$ is an \bb-trap, $X^i_\bb$, $Z^i$ and $Z_\bb^i$ are \nb-traps in $V$. $Z^i$ is in \nb-trap in $X_\bb$ and $Z_\nb^i, Z_\bb^i$ are \bb and \nb traps in $Z^i$, respectively.
    Therefore by Lem.~\ref{app-lem:Kuesters6.2}, $\mathcal{G}^\ell[Y]$ is a subgame of $\mathcal{G}^\ell$ with $Y$ being any of these sets. 
\end{observation}

\begin{lemma}[Sec.~\ref{sec:zielonka:correct} Obs.~\ref{it:obs2}]\label{app-lem:X_nb-equivalence}
    $X_\nb^{i} \cup \SafeReach_\nb^f(X^{i}_\bb, Z_\nb^{i}, \mathcal{G}^\ell) =  \SafeReach_\nb^f(V, X_\nb^{i} \cup Z_\nb^{i}, \mathcal{G}^\ell) $
\end{lemma}
\begin{proof}
   $ \mathbf{(\subseteq )}$ Trivially, $X_\nb^{i} \subseteq \SafeReach_\nb^f(V, X_\nb^{i} \cup Z_\nb^{i}, \mathcal{G}^\ell)$.
    Similarly a \\$v \in  \SafeReach_\nb^f(X^{i}_\bb, Z_\nb^{i}, \mathcal{G}^\ell)$, can be made by $\nb$ to reach $Z_\nb^i$ while staying in $X_\bb^i$. Then $v$ is trivially in the righthand side equation as well.
    
    \noindent $ \mathbf{(\supseteq )}$ 
    Let $v \in \SafeReach_\nb^f(V, X_\nb^{i} \cup Z_\nb^{i}, \mathcal{G}^\ell) \setminus X_\nb^i$.%, be in $X^j \setminus X^{j-1}$ where $X^j$ is the value of the $X$ variable after the $j^{th}$ iteration of the fixed-point computation from formula~\eqref{eq:\SafeReachEven}. that is $\rank{v} = j$ and $v \in X_\Even^{i} \cup Z_\Even^{i} \cup \Cpre_\Even(X^{j-1}) \cup \Lpre^\exists(X^{j-1})$.
     Since $v \in X_\bb^i$ and $X_\bb^i$ is an \nb-trap in $V$, if $v \in V_\bb$ it has one outgoing edge not leading to $X_\nb^i$ and 
    if $v \in V_\nb$, no outgoing edge of $v$ lead to $X_\nb^i$. That is, $v$ can either be made by \nb to reach $Z^i_\nb$ by staying in $X_\bb^i$ (i.e. it is in $\SafeReach^f_\nb(X_\bb^i, Z_\nb^i, \mathcal{G}^\ell)$),
    or $\bb = \Odd$ there exists a sequence of outgoing live edges that make $v$ reach $X_\nb^i$. This is not possible since there exists no live edges from $X_\Odd^i$ to $X_\Even^i$ due to Lem.~\ref{app-lem:safereacheven-noliveedges}.
\end{proof}
\begin{corollary}[Sec.~\ref{sec:zielonka:correct} Obs.~\ref{it:obs3}]\label{app-cor:increasing-decreasing-sequences}
    Due to Lem.~\ref{app-lem:X_nb-equivalence}, $\{X_\nb^{i}\}_{i\in \mathbb{N}}$ is an increasing sequence. Consequently, $\{X_\bb^{i}\}_{i\in \mathbb{N}}$ is a decreasing sequence. 
\end{corollary}
Since $V$ is finite, the corollary immediately implies that these sequences reach saturation value for some, and in fact the same, $k$. 

% \vspace{0.3cm}
\smallskip
\noindent\textbf{Part 1.}
We first assume an odd number $n$ is the maximum priority in $\mathcal{G}^\ell$.
% \vspace{0.2cm}
% 
%We construct a decreasing series of subsets of $V$, $\{X_\Odd^i\}_{i\in \mathbb{N}}$
%by assigning the following sets (Fig.~\ref{fig:X_Even}): % V =: &X_\Odd^0 \supsetneq X_\Odd^1 \supsetneq \ldots \supsetneq  X_\Odd^k =  X_\Odd^{k+1}
%\vspace{0.3cm}
% 
%\noindent Initially set $X^0_\Even = \emptyset$. For all $i \in \mathbb{N}$, set 
%\begin{align*}
%    &X_\Odd^i = V \setminus X_\Even^i, \quad \quad N^i = \{v \in X^i_\Odd \mid \chi(v) = n\}\\
%    &Z^i= X^i_\Odd \setminus \SafeReach^f_\Odd(X^i_\Odd, N^i, \mathcal{G}^\ell) \\
%    &X^{i+1}_\Even :=  \SafeReach_\Even^f(V, X_\Even^{i} \cup Z_\Even^{i}, \mathcal{G}^\ell) % X_\Even^{i} \cup \SafeReach_\Even^f(X^{i}_\Odd, Z_\Even^{i}, \mathcal{G}^\ell) )%\text{\todo{IS: I know the equality is not completely justified. The first one is cheaper for an algorithm pov, whereas the second one is easier to justify that $X^i_\Odd$ is an \Even-trap.}} 
%\end{align*}
% 
%where $Z_\Even^i$ is the \Even winning region in the subgame $\mathcal{G}^\ell[Z^i]$, assuming it is a subgame. 
%First let's show that these sets are well-defined with some properties.
%\begin{lemma}
%The sets $X_\Odd^i, X_\Even^i, N^i, Z^i, Z_\Even^i$ and $Z_\Odd^i$ are well defined for all $i \in \mathbb{N}$.
%\end{lemma}
%\begin{proof} We will prove this by induction. For the base case $i = 0$, 
%$X_\Odd^0 = V$ is trivially an \Even trap in $V$ and $\mathcal{G}^\ell[X^0_\Odd]$ is trivially a subgame of $\mathcal{G}^\ell$. 
%By Lem.~\ref{lem:Kuesters6.4}, $Z^0$ is an \Odd-trap in $X^0_\Odd$, and thus by Lem.~\ref{lem:Kuesters6.3}-1, $\mathcal{G}^\ell[Z^0]$ is a subgame of $\mathcal{G}^\ell$. 
%Due to Corollary~\ref{cor:determinacy}, we know $\mathcal{G}^\ell[Z^0]$ is divided into an \Odd paradise and \Even paradise. Therefore,  
%$Z^0_\Odd$ and $Z^0_\Even$ are also well-defined.  
% 
%By induction on $i$, we get by Lem.~\ref{lem:Kuesters6.4} that $X^i_\Odd$ is an \Even trap in $V$, and by Lem.~\ref{lem:Kuesters6.3}-1 $\mathcal{G}^\ell[X_\Odd^i]$ is a subgame of $\mathcal{G}^\ell$. $Z^i$ is an \Odd-trap in $\mathcal{G}^\ell[X^i_\Odd]$, and thus by Lem.~\ref{lem:Kuesters6.2}, $\mathcal{G}^\ell[Z^i]$ is a subgame in $\mathcal{G}^\ell$.
%Therefore $Z_\Even^i$ and $Z_\Odd^i$ are well-defined.
%\end{proof}
%We also derived the following observations from the proof \IS{Use and prove if needed}
% 
%\begin{observation}\label{obs:traps-subgames}
%    $X^i_\Even$ is an \Odd-trap, $X^i_\Odd$, $Z^i$ and $Z_\Odd^i$ are \Even-traps in $V$. $Z^i$ is in \Even-trap in $X_\Odd$ and $Z_\Even^i, Z_\Odd^i$ are \Odd and \Even traps in $Z^i$, respectively.
%    Therefore by Lem.~\ref{lem:Kuesters6.2}, $\mathcal{G}^\ell[Y]$ is a subgame of $\mathcal{G}^\ell$ with $Y$ being any of these sets. 
%\end{observation}
% 
%\begin{lemma}\label{lem:X_Even-equivalence}
%    $X_\Even^{i} \cup \\SafeReach_\Even^f(X^{i}_\Odd, Z_\Even^{i}, \mathcal{G}^\ell) =  \\SafeReach_\Even^f(V, X_\Even^{i} \cup Z_\Even^{i}, \mathcal{G}^\ell) $
%\end{lemma}
%\begin{proof}
%   $ \mathbf{(\subseteq )}$ Trivially, $X_\Even^{i} \subseteq \\SafeReach_\Even^f(V, X_\Even^{i} \cup Z_\Even^{i}, \mathcal{G}^\ell)$
%    Similarly a $v \in  \\SafeReach_\Even^f(X^{i}_\Odd, Z_\Even^{i}, \mathcal{G}^\ell)$, can be made to reach $Z_\Even^i$ while staying in $X_\Odd^i$. Then $v$ is trivially in the righthand side equation as well.
%     
%    \noindent $ \mathbf{(\supseteq )}$ 
%    Let $v \in \SafeReach_\Even^f(V, X_\Even^{i} \cup Z_\Even^{i}, \mathcal{G}^\ell) \setminus X_\Even^i$.%, be in $X^j \setminus X^{j-1}$ where $X^j$ is the value of the $X$ variable after the $j^{th}$ iteration of the fixed-point computation from formula~\eqref{eq:\SafeReachEven}. that is $\rank{v} = j$ and $v \in X_\Even^{i} \cup Z_\Even^{i} \cup \Cpre_\Even(X^{j-1}) \cup \Lpre^\exists(X^{j-1})$.
%     Since $v \in X_\Odd^i$ and $X_\Odd^i$ is an \Even-trap, if $v \in V_\Odd$ it has one outgoing edge not leading to $X_\Even^i$ and 
%    if $v \in V_\Even$, no outgoing edge of $v$ lead to $X_\Even^i$. That is, $v$ can either be made by \Even to reach $Z^i_\Even$ by staying in $X_\Odd^i$ (i.e. it is in $\SafeReach^f_\Even(X_\Odd^i, Z_\Even^i, \mathcal{G}^\ell)$),
%   and there exists a sequence of outgoing live edges that make $v$ reach $X_\Even^i$. This is not possible since there exists no live edges from $X_\Odd^i$ to $X_\Even^i$ due to Lem.~\ref{lem:safereacheven-noliveedges}.
%\end{proof}
% 
%\begin{corollary}\label{cor:increasing-decreasing-sequences}
%    Due to Lem.~\ref{lem:X_Even-equivalence}, $\{X_\Even^{i}\}_{i\in \mathbb{N}}$ is an increasing sequence. Consequently, $\{X_\Odd^{i}\}_{i\in \mathbb{N}}$ is a decreasing sequence. 
%\end{corollary}
%Since $V$ is finite, the corollary immediately implies that these sequences reach saturation value for some, and in fact the same, $k$. 
%% Figure environment removed
%------------
Cor.~\ref{app-cor:increasing-decreasing-sequences} gives that $\{X_\Odd^i\}_{i\in \mathbb{N}}$ is an increasing sequence and saturates at some index $k$.
Observe that $X_\Odd^k$ is the saturation value if and only if $Z_\Even^k = \emptyset$.
% 
The following proposition states that, \Odd safe reachability set of the saturation value $X_\Odd^k$ gives us \Wo.
\begin{proposition}\label{app-prop:n-odd}
    If $Z_\Even^k = \emptyset$, then $\SafeReach^f_\Odd(V, X^k_\Odd, \mathcal{G}^\ell)$ is an \Odd-paradise and $V \setminus \SafeReach^f_\Odd(V, X^k_\Odd, \mathcal{G}^\ell)$ is an \Even-paradise in $\mathcal{G}^\ell$.
\end{proposition}
We give the proof of Prop.~\ref{app-prop:n-odd} in three parts: First we prove $X^k_\Odd$ is an \Odd-paradise, then we show $\SafeReach^f_\Odd(V, X^k_\Odd, \mathcal{G}^\ell)$ is an \Odd-paradise, and lastly we prove that $V \setminus \SafeReach^f_\Odd(V, X^k_\Odd, \mathcal{G}^\ell)$ is an \Even-paradise. 

\begin{proof}\noindent \textbf{\textbf{ ($X^k_\Odd$ is an \Odd-paradise)}} 

\noindent Let $z$ be the winning \Odd strategy template on $Z^k = Z_\Odd^k$ in game $\mathcal{G}^\ell[Z^k]$. Any play $\pi$ that starts and stays in $Z^k$, and is compliant with $z$ is clearly \Odd winning.
However, $z$ is not necessarily an \Odd strategy template in $\mathcal{G}^\ell$ since there are possibly some $(v,w) \in E$ with $v \in Z^k \cap V_\Even$  and $w \not \in Z^k$.
For all such $(v,w)$, $w \in \SafeReach^f_\Odd(X^k_\Odd, N^k, \mathcal{G}^\ell)$ since $X^k_\Odd$ is an \Even-trap in $V$. Let $sr$ be the partial \Odd strategy template on $\SafeReach^f_\Odd(X^k_\Odd, N^k, \mathcal{G}^\ell)$, defined via the ranking function as presented during the introduction of safe reachability sets. 
Every (finite) play that starts in $\SafeReach^f_\Odd(X^k_\Odd, N^k, \mathcal{G}^\ell)$ compliant with $sr$ reaches $N^k$ in finitely many steps. The nodes in $N^k$ are dead ends in $sr$. 
Define an \Odd strategy template on $X^k_\Odd$ with the edge set $\e$ defined as follows:
$$ (v,w) \in \e \text{ if }\begin{cases} (v,w) \in z \cup sr,\\
    (v,w) \in E \text{ and } v \in V_\Even \cap X^k_\Odd,\\
    w = v_r \text{ if } v \in N^k \cap V_\Odd
\end{cases}
$$ where $v_r$ is a randomly chosen fixed successor for each $v\in  N^k \cap V_\Odd$, that is inside $X^k_\Odd$. Such a successor is guaranteed to exist since $X^k_\Odd$ is an \Even-trap.
Observe that all edges in $\e$ are in $X^k_\Odd \times X^k_\Odd$. However $(X^k_\Odd, \e)$ is not necessarily an \Odd strategy template in $\mathcal{G}^\ell$ since
there may be some $v \in V^\ell$ that lie on a cycle in $(X^k_\Odd, \e)$ but $\e$ does not contain their live outgoing edges. 
We will expand the edge set $\e$ to add the necessary live edges iteratively, like we did in ~\ref{const:S} (S3)-(S4).
$\overline{\e}$ is defined to be the saturation value of $\overline{e}^j$ such that:
$$\overline{e}^0 = \e, \quad  \quad \overline{e}^j = \overline{e}^{j-1} \cup \{(v, w) \in V^\ell \mid v \text{ lies on a cycle in } (X_\Odd^k, \overline{e}^{j-1})\}.$$

\vspace{0.1cm}
With this construction $\mathcal{S} = (X_\Odd^k, \overline{\e})$ is an \Odd strategy template in $\mathcal{G}^\ell$. We claim it is also a winning one. 

The underlying observation of the proof of the claim is that every play starting  $X_\Odd^k$ compliant with $\mathcal{S}$ that eventually stops seing a newly added cycle (one that is not in $z \cup sr$), stays in $Z^k$ and is won by \Odd obeying $z$; and every play that takes a newly added cycle infinitely often must see priority $n$ infinitely often, and is thus won by \Odd.

Let us look at a play $\pi$ compliant with $\mathcal{S}$. If $\pi$ eventually does not see a newly added cycle, it is clear that it wins by eventually obeying $z$ (since $sr$ does not contain any cycles).


Observe that for all newly added edges $(v,w)$ either (i)  $v \in V_\Even \cap Z^k$ and $w \in \SafeReach^f_\Odd(X^k_\Odd, N^k, \mathcal{G}^\ell)$, (ii) $v \in N^k$ or (iii) $(v, w) \in E^\ell $ where $v$ does not lie on a cycle in $z \cup sr$ and has a unique edge $(v,w') \in z \cup sr$, and this edge lies on a cycle in $\mathcal{S}$.

All the newly added cycles have to contain a newly added edge. 
If $\pi$ sees a new edge infinitely often, it visits $N^k$ infinitely often, and is thus won by \Odd. This is clear for edges of kind (ii).
Let $\pi$ see an edge of kind (iii) infinitely often. If $w\in V_\Even$, then all its outgoing edges achieves positive progress towards $N^k$, and if $w \in V_\Odd$, then it has an edge that achieves positive progress. Since $w$ is taken infinitely often, an edge that achieves positive progress towards $N^k$ will eventually be taken. Thus, $N^k$ will eventually be reached. That is, $\pi$ will visit $N^k$ infinitely often.
Finally let $\pi$ see an edge $(v,w)$ of kind (i) infinitely often. Then $(v,w')$ is also seen infinitely often. Let $C^1$ be the cycle that contains $(v,w')$. Since $C^1$ is also newly added, it contains a newly added edge $(v_1, w_1) \neq (v,w)$ since $C^1$ exists in $\overline{\e}$ before $(v,w)$ is added. If $(v_1, w_1)$ is of kind (i) or (ii), we are done. Assume the edge is of kind (iii)
and let $(v_1, w'_1)$ be the unique outgoing edge of $v_1$ in $z \cup sr$. $(v_1, w'_1)$ lies on a newly added cycle $C^2$. Let $(v_2, w_2) \not \in \{(v, w), (v_1, w_1)\} $ be the newly added edge in $C^2$. 
Carry on in this manner, assuming all newly added edges $(v_i, w_i)$ are of kind (iii). Since all $(v_i, w_i)$ are distinct and there are a finite number of live edges, for some $C^r$, $(v_r, w_r)$ should be of kind (i) or (ii).
Since $\pi$ sees $v$ infinitely often it should see all $C^i$ infinitely often, and since $C^r$ visits $N^k$, $\pi$ visists $N^k$ infinitely often. Thus, $\pi$ is won by \Odd.

%Observe that all newly added cycles are either due to a newly added outgoing edge of $N^i$, or a newly added live edge.
%If $\pi$ infinitely sees a cycle of the first kind, then it sees $n$ infinitely often, and is thus won by \Odd.
%Say all the infinitely seen newly added cycles in $\pi$ are due to newly added live edges. We claim, then again $\pi$ needs to see $n$ infinitely often.
%To see this, assume the play sees a newly added live edge $(v,w')$ infinitely often. If $v \in \SafeReach_\Odd(X^i_\Odd, N^i, \mathcal{G}^\ell)$, then due to $\pi$ being compliant with $\mathcal{S}$, the outgoing edges of $v$ in $sr$ need to be taken infinitely often as well. Thus, $\pi$ visits $N^i$ infinitely often.
%Now assume $v \in X_\Odd \setminus \SafeReach_\Odd(X^i_\Odd, N^i, \mathcal{G}^\ell)$. Since $(v,w')$ is newly added, $v$ does not lie on a cycle in $z \cup sr$. Thus, it has a unique outgoing edge in $z \cup sr$. Let $(v,w)$ be this edge.
%$v$ lies on a cycle $C$ in $\mathcal{S}$. Then $C$ contains a newly added edge $(v',u')$. Then either $v' \in V_\Even \cup X_\Odd$ and $u' \in \SafeReach(X^i_\Odd, )$

%Then the play sees a newly added live edge $(v,w')$ with $v \in X_\Odd \setminus N^i$ infinitely often. Let $(v, w)$ be outgoing edge of $v$ in $z \cup sr$.
%Then $\pi^{\rho, \phi}$ takes $(v, w)$ infinitely often as well. The fact that $(v,w')$ is newly added implies that the edge $(v,w)$ does not lie on a cycle in $z$. Then the reason $\pi^{\rho,\phi}$ comes back to $v$ again after $(v,w)$ is the \Even strategy $\phi$ taking an edge $(v', w')$ where $v' \in V_\Even \cap Z^i $ and $ w' \in \SafeReach_\Odd(X^i_\Odd, N^i, \mathcal{G}^\ell)$. 
%Then, $\pi^{\rho, \phi}$ reaches $N^i$ in finitely many steps (observe that at this point, it doesn't matter whether there is a new cycle in $\SafeReach_\Odd(X^i_\Odd, N^i, \mathcal{G}^\ell)$, since all nodes in this set reaches $N^i$ in finitely many steps). 
%Thus, $\pi^{\rho, \phi}$ sees $N^i$ infinitely often and is thus won by \Odd. 

\vspace{0.2cm}
\noindent \textbf{($\SafeReach^f_\Odd(V, X^k_\Odd, \mathcal{G}^\ell)$ is an \Odd-paradise)} 

\noindent Since $X_\Odd^k$ is an \Odd-paradise in $\mathcal{G}^\ell$, by Lem.~\ref{app-lem:safe-reach-Odd-paradise} we get that $\SafeReach^f_\Odd(V, X_\Odd^k, \mathcal{G}^\ell)$ is again an \Odd-paradise in $\mathcal{G}^\ell$.
\vspace{0.2cm}

\noindent\textbf{($V \setminus \SafeReach^f_\Odd(V, X^k_\Odd, \mathcal{G}^\ell)$ is an \Even-paradise)} 

\noindent Let $T:=\SafeReach^f_\Odd(V, X^k_\Odd, \mathcal{G}^\ell)$ and 
$\Xsr_\Even^i := \SafeReach^f_\Even(X_\Odd^i, Z_\Even^i, \mathcal{G}^\ell)$. Let the partial \Even strategy template on $\Xsr_\Even^i$ be denoted by $sr^i$ and the winning \Even strategy on $Z_\Even^i$ in game $\mathcal{G}^\ell[Z^i]$ be denoted by $z^i$. By Lem.~\ref{app-lem:Kuesters6.4}, $V \setminus T$ is an \Odd-trap.
Cor.~\ref{app-cor:increasing-decreasing-sequences} gives us that $\{X_\Even^i\}_{i \in \mathbb{N}}$ is an increasing sequence. 
Furthermore by Lem.~\ref{app-lem:X_nb-equivalence}, which gives an alternative definition for $X_\Even^{i+1}$, we observe that each $v \in X^k_\Even$ belongs to $\Xsr^j$ for some $j < k$.
Moreover, we can observe that $X^i_\Even$ and $\Xsr^i$ are disjoint sets, due to $X_\Even^i$ and $X_\Odd^i$ being disjoint. Therefore, we conclude that 
each $v \in X_\Even^k$ belongs to a unique $Xsr^j$. The same clearly holds for $v \in V\setminus T$, since $(V\setminus T) \subseteq X_\Even^k$.
Furthermore, since $V\setminus T$ is an \Odd-trap, for all $(v,w) \in E$ with $v \in V_\Odd \cap (V \setminus T)$, $w \in (V\setminus T)$.

We construct the \Even strategy template $\mathcal{S} = (X, \e)$ where $\e$ is defined as follows: $(v,w) \in E $ is in $\e$ if, 
$$\begin{cases}v \in V_\Odd\\
    v \in Z_\Even^i \text{ and } (v,w) \text{ is the unique outgoing edge of }v \text{ in } z^i\\
    v \in \Xsr_\Even^i \setminus Z_\Even^i \text{ and } (v,w) \text{ is the unique outgoing edge of }v \text{ in } sr^i\\
\end{cases}$$

It is clear that $\mathcal{S}$ is an \Even strategy template since it contains all outgoing edges of \Odd nodes in $V \setminus T$, and a unique outgoing edge for each \Even node in $V \setminus T$. We claim that $\mathcal{S}$ is also winning.
To prove this claim we will need the following two observations. 

Let $\pi = v_1 v_2 \ldots $ be a fair play that start in $V \setminus T$ and is compliant with $\mathcal{S}$. Let $\Xsr(\pi) = \Xsr_1\Xsr_2\Xsr_3\ldots$ be such that $\Xsr_i$ is the unique $\Xsr^j$, $v_i$ belongs to.

(1) If $v_t \in Z_\Even^i$, then $v_{t+1}$ is either in $Z_\Even^i$ or in $\Xsr^r$ for some $t < i$. This follows from $Z_\Even^i$ being an \Odd-trap in $X_\Odd^i$ (by Obs.~\ref{app-obs:traps-subgames}).

(2) If $\Xsr^i$ is seen infinitely often in $\Xsr(\pi)$, then $Z_\Even^i$ is seen infinitely often as well. Due to the pigeonhole principle, $\Xsr^i$ being visited infinitely often in $\Xsr(\pi)$ implies that some $v \in \Xsr^i$ is visited infinitely often.
If $v \not \in Z_\Even^i$, it is in $\Xsr^i \setminus Z_\Even^i$. Say $v \in V_\Even$, then the unique $(v,w) \in \e$ causes positive progress towards $Z_\Even^i$. If $v \in V_\Odd \setminus V^\ell$, then all of the outgoing edges of $v$ cause positive progress towards $Z_\Even^i$.
If $v \in V^\ell$, there is at least one $(v,w) \in E^\ell$ causing positive progress towards $Z_\Even^i$. Since $v$ is seen infinitely often in $\pi$, this edge is taken infinitely often as well. 
By induction, $\pi$ visits $Z_\Even^i$ infinitely often.

\vspace{0.2cm}
\noindent \emph{Claim:} Any fair play $\pi$ starting in $X$ and compliant with $\mathcal{S}$ eventually stays in $Z_\Even^i$ for some $i$.

\noindent \emph{Proof of Claim.}
Let $i$ be the minimum index for which $\Xsr^i$ appears infinitely often in $\Xsr(\pi)$. By observation (2), $\pi$ sees a set of nodes $P \subseteq Z_\Even^i$ infinitely often. Let $v_t \in P$. By observation (1), $v_{t+1}$ is either in $Z_\Even^i$ or in $\Xsr^r$ for some $r < i$.
Since $i$ is the minimum index for which $\Xsr^i$ is seen infinitely often in $\Xsr(\pi)$, after some $t' \in \mathbb{N}$, for all $v_{t'}\in P$, $v_{t'+1} \in Z_\Even^i$.

Since $\pi$ eventually stays in $Z_\Even^i$, the strategy $\mathcal{S}$ eventually collapses to $z_\Even^i$ and thus, \Even wins $\pi$.
\end{proof}

With this, we have proven Prop.~\ref{app-prop:n-odd}, and therefore have given an algorithm to calculate \We and \Wo on an \Odd-fair parity game with 
an odd upper bound $n$ on the priorities in the game graph. The algorithm however requires a sibling-algorithm that does the same for an \Odd-fair parity game with an upper bound $n-1$ on its priorities. In the second part that follows, we give this sibling-algorithm.

% \vspace{0.5cm}
\smallskip
\noindent\textbf{Part 2.}
We now assume an even number $n$ is the maximum priority in $\mathcal{G}^\ell$. %We construct a decreasing series of subsets of $V$, $\{X_\Odd^i\}_{i\in \mathbb{N}}$ by assigning the following sets (Fig.~\ref{fig:X_Even}):
% \vspace{0.3cm}
%Now we know how to get the \Odd and \Even winning regions in a game $\mathcal{G}^\ell$, given that the highest priority $n$ in the game is odd, by reducing it down to solving subgames where the highes priority is $n-1$.
%We should show how to get \Odd and \Even winning regions in a game where the highest priority is even, then we will be able to apply these two constructions recursively to get an variant of Zielonka's algorithm that solves \Odd-fair parity games. 
% 
%This part of the proof is almost identical to the proof of Zielonka's algorithm for regular parity games. %We use strategy te
%But we will give out a full proof for the sake of completeness. 
% 
We set the sets as before, and because $n$ is even, this time $\{X_\Odd^i\}_{i \in \mathbb{N}}$ is an increasing sequence and $\{X_\Even^i\}_{i \in \mathbb{N}}$ is a decreasing one (Fig.~\ref{fig:kuesters-figure-extended}).
Both sequences saturate at some index $k$, and for this $k$, $Z_\Odd^k = \emptyset$. Furthermore, $X_\Even^k$ and $X_\Odd^k$ are \We and \Wo, respectively.
%We again set the sets with the same names as before, the division can be seen in Fig...
%Let $W_O$ be an \Odd paradise in $\mathcal{G}^\ell$, and let $X_\Odd = \SafeReach_\Odd(V, W_E, \mathcal{G}^\ell)$. It is clear that $X_\Odd$ is an \Odd paradise in $\mathcal{G}^\ell$.
%Let $X_\Even = V \setminus X_\Odd$. $X_\Even$ is an \Odd-trap. $N$ is defined as before and $Z:= X_\Even \setminus \SafeReach_\N(X_\Even, N, \mathcal{G}^\ell)$. $Z$ is an \Even-trap in $X_\Even$, and thus $\mathcal{G}^\ell[Z]$ is a subgame. 
%Let $Z_\Even$ and $Z_\Odd$ be the winning regions of \Even and \Odd in $\mathcal{G}^\ell[Z]$ as before. 


\begin{proposition} For all $i$, $Z^i_\Odd \cup X^i_\Odd$ is an \Odd-paradise in $\mathcal{G}^\ell$. 
\end{proposition}
\begin{proof}
    The fact that $Z^i_\Odd \cup X^i_\Odd$ is an \Even-trap follows from the observations in \ref{app-obs:traps-subgames}.
%The union is an \Even trap: $X_\Odd$ is an \Even trap in $\mathcal{G}^\ell$ and $Z_\Odd$ is an \Even trap in $X_\Even$ since it is an \Even trap in  $Z$ and $Z$ is an \Even trap in $X_\Even$ (lemma..).
%Then, $v \in V_\Odd \cap (Z_\Odd \cup X_\Odd)$ has an outgoing edge to staying inside the set, and for  $v \in V_\Even \cap (Z_\Odd \cup X_\Odd)$ all the outgoing edges are either from $X_\Odd$ to $X_\Odd$, $Z_\Odd$ to $Z_\Odd$ or from $Z_\Odd$ to $X_\Odd$.
%Therefore,  $Z_\Odd \cup X_\Odd$ is an \Even trap in $\mathcal{G}^\ell$.

Let us denote the winning \Odd strategy template on $ Z^i_\Odd $ in $\mathcal{G}^\ell[Z^i]$ with $z$ 
and the strategy template on $X_\Odd^i$ in $\mathcal{G}^\ell$ by $x$. 
Let $\e$ be the edge set that contains all edges in $z \cup x$, together with all $\{(v,w) \in E \mid v \in V_\Even \cap (Z^i_\Odd \cup Z_\Odd^i) \}$.
Due to $X_\Odd^i$ being an \Even-trap in $V$, all outgoing edges of \Even nodes in $X_\Odd^i$, stay in $X_\Odd^i$. Then, $\e$ does not introduce any new cycles to $z \cup x$ since all the newly added edges are in one direction, from $Z^i_\Odd$ to $X^i_\Odd$. Thus, $\mathcal{S} = (X_\Odd^i \cup Z_\Odd^i, \e)$ is an \Odd strategy template in $\mathcal{G}^\ell$.
We claim it is also a winning one. 
A play $\pi$ starting in $X_\Odd^i$ and compliant with $\mathcal{S}$ stays in $X_\Odd^i$ and therefore wins by obeying $x$. 
If $\pi$ starts in $Z_\Odd^i$, it either eventually reaches $X_\Odd^i$ and therefore wins by the previous argument. Or, it stays in $Z_\Odd^i$ and wins by obeying $z$.
\end{proof}

\begin{proposition} If $Z^i_\Odd = \emptyset$, $X^i_\Even$ is an \Even-paradise in $\mathcal{G}^\ell$.
\end{proposition}

\begin{proof}
We know $X^i_\Even$ is an \Odd-trap~\ref{app-obs:traps-subgames}. Let $z$ be the winning \Even strategy on $Z^i_\Even$ in subgame $\mathcal{G}^\ell[Z^i]$ and $sr$ be the partial strategy template on $\SafeReach^f_\Even(X^i_\Even, N^i, \mathcal{G}^\ell)$ where all nodes in $\SafeReach^f_\Even(X^i_\Even, N^i, \mathcal{G}^\ell) \setminus N^i$ are forced to positive progress towards $N^i$ in the next step, and nodes in $N^i$ are dead-ends.

We construct an \Even strategy template $\mathcal{S} = (X_\Even^i, \e)$ where $\e$ is defined as follows: 
$$
(v,w) \in \e \text{ if }\begin{cases}
    (v, w) \in z \cup sr, \\
    (v, w)\in E \text{ and } v\in V_\Odd \cap X_\Even,\\
    w = v_r \text{ if } v \in N^i \cap V_\Even
\end{cases}
$$ where $v_r$ is a randomly chosen fixed successor for each $v \in N^i \cap V_\Even$, that is inside $X_\Even^i$. Such a successor is guaranteed to exist since $X_\Even^i$ is an \Odd-trap.

$\mathcal{S}$ is clearly an \Even strategy template in $\mathcal{G}^\ell$ since all \Odd nodes in $X_\Even^i$ have all their outgoing edges in $\mathcal{S}$ and all \Even nodes have a unique outgoing edge.
We claim it is also winning. 

Let $\pi$ be a play that starts in $X_\Even^i$ and is compliant with $\mathcal{S}$. We claim $\pi$ either (i) eventually stays in $Z_\Even^i$, and therefore eventually obeys $z$ or (ii) it sees $N^i$ infinitely often.
It is easy to see that in both of these cases $\pi$ is \Even winning. We will try to show that one of these cases must occur.
Assume $\pi$ does not eventually stay in $Z^i_\Even$. Then $\pi$ visits some $ v \in \SafeReach^f_\Even(X^i_\Even, N^i, \mathcal{G}^\ell)$ infinitely often. If $v \in V_\Odd$, all outgoing edges of $v$ are in $sr$ make positive progress towards $N^i$, and if $v \in V_\Even $ the unique successor of $v$ in $sr$ make positive progress towards $N^i$. 
Thus, $\pi$ visists $N^i$ after finitely many steps. Since $v$ is visited infinitely often by $\pi$, $N^i$ is also visited infinitely often.
\end{proof}


\subparagraph{Corrrectness of Alg.~\ref{algo:fair-zielonka-bb}.}
The $X$ set in $\SOLVE_\Odd(n, \mathcal{G}^\ell)$ holds the value of $X_\Odd^i$ and
the $X$ set in $\SOLVE_\Even(n, \mathcal{G}^\ell)$ holds the value of $X_\Even^i$ at the $i^{th}$ iteration of their respective \emph{while} loops. 
Note that both of these sequences are initialized at $V$ and are strictly decreasing, until they reach their saturation value $X_\Odd^k$ or $X_\Even^{k'}$. When these saturation values are reached $Z_\Even^k = \emptyset $ in the $\SOLVE_\Odd$ procedure and $Z_\Odd^{k'} = \emptyset $ in the $\SOLVE_\Even$ procedure. 
This is exactly when $\SOLVE_\Even$ returns $X_\Even^{k}$ and $\SOLVE_\Odd$ returns $\SafeReach^f_\Odd(V, X_\Odd^{k'}, \mathcal{G}^\ell)$; correctfully returning their respective winning regions according to the correctness proof of Thm.~\ref{thm:solvebb}.

%Let $X_\Even := \SafeReach^f(V, Z_\Even^{i-1}, \mathcal{G}^\ell)$ and $X_\Odd := V \setminus X_\Even$.
%Set $N = \{v \in X_\Odd \mid \chi(v) = n\}$ and $Z= X_\Odd \setminus \SafeReach^f_\Odd(X_\Odd, N, \mathcal{G}^\ell)$. 
%Since $X_\Odd$ is an \Even trap in $\mathcal{G}^\ell$, by Lem.~\ref{lem:Kuesters6.3}-1, $X_\Odd$ is a subgame of $\mathcal{G}^\ell$. (Since $Z$ is the complement of a $\SafeReach^f_\Odd$ set in $X_\Odd$, by Lem.~\ref{lem:Kuesters6.4}-1, it is an $\Odd-$trap in $X_\Odd$.
%By Lem.~\ref{lem:Kuesters6.3}-1, $Z$ is a subgame in $X_\Odd$; and thus a subgame in $\mathcal{G}^\ell$. )
%Let $Z^i_\Even$ and $Z^i_\Odd$ be the winning regions of \Even and \Odd in $\mathcal{G}^\ell[Z]$. Naturally, $Z^i_\Even$ is an $\Even-$paradise, and $Z^i_\Odd$ is a $\Odd-$paradise in $Z$. %Let us call the winning $a$-strategy template on $Z_a$ in $Z$, $z_a$; and the winning $\neg a$-strategy template on $Z_{\neg a}$ in $Z$, $z_{\neg a}$. 

%Before presenting a recursive algorithm to solve \Odd-fair parity games, we present the last lemma of \cite{Kuesters2002} that does not work in the \Odd-fair setting. 
%The last lemma of \cite{Kuesters2002} is the following:
%\begin{lemma}[\cite{Kuesters2002}, Lemma 6.9]\label{lem:Kuesters6.9}
%If $Z_{\neg a} = \emptyset$, then $X_a$ is an $a-$paradise in $\mathcal{G}$.
%\end{lemma}

%\textbf{Counter example.} Figure \ref{fig:counter_ex} depicts a counter example of Lem.~\ref{lem:Kuesters6.9}.
%In the figure $a= \Odd$, $X_{\neg a} = X_\Even = \{v_0\}$ which is an $\Even-$paradise in $\mathcal{G}^\ell$. $X_\Odd = V \setminus X_\Even$ is an $\Even-$trap. $N = \{v_2\}$ for $n = 3$, and $\SafeReach^f_{\Odd}(X_\Odd, N, \mathcal{G}^\ell) = \{v_1, v_2 \}$; thus, $Z = \emptyset$. Consequently $Z_{\Even} = \emptyset$. But unlike Lem. 6.9 suggests, $X_\Odd$ is not an $\Odd-$paradise in $\mathcal{G}^\ell$. In fact, the whole game is winning for $\Even$.

%% Figure environment removed

%\noindent \textbf{Showing the algorithm corresponds to the proof \IS{I didn't touch the part below yet}}

%Let us go over Alg.~\ref{alg:fair-zielonka} and show that the calculations correspond to that of the described recursive algorithm. Let us represent the values of the variables $N,Z,Z_\Odd $ and $G$ at the $i^{th}$ iteration of the while loop between lines \ref{line:while_start} - \ref{line:while_end} by $N^i, Z^i, Z_\Odd^i $ and $G^i$.
%In line \ref{line:N} we calculate set $N^i$ as defined above.
%In line \ref{line:Z} we calculate $Z^i = X_{\Even}^i \setminus \SafeReach^f_\Even(X_\Even^i, N^i, \mathcal{G}^\ell)$. This works since $G^i$ equals $X_\Even^i$ throughout the algorithm. 
%In line \ref{line:Z_Odd}, we recursively compute the winning region for \Odd in $\mathcal{G}[Z^i]$ as $Z^i_\Odd$ by decreasing the highest priority in the game $\mathcal{G}[Z^i]$ by one. The reason behind is that $Z^i$ does not contain any nodes with priority $n$.
%Finally, in line \ref{line:G_update}, we remove $\SafeReach_\Odd^f(G, (V\setminus G ) \cup Z_\Odd, \mathcal{G}^\ell)$ from $G$. $G^i$ holds the value of $X_\Even^i$ until this line, whereas in this line it is updated to $X_\Even^{i+1}$.
%This is equivalent to the computation in the recursive algorithm described above for the following reason: 
%$W_\Odd^{i+1} = X_\Odd^i \cup Z_\Odd^i$ and $X_\Odd^{i+1} = \SafeReach^f_\Odd(X_\Even^i, W_\Odd^{i+1}, \mathcal{G}^\ell)$ are given in the description of the recursive algorithm.
%Then,  $X_\Odd^{i+1} = \SafeReach^f_\Odd(X_\Even^{i},X_\Odd^i \cup Z_\Odd^i, \mathcal{G}^\ell)$.
%Since $X^i_\Odd = V \setminus X^i_\Even$, $X^{i+1}_\Odd = \SafeReach^f_\Odd(X_\Even^i, (V \setminus G^i) \cup Z_\Odd^i, \mathcal{G}^\ell)$, and since $V = X^0_\Even \supseteq X^1_\Even \supseteq \ldots X^k_\Even = X^{k+1}_\Even$ is a decreasing chain, 
%$X_\Even^{i+1} = V \setminus X_\Odd^{i+1} = X^{i}_\Even \setminus \SafeReach^f_\Odd(X_\Even^i, (V \setminus G^i) \cup Z_\Odd^i, \mathcal{G}^\ell)$.

%When $G^i = X_\Even^i$ is empty, then the winning set is trivially empty. As proved in above claim, when $Z_\Odd^i$ is empty, $G = X_\Even^i$ is an $\Even-$paradise, and the algorithm correctly returns $G$ to be the winning set of Even.

