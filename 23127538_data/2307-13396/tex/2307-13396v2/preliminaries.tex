\vspace*{-0.3cm}
\section{Preliminaries}
\noindent\textbf{Notation.}
We use $\mathbb{N}$ to denote the set of natural numbers including zero and $\mathbb{N}^+$ to denote positive integers. Let $\Sigma$ be a finite set. Then  $\Sigma^*$ and $\Sigma^\omega$ denote the sets of finite and infinite words over $\Sigma$, respectively. %, and $\Sigma^\infty$ is equal to $\Sigma^*\cup \Sigma^\omega$.
% Given two natural numbers $a,b\in\mathbb{N}$ with $a<b$, we use $[a;b]$ to denote the set $\set{n\in\mathbb{N} \mid a\leq n\leq b}$.
% For any given set $[a;b]$, we write $i\ineven [a;b]$ and $i\inodd [a;b]$ as short hand for $i\in [a;b]\cap \set{0,2,4,\ldots}$ and $i\in [a;b]\cap \set{1,3,5,\ldots}$ respectively.
% Given two sets $A$ and $B$, a relation $R\subseteq A\times B$, and an element $a\in A$, we write $R(a)$ to denote the set $\set{b\in B\mid (a,b)\in R}$.


\smallskip
\noindent\textbf{Game graphs.}
A \emph{game graph} is a tuple $\gamegraph= \tup{V,V^0,V^1,E}$ where $(V,E)$ is a finite directed graph with  \emph{edges} $ E $ and \emph{vertices} $ V $ partioned into player $0$ and player $1$ vertices, $V^0$ and $V^1$, respectively. 
Without loss of generality, we can assume
that all nodes in $V$ have at least one outgoing edge. Under this assumption, there exist plays from each vertex.
A \emph{play} originating at a vertex $v_0$ is an infinite sequence of vertices $\pi=v_0v_1\ldots \in V^\omega$. 
For $v \in V$, $E(v)$ denotes its successor set $\{w \mid (v, w) \in E\}$. 

\smallskip
\noindent\textbf{LTL winning conditions.}
Given a game graph $\gamegraph$, we consider winning conditions specified using a formula $\spec$ in \emph{linear temporal logic} (LTL) over the vertex set $V$, that is, we consider LTL formulas whose atomic propositions are sets of vertices. 
In this case the set of desired infinite plays is given by the semantics of $\spec$ which is an $\omega$-regular language $\lang(\spec)\subseteq V^\omega$. 
%In this case, an $\omega$-regular winning condition expressed as the LTL formula $\spec$ can be interpreted as the $\omega$-regular language $\lang(\spec)\subseteq V^\omega$ containing all desired infinite sequences of vertices. 
The standard definitions of $\omega$-regular languages and LTL are omitted for brevity and can be found in standard textbooks \cite{baierbook}. A game graph $\gamegraph$ under the winning condition $\spec$ is written as $\ltup{\gamegraph, \spec}$. A play $\pi$ is winning for player $0$ in $\ltup{\gamegraph, \spec}$ if $\pi\in\lang(\spec)$, i.e. $\pi\models\spec$.

\smallskip
\noindent \textbf{Strategies.}
A \emph{strategy} for player $j$ over the game graph $\gamegraph$ is a function $\rho^j : V^* \cdot V^j \to V$ with the constraint that for all $u\cdot v \in V^* \cdot V^j$ it holds that $\rho^j(u \cdot v) \in E(v)$. A play $\pi=v_0v_1\ldots \in V^\omega$ is compliant with $\rho^j$ if for all $i\in \mathbb{N}$ holds that $v_i\in V^j$ implies $v_{i+1}=\rho^j(v_0 \ldots v_i)$. A strategy $\rho^j$ is winning from a subset $V'$ of vertices of the game $\ltup{\gamegraph,\Psi}$ if all plays $\pi$ in $G$ that start at a vertex in $V'$ and are compliant with $\rho^j$ are winning w.r.t.\ $\Psi$. 
A strategy $\rho$ is called \emph{positional} iff for all $w_1, w_2 \in V^*$, $\rho(w_1 \cdot v) = \rho(w_2 \cdot v)$. %That is, a positional strategy $\rho_0$ for
 %\Even can be defined from $\Ve \to V$, instead of $V^* \cdot \Ve \to V$. 
%\AKS{misses definition of compliant plays}

\smallskip
\noindent\textbf{Parity Games.}
Parity games are particular two player games over a game graph $\gamegraph$ where the winning condition is given by a particular mapping of vertices. Formally, a parity game is a tuple $\mathcal{G} = \ltup{ V, \Ve, \Vo, E, \chi}$, where $\tup{V, \Ve, \Vo, E}$ is a game graph and $\chi : V \to \mathbb{N}^+$ is a function which labels each vertex with an integer value, called a \emph{priority}. The players $0$ and $1$ are called $\Even$ and $\Odd$ in a parity game and 
% 
% between player \Even and player \Odd, where
% \begin{inparaenum}[(i)]
%     \item $V$ is the vertex set, which is a disjunction of $\Ve$, the vertices owned by player \Even and $\Vo$, the vertices owned by player \Odd;
%     \item $E \subseteq V \times V $ is the edge set; and
%     \item $\chi : V \to \mathbb{N}^+$ is the priority function, which labels each vertex with an integer value.
% \end{inparaenum}
% % 
% The two players -- \Even and \Odd -- get to chose the next edge to take from their respective vertices $\Ve$ and $\Vo$. A play is
% an infinite sequence $v_1 v_2 v_3 \ldots \in V^\omega$ 
% \AKS{Usually we start counting at $0$, should we change it?}
% of vertices, where for all $i\in\mathbb{N}$, $(v_i, v_{i+1}) \in E$. Without loss of generality, we can assume
% that all nodes in $V$ have at least one outgoing edge. Under this assumption, there exists plays from each vertex.
a play $\pi = v_1 v_2 \ldots$ is winning for \Even iff $\max\{\inf(\pi)\}$ is \emph{even}, where $\inf(\pi)$ is the set of vertices visited infinitely often in $\pi$. Otherwise the play is winning for \Odd. %We define the sets $C_i := \{ v \in V \mid \chi(v) = i\}$ and $\overline{C_i} := V \setminus C_i$ to ease notation.

% \smallskip
% \noindent \textbf{Winning Parity Games.}
A node $v\in V$ is said to be won by \Even, if \Even has a (winning) strategy $\rho$ such that 
all plays $\pi = v\cdot \pi'$ that are compliant with $\rho$ are won by \Even. %\IS{Q: Is there a difference between saying (i) 'all plays that are compliant wih $\rho_0$ are won by... and (ii) for all player \Odd strategies $\rho_1$, all plays that are compliant with $\rho_0$ and $\rho_1$ are won by...}
The winning region of \Even is the set of all nodes won by \Even and is denoted by $\We$. The winning region of \Odd, $\Wo$, is defined similarly. 
% 
It is well-known that parity games are determined, that is, all nodes are either in $\mathcal{W}_{Even}$ or in $\mathcal{W}_{Odd}$; and that both players have positional winning strategies from their respective winning regions \cite{EJ89}.

\begin{comment}
\smallskip
\noindent\textbf{Fair Parity Games.}
A \emph{fair parity game} $\mathcal{G}^\ell$ is a tuple $\ltup{\mathcal{G},E^\ell}$, where $\mathcal{G} = \langle V, \Ve, \Vo, E, \chi\rangle$ is a parity game, $E^\ell \subseteq E $ is a set of \emph{live edges} %inducing a \emph{strong transition fairness constraint}. 
and $V^\ell$, the domain of the relation $E^\ell$, is the set of \emph{live vertices}. The live edges induce a \emph{strong transition fairness constraint} -- whenever a live vertex $v$ is visited infinitely often, every outgoing live edge $(v, w') \in E^\ell$ needs to be taken infinitely often.
Formally, a play $\pi$ in $\mathcal{G}$ \emph{complies} with $E^\ell$ if the LTL formula
\begin{equation}\label{eq:fairness-ltl}
    \textstyle\alpha := \bigwedge_{(v, w)\in E^\ell} (\,\square\, \diamondsuit \,v \implies \square\, \diamondsuit\, (v \wedge \bigcirc w))\vspace{-1mm}
\end{equation} 
holds along $\pi$, i.e.\ $\pi\models\alpha$.%\footnote{$\square$ stands for "always", $\diamondsuit$ for "eventually" and $\bigcirc$ for "next". For more detailed information on LTL \cite{}}
 A play $\pi$ is winning for player \Even if and only if the play does not comply with $E^\ell$ or $\max\{\inf(\pi)\}$ is even.

\smallskip
\noindent\textbf{\Odd-Fair Parity Games.}
An \emph{\Odd-fair parity game} is a particular instance of a fair parity game, where only player \Odd is restricted by an additional \emph{strong transition fairness constraint}, i.e., $V^\ell\subseteq\Vo$. In this case, a play $\pi$ is winning for \Even in $\mathcal{G}^\ell$ if and only if $\pi \models \neg\alpha$ or $\max\{\inf(\pi)\}$ is \emph{even}. Dually, $\pi$ is winning for \Odd iff $\pi \models \alpha$ and $\max\{\inf(\pi)\}$ is odd. A strategy $\rho$ over $\gamegraph$ is therefore winning for \Even (resp.\ \Odd) in $\mathcal{G}^\ell$ if all plays compliant with $\rho$ are winning for \Even (resp. \Odd) in $\mathcal{G}^\ell$.

As the winning condition of a parity game can be equivalently modeled by a suitably defined LTL winning condition, we see that \Odd-fair parity games are a special deterministic $\omega$-regular game with perfect information. This implies that \Odd-fair parity games are determined (by the Borel determinacy theorem \cite{Martin75}) and whenever there exists a winning strategy for \Even/\Odd in such a game, then there also exists one with \emph{finite} memory \cite{GH82}. 
\end{comment}
\smallskip
\noindent\textbf{\Odd-Fair Parity Games.} %\todo{I removed the 'fair parity' part upon the correct complaints of a review that they are not well-defined.}
An \emph{\Odd-fair parity game} 
$\mathcal{G}^\ell$ is a tuple $\ltup{\mathcal{G},E^\ell}$, where $\mathcal{G} = \langle V, \Ve, \Vo, E, \chi\rangle$ is a parity game, $E^\ell \subseteq E $ is a set of \emph{live edges} that originate from \Odd player vertices %inducing a \emph{strong transition fairness constraint}. 
and $V^\ell \subseteq \Vo$, the domain of the relation $E^\ell$, is the set of \emph{live vertices}.
The live edges induce a \emph{strong transition fairness constraint} -- whenever a live vertex $v$ is visited infinitely often, every outgoing live edge $(v, w') \in E^\ell$ needs to be taken infinitely often.
Formally, a play $\pi$ in $\mathcal{G}$ \emph{complies} with $E^\ell$ if the LTL formula%
\footnote{Here, $\square$, $\diamondsuit$ and $\bigcirc$ stand for the LTL operators 'always', 'eventually' and 'next'.}
\begin{equation}\label{eq:fairness-ltl}
    \textstyle\alpha := \bigwedge_{(v, w)\in E^\ell} (\,\square\, \diamondsuit \,v \implies \square\, \diamondsuit\, (v \wedge \bigcirc w))\vspace{-1mm}
\end{equation} 
holds along $\pi$, i.e.\ $\pi\models\alpha$. %\footnote{$\square$ stands for "always", $\diamondsuit$ for "eventually" and $\bigcirc$ for "next". For more detailed information on LTL \cite{}}
A play $\pi$ is winning for \Even in $\mathcal{G}^\ell$ if and only if $\pi \models \neg\alpha$ or $\max\{\inf(\pi)\}$ is \emph{even}. Dually, $\pi$ is winning for \Odd iff $\pi \models \alpha$ and $\max\{\inf(\pi)\}$ is odd. A strategy $\rho$ over $\gamegraph$ is therefore winning for \Even (resp.\ \Odd) in $\mathcal{G}^\ell$ if all plays compliant with $\rho$ are winning for \Even (resp. \Odd) in $\mathcal{G}^\ell$.

As the winning condition of a parity game can be equivalently modeled by a suitably defined LTL winning condition, we see that \Odd-fair parity games are a special $\omega$-regular game with perfect information. This implies that \Odd-fair parity games are determined (by the Borel determinacy theorem \cite{Martin75}) and whenever there exists a winning strategy for \Even/\Odd in such a game, then there also exists one with \emph{finite} memory \cite{GH82}. 


% 
% It was recently shown in \cite{banerjee2022fast} that winning strategies of \Even in \Odd-fair parity games are positional and can be deduced from the ranking induced by a new fixed-point algorithm in the $\mu$-calculus computing the winning region of \Even in \Odd-fair parity games.

% =======
% 
% 
% We define a parity game graph as a tuple $\mathcal{G} = <V, V_0, V_1, E, \chi>$ between \textit{Even} (or, player $0$)  and \textit{Odd} (or, player $1$) where
% \begin{itemize}
%     \item $V$ is the vertex set, which is a dusjunction of $V_0$ (or, $V_{Even}$), the vertices owned by player \textit{Even} and $V_1$ (or, $V_{Odd}$), the vertices owned by player \textit{Odd}.
%     \item $E \subseteq V \times V $ is the edge set
%     \item $\chi : V \to \mathbb{N}^+$ is the priority function, which labels each vertex with an integer value.
% \end{itemize}
% 
% Two players \textit{Even} and \textit{Odd} get to chose the next edge to take from their respective vertices $V_0$ and $V_1$. A play is
% an infinite sequence $v_1 v_2 v_3 \ldots $ of vertices, where for all $i$, $(v_i, v_{i+1}) \in E$. Without loss of generality, we can assume
% all the nodes in $V$ obtain at least one outgoing edge. Under this assumption, there exists plays from each vertex.
% A play $\pi = v_1 v_2 \ldots$ is winning for \textit{Even} iff $max\{inf(\pi)\}$ is even, where $inf(\pi)$ is the set of vertices visited infinitely often in $\pi$. Otherwise the play is winning for \textit{Odd}
% 
% A \textbf{strategy} for \textit{Even} is a function $\rho_0 : V^* \cdot V_0 \to V$ with the constraint that for all $u,v \in V^* \cdot V_0$, $\rho(u \cdot v) \in E(v)$, and similarly for \textit{Odd}.
% A strategy $\rho$ is called a \textbf{positional strategy} iff for all $w_1, w_2 \in V^*$, $\rho(w_1 \cdot v) = \rho(w_2 \cdot v)$. That is a positional strategy $\rho_0$ for
%  \textit{Even} can be defined from $V_0 \to V$, instead of $V^* \cdot V_0 \to V$. 
% 
% A node $v\in V$ is said to be won by \textit{Even}, if \textit{Even} has a (winning) strategy $\rho_0$ such that 
% all plays $\pi = v\cdot \pi'$ that are compliant with $\rho_0$ are won by \textit{Even}. %\IS{Q: Is there a difference between saying (i) 'all plays that are compliant wih $\rho_0$ are won by... and (ii) for all player \textit{Odd} strategies $\rho_1$, all plays that are compliant with $\rho_0$ and $\rho_1$ are won by...}
% The winning region of \textit{Even} is the set of all nodes won by \textit{Even} and is denoted by $\mathcal{W}_0$ or $\mathcal{W}_{Even}$. Winning region of \textit{Odd}, $\mathcal{W}_1$ or $\mathcal{W}_{Odd}$ is defined similarly. 
% 
% It is well-known that parity games are determined, that is, all nodes are either in $\mathcal{W}_0$ or $\mathcal{W}_1$; and that both players have positional winning strategies from their respective winning regions. \IS{ref}
% 
% \textbf{Fair parity games} are given by a parity game graph $\mathcal{G}$, and a subset of the edge set $E^l \subseteq E \cap (V_1 \times V)$, which are called \textit{live edges}.
% Let $\mathcal{G}$ be a parity game and $E^l$ be a given set of \textit{live edges}. We call $V^l := dom(E^l) \subseteq V$ the \textit{live vertices}, which are all \textit{Odd} vertices. Intuitively, $E^l$ represents a fairness condition that is expected for \textit{Odd} to satisfy: whenever 
% a node in $V^l$ is visited infinitely often, we expect \textit{Odd} to take every outgoing edge $(v, w') \in E^l$ infinitely often. Formally, a fair parity game is represented by $\mathcal{G}^l := <\mathcal{G}, E^l>$ and a play $\pi$
% in $\mathcal{G}^l$ is 'fair' if it satisfies the following LTL formula
% \begin{equation}\label{eq:fairness-ltl}
%     \alpha := \bigwedge_{(v, w)\in E^l} (\square \diamond v \implies \square \diamond (v \wedge \bigcirc w))
% \end{equation}
% 
% Given a fair parity game $\mathcal{G}^l$, \textit{Even} wins a play $\pi$ on $\mathcal{G}^l$ if and only if either $\pi \models \neg\alpha$ or
% $max\{inf(\pi)\}$ is even. Dually, \textit{Odd} wins $\pi$ iff $\pi \models \alpha$ and $max\{inf(\pi)\}$ is odd.
% 
% % ---------------
% 
% A \textbf{fair strategy} for player $i$ is a function $\rho_i: V^* \cdot V_i \to V$ in a fair parity game with the constraint that for all
% $u,v \in V^* \cdot V_i, \rho_i(u\cdot v) \in E(v)$. A play $\pi$ is compliant with the 
% fair strategy $\rho_i$ iff the following holds for all $u, v \in V^* \cdot V_i$,
% \begin{align*}
% v  \not \in inf(\pi) \implies &\forall j \text{ s.t. } v_j = v, v_{j+1} = \rho_i(u \cdot v)\\
% v \in inf(\pi) \implies &\forall w \in \{\rho_i(u \cdot v)\} \cup \{w\mid (v,w) \in E^l\},  \\
% & \{j \mid v_j = v \text{ and } v_{j+1} = w \} \text{ is an infinite set and }\\
% & \forall j \text{ s.t. } v_j = v, \quad v_{j+1} \in \{\rho_i(u \cdot v)\} \cup \{w\mid (v,w) \in E^l\}
% \end{align*}
% Note that, since for $v \in V_{Even}$, $E^l(v) = \emptyset$, a play $\pi$ is compliant with $\rho_0$ iff it satisfies 
% \begin{align*}
%     \forall j, \quad v_j = v  \implies v_{j+1} = \rho_0(u\cdot v) 
% \end{align*}
% 
% \begin{proposition} 
%     Fair parity games are determined.
% \end{proposition}
% %The fairness condition in fair parity games can be easily interpreted as a Muller condition. Thus, fair parity games are determined and both of the players have finite memory winning strategies \cite{GH82}. 
% Fair parity games are infinite two player games with perfect information and no randomness where the winning conditions are defined by LTL-formulas (thus, are $\omega-$regular). It follows by Borel determinacy
% theorem that they are determined. % i.e. for all $v\in V_i \cap \mathcal{W}_i$, player $i$ has a finite memory winning strategy.  
% 
% %\vspace{0.3cm}
% %Let us give some definitions before proceeding to \textit{fair positional strategies}.
% % ------------------
% %\begin{definition}
% %Let $\pi = v_1 v_2 \ldots$ be an infinite play. Then $\pi[k] := v_k v_{k+1} \ldots$.
% %\end{definition}
% %\begin{definition}
% % $ind_{\pi}(v):= \{i \mid i\in \mathbb{N}^+ \text{ and } v_i = v\}$ and 
% %$$occ_\pi(v, i) = \begin{cases} 0, \text{if } v_i \neq v, \\
% %    n, \text{ if $v_i = v$ and $v_i$ is the } n^{th} \text{ occurrence of } v \text{ in }\pi.
% %\end{cases}$$
% %\end{definition}
% % \begin{notation}
% $E(v) := \{ w \mid (v,w) \in E\}$ and $E^l(v) := \{ w \mid (v,w) \in E^l\}$.
% % \end{notation}
% 
% %A fair strategy $\rho$ is called \textit{fair positional} iff for all $w_1, w_2 \in V^*$, $\rho(w_1 \cdot v) = \rho(w_2 \cdot v)$. That is, 
% %a fair positional strategy for player $i$ can be defined as $\rho_i : V_i \to V$. 
% 
% %\begin{proof}
% %	\IS{Zielonka's algorithm is a proof }
% %\end{proof}
% >>>>>>> e435efc (proof is almost there)

