\subsection{Proof of Prop.~\ref{prop:mainresult}}\label{app:counter-strategy-templates}
    We will restate the fixed-point formula that calculates the \Odd winning region and the main proposition for the sake of self-containment.
    
% restating prop 2
\begingroup
\def\theproposition{\ref{eq:fp-odd}}
\begin{proposition}\label{app-eq:fp-odd}
    Given an \Odd-fair parity game $\mathcal{G}^\ell = (\ltup{V, \Ve, \Vo, E, \chi}, E^\ell)$ with least even upper bound $l\geq 0$ it holds that $Z=\Wo$, where
    \begin{align}
        Z &:=\textstyle \mu {Y_l}.~  \nu {X_{l-1}}.~  \ldots \mu{Y_2}.~  \nu{X_1}.~  \bigcap_{j \in \ev{2}{l}} \B_j[Y_j, X_{j-1}], \\
        &\text{ where} \quad
        \B_j[\mathbf{Y}, \mathbf{X}] := \left(\textstyle\bigcup_{i \in [j+1,l]} C_i\right) \cup \left(\overline{C_j} \cap \Npre(\mathbf{Y}, \mathbf{X}) \right) \cup \left(C_j \cap \Cpre_\Odd(\mathbf{Y})\right).\nonumber
    \end{align}
     then $Z=\Wo$.
     Further, it takes $\mathcal{O}(n^{l+1})$ symbolic steps to compute $Z$.
\end{proposition}
\addtocounter{proposition}{-1} % decrease the counter that holds proposition numbers, so that the previous restated proposition is not seen.
\endgroup
    
    \begingroup
    \def\theproposition{\ref{prop:mainresult}}
    \begin{proposition}\label{app-prop:mainresult}
        Every player \Odd strategy compliant with $\Sc^{\mathcal{G}^\ell}$ is winning for \Odd in $\mathcal{G}^\ell$.

    \end{proposition}
    \addtocounter{proposition}{-1} % decrease the counter that holds proposition numbers, so that the previous restated proposition is not seen.
    \endgroup

    The main observation behind the proof of Prop.~\ref{app-prop:mainresult} is similar to the main observation in Sec.~\ref{sec:zielonka}, leading to the proof of Alg.~\ref{algo:fair-zielonka-bb}.
    That is, there exists a core subset of the \Odd winning region $\Wo'\subseteq \Wo$, that is added to $Z$ in the first iteration of the 
    fixed-point calculation in ~\eqref{eq:fp-odd}, to which each $v \in \Wo$ can be made to reach by \Odd. 
    Here in particular, we show that any \Odd strategy compliant with $\Sc^{\mathcal{G}^\ell}$ reaches $\Wo'$ (infinitely often) while obeying the fairness condition, and is thus winning for \Odd.

    The proof of Prop.~\ref{prop:mainresult} consists of $3$ main propositions. Before we present them, we will gather some observations from the fixed-point formula ~\eqref{app-eq:fp-odd} and present them as lemmas.
    
    According to our previous definitions, $Y_m^{r_l, r_{l-1}, \ldots, r_m}$ denotes the value of $Y_m$ variable after the $r_{m}^{th}$ iteration on it, while $Y_i, X_i$ variables for $i>m$ are in their ${{r_i}+1}^{th}$ iterations.
    If we flatten this formula we get the following equality: $Y_m^{r_l, r_{l-1}, \ldots, r_m} = $
    $$\nu X_{m-1}\ldots \mu Y_2 \nu X_1. \bigcap_{j \in \ev{m+2}{l}} \B_j[Y_j^{r_j}, X_{j-1}^{r_{j-1}}] \cap \B_m[Y_m^{r_m-1}, X_{m-1}] \cap \bigcap_{j \in \ev{2}{m-2}} \B_j[Y_j, X_{j-1}]$$

    Observe that when the fixed-point above is calculated, all $X_{j}, Y_j$ values for $j < m$ will saturate at the same value,
    which is the final result of the computation. That is, 
    \begin{lemma}\label{app-obs:flat-Z}
    $$ Y_m^{r_l, \ldots, r_m} = \bigcap_{j \in \ev{m+2}{l}} \B_j[Y_j^{r_j}, X_{j-1}^{r_{j-1}}] \cap \B_m[Y_m^{r_m-1},  Y_m^{r_l, \ldots, r_m}] \cap \bigcap_{j \in \ev{2}{m-2}} \B_j[Y_m^{r_l, \ldots, r_m}, Y_m^{r_l, \ldots, r_m}]$$
    \end{lemma}

    \begin{lemma}\label{app-lem:intersection_of_Y_m}   
        For all $v \in \Wo$ with $\rank{v} = (r_l, 0, \ldots, r_2, 0)$. Then,
        $$v \in \bigcap_{j \in \ev{2}{l}} Y_j^{r_l-1, 0, r_{l-2}-1, 0, \ldots, r_{j-2}-1, 0, r_j}$$
    \end{lemma}
        This is similar to our previous observation. $\rank{v} = (r_l, 0, \ldots, r_2, 0)$ implies $v$ was added to the formula while
        $Y_j$ variable was on it's $r_j^{th}$ iteration for all $j \in \ev{2}{l}$. Since $X_{j-1}^0 = V$, the iteration values of $X$ variables can be safely ignored. 
    
    \begin{lemma}\label{app-obs:v-Even-Odd-inequalities} 
        $\quad \text{if } v \in V_\Even, \quad \quad \forall(v, w)\in E, \rank{v}\geq_{l+1-\chi(v)} \rank{w}$
        $$\text{if } v \in V_\Odd, \quad \quad \exists(v, w)\in E, \rank{v}\geq_{l+1-\chi(v)} \rank{w}$$
       % \begin{align*}
       %     \quad\quad\quad\quad\quad\quad\text{if } v \in V_\Even, \quad \quad \forall(v, w)\in E, \rank{v}\geq_{l+1-\chi(v)} \rank{w}\\
       %     \quad\quad\quad\quad\quad\quad\text{if } v \in V_\Odd, \quad \quad \exists(v, w)\in E, \rank{v}\geq_{l+1-\chi(v)} \rank{w}
       % \end{align*}
        where $\rank{v} \geq_b \rank{w}$ denotes the $\geq$ relation in the lexicographic ordering, restricted to the first b elements of the tuple. If $\chi(v)$ is even, the inequalities are strict. 
    \end{lemma}
    \begin{proof}
        Consider a $v$ with $\chi(v) \in \{m-1, m\}$ for some even $m$ and let $\rank{v} = (r_l, 0, \ldots, r_2, 0)$.
        By Lem.~\ref{app-lem:intersection_of_Y_m}, $v \in Y_m^{r_l-1, 0, \ldots, r_{m-2}-1, 0, r_m}$. If we look at the flattening of this formula in Lem.\ref{app-obs:flat-Z}, $v$ is in particular, inside the middle term of this formula. That is,
         \\$v \in \B_m[Y_m^{r_l-1, \hdots, r_m-1}, Y_m^{r_l-1,\hdots, r_m }]$. If we go through the definition of this term we get,
            $$(\bigcup_{i \in [m+1, l]} C_i) \cup (\overline{C_m} \cap \Npre(Y_m^{r_l-1, \hdots, r_m-1}, Y_{m}^{r_l-1,0, \hdots, r_m})) \cup (C_{m} \cap \Cpre_\Odd(Y_m^{r_l-1, 0, \hdots, r_m-1}))$$
        
            \vspace{-7mm}
            \begin{align*}
               \text{ That gives us, } \quad \quad &\text{if } \chi(v) = m, \quad \quad \quad\quad\quad \,\,\, v \in \Cpre_\Odd(Y_m^{r_l-1, 0, \ldots, r_m-1}) \\
                &\text{if } \chi(v) = m-1, \quad \quad \quad\quad v \in \Npre(Y_m^{r_l-1, 0, \ldots, r_m-1}, Y_m^{r_l-1, 0, \hdots, r_m}) \\
        \end{align*}

        By the definition of $\Npre$ we get, if $\chi(v) = m-1$ then $v \in \Cpre_\Odd(Y_m^{r_l-1, 0, \hdots, r_m})$.
       Since odd indices get $0$-ranks, the claim of the lemma follows from the definition of $\Cpre_\Odd$ together with the observation $\rank{v} \geq_{l+1-m} \rank{w} \Leftrightarrow \rank{v} \geq_{l+1-(m-1)} \rank{w}$.
            %from the fact that, if $w \in Y_m^{r_l-1, 0, \hdots, r_m}$, then then first $l+1-m$ of $\rank{w}$ is less than or equal to that of $\rank{v}$. Since odd indices are always $0$, $\rank{v} \geq_{l+1-m} \rank{w} \iff \rank{v} \geq_{l+1-(m-1)} \rank{w}$. 
    \end{proof}

    Now we are ready to introduce the first of our three main propositions:
    
    \begin{proposition}\label{app-prop:Mexists}
        If $\Wo \neq \emptyset$, there exists a non empty set $M := \{ v \in \Wo \mid \rank{v} = (1, 0, 1, 0, \ldots, 1, 0)\}$. Furthermore, for all $v\in M$, $\chi(v)$ is odd.
    \end{proposition}
    Observe that $(1,0,1,0,\ldots, 1,0)$ is the smallest rank possible. Therefore, $v\in M$ are the vertices that were added to $Z$ in ~\eqref{app-eq:fp-odd} in the first iteration of the fixed-point calculation and were never removed.
    The first part of the proposition follows from the monotonicity of fixed-point calculation. That is, if $M$ was empty $Z$ would be empty as well.
    
    For the second part, observe that in the first iteration of the formula, for all $j$, $Y_j = \emptyset$. Also, $\Cpre_\Odd(\emptyset) = \emptyset$. 
    Then from~\eqref{app-eq:fp-odd}, $Z$ does not contain any $v$ with even priority. 
    
    \begin{proposition}\label{app-prop:cycle-through-M}
        All cycles in $\Sc^{\mathcal{G}^\ell}$ that pass through a vertex in $M$ are \Odd winning.% (i.e. the largest priority in the cycle is odd).
    \end{proposition}

    To see why Prop.~\ref{app-prop:cycle-through-M} holds, we make an observation.
    For an even $m\leq l$, let $Y_m^\1$ denote the value of $Y_m$ after the first ever iteration over it is completed, during the computation of ~\ref{eq:fp-odd}.
    I.e. $Y_m^\1 = Y^{0,0,\ldots ,0, 1} $.
    %$$\nu X_{m-1}\ldots \mu Y_2 \nu X_1. \bigcap_{j \in \ev{m+2}{l}} \B_j[\emptyset, V] \cap \B_m[\emptyset, X_{m-1}] \cap \bigcap_{j \in \ev{2}{m-2}} \B_j[Y_j, X_{j-1}]$$
    %In the first term $\B_j$ takes $\emptyset$ and $V$ as arguments. This is due to all $Y_{j}, X_{j-1}$ variables for $j \leq m$ having the values they are initialized with. Observe that when the fixed-point above is calculated, all $X_{j-1}, Y_j$ values for $j < m$ will saturate at the same value,
    %which is the final result of the computation. That is, 
    Since for all $j$, $Y_j^0 = \emptyset$ and $ X^0_{j-1} = V$, Lem.~\ref{app-obs:flat-Z} gives,
    \begin{equation}\label{eq:Ym1}
    Y_m^\1 = \bigcap_{j \in \ev{m+2}{l}} \B_j[\emptyset, V] \cap \B_m[\emptyset, Y_m^\1] \cap \bigcap_{j \in \ev{2}{m-2}} \B_j[Y_m^\1 Y_m^\1]
    \end{equation}
    If we go through the definition of $\B_j$ we see that: the first term of this formula adds or deletes $v \in C_j$ with $j > m$. It adds all the ones with odd $j$  and removes all the ones with even $j$.%is equal to $\bigcup_{j \in \ev{m+2}{l}}C_{j-1} \cup \bigcup_{j \in [1, m+1]}C_j$. That is, the first term eliminates all $v \in C_j$ with even $j>m$ from $Y_m^\1$ and add all $C_j$ with odd $j>m$.
     The last term adds and removes $v \in C_j$ for $j \leq m-2$. It adds the ones in $\Cpre_\Odd(Y_m^\1)$ and removes the ones that are not. The middle term eliminates $C_m$ and all $v \in C_j \cap \neg \Npre(\emptyset, Y_m^\1)$ for $j < m$, and adds $v \in C_{m-1} \cap \Npre(\emptyset, Y_m^\1)$.
    If we go through the definition of $\Npre$, we see that $\Npre(\emptyset, Y_m^\1) = \Cpre_\Odd(Y_m^\1) \cap (V_\Even \cup \Lpre^\forall(Y_m^\1))$.
    This gives,
    \begin{equation}\label{app-eq:obs}
         v \in Y_m^\1 \iff \chi(v)>m\text{ and is odd, or } \chi(v)< m \text{ and } v\in \Npre(\emptyset, Y_m^\1)
    \end{equation}
    %$ Y_m^\1$ consists of $v$ with either odd $\chi(v)>m$, or in $\Npre(\emptyset, Y_m^\1)$.

    Then for all $v \in M$, $v \in Y_m^\1$ for each even $m \leq l$. In particular, $ v \in Y_n^\1$ where $n$ is such that $\chi(v) = n-1$.
    It follows that $ v \in \B_n[\emptyset, Y_n^\1]$. Then, $v \in \Cpre_\Odd(Y_n^\1) \cap (V_\Even \cup \Lpre^\forall(Y_n^\1))$.
    Since all live outgoing edges of $v$ are in $Y_n^\1$, for all $(v,w)$ in $\Sc^{\mathcal{G}^\ell}$, $w \in Y_n^\1$.

    By our previous observation $w$ either has an odd priority larger than $n$, or is in $ \Cpre_\Odd(Y_n^\1) \cap (V_\Even \cup \Lpre^\forall(Y_n^\1))$.
    If $\chi(w)>n$ is odd, then $w \in Y^\1_{\chi(w)+1}$, and we repeat the same argument to conclude the highest priority seen is always odd.

    %Before we present the third proposition, we need a lemma and some observations obtained from formula ~\eqref{eq:fp-odd}.
    %\begin{lemma}\label{obs:v-Even-Odd-inequalities} 
    %%    \begin{align*}OBS
     %       &\quad \quad \quad \quad  \quad \quad \text{if } v \in V_\Even, \quad \quad \forall(v, w)\in E, \rank{v}\geq_{l+1-\chi(v)} \rank{w}\\
     %       &\quad \quad \quad  \quad \quad \quad  \text{if } v \in V_\Odd, \quad \quad \exists(v, w)\in E, \rank{v}\geq_{l+1-\chi(v)} \rank{w}
     %   \end{align*}
     %   where $\rank{v} \geq_b \rank{w}$ denotes the $\geq$ relation in the lexicographic ordering, restricted to the first b elements of the tuples $\rank{v}$ and $\rank{w}$. If $\chi(v)$ is odd, the inequalities are strict. 
    %\end{lemma}
    \begin{definition}
        We call a play $\pi = v_1 v_2 \ldots$ in $\Sc^{\mathcal{G}^\ell}$ \emph{minimal} if for all $v_i \in V_\Odd$, $v_{i+1}$ is the minimum ranked successor of $v_i$. A minimal cycle is a section of a minimal play.
    \end{definition}
    \begin{lemma}\label{app-lem:minimalplayOddwinning}
        Every minimal play is \Odd winning.
    \end{lemma}
    A minimal play only sees minimal cycles. Let $\delta = w_1 w_2 \ldots w_1$ be such a cycle. 
    $\delta$ cannot be an \Even winning cycle: Assume $b := \max\{ \chi(w) \mid w \in \delta\} $ is even. Let $w_i\in \delta$ have priority $b$. By Obs. \ref{app-obs:v-Even-Odd-inequalities}, $\rank{w_i} >_{l+1-b} \rank{w_{i+1}} \geq_{l+1 - \chi(w_{i+1})} \ldots \geq_{l+1-\chi(w_{i-1})} \rank{w_i}$. Since for all $w_j \in \delta$, $\chi(w_{j})\leq b$, the inequality yields $\rank{w_i} >_{l+1-b} \rank{w_i}$, which is a contradiction.

    %The last proposition states that all $\pi$ that starts in \Wo and is compliant with $\Sc^{\mathcal{G}^\ell}$, visits $M$ infinitely often. 
    \begin{proposition}\label{app-prop:minimal-play-visits-M}
        Any minimal play compliant with $\Sc^{\mathcal{G}^\ell}$ visits $M$ infinitely often.
    \end{proposition}
    %Any minimal play sees minimal cycles infinitely often. 
    Let $\delta = w_1 w_2 \ldots w_1$ be a minimal cycle and $w_k$ its vertex with maximum priority. We will show that $w_k \in M$. Since $\pi = \delta \delta \ldots$ is a minimal play, by Lemma.~\ref{app-lem:minimalplayOddwinning} we know $\chi(w_k)$ is odd. Furthermore, we have observed in \ref{app-eq:obs} that $w_k \in Y_m^\1$ for all $m > \chi(w_k)$. 
    If we can show that $w_k \in Y_m^\1$ also for $m < \chi(w_k)$, then we have $w_k \in M$. We will now show this. 
    
    Assume to the contrary that $w_k \not \in M$ and let $j$ be the largest non-trivial index of $\rank{w_k}$. 
    That is $j < l$ is the largest even integer such that $w_k \not \in Y_j^\1$. Let $t$ be the value of this index, i.e. $w_k \in Y_j^{0,\ldots, 0,t} \setminus Y_j^{0,\ldots, 0,{t-1}}$. 
    Let us denote $Y_j^{0, \ldots, 0, t}$ by $Y_j^\te$ for short. 

    Since $\delta$ is minimal, Lem.~\ref{app-obs:v-Even-Odd-inequalities} gives $\rank{w_i} \geq_{l+1 - \chi(w_i)} \rank{w_{i+1}}$ for all $w_i \in \delta$. Since $\chi(w_i) \leq \chi(w_k)$ for all $i$ and $\chi(w_k) < j$; $\rank{w_i} \geq_{l+1-j} \rank{w_{i+1}}$ for all $w_i \in \delta$. 
     This implies $\rank{w} =_{l+1-j} \rank{w'}$ for all $w, w'\in \delta$. It follows that for all $w\in \delta$, $w \in Y_j^\te \setminus Y_j^{\mathbf{t-1}}$.
 
     %We can follow the same steps in equation~\eqref{eq:Ym1} to observe that 
   %$\forall (v,w)$ in $\Sc^{\mathcal{G}^\ell}$, $w \in \Npre(Y^{\mathbf{t-1}}_j, Y^\te_j) = \Cpre_\Odd(Y^\te_j) \cap (V_\Even \cup \Lpre^\forall(Y^\te_j) \cup Pre^\exists_\Odd(Y^\mathbf{t-1}_j))$.
   %If $w \in Pre^\exists_\Odd(Y_j^{\mathbf{t-1}})$, since $\delta$ is a minimal cycle, $\delta$ will have an element from $Y_j^{\mathbf{t-1}}$. However, this contradicts our observation that $\delta$ lies in $Y_j^\te \setminus Y_j^{\mathbf{t-1}}$.
   %On the other hand if non of the $w \in \delta $ lie in $ Pre^\exists_\Odd(Y_j^{\mathbf{t-1}})$ this implies that they all get into the formula due to reaching other nodes in $Y_j^\te \setminus Y_j^\mathbf{t-1}$. This is not possible since a node in $\delta$ has to be added to $Y_j^\te \setminus Y_j^\mathbf{t-1}$ as the first node and thus, have to have a successor in $Y_j^\mathbf{t-1}$. % while $Y_j^\te \setminus Y_j^{\mathbf{t-1}}$ is empty.
   %  Therefore, $w_k\in M$.

     Once more by Lem.~\ref{app-obs:flat-Z} we get that for all $w \in \delta$, 
     $$w \in \B_j[Y_j^{\mathbf{t-1}}, Y_j^\te] = (\bigcup_{i\in[j+1, l]} C_i) \cup ( \overline{C_j} \cap \Npre(Y_j^{\mathbf{t-1}}, Y_j^\te) \cup (C_j \cap \Cpre_\Odd(Y_j^{\mathbf{t-1}})))$$
     Since $\chi(w) < j$, this implies $$w \in \Npre(Y_j^{\mathbf{t-1}}, Y_j^\te) = \Cpre_\Odd(Y_j^\te) \cap (V_\Even \cup \Lpre^\forall(Y_j^\te) \cap \Pre^\exists_\Odd(Y_j^{\mathbf{t-1}}) )$$
     %Since we assumed all $w \in Y_j^\te \setminus Y_j^{\mathbf{t-1}}$, $w$ cannot be in the last term.
     Now consider the set $Y_j^\te \setminus Y_j^{\mathbf{t-1}}$, which is initially empty. Then the first term in $\delta$ that gets in $Y_j^\te \setminus Y_j^{\mathbf{t-1}}$ has to be in $\Pre^\exists_\Odd(Y_j^{\mathbf{t-1}})$. 
     This contradicts our assumption that all $w_i \in  Y_j^\te \setminus Y_j^{\mathbf{t-1}}$ and proves that $w_k \in M$.
     We are now ready to prove the main theorem.
     \begin{proof}[Proof of Thm. \ref{prop:mainresult}]
          Let $\pi= v_0v_1\ldots$ be a play compliant with $\Sc^{\mathcal{G}^\ell}$ with $v_0 \in \Wo$. %Since all plays compliant with \Odd strategy templates obey the fairness condition, we only need to show that the maximum priority seen infinitely  in $\pi$ is odd.
          Since $\pi$ is compliant with an \Odd strategy template, it is a fair play. 
          For a node $v \in \Wo$, let $v_{\min}$ be the minimum ranked successor of $v$.
          Since $\pi$ is fair, for all $v$ that is visited infinitely often in $\pi$, $v_{\min}$ is visited infinitely often as well. 
          This gives us an infinite subsequence of $\pi$ that is minimal. Since all minimal plays visit $M$ infinitely often (Prop.~\ref{app-prop:minimal-play-visits-M}), 
          $\pi$ visists $M$ infinitely often. Then there must exist an $x \in M$ that $\pi$ visits infinitely often. 
          Then a tail of $\pi$ is consisted of consecutive cycles over $x$. Since all cycles that pass through $M$ are \Odd winning (Prop.~\ref{app-prop:cycle-through-M}), $\pi$ is \Odd-winning.
   \end{proof}
    
%     
%     
% %   The main insight behind the proof of Prop.~\ref{prop:mainresult} is actually similar to the one enabling the proof of the \Odd-fair Zielonka's algorithm given in Sec.~\ref{} and consists of three steps: 
%   
%   It shows that there exists a core subset of the \Odd winning region $M\subseteq \Wo$, that is added to $Z$ in the first iteration of the 
%     fixed-point calculation in ~\eqref{eq:fp-odd}, to which each $v \in \Wo$ can be forced to reach by \Odd. 
%     
%     
%     
%     Here in particular, we show that any \Odd strategy compliant with $\Sc^{\mathcal{G}^\ell}$ reaches $\Wo'$ (infinitely often) while obeying the fairness condition, and is thus winning for \Odd.
% 
%     The full proof of Prop.~\ref{prop:mainresult} can be found in App.~\ref{??}. It consists of $3$ main propositions which we present here one-by-one along with some intuition on why they hold.
% 
%     \begin{proposition}\label{prop:Mexists}
%         If $\Wo \neq \emptyset$, there exists a non empty set $M := \{ v \in \Wo \mid \rank{v} = (1, 0, 1, 0, \ldots, 1, 0)\}$.
%     \end{proposition}
%     Observe that $(1,0,1,0 ,\ldots, 1, 0)$ is the minimum rank the ranking function assigns to a vertex. Also, the vertices in $M$ are exactly the vertices that are added to $Z$ during the
%     first iteration of the fixed-point calculation and are never removed. The existence of such a set is apparent from the fact that, each vertex $v \in \Wo$ that has a non-minimum rank, is in the set because 
%     of the vertices with smaller ranks, i.e. the vertices that got included to $Z$ prior to $v$. This requires a set of vertices that were added to $Z$ prior to all the others. 
% %     
%     Additionally, from \eqref{eq:fp-odd} we gather the observation that all $v \in M$ have odd priorities.
%     \begin{proposition}\label{prop:cycle-through-M}
%         All cycles in $\Sc^{\mathcal{G}^\ell}$ that pass through a vertex in $M$ are \Odd winning.% (i.e. the largest priority in the cycle is odd).
%     \end{proposition}
% 
%     To see why Prop.~\ref{prop:cycle-through-M} holds, we need to make an observation. 
%     For an even $m\leq l$, let $Y_m^\1$ denote the value of $Y_m$ after the first ever iteration over it is completed, during the computation of \eqref{eq:fp-odd}.
%     I.e. $Y_m^\1 = Y^{0,0,\ldots ,0, 1} = $
%     $$\nu X_{m-1}\ldots \mu Y_2 \nu X_1. \bigcap_{j \in \ev{m+2}{l}} \B_j[\emptyset, V] \cap \B_m[\emptyset, X_{m-1}] \cap \bigcap_{j \in \ev{2}{m-2}} \B_j[Y_j, X_{j-1}].$$
%     In the first term $\B_j$ takes $\emptyset$ and $V$ as arguments. This is due to all $Y_{j}, X_{j-1}$ variables for $j \leq m$ having the values they are initialized with. Observe that when the fixed-point above is calculated, all $X_{j-1}, Y_j$ values for $j < m$ will saturate at the same value,
%     which is the final result of the computation. That is, 
%     \begin{equation}\label{eq:Ym1}
%     Y_m^\1 = \bigcap_{j \in \ev{m+2}{l}} \B_j[\emptyset, V] \cap \B_m[\emptyset, Y_m^\1] \cap \bigcap_{j \in \ev{2}{m-2}} \B_j[Y_m^\1 Y_m^\1].
%     \end{equation}
%     If we go through the definition of $\B_j$ we see that: the first term of this formula adds or deletes $v \in C_j$ with $j > m$. It adds all the ones with odd $j$  and removes all the ones with even $j$.%is equal to $\bigcup_{j \in \ev{m+2}{l}}C_{j-1} \cup \bigcup_{j \in [1, m+1]}C_j$. That is, the first term eliminates all $v \in C_j$ with even $j>m$ from $Y_m^\1$ and add all $C_j$ with odd $j>m$.
%      The last term adds and removes $v \in C_j$ for $j \leq m-2$. It adds the ones in $\Cpre_\Odd(Y_m^\1)$ and removes the ones that are not. The middle term eliminates $C_m$ and all $v \in C_j \cap \neg \Npre(\emptyset, Y_m^\1)$ for $j < m$, and adds $v \in C_{m-1} \cap \Npre(\emptyset, Y_m^\1)$.
%     If we go through the definition of $\Npre$, we see that $\Npre(\emptyset, Y_m^\1) = \Cpre_\Odd(Y_m^\1) \cap (V_\Even \cup \Lpre^\forall(Y_m^\1))$.
%     This gives us the observation,
%     \begin{equation}\label{eq:obs}
%         \text{If  } v \in Y_m^\1 \text{ then either } \chi(v)>m\text{ and is odd, or }  v\in \Npre(\emptyset, Y_m^\1).
%     \end{equation}
%     %$ Y_m^\1$ consists of $v$ with either odd $\chi(v)>m$, or in $\Npre(\emptyset, Y_m^\1)$.
% 
%     Now observe that, for every $v \in M$, $v \in Y_m^\1$ for each even $m \leq l$. In particular, $ v \in Y_n^\1$ where $n$ is the even number for which $\chi(v) = n-1$.
%     It follows that $ v \in \B_n[\emptyset, Y_n^\1]$. Then, $v \in \Cpre_\Odd(Y_n^\1) \cap (V_\Even \cup \Lpre^\forall(Y_n^\1))$.
%     Since all live outgoing edges of $v$ are in $Y_n^\1$, for all $(v,w)$ in $\Sc^{\mathcal{G}^\ell}$, $w \in Y_n^\1$.
% 
%     By our previous observation $w$ either has an odd priority larger than $n$, or is in $ \Cpre_\Odd(Y_n^\1) \cap (V_\Even \cup \Lpre^\forall(Y_n^\1))$.
%     If $\chi(w)>n$ is odd, then $w \in Y^\1_{\chi(w)+1}$, and we repeat the same argument to conclude the highest priority seen is always odd.
% 
%     Before we present the third proposition, we need a lemma and some observations obtained from formula ~\eqref{eq:fp-odd}.
%     \begin{observation}\label{obs:v-Even-Odd-inequalities} 
%         \begin{align*}
%             &\quad \quad \quad \quad  \quad \quad \text{if } v \in V_\Even, \quad \quad \forall(v, w)\in E, \rank{v}\geq_{l+1-\chi(v)} \rank{w}\\
%             &\quad \quad \quad  \quad \quad \quad  \text{if } v \in V_\Odd, \quad \quad \exists(v, w)\in E, \rank{v}\geq_{l+1-\chi(v)} \rank{w}
%         \end{align*}
%         where $\rank{v} \geq_b \rank{w}$ denotes the $\geq$ relation in the lexicographic ordering, restricted to the first b elements of the tuples $\rank{v}$ and $\rank{w}$. If $\chi(v)$ is odd, the inequalities are strict. 
%     \end{observation}
%     We call a play $\pi = v_1 v_2 \ldots$ in $\Sc^{\mathcal{G}^\ell}$ \emph{minimal} if for all $v_i \in V_\Odd$, $v_{i+1}$ is the minimum ranked successor of $v_i$. We call a cycle minimal, if it is a section of a minimal play.
%     \begin{lemma}\label{lemma:minimalplayOddwinning}
%         Every minimal play is \Odd winning.
%     \end{lemma}
%     This lemma follows from Obs. \ref{obs:v-Even-Odd-inequalities}. A minimal play only sees minimal cycles. Let $\delta = w_1 w_2 \ldots w_1$ be such a cycle. 
%     $\delta$ cannot be an \Even winning cycle: Assume $b := max\{ \chi(w) \mid w \in \delta\} $ is even. Let $w_i\in \delta$ have priority $b$. Then by Obs. \ref{obs:v-Even-Odd-inequalities}, $\rank{w_i} >_{l+1-b} \rank{w_{i+1}} \geq_{l+1 - \chi(w_{i+1})} \ldots \geq_{l+1-\chi(w_{i-1})} \rank{w_i}$. Since for all $w_j \in \delta$, $\chi(w_{j})\leq b$, the previous inequality yields $\rank{w_i} >_{l+1-b} \rank{w_i}$, which is a contradiction.
% 
%     %The last proposition states that all $\pi$ that starts in \Wo and is compliant with $\Sc^{\mathcal{G}^\ell}$, visits $M$ infinitely often. 
%     \begin{proposition}\label{prop:minimal-play-visits-M}
%         Any minimal cycle in $\Sc^{\mathcal{G}^\ell}$ visits $M$.
%     \end{proposition}
%     Let $\delta = w_1 w_2 \ldots w_1$ be a minimal cycle and $w_k$ it vertex with a maximum prioirty. We will show that $w_k \in M$. By Lemma.~\ref{lemma:minimalplayOddwinning} we know $\chi(w_k)$ is odd. Furthermore, we have observed in \ref{eq:obs} that $w_k \in Y_m^\1$ for all $m > \chi(w_k)$. 
%     If we can show that $w_k \in Y_m^\1$ also for $m < \chi(w_k)$, then clearly $w_k \in M$. We will now show this. 
%     Assume to the contrary that $w_k \not \in M$ and let $j$ be the largest non-trivial index of $\rank{w_k}$. 
%     That is $j < l$ is the largest even integer such that $w_k \not \in Y_j^\1$. Let $t$ be the value of this index, i.e. $w_k \in Y_j^{0,\ldots, 0,t} \setminus Y_j^{0,\ldots, 0,{t-1}}$. 
%     Let us denote $Y_j^{0, \ldots, 0, t}$ by $Y_j^\te$ for short. 
%     
%     Since $\delta$ is minimal, Obs.~\ref{obs:v-Even-Odd-inequalities} gives us $\rank{w_i} \geq_{l+1 - \chi(w_i)} \rank{w_{i+1}}$ for all $w_i \in \delta$. Since $\chi(w_i) \leq \chi(w_k)$ for all $i$ and $\chi(w_k) < j$; $\rank{w_i} \geq_{l+1-j} \rank{w_{i+1}}$ for all $w_i \in \delta$. 
%     Since $\delta$ is a cycle, thise implies $\rank{w} =_{l+1-j} \rank{w'}$ for all $w, w'\in \delta$. This gives us that, for all $w\in \delta$, $w \in Y_j^\te \setminus Y_j^{\mathbf{t-1}}$.
% 
%     We can follow the same steps in equation~\eqref{eq:Ym1} to observe that 
%   $\forall (v,w)$ in $\Sc^{\mathcal{G}^\ell}$, $w \in \Npre(Y^{\mathbf{t-1}}_j, Y^\te_j) = \Cpre_\Odd(Y^\te_j) \cap (V_\Even \cup \Lpre^\forall(Y^\te_j) \cup Pre^\exists_\Odd(Y^\mathbf{t-1}_j))$.
%   If $w \in Pre^\exists_\Odd(Y_j^{\mathbf{t-1}})$, since $\delta$ is a minimal cycle, $\delta$ will have an element from $Y_j^{\mathbf{t-1}}$. However, this contradicts our observation that $\delta$ lies in $Y_j^\te \setminus Y_j^{\mathbf{t-1}}$.
%   On the other hand if non of the $w \in \delta $ lie in $ Pre^\exists_\Odd(Y_j^{\mathbf{t-1}})$ this implies that they all get into the formula due to reaching other nodes in $Y_j^\te \setminus Y_j^\mathbf{t-1}$. This is not possible since a node in $\delta$ has to be added to $Y_j^\te \setminus Y_j^\mathbf{t-1}$ as the first node and thus, have to have a successor in $Y_j^\mathbf{t-1}$. % while $Y_j^\te \setminus Y_j^{\mathbf{t-1}}$ is empty.
%     Therefore, $w_k\in M$.
% 
% %    We are now ready to prove the main theorem.
% %    \begin{proof}[Proof of Thm. \ref{thm:mainresult}]
% %         Let $\pi= v_0v_1\ldots$ be a play compliant with $\Sc^{\mathcal{G}^\ell}$ and $v_0 \in \Wo$. %Since all plays compliant with \Odd strategy templates obey the fairness condition, we only need to show that the maximum prioirty in $\pi$ is odd.
% %         Since $\pi$ is compliant with an \Odd strategy template, it obeys the fairness condition. 
% %         The fact that $\pi$ visits minimal cycles infinitely often follows from the fact that whenever a $v \in V_\Odd$ is seen infinitely often in $\pi$, $(v, v_{min})$ should be seen infinitely often as well.
% %         This gives by induction that, a minimal cycle that passes through $v$ should be visited infinitely often. Due to Prop.~\ref{prop:minimal-play-visits-M}, we know that therefore, $\pi$ visits $M$ infinitely often. Since $M$ is finite, $\pi$ visits a $x \in M$ infinitely often. Thus, a tail of $\pi$ can be seen as consecutive cycles over $x$. Since by Prop.~\ref{prop:cycle-through-M}, all cycles that pass through $M$ are \Odd winning, $\pi$ is \Odd winning.  
% %     \end{proof}
