\subsection{Experimental Results}
% \vspace{-1mm}

We conducted an experimental study to empirically validate the claim that our new \Odd-fair Zielonka's algorithm retains its efficiency in practice (see App.~\ref{app:experiments} for details). 

We generated \Odd-fair parity instances manipulating $286$ benchmark instances of PGAME$\_$ Synth$\_$2021 dataset of the SYNTCOMP benchmark suite~\cite{syntcomp} and $51$ instances of PGSolver dataset of Keiren's benchmark suite~\cite{keirens} by adding live edges to the given (normal) parity games.
%We generated $286$ benchmark instances from the PGAME$\_$Synth$\_$2021 dataset of the SYNTCOMP benchmark suite~\cite{syntcomp} and $51$ benchmark instances from the PGSolver dataset of Keiren's benchmark suite~\cite{keirens} by adding live edges to the given (normal) parity games. 
We empirically compared the (non-optimized\footnote{While optimized version of \texttt{N-ZL} and \texttt{N-FP} are available in \texttt{oink} \cite{oink} our goal is a conceptual comparison, which is better achieved by similar (non-optimized) implementations for all algorithms.}) C++-based implementations of 
% For this, we implemented the following algorithms  in C++:
\begin{inparaenum}[(i)]
 \item the \Odd-fair Zielonka's algorithm (\texttt{OF-ZL}) from Alg.~\ref{algo:fair-zielonka-bb},
 \item the \enquote{normal} Zielonka's algorithm (\texttt{N-ZL}) from~\cite{Zielonka98}, 
 \item the fixed-point algorithm for \Odd-fair parity games (\texttt{OF-FP}) from~\cite{banerjee2022fast} implementing~\eqref{eq:fp-odd}, and
 \item the \enquote{normal} fixed-point algorithm (\texttt{N-FP}) for \enquote{normal} parity games from~\cite{EJ91}. % \todo{IS: a reviewer correctly asked us to add \texttt{N-FP} to the implemented algorithms list as well, and I did. Here for normal parity fixed-point algorithm should I cite an old paper that gives the parity fixed-point, or is it enough to cite Banerjee et. al. agan (as I did here) since the fixed-point formulation for regular parity is given there as well.}
\end{inparaenum}
On the \emph{SYNTCOMP benchmarks}, the time-out rates are: $82$ instances for \texttt{OF-FP}, $58$ for \texttt{OF-ZL}; $73$ for \texttt{N-FP} and $47$ for \texttt{N-ZL}. On the 204 instances that neither of the algorithms time out the average computation times are: $122.7$ seconds for \texttt{OF-FP}, $4.6$ seconds for \texttt{OF-ZL}, $45.2$ seconds for \texttt{N-FP} and $3.6$ seconds for \texttt{N-ZL}.   
For all instances that did not time out for all four algorithms, Fig.~\ref{fig:logscale-main} shows scatter plots comparing the computation times of \texttt{OF-ZL} with \texttt{OF-FP} (left) and  \texttt{OF-ZL} with \texttt{N-ZL} (right) using logarithmic scaling. The diagonal shows instances with similar computation times. Points above the diagonal show superior performance of \texttt{OF-ZL}.
For the \emph{PGSolver dataset} \texttt{OF-FP} timed out on all generated instances, whereas \texttt{OF-ZL} took $24.9$ seconds on average to terminate.

% Figure environment removed

We clearly see that  \texttt{OF-ZL} performs up to one order of magnitude better than \texttt{OF-FP} in many instances while \texttt{OF-ZL} and \texttt{N-ZL} perform very similar on the given benchmark instances. In addition, we observe that \texttt{OF-FP} starts timing out as soon as the examples became more complex. %, being especially sensitive to the increase in the number of priorities. 
%  On the other hand, \texttt{OF-ZL} preserves its performance considerably in the face of the increase in the same parameters. 
These outcomes match the known comparison results between the naive fixed-point calculation versus Zielonka's algorithm, on normal parity games.  


% \IS{In this section I just filled the previous experiment results with the new results. The new comparisons are in the appendix. Moreover, we did not run out examples on Keiren's benchamrk suite yet, but I told Munko to do it for examples with $\leq 1000$ nodes, as a last thing.}
% 
% We have implemented our \Odd-fair Zielonka algorithm (called \textsc{OF-ZL}) and the fixed-point algorithm from \eqref{eq:fp-odd} (called \textsc{OF-FP}) in a C++-based prototype tool for empirical evaluation over the PGAME$\_$Synth$\_$2021 dataset of the SYNTCOMP benchmark suite~\cite{syntcomp} and the small examples\footnote{at most 1000 nodes} of PGSolver dataset of Keiren's benchmark suite~\cite{keirens}. These datasets contain examples of normal parity games. For each example we have generate two \Odd-fair parity game instances by randomly selecting live edges (see App.~\ref{app:experiments} for details). All experiments were performed on a machine equipped with Intel(R) Core(TM) i5-6600 CPU @ 3.30GHz and 8GB RAM. We declare a timeout when the calculation of an example exceeds 1 hour.
% 
% 
% We first observed that \textsc{OF-FP} times out on \emph{all} instances we take from the PGSolver dataset, while \textsc{OF-ZL} terminated on all instances with $--$\todo{IS: waiting for Munko's results} seconds average computation time. Further, \textsc{OF-FP} times out on $82$ out of $286$ games from the PGAME$\_$Synth$\_$2021 dataset\todo{IS: though we cannot take all the games and maybe we should somehow mention this}, while \textsc{OF-ZL} only times out on $58$ instances. For the instances where both solvers compute a solution, \textsc{OF-ZL} is faster in all cases.
% When considering only the instances where both algorithms do not time out, we obtain an average computation time of $72.3$ seconds for \textsc{OF-FP} and $3.5$ seconds for \textsc{OF-ZL}. %This re-confirms the computational adantage of Zielonka's algorithm over fixed-point algorithms known from normal parity games. 
% Further, we also observed a dependence of \textsc{OF-FP} on the size of the game graph, while \textsc{OF-ZL} preserves its performance (see App.~\ref{app:experiments} for details).\todo{IS: Zl also times out when the size of the graph is too big.}
% These outcomes match the known comparison results between the fixpoint calculation versus Zielonka's algorithm, on normal parity games.  

%PGSolver dataset of Keiren's benchmark suite contains parity games generated using PGSolver. %It includes random games, and games that are hard for certain algorithms. 
%All the parity game graphs here have $1000$ nodes and the nodes have priorities between $0$ and $10$.
%On PGSolver dataset, due to an increased complexity in the size of the games, the fixed-point algorithm timed out on all of the examples. Whereas, Zielonka's algorithm took 53 sec. on average to compute. 
% Details of the experiments can be found in the appendix, Sec.~\ref{app:experiments}.
