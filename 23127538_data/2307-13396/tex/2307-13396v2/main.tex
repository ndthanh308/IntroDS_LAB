    \documentclass[a4paper,UKenglish,cleveref, autoref, thm-restate]{lipics-v2021}
%This is a template for producing LIPIcs articles. 
%See lipics-manual.pdf for further information.
%for A4 paper format use option "a4paper", for US-letter use option "letterpaper"
%for british hyphenation rules use option "UKenglish", for american hyphenation rules use option "USenglish"
%for section-numbered lemmas etc., use "numberwithinsect"
%for enabling cleveref support, use "cleveref"
%for enabling autoref support, use "autoref"
%for anonymousing the authors (e.g. for double-blind review), add "anonymous"
%for enabling thm-restate support, use "thm-restate"

%\graphicspath{{./graphics/}}%helpful if your graphic files are in another directory

\bibliographystyle{plainurl}% the mandatory bibstyle

\usepackage{hyperref}


% Packages
\usepackage{amsmath,amssymb,amsfonts}
\usepackage{graphicx}
\usepackage{textcomp}
\usepackage[binary-units=true]{siunitx}
\usepackage{comment}
\usepackage{amsmath}
%\usepackage{amsthm}
\usepackage{etoolbox}
\usepackage{color,colortbl}
\usepackage{mathtools}
\usepackage{enumerate}
\usepackage{algorithm, algpseudocode}
\usepackage{booktabs}
\usepackage[normalem]{ulem}
\usepackage{pgf}
\usepackage{pgfplotstable}
\usepackage{tikz,pgfplots}
\usetikzlibrary{trees,decorations,arrows,arrows.meta,automata,shadows,positioning,plotmarks,backgrounds,shapes,shapes.misc}
\usetikzlibrary{calc,matrix,fit,petri,decorations.pathmorphing,patterns}
\usetikzlibrary{decorations.pathreplacing,decorations.markings,shapes.geometric,calc}
\usetikzlibrary{tikzmark}
\usepackage{paralist}
\usepackage{stmaryrd}
\usepackage{xspace}
\usepackage{graphicx}
\usepackage{float}
\usepackage[utf8]{inputenc} 
  \usepackage{csquotes} 
%  \usepackage{algorithm}
% \usepackage{subfig}
\usepackage{multirow}
\usepackage{array}
\usepackage{dsfont}
 \usepackage{array}
\usepackage{xifthen}
\usepackage{relsize}
\usepackage{xfrac}
\usepackage{wrapfig}
\usepackage{rotating}
\usepackage{paralist}

\usepackage{longtable}
\usepackage{caption}
\usepackage[position=b]{subcaption}

%%%%
% Provide the command \fpeval as a copy of the code-level \fp_eval:n.
%\usepackage{expl3}[2012-07-08]
%\ExplSyntaxOn
%\cs_new_eq:NN \fpeval \fp_eval:n
%\ExplSyntaxOff
%%%%

%---------- TikZ setup ----------
\colorlet{darkgreen}{green!80!black}
\colorlet{darkred}{red!80!black}
\usetikzlibrary{arrows, automata, shapes}
\tikzset{auto, >= stealth}
\tikzset{every edge/.append style={thick, shorten >= 1pt}}
\tikzset{initial/.style={draw, thick, <-, shorten <=1pt}}
\tikzset{player0/.style = {draw, thick, shape=circle, minimum size=5mm}}
\tikzset{player1/.style = {draw, thick, shape=rectangle, minimum size=5mm}}
\newcommand\pos{1.4}


%######################REMOVE BEFORE PUBLISHING####################################

% -------------------Margin notes--------------------------

\usepackage{marginnote}
\usepackage{schemata}


% ---------------------Todo setup--------------------------------------

\usepackage[
%  disable, %turn off todonotes
colorinlistoftodos, %enable a coloured square in the list of todos
textwidth=\marginparwidth, %set the width of the todonotes
textsize=scriptsize, %size of the text in the todonotes
]{todonotes}


% 2 player games
\newcommand{\p}[1]{\mathit{Player}~#1}
\newcommand{\pz}{\p{0}}
\newcommand{\po}{\p{1}}
\newcommand{\game}{\mathcal{G}}
\newcommand{\vertex}{V}
\newcommand{\vertexz}{\vertex^0}
\newcommand{\vertexo}{\vertex^1}
\newcommand{\vertexi}{\vertex^i}
\newcommand{\vin}{v_{\mathsf{init}}}
\newcommand{\win}{\Phi}
%\newcommand{\conc}{\mathit{Conc}}
\newcommand{\gamegraph}{G}
\newcommand{\spec}{\Phi}
%\newcommand{\Bh}{\widehat{B}}
\newcommand{\edge}{E}
\newcommand{\stratz}{\pi^0}
\newcommand{\strato}{\pi^1}
\newcommand{\strati}{\pi^i}
\newcommand{\Stratz}{\Pi^0}
\newcommand{\Strato}{\Pi^1}
\newcommand{\Strati}{\Pi^i}
\newcommand{\Stratzml}{\Pi^{0,\mathrm{ML}}}
\newcommand{\Stratoml}{\Pi^{1,\mathrm{ML}}}
\newcommand{\play}{\rho}
\newcommand{\col}{\mathit{Col}}
\makeatletter
\newsavebox{\@brx}
\newcommand{\llangle}[1][]{\savebox{\@brx}{\(\m@th{#1\langle}\)}%
  \mathopen{\copy\@brx\kern-0.5\wd\@brx\usebox{\@brx}}}
\newcommand{\rrangle}[1][]{\savebox{\@brx}{\(\m@th{#1\rangle}\)}%
  \mathclose{\copy\@brx\kern-0.5\wd\@brx\usebox{\@brx}}}
\renewcommand{\algorithmicrequire}{\textbf{Input:}}
\renewcommand{\algorithmicensure}{\textbf{Output:}}
\makeatother
\newcommand{\team}[1]{\llangle #1\rrangle}
\newcommand{\src}{\mathit{src}}
\newcommand{\trg}{\mathit{trg}}
\newcommand{\rankfun}{\mathsf{rank}}
\newcommand{\front}{\mathit{front}}
\newcommand{\recurring}{\mathit{inf}}


\newcommand{\set}[1]{\left\lbrace #1\right\rbrace}
\newcommand{\Set}[1]{\set{#1}}
%\newcommand{\tup}[1]{\left( #1\right)}
%\newcommand{\ltup}[1]{\langle #1\rangle}
\newcommand{\ap}{\mathit{AP}}
\newcommand{\inodd}{\in_{\mathrm{odd}}}
\newcommand{\ineven}{\in_{\mathrm{even}}}
\newcommand{\false}{\mathsf{False}}
\newcommand{\lang}{\mathcal{L}}

\newcommand{\A}{\mathcal{A}}
\newcommand{\Sc}{\mathcal{S}}
% \newcommand{\N}{\mathcal{N}}
\newcommand{\B}{\mathcal{B}}
\newcommand{\Bb}{\mathbf{B}}
\newcommand{\Bd}{\dot{\mathbf{B}}}
% \newcommand{\Xo}{\overline{X}}
% \newcommand{\Yo}{\overline{Y}}

\newcommand\calF{\mathcal{F}}
\newcommand\calG{\mathcal{G}}
\newcommand\calM{\mathcal{M}}
\newcommand\calV{\mathcal{V}}
\newcommand\calU{\mathcal{U}}
\newcommand\calW{\mathcal{W}}
\newcommand\calP{\mathcal{P}}
\newcommand\calD{\mathbb{D}}
%%%%%%%%%%%%%%%%%
%% macros introduced by Luke 
\newcommand\mydef[1]{{\bf\em #1}}
%%%%%%%%%%%%%%%%%

\newcommand{\numviparams}{{| \lambda |}}
\newcommand{\scoreaccvars}[1]{s_1^{#1}, \ldots, s_{\numviparams}^{#1}}
\newcommand{\scoreaccvar}[2]{s_{#1}^{#2}}
\newcommand{\isdeterm}[1]{\text{Deterministic}({#1})}


\newcommand{\expect}[1]{\mathbb{E}\left[{#1}\right]}
\newcommand{\var}[1]{\mathbb{V}\left[ {#1} \right]}
\newcommand{\expectdist}[2]{\mathbb{E}_{#1}\left[ {#2} \right]}
\newcommand{\vardist}[2]{\mathbb{V}_{#1}\left[ {#2} \right]}
\newcommand{\cov}[2]{\mathbb{C}\text{ov}[{#1}][{#2}]}
\newcommand{\covv}[1]{\mathbb{C}\text{ov}[{#1}]}
\newcommand{\corr}[1]{\mathbb{C}\text{orr}[{#1}]}

\newcommand{\fix}[1]{\mathit{fix}\left({#1}\right)}
\newcommand{\sbr}[1]{\left\llbracket {#1} \right\rrbracket}
\newcommand{\ctxtype}[3]{{#1} \cong_\text{ctx} {#2} : {#3}}
\newcommand{\bigstep}[3]{{#1} \Downarrow_{#2} {#3}}


% PCF types
\newcommand{\bool}{\mathit{bool}}
\newcommand{\nat}{\mathit{nat}}

\newcommand{\ctx}[1]{\mathcal{C}\left[ {#1}\right] }
\newcommand{\pcft}[1]{\text{PCF}_{#1}}

\newcommand{\nfl}{\mathbb{N}_\bot}
\newcommand{\bfl}{\mathbb{B}_\bot}

% PCF constructs
\newcommand{\succc}[1]{\mathbf{succ}({#1})}
\newcommand{\succcn}[2]{\mathbf{succ}^{#1}({#2})}
\newcommand{\zero}{\mathbf{0}}
\newcommand{\zerotest}[1]{\mathbf{zero}\left({#1}\right)}
\newcommand{\pred}[1]{\mathbf{pred}\left( {#1} \right)}
\newcommand{\predn}[2]{\mathbf{pred}^{#1}\left( {#2} \right)}
\def\solvable{\#}

\newcommand{\true}{\mathbf{true}}
\newcommand{\false}{\mathbf{false}}
\newcommand{\pcffix}[1]{\mathbf{fix}\left({#1}\right)}
\newcommand{\pcffn}[3]{\mathbf{fn}~{#1}:{#2}\mathpunct{.}{#3}}
\newcommand{\pairtype}[2]{{#1} * {#2}}
\newcommand{\pairexp}[2]{\mathbf{pair}({#1}, {#2})}
\newcommand{\leftexp}[1]{\mathbf{left}({#1})}
\newcommand{\rightexp}[1]{\mathbf{right}({#1})}

\newcommand{\RationalPos}{\mathbb{Q}^{+}}

\newcommand{\meas}[1]{\mathbb{M}\left( {#1} \right) }
\newcommand{\integ}[1]{\sbr{#1}_I}

\newcommand{\notbigstep}[2]{{#1}~\cancel{\Downarrow}_{#2}}
\newcommand{\subtrace}[3]{{#1}^{{#2} \ldots {#3}}}
\newcommand{\supp}[1]{\textsf{supp}\left({#1}\right)}
\newcommand{\dom}[1]{\textsf{Dom}\left({#1}\right)}
\newcommand{\suppk}[2]{\textsf{Supp}^{#1}\left({#2}\right)}
\newcommand{\tracespace}{\bigcup_{n \in \mathbb{N}}[0, 1]^n}
\newcommand{\generictracespace}{\mathbb{T}}
\newcommand{\nnreals}{\mathbb{R}_{\geq 0}}
\newcommand{\posreals}{\mathbb{R}_{> 0}}
\newcommand{\reals}{\mathbb{R}}

\newcommand{\unrollkM}[2]{\textsf{unroll}_{#1}\left({#2}\right)}
\newcommand{\nphmcint}[5]{\Psi_\textsf{NP}\left({#1}, {#2}, {#3}, {#4}, {#5}\right)}

%SPCF constructs
\newcommand{\spcfvalues}{\Lambda^0_v}

\newcommand{\prevalueM}[1]{\textsf{value}^{-1}_{#1}(\spcfvalues{})}
\newcommand{\num}[1]{\underline{#1}}

% \theoremstyle{definition}
% \newtheorem{thm}{Theorem}
% \newtheorem{lem}{Lemma}
% \newtheorem{defn}{Definition}
% \newtheorem{conj}{Conjecture}
% \newtheorem{prop}{Proposition}

%\theoremstyle{definition}
%\newtheorem{defn}{Definition}[section]
%\newtheorem{example}[defn]{Example}
%
%
%\theoremstyle{plain}
%\newtheorem{thm}{Theorem}[section]
%\newtheorem{lem}[thm]{Lemma}
%\newtheorem{cor}[thm]{Corollary}
%\newtheorem{conj}[thm]{Conjecture}
%\newtheorem{prop}[thm]{Proposition}
%\newtheorem{remark}[thm]{Remark}

%% Proofs
%\let\oldproof\proof
%\renewcommand{\proof}{\color{blue}\oldproof}


\definecolor{codegreen}{rgb}{0,0.6,0}
\definecolor{codegray}{rgb}{0.5,0.5,0.5}
\definecolor{codepurple}{rgb}{0.58,0,0.82}
\definecolor{backcolour}{rgb}{0.95,0.95,0.92}

\lstdefinestyle{myStyle}{
    belowcaptionskip=1\baselineskip,
    breaklines=true,
    frame=none,
    basicstyle=\footnotesize\ttfamily,
    keywordstyle=\bfseries\color{green!40!black},
    commentstyle=\itshape\color{purple!40!black},
    identifierstyle=\color{blue},
    backgroundcolor=\color{gray!10!white},
    %backgroundcolor=\color{backcolour}, 
    numberstyle=\tiny\color{codegray},
    stringstyle=\color{codepurple},
    breakatwhitespace=false,                          
    keepspaces=true,                 
    numbers=left,       
    numbersep=5pt,                  
    showspaces=false,                
    showstringspaces=false,
    showtabs=false,                  
    tabsize=2,
}

% argmin/argmax
\DeclareMathOperator*{\argmax}{arg\,max}
\DeclareMathOperator*{\argmin}{arg\,min}

% Concatenation of lists
\newcommand\doubleplus{+\kern-1.3ex+\kern0.8ex}

% Program configurations
\newcommand{\tuple}[1]{\ensuremath{\langle #1 \rangle}}
% Rule based definitions
\newcommand{\Rule}[4][]{\ensuremath{\inferrule*[lab={\hypertarget{#2}{(\TirName{#2})}},#1]{#3}{#4}}}

% Calligraphic symbols
\newcommand{\calI}{{\mathcal I}} 
\newcommand{\calT}{{\mathcal T}}

%  Macro for new Y operator.
\newcommand{\yBounded}[3]{\mu^{#1}_{#2}\rvert_{#3}}

%%%%%%%%%%%%%%%%%
 
%%%%%%%%%%%%%%%%%

\newcommand{\expv}{\mathbb{E}}

\newcommand{\combTr}[2]{\left[\begin{matrix}
		#1\\
		#2
	\end{matrix} \right]}

\newcommand{\exType}[2]{\left\{\begin{matrix}
		#1\\
		#2
	\end{matrix} \right\}}
\newcommand{\myint}[1]{ [#1]}
\newcommand{\Uniform}{\ensuremath{\mathrm{Uniform}}}
\newcommand{\Normal}{\ensuremath{\mathrm{normal}}}
\DeclareMathOperator{\abs}{abs}
\DeclareMathOperator{\pdf}{pdf}

\newcommand{\intConf}[1]{\lceil#1\rceil}
\newcommand{\tr}{\boldsymbol{t}}

\newcommand{\sample}{\tt{sample}}
%\newcommand{\fix}{\texttt{fix}}
%\newcommand{\num}[1]{\underline{#1}}
\newcommand{\myif}{\texttt{if}}
\newcommand{\mylet}{\texttt{let} \, }
\newcommand{\myin}{\, \texttt{in} \,}
\newcommand{\mythen}{\, \texttt{then} \,}
\newcommand{\myelse}{\, \texttt{else} \,}
\newcommand{\score}{\tt{score}}
\newcommand{\tick}{\tt{tick}}

\newcommand{\term}{\tt{term}}
\newcommand{\pv}{\mathbf{v}}
\newcommand{\rv}{\mathbf{r}}

\newcommand{\interval}{\mathfrak{I}}

\newcommand{\typeReal}{\textbf{\textsf{R}}}

\newcommand{\symbolInt}{\myint{\cdot}}

\newcommand{\LambdaInterval}{\Lambda_{\interval}}
\newcommand{\LambdaSymbolic}{\Lambda_{\text{sym}}}

\newcommand{\toIntervalTerm}[1]{#1^{2\interval}}

%Others
\newcommand{\Sset}{\mathbb{S}}
\newcommand{\Iset}{\mathbb{I}}
\newcommand{\Rset}{\mathbb{R}}
\newcommand{\Nset}{\mathbb{N}}
\newcommand{\Zset}{\mathbb{Z}}

\newcommand{\Term}{\mathbb{T}}
\newcommand{\prob}{\mathbb{P}}
\newcommand{\expt}{\mathbb{E}}


\newcommand{\Leb}{\tt{Leb}}
\newcommand{\Red}{\tt{Red}}
\newcommand{\cost}{\text{cost}}

%\newcommand{\intervalab}[2]{\underline{[#1,#2]}}
\newcommand{\intervalab}{\underline{[a,b]}}
\newcommand{\interI}{\mathcal{I}}
\newcommand{\trans}{\mathcal{T}}

\newcommand{\iv}{\mathbb{I}}

% Programming language constructs
\newcommand{\lit}[1]{\underline{#1}}
\newcommand{\letIn}[1]{\mathsf{let}\,{#1}\,\mathsf{in}\,}
\newcommand{\fixLam}[2]{\mu {#1} {#2}.}
\newcommand{\ifElse}[3]{\mathsf{if} (#1 \le \num{0}) \, {#2} \,\mathsf{else}\, {#3}}

%%Basic notions
\newcommand{\pspace}{(\Omega,\mathcal{F},\probm)}
\newcommand{\probm}{\mathbb{P}}
\newcommand{\condexpv}[2]{{\expt}{\left[{#1} \mid {#2}\right]}}

\newcommand{\stdConf}[1]{(#1)}
%\newcommand{\intConf}[1]{\lceil#1\rceil}
%\newcommand{\intConf}[1]{(#1)}
%\newcommand{\symConf}[1]{\langle\!\langle  #1 \rangle\!\rangle}
%\newcommand\symPath[1]{(#1)}
\newcommand{\symPath}[1]{\langle\!\langle  #1 \rangle\!\rangle}
\newcommand\symConf[1]{(#1)}

\newcommand{\ifSimple}[3]{\mathsf{if}(#1, #2, #3)}
%\newcommand{\ifElse}[3]{\mathsf{if} (#1 \le 0) \, \allowbreak {#2} \, \allowbreak \mathsf{else}\, {#3}}
%\newcommand{\ifElse}[3]{\ifSimple{#1}{#2}{#3}}

%\newcommand{\trace}{\mathsf{s}}
%
%\newcommand\defn[1]{{\bf \em #1}}
\newcommand{\traces}{\mathbb{T}}
%
%\newcommand{\stdConf}[1]{(#1)}
%%\newcommand{\intConf}[1]{\lceil#1\rceil}
%\newcommand{\intConf}[1]{(#1)}
%%\newcommand{\symConf}[1]{\langle\!\langle  #1 \rangle\!\rangle}
%%\newcommand\symPath[1]{(#1)}
%\newcommand{\symPath}[1]{\langle\!\langle  #1 \rangle\!\rangle}
%\newcommand\symConf[1]{(#1)}

\newcommand{\valueSem}[1]{\mathsf{val}_{#1}} % value (semantics)
\newcommand{\weightSem}[1]{\mathsf{wt}_{#1}} % weight (semantics)
\newcommand{\measureSem}[1]{\llbracket #1 \rrbracket}
\newcommand{\posterior}{\mathsf{posterior}}


%%%%%%%%%
% 
%%%%%%%%
\newcommand{\loc}{\ell}
\newcommand{\locs}{\mathit{L}}
\newcommand{\blocs}{\mathit{L}_{\mathrm{b}}}

\newcommand{\iflocs}{\mathit{L}_{\mathrm{if}}}
\newcommand{\looplocs}{\mathit{L}_{\mathrm{while}}}

\newcommand{\alocs}{\mathit{L}_{\mathrm{a}}}
\newcommand{\wlocs}{\mathit{L}_{\mathrm{w}}}
\newcommand{\rlocs}{\mathit{L}_{\mathrm{r}}}
\newcommand{\Alocs}[1]{\mathit{L}_{\mathrm{A}}^{\mathsf{#1}}}
\newcommand{\Dlocs}{\mathit{L}_{\mathrm{nd}}}
\newcommand{\transitions}{{\rightarrow}}

%%% 
\newcommand{\plocs}{\mathit{L}_{\mathrm{p}}}
\newcommand{\tlocs}{\mathit{L}_{\mathrm{t}}}

\newcommand{\lin}{\loc_\mathrm{init}}
\newcommand{\lout}{\loc_\mathrm{out}}
\newcommand{\val}[1]{\mbox{\sl Val}_{#1}}

\newcommand{\pvars}{V_\mathrm{p}}
\newcommand{\rvars}{V_{\mathrm{r}}}
\newcommand{\pre}{\mathrm{pre}}

\newcommand{\sle}{\sqsubseteq}
\newcommand{\sge}{\sqsupseteq}

\newcommand{\lfp}{\mathrm{lfp}}
\newcommand{\gfp}{\mathrm{gfp}}

\newcommand{\rdvarjdis}{\mathcal D}
\newcommand{\sampset}{\textit{supp}}

\newcommand{\upd}{\mbox{\sl upd}}
\newcommand{\wet}{\mbox{\sl wt}}
\newcommand{\transset}{\mathfrak T}
\newcommand{\valin}{\pv_{\mathrm{init}}}
\newcommand{\ret}{\mbox{\sl ret}}

\newcommand{\win}{w_{\mathrm{init}}}

\newcommand{\sampdpd}{\overline{\Upsilon}}

\newcommand{\outmap}{\text{O}}
\newcommand{\sat}[1]{\langle #1 \rangle}
\newcommand{\monoid}{\mbox{\sl Monoid}}
\newcommand{\handelmanformat}{(\dagger)}

\newcommand{\trunc}{\mathcal{B}}

\newcommand{\ewt}{\mbox{\sl ewt}}
\newcommand{\statemap}{\text{St}}

\newcommand{\valrd}{{\mathbf{r}}}
\newcommand{\frmloc}{\ell^{\mathrm{src}}}
\newcommand{\toloc}{\ell^{\mathrm{dst}}}

\newcommand{\monomials}{\mathbf{M}}

\title{Solving Odd-Fair Parity Games}%TODO Please add

% \titlerunning{Dummy short title} %TODO optional, please use if title is longer than one line

\author{Irmak Sa\u{g}lam}{Max Planck Institute for Software Systems (MPI-SWS), Kaiserslautern, Germany}{isaglam@mpi-sws.org}{[orcid]}{}

\author{Anne-Kathrin Schmuck}{Max Planck Institute for Software Systems (MPI-SWS), Kaiserslautern, Germany}{akschmuck@mpi-sws.org}{[orcid]}{}

\authorrunning{I. Sa\u{g}lam, A.-K. Schmuck} %TODO mandatory. First: Use abbreviated first/middle names. Second (only in severe cases): Use first author plus 'et al.'

% \author{John Q. Public}{Dummy University Computing Laboratory, [optional: Address], Country \and My second affiliation, Country \and \url{http://www.myhomepage.edu} }{johnqpublic@dummyuni.org}{https://orcid.org/0000-0002-1825-0097}{(Optional) author-specific funding acknowledgements}%TODO mandatory, please use full name; only 1 author per \author macro; first two parameters are mandatory, other parameters can be empty. Please provide at least the name of the affiliation and the country. The full address is optional

% \author{Joan R. Public\footnote{Optional footnote, e.g. to mark corresponding author}}{Department of Informatics, Dummy College, [optional: Address], Country}{joanrpublic@dummycollege.org}{[orcid]}{[funding]}

\Copyright{Irmak Sa\u{g}lam and Anne-Kathrin Schmuck} %TODO mandatory, please use full first names. LIPIcs license is "CC-BY";  http://creativecommons.org/licenses/by/3.0/

\ccsdesc[500]{Theory of computation~Solution concepts in game theory}
%\ccsdesc[100]{\textcolor{red}{Replace ccsdesc macro with valid one}} %TODO mandatory: Please choose ACM 2012 classifications from https://dl.acm.org/ccs/ccs_flat.cfm 

\keywords{parity games, strong transition fairness, algorithmic game theory} %TODO mandatory; please add comma-separated list of keywords

%\keywords{games, parity games, strong transition fairness, algorithmic game theory, algorithmic fairness} %TODO mandatory; please add comma-separated list of keywords

%\category{} %optional, e.g. invited paper

%\relatedversion{} %optional, e.g. full version hosted on arXiv, HAL, or other respository/website
%\relatedversion{A full version of the paper is available at \url{...}.}

%\supplement{}%optional, e.g. related research data, source code, ... hosted on a repository like zenodo, figshare, GitHub, ...

\funding{This work was partially supported by the DFG projects SCHM 3541/1-1 and 389792660 TRR 248–CPEC.}

\acknowledgements{We are grateful for the immense support provided by Munko Tsyrempilon for the experimental validation.}

\nolinenumbers %uncomment to disable line numbering

\hideLIPIcs  %uncomment to remove references to LIPIcs series (logo, DOI, ...), e.g. when preparing a pre-final version to be uploaded to arXiv or another public repository

%Editor-only macros:: begin (do not touch as author)%%%%%%%%%%%%%%%%%%%%%%%%%%%%%%%%%%
% \EventEditors{John Q. Open and Joan R. Access}
% \EventNoEds{2}
% \EventLongTitle{42nd Conference on Very Important Topics (CVIT 2016)}
% \EventShortTitle{CVIT 2016}
% \EventAcronym{CVIT}
% \EventYear{2016}
% \EventDate{December 24--27, 2016}
% \EventLocation{Little Whinging, United Kingdom}
% \EventLogo{}
% \SeriesVolume{42}
% \ArticleNo{23}
%%%%%%%%%%%%%%%%%%%%%%%%%%%%%%%%%%%%%%%%%%%%%%%%%%%%%%

\begin{document}

\maketitle


\begin{abstract}
  This paper discusses the problem of efficiently solving parity games where player \Odd~has to obey an additional \emph{strong transition fairness constraint} on its vertices -- given that a player \Odd~vertex $v$ is visited infinitely often, a particular subset of the outgoing edges (called \emph{live edges}) of $v$ has to be taken infinitely often. Such games, which we call \emph{\Odd-fair parity games}, naturally arise from abstractions of cyber-physical systems for planning and control. 
%   and can currently not satisfactory been dealt with.}
  %for planning and control. %as well as in planning and in resource management.\todo{todo: if we don't add the explanation about resource planning, we should remove it from here}

  In this paper, we present a new Zielonka-type algorithm for solving \Odd-fair parity games. This algorithm not only shares \emph{the same worst-case time complexity} as Zielonka's algorithm for (normal) parity games but also preserves the algorithmic advantage Zielonka's algorithm possesses over other parity solvers with exponential time complexity. %, such as the recently introduced \emph{fixed-point algorithm} for \Odd-fair parity games by Banerjee et. al.~\cite{banerjee2022fast}.
 
 We additionally introduce a formalization of \Odd~player winning strategies in such games, which were unexplored previous to this work. %To the best of our knowledge, this is the first time such a formalization has been introduced.  %We represent these strategies as \enquote{almost positional} strategy templates, and prove their existence from all player \Odd~winning vertices. 
 This formalization serves dual purposes: firstly, it enables us to prove our Zielonka-type algorithm; secondly, it stands as a noteworthy contribution in its own right, augmenting our understanding of additional fairness assumptions in two-player games.\end{abstract}
 

\section{Introduction}
Current quantum hardware is unable to carry out universal quantum computations due to the buildup of errors that occur during the computation. 
The magnitude of the individual error is currently above the value that the Threshold Theorem requires in order to kick-start quantum error correction and fault-tolerant quantum computation~\cite[Section 10.6]{nielsen_chuang_2010}. 
Although the experimentally achieved fidelity rates are promising and the error bounds are inching closer to the required threshold, we will have to work for the foreseeable future with quantum hardware with errors that build-up during the computation.  This implies that we can only do a limited number of steps before the output of the computation has become completely uncorrelated with the intended one.

For fault-tolerant quantum computing, we repeat four steps: 
1) We apply a number of single and two-qubit quantum gates, in parallel whenever possible; 
2) We perform a syndrome measurement on a subset of the qubits; 
3) We perform fast classical computations to determine which errors have occurred and how to correct them; 
and, 4) We apply correction terms based on the classical computations.
We then repeat these four steps with a next sequence of gates. 
These four steps are essential to fault-tolerant quantum computing. 


The starting point of this work is to use the four steps outlined above, not to carry out error correction and fault-tolerant computation, but to enhance short, constant-depth, {\em uncorrected} quantum circuits that perform single qubit gates and {\em nearest-neighbor} two qubit gates. 
Since in the long run we will have to implement error-correction and fault-tolerant computation anyhow, and this is done by such a four-step process, why not make other use of this architecture? Moreover, on some of the quantum hardware platforms, these operations are already in place.
Embracing this idea we naturally arrive at the question: what is the computational power of \textit{low-depth} quantum-classical circuits organized as in the four steps outlined above? 
We thus investigate circuits that execute a small, ideally constant, number of stages, where at each stage we may apply, in parallel, single qubit gates and {\em nearest-neighbor} two qubit gates, followed by measurements, followed by low-depth classical computations of which the outcome can control quantum gates in later stages. 
It is not clear, at first, whether such circuits, especially with constant depth, can do anything remotely useful. 
But we will see that this is indeed the case: many quantum computations can be done by such circuits in constant depth. 
By parallelizing quantum computations in this way, we improve the overall computational capabilities of these circuits, as we do not incur errors on qubits that are idle, simply because qubits are not idle for a very long time. 
Furthermore, reducing the depth of quantum circuits, at the cost of increasing width, allows the circuit to be run faster even if errors occur.

The first usage of such a four-step layout, not to do error correction, but to perform computations, can be found in the paradigm of measurement-based quantum computing~\cite{gottesman1999demonstrating,raussendorf2001one,jozsa2006introduction,clark2007generalised}: 
A universal form of quantum computing where a quantum state is prepared and operations are performed by measuring qubits in different bases, depending on previous measurements and intermediate measurements.

\citeauthor{PhamSvore2013} were the first to formalize the four-step protocol for performing computations~\cite{PhamSvore2013}. They included specific hardware topologies by considering two-dimensional graphs for imposing constraints on qubit interactions. In their model, they develop circuits for particularly useful multi-qubit gates, including specifying costs in the width, number of qubits, depth, number of concurrent time steps, size, and total number of non-Identity operations.
As a result, they find an algorithm that factors integers in polylogarithmic depth.
\citeauthor{Browne:2011} showed that the main tool in the work by \citeauthor{PhamSvore2013}, the fan-out gate, can also be replaced by additional log-depth classical computations in the measurement-based quantum computing setting~\cite{Browne:2011}.

More recently, \citeauthor{Cirac:2021} introduced a scheme to implement unitary operations involving quantum circuits combined with Local Operations and Classical Communication ($\mathsf{LOCC}$) channels: $\mathsf{LOCC}$-assisted quantum circuits~\cite{Cirac:2021}. Similarly to the four-step scheme we just described, they allow for a short depth circuit to be run on the qubits, followed by one round of $\mathsf{LOCC}$, in which ancilla qubits are measured and local unitaries are applied based on the measurement outcomes. They show that in this model any 1D transitionally invariant matrix-product state (MPS) with fixed bond dimension is in the same phase of matter as the trivial state. Similar ideas can be found in~\cite{TVV_NonAbelianTopologicalOrder_2022, tantivasadakarn2021long}.

In this work, we introduce a new model, called \textit{Local Alternating Quantum-Classical Computations} ($\LAQCC$). In this model we alternate between running quantum circuits (constrained by locality), ending in the measurement of a subset of qubits, and fast classical computations based on the measurement results. The outcome of the classical computations are then used to control future quantum circuits. We allow for flexibility in this model, by giving different constraints to the power of both the quantum circuits and the classical circuits as well as the number of alternations between them. 
Most attention will be given to $\LAQCC$ containing quantum circuits of constant depth, classical circuits of logarithmic depth and at most a constant number of alternations between them. 
Any circuit constructed in this model is considered to be of constant depth. 
We restrict ourselves to logarithmic depth classical computations, as this is the first natural and non-trivial extension beyond constant-depth classical computations. 
Constant-depth classical computations do however also have an equivalent constant-depth quantum implementation.

The definition of $\LAQCC$ sharpens the original definition of \citeauthor{PhamSvore2013} by adding constraints to the intermediate classical computations. This allows us to bound the power of $\LAQCC$ from above. 

The main result of \citeauthor{Cirac:2021}, that 1D translational invariant MPS with fixed bond dimension can be prepared by $\mathsf{LOCC}$-assisted circuits, relies on local symmetries of the MPS. These symmetries allow them to prepare local states (on a constant number of qubits) and glue them together by doing one round of the appropriate entangling measurement and corrections, after which they run a round of local unitaries to get the desired result. This general scheme for preparing states that exhibit an MPS description with the appropriate local symmetries requires only geometrically local unitaries and one round of measurement and corrections an therefore is accessible in $\LAQCC$. Studying different local symmetries, known as Symmetry Protected Topological (SPT) phases of matter, to find measurement-based constant depth circuits for states is a broad ongoing field of research~\cite{TVV_NonAbelianTopologicalOrder_2022, tantivasadakarn2021long, smith2023deterministic}. 
All these schemes have a $\LAQCC$ implementation.

%$\LAQCC$-circuits also exist for general schemes of preparing local states, based on the local tensors, and gluing them together using one round of entangled measurement and corrections, based on the local symmetry. 
%The main result of \citeauthor{Cirac:2021}, that 1D translational invariant MPS with fixed bond dimension can be prepared by $\mathsf{LOCC}$-assisted circuits, relies heavily on local symmetries of the MPS and as a result also has an equivalent $\LAQCC$ implementation. 
%The corrections applied after the measurement round are local unitaries depending on the local symmetries of the MPS. 

 

%This general scheme of preparing local states, based on the local tensors, and gluing it together by doing one round of entangled measurement and corrections, based on the local symmetry, is accessible in $\LAQCC$.
Note however that \citeauthor{Cirac:2021} also suggest a circuit for the $W$-state.
This circuit uses sequentially and dependent measurement-based corrections of the ancilla qubits. 
These dependent measurements translate to sequential alternations between the quantum and classical circuits and therefore increase the total depth to linear depth, exceeding the constant-depth constraints imposed by $\LAQCC$-circuits. 

We study the power of the $\LAQCC$ model with respect to state preparation, showing that even with only constant quantum-depth and logarithmic classical depth it remains possible to prepare states with long-range entanglement.
Another surprising result is that it is unlikely that $\LAQCC$ circuits are classically simulatable. We show that any instantaneous quantum polynomial-time (IQP) circuit~\cite{Bremner2010,Shepherd2009} has an $\LAQCC$ implementation.
Classical simulation of IQP circuits implies the collapse of the polynomial hierarchy to the third level, which is not believed to be true~\cite{Bremner2017}. Therefore, we expect that $\LAQCC$ circuits are unlikely to be classically simulatable. We bound the power of $\LAQCC$ by showing that it is contained in $\QNC^1$, the class of polynomial-size, log-depth circuits.

Next, we also study the power that intermediate classical calculations can add to quantum computations, by considering a new model that alternates between polynomially many polynomial-depth quantum circuits and unbounded classical computations
We study this model by doing a complexity theoretical analysis, where we draw inspiration from the notions of complexity given by \citeauthor{RosenthalYuen:2022}, \citeauthor{MetgerYuen:2023}, and \citeauthor{Aaronson:2004}.
All three complexity notions are based on the notion of state preparation, instead of more traditional definition of complexity such as the decidability of a computational problem. 
The first two consider classes based on sequences of quantum states preparable by a polynomial-sized quantum circuit, where the circuits are uniformly generated by a computational class, for instance, the class $\mathsf{PSPACE}$, which results in the complexity class $\mathsf{StatePSPACE}$~\cite{RosenthalYuen:2022,MetgerYuen:2023}.
The third notion considers a relative complexity, where the complexity is measured between two given states, and is measured by the number of gates, from a given gate-set, required to transform one state in another state~\cite{Aaronson:2004}. 
For our definition of state preparation complexity, we drop the uniformity constraint from~\cite{RosenthalYuen:2022,MetgerYuen:2023} and define a class as $\mathsf{StateX}$, which refers to states preparable by circuits of type $\mathsf{X}$. 
As an example, if $\mathsf{X} = \QNC^0$, this results in the class $\mathsf{StateQNC^0}$, which is the set of states preparable from the $\ket{0}^n$ state by poly-size constant-depth circuits. 
This notion is similar to the relative complexity from~\cite{Aaronson:2004}, where one state is the  $\ket{0}^n$ state and instead of counting the number of gates we consider the set of states preparable by a fixed number of gates. Using this notion of complexity we show that any state preparable by an $\LAQCC^*$ circuit is also preparable by a $\mathsf{PostQPoly}$ circuit, the class of circuits of polynomial depth with an additional post-selection gate. 

All Clifford circuits have a constant-depth $\LAQCC$ implementation, implying that any stabilizer state can be implemented by a constant-depth $\LAQCC$ circuit, see Section~\ref{sec:clifford_circuits} for a proof of this statement. 
Efficient circuits for stabilizer states have been known already through measurement-based quantum computing. Therefore this paper focuses on the preparation of non-stabilizer states, and as a surprising result we find novel constant-depth protocols for four very natural classes of non-stabilizer states.
Despite the extensive research into these four classes of non-stabilizer states and the many applications of them, no efficient constant- or low-depth state preparation protocols are known yet. We specifically consider these four classes as they are all often used as initial states in other algorithms.

The first state is a uniform superposition over an arbitrary number of states. 
This state finds applications in many quantum algorithms, as they often start with a uniform superposition over multiple states. 
This superposition is often achieved by applying Hadamard gates to every qubit due to its simplicity to prepare. 
Yet, the analysis of many algorithms, such as Shor's algorithm~\cite{Shor:1997}, would benefit from a different initial superposition. 
The circuit to prepare the uniform superposition over an arbitrary number of states uses an exact version of Grover search as a subroutine, that turns a probabilistic circuit, with a known constant probability of success, into a deterministic circuit. 
We use the circuit for preparing a uniform superposition over an arbitrary number of states as a subroutine in the next two quantum state preparation protocols. 

The second state is the $W$-state, the uniform superposition over all computational basis states of Hamming-weight~$1$, a natural long-ranged entangled state that displays a fundamentally nonequivalent type of entanglement from the Greenberger–Horne–Zeilinger state~\cite{WState:2000}, for which $\LAQCC$-type constant-depth circuits were previously known~\cite{PhamSvore2013, Cirac:2021}. 
The $W$-state is often used as benchmark for new quantum hardware~\cite{Haffner2005,Neeley2010,GarciaPerez:2021}. 
A novel way to prepare the $W$-state therefore gives a new way to benchmark different quantum devices with each other. 
A circuit for preparing the $W$-state was given in~\cite{Cirac:2021}, but this implementation requires sequentially alternating measurements followed by local unitaries, which in the $\LAQCC$ model is not considered to be of constant depth. 
We improve this protocol by giving an $\LAQCC$ implementation of the $W$-state, based on a compress-uncompress method that links the one-hot and binary encoding of integers.

The third state considered is the Dicke state, a generalization of the $W$-state, a superposition over all computational basis states with Hamming-weight $k$~\cite{Dicke:1954}. 
Dicke states have relevance in various practical settings.
For instance, for quantum game theory~\cite{zdemir2007}, quantum storage~\cite{Bacon_Compress:2006,Plesch:2010}, quantum error correction~\cite{ouyang2014permutation}, quantum metrology~\cite{toth2012multipartite}, and quantum networking~\cite{prevedel2009experimental}. 
Dicke states have been used as a starting state for variational optimization algorithms, most notably Quantum Alternating Operator Ansatz (QAOA)~\cite{Hadfield2019}, to find solutions to problems such as Maximum k-vertex Cover~\cite{Brandhofer2022,cook2020quantum}.
The ground states of physical Hamiltonians describing one-dimensional chains tend to show a resemblance to Dicke states such as states resulting from the Bethe ansatz, making them an ideal starting state when investigating the ground state behavior of these Hamiltonians~\cite{TDL_BetheAnsatzDerivation:2010,B_ExcitedStateQuantumPhaseTransitions:2013,DickeTransitions:2021}. 
For instance, the algorithm by \citeauthor{van2021preparing}, who give an algorithm to prepare the Bethe ansatz eigenstates of the spin-1/2 XXZ spin chain, starts by first preparing a Dicke state~\cite{van2021preparing}. 
A Dicke-state preparation protocol based on the compress-uncompress methodology used in the $W$-state furthermore finds applications in entanglement distillation, where the entanglement of a large state is concentrated on only a few qubits. 
Efficient deterministic circuits for preparing Dicke states have been proposed by \citeauthor{bartschi2019deterministic}~\cite{bartschi2019deterministic, bartschi2022deterministic_short_depth}. 
They provide a quantum circuit of depth $\mathO(k \log(\frac{n}{k}))$, allowing arbitrary connectivity, to prepare a Dicke state, which they conjecture to be optimal when $k$ is constant. 
In this work, we provide a constant-depth $\LAQCC$ circuit below their conjectured bound already for constant $k$. 
However, this does not directly disprove their conjecture, as we allow for intermediate measurements and classical computations. 
More significantly, we even construct constant-depth $\LAQCC$ circuits for $k = \mathO(\sqrt{n})$ greatly improving their bound.
This construction extends the compress-uncompress method for the $W$-state combined with additional subroutines. 

We continue with a log-depth state preparation protocol for the Dicke-state for arbitrary $k$. 
This protocol implements an efficient transformation between the factoradic number representation and the combinatorial number representation of a positive integer. 
The combinatorial number representation relates directly to the Dicke state. 
The provided efficient transformation between number representation systems might be of independent interest. 

We conclude by modifying our protocol for preparing a Dicke-state to a protocol that prepares quantum many-body scar states in constant-depth. 
These states have low entanglement and longer coherence times than states with similar energy density.
These characteristics make many-body scar states interesting to analyze and relevant within physics.
Many-body scar states appear for instance in the AKLT model~\cite{AKLT:1987,MRBAR:2018,MRB:2018} and different spin models~\cite{SI:2019,MOBFR:2020}.
Known methods for preparing these states have polynomial-depth~\cite{Gustafson:2023}, whereas our circuit has constant depth. 

% We conclude by studying the power that intermediate classical calculations can add to quantum computations. 
% In this study, we define a new model that relaxes constant-depth quantum circuits to polynomial depth quantum circuits, log-depth classical calculations to unbounded classical computations and a constant number of alternations to a polynomial number of alternations. 
% We call this model $\LAQCC^*$. 
% We study this model by doing a complexity theoretical analysis, where we draw inspiration from the notions of complexity given by \citeauthor{RosenthalYuen:2022}, \citeauthor{MetgerYuen:2023}, and \citeauthor{Aaronson:2004}.
% All three complexity notions are based on the notion of state preparation, instead of more traditional definition of complexity such as the decidability of a computational problem. 
% The first two consider classes based on sequences of quantum states preparable by a polynomial-sized quantum circuit, where the circuits are uniformly generated by a computational class, for instance, the class $\mathsf{PSPACE}$, which results in the complexity class $\mathsf{StatePSPACE}$~\cite{RosenthalYuen:2022,MetgerYuen:2023}.
% The third notion considers a relative complexity, where the complexity is measured between two given states, and is measured by the number of gates, from a given gate-set, required to transform one state in another state~\cite{Aaronson:2004}. 
% For our definition of state preparation complexity, we drop the uniformity constraint from~\cite{RosenthalYuen:2022,MetgerYuen:2023} and define a class as $\mathsf{StateX}$, which refers to states preparable by circuits of type $\mathsf{X}$. 
% As an example, if $\mathsf{X} = \QNC^0$, this results in the class $\mathsf{StateQNC^0}$, which is the set of states preparable from the $\ket{0}^n$ state by poly-size constant-depth circuits. 
% This notion is similar to the relative complexity from~\cite{Aaronson:2004}, where one state is the  $\ket{0}^n$ state and instead of counting the number of gates we consider the set of states preparable by a fixed number of gates. Using this notion of complexity we show that any state preparable by an $\LAQCC^*$ circuit is also preparable by a $\mathsf{PostQPoly}$ circuit, the class of circuits of polynomial depth with an additional post-selection gate. 

\paragraph{Summary of results}
\begin{itemize}
    \item We give a new definition of a computational model that captures the power of the four step process: applying a constant number of layers of one- and two-qubit gates; performing a syndrome measurement; perform a fast classical computation determining corrections; apply corrections. We call this model \emph{Local Alternating Quantum Classical Computations}, or $\LAQCC$ for short. In this model we bound the allowed quantum operations, intermediate classical calculations, and number of rounds separately. In Section~\ref{sec:LAQCC_model} we define this model and give a list of operations based on results from literature contained in this computational model. In some of these operations we explicitly use that we allow for multiple, but at most constant, rounds  of corrections.
    \item  We show show that there exist $\LAQCC$ circuits that can not be weakly simulated in Section~\ref{sec:IQP_in_LAQCC}. We further show that for every $\LAQCC$ circuit there exists a $\QNC^1$ circuit simulating it perfectly, in Section~\ref{sec:LAQCC_in_QNC1}.
    \item We introduce a new type computational complexity for preparing states and show that the extension of $\LAQCC$ where we allow a polynomial number of rounds and unbounded classical computation, is contained in $\mathsf{PostQPoly}$, the class of polynomial circuits with post-selection, in Section~\ref{sec:Complexity results}.
    \item We show a protocol to prepare the uniform superposition state of size $q$ in $\LAQCC$ using $\mathO(\ceil{\log_2(q)}^2)$ qubits in Section~\ref{sec:superposition_modulo_q}. 
    \item We show a protocol to prepare the $W_n$ state in $\LAQCC$ using $\mathO(n\log(n))$ qubits in Section~\ref{sec:W_state_in_LAQCC}.
    \item We show two ways of preparing the Dicke-$(n,k)$ state. The first method is in $\LAQCC$, works up to $k = \mathO(\sqrt{n})$, uses $\mathO(n^2\log(n))$ qubits, and is found in Section~\ref{sec:dicke:small_k}. The second method is in $\LAQCC\text{-}\mathsf{LOG}$ (an extension of $\LAQCC$ allowing for logarithmic number of alterations instead of constant), works for any $k$, uses $\mathO(\text{poly}(n))$ qubits, and is found in Section~\ref{sec:Dicke_in_LAQCC_LOG}. 
    \item We extend on our $\LAQCC$ method of generating Dicke-$(n,k)$ states for $k = \mathO(\sqrt{n})$ and show a protocol to generate many-body scar states for a particular Hamiltonian in $\LAQCC$ (Section~\ref{sec:many_body_scar}). 
\end{itemize}
Summarized in a table, we provide the following state generation protocols:
\begin{table}[htb]
\centering
\begin{tabular}{l|l|l|l}
\textbf{State description} & \textbf{Width} & \textbf{Depth} & \textbf{Implementation}\\
\hline 
Uniform superposition mod $q$: $\frac{1}{\sqrt{q}} \sum_{i = 0}^{q-1}\ket{i}$ & $\mathO(\ceil{\log^2 q})$ & $\mathO(1)$ & Section~\ref{sec:superposition_modulo_q}\\

$W$-state: $\frac{1}{\sqrt{n}}\sum_{i = 0}^{n-1}\ket{e_i}$ & $\mathO(n \log n)$ & $\mathO(1)$ & Section~\ref{sec:W_state_in_LAQCC}\\

Dicke-$(n,k)$, $k = \mathO(\sqrt{n})$: $\binom{n}{k}^{-1/2}\sum_{x \in \{0,1\}^n: |x| = k} \ket{x}$ &  $\mathO(n^2\log n)$ & $\mathO(1)$ 
&Section~\ref{sec:dicke:small_k}\\

Dicke-$(n,k)$: $\binom{n}{k}^{-1/2}\sum_{x \in \{0,1\}^n: |x| = k} \ket{x}$ & $\mathO(\text{poly}(n))$ & $\mathO(\log n)$ &Section~\ref{sec:Dicke_in_LAQCC_LOG}\\

QMBS: $\ket{S_k} = \frac{1}{k! \sqrt{\mathcal N(n,k)}}(Q^\dagger)^k \ket{\Omega}$ &  $\mathO(n^2\log n)$ & $\mathO(1)$  &  Section~\ref{sec:many_body_scar}
\end{tabular}
\caption{Summary of state preparation protocols given in this paper.}
\label{tab:sate_prep}
\end{table}
In the entry for the quantum many-body scar state $Q$ denotes the raising operator and $\mathcal N(n,k)=\binom{n-k-1}{k}$. 
Section~\ref{sec:many_body_scar} will provide more details on the variables and the implementation. 

\paragraph{Organization of the paper}
\noindent We first introduce relevant preliminaries in Section~\ref{sec:preliminaries}. 
In Section~\ref{sec:LAQCC_model} we formally define the class of Local Alternating Quantum-Classical Computations ($\LAQCC$). We also show that any Clifford circuit can be implemented in constant depth $\LAQCC$ (a result based on a result from measurement-based quantum computing~\cite{jozsa2006introduction}). 
This result allows us to give many useful multi-qubit gates and routines in Section~\ref{sec:gates_created_in_LAQCC}. 
Beyond that we show that constant depth $\LAQCC$ circuits are contained in $\QNC^1$ and that any $\mathsf{IQP}$ circuit has an $\LAQCC$ implementation.
We conclude this section with an analysis of a more powerful instantiation of $\LAQCC$ and show an inclusion with respect to the class $\mathsf{PostQPoly}$, which is the class of circuits of polynomial depth with one additional post-selection gate. 
In Section~\ref{sec:state_prep_in_LAQCC} we give $\LAQCC$ circuit implementations for preparing the uniform superposition over an arbitrary number of states, the $W$-state and the Dicke state up to $k = \mathO(\sqrt{n})$. We furthermore give a log-depth circuit implementation for preparing the Dicke state for any $k$. We conclude by showing a $\LAQCC$ circuit for generating many body scar states of a particular type of Hamiltonian.


\section{Preliminaries}
In this section, we describe the necessary background for automated planning and the significance of the International Planning Competition. 

% \subsection{Ontology}
% A formal ontology is typically represented as a set of concepts, relations, and axioms. A concept represents a set of objects or entities that share common properties, while a relation represents a connection or association between two or more concepts. Axioms are statements that define the relationships between concepts and relations. It is a formal representation of knowledge that is designed to facilitate automated reasoning and information processing. It acts as a structured vocabulary that describes a domain and promotes interoperability, data integration, and communication between humans and machines. Formally, an ontology $O$ can be represented as a tuple $(C, R, A)$, where $C$ is the set of concepts, $R$ is the set of relations, and $A$ is the set of axioms. Each concept \textit{c} $\in$ $C$ can be represented as a set of attributes, denoted as $Att(c)$. Similarly, each relation \textit{r} $\in$ $R$ can be represented as a set of attributes, denoted as $Att(r)$.

% Ontology is a branch of philosophy that deals with the nature of existence and being. In the field of computer science, however, ontology refers to a formal representation of knowledge that is designed to facilitate automated reasoning and information processing. It is a structured vocabulary that describes a domain and promotes interoperability, data integration, and communication between humans and machines. Various tools and methodologies, including Protege and ontology editors, are available for ontology creation. Ontologies are increasingly important in artificial intelligence, knowledge engineering, and the semantic web, and researchers are exploring their potential in diverse domains and applications.

% Figure environment removed

\subsection{Automated Planning}

Automated planning, also known as AI planning, is the process of finding a sequence of actions that will transform an initial state of the world into a desired goal state \cite{ghallab2004automated}. It involves constructing a plan or a sequence of actions that will achieve a specified objective while respecting any constraints or limitations that may be present. Formally, automated planning can be defined as a tuple $(S, A, T, I, G)$, where:
\begin{itemize}
    \item $S$ is the set of possible states of the world
    \item $A$ is the set of possible actions that can be taken
    \item $T$ is the transition function that describes the effects of taking an action on the current state of the world
    \item $I$ is the initial state of the world
    \item $G$ is the desired goal state
\end{itemize}
Using this notation, the problem of automated planning can be framed as finding a sequence of actions $\prec a_1, a_2, ..., a_k\succ$ that will transform the initial state $I$ into the goal state $G$, while respecting any constraints or limitations on the actions. 
 % In automated planning, 
 A problem is defined in terms of a domain and a problem instance. The domain defines the possible actions that can be taken and the effects of each action, while the problem instance specifies the initial state of the world and the desired goal state. 
Various techniques can be used to solve the planning problem, such as search algorithms, constraint-based reasoning, and optimization methods. These techniques involve exploring the space of possible plans and selecting the one that satisfies the objective and any constraints. Figure \ref{fig:planning_bw} illustrates an automated planning scenario for the blocksworld domain, where an initial state can be transformed into a goal state by executing a sequence of actions.

% \noindent \textbf{Attributes modeled about a domain.}
%   %\noindent \textbf{Attributes modeled in a domain file}
%  \begin{enumerate}
%      \item \textbf{Requirements:} A list of requirements that the planner must satisfy in order to solve the domain. Requirements include durative actions, conditional effects, or negative preconditions. For example, in blocksworld domain with types involved, one of the requirements is \emph{typing}.
%     \item \textbf{Predicates:} Predicates are fundamental elements in the planning domain that define the properties of the world. They are used to describe the initial and goal states, as well as the preconditions and effects of actions. Predicates are usually defined as logical expressions over a set of variables, where each variable can take on a finite number of values. In the context of planning, predicates are typically used to represent facts about the world that can be true or false, such as the location of an object or the status of a machine. For example, in blocksworld domain, the predicate \verb|(on b1 b2)| could indicate that block 'b2' is on top of block 'b1'.
%      \item \textbf{Actions:} Actions are the basic units of change in the planning domain. They represent atomic operations that can be performed to transform the world from one state to another. Each action has a name, a set of parameters, preconditions that must be satisfied before the action can be executed, and effects that describe the changes that the action makes to the world. Actions can be used to model a wide variety of operations, ranging from simple movements or transformations to complex processes such as planning or decision-making. For example, in blocksworld domain, the action \verb|unstack b2 b1| can be used to unstack block 'b2' from block 'b1'. 
     
%      \item \textbf{Preconditions:} Preconditions are the conditions that must be true before an action can be executed. They are usually defined using predicates and can involve multiple variables. Preconditions can also be negative, which means that a certain condition must not be true for an action to be executed. In planning, preconditions ensure that actions are only executed when the necessary conditions have been met, such as ensuring that a machine is turned off before it is serviced. For example, in blocksworld domain, the action \verb|unstack b2 b1| has a precondition of \verb|(on b1 b2)|, meaning that for the action to be valid, the block 'b2' should be on top of block 'b1'.
     
%      \item \textbf{Effects:} Effects describe the changes that an action makes to the world. They are usually defined using predicates and can involve multiple variables. Effects can be positive, which means that a certain condition becomes true after the action is executed, or negative, which means that a certain condition becomes false after the action is executed. In the context of planning, effects are used to model the changes that result from executing an action, such as moving an object from one location to another or turning a machine on. For example, in blocksworld domain, when the action \verb|unstack b2 b1| is executed, one of its effect is \verb|(not (on b1 b2))|, indicating that block 'b2' is no longer on top of block 'b1'.
     
%      \item \textbf{Constants:} Constants are values that are fixed and do not change during the execution of the planning problem. They are used to represent objects or entities in the world that have a fixed value, such as the speed limit on a road. Constants can be used to simplify the planning problem by reducing the number of variables that need to be considered and by providing a fixed set of values that can be used in predicates and actions. For example, in blocksworld domain, the constant \emph{table} could represent the surface on which the blocks are initially placed.
     
%      \item \textbf{Types:} Types are used to classify objects or entities in the world based on their attributes or properties. They are used to define the domain of values that a variable can take on and can be used to constrain the values that are assigned to variables. In the context of planning, types are typically used to group related objects or entities together, such as cars or bicycles, and to specify the properties that are common to all members of a type, such as their color or size. For example, in blocksworld domain with types involved, one can represent the predicate as \verb|(on ?x - block ?y - block)| stating that the parameters in the predicate are of type \emph{block}.

%  \end{enumerate}


% ######### Shorter version for AI Planning preliminaries
% \subsection{Automated Planning}

% Automated planning, also known as AI planning, finds actions transforming an initial world state into a goal state \cite{ghallab2004automated}. It involves creating a plan, respecting constraints, defined as $(S, A, T, I, G)$ where $S$ is the world states set, $A$ is the actions set, $T$ is the state transition function, $I$ is the initial state, and $G$ is the goal state. The challenge is to find actions $\prec a_1, a_2, ..., a_k\succ$ converting $I$ to $G$ under constraints. 

% A problem has a domain (defining actions and effects) and an instance (specifying initial and goal states). Various techniques can be used to solve the planning problem, such as search algorithms, constraint-based reasoning, and optimization methods. These techniques involve exploring the space of possible plans and selecting the one that satisfies the objective and any constraints. Figure \ref{fig:planning_bw} illustrates an automated planning scenario for the blocksworld domain, where an initial state can be transformed into a goal state by executing a sequence of actions.

\noindent \textbf{Attributes modeled about a domain.}
 \begin{enumerate}
     \item \textbf{Requirements:} A list of requirements that the planner must satisfy to solve the given domain, e.g., \emph{typing} in blocksworld with types.
     \item \textbf{Predicates:} Define world properties, e.g., \verb|(on b1 b2)| in blocksworld.
     \item \textbf{Actions:} Units of change with preconditions and effects, e.g., \verb|unstack b2 b1| in blocksworld.
     \item \textbf{Preconditions:} Conditions for action execution, e.g., \verb|(on b1 b2)| for \\ \verb|unstack b2 b1|.
     \item \textbf{Effects:} Post-action world changes, e.g., \verb|(not (on b1 b2))| after \\ \verb|unstack b2 b1|.
     \item \textbf{Constants:} Fixed values, e.g., \emph{table} in blocksworld.
     \item \textbf{Types:} Classifications based on attributes, e.g., \\ \verb|(on ?x - block ?y - block)| in typed blocksworld.
 \end{enumerate}

\noindent \textbf{Attributes modeled about a problem instance from a domain.}
\begin{enumerate}
    \item \textbf{Name:} The name of the planning problem.
    \item \textbf{Domain:} The name of the planning domain that the problem belongs to.
    \item \textbf{Objects:} A list of objects that are present in the planning problem. Objects are typically defined in terms of their type and name. In the example shown in Figure \ref{fig:planning_bw}, objects are b1, b2, and b3.
    \item \textbf{Initial State:} A description of the initial state of the world, including the values of all relevant predicates. Figure \ref{fig:planning_bw} represents an example initial state.
    \item \textbf{Goal State:} A description of the desired goal state of the world, including the values of all relevant predicates. Figure \ref{fig:planning_bw} represents an example goal state.
\end{enumerate}

% \vspace{2cm}
\subsection{International Planning Competition (IPC)}

% IPC serves as a significant means of assessing and comparing various planning systems. By presenting new planners and benchmark problems each year, the competitions aim to stimulate the advancement of new planning methodologies and reflect current trends and challenges in the field. The competition comprises multiple tracks, each covering various planning problems such as classical, temporal, and probabilistic planning. These tracks include benchmark problems that evaluate the performance of planners concerning parameters such as plan quality, plan length, and run time. The results of these competitions provide insights into the current state-of-the-art in planning and help identify the strengths and weaknesses of different planning systems. IPC can serve as an excellent starting point for building a planning-related ontology as the benchmark problems used in these competitions can provide a comprehensive overview of the domain and the types of problems that planners need to solve. 

IPC is pivotal for evaluating and contrasting planning systems. Introducing new planners and benchmarks, it promotes innovative planning methodologies and reflects the field's evolving challenges. The competition has multiple tracks, such as classical and probabilistic planning, with benchmarks assessing plan quality, length, and run time. IPC results offer a glimpse into the latest in planning, highlighting system pros and cons. The benchmarks from IPC are ideal for crafting a planning-related ontology, encapsulating the domain's breadth and planners' challenges.


\vspace{-0.15cm}

\section{Strategy Templates}\label{sec:templates}

In this section, we introduce a formalization of player \Odd strategies in \Odd-fair parity games via \emph{strategy templates}.
% 
In contrast to player \Even, player \Odd winning strategies are no longer positional in \Odd-fair parity games, as illustrated by the following example. %that requires the same number of symbolic steps as the algorithm computing winning strategies for \Even in \enquote{normal} parity games.
% \vspace{-0.5em}
\begin{example}\label{ex:strategytemplates}
Consider the three different parity games depicted in Fig.~\ref{fig:Oddstrategies1}. %, three \Odd-fair parity games are depicted, with circles indicating \Ve and squares indicating \Vo. Edges in $E^\ell$ are shown by dashed lines. All nodes are labeled with their priorities.
   In all three games, \Odd has a winning strategy from all vertices, i.e., $\mathcal{W}_{Odd}=V$. %The red-colored edges indicate \Odd's strategy: if \Odd takes the red edges alternatingly from the source nodes, it wins from all nodes. 
  However, in order to win, the vertex $3$ has to be seen infinitely often in game (a) and (b), which forces \Odd to use its live edge\textbackslash s infinitely often. This prevents the existence of a positional strategy for \Odd in games (a) and (b): In (a) it needs to somehow alternate between (it's only) live edge to $4$ and a \enquote{normal} edge to $7$ (both indicated in red) in order to win, and in (b) it needs to somehow alternate between all its live edges (also indicated in red). In the game (c), \Odd can win by 'escaping' its live vertex $3$ to a \enquote{normal} vertex $5$, and thereby has a positional strategy. % (again indicated in red).
   
  Now consider the subgraph of each game formed by all colored edges (red and blue), which include the strategy choices from \Vo and \emph{all} outgoing edges from \Ve. As we have seen that \Odd needs to play all red edges repeatably, this subgraph represents the paths that \emph{can} be seen in the game depending on the \Even strategy. Hence, a node $v\in\Vl\subseteq\Vo$ can be seen infinitely often in a play (compliant with \Odd's strategy), if it lies on a cycle in this subgraph. We observe that, in games (a) and (b), node $3$ lies on cycles in this subgraph, whereas in game (c), it does not. 
  We further see that whenever a vertex  $v\in\Vl$ lies on a cycle, \Odd needs to take all its outgoing live edges (as for vertex $3$ in example (b)) and possibly one more edge (as for vertex $3$ in example (a)), for all other vertices in $\Vo$ a positional strategy suffices (as for vertex $5$ in all examples, and for vertex $3$ in example (c)). This shows that \Odd strategies are intuitively still \enquote{almost positional}.
% 
\end{example}

% Figure environment removed


\vspace*{-0.2cm}

The intuitions conveyed by Ex.~\ref{ex:strategytemplates} are formalized by the following definitions. % for \Odd strategy templates.


\begin{definition}[\Odd Strategy Template]\label{def:Oddstrategytemplate}
 Given an \Odd-fair parity game $\mathcal{G}^\ell = \ltup{\mathcal{G}, E^\ell}$ with \newline $\mathcal{G} = \langle V, \Ve, \Vo, E, \chi\rangle$, an \Odd \emph{strategy template} $\mathcal{S}$ over $\mathcal{G}^\ell$ is a subgraph of $\mathcal{G}$ given as follows: $\mathcal{S}:=\tup{V',E'}$ where $V'\subseteq V$ and $E'\subseteq E \cap (V' \times V')$ such that the following hold,
\begin{compactitem}\label{item:Oddstrtemprules}
 \item if $v \in \Vo \cap V'$ does not lie on a cycle in $(V',E')$, then $|E'(v)|=1$,
 \item if $v \in \Vo \cap V'$ lies on a cycle in $(V',E')$ then $E^\ell(v) \subseteq E'(v)$ and  $1\leq |E'(v)|\leq |E^\ell(v)| + 1$,
 \item if $v \in \Ve \cap V'$, then  $E'(v) = E(v)$.
\end{compactitem}
\end{definition}
%
\begin{definition}\label{def:compliantstrat}
 Let  $\mathcal{G}^\ell = \ltup{\mathcal{G}, E^\ell}$ be an \Odd-fair parity game with \Odd strategy template $\mathcal{S}=\tup{V',E'}$, and $V'_\Odd := V' \cap V_\Odd$. Then an
\Odd strategy $\rho$ is said to be \textbf{compliant} with $\mathcal{S}$ if  
it is a winning strategy in the game $\ltup{\gamegraph,\alpha'}$ where $\gamegraph= \tup{V,\Ve,\Vo,E}$ and 
\begin{subequations}
 \begin{align}
 \alpha':= &\textstyle\bigwedge_{v\in\Vo'}(\,\square\, (\,v \implies \bigvee_{(v,w)\in E'} \bigcirc\, w\,))\,\label{equ:alpha:a}\\
 & \textstyle\wedge \bigwedge_{v\in\Vo'} (\,\square \,\diamondsuit\, v \implies \bigwedge_{(v,w)\in E'}\square\, \diamondsuit\, (\,v \wedge \bigcirc \,w\,)).\label{equ:alpha:b}
\end{align}
\end{subequations}
\end{definition}

Intuitively, for all \Odd vertices in $\mathcal{S}$, the strategy $\rho$ compliant with $\mathcal{S}$ takes only their outgoing edges in $\mathcal{S}$ \eqref{equ:alpha:a}, and if a play visits an \Odd node $v$ infinitely often, then $\rho$ takes each of $v$'s outgoing edges in $\mathcal{S}$ infinitely often \eqref{equ:alpha:b}.
% 
For an \Odd strategy template $\mathcal{S}$, if $v \in V'_\Odd$ lies on a cycle in $\mathcal{S}$, then by Def. \ref{def:Oddstrategytemplate}, $\mathcal{S}$ contains all live outgoing edges of $v$. By \eqref{equ:alpha:b} any \Odd strategy $\rho$ compliant with $\mathcal{S}$ satisfies the fairness condition in \eqref{eq:fairness-ltl} for $v$. 
On the other hand, if $v \in V'_\Odd$ does not lie on a cycle in $\mathcal{S}$, then by \eqref{equ:alpha:a} any such $\rho$ sees $v$ at most once. Thus $\rho$ trivially satisfies \eqref{eq:fairness-ltl} for $v$. 
This observation is stated in the following proposition.


\begin{proposition}
 Given the premisses of Def.~\ref{def:compliantstrat} let $\pi$ be a play starting from a node in $V'$ that complies with $\rho$. Then $\pi \models \alpha$ where $\alpha$ if the LTL formula in~\eqref{eq:fairness-ltl}.%\vspace{-2mm}
\end{proposition}

Next, we define \Even strategy templates. Each \Even strategy template encodes a unique \Even positional strategy, which is known to exist in \Odd-fair parity games \cite{banerjee2022fast}, due to the lack of fair edges defined on \Even vertices. %, \Even strategy templates are very simple\footnote{In fact, \Even strategy templates simply encode a positional strategy and are only re-defined to make further arguments more symmetric for both players.}.
\begin{definition}\label{def:Evenstrategytemplate}
    Given an \Odd-fair parity game $\mathcal{G}^\ell = \ltup{\mathcal{G}, E^\ell}$ with \newline $\mathcal{G} = \langle V, \Ve, \Vo, E, \chi\rangle$, an \Even \emph{strategy template} $\mathcal{S}$ over $\mathcal{G}^\ell$ is a subgraph of $\mathcal{G}$ given as $\mathcal{S}:=\tup{V', E'}$ where $V'\subseteq V$ and $E'\subseteq E \cap (V' \times V')$ such that,    \begin{compactitem}\label{item:Evenstrtemprules}
     \item if $v \in \Ve \cap V'$, then $|E'(v)|=1$,
     \item if $v \in \Vo \cap V'$, then  $E'(v) = E(v)$.
    \end{compactitem}
\end{definition}

\vspace*{-0.1cm}

An \Even strategy $\rho$ is compliant with the \Even strategy template $\mathcal{S} = \tup{V', E'}$ if for all $v \in V'_\Even$, $\rho(v) = E'(v)$. In other words, $\rho$ is the positional strategy defined by $\mathcal{S}$.

Let $\rho$ be an \Odd (\Even) strategy, compliant with the \Odd (\Even) strategy template $\mathcal{S}$ and let $\pi$ be a play compliant with $\rho$. Then we call $\pi$ a play \emph{compliant with $\mathcal{S}$}.

\vspace*{-0.1cm}

\begin{definition}
An \Odd (\Even) strategy template $\mathcal{S}=\ltup{V', E'}$ is \emph{winning} in the \Odd-fair parity game $\mathcal{G}^\ell$ if all \Odd (\Even) strategies $\rho$ compliant with $\mathcal{S}$ are winning for player \Odd (\Even) in $\mathcal{G}^\ell$ from $V'$. A winning \Odd (\Even) strategy template $\mathcal{S}$ is called \emph{maximal} if $V'=\Wo$ ($\We$).%\vspace{-2mm}
\end{definition}

\vspace*{-0.2cm}
We note that maximal winning \Odd (\Even) strategy templates $\mathcal{S}$ immediately imply that for every vertex $v\in \Wo$ ($\We$) there exists a winning strategy for player \Odd (\Even) from $v$ that is compliant with $\mathcal{S}$.
% 
The existence of maximal winning \Even strategy templates follows from the existence of positional \Even strategies \cite{banerjee2022fast}. 
% 
The first main contribution of this paper is a constructive proof showing the existence of maximal winning \Odd strategy templates given in the next section. 
This result is then used in Sec.~\ref{sec:zielonka} to prove the correctness of \Odd-fair Zielonka's algorithm, which is introduced there.





\vspace*{-0.25cm}
\section{Existence of Maximal Winning \Odd Strategy Templates}\label{sec:strat-templates}

This section proves the existence of maximal winning \Odd strategy templates\footnote{In the rest of this section, we will sometimes call \Odd strategy templates simply, \emph{strategy templates}, since these are the only strategy templates we will be dealing with.} in \Odd-fair parity games, formalized in the following theorem.

\begin{theorem}\label{thm:existence-maximaloddstrategytemplates}
    Given an \Odd-fair parity game $\mathcal{G}^\ell$, there exists a maximal winning \Odd strategy template. 
\end{theorem}

We prove Thm.~\ref{thm:existence-maximaloddstrategytemplates} by giving an algorithm which constructs $\mathcal{S}$ from a ranking function induced by a fixed-point algorithm in the $\mu$-calculus which computes \Wo. Towards this goal, Sec.~\ref{sec:assump:prelim} first introduces necessary preliminaries, Sec.~\ref{sec:templates:solving} gives the fixed-point algorithm to compute \Wo and Sec.~\ref{sec:templates:ranking} formalizes how to extract a strategy template $\mathcal{S}$ from the ranking induced by this fixed-point and proves that $\mathcal{S}$ is indeed maximal and winning. % from this computation and Sec.~\ref{sec:templates:result} finally shows how this ranking can be used to construct maximal winning strategy templates and proves their correctness. 

While this section uses fixed-point algorithms extensively to \emph{construct} a maximal winning \Odd strategy template towards a \emph{proof} of Thm.~\ref{thm:existence-maximaloddstrategytemplates}, we note again that the proof of the new Zielonka's algorithm given in Sec.~\ref{sec:zielonka} only uses the \emph{existence} of templates (i.e., the fact that Thm.~\ref{thm:existence-maximaloddstrategytemplates} holds) and does not utilize their \emph{construction} via the algorithm presented here. %.\todo{But in the proofs we sometimes construct a strategy template, no? Maybe we can say "does not construct one" instead of "does not require the construction"}


\subsection{Preliminaries on Fixed-Point Algorithms}\label{sec:assump:prelim}
% 

This subsection contains the basic notation used in this section. 

\smallskip
\noindent\textbf{Set Transformers.}  Let $ \gamegraph=(V,\Ve, \Vo, E) $ be a game graph, $ S,T\subseteq V $ and $\bb$ be the player index\footnote{$\bb \in \{\Even,\Odd\} $ where $\bb=\Even$ implies $\neg \bb=\Odd$, and vice versa}. Then we define the following predecessor operators: 
\begin{subequations}\label{equ:Pres}
 \begin{align*}    
    \Pre_\bb^\exists(S) &:= \{ v \in V_\bb \mid E(v) \cap S \neq \emptyset \} && 
        \Lpre^\exists(S) := \{v \in \Vo \mid E^\ell(v) \cap S \neq \emptyset\} \notag \\ 
        \Pre_\bb^\forall(S) &:= \{ v \in V_\bb \mid E(v) \subseteq S  \} &&
    \Lpre^\forall(S) := \{v \in \Vo \mid E^\ell(v) \subseteq S \}\quad  (3)
     \end{align*}
\end{subequations}

The predecessor operators $\Pre_\bb^\exists(S) $ and $\Pre_\bb^\forall(S)$ compute the sets of vertices with \emph{at least one} successor and with \emph{all} successors in $ S $, respectively. The live predecessor operators  $ \Lpre^\exists(S) $ and $\Lpre^\forall(S)$ restrict this analysis to live edges.
We see that 
% 
 \begin{align}    \label{equ:Preseq}
   \neg \Pre_\bb^{\exists}(\neg S)&= V_{\neg \Lambda} \cup \Pre_{\neg \bb}^{\forall}(S)&&\text{and}&&
   \neg \Lpre^{\exists}(\neg S)= \Ve \cup \Lpre^{\forall}(S)%\vspace{-2mm}
 \end{align}
% 
where for a set $X \subseteq V$, $\neg X$ stands for $V \setminus X$. We combine the pre-operators from \eqref{equ:Pres} into the combined set transformer\footnote{Note that $\Apre(S,T)$ and $\Npre(S,T)$ are meaningful only when $T \subseteq S$ and $S \subseteq T$, respectively. Otherwise they are equivalent to $\Cpre_\Even(T)$ and $\Cpre_\Odd(T)$. We note that these preconditions will always be satisfied in our calculations due to the monotonicity of fixed-point computations.}:
\begin{subequations}\label{equ:combindedPres}     
     \begin{align}
    \Cpre_\bb(S) &:= \Pre_\bb^\exists(S) \cup \Pre_{\nb}^\forall(S)\label{equ:cpre}\\
    \Apre(S, T) &:= \Cpre_\Even(T) \cup (\Lpre^{\exists}(T) \cap \Pre_\Odd^{\forall}(S))\label{equ:apre}\\
    \Npre(S,T) &:= \Cpre_\Odd(T) \cap (\Ve \cup \Lpre^\forall(T) \cup \Pre_\Odd^{\exists}(S))\label{equ:npre}       
    \end{align}
\end{subequations}
% 
The \emph{controllable predecessor operator} $\Cpre_\bb(S)$ computes the set of vertices from which player $\bb$ can force visiting $ S $ in \emph{one} step. It immediately follows that 
%
\begin{align}
\neg \Cpre_\Even(\neg S)&:= \Cpre_\Odd(S)\label{equ:cpre_equal}.
\end{align}
% 
The \emph{almost-sure controllable predecessor} operator $\Apre(S,T)$ computes the set of states that can be controlled by Player \Even to stay in $T$ (via $\Cpre_\Even(T ))$ as well as all Player \Odd states in $V^\ell$ that
(a) will eventually make progress towards $T$ if Player \Odd obeys its fairness-assumptions (via $\Lpre^{\exists}$) and (b) will never leave $S$ in the \enquote{meantime} (via $\Pre_\Odd^{\forall}(S))$). Using \eqref{equ:Preseq} and \eqref{equ:cpre_equal} we have 
  $\Npre(S,T):= \neg \Apre(\neg S, \neg T)$.


\smallskip
\noindent\textbf{Fixed-point Algorithms in the $ \mu $-calculus.} 
$ \mu $-calculus offers a succinct representation of symbolic algorithms (i.e., algorithms manipulating sets of vertices instead of individual vertices) over a game graph $ \gamegraph $. 
We omit the (standard) syntax and semantics of $ \mu $-calculus formulas (see \cite{Kozen:muCalculus}) and only discuss their evaluation
%  
on an example fixed-point algorithm given by a 2-nested $ \mu $-calculus formula of the form $Z=\mu Y.~\nu X.~\phi(X,Y)$, where  $ X,Y \subseteq V$ are subsets of vertices
 and $ \mu $ and $ \nu $ denote, respectively, the least and the greatest fixed-point. $\phi$ is a formula composed from the \emph{monotone set transformers} in  \eqref{equ:Pres} and \eqref{equ:combindedPres}. % of the functional defined as $ X\mapsto \phi(X) $. 
 
 Given this formula, first, both formal variables $X$ and $Y$ are initialized. As $Y$ (resp. $X$) is preceded by $\mu$ (resp. $\nu$) it is initialized with $Y^0:=\emptyset$ (resp. $X^0:=V$). Now we first keep $Y$ at its initial value and iteratively compute $X^k=\phi(X^{k-1},Y^0) $ until $X^{k+1}=X^k$. At this point $X$ saturates, denoted by $X^\infty$. We then \enquote{copy} $X^\infty$, to $Y$, i.e., have $Y^1:=X^\infty$, reinitialize $X^0:=\emptyset$, and re-evaluate $X^k=\phi(X^{k-1},Y^1) $ with the new value of $Y$. This calculation terminates if $Y$ saturates, i.e.,  $Y^\infty=Y^{l+1}=X^l$ for some $l\geq 0$, and outputs $Z=Y^\infty$. In order to remember all intermediate values of $X$ we use $X^{l,k}$ to denote the set computed in the $k$-th iteration over $X$ during the computation of $Y^l$. I.e., $Y^l=X^{l,\infty}$.

\smallskip
\noindent\textbf{Additional Notation.} 
We will use the letters $l,m$ and $n$ exclusively to denote \emph{even} positive integers. For $a \leq b \in \mathbb{N}$, we will use the regular set symbol $[a,b]$ to denote the set of all integers between $a$ and $b$, i.e., $[a,b]:=\{a, a+1 , \ldots , b\}$; and $\ev{a}{b}$ to denote all the \emph{even} integers between $a$ and $b$. %, including $a$ or $b$ as well given that it is even,
E.g. $\ev{2}{7} = \{2, 4,  6\}$.
In addition, given an \Odd-fair parity game $\mathcal{G}^\ell$, we define the sets $C_i := \{ v \in V \mid \chi(v) = i\}$ and $\overline{C_i} := V \setminus C_i$ to ease notation. We say $\mathcal{G}^\ell$ has 
the least even upperbound $l$ if $C_l \cup C_{l-1}\neq\emptyset$ and $C_i=\emptyset$ for all $i>l$.


\vspace{-0.1cm}
\subsection{A Fixed-Point Algorithm for $\mathcal{W}_{\Odd}$}\label{sec:templates:solving}


Given an  \Odd-fair parity game $\mathcal{G}^\ell = \ltup{\langle V, \Ve, \Vo, E, \chi \rangle, E^\ell}$ this section presents a fixed-point algorithm in the $\mu$-calculus which computes the winning region $\Wo$ of player $\Odd$ in \Odd-fair parity games. It is obtained by negating the fixed-point formula computing \We \,in~\cite{banerjee2022fast}, formalized in the following proposition and proven in App.~\ref{app:fp-proof}.

\begin{proposition}\label{prop: W_Odd}
Given an \Odd-fair parity game $\mathcal{G}^\ell = (\ltup{V, \Ve, \Vo, E, \chi}, E^\ell)$ with least even upperbound $l\geq 0$ it holds that $Z=\Wo$, where
\begin{small}
\begin{align}\label{eq:fp-odd}
    Z &:=\textstyle \mu {Y_l}.~  \nu {X_{l-1}}.~  \ldots \mu{Y_2}.~  \nu{X_1}.~  \bigcap_{j \in \ev{2}{l}} \B_j[Y_j, X_{j-1}], \\ \vspace{0.1cm}
    &\text{ where} \quad
    \B_j[\mathbf{Y}, \mathbf{X}] := \left(\textstyle\bigcup_{i \in [j+1,l]} C_i\right) \cup \left(\overline{C_j} \cap \Npre(\mathbf{Y}, \mathbf{X}) \right) \cup \left(C_j \cap \Cpre_\Odd(\mathbf{Y})\right).\nonumber
\end{align}
\end{small}
% 
\end{proposition}


Before utilizing \eqref{eq:fp-odd} we illustrate its computations via an example. %IRMAAAAK IRMAK


\begin{example}\label{ex:1}
Consider the \Odd-fair parity game $\mathcal{G}^\ell $ depicted in Fig.~\ref{fig:ex1} (left). Here, the name of the vertices coincide with their priorities, e.g., $C_2=\set{2a, 2b, 2c}$. $\Ve$ and $\Vo$ are indicated by circles and squares, respectively. Edges in $E^\ell$ are shown by dashed lines. 
% 
As the least even upperbound in this example is $l=4$, 

\vspace*{-0.3cm}
\begin{small}
\begin{align}\label{equ:fpexample}
    &Z = \mu Y_4.~ \nu X_3.~ \mu Y_2.~ \nu X_1.~ \Phi^{Y_4, X_3, Y_2, X_1}~\quad  \text{where}\\
    &\Phi^{Y_4, X_3, Y_2, X_1}:= (\overline{C_4} \cap \Npre(Y_4, X_3)) \cup (C_4 \cap \Cpre_\Odd(Y_4)))\nonumber\\
    & \hspace{2.03cm}\cap (\overline{C_2} \cap \Npre(Y_2, X_1)) \cup (C_2 \cap \Cpre_\Odd(Y_2)) \cup C_4 \cup C_3)\nonumber.
\end{align}
\end{small}

\vspace{-0.2cm}

% Figure environment removed

\vspace{-0.2cm}

Using the notation defined in Sec.~\ref{sec:assump:prelim}, we initialize  \eqref{equ:fpexample} by $Y_4^{0} = \emptyset$, $X_3^{0, 0} = V$, $Y_2^{0,0,0} = \emptyset$ and $X_1^{0,0,0,0} = V$ and observe from \eqref{equ:combindedPres} that  $\Cpre_\Odd(\emptyset)=\emptyset$ and $\Npre(\emptyset, V)=V$. We obtain 
\begin{small}
\begin{align*}
 X_1^{0,0,0,1} &= \Phi^{Y_4^{0}, X_3^{0, 0}, Y_2^{0,0,0}, X_1^{0,0,0,0} }
 =((\overline{C_4} \cap \Npre(\emptyset, V)) \cup (C_4 \cap \Cpre_\Odd(\emptyset)))\cap ((\overline{C_2} \cap \Npre(\emptyset, V)) \\ 
 & \quad \,\, \cup (C_2 \cap \Cpre_\Odd(\emptyset)) \cup C_4 \cup C_3) =(\overline{C_4} ) \cap (\overline{C_2} \cup C_4 \cup C_3) =C_3 \cup C_1\\
%   
%   
  X_1^{0,0,0,2} &= \Phi^{Y_4^{0}, X_3^{0, 0}, Y_2^{0,0,0}, X_1^{0,0,0,1} }\\
  &= C_3 \cup (C_1 \cap \Npre(Y_2^{0,0,0}, X_1^{0,0,0,1})) = C_3 \cup (C_1 \cap \Npre(\emptyset, C_3\cup C_1))=C_3
%   
\end{align*}
\end{small}

\vspace{-0.1cm}

where $ \Npre(\emptyset, C_3\cup C_1)=\emptyset$ as $v \in \Npre(\emptyset, C_3 \cup C_1)$ implies $v\in \Cpre_\Odd(C_3 \cup C_1) = \{2b,4a\}$ and $v\in \Ve \cup \Lpre^\forall(C_3 \cup C_1)$. However, $2b, 4a$ are \Odd vertices with live outgoing edges to $2a,2c\in (V \setminus (C_3 \cup C_1))$.
% 
In the next iteration, we again get $X_1^{0,0,0,3} = C_3$ and thus $X_1$ saturates with $C_3$. Therefore, $Y_2^{0,0,1}=C_3$. Now the next round of computations of $\Phi$ results in 
\begin{small}
\begin{align*}
   X_1^{0,0,1,1} &= \Phi^{Y_4^{0}, X_3^{0, 0}, Y_2^{0,0,1}, X_1^{0,0,1,0} } =  C_3 \cup (C_1 \cap \Npre(Y_2^{0,0,1}, X_1^{0,0,1,0})) \cup (C_2 \cap \Cpre_\Odd(Y_2^{0,0,1}))\\
 & =C_3 \cup (C_1 \cap \Npre(C_3, V)) \cup (C_2 \cap \Cpre_\Odd(C_3))=C_3 \cup C_1 \cup \{2b\}\\
    X_1^{0,0,1,2} &= \Phi^{Y_4^{0}, X_3^{0, 0}, Y_2^{0,0,1}, X_1^{0,0,1,1} } = C_3 \cup \{2b\}=X_1^{0,0,1,3} 
% 
\end{align*}
\end{small}
Here $C_1$ and $\{2b\}$ get added in $X_1^{0,0,1,1}$ as $1a \in \Npre(C_3, V)$ trivially and $2b \in \Cpre_\Odd (C_3)$ due to the edge $(2b,3b)$. $C_1$ is removed from $X_1^{0,0,1,2}$ since
$1a$ cannot be forced by \Odd to $C_1 \cup C_3 \cup \{2b\}$ in the next step.
%$1a \not \in ( C_1 \cap \Npre(Y_2^{0,0,1}, X_1^{0,0,1,1} )) = (C_1 \cap \Npre(C_3, C_1 \cup C_3 \cup \{2b\}))$ since $1a \not \in \Cpre_\Odd(C_1\cup C_3 \cup \{2b\})$.
The fixed-point calculation proceeds in a similar fashion, until $Y_4$ reaches its saturation value $V \setminus \{2a\}$. 
The full computation of $Z$ is given in App.~\ref{app:example}. %\vspace{-2mm}
\end{example}


\subsection{Construction of a Rank-based Strategy Template}\label{sec:templates:ranking}
Given an \Odd-fair parity game $\mathcal{G}^\ell$ with least even upperbound $l\geq 0$, we define a ranking function $\rank{}: \Wo \to \mathbb{N}^{l}$. Intuitively, $\rank{v}$ indicates in which iteration $v$ was added to $Z$ in \eqref{eq:fp-odd} and  never got removed from $Z$ again, as illustrated by the following example. %We show that there exists a strategy template $\Sc=(V',\Ve',\Vo', E')$ of $\mathcal{G}^\ell$, constructed according to $r$, for which all compliant player \Odd strategies are winning in $\mathcal{G}^\ell$.


\begin{example}\label{ex:2}
 Consider again the \Odd-fair parity game depicted in Fig.~\ref{fig:ex1}. Here, $\rank{v}$ of each $v \in \Wo = V \setminus \{2a\}$ is shown in red next to the node in the figure. Intuitively, the $4-$tuple is associated with the subscript $Y_4,Y_3,Y_2,Y_1$ of $\Phi$ in \eqref{equ:fpexample}. For instance $\rank{3a}=(2,0,1,0)$ indicates that $3a$ was added to $Z$ 
 during the first iteration of $Y_2$ inside the second iteration of $Y_4$.
 More concretely, $3a \not \in Y_4^{0}, 3a \not \in Y_4^1, 3a \in Y_4^2$. So $2$ is the first iteration of the $Y_4$ variable in which $3a$ got included in the variable. For $Y_2$, $3a \not \in Y_2^{2,0, 0}$ and $3a \in Y_2^{2,0,1}$, and therefore $\rank{3a} =  (2,0,1,0)$.
\end{example}

The intuition of Ex.~\ref{ex:2} is formalized in the following definition.

\begin{definition}[rank]\label{def:rank}
Given an \Odd-fair parity game $\mathcal{G}^\ell = (\ltup{V, \Ve, \Vo, E, \chi}, E^\ell)$ with least even upperbound $l\geq 0$ and winning region $\Wo\subseteq V$, we define the ranking function $\rank{}: \Wo \to \mathbb{N}^{l}$ for $v\in \Wo$ such that 
 \begin{equation}\label{eq:rank}
  \textstyle\rank{v}=(r_l,0,r_{l-1},0\hdots r_2, 0) \quad\text{if}\quad v\in \bigcap_{j\in\ev{2}{l}}Y_j^{r_l,0,\hdots ,r_j}\setminus Y_j^{r_l,0,\hdots ,r_j-1}.
 \end{equation}
where the valuations of $Y_j$ variables are obtained from the iterations of the fixed-point calculation in~\eqref{eq:fp-odd} as illustrated in Ex.~\ref{ex:1}.
\end{definition}


% \section{Strategy Templates}
A ranking function obtained from a fixed-point computation as in \eqref{eq:rank} naturally gives rise to a positional winning strategy for the respective player in (normal) $\omega$-regular games that allow for positional strategies. The corresponding positional strategy is obtained by always choosing a \emph{minimum ranked successor} in the winning region\footnote{See \cite{banerjee2022fast} for a similar construction of the positional winning strategy of \Even in \Odd-fair parity games}.
% 
We use this insight to obtain a \emph{candidate} maximal strategy template for player \Odd (which we prove to be also \emph{winning} in Prop.~\ref{prop:mainresult}) as follows.
% 
We start with a subgraph on \Wo defining the minimum ranked successor strategy for \Odd induced by the ranking in \eqref{eq:rank}, and then iteratively add all live edges of nodes that lie on a cycle in the subgraph, to the subgraph. The saturated subgraph then defines a strategy template for \Odd, as formalized next. 

\begin{definition}[Rank-based Strategy Template]\label{def:S}
    Given an \Odd-fair parity game $\mathcal{G}^\ell = (\ltup{V, \Ve, \Vo, E, \chi}, E^\ell)$ with least even upper bound $l\geq 0$ on the priorities of nodes, winning region $\Wo\subseteq V$ and the ranking function $\rank{}: \Wo \to \mathbb{N}^{l}$ from Defn.~\ref{def:rank}, we define a strategy template $\Sc^{\mathcal{G}^\ell}=(\Wo,E')$ where $E'$ is constructed as follows:
   \begin{enumerate}\label{const:S}
   \item[(S1)] for all $v \in \Ve \cap \Wo$, add all $(v, w)\in E$ to $E'$;
   \item[(S2)] for all $v \in \Vo \cap \Wo$, add $(v,w)\in E$ to $E'$ for a $w$ with %$w$ is the successor of $v$ with minimum rank, i.e., 
   $w=argmin_{w'\in E(v)}\rank{w'}$ ($w$ is arbitrarily picked amongst the successors with the mimimum ranking);
   \item[(S3)] for all $v \in V^\ell\cap \Wo$, add all $(v,w)\in E^\ell$ to $E'$ if $v$ lays on a cycle in $\mathcal{S}^{\mathcal{G}^\ell}$;
   \item[(S4)] repeat item (S3) until no new edges are added.
   \end{enumerate}
   We call $\Sc^{\mathcal{G}^\ell}$ the \emph{minimum rank based maximal \Odd strategy template of $\mathcal{G}^\ell$}.
   \end{definition}
   
      \begin{example}\label{ex:3}
    $\Sc^{\mathcal{G}^\ell}$ for $\mathcal{G}^\ell$ from Ex.~\ref{ex:1} is depicted in Fig.~\ref{fig:ex1} (right). %We see that the live edges originating from vertex $4a$ are not contained in the template as $4a$ cannot be seen infinitely often if player \Odd chooses the minimal rank successor (i.e., moves to $2b$) upon the first visit to $4a$. After that, $4a$ cannot be visited again if player \Odd plays a strategy compliant with the strategy template.
   \end{example}
   
   It is clear from the definition that $\Sc^{\mathcal{G}^\ell}$ is an \Odd strategy template in $\mathcal{G}^\ell$. It is also maximal since each $v \in \Wo$ is assigned a rank. %The next subsection proves that it is also \emph{winning}.
   It remains to show that it is winning:
   
   
    \begin{proposition}\label{prop:mainresult}
        Every player \Odd strategy compliant with $\Sc^{\mathcal{G}^\ell}$ is winning for \Odd in $\mathcal{G}^\ell$.
    \end{proposition}
    
    \noindent The full proof of Prop.~\ref{prop:mainresult} can be found in App.~\ref{app:counter-strategy-templates} and we only give a proof-sketch here.
    
    First, recall that $\Sc^{\mathcal{G}^\ell}$ is obtained by extending a minimum-rank based strategy as formalized in Def.~\ref{def:S}. Based on this we call a play $v_1 v_2 \ldots$ in $\Sc^{\mathcal{G}^\ell}$ \emph{minimal} if for all $v_i \in V_\Odd$, $v_{i+1}$ is the minimum ranked successor of $v_i$. We further call a cycle minimal, if it is a section of a minimal play.
%     
    Now consider a play $\pi= v_0v_1\ldots$ which is compliant with $\Sc^{\mathcal{G}^\ell}$ and $v_0 \in \Wo$.  Since $\pi$ is compliant with an \Odd strategy template, it obeys the fairness condition. It is left to show that $\pi$ is \Odd winning.
    %
    We do this by a chain of three observations,% formally proven in App.~\ref{app:counter-strategy-templates}:
    \begin{enumerate}
     \item If $\Wo \neq \emptyset$, there exists a non empty set $M := \{ v \in \Wo \mid \rank{v} = (1, 0, 1, 0, \ldots, 1, 0)\}$ (see Prop.~\ref{app-prop:Mexists}).
     \item All cycles in $\Sc^{\mathcal{G}^\ell}$ that pass through a vertex in $M$ are \Odd winning (see Prop.~\ref{app-prop:cycle-through-M}).
     \item All infinite minimal plays in $\Sc^{\mathcal{G}^\ell}$ visit $M$ infinitely often (see Prop.~\ref{app-prop:minimal-play-visits-M}).
    \end{enumerate}
    
    While item 1 simply follows from the observation that $(1,0,1,0 ,\ldots, 1, 0)$ is the minimum rank the ranking function assigns to a vertex and the set of nodes with this rank cannot be empty due to the monotonicity of \eqref{eq:fp-odd}, the proofs for item 2 and 3 are rather technical. %They require a careful analysis of the fixed-point algorithm in \eqref{eq:fp-odd} w.r.t.\ ranks over cycles within $\Sc^{\mathcal{G}^\ell}$ and are given in full detail in App.~\ref{app:counter-strategy-templates}.
    
    With the observations in item 1-3 being proven, we are ready to show that $\pi$ is \Odd winning. 
    Observe that $\pi$ \enquote{embeds} an infinite minimal play, that is, there exists a subsequence $\pi' = v_{j_1} v_{j_2} \ldots$ of $\pi$ where $j_1 < j_2 < \ldots$ that is a minimal play. This is because whenever a $v \in V_\Odd \cap \Wo$ is seen infinitely often in $\pi$, $(v, v_{\min})$ should be seen infinitely often as well, where $v_{\min}$ is the minimum-rank successor of $v$ in $\Sc^{\mathcal{G}^\ell}$.
    Since $\pi'$ visits $M$ infinitely often (from item 3), $\pi$ does so too.
    %
    Then due to pigeonhole principle, there exists an $x\in M$ that is visited infinitely often by $\pi$. Thus, a tail of $\pi$ can be seen as consecutive cycles over $x$. Since all cycles that pass through $M$ are \Odd winning (from item 2), we conclude that $\pi$ is \Odd winning.  
    
    Thm.~\ref{thm:existence-maximaloddstrategytemplates} now follows as a corollary of Prop.~\ref{prop:mainresult}.
    


\section{Zielonka's Algorithm for \Odd-Fair Parity Games}\label{sec:zielonka}
In this section, we construct a Zielonka-like algorithm that solves \Odd-fair parity games. We call this algorithm \emph{\Odd-fair Zielonka's algorithm}. We first recall the original Zielonka's algorithm in Sec.~\ref{sec:zielonka:orig} and outline the changes imposed for our new \Odd-fair version in Sec.~\ref{sec:zielonka:fair}. We then discuss the correctness of this new algorithm in Sec.~\ref{sec:zielonka:correct}.


For the remainer of this section, we will take $\mathcal{G}^\ell = \ltup{(V, V_\Even, V_\Odd, E, \chi), E^\ell}$ to be an \Odd-fair parity game. 

\subsection{The Original Zielonka's Algorithm}\label{sec:zielonka:orig}
Intuitively, Zielonka's algorithm consists of two nested recursive functions, 
$\SOLVE_{\Even}(n,\mathcal{G})$ and $\SOLVE_{\Odd}(n,\mathcal{G})$ which compute \We and \Wo in a given parity game $\mathcal{G}$ with, respectively, even or odd upper bound priority $n$. Both functions recursively call each other on a sequence of sub-games that is constructed during the run of the algorithm. 

The main difference between the original Zielonka's algorithm \cite{Zielonka98} and our new \Odd-fair version in Alg.~\ref{algo:fair-zielonka-bb}  
is the computation of the safe reachability set, denoted by $\SafeReach^f_\bb$ within the algorithms. 
Intuitively, the safe reachability set of player \bb  
is the set of vertices from which \bb has a strategy to force the game into the reach set $R\subseteq V$, while staying in the safety set $S\subseteq V$. 
% 
In a (normal) parity game $\mathcal{G}$ (without live edges), this set  
can be computed via the single-nested fixed-point formula
\begin{equation}\label{equ:Xsr1}
 \Xsr_\bb:=\mu X~.~(S \cap (R \cup \Cpre_\bb(X))).
\end{equation}
If one interpretes Alg.~\ref{algo:fair-zielonka-bb} over (normal) parity games $\mathcal{G}$, defines $\SafeReach^f_\bb$ via \eqref{equ:Xsr1} for the respective player, and replaces $\SafeReach^f_\Odd(\cdot,X,\cdot)$ in the last return statement with $X$ (so, the algorithm returns $X$ for any $\Lambda$), one gets exactly Zielonka's algorithm for parity games. 


\vspace*{-0.5cm}

% Figure environment removed


\subsection{The \Odd-fair Zielonka's Algorithm}\label{sec:zielonka:fair}
We are now considering an \Odd-fair parity game $\mathcal{G}^\ell$. % with live edges on \Odd player vertices.
As discussed before, the main difference of the \Odd-fair Zielonka's algorithm from the original one lies in the construction of the safe reachability sets denoted by $\SafeReach^f_\bb$ in Alg.~\ref{algo:fair-zielonka-bb}. We therefore start by discussing its computation for both players.

\smallskip
\noindent\textbf{The \Odd Player.}
The first, somehow surprising, observation is that for player \Odd in \Odd-fair parity game $\mathcal{G}^\ell$, the safe reachability set $\Xsr_\Odd$ can still be computed via \eqref{equ:Xsr1}. This is due to the fact that $R$ only needs to be visited once, and 
\Even vertices do not have live outgoing edges that might prevent player \Odd from forcing a visit to $R$. 

In addition, we can extract a \emph{partial strategy template} for player \Odd from the iterative computation of \eqref{equ:Xsr1} via a similar, but much simpler ranking argument as used in Sec.~\ref{sec:strat-templates}. Here, $\rank{v} = 1$ for $v \in R$ and for the remaining vertices, 
$\rank{v}$ is the minimum integer $j$ for which $v \in X^j:=(S \cap (R \cup \Cpre_\Odd(X^{j-1})))$ where $X^0=\emptyset$. The positional strategy of \bb is then to take the minimum ranked successor from each \Odd node. 

Another way to think about this strategy is in the form of an acyclic subgraph of $\mathcal{G}^\ell$ on $\Xsr_\Odd$, where nodes in $R$ have no outgoing edges,
and for the remaining nodes, \Odd nodes have one outgoing edge and \Even nodes have all their outgoing edges. This is because if $v \in X^j\cap \Ve$, all outgoing edges achieve positive progress towards $R$, i.e. for all $(v, w) \in E$, $w \in X^{j-1}$.
Now it is easy to see that this subgraph almost defines a strategy template, i.e., on $\Xsr_\Odd\setminus R$, \Even nodes have all their outgoing edges in the subgraph, no \Odd node lies on a cycle and all of them have one outgoing edge. However, vertices in $R$ are dead-ends. We therefore call the strategy template induced by  \eqref{equ:Xsr1} \emph{partial} and denote it by $sr$. %\AKS{I think we need to properly formalize this template to use it in the next section}


\smallskip
\noindent\textbf{The \Even Player.}
It follows from the results of Banerjee et. al.~\cite{banerjee2022fast} that the safe reachability set $\Xsr_\Even$ of player \Even in \Odd-fair parity games requires the 2-nested fixed-point formula $\nu Y.\mu X.S \cap (R \cup \Apre(Y,X))$, which (via the operators defined in Sec.~\ref{sec:assump:prelim}) equals
%  
\begin{equation}\label{equ:Xsr2}
 \Xsr_\Even: =~\nu Y~.~\mu X~.~S \cap (R \cup (\Cpre_\Even(X) \cup (\Lpre^{\exists}(X) \cap \Pre_\Odd^{\forall}(Y))))
\end{equation}
Intuitively, the necessity of a 2-nested formula arises from the following lack of information: we do not know in advance, which \Odd nodes need to lie on a cycle on a strategy template required for \Odd to win. If any positional strategy that lets \Odd win (i.e., to avoid $R$ or leave $S$) from a $v\in V^\ell$
requires $v$ to lie on a cycle, then \Odd has to take $v$'s live outgoing edges as well, and thus, it can enter $\Xsr_\Even$ and lose.
The calculation of \eqref{equ:Xsr2} starts with $Y^0 := V$, resulting in $\Pre_\Odd^{\forall}(V)=V$, hence 
% \vspace{-0.5cm}
\begin{equation}\label{equ:Xsr2a}
 Y^{1}:=\mu X~.~S\cap (R \cup \Cpre_\Even(X) \cup \Lpre^\exists(X)).
\end{equation}
% \vspace{-0.1cm}
Due to the disappearence of $\Pre^\forall_\Odd(Y)$ in this iteration, intuitively all $v \in V^\ell$ are treated as if they do not have any positional winning \Odd strategy on them, so as if all \Odd strategies have to take all the live edges in the game. 
% 
$Y^1$ includes any \Odd vertex that progresses towards $R$ while staying in $S$ with using either all its edges (due to $\Cpre_\Even(X)$) or through one live edge (due to $\Lpre^\exists(X)$). Thus, any vertex that manages to stay in $V \setminus Y^1$ does so due to being won by \Odd even if \Even could force all the live outgoing edges to be taken. 
Note that due to the monotonicity of fixed-point operators, for all $j$, $V \setminus Y^1 \subseteq V \setminus Y^j$.

Throughout the calculation, $ V \setminus Y^j$ keeps track of the nodes that have managed to escape $S$ or avoid $R$ in the previous iteration, so are `already' won by \Odd in the first $j$ iterations. The inner fixed-point calculation in the $(j+1)^{th}$ iteration treats $V \setminus Y^j$ as a subset of \Odd's winning region and it deems any node that can be forced by \Odd to reach $V \setminus Y^j$, lost by \Even.
When the algorithm saturates, $Y^\infty$ contains only those \Odd nodes that cannot be forced by \Odd to reach $V \setminus Y^\infty$, i.e., are won by \Even. Here it is important to observe that, $V \setminus Y^\infty$ contains some \Odd nodes that are not $V \setminus Y^1$. Since they are in $Y^1$, these nodes inductively %\todo{IS:iteratively? any other word?} 
reach \Even winning vertices through live edges.  
This reveals that, all nodes in $V \setminus Y^j$ but not in $V \setminus Y^1$ win due to a positional \Odd strategy that reaches $V \setminus Y^{j-1}$. 
Iteratively, this reveals that all such nodes have positional \Odd strategies that make them reach $V \setminus Y^1$.

The above alternative interpretation of the computation of $\Xsr_\Even$ in \eqref{equ:Xsr2} is the key insight that we utilize to define our new \Odd-fair Zielonka's algorithm, as discussed next.

\smallskip
\noindent\textbf{The \Odd-fair Zielonka's Algorithm.}
%
Following up on the previous discussion, we use the following insight within the construction of the \Odd-fair Zielonka's algorithm. We assume the existence of a core subset $\Wo' \subseteq \Wo$  
that player \Odd can force all nodes in $\Wo$ %(i.e., the winning region of \Odd in the \Odd-fair parity game $\mathcal{G}^\ell$) 
to, that is winning for \Odd even under the assumption that \Even can force all the live edges in the game to be taken. %\footnote{We note that this is a similar insight used in the proof of winning strategy templates discussed in Sec.~\ref{sec:strat-templates}.}.  
% 
Since Zielonka's algorithm solves parity games by a sequence of nested safe-reachability calculations for alternating players, we apply the following trick:
Instead of computing $\Xsr_\Even$ via \eqref{equ:Xsr2} in each recursive call of Alg.~\ref{algo:fair-zielonka-bb}, we only compute $Y^1$ via \eqref{equ:Xsr2a} and use it as an \emph{overapproximation} of $\Xsr_\Even$ (which it truly is due to the monotonicity of \eqref{equ:Xsr2} in $Y$). 
That is, while we take the \Odd safe reachability set $\SafeReach^f_\Odd$ as the original (linear) \Odd safe reachability computation known for these games (given in~\eqref{equ:Xsr1}), we do not take \Even safe reachability formula $\SafeReach^f_\Even$ to be the (quadratic) \Even safe reachability computation known for these games (given in~\eqref{equ:Xsr2}),
but we instead take it as its (linear) subformula given in~\eqref{equ:Xsr2a} and arrive at an overapproximation of the \Even safe reachability region at the end of each $\SafeReach^f_\Even$ calculation. We finalize the recursive call $\SOLVE_\Odd$ by an extra call of $\SafeReach^f_\Odd$ applied to the (thus) underapproximated \Odd winning region in the sub-game, therefore expanding the returned \Odd winning region of the sub-game.  

By this, it turns out that the recursive call of $\SOLVE_\Odd(n, \mathcal{G}^\ell)$ actually computes $\Wo'$ as the set $X$ and we ensure that $\Wo$ is returned by the additional (linear) computation of $\SafeReach^f_\Odd$ over $X$ in the last return statement of Alg.~\ref{algo:fair-zielonka-bb}.
% 
This instantiation of the safe-reachability computations is formalized next.

\begin{definition}\label{def:safereach}
 Given an \Odd-fair parity game $\mathcal{G}^\ell=\ltup{(V, V_\Even, V_\Odd, E, \chi), E^\ell}$ the safe-reachability procedures $\SafeReach^f_\Odd(S, R, \mathcal{G}^\ell)$ and $\SafeReach^f_\Even(S, R, \mathcal{G}^\ell)$ in Alg.~\ref{algo:fair-zielonka-bb} denote the iterative fixed-point computations in \eqref{equ:Xsr1} for \Odd and \eqref{equ:Xsr2a} for \Even.
\end{definition} 

\subsection{Complexity of the \Odd-fair Zielonka's Algorithm}\label{sec:zielonka:complexity}
The safe-reachability computations defined in Def.~\ref{def:safereach} have the same complexity as their computations via \eqref{equ:Xsr1} in the original Zielonka's algorithm. The only difference is in the number of calculated $\Pre$ operations: while $\SafeReach_\Even$ from original Zielonka's algorithm~\eqref{equ:Xsr1} require the calculation of only one $\Pre$ operator, $\SafeReach_\Even^f$ from~\eqref{equ:Xsr2a} requires the calculation of 2 $\Pre$ operators. The additional final call of $\SafeReach^f_\Odd$ in $\SOLVE_\Odd$ procedure also has linear complexity and requires one $\Pre$ calculation. 
Therefore, not only the worst-case time complexity of Alg.~\ref{algo:fair-zielonka-bb} is equivalent to that of original Zielonka's algorithm (which would be the case even if we used the quadratic safe reachability formula from~\eqref{equ:Xsr2} for \Even since the overall complexity of the algorithm is exponential) but we create almost no additional computational overhead in the algorithm by introducing the fairness assumptions.

We further remark that Alg.~\ref{algo:fair-zielonka-bb} is not a straight-forward interpretation of the nested fixed-point in~\eqref{eq:fp-odd}, and its negation (see~\eqref{eq:fp-even} in App.~\ref{app:fp-proof}) in the form of Zielonka's algorithm. 

Firstly, such a straightforward approach is non-trivial due to $\Apre$ and $\Npre$ operators taking two variables from two different iterations of the fixed-point calculation. 
Furthermore, 
at each \Even safe-reachability call of Alg.~\ref{algo:fair-zielonka-bb}, as mentioned we compute 2 $\Pre$ operators (equation~\ref{equ:Xsr2a}), whereas in each such corresponding step in the fixed-point iteration, we would have to compute 3 $\Pre$ operators due to the expansion of $\Apre$~\eqref{equ:apre} and $\Npre$~\eqref{equ:npre}.

It remains to show that \Odd-fair Zielonka's algorithm solves \Odd-fair parity games. 

\subsection{Correctness of the \Odd-fair Zielonka's Algorithm}\label{sec:zielonka:correct}
% 
% 
% Now we will try to convey the idea of the correctness proof of Alg.~\ref{alg:fair-zielonka}. 
We first recall that \Odd-fair parity games are determined. %the results of Banerjee et. al.~\cite{banerjee2022fast}. Given an \Odd-fair parity game $\mathcal{G}^\ell=\ltup{(V, V_\Even, V_\Odd, E, \chi), E^\ell}$, we therefore know that \We and \Wo partition $V$. Following the original Zieloka's algorithms proof, it therefore remains to show that $\SOLVE_{\Even}(n,\mathcal{G}^\ell)$ and $\SOLVE_{\Odd}(n,\mathcal{G}^\ell)$ as in Alg.~\ref{algo:fair-zielonka-odd} and Alg.~\ref{algo:fair-zielonka-even} actually compute \We and \Wo, respectively.
Next, we prove the correctness of the algorithm by induction on $n$. Since in the base case $n = 0$ the calls correctly return $\emptyset$, it suffices to prove the correctness of each function, assuming the correctness of the other. This is formalized next. %for $\SOLVE_{EVEN}$ (Alg.~\ref{algo:fair-zielonka-even}), where $\SOLVE_{ODD}$ (Alg.~\ref{algo:fair-zielonka-odd}) follows from a symmetrical argument with odd $n$.
\begin{comment}
\begin{theorem}[Correctness of $\SOLVE_{\bb}$, Alg.~\ref{algo:fair-zielonka-bb}]\label{thm:solvebb}
Let $\mathcal{G}^\ell=\ltup{(V, V_\Even, V_\Odd, E, \chi), E^\ell}$ be an \Odd-fair parity game with \textsf{parity(\bb)}\footnote{\textsf{parity(\Odd)} is odd and  \textsf{parity(\Even)} is even.} upper bound priority $n$. Further, assume that for any \Odd-fair parity game $\mathcal{G'}^\ell$ with  \textsf{parity($\bb$)} upper bound priority $n'<n$ holds that 
 $\mathcal{W}_\bb[\mathcal{G'}^\ell]=\SOLVE_{\bb}(n',\mathcal{G'}^\ell)$ and for any  \Odd-fair parity game $\mathcal{G''}^\ell$ with \textsf{parity($\neg \bb$)} upper bound priority $n''<n$ holds that 
 $\mathcal{W}_{\neg \bb}[\mathcal{G''}^\ell]=\SOLVE_{\neg \bb}(n'',\mathcal{G''}^\ell)$. Then, $\mathcal{W}_\bb[\mathcal{G}^\ell]=\SOLVE_{\bb}(n,\mathcal{G}^\ell)$.
\end{theorem}
\end{comment}

\begin{theorem}[Correctness of $\SOLVE_{\bb}$, Alg.~\ref{algo:fair-zielonka-bb}]\label{thm:solvebb}
Assume that for any \Odd-fair parity game $\mathcal{G}'^\ell$ where $n' < n$ is an odd (resp. even) upper bound on the priorities of the game, $SOLVE_\Odd(n', \mathcal{G}')^\ell$ correctly returns the \Odd winning region (resp. $\SOLVE_\Even(n', \mathcal{G}')^\ell$ correctly returns the \Even winning region) in $\mathcal{G}'^\ell$. Then $SOLVE_\bb(n, \mathcal{G}^\ell)$ correctly returns the winning region of player $\bb$ where $n$ is even if $\bb= \Even$ and odd if $\bb = \Odd$.
\end{theorem}

%While the proof of Alg.~\ref{algo:fair-zielonka-odd} follows essentially the proof by Ralf K{\"u}sters \cite{Kuesters2002} of the original Zielonka's algorithm \cite{zielonkas-alg}, the proof of Alg.~\ref{algo:fair-zielonka-even} formalized by Thm.~\ref{thm:solveodd} becomes substantially more complex. First, our instantiation of $\SafeReach^f_\Even(S, R, \mathcal{G}^\ell)$ via \eqref{equ:Xsr2} only computes an \emph{overapproximation} of the safe reachability set $\Xsr_\Even$, and second, we must use \Odd \emph{winning strategy templates} instead of positional winning strategies, to prove a vertex to be winning. While the complete correctness proofs of both algorithms can be found in App.~\ref{app:zielonka-proof}, we give the intuition of Thm.~\ref{thm:solveodd} here, as this is the main contribution of this section. % In order to do so, we first define some preliminaries in Sec.~\ref{sec:zielonka:correctness:prelim}.
% We follow the notation of Ralf K{\"u}sters proof \cite{Kuesters2002} of the original Zielonka's algorithm \cite{Zielonka98}.% Let us first set up some preliminaries.

% \vspace{0.2cm}

\noindent\textbf{Notation.}
We follow the notation of K{\"u}sters' proof \cite{Kuesters2002} of Zielonka's original algorithm \cite{Zielonka98}.
%For the remainer of this section, take $\mathcal{G}^\ell = \ltup{(V, V_\Even, V_\Odd, E, \chi), E^\ell}$. 
Recall that $\mathcal{G}^\ell$ has no dead-ends. For some $X \subseteq V$, we call $\mathcal{G}^\ell[X] = \ltup{(X, X \cap V_\Even, X \cap V_\Odd, X \times X \subseteq E, \chi \mid_X), X \times X \subseteq E^\ell }$ 
a \emph{subgame} of $\mathcal{G}^\ell$ if it has no dead-ends. Here, $\chi\mid_X$ is the priority function $\chi : V \to \mathbb{N}$ restricted to domain $X$. Let $n$ be an upper bound on the priorities in $V$. If the parity of $n$ is even, set $\bb$ to $\Even$; if it's odd, set $\bb$ to \Odd. 

\vspace{0.2cm}

\noindent\textbf{\bb-trap and \bb-paradise.}
A $\bb$-trap is a subset $T \subseteq V$ for $\bb \in \{\Even, \Odd\}$ such that,
$\forall v \in T \cap V_{\nb},\,\, \exists (v, w)\in E \,\,\text{ with } w \in T$ and $\forall v \in T \cap V_{\bb},\,\, (v, w) \in E \implies w \in T$. 
A $\bb$-paradise in $\mathcal{G}^\ell$ is a subset $T \subseteq V$ which is a $\nb$-trap in $V$ and there exists a winning $\bb$ strategy template $(T, \e)$ in $\mathcal{G}^\ell$. %\AKS{don't we need that the vertices contained in this template are precisely $T$.}

\vspace{0.2cm}

The recursive calls of $\SOLVE_\bb$ and $\SOLVE_{\neg \bb}$ on subgames within Alg.~\ref{algo:fair-zielonka-bb} induce a characteristic partition of the game graph. For the correctness proof, 
we need to remember a series of these subgames that are constructed through previous recursive calls. The partition of these subsets is illustrated in Fig.~\ref{fig:kuesters-figure-extended} and formalized as follows.
% 
% % Figure environment removed

\vspace{-0.4cm}

\begin{align}\label{equ:seriesZielonka}
    &X_\bb^i := V \setminus X_\nb^i \quad \quad \quad &&N^i:= \{v \in X^i_\bb \mid \chi(v) = n\}\\
    &Z^i:= X^i_\bb \setminus \SafeReach^f_\bb(X^i_\bb, N^i, \mathcal{G}^\ell) \quad &&X^{i+1}_\nb :=  \SafeReach^f_\nb(V, X_\nb^{i} \cup Z_\nb^{i}, \mathcal{G}^\ell)\nonumber % X_\Even^{i} \cup \SafeReach_\Even^f(X^{i}_\Odd, Z_\Even^{i}, \mathcal{G}^\ell) )%\text{\todo{IS: I know the equality is not completely justified. The first one is cheaper for an algorithm pov, whereas the second one is easier to justify that $X^i_\Odd$ is an \Even-trap.}} 
\end{align}

%\vspace{-0.3cm}

where, in addition $Z_\bb^i$ is the \bb winning region in the subgame $\mathcal{G}^\ell[Z^i]$. Intuitively, the sets constructed in \eqref{equ:seriesZielonka} correspond to the sets with the same name within Alg.~\ref{algo:fair-zielonka-bb}.
    
We collect the following observations on these sets, which are proven in App.~\ref{app:zielonka-proof}. %and mimic the corresponding properties in the proof of the original Zielonka's proof \cite{Kuesters2002}.
\begin{enumerate}\label{it:zlk-observations}
\item  \textbf{(App. - Obs.~\ref{app-obs:traps-subgames})} $X^i_\nb$ is an \bb-trap, $X^i_\bb$, $Z^i$ and $Z_\bb^i$ are \nb-traps in $V$. $Z^i$ is in \nb-trap in $X_\bb$ and $Z_\nb^i, Z_\bb^i$ are \bb- and \nb-traps in $Z^i$, respectively. Therefore, $\mathcal{G}^\ell[Y]$ is a subgame of $\mathcal{G}^\ell$ with $Y$ being any of these sets.\label{it:obs1} %(see Obs.~\ref{obs:traps-subgames} in App.~\ref{}).
 \item \textbf{(App. - Lem.~\ref{app-lem:X_nb-equivalence})} $X_\nb^{i} \cup \SafeReach_\nb^f(X^{i}_\bb, Z_\nb^{i}, \mathcal{G}^\ell) =  \SafeReach_\nb^f(V, X_\nb^{i} \cup Z_\nb^{i}, \mathcal{G}^\ell)$.\label{it:obs2}%(see Lem.~\ref{lem:X_nb-equivalence} in App.~\ref{}).
 \item \textbf{(App.  - Cor.~\ref{app-cor:increasing-decreasing-sequences})} As a consequence of the previous item, $\{X_\nb^{i}\}_{i\in \mathbb{N}}$ is an increasing sequence. Consequently, $\{X_\bb^{i}\}_{i\in \mathbb{N}}$ is a decreasing sequence. As $V$ is finite, this immediately implies that these sequences reach a saturation value for some, and in fact the same, $k$. \label{it:obs3}
 \item \textbf{(App.  - Lem.~\ref{app-lem:safe-reach-Odd-paradise})} If $R \subseteq V$ is an \Odd-paradise in $\mathcal{G}^\ell$, then $\SafeReach^f_\Odd(V, R, \mathcal{G}^\ell)$ is also an \Odd-paradise in $\mathcal{G}^\ell$.\label{it:obs4}
 \item \textbf{(App.  - Lem.~\ref{app-lem:safereacheven-noliveedges})} The set $U \setminus \SafeReach_\bb(U, R, \mathcal{G}^\ell)$ is a $\bb$-trap in $U$. \label{it:obs5}
\end{enumerate}

\vspace{0.1cm}

In contrast to Zielonka's original algorithm, the proof of the procedures $\SOLVE_\Even$ and $\SOLVE_\Odd$ is not identical in \Odd-fair Zielonka's algorithm. This is due to the different safe-reachability set constructions used. Next we sketch the correctness proof of Thm.~\ref{thm:solvebb} for $\bb:=\Odd$, corresponding to the correctness of procedure $\SOLVE_\Odd$. The proof for $\bb:=\Even$ is left to the appendix, as it resembles the proof Zielonka's original algorithm more.
% % \subsubsection{Correctness of $\SOLVE_\Odd$ --  Thm.~\ref{thm:solvebb}}\label{sec:zielonka:correctness:odd}

\begin{proposition}\label{prop:n-odd}
Given the premisses of Thm.~\ref{thm:solvebb} for $\bb = \Odd$, if $Z_\Even^k = \emptyset$ then $\SafeReach^f_\Odd(V, X^k_\Odd, \mathcal{G}^\ell)$ is an \Odd-paradise and $V \setminus \SafeReach^f_\Odd(V, X^k_\Odd, \mathcal{G}^\ell)$ is an \Even-paradise in $\mathcal{G}^\ell$.
\end{proposition}

Within Prop.~\ref{prop:n-odd}, the fact that $Z_\Even^k = \emptyset$ refers to the termination of the recursive call in Alg.~\ref{algo:fair-zielonka-bb} which results in the saturation of the sequence $\{X_\Odd^i\}_{i\in \mathbb{N}}$ with $X_\Odd^k$. This implies that $\SOLVE_\Odd$ returns 
$ T:=\SafeReach^f_\Odd(V, X^k_\Odd, \mathcal{G}^\ell) $, which is an \Odd-paradise and $V \setminus T$ an \Even-paradise. With this, Thm.~\ref{thm:solvebb} follows from Prop.~\ref{prop:n-odd} for $\bb=\Odd$. % Alg.~\ref{algo:fair-zielonka-bb} for $\bb = \Odd$.
We now give a proof sketch of Prop.~\ref{prop:n-odd}.

We first recall from observation~\ref{it:obs1} that $T$ and $V\setminus T$ are \Even- and \Odd-traps in $V$, respectively. In order to prove Prop.~\ref{prop:n-odd}, it remains to show that there exists an \Odd (resp. \Even) strategy template which is winning in $\mathcal{G}^\ell$ and maximal on $T$ (resp. $V\setminus T$). We next give the construction of these templates and a high-level intuition on why they are actually \emph{winning}. 

\smallskip
\noindent\textbf{Winning \Odd Strategy Templates.} 
As $X^k_\Odd$ is known to be an \Even-trap, it can be proven to be an \Odd-paradise by constructing a winning maximal strategy template on it. It then follows from observation~\ref{it:obs4} that $T$ is also an \Odd-paradise.

Towards a construction of a maximal winning \Odd strategy template on $X_\Odd$, we first observe that $X^k_\Odd=Z_\Odd^k\cup \SafeReach^f_\Odd(X^k_\Odd, N^k, \mathcal{G}^\ell)$ (as $Z_\Even^k=\emptyset$). % Now we first consider $Z^k = Z_\Odd^k$ (as $Z_\Even^k=\emptyset$)\todo{IS: why do we 'consider' this, isn't this given for $k$?}
 Then there exists a maximal winning \Odd strategy template $z$ on $Z^k = Z_\Odd^k$ in game $\mathcal{G}^\ell[Z^k]$. % and the definition of $Z_\Odd^k$. 
 Any play $\pi$ compliant with $z$ that starts and stays in $Z^k$ is clearly \Odd winning.
However, $z$ is not necessarily an \Odd strategy template in $\mathcal{G}^\ell$ since there are possibly some $(v,w) \in E$ with $v \in Z^k \cap V_\Even$  and $w \not \in Z^k$.
For all such edges, $w \in \SafeReach^f_\Odd(X^k_\Odd, N^k, \mathcal{G}^\ell)$ since $X^k_\Odd$ is an \Even-trap in $V$.
% 
For the state set $\Xsr_\Odd:=\SafeReach^f_\Odd(X^k_\Odd, N^k, \mathcal{G}^\ell)$, recall from Sec.~\ref{sec:zielonka:fair} that there exists partial strategy template $sr$ defined on $\Xsr_\Odd$ with dead ends in $N^k$.

Using the templates $z$ and $sr$, we can construct a maximal candidate \Odd strategy template on $X^k_\Odd$. Following the intuition behind the construction of $\mathcal{S}^{\mathcal{G}^\ell}$ in Def.~\ref{def:S}, we first define a base subgraph $(X^k_\Odd,\e)$ with $\e\subseteq E$ s.t.\
 $(v,w) \in E $ is in $\e$ if either (i) $(v,w) \in z \cup sr$, (ii) $v \in V_\Even \cap X^k_\Odd$, or (iii) $v \in N^k \cap V_\Odd$ and $w = v_r$
where $v_r$ is a random fixed successor of $v$, that is in $X^k_\Odd$.
Such a successor is guaranteed to exist since $X^k_\Odd$ is an \Even-trap.
We now extend the subgraph $(X^k_\Odd,\e)$ to an \Odd strategy template by adding all live edges originating in vertices  $X^k_\Odd\cap V^\ell$ that lie on a cycle in $\e$, similar to Def.~\ref{const:S} (S3)-(S4). This results in a subgraph $\Sc=(X^k_\Odd,\overline{\e})$ %where $\overline{e}$ is defined to be the saturation value of the sequence $\overline{e}^j = \overline{e}^{j-1} \cup \{(v, w) \in V^\ell \mid v \text{ lies on a cycle in } (X_\Odd^k, \overline{e}^{j-1})\}$ where $\overline{e}^0 = e$.
that is a maximal \Odd strategy template. %\AKS{Why don't we need $\Sc$ to be maximal on $T$?}
% 
The underlying idea behind $\mathcal{S}$ being winning %(formally proven in App.~\ref{app:zielonka-proof}) 
is the following: Any play that starts in $X_\Odd^k$ either stays in $Z^k$ after some point and is won by $\mathcal{S}$ collapsing to $z$, or sees a newly added cycle (one that is not in $z \cup sr$) infinitely often. All such cycles contain a newly added edge. An analysis of newly added edges reveal that, 
all of them -- when seen infinitely often -- eventually drag a play towards $N^i$. Thus, every play that sees a new cycle infinitely often sees $n$ infinitely often, and thus won by \Odd.

\smallskip

\noindent\textbf{Winning \Even Strategy Templates.} 
Here we show that $V \setminus T$ is an \Even-paradise in $\mathcal{G}^\ell$. 
We first define $\Xsr^i_\Even:=\SafeReach^f_\Even(X_\Odd^i, Z_\Even^i, \mathcal{G}^\ell)$ and denote by $sr^i$ the partial \Even strategy template defined on $\Xsr^i_\Even$. We further denote the winning \Even strategy on $Z_\Even^i$ in game $\mathcal{G}^\ell[Z^i]$ by $z^i$. 
% 
We can now construct the \Even strategy template $\mathcal{S} = (V \setminus T, \e)$ where $\e$ is the combination of edges in $sr^i \cup z^i$ with $\{(v,w) \in E \mid v \in V_\Odd \cap (V \setminus T)\}$.
Since $V\setminus T$ is an \Odd-trap by observation~\ref{it:obs5}, the edge set $\e$ stays within $V \setminus T$, i.e. $\e \subseteq V\setminus T \times V \setminus T$. Then clearly, $\mathcal{S}$ is an \Even strategy template.
To see $\mathcal{S}$ is winning we first observe that each $v \in V \setminus T$ there exists a unique $i<k$ such that $v \in \Xsr^i_\Even$. Let $\pi = v_1 v_2 \ldots$ be a play compliant with $\mathcal{S}$ and let $s = \Xsr_1 \Xsr_2 \ldots$ be the sequence such that $v_i \in \Xsr$.
(1) If $v_j \in Z_\Even^i$, $v_{j+1} \in Z_\Even^i \cup \{\Xsr_\Even^r \mid r < i\}$. This follows from $Z_\Even^i$ being an \Odd-trap in $X_\Odd^i$.
(2) If $\pi$ visits $v \in \Xsr^i$ infinitely often, $\pi$ visits $Z_\Even^i$ infinitely often: This is because $\pi$ visits the $(v,w)$ in $\mathcal{S}$ that makes positive progress towards $Z_\Even^i$ infinitely often as well. 
% 
Let $i$ be the minimum index such that $\Xsr_\Even^i$ is seen infinitely often in $s$. By (1), $\pi$ visits $Z_\Even^i$ infinitely often and by (1) and the minimality of $i$, it should eventually stay in $Z_\Even^i$.
Thus $\mathcal{S}$ eventually collapses to $z_\Even^i$ on $\pi$ and the play is won by \Even.

\vspace{-0.2cm}
%\smallskip
%\noindent \textbf{The Algorithm.} Observe that $\SOLVE_\Odd(n, \mathcal{G}^\ell)$ (Alg.~\ref{algo:fair-zielonka-odd}) calculates the sets as given in the construction(Fig.~\ref{fig:kuesters-figure-extended}) where $X$ holds the value of $X_\Odd^i$ at the end of the $i^{th}$ iteration of it's \emph{while} loop. $\{X_\Odd^i\}_{i \in \mathbb{N}}$ is a decreasing sequence which saturates at some $X_\Odd^k$ where $Z_\Even^k = \emptyset$.
%$\SOLVE_\Odd(n, \mathcal{G}^\ell)$ returns  $\SafeReach(V, X_\Odd^k, \mathcal{G}^\ell)$, which is shown to be equal to \Wo by Prop.~\ref{prop:n-odd}. 

%Similarly, $X$ in $\SOLVE_\Even(n, \mathcal{G}^\ell)$ (Alg.~\ref{algo:fair-zielonka-even}) holds the value of $X_\Even^i$ after the $i^{th}$ iteration. The constructive proof of Thm.~\ref{thm:solveeven} (App.~\ref{app:zielonka-proof}) shows that the saturation value $X^{k'}_\Even$ (where $Z^{k'}_\Odd = \emptyset$) is equal to \We, and this is exacly the value $\SOLVE_\Even(n, \mathcal{G}^\ell)$ returns.
%


\section{Experiments}
% \haizhou{Follow the same way of introduction as we did in Section2.}
% \noindent In this section, we will introduce datasets and experimental setups that we used. Then we evaluate our method, other self-supervised methods, and supervised methods under different distribution shifts (\ie, concept shifts and covariate shifts) under common settings (\ie, transductive, inductive settings). It has to note that we focus on node-level tasks (\eg, node classification) in this work. As for graph-level tasks, we leave it as our future work and some simple experiments can be found in Appendix~\ref{app:graph_classification}. 
In this section, we first introduce the experimental setup including datasets, training, and evaluation protocol in Section~\ref{sec:dataset}~and~\ref{sec:unsupervised}. 
% Next, we present our experimental setup and conduct extensive experiments to evaluate our method in Section~\ref{sec:unsupervised}. 
We then perform an ablation study to demonstrate the effectiveness of each proposed component in Section~\ref{sec:ablation}. 
Additionally, we analyze the impact of important hyper-parameters in Section~\ref{sec:sensitivity}. 
Subsequently, we integrate our method with various encoding models, showcasing the model-agnostic nature of our recipe in Section~\ref{sec:other_models}. 
Finally, we provide some qualitative results such as feature visualization in Section~\ref{sec:vis}.
It is important to note that we focus on node-level tasks (\eg, node classification) in this work. As for graph-level tasks, we leave it as our future work, while some simple experiments are also provided in Appendix~\ref{app:graph_classification}.

\subsection{Datasets}\label{sec:dataset}
There exist some benchmarks for evaluating graph out-of-distribution generalization~\cite{good,ji2022drugood,gds}. 
Among them, GOOD~\cite{good} is the most representative and comprehensive benchmark that curates more diverse graph datasets with diverse tasks, including single/multi-task graph classification, graph regression, and node classification involving more distribution shifts (\ie, concept shifts and covariate shifts). Hence in this work, we follow the evaluation protocol proposed in \cite{good}. Furthermore, we validate the effectiveness of our method in the datasets (\ie, Amazon-Photo, Elliptic) that are used in EERM~\cite{eerm}. The statistics and detailed introduction to these datasets can be found in Table~\ref{tab:dataset} and Appendix~\ref{app:datasets}.

\begin{table*}[htp]
\caption{The descriptions of datasets. ``Domain-Level'' means splitting by graphs, ``Time-Aware'' denotes splitting according to chronological order.``Word'' and ``Degree'' represent splitting according to word diversity and node degree respectively. ``Language'' means splitting by user language, suggesting the prediction should not be impacted by the language the user use. ``University'' denotes splitting according to the domain university, implying that the prediction of webpages should be based on word contents and link connections rather than university features. ``Color'' means that nodes are split according to node differences in covariate shift and color-label correlations in concept shift.}
\label{tab:dataset}
\centering
\begin{tabular}{cccccccc}
\toprule
Datasets     & Network Type        & \#Nodes & \#Edges & \#Attributes &\#Classes& Train/Val/Test Split     & Metric   \\
% Cora         & Artificial Transformation & 2,703   &         &              &         &                      & Accuracy \\
Amazon-Photo\footnotemark
             & Co-purchasing network      & 7,650   & 119,081   & 755          & 10      & Domain-Level         & Accuracy \\
Elliptic\footnotemark  
             & Bitcoin transactions       & 203,769 & 234,355   & 165          & 2       & Time-Aware           & F1-Score \\
GOOD-Cora    & Scientific publications    & 19,793  & 126,842   & 8,710         & 70      & Word/Degree          & Accuracy \\
% GOOD-Arxiv   & arXiv papers               & 169,343 & 2,315,598 & 128          & 40      & Time/Degree          & Accuracy \\
GOOD-Twitch  & Gamer network              & 34,120  & 892,346   & 128          & 2       & Language             & ROC-AUC  \\
GOOD-CBAS    & A BA-house graph           & 700     & 3,962     & 4             & 4       & Color                & Accuracy \\
GOOD-WebKB   & Webpage network            & 617     & 1,138     & 1,703         & 5       & University           & Accuracy \\
\bottomrule
\end{tabular}
\end{table*}
\footnotetext[5]{This dataset is adopted from~\cite{yang2016revisiting}. \cite{eerm} constructs ten graphs with different environment id’s for each graph.} 
\footnotetext[6]{The original is available on \hyperlink{https://www.kaggle.com/ellipticco/elliptic-data-set}{https://www.kaggle.com/ellipticco/elliptic-data-set}}

\subsection{Unsupervised Representation Learning}\label{sec:unsupervised}
\subsubsection{Transductive Setting}~\label{sec:trans}
% \noindent\textbf{Baselines.}\quad We conduct experiments with 12 baselines which consist of three categories: supervised methods and self-supervised generative methods, self-supervised contrastive methods. Specifically, we compare with three supervised baselines: empirical risk minimization~(ERM)~\cite{erm}, invariant risk minimization (IRM)~\cite{irm}, and a recent proposed graph OOD method dubbed EERM~\cite{eerm}. We also compare various unsupervised node-level representation learning methods: three self-supervised generative methods including GAE~\cite{gae}, VGAE~\cite{gae}, GraphMAE~\cite{gmae} and seven self-supervised contrastive methods: DGI~\cite{dgi}, MVGRL~\cite{mvgrl}, GRACE~\cite{grace}, RoSA~\cite{rosa}, BGRL~\cite{bgrl}, COSTA~\cite{costa}, SwAV~\cite{swav}. The descriptions of these methods can be found in Appendix~\ref{app:baselines}.
In this subsection, we focus on validating our proposed algorithm under the transductive setting, where the test nodes will participate in message passing~\cite{gilmer2017neural} during training following~\cite{good}. 

\noindent\textbf{Baselines.} We conduct experiments with 12 baselines from three categories: (i)~supervised methods, including empirical risk minimization~(\textbf{ERM})~\cite{erm}, invariant risk minimization (\textbf{IRM})~\cite{irm}, and a recent proposed graph OOD method \textbf{EERM}~\cite{eerm}; (ii)~self-supervised generative methods including Graph Autoencoder (\textbf{GAE})~\cite{gae}, Variational Graph Autoencoder (\textbf{VGAE})~\cite{gae}, Self-Supervised Masked Graph Autoencoders (\textbf{GraphMAE})~\cite{gmae}; (iii)~self-supervised contrastive methods including Deep Graph Infomax (\textbf{DGI})~\cite{dgi}, Contrastive Multi-View Representation Learning on Graphs (\textbf{MVGRL})~\cite{mvgrl}, Deep Graph Contrastive Representation Learning (\textbf{GRACE})~\cite{grace}, A Robust Self-Aligned Framework for Node-Node Graph Contrastive Learning (\textbf{RoSA})~\cite{rosa}, Bootstrapped Representation Learning on Graphs (\textbf{BGRL})~\cite{bgrl}, Covariance-Preserving Feature Augmentation for Graph Contrastive Learning (\textbf{COSTA})~\cite{costa}, Unsupervised Learning of Visual Features by Contrasting Cluster Assignments (\textbf{SwAV})~\cite{swav}. The detailed descriptions of these baselines can be found in Appendix~\ref{app:baselines}.

\noindent\textbf{Experimental setup.} We use the same graph encoder across different datasets for a fair comparison following~\cite{good}. We use grid search to find other hyper-parameters (\eg, learning rate, epochs) for different methods. For all experiments, we select the best checkpoints for ID and OOD tests according to results on ID and OOD validation sets following~\cite{good}, respectively. Experimental details and hyper-parameter selections are provided in Appendix~\ref{app:hyper}. For evaluating unsupervised methods, a linear classifier will be built on the frozen trained encoder after finishing pre-training. The reported results are the mean performance with standard deviation after 10 runs following~\cite{good}.

\noindent\textbf{Analysis.}\quad Based on the experimental results listed in Table~\ref{tab:trans_concept} and \ref{tab:trans_covariate}, we can draw the following conclusions: firstly, we find strong self-supervised methods (\eg, GRACE, BGRL, COSTA) are more robust to distribution shifts (concept shift in Table~\ref{tab:trans_concept} and covariate shift in Table~\ref{tab:trans_covariate}) compared to supervised methods. For instance, on GOOD-CBAS and GOOD-WebKB datasets, GRACE surpasses the best supervised method by large margins (over 6\% absolute improvement). Interestingly, we find the methods designed for OOD generalization (\ie, IRM) and graph OOD generalization (\ie, EERM) do not attain superior performance than the standard ERM on most of the datasets. For example, EERM shows superior OOD performance compared to ERM in only one experiment, and IRM outperforms ERM in four out of ten experiments across the conducted evaluations. This phenomenon is also observed in \cite{good,ahuja2020empirical,rosenfeld2021risks}, showcasing the challenge of achieving invariant prediction in non-Euclidean graph settings. 

Furthermore, our method surpasses other SOTA self-supervised methods on the OOD test set of all datasets by a considerable margin while achieving comparable performance in the in-distribution test set. For instance, on small datasets such as GOOD-CBAS and GOOD-WebKB, our method outperforms GRACE\footnote{MARIO is built up on GRACE according to our recipe. So, we make a comparison with GRACE here.} by over 2\% absolute accuracy on the OOD test set. On larger datasets such as GOOD-Cora and GOOD-Twitch, our method still outperforms other methods which shows its superiority. For instance, under covariate shift, MARIO surpasses other methods by over 7\% absolute accuracy on the GOOD-Twitch OOD test set. These statistics prove the effectiveness of our design.


\begin{table*}[htp]
\caption{Experimental results of all methods under concept shift. The bold font means the top-1 performance and the underline represents the second performance across the unsupervised methods. 'ID' represents in-distribution test performance and 'OOD' means out-of-distribution test performance. (OOM: out-of-memory on a GPU with 24GB memory)}
\label{tab:trans_concept}
\centering
\scalebox{0.95}{
\begin{tabular}{l|cc|cc|cc|cc|cc}
\toprule
\toprule
\multirow{3}{*}{concept shift} & \multicolumn{4}{c|}{GOOD-Cora}                   & \multicolumn{2}{c|}{GOOD-CBAS} & \multicolumn{2}{c|}{GOOD-Twitch} & \multicolumn{2}{c}{GOOD-WebKB} \\
                           & \multicolumn{2}{c}{word} & \multicolumn{2}{c|}{degree}& \multicolumn{2}{c|}{color}    & \multicolumn{2}{c|}{language}   & \multicolumn{2}{c}{university} \\
                           & ID         & OOD         & ID          & OOD          & ID            & OOD           & ID             & OOD            & ID            & OOD            \\
\midrule
ERM                        & 66.38±0.45 & 64.44±0.18  & 68.60±0.40  & 60.76±0.34   & 89.79±1.39    & 83.43±1.19    & 80.80±1.00     & 56.92±0.92     & 62.67±1.53    & 26.33±1.09     \\
IRM                        & 66.42±0.41 & 64.29±0.31  & 68.57±0.35  & 61.45±0.24   & 89.64±1.21    & 82.29±1.14    & 78.87±1.04     & 59.30±1.79     & 62.67±1.10    & 26.88±1.42     \\
EERM                       & 65.10±0.44 & 62.45±0.19  & 66.95±0.44  & 56.58±0.25   & 79.07±2.12    & 64.50±1.01    & OOM            & OOM            & 62.50±2.01    & 28.07±3.23      \\
\midrule
% Random-Init                & 37.53±1.74 & 32.12±1.24  & 37.82±1.71  & 27.74±1.14   &               &               &                &                & 60.33±2.21    & 27.07±1.70     \\
GAE                        & 60.65±0.89 & 58.00±0.55  & 62.59±1.11  & 53.44±0.80   & 75.28±1.36    & 68.07±2.05    & 81.25±0.81     & 51.51±1.05     & 62.17±3.34    & 25.78±1.85     \\
VGAE                       & 63.19±0.53 & 60.35±0.47  & 61.65±0.66  & 54.28±0.28   & 76.50±0.50    & 59.07±0.56    & 80.46±0.53     & 55.56±4.53     & 62.50±2.38    & 24.40±2.57     \\
GraphMAE                   & \underline{66.44±0.46} & \underline{64.87±0.30}  & 67.95±0.46  & 59.41±0.39   & 89.14±0.89    & 82.93±0.93    & 80.05±0.64     & 59.38±1.49     & 61.83±3.37    & 29.27±2.15     \\
DGI                        & 63.33±0.56 & 60.71±0.49  & 65.93±1.02  & 55.83±0.53   & 91.22±1.47    & 85.00±1.66    & 80.05±0.87     & 59.16±1.88     & 61.83±2.83    & 28.63±1.92      \\
MVGRL                      & OOM        & OOM         & OOM         & OOM          & 88.57±1.15    & 76.50±1.17    & OOM            & OOM            & 62.00±3.79    & 28.26±4.20     \\
GRACE                      & 65.61±0.61 & 63.92±0.44  & \textbf{68.59±0.35}  & 60.15±0.45   & 92.00±1.39    & 88.64±0.67    & \textbf{83.43±0.63}     & \underline{60.45±1.46}     & 64.00±3.43    & \underline{34.86±3.43}  \\
RoSA                       & 64.06±0.67 & 62.44±0.39  & 67.07±0.65  & 57.68±0.44   & 90.78±2.27    & 85.93±2.14    & 82.39±0.42     & 57.45±2.16     & 64.17±4.10    & 32.20±2.15     \\
BGRL                       & 65.18±0.43 & 63.43±0.45  & 66.83±0.80  & 59.63±0.38   & 92.36±1.16    & 87.14±1.60    & 82.52±0.60     & 55.48±1.48     & 63.67±2.33    & 31.47±3.43     \\
COSTA                      & 65.05±0.80 & 62.37±0.45  & 66.76±0.87  & 55.73±0.36   & \underline{93.50±2.62}    & \underline{89.29±3.11}    & 83.15±0.30 & 55.03±3.22     & 61.66±2.58    & 32.39±2.13 \\
% ArCL                       &            &             & 67.64±0.57  & 59.71±0.44   &               &               &                &                & 65.00±3.94    & 35.41±1.97 \\      
SwAV                       & 62.22±0.53 & 59.79±0.53  & 64.65±0.94  & 55.06±0.39   & 89.00±0.79    & 81.72±0.66    & \underline{83.32±0.15}     & 59.69±1.97     & \underline{65.17±3.76}    & 29.36±2.01    \\
\midrule
MARIO                       & \textbf{67.11±0.46} & \textbf{65.28±0.34}  & \underline{68.46±0.40}  & \textbf{61.30±0.28}   & \textbf{94.36±1.21}    & \textbf{91.28±1.10}    & 82.31±0.54     & \textbf{63.33±1.72}     & \textbf{65.67±2.81}    & \textbf{37.15±2.37}     \\
\bottomrule
\end{tabular}}
\end{table*}

\begin{table*}[htp]
\caption{Experimental results of all methods under covariate shift. The bold font means the top-1 performance and the underline represents the second performance across the unsupervised methods. 'ID' represents in-distribution test performance and 'OOD' means out-of-distribution test performance. (OOM: out-of-memory on a GPU with 24GB memory)}
\label{tab:trans_covariate}
\centering
\scalebox{0.95}{
\begin{tabular}{l|cc|cc|cc|cc|cc}
\toprule
\toprule
\multirow{3}{*}{covariate shift} & \multicolumn{4}{c|}{GOOD-Cora}                                   & \multicolumn{2}{c|}{GOOD-CBAS} & \multicolumn{2}{c|}{GOOD-Twitch} & \multicolumn{2}{c}{GOOD-WebKB} \\
                           & \multicolumn{2}{c}{word} & \multicolumn{2}{c|}{degree}& \multicolumn{2}{c|}{color}    & \multicolumn{2}{c|}{language}   & \multicolumn{2}{c}{university} \\
                           & ID         & OOD         & ID          & OOD          & ID            & OOD           & ID             & OOD            & ID            & OOD            \\
\midrule
ERM                        & 70.50±0.41 & 64.69±0.33  & 72.46±0.49  & 55.53±0.50   & 92.00±3.08    & 77.57±1.29    & 70.98±0.41     & 49.35±5.09     & 39.34±1.79    & 14.52±3.14   \\
IRM                        & 70.48±0.26 & 64.53±0.57  & 71.98±0.34  & 53.72±0.46   & 90.86±2.41    & 78.86±1.67    & 69.81±0.95     & 49.11±2.82     & 38.52±3.30    & 13.97±2.80     \\
EERM                       & OOM        & OOM         & OOM         & OOM          & 65.00±2.57    & 57.43±3.60    & OOM            & OOM            & 46.07±4.55    & 27.40±7.65     \\
\midrule
GAE                        & 56.63±0.79 & 48.93±0.93  & 66.30±0.88  & 34.01±0.87   & 73.00±2.16    & 60.86±3.01    & 67.24±1.23     & 47.65±2.49     & 45.08±6.32    & 28.02±6.29    \\
VGAE                       & 62.02±0.66 & 54.12±0.86  & 69.41±0.57  & 44.20±1.29   & 62.29±2.04    & 63.29±1.11    & 66.99±1.43     & \underline{50.48±4.58}     & 48.85±4.68    & 20.87±6.69     \\
GraphMAE                   & 68.14±0.43 & 64.00±0.33  & \textbf{73.36±0.56}  & 53.75±0.55   & 67.28±3.03    & 67.28±1.49    & 68.84±1.20     & 48.02±2.79     & 48.03±4.34    & 30.00±8.09     \\
DGI                        & 60.85±0.75 & 57.03±0.67  & 68.97±0.41  & 41.75±0.88   & 69.57±4.09    & 59.71±3.43    & 68.43±1.05     & 44.83±1.61     & 48.52±5.04    & 21.11±7.50     \\
MVGRL                      & OOM        & OOM         & OOM         & OOM          & 65.00±1.94    & 64.15±0.77    & OOM            & OOM           & \textbf{54.10±5.39}    & 16.59±6.51     \\
GRACE                      & \underline{68.77±0.33} & \underline{64.21±0.41}  & 72.69±0.34  & \underline{56.10±0.63}   & \underline{93.57±1.83}    & \underline{89.29±3.40}    & \underline{71.12±0.87} & 46.21±1.54 & 49.67±5.82    & 28.10±4.68    \\
RoSA                       & 68.19±0.56 & 62.48±0.61  & 71.04±0.62  & 52.72±0.79   & 84.71±4.14    &79.14±3.51     & 70.58±0.36     & 45.83±1.72     & 52.30±4.24    & \underline{34.24±7.92}     \\
BGRL                       & 67.23±0.43 & 61.33±0.36  & 72.11±0.39  & 49.15±0.73   & 89.00±2.56    & 79.86±3.29    & \textbf{71.43±0.53}     & 43.86±0.94     & 51.80±5.55    & 30.32±7.61    \\
COSTA                      & 65.28±0.60 & 60.33±0.53  & 70.65±0.62  & 54.03±0.28   & 92.29±1.59    & 82.71±2.74    & 69.29±1.37     & 49.07±2.13     & 50.49±3.01    & 29.84±4.75   \\
SwAV                       & 63.29±1.01 & 56.98±0.94  & 70.27±0.73  & 43.00±0.52   & 89.57±1.12    & 81.43±1.69    & 69.19±0.93     & 49.37±2.96     & 49.84±4.82    & 30.55±6.72   \\
\midrule
MARIO                       & \textbf{69.99±0.54} & \textbf{65.06±0.34}  & \underline{72.73±0.43}  & \textbf{57.73±0.45}  & \textbf{94.57±2.46}    & \textbf{91.00±2.48}     & 68.31±0.78 & \textbf{57.37±1.37}     & \underline{53.94±3.23}    & \textbf{35.24±4.98}   \\
\bottomrule
\end{tabular}}

\end{table*}

\subsubsection{Inductive Setting}
In this subsection, we conduct experiments under the inductive settings, where the test nodes are kept unseen during training. This setting is more suitable for domain generalization.
% But we think it is more convincing that conduct experiments under inductive settings which means test nodes are unseen during training. This setting is more appropriate for domain generalization.

\noindent\textbf{Baselines:} For GOOD-WebKB and GOOD-CBAS datasets, we adopt ERM, IRM, GraphMAE, and GRACE as our baselines. And for Amazon-Photo and Elliptic datasets, we select ERM, EERM, and GRACE as our baselines.

\noindent\textbf{Experimental setup:} For GOOD-WebKB and GOOD-CBAS datasets, we use the same model configuration in Section~\ref{sec:trans}.
% Besides, we add experiments on Amazon-Photo dataset~\cite{yang2016revisiting} and Elliptic~\cite{elliptic} dataset in this subsection. 
For Amazon-Photo dataset~\cite{yang2016revisiting} and Elliptic~\cite{elliptic} dataset, they consist of many snapshots (training data and testing data use different snapshots) which are naturally inductive. For Amazon-Photo dataset, we use 2-layer GCN~\cite{gcn} as the encoder and for elliptic dataset, we use 5-layer GraphSAGE~\cite{sage} as encoder following~\cite{eerm}.

% Figure environment removed

\noindent\textbf{Analysis:}
According to Figure~\ref{fig:amazon},\ref{fig:elliptic},\ref{fig:ind_con},\ref{fig:ind_cov}, we can draw following conclusions:
firstly, based on Figure~\ref{fig:amazon}, it is evident that our method outperforms other representative supervised and self-supervised methods on all test graphs (T1$\sim$T8). This superiority is reflected in the larger median value of our method compared to others. For instance, MARIO achieves over a 3\% absolute improvement compared to ERM in terms of the mean value of eight median values. Additionally, our method demonstrates higher stability across different random initializations, as indicated by the closer proximity of the first and third quartile values to the median value~(\eg, the difference of first and third quartile values of ERM, EERM, GRACE and MARIO are 4.2, 3.3, 6.7 and 1.0 on T8 respectively which indicates MARIO is much more stable than other methods). Furthermore, our method exhibits consistent performance across different graphs (\eg, The standard deviation of median values on T1$\sim$T8 for ERM, EERM, GRACE, and MARIO are 0.4, 1.1, 1.2, and 0.3, respectively.), indicating its robustness to environmental variations and its ability to extract invariant features: $g(G^e) \approx g(G^{e'})$ for all $e, e' \in \mathcal{E}^\text{train}$. In summary, our method showcases enhanced OOD generalization capabilities.
% $g(G^e)g(G^e^\prime)$ where $any e, e^\prime in \mathcal{E}^{train}$

Secondly, from the results presented in Figure~\ref{fig:elliptic}, we can observe that our method averagely harvests 10.9\% absolute improvement over GRACE and 12.5\% absolute improvement over EERM in terms of F1 scores on Elliptic dataset. This demonstrates the effectiveness of our method in handling distribution shifts and improving performance compared to existing approaches. It is worth noting that GRACE's performance worsens over time, indicating its inability to handle distribution shifts effectively. In contrast, our method consistently achieves better F1 scores, except for T9, which is caused by the dark market shutdown occurred after T7~\cite{elliptic}. The emergence of such an event introduces significant variations in data distributions, which subsequently results in performance degradation for all methods. Indeed, this event serves as an unpredictable external factor that introduces significant challenges for models trained on limited training data. The results indicate that the performance heavily depends on available training data. Nonetheless, our approach outperforms other methods even in such an extreme case. This highlights the effectiveness of our method in addressing distribution shifts and improving generalization performance.

Finally, based on the observations from Figure~\ref{fig:ind_con} and Figure~\ref{fig:ind_cov} MARIO demonstrates the best performances on both ID and OOD test sets for GOOD-WebKB and GOOD-CBAS datasets, under both concept shift and covariate shift. Notably, MARIO outperforms other methods by more than 3\% and 10\% absolute improvement on GOOD-WebKB and GOOD-CBAS, respectively, under covariate shift. We can draw similar conclusions as discussed in Section~\ref{sec:trans}. Even under the inductive setting, our method continues to demonstrate excellent OOD generalization capabilities and achieves comparable or even improved in-distribution test performance. These statistical results further validate the effectiveness of our method in handling distribution shifts and enhancing generalization performance.

Overall, the observations we have made provide strong evidence of the great capacity of our method for handling distribution shifts, validating its effectiveness and potential for real-world applications.



% Figure environment removed

% Figure environment removed


% Figure environment removed


\subsection{Ablation Studies}\label{sec:ablation}
\noindent Table~\ref{tab:aba} provides a detailed analysis of the effect of each component according to our proposed recipe for improving OOD generalization in graph contrastive learning. Let's examine the different variants of our method and their impact on performance.
Specifically, MARIO~(w/o ad) represents MARIO without  adversarial augmentation. MARIO~(w/o cmi) denotes we only maximize the mutual information between positive pairs without considering conditional mutual information. MARIO~(w/o cmi, ad) means a vanilla graph contrastive method that is similar to GRACE. 

From Table~\ref{tab:aba}, we can find MARIO~(w/o cmi) lags far behind MARIO on OOD test set which demonstrates appropriately minimizing the redundant information (\ie, conditional mutual information) is essential to improve OOD generalization of GCL methods. And adversarial augmentation can also boost OOD generalization because it can approximately serve as a supermum operator to learn more invariant features  discussed in Section~\ref{sec:aug}. Based on the analysis of these variants, it is evident that the proposed improvements on data augmentation and contrastive loss in the recipe are both effective in enhancing graph OOD generalization. Each component contributes to the overall performance improvement, and their combination leads to a stronger self-supervised graph learner in terms of graph OOD generalization. 

In short, the findings from Table~\ref{tab:aba} support the rationale behind your proposed recipe and provide empirical evidence of the effectiveness of each proposed component. By incorporating these enhancements, our method achieves superior performance in handling distribution shifts and improving graph OOD generalization in graph contrastive learning.
\begin{table*}[htp]
\caption{Ablation studies for MARIO by masking each component.}
\label{tab:aba}
\centering
\scalebox{0.9}{
\begin{tabular}{l|cc|cc|cc|cc|cc}
\toprule
\toprule
\multirow{3}{*}{concept shift} & \multicolumn{4}{c|}{GOOD-Cora}                       & \multicolumn{2}{c|}{GOOD-CBAS} & \multicolumn{2}{c|}{GOOD-Twitch} & \multicolumn{2}{c}{GOOD-WebKB} \\
                           & \multicolumn{2}{c}{word} & \multicolumn{2}{c|}{degree}& \multicolumn{2}{c|}{color}    & \multicolumn{2}{c|}{language}   & \multicolumn{2}{c}{university} \\
                           & ID         & OOD         & ID          & OOD          & ID            & OOD           & ID             & OOD            & ID            & OOD            \\
\midrule
MARIO                      & \textbf{67.11±0.46} & \textbf{65.28±0.34}  & \textbf{68.46±0.40}  & \textbf{61.30±0.28}      & \textbf{94.36±1.21}  & \textbf{91.28±1.10}    & 82.31±0.54     & \textbf{63.33±1.72}     & \textbf{65.67±2.81}    & \textbf{37.15±2.37}     \\
MARIO(w/o ad)              & 66.23±0.53 & 64.02±0.18  & 67.88±0.38  & 60.46±0.29   & 93.21±1.25    & 90.29±0.91    & 82.42±0.73     & 60.50±1.02     & 64.83±2.83    & 36.51±3.25    \\
MARIO(w/o cmi)             & 65.32±0.60 & 63.51±0.32  & 68.14±0.32  & 61.19±0.34   & 94.15±1.23    & 90.57±1.96    & \textbf{82.51±0.56}     & 61.41±2.63     & 64.50±4.35    & 35.78±2.53     \\
MARIO(w/o cmi, ad)         & 64.67±0.55 & 63.11±0.32  & 67.95±0.65  & 60.01±0.57   & 93.36±1.66    & 89.64±1.73    & 81.90±0.75     & 60.12±1.60     & 64.17±3.67    & 34.13±2.38     \\
\bottomrule
\end{tabular}}
\end{table*}
% & 65.32±0.60 & 63.51±0.32 exchange 64.67±0.55 & 63.11±0.32
% 68.14±0.32       id ood test: 60.95±0.43       ood ood test: 61.19±0.34


\subsection{Sensitivity Analysis}\label{sec:sensitivity}
\noindent In this subsection, we will analyze some important hyper-parameters of our method. We conduct sensitivity analysis on GOOD-WebKB dataset with concept shift, we chose two sensitive hyper-parameters (\ie, the coefficient $\gamma$ of condition mutual information in Equation~\ref{equ:cmi} and the number of prototypes $|C|$ in Equation~\ref{equ:pq}). The coefficient of CMI range in $[0.001, 0.01, 0.1, 0.5, 1]$ and the number of prototypes $|C|$ ranges in $[10, 50, 100, 200, 300]$. From Figure~\ref{fig:sensitivity}, we can observe that $\gamma$ reaches 0.1 and $|C|$ reaches 100 or 200 can achieve the best OOD test accuracy. Both higher and lower values of $\gamma$ result in suboptimal performance. This finding aligns with previous research such as DIB~\cite{dib}, indicating that an appropriate compression level is crucial for achieving optimal performance. Extremely high or low compression values are not ideal. 

Regarding the number of prototypes $|C|$, based on the results shown in Figure~\ref{fig:sensitivity}, it is found that setting $|C|=100$ leads to the best performance in terms of OOD test accuracy. This choice provides a moderate number of pseudo labels, which is beneficial for the learning process. 

Based on the sensitivity analysis, we determined that setting $\gamma=0.1$ and $|C|=100$ on most datasets. These hyperparameter values strike a balance between compression level and the number of prototypes, resulting in improved graph OOD generalization.
% Figure environment removed


\subsection{Integrated with Other Models}\label{sec:other_models}
% Figure environment removed

\begin{table}[htp]
\caption{Results of different learning approaches with different encoding models (\ie, GCN, GraphSAGE, GAT).}
\label{tab:others}
\centering
\scalebox{0.9}{
\begin{tabular}{cc|cc|cc}
\toprule
\toprule
\multirow{3}{*}{Model}& \multirow{3}{*}{Method} & \multicolumn{2}{c|}{GOOD-CBAS} & \multicolumn{2}{c}{GOOD-WebKB} \\
                & & \multicolumn{2}{c|}{color}    & \multicolumn{2}{c}{university} \\
                &   & ID          & OOD         & ID          & OOD            \\
\midrule
\multirow{3}{*}{GCN} 
&ERM               & 89.79±1.39 & 83.43±1.19  &  62.67±1.53 & 26.33±1.09         \\
&GRACE             & 92.00±1.39 & 88.64±0.67  &  64.00±3.43 & 34.86±3.43        \\
&MARIO             & 94.36±1.21 & 91.28±1.10  &  65.67±2.81 & 37.15±2.37        \\ \bottomrule
\multirow{3}{*}{SAGE} 
&ERM               & 95.07±1.51 & 75.14±1.19  & 73.67±2.08  & 46.33±3.42       \\
&GRACE             & 95.29±1.11 & 74.43±2.36  & 70.50±5.06  & 49.54±3.83        \\
&MARIO             & 96.00±1.07 & 76.29±3.01  & 71.00±3.82  & 51.74±4.63        \\ \bottomrule
\multirow{3}{*}{GAT} 
&ERM               & 78.64±3.63 & 72.93±2.64  & 61.33±3.71  & 28.99±2.63        \\
&GRACE             & 84.57±1.79 & 78.36±1.60  & 59.50±2.36  & 35.78±3.26        \\
&MARIO             & 84.93±1.95 & 80.43±1.89  & 62.17±4.78  & 38.17±3.10        \\
\bottomrule
\end{tabular}}
\end{table}



\noindent In the subsection, we demonstrate the model-agnostic nature of the recipe by integrating it with various graph neural network (GNN) models, including GCN, GraphSAGE, and GAT.

From Table~\ref{tab:others}, it can be observed that regardless of the specific GNN model used as the encoder, our method consistently achieves the best performance on the OOD test set. This indicates the effectiveness and robustness of our method across different GNN models.
By achieving superior performance across different GNN models, MARIO demonstrates its versatility and ability to improve the OOD generalization of various graph neural models. This highlights the broad applicability and effectiveness of our recipe in enhancing the performance of different GNN encoders.

Furthermore, we integrate our recipe with other GCL methods in Appendix~\ref{app:other_methods}. The results demonstrate our recipe can boost the OOD generalization ability of various GCL methods which means our recipe can serve as a plug-in for many current classical GCL methods.

% Figure environment removed

\subsection{Visualization}\label{sec:vis}
\subsubsection{Metric Score Curves}
We present metric score curves for ERM and MARIO, including training, ID validation, ID testing, OOD validation, and OOD testing accuracy, in Figure~\ref{fig:curve2}. Notably, MARIO demonstrates superior convergence with approximately 10\% absolute improvement on the OOD test set compared to ERM. Furthermore, MARIO effectively narrows the performance gap between in-distribution and out-of-distribution performance, showcasing its efficacy in enhancing OOD generalization for graph data. More metric score curves can be found in Appendix~\ref{app:curves}.


\subsubsection{Feature Visualization}
In order to assess the quality of learned embeddings, we adopt t-SNE~\cite{tsne} to visualize the node embedding on GOOD-Cora dataset (concept shift in word domain) using random-init of GCN, EERM, GRACE, and MARIO, where different classes have different colors in Figure~\ref{fig:vis}. For clarity, we select eight classes with the largest number of nodes to enhance the informativeness and interpretability of the visualization. We can observe that the 2D projection of node embeddings learned by MARIO has a better separation of clusters, which indicates the model can help learn representative features for downstream tasks. It has to note that we depict both ID nodes and OOD nodes in the same figure. 

Besides, we also separately visualize ID nodes and OOD nodes in the different figures in the Appendix~\ref{app:feature}. And we can find MARIO performs a clearer separation of clusters whether on ID nodes or OOD nodes compared to other methods.




\bibliography{references-wo-doi-url}
%
%\begin{thebibliography}{8}
%\bibitem{ref_article1}
%Author, F.: Article title. Journal \textbf{2}(5), 99--110 (2016)

%\bibitem{ref_lncs1}
%Author, F., Author, S.: Title of a proceedings paper. In: Editor,
%F., Editor, S. (eds.) CONFERENCE 2016, LNCS, vol. 9999, pp. 1--13.
%Springer, Heidelberg (2016). \doi{10.10007/1234567890}

%\end{thebibliography}
\newpage 

\appendix
\section{Appendix}
\subsection{Proof of the Fixed-point Formula for \Wo}\label{app:fp-proof}
It was recently shown in \cite{banerjee2022fast} that the winning region $\We$ for \Even in an \Odd-fair parity game $\mathcal{G}^\ell$ with least even upper bound priority $l\geq 0$ can be computed by the fixed-point algorithm 
% 
\begin{align}\label{eq:fp-even}
 \We = &\nu {Y_l}.~ \mu X_{l-1}.~ \ldots \nu{Y_2}.~ \mu{X_1}.~ \bigcup_{j \in \ev{2}{l}} \A_j \quad \\
 & \text{ where, } \quad \A_j := \left(C_j \cap Cpre_\Even(Y_j)\right) \cup \left(\left(\textstyle\bigcup_{i \in [1,j-1]}C_i\right) \cap \Apre(Y_j, X_{j-1})\right)\nonumber
\end{align}

As  \Odd-fair parity games are determined, we can simply compute the winning region for player $\Odd$ by negating \eqref{eq:fp-even}, which leads to Prop.~\ref{prop: W_Odd}. For the sake of self-containment, we restate Prop.~\ref{prop: W_Odd} here.

% restating prop 2
\begingroup
\def\theproposition{\ref{prop: W_Odd}}
\begin{proposition}
    Given an \Odd-fair parity game $\mathcal{G}^\ell = (\ltup{V, \Ve, \Vo, E, \chi}, E^\ell)$ with least even upper bound $l\geq 0$ and
\begin{align}\label{eq:fp-odd-app}
    Z := &\mu {Y_l}.~  \nu {X_{l-1}}.~  \ldots \mu{Y_2}.~  \nu{X_1}.~  \bigcap_{j \in \ev{2}{l}} \B_j, \\
    &\text{ where} \quad
    \B_j := \left(\textstyle\bigcup_{i \in [j+1,l]} C_i\right) \cup \left(\overline{C_j} \cap \Npre(Y_j, X_{j-1}) \right) \cup \left(C_j \cap \Cpre_\Odd(Y_j)\right)\nonumber
\end{align}
then $\Phi=\Wo$.
Further, it takes $\mathcal{O}(n^{l+1})$ symbolic steps to compute $\Wo$ via \eqref{eq:fp-odd}.
\end{proposition}
\addtocounter{proposition}{-1} % decrease the counter that holds proposition numbers, so that the previous restated proposition is not seen.
\endgroup

\begin{proof}
We use the negation rule of the $\mu$-calculus, i.e., $\neg (\mu X~.~F(X))=\nu X~.~\neg F(\neg X)$, to negate \eqref{eq:fp-even}. Using the equivalences in \eqref{equ:Preseq} and \eqref{equ:cpre_equal} and common De-Morgan laws, we get 
\begin{subequations}
\begin{align}
 \neg \A_j(\neg Y_j,\neg X_{j-1})=&\left(\overline{C_j} \cup \Cpre_\Odd(Y_j)\right) \cap \left(\left(\textstyle\bigcup_{i \in [j,l]} C_i\right) \cup \Npre(Y_j, X_{j-1})\right)\\
 =&\left(\textstyle\bigcup_{i \in [j+1, l]} C_i\right) \cup \left(\overline{C_j} \cap \Npre(Y_j, X_{j-1})\right) \nonumber\\
 &\cup \left(C_j \cap \Cpre_\Odd(Y_j)\right) \cup \left(\Cpre_\Odd(Y_j) \cap \Npre(Y_j, X_{j-1})\right)\label{eq2}\\
 =&\left(\textstyle\bigcup_{i \in [j+1,l]} C_i\right) \cup \left(\overline{C_j} \cap \Npre(Y_j, X_{j-1}) \right) \cup \left(C_j \cap \Cpre_\Odd(Y_j)\right)
\end{align}\end{subequations}
where the last equivalence follows from the observation that the last term of \eqref{eq2} is redundant since it is a subset of both $ \Npre(Y_j, X_{j-1})$ and $\Cpre_\Odd(Y_j)$: If a $v$ is in the last term, it either has priority $j$, in which case it is already in $C_j \cap \Cpre_\Odd(Y_j)$, or it has a different priority, in which case it is already in $\Npre(Y_j, X_{j-1})$. %As all formal variables can have arbitrary symbols, we just rename them to their non-overlined versions but swap the preceding $\mu$/$\nu$ operators. This yields \eqref{eq:fp-odd} from negating \eqref{eq:fp-even}.
\end{proof}

\subsection{Proof of Prop.~\ref{prop:mainresult}}\label{app:counter-strategy-templates}
    We will restate the fixed-point formula that calculates the \Odd winning region and the main proposition for the sake of self-containment.
    
% restating prop 2
\begingroup
\def\theproposition{\ref{eq:fp-odd}}
\begin{proposition}\label{app-eq:fp-odd}
    Given an \Odd-fair parity game $\mathcal{G}^\ell = (\ltup{V, \Ve, \Vo, E, \chi}, E^\ell)$ with least even upper bound $l\geq 0$ it holds that $Z=\Wo$, where
    \begin{align}
        Z &:=\textstyle \mu {Y_l}.~  \nu {X_{l-1}}.~  \ldots \mu{Y_2}.~  \nu{X_1}.~  \bigcap_{j \in \ev{2}{l}} \B_j[Y_j, X_{j-1}], \\
        &\text{ where} \quad
        \B_j[\mathbf{Y}, \mathbf{X}] := \left(\textstyle\bigcup_{i \in [j+1,l]} C_i\right) \cup \left(\overline{C_j} \cap \Npre(\mathbf{Y}, \mathbf{X}) \right) \cup \left(C_j \cap \Cpre_\Odd(\mathbf{Y})\right).\nonumber
    \end{align}
     then $Z=\Wo$.
     Further, it takes $\mathcal{O}(n^{l+1})$ symbolic steps to compute $Z$.
\end{proposition}
\addtocounter{proposition}{-1} % decrease the counter that holds proposition numbers, so that the previous restated proposition is not seen.
\endgroup
    
    \begingroup
    \def\theproposition{\ref{prop:mainresult}}
    \begin{proposition}\label{app-prop:mainresult}
        Every player \Odd strategy compliant with $\Sc^{\mathcal{G}^\ell}$ is winning for \Odd in $\mathcal{G}^\ell$.

    \end{proposition}
    \addtocounter{proposition}{-1} % decrease the counter that holds proposition numbers, so that the previous restated proposition is not seen.
    \endgroup

    The main observation behind the proof of Prop.~\ref{app-prop:mainresult} is similar to the main observation in Sec.~\ref{sec:zielonka}, leading to the proof of Alg.~\ref{algo:fair-zielonka-bb}.
    That is, there exists a core subset of the \Odd winning region $\Wo'\subseteq \Wo$, that is added to $Z$ in the first iteration of the 
    fixed-point calculation in ~\eqref{eq:fp-odd}, to which each $v \in \Wo$ can be made to reach by \Odd. 
    Here in particular, we show that any \Odd strategy compliant with $\Sc^{\mathcal{G}^\ell}$ reaches $\Wo'$ (infinitely often) while obeying the fairness condition, and is thus winning for \Odd.

    The proof of Prop.~\ref{prop:mainresult} consists of $3$ main propositions. Before we present them, we will gather some observations from the fixed-point formula ~\eqref{app-eq:fp-odd} and present them as lemmas.
    
    According to our previous definitions, $Y_m^{r_l, r_{l-1}, \ldots, r_m}$ denotes the value of $Y_m$ variable after the $r_{m}^{th}$ iteration on it, while $Y_i, X_i$ variables for $i>m$ are in their ${{r_i}+1}^{th}$ iterations.
    If we flatten this formula we get the following equality: $Y_m^{r_l, r_{l-1}, \ldots, r_m} = $
    $$\nu X_{m-1}\ldots \mu Y_2 \nu X_1. \bigcap_{j \in \ev{m+2}{l}} \B_j[Y_j^{r_j}, X_{j-1}^{r_{j-1}}] \cap \B_m[Y_m^{r_m-1}, X_{m-1}] \cap \bigcap_{j \in \ev{2}{m-2}} \B_j[Y_j, X_{j-1}]$$

    Observe that when the fixed-point above is calculated, all $X_{j}, Y_j$ values for $j < m$ will saturate at the same value,
    which is the final result of the computation. That is, 
    \begin{lemma}\label{app-obs:flat-Z}
    $$ Y_m^{r_l, \ldots, r_m} = \bigcap_{j \in \ev{m+2}{l}} \B_j[Y_j^{r_j}, X_{j-1}^{r_{j-1}}] \cap \B_m[Y_m^{r_m-1},  Y_m^{r_l, \ldots, r_m}] \cap \bigcap_{j \in \ev{2}{m-2}} \B_j[Y_m^{r_l, \ldots, r_m}, Y_m^{r_l, \ldots, r_m}]$$
    \end{lemma}

    \begin{lemma}\label{app-lem:intersection_of_Y_m}   
        For all $v \in \Wo$ with $\rank{v} = (r_l, 0, \ldots, r_2, 0)$. Then,
        $$v \in \bigcap_{j \in \ev{2}{l}} Y_j^{r_l-1, 0, r_{l-2}-1, 0, \ldots, r_{j-2}-1, 0, r_j}$$
    \end{lemma}
        This is similar to our previous observation. $\rank{v} = (r_l, 0, \ldots, r_2, 0)$ implies $v$ was added to the formula while
        $Y_j$ variable was on it's $r_j^{th}$ iteration for all $j \in \ev{2}{l}$. Since $X_{j-1}^0 = V$, the iteration values of $X$ variables can be safely ignored. 
    
    \begin{lemma}\label{app-obs:v-Even-Odd-inequalities} 
        $\quad \text{if } v \in V_\Even, \quad \quad \forall(v, w)\in E, \rank{v}\geq_{l+1-\chi(v)} \rank{w}$
        $$\text{if } v \in V_\Odd, \quad \quad \exists(v, w)\in E, \rank{v}\geq_{l+1-\chi(v)} \rank{w}$$
       % \begin{align*}
       %     \quad\quad\quad\quad\quad\quad\text{if } v \in V_\Even, \quad \quad \forall(v, w)\in E, \rank{v}\geq_{l+1-\chi(v)} \rank{w}\\
       %     \quad\quad\quad\quad\quad\quad\text{if } v \in V_\Odd, \quad \quad \exists(v, w)\in E, \rank{v}\geq_{l+1-\chi(v)} \rank{w}
       % \end{align*}
        where $\rank{v} \geq_b \rank{w}$ denotes the $\geq$ relation in the lexicographic ordering, restricted to the first b elements of the tuple. If $\chi(v)$ is even, the inequalities are strict. 
    \end{lemma}
    \begin{proof}
        Consider a $v$ with $\chi(v) \in \{m-1, m\}$ for some even $m$ and let $\rank{v} = (r_l, 0, \ldots, r_2, 0)$.
        By Lem.~\ref{app-lem:intersection_of_Y_m}, $v \in Y_m^{r_l-1, 0, \ldots, r_{m-2}-1, 0, r_m}$. If we look at the flattening of this formula in Lem.\ref{app-obs:flat-Z}, $v$ is in particular, inside the middle term of this formula. That is,
         \\$v \in \B_m[Y_m^{r_l-1, \hdots, r_m-1}, Y_m^{r_l-1,\hdots, r_m }]$. If we go through the definition of this term we get,
            $$(\bigcup_{i \in [m+1, l]} C_i) \cup (\overline{C_m} \cap \Npre(Y_m^{r_l-1, \hdots, r_m-1}, Y_{m}^{r_l-1,0, \hdots, r_m})) \cup (C_{m} \cap \Cpre_\Odd(Y_m^{r_l-1, 0, \hdots, r_m-1}))$$
        
            \vspace{-7mm}
            \begin{align*}
               \text{ That gives us, } \quad \quad &\text{if } \chi(v) = m, \quad \quad \quad\quad\quad \,\,\, v \in \Cpre_\Odd(Y_m^{r_l-1, 0, \ldots, r_m-1}) \\
                &\text{if } \chi(v) = m-1, \quad \quad \quad\quad v \in \Npre(Y_m^{r_l-1, 0, \ldots, r_m-1}, Y_m^{r_l-1, 0, \hdots, r_m}) \\
        \end{align*}

        By the definition of $\Npre$ we get, if $\chi(v) = m-1$ then $v \in \Cpre_\Odd(Y_m^{r_l-1, 0, \hdots, r_m})$.
       Since odd indices get $0$-ranks, the claim of the lemma follows from the definition of $\Cpre_\Odd$ together with the observation $\rank{v} \geq_{l+1-m} \rank{w} \Leftrightarrow \rank{v} \geq_{l+1-(m-1)} \rank{w}$.
            %from the fact that, if $w \in Y_m^{r_l-1, 0, \hdots, r_m}$, then then first $l+1-m$ of $\rank{w}$ is less than or equal to that of $\rank{v}$. Since odd indices are always $0$, $\rank{v} \geq_{l+1-m} \rank{w} \iff \rank{v} \geq_{l+1-(m-1)} \rank{w}$. 
    \end{proof}

    Now we are ready to introduce the first of our three main propositions:
    
    \begin{proposition}\label{app-prop:Mexists}
        If $\Wo \neq \emptyset$, there exists a non empty set $M := \{ v \in \Wo \mid \rank{v} = (1, 0, 1, 0, \ldots, 1, 0)\}$. Furthermore, for all $v\in M$, $\chi(v)$ is odd.
    \end{proposition}
    Observe that $(1,0,1,0,\ldots, 1,0)$ is the smallest rank possible. Therefore, $v\in M$ are the vertices that were added to $Z$ in ~\eqref{app-eq:fp-odd} in the first iteration of the fixed-point calculation and were never removed.
    The first part of the proposition follows from the monotonicity of fixed-point calculation. That is, if $M$ was empty $Z$ would be empty as well.
    
    For the second part, observe that in the first iteration of the formula, for all $j$, $Y_j = \emptyset$. Also, $\Cpre_\Odd(\emptyset) = \emptyset$. 
    Then from~\eqref{app-eq:fp-odd}, $Z$ does not contain any $v$ with even priority. 
    
    \begin{proposition}\label{app-prop:cycle-through-M}
        All cycles in $\Sc^{\mathcal{G}^\ell}$ that pass through a vertex in $M$ are \Odd winning.% (i.e. the largest priority in the cycle is odd).
    \end{proposition}

    To see why Prop.~\ref{app-prop:cycle-through-M} holds, we make an observation.
    For an even $m\leq l$, let $Y_m^\1$ denote the value of $Y_m$ after the first ever iteration over it is completed, during the computation of ~\ref{eq:fp-odd}.
    I.e. $Y_m^\1 = Y^{0,0,\ldots ,0, 1} $.
    %$$\nu X_{m-1}\ldots \mu Y_2 \nu X_1. \bigcap_{j \in \ev{m+2}{l}} \B_j[\emptyset, V] \cap \B_m[\emptyset, X_{m-1}] \cap \bigcap_{j \in \ev{2}{m-2}} \B_j[Y_j, X_{j-1}]$$
    %In the first term $\B_j$ takes $\emptyset$ and $V$ as arguments. This is due to all $Y_{j}, X_{j-1}$ variables for $j \leq m$ having the values they are initialized with. Observe that when the fixed-point above is calculated, all $X_{j-1}, Y_j$ values for $j < m$ will saturate at the same value,
    %which is the final result of the computation. That is, 
    Since for all $j$, $Y_j^0 = \emptyset$ and $ X^0_{j-1} = V$, Lem.~\ref{app-obs:flat-Z} gives,
    \begin{equation}\label{eq:Ym1}
    Y_m^\1 = \bigcap_{j \in \ev{m+2}{l}} \B_j[\emptyset, V] \cap \B_m[\emptyset, Y_m^\1] \cap \bigcap_{j \in \ev{2}{m-2}} \B_j[Y_m^\1 Y_m^\1]
    \end{equation}
    If we go through the definition of $\B_j$ we see that: the first term of this formula adds or deletes $v \in C_j$ with $j > m$. It adds all the ones with odd $j$  and removes all the ones with even $j$.%is equal to $\bigcup_{j \in \ev{m+2}{l}}C_{j-1} \cup \bigcup_{j \in [1, m+1]}C_j$. That is, the first term eliminates all $v \in C_j$ with even $j>m$ from $Y_m^\1$ and add all $C_j$ with odd $j>m$.
     The last term adds and removes $v \in C_j$ for $j \leq m-2$. It adds the ones in $\Cpre_\Odd(Y_m^\1)$ and removes the ones that are not. The middle term eliminates $C_m$ and all $v \in C_j \cap \neg \Npre(\emptyset, Y_m^\1)$ for $j < m$, and adds $v \in C_{m-1} \cap \Npre(\emptyset, Y_m^\1)$.
    If we go through the definition of $\Npre$, we see that $\Npre(\emptyset, Y_m^\1) = \Cpre_\Odd(Y_m^\1) \cap (V_\Even \cup \Lpre^\forall(Y_m^\1))$.
    This gives,
    \begin{equation}\label{app-eq:obs}
         v \in Y_m^\1 \iff \chi(v)>m\text{ and is odd, or } \chi(v)< m \text{ and } v\in \Npre(\emptyset, Y_m^\1)
    \end{equation}
    %$ Y_m^\1$ consists of $v$ with either odd $\chi(v)>m$, or in $\Npre(\emptyset, Y_m^\1)$.

    Then for all $v \in M$, $v \in Y_m^\1$ for each even $m \leq l$. In particular, $ v \in Y_n^\1$ where $n$ is such that $\chi(v) = n-1$.
    It follows that $ v \in \B_n[\emptyset, Y_n^\1]$. Then, $v \in \Cpre_\Odd(Y_n^\1) \cap (V_\Even \cup \Lpre^\forall(Y_n^\1))$.
    Since all live outgoing edges of $v$ are in $Y_n^\1$, for all $(v,w)$ in $\Sc^{\mathcal{G}^\ell}$, $w \in Y_n^\1$.

    By our previous observation $w$ either has an odd priority larger than $n$, or is in $ \Cpre_\Odd(Y_n^\1) \cap (V_\Even \cup \Lpre^\forall(Y_n^\1))$.
    If $\chi(w)>n$ is odd, then $w \in Y^\1_{\chi(w)+1}$, and we repeat the same argument to conclude the highest priority seen is always odd.

    %Before we present the third proposition, we need a lemma and some observations obtained from formula ~\eqref{eq:fp-odd}.
    %\begin{lemma}\label{obs:v-Even-Odd-inequalities} 
    %%    \begin{align*}OBS
     %       &\quad \quad \quad \quad  \quad \quad \text{if } v \in V_\Even, \quad \quad \forall(v, w)\in E, \rank{v}\geq_{l+1-\chi(v)} \rank{w}\\
     %       &\quad \quad \quad  \quad \quad \quad  \text{if } v \in V_\Odd, \quad \quad \exists(v, w)\in E, \rank{v}\geq_{l+1-\chi(v)} \rank{w}
     %   \end{align*}
     %   where $\rank{v} \geq_b \rank{w}$ denotes the $\geq$ relation in the lexicographic ordering, restricted to the first b elements of the tuples $\rank{v}$ and $\rank{w}$. If $\chi(v)$ is odd, the inequalities are strict. 
    %\end{lemma}
    \begin{definition}
        We call a play $\pi = v_1 v_2 \ldots$ in $\Sc^{\mathcal{G}^\ell}$ \emph{minimal} if for all $v_i \in V_\Odd$, $v_{i+1}$ is the minimum ranked successor of $v_i$. A minimal cycle is a section of a minimal play.
    \end{definition}
    \begin{lemma}\label{app-lem:minimalplayOddwinning}
        Every minimal play is \Odd winning.
    \end{lemma}
    A minimal play only sees minimal cycles. Let $\delta = w_1 w_2 \ldots w_1$ be such a cycle. 
    $\delta$ cannot be an \Even winning cycle: Assume $b := \max\{ \chi(w) \mid w \in \delta\} $ is even. Let $w_i\in \delta$ have priority $b$. By Obs. \ref{app-obs:v-Even-Odd-inequalities}, $\rank{w_i} >_{l+1-b} \rank{w_{i+1}} \geq_{l+1 - \chi(w_{i+1})} \ldots \geq_{l+1-\chi(w_{i-1})} \rank{w_i}$. Since for all $w_j \in \delta$, $\chi(w_{j})\leq b$, the inequality yields $\rank{w_i} >_{l+1-b} \rank{w_i}$, which is a contradiction.

    %The last proposition states that all $\pi$ that starts in \Wo and is compliant with $\Sc^{\mathcal{G}^\ell}$, visits $M$ infinitely often. 
    \begin{proposition}\label{app-prop:minimal-play-visits-M}
        Any minimal play compliant with $\Sc^{\mathcal{G}^\ell}$ visits $M$ infinitely often.
    \end{proposition}
    %Any minimal play sees minimal cycles infinitely often. 
    Let $\delta = w_1 w_2 \ldots w_1$ be a minimal cycle and $w_k$ its vertex with maximum priority. We will show that $w_k \in M$. Since $\pi = \delta \delta \ldots$ is a minimal play, by Lemma.~\ref{app-lem:minimalplayOddwinning} we know $\chi(w_k)$ is odd. Furthermore, we have observed in \ref{app-eq:obs} that $w_k \in Y_m^\1$ for all $m > \chi(w_k)$. 
    If we can show that $w_k \in Y_m^\1$ also for $m < \chi(w_k)$, then we have $w_k \in M$. We will now show this. 
    
    Assume to the contrary that $w_k \not \in M$ and let $j$ be the largest non-trivial index of $\rank{w_k}$. 
    That is $j < l$ is the largest even integer such that $w_k \not \in Y_j^\1$. Let $t$ be the value of this index, i.e. $w_k \in Y_j^{0,\ldots, 0,t} \setminus Y_j^{0,\ldots, 0,{t-1}}$. 
    Let us denote $Y_j^{0, \ldots, 0, t}$ by $Y_j^\te$ for short. 

    Since $\delta$ is minimal, Lem.~\ref{app-obs:v-Even-Odd-inequalities} gives $\rank{w_i} \geq_{l+1 - \chi(w_i)} \rank{w_{i+1}}$ for all $w_i \in \delta$. Since $\chi(w_i) \leq \chi(w_k)$ for all $i$ and $\chi(w_k) < j$; $\rank{w_i} \geq_{l+1-j} \rank{w_{i+1}}$ for all $w_i \in \delta$. 
     This implies $\rank{w} =_{l+1-j} \rank{w'}$ for all $w, w'\in \delta$. It follows that for all $w\in \delta$, $w \in Y_j^\te \setminus Y_j^{\mathbf{t-1}}$.
 
     %We can follow the same steps in equation~\eqref{eq:Ym1} to observe that 
   %$\forall (v,w)$ in $\Sc^{\mathcal{G}^\ell}$, $w \in \Npre(Y^{\mathbf{t-1}}_j, Y^\te_j) = \Cpre_\Odd(Y^\te_j) \cap (V_\Even \cup \Lpre^\forall(Y^\te_j) \cup Pre^\exists_\Odd(Y^\mathbf{t-1}_j))$.
   %If $w \in Pre^\exists_\Odd(Y_j^{\mathbf{t-1}})$, since $\delta$ is a minimal cycle, $\delta$ will have an element from $Y_j^{\mathbf{t-1}}$. However, this contradicts our observation that $\delta$ lies in $Y_j^\te \setminus Y_j^{\mathbf{t-1}}$.
   %On the other hand if non of the $w \in \delta $ lie in $ Pre^\exists_\Odd(Y_j^{\mathbf{t-1}})$ this implies that they all get into the formula due to reaching other nodes in $Y_j^\te \setminus Y_j^\mathbf{t-1}$. This is not possible since a node in $\delta$ has to be added to $Y_j^\te \setminus Y_j^\mathbf{t-1}$ as the first node and thus, have to have a successor in $Y_j^\mathbf{t-1}$. % while $Y_j^\te \setminus Y_j^{\mathbf{t-1}}$ is empty.
   %  Therefore, $w_k\in M$.

     Once more by Lem.~\ref{app-obs:flat-Z} we get that for all $w \in \delta$, 
     $$w \in \B_j[Y_j^{\mathbf{t-1}}, Y_j^\te] = (\bigcup_{i\in[j+1, l]} C_i) \cup ( \overline{C_j} \cap \Npre(Y_j^{\mathbf{t-1}}, Y_j^\te) \cup (C_j \cap \Cpre_\Odd(Y_j^{\mathbf{t-1}})))$$
     Since $\chi(w) < j$, this implies $$w \in \Npre(Y_j^{\mathbf{t-1}}, Y_j^\te) = \Cpre_\Odd(Y_j^\te) \cap (V_\Even \cup \Lpre^\forall(Y_j^\te) \cap \Pre^\exists_\Odd(Y_j^{\mathbf{t-1}}) )$$
     %Since we assumed all $w \in Y_j^\te \setminus Y_j^{\mathbf{t-1}}$, $w$ cannot be in the last term.
     Now consider the set $Y_j^\te \setminus Y_j^{\mathbf{t-1}}$, which is initially empty. Then the first term in $\delta$ that gets in $Y_j^\te \setminus Y_j^{\mathbf{t-1}}$ has to be in $\Pre^\exists_\Odd(Y_j^{\mathbf{t-1}})$. 
     This contradicts our assumption that all $w_i \in  Y_j^\te \setminus Y_j^{\mathbf{t-1}}$ and proves that $w_k \in M$.
     We are now ready to prove the main theorem.
     \begin{proof}[Proof of Thm. \ref{prop:mainresult}]
          Let $\pi= v_0v_1\ldots$ be a play compliant with $\Sc^{\mathcal{G}^\ell}$ with $v_0 \in \Wo$. %Since all plays compliant with \Odd strategy templates obey the fairness condition, we only need to show that the maximum priority seen infinitely  in $\pi$ is odd.
          Since $\pi$ is compliant with an \Odd strategy template, it is a fair play. 
          For a node $v \in \Wo$, let $v_{\min}$ be the minimum ranked successor of $v$.
          Since $\pi$ is fair, for all $v$ that is visited infinitely often in $\pi$, $v_{\min}$ is visited infinitely often as well. 
          This gives us an infinite subsequence of $\pi$ that is minimal. Since all minimal plays visit $M$ infinitely often (Prop.~\ref{app-prop:minimal-play-visits-M}), 
          $\pi$ visists $M$ infinitely often. Then there must exist an $x \in M$ that $\pi$ visits infinitely often. 
          Then a tail of $\pi$ is consisted of consecutive cycles over $x$. Since all cycles that pass through $M$ are \Odd winning (Prop.~\ref{app-prop:cycle-through-M}), $\pi$ is \Odd-winning.
   \end{proof}
    
%     
%     
% %   The main insight behind the proof of Prop.~\ref{prop:mainresult} is actually similar to the one enabling the proof of the \Odd-fair Zielonka's algorithm given in Sec.~\ref{} and consists of three steps: 
%   
%   It shows that there exists a core subset of the \Odd winning region $M\subseteq \Wo$, that is added to $Z$ in the first iteration of the 
%     fixed-point calculation in ~\eqref{eq:fp-odd}, to which each $v \in \Wo$ can be forced to reach by \Odd. 
%     
%     
%     
%     Here in particular, we show that any \Odd strategy compliant with $\Sc^{\mathcal{G}^\ell}$ reaches $\Wo'$ (infinitely often) while obeying the fairness condition, and is thus winning for \Odd.
% 
%     The full proof of Prop.~\ref{prop:mainresult} can be found in App.~\ref{??}. It consists of $3$ main propositions which we present here one-by-one along with some intuition on why they hold.
% 
%     \begin{proposition}\label{prop:Mexists}
%         If $\Wo \neq \emptyset$, there exists a non empty set $M := \{ v \in \Wo \mid \rank{v} = (1, 0, 1, 0, \ldots, 1, 0)\}$.
%     \end{proposition}
%     Observe that $(1,0,1,0 ,\ldots, 1, 0)$ is the minimum rank the ranking function assigns to a vertex. Also, the vertices in $M$ are exactly the vertices that are added to $Z$ during the
%     first iteration of the fixed-point calculation and are never removed. The existence of such a set is apparent from the fact that, each vertex $v \in \Wo$ that has a non-minimum rank, is in the set because 
%     of the vertices with smaller ranks, i.e. the vertices that got included to $Z$ prior to $v$. This requires a set of vertices that were added to $Z$ prior to all the others. 
% %     
%     Additionally, from \eqref{eq:fp-odd} we gather the observation that all $v \in M$ have odd priorities.
%     \begin{proposition}\label{prop:cycle-through-M}
%         All cycles in $\Sc^{\mathcal{G}^\ell}$ that pass through a vertex in $M$ are \Odd winning.% (i.e. the largest priority in the cycle is odd).
%     \end{proposition}
% 
%     To see why Prop.~\ref{prop:cycle-through-M} holds, we need to make an observation. 
%     For an even $m\leq l$, let $Y_m^\1$ denote the value of $Y_m$ after the first ever iteration over it is completed, during the computation of \eqref{eq:fp-odd}.
%     I.e. $Y_m^\1 = Y^{0,0,\ldots ,0, 1} = $
%     $$\nu X_{m-1}\ldots \mu Y_2 \nu X_1. \bigcap_{j \in \ev{m+2}{l}} \B_j[\emptyset, V] \cap \B_m[\emptyset, X_{m-1}] \cap \bigcap_{j \in \ev{2}{m-2}} \B_j[Y_j, X_{j-1}].$$
%     In the first term $\B_j$ takes $\emptyset$ and $V$ as arguments. This is due to all $Y_{j}, X_{j-1}$ variables for $j \leq m$ having the values they are initialized with. Observe that when the fixed-point above is calculated, all $X_{j-1}, Y_j$ values for $j < m$ will saturate at the same value,
%     which is the final result of the computation. That is, 
%     \begin{equation}\label{eq:Ym1}
%     Y_m^\1 = \bigcap_{j \in \ev{m+2}{l}} \B_j[\emptyset, V] \cap \B_m[\emptyset, Y_m^\1] \cap \bigcap_{j \in \ev{2}{m-2}} \B_j[Y_m^\1 Y_m^\1].
%     \end{equation}
%     If we go through the definition of $\B_j$ we see that: the first term of this formula adds or deletes $v \in C_j$ with $j > m$. It adds all the ones with odd $j$  and removes all the ones with even $j$.%is equal to $\bigcup_{j \in \ev{m+2}{l}}C_{j-1} \cup \bigcup_{j \in [1, m+1]}C_j$. That is, the first term eliminates all $v \in C_j$ with even $j>m$ from $Y_m^\1$ and add all $C_j$ with odd $j>m$.
%      The last term adds and removes $v \in C_j$ for $j \leq m-2$. It adds the ones in $\Cpre_\Odd(Y_m^\1)$ and removes the ones that are not. The middle term eliminates $C_m$ and all $v \in C_j \cap \neg \Npre(\emptyset, Y_m^\1)$ for $j < m$, and adds $v \in C_{m-1} \cap \Npre(\emptyset, Y_m^\1)$.
%     If we go through the definition of $\Npre$, we see that $\Npre(\emptyset, Y_m^\1) = \Cpre_\Odd(Y_m^\1) \cap (V_\Even \cup \Lpre^\forall(Y_m^\1))$.
%     This gives us the observation,
%     \begin{equation}\label{eq:obs}
%         \text{If  } v \in Y_m^\1 \text{ then either } \chi(v)>m\text{ and is odd, or }  v\in \Npre(\emptyset, Y_m^\1).
%     \end{equation}
%     %$ Y_m^\1$ consists of $v$ with either odd $\chi(v)>m$, or in $\Npre(\emptyset, Y_m^\1)$.
% 
%     Now observe that, for every $v \in M$, $v \in Y_m^\1$ for each even $m \leq l$. In particular, $ v \in Y_n^\1$ where $n$ is the even number for which $\chi(v) = n-1$.
%     It follows that $ v \in \B_n[\emptyset, Y_n^\1]$. Then, $v \in \Cpre_\Odd(Y_n^\1) \cap (V_\Even \cup \Lpre^\forall(Y_n^\1))$.
%     Since all live outgoing edges of $v$ are in $Y_n^\1$, for all $(v,w)$ in $\Sc^{\mathcal{G}^\ell}$, $w \in Y_n^\1$.
% 
%     By our previous observation $w$ either has an odd priority larger than $n$, or is in $ \Cpre_\Odd(Y_n^\1) \cap (V_\Even \cup \Lpre^\forall(Y_n^\1))$.
%     If $\chi(w)>n$ is odd, then $w \in Y^\1_{\chi(w)+1}$, and we repeat the same argument to conclude the highest priority seen is always odd.
% 
%     Before we present the third proposition, we need a lemma and some observations obtained from formula ~\eqref{eq:fp-odd}.
%     \begin{observation}\label{obs:v-Even-Odd-inequalities} 
%         \begin{align*}
%             &\quad \quad \quad \quad  \quad \quad \text{if } v \in V_\Even, \quad \quad \forall(v, w)\in E, \rank{v}\geq_{l+1-\chi(v)} \rank{w}\\
%             &\quad \quad \quad  \quad \quad \quad  \text{if } v \in V_\Odd, \quad \quad \exists(v, w)\in E, \rank{v}\geq_{l+1-\chi(v)} \rank{w}
%         \end{align*}
%         where $\rank{v} \geq_b \rank{w}$ denotes the $\geq$ relation in the lexicographic ordering, restricted to the first b elements of the tuples $\rank{v}$ and $\rank{w}$. If $\chi(v)$ is odd, the inequalities are strict. 
%     \end{observation}
%     We call a play $\pi = v_1 v_2 \ldots$ in $\Sc^{\mathcal{G}^\ell}$ \emph{minimal} if for all $v_i \in V_\Odd$, $v_{i+1}$ is the minimum ranked successor of $v_i$. We call a cycle minimal, if it is a section of a minimal play.
%     \begin{lemma}\label{lemma:minimalplayOddwinning}
%         Every minimal play is \Odd winning.
%     \end{lemma}
%     This lemma follows from Obs. \ref{obs:v-Even-Odd-inequalities}. A minimal play only sees minimal cycles. Let $\delta = w_1 w_2 \ldots w_1$ be such a cycle. 
%     $\delta$ cannot be an \Even winning cycle: Assume $b := max\{ \chi(w) \mid w \in \delta\} $ is even. Let $w_i\in \delta$ have priority $b$. Then by Obs. \ref{obs:v-Even-Odd-inequalities}, $\rank{w_i} >_{l+1-b} \rank{w_{i+1}} \geq_{l+1 - \chi(w_{i+1})} \ldots \geq_{l+1-\chi(w_{i-1})} \rank{w_i}$. Since for all $w_j \in \delta$, $\chi(w_{j})\leq b$, the previous inequality yields $\rank{w_i} >_{l+1-b} \rank{w_i}$, which is a contradiction.
% 
%     %The last proposition states that all $\pi$ that starts in \Wo and is compliant with $\Sc^{\mathcal{G}^\ell}$, visits $M$ infinitely often. 
%     \begin{proposition}\label{prop:minimal-play-visits-M}
%         Any minimal cycle in $\Sc^{\mathcal{G}^\ell}$ visits $M$.
%     \end{proposition}
%     Let $\delta = w_1 w_2 \ldots w_1$ be a minimal cycle and $w_k$ it vertex with a maximum prioirty. We will show that $w_k \in M$. By Lemma.~\ref{lemma:minimalplayOddwinning} we know $\chi(w_k)$ is odd. Furthermore, we have observed in \ref{eq:obs} that $w_k \in Y_m^\1$ for all $m > \chi(w_k)$. 
%     If we can show that $w_k \in Y_m^\1$ also for $m < \chi(w_k)$, then clearly $w_k \in M$. We will now show this. 
%     Assume to the contrary that $w_k \not \in M$ and let $j$ be the largest non-trivial index of $\rank{w_k}$. 
%     That is $j < l$ is the largest even integer such that $w_k \not \in Y_j^\1$. Let $t$ be the value of this index, i.e. $w_k \in Y_j^{0,\ldots, 0,t} \setminus Y_j^{0,\ldots, 0,{t-1}}$. 
%     Let us denote $Y_j^{0, \ldots, 0, t}$ by $Y_j^\te$ for short. 
%     
%     Since $\delta$ is minimal, Obs.~\ref{obs:v-Even-Odd-inequalities} gives us $\rank{w_i} \geq_{l+1 - \chi(w_i)} \rank{w_{i+1}}$ for all $w_i \in \delta$. Since $\chi(w_i) \leq \chi(w_k)$ for all $i$ and $\chi(w_k) < j$; $\rank{w_i} \geq_{l+1-j} \rank{w_{i+1}}$ for all $w_i \in \delta$. 
%     Since $\delta$ is a cycle, thise implies $\rank{w} =_{l+1-j} \rank{w'}$ for all $w, w'\in \delta$. This gives us that, for all $w\in \delta$, $w \in Y_j^\te \setminus Y_j^{\mathbf{t-1}}$.
% 
%     We can follow the same steps in equation~\eqref{eq:Ym1} to observe that 
%   $\forall (v,w)$ in $\Sc^{\mathcal{G}^\ell}$, $w \in \Npre(Y^{\mathbf{t-1}}_j, Y^\te_j) = \Cpre_\Odd(Y^\te_j) \cap (V_\Even \cup \Lpre^\forall(Y^\te_j) \cup Pre^\exists_\Odd(Y^\mathbf{t-1}_j))$.
%   If $w \in Pre^\exists_\Odd(Y_j^{\mathbf{t-1}})$, since $\delta$ is a minimal cycle, $\delta$ will have an element from $Y_j^{\mathbf{t-1}}$. However, this contradicts our observation that $\delta$ lies in $Y_j^\te \setminus Y_j^{\mathbf{t-1}}$.
%   On the other hand if non of the $w \in \delta $ lie in $ Pre^\exists_\Odd(Y_j^{\mathbf{t-1}})$ this implies that they all get into the formula due to reaching other nodes in $Y_j^\te \setminus Y_j^\mathbf{t-1}$. This is not possible since a node in $\delta$ has to be added to $Y_j^\te \setminus Y_j^\mathbf{t-1}$ as the first node and thus, have to have a successor in $Y_j^\mathbf{t-1}$. % while $Y_j^\te \setminus Y_j^{\mathbf{t-1}}$ is empty.
%     Therefore, $w_k\in M$.
% 
% %    We are now ready to prove the main theorem.
% %    \begin{proof}[Proof of Thm. \ref{thm:mainresult}]
% %         Let $\pi= v_0v_1\ldots$ be a play compliant with $\Sc^{\mathcal{G}^\ell}$ and $v_0 \in \Wo$. %Since all plays compliant with \Odd strategy templates obey the fairness condition, we only need to show that the maximum prioirty in $\pi$ is odd.
% %         Since $\pi$ is compliant with an \Odd strategy template, it obeys the fairness condition. 
% %         The fact that $\pi$ visits minimal cycles infinitely often follows from the fact that whenever a $v \in V_\Odd$ is seen infinitely often in $\pi$, $(v, v_{min})$ should be seen infinitely often as well.
% %         This gives by induction that, a minimal cycle that passes through $v$ should be visited infinitely often. Due to Prop.~\ref{prop:minimal-play-visits-M}, we know that therefore, $\pi$ visits $M$ infinitely often. Since $M$ is finite, $\pi$ visits a $x \in M$ infinitely often. Thus, a tail of $\pi$ can be seen as consecutive cycles over $x$. Since by Prop.~\ref{prop:cycle-through-M}, all cycles that pass through $M$ are \Odd winning, $\pi$ is \Odd winning.  
% %     \end{proof}

\subsection{Zielonka's Algorithm for \Odd-Fair Parity Games}\label{app:zielonka-proof}

% This section is intended to provide a detailed proof of Thm.~\ref{thm:solvebb}.
% 
% We refer the reader to Sec.~\ref{sec:zielonka} for the introduction and motivation of the proofs. We will not repeat the definitions of safe reachability sets and partial strategy templates in this section. Once more we refer the reader to Sec.~\ref{sec:zielonka} for these preliminaries. 
% That said,  we will not follow the same lay-out on the proofs as in Sec.~\ref{sec:zielonka}. Moreover, we will present new definitions and lemmas liberally. So this section should be perceived as \emph{somehow} stand-alone, with the exception of the aforementioned dependencies. 
% The proof roughly follows the foot steps of ~\cite{Kuesters2002}.% We will cite some lemmas from this work and leave them unproven if the proof is unaffected by the fairness condition. 

This section provides a detailed proof of Thm.~\ref{thm:solvebb}.
% 
However, we will not follow the lay-out given for this proof in Sec.~\ref{sec:zielonka} but rather follow the foot steps of the correctness proof of the \enquote{normal} Zielonka's algorithm from  ~\cite{Kuesters2002}. Hence, this section should be perceived as stand-alone, with the exception of the definitions of safe reachability sets and partial strategy templates, which can be found in Sec.~\ref{sec:zielonka}. While we do not follow the same lay-out, the motivation and intuition given for the proof in Sec.~\ref{sec:zielonka} still carries over to this section.


% \subsubsection{Preliminaries} 
% 
%We start by extending the strategy template definition given in section 4 to \Even strategies (just \Even and \Odd are swapped in the definition). Note that, since \Even nodes do not have live outgoing edges, for all $v \in V'_\Odd$, $E(v) \subseteq E'$ and for all $v \in V'_\Even$, $|E'(v)| = 1$. Consequently, a positional \Even strategy $\rho$ is equal to the unique strategy compliant with the strategy template $(V', V'_\Even, V'_\Odd, E')$ for which $E'(v) = \rho(v)$ for all $v \in V'_\Even$. We use a positional \Even strategy $\rho$ to define a strategy template for \Even.
%To distinguish strategy templates defining \Odd strategies from the ones defining \Even strategies, we denote \Odd strategy templates by $\mathcal{S}^\Odd$ and \Even ones with $\mathcal{S}^\Even$. 
% 
%Furthermore, we obtain a partial strategy template by taking a strategy template and a subset $V'$ of its vertex set, and removing all the outgoing edges of all vertices in $V'$. \IS{?}
%Let $a \in \{\Even, \Odd\}$ and $\mathcal{S}^a = (V', V'_\Even, V'_\Odd, E')$ be an $a-$strategy template. We obtain a partial strategy template as follows: Let $V'' \subseteq V'$ be such that, for all $v \in V''$, $v$ does not lay on a cycle in $E'$. Now for all $v \in V''$, remove all outgoing edges of $v$ from $E'$ to obtain $E''$. Then $(V', V'_\Even, V'_\Odd, E'')$ is a partial strategy template for $a$. We denote a partial strategy template of $a$ by $\mathcal{P}^a$.

\subsubsection{Preliminaries} 
% This section proves important observations w.r.t.\ the \bb paradises in \odd-fair parity games. In order to do so, we restate the definitions of an $\bb-$trap (Def.~\ref{def:atrap}) and a  \bb paradise (Def.~\ref{}) from Sec.~\ref{sec:zielonka}. We also estate three lemmas (Lem.~\ref{app-lem:Kuesters6.2}, \ref{app-lem:Kuesters6.3} and \ref{app-lem:Kuesters6.4}) exactly as they appear in \cite{Kuesters2002} and omit the proofs since the statements of these lemmas are only concerned with the properties of the subsets of $V$, and are therefore unaffected by the fairness condition.
We emphasize again that we assume the underlying game graph of the fair parity game $\mathcal{G}^\ell$ to be deadend-free.

\smallskip
\noindent\textbf{Subgames.}
For some $U \subseteq V$ we denote by $E \mid_U = \{(v, w) \in E \mid v, w \in U \}$ and by $\chi\mid_U$ we denote the restriction of the function $\chi$ to the domain $U$. %With this, we formally define subgames as follows.

%We call a $\mathcal{P} = (V', E')$ a \emph{partial strategy template in $\mathcal{G}^\ell$} if it is a strategy template, with the exception of some dead-ends. That is, $\mathcal{P}$ is a partial \Odd (\Even) strategy template if for each $v \in V'$, either $E'(v) = \emptyset$ or $v$ satisfies~\ref{item:Oddstrtemprules} (\ref{item:Evenstrtemprules}).

\begin{definition}[Subgames]
    Let $U \subseteq V$. The subgraph of $\mathcal{G}^\ell$ induced by $U$ is shown as $\mathcal{G}^\ell[U]$ and is the restriction of the game graph to $U$, i.e.
    $\mathcal{G}^\ell[U] = \ltup{ \langle U, \Ve \cap U, \Vo \cap U, E|_U, \chi|_U\rangle, E^\ell|_U} $. $\mathcal{G}^\ell[U]$ is a subgame of $\mathcal{G}^\ell$ if and only if $\mathcal{G}^\ell[U]$ is deadend-free.
\end{definition}

\begin{lemma}[\cite{Kuesters2002}, Lemma 6.2]\label{app-lem:Kuesters6.2}
    If $U, U' \subseteq V$ where $\mathcal{G}^\ell[U]$ is a subgame of $\mathcal{G}^\ell$ and $(\mathcal{G}^\ell[U])[U']$ is a subgame of $\mathcal{G}^\ell[U]$, then $\mathcal{G}^\ell[U']$ is a subgame of $\mathcal{G}^\ell$.
\end{lemma}

The above lemma (as well as the following two lemmas  \ref{app-lem:Kuesters6.3} and \ref{app-lem:Kuesters6.4}) are restated exactly as they appear in \cite{Kuesters2002}. We omit their proofs since the statements of these lemmas are only concerned with the properties of the subsets of $V$, and are therefore unaffected by the fairness condition.

%\begin{definition}[Safe Reachability Game] A safe reachability game on game $\mathcal{G}$ is given by the following LTL formula
%    $$ \alpha^{sr} = \square \,\, p_s \,\wedge \,\diamond \,\,p_r $$
%    where $S$ and $R$ are two subsets of $V$, called the safety and reachabilty sets respectively. For $v \in V$, $v \models p_s$ iff $v\in S$ and $v \models p_r$ iff $v\in R$. 
%    $\SafeReach_\Even(S, R, \mathcal{G})$ denotes the winning region of \Even in this game and  $\SafeReach_\Odd(S, R, \mathcal{G})$ denotes the winning region of \Odd.
%\end{definition}
%Given a game graph $G = (V, V_\Even, V_\Odd, E)$, reachability set $R\subseteq V$ and safety set $S \subseteq V$, the following fixed-point formula computes the winning region for player $\bb$ where $\bb \in \{\Even, \Odd\}$ \cite{banerjee2022fast}:
%$$ \SafeReach_\bb(S, R, \mathcal{G}) = \mu X. (p_s \wedge (p_r \vee \\Cpre_\bb(X))) $$
%Note that there is an obvious positional strategy for player $\bb$, which makes it progress towards $R$ while staying in $S$ from each vertex in $\SafeReach_\bb(S, R, \mathcal{G}) \setminus R$. This
%strategy can be obtained by assigning a ranking (where
%\newline $rank: \SafeReach_\bb(S, R, \mathcal{G}) \to \mathbb{N}^+$ to each node $ v \in \SafeReach_\bb(S, R, \mathcal{G})$ that represents the minimum iteration of the least fixed-point variable $X$, in which they got added into the formula. That is, 
%$rank(v) = k $ iff $ v \in X^k \setminus X^{k-1}$. 
%Then, the strategy of player $\bb$ is simply to take the minimum ranked successor from each $v \in V_\bb \cap \SafeReach_\bb(S, R, \mathcal{G}) \setminus R$. Note that $rank(v) = 1$ iff $v\in R$.
%This strategy defines a partial strategy template $\mathcal{P}^\bb = \tup{V', E' }$ where $V' = SafeReach_\bb(S, R, \mathcal{G})$ and $v \in R$ are dead-ends. For all $v \in V'_\bb$, $E'(v)$ only contains a minimum ranked successor of $v$ and for all $v \in V'_{\nb}$, $E'(v) = E(v)$. Note that this partial strategy template gives out a minimum-rank strategy for player \bb. Additionally, $\mathcal{P}^\bb$ contains no cycles, since in a play $\pi = v_0 v_1 \ldots v_k$ compliant with $\mathcal{P}^\bb$, where $v_0 \in SafeReach_\bb(S,R,\mathcal{G}^\ell) \setminus R$ and $v_k \in R$; $rank(v_{i+1})< rank(v_i)$ for all $i \in [0,k-1]$ in order for the play to be able to eventually reach $R$. This is the case for all strategies of $\nb$, whereas $\bb$ nodes only have one outgoing edge. This is only possible in case $\mathcal{P}^\bb$ is cycle-free.

%On a fair game graph $\mathcal{G}^\ell := \ltup{\mathcal{G}, E^\ell}$, the safe reachability set of \Odd is calculated by the same formula as in regular games; hewever, the safe reachability set of \Even requires a 2-nested fixed-point formula:
%\begin{align*}
%& SafeReach^l_\Odd(S, R, \mathcal{G}^\ell) = \mu X. (p_s \wedge (p_r \vee \\Cpre_\Odd(X)))\\
%& SafeReach^l_\Even(S, R, \mathcal{G}^\ell) = \nu Y \mu X. (p_s  \wedge (p_r \vee \Apre(Y,X)))\\
%\end{align*}
%The change is due to the increased power of \Even in \Odd-fair games. Namely, \Even can force a $v\in V_\Odd$ into a region $R$ in the next step not only if all the outgoing edges of $v$ land in $R$, but also if $v$ has a live edge to $R$ and the \Odd strategy has a cycle passing through $v$.
%$SafeReach^l_\Even(S, R, \mathcal{G}^\ell)$ does the following with it's nested fixed-point calculation:
%It begins by setting $Y^0:= V$ and $X^{0,0} := \emptyset$. Then calculates all nodes that can be forced to reach $R$, assuming all $v \in V^\ell$ have a cycle passing through them (this is because $\\Apre(V,X)$ adds $v \in V^\ell$ to $X^{0, i}$ if it has a live outgoing edge to $X^{0, i-1}$).
%The saturation value of $X^{0,k}$ is set to $Y^1$. $Y^1$ contains elements that can reach $R$, if all $V^\ell$ had lied on a cycle, and therefore had to eventually take their live edges. In the next iteration, the formula calculates the set of nodes that can be forced by \Even to reach $R$, assuming that for all $v \in V^\ell$, if $v$ has an outgoing edge to
%$V \setminus Y^1$, it takes this outgoing edge and avoids $R$ (so it wins by a 1-step strategy from $v$ and therefore, $v$ does not lie on a cycle the \Odd strategy), and if it does not have such an edge, then we assume it lies on a cycle and it is included in $X^i$ if it has a live outgoing edge leading to $X^{i-1}$.
%The saturation value of $X^{1, k}$ is set to $Y^2$ and the calculation is repeated until the saturation of the $Y$ variable. Once $Y$ has saturated, all nodes that are included in $Y$ are those that can be forced by \Even to reach $R$. If $v \in V^\ell \cap Y$ then 
% all outgoing edges of $v$ lead to $Y$. \IS{Anne: Maybe a better explanation instead of the last sentence?}

% In \Odd-fair Zielonka algorithm, we will use the safe reachability sets for \Even and \Odd denoted by $SafeReach^f_\Even$ and $SafeReach^f_\Odd$. The \Odd safe reachability set will remain as before, i.e. $SafeReach^f_\Odd(S, R, \mathcal{G}^\ell) := SafeReach^l_\Odd(S, R, \mathcal{G}^\ell)$.
%However, we will use a simpler variant of the \Even safe reachability set. In this variant intuitively, we act as if all the $v \in V^\ell$ have a cycle passing through them, and add them to the \Even safe reachability set if there is a live edge $(v,w)$ that has positive progress towards $R$.
% \begin{equation}\label{app-eq:SafeReachEven}
% SafeReach^f_\Even(S, R, \mathcal{G}^\ell) = \mu X. (p_s  \wedge (p_r \vee \Cpre_\Even(X) \vee Lpre^\exists(X)))
%% \end{equation}
% That is, $v \in X^i$ if $\exists (v,w)$ with $w \in X^{i-1}$. Using this fixed-point formula to calculate the \Even reachability sets allows us to calculate an over-approximation, say $W$, of the \Even winning region $\We$ at the end of the algorithm. The complement of $W$ is an under approximation of $\Wo$ that contains no live edges leading to $W$.
% Then we show that $\Wo$ is exactly $SafeReach^f_\Odd(V, V \setminus W, \mathcal{G}^\ell)$ and $\We$ is the complement of this set. 
%\IS{some connecting sentence}
 %Let's give some more preliminaries before moving to the proof.
 
\smallskip
\noindent\textbf{\bb-Trap.} We restate the definition of an $\bb$-trap from Sec.~\ref{sec:zielonka}. and subsequently show important observations w.r.t.\ $\bb$-traps in \Odd-fair parity games.
 
\begin{definition}[\bb-trap]\label{def:atrap}
An $\bb$-trap is a subset $T \subseteq V$ for $\bb \in \{\Even, \Odd\}$ such that,
\begin{align*} &\forall v \in T \cap V_{\nb}, \quad \exists (v, w)\in E \text{ with } w \in T, \\
   & \forall v \in T \cap V_{\bb}, \quad \, (v, w) \in E \implies w \in T.
 \end{align*}
\end{definition}

%So any strategy $\rho$ that is compliant with a strategy template $\mathcal{S}^\Odd$, for which 

% Now we will give some lemmas, the proofs of which can be found in \cite{Kuesters2002} and the fairness condition do not change the proofs. For the following, let $\mathcal{G}^\ell = \tup{\mathcal{G}, E^\ell}$ be an \Odd-Fair Parity game. 

\begin{lemma}[\cite{Kuesters2002} Lemma 6.3]\label{app-lem:Kuesters6.3}
    \begin{enumerate}
        \item For every $\bb$-trap $U$ in $\mathcal{G}^\ell$, $\mathcal{G}^\ell[U]$ is a subgame.
        \item If $X$ is an $\bb$-trap in $\mathcal{G}^\ell$ and $Y\subseteq X$ is an $\bb$-trap in  $\mathcal{G}^\ell[X]$, then $Y$ is an $\bb$-trap in $\mathcal{G}^\ell$.
    \end{enumerate}
\end{lemma}

%\noindent For the following, let $U \subseteq V$  such that $\mathcal{G}^\ell[U]$  is a subgame of $\mathcal{G}^\ell$.
\begin{lemma}[\cite{Kuesters2002}, Lemma 6.4 -- Sec.~\ref{sec:zielonka:correct} Obs.~\ref{it:obs5}]\label{app-lem:Kuesters6.4}
 The set $U \setminus \SafeReach^f_\bb(U, R, \mathcal{G}^\ell)$ is an $\bb$-trap in $U$.
\end{lemma}
\begin{lemma}\label{app-lem:safereacheven-noliveedges} 
    Let $W = U \setminus \SafeReach^f_\Even(U, R, \mathcal{G}^\ell)$. There exists no $(v,w) \in E^\ell$ with $v \in W$ and $w \in \SafeReach^f_\Even(U, R, \mathcal{G}^\ell)$.
\end{lemma}
\begin{proof}
%(1.) For $a= \Odd$, the proof is the same as the original proof. Let $a= \Even$. Assume $V \setminus \SafeReach^f_\Even(U, R, \mathcal{G}^\ell)$ is not an $\Even-$trap. Then there is either a $v \in V_\Even|_U \setminus \SafeReach^f_\Even(U, R, \mathcal{G}^\ell) $ such that for all $(v,w) \in E|_U$, $w \in \SafeReach^f_\Even(U, R, \mathcal{G}^\ell)$;
%or, there exists a $v \in V_\Odd|_U \setminus \SafeReach^f_\Even(U, R, \mathcal{G}^\ell) $ such that there exists a $(v, w) \in E|_U$ like that. For both of these cases, $v \models (p_U \vee \Cpre_\Even(\mathbf{X}))$ where $\mathbf{X} = \SafeReach^f_\Even(U, R, \mathcal{G}^\ell)$, as it is the value it gets in its saturation. But this requires $v$ to be in $\SafeReach^f_\Even(U, R, \mathcal{G}^\ell)$ as well, through the definition of $\Apre$ and the fixed-point calculation. Thus, no such $v$ exists and the set is an $a-$trap.\\
%(2.) For $a= \Even$, the proof  is the same as the original proof. Let $a= \Odd$. Assume that the given set is not an \Odd-trap. Then either there exists a $v \in V_\Odd|_{U} \cap \SafeReach^f_{Even}(U, R, \mathcal{G}^\ell)$ such that $\exists (v,w) \in E|_{U}$ with $w \in U \setminus \SafeReach^f_\Even(U, R, \mathcal{G}^\ell)$, or there exists a $v \in V_\Even |_U \cap \SafeReach^f_{Even}(U, R, \mathcal{G}^\ell)$ such that all $(v,w) \in E|_{U}$ satisfy that. For both of these cases, $w \not \in \mathbf{X}, \mathbf{Y} = ^f_\Even(U, R, \mathcal{G}^\ell)$, for the saturation values of the fixed-point variables. Thus, $v \not \models \Apre(\mathbf{Y}, \mathbf{X})$. I.e. $v \not \in \SafeReach^f_\Even(U, R, \mathcal{G}^\ell)$.\\
A node $v \in U \setminus \SafeReach^f_\bb(U, R, \mathcal{G}^\ell) \cap V_\bb$ cannot have an edge that leads to $\SafeReach^f_\bb(U, R, \mathcal{G}^\ell)$, since then $v$ itself must be in this set.
Similarly a node  $v \in U \setminus \SafeReach^f_\bb(U, R, \mathcal{G}^\ell) \cap V_{\nb}$ must have an edge that leads to $ U \setminus \SafeReach^f_\bb(U, R, \mathcal{G}^\ell)$, or else $v$ would be in  $\SafeReach^f_\bb(U, R, \mathcal{G}^\ell)$.
\end{proof}

\begin{lemma}\label{app-lem:SafeReachOdd_of_an_even_trap_is_an_even_trap}
    If $R$ is an \Even-trap in $U$, then so is $\SafeReach^f_{\Odd}(U, R, \mathcal{G}^\ell)$.
\end{lemma}
\begin{proof}
    This is easy to observe from the definition of a partial strategy template $sr_\Odd$ on $\SafeReach^f_\Odd(U, R, \mathcal{G}^\ell)$.
    All $(v, w) \in E$ with $v \in V_\Even \cap \SafeReach^f_\Odd(U, R, \mathcal{G}^\ell)\setminus R$, are in $sr_\Odd$. That is, $w \in \SafeReach^f_\Odd(U, R, \mathcal{G}^\ell)$. For all $v \in V_\Even \cap R$, all $(v,w) \in E\subseteq U \times U$ are in $R$ since $R$ is an \Even-trap in $U$. 
    Thus for all \Even nodes in $\SafeReach^f_\Odd(U, R, \mathcal{G}^\ell)$, all their successors in $U$ are in the set again. 
    We can similarly observe that for all $v \in V_\Odd \cap \SafeReach^f_\Odd(U, R, \mathcal{G}^\ell)$ they have at least one successor in the set. 
    Thus this set is an \Even-trap in $U$. 
\end{proof}

\smallskip
\noindent\textbf{\bb-Paradise.} We restate the definition of an $\bb$-paradise from Sec.~\ref{sec:zielonka} and subsequently show important observations w.r.t.\ $\bb-$paradises in \Odd-fair parity games.

\begin{definition}[$\bb$-paradise]\
An $\bb$-paradise of an \Odd-fair parity game $\mathcal{G}^\ell$ is a region $P \subseteq V$ from which player $\nb$ cannot escape (i.e. $P$ is a $\nb$-trap) and player $\bb$ has a strategy to win from all $v\in P$. As we have proven in section 5, this implies that there exists a strategy template $\mathcal{S}^\bb$ with the vertex set $P$ such that all player $\bb$ strategies compliant with $\mathcal{S}^\bb$ are winning for player $\bb$.

Formally $P \subseteq V$  is an $\bb$-paradise if:
\begin{itemize}
\item $P$ is a $\neg \bb$-trap and, 
\item There exists a winning \bb strategy template $\mathcal{S}^\bb = \ltup{P, E'}$ on $\mathcal{G}^\ell$.
\end{itemize}
\end{definition}
Note that if $P$ is an $\bb$-paradise, and play $\pi$ starting in $P$ and is compliant with $\mathcal{S}^a$, stays in $P$ and is won by \bb.

\begin{lemma}[Sec.~\ref{sec:zielonka:correct} Obs.~\ref{it:obs4}]\label{app-lem:safe-reach-Odd-paradise}
    If $R\subseteq V$ is an \Odd-paradise in $\mathcal{G}^\ell$, then $\SafeReach_\Odd^f(V, R, \mathcal{G}^\ell)$ is also an \Odd-paradise in $\mathcal{G}^\ell$.
\end{lemma}
\begin{proof}
Due to Lem.~\ref{app-lem:SafeReachOdd_of_an_even_trap_is_an_even_trap}, $\SafeReach_\Odd^f(V, R, \mathcal{G}^\ell)$ is an \Even-trap in $V$.
The winning \Odd strategy template on it is just a combination of the winning \Odd strategy template $\mathcal{S}$ on $R$ and the partial \Odd strategy template $sr_\Odd$ on $\SafeReach^f_\Odd(V, R, \mathcal{G}^\ell)$, on which nodes in $R$ are dead-ends and  
all $v \in \SafeReach^f_\Odd(V, R, \mathcal{G}^\ell) \setminus R$ are guaranteed to reach $R$ in finitely many steps. %Remember that $sr_\Odd$ is acyclic. 
Let $\e$ be the combination of edges in $sr_\Odd$ and $\mathcal{S}$. 
Since $R$ is an \Even-trap in $V$, all outgoing edges of \Even nodes in $R$ stay in $R$. All outgoing edges of \Even nodes in 
$\SafeReach^f_\Odd(V, R, \mathcal{G}^\ell) \setminus R$ are in $sr_\Odd$. Therefore all outgoing edges of \Even nodes in $\SafeReach^f_\Odd(V, R, \mathcal{G}^\ell)$ are in $\e$.
It's easy to see $\e$ introduces no new cycles to $sr_\Odd \cup \mathcal{S}$. Therefore $\mathcal{S}' = 
(\SafeReach^f_\Odd(V, R, \mathcal{G}^\ell), \e)$ is an \Odd strategy template in $\mathcal{G}^\ell$.
$\mathcal{S}'$ is winning because any play starting in $\SafeReach^f_\Odd(V, R, \mathcal{G}^\ell)\setminus R$ reaches $R$ in finitely many steps and from there on stays in $R$. 
Since from that point on $\mathcal{S}'$ collapses to $\mathcal{S}$, the game is won by \Odd. 

\end{proof}
\begin{corollary}\label{app-cor:determinacy}
    For an \Odd-fair parity game $\mathcal{G}^\ell$, $V$ is partitoned into an \Even-paradise and an \Odd-paradise. 
\end{corollary}
The corollary follows from the fixed-point equations~\eqref{eq:fp-even} and~\eqref{eq:fp-odd}. Winning region of player $\bb$ is by definition an $\bb$-paradise. \We is the \Even-paradise with the strategy template defined by the positional strategy acquired from the fixed-point formula in~\eqref{eq:fp-even}. The calculation of the positional strategy is closely related to the ranking function and strategy template computation in Sec.~\ref{sec:strat-templates}, and a brief introduction of the calculation can be found in \cite{banerjee2022fast}.
$\Wo = V \setminus \We$ is the \Odd-paradise. The calculation of the strategy template for \Odd is given in Section 5. 


\subsubsection{Computing Winning Regions $\mathcal{W}_\bb$}
Now we will give a construction to calculate $\Wo$ and $\We$ in $\mathcal{G}^\ell$. The construction corresponds to the \Odd-fair Zielonka's algorithm given in Alg.~\ref{algo:fair-zielonka-bb}.
We will give the construction in two parts. First we will take an \Odd-fair parity game $\mathcal{G}^\ell$ and an \emph{odd} integer $n$ where $n$ is an upper bound on the priorities seen in the vertex set of $\mathcal{G}^\ell$. Then we will show how to obtain $\Wo$ and $\We$ in $\mathcal{G}^\ell$ in the existence of a procedure
that can do the same on a subgame $\mathcal{G}^\ell[X]$ of $\mathcal{G}^\ell$ where $n-1$ is an upperbound of the priorities seen in $\mathcal{G}^\ell[X]$. 
In the second part we will show the same for $\mathcal{G}^\ell$ with an \emph{even} $n$. The combination of these two procedures with a base case, will give the recursive algorithm we need to solve \Odd-fair parity games. 
We will count on strategy templates in the proof of both parts. However, the second part of the algorithm follows roughly the same principles in Zielonka's original algorithm, whereas the
 the first part requires an essential change in reasoning, due to the adoption of $\SafeReach^f_\Even$. Even though the reasoning required to prove the first part is fairly different than Zielonka's original algorithm, 
a computationally cheap addition to the original algorithm is sufficient to get the correct computation for the \Odd-fair variant. Surprisingly, the trick is cheap enough not to alter the complexity of the original algorithm at all!

\smallskip
\noindent\textbf{Subsets and Sequences.}
Let $n$ be an upper bound on the priorities seen in $V$. If $n$ is \Even, set $\bb:=\Even$, otherwise $\bb:=\Odd$.
Further, we construct a decreasing series of subsets of $V$, $\{X_\bb^i\}_{i\in \mathbb{N}}$
by assigning the following sets (see Fig.~\ref{fig:kuesters-figure-extended} for an illustration): % V =: &X_\Odd^0 \supsetneq X_\Odd^1 \supsetneq \ldots \supsetneq  X_\Odd^k =  X_\Odd^{k+1}
\vspace{0.3cm}

\noindent Initially set $X^0_\nb = \emptyset$. For all $i \in \mathbb{N}$, set 
\begin{subequations}\label{equ:seriesZielonka}
    \begin{align*}
   &X_\bb^i := V \setminus X_\nb^i \quad \quad \quad &N^i:= \{v \in X^i_\bb \mid \chi(v) = n\}\\
       &Z^i:= X^i_\bb \setminus \SafeReach^f_\bb(X^i_\bb, N^i, \mathcal{G}^\ell) \quad &X^{i+1}_\nb :=  \SafeReach^f_\nb(V, X_\nb^{i} \cup Z_\nb^{i}, \mathcal{G}^\ell) % X_\Even^{i} \cup \SafeReach_\Even^f(X^{i}_\Odd, Z_\Even^{i}, \mathcal{G}^\ell) )%\text{\todo{IS: I know the equality is not completely justified. The first one is cheaper for an algorithm pov, whereas the second one is easier to justify that $X^i_\Odd$ is an \Even-trap.}} 
   \end{align*}
   \end{subequations}
where $Z_\nb^i$ is the \nb winning region in the subgame $\mathcal{G}^\ell[Z^i]$, assuming it is a subgame. 
First let's show that these sets are well-defined.
\begin{lemma}
The sets $X_\bb^i, X_\nb^i, N^i, Z^i, Z_\nb^i$ and $Z_\bb^i$ are well defined for all $i \in \mathbb{N}$.
\end{lemma}
\begin{proof} We will prove this by induction. For the base case $i = 0$, 
$X_\bb^0 = V$ is trivially an \nb-trap in $V$ and $\mathcal{G}^\ell[X^0_\bb]$ is trivially a subgame of $\mathcal{G}^\ell$. 
By Lem.~\ref{app-lem:Kuesters6.4}, $Z^0$ is an \bb-trap in $X^0_\bb$, and thus by Lem.~\ref{app-lem:Kuesters6.3}-1, $\mathcal{G}^\ell[Z^0]$ is a subgame of $\mathcal{G}^\ell$. 
Due to Corollary~\ref{app-cor:determinacy}, we know $\mathcal{G}^\ell[Z^0]$ is divided into an \bb-paradise and \nb-paradise. Therefore,  
$Z^0_\bb$ and $Z^0_\nb$ are also well-defined.  

By induction on $i$, we get by Lem.~\ref{app-lem:Kuesters6.4} that $X^i_\bb$ is an \nb-trap in $V$, and by Lem.~\ref{app-lem:Kuesters6.3}-1 $\mathcal{G}^\ell[X_\bb^i]$ is a subgame of $\mathcal{G}^\ell$. $Z^i$ is an \bb-trap in $\mathcal{G}^\ell[X^i_\bb]$, and thus by Lem.~\ref{app-lem:Kuesters6.2}, $\mathcal{G}^\ell[Z^i]$ is a subgame in $\mathcal{G}^\ell$.
Therefore $Z_\nb^i$ and $Z_\bb^i$ are well-defined.
\end{proof}
We also derived the following observations from the proof:

\begin{observation}[Sec.~\ref{sec:zielonka:correct} Obs.~\ref{it:obs1}]\label{app-obs:traps-subgames}
    $X^i_\nb$ is an \bb-trap, $X^i_\bb$, $Z^i$ and $Z_\bb^i$ are \nb-traps in $V$. $Z^i$ is in \nb-trap in $X_\bb$ and $Z_\nb^i, Z_\bb^i$ are \bb and \nb traps in $Z^i$, respectively.
    Therefore by Lem.~\ref{app-lem:Kuesters6.2}, $\mathcal{G}^\ell[Y]$ is a subgame of $\mathcal{G}^\ell$ with $Y$ being any of these sets. 
\end{observation}

\begin{lemma}[Sec.~\ref{sec:zielonka:correct} Obs.~\ref{it:obs2}]\label{app-lem:X_nb-equivalence}
    $X_\nb^{i} \cup \SafeReach_\nb^f(X^{i}_\bb, Z_\nb^{i}, \mathcal{G}^\ell) =  \SafeReach_\nb^f(V, X_\nb^{i} \cup Z_\nb^{i}, \mathcal{G}^\ell) $
\end{lemma}
\begin{proof}
   $ \mathbf{(\subseteq )}$ Trivially, $X_\nb^{i} \subseteq \SafeReach_\nb^f(V, X_\nb^{i} \cup Z_\nb^{i}, \mathcal{G}^\ell)$.
    Similarly a \\$v \in  \SafeReach_\nb^f(X^{i}_\bb, Z_\nb^{i}, \mathcal{G}^\ell)$, can be made by $\nb$ to reach $Z_\nb^i$ while staying in $X_\bb^i$. Then $v$ is trivially in the righthand side equation as well.
    
    \noindent $ \mathbf{(\supseteq )}$ 
    Let $v \in \SafeReach_\nb^f(V, X_\nb^{i} \cup Z_\nb^{i}, \mathcal{G}^\ell) \setminus X_\nb^i$.%, be in $X^j \setminus X^{j-1}$ where $X^j$ is the value of the $X$ variable after the $j^{th}$ iteration of the fixed-point computation from formula~\eqref{eq:\SafeReachEven}. that is $\rank{v} = j$ and $v \in X_\Even^{i} \cup Z_\Even^{i} \cup \Cpre_\Even(X^{j-1}) \cup \Lpre^\exists(X^{j-1})$.
     Since $v \in X_\bb^i$ and $X_\bb^i$ is an \nb-trap in $V$, if $v \in V_\bb$ it has one outgoing edge not leading to $X_\nb^i$ and 
    if $v \in V_\nb$, no outgoing edge of $v$ lead to $X_\nb^i$. That is, $v$ can either be made by \nb to reach $Z^i_\nb$ by staying in $X_\bb^i$ (i.e. it is in $\SafeReach^f_\nb(X_\bb^i, Z_\nb^i, \mathcal{G}^\ell)$),
    or $\bb = \Odd$ there exists a sequence of outgoing live edges that make $v$ reach $X_\nb^i$. This is not possible since there exists no live edges from $X_\Odd^i$ to $X_\Even^i$ due to Lem.~\ref{app-lem:safereacheven-noliveedges}.
\end{proof}
\begin{corollary}[Sec.~\ref{sec:zielonka:correct} Obs.~\ref{it:obs3}]\label{app-cor:increasing-decreasing-sequences}
    Due to Lem.~\ref{app-lem:X_nb-equivalence}, $\{X_\nb^{i}\}_{i\in \mathbb{N}}$ is an increasing sequence. Consequently, $\{X_\bb^{i}\}_{i\in \mathbb{N}}$ is a decreasing sequence. 
\end{corollary}
Since $V$ is finite, the corollary immediately implies that these sequences reach saturation value for some, and in fact the same, $k$. 

% \vspace{0.3cm}
\smallskip
\noindent\textbf{Part 1.}
We first assume an odd number $n$ is the maximum priority in $\mathcal{G}^\ell$.
% 
Cor.~\ref{app-cor:increasing-decreasing-sequences} gives that $\{X_\Odd^i\}_{i\in \mathbb{N}}$ is an increasing sequence and saturates at some index $k$.
Observe that $X_\Odd^k$ is the saturation value if and only if $Z_\Even^k = \emptyset$.
% 
The following proposition states that, \Odd safe reachability set of the saturation value $X_\Odd^k$ gives us \Wo.
\begin{proposition}\label{app-prop:n-odd}
    If $Z_\Even^k = \emptyset$, then $\SafeReach^f_\Odd(V, X^k_\Odd, \mathcal{G}^\ell)$ is an \Odd-paradise and $V \setminus \SafeReach^f_\Odd(V, X^k_\Odd, \mathcal{G}^\ell)$ is an \Even-paradise in $\mathcal{G}^\ell$.
\end{proposition}
We give the proof of Prop.~\ref{app-prop:n-odd} in three parts: First we prove $X^k_\Odd$ is an \Odd-paradise, then we show $\SafeReach^f_\Odd(V, X^k_\Odd, \mathcal{G}^\ell)$ is an \Odd-paradise, and lastly we prove that $V \setminus \SafeReach^f_\Odd(V, X^k_\Odd, \mathcal{G}^\ell)$ is an \Even-paradise. 

\begin{proof}\noindent \textbf{\textbf{ ($X^k_\Odd$ is an \Odd-paradise)}} 

\noindent Let $z$ be the winning \Odd strategy template on $Z^k = Z_\Odd^k$ in game $\mathcal{G}^\ell[Z^k]$. Any play $\pi$ that starts and stays in $Z^k$, and is compliant with $z$ is clearly \Odd winning.
However, $z$ is not necessarily an \Odd strategy template in $\mathcal{G}^\ell$ since there are possibly some $(v,w) \in E$ with $v \in Z^k \cap V_\Even$  and $w \not \in Z^k$.
For all such $(v,w)$, $w \in \SafeReach^f_\Odd(X^k_\Odd, N^k, \mathcal{G}^\ell)$ since $X^k_\Odd$ is an \Even-trap in $V$. Let $sr$ be the partial \Odd strategy template on $\SafeReach^f_\Odd(X^k_\Odd, N^k, \mathcal{G}^\ell)$, defined via the ranking function as presented during the introduction of safe reachability sets. 
Every (finite) play that starts in $\SafeReach^f_\Odd(X^k_\Odd, N^k, \mathcal{G}^\ell)$ compliant with $sr$ reaches $N^k$ in finitely many steps. The nodes in $N^k$ are dead ends in $sr$. 
Define an \Odd strategy template on $X^k_\Odd$ with the edge set $\e$ defined as follows:
$$ (v,w) \in \e \text{ if }\begin{cases} (v,w) \in z \cup sr,\\
    (v,w) \in E \text{ and } v \in V_\Even \cap X^k_\Odd,\\
    w = v_r \text{ if } v \in N^k \cap V_\Odd
\end{cases}
$$ where $v_r$ is a randomly chosen fixed successor for each $v\in  N^k \cap V_\Odd$, that is inside $X^k_\Odd$. Such a successor is guaranteed to exist since $X^k_\Odd$ is an \Even-trap.
Observe that all edges in $\e$ are in $X^k_\Odd \times X^k_\Odd$. However $(X^k_\Odd, \e)$ is not necessarily an \Odd strategy template in $\mathcal{G}^\ell$ since
there may be some $v \in V^\ell$ that lie on a cycle in $(X^k_\Odd, \e)$ but $\e$ does not contain their live outgoing edges. 
We will expand the edge set $\e$ to add the necessary live edges iteratively, like we did in ~\ref{const:S} (S3)-(S4).
$\overline{\e}$ is defined to be the saturation value of $\overline{e}^j$ such that:
$$\overline{e}^0 = \e, \quad  \quad \overline{e}^j = \overline{e}^{j-1} \cup \{(v, w) \in V^\ell \mid v \text{ lies on a cycle in } (X_\Odd^k, \overline{e}^{j-1})\}.$$

\vspace{0.1cm}
With this construction $\mathcal{S} = (X_\Odd^k, \overline{\e})$ is an \Odd strategy template in $\mathcal{G}^\ell$. We claim it is also a winning one. 

The underlying observation of the proof of the claim is that every play starting  $X_\Odd^k$ compliant with $\mathcal{S}$ that eventually stops seing a newly added cycle (one that is not in $z \cup sr$), stays in $Z^k$ and is won by \Odd obeying $z$; and every play that takes a newly added cycle infinitely often must see priority $n$ infinitely often, and is thus won by \Odd.

Let us look at a play $\pi$ compliant with $\mathcal{S}$. If $\pi$ eventually does not see a newly added cycle, it is clear that it wins by eventually obeying $z$ (since $sr$ does not contain any cycles).


Observe that for all newly added edges $(v,w)$ either (i)  $v \in V_\Even \cap Z^k$ and $w \in \SafeReach^f_\Odd(X^k_\Odd, N^k, \mathcal{G}^\ell)$, (ii) $v \in N^k$ or (iii) $(v, w) \in E^\ell $ where $v$ does not lie on a cycle in $z \cup sr$ and has a unique edge $(v,w') \in z \cup sr$, and this edge lies on a cycle in $\mathcal{S}$.

All the newly added cycles have to contain a newly added edge. 
If $\pi$ sees a new edge infinitely often, it visits $N^k$ infinitely often, and is thus won by \Odd. This is clear for edges of kind (ii).
Let $\pi$ see an edge of kind (iii) infinitely often. If $w\in V_\Even$, then all its outgoing edges achieves positive progress towards $N^k$, and if $w \in V_\Odd$, then it has an edge that achieves positive progress. Since $w$ is taken infinitely often, an edge that achieves positive progress towards $N^k$ will eventually be taken. Thus, $N^k$ will eventually be reached. That is, $\pi$ will visit $N^k$ infinitely often.
Finally let $\pi$ see an edge $(v,w)$ of kind (i) infinitely often. Then $(v,w')$ is also seen infinitely often. Let $C^1$ be the cycle that contains $(v,w')$. Since $C^1$ is also newly added, it contains a newly added edge $(v_1, w_1) \neq (v,w)$ since $C^1$ exists in $\overline{\e}$ before $(v,w)$ is added. If $(v_1, w_1)$ is of kind (i) or (ii), we are done. Assume the edge is of kind (iii)
and let $(v_1, w'_1)$ be the unique outgoing edge of $v_1$ in $z \cup sr$. $(v_1, w'_1)$ lies on a newly added cycle $C^2$. Let $(v_2, w_2) \not \in \{(v, w), (v_1, w_1)\} $ be the newly added edge in $C^2$. 
Carry on in this manner, assuming all newly added edges $(v_i, w_i)$ are of kind (iii). Since all $(v_i, w_i)$ are distinct and there are a finite number of live edges, for some $C^r$, $(v_r, w_r)$ should be of kind (i) or (ii).
Since $\pi$ sees $v$ infinitely often it should see all $C^i$ infinitely often, and since $C^r$ visits $N^k$, $\pi$ visists $N^k$ infinitely often. Thus, $\pi$ is won by \Odd.


\vspace{0.2cm}
\noindent \textbf{($\SafeReach^f_\Odd(V, X^k_\Odd, \mathcal{G}^\ell)$ is an \Odd-paradise)} 

\noindent Since $X_\Odd^k$ is an \Odd-paradise in $\mathcal{G}^\ell$, by Lem.~\ref{app-lem:safe-reach-Odd-paradise} we get that $\SafeReach^f_\Odd(V, X_\Odd^k, \mathcal{G}^\ell)$ is again an \Odd-paradise in $\mathcal{G}^\ell$.
\vspace{0.2cm}

\noindent\textbf{($V \setminus \SafeReach^f_\Odd(V, X^k_\Odd, \mathcal{G}^\ell)$ is an \Even-paradise)} 

\noindent Let $T:=\SafeReach^f_\Odd(V, X^k_\Odd, \mathcal{G}^\ell)$ and 
$\Xsr_\Even^i := \SafeReach^f_\Even(X_\Odd^i, Z_\Even^i, \mathcal{G}^\ell)$. Let the partial \Even strategy template on $\Xsr_\Even^i$ be denoted by $sr^i$ and the winning \Even strategy on $Z_\Even^i$ in game $\mathcal{G}^\ell[Z^i]$ be denoted by $z^i$. By Lem.~\ref{app-lem:Kuesters6.4}, $V \setminus T$ is an \Odd-trap.
Cor.~\ref{app-cor:increasing-decreasing-sequences} gives us that $\{X_\Even^i\}_{i \in \mathbb{N}}$ is an increasing sequence. 
Furthermore by Lem.~\ref{app-lem:X_nb-equivalence}, which gives an alternative definition for $X_\Even^{i+1}$, we observe that each $v \in X^k_\Even$ belongs to $\Xsr^j$ for some $j < k$.
Moreover, we can observe that $X^i_\Even$ and $\Xsr^i$ are disjoint sets, due to $X_\Even^i$ and $X_\Odd^i$ being disjoint. Therefore, we conclude that 
each $v \in X_\Even^k$ belongs to a unique $Xsr^j$. The same clearly holds for $v \in V\setminus T$, since $(V\setminus T) \subseteq X_\Even^k$.
Furthermore, since $V\setminus T$ is an \Odd-trap, for all $(v,w) \in E$ with $v \in V_\Odd \cap (V \setminus T)$, $w \in (V\setminus T)$.

We construct the \Even strategy template $\mathcal{S} = (X, \e)$ where $\e$ is defined as follows: $(v,w) \in E $ is in $\e$ if, 
$$\begin{cases}v \in V_\Odd\\
    v \in Z_\Even^i \text{ and } (v,w) \text{ is the unique outgoing edge of }v \text{ in } z^i\\
    v \in \Xsr_\Even^i \setminus Z_\Even^i \text{ and } (v,w) \text{ is the unique outgoing edge of }v \text{ in } sr^i\\
\end{cases}$$

It is clear that $\mathcal{S}$ is an \Even strategy template since it contains all outgoing edges of \Odd nodes in $V \setminus T$, and a unique outgoing edge for each \Even node in $V \setminus T$. We claim that $\mathcal{S}$ is also winning.
To prove this claim we will need the following two observations. 

Let $\pi = v_1 v_2 \ldots $ be a fair play that start in $V \setminus T$ and is compliant with $\mathcal{S}$. Let $\Xsr(\pi) = \Xsr_1\Xsr_2\Xsr_3\ldots$ be such that $\Xsr_i$ is the unique $\Xsr^j$, $v_i$ belongs to.

(1) If $v_t \in Z_\Even^i$, then $v_{t+1}$ is either in $Z_\Even^i$ or in $\Xsr^r$ for some $t < i$. This follows from $Z_\Even^i$ being an \Odd-trap in $X_\Odd^i$ (by Obs.~\ref{app-obs:traps-subgames}).

(2) If $\Xsr^i$ is seen infinitely often in $\Xsr(\pi)$, then $Z_\Even^i$ is seen infinitely often as well. Due to the pigeonhole principle, $\Xsr^i$ being visited infinitely often in $\Xsr(\pi)$ implies that some $v \in \Xsr^i$ is visited infinitely often.
If $v \not \in Z_\Even^i$, it is in $\Xsr^i \setminus Z_\Even^i$. Say $v \in V_\Even$, then the unique $(v,w) \in \e$ causes positive progress towards $Z_\Even^i$. If $v \in V_\Odd \setminus V^\ell$, then all of the outgoing edges of $v$ cause positive progress towards $Z_\Even^i$.
If $v \in V^\ell$, there is at least one $(v,w) \in E^\ell$ causing positive progress towards $Z_\Even^i$. Since $v$ is seen infinitely often in $\pi$, this edge is taken infinitely often as well. 
By induction, $\pi$ visits $Z_\Even^i$ infinitely often.

\vspace{0.2cm}
\noindent \emph{Claim:} Any fair play $\pi$ starting in $X$ and compliant with $\mathcal{S}$ eventually stays in $Z_\Even^i$ for some $i$.

\noindent \emph{Proof of Claim.}
Let $i$ be the minimum index for which $\Xsr^i$ appears infinitely often in $\Xsr(\pi)$. By observation (2), $\pi$ sees a set of nodes $P \subseteq Z_\Even^i$ infinitely often. Let $v_t \in P$. By observation (1), $v_{t+1}$ is either in $Z_\Even^i$ or in $\Xsr^r$ for some $r < i$.
Since $i$ is the minimum index for which $\Xsr^i$ is seen infinitely often in $\Xsr(\pi)$, after some $t' \in \mathbb{N}$, for all $v_{t'}\in P$, $v_{t'+1} \in Z_\Even^i$.

Since $\pi$ eventually stays in $Z_\Even^i$, the strategy $\mathcal{S}$ eventually collapses to $z_\Even^i$ and thus, \Even wins $\pi$.
\end{proof}

With this, we have proven Prop.~\ref{app-prop:n-odd}, and therefore have given an algorithm to calculate \We and \Wo on an \Odd-fair parity game with 
an odd upperbound $n$ on the priorities in the game graph. The algorithm however requires a sibling-algorithm that does the same for an \Odd-fair parity game with an upperbound $n-1$ on its priorities. In the second part that follows, we give this sibling-algorithm.

% \vspace{0.5cm}
\smallskip
\noindent\textbf{Part 2.}
We now assume an even number $n$ is the maximum priority in $\mathcal{G}^\ell$. %We construct a decreasing series of subsets of $V$, $\{X_\Odd^i\}_{i\in \mathbb{N}}$ by assigning the following sets (Fig.~\ref{fig:X_Even}):
%  
We set the sets as before, and because $n$ is even, this time $\{X_\Odd^i\}_{i \in \mathbb{N}}$ is an increasing sequence and $\{X_\Even^i\}_{i \in \mathbb{N}}$ is a decreasing one (Fig.~\ref{fig:kuesters-figure-extended}).
Both sequences saturate at some index $k$, and for this $k$, $Z_\Odd^k = \emptyset$. Furthermore, $X_\Even^k$ and $X_\Odd^k$ are \We and \Wo, respectively.


\begin{proposition} For all $i$, $Z^i_\Odd \cup X^i_\Odd$ is an \Odd-paradise in $\mathcal{G}^\ell$. 
\end{proposition}
\begin{proof}
    The fact that $Z^i_\Odd \cup X^i_\Odd$ is an \Even-trap follows from the observations in \ref{app-obs:traps-subgames}.

Let us denote the winning \Odd strategy template on $ Z^i_\Odd $ in $\mathcal{G}^\ell[Z^i]$ with $z$ 
and the strategy template on $X_\Odd^i$ in $\mathcal{G}^\ell$ by $x$. 
Let $\e$ be the edge set that contains all edges in $z \cup x$, together with all $\{(v,w) \in E \mid v \in V_\Even \cap (Z^i_\Odd \cup Z_\Odd^i) \}$.
Due to $X_\Odd^i$ being an \Even-trap in $V$, all outgoing edges of \Even nodes in $X_\Odd^i$, stay in $X_\Odd^i$. Then, $\e$ does not introduce any new cycles to $z \cup x$ since all the newly added edges are in one direction, from $Z^i_\Odd$ to $X^i_\Odd$. Thus, $\mathcal{S} = (X_\Odd^i \cup Z_\Odd^i, \e)$ is an \Odd strategy template in $\mathcal{G}^\ell$.
We claim it is also a winning one. 
A play $\pi$ starting in $X_\Odd^i$ and compliant with $\mathcal{S}$ stays in $X_\Odd^i$ and therefore wins by obeying $x$. 
If $\pi$ starts in $Z_\Odd^i$, it either eventually reaches $X_\Odd^i$ and therefore wins by the previous argument. Or, it stays in $Z_\Odd^i$ and wins by obeying $z$.
\end{proof}

\begin{proposition} If $Z^i_\Odd = \emptyset$, $X^i_\Even$ is an \Even-paradise in $\mathcal{G}^\ell$.
\end{proposition}

\begin{proof}
We know $X^i_\Even$ is an \Odd-trap~\ref{app-obs:traps-subgames}. Let $z$ be the winning \Even strategy on $Z^i_\Even$ in subgame $\mathcal{G}^\ell[Z^i]$ and $sr$ be the partial strategy template on $\SafeReach^f_\Even(X^i_\Even, N^i, \mathcal{G}^\ell)$ where all nodes in $\SafeReach^f_\Even(X^i_\Even, N^i, \mathcal{G}^\ell) \setminus N^i$ are forced to positive progress towards $N^i$ in the next step, and nodes in $N^i$ are dead-ends.

We construct an \Even strategy template $\mathcal{S} = (X_\Even^i, \e)$ where $\e$ is defined as follows: 
$$
(v,w) \in \e \text{ if }\begin{cases}
    (v, w) \in z \cup sr, \\
    (v, w)\in E \text{ and } v\in V_\Odd \cap X_\Even,\\
    w = v_r \text{ if } v \in N^i \cap V_\Even
\end{cases}
$$ where $v_r$ is a randomly chosen fixed successor for each $v \in N^i \cap V_\Even$, that is inside $X_\Even^i$. Such a successor is guaranteed to exist since $X_\Even^i$ is an \Odd-trap.

$\mathcal{S}$ is clearly an \Even strategy template in $\mathcal{G}^\ell$ since all \Odd nodes in $X_\Even^i$ have all their outgoing edges in $\mathcal{S}$ and all \Even nodes have a unique outgoing edge.
We claim it is also winning. 

Let $\pi$ be a play that starts in $X_\Even^i$ and is compliant with $\mathcal{S}$. We claim $\pi$ either (i) eventually stays in $Z_\Even^i$, and therefore eventually obeys $z$ or (ii) it sees $N^i$ infinitely often.
It is easy to see that in both of these cases $\pi$ is \Even winning. We will try to show that one of these cases must occur.
Assume $\pi$ does not eventually stay in $Z^i_\Even$. Then $\pi$ visits some $ v \in \SafeReach^f_\Even(X^i_\Even, N^i, \mathcal{G}^\ell)$ infinitely often. If $v \in V_\Odd$, all outgoing edges of $v$ are in $sr$ make positive progress towards $N^i$, and if $v \in V_\Even $ the unique successor of $v$ in $sr$ make positive progress towards $N^i$. 
Thus, $\pi$ visists $N^i$ after finitely many steps. Since $v$ is visited infinitely often by $\pi$, $N^i$ is also visited infinitely often.
\end{proof}


\subparagraph{Corrrectness of Alg.~\ref{algo:fair-zielonka-bb}.}
The $X$ set in $\SOLVE_\Odd(n, \mathcal{G}^\ell)$ holds the value of $X_\Odd^i$ and
the $X$ set in $\SOLVE_\Even(n, \mathcal{G}^\ell)$ holds the value of $X_\Even^i$ at the $i^{th}$ iteration of their respective \emph{while} loops. 
Note that both of these sequences are initialized at $V$ and are strictly decreasing, until they reach their saturation value $X_\Odd^k$ or $X_\Even^{k'}$. When these saturation values are reached $Z_\Even^k = \emptyset $ in the $\SOLVE_\Odd$ procedure and $Z_\Odd^{k'} = \emptyset $ in the $\SOLVE_\Even$ procedure. 
This is exactly when $\SOLVE_\Even$ returns $X_\Even^{k}$ and $\SOLVE_\Odd$ returns $\SafeReach^f_\Odd(V, X_\Odd^{k'}, \mathcal{G}^\ell)$; correctfully returning their respective winning regions according to the correctness proof of Thm.~\ref{thm:solvebb}.



\subsection{Details on Experimental Results}\label{app:experiments}

We conducted an experimental study to empirically validate the claim that our new \Odd-fair Zielonka's algorithm retains its efficiency in practice. For this, we implemented the following algorithms (non-optimized) in C++:
\begin{itemize}
 \item \texttt{OF-ZL}: \Odd-fair Zielonka's algorithm (Alg.~\ref{algo:fair-zielonka-bb}),
 \item \texttt{N-ZL}: \enquote{normal} Zielonka's algorithm from \cite{Zielonka98} (i.e., Alg.~\ref{algo:fair-zielonka-bb} with the simplifications described in Sec.~\ref{sec:zielonka:orig}),
 \item \texttt{OF-FP}: the fixed-point algorithm for \Odd-fair parity games implementing \eqref{eq:fp-odd} ,
 \item \texttt{N-FP}: the fixed-point algorithm for \enquote{normal} parity games from \cite{EJ91}.
\end{itemize}
Of course, for both \texttt{N-ZL} and \texttt{N-FP} there exist optimized implementations in the tool \texttt{oink}~\cite{oink}. However, the goal of this section is to show a conceptual comparison, rather than evaluating best computation times. We believe that this is better achieved by using similar (non-optimized) implementations for all algorithms. In particular, we run the following three sets of experiments to show that:
\begin{enumerate}
 \item \texttt{OF-ZL}: is largely insensitive to the number of priorities and the number of fair edges (see Fig.~\ref{fig:percentages-colours}),
 \item \texttt{OF-ZL}: significantly outperforms \texttt{OF-FP} on almost all benchmarks (see Fig.~\ref{fig:zoomed_out} (right))
\item the performance of \texttt{OF-ZL} and \texttt{N-ZL} on the given benchmark set is very similar (see Fig.~\ref{fig:zielonkas_comparison}),
 \item the comparative performance of \texttt{OF-ZL} and \texttt{N-ZL} w.r.t.\ their respective fixed-point versions \texttt{OF-FP} and \texttt{N-FP}), respectively, is very similar (see Fig.~\ref{fig:logscale}).
\end{enumerate}
All experiments where run on a large benchmark suite explained in Sec.~\ref{app:experiments:benchmarks}. To perform our experiments we used a machine equipped with Intel(R) Core(TM) i5-6600 CPU @ 3.30GHz and 8GB RAM. We declare a timeout when the calculation of an example exceeds 1 hour.

\subsubsection{Benchmark}\label{app:experiments:benchmarks}
We generated \Odd-fair parity game instances manipulating $286$ benchmark instances of PGAME$\_$Synth$\_$2021 dataset of the SYNTCOMP benchmark suite~\cite{syntcomp} and $51$ benchmark instances of the PGSolver dataset of Keiren's benchmark suite~\cite{keirens}. Within the latter, we restricted ourselfs to instances with $\leq 5000$ nodes.
Both datasets contain examples of normal parity games. For each selected
example, we generate \Odd-fair parity game instances for a particular liveness percentage $\alpha$. For a $\alpha\%$-liveness instant, we fix $\alpha\%$ of the \Odd nodes in the game, and turn $\alpha\%$ of each of their outgoing edges to live edges. %, where $\alpha$ is either $30$ or $50$. 
In addition, we also generated \Odd-fair parity game instances with varying number of priorities $p$ by partitioning the nodes of the games uniformly at random according to the number of priorities.

%
Detailed run-times of all algorithms on a representative selection of examples from the instances fenerated from SYNTCOMP benchmark suite are listed in Table~\ref{table:FPvsOddfairzlk}. 
% 
On the \Odd-fair instances with $50\%-$liveness generated from the SYNTCOMP benchmark suite, there are 204 instances where neither of the algorithms \texttt{OF-FP}, \texttt{OF-ZL}, \texttt{N-FP} or \texttt{N-ZL} timed out. On these instances, \texttt{OF-ZL} gives an average computation time of $4.6$ seconds while \texttt{OF-FP} took $122.7$ seconds on average. 
On the same examples, \texttt{N-ZL} takes on average $3.6$ seconds to compute while \texttt{N-FP} gives an average of $45.2$ seconds. 
For the PGSolver dataset \texttt{OF-FP} timed out on all generated instances, whereas \texttt{OF-ZL} took $24.9$ seconds on average to terminate.




% In order to compare the algorithmic advantage Zielonka's algorithm posesses over the fixed-point implementation, 
% we have implemented a naive fixed-point algorithm in the same fashion with \texttt{OF-FP} and our own version of Zielonka's algorithm for regular parity games in the same fashion with \texttt{OF-ZL}.
% We call these these algorithms \texttt{N-FP} and \texttt{N-ZL} respectively, where \texttt{R} stands for `regular'.  
% We do not use any efficient online parity solvers and instead use our 'non-optimal' code that is written similarly to their fair-parity counterparts in order for the comparison to be meaningful.


\subsubsection{Sensitivity}\label{app:experiments:sensitivity}
To monitor the sensitivity of \texttt{OF-ZL} to the change in number of priorities as well as the percentage of live edges in the game, we picked $12$ parity game instances from the SYNTCOMP dataset which did not timeout (after one hour). %\AKS{are those also depicted in the table? if so, could we mark the rows?}\IS{They are mostly not in the table.}
With priorities $3-4-5-6$ and liveness degrees 0$\%$\footnote{regular parity game}-30$\%$-50$\%$-80$\%$ we get 192 different \Odd-fair parity instances. Fig.~\ref{fig:percentages-colours} shows the runtime of \texttt{OF-ZL} on these instances.

We can see that the runtimes of instances with different priority and liveness percentages are distributed in a seemingly random manner.
This tells us that \Odd-fair Zielonka's algorithm is highly insensitive to a change in the percentage of live edges and the number of priorities. %\todo{IS:whereas fixed-point algorithm...?} 
This observation is inline with the known insensitivity of Zielonka's algorithm for the number of priorities.

% Figure environment removed




\subsubsection{Comparative Evaluation}\label{app:experiments:comparison}
In order to validate the computational advantage of \texttt{OF-ZL} over \texttt{OF-FP}, we have run both algorithms on all 50$\%$-liveness instances generated from the SYNTCOMP benchmark dataset. On 58 of these instances, both algorithms time out. The run-times for all other instances are depicted in Fig.~\ref{fig:zoomed_out} (right),~\ref{fig:zoomed_in} (right) and~\ref{fig:logscale} (right). The left plots in Fig.~\ref{fig:zoomed_out}-\ref{fig:logscale} show the same comparison for the \enquote{normal} parity algorithms \texttt{N-ZL} and \texttt{N-FP}. 
In both cases, Fig.~\ref{fig:zoomed_in} shows the zoomed-in version of the respective plot in Fig.~\ref{fig:zoomed_out}. Fig.~\ref{fig:logscale} shows the data-points from the respective plot in Fig.~\ref{fig:zoomed_in} as a scatter plot in log-scale. 
The examples on which only \texttt{x-FP} times out, can be seen as the dots on the ceiling of the plots in Fig.~\ref{fig:zoomed_out}. In all plots, points above the diagonal correspond to instances where Zielonka's algorithm outperforms the fixed-point algorithm.

We clearly see in Fig.~\ref{fig:zoomed_out}-\ref{fig:logscale} that Zielonka's algorithm performs significantly better than the fixed-point version, both in the \Odd-fair (right) and in the normal (left) case. More importantly, the overall performance comparison between  \texttt{OF-ZL} over \texttt{OF-FP} (right plots) mimics the comparison between \texttt{N-ZL} over \texttt{N-FP}. This allows us to conclude that our new \Odd-fair Zielonka's algorithm retains the computational advantages of Zielonka's algorithm. % and handling fairness only introduces a mild overhead. 

In addition, Table~\ref{table:FPvsOddfairzlk} shows that \texttt{OF-ZL} results in almost the same run-time as \texttt{N-ZL}, showing that our changes in the algorithm incur almost no computational disadvantages over the original algorithm. This allows us to handle transition fairness for almost free in practice.



% Figure environment removed

% Figure environment removed

% Figure environment removed

% Figure environment removed

% 
% 
% 
% 
% %TABLE
% In Table~\ref{table:FPvsOddfairzlk}, we aim to make two different comparisons visible, on generated instances from 20 representative examples we selected from PGAME$\_$Synth$\_$2021 dataset. First comparison is between the performance of the two algorithms, \texttt{OF-FP} and \texttt{OF-ZL} for fair parity games and 
% the second comparison is between the speed-up provided by Zielonka's algorithm in regular parity games and the speed-up provided by fair Zielonka's algorithm in fair parity games with respect to their respective fixed-point algorithms. 
% 
% A generated instance is prefixed by \enquote{30$\%$-} (\enquote{50$\%$-}) if it's a fair-parity instance with 30$\%$-(50$\%$-)-liveness, created from the representative example with the original name. The listed runtimes for \texttt{FP} and \texttt{ZL} correspond to the runtimes of \texttt{OF-FP} and \texttt{OF-ZL} on these instances.  
% 
% %with 30$\%$-liveness or "50$\%$-" to the original example's name for the first instance (that turns $30\%$ of the \Odd nodes and edges to live), and the second instance (that do the same for $50\%$), respectively. We run the \Odd-fair fixed-point (our naive implementation of the fixed-point algorithm from~\eqref{eq:fp-odd}) and \Odd-fair Zielonka (Alg.~\ref{algo:fair-zielonka-odd}-\ref{algo:fair-zielonka-even}) on these instances and the listed runtimes for \texttt{FP} and \texttt{ZL} correspond to the runtimes of \texttt{OF-FP} and \texttt{OF-ZL}, respectively.  
% If an instance is listed without any prefix, it corresponds to the original representative example, which is as a parity game without live edges. The listed runtimes for \texttt{FP} and \texttt{ZL} correspond to the runtimes of \texttt{N-FP} and \texttt{N-ZL} on these instances.  
% 
% \IS{change the numbers according to results:} On 3 of these instances, the fixed-point algorithm timed out whereas Zielonka's algorithm took on average 90 seconds to terminate. 



\smallskip
\textbf{Conclusion:}
The results show that Zielonka's algorithm is significantly faster in solving \Odd-fair parity games compared to the calculation performed by the fixed-point algorithm, as is the case in normal parity games. 
The fixed-point algorithm started timing out as soon as the examples became more complex, being especially sensitive to  
the increase in the number of priorities. Whereas, Zielonka's algorithm preserves its performance considerably in the face of the increase in the same parameters. These 
outcomes match the known comparison results between the naive fixed-point calculation versus Zielonka's algorithm, on normal parity games.  



\newpage
% \begin{center}

     \begin{longtable}{||c |c |c |c |c |c ||} %begin{table}
     \caption{Detailed run-time comparison of \texttt{N-FP} and \texttt{N-ZL} on the original parity game instances (yellow rows) with \texttt{OF-FP} and \texttt{OF-ZL} on their respective $30\%$- and $50\%$-liveness \Odd-fair parity game instances (white rows). The instance name is taken from the original benchmark suite.
%      parity examples (the rows with the original example's names)
% and \Odd-fair instances of these examples with 30$\%$- and 50$\%$-liveness (the rows with the original example's names prefixed by \enquote{30$\%$-} or \enquote{50$\%$-})
% solved via a naive implementation of their corresponding fixed point and Zielonka's algorithms. For the regular parity examples \texttt{FP} and \texttt{ZL} correspond to \texttt{N-FP} and \texttt{N-ZL}, and for \Odd-fair parity instances they correspond to \texttt{OF-FP} and \texttt{OF-ZL}, respectively.
}\label{table:FPvsOddfairzlk}\\
     
    %      \begin{tabular}{||c |c |c |c |c |c ||}
 \hline
 Name &  $\#$  &  $\#$  &  $\#$  & \texttt{FP} & \texttt{ZL}  \\ [0.5ex] 
      &    nodes    &   edges     &   priorities &     (sec.)       & (sec.)  \\
 \hline  \hline \rowcolor{Highlight}
 EscalatorCountingInit & 99 & 148 & 3 & 0.064 & 0.012  \\ 
 \hline
 $30\%$-EscalatorCountingInit & 99 & 148 & 3 & 0.075 & 0.018  \\ 
 \hline
 $50\%$-EscalatorCountingInit & 99 & 148 & 3 & 0.072 & 0.02 \\
 \hline \rowcolor{Highlight}
 KitchenTimerV1 & 80 & 124 & 3 & 0.055 & 0.008 \\
 \hline
 $30\%$-KitchenTimerV1 & 80 & 124 & 3 & 0.068 & 0.012 \\
 \hline
 $50\%$-KitchenTimerV1 & 80 & 124 & 3 & 0.21 & 0.009  \\
 \hline \rowcolor{Highlight}
 KitchenTimerV6 & 4099 &  6560 & 3 & 87 & 11 \\
 \hline
 $30\%$-KitchenTimerV6 & 4099 &  6560 & 3 & 88 & 11 \\
 \hline
 $50\%$-KitchenTimerV6 & 4099 &  6560 & 3 & 352 & 18 \\
 \hline \rowcolor{Highlight}
 MusicAppSimple & 344 &  562 & 3 & 0.488 & 0.073 \\
 \hline
 $30\%$-MusicAppSimple & 344 &  562 & 3 & 0.496 & 0.082 \\
 \hline
 $50\%$-MusicAppSimple & 344 &  562 & 3 & 0.799 & 0.089 \\
 \hline \rowcolor{Highlight}
 TwoCountersRefinedRefined & 1933 & 3140 & 3 & 14.9 & 2.5 \\
 \hline
 $30\%$-TwoCountersRefinedRefined & 1933 & 3140 & 3 & 15 & 1.2 \\
 \hline 
 $50\%$-TwoCountersRefinedRefined & 1933 & 3140 & 3 & 74 & 3.72 \\
 \hline \rowcolor{Highlight}
 Zoo5 & 479 & 768 & 3 & 0.96 & 0.135 \\
 \hline
 $30\%$-Zoo5 & 479 & 768 & 3 & 0.981 & 0.152 \\
 \hline 
 $50\%$-Zoo5 & 479 & 768 & 3 & 1.57 & 0.172 \\
 \hline \rowcolor{Highlight}
 amba$\_$decomposed$\_$lock$\_$3 & 1558 & 2336 & 3 & 72 & 1.5  \\
 \hline
 $30\%$-amba$\_$decomposed$\_$lock$\_$3 & 1558 & 2336 & 3 & 73 & 1.5  \\
 \hline 
 $50\%$-amba$\_$decomposed$\_$lock$\_$3 & 1558 & 2336 & 3 & 56 & 2.9 \\
 \hline \rowcolor{Highlight}
 full$\_$arbiter$\_$2 & 204 & 324 & 3 & 0.59 & 0.049 \\
 \hline
 $30\%$-full$\_$arbiter$\_$2 & 204 & 324 & 3 & 0.602 & 0.047 \\
 \hline 
 $50\%$-full$\_$arbiter$\_$2 & 204 & 324 & 3 & 5 & 0.059 \\
 \hline \rowcolor{Highlight}
 full$\_$arbiter$\_$3 & 1403 & 2396 & 3 & 21.18 & 2 \\
 \hline
 $30\%$-full$\_$arbiter$\_$3 & 1403 & 2396 & 3 & 21.5 & 2 \\
 \hline 
 $50\%$-full$\_$arbiter$\_$3 & 1403 & 2396 & 3 & 93 & 3.46 \\
 \hline \rowcolor{Highlight}
 lilydemo06 & 369 & 548 & 3 & 8.1 & 0.18 \\
 \hline 
 $30\%$-lilydemo06 & 369 & 548 & 3 & 8.13 & 0.206 \\ 
 \hline 
 $50\%$-lilydemo06 & 369 & 548 & 3 & 18 & 0.212 \\
 \hline \rowcolor{Highlight}
 lilydemo07 & 78 & 108 & 3 & 0.27 & 0.01 \\ 
 \hline
 $30\%$-lilydemo07 & 78 & 108 & 3 & 0.284 & 0.017 \\ 
 \hline 
 $50\%$-lilydemo07 & 78 & 108 & 3 & 0.33 & 0.008 \\ 
 \hline \rowcolor{Highlight}
 simple$\_$arbiter$\_$unreal1 & 2178 & 3676 & 3 & 22.8 & 3 \\
 \hline
 $30\%$-simple$\_$arbiter$\_$unreal1 & 2178 & 3676 & 3 & 23 & 3 \\
 \hline 
 $50\%$-simple$\_$arbiter$\_$unreal1 & 2178 & 3676 & 3 & 254 & 7 \\ 
 \hline \rowcolor{Highlight}
 amba$\_$decomposed$\_$arbiter$\_$2 & 141 & 212 & 4 & 0.72 & 0.03  \\
 \hline
 $30\%$-amba$\_$decomposed$\_$arbiter$\_$2 & 141 & 212 & 4 & 0.73 & 0.06  \\
 \hline
 $50\%$-amba$\_$decomposed$\_$arbiter$\_$2 & 141 & 212 & 4 & 1 & 0.035  \\
 \hline \rowcolor{Highlight}
 loadfull3 & 1159 & 2030 & 4 & 5.62 & 0.609 \\
 \hline
 $30\%$-loadfull3 & 1159 & 2030 & 4 & 5 & 0.614 \\
 \hline 
 $50\%$-loadfull3 & 1159 & 2030 & 4 & 5 & 0.754 \\
 \hline \rowcolor{Highlight}
 ltl2dba01 & 101 & 152 & 4 & 0.074 & 0.031  \\
 \hline
 $30\%$-ltl2dba01 & 101 & 152 & 4 & 0.075 & 0.030  \\
 \hline 
 $50\%$-ltl2dba01 & 101 & 152 & 4 & 1.4 & 0.028 \\
 \hline \rowcolor{Highlight}
 ltl2dba14 & 97 & 144 & 4 & 0.18 & 0.016 \\
 \hline
 $30\%$-ltl2dba14 & 97 & 144 & 4 & 0.181 & 0.013 \\
 \hline 
 $50\%$-ltl2dba14 & 97 & 144 & 4 & 0.574 & 0.012 \\
 \hline \rowcolor{Highlight}
 ltl2dba22 & 21 & 30 & 4 & 0.037 & 0.002 \\
 \hline
 $30\%$-ltl2dba22 & 21 & 30 & 4 & 0.036 & 0.002 \\
 \hline 
 $50\%$-ltl2dba22 & 21 & 30 & 4 & 0.03 & 0.0009 \\
 \hline \rowcolor{Highlight}
 prioritized$\_$arbiter$\_$unreal2 & 851 & 1412 & 4 & 15.8 & 0.73 \\
 \hline
 $30\%$-prioritized$\_$arbiter$\_$unreal2 & 851 & 1412 & 4 & 16 & 0.759 \\
 \hline 
 $50\%$-prioritized$\_$arbiter$\_$unreal2 & 851 & 1412 & 4 & 126 & 1.2 \\
 \hline \rowcolor{Highlight}
 lilydemo17 & 3102 & 5334 & 7 & 1237 & 41 \\
 \hline
 $30\%$-lilydemo17 & 3102 & 5334 & 7 & Timeout & 41 \\
 \hline 
 $50\%$-lilydemo17 & 3102 & 5334 & 7 & Timeout & 24 \\
 \hline \rowcolor{Highlight}
 lilydemo18 & 449 & 728 & 9 & 220 & 0.6 \\
 \hline
 $30\%$-lilydemo18 & 449 & 728 & 9 & 224 & 0.621 \\
 \hline 
 $50\%$-lilydemo18 & 449 & 728 & 9 & Timeout & 0.552 \\[1ex] 
 \hline
%\end{tabular}
%\caption{Performance comparison between parity examples solved via \texttt{N-FP} and \texttt{N-ZL} (the rows with the original example's names),
%and \Odd-fair instances of the example with 30$\%$- and 50$\%$-liveness solved via \texttt{OF-FP} and \texttt{OF-ZL} (the rows with the original example's names prefixed by \enquote{30$\%$-} or \enquote{50$\%$-}).}


\end{longtable} %\end{table}
% \end{center}
     
% 2 figures side-byy-side using minipages:
%% Figure environment removed

\newpage
\subsection{Additional material for Ex. \ref{ex:1}}\label{app:example}
Below we present an extended version of the fixed-point calculation in \eqref{equ:fpexample}, 
\begin{align*}
&   Y_4^{0} = \emptyset \\
&    \quad X_3^{0, 0} = V \\
&    \quad \quad Y_2^{0,0,0} = \emptyset \\
&    \quad \quad \quad X_1^{0,0,0,0} = V \\
&    \quad \quad \quad X_1^{0,0,0,1} = \Phi^{Y_4^{0}, X_3^{0, 0}, Y_2^{0,0,0}, X_1^{0,0,0,0} } = C_3 \cup C_1 \\
&    \quad \quad \quad X_1^{0,0,0,2} = \Phi^{Y_4^{0}, X_3^{0, 0}, Y_2^{0,0,0}, X_1^{0,0,0,1} } = C_3 \cup (C_1 \cap \Npre(Y_2^{0,0,0}, X_1^{0,0,0,1})) = C_3\\
&    \quad \quad \quad X_1^{0,0,0,3} = \Phi^{Y_4^{0}, X_3^{0, 0}, Y_2^{0,0,0}, X_1^{0,0,0,1} } = C_3 \cup (C_1 \cap \Npre(Y_2^{0,0,0}, X_1^{0,0,0,2})) = C_3\\
&    \quad \quad Y_2^{0,0,1} = X_1^{0,0,0,\infty} = C_3\\
&    \quad \quad \quad X_1^{0,0,1,0} = V \\
&    \quad \quad \quad X_1^{0,0,1,1} = \Phi^{Y_4^{0}, X_3^{0, 0}, Y_2^{0,0,1}, X_1^{0,0,0,0} } = C_3 \cup C_1 \cup \{2b\}\\
&    \quad \quad \quad X_1^{0,0,1,2} = \Phi^{Y_4^{0}, X_3^{0, 0}, Y_2^{0,0,1}, X_1^{0,0,0,1} } = C_3 \cup \{2b\}\\
&    \quad \quad \quad X_1^{0,0,1,3} = \Phi^{Y_4^{0}, X_3^{0, 0}, Y_2^{0,0,1}, X_1^{0,0,0,2} } = C_3 \cup \{2b\}\\
&    \quad \quad Y_2^{0,0,2} = X_1^{0,0,1,\infty} = C_3 \cup \{2b\}\\
&    \quad \quad \quad X_1^{0,0,2,0} = V \\
&    \quad \quad \quad X_1^{0,0,2,1} = \Phi^{Y_4^{0}, X_3^{0, 0}, Y_2^{0,0,2}, X_1^{0,0,0,0} } = C_3 \cup C_1 \cup \{2b, 2c\}\\
&    \quad \quad \quad X_1^{0,0,2,2} = \Phi^{Y_4^{0}, X_3^{0, 0}, Y_2^{0,0,2}, X_1^{0,0,0,1} } = C_3 \cup \{2b,2c\}\\
&    \quad \quad \quad X_1^{0,0,2,3} = \Phi^{Y_4^{0}, X_3^{0, 0}, Y_2^{0,0,2}, X_1^{0,0,0,2} } = C_3 \cup \{2b,2c\}\\
&    \quad \quad Y_2^{0,0,3} = X_1^{0,0,2,\infty} = C_3 \cup \{2b, 2c\}\\
&    \quad \quad \ldots\\
&    \quad \quad Y_2^{0,0,4} = X_1^{0,0,3,\infty} = C_3 \cup \{2b, 2c\}\\
&    \quad X_3^{0, 1} = Y_2^{0,0,\infty} =  C_3 \cup \{2b, 2c\} \\
&    \quad \quad Y_2^{0,1,0} = \emptyset \quad  \\
&    \quad \quad Y_2^{0,1,1} = X_1^{0,1,0,\infty} = \{3b\} \quad \\
&    \quad \quad Y_2^{0,1,2} = X_1^{0,1,1,\infty} = \{2b,3b\}\ \quad \\
&    \quad \quad Y_2^{0,1,3} = Y_2^{0,1,4} = X_1^{0,1,2,\infty} = X_1^{0,1,3,\infty} = \{2b,2c,3b\}\\
&    \quad X_3^{0, 2} = Y_2^{0,1,\infty} = \{2b,2c,3b\} \\
&    \quad \ldots \\
&    \quad X_3^{0, 3} = Y_2^{0,2,\infty} = \{2b,2c,3b\} \\
&    Y_4^{1} = X_3^{0,\infty} = \{2b,2c,3b\}\\
&    \quad X_3^{1, 0} = V \\
&    \quad \quad Y_2^{1,0,0} = \emptyset \\
&    \quad \quad Y_2^{1,0,1} = X_1^{1,0,0,\infty} = C_3 \cup C_4\\
\end{align*}
\begin{align*}
&    \quad \quad Y_2^{1,0,2} = X_1^{1,0,1,\infty} = C_3 \cup C_4 \cup \{2b\}\\
&    \quad \quad Y_2^{1,0,4} =  Y_2^{1,0,3} = C_1 \cup C_3 \cup C_4 \cup \{2b, 2c\}\\
&    \quad X_3^{1, 1} =  Y_2^{1,0,\infty} = C_1 \cup C_3 \cup C_4 \cup \{2b, 2c\}\\
&    \quad \quad Y_2^{1,1,0} = \emptyset \\
&    \quad \quad Y_2^{1,1,1} = C_3 \cup C_4 \\
&    \quad \quad Y_2^{1,1,2} = C_3 \cup C_4 \cup \{2b\}\\
&    \quad \quad Y_2^{1,1,3} = Y_2^{1,1,4} = C_1 \cup C_3 \cup C_4 \cup \{2b, 2c\}\\
&    \quad X_3^{1, 2} =  Y_2^{1,1,\infty} = C_1 \cup C_3 \cup C_4 \cup \{2b, 2c\}\\
&   Y_4^{2} = X_3^{1,\infty} = C_1 \cup C_3 \cup C_4 \cup \{2b, 2c\}\\
&   \ldots \\
&   Y_4^{3} = C_1 \cup C_3 \cup C_4 \cup \{2b, 2c\}
\end{align*}
And finally,
\begin{equation*}
   \mathcal{W}_{Odd} = Y_4^\infty = C_1 \cup C_3 \cup C_4 \cup \{2b, 2c\} = V \setminus \{2a\}
\end{equation*}


\end{document}

