

% \section{Strong Transitions Fairness in Parity Games}
% This section introduces a new class of parity games where the players are additionally constrained by strong transition fairness. 




% \smallskip
% \noindent\textbf{Problem Statement and Outline.}
% Motivated by the known practical importance of exponential algorithms, such as Zielonka's algorithm, for solving Parity games fast in practice, the goal of this paper is to extend Zielonka's algorithm to \Odd-fair Parity games while retaining its efficiency. However, the 
% the computational flow of Zielonka's algorithm is quite different from the known fixed-point algorithm given in \cite{banerjee2022fast} - while the latter computes the winning region only for player \Even by repeatably looking at the entire game graph, Zielonka's algorithm (and in fact most known exponential algorithms) is based on a symmetric solution where winning regions of both players are incrementally constructed at the same time.
% % 
% Therefore, the extension of Zielonka's algorithm to \Odd-fair parity games requires the understanding (and formalization) of player \Odd strategies, which are not positional anymore as shown in example~\ref{}. 
% 
% The main contribution of this paper is therefore to formalize player \Odd strategies in \Odd-fair parity games via strategy templates, as introduced in Sec.~\ref{sec:templates}, and to show that there exists a winning strategy template for player \Odd from every vertex of its winning region that renders all strategies that are complaint with the template winning for \Odd. In order to 
% 
% % Within this paper, we compute winning strategies of player \Odd instead. In particular, we show a fixed-point algorithm in the $\mu$-calculus to compute the winning region $\Wo$ of player $\Odd$ in an \Odd-fair parity game $\mathcal{G}^\ell$, which again has the same number of symbolic steps as the respective fixed-point for \enquote{normal} parity games. 
% % % 
% % We then show that the ranking induced by this fixed-point over vertices in $\Wo$ can be used to define \emph{strategy templates}. Strategy templates define an infinite number of player $\Odd$ strategies which are all winning in $\mathcal{G}^\ell$. Intuitively, strategy templates show that knowing how player \Odd chooses to obey a given strong transition fairness constraint is not important for winning -- it suffices to know which fairness constraints need to be obeyed, which is part of the computed strategy template. 

\vspace{-0.15cm}

\section{Strategy Templates}\label{sec:templates}
% \vspace{-0.5em}

%Motivated by the known practical importance of exponential algorithms, such as Zielonka's algorithm, for solving Parity games fast in practice, the goal of this paper is to extend Zielonka's algorithm to \Odd-fair Parity games while retaining its efficiency. This, however, requires the formalization of player \Odd wining strategies in \Odd-fair parity games.
In this section, we introduce a formalization of player \Odd strategies in \Odd-fair parity games via \emph{strategy templates}.
% 
% under the name \Odd \emph{strategy templates}. This formalization will be needed in Sec. 4 while proving the correctness of Zielonka's algorithm for \Odd-fair parity games. 
% We state the existence of winning \Odd-strategy templates on \Wo in Prop~\ref{prop:existence-maximaloddstrategytemplates} but do not provide its proof until Sec. 5. 
% This is to avoid dividing the reader's attention from the extension of Zielonka's algorithm to Odd-fair parity games, which is the main emphasis of this paper and is introduced in Section 4.
% 
In contrast to player \Even, player \Odd winning strategies are no longer positional in \Odd-fair parity games, as illustrated by the following example. %that requires the same number of symbolic steps as the algorithm computing winning strategies for \Even in \enquote{normal} parity games.
% \vspace{-0.5em}
\begin{example}\label{ex:strategytemplates}
Consider the three different parity games depicted in Fig.~\ref{fig:Oddstrategies1}. %, three \Odd-fair parity games are depicted, with circles indicating \Ve and squares indicating \Vo. Edges in $E^\ell$ are shown by dashed lines. All nodes are labeled with their priorities.
   In all three games, \Odd has a winning strategy from all vertices, i.e., $\mathcal{W}_{Odd}=V$. %The red-colored edges indicate \Odd's strategy: if \Odd takes the red edges alternatingly from the source nodes, it wins from all nodes. 
  However, in order to win, the vertex $3$ has to be seen infinitely often in game (a) and (b), which forces \Odd to use its live edge\textbackslash s infinitely often. This prevents the existence of a positional strategy for \Odd in games (a) and (b): In (a) it needs to somehow alternate between (it's only) live edge to $4$ and a \enquote{normal} edge to $7$ (both indicated in red) in order to win, and in (b) it needs to somehow alternate between all its live edges (also indicated in red). In the game (c), \Odd can win by 'escaping' its live vertex $3$ to a \enquote{normal} vertex $5$, and thereby has a positional strategy. % (again indicated in red).
   
  Now consider the subgraph of each game formed by all colored edges (red and blue), which include the strategy choices from \Vo and \emph{all} outgoing edges from \Ve. As we have seen that \Odd needs to play all red edges repeatably, this subgraph represents the paths that \emph{can} be seen in the game depending on the \Even strategy. Hence, a node $v\in\Vl\subseteq\Vo$ can be seen infinitely often in a play (compliant with \Odd's strategy), if it lies on a cycle in this subgraph. We observe that, in games (a) and (b), node $3$ lies on cycles in this subgraph, whereas in game (c), it does not. 
  We further see that whenever a vertex  $v\in\Vl$ lies on a cycle, \Odd needs to take all its outgoing live edges (as for vertex $3$ in example (b)) and possibly one more edge (as for vertex $3$ in example (a)), for all other vertices in $\Vo$ a positional strategy suffices (as for vertex $5$ in all examples, and for vertex $3$ in example (c)). This shows that \Odd strategies are intuitively still \enquote{almost positional}.
% %   
%   
% Another intuition we gather is that, in all of these examples for any node in \Wo it is sufficient for \Odd to take either one outgoing edge (i.e. \Odd had a positional strategy on the node) or, to take all its outgoing live edges and possibly one more edge. We express this feature of a strategy by \enquote{almost positionality}.
   
%    The difference between the games (a)-(b) and (c) is that, in the first two node $3$ has to be seen infinitely often whereas in the later this is not the case. 
%    In game (a), \Odd does not have a positional winning strategy from node $3$; however, it has a winning strategy that allows it to take its live edge to $4$ together with its edge to $7$ infinitely often to win the game.
% In (b), once more \Odd does not have a positional winning strategy from node $3$, since in all winning plays it needs to take both its live outgoing edges infinitely often.
% On the other hand game (c) is an example where \Odd has a positional winning strategy that ignores the live outgoing edge of $3$.
\end{example}%\vspace{-2em}

% Figure environment removed

% The difference between the games (a)-(b) and (c) is that, in the first two node $3$ has to be seen infinitely often whereas in the later this is not the case. Observe that, for each game in Fig.~\ref{fig:Oddstrategies1} if we mark all outgoing edges of \Even nodes with red in addition to the red \Odd edges, we get a subgraph of the game.

\vspace*{-0.2cm}

The intuitions conveyed by Ex.~\ref{ex:strategytemplates} are formalized by the following definitions. % for \Odd strategy templates.

% are captured by so called \emph{strategy templates}, data structures that allow us to define an infinite number of winning player \Odd strategies for an \Odd-fair parity game in a finitary manner that is \emph{almost positional}.

\begin{definition}[\Odd Strategy Template]\label{def:Oddstrategytemplate}
 Given an \Odd-fair parity game $\mathcal{G}^\ell = \ltup{\mathcal{G}, E^\ell}$ with \newline $\mathcal{G} = \langle V, \Ve, \Vo, E, \chi\rangle$, an \Odd \emph{strategy template} $\mathcal{S}$ over $\mathcal{G}^\ell$ is a subgraph of $\mathcal{G}$ given as follows: $\mathcal{S}:=\tup{V',E'}$ where $V'\subseteq V$ and $E'\subseteq E \cap (V' \times V')$ such that the following hold,
\begin{compactitem}\label{item:Oddstrtemprules}
 \item if $v \in \Vo \cap V'$ does not lie on a cycle in $(V',E')$, then $|E'(v)|=1$,
 \item if $v \in \Vo \cap V'$ lies on a cycle in $(V',E')$ then $E^\ell(v) \subseteq E'(v)$ and  $1\leq |E'(v)|\leq |E^\ell(v)| + 1$,
 \item if $v \in \Ve \cap V'$, then  $E'(v) = E(v)$.
\end{compactitem}
\end{definition}
%Intuitively, whenever an \Odd vertex lays on a cycle in $(V', E')$, we expect it
%to have all its outgoing live edges, and possibly one more edge. Whenever it does not lay on a cycle, it has exactly one of its outgoing edges. All vertices $v\in\Ve$ have all of their outgoing edges. Moreover, all of the vertices in $V'$ have at least one outgoing edge in $E'$. \AKS{maybe move to the example above}\IS{I am not sure, we can remove if space doesn't suffice.}
% We call a tuple $(V', V'_\Even, V'_\Odd, E')$ a \textit{plausible} strategy template to indicate that the sets $V', V'_\Even, V'_\Odd $ and $E'$ satisfy the restrictions given in the definition above. \IS{??}
\begin{definition}\label{def:compliantstrat}
 Let  $\mathcal{G}^\ell = \ltup{\mathcal{G}, E^\ell}$ be an \Odd-fair parity game with \Odd strategy template $\mathcal{S}=\tup{V',E'}$, and $V'_\Odd := V' \cap V_\Odd$. Then an
\Odd strategy $\rho$ is said to be \textbf{compliant} with $\mathcal{S}$ if % the restriction of $\rho$ to $V'$ is a winning strategy in the game $\ltup{\mathcal{S},\alpha'}$ where 
it is a winning strategy in the game $\ltup{\gamegraph,\alpha'}$ where $\gamegraph= \tup{V,\Ve,\Vo,E}$ and 
\begin{subequations}
 \begin{align}
 \alpha':= &\textstyle\bigwedge_{v\in\Vo'}(\,\square\, (\,v \implies \bigvee_{(v,w)\in E'} \bigcirc\, w\,))\,\label{equ:alpha:a}\\
 & \textstyle\wedge \bigwedge_{v\in\Vo'} (\,\square \,\diamondsuit\, v \implies \bigwedge_{(v,w)\in E'}\square\, \diamondsuit\, (\,v \wedge \bigcirc \,w\,)).\label{equ:alpha:b}
\end{align}
\end{subequations}
\end{definition}

Intuitively, for all \Odd vertices in $\mathcal{S}$, the strategy $\rho$ compliant with $\mathcal{S}$ takes only their outgoing edges in $\mathcal{S}$ \eqref{equ:alpha:a}, and if a play visits an \Odd node $v$ infinitely often, then $\rho$ takes each of $v$'s outgoing edges in $\mathcal{S}$ infinitely often \eqref{equ:alpha:b}.
% 
For an \Odd strategy template $\mathcal{S}$, if $v \in V'_\Odd$ lies on a cycle in $\mathcal{S}$, then by Def. \ref{def:Oddstrategytemplate}, $\mathcal{S}$ contains all live outgoing edges of $v$. By \eqref{equ:alpha:b} any \Odd strategy $\rho$ compliant with $\mathcal{S}$ satisfies the fairness condition in \eqref{eq:fairness-ltl} for $v$. 
On the other hand, if $v \in V'_\Odd$ does not lie on a cycle in $\mathcal{S}$, then by \eqref{equ:alpha:a} any such $\rho$ sees $v$ at most once. Thus $\rho$ trivially satisfies \eqref{eq:fairness-ltl} for $v$. 
% That is, any \Odd strategy $\rho$ compliant with an \Odd strategy template satisfy the fairness condition \eqref{eq:fairness-ltl}. 
This observation is stated in the following proposition.

%With this, it becomes clear that any \Odd strategy $\rho$ compliant with any strategy template $\mathcal{S}$, obeys the fairness condition.
% \vspace{-1mm}
\begin{proposition}%\footnote{The proof of the proposition can be found in the appendix, Sec.~\ref{app:S-proof}. }\label{prop:S:fair}
 Given the premisses of Def.~\ref{def:compliantstrat} let $\pi$ be a play starting from a node in $V'$ that complies with $\rho$. Then $\pi \models \alpha$ where $\alpha$ is the LTL formula in~\eqref{eq:fairness-ltl}.%\vspace{-2mm}
\end{proposition}

Next, we define \Even strategy templates. Each \Even strategy template encodes a unique \Even positional strategy, which is known to exist in \Odd-fair parity games~\cite{Klarlund94}, due to the lack of fair edges defined on \Even vertices. %, \Even strategy templates are very simple\footnote{In fact, \Even strategy templates simply encode a positional strategy and are only re-defined to make further arguments more symmetric for both players.}.
\begin{definition}\label{def:Evenstrategytemplate}
    Given an \Odd-fair parity game $\mathcal{G}^\ell = \ltup{\mathcal{G}, E^\ell}$ with \newline $\mathcal{G} = \langle V, \Ve, \Vo, E, \chi\rangle$, an \Even \emph{strategy template} $\mathcal{S}$ over $\mathcal{G}^\ell$ is a subgraph of $\mathcal{G}$ given as $\mathcal{S}:=\tup{V', E'}$ where $V'\subseteq V$ and $E'\subseteq E \cap (V' \times V')$ such that,    \begin{compactitem}\label{item:Evenstrtemprules}
     \item if $v \in \Ve \cap V'$, then $|E'(v)|=1$,
     \item if $v \in \Vo \cap V'$, then  $E'(v) = E(v)$.
    \end{compactitem}
\end{definition}

\vspace*{-0.1cm}

An \Even strategy $\rho$ is compliant with the \Even strategy template $\mathcal{S} = \tup{V', E'}$ if for all $v \in V'_\Even$, $\rho(v) = E'(v)$. In other words, $\rho$ is the positional strategy defined by $\mathcal{S}$.

Let $\rho$ be an \Odd (\Even) strategy, compliant with the \Odd (\Even) strategy template $\mathcal{S}$ and let $\pi$ be a play compliant with $\rho$. Then we call $\pi$ a play \emph{compliant with $\mathcal{S}$}.

\vspace*{-0.1cm}

\begin{definition}
An \Odd (\Even) strategy template $\mathcal{S}=\ltup{V', E'}$ is \emph{winning} in the \Odd-fair parity game $\mathcal{G}^\ell$ if all \Odd (\Even) strategies $\rho$ compliant with $\mathcal{S}$ are winning for player \Odd (\Even) in $\mathcal{G}^\ell$ from $V'$. A winning \Odd (\Even) strategy template $\mathcal{S}$ is called \emph{maximal} if $V'=\Wo$ ($\We$).%\vspace{-2mm}
\end{definition}
%\IS{in the previous defn, we used \Even strategy templates without defining them first. Either define them beforehand or remove them from this definition.}

\vspace*{-0.2cm}
We note that maximal winning \Odd (\Even) strategy templates $\mathcal{S}$ immediately imply that for every vertex $v\in \Wo$ ($\We$) there exists a winning strategy for player \Odd (\Even) from $v$ that is compliant with $\mathcal{S}$.
% 
The existence of maximal winning \Even strategy templates follows from the existence of positional \Even strategies~\cite{Klarlund94}. 
% 
The first main contribution of this paper is a constructive proof showing the existence of maximal winning \Odd strategy templates given in the next section. 
This result is then used in Sec.~\ref{sec:zielonka} to prove the correctness of \Odd-fair Zielonka's algorithm, which is introduced there.

% The following proposition claims the existence of maximal winning \Odd strategy templates. We will postpone the proof until Sec. 5. In the following chapter we will assume Prop.~\ref{prop:existence-maximaloddstrategytemplates} holds and build a variant of Zielonka's algorithm that solves \Odd-fair parity games based on the assumption.  
% \begin{proposition}\label{prop:existence-maximaloddstrategytemplates}
%     Given an \Odd-fair parity game $\mathcal{G}^\ell$, there exists a maximal winning \Odd strategy template. 
% \end{proposition}
%The remainder of this paper shows an algorithm to compute \emph{maximal winning strategy templates} for \Odd in \Odd-fair parity games.
% \vspace{-4mm}
% 
% \begin{proposition}
% Let $\mathcal{W}_{Odd}$ be the winning region of \textit{Odd} in $\mathcal{G}^\ell= ((V,E, \chi),E^\ell)$. Then, there exists a strategy template
% $(\mathcal{W}_{Odd}, E')$ of $\mathcal{G}^\ell$ such that all strategies in $S_1((\mathcal{W}_{Odd},E'))$ are winning for \textit{Odd} from $\mathcal{W}_{Odd}$.
% \end{proposition}
% %\begin{proof}
% %Let $w \in W_1$. Then there exists a winning strategy $\sigma_1: V^* \cdot V_1 \to V$ that wins the game from $w$.
% %\end{proof}
% To prove this proposition, we introduce the fixpoint formula calculating $\mathcal{W}_{Odd}$ and obtain a ranking function $r:\mathcal{W}_{Odd} \to \mathbb{N}^\ell$ from the formula, where $l$ is the least even upper bound of the assigned priorities. 
% The ranking function is an adaptation of Jurdzinski's small progress measures to fair parity games. 
% Then using the ranking $r$ we will prove that there exists a strategy template $(\mathcal{W}_{Odd}, E')$ of $\mathcal{G}^\ell$ such that all strategies in $S_1((\mathcal{W}_{Odd}, E'))$ are winning for \textit{Odd}.
%\begin{definition}
%Let $(V', E')$ be a strategy template of a game $\mathcal{G}^\ell$ with priorities $\leq$ d. A function $\rho: V'_0 \cup V'_1 \to \mathbb{N}^{d+1}$ is a \textbf{fair parity progress measure} for $(V',E')$ if the following hold for all $v\in V'$:
%\begin{itemize}
%    \item if $v \in V'_1 \cap V^\ell$ lays on a cycle in $(V',E')$:
%    \begin{itemize}
%        \item if $\chi(v)$ is odd: \\
%        $\exists (v,w) \in E'$ with $ \rho(v) >_{l+1 -\chi(v)} \rho(w)$  and 
%        $ \forall (v,w) \in E', \rho(v) \geq_{l -\chi(v)} \rho(w)$, 
%        \item if $\chi(v)$ is even: \\$\exists (v,w) \in E'$ with $ \rho(v) >_{l+1 -\chi(v)} \rho(w)$ and $\forall (v,w) \in E', \rho(v) \geq_{l + 2 - \chi(v)} \rho(w)$.
%    \end{itemize}
%    \item otherwise:
%    \begin{itemize}
%       \item if $\chi(v)$ is odd, $\forall (v,w) \in E', \rho(v) \geq_{l+1-\chi(v)} \rho(w)$
%        \item if $\chi(v)$ is even, $\forall (v,w) \in E', \rho(v) >_{l+1-\chi(v)} \rho(w)$
%    \end{itemize}
%\end{itemize}
%\end{definition}

%\begin{lemma}
%If there is a fair parity progress measure for a strategy template $(V', E')$ for a game $\mathcal{G}^\ell$, then in each strongly connected component, the maximum priority of a simple cycle is odd.
%\end{lemma}

