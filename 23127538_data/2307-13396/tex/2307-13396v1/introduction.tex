
\vspace*{-0.3cm}
\section{Introduction}
\vspace{-1mm}

\emph{Parity games} are a canonical representation of $\omega$-regular two-player games over finite graphs, which arise from many core computational problems in the context of correct-by-construction synthesis of reactive software or hardware. In particular, two player games on graphs have been extensively used in the context of cyber-physical system design \cite{Tabuada2009,belta2017formal}, showing their practical importance. 
% 
\emph{Fairness}, on the other hand, is a property that widely occurs in this context - both as a desired property to be enforced (e.g., requiring a synthesized scheduler to fairly serve its clients), as well as a common assumption on the behavior of other components (i.e., assuming the network to always eventually deliver a data packet). 
While \emph{strong fairness} encoded by a Streett condition necessarily incurs a high additional cost in synthesis \cite{EJ99}, it is known that the general reactivity(1) (GR(1)) fragment of linear temporal logic (LTL)~\cite{BJPPS2006} allows for efficient synthesis in the presence of very restricted fairness conditions. Due to its efficiency, it is extensively used in the context of cyber-physical system design, e.g.~\cite{WEK18,AMT2013,MR2015,KFP2007,KFP2009,SKCCB2015}. %The GR(1) fragment was particularly well received in the robotics and cyber-physical systems community and was extensively used to synthesize controllers for physical systems \cite{}.

\begin{comment}

Despite the omnipresence of fairness in synthesis problems and the success of the GR(1) fragment, not much else is known about tractable fairness constraints in synthesis. A notable exception is the recent work by Banerjee et. al.~\cite{banerjee2022fast} which shows that the sub-class of \emph{strong transition fairness assumptions}~\cite{QS83,Francez,baierbook} can be handled efficiently in synthesis. In particular, they consider Rabin games where the environment player has to obey additional strong transition fairness constraints on its vertices  -- i.e., if the environment player vertex $v$ is visited infinitely often, a particular subset of the outgoing edges (called \emph{live edges}) of $v$ has to be taken infinitely often.

Banerjee et. al.~\cite{banerjee2022fast} show that such games can be solved via a symbolic fixed-point algorithm in the $\mu$-calculus that has almost the same computational worst case complexity as the algorithm for (normal) Rabin games. %This makes strong transition fairness a promising candidate for a tractable class of fairness constraints in synthesis. 
% The \emph{main contribution} of this paper is to show that this insight caries over to \emph{Parity games under strong transition fairness assumptions}. Here, player \Odd has to obey the additional strong transition fairness constraint. We call such games \emph{\Odd-fair Parity games}.
% \todo{we repeat \emph{the main contribution} twice and say different things. Maybe we should use something else here... Or express it in some other way.}
% 
The interesting insight that Banerjee et. al.~\cite{banerjee2022fast} exploit, is that the required modification of the symbolic algorithm are very local and syntactic. In particular, Banerjee et. al. also show that this intuition carries over to symbolic fixed-point algorithms for many other games. In particular, this provides a symbolic algorithm for \Odd-fair parity games.\todo{IS : I think we can make this paragraph. An alternative is below}
\end{comment}

Despite the omnipresence of fairness in such synthesis problems and the success of the GR(1) fragment, not much else is known about tractable fairness constraints in synthesis via two player games on graphs. A notable exception is the recent work by Banerjee et.\ al.~\cite{banerjee2022fast} which consideres the sub-class of \emph{strong \textbf{transition} fairness assumptions}~\cite{QS83,Francez,baierbook} which require that whenever the environment player vertex $v$ is visited infinitely often, a particular subset of the outgoing edges (called \emph{live edges}) of $v$ has to be taken infinitely often. In other words, \emph{strong \textbf{transition} fairness assumptions} limit \emph{strong fairness assumptions} to individual transitions.
Despite their limited expressive power, such restricted fairness constrains do naturally arise in resource management \cite{CAFMR13}, in abstractions of continuous-time physical processes for planning \cite{CPRT03, DTV99, PT01, DIRS18, RS14, AGR20} and controller synthesis \cite{MMSS2021, NOL17}, which makes them interesting to study.
 
Concretely, Banerjee et.\ al.~\cite{banerjee2022fast} show that \emph{parity games} with strong transition fairness assumptions on player \Odd\ -- which we call \emph{\Odd-fair parity games} -- can be solved via a symbolic fixed-point algorithm in the $\mu$-calculus with almost the same computational worst case complexity as the algorithm for the \enquote{normal} version of the same game. %This makes strong transition fairness a promising candidate for a tractable class of fairness constraints in synthesis. 
The existence of quasi-polynomial time solution algorithms for \Odd-fair parity games then follows as a corollary of their nested fixed-point characterization~\cite{HS21, ANP21, JMT22}. %\todo{IS : Here, I have removed the intuition behind Banerjee et al.'s construction (syntactic and local change)}
% 
Unfortunately, it is well known that symbolic fixed-point computations become cumbersome very fast for parity games, as the number of priorities in the game graph increases, leading to high computation times in practice. 
Given the known inefficiency of existing quasi-polynomial algorithms for parity games \cite{oink, Parys19}, despite their theoretical advantages, they are not viable candidates for adoption in the development of efficient solution algorithms for \Odd-fair parity games either.
% 
For (normal) parity games, computational tractability can be achieved by other algorithms, such as Zielonka's algorithm \cite{Zielonka98}, tangle learning \cite{van-Dijk-tangle-learning} or strategy-improvement \cite{Schewe-strategy-improvement}, implemented in the state-of-the-art tool \texttt{oink}~\cite{oink}, with Zielonka's algorithm being widely recognized as the most prominent approach.

The \emph{\textbf{main contribution}} of this paper is a Zielonka-type algorithm, referred to as \enquote{\emph{\Odd-fair Zielonka's algorithm}}, for solving \Odd-fair parity games. This novel algorithm meets the efficiency of Zielonka's algorithm while maintaining the same computational worst-case complexity (which is exponential just like the worst-case complexity of the fixed-point algorithm from \cite{banerjee2022fast}).
Using a prototype implementation, we experimentally verify that it retains the algorithmic advantage Zielonka's algorithm holds over fixed-point algorithms for classical parity games. %exhibits similar advantages to Zielonka's algorithm over fixed-point algorithms for classical parity games,


In contrast to the work by Banerjee et. al.~\cite{banerjee2022fast}, the adaptation and the correctness proof of \Odd-fair \emph{Zielonka's algorithm} requires the understanding of \Odd player strategies, while \cite{banerjee2022fast} studies the solution of such games solely from the \Even player's perspective. Unfortunately, \Odd strategies are substantially more complex than \Even strategies in such games, as they are not positional -- while player \Even strategies still are (see \cite[Thm.3.10]{banerjee2022fast}). %Positionality of \Even strategies is due to live edges only originating from \Odd vertices, and can be achieved as... . 
The \emph{\textbf{second contribution}} of this paper is therefore the formalization of \Odd player strategies in \Odd-fair parity games, via so called \emph{strategy templates}, which was unexplored prior to this work.
We give a constructive proof for the existence of strategy templates winning for \Odd from all vertices in the winning region of \Odd.
This serves dual purposes: firstly, it enables us to prove the correctness of the \Odd-fair Zielonka's algorithm; secondly, it stands as a noteworthy contribution in its own right, augmenting our understanding of additional fairness assumptions in two-player games which are currently only unsatisfactorily adressed in various practically motivated synthesis problems.


