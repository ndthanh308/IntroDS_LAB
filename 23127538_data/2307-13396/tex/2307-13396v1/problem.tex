
\vspace{-0.15cm}

\section{Strategy Templates}\label{sec:templates}

In this section, we introduce a formalization of player \Odd strategies in \Odd-fair parity games via \emph{strategy templates}.
% 
In contrast to player \Even, player \Odd winning strategies are no longer positional in \Odd-fair parity games, as illustrated by the following example. %that requires the same number of symbolic steps as the algorithm computing winning strategies for \Even in \enquote{normal} parity games.
% \vspace{-0.5em}
\begin{example}\label{ex:strategytemplates}
Consider the three different parity games depicted in Fig.~\ref{fig:Oddstrategies1}. %, three \Odd-fair parity games are depicted, with circles indicating \Ve and squares indicating \Vo. Edges in $E^\ell$ are shown by dashed lines. All nodes are labeled with their priorities.
   In all three games, \Odd has a winning strategy from all vertices, i.e., $\mathcal{W}_{Odd}=V$. %The red-colored edges indicate \Odd's strategy: if \Odd takes the red edges alternatingly from the source nodes, it wins from all nodes. 
  However, in order to win, the vertex $3$ has to be seen infinitely often in game (a) and (b), which forces \Odd to use its live edge\textbackslash s infinitely often. This prevents the existence of a positional strategy for \Odd in games (a) and (b): In (a) it needs to somehow alternate between (it's only) live edge to $4$ and a \enquote{normal} edge to $7$ (both indicated in red) in order to win, and in (b) it needs to somehow alternate between all its live edges (also indicated in red). In the game (c), \Odd can win by 'escaping' its live vertex $3$ to a \enquote{normal} vertex $5$, and thereby has a positional strategy. % (again indicated in red).
   
  Now consider the subgraph of each game formed by all colored edges (red and blue), which include the strategy choices from \Vo and \emph{all} outgoing edges from \Ve. As we have seen that \Odd needs to play all red edges repeatably, this subgraph represents the paths that \emph{can} be seen in the game depending on the \Even strategy. Hence, a node $v\in\Vl\subseteq\Vo$ can be seen infinitely often in a play (compliant with \Odd's strategy), if it lies on a cycle in this subgraph. We observe that, in games (a) and (b), node $3$ lies on cycles in this subgraph, whereas in game (c), it does not. 
  We further see that whenever a vertex  $v\in\Vl$ lies on a cycle, \Odd needs to take all its outgoing live edges (as for vertex $3$ in example (b)) and possibly one more edge (as for vertex $3$ in example (a)), for all other vertices in $\Vo$ a positional strategy suffices (as for vertex $5$ in all examples, and for vertex $3$ in example (c)). This shows that \Odd strategies are intuitively still \enquote{almost positional}.
% 
\end{example}

% Figure environment removed


\vspace*{-0.2cm}

The intuitions conveyed by Ex.~\ref{ex:strategytemplates} are formalized by the following definitions. % for \Odd strategy templates.


\begin{definition}[\Odd Strategy Template]\label{def:Oddstrategytemplate}
 Given an \Odd-fair parity game $\mathcal{G}^\ell = \ltup{\mathcal{G}, E^\ell}$ with \newline $\mathcal{G} = \langle V, \Ve, \Vo, E, \chi\rangle$, an \Odd \emph{strategy template} $\mathcal{S}$ over $\mathcal{G}^\ell$ is a subgraph of $\mathcal{G}$ given as follows: $\mathcal{S}:=\tup{V',E'}$ where $V'\subseteq V$ and $E'\subseteq E \cap (V' \times V')$ such that the following hold,
\begin{compactitem}\label{item:Oddstrtemprules}
 \item if $v \in \Vo \cap V'$ does not lie on a cycle in $(V',E')$, then $|E'(v)|=1$,
 \item if $v \in \Vo \cap V'$ lies on a cycle in $(V',E')$ then $E^\ell(v) \subseteq E'(v)$ and  $1\leq |E'(v)|\leq |E^\ell(v)| + 1$,
 \item if $v \in \Ve \cap V'$, then  $E'(v) = E(v)$.
\end{compactitem}
\end{definition}
%
\begin{definition}\label{def:compliantstrat}
 Let  $\mathcal{G}^\ell = \ltup{\mathcal{G}, E^\ell}$ be an \Odd-fair parity game with \Odd strategy template $\mathcal{S}=\tup{V',E'}$, and $V'_\Odd := V' \cap V_\Odd$. Then an
\Odd strategy $\rho$ is said to be \textbf{compliant} with $\mathcal{S}$ if  
it is a winning strategy in the game $\ltup{\gamegraph,\alpha'}$ where $\gamegraph= \tup{V,\Ve,\Vo,E}$ and 
\begin{subequations}
 \begin{align}
 \alpha':= &\textstyle\bigwedge_{v\in\Vo'}(\,\square\, (\,v \implies \bigvee_{(v,w)\in E'} \bigcirc\, w\,))\,\label{equ:alpha:a}\\
 & \textstyle\wedge \bigwedge_{v\in\Vo'} (\,\square \,\diamondsuit\, v \implies \bigwedge_{(v,w)\in E'}\square\, \diamondsuit\, (\,v \wedge \bigcirc \,w\,)).\label{equ:alpha:b}
\end{align}
\end{subequations}
\end{definition}

Intuitively, for all \Odd vertices in $\mathcal{S}$, the strategy $\rho$ compliant with $\mathcal{S}$ takes only their outgoing edges in $\mathcal{S}$ \eqref{equ:alpha:a}, and if a play visits an \Odd node $v$ infinitely often, then $\rho$ takes each of $v$'s outgoing edges in $\mathcal{S}$ infinitely often \eqref{equ:alpha:b}.
% 
For an \Odd strategy template $\mathcal{S}$, if $v \in V'_\Odd$ lies on a cycle in $\mathcal{S}$, then by Def. \ref{def:Oddstrategytemplate}, $\mathcal{S}$ contains all live outgoing edges of $v$. By \eqref{equ:alpha:b} any \Odd strategy $\rho$ compliant with $\mathcal{S}$ satisfies the fairness condition in \eqref{eq:fairness-ltl} for $v$. 
On the other hand, if $v \in V'_\Odd$ does not lie on a cycle in $\mathcal{S}$, then by \eqref{equ:alpha:a} any such $\rho$ sees $v$ at most once. Thus $\rho$ trivially satisfies \eqref{eq:fairness-ltl} for $v$. 
This observation is stated in the following proposition.


\begin{proposition}
 Given the premisses of Def.~\ref{def:compliantstrat} let $\pi$ be a play starting from a node in $V'$ that complies with $\rho$. Then $\pi \models \alpha$ where $\alpha$ if the LTL formula in~\eqref{eq:fairness-ltl}.%\vspace{-2mm}
\end{proposition}

Next, we define \Even strategy templates. Each \Even strategy template encodes a unique \Even positional strategy, which is known to exist in \Odd-fair parity games \cite{banerjee2022fast}, due to the lack of fair edges defined on \Even vertices. %, \Even strategy templates are very simple\footnote{In fact, \Even strategy templates simply encode a positional strategy and are only re-defined to make further arguments more symmetric for both players.}.
\begin{definition}\label{def:Evenstrategytemplate}
    Given an \Odd-fair parity game $\mathcal{G}^\ell = \ltup{\mathcal{G}, E^\ell}$ with \newline $\mathcal{G} = \langle V, \Ve, \Vo, E, \chi\rangle$, an \Even \emph{strategy template} $\mathcal{S}$ over $\mathcal{G}^\ell$ is a subgraph of $\mathcal{G}$ given as $\mathcal{S}:=\tup{V', E'}$ where $V'\subseteq V$ and $E'\subseteq E \cap (V' \times V')$ such that,    \begin{compactitem}\label{item:Evenstrtemprules}
     \item if $v \in \Ve \cap V'$, then $|E'(v)|=1$,
     \item if $v \in \Vo \cap V'$, then  $E'(v) = E(v)$.
    \end{compactitem}
\end{definition}

\vspace*{-0.1cm}

An \Even strategy $\rho$ is compliant with the \Even strategy template $\mathcal{S} = \tup{V', E'}$ if for all $v \in V'_\Even$, $\rho(v) = E'(v)$. In other words, $\rho$ is the positional strategy defined by $\mathcal{S}$.

Let $\rho$ be an \Odd (\Even) strategy, compliant with the \Odd (\Even) strategy template $\mathcal{S}$ and let $\pi$ be a play compliant with $\rho$. Then we call $\pi$ a play \emph{compliant with $\mathcal{S}$}.

\vspace*{-0.1cm}

\begin{definition}
An \Odd (\Even) strategy template $\mathcal{S}=\ltup{V', E'}$ is \emph{winning} in the \Odd-fair parity game $\mathcal{G}^\ell$ if all \Odd (\Even) strategies $\rho$ compliant with $\mathcal{S}$ are winning for player \Odd (\Even) in $\mathcal{G}^\ell$ from $V'$. A winning \Odd (\Even) strategy template $\mathcal{S}$ is called \emph{maximal} if $V'=\Wo$ ($\We$).%\vspace{-2mm}
\end{definition}

\vspace*{-0.2cm}
We note that maximal winning \Odd (\Even) strategy templates $\mathcal{S}$ immediately imply that for every vertex $v\in \Wo$ ($\We$) there exists a winning strategy for player \Odd (\Even) from $v$ that is compliant with $\mathcal{S}$.
% 
The existence of maximal winning \Even strategy templates follows from the existence of positional \Even strategies \cite{banerjee2022fast}. 
% 
The first main contribution of this paper is a constructive proof showing the existence of maximal winning \Odd strategy templates given in the next section. 
This result is then used in Sec.~\ref{sec:zielonka} to prove the correctness of \Odd-fair Zielonka's algorithm, which is introduced there.



