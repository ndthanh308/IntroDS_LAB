\vspace*{-0.3cm}
\section{Preliminaries}
\noindent\textbf{Notation.}
We use $\mathbb{N}$ to denote the set of natural numbers including zero and $\mathbb{N}^+$ to denote positive integers. Let $\Sigma$ be a finite set. Then  $\Sigma^*$ and $\Sigma^\omega$ denote the sets of finite and infinite words over $\Sigma$, respectively. %, and $\Sigma^\infty$ is equal to $\Sigma^*\cup \Sigma^\omega$.


\smallskip
\noindent\textbf{Game graphs.}
A \emph{game graph} is a tuple $\gamegraph= \tup{V,V^0,V^1,E}$ where $(V,E)$ is a finite directed graph with  \emph{edges} $ E $ and \emph{vertices} $ V $ partioned into player $0$ and player $1$ vertices, $V^0$ and $V^1$, respectively. 
Without loss of generality, we can assume
that all nodes in $V$ have at least one outgoing edge. Under this assumption, there exist plays from each vertex.
A \emph{play} originating at a vertex $v_0$ is an infinite sequence of vertices $\pi=v_0v_1\ldots \in V^\omega$. 
For $v \in V$, $E(v)$ denotes its successor set $\{w \mid (v, w) \in E\}$. 

\smallskip
\noindent\textbf{LTL winning conditions.}
Given a game graph $\gamegraph$, we consider winning conditions specified using a formula $\spec$ in \emph{linear temporal logic} (LTL) over the vertex set $V$, that is, we consider LTL formulas whose atomic propositions are sets of vertices. 
In this case the set of desired infinite plays is given by the semantics of $\spec$ which is an $\omega$-regular language $\lang(\spec)\subseteq V^\omega$. 
The standard definitions of $\omega$-regular languages and LTL are omitted for brevity and can be found in standard textbooks \cite{baierbook}. A game graph $\gamegraph$ under the winning condition $\spec$ is written as $\ltup{\gamegraph, \spec}$. A play $\pi$ is winning for player $0$ in $\ltup{\gamegraph, \spec}$ if $\pi\in\lang(\spec)$, i.e. $\pi\models\spec$.

\smallskip
\noindent \textbf{Strategies.}
A \emph{strategy} for player $j$ over the game graph $\gamegraph$ is a function $\rho^j : V^* \cdot V^j \to V$ with the constraint that for all $u\cdot v \in V^* \cdot V^j$ it holds that $\rho^j(u \cdot v) \in E(v)$. A play $\pi=v_0v_1\ldots \in V^\omega$ is compliant with $\rho^j$ if for all $i\in \mathbb{N}$ holds that $v_i\in V^j$ implies $v_{i+1}=\rho^j(v_0 \ldots v_i)$. A strategy $\rho^j$ is winning from a subset $V'$ of vertices of the game $\ltup{\gamegraph,\Psi}$ if all plays $\pi$ in $G$ that start at a vertex in $V'$ and are compliant with $\rho^j$ are winning w.r.t.\ $\Psi$. 
A strategy $\rho$ is called \emph{positional} iff for all $w_1, w_2 \in V^*$, $\rho(w_1 \cdot v) = \rho(w_2 \cdot v)$. %That is, a positional strategy $\rho_0$ for

\smallskip
\noindent\textbf{Parity Games.}
Parity games are particular two player games over a game graph $\gamegraph$ where the winning condition is given by a particular mapping of vertices. Formally, a parity game is a tuple $\mathcal{G} = \ltup{ V, \Ve, \Vo, E, \chi}$, where $\tup{V, \Ve, \Vo, E}$ is a game graph and $\chi : V \to \mathbb{N}^+$ is a function which labels each vertex with an integer value, called a \emph{priority}. The players $0$ and $1$ are called $\Even$ and $\Odd$ in a parity game and 
a play $\pi = v_1 v_2 \ldots$ is winning for \Even iff $\max\{\inf(\pi)\}$ is \emph{even}, where $\inf(\pi)$ is the set of vertices visited infinitely often in $\pi$. Otherwise the play is winning for \Odd. %We define the sets $C_i := \{ v \in V \mid \chi(v) = i\}$ and $\overline{C_i} := V \setminus C_i$ to ease notation.

A node $v\in V$ is said to be won by \Even, if \Even has a (winning) strategy $\rho$ such that 
all plays $\pi = v\cdot \pi'$ that are compliant with $\rho$ are won by \Even. %\IS{Q: Is there a difference between saying (i) 'all plays that are compliant wih $\rho_0$ are won by... and (ii) for all player \Odd strategies $\rho_1$, all plays that are compliant with $\rho_0$ and $\rho_1$ are won by...}
The winning region of \Even is the set of all nodes won by \Even and is denoted by $\We$. The winning region of \Odd, $\Wo$, is defined similarly. 
% 
It is well-known that parity games are determined, that is, all nodes are either in $\mathcal{W}_{Even}$ or in $\mathcal{W}_{Odd}$; and that both players have positional winning strategies from their respective winning regions \cite{EJ89}.

\begin{comment}
\smallskip
\noindent\textbf{Fair Parity Games.}
A \emph{fair parity game} $\mathcal{G}^\ell$ is a tuple $\ltup{\mathcal{G},E^\ell}$, where $\mathcal{G} = \langle V, \Ve, \Vo, E, \chi\rangle$ is a parity game, $E^\ell \subseteq E $ is a set of \emph{live edges} %inducing a \emph{strong transition fairness constraint}. 
and $V^\ell$, the domain of the relation $E^\ell$, is the set of \emph{live vertices}. The live edges induce a \emph{strong transition fairness constraint} -- whenever a live vertex $v$ is visited infinitely often, every outgoing live edge $(v, w') \in E^\ell$ needs to be taken infinitely often.
Formally, a play $\pi$ in $\mathcal{G}$ \emph{complies} with $E^\ell$ if the LTL formula
\begin{equation}\label{eq:fairness-ltl}
    \textstyle\alpha := \bigwedge_{(v, w)\in E^\ell} (\,\square\, \diamondsuit \,v \implies \square\, \diamondsuit\, (v \wedge \bigcirc w))\vspace{-1mm}
\end{equation} 
holds along $\pi$, i.e.\ $\pi\models\alpha$.%\footnote{$\square$ stands for "always", $\diamondsuit$ for "eventually" and $\bigcirc$ for "next". For more detailed information on LTL \cite{}}
 A play $\pi$ is winning for player \Even if and only if the play does not comply with $E^\ell$ or $\max\{\inf(\pi)\}$ is even.

\smallskip
\noindent\textbf{\Odd-Fair Parity Games.}
An \emph{\Odd-fair parity game} is a particular instance of a fair parity game, where only player \Odd is restricted by an additional \emph{strong transition fairness constraint}, i.e., $V^\ell\subseteq\Vo$. In this case, a play $\pi$ is winning for \Even in $\mathcal{G}^\ell$ if and only if $\pi \models \neg\alpha$ or $\max\{\inf(\pi)\}$ is \emph{even}. Dually, $\pi$ is winning for \Odd iff $\pi \models \alpha$ and $\max\{\inf(\pi)\}$ is odd. A strategy $\rho$ over $\gamegraph$ is therefore winning for \Even (resp.\ \Odd) in $\mathcal{G}^\ell$ if all plays compliant with $\rho$ are winning for \Even (resp. \Odd) in $\mathcal{G}^\ell$.

As the winning condition of a parity game can be equivalently modeled by a suitably defined LTL winning condition, we see that \Odd-fair parity games are a special deterministic $\omega$-regular game with perfect information. This implies that \Odd-fair parity games are determined (by the Borel determinacy theorem \cite{Martin75}) and whenever there exists a winning strategy for \Even/\Odd in such a game, then there also exists one with \emph{finite} memory \cite{GH82}. 
\end{comment}
\smallskip
\noindent\textbf{\Odd-Fair Parity Games.}
An \emph{\Odd-fair parity game} 
$\mathcal{G}^\ell$ is a tuple $\ltup{\mathcal{G},E^\ell}$, where $\mathcal{G} = \langle V, \Ve, \Vo, E, \chi\rangle$ is a parity game, $E^\ell \subseteq E $ is a set of \emph{live edges} that originate from \Odd player vertices %inducing a \emph{strong transition fairness constraint}. 
and $V^\ell \subseteq \Vo$, the domain of the relation $E^\ell$, is the set of \emph{live vertices}.
The live edges induce a \emph{strong transition fairness constraint} -- whenever a live vertex $v$ is visited infinitely often, every outgoing live edge $(v, w') \in E^\ell$ needs to be taken infinitely often.
Formally, a play $\pi$ in $\mathcal{G}$ \emph{complies} with $E^\ell$ if the LTL formula%
\footnote{Here, $\square$, $\diamondsuit$ and $\bigcirc$ stand for the LTL operators 'always', 'eventually' and 'next'.}
\begin{equation}\label{eq:fairness-ltl}
    \textstyle\alpha := \bigwedge_{(v, w)\in E^\ell} (\,\square\, \diamondsuit \,v \implies \square\, \diamondsuit\, (v \wedge \bigcirc w))\vspace{-1mm}
\end{equation} 
holds along $\pi$, i.e.\ $\pi\models\alpha$. %\footnote{$\square$ stands for "always", $\diamondsuit$ for "eventually" and $\bigcirc$ for "next". For more detailed information on LTL \cite{}}
A play $\pi$ is winning for \Even in $\mathcal{G}^\ell$ if and only if $\pi \models \neg\alpha$ or $\max\{\inf(\pi)\}$ is \emph{even}. Dually, $\pi$ is winning for \Odd iff $\pi \models \alpha$ and $\max\{\inf(\pi)\}$ is odd. A strategy $\rho$ over $\gamegraph$ is therefore winning for \Even (resp.\ \Odd) in $\mathcal{G}^\ell$ if all plays compliant with $\rho$ are winning for \Even (resp. \Odd) in $\mathcal{G}^\ell$.

As the winning condition of a parity game can be equivalently modeled by a suitably defined LTL winning condition, we see that \Odd-fair parity games are a special deterministic $\omega$-regular game with perfect information. This implies that \Odd-fair parity games are determined (by the Borel determinacy theorem \cite{Martin75}) and whenever there exists a winning strategy for \Even/\Odd in such a game, then there also exists one with \emph{finite} memory \cite{GH82}. 

