\subsection{Experimental Results}
% 

We conducted an experimental study to empirically validate the claim that our new \Odd-fair Zielonka's algorithm retains its efficiency in practice (see App.~\ref{app:experiments} for details). 

We generated \Odd-fair parity instances manipulating $286$ benchmark instances of PGAME$\_$ Synth$\_$2021 dataset of the SYNTCOMP benchmark suite~\cite{syntcomp} and $51$ instances of PGSolver dataset of Keiren's benchmark suite~\cite{keirens} by adding live edges to the given (normal) parity games.
%
We empirically compared the (non-optimized\footnote{While optimized version of \texttt{N-ZL} and \texttt{N-FP} are available in \texttt{oink} \cite{oink} our goal is a conceptual comparison, which is better achieved by similar (non-optimized) implementations for all algorithms.}) C++-based implementations of 
% 
\begin{inparaenum}[(i)]
 \item the \Odd-fair Zielonka's algorithm (\texttt{OF-ZL}) from Alg.~\ref{algo:fair-zielonka-bb},
 \item the \enquote{normal} Zielonka's algorithm (\texttt{N-ZL}) from~\cite{Zielonka98}, 
 \item the fixed-point algorithm for \Odd-fair parity games (\texttt{OF-FP}) from \cite{banerjee2022fast} implementing~\eqref{eq:fp-odd}, and
 \item the \enquote{normal} fixed-point algorithm (\texttt{N-FP}) for \enquote{normal} parity games from~\cite{EJ91}.% \todo{IS: a reviewer correctly asked us to add \texttt{N-FP} to the implemented algorithms list as well, and I did. Here for normal parity fixed-point algorithm should I cite an old paper that gives the parity fixed-point, or is it enough to cite Banerjee et. al. agan (as I did here) since the fixed-point formulation for regular parity is given there as well.}
\end{inparaenum}
On the \emph{SYNTCOMP benchmarks}, the time-out rates are: $82$ instances for \texttt{OF-FP}, $58$ for \texttt{OF-ZL}; $73$ for \texttt{N-FP} and $47$ for \texttt{N-ZL}. On the 204 instances that neither of the algorithms time out the average computation times are: $122.7$ seconds for \texttt{OF-FP}, $4.6$ seconds for \texttt{OF-ZL}, $45.2$ seconds for \texttt{N-FP} and $3.6$ seconds for \texttt{N-ZL}.   
For all instances that did not time out for all four algorithms, Fig.~\ref{fig:logscale-main} shows scatter plots comparing the computation times of \texttt{OF-ZL} with \texttt{OF-FP} (left) and  \texttt{OF-ZL} with \texttt{N-ZL} (right) using logarithmic scaling. The diagonal shows instances with similar computation times. Points above the diagonal show superior performance of \texttt{OF-ZL}.
For the \emph{PGSolver dataset} \texttt{OF-FP} timed out on all generated instances, whereas \texttt{OF-ZL} took $24.9$ seconds on average to terminate.

% Figure environment removed

We clearly see that  \texttt{OF-ZL} performs up to one order of magnitude better than \texttt{OF-FP} in many instances while \texttt{OF-ZL} and \texttt{N-ZL} perform very similar on the given benchmark instances. In addition, we observe that \texttt{OF-FP} starts timing out as soon as the examples became more complex. %, being especially sensitive to the increase in the number of priorities. 
These outcomes match the known comparison results between the naive fixed-point calculation versus Zielonka's algorithm, on normal parity games.  

